% CVPR 2023 Paper Template
% based on the CVPR template provided by Ming-Ming Cheng (https://github.com/MCG-NKU/CVPR_Template)
% modified and extended by Stefan Roth (stefan.roth@NOSPAMtu-darmstadt.de)

\documentclass[10pt,twocolumn,letterpaper]{article}
% \usepackage[accsupp]{axessibility} 
%%%%%%%%% PAPER TYPE  - PLEASE UPDATE FOR FINAL VERSION
% \usepackage[review]{cvpr}      % To produce the REVIEW version
\usepackage{cvpr}              % To produce the CAMERA-READY version
% \usepackage[pagenumbers]{cvpr} % To force page numbers, e.g. for an arXiv version

\usepackage{graphicx}
\usepackage{amsmath}
\usepackage{amssymb}
\usepackage{amsthm}
\usepackage{booktabs}
\usepackage{appendix}
\usepackage{multirow}
\usepackage{graphicx}
\usepackage{makecell}
\usepackage{xcolor,colortbl}
\usepackage[accsupp]{axessibility}  % Improves PDF readability for those with disabilities.
% \usepackage[linesnumbered,ruled,vlined]{algorithm2e}
\usepackage[linesnumbered,ruled]{algorithm2e}

% It is strongly recommended to use hyperref, especially for the review version.
% hyperref with option pagebackref eases the reviewers' job.
% Please disable hyperref *only* if you encounter grave issues, e.g. with the
% file validation for the camera-ready version.
%
% If you comment hyperref and then uncomment it, you should delete
% ReviewTempalte.aux before re-running LaTeX.
% (Or just hit 'q' on the first LaTeX run, let it finish, and you
%  should be clear).

\usepackage[pagebackref,breaklinks,colorlinks]{hyperref}

\usepackage[capitalize]{cleveref}
\crefname{section}{Sec.}{Secs.}
\Crefname{section}{Section}{Sections}
\Crefname{table}{Table}{Tables}
\crefname{table}{Tab.}{Tabs.}


% \theoremstyle{plain}
\newtheorem{theorem}{Theorem}
\newtheorem{proposition}{Proposition}
\newtheorem{lemma}{Lemma}
\newtheorem{corollary}{Corollary}
\theoremstyle{definition}
\newtheorem{definition}{Definition}
\newtheorem{assumption}{Assumption}
% \theoremstyle{remark}
\newtheorem{remark}{Remark}

\DeclareMathOperator{\lap}{Lap} 
\DeclareMathOperator{\Clip}{Clip} 
\DeclareMathOperator{\topk}{top} 
\DeclareMathOperator{\blurs}{blurs} 
\DeclareMathOperator{\blur}{blur} 
\DeclareMathOperator{\lus}{lus} 
\DeclareMathOperator{\randk}{rand} 
\DeclareMathOperator{\spar}{spar}

%%%%%%%%% PAPER ID  - PLEASE UPDATE
\def\cvprPaperID{4552} % *** Enter the CVPR Paper ID here
\def\confName{CVPR}
\def\confYear{2023}


\newcommand{\ls}[1]{{\color{red}{\bf\sf [LS: #1]}}}
\newcommand{\syf}[1]{{\color{blue}{\bf\sf [SYF: #1]}}}

\begin{document}


%%%%%%%%% TITLE - PLEASE UPDATE
\title{Make Landscape Flatter in Differentially Private Federated Learning}

\author{
Yifan Shi\textsuperscript{\rm 1} 
\quad
Yingqi Liu\textsuperscript{\rm 2}
\quad
Kang Wei\textsuperscript{\rm 2}
\quad
Li Shen\textsuperscript{\rm 3,}\thanks{Corresponding authors: Li Shen and Xueqian Wang}
\quad
Xueqian Wang\textsuperscript{\rm 1,*}
\quad
Dacheng Tao\textsuperscript{\rm 3}
\\
\textsuperscript{\rm 1}Tsinghua University, Shenzhen, China; \textsuperscript{\rm 3}JD Explore Academy, Beijing, China\\
\textsuperscript{\rm 2}Nanjing University of Science and Technology, Nanjing, China\\
%\textsuperscript{\rm 3}JD Explore Academy, Beijing, China\\
{\tt\small shiyf21@mails.tsinghua.edu.cn;
lyq@njust.edu.cn;
kang.wei@njust.edu.cn; 
}\\
{\tt\small mathshenli@gmail.com; wang.xq@sz.tsinghua.edu.cn; dacheng.tao@gmail.com
}
}
\maketitle

%%%%%%%%% ABSTRACT
\begin{abstract}
To defend the inference attacks and mitigate the sensitive information leakages in Federated Learning (FL), client-level Differentially Private FL (DPFL) is the de-facto standard for privacy protection by clipping local updates and adding random noise. However, existing DPFL methods tend to make a sharper loss landscape and have poorer weight perturbation robustness, resulting in severe performance degradation. To alleviate these issues, we propose a novel DPFL algorithm named DP-FedSAM, which leverages gradient perturbation to mitigate the negative impact of DP. Specifically, DP-FedSAM integrates Sharpness Aware Minimization (SAM) optimizer to generate local flatness models with better stability and weight perturbation robustness, which results in the small norm of local updates and robustness to DP noise, thereby improving the performance. From the theoretical perspective, we analyze in detail how DP-FedSAM mitigates the performance degradation induced by DP. Meanwhile, we give rigorous privacy guarantees with Rényi DP and present the sensitivity analysis of local updates. At last, we empirically confirm that our algorithm achieves state-of-the-art (SOTA) performance compared with existing SOTA baselines in DPFL. Code is available at \url{https://github.com/YMJS-Irfan/DP-FedSAM}
\end{abstract}

%%%%%%%%% BODY TEXT
\section{Introduction}
\label{sec:intro}
\begin{figure}[t]
\begin{center}
    \includegraphics[width=1\linewidth]{figures/teaser.pdf}
\end{center}
\vspace{-0.1in}
\caption{\textbf{{\em Foggy} vs {\em Clear} NeRF.} Our \ournerf gets rid of reconstruction errors manifested as foggy ``floaters" in the density volume without additional input or significant computational overhead. 
%
Below are density profiles along a given ray before and after our geometry correction procedure, where we discard density peaks corresponding to floaters.
}
\label{fig:teaser}
\vspace{-0.2in}
\end{figure}



%The emergence of 
Neural Radiance Fields (NeRFs)~\cite{mildenhall2020nerf}  %and its variants 
have made revolutionary contributions in %photo-realistic 
novel view synthesis~\cite{barron2021mip,barron2022mip}, 
autonomous driving~\cite{rematas2022urban,tancik2022block}, digital human~\cite{hong2022headnerf,zhao2022humannerf}, and 3D content generation~\cite{eg3d,poole2022dreamfusion,lin2022magic3d}.
%by leveraging a multi-layer perceptron (MLP) to implicitly model the mapping from input 5D coordinates (i.e., 3D coordinates $\mathbf{x} = (x,y,z)$ and 2D viewing directions $\mathbf{d}=(\theta,\phi)$) to volume density $\sigma$ and view-dependent emitted radiance color $\mathbf{c} = (r,g,b)$. 
%
%They then use traditional volume rendering mechanisms on the obtained continuous 5D function (i.e., MLP) to generate novel views. 
To date, unfortunately, most NeRF-based methods encounter challenges when tackling large-scale cluttered scenes (e.g., Fig.~\ref{fig:teaser}):
\begin{enumerate}[leftmargin=0.16in, topsep=2pt,itemsep=-1ex,partopsep=1ex,parsep=1ex]
\item Input observations used for NeRF are often too sparse  compared to forward-facing or synthetic looking-inward scenes;
%\item Recovering fine-grained objects within a large volume is challenging for NeRF; %in capturing details accurately.
\item View-dependent visual effects give rise to ambiguity, resulting in a ``foggy" density field as shown in Fig.~\ref{fig:teaser}. 
%
Such artifacts are particularly pronounced in indoor scenes strewn with view-dependent appearances, such as specular highlights, glossy surface reflections from man-made objects. 
\end{enumerate}

Despite attempts to enhance NeRF's rendering quality given suboptimal input, such as using 3D conical frustums~\cite{barron2021mip,barron2022mip}, physically-grounded augmentations~\cite{chen2022aug}, and misalignment correction~\cite{jiang2022alignerf},  these challenges have yet to be fully resolved.
%
Depth supervision~\cite{deng2022depth, wei2021nerfingmvs} or proxy geometry~\cite{xu2021scalable,wu2022scalable} images can help alleviate the challenges in handling large-scale with sparse input, at the expense of %but they come at the cost of requiring 
expensive pre-processing or additional input.
%
Another line of work~\cite{wang2021neus, oechsle2021unisurf, wang2022neuris} achieves better reconstruction of surface geometry by using signed distances instead of volume density as scene representation. However, they sacrifice the ability to synthesize photo-realistic novel views.

%We observe that NeRF has been suffering from foggy ``floater" artifacts in large-scale cluttered scenes.
%
%Such artifacts are particularly pronounced in indoor scenes strewn with view-dependent appearances from man-made objects. 
%
To address the above issues, we propose an extension to NeRF, dubbed as {\bf \ournerf}, which enforces effective {\em appearance} and {\em geometry} constraints conducive to accurate colors and 3D densities estimation. We believe \ournerf can contribute beyond novel view synthesis, such as NeRF object detection~\cite{hu2022nerf}, NeRF object segmentation~\cite{zhi2021place, liu2022unsupervised, fan2022nerf,ren2022neural}, and NeRF registration~\cite{goli2022nerf2nerf}, where the rooms for improvement are substantial if more accurate color and density estimation are available.

Correspondingly, there are two steps in \ournerf. First, for appearance correction, the view-independent and view-dependent color components are predicted from the underlying 3D scene, which is combined to produce the final color estimation (Fig.~\ref{fig:toaster}).
%
The view-independent component (diffuse color and shading) captures the overall scene color, while the view-dependent component (highlights or reflections) captures color variations due to changes in viewing angle.
%
\ournerf then discards these view-dependent appearances in the training views to prevent them from interfering with the density estimation.
%
Second, a simple and effective geometry correction procedure will be performed to further eliminate the foggy ``floaters" or density errors. This geometry correction procedure is based on an assumption in line with traditional ray tracing in computer graphics.
\begin{comment}
% xh: basically copying method
On the other hand, ClearNeRF performs a geometric correction procedure performed on each traced ray during inference to refine the density estimation and better tackle the floater artifacts. 
%
The geometry correction procedure assumes that there should only be one salient peak along each traced ray during NeRF inference. 
Only the salient peak closest to the ray origin (the camera center) corresponds to  true geometry while the others will be manifested as foggy floaters hovering in the density volume. 
%
This assumption is in line with traditional ray tracing in computer graphics where in the absence of noise, only one intersection per ray should be returned to indicate the closest ray-object intersection.
%
\end{comment}
%%%%%%%%%%%
%As shown in Fig.~\ref{fig:teaser}, when reconstructing an indoor scene with sparse input and highly view-dependent objects, NeRF produces severe floating artifacts due to its attempt to explain view-dependent appearances.
%
Experiments verify that our proposed \ournerf can effectively get rid of floater artifacts without additional input.% or significant computational overhead. 


In summary, our contributions include the following:
\begin{itemize}[leftmargin=0.16in, topsep=2pt,itemsep=-1ex,partopsep=1ex,parsep=1ex]
    \item We propose a concise method for decomposing view-independent and view-dependent appearance during NeRF training and eliminate the interference of view-dependent appearance.
    \item We propose a geometric correction procedure performed on each traced ray during inference to refine the density estimation and better tackle the floater artifacts.
    \item Extensive experiments and ablations verify the effectiveness of our core designs and results in improvements over the vanilla NeRF and other state-of-the-art alternatives.
    %without additional computational resources or other inputs.
\end{itemize}




\section{Related Work}
\label{section:related_work}

\import{tables/}{sota_comparison}

There have been many work focusing on fault injection for \titlecaseabbreviationpl{dnn}, focusing on the tools required for fault injection as well as the different sampling models. We show a comparison in Table \ref{table:sota_comparison}.
enpheeph \cite{colucciEnpheephFaultInjection2022a}, TensorFI \cite{liTensorFIConfigurableFault2018} and LLTFI \cite{agarwalLLTFIFrameworkAgnostic2022} focus on providing innovative fault injection frameworks to speed up the overhead of running fault injection compared to normal \titlecaseabbreviation{dnn} execution. However, they do not employ any specific algorithm for sampling, resorting to random uniform sampling. At the same time, they are able to cover the whole search space, as they do not filter the possible faults.
On the other hand, BinFI \cite{chenBinFIEfficientFault2019} and AVFI \cite{jhaMLBasedFaultInjection2019} provide improved fault models, not focusing on how the fault injection is executed as the previous works, but how to choose the faults using external knowledge, in their case human knowledge. However, they still employ random uniform sampling, but as they decrease the extension of the fault search space, they reach higher precision than the previous tools.
\emphasizedworkname{} employs a novel sampling method based on importance sampling, hence it is capable of achieving similar precision while still requiring no human knowledge and without reducing the fault search space.
\section{Preliminary}

In this section, we first give the problem setup of FL, and then introduce the several terminologies in DP. 

\subsection{Federated Learning}
Consider a general FL system consisting of $M$ clients, in which each client owns its local dataset.
Let $\mathcal D^{\text{train}}_i$, $\mathcal D^{\text{val}}_i$ and $\mathcal D^{\text{test}}_i$ denote the training dataset, validation dataset and testing dataset, held by client $i$, respectively, where $i\in \mathcal{U} = \{1, 2,\ldots, M\}$.
Formally, the FL task is expressed as:
\begin{equation}
\small
\mathbf{w}^{\star} = \mathop{\arg\min}_{\mathbf{w}}\sum_{i\in \mathcal{U}}p_{i}f_i(\mathbf{w}, \mathcal D^{\text{train}}_{i}),
\end{equation}
where $p_{i} = \vert \mathcal D^{\text{train}}_i\vert/\vert \mathcal D^{\text{train}}\vert\geq 0$ with $\sum_{i\in \mathcal{U}}{p_{i}}=1$, and $f_i(\cdot)$ is the local loss function with $f_i(\mathbf{w}) = F_i(\mathbf{w}; \xi_i)$, $\xi_i$ is a batch sample data in client $i$. $\vert \mathcal D^{\text{train}}_{i}\vert $ is the size of training dataset $\mathcal D^{\text{train}}_i$ and $\vert \mathcal D^{\text{train}}\vert = \sum_{i\in \mathcal{U}}{\vert \mathcal D_{i}^{\text{train}}\vert}$ is the total size of training datasets, respectively.
For the $i$-th client, a local model is learned on its private training data $\mathcal D^{\text{train}}$ via:
\begin{equation}
\small
\mathbf{w}_{i}=\mathbf{w}_{i}^{t}-\eta \nabla f_i(\mathbf{w}_{i}, \mathcal D^{\text{train}}_{i}).
\end{equation}
Generally, the local loss function $f_i(\cdot)$ has the same expression across each client.
Then, the $M$ associated clients collaboratively learn a global model $\mathbf{w}$ over the heterogeneous training data $\mathcal D^{\text{train}}_{i}$, $\forall i \in \mathcal{U}$. 


\subsection{Differential Privacy}
Differential Privacy (DP) \cite{Dwork2014the} is a rigorous privacy notion for measuring privacy risk. In this paper, we consider a relaxed version: Rényi DP (RDP)~\cite{mironov2017renyi}.
% \begin{definition}
% ($(\epsilon, \delta)$-DP, \cite{Dwork2014the}). Given privacy parameters $\epsilon > 0$ and $0 \le \delta < 1$, a randomized mechanism $\mathcal{M}$ satisfies $(\epsilon, \delta)$-DP if for any pair of adjacent datasets $\mathcal D$, $\mathcal D^{\prime}$ , and any subset of outputs $O \subseteq range(\mathcal{M})$:
% \begin{equation}
% \operatorname{Pr}[\mathcal{M}(\mathcal D) \in O] \leq e^{\epsilon} \operatorname{Pr}\left[\mathcal{M}\left(\mathcal D^{\prime}\right) \in O\right]+\delta,
% \end{equation}
% where adjacent datasets are constructed by adding or removing any record. In addition,  $(\epsilon, \delta)$-DP is called $\epsilon$-DP, or pure DP when $\delta = 0$.
% \end{definition}
\begin{definition}
(Rényi DP, \cite{mironov2017renyi}). Given a real number $\alpha \in (1, \infty )$ and privacy parameter $\rho \ge 0$, a randomized mechanism $\mathcal{M}$ satisfies $(\alpha, \rho)$-RDP if for any two neighboring datasets $\mathcal D$, $\mathcal D'$ that differ in a single record, the Rényi $\alpha$-divergence between $\mathcal{M}(\mathcal D)$ and $\mathcal{M}(\mathcal D^{\prime})$ satisfies:
\begin{equation}
\small
\!\!\!D_{\alpha}\left[\mathcal{M}(\mathcal D) \| \mathcal{M}\left(\mathcal D^{\prime}\right)\right]\!:=\!\frac{1}{\alpha\!-\!1} \log \mathbb{E}\left[\left(\frac{\mathcal{M}(\mathcal D)}{\mathcal{M}\left(\mathcal D^{\prime}\right)}\right)^{\alpha}\!\right] \!\leq \!\rho,
\end{equation}
where the expectation is taken over the output of $\mathcal{M}(\mathcal D^{\prime})$.
\end{definition}
Rényi DP is a useful analytical tool to measure the privacy and accurately represent guarantees on the tails of the privacy loss, which is strictly stronger than $(\epsilon, \delta)$-DP for $\delta > 0 $. Thus, we provide the privacy analysis based on this tool for each user's privacy loss.
% There are wide range of mechanisms to quantify data privacy such as Laplace Mechanism and Gaussian Mechanism. The privacy loss random variable after composing multiple Gaussian Mechanisms also follows a Gaussian distribution, and this nice featuremakes composition analysis in the training of deep models more friendly. As a result, given the same
% privacy budget, we can achieve tighter privacy guarantee by composing multiple Gaussian Mechanisms in deep learning

\begin{definition}
\text{\rm ($l_2$ Sensitivity).}\label{sensitivity}
\cite[Definition 2]{cheng2022differentially}
Let $\mathcal{F}$ be a function, the $L_{2}$-sensitivity of $\mathcal{F}$ is defined as $\mathcal{S}=\max _{D \simeq D^{\prime} } \| \mathcal{F}\left(D\right)-$ $\mathcal{F}\left(D^{\prime}\right) \|_{2}$, where the maximization is taken over all pairs of adjacent datasets.
\end{definition}

The sensitivity of a function $\mathcal{F}$ captures the magnitude which a single individual’s data can change the function $\mathcal{F}$ in the worst case.
Therefore, it plays a crucial role in determining the magnitude of noise required to ensure DP.

\begin{definition}(Client-level DP) \cite[Definition 1]{mcmahan2017learning}. 
 A randomized algorithm $\mathcal{M}$ is $(\epsilon, \delta)$-DP if for any two adjacent datasets $U$, $U^{\prime}$ constructed by adding or removing all records of any client, and every possible subset of outputs $O$ satisfy the following inequality:
\begin{equation}
\operatorname{Pr}[\mathcal{M}(U) \in O] \leq e^{\epsilon} \operatorname{Pr}\left[\mathcal{M}\left(U^{\prime}\right) \in O\right]+\delta.
\end{equation}
\end{definition}

In client-level DP, we aim to ensure participation information for any clients. Therefore, we need to make local updates similar whether one client participates or not.

% In the client-level DP, we aim to ensure the participation information of any client. Thus, we need to make the local update similar whatever any client participates in or not.

% \ls{at the end of each definition, several discussion are needed. Why do we introduce these definitions.}
\section{Method}\label{sec:method}
\begin{figure*}
    \centering
    \includegraphics[width=\linewidth,keepaspectratio]{figures/pipelines/pipeline_figure_full.pdf}
    %\vspace{-0.7cm}
    \caption[]{
    \OURS{} first generates a set of pseudo masks (top) to initiate self-training (bottom) for unsupervised 3D instance segmentation.
    We leverage features from 3D self-supervised pre-training in combination with 2D self-supervised features on an input mesh.
    These multi-modal features are then aggregated on geometric segment primitives, integrating low- and high-level signals for pseudo mask segmentation.
    These initial pseudo masks are then used as supervision for a 3D transformer-based model to produce updated instance masks that are integrated into the supervision of multiple self-training cycles.
    Finally, we obtain clean and dense instance segmentation without using any manual annotations.
    } 
    \label{fig:full_pipeline}
\end{figure*}

\paragraph{Problem definition}
We propose an unsupervised learning-based method for 3D instance segmentation. Formally, we assume a set of training 3D scenes $\{X_i\}_{i=1}^{n_t}$, represented as meshes, where each scene $X_i$ contains an unknown set of $n_i$ objects. We aim to train a model that can predict for a previously unseen input scene $X$, a set of 3D masks representing the different object instances in that scene. 

\paragraph{Method overview}
We employ a two-stage scheme for unsupervised 3D instance segmentation.
First, we group scene points into contiguous regions based on geometric primitives, and aggregate self-supervised features in these regions to generate an initial set of pseudo masks using Normalized Cut.  This grouping technique enables efficient handling of high-dimensional 3D data.
%
We then follow a series of self-training cycles to refine the pseudo annotations.
An overview of our approach is shown in Figure~\ref{fig:full_pipeline}.

\subsection{Geometric oversegmentation}\label{sec:oversegmentation}
Normalized Cut~\cite{shi2000normalized_cut} (NCut) with deep features has been successfully applied in the 2D domain on dense graphs generated from image patches \cite{wang2022tokencut,lis2022attentropy,wang2023cut}. However, adopting this directly to 3D would be computationally infeasible due to the cubic growth with dimensionality.

We thus propose a solution in the form of geometric oversegmentation through graph coarsening. We begin by creating a graph where each node represents a mesh vertex. Then, we aggregate nodes with similar normal direction and color values and cluster them into contiguous mesh segments using the efficient method proposed by \cite{felzenszwalb2004efficient}. This process reduces the graph size by multiple orders of magnitude. 
%
Our geometry aware segments, as opposed to voxels or points, additionally provide regularization to the subsequent stage of feature aggregation for generating pseudo masks, which we will describe in the following section.

\subsection{Initial pseudo mask generation}\label{sec:dataset_gen}

We first predict an initial set of pseudo masks.  Additional masks will be added during the self-training phase described in Section~\ref{sec:self_train}. 
Thus, we favor generating a reliable set of masks at the cost of restricting to a sparse initial set (i.e., missing potential instances rather than generating noisy masks for them).

\paragraph{Feature aggregation}

We aim to employ a strong set of features for pseudo mask generation and thus consider complementary geometric and color signals from RGB-D scan data.
We leverage geometric 3D self-supervised features from a state-of-the-art 3D pre-training approach, Contrastive Scene Contexts (CSC)~\cite{hou2021exploring}.
We additionally consider 2D self-supervised features from DINO~\cite{caron2021emerging_dino}, extracted from the RGB images and projected to 3D using the corresponding camera poses.
Both the 3D and 2D features are aggregated within each of our geometry-aware segments. 

\paragraph{Masked foreground separation}
We apply NCut to our aggregated features to extract foreground regions $M$ as our initial pseudo masks. 
%
Starting with an empty set $M^0=\{\}$, we iteratively compute the adjacency matrix and retrieve the masks. 
%
That is, we start from $N$ geometric segments with their corresponding $D$-dimensional features $\mathcal{F} \in \mathcal{R}^{N\times D}$, and construct the similarity matrix $A = sim(\mathcal{F})$, where $sim$ denotes cosine similarity. 
Additionally, for the multi-modal setup we calculate similarity matrices $A_{2D}$ and $A_{3D}$ independently and take their weighted average to obtain the final scores. 
Empirically, we found this to be more robust than direct feature fusion of the different modalities, due to their different statistical characteristics.
%
\indent We obtain $W_j$ for Equation~\ref{eq:general_eigenval} by thresholding $A$ at $\tau_{cut}$, where $j$ denotes the $j^{th}$ NCut iteration. 
Using $W_j$, we solve for the second eigenvector $v_j$ and threshold it to retrieve the partition $m_j$. 
We keep all separated foregrounds in $M^0$, where for each upcoming iteration, we mask out the row and column vectors from $W_i$, where $m_i \in M^0$ was already accepted as a foreground instance and $i$ being the segment ids. 
This allows greedy separation of instances in order of confidence  in every cut iteration.
Examples of our generated pseudo masks are visualized in Figures \ref{fig:freemask_vs_ncut} and \ref{fig:self_training_refinement}. \\
%
\indent As the adjacency graph is unaware of the mesh connectivity, NCut often results in masks that span  spatially separated scene regions. 
In 3D, we can leverage knowledge of physical distance to constrain masks to be contiguous in the coarsened scene connectivity graph. We thus filter masks that have separated components, keeping only the ones that contain the item with the maximum absolute value in  $v_j$. Separation is performed before saving $m_j$ into $M^0$, thus allowing for repeated separation of every component. 
We iterate until the maximum number of instances $M^0 = \{m_i\}_{i=1}^{N_m}$ are obtained, or there are no segments left in the scene. 

\subsection{Self-Training}\label{sec:self_train}

Our initial pseudo masks can provide a set of proposed instances $M^0$; however, these pseudo masks are quite sparse in the scenes and sometimes over- or under-split nearby instances.
We thus refine the pseudo mask data through an iterative self-training strategy, producing final instance segmentation predictions $M'$ with more dense and complete instance proposals.

We leverage a state-of-the-art 3D transformer-based backbone~\cite{Schult23mask3d} for our self-training from pseudo mask data as supervision. 
Through multiple training cycles we save the proposals of the $t^{th}$ iteration into $M^{t}$, from the self-trained model, and save these masks as an extension to the original pseudo dataset obtaining $M^t \supseteq M^0$. 
From the second training iteration, we can extract the most confident $K$ predictions and sample these new instance proposals as an addition to the pseudo annotations. 
Further, we only accept new instances if the added information value is larger than a minimum threshold, which we measure by simple segment IoU scores. This way, we can effectively densify the originally sparse annotations, but without limiting the quality of the originally clean pseudo masks. 
%
\paragraph{Loss} \label{par:losses}
We adapt DropLoss \cite{wang2023cut} for our self-training cycles, which is robust to sparse data and missing annotations. 
In particular, we use a weighted combination of cross-entropy and Dice \cite{sudre2017generalised_diceloss} losses for bipartite-matching with pseudo annotations.
We then drop losses for backpropagation which do not have at least $\tau_{drop}$ overlap with the annotations from the previous cycle.

\subsection{Implementation Details}\label{sec:implementation}
%
\paragraph{Backbones.} 
We use a Res16UNet34C sparse-voxel UNet implemented in the MinkowskiEngine~\cite{choy20194d} for 3D pre-trained feature extraction as well as for the 3D transformer during self-training. 

\paragraph{Self-training.} We employ the 3D transformer architecture of \cite{Schult23mask3d}, initialized from scratch. 
The first self-training cycle is trained for 600 epochs with a batch size of 8 until convergence, which takes $\approx 3$ days on a single NVIDIA RTX A6000 GPU. 
Further self-training cycles are all initialized from the previous state and finetuned for an additional 50 epochs in $\approx 4$ hours and for a total of 4 training cycles to produce the final set of instance predictions $S$. 
For the Hungarian assignment, we take the original weighted combination of dice and binary cross-entropy losses and only apply the DropLoss condition in the backpropagation phase.
\section{Theoretical Analysis}\label{th}

In this section, we give a rigorous analysis of DP-FedSAM, including its sensitivity, privacy, and convergence rate. The detailed proof is placed in \textbf{Appendix} \ref{appendix_th}. Below, we first give several necessary assumptions.

\begin{assumption} \label{a1}
(Lipschitz smoothness). The function $F_i$ is differentiable and $\nabla F_i$ is $L$-Lipschitz continuous, $\forall i \in \{1,2,\ldots,M\}$, i.e.,
$\|\nabla F_i({\bf x}) - \nabla F_i({\bf y})\| \leq L \|{\bf x} - {\bf y}\|,$
for all ${\bf x}, {\bf y} \in \mathbb{R}^d$.
% The first-order Lipschitz assumption is commonly used in the ML community. Here, for simplicity, we suppose all functions enjoy the same Lipschitz constant $L$.
\end{assumption}

\begin{assumption} \label{a2}
(Bounded variance). The gradient of the function $f_i$ have $\sigma_l$-bounded variance, i.e.,
$\mathbb{E}_{\xi_i}\left\|\nabla F_i (\mathbf{w}^k(i);\xi_i ) -\nabla F_i (\mathbf{w}(i))\right  \|^2 \leq \sigma_l^2$, $\forall i \in \{1,2,\ldots,M\}, k \in \{1, ..., K-1\},$
the global variance is also bounded, i.e., $\frac{1}{M} \sum_{i=1}^M \|\nabla f_i({\bf w}) - \nabla f({\bf w})\|^2 \leq \sigma_{g}^2$ for all ${\bf w} \in \mathbb{R}^d$. It is not hard to verify that the $\sigma_g$ is smaller than the homogeneity parameter $\beta$, i.e., $\sigma_g^2 \leq \beta^2$.
\end{assumption}

\begin{assumption}\label{a3}
(Bounded gradient). For any $i \!\in\! \{1,2,\ldots,M\}$ and ${\bf w}\!\in\! \mathbb{R}^d$, we have $\|\nabla f_i({\bf w})\|\!\leq\! B $.
\end{assumption}

\begin{assumption}\label{a4}
(Unbiased Gradient Estimator). For any data sample $z$ from $\mathcal{D}_i$ and ${\bf w}\!\in\! \mathbb{R}^d$, the local gradient estimator is unbiased, i.e., $\mathbb{E}[\nabla f_i(\mathbf{w}; z)]= \mathbb{E}[\nabla f_i(\mathbf{w})]$.
\end{assumption}

Note that the above assumptions are mild and commonly used in characterizing the convergence rate of FL \cite{Sun2022Decentralized,shi2023improving,ghadimi2013stochastic,yang2021achieving,bottou2018optimization,reddi2020adaptive, huang2022achieving, Qu2022Generalized, cheng2022differentially,hu2022federated}. 
Furthermore, gradient clipping operation in DL is often used to prevent the gradient explosion phenomenon, thereby the gradient is bounded. 
The technical difficulty for DP-FedSAM lies in: (i) how SAM mitigates the impact of DP; (ii) how to analyze in detail the impacts of the consistency among clients and the on-average norm of local updates caused by clipping operation.  

%Below, we first state the theoretical result of the sensitivity analysis of the local update for each client.

\subsection{Sensitivity Analysis}
At first, we study the sensitivity of local update  $\Delta^t_i$ from any client $i \in \{1,2,..., M\}$ before clipping at $t$-th communication round in DP-FedSAM. This upper bound of sensitivity can roughly measure the degree of privacy protection.
% The operations for DP are mainly clipping of local updates and adding noise \cite{cheng2022differentially}, thus we start by studying the norm of local update $\|\Delta^t_i\|_2$ from any client $i \in \{1,2,...,M\}$ before clipping at $t$-th communication round.
% \begin{theorem}
% \rm ($l_2$ Norm). Under assumptions, $\|\Delta^t_i\|_2$ in our algorithm can be computed as follows:
% \begin{align}
%     \mathbb{E}\|\Delta^t_i\|^2_2  \leq 3\eta^2K(L^2\rho^2+\sigma_g^2+B^2),
% \end{align}
% \end{theorem} 
% Meanwhile, we present the norm of local update with SGD optimizer in DPFL, that is $\mathbb{E}\|\Delta^t_{i, SGD}\|^2_2  \leq3\eta^2K(B^2\sigma_l^2+\sigma_g^2+B^2)$. 
% The detailed proof is presented in Appendix \ref{appendix_th}. 
% \begin{remark}
% we can find the norm in our scheme is smaller than that in baselines. When the perturbation amplitude $\rho$
% proportional to the learning rate, e.g., $\rho = \mathcal{O}(\frac{1}{\sqrt{T}})$, $L^2\rho^2=\frac{L^2}{T}< B^2\sigma_l^2$. Therefore, the norm of local update in DP-FedSAM is smaller than that in baselines, which means that the impact of clipping is alleviated.
% \end{remark}
Under Definition \ref{sensitivity}, the sensitivity can be denoted by $\mathcal{S}_{\Delta_i^t}$ in client $i$ at $t$-th communication round. 
\begin{theorem} (Sensitivity).\label{th:sensitivity}
Denote $\Delta_i^t(\bf x)$ and $\Delta_i^t(\bf y)$ as the local update at $t$-th communication round, the model $\mathbf{x}(i)$ and $\mathbf{y}(i)$ is conducted on two sets which differ at only one sample. Assume the initial model parameter $\mathbf{w}^t(i)=\mathbf{x}^{t, 0}(i) =\mathbf{y}^{t, 0}(i)$. With the above assumptions, the expected squared  
sensitivity $\mathcal{S}^2_{\Delta_i^t}$ of local update is upper bounded,
\begin{align}
\small
    \mathbb{E}\mathcal{S}^2_{\Delta_i^t} \leq
     \frac{6\eta^2\rho^2KL^2(12K^2L^2\eta^2+ 10)}{1-2\eta^2L^2 K}
\end{align}
When the local adaptive learning rate satisfies $\eta=\mathcal{O}({1}/{L\sqrt{KT}})$ and the perturbation amplitude $\rho$
proportional to the learning rate, e.g., $\rho = \mathcal{O}(\frac{1}{\sqrt{T}})$, we have
\begin{align}
\small
    \mathbb{E}\mathcal{S}^2_{\Delta_i^t} \leq
    \mathcal{O}\left(\frac{1}{T^2}\right). 
\end{align}
For comparison, we also present the expected squared sensitivity of local update with SGD in DPFL, that is $ \mathbb{E}\mathcal{S}^2_{\Delta_i^t, SGD} \leq \frac{6\eta^2\sigma_l^2K}{1-3\eta^2KL^2}$.
% \begin{equation} 
% \small
%     \mathbb{E}\mathcal{S}^2_{\Delta_i^t, SGD} \leq \frac{6\eta^2\sigma_l^2K}{1-3\eta^2KL^2}.
% \end{equation}
Thus $\mathbb{E}\mathcal{S}^2_{\Delta_i^t, SGD} \leq \mathcal{O}(\frac{\sigma_l^2}{KL^2T})$ when $\eta=\mathcal{O}({1}/{L\sqrt{KT}})$.
\begin{remark}
It is clearly seen that the upper bound in $  \mathbb{E}\mathcal{S}^2_{\Delta_i^t, SAM}$ is tighter than that in $\mathbb{E}\mathcal{S}^2_{\Delta_i^t, SGD}$. From the perspective of privacy protection, it means DP-FedSAM has a better privacy guarantee than DP-FedAvg.
% \ls{Not SAM. It is DP-SAM better than DP-SGD. Should we mention that the DP-SAM is of independent interest.}.
Another perspective from local iteration, which means both better model consistency among clients and training stability.
\end{remark}
\end{theorem}

%----------------------------------------------------

\subsection{Privacy Analysis}
To achieve client-level privacy protection, we derive the sensitivity of the aggregation process at first after clipping the local updates.
% moments accountants are used to verify that it satisfies client-level DP. 
\begin{lemma}
The sensitivity of client-level DP in DP-FedSAM can be expressed as $C/m$.
\end{lemma}
\begin{proof}
Given two adjacent batches $\mathcal{W}^{t}$ and $\mathcal{W}^{t,\rm{adj}}$, that $\mathcal{W}^{t,\rm{adj}}$ has one more or less client, we have
\begin{equation} \small
\small
\left\Vert \frac{1}{m}\sum_{i\in \mathcal{W}^{t}}\Delta^{t}_{i}-\frac{1}{m}\sum_{j\in \mathcal{W}^{t,\rm{adj}}} \Delta^{t}_{j} \right\Vert_{2} = \frac{1}{m}\left\Vert \Delta^{t}_{j'} \right\Vert_{2}\leq \frac{C}{m},
\end{equation}
where $\Delta^{t}_{j'}$ is the local update of one more or less client.
\end{proof}
\begin{remark}
The value of this sensitivity can determine the amount of variance for adding random noise.
\end{remark}
After adding Gaussian noise, we calculate the accumulative privacy budget~\cite{Yousefpour2021Opacus} along with training as follows. 
\begin{theorem} \label{th:privacy}
After $T$ communication rounds, the accumulative privacy budget is calculated by:
\begin{equation} \small\label{eq:accumulative_eps}
\begin{aligned}
\epsilon = \overline{\epsilon} + \frac{(\alpha-1)\log(1-\frac{1}{\alpha})-\log(\alpha)-\log(\delta)}{\alpha-1},
\end{aligned}
\end{equation}
where
\begin{equation} \small
\begin{aligned}
\overline{\epsilon} &= \frac{T}{\alpha-1}\ln {\mathbb{E}_{z\sim \mu_{0}(z)}\left[\left(1-q+\frac{q \mu_{1}(z)}{\mu_{0}(z)}\right)^{\alpha}\right]},
\end{aligned}
\end{equation}
and $q$ is sample rate for client selection, $\mu_{0}(z)$ and $\mu_{1}(z)$ denote the Gaussian probability density function (PDF) of $\mathcal{N}(0,\sigma)$ and the mixture of two Gaussian distributions $q\mathcal{N}(1,\sigma)+(1-q)\mathcal{N}(0,\sigma)$, respectively, $\sigma$ is the noise STD, $\alpha$ is a selectable variable.
\end{theorem}
\begin{remark}
It can be seen that a small sampling rate $q$ can enhance the privacy guarantee by decreasing the privacy budget, but it may also degrade the training performance due to the number of participating clients being reduced in each communication round. So a better trade-off is needed.
\end{remark}
% \ls{it seems that the results in this subsection are unrelated to the DP-FedSAM.}
% \ls{some results are merely intermediate results. It is better to use "Lemma" or "Proposition"}

%----------------------------------------------------
\subsection{Convergence Analysis}
Below, we give a convergence analysis of how DP-FedSAM mitigates the negative impacts of DP. The technical contribution also is combining the impacts of the on-average norm of local updates $\overline{\alpha}^{t}$ and local update consistency among clients $\tilde{\alpha}^t$ on the rate. Moreover, we also empirically confirm these results in Section \ref{exper_DP}.

\begin{theorem}\label{th:conver}
Under assumptions 1-4, local learning rate satisfies $\eta=\mathcal{O}({1}/{L\sqrt{KT}})$ and $f^{*}$ is denoted as the minimal value of $f$, i.e., $f(x)\ge f(x^*)=f^*$ for all $x\in \mathbb{R}^{d}$. 
When the perturbation amplitude $\rho$ is
proportional to the learning rate, e.g., $\rho = \mathcal{O}(1/\sqrt{T})$,
the sequence of outputs $\{\mathbf{w}^t\}$ generated by Alg. \ref{DFedAvg_DP}, we have:
\begin{equation*}
\small
\begin{split}
\small
    & \frac{1}{T} \sum_{t=1}^T
    \mathbb{E}\left[\overline{\alpha}^{t}\left\|\nabla f\left(\mathbf{w}^{t}\right)\right\|^{2}\right]   \leq  
    \underbrace{\mathcal{O}\left(\frac{2L(f({\bf w}^{1})-f^{*})}{\sqrt{KT}} + \frac{ L^2\sigma_{l}^2}{KT^2}\right)}_{\text{From FedSAM}} \\ 
    &\qquad\quad +
    \underbrace{
     \underbrace{\mathcal{O}\left( \frac{\sum_{t=1}^T(\overline{\alpha}^t  \sigma_{g}^2 + \tilde{\alpha}^t  L^2  )}{T^2}  \right)}_{\text{Clipping}}
    + \underbrace{ \mathcal{O}\left(\frac{L^2 \sqrt{T}\sigma^2C^2d}{m^2\sqrt{K}} \right)}_{\text{Adding noise}}
    }_{\text{From operations for DP}} 
    \end{split}
\end{equation*}
where
\begin{equation} \small
\begin{split}
    \overline{\alpha}^{t} :=\frac{1}{M} \sum_{i=1}^{M} \alpha^t_i~~~ \text{and}  ~~~ \tilde{\alpha}^t :=\frac{1}{M}\sum_{i=1}^{M} |\alpha_i^t - \overline{\alpha_i^t}|, 
\end{split}
\end{equation}
with $\alpha^t_i = \min (1, \frac{C}{ \eta  \|  \sum_{k=0}^{K-1}  \tilde{\mathbf{g}}^{t,k}(i) \|} ) $, respectively. Note that $\overline{\alpha}^{t}$ and $\tilde{\alpha}^t$ measure the on-average norm of local updates and local update consistency among clients before clipping and adding noise operations in DP-FedSAM, respectively.
\end{theorem}
\begin{table*}
\caption{Averaged training accuracy (\%) and testing accuracy (\%) on two data in both IID and Non-IID settings for all compared methods.}
\vspace{-0.2cm}
\centering
% \small
\scriptsize
\renewcommand{\arraystretch}{0.5}
\label{table:all_baselines}
\resizebox{0.9\linewidth}{!}{
\begin{tabular}{cccccccc} 
\toprule
\multirow{2}{*}{\textbf{Task}} & \multirow{2}{*}{\textbf{Algorithm}} & \multicolumn{2}{c}{Dirichlet~0.3} & \multicolumn{2}{c}{Dirichlet~0.6} & \multicolumn{2}{c}{IID}  \\ 
\cmidrule{3-8}
                      &                            & Train         & Validation          & Train         & Validation          & Train      & Validation           \\ 
\midrule 
       & DP-FedAvg       & 99.28$\pm$0.02 & 73.10$\pm$0.16          & 99.55$\pm$0.02 & 82.20$\pm$0.35          & 99.66$\pm$0.40 & 81.90$\pm$0.86           \\
       & Fed-SMP-$\randk_k$ & 99.24$\pm$0.02 & 73.72$\pm$0.53          & 99.71$\pm$0.01 & 82.18$\pm$0.73          & 99.71$\pm$0.61 & 84.16$\pm$0.83           \\
       & Fed-SMP-$\topk_k $    & 99.31$\pm$0.04 & 75.75$\pm$0.35          & 99.72$\pm$0.02 & 83.41$\pm$0.91          & 99.73$\pm$0.40 & 83.32$\pm$0.52           \\
EMNIST & DP-FedAvg-$\blur$      & 99.12$\pm$0.02 & 73.71$\pm$0.02          & 99.66$\pm$0.00 & 83.20$\pm$0.01          & 99.67$\pm$0.03 & 82.92$\pm$0.49           \\
       & DP-FedAvg-$\blurs$     & 99.63$\pm$0.08 & 76.25$\pm$0.35          & 99.72$\pm$0.02 & 83.41$\pm$0.91          & 99.74$\pm$0.45 & 82.92$\pm$0.49           \\
       & DP-FedSAM       & 96.28$\pm$0.64 & 76.81$\pm$0.81          & 95.07$\pm$0.45 & 84.32$\pm$0.19 & 95.61$\pm$0.94 & 85.90$\pm$0.72           \\
       & DP-FedSAM-$\topk_k $  & 94.77$\pm$0.11 & \textbf{77.27$\pm$0.67} & 95.87$\pm$1.52 & \textbf{84.80$\pm$0.60 }         & 96.12$\pm$0.85 & \textbf{87.70$\pm$0.83}  \\
\midrule
                      & DP-FedAvg                  & 93.65$\pm$0.47    & 47.98$\pm$0.24          & 93.65$\pm$0.42    & 50.05$\pm$0.47          & 93.65$\pm$0.15 & 50.90$\pm$0.86           \\
                      & Fed-SMP-$\randk_k$              & 95.46$\pm$0.43    & 48.14$\pm$0.12          & 95.36$\pm$0.06    & 51.33$\pm$0.36          & 95.36$\pm$0.06 & 50.61$\pm$0.20           \\
                      & Fed-SMP-$\topk_k $           & 95.49$\pm$0.14    & 49.93$\pm$2.29          & 95.49$\pm$0.09    & 54.11$\pm$0.83          & 95.49$\pm$0.10 & 53.30$\pm$0.45           \\
CIFAR-10              & DP-FedAvg-$\blur$              & 95.47$\pm$0.12    & 47.66$\pm$0.01          & 99.66$\pm$0.42    & 51.05$\pm$0.01          & 94.50$\pm$0.05 & 52.56$\pm$0.47           \\
                      & DP-FedAvg-$\blurs$          & 96.79$\pm$0.51    & 51.23$\pm$0.66          & 99.72$\pm$0.09    & 54.11$\pm$0.83          & 96.45$\pm$0.30 & 53.48$\pm$0.76           \\
                      & DP-FedSAM                  & 90.38$\pm$0.90    & 53.92$\pm$0.55          & 90.83$\pm$0.15    & 54.14$\pm$0.60          & 90.83$\pm$0.16 & 55.58$\pm$0.50           \\
                      & DP-FedSAM-$\topk_k $            & 93.25$\pm$0.60    & \textbf{54.85$\pm$0.86} & 92.60$\pm$0.65    & \textbf{57.00$\pm$0.69} & 91.52$\pm$0.11 & \textbf{58.82$\pm$0.51}  \\
\bottomrule
\end{tabular}}
\vspace{-0.2cm}
\end{table*}


\begin{remark}
The proposed DP-FedSAM can achieve a tighter bound in general non-convex setting compared with previous work \cite{cheng2022differentially, hu2022federated} ($\small \mathcal{O}\left( \frac{1}{\sqrt{KT}} + \frac{6K\sigma_{g}^2 + \sigma_l^2}{T} + \frac{B^2\sum_{t=1}^T(\overline{\alpha}^t + \tilde{\alpha}^t)}{T}+\frac{L^2 \sqrt{T}\sigma^2C^2d}{m^2\sqrt{K}}\right)$ in \cite{cheng2022differentially} and $\small \mathcal{O}\left( \frac{1}{\sqrt{KT}} + \frac{3\sigma_{g}^2 + 2\sigma_l^2}{\sqrt{KT}} +\frac{4L^2 \sqrt{T}\sigma^2C^2d}{m^2\sqrt{K}}\right)$ in \cite{hu2022federated}) and reduce the impacts of the local and global variance $\sigma_l^2$, $\sigma_g^2$. 
Meanwhile, 
we are the first to theoretically analyze these two impacts of both the on-average norm of local updates $\overline{\alpha}^t$ and local update inconsistency among clients $\tilde{\alpha}^t$ on convergence. And the negative impacts of $\overline{\alpha}^t$ and $\tilde{\alpha}^t$ are also significantly mitigated upon convergence compared with previous work \cite{cheng2022differentially} due to local SAM optimizer being adopted.
It means that we effectively alleviate performance degradation caused by clipping operation in DP and achieve better performance under symmetric noise. This theoretical result has also been empirically verified on several real-world data (see Section \ref{eva} and \ref{exper_DP}).

\end{remark}

\section{Experiments}
In this section, we conduct extensive experiments to verify the effectiveness of DP-FedSAM. 

\subsection{Experiment Setup}

\noindent
\textbf{Dataset and Data Partition.}\  
The efficacy of DP-FedSAM is evaluated on three datasets, including \textbf{EMNIST} \cite{cohen2017emnist}, \textbf{CIFAR-10} and \textbf{CIFAR-100} \cite{krizhevsky2009learning}, in both IID and Non-IID settings. Specifically, Dir Partition \cite{hsu2019measuring} is used for simulating Non-IID across federated clients, where the local data of each client is partitioned by splitting the total dataset through sampling the label ratios from the Dirichlet distribution Dir($\alpha$) with parameters $\alpha=0.3$ and $\alpha=0.6$. 


\noindent
\textbf{Baselines.} We focus on DPFL methods that ensure client-level DP. Thus we compare DP-FedSAM with existing DPFL baselines: \textbf{DP-FedAvg} \cite{McMahan2018learning} ensures client-level DP guarantee by directly employing Gaussian mechanism to the local updates. \textbf{DP-FedAvg-blur} \cite{cheng2022differentially} adds regularization method (BLUR) based on DP-FedAvg. \textbf{DP-FedAvg-blurs} \cite{cheng2022differentially} uses local update sparsification (LUS) and BLUR for improving the performance of DP-FedAvg.
\textbf{Fed-SMP-$\randk_k$} and \textbf{Fed-SMP-$\topk_k$} \cite{hu2022federated} leverage random sparsification and $\topk_k$ sparsification technique for reducing the impact of DP noise on model accuracy, respectively.

\noindent
\textbf{Configuration.}
For EMNIST, we use a simple CNN model and train $200$ communication round. For CIFAR-10 and CIFAR-100 datasets, we use the ResNet-18 \cite{he2016deep} backbone and train $300$ communication round. 
For all experiments, we set the number of clients $M$ to $500$. The default sample ratio $q$ of the client is $0.1$. The local learning rate $\eta$ is set to 0.1 with a decay rate $0.005$ and momentum $0.5$, and the number of training epochs is $30$. For privacy parameters, noise multiplier $\sigma$ is set to $0.95$ and  the privacy failure probability $\delta=\frac{1}{M}$. The clipping threshold $C$ is selected by grid search from set $\{0.1,0.2,0.4,0.6,0.8\}$, and we find that the gradient explosion phenomenon will occur when $C \ge 0.6$ on EMNIST and $C=0.2$ performs better on three datasets.  The weight perturbation ratio is set to $\rho = 0.5$. We run each experiment $3$ trials and report the best-averaged testing accuracy in each experiment.

The results on EMNIST and CIFAR-10 datasets are placed in the main paper, while the results on  CIFAR-100 are placed in \textbf{Appendix} \ref{imp_detail}.
For a more comprehensive and fair comparison, we integrate DP-FedSAM with $\topk_k$ sparsification technique \cite{hu2022federated, cheng2022differentially}, named DP-FedSAM-$\topk_k$ (More discussion in \textbf{Appendix} \ref{DP-sam-topk}), to compare with baselines that also use local sparsification technique, such as Fed-SMP-$\randk_k$, Fed-SMP-$\topk_k$, and DP-FedAvg-$\blurs$.


\begin{figure*}[t]
\centering
    \begin{subfigure}{1\linewidth}
    \centering
        \includegraphics[width=0.95\textwidth]{figures/emnist.pdf}
        \caption{EMNIST}
        \label{fig:emnist}
    \end{subfigure}
    % \hfill
    % \quad
    \begin{subfigure}{1\linewidth}
    \centering
        \includegraphics[width=0.95\textwidth]{figures/cifar10.pdf}
        \caption{CIFAR-10}
        \label{fig:cifar10}
    \end{subfigure}
    \vspace{-0.55cm}
    \caption{\small The averaged testing accuracy on \textit{EMNIST} and \textit{CIFAR-10} dataset under symmetric noise for all compared methods. }
 \label{fig:all}
\vspace{-0.6cm}
\end{figure*}
% FedAvg, DP-FedAvg, DP-FedAvg-$\blurs$, Fed-SMP-$\topk_k$ and Fed-SMP-$\randk_k$.
% DP-FedSAM

\subsection{Experiment Evaluation}\label{eva}
% \begin{figure*}[t]
%     \centering
%         \includegraphics[width=1\textwidth]{figures/emnist.pdf}
%     \caption{Averaged testing accuracy on \textit{EMNIST} dataset under symmetric noise for all compared methods. }
%     \label{emnist}
% \end{figure*}



% \usepackage{multirow}
% \usepackage{booktabs}







\noindent
\textbf{Performance with compared baselines.}
In Table \ref{table:all_baselines} and Figure \ref{fig:all}, we evaluate DP-FedSAM and DP-FedSAM-$\topk_k$ on EMNIST and CIFAR-10 datasets in both settings compared with all baselines from DP-FedAvg to DP-FedAvg-$\blurs$. The baseline methods seem to be overfitting in Table \ref{table:all_baselines}, especially on more complex data (e.g., CIFAR-100 in Table \ref{table:cifar100_all_baselines}). The main reasons are the random noise introduced by DP and the over-fitting and inconsistency of the local models. From all these results, it is seen that our proposed algorithms outperform other baselines under symmetric noise both on accuracy and generalization perspectives.
It means that we significantly improve the performance and generate a better trade-off between performance and privacy in DPFL.
% Specifically, in Table 1, we can see that the performance improvement is more obvious than all other baselines on EMNIST and CIFAR-10 with the same communication round.  
For instance, the averaged testing accuracy is $85.90\%$ in DP-FedSAM and $87.70\%$ in DP-FedSAM-$\topk_k$ on EMNIST in the IID setting, which are better than other baselines. Meanwhile, the difference between training accuracy and test accuracy is $9.71\%$ in DP-FedSAM and $8.40\%$ in DP-FedSAM-$\topk_k$, while that is $17.74\%$ in DP-FedAvg and $16.41\%$ in Fed-SMP-$\topk_k$, respectively. That means our algorithms significantly mitigate the performance degradation issue caused by DP.


\noindent
\textbf{Impact of Non-IID levels.}
In the experiments under different participation cases as shown in Table \ref{table:all_baselines}, we further prove the robust generalization of the proposed algorithms. Heterogeneous data distribution of local clients is set to various participation levels from IID, Dirichlet 0.6, and Dirichlet 0.3, which makes the training of the global model more difficult. On EMNIST, as the Non-IID level decreases, DP-FedSAM achieves better generalization than DP-FedAvg, and the difference between training accuracy and test accuracy in DP-FedSAM $\{19.47\%, 10.75\%, 9.71\%\}$ are lower than that in DP-FedAvg $\{26.18\%, 17.35\%, 17.74\%\}$. Similarly, the difference values in DP-FedSAM-$\topk_k$ $\{17.50\%, 11.07\%, 8.40\%\}$ are also lower than that in Fed-SMP-$\topk_k$ $\{23.56\%, 16.31\%, 16.41\%\}$. These observations confirm that our algorithms are more robust than baselines in various degrees of heterogeneous data.
\subsection{Discussion for DP with SAM in FL}\label{exper_DP}
In this subsection, we empirically discuss how SAM mitigates the negative impacts of DP from the aspects of the norm of local update and the visualization of loss landscape and contour. Meanwhile, we also observe the training performance with SAM under different privacy budgets $\epsilon$ compared with all baselines. These experiments are conducted on CIFAR-10 with ResNet-18 \cite{he2016deep} and Dirichlet $\alpha=0.6$.
% And we conduct experiments on CIFAR-10 with ResNet-18 \cite{he2016deep} and Dirichlet $\alpha=0.6$.

\begin{figure}
\centering
\includegraphics[width=0.48\textwidth]{figures/norm1.pdf}
\vspace{-0.4cm}
\caption{\small Norm distribution and average norm of local updates.
}
\vspace{-0.4cm}
\label{fig:norm}
\end{figure}
\begin{figure*}[ht]
\centering
\includegraphics[width=0.95\textwidth]{figures/abla.pdf}
\vspace{-0.35cm}
\caption{\small Impact of hyper-parameters: perturbation radius $\rho$, local iteration steps $K$, total clients size $M$.}
\vspace{-0.35cm}
\label{fig:abla}
\end{figure*}
% \begin{figure}
% \centering
% \begin{subfigure}{0.57\linewidth}
% \centering
% \includegraphics[width=1\textwidth]{figures/sam_landscape.pdf}
% \caption{\small Loss landscape}
% \end{subfigure}
% % \quad
% \hfill
% \begin{subfigure}{0.41\linewidth}
% \centering
% \includegraphics[width=1\textwidth]{figures/contour_c_comparsion.pdf}
% \caption{\small Loss surface contour}
% \end{subfigure}
% % \begin{subfigure}{0.45\linewidth}
% % \centering
% % \includegraphics[width=1\textwidth]{figures/norm1.pdf}
% % \caption{\small Norm distribution and average norm of local updates.
% % }
% % \end{subfigure}
% \vspace{-0.2cm}
% \caption{\small Loss landscape and surface contour of DP-FedSAM. Compared with DP-FedAvg in the left of Figure \ref{landscape_fedavg_dpfedavg} (a) and (b) with the same setting, DP-FedSAM has a flatter landscape with both better generalization ability (flat minima, see Figure \ref{sam_land} (a)) and weight perturbation robustness (see Figure \ref{sam_land} (b)).
% }
% \vspace{-0.2cm}
% \label{sam_land}
% \end{figure}


\noindent
\textbf{The norm of local update.}
To validate the theoretical results for mitigating the adverse impacts of the norm of local updates, we conduct experiments on DP-FedSAM and DP-FedAvg with clipping threshold $C=0.2$ as shown in Figure \ref{fig:norm}. We show the norm $\Delta_i^t$ distribution and average norm $\overline{\Delta}^t$ of local updates before clipping during the communication rounds. In contrast to DP-FedAvg, most of the norm is distributed at the smaller value in our scheme in Figure \ref{fig:norm} (a), which means that the clipping operation drops less information. Meanwhile, the on-average norm $\overline{\Delta}^t$ is smaller than DP-FedAvg as shown in Figure \ref{fig:norm} (b). These observations are also consistent with our theoretical results in Section \ref{th}.




\begin{table}
\centering
\caption{Performance comparison under different privacy budgets.}
\vspace{-0.2cm}
\label{privacy}
\small 
\renewcommand{\arraystretch}{0.9}
\resizebox{\linewidth}{!}{
\begin{tabular}{ccccc} 
\toprule
\multirow{2}{*}{Algorithm} & \multicolumn{4}{c}{Averaged test accuracy (\%) under different privacy budgets $\epsilon$}                                                                                                            \\ 
\cmidrule{2-5}
                           &  $\epsilon$ = 4                      & $\epsilon$ = 6                                  &$\epsilon$ =  8                                  & $\epsilon$ = 10                                  \\ 
\midrule
DP-FedAvg                  & 38.23 $\pm$
  0.15 & 43.87 $\pm$ 0.62 & 46.74 $\pm$ 0.03 & 49.06 $\pm$ 0.49  \\
Fed-SMP-$\randk_k$              & 33.78 $\pm$ 0.92    & 42.21 $\pm$ 0.21 & 48.20 $\pm$ 0.05 & 50.62 $\pm$ 0.14  \\
Fed-SMP-$\topk_k$               & 38.99 $\pm$0.50     & 46.24 $\pm$ 0.80 & 49.78 $\pm$ 0.78 & 52.51 $\pm$ 0.83  \\
DP-FedAvg-$\blur $            & 38.23 $\pm$ 0.70    & 43.93 $\pm$ 0.48 & 46.74 $\pm$ 0.92 & 49.06 $\pm$ 0.13  \\
DP-FedAvg-$\blurs $             & 39.39 $\pm$0.43     & 46.64 $\pm$ 0.36  & 50.18 $\pm$ 0.27 & 52.91 $\pm$ 0.57  \\
DP-FedSAM                  & \textbf{39.89$\pm$ 0.17}   & 47.92 $\pm$ 0.23 & 51.30 $\pm$ 0.95 & 53.18 $\pm$ 0.40  \\
DP-FedSAM-$\topk_k$              & 38.96 $\pm$ 0.61    & \textbf{49.17 $\pm$ 0.15} & \textbf{53.64 $\pm$ 0.12} & \textbf{56.36 $\pm$ 0.36}  \\
\bottomrule
\end{tabular}
}
\vspace{-0.45cm}
\end{table}

\noindent
\textbf{Performance under different privacy budgets $\epsilon$.}
Table \ref{privacy} shows the test accuracies for the different level privacy guarantees. Note that $\epsilon$ is not a hyper-parameter, but can be obtained by the privacy design in each round such as Eq. (\ref{eq:accumulative_eps}).
Our methods consistently outperform the previous SOTA methods under various privacy budgets $\epsilon$.
Specifically, DP-FedSAM and DP-FedSAM-$\topk_k$ significantly improve the accuracy of DP-FedAvg and Fed-SMP-$\topk_k$ by $1\% \sim 4\%$ and $3\% \sim 4\%$ under the same $\epsilon$, respectively. 
Furthermore, the test accuracy can improve as the privacy budget $\epsilon$ increases, which suggests us a better balance is necessary between training performance and privacy. There is a better trade-off when achieving better accuracy under the same $\epsilon$ or a smaller $\epsilon$ under approximately the same accuracy.


\noindent
\textbf{Loss landscape and contour.}
The discussion and visualizing results are placed in \textbf{Appendix} \ref{exper_DP_appendix} due to limited space.
% To visualize the sharpness of the flat minima and observe robustness to DP noise obtained by DP-FedSAM, we show the loss landscape and surface contour following by the plotting method \cite{li2018visualizing} in Figure \ref{sam_land}. It is clear that DP-FedSAM has flatter minima and better robustness to DP noise than DP-FedAvg in the left of Figure \ref{landscape_fedavg_dpfedavg} (a) and (b), respectively. It indicates that our proposed algorithm achieves better generalization and makes the training process more adaptive to the DPFL setting.
% thereby getting a better balance between performance and privacy.

\subsection{Ablation Study}
We verify the influence of hyper-parameters in DP-FedSAM on EMNIST with Dirichlet partition  $\alpha=0.6$. 

\noindent
\textbf{Perturbation weight $\rho$.}
Perturbation weight $\rho$ has an impact on performance as the adding perturbation is accumulated when the communication round $T$ increases. 
% It is a trade-off between performance and generalization. 
To select a proper value for our algorithms, we conduct some experiments on various perturbation radius from the set $\{ 0.01, 0.1, 0.3, 0.5, 0.7, 1.0\}$ in Figure \ref{fig:abla} (a). As $\rho = 0.5$, we achieve better convergence and performance. 
% Meanwhile, $\rho = \mathcal{O}(\frac{1}{\sqrt{T}})$ can make a speedup on convergence.

\noindent
\textbf{Local iteration steps $K$.}
Large local iteration steps $K$ can help the convergence in previous DPFL work \cite{cheng2022differentially} with the theoretical guarantees. To investigate the acceleration on $T$ by adopting a larger $K$, we fix the total batchsize and change local training epochs. In Figure \ref{fig:abla} (b), our algorithm can accelerate the convergence in Theorem \ref{th:conver} as a larger $K$ is adopted, that is, use a larger epoch value. However, the adverse impact of clipping on training increases as $K$ is too large, e.g., epoch 
= $40$. Thus we choose $30$ local epoch.

\noindent
\textbf{Client size $M$.}
We compare the performance between different numbers of client participation $m=\{ 100, 200, 300, 500, 700\}$ with the same hyper-parameters in Figure \ref{fig:abla} (c). Compared with larger $m=500$, the smaller $m$ may have worse performance due to large variance $\sigma^2 C^2/m$ in DP noise with the same setting. Meanwhile, when $m$ is too large such as $M=700$, the performance may degrade as the local data size decreases. 
\begin{table}
\caption{The averaged training accuracy and testing accuracy.}
\vspace{-0.2cm}
\label{ab:sam}
\small 
\centering
\renewcommand{\arraystretch}{0.9}
\resizebox{\linewidth}{!}{
\begin{tabular}{cccc}
\toprule
Algorithm  & Train  (\%)& Validation (\%) & Differential value (\%)  \\
\midrule
DP-FedAvg  &    99.55$\pm$0.02   &       82.20$\pm$ 0.35        &   17.35$\pm$0.32              \\
DP-FedSAM  &    95.07$\pm$0.45   &    84.32$\pm$0.19 $\uparrow$        &     10.75$\pm$0.26              $\downarrow$  \\
Fed-SMP-$\topk_k$   &   99.72$\pm$0.02    &     83.41  $\pm$ 0.91     &        16.31$\pm$ 0.89            \\
DP-FedSAM-$\topk_k$ & 95.87$\pm$0.52        &     84.80$\pm$0.60  $\uparrow$      & 11.07$\pm$0.08  $\downarrow$  \\   
\bottomrule
\end{tabular}}
\vspace{-0.4cm}
\end{table}

\noindent
\textbf{Effect of SAM.}
% 
As shown in Table \ref{ab:sam}, it is seen that DP-FedSAM and DP-FedSAM-$\topk_k$ can achieve performance improvement and better generalization compared with DP-FedAvg and Fed-SMP-$\topk_k$ as SAM optimizer is adopted with the same setting, respectively.

% \vspace{-0.25cm}
 \section{Conclusion}
 In this paper, we have presented a tactile manipulation system that is able to rotate different objects without vision. We showed an end-to-end reinforcement learning framework to learn tactile dexterity over the proposed system. We carried out experiments both in simulation and real to demonstrate its effectiveness. Our work demonstrated that we are able to achieve tactile dexterity as humans in real for the first time. In the future, there are many promising future directions to investigate, such as exploring the use of a more dense contact sensor array and scaling up the system to solve more diverse tasks. We hope that our work can pave the way for more intelligent robot hands.

% \clearpage
{
\small
\bibliographystyle{ieee_fullname}
\bibliography{ref}
}

\clearpage
\section{Appendix for Proofs}

\paragraph{Proof of Theorem \ref{thm:main}.}

\begin{proof}
\label{proof:main}
Our proof has two steps. In Step 1, we will show that SimCLR is equivalent to minimizing the cross entropy loss defined in Eqn.~(\ref{eqn:cross-entropy}). 
In Step 2, we will show  that minimizing the cross-entropy loss 
is equivalent to spectral clustering on $\bfpi$. 
Combining the two steps together, we have proved our theorem. 

\textbf{Step 1: } SimCLR is equivalent to minimizing the cross entropy loss.

The cross-entropy loss takes expectation over 
$\bfW_\bfX\sim \mathbb{P}(\cdot ; \bfpi)$, 
which means $\bfW_\bfX$ has exactly one non-zero entry in each row $i$. By Lemma~\ref{lem:multinomial}, we know every row $i$ of $\bfW_\bfX$ is independent of other rows. Moreover, 
$\bfW_{\bfX,i}\sim \mathcal{M}(1, \bfpi_i/\sum_j \bfpi_{i,j})=\mathcal{M}(1, \bfpi_i)$, because $\bfpi_i$ itself is a probability distribution.
Similarly, we know $\bfW_\bfZ$ also has the row-independent property by sampling over $\mathbb{P}(\cdot;\bfK_\bfZ)$.
Therefore, by Lemma~\ref{lem:cross_split}, we know Eqn.~(\ref{eqn:cross-entropy}) is equivalent to:
\[
 -\sum_{i=1}^n \mathbb{E}_{\bfW_{\bfX,i}}[\log \mathbb{P}(\bfW_{\bfZ,i}=\bfW_{\bfX,i};\bfK_\bfZ)],
\]

This expression takes expectation over $\bfW_{\bfX,i}$ for the given row $i$. Notice that 
$\bfW_{\bfX,i}$ has exactly one non-zero entry, which equals $1$ (same for $\bfW_{\bfZ,i}$). 
As a result
we expand the above expression to be:
\begin{equation}
 -\sum_{i=1}^n \sum_{j\neq i} \Pr(\bfW_{\bfX,i,j}=1)\log \Pr(\bfW_{\bfZ,i,j}=1).
\label{eqn:detailed-expansion}    
\end{equation}


By Lemma~\ref{lem:multinomial}, $\Pr(\bfW_{\bfZ,i,j}=1)=\bfK_{\bfZ,i,j}/\|\bfK_{\bfZ,i}\|_1$ for $j\neq i$. Recall that $\bfK_\bfZ=(k(\bfZ_i-\bfZ_j))_{(i,j)\in[n]^2}$, which means 
$\bfK_{\bfZ,i,j}/\|\bfK_{\bfZ,i}\|_1=\frac{\exp(-\|\bfZ_i-\bfZ_j\|^2/{2\tau})}{\sum_{k\neq i}
\exp(-\|\bfZ_i-\bfZ_k\|^2/{2\tau})
}$ for $j\neq i$, when $k$ is the Gaussian kernel with variance $\tau$. 

Notice that $\bfZ_i=f(\bfX_i)$, so we know
\begin{equation}
-\log \Pr(\bfW_{\bfZ,i,j}=1)=
-\log \frac{\exp(-\|f(\bfX_i)-f(\bfX_j)\|^2/{2\tau})}{\sum_{k\neq i}
\exp(-\|f(\bfX_i)-f(\bfX_k)\|^2/{2\tau}),
}
\label{eqn:infonce-equivalence}    
\end{equation}


The right hand side is exactly the InfoNCE loss defined in Eqn.~(\ref{eqn:infonce}).
Inserting Eqn.~(\ref{eqn:infonce-equivalence}) into Eqn.~(\ref{eqn:detailed-expansion}), we get the SimCLR algorithm, which first samples augmentation pairs $(i,j)$ with $\Pr(\bfW_{\bfX,i,j}=1)$ for each row $i$, and then optimize the InfoNCE loss. 

\textbf{Step 2: } minimizing the cross entropy loss 
is equivalent to spectral clustering on $\bfpi$.


By Lemma~\ref{lem:convert_to_spectral}, we may further convert the loss to 
\begin{equation}
\label{eqn:main-theorem-repul-attr}
\min_{\bfZ}
-\sum_{(i,j)\in [n]^2} \mathbf{P}_{i,j}
\log k (\bfZ_i-\bfZ_j)+\log \mathbf{R}(\bfZ).
\end{equation}
Since $k$ is the Gaussian kernel, this reduces to \[
\min_\bfZ \mathrm{tr}(\bfZ^\top \mathbf{L}(\bfpi) \bfZ)
+\log \mathbf{R}(\bfZ),
\]

where we use the fact that $\mathbb{E}_{\bfW_\bfX\sim \mathbb{P}(\cdot; \bfpi)}[\mathbf{L}(\bfW_\bfX)]
=\mathbf{L}(\bfpi)
$, because the Laplacian operator is linear and $
\mathbb{E}_{\bfW_\bfX\sim \mathbb{P}(\cdot; \bfpi)}(\bfW_\bfX)=\bfpi
$.
\end{proof}

\paragraph{Proof of Theorem \ref{thm:clip}.}
\begin{proof}
Since $\bfW_\bfX\sim \mathbb{P}(\cdot;\bfpi_{\mathbf{A}, \mathbf{B}})$, we know 
$\bfW_\bfX$ has exactly one non-zero entry in each row, denoting the pair that got sampled. 
A notable difference compared to the previous proof is we now have $n_\mathcal{A}+n_\mathcal{B}$ objects in our graph. CLIP deals with this by taking a mini-batch of size $2N$, 
such that $n_\mathcal{A}=n_\mathcal{B}=N$, and adding the $2N$ InfoNCE losses together. We label the objects in $\mathcal{A}$ as $[n_\mathcal{A}]$, and the objects in $\mathcal{B}$ as $\{n_\mathcal{A}+1, \cdots, n_\mathcal{A}+n_\mathcal{B}\}$. 

Notice that $\bfpi_{\mathbf{A}, \mathbf{B}}$ is a bipartite graph, so the edges of objects in $\mathcal{A}$ will only connect to object in $\mathcal{B}$ and vice versa. We can define the similarity matrix in $\cZ$ as $\bfK_\bfZ$, 
where $\bfK_\bfZ(i, j+n_\mathcal{A})=\bfK_\bfZ(j+n_\mathcal{A},i)= k(\bfZ_i-\bfZ_j)$ for $i\in [n_\mathcal{A}], j\in [n_\mathcal{B}]$, and otherwise we set $\bfK_\bfZ(i,j)=0$. 
The rest is same as the previous proof. 
\end{proof}

\paragraph{Proof of Theorem \ref{thm:exponential}.}

\begin{proof}
\label{proof:exponential}
Since the objective function consists of a linear term combined with an entropy regularization, which is a strongly concave function, the maximization problem is a convex optimization problem. Owing to the implicit constraints provided by the entropy function, the problem is equivalent to having only the equality constraint. We then introduce the Lagrangian multiplier $\lambda$ and obtain the following relaxed problem:

$$
\widetilde{E}(\boldsymbol{\alpha})=\psi_{1}-\sum_{i=1}^n \alpha_{i} \psi_{i}+\tau \sum_{i=1}^n \alpha_{i}\log \alpha_{i}+\lambda\left(\boldsymbol{\alpha}^{\top} \mathbf{1}_n-1\right).
$$

As the relaxed problem is unconstrained, taking the derivative with respect to $\alpha_{i}$ yields

$$
\frac{\partial \widetilde{E}(\boldsymbol{\alpha})}{\partial \alpha_{i}}=-\psi_{i}+\tau\left(\log \alpha_{i}+\alpha_{i} \frac{1}{\alpha_{i}}\right)+\lambda=0.
$$

Solving the above equation implies that $\alpha_{i}$ takes the form
$
\alpha_{i}=\exp \left(\frac{1}{\tau} \psi_{i}\right) \exp \left(\frac{-\lambda}{\tau}-1\right).
$ Since $\alpha_{i}$ lies on the probability simplex, the optimal $\alpha_{i}$ is explicitly given by
$
\alpha^{*}_{i}=\frac{\exp \left(\frac{1}{\tau} \psi_{i}\right)}{\sum_{i^{\prime}=1}^n \exp \left(\frac{1}{\tau} \psi_{i^{\prime}}\right)} .
$ Substituting the optimal point into the objective function, we obtain
$$
\begin{aligned}
E\left(\boldsymbol{\alpha}^*\right)  &=\psi_1-\sum_{i=1}^n \frac{\exp \left(\frac{1}{\tau} \psi_{i}\right)}{\sum_{i^{\prime}=1}^n \exp \left(\frac{1}{\tau} \psi_{i^{\prime}}\right)} \psi_{i}+\tau \sum_{i=1}^n \frac{\exp \left(\frac{1}{\tau} \psi_{i}\right)}{\sum_{i^{\prime}=1}^n \exp \left(\frac{1}{\tau} \psi_{i^{\prime}}\right)}\log \frac{\exp \left(\frac{1}{\tau} \psi_{i}\right)}{\sum_{i^{\prime}=1}^n \exp \left(\frac{1}{\tau} \psi_{i^{\prime}}\right)} \\
& =\psi_1 - \tau \log \left(\sum_{i=1}^n \exp \left(\frac{1}{\tau} \psi_{i}\right)\right).
\end{aligned}
$$
Thus, the Lagrangian dual function is given by
\begin{equation*}
-E\left(\boldsymbol{\alpha}^*\right)= -\tau \log \frac{\exp \left(\frac{1}{\tau} \psi_{1}\right)}{\sum_{i=1}^n \exp \left(\frac{1}{\tau} \psi_{i}\right)}.\qedhere
\end{equation*}
\end{proof}



\section{More on Experiments} \label{section: experiment_details}

\paragraph{CIFAR-10 and CIFAR-100} CIFAR-10 ~\citep{krizhevsky2009learning} and CIFAR-100 ~\citep{krizhevsky2009learning} are well-known classic image classification datasets. Both CIFAR-10 and CIFAR-100 contain a total of 60k $32 \times 32$ labeled images of different classes, with 50k for training and 10k for testing. CIFAR-10 is similar to CIFAR-100, except there are 10 different classes in CIFAR-10 and 100 classes in CIFAR-100.

\paragraph{TinyImageNet} TinyImageNet ~\citep{le2015tiny} is a subset of ImageNet ~\citep{deng2009imagenet}. There are 200 different object classes in TinyImageNet, with 500 training images, 50 validation images, and 50 test images for each class. All the images in TinyImageNet are colored and labeled with a size of $64 \times 64$.

\textbf{Pseudo-code.} Algorithm \ref{alg:Training Procedure} presents the pseudo-code for our empirical training procedure.

\begin{algorithm}[!htbp]
\caption{Training Procedure}
\label{alg:Training Procedure}
\begin{algorithmic}[1]
\REQUIRE trainable encoder network $f$, batch size $N$, augmentation strategy \textit{aug}, loss function $L$ with hyperparameters \textit{args}
\FOR {sampled minibatch ${x_i}_{i=1}^N$}
\FORALL{$i \in { 1, ..., N }$}
\STATE draw two augmentations $t_i = \textit{aug}\left(x_i\right) $, $t_i' = \textit{aug}\left(x_i\right) $
\STATE $z_i = f\left(t_i\right)$, $z_i' = f\left(t_i'\right)$
\ENDFOR
\STATE compute loss $\mathcal{L} = L(N, z, z', \textit{args})$
\STATE update encoder network $f$ to minimize $\mathcal{L}$
\ENDFOR
\STATE \textbf{Return} encoder network $f$
\end{algorithmic}
\end{algorithm}

We also provide the pseudo-code for our core loss function used in the training procedure in Algorithm \ref{alg:Core loss}. The pseudo-code is almost identical to SimCLR's loss function, with the exception of an extra parameter $\gamma$.

\begin{algorithm}[!htbp]
\caption{Core loss function $\mathcal{C}$}
\label{alg:Core loss}
\begin{algorithmic}[1]
\REQUIRE batch size $N$, two encoded minibatches $z_1, z_2$, $\gamma$, temperature $\tau$
\STATE $z = \textit{concat}\left(z_1, z_2\right)$
\FOR {$i \in {1, ..., 2N }, j \in {1, ..., 2N}$ }
\STATE $s_{i,j} = \Vert z_i - z_j \Vert_2^{\gamma}$
\ENDFOR
\STATE \textbf{define} $l(i, j)$ \textbf{as} $l(i, j) = - \log \frac{exp\left(s_{i,j}/\tau \right)}{\sum_{k=1}^{2N} \mathbf{1}{[k \ne i]} exp\left(s{i, j} / \tau \right)} $
\STATE \textbf{Return} $\frac{1}{2N} \sum_{k=1}^N\left[l(i, i+N) + l(i+N, i)\right]$
\end{algorithmic}
\end{algorithm}

Utilizing the core loss function $\mathcal{C}$, we can define all kernel loss functions used in our experiments in Table \ref{table: loss definition}. For all $z_i \in z$ with even dimensions $n$, we define $z_{L_i} = z_i\left[0:n/2\right]$ and $z_{R_i} = z_i\left[n/2:n\right]$.

\begin{table}[ht]
\centering
\begin{tabular}{{@{}l|l@{}}}
Kernel  &  Loss function \\ \midrule
Laplacian & $\mathcal{C}\left(N, z, z', \gamma=1, \tau\right)$\\ \midrule
Sum       & $\lambda * \mathcal{C}\left(N, z, z', \gamma=1, \tau_1\right) + (1-\lambda) * \mathcal{C}\left(N, z, z', \gamma=2, \tau_2\right)$  \\ \midrule
Concatenation Sum&$\lambda * \mathcal{C}\left(N, z_L, z'_L, \gamma=1, \tau_1\right) + (1-\lambda) * \mathcal{C}\left(N, z_R, z'_R, \gamma=2, \tau_2\right)$\\ \midrule
$\gamma = 0.5$ & $\mathcal{C}\left(N, z, z', \gamma=0.5, \tau\right)$          \\ 

\end{tabular}

\caption{Definition of kernel loss functions in our experiments}
\label {table: loss definition}
\end{table}

\textbf{Baselines.} We reproduce the SimCLR algorithm using PyTorch Lightning~\citep{PytorchLightning}.

\textbf{Encoder details.}
The encoder $f$ consists of a backbone network and a projection network. We employ ResNet50~\citep{ResNet} as the backbone and a 2-layer MLP (connected by a batch normalization~\citep{ioffe2015batch} layer and a ReLU \cite{nair2010rectified} layer) with hidden dimensions 2048 and output dimensions 128 (or 256 in the concatenation kernel case).

\textbf{Encoder hyperparameter tuning.}
For each encoder training case, we randomly sample 500 hyperparameter groups (sample details are shown in Table \ref{table: Hyperparameter sample}) and train these samples simultaneously using Ray Tune ~\citep{RayTune}, with the ASHA scheduler~\citep{li2018massively}. Ultimately, the hyperparameter group that maximizes the online validation accuracy (integrated in PyTorch Lightning) within 5000 validation steps is chosen for the given encoder training case.

\begin{table}[ht]
\centering

\begin{tabular}{@{}l|l|l@{}}
\midrule
Hyperparameter  & Sample Range & Sample Strategy \\ \midrule
start learning rate & $\left[10^{-2}, 10\right]$ & log uniform \\ \midrule
$\lambda$       & $\left[0, 1\right]$ & uniform \\ \midrule
$\tau$, $\tau_1$, $\tau_2$ & $\left[0, 1\right]$ & log uniform \\ \midrule
\end{tabular}

\caption{Hyperparameters sample strategy}
\label {table: Hyperparameter sample}
\end{table}

\textbf{Encoder training.} 
We train each encoder using the LARS optimizer~\citep{LARSOptimizer}, LambdaLR Scheduler in PyTorch, momentum 0.9, weight decay $10^{-6}$, batch size 256, and the aforementioned hyperparameters for 400 epochs on a single A-100 GPU.

\textbf{Image transformation.} The image transformation strategy, including augmentation, is identical to the default transformation strategy provided by PyTorch Lightning.

\textbf{Linear evaluation.}
The linear head is trained using the SGD optimizer with a cosine learning rate scheduler, batch size 64, and weight decay $10^{-6}$ for 100 epochs. The learning rate starts at $0.3$ and ends at $0$.

\textbf{Moco Experiments.} We also tested our method based on MoCo~\citep{he2019moco}. The results are summarized in Table \ref{tab:results-moco}. Here we choose ResNet18~\citep{ResNet} as the backbone and set a temperature of $0.1$ as default. For our simple sum kernel, we set $\lambda=0.8$. The results show that our method outperforms the original MoCo method.

\begin{table}[thb]
\centering
\caption{MoCo Experiment Results on CIFAR-10 and CIFAR-100.}
\label{tab:results-moco}
\resizebox{\textwidth}{!}{%
\begin{tabular}{@{}c|ccc|ccc@{}}
\toprule
\multirow{3}{*}{Method} & \multicolumn{3}{c|}{CIFAR-10} & \multicolumn{3}{c}{CIFAR-100} \\ \cmidrule(lr){2-4} \cmidrule(lr){5-7} 
                        & 200 epochs & 400 epochs    & 1000 epochs   & 200 epochs & 400 epochs & 1000 epochs         \\ \midrule
MoCo (repro.)         & $76.41 \pm 0.12$    & $80.01 \pm 0.15$          & $84.45 \pm 0.08$    & $\mathbf{47.02 \pm 0.11}$ & $52.50 \pm 0.07$ & $57.62 \pm 0.15$            \\
\midrule
Laplacian Kernel        & ${78.09 \pm 0.10}$    & $\mathbf{83.85 \pm 0.09}$          & $\mathbf{88.34 \pm 0.16}$    & $46.12 \pm 0.22$   & $53.44 \pm 0.17$ & $59.10 \pm 0.14$        \\
Simple Sum Kernel & $\mathbf{78.12 \pm 0.15}$   & $83.23 \pm 0.18$ & $87.50 \pm 0.20$ & $46.65 \pm 0.06$ & $\mathbf{53.62 \pm 0.19}$ & $\mathbf{59.83 \pm 0.12}$\\
\bottomrule
\end{tabular}
}
\end{table}



\section{More Experiments on Synthetic Data}


Consider a scenario with $n$ clusters, each containing $k$ vertices. Let the probability of vertices $u$ and $v$ from the same cluster belonging to $\bfpi$ be $p$. Conversely, for vertices $u$ and $v$ from different clusters, let the probability of belonging to $\pi$ be $q$. We generate the graph $\bfpi$ randomly, based on $p$ and $q$. We experiment with values of $k=100$ and $n=6$ for ease of visualization, embedding all points in a two-dimensional space. Each vertex's initial position originates from a normal distribution. In each iteration, we sample a subgraph of $\bfpi$ uniformly, ensuring each vertex has an out-degree of $1$. We then optimize the corresponding vectors using InfoNCE loss with an SGD optimizer and iterate until convergence. Our experimental setup consists of an SGD learning rate of $1$, an InfoNCE loss temperature of $0.5$, and a batch size of $50$. We evaluate two scenarios with different $p$ and $q$ values: $p=1$, $q=0$, and $p=0.75$, $q=0.2$. The results of these experiments are visualized in Figure \ref{fig:vis-spectral-cluster}. The obtained embeddings exhibit the hallmark pattern of spectral clustering of graph $\bfpi$.

\begin{figure}[!tb]
\centering
\subfigure{
\includegraphics[width=1\textwidth]{Figures/cluster_pi.png}
\label{fig:vis-cluster}
}
\subfigure{
\includegraphics[width=1\textwidth]{Figures/noised_cluster_pi.png}
\label{fig:vis-noised-cluster}
}
\caption{Visualizations of the optimization process using InfoNCE Loss on the vectors corresponding to $\bfpi$. Points of identical color belong to the same cluster within $\bfpi$. To showcase the internal structure of $\bfpi$, we randomly select 10 vertices from each cluster to display the edge distribution of $\bfpi$.}
\label{fig:vis-spectral-cluster}
\end{figure}



\end{document}
