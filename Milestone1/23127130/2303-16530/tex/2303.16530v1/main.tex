\documentclass[conference]{IEEEtran}
\usepackage{cite}
\usepackage{amsmath,amssymb,amsfonts}
\usepackage{algorithmic}
\usepackage{graphicx}
\usepackage{textcomp}
\usepackage{xcolor}
\usepackage{balance}
\def\BibTeX{{\rm B\kern-.05em{\sc i\kern-.025em b}\kern-.08em
    T\kern-.1667em\lower.7ex\hbox{E}\kern-.125emX}}

\usepackage{caption}
\usepackage{subcaption}
\usepackage{url}
\usepackage{enumitem}
\usepackage{booktabs}
\usepackage{array}
\usepackage{quoting}
\usepackage{hyperref}
\newcolumntype{L}[1]{>{\raggedright\let\newline\\\arraybackslash\hspace{0pt}}m{#1}}
\newcolumntype{C}[1]{>{\centering\let\newline\\\arraybackslash\hspace{0pt}}m{#1}}
\newcolumntype{R}[1]{>{\raggedleft\let\newline\\\arraybackslash\hspace{0pt}}m{#1}}

\usepackage{amssymb}
\usepackage{amsfonts}

%%%%%%%%%%%%%%%%%%%%%%%%%%%%%%%%%%%%%%%%%%%%%%%%%%%%%%%%%%%%%%%%%%%%%%%%%%%%%%%
\makeatletter
\def\ps@IEEEtitlepagestyle{
	\def\@oddfoot{\mycopyrightnotice}
	\def\@evenfoot{}
}
\def\mycopyrightnotice{
	{\footnotesize
		\begin{minipage}{\textwidth}
			\centering
			\textcopyright~2023 IEEE.  Personal use of this material is permitted.  Permission from IEEE must be obtained for all other uses, in any current or future media, including reprinting/republishing this material for advertising or promotional purposes, creating new collective works, for resale or redistribution to servers or lists, or reuse of any copyrighted component of this work in other works.
		\end{minipage}
	}
}
%%%%%%%%%%%%%%%%%%%%%%%%%%%%%%%%%%%%%%%%%%%%%%%%%%%%%%%%%%%%%%%%%%%%%%%%%%%%%%%


\usepackage[many]{tcolorbox}   
\newtcolorbox{boxA}{
    colback = white,
    boxrule = 1pt,
    colframe = black, 
    boxsep=1pt, 
    left=2pt,
    right=2pt,
    top=2pt,
    bottom=2pt
}
\begin{document}

\title{Runtime Verification of Self-Adaptive Systems with Changing Requirements}

\author{
	\IEEEauthorblockN{
            Marc Carwehl\IEEEauthorrefmark{1},
		Thomas Vogel\IEEEauthorrefmark{1},
		Genaína Nunes Rodrigues\IEEEauthorrefmark{2},
		and
		Lars Grunske\IEEEauthorrefmark{1}
	}
	\IEEEauthorblockA{\IEEEauthorrefmark{1}\textit{Department of Computer Science}, \textit{Humboldt-Universit\"at zu Berlin}, Berlin, Germany\\
		Email: carwehl@informatik.hu-berlin.de, thomas.vogel@informatik.hu-berlin.de, grunske@informatik.hu-berlin.de}
	\IEEEauthorblockA{\IEEEauthorrefmark{2}\textit{Department of Computer Science, University of Bras\'ilia, Brazil}\\
		Email: genaina@unb.br}
}

\maketitle

\begin{abstract}

To accurately make adaptation decisions, a self-adaptive system needs precise means to analyze itself at runtime. To this end, runtime verification can be used in the feedback loop to check that the managed system satisfies its requirements formalized as temporal-logic properties. These requirements, however, may change due to system evolution or uncertainty in the environment, managed system, and requirements themselves. Thus, the properties under investigation by the runtime verification have to be dynamically adapted to represent the changing requirements while preserving the knowledge about requirements satisfaction gathered thus far, all with minimal latency. To address this need, we present a runtime verification approach for self-adaptive systems with changing requirements. Our approach uses property specification patterns to automatically obtain automata with precise semantics that are the basis for runtime verification. The automata can be safely adapted during runtime verification while preserving intermediate verification results to seamlessly reflect requirement changes and enable continuous verification. We evaluate our approach on an Arduino prototype of the Body Sensor Network and the Timescales benchmark. Results show that our approach is over five times faster than the typical approach of redeploying and restarting runtime monitors to reflect requirements changes, while improving the system's trustworthiness by avoiding interruptions of verification.

\end{abstract}

\begin{IEEEkeywords}
Runtime verification; requirement changes; property specification patterns; self-adaptive systems;
\end{IEEEkeywords}

% Importance and appeal of children's drawings
Children's depictions of the human figure are highly expressive and varied.
As one of the very first subjects children attempt to draw, the representation begins as an almost unintelligible cloud of scribbles. 
As the child grows, their representation of the human figure becomes more developed and is extended to graphically represent many different types of characters: people, animals, and even personified objects (see Figure 1).

Who among us has not wished, either as a child or as an adult, to see such figures come to life and move around on the page?
Sadly, while it is relatively fast to produce a single drawing, creating the sequence of images necessary for animation is a much more tedious endeavor, requiring discipline, skill, patience, and sometimes complicated software.
As a result, most of these figures remain static upon the page.

% We built a system to animate them.
Inspired by the importance and appeal of the drawn human figure, we design and build a system to automatically animate it given an in-the-wild photograph of a child's drawing. 
Our system is fast, intuitive, and robust to much of the variation present in these types of drawings, making it well-suited to allow our target audience--children--to see their own characters coming to life.
The system is comprised of four stages: figure detection, segmentation masking, pose estimation/rigging, and animation. 
We describe each stage and identify common causes of failure in each. 
For object detection and pose estimation, we make use of existing computer vision models designed to detect human figures and joints in photographs; we fine-tune these models for use with children's drawings.
For segmentation, we present a straightforward, image processing-based method that, for animation purposes, is more useful and accurate than segmentation masks obtained from a fine-tuned object detection model.
During the animation step, we take advantage of the \textit{twisted perspective} commonly seen in children’s drawings to retarget motion capture data onto the character in a novel and appealing way.

% We use existing machine learning models. However, given the wide domain gap it's not clear how much fine-tuning data was needed. So we ran some experiments to find out and report it.
While our system leverages existing models and techniques, most are not directly applicable to the task due to the many differences between photographic images and simple pen and paper representations. 
To this end, we couple the presentation of our system with a set of experiments exploring the relationship between fine-tuning training set size and success rates.
We also include a perceptual study validating viewer preference for incorporating \textit{twisted perspective} into the motion retargeting step.

We validate the desirability and appeal of our system by building and publicly releasing a version of it as the \AD Demo \,\cite{animateddrawings}.
Launched in December 2021, this demo has been used by millions of people around the world to animate their children's drawings.
Inspired by this reception, our second contribution is The Amateur Drawings Dataset: \hjs{180,000 drawings and user-accepted annotations collected, with consent, through the demo. See Section \ref{sec:UI} for a description of how the annotations were generated.}
We believe this dataset will be a resource to researchers from various fields seeking to better understand the space of amateur drawings, evaluate new algorithms in this domain, or develop new drawing-based tools in general.

To summarize, our contributions are as follows:
\begin{enumerate}
    \item 
    We explore the problem of automatic sketch-to-animation for children's drawings of human figures and present a framework that achieves this effect. We also present a set of experiments determining the amount of training data necessary to achieve high levels of success and a perceptual study validating the usefulness of our motion retargeting technique.
    \item To encourage additional research in the domain of amateur drawings, we present a first-of-its-kind dataset of 180,000 user-submitted amateur drawings, along with user-accepted bounding box, segmentation mask, and joint location annotations.
\end{enumerate}

Upon acceptance of this paper, we plan to publicly release the Amateur Drawings Dataset, project code, and fine-tuned model weights.

Our work builds on existing methods from several fields but is, to our knowledge, the first work focused specifically on fully automatic animation of children's drawings of human figures. 
To ground the work, we provide a summary of salient observations from the field of children's art analysis.
In addition, we briefly review related work on 2D image-to-animation and object and pose estimation for abstract images. 


\subsection{Analysis of Children's Drawings}

\hjs{
Children's drawings have long been of interest to the scientific community.
For well over a century, researchers from multiple fields have 
collected\,\cite{IndianaS55:online,kellogg1967rhoda,AWebbasedDatabaseforDrawingsofGods,geist2002they}
and analyzed them, seeking to understand and measure children's thought processes\,\cite{sully2021studies,barnes1892study,clark1897child,buhler2013mental}, 
intellectual development\,\cite{goodenough1926measurement},
and perceptions\,\cite{chambers1983stereotypic,doi:10.1080/01443410500344167}.
}
Particular attention has been given to drawings of human figures, one of the first and most frequently drawn subjects throughout childhood\,\cite{cox2013children}.

As the child develops, the schemas they employ to represent the human form become more complete (see Figure \ref{fig:tadpole-transitional-conventional}).
Even within these schemas, there is significant variation.
In addition to asymmetries and variation in color and proportion, many body parts appear optional to include; a study of drawings by 4-6 year old children showed that, while heads, legs, and eyes are almost universally present, other body parts (including torsos, arms, hands, and feet) were frequently absent\,\cite{cox2013children}.
Inversely, non-human body parts are frequently added in order to represent other subject classes\,\cite{kellogg1969analyzing}. With the addition of large ears, the figure may represent a cat or bear (Figures \ref{fig:maskrcnn_before_after}.m and \ref{fig:maskrcnn_before_after}.g); with the addition of a crown, it can represent a pineapple (Figure \ref{fig:maskrcnn_before_after}.n).
All of these sources of character variation make automatic character animation from drawings a non-trivial task.

\begin{figure}
\includegraphics[width=\linewidth]{images/tadpole-transition-conventional.png}
\caption{
As children learn to draw the human figure, the morphologies of the schemas they employ vary and evolve considerably\,\cite{cox2014drawings}.
Children frequently begin by drawing a \textit{tadpole figure}, a circular head region from which arms and legs extend. 
Some will progress to a \textit{transitional figure}, dropping the arms down so they extend from the legs. 
When a line is drawn between the legs, creating the separate torso region, the \textit{conventional figure} is formed.
Though these are small changes from the perspective of the drawer, they result in significantly different character morphologies when viewed through the lens of character animation.
A successful drawing-to-animation system must be robust to these types of variations.}
\label{fig:tadpole-transitional-conventional}
\end{figure}

Many researchers have focused closely on the unique style of children's drawings.
The psychologist and artist John Willats argues that, in order to understand the style of children's drawings, one must understand that the primary picture primitives employed by children are \textit{regions}, or 2D areas\,\cite{willats2006making}.
A squat volume, such as a head or torso, may be represented by a circular or ellipsoid region, whereas an elongated volume, such as a leg, may be represented by a long, thin region or even a single line.
These regions are not depictions of the object from any particular point of view. 
Rather, they are \textit{3D volumetric object-centered descriptions}\,\cite{marr1982vision},
2D areas with attributes perceptually similar to those of 3D object they are meant to represent.
%The regions begin as circles and lines, but later become modified to better reflect the perceptually impactful aspects of the objects they represent; a region representing a sugar cube or die may be given square corners, and a long region representing an arm may be given a bend to depict the elbow or split at the end to represent fingers (CITE Willats, 2005).

There are two stylistic outcomes of these \textit{object-centered descriptions} that bear mention.
First, the use of foreshortening is very rare in children's drawings \,\cite{piaget1956, willats1992representation}. 
This design choice is understandable; foreshortening a long region, such as a limb, results in a short region which does not adequately reflect the \textit{longness} of the object.
Second, the human figure may appear to have been drawn from many different perspectives, so as to make each part of the character maximally recognizable.
For example, the head and torso may face forward while the legs and feet are pointed to the side.
This technique, often referred to as \textit{twisted perspective}, is frequently seen and well-documented\,\cite{dziurawiec1992twisted}.
Both of these stylistic aspects are used to guide the design decisions of our system when applying human motion capture data onto the character.


\subsection{2D Image to Animation}

Previous researchers have proposed methods to animate drawings or photographs, many of which rely upon additional modes of user input.
Hornung et al. present a method to animate a 2D character in a photograph, given user-annotated joint locations\,\cite{Hornung2007anim2Dpicmotion}.
Pan and Zhang demonstrate a method to animate 2D characters with user-annotated joint locations via a variable-length needle model\,\cite{Pan2011}.
Jain et al. present an integrated approach to generate 3D proxies for animation given joint locations, segmentation masks, and per-part bounding boxes\,\cite{jain:2012}. 
Levi and Gotsman provide a method to create an articulated 3D object from a set of annotated 2D images and an initial 3D skeletal pose\,\cite{ArtiSketch}.
\textit{Live Sketch}\,\cite{su2018livesketch}
tracks control points from a video and applies their motion to user-specified control points upon a character.
Other approaches allow the user to specify character motions through a puppeteer interface, using RGB or RGB-D cameras\,\cite{held20123d,barnes2008video}.
\textit{ToonCap}\,\cite{Fan:2018:TAL} focuses on an inverse problem, capturing poses of a known cartoon character, given a previous image of the character annotated with layers, joints, and handles. 


\textit{Toonsynth}\,\cite{Dvoroznak18-SIG} and \textit{Neural Puppet}\,\cite{poursaeed2020neural} both present methods to synthesize animations of hand-drawn characters given a small set of drawings of the character in specified poses.
Hinz et al. train a network to generate new animation frames of a single character given 8-15 training images with user-specified keypoint annotations\,\cite{hinz2022charactergan}.

\textit{Monster Mash}\,\cite{Dvoroznak20-SA} presents an intuitive framework for sketch-based modeling and animation, and \textit{2.5D Cartoon Models}\,\cite{10.1145/1778765.1778796} presents a novel method of constructing 3D-like characters from a small number of 2D representations. 
Both of these are intuitive and well designed animation tools targeted towards amateur users.


\hjs{
Some animation methods are specifically tailored toward particular forms, such as faces\,\cite{elor2017bringingPortraits}, coloring book characters\,\cite{magnenat2015live}, or characters with human-like proportions. 
One notable work that is focused on the human form is \textit{Photo Wake Up}\,\cite{weng2019photo}. 
The authors show a method for creating a rigged and textured 3D mesh from a single image of a human-like figure.
Similar to us, their end goal is to allow users to seamlessly bring 2D characters to life; their work does an impressive job of accomplishing this.
Our method differs in two significant ways. 
First, while their work is focused on creating a 3D model for a mixed reality use case, 
ours is specifically focused on animating twisted perspective figures while staying within a 2D plane.
Second, children's drawings are much more abstract, incorrectly proportioned, and non human-like than the examples demonstrated in the paper.
We test our method upon the more abstract examples demonstrated in their paper and, with minor segmentation cleaning, they were successfully animated by our method.
}












\hjs{While the approaches listed here are wonderful tools to ease the burden of animation, none were perfectly suited to our use case.
Some require additional user input beyond the drawing itself, making the animation process more complex.
Others require the user to consistently draw the same character in multiple poses, which is beyond the skills of young children.
Others are focused on animating specific forms, precluding their use on children's drawings of the human figure.}


%Siarohin and colleagues propose a method for animating arbitrary classes of subjects,
%but require training videos of class members moving\,\cite{Siarohin_2019_NeurIPS}, making it unsuitable children's drawings.


\subsection{Detection, Segmentation, and Pose Estimation on Non-Photorealistic Images}

\hjs{
Aided by the the existence of large annotated datasets\,\cite{lin2014microsoft,6909866,6682899}, researchers have made considerable progress solving the problems of object detection, segmentation, and pose estimation from photographs. See, for example\,\cite{MaskRCNNhe2017mask,openpose19,guler2018densepose,alphapose,toshev2014deeppose}.
We explain the methods in this area that we leverage in Sections \ref{sec:character_detection} and \ref{sec:joint_detection}.

While traditional methods for detection, segmentation, and pose estimation of non-photorealistic images exist\,\cite{choi2012retrieval,bregler2002turning,davis2006sketching,eitz2012humans}, the lack of easily available datasets has resulted in slower adoption of deep learning models.
Some researchers are addressing this problem by developing methods and releasing datasets focused on the domain of anime characters\,\cite{chen2022bizarre,10.1145/3011549.3011552}, professional sketches\,\cite{brodt2022sketch2pose}, and mouse doodles\,\cite{ha2017neural}.
Other researchers have presented a non-deep learning method for inferring character poses from \textit{gesture drawings}\,\cite{Gesture3D}.
}
Because the Amateur Drawings Dataset is comprised of in-the-wild photographs of drawings created by the general public, we believe it will complement the value of existing datasets and allow for new dimensions of exploration and analysis.

\section{Runtime Verification of Self-Adaptive Systems with Changing Requirements}\label{sec:approach}

In this section, we discuss our approach to runtime verification of SAS with changing requirements. 
First, we introduce the \textit{Property Specification and Adaptation Patterns} for runtime verification. Then, we detail how our approach uses these patterns in SAS for automata-based runtime verification, where requirements changes imply adaptations of properties and observers. We use the BSN as a running example.

\subsection{Property Specification and Adaptation Patterns}\label{sec:approach:catalog}

In our work, we focus on verifying SAS against properties expressed in MTL. To ease the formalization of requirements as MTL properties, we leverage the \textit{Property Specification Patterns} (PSP) from literature (Section~\ref{sec:background:psp}). We particularly reuse the Structured English Grammar and mapping to MTL formula templates from Autili et al.~\cite{AutiliGLPT15}. Thus, a user formalizes a requirement in structured English, which is automatically translated to an MTL property. 

Additionally, to realize an \textit{automata-based runtime verification}, we need to construct observers for MTL properties (Section~\ref{sec:background:rv}). To automate the construction of observers with PSP, we built upon our PSP catalog that provides observer templates (UPPAAL timed automata) for TCTL properties and focuses on design-time model checking with UPPAAL~\cite{vogel2023property}. However, direct reuse of these observer templates for runtime verification is not feasible, e.g., due to non-determinism in these observers and their focus on design-time model checking. 
Thus, we created new observer templates for MTL properties that are deterministic and focus on
runtime verification. Still, the existing templates have been a solid basis to obtain the new templates. These templates allow us to automatically construct observers for properties expressed in Structured English.

Our novel PSP catalog for runtime verification with its observer templates builds on the mapping from natural language to MTL~\cite{AutiliGLPT15} and observer techniques~\cite{vogel2023property}. Therefore, the catalog offers precise semantics in expressing properties and representing them in observers. 
In particular, we systematically created the observer templates by manually analyzing all possible types of traces that would violate a property, and generalizing these traces to a timed automaton. Thus, such an observer template represents a set of traces, of which some violate the corresponding property. To distinguish violating and satisfying traces, the observer template contains an \textit{error} state that is reached if and only if a trace violates the property.  

Additionally, with our catalog we propose \textit{Property Adaptation Patterns} (PAP) that define at the PSP level how observers should be adapted to represent changes of requirements. Thus, adaptations of properties are accurately reflected in the observers. Technically, PAP are defined by graph transformations (cf.~\cite{Giese+2012}) on observer templates that are instantiated to adaptation rules for concrete observers.
Thus, our catalog does not only provide precise semantics for specifying (using PSP) but also for adapting properties (using PAP).
The PSP/PAP catalog is publicly available\footnote{\url{https://github.com/hub-se/PAP/wiki}} and  detailed in the following sections.


\subsection{Architectural Overview}\label{sec:approach:architecture}

Fig.~\ref{fig:architecture} shows an architectural overview of our approach. We consider a SAS to be split into a managed and managing system operating in an environment. The managing system is split into two layers, each implementing a MAPE-K feedback loop: the \textit{Change Manager} and the \textit{Requirements Manager}.

\begin{figure}
    \centering
    \includegraphics[width=.9\columnwidth]{figures/architecture.png}
    \caption{Architectural Overview.}
    \label{fig:architecture}
    \vspace{-2em}
\end{figure}

\paragraph{Change Manager}
The change manager (shaded in blue in Fig.~\ref{fig:architecture}) adapts the managed system so that the system satisfies its requirements despite uncertainty. Adaptation is needed if requirements are violated. To determine such violations, the change manager performs \textit{automata-based runtime verification} (Section~\ref{sec:background:rv}).
For this purpose, it monitors the managed system and environment. The \textit{Event Monitor} adds events representing changes of the system and environment to the first-in-first-out \textit{Queue}. 
In the \textit{Analyze} step, the \textit{Runtime Verifier} consumes the events from the queue and matches them against the \textit{Observers}, each representing a property.

If a property is violated, adaptation of the managed system is needed so that the \textit{Plan} and \textit{Execute} steps are performed. 
The change manager interacts with the \textit{Requirements Manager} by notifications about property violations. The other way around, the requirements manager initializes the runtime verification by providing observers to the change manager and dynamically adapts these observers if requirements change. 


\paragraph{Requirements Manager}
The requirements manager (top layer shaded in gray in Fig.~\ref{fig:architecture}) is in charge of formalizing requirements given by a human in Structured English to MTL properties and corresponding observers. It uses our \textit{Property Specification Pattern} (PSP) catalog comprising mappings to MTL and observer templates. The generated observers are provided to the change manager for runtime verification. 

At runtime, the requirements manager monitors the change manager, managed system, and environment to identify with the help of a human changes of requirements, for instance, triggered by a human or the change manager notifying about violations of requirements. 
In the \textit{Analyze} and \textit{Plan} steps, the requirements manager and human determine which requirements have changed and how they have changed to adapt the properties and corresponding observers used by the change manager accordingly. To accurately adapt properties and observers for runtime verification, our \textit{Property Adaptation Pattern} (PAP) catalog comprises adaptation templates for each observer template. These adaptation templates are instantiated to adaptation rules that are automatically and safely executed on the change manager's observers to reflect the changed requirement for runtime verification.
Changes of requirements may also mean that new requirements emerge or existing requirements become irrelevant. Thus, the requirements manager has to synthesize new properties and observers that are provided to the change manager or respectively remove existing observers from the change manager. 

We believe that fully automating the requirements manager, especially the analyze and plan steps, is challenging and also possibly not desired. Therefore, our current proposal includes the \textit{human in the loop} who is in charge of decision-making regarding identifying requirements changes and determining how properties/observers need to be adapted to reflect these changes. However, our PSP/PAP catalog supports the human by easing the specification and adaptation of properties with their observers while providing precise semantics for them. Moreover, the execution of dynamic and safe adaptations of properties/observers is performed automatically.

In the following, we detail how runtime verification is initialized and performed, and how properties are adapted.
In general, the system may have multiple requirements and each requirement may be expressed by multiple properties. For readability purposes, we describe in the following section how our approach handles one requirement expressed by one property. 
Nevertheless, our approach works with multiple requirements and multiple properties by deploying multiple independent instances of the \textit{Event Monitor}, including the \textit{Queue}, and the \textit{Runtime Verifier}.

\subsection{Initializing Runtime Verification}\label{subsec:initialzingRV}

Our PSP catalog in the \textit{Requirements Manager} is used when a stakeholder expresses a requirement in Structured English. The catalog then generates an MTL property formalizing the requirement. Moreover, the corresponding observer template is retrieved and instantiated to an observer that represents this property (an example is shown in Fig.~\ref{fig:obs-all}). The resulting observer is eventually provided to the \textit{Change Manager}'s \textit{Knowledge} and used by the \textit{Runtime Verifier} (cf. Fig.~\ref{fig:architecture}).

Conceptually, an observer contains a set of states, one of which is the current state representing the current state of the managed system and environment. Each of the observer's states has a set of outgoing transitions. Such a transition points to another state (the transition's \textit{target}) and may have a guard condition over clocks and an action to reset clocks. 
Additionally, transitions may be labeled with an event type.

After deployment, the runtime verifier sets the observer's current state to the initial state.
Additionally, based on the property, a list of event types relevant to evaluate the property is provided to the change manager's knowledge and used by the event monitor to filter relevant events emitted from the managed system and environment.\footnote{Technically, the managed system and environment do not need to emit events, but the monitor can sense the state of the system and environment and create corresponding events whenever relevant state changes occur.} Filtered events are added to the \textit{Queue}, from where they are processed for runtime verification by the runtime verifier (see Section~\ref{subsec:performingRV}).

\begin{figure}[t!]
    \centering
    \begin{subfigure}{0.5\textwidth}
        \centering
        \includegraphics[width=.9\columnwidth]{figures/observer-template.png}
        \caption{Observer template.}
        \label{fig:obs-template}
    \end{subfigure}%

    \begin{subfigure}{0.5\textwidth}
        \centering
        \includegraphics[width=.9\columnwidth]{figures/observer2.png}
        \caption{Instantiated observer from the template.}
        \label{fig:obs}
    \end{subfigure}%
    \caption{Observer for Timed Response Chain w/ Between scope. }
    \label{fig:obs-all}
    \vspace{-1em}
\end{figure}

\textbf{Example:}
In the BSN, the developer is aware that network congestions may arise. To this end, the change manager adapts the managed system to decrease network usage by reducing the number of scheduler cycles and therefore limiting how often the BodyHub requests data. Nevertheless, the developer wants to specify that in any case, both sensor nodes (thermometer and pulse sensor) shall send data to the BodyHub when the BodyHub requests such data. The developer sets a time limit of $2s$ within which both sensors shall respond. This behavior is expected to be repeated for every scheduler cycle. To formalize this requirement, the developer selects a suitable pattern (\textit{Timed Response Chain} with the \textit{Between} scope) from the PSP catalog and expresses the requirement using the structured English grammar:

\begin{quote}
    Between the scheduler cycle starting and elapsing, if \emph{the BodyHub requests data}, then in response \emph{the thermometer sends} eventually  \emph{within 2s }followed by \emph{the pulse sensor sends} \emph{within 2s}.
\end{quote}

The Timed Response Chain covers requirements that expect an ordered chain of events within a time window in response to a request event. The Between scope further requires that the request and responses together are surrounded by two events, in this case, describing the start and end of a scheduler cycle.
%
The requirements manager uses the PSP catalog to provide the MTL formula template
\begin{equation}
\begin{split}
        & \square (( Q \wedge \lozenge R ) \rightarrow \\ 
        & (P \rightarrow \linebreak (\neg R \mathcal{U}^{[0,t]} (S_1 \wedge \neg R \wedge (\lozenge ^{[0, t]} (S_2 ))  ))) \mathcal{U} R)
\end{split}
\end{equation}
as well as the corresponding observer template shown in Fig.~\ref{fig:obs-template} that both correspond to the selected pattern. Both templates are instantiated and the placeholders $P, Q, R, S_1, S_2$ are replaced with actual values from the Structured English requirement. This results in the following MTL formula:
\begin{equation}\label{eq:req1}
\begin{split}\raisetag{2.8em}
        & \square (( \text{cycle\_starting} \wedge \lozenge \text{cycle\_ending} ) \rightarrow (\text{request} \rightarrow \\
        & (\neg \text{cycle\_ending} \text{ }\mathcal{U}^{[0,2]} (\text{thermometer\_reply} \text{ }\wedge \\ 
        & \neg \text{cycle\_ending} \wedge (\lozenge ^{[0, 2]} (\text{pulse\_reply} ))  ))) \text{ }\mathcal{U}\text{ } \text{cycle\_ending})
\end{split}
\end{equation}
and the observer shown in Fig.~\ref{fig:obs}.
 
In the observer, there are states corresponding to the scheduler cycle not having started (\emph{closed}), the cycle having started (\emph{open}), the BodyHub requesting data (\emph{waiting$_1$}), and the first sensor, but not yet the second sensor sending data (\emph{waiting$_2$}). Additionally, there is an \emph{error}-state that is reached if and only if the scheduler cycle elapses before both sensors send data, or if the time bound of $2s$ elapses. 

The event types that need to be monitored to evaluate the property are: (i) the scheduler cycle starts, (ii) the scheduler cycle ends, (iii) the BodyHub requests data, (iv) the pulse sensor sends, and (v) the thermometer sends.
The observer's current state is set to its initial state, which is \emph{closed}. 

 
\subsection{Performing Runtime Verification}\label{subsec:performingRV}

Once the observer is deployed to the change manager, the \textit{runtime verifier} traverses the observer to verify the managed system against the property encoded in the observer. To this end, it checks if outgoing transitions of the observer's current state are enabled. 
A transition is enabled if its guard condition is evaluated to \textit{true}. In general, our observers only have guard conditions that refer to clock valuations against time bounds. Therefore, an observer has a clock that can be reset when a transition is taken.  
Transitions equipped with a label are only enabled when an event instance of the labeled event type is processed. 
Disabled transitions with a guard condition that are unlabeled may become enabled simply because time progresses. To take such a transition as soon as the guard's valuation changes from \textit{false} to \textit{true}, the runtime verifier analyzes all such transitions starting in the observer's current state upon entering it and determines the amount of time that needs to pass for each transition to become enabled. For the smallest such time, a timer is set that will generate an event that triggers the runtime verifier to force progress in the observer. Therefore, the runtime verifier only needs to access the observer when an event is monitored, either stemming from the managed system or environment, or from an elapsed timer. If the state in the observer is switched due to events from the managed system or environment and before the timer elapses, the timer is discarded. 


To perform runtime verification, the \textit{event monitor} observes the managed system and environment and puts the events determined as relevant in the first-in-first-out event queue, which serves as a buffer of events, maintained by the knowledge component.\footnote{For runtime verification, we assume that monitored events from the managed system and environment have a strict order, that is, one event is monitored after another. Thus, an event is a tuple of an instance of some general event type and a timestamp. A trace is a list of such events.} The \textit{runtime verifier} processes these events from the queue one after the other. For each consumed event, it checks whether any outgoing transition of the observer's current state is enabled. 
If a transition is enabled, it is taken and the observer's current state is set to the transition's target. 
Otherwise, when no transition is enabled, the observer remains in its current state. At this point, the processed event is discarded. 
Upon reaching a new state, again all outgoing transitions are checked. If no transition is enabled, the observer remains in its current state and a timer is set according to the outgoing transitions' guard conditions (if applicable). 
Therefore, while the observer's structure represents the property, its current state represents previously obtained knowledge about the execution of the managed system and its environment until now.

If and only if the observer reaches an error state during the runtime verification, the managed system in its environment violates the property. 
In the change manager, detecting such a violation can act as a stimulus that, among others, triggers it to plan and execute an adaptation of the managed system. The change manager may also notify the requirements manager of the violation.
We designed this observer-based verification technique to be used in an online setting, that is, the verification is performed alongside the running managed system and environment that provide a continuing stream of events. However, the technique can be used without modifications for offline verification when a trace of events is provided later. 

\textbf{Example:}
Suppose the following execution of the BSN regarding the property specified above: 
First, a scheduler cycle starts. The monitor adds the corresponding event to the queue before it is analyzed. For the observer's current state, there is an enabled transition for the monitored event. The transition is taken and the observer progresses to state \emph{open}. $100ms$ later, the \emph{BodyHub requests} data from the sensors. Processing this event, the observer is progressed to state \emph{waiting$_1$} and its clock $c$ is reset. Upon analyzing the current state's outgoing transitions, a timer is set to $2,000ms$. If the thermometer and pulse sensor do not send data in return before the timer elapses, the observer progresses to the \textit{error} state. 
During the BSN's execution, the change manager may perform adaptations in the scheduler, that is, the number of scheduler cycles may be decreased to improve confidence in the obtained data, or increased to reduce energy consumption and network usage. 


\subsection{Adapting the Property During Runtime Verification}\label{sec:approach:pap}
At runtime, the requirements manager monitors the change manager, managed system, and environment to identify and handle requirements changes with the help of a human. 
It can further react to notifications from the change manager that the managed system currently violates the property. 
%
If the requirements manager finds that the previously specified property is no longer adequate, it utilizes the \textit{property adaptation patterns}~(PAP) to systematically adapt the existing property. These PAP extend our PSP catalog to define adaptations of properties and observers at the pattern level (Section~\ref{sec:approach:catalog}). Therefore, they provide precise semantics for such adaptations.

Particularly, the requirements manager selects the PAP that appropriately reflects the requirements change in the property and observer, instantiates the PAP to an adaptation rule, and applies this rule to dynamically adapt the observer deployed in the change manager. 
Thus, the observer representing the requirement seamlessly co-evolves with the requirement, which contrasts discarding and redeploying a new observer in the case of requirements changes.
Therefore, the observer's current state can persist through adaptation, which is beneficial as it reflects information obtained previously about the execution of the managed system and environment until the adaptation. This leverages an incremental verification (cf.~\cite{ghezzi2012evolution}) where previous knowledge is preserved for the runtime verification. 

Nevertheless, enacting an adaptation of the observer has to be synchronized with the runtime verification that uses the same observer so that the adaptation is \textit{safe}. Otherwise, adapting the observer while the runtime verifier traverses the observer and performs state transitions could lead to inconsistencies. To achieve safe adaptations, the observer can only be adapted when it is quiescent (cf.~\cite{Kramer:TSE90}). 
Therefore, the requirements manager adds a dedicated \textit{adaptation event} to the event queue of the change manager. When this event is processed by the runtime verifier, the adaptation of the observer is performed instead of a verification step. After the adaptation, the runtime verifier continues processing the monitored events and performing verification steps. This approach also ensures that all events monitored before the adaptation event are processed with the unchanged observer.
 
Since our observers are based on PSP, both the original and adapted property can be expressed in structured natural language to describe the adaptation. The PAP range from \textit{parameter} (i.e., updating time bounds or replacing event types corresponding to placeholders in MTL formula templates of PSP) to \textit{structural} adaptations (i.e., the structure of the underlying property and observer are adapted, e.g., by adding or removing a response in a response-chain property resulting  in novel or obsolete states in the observer).

In the following, we present five PAP. We outline them in natural language and formalize exemplarily two of them with graph transformations on observers.
For these PAP, we noticed that a seamless adaptation of the observer preserving the already obtained knowledge in contrast to redeployments and restarts of verification processes is beneficial. 
Still, we do not claim that there is no situation in which a redeployment and restart of the observer can be appropriate.

We present the following five PAP: 
a) updating a time guard,
b) updating an event,
c) adding a response to a chain,
d) removing a response from the chain,
and e) splitting the response chain into multiple response properties.
While patterns (a) and (b) cover parametric changes of the requirement, patterns (c), (d), and (e) cover structural changes. 


\paragraph{Updating a Time Guard}
This PAP covers changes of a deadline in a real-time requirement by adapting a property's time guard. 
Such an adaptation is performed by updating the corresponding guards in the observer and all timers. Adapting a time interval might change the valuation of guard conditions, and therefore enable previously disabled transitions. Thus, adapting a time interval might yield an immediate violation of the property. 
This PAP can be applied in this fashion to real-time properties of most patterns from the PSP catalog such as the Response, Existence, Absence, and Recurrence. 

\paragraph{Updating an Event}
This PAP is used to exchange one of the events specified in the property. In the observer template, such an adaptation can be performed by updating the labels of transitions from their old value, such as \textit{P}, to their new values, such as \textit{P'}.
This PAP can be applied to any pattern and observer in our catalog. 


\paragraph{Adding a Response to the Chain}

\begin{figure}[t!]
    \centering
    \begin{subfigure}{0.5\textwidth}
        \centering
        \includegraphics[width=.7\columnwidth]{figures/observer-template-add.png}
        \caption{Graph transformation rule for the PAP.}
        \label{fig:obs-template-add}
    \end{subfigure}%

    \begin{subfigure}{0.5\textwidth}
        \centering
        \includegraphics[width=.9\columnwidth]{figures/observer3.png}
        \caption{Resulting observer after adding a glucose sensor.}
        \label{fig:obs3}
    \end{subfigure}%
    \caption{PAP of adding a response to a response chain.}
    \label{fig:obs-templates}
    \vspace{-2em}
\end{figure}

This PAP is used when an additional response is expected to occur in the chain of responses, which extends the property. 
In essence, the response chain pattern defines a list of responses. The PAP allows the addition of a response to the end of the list, which is defined by the graph transformation rule (cf.~\cite{Giese+2012}) shown in Fig.~\ref{fig:obs-template-add}. The response to be added is defined by the rule's parameter $S_3$. This PAP can be applied to the observer template of the Response Chain pattern shown in Fig.~\ref{fig:obs-template} by matching the black and red elements of the rule in the observer template and afterwards performing the side effects of the rule, that is, removing the red elements (that are further annotated with $--$) and adding the green elements (that are further annotated with $++$) to the observer template. Accordingly, a new state $waiting_3$ is added to the observer template during which the new response $S_3$ is expected after response $S_2$ has occurred to move to state \textit{open}, otherwise to move to the \textit{error} state if the time bound has passed. 

However, the graph transformation rule shown in Fig.~\ref{fig:obs-template-add} is mainly considered as a specification of an observer adaptation at the pattern/template level that guarantees precise semantics of the adaptation according to the PSP. 
Thus, in a SAS, this rule is not applied on an observer template but it is rather itself a template. It will be instantiated for adapting a concrete property and observer. Instantiating and applying the rule to the observer of the BSN shown in Fig.~\ref{fig:obs} results in the observer shown in Fig.~\ref{fig:obs3}. Particularly, this adaptation reflects the requirements change that the BSN has to consider the data sensed and sent as \textit{glucose reply} by the glucose sensor, which have not been considered before. The response \textit{glucose reply} is expected to occur as the last response of the chain.

The observer's current state persists through the adaptation process, that is, the runtime verifier preserves the knowledge of which other responses of the chain have already occurred.  


\paragraph{Removing a Response from the Chain}
This PAP is used when an existing response is not needed anymore in the chain.
It is the counterpart of the previous PAP (adding a response).  
We now consider the removal of a response $S_1$ in the middle of the chain as shown by the graph transformation rule in Fig.~\ref{fig:obs-template-rem}. 
The rule is instantiated and applied similarly to the rule shown in Fig.~\ref{fig:obs-template-add}, but in this case to adapt the observer shown in Fig.~\ref{fig:obs3} to obtain the observer shown in Fig.~\ref{fig:obs2'}. This adaptation reflects the requirements change in the BSN that the thermometer is not needed anymore and therefore, the \textit{thermometer reply} is not expected to happen anymore.  

However, since this adaptation removes a state from the observer, we have to take into account that the state to be removed can be the \textit{current} state of the observer. If this is the case, a new current state has to be determined and set by the adaptation. 
If the current state is $waiting_i$ that should be removed by adaptation, we know that the property's scope is open (scheduler cycle has started), a request has occurred, and all responses prior to $S_i$ have happened.
In the adapted observer, the same information is represented by state $waiting_{i+1}$, since the state $waiting_i$ together with the response $S_i$ have been removed by adaptation. Thus, in the case that the \textit{current} state was removed, the current state of the observer after adaptation is set to $waiting_{i+1}$.

\begin{figure}[t!]
    \centering
    \begin{subfigure}{0.45\textwidth}
        \centering
        \includegraphics[width=.7\columnwidth]{figures/observer-template-remove.png}
        \caption{Graph transformation rule for the PAP.}
        \label{fig:obs-template-rem}
    \end{subfigure}%
     
    \begin{subfigure}{0.45\textwidth}
        \centering
        \includegraphics[width=.9\columnwidth]{figures/observer2+.png}
        \caption{Resulting observer after removing a thermometer.}
        \label{fig:obs2'}
    \end{subfigure}%
    \caption{PAP of removing a response from the chain.}
    \vspace{-2em}
\end{figure}

\paragraph{Splitting the Response Chain}
In a response chain, the order of expected responses is specified. If a specific order of the responses is no longer required, this PAP splits the chain into multiple, independent responses, that is, multiple response chains with only one response for each chain. Therefore, it generates multiple observers, one for each response, and maintains the information represented by the existing observer by selecting the current state for each of those new observers depending on the current state of the existing observer. 

Consider, for example, a response chain with two responses as shown in Fig.~\ref{fig:obs}. The observer's state \emph{closed} represents that the scope is closed, \emph{open} represents that the scope is open but no response is required because any previous request has already been replied, \emph{error} represents a violation of the property. Each of these three states can also be found in the new observers, where they still represent the same information. Hence, if the existing observer is in any of these states, the new observers will be set to that state as well. 
If the existing observer's current state is $waiting_i$, i.e., any of the states where a request has been sent but not all replies have occurred, the information for the new observers differs according to the response they are representing. Consider, for example, that $waiting_2$ is the current state in the existing observer. This means that the first response has occurred, but not yet the second. Thus, the new observer for the first response should have \emph{open} as the current state, representing that all requests have been addressed by a response, while the other new observer (for the second response) should have \emph{waiting} as its current state because a request has occurred but not yet the corresponding response (i.e., $S_2$). In general, all observers representing properties regarding responses $S_j$ with $j < i$ will have their current state set to \textit{open}, while the remaining observers will be in state \textit{waiting}. 

\smallskip

After an observer adaptation regardless of the PAP that is used, the runtime verifier checks all outgoing transitions of the current state in the observer since any adaptation may enable previously disabled transitions. Additionally, existing timers are discarded and new timers are set accordingly for updated timed guards in the observer. Afterwards the runtime verifier continues with regular verification steps by consuming monitored events from the queue (see Section~\ref{subsec:performingRV}). 

\section{Evaluation}\label{sec:evaluation}

We evaluate our observer-based runtime verification approach by investigating the following research questions: 
\begin{itemize}
    \item [RQ1:] How efficient is the observer-based runtime verification in terms of time for processing monitored events with observers and memory needed to represent observers?
    \item [RQ2:] How accurate is the observer-based runtime verification in detecting violations of properties? 
    \item [RQ3:] How fast is a dynamic adaptation of an observer at runtime compared to a redeployment of the observer? 
    \item [RQ4:] Can the observer-based runtime verification with adapting observers increase the trustworthiness of SAS?
\end{itemize}

To perform our evaluation, we implemented our runtime verification approach with its  observers and adaptations following the PSP and PAP with C++ on Arduino.\footnote{The PSP/PAP catalog, implementation, and replication package for the evaluation are available at: \url{https://www.github.com/HUB-SE/PAP/}} 
We deployed the code on multiple Arduino Mega\footnote{Arduino Mega 2560 Rev3, 8 KB SRAM, 16MHz clock speed.}. 

\noindent
\textbf{RQ1 } 
This research question addresses the efficiency of our observer-based runtime verification. To perform verification online, the runtime verifier is desired to process monitored events faster than the managed system and environment emit them. Thus, the verifier can provide fast results without the possibility of an overflowing event queue. 

To determine how much time it takes for our observer-based verification technique to process events, we generated ten artificial traces containing events of five different types. The traces have a length of $50,000$ events. We measured the time that our technique took with an artificial observer to process these traces. This artificial observer contains five states. Each state has five outgoing transitions, each labeled with one of the five event types.\footnote{We omit evaluating the costs of managing timers due to guarded transitions in the observer because they are similar to processing an event requiring in both cases to check all outgoing transitions of the observer's current state.} Thus, with each processed event of the trace, one transition will be enabled regardless of the current state of the observer.
For each processed event, any outgoing transition of the current state has to be checked until the transition with the matching label is found. Overall, the artificial observer has 25 transitions, which  in our experience is a realistic upper bound for an observer~\cite{vogel2023property}. 

We execute our runtime verification technique with the artificial observer on Arduino Mega against the ten artificial traces. On average, our technique took $6571.1ms$ to process a trace with a standard deviation of $\mu$\,$\leq$\,$7.2ms$.
Thus, on average it takes $0.13ms$ ($6571.1ms$/$50,000$ events) for our technique to process a single event with an observer. 
Prominent benchmarks for runtime verification such as Timescales~\cite{ulus2019timescales} provide traces that contain one event per \textit{ms}. Thus, we conclude that our observer-based runtime verification technique is sufficiently efficient concerning the execution time. This especially holds since we check only monitored events representing changes of the managed system or environment, where we consider a rate of one change per $ms$ as extraordinarily high with respect to our experience with the BSN. 

We also investigate the memory needed to represent an observer in a data structure implemented on Arduino in terms of SRAM usage. For this purpose, we use observers for nine properties with different combinations of patterns and scopes as well as the artificial observer discussed previously.
As shown in Table~\ref{tab:memory}, the observers consume between \textit{355} and \textit{1,136 bytes} of memory. For each observer, we list the number of states and transitions to illustrate the size of the observer. Such sizes are representative of properties following the PSP. 
We conclude that observers can be efficiently represented given their size in terms of states and transitions and multiple observers each representing a property can be deployed to one Arduino Mega that has \textit{8KB} of SRAM. 
\begin{boxA}
    Our runtime verification is efficient as it just requires $0.13ms$ on average to process a monitored event and between $355$ and $1,136$ \textit{bytes} of memory to represent an observer. 
\end{boxA}

\begin{table}[tbp]
    \begin{center}
        \caption{Size of observers for properties in terms of numbers of states (\#S) and transitions (\#T), and memory usage in bytes.}
        \label{tab:memory}
        \vspace{-.5em}
        \begin{tabular}{c c c c}
            \toprule
            \textbf{Pattern + Scope} & \textbf{\#S} & \textbf{\#T} & \textbf{Memory (bytes)}\\
            \midrule
            Absence After & 5 & 4 & 614 \\
            Absence Before & 5 & 4 & 558 \\
            Absence Between & 6 & 8 & 866 \\
            Recurrence Globally & 2 & 2 & 355 \\
            Recurrence Between & 4 & 5 & 605 \\
            Response Globally & 3 & 3 & 458 \\
            Response Between & 4 & 6 & 652 \\
            Response Chain Between, 2 responses & 6 & 11 & 940 \\
            Response Chain Between, 3 responses & 7 & 14 & 1,136 \\ \midrule
            Artificial observer & 5 & 25 & 1,047 \\
            \bottomrule
            \end{tabular}
    \end{center}
    \vspace{-2.25em}
\end{table}

\noindent
\textbf{RQ2 } 
In this research question, we investigate the correctness of our runtime verification. To this end, we implemented a trace generator according to the grammar of Timescales~\cite{ulus2019timescales}, which is a runtime verification benchmark. Each trace targets a property based on the PSP catalog and can be generated to either satisfy or violate the property. Thus, the generator provides the ground truth of whether a generated trace violates or satisfies the property.
We considered nine properties that follow the patterns and scopes shown in Table~\ref{tab:memory}.
For each property, we generated 20 different traces, each consisting of about 60 events. Ten of them satisfy and ten violate the property. 
Afterward, we instantiated the observer template of our PSP catalog for the property. We deployed the resulting observer and evaluated, whether it reaches an \textit{error} state when processing the trace. 
We found that our observers classified each of the 20 traces correctly for each of the nine properties. 
\begin{boxA}
    We can report 100\% accuracy in detecting property violations since our observer-based runtime verification has provided correct results for all 180 runs of the experiment (20 traces for each of the nine properties).
\end{boxA}

\begin{table*}[tbp]
    \begin{center}
        \caption{Requirements changes for the BSN. Added/updated parts of the property along the changes are highlighted in blue.}
        \label{tab:scenarios}
        \begin{tabular*}{\textwidth}{c L{0.12\textwidth} L{0.13\textwidth} L{0.67\textwidth}}
            \toprule
            \textbf{\#} & \textbf{Req. Change} & \textbf{PAP} & \textbf{MTL Property} \\
            \midrule
            0 & Initial situation (cf. Eq.~\ref{eq:req1}) & -- &$\square (( \text{cycle\_starting} \wedge \lozenge \text{cycle\_ending} ) \rightarrow (\text{request} \rightarrow (\neg \text{cycle\_ending } \text{ }\mathcal{U}^{[0,2]}\text{ } (\text{thermometer\_reply} \wedge \linebreak \neg \text{cycle\_ending} \wedge (\lozenge ^{[0, 2]} (\text{pulse\_reply} ))  ))) \text{ }\mathcal{U}\text{ } \text{cycle\_ending})$\\ \midrule
            1 & Add a glucometer & Adding a Response to the Chain & $\square (( \text{cycle\_starting} \wedge \lozenge \text{cycle\_ending} ) \rightarrow (\text{request} \rightarrow  (\neg \text{cycle\_ending} \text{ }\mathcal{U}^{[0,2]}\text{ }  (\text{thermometer\_reply} \wedge \linebreak \neg \text{cycle\_ending} \wedge (\lozenge ^{[0, 2]} (\text{pulse\_reply} )) \textcolor{blue}{\wedge \neg \text{cycle\_ending} \wedge (\lozenge ^{[0, 2]} (\text{glucose\_reply} ))} ))) \text{ }\mathcal{U} \text{ }\text{cycle\_ending})$\\ \midrule
            2 & Update time guard & Updating a Time Guard & $\square (( \text{cycle\_starting} \wedge \lozenge \text{cycle\_ending} ) \rightarrow (\text{request} \rightarrow (\neg \text{cycle\_ending} \text{ }\mathcal{U}^{[0,\textcolor{blue}{3}]}\text{ } (\text{thermometer\_reply} \wedge \neg \text{cycle\_ending} \wedge  (\lozenge ^{[0, \textcolor{blue}{3}]} (\text{pulse\_reply} )) \wedge \neg \text{cycle\_ending} \wedge (\lozenge ^{[0, \textcolor{blue}{3}]} (\text{glucose\_reply} )) ))) \text{ }\mathcal{U}\text{ } \text{cycle\_ending})$\\ \midrule
            3 & Remove the thermometer & Rem. a Response from the Chain & $\square (( \text{cycle\_starting} \wedge \lozenge \text{cycle\_ending} ) \rightarrow (\text{request} \rightarrow  \neg \text{cycle\_ending} \wedge  (\lozenge ^{[0, 3]} (\text{pulse\_reply} )) \wedge \neg \text{cycle\_ending} \wedge (\lozenge ^{[0, 3]} (\text{glucose\_reply} )) ) \text{ }\mathcal{U}\text{ } \text{cycle\_ending})$\\ \midrule
            4 & Scheduler requests data & Updating an Event & $\square (( \text{cycle\_starting} \wedge \lozenge \text{cycle\_ending} ) \rightarrow (\text{\textcolor{blue}{s\_request}} \rightarrow  (\neg \text{cycle\_ending} \text{ }\mathcal{U}^{[0,3]}\text{ } (\text{pulse\_reply} \wedge \neg \text{cycle\_ending} \wedge (\lozenge ^{[0, 3]} (\text{glucose\_reply} ))  ))) \text{ }\mathcal{U} \text{ }\text{cycle\_ending})$\\ \midrule     
            5 & Neglect order of sensors & Splitting the Response Chain & $\square (( \text{cycle\_starting} \wedge \lozenge \text{cycle\_ending} ) \rightarrow (\text{s\_request} \rightarrow  (\neg \text{cycle\_ending} \text{ }\mathcal{U}^{[0,3]}\text{ } (\text{pulse\_reply}))) \text{ }\mathcal{U}\text{ } \text{cycle\_ending})$ -- \textit{and a similar property for {glucose\_reply}}\\

            \bottomrule
            \end{tabular*}
    \end{center}
    \vspace{-2em}

\end{table*}

\noindent
\textbf{RQ3 }
This research question addresses the performance of a dynamic adaptation of a property at runtime. Thus, we compare the runtime efficiency of a \textit{dynamic adaptation} based on our PAP and a \textit{redeployment} of an observer. A redeployment comprises invoking the destructor to free up the memory consumed by the observer and the constructor to instantiate and represent the new observer in freshly allocated memory. 
For this experiment, we use the Response Chain property shown in Eq.~\ref{eq:req1}. For requirements changes, we alternate between adding and removing responses from the chain as well as updating response events in the chain. Such changes can easily be repeated multiple times on an observer to achieve reliable time measurements. For one run, we alternate between the changes until each of them is performed $1,000$ times resulting in a total of $3,000$ changes that are either realized by $3,000$ dynamic adaptations or $3,000$ redeployment of the observer. We repeat both runs ten times. The runs are all executed on Arduino.

On average across the ten runs, the $3,000$ dynamic adaptations of the observer took in total $3.034s$ (stdev $\mu_1$\,$\leq$\,$0.49ms$). 
Thus, one dynamic adaptation of an observer takes on average \textit{1.01ms}. 
In contrast, the $3,000$ redeployments of the observer took on average $15.330s$ (stdev $\mu_2$\,$\leq$\,$0.85ms$), that is, on average \textit{5.11ms} for one redeployment of an observer. 

\begin{boxA}
    A dynamic adaptation of an observer ($1.01ms$) is more than five times faster than a redeployment ($5.11ms$). 
\end{boxA}

\noindent
\textbf{RQ4 }
For the last research question, we investigate how our approach of dynamically adapting observers to reflect requirements changes can increase the trustworthiness of a SAS. To this end, we use a port of the BSN artifact~\cite{BSN} we implemented for the Arduino platform.
Starting with an initial situation of the BSN described by the requirement discussed in Section~\ref{subsec:initialzingRV} and formalized by the MTL property in Eq.~\ref{eq:req1}, we consider a sequence of five requirements changes shown in Table~\ref{tab:scenarios}. For each requirements change, the table shows the PAP to specify the adaptation of the property/observer and the MTL property after adaptation. We execute the BSN alongside our runtime verification approach and use the PAP to specify and perform the dynamic adaptations of the observer to reflect sequentially these five requirements changes in the verification.

With this demonstration of our approach, we show that using PAP allows us to specify adaptations of observers with precise semantics as shown by the corresponding MTL properties before and after an adaptation (cf.~Table~\ref{tab:scenarios}). Such adaptations of observers are dynamically and safely performed so that our approach preserves the knowledge---in terms of intermediate verification results as progress made in the observer---without compromising the integrity of the observer. 
For the given property that is adapted (Table~\ref{tab:scenarios}), the knowledge preserved in the observer comprises whether a scheduler cycle has already started and if so, which of the sensors already have and which still need to send data to the BodyHub. 
Thus, our approach achieves an incremental, continuous verification of the currently executing scheduler cycle against the adapted property.
Without preserving this knowledge (e.g., by a redeployment), the currently executing scheduler cycle remains unverified against the adapted property as the observer is reset to its initial state where it expects a novel cycle to start (cf.~\textit{cycle\_starting} event). In such a situation, there is no verification evidence about the safety of the BSN. 
\begin{boxA}
    Applying the PAP enables a continuous, incremental verification of the BSN that increases the trustworthiness of the BSN when requirements changes occur. 
\end{boxA}

\noindent
\textbf{Discussion}
In our evaluation, we have shown the efficiency of our observer-based runtime verification ($0.13ms$ to process a monitored event and at most $1.136$ bytes to represent a monitor) and adaptation of properties based on PAP ($1.01ms$ to dynamically adapt an observer). Thus, our approach can efficiently be used on microcontrollers such as Arduino. 

Moreover, we have shown empirically the correctness of our observer-based runtime verification using a benchmark based on Timescales~\cite{ulus2019timescales} as ground truth. Since there is no ground truth for verifying a running system against adapted properties, we cannot validate the correctness of our verification approach under changing requirements. Thus, we demonstrated qualitatively the benefits of continuous, incremental runtime verification on the trustworthiness of the~BSN~\cite{BSN}. 

\noindent
\textbf{Threats to Validity}
Threats to the validity of our study are as follows. 
\textit{Construct:}
Potential errors in our implementation of the observers and the Timescales grammar cause a threat to the validity of our reported results on correctness. We address this threat by having reviewed the observers and making the implementation and replication package publicly available. 
\textit{Internal:}
Threats of this category concern the experiments and measurements we conducted. To mitigate measurement errors and obtain reliable results, we repeated experiments and performed them on a SEAMS artifact ported to Arduino and on a benchmark based on Timescales that is used by runtime verification research community. 
Moreover, requirements formalized with the Structured English Grammar~\cite{AutiliGLPT15} might not match the stakeholders' intentions, which is also true for the properties/observers and eventually for the verification results. In this context, we rely on our expertise on the BSN~\cite{BSN,Solano+2019} and property specification patterns~\cite{AutiliGLPT15,vogel2023property}.  
\textit{External:}
We considered only the BSN with one requirement that changes in five ways in our study. Thus, our results may not generalize to other SAS, requirements, and changes.
Finally, our PSP/PAP catalog currently supports four PSP with different scopes and five PAP, two of which can be applied to all four PSP and three only to the Response Chain pattern. Thus, we cannot generalize our catalog to other patterns collected in~\cite{AutiliGLPT15}.

\section{Related Work}

\textbf{Topological Map in Exploration and Navigation.} Inspired by the animal and human psychology~\cite{tolman1948cognitive}, a large amount of work has recently proposed to build topological map to represent an environment~\cite{graphtopoexplore,Murphy08ICRA,neural_topomap,beeching2020learning,savinov2018semiparametric,VisualGraphMem_ICCV21,MRTopoMap,Savarese-RSS-19}. They use the topological map for tasks such as navigation~\cite{neural_topomap,learn2explore_iclr20,savinov2018semiparametric,VisualGraphMem_ICCV21,Savarese-RSS-19}, exploration~\cite{learn2explore_iclr20,graphtopoexplore,Murphy08ICRA,savinov2018semiparametric,MRTopoMap,TSGM} and planning~\cite{beeching2020learning}. To build the topological map, they combine various sensors such as RGB image, depth map~\cite{TSGM,Savarese-RSS-19}, pose~\cite{neural_topomap,learn2explore_iclr20,beeching2020learning} and even LiDAR scanner~\cite{MRTopoMap,graphtopoexplore}. Some of them further adopt data-hungry and computation-demanding Reinforcement Learning~(RL) techniques to train the model to construct the topological map~\cite{neural_topomap,learn2explore_iclr20,VisualGraphMem_ICCV21}. Kwon \textit{et al.}~\cite{VisualGraphMem_ICCV21} combine imitation learning~(IL) and RL to train the model. Some of these methods~\cite{neural_topomap,learn2explore_iclr20,beeching2020learning} involve metric information to construct the topological map. N.~Savinov~\textit{et. al.}~\cite{savinov2018semiparametric} use the random walk to construct the topological map, which inevitably leads to an inefficient topological map. TSGM~\cite{TSGM} jointly adds surrounding objects during topological map construction. Unlike these prior works, our \textit{\acronym{}} is completely metric-free and simple in experimental configuration~(just RGB image, much smaller expert demonstration size).

\textbf{Hallucinating Future Feature.} The idea of hallucinating future latent features has been discussed in other application domains. Previous work has utilized this idea of visual anticipation in video prediction/human action prediction~\cite{16Vondrick,17Zeng,20Chang,21Fernando,Suris2021LearningTP}, and researchers have applied similar ideas to robot motion and path planning~\cite{Jain2016RecurrentNN, Koppula2016, Carlone2019, Park2016}. As stated in~\cite{16Vondrick,17Zeng,Suris2021LearningTP}, visual features in the latent space provide an efficient way to encode semantic/high-level information of scenes, allowing us to do planning in the latent space, which is considered more computationally efficient when dealing with high-dimensional data as input~\cite{Lippi2020,Ichter2019}. Different from previous robotics work, we take advantage of this efficient representation by adding deep supervision when anticipating the next visual feature, which was computationally intractable if we were to operate at the pixel level.

\textbf{Deeply-Supervised Learning} has been extensively explored~\cite{deeply_supervised_nets,knowledge_synergy,li2017deep,li2018deep} during the past several years. The main idea is to add extra supervision to various intermediate layers of a deep neural network in order to more effectively train deeper neural networks. In our work, we adopt a similar idea to deeply supervise the training of feature hallucination and action generation.


\begin{figure*}[t]
    \centering
    \includegraphics[width=0.8\linewidth]{Topo_map.pdf}
    \caption{\textbf{Training and inference for task and motion imitation.} Feature extractor $g_\psi$ takes image $I_t$ as input and generates the corresponding feature vector $f_t$. \textit{TaskPlanner} $\pi_{\theta_T}$ is a recurrent neural network (RNN) consuming a sequence of features $\{ f_{t-10}, \cdots, f_t\}$ to hallucinate the next best feature to visit $\hat{f}_{t+1}$. \textit{MotionPlanner} $\pi_{\theta_M}$ consumes the concatenation (denoted by $\bigoplus$) of ${f}_{t}$ and $\hat{f}_{t+1}$ and generates the action to move the agent towards the hallucinated feature. During training, we supervise all the intermediate outputs including the intermediate hallucinated features $\{ \hat{f}_{t-9}, \cdots, \hat{f}_{t} \}$ and the intermediate actions $\{ \hat{a}_{t-10}, \cdots, \hat{a}_{t-1} \}$, in addition to the final output $\hat{f}_{t+1}$ and $\hat{a}_t$. During inference, current observation $I_t$ is firstly encoded and fed into $\pi_{\theta_T}$ to hallucinate $\hat{f}_{t+1}$, and then $\hat{f}_{t+1}$ combined with the ${f}_{t}$ is fed into $\pi_{\theta_M}$ for motion planning. $\mathcal{L}_T$ is $L_2$ loss and $\mathcal{L}_M$ is cross entropy loss (the subscripts $T$ and $M$ denote \textbf{T}ask and \textbf{M}otion respectively). $h_t$ denotes the hidden state of RNN.} 
    \label{fig:pipeline}
    \vspace{-5mm}
\end{figure*}

\textbf{Task and Motion Planning.} Task and motion planning (TAMP) divides a robotic planning problem into high-level task allocation (task planning) and low-level action for task execution (motion planning). This hierarchical framework is adopted in many robotic tasks such as manipulation \cite{chitnis2016guided,mcdonald2022guided} exploration~\cite{Cao-RSS-21} and navigation \cite{lo2018petlon,thomas2021mptp}. Such a framework allows us to leverage high-level information about the scenes to tackle challenges in local control techniques~\cite{bansal2019-lb-wayptnav}. In this work, to perform active topological mapping of a novel environment, the agent firstly reasons at the highest level about the regions to navigate: hallucinate the next best feature point to visit. Afterward, the agent takes an action to get to the target feature. The whole procedure is totally implemented in feature space without any metric information.

\textbf{Imitation Learning} aims to mimic human behavior or expert demonstrations for a given specific task~\cite{imitation_learning,il_legged,il_planning}. The agent is trained to perform tasks by directly observing demonstrations~\cite{il_legged,il_planning}. In our work, the expert demonstration is a set of image-action pair sequences that an agent would observe along a route that efficiently covers an environment. It is widely accessible in either real-world or simulated environments~(e.g. from human experts or maps of environments). 
%\section{}
%\label{sec:resDir}


\section{Conclusion}
\label{sec:conclusion}
% <>
Since its advent in 1931, Koopman operator theory \cite{koopman:1931} has only recently been actively utilized for solving practical problems, thanks to the introduction of the DMD algorithm in 2008 \cite{schmid:2008}. Since then, a multitude of DMD algorithm variations have risen to prominence and found utility across various fields. A notable feature of our survey paper was reviewing and categorizing the results of over 100 research papers based on both application and algorithm type in smart mobility and vehicle engineering  (see Table~\ref{tab1} and Section~\ref{sec:vehicApp}).  Additionally, this survey paper identified potential research gaps in smart mobility and vehicular engineering applications (Remarks~\ref{remGap1}--\ref{remGap6}). Finally, this review paper discussed theoretical aspects of Koopman operator theory that have been largely neglected by the smart mobility and vehicle engineering community and yet have large potential for contributing to solving open problems in these areas (see Section~\ref{subsec:theorIssue}).

\noindent{\textbf{Future Research Directions.}}	Given the emergence of cyber-threats against connected and autonomous vehicles as well as robotic systems (see, e.g.,~\cite{nekouei2021randomized,mohammadi2022generation}), a future research direction might include utilizing Koopman operator-based algorithms for designing cyber-resilient vehicular and smart mobility applications (see, e.g.,~\cite{taheri2022data} for a related line of research). Another potential research direction is using Koopman operator-based algorithms for predicting the motion of vulnerable road users (VRUs), e.g., pedestrians and cyclists (see, e.g.,~\cite{pool2019context,scholler2020constant}). Finally, rehabilitation robotics and robotic exoskeletons can be the benefactors of the predictive capabilities of Koopman operator-based algorithms for detecting tripping events and/or system  identification in various modes of locomotion (see, e.g.,~\cite{kumar2019extremum,aprigliano2019pre}).



%Fig. 1 depicts the accumulation of such algorithms since 2014, which are particular to vehicle engineering and smart mobility, i.e., the focus of this review. Table 1 summarizes the varieties of relevant algorithms developed in those studies. Furthermore, we have highlighted theoretical issues, whose expansion will have potential applications to the wide research area of smart mobility and vehicle engineering.  

%Although fairly comprehensive, we have found several gaps in this research area. In particular, we could not find any studies related to elevators, robots/vehicles employing crawling, slithering, hopping or peristaltic locomotion, arctic or special-terrain vehicles such as those employing screws or tracks, hovercraft and other amphibious vehicles or subsystems which tolerate flexible environments, classification or guidance systems related to vehicles for drilling or agriculture, or for current-ripple, power-split, battery health monitoring, nuclear propulsion, exoskeletons/prosthetics, personal mobility, motorsports, specialized rovers or similar open problems in emerging areas.  These examples are, of course, not exhaustive.  
%
%The purely data-driven nature of Koopman operators holds the promise of capturing unknown and complex dynamics for reduced-order model generation and system identification, through which the rich machinery of linear control techniques can be utilized. The emergent nature of the smart mobility and vehicular-related applications, where  the Koopman operator  in each particular application needs to be approximated, implies that the development of various Koopman operator approximation  algorithms is expected to grow along with the vehicular problems they aim to solve.  Given the ongoing development of this research area and the many existing open problems in the fields of smart mobility and vehicle engineering, a survey of techniques and open challenges of applying Koopman operator theory to this vibrant area is warranted.  To the best of our knowledge, this survey paper is the \emph{first of its kind} reviewing the applications of Koopman operator theory within a focused research area, namely, smart mobility and vehicle engineering applications. A \emph{notable feature} of our survey paper is reviewing and categorizing the results of over 100 research papers based on both application and algorithm type  (see Tables~\ref{tab1}--~\ref{tab4} and Section~\ref{sec:vehicApp}) that are concerned with the applications of Koopman operator theory to the field of smart mobility and vehicular engineering. Such a \emph{comprehensive and  detailed categorization} will be beneficial to the research practitioners working in the field.  Furthermore, this review paper discusses theoretical aspects of Koopman operator theory that have been largely neglected by the smart mobility and vehicle engineering community and yet have large potential for contributing to solving open problems in these areas. Additionally, our survey paper seeks to \emph{identify gaps} in the smart mobility and vehicle engineering research where new and existing Koopman operator-based methods have the potential to further develop and address unsolved problems  potentially benefiting from the perspectives of nonlinear system identification, control, global linearization, and the predictive powers that Koopman operator theory has to offer (see, e.g., Remarks~\ref{remGap1}--\ref{remGap6}). 


\clearpage
\balance
\bibliographystyle{IEEEtran}
\bibliography{bibliography}
\end{document}
