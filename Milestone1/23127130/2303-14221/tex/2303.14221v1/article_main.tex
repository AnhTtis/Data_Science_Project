\documentclass[fleqn,10pt]{wlscirep}

\usepackage{authblk}
\usepackage{capt-of}
\usepackage{subfigure}
\usepackage{wrapfig}
\usepackage{hyperref}       % hyperlinks
\usepackage{url}            % simple URL typesetting
\usepackage{booktabs}       % professional-quality tables
\usepackage{amsfonts}       % blackboard math symbols
\usepackage{nicefrac}       % compact symbols for 1/2, etc.
\usepackage{microtype}      % microtypography
\usepackage{lipsum}
\usepackage{listings}
\usepackage{fancyhdr}       % header
\usepackage{graphicx}       % graphics
\usepackage{chemformula}
\graphicspath{{png/}}    % organize your images and other figures under specified folder
\usepackage[section]{placeins}
\usepackage{tabularx}
\usepackage{subfigure}
% \hbadness=1150
\raggedbottom

\usepackage[utf8]{inputenc}
\usepackage[T1]{fontenc}
\usepackage{caption}
% \usepackage{subcaption}
\usepackage{booktabs}
\usepackage{float}
\usepackage{parskip}
\usepackage{hyperref}
\usepackage{multirow}
\usepackage{tablefootnote}
\usepackage{threeparttable}
\usepackage{chemformula}
\usepackage{xcolor}
\usepackage[section]{placeins}
% \usepackage{natbib}
% \hbadness=1150
% \raggedbottom
\usepackage{setspace}
% Keywords command
% \providecommand{\keywordsdisplay}[1]
% {
%   \textbf{\textit{Keywords---}} #1
% }
\begin{document}
\title{The Battle of Information Representations: Comparing Sentiment and Semantic Features for Forecasting Market Trends}

\author[1]{Andrei ZAICHENKO}
\author[2]{Aleksei KAZAKOV}
\author[2]{Elizaveta KOVTUN}
\author[2,3,*]{Semen BUDENNYY}

\affil[1]{Higher School of Economics University, Moscow}
\affil[2]{Sber AI Lab, Moscow}
\affil[3]{Artificial Intelligence Research Institute (AIRI), Moscow}

\affil[*]{Corresponding author: budennyysemen@gmail.com }

\begin{abstract}
% The study of the stock market using machine learning techniques is an important task that can reveal patterns and dynamics that are difficult to detect using traditional methods. This knowledge can help researchers better understand the behavior of financial markets and develop new models to explain their movements. Despite its significance, most research in stock movement prediction relies on extracting sentiment from the text and using it as the primary feature for prediction, leaving little exploration of the potential of semantic features and context in financial market trends. In this paper, we aim to fill this gap in research by testing the hypothesis that semantic features and context are important for stock price prediction. Our approach uses sentence embeddings extracted from Twitter data to capture semantic information and contextual relationships with financial market trends. We compare our approach to traditional sentiment-based solutions to evaluate its performance and potential impact on the field of stock price prediction.
% To achieve this, we train multiple models with a custom loss function on a dataset consisting of Twitter data and stock prices. Our results show that the use of sentence embeddings in our experiment lead us to a discovery that eliminates the need to solve the problem of sentiment extraction without a significant loss in quality for final predictions. Our findings contribute to the field of stock price prediction by highlighting the advantages and limitations of the use of sentence embeddings in this task. They provide a valuable insight into the cases where it delivers better results in comparison with baseline sentiment approach. In multiple experiments both sentiment and semantic approaches demonstrated similar performance proving the initial hypothesis of the semantic features importance in predicting financial market trends. In the future works section, we suggest possible continuations for further research to explore alternative methods for incorporating semantic and contextual information into stock price prediction models.

The study of the stock market with the attraction of machine learning approaches is a major direction for revealing hidden market regularities. This knowledge contributes to a profound understanding of financial market dynamics and getting behavioural insights, which could hardly be discovered with traditional analytical methods. Stock prices are inherently interrelated with world events and social perception. Thus, in constructing the model for stock price prediction, the critical stage is to incorporate such information on the outside world, reflected through news and social media posts. To accommodate this, researchers leverage the implicit or explicit knowledge representations: (1) sentiments extracted from the texts or (2) raw text embeddings. However, there is too little research attention to the direct comparison of these approaches in terms of the influence on the predictive power of financial models. In this paper, we aim to close this gap and figure out whether the semantic features in the form of contextual embeddings are more valuable than sentiment attributes for forecasting market trends. We consider the corpus of Twitter posts related to the largest companies by capitalization from NASDAQ and their close prices. To start, we demonstrate the connection of tweet sentiments with the volatility of companies' stock prices. Convinced of the existing relationship, we train Temporal Fusion Transformer models for price prediction supplemented with either tweet sentiments or tweet embeddings. Our results show that in the substantially prevailing number of cases, the use of sentiment features leads to higher metrics. Noteworthy, the conclusions are justifiable within the considered scenario involving Twitter posts and stocks of the biggest tech companies. 

% However, for the most part, two approaches demonstrate similar performance, leaving us with the option to omit an additional model for sentiment extraction and use text embeddings. 
% We provide the characteristics of the conditions under which the particular information representation might be more beneficial. 

% For one who faces choice between two information representations, we provide the description of the situations when one form might be more beneficial than another. 

% It is important the limitation of conducted research is that we make a comparison only within restricted use case - Twitter posts and stocks of companies from NASDAQ index.  

\end{abstract}

\maketitle

\begin{keywordname}
stock market, embedding, time series, event study, NLP
\end{keywordname}
\section{Introduction}

Stock price prediction is a challenging problem in the financial domain that draws significant interest from researchers. There is a massive amount of groundwork on machine learning applications for forecasting financial market trends. The developed methods could be divided into two streams: fundamental and technical analysis. 

Technical analysis relies solely on historical structured data of stock markets. Such analysis is extensively used with machine learning techniques
\cite{AYALA2021107119, PENG2021100060}. These studies demonstrate that technical indicators such as Exponential Moving Average (EMA) and Moving Average
Convergence/Divergence (MACD) can be used to increase the profitability of trading signals.

In contrast, fundamental analysis focuses on any useful information outside of historical market data regarding
the stock in question, such as the financial environment, law regulations, social networks, geopolitical stability, and news. 
While both approaches are usually used separately, recent studies show that a combination of the two can yield more accurate predictions \cite{Nti2020}.

With the advancements in machine learning and natural language processing (NLP), researchers explore the leveraging
of various features extracted from textual data for predicting stock prices. Such NLP technique as sentiment analysis allows the extraction of the sentiment or emotion from a piece of text, which can be used to infer the overall sentiment of the market towards a particular company or stock. 
Information representation in the context of financial markets is proposed to be viewed as explicit or implicit. 

The implicit representation involves extracting sentiment polarity directly from text (by a pre-trained NLP classifier) and then using this information to assess the expected reaction of a signal. This typically involves classifying the text into positive or negative categories based on its perceived sentiment towards market and using this classification as a feature in a further machine learning model to predict price movements. This approach has the advantage of being relatively simple and straightforward to implement, but it has the drawback of losing important semantic features and contextual information in the text. Moreover, the implicit approach relies on the product of the used sentiment analysis algorithm, whose inaccurate operation can introduce distortions into the work of the final predictive model. 

The explicit approach involves creating an embedding from the text directly and then using this embedding to predict the reaction for signal change \cite{10.1007/978-3-030-34223-4_5}. An embedding is a mathematical representation of a text in a lower-dimensional space, which can capture the meaning and context of the word or phrase. By creating an embedding from text, this approach is able to retain more of the semantic links contained in the text, which may be useful for predicting financial market trends. However, this approach may be more complex and time-consuming to implement, as it involves creating and training an embedding model on the text data.

After analyzing two ways of representing information related to the stock market, we can conclude that each of them has its own strengths and weaknesses. In this work, our aim is to compare the performance of the stock price prediction model built on the basis of either implicit or explicit knowledge representations.



% \input{sections/introduction_revised}
\section{Contribution}

In this research, our goal is to explore the effectiveness of the explicit embedding vector approach and compare it with the more established implicit sentiment solution. The main contributions achieved during the study of this topic are as follows:

\begin{itemize}
    \item Demonstrated the statistical dependence between stock price volatility and Twitter post sentiments.
    \item Proved intrinsic linkage between two kinds of representations 
    by showing that embeddings hold information on sentiments.
    \item Discovered the superiority of the sentiment extraction approach over the usage of embedding vectors in the prevailing number of cases. 
\end{itemize}

\section{Related work}
% Stock price prediction is a challenging problem in the financial domain and has attracted significant attention from researchers
% over the years. 

One of the main streams of the research is dedicated to studying the usage of purely technical analysis trading indicators and historical data in combination with statistical methods for stock price prediction with machine learning. Many researchers employ the GRU-based models in this tasks \cite{Aseeri_2023, app13010222, GUPTA2022117986} while the others explore the transformer architecture in this field \cite{WANG2022118128, numhtml, 10027785}. The goal of such research is to provide investors and other financial actors with an insight of price movements to make more informed financial decisions relying on technical analysis as the foundation.

With the development of NLP methods, the number of works aiming to predict stock price trends and volatility proposed a combination of financial news and social media data is increasing \cite{Khan2022}. \cite{10.1145/3159652.3159690} pointed out the lack of trustworthiness and comprehensiveness of online content collected from social media and low quality news sources. Clinical trial announcements were used as a source of sentiment in pursuit of predicting pharma stock market price changes by \cite{https://doi.org/10.48550/arxiv.2208.07248}.  Another research used Valence Aware Dictionary and Sentiment Reasoning (VADER) \cite{hutto2014vader} for sentiment analysis \cite{math10122001}. \cite{Li2022} proposed a novel Deep Learning Transformer Encoder Attention (TEA) model. \cite{https://doi.org/10.48550/arxiv.2005.02527} considered such under-explored content as Environmental, Social, and Corporate Governance (ESG) news flow for volatility forecasting. \cite{AUDRINO2020334} analyzed the impact of sentiment and attention variables on the stock market volatility by adding search engine and information consumption data on top of widely used social media and news texts.

Sentiment analysis involves extracting the sentiment
or emotion from a piece of text, which can be used to infer the overall sentiment of the market towards a particular company
or stock. Sentiment scores are commonly used in stock price prediction studies \cite{Khan2022, 10.1145/3159652.3159690, math10122001, Li2022, https://doi.org/10.48550/arxiv.2005.02527, AUDRINO2020334} as they are easy to compute and provide a
simple metric to gauge market sentiment.
On the other hand, embedding vectors are dense numerical representations of words or phrases that capture the semantic
meaning of the text. They are created by mapping words or phrases to high-dimensional vectors in a way that similar words or phrases are located close to each other in this vector space using techniques like Word2vec \cite{https://doi.org/10.48550/arxiv.1301.3781} or GloVe \cite{pennington-etal-2014-glove}, BERT \cite{https://doi.org/10.48550/arxiv.1810.04805} and GPT \cite{https://doi.org/10.48550/arxiv.2202.08904}, and have been shown to
be effective in capturing complex relationships between words and phrases. 

There are several works that concentrate on the usage of text semantics in the context of stock price prediction. \cite{10.1007/978-3-030-34223-4_5} proposed a Multi-head Attention Fusion Network to exploit aspect-level semantic information from texts to enhance prediction. \cite{Lin2022} developed a Spatial-temporal attention-based convolutional network. The authors converted news articles into 300-dimensional vector embeddings and used them as a feature in their model. They noted that in the case of the utilization of the preprocessed text features, latent information in the text is lost because the relationships between the text and stock price are not considered. \cite{Chandola2022} employed Word2Vec and LSTM algorithms, while \cite{Chen2022} adopted a transformer architecture using high level textual features. The main conclusion derived from these papers can be formulated as follows - the effectiveness of exploiting and fusing semantic aspect-level textual information leads to an improved performance upon the baselines. This fact means that the topic needs further investigation and refining.

% The main advantage of the embeddings over sentiment analysis is that they capture more complex relationships in the data. While sentiment analysis can indicate whether a tweet or news article about a certain company has either negative or positive attitude towards the stock the embeddings try to capture how that text relates to other pieces of the information allowing for a more deep and sophisticated analysis without dropping the important semantic features. For example in one recent study \cite{10.1007/978-3-030-34223-4_5} the authors claim that their Multi-head Attention Fusion Network utilizing text semantics shows the superior performance against several strong baselines.

The main advantage of embeddings over sentiment analysis is that they capture more complex relationships in the data. Despite the promising results of using embedding vectors as a feature \cite{Chandola2022, Chen2022}, there is still a research gap in the comparison of the effectiveness of using embedding vectors versus sentiment scores in stock price prediction. Moreover, reserachers tend to experiment with different datasets and methodologies, making it difficult to draw meaningful comparisons.

% While sentiment scores are commonly used as a feature in stock price prediction, they have limitations. For
% instance, they rely on the accuracy of the sentiment analysis algorithm used and may not capture the nuances of market
% sentiment. In contrast, embedding vectors are based on the underlying semantic relationships between words and phrases and
% are less prone to such limitation.

% In conclusion, while both sentiment scores and embedding vectors have been shown to be effective features for stock price
% prediction, there is a need for more research to directly compare the effectiveness of the two approaches. This research gap
% presents an opportunity for future studies to explore the potential of using embedding vectors as a feature in stock price
% prediction and compare their effectiveness with that of sentiment scores

% \input{sections/introduction_revised}
% \input{sections/research_gap}


% \input{sections/related_works_revised}
% \input{sections/data_description}
\section{PoseRAC Model}
\label{sec4}

\begin{figure*}[t]
\centering
\includegraphics[width=1.0\textwidth]{figure5.pdf}
\caption{Overview of our proposed PoseRAC. For a input video, the repetitive count can be obtained through Pose Estimation, Transformer Encoder, Pose Mapping and Action-trigger, where only the Encoder and the Pose Mapping need to be trained. We use Triplet Margin Loss to train the Encoder while Binary Cross Entropy Loss to train both the Encoder and the Pose Mapping. In addition to achieving the state-of-the-art performance so far, the biggest highlight of our PoseRAC is that it is lightweight enough to be easily trained on a CPU.}
\label{fig5}
\end{figure*}

Given a video $V={\{x_i\}}^{T}_{1}\in \mathbb{R}^{C\times H\times W\times T}$ with $T$ RGB frames, repetitive action counting model aims to predict a certain value $Y$, which is the number of repetitive actions. In this section, we will introduce our PoseRAC in detail.

\subsection{Model Overview}

As shown in Figure \ref{fig5}, PoseRAC consists of four parts. 

\begin{itemize}

\item The first is a state-of-the-art and lightweight Pose Estimation Network~($\S\ref{first}$), which is used to estimate the poses represented by lots of human pose key points from each frame of the original video sequence. 

\item The second is a simple Transformer Encoder~($\S\ref{second}$) to embed the key points of poses into high-level feature space, where the same class have similar distances, while the distances of different classes are far apart.

\item The third is a Pose Mapping Module~($\S\ref{third}$), where the unique mapping relationship between the salient poses and the action classes can be learned. Each pose can be mapped to the action class with the highest probability after the previous encoding.

\item The fourth part is a lightweight Action-trigger Module~($\S\ref{fourth}$). When we get the salient action classification results of all frames of the entire video sequence, we can use this module to calculate the repetition count in a short time.

\end{itemize}

\subsection{Pose Estimation Network}
\label{first}
Our model first converts the video sequence into a sequence of human pose key points, which can be defined as: 
\begin{equation}
\begin{split}
&V={\{x_i\}}^{T}_{1}\in \mathbb{R}^{C\times H\times W\times T}\\
&V\xrightarrow{\mathrm{Pose Estimation}} P={\{p_i\}}^{T}_{1}\in \mathbb{R}^{D\times K\times T}
\end{split}
\label{eq1}
\end{equation}
where each $x_i$ represents a single RGB frame, and each $p_i$ represents the key points of each frame. To express the key points of each frame, we use $D\times K$ sequence, which includes two parts, one ($K$) is the number of key points to fully represent the current pose, the other ($D$) is the dimension of each key point, generally three, which are the two coordinates of the planes and the depth estimation.

Here we use state-of-the-art pose estimation models such as Vitpose\cite{xu2022vitpose} and BlazePose\cite{bazarevsky2020blazepose}. The pose estimation algorithms themselves are not designed by us, but we introduce pose information into the action counting task, which is a novel design not explored by previous work.

Moreover, our pose-level poses estimation processes the primitive information of video, which is similar to the feature extraction network in all video-level algorithms such as I3D\cite{carreira2017quo}, VideoSwinTransformer\cite{liu2022video}, and TSN\cite{wang2016temporal}. But the difference is that the result of video-level incorporates all information, while pose-level only produces core information, which greatly improves the performance. Additionally, using pose information can contribute to the lightweight of model. For instance, for a 1024-frame video, video-level feature extraction with an output dimension of 512 would produce a data volume of $1024\times 512=524288$, while using pose information with 33 key points produces a data volume of only $1024\times 33 \times 3=101376$.

\subsection{Encoding Poses with Transformer}
\label{second}
Here we specify our data representation for the Transformer Encoder, which requires input batch size, sequence length, and embedding dimensions. In our pose-level approach, each frame is a batch, the number of key points in each frame is the sequence length, and the feature dimension of each key point is the embedding dimension.

First we get the pose of each frame ${p_i}\in \mathbb{R}^{D\times K}$ through the Pose Estimation Network, where $i\in {1, 2, \dots, T}$ is the frame index, $K$ is the number of key points, and $D$ is the dimension of each key point. We further define $p_i = {\{k_j\}}^{K}_{1}$ to represent each key point, where $k_j\in \mathbb{R}^D$, and we embed it to obtain richer information. Our embedding projection $\mathrm{\bf{E}}$ is a simple MLP network with ReLU as the activation function. These calculations can be defined as:
\begin{equation}
\begin{split}
\mathrm{\bf{Z}}^0 = [\mathrm{\bf{E}}(k_1), \mathrm{\bf{E}}(k_2), \dots, \mathrm{\bf{E}}(k_K)]^T
\end{split}
\end{equation}
where $\mathrm{\bf{E}}(k_j)\in \mathbb{R}^{D^{\prime}}$ is the embedding feature. Then the next Transformer takes $\mathrm{\bf{Z}}^0$ as input and encodes it with self-attention. Given $\mathrm{\bf{Z}}^0\in \mathbb{R}^{K\times D^{\prime}}$ with $K$ key point features, each of which is $D^{\prime}$-dimensional, $\mathrm{\bf{Z}}^0$ is projected using $\mathrm{\bf{W}}_Q\in \mathbb{R}^{D^{\prime}\times D_q}$, $\mathrm{\bf{W}}_K\in \mathbb{R}^{D^{\prime}\times D_k}$, $\mathrm{\bf{W}}_V\in \mathbb{R}^{D^{\prime}\times D_v}$, where $D_k=D_q$, to extract feature representations query($\mathrm{\bf{Q}}$), key($\mathrm{\bf{K}}$) and value($\mathrm{\bf{V}}$), which can be defined as:
\begin{equation}
\begin{split}
&\mathrm{\bf{Q}}=\mathrm{\bf{Z}}^0\times \mathrm{\bf{W}}_Q\\
&\mathrm{\bf{K}}=\mathrm{\bf{Z}}^0\times \mathrm{\bf{W}}_K\\
&\mathrm{\bf{V}}=\mathrm{\bf{Z}}^0\times \mathrm{\bf{W}}_V
\end{split}
\end{equation}
and the output of self-attention can be computed as:
\begin{equation}
\begin{split}
\mathrm{\bf{Attn}}=\mathrm{Softmax}(\frac{\mathrm{\bf{Q}}\mathrm{\bf{K}}^T}{\sqrt{D_q}})\mathrm{\bf{V}}
\end{split}
\end{equation}
where $\mathrm{\bf{Attn}}\in \mathbb{R}^{K\times D^{\prime}}$. Also, we use common multi-head self-attention (MHSA) to make several self-attention operations calculate in parallel.

Now we introduce the overall architecture of Transformer Encoder, which has $L$ layers with each layer consisting of MHSA and MLP blocks. Also, LayerNorm and Residual Connection are applied before and after every MHSA or MLP block, respectively. Because the number of key points of each frame is  a bit less, so our encoder does not include the downsampling module that other models may have. The overall process can be defined as:
\begin{equation}
\begin{split}
&\mathrm{\bf{\hat{Z}}}^l = \mathrm{MHSA}(\mathrm{LN}(\mathrm{\bf{Z}}^{l-1})) + \mathrm{\bf{Z}}^{l-1}\\
&\mathrm{\bf{Z}}^l = \mathrm{MLP}(\mathrm{LN}(\mathrm{\bf{\hat{Z}}}^l)) + \mathrm{\bf{\hat{Z}}}^l
\end{split}
\end{equation}
where $\mathrm{\bf{Z}}^{l-1}$, $\mathrm{\bf{\hat{Z}}}^l$, $\mathrm{\bf{Z}}^l\in \mathbb{R}^{K\times D^{\prime}}$.


\subsection{Pose Mapping}
\label{third}
Taking the Encoder output $\mathrm{\bf{Z}}^L\in \mathbb{R}^{K\times D^{\prime}}$ as input, Pose Mapping module outputs probability scores $\mathrm{\bf{S}}\in \mathbb{R}^{C}$ of the current frame over all action classes. We perform binary classification after Sigmoid activation for each class, with the two salient poses of each class represented by the same bit data. To realize such a module, we use a very lightweight MLP network, which avoids the complexity. First, the two dimensions $K$ and $D^{\prime}$ of $\mathrm{\bf{Z}}^L$ are flattened into $\mathbb{R}^{KD^{\prime}}$, and then it passes through an MLP module, where the output channels is set to $C$, which can be defined as:
\begin{equation}
\begin{split}
\mathrm{\bf{S}} = \sigma(\mathrm{MLP}(\mathrm{Flatten}(\mathrm{\bf{Z}}^L)))
\end{split}
\end{equation}
where $\sigma$ represents the Sigmoid activation function.

With such Pose Mapping, we can obtain the scores of single frame. It should be noted that we extract the poses of all frames, and use the convenience of matrix operations to obtain scores in parallel, which is actually consistent with the idea of mini batch. So at last, we combine the scores of all frames to get the video score matrix $\mathrm{\bf{\hat{S}}}\in \mathbb{R}^{C\times T}$, where $T$ represents the number of frames in the current video. 


\subsection{Action-trigger Module}
\label{fourth}
We use the lightweight Action-trigger Module to obtain the final output $Y$, the repetitive action count, which has a time complexity of $\mathcal{O}(n)$. First, we get the scores $S_c\in \mathbb{R}^T$ of a given action class from $\mathrm{\bf{\hat{S}}}$. Then, we scan all frames and use the action-trigger mechanism to count when the two salient poses of the action class occur sequentially. We set upper and lower bounds to distinguish the scores of the two salient poses, which cluster non-salient poses in the middle and easily classify the salient poses to the two ends.

\subsection{Losses and Metric Learning}

The modules need to be trained are Embedding, Transformer Encoder and Pose Mapping, and because we perform binary classification for each class, so we use the Binary Cross Entropy Loss, which can be defined as follows:
\begin{gather}
\mathcal{L}_{bce} = -\frac{1}{N}\sum\limits_{i=1}^{N}(\frac{1}{C}\sum\limits_{j=1}^{C}loss(i,j))  \\
 loss(i,j)=y_{ij}\log p_{ij} + (1-y_{ij})\log(1-p_{ij})
\end{gather}
where $N$ represents the batch size (in our method, each frame is a batch), $C$ represents the number of classes, $y$ and $p$ are the labels and our predictions, respectively.

Moreover, we use Metric Learning to improve our Encoder and introduce the Pose Triplet Loss. Given a pose, Encoder produces higher-level features $\mathrm{\bf{Z}}^L$, which should be more representative. As shown in Figure \ref{fig5}, we achieve this with Triplet Margin Loss function, which selects anchors, same class positive samples, and different classes negative samples in a batch. It can be expressed as:
\begin{equation}
\begin{split}
\mathcal{L}_{tri} = \mathrm{max}(\mathrm{CS}(a,p)-\mathrm{CS}(a,n)+\mathrm{margin},0)
\end{split}
\end{equation}
where $a$, $p$, $d$ are anchors, positive and negative samples, and $\mathrm{CS}$ represents the Cosine Similarity to measure the distance between features. We pay more attention to hard samples, where the distances between anchors and negative samples are even smaller than those of positive samples. After Metric Learning, the poses of each action can be distinguishable, which cluster in the high-level space.

At last, our overall training combines these two losses:
\begin{equation}
\begin{split}
\mathcal{L} = \mathcal{L}_{bce} + \alpha\mathcal{L}_{tri}
\end{split}
\end{equation}
where $\alpha$ is the weight factor to control the two losses in the same numeric scale.
\subsection{Implementation Details}

\noindent{\bf Training.} We use the \emph{RepCount-pose} and \emph{UCFRep-pose} dataset we created to train our model. Only the frames with salient poses are inputted into the network instead of the entire video to speed up the fitting.

\noindent{\bf Inference.} During inference, the entire video sequence is inputted into the model. The poses of all frames pass through the Encoder and Pose Mapping, and then enter the Action-trigger Module to output the repetitive count.
\section{Results}
\label{results}

\begin{figure*}[ht]
    \centering
    \includegraphics[scale=0.15,trim={0 2.5cm 0 5cm},clip]{images/aoi-single_burst}
    \caption{The time average peak Age of Information with burst and \gls{soa} loss values against the dynamic reliability logic for different network topologies.}
    \label{fig:aoi_burst}\vspace{-0.4cm}
\end{figure*}


This paper focuses on both transport layer and application layer metrics to determine the feasibility of dynamic reliability. For this, we have selected the session packet volume, as transmitted, retransmitted, lost and backlogged packets as \glspl{kpi} for the transport layer; while focusing on the \gls{aoi} for the application layer. The \gls{aoi} was chosen as a crucial indicator for the freshness of packets in real-time applications. More specifically, this work adopts the time average peak \gls{aoi} equation \cite{aoi_equation} depicted in Eq. \ref{aoi}, where $\Delta(r_{i+1})$ is the $i$th update at the time it was received at the server, for a session time period of $\tau$.

\begin{equation}
    \label{aoi}
    \gls{aoi}_\tau = \frac{1}{n-1}\sum_{i=1}^{n-1} \Delta(r_{i+1})
\end{equation}

We include a comparison between the vanilla QUIC implementation which does not enjoy the dynamic reliability extension, with a number of dynamic reliability policies. The tests were run a number of times for statistical significance, with the mean value of vanilla implementation used as a baseline for comparison. The topology utilised both random loss and bursty loss to explore the bounds of dynamic reliability. The \gls{soa} loss in the figures correspond to the loss values presented in Table. \ref{tab:path_char}, for ease of comparison between bursty and random loss scenarios.

\subsection{Transport-Layer KPIs}

To analyse the performance gain at the transport layer due to dynamic reliability, the volume of transmitted and backlogged packets is examined. The figures are in the form of boxplots, which take the vanilla implementation as a benchmark, depicted as the red dashed line.

As seen in Fig. \ref{fig:sent_burst}, the loss plays a crucial role in the performance of the reliability policies. The policies under random loss did incredibly well for the networks with a larger capacity, namely \gls{mmwave} and Sub-6~GHz, whereas for burst loss, the lower network capacities had a larger packet reduction. With the increase in burst loss, the behaviour of the set split reliable policies became unpredictable, if a reliable assignment happened to coincide with a burst loss, the number of transmitted packets increases, and vice versa. On the other hand, in smarter policies, such as Loss-Aware, the performance lightly matched the vanilla baseline, as the reliable assignment dominated the session to compensate for a higher burst loss. Not only that but, the burst loss also impacted the variance of the transmitted packets for the policies.

Unsurprisingly, the unreliable focused policy, 80-20 split, outperformed other policies for all topologies in random and bursty loss scenarios, with an approximate reduction of 80\%. That being said, the majority of the policies reduced the transmitted packets on the link by approximately 70\% for random loss, while the reduction started at $\approx 15\%$ and decreased as the loss increased for the burst loss scenario.

The retransmitted and lost packets, not shown due to space limitations, followed the same trend as the transmitted packets for the random loss scenarios. However, for the burst loss scenarios, the larger capacity networks had a lower reduction in the retransmitted and lost packets. This can be seen as a favorable outcome since the lower capacity networks are scarce on resources. It is important to note that the Loss-Aware policy mimicked the vanilla approach as the burst loss increased, signifying the overwhelming appointment of reliable packets in adapting to the harsh burst loss conditions.
 
Alternatively, Fig. \ref{fig:backlog_burst} clearly shows a stark comparison between the policies and loss scenario in the reduction of the backlogged packets. The Loss-Aware policy for random loss scenario reduced the backlogged packets by up to 50\%, beating all other policies by approximately 30\%. Furthermore, it is clear that the unreliability focused policies resulted in the lowest backlog for the session. In comparison, we notice that the burst loss and the backlogged frequency have a positive correlation, where the maximum reduction of the backlogged packets for the policies is at most 20\%. Much like the transmitted packets, the probability of a burst loss occurrence plays a vital role in the number of retransmissions sent and by extension the number of backlogged packets. Thus, we can conclude that the stress placed on the buffer is a result of the reliable packets which is tightly coupled with the congestion on the session. Whereas, unreliable focused policies did not encounter such a phenomenon regardless if it was experiencing a burst loss.


\subsection{Application-Layer KPIs}

The feasibility of dynamic reliability for real-time applications can be determined by the \gls{aoi}, with comparison across different topologies and policies. If we take a strict approach and consider anything below $10$~ms is real-time \cite{real-time}, then all the reliability policies passed that requirement, which is attractive for real-time applications, as shown in Fig. \ref{fig:aoi_burst}. Utilising the median as an estimate of the runs, the policies in the WLAN and Sub-6~GHz topology with random loss floated around $4-5$~ms with negligible difference, while the \gls{aoi} for \gls{mmwave} was $\approx 2-3$~ms. It is clear that the \gls{aoi} and the network capacity have a negative correlation, as the network capacity decreases, the \gls{aoi} increases. The same correlation is extended to the bursty loss scenarios, where \gls{mmwave} dominated the other topologies. That being said, it is crucial to note that the \gls{aoi} for the reliability policies is often slightly better than or equal to the \gls{aoi} of the vanilla implementation, proving that dynamic reliability reduces the congestion of the session at no cost to the \gls{aoi}.

\section{Conclusions}

This study provides evidence that information obtained from Twitter posts serves as a strong indicator for stock price movements. In particular, Twitter sentiment scores are highly correlated with price volatility and improve the performance of predictive models for financial markets. We conclude that sentiments provide valuable insights on events around and contribute to better capturing underlying market dynamics. 

The main goal of this paper is to investigate whether sentence embeddings yield better results in stock price prediction compared to the sentiment analysis approach. In the majority of conducted experiments, the sentiment approach outperforms the embedding vectors method. This fact might be counterintuitive because embeddings seem to encompass more valuable contextual information. However, sentiments tend to represent information in a more concise way, bringing less noise into the prediction model. Nevertheless, the embedding approach still has an advantage that it does not require an additional model for sentiment extraction and the consequent quality verification of that procedure. It is important to note the limitation of the conducted research. We made a comparison analysis only within restricted use case of Twitter posts and stocks of top companies from NASDAQ.  

\section{Statement on computational resources and environmental impact}  

We used a NVIDIA GeForce RTX 3080 Ti GPU to train the models, extract the sentiment score and make embedding vectors from the tweets using BERT model \cite{all-MiniLM-L6-v2}. Two NVIDIA A100 80GB PCIe GPUs were used for testing out the MPNet \cite{all-mpnet-base-v2} approach and inferencing. This work contributed totally 6.12 kg equivalent $CO_{2}$ emissions. The carbon emissions information was generated using the open-source library \textit{eco2AI}\footnote{Source code for \textit{eco2AI} is available at \url{https://github.com/sb-ai-lab/Eco2AI}} \cite{budennyy2023eco2ai}. 

% The embedding vectors method demonstrated similar performance in comparison with sentiment approach for the vast majority of conducted experiments. It is especially evident for the 3 day prediction window experiment directly eliminating the need to solve the problem of sentiment extraction without a significant loss in quality for final predictions.

% Sentiment extraction requires a rigorous process of feature extraction and quality verification in case we don't have manually labeled data, which can be time-consuming. On the other hand, sentence embedding approach does not require verification and can produce similar results to sentiment extraction retaining more of the semantic and contextual information contained in the text. Nevertheless, the model training time for sentence embedding is significantly longer. These findings suggest that sentence embedding could be considered a robust solution for stock price prediction, due to its similar performance to sentiment extraction, despite the longer model training time.

% On the other hand the results of the experiments demonstrated that the sentiment polarity extraction approach still slightly outperforms sentence embeddings in terms of accuracy and training time for predicting the stock closing price 3 and 5 days ahead. Also, increasing the dimensionality of sentence embeddings did not yield better results and instead worsened the performance of the Temporal Fusion Transformer model. Additionally, in case of the 3 days ahead approach, it received an improvement for sentence embeddings only for companies where sentiment was not a good predictor in the first place.

% This suggests that the choice between sentiment polarity extraction and sentence embeddings as the preferred approach for closing price prediction can possibly depend on the specific task and the prediction horizon, as well as the effectiveness of sentiment as a predictor in the given context.
%\input{sections/future_works}
% \clearpage
% \subsection{Previous scenari}

\begin{table}[h]
    \centering
    \begin{tabular}{|c|c|c|c|c|c|c|c|}
            \hline
            scenario & time horizon & Cap on CO2 & Cost of CO2 & ENS ALLOWED & Cost ENS & Pipe and/or Boat & Objective \\ \hline
            1 & 8760 & 0.0 & 0.0 & False & - & pipe and boat & 39992.71 \\
            2 & 8760 & None & 0.08 & True & 3.0 & pipe and boat & 37515.28 \\
            3 & 8760 & None & 0.0 & True & 3.0 & pipe and boat & 36107.64 \\
            4 & 8760 & None & 0.08 & True & 3.0 & only pipe & 37552.58 \\
            5 & 8760 & None & 0.08 & True & 3.0 & only carrier & 37539.88 \\
            6 & 8760 & 0.0 & 0.0 & False & - & pipe and boat & 52345.09 \\
            7 & 8760 & None & 0.08 & False & - & pipe and boat & 39050.49 \\
            8 & 8760 & None & 0.1648981 & False & - & pipe and boat & 39992.71 \\
            \hline
            
    \end{tabular}
    \caption{Scenari parameters}
    \label{tab:scenario_parameters_prev}
\end{table}


\begin{table}[h]
    \centering
    \begin{tabular}{|c|c|c|c|c|c|c|c|}
         \hline
            scenario & wind on & wind off & solar\_be & ccgt\_be & wind\_gl & wind\_nz & solar\_nz \\ \hline
            1 & 8.40 & 7.71 & 9.56 & 22.41 & 0.00 & 98.13 & 91.39 \\ 
            2 & 8.40 & 8.00 & 13.83 & 17.39 & 0.00 & 87.31 & 81.09 \\ 
            3 & 8.40 & 8.00 & 13.29 & 17.32 & 0.00 & 86.46 & 80.49 \\ 
            4 & 8.40 & 8.00 & 13.42 & 17.30 & 0.00 & 86.42 & 80.45 \\ 
            5 & 8.40 & 8.00 & 13.88 & 17.50 & 0.00 & 87.93 & 81.63 \\ 
            6 & 8.40 & 8.00 & 15.68 & 19.58 & 126.54 & 0.00 & 0.00 \\ 
            7 & 8.40 & 7.19 & 8.95 & 22.06 & 0.00 & 93.04 & 86.40 \\ 
            8 & 8.40 & 7.71 & 9.56 & 22.41 & 0.00 & 98.18 & 91.43 \\  
            \hline
            
    \end{tabular}
    \caption{Total Power installation in GW}
    \label{tab:power_prev}
\end{table}



\begin{table}[h]
    \centering
    \begin{tabular}{|c|c|c|c|c|c|}
            \hline
            scenario & PCCC & PCCC CCGT & DAC NZ & DAC GR \\ \hline
            1 & 4.11 & 2.59 & 1.26 & 0.00 \\ 
            2 & 4.11 & 1.55 & 0.00 & 0.00 \\ 
            3 & 5.00 & 0.44 & 0.00 & 0.00 \\ 
            4 & 4.11 & 1.33 & 0.00 & 0.00 \\ 
            5 & 4.11 & 1.59 & 0.00 & 0.00 \\ 
            6 & 4.11 & 2.59 & - & 1.22 \\ 
            7 & 4.11 & 1.76 & 0.00 & 0.00 \\ 
            8 & 4.11 & 2.59 & 1.27 & 0.00 \\
            \hline
            
    \end{tabular}
    \caption{Technology of capture with capacity in kt/h}
    \label{tab:capture_co2_prev}
\end{table}


\begin{table}[h]
    \centering
    \begin{tabular}{|c|c|c|c|c|}
            \hline
            Scenario & pipe nz & carrier nz & pipe gr & carrier gr \\
            \hline
            1 & 3.196 & 2.363 & 0.000 & 0.000 \\
            2 & 2.158 & 5.626 & 0.000 & 0.000 \\
            3 & 5.445 & 0.000 & 0.000 & 0.000 \\
            4 & 5.442 &  - & 0.000 &  - \\
            5 &  - & 9.280 &  - & 0.000 \\
            6 & 0.000 & 0.000 & 0.000 & 7.518 \\
            7 & 2.374 & 5.833 & 0.000 & 0.000 \\
            8 & 3.196 & 2.344 & 0.000 & 0.000 \\
            \hline
    \end{tabular}
    \caption{Transport technology with capacity in kt/h}
    \label{tab:transport_co2_prev}
\end{table}

\subsection{Technologies}\label{subsec:technologies}
In this sub-section, the different types of technologies are described, namely conversion and storage technology and balances (respectively represented by nodes and hyperedges in \autoref{sec:modelling}). The different constraints are also thouroughly described. The notation follows the convention taken in \cite{Berger2021} where the Latin letters denote optimisation variables and indices, while Greek letters indicate parameters. 

\subsubsection{Nodes}

% \textbf{Variable Energy Sources}: A set $\mathcal{P}_R = \{PV, W_{on}, W_{off}\}$:
    % $$
    % \begin{aligned}
    % & P_{E, t}^p \leq \pi_t^p\left(\kappa_0^p+K_E^p\right), \quad \forall t \in \mathcal{T}, \quad \forall p \in \mathcal{P}_R \\
    % & K_E^p \leq \kappa_{\max }^p, \quad \forall p \in \mathcal{P}_R
    % \end{aligned}
    % $$
    % with $\kappa_0^p$ represents the pre-installed capacity and $K_E^p$ the newly installed capacity. The parameter $\pi_t^p$ represents the load factor at timestep $t$. 
    % Investment and operating costs are described as
    % $$
    % C^p=\left(\zeta^p+\theta_f^p\right) K_E^p+\sum_{t \in \mathcal{T}} \theta_v^p P_{E, t}^p \delta t, \quad \forall p \in \mathcal{P}_R
    % $$
    % with 

% \textbf{Dispatchable Technologies}
    % \begin{equation}
        % \begin{aligned}
        % & P_{E, t}^p \leq \kappa_0^p+K_E^p, \quad \forall p \in \mathcal{P}_D \\
        % & P_{E, t}^p-P_{E, t-1}^p \leq \Delta_{+}^p\left(\kappa_0^p+K_E^p\right), \quad \forall p \in \mathcal{P}_D \\
        % & P_{E, t}^p-P_{E, t-1}^p \geq-\Delta_{-}^p\left(\kappa_0^p+K_E^p\right), \quad \forall p \in \mathcal{P}_D \\
        % & \mu^p\left(\kappa_0^p+K_E^p\right) \leq P_{E, t}^p, \quad \forall p \in \mathcal{P}_D \\
        % & Q_{\mathrm{CO}_2, t}^p=\frac{\nu^p P_{E, t}^p}{\eta^p}, \quad p \in\{\mathrm{BM}, % \mathrm{WS}\}
        % \end{aligned}
    % \end{equation}

\textbf{Conversion technologies}: 
Let us consider a given node $n \in \mathcal{N}$ wich is a conversion technology \textit{i.e.} which takes as input a given commodity $i$ flow and outputs another commodity $r$ flow. A conversion technology is described by an internal variable representing the capacity of a guven technology associated with a given commodity and by external variables linking the flows in and out. These relations are expressed as 

\begin{equation}
q_{rt}^n - \phi_{i}^n q_{i(t+\tau_{i}^n)}^n = 0, \mbox{ } \forall i \in \mathcal{I}^n\setminus\{r\}, \mbox{ } \forall t \in \mathcal{T}^n,
\label{eq:conversion}
\end{equation}
where $q_{it}^n \in \mathbb{R}^{+}$ is the flow of a given commodity $i$, $\phi_{i}^n$ is the conversion factor from $i$ to $r$ and $\tau_{i}^n$ is the time for the process to occur. Other constraints occur in conversion technology such as the maximum capacity expressed as

\begin{equation}
K_0^n - K_t^n = 0, \mbox{ } \forall t \in \mathcal{T}\setminus\{0\},
\label{eq:conversioncapacitystatic}
\end{equation}
where $K_t^n$ is the capacity at each time step of the time horizon considered. The maximum capacity remains constant over the entire horizon because we use a static investment policy. In the following $K^{n}$ will be a shorthand for $K_0^n$ and represents the new capacity installed. 

Some conversion technology are not dispatchable (\textit{e.g.} variable renewable energy). Therefore, the availability of such a technology is expressed as 

\begin{equation}
q_{r't}^n - \pi_{t}^n (\underbar{$\kappa$}^{n} + K^{n}) \le 0, \mbox{ } \forall t \in \mathcal{T},
\label{eq:sizing}
\end{equation}
where $\pi_{t}^n \in [0,1]$ is the normalized capacity factor of the technology $n$ at timestep $t$, $\underbar{$\kappa$}^{n}$ is the pre-installed capacity. 

Some technology capacities are bounded by a maximum potential capacity. This constraint reads as
\begin{equation}
(\underbar{$\kappa$}^{n} + K^{n}) - \bar{\kappa}^{n} \le 0,
\label{eq:potential}
\end{equation}
where $\bar{\kappa}^{n}$ is the maximum capacity of technology $n$.

To run a technology $n$ a minimal input flow may be needed and is expressed as
\begin{equation}
\mu^{n} (\underbar{$\kappa$}^{n} + K^{n}) - \frac{\phi_{i}^n}{\phi_{r'}^n}q_{it}^n \le 0, \mbox{ } \forall t \in \mathcal{T},
\label{eq:mustrun}
\end{equation}
where $\mu^{n} \in [0,1]$ represents the minimum operating level (as a fraction of the installed capacity). \textcolor{red}{Since the technology is sized with respect to the flow of commodity r', the flow of a commodity $i \neq r'$ must be scaled by the ratio of conversion factors in \ref{eq:mustrun}. }

Some conversion technologies are limited in the rate at which they can change a commodity flow. These are the so-called ramping up constraint defined as
\begin{equation}
\frac{\phi_{i}^n}{\phi_{r'}^n}(q_{it}^n - q_{i(t-1)}^n) - \Delta_{i,+}^{n} (\underbar{$\kappa$}^{n} + K^{n}) \le 0, \mbox{ } \forall t \in \mathcal{T}\setminus\{0\},
\label{eq:rampup}
\end{equation}

and ramping down constraint defined as
\begin{equation}
\frac{\phi_{i}^n}{\phi_{r'}^n}(q_{i(t-1)}^n - q_{it}^n) - \Delta_{i,-}^{n} (\underbar{$\kappa$}^{n} + K^{n}) \le 0, \mbox{ } \forall t \in \mathcal{T}\setminus\{0\},
\label{eq:rampdown}
\end{equation}
where $\Delta_{i,+}^{n} \in [0,1]$ and $\Delta_{i,-}^{n} \in [0,1]$ the maximum rates at which flows can be ramped up and down.

Finally, we get the local objective function for technology $n$ written as
\begin{equation}
F_n = \nu (\zeta^{n} + \theta_f^n) K^{n} + \sum_{t \in \mathcal{T}} \theta_{t,v}^n q_{r't}^n \delta t,\label{eq:objectiveconversion}
\end{equation}
where $\nu \in \mathbb{N}$ is the number of years spanned by the optimisation horizon, $\zeta^n \in \mathbb{R}_+$ represents the (annualised) investment cost (also known as capital expenditure, CAPEX), $\theta_f^n \in \mathbb{R}_+$ models fixed operation and maintenance (FOM) costs and $\theta_{t,v}^n \in \mathbb{R}_+$ represents variable operation and maintenance (VOM) costs, which may be time-dependent.


%%%%%%%%%%%%%%%
%   STORAGE   %
%%%%%%%%%%%%%%%

\textbf{Storage}: Let $n \in \mathcal{N}$ be a node representing a given storage technology. This later internal variable represents the current amount of commodity stored while the external variables are the flows in (charge) and out (discharge) of this technology as well as other commodity input flow needed. The dynamics of charge and discharge is described as follows

\begin{equation}
e_{t+1}^n - (1-\eta_{S}^n) e_{t}^n - \eta_{+}^{n} q_{it}^n + \frac{1}{\eta_{-}^{n}} q_{jt}^n = 0, \mbox{ } \forall t \in \mathcal{T} \setminus\{T-1\},
\label{eq:storagedynamics}
\end{equation}
where $e_{t}^n \in \mathbb{R}_+$ is the inventory level at time $t$, $q_{i_ut}^n \in \mathbb{R}_+$ and $q_{i_yt}^n \in \mathbb{R}_+$ represent commodity in and outflows at time $t$, respectively, $\eta_S^n \in [0, 1]$ is the self-discharge rate, $\eta_+^n \in [0, 1]$ is the charge efficiency and $\eta_-^n \in [0, 1]$ is the discharge efficiency. The consumption of other commodity may be modelled with

\begin{equation}
q_{lt}^n - \phi_{i}^n q_{it}^n = 0, \mbox{ } \forall t \in \mathcal{T}.
\label{eq:conversionstorage}
\end{equation}

To avoid board effect, we impose that the stock at the begininng of the time horizon is equal at the end of the time horzion. Thsi prevents the model to use freely a commodity stored. The constraint can be written as
\begin{equation}
e_{0}^n - e_{T-1}^n = 0.
\label{eq:storagecyclicity}
\end{equation}
where $e_{0}^n$ is the inventory level at timestep zero and $e_{T-1}^n$ is the inventory level at the end of the spanned time horizon. 

As for conversion technology (\textit{cfr} \autoref{eq:conversioncapacitystatic}) the policy of investment is static. Thus it can be expressed as 
\begin{equation}
E_0^n - E_t^n = 0, \mbox{ } \forall t \in \mathcal{T}\setminus\{0\},
\label{eq:storagestockstatic}
\end{equation}
where $E_0^n$ is the newly installed capacity. In the later, the shorthand $E^n$ to denote $E_t^n$ will be used. 
The total storage capacity is defined as

\begin{equation}
e_{t}^n - (\underbar{$\epsilon$}^{n} + E^{n}) \le 0, \mbox{ } \forall t \in \mathcal{T},
\label{eq:storagestocksizing}
\end{equation}

As stated in \autoref{eq:potential} for conversion technologies, storage technology might have a maximum storage capacity. This is expressed as 
\begin{equation}
(\underbar{$\epsilon$}^{n} + E^{n}) - \bar{\epsilon}^{n} \le 0,
\label{eq:storagepotential}
\end{equation}
where $\bar{\epsilon}^n \in \mathbb{R}_+$ represents the maximum stock capacity that may be deployed. 

A minimum inventory level may be used and described as follows


\begin{equation}
\sigma^{n}(\underbar{$\epsilon$}^{n} + E^{n}) - e_{t}^n \le 0, \mbox{ } t \in \mathcal{T},
\label{eq:storageminimumsoc}
\end{equation}
where $\sigma^{n} \in [0, 1]$ is the mimum level of the inventory as a proportion of the installed capacity. 

The maximum inflow is modelled using an internal variable that is also constant throughout the time horizon considered, as in Eq. \eqref{eq:conversioncapacitystatic}. It is defined as follows
\begin{equation}
q_{it}^n - (\underbar{$\kappa$}^{n} + K^{n}) \le 0, \mbox{ } \forall t \in \mathcal{T},
\label{eq:storageflowplussizing}
\end{equation}
where $\underbar{$\kappa$}^n \in \mathbb{R}_+$ denotes the existing flow capacity and $K^n \in \mathbb{R}_+$ is used as shorthand for the new capacity. The maximum in- and outflows may be asymmetric, depending on the properties of the underlying technology, which is modelled via
\textcolor{red}{The maximum in- and outflows may be asymmetric, depending on the properties of the underlying technology, which is modelled via}
\begin{equation}
q_{jt}^n - \rho^{n} (\underbar{$\kappa$}^{n} + K^{n}) \le 0, \mbox{ } \forall t \in \mathcal{T},
\label{eq:storageflowminussizing}
\end{equation}
where $\rho^n \in \mathbb{R}_+$ represents the maximum discharge-to-charge ratio.
Finally the local objective for node $n$ can be written as follows
\begin{equation}
F^n = \Big[\nu (\varsigma^{n} + \vartheta^{n}_{f}) E^{n} + \sum_{t \in \mathcal{T}} \vartheta_{t,v}^n e_{t}^n\Big] + \Big[\nu (\zeta^{n} + \theta^n_f) K^{n} + \sum_{t \in \mathcal{T}} \theta_{t,v}^n q_{it}^n \delta t \Big].
\label{eq:objectivestorage}
\end{equation}
where $\varsigma^n \in \mathbb{R}_+$ and $\zeta^n \in \mathbb{R}_+$ represent the stock and flow components of CAPEX, $\vartheta_f^n \in \mathbb{R}_+$ and $\theta_f^n \in \mathbb{R}_+$ model the stock and flow components of FOM costs, while $\vartheta_{t,v}^n \in \mathbb{R}_+$ and $\theta_{t,v}^n \in \mathbb{R}_+$ represent the stock and flow components of VOM costs, which may be time-dependent.

\subsubsection{Hyperedges}
\textbf{Conservation}: Let $e\in \mathcal{E}$ be a conservation hyperedge which ensure that all the flows going in and out of a given commodity are respected. 
\begin{equation}
\sum_{n \in e_T} q_{it}^n - \sum_{n \in e_H} q_{it}^n - \lambda_{t}^e = 0, \mbox{ } \forall t \in \mathcal{T},
\label{eq:conservationhyperedge}
\end{equation}



% \clearpage
\newpage
% \bibliographystyle{apalike}
\bibliography{source.bib}

\newpage
\subsection{Previous scenari}

\begin{table}[h]
    \centering
    \begin{tabular}{|c|c|c|c|c|c|c|c|}
            \hline
            scenario & time horizon & Cap on CO2 & Cost of CO2 & ENS ALLOWED & Cost ENS & Pipe and/or Boat & Objective \\ \hline
            1 & 8760 & 0.0 & 0.0 & False & - & pipe and boat & 39992.71 \\
            2 & 8760 & None & 0.08 & True & 3.0 & pipe and boat & 37515.28 \\
            3 & 8760 & None & 0.0 & True & 3.0 & pipe and boat & 36107.64 \\
            4 & 8760 & None & 0.08 & True & 3.0 & only pipe & 37552.58 \\
            5 & 8760 & None & 0.08 & True & 3.0 & only carrier & 37539.88 \\
            6 & 8760 & 0.0 & 0.0 & False & - & pipe and boat & 52345.09 \\
            7 & 8760 & None & 0.08 & False & - & pipe and boat & 39050.49 \\
            8 & 8760 & None & 0.1648981 & False & - & pipe and boat & 39992.71 \\
            \hline
            
    \end{tabular}
    \caption{Scenari parameters}
    \label{tab:scenario_parameters_prev}
\end{table}


\begin{table}[h]
    \centering
    \begin{tabular}{|c|c|c|c|c|c|c|c|}
         \hline
            scenario & wind on & wind off & solar\_be & ccgt\_be & wind\_gl & wind\_nz & solar\_nz \\ \hline
            1 & 8.40 & 7.71 & 9.56 & 22.41 & 0.00 & 98.13 & 91.39 \\ 
            2 & 8.40 & 8.00 & 13.83 & 17.39 & 0.00 & 87.31 & 81.09 \\ 
            3 & 8.40 & 8.00 & 13.29 & 17.32 & 0.00 & 86.46 & 80.49 \\ 
            4 & 8.40 & 8.00 & 13.42 & 17.30 & 0.00 & 86.42 & 80.45 \\ 
            5 & 8.40 & 8.00 & 13.88 & 17.50 & 0.00 & 87.93 & 81.63 \\ 
            6 & 8.40 & 8.00 & 15.68 & 19.58 & 126.54 & 0.00 & 0.00 \\ 
            7 & 8.40 & 7.19 & 8.95 & 22.06 & 0.00 & 93.04 & 86.40 \\ 
            8 & 8.40 & 7.71 & 9.56 & 22.41 & 0.00 & 98.18 & 91.43 \\  
            \hline
            
    \end{tabular}
    \caption{Total Power installation in GW}
    \label{tab:power_prev}
\end{table}



\begin{table}[h]
    \centering
    \begin{tabular}{|c|c|c|c|c|c|}
            \hline
            scenario & PCCC & PCCC CCGT & DAC NZ & DAC GR \\ \hline
            1 & 4.11 & 2.59 & 1.26 & 0.00 \\ 
            2 & 4.11 & 1.55 & 0.00 & 0.00 \\ 
            3 & 5.00 & 0.44 & 0.00 & 0.00 \\ 
            4 & 4.11 & 1.33 & 0.00 & 0.00 \\ 
            5 & 4.11 & 1.59 & 0.00 & 0.00 \\ 
            6 & 4.11 & 2.59 & - & 1.22 \\ 
            7 & 4.11 & 1.76 & 0.00 & 0.00 \\ 
            8 & 4.11 & 2.59 & 1.27 & 0.00 \\
            \hline
            
    \end{tabular}
    \caption{Technology of capture with capacity in kt/h}
    \label{tab:capture_co2_prev}
\end{table}


\begin{table}[h]
    \centering
    \begin{tabular}{|c|c|c|c|c|}
            \hline
            Scenario & pipe nz & carrier nz & pipe gr & carrier gr \\
            \hline
            1 & 3.196 & 2.363 & 0.000 & 0.000 \\
            2 & 2.158 & 5.626 & 0.000 & 0.000 \\
            3 & 5.445 & 0.000 & 0.000 & 0.000 \\
            4 & 5.442 &  - & 0.000 &  - \\
            5 &  - & 9.280 &  - & 0.000 \\
            6 & 0.000 & 0.000 & 0.000 & 7.518 \\
            7 & 2.374 & 5.833 & 0.000 & 0.000 \\
            8 & 3.196 & 2.344 & 0.000 & 0.000 \\
            \hline
    \end{tabular}
    \caption{Transport technology with capacity in kt/h}
    \label{tab:transport_co2_prev}
\end{table}

\subsection{Technologies}\label{subsec:technologies}
In this sub-section, the different types of technologies are described, namely conversion and storage technology and balances (respectively represented by nodes and hyperedges in \autoref{sec:modelling}). The different constraints are also thouroughly described. The notation follows the convention taken in \cite{Berger2021} where the Latin letters denote optimisation variables and indices, while Greek letters indicate parameters. 

\subsubsection{Nodes}

% \textbf{Variable Energy Sources}: A set $\mathcal{P}_R = \{PV, W_{on}, W_{off}\}$:
    % $$
    % \begin{aligned}
    % & P_{E, t}^p \leq \pi_t^p\left(\kappa_0^p+K_E^p\right), \quad \forall t \in \mathcal{T}, \quad \forall p \in \mathcal{P}_R \\
    % & K_E^p \leq \kappa_{\max }^p, \quad \forall p \in \mathcal{P}_R
    % \end{aligned}
    % $$
    % with $\kappa_0^p$ represents the pre-installed capacity and $K_E^p$ the newly installed capacity. The parameter $\pi_t^p$ represents the load factor at timestep $t$. 
    % Investment and operating costs are described as
    % $$
    % C^p=\left(\zeta^p+\theta_f^p\right) K_E^p+\sum_{t \in \mathcal{T}} \theta_v^p P_{E, t}^p \delta t, \quad \forall p \in \mathcal{P}_R
    % $$
    % with 

% \textbf{Dispatchable Technologies}
    % \begin{equation}
        % \begin{aligned}
        % & P_{E, t}^p \leq \kappa_0^p+K_E^p, \quad \forall p \in \mathcal{P}_D \\
        % & P_{E, t}^p-P_{E, t-1}^p \leq \Delta_{+}^p\left(\kappa_0^p+K_E^p\right), \quad \forall p \in \mathcal{P}_D \\
        % & P_{E, t}^p-P_{E, t-1}^p \geq-\Delta_{-}^p\left(\kappa_0^p+K_E^p\right), \quad \forall p \in \mathcal{P}_D \\
        % & \mu^p\left(\kappa_0^p+K_E^p\right) \leq P_{E, t}^p, \quad \forall p \in \mathcal{P}_D \\
        % & Q_{\mathrm{CO}_2, t}^p=\frac{\nu^p P_{E, t}^p}{\eta^p}, \quad p \in\{\mathrm{BM}, % \mathrm{WS}\}
        % \end{aligned}
    % \end{equation}

\textbf{Conversion technologies}: 
Let us consider a given node $n \in \mathcal{N}$ wich is a conversion technology \textit{i.e.} which takes as input a given commodity $i$ flow and outputs another commodity $r$ flow. A conversion technology is described by an internal variable representing the capacity of a guven technology associated with a given commodity and by external variables linking the flows in and out. These relations are expressed as 

\begin{equation}
q_{rt}^n - \phi_{i}^n q_{i(t+\tau_{i}^n)}^n = 0, \mbox{ } \forall i \in \mathcal{I}^n\setminus\{r\}, \mbox{ } \forall t \in \mathcal{T}^n,
\label{eq:conversion}
\end{equation}
where $q_{it}^n \in \mathbb{R}^{+}$ is the flow of a given commodity $i$, $\phi_{i}^n$ is the conversion factor from $i$ to $r$ and $\tau_{i}^n$ is the time for the process to occur. Other constraints occur in conversion technology such as the maximum capacity expressed as

\begin{equation}
K_0^n - K_t^n = 0, \mbox{ } \forall t \in \mathcal{T}\setminus\{0\},
\label{eq:conversioncapacitystatic}
\end{equation}
where $K_t^n$ is the capacity at each time step of the time horizon considered. The maximum capacity remains constant over the entire horizon because we use a static investment policy. In the following $K^{n}$ will be a shorthand for $K_0^n$ and represents the new capacity installed. 

Some conversion technology are not dispatchable (\textit{e.g.} variable renewable energy). Therefore, the availability of such a technology is expressed as 

\begin{equation}
q_{r't}^n - \pi_{t}^n (\underbar{$\kappa$}^{n} + K^{n}) \le 0, \mbox{ } \forall t \in \mathcal{T},
\label{eq:sizing}
\end{equation}
where $\pi_{t}^n \in [0,1]$ is the normalized capacity factor of the technology $n$ at timestep $t$, $\underbar{$\kappa$}^{n}$ is the pre-installed capacity. 

Some technology capacities are bounded by a maximum potential capacity. This constraint reads as
\begin{equation}
(\underbar{$\kappa$}^{n} + K^{n}) - \bar{\kappa}^{n} \le 0,
\label{eq:potential}
\end{equation}
where $\bar{\kappa}^{n}$ is the maximum capacity of technology $n$.

To run a technology $n$ a minimal input flow may be needed and is expressed as
\begin{equation}
\mu^{n} (\underbar{$\kappa$}^{n} + K^{n}) - \frac{\phi_{i}^n}{\phi_{r'}^n}q_{it}^n \le 0, \mbox{ } \forall t \in \mathcal{T},
\label{eq:mustrun}
\end{equation}
where $\mu^{n} \in [0,1]$ represents the minimum operating level (as a fraction of the installed capacity). \textcolor{red}{Since the technology is sized with respect to the flow of commodity r', the flow of a commodity $i \neq r'$ must be scaled by the ratio of conversion factors in \ref{eq:mustrun}. }

Some conversion technologies are limited in the rate at which they can change a commodity flow. These are the so-called ramping up constraint defined as
\begin{equation}
\frac{\phi_{i}^n}{\phi_{r'}^n}(q_{it}^n - q_{i(t-1)}^n) - \Delta_{i,+}^{n} (\underbar{$\kappa$}^{n} + K^{n}) \le 0, \mbox{ } \forall t \in \mathcal{T}\setminus\{0\},
\label{eq:rampup}
\end{equation}

and ramping down constraint defined as
\begin{equation}
\frac{\phi_{i}^n}{\phi_{r'}^n}(q_{i(t-1)}^n - q_{it}^n) - \Delta_{i,-}^{n} (\underbar{$\kappa$}^{n} + K^{n}) \le 0, \mbox{ } \forall t \in \mathcal{T}\setminus\{0\},
\label{eq:rampdown}
\end{equation}
where $\Delta_{i,+}^{n} \in [0,1]$ and $\Delta_{i,-}^{n} \in [0,1]$ the maximum rates at which flows can be ramped up and down.

Finally, we get the local objective function for technology $n$ written as
\begin{equation}
F_n = \nu (\zeta^{n} + \theta_f^n) K^{n} + \sum_{t \in \mathcal{T}} \theta_{t,v}^n q_{r't}^n \delta t,\label{eq:objectiveconversion}
\end{equation}
where $\nu \in \mathbb{N}$ is the number of years spanned by the optimisation horizon, $\zeta^n \in \mathbb{R}_+$ represents the (annualised) investment cost (also known as capital expenditure, CAPEX), $\theta_f^n \in \mathbb{R}_+$ models fixed operation and maintenance (FOM) costs and $\theta_{t,v}^n \in \mathbb{R}_+$ represents variable operation and maintenance (VOM) costs, which may be time-dependent.


%%%%%%%%%%%%%%%
%   STORAGE   %
%%%%%%%%%%%%%%%

\textbf{Storage}: Let $n \in \mathcal{N}$ be a node representing a given storage technology. This later internal variable represents the current amount of commodity stored while the external variables are the flows in (charge) and out (discharge) of this technology as well as other commodity input flow needed. The dynamics of charge and discharge is described as follows

\begin{equation}
e_{t+1}^n - (1-\eta_{S}^n) e_{t}^n - \eta_{+}^{n} q_{it}^n + \frac{1}{\eta_{-}^{n}} q_{jt}^n = 0, \mbox{ } \forall t \in \mathcal{T} \setminus\{T-1\},
\label{eq:storagedynamics}
\end{equation}
where $e_{t}^n \in \mathbb{R}_+$ is the inventory level at time $t$, $q_{i_ut}^n \in \mathbb{R}_+$ and $q_{i_yt}^n \in \mathbb{R}_+$ represent commodity in and outflows at time $t$, respectively, $\eta_S^n \in [0, 1]$ is the self-discharge rate, $\eta_+^n \in [0, 1]$ is the charge efficiency and $\eta_-^n \in [0, 1]$ is the discharge efficiency. The consumption of other commodity may be modelled with

\begin{equation}
q_{lt}^n - \phi_{i}^n q_{it}^n = 0, \mbox{ } \forall t \in \mathcal{T}.
\label{eq:conversionstorage}
\end{equation}

To avoid board effect, we impose that the stock at the begininng of the time horizon is equal at the end of the time horzion. Thsi prevents the model to use freely a commodity stored. The constraint can be written as
\begin{equation}
e_{0}^n - e_{T-1}^n = 0.
\label{eq:storagecyclicity}
\end{equation}
where $e_{0}^n$ is the inventory level at timestep zero and $e_{T-1}^n$ is the inventory level at the end of the spanned time horizon. 

As for conversion technology (\textit{cfr} \autoref{eq:conversioncapacitystatic}) the policy of investment is static. Thus it can be expressed as 
\begin{equation}
E_0^n - E_t^n = 0, \mbox{ } \forall t \in \mathcal{T}\setminus\{0\},
\label{eq:storagestockstatic}
\end{equation}
where $E_0^n$ is the newly installed capacity. In the later, the shorthand $E^n$ to denote $E_t^n$ will be used. 
The total storage capacity is defined as

\begin{equation}
e_{t}^n - (\underbar{$\epsilon$}^{n} + E^{n}) \le 0, \mbox{ } \forall t \in \mathcal{T},
\label{eq:storagestocksizing}
\end{equation}

As stated in \autoref{eq:potential} for conversion technologies, storage technology might have a maximum storage capacity. This is expressed as 
\begin{equation}
(\underbar{$\epsilon$}^{n} + E^{n}) - \bar{\epsilon}^{n} \le 0,
\label{eq:storagepotential}
\end{equation}
where $\bar{\epsilon}^n \in \mathbb{R}_+$ represents the maximum stock capacity that may be deployed. 

A minimum inventory level may be used and described as follows


\begin{equation}
\sigma^{n}(\underbar{$\epsilon$}^{n} + E^{n}) - e_{t}^n \le 0, \mbox{ } t \in \mathcal{T},
\label{eq:storageminimumsoc}
\end{equation}
where $\sigma^{n} \in [0, 1]$ is the mimum level of the inventory as a proportion of the installed capacity. 

The maximum inflow is modelled using an internal variable that is also constant throughout the time horizon considered, as in Eq. \eqref{eq:conversioncapacitystatic}. It is defined as follows
\begin{equation}
q_{it}^n - (\underbar{$\kappa$}^{n} + K^{n}) \le 0, \mbox{ } \forall t \in \mathcal{T},
\label{eq:storageflowplussizing}
\end{equation}
where $\underbar{$\kappa$}^n \in \mathbb{R}_+$ denotes the existing flow capacity and $K^n \in \mathbb{R}_+$ is used as shorthand for the new capacity. The maximum in- and outflows may be asymmetric, depending on the properties of the underlying technology, which is modelled via
\textcolor{red}{The maximum in- and outflows may be asymmetric, depending on the properties of the underlying technology, which is modelled via}
\begin{equation}
q_{jt}^n - \rho^{n} (\underbar{$\kappa$}^{n} + K^{n}) \le 0, \mbox{ } \forall t \in \mathcal{T},
\label{eq:storageflowminussizing}
\end{equation}
where $\rho^n \in \mathbb{R}_+$ represents the maximum discharge-to-charge ratio.
Finally the local objective for node $n$ can be written as follows
\begin{equation}
F^n = \Big[\nu (\varsigma^{n} + \vartheta^{n}_{f}) E^{n} + \sum_{t \in \mathcal{T}} \vartheta_{t,v}^n e_{t}^n\Big] + \Big[\nu (\zeta^{n} + \theta^n_f) K^{n} + \sum_{t \in \mathcal{T}} \theta_{t,v}^n q_{it}^n \delta t \Big].
\label{eq:objectivestorage}
\end{equation}
where $\varsigma^n \in \mathbb{R}_+$ and $\zeta^n \in \mathbb{R}_+$ represent the stock and flow components of CAPEX, $\vartheta_f^n \in \mathbb{R}_+$ and $\theta_f^n \in \mathbb{R}_+$ model the stock and flow components of FOM costs, while $\vartheta_{t,v}^n \in \mathbb{R}_+$ and $\theta_{t,v}^n \in \mathbb{R}_+$ represent the stock and flow components of VOM costs, which may be time-dependent.

\subsubsection{Hyperedges}
\textbf{Conservation}: Let $e\in \mathcal{E}$ be a conservation hyperedge which ensure that all the flows going in and out of a given commodity are respected. 
\begin{equation}
\sum_{n \in e_T} q_{it}^n - \sum_{n \in e_H} q_{it}^n - \lambda_{t}^e = 0, \mbox{ } \forall t \in \mathcal{T},
\label{eq:conservationhyperedge}
\end{equation}



\end{document}