\section{Conclusions}

This study provides evidence that information obtained from Twitter posts serves as a strong indicator for stock price movements. In particular, Twitter sentiment scores are highly correlated with price volatility and improve the performance of predictive models for financial markets. We conclude that sentiments provide valuable insights on events around and contribute to better capturing underlying market dynamics. 

The main goal of this paper is to investigate whether sentence embeddings yield better results in stock price prediction compared to the sentiment analysis approach. In the majority of conducted experiments, the sentiment approach outperforms the embedding vectors method. This fact might be counterintuitive because embeddings seem to encompass more valuable contextual information. However, sentiments tend to represent information in a more concise way, bringing less noise into the prediction model. Nevertheless, the embedding approach still has an advantage that it does not require an additional model for sentiment extraction and the consequent quality verification of that procedure. It is important to note the limitation of the conducted research. We made a comparison analysis only within restricted use case of Twitter posts and stocks of top companies from NASDAQ.  

\section{Statement on computational resources and environmental impact}  

We used a NVIDIA GeForce RTX 3080 Ti GPU to train the models, extract the sentiment score and make embedding vectors from the tweets using BERT model \cite{all-MiniLM-L6-v2}. Two NVIDIA A100 80GB PCIe GPUs were used for testing out the MPNet \cite{all-mpnet-base-v2} approach and inferencing. This work contributed totally 6.12 kg equivalent $CO_{2}$ emissions. The carbon emissions information was generated using the open-source library \textit{eco2AI}\footnote{Source code for \textit{eco2AI} is available at \url{https://github.com/sb-ai-lab/Eco2AI}} \cite{budennyy2023eco2ai}. 

% The embedding vectors method demonstrated similar performance in comparison with sentiment approach for the vast majority of conducted experiments. It is especially evident for the 3 day prediction window experiment directly eliminating the need to solve the problem of sentiment extraction without a significant loss in quality for final predictions.

% Sentiment extraction requires a rigorous process of feature extraction and quality verification in case we don't have manually labeled data, which can be time-consuming. On the other hand, sentence embedding approach does not require verification and can produce similar results to sentiment extraction retaining more of the semantic and contextual information contained in the text. Nevertheless, the model training time for sentence embedding is significantly longer. These findings suggest that sentence embedding could be considered a robust solution for stock price prediction, due to its similar performance to sentiment extraction, despite the longer model training time.

% On the other hand the results of the experiments demonstrated that the sentiment polarity extraction approach still slightly outperforms sentence embeddings in terms of accuracy and training time for predicting the stock closing price 3 and 5 days ahead. Also, increasing the dimensionality of sentence embeddings did not yield better results and instead worsened the performance of the Temporal Fusion Transformer model. Additionally, in case of the 3 days ahead approach, it received an improvement for sentence embeddings only for companies where sentiment was not a good predictor in the first place.

% This suggests that the choice between sentiment polarity extraction and sentence embeddings as the preferred approach for closing price prediction can possibly depend on the specific task and the prediction horizon, as well as the effectiveness of sentiment as a predictor in the given context.