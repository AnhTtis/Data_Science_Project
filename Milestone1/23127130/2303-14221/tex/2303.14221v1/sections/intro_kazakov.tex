\section{Introduction}

Stock price prediction is a challenging problem in the financial domain that draws significant interest from researchers. There is a massive amount of groundwork on machine learning applications for forecasting financial market trends. The developed methods could be divided into two streams: fundamental and technical analysis. 

Technical analysis relies solely on historical structured data of stock markets. Such analysis is extensively used with machine learning techniques
\cite{AYALA2021107119, PENG2021100060}. These studies demonstrate that technical indicators such as Exponential Moving Average (EMA) and Moving Average
Convergence/Divergence (MACD) can be used to increase the profitability of trading signals.

In contrast, fundamental analysis focuses on any useful information outside of historical market data regarding
the stock in question, such as the financial environment, law regulations, social networks, geopolitical stability, and news. 
While both approaches are usually used separately, recent studies show that a combination of the two can yield more accurate predictions \cite{Nti2020}.

With the advancements in machine learning and natural language processing (NLP), researchers explore the leveraging
of various features extracted from textual data for predicting stock prices. Such NLP technique as sentiment analysis allows the extraction of the sentiment or emotion from a piece of text, which can be used to infer the overall sentiment of the market towards a particular company or stock. 
Information representation in the context of financial markets is proposed to be viewed as explicit or implicit. 

The implicit representation involves extracting sentiment polarity directly from text (by a pre-trained NLP classifier) and then using this information to assess the expected reaction of a signal. This typically involves classifying the text into positive or negative categories based on its perceived sentiment towards market and using this classification as a feature in a further machine learning model to predict price movements. This approach has the advantage of being relatively simple and straightforward to implement, but it has the drawback of losing important semantic features and contextual information in the text. Moreover, the implicit approach relies on the product of the used sentiment analysis algorithm, whose inaccurate operation can introduce distortions into the work of the final predictive model. 

The explicit approach involves creating an embedding from the text directly and then using this embedding to predict the reaction for signal change \cite{10.1007/978-3-030-34223-4_5}. An embedding is a mathematical representation of a text in a lower-dimensional space, which can capture the meaning and context of the word or phrase. By creating an embedding from text, this approach is able to retain more of the semantic links contained in the text, which may be useful for predicting financial market trends. However, this approach may be more complex and time-consuming to implement, as it involves creating and training an embedding model on the text data.

After analyzing two ways of representing information related to the stock market, we can conclude that each of them has its own strengths and weaknesses. In this work, our aim is to compare the performance of the stock price prediction model built on the basis of either implicit or explicit knowledge representations.


