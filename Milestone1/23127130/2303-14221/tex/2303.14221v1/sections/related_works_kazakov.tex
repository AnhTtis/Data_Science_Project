\section{Related work}
% Stock price prediction is a challenging problem in the financial domain and has attracted significant attention from researchers
% over the years. 

One of the main streams of the research is dedicated to studying the usage of purely technical analysis trading indicators and historical data in combination with statistical methods for stock price prediction with machine learning. Many researchers employ the GRU-based models in this tasks \cite{Aseeri_2023, app13010222, GUPTA2022117986} while the others explore the transformer architecture in this field \cite{WANG2022118128, numhtml, 10027785}. The goal of such research is to provide investors and other financial actors with an insight of price movements to make more informed financial decisions relying on technical analysis as the foundation.

With the development of NLP methods, the number of works aiming to predict stock price trends and volatility proposed a combination of financial news and social media data is increasing \cite{Khan2022}. \cite{10.1145/3159652.3159690} pointed out the lack of trustworthiness and comprehensiveness of online content collected from social media and low quality news sources. Clinical trial announcements were used as a source of sentiment in pursuit of predicting pharma stock market price changes by \cite{https://doi.org/10.48550/arxiv.2208.07248}.  Another research used Valence Aware Dictionary and Sentiment Reasoning (VADER) \cite{hutto2014vader} for sentiment analysis \cite{math10122001}. \cite{Li2022} proposed a novel Deep Learning Transformer Encoder Attention (TEA) model. \cite{https://doi.org/10.48550/arxiv.2005.02527} considered such under-explored content as Environmental, Social, and Corporate Governance (ESG) news flow for volatility forecasting. \cite{AUDRINO2020334} analyzed the impact of sentiment and attention variables on the stock market volatility by adding search engine and information consumption data on top of widely used social media and news texts.

Sentiment analysis involves extracting the sentiment
or emotion from a piece of text, which can be used to infer the overall sentiment of the market towards a particular company
or stock. Sentiment scores are commonly used in stock price prediction studies \cite{Khan2022, 10.1145/3159652.3159690, math10122001, Li2022, https://doi.org/10.48550/arxiv.2005.02527, AUDRINO2020334} as they are easy to compute and provide a
simple metric to gauge market sentiment.
On the other hand, embedding vectors are dense numerical representations of words or phrases that capture the semantic
meaning of the text. They are created by mapping words or phrases to high-dimensional vectors in a way that similar words or phrases are located close to each other in this vector space using techniques like Word2vec \cite{https://doi.org/10.48550/arxiv.1301.3781} or GloVe \cite{pennington-etal-2014-glove}, BERT \cite{https://doi.org/10.48550/arxiv.1810.04805} and GPT \cite{https://doi.org/10.48550/arxiv.2202.08904}, and have been shown to
be effective in capturing complex relationships between words and phrases. 

There are several works that concentrate on the usage of text semantics in the context of stock price prediction. \cite{10.1007/978-3-030-34223-4_5} proposed a Multi-head Attention Fusion Network to exploit aspect-level semantic information from texts to enhance prediction. \cite{Lin2022} developed a Spatial-temporal attention-based convolutional network. The authors converted news articles into 300-dimensional vector embeddings and used them as a feature in their model. They noted that in the case of the utilization of the preprocessed text features, latent information in the text is lost because the relationships between the text and stock price are not considered. \cite{Chandola2022} employed Word2Vec and LSTM algorithms, while \cite{Chen2022} adopted a transformer architecture using high level textual features. The main conclusion derived from these papers can be formulated as follows - the effectiveness of exploiting and fusing semantic aspect-level textual information leads to an improved performance upon the baselines. This fact means that the topic needs further investigation and refining.

% The main advantage of the embeddings over sentiment analysis is that they capture more complex relationships in the data. While sentiment analysis can indicate whether a tweet or news article about a certain company has either negative or positive attitude towards the stock the embeddings try to capture how that text relates to other pieces of the information allowing for a more deep and sophisticated analysis without dropping the important semantic features. For example in one recent study \cite{10.1007/978-3-030-34223-4_5} the authors claim that their Multi-head Attention Fusion Network utilizing text semantics shows the superior performance against several strong baselines.

The main advantage of embeddings over sentiment analysis is that they capture more complex relationships in the data. Despite the promising results of using embedding vectors as a feature \cite{Chandola2022, Chen2022}, there is still a research gap in the comparison of the effectiveness of using embedding vectors versus sentiment scores in stock price prediction. Moreover, reserachers tend to experiment with different datasets and methodologies, making it difficult to draw meaningful comparisons.

% While sentiment scores are commonly used as a feature in stock price prediction, they have limitations. For
% instance, they rely on the accuracy of the sentiment analysis algorithm used and may not capture the nuances of market
% sentiment. In contrast, embedding vectors are based on the underlying semantic relationships between words and phrases and
% are less prone to such limitation.

% In conclusion, while both sentiment scores and embedding vectors have been shown to be effective features for stock price
% prediction, there is a need for more research to directly compare the effectiveness of the two approaches. This research gap
% presents an opportunity for future studies to explore the potential of using embedding vectors as a feature in stock price
% prediction and compare their effectiveness with that of sentiment scores