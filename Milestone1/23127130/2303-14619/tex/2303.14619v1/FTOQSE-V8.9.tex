%% LyX 2.3.7 created this file.  For more info, see http://www.lyx.org/.
%% Do not edit unless you really know what you are doing.
\documentclass[twocolumn,english,pra, aps, twocolumn, superscriptaddress]{revtex4-1}
\usepackage[T1]{fontenc}
\usepackage[latin9]{inputenc}
\setcounter{secnumdepth}{3}
\usepackage{color}
\usepackage{mathrsfs}
\usepackage{amstext}
\usepackage{amssymb}
\usepackage{graphicx}
\usepackage{esint}

\makeatletter
%%%%%%%%%%%%%%%%%%%%%%%%%%%%%% User specified LaTeX commands.
\usepackage{xcolor}
\usepackage[colorlinks,citecolor=red,linkcolor=blue]{hyperref}
%\textwidth=300pt

\makeatother

\usepackage{babel}
\begin{document}
\title{Finite-Time Optimization of Quantum Szilard heat engine}
\author{Tan-Ji Zhou}
\address{Graduate School of China Academy of Engineering Physics, No. 10 Xibeiwang
East Road, Haidian District, Beijing, 100193, China}
\address{Department of Physics, Peking University}
\author{Yu-Han Ma}
\email{yhma@bnu.edu.cn}

\address{Department of Physics, Beijing Normal University, Beijing 100875,
China}
\address{Graduate School of China Academy of Engineering Physics, No. 10 Xibeiwang
East Road, Haidian District, Beijing, 100193, China}
\author{C. P. Sun}
\email{suncp@gscaep.ac.cn}

\address{Graduate School of China Academy of Engineering Physics, No. 10 Xibeiwang
East Road, Haidian District, Beijing, 100193, China}
\address{Beijing Computational Science Research Center, Beijing 100193, China}
\begin{abstract}
We propose a finite-time quantum Szilard engine (QSE) with a quantum
particle with spin as the working substance (WS) to accelerate the
operation of information engines. A Maxwell's demon (MD) is introduced
to probe the spin state within a finite measurement time $t_{{\rm M}}$
to capture the which-way information of the particle, quantified by
the mutual information $I(t_{\rm{M}})$ between WS and MD. We establish
that the efficiency $\eta$ of QSE is bounded by $\eta\leq1-(1-\eta_{\rm{C}}){\rm ln}2/I(t_{{\rm M}})$,
where $I(t_{{\rm M}})/\rm {ln}2$ characterizes the
ideality of quantum measurement, and approaches $1$ for the Carnot
efficiency reached under ideal measurement in quasi-static regime.
We find that the power of QSE scales as $P\propto t_{{\rm M}}^{3}$ in the
short-time regime and as $P\propto t_{\rm M}^{-1}$ in the long-time regime.
Additionally, considering the energy cost for erasing the MD's memory
required by Landauer's principle, there exists a threshold time that
guarantees QSE to output positive work.
\end{abstract}
\maketitle
\textit{\textcolor{blue}{Introduction.}}---The field of information
heat engines \citep{1929Szilard,2009MaruyamaRMP.81.1} was developed
to establish the fundamental role of information in thermodynamics
\citep{2009MaruyamaRMP.81.1,2015thermodynamicsinformation}. These
engines operate by generating work through information processing,
rather than conventional heat absorption. A well-known example is
the Szilard engine \citep{1929Szilard}, which involves the Maxwell's
demon (MD) probing the position of particles in a single heat reservoir
to extract work. The apparent violation of the second law of thermodynamics
in this case is resolved \citep{quan2006maxwell,2009MaruyamaRMP.81.1,dong2011quantum,kim2011quantum,cai2012multiparticle,2015thermodynamicsinformation}
by taking into account the energy cost of erasing the demon's memory,
as per Landauer's principle \citep{1961Landauer5392446}. Several
quantum Szilard engines (QSEs) have been proposed \citep{quan2006maxwell,2009MaruyamaRMP.81.1,dong2011quantum,kim2011quantum,cai2012multiparticle,2015thermodynamicsinformation}
to explore the connection between information and thermodynamics in
quantum regime. 
\begin{figure}[h]
\centering{}\includegraphics[width=8.5cm]{Fig1}\caption{\label{fig:Schematic-diagram-of}Schematic illustration of the Szilard
engine cycle. MD first observes the position of the particle (through
the detection of the particle's spin) in the box (stage I). Then proper
operation is chosen to extract work from collisions of particle and
baffle (stage II). After the work done process, the information of
the particle's position recorded by MD is erased, and the entire system
returns to its initial state (stage III).}
\end{figure}

In QSEs, the demon's recording of the particle's position is equivalent
to the establishment of entanglement between MD and the particle.
However, high-speed cyclic operation needs to be accomplished within
a short measurement time to achieve optimal power of QSE. This makes
it impossible to carry out an ideal quantum measurement, according
to the decoherence theory for quantum measurement \citep{joos2013decoherence,zeh1970interpretation,zurekDecoherenceEinselectionQuantum2003,schlosshauerDecoherenceMeasurementProblem2005}.

In this Letter, we propose and study a finite-time QSE model to analyze
the influence of non-ideal quantum measurement on engine performance.
We investigate the information correlation between the particle and
MD during quantum measurement and analyze the dynamics of the which-way
information recorded by MD. Our results demonstrate the fundamental
trade-off between the power and efficiency of QSE due to non-ideal
measurement. Although information recording does not consume additional
energy, the non-ideality of measurement affects the thermodynamic
signatures of the QSE. 

\textit{\textcolor{blue}{Operation cycle of a QSE.}}---As illustrated
in Fig. \ref{fig:Schematic-diagram-of}, the working substance (WS)
of the considered QSE is specified as a single particle with spin-1/2.
The MD probes the particle's position through the detection of its
spin state to obtain the which-way information of the particle, which
can be further used to output work. The operation cycle of the QSE
involves three stages, namely, information recording through quantum
measurement (stage I), work outputting (stage II) and information
erasing (stage III).

\textbf{In stage I}, the particle is injected into a cylinder, and
the inhomogeneous magnetic field which perpendicular to the direction
of the particle's incidence is turned on. This results in the entanglement
of the particle's spin and spatial states, enabling the Maxwell's
demon to detect the particle's position through measurement of its
spin state.

After stage I, the spin-magnetic field interaction gives rise to a
time-dependent correlation $I(t_{M})$ between the spin state and
the spatial state, and the deflection of the particle can be inferred
from the spin detection. Hereafter, the duration of stage I $t_{M}$
is called the measurement time. We qualify such correlation with the
mutual information \citep{kullback1997information} as

\begin{equation}
I(t_{\rm{M}})=I_{\rm{DP}}\equiv S_{{\rm D}}+S_{{\rm p}}-S_{{\rm DP}},
\end{equation}
where $S_{{\rm DP}}$, $S_{{\rm D}}$ and $S_{{\rm p}}$ are the Shannon
entropy of the total system, the MD and the particle, respectively.

\textbf{In stage II}, the magnetic field is turned off and a suitable
protocol is chosen according to MD's memory to enable the extraction
of work from collisions of the particle and baffle. The engine is
contacting with a heat reservoir with temperature $T_{{\rm H}}$ when
it outputs work. At the end of stage II, the correlation between the
particle and MD vanished due to their thermalization in the reservoir,
thereby contributing to an increase of their total entropy, i.e.,
$\Delta S_{{\rm DP}}=I(t_{\rm{M}})$ \citep{li2017production,Still2020PRL.124.050601}.

The generalized free energy \citep{crooks2007beyond,Sagawa2008PRLFreeEnergy,takara2010generalization,lostaglio2015description,deffner2016quantum,Still2020PRL.124.050601}
of the total system containing the particle and MD

\begin{equation}
F\equiv U-k_{{\rm B}}T_{{\rm H}}S_{{\rm DP}}\label{eq:F}
\end{equation}
is defined by Shannon entropy $S_{{\rm DP}}$ instead of the thermodynamic
entropy in conventional cases. Here, $U$ is the energy of the total
system. Consequently, it follows from Eq. (\ref{eq:F}) and the 2nd
law of thermodynamics ($W\text{\ensuremath{\leq-\Delta F}}$) that
the output work is bounded above by the spin-position correlation
of the particle as

\begin{equation}
W_{{\rm O}}\leq k_{{\rm B}}T_{{\rm H}}I(t_{M}),\label{eq:Wo}
\end{equation}
where the facts that the internal energy of an ideal gas particle
is unchanged in the isothermal process and MD's energy remains unchanged
throughout the whole cycle have been used. As a result, heat in the
amount of $Q_{{\rm O}}=W_{{\rm O}}$ is absorbed from the heat reservoir.

\textbf{In stage III}, the information recorded by MD is erased, and
the entire system returns to its initial state. In this process, the
engine is in contact with another reservoir with temperature $T_{{\rm C}}$.
According to Landauer\textquoteright s principle, work in the amount
of $W_{{\rm E}}\geq k_{{\rm B}}T_{{\rm C}}{\rm ln}2$ must beapplied
to the system \citep{1961Landauer5392446} to erase MD's memory (to
initialize the spin state in our case). The same amount of heat $Q_{{\rm E}}=W_{{\rm E}}$
is dissipated into the reservoir according to the first law of thermodynamics
in this isothermal process.

\textit{\textcolor{blue}{Realization of the cycle.}}---Taken Stern-Gerlach
experiment \citep{gerlach1922experimentelle,bohm2012quantum,platt1992modern,griffithsIntroductionQuantumMechanics2018}
as an realization of stage I, the correlation between spin state and
the spatial state of the particle can be specifically obtained (See
Sec. I of the Supplementary Materials (SM) for details \citep{S-M}).
We define the measurement ideality as $\text{\ensuremath{\mathcal{M}\equiv I/{\rm ln}2}}$
and it is explicitly written as \citep{S-M}

\begin{equation}
\mathcal{M}(\tilde{t})=1+\frac{p(\tilde{t}){\rm ln}p(\tilde{t})+\left[1-p(\tilde{t})\right]{\rm ln}\left[1-p(\tilde{t})\right]}{{\rm ln}2}
\end{equation}
with

\begin{equation}
p(\tilde{t})=\frac{1}{2}\left[1+{\rm erf}\left(\frac{\alpha\tilde{t}^{2}}{\sqrt{2\tilde{t}^{2}+8}}\right)\right]
\end{equation}
the possibility that MD can accurately infer the particle's position.
Here, ${\rm erf}(z)=2\pi^{-1/2}\int_{0}^{z}\mathrm{exp}(-u^{2})\text{{\rm d}}u$
is the Gaussian error function, $\tilde{t}\equiv t_{\rm{M}}/\tau_{\rm{M}}$
is the dimensionless measurement time. The characteristic time of
the measurement process $\tau_{\rm{M}}\equiv ma^{2}/\hbar$ is determined
by the particle mass $m$ and the width of the particle's spatial
wave packet $a$. In addition, $\text{\ensuremath{\alpha}}\equiv ma^{3}f/\hbar^{2}$
is proportional to the magnetic force $f$ acting on the spin.

We further plot $\mathcal{M}(\tilde{t})$ as a function of $\tilde{t}$
in Fig. \ref{fig:Normalized-Ideality_Efficiency-and-power} (a). It
is clearly seen in this figure that for a given measurement time,
larger $\alpha$ results in a greater degree of correlation generated
between the spin degrees of freedom and the spatial degrees of freedom
of the particle, and thus MD can more accurately distinguish the position
of the particle. As the measurement time increases, the spin-position
correlation of the particle monotonically increases. In the ideal
measurement process, we have $\lim_{\tilde{t}\rightarrow\infty}\mathcal{M}(\tilde{t})=1$
($p(\tilde{t})\rightarrow1$), which is the plateau approached by
the three curves in Fig. \ref{fig:Normalized-Ideality_Efficiency-and-power}(a).

\begin{figure}
\begin{raggedright}
\includegraphics[width=4cm]{Fig2a}\includegraphics[width=4cm]{Fig2b}
\par\end{raggedright}
\begin{centering}
\includegraphics[width=8.5cm]{Fig2c}
\par\end{centering}
\centering{}\caption{\label{fig:Normalized-Ideality_Efficiency-and-power}(a) Measurement
ideality$\text{\ensuremath{\mathcal{M}(t)=I(t)/{\rm ln}2}}$ with
different $\alpha$. The blue dashed curve, orange solid curve, and
yellow dotted curve are plotted with $\alpha=0.2$, $\alpha=0.4$,
and $\alpha=2$, respectively. (b) Efficiency and power of the engine
as the function of dimensionalized measurement time, where $\eta_{{\rm C}}=0.6$,
$\alpha=0.4$ are used. The grey solid curve and the black dotted
curve mark $P(\tilde{t})$ and $P_{\mathrm{O}}(\tilde{t})$ respectively,
which are normalized by $\max\left[P_{\mathrm{O}}(\tilde{t})\right]$,
and the red dashed line marks $\eta(\tilde{t})$ which is normalized
by $\eta_{\mathrm{C}}$. (c) The threshold time $t_{0}$ as a function
of the Carnot efficiency $\eta_{\mathrm{C}}$ and the characteristic
parameter $\text{\ensuremath{\alpha}}=ma^{3}f/\hbar^{2}$. The yellow
area corresponds to higher value of $t_{0}$ and the blue area corresponds
to the lower part. $t_{0}$ is numerically solved from Eq. (\ref{eq:W>0})
for each given set of $\eta_{{\rm C}}$ and $\text{\ensuremath{\alpha}}$.}
\end{figure}

We emphasize here that the essential difference between the Szilard
engine with finite-time quantum measurement proposed in this work
and the conventional one is reflected in the maximum extracted work
allowed by the information recording process. In the conventional
Szilard engine, MD can deterministically observe the position of the
particle, enabling a precise judgment on which operation to choose
to extract work. However, in our case, the dynamics of the measurement
renders the inference of the particle's position from its spin state
probabilistic. The possibility of selecting an incorrect operation
for extracting work will reduce the output work, which has been recently
discussed for the correlated variable Szilard engine in non-dynamical
case \citep{Still2020PRL.124.050601}.

In particular, $\tilde{t}\rightarrow0$ is associated with invalid
measurement with $\mathcal{M}\rightarrow0$. Consequently, MD knows
nothing about the particle's position, and thus no work can be extracted.
On the other hand, when an ideal measurement with $\mathcal{M}=1$
is achieved in the quasi-static regime ($\tilde{t}\rightarrow\infty$),
Eq. (\ref{eq:Wo}) recovers the result $W_{{\rm O}}^{{\rm ideal}}=k_{{\rm B}}T_{{\rm H}}\ln2$
of the conventional Szilard engine. Hence we can rewrite the upper
bound of the outputting work in Eq. (\ref{eq:Wo}) as $W_{{\rm O}}\leq\mathcal{M}W_{{\rm O}}^{{\rm ideal}}.$

\textit{\textcolor{blue}{Efficiency and power of the QSE.}}---With
the energy conversion relations in the above cycle analysis in mind,
the total work of the QSE performing in one cycle reads

\begin{equation}
W=W_{{\rm O}}-W_{{\rm E}}\leq\mathcal{M}W_{{\rm O}}^{{\rm ideal}}-k_{{\rm B}}T_{{\rm C}}{\rm ln}2,\label{eq:work}
\end{equation}
and thus the efficiency of the QSE $\eta\equiv W/Q_{{\rm O}}=W/W_{{\rm O}}$
satisfies

\begin{equation}
\eta\leq1-\frac{k_{{\rm B}}T_{{\rm C}}{\rm ln}2}{\mathcal{M}W_{{\rm O}}^{{\rm ideal}}}=1-\frac{\left(1-\eta_{{\rm C}}\right).}{\mathscr{\mathcal{M}}}\label{eq:eta}
\end{equation}
The above inequality can be reorganized in terms of the Carnot efficiency
$\eta_{{\rm C}}\equiv1-T_{{\rm C}}/T_{{\rm H}}$ as

\begin{equation}
\frac{1-\eta_{{\rm C}}}{1-\eta}\leq\mathcal{M},\label{eq:eta_ineq}
\end{equation}
which is the main result of this Letter, revealing fundamental upper
bound on energy conversion efficiency set by information measurement
ideality. Here, the equal sign is saturated if and only if there is
no irreversibility in stage II and stage III.

By definition, the power of the QSE is $P\equiv W/t_{{\rm cycle}}$,
where the cycle time is $t_{{\rm cycle}}=t_{{\rm M}}+t_{{\rm O}}+t_{{\rm E}}$
with the duration of the work outputting (information erasing) process
to be $t_{{\rm O}}$ ($t_{{\rm E}}$). In order to significantly demonstrate
the impact of finite-time measurements on the performance of QSE,
we first consider that the coupling between the system and measurement
instrument is weak enough so that the finite-time behavior of the
information recording dominates in the considered time scale ($t_{{\rm E}},t_{{\rm O}}\ll t_{{\rm M}}$)
\citep{S-M}. In this case, the irreversibility of the work outputting
and information erasing processes can be ignored, and the power of
the engine reduces to

\begin{equation}
P=\frac{W}{t_{{\rm M}}}\leq\frac{\mathcal{M}W_{{\rm O}}^{{\rm ideal}}-k_{{\rm B}}T_{{\rm C}}{\rm ln}2}{t_{{\rm M}}}=\frac{\eta(\tilde{t})k_{{\rm B}}T_{{\rm H}}I(\tilde{t})}{\tau_{{\rm M}}\tilde{t}}.
\end{equation}
We further define the output power associated with the work outputting
process as $P_{{\rm O}}(\tilde{t})\equiv W_{{\rm O}}/t_{{\rm M}}=k_{{\rm B}}T_{{\rm H}}I(\tilde{t})/(\tau_{{\rm M}}\tilde{t})$,
where the equal sign in Eq. (\ref{eq:eta_ineq}) is taken.

We illustrate $\eta(\tilde{t})$, $P(\tilde{t})$, and $P_{{\rm O}}(\tilde{t})$
as functions of non-dimensionalized measurement time $\tilde{t}$
in Fig. \ref{fig:Normalized-Ideality_Efficiency-and-power} (b), where
$P(\tilde{t})$ and $P_{{\rm O}}(\tilde{t})$ are non-dimensionalized
by the maximum value of $P_{{\rm O}}(\tilde{t})$, and $\eta(\tilde{t})$
is normalized by $\eta_{\mathrm{C}}$. As illustrated in this figure,
$P_{{\rm O}}(\tilde{t})$ increases from zero with the increase of
measurement time $\tilde{t}$ until it reaches its maximum value and
then decreases gradually, which is positive for all $\tilde{t}>0$.
In the short time regime, $\mathcal{M}(\tilde{t})\propto\alpha^{2}\tilde{t}^{4}$
\citep{S-M}, and thus $P_{{\rm O}}(\tilde{t}\ll1)\propto\alpha^{2}\tilde{t}^{3}$,
which indicates the output power is proportional to the square of
the strength of measurement ($\alpha^{2}$). On the other hand, in
the long time regime, $\mathcal{M}(\tilde{t})$ tends to $1$ exponentially
(see Fig. \ref{fig:Normalized-Ideality_Efficiency-and-power} (a))
such that $P_{{\rm O}}(\tilde{t}\gg1)\propto\tilde{t}^{-1}$.

\textit{\textcolor{blue}{Threshold time for positive work.}}--- It
is observed in Fig. \ref{fig:Normalized-Ideality_Efficiency-and-power}
(b) that the power $P(\tilde{t})$ and the efficiency $\eta(\tilde{t})$
of the information heat engine are positive only when the time exceeds
a threshold value. This can be understood from Eq. (\ref{eq:work})
that the positive work condition $W\geq0$ sets a lower bound for
measurement ideality as

\begin{equation}
\mathcal{M}(\tilde{t})\geq1-\eta_{{\rm C}}.\label{eq:W>0}
\end{equation}
Naturally, we define the threshold time $t_{0}\equiv\mathcal{M}^{-1}(1-\eta_{{\rm C}})$
as the lower bound of the dimensionless measurement time that holds
the above inequality. The positive work condition for the engine is
thus converted to a restriction on the measurement time $\tilde{t}>t_{0}$.

For different value of $\eta_{{\rm C}}$, the shape of the curves
in Fig. \ref{fig:Normalized-Ideality_Efficiency-and-power} (b) will
change on the premise that the overall characteristics remain unchanged,
resulting in different values of $t_{0}$. We numerically solve $t_{0}$
from Eq. (\ref{eq:W>0}) for each given set of parameters $\left\{ \eta_{{\rm C}}\in(0.1,1),\ensuremath{\alpha\in(0.1,1)}\right\} $,
and then plot $t_{0}$ as a function of $\eta_{\mathrm{C}}$ and $\text{\ensuremath{\alpha}}$
in Fig. \ref{fig:Normalized-Ideality_Efficiency-and-power} (c). This
figure shows that the threshold time increase remarkably as $\alpha$
grows beyond a certain value, while it decreases rather gradually
with the increase of $\eta_{{\rm C}}$ for relatively large $\text{\ensuremath{\alpha}}$.
Therefore, we can increase $\alpha$ to achieve a stronger measurement,
thereby effectively reducing $t_{0}$. By doing this, the engine can
output positive work in a shorter cycle time. In other words, increasing
$\alpha$ can speed up the engine operation without decreasing the
output work, thereby improving the power of the engine.

\textit{\textcolor{blue}{Optimal operation of the QSE.}}--- When
the power of the information heat engine $P(\tilde{t})$ reaches its
maximum at the measurement time $\tilde{t}=\tilde{t}^{*}$, i.e.,
$P(\tilde{t}^{*})={\rm max}\left\{ P(\tilde{t})|\tilde{t}>0\right\} \equiv P_{{\rm max}}$,
the corresponding efficiency is referred to as the efficiency at maximum
power (EMP). In Fig. \ref{fig:EMP} (a), the EMP of the engine $\text{\ensuremath{\eta_{{\rm MP}}}}$as
a function of $\eta_{{\rm C}}$ is plotted with the blue solid curve.
The typical efficiency $\eta_{+}=\eta_{\mathrm{C}}/(2-\eta_{\mathrm{C}})$,
known as the upper bound of the EMP of low-dissipation Carnot heat
engine \citep{EspositoPRL2010}, is marked with the orange dash-dotted
curve. It can be observed from Fig. \ref{fig:EMP} that the EMP of
QSE can exceed $\eta_{+}$ and finally approaches the Carnot efficiency
$\eta_{\mathrm{C}}=1-T_{\mathrm{C}}/T_{\mathrm{H}}$ (red dashed line)
as $\eta_{{\rm C}}\to1$. In this example, $\alpha=0.4$ is used.
Note that $\eta_{+}$ is the overall upper bound for EMP of all near-equilibrium
engines \citep{ZCTuCPB}, the significantly higher EMP of our QSE
indicates the quantum thermodynamical advantage away from equilibrium
regime. In addition, the dependence of $\eta_{{\rm MP}}/\eta_{{\rm C}}$
on $\alpha$ is shown in the inset figure, indicating that the EMP
of QSE gradually increases with $\alpha$ and approaches the Carnot
efficiency very closely. 
\begin{figure}
\begin{raggedright}
\includegraphics[width=8.5cm]{Fig3a}
\par\end{raggedright}
\begin{raggedright}
\includegraphics[width=8.5cm]{Fig3b}
\par\end{raggedright}
\centering{}\caption{\label{fig:EMP} (a) Efficiency at maximum power of QSE as a function
of $\eta_{C}$ . The blue solid curve represents the efficiency at
maximum power (EMP) of the information engine, the typical bound of
EMP $\eta_{+}=\eta_{\rm{C}}/(2-\eta_{\rm{C}})$ is plotted
with the orange dash-dotted curve, and the red dashed line marks the
Carnot efficiency $\eta_{\rm{C}}=1-T_{\rm{C}}/T_{\rm{H}}$,
in this plot $\alpha=0.4$ is used. In the inset figure, EMP as a
function of $\alpha$ is plotted with the blue dotted curve, and $\eta_{{\rm C}}=0.6$
is used. (b) Power-efficiency trade-off relation of QSE with $\alpha=0.4$
and $\eta_{{\rm C}}=0.6$. In this figure, only the finite-time effect
of the information recording process (stage I) is considered. In the
inset figure, the maximum power $P_{\rm{max}}$ as a function
of $\alpha$ is plotted with the blue circle-dotted curve.}
\end{figure}

Moreover, the power-efficiency trade-off relation \citep{holubecMaximumEfficiencyLowdissipation2016,TradeoffrelationShiraishi,CavinaPRLtradeoffrelation,Constraintrelationyhma}
of our QSE is illustrated in Fig. \ref{fig:EMP} (b) with $\alpha=0.4$
and $\eta_{{\rm C}}=0.6$. To obtain this trade-off, $P(\tilde{t})$
and $\eta(\tilde{t})$ are calculated numerically by varying the measurement
time $\tilde{t}\in[1,10^{5}]$. Increasing $\tilde{t}$ results in
an increase in $\eta$, while $P$ first increases to reach its maximum
value and then decreases, as shown in Fig. \ref{fig:Normalized-Ideality_Efficiency-and-power}
(a). The EMP of the engine is associated with the rightmost edge of
the curve (red squared dot), which is pointed by the red arrow. The
maximum power $P_{\mathrm{max}}$ as a function of $\alpha$ is demonstrated
in the inset figure, which indicates that an increase in $\alpha$
will lead to a significant increase in $P_{\mathrm{max}}$.

Overall, the above obtained results demonstrate that the dynamics
of information recording realized by quantum measurement has a significant
impact on the operation of the Szilard engine. Specifically, more
complete information recording (larger $\alpha$) leads to higher
maximum power as well as higher EMP of the engine.

\textit{\textcolor{blue}{Remarks on finite-time effects in stages
II and III.}}\textit{---}In the above discussion, only the finite-time
behavior of the information recording process is taken into account.
When we consider the finite-time effects in both the information recording
and information erasing processes, we find that \textbf{i)} the EMP
of the engine can also exceed the upper limit of the efficiency of
low-dissipation Carnot heat engine at a certain range of $\eta_{{\rm C}}$;
\textbf{ii)} the QSE can operate anywhere within the envelop of the
power-efficiency trade-off, which is quite different from the single-variable
case shown in Fig. \ref{fig:EMP} (b) where the system can only operate
with power and efficiency located on the curve. On the other hand,
if the information recording is complete with an ideal measurement
with $I(t_{{\rm M}})=\ln2$, and the operation time of the measurement
process can be ignored in comparison with that of the work outputting
and information erasing process, then the information heat engine
can be exactly mapped into a Carnot-like heat engine, whose EMP satisfies
$\eta_{\mathrm{C}}/2\leq\eta_{\rm{MP}}\leq\eta_{\mathrm{C}}/(2-\eta_{\mathrm{C}})$.
The detailed calculations and discussion on these issues are given
in Sec. II of the SM \citep{S-M}.

\textit{\textcolor{blue}{Conclusions and outlooks.}}---In summary,
we propose a finite-time quantum Szilard heat engine model. The non-ideal
measurement of the working substance carried out by the demon allows
the engine cycle to run fast with high power at the expense of efficiency.
It is found that the engine is able to output positive work only after
the measurement time reaches a threshold, which reflects the dynamic
competition between the output work achieved by information recording
and the work consumed for erasing information. The optimal performance
of such information engine, characterized with the EMP and the power-efficiency
trade-off relation, is obtained quantitatively. These results demonstrate
that quantum measurements have potentially observable thermodynamic
effects, which have a critical impact on thermodynamic cycles of exchanging
information for energy.

This study bridges the gap between the dynamics of quantum measurement
and non-equilibrium thermodynamics, and shall bring new insights in
providing thermodynamic evidence for identifying different interpretations
of quantum measurement \citep{everett1957relative,griffiths1984consistent,zurekDecoherenceEinselectionQuantum2003,mittelstaedt2004interpretation,allahverdyan2013understanding}.
As potential extensions, the optimal control protocol \citep{yhmaoptimalcontrol,2020Optimal}
and the geometric optimization \citep{Crooks2007,maMinimalEnergyCost2022}
of the cycle, the finite-sized effect of the reservoir \citep{ondrechen1981maximum,izumida2014work,2020-finite-size,yuan2022optimizing}
involved in the work output processes and erase process, the many-body
effect of the working substance \citep{YHMaQPTHE} are expected to
be taken into future consideration. In addition, whether there is
a thermodynamic criterion for objectivity in quantum measurement \citep{zurekDecoherenceEinselectionQuantum2003,zurek2009quantum,riedel2016objective,li2018objectivity}
is a fundamental question worth exploring. Finally, inspired by the
obtained results on thermodynamic signatures of information recording
in quantum measurement, we leave a fascinating and challenging question
to end this paper: \textit{What thermodynamic consequences could be
caused by the finite-time memory of intelligent life.}

\textit{\textcolor{blue}{Acknowledgment}}\textcolor{blue}{.}---This
work is supported by the National Natural Science Foundation of China
(Grants No. 12088101, No. U1930402, and No. U1930403). Y.-H. Ma acknowledges
support from the China Postdoctoral Science Foundation (Grant No.
BX2021030).

\bibliographystyle{apsrev}
\bibliography{FTOSHE}

\end{document}
