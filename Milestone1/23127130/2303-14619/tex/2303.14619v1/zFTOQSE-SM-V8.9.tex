%% LyX 2.3.7 created this file.  For more info, see http://www.lyx.org/.
%% Do not edit unless you really know what you are doing.
\documentclass[english]{revtex4-2}
\usepackage[T1]{fontenc}
\usepackage[latin9]{inputenc}
\setcounter{secnumdepth}{3}
\usepackage{amsmath}
\usepackage{amssymb}
\usepackage{graphicx}

\makeatletter
%%%%%%%%%%%%%%%%%%%%%%%%%%%%%% User specified LaTeX commands.
\usepackage{xcolor}
\usepackage[colorlinks,citecolor=red,linkcolor=blue]{hyperref}
%\textwidth=300pt
\renewcommand\theequation{S\arabic{equation}}
%\renewcommand\theequation{S\arabic{figure}}

\@ifundefined{showcaptionsetup}{}{%
 \PassOptionsToPackage{caption=false}{subfig}}
\usepackage{subfig}
\makeatother

\usepackage{babel}
\begin{document}
\title{\textit{\Large{}Supplementary Materials for}{\Large{} ``Finite-Time
Optimization of Quantum Szilard heat engine''}}
\author{Tan-Ji Zhou}
\address{Graduate School of China Academy of Engineering Physics, No. 10 Xibeiwang
East Road, Haidian District, Beijing, 100193, China}
\address{Department of Physics, Peking University}
\author{Yu-Han Ma}
\email{yhma@bnu.edu.cn}

\address{Department of Physics, Beijing Normal University, Beijing 100875,
China}
\address{Graduate School of China Academy of Engineering Physics, No. 10 Xibeiwang
East Road, Haidian District, Beijing, 100193, China}
\author{C. P. Sun}
\email{suncp@gscaep.ac.cn}

\address{Graduate School of China Academy of Engineering Physics, No. 10 Xibeiwang
East Road, Haidian District, Beijing, 100193, China}
\address{Beijing Computational Science Research Center, Beijing 100193, China}

\maketitle
This document is devoted to providing the detailed derivations and
the supporting discussions to the main content of the Letter. In Sec.\ref{sec:Information-Recording},
we discuss the information recording in the Stern-Gerlach experiment
and obtain the spatial wave function, overlapping integral of the
wave functions, the conditional spatial distribution probability and
the correlation (mutual entropy $I(\tilde{t})$) between MD and the
spatial state of the particle. In Sec. \ref{sec:Thermodynamic-signature-of-full-finite-time-description},
we calculate the EMP of the Szilard heat engine in more general cases
with full finite time description of the cycle.

\section{\label{sec:Information-Recording}Information recording in the Stern-Gerlach
experiment}

The Schematic illustration of the Stern-Gerlach experiment \citep{gerlach1922experimentelle,bohm2012quantum,platt1992modern,griffithsIntroductionQuantumMechanics2018}
is shown in Fig. \ref{fig:Schematic-diagram-of-1}. Non-interacting
particles with mass $m$, momentum $p$, and magnetic moment $\vec{\mu}$
are injected along $z$ direction with velocity $v_{z}$. Physically,
the applied inhomogeneous magnetic field $B(x)$ along $x$ direction
causes entanglement between the spin and the spatial degrees of freedom
of the particles. Such entanglement leads to quantum correlation between
these two kinds of degrees of freedom. 
\begin{figure}
\includegraphics[width=8.5cm]{SM_Fig1}\caption{\label{fig:Schematic-diagram-of-1}Schematic illustration of Stern-Gerlach
experiment. The inhomogeneous magnetic field along $x$ direction
results in the entanglement between the spin degrees of freedom and
the spatial degrees of freedom of the particles, and thus separate
particles with different spins in space.}
\end{figure}

In the ideal case where the spatial wave packets corresponding to
the spin-up and spin-down states of the particle are completely separated,
once the state of the spin (spatial) degrees of freedom is determined,
the state of the spatial (spin) degrees of freedom can be immediately
inferred. Nevertheless, in realistic cases, when the position of the
screen for observing the particle distribution is fixed, there is
usually overlap between the spatial wave packets associated with different
spin states. Intuitively, increasing the movement time of the particle
in the magnetic field (or the gradient of the magnetic field) will
make the spatial wave packets farther apart along the gradient of
the magnetic field, which results in a better distinction between
different spin degrees of freedom. In what follows, to quantify the
correlation between spin and spatial degrees of freedom, we first
study the evolution of the particle's spatial wave packet.

\subsection{\label{subsec:Time-evolution}Time evolution of the spatial wave
packet}

The Hamiltonian of the particle reads
\begin{equation}
\hat{H}=\dfrac{\hat{p}^{2}}{2m}-\mu_{x}B(x)\hat{\sigma}_{x},\label{eq:H}
\end{equation}
where $\hat{\sigma}_{x}=\left(|\uparrow\rangle\left\langle \uparrow\right|-|\downarrow\rangle\left\langle \downarrow\right|\right)$
is Pauli matrix in $x$ direction with the corresponding spin up (down)
state $|\uparrow\rangle(|\downarrow\rangle)$, and $\mu_{x}$ is the
$x$ component of the magnetic moment. Suppose the initial state of
particle in spin space and real space are non-entangled, and are respectively
$|s\rangle=c_{+}|\uparrow\rangle+c_{-}|\downarrow\rangle$ and $|\psi(x,0)\rangle$,
the full initial state of the particle is thus

\begin{equation}
|\Psi(0)\rangle=(c_{+}|\uparrow\rangle+c_{-}|\downarrow\rangle)\otimes|\psi(x,0)\rangle.
\end{equation}
In the coordinate representation, one has

\begin{equation}
\langle x|\Psi(0)\rangle=(c_{+}|\uparrow\rangle+c_{-}|\downarrow\rangle)\langle x|\psi(x,0)\rangle.
\end{equation}
According to the Schrodinger's equation, the time evolution of total
state of the particle follows as

\begin{align}
\langle x|\Psi(t)\rangle & =\langle x|e^{-i\hat{H}t/\hbar}|\Psi(0)\rangle\nonumber \\
 & =c_{+}\psi_{+}(x,t)|\uparrow\rangle+c_{-}\psi_{-}(x,t)|\downarrow\rangle,\label{eq:spatial}
\end{align}
where $\psi_{+}(x,t)$ and $\psi_{-}(x,t)$ are the normalized spatial
wave function corresponding to spin state $|\uparrow\rangle$ and
$|\downarrow\rangle$, respectively. Without loss of generality, we
focus on the the symmetric case that the superposition coefficients
of the initial states are $c_{+}=c_{-}=1/\sqrt{2}$.

In addition, we assume that the inhomogeneous magnetic field is slowly
varying over the range of particle motion, and hence $B(x)$ can be
expanded to the first-order of $x$ as $B(x)\approx B(0)+[(\partial B/\partial x)|_{x=0}]x$
\citep{platt1992modern}. In this sense, the Hamiltonian in Eq. (\ref{eq:H})
is approximated as

\begin{equation}
\hat{H}=\dfrac{\hat{p}^{2}}{2m}-fx\hat{\sigma}_{x},\label{eq:H-1}
\end{equation}
where $f\equiv\mu_{x}(\partial B/\partial x)|_{x=0}$ is the magnetic
force applied on the particle in classical picture. The spatial wave
function can be expressed with the path integral approach as \citep{feynmanQuantumMechanicsPath2010}

\begin{equation}
\psi_{\pm}(x,t)=\int_{-\infty}^{\infty}K_{\pm}(x,t;x',t')\psi(x',t'){\rm d}x',\label{eq:pathintegral}
\end{equation}
where $\psi(x',t')$ is the wave function at space-time coordinate
$(x',t')$, and $K_{\pm}(x,x';t)$ is the propagator determined by
the Hamiltonian of the system. In our case, without loss of generality,
we set initial time $t'=0$, and choose the initial wave packet as
a Gaussian wave packet with width $a$ as

\begin{equation}
\psi(x',0)\equiv\langle x|\psi(x',0)\rangle=(\dfrac{1}{2\pi a^{2}})^{1/4}e^{-\frac{x'^{2}}{4a^{2}}}.\label{eq:waveinitial}
\end{equation}
which is plotted in Fig. \ref{fig:Evolution-of-the} as the purple
solid curve in the $z=0$ plane. Note that the Hamiltonian in Eq.
(\ref{eq:H-1}) can be reduced to the coordinate space as

\begin{equation}
\hat{H}_{+}^{s}\equiv\left\langle \uparrow\right|\hat{H}|\uparrow\rangle=\dfrac{\hat{p}^{2}}{2m}-f\hat{x},\;\hat{H}_{-}^{s}\equiv\left\langle \downarrow\right|\hat{H}|\downarrow\rangle=\dfrac{\hat{p}^{2}}{2m}+f\hat{x},
\end{equation}
the propagator $K_{\pm}(x,x';t)\equiv\left\langle x\right|\exp(-i\hbar^{-1}\int_{0}^{t}\hat{H}_{\pm}^{s}dt)|x'\rangle$
associated with this linear potential reads \citep{feynmanQuantumMechanicsPath2010,hsuSternGerlachDynamicsQuantum2011}

\begin{equation}
K_{\pm}(x,x';t)=\sqrt{\frac{m}{2\pi i\hbar t}}{\rm exp}\left(-\frac{m(x-x')^{2}}{2i\hbar t}\pm\frac{f(x+x')t}{2i\hbar}+\frac{f^{2}t^{3}}{24i\hbar m}\right).\label{eq:K}
\end{equation}

Substituting Eqs. (\ref{eq:K}) and (\ref{eq:waveinitial}) into Eq.
(\ref{eq:pathintegral}), the wave function is obtained as

\begin{align}
\psi_{\pm}(x,t) & =(\dfrac{1}{2\pi a^{2}})^{1/4}\sqrt{\frac{m}{2\pi i\hbar t}}\int_{-\infty}^{\infty}{\rm exp}\left(-\frac{m(x-x')^{2}}{2i\hbar t}\pm\frac{f(x+x')t}{2i\hbar}+\frac{f^{2}t^{3}}{24i\hbar m}\right)e^{-\frac{x'^{2}}{4a^{2}}}{\rm d}x',\nonumber \\
 & =(\dfrac{1}{2\pi a^{2}})^{1/4}\sqrt{\frac{m}{2\pi i\hbar t}}\int_{-\infty}^{\infty}{\rm exp}\left(-\frac{x'^{2}}{4a^{2}}-\frac{m(x-x')^{2}}{2i\hbar t}\pm\frac{f(x+x')t}{2i\hbar}+\frac{f^{2}t^{3}}{24i\hbar m}\right){\rm d}x'.
\end{align}
Using the one-dimensional Gaussian integral 
\begin{equation}
\int_{-\infty}^{\infty}e^{-bz^{2}+cz}{\rm d}z=\sqrt{\frac{\pi}{b}}{\rm exp}\left(\frac{c^{2}}{4b}\right),{\rm Re({\it b})>0},\label{eq:Gaussian}
\end{equation}
the spatial wave function is straightforward calculated as

\begin{equation}
\psi_{\pm}(x,t)=\dfrac{(a^{2}/2\pi)^{1/4}}{\sqrt{a^{2}+i\hbar t/2m}}{\rm exp}\left\{ -\dfrac{if^{2}t^{3}}{6\hbar m}-\dfrac{[x\mp ft^{2}/(2m)]^{2}}{4(a^{2}+i\hbar t/2m)}\pm\frac{iftx}{\hbar}\right\} ,\label{eq:psi_pm}
\end{equation}

\begin{flushleft}
The norms of the above wave functions at $t=0,3,6$ are illustrated
in Fig. \ref{fig:Evolution-of-the}, where (a) is plotted with $f=0.5$
and (b) is plotted with $f=1.5$, respectively. All other parameters
are set to $1$ in this example. The red dash-dotted curves and blue
dashed curves are associated with $\psi_{+}(x,t)$ and $\psi_{-}(x,t)$,
respectively. It is clearly seen in this figure that the degree of
separation of the wave packets corresponding to different spin states
increases with time. And for a given $t$, stronger magnetic field
gradient, i.e., $f\uparrow$, causes the wave packets to be separated
farther apart.
\par\end{flushleft}

\begin{center}
\begin{figure}
\begin{centering}
\subfloat[]{\includegraphics[width=8.5cm]{SM_Fig2a}

}\subfloat[]{\includegraphics[width=8.5cm]{SM_Fig2b}

}
\par\end{centering}
\caption{\label{fig:Evolution-of-the}Evolution of the spatial wave packet
with $f=0.5$(a) and $f=1.5$(b). The purple solid curve represents
the initial wave packet in Eq. (\ref{eq:waveinitial}). The red dash-dotted
(blue dashed) curves mark $|\psi_{+}(x,t)|$ ($|\psi_{-}(x,t)|$)
at $t=3$ and $t=6$. In this figure, we choose $a=m=\hbar=1$, the
velocity of the particle in the $z$ axis is set to $v_{z}=1$.}
\end{figure}
\par\end{center}

After the measurement process of duration $t_{{\rm M}}$, the spacial
overlapping integral $F\equiv|\langle\psi_{+}|\psi_{-}\rangle|$ is 

\begin{align}
F(t_{{\rm M}}) & =\int_{-\infty}^{\infty}\psi_{+}^{*}(x,t_{{\rm M}})\psi_{-}(x,t_{{\rm M}})dx\nonumber \\
 & =\frac{\left(a^{2}/2\pi\right)^{1/2}}{\sqrt{a^{4}+\hbar^{2}t_{M}^{2}/4m^{2}}}\int_{-\infty}^{\infty}{\rm exp}\left\{ -\dfrac{[x-ft_{M}^{2}/(2m)]^{2}}{4(a^{2}-i\hbar t_{{\rm M}}/2m)}-\dfrac{[x+ft_{M}^{2}/(2m)]^{2}}{4(a^{2}+i\hbar t_{{\rm M}}/2m)}-2\frac{ift_{{\rm M}}x}{\hbar}\right\} dx\nonumber \\
 & =\frac{\left(a^{2}/2\pi\right)^{1/2}}{\sqrt{a^{4}+\hbar^{2}t_{M}^{2}/4m^{2}}}\int_{-\infty}^{\infty}{\rm exp}\left\{ -\dfrac{a^{2}\left(x^{2}+f^{2}t_{M}^{4}/4m^{2}\right)-i\hbar fxt_{M}^{3}/2m^{2}}{2(a^{4}+\hbar^{2}t_{M}^{2}/4m^{2})}-2\frac{ift_{{\rm M}}x}{\hbar}\right\} dx,
\end{align}
which is straightforward calculated as, by using Eq. (\ref{eq:Gaussian}),

\begin{equation}
F(\tilde{t})={\rm exp}\left[-\left(2\alpha^{2}\tilde{t}^{2}+\frac{\alpha^{2}\tilde{t}^{4}}{8}\right)\right].\label{eq:Ft}
\end{equation}
Here, $\tilde{t}\equiv t_{{\rm M}}/\tau_{\mathrm{M}}$ is the dimensionless
measurement time, $\tau_{\mathrm{M}}\equiv ma^{2}/\hbar$ is the characteristic
time of the measurement process. $\text{\ensuremath{\alpha}}\equiv ma^{3}f/\hbar^{2}$,
which characterizes the strength of decoherence after measurement,
is proportional to the magnetic force $f$. For given $\tilde{t}$,
a stronger magnetic field (larger $f$) results in a smaller $F(t_{{\rm M}})$,
i.e. a more pronounced decoherence corresponds to farther apart wave
packets in Fig. \ref{fig:Evolution-of-the}. Obviously, $F(\tilde{t})\rightarrow0$
in the long-time limit $\tilde{t}\to\infty$ of the measurement process.
In this case, there is no overlapping between $\psi_{+}(x,t)$ and
$\psi_{-}(x,t)$, and the spin of the particle can be perfectly distinguished
by observing its spatial wave function in $x$ direction. Therefore,
an ideal measurement of the spin is achieved in the Stern-Gerlach
experiment.

\subsection{\label{subsec:Conditional-spatial-distribution-probability}Conditional
spatial distribution probability}

To discuss the non-ideal measurement within finite measuring duration,
we first study the conditional probability $p(L|\uparrow)$ that a
spin-up particle appears in the ``left'' area of the screen, and
the conditional probability $p(R|\downarrow)$ that a spin-down particle
appears in the right'' area of the screen. Here, left and right are
divided with respect to the $x$-direction, which corresponds to the
area with $x>0$ and $x<0$, respectively. By definitions, the probabilities
are

\begin{align}
p(L|\uparrow)\equiv & \int_{0}^{\infty}\langle x|\left[\langle\uparrow|\Psi(t)\rangle\langle\Psi(t)|\uparrow\rangle\right]|x\rangle dx\nonumber \\
= & \int_{0}^{\infty}\left|\psi_{+}(x,t_{{\rm M}})\right|^{2}\text{{\rm d}}x,\label{eq:p(R_up)}
\end{align}
and

\begin{align}
p(R|\downarrow)\equiv & \int_{-\infty}^{0}\langle x|\left[\langle\downarrow|\Psi(t)\rangle\langle\Psi(t)|\downarrow\rangle\right]|x\rangle dx\nonumber \\
= & \int_{-\infty}^{0}\left|\psi_{-}(x,t_{{\rm M}})\right|^{2}\text{{\rm d}}x.
\end{align}
Obviously, due to the symmetry of the spatial wave functions obtained
in Eq. (\ref{eq:psi_pm}), the above two probabilities equals to each
other, i.e., $p(L|\uparrow)=p(R|\downarrow)\equiv p(t_{{\rm M}})$.
Substituting Eq. (\ref{eq:psi_pm}) into Eq. (\ref{eq:p(R_up)}) yields 

\begin{align}
p\left(t_{{\rm M}}\right) & =\int_{0}^{\infty}\left|\psi_{+}(x,t_{{\rm M}})\right|^{2}\text{{\rm d}}x\nonumber \\
 & =\frac{\left(a^{2}/2\pi\right)^{1/2}}{\sqrt{a^{4}+\hbar^{2}t_{{\rm M}}^{2}/4m^{2}}}\int_{0}^{\infty}{\rm exp}\left[-\frac{a^{2}}{2\left(a^{4}+\hbar^{2}t_{{\rm M}}^{2}/4m^{2}\right)}\left(x-\frac{ft_{{\rm M}}^{2}}{2m}\right)^{2}\right]\text{{\rm d}}x\nonumber \\
 & =\frac{\left(a^{2}/2\pi\right)^{1/2}}{\sqrt{a^{4}+\hbar^{2}t_{{\rm M}}^{2}/4m^{2}}}\int_{-\frac{ft_{{\rm M}}^{2}}{2m}}^{\infty}{\rm exp}\left[-\frac{a^{2}}{2\left(a^{4}+\hbar^{2}t_{{\rm M}}^{2}/4m^{2}\right)}y^{2}\right]\text{{\rm d}}y\label{eq:Ptappendix}
\end{align}
Using the Gauss error function 
\begin{equation}
{\rm erf}\left(z\right)\equiv\frac{2}{\sqrt{\pi}}\int_{0}^{z}e^{-u^{2}}{\rm d}u,
\end{equation}
we simplify Eq. (\ref{eq:Ptappendix}) as

\begin{equation}
p\left(t_{{\rm M}}\right)=p(\tilde{t})=\frac{1}{2}\left[1+{\rm erf}\left(\frac{\alpha\tilde{t}^{2}}{\sqrt{2\tilde{t}^{2}+8}}\right)\right],\label{eq:P(T)}
\end{equation}
where ${\rm erf}(z)=2\pi^{-1/2}\int_{0}^{z}\mathrm{exp}(-u^{2})\text{{\rm d}}u$
is the Gauss error function. When the dimensionless measurement time
$\tilde{t}$ approaches infinity and zero we have $\lim_{\tilde{t}\rightarrow\infty}p(\tilde{t})=1$
and $\lim_{\tilde{t}\rightarrow0}p(\tilde{t})=1/2$ , respectively.
Correspondingly, the probability of a spin-up particle appears in
the ``left'' area, and the probability of a spin-down particle appears
in the \textquotedblleft right'' area are
\begin{equation}
p(R|\uparrow)=\int_{-\infty}^{0}\left|\psi_{+}(x,t_{{\rm M}})\right|^{2}\text{{\rm d}}x=1-p(\tilde{t}),\label{eq:Prup}
\end{equation}
and

\begin{equation}
p(L|\downarrow)=\int_{0}^{\infty}\left|\psi_{-}(x,t_{{\rm M}})\right|^{2}\text{{\rm d}}x=1-p(\tilde{t}).\label{eq:Pldown}
\end{equation}
Here the normalized condition for conditional probabilities $p(L|\theta)+p(R|\theta)=1,\theta\in\{\uparrow,\downarrow\}$
has been used.

In the long-time regime of $\tilde{t}\gg1$, we have $p(\tilde{t})\rightarrow1$,
which results in the conditional probabilities $p(L|\uparrow)=p(R|\downarrow)\rightarrow1.$
In this case, we can achieve an ideal measurement for the spin state
through the measurement of the space state: when a particle is observed
to be on the left (right) side of the screen, we can definitely determine
that its spin state is $|\uparrow\rangle$ ($|\downarrow\rangle$).
Conversely, we can also determine the spatial position of a particle
by measuring its spin state. However, for an arbitrary $\tilde{t}$
away from the long-time regime, even if the spin state of the particle
has been accurately obtained, we can only infer the spatial state
of the particle probabilistically. For example, when we find the spin
state of the particle is $|\uparrow\rangle$, we therefore infer that
the particle is on the left side of the screen. According to Eq. (\ref{eq:Prup}),
the probability that this inference is correct is $p_{c}=p(\tilde{t})$.
On the contrary, the probability of this inference being wrong is
$p_{w}=1-p(t)$, which corresponds to the case where the particle
is actually on the right side of the screen.

\subsection{\label{subsec:Correlation-between-spin-state-and-spatial-state}Measurement
ideality}

As we mentioned in the main text, the measurement of the position
of the particle by MD is realized through the detection of the spin
state. After the measurement, we assume that the instrument detects
the spin perfectly, then the MD state $|D_{\theta}\rangle$ becomes
the relative state of the spin state $|\theta\rangle$, and the conditional
probability of spin state given MD state $p(\theta|D)$ satisfies
\begin{equation}
p(\uparrow|D_{\uparrow})=p(\downarrow|D_{\downarrow})=1,p(\uparrow|D_{\downarrow})=p(\downarrow|D_{\uparrow})=0
\end{equation}

If the observation time is not long enough (in comparison with $\tau_{M}$),
the quantum measurement of the position of the particle is non-ideal.
We quantify the correlation information of the demon and the particle
with the mutual entropy \citep{kullback1997information,2002Quantum,li2017production,Still2020PRL.124.050601,ma2021works}

\begin{equation}
I=S_{P}+S_{D}-S_{DP}.
\end{equation}
Here, $S_{Y}=-\sum_{y\in Y}p(y){\rm ln}p(y)$ is the Shannon entropy
with respect to the variable $Y$, $S_{DP}=-\sum_{P}\sum_{D}p\left(P,D\right){\rm ln}p\left(P,D\right)$
is the joint entropy of the particle and the demon, where the joint
probability is $p\left(P,D\right)=p(P|D)p(D)$. Using such definition
and Eqs. (\ref{eq:Prup}) and (\ref{eq:Pldown}), the mutual information
is obtained as

\begin{align}
I & =S_{P}+S_{D}-S_{DP}\nonumber \\
 & =\sum_{P\in\{R,L\}}\sum_{D\in\{D_{\uparrow},D_{\downarrow}\}}p(P,D){\rm log}\frac{p(P|D)}{p(P)}\nonumber \\
 & =\sum_{P\in\{R,L\}}\sum_{D\in\{D_{\uparrow},D_{\downarrow}\}}\sum_{\theta\in\{\uparrow,\downarrow\}}p(P|D)p(D){\rm log}\frac{p(P|D)}{p(P)}\\
 & =\sum_{P\in\{R,L\}}\sum_{D\in\{D_{\uparrow},D_{\downarrow}\}}\sum_{\theta\in\{\uparrow,\downarrow\}}p(P|\theta)p(\theta|D)p(D){\rm log}\frac{p(P|\theta)p(\theta|D)}{p(P)}\\
 & =p(R|\uparrow)p(\uparrow){\rm log}\frac{p(R|\uparrow)}{p(R)}+p(R|\downarrow)p(\downarrow){\rm log}\frac{p(R|\downarrow)}{p(R)}+p(L|\uparrow)p(\uparrow){\rm log}\frac{p(L|\uparrow)}{p(L)}+p(L|\downarrow)p(\downarrow){\rm log}\frac{p(L|\downarrow)}{p(L)}\nonumber \\
 & ={\rm ln}2+p(\tilde{t}){\rm ln}p(\tilde{t})+\left[1-p(\tilde{t})\right]{\rm ln}\left[1-p(\tilde{t})\right],\label{eq:I_PD}
\end{align}
where $p(\uparrow)=p(\downarrow)=1/2$ has beenused according to the
symmetric choose of superposition coefficients in the initial states. 

According to the above equation and Eq. (\ref{eq:P(T)}), for the
example shown in Fig. \ref{fig:Evolution-of-the} (a) with $f=0.5$,
$I_{\tilde{t}=3}=0.355$ and $I_{\tilde{t}=6}=0.677$; while for Fig.
\ref{fig:Evolution-of-the} (b) with $f=1.5$, $I_{\tilde{t}=3}=0.692$
and $I_{\tilde{t}=6}=0.693$. Moreover, we define the measurement
ideality as $\text{\ensuremath{\mathcal{M}(\tilde{t})\equiv I(\tilde{t})/{\rm ln}2}}$,
namely,

\begin{equation}
\mathcal{M}(\tilde{t})=1+\frac{p(\tilde{t}){\rm ln}p(\tilde{t})+\left[1-p(\tilde{t})\right]{\rm ln}\left[1-p(\tilde{t})\right]}{{\rm ln}2}
\end{equation}
which is illustrated in Fig.2 of the main text.

The long-time and short-time behaviors of the mutual entropy are discussed
below. In the long-time regime of $\tilde{t}\gg1$, we have $p\left(\tilde{t}\right)\rightarrow1$,
keeping to the first order of $1-p\left(\tilde{t}\right)$, we have

\begin{align}
\mathcal{M}\left(\tilde{t}\gg1\right) & \approx1+\frac{{\rm ln}\left[1+p\left(\tilde{t}\right)-1\right]}{{\rm ln}2}\nonumber \\
 & \approx1+\frac{p\left(\tilde{t}\right)-1}{{\rm ln}2}\nonumber \\
 & =1+\frac{1}{2{\rm ln}2}\left[{\rm erf}\left(\frac{\alpha\tilde{t}^{2}}{\sqrt{8+\tilde{t}^{2}}}\right)-1\right]\nonumber \\
 & \approx1-\frac{1-{\rm erf}\left(\alpha\tilde{t}\right)}{2{\rm ln}2}.
\end{align}
On the other hand, in the short-time regime of $\tilde{t}\ll1$, we
have

\begin{align}
p\left(\tilde{t}\ll1\right) & =\frac{1}{2}\left[{\rm 1+erf}\left(\frac{\alpha\tilde{t}^{2}}{\sqrt{8+\tilde{t}^{2}}}\right)\right]=\frac{1}{2}+\frac{1}{\sqrt{\pi}}\int_{0}^{\frac{\alpha\tilde{t}^{2}}{\sqrt{8+\tilde{t}^{2}}}}e^{-u^{2}}{\rm d}u\nonumber \\
 & \approx\frac{1}{2}+\frac{1}{\sqrt{\pi}}\int_{0}^{\frac{\alpha\tilde{t}^{2}}{2\sqrt{2}}}(1-u^{2}){\rm d}u\nonumber \\
 & \approx\frac{1}{2}+\frac{\alpha\tilde{t}^{2}}{2\sqrt{2\pi}}.
\end{align}
Then the mutual entropy is approximated as

\begin{align}
\mathcal{M}\left(\tilde{t}\ll1\right) & =2+\left(\frac{1}{2{\rm ln}2}+\frac{\alpha\tilde{t}^{2}}{2{\rm ln}2\sqrt{2\pi}}\right){\rm ln}\frac{\frac{1}{2}+\frac{\alpha\tilde{t}^{2}}{2\sqrt{2\pi}}}{2}+\left(\frac{1}{2{\rm ln}2}-\frac{\alpha\tilde{t}^{2}}{2{\rm ln}2\sqrt{2\pi}}\right){\rm ln}\left[\frac{\frac{1}{2}-\frac{\alpha\tilde{t}^{2}}{2\sqrt{2\pi}}}{2}\right]\nonumber \\
 & =\left(\frac{1}{2{\rm ln}2}+\frac{\alpha\tilde{t}^{2}}{2{\rm ln}2\sqrt{2\pi}}\right){\rm ln}\left(1+\frac{\alpha\tilde{t}^{2}}{\sqrt{2\pi}}\right)+\left(\frac{1}{2{\rm ln}2}-\frac{\alpha\tilde{t}^{2}}{2{\rm ln}2\sqrt{2\pi}}\right){\rm ln}\left(1-\frac{\alpha\tilde{t}^{2}}{\sqrt{2\pi}}\right)\nonumber \\
 & \approx\left(\frac{1}{2{\rm ln}2}+\frac{\alpha\tilde{t}^{2}}{2{\rm ln}2\sqrt{2\pi}}\right)\left[\frac{\alpha\tilde{t}^{2}}{\sqrt{2\pi}}-\frac{1}{2}\left(\frac{\alpha\tilde{t}^{2}}{\sqrt{2\pi}}\right)^{2}\right]+\left(\frac{1}{2{\rm ln}2}-\frac{\alpha\tilde{t}^{2}}{2{\rm ln}2\sqrt{2\pi}}\right)\left[-\frac{\alpha\tilde{t}^{2}}{\sqrt{2\pi}}-\frac{1}{2}\left(\frac{\alpha\tilde{t}^{2}}{\sqrt{2\pi}}\right)^{2}\right]\nonumber \\
 & =\frac{\alpha^{2}\tilde{t}^{4}}{8\pi{\rm ln}2}=\frac{m^{2}a^{6}f^{2}\tilde{t}^{4}}{8\pi{\rm ln}2\hbar^{4}},
\end{align}
which is proportional to $\tilde{t}^{4}$. The proportional coefficient
increases with the square of the magnetic force and with the sixth
power of the initial width of the wave packet.

\section{\label{sec:Thermodynamic-signature-of-full-finite-time-description}Thermodynamic
signature of full finite-time description of quantum Szilard heat
engine}

In the main text, we mainly focus on the finite-time effect in the
measurement process (Stage I). In this section, we discuss the finite-time
behaviors of the other processes (Stages II and III) in the engine
cycle.

\subsection{\label{subsec:Information-recording-and-Information-erasing}Thermodynamic
signatures of information recording and information erasing process}

In this part, we further consider the finite-time effect in both the
information recording and information erasing processes. According
to the finite-time Landauer's principle \citep{berut2012experimental,proesmansFiniteTimeLandauerPrinciple2020,ma2022minimal},
the non-ideal erasing process leads to irreversible energy dissipation
which is inversely proportional to the erasing time $t_{{\rm E}}$,
namely,

\begin{equation}
W_{\mathrm{E}}(t_{\mathrm{E}})=k_{\mathrm{B}}T_{{\rm C}}{\rm ln}2\left(1+\frac{C_{{\rm E}}}{t_{\mathrm{E}}}\right),\label{eq:WE-1}
\end{equation}
where $C_{{\rm E}}$ is a constant determined by specific erasing
process \citep{proesmansFiniteTimeLandauerPrinciple2020,maMinimalEnergyCost2022}.
In this manner, we rewrite the total work of the QSE performs in one
cycle as
\begin{equation}
W=\mathcal{M}(t_{{\rm M}})k_{{\rm B}}T_{{\rm H}}{\rm ln}2-k_{\mathrm{B}}T_{{\rm C}}{\rm ln}2\left(1+\frac{C_{{\rm E}}}{t_{\mathrm{E}}}\right)\label{eq:W-1}
\end{equation}
Then, the power and efficiency of the information heat engine becomes

\begin{equation}
\eta(t_{{\rm M}},t_{{\rm E}})=\frac{W}{\mathcal{M}(t_{{\rm M}})k_{{\rm B}}T_{{\rm H}}{\rm ln}2}=1-\frac{1-\eta_{{\rm C}}}{\mathcal{M}(t_{{\rm M}}){\rm ln}2}\left({\rm ln}2+\frac{C_{{\rm E}}}{t_{\mathrm{E}}}\right),\label{eq:eta-1}
\end{equation}
and

\begin{equation}
P(t_{{\rm M}},t_{{\rm E}})=\frac{\mathcal{M}(t_{{\rm M}})k_{{\rm B}}T_{{\rm H}}{\rm ln}2-k_{\mathrm{B}}T_{{\rm C}}{\rm ln}2\left(1+C_{{\rm E}}t_{\mathrm{E}}^{-1}\right)}{t_{{\rm M}}+t_{{\rm E}}}\label{eq:P}
\end{equation}

Similar to the situation where only the finite-time effect of the
information recording process is taken into account, the EMP of the
engine in this situation can also exceed the upper limit of the efficiency
of low-dissipation Carnot heat engine $\eta_{+}=\eta_{\mathrm{C}}/(2-\eta_{\mathrm{C}})$
at certain range of $\eta_{{\rm C}}$. In order to quantify the influence
of parameters $\alpha$ and $C_{{\rm E}}$ on the EMP, in Fig. \ref{fig:EMP-1},
we show the width of the range of $\eta_{C}$ within which EMP can
exceed $\eta_{+}$ in the two-dimensional parameter space $\left(\alpha,C_{{\rm E}}\right)$.
It can be seen from the figure that wider range of $\eta_{{\rm C}}$
requires relative large value of $\alpha$ and small value of $C_{{\rm E}}$,
which consistent well with physical intuition: for a given operation
time, a larger $\alpha$ results in a more complete recording of information,
while a smaller $C_{{\rm E}}$ corresponds to a more reversible erasing
process with less energy dissipation.

\begin{figure}
\centering{}\includegraphics[width=8.5cm]{SM_Fig3} \caption{\label{fig:EMP-1} The width of the range of $\eta_{{\rm C}}$ that
the efficiency at maximum power(EMP) can exceed $\eta_{+}$ in two-dimensional
parameter space $\left(\alpha,C_{{\rm E}}\right)$. Lighter area corresponds
to wider range of $\eta_{{\rm C}}$ within which EMP can exceed the
typical bound of EMP $\eta_{+}=\eta_{\mathrm{C}}/(2-\eta_{\mathrm{C}})$
.}
\end{figure}

The power-efficiency trade-off relation is shown in Fig. \ref{fig:E-P tradeoff2}
by calculating the efficiency $\eta/\eta_{\mathrm{C}}$ and the power
$P/P_{\mathrm{max}}$ corresponding to $10^{7}$ of random measurement
time and erasing time pairs $\left(\tilde{t},t_{{\rm E}}\right)$
from$\left(1,1\right)$ to $\left(4,400\right)$. Each blue dot in
Fig. \ref{fig:E-P tradeoff2} represents a heat engine cycle with
different $\left(\tilde{t},t_{{\rm E}}\right)$, and the red solid
curve is the envelop of these blue dots which shows the boundary of
the trade-off relation. Different from the single-variable case shown
in Fig. 5(b) in the main text where the system can only operate with
power and efficiency locate on the curve, the heat engine with the
finite-time effects in both the information recording and information
erasing processes taken into consideration can operate at anywhere
within the envelop. The efficiency at an arbitrary given power $P_{a}$
is bounded by two intersection points of the vertical line $P=P_{a}$
and the red solid curve. As $P/P_{\mathrm{max}}\to0$, $\eta/\eta_{{\rm C}}$
approaches $1$, which covers the quasi-static regime of the information
heat engine.

\begin{figure}
\centering{}\includegraphics[width=8.5cm]{SM_Fig4}\caption{\label{fig:E-P tradeoff2} Power-efficiency trade-off relation of
the information heat engine, where the finite-time effects of both
the information recording and information erasing process are considered.
The Blue dots are obtained numerically from Eqs. (\ref{eq:eta-1})
and (\ref{eq:P}) with $10^{7}$ sets of measurement time and erasing
time pair $\left(\tilde{t},t_{{\rm E}}\right)$ chosen randomly from
$\tilde{t}\in\left(1,4\right)$ and $t_{{\rm E}}\in\left(1,400\right)$.
The Red solid curve marks the envelop of these blue dots, namely,
the boundary of the power-efficiency trade-off. In this plot, we choose
$\eta_{{\rm C}}=0.6$, $\alpha=0.4$ and $C_{{\rm E}}=0.5$.}
\end{figure}


\subsection{\label{subsec:Work-ouputting-and-Information-erasing}Thermodynamic
signature of work outputting and information erasing process}

In the last case to be studied, we assume $\tau_{\mathrm{M}}\ll t_{{\rm M}}\ll t_{{\rm O}}(t_{{\rm E}})$.
In this regime, the information recording is complete with $\mathcal{M}(t_{{\rm M}})\approx1$,
and operation time of the measurement process can be ignored in comparison
with that of the work outputting and information erasing process.
Below, we focus on the EMP of the information engine.

In the work outputting process, the information engine is contact
with a heat reservoir of temperature $T_{{\rm H}}$, and such a process
is the same as the finite-time isothermal process in the Carnot-like
cycle \citep{CA,schmiedl2008efficiency,EspositoPRL2010,Constraintrelationyhma}.
In the low-dissipation regime \citep{EspositoPRL2010,Constraintrelationyhma,yhmaoptimalcontrol,2020Optimal,ma2020experimental},
the output work follows as

\begin{equation}
W_{\mathrm{O}}\left(t_{\mathrm{O}}\right)=T_{{\rm H}}\Delta S\left(1-\frac{C_{\mathrm{O}}}{t_{\mathrm{O}}}\right),\label{eq:WO}
\end{equation}
where $\Delta S=k_{\mathrm{B}}\ln2$ is the reversible entropy change
\citep{dong2011quantum,kim2011quantum} and $C_{\mathrm{O}}$ is the
dissipation constant. It follows from Eqs. (\ref{eq:WO}) and (\ref{eq:W-1})
that the power of engine becomes a function of $t_{\mathrm{O}}$ and
$t_{\mathrm{E}}$ as

\begin{align}
P & \left(t_{\mathrm{O}},t_{\mathrm{E}}\right)=\frac{W_{\mathrm{O}}\left(t_{\mathrm{O}}\right)-W_{\mathrm{E}}\left(t_{\mathrm{E}}\right)}{t_{\mathrm{O}}+t_{\mathrm{E}}}\\
 & =T_{{\rm H}}\Delta S\frac{\eta_{\mathrm{C}}-C_{{\rm O}}t_{\mathrm{O}}^{-1}-\left(1-\eta_{\mathrm{C}}\right)C_{{\rm E}}t_{\mathrm{E}}^{-1}}{t_{\mathrm{O}}+t_{\mathrm{E}}}.
\end{align}
It is easy to check that the maximum power $P=P_{\mathrm{max}}$ is
achieved when $\partial P/\partial t_{\mathrm{O}}=0$ and $\partial P/\partial t_{\mathrm{E}}=0$,
and the optimal operation times at maximum power are obtained as

\begin{equation}
t_{\mathrm{O}}^{*}=\frac{2C_{\mathrm{O}}}{\eta_{\mathrm{C}}}\left(1+\sqrt{\frac{C_{\mathrm{E}}}{C_{\mathrm{O}}}}\right),\;t_{\mathrm{E}}^{*}=t_{\mathrm{O}}^{*}\sqrt{\frac{C_{\mathrm{E}}}{C_{\mathrm{O}}}}
\end{equation}
Correspondingly, the efficiency at maximum power (EMP) for this information
engine is

\begin{equation}
\eta_{MP}=1-\frac{W_{\mathrm{E}}\left(t_{\mathrm{E}}^{*}\right)}{W_{\mathrm{O}}\left(t_{\mathrm{O}}^{*}\right)}=\frac{\eta_{\mathrm{C}}}{2-\kappa\eta_{\mathrm{C}}}.\label{eq:EMP}
\end{equation}
Here $\kappa=\left[1+\sqrt{C_{\mathrm{E}}/\left(C_{\mathrm{O}}\right)}\right]^{-1}$
depends on the asymmetry of dissipation when the heat engine is in
contact with two heat reservoirs. Since $0\leq\kappa\leq1$ for different
$\Sigma_{\mathrm{c}}/\Sigma_{\mathrm{h}}$, $\eta_{MP}$ is found
to satisfy the following inequality

\begin{equation}
\eta_{\mathrm{L}}\equiv\frac{\eta_{\mathrm{C}}}{2}\leq\eta_{MP}\leq\frac{\eta_{\mathrm{C}}}{2-\eta_{\mathrm{C}}}\equiv\eta_{\mathrm{U}}.\label{eq:inequality}
\end{equation}
In the above relation, $\eta_{\mathrm{U}}$ and $\eta_{\mathrm{L}}$
are respectively the upper and lower bound of EMP. 

Equation (\ref{eq:inequality}) covers the result obtained with the
low-dissipation model for finite-time Carnot engine \citep{EspositoPRL2010}.
This indicates that when the information recording is ideal and the
corresponding duration is ignored, the finite-time Szilard engine
is mapped to a low-dissipation Carnot heat engine. When the dissipation
coefficients in the work outputting process and information erasing
process are the same as each other, i.e., $C_{\mathrm{E}}/C_{\mathrm{O}}=1$,
the EMP of Eq. (\ref{eq:EMP}) recovers the so-called Curzon-Ahlborn
efficiency, i.e., $\eta_{\mathrm{CA}}=1-\sqrt{T_{\mathrm{C}}/T_{\mathrm{H}}}$
\citep{CA}.

\bibliographystyle{apsrev}
\bibliography{FTOSHE}

\end{document}
