\documentclass[11pt,oneside,english]{amsart}
\RequirePackage{amsthm,amsmath,mathtools}

\usepackage[T1]{fontenc}
\usepackage{geometry,comment}
%\geometry{verbose,tmargin=3cm,bmargin=3cm,lmargin=3.2cm,rmargin=3.2cm}
\geometry{verbose,tmargin=2.5cm,bmargin=2.5cm,lmargin=2.5cm,rmargin=2.5cm}

\usepackage{subfig}
\usepackage{setspace}
\usepackage{babel,verbatim}
\usepackage{float}
\usepackage{amstext}
\usepackage{amssymb,mathrsfs}
\usepackage{graphicx}
\usepackage[numbers]{natbib}
\RequirePackage[colorlinks=true,linkcolor=black,
	citecolor=blue,urlcolor=blue]{hyperref}

\providecommand{\tabularnewline}{\\}

%%%%%%%%%%%%%%%%%%%%%%%%%%%%%% Textclass specific LaTeX commands.
\numberwithin{equation}{section}
\numberwithin{figure}{section}
  \theoremstyle{plain}
  \newtheorem{lyxalgorithm}{\protect\algorithmname}
  \theoremstyle{plain}
  \newtheorem*{lyxalgorithm*}{\protect\algorithmname}
\theoremstyle{plain}
\newtheorem{thm}{\protect\theoremname}
  \theoremstyle{remark}
  \newtheorem*{rem*}{\protect\remarkname}
  \theoremstyle{remark}
  \newtheorem{rem}[]{\protect\remarkname}
  \theoremstyle{plain}
  \newtheorem*{assumption*}{\protect\assumptionname}
  \theoremstyle{plain}
  \newtheorem{lem}[thm]{\protect\lemmaname}
  \theoremstyle{plain}
  \newtheorem{cor}[thm]{\protect\corollaryname}
  \theoremstyle{plain}
  \newtheorem{prop}[thm]{\protect\propositionname}
  \theoremstyle{definition}
  \newtheorem*{example*}{\protect\examplename}
  \newtheorem{ex}[]{\protect\examplename}
  \theoremstyle{definition}
  \newtheorem*{algorithm}{\protect\algorithmname}

\usepackage{tikz}
\usetikzlibrary{shapes.multipart,shapes,arrows,decorations.pathreplacing}

\usepackage{algorithm,algpseudocode,multirow,bbm,enumitem}
\usepackage{array}
\newcolumntype{L}{>{\centering\arraybackslash}m{1.55cm}}
\newcolumntype{C}{>{\centering\arraybackslash}m{4.35cm}}
\newcolumntype{T}{>{\centering\arraybackslash}m{4.85cm}}

\newcommand{\Oh}{\mathcal{O}}
\newcommand{\C}{\mathcal{C}}
\newcommand{\F}{\mathcal{F}}
\newcommand{\U}{\mathrm{U}}
\newcommand{\Exp}{\mathrm{Exp}}
\newcommand{\e}{\mathbb{E}}
\newcommand{\V}{\mathbb{V}}
\newcommand{\p}{\mathbb{P}}
\newcommand{\Z}{\mathbb{Z}}
\newcommand{\mS}{\mathcal{S}}
\newcommand{\R}{\mathbb{R}}
\newcommand{\N}{\mathbb{N}}
\newcommand{\ov}[1]{\overline{#1}}
\newcommand{\1}{\mathbbm{1}}
\newcommand{\D}{\mathrm{d}}

\newcommand{\SB}{{\mathrm{SB}}}
\newcommand{\MC}{{\mathrm{MC}}}
\newcommand{\ML}{{\mathrm{ML}}}
\newcommand{\sgn}{{\mathrm{sgn}}}
\newcommand{\eqd}{\overset{d}{=}}
\newcommand{\cov}{\mathrm{cov}}
\newcommand{\blambda}{{\boldsymbol\lambda}}
\newcommand{\bzero}{{\boldsymbol0}}
\newcommand{\ISE}{\mathrm{IS}}
\newcommand{\STE}{\mathrm{ST}}
\newcommand{\mL}{\mathcal{L}}

 \usepackage{relsize,scalefnt}
\newcommand{\pp}{{\mathsmaller{(+)}}}
\newcommand{\pn}{{\mathsmaller{(-)}}}
\newcommand{\ppm}{{\mathsmaller{(\pm)}}}

\DeclareRobustCommand{\stirling}{\genfrac\{\}{0pt}{}}

\usepackage{lineno}
\usepackage{pgfplots} % Graphs
\usepgfplotslibrary{fillbetween}

\newcommand{\BrownianMotion}[7]{% initial points (2), points, advance, rand factor, options, end label
\draw[#6] (#1,#2)
\foreach \x in {1,...,#3}
{   -- ++(#4,rand*#5)
}
node[below right] {#7};
}

\linespread{1.5}
\usepackage{caption}
\usepackage{nameref}
\DeclareCaptionLabelFormat{algnonumber}{}
\captionsetup[algorithm]{labelformat=algnonumber,font=bf}
\makeatother
  \providecommand{\algorithmname}{Algorithm}
  \providecommand{\assumptionname}{Assumption}
  \providecommand{\examplename}{Example}
  \providecommand{\lemmaname}{Lemma}
  \providecommand{\propositionname}{Proposition}
  \providecommand{\remarkname}{Remark}
\providecommand{\corollaryname}{Corollary}
\providecommand{\theoremname}{Theorem}

\begin{document}

\title{Fast exact simulation of the first passage of a
tempered stable subordinator across a non-increasing
function\\
--Reply to referees--}

\author{Jorge I. Gonz\'{a}lez C\'{a}zares \& Feng Lin \& Aleksandar Mijatovi\'{c}}

\maketitle

We want to thank the referees and AE for their thoughtful and thorough reading of our paper which helped improve the presentation of the paper and increase the clarity in our algorithms. Below we detail how we addressed each of their comments.

\subsection*{Referee 1}
{\bf Major comments.}
\begin{itemize}
    \item[1.] We thank the referee for this question. As it turns out, this method is also viable if implemented appropriately (i.e. using Householder's algorithm, which reduces the number of times the numerical integral needs to be evaluated). However, we have opted for our algorithm in its current form as it is, on average, 32 times faster and is less sensitive to errors in the numerical integrals, see details in new Subsection~2.6. We believe that the difference in speed is mainly due to the fact that our algorithm performs numerical integration fewer times. A similar phenomenon arises when considering the inversion-rejection method, as it consists of splitting the domain and finding (often constant) bounding densities to draw samples from for each subdomain. Our partition of the domain is, instead, done using analytic properties of the specific density at hand.
    \item[2.] We reduced the discussion of `exact simulation' in the main body of the paper, leaving an in-depth discussion in Appendix C. This discussion includes a comparison to the discussion in \S7.2 of the book by H\"ormann, W., Leydold, J., and Deringer, G. (2000), on the same topic (we thank the referee for pointing us to this reference).
\end{itemize}

{\bf Minor comments.}
\begin{itemize}
    \item[1.] The comment has been implemented.
    \item[2.] The comment has been implemented.
    \item[3.] We now clarify that the binary search ensures that this condition holds for the initial estimate $x_0$ used in the Newton--Raphson method.
\end{itemize}

\subsection*{Referee 2}

\begin{itemize}
    \item {\bf  Organisation of the paper.} We agree with the referee's assessment that the number of modules were overwhelming and the algorithms' justifications were too far apart from the algorithms themselves. We have now included new Figure 2.1 to illustrate the dependence between the algorithms in the paper. We have also included a short high-level arguments for the validity of each algorithm, placed next to them, to explain why and how they work.
    \item {\bf Complexity Analysis of the numerical inversion, root-finding, and quadrature.} We have elaborated on why we assume the cost of the numerical quadrature to be constant (dependent \emph{only} on the number of precision digits $N$), see Remark 1.
    \item {\bf Proper Introduction of Notations and Definitions before Using Them.} We apologise for the inconsistencies in our notation and for terms used before being introduced. We have carefully read our work to make sure this is no longer an issue.
\end{itemize}

{\bf Minor comments}
\begin{itemize}
    \item[1.] Equation numbering now explicitly refers the section in which the equation lies, so that such confusions do not arise. 
    \item[2.] We only use the subscript notation $S_t$.
    \item[3.] All typos have been fixed.
\end{itemize}

{\bf Associate Editor.}
\begin{itemize}
    \item[1.] We have included definitions of all such technical terms. We have also fixed the typo in the second sentence.
    \item[2.] This errors have been fixed.
    \item[3.] We now explain what the role of $R$ is, how it is selected, and the fact that its value does not change during the algorithm's run.
    \item[4.] The function $B$ is now defined inside the algorithm.
    \item[5.] We now give a reference to Lemma 1, where this property is proved.
\end{itemize}

{\bf Further changes.} We have decided to combine the old Algorithms 3 and 4 into the new FPS-Alg (and thus Theorem 2 and Corollary 2 into the new Theorem 2). Further

\bibliographystyle{abbrv}
\bibliography{ref_first_passage}
\end{document}