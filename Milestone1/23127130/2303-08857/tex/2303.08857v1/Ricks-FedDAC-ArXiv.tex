% ****** Start of file apssamp.tex ******
%
% This file is part of the APS files in the REVTeX 4.2 distribution.
% Version 4.2a of REVTeX, December 2014
%
% Copyright (c) 2014 The American Physical Society.
%
% See the REVTeX 4 README file for restrictions and more information.
%
% TeX'ing this file requires that you have AMS-LaTeX 2.0 installed
% as well as the rest of the prerequisites for REVTeX 4.2
%

% See the REVTeX 4 README file
% It also requires Running BibTeX. The commands are as follows:
%
% 1) latex apssamp.tex
% 2) bibtex apssamp
% 3) latex apssamp.tex
% 4) latex apssamp.tex
%
\documentclass[%
 reprint,
superscriptaddress,
%groupedaddress,
%unsortedaddress,
%Runinaddress,
%frontmatterverbose, 
%preprint,
%preprintnumbers,
%nofootinbib,
%nobibnotes,
%bibnotes,
 amsmath,amssymb,
 aps,
%pra,
%prb,
%rmp,
%prstab,
%prstper,
%floatfix,
]{revtex4-2}

\usepackage{graphicx}% Include figure files
\usepackage{dcolumn}% Align table columns on decimal point
\usepackage{bm}% bold math
\usepackage{xcolor}
\usepackage{soul}
\usepackage{gensymb}
\newcommand{\RLS}[1] {{\color{red} #1}}
\usepackage[symbol]{footmisc}

%\usepackage{hyperref}% add hypertext capabilities
%\usepackage[mathlines]{lineno}% Enable numbering of text and display math
%\linenumbers\relax % Commence numbering lines

%\usepackage[showframe,%Uncomment any one of the following lines to test 
%%scale=0.7, marginratio={1:1, 2:3}, ignoreall,% default settings
%%text={7in,10in},centering,
%%margin=1.5in,
%%total={6.5in,8.75in}, top=1.2in, left=0.9in, includefoot,
%%height=10in,a5paper,hmargin={3cm,0.8in},
%]{geometry}




\begin{document}

\preprint{APS/123-QED}

\title{\textit{In situ} study of iron phase transitions at high pressure and temperature over millisecond timescales via time-resolved X-ray diffraction}

\author{Matthew Ricks}
\affiliation{Department of Physics and Astronomy, Brigham Young University, Provo, Utah 84602, USA}

\author{Arianna E. Gleason}
\affiliation{SLAC National Accelerator Laboratory, 2575 Sand Hill Rd., Menlo Park, CA, 94025, USA}

\author{Francesca Miozzi}
\affiliation{Earth and Planets Laboratory, Carnegie Institution for Science, Washington, DC 20015}

\author{Hong Yang}
\affiliation{Department of Earth and Planetary Sciences, Stanford University, 450 Jane Stanford Way,  Building 320, Stanford, CA 94305}

\author{Stella Chariton}
\affiliation{Center for Advanced Radiation Sources, The University of Chicago, Chicago, IL 60637}

\author{Vitali B. Prakapenka}
\affiliation{Center for Advanced Radiation Sources, The University of Chicago, Chicago, IL 60637}

\author{Richard L. Sandberg}
\affiliation{Department of Physics and Astronomy, Brigham Young University, Provo, Utah 84602, USA}

\author{Wendy L. Mao}
\affiliation{Department of Earth and Planetary Sciences, Stanford University, 450 Jane Stanford Way,  Building 320, Stanford, CA 94305}

\author{Silvia Pandolfi}
\affiliation{SLAC National Accelerator Laboratory, 2575 Sand Hill Rd., Menlo Park, CA, 94025, USA}
\affiliation{Now at: Sorbonne Université, Muséum National d’Histoire Naturelle, UMR CNRS 7590, Institut de Minéralogie, de Physique des Matériaux et de Cosmochimie, IMPMC, 75005 Paris, France}

\date{\today}

\begin{abstract}
We investigate the phase transitions of iron at high pressure and high temperature conditions using a fast-loading dynamic-diamond anvil cell (dDAC) setup. Using the dDAC apparatus coupled with \textit{in situ} X-ray-diffraction at the 13-IDD beamline at Advanced Photon Source in Argonne National Laboratory, we demonstrate compression rates of hundreds of GPa/s and monitor the structural evolution with millisecond time resolution. This technique allows us to cover an intermediate compression rate between conventional static- and dynamic-compression experiments, providing new insight on the kinetic effects influencing iron phase transitions. Crucially, the dDAC setup is compatible with doubled sided laser heating, enabling a detailed investigation of the pressure-temperature phase diagram under dynamic compression, as opposed to shock-compression techniques, which are constrained along the Hugoniot curve. We provide thus the first insight on the $\gamma-\epsilon$ phase transition (\textit{i.e.}, \textit{fcc} to \textit{hcp}) of iron under dynamic loading and compare the results with the trends observed for the $\alpha-\epsilon$ (\textit{i.e.}, \textit{bcc} to \textit{hcp}) phase transition. Our results demonstrate that the specific deformation mechanism strongly influences the response under dynamic loading.
\end{abstract}


\maketitle

%\tableofcontents

\section{\label{sec:intro}Introduction}
Iron (Fe) is the main constituent of the Earth's core, and its behavior at extreme conditions has been extensively studied, both experimentally and theoretically. Fe is also one of the most commonly used materials in the industrial sector. Its versatility, strength, and durability make it an essential component for various applications.  The stable structure of Fe at ambient conditions, the so-called $\alpha$ phase, with body-centered cubic (\textit{bcc}) structure, transforms into an hexagonal-close packed (\textit{hcp}) structure, the $\epsilon$ phase, under high-pressure (\textit{HP}) \cite{Shen1998}, while at high-temperature (\textit{HT}), a face-centered cubic (\textit{fcc}) structure is stabilized, \textit{i.e.}, the $\gamma$ phase. Upon compression at high-pressure and high-temperature (\textit{HP}-\textit{HT}), the $\gamma$ phase also transforms into the $\epsilon$ phase. This $\epsilon$ phase remains stable up to multi-Mbar pressures and is believed to be the phase present in the Earth's solid core \cite{anderson1986}.

Numerous static experiments have been conducted to explore Fe \textit{HP-HT} equilibrium phase diagram \cite{Boehler1993,Shen1998,Komabayashi2009,Kubo2003,Ma2004,Funamori1996, Miozzi2020, Bancroft1956}. In these experiments, efforts were made to ensure hydrostatic conditions in order to mimic the environment of the Earth's interior as closely as possible, and to avoid uniaxial strain and temperature gradients that make it difficult to accurately estimate pressure (\textit{P}) and temperature (\textit{T}). Although understanding Fe behaviour under the hydrostatic conditions has important implications for our understanding of the Earth's core, not all geological environments on Earth or extraterrestrial planets are static: shearing in subduction zones and planetary impacts are examples of dynamic geophysical processes \cite{Kon2015}. Dynamic compression techniques, such as gas gun and laser ablation compression, have been extensively used to characterize Fe deformation and melting at ultrafast timescales, from $\mu$s down to ps \cite{Yoo1993,Benuzzi2002,Ping2013,Amadou2015,Harmand2015,Denoeud2016,Torchio2016,Merkel2021,Branch2018, Boettger1997, Hawreliak2008}. These approaches can attain pressures up to several TPa \cite{Smith2014} (\textit{i.e.,} tens of millions atmospheres); however, single-shock experiments are constrained to probe states along the Hugoniot curve in the \textit{P}-\textit{T} space \cite{Duffy2019}. For Fe, the Hugoniot curve crosses the  $\alpha-\epsilon$ phase boundary, and the transition has been observed at ca. 13 GPa, a slightly higher pressure compared to the static value \cite{Hawreliak2006,Kalantar2005,Hawreliak2021}. Despite recent developments that allow us to deploy more complex, off-Hugoniot compression profiles \cite{Wang2013,Smith2018,Kraus2022}, the $\gamma-\epsilon$ transition remains inaccessible using conventional dynamic compression techniques. % maybe add some references about current planetary body impact relevance, even more foundationally also general material understanding (more holistic equation of state)

Between the timescales characteristics of static and shock compression experiments lies a vast region of unexplored compression rates. Recent development in dynamic diamond-anvil cell (dDAC) technology has started to fill in this gap. Using either gas supplied membrane or electromechanical piezoelectric actuators 
% check timescales on papers with piezo for compression (maybe jessie or zolt)
to control pressure, dDACs can access compression rates on the second to millisecond timescales \cite{sinogeikin2015} For example Kon\^opkov\'a \textit{et al.} investigated the  $\alpha-\epsilon$ transition at compression rates up to 4.1 GPa/s at ambient temperature using a dDAC equipped with a membrane enclosure \cite{Kon2015}. They find this phase boundary to be close to the values found in static compression experiments if the transition happens under quasi-hydrostatic conditions, when a pressure transmitting medium is used in the sample chamber. % don't add mention of the double sided laser heating, highlight the contrast of the between us and konopkova, "in compliment to konopkova" Double-sided laser heating compatible dDACs enable simultaneous heating and compression to dynamically explore extreme \textit{P-T} states. 

Here, we provide the first insight on the  $\gamma-\epsilon$ phase transition of Fe under dynamic compression, using a dDAC setup to attain millisecond (ms) compression timescales at HT. The phase transition onset at these compression rates was measured for two different temperatures, and the values agree with previous static compression experiments from Shen \textit{et al.} \cite{Shen1998}, while it also shows a phase transition lowering with respect to results from more recent studies \cite{Komabayashi2009, Funamori1996, Kubo2003}. We also investigated the $\alpha-\epsilon$ phase transition at ambient temperature; in this case, a noticeable increase of the transition offset is observed at the increased compression rate. Our results demonstrate that the extent and nature of kinetic effects on phase transition boundaries is strongly influenced by the specifics of the transition mechanism, which should be taken into account when using static- or dynamic- compression data to inform geological models.

\section{\label{sec:methods}Methods}

\begin{figure}
\includegraphics[width=\columnwidth]{setup.pdf}
\caption{\label{fig:setup} Schematic view of the experimental setup used on the 13-IDD beamline at the APS synchrotron. Loading in the DAC was performed using a membrane and an enclosure compatible with the mini-BX80 cells; the intermediate buffer allowed to perform fast (ms) compression runs. The structure of the sample was monitored in-real-time using XRD, and the X-ray beam was spatially overlapped with the laser-heating spot.}
\end{figure}


Samples consisted of reagent grade Fe powder with micrometer-sized grains, commercially  available from Alfa Aesar. The samples were loaded in pre-indented stainless steel gaskets (80 $\mu$m-thick SS 304, initial thickness 250 $\mu$m). The diamonds had flat culets of 250  and 300 $\mu$m. The sample chamber (100 $\mu$m diameter) was drilled using an electrical discharge drilling system. Cold-pressed flakes of KCl were placed on either side of the Fe grains, 
providing thermal insulation form the diamond culets as well as acting as  a pressure transmitting medium and a pressure calibrant with a well known equation of state \cite{tateno2019}.

The sample assemblages were loaded in a mini-BX80 DAC developed by DAC Tools, which is a modified version of the mini-BX90 apparatus \cite{Kantor2012},and it was equipped with tungsten carbide seats. The mini-BX80 allows for pressure control either by screws or remotely, \textit{e.g.}, using a membrane enclosure and an online remote control system\cite{sinogeikin2015}. To enable 
dynamic loading of the DAC using the membrane gas supply, our experimental setup included an intermediate buffer. The buffer allows us to pre-load the gas supply to a desired pressure, and it is located near the DAC. An electric solenoid valve allows release of the gas in the membrane on a short ($\sim$ms) timescale for high compression rate experiments (FIG.\ref{fig:setup}). Analogous setups for 
fast compression and \textit{in situ} characterization using synchrotron radiation have already been demonstrated at other facilities \cite{Velisavljevic_2014,haberl2014}.

\begin{figure}
\includegraphics[width=0.9\columnwidth]{comp_rate.pdf}
\caption{\label{fig:comp_rate} Temporal evolution of pressure during dynamic loading for the four experimental runs; pressure was measured using the known EOS of KCl. For each run, temperature values as well as compression rates are reported; in Run 4, no intermediate buffer was used, resulting in a $\sim$100 times lower compression rate.}
\end{figure}

Experiments were conducted at the 13-IDD beamline of the GSECARS sector of the 
Advanced Photon Source \cite{Shen2005}. The P-T conditions probed during our experiments, as well as the compression rates attained during dynamic loading, are shown in FIG.\ref{fig:comp_rate} In Run 1, compression was performed at 2000K, with an average compression rate of 400 GPa/s: Run 2 reached 530 GPa/s at 1400K. Run 3  and Run 4 were performed at 300 K, and reached 360 GPa/s and 2.5 GPa/s compression rates, respectively; it should be noted that in Run 4 the compression was performed without using the intermediate buffer.

\begin{figure*}
\includegraphics[width=0.7\textwidth]{waterfall_plots.pdf}
\caption{\label{fig:waterfall_plots} 
Azimuthally-integrated XRD patterns as a function of time show the structural changes in the sample upon compression. The XRD data is shown in the same colors used in FIG.\ref{fig:comp_rate} to represent pressure evolution. (a) and (b): HT experiments, showing the $\gamma$-$\epsilon$ transition at 2000 K and 1400 K, respectively. (c) and (d): experiments at 300 K, showing the $\alpha$-$\epsilon$ transition for compression at 360 GPa/s and 2.5 GPa/s, respectively. For all patterns, the peaks of the observed Fe phases, as well as those of KCl, are indexed; the time values are measured with respect to the beginning of the dDAC compression run.}
\end{figure*}

X-ray diffraction data (XRD) were collected \textit{in situ} at HP-HT using a monochromatic X-ray beam with energy of either 37 keV or 42 keV and a Pilatus3 X 1M CdTe detector. LaB$_6$ was used as reference to calibrate the detector distance and geometry. 1D integration from 2D detector images was done using the Fit2D and Dioptas software packages \cite{Hammersley1996,Clemens2015}. Peak identification and fitting was conducted in the PDindexer software package \cite{SETO2010}. During Run 1-3, the acquisition time was 2 ms/pattern; this ensured good temporal resolution for the characterization of the high-pressure phase transitions and an accurate determination of the transition onset. During Run 4, given the lower compression rate, the acquisition time was increased to 20 ms/pattern. Changes in pressure were monitored using the known EOS of the B2 phase of KCl \cite{tateno2019} and fitting of the KCl (110) reflection; this choice was dictated by the visibility of the KCl (110) reflection throughout the whole experiment and the lack of superposition with any Fe peaks.

Thanks to the short (less than 20 mm) working distance on both sides, the mini-BX80 cell equipped with membrane enclosure is compatible with the double-sided laser heating setup of the 13-IDD beamline \cite{prakapenka2008}. Temperature was measured on both sides of the DAC with 300 ms temporal resolution by fitting a Planck equation to the thermal radiation spectrum \cite{prakapenka2008}. The size of X-ray beam on the sample was 2$\mu$m (V) x 3.5$\mu$m (H) and the size of the flat-top laser-heating spot was 12 $\mu$m in diameter; the beams were coaxially aligned to spatially overlap. 
% Richard had some questions here, but I think they are answered in prakapenka2008. Do you think we should any any addition explanation?

In all runs, the sample was initially compressed to
7 GPa, well past the phase boundary in KCl from the B1 to the B2 phase. This 
ensured more accurate pressure measurements, as the KCl does not undergo any 
structural transition over the explored pressure range. In Run 1 and Run 2, the sample was first heated, and then pressure was increased using the buffer at HT. With this experimental approach, we demonstrate a peak compression rate of over 500 GPa/s over 30 ms.

\section{\label{sec:results}Results}

We performed time-resolved XRD to characterize both the $\alpha$-$\epsilon$ and the $\gamma$-$\epsilon$ phase transitions of Fe. Experiments were performed at HT and 300 K, respectively; the \textit{in situ} data is reported in FIG.\ref{fig:waterfall_plots}. For each experiment, the sample was compressed statically up to a pressure value close to the phase boundary before launching the dynamic compression run, as to ensure completion of the transformation during loading. In Run 1 (FIG.\ref{fig:waterfall_plots}(a)) measurements were performed keeping temperature at 2000 K, and the sample was pre-compressed up to 37 GPa; at these conditions only the $\gamma$ phase is present. During loading, we observe the appearance of the $\epsilon$-(010) reflection ($\sim$9.2$\degree$) after 12 ms, at 40 GPa. Coexistence of the $\gamma$ and $\epsilon$ phases is observed up to 46 GPa, and progression of the transformation is confirmed by the changes in relative intensity of the correspondent peaks. At 46 GPa, the peak observed at $\sim$9.7$\degree$ can be indexed as either the $\gamma$-(111) or the $\epsilon$-(002) reflections. The crystalline texture, which can be inferred from the 2D-XRD pattern by examining the intensity distribution along the Debye-Scherrer ring (see also FIG.\ref{fig:long_aq} and correspondent discussion), is more consistent with the $\gamma$ phase. Moreover, as observed in previous experiments, the $\epsilon$ phase is expected to grow along a preferred orientation in a DAC, which results in a decrease of the $\epsilon$-(002) reflection's intensity \cite{Singh2006}; thus, we do not expect to observe signal from the $\epsilon$-(002) peak over the 2 ms integration time. In Run 2 (FIG.\ref{fig:waterfall_plots}(b)), the sample was pre-compressed up to 18 GPa and temperature was maintained at 1400 K during dynamic loading. At $t = 0$ ms, all the Fe peaks can be indexed as reflections from the $\gamma$ phase. The $\epsilon$-(010) peak becomes visible after 10 ms, at 23 GPa. The transition takes place over a few GPa, and at 28 GPa all of the $\gamma$ phase reflections, with the exception of the $\gamma$-(111) (plus $\epsilon$-(002)) have disappeared. As mentioned for Run 1, also in this case the peak at $\sim$9.5$\degree$ is more likely to indicate the persistence of small amounts of the $\gamma$ phase rather than corresponding to the $\epsilon$-(002) reflection. In Run 3 (FIG.\ref{fig:waterfall_plots}(c)) the sample was initially compressed to about 11 GPa, a pressure at which only the $\alpha$ phase is present, and dynamic loading was performed at 300 K. At 13.6 GPa, multiple reflections from the $\epsilon$ phase appear, namely, $\epsilon$-(010), $\epsilon$(011) and $\epsilon$-(012). It is worth noting that the value of the transition onset for this run (with compression rate of 360 GPa/s) is higher than the pressures reported in previous static compression experiments \cite{Shen1998} At 14.6 GPa, the $\alpha$-(110) peak is most likely still present, and superimposed with the $\epsilon$-(002). In Run 4 (FIG.\ref{fig:waterfall_plots}(d)), compression was performed at 300 K without the use of the intermediate buffer, resulting in a compression rate reduction by a factor $\sim$100 (2.5 GPa/s as compared to 360 GPa/s in Run 3). The sample was compressed statically up to 7 GPa, and at this compression rate the emergence of the $\epsilon$-(010) and $\epsilon$-(011) reflections was observed at pressures as low as 10.8 GPa., a value that is in agreement with the $\alpha$-$\epsilon$ phase boundary identified in static compression experiments \cite{Shen1998}.  

\begin{figure*}
\includegraphics[width=\textwidth]{long_aq.pdf}
\caption{\label{fig:long_aq}
2D XRD data projected onto the 2$\theta$-$\phi$ (azimuthal angle) space is overlaid with the 1D azimuthally integrated data. Data was acquired using 1s integration time. (a) and ( d):
structure and texture of the sample after static compression to 7 GPa at 300 K. (b) and (e): structural and textural changes occurring after heating at HP up to 1360 K and 1450 K, respectively. (c) and (f): samples' structure after dynamic loading and quench down to 300 K; in both cases, the final Fe structure corresponds to the $\epsilon$ phase. }
\end{figure*}

% When talking about figure 4, group same situation together. In the text use less of the specific references to the numbers. Let the labels in the figures themselves be the primary source for knowing what the state is in each frame
The quality of the XRD data collected \textit{in situ} with ms aquisition time allows identification of the phases present at each investigated pressure. Additionally, longer-acquisition data with higher quality was also collected before and after dynamic compression for more detailed analysis. Representative data from two distinct runs is shown in FIG.\ref{fig:long_aq}, in which 2D XRD data projected onto the 2$\theta$-$\phi$ (azimuthal angle) space is overlaid with the 1D azimuthally integrated data to highlight the texture corresponding to each Fe phase. FIG.\ref{fig:long_aq} (a, d) shows the structure of the sample at 7 GPa and 300 K, prior to being heated or compressed dynamically; at these conditions, no phase transition is observed and the sample maintains the ambient Fe structure ($\alpha$ phase). FIG.\ref{fig:long_aq} (b, e) shows the sample transformation upon annealing at different temperatures; in particular, we notice that a higher temperature is required to ensure completion of the $\alpha$-$\gamma$ transformation at HT ( 1450 K rather than 1360 K). The reflections corresponding to the $\gamma$ phase show a non-uniform intensity distribution over the Debye-Scherrer rings, along the $\phi$ direction. This indicates that this phase crystallizes in large grains that do not cover the whole range of orientations with respect to the X-ray beam. FIG.\ref{fig:long_aq} (c, f) shows the final state reached by the sample, after dynamic loading and quenching. In both cases, complete transition to the $\epsilon$ phase is observed; this phase appears to have wider XRD peaks with more uniform azimuthal intensity distribution compared with its precursor, which is indicative of a finely-grained powder. However, a strongly preferred orientation can be inferred by the low intensity of the $\epsilon$-(002) reflection; this is expected, as the $\epsilon$[002] direction is the most compressible one in this phase. It is also interesting to note that, in the case of a pure $\gamma$ precursor (FIG.\ref{fig:long_aq}(c,f)), the $\epsilon$ appears more textured than in the case of a mixed $\alpha$ and $\gamma$ precursor (FIG.\ref{fig:long_aq}(b,c)). It is thus possible that the lower intensity of the peaks corresponding to the $\epsilon$ phase observed in the $\gamma$-$\epsilon$ transition (FIG.\ref{fig:waterfall_plots}(a,b)) with respect to the $\alpha$-$\gamma$ transition (FIG.\ref{fig:waterfall_plots}(c)) may also be due to the difference in grain size imparted by the starting phase. Indeed, bigger grains may results in fewer crystallites in the Bragg condition contributing to the peaks' intensity; in particular, in Run 1 only the $\epsilon$(010) reflection is visible \textit{in situ} (FIG.\ref{fig:waterfall_plots}(a)).

\section{\label{sec:discussion}Discussion}

\begin{figure}
\includegraphics[width=\columnwidth]{phase_diagram.pdf}
\caption{\label{fig:phase_dia} Experimental results from dDAC experiments compared with the state-of-the-art equilibrium phase diagram of Fe. Data are represented using markers of different shapes for each run, while the colours correspond to different Fe phases. The solid line is the equilibrium phase diagram by Shen 
\textit{et al.} \cite{Shen1998}; the dotted lines represent the $\gamma-\epsilon$ equilibrium boundary from other experimental studies \cite{Komabayashi2009, Kubo2003, Funamori1996}}
\end{figure}

The experiments here presented use a dDAC apparatus to perform HP-HT experiments and reach compression rates up to $\sim$500 GP/s to investigate the influence of the loading timescale on Fe behaviour at extreme conditions. The transition onsets measured at different temperatures for both the \textit{fcc}-\textit{hcp} ($\gamma$-$\epsilon$) and the \textit{bcc}-\textit{hcp} ($\alpha$-$\epsilon$) phase transitions of Fe are shown in FIG.\ref{fig:phase_dia} and overlaid with previous results from static compression experiments  \cite{Shen1998,Komabayashi2009,Kubo2003,Funamori1996}. For the $\gamma$-$\epsilon$ transition, results from dDAC are in very good agreement with the phase boundaries proposed by Shen \textit{et al.} \cite{Shen1998}, while they exhibit a lower transition onset compared with other static compression experiments \cite{Komabayashi2009, Kubo2003, Funamori1996}. It is worth noting that, independently of the considered reference for the equilibrium phase boundary, no increase in the phase transition onset is observed under dynamic loading. On the contrary, for a compression rate of 360 GPa/s at 300 K, the $\alpha$-$\epsilon$ transition is observed at higher P, \textit{i.e.}, 13.6 GPa, as compared with the static phase boundary. To confirm that the shift observed in the $\alpha$-$\epsilon$ transition is due to the compression timescale, we have performed an additional experiment (Run 4, not shown in FIG.\ref{fig:phase_dia}) at $\sim$100 lower compression rate; the XRD data confirms that, at 2.5 GPa/s the transition onset is 10.8 GPa, in much closer agreement with the equilibrium value. Our results thus demonstrate that the dDAC apparatus allows us to perform compression in a wide range of timescales (s down to ms) over which noticeable kinetic effects can be studied.

In previous HP studies, deviations of the transition pressure from the equilibrium phase diagram has been attributed to the presence of non-hydrostatic stresses in the DAC; in particular, sluggishness in the $\alpha$-$\epsilon$ transition has been observed \cite{Taylor1991}. To ensure that the results obtained in our HP-HT experiments are not influenced by non-hydrostatic stresses, we have investigated the hydrostaticity. The hexagonal unit cell is described by two lattice parameters, \textit{a} in the hexagonal plane, and \textit{c} in the stacking direction; as mentioned earlier, the \textit{c} axis is the most compressible one, and it tends to align with the compression axis in DAC experiments (see also FIG.\ref{fig:long_aq}). Thus, a deviation from hydrostaticity could be detected by analyzing the \textit{c/a} ratio, as a lowering of its value would indicate the presence of a stress gradient and a higher pressure along the compression axis \cite{Singh2006,Hawreliak2011}. FIG.\ref{fig:coa_ratio} shows the \textit{c/a} ratio of the $\epsilon$ phase calculated from our experimental data and compared with previous results from the literature \cite{Kon2015,Fischer2015} In particular,  Kon\^opkov\'a \textit{et al.} measured the \textit{c/a} ratio both in hydrostatic and non-hydrostatic conditions, \textit{i.e.}, with and without the use of a pressure transmitting medium, giving a reliable reference for the \textit{c/a} values measured in either condition. Our results are consistent with a hydrostatic compression state in the DAC sample chamber; thus, any deviation from the equilibrium phase diagram here observed can be ascribed to the effects of the strain rate.

The influence of strain rate on HP phase transitions has already been analyzed in several systems; however, the effects of fast compression on the phase boundaries are not univocal, and they strongly depend on the system and on the specifics of the deformation mechanism. For example, in certain systems higher compression rates can cause an increase of the transition pressure, as the fast loading hinders the rearrangement of the atoms (so-called \textit{kinetic hindrance}) \cite{Husband123}. In contrast, several systems have been observed to exhibit a phase transition lowering under dynamic compression. Silicon exhibits a lowering of the Si-I to Si-II transition under laser driven shock-compression \cite{McBride2019}, which our recent work has demonstrated to be due to a defect-free \textit{inelastic} deformation mechanism activated at ultrafast (ns) timescales \cite{Pandolfi2022}. Characterization of bismuth and antimony under dynamic compression has shown that certain transitions take place at pressures lower than the static phase boundary \cite{pepin2019,Coleman2019}. Interestingly, the lowering in pressure is observed for \textit{displacive} transitions, \textit{i.e.}, there is no change in unit cell volume through the transformation, which requires only small displacements of the atoms.

Our results evidence two distinct trends in the $\alpha$-$\epsilon$ and $\gamma$-$\epsilon$ phase transitions of Fe under dynamic loading: compared to static compression experiments, the onset of $\epsilon$ formation is increased and unchanged (or lowered), respectively. Based on previous results from dynamic compression experiments, this could be due to differences in the transition mechanisms that govern the transformations at the atomic level. Indeed, the \textit{bcc}-\textit{hcp} ($\gamma$-$\epsilon$) transformation happens via a combination of compression along one axis and shuffle of the planes \cite{Hawreliak2006, Kalantar2005}, and recent experiments have confirmed that the completion of the transformation requires two step: a displacive seeding followed by a \textit{reconstructive} (\textit{i.e.}, involving bond breaking) deformation \cite{BOULARD202030}. Under dynamic compression, a reconstructive transformation is expected to exhibit kinetic hindrance, as also suggested by molecular dynamics simulations of Fe $\alpha$-$\epsilon$ transition\cite{shao2018}. On the other hand, the \textit{fcc} and \textit{hcp} structures are more closely related, as they only differ for the stacking of the planes along one direction; the transformation is thus expected to happen via a purely displacive deformation. An increase of the compression rate can cause a high density of stacking faults, which could result in a high number of nucleation sites and ultimately favor the phase transition \cite{pepin2019, Ling2022, Sharma2020, SharmaPRL2020, SharmaPRX2020}. It is also worth noting that, at the strain-rates characteristics of our experiments, Fe plastic deformation is predominantly driven by thermally-activated dislocation flow \cite{Smith2011}, so the generation and diffusion of crystalline defects could strongly influence structural transformations in our HP-HT dDAC experiments.

\begin{figure}
\includegraphics[width=\columnwidth]{coa_ratio.pdf}
\caption{\label{fig:coa_ratio} Hydrostaticity analysis, by measuring the \textit{c/a} ratio of the \textit{hcp} cell parameters as a function of pressure. Our data (pink symbols) are compared with previous experiments in hydrostatic conditions by Fischer \textit{et al.} \cite{Fischer2015} and with results from Kon\^opkov\'a \textit{et al.}\cite{Kon2015}. In the latter study, results for both hydrostatic (solid symbols) and non-hydrostatic (empty symbols) were reported.
}
\end{figure}

\section{\label{sec:conclusion}Conclusion}

In this study, we demonstrate for the first time dynamic compression of a material in a dDAC setup coupled with stable laser-heating. Compression rates of hundreds of GPa/s were attained while simultaneously maintaining  high temperatures up to 2000K. Collection of time-resolved XRD data with millisecond time resolution enabled characterization of the phase transitions of Fe \textit{in situ}. Interestingly, the dDAC-laser-heating setup allows to explore (quasi)isothermal compression of a material, a pathway not attainable using conventional shock-compression techniques. We provide the first insight on the $\gamma$-$\epsilon$ phase transition of Fe, and compare our results with those obtained for the $\alpha$-$\epsilon$ transition, as well as the equilibrium phase diagram. We observe that the increase in strain-rate affects differently the phase transitions of Fe, and we attribute the differences to the specific deformation mechanisms. Indeed, no discernible change with respect to static compression experiments is observed for the displacive $\gamma$-$\epsilon$ phase transition up to 500 GPa/s. In contrast, the reconstructive $\alpha-\epsilon$ transition exhibits a marked shift of the transition onset with the compression timescale. This study demonstrates a new approach for exploration of HP-HT phase transitions under dynamic loading, covering an intermediate timescale between the well-established static- and shock- compression methods. Insight at these intermediate (ms) timescales will provide a more complete understanding of matter deformation at extreme conditions and dynamic geophysical processes. Our results also demonstrate that the strain rate affects differently phase transformations depending on the deformation mechanism, thus particular care should be taken when using experimental data to model geological processes at different timescales.


\begin{acknowledgments}
This work was carried out at the GeoSoilEnviroCARS (The University of Chicago,
Sector 13), Advanced Photon Source (Argonne National Laboratory). GeoSoilEnviroCARS is
supported by the National Science Foundation—Earth Sciences (No. EAR-1634415). The
Advanced Photon Source is a U.S. Department of Energy (DOE) Office of Science User Facility operated for the DOE Office of Science by Argonne National Laboratory under Contract No. DE-AC02-06CH11357. A.E.G., R.L.S., and S.P. acknowledge support from 2019 DOE FES ECA. A.E.G. and W.L.M. acknowledge support by the Geophysics Program at NSF (EAR2049620). M.R. acknowledges support from the College of Physical and Mathematical Sciences at Brigham Young University and DOE SULI 2021 at SLAC National Lab.
\end{acknowledgments}

\nocite{*}

\bibliography{bibtexfile.bib}% Produces the bibliography via BibTeX.

\end{document}
%
% ****** End of file version1.tex ******

