\section{Problem Formulation}
\label{formulation}

\noindent\textbf{Graphic Layout.}
A graphic layout $x$ is composed of a set of graphic elements $\left\{\boldsymbol{e}_i\right\}^N_{i=1}$, where $N$ denotes the number of elements.
Each element $\boldsymbol{e}_i$ has an element type $c_i$ and a bounding box indicating its left $l_i$, top $t_i$, right $r_i$, and bottom $b_i$ coordinates.
Following the advanced layout generation methods~\cite{vtn,c2f,layouttransformer,canvasvae,nguyen2021diverse, rahman2021ruite}, we represent an element as a sequence with 5 discrete tokens, i.e., $\boldsymbol{e}_i = \left\{c_i l_i t_i r_i b_i\right\}$, where the continuous bounding box coordinates are uniformly discretized into integers between $[0,K)$. Then, we represent a layout as a concatenation of element sequences:
\vspace{-15px}
\begin{align}\label{eq:layout}
    \mathbf{x} = \left\{\langle \text{sos}\rangle c_1 l_1 t_1 r_1 b_1\|\dots\|c_N l_N t_N r_N b_N\langle \text{eos}\rangle\right\},
\end{align}
where $\langle \text{sos}\rangle$ and $\langle \text{eos}\rangle$ are special tokens indicating the start and end of a sequence, and token $\|$ indicates the separator between any two elements.
Obviously, the layout sequence is \emph{heterogeneous}.
The element type tokens are \emph{categorical}, while the coordinate tokens are \emph{ordinal}.


\noindent \textbf{Graphic Layout Generation.}
In this work, we primarily focus on unconditional layout generation.
Specifically, we learn a generative model $p_\theta(\mathbf{x})$ parameterized by $\theta$, which synthesizes diverse and high-quality graphic layouts.




