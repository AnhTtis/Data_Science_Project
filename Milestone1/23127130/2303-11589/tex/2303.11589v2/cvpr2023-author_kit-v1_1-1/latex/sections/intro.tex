\section{Introduction}
\label{sec:intro}



\begin{figure}[t]
  \centering
  \begin{subfigure}{1\linewidth}
   \includegraphics[width=1\linewidth]{cvpr2023-author_kit-v1_1-1/latex/pics/_all_3_ours.jpg}
    \caption{Mild forward corruption process}
    \label{noise_ours}
  \end{subfigure}
  \hfill
  \begin{subfigure}{1\linewidth}
   \includegraphics[width=1\linewidth]{cvpr2023-author_kit-v1_1-1/latex/pics/_all_3_absorb.jpg}
    \caption{Absorbing~\cite{d3pm}}
    \label{noise_absorb}
  \end{subfigure}
  \hfill
  \begin{subfigure}{1\linewidth}
   \includegraphics[width=1\linewidth]{cvpr2023-author_kit-v1_1-1/latex/pics/_all_3_uniform.jpg}
    \caption{Uniform~\cite{2015}}
    \label{noise_uniform}
    \hfill
        \begin{subfigure}{1\linewidth}
    \centering
   \includegraphics[width=1\linewidth]{cvpr2023-author_kit-v1_1-1/latex/pics/legend.pdf}
    \caption{The mapping between colors and element types used in \cref{noise_ours,noise_absorb,noise_uniform}}
    \label{graphic layout eg}
  \end{subfigure}
  \end{subfigure}
 \vspace{-15px}
\caption{Comparison of different forward corruption processes. We sample the layouts at the timesteps 0, 1/6, 2/6, 3/6, 4/6, 5/6, and 1 of the total timestep. The blank page is used when the format of the layout sequence is destroyed.} %
   \label{noise}
   \vspace{-12.5px}
\end{figure}


Graphic layout, i.e., the \emph{sizes and positions} of elements, is important to the interaction between the viewer and the information.
Recently, layout generation attracts growing research interest.
Leading approaches~\cite{layouttransformer,unilayout,blt,c2f} often represent a layout as a sequence of elements and leverage Transformer~\cite{vaswani2017attention} to model element relationships.
As the placement of one element could depends on any part of a layout, \emph{global context modeling} plays a critical role in layout generation. 
However, there is no satisfactory solution to it.
Some studies simply consider biased context~\cite{layouttransformer,unilayout,vtn,c2f}.
They generate layout sequences autoregressively, where the generation order for elements is predefined and the placement of one element only depends on a certain part of layout.
A few other studies try to utilize global context by non-autoregressive generation~\cite{ndn}.
Unfortunately, they fail to improve the generation quality significantly since it is too challenging to generate a sequence in a single pass~\cite{ghazvininejad2019mask}.


Meanwhile, the emerging diffusion probabilistic model (DDPM)~\cite{ddpm,song2020score} achieves amazing performance on many generation tasks~\cite{amazingdiffusion1,amazingdiffusion2,amazingdiffusion3,amazingdiffusion4,amazingdiffusion5,amazingdiffusion6,amazingdiffusion7}.
It consists of multiple rounds, each of which gradually denoises the latent variables towards the desired data distribution.
This sort of process seems to be a promising solution to layout generation.
First, the layout generated in the last round could serve as the global context for the generation in the next round.
Second, by multiple rounds of denoising, a layout could be refined iteratively, overcoming the challenge of single-pass generation from non-autoregressive models.


To this end, we propose \emph{LayoutDiffusion} to improve graphic layout generation.
As a layout is represented as a sequence of discrete tokens~\cite{unilayout,blt,layouttransformer}, we formulate layout generation as a discrete diffusion process.
Roughly speaking, it samples a layout by reversing a forward process.
The forward process corrupts the real data into a sequence of increasingly noisy latent variables by a fixed Markov Chain.
The reverse process starts from noise and denoises it step by step via learning the posterior distribution.

To ease the estimation of the posterior distribution, it is critical to design a \emph{mild} forward corruption process~\cite{improvedDdpm}, in which latent variables in neighboring steps do not differ too much and become increasingly chaotic with the growth of forward steps (see~\cref{noise_ours}).
However, designing such a process for layout is non-trivial, due to the \emph{heterogeneous} nature of the layout sequence, where the tokens representing element types are \emph{categorical} while the tokens representing element coordinates are \emph{ordinal}.
Existing discrete forward processes hardly consider heterogeneous tokens.
Directly applying them to layout data often leads to harsh corruptions, where a layout is changed dramatically at each step (see \cref{noise_absorb,noise_uniform}).
For example, the uniform process in~\cref{noise_uniform} will transition an element type token to a coordinate token, drastically violating the layout semantics.



To realize a mild corruption process for layout, we make three important observations.
\emph{(i) Legality}. 
The transition between type tokens and coordinate tokens will lead to an illegal layout sequence, resulting in an unpredictable change between forward steps.
Hence, it is vital to impose legality during the corruption process.
\emph{(ii) Coordinate Proximity}.
Coordinate tokens are ordinal, and thus transitioning a coordinate token to its proximal tokens (\eg, from $0$ to $1$) will introduce a milder change to a layout compared with transitioning to distant ones (\eg, from $0$ to $127$).
\emph{(iii) Type Disruption}.
Unlike coordinate tokens, type tokens are categorical and do not have particular proximity.
Simply transitioning one type to another may cause abrupt semantic changes to a layout (\eg, from a button to a background image).

Motivated by the above observations, we propose a block-wise transition matrix coupled with a piece-wise linear noise schedule in LayoutDiffusion.
The transition matrix is designed as follows.
First, to achieve legality, we only allow the internal transition between coordinate tokens and that between type tokens.
Second, regarding coordinate proximity, we leverage discretized Gaussian~\cite{d3pm}, where the transition between more proximal tokens takes a higher probability, for the transition between coordinate tokens.
Third, as for type disruption, we introduce absorbing state~\cite{d3pm}.
Each type token either stays the same or transitions to the absorbing state.
To further alleviate type disruption, we propose a piece-wise linear noise schedule to make the transition for element types only occur in the late stage of the forward process.
With above techniques, LayoutDiffusion achieves the mild forward process shown in~\cref{noise_ours}.


Our design also enables LayoutDiffusion to perform certain conditional layout generation tasks in a \emph{plug-and-play} manner without re-training, which has never been explored by previous work.
Specifically, owning to the mild forward process achieved by LayoutDiffusion, its reverse process is to iteratively improve a layout, which naturally supports the task of layout refinement~\cite{rahman2021ruite}.
Besides, as the transition of element types only occurs in the late forward process, LayoutDiffusion will determine the element types in a layout quickly in the reverse process.
Thus, it can perform generation conditioned on types by simply keeping the types fixed and running the reverse process.

In summary, this work makes four key contributions:
\begin{compactenum}

\item We formulate layout generation as a discrete diffusion process, which addresses biased context modeling by iterative refinement from a non-autoregressive model.

\item We design a new diffusion process based on the heterogeneous nature of layout sequence (legality, coordinate proximity and type disruption). It not only better suits layout data but also showcases a promising way of applying diffusion models to other heterogeneous data.

\item We enable certain conditional layout generation tasks in a plug-and-play manner without re-training. 

\item We make extensive experiments and user studies.
LayoutDiffusion outperforms existing methods on all the tasks in terms of most evaluation metrics, even if it is not re-trained for conditional generation tasks. 

\end{compactenum}