\documentclass[10pt,twocolumn,letterpaper]{article}

\usepackage[pagenumbers]{cvpr} 


\usepackage{graphicx}
\usepackage{amsmath}
\usepackage{amssymb}
\usepackage{booktabs}
\usepackage{multirow}
\usepackage{makecell}
\usepackage{float}
\usepackage{diagbox}

\usepackage{graphics}
% \usepackage{floatrow}
\usepackage{sidecap}
\usepackage{setspace}
\usepackage[linesnumbered,ruled,vlined]{algorithm2e}

\usepackage[pagebackref,breaklinks,colorlinks]{hyperref}
\usepackage{paralist}

% Support for easy cross-referencing
\usepackage[capitalize]{cleveref}
\crefname{section}{Sec.}{Secs.}
\Crefname{section}{Section}{Sections}
\Crefname{table}{Table}{Tables}
\crefname{table}{Tab.}{Tabs.}



\begin{document}

%%%%%%%%% TITLE - PLEASE UPDATE
\title{LayoutDiffusion:  Improving Graphic Layout Generation by \\ Discrete Diffusion Probabilistic Models}

\author{
Junyi Zhang$^1$ \qquad Jiaqi Guo$^2$ \qquad Shizhao Sun$^2$ \qquad Jian-Guang Lou$^2$ \qquad Dongmei Zhang$^2$\\
$^1$Shanghai Jiao Tong University \qquad $^2$Microsoft Research Asia\\
{\tt\small junyizhang@sjtu.edu.cn \qquad \{jiaqiguo,~shizsu,~jlou,~dongmeiz\}@microsoft.com}
}
\maketitle

%%%%%%%%% ABSTRACT
\begin{abstract}
    Creating graphic layouts is a fundamental step in graphic designs.
    In this work, we present a novel generative model named LayoutDiffusion for automatic layout generation.
    As layout is typically represented as a sequence of discrete tokens, LayoutDiffusion models layout generation as a discrete denoising diffusion process.
    It learns to reverse a mild forward process, in which layouts become increasingly chaotic with the growth of forward steps and layouts in the neighboring steps do not differ too much.
    Designing such a mild forward process is however very challenging as layout has both categorical attributes and ordinal attributes.
    To tackle the challenge, we summarize three critical factors for achieving a mild forward process for the layout, i.e., legality, coordinate proximity and type disruption.
    Based on the factors, we propose a block-wise transition matrix coupled with a piece-wise linear noise schedule.
    Experiments on RICO and PubLayNet datasets show that LayoutDiffusion outperforms state-of-the-art approaches significantly. 
    Moreover, it enables two conditional layout generation tasks in a plug-and-play manner without re-training and achieves better performance than existing methods.

\end{abstract}
\vspace{-5px}
%%%%%%%%% BODY TEXT
\section{Introduction}

The increasing complexity of source code poses a key challenge to the reliability of large-scale software systems. Software bugs in these systems can lead to safety issues~\cite{bug_safety} for users around the world as well as cause non-negligible financial losses~\cite{bug_loss}. As such, developers have to spend a large amount of time and effort on bug fixing. Consequently, \aprfull (\apr), designed to automatically generate patches to fix software bugs, has attracted wide attention from both academia and industry~\cite{long2016prophet, legoues2012genprog, long2015spr, lou2020can, tufano2018empstudy}. 


To achieve \apr, one popular approach is known as Generate-and-Validate (G\&V)~\cite{qi2015gv, ghanbari2019prapr, lou2020can, le2016hdrepair, legoues2012genprog, wen2018capgen, hua2018sketchfix, martinez2016astor, koyuncu2020fixminder, liu2019tbar, liu2019avatar}, which is typically based on the following pipeline: First, fault localization techniques~\cite{wong2016fl, abreu2007ochiai, zhang2013injecting, papadakis2015metallaxis, li2019deepfl, li2017transforming} are applied to determine the suspicious locations in programs where bugs are likely to exist. Then, the buggy locations are used by the \apr tools to generate a list of patches that replace buggy lines with correct lines. Afterward, each patch is validated against the original test suite to identify any \emph{plausible patches} (i.e., passing all tests in the test suite). Finally, to determine the \emph{correct patches}, developers examine the list of plausible patches to see if any of them can correctly fix the bug. 

Traditional \apr tools can mainly be categorized into heuristic-based~\cite{legoues2012genprog, le2016hdrepair, wen2018capgen}, constraint-based~\cite{mechtaev2016angelix, le2017s3, demacro2014nopol, long2015spr} and \template~\cite{ghanbari2019prapr, hua2018sketchfix, martinez2016astor, liu2019tbar, liu2019avatar}. Among these traditional tools, \template \apr tools~\cite{ghanbari2019prapr, liu2019tbar, benton2020effectiveness} have been able to achieve state-of-the-art results. \Template \apr tools typically leverage pre-defined templates (e.g., adding a nullness check) for bug fixing. However, since these fix templates are typically handcrafted, the number and types of bugs they are able to fix can be limited. 



To address the limitations of traditional \apr, researchers have proposed various \learning \apr tools~\cite{li2020dlfix, chen2018sequencer, jiang2021cure, lutellier2020coconut, zhu2021recoder, ye2022rewardrepair} based on the \nmtfull (\nmt) architecture~\cite{sutskever2014mt} where the input is the buggy code snippets and the goal is to translate the buggy code snippets into a fixed version. To accomplish this, \learning \apr tools require supervised training datasets with pairs of both buggy and fixed code snippets in order to learn how to perform this translation step. These training data are usually obtained by mining historical bug fixes using heuristics/keywords~\cite{dallmeier2007benchmark}, which can be imprecise for identifying bug-fixing commits; even the actual bug-fixing commits can include irrelevant code changes, leading to further pollution in the dataset~\cite{xia2022alpharepair}.
% 
Moreover, it can be hard for such \apr tools to generalize and fix bug types unseen during training. 



To better leverage recent advances in \plmfull{s} (\plm{s}), researchers~\cite{xia2022alpharepair, xia2023repairstudy, kolak2022patch, prenner2021codexws} have directly applied \plm{s} to generate patches without bug-fixing datasets. These \llm-based \apr tools work by either directly generating a complete code function~\cite{prenner2021codexws, xia2023repairstudy} or predict/infill the correct code snippet given its surrounding context~\cite{xia2022alpharepair, xia2023repairstudy}. By directly using \llm{s} that are pre-trained on billions of open-source code snippets, \llm-based \apr tools can achieve state-of-the-art performance on many repair datasets~\cite{xia2022alpharepair}. 


% 
%
%

Traditional \apr tools have long used the insight of the \emph{plastic surgery hypothesis}~\cite{barr2014plastic} where it states that the code ingredients to fix a bug already exist within the same project. Traditional \apr tools have manually designed pattern-~\cite{ghanbari2019prapr, saha2017elixir} or heuristic-based~\cite{jiang2018simfix, legoues2012genprog} approaches to finding and using such relevant code ingredients to generate fixes for bugs. However, the plastic surgery hypothesis has been largely ignored in \llm-based \apr. In fact, \llm provides a unique opportunity to fully automate the plastic surgery hypothesis idea via fine-tuning (learning project-specific information via model updates from the buggy project) and prompting (directly providing relevant code ingredients to the model), and make it directly applicable to different languages (since the \llm{s} are typically multi-lingual).%
Moreover, despite the intensive manual efforts involved, traditional \apr tools still cannot fully leverage project-specific information due to large search space for leveraging/composing existing code ingredients. In contrast, the project-specific information can effectively leveraged by \llm{s} due to their power in code understanding/vectorization, e.g., even partial/imprecise information may still guide \llm{s} in correct patch generation!
 To this end, we ask the question: \emph{How useful is the plastic surgery hypothesis in the era of \plm{s}}?








\mypara{Our Work.} To answer the question, we present \ourtech{\xspace} -- a \llm-based approach that automatically utilizes the plastic surgery hypothesis by systematically combining multiple fine-tuning and prompting strategies for \apr. \ourtech fine-tunes \plm{s} using two novel domain-specific training strategies: \textbf{\epfinetune} -- we fine-tune using the original buggy project by aggressively masking out a high percentage of tokens, which allows \plm to learn project-specific code tokens and programming styles; and \textbf{\rofinetune} -- which only masks out a single continuous code sequence per training sample, allowing the model to get used to the final \csapr task of predicting a single continuous code sequence. Furthermore, we directly leverage the ability for \plm{s} to understand natural language instructions and introduce a novel prompting strategy, \textbf{\idprompting}, which uses information retrieval and static analysis to obtain a list of relevant identifiers for the buggy lines. While such relevant identifiers are critical for fixing some difficult bugs, they may not be seen by the \llm during inference due to limited context window size. Through the use of prompting, we directly tell the model to use these extracted identifiers (relevant code ingredients) to generate the correct code. Finally, to perform repair, we combine all four model variants (including the base model, both fine-tuned models and the base model with prompting) for the final repair.





While our insight of leveraging the plastic surgery hypothesis for \llm-based \apr is generalizable across different types of \plm{s}, to implement \ourtech, we choose a recent \plm{\xspace}, \ctfive~\cite{wang2021codet5}, which is pre-trained on millions of open-source code snippets. \ctfive is an encoder-decoder model trained using \mspfull (\msp) objective where a percentage of tokens are masked out and each continuous masked token sequence is referred to as a masked span. Also, although we only extract relevant identifiers from the current buggy project (since this paper focuses on the plastic surgery hypothesis), our work can be easily extended to obtain other code information (such as relevant statements or functions) from other sources, such as  the massive pre-training corpora~\cite{husain2020codesearchnet} or historical bug-fixing datasets~\cite{jiang2019infer}, which can provide more coding knowledge for \llm{s}. Besides, although we mainly focus on using traditional string comparison algorithms for information retrieval in this paper, these techniques can be easily replaced by other frequency-based retrieval~\cite{robertson2009probabilistic} and neural search (or embedding-based search)~\cite{reimers2019sentence}.
  In summary, this paper makes the following contributions:


%


\begin{itemize}[noitemsep, leftmargin=*, topsep=0pt]
    \item \textbf{Dimension.} This paper is the first to revisit the important plastic surgery hypothesis in the era of \llm{s}. It opens up a new dimension for \llm-based \apr to incorporate previously neglected information from the buggy project itself to boost \apr performance. Furthermore, it demonstrates the promising future of retrieval-based prompting for modern \llm-based \apr.
    \item \textbf{Implementation.} We implement \ourtech based on the recent \ctfive model. We augment the model using two novel fine-tuning strategies: \epfinetune and \rofinetune, along with a novel prompting strategy based on information retrieval and static analysis: \idprompting. We combine the patches generated by all four models together and perform patch ranking to speed up \apr.% 
    \item \textbf{Evaluation Study.} We conduct an extensive evaluation against state-of-the-art \apr tools. On the widely studied \dfj 1.2 and 2.0 datasets~\cite{just2014dfj}, \ourtech is able to achieve the new state-of-the-art results of 89 and 44 correct bug fixes (15 and 8 more than best baseline) respectively.  Furthermore, we perform a broad ablation study to justify our design. \ourtech demonstrates for the first time that the plastic surgery hypothesis can substantially boost \llm-based \apr and advance state-of-the-art \apr, while being fully automated and general. Moreover, even partial/imprecise code ingredients may still effectively guide \llm{s} for \apr!
\end{itemize}


\section{Related work}
\noindent \textbf{Video foundation models.}
With sufficient computational power and an abundant source of data, there have been attempts to build a single large-scale foundation model that can be adapted to diverse downstream tasks.
Along with the success of foundations models in the natural language processing domain~\cite{brown2020language,chen2021evaluating,devlin2019bert} and in computer vision~\cite{bertasius2021space,jia2021scaling,radford2021learning}, video data has become another data type of interest, as it has grown in scale due to numerous internet video-sharing platforms.
Accordingly, several methods to train a video foundation model have been proposed.
Due to the innate multi-modality of video data, \textit{i.e.}, a combination of visual $\cdot$ vocal $\cdot$ textual context, most works have centered around the variations of the cross-modal attention mechanism \cite{akbari2021vatt,bertasius2021space,gabeur2020multi,luo2020univl,neimark2021video,tan2021look,wei2020multi,yang2021taco}.
In addition, as most video data lack proper labels or descriptions, contrastive learning methods were studied to learn meaningful feature representations or enhance video-text alignment in a self-supervised manner \cite{akbari2021vatt,kuang2021video,luo2020univl,yang2021taco}.

More specifically, MERLOT \cite{zellers2021merlot} proposed a multi-modal representation learning method for visual commonsense reasoning, which also performed well in twelve video reasoning tasks.
VATT \cite{akbari2021vatt} introduced a multi-modal learning method via contrastive learning. 
The pre-trained model performed well in a variety of vision tasks from image classification to video action recognition and zero-shot video retrieval.
Another representative work, UniVL \cite{luo2020univl} proposed a straightforward pre-training method with auxiliary loss functions. 
After fine-tuning on a specific task, the pre-trained model performed outstandingly in a wide range of tasks of text-to-video retrieval, action segmentation, action step localization, video sentiment analysis, and video captioning.
Other foundation models for multiple video tasks include \cite{li2020hero,sun2019learning,sun2019videobert,zhu2020actbert,fu2021violet,wang2022all}. 

\noindent \textbf{Auxiliary learning.}
In order to enhance the performance of one or a multitude of primary tasks, auxiliary learning methods can be incorporated.
\cite{ruder2017overview} introduced Multi-task learning (MTL) to the deep neural networks by training a single model with multiple task losses to assist learning on the main task.
Such a method is generally adapted to pre-train the foundation models in the self-supervised manner~\cite{li2020hero,sun2019learning,sun2019videobert,zhu2020actbert,fu2021violet,wang2022all}.
However, these various pretext task losses used in the pre-training phase are ignored in the fine-tuning phase, and only the primary task loss is minimized.

Recently, meta-learning methods have been introduced for auxiliary learning.
\cite{liu2019self,navon2020auxiliary,shu2019meta} proposed a meta-learning method in which the model learns auxiliary tasks to generalize well to unseen data. 
In these settings, a separate subset of data is held out as the primary task, while the others are used as auxiliary tasks that aid the primary task's performance.
Similar methods were adopted for computer vision tasks such as semantic segmentation \cite{xu2021leveraging}.
Other domain applications include navigation tasks with reinforcement learning \cite{ye2021auxiliary}, or self-supervised learning methods on graph data \cite{hwang2020self}.
%!TEX root = ../main.tex

\section{Inductive Conformal Prediction}
\label{sec:pre:icp}
Given a set $\{ z_i = ( x_i, y_i ) \}_{i=1}^l$ with observation $x_i \in \calX$ and label $y_i \in \calY$ such that each $z_i \in \calZ := \calX \times \calY $ is drawn i.i.d. from an \emph{unknown} distribution on $\calZ$, inductive conformal prediction (ICP) provides 
% a simple yet powerful framework to learn 
a \emph{set prediction} $\Feps(x) \subseteq \calY$, parameterized by an error rate $0 < \epsilon <1$, such that given a new sample $z_{l+1} = (x_{l+1},y_{l+1})$ satisfying an \emph{exchangeability} condition (elaborated in Theorem~\ref{thm:icp-validity}), we have
\bea\label{eq:icpmiscoverage}
\probof{ y_{l+1} \in \Feps(x_{l+1}) } \geq 1-\epsilon, 
\eea
\ie, the prediction set $\Feps$ guarantees to contain the true label $y_{l+1}$ with probability at least $1-\epsilon$. 

% In order to achieve the probabilistic coverage in~\eqref{eq:icpmiscoverage}, ICP performs the following three steps.

{\bf Training}. We start by dividing the dataset into a \emph{proper training set} $\{ z_1,\dots,z_m \}$ and a \emph{calibration set} $\{ z_{m+1},\dots,z_{l} \}$. We shorthand $n = l - m$ as the size of the calibration set.
We learn a prediction function $f: \calX \rightarrow \tcalY$ from the proper training set using \emph{any} architecture, which allows us to fully exploit the power of modern deep learning. The prediction space $\tcalY$ can be the same as the label space $\calY$, or can contain auxiliary information such as a heuristic notion of uncertainty (\eg, softmax scores in classification or a heatmap in the case of keypoint detection). 

{\bf Conformal calibration}. 
% Leveraging the learned $f$, 
We define a \emph{nonconformity} function $S: \calZ^{m} \times \calZ \rightarrow \Real{}$ to measure how well a given sample $z = (x,y)$ \emph{conforms} to the proper training set. A popular instance of $S$ leverages the learned prediction $f$:
\bea \label{eq:nonconformity}
S\parentheses{\cbrace{z_1,\dots,z_m},(x,y)} \stackrel{\eg}{=} r(y,f(x)),
\eea
where $r: \calY \times \tcalY \rightarrow \Real{}$ is a measure of disagreement between the label $y$ and the prediction $f(x)$. For example, consider $\calY = \tcalY = \Real{}$, one can design $r(y,f(x)) = \abs{y - f(x)}$: if $(x,y)$ poorly conforms to the training set, $f$ will incur large errors.   
While the function $S$ can be arbitrary (\eg, a learnable neural network~\cite{stutz22iclr-learnconformal}), \eqref{eq:nonconformity} is a convenient definition since $f$ is implicitly dependent on $\{z_i\}_{i=1}^m$ and $r$ can incorporate domain-specific knowledge.
We then compute the nonconformity scores on the calibration set as $\alpha_i = r(y_i,f(x_i)), i = m+1,\dots,l$,
and sort them in \emph{nonincreasing} order $\alpha_{\pi(1)}\geq\dots \geq \alpha_{\pi(n)}$, where $\pi(i) \in \{m+1,\dots,l\}$ is an index permutation.
 % (offset by $m$).

{\bf Conformal prediction}. Given a new observation $x_{l+1}$ (with an unknown $y_{l+1}$) and a user-specified $\epsilon \in (0,1)$, we compute the inductive conformal prediction (ICP) set as
\bea\label{eq:icpcompute}
\Feps \parentheses{x_{l+1}} = \cbrace{y \in \calY \mid \alpha^y \leq \alpha_{\pi(\floor{(n+1)\epsilon})}},
\eea
where $\alpha^y = r(y,f(x_{l+1}))$
is the nonconformity score of the new sample when fixing the true label to be $y$. In other words, the ICP set~\eqref{eq:icpcompute} outputs the set of all labels that make the nonconformity score of the new sample no greater than $\alpha_{\pi(\floor{(n+1)\epsilon})}$ -- the $\floor{(n+1)\epsilon}$-th largest nonconformity score in the calibration set. 
% By doing so, ICP ensures that there are at least $\floor{(n+1)\epsilon}$ samples in the calibration set that are less conforming than the new sample. 
We have the following result stating the probabilistic coverage of the ICP set~\eqref{eq:icpcompute}.
% provides a valid statistical coverage of the true label $y_{l+1}$.

\begin{theorem}[Validity of ICP Coverage {\cite{vovk05book-conformal,lei18jasa-conformal,vovk12acml-icpconditional}}] \label{thm:icp-validity}
If $z_{m+1},\dots,z_l$, $z_{l+1} = (x_{l+1},y_{l+1})$ are exchangeable, \ie, their distribution is invariant under permutation, then
\bea\label{eq:icpvalidity}
1 - \epsilon \leq \probof{y_{l+1} \in \Feps(x_{l+1})} \leq 1 - \epsilon + 1/(n+1)
\eea
for any $\epsilon \in (0,1)$. Furthermore, when conditioned on the calibration set, calling $h = \floor{(n+1)\epsilon}$, we have
\begin{equation}\label{eq:beta}
\hspace{-4mm}\probof{y_{l+1}\!\in\!\Feps(x_{l+1})\!\mid\!\{z_{m+1},\dots,z_l\}}\!\sim\!\mathrm{Beta}(n+1\!-\!h,h).
\end{equation}
\end{theorem}
A few remarks are in order about Theorem~\ref{thm:icp-validity}.
First, asking $z_{m+1},\dots,z_l,z_{l+1}$ to be exchangeable is weaker than asking them to be independent. However, this assumption typically fails when the calibration set is a single video sequence, where the image frames $\{z_{m+1},\dots,z_l\}$ are temporally correlated~\cite{luo21arxiv-conformalsafety}. Fortunately, as we detail in Section~\ref{sec:experiments}, the way the LineMOD Occlusion dataset~\cite{brachmann14eccv-linemodocc} was collected makes the exchangeability condition easily satisfied, which also suggests best practices to make the exchangeability condition hold in computer vision. 
Second, the lower bound in~\eqref{eq:icpvalidity} can be intuitively proved because under exchangeability, $\alpha_{l+1} := r(y_{l+1},f(x_{l+1}))$ --the nonconformity score of the new sample with the true label-- is \emph{exchangeable} with the nonconformity scores of the calibration samples, and hence \emph{equally likely} to fall in anywhere between the scores $\{ \alpha_{\pi(i)}\}_{i=1}^n$. Consequently, $\probof{y_{l+1} \in \Feps(x_{l+1})} = \probof{\alpha_{l+1} \leq \alpha_{\pi(\floor{(n+1)\epsilon})}} = 1 - \floor{(n+1)\epsilon}/(n+1) \geq 1 - \epsilon$. The upper bound in \eqref{eq:icpvalidity} states that $1-\epsilon$ is not overly conservative (indeed tight if $n$ is large). 
Lastly, the probabilistic guarantee in \eqref{eq:icpvalidity} is \emph{marginal} over the randomness of the calibration set, meaning if one chooses an infinite number of calibration sets,  the \emph{average} empirical coverage will converge to $1-\epsilon$. This, however, implies that the empirical coverage given one calibration set is a random variable that fluctuates as the Beta distribution~\eqref{eq:beta}. Fig.~\ref{fig:beta-distribution} plots the Beta distribution at $\epsilon=0.1$ with different sizes of the calibration set. We observe that as $n$ increases the empirical coverage becomes more concentrated at $1-\epsilon$. Our experiments show that even with a small ($n=200$) calibration set, the empirical coverage is close to, and mostly higher than, $1-\epsilon$.

% \begin{proposition}[Conditional Validity of ICP {\cite{vovk12acml-icpconditional}}] \label{prop:icp-conditional-validity}
% \red{To be filled out}
% \end{proposition}


% Proposition~\ref{prop:icp-validity} states that, if the new observation $z_{l+1}$ is exchangeable with the calibration set (which is a weaker condition than requiring $z_{l+1}$ is jointly i.i.d. with the calibration set), then no matter which prediction function $f$ has been learned from the proper training set, and which function $A$ has been chosen to compute the nonconformity score, we have at least $1-\epsilon$ confidence that the ICP $\Feps$ defined in \eqref{eq:icp} contains the true label. Of course, the caveat here is that the quality of the learned prediction function $f$ and the nonconformity function $A$ will decide the conservativeness of the ICP $\Feps$. For example, if $f$ has poor predictive power, then the set $\Feps$ may be arbitrarily large so that it tells little information about the true label $y$. \red{Fortunately, as we will show in experiments, with modern deep learning architectures for learning $f$, we can obtain ICPs that are both confident and tight.}

%!TEX root = ../main.tex
% \begin{figure}
% \hspace{-4mm}\includegraphics[width=1.1\columnwidth]{icp-overview-half.pdf}
% \caption{Given a learned prediction function and a calibration set of $n$ samples, conformal calibration uses a nonconformity function~\eqref{eq:nonconformity} to compute and sort nonconformity scores $\{ \alpha_{\pi(i)}\}_{i=1}^n$. Given a new observation and an error rate $\epsilon$, conformal prediction~\eqref{eq:icpcompute} outputs a prediction set of all labels under which the nonconformity score of the new sample is no larger than $\alpha_{\pi(\floor{(n+1)\epsilon})}$.
% \label{fig:icp-overview}}
% \end{figure}
\begin{figure}
\vspace{-4mm}
\begin{center}
\includegraphics[width=0.6\columnwidth]{beta.pdf}
\end{center}
\vspace{-6mm}
\caption{Beta distribution of the conditional coverage in~\eqref{eq:beta} with $\epsilon=0.1$ and different $n$. Notice how the conditional probability becomes more concentrated around $1-\epsilon$ when $n$ increases.
\label{fig:beta-distribution}}
\vspace{-7mm}
\end{figure}

\subsection{Approach}
This paper describes three concepts with the aim 
of a robust, energy-efficient robot control. 
While these concepts are rather straightforward and intuitive, they 
are not yet utilised in mainstream manipulator control.
It is not argued that all of these principles 
need to be used, but if the (sub)task allows it, using any 
of these principles can have a positive impact on the energy-efficiency.

\vspace{-4mm}
\subsubsection{Contextual prior knowledge:}
When humans perform a transportation task, they do  
not perform strict PTP motions such as traditional industrial 
robots. Instead, movements with a certain tolerance on the position 
are performed. 
This allows the natural dynamics of the system to be exploited, 
as will be explained in the following subsection. Typically, the 
tolerances come from 
knowledge about both the environment and the task context. For example, 
the spatial constraints, 
fragility of the payload, 
if a certain part of the task requires a higher precision, 
etc. 
It is clear that this knowledge precedes the task execution and 
determines how the human will perform the task.
The execution is generally done in multiple states, e.g., picking up 
the payload, moving and placing near the target position, making small 
adjustments when necessary.

This knowledge is used to split up the task in multiple 
subtasks and identify the different requirements. Robust 
controllers and monitors are then developed to perform and coordinate 
between these subtasks. 
Examples of such requirements are crane like operations such as:
 lifting the load to a certain 
height, transporting it without colliding, and lowering the load 
until contact is made.

The task also does not require high control precision throughout, 
but only for the initial grasping and final placement. 
In addition, this does not need to come only from the 
control.
Geometric constraints such as the environment or a previously placed
payload can be used to achieve this accuracy by sliding against them. 
This is further explained in section \ref{sec:discrete_control}.

By using this knowledge, lower-cost (and often also lower-weight) 
hardware can be used, so that a more robust, 
energy efficient execution can be developed. 
Thus, for a repetitive task, the cost of 
designing and implementing a task-specific controller is not 
necessarily higher than a generic, less energy-efficient controller.\\

\vspace{-8mm}
\subsubsection{Exploiting natural dynamics}
In this work, the natural dynamics of the system are used to inject 
as little energy as possible, resulting in energy-efficient 
motions.
However, precise control of the timing is lost when the system freely
follows its natural dynamics.

Due to the layout of the used cable robot (Fig.1), when the end effector is
in a fully constrained position, releasing the power of one (or more) 
of the motors, will result in a pendulum-like swing around the 
cables that are still powered, or braked. 
This swing is used in the control strategy to cover the horizontal 
distance while consuming a minimal amount of energy. 

\vspace{-4mm}
\subsubsection{Active use of brakes}
Based on the context, certain subtasks may occur where a joint 
does not need to move. Instead of producing a constant standstill
torque, it can also be opted to brake the joint.
Another case occurs when the demanded motion is in line with 
external forces such as gravity. In case of a continuous brake,
the brake force can be directly controlled to achieve a certain 
resulting force. 
With a discrete brake, a tolerance region can be determined between  
which the brake switches on-and-off to achieve a similar effect.
Section \ref{sec:continuous_control} utilises this concept to drop the
payload without driving the motor. The brakes can also be used to stop 
the natural dynamics, if necessary.
\begin{table*}
    \centering
    \setlength{\tabcolsep}{3.5mm}{
    \renewcommand{\arraystretch}{0.95}
    \begin{small}
        \resizebox{\textwidth}{!}{
            \begin{tabular}{llcccccccc}
                \specialrule{1.1pt}{0pt}{1pt}
                                            &                       & \multicolumn{4}{c}{RICO}   & \multicolumn{4}{c}{PubLayNet}                                                                                                                                                                                 \\ \cmidrule(l){3-6} \cmidrule(l){7-10}
                Subtasks                    & \makecell[c]{Methods} & mIoU ($\uparrow$)           &   Overlap ($\rightarrow$)              & Align. ($\rightarrow$)        & FID ($\downarrow$)& mIoU ($\uparrow$)            &   Overlap ($\rightarrow$)            & Align. ($\rightarrow$)        & FID ($\downarrow$)      \\ \midrule
                \multirow{8}{*}{Un-Gen}      
                                            
                                            & LayoutTransformer     &0.587 & \underline{0.542} & {0.037} & 24.320   & 0.359 & \underline{0.0045} & 0.067 & 30.048      \\
                                            & VTN                   &0.336 & 0.561 & 0.477 & 88.115     & 0.312 & 0.221  & 0.207 & 105.909     \\ 
                                            &Coarse2Fine           & 0.360 & 0.676 & \underline{0.128} & 46.483& 0.361 & 0.142  & 0.221 & 50.854     \\ 
                                            &LayoutFormer++           & \underline{0.634} & 0.546 & 0.051 & 20.198  & 0.401 & {0.0010} & {\textbf{0.028}} & 47.082  \\ \cmidrule{2-10}
                                            
                                            &Diffusion-LM$^\diamond$    & {\textbf{0.662}} & 0.631 & 0.184 & 11.448  & {\textbf{0.439}} & 0.0125 & 0.076 & 11.895 \\
                                            &D3PM (absorbing)$^\diamond$ &0.585 & 0.619 & 0.157 & \underline{4.985}  & 0.401 & 0.0427 & 0.075 & 12.218  \\
                                            &D3PM (uniform)$^\diamond$   &0.595 & 0.658 & 0.229 & 5.576   & 0.405 & 0.0571 & 0.099 & \underline{11.212}  \\
                                            &LayoutDiffusion$^\diamond$ (ours)
                                            &0.620 & {\textbf{0.502}} & {\textbf{0.069}} & {\textbf{2.490}}   & \underline{0.417} & {\textbf{0.0030}} & \underline{0.065} & {\textbf{8.625}}  \\
                                             \midrule

                \multirow{8}{*}{Gen-Type}   
                                            & NDN-none   & \underline{0.35}  & 0.55  & 0.56  & 13.76     & 0.31  & 0.17  & 0.35  & 35.67     \\ 
                                            & LayoutGan++                  & 0.298 & 0.620 & 0.261 & 5.954     & 0.297 & 0.148 & 0.124 & 14.875   \\ 
                                            & BLT           & 0.216 & 0.983 & 0.150 & 25.633    & 0.140 & 0.196 & 0.036 & 38.684    \\ 
                                            & LayoutFormer++           & {\textbf{0.377}} & \underline{0.537} & \underline{0.124} & \underline{2.483}   & 0.333 & \underline{0.009} & {\textbf{0.025}} & 10.151    \\ \cmidrule{2-10}
                                            & Diffusion-LM$^\diamond$    & 0.324 & 0.574 & 0.199 & 6.530   & 0.316 & 0.026 & 0.046 & \underline{7.396}        \\
                                            & D3PM (absorbing)$^\diamond$ & 0.337 & 0.594 & 0.192 & 3.506  & 0.333 & 0.058 & 0.057 & 8.858 \\
                                            & D3PM (uniform)$^\diamond$   & 0.317 & 0.621 & 0.218 & 5.771  & {\textbf{0.351}} & 0.063 & 0.064 & 8.275      \\
                                            
                                            & LayoutDiffusion$^\diamond$ (ours) & {0.345} & {\textbf{0.491}} & {\textbf{0.124}} & {\textbf{1.557}}   & \underline{0.343} & {\textbf{0.005}} & \underline{0.029} & {\textbf{3.731}} \\
                                             \midrule 


                \multirow{5}{*}{Refinement}  
                                            & RUITE      &\underline{0.658} & \underline{0.492} & 0.177 & 7.926 & 0.637 & 0.0375 & 0.073 & 7.890  \\ %
                                            & LayoutFormer++           & 0.656 & 0.503 & \underline{0.141} & \underline{3.666} & \underline{0.642} & \underline{0.0126} & \underline{0.042} & \underline{2.937}  \\ \cmidrule{2-10}
                                            & Diffusion-LM$^\diamond$    & 0.621 & 0.499 & 0.181 & 8.578 & 0.573 & 0.0408 & 0.140 & 13.985 \\
                                            & D3PM (uniform)$^\diamond$   & 0.568 & 0.526 & 0.266 & 8.407 & 0.564 & 0.0515 & 0.110 & 14.553 \\
                                            & LayoutDiffusion$^\diamond$ (ours) &{\textbf{0.719}} & {\textbf{0.469}} & {\textbf{0.102}} & {\textbf{0.549}} & {\textbf{0.660}} & {\textbf{0.0079}} & {\textbf{0.035}} & {\textbf{2.045}} \\
                                            \midrule

                
              Real Data  &&-& 0.466 & 0.093 & -  & - & 0.0031 & 0.022 & -     \\
                \specialrule{1.1pt}{1pt}{0pt}
            \end{tabular}}
    \end{small}}
     \vspace{-5px}
        \caption{Quantitative results. Methods with $\diamond$ are diffusion-based, which achieve conditional generation (i.e., Gen-Type and Refinement) in a \textit{plug-and-play} manner, while other methods require re-training for each subtask. The best and the second best values of each metric are \textbf{bold} and \underline{underlined} respectively. For mIoU, the higher the score, the better the performance (indicated by $\uparrow$). For Overlap and Align, the closer to real data, the better (indicated by $\rightarrow$). For FID, the lower the score, the better the performance (indicated by $\downarrow$).
        }
    \label{Tab:quantitative_results}
 \vspace{-15px}
\end{table*}








                



\section{Experiments}
\label{sec:exp}

In this section, we demonstrate the wide range of applications and the high capabilities of Uni-Fusion. 
First, we evaluate Uni-Fusion in application 1) Incremental surface and color reconstruction, comparing its performance with SOTAs.
%
For applications 2) and 5), which are new topics, no specific benchmarks are available. 
Therefore, we showcase the performance on existing results.
%
Next, we implement application 3) and compare it with SOTA zero-shot semantic segmentation models.
%
Finally, for application 4), since infrared data is not commonly used, we collect our own dataset containing infrared values and show all applications on this data.

\subsection{Implementation Details}
\label{sec:exp:details}

In the experiments, we use our sample-based GPIS for local geometry encoding.
For each point, two additional points are sampled along normal direction, one positive and one negative, with distance $d_s=0.1$ in the local voxel's normalized space. 
Compared to derivative-based GPIS, our sample-based GPIS is more efficient in both space and time. 
For the encoder, we randomly sample $256$ anchor points from the range $[-0.5,0.5]^3$.
We utilize the first $20$ eigenpairs, resulting in a feature dimension of $20$.
The model selection process is discussed in the ablation study.

Different latent maps use different granularities.
For the surface LIM, we use a voxel size of $5\si{\centi\meter}$. 
For color which requires later comparison to NeRF, we use a voxel size of $2\si{\centi\meter}$.
For other property LIM and feature LIM, we use a voxel size of $10\si{\centi\meter}$.

For smooth reconstruction, the encoded voxel is designed overlapped following~\cite{huang2021di}.
The encoded voxel uses twice the voxel size, resulting in a half-space overlap with each neighboring voxel.
During meshing, SDFs are retrieved and interpolated from the overlapped voxels~\cite{huang2021di}.
While for the remaining properties, we sample only from its own voxel part.

The implementation runs on a PC with AMD Ryzen 9 5950X 16-core CPU and an Nvidia Geforce RTX 3090 GPU (24 GB).

\subsection{Datasets}

We evaluate incremental reconstruction on the ScanNet dataset~\cite{dai2017scannet}, TUM RGB-D dataset~\cite{sturm2012benchmark}, and Replica dataset~\cite{sucar2021imap}.
Using MSG-Net~\cite{zhang2018multi}'s material set, we transfer styles to the 3D canvas.
For open-vocabulary scene understanding, we evaluate on ScanNet segmentation data~\cite{qi2017pointnet++} and S3DIS dataset~\cite{armeni20163d}.

\subsubsection{ScanNet~\cite{dai2017scannet}}

ScanNet is a densely annotated RGB-D video dataset.
It is captured with the structure sensor~\cite{occipital} and contains 1513 scenes for training and validation.
For each scene, both images and a 3D mesh is provided, along with their 2D and 3D semantic annotations. 

ScanNet provides 312 scenes for validation, which contains a wide range of different room structures.
It has now been widely used in the thorough evaluation of the performance of reconstruction and semantic segmentation.

\subsubsection{TUM RGB-D~\cite{sturm2012benchmark}}

TUM RGB-D is a benchmark to mainly evaluate the tracking performance.
It is captured with Microsoft Kinect sensor together with ground-truth trajectory from the sensor.

\subsubsection{Replica~\cite{sucar2021imap}}

The Replica dataset refers to iMAP's pre-processed dataset~\cite{sucar2021imap}.
It is a synthetic rendered RGB-D dataset from given 3D models.
The advantage of including this dataset is that Replica does not have motion blur. 
This is better to evaluate the capability of the algorithms on reconstructing surface color.

\subsubsection{MSG-Net Style~\cite{zhang2018multi}}

MSG-Net provides material images for transfering the styles.
We select 21style fold for demonstration.
These images are given in \cref{fig:style} together with our result.

\subsubsection{ScanNet Point Cloud Segmentation Data~\cite{qi2017pointnet++}}

For point cloud semantic segmentation benchmarking, PointNet++~\cite{qi2017pointnet++} preprocesses the original ScanNet~\cite{dai2017scannet} and generates subsampled point clouds and corresponding annotations for each scene.

\subsubsection{S3DIS~\cite{armeni20163d} and 2D-3D-S~\cite{armeni2017joint}}

S3DIS is a semantic segmentation dataset for 3D point clouds.
Which is also a subset of the 2D-3D-S dataset.
The 2D-3D-S dataset is a multi-modality dataset containing 2D, 2.5D and 3D domains. 
This dataset is densely annotated with semantic labels.

Note that 2D-3D-S's 2D captures is not a RGB-D video as ScanNet.
2D-3D-S's images only have small overlap. 
Therefore, it is only suitable for semantic segmentation and not for incremental reconstruction.

\subsection{Baselines}

For online surface mapping evaluation, we select TSDF-Fusion~\cite{curless1996volumetric}, iMAP~\cite{sucar2021imap}, SOTA DI-Fusion~\cite{huang2021di} and BNV-Fusion~\cite{li2022bnv} as four baseline methods.

For the color field, we choose TSDF-Fusion~\cite{curless1996volumetric}, $\sigma$-Fusion~\cite{rosinol2023probabilistic}, iMAP~\cite{sucar2021imap}, NICE-SLAM~\cite{zhu2022nice} and even the recent hot Neural Radiance Fields model NeRF-SLAM~\cite{rosinol2022nerf} as five baselines.
While including NeRF in the comparison may not be entirely fair, we want to show how Uni-Fusion narrows the performance gap.

For the scene understanding application, we evaluate generalized zero-shot point cloud semantic segmentation with ZSLPC~\cite{cheraghian2019zero}, DeViSe~\cite{frome2013devise} and SOTA 3DGenZ~\cite{michele2021generative} for comparison.

\subsection{Metrics}

For incremental reconstruction, we evaluate the geometric reconstruction using \textbf{Accuracy}, \textbf{Completeness}, and \textbf{F1 score} according to SOTA BNV-Fusion. It firstly uniformly samples $100,000$ points from the reconstruction and ground truth meshes respectively.
Then \textbf{Accuracy} (\textbf{Completeness}) measures the percentage of reconstruction-to-groundtruth (groundtruth-to-reconstruction) distances that are lower than $2.5\si{\centi\meter}$ threshold. \textbf{F1 score} is the harmonic mean of accuracy and completeness.
For tracking performance, we use \textbf{ATE RMSE}.

To evaluate color reconstruction, we follow SOTA on this topic, NeRF to render both depth and RGB images to compare the image level \textbf{Depth L1} and \textbf{RGB PSNR}.

To compare scene understanding, we follow zero-shot point cloud semantic segmentation SOTA 3DGenZ to evaluate the \textbf{Intersection-of-Union (IoU)} and \textbf{Accuracy}.


\subsection{Reconstruction Results}

For evaluation, we first use the ScanNet validation set with 312 sequences to thoroughly test the geometric reconstruction on a large variant of scenes.
%
Then, we use TUM RGB-D to compare our modified tracking model with related works.
Because this part is not the main contribution, we give a rough overview of the tracking results.
%
To fairly evaluate the color reconstruction, we use the high quality rendered Replica dataset to compare with related works, including NeRF.

%\subsubsection{Object}
% on instance-gp
% Objective data usually has more fine detail
% 1. for detail precision
% A: no, object reconstruction is not as good as instance-ngp, so cancelled.
\begin{table*}[!]
	\centering
	\caption{Comparison to ScanNet~\cite{dai2017scannet}.
       Our method generalizes better to various scenes.
       $^*$ indicates the result from our runs of the official BNV-Fusion code.}
	\small
	%\setlength{\tabcolsep}{5mm}
	\setlength{\tabcolsep}{0.9em}
		%\resizebox{\textwidth}{!}{
		\begin{tabular}{l  c c c| c c c }
			\toprule
			Method & \begin{tabular}{@{}c@{}}Pre-Train\\ with extra dataset\end{tabular} & \begin{tabular}{@{}c@{}}Train \\ with sequences\end{tabular} & Real-time & Accuracy (\%) $\uparrow$ & Completeness (\%) $\uparrow$ & F1 score $\uparrow$ \\
			\midrule
			TSDF Fusion~\cite{zhou2018open3d} & None & None & $\checkmark$ &73.83 & 85.85 & 78.84 \\
			iMAP~\cite{sucar2021imap} & None & Online train& &68.96 & 82.12 & 74.96 \\
			DI-Fusion~\cite{huang2021di} &Object Pretrain & None & $\checkmark$&66.34 & 79.65 & 72.97 \\
			BNV-Fusion~\cite{li2022bnv} &Object Pretrain &  Post Optimization& &{74.90} & \textbf{88.12} & {80.56} \\
			BNV-Fusion$^{*}$~\cite{li2022bnv} &Object Pretrain & Post Optimization &&{73.42} & {81.75} & {77.18} \\
			\textbf{Uni-Fusion (Ours)} &None &None &$\checkmark$&\textbf{80.43} & {84.91} & \textbf{82.44} \\
			\bottomrule
		\end{tabular}
	  %}
	\label{tab:scannet}
	\vspace{-.6cm}
\end{table*}
\begin{figure*}[t]
	\subfloat[width=.33\textwidth][Accuracy]{
		\centering
		\includegraphics[width=.22\linewidth]{im/exp/recons/scannet/scannet_acc.png}
		\includegraphics[width=.1\linewidth]{im/exp/recons/scannet/scannet_acc_box.png}
	}
	\subfloat[width=.33\textwidth][Completeness]{
		\centering
		\includegraphics[width=.22\linewidth]{im/exp/recons/scannet/scannet_comp.png}
		\includegraphics[width=.1\linewidth]{im/exp/recons/scannet/scannet_comp_box.png}
	}
	\subfloat[width=.33\textwidth][F1 score]{
		\centering
		\includegraphics[width=.22\linewidth]{im/exp/recons/scannet/scannet_F1.png}
		\includegraphics[width=.1\linewidth]{im/exp/recons/scannet/scannet_F1_box.png}
	}
	\label{fig:recon:scannet:elementwise}
	\caption{Quantitative comparison on 312 scenes of the ScanNet validation set.
       We demonstrate the performance of SOTA BNV-Fusion and our Uni-Fusion.
       We sort our evaluation value and reordered all of the scores.
       The zigzag pink is the BNV-Fusion result;
       we also plot a deep-pink smoothed curve for better visualization.}
\end{figure*}

\subsubsection{Evaluation on ScanNet Dataset~\cite{dai2017scannet}}
\label{sec:exp:scannet}

We use the 312 diversified scenes from the ScanNet validation set to evaluate the performance of surface reconstruction. 
We follow the pure mapping SOTA BNV-Fusion to take every 10th posed frame as input. 
%
Without using any learning (in contrast iMAP, DI-Fusion, and BNV-Fusion do) or any post optimization (as BNV-Fusion does), our Uni-Fusion is capable to achieve precise continuous mapping performance. 

As shown in~\cref{tab:scannet}, our Uni-Fusion achieves \textbf{$+6$ higher accuracy} than the incremental surface reconstruction SOTA BNV-Fusion.
Our model does not exceed on completeness comparing to BNV-Fusion that support completion in post-optimization.
Though, Uni-Fusion's completion is still much higher than one other optimization based model iMAP.
%We consider it because our model does not support hole-completion as the optimization based models iMap and BNV-Fusion.
Overall, our Uni-Fusion model achieves higher F1-scores that quantifies the overall quality.

Please note that, SOTA BNV-Fusion is not real-time capable, since it requires post optimization of all fed frames.
Without the post-optimization, the real-time model Di-Fusion shows much worse results.
However, our \textbf{real-time} model \textbf{Uni-Fusion} is able to achieves \textbf{much better} reconstruction quality than these approaches even without post-optimization. 

\newcommand{\scannetImSize}{.16}
\begin{figure*}[t!]
	\centering
	\setlength{\tabcolsep}{0.1em}
	\renewcommand{\arraystretch}{.1}
	\begin{tabular}{|c | c |c |||c |c | c|}
		\hline
		{\Large{BNV-Fusion}} & {\Large{Uni-Fusion}} &{\Large{Ground Truth}} & {\Large{BNV-Fusion}} &{\Large{Uni-Fusion}} & {\Large{Ground Truth}} \\ \hline \hline
		
\includegraphics[width=\scannetImSize\linewidth]{im/exp/recons/scannet_qualifi/scene0568_00_bnv.png}
		&\includegraphics[width=\scannetImSize\linewidth]{im/exp/recons/scannet_qualifi/scene0568_00_mine.png}
		&\includegraphics[width=\scannetImSize\linewidth]{im/exp/recons/scannet_qualifi/scene0568_00_gt.png}
		&		\includegraphics[width=\scannetImSize\linewidth]{im/exp/recons/scannet_qualifi/scene0164_00_bnv.png}
		&\includegraphics[width=\scannetImSize\linewidth]{im/exp/recons/scannet_qualifi/scene0164_00_mine.png}
		&\includegraphics[width=\scannetImSize\linewidth]{im/exp/recons/scannet_qualifi/scene0164_00_gt.png}\\
		
\includegraphics[width=\scannetImSize\linewidth]{im/exp/recons/scannet_qualifi/scene0249_00_bnv.png}
		&\includegraphics[width=\scannetImSize\linewidth]{im/exp/recons/scannet_qualifi/scene0249_00_mine.png}
		&\includegraphics[width=\scannetImSize\linewidth]{im/exp/recons/scannet_qualifi/scene0249_00_gt.png}
		&		\includegraphics[width=\scannetImSize\linewidth]{im/exp/recons/scannet_qualifi/scene0435_00_bnv.png}
		&\includegraphics[width=\scannetImSize\linewidth]{im/exp/recons/scannet_qualifi/scene0435_00_mine.png}
		&\includegraphics[width=\scannetImSize\linewidth]{im/exp/recons/scannet_qualifi/scene0435_00_gt.png}\\
		
\includegraphics[width=\scannetImSize\linewidth]{im/exp/recons/scannet_qualifi/scene0046_00_bnv.png}
		&\includegraphics[width=\scannetImSize\linewidth]{im/exp/recons/scannet_qualifi/scene0046_00_mine.png}
		&\includegraphics[width=\scannetImSize\linewidth]{im/exp/recons/scannet_qualifi/scene0046_00_gt.png}
		&		\includegraphics[width=\scannetImSize\linewidth]{im/exp/recons/scannet_qualifi/scene0050_00_bnv.png}
		&\includegraphics[width=\scannetImSize\linewidth]{im/exp/recons/scannet_qualifi/scene0050_00_mine.png}
		&\includegraphics[width=\scannetImSize\linewidth]{im/exp/recons/scannet_qualifi/scene0050_00_gt.png}\\
		\hline
	\end{tabular}
	%\captionof{figure}
	\caption{Surface reconstruction on ScanNet dataset.}
	\label{fig:recons:scannet_demo}
	\vspace{-.5cm}
\end{figure*}

We additionally run BNV-Fusion's official implementation (emphasized with $^*$) on the 312 videos of ScanNet and do a post element-wise comparison in \cref{fig:recon:scannet:elementwise}. 
Our result is the {\color{Cyan}light blue} curve, BNV-Fusion's result is colored with {\color{Lavender}pink}.
Scene index is sorted corresponding to the score value of Uni-Fusion.
For better visualization, we smooth BNV-Fusion's curve and plot it with dark pink.
It is obvious that the score of Uni-Fusion is overall higher than BNV-Fusion's. 
Moreover, we use box-plot to conclude the statistics besides the curve plot. Uni-Fusion's scores are distributed in a higher region. For completeness which is less obvious better, Uni-Fusion's box is smaller while in a relative higher position. This means that Uni-Fusion has more stable completeness result while BNV-Fusion is more likely to get low completeness in some cases.

To summarize, our model is almost better on all 312 scenes on all accuracy, completeness and F1-score.
Which is also revealed in \cref{tab:scannet} with BNV-Fusion$^*$, that the BNV-Fusion's official implementation does not exceed Uni-Fusion on all metrics.

We plot reconstruction on selected scenes from ScanNet in~\cref{fig:recons:scannet_demo}. 
Both BNV-Fusion and our Uni-Fusion are able to produce high quality reconstruction.
We see that BNV-Fusion gives lots of small meshes on walls, which are shown as small particles in the reconstruction. 
We consider it is because BNV-Fusion use very small voxel size ($0.02\si{\meter}$) to get a high score.
This is also revealed by their \textbf{\SI{247}{MB}} mesh in average, while ours is only \textbf{\SI{54}{Mb}} in average.
Furthermore, our Uni-Fusion's mesh is more smooth and
%Both BNV-Fusion and Uni-Fusion demonstrate high quality result.
also provides high-precise color to the mesh which is not available for the Surface SOTA.

%In this test, we purely evaluate the surface reconstruction capacity with SOTAs. 
%While reconstruction is not merely surface.  
%Thus in the following, we find benchmarks for both surface and color.


\subsubsection{Tracking Evaluation on TUM RGB-D Dataset~\cite{sturm2012benchmark}}
% follow nice-slam

In the above test, we compare the performance of pure mapping.
While tracking is not the contribution focus in our paper, it is part of the reconstruction model. We follow the novel reconstruction model NICE-SLAM~\cite{zhu2022nice} to evaluate the camera tracking on the small-scale TUM RGB-D dataset.
Our Uni-Fusion uses a coarse-to-fine strategy for 3D reconstruction tracking.
From~\cref{tab:tum_rmse}, it demonstrates overall better ATE RMSE than other implicit representation models.

\begin{table}[]
		\caption{Tracking on TUM RGB-D~\cite{sturm2012benchmark}.
		ATE RMSE [$\si{\centi\meter}$] ($\downarrow$) is used as the evaluation metric.
	}
	\centering
	\footnotesize
	\setlength{\tabcolsep}{0.7em}
	\resizebox{\linewidth}{!}{
		\begin{tabular}{l|ccc}
			\hline
			& \tt{fr1/desk} &  \tt{fr2/xyz} &  \tt{fr3/office} \\
			
			\hline
			{iMAP}~\cite{sucar2021imap}      & 4.9 & 2.0 & 5.8  \\
			{iMAP$^*$}~\cite{sucar2021imap} & 7.2 & 2.1  & 9.0 \\
			{DI-Fusion~\cite{huang2021di}} & 4.4 & 2.3 & 15.6 \\
			NICE-SLAM~\cite{zhu2022nice}           & 2.7 & 1.8 & 3.0 \\
			Ours& 1.8& 0.5& 2.1 \\
			\hline
			{BAD-SLAM}\cite{schops2019bad} & 1.7  & 1.1  & 1.7 \\
			{Kintinuous}\cite{whelan2012kintinuous} & 3.7  &  2.9  & 3.0 \\
			{ORB-SLAM2}\cite{mur2017orb} & \bf 1.6  & \bf 0.4  & \bf 1.0 \\
			\hline
	\end{tabular}}
	\vspace{2pt}

	\label{tab:tum_rmse}
\end{table}

On the other hand, there also exist high accuracy algorithms from SLAM. 
By additional using Bundle Adjustment and Loop-closing techniques, their tracking quality is much better than all of the implicit based models.

%But it is dangerous to directly apply SLAM result on reconstruction. Please find our demonstration in Fig [?]. Which explains the more widely used frame-to-model strategy in 3D reconstruction.
Even though, our coarse-to-fine strategy firstly ensure not easy to lose track. Secondly, it is more suitable for surface fitting.

Which further support our test in Replica dataset.



\begin{table*}[t!]
	\centering
	\caption{Geometric (L1) and Photometric (PSNR) evaluation on the Replica dataset~\cite{sucar2021imap}.}
	\footnotesize
	\setlength{\tabcolsep}{0.36em}
	\renewcommand{\arraystretch}{1.2}
	\begin{tabular}{clcccccccccccccccccc}
		\toprule
		& & \multicolumn{1}{c}{\makecell{\tt{office-0}}} & \multicolumn{1}{c}{\makecell{\tt{office-1}}} & \multicolumn{1}{c}{\makecell{\tt{office-2}}}& \multicolumn{1}{c}{\makecell{\tt{office-3}}} & \multicolumn{1}{c}{\makecell{\tt{office-4}}} & \multicolumn{1}{c}{\makecell{\tt{room-0}}} & \multicolumn{1}{c}{\makecell{\tt{room-1}}} &  \multicolumn{1}{c}{\makecell{\tt{room-2}}} & Avg. \\
		\midrule
		\multicolumn{5}{l}{\textit{Non-continuous mapping method}}\\
		\multirow{2}{*}{\makecell{\textbf{TSDF-Fusion}~\cite{curless1996volumetric}}}
		& {\bf Depth L1} [$\si{\centi\meter}$] $\downarrow$
	 & 14.11 & 10.50 & 30.89 & 28.92 & 22.83	& 23.51 & 20.94 & 23.34 & 21.88 \\
		& {\bf PSNR } [$\si{\dB}$] $\uparrow$
		& 11.16 & 15.92 & 4.86 & 5.68 & 5.46 & 3.43 & 4.51 & 5.57 & 7.07 \\
		
		\midrule
		\multirow{2}{*}{\makecell{\textbf{$\sigma$-Fusion}\cite{rosinol2023probabilistic} }}
		& {\bf Depth L1} [$\si{\centi\meter}$] $\downarrow$
		 & 13.80 & 10.21 & 22.27 & 28.70 & 22.21& 21.92 & 19.28 & 22.40 & 20.10 \\
		& {\bf PSNR } [$\si{\dB}$] $\uparrow$
		 & 11.16 & 15.92 & 4.86 & 5.69 & 5.46& 3.45  & 4.51 & 5.57 & 7.08 \\
		
		
		
		
		
		\midrule
		\midrule
		\multicolumn{5}{l}{\textit{Continuous mapping method}}\\
		\multirow{2}{*}{\makecell{\textbf{iMAP$^*$}~\cite{sucar2021imap}}}
		& {\bf Depth L1} [$\si{\centi\meter}$] $\downarrow$
		 & 6.43 & 7.41 & 14.23 & 8.68 & 6.80& 5.70 & 4.93 & 6.94 & 7.64\\
		& {\bf PSNR } [$\si{\dB}$] $\uparrow$
		& 7.39 & 11.89 & 8.12 & 5.62 & 5.98& 5.66 & 5.31 & 5.64  & 6.95\\
		\midrule
		\multirow{2}{*}{{\makecell{\textbf{Nice-SLAM}~\cite{zhu2022nice} }}}
		& {\bf Depth L1} [$\si{\centi\meter}$] $\downarrow$
		& { 1.51 } & { 0.93 } & { 8.41 } & { 10.48 } & {2.43} & { 2.53 } & { 3.45 } & { 2.93 }  & { 4.08 } \\
		& {\bf PSNR } [$\si{\dB}$] $\uparrow$
		 & { 22.44 } & { 25.22 } & { 22.79 } & { 22.94 } & { 24.72 } & \textbf{ 29.90 } & \textbf{ 29.12 } & { 19.80 }& { 24.61 } \\
		
		
		
		
		\midrule	
		\multirow{2}{*}{{\makecell{\textbf{Uni-Fusion} (Ours) }}}
		% using abs(diff)
		%	& {\bf Depth L1} [$\si{\centi\meter}$] $\downarrow$ &\textbf{1.98}&\textbf{1.18}&\textbf{1.64}&\textbf{1.23}&\textbf{0.84}&\textbf{1.61}&\textbf{3.01}&\textbf{1.60} &\textbf{1.64}
		% follow nerf-slam to remove outlier gt first
		& {\bf Depth L1} [$\si{\centi\meter}$] $\downarrow$
		& \textbf{0.79}&\textbf{0.56}&\textbf{1.59}&\textbf{2.71}&\textbf{1.66}&\textbf{1.94}&\textbf{0.69}&\textbf{1.80}& \textbf{1.47}
		\\
		& {\bf PSNR } [$\si{\dB}$] $\uparrow$ &\textbf{33.88}&\textbf{33.31}&\textbf{25.84}&\textbf{26.01}&\textbf{28.14}&24.02&26.20&\textbf{27.17} &\textbf{28.07}
		\\
		
		\midrule
		\midrule
		\multicolumn{5}{l}{\textit{Neural radiance field method}}\\
		\multirow{2}{*}{{\makecell{\textbf{NeRF-SLAM}~\cite{rosinol2022nerf} }}}
		& {\bf Depth L1} [$\si{\centi\meter}$] $\downarrow$
	 & {2.49}   & {1.98}  & {9.13}  & {10.58} & {3.59}	& {2.97}  & {2.63}  & {2.58}  & {4.49} \\
		& {\bf PSNR } [$\si{\dB}$] $\uparrow$
	 & \textbf{48.07}  & \textbf{53.44} & \textbf{39.30} & \textbf{38.63} & \textbf{39.21} 	& \textbf{34.90} & \textbf{36.95} & \textbf{40.75}& \textbf{41.40} \\
		
		\bottomrule
	\end{tabular}%
	
	\label{tab:replica_per_scene}
\end{table*}


\begin{table*}[t!]
	\centering
	\caption{Differences among different Surface \& Color reconstruction models.}
	\small
	\setlength{\tabcolsep}{.6em}
	%{
		%\resizebox{\textwidth}{!}{
			\begin{tabular}{l | c c c c c c }
				\toprule
				Method & 
				\begin{tabular}{@{}c@{}}Pre-Train\\ with extra dataset\end{tabular}
				& \begin{tabular}{@{}c@{}}Train\\ with sequences\end{tabular}
				& Real-time	
				& Direct Output &  \begin{tabular}{@{}c@{}}Light\\ direction\end{tabular} 
				&Render\\
				\hline 
				TSDF-Fusion & None & None & $\checkmark$& Discrete TSDF &  &Ray Rasterization\\\hline
				$\sigma$-Fusion & None & None &$\checkmark$&Discrete TSDF  && Ray Rasterization\\\hline
				iMAP & None & Online Train && MLPs  & &Volumetric Rendering\\\hline
				NICE-SLAM & \begin{tabular}{@{}c@{}}Pretrain\\ with indoor dataset\end{tabular} & Online Train&& Neural Implicit Grid&  & Volumetric Rendering\\\hline
				
				NeRF-SLAM & None & Train hundred epochs &-&NeRF & $\checkmark$ &Volumetric Rendering \\\hline
				
				\textbf{Uni-Fusion} & None & None&$\checkmark$& Latent Implicit Map && Ray Rasterization\\				
				\hline
			\end{tabular}
		%}
	%}
	\label{tab:replica_diff}
\end{table*}
\subsubsection{Evaluation on Replica RGB-D Dataset~\cite{sucar2021imap}}
In this evaluation, we compare with implicit reconstruction (TSDF-Fusion, $\sigma$-Fusion) and latent implicit reconstruction models (iMAP, NICE-SLAM) that support colors. 
We also add a large-scale NeRF model, NeRF-SLAM in to the table.  
NeRF is SOTA in view-synthesis task, which is unfair to direct compare with the rest. As the rest model does not even model light directions.
We add NeRF in this part to demonstrate that Uni-Fusion strongly reduce the gap.
Note that, NeRF-SLAM embeds external tracking model ~\cite{teed2021droid} to provide poses while using SOTA NeRF implementation Instance-ngp~\cite{muller2022instant} for NeRF construction.
%Therefore it is considered the SOTA to model the colors.

Uni-Fusion track and follow our previous setting in ScanNet test to take every 10 frames for mapping.
NICE-SLAM and NeRF-SLAM produce depth and color by rendering,
To obtain result from Uni-Fusion, we cast rays from virtual camera to our result surface mesh for depth image. 
Then Uni-Fusion infer the cast points in Uni-Fusion's color LIM for color result.

From~\cref{tab:replica_per_scene}, Uni-Fusion demonstrate
best Depth L1 on all scenes with an average of \textbf{$\pmb{1.47}$$\si{\centi\meter}$ depth L1}. Which is \textbf{$\pmb{177\%}$ boost} comparing to the second best.

Moreover, excluding NeRF, our Uni-Fusion also shows the best performance to model the colors with an average of $28.07$$\si{\dB}$ PSNR.

However, it is strange that NICE-SLAM lost details while in two cases, it shows better PSNR than Uni-Fusion. 
To highlight the true result,
we plot the rendered image in \cref{fig:replica_render}.
It is obvious that our Uni-Fusion models the details of painting, carpet and quilt well. 
While NICE-SLAM just roughly models the average color.

Moreover, from the  \cref{fig:replica_render}, our Uni-Fusion's rendering quality is as precise as NeRF. 
Please also find the painting, carpet and quilt, Uni-Fusion recovered the original appearance well.
Please find the {\color{green} green window} for the emphasized region.
Uni-Fusion reproduce the high-quality appearance which is very close to NeRF on qualitative evaluation.
%It can hardly find difference between the results from NeRF-SLAM, Uni-Fusion and Ground Truth.
%
But, Uni-Fusion still has a quantitative score gap to the color rending of NeRF ($41.4$$\si{\dB}$).
Though the Uni-Fusion's rendering result is highly close to NeRF and ground truth.
%
We consider the main reasons are that \textbf{1.} Uni-Fusion does not model the light directions to points, which is essential to NeRF.
\textbf{2.} NeRF optimizes on the rendering image quality by focussing mainly on color while less on depth.
It can be revealed by the higher color rendering score with much worse depth rendering than our Uni-Fusion.
\textbf{3.} our Uni-Fusion does not support hole filling.
This directly leads to black holes in our rendered images.

We summarize the differences in \cref{tab:replica_diff}.
Similar to TSDF-Fusion and $\sigma$-Fusion, our Uni-Fusion is a forward method which, does not need any training, i.e., pre- or online training. 
Uni-Fusion also produces similar to NICE-SLAM and NeRF-SLAM an implicit map with set of latent that outputs results at arbitrary resolution.
However, we differ on the extracting of the signed distance field.
%FIXME: I do not understand the next sentence.
Uni-Fusion's latent feature rule its own region independently.
Each query value is directly inferred with the corresponding ruling latent.
While NICE-SLAM and NeRF-SLAM use a much denser grid to interpolate query features. This requires volumetric rendering for inference.

Similar to TSDF-Fusion, $\sigma$-Fusion, our Uni-Fusion is also a real-time algorithm.
iMAP, NICE-SLAM and NeRF-SLAM run hardly in real-time.
NeRF-SLAM is claiming to be real-time, which is questionable as they still need hundreds of epochs training after feeding the data.

Nevertheless, optimization with backpropagation learns pixel-to-pixel well.
It is theoretically advanced for a regression-and-fusion strategy. 
Though Uni-Fusion demonstrates its high capability to model the color, NeRF-like post-optimization would still be a good direction for further improvements of Uni-Fusion.

\newcommand{\replicaImSize}{.24}
\begin{figure*}[t]
	\centering
	\setlength{\tabcolsep}{0.1em}
	\renewcommand{\arraystretch}{.1}
	\begin{tabular}{|c | c |c |c| }
		 \hline
		{\Large{NICE-SLAM}} &{\Large{NeRF-SLAM}}&\textbf{\Large{Uni-Fusion}}&\Large{Ground Truth}\\
		%		\hline
		%		\includegraphics[width=\replicaImSize\linewidth]{im/exp/recons/replica/nice-slam/of2_1286.png} &
		%		\includegraphics[width=\replicaImSize\linewidth]{im/exp/recons/replica/mine/of2_1286.jpg} &
		%		\includegraphics[width=\replicaImSize\linewidth]{im/exp/recons/replica/mine/of2_1286.jpg} &
		%		\includegraphics[width=\replicaImSize\linewidth]{im/exp/recons/replica/gt/of2_1286.jpg} \\
		
		\hline
		\includegraphics[width=\replicaImSize\linewidth]{im/exp/recons/replica/nice-slam/rm0_769_window.png} &
		\includegraphics[width=\replicaImSize\linewidth]{im/exp/recons/replica/nerf-slam/rm0_769_window.jpg} &
		\includegraphics[width=\replicaImSize\linewidth]{im/exp/recons/replica/mine/rm0_769_window.jpg} &
		\includegraphics[width=\replicaImSize\linewidth]{im/exp/recons/replica/gt/rm0_769_window.jpg} \\
		\hline
		\includegraphics[width=\replicaImSize\linewidth]{im/exp/recons/replica/nice-slam/of3_575_window.png} &
		\includegraphics[width=\replicaImSize\linewidth]{im/exp/recons/replica/nerf-slam/of3_575_window.jpg} &
		\includegraphics[width=\replicaImSize\linewidth]{im/exp/recons/replica/mine/of3_575_window.jpg} &
		\includegraphics[width=\replicaImSize\linewidth]{im/exp/recons/replica/gt/of3_575_window.jpg} \\
		\hline
		\includegraphics[width=\replicaImSize\linewidth]{im/exp/recons/replica/nice-slam/rm1_425_window.png} &
		\includegraphics[width=\replicaImSize\linewidth]{im/exp/recons/replica/nerf-slam/rm1_425_window.jpg} &
		\includegraphics[width=\replicaImSize\linewidth]{im/exp/recons/replica/mine/rm1_425_window.jpg} &
		\includegraphics[width=\replicaImSize\linewidth]{im/exp/recons/replica/gt/rm1_425_window.jpg} \\
		\hline
		\includegraphics[width=\replicaImSize\linewidth]{im/exp/recons/replica/nice-slam/rm2_1085_window.png} &
		\includegraphics[width=\replicaImSize\linewidth]{im/exp/recons/replica/nerf-slam/rm2_1085_window.jpg} &
		\includegraphics[width=\replicaImSize\linewidth]{im/exp/recons/replica/mine/rm2_1085_window.jpg} &
		\includegraphics[width=\replicaImSize\linewidth]{im/exp/recons/replica/gt/rm2_1085_window.jpg} \\		
		
	\end{tabular}
	%\captionof{figure}
	\caption{Demonstration of color rendering on the Replica dataset. Fine appearances are highlighted in {\color{green}green window}. Small flaws are in a {\color{red}red} box.}
	\label{fig:replica_render}
	\vspace{-.5cm}
\end{figure*}

%(2) NeRF model learning radiance field that model the light on different direction on surface. While Uni-Fusion naturally treat different directional light the same color.

%\begin{table*}[t!]
%	\centering
%	\setlength{\tabcolsep}{0.1em}
%	\renewcommand{\arraystretch}{.1}
%	\begin{tabular}{c | c |c |c |c }
%		\hline 
%		\rotatebox{90}{\large{NICE-SLAM}} &
%		\includegraphics[width=\replicaImSize\linewidth]{im/exp/recons/replica/nice-slam/of3_575.png} &
%		\includegraphics[width=\replicaImSize\linewidth]{im/exp/recons/replica/nice-slam/rm0_769.png} &
%		\includegraphics[width=\replicaImSize\linewidth]{im/exp/recons/replica/nice-slam/rm1_425.png} &
%		\includegraphics[width=\replicaImSize\linewidth]{im/exp/recons/replica/nice-slam/rm2_1085.png} \\
%		\hline
%		\rotatebox{90}{\large{NeRF-SLAM}} &
%		\includegraphics[width=\replicaImSize\linewidth]{im/exp/recons/replica/mine/of3_575.jpg} &
%		\includegraphics[width=\replicaImSize\linewidth]{im/exp/recons/replica/mine/rm0_769.jpg} &
%		\includegraphics[width=\replicaImSize\linewidth]{im/exp/recons/replica/mine/rm1_425.jpg} &
%		\includegraphics[width=\replicaImSize\linewidth]{im/exp/recons/replica/mine/rm2_1085.jpg} \\	
%		\hline
%		\rotatebox{90}{\textbf{\Large{Uni-Fusion}}} &
%		\includegraphics[width=\replicaImSize\linewidth]{im/exp/recons/replica/mine/of3_575.jpg} &
%		\includegraphics[width=\replicaImSize\linewidth]{im/exp/recons/replica/mine/rm0_769.jpg} &
%		\includegraphics[width=\replicaImSize\linewidth]{im/exp/recons/replica/mine/rm1_425.jpg} &
%		\includegraphics[width=\replicaImSize\linewidth]{im/exp/recons/replica/mine/rm2_1085.jpg} \\
%		\hline
%		\rotatebox{90}{\large{Ground Truth}} &
%		\includegraphics[width=\replicaImSize\linewidth]{im/exp/recons/replica/gt/of3_575.jpg} &
%		\includegraphics[width=\replicaImSize\linewidth]{im/exp/recons/replica/gt/rm0_769.jpg} &
%		\includegraphics[width=\replicaImSize\linewidth]{im/exp/recons/replica/gt/rm1_425.jpg} &
%		\includegraphics[width=\replicaImSize\linewidth]{im/exp/recons/replica/gt/rm2_1085.jpg} \\
%		\hline		
%		
%	\end{tabular}
%	\captionof{figure}{Demonstration of color rendering on Replica dataset.}
%\end{table*}

\subsection{Ablation study}
\label{exp:surface:ablation}

\begin{figure}[]
	\centering
%		\subfloat[width=\textwidth][Sample based]{
%		\centering
%		\includegraphics[width=.7\linewidth]{im/exp/ablation/GPIS/seq3_sample_color.png}
%	}\\
%	\subfloat[width=\textwidth][Derivative based]{
%		\centering
%		\includegraphics[width=.7\linewidth]{im/exp/ablation/GPIS/seq3_derivative_color.png}
%	}
		\includegraphics[width=.7\linewidth]{im/exp/ablation/GPIS/seq3_sample_color_a.png}
		\includegraphics[width=.7\linewidth]{im/exp/ablation/GPIS/seq3_derivative_color_b.png}
	\caption{Ablation study on surface construction basis. (a) Sample based. (b) Derivative based.}
	\label{fig:ablation:GPIS}
\end{figure}


\begin{table}[]
	\caption{Ablation study on tracking.
	}
	\centering
	\footnotesize
	\setlength{\tabcolsep}{0.7em}
	\resizebox{\linewidth}{!}{
		\begin{tabular}{l|ccc}
			\hline
			& \tt{fr1/desk} &  \tt{fr2/xyz} &  \tt{fr3/office} \\
			\hline
			External& 2.1& 0.5& 2.5 \\
			External+Internal&1.8& 0.5& 2.1 \\
			\hline
	\end{tabular}}
	\vspace{-2pt}
	%\vspace{-1cm}
	\label{tab:tum_rmse2}
\end{table}

\begin{figure}
	\centering
	\includegraphics[width=.7\linewidth]{im/exp/ablation/voxel_size/seq_voxel_size.png}
	%	\subfloat[width=.33\textwidth][0.1]{
		%		\centering
		%		\includegraphics[width=.33\linewidth]{im/exp/ablation/voxel_size/seq3_0_1_color.png}
		%	}
	%	\subfloat[width=.3\textwidth][0.05]{
		%		\centering
		%		\includegraphics[width=.33\linewidth]{im/exp/ablation/GPIS/seq3_sample_color.png}
		%	}
	%	\subfloat[width=.33\textwidth][0.02]{
		%	\centering
		%	\includegraphics[width=.33\linewidth]{im/exp/ablation/voxel_size/seq3_0_02_color.png}
		%	}
	\caption{Ablation study on voxel size.}
	\label{fig:ablation:voxel_size}
\end{figure}




 \newcommand{\styleImSize}{.2}
\begin{figure*}[b!]
	\vspace{-.5cm}
	\centering
	\setlength{\tabcolsep}{0.1em}
	\renewcommand{\arraystretch}{.1}
	\resizebox{\textwidth}{!}{\begin{tabular}{ccccc}
			%		\includegraphics[width=\styleImSize\line]{im/exp/style/style/0} &
			%		\includegraphics[width=\styleImSize\linewidth]{im/exp/style/style/1} &
			%		\includegraphics[width=\styleImSize\linewidth]{im/exp/style/style/2} &
			%		\includegraphics[width=\styleImSize\linewidth]{im/exp/style/style/3} &
			%		\includegraphics[width=\styleImSize\linewidth]{im/exp/style/style/4} &
			%		\includegraphics[width=\styleImSize\linewidth]{im/exp/style/style/5} &
			%		\includegraphics[width=\styleImSize\linewidth]{im/exp/style/style/6} \\
			\hline\hline
			\includegraphics[width=\styleImSize\linewidth]{im/exp/style/processed/office0_0.png} &
			\includegraphics[width=\styleImSize\linewidth]{im/exp/style/processed/office0_1.png} &
			\includegraphics[width=\styleImSize\linewidth]{im/exp/style/processed/office0_2.png} &
			\includegraphics[width=\styleImSize\linewidth]{im/exp/style/processed/office0_3.png} &
			\includegraphics[width=\styleImSize\linewidth]{im/exp/style/processed/office0_4.png} \\
			\includegraphics[width=\styleImSize\linewidth]{im/exp/style/processed/office0_5.png} &
			\includegraphics[width=\styleImSize\linewidth]{im/exp/style/processed/office0_6.png} &
			\includegraphics[width=\styleImSize\linewidth]{im/exp/style/processed/office0_7.png} &
			\includegraphics[width=\styleImSize\linewidth]{im/exp/style/processed/office0_8.png} &
			\includegraphics[width=\styleImSize\linewidth]{im/exp/style/processed/office0_9.png} \\
			\includegraphics[width=\styleImSize\linewidth]{im/exp/style/processed/office0_10.png} &
			\includegraphics[width=\styleImSize\linewidth]{im/exp/style/processed/office0_11.png} &
			\includegraphics[width=\styleImSize\linewidth]{im/exp/style/processed/office0_12.png} &
			\includegraphics[width=\styleImSize\linewidth]{im/exp/style/processed/office0_13.png} &
			\includegraphics[width=\styleImSize\linewidth]{im/exp/style/processed/office0_14.png} \\
			\includegraphics[width=\styleImSize\linewidth]{im/exp/style/processed/office0_15.png} &
			%\includegraphics[width=\styleImSize\linewidth]{im/exp/style/processed/office0_16.png} &
			\includegraphics[width=\styleImSize\linewidth]{im/exp/style/processed/office0_17.png} &
			\includegraphics[width=\styleImSize\linewidth]{im/exp/style/processed/office0_18.png} &
			\includegraphics[width=\styleImSize\linewidth]{im/exp/style/processed/office0_19.png} &
			\includegraphics[width=\styleImSize\linewidth]{im/exp/style/processed/office0_20.png} 
			\\	\hline
		\end{tabular}
	}
	%\captionof{figure}
	\caption{Style transfer on 3D canvas.}
	\label{fig:style}
\end{figure*}


\subsubsection{ Sample-based or Derivative-based}

We select the surface model with our own captured sequences. 
All settings are detailed in \cref{sec:exp:details}.
As shown in~\cref{fig:ablation:GPIS}, reconstruction of Yijun's office is demonstrated. 
Both models are able to construct, but the derivative-based model produces a lot of noise on the surface.
This is because for smoothness purpose, we build voxels that are overlapped to its neighbor, which causes redundant voxels near the surface.
For those redundant voxels, no center sample is provided and thus the derivative based surface construction builds bad SDFs on unknow region of the voxels.


Instead, sample-based surface construction does not have this problem as it adds more points in voxels and is able to construct highly-smooth surfaces.
From which, we find well constructed and colored white board, chair, school bag and even the oranges.

\subsubsection{Tracking}


Our Uni-Fusion use a coarse-to-fine strategy for tracking. 
An external tracking model is running in one thread aside from the mapping thread.
In the mapping thread, it takes pose result from the external tracking and applies the internal tracking for colored point cloud.

The result is demonstrated in~\cref{tab:tum_rmse2}. 
The coarse-to-fine is relatively better on trajectory estimation.

\subsubsection{Voxel size}

Testing the office scene, we vary the voxel size from low to high. 
From~\cref{fig:ablation:voxel_size}, when low voxel size $0.02$m is used, the surface is rough.
Then when voxel size goes larger, the smoothness is improved.
However, when we use $0.1$m voxel size, the surface color is blur. 
Considering Uni-Fusion produces a surface color field, the quality of surface directly affect the coloring.
Thus, continuing enlarging the voxel size also results in worse surface results.

Therefore, in the above experiments, $0.05$m voxel size is utilized for surface construction.
In addition, each voxel for encoding are actually with size $0.1$m, since we use overlapped voxel.

%\subsubsection{Anchor number and feature dimension}


\subsection{Application: 2D-to-3D Transfer}
\label{sec:fabircated_prop}

Applications such as 2) and 4) can be easily integrated with application 1) incremental reconstruction (\cref{sec:incremental_reconstruction}) by incorporating the fabricated result together with the point cloud.
%
For instance, given RGB-D frames, we detect saliency or transfer image styles to generate a fabricated $X$ image. Here, $X$ represents saliency, style, or other properties. 
By combining $X$ with depth information through unprojection,
we assign
the fabricated values to corresponding points, resulting in point pairs ($\V X$, $\V Q_{X}$).

Similar to the reconstruction pipeline in~\cref{fig:recons_and_scene_understanding}, we employ encoding (\cref{sec:encoder}) and fusion (\cref{eq:fuse}) to construct a global LIM for the fabricated properties $X$.
This global LIM represents a surface $X$ fields that is utilized for subsequent inference.

While it is possible to similarly transfer a 2D semantic image to 3D,
it may not be feasible in practice due to the need for multiple passes of different categories of semantic information 
 on the same dataset (such as object, usability, etc.).
Therefore, in the following section, we demonstrate the construction of a surface feature field for scene understanding application that satisfies various 
requirements through a single mapping pass.

\begin{table*}[b!]
	%\renewcommand{\arraystretch}{0.9}
	%\setlength{\tabcolsep}{3pt}
	\caption{GZSL semantic segmentation results. Scores are in \%.
	  $^\dagger$ indicate 3DGenZ's adaption of the method.
       Note that, Uni-Fusion-SU does not even train with the seen classes.}
	\centering
	\begin{tabular}{l|c|c |c ||ccc|ccc}
		\toprule
		\multicolumn{1}{c}{}& \multicolumn{2}{c|}{Training set} & Inference input &\multicolumn{3}{c|}{ScanNet } & \multicolumn{3}{c}{S3DIS}\\
		& Backbone & Classifier & &$Seen$& $Unseen$ & $All$&$Seen$& $Unseen$ & $All$
%		\multicolumn{3}{c|}{mIoU} & 
%		\multicolumn{3}{c|}{mIoU} \\ 
%		&&&& $Seen$& $Unseen$ & $All$&$Seen$& $Unseen$ & $All$
		%\cellcolor{white}{}  
		%\cellcolor{white}{}
		\\
		\midrule
		
		\multicolumn{5}{l}{\textit{Supervised methods with different levels of supervision}}\\
		
		Full supervision & $seen \cup unseen$ & $seen \cup unseen$ & Point Cloud &43.3&51.9 &45.1&74.0&50.0&66.6 \\
		
		ZSL backbone & $seen$ & $seen \cup unseen$  &Point Cloud&41.5&39.2 & 40.3&60.9& 21.5&  48.7 \\
		
		ZSL-trivial & $seen$ & $seen$ &Point Cloud&39.2&0.0&31.3&70.2 &0.0&48.6  \\
		\midrule
		\multicolumn{5}{l}{\textit{Generalized zero-shot-learning methods}}\\
		
		ZSLPC-Seg~\cite{cheraghian2019zero}$^\dagger$ & $seen$ & $unseen$  &Point Cloud&28.2&0.0& 22.6&65.6 &0.0& 45.3\\
		
		DeViSe-3DSeg~\cite{frome2013devise}$^\dagger$ & $seen$ & $unseen$   &Point Cloud &20.0&0.0&16.0&70.2&0.0& 48.6\\ 
		%ZSLPC-Seg~\cite{cheraghian2019zero}$^\dagger$ & $seen$ & $unseen$  &  4.0&13.9\\
		%DeViSe-3DSeg~\cite{frome2013devise}$^\dagger$ & $seen$ & $unseen$   &  3.0&10.9\\
		3DGenZ~\cite{michele2021generative} & $seen$ & $seen \cup \hat{unseen}$  &Point Cloud &32.8&7.7& {27.8}&53.1&7.3&   \textbf{39.0} \\
		\midrule
		\multicolumn{5}{l}{\textit{Zero-shot learning + map fusion}}\\
		Uni-Fusion-SU (Ours) &None&None&Sparse Frames&31.0&\textbf{41.9}&\textbf{32.9} &31.3&\textbf{24.0}&29.0\\
		\bottomrule
		\multicolumn{1}{l}{}\\[-7pt]
	\end{tabular}

	\label{tab:sem_seg_overview}
\end{table*}

\begin{figure*}[t!]
	\centering
	\setlength{\tabcolsep}{0.1em}
	\renewcommand{\arraystretch}{.1}
	\begin{tabular}{|c | c |c |||c |c | c|}
		\toprule
		{\Large{3DGenZ}} & {\Large{Uni-Fusion}} &{\Large{Ground Truth}} & {\Large{3DGenZ}} &{\Large{Uni-Fusion-SU}} & {\Large{Ground Truth}} \\ \midrule
		
		\includegraphics[width=\scannetImSize\linewidth]{im/exp//ss/gen3dz_0568.png}
		&\includegraphics[width=\scannetImSize\linewidth]{im/exp//ss/mine_0568.png}
		&\includegraphics[width=\scannetImSize\linewidth]{im/exp//ss/gt_0568.png}
		&		\includegraphics[width=\scannetImSize\linewidth]{im/exp//ss/gen3dz_0164.png}
		&\includegraphics[width=\scannetImSize\linewidth]{im/exp//ss/mine_0164.png}
		&\includegraphics[width=\scannetImSize\linewidth]{im/exp//ss/gt_0164.png}\\
		
		
		\includegraphics[width=\scannetImSize\linewidth]{im/exp//ss/gen3dz_0249.png}
		&\includegraphics[width=\scannetImSize\linewidth]{im/exp//ss/mine_0249.png}
		&\includegraphics[width=\scannetImSize\linewidth]{im/exp//ss/gt_0249.png}
		&		\includegraphics[width=\scannetImSize\linewidth]{im/exp//ss/gen3dz_0435.png}
		&\includegraphics[width=\scannetImSize\linewidth]{im/exp//ss/mine_0435.png}
		&\includegraphics[width=\scannetImSize\linewidth]{im/exp//ss/gt_0435.png}\\
		
		
		\includegraphics[width=\scannetImSize\linewidth]{im/exp//ss/gen3dz_0046.png}
		&\includegraphics[width=\scannetImSize\linewidth]{im/exp//ss/mine_0046.png}
		&\includegraphics[width=\scannetImSize\linewidth]{im/exp//ss/gt_0046.png}
		&		\includegraphics[width=\scannetImSize\linewidth]{im/exp//ss/gen3dz_0050.png}
		&\includegraphics[width=\scannetImSize\linewidth]{im/exp//ss/mine_0050.png}
		&\includegraphics[width=\scannetImSize\linewidth]{im/exp//ss/gt_0050.png}\\
		\bottomrule
		
	\end{tabular}
	\includegraphics[width=\linewidth]{im/ss_colorbar}
	%\captionof{figure}
	\caption{Demonstration of semantic segmentation on the ScanNet dataset.
       Selected scenes are consistent with~\cref{fig:recons:scannet_demo}}
	\label{fig:segmentation_demo}
	
\end{figure*}

\subsection{Scene Understanding Results}

Saliency detection effectively highlights the objects of interest.
This is also considered part of 3D semantic understanding.
However, as the semantics categories vary, fusing different categories of semantics into multiple LIMs can be inefficient.
%
Therefore, in this section, we utilize Uni-Fusion to fuse and construct a surface field for high-dimensional CLIP embeddings.
With a single LIM, we can generate different semantic results based on corresponding commands.
%
Since now our Uni-Fusion works with OpenSeg for scene understanding purposes, we call it Uni-Fusion-SU.

\subsubsection{Semantic Segmentation}
\label{sec:exp:semantic}

We first evaluate our model on generalized zero-shot point cloud semantic segmentation application.
Generalized Zero-Shot Learning (GZSL) differs from Zero-Shot Learning (ZSL) in that ZSL only predicts classes unseen during training, while GZSL predicts both unseen and seen classes~\cite{michele2021generative}.
Therefore, comparing our results with GZSL SOTAs provides a better understanding of the potential of Uni-Fusion-SU, as it does not train on both seen and unseen. 

This test uses ScanNet and S3DIS datasets for benchmarking. 
It is important to note that the \textbf{compared baselines are trained on the corresponding datasets}.
Our Uni-Fusion-SU uses OpenSeg to provide the 2D image level feature ebmedding.
Although \textbf{Uni-Fusion-SU} is also zero-shot, \textbf{it does not touch any ScanNet or S3DIS annotations}.

We demonstrate the mIoU scores in~\cref{tab:sem_seg_overview}.
In particular, our model achieves best results among the zero-shot learning methods on the ScanNet dataset and remains competitive with fully supervised methods.

Furthermore, we provide results specifically for the unseen classes in~\cref{sup:tab:sn_acc_miou}.
Although not as good as the fully supervised approach, Uni-Fusion-SU performs much better than 3DGenZ.
In addition, our Uni-Fusion-SU demonstrates high precision in classes such as sofa and Toilet, even when compared to the fully supervised model.

\begin{table}[htbp]
		\caption{Classwise GZSL semantic segmentation performance (\%) on the ScanNet unseen split.}
	\centering
	\newcommand*\rotext{\multicolumn{1}{R{45}{1em}}}
	\setlength{\tabcolsep}{1.7pt}
	\begin{tabular}{@{}l@{~}c|rrrr|r@{}}
		\toprule		
		& &
		{\textbf{Bookshelf}} & {\textbf{Desk}} & {\textbf{Sofa}} & {\textbf{Toilet}} & \stackbox{mean} \\
		
		\midrule
		FSL (Fully supervise) & IoU & 	56.9&	30.0&	57.4&	63.4 & 51.9
		\\ 
		3DGenZ (Zero-shot) & IoU & 	6.3&	3.3&	13.1&	8.1 & 7.7
		\\
		Uni-Fusion-SU (Ours) & IoU &38.3&16.8&51.7&60.9&41.9
	\\ \midrule 
	3DGenZ (Zero-shot)& Acc. & 	13.4&	5.9&	49.6&	26.3 &23.8
	\\
	Uni-Fusion-SU (Ours) & Acc. &61.9&29.6&67.4&91.6& 62.6
		\\
		\bottomrule
	\end{tabular}

	\label{sup:tab:sn_acc_miou}
\end{table}

However, in the S3DIS dataset, our model does not outperform 3DGenZ and other methods as shown in~\cref{tab:sem_seg_overview}.

Even in the result of unsceened data, as presented in \cref{sup:tab:s3dis_acc_miou}, we observe that Uni-Fusion-SU hardly finds some classed, e.g. Beam and Column, which are not commonly annotated objects. 
However, for common objects like sofa and window, our model performs much better.

\begin{table}[htbp]
		\caption{Classwise GZSL semantic segmentation performance (\%) on the S3DIS unseen split.}
	\centering
	\newcommand*\rotext{\multicolumn{1}{R{45}{1em}}}
	\setlength{\tabcolsep}{1.7pt}
	\begin{tabular}{@{}l@{~}c|rrrr|r@{}}
		\toprule		
		& &
		{\textbf{Beam}} & {\textbf{Column}} & {\textbf{Sofa}} & {\textbf{Window}} & \stackbox{mean} \\
		
		\midrule
		FSL (Fully supervise) & IoU & 	63.1&	10.2&	54.1&	72.4 & 50.0
		\\ 
		3DGenZ (Zero-shot) & IoU & 	13.9&	2.4&4.9&	8.1 &7.3
		\\
		Uni-Fusion-SU (Ours) & IoU &5.5&0.02&57.4&32.9&	24.0
		\\ \midrule 
		3DGenZ (Zero-shot) & Acc. & 	20.0&	9.1&	62.4&	23.7 &28.8
		\\
		Uni-Fusion-SU (Ours) & Acc. &41.5&0.02&78.3&42.1& 40.5
		\\	
		\bottomrule
	\end{tabular}

	\label{sup:tab:s3dis_acc_miou}
\end{table}

We present the results of the semantic segmentation in~\cref{fig:segmentation_demo}. 
It is evident that, 3DGenZ's result contains more noise, as seen in the spotted sofa, bed and other objects.
Conversely, Uni-Fusion-SU's results are generally smoother and more precise.

%
%\begin{figure*}[htbp]
%	\centering
%	\includegraphics[width=.3\linewidth]{example-image-golden}
%	\includegraphics[width=.3\linewidth]{example-image-golden}
%	\includegraphics[width=.3\linewidth]{example-image-golden}
%	\\
%	\includegraphics[width=.3\linewidth]{example-image-golden}
%	\includegraphics[width=.3\linewidth]{example-image-golden}
%	\includegraphics[width=.3\linewidth]{example-image-golden}
%	
%	\caption{Semantic segmentation result on ScanNet.}
%\end{figure*}
%
%\begin{figure*}[htbp]
%	\centering
%	\includegraphics[width=.3\linewidth]{example-image-golden}
%	\includegraphics[width=.3\linewidth]{example-image-golden}
%	\includegraphics[width=.3\linewidth]{example-image-golden}
%	\\
%	\includegraphics[width=.3\linewidth]{example-image-golden}
%	\includegraphics[width=.3\linewidth]{example-image-golden}
%	\includegraphics[width=.3\linewidth]{example-image-golden}
%	
%	\caption{Semantic segmentation result on S3DIS.}
%\end{figure*}

\subsubsection{Scene Understanding with Different Properties}

\begin{figure*}[t!]
	\centering
	\setlength{\tabcolsep}{0.1em}
	\renewcommand{\arraystretch}{.1}
	\resizebox{\textwidth}{!}{\begin{tabular}{|c | c | c | c | c | c|}
			\toprule 
			& \textbf{scene0568\_00} & \textbf{scene0249\_00} & \textbf{scene0435\_00} & \textbf{office3} & \textbf{room0}\\
			\midrule
			{} &
			\raisebox{-.5\height}{\includegraphics[width=\fabImSize\linewidth]{im/exp/fab/scannet/0568_color.png}} & %\raisebox{-.5\height}{\includegraphics[width=\fabImSize\linewidth]{im/exp/fab/scannet/0164_color.png}} &
			\raisebox{-.5\height}{\includegraphics[width=\fabImSize\linewidth]{im/exp/fab/scannet/0249_color.png}} & \raisebox{-.5\height}{\includegraphics[width=\fabImSize\linewidth]{im/exp/fab/scannet/0435_color.png}}
			&
			\raisebox{-.5\height}{\includegraphics[width=\fabImSize\linewidth]{im/exp/fab/replica/office3_color.png}}
			&
			\raisebox{-.5\height}{\includegraphics[width=\fabImSize\linewidth]{im/exp/fab/replica/room0_color.png}}\\ %\raisebox{-.5\height}{\includegraphics[width=\fabImSize\linewidth]{im/exp/fab/scannet/0050_color.png}} %\includegraphics[width=\fabImSize\linewidth]{im/exp/fab/replica/office3_color.png}
			\\
			\textbf{Desk}  &
			\raisebox{-.5\height}{\includegraphics[width=\fabImSize\linewidth]{im/exp/fab/scannet/0568_lt_desk.png}}&
			%\raisebox{-.5\height}{\includegraphics[width=\fabImSize\linewidth]{im/exp/fab/scannet/0164_lt_desk.png}}&
			\raisebox{-.5\height}{\includegraphics[width=\fabImSize\linewidth]{im/exp/fab/scannet/0249_lt_desk.png}}&
			\raisebox{-.5\height}{\includegraphics[width=\fabImSize\linewidth]{im/exp/fab/scannet/0435_lt_desk.png}}
			&
			\raisebox{-.5\height}{\includegraphics[width=\fabImSize\linewidth]{im/exp/fab/replica/office3_lt_desk.png}}
			&
			\raisebox{-.5\height}{\includegraphics[width=\fabImSize\linewidth]{im/exp/fab/replica/room0_lt_desk.png}}\\
			%\raisebox{-.5\height}{\includegraphics[width=\fabImSize\linewidth]{im/exp/fab/scannet/0050_lt_desk.png}}
			%\includegraphics[width=\fabImSize\linewidth]{im/exp/fab/replica/office3_saliency.png}
			\\
			
			\textbf{Sofa} &
			\raisebox{-.5\height}{\includegraphics[width=\fabImSize\linewidth]{im/exp/fab/scannet/0568_lt_sofa.png}} &
			%\raisebox{-.5\height}{\includegraphics[width=\fabImSize\linewidth]{im/exp/fab/scannet/0164_lt_sofa.png}} &
			\raisebox{-.5\height}{\includegraphics[width=\fabImSize\linewidth]{im/exp/fab/scannet/0249_lt_sofa.png}} &
			\raisebox{-.5\height}{\includegraphics[width=\fabImSize\linewidth]{im/exp/fab/scannet/0435_lt_sofa.png}}&
			\raisebox{-.5\height}{\includegraphics[width=\fabImSize\linewidth]{im/exp/fab/replica/office3_lt_sofa.png}}
			&
			\raisebox{-.5\height}{\includegraphics[width=\fabImSize\linewidth]{im/exp/fab/replica/room0_lt_sofa.png}}\\
			%\raisebox{-.5\height}{\includegraphics[width=\fabImSize\linewidth]{im/exp/fab/scannet/0050_lt_sofa.png}}
			%\includegraphics[width=\fabImSize\linewidth]{im/exp/fab/replica/office3_style.png}
			\\
			\textbf{Work} &
			\raisebox{-.5\height}{\includegraphics[width=\fabImSize\linewidth]{im/exp/fab/scannet/0568_lt_work.png}} &
			%\raisebox{-.5\height}{\includegraphics[width=\fabImSize\linewidth]{im/exp/fab/scannet/0164_lt_work.png}} &
			\raisebox{-.5\height}{\includegraphics[width=\fabImSize\linewidth]{im/exp/fab/scannet/0249_lt_work.png}} &
			\raisebox{-.5\height}{\includegraphics[width=\fabImSize\linewidth]{im/exp/fab/scannet/0435_lt_work.png}}&
			\raisebox{-.5\height}{\includegraphics[width=\fabImSize\linewidth]{im/exp/fab/replica/office3_lt_work.png}}
			&
			\raisebox{-.5\height}{\includegraphics[width=\fabImSize\linewidth]{im/exp/fab/replica/room0_lt_work.png}}\\
			%\raisebox{-.5\height}{\includegraphics[width=\fabImSize\linewidth]{im/exp/fab/scannet/0050_lt_work.png}}
			%\includegraphics[width=\fabImSize\linewidth]{im/exp/fab/replica/office3_style.png}
			\\
			\textbf{Sittable} &
			\raisebox{-.5\height}{\includegraphics[width=\fabImSize\linewidth]{im/exp/fab/scannet/0568_lt_sit.png}} &
			%\raisebox{-.5\height}{\includegraphics[width=\fabImSize\linewidth]{im/exp/fab/scannet/0164_lt_sit.png}} &
			\raisebox{-.5\height}{\includegraphics[width=\fabImSize\linewidth]{im/exp/fab/scannet/0249_lt_sit.png}} &
			\raisebox{-.5\height}{\includegraphics[width=\fabImSize\linewidth]{im/exp/fab/scannet/0435_lt_sit.png}}&
			\raisebox{-.5\height}{\includegraphics[width=\fabImSize\linewidth]{im/exp/fab/replica/office3_lt_sit.png}}
			&
			\raisebox{-.5\height}{\includegraphics[width=\fabImSize\linewidth]{im/exp/fab/replica/room0_lt_sit.png}}\\
			%\raisebox{-.5\height}{\includegraphics[width=\fabImSize\linewidth]{im/exp/fab/scannet/0050_lt_sit.png}}
			%\includegraphics[width=\fabImSize\linewidth]{im/exp/fab/replica/office3_style.png}
			\\
			\textbf{Wood} &
			\raisebox{-.5\height}{\includegraphics[width=\fabImSize\linewidth]{im/exp/fab/scannet/0568_lt_wood.png}} &
			%\raisebox{-.5\height}{\includegraphics[width=\fabImSize\linewidth]{im/exp/fab/scannet/0164_lt_wood.png}} &
			\raisebox{-.5\height}{\includegraphics[width=\fabImSize\linewidth]{im/exp/fab/scannet/0249_lt_wood.png}} &
			\raisebox{-.5\height}{\includegraphics[width=\fabImSize\linewidth]{im/exp/fab/scannet/0435_lt_wood.png}}&
			\raisebox{-.5\height}{\includegraphics[width=\fabImSize\linewidth]{im/exp/fab/replica/office3_lt_wood.png}}
			&
			\raisebox{-.5\height}{\includegraphics[width=\fabImSize\linewidth]{im/exp/fab/replica/room0_lt_wood.png}}\\
			%\raisebox{-.5\height}{\includegraphics[width=\fabImSize\linewidth]{im/exp/fab/scannet/0050_lt_wood.png}}
			%\includegraphics[width=\fabImSize\linewidth]{im/exp/fab/replica/office3_style.png}
			\\
			
			
			\bottomrule
		\end{tabular}
	}
	%\captionof{figure}
	\caption{Demonstration of the original mesh, highlighted semantic mesh given various queries.}
	\label{fig:fab_lt}
	\vspace{-.5cm}
\end{figure*}

The main contribution of this application is that, Uni-Fusion is the first model to construct a continuous mapping of high-dimensional embeddings onto the surface without the need for any training of the map representation.
%
In the previous experiment (\cref{sec:exp:semantic}), we evaluate the performance of generalized zero-shot semantic segmentation.
However, the potential of Uni-Fusion goes beyond semantic segmentation.
%
By constructing a LIM, we obtain a surface CLIP feature field.
This enables us to query various semantic categories such as 
%without the need of multiple LIMs or rerun for other properties, we query 
\textbf{Object, Room Type, Material, Affordance and Activity} without requiring multiple LIMs or re-running the model.

We present the results in \cref{fig:fab_lt}, 
where we query object (desk, sofa), activity (work), affordance (sittable), and material (wood).
Uni-Fusion-SU accurately identifies and highlights the object and material regions.
However, for less specific commands such as work or sittable, the model provides a wider range of results with less confidence (indicated by dull yellow).
Nevertheless, the suggested options are also roughly correct.









\subsection{Time}

We run all of the applications in a single pass using our captured office sequences and evaluate the time cost of construction and fusion of each LIM. 
The average time cost across frames is shown in~\cref{tab:time}.

\begin{table}[htbp]
	\caption{Time required for each frame.
	}
	\centering
	\footnotesize
	\setlength{\tabcolsep}{0.7em}
	\resizebox{\linewidth}{!}{
		\begin{tabular}{l|ccccccc}
			\toprule
			&Surface & Color & Infrared & Style & Saliency & Latent&Internal Track \\ \midrule
			Time ($\si{\second}$)&0.100 & 0.038 & 0.045 & 0.048 & 0.045 &0.011 &0.225 \\ \bottomrule
	\end{tabular}}
	
	\label{tab:time}
\end{table}

\newcommand{\mineImSize}{.32}
%\begin{table*}[t!]
%	\centering
%	\setlength{\tabcolsep}{0.1em}
%	\renewcommand{\arraystretch}{.1}
%	\begin{tabular}{|c | c |c |}
%		\hline 
%		{Color} &{Infrared} & {Saliency} \\
%		\includegraphics[width=\mineImSize\linewidth]{im/exp/fab/mine/office/seq3_color.png} &
%		\includegraphics[width=\mineImSize\linewidth]{im/exp/fab/mine/office/seq3_color.png} &
%		\includegraphics[width=\mineImSize\linewidth]{im/exp/fab/mine/office/seq3_saliency.png} \\
%		{Style 1}&{Style 1}&{Style 1}\\
%		\includegraphics[width=\mineImSize\linewidth]{im/exp/fab/mine/office/seq3_style.png}&
%		\includegraphics[width=\mineImSize\linewidth]{im/exp/fab/mine/office/seq3_style.png}&
%		\includegraphics[width=\mineImSize\linewidth]{im/exp/fab/mine/office/seq3_style.png}\\
%		{Sofa}&{Desk}&{Soft}\\
%		\includegraphics[width=\mineImSize\linewidth]{im/exp/fab/mine/office/seq3_lt_sofa.png}&
%		\includegraphics[width=\mineImSize\linewidth]{im/exp/fab/mine/office/seq3_lt_desk.png}&
%		\includegraphics[width=\mineImSize\linewidth]{im/exp/fab/mine/office/seq3_lt_soft.png}\\		
%		\hline
%	\end{tabular}
%	\captionof{figure}{Demonstration on captured office data.}
%	\label{fig:mine_demo}
%\end{table*}
%\begin{figure*}[t!]
%	\centering
%	\setlength{\tabcolsep}{0.1em}
%	\renewcommand{\arraystretch}{.1}
%	\begin{tabular}{|c | c |c |}
%		\hline \hline
%		\includegraphics[width=\mineImSize\linewidth]{im/exp/fab/mine/office/seq3_w_slam_color.png}&	\includegraphics[width=\mineImSize\linewidth]{im/exp/fab/mine/office/seq3_w_slam_ir.png}&	\includegraphics[width=\mineImSize\linewidth]{im/exp/fab/mine/office/seq3_w_slam_saliency.png}\\
%		{Color} &{Infrared} & {Saliency}\\
%<<<<<<< HEAD


%=======
%		
%		\includegraphics[width=\mineImSize\linewidth]{im/exp/fab/mine/office/seq3_w_slam_style.png}
%		&\includegraphics[width=\mineImSize\linewidth]{im/exp/fab/mine/office/seq3_w_slam_lt_desk.png}
%		&\includegraphics[width=\mineImSize\linewidth]{im/exp/fab/mine/office/seq3_w_slam_lt_wood.png}\\
%		{Style} & {Object-desk} & {Material-wood} \\\hline
%>>>>>>> e014bc950c14dec9ffa1d2d7a6de9b7abfefabdd
%	\end{tabular}
%	%\captionof{figure}
%	\caption{Demonstration on captured Office data.}
%	\label{fig:office}
%\end{figure*}

\begin{figure*}[]
	\centering
	\setlength{\tabcolsep}{0.1em}
	\renewcommand{\arraystretch}{.1}
	\begin{tabular}{|c | c |c |}
	\hline \hline
	\includegraphics[width=\mineImSize\linewidth]{im/exp/fab/mine/appartment2/appartment2_color.png}&	\includegraphics[width=\mineImSize\linewidth]{im/exp/fab/mine/appartment2/appartment2_ir.png}&	\includegraphics[width=\mineImSize\linewidth]{im/exp/fab/mine/appartment2/appartment2_saliency.png}\\
		{Color} &{Infrared} & {Saliency}\\
	\includegraphics[width=\mineImSize\linewidth]{im/exp/fab/mine/appartment2/appartment2_style.png}
&\includegraphics[width=\mineImSize\linewidth]{im/exp/fab/mine/appartment2/appartment2_lt_sofa.png}
&\includegraphics[width=\mineImSize\linewidth]{im/exp/fab/mine/appartment2/appartment2_lt_desk.png}
\\
{Style} & {Object-sofa} & {Object-desk}\\
\includegraphics[width=\mineImSize\linewidth]{im/exp/fab/mine/appartment2/appartment2_lt_coat.png}
&\includegraphics[width=\mineImSize\linewidth]{im/exp/fab/mine/appartment2/appartment2_lt_sit.png}
&\includegraphics[width=\mineImSize\linewidth]{im/exp/fab/mine/appartment2/appartment2_lt_wood.png}\\
{Object-coat} & {Affordance-sit} & {Material-wood} \\\hline
	\end{tabular}
%\captionof{figure}
\caption{Demonstration on the captured apartment data.}
\label{fig:appartment}
%\vspace{-.5cm}
\end{figure*}


Using depth and property images of size $720\times1280$ as input, it is evident from the table, that our model operates at a frequency of $\sim10\si{\hertz}$ for  surface (sample mode) LIM construction and integration. 
It alse achieves a frequency of over $20\si{\hertz}$ for color, infrared, style, and saliency.
These results demonstrate the suitability of Uni-Fusion for real-time applications.

However, our internal tracking process takes around $0.225\si{\second}$ per frame, which is relatively slower compared to the mapping module. 
Nevertheless, Uni-Fusion uses external tracking to prevent tracking loss, enabling our internal tracking and mapping to operate at a lower frequency.
As a result, the entire model can be effectively applied in real-time in various scenarios.

\section{Extensive experiment on our own dataset}

In previous experiments, we evaluate the capabilities of Uni-Fusion in different applications. 
To further demonstrate its effectiveness in robotic environmental understanding, we capture our own dataset to show all applications together.

We capture two scenes: The office and apartment of the first author using a Microsoft Kinect Azure. 
%
RGB-D and infrared video are captured. After calibration, RGB, depth, infrared inputs have resolution of $720\times1280$.
Uni-Fusion tracks and reconstructs all applications in one pass.
%
While office data has been involved in ablation study (\cref{exp:surface:ablation}), we showcase all applications using the apartment dataset, as depicted in~\cref{fig:appartment}.

For better visualization, the ceiling of reconstruction is removed.
The top row of images presents the colored mesh with room details, the infrared mesh revealing the lighting effect, and the saliency reconstruction highlighting objects crucial for navigation.
Additionally, we select the second style from~\cref{fig:style} for style transfer to the apartment canvas.
%
As a result, the wooden floor in the room is colored with dark green.
The whole apartment is in a warm style.

The remaining results are generated from the surface field of the CLIP embeddings. 
We issue commands to locate objects, e.g., where is the sofa, desk and coat.
In addition, it easily identifies affordances such as being sittable.
For material, it successfully detects the wooden floor in each room.


\section{Conclusion}
\vspace{-2.5px}
In this work, we propose \textit{LayoutDiffusion} to improve graphic layout generation by discrete diffusion models. 
The core of our method lies in realizing a mild forward process by considering the heterogeneous characteristics of the layout. 
Our method also enables two conditional generation tasks without re-training. 
Experiments demonstrate the superiority of LayoutDiffusion over leading approaches for layout generation and existing diffusion models.
In the future, we plan to incorporate diverse conditions~\cite{nichol2021glide,ho2022classifier} in LayoutDiffusion.
Besides, we will also explore how to extend LayoutDiffusion to handle other heterogeneous data.

{\small
\bibliographystyle{ieee_fullname}
\bibliography{egbib}
}
\clearpage
\appendix

\onecolumn

\begin{spacing}{1}
\tableofcontents
\end{spacing}

\newpage


\section{Additional Implementation Details}

\subsection{More Implementation Details for LayoutDiffusion}

\noindent \textbf{Noise Schedule.} We investigate the effectiveness of our proposed schedule $\beta_t=g/(T-t+\epsilon)^h$ for the discretized Gaussian transition matrix by comparing with the original linear schedule $\beta_t=bt/T$ used in~\cite{d3pm}. 
Notably, here $\beta_t$ is not a variance term bounded in $[0,1]$\footnote{In fact, as $\beta_t$ tends to positive infinity, $\mathbf{Q}_t^{coord}$ will approach a transtion matrix for uniform noise as described by Sohl-Dickstein \etal~\cite{2015}.}, and with the growth of forward steps, the cumulative matrix $\overline{\mathbf{Q}}_t^{coord}$ converges to uniform distribution.
Thus, we attempt to analyse the noise process by observing the standard derivation of the cumulative matrix.
A higher std. indicates a more sparse matrix and hence a lower transition probability to other coordinate tokens. 
As shown in~\cref{fig:noise schedule}, 
our schedule presents a gentler noising process and a more stable convergence state compared to the original linear schedule (as suggested by a higher std. at the beginning of the forward process, and a lower std. at the very end of the process).

\vspace{-11px}
\begin{figure}[th]
\centering

   \includegraphics[width=0.35\linewidth]{cvpr2023-author_kit-v1_1-1/latex/suppl_pics/noise.pdf}%
\vspace{-5px}
   \caption{Standard derivation of the cumulative discretized Gaussian matrix $\overline{\mathbf{Q}}_t^{coord}$ of our proposed $\beta_t$ and the original linear schedule. }
\vspace{-5px}%
   \label{fig:noise schedule} 
\end{figure}



\noindent \textbf{Denoising Model.}
We present the model architecture as in~\cref{model arch}. We embed the input sequence, input timestep, and positions with 768, 128, and 768 dimensions respectively. The dropout rate is set as 0.1. For the transformer encoder, we simply adopt the same hyperparameters of BERT-base~\cite{bert} encoder, i.e., 12 layers, 12 attention heads, 768 hidden size, and 3,072 dimensions for the feed forward layer.

\begin{figure}[th]
  \centering
   \includegraphics[width=1\linewidth]{cvpr2023-author_kit-v1_1-1/latex/suppl_pics/model.pdf}
   \vspace{-10px}
\caption{Model architecture of $p_{\theta}(x_0|x_t)$.}
\vspace{-5px}
   \label{model arch} 
\end{figure}





\noindent \textbf{Hyperparameters.} We train our model using AdamW optimizer~\cite{adam} with $lr=0.00004$, $betas=(0.9,0.999)$, and zero $weight\_decay$. We also apply an exponential moving average (EMA) over model parameters with a rate of $0.9999$.
We set the batch size as 64. For RICO~\cite{rico}, we train the model with 2 V100 GPUs for 175,000 steps to achieve the best results; and for PublayNet~\cite{publaynet}, we train the model with 4 V100 GPUs for 350,000 steps.
For the schedule of type tokens, we set $\Tilde{T}=160$ in~\cref{late absorb}. For the schedule of coordinate tokens, we set $T=\Tilde{T}=160$, $g=12.4$, $h=2.48$, and $\epsilon=0.0001$ in $\beta_t=g/(T-t+\epsilon)^h$. We also provide the hyperparameters for the implementation of different timesteps apart from 200 steps (we implement these variants in the \cref{tbl:speed} of the main paper), as shown in~\cref{tbl:schedule}.


\begin{table}[h]
\centering
\setlength{\tabcolsep}{3mm}{
  \begin{tabular}{lcc}
    \toprule
    Total timesteps &  $\Tilde{T}$ for type schedule &   $\beta_t$ for coordinate schedule \\
    \midrule
    100 &80 & $20.0/(80-t+0.0001)^{2.96}$   \\
    200 &160 & $12.4/(160-t+0.0001)^{2.48}$   \\
    500 &400 & $6.2/(400-t+0.0001)^{2.00}$  \\ 
    1000 & 800 & $3.5/(800-t+0.0001)^{1.76}$   \\
    2000 & 1600 & $2.0/(1600-t+0.0001)^{1.52}$   \\ 
    \bottomrule
  \end{tabular}}
\caption{Hyperparameters for variants of different timesteps.}
\vspace{-5px}
\label{tbl:schedule}
\end{table}


\subsection{Implementation of other Diffusion Baselines}
\noindent \textbf{Diffusion-LM~\cite{diffusionlm}.} We implement Diffusion-LM based on the official repository\footnote{\url{https://github.com/XiangLi1999/Diffusion-LM}}. We realize the layout generation task via Diffusion-LM by feeding the layout as a sequence and reconverting the output sequence to a layout. For the hyperparameters, we simply adopt the default setting that Diffusion-LM adopts on E2E~\cite{e2e} dataset, since only relatively small vocabulary size ($\sim 150$) is required for the tokens that represent layout sequence.

For conditional generation tasks, we apply the similar idea as in LayoutDiffusion. Specifically, for Gen-Type, we fix the type tokens by feeding the target in each timestep and run the whole reverse denoising process. For refinement task, we embed the layout sequence and set the embedded latent as the input of start timestep $T_{\text{refine}}$, and then run the remaining reverse process with type fixed.
For the choice of $T_{\text{refine}}$, we traverse through [250,500,750,1000,1250,1500] of the total 2000 steps, and find the best result is achieved when $T_{\text{refine}}=1000$.








\vspace{5px}
\noindent \textbf{D3PM uniform~\cite{d3pm}.} We implement D3PMs based on both the official repository of D3PMs\footnote{\url{https://github.com/google-research/google-research/tree/master/d3pm}} and the official repository of another method concerning discrete diffusion model, i.e., VQ-Diffusion\footnote{\url{https://github.com/microsoft/VQ-Diffusion}}, since our implementation is based on PyTorch~\cite{paszke2019pytorch} rather than JAX~\cite{jax2018github}. We realize the layout generation task in a similar way as LayoutDiffusion, i.e., feeding the layout as a sequence of tokens and treating each token as a discrete state.

For the hyperparameters of the diffusion framework, we set the total diffusion timesteps $T=1000$, the schedule $\beta_t=(T-t+1)^{-1}$, and the auxiliary loss weight $\lambda=0.0001$, all follow the setting as reported in D3PMs paper. For the denoising model, we apply the similar model as in LayoutDiffusion and Diffusion-LM for a fair comparison.

For conditional generation tasks, we implement the Gen-Type and refinement the same way as in our implementation of Diffusion-LM, since both methods are replace-based diffusion methods. For the start timestep for refinement task $T_{\text{refine}}$, we sweep from [200,400,600,800] of the total 1000 steps, and find $T_{\text{refine}}=400$ is the optimal choice.


\vspace{5px}
\noindent \textbf{D3PM absrobing~\cite{d3pm}.} We implement the D3PM (absorbing) the same way as in D3PM (uniform). All settings except for the model are also referenced from the original D3PMs paper.

One major difference lies in the implementation of conditional generation tasks. 
For the Gen-Type task, we apply the similar idea as in LayoutDiffusion. To be more specific, we feed the given type set at the beginning step $T$, and run the whole reverse process. 
It is noteworthy that, for D3PM (absorbing), all coordinates start to recover strictly from timestep $T$.
Hence, it cannot save steps as the same strategy in LayoutDiffusion by picking a timestep $T_{\text{Gen-Type}}$ which is smaller than $T$. 
Besides, in the reverse process of D3PM (absorbing), the sampled tokens cannot transition into \texttt{MASK} or other tokens, 
thus, it is unable to perform the refinement task as in LayoutDiffusion. 




\subsection{Settings on Conditional Generation Tasks}
\label{sec:details cond tasks}
We present below the settings on conditional generation tasks in the main paper.
For further experiments on conditional generation tasks, please refer to~\cref{more cond}.
\vspace{2.5px}

\noindent \textbf{Gen-Type.}
We follow the convention in \cite{blt,unilayout}.
To be more specific, for a layout in the test set, we extract its type set as the input and let the model generate the bounding box attributes of each element .


\vspace{2.5px}
\noindent \textbf{Refinement.} 
In the real scenario, the noise level of the user-given flawed layout cannot be known in advance.
Besides, different flawed layouts may have different noise levels.
To simulate the real scenario, we improve the setting used in RUITE~\cite{rahman2021ruite}.
Specifically, the setting in RUITE is to construct a test set by adding random noises to the position and size of each element, where the noise is sampled from a normal distribution with mean $0$ and the standard variance $0.01$.
In our improved setting, we modify the standard variance of the noise to be uniformly sampled from $[0.005,0.01,0.015,0.02,0.025]$. 
Besides, for the baselines, we train the model with input noise of 0.01 standard variance; for LayoutDiffusion, we apply the same inference steps $T_{\text{refine}}$ for different levels of noise, unlike the settings in~\cref{sec:suppl_refine}.

\subsection{Details about Classification Model for FID Evaluation}
We use the same layout classification model as LayoutGan++~\cite{layoutganpp} and LayoutFormer++~\cite{unilayout}. Specifically, the model includes an encoder and a decoder, both of which are based on the Transformer architecture.
The encoder takes in bounding box coordinates and corresponding labels and produces a feature representation, while the decoder uses the feature representation to predict the class probabilities and bounding box coordinates for each layout element.
We implement the model based on the official repository of LayoutGan++\footnote{\url{https://github.com/ktrk115/const_layout}} and train it using the methods described in their paper.
Our evaluation results using this model are consistent with those reported in the LayoutFormer++.

\section{Discussion on Diversity} %
\label{sec:selfsim}
\subsection{Metric SelfSim}
Diversity is a key but often overlooked aspect in layout generation tasks. In this section, we propose a new metric called \emph{SelfSim} to measure the self-similarity of generated layouts, which serves as an indicator of diversity. The intuition behind this metric is that more diverse generated layouts should be less self-similar. Specifically, we calculate the average Intersection over Union (IoU) between any pairs of generated layouts with the same set of element types.

\subsection{Algorithm of SelfSim}
Inspired by the metrics for evaluating diversity in NLP (\eg, diverse 4-gram~\cite{div4} and self-BLEU~\cite{selfB}), we propose to assess the diversity of generated layouts by measuring the self-similarity of the generated layout set. Specifically, we partition the generated layouts into different subsets based on their type sets and then count the similarity of the layouts within each subset. The similarity is calculated by averaging the intersection of union (IoU) of the bounding boxes for each pair of layouts in the subset.
We present the algorithm for calculating SelfSim as in \cref{algo_selfsim}.

\IncMargin{1em}
\begin{algorithm}[h]
\label{algo_selfsim}
	\caption{Calculation of the Self-Similarity score}
    \vspace{1px}
	\KwIn{A set of graphic layouts $\mathbb{X}=\{\mathbf{x}_1,\mathbf{x}_2,\dots,\mathbf{x}_m\}$} 
	\vspace{1px}
	\KwOut{The Self-Similarity score of the given layout set}
	 \BlankLine 
	 
	 \emph{partition $\mathbb{X}$ by the layouts' \textbf{type set}, denote the partition as $\mathbb{P}=\{\mathbb{X}_1,\mathbb{X}_2,\dots,\mathbb{X}_n\}$}\; 
	 \For(\tcc*[f]{traverse each subset $\mathbb{X}_i \in \mathbb{P}$}){$i\leftarrow 1$ \KwTo $n$}{ 
	 	\emph{count the number of elements in subset $\mathbb{X}_i$, denoted as $l_i$}\; 
	 	\If(\tcc*[f]{only one layout in the subset}){$l_i=1$}{set the Self-Similarity score of subset $\mathbb{X}_i$ as $S_i=0$\;}
	 	\Else(\tcc*[f]{more than one layout has this \textbf{type set}}){
	 	\For{each pair $(x_j^i,x_k^i) (j \neq k)$ in $\mathbb{X}_i$}{ 
	 	calculate the IoU of bounding boxes between $x_j^i$ and $x_k^i$, denoted as $U_{jk}^i$\; }
average the $U_{jk}^i$ of the total ${l_i \choose 2}$ pairs to get the mean $S_i$\;}
 	 }
 	 \KwRet{the weighted average of all subsets' Self-Similarity score $\dfrac{\sum_{i=1}^n l_i S_i}{\sum_{i=1}^n l_i}$\;}

 	 \end{algorithm}
 \DecMargin{1em} 
 \vspace{-5px}

 
\begin{figure}[t]
  \centering
  \begin{subfigure}{0.28\linewidth}
    \includegraphics[width=1\linewidth]{cvpr2023-author_kit-v1_1-1/latex/suppl_pics/0.png}
    \caption{$S_i=0.0$}
    \label{fig:00}
  \end{subfigure}
  \hspace{10px}
  \begin{subfigure}{0.28\linewidth}
\includegraphics[width=1\linewidth]{cvpr2023-author_kit-v1_1-1/latex/suppl_pics/2.png}
    \caption{$S_i=0.2$}
    \label{fig:02}
  \end{subfigure}
  \hspace{7.5px}
    \begin{subfigure}{0.38\linewidth}
    \includegraphics[width=1\linewidth]{cvpr2023-author_kit-v1_1-1/latex/suppl_pics/4.png}
    \caption{$S_i=0.4$}
    \label{fig:04}
  \end{subfigure}
    \vspace{5px}
    
  \begin{subfigure}{0.29\linewidth}
    \includegraphics[width=1\linewidth]{cvpr2023-author_kit-v1_1-1/latex/suppl_pics/6.png}
    \caption{$S_i=0.6$}
    \label{fig:06}
  \end{subfigure}
  \hspace{10px}
  \begin{subfigure}{0.29\linewidth}
\includegraphics[width=1\linewidth]{cvpr2023-author_kit-v1_1-1/latex/suppl_pics/8.png}
    \caption{$S_i=0.8$}
    \label{fig:08}
  \end{subfigure}
  \hspace{10px}
    \begin{subfigure}{0.195\linewidth}
\includegraphics[width=1\linewidth]{cvpr2023-author_kit-v1_1-1/latex/suppl_pics/10.png}
    \caption{$S_i=1.0$}
    \label{fig:10}
  \end{subfigure}
  \caption{Examples of subsets with different SelfSim. As SelfSim goes from 0 to 1, the layouts in the subset go from totally different to completely identical.}
  \vspace{-5px}
  \label{fig:selfsim}
\end{figure}

\subsection{Case Study of SelfSim}
To visually demonstrate the effectiveness of SelfSim, we show some subsets with different SelfSims (i.e., subsets with different $S_i$) in~\cref{fig:selfsim}. We observe that subsets with higher SelfSims tend to have more similar layouts, while those with lower SelfSims have more diverse layouts. This further supports the effectiveness of the SelfSim metric for assessing the diversity of generated layouts.



\subsection{SelfSim Comparison with Existing Methods}
\cref{tbl:self-sim} compares LayoutDiffusion with existing layout methods and the diffusion-based method using SelfSim. While LayoutFormer++, Diffusion-LM, and LayoutTransformer have advantages in certain aspects of quality (as shown in \cref{Tab:quantitative_results} in the main paper), they suffer from obvious diversity issues, which aligns with our user study findings (as shown in \cref{user_study}). On the other hand, although D3PM performs slightly better in diversity on the PubLayNet dataset, it lags behind in terms of quality (see \cref{Tab:quantitative_results}). These results suggest that our proposed method achieves a better quality-diversity trade-off.

\begin{table}[t]
\centering
\renewcommand{\arraystretch}{0.9}
\setlength{\tabcolsep}{4pt}
\begin{tabular}{lccccccc}
\toprule

SelfSim$\downarrow$ & & LayoutTransformer & LayoutFormer++ & Diffusion-LM & D3PM (absorbing) & D3PM (uniform) & LayoutDiffusion\\
\midrule
RICO & & 0.318 & \textit{0.581} & \textit{0.326} & 0.157 & 0.165 & 0.157 \\
PublayNet & & \textit{0.314} & \textit{0.328} & 0.222 & 0.194 & 0.189 & 0.198 \\
\bottomrule
\end{tabular}
\vspace{-5px}
\caption{Comparison of SelfSim scores for unconditional generation on RICO and PublayNet datasets. Lower SelfSim scores indicate better diversity. \textit{Italic font} denotes the two worst-performing methods.}
\label{tbl:self-sim}
\end{table}


\section{Additional Experiments on Conditional Generation}
\label{more cond}
In this section, we design two sets of experiments to investigate LayoutDiffusion's robustness in the refinement task and its diversity performance in the generation conditioned on type (Gen-Type) task.

\subsection{Refinement}
\label{sec:suppl_refine}

\begin{table}[h]
    \centering
    \setlength{\tabcolsep}{7.5mm}{
    \renewcommand{\arraystretch}{1}
    \begin{small}
        \resizebox{\textwidth}{!}{
            \begin{tabular}{llcccc}
                \specialrule{1.1pt}{0pt}{1pt}

Noise level & \makecell[c]{Methods}       & mIoU $\uparrow$  &   Overlap$\rightarrow$ & Align.  $\rightarrow$  & FID  $\downarrow$  \\ \midrule
\multirow{3}{*}{std.=0.005} & RUITE                 & 0.743 & 0.473   & 0.126               & 1.244   \\
      & UniLayout             & 0.722 & 0.479   & 0.119               & 1.043   \\
      & LayoutDiffusion   (30 steps)  & \textbf{0.787} & \textbf{0.467}   & \textbf{0.095}               & \textbf{0.499}   \\\midrule
\multirow{3}{*}{std.=0.01}  & RUITE                  & 0.716 & 0.483   & 0.139               & 1.475   \\
      & UniLayout             & 0.704 & 0.487   & 0.123               & 1.124   \\
      & LayoutDiffusion   (40 steps)  & \textbf{0.759} & \textbf{0.467}   & \textbf{0.098}               & \textbf{0.500}   \\\midrule
\multirow{3}{*}{std.=0.02}  & RUITE                  & 0.611 & 0.507   & 0.203               & 13.633  \\
      & UniLayout              & 0.621 & 0.514   & 0.157               & 4.981   \\
      & LayoutDiffusion   (50 steps)  & \textbf{0.748} & \textbf{0.469}   & \textbf{0.097}               & \textbf{0.496}   \\\midrule
      & Real Data  &-& 0.466 & 0.093 & -       \\
                \specialrule{1.1pt}{1pt}{0pt}
            \end{tabular}}
    \end{small}}

        \caption{Qualitative comparison under different noise levels on RICO. The content in the brackets denotes for the number of inference steps. The best results are \textbf{bold}.}
    \label{Tab:refine_table}
\vspace{-5px}
\end{table}
\vspace{-5px}

As introduced in~\cref{sec:details cond tasks}, in the main paper, our experiments for the refinement task apply a mixture of different levels of noise as input. 
We suppose that the excellent results achieved by LayoutDiffusion are due to its capability of handling various levels of noise.
To investigate the model's robustness to the noise, in this section, we compare with the two strongest baselines and further study the performance of the methods under each specific noise levels.

Specifically, we evaluate the performance of different methods under the conditions that the standard deviation of the noise is 0.005, 0.01, and 0.02, respectively. Plus, for a fair comparison, we train the baselines with the input noise of 0.01 standard deviation, and apply the same model for inference.



\noindent \textbf{Quantitative results.} As shown in~\cref{Tab:refine_table}, for the two baselines (RUITE~\cite{rahman2021ruite} and LayoutFormer++~\cite{unilayout}), the models exhibit favorable performances when dealing with noise levels less than or equal to the training level (std.=0.005 and std.=0.01).
However, when the testing noise level is greater than the training's (std.=0.02), the models suffer a significant performance drop.
For LayoutDiffusion, it not only surpasses the baseline in all 12 competitions (3 levels $\times$ 4 metrics), but also consistently presents excellent performance as the noise level varies, indicating that LayoutDiffusion is highly robust to noise levels.


\begin{figure}[th]
  \centering

   \includegraphics[width=1.01\linewidth]{cvpr2023-author_kit-v1_1-1/latex/suppl_pics/suppl_refine.pdf}

   \caption{Qualitative comparison under different noise levels on RICO. Each row shares the same noise levels while each column shares the same method. For more quantitative result of LayoutDiffusion on Refinement, please refer to~\cref{more refine}.}
   \label{suppl refine} 
   \vspace{-5px}
\end{figure}
\noindent \textbf{Qualitative results.} We provide the quantitative results in~\cref{suppl refine}. We can conclude that as the input gets more chaotic, LayoutDiffusion consistently produces pleasing layouts, while other baselines fail to achieve so, which is in line with the quantitative results.


\subsection{Generation Conditioned on Type}
As described in~\cref{sec:details cond tasks}, in the main paper, our experiments for the Gen-Type task only generate one sample for each input type set.
In this section, we study whether they can generate multiple diverse layouts for a given type set to further explore the diversity performance of each method under Gen-Type task.

Specifically, we first find all the different type sets in the layouts of the test set, and then equally generate 5 layouts for each given type set\footnote{In practice, we find 2714 different type sets out of 3729 layouts in the test set of RICO, and 1339 type sets out of 10998 layouts in PublayNet's test set.}. 
We compare to the baseline with the best quality performance (i.e., LayoutFormer++~\cite{unilayout}), which can generate multiple layouts with top-k sampling~\cite{fan-etal-2018-hierarchical}. LayoutDiffusion can generate multiple layouts by simply running the inference process multiple times.
We apply the metric SelfSim (as discussed in~\cref{sec:selfsim}) for the evaluation of diversity.

\begin{table}[H]
    \centering
    \setlength{\tabcolsep}{6mm}{
    \renewcommand{\arraystretch}{1}
    \begin{small}
        \resizebox{\textwidth}{!}{
            \begin{tabular}{llccccc}
                \specialrule{1.1pt}{0pt}{1pt}

Dataset & \makecell[c]{Methods}       & mIoU $\uparrow$  &   Overlap$\downarrow$ & Align.  $\downarrow$  & FID  $\downarrow$ &SelfSim  $\downarrow$  \\ \midrule
\multirow{2}{*}{RICO}
      & UniLayout           & \textbf{{0.375}}
&
0.563
&
0.125
&
9.786
&
0.536
  \\
      & LayoutDiffusion    & 0.357
&
\textbf{{0.490}}
&
\textbf{{0.062}}
&
\textbf{8.973}
&
\textbf{{0.268}}
 \\\midrule
\multirow{2}{*}{PublayNet} &  UniLayout           & \textbf{{0.315}}
&
0.025
&
0.030
&
31.121
&
0.224
   \\
      & LayoutDiffusion     &0.312
&
\textbf{{0.007}}
&
\textbf{{0.029}}
&
\textbf{{21.522}}
&
\textbf{{0.189}}
   \\
                \specialrule{1.1pt}{1pt}{0pt}
            \end{tabular}}
    \end{small}}

        \caption{Qualitative comparison under new sampling strategy (5 samples for each type set). Since the type distribution of the generated layouts differs from that of the test set, we simply assume here that the less misalignment and overlap is better.}
    \label{Tab:type_table}
\vspace{-5px}
\end{table}

\noindent \textbf{Quantitative results.} The quantitative results is given in~\cref{Tab:type_table}. Compared to LayoutFormer++, LayoutDiffusion performs significantly better in diversity (as suggested by SelfSim), while achieving comparable quality performance (as suggested by Overlap and Align.).
We hypothesize that the gap between diversity is due to the probability accumulation of the autoregressive model while LayoutDiffusion samples each layout from independent noise.

\noindent \textbf{Qualitative results.} As show in~\cref{type suppl}, despite equipped with top-k sampling, LayoutFormer++ still suffers severe diversity problem (duplication occurs in the first three row and the last row. Besides, the generated layouts in the fourth row share similar patterns). 
While for LayoutDiffusion, all the 5 samples of each type set are both pleasing and in great diversity, further demonstrating the superiority of LayoutDiffusion on Gen-Type task.



\begin{figure}[th]
  \centering
  \vspace{-10px}
   \includegraphics[width=0.95\linewidth]{cvpr2023-author_kit-v1_1-1/latex/suppl_pics/type_suppl.pdf}

   \caption{Qualitative comparison of diversity on RICO. Each type set corresponds to five samples from LayoutFormer++ (left) and five samples from LayoutDiffusion (right). For more qualitative results of LayoutDiffusion on Gen-Type, please refer to~\cref{more gen-type}.}
   \label{type suppl} 
\end{figure}


\section{Additional Ablation Studies}
\label{more ablation}
\subsection{Additional Ablation Studies on Conditional Generation Tasks}
In addition to the ablation studies on unconditional layout generation discussed in the main paper (see \cref{ablation}), we also conducted ablation studies with the variations on two conditional generation tasks, i.e., Gen-Type and Refinement, to further investigate the effectiveness of our design.
The quantitative comparison is shown in \cref{Tab:ablation_condition}. 

LayoutDiffusion consistently outperforms all the variations on almost all metrics in both tasks. This result indicates that our design is superior and the reverse generation process is well-suited to both tasks, allowing the model to better leverage the given conditions.
Notably, the comparison between LayoutDiffusion and Uniform $\mathbf{Q}_t^{\text{type}}$ as well as Linear $\overline{\gamma}_t$ highlights the importance of our handling of the type tokens, which considers type corruption factor. This factor leads to better utilization of type information in the Gen-Type task.
Moreover, the comparison between LayoutDiffusion and the two variations of $\mathbf{Q}_t^{\text{coord}}$ as well as Linear $\beta_t$ in the Refinement task demonstrates the importance of our design for the coordinate tokens, which helps us model the precise details of the layout and achieve better performance in the Refinement task.

\begin{table}[H]
    \centering
    \setlength{\tabcolsep}{3.5mm}{
    \renewcommand{\arraystretch}{0.75}
    \begin{small}
        \resizebox{\textwidth}{!}{
            \begin{tabular}{lcccccccc}
                \specialrule{1.1pt}{0pt}{1pt}
                                                                   & \multicolumn{4}{c}{Gen-Type}   & \multicolumn{4}{c}{Refinement}                                                                                                                                                                                 \\ \cmidrule(l){2-5} \cmidrule(l){6-9}
                 \makecell[c]{Methods} & mIoU $\uparrow$           &   Overlap $\downarrow$             & Align. $\downarrow$        & FID $\downarrow$& mIoU $\uparrow$          &   Overlap $\downarrow$          & Align.$\downarrow$       & FID$\downarrow$      \\ \midrule
                      
                                            
                                             Uniform $\mathbf{Q}_t^{\text{coord}}$     &0.324 & 0.601 & 0.141 & 2.944   & 0.645 & 0.487 & 0.199 & 4.312      \\
                                             Absorbing $\mathbf{Q}_t^{\text{coord}}$                   &0.336 & 0.587 & 0.137 & 2.846     & - & -  & - & -     \\ 
                                            Uniform $\mathbf{Q}_t^{\text{type}}$          & 0.320& 0.532 &  0.188& 3.070 &0.698 & 0.477 &  0.167 & 2.443       \\ \midrule
                                            Linear $\overline{\gamma}_t$   & 0.308 & 0.513 & 0.164 & 2.768  & 0.667 & \textbf{0.467} & 0.133 &1.451 \\
                                            Linear $\beta_t$ &0.317 & 0.527 & 0.191 & 2.273  &0.659 & 0.491 & 0.185 &  1.835 \\\midrule
                                            LayoutDiffusion (ours)   &\textbf{0.345} & \textbf{0.491} & \textbf{0.124} & \textbf{1.557}   & \textbf{0.719} & 0.469 & \textbf{0.102} & \textbf{0.549}  \\
                                             
                \specialrule{1.1pt}{1pt}{0pt}
            \end{tabular}}
    \end{small}}
     \vspace{-5px}
        \caption{Quantitative results on conditional generation tasks for LayoutDiffusion and its ablations on RICO. The variation of absorbing $\mathbf{Q}_t^{\text{coord}}$ do not support refinement, as the coordinates are fixed during generation. The best result is in \textbf{bold}.
        }
    \label{Tab:ablation_condition}

\end{table}


\subsection{Additional Ablation Studies on Noise Schedule of Type Tokens}
In this section, we further investigate the effectiveness of our type schedule by experimenting other different $\overline{\gamma}_t$ schedules.

Recall that in the main paper, we follow the insight that type changes in the early stage may bring large semantic shift to the layout, thus, we set the noise schedule for $\overline{\gamma}_t$ as:
\begin{equation}
    \overline{\gamma}_t = \begin{cases}
    0, &t< \Tilde{T} \\
(t-\Tilde{T})/(T-\Tilde{T}), &t \ge \Tilde{T} 
\label{late absorb}
\end{cases}
\end{equation}

We denote this kind of schedule as ``late absorb $\Tilde{T}$", since under this schedule, all type tokens stay unchanged until timestep $\Tilde{T}$ when they start to absorb, and at the terminal step $T$, all type tokens reach the absorbed state. Follow this idea, we can come to a similar noise schedule, ``early absorb $\Tilde{T}'$", where the type tokens start to absorb at the beginning and fully adsorbed in the early stage, and it can be defined as follows:

\begin{equation}
    \overline{\gamma}_t = \begin{cases}
    t/\Tilde{T}', &t< \Tilde{T}' \\
1, &t \ge \Tilde{T}' \\
\end{cases}
\end{equation}

Note that, when $\Tilde{T}'=T$ and $\Tilde{T}=0$, two schedules becomes the same and is experimented in the ablation studies of the main paper (denoted as ``linear $\overline{\gamma}_t$"). Here, we provide a detailed experiment on $\overline{\gamma}_t$, including different choices of ``late absorb $\Tilde{T}$" and ``early absorb $\Tilde{T}'$".



\begin{table}[t]
\vspace{-0.2cm}
\caption{Ablation study on matching and task-specific adaptation.}
\vspace{-0.4cm}
\label{tab:abaltion_table}
\begin{center}
    \renewcommand{\arraystretch}{1.5}
    \renewcommand{\aboverulesep}{0pt}
    \renewcommand{\belowrulesep}{0pt}
    \setlength\tabcolsep{2pt}
    \fontsize{8pt}{8} \selectfont
    \begin{tabular}{c|cc|cc|cc|cc|cc}
        \toprule
        \multirow{4}{*}{Model} &
        \multicolumn{10}{c}{Tasks} \\
        
        \cmidrule{2-11}
        &
        \multicolumn{2}{c|}{Fold 1} & \multicolumn{2}{c|}{Fold 2} & \multicolumn{2}{c|}{Fold 3} & 
        \multicolumn{2}{c|}{Fold 4} & \multicolumn{2}{c}{Fold 5} \\
        
        \cmidrule{2-11}
        &
        SS & SN & ED & ZD & TE & OE & K2 & K3 & RS & PC \\
        &
        mIoU ↑ & mErr ↓ & RMSE ↓ & RMSE ↓ & RMSE ↓ & RMSE ↓ & RMSE ↓ & RMSE ↓ & RMSE ↓ & RMSE ↓ \\
        
        \midrule
        Ours w/o Matching &
        0.2681 & 13.0704 & 0.1111 & 0.0404 & \textbf{0.0778} & 
		0.1061 & \textbf{0.0613} & 0.0537 & 0.1559 & 0.0445 \\
        
        Ours w/o Adaptation &
        0.0002 & 23.4212 & 0.1515 & 0.0641 & 0.1513 & 
		0.1152 & 0.1110 & 0.0625 & 0.2033 & 0.0632 \\
        
        Ours &
        \textbf{0.4097} & \textbf{11.4391} & \textbf{0.0741} & \textbf{0.0316} & 0.0791 & 
		\textbf{0.0912} & 0.0639 & \textbf{0.0519} & \textbf{0.1089} & \textbf{0.0420} \\
        
        \bottomrule
        
    \end{tabular}
\end{center}
\vspace{-0.1cm}
\end{table}

As shown in the first group of~\cref{Tab:ablation_table}, we can conclude that as the type starts absorb later (from top to bottom in the table), the overall generation performance becomes better. Specifically, with $\Tilde{T}$ for the late absorb decreases, the quality of the generated layouts gets worse (as suggested by mIoU, Align., and FID). When it comes to early absorb, the performance drops as $\Tilde{T}'$ decreases.
The experiment empirically supports the insight we discuss above.


\subsection{Ablation Studies on Sequence Ordering}
In the main paper, we sort the layout sequence according to the alphabetical order of the elements' type in the layout (denoted as ``lexico"). Other choices are the positional ordering of the elements' bounding boxes (denoted as ``position") or simply randomly sorting the elements (denoted as ``random"). 
Besides, for each element in the layout, we represent its bounding box by the left, top, right, bottom coordinates (denoted as ``ltrb'').
One can also represent the bounding box using an element's left coordinate, top coordinate, width and height (denoted as ``ltwh'').
Here, we provide the results of all these options for sequence ordering.

As shown in the second group of~\cref{Tab:ablation_table}, for the format representing the coordinates, the ltrb group exhibits better overall performance than the ltwh group. One explanation is that ltrb format may provide more straightforward information for the precise alignment of the bounding boxes. 
For the ordering of elements, the alphabetical ordering and positional ordering are slightly better than the random ordering. We hypothesize that the model can exploit the additional ordering information with positional embedding. 
However, note that for conditional generation, the positional ordering of the elements is unknown, so to enable both unconditional and conditional generation, we apply the alphabetical ordering to sort the elements.





\section{Qualitative Results of LayoutDiffusion}
\label{sec:suppl_quali}
In this section, we provide more generated samples covering three unconditional and conditional tasks on two datasets. 
\subsection{Unconditional Generation}
\begin{figure}[H]
  \centering
   \includegraphics[width=0.9\linewidth]{cvpr2023-author_kit-v1_1-1/latex/suppl_pics/ungen_rico.png}
   \caption{Examples of unconditional generation on RICO dataset.}
   \label{ungen_rico}
\end{figure}
\begin{figure}[H]
  \centering
   \includegraphics[width=0.9\linewidth]{cvpr2023-author_kit-v1_1-1/latex/suppl_pics/pub_ungen.png}
   \caption{Examples of unconditional generation on PublayNet dataset.}
   \label{ungen_pub}
\end{figure}


\subsection{Refinement}

\label{more refine}
\begin{figure}[H]
  \centering
   \includegraphics[width=0.925\linewidth]{cvpr2023-author_kit-v1_1-1/latex/suppl_pics/rico_refine.pdf}
\caption{Examples of refinement on RICO. The left side of each pair is the input layout while the right side is the generated layout.}
   \label{refine rico} 
\end{figure}

\begin{figure}[H]
  \centering
   \includegraphics[width=0.925\linewidth]{cvpr2023-author_kit-v1_1-1/latex/suppl_pics/pub_refine.pdf}
\caption{Examples of refinement on PublayNet. The left side of each pair is the input layout while the right side is the generated layout.}
   \label{refine pub}
\end{figure}


\subsection{Generation Conditioned on Type}
\label{more gen-type}
\vspace{-10px}
\begin{figure}[H]
  \centering
   \includegraphics[width=0.9\linewidth]{cvpr2023-author_kit-v1_1-1/latex/suppl_pics/type_rico.pdf}
\caption{Examples of Gen-Type on RICO. Each given type set corresponds to four generated layouts.}
   \label{type rico}
\vspace{-5px}
\end{figure}

\begin{figure}[H]
  \centering
   \includegraphics[width=0.93\linewidth]{cvpr2023-author_kit-v1_1-1/latex/suppl_pics/type_pub.pdf}
   \vspace{-5px}
\caption{Examples of Gen-Type on PublayNet. Each given type set corresponds to four generated layouts.}
   \label{type pub} 
\end{figure}

\section{Fine-Grained Visualization of the Forward Diffusion Process}
\begin{figure}[H]
  \centering
   \includegraphics[width=0.925\linewidth]{cvpr2023-author_kit-v1_1-1/latex/suppl_pics/noise_steps.jpg}
   \caption{An example of the forward process of LayoutDiffusion on RICO. We sample 99 steps uniformly from the total 200 timesteps.}
   \label{noise steps} 
\end{figure}

\begin{figure}[H]
  \centering
   \includegraphics[width=1.01\linewidth]{cvpr2023-author_kit-v1_1-1/latex/suppl_pics/forward_suppl.pdf}
   \caption{Examples of the forward process of LayoutDiffusion on RICO. We sample 22 steps uniformly from the total 200 timesteps.}
   \label{noise steps more} 
\end{figure}

\section{Fine-Grained Visualization of the Reverse Generation Process}
\begin{figure}[H]
  \centering
   \includegraphics[width=0.925\linewidth]{cvpr2023-author_kit-v1_1-1/latex/suppl_pics/gen_steps.jpg}
   \caption{An example of the reverse process of LayoutDiffusion on RICO. We sample 99 steps uniformly from the total 200 timesteps.}
   \label{reverse steps} 
\end{figure}

\begin{figure}[H]
  \centering
   \includegraphics[width=1.01\linewidth]{cvpr2023-author_kit-v1_1-1/latex/suppl_pics/reverse_suppl.pdf}
   \caption{Examples of the reverse process of LayoutDiffusion on RICO. We sample 22 steps uniformly from the total 200 timesteps.}
   \label{reverse steps more} 
\end{figure}

\end{document}
