\section{Discussion}
\label{sec_discussion}
Having shown a savings potential through price responsive \loadshifting, the following topics deserve attention: the step from simulation to reality, potential performance improvements, control of the \hp, heat scheduling using an average indoor temperature and minimizing operation costs rather than discomfort or indirect \coTwo-emissions.

A large amount of papers assume perfect forecasting when conducting simulation studies of \MPC\ with the consequence that results reflect upper performance boundaries. This is avoided in a real implementation. Nevertheless, the problem then shifts to estimating the true cost reduction or \savingRateText. \figref{fig_savings_development} shows the extent of the challenge since the \savingRateText\ has not converged after 10-12 days. Even after 55 days this is not fully the case. The reason is the high volatility in daily savings and losses which are in the range $\pm$\euro3. The implications are that short-term studies (of the order of days) are at risk of reporting \savingRateText s which diverge severely from the true rate. If the so called ``File Drawer Effect'' (Failing to publish negative results) is at play, the bias might be towards too high savings potential. The strategy applied here is to rely on benchmark data collected from the prior heating season. However, even with a full season of data, there are holes in the coverage, meaning that there are experiment days without counterparts in the benchmark dataset. Ideally, the sensing equipment needs to be installed several seasons before the experimental controller is applied in order to have a reliable dataset.
%Relying on forecasts, a weakness of the \MPC\ controller has been noticed. It has issues dealing with sunny days, even though the predictive nature should make the \MPC\ controller superior. \figref{fig_cost_compare} clearly reveals that days with significant loses tend to have high sun intensity. 
%But not only does the controller perform worse, the many random variables make it difficult to estimate the true savings potential of a particular algorithm. Our strategy has been to collect benchmark data for a full heating season and use it for the evaluation, and test for an extensive period. 

The main focus areas for improved performance are \hp\ control and modelling. The performance of the \MPC\ was degraded by problems listed below.% with start-up timing, slow reference control and unanticipated shut-downs. 
%\subsection{Unsolved control challenges and suggested solutions}
%\label{control_challenge}
%Having a controller running in a real setting, it is often that problems and challenges are discovered which are overlooked in simulation. This section briefly touches upon some practical lessons learned by the authors.
%This section presents some practical challenges and problems, which are typically neglected in simulation studies.

%During the test campaign several issues have been discovered with some largely solved but others need further attention. This section provides an overview the problems\\\\
\begin{itemize}
    \item A side effect of using the compressor block function is that the \hp\ attempts to heat the \DHW\ using the electric heating rod, which has a power output of \SI{10}{kW}. This is far from ideal, since just a few minutes in this state is costly. \textbf{Suggested solution}: block the compressor for space heating only.
    \item When the \hp\ defrosts, the measured heat flow reverses. Any controller regulating the heat output needs to be able to detect and handle such a situation. \textbf{Suggested solution}: put  the control in standby mode.
    \item It is not possible to start the heat pump on demand, the only option is to release the compressor brake and wait. The waiting time is observed to be between 60 and 120 min. \textbf{Suggested solution}: adapt the block release for best start-up timing or introduce/use open HP controller standards
    \item In certain situations the heat pump shuts down before it should. It is assumed that a combination of high ambient temperature and low flow caused the internal controllers to shut it down. \textbf{Suggested solution}: use data to figure out what events cause a shut down.
    \item A low pass filter and other unknown internal states make control through ambient temperature overwrite particularly challenging. \textbf{Suggested solution}: use a heat pump with reference control for heat.
\end{itemize}
%A side effect of using the compressor block function is that the \hp\ attempts to heat the \DHW\ using the electric heating rod which have a power output of 10 kW. This is far from ideal, since just a few minutes in this state is costly. \textbf{Suggested solution}: Block the compressor for space heating only.

%When the \hp\ defrosts the measured heat flow reverse. Any controller regulating the heat output needs to be able to detect and handle such a situation. \textbf{Suggested solution}: Pause control.

%It is not possible to start the heat pump on the demand, the only option is to release the compressor brake and wait. The waiting time is observed to be between 60 and 120 min. \textbf{Suggested solution}: Adapt the block release for best start-up timing.


%In certain situations the heat pump shut down before this being requested. It is assumed that a combination of high ambient temperature and low flow caused the internal controllers to shut it down. \textbf{Suggested solution}: Use data to figure out what events causes a shut down.

%The input-manipulation is performed through a filter with a large time-constant making control and other internal states are unknown making control unnecessarily difficult.
All of these effects could have been prevented if the \hp\ had an interface for set-point control. The path forward for commercial heat pumps should be to provide an interface for reference control which would allow the \hp\ to operate in a near-optimal state while being part of a coordinated and cooperative control scheme.%A robust scheme should be derived for night heating to prevent overdosing in cases where sun intensity have been severely underestimated.


Using an averaged room temperature in the upper control level has proven to be completely viable with respect to comfort. Controlling this way does conflict with the idea that each room should be controlled strictly after individual references, but it is our experience that large temperature differences within the thick shell of a low energy house are difficult to obtain in any case.

Given that prices are the result of market mechanisms rather than purely physical processes, it is exceedingly difficult to forecast future prices. This controller can become more efficient if the daily price-volatility increases, on the other hand, lower prices can make the controller superfluous. This said, low prices are good for the consumer, so one might think of the controller as an insurance against long periods of high, volatile prices.

Lastly, a look at some economic aspects with respect to the current price situation and controller performance. If the heating season lasts 4 months ($\approx$ 120 days), an average cost reduction of \currency1 per day would translate into \currency1200 over a 10 year period.Although it is hard to predict, this could be a reasonable price for the controller. The \MPC\ yielded a reduction of \currency\savingRate\ per day, meaning that it would take \yearsEarning\ years to save \currency1200 under current price conditions. It has to be noted that this case involves a low energy house, and this calculation is not meant to be extrapolated to less energy efficient houses. 

The analysis of the savings potential for the \MPC\ controller would be incomplete if the effect of the relatively large \cookingPeak\ price is ignored. The daily recurrence of the \cookingPeak\ tariff begs the question: \textit{What is the \savingRateText\ from simply blocking the \hp\ in the \cookingPeak\ period?} Using the method from section \ref{sec_validation}, the benchmark controller has used an extra \extraPeakElec\ \kwh\ electricity in the timespan 17:00-21:00 translating to an extra cost of \currency\extraPeakCost\ over the test period compared to the \MPC\ approach. Postponing the electricity consumption to the hours following the \cookingPeak\ would cost \currency\costAfter, based on the average price between 21:00 and 01:00, resulting in an overall reduction of \currency\savePeakBlock. This is \percPeakSave\% of the estimated savings provided by the \MPC. Two things have to be noted, the peak block saving assumes a \cop\ of 4.2, which can only be achieved at moderate heating loads, and the cost reduction of the experiment is calculated based on all comfort levels.