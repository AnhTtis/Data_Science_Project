\section{Interpretation of the results}
\label{sec_result_interpret}
In this section the authors provide their interpretation of the data and results presented in the former section. Starting with Table \ref{table_cmp_cost}, where comfort levels 1 and 2 (\comfLvlTriangleOne\ and \comfLvlTriangleTwo, respectively) show a clear percentage-vise savings potential. At comfort level 1, the indoor temperature was uncomfortably low, so this result is ignored. Test period 2 (\comfLvlTriangleTwo) is more interesting since the residents experienced a comfortable indoor climate while saving on heating costs. This raises the question: \textit{Did the price responsiveness cause the economical
savings?} The answer is unknown since the lower average indoor temperature, and thereby lower heat demand, could have been the main reason. The main take-away from comfort level 2 is that even a \NZEB\  can gain by lowering indoor temperature. Test period three (\comfLvlTriangleThree) was executed with an average indoor temperature of \SI{0.2}{\degC} higher than the one of the benchmark data, meaning that the \comfThreePercSave\% savings are likely to be contributed to the controller.

Having established that overall savings are possible under the current Danish price scheme, the next part investigates situations which are particularly favorable or unfavorable for the controller. 
%This is unfortunate, but it is also interesting for several reasons. Despite the \MPC\ breaking even with respect costs there are situations or events where it is able to generate savings consistently and situations with consistent losses.
Before reading on, keep in mind that \textit{large savings can only originate from situations with large potential costs}. For the analysis Figs.~\ref{fig_avg_temp_1}, \ref{fig_cost_compare}, \ref{fig_savings_analysis}, \ref{fig_cop_investigation_2}, Table \ref{table_cmp_electricity} and \ref{table_cmp_energy} are used. \figref{fig_cost_compare} reveals that the largest share of consistent savings are generated between 0.7 and \SI{4.0}{\degreeCelsius}. The large loss seen at \SI{1.1}{\degreeCelsius} is the transition day between test period 2 and 3 where extra energy was needed to lift the average indoor temperature. The lower ambient temperature increases the demand for heat which increases the cost. However, as this is true for all buildings in the region not only the consumption dependent costs are driven up, so are the electricity prices. This is visible in the upper graph of \figref{fig_savings_analysis}, where the average daytime price is inversely correlated with the temperature. The result is that heat demand and price amplifies each other which increase the daily cost dramatically when the ambient temperature is below \SI{5}{\degreeCelsius}.

Having established the factors driving up costs, we turn the attention to daily price variation which also impacts the potential for cost savings, see the lower graph in \figref{fig_savings_analysis}. Although, the results are more scattered than in the upper graph, three trends can be seen. First, the day/night price-ratio is more likely to be higher at high ambient temperatures. Second, at ratios above three, the controller is likely to save money, albeit these are mostly warm low cost days. Third, the most interesting trend is the range 0 to \SI{5}{\degreeCelsius}, where the ratio often was above 2, securing significant percentage-vise savings.

At this point the price conditions for savings are established. Hence, we return focus to test periods 3 and 4 (\comfLvlTriangleThree\ and \comfLvlTriangleFour, respectively) and ask: \textit{Is the potential for savings larger than presented here?} The controller responded to the price signal by changing the daily heating pattern significantly, as seen in \figref{fig_production}. Several things impact savings: model errors, forecasts errors, lack of controller robustness and more. This said, the loss of \hp\ efficiency, mentioned in Section \ref{sec_heat_pump_efficiency}, stands out as a major plausible limiter. % Figure \ref{fig_cop_investigation_2} shows the initial fit (dotted lines) against the new fit obtained and implemented a few weeks after starting the controller.The controller shows a noticeable fall in efficiency.This means that the controller not only loses efficiency by operating the compressor at higher loads it also loses efficiency by controlling it differently than default mode, which is a large disadvantage.
The efficiency loss originates not only from the higher loads but also due to the dynamic operations. The loss from dynamic operations puts load shifting at a disadvantage. When the supervisory controller calculates the heating plan it also considers the continuous approach featured by the benchmark controller but discards it as inefficient compared to the night heating approach. This happens because the controller relies on the \hp\ efficiency model, which does not inform that the default controller---the one from the manufacturer---can operate the \hp\ more efficiently. This can be used as a critique of the presently implemented heat controller, yet, it can also be posed as a question to why the manufacturers of domestic \hp's do not let them be controlled according to a heat reference as an alternative to the ambient temperature heating curve.

A weakness has been noticed in the \MPC s reliance on forecasts. The procedure has issues dealing with sunny days, even though the predictive nature should ensure superiority of the \MPC. \figref{fig_cost_compare} clearly reveals that days with significant loses tend to have high sun intensity. 
%In Figure \ref{fig_cost_compare} it is clear that the \MPC\ consistently generate loses on sunny days. 
We expect this to be due a combination of several factors which coalesce with unfortunate outcomes. The low electricity prices invite the controller to boost heating intensely between 00:00 and 06:00 to avoid electricity consumption during more expensive day hours. If the model and forecasts were perfect the heat would be boosted accurately. However, in practice an overheating event is likely to occur if a thick cloud cover is wrongfully predicted, and intense sunshine happens instead. The cloud cover data from the weather service has several times been unreliable even at time-of-use. This effect can be mitigated by correcting the forecast with live \pv\ data. Further, a robust control approach which restrains night boosting slightly should be applied.
We provide some comments on the growth conditions which constituted the majority of our analysis in sections \ref{sec:Hmixing} and \ref{sec:Hsigma}. In the simplest cases of Lemma \ref{lemma:unstableGrowth}, growth was established in an analogous fashion to the old one-step expansion condition (\ref{eq:oldOneStepExpansion}), finding the relevant Jacobians $M_j$ and checking that their expansion factors $K(M_j)$ satisfy
\begin{equation}
    \label{eq:discussionOneStep}
    \sum_j \frac{1}{K(M_j)} <1.
\end{equation}
For the more complicated cases, the inductive method used to establish growth near the accumulation points in Lemma \ref{lemma:unstableGrowth} and the weakened one-step expansion condition (\ref{eq:oneStep}) both address the same fundamental issue: the splitting of unstable curves by singularities into an unbounded number of small components. They circumvent this obstacle in rather different ways, however. While (\ref{eq:oneStep}) generalises (\ref{eq:discussionOneStep}) to ensure an growth of unstable curves `on average' (see \cite{chernov_statistical_2009} for a precise statement), our inductive method is a more direct adaptation of (\ref{eq:discussionOneStep}), using it to generate contradictory geometric conditions which a hypothetical non-growing unstable curve must satisfy. It may be possible to prove Theorem \ref{sec:Hmixing} using (\ref{eq:oneStep}) as the basis for growth. Since we required (\ref{eq:oneStep}) anyway for proving Theorem \ref{thm:HsigmaExp}, this could potentially condense our analysis, but only to a minor extent. A convenience of the method used in section \ref{sec:Hmixing} is that, by way of the `simple intersection' property, it naturally gives geometric information on the images of manifolds, useful for proving the property \textbf{(M)} of Theorem \ref{thm:katok-strelcyn}.

We expect that essentially analogous analysis can be applied to establish mixing properties in a wide class of piecewise linear non-uniformly hyperbolic maps, including those (like the OTM) which sit on the boundary of ergodicity and beyond. While we have relied on the precise partition structure of $H_\sigma$, its fundamental feature (self-similar sequences of elements $A^k$, sharing boundaries with its neighbours $A^{k-1},A^{k+1}$ and accumulating onto some point $p$) is quite typical to return map systems. See, for example, those of various stadium billiards \cite{chernov_chaotic_2006,chernov_improved_2008,chernov_statistical_2009} and LTMs \cite{springham_polynomial_2014}. Indeed, the same method can be used to prove the Bernoulli property for non-monotonic LTMs \cite{myers_hill_mixing_2022}, where monotonicity of the manifold images cannot be assumed and the classical argument \cite{sturman_mathematical_2006} fails. The OTM is the pointwise limit of these maps as the boundary shrinks to null measure. It further has utility in proving growth conditions for maps which are uniformly hyperbolic but possess regions $A_j$ where the hyperbolicity is very weak, signified by $K(M_j) \approx 1$, so that (\ref{eq:discussionOneStep}) fails. Typically this leads to suboptimal bounds on mixing windows, see e.g. \cite{wojtkowski_model_1981,przytycki_ergodicity_1983,myers_hill_family_2022}. The map $H_{(\eta,\eta)}$ for $\eta \approx 1/2$ is another example, possessing weak hyperbolicity over $A_2, A_3$. Letting $\varepsilon = |\eta-1/2|>0$, there is an upper bound $N = N(\varepsilon)$ on escape times from the intersections $A_2\cap \sigma, A_3 \cap \sigma$. The growth lemma then follows by applying the inductive step roughly $N$ times and can be established for arbitrarily small $\varepsilon$, opening the door to establishing optimal mixing windows.

The above gives two examples of piecewise linear perturbations to $H$ where mixing with respect to Lebesgue is preserved and our methods can be applied. Nonlinear perturbations to the shear profiles complicate the analysis in several ways. Firstly as the map's Jacobians takes on a broader range of values, cone invariance becomes an increasingly harder condition to establish. Cones must be widened, giving looser bounds on expansion factors, which may already be weak due to new regions of weaker stretching. This, together with the change from polygonal to curvilinear return time partition elements and nonlinear local manifolds, adds some complexity to showing growth conditions. This does not rule out certain (small) nonlinear perturbations however. There is some leeway in the inequalities which govern cone invariance and growth of local manifolds, the latter of which is not too dissimilar from the piecewise linear setting (see Lemmas \ref{lemma:piecewiseApprox}, \ref{lemma:componentLength}). Certain small perturbations would not alter the \emph{topological} structure of the return time partition, i.e. which elements share boundaries, the key information needed for setting up the induction. Finally while the partition elements would no longer be polygonal, only coarse geometric information is required for verifying each inductive step. Following the above, a potential perturbation could be to replace the linear portions of each shear by a cubic, perturbing the tent profile
\[  f(t) = \begin{cases} 2t & 0 \leq t \leq 1/2, \\ 2(1-t) & 1/2 \leq t \leq 1 ,\end{cases} \]
of the OTM shears to
\[  f_a(t) = \begin{cases} \frac{1}{8} t \left(16 - a + 6at - 8at^{2} \right) & 0 \leq t \leq 1/2, \\ \frac{1}{8}\left(1-t\right)\left( 16 - a + 6a\left(1-t\right) - 8a\left(1-t\right)^{2}\right)  & 1/2 \leq t \leq 1, \end{cases}   \]
for $a>0$. For small enough $a$ the gradient range $f'(t)$ is restricted to small neighbourhoods of $\{ 2, -2\}$ and the escape time partition retains a similar structure. We illustrate this in Figure \ref{fig:perturbations}, showing escapes from the square $S_3$ under the map $G \circ F$, equivalent to escapes from the perturbed $A_3$ under the $G \circ F$, but with a cleaner geometry for comparison. When $a$ is too large the analogy to the OTM breaks down. At $a=16$ the map is twice differentiable everywhere and features a new source of slowed mixing, the Jacobian is the identity at the corner points $x,y \in \{  0, 1/2 \}$ giving locally parabolic behaviour (visible in the escape time partition). 

\begin{figure}
    \centering
    \includegraphics[width=0.24 \linewidth]{0.png}
    \includegraphics[width=0.24 \linewidth]{4.png}
    \includegraphics[width=0.24 \linewidth]{8.png}
    \includegraphics[width=0.24 \linewidth]{16.png}
    \caption{Partition of escape times from $S_3$ under the mapping $F \circ G$ for $a= 0,4,8,16$. }
    \label{fig:perturbations}
\end{figure}

\section{Conclusion}
\label{sec_conclusion}
During this study an implementation oriented, price-responsive \MPC\ controller has been tested on a commercial \hp, over the course of four months in the winter 2022-2023. The results show that \loadshifting\ can reduce heating costs by at least \comfThreeCostPerc\%, only by activating the heat capacity of the building structure and without reducing the indoor temperature. The production patterns have been shifted to support the grid through increased consumption at night and by shutting the \hp\ down in the \cookingPeak. Further, it has been established that under the current danish price scheme the \cookingPeak\ is the decisive cost factor, and about \percPeakSave\% of the savings provided by the \MPC\ can be obtained just by blocking the \hp\ in the \cookingPeak. Full or partial shutdown in the \cookingPeak should immediately be broadly implemented. This rule creates correlated consumption patterns, which might become problematic for the grid later. In case the grid operators wish to use more coordinated approach controllers of the type presented here are needed, but, at the moment the cost reduction obtained from price responsiveness cannot cover the costs of acquiring such capabilities, so more financial incentive needs to be provided.

The ambient temperature overwrite applied to control the heat flow of the \hp\ has proven to be a functional but inefficient way to make the it \smartGrid-ready. A dedicated input for reference control as a standard is to be desired if advanced control of \hp s should be the norm. 

Several publications have suggested that the upper layer, in a hierarchical control structure can be controlled using a building model having only one heating zone without degrading indoor comfort. We can report that the results presented here support this idea. Although, it has to be mentioned that the highly insulated shell of the house might me a large contributor.

Future work is to automate the process of gathering quality data from the sensors and apply an update the models for building, \hp\ and \pv\ regularly. The next natural step for the \MPC\ is to upgrade the \hp\ retrofit to include control of domestic hot water production which, at this moment, is a randomly occurring process, often taking place in the \cookingPeak.% Further, the house should be equipped with a battery to study the difference between this \MPC\ and one which features an energy storage medium.