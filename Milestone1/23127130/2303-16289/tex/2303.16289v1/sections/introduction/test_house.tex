% \section{Case study}
% \label{sec_test_house}
% The case study house is a \qty{230}{\meter\squared}, two-story \singleFam\ house from 2018; see \figref{fig_blueprint}. According to the Danish building regulation, it is classified as a low energy class building (BR2020), which, among other requirements, implies a maximum annual heat demand of \qty{20}{kWh\per\meter\squared} \cite{br18}. It is located on Sjælland (Zealand) in Denmark, with a south view over the sea. A south facing photovoltaic system is placed on the roof with a measured peak output of \qty{4}{kW} in end of December and \qty{5.5}{kW} in June. Space heating and domestic hot water is provided by a \textit{Bosch Compress 7000i} AW (air-to-water) heat pump. Domestic hot water takes priority over space heating. Based on measured data the nominal electric consumption ranges from \qty{200}{W} to \qty{2500}{W}. Floor heating, embedded in concrete, is installed throughout the house. The floor heating system is controlled by a Wavin controller and consists of 15 circuits delivering heat to 11 heating zones. Each zone has one thermostat assigned, meaning that if more circuits are supplying the same zone all valves in the particular zone opens when heat is requested. The circuits are ON/OFF controlled based on deviations from the temperature reference provided for each zone. The heat pump is controlled by an ambient temperature compensated heat curve. The household has a variable electricity price contract, which is based on the \nordpool\ market spot-price.
% \begin{figure}[h]
% 	\centering
% 	\includegraphics[width=.8\columnwidth]{figs/blueprint.pdf}
% 	\label{fig_blueprint}
% 	\caption{Blueprint of the floor plan with area distribution. Blue numbers indicate the number of \fh\ circuits in each zone}
% \end{figure}