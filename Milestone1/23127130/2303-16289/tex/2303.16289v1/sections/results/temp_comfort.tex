\subsection{Temperature comfort}
\label{sec_temp_comfort}
The temperature distributions in \figref{fig_avg_temp_1} show that the indoor temperature has not been impacted by the \MPC\ controller providing price-led load shifting. This is particularly clear in the lower left and right plot which shows the distribution of minimum and maximum room temperatures, respectively. The min./max. room temperature are defined as $\min/\max\left(\Tri{1}(t), \dots, \Tri{\numRooms}(t)\right)$. The lower minimum temperature is caused by one room where the reference was set to 19 \degC.

Since the house was occupied throughout the test, the residents were sent a questionnaire about the experienced indoor climate on the $11^\text{th}$ Jan. 2023. The questions and answers can be read in Appendix \ref{app_resident_statement}.

Although each room has an assigned temperature reference, not much attention has been given to individual rooms besides responding to complaints, which  was only necessary once, at comfort level 1. Two rooms, hobby and bedroom, had reference settings at 19 and 21 \degC, respectively, and the rest had 22.5 \degC. The hobby room is partly detached from the rest of the house, and it was thus easy to keep the temperature low. The bedroom could not be kept at 21 \degC, even though the floor heating circuit was seldom on. This shows, as pointed out by \cite{vogler-finck_inverse_2019}, that it is difficult to maintain large discrepancies between room temperatures within a \NZEB.
% \begin{figure}[h]
% 	\centering
% 	\includegraphics[width=\columnwidth]{figs/avg_room_temps.pdf}
% 	\caption{\figUpper\ Shows the distribution of mean indoor temperature compared to the benchmark period. \figLowerLeft\ Shows a histogram of the lowest measured temperature in the rooms. \figLowerRight\ Shows the highest measured temperature in the rooms.}
% 	\label{fig_avg_temp}
% \end{figure}
% \subsubsection*{Resident statement}
% The $11^\text{th}$ Jan. 2023, the residents were sent a questionnaire about the experienced indoor climate comfort. Before reading the questionnaire, note that it was not conducted anonymously. The questions and answers are as follows.
% \begin{enumerate}
%     \item \textit{Have you experienced an increase in discomfort with respect to indoor climate when you compare with last heating season?}
%     \begin{description}
%     \item [Answer:] I actually think that we have felt a better comfort than I remember from last year. There was an experience of discomfort in the beginning of the experiment, which we talked about, thereafter it has been quite comfortable.
%     \end{description}
%     \item \textit{When you have experienced discomfort, has it been too warm, too cold or do you experience both too warm and too cold periods?}
%     \begin{description}
%     \item [Answer:] No, we don't have any periods with unpleasantly low temperatures. We have as always lower temperatures in the cinema/hobby-room, but that has been fine.
%     \end{description}
%     \item \textit{Is there any time of day where the discomfort most often occurs?}
%     \begin{description}
%     \item [Answer:] No comment
%     \end{description}
%     \item What have you noticed with respect to floor temperatures?
%     \begin{description}
%     \item [Answer:] It has actually been pleasantly warm, and I don't think that we have experienced cold floors, which we typically experience at middle-high outdoor temperatures.
%     \end{description}
%     \item Have you experienced that the radiation from the floors have been to high?
%     \begin{description}
%     \item [Answer:] No
%     \end{description}
%     \item Have you experienced that the radiation was too low? A feeling of being cold even though the room temperature was high.
%     \begin{description}
%     \item [Answer:] No, as said, we have not experienced that for long.
%     \end{description}
% \end{enumerate}