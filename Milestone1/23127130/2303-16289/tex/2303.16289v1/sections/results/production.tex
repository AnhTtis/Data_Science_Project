\subsection{Space-heating production patterns}
\label{sec_production}
In this section the daily heat production patterns are presented and compared to the benchmark data. The upper graph in \figref{fig_production} shows the daily heat production curves normalized with respect to part of the full day. Lower shows the actual hourly production in kWh.
\begin{figure}[H]
	\centering
	\includegraphics[width=\columnwidth]{figs/production_pattern_Qhp_kWh.pdf}
	\caption{\figUpper\ Distribution of daily production patterns from the benchmark controller compared with the \MPC\ controller. The y-axis is the hourly fraction of total production for a given day. \figLower\ Actual hourly heat production}
	\label{fig_production}
\end{figure}
The heat production has increased remarkably at night meaning that the controller bets against the classic night setback strategy despite it generally being colder at night causing the \hp\ to be less efficient. The midday production has increased, but most notably is the complete lack of heating in the \cookingPeak\ period between 17:00 and 21:00.
%\subsection{Cooking peak cost}
% \newcommand{\extraPeakCost}{31.2}
% \newcommand{\extraPeakElec}{52.2}
% \newcommand{\avgEveningPrice}{0.28}
% \newcommand{\copPeak}{4.2}
% \newcommand{\costAfter}{14.8}
% \newcommand{\savePeakBlock}{16.4}
% The analysis of the savings potential for the \MPC\ controller would be incomplete if the effect of the relatively large \cookingPeak\ price is ignored. The daily recurrence of the \cookingPeak\ tariff begs the question: \textit{What is the \savingRate\ from simply blocking the \hp\ in the \cookingPeak\ period?} Using the method from section \ref{sec_validation}, the benchmark controller has used an extra \extraPeakElec\ \kwh\ electricity in the timespan 17:00-21:00 translating to an extra cost of \currency\extraPeakCost\ over the test period compared to the \MPC approach. Postponing the electricity consumption to the hours following the \cookingPeak\ would cost \currency\costAfter, based on the average price between 21:00 and 01:00, resulting in an overall reduction of \currency\savePeakBlock. This is \percPeakSave\% of the estimated savings provided by the \MPC. Two things have to be noted, the peak block saving assumes a \cop\ of 4.2, which can only be achieved at moderate heating loads, and the cost reduction of the experiment is calculated based on all comfort levels.