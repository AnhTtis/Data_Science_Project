\section{Experiment description}
\label{sec_experiment}
\newcommand{\numComfLvlWord}{four}
\newcommand{\numComfLvl}{4}
\newcommand{\TaBound}{0.5}
\newcommand{\TaBoundPerc}{5}
\newcommand{\EpvBound}{2.0}
\newcommand{\EpvBoundPerc}{10}
The experiment was conducted over \numTestDays\ days in the period 2022-11-07 to \expEndDate. During the experiment \numComfLvlWord\ combinations of hourly discomfort cost, $\priceComf \in \realv{24}$, and average room temperature reference levels, $\Tref \in \realv{24}$, were applied, see \figref{fig_comfort_cost}. A pair consisting of a temperature reference and a discomfort cost makes out a comfort level. The cost and reference are used in the quadratic cost term $\sum_{i=0}^{23} \priceComfi{i} \left(\Tri{i} - \Trefi{i} \right)$, where $i$ is the hour. %The colors of the curves are referred in the lower plot of Figure \ref{fig_cost_compare} using triangles.
\begin{figure}[ht]
	\centering
	\includegraphics[width=1\columnwidth]{figs/hourly_reference.pdf}
	\caption{\figUpper Hourly mean temperature reference calculated from the collection of zone references. \figLower Hourly virtual discomfort price. Both vectors are used in equation system \eqref{eq_concept_miocp}. Each comfort-cost combination has assigned a colored triangle  ({\color{ref1}\FilledBigTriangleUp}, {\color{ref2}\FilledBigTriangleUp}, {\color{ref3}\FilledBigTriangleUp}, {\color{ref4}\FilledBigTriangleUp}) and denoted a comfort level.} 
	\label{fig_comfort_cost}
\end{figure}
Having \numComfLvlWord\ comfort levels is a result of gradually adjusting the overall average indoor temperature to be similar to the one from the benchmark data in order to reduce a variable with respect to the cost analysis. Consistency of indoor temperature was achieved at comfort level 4 as seen in the upper graph in \figref{fig_avg_temp_1} where the average temperature distributions are plotted. The benchmark and experiment periods are shown in \figref{fig_calendar}.
%Comfort level 1 spanned 13 days (2023-11-07 to 2022-11-19), level 2: 12 days (2022-11-20 to 2022-12-06)), level 3: 55 days (2022-12-07 to 2023-02-01), level 4: \comfFourNumDays\ days (2023-02-02 to \expEndDate).
\begin{figure}[h]
	\centering
	\includegraphics[width=\columnwidth]{figs/data_calendar.pdf}
	\caption{Calendar overview of benchmark data and experiment periods. Green color is benchmark days and other colors are the comfort levels during the experiment.}
	\label{fig_calendar}
\end{figure}
\begin{figure}[h]
	\centering
	\includegraphics[width=\columnwidth]{figs/avg_room_temps.pdf}
	\caption{\figUpper\ Distribution of mean indoor temperature compared to the benchmark period. \figLowerLeft\ Histogram of the lowest measured temperature in the rooms. Dashed line shows the mean temperature. \figLowerRight\ Highest measured temperature in the rooms.}
	\label{fig_avg_temp_1}
\end{figure}
The benchmark dataset used for comparison consists of 193 days with daily mean ambient temperatures in the range $-2.5$ to \SI{15}{\celsius} and daily \pv\ production in the range 0 to \SI{39}{\kilo\watt\hour}. The search bound for finding similar benchmark days are 0.5 \degC\ for ambient temperature making it \TaBoundPerc\% of the full range and \SI{\EpvBound}{\kilo\watt\hour}\ for daily \pv\ electricity production which is \EpvBoundPerc\% of the full range.
%Two low level heat controllers were tested; a reactive PI-controller directly using the heat output as feedback and a linear \MPC\ with a four hour control horizon. Since the heat controller is not in scope of this paper no more information is provided. 

The \hp\ model was re-calibrated twice during the experiment. First time was after a few weeks of running \MPC\ and second the $27^{\text{th}}$ of January. The building model was changed and refitted on the $27^{\text{th}}$ of January.

The optimization problems are implemented using Casadi \cite{Andersson2019} and solved with the mixed integer non-linear programming solver Bonmin \cite{bonmin}.
