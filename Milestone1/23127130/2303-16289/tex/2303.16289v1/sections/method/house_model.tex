\subsection{Single-zone lumped parameter house model}
\label{sec_house_model}
 The single-zone house model, seen in \eqref{eq_heating_model_2order}, has, as argued in \cite{killian_comprehensive_2018}, two dynamic states which describe an averaged room (\Tr) and floor temperature (\Tf). The reason for modelling using only a single zone is given in \cite{vogler-finck_inverse_2019}. Here the affects caused by position of doors and air stratification led the authors to conclude that a single-zone model is as useful as multi-zone model for \MPC. In \cite{amato_dual-zone_2023} a volume weighted average temperature is used since only the central heat meter is available. The purpose of the model is to make the \MPC\ responsive to the impact of high sun intensity, ambient temperature and heat created by household appliances. These three aspects should be included if a forecast is available, otherwise they can be omitted at the cost of increased uncertainty. In this work the forecast for heat produced by household appliances and occupation is left out. The reason is partly technical, but also driven by privacy concerns.
\begin{subequations}
	\label{eq_heating_model_2order}
	\begin{align}
		\Cr\dotTr &= \Ur \left(\Tf - \Tr\right) + \Ur \left( \Tf - \Tr \right) + \dotQsun  \label{subeq_room}\\
		\Cf\dotTf &=  \Ur \left( \Tf - \Tr \right) + \dotQhp
	\end{align}
\end{subequations}
The control input to the model is heat flow $\dotQhp$ measured over the floor heating system. The two-state formulation allows for estimating the overall heat capacity of the building through $\Cf$ and to capture the rapid air temperature changes, caused by sun radiation, in $\Cr$. The state space formulation is given as,
\newcommand{\hModelSpace}{\hspace{0.2cm}}
\begin{subequations}
\label{eq_all_ss}
\begin{align}
	\label{eq_ss_model_final}
	&\dot{\xv}(t) = \Am\xv(t) + \Bm \uv(t) + \Em \dv(t)\\
	&\yv(t) = \Cm \xv(t) \hspace{1cm} \Cm = \begin{bmatrix} 1 & 0\end{bmatrix} 
\end{align}
\begin{align}
	\label{eq_ss_matrices}
	&\Am = \Ccal^{-1}\mathcal{U} \hspace{1cm} \Bm = \Ccal^{-1}\mathcal{B} \hspace{1cm} \Em = \Ccal^{-1}\mathcal{E}\\
	&\Ccal = \begin{bmatrix} \Cr & 0 \\ 0 & \Cf\ \end{bmatrix} \hModelSpace \Ucal = \begin{bmatrix} -\left(\Ur + \Ua\right) & \Ur \\ \Ur & -\Ur  \end{bmatrix} \hModelSpace 
	\Bcal = \begin{bmatrix} 0 \\ 1  \end{bmatrix}\\ &\Ecal = \begin{bmatrix}  \Ua & \sunParam{1} & \sunParam{2}\\ 0 & 0 & 0 \end{bmatrix} \hModelSpace  \xv = \begin{bmatrix} \Tr \\ \Tf  \end{bmatrix} \hModelSpace \uv = \begin{bmatrix}  \dotQhp  \end{bmatrix} \hModelSpace \dv = \begin{bmatrix}  \Ta \\ \forIsun  \\ \forIsundir\end{bmatrix} \label{eq_state_space_end}
\end{align}
\end{subequations}
The input is total heat flow, \dotQhp, and the disturbances are ambient temperature, direct sun irradiation. The common indoor temperature is an area weighted average of all room temperatures,
\begin{align}
    \label{eq_common_Tr}
    \Tr = \frac{\Arj{1}\Tri{1} + \cdots + \Arj{\numRooms}\Tri{\numRooms}}{\Arj{1} + \cdots + \Arj{\numRooms}} 
\end{align}
The power from sun radiation can be estimated in many ways, but is here chosen to be:
\begin{align}
    \dotQsun &= \sunParam{1} \forIsun + \sunParam{2}\forIsundir\\
    \forIsun &= \forIsundir (1 -\forCloudCover)
\end{align}
where \forIsundir\ [\si{\watt\per\meter\squared}] is direct sun and $\forCloudCover$ is the fraction of cloud cover. This particular formulation gives short but intense bursts of sunlight.

The model is discretized using \zoh\ (\ZOH) discretization. Figures of the parameter fits are shown in Appendix \ref{app_house_model}.
%In order to get a finite dimension optimization problem the model needs to be discretized. Since it is linear and the input held steady during the entire control period a straight forward choice is \zoh\ (\ZOH) discretization.
