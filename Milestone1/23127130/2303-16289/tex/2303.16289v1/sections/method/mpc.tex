\subsubsection{Design of supervisory controller}
\label{sec_mpc}
The conceptualized version of the \textit{\miocp} (\MIOCP) at the core of the \textit{\mimpc} (\MIMPC)\ is seen in equation system \eqref{eq_concept_miocp} on a form which describes the functionality of the cost and various constraints rather than the implementation. Note that \eqref{eq_concept_miocp} contains two sub-versions decided by the indicator variable $\delta_{\op}$. It must be stressed that the value of \deltaOp\ is chosen before implementing the problem, it is not an optimization variable.  The difference between the two versions is the \hp\ \efficiencyModel.
\setlength{\jot}{7pt}
\vspace{-0mm}
\begin{subequations}
	\label{eq_concept_miocp}
	\begin{align}
		%&\min_{\Php, z_{\hp}, \delta_{\hp}} \fg(\Pg) + \fcomf(\Tr, \Tref) + c_\slack^\tra \slack\\
		%&J(u_\op^*) = \min_{u_\op} \fg(\Pg) + \Vert\Tr-\Tref\Vert^2_\costRoom \\
		&J(u_\op^*) = \min_{u_\op} \fgrid(\Pg) + \fcoTwo(\Pg) + \fcomf(\Tr, \Tref) \\
	& u_\op = \begin{cases} (\Php, \delta_{\hp}, \slackVar) & \delta_\op = 1 \hspace{0.1cm} \\ (\dotQhp, \delta_{\hp}, \slackVar) & \delta_\op = 0 \end{cases} \label{subeq_concept_input_var}\\
	 &u_\op \in \realv{\Nhor} \times \{0,1\}^\Nhor \times \realv{\Nslack}_+\\
		&\text{s.t.} \nonumber\\
		%& Q_\text{budget} = \sum_{i=0}^{N} z_{\Q,\iterOp}\\
		%&  \Pg^+ = \max\left(\Php - \Ppv + \Papp, 0\right)\\
		&  \Pg = \Php - \forPpv + \Papp \label{subeq_elec_balance}\\
		%& z_{\hp,\iterOp} \geq f(\Phpi{\iterOp},\Tai{\iterOp})-M(1-\delta_{\hp,\iterOp})\\
		& \xv_{\iterOp+1} = \Am\xv_{\iterOp} + \Bm \dotQhpi{\iterOp} + \Em d_\iterOp, \hspace{0.5cm} \Tri{\iterOp} = \Cm \xv_{\iterOp} \label{subeq_conccept_dynamics}\\
		&\text{if} \hspace{2mm} \delta_{\op} = 1 \hspace{2mm} \text{then} \hspace{2mm} \dotQhpi{\iterOp} = \fhpi{\dotQ}(\Phpi{\iterOp}, \Tai{\iterOp})\deltahpi{i} \label{subeq_concept_if_Q} \\
		&\text{if} \hspace{2mm} \delta_{\op} = 0 \hspace{2mm} \text{then} \hspace{2mm} \Phpi{\iterOp} = \fhpi{\pow}(\dotQhpi{\iterOp}, \Tai{\iterOp})\deltahpi{i} \label{subeq_concept_if_P} \\
		& \Phpi{\iterOp} \in  \begin{cases} [\Phpmin,\Phpmax] & \delta_{\hp,\iterOp} = 1 \\ 0 & \delta_{\hp,\iterOp} = 0 \end{cases}\label{subeq_concept_php_range} \\
		&\Delta\delta_{\hp,\iterOp} = \delta_{\hp,\iterOp} - \delta_{\hp,\iterOp-1} \hspace{0.3cm} \Delta \Phpi{\iterOp} = \Phpi{\iterOp} - \Phpi{\iterOp-1}\\
		&\Delta \Phpi{\iterOp} \in  \begin{cases} [\Delta\Phpmin,\Delta\Phpmax] & \delta_{\hp,  \iterOp} = 1 \\ (-\infty,\Delta\Phpmax] & \delta_{\hp,  \iterOp} = 0 \end{cases} \label{subeq_concept_delta_php_set}\\
		 %\label{subeq_concept_delta_php}\\
 		&\Delta\delta_{\hp,\iterOp} = -1 \implies \delta_{\hp,\iterOp+1},\dots,\delta_{\hp,\iterOp+M-1} = 0 \label{subeq_concept_offtime}
	\end{align}
\end{subequations}
The cost function is the sum of three functions. First, a linear term, $\fgrid(\Pg)$, describing the differentiated cost of either importing from or exporting to the electricity grid. The input is consumed electricity from the grid, \Pg, with positive values indicating import. The prices for buying and selling to the grid are given as $\priceElec^+ > 0$  and $\priceElec^- > 0$, respectively. The second term is a self-imposed $\coTwo$-tax. The third is the comfort term which punishes deviations from the desired temperature. Slack variables are used to ensure feasibility. % It places a large cost on all slack variables that have been introduced to ensure repeated feasibility.
Together the terms make out a convex cost-function.
 
The constraint \eqref{subeq_elec_balance} describes the electricity balance were the amount of electricity bought from the grid (\grid) is calculated. Constraint \eqref{subeq_conccept_dynamics} describes the linear dynamics of the house. Constraints \eqref{subeq_concept_if_Q}-\eqref{subeq_concept_offtime} models the properties of the \hp\ presented in Section \ref{sec_hp_object}. Constraints \eqref{subeq_concept_if_Q} and \eqref{subeq_concept_if_P} both describes the \hp\ efficiency, but only one is active dependent on the initial choice of \deltaOp. Constraint \eqref{subeq_concept_php_range} describes the piece-wise function where the compressor either is off, or operating in the range $[\Phpmin,\Phpmax]$. The constraint \eqref{subeq_concept_delta_php_set} limits the rate off change between control periods. To meet the requirement that the \hp\ can be turned off from any operational state, the down rate is set to $-\infty$ when $\deltahpi{\iterOp} = 0$. Last \eqref{subeq_concept_offtime} forces the \hp\ to stay turned off for minimum $M$ sample times. Having described the functionality of the \OP\, the next part focuses on implementation aspects.

The guiding principle for the implementation is that the structure of the problem is convex if the problem is relaxed, meaning that if integer variables are replaced with continuous ones, the problem is convex. The cost function from equation system\eqref{eq_concept_miocp} is implemented as:
%\begin{subequations}
\vspace{-0mm}
\begin{align}
    &J(u_\op) = \priceElec^{-\tra}\Pg + \Delta\priceElec^{+\tra} \Pg^+ + \zcomf + c_\slack^T \slackVar
    % & u_\op = \begin{cases} \Php, \delta_{\hp} & \delta_\op = 1 \hspace{0.1cm} \\ \dotQhp, \delta_{\hp} & \delta_\op = 0 \end{cases}\\ 
    % &u_\op \in \realv{\Nhor} \times \{0,1\}^\Nhor
\end{align}
Here the auxiliary variables $\Pgplus, \zcomf \in \realv{\Nwin}_+$ are introduced. The variable $\Pgplus$ is defined as entry-wise $\max(0, \Pg)$ and \zcomf\ has to be larger than any competing comfort constraints. %The expression $\max(0, \Pg)$ can be converted to linear constraints by introducing a positive auxiliary variable.
%The role of \zcomf\ is explained in more detail in. 
The vector $\Delta\priceElec^{+} = \priceElec^{+} - \priceElec^{-} > 0$ describes the positive difference between buying price and selling price. Note that the buying price needs to be higher than the selling price, otherwise the solution to the \OP\ entails buying excessive amounts of electricity just to sell it again in the same instance. %The buying price $\priceElec^{+}$ is the sum of hourly spot price \priceSpot, electricity tariffs \priceTariff\ and an artificial self imposed \coTwoTax\ \priceCoTwo\ as seen in
%\begin{align}
%     \label{eq_price_buy}
%     \priceElecBuy = \priceSpot + \priceTariff + w_{\coTwo} \priceCoTwo 
% \end{align}
%where \priceSpot\ and \priceTariff\ are given in [euro/kWh] the \coTwoTax is given in [euro/kg], therefor the variable $w_{\coTwo}$ is the hourly estimated \coTwo\ emission in kg per kWh consumed electricity from the grid.\\
% To encode the definition for $\Pg^+$ the constraint in \eqref{subeq_elec_balance_plus} is used, to conclude the price model electricity balance from \eqref{subeq_elec_balance} is included.
% \begin{align}
%     &  \Pg^+ = \Php - \Ppv + \Papp + z_\elec \hspace{0.5cm} z_\elec, \Pg^+ \geq 0 \label{subeq_elec_balance_plus}
% \end{align}
The auxiliary variable \zcomf\ encodes the expression $\max(\fcomfi{1}(\Tr, \Tref), \cdots, \fcomfi{\Ncomf}(\Tr, \Tref))$
% \begin{align}
% %\fcomf = \max(\fcomfi{1}(\Tr, \Tref), \cdots, \fcomfi{\Ncomf}(\Tr, \Tref))
% \max(\fcomfi{1}(\Tr, \Tref), \cdots, \fcomfi{\Ncomf}(\Tr, \Tref))
% \end{align}
where $\fcomfi{i}(\Tr, \Tref)$ with $i \in \{1,...,\Ncomf\}$ is either an affine or quadratic positive definite function. This formulation gives room for skewed functions which can for instance penalize either over- or underheating. Note that the artificial \coTwoTax\ term is not missing, it is merely incorporated into the buying price as described in Section \ref{sec_price_model}.

The \hp\ \efficiencyModel\ in either \eqref{subeq_concept_if_P} or \eqref{subeq_concept_if_Q} is implemented using the known \mld\ technique from \cite{bemporad_control_1999} where an auxiliary variable is introduced $\zhpi{i}$ to either be zero of mirror the value of the function dependent on $\deltahpi{i}$. To preserve convexity of the input set, only an inequality is used instead of the original equality seen in \eqref{subeq_concept_if_P}. if $\deltaOp = 0$ then $\Phpi{\iterOp} \geq \fhpi{\pow}(\dotQhpi{\iterOp}, \Tai{\iterOp})$ and if $\deltaOp = 0$ then $\dotQhpi{\iterOp} \leq \fhpi{\dotQ}(\Phpi{\iterOp}, \Tai{\iterOp})\deltahpi{i}$. The structure of the problem forces the solution onto the curve emulating the equality constraint. When $\deltaOp = 1$, there are a few cases where $\dotQhp$ deviates from the curve to avoid the cost of overheating. To avoid this an equality constraint can be implemented with the added computational cost. 
%If $\deltaOp = 0$, then \Php\ is free to be larger than the efficiency model $\fhpi{\pow}$, but the result becomes equality since other solutions are sub-optimal. If $\deltaOp = 1$ \dotQhp\ is not always equal to $\fhpi{\dotQ}$. There are transitional situations where it deviates, to avoid the cost of overheating. To avoid thislternatively one can implement an equality constraint with the added computational cost.
%Take the case where $\deltaOp = 0$, then \Php\ is free to be larger than $\fhpi{\pow}$, but that is sub-optimal since electricity costs, therefore \Php\ will be equal to $\fhpi{\pow}$. In the case where $\deltaOp = 1$, \dotQhp\ is not always equal to $\fhpi{\dotQ}$, there are situations where it deviates, to avoid the cost of overheating. This is of course sub-optimal, alternatively one can implement an equality constraint with the added computational cost.
The constraint in \eqref{subeq_concept_php_range} is implemented as e.g. in \cite{kuboth_economic_2019,lee_mixed-integer_2019,parisio_model_2014}, so is the constraint in \eqref{subeq_concept_delta_php_set}. The down-time model constraint in \eqref{subeq_concept_offtime} can be implemented as shown in \cite{parisio_model_2014}.
The problem can be summed up to
%\end{subequations}
\vspace{-0mm}
\begin{subequations}
    \label{eq_miocp}
    \begin{align}
        %&\min_{\uv \in \realv{N}, } \xv^\tra\Qm \xv + \uv \Rm^\tra \uv + \cvec^\tra \xv\\
        &\min_{\uv \in \realv{N}, } J(\xv_0, \uv)\\
        &\text{s.t.}\\
        &\SSdx{k}\\
        &\yv_1 = \Cm\xv + \Dm\uv\\
        &\xv_{k} \in  \mathcal{X}\hspace{1cm}\xv_0 = \xv(t)\\
        &\yv_2 \geq \textbf{f}_\text{convex}(\xv, \uv)\\
        &\yv_3 \leq \textbf{f}_\text{concave}(\xv, \uv)
    \end{align}
\end{subequations}
where $J$ is the convex cost function. Section \ref{sec_model} details the models that specify the \MPC\ formulation given here.
