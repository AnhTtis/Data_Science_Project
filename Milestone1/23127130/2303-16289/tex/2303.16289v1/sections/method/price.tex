\subsection{Price model}
The hourly price models for buying and selling electricity from/to the grid is given in \eqref{eq_price_buy} and \eqref{eq_price_sell}, respectively. The models are used in the supervisory controller and for evaluation. %The buying price, $\priceElec^{+}$, is the sum of hourly spot price \priceSpot, electricity tariffs \priceTariff\ and an artificial self imposed \coTwoTax\ \priceCoTwo\ as seen in \eqref{eq_price_buy} and \eqref{eq_price_sell}, 
The price for buying electricity is
\begin{align}
    \price^{+}_{\text{excl. \VAT}} &= \priceSpot + \priceTariff + w_{\coTwo} \priceCoTwo + \priceTransport \nonumber\\ 
    \priceElecBuy &=  \price^{+}_\text{excl. VAT} + 0.25\price^{+}_{\text{excl. \VAT}}
    \label{eq_price_buy}
\end{align}
where the spot price, \priceSpot, distribution tariff, \priceTariff, transport tariff, \priceTransport, and are given in [\si{\text{\euro}\per\kWh}]. The self-imposed artificial \coTwoTax, \priceCoTwo, is given in [\si{\text{\euro}\per\kilogram}]. Hence, the variable $w_{\coTwo}$ is the hourly estimated \coTwo\ emission in kg per kWh electricity. The Danish VAT rate is 25\% of the full price and the \tsos\ (\TSOs) tariff is a fixed rate of \euro 0.02. The selling price model is
\begin{align}
    \label{eq_price_sell}
    \priceElecSell = \priceSpot
\end{align}
The distribution tariffs, \priceTariff, chosen for the test are based on future signaled prices for January 1st, 2023 in Denmark. The exact tariffs vary between Distribution Systems Operators (\DSOs s), but the pattern is low prices at night, a higher daily price with a sharp increase in the cooking peak. The chosen model is inspired by \cite{net_tariffs_cerius}.
\label{sec_price_model}
\begin{align}
    \priceTariff =  \begin{cases} 
        0.027 \text{\euro} & t \in {[00.00,06.00)} \\ 
        0.081 \text{\euro} & t \in {[06.00,17.00),[21.00,00.00)} \\
        0.26 \text{\euro} & t \in {[17.00,21.00)}
    \end{cases}
\end{align}
It is worth noting that the tariffs need to be realistic, since the choice of values has a large impact on savings potential. If an unrealistic price of \euro 10 is used for the \cookingPeak instead of \euro 0.26, the price-aware controller shuts the \hp\ off in this period and gains and unfair advantage over the price-unaware.%, since it inevitably will heat in this period and pay the large price.

The second part of the price model regards \pv\ produced electricity and the impact the \hp\ has on self-consumption. In the test house the electricity is phase-metered, but the exact per phase import and export is unknown since the numbers are aggregated and stored on hourly basis. Since the data is aggregated, the meter is instead treated as a summation meter with one-hour reporting. The netting interval is unknown even though it is important for the measure of import and export, as shown in \cite{ziras_effect_2021}. The available signals are hourly import, $\Eimp(k)$, hourly export, $\Eexp(k)$, hourly production from the \pv, $\Epv$ and consumption from \hp, \Ehp(k). The difference between export and import, seen in \eqref{eq_net_import}, is the net import, $\Delta\Eg(k)$, which is the billable amount. For notational purposes the hour indicator $k$ is implied hence on.
\begin{align}
    \label{eq_net_import}
    &\Delta\Eg = \Eimp - \Eexp
\end{align}
The sun power corrected cost associated with running the \hp\ is then:
\vspace{-1mm}
\begin{align}
    \Ehp^{*} =  \begin{cases} 
        0 & \Delta\Eg \leq 0 \\ 
        \min\left( \Ehp, \Delta\Eg\right) & \Delta\Eg > 0
    \end{cases}
\end{align}
The same amount of available \pv\ produced solar power and consumption is imposed on similar/comparable days (definition in section \ref{sec_validation}), which are used in the controller evaluation. The consumption of similar days is corrected using the difference in \hp\ consumption: 
\vspace{-1mm}
\begin{align}
    \label{eq_hp_diff}
    \Delta\Ehp =  \Ehpcmp - \Ehpexp
\end{align}
with \Ehpexp\ being the \hp\ consumption for the experiment and \Ehpcmp the similar day. The virtual net import increases when the \hp\ consumes more in hour $k$ and vice versa, as seen in \eqref{eq_Eq_corr}.
\vspace{-1mm}
\begin{align}
    \label{eq_Eq_corr}
    \Delta\Eg^{*}(k)  =  \Delta\Eg + \Delta\Ehp(k)
\end{align}
The corrected net import for similar/comparison days is:
\begin{align}
    \Ehp^{*} =  \begin{cases} 
        0 & \Delta\Eg^{*} \leq 0 \\ 
        \min\left( \Ehpcmp, \Delta\Eg^{*}\right) & \Delta\Eg^{*} > 0
    \end{cases}
\end{align}
The idea behind this mode of calculating the \hp\ consumption is that other appliances use the self-produced electricity too, and the \hp\ should ideally consume less than the excess capacity. 


% The self-consumption for hour $k$, $\Eself(k)$, is given as
% \begin{align}
%     \label{eq_self_cons}
%     &\Eself(k) = \Epv(k) - \Eexp(k)
% \end{align}
% The total consumption \Econ\ in the household is then
% \begin{align}
%     \label{eq_consumption}
%     &\Econ(k) = \Eimp(k) + \Eself(k)
% \end{align}
% which can be split into electricity consumed by the \hp, and remaining electric appliances \Eapp
% \begin{align}
%     &\Econ(k) = \Eapp(k) + \Ehp(k)
% \end{align}
% It is the self-consumption which brings the cost down, therefore the self consumption is split between the appliances and the heat pump according to their share of total consumption, and the corrected electricity consumption of the \hp\ is therefore
% \begin{align}
%     \Ehp^{*} (k) = \Ehp(k) - \frac{\Ehp(k)}{\Econ(k)}\Eself(k)
% \end{align}
% which is always positive since $\Eself \leq \Econ$. The desired consequence of this formulation is that it is better to run the heat pump in sun hours when the consumption from other equipment is low.