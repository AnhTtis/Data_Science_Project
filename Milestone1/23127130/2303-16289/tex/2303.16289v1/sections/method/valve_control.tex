\subsection{Valve selector}
\vspace{-2mm}
The valve selector, or dispatcher, is a mixed integer linear programming problem tasked with providing the flow, $\flow$, as requested, $\flow_\reference$, and distribute the water to the most suitable rooms. Note that the Valve selector is only reactive control. The optimization problem is
\label{sec_sub_control}
\newcommand{\valVSpace}{\vspace{0mm}}
\begin{subequations}
	\label{eq_flow_op}
	\begin{align}
		&\min_{\val \in \{0,1\}^M}  J(\val)=\min_{\val \in \{0,1\}^M} c_{\flow} \Vert \flow_\reference - \flow \Vert^2_2 + \priceComf^\tra \val \label{eq_cost}\valVSpace\\
		&\text{s.t.}\valVSpace\\
		& \nomflow = \sum_{i = 1}^{M} \val_i\nomflow_i, \label{constraint_q}\valVSpace\\
		&\flow = c_0 + c_1\nomflow + c_2\nomflow^2 \label{constraint_q_poly}\valVSpace\\
		& \flow \geq \flow_{\min}  \label{constraint_min_q},\valVSpace\\
		&  r_m \geq \sum_{i = 1}^{M} \max(\val_{0,i}-\val_{i}, 0), \label{constraint_limit_closed}\valVSpace\\
		&\val_j = 1, \hspace{1cm} \forall j \in \mathcal{J}.	\label{constraint_force_open}
	\end{align}
\end{subequations}
The cost function is a trade-off between following the flow reference and delivering the heat to the right rooms. The two terms are weighted by $c_{\flow} \in \mathbb{R}_+$ and $\priceComf \in \realv{\Nroom}$. The comfort cost is pre-calculated as $\priceComfi{i} = a(\Tri{i} - \Trefi{i})$ with $a > 0$ such that cold rooms get priority. The expressions \eqref{constraint_q} and \eqref{constraint_q_poly} describe the flow as a function of valve configuration. In \eqref{constraint_q} the flow is a sum of contributions, but since the flow saturates as more valves open a second order polynomial is used in \eqref{constraint_q_poly} to model this effect. The term $\nomflow^2$ seems to deliver a range of square and \bilinear\ terms, which is inconsistent with \MILP. Luckily, $\val_i$ is binary, meaning that $\val_i \val_j$ is an AND statement ($v_i \land v_j$) can be encoded by \mld (\MLD) as a linear inequality \cite{bemporad_control_1999}. The squared terms are unproblematic since $\val_i^2 = \val_i$. Encoding the binary polynomial has a cost in form of added binary auxiliary variables. The added number of binary variables is ${M \choose 2}$, which in this case is 55, making a total of 66 variables. Constraint \eqref{constraint_min_q} forces a minimum flow, \eqref{constraint_limit_closed} limits the number of valves that can be closed in one iteration and \eqref{constraint_force_open} forces circuits open which belong to too cold rooms.