\subsection{Air-to-water heat pump efficiency model}
%The \atow\ \hp\ efficiency model, \fhp, bridges between space heating and the cost of said heating in form of electricity consumption. Although a heat pump is a thermal system with dynamics, its time constants are considered fast enough to be neglected when the sample time is one hour. In an optimization context this requires a relationship between $\dotQhp$ and $\Php$ to be defined. In this section two examples are provided which fit to the framework: one where heat is a function of electricity and one where it is opposite.

%describes the relationship between consumed electricity and produced heat given the ambient temperature.
In order to inform the supervisory controller on the efficiency of the \hp\ a relation between electricity consumption and heat production is formulated. It is based on the formulation provided in \cite{wimmer_regelung_2004}, where the \hp\ efficiency is provided by the Carnot coefficient of performance, $\copcarnot = \frac{\T_\hot}{\T_\hot-\T_\cold}$, and the efficiency of the compressor, \etahp:
\begin{align}
    \label{eq_cop_basic}
    \dotQhp &= \etahp(\Php) \copcarnot \Php \nonumber \\
    &= \cophp \Php
\end{align}
with \cophp\ being the overall efficiency for a given \hp. The expression for heat as a function of electricity (direct way) is denoted as $\fhpi{\dotQ}$ and the reverse way where electricity is calculated from heat is  $\fhpi{\pow}$.
\subsubsection{Heat as a function of electricity: $\fhpi{\dotQ}$}
The expression chosen for the \cophp\ is:
\begin{align}
	\label{eq_regr_cop_2}
	&\cophp = \frac{k}{\Php } +
	\left(\frac{k_0}{\Php } + k_1 + k_2\Php\right)\copcarnot
\end{align}
with
\begin{align}
\copcarnot \equiv \frac{\TF+\SI{273.15}{\celsius}}{\TF-\Ta}
\end{align}
and the requirement that the coefficient $k_2$ is negative and the forward temperature is constant \nomTF. The reason for this choice is that the heat function $\fhp(\Php)$ contains a second order polynomial when the power \Php\ is multiplied onto \eqref{eq_regr_cop_2}:
\begin{align}
	\label{eq_regr_Q_2}
	\dotQhp = k + (k_0 + k_1\Php + k_2\Php^2)\copcarnot
\end{align}
and taking the second order partial derivative of \eqref{eq_regr_Q_2}
with respect to \Php\ shows that 
\begin{align}
	\label{eq_hessian_Q_2}
	\frac{\partial^2\Qfh}{\partial^2 \Php} = k_2\copcarnot < 0
\end{align}
implying that if all other variables are constants then \eqref{eq_regr_Q_2} is concave.
% Since the koefficient $k_2 < 0$ the expression \eqref{eq_regr_Q_2}
%  is concave.
\subsubsection{Electricity as a function of heat: $\fhpi{\pow}$}
The inverse formulation, seen in \eqref{eq_regr_P_2}, where electricity is the dependent variable, is concave if the coefficient $k_2 > 0$. 
\begin{align}
	\label{eq_regr_P_2}
	\Php = k + (k_0 + k_1\dotQhp + k_2\dotQhp^2)\frac{1}{\copcarnot}
\end{align}