\section{Parameter Identification}
\label{sec_parameter_est}
\subsection{parameter estimation for house model}
The common room temperature, seen in \eqref{eq_common_Tr} is an average of all room temperatures weighted by the room areas.
\begin{align}
    \label{eq_common_Tr}
    \Tr = \Arj{1}\Tri{1} + \cdots + \Arj{\numRooms}\Tri{\numRooms} 
\end{align}
The identification for the model is performed in two steps, first, a grey-box estimation of the parameters and then a state space sub space method SSID. An advantage of using only two states for the model is that it is possible to solve for the grey-box parameters although they are not given, and thereby anchor the solution physically. The second important signal is the sun irradiation seen in
\begin{align}
    \Isun = \Isundir (1 -\perc_\cloud)
\end{align}
where \Isundir is direct sun $[W/m^2]$ and $\perc_\cloud$ is the fraction of cloud cover. This particular formulation gives short but intense bursts of sunlight. This lets the model partly express the rapid changes to air temperature which the direct sunlight can cause.
%\subsection{Parameters identification for photovoltaic model}
%\subsection{Parameters identification for heat pump model}




\begin{itemize}
    \item externalities 
\end{itemize}