\subsection{Evaluation procedure}
\label{sec_validation}
\newcommand{\iterDayCmp}{j}
\newcommand{\iterDayExp}{i}
\newcommand{\iterDay}{i}
The objective of the evaluation procedure is to answer whether the new price- and forecast-aware controller saves money when compared to the existing benchmark controller described in Section \ref{sec_test_house}. The key performance indicator is daily cost given the weather conditions. It is inherently difficult to benchmark and validate the performance of a controller operating in a complex environment with many uncontrollable external factors such as weather and occupant activities. Further, the long time-constants play a significant role by demanding long test periods. Ideally, the benchmark  and \MPC-controller should be run in parallel on exact copies of the same building placed at the same location, with occupants doing the same activities. Although, some buildings support such circumstances, this can obviously not be asked of the occupants. Instead, a benchmark \dataSet\ from the same house is used for the evaluation. The benchmark \dataSet\ is based on data collected from the former heating period (2021-2022) where the original benchmark controller was operating. The data is sorted into full days creating a collection of comparison days $\daySetCmp$, seen in \eqref{eq_days}, from which appropriate subsets can be selected. The daily generated data on set form is:
% \begin{align}
%     \label{eq_days}
%     \daily^i = \{\Eg^\iterDay, \Ta^\iterDay, \priceElecBuySeti{\iterDay}, \Epv^\iterDay \in \realv{\Nday}\}, \hspace{0.1cm} \daily^i \in \mathcal{D}
% \end{align}
\newcommand{\iterDayOne}{n}
\begin{align}
    \label{eq_days}
    &\mathcal{D}_\compare = \left\{day^\iterDayOne = \left(\Eg^\iterDayOne, \Ta^\iterDayOne, \priceElecBuySeti{\iterDayOne},\Epv^\iterDayOne  \right)\right\}\\
    &\iterDayOne = 1,\dots,\Ncmp, \hspace{4mm} \Eg^\iterDayOne, \Ta^\iterDayOne, \priceElecBuySeti{\iterDayOne},\Epv^\iterDayOne \in \realv{\Nday}
    %\daily^i = \{\Eg^\iterDay, \Ta^\iterDay, \priceElecBuySeti{\iterDay}, \Epv^\iterDay \in \realv{\Nday}\}, \hspace{0.1cm} \daily^i \in \mathcal{D}
\end{align}
with $\Eg^\iterDay$ and $\Epv^\iterDay$ being the electricity consumption from grid and production from \pv\ in \si{\kWh} during day $i$, respectively, $\Ta^\iterDay$ the ambient temperature, $\priceElecBuySeti{\iterDay}$ the hourly electricity price for day $i$, and $\Nday = 24$. Note that benchmark days where the system has been manipulated or a significant amount of data is missing are dropped to minimise pollution of the results. A similar data collection, $\daySetExp$, is generated from the experiment period. %To evaluate daily controller performance a subset of days, $\daySetCmp^\iterDayExp \subset \daySetCmp$, with similar average ambient temperature and sun irradiation are drawn from the comparison collection $\daySetCmp$ for each experiment day $\iterDayExp$:
The \MPC-controller is evaluated daily by comparing the operation cost of day $\iterDayExp$ to a subset of benchmark days, $\daySetCmp^\iterDayExp \subset \daySetCmp$, drawn from the full benchmark \dataSet. The subset, $\daySetCmp^\iterDayExp$, is drawn according to the following rule:
% \begin{align}
%     \label{eq_subset_cmp_days}
%     \mathcal{D}_{\compare}^\iterDayExp &= \{day^\iterDayCmp \hspace{1mm}\vert\hspace{1mm}  \nonumber\\ &-\Delta\avgTai{\dn} \leq \avgTaCmp^{\iterDayCmp} - \avgTaExp^\iterDayExp \leq \Delta\avgTai{\up}, \nonumber\\
%     &-\Delta\Epvi{\dn}  \leq \Sigma \EpvCmp^{\iterDayCmp}-\Sigma\EpvExp^\iterDayExp \leq \Delta\Epvi{\up} \nonumber \\
%     &day^\iterDayCmp \in \daySetCmp \}
% \end{align}
\begin{align}
    \label{eq_subset_cmp_days}
    \mathcal{D}_{\compare}^\iterDayExp &= \{day \hspace{1mm}\vert\hspace{1mm}  \nonumber\\ &-\Delta\avgTai{\dn} \leq \avgTaCmp - \avgTaExp^\iterDayExp \leq \Delta\avgTai{\up}, \nonumber\\
    &-\Delta\Epvi{\dn}  \leq \Sigma \EpvCmp-\Sigma\EpvExp^\iterDayExp \leq \Delta\Epvi{\up}, \nonumber \\
    & \avgTaCmp,\Sigma \EpvCmp \in day \in \daySetCmp \}
\end{align}
with $\avgTa$, $\Sigma \Epv$ being average ambient temperature and accumulated electricity production from \pv, respectively. The constants $\Delta\avgTai{\dn}$ and $\Delta\avgTai{\up}$ are the down- and up-search range for ambient temperature, respectively. Similar, $\Delta\Epvi{\dn}$, $\Delta\Epvi{\up}$ makes out the search-range for accumulated electricity produced by the \pv. Here the \pv\ is used as an indicator for sun radiation. This is not a perfect indicator, since the sun altitude and intensity vary with the seasons, thereby creating a bias. However, it is found to be a good indicator for dealing with cloud conditions on-site, since it directly measures the level of shadow on the building. With ambient temperature and sun irradiation accounted for, factors such as occupant behavior and previous day heating patterns are left out. This undeniably causes noise, making the electricity consumption of the \hp\ distribute randomly for any given day. To decrease the influence of the noise, the controller is run over a long period to obtain more consistent results.

We calculate a virtual cost for benchmark day $\iterDayCmp$, with respect to experiment day $\iterDayExp$,
\begin{align}
    \label{eq_cost_comp}
    &\costElecCmp^\iterDayCmp = \sum_{k = 0}^{\Nday} \priceElecBuySeti{\iterDay}(k)\Eg^\iterDayCmp(k)\nonumber \\  &\priceElecBuySeti{\iterDay} \in \dailyExp^\iterDayExp, \hspace{0.4cm} \Eg^\iterDayCmp \in \dailyCmp^\iterDayCmp \in \daySetCmp^\iterDayExp
\end{align}
% \begin{align}
%     \label{eq_cost_comp}
%     \costElecCmp^\iterDayCmp = &\sum_{k = 0}^{\Nday-1} \priceElecBuySeti{\iterDay}(k)\Eg(k) \nonumber \\ &  \priceElecBuySeti{\iterDay} \in \dailyExp^\iterDayExp,\hspace{0.1cm} \Eg \in \dailyCmp \in \daySetCmp^\iterDayExp
% \end{align}
It simply means that electricity consumption from similar benchmark days are imposed onto the price of the experiment day to calculate the virtual cost. This provides a plausible alternate outcome for the case where the benchmark controller had been running instead. This is done since the benchmark controller is price ignorant and thereby acts independently of the price. This manoeuvre would not be possible if the comparison was between two price-aware controllers. In that case price curves would have to be accounted for as well. The cost of the experiment day $i$, $\costElecExp^\iterDayExp$, is of course calculated using the actual electricity consumption for the day.