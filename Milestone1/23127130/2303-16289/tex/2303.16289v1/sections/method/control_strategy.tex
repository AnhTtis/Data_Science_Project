\section{Control}
\label{sec_control_strategy}
The control objective is to provide the required comfort level at the lowest cost feasible. To accomplish this, the controller needs to make two high-level control actions. First, it must choose the heat pump heat flow $\dotQhp(t) \in \mathbb{R}$ and the \fh\ water flow $q(t) \in \mathbb{R}$. Second, it must guide the water to the most suitable rooms. It is not possible to control the heat and water flow directly, but it is possible to influence them indirectly. The heat production can be indirectly controlled using ambient temperature $\Ta$, and valve positions $\val$ affect flow:
\begin{align}
    \label{eq_control_inputs}
    \dotQhp(t) = f(\Ta, \cdot),  \hspace{0.3cm} \flow = g(\val, \cdot), \hspace{0.2cm} \val \in \mathbb{R}^N, \hspace{0.2cm} \Ta \in \mathbb{R}
\end{align}
The $(\cdot,\cdot)$ notation indicates that heat and water flow are not only functions of ambient temperature and valve positions, but other factors too.% Later in the text we will elaborate on these relationships.
\subsection{Control hierarchy}
\begin{figure}[h]
	\centering
	\includegraphics[width=\columnwidth]{figs/control_diagram_4.pdf}
	\caption{Shows the control diagram with signals. Blue indicates computations conducted remotely and green indicates onsite units and yellow the physical components. The grey boxes contain the control layers in the hierarchical control structure}
	\label{fig_control_diagram}
\end{figure}
The control concept comprises three control levels, see \figref{fig_control_diagram}. The upper layer contains the supervisory controller that is aware of energy assets connected to the system as well as important externalities such as weather and electricity prices. It treats the energy assets as objects with properties which can be utilized for optimal control. A key feature of the supervisory controller is that it knows what the energy assets can do, and why they should do it, but not how to make them do it. The middle layer is tasked with tracking the heat reference, delivered by the supervisory controller, and distributing the heat to appropriate rooms. This layer knows how to deliver the demanded energy, but not why it does it. Based on the heat reference and room temperatures, the valve controller selects the valves to be opened in order to provide a flow, which works as an operating point for the heat controller, and to transport the heat to the rooms that need it the most. The heat controller follows the heat reference by providing an artificial ambient temperature to the \hp\ to indirectly control the compressor speed.

The lowest layer handles the interface between the control signal and the actual hardware. The Heat publisher translates the artificial ambient temperature provided by the heat controller to a voltage which emulates the outdoor temperature sensors output at given temperature. The valve publisher translates the valve selection into room temperature references designed to force circuits open or closed.%It knows the hardware and how to replicate the control signal in the physical system.
\subsection{Supervisory controller}
\figref{fig_supervisory_controller} presents the concept for the supervisory controller in the upper layer. 
\begin{figure}[H]
	\centering
	%\includegraphics[width=.5\columnwidth]{figs/dQ_fit.pdf}
	\includegraphics[width=0.65\columnwidth]{figs/control_concept_dotQhp_R.pdf}
	\caption{Overview of the top layer supervisory controller.}
 	\label{fig_supervisory_controller}
\end{figure}
The controller relies on three main components, forecasts, models and measurements. Based on these, the controller computes a heat reference (or ``budget''), $\Qref$, which is dispatched to the lower level controllers. The hierarchical structure makes the supervisory controller more flexible than a monolithic structure, since it can calculate the heat reference without concern for how the heat is delivered---it just needs to know at which efficiency and rate the heat can be delivered. The biggest drawback of using heat for the interface is that it needs to be measured, and adding a heat flow sensor to the hydraulic network is costly.
%\subsection{Heat and distribution controller}
%\subsection{Control publisher}