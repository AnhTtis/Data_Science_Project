\subsection{Heat controller}
The heat controller, placed in the middle layer, is required to deliver heat, \Qhp, according to the heat reference. Further, it suppresses the compressor in periods with no demand and is responsible for timing the \hp\ start. The diagram is seen in \figref{fig_heat_control}.
\begin{figure}[h]
	\centering
	\includegraphics[width=0.7\columnwidth]{figs/heat_controller.pdf}
	\caption{Shows the heat controller with feedback}
	\label{fig_heat_control}
\end{figure}
The signals are: reference vector $\Qref \in \realv{N}$, artificial ambient temperature, \artTa, the voltage representing the said ambient temperature, $\Volta$, the measurement vector, $y_\hp = \begin{bmatrix} \dotQhp & \Php \end{bmatrix}$, and the binary compressor blocking signal, $b_\text{c}$. During the test period a PID-controller and a short horizon \MPC\ were tested. The PID-controller uses the measured heat flow and the reference in regular feedback. The \MPC\ accumulates the delivered heat flow over the hour to match the heat reference given for that hour. Beyond heat control, the two controllers need to handle defrosting periods, \DHW\ production and start delays as mentioned in Section \ref{sec_hp_object}. Defrost periods and \DHW\ production are handled by detecting the event and setting the controller to standby-mode. After releasing the compressor block, it takes about 1.5 hour before the heat pump starts, therefore the reference vector is used to remove the blockage a defined time-span before the actual control takes place. More detail is given in a parallel paper in progress.