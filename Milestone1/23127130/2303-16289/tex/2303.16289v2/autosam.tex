% autosam.tex
% Annotated sample file for the preparation of LaTeX files
% for the final versions of papers submitted to or accepted for 
% publication in AUTOMATICA.

% See also the Information for Authors.

% Make sure that the zip file that you send contains all the 
% files, including the files for the figures and the bib file.

% Output produced with the elsart style file does not imitate the
% AUTOMATICA style. The style file is generic for all Elsevier
% journals and the output is laid out for easy copy editing. The
% final document is produced from the source file in the
% AUTOMATICA style at Elsevier.

% You may use the style file autart.cls to obtain a two-column 
% document (see below) that more or less imitates the printed 
% Automatica style. This may helpful to improve the formatting 
% of the equations, tables and figures, and also serves to check 
% whether the paper satisfies the length requirements.

% Please note: Authors must not create their own macros.

% For further information regarding the preparation of LaTeX files 
% for Elsevier, please refer to the "Full Instructions to Authors" 
% from Elsevier's anonymous ftp server on ftp.elsevier.nl in the
% directory pub/styles, or from the internet (CTAN sites) on
% ftp.shsu.edu, ftp.dante.de and ftp.tex.ac.uk in the directory
% tex-archive/macros/latex/contrib/supported/elsevier.


%\documentclass{elsart}               % The use of LaTeX2e is preferred.

\documentclass[twocolumn]{autart}    % Enable this line and disable the 
                                     % preceding line to obtain a two-column 
                                     % document whose style resembles the
                                     % printed Automatica style.

\usepackage{siunitx}
\usepackage{textcomp}
\usepackage{eurosym}
\usepackage{rotating}
\usepackage{tikz}
\usepackage{graphicx}          % Include this line if your 
                               % document contains figures,
%\usepackage[dvips]{epsfig}    % or this line, depending on which
                               % you prefer.
% \usepackage[latin1]{inputenc}
\usepackage[british]{babel}
\usepackage[all]{xy}
\usepackage{amscd}
\usepackage{amssymb}
\usepackage{amsthm}
\usepackage{enumitem}
\usepackage{mathrsfs,bbm}
\usepackage{xcolor,graphicx}
\usepackage{graphics}
\usepackage{soul}
\usepackage{comment}
\usepackage[all]{xy}
\usepackage{amscd}
\usepackage{amssymb,amsmath,latexsym}
\usepackage{amsthm}
\usepackage{enumitem}
\usepackage{mathrsfs,bbm}
\usepackage{dsfont}
\usepackage{tikz-cd}
\usepackage[T1]{fontenc}
\usepackage[utf8]{inputenc}  
 %
%%%%%%%%%%%%%%%%%%%%%%%%%%%%%%%%%%
%pagestyle
%%%%%%%%%%%%%%%%%%%%%%%%%%%%%%%%%%
%\pagestyle{plain}
\textwidth=430pt
\headsep=.7cm
\evensidemargin=15pt
\oddsidemargin=15pt
\leftmargin=0cm
\rightmargin=0cm
%%
%%%%%%%%%%%%%%%%%%%%%%%
\newcommand*\fixitem {\item[]%
  \refstepcounter{enumi}\hskip-\leftmargin\labelenumi\hskip\labelsep}
\newtheorem*{mainthm}{Main Theorem}
\newtheorem*{mainthm1}{Theorem}
\newtheorem*{maincor}{Corollary}
\usepackage[colorlinks=true]{hyperref}
\DeclareMathOperator{\Forall}{\forall}
\DeclareMathOperator{\Exists}{\exists}
\DeclareMathOperator{\ord}{ord}
\newcommand{\phiD}{\varphi_D}
\newcommand{\phiDI}{\varphi_{\mathbf{D}_I}}
\newcommand{\phiDIj}{\varphi_{\mathbf{D}_I (j)}}
\newcommand{\phiH}{\varphi_H}
\newcommand{\phiTimes}{\phiD \otimes \phiH}
\newcommand{\phiTimesDI}{\varphi_{\mathbf{D}_I} \otimes \phiH}
\newcommand{\R}{\mathscr{A}}
\newcommand{\X}{\mathscr{X}}
\newcommand{\Xf}{\mathscr{X}_{(k_0 ,i)}[r_0]}
\newcommand{\Xfr}{\mathscr{X}_{(k_0,i)}[r]}
\newcommand{\hotimes}{\widehat{\otimes}}
\newcommand{\C}{\mathbb{C}_p}
\newcommand{\V}{\mathscr{V}}
\newcommand{\B}{\mathscr{B}}
\newcommand{\dualD}{\mathfrak{D}}
\newcommand{\Dg}{\mathbf{D}}
\newcommand{\DD}{\mathcal{D}^0}
\newcommand{\DDg}{\mathcal{D}}
\newcommand{\DV}{\mathcal{D}}
\newcommand{\W}{\mathscr{W}_N}
\newcommand{\Ao}{\mathbf{A}^\circ}
\newcommand{\AoK}{\mathbf{A}^\circ_{\K}}
\newcommand{\AK}{\mathbf{A}_{/\K}}
\newcommand{\OOO}{\mathscr{A}^\circ}
\newcommand{\K}{\mathcal{K}} 
\newcommand{\OK}{\mathcal{O}_{\K}}
\newcommand{\varprojlog}[1]{\underleftarrow{\log\!^{#1}}}
\newcommand{\T}{\mathscr{T}}
\newcommand{\TT}{\mathbf{T}}
\newcommand{\VV}{\mathbf{V}}
\newcommand{\HH}{\mathcal{H}}
\newcommand{\hh}{\mathcal{H}^+}
\newcommand{\HG}[2]{\mathcal{H}_{#1}(#2)}
\newcommand{\hhl}{\mathcal{H}^{+,[l]}}
\newcommand{\hhj}{\mathcal{H}^{+,[j]}}
\newcommand{\hhjj}{\mathcal{H}^{+,[l,l']}}
\newcommand{\GS}{G_{\mathbb{Q},S}}
\newcommand{\Rf}{R_{(k_0 ,i)}[r_0]}
\newcommand{\Rfr}{R_{(k_0 ,i)}[r]}
\newcommand{\parT}{\langle T\rangle}
\newcommand{\Zf}{Z_{(k_0 ,i)}[r_0]}
\newcommand{\Zfr}{\mathscr{Z}_{(k_0 ,i)}[r]}
\newcommand{\ZFf}{\mathscr{Z}_{(k_0 ,i)}[r_0]}
\newcommand{\ZFfr}{\mathscr{Z}_{(k_0 ,i)}[r]}
\newcommand{\ZF}{\mathscr{Z}}
%\usepackage[square,numbers]{natbib}
\usepackage[square, numbers]{natbib}
\usepackage{hyperref}
%\usepackage{multicol}
%\usepackage{draftwatermark}
\begin{document}
%\SetWatermarkText{Confidential}
%\SetWatermarkScale{10}
%\SetWatermarkVerCenter{15cm}
%\SetWatermarkHorCenter{0.7cm}
%\SetWatermarkAngle{90}
% Symbol definitions
% Symbol definitions
\newcommand{\perk}[1]{\ensuremath{\{k#1\}}}
\newcommand{\samk}[1]{\ensuremath{[k#1]}}
\newcommand{\timt}[1]{\ensuremath{(t#1)}}

%% Fundamental variables
\newcommand{\T}{\ensuremath{T}}             % Temperature
\newcommand{\dotT}{\ensuremath{\dot{\T}}}   % Derivative of temperature
\newcommand{\Pre}{\ensuremath{p}}           % Presure
\newcommand{\pow}{\ensuremath{P}}           % Power
\newcommand{\energy}{\ensuremath{E}}        % Energy
\newcommand{\perc}{\ensuremath{\alpha}}     % Percentage
\newcommand{\qprop}{\ensuremath{\alpha}}
\newcommand{\nomqprop}{\ensuremath{\overline{\qprop}}}
\newcommand{\U}{\ensuremath{U}}             % 
\newcommand{\C}{\ensuremath{C}}
\newcommand{\Q}{\ensuremath{Q}}
\newcommand{\dotQ}{\dot{\Q}}
\newcommand{\vol}{\ensuremath{V}}
\newcommand{\dotV}{\dot{\V}}
\newcommand{\val}{\ensuremath{v}}
\newcommand{\Area}{\ensuremath{A}}
\newcommand{\mass}{\ensuremath{m}}
%\newcommand{\dotm}{\ensuremath{\dot{\m}}}
\newcommand{\dotm}{\ensuremath{q}}
\newcommand{\nomdotm}{\overline{\dotm}}
\newcommand{\nomflow}{\overline{\flow}}
\newcommand{\res}{\ensuremath{R}}			% Photovoltaic
\newcommand{\air}{\ensuremath{\text{a}}}
\newcommand{\spu}{\ensuremath{u}}			
\newcommand{\price}{\ensuremath{c}}
\newcommand{\dt}{\ensuremath{\Delta t}}
\newcommand{\len}{\ensuremath{\ell}}
\newcommand{\relative}{\ensuremath{\text{rel}}}
\newcommand{\vel}{\ensuremath{v}}           % velocity
\newcommand{\maxi}{\text{max}}
\newcommand{\Irr}{\ensuremath{I}}
\newcommand{\deltahp}{\ensuremath{\delta_{\hp}}}
\newcommand{\deltahpi}[1]{\ensuremath{\delta_{\hp,#1}}}
\newcommand{\costRoom}{\Rm}
\newcommand{\tsample}{\ensuremath{t_s}}

% MPC and state space
\newcommand{\xss}{\ensuremath{x}}			% States
\newcommand{\uss}{\ensuremath{u}}			% Inputs
\newcommand{\fss}{\ensuremath{\tilde{d}}}			% Forecasts
\newcommand{\dss}{\ensuremath{\hat{d}}}

% Base Components
\newcommand{\room}{\ensuremath{\text{r}}} 			% Room
\newcommand{\floor}{\ensuremath{\text{f}}}			% Floor

% Heating system
\newcommand{\brine}{\ensuremath{\text{B}}}			% Brine
\newcommand{\forward}{\ensuremath{\text{F}}}		% Forward
\newcommand{\return}{\ensuremath{\text{R}}}			% Return
\newcommand{\fh}{\ensuremath{\text{FH}}}			% Floor heating
\newcommand{\hp}{\ensuremath{\text{HP}}}			% Heat Pump
\newcommand{\area}{\ensuremath{\text{A}}}			% Area
\newcommand{\water}{\ensuremath{\text{w}}}			% Water
\newcommand{\hot}{\ensuremath{\text{H}}}			% Hot
\newcommand{\cold}{\ensuremath{\text{C}}}			% Cold
\newcommand{\tank}{\ensuremath{\text{ta}}}			% Tank
\newcommand{\pv}{\ensuremath{\text{PV}}}			%
\newcommand{\sensor}{\ensuremath{\text{s}}}			% Photovoltaic
\newcommand{\envel}{\ensuremath{\text{e}}}			% Photovoltaic
\newcommand{\cop}{\ensuremath{\text{COP}}}			% Photovoltaic
\newcommand{\comp}{\ensuremath{\text{com}}}			% Photovoltaic
\newcommand{\evap}{\ensuremath{\text{eva}}}			% Photovoltaic
\newcommand{\source}{\ensuremath{\text{src}}}		% Photovoltaic
\newcommand{\coTwo}{\ensuremath{\text{CO}_2}}
\newcommand{\coTwoTax}{\coTwo-tax}
\newcommand{\pipe}{\ensuremath{\text{p}}}
\newcommand{\reference}{\ensuremath{\text{ref}}}
\newcommand{\grid}{\ensuremath{\text{G}}}
\newcommand{\diag}{\ensuremath{\text{diag}}}
\newcommand{\bess}{\ensuremath{\text{b}}}
\newcommand{\comf}{\ensuremath{\text{cmf}}}
\newcommand{\expo}{\ensuremath{\text{EX}}}
\newcommand{\impo}{\ensuremath{\text{IM}}}
\newcommand{\up}{\ensuremath{\text{up}}}
\newcommand{\dn}{\ensuremath{\text{dn}}}
\newcommand{\app}{\ensuremath{\text{APP}}}
\newcommand{\charge}{\ensuremath{\text{c}}}
\newcommand{\transport}{\ensuremath{\text{tr}}}
\newcommand{\partition}{\ensuremath{\text{part}}}
\newcommand{\consumption}{\ensuremath{\text{con}}}
%\newcommand{\export}{\ensuremath{-}
%\newcommand{\import}{\ensuremath{+}
\newcommand{\selfCons}{\ensuremath{\text{self}}}
\newcommand{\spot}{\ensuremath{\text{spot}}}
\newcommand{\tariff}{\ensuremath{\text{tariff}}}
\newcommand{\daily}{\ensuremath{\text{day}}}
\newcommand{\elec}{\ensuremath{\text{E}}}
\newcommand{\effic}{\ensuremath{\eta}}
\newcommand{\slack}{\ensuremath{\text{s}}}
\newcommand{\slackVar}{\ensuremath{S}}
\newcommand{\direct}{\ensuremath{\text{dir}}}
\newcommand{\Nslack}{\ensuremath{N_\slack}}
\newcommand{\Ncomf}{\ensuremath{N_\comf}}
\newcommand{\Nroom}{\ensuremath{N_\room}}
\newcommand{\TSO}{\ensuremath{\text{tso}}}
\newcommand{\VAT}{\ensuremath{\text{vat}}}
\newcommand{\experiment}{\ensuremath{\text{exp}}}
\newcommand{\compare}{\ensuremath{\text{cmp}}}
\newcommand{\cost}{\ensuremath{\text{cost}}}
\newcommand{\Ehp}{\ensuremath{\energy_{\hp}}}
\newcommand{\Ehpexp}{\ensuremath{\energy_{\hp, \experiment}}}
\newcommand{\Ehpcmp}{\ensuremath{\energy_{\hp, \compare}}}
\newcommand{\Epv}{\ensuremath{\energy_{\pv}}}
\newcommand{\Epvi}[1]{\ensuremath{\energy_{\pv,#1}}}
\newcommand{\Eg}{\ensuremath{\energy_{\grid}}}
\newcommand{\Econ}{\ensuremath{\energy_{\consumption}}}
\newcommand{\Eapp}{\ensuremath{\energy_{\app}}}
\newcommand{\Eself}{\ensuremath{\energy_{\selfCons}}}
\newcommand{\Eexp}{\ensuremath{\energy_{\expo}}}
\newcommand{\Eimp}{\ensuremath{\energy_{\impo}}}
\newcommand{\dailyExp}{\ensuremath{\text{day}_{\experiment}}}
\newcommand{\dailyCmp}{\ensuremath{\text{day}_{\compare}}}
\newcommand{\EpvCmp}{\Epvi{\compare}}
\newcommand{\EpvExp}{\Epvi{\experiment}}
\newcommand{\kf}{\ensuremath{\text{Kalman Filter}}}
\newcommand{\ukf}{\ensuremath{\text{Unscented }\kf}}
\newcommand{\Qref}{\ensuremath{\Q_{\reference}}}
% Sets
% Set of open valves
\newcommand{\setVal}{\ensuremath{\mathcal{V}}}

% matrices and vectors
\newcommand{\vval}{\ensuremath{{v}}}			% Photovoltaic

\newcommand{\pperA}{\ensuremath{g_\pipe}}
\newcommand{\dtr}{\ensuremath{\dt_\room}}
\newcommand{\dtri}[1]{\ensuremath{\dt_{\room,#1}}}
\newcommand{\dtest}{\ensuremath{\hat{\dt}}}
\newcommand{\dtrest}{\ensuremath{\dtest_\room}}
\newcommand{\dtrestj}[1]{\ensuremath{\dtest_{\room,#1}}}
\newcommand{\dttr}{\ensuremath{\dt_\transport}}
\newcommand{\dttrest}{\ensuremath{\dtest_\transport}} 
\newcommand{\dttri}[1]{\ensuremath{\dt_{\transport,#1}}}
\newcommand{\pperAavg}{\bar{\pperA}}

\newcommand{\lentr}{\len_{\transport}}
\newcommand{\lenR}{\len_{\return}}
\newcommand{\lentrest}{\hat{\len}_{\transport}}
\newcommand{\lentrestj}[1]{\hat{\len}_{\transport,#1}}
\newcommand{\lentrj}[1]{\len_{\transport,#1}}
\newcommand{\lenpart}{\len_\text{part}}
% Disturbances
\newcommand{\amb}{\ensuremath{\text{a}}}			% Ambient
\newcommand{\inter}{\ensuremath{\text{int}}}			% Input
\newcommand{\occ}{\ensuremath{\text{occ}}}			% Input
\newcommand{\loss}{\ensuremath{\text{loss}}}		% Input
\newcommand{\SUN}{\ensuremath{\text{s}}}			% Input
\newcommand{\sun}{\ensuremath{\text{\SUN}}}
\newcommand{\cloud}{\ensuremath{\text{cloud}}}

\newcommand{\IN}{\ensuremath{\text{in}}}			% Input
\newcommand{\OUT}{\ensuremath{\text{out}}}			% Input

\newcommand{\subjectTo}{\ensuremath{\text{s.t.}}}   % Input
\newcommand{\priceElec}{\ensuremath{\price_\elec}}
\newcommand{\priceElecBuy}{\ensuremath{\price_\elec^{+}}}
\newcommand{\priceElecBuySeti}[1]{\ensuremath{\price_{\elec}^{+,#1}}}
\newcommand{\priceElecSell}{\ensuremath{\price_\elec^{-}}}
\newcommand{\priceComf}{\ensuremath{\price_\comf}}
\newcommand{\priceComfi}[1]{\ensuremath{\price_{\comf,#1}}}
\newcommand{\priceSlack}{\ensuremath{\price_\slack}}
\newcommand{\priceCoTwo}{\ensuremath{\price_{\coTwo}}}
\newcommand{\priceSpot}{\ensuremath{\price_{\spot}}}
\newcommand{\priceTariff}{\ensuremath{\price_{\tariff}}}
\newcommand{\priceTransport}{\ensuremath{\price_{\TSO}}}
\newcommand{\priceVat}{\ensuremath{\price_{\VAT}}}
\newcommand{\priceElecBuyi}[1]{\ensuremath{\price_{\elec,#1}^{+}}}
\newcommand{\priceElecBuyExp}{\priceElecBuyi{\experiment}}
\newcommand{\priceElecBuyCmp}{\priceElecBuyi{\compare}}
\newcommand{\costElecCmp}{\ensuremath{\cost_{\compare}}}
\newcommand{\costElecExp}{\ensuremath{\cost_{\experiment}}}
\newcommand{\deltaOp}{\ensuremath{\delta_{\op}}}

\newcommand{\priceEleci}[1]{\ensuremath{\price_{\elec,#1}}}


% Derived variables
\newcommand{\Ar}{\A_\room}
\newcommand{\Vr}{\V_\room}
\newcommand{\Tr}{\ensuremath{\T_{\room}}}
\newcommand{\Tcold}{\ensuremath{\T_{\cold}}}
\newcommand{\Thot}{\ensuremath{\T_{\hot}}}
\newcommand{\flow}{\ensuremath{q}}
\newcommand{\Tref}{\ensuremath{\T_{\room,\reference}}}
\newcommand{\TFref}{\ensuremath{\T_{\forward,\reference}}}
\newcommand{\dotTr}{\ensuremath{\dot{\T}_{\room}}}
\newcommand{\Tf}{\ensuremath{\T_{\floor}}}
\newcommand{\dotTf}{\ensuremath{\dot{\T}_{\floor}}}
\newcommand{\Tw}{\ensuremath{\T_{\water}}}
\newcommand{\dotTw}{\ensuremath{\dot{\T}_{\water}}}
\newcommand{\Te}{\ensuremath{\T_{\envel}}}
\newcommand{\dotTe}{\ensuremath{\dot{\T}_{\envel}}}
\newcommand{\dotTR}{\ensuremath{\dot{\T}_{\return}}}
\newcommand{\nomPerc}{\rpipe}
\newcommand{\Ta}{\ensuremath{\T_{\amb}}}
\newcommand{\artTa}{\ensuremath{\hat{\T_{\amb}}}}
\newcommand{\avgTa}{\ensuremath{\overline{\T}_{\amb}}}
\newcommand{\avgTai}[1]{\ensuremath{\overline{\T}_{\amb,#1}}}
\newcommand{\avgTaCmp}{\avgTai{\compare}}
\newcommand{\avgTaExp}{\avgTai{\experiment}}
\newcommand{\Tsr}{\ensuremath{\T_{\sensor,\room}}}
\newcommand{\Ts}{\ensuremath{\T_{\sensor}}}
\newcommand{\dotTs}{\ensuremath{\dot{\T}_{\sensor}}}
\newcommand{\TRr}{\ensuremath{\T_{\return,\room}}}
\newcommand{\radius}{\ensuremath{r}}
\newcommand{\Isun}{\ensuremath{\Irr_{\SUN}}}
\newcommand{\Isundir}{\ensuremath{\Irr_{\SUN,\direct}}}
\newcommand{\forIsun}{\ensuremath{\tilde{\Irr}_{\SUN}}}
\newcommand{\forIsundir}{\ensuremath{\tilde{\Irr}_{\SUN,\direct}}}
\newcommand{\Qday}{\ensuremath{\Q_{\daily}}}
\newcommand{\efficHP}{\ensuremath{\eta_{\hp}}}
\newcommand{\copcarnot}{\ensuremath{\cop_\text{CARNOT}}}
\newcommand{\cophp}{\ensuremath{\cop_\hp}}
\newcommand{\fhp}{\ensuremath{f_\hp}}
\newcommand{\fgrid}{\ensuremath{f_\grid}}
\newcommand{\fcomf}{\ensuremath{f_\comf}}
\newcommand{\fcomfi}[1]{\ensuremath{f_{\comf,#1}}}
\newcommand{\zcomf}{\ensuremath{z_\comf}}
\newcommand{\fslack}{\ensuremath{f_\slack}}
\newcommand{\fcoTwo}{\ensuremath{f_{\coTwo}}}
\newcommand{\Pgplus}{\ensuremath{\Pg^+}}
\newcommand{\zhp}{\ensuremath{z_\hp}}
\newcommand{\zhpi}[1]{\ensuremath{z_{\hp,#1}}}
\newcommand{\Volt}{\mathcal{U}}
\newcommand{\Volta}{\Volt_\amb}
\newcommand{\forCloudCover}{\tilde{\perc}_\cloud}


\newcommand{\etahp}{\ensuremath{\eta_{\hp}}}
\newcommand{\Nwin}{\ensuremath{N}}
%\newcommand{\dotTR}{\ensuremath{\dot{\T}_{\return}}}

\newcommand{\TB}{\ensuremath{\T_{\brine}}}
\newcommand{\TF}{\ensuremath{\T_{\forward}}}
\newcommand{\nomTF}{\ensuremath{\hat{\T}_{\forward}}}
\newcommand{\TR}{\ensuremath{\T_{\return}}}
\newcommand{\Php}{\ensuremath{\pow_{\hp}}}
\newcommand{\Pg}{\ensuremath{\pow_{\grid}}}
\newcommand{\Ppv}{\ensuremath{\pow_{\pv}}}
\newcommand{\forPpv}{\ensuremath{\tilde{\pow}_{\pv}}}
\newcommand{\Pb}{\ensuremath{\pow_{\bess}}}
\newcommand{\Eb}{\ensuremath{\energy_{\bess}}}
\newcommand{\Pbc}{\ensuremath{\pow_{\bess, c}}}
\newcommand{\Pbd}{\ensuremath{\pow_{\bess, d}}}
\newcommand{\Papp}{\ensuremath{\pow_{\app}}}
\newcommand{\cg}{\price_{\grid}}
\newcommand{\dotQhp}{\ensuremath{\dotQ_{\hp}}}
\newcommand{\dotQsun}{\ensuremath{\dotQ_{\sun}}}
\newcommand{\Qhp}{\ensuremath{\Q_{\hp}}}
\newcommand{\dotQfh}{\ensuremath{\dotQ_{\fh}}}
\newcommand{\dotQsc}{\ensuremath{\hat{\dotQ}}}
\newcommand{\flowsc}{\ensuremath{\hat{\flow}}}
\newcommand{\dotQfhsc}{\ensuremath{\hat{\dotQ}_{\fh}}}
\newcommand{\Qfh}{\ensuremath{\Q_{\fh}}}
\newcommand{\heatFHRel}{\ensuremath{\dotQ_{\fh,\relative}}}
\newcommand{\QFHRel}{\ensuremath{\Q_{\fh,\relative}}}
\newcommand{\QFHRelj}[1]{\ensuremath{\Q_{\fh,\relative,#1}}}
\newcommand{\percj}[1]{\ensuremath{\perc_{#1}}}
\newcommand{\percFHRelj}[1]{\ensuremath{\perc_{\fh,\relative,#1}}}



\newcommand{\Ua}{\ensuremath{\U_{\area}}}
\newcommand{\Uf}{\ensuremath{\U_{\floor}}}
\newcommand{\Ur}{\ensuremath{\U_{\room}}}
\newcommand{\UR}{\ensuremath{\U_{\return}}}
\newcommand{\Uamb}{\ensuremath{\U_{\amb}}}
\newcommand{\Uw}{\ensuremath{\U_{\water}}}
\newcommand{\Ue}{\ensuremath{\U_{\envel}}}
\newcommand{\propU}{\hat{\U}}
\newcommand{\propUf}{\ensuremath{\propU_{\floor}}}
\newcommand{\propUr}{\ensuremath{\propU_{\room}}}
\newcommand{\propUR}{\ensuremath{\propU_{\return}}}
\newcommand{\propUa}{\ensuremath{\propU_{\amb}}}
\newcommand{\propUw}{\ensuremath{\propU_{\water}}}
\newcommand{\propUe}{\ensuremath{\propU_{\envel}}}
\newcommand{\URr}{\ensuremath{\U_{\return,\room}}}
\newcommand{\Usr}{\ensuremath{\U_{\sensor,\room}}}


\newcommand{\RA}{\ensuremath{\res_{\area}}}
\newcommand{\Rf}{\ensuremath{\res_{\floor}}}
\newcommand{\Ra}{\ensuremath{\res_{\amb}}}
\newcommand{\Rw}{\ensuremath{\res_{\water}}}
\newcommand{\Rsr}{\ensuremath{\res_{\sensor,\room}}}
\newcommand{\RR}{\ensuremath{\res_{\return}}}
\newcommand{\RRr}{\ensuremath{\res_{\return,\room}}}

\newcommand{\uarea}{\ensuremath{\su_{\area}}}
\newcommand{\uf}{\ensuremath{\spu_{\floor}}}
\newcommand{\ur}{\ensuremath{\spu_{\room}}}
\newcommand{\ua}{\ensuremath{\spu_{\amb}}}
\newcommand{\uw}{\ensuremath{\spu_{\water}}}
\newcommand{\ue}{\ensuremath{\spu_{\envel}}}
\newcommand{\utr}{\ensuremath{\spu_{\transport}}}
\newcommand{\uR}{\ensuremath{\spu_{\return}}}
\newcommand{\uRr}{\ensuremath{\spu_{\return,\room}}}

\newcommand{\Cr}{\ensuremath{\C_{\room}}}
\newcommand{\propCr}{\ensuremath{\hat{\C}_{\room}}}
\newcommand{\CR}{\ensuremath{\C_{\return}}}
\newcommand{\Cf}{\ensuremath{\C_{\floor}}}
\newcommand{\propCf}{\ensuremath{\hat{\C}_{\floor}}}
\newcommand{\Cw}{\ensuremath{\C_{\water}}}
\newcommand{\Ctr}{\ensuremath{\C_{\transport}}}
\newcommand{\Cs}{\ensuremath{\C_{\sensor}}}
\newcommand{\Ce}{\ensuremath{\C_{\envel}}}

\newcommand{\cw}{\ensuremath{c_{\water}}}
\newcommand{\cair}{\ensuremath{c_{\air}}}

\newcommand{\rhow}{\ensuremath{\rho_{\water}}}
\newcommand{\rpipe}{\ensuremath{\radius_{\pipe}}}

\newcommand{\dotQsrc}{\ensuremath{\dot{\Q}_{\source}}}
\newcommand{\Psrc}{\ensuremath{\pow_{\source}}}

\newcommand{\lenpipe}{\ensuremath{\len_{\pipe}}}
\newcommand{\lenroom}{\ensuremath{\len_{\room}}}
\newcommand{\Apipe}{\ensuremath{\A_{\pipe}}}


\newcommand{\Phpi}[1]{\ensuremath{\pow_{\hp,#1}}}

% Nominal flow distribution

% Derived variables 2
\newcommand{\Tri}[1]{\ensuremath{\T_{\room,#1}}}
\newcommand{\dotTri}[1]{\ensuremath{\dot{\T}_{\room,#1}}}


\newcommand{\Tfi}[1]{\ensuremath{\T_{\floor,#1}}}
\newcommand{\dotTfi}[1]{\ensuremath{\dot{\T}_{\floor,#1}}}

\newcommand{\Twi}[1]{\ensuremath{\T_{\water,#1}}}
\newcommand{\dotTwi}[1]{\ensuremath{\dot{\T}_{\water,#1}}}

\newcommand{\Tei}[1]{\ensuremath{\T_{\envel,#1}}}
\newcommand{\dotTei}[1]{\ensuremath{\dot{\T}_{\envel,#1}}}

\newcommand{\Ttri}[1]{\ensuremath{\T_{\transport,#1}}}
\newcommand{\dotTtri}[1]{\ensuremath{\dot{\T}_{\transport,#1}}}

\newcommand{\TRi}[1]{\ensuremath{\T_{\return,#1}}}
\newcommand{\dotTtRi}[1]{\ensuremath{\dot{\T}_{\return,#1}}}

\newcommand{\TrRefi}[1]{\ensuremath{\T_{\room,\reference,#1}}}


\newcommand{\Tai}[1]{\ensuremath{\T_{\amb,#1}}}
\newcommand{\TRj}[1]{\ensuremath{\T_{\return,#1}}}
\newcommand{\UAi}[1]{\ensuremath{\U_{\area,#1}}}
\newcommand{\Cri}[1]{\ensuremath{\C_{\room,#1}}}
\newcommand{\Cfi}[1]{\ensuremath{\C_{\floor,#1}}}
\newcommand{\Cwi}[1]{\ensuremath{\C_{\water,#1}}}
\newcommand{\Ctri}[1]{\ensuremath{\C_{\transport,#1}}}
\newcommand{\Ti}[2]{\ensuremath{\T_{#1,#2}}}
\newcommand{\Ci}[2]{\ensuremath{\C_{#1,#2}}}
\newcommand{\Ui}[2]{\ensuremath{\U_{#1,#2}}}
\newcommand{\dotQij}[2]{\ensuremath{\dotQ_{#1,#2}}}
\newcommand{\dotUi}[2]{\ensuremath{\dotU_{#1,#2}}}
\newcommand{\dotTi}[2]{\ensuremath{\dotT_{#1,#2}}}
\newcommand{\Rwi}[1]{\ensuremath{\res_{\water,#1}}}
\newcommand{\cgi}[1]{\ensuremath{\price_{\grid,#1}}}
\newcommand{\ci}[1]{\ensuremath{c_{#1}}}
\newcommand{\lenpj}[1]{\ensuremath{\len_{\room,#1}}}
\newcommand{\Apj}[1]{\ensuremath{\Area_{\pipe,#1}}}
\newcommand{\Arj}[1]{\ensuremath{\Area_{\room,#1}}}
\newcommand{\dtrj}[1]{\ensuremath{\dt_{\room,#1}}}
\newcommand{\dttrj}[1]{\ensuremath{\dt_{\transport,#1}}}
\newcommand{\dotQfhj}[1]{\ensuremath{\dotQ_{\fh,#1}}}
\newcommand{\nomvali}[1]{\ensuremath{\nomqprop_{#1}\val_{#1}}}
\newcommand{\dotQhpi}[1]{\ensuremath{\dotQ_{\hp,#1}}}

\newcommand{\Ufi}[1]{\ensuremath{\U_{\floor,#1}}}
\newcommand{\Uai}[1]{\ensuremath{\U_{\amb,#1}}}
\newcommand{\Uri}[1]{\ensuremath{\U_{\room,#1}}}
\newcommand{\Uwi}[1]{\ensuremath{\U_{\water,#1}}}
\newcommand{\Uei}[1]{\ensuremath{\U_{\envel,#1}}}
\newcommand{\RAi}[1]{\ensuremath{\res_{\area,#1}}}
\newcommand{\Rfi}[1]{\ensuremath{\res_{\floor,#1}}}
\newcommand{\Rai}[1]{\ensuremath{\res_{\amb,#1}}}
\newcommand{\Rri}[1]{\ensuremath{\res_{\room,#1}}}
\newcommand{\Rtri}[1]{\ensuremath{\res_{\transport,#1}}}
\newcommand{\flowi}[1]{\ensuremath{q_{#1}}}
\newcommand{\Trefi}[1]{\ensuremath{\T_{\room,\reference,#1}}}
\newcommand{\QFHReli}[1]{\ensuremath{\Q_{\fh,\relative,#1}}}
\newcommand{\dotQFHReli}[1]{\ensuremath{\dotQ_{\fh,\relative,#1}}}

\newcommand{\trp}{\ensuremath{\text{T}}}


\newcommand{\cgex}{\cgi{\expo}}
\newcommand{\cgim}{\cgi{\impo}}

\newcommand{\Hp}{\ensuremath{h_p}}

% Optimization problem
\newcommand{\fhpmin}{\ensuremath{\underline{f_{\hp}}}}
\newcommand{\fhpmax}{\ensuremath{\overline{f_{\hp}}}}
\newcommand{\Phpmin}{\ensuremath{\underline{\Php}}}
\newcommand{\Phpmax}{\ensuremath{\overline{\Php}}}
\newcommand{\fhpi}[1]{\ensuremath{f_{\hp,#1}}}

% Matrices
\newcommand{\Am}{\ensuremath{\mathbf{A}}}
\newcommand{\Bm}{\ensuremath{\mathbf{B}}}
\newcommand{\Cm}{\ensuremath{\mathbf{C}}}
\newcommand{\Dm}{\ensuremath{\mathbf{D}}}
\newcommand{\Em}{\ensuremath{\mathbf{E}}}
\newcommand{\Nm}{\ensuremath{\mathbf{N}}}
\newcommand{\Um}{\ensuremath{\mathbf{U}}}
\newcommand{\Vm}{\ensuremath{\mathbf{V}}}
\newcommand{\ident}{\ensuremath{\mathbf{I}}}
\newcommand{\Wm}{\ensuremath{\mathbf{W}}}
\newcommand{\Zm}{\ensuremath{\mathbf{Z}}}
\newcommand{\Pm}{\ensuremath{\mathbf{P}}}
\newcommand{\Tm}{\ensuremath{\mathbf{T}}}
\newcommand{\Qm}{\ensuremath{\mathbf{Q}}}
\newcommand{\Rm}{\ensuremath{\mathbf{R}}}

\newcommand{\Acal}{\ensuremath{\mathcal{A}}}
\newcommand{\Bcal}{\ensuremath{\mathcal{B}}}
\newcommand{\Ccal}{\ensuremath{\mathcal{C}}}
\newcommand{\Dcal}{\ensuremath{\mathcal{D}}}
\newcommand{\Ecal}{\ensuremath{\mathcal{E}}}
\newcommand{\Ncal}{\ensuremath{\mathcal{N}}}
\newcommand{\Ucal}{\ensuremath{\mathcal{U}}}
\newcommand{\Vcal}{\ensuremath{\mathcal{V}}}
\newcommand{\Wcal}{\ensuremath{\mathcal{W}}}
\newcommand{\Zcal}{\ensuremath{\mathcal{Z}}}
\newcommand{\Pcal}{\ensuremath{\mathcal{P}}}
\newcommand{\Tcal}{\ensuremath{\mathcal{T}}}


\newcommand{\avm}{\ensuremath{\mathbf{a}}}
\newcommand{\Sigm}{\ensuremath{\mathbf{\Sigma}}}
\newcommand{\Aouter}{\ensuremath{\A\A^*}}
\newcommand{\Ainner}{\ensuremath{\A^*\A}}
\newcommand{\ARinner}{\ensuremath{\A^\trp\A}}
\newcommand{\ZRinner}{\ensuremath{\Z^\trp\Z}}
% Vectors
\newcommand{\uv}{\ensuremath{\mathbf{u}}}
\newcommand{\vv}{\ensuremath{\mathbf{v}}}
\newcommand{\qv}{\ensuremath{\mathbf{q}}}
\newcommand{\xv}{\ensuremath{\mathbf{x}}}
\newcommand{\yv}{\ensuremath{\mathbf{y}}}
\newcommand{\zv}{\ensuremath{\mathbf{z}}}
\newcommand{\nv}{\ensuremath{\mathbf{n}}}
\newcommand{\bv}{\ensuremath{\mathbf{b}}}
\newcommand{\cvec}{\ensuremath{\mathbf{c}}}
\newcommand{\dv}{\ensuremath{\mathbf{d}}}

\newcommand{\rv}{\ensuremath{\mathbf{r}}}
\newcommand{\alphav}{\ensuremath{\mathbf{\alpha}}}

\newcommand{\realv}[1]{\ensuremath{\mathbb{R}^{#1}}}
\newcommand{\binv}[1]{\ensuremath{\mathbb{B}^{#1}}}
\newcommand{\iter}{\ensuremath{j}}

\newcommand{\placehold}{\ensuremath{\{\cdot\}}}

\newcommand{\rhc}{relative heating coefficient}
\newcommand{\RHC}{RHC}
\newcommand{\rmse}{root mean square error}
\newcommand{\RMSE}{RMSE}
\newcommand{\rhcs}{\beta}
\newcommand{\rhcsj}[1]{\ensuremath{\beta_{#1}}}
\newcommand{\rhcsq}{\rhcs_{\dotm}}
\newcommand{\rhcsqj}[1]{\rhcs_{\dotm,#1}}
\newcommand{\rhcsA}{\rhcs_{\Ar}}
\newcommand{\rhcsAj}[1]{\rhcs_{\Ar,#1}}
\newcommand{\rhcsAqj}[1]{\rhcs_{\Ar\dotm,#1}}

%\newcommand{\Ardt}[]{Ar,j∆tr,j}

\newcommand{\dtsim}{\ensuremath{dt}}
%\newcommand{\sensor}{\ensuremath{\text{s}}}
\newcommand{\percPpart}{\ensuremath{\alpha_\pipe}}

\newcommand{\gParam}{\ensuremath{g}}
\newcommand{\fParam}{\ensuremath{\gParam_\floor}}
\newcommand{\eParam}{\ensuremath{\gParam_\envel}}
\newcommand{\rParam}{\ensuremath{\gParam_\room}}
\newcommand{\pParam}{\ensuremath{\gParam_\pipe}}
\newcommand{\qParam}{\ensuremath{\gParam_\dotm}}
\newcommand{\sunParam}[1]{\ensuremath{\gParam_{\SUN,#1}}}
\newcommand{\propsunParam}[1]{\ensuremath{\hat{\gParam}_{\SUN,#1}}}

\newcommand{\eqDist}{0.5cm}
\newcommand{\numPart}{\ensuremath{N_\partition}}
\newcommand{\numTransportPart}{\ensuremath{M_\partition}}
\newcommand{\Ncirc}{N}
\newcommand{\numRooms}{\ensuremath{\Ncirc_r}}
\newcommand{\Nhor}{\ensuremath{N}}
\newcommand{\Nday}{\ensuremath{N_{\daily}}}
\newcommand{\Ncmp}{N_{\compare}}
%%%%%%%%%%%%%%%%%%%%%%%%%%%%%%%%%%%%%%%%%%%%%%%%%%%%%%%%%%%%%%%%%%%%%%%%%%%%%
% METHOD DEFS
%%%%%%%%%%%%%%%%%%%%%%%%%%%%%%%%%%%%%%%%%%%%%%%%%%%%%%%%%%%%%%%%%%%%%%%%%%%%%
\newcommand{\share}{\mathcal{P}}
\newcommand{\shareFHRelj}[1]{\ensuremath{\share_{\fh,\relative,#1}}}
\newcommand{\maxLike}{Maximum Likelihood}
\newcommand{\ML}{ML}
\newcommand{\leastSq}{Least Squares}
\newcommand{\LS}{LS}
\newcommand{\UKF}{UKF}

% MPC
\newcommand{\mpc}{Model Predictive Control}
\newcommand{\MPC}{MPC}

% Mixed integer optimal control
\newcommand{\mimpc}{Mixed Integer Model Predictive Control}
\newcommand{\MIMPC}{MIMPC}

\newcommand{\milp}{Mixed Integer Linear Programming}
\newcommand{\MILP}{MILP}

% Mixed integer optimal control
\newcommand{\miocp}{Mixed Integer Optimal Control Problem}
\newcommand{\MIOCP}{MIOCP}

% Mixed integer optimal control
\newcommand{\ocp}{optimal control problem}
\newcommand{\Ocp}{Optimal control problem}
\newcommand{\OCP}{OCP}

\newcommand{\efficiencyModel}{efficiency model}

% Zero order hold
\newcommand{\zoh}{Zero Order Hold}
\newcommand{\ZOH}{ZOH}

\newcommand{\tes}{thermal energy storage}
\newcommand{\TES}{TES}

\newcommand{\ebat}{electric battery}
\newcommand{\BESS}{BAT}

\newcommand{\tra}{\ensuremath{\text{T}}}

\newcommand{\nzeb}{near zero-emission building}
\newcommand{\NZEB}{NZEB}
\newcommand{\backend}{backend}

\newcommand{\atow}{air-to-water}

\newcommand{\lkf}{Linear Kalman Filter}
\newcommand{\LKF}{LKF}

\newcommand{\singleFam}{single-family}
\newcommand{\SingleFam}{Single-family}
\newcommand{\dataSet}{data-set}
\newcommand{\DataSet}{Data-set}
\newcommand{\spaceHeating}{space-heating}
\newcommand{\SpaceHeating}{Space-heating}
\newcommand{\edmdc}{Extended Dynamic Mode Decomposition with Control}
\newcommand{\EDMDC}{EDMDc}


\newcommand{\DHW}{DHW}
\newcommand{\dhw}{domestic hot water}

\newcommand{\eu}{European Union}
\newcommand{\EU}{EU}

\newcommand{\mld}{Mixed Logic Dynamics}
\newcommand{\MLD}{MLD}

\newcommand{\OP}{optimization problem}
\newcommand{\op}{\ensuremath{\text{OP}}}

\newcommand{\nordpool}{Nord Pool}

%\newcommand{\currency}{DKK}
\newcommand{\currency}{\euro}
\newcommand{\cookingPeak}{evening peak}
\newcommand{\CookingPeak}{Evening peak}


\newcommand{\loadShift}{load shift}
\newcommand{\LoadShift}{Load shift}

\newcommand{\loadshifting}{\loadShift ing}
\newcommand{\Loadshifting}{\LoadShift ing}

\newcommand{\dsm}{demand side management}
\newcommand{\Dsm}{Demand side management}
\newcommand{\DSM}{DSM}

\newcommand{\smartGrid}{smart grid}
\newcommand{\SmartGrid}{Smart grid}

\newcommand{\savingRateText}{saving rate}
\newcommand{\SavingRateText}{Saving rate}

\newcommand{\bilinear}{bilinear}

\newcommand{\ipcc}{The Intergovernmental Panel on Climate Change}
\newcommand{\IPCC}{IPCC}

\newcommand{\tsos}{Transport Service Operator}
\newcommand{\TSOs}{TSO}
\newcommand{\dsos}{Distribution Service Operator}
\newcommand{\DSOs}{DSO}

\newcommand{\daySetCmp}{\mathcal{D}_\compare}
\newcommand{\daySetExp}{\mathcal{D}_\experiment}
%%%%%%%%%%%%%%%%%%%%%%%%%%%%%%%%%%%%%%%%%%%%%%%%%%%%%%%%%%%%%%%%%%%%%%%%%%%%%%%%
% GENERAL EQUATIONS
%%%%%%%%%%%%%%%%%%%%%%%%%%%%%%%%%%%%%%%%%%%%%%%%%%%%%%%%%%%%%%%%%%%%%%%%%%%%%%%%
\newcommand{\SSdx}[1]{\xv_{#1+1} = \Am \xv_{#1} + \Bm \uv_{#1} + \Em \dv_{#1}}
\newcommand{\SScx}[1]{\dot{\xv}_{#1} = \Am \xv_{#1} + \Bm \uv_{#1} + \Em \dv_{#1}}

\newcommand{\iterOp}{i}

\newcommand{\figSpace}{-7mm}
\newcommand{\SecSpace}{-3mm}
\newcommand{\tableSpace}{-6mm}

%\newcommand{\todo}{\textbf{todo}}

\newcommand{\degC}{\si{\celsius}}
\newcommand{\kwh}{\si{\kilo\watt\hour}}
%%%%%%%%%%%%%%%%%%%%%%%%%%%%%%%%%%%%%%%%%%%%%%%%%%%%%%%%%%%%%%%%%%%%%%%%%%%%%%%%
% Figures
%%%%%%%%%%%%%%%%%%%%%%%%%%%%%%%%%%%%%%%%%%%%%%%%%%%%%%%%%%%%%%%%%%%%%%%%%%%%%%%%

\newcommand{\figUpper}{\text{Upper: }}
\newcommand{\figLower}{\text{Lower: }}
\newcommand{\figLeft}{\text{Left: }}
\newcommand{\figRight}{\text{Right: }}
\newcommand{\figLowerLeft}{\text{Lower left: }}
\newcommand{\figLowerRight}{\text{Lower right: }}
\newcommand{\figUpperLeft}{\text{Upper left: }}
\newcommand{\figUpperRight}{\text{Upper right: }}

\newcommand{\questionTwo}{\textit{What impact does the controller have on indoor climate?}}
\newcommand{\questionThree}{\textit{Is it possible to load the floor with heat during high irradiation periods from a low hanging sun without severely compromising comfort?}}
\newcommand{\questionFour}{\textit{How well does the efficiency model of the heat pump predict the actual efficiency?}}
\newcommand{\questionFive}{\textit{How well does the efficiency model of the heat pump predict the actual efficiency?}}

\newcommand{\neogrid}{Neogrid Technologies}

\newcommand{\comfLvlTriangleOne}{{\color{ref1}\FilledBigTriangleUp}}
\newcommand{\comfLvlTriangleTwo}{{\color{ref2}\FilledBigTriangleUp}}
\newcommand{\comfLvlTriangleThree}{{\color{ref3}\FilledBigTriangleUp}}
\newcommand{\comfLvlTriangleFour}{{\color{ref4}\FilledBigTriangleUp}}

% Used values
\newcommand{\rsquaredInit}{0.92}
\newcommand{\rsquaredUpd}{0.96}


% Daily updated values
\newcommand{\percSaving}{11}
\newcommand{\numTestDays}{97}
\newcommand{\totalSavings}{27.3}
\newcommand{\savingRate}{0.28}
\newcommand{\yearsEarning}{36}
\newcommand{\comfTOnePercSave}{32.8}
\newcommand{\comfTwoPercSave}{29.1}
\newcommand{\comfThreePercSave}{2.3}
\newcommand{\comfFourPercSave}{14.3}
\newcommand{\extraPeakCost}{34.9}
\newcommand{\extraPeakElec}{60.0}
\newcommand{\avgEveningPrice}{0.27}
\newcommand{\copPeak}{4.2}
\newcommand{\costAfter}{16.1}
\newcommand{\savePeakBlock}{18.8}
\newcommand{\percPeakSave}{69}
\newcommand{\comfOneCostPerc}{32.8}
\newcommand{\comfTwoCostperc}{29.1}
\newcommand{\comfThreeCostPerc}{2.3}
\newcommand{\comfFourCostPerc}{14.3}
\newcommand{\comfOneCostPercNoDec}{32}
\newcommand{\comfTwoCostpercNoDec}{29}
\newcommand{\comfThreeCostPercNoDec}{2}
\newcommand{\comfFourCostPercNoDec}{14}
\newcommand{\expEndDate}{2023-03-05}
\newcommand{\comfFourNumDays}{31}
\newcommand{\comfOnePercSaveNoDec}{33}
\newcommand{\comfTwoPercSaveNoDec}{29}
\newcommand{\comfThreePercSaveNoDec}{2}
\newcommand{\comfFourPercSaveNoDec}{14}

\begin{frontmatter}
%\runtitle{Insert a suggested running title}  % Running title for regular 
                                              % papers but only if the title  
                                              % is over 5 words. Running title 
                                              % is not shown in output.

%\title{Realizing mixed-integer model predictive control with weather and price compensation for an occupied single-family house featuring a heat pump, floor heating and photovoltaic\thanksref{footnoteinfo}}
%\title{Realizing mixed-integer predictive control with weather and price compensation for an occupied single-family house\thanksref{footnoteinfo}}
\title{Long-term experimental study of price responsive predictive control in a real occupied single-family house with heat pump\thanksref{footnoteinfo}}
% Title, preferably not more 
                                                % than 10 words.
% \title{Practical realization of mixed-integer model predictive control for an occupied single-family house featuring a heat pump, floor heating and photovoltaic}

\thanks[footnoteinfo]{This work is funded the Danish Energy Agency through the EUDP project OPSYS 2.0 (Case num.: 64018-0581) and the Department of Electronic Systems at Aalborg University.}

\author[aau]{Simon Thorsteinsson}\ead{sith@es.aau.dk},    % Add the   % (ead) as shown
\author[neogrid]{Alex Arash Sand Kalaee}\ead{ask@neogrid.dk},
\author[neogrid]{Pierre Vogler-Finck}\ead{pvf@neogrid.dk},
%\author[neogrid]{Morten Veis Donnerup}\ead{mvd@neogrid.dk},
\author[neogrid]{Henrik Lund St\ae rmose}\ead{hls@neogrid.dk},
\author[dti]{Ivan Katic}\ead{ik@teknologisk.dk},
\author[aau]{Jan Dimon Bendtsen}\ead{dimon@es.aau.dk}

\address[aau]{Section for Automation and Control, Aalborg University, Aalborg, Denmark}  % Please supply         
\address[neogrid]{Neogrid Technologies ApS, Aalborg, Denmark}  % Please supply
\address[dti]{Danish Technological Institute, Taastrup, Denmark}
% here.

          
% \begin{keyword}                           % Five to ten keywords,  
% Heat pump; Model predictive control; Mixed-integer; Hierarchical; Near zero emission building; Floor heating; Photovoltaic; Real experiment; Demonstration; Load shifting; Price responsive; Demand side management.               % chosen from the IFAC 
% \end{keyword}                             % keyword list or with the 
                                          % help of the Automatica 
\begin{keyword}                           % Five to ten keywords,  
Hierarchical model predictive control; Heat pump; Retrofit building control; Real experiment; Load shifting; Demand side management.
\end{keyword}                               % keyword wizard

% \begin{abstract}                          % Abstract of not more than 200 words.
% The accelerating introduction of renewable electricity and increased consumption through electrification of the transport and heating sector challenges grid stability. The development requires new solutions, where one is \loadshifting\ through \dsm. This work presents a four month experimental study of a low complexity, hierarchical \mpc\ (\MPC) implemented on a real \singleFam\ house occupied by its residents. The house---located in Denmark---is a \nzeb\ heated by a commercial, weather compensated \atow\ heat pump in combination with floor heating. % and equipped with photovoltaic panels on the roof.
%  The upper-layer \MPC\ uses the price signal, weather forecast, a single-zone building model and a non-linear heat pump efficiency model to generate a \spaceHeating\ schedule. The commercial heat pump is made to act \smartGrid-ready through input-manipulation to enable heat boosting. 
% %The pre-installed heat pump is then input-manipulated, using a fake ambient temperature signal, in order to ramp production up and down according to a provided heating schedule, which, in effect, makes the \hp\ act \smartGrid\ ready.
% %The price signal is modelled based on the actual market spot-price, for the price area DK2, and the, so called, \textit{Tariff model 3.0} providing a realistic foundation for the experiment.

% The question, today, is not whether a complex \MPC\ algorithm is superior to rule-based controllers---this is known---but whether the benefits outweigh the costs and whether the method will ever be broadly accepted and implemented. To answer this, we need to carry out realistic, long running studies to prove the benefits, discuss the costs and uncover the performance loses going from simulation to reality, to finally determine whether this is a viable path forward.

% The cost reduction, gained from the controller, ranged from \comfThreeCostPercNoDec-\comfOneCostPercNoDec\% depending on the comfort level, we share empirical evidence of the following: (1) The test house have a large time-of-heating flexibility without having an energy storage. (2) It is still possible to save a reasonable amount of money, in a low energy house, just by lowering the temperature reference half a degree; (3) The current price scheme provides incentive for Danish end-consumers to shift heating loads. (4) Concrete floors can be boosted with heat without degrading indoor comfort.
% %The study contains the following: (1) Proof of concept that existing commercial \hp\ with \fh\ can be retrofitted for coordinated control instead of the individual control widely implemented today; (2) Design  (2) Investigation of energy flexibility of a \NZEB\ house; (3) Experimental evaluation of the chosen control algorithm.
% Although, the experiment itself is a success, the authors have identified several issues related to controlling the heat pump through input-manipulation and proposes a more prosperous path forward, which entails coordination with the manufactures, in order to make commercial heat pumps more \smartGrid\ ready.% Despite the solution presented here concerns future perspectives for \DSM\ two immediately implementable rules, for the Danish end-consumers, have been derived from the data: The rules are: \textit{If comfort allows it, block the heat pump in the four hour cooking-peak period} and \textit{All thermostats in a low energy house should be lowered with half a degree to see if comfort is reduced}.
% %Although, the experiment itself is a success, the authors have identified a gab between general expectations from the academic community with respect to \smartGrid readiness of fielded heat pumps and their actual control capabilities. 
% %This gab can partly be closed with the method presented here, but the view of the authors is that a more prosperous path forward is to include the manufactures in the process and develop a heat pump which can mitigate future stability challenges for the electric grid.
% %Although, the experiment itself is a success, the authors have identified a gab between general expectations from the academic community with respect to \smartGrid readiness heat pumps and the actual control capabilities. 
% %several issues related to controlling the the heat pump through input- and proposes an alternate path forward for the sector. 
% %Having shown that a \NZEB\ house can \loadShift\ \spaceHeating and thereby support the electricity grid.
% %Last, based on the obtained results a simple, immediately implementable, cost saving control rule has been derived for the Danish energy market. The rule is: \textit{If comfort allows it, block the heat pump in the four hour cooking-peak period}
% \end{abstract}

% \begin{abstract} % Original
% The continuous introduction of renewable electricity and increased consumption through electrification of the transport and heating sector pose stability challenges to the grid.
% This study investigates load shifting through demand side management as a solution. We present a four-month experimental investigation of a low-complexity, hierarchical Model Predictive Control approach for demand side management in a near-zero emission single-family house in Denmark. The control algorithm uses a price signal, weather forecast, a building model, and a non-linear heat pump efficiency model to generate a space-heating schedule. The cost reduction from the controller ranged from 3-36\% depending on the comfort level. The experiment demonstrates that load shifting is feasible and cost-effective, even without energy storage, and that the current price scheme provides an incentive for Danish end-consumers to shift heating loads. However, issues related to controlling the heat pump through input-manipulation were identified, and the authors propose a more prosperous path forward involving coordination with manufacturers and regulators to make commercial heat pumps more smart grid-ready.
% \end{abstract}

\begin{abstract}
The continuous introduction of renewable electricity and increased consumption through electrification of the transport and heating sector challenges grid stability. This study investigates \loadshifting\ through \dsm\ as a solution. We present a four-month experimental study of a low-complexity, hierarchical Model Predictive Control approach for \dsm\ in a near-zero emission occupied single-family house in Denmark. The control algorithm uses a price signal, weather forecast, a single-zone building model, and a non-linear heat pump efficiency model to generate a \spaceHeating\ schedule. The weather-compensated, commercial heat pump is made to act \smartGrid-ready through outdoor temperature input override to enable heat boosting and forced stops to accommodate the heating schedule. The cost reduction from the controller ranged from \comfThreePercSaveNoDec-\comfFourPercSaveNoDec\% depending on the chosen comfort level. The experiment demonstrates that \loadshifting\ is feasible and cost-effective, even without energy storage, and that the current price scheme provides an incentive for Danish end-consumers to shift heating loads. However, issues related to controlling the heat pump through input-manipulation were identified, and the authors propose a more promising path forward involving coordination with manufacturers and regulators to make commercial heat pumps truly smart grid-ready.% Besides being expensive and time consuming, long-term experiments are difficult to evaluate because of the many uncontrollable externalities, but due to the fundamental importance of field experiments extra emphasis is put on the evaluation procedure and subsequent discussion.% This is done to invite scrutiny of method and provide direction for future experiments of such kind.
%Long-term experiments are rare because they are expensive and time consuming, Further, the many uncontrollable externalities make them difficult to evaluate, but due to their fundamental importance extra emphasis is put on the evaluation procedure and subsequent discussion.
\end{abstract}

\end{frontmatter}

%\begin{multicols}{2}
\section{Introduction}
\label{sec:introduction}
% \begin{itemize}
%     % Diffusion of FL
%     \item {\st{Diffusion of FL}}
%     % Security threats to FL
%     \item {\st{Security threats to FL with particular focus on model poisoning}}
%     % Limitations of existing countermeasures
%     \item {\st{Current countermeasures (e.g., KRUM) and their limitations}}
%     % Proposed method and its advantages
%     \item {\st{Intuitive description of the proposed method and its difference (i.e., advantages) w.r.t. state of the art}}
%     % Main contributions
%     \item {\st{Summary of the main contributions of this work}}
%     % Paper's structure and organization
%     \item {\st{Paper's structure and organization}}
% \end{itemize}

% Diffusion of FL
Recently, {\em federated learning} (FL) has emerged as the leading paradigm for training distributed, large-scale, and privacy-preserving machine learning (ML) systems~\cite{mcmahan2017googleai,mcmahan2017aistats}. 
The core idea of FL is to allow multiple edge clients to collaboratively train a shared, global model without disclosing their local private training data.
%Specifically, an FL system consists of a central server and many edge clients; 
A typical FL round involves the following steps: {\em(i)} the server randomly picks some clients and sends them the current, global model; {\em(ii)} each selected client locally trains its model with its own private data; then, it sends the resulting local model to the server;\footnote{Whenever we refer to global/local model, we mean global/local model {\em parameters}.} {\em(iii)} the server updates the global model by computing an \emph{aggregation function}, usually the average (FedAvg), on the local models received from clients.
% \begin{enumerate}
%     \item[{\em(i)}] the server sends the current, global model to the clients and appoints some of them for training;
%     \item[{\em(ii)}] each selected client locally trains its copy of the global model with its own private data; then, it sends the resulting local model back to the server;\footnote{Whenever we refer to global/local model, we mean global/local model {\em parameters}.}
%     \item[{\em(iii)}] the server updates the global model by computing an \emph{aggregation function} on the local models received from clients (by default, the average, also referred to as FedAvg~\cite{mcmahan2017aistats}).
% \end{enumerate}
This process goes on until the global model converges. %(e.g., after a certain number of rounds or other similar stopping criteria).
%\\
% The advantages of FL over the traditional, centralized learning paradigm are undoubtedly clear in terms of flexibility/scalability (clients can join/disconnect from the FL network dynamically), network communications (only model weights\footnote{We will use \textit{parameters} and \textit{weights} interchangeably.} are exchanged between clients and server), and privacy (each client's private training data is kept local at the client's end and not uploaded to the server).
\\
% Security threats to FL
%However, the growing adoption of FL also raises security concerns~\cite{costa2022covert}, particularly about its confidentiality, integrity, and availability.
Although its advantages over standard ML, FL also raises security concerns~\cite{costa2022covert}. %, particularly about its confidentiality, integrity, and availability~\cite{costa2022covert}.
% OLD, LONG VERSION
% Indeed, some work deals with privacy leakage that may expose the local data of some clients~\cite{melis2019sp}. 
% A large body of work, instead, investigates attacks that usually aim to detriment the predictive accuracy of the learned global model. For instance, \emph{data poisoning} attacks achieve this goal by letting an adversary pollute the training set of some corrupt FL clients with maliciously crafted examples~\cite{jagielski2018sp}.
% Similarly, in \emph{model poisoning} the attacker attempts to tweak the global model weights~\cite{bhagoji2019pmlr} by directly perturbing the local model's weights of some infected FL clients before these are sent to the central server for aggregation, usually via so-called Byzantine attacks. 
% It turns out that Byzantine model poisoning attacks severely impact standard FedAvg; therefore, more robust aggregation functions must be designed to make FL systems secure.
Here, we focus on \emph{untargeted model poisoning} attacks~\cite{bhagoji2019pmlr}, where an adversary attempts to tweak the global model weights %\footnote{We will use the terms \textit{parameters} and \textit{weights} interchangeably.} 
by directly perturbing the local model's parameters of some infected clients before these are sent to the central server for aggregation.
In doing so, the adversary aims to jeopardize the global model \textit{indiscriminately} at inference time.
Such model poisoning attacks severely impact standard FedAvg; therefore, more robust aggregation functions must be designed to secure FL systems.
\\
% In this paper, we focus on designing a novel robust aggregation scheme at the server's end to contrast the effect of Byzantine model poisoning attacks.
%
% Current countermeasures and their limitations
%Several countermeasures have been proposed in the literature to combat model poisoning attacks on FL systems.
% Some methods use simple statistics more robust than plain average to smooth the impact of malicious updates (e.g., Trimmed Mean and FedMedian~\cite{yin2018icml}). 
% Other defenses implement outlier detection techniques to discard malicious updates from the aggregation performed at the server's end. Those are either based on heuristics (e.g., Krum/Multi-Krum~\cite{blanchard2017nips} and Bulyan~\cite{mhamdi2018pmlr}) or data-driven approaches (e.g., K-means clustering~\cite{shen2016acm} or DnC via spectral analysis~\cite{shejwalkar2021ndss}). 
% Finally, some strategies rely on a centralized ``source of trust'' to spot potential malicious updates (e.g., FLTrust~\cite{cao2020fltrust}).
% Several countermeasures have been proposed in the literature to combat model poisoning attacks on FL systems, i.e., to discard possible malicious local updates from the aggregation performed at the server's end. 
% These techniques range from simple statistics more robust than plain average (e.g., Trimmed Mean and FedMedian~\cite{yin2018icml}) to outlier detection heuristics (e.g., Krum/Multi-Krum~\cite{blanchard2017nips} and Bulyan~\cite{mhamdi2018pmlr}) or data-driven approaches (e.g., spectral analysis via K-means clustering~\cite{shen2016acm} or spectral analysis), or methods based on ``source of trust'' (e.g., FLTrust~\cite{cao2020fltrust}).
% OLD, LONG VERSION
%Several countermeasures have been proposed in the literature to combat Byzantine model poisoning attacks on FL systems.
% Descriptive statistics
% For example, Trimmed Mean and FedMedian aggregate local model updates using more robust statistics than standard average~\cite{yin2018icml}.
%
% % Heuristics for outlier detection
% Many existing Byzantine-resilient strategies implement some outlier detection heuristics to discard the model updates sent by potentially malicious clients from the input of the aggregation function.
% One of the most popular heuristics is Krum~\cite{blanchard2017nips}.
% This strategy tries to mitigate the impact of Byzantine attacks by selecting as a global model the local model with the smallest sum of Euclidean distances to {\em all} the other local models.
% Although powerful, Krum requires the server to know (or, at least, estimate) the number of malicious FL clients upfront, which is generally impossible in a realistic attack scenario. %
% Moreover, Krum may become ineffective for complex, high-dimensional model parameter spaces due to the curse of dimensionality.
% Bulyan~\cite{mhamdi2018pmlr} tries to overcome this issue by combining Krum with a variant of Trimmed Mean.
% % Data-driven outlier detection
% Other strategies use data-driven outlier detection techniques -- e.g., via K-means clustering~\cite{shen2016acm} -- to spot potential malicious local model updates. 
% %For instance, Shen et al. propose to cluster local model updates with K-means and thus identify outliers.
%
% % Other techniques
% As far as the server is concerned, any local model received can be from a potential malicious client. 
% FLTrust~\cite{cao2020fltrust} assumes the server acts as a client, i.e., trains a local model on an additional {\em trustworthy} dataset at the server's end and compares it against all the local models from other clients. 
% This way, the server can rely on some ``source of trust'' when discarding potentially malicious clients.
%\\
% Limitations of existing Byzantine-resilient strategies
Unfortunately, existing defense mechanisms either rely on simple heuristics (e.g., Trimmed Mean and FedMedian by~\cite{yin2018icml}) or need strong and unrealistic assumptions to work effectively (e.g., foreknowledge or estimation of the number of malicious clients in the FL system, as for Krum/Multi-Krum~\cite{blanchard2017nips} and Bulyan~\cite{mhamdi2018pmlr}, which, however, cannot exceed a fixed threshold).
Furthermore, outlier detection methods using K-means clustering~\cite{shen2016acm} or spectral analysis like DnC~\cite{shejwalkar2021ndss} do not directly consider the temporal evolution of local model updates received.
Finally, strategies like FLTrust~\cite{cao2020fltrust} require the server to collect its own dataset and act as a proper client, thereby altering the standard FL protocol.
\\
% OLD, LONG VERSION
% Overall, existing Byzantine-resilient strategies are either simple heuristics (e.g., FedMedian) or, if they are more complex, they rely on strong and unrealistic assumptions to work effectively (e.g., knowing the number of malicious clients in the FL system in advance, as for Krum and alike).
% Furthermore, data-driven outlier detection methods do not consider the temporary evolution of local model updates received (e.g., K-means clustering). 
% Finally, strategies like FLTrust requires the server to collect its own dataset and act as a proper client, thereby altering the standard FL protocol.
%
% Description of the proposed method
This work introduces a novel pre-aggregation \textit{filter} robust to untargeted model poisoning attacks. Notably, this filter $(i)$ operates without requiring prior knowledge or constraints on the number of malicious clients and $(ii)$ inherently integrates temporal dependencies. 
The FL server can employ this filter as a preprocessing step before applying \textit{any} aggregation function, be it standard like FedAvg or robust like Krum or Bulyan.
Specifically, we formulate the problem of identifying corrupted updates as a multidimensional (i.e., matrix-valued) time series anomaly detection task. 
The key idea is that legitimate local updates, resulting from well-calibrated iterative procedures like stochastic gradient descent (SGD) with an appropriate learning rate, show \textit{higher predictability} compared to malicious updates. This hypothesis stems from the fact that the sequence of gradients (thus, model parameters) observed during legitimate training exhibit regular patterns, as validated in Section~\ref{subsec:intuition}. %until convergence. 
%This regularity may be more pronounced for smooth convex loss functions, but it can still be captured within an appropriate time window, even for more complex and convoluted loss surfaces. 
%We provide evidence of this claim in Appendix~B, where we show that the average mutual information (i.e., ``predictability''), calculated over pairs of legitimate model updates sent at different FL rounds, is significantly higher than the corresponding computation for a malicious client.
\\
Inspired by the matrix autoregressive (MAR) framework for multidimensional time series forecasting~\cite{chen2021je}, we propose the FLANDERS ({\em \textbf{F}ederated \textbf{L}earning meets \textbf{AN}omaly \textbf{DE}tection for a \textbf{R}obust and \textbf{S}ecure}) filter.
The main advantages of FLANDERS over existing strategies like FLDetector~\cite{zhao2020multivariate} are its resilience to large-scale attacks, where $50\%$ or more FL participants are hostile, and the capability of working under realistic non-iid scenarios.
We attribute such a capability to two key factors: $(i)$ FLANDERS works without knowing a priori the ratio of corrupted clients, and $(ii)$ it embodies temporal dependencies between intra- and inter-client updates, quickly recognizing local model drifts caused by evil players. Below, we summarize our main contributions:

\begin{itemize}
\item[{\em(i)}]
We provide empirical evidence that the sequence of models sent by legitimate clients is more predictable than those of malicious participants performing untargeted model poisoning attacks.
\\
\item[{\em(ii)}] 
We introduce FLANDERS, the first pre-aggregation filter for FL robust to untargeted model poisoning based on multidimensional time series anomaly detection.
\\
\item[{\em(iii)}] 
We integrate FLANDERS into Flower,\footnote{\scriptsize{\url{https://flower.dev/}}} a popular FL simulation framework for reproducibility.
\\
\item[{\em(iv)}] 
We show that FLANDERS improves the robustness of the existing aggregation methods under multiple settings: different datasets, client's data distribution (non-iid), models, and attack scenarios.
\\
\item[{\em(v)}] 
We publicly release all the implementation code of FLANDERS along with our experiments.\footnote{\scriptsize{\url{https://anonymous.4open.science/r/flanders_exp-7EEB}}}
\end{itemize}

% Paper's structure and organization
The remainder of the paper is structured as follows. %some related work and the current state-of-the-art solutions to security issues that FL entails. 
Section~\ref{sec:background} covers background and preliminaries. 
In Section~\ref{sec:related}, we discuss related work.
Section~\ref{sec:problem} and Section~\ref{sec:method} describe the problem formulation and the method proposed. % to tackle it. 
Section~\ref{sec:experiments} gathers experimental results. %, and Section~\ref{sec:limitations} discusses some limitations of this work.
Finally, we conclude in Section~\ref{sec:conclusion}.
 %discusses the limitations of this work and draws future research directions.
%reports conclusions and draws perspectives for future research directions.

%%%%%%% OLD %%%%%%%
%to overcome the resilience of Byzantine failures in distributed Stochastic Gradient Descent computations. 
% The strength of Krum is its time complexity, which is linear in the gradient dimension. 
% However, the robustness of the approach is guaranteed for gradient-based learning applications only when the majority of the clients are not compromised. 
% Besides, the aggregation mechanism of Krum, as well as that of similar methods, is robust from a coarse-grained perspective and does not provide solutions to errors and perturbations that may occur at inference time.
%A related approach to~\cite{blanchard2017nips} is the work of Su et al.~\cite{su2016dc}. Here, the authors propose an iterated approximate agreement to tackle a multi-layer scenario attacked by Byzantine agents. 
%However, the method works efficiently on the sole discrete context and it is inapplicable to continuous state environments.
%\gabri{Maybe, we should just talk about the main limitations of existing countermeasures without digging into their details (or, we can just mention Krum as this is the most popular one). I will move the description of all these methods to the Related Work section.}
\section{Method}
\label{sec:method}

% \ml{``Inconsistent'' to ``large variation''}

% In this section, we propose our methods based on the observations in Section \ref{sec:motivation}.
In this section, we propose two techniques to further enhance the strong baseline to capture the variation of activation distributions better.
We first introduce spatial re-scaling to adapt the network to pixel-to-pixel variation.
We then propose channel-wise shifting and re-scaling to better capture the channel-to-channel variation.
Meanwhile, as both of the two methods are image-dependent, the image-to-image variation can be captured naturally.
By combining the two methods with our strong baseline, we build our enhanced BNN for SR, named EBSR.

% Because the activation distributions among pixels, channels and images have large variations \red{**are highly inconsistent} in SR networks, we introduce spatial re-scaling to adapt to pixel-wise variations and channel shift and re-scaling to adapt to channel-wise variations. And both of them are image-dependent to adapt to image-wise variations, which means during inference our network re-scales and shifts the distributions of activations flexibly for different input images. Based on these methods, we build an enhanced binary neural network for image super-resolution (EBSR).

% According to [3], the difference of activation magnitudes indicates different scaling factors are needed for each pixel.

\subsection{Spatial Re-scaling}
% It is better to use different scaling factors for different pixels to reduce the quantization error and retain more detailed information for image super-resolution. 

% \ml{In the main method, we do not need to introduce the previous works but can focus on introducing our own method. Channel rescaling in Real-to-binary Net is not relevant in this context.}

% Re-scaling the output of binary convolutions was proposed at the birth of BNN in XNOR-Net \cite{rastegari2016xnor} to reduce quantization error and improve accuracy for image classification tasks.
% It is computed as below:
% \begin{equation}
% \mathcal{A} * \mathcal{W} \approx(\operatorname{sign}(\mathcal{A}) \circledast \operatorname{sign}(\mathcal{W})) \odot \mathcal{K} \alpha
% \label{eq:xnor-net rescale}
% \end{equation}
% where $\circledast$ denotes the binary convolution and $\odot$ denotes the element-wise multiplication.
% $\mathcal{A}$, $\mathcal{W}$, $\alpha$, and $\mathcal{K}$ denote the activation, weight, weight scaling factor, and activation scaling factor, respectively.
%  Later in XNOR-Net++ \cite{bulat2019xnor}, Bulat et al. fuse the activation and weight scaling factors into a single one that is learned end-to-end based on gradients and this improves the classification accuracy on ImageNet dataset.

% % It is computed as Eq.~\ref{eq:xnor-net rescale}, where $\circledast$ denotes 
% %  the binary convolution and $\odot$ denotes the element-wise multiplication. The binary convolution of $\mathcal{A}$ and $\mathcal{W}$ is rescaled by the weight scaling factor $\alpha$ and the activation scaling factor $\mathcal{K}$, both of which are calculated analytically.


% \zc{Similarly, you should explain the meaning of A, W and the operators $\circledast$ in the formula}
% Then in Real-to-binary Net \cite{martinez2020training}, Martinez et al. used a data-driven channel re-scaling module that takes the pre-convolution activations as input to predict the activation scaling factor. Unlike that in XNOR-Net++ \cite{bulat2019xnor}, these scaling factors are not fixed during inference but rather inferred from data. By doing this, they further improved the classification accuracy on ImageNet over XNOR-Net++. 
As is shown in Figure \ref{fig:pixel}, activation distributions have large pixel-to-pixel variation in SR networks
and the difference of activation magnitudes indicates different scaling factors are preferred for different pixels.
Inspired by \cite{martinez2020training}, we propose spatial re-scaling to better adapt the network to the spatial variation
of activation distributions in SR networks.
% fit the various pixel-wise distributions in SR networks.
We take the real-valued activations $A$ before convolution as input and predict pixel-wise scaling factors $S(A)$, which re-scale the binary convolution output. Spatial re-scaling process can be formulated as follows:
\begin{equation}
A * W \approx(\operatorname{sign}(A) \circledast \operatorname{sign}(W)) \odot \alpha \odot S(A)
\label{eq:spatial rescale}
\end{equation}
where $\circledast$ denotes 
the binary convolution and $\odot$ denotes the element-wise multiplication. $A$, $W$, $\alpha$, and $S\left(A\right)$ denote real-valued activations, weights, the scaling factor of weights, and the spatial-wise scaling factor of activations respectively. $S\left(A\right) \in \mathbb{R}^{1\times H\times W}$ can be calculated with a convolution and a sigmoid function.
% as $\sigma\left( CONV\left(A\right)\right)$. 
As shown in Figure \ref{fig:method}(a), real-valued activations first go through a convolution layer,
which has an input channel of $C$ and an output channel of 1, 
and then pass through a sigmoid function to produce the scaling factors $S(A)$ along the spatial dimension.
During inference, the scaling factor will change dynamically according to different input feature maps.
By re-scaling binary convolution output using $S(A)$, we can reduce the quantization error and the original pixel-wise information in FP activation
will be preserved much better.
Spatial re-scaling leads to a large PSNR improvement of 0.24 dB (from 30.30 dB to 31.54 dB) on Set5 and 0.22 dB (from 25.09 dB to 25.31 dB)
on Urban100 compared with our strong baseline. 

\subsection{Channel-wise Shifting and Re-scaling}

\begin{table}[!tb]
\centering
\caption{Comparison between whether to fuse channel-wise shifting and re-scaling or not based on our baseline with spatial re-scaling. }
\label{tab:fusing}

\scalebox{0.65}{
\begin{tabular}{c|cc|cc|cc}
\hline
\multirow{2}{*}{Method}     & \multirow{2}{*}{OPs} & \multirow{2}{*}{Params} & \multicolumn{2}{c|}{Set5} & \multicolumn{2}{c}{Urban100} \\ \cline{4-7} 
                            &                      &                         & PSNR        & SSIM        & PSNR          & SSIM         \\ \hline
Baseline + spatial re-scale & 2.16G                & 0.05M                   & 31.54       & 0.883       & 25.31         & 0.759        \\
+ channel-wise shift and re-scale             & 2.34G                & 0.09M                   & 31.61       & 0.885       & 25.35         & 0.761        \\
+ w/ fusing                   & 2.27G                & 0.08M                   & \textbf{31.64}       & \textbf{0.885}       & \textbf{25.36}         & \textbf{0.761}        \\ \hline
\end{tabular}
}
\end{table}

In SR networks, activation distributions exhibit larger channel-to-channel variation (Figure \ref{fig:chl}).
Both the mean and magnitude of the activation distributions vary significantly across channels.
% Thus we use channel-wise shifting and re-scaling to adapt to various channel-wise distributions. 
\cite{martinez2020training} has proposed the data-driven channel re-scaling, 
but our method differs from them in further introducing data-driven thresholds to handle the channel-wise variation of both mean and magnitude.
Since the blocks to generate the scaling factors and thresholds are very similar, we further propose to fuse them into one module.
% and fusing channel-wise shifting and re-scaling into one module.
We evaluate the effect of fusing the two blocks in Table \ref{tab:fusing}.
With channel-wise shifting and re-scaling fused, our models have fewer operations and parameters overhead and slightly higher performance.

For the specific process, we take the real-valued activations as input and predict different thresholds and scaling factors for each channel. They are also image dependent, e.g., $\beta_{i}$ in Eq.\ref{eq:act_binarize} is no longer fixed during inference but generated according to different input feature maps. Channel-wise shifting and re-scaling can be formulated as follows:
\begin{equation}
A * W \approx(\operatorname{sign}(A-C_s(A)) \circledast \operatorname{sign}(W)) \odot \alpha \odot C_r(A)
\label{eq:channel-wise_shift_and_rescale}
\end{equation}
where $\circledast$ denotes 
the binary convolution and $\odot$ denotes the element-wise multiplication. $C_s(A), C_r(A) \in \mathbb{R}^{C\times1\times1}$ denote the channel-wise threshold and scaling factor, respectively. 
We show the block diagram in Figure \ref{fig:method}(b).
The real-valued input feature map is first squeezed to a ${C\times1\times1}$ vector by a global average pooling (GAP) layer.
The subsequent fully connected layers and ReLU learn the channel-wise information and output a ${2C\times1\times1}$ vector.
Then the ${2C\times1\times1}$ vector is split into two ${C\times1\times1}$ vectors.
We use the first $C$ channels as the channel-wise bias and pass the last $C$ channels through a sigmoid layer 
as the channel-wise scaling factor, which are used to shift the real-valued activations and re-scale the binary convolution output, respectively. 


% \ml{We can mention previously, channel-wise re-scale has been proposed. We propose to fuse them. Add the comparison between fuse v.s. no fuse.}

\begin{figure}[!tbp]%
  \centering
    \includegraphics[width=0.4\textwidth]{fig/methods.png}
  
% \subfloat[channel-wise shifting\&re-scale]{
%     \label{subfig:channel-wise shifting and re-scale}
%     \includegraphics[width=0.2\textwidth]{fig/chl shift and rescale.png}
%   }

  \caption{Block diagram for spatial re-scaling, and channel-wise shifting and re-scaling.} 
  % Input A is the real-valued activation tensor and C, H, and W denote its dimension. GAP stands for global average pooling. The reduction ratio r is set to 16 for a better trade-off between the performance and the number of operations and parameters.}
  \label{fig:method}
\end{figure}


\subsection{Network Structure}

Combining the spatial re-scaling and the channel-wise shifting and re-scaling methods, we construct the enhanced convolution layer (E-Conv).
Then we build our EBSR model based on E-Conv.
In Figure \ref{fig:E-conv}, we compare the binary convolution layer used in the baseline network and our proposed E-Conv.
We use spatial and channel-wise scaling factors to re-scale the binary convolution output,
and use channel-wise shifting to learn appropriate thresholds for each channel before binarization.
The scaling factors and threshold used in E-Conv are learnable and depend on the real-valued input activations.
In this way, our proposed EBSR can adapt to pixel-to-pixel, channel-to-channel, and image-to-image variations
to reduce the large binarization error and preserve more details.
% In this way, our proposed E-Conv reduces the large quantization error caused by binarization and keeps the original information of input feature maps to a large extent.


\begin{figure}[!tb]%
  \centering

    \includegraphics[width=0.5\textwidth]{fig/E-conv.png}

  \caption{Comparison of (a) the binary convolution layer with a skip connection used in our baseline network and (b) the proposed E-Conv.}
  \label{fig:E-conv}
\end{figure}


Figure \ref{fig:network} shows the basic block based on the E-Conv and our EBSR composed of the basic blocks. Following existing works, the convolution layers in the head and tail modules are not binarized. We choose the lightweight EDSR which has 16 basic blocks and 64 channels, and EDSR which has 32 basic blocks and 256 channels as our backbones, which correspond to EBSR-light and EBSR, respectively.

\begin{figure}[!tb]%
  \centering
  {
    \includegraphics[width=0.35\textwidth]{fig/network.png}
  }
  
  \caption{The structure of our proposed EBSR.  Convolution layers in purple are real-valued vanilla 3x3 convolutions.}
  \label{fig:network}
\end{figure}
\section{Results}
\label{results}

\begin{figure*}[ht]
    \centering
    \includegraphics[scale=0.15,trim={0 2.5cm 0 5cm},clip]{images/aoi-single_burst}
    \caption{The time average peak Age of Information with burst and \gls{soa} loss values against the dynamic reliability logic for different network topologies.}
    \label{fig:aoi_burst}\vspace{-0.4cm}
\end{figure*}


This paper focuses on both transport layer and application layer metrics to determine the feasibility of dynamic reliability. For this, we have selected the session packet volume, as transmitted, retransmitted, lost and backlogged packets as \glspl{kpi} for the transport layer; while focusing on the \gls{aoi} for the application layer. The \gls{aoi} was chosen as a crucial indicator for the freshness of packets in real-time applications. More specifically, this work adopts the time average peak \gls{aoi} equation \cite{aoi_equation} depicted in Eq. \ref{aoi}, where $\Delta(r_{i+1})$ is the $i$th update at the time it was received at the server, for a session time period of $\tau$.

\begin{equation}
    \label{aoi}
    \gls{aoi}_\tau = \frac{1}{n-1}\sum_{i=1}^{n-1} \Delta(r_{i+1})
\end{equation}

We include a comparison between the vanilla QUIC implementation which does not enjoy the dynamic reliability extension, with a number of dynamic reliability policies. The tests were run a number of times for statistical significance, with the mean value of vanilla implementation used as a baseline for comparison. The topology utilised both random loss and bursty loss to explore the bounds of dynamic reliability. The \gls{soa} loss in the figures correspond to the loss values presented in Table. \ref{tab:path_char}, for ease of comparison between bursty and random loss scenarios.

\subsection{Transport-Layer KPIs}

To analyse the performance gain at the transport layer due to dynamic reliability, the volume of transmitted and backlogged packets is examined. The figures are in the form of boxplots, which take the vanilla implementation as a benchmark, depicted as the red dashed line.

As seen in Fig. \ref{fig:sent_burst}, the loss plays a crucial role in the performance of the reliability policies. The policies under random loss did incredibly well for the networks with a larger capacity, namely \gls{mmwave} and Sub-6~GHz, whereas for burst loss, the lower network capacities had a larger packet reduction. With the increase in burst loss, the behaviour of the set split reliable policies became unpredictable, if a reliable assignment happened to coincide with a burst loss, the number of transmitted packets increases, and vice versa. On the other hand, in smarter policies, such as Loss-Aware, the performance lightly matched the vanilla baseline, as the reliable assignment dominated the session to compensate for a higher burst loss. Not only that but, the burst loss also impacted the variance of the transmitted packets for the policies.

Unsurprisingly, the unreliable focused policy, 80-20 split, outperformed other policies for all topologies in random and bursty loss scenarios, with an approximate reduction of 80\%. That being said, the majority of the policies reduced the transmitted packets on the link by approximately 70\% for random loss, while the reduction started at $\approx 15\%$ and decreased as the loss increased for the burst loss scenario.

The retransmitted and lost packets, not shown due to space limitations, followed the same trend as the transmitted packets for the random loss scenarios. However, for the burst loss scenarios, the larger capacity networks had a lower reduction in the retransmitted and lost packets. This can be seen as a favorable outcome since the lower capacity networks are scarce on resources. It is important to note that the Loss-Aware policy mimicked the vanilla approach as the burst loss increased, signifying the overwhelming appointment of reliable packets in adapting to the harsh burst loss conditions.
 
Alternatively, Fig. \ref{fig:backlog_burst} clearly shows a stark comparison between the policies and loss scenario in the reduction of the backlogged packets. The Loss-Aware policy for random loss scenario reduced the backlogged packets by up to 50\%, beating all other policies by approximately 30\%. Furthermore, it is clear that the unreliability focused policies resulted in the lowest backlog for the session. In comparison, we notice that the burst loss and the backlogged frequency have a positive correlation, where the maximum reduction of the backlogged packets for the policies is at most 20\%. Much like the transmitted packets, the probability of a burst loss occurrence plays a vital role in the number of retransmissions sent and by extension the number of backlogged packets. Thus, we can conclude that the stress placed on the buffer is a result of the reliable packets which is tightly coupled with the congestion on the session. Whereas, unreliable focused policies did not encounter such a phenomenon regardless if it was experiencing a burst loss.


\subsection{Application-Layer KPIs}

The feasibility of dynamic reliability for real-time applications can be determined by the \gls{aoi}, with comparison across different topologies and policies. If we take a strict approach and consider anything below $10$~ms is real-time \cite{real-time}, then all the reliability policies passed that requirement, which is attractive for real-time applications, as shown in Fig. \ref{fig:aoi_burst}. Utilising the median as an estimate of the runs, the policies in the WLAN and Sub-6~GHz topology with random loss floated around $4-5$~ms with negligible difference, while the \gls{aoi} for \gls{mmwave} was $\approx 2-3$~ms. It is clear that the \gls{aoi} and the network capacity have a negative correlation, as the network capacity decreases, the \gls{aoi} increases. The same correlation is extended to the bursty loss scenarios, where \gls{mmwave} dominated the other topologies. That being said, it is crucial to note that the \gls{aoi} for the reliability policies is often slightly better than or equal to the \gls{aoi} of the vanilla implementation, proving that dynamic reliability reduces the congestion of the session at no cost to the \gls{aoi}.

\section{Conclusion}\label{sec:conclusion}
In this work, we focus on addressing the fundamental challenge of OOD detection tasks, which is how to fully understand the semantic discrepancy between the ID/OOD samples. We reveal that the key to success in the realistic SCOOD task is to allocate as many ID samples in the unlabeled set correctly as possible. To this end, we propose a novel uncertainty-aware optimal transport scheme that introduces class-specific energy scores as guidance for effective label assignment. Experimental results show that our method achieves better performance than previous state-of-the-art methods on SCOOD benchmarks.

\textbf{Limitations.} In addition to temperature scaling, other techniques such as feature clipping applied in ReAct~\cite{sun2021react} also enhance the performance of energy score, so how to obtain an OOD score that best fits the SCOOD task can be further explored. Moreover, a setting highly related to SCOOD has been proposed in \cite{katz2022training} and formulated as a constrained optimization problem. We will also theoretically analyze these practical OOD settings in our feature work.

% \section*{Acknowledgments}
\textbf{Acknowledgments.} 
This work is supported by National Key R\&D Program of China under Grant 2020AAA0105701, National Natural Science Foundation of China (NSFC) under Grants 61872327, Major Special Science and Technology Project of Anhui, National Natural Science Foundation of China (62033012) and Ant Group through Ant Research Intern Program.

% \author[aau]{Simon Thorsteinsson}\ead{sith@es.aau.dk},    % Add the   % (ead) as shown
% \author[neogrid]{Alex Arash Sand Kalaee}\ead{ask@neogrid.dk},
% \author[neogrid]{Pierre Vogler-Finck}\ead{pvf@neogrid.dk},
% \author[neogrid]{Henrik Lund Stærmose}\ead{hls@neogrid.dk},
% \author[dti]{Ivan Katic}\ead{ik@teknologisk.dk},
% \author[aau]{Jan Dimon Bendtsen}\ead{dimon@es.aau.dk}
\section*{Credit author statement}
\textbf{Simon Thorsteinsson:} Conceptualization, Methodology, Software, Formal analysis, Investigation, Data Curation, Writing - Original Draft, Visualization \textbf{Alex Kalaee:} Conceptualization, Writing - Review \& Editing \textbf{Pierre Vogler-Finck:} Conceptualization, Writing - Review \& Editing \textbf{Henrik Stærmose:} Resources, Conceptualization, Writing - Review \& Editing, Funding acquisition \textbf{Ivan Katic:} Conceptualization, Writing - Review \& Editing, Project administration, Funding acquisition  \textbf{Jan Bendtsen:} Conceptualization, Resources, Writing - Review \& Editing, Supervision, Funding acquisition.
\section*{Declaration of Competing Interest}
The authors have no competing interests to declare.

\begin{ack}                               % Place acknowledgements
The authors would like to express their sincere gratitude to the family for risking money and comfort by letting us carry out this experiment on their house. This work is funded the Danish Energy Agency through the EUDP project OPSYS 2.0 (Case num.: 64018-0581).  % here.
\end{ack}

%\end{multicols}


% \begin{figure}
% \begin{center}
% \includegraphics[height=4cm]{jcaesar.eps}    % The printed column  
% \caption{Gaius Julius Caesar, 100--44 B.C.}  % width is 8.4 cm.
% \label{fig1}                                 % Size the figures 
% \end{center}                                 % accordingly.
% \end{figure}
% OR

%\begin{figure}
%\begin{center}
%\epsfig{file=jcaesar,width=7cm}
%\caption{Gaius Julius Caesar, 100--44 B.C.}
%\label{fig1}
%\end{center}
%\end{figure}


% \begin{thm}
% The square of the length of the hypotenuse of a right triangle equals the sum of the squares 
% of the lengths of the other two sides.
% \end{thm}



%\bibliographystyle{plain}        % Include this if you use bibtex 
 % set the citation style to ``numbers''.

\bibliographystyle{plainnat}

\setcitestyle{numbers}
%\bibliography{MalariaBib,MST} % now the reference list has numbers
%\bibliography{common/mpc_dh1}
\bibliography{autosam,mpc_dh1,software}           % and a bib file to produce the
\appendix
\section{Appendix for Proofs}

\paragraph{Proof of Theorem \ref{thm:main}.}

\begin{proof}
\label{proof:main}
Our proof has two steps. In Step 1, we will show that SimCLR is equivalent to minimizing the cross entropy loss defined in Eqn.~(\ref{eqn:cross-entropy}). 
In Step 2, we will show  that minimizing the cross-entropy loss 
is equivalent to spectral clustering on $\bfpi$. 
Combining the two steps together, we have proved our theorem. 

\textbf{Step 1: } SimCLR is equivalent to minimizing the cross entropy loss.

The cross-entropy loss takes expectation over 
$\bfW_\bfX\sim \mathbb{P}(\cdot ; \bfpi)$, 
which means $\bfW_\bfX$ has exactly one non-zero entry in each row $i$. By Lemma~\ref{lem:multinomial}, we know every row $i$ of $\bfW_\bfX$ is independent of other rows. Moreover, 
$\bfW_{\bfX,i}\sim \mathcal{M}(1, \bfpi_i/\sum_j \bfpi_{i,j})=\mathcal{M}(1, \bfpi_i)$, because $\bfpi_i$ itself is a probability distribution.
Similarly, we know $\bfW_\bfZ$ also has the row-independent property by sampling over $\mathbb{P}(\cdot;\bfK_\bfZ)$.
Therefore, by Lemma~\ref{lem:cross_split}, we know Eqn.~(\ref{eqn:cross-entropy}) is equivalent to:
\[
 -\sum_{i=1}^n \mathbb{E}_{\bfW_{\bfX,i}}[\log \mathbb{P}(\bfW_{\bfZ,i}=\bfW_{\bfX,i};\bfK_\bfZ)],
\]

This expression takes expectation over $\bfW_{\bfX,i}$ for the given row $i$. Notice that 
$\bfW_{\bfX,i}$ has exactly one non-zero entry, which equals $1$ (same for $\bfW_{\bfZ,i}$). 
As a result
we expand the above expression to be:
\begin{equation}
 -\sum_{i=1}^n \sum_{j\neq i} \Pr(\bfW_{\bfX,i,j}=1)\log \Pr(\bfW_{\bfZ,i,j}=1).
\label{eqn:detailed-expansion}    
\end{equation}


By Lemma~\ref{lem:multinomial}, $\Pr(\bfW_{\bfZ,i,j}=1)=\bfK_{\bfZ,i,j}/\|\bfK_{\bfZ,i}\|_1$ for $j\neq i$. Recall that $\bfK_\bfZ=(k(\bfZ_i-\bfZ_j))_{(i,j)\in[n]^2}$, which means 
$\bfK_{\bfZ,i,j}/\|\bfK_{\bfZ,i}\|_1=\frac{\exp(-\|\bfZ_i-\bfZ_j\|^2/{2\tau})}{\sum_{k\neq i}
\exp(-\|\bfZ_i-\bfZ_k\|^2/{2\tau})
}$ for $j\neq i$, when $k$ is the Gaussian kernel with variance $\tau$. 

Notice that $\bfZ_i=f(\bfX_i)$, so we know
\begin{equation}
-\log \Pr(\bfW_{\bfZ,i,j}=1)=
-\log \frac{\exp(-\|f(\bfX_i)-f(\bfX_j)\|^2/{2\tau})}{\sum_{k\neq i}
\exp(-\|f(\bfX_i)-f(\bfX_k)\|^2/{2\tau}),
}
\label{eqn:infonce-equivalence}    
\end{equation}


The right hand side is exactly the InfoNCE loss defined in Eqn.~(\ref{eqn:infonce}).
Inserting Eqn.~(\ref{eqn:infonce-equivalence}) into Eqn.~(\ref{eqn:detailed-expansion}), we get the SimCLR algorithm, which first samples augmentation pairs $(i,j)$ with $\Pr(\bfW_{\bfX,i,j}=1)$ for each row $i$, and then optimize the InfoNCE loss. 

\textbf{Step 2: } minimizing the cross entropy loss 
is equivalent to spectral clustering on $\bfpi$.


By Lemma~\ref{lem:convert_to_spectral}, we may further convert the loss to 
\begin{equation}
\label{eqn:main-theorem-repul-attr}
\min_{\bfZ}
-\sum_{(i,j)\in [n]^2} \mathbf{P}_{i,j}
\log k (\bfZ_i-\bfZ_j)+\log \mathbf{R}(\bfZ).
\end{equation}
Since $k$ is the Gaussian kernel, this reduces to \[
\min_\bfZ \mathrm{tr}(\bfZ^\top \mathbf{L}(\bfpi) \bfZ)
+\log \mathbf{R}(\bfZ),
\]

where we use the fact that $\mathbb{E}_{\bfW_\bfX\sim \mathbb{P}(\cdot; \bfpi)}[\mathbf{L}(\bfW_\bfX)]
=\mathbf{L}(\bfpi)
$, because the Laplacian operator is linear and $
\mathbb{E}_{\bfW_\bfX\sim \mathbb{P}(\cdot; \bfpi)}(\bfW_\bfX)=\bfpi
$.
\end{proof}

\paragraph{Proof of Theorem \ref{thm:clip}.}
\begin{proof}
Since $\bfW_\bfX\sim \mathbb{P}(\cdot;\bfpi_{\mathbf{A}, \mathbf{B}})$, we know 
$\bfW_\bfX$ has exactly one non-zero entry in each row, denoting the pair that got sampled. 
A notable difference compared to the previous proof is we now have $n_\mathcal{A}+n_\mathcal{B}$ objects in our graph. CLIP deals with this by taking a mini-batch of size $2N$, 
such that $n_\mathcal{A}=n_\mathcal{B}=N$, and adding the $2N$ InfoNCE losses together. We label the objects in $\mathcal{A}$ as $[n_\mathcal{A}]$, and the objects in $\mathcal{B}$ as $\{n_\mathcal{A}+1, \cdots, n_\mathcal{A}+n_\mathcal{B}\}$. 

Notice that $\bfpi_{\mathbf{A}, \mathbf{B}}$ is a bipartite graph, so the edges of objects in $\mathcal{A}$ will only connect to object in $\mathcal{B}$ and vice versa. We can define the similarity matrix in $\cZ$ as $\bfK_\bfZ$, 
where $\bfK_\bfZ(i, j+n_\mathcal{A})=\bfK_\bfZ(j+n_\mathcal{A},i)= k(\bfZ_i-\bfZ_j)$ for $i\in [n_\mathcal{A}], j\in [n_\mathcal{B}]$, and otherwise we set $\bfK_\bfZ(i,j)=0$. 
The rest is same as the previous proof. 
\end{proof}

\paragraph{Proof of Theorem \ref{thm:exponential}.}

\begin{proof}
\label{proof:exponential}
Since the objective function consists of a linear term combined with an entropy regularization, which is a strongly concave function, the maximization problem is a convex optimization problem. Owing to the implicit constraints provided by the entropy function, the problem is equivalent to having only the equality constraint. We then introduce the Lagrangian multiplier $\lambda$ and obtain the following relaxed problem:

$$
\widetilde{E}(\boldsymbol{\alpha})=\psi_{1}-\sum_{i=1}^n \alpha_{i} \psi_{i}+\tau \sum_{i=1}^n \alpha_{i}\log \alpha_{i}+\lambda\left(\boldsymbol{\alpha}^{\top} \mathbf{1}_n-1\right).
$$

As the relaxed problem is unconstrained, taking the derivative with respect to $\alpha_{i}$ yields

$$
\frac{\partial \widetilde{E}(\boldsymbol{\alpha})}{\partial \alpha_{i}}=-\psi_{i}+\tau\left(\log \alpha_{i}+\alpha_{i} \frac{1}{\alpha_{i}}\right)+\lambda=0.
$$

Solving the above equation implies that $\alpha_{i}$ takes the form
$
\alpha_{i}=\exp \left(\frac{1}{\tau} \psi_{i}\right) \exp \left(\frac{-\lambda}{\tau}-1\right).
$ Since $\alpha_{i}$ lies on the probability simplex, the optimal $\alpha_{i}$ is explicitly given by
$
\alpha^{*}_{i}=\frac{\exp \left(\frac{1}{\tau} \psi_{i}\right)}{\sum_{i^{\prime}=1}^n \exp \left(\frac{1}{\tau} \psi_{i^{\prime}}\right)} .
$ Substituting the optimal point into the objective function, we obtain
$$
\begin{aligned}
E\left(\boldsymbol{\alpha}^*\right)  &=\psi_1-\sum_{i=1}^n \frac{\exp \left(\frac{1}{\tau} \psi_{i}\right)}{\sum_{i^{\prime}=1}^n \exp \left(\frac{1}{\tau} \psi_{i^{\prime}}\right)} \psi_{i}+\tau \sum_{i=1}^n \frac{\exp \left(\frac{1}{\tau} \psi_{i}\right)}{\sum_{i^{\prime}=1}^n \exp \left(\frac{1}{\tau} \psi_{i^{\prime}}\right)}\log \frac{\exp \left(\frac{1}{\tau} \psi_{i}\right)}{\sum_{i^{\prime}=1}^n \exp \left(\frac{1}{\tau} \psi_{i^{\prime}}\right)} \\
& =\psi_1 - \tau \log \left(\sum_{i=1}^n \exp \left(\frac{1}{\tau} \psi_{i}\right)\right).
\end{aligned}
$$
Thus, the Lagrangian dual function is given by
\begin{equation*}
-E\left(\boldsymbol{\alpha}^*\right)= -\tau \log \frac{\exp \left(\frac{1}{\tau} \psi_{1}\right)}{\sum_{i=1}^n \exp \left(\frac{1}{\tau} \psi_{i}\right)}.\qedhere
\end{equation*}
\end{proof}



\section{More on Experiments} \label{section: experiment_details}

\paragraph{CIFAR-10 and CIFAR-100} CIFAR-10 ~\citep{krizhevsky2009learning} and CIFAR-100 ~\citep{krizhevsky2009learning} are well-known classic image classification datasets. Both CIFAR-10 and CIFAR-100 contain a total of 60k $32 \times 32$ labeled images of different classes, with 50k for training and 10k for testing. CIFAR-10 is similar to CIFAR-100, except there are 10 different classes in CIFAR-10 and 100 classes in CIFAR-100.

\paragraph{TinyImageNet} TinyImageNet ~\citep{le2015tiny} is a subset of ImageNet ~\citep{deng2009imagenet}. There are 200 different object classes in TinyImageNet, with 500 training images, 50 validation images, and 50 test images for each class. All the images in TinyImageNet are colored and labeled with a size of $64 \times 64$.

\textbf{Pseudo-code.} Algorithm \ref{alg:Training Procedure} presents the pseudo-code for our empirical training procedure.

\begin{algorithm}[!htbp]
\caption{Training Procedure}
\label{alg:Training Procedure}
\begin{algorithmic}[1]
\REQUIRE trainable encoder network $f$, batch size $N$, augmentation strategy \textit{aug}, loss function $L$ with hyperparameters \textit{args}
\FOR {sampled minibatch ${x_i}_{i=1}^N$}
\FORALL{$i \in { 1, ..., N }$}
\STATE draw two augmentations $t_i = \textit{aug}\left(x_i\right) $, $t_i' = \textit{aug}\left(x_i\right) $
\STATE $z_i = f\left(t_i\right)$, $z_i' = f\left(t_i'\right)$
\ENDFOR
\STATE compute loss $\mathcal{L} = L(N, z, z', \textit{args})$
\STATE update encoder network $f$ to minimize $\mathcal{L}$
\ENDFOR
\STATE \textbf{Return} encoder network $f$
\end{algorithmic}
\end{algorithm}

We also provide the pseudo-code for our core loss function used in the training procedure in Algorithm \ref{alg:Core loss}. The pseudo-code is almost identical to SimCLR's loss function, with the exception of an extra parameter $\gamma$.

\begin{algorithm}[!htbp]
\caption{Core loss function $\mathcal{C}$}
\label{alg:Core loss}
\begin{algorithmic}[1]
\REQUIRE batch size $N$, two encoded minibatches $z_1, z_2$, $\gamma$, temperature $\tau$
\STATE $z = \textit{concat}\left(z_1, z_2\right)$
\FOR {$i \in {1, ..., 2N }, j \in {1, ..., 2N}$ }
\STATE $s_{i,j} = \Vert z_i - z_j \Vert_2^{\gamma}$
\ENDFOR
\STATE \textbf{define} $l(i, j)$ \textbf{as} $l(i, j) = - \log \frac{exp\left(s_{i,j}/\tau \right)}{\sum_{k=1}^{2N} \mathbf{1}{[k \ne i]} exp\left(s{i, j} / \tau \right)} $
\STATE \textbf{Return} $\frac{1}{2N} \sum_{k=1}^N\left[l(i, i+N) + l(i+N, i)\right]$
\end{algorithmic}
\end{algorithm}

Utilizing the core loss function $\mathcal{C}$, we can define all kernel loss functions used in our experiments in Table \ref{table: loss definition}. For all $z_i \in z$ with even dimensions $n$, we define $z_{L_i} = z_i\left[0:n/2\right]$ and $z_{R_i} = z_i\left[n/2:n\right]$.

\begin{table}[ht]
\centering
\begin{tabular}{{@{}l|l@{}}}
Kernel  &  Loss function \\ \midrule
Laplacian & $\mathcal{C}\left(N, z, z', \gamma=1, \tau\right)$\\ \midrule
Sum       & $\lambda * \mathcal{C}\left(N, z, z', \gamma=1, \tau_1\right) + (1-\lambda) * \mathcal{C}\left(N, z, z', \gamma=2, \tau_2\right)$  \\ \midrule
Concatenation Sum&$\lambda * \mathcal{C}\left(N, z_L, z'_L, \gamma=1, \tau_1\right) + (1-\lambda) * \mathcal{C}\left(N, z_R, z'_R, \gamma=2, \tau_2\right)$\\ \midrule
$\gamma = 0.5$ & $\mathcal{C}\left(N, z, z', \gamma=0.5, \tau\right)$          \\ 

\end{tabular}

\caption{Definition of kernel loss functions in our experiments}
\label {table: loss definition}
\end{table}

\textbf{Baselines.} We reproduce the SimCLR algorithm using PyTorch Lightning~\citep{PytorchLightning}.

\textbf{Encoder details.}
The encoder $f$ consists of a backbone network and a projection network. We employ ResNet50~\citep{ResNet} as the backbone and a 2-layer MLP (connected by a batch normalization~\citep{ioffe2015batch} layer and a ReLU \cite{nair2010rectified} layer) with hidden dimensions 2048 and output dimensions 128 (or 256 in the concatenation kernel case).

\textbf{Encoder hyperparameter tuning.}
For each encoder training case, we randomly sample 500 hyperparameter groups (sample details are shown in Table \ref{table: Hyperparameter sample}) and train these samples simultaneously using Ray Tune ~\citep{RayTune}, with the ASHA scheduler~\citep{li2018massively}. Ultimately, the hyperparameter group that maximizes the online validation accuracy (integrated in PyTorch Lightning) within 5000 validation steps is chosen for the given encoder training case.

\begin{table}[ht]
\centering

\begin{tabular}{@{}l|l|l@{}}
\midrule
Hyperparameter  & Sample Range & Sample Strategy \\ \midrule
start learning rate & $\left[10^{-2}, 10\right]$ & log uniform \\ \midrule
$\lambda$       & $\left[0, 1\right]$ & uniform \\ \midrule
$\tau$, $\tau_1$, $\tau_2$ & $\left[0, 1\right]$ & log uniform \\ \midrule
\end{tabular}

\caption{Hyperparameters sample strategy}
\label {table: Hyperparameter sample}
\end{table}

\textbf{Encoder training.} 
We train each encoder using the LARS optimizer~\citep{LARSOptimizer}, LambdaLR Scheduler in PyTorch, momentum 0.9, weight decay $10^{-6}$, batch size 256, and the aforementioned hyperparameters for 400 epochs on a single A-100 GPU.

\textbf{Image transformation.} The image transformation strategy, including augmentation, is identical to the default transformation strategy provided by PyTorch Lightning.

\textbf{Linear evaluation.}
The linear head is trained using the SGD optimizer with a cosine learning rate scheduler, batch size 64, and weight decay $10^{-6}$ for 100 epochs. The learning rate starts at $0.3$ and ends at $0$.

\textbf{Moco Experiments.} We also tested our method based on MoCo~\citep{he2019moco}. The results are summarized in Table \ref{tab:results-moco}. Here we choose ResNet18~\citep{ResNet} as the backbone and set a temperature of $0.1$ as default. For our simple sum kernel, we set $\lambda=0.8$. The results show that our method outperforms the original MoCo method.

\begin{table}[thb]
\centering
\caption{MoCo Experiment Results on CIFAR-10 and CIFAR-100.}
\label{tab:results-moco}
\resizebox{\textwidth}{!}{%
\begin{tabular}{@{}c|ccc|ccc@{}}
\toprule
\multirow{3}{*}{Method} & \multicolumn{3}{c|}{CIFAR-10} & \multicolumn{3}{c}{CIFAR-100} \\ \cmidrule(lr){2-4} \cmidrule(lr){5-7} 
                        & 200 epochs & 400 epochs    & 1000 epochs   & 200 epochs & 400 epochs & 1000 epochs         \\ \midrule
MoCo (repro.)         & $76.41 \pm 0.12$    & $80.01 \pm 0.15$          & $84.45 \pm 0.08$    & $\mathbf{47.02 \pm 0.11}$ & $52.50 \pm 0.07$ & $57.62 \pm 0.15$            \\
\midrule
Laplacian Kernel        & ${78.09 \pm 0.10}$    & $\mathbf{83.85 \pm 0.09}$          & $\mathbf{88.34 \pm 0.16}$    & $46.12 \pm 0.22$   & $53.44 \pm 0.17$ & $59.10 \pm 0.14$        \\
Simple Sum Kernel & $\mathbf{78.12 \pm 0.15}$   & $83.23 \pm 0.18$ & $87.50 \pm 0.20$ & $46.65 \pm 0.06$ & $\mathbf{53.62 \pm 0.19}$ & $\mathbf{59.83 \pm 0.12}$\\
\bottomrule
\end{tabular}
}
\end{table}



\section{More Experiments on Synthetic Data}


Consider a scenario with $n$ clusters, each containing $k$ vertices. Let the probability of vertices $u$ and $v$ from the same cluster belonging to $\bfpi$ be $p$. Conversely, for vertices $u$ and $v$ from different clusters, let the probability of belonging to $\pi$ be $q$. We generate the graph $\bfpi$ randomly, based on $p$ and $q$. We experiment with values of $k=100$ and $n=6$ for ease of visualization, embedding all points in a two-dimensional space. Each vertex's initial position originates from a normal distribution. In each iteration, we sample a subgraph of $\bfpi$ uniformly, ensuring each vertex has an out-degree of $1$. We then optimize the corresponding vectors using InfoNCE loss with an SGD optimizer and iterate until convergence. Our experimental setup consists of an SGD learning rate of $1$, an InfoNCE loss temperature of $0.5$, and a batch size of $50$. We evaluate two scenarios with different $p$ and $q$ values: $p=1$, $q=0$, and $p=0.75$, $q=0.2$. The results of these experiments are visualized in Figure \ref{fig:vis-spectral-cluster}. The obtained embeddings exhibit the hallmark pattern of spectral clustering of graph $\bfpi$.

\begin{figure}[!tb]
\centering
\subfigure{
\includegraphics[width=1\textwidth]{Figures/cluster_pi.png}
\label{fig:vis-cluster}
}
\subfigure{
\includegraphics[width=1\textwidth]{Figures/noised_cluster_pi.png}
\label{fig:vis-noised-cluster}
}
\caption{Visualizations of the optimization process using InfoNCE Loss on the vectors corresponding to $\bfpi$. Points of identical color belong to the same cluster within $\bfpi$. To showcase the internal structure of $\bfpi$, we randomly select 10 vertices from each cluster to display the edge distribution of $\bfpi$.}
\label{fig:vis-spectral-cluster}
\end{figure}





%\bibliography{mpc_dh1}  
                                 % bibliography (preferred). The
                                 % correct style is generated by
                                 % Elsevier at the time of printing.

%\begin{thebibliography}{99}     % Otherwise use the  
                                 % thebibliography environment.
                                 % Insert the full references here.
                                 % See a recent issue of Automatica 
                                 % for the style.
%  \bibitem[Heritage, 1992]{Heritage:92}
%     (1992) {\it The American Heritage. 
%     Dictionary of the American Language.}
%     Houghton Mifflin Company.
%  \bibitem[Able, 1956]{Abl:56}
%     B.~C.~Able (1956). Nucleic acid content of macroscope. 
%     {\it Nature 2}, 7--9. 
%  \bibitem[Able {\em et al.}, 1954]{AbTaRu:54}   
%     B.~C. Able, R.~A. Tagg, and M.~Rush (1954).
%     Enzyme-catalyzed cellular transanimations.
%     In A.~F.~Round, editor, 
%     {\it Advances in Enzymology Vol. 2} (125--247). 
%     New York, Academic Press.
%  \bibitem[R.~Keohane, 1958]{Keo:58}
%     R.~Keohane (1958).
%     {\it Power and Interdependence: 
%     World Politics in Transition.}
%     Boston, Little, Brown \& Co.
%  \bibitem[Powers, 1985]{Pow:85}
%     T.~Powers (1985).
%     Is there a way out?
%     {\it Harpers, June 1985}, 35--47.

%\end{thebibliography}

%\appendix
%\section{A summary of Latin grammar}    % Each appendix must have a short title.
%\section{Some Latin vocabulary}         % Sections and subsections are supported  
                                        % in the appendices.

\end{document}