%\subsection{Daily heating cost and mean indoor temperature}
\subsection{Heating costs and energy consumption}
\label{sec_res_cost}
This section is dedicated to the investigation of the savings potential.
%, but before presenting cost results a look at Figure \ref{fig_avg_temp_1} shows the reason that three different comfort levels have been applied during the test period. To reduce variables in the cost analysis 
%Since the indoor temperature impacts the consumption and thereby costs it 
%Remember, that all benchmark costs provided in this section are estimates since they virtual and calculated based on the evaluation model described in Section \ref{sec_validation}. 
The section consists of Table \ref{table_cmp_cost}, which sums up the savings accumulated during the test periods and \figref{fig_cost_compare} presents costs with respect to individual days. Note that a row in each table has been dedicated the combined analysis of comfort level 3 and 4.
%\onecolumn
\newcommand{\tablePercCostCmp}{0.145}
\begin{table}[H]
\centering
\caption{Shows the accumulated economical savings estimate for space heating over the test periods.}
\begin{tabular}{p{0.125\linewidth} p{0.19\linewidth} p{\tablePercCostCmp\columnwidth} 
p{\tablePercCostCmp\columnwidth} p{\tablePercCostCmp\columnwidth}}
\toprule[1.5pt]
Comfort level & Average benchmark cost [\currency]  & Exp. cost [\currency] & Reduction [\currency] & \SavingRateText\ [\%]\\ %[0.5ex] 
\midrule[1.5pt]
% 1097 \currency  & 948 \currency  & 149 \currency & 13.6\%\\ 
%%%%%%% COST  %%%%%%%%%%%%%%%%%%%%%%%%%%%%%%%%%%%%%%%%%%%%%%%%%%%%%%%%%%%%%%%%%%%%%%%%%%%%%%%%%%%%%%%%%%%
% 1 ({\color{ref1}\FilledBigTriangleUp}) & 10.92 &  7.33 & 3.59  &32.8 \\ 
% 2 ({\color{ref2}\FilledBigTriangleUp})& 49.84 & 35.34 & 14.50  &29.1 \\ 
% 3 ({\color{ref3}\FilledBigTriangleUp})& 126.42 & 123.49 & 2.93  &2.3 \\ 
% 4 ({\color{ref4}\FilledBigTriangleUp}) & 42.65 & 35.23 & 7.42  &17.4 \\ 
% \midrule[1.5pt]
% Total & 229.83  & 201.39  & 28.43  & 12.4 \\
1 ({\color{ref1}\FilledBigTriangleUp}) & 10.92 &  7.33 & 3.59  &32.8 \\ 
2 ({\color{ref2}\FilledBigTriangleUp})& 49.84 & 35.34 & 14.50  &29.1 \\ 
3 ({\color{ref3}\FilledBigTriangleUp}) & 126.42 & 123.49 & 2.93  &2.3 \\ 
4 ({\color{ref4}\FilledBigTriangleUp}) & 42.65 & 35.23 & 7.42  &17.4 \\ 
\midrule[1.5pt]
3 and 4 & 169.07 & 158.72 & 10.35  &6.1 \\
\midrule[1.5pt]
All & 229.83  & 201.39  & 28.43  & 12.4 \\
%%%%%%% COST %%%%%%%%%%%%%%%%%%%%%%%%%%%%%%%%%%%%%%%%%%%%%%%%%%%%%%%%%%%%%%%%%%%%%%%%%%%%%%%%%%%%%%%%%%%
\bottomrule[1.5pt]
\end{tabular}
\label{table_cmp_cost}
\end{table}
Table \ref{table_cmp_cost} shows a significant \percSaving\ percentage point saving on the heating bill, but looking at the absolute savings of \currency\totalSavings, derived from \numTestDays\ operation days, the results are more modest. Further, it can be seen that the comfort level has a significant impact on savings. \figref{fig_savings_development} shows the development of the estimated \savingRateText\ for each comfort level. The large variance in daily savings means that the long-term expected \savingRateText\ has not settled after 10 days.
\begin{figure}[H]
	\centering
	\includegraphics[width=1\columnwidth]{figs/savings_development.pdf}
	\caption{Day-to-day development of the total accumulated \savingRateText\ for each comfort level.}
	\label{fig_savings_development}
\end{figure}
\figref{fig_cost_compare} contains the cost results broken down into individual test days (One test day per column), which are analysed with respect to average ambient temperature and sun intensity. Dots are similar benchmark days derived according to description of section \ref{sec_validation} and the black lines in each column represents the average cost of similar benchmark days. The results reveal three main ambient temperature regions: the warm (6 to 13 \si{\degreeCelsius}), the medium (0 to 6 \si{\degreeCelsius}) and the cold (-5 to 0 \si{\degreeCelsius}). In the warm region, the heating demand is so low that percentage losses or gains amounts to very small differences in savings or losses. The medium region shows the highest potential for savings. %It is in this region where the controller really can take advantage of the flexibility offered by the house. 
The results from cold, sunny days are difficult to assess due to a sparse amount of similar days present in the benchmark data, but the immediate results point at consistent losses. Further, it can be seen that sunny days (reddish dots and crosses) reduce costs since they drop lower than their more cloudy counterparts. This is of course related to overheating events, which might be uncomfortable. The plot also shows that the costs increase as temperature decreases.
%Figure \ref{fig_cost_compare} presents the space heating cost broken down into individual test days, with each day having its own dedicated column.
\begin{sidewaysfigure*}
%\begin{figure*}[t]
	\centering
	\includegraphics[width=\textwidth]{figs/cost_compare.pdf}
	\caption{\figUpper Daily space heating cost during the experiment period (crosses) compared with the virtual costs, calculated based on Section \ref{sec_validation}, from similar days (dots). Each experiment day is assigned one column and are sorted according to daily mean temperature. The black horizontal line (-) marks the mean benchmark cost. \figLower Savings between mean benchmark cost and experiment cost. The colors of the triangles refer to the comfort level at given experiment day.}
%\end{figure*}
    \label{fig_cost_compare}
\end{sidewaysfigure*}
%The efficiency loss of operating the heat pump by manipulating the ambient temperature signal superseeds the potential gains from utilizing prices differences.

Table \ref{table_cmp_electricity} shows the electricity consumption. It is common that studies experience an increase in primary energy consumption when applying price responsive control (15.8\% more electricity in \cite{hu_price-responsive_2019}, 10.3\% more heat in \cite{amato_dual-zone_2023}), as is the case for comfort level 1 and 4, albeit the values observed here are significantly higher. Comfort level 2 and 4 show a reduction, which is likely to be connected to the lower indoor temperature in comfort level 1 and a sequence of sunny days which the MPC controller could capitalize on.
\renewcommand{\arraystretch}{1.5}
%%%%%%%%%%%%%%%%%%%%%%%%%%%%%%%%%%%%%%%%%%%%%%%%%%%%%
% ELECTRICITY
%%%%%%%%%%%%%%%%%%%%%%%%%%%%%%%%%%%%%%%%%%%%%%%%%%%%%
\newcommand{\tablePercElectricity}{0.23}
\begin{table}[H]
\centering
\caption{Accumulated electricity consumption from the experiment and benchmark data}
\begin{tabular}{p{0.13\linewidth} p{\tablePercElectricity\linewidth} p{\tablePercElectricity\columnwidth} 
p{0.20\columnwidth}}
\toprule[1.5pt]
Comfort level & Accumulated average electricity (Benchmark) [\si{kWh}] & Accumulated Electricity
(Experiment) [\si{kWh}]   & Percentage increase [\%]\\
\midrule[1.5pt]
%%%%%%% ELEC  %%%%%%%%%%%%%%%%%%%%%%%%%%%%%%%%%%%%%%%%%%%%%%%%%%%%%%%%%%%%%%%%%%%%%%%%%%%%%%%%%%%%%%%%%%%
% 1 ({\color{ref1}\FilledBigTriangleUp}) & 45.9 &  41.4  &11.0 \\ 
% 2 ({\color{ref2}\FilledBigTriangleUp})& 96.3 &  110.5   &-12.8 \\ 
% 3 ({\color{ref3}\FilledBigTriangleUp})& 492.9 &  398.9   &23.6 \\ 
% 4 ({\color{ref4}\FilledBigTriangleUp}) & 178.0 &  191.8   &-7.2 \\ 
% \midrule[1.5pt]
% Total & 813  & 743  & 9.5 \\
1 ({\color{ref1}\FilledBigTriangleUp})  &  41.4 & 45.9 &11.0 \\ 
2 ({\color{ref2}\FilledBigTriangleUp}) &  110.5 & 96.3  &-12.8 \\ 
3 ({\color{ref3}\FilledBigTriangleUp}) &  398.9 & 492.9  &23.6 \\ 
4 ({\color{ref4}\FilledBigTriangleUp})  &  191.8 & 178.0  &-7.2 \\ 
\midrule[1.5pt]
3 and 4 & 590.72 & 670.96  & 13.6 \\ 
\midrule[1.5pt]
All   & 743 & 813 & 9.5 \\
%%%%%%%%%%%%%%%%%%%%%%%%%%%%%%%%%%%%%%%%%%%%%%%%%%%%%%%%%%%%%%%%%%%%%%%%%%%%%%%%%%%%%%%%%%%%%%%%%%%%%%%%%
\bottomrule[1.5pt]
\end{tabular}
\label{table_cmp_electricity}
\end{table}
As with electricity, the heat production (Table \ref{table_cmp_energy})  has increased, but percentage-wise, not as much. This can be explained by the lower \cop\ causing less heat to be produced for the electricity.
%%%%%%%%%%%%%%%%%%%%%%%%%%%%%%%%%%%%%%%%%%%%%%%%%%%%%
% HEAT
%%%%%%%%%%%%%%%%%%%%%%%%%%%%%%%%%%%%%%%%%%%%%%%%%%%%%
\newcommand{\tablePercEnergy}{0.23}
\begin{table}[H]
\centering
\caption{Accumulated heat produced by the \hp.}
\begin{tabular}{p{0.13\columnwidth} p{\tablePercEnergy\columnwidth} p{\tablePercEnergy\columnwidth} 
p{0.2\columnwidth}}
\toprule[1.5pt]
Comfort level & Accumulated Average heat (Benchmark) [\si{kWh}] & Accumulated heat (Experiment) [\si{kWh}]   & Percentage Increase [\%]\\
\midrule[1.5pt]
%%%%% HEAT %%%%%%%%%%%%%%%%%%%%%%%%%%%%%%%%%%%%%%%%%%%%%%%%%%%%%%%%%%%%%%%%%%%%%%%%%%%%%%%%%%%%%%%%%%%%%%
% 1 ({\color{ref1}\FilledBigTriangleUp}) & 218.3 &  192.6  &13.4 \\ 
% 2 ({\color{ref2}\FilledBigTriangleUp})& 385.7 &  455.1   &-15.2 \\ 
% 3 ({\color{ref3}\FilledBigTriangleUp})& 1919.0 &  1667.2   &15.1 \\ 
% 4 ({\color{ref4}\FilledBigTriangleUp}) & 707.5 &  799.9   &-11.5 \\ 
% \midrule[1.5pt]
% 3 and 4 & 2467.13 & 2626.56  & 6.5 \\
% \midrule[1.5pt]
% Total & 3231  & 3115  & 3.7 \\
1 ({\color{ref1}\FilledBigTriangleUp})  &  192.6 & 218.3 &13.4 \\ 
2 ({\color{ref2}\FilledBigTriangleUp}) &  455.1 & 385.7  &-15.2 \\ 
3 ({\color{ref3}\FilledBigTriangleUp}) &  1667.2 & 1919.0  &15.1 \\ 
4 ({\color{ref4}\FilledBigTriangleUp})  &  799.9 & 707.5  &-11.5 \\ 
\midrule[1.5pt]
3 and 4 & 2467.13 & 2626.56  & 6.5 \\ 
\midrule[1.5pt]
All  & 3115  & 3231 & 3.7 \\
% 1 ({\color{ref1}\FilledBigTriangleUp}) & 218.3 &  192.6  &13.4 \\ 
% 2 ({\color{ref2}\FilledBigTriangleUp})& 385.7 &  455.1   &-15.2 \\ 
% 3 ({\color{ref3}\FilledBigTriangleUp})& 1919.0 &  1667.2   &15.1 \\ 
% 4 ({\color{ref4}\FilledBigTriangleUp}) & 707.5 &  799.9   &-11.5 \\ 
% \midrule[1.5pt]
% Total & 3231  & 3115  & 3.7 \\
%%%%%%%%%%%%%%%%%%%%%%%%%%%%%%%%%%%%%%%%%%%%%%%%%%%%%%%%%%%%%%%%%%%%%%%%%%%%%%%%%%%%%%%%%%%%%%%%%%%%%%%%%
\bottomrule[1.5pt]
\end{tabular}
\label{table_cmp_energy}
\end{table}
% \begin{sidewaysfigure*}
% %\begin{figure*}[t]
% 	\centering
%     \label{fig_cost_compare}
% 	\includegraphics[width=\textwidth]{figs/cost_compare.pdf}
% 	\caption{\figUpper Shows the daily space heating cost during the experiment period (crosses) compared with the virtual costs, calculated based on Section \ref{sec_validation}, from similar days (dots). Each experiment day is assigned one column and are sorted according to daily mean temperature. The black horizontal line (-) marks the mean comparable cost. \figLower Shows the savings between mean compare cost and experiment cost. The colors of the triangles refers the average room temperature reference scheme and comfort cost applied to given experiment day presented in Fig. \ref{fig_comfort_cost}.}
% %\end{figure*}
% \end{sidewaysfigure*}
\figref{fig_cost_compare} gives a sense of monetary savings potential, but it hides some important factors leading to significant savings larger than \euro 1. \figref{fig_savings_analysis} explores these factors. 
%\twocolumn
\begin{figure}[H]
	\centering
	\includegraphics[width=1\columnwidth]{figs/savings_analysis.pdf}
	\caption{\figUpper Average daytime price as a function of the average temperature. \figLower Daily savings formulated in percentage as a function of the day over night price ratio. Colors indicate the daily average ambient temperature and the contours the density.}
	\label{fig_savings_analysis}
\end{figure}
%The upper figure, showing the averaged daytime price plotted against the daily average ambient temperature, clearly shows that the price for electricity---during this heating season---increases as temperatures lower.
The upper figure clearly shows an inverse correlation between daily average ambient temperature and the average daytime electricity price during this heating season. This partly explains the larger savings potential between 1 and \SI{5}{\degreeCelsius} seen in \figref{fig_cost_compare}. The price is in the high end while the house still maintains flexibility. The next influential factor is volatility in prices which is explored in the lower graph of \figref{fig_savings_analysis}. Here the daily relative savings are plotted against the day over night average price ratio. Nighttime price is defined as the average price between 00:00 and 06:00, and daytime is given as the average over the remaining period. Although the dots are more scattered here, some important trends can be seen. First, the price ratio decreases with colder temperatures. Second, most days with a price ratio above three resulted in savings and, third, cold days gave significant loses.