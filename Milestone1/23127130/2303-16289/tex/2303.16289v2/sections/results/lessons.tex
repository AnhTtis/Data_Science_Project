%\subsection{Unsolved control challenges and suggested solutions}
%\label{control_challenge}
%Having a controller running in a real setting, it is often that problems and challenges are discovered which are overlooked in simulation. This section briefly touches upon some practical lessons learned by the authors.
%This section presents some practical challenges and problems, which are typically neglected in simulation studies.

%During the test campaign several issues have been discovered with some largely solved but others need further attention. This section provides an overview the problems\\\\
\begin{itemize}
    \item A side effect of using the compressor block function is that the \hp\ attempts to heat the \DHW\ using the electric heating rod, which has a power output of \SI{10}{kW}. This is far from ideal, since just a few minutes in this state is costly. \textbf{Suggested solution}: block the compressor for space heating only.
    \item When the \hp\ defrosts, the measured heat flow reverses. Any controller regulating the heat output needs to be able to detect and handle such a situation. \textbf{Suggested solution}: put  the control in standby mode.
    \item It is not possible to start the heat pump on demand, the only option is to release the compressor brake and wait. The waiting time is observed to be between 60 and 120 min. \textbf{Suggested solution}: adapt the block release for best start-up timing or introduce/use open HP controller standards
    \item In certain situations the heat pump shuts down before it should. It is assumed that a combination of high ambient temperature and low flow caused the internal controllers to shut it down. \textbf{Suggested solution}: use data to figure out what events cause a shut down.
    \item A low pass filter and other unknown internal states make control through ambient temperature overwrite particularly challenging. \textbf{Suggested solution}: use a heat pump with reference control for heat.
\end{itemize}
%A side effect of using the compressor block function is that the \hp\ attempts to heat the \DHW\ using the electric heating rod which have a power output of 10 kW. This is far from ideal, since just a few minutes in this state is costly. \textbf{Suggested solution}: Block the compressor for space heating only.

%When the \hp\ defrosts the measured heat flow reverse. Any controller regulating the heat output needs to be able to detect and handle such a situation. \textbf{Suggested solution}: Pause control.

%It is not possible to start the heat pump on the demand, the only option is to release the compressor brake and wait. The waiting time is observed to be between 60 and 120 min. \textbf{Suggested solution}: Adapt the block release for best start-up timing.


%In certain situations the heat pump shut down before this being requested. It is assumed that a combination of high ambient temperature and low flow caused the internal controllers to shut it down. \textbf{Suggested solution}: Use data to figure out what events causes a shut down.

%The input-manipulation is performed through a filter with a large time-constant making control and other internal states are unknown making control unnecessarily difficult.