\subsection{Heat pump efficiency model}
\label{sec_heat_pump_efficiency}
%This section deals with question four presented in the introduction: \questionFour\ 
As the \hp\ efficiency model informs the \MPC\ on the trade-off between heat boosting and efficiency loss, it is highly important that it is accurate. Figure \ref{fig_cop_investigation_2} shows that the general \cop\ fell when the \hp\ was operated according to the new controller, leading to inaccurate estimates.
\begin{figure}[H]
	\centering
	\includegraphics[width=\columnwidth]{figs/cop_model.pdf}
	\caption{The two parameter fits for the \hp\ efficiency model during the experiment. The dashed lines describe the \cop\ as a function of electricity consumption at varies fixed ambient temperatures. The solid lines show the updated fit, which matches the data (scatter dots), obtained during the experiment, better than the original. The coloring indicates the ambient temperature level with blue signaling cold and red warm.}
	\label{fig_cop_investigation_2}
\end{figure}
The dashed lines represents the original fit, which is based on data from the benchmark controller at various fixed ambient temperatures, the scattered dots represents data points obtained in the test period and the solid lines are from an updated parameter fit. The original fit has an $R^2$ value of \rsquaredInit, but since it performed poorly during operation---often overestimating the efficiency---the fit was updated (solid lines), which resulted in lower predicted efficiencies. The main take away is that it is necessary to update the model repeatedly when the operational style changes, otherwise severe miscalculations are introduced. \figref{fig_cop_meas_est} present the results of the updated fit.
\begin{figure}[H]
	\centering
	\includegraphics[width=\columnwidth]{figs/cop_meas_est_ransac.pdf}
	\caption{The red contour lines shows the distribution of the new data calculated using the pre-experiment fit.}
	\label{fig_cop_meas_est}
\end{figure}
%which was based on data from the benchmark controller and used in the beginning of the experiment.  