\section{Interpretation of the results}
\label{sec_result_interpret}
In this section the authors provide their interpretation of the data and results presented in the former section. Starting with Table \ref{table_cmp_cost}, where comfort levels 1 and 2 (\comfLvlTriangleOne\ and \comfLvlTriangleTwo, respectively) show a clear percentage-vise savings potential. At comfort level 1, the indoor temperature was uncomfortably low, so this result is ignored. Test period 2 (\comfLvlTriangleTwo) is more interesting since the residents experienced a comfortable indoor climate while saving on heating costs. This raises the question: \textit{Did the price responsiveness cause the economical
savings?} The answer is unknown since the lower average indoor temperature, and thereby lower heat demand, could have been the main reason. The main take-away from comfort level 2 is that even a \NZEB\  can gain by lowering indoor temperature. Test period three (\comfLvlTriangleThree) was executed with an average indoor temperature of \SI{0.2}{\degC} higher than the one of the benchmark data, meaning that the \comfThreePercSave\% savings are likely to be contributed to the controller.

Having established that overall savings are possible under the current Danish price scheme, the next part investigates situations which are particularly favorable or unfavorable for the controller. 
%This is unfortunate, but it is also interesting for several reasons. Despite the \MPC\ breaking even with respect costs there are situations or events where it is able to generate savings consistently and situations with consistent losses.
Before reading on, keep in mind that \textit{large savings can only originate from situations with large potential costs}. For the analysis Figs.~\ref{fig_avg_temp_1}, \ref{fig_cost_compare}, \ref{fig_savings_analysis}, \ref{fig_cop_investigation_2}, Table \ref{table_cmp_electricity} and \ref{table_cmp_energy} are used. \figref{fig_cost_compare} reveals that the largest share of consistent savings are generated between 0.7 and \SI{4.0}{\degreeCelsius}. The large loss seen at \SI{1.1}{\degreeCelsius} is the transition day between test period 2 and 3 where extra energy was needed to lift the average indoor temperature. The lower ambient temperature increases the demand for heat which increases the cost. However, as this is true for all buildings in the region not only the consumption dependent costs are driven up, so are the electricity prices. This is visible in the upper graph of \figref{fig_savings_analysis}, where the average daytime price is inversely correlated with the temperature. The result is that heat demand and price amplifies each other which increase the daily cost dramatically when the ambient temperature is below \SI{5}{\degreeCelsius}.

Having established the factors driving up costs, we turn the attention to daily price variation which also impacts the potential for cost savings, see the lower graph in \figref{fig_savings_analysis}. Although, the results are more scattered than in the upper graph, three trends can be seen. First, the day/night price-ratio is more likely to be higher at high ambient temperatures. Second, at ratios above three, the controller is likely to save money, albeit these are mostly warm low cost days. Third, the most interesting trend is the range 0 to \SI{5}{\degreeCelsius}, where the ratio often was above 2, securing significant percentage-vise savings.

At this point the price conditions for savings are established. Hence, we return focus to test periods 3 and 4 (\comfLvlTriangleThree\ and \comfLvlTriangleFour, respectively) and ask: \textit{Is the potential for savings larger than presented here?} The controller responded to the price signal by changing the daily heating pattern significantly, as seen in \figref{fig_production}. Several things impact savings: model errors, forecasts errors, lack of controller robustness and more. This said, the loss of \hp\ efficiency, mentioned in Section \ref{sec_heat_pump_efficiency}, stands out as a major plausible limiter. % Figure \ref{fig_cop_investigation_2} shows the initial fit (dotted lines) against the new fit obtained and implemented a few weeks after starting the controller.The controller shows a noticeable fall in efficiency.This means that the controller not only loses efficiency by operating the compressor at higher loads it also loses efficiency by controlling it differently than default mode, which is a large disadvantage.
The efficiency loss originates not only from the higher loads but also due to the dynamic operations. The loss from dynamic operations puts load shifting at a disadvantage. When the supervisory controller calculates the heating plan it also considers the continuous approach featured by the benchmark controller but discards it as inefficient compared to the night heating approach. This happens because the controller relies on the \hp\ efficiency model, which does not inform that the default controller---the one from the manufacturer---can operate the \hp\ more efficiently. This can be used as a critique of the presently implemented heat controller, yet, it can also be posed as a question to why the manufacturers of domestic \hp's do not let them be controlled according to a heat reference as an alternative to the ambient temperature heating curve.

A weakness has been noticed in the \MPC s reliance on forecasts. The procedure has issues dealing with sunny days, even though the predictive nature should ensure superiority of the \MPC. \figref{fig_cost_compare} clearly reveals that days with significant loses tend to have high sun intensity. 
%In Figure \ref{fig_cost_compare} it is clear that the \MPC\ consistently generate loses on sunny days. 
We expect this to be due a combination of several factors which coalesce with unfortunate outcomes. The low electricity prices invite the controller to boost heating intensely between 00:00 and 06:00 to avoid electricity consumption during more expensive day hours. If the model and forecasts were perfect the heat would be boosted accurately. However, in practice an overheating event is likely to occur if a thick cloud cover is wrongfully predicted, and intense sunshine happens instead. The cloud cover data from the weather service has several times been unreliable even at time-of-use. This effect can be mitigated by correcting the forecast with live \pv\ data. Further, a robust control approach which restrains night boosting slightly should be applied.