\subsection{Retrofit architecture}
The retrofit architecture, which is built and implemented by \neogrid, is seen in \figref{fig_hardware}. 
\begin{figure}
	\centering
	\includegraphics[width=0.9\columnwidth]{figs/hardware_architecture_2.pdf}
	\caption{Shows the overview of the hardware and communication protocols. Blue color is for preinstalled hardware, green for installed sensors, yellow is the off-site infrastructure and purple are data services.}
 	\label{fig_hardware}
\end{figure}
The infrastructure consists of an onsite part and a \backend\ with the control box acting as gateway between them. The \backend\ is responsible for refining, organizing, downloading data from weather and price services, and storing data, which is used for analysis and model fitting. The control box is responsible for providing control signals and collecting measurements from all units. In this case, it means to provide the artificial ambient temperature overwrite, via a digital to analog converter (DAC) and blocking the compressor using a relay. The datahub (Eloverblik) is an online platform, provided by the danish publicly owned company Energinet, where electricity-customers can get an overview of their consumption or share data with third-party. We use the BACnet and Modbus protocols to communicate with the floor heating controller for collecting room temperatures and other data from the floor heating system, and sending set-points to the valves.
