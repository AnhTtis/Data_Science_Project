\section{Shift in continuum LLs}

It is known that for $C$ filled Landau levels in the continuum, the shift $\mathscr{S}$ can be written as $\mathscr{S} = \frac{C^2}{2}$, where $C$ is the Chern number (Wen-Zee, 1992). In particular, the $n$th LL contributes $n+1/2$ to the shift, where $n = 0,1,2,\dots$. To study what happens at the boundary between two systems with the same $C$ but different $\mathscr{S}$, we can use the following idea. First consider a system with 2 filled LLs. This has total Chern number 2 and total shift $1/2 + 3/2 = 2$. Now consider a second system which has 2 copies of the lowest LL. This system also has Chern number 2, but total shift $1/2 + 1/2 = 1$. 

We will now define a radial configuration in which the first system is located at $r>>r_0$ and the second is at $r<<r_0$. The two systems share a boundary around $r=r_0$. Then there will be two counterpropagating gapless boundary states in this region, one from each system. The question is whether these states can be gapped out by symmetry-preserving backscattering terms. We will numerically simulate this and show that they cannot.

\begin{figure}[t]
    \centering
    \includegraphics[width=0.45\textwidth]{LL_1.png}
    \includegraphics[width=0.45\textwidth]{LL_2.png}
    \caption{Energy as a function of $m-n$ (which is a monotonic function of $r^2$) for two systems with radial potentials described by Eq.~\eqref{eq:radpot}. The zero energy crossing points in the top and bottom figures are $m-n = 57,62$ respectively; $r_0$ corresponds to taking $m-n \sim 60$. When $r  << r_0$, the full system resembles a stack of two $C=1$ IQH states. When $r>> r_0$, the full system resembles a stack of a trivial system with a $C=2$ IQH state. The value of the shift changes at the boundary, because the two zero-energy modes have different values of $m-n$ and cannot be scattered into each other.}
    \label{fig:LLgraph}
\end{figure}

To model this configuration, we use the potentials
\begin{equation}\label{eq:radpot}
    V_i(\hat{r}^2) = - \mu_i + K_i \tanh\left( \frac{\hat{r}^2 - r_0^2}{l^2}\right)
\end{equation}
where $i=1,2$. The eigenvalues of $H_0 +V_1$ and $H_0+V_2$ are plotted as functions of $m-n$ (which increases with the radius $r$) in Fig. \ref{fig:LLgraph}. For the first system, $\mu_1 = -1.5, K_1 <0$. That is, when $r>>r_0$, there are two LLs below zero energy, and when $r<<r_0$, only the LLL is below zero energy. For the second system, $\mu_2 = -0.5, K_2 >0$. That is, when $r>>r_0$, there are no LLs below zero energy, and when $r<<r_0$, only the LLL is below zero energy. 

We can see that when $r>>r_0$, the total Chern number equals 2, while the total shift equals $1/2 + 3/2 = 2$. When $r<<r_0$, the total Chern number equals 2, while the total shift equals $1/2 + 1/2 = 1$. This indeed realizes our model configuration.  

The crucial point is that the two zero energy states have different values of the conserved quantum number $m-n$. Since radial perturbations must conserve $m-n$, they cannot scatter the zero-energy states into each other. This can only change if we introduce terms in the potential that are not radial, i.e. terms that break the $U(1)$ spatial rotation symmetry.

% In the first system, the choice of $\mu_1$ implies that we should take $n=1$Thus each system contributes a state at almost zero energy, but these two states have different values of $m$. These states cannot be scattered into each other by any symmetric (radial) perturbations, since $\hat{r}^2$ conserves $m$. 

% Suppose we only consider the diagonal terms in $V$. The zero energy state $\ket{n,m^*}$ in each system satisfies
% \begin{equation}
%     n + \frac{1}{2} - \mu + V_0 \tanh \left( \frac{2(2n + m^* -1) - r_0^2}{l^2}\right) = 0.
% \end{equation}
% The value of $m^*$ depends on $n$. In our case, we have chosen that $n + \frac{1}{2} - \mu=0$ for the first system when $n=0$, and for the second system when $n=1$. By setting the argument of the $\tanh$ term to zero and simplifying, we find that $m^*(n=1) = m^*(n=0)-2$. We can confirm this numerically (we choose the states with energy closest to zero; the energy is $\sim V_0/l^2$). If we include the off-diagonal terms in $V$, $n$ is no longer a good quantum number for the zero energy state, but we find numerically that $\Delta m^* = 1$ for the two systems. 