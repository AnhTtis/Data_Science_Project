In this brief appendix, we explain some motivation for the results of this article from the perspective of Coulomb branches of $4$d $\cN=2$ and $5$d $\cN=1$ gauge theories with a generic choice of complex structure. The goal here is not to be precise, but instead explain some motivation for the ideas in this article. While reading this appendix, the reader should keep in mind that I know very little physics!
In \cite{bfn-ii, nakajima-coulomb} (see also \cite{nakajima-intro}), it is argued that the Coulomb branch of $3$d $\cN=4$ pure gauge theory on $\RR^3$ can be modeled by the algebraic symplectic variety $\cM_C := \spec \H^{G_c}_\ast(\Gr_G; \cc)$ over $\cc$. This is in turn isomorphic by \cite[Theorem 3]{bf-derived-satake} (reproved here as \cref{cor: loop-rot Gr and biWhit}) to the phase space of the Toda lattice for $\ld{G}$, as well as (by \cite[Theorem A.1]{bfn-ii}) to the moduli space of solutions of Nahm's equations on $[-1,1]$ for a compact form of $\ld{G}$ with an appropriate boundary condition.
The \textit{quantized} Coulomb branch of $3$d $\cN=4$ pure gauge theory on $\RR^3$ is then modeled by $\cA_\epsilon := \H^{G_c\times S^1_\rot}_\ast(\Gr_G; \cc)$. Note that $\cA_\epsilon$ is isomorphic to the algebra of operators of the quantized Toda lattice for $\ld{G}$.

The physical reason for the definition of $\cA_\epsilon$ is the ``$\Omega$-background'' (introduced in \cite{nek-shat}); we refer the reader to \cite{ben-zvi-susy, teleman-icm} for helpful expositions on this topic. The essential idea is as follows: the equivariant homology $C^G_\ast(\Gr_G; \cc)$ admits the structure of an $\Efr{3}$-algebra. In particular, the $\E{3}$-algebra structure on $C^G_\ast(\Gr_G; \cc)$ is equivariant for the action of $S^1$ on $C^G_\ast(\Gr_G; \cc)$ via loop rotation, and the action of $S^1$ on $\E{3}$ via rotation about a line $\ell\subseteq \RR^3$. Using the fact that the fixed points of the $S^1$-action on $\RR^3$ are given by the line $\ell$, it is argued in \cite{ben-zvi-susy} that the homotopy fixed points of $C^G_\ast(\Gr_G; \cc)$ admits the structure of an $\E{1}$-$C_{S^1}^\ast(\ast; \cc)$-algebra. Furthermore, the associative multiplication on $C^{G_c\times S^1_\rot}_\ast(\Gr_G; \cc)$ degenerates to the $2$-shifted Poisson bracket on $\H^{G_c}_\ast(\Gr_G; \cc)$ obtained from the $\E{3}$-algebra structure. The ``$\Omega$-background'' is supposed to refer to the compatibility of the $S^1$-action on $C^G_\ast(\Gr_G; \cc)$ with the $S^1$-action on the $\E{3}$-operad.

From the mathematical perspective, the idea that $S^1$-actions can be viewed as deformation quantizations has been made precise by \cite{preygel, toen-icm}, and more recently in \cite{butson-i, butson-ii}, at least in characteristic zero.  Although often not said explicitly, the idea has been a cornerstone of the development of Hochschild homology and its relatives. (The reader can skip the following discussion, since it will not be necessary in the remainder of this section; we only include it for completeness.) 

Consider a smooth $\cc$-scheme $X$, so that the Hochschild-Kostant-Rosenberg theorem gives an isomorphism $\HH(X/\cc) \simeq \Sym(\Omega^1_{X/\cc}[1])$. There is an isomorphism $\Sym(\Omega^1_{X/\cc}[1]) \simeq \bigoplus_{n\geq 0} (\wedge^n \Omega^1_{X/\cc})[n]$, so $\Sym(\Omega^1_{X/\cc}[1])$ can be understood as a shearing of the algebra $\Omega^\ast_{X/\cc} = \bigoplus_{n\geq 0} (\wedge^n \Omega^1_{X/\cc})[-n]$ of differential forms. The Hochschild-Kostant-Rosenberg theorem further states that the $S^1$-action on $\HH(X/\cc)$ is a shearing of the de Rham differential on $\Omega^\ast_{X/\cc}$. 
    
The Koszul dual of the algebra $\HH(X/\cc) \simeq \Sym(\Omega^1_{X/\cc}[1])$ is $\Sym(T_{X/\cc}[-2]) \simeq \co_{T^\ast[2] X}$; in the same way, the sheaf of differential operators on $X$ is Koszul dual to the de Rham complex of $X$. This can be drawn pictorially as follows:
$$\xymatrix{
\Sym(T_{X/\cc}[-2]) \simeq \co_{T^\ast[2] X} \ar@{~>}[r]^-{\text{def. quant}} \ar@{~>}[d]_-{\text{Koszul dual}} & \cd^\hbar_{X/\cc} \ar@{~>}[d]^-{\text{Koszul dual}} \\
\Sym_{\co_X}(\Omega^1_{X/\cc}[1]) \simeq \HH(X/\cc) \ar@{~>}[r]_-{S^1\text{-action}} & \text{shearing of }(\Omega^\ast_{X/k}, d_\dR).
}$$
Since the algebra $\cd_X^\hbar$ of differential operators is a quantization of $T^\ast[2] X$, this diagram illustrates the idea that the $S^1$-action on Hochschild homology plays the role of a Koszul dual to deformation quantization.

\begin{example}\label{ex: 3d-sl2}
    We will keep $G = \PGL_2$ as a running example in discussing Coulomb branches (see also \cite[Section 2]{seiberg-witten-coulomb}), so that $\ld{G} = \SL_2$. In this case,
    $$\cM_C \cong \spec \cc[x, a^{\pm 1}, \tfrac{a-a^{-1}}{x}]^{\Z/2} \cong \spec \cc[x^2, a+a^{-1}, \tfrac{a-a^{-1}}{x}]$$
    by \cref{thm: ordinary hmlgy reg centr}, where $\Z/2$ acts on $\cc[x, a^{\pm 1}, \tfrac{a-a^{-1}}{x}]$ by $x\mapsto -x$ and $a\mapsto a^{-1}$. 
    This is the regular centralizer group scheme of $\SL_2$. %atiyah hitchin 
    Let us denote by 
    \begin{align*}
        \Phi & = x^2,  \\
        U & = a + a^{-1}, \\
        V & = \tfrac{a-a^{-1}}{x}.
    \end{align*}
    Then we have the single relation
    $$U^2 - \Phi V^2 = (a + a^{-1})^2 - (a - a^{-1})^2 = 4,$$
    so $\cM_C$ is isomorphic to the subvariety of $\AA^3_\cc$ cut out by the above equation. 
    %Alternatively, and perhaps more suggestively, this equation can be rewritten as follows:
    %$$(U+2)(U-2) = \Phi V^2.$$
    This is known as the \textit{Atiyah-Hitchin manifold}, and was studied in great detail in \cite{atiyah-hitchin} (see \cite[Page 20]{atiyah-hitchin} for the definition). In \cite[Theorem A.1]{bfn-ii}, it was shown that the Atiyah-Hitchin manifold is isomorphic to the moduli space of solutions of Nahm's equations on $[-1,1]$ for $\mathrm{SU}(2)$ with an appropriate boundary condition.

    Since a normal vector to the defining equation of $\cM_C$ is $2U\partial_U - V^2 \partial_\phi - 2V\Phi \partial_V$, the standard holomorphic $3$-form $dU \wedge d\Phi \wedge dV$ on $\AA^3_\cc$ induces a holomorphic symplectic form $\tfrac{d\Phi \wedge dV}{2U}$ on $\cM_C$. (This can also be written as $\tfrac{dU \wedge dV}{V^2}$ or as $\tfrac{d\Phi \wedge dU}{2\Phi V}$.) The associated Poisson bracket on $\co_{\cM_C} \cong \H^{G_c}_\ast(\Gr_G; \cc)$ agrees with the $2$-shifted Poisson bracket arising from the $\E{3}$-structure on $C^{G_c}_\ast(\Gr_G; \cc)$.

    The quantized algebra $\cA_\epsilon$ can be described explicitly as follows. Let us write $\theta = \tfrac{1}{x}(s-1)$, where $s$ is the simple reflection generating the Weyl group of $\SL_2$. Then $\cA_\epsilon$ is generated as an algebra over $\cc\pw{\hbar}$ by $\Z/2$-invariant polynomials in $x$, $a^{\pm 1}$, and $\theta$, where $x$ is to be viewed as $a\partial_a$. Moreover, under the isomorphism $\cA_\epsilon/\hbar \cong \co_{\cM_C}$, the class $x$ is sent to $x$, and $\theta$ is sent to $\tfrac{a-1}{x}$. We then have the commutation relation $[x,a^{\pm 1}] = \pm \hbar a^{-1}$, induced by $[\partial_a, a] = \hbar$; see \cref{ex: ordinary quantized diffop}. This implies that 
    $$[x^2, a^{\pm 1}] = \hbar^2 a^{\pm 1} \pm 2\hbar a^{\pm 1} x,$$
    which in turn implies that $\cA_\epsilon$ is the quotient of the free associative $\cc\pw{\hbar}$-algebra on $\Phi$, $U$, and $V = \tfrac{1}{x}(a-a^{-1})$ subject to the relations
    \begin{align*}
        [\Phi,V] & = 2\hbar U - \hbar^2 V,\\
        [\Phi, U] & = 2\hbar \Phi V - \hbar^2 U, \\
        [U,V] & = \hbar V^2,\\
        U^2 - 4 & = \Phi V^2 - \hbar UV.
    \end{align*}
    Note that the commutation relations for $[\Phi, U]$ and $[U,V]$ in \cite[Equation B.3]{dimofte-garner} have typos, but it is stated correctly in \cite[Equation 5.51]{bullimore-dimofte-gaiotto}. 
\end{example}
\begin{example}
    When $G = \SL_2$, we can identify $\cM_C$ with the quotient of the scheme of \cref{ex: 3d-sl2} by the free $\Z/2$-action sending $U\mapsto -U$ and $V\mapsto -V$; so
    $$\cM_C \cong \spec \cc[x^2, (a+a^{-1})^2, \left(\tfrac{a-a^{-1}}{2x}\right)^2, \tfrac{(a+a^{-1})(a-a^{-1})}{2x}].$$
    This is the regular centralizer group scheme for $\PGL_2$. Note that if we denote
    \begin{align*}
        \Phi & = x^2,\\
        A & = (a+a^{-1})^2,\\
        B & = 4\left(\tfrac{a-a^{-1}}{2x}\right)^2 = \tfrac{(a-a^{-1})^2}{x^2},\\
        C & = 2\tfrac{(a+a^{-1})(a-a^{-1})}{2x} = \tfrac{(a+a^{-1})(a-a^{-1})}{x},
    \end{align*}
    then we have relations
    \begin{align*}
        AB & = C^2,\\
        A - \Phi B & = 4.
    \end{align*}
    In particular, $\cM_C$ is cut out in $\AA^3_\cc$ (with coordinates $\Phi$, $B$, and $C$) via the equation
    $$C^2 - \Phi B^2 = 4B.$$
    Note the similarity to the manifold from \cref{ex: 3d-sl2}: in fact, it is the quotient of the aforementioned manifold by the free $\Z/2$-action sending $U\mapsto -U$ and $V\mapsto -V$. In terms of these coordinates, $B = V^2$ and $C = UV$. (Sometimes, this quotient is also referred to as the Atiyah-Hitchin manifold.) It is also possible to describe $\cA_\epsilon$; we leave this to the reader, since it is rather tedious.
\end{example}

\begin{heuristic}\label{heuristic: 4d-n2}
    An unpublished conjecture of Gaiotto (which I learned about from Nakajima) says that the Coulomb branch of $4$d $\cN=2$ pure gauge theory over $\RR^3 \times S^1$ with a generic choice of complex structure can be modeled by $\cM_C^\fourd := \spec \KU^{G_c}_0(\Gr_G) \otimes_\Z \cc$. Although I do not know Gaiotto's motivation for this conjecture (it is probably inspired by \cite{seiberg-witten-coulomb}), my attempt at heuristically justifying it goes as follows. (In \cite[Appendix C(b)]{ku-rel-langlands}, I suggest that it might be slightly better to consider $\spec \ku^{G_c}_\ast(\Gr_G) \otimes_\Z \cc$ instead, where $\ku$ denotes \textit{connective} complex K-theory. The Bott class generating $\pi_2(\ku)$ plays the role of the radius of the circle $S^1$.)
    
    Recall that $\Gr_G/G\pw{t}$ can be viewed as $\Bun_G(S^2)$. It is reasonable to view $\KU_0(\Bun_G(S^2)) \otimes \cc$ as closely related to $\H_\ast(L\Bun_G(S^2); \cc)$, where $L\Bun_G(S^2)$ denotes the topological free loop space of $\Bun_G(S^2)$. Since $LBG \simeq BLG$, we have $L\Bun_G(S^2) \simeq \Bun_{LG}(S^2)$, so one might view $\H_\ast(L\Bun_G(S^2); \cc)$ as the ring of functions on the ``Coulomb branch of $3$d $\cN=4$ pure gauge theory on $\RR^3$ with gauge group $LG$''.

    Making precise sense of this phrase seems difficult, but one possible workaround could be the following. It is often useful to view gauge theory with gauge group $LG$ as ``finite temperature'' gauge theory with gauge group $G$. Recall that Wick rotation relates $(3+1)$-dimensional quantum field theory at a finite temperature $T$ to statistical mechanics over $\RR^3 \times S^1$ where the circle has radius $\tfrac{1}{2\pi T}$. This suggests that $\H_\ast(L\Bun_G(S^2); \cc)$ (which is more precisely to be replaced by $\KU^{G_c}_0(\Gr_G) \otimes \cc$) can be viewed as the ring of functions on the ``Coulomb branch of $4$d $\cN=2$ pure gauge theory on $\RR^3 \times S^1$ with gauge group $G$''. See \cite[Remark 3.14]{bfn-ii}. In \cite{bfm}, $\spec \KU^{G_c}_0(\Gr_G) \otimes \cc$ was identified with the phase space of the relativistic Toda lattice for $\ld{G}$.

    One can also define a quantization of $\cM_C^\fourd$ via $\cA_\epsilon^\fourd := \KU^{G_c \times S^1_\rot}_0(\Gr_G) \otimes \cc$; this can be viewed as a model for the quantized Coulomb branch of $4$d $\cN=2$ pure gauge theory on $\RR^3\times S^1$. The algebra $\cA_\epsilon^\fourd$ can be identified with the algebra of operators of the quantized relativistic Toda lattice for $\ld{G}$.
\end{heuristic}
\begin{example}\label{ex: 4d-sl2}
    When $G = \PGL_2$, \cref{thm: ku hmlgy reg centr} tells us that 
    $$\cM_C^\fourd \cong \spec \cc[x^{\pm 1}, a^{\pm 1}, \tfrac{a-a^{-1}}{x-1}]^{\Z/2} \cong \spec \cc[x+x^{-1}, a+a^{-1}, \tfrac{(a-a^{-1})(x+1)}{x-1}],$$
    where $\Z/2$ acts by $x\mapsto x^{-1}$ and $a\mapsto a^{-1}$.
    For simplicity, let us consider instead a slight variant of $\cM_C^\fourd = \spec \KU^{\mathrm{PSU}(2)}_0(\Gr_{\PGL_2}) \otimes_\Z \cc$, given by ${\cM'}_C^\fourd = \spec \KU^{\SU(2)}_0(\Gr_{\PGL_2}) \otimes_\Z \cc$. Then
    $${\cM'}_C^\fourd \cong \spec \cc[x^{\pm 1}, a^{\pm 1}, \tfrac{a-a^{-1}}{x-x^{-1}}]^{\Z/2} \cong \spec \cc[x+x^{-1}, a+a^{-1}, \tfrac{a-a^{-1}}{x-x^{-1}}].$$
    Let us write $\Psi = x + x^{-1}$, $W = a+a^{-1}$, and $Z = \tfrac{a-a^{-1}}{x-x^{-1}}$. Then, one easily verifies that ${\cM'}_C^\fourd$ is the subvariety of $\AA^3_\cc$ cut out by the equation
    $$W^2 - (\Psi^2 - 4)Z^2 = 4.$$
    %Alternatively, and perhaps more suggestively:
    %$$(W+2)(W-2) = (\Psi+2)(\Psi-2)Z^2.$$
    This may be regarded as a multiplicative analogue of the Atiyah-Hitchin manifold. It would be very interesting to understand a relationship between this manifold and the moduli space of solutions to some analogue of Nahm's equations for $\mathrm{PSU}(2)$ with an appropriate boundary condition. The complex manifold ${\cM'}_C^\fourd$ has a holomorphic symplectic form given by $\tfrac{d\Psi \wedge dZ}{W}$, which can also be written as $\tfrac{d\Psi \wedge dW}{(\Psi^2-4)Z}$ or as $\tfrac{dZ \wedge dW}{\Psi Z^2}$.

    It is also possible to explicitly describe the quantized algebra $\cA_\epsilon^\fourd$. The resulting description is not very enlightening, so we will only indicate how one reaches the answer. In this case, instead of the relation $[\partial_a, a] = \hbar$ which appeared in \cref{ex: 3d-sl2}, we have the relation $xa = qax$ (i.e., $xax^{-1} a^{-1} = q$); see \cref{ex: q quantized diffop}. In particular, $xa^{-1} = q^{-1} a^{-1} x$, $x^{-1} a = q^{-1} a x^{-1}$, and $x^{-1} a^{-1} = qa^{-1}x^{-1}$. It follows after some tedious calculation that $\cA_\epsilon^\fourd$ is the quotient of the free associative $\cc[q^{\pm 1}]$-algebra on $\Psi$, $W$, and $\tfrac{x+1}{x-1} (a-a^{-1})$ subject to four relations.

    Suppose we consider instead the variant of $\cA_\epsilon^\fourd$ defined by ${\cA'}_\epsilon^\fourd = \KU^{\SU(2) \times S^1_\rot}_0(\Gr_{\PGL_2}) \otimes \cc$.  Then ${\cA'}_\epsilon^\fourd$ is the quotient of the free associative $\cc[q^{\pm 1}]$-algebra on $\Psi$, $W$, and $Z = \tfrac{1}{x-x^{-1}} (a-a^{-1})$ subject to the relations
    \begin{align*}
        [\Psi, W] & = (q-1)(\Psi^2-4)Z - \tfrac{(q-1)^2}{2q} ((\Psi^2-4)Z + \Psi W), \\
        [\Psi, Z] & = (q-1)W - \tfrac{(q-1)^2}{2q}(\Psi Z + W),\\
        [Z,W] & = (q-1) \Psi Z^2 - \tfrac{(q-1)^2}{2q} (\Psi Z + W)Z, \\
        W^2 - 4 & = (\Psi^2-4)Z^2 - \tfrac{(q-1)^2}{2q} (\Psi^2-4)Z^2 + \tfrac{q^2-1}{2q} \Psi WZ.
    \end{align*}
\end{example}
\begin{remark}
    As always, one can also replace $\KU$ by connective complex K-theory $\ku$. This introduces a new ``Bott'' parameter $\beta$ which recovers $\KU$ when $\beta$ is set to $1$ (informally speaking), and recovers ordinary cohomology when $\beta$ is killed. In \cite[Appendix C(b)]{ku-rel-langlands}, I have suggested that working with $\ku$ instead of $\KU$ in \cref{heuristic: 4d-n2}  should produce the ``Coulomb branch of $4$d $\cN=2$ pure gauge theory on $\RR^3 \times S^1$ with gauge group $G$'' where the Bott parameter $\beta$ identifies with the radius of the circle $S^1$.
    
    In the case of \cref{ex: 4d-sl2}, the resulting manifold ${\cM'}_C^\beta = \spec \ku^{\SU(2)}_\ast(\Gr_{\PGL_2}) \otimes_{\Z[\beta]} \cc[\beta]$ is given by
    $${\cM'}_C^\beta \cong \spec \cc[\beta, x, \tfrac{1}{1+\beta x}, a^{\pm 1}, \tfrac{a-a^{-1}}{x-\ol{x}}]^{\Z/2} \cong \spec \cc[x+\ol{x}, a+a^{-1}, \tfrac{a-a^{-1}}{x-\ol{x}}].$$
    Here, $\ol{x}$ is the inverse of $x$ in the group law $x + y + \beta xy$, so that $\ol{x} = -\tfrac{x}{1+\beta x}$. 
    Let us write $\Psi' = x + \ol{x}$, $W = a+a^{-1}$, and $Z = \tfrac{a-a^{-1}}{x-\ol{x}}$; then ${\cM'}_C^\beta$ is the subvariety of $\AA^3_\cc \times \AA^1_\beta$ cut out by the equation
    $$W^2 - (\beta^2 \Psi' - 4)\Psi' Z^2 = 4.$$
    The complex manifold ${\cM'}_C^\beta$ has a holomorphic symplectic form given by $\tfrac{d\Psi' \wedge dZ}{W}$, which can also be written as $\tfrac{d\Psi' \wedge dW}{2Z\Psi' (\beta^2 \Psi'-4)}$ or as $\tfrac{dZ \wedge dW}{2Z^2(\beta^2 \Psi' - 2)}$.
    When $\beta$ is set to $1$, one can identify $\Psi'$ with $\Psi + 2$ with $\Psi$ as in \cref{ex: 4d-sl2}; then ${\cM'}_C^\beta$ recovers the multiplicative Atiyah-Hitchin manifold. When $\beta$ is killed, ${\cM'}_C^\beta$ is the usual Atiyah-Hitchin manifold. Again, it would be very interesting to understand a relationship between ${\cM'}_C^\beta$ and the moduli space of solutions to some analogue of Nahm's equations for $\mathrm{PSU}(2)$ with an appropriate boundary condition.  It is also possible to compute the loop-rotation equivariant version of ${\cM'}_C^\beta$, but we leave this to the reader.
\end{remark}
\begin{example}
    When $G = \SL_2$, one can view $\cM_C^\fourd$ with the quotient of the scheme ${\cM'}_C^\fourd$ of \cref{ex: 4d-sl2} by the free $\Z/2$-action sending $W\mapsto -W$ and $Z\mapsto -Z$; so
    $$\cM_C^\fourd \cong \spec \cc[x + x^{-1}, (a + a^{-1})^2, \left(\tfrac{a - a^{-1}}{x - x^{-1}}\right)^2, \tfrac{(a - a^{-1})(a + a^{-1})}{x - x^{-1}}].$$
    This is the regular centralizer group scheme for $\PGL_2$. Note that if we denote
    \begin{align*}
        \Psi & = x + x^{-1},\\
        A & = (a+a^{-1})^2 = a^2 + a^{-2} + 2,\\
        B & = \left(\tfrac{a-a^{-1}}{x-x^{-1}}\right)^2 = \tfrac{a^2+a^{-2}-2}{x^2 + x^{-2} - 2},\\
        C & = \tfrac{(a+a^{-1})(a-a^{-1})}{x - x^{-1}} = \tfrac{a^2-a^{-2}}{x-x^{-1}},
    \end{align*}
    then we have relations
    \begin{align*}
        AB & = C^2,\\
        A - (\Psi^2 - 4) B & = 4.
    \end{align*}
    In particular, $\cM_C^\fourd$ is cut out in $\AA^3_\cc$ (with coordinates $\Psi$, $B$, and $C$) via the equation
    $$C^2 - (\Phi^2 - 4) B^2 = 4B.$$
    Note the similarity to \cref{ex: 4d-sl2}. It is also possible to describe $\cA_\epsilon^\fourd$; again, we leave this to the reader, since it is rather tedious.

    Let us again note the variant involving connective complex K-theory $\ku$: in this case, 
    $$\cM_C^\beta \cong \spec \cc[\beta, x + \ol{x}, (a + a^{-1})^2, \left(\tfrac{a - a^{-1}}{x - \ol{x}}\right)^2, \tfrac{(a - a^{-1})(a + a^{-1})}{x - \ol{x}}].$$
    If we denote
    \begin{align*}
        \Psi' & = x + \ol{x},\\
        A & = (a+a^{-1})^2 = a^2 + a^{-2} + 2,\\
        B' & = \left(\tfrac{a-a^{-1}}{x-\ol{x}}\right)^2 = \tfrac{a^2+a^{-2}-2}{x^2 + \ol{x}^2 - 2x\ol{x}},\\
        C' & = \tfrac{(a+a^{-1})(a-a^{-1})}{x - \ol{x}} = \tfrac{a^2-a^{-2}}{x-\ol{x}},
    \end{align*}
    then we have relations
    \begin{align*}
        AB' & = {C'}^2,\\
        A - (\beta^2 \Psi' - 4)\Psi' B & = 4.
    \end{align*}
    In particular, $\cM_C^\beta$ is cut out in $\AA^3_\cc \times \AA^1_\beta$ (with coordinates $\Psi$, $B$, $C$, and $\beta$) via the equation
    $$C^2 - (\beta^2 \Psi' - 4)\Psi' B^2 = 4B.$$
\end{example}

Consider an elliptic curve $E(\cc)$ over $\cc$. Motivated by \cref{heuristic: 4d-n2} and \cite{nakajima-yoshioka}, one might expect that (in some specific complex structure) the Coulomb branch of $5$d $\cN=1$ pure gauge theory over $\RR^3 \times E(\cc)$ can be modeled by the complexification of the $G_c$-equivariant $k$-homology of $\Gr_G$, where $k$ is an elliptic cohomology theory associated to a putative integral lift of $E$. Unfortunately, a classical result of Tate says that there are no smooth elliptic curves over $\Z$ (see \cite{ogg-ell-curve-over-Z} for an elementary proof); so $E(\cc)$ cannot literally lift to $\Z$ (i.e., $\pi_0(k)$ cannot be $\Z$).

As a fix, one can more generally simultaneously consider all possible ``Coulomb branches'' $\cM_C^\fived := \spec \pi_0 \cf_G(\Gr_G)^\vee \otimes \cc$ associated to every complex-oriented $2$-periodic $\Eoo$-ring $k$ equipped with an oriented elliptic curve (this is very nearly the same as considering the universal example $\spec \tmf^{G_c}_0(\Gr_G) \otimes \cc$). We have described $\spec \pi_0 \cf_T(\Gr_G)^\vee \otimes \cc$ in \cref{thm: elliptic hmlgy reg centr}, from which one can calculate $\cM_C^\fived$. Similarly, one can even use \cref{thm: ell loop-rot flag} to calculate $\pi_0 \cf_{T\times \GG_m^\rot}(\Gr_G)^\vee \otimes \cc$ and $\cA_\epsilon^\fived := \pi_0 \cf_{G\times \GG_m^\rot}(\Gr_G)^\vee \otimes \cc$, but this is already incredibly complicated for $G = \SL_2$.

It would be very interesting to give a physical interpretation to $\pi_0 \cf_G(\Gr_G)^\vee \otimes \cc$ and $\pi_0 \cf_{G\times \GG_m^\rot}(\Gr_G)^\vee \otimes \cc$ for other $2$-periodic $\Eoo$-rings $k$, although we expect this to be very difficult. Indeed, most other chromatically interesting generalized equivariant cohomology theories only exist after profinite or $p$-adic completion, and do not admit transcendental analogues; but see \cref{rmk: morava e-theory}. It would also be very interesting to describe the analogue of our calculations for the ind-schemes $\cR_{G,\mathbf{N}}$ introduced in \cite{bfn-ii}. By adapting the methods of \cite[Section 4]{bfn-ii}, this is easy when $G$ is a torus. We expect it to lead to interesting geometry for nonabelian $G$. In \cite{ku-rel-langlands}, we extend the discussion of this paper (at least, the parts concerning ordinary cohomology and K-theory) to connective K-theory, and suggest an analogue of the relative Langlands program of \cite{bzsv} in this setting; as mentioned in the introduction of \cite{ku-rel-langlands}, the story therein also admits an elliptic variant.