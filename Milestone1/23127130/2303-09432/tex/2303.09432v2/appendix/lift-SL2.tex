In this brief section, we study the question of lifting $\SL_2$ as a group scheme over $\Z$ to other $\Eoo$-rings like K-theory $\KU$ or the sphere spectrum $S^0$. To set up the question, let us first make the notion of ``lifting'' precise: if $X$ is a scheme over $\Z$ with structure sheaf $\co_X$ and $k$ is an $\Eoo$-ring equipped with an $\Eoo$-map $k \to \Z$, a flat lifting of $X$ to $k$ as an $\E{n}$-scheme (see \cite{francis-thesis}) will mean the data of a sheaf $\co_X^\top$ of $\E{n}$-$k$-algebras on $X$ along with an isomorphism $\co_X^\top \otimes_k \Z \xrightarrow{\simeq} \co_X$. 
%If $G$ is a group scheme over $\Z$, then a flat lifting of $G$ to $k$ as a group object in $\E{n}$-schemes will mean the data of a flat lifting of $G$ to $k$ as an $\E{n}$-scheme along with an $\E{1}$-coalgebra structure on $\co_G^\top$ (in the category of sheaves of $\E{n}$-$k$-algebras on $X$) whose transport along the isomorphism $\co_G^\top \otimes_k \Z \xrightarrow{\simeq} \co_G$ identifies with the existing coalgebra structure on $\co_G$. In particular, the coproduct $\co_G^\top \to \co_G^\top \otimes_k \co_G^\top$ must be a map of sheaves of $\E{n}$-$k$-algebras. (If $G$ is commutative, one can ask for further refinements, like an $\E{j}$-coalgebra structure on the sheaf $\co_G^\top$ with $j\geq 2$; but we will not need this here.)
It is easy to lift $\GL_n$ to the sphere as an $\Eoo$-scheme (i.e., a spectral scheme in the sense of \cite{SAG}), because it is an open subset in $\AA^{n^2}$. However, we will see in a moment that a simple calculation proves that $\SL_2$ itself cannot be lifted in a natural way (and slight variants of this question lead to very subtle issues that we have been unable to resolve).

Observe that many schemes associated to $\SL_2$ lift all the way to the sphere spectrum. For instance, each choice of Borel subgroup $B\subseteq \SL_2$ (with unipotent radical $U$) defines a surjection $\SL_2 \to \AA^2 - \{0\}$, given by quotienting on the left or the right by $U$. The scheme $\AA^2$ admits a flat lift to a spectral scheme $(\AA^2)_{S^0}$ over $S^0$, given by $\spec S^0[\Z_{\geq 0} \times \Z_{\geq 0}]$. The simple observation is the following:
\begin{prop}\label{prop: E4-lifting SL2}
    There is no flat lifting $(\SL_2)_{S^0}$ of $\SL_2$ to $S^0$ (or even to connective complex K-theory $\ku$) as an $\E{4}$-scheme along with a lifting $(\SL_2)_{S^0} \to (\AA^2)_{S^0}$ of the maps $\SL_2 \to \AA^2 - \{0\} \subseteq \AA^2$.
\end{prop}
\begin{proof}
    Fix a prime $p$, and let $n\geq 1$.
    A flat lifting to $\ku$ of an affine (say) scheme $X = \spec(R)$ over $\Z$ defines power operations on $R$. Indeed, if $\tilde{R}$ is the $\E{n}$-$\ku$-algebra lifting $R$, then $R^\wedge_p \cong \pi_0(L_{K(1)} \tilde{R})$. If $A$ is a $K(1)$-local $\E{2n+1}$-$\KU$-algebra, then $\pi_0(A)$ admits the structure of a ``weak $\delta_n$-ring'', in the sense that there is a map $\delta: \pi_0(A) \to \pi_0(A/p^n)$ of sets (where $A/p^n$ denotes the derived quotient) such that 
    \begin{align*}
        \delta(x+y) & = \delta(x) + \delta(y) - \tfrac{1}{p}((x+y)^p - x^p - y^p) \pmod{p^{n-1}}, \\
        \delta(xy) & = \delta(x) y^p + \delta(y) x^p + p\delta(x) \delta(y) \pmod{p^{n-1}}.
    \end{align*}
    If $A$ refines to an $\E{2n+2}$-$\KU$-algebra, then $\pi_0(A)$ further admits the structure of a ``$\delta_n$-ring'', meaning that the above relations hold modulo $p^n$. (I am grateful to Ishan Levy for a discussion about this.) In this case, the map $\psi: \pi_0(A) \to \pi_0(A/p^{n+1})$ sending $x\mapsto x^p + p\delta(x)$ is a ring map lifting the Frobenius. Observe that
    $$\delta(-x) = \begin{cases}
        -\delta(x) - x^2 & p=2, \\
        -\delta(x) & p>2.
    \end{cases}$$
    The operation $\delta$ is furthermore natural in maps of $K(1)$-local $\E{2n+1}$-$\KU$-algebras. When $n = \infty$, the power operation $\delta$ is constructed in \cite{k1local}, and its construction for finite $n$ is nearly identical.

    The $\Z$-algebra $R = \Z[x]$ admits a canonical lifting to $S^0$ as an $\Eoo$-ring (via $S^0[\Z_{\geq 0}] = S^0[x]$). The corresponding $\delta$-operation on $R$ is simply given by $\delta(x) = 0$. By choosing $U \subseteq \SL_2$ to be the subgroup of upper or lower triangular matrices, one obtains two maps $\SL_2 \to \AA^2$ which send a matrix $\begin{psmallmatrix}
        a & b \\
        c & d
    \end{psmallmatrix}$ to $(a,b)$ and $(d,b)$. The resulting map $f: \co_{\AA^2} \otimes_\Z \co_{\AA^2} \to \co_{\SL_2}$ is a surjection, with kernel given by the determinant ideal $(ad - bc - 1)$. If $\SL_2$ admits a lift to an $\E{4}$-$\ku$-algebra compatibly with the two maps $\SL_2 \to \AA^2$, then the map $f$ must be one of $\delta_1$-rings. It follows that $\delta$ vanishes on the generators $a,b,c,d\in \co_{\SL_2}$. In particular, $\delta(ad) = \delta(a) d^p + \delta(d) a^p + p\delta(a) \delta(d)$ must vanish in $\co_{\SL_2}/p$; similarly for $\delta(bc)$.

    If $p=2$, then
    \begin{align*}
        \delta(ad - bc) & = \delta(ad) + \delta(-bc) + adbc \\
        & = \delta(ad) - \delta(bc) - (-bc)^2 + adbc = bc,
    \end{align*}
    where the final equality is because $\delta(ad) = \delta(bc) = 0$ and $ad - bc = 1$. Similarly, if $p>2$, then
    \begin{align*}
        \delta(ad - bc) & = \delta(ad) + \delta(-bc) - \tfrac{1}{p} ((ad - bc)^p - (ad)^p - (-bc)^p) \\
        & = \tfrac{1}{p} ((bc+1)^p - b^p c^p - 1),
    \end{align*}
    again because $\delta(ad) = \delta(bc) = 0$. The fact that $\delta(ad - bc) = \delta(1) = 0$ implies that for any commutative $\FF_p$-algebra $R$ and a matrix $\begin{psmallmatrix}
        a & b \\
        c & d
    \end{psmallmatrix} \in \SL_2(R)$, the polynomial $\tfrac{1}{p} ((bc+1)^p - b^p c^p - 1)$ must vanish in $R$. This is clearly false: take $R = \FF_p[x]$ and the matrix $\begin{psmallmatrix}
        x+1 & x \\
        1 & 1
    \end{psmallmatrix}$. (One could of course use any prime $p$ to obtain this contradiction; but we allow flexibility in the choice of $p$ to assuade any worries about $\SL_2$ being liftable upon localization at some primes but not others.)
\end{proof}
\begin{remark}\label{rmk: delta and PD}
    Since $\delta(x)$ behaves like the $p$th divided power $-\gamma_p(x)$, the argument of \cref{prop: E4-lifting SL2} can alternatively be interpreted as showing that the ideal which cuts out $\SL_2 \hookrightarrow \GL_2$ does not have a divided power structure, even over $\Z/p^2$.
\end{remark}
Since a weak $\delta_1$-ring structure on a commutative ring $R$ is just a map of sets $\delta: R \to R/p$ satisfying no relations, the above argument does not prove the analogue of \cref{prop: E4-lifting SL2} with $\E{4}$ replaced by $\E{3}$ or $\E{2}$. In particular, \cref{prop: E4-lifting SL2} intriguingly leaves open the possibility that $\SL_2$ admits a lift as an $\E{2}$- or $\E{3}$-scheme to $S^0$ or $\ku$ compatibly with its natural actions on $\AA^2$. Note that the same argument in \cref{prop: E4-lifting SL2} shows that $\SL_n$ also cannot be lifted as an $\E{4}$-scheme to to $S^0$ (or even to connective complex K-theory $\ku$) for any $n\geq 2$ compatibly with its natural actions on $\AA^n$.
\begin{remark}
    The argument of \cref{prop: E4-lifting SL2} is very robust. It can be used to show, for instance, that if $1<k<n-1$, then the Grassmannian $\Gr_k(\AA^n)$ over $\Z$ cannot be lifted as an $\E{4}$-scheme to $S^0$, or even to $\ku$, compatibly with its Pl\"ucker embedding into $\PP(\wedge^k \AA^n) \cong \PP^{\binom{n}{k}-1}$ (which is lifted via the flat projective space of \cite[Construction 5.4.1.3]{SAG}). In fact, an even stronger statement is true: \cite[Theorem 6]{frobenius-on-toric} implies that for a semisimple algebraic group $G$, the flag variety $G/P$ over $\Z/p^2$ does not have a lift of Frobenius as long as the parabolic subgroup $P$ is contained in one of the maximal parabolics enumerated in \cite[Examples 4.3.1-4.3.7]{frobenius-on-toric}. This, in particular, recovers the statement about Grassmannians above. The proof of the general claim uses Bott vanishing, which is more sophisticated than the hands-on approach of \cref{prop: E4-lifting SL2}.
    
    For concreteness, let us demonstrate this non-liftability for $\Gr_2(\AA^4)$, which is cut out inside $\PP^5$ (with coordinates $[x_0: x_1: x_2: y_0: y_1: y_2]$) via the formula $x_0 y_0 - x_1 y_1 + x_2 y_2 = 0$. Let $a = x_0 y_0$, $b = x_1 y_1$, and $c = x_2 y_2$, so that $b = a+c$. Again, $\delta(a) = \delta(b) = \delta(c) = 0$. When $p=2$, for instance, this implies that
    \begin{align*}
        \delta(a - b + c) & = \delta(a) + \delta(-b) + \delta(c) + ab + bc - ac\\
        & = -b^2 + ab + bc - ac = -ac.
    \end{align*}
    For a general prime, one has $\delta(a - b + c) = \tfrac{1}{p} (a^p + c^p - (a+c)^p)$.
    Since $\delta(a-b+c) = \delta(0) = 0$, this implies that $\tfrac{1}{p} (a^p + c^p - (a+c)^p) = 0$; since this function is not identically zero on $\Gr_2(\AA^4)_{\FF_p}$, we obtain the desired contradiction. Note that, just as in \cref{rmk: delta and PD}, this argument says that the ideal cut out by $x_0 y_0 - x_1 y_1 + x_2 y_2$ in $\Z[x_0, \cdots, y_2]/p^2$ does not admit a divided power structure; from this perspective, the above observation should be attributed to Koblitz \cite[Section 3.3(4)]{berthelot-ogus}. 
\end{remark}
\begin{corollary}
    Let $(\GL_2)_{S^0}$ denote the spectral scheme $(\AA^4)_{S^0}[\tfrac{1}{ad- bc}]$, and let $(\GL_1)_{S^0} = \spec S^0[x^{\pm 1}]$. Then the map $\det: \GL_2 \to \GG_m$ over $\Z$ does not lift to a map $(\GL_2)_{S^0} \to (\GL_1)_{S^0}$ exhibiting $(\GL_2)_{S^0}$ as an $\E{4}$-scheme over $(\GL_1)_{S^0}$; in fact, such a lifting is prohibited even over $\ku$.
\end{corollary}
\begin{proof}
    If there was a lifting $(\GL_2)_\ku \to (\GL_1)_\ku$ which exhibits $(\GL_2)_\ku$ as an $\E{4}$-scheme over $(\GL_1)_\ku$, then there would be a map $\ku[x^{\pm 1}] \to \co_{(\GL_2)_\ku} = \ku[a,b,c,d,\tfrac{1}{ad - bc}]$ sending $x\mapsto ad - bc$ which exhibits $\co_{(\GL_2)_\ku}$ as an $\E{4}$-$\ku[x^{\pm 1}]$-algebra. Base-changing along the map $\ku[x^{\pm 1}] \to \ku$ sending $x\mapsto 1$ would then produce an $\E{4}$-$\ku$-algebra $\co_{(\GL_2)_\ku} \otimes_{\ku[x^{\pm 1}]} \ku$ which lifts $\co_{\SL_2}$ to $\ku$. Such a lifting is prohibited by \cref{prop: E4-lifting SL2}. 
\end{proof}
If $(\GL_1^\free)_{S^0} = \spec S^0\{x\}[1/x]$ where $S^0\{x\}$ denotes the free $\Eoo$-ring on one generator, then there \textit{is} a map $(\GL_2)_{S^0} \to (\GL_1^\free)_{S^0}$ exhibiting $(\GL_2)_{S^0}$ as an $\Eoo$-scheme over $(\GL_1^\free)_{S^0}$. However, $(\GL_1^\free)_{S^0}$ is not a flat lift of $\GL_1$ to $S^0$. In other words, there is no reasonable way to construct ``strict'' determinants over the sphere spectrum (or even over $\ku$), at least in the setting of spectral algebraic geometry of $\E{4}$-schemes. 
\begin{remark}
    There \textit{is} a lifting of $\det$ to a map $(\GL_2)_\MU \to (\GL_1)_\MU$ of $\E{2}$-$\MU$-schemes\footnote{However, it does \textit{not} necessarily exhibit $(\GL_2)_\MU$ as an $\E{2}$-scheme over $(\GL_1)_\MU$; so the fiber product $(\GL_2)_\MU \times_{(\GL_1)_\MU} \spec(\MU)$ may not exist in $\E{2}$-$\MU$-schemes. This is one of the quirks of spectral algebraic geometry with $\E{n}$-rings for finite $n$: even if $X$ and $Y$ are spectral schemes with sheaves of $\Eoo$-rings, a map $f: X \to Y$ of $\E{n}$-schemes may not exhibit $X$ as an $\E{n}$-$Y$-scheme. This is because an $\E{n}$-map $A \to B$ between $\Eoo$-rings need not exhibit $B$ as an $\E{n}$-$A$-algebra.}. In fact, any $\E{1}$-$\MU$-algebra map from $\MU[x^{\pm 1}]$ to an even $\E{2}$-$\MU$-algebra $A$ can be refined to an $\E{2}$-$\MU$-algebra map. Indeed, an $\E{1}$-$\MU$-algebra map $f: \MU[x^{\pm 1}] \to A$ can be viewed as the map $g: S^1 \to \BGL_1(A)$ which detects $f(x) \in \pi_1(\BGL_1(A)) = \pi_0(A)^\times$. The map $f$ can be upgraded to an $\E{2}$-$\MU$-algebra map if and only if $g$ can be delooped once. Since $BS^1 = \CP^\infty$ has an even cell structure and $B^2\GL_1(A)$ has even homotopy, there are no obstructions to extending the map $S^2 \to B^2\GL_1(A)$ detecting $f(x)$ along the inclusion $S^2 \to BS^1$. It follows from this discussion that there is an $\E{2}$-$\MU$-algebra map $\det_\MU: \MU[x^{\pm 1}] \to \co_{(\GL_2)_\MU} = \MU[a,b,c,d,\tfrac{1}{ad - bc}]$ which sends $x\mapsto ad - bc$. This, however, only exhibits $\co_{(\GL_2)_\MU}$ as an $\E{1}$-$\MU[x^{\pm 1}]$-algebra. To exhibit it as an $\E{2}$-$\MU[x^{\pm 1}]$-algebra, one would need to upgrade $\det_\MU$ to an $\E{3}$-$\MU$-algebra map, but this seems quite difficult.
\end{remark}

The argument for \cref{prop: E4-lifting SL2} does not require any knowledge of the coalgebra structure on $\co_{\SL_2}$, so it is possible that $\SL_2$ lifts as an $\E{3}$-scheme to $S^0$ or $\ku$, but does not lift as an $\E{1}$-group object therein. The group scheme $\GG_a$ provides a simple example of this phenomenon of lifting as a spectral scheme, but not as a group scheme:
\begin{prop}\label{prop: lifting Ga as group scheme}
    There is no flat lifting $(\GG_a)_{S^0}$ of $\GG_a$ to $S^0$ (or even to connective complex K-theory $\ku$) as an $\E{1}$-group object in $\E{4}$-schemes.
\end{prop}
\begin{proof}
    The $\E{4}$-$\ku$-algebra of functions on the flat lifting $(\GG_a)_\ku$ must be given by $\ku[x]$. The group law on $\GG_a$ is given by the coproduct $\Z[x] \to \Z[x,y]$ sending $x\mapsto x + y$. We therefore need to show that there is no $\E{4}$-$\ku$-algebra map $\Delta:\ku[x] \to \ku[x,y]$ given by $\Delta(x) =  x+y$ on $\pi_0$. This follows from the $\delta_1$-ring structure: we need $\Delta(\delta(x)) = \delta(\Delta(x))$ in $\FF_p[x]$. But $\delta(x) = \delta(y) = 0$, so $\delta(\Delta(x)) = \tfrac{1}{p} (x^p + y^p - (x+y)^p)$ must vanish in $\FF_p[x,y]$, which is a contradiction.
\end{proof}
It was already shown in \cite[Proposition 1.6.20]{elliptic-ii} that there is no flat lifting $(\GG_a)_{S^0}$ of $\GG_a$ to the truncation $\tau_{\leq 1}(S^0)$ as a group scheme; this of course prohibits such a lifting to $S^0$, too. The proof in \textit{loc. cit.} uses the nontriviality of the Hopf element $\eta \in \pi_1(S^0)$. Since this element vanishes in $\ku$, the proof therein cannot be directly adapted to prove \cref{prop: lifting Ga as group scheme}. 
However, let us note that using \cite[Proposition 5.4.9]{rotinv}, one can show that the additive group over $\Z$ admits a flat lifting to a group object in $\E{2}$-schemes over $S^0$.