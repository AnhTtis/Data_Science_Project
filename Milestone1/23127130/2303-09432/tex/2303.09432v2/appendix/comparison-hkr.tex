The calculations of this article (more precisely, the perspective of \cref{rmk: 1-shifted cartier}) were motivated by the work of Hopkins-Kuhn-Ravenel \cite{hkr}, who study the case of finite groups.
In this section, we will describe a relationship to their work. Our discussion will be rather heuristic, and we will sweep a few details under the rug to keep the exposition readable.

Before proceeding, the first thing to note is that while the present article only discusses \textit{connected} compact Lie groups, Hopkins-Kuhn-Ravenel only study \textit{discrete} compact Lie groups (that is, finite groups).  Next, the work of \cite{hkr} only deals with Borel-equivariant cohomology. This means that one does \textit{not} need to assume that the complex-oriented $2$-periodic $\Eoo$-ring $k$ is equipped with an oriented commutative $k$-group $\GG$; recall from \cref{sec: equiv coh} that the purpose of $\GG$ is to provide a decompletion of Borel-equivariant cohomology for compact abelian Lie groups. All that is needed is the formal completion $\hat{\GG}$ of $\GG$ at the identity section. Note that this is not extra data associated to $k$, since $\hat{\GG} = \spf k^{\CP^\infty_+}$. Let $\hat{\GG}_0$ denote the underlying $1$-dimensional formal group over $\pi_0(k)$.

In fact, an even more stringent condition is required of $k$ in \cite{hkr}: not only is it required to be complex-oriented and $2$-periodic, but $\pi_0(k)$ is required to be a complete local Noetherian domain with maximal ideal $\fr{m}$ whose residue field $\pi_0(k)/\fr{m}$ is of characteristic $p>0$, such that $p$ is not nilpotent in $\pi_0(k)$. Let $n$ denote the height of the formal group $\hat{\GG}_0$ base-changed along $\pi_0(k) \to \pi_0(k)/\fr{m}$. In the following discussion, we will simply write $k^0(X)$ to denote $\pi_0$ of the the $k$-cochains on $X$ (instead of the more cumbersome notation $\pi_0 \cf(X)$).

Let $\cc_p$ denote the completion of the algebraic closure of $\QQ_p$, and choose a continuous embedding $\pi_0(k) \to \co_{\cc_p}$. The base-change of $\hat{\GG}_0$ to $\co_{\cc_p}$ defines a formal group law on the maximal ideal of $\co_{\cc_p}$; assume that the base-change of $\hat{\GG}_0$ along the map $\pi_0(k) \to \pi_0(k)/\fr{m}$ has finite height. Then, there exists an exponential isomorphism 
\begin{equation}\label{eq: exponential iso}
    e: (\QQ_p/\Z_p)^n \xrightarrow{\sim} (\fr{m}_{\co_{\cc_p}}, +_{\hat{\GG}_0}),
\end{equation}
where $n$ is the height of $\hat{\GG}_0$. The basic calculation driving the results of \cite{hkr} is the following.
\begin{prop}\label{prop: coh of BZpj}
    There is an isomorphism 
    $$k^0(B\Z/p^j) \cong \pi_0(k)\pw{t}/[p^j](t),$$
    where $[p^j](t) \in \pi_0(k)\pw{t}$ is the $p^j$-series of the formal group law $\hat{\GG}_0$, and $t$ is the first Chern class of the standard character $\Z/p^j \cong \mu_{p^j} \subseteq S^1$. That is, there is an isomorphism $\spf k^0(B\Z/p^j) \cong \hat{\GG}_0[p^j]$.
\end{prop}
\begin{construction}\label{cstr: hkr}
    \cref{prop: coh of BZpj} and the discussion preceding it gives an isomorphism
    $$\spf(k^0(B\Z/p^j)) \otimes_{\spf \pi_0(k)} \spec \cc_p \cong \tfrac{1}{p^j}\Z/\Z,$$
    where the right-hand side denotes the constant group scheme over $\cc_p$. A choice of generator (e.g., $\tfrac{1}{p^j}$) of this group therefore gives a map $k^0(B\Z/p^j) \to \cc_p$.
    Now let $F$ be a finite group, and let $f: \Z_p^n \to F$ be a homomorphism. Then $f$ factors as a map $\Z_p^n \to (\Z/p^j)^n \to F$ for some $j$, so there is a map $k^0(BF) \to k^0(B(\Z/p^j)^n)$. Taking the product of the maps $k^0(B\Z/p^j) \to \cc_p$ described above gives a map $k^0(B(\Z/p^j)^n) \to \cc_p$, which finally defines a composite map
    $$k^0(BF) \to k^0(B(\Z/p^j)^n) \to \cc_p.$$
    This composite depends only on the conjugacy class of $f$, and so this construction defines a map $\Hom(\Z_p^n, F)\mmod F \to \Map(k^0(BF), \cc_p)$, whose adjoint is a map $k^0(BF) \to \Map(\Hom(\Z_p^n, F)\mmod F, \cc_p)$. Here, $F$ acts on $\Hom(\Z_p^n, F)$ by conjugation.
\end{construction}
In the discussion below, $F$ will be a finite group. For simplicity, we will further assume that $k^\ast(BF)$ is concentrated in even degrees (so, by the $2$-periodicity of $k$, it is completely determined by $k^0(BF)$). If $X$ is an $F$-space, the homotopy orbits of $X$ will be denoted $X_{hF}$, while the ordinary quotient of $X$ by the $F$-action will be denoted $X\mmod F$.
\begin{theorem}[Hopkins-Kuhn-Ravenel]\label{thm: hkr}
    The map from \cref{cstr: hkr} defines an isomorphism 
    $$k^0(BF) \otimes_{\pi_0(k)} \cc_p \xrightarrow{\cong} \Map(\Hom(\Z_p^n, F)\mmod F, \cc_p).$$
    The quotient $\Hom(\Z_p^n, F)\mmod F$ can be replaced by the homotopy orbits $\Hom(\Z_p^n, F)_{hF}$, since $F$ is a finite group and its order is invertible in $\cc_p$.
\end{theorem}
Note that the homotopy orbits $\Hom(\Z_p^n, F)_{hF}$ can be identified with $\Map(BT_p^n, BF)$, where $T_p^n = (\QQ_p/\Z_p)^n$ is the $p$-adic $n$-torus.
One can use a ring smaller than $\cc_p$ in \cref{thm: hkr}; essentially, one only needs to extend scalars to the rationalization of the smallest ring containing $\pi_0(k)$ over which the exponential isomorphism \cref{eq: exponential iso} holds.

In \cite{elliptic-iii}, Lurie observes that the isomorphism of \cref{thm: hkr} can be categorified, at least if one assumes the data of a decompletion $\GG$ of $\hat{\GG}$. (We refer the reader to \cite{elliptic-iii} for further details, since the specific setup will not concern us much below.) Namely, if $F$ is a finite group, Lurie defines an $\infty$-category $\Loc_F(\ast; k)$ (denoted by $\mathrm{LocSys}_\GG(BF)$ in \textit{loc. cit.}), and proves the following as (a consequence of) \cite[Theorem 6.4.1]{elliptic-iii}:
\begin{theorem}[Lurie]\label{thm: lurie tempered}
    Fix an embedding $\pi_0(k) \to \cc_p$, so it defines an $\Eoo$-map $k \to \cc_p[u^{\pm 1}]$.
    There is a symmetric monoidal fully faithful embedding
    $$\Loc_F(\ast; k) \otimes_k \cc_p[u^{\pm 1}] \hookrightarrow \Loc(\Map(BT_p^n, BF); \cc_p[u^{\pm 1}]).$$
\end{theorem}
The essential image of the above embedding is described in \cite[Theorem 6.5.13]{elliptic-iii}.

Let us examine the isomorphism \cref{thm: hkr} and the embedding \cref{thm: lurie tempered} further; we will rephrase the right-hand sides of both results as algebro-geometric objects. To do this, note that the exponential isomorphism between $\hat{\GG}_0 \otimes_{\pi_0(k)} \cc_p$ and $(\QQ_p/\Z_p)^n$ defines an isomorphism between $\bD(\hat{\GG}_0) \otimes_{\pi_0(k)} \cc_p$ and $\Z_p^n$. Here, $\bD(\hat{\GG}_0) = \Hom(\hat{\GG}_0, \GG_m)$ is the Cartier dual of $\hat{\GG}_0$.  Note that the $1$-shifted Cartier dual $\hat{\GG}_0^\vee$ can be identified with the classifying stack of $\bD(\hat{\GG}_0)$.

View the finite group $F$ as defining a constant group scheme $\ul{F}$ over $\cc_p$. Since $\hat{\GG}_0^\vee \otimes_{\pi_0(k)} \cc \cong \Z_p^n$, the mapping stack $\Map(\hat{\GG}_0^\vee, B\ul{F})$ is the quotient of the discrete scheme $\ul{\Hom(\Z_p^n, F)}$ by the constant group scheme $\ul{F}$ acting by conjugation. It follows that the $\cc_p$-algebra $\Map(\Hom(\Z_p^n, F)\mmod F, \cc_p)$ is the algebra of functions on the mapping stack $\Map(\hat{\GG}_0^\vee, B\ul{F})$. (Not that since the order of $F$ is invertible in $\cc_p$, the derived and classical algebras of functions agree.) Similarly, $\Loc(\Map(BT_p^n, BF); \cc_p)$ can be viewed as the category of quasicoherent sheaves on the mapping stack $\Map(\hat{\GG}_0^\vee, B\ul{F})$. Therefore, \cref{thm: hkr} and \cref{thm: lurie tempered} can be restated as:
\begin{align}
    \pi_0 k_F \otimes_{\pi_0(k)} \cc_p & \xrightarrow{\cong} \Gamma(\Map(\hat{\GG}_0^\vee, B\ul{F}); \co), \label{eq: restated hkr} \\
    \Loc_F(\ast; k) \otimes_k \cc_p[u^{\pm 1}] & \hookrightarrow \QCoh(\Map(\hat{\GG}_0^\vee, B\ul{F})) \otimes_{\Mod_{\cc_p}} \Mod_{\cc_p[u^{\pm 1}]}. \label{eq: restated lurie tempered}
\end{align}
One can even replace $\hat{\GG}_0$ in the above by $\GG_0$. Observe, now, that $\Map(\GG_0^\vee, B\ul{F})$ is simply the stack $\Bun_{\ul{F}}(\GG_0^\vee)$. 

We can now compare \cref{eq: restated hkr} and \cref{eq: restated lurie tempered} to the discussion in the body of this article. Assume now that $k$ is either $\QQ[u^{\pm 1}]$, $\KU$, or elliptic cohomology. If $G_c$ was instead  a connected compact Lie group, the analogue of \cref{eq: restated hkr} states that $\pi_0 k_{G_c} \otimes_{\pi_0(k)} \cc$ is the ring of (classical, not derived!) global sections of the structure sheaf on $\Bun_G^\ss(\GG_0^\vee)$, where $G$ is the complex reductive group corresponding to $G_c$. This is clear when $\GG_0$ is $\GG_a$ (and $k = \QQ[u^{\pm 1}]$) or $\GG_m$ (and $k = \KU$). In the case when $\GG_0$ is an elliptic curve, this is essentially part of the definition of equivariant elliptic cohomology as sketched in \cite{survey} and constructed in \cite{gepner-meier, t-equiv-tmf}.

Let us continue to assume that $G_c$ is a connected compact Lie group, and further impose that it is simply-laced and almost simple. We will now a give a heuristic argument suggesting that \cref{thm: intro omnibus} -- or rather, its variant from \cref{rmk: g-equiv regular satake elliptic} describing $\Loc_{G_c}^\gr(\Gr_G; k)$ -- can be viewed as an analogue of \cref{eq: restated lurie tempered}. 

Indeed, the rephrasing of \cref{rmk: g-equiv regular satake elliptic} from \cref{rmk: 1-shifted cartier} states that there is an equivalence
\begin{equation}\label{eq: appendix 1-shifted cartier}
    \Loc_{\ld{G}_c}^\gr(\Gr_G; k) \otimes_{\pi_0(k)} \cc \simeq \QCoh(\Bun_{\ld{G}}^\ss(\GG_0^\vee)^\reg).
\end{equation}
The regular locus $\Bun_{\ld{G}}^\ss(\GG_0^\vee)^\reg$ is an open substack of $\Bun_{\ld{G}}^\ss(\GG_0^\vee)$ (whose complement has codimension $\geq 2$, as proved in \cite[Proposition 3.1.16]{davis-elliptic-springer}), and so there is a fully faithful embedding $\QCoh(\Bun_{\ld{G}}^\ss(\GG_0^\vee)^\reg) \hookrightarrow \QCoh(\Bun_{\ld{G}}^\ss(\GG_0^\vee))$. That is, there is a fully faithful embedding
\begin{equation}\label{eq: rephrased-fully faithful}
    \Loc_{\ld{G}_c}^\gr(\Gr_G; k) \otimes_{\pi_0(k)} \cc \hookrightarrow \QCoh(\Bun_{\ld{G}}^\ss(\GG_0^\vee)).
\end{equation}
Assume for the moment that \cref{eq: rephrased-fully faithful} holds if $\ld{G}$ is a finite group $\ld{F}$ (and replace $\cc$ above by $\cc_p$). Of course, it is not clear what the Langlands dual $F$ of $\ld{F}$ should mean; but it is reasonable to believe that, whatever it is, $F$ should be a finite group (or perhaps a finite group scheme). In any case, $\Gr_F$ will just be a point, so the left-hand side of \cref{eq: rephrased-fully faithful} is simply $\Loc_{\ld{F}}^\gr(\ast; k) \otimes_{\pi_0(k)} \cc$. It is reasonable to expect that, thanks to a formality-type statement, the $2$-periodification of the category $\Loc_{\ld{F}}^\gr(\ast; k) \otimes_{\pi_0(k)} \cc$ is equivalent to $\Loc_{\ld{F}}(\ast; k) \otimes_{k} \cc[u^{\pm 1}]$.

Turning to the right-hand side of \cref{eq: rephrased-fully faithful}, note that because $\ld{F}$ is a finite group, there is no meaningful notion of semistability, and so $\Bun_{\ld{F}}^\ss(\GG_0^\vee) = \Bun_{\ld{F}}(\GG_0^\vee)$. With these translations made (so the left-hand side of \cref{eq: rephrased-fully faithful} is replaced by $\Loc_{\ld{F}}(\ast; k) \otimes_{k} \cc[u^{\pm 1}]$, and the right-hand side by the $2$-periodification of $\QCoh(\Bun_{\ld{F}}(\GG_0^\vee))$), \cref{eq: rephrased-fully faithful} is precisely of the form \cref{eq: restated lurie tempered}, as claimed.

\begin{remark}\label{rmk: Bun S2 finite group}
    The above comparison between the quotient $\Gr_G/G\pw{t}$ for a connected compact Lie group $G_c$ and the classifying space $BF$ for a finite group $F$ can be made more precise by noting that $\Gr_G/G\pw{t}$ is homotopy equivalent to the mapping space $\Map(S^2, BG_c) = \Bun_{G_c}(S^2)$, and that if $F$ is a finite group, then $\Bun_F(S^2) = BF$. 
\end{remark}

The work of Hopkins-Kuhn-Ravenel in fact proves a statement which is much more general than \cref{thm: hkr} (and similarly, Lurie's work in \cite{elliptic-iii} yields a much stronger statement than \cref{thm: lurie tempered}). Namely, they prove the following.
\begin{theorem}[Hopkins-Kuhn-Ravenel]\label{thm: upgraded hkr}
    Let $F$ be a finite group, and let $X$ be a finite $F$-space. For each homomorphism $\alpha: \Z_p^n \to F$, let $X^\alpha$ denote the fixed locus of $\im(\alpha)$. Then there is an isomorphism
    $$k^\ast(X_{hF}) \otimes_{\pi_0(k)} \cc_p \xrightarrow{\cong} \H^\ast\left(\left(\sqcup_{\alpha\in \Hom(\Z_p^n, F)} X^\alpha\right)\mmod F;  \cc_p[u^{\pm 1}]\right).$$
\end{theorem}
The isomorphism of \cref{thm: hkr} is the special case when $X$ is a point. In \cite{elliptic-iii}, Lurie shows that \cref{thm: upgraded hkr} is a consequence of a more general statement. If $X$ is a finite $F$-space, Lurie defines an $\infty$-category $\Loc_F(X; k)$ (denoted by $\mathrm{LocSys}_\GG(X//F)$ in \textit{loc. cit.}), and proves the following as \cite[Theorem 6.4.1]{elliptic-iii}:
\begin{theorem}[Lurie]\label{thm: upgraded lurie tempered}
    There is a symmetric monoidal fully faithful embedding
    $$\Loc_F(X; k) \otimes_k \cc_p[u^{\pm 1}] \hookrightarrow \Loc(\Map(BT_p^n, X_{hF}); \cc_p[u^{\pm 1}]).$$
\end{theorem}
The essential image of the above embedding is described in \cite[Theorem 6.5.13]{elliptic-iii}. For the reader interested in chasing down references: specifically, \cref{thm: upgraded lurie tempered} generalizes \cite[Theorem 4.3.2]{elliptic-iii}; the latter implies \cref{thm: upgraded hkr} by \cite[Corollary 4.3.4]{elliptic-iii}. The basic observation is that the mapping space $\Map(BT_p^n, X_{hF})$ is equivalent to $\left(\sqcup_{\alpha\in \Hom(\Z_p^n, F)} X^\alpha\right)_{hF}$. Note that the homotopy quotient $\Hom(\Z_p^n, F)_{hF}$ can be written as a disjoint union $\sqcup_{[\alpha]} BZ(\alpha)$ ranging over conjugacy classes of homomorphisms $\alpha: \Z_p^n \to F$; here $Z(\alpha)$ denotes the centralizer of the image of $\alpha$. Similarly, the homotopy orbits $\left(\sqcup_{\alpha\in \Hom(\Z_p^n, F)} X^\alpha\right)_{hF}$ can be rewritten as the disjoint union $\sqcup_{[\alpha]} X^\alpha_{hZ(\alpha)}$.

\begin{remark}
    One could contemplate a variant of \cref{thm: upgraded hkr} and \cref{thm: upgraded lurie tempered} which replaces $\cc_p$ by other $\Eoo$-$k$-algebras (e.g., over which the base-change of $\hat{\GG}_0$ is not necessarily isomorphic to $(\QQ_p/\Z_p)^n$, but over which it has $(\QQ_p/\Z_p)^j$ as a summand for some $j<n$). The analogues of \cref{thm: upgraded hkr} and \cref{thm: upgraded lurie tempered} in this generality were proved in \cite{stapleton-hkr, stapleton-2} and \cite{elliptic-iii}. 
\end{remark}

Given the analogy between \cref{thm: lurie tempered} and \cref{thm: intro omnibus}, it is natural to ask for an analogue of \cref{thm: upgraded lurie tempered} for connected compact Lie groups. In the following discussion, we suggest an analogy: namely, one could view the $k$-theoretic variant (described for $k = \ku$ in \cite{ku-rel-langlands}) of the local unramified relative Langlands conjecture of \cite{bzsv} as an analogue of the aforementioned results.

To understand this, let us again massage \cref{thm: upgraded hkr} and \cref{thm: upgraded lurie tempered} to a form more suited to algebro-geometric considerations.  We will continue to assume for simplicity that $k^\ast(BF)$ is concentrated in even degrees.
\cref{thm: upgraded hkr} describes how, under the isomorphism of \cref{thm: hkr}, the $k^0(BF) \otimes_{\pi_0(k)} \cc_p$-module $k^\ast(X_{hF}) \otimes_{\pi_0(k)} \cc_p$ decomposes as a module over $\Gamma(\Map(\hat{\GG}_0^\vee, B\ul{F}); \co)$. Similarly, \cref{thm: upgraded lurie tempered} says that there is an explicit $\QCoh(\Map(\hat{\GG}_0^\vee, B\ul{F}))$-module category $\tilde{\cC}_X$ and a fully faithful $\Loc_F(\ast; k) \otimes_k \cc_p[u^{\pm 1}]$-linear embedding
$$\Loc_F(X; k) \otimes_k \cc_p[u^{\pm 1}] \hookrightarrow \tilde{\cC}_X \otimes_{\cc_p} \cc_p[u^{\pm 1}].$$ 
Note that one source of $\QCoh(\Map(\hat{\GG}_0^\vee, B\ul{F}))$-module categories are maps $\ld{L} \to \Map(\hat{\GG}_0^\vee, B\ul{F})$: namely, $\QCoh(\ld{L})$ is a $\QCoh(\Map(\hat{\GG}_0^\vee, B\ul{F}))$-module category. That is, one could imagine that $\tilde{\cC}_X$ is of the form $\QCoh(\ld{L})$ for some such $\ld{L}$ as above which is associated to $X$. (While one can give a somewhat \textit{ad hoc} definition of $\ld{L}$ in terms of the fixed point spaces $X^\alpha$ and their (co)homology\footnote{For instance, take $\ld{L}$ to be the stack $\sqcup_{[\alpha]} \spec(\H^\ast(X^\alpha; \cc_p))/\ul{Z(\alpha)}$. }, it should be rather interesting to intrinsically understand the algebro-geometric properties of $\ld{L}$ directly.)

More generally, recall that the data of a $k$-linear $\infty$-category with $F$-action is just a $\Fun(BF, \Mod_k)$-module category. Since $\Fun(BF, \Mod_k)$ is a completion of the $\infty$-category $\Loc_F(\ast; k)$, one might view the data of a $\Loc_F(\ast; k)$-module category $\cC$ as a decompletion of the notion of a $k$-linear $\infty$-category with $F$-action. One example of such a category is $\Loc_F(X; k)$ for a finite $F$-space $X$. If $\unit_\cC$ is a distinguished object of $\cC$, then $\End_\cC(\unit_\cC) \otimes_{\pi_0(k)} \cc_p$ is a $k^0(BF) \otimes_{\pi_0(k)} \cc_p$-module, and hence a $\Gamma(\Map(\hat{\GG}_0^\vee, B\ul{F}); \co)$-module. One could now ask for a description of this module structure; when $\cC = \Loc_F(X; k)$ and $\unit_\cC$ is the constant sheaf therein, this is precisely answered by \cref{thm: upgraded hkr}.  Similarly, one could ask for an analogue of \cref{thm: upgraded lurie tempered} in this generalized context.
Summarizing, both \cref{thm: upgraded hkr} and \cref{thm: upgraded lurie tempered} can be understood as describing how a $\Loc_F(\ast; k)$-module category decomposes over the mapping stack $\Map(\hat{\GG}_0^\vee, B\ul{F})$. 

Let $G_c$ be a connected, almost simple, simply-laced compact Lie group. Then, as discussed above, the analogue of $\Loc_F(\ast; k)$ is the $\infty$-category $\Loc_{\ld{G}_c}(\Gr_G; k)$. Moreover, the analogue of the tensor product on $\Loc_F(\ast; k)$ is the \textit{convolution} tensor product on $\Loc_{\ld{G}_c}(\Gr_G; k)$ coming, for instance, from the $\ld{G}_c$-equivariant $\E{2}$-space structure on $\Gr_G \cong \Omega G_c$. As mentioned in \cref{rmk: 1-shifted cartier}, the equivalence of \cref{eq: appendix 1-shifted cartier} is monoidal for the convolution tensor product on $\Loc_{\ld{G}_c}(\Gr_G; k)$ and the ordinary tensor product of quasicoherent sheaves on $\Bun_{\ld{G}}^\ss(\GG_0^\vee)^\reg$.

Based on the discussion above, one can interpret the following question as an analogue of \cref{thm: upgraded hkr} and \cref{thm: upgraded lurie tempered}: how does a $\Loc_F(\ast; k)$-module category decompose over $\Bun_{\ld{G}}^\ss(\GG_0^\vee)^\reg$? More precisely, any finite $G_c$-space $X$ should:
\begin{enumerate}
    \item define a $\Loc_{\ld{G}_c}(\Gr_G; k)$-module category $\cC_X$; this is the analogue of the $\Loc_F(\ast; k)$-module category $\Loc_F(X; k)$.
    \item define a fully faithful embedding $\cC_X \hookrightarrow \tilde{\cC}_X$ into an explicit $\QCoh(\Bun_{\ld{G}}^\ss(\GG_0^\vee))$-module category $\tilde{\cC}_X$; this is the analogue of the fully faithful embedding $\Loc_F(X; k) \hookrightarrow \bigoplus_{[\alpha]} \Loc(X^\alpha_{hZ(\alpha)}; \cc_p)$ from \cref{thm: upgraded lurie tempered}.
\end{enumerate}

In the following discussion, we will quietly replace $\Loc_{\ld{G}_c}(\Gr_G; k)$ by $\Loc_{G_c}(\Gr_G; k)$ for conceptual simplicity; this, of course, changes the quasicoherent side, but to avoid getting into more detail than is necessary, we will pretend that the dual side remains unchanged\footnote{If $k = \QQ[u^{\pm 1}]$ and $\GG = \GG_a$, the object $\Bun_{\ld{G}}^\ss(\GG_0^\vee) = \ld{\g}/\ld{G}$ must be replaced by $\ld{\g}^\ast/\ld{G} = \g/\ld{G}$; and similarly, if $k = \KU$ and $\GG = \GG_m$, the object $\Bun_{\ld{G}}^\ss(\GG_0^\vee) = \ld{G}/\ld{G}$ must be replaced by $G/\ld{G}$.}. To describe a candidate for $\cC_X$, recall that the quotient $\Gr_G/G\pw{t}$ is homotopy equivalent to the mapping space $\Map(S^2, BG_c) = \Bun_{G_c}(S^2)$. This, in turn, can be described as the double coset stack $G_c\backslash (LG_c)/G_c$, where $LG_c$ denotes the (topological) free loop space of $G_c$.  Any $G_c$-space $X$ defines an $LG_c$-space $LX$, and the stack $G_c\backslash (LG_c)/G_c$ acts on $(LX)/G_c$ by convolution. That is, the $\infty$-category $\Loc_{G_c}(\Gr_G; k)$ with its convolution tensor product acts on $\Loc_{G_c}(LX; k)$. One could therefore regard the latter category as a candidate for $\cC_X$, and further ask for the following strengthening of (a) and (b) above:
\begin{itemize}
    \item there should be a stack $\ld{L}^\reg$ equipped with a map $\ld{L}^\reg \to \Bun_{\ld{G}}^\ss(\GG_0^\vee)^\reg$ such that there is an equivalence
    $$\cC_X = \Loc_{G_c}(LX; k) \simeq \QCoh(\ld{L}^\reg).$$
    \item the stack $\ld{L}^\reg$ should be an open substack of a larger stack $\ld{L}$, and the map $\ld{L}^\reg \to \Bun_{\ld{G}}^\ss(\GG_0^\vee)^\reg$ extends to a map $\ld{L} \to \Bun_{\ld{G}}^\ss(\GG_0^\vee)$. This gives a fully faithful embedding 
    $$\cC_X \hookrightarrow \tilde{\cC}_X := \QCoh(\ld{L}).$$
\end{itemize}
Note that $\Bun_{\ld{G}}^\ss(\GG_0^\vee)$ is the quotient of $\Bun_{\ld{G}}^\ss(\GG_0^\vee)_\triv$ by $\ld{G}$, so one could equivalently view $\ld{L}$ as the data of a $\ld{G}$-stack $\ld{M}$ equipped with a $\ld{G}$-equivariant map 
$$\mu: \ld{M} \to \Bun_{\ld{G}}^\ss(\GG_0^\vee)_\triv.$$
The relation between $\ld{L}$ and $\ld{M}$ is that $\ld{L} = \ld{M}/\ld{G}$. (There is more to say, regarding shifted symplectic structures \cite{ptvv}, but we refer the reader to \cite[Section 5.2]{ku-rel-langlands} for further discussion.)
\begin{example}
    If $k = \QQ[u^{\pm 1}]$ and $\GG_0 = \GG_a$, then $\ld{M}$ is simply a $\ld{G}$-stack equipped with a $\ld{G}$-equivariant map $\mu: \ld{M} \to \ld{\g}^\ast$. Similarly, if $k = \KU$ and $\GG_0 = \GG_m$, then $\ld{M}$ is simply a $\ld{G}$-stack equipped with a $\ld{G}$-equivariant map $\mu: \ld{M} \to G$.
\end{example}

Suppose $X$ is the analytification of an affine $G$-variety $X_\cc$. In \cite{bzsv}, Ben-Zvi--Sakellaridis--Venkatesh study (under certain additional conditions on $X_\cc$) the full $\infty$-category $\Shv_{G\pw{t}}(X_\cc\ls{t}; \cc)$ as a module over $\Shv_{G\pw{t}}(\Gr_G; \cc)$. The local unramified geometric conjecture of \cite{bzsv} (see \cite[Conjecture 7.5.1]{bzsv}) says -- up to the issue of shearing, which we will ignore here -- that associated to $X_\cc$ is a Hamiltonian $\ld{G}$-stack $\ld{M}$ such that there is an equivalence of categories $\Shv_{G\pw{t}}(X_\cc\ls{t}; \cc) \simeq \QCoh(\ld{M}/\ld{G})$. The data of a Hamiltonian $\ld{G}$-structure on $\ld{M}$ gives, in particular, an $\ld{G}$-equivariant moment map $\ld{M} \to \ld{\g}^\ast$ which makes $\QCoh(\ld{M}/\ld{G})$ into a $\QCoh(\ld{\g}^\ast/\ld{G})$-module category. Moreover, under certain assumptions on $X_\cc$, there is a fully faithful embedding $\Loc_{G_c}(LX; \cc) \hookrightarrow \Shv_{G\pw{t}}(X_\cc\ls{t}; \cc)$. Putting this together, we find a picture exactly like the one described in the preceding paragraph: namely, assuming \cite[Conjecture 7.5.1]{bzsv}, there is a fully faithful embedding
$$\Loc_{G_c}(LX; \cc) \hookrightarrow \Shv_{G\pw{t}}(X_\cc\ls{t}; \cc) \simeq \QCoh(\ld{M}/\ld{G})$$
of $\Loc_{G_c}(LX; \cc)$ into an explicit $\QCoh(\ld{\g}^\ast/\ld{G})$-module category. Therefore, one could view (the $2$-periodification of) \cite[Conjecture 7.5.1]{bzsv} as a conjectural analogue for connected compact Lie groups and $k = \cc[u^{\pm 1}]$ of \cref{thm: upgraded hkr} and \cref{thm: upgraded lurie tempered}.\footnote{Of course, since $F$ is a finite group, \cref{thm: upgraded hkr} and \cref{thm: upgraded lurie tempered} are contentless if $k = \cc[u^{\pm 1}]$; so what we mean by the analogy between \cite[Conjecture 7.5.1]{bzsv} and \cref{thm: upgraded lurie tempered} is that the latter admits a conjectural generalization to connected compact Lie groups, and that the resulting statement specialized to $k = \cc[u^{\pm 1}]$ is still interesting and bears analogy to \cite[Conjecture 7.5.1]{bzsv}.} Motivated by this discussion, we propose in \cite{ku-rel-langlands} that there should be a variant of \cite[Conjecture 7.5.1]{bzsv} for sheaves with coefficients in other $\Eoo$-rings (like connective complex K-theory $\ku$ or elliptic cohomology).