In \cite{brylinski-zhang}, Brylinski-Zhang compute the $G_c$-equivariant complex K-theory of $G_c$ for a connected compact Lie group $G_c$ with torsion-free fundamental group as the ring $\Omega^\ast_{\mathrm{RU}(G)/\Z} = \Omega^\ast_{T\mmod W/\Z}$ of K\"ahler differentials on the complex representation ring of $G$. Our goal in this section is to describe the relationship between this calculation and (the proof of) \cref{thm: intro omnibus}. 

We begin by stating an obvious corollary of \cref{thm: intro omnibus}. Recall that if $\GG_0$ is either $\GG_a$, $\GG_m$, or an elliptic curve $E$, and $\cM_{T,0} = \Hom(\bX^\ast(T), \GG_0)$, there is a Kostant section $\kappa: \cM_{T,0} \to \Bun_{\ld{B}}^0(\GG_0^\vee)$ as described in \cref{def: additive kostant slice}, \cref{def: mult kostant slice}, and \cref{elliptic-kostant}. Recall that $F$ is an algebraically closed field of characteristic zero containing $\pi_0(k)$.
\begin{theorem}\label{thm: T-equiv loc on G}
    Let $G$ be a connected almost simple simply-laced group. Let $k$ denote either $\QQ[u^{\pm 1}]$, $\KU$, or elliptic cohomology, and let $\GG_0$ be either $\GG_a$, $\GG_m$, or an elliptic curve $E$ over $\pi_0(k)$, respectively.
    Then there is an equivalence 
    $$\Loc_{\ld{T}_c}^\gr(G_c; k) \otimes_{\pi_0(k)} F \simeq \QCoh(\cM_{\ld{T},0} \times_{\Bun_{\ld{B}}^0(\GG_0^\vee)} \cM_{\ld{T},0}),$$
    where the right-hand side denotes the self-intersection of the Kostant slice.
\end{theorem}
\begin{proof}
    Recall from \cref{def: graded Loc} that
    $$\Loc_{\ld{T}_c}^\gr(G_c; k) = \LMod_{\pi_0(\cf_{\ld{T}}(\Gr_G)^\vee)}(\QCoh(\cM_{\ld{T},0})).$$
    In \cref{thm: ordinary hmlgy reg centr}, \cref{thm: ku hmlgy reg centr}, and \cref{thm: elliptic hmlgy reg centr}, we showed that $\spec_{\cM_{\ld{T},0}}(\pi_0(\cf_{\ld{T}}(\Gr_G)^\vee))$ is isomorphic to the self-intersection $\cM_{\ld{T},0} \times_{\Bun_{\ld{B}}^0(\GG_0^\vee)} \cM_{\ld{T},0}$, so the desired equivalence follows.
\end{proof}
In the same way, if $G$ is further assumed to have torsion-free fundamental group, and $\cM_{G,0}$ denotes the moduli \textit{space} of semistable $G$-bundles on $\GG_0^\vee$, there is a Kostant section $\kappa: \cM_{G,0} \to \Bun_{\ld{G}}^\ss(\GG_0^\vee)$. In the additive and multiplicative cases, this follows from \cref{def: additive kostant slice}, \cref{def: mult kostant slice}, and in the elliptic case, it can be deduced from \cite{davis-elliptic-springer} as in \cref{elliptic-kostant}. Just as in \cref{thm: T-equiv loc on G}, there is an equivalence 
\begin{equation}\label{eq: G-equiv loc on G}
    \Loc_{\ld{G}_c}^\gr(G_c; k) \otimes_{\pi_0(k)} F \simeq \QCoh(\cM_{\ld{G},0} \times_{\Bun_{\ld{G}}^\ss(\GG_0^\vee)} \cM_{\ld{G},0})
\end{equation}
where the right-hand side denotes the self-intersection of the Kostant slice. Under this equivalence, the ``constant sheaf'' in $\Loc_{\ld{G}_c}^\gr(G_c; k)$ is sent to the pushforward of the structure sheaf under the relative diagonal
$$\delta: \cM_{\ld{G},0} \to \cM_{\ld{G},0} \times_{\Bun_{\ld{G}}^\ss(\GG_0^\vee)} \cM_{\ld{G},0}.$$

In the remainder of this section, we will explain how \cref{eq: G-equiv loc on G} implies the calculation of \cite{brylinski-zhang}, as well as the relationship to the Hochschild-Kostant-Rosenberg theorem. (This, of course, is a triple of authors distinct from Hopkins-Kuhn-Ravenel with initials ``HKR''!)
For simplicity, we will only focus on the case when $k$ is $\QQ[u^{\pm 1}]$ or $\KU$ (so $\GG_0$ is either $\GG_a$ or $\GG_m$, and $\Bun_{\ld{G}}^\ss(\GG_0^\vee)$ is either $\ld{\g}/\ld{G}$ or $\ld{G}/\ld{G}$). With a little bit of elbow grease, one can show that most of the results below continue to work for elliptic cohomology, too.

Recall that $\Loc_{{G}_c}^\gr(G_c; k)$ is intended to be an approximation to a $k$-linear $\infty$-category of $G_c$-equivariant local systems on $G_c$. The algebra of endomorphisms of the constant sheaf in this $\infty$-category is given by the equivariant cochains $\cf_G(G_c)$. This is a quasicoherent sheaf over the spectral $k$-scheme $\cM_G$, and it can be described explicitly as follows. If $\fr{X}_k$ is a spectral prestack over $k$, let $\cL \fr{X}_k$ denote the free loop space of $\fr{X}_k$, i.e., the mapping prestack $\Map(B\Z, \fr{X}_k)$. Here, $\Z$ is viewed as a constant stack over $k$. The global sections of the structure sheaf of $\cL \fr{X}_k$ computes the Hochschild homology $\HH(\fr{X}_k/k)$.
\begin{prop}\label{prop: G-equiv coh of G and HH}
    Assume (for simplicity) that $k$ is either $\QQ[u^{\pm 1}]$ or $\KU$. If $G$ is connected, then there is an isomorphism of spectral $k$-schemes
    $$\spec_{\cM_G}(\cf_G(G_c)) \cong \cL \cM_G.$$
    In particular, there is an isomorphism of $\Eoo$-$k_{G_c}$-algebras
    $$\cf_G(G_c) \cong \HH(\cM_G/k).$$
\end{prop}
\begin{proof}
    Recall that $B\Z$ is isomorphic to the constant $k$-stack $S^1$, which can be written as the pushout $\ast \sqcup_{\ast \sqcup \ast} \ast$. Therefore, since $\cM_G = \spec k_{G_c}$ is affine (because $k$ is either $\QQ[u^{\pm 1}]$ or $\KU$), we may wite $\cL \cM_G = \spec (k_{G_c} \otimes_{k_{G_c} \otimes_k k_{G_c}} k_{G_c})$. Since the functor $\cf_G: \Top(G_c)^\op \to \Mod_{k_{G_c}}$ sends finite products of connected finite $G$-spaces to tensor products, we find that 
    $$k_{G_c} \otimes_{k_{G_c} \otimes_k k_{G_c}} k_{G_c} \cong \cf_G(\ast) \otimes_{\cf_{G \times G}(\ast)} \cf_G(\ast) \cong \cf_G(G_c),$$
    since there is an isomorphism of orbispaces
    $$\ast/G_c\times_{\ast/(G_c \times G_c)} \ast/G_c \cong G_c/G_c.\qedhere$$
\end{proof}
\begin{remark}\label{rmk: equiv coh of G mod K and HH}
    The approach of \cref{prop: G-equiv coh of G and HH} can be used to compute the equivariant cohomology $\cf_G(\Omega G_c)$, too. Namely, observe that there is an isomorphism of orbispaces
    $$\ast/G_c\times_{\ast/G_c\times_{\ast/(G_c \times G_c)} \ast/G_c} \ast/G_c \cong (\Omega G_c)/G_c,$$
    so that there is an isomorphism
    $$\cf_G(\Omega G_c) = k_{G_c} \otimes_{k_{G_c} \otimes_{k_{G_c} \otimes_k k_{G_c}} k_{G_c}} k_{G_c}.$$
    The right-hand side can be expressed more succinctly as the factorization homology $\int_{S^2}(k_{G_c}/k)$.

    More generally, observe that if $K_c \subseteq G_c$ is a closed subgroup such that $G_c/K_c$ is a finite $K_c$-space (where $K_c$ acts on the left by multiplication), and $L(G_c/K_c)$ denotes the (topological) free loop space of $G_c/K_c$, then
    $$G_c\backslash L(G_c/K_c) \simeq K_c \backslash \Omega(G_c/K_c) \simeq (\ast \times_{\ast \times_{\ast/G_c} \ast/K_c} \ast)/K_c \simeq \ast/K_c \times_{\ast/K_c \times_{\ast/G_c} \ast/K_c} \ast/K_c.$$
    It follows that there is an isomorphism
    $$\cf_G(\cL(G_c/K_c)) = k_{K_c} \otimes_{k_{K_c} \otimes_{k_{G_c}} k_{K_c}} k_{K_c}.$$
    The right-hand side can be expressed more succinctly as the relative Hochschild homology $\HH(\cM_K/\cM_G)$, so that there is an isomorphism of spectral $k$-schemes
    $$\spec_{\cM_G}(\cf_G(G_c/K_c)) \cong \cL(\cM_K/\cM_G) \cong \cL(\cM_K) \times_{\cL(\cM_G)} \cM_G.$$
    The discussion above computing $\cf_K(\Omega K_c)$ is the special case of the above calculation when $G_c = K_c \times K_c$, with $K_c$ embedded diagonally.
\end{remark}
\begin{example}
    Let $k = \QQ[u^{\pm 1}]$. Then the preceding discussion shows that $C^\ast_{G_c}(\Omega G_c; \QQ[u^{\pm 1}])$ is isomorphic to the factorization homology $\int_{S^2}(k_{G_c}/k) = \HH(k_{G_c}/k_{G_c} \otimes_k k_{G_c})$. The latter has a Hochschild-Kostant-Rosenberg filtration whose associated graded is given by the $2$-periodification $L\Omega^\ast_{\fr{t}\mmod W/(\fr{t}\mmod W \times_{\spec \QQ} \fr{t}\mmod W)}[u^{\pm 1}]$ of the derived Hodge complex of $\fr{t}\mmod W$ embedded diagonally into $\fr{t}\mmod W \times_{\spec \QQ} \fr{t}\mmod W$.
    Since we are working rationally, the Hochschild-Kostant-Rosenberg filtration splits, and so there is an isomorphism
    $$\int_{S^2}(k_{G_c}/k) \cong L\Omega^\ast_{\fr{t}\mmod W/(\fr{t}\mmod W \times_{\spec \QQ} \fr{t}\mmod W)}[u^{\pm 1}].$$
    Note that if $X$ (like $\fr{t}\mmod W$) is an affine space over a commutative ring $R$, then $L\Omega^\ast_{X/(X \times_{\spec(R)} X)} \cong \Gamma^\ast(\Omega^1_{X/R})$; so the above isomorphism could instead be stated as
    $$\int_{S^2}(k_{G_c}/k) \cong \Sym_{\co_{\fr{t}\mmod W}}(\Omega^1_{\fr{t}\mmod W})[u^{\pm 1}] = \co_{T(\fr{t}\mmod W)}[u^{\pm 1}],$$
    where $T(\fr{t}\mmod W)$ is the tangent bundle of $\fr{t}\mmod W$. It follows that there is an isomorphism
    $$\spec C^\ast_{G_c}(\Omega G_c; \QQ[u^{\pm 1}]) \cong T(\fr{t}\mmod W) \times_{\spec(\QQ)} \spec(\QQ[u^{\pm 1}]).$$
    This recovers the $\hbar = 0$ case of \cite[Theorem 1]{bf-derived-satake}. The case with loop-rotation equivariance included, i.e., when $\hbar$ need not be zero, follows from \cref{lem: hochschild and def to nc} below, which recovers the description of $\spec C^\ast_{G_c \times S^1_\rot}(\Omega G_c; \QQ[u^{\pm 1}])$ as the deformation to the normal cone of the diagonal embedding $\fr{t}\mmod W \hookrightarrow \fr{t}\mmod W \times \fr{t}\mmod W$.
\end{example}
The following statement is essentially Koszul dual to the usual Hochschild-Kostant-Rosenberg theorem describing Hochschild homology with its circle action via the de Rham complex:
\begin{lemma}\label{lem: hochschild and def to nc}
    Let $X = \spec(A)$ be a smooth affine scheme over $\QQ$, and let $\Def_\hbar^\Delta(X)$ denote the deformation to the normal cone of the diagonal embedding $X \hookrightarrow X \times X$. Then there is an isomorphism
    $$\spec \pi_\ast \left(\int_{S^2}(A/\QQ)\right)^{hS^1} \cong \Def_\hbar^\Delta(X)$$
    of $\pi_\ast \QQ^{hS^1} \cong \QQ\pw{\hbar}$-algebras.
\end{lemma}
\begin{proof}
    By standard arguments, it suffices to check the claim when $A$ is a finitely generated polynomial algebra. Let us demonstrate the claim when $A$ is a polynomial algebra on a single class; an easy modification of this argument will prove the claim in general when $A = \QQ[V]$ for some finite-dimensional $\QQ$-vector space $V$.
    Let us identify $\QQ[x] \otimes \QQ[x] = \QQ[x,y]$, so that the standard resolution of $\QQ[x]$ as a $\QQ[x,y]$-algebra identifies 
    $$\QQ[x] \otimes_{\QQ[x] \otimes \QQ[x]} \QQ[x] \cong \QQ[x, \sigma(x-y)]/(\sigma(x-y)^2),$$
    where $\sigma(x-y)$ is in degree $1$. This implies that 
    $$\int_{S^2}(A/\QQ) \simeq \QQ[x] \otimes_{\QQ[x] \otimes_{\QQ[x] \otimes \QQ[x]} \QQ[x]} \QQ[x] \cong \QQ[x, \sigma^2(x-y)],$$
    with $\sigma^2(x-y)$ in degree $2$. Note that $\int_{S^2}(A/\QQ)$ is an $S^1$-equivariant $\Eoo$-$\QQ[x,y]$-algebra, so that $\pi_\ast\left(\int_{S^2}(A/\QQ)\right)^{hS^1}$ is an $\Eoo$-$\QQ[x,y]$-algebra; let us now determine this algebra structure. Since this ring is concentrated in even degrees, the homotopy fixed point spectral sequence computing $\pi_\ast\left(\int_{S^2}(A/\QQ)\right)^{hS^1}$ degenerates, and we find that $\pi_\ast\left(\int_{S^2}(A/\QQ)\right)^{hS^1} \cong \QQ\pw{\hbar}[x, \sigma^2(x-y)]$ with $\sigma^2(x-y)$ in weight $2$ and $\hbar$ in weight $-2$. The $\QQ[x,y]$-algebra structure is given by the observation that 
    $$x - y = \hbar \sigma^2(x-y);$$
    this relation is true for abstract reasons (as explained, for instance, in \cite[Appendix A]{hahn-wilson-bpn}). It follows that there is an isomorphism
    $$\pi_\ast\left(\int_{S^2}(A/\QQ)\right)^{hS^1} \cong \QQ\pw{\hbar}[x, y, \tfrac{x-y}{\hbar}]$$
    of $\QQ[x,y]$-algebras. The spectrum of the right-hand side identifies with $\Def_\hbar^\Delta(\AA^1)$, as desired.
\end{proof}
\begin{remark}
    More generally, suppose $A$ is a commutative $\Z$-algebra, and let $X = \spec(A)$. Let $\Def_\hbar^\Delta(X)$ denote the divided power deformation to the normal cone of the diagonal embedding $X \hookrightarrow X \times X$, so that its fiber over $\hbar = 0$ is the PD-hull $T_X^\sharp$ of the (derived) tangent bundle of $X$. Then there is a filtration on $\left(\int_{S^2}(A/\Z)\right)^{hS^1}$ whose associated graded is given by $\co_{\Def_\hbar^\Delta(X)}$. The proof is exactly as in \cref{lem: hochschild and def to nc}: the desired filtration is given by left Kan extending the Postnikov filtration $\tau_{\geq \star} \left(\int_{S^2}(A/\Z)\right)^{hS^1}$ from polynomial $\Z$-algebras to all commutative $\Z$-algebras.
\end{remark}

Let us now discuss the relationship between \cref{prop: G-equiv coh of G and HH} and \cref{eq: G-equiv loc on G}. Although the cases $k = \QQ[u^{\pm 1}]$ and $k = \KU$ can be treated simultaneously, we will present the discussion separately for both for the sake of clarity. The upshot of this discussion is that the approximation to $\cf_G(G_c)$ afforded by the degeneration of $\Loc_{G_c}(G_c; k)$ to $\Loc_{{G}_c}^\gr(G_c; k)$ identifies, under \cref{prop: G-equiv coh of G and HH} and \cref{eq: G-equiv loc on G}, with the Hochschild-Kostant-Rosenberg spectral approximation of $\pi_\ast \HH(\cM_G/k)$ by $\Omega^\ast_{\cM_{G,0}/\pi_0(k)}$.
\begin{lemma}\label{lem: endomorphisms of delta sheaf}
    Let $H$ be a smooth affine group scheme over an affine scheme $S = \spec(R)$, let $\delta: S \to H$ denote the zero section, and let $\fr{h}$ denote its Lie algebra (viewed as a vector bundle over $S$). Then $\End_{\QCoh(H)}(\delta_\ast \co_S)$ has a filtration whose associated graded is isomorphic to $\co_{\fr{h}^\ast[1]}$. If $R$ is a $\QQ$-algebra, this filtration splits.
\end{lemma}
\begin{proof}
    The endomorphism algebra $\End_{\QCoh(H)}(\delta_\ast \co_S)$ is isomorphic to the $R$-linear dual of $\co_{S \times_H S}$. The derived scheme $S \times_H S$ depends only on the formal completion $\hat{H}$. Note that $\hat{H}$ admits a filtration (coming from powers of the ideal sheaf of the zero section of $\hat{H}$) whose associated graded is isomorphic to $\hat{\fr{h}}$; furthermore, the exponential map defines a splitting of this filtration when $R$ is a $\QQ$-algebra. This defines a filtration on $S \times_H S$ whose associated graded is isomorphic to $S \times_{\fr{h}} S = \fr{h}[-1]$. Therefore, the $R$-linear dual of $\co_{S \times_H S}$ is isomorphic to $\co_{\fr{h}^\ast[1]}$.
\end{proof}
\begin{example}\label{ex: rational coh of G mod G}
    Suppose $k = \QQ[u^{\pm 1}]$, and let $\ld{J} = \fr{t}\mmod W \times_{\ld{\g}^\ast/\ld{G}} \fr{t}\mmod W$. Then \cref{eq: G-equiv loc on G} states that there is an equivalence 
    $$\Loc_{{G}_c}^\gr(G_c; k) \otimes_{\QQ} F \simeq \QCoh(\ld{J}),$$
    and the ``constant sheaf'' $\ul{k}^\gr$ in $\Loc_{\ld{G}_c}^\gr(G_c; k)$ is sent to the pushforward of the structure sheaf under the identity section $\delta: \fr{t}\mmod W \to \ld{J}$. Taking endomorphisms, we find that
    $$\End_{\Loc_{{G}_c}^\gr(G_c; k)}(\ul{k}^\gr) \otimes_{\QQ} F \cong \End_{\QCoh(\ld{J})}(\delta_\ast \co_{\fr{t}\mmod W}).$$
    By \cref{lem: endomorphisms of delta sheaf}, the right-hand side admits a (split) filtration whose associated graded is isomorphic to the algebra of functions on $\Lie_{\fr{t}\mmod W}(\ld{J})^\ast[1]$. By \cite[Theorem 3.4.2]{riche}, one finds that the Lie algebra $\Lie_{\fr{t}\mmod W}(\ld{J})$ is isomorphic to the cotangent bundle $T^\ast(\fr{t}\mmod W)$, so that $\Lie_{\fr{t}\mmod W}(\ld{J})^\ast[1]$ is isomorphic to $T[1](\fr{t}\mmod W)$. Its ring of functions is precisely the Hodge cohomology $\Omega^\ast_{\fr{t}\mmod W/F} = \bigoplus (\Omega^i_{\fr{t}\mmod W/F})[-i]$ of $\fr{t}\mmod W$. Summarizing, we have found that there is an isomorphism
    $$\End_{\Loc_{{G}_c}^\gr(G_c; k)}(\ul{k}^\gr) \otimes_{\QQ} F \cong \Omega^\ast_{\fr{t}\mmod W/F}.$$
    
    On the other hand, it follows from the constructions in \cref{sec: degenerations} that there is a filtration on $\cf_G(G_c) = \End_{\Loc_{{G}_c}(G_c; k)}(\ul{k})$ whose associated graded is $\End_{\Loc_{{G}_c}^\gr(G_c; k)}(\ul{k}^\gr)[u^{\pm 1}]$. By the above discussion, the latter is $\Omega^\ast_{\fr{t}\mmod W/F}$. \cref{prop: G-equiv coh of G and HH} shows that $\cf_G(G_c) \otimes_k F[u^{\pm 1}]$ is isomorphic to the Hochschild homology $\HH(\fr{t}\mmod W/F)[u^{\pm 1}]$. There is therefore a filtration on $\HH(\fr{t}\mmod W/F)[u^{\pm 1}]$ whose associated graded is $\Omega^\ast_{\fr{t}\mmod W/F}[u^{\pm 1}]$. This filtration is precisely the Hochschild-Kostant-Rosenberg filtration on Hochschild homology (see, e.g., \cite{antieau-filtration-HP, raksit, toen-hkr} for modern references).
\end{example}
\begin{example}\label{ex: KU of G mod G}
    Suppose $k = \KU$, and assume $G$ is simply-laced and has torsion-free fundamental group. Let $\ld{J}_\mu = T\mmod W \times_{G/\ld{G}} T\mmod W$. Then \cref{eq: G-equiv loc on G} states that there is an equivalence 
    $$\Loc_{{G}_c}^\gr(G_c; \KU) \otimes_{\Z} F \simeq \QCoh(\ld{J}_\mu),$$
    and the ``constant sheaf'' $\ul{\KU}^\gr$ in $\Loc_{\ld{G}_c}^\gr(G_c; \KU)$ is sent to the pushforward of the structure sheaf under the identity section $\delta: T\mmod W \to \ld{J}_\mu$. Taking endomorphisms, we find that
    $$\End_{\Loc_{{G}_c}^\gr(G_c; \KU)}(\ul{\KU}^\gr) \otimes_\Z F \cong \End_{\QCoh(\ld{J}_\mu)}(\delta_\ast \co_{T\mmod W}).$$
    By \cref{lem: endomorphisms of delta sheaf}, the right-hand side admits a (split) filtration whose associated graded is isomorphic to the algebra of functions on $\Lie_{T\mmod W}(\ld{J}_\mu)^\ast[1]$. There is a multiplicative analogue of \cite[Theorem 3.4.2]{riche} which states the Lie algebra $\Lie_{T\mmod W}(\ld{J}_\mu)$ is isomorphic to the cotangent bundle $T^\ast(T\mmod W)$. In particular, $\Lie_{T\mmod W}(\ld{J}_\mu)^\ast[1]$ is isomorphic to $T[1](T\mmod W)$. Its ring of functions is precisely the Hodge cohomology $\Omega^\ast_{T\mmod W/F} = \bigoplus (\Omega^i_{T\mmod W/F})[-i]$ of $T\mmod W$. Summarizing, we have found that there is an isomorphism
    $$\End_{\Loc_{{G}_c}^\gr(G_c; k)}(\ul{k}^\gr) \otimes_{\Z} F \cong \Omega^\ast_{T\mmod W/F}.$$

    On the other hand, it follows from the constructions in \cref{sec: degenerations} that there is a filtration on $\cf_G(G_c) = \End_{\Loc_{{G}_c}(G_c; \KU)}(\KU)$ with associated graded given by $\End_{\Loc_{{G}_c}^\gr(G_c; \KU)}(\ul{\KU}^\gr)[u^{\pm 1}]$. By the above discussion, the latter is $\Omega^\ast_{T\mmod W/F}$. \cref{prop: G-equiv coh of G and HH} shows that $\cf_G(G_c) \otimes_{\KU} F[u^{\pm 1}]$ is isomorphic to the Hochschild homology $\HH(T\mmod W/F)[u^{\pm 1}]$. There is therefore a filtration on $\HH(T\mmod W/F)[u^{\pm 1}]$ whose associated graded is $\Omega^\ast_{T\mmod W/F}[u^{\pm 1}]$. Again, this filtration is precisely the Hochschild-Kostant-Rosenberg filtration on Hochschild homology.
\end{example}
\begin{remark}
    While we are on the topic of the equivariant K-theory of $G_c$, let us note the relationship between \cref{eq: G-equiv loc on G} and the work of Freed-Hopkins-Teleman \cite{fht-i, fht-ii, fht-iii, fht-complex, fht-categorified}.\footnote{Nearly the same perspective can also be found in some of Teleman's talks; e.g., \cite{teleman-slides}.} We will be brief, since we will not use these results below. Associated to a class $\tau \in \H^4(BG_c; \Z)$ is the ``twisted equivariant K-homology'' $\KU^G_\tau(G_c)$. When $\tau$ is sufficiently nondegenerete, Freed-Hopkins-Teleman computed that $\pi_\ast \KU^G_\tau(G_c)$ is isomorphic to $\RU(G)/I^\tau$ for a particular ideal $I^\tau$ (called the ``Verlinde ideal''). The categorification of this isomorphism from \cite{fht-categorified} shows that, associated to $\tau$, there is a map $f: T\mmod W \cong \spec \pi_0 \KU_G \to \AA^1$ such that (under certain hypotheses on $\tau$), there is an isomorphism between $\pi_\ast \KU^G_\tau(G_c) \otimes_\Z F$ and the Jacobian ring of $f$.
    
    This is related to \cref{eq: G-equiv loc on G} in the following manner. Below, we will implicitly base-change all rings from $\Z$ to $F$, to avoid cumbersome notation. Recall from \cite[Equation 3]{fht-i} that there is a spectral sequence
    \begin{equation}\label{eq: twisted k-theory sseq}
        E_1^{\ast,\ast} \cong \pi_\ast \KU_G \otimes_{\pi_\ast \cf_G(\Gr_G)^\vee} \pi_\ast \KU_G \Rightarrow \pi_\ast \KU^G_\tau(G_c).
    \end{equation}
    The tensor product is derived; moreover, the class $\tau$ defines a particular $\pi_\ast \cf_G(\Gr_G)^\vee$-module structure on $\pi_\ast \KU_G$, and one of the tensor factors is given this module structure. (The other tensor factor is given the module structure coming from the augmentation.) Using \cref{thm: ku hmlgy reg centr}, let us view $\spec \pi_\ast \cf_G(\Gr_G)^\vee$ as the ($2$-periodification of) $\ld{J}_\mu$. Then $\tau$ defines a particular closed subscheme $T\mmod W \cong L_\tau \hookrightarrow \ld{J}_\mu$ (which is in fact a Lagrangian), and the $E_1$-page of this spectral sequence can be identified with (the $2$-periodification of) the ring of functions on $L_\tau \times_{\ld{J}_\mu} T\mmod W$. If $L_\tau$ lies in the formal neighborhood of $\ld{J}_\mu$, then we may replace $\ld{J}_\mu$ in this fiber product by its formal completion $\hat{\ld{J}}_\mu$ at the zero section. Since we have implicitly base-changed everything to the characteristic zero field $F$, the argument of \cref{lem: endomorphisms of delta sheaf} further lets us replace $\hat{\ld{J}}_\mu$ by its Lie algebra, which (as mentioned in \cref{ex: KU of G mod G}) is given by $T^\ast(T\mmod W)$. Under this replacement, the map $T\mmod W \cong L_\tau \to \hat{\ld{J}}_\mu$ becomes identified with the map $df: T\mmod W \to T^\ast(T\mmod W)$, where $f: T\mmod W \to \AA^1$ is the map from \cite{fht-categorified}. The derived fiber product $L_\tau \times_{T^\ast(T\mmod W)} T\mmod W$ is precisely the Jacobian ring of $f$; that is to say, the $E_1$-page of the spectral sequence \cref{eq: twisted k-theory sseq} identifies with the Jacobian ring of $f$. If the spectral sequence \cref{eq: twisted k-theory sseq} degenerates at the $E_1$-page, then we conclude that $\pi_\ast \KU^G_\tau(G_c)$ is isomorphic to the Jacobian ring of $f$, as desired.
\end{remark}

In fact, the Hochschild-Kostant-Rosenberg filtrations on $\HH(\fr{t}\mmod W/F)$ and $\HH(T\mmod W/F)$ from \cref{ex: rational coh of G mod G} and \cref{ex: KU of G mod G} both split, since $F$ is of characteristic zero and $\fr{t}\mmod W$ and $T\mmod W$ are smooth schemes. We therefore conclude that there are isomorphisms
\begin{align*}
    \H^\ast_{G_c}(G_c; F[u^{\pm 1}]) & \cong \Omega^\ast_{\fr{t}\mmod W/F}[u^{\pm 1}], \\
    \KU^\ast_{G_c}(G_c) \otimes_\Z F & \cong \Omega^\ast_{T\mmod W/F}[u^{\pm 1}],
\end{align*}
the latter for $G_c$ being simply-laced. (This assumption can be removed with further work.)
The final isomorphism above recovers (the base-change to $F$ of) the isomorphism of Brylinski-Zhang. Arguing as above, one also finds that if $k$ is an elliptic cohomology theory, $G_c$ is simply-laced and has torsion-free fundamental group, and $i: \spec(F[u^{\pm 1}]) \to \cM_G$ is a map with $F$ being an algebraically closed field of characteristic zero, there is an isomorphism of quasicoherent sheaves over $\spec(F[u^{\pm 1}])$:
\begin{equation}\label{eq: kahler diff and G mod G}
    \pi_\ast i^\ast \cf_G(G_c) \cong \Omega^\ast_{\cM_{G,0}/F}[u^{\pm 1}].
\end{equation}
As stated, \cref{eq: kahler diff and G mod G} holds if $k$ is $\QQ[u^{\pm 1}]$, complex K-theory, or elliptic cohomology. One could ask whether \cref{eq: kahler diff and G mod G} holds over the sphere spectrum.
\begin{remark}\label{rmk: miller splitting}
    At least in the case of classical groups, additive isomorphisms of the form discussed in this section follow from stronger statements about splittings of the suspension spectrum $(G_c)_+$. Such statements were proved in \cite{miller-stable-splittings}; let us illustrate this when $G = \GL_n$. For $j\leq n$, let $\Gr_j(\cc^n) = \U(n)/(\U(j) \times \U(n-j))$, and let $\Gr_j(\cc^n)^{\fr{u}(j)}$ denote the Thom spectrum of the vector bundle over $\Gr_j(\cc^n)$ given by the pulling back the adjoint representation of $\U(j)$ along the map $\Gr_j(\cc^n) \to \BU(j)$. Then there is a $\U(n)$-equivariant splitting
    $$(G_c)_+ \simeq \bigoplus_{j=0}^n \Gr_j(\cc^n)^{\fr{u}(j)}.$$
    This induces a splitting of $\cf_G(G_c)$, and hence of $i^\ast \cf_G(G_c)$ for any map $i: \spec(F[u^{\pm 1}]) \to \cM_G$ with $F$ being an algebraically closed field of characteristic zero. One can show that there is an isomorphism
    $$\pi_\ast i^\ast \cf_G(\Gr_j(\cc^n)^{\fr{u}(j)}) \cong \Omega^j_{\cM_{\GL_n,0}/F}[u^{\pm 1}],$$
    so taking the direct sum over $j = 0, \cdots, n$ gives an additive equivalence of the form \cref{eq: kahler diff and G mod G}.
\end{remark}
Although such splittings of $(G_c)_+$ were proved in \cite{miller-stable-splittings} only for classical groups, they can also be extended with some work to the exceptional groups, too.
However, one encounters an important difficulty in trying to extend \cref{eq: kahler diff and G mod G} to a statement about the stable homotopy type of $G_c$ itself. Namely, suppose that \cref{eq: kahler diff and G mod G} holds for $F$ of arbitrary characteristic; in fact, let us even suppose that the \textit{non}equivariant version of \cref{eq: kahler diff and G mod G} holds, i.e., that there is an isomorphism
\begin{equation}\label{eq: ansatz all char}
    \pi_\ast i^\ast \cf(G_c) \cong \Omega^\ast_{\cM_{G,0}/F}[u^{\pm 1}] \otimes_{\co_{\cM_{G,0}}} F
\end{equation}
for any map $i: \spec(F[u^{\pm 1}]) \to \cM_G$.
Motivated by the example of \cref{rmk: miller splitting}, it is natural to wonder whether this putative splitting could arise from a(n additive) splitting of $(G_c)_+$ itself, where the summands of $(G_c)_+$ realize the individual summands $\Omega^j_{\cM_{G,0}/F}[u^{\pm 1}] \otimes_{\co_{\cM_{G,0}}} F$. Unfortunately, this turns out to be impossible, at least if interpreted naively.
\begin{example}
    Suppose $G = \G_2$. Since $\G_2$ is a framed manifold, its top cell stably splits, and so there is a splitting
    $$(\G_2)_+ \simeq S^0 \oplus X \oplus S^{\g_2},$$
    where $S^{\g_2}$ is the one-point compactification of the adjoint representation of $\G_2$, and $X$ is a finite CW-complex with partial cell diagram shown in \cref{fig: cell diagram G2}.
    \begin{figure}[H]
    \begin{tikzpicture}[scale=0.75]
    \draw [fill] (0, 0) circle [radius=0.05];
    \draw [fill] (2, 0) circle [radius=0.05];
    \draw [fill] (3, 0) circle [radius=0.05];
    \draw [fill] (5, 0) circle [radius=0.05];
    \draw [fill] (6, 0) circle [radius=0.05];
    \draw [fill] (8, 0) circle [radius=0.05];
    
    \draw (0,0) to[out=-90,in=-90] node[below] {\footnotesize{$\Sq^2$}} (2,0);
    \draw (2,0) to node[below] {\footnotesize{$\Sq^1$}} (3,0);
    \draw (5,0) to node[below] {\footnotesize{$\Sq^1$}} (6,0);
    \draw (6,0) to[out=-90,in=-90] node[below] {\footnotesize{$\Sq^2$}} (8,0);
    \end{tikzpicture}
    \caption{A partial cell diagram for the stable summand $X$ of $(\G_2)_+$. The dots indicate the cells; starting from the left, the cells lie in dimensions $3,5,6,8,9$, and $11$. The labels represent the action of the Steenrod operations in mod $2$ cohomology.}
    \label{fig: cell diagram G2}
    \end{figure}
    Let us now take $F$ to be a field of characteristic $2$. Then $\H^\ast_{\G_2}(\ast; F) \cong F[w_4, w_6, w_7]$, where the subscript indicates the cohomological degree. This implies that 
    $$\Omega^\ast_{\H^\ast_{\G_2}(\ast; F)/F} \otimes_{\H^\ast_{\G_2}(\ast; F)} F \cong \Lambda(dw_4, dw_6, dw_7),$$ 
    where $\Lambda$ denotes the exterior algebra on the classes $dw_4$, $dw_6$, and $dw_7$. According to \cref{eq: ansatz all char}, these classes would contribute to $\H^\ast(\G_2; F)$ in cohomological degrees $3,5$, and $6$ respectively. First of all, let us observe that the above ring is \textit{not} isomorphic to $\H^\ast(\G_2; F)$: instead, there is an isomorphism
    $$\H^\ast(\G_2; F) \cong F[dw_4, dw_6]/((dw_4)^4, (dw_6)^2).$$
    Nevertheless, there is an additive isomorphism between $\Omega^\ast_{\H^\ast_{\G_2}(\ast; F)/F} \otimes_{\H^\ast_{\G_2}(\ast; F)} F$ and $\H^\ast(\G_2; F)$, so we can still ask for a stable splitting of $(\G_2)_+$ which realizes the individual summands $\Omega^j_{\H^\ast_{\G_2}(\ast; F)/F} \otimes_{\H^\ast_{\G_2}(\ast; F)} F$. This already fails for $j=1$. Indeed, the $6$-skeleton of $X$ provides a CW-complex $Y$ with a map $Y \to \G_2$ which realizes the inclusion of the subspace $\Omega^1_{\H^\ast_{\G_2}(\ast; F)/F} \otimes_{\H^\ast_{\G_2}(\ast; F)} F \cong F\{dw_4, dw_6, dw_7\}$ into $\H^\ast(\G_2; F)$. However, the map $Y \to \G_2$ cannot stably split (in particular, the $6$-skeleton of $X$ does not stably split off $X$). This was proved ``by hand'' in \cite[Theorem 1.10]{cohen-peterson} using Dyer-Lashof operations.
\end{example}
The preceding example is not special to the non-simply-laced case; one can show that a similar result holds for $G$ of type $E$, too.