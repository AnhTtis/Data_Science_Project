In this section, we will quickly review the derived geometric Satake equivalence following \cite{bf-derived-satake} and \cite{arinkin-gaitsgory-singsupp}. Let $k$ denote a commutative $\QQ$-algebra; all Langlands dual objects will be assumed to live over $k$, and are base-changes of their ``split forms'' over $\QQ$.
\begin{setup}
    Let $G$ be a connected reductive group (over $\cc$, always), and let $\Gr_G = G\ls{t}/G\pw{t}$ denote the affine Grassmannian. There is a canonical left action of $G\ls{t}$ on $\Gr_G$, and hence an action of $G\pw{t} \subseteq G\ls{t}$. The affine Grassmannian is a union of $G\pw{t}$-invariant closed subschemes $X_\alpha$ of finite type, and one defines $\Shv_{G\pw{t}}(\Gr_G; k) = \colim_\alpha \Shv_{G\pw{t}}(X_\alpha; k)$.
    Inside $\Shv_{G\pw{t}}(\Gr_G; k)$ are two full subcategories: 
    \begin{itemize}
        \item $\Shv_{G\pw{t}}(\Gr_G; k)^\lcc$ is the full subcategory of objects whose image under the forgetful functor $\Shv_{G\pw{t}}(\Gr_G; k) \to \Shv(\Gr_G; k)$ is compact. Such objects are called ``locally compact''.
        \item $\Shv_{G\pw{t}}(\Gr_G; k)^\omega$ of compact objects in $\Shv_{G\pw{t}}(\Gr_G; k)$.
    \end{itemize}
    The $\infty$-category $\Shv_{G\pw{t}}(\Gr_G; k)$ admits a monoidal structure, which in fact restricts to a monoidal structure on each of the full subcategories above.
\end{setup}
\begin{setup}
    Let $(e,f,h)$ denote a principal $\sl_2$-triple in the Langlands dual Lie algebra $\ld{\g}$. The element $f$ defines a nondegenerate character $\psi: \ld{\fr{n}} \to \AA^1$. Let $\ld{\g}^{\ast,e}$ denote the orthogonal complement to the subspace $[e,\ld{\g}] \subseteq \ld{\g}$. This defines the \textit{Kostant slice} $\psi + \ld{\g}^{\ast,e} \subseteq \ld{\g}^\ast$; we will denote this inclusion by $\kappa$. Composing the invariant-theoretic quotient map $\chi: \ld{\g}^\ast \to \ld{\g}^\ast\mmod \ld{G}$ with the Kostant slice defines an isomorphism. In other words, the following composite is an isomorphism:
    $$\psi + \ld{\g}^{\ast,e} \xar{\kappa} \ld{\g}^\ast \xar{\chi} \ld{\g}^\ast\mmod \ld{G}.$$
    It will be convenient to identify $\psi + \ld{\g}^{\ast,e}$ with $\ld{\g}^\ast\mmod \ld{G}$ under this isomorphism. If the vector space $\ld{\g}^\ast$ is placed in weight $2$, the map $\kappa$ can be checked to give a \textit{graded} map
    $$\kappa: \ld{\g}^\ast(2)\mmod \ld{G} \to \ld{\g}^\ast(2).$$
    Shearing this graded map (in the sense of \cite[Section 2.1]{ku-rel-langlands}) defines a map $\ld{\g}^\ast[2]\mmod \ld{G} \to \ld{\g}^\ast[2]$, which we will also denote by $\kappa$.
\end{setup}
\begin{lemma}[Chevalley restriction]
    There is an isomorphism $\ld{\g}^\ast\mmod \ld{G} \cong \fr{t}\mmod W$, which refines to an isomorphism of graded schemes
    $$\ld{\g}^\ast(2)\mmod \ld{G} \cong \fr{t}(2)\mmod W \cong \spec \H^\ast_G(\ast; \cc).$$
\end{lemma}
The first part of the following result is \cite[Theorem 5]{bf-derived-satake}, and the second part is \cite[Theorem 12.5.3]{arinkin-gaitsgory-singsupp}.
\begin{theorem}[Bezrukavnikov-Finkelberg, Arinkin-Gaitsgory]\label{thm: derived satake}
    There is a monoidal equivalence
    $$\Shv_{G\pw{t}}(\Gr_G; k)^\lcc \simeq \Perf(\ld{\g}^\ast[2]/\ld{G}),$$
    which restricts to a monoidal equivalence
    $$\Shv_{G\pw{t}}(\Gr_G; k)^\omega \simeq \Perf_{\ld{\cN}/\ld{G}}(\ld{\g}^\ast[2]/\ld{G}),$$
    where the right-hand side is the full subcategory of those perfect complexes which are set-theoretically supported on the nilpotent cone of $\ld{\g}^\ast$.
    Furthermore, there is a commutative diagram
    $$\xymatrix{
    \Ind(\Shv_{G\pw{t}}(\Gr_G; k)^\lcc) \ar[r]^-\sim \ar[d]_{p_!} & \QCoh(\ld{\g}^\ast[2]/\ld{G}) \ar[d]^-{\kappa^\ast} \\
    \Shv_{G\pw{t}}(\ast; k) \ar[r]_-\sim & \QCoh(\ld{\g}^\ast[2]\mmod \ld{G}),
    }$$
    where $p: \Gr_G \to \ast$ is the canonical map to a point and $\kappa^\ast$ is pullback along the (shifted) Kostant slice.
\end{theorem}
We will refer to the first equivalence of \cref{thm: derived satake} as the \textit{derived geometric Satake equivalence}, or more colloquially as ``derived Satake''.
\begin{definition}
    A point $x\in \ld{\g}^\ast$ is called \textit{regular} if its centralizer $Z_{\ld{G}}(x)\subseteq \ld{G}$ has dimension given by the rank of $\ld{G}$. 
    Let $\ld{\g}^{\ast,\reg}$ denote the locus of regular elements; this is an open subscheme whose complement is of codimension $3$.
\end{definition}
\begin{theorem}[{Kostant, \cite{kostant-lie-group-reps}}]\label{thm: kostant reg locus}
    The $\ld{G}$-orbit of the Kostant slice $\kappa: \ld{\g}^\ast\mmod \ld{G} \to \ld{\g}^\ast$ identifies with the regular locus $\ld{\g}^{\ast,\reg}$.
\end{theorem}
\begin{corollary}\label{cor: reg locus satake}
    Let $\ul{k}_{\Gr_G} \in \Ind(\Shv_{G\pw{t}}(\Gr_G; k)^\lcc)$ denote the constant sheaf, and let $\Loc_{G\pw{t}}(\Gr_G; k)$ denote the full subcategory generated by $\ul{k}_{\Gr_G}$. Then there is an equivalence
    $$\Loc_{G\pw{t}}(\Gr_G; k) \simeq \QCoh(\ld{\g}^{\ast,\reg}[2]/\ld{G}).$$
\end{corollary}
\begin{proof}
    Observe that $\ul{k}_{\Gr_G}$ is the pullback $p^\ast \ul{k}$ of the (necessarily constant) sheaf $\ul{k} \in \Shv_{G\pw{t}}(\ast; k)$. Since $p^\ast$ is the right adjoint to $p_!$ (and $\kappa_\ast$ is the right adjoint to $\kappa^\ast$), the commutative diagram of \cref{thm: derived satake} says that $\Loc_{G\pw{t}}(\Gr_G; k)$ is equivalent to the full subcategory of $\QCoh(\ld{\g}^\ast[2]/\ld{G})$ generated by $\kappa_\ast \co_{\ld{\g}^\ast[2]\mmod \ld{G}}$. However, \cref{thm: kostant reg locus} implies that this full subcategory is equivalent to $\QCoh(\ld{\g}^{\ast,\reg}[2]/\ld{G})$, as desired.
\end{proof}
A parallel story holds for the Arkhipov-Bezrukavnikov-Ginzburg (called ``ABG'' in this article) equivalence from \cite{abg-iwahori-satake}. 
\begin{recall}
    Let $\tilde{\ld{\g}}$ denote the Grothendieck-Springer resolution, so that $\tilde{\ld{\g}} = T^\ast(\ld{G}/\ld{N})/\ld{T}$. The action of $\ld{G}$ on $T^\ast(\ld{G}/\ld{N})$ defines the moment map $\mu: \tilde{\ld{\g}} \to \ld{\g}^\ast$. Let $\tilde{\ld{\g}}^{\reg}$ denote the preimage of the regular locus $\ld{\g}^{\ast,\reg} \subseteq \ld{\g}^\ast$ under the moment map $\mu$.
\end{recall}
\begin{prop}\label{prop: psi + t}
    There is an isomorphism $\tilde{\ld{\g}} \cong \ld{G} \times^{\ld{B}} \ld{\fr{n}}^\perp$, as well as a map $\kappa: \psi + \ld{\fr{t}}^\ast \subseteq \ld{\fr{n}}^\perp$ which fits into a Cartesian square
    $$\xymatrix{
    \psi + \ld{\fr{t}}^\ast \ar[r] \ar[d] & \ld{\fr{n}}^\perp \ar[r] & \tilde{\ld{\g}} \ar[d]^-\mu \\
    \psi + \ld{\g}^{\ast,e} \ar[rr] & & \ld{\g}^\ast.
    }$$
\end{prop}
\begin{proof}
    Let $\ld{M}$ be a Hamiltonian $\ld{G}$-scheme with moment map $\mu: \ld{M} \to \ld{\g}^\ast$. Then the pullback $\ld{M} \times_{\ld{\g}^\ast} (\psi + \ld{\g}^{\ast,e})$ can be identified with the Whittaker reduction $\ld{M}/_\psi \ld{N}$. Indeed, a theorem of Kostant's from \cite{kostant-whittaker} identifies $\psi + \ld{\g}^{\ast,e}$ with $(\psi + \ld{\fr{n}}^{-,\perp})/\ld{N}^-$, so that there are isomorphisms
    \begin{align*}
        \ld{M} \times_{\ld{\g}^\ast} (\psi + \ld{\g}^{\ast,e}) & \cong \ld{M}/\ld{G} \times_{\ld{\g}^\ast/\ld{G}} (\psi + \ld{\g}^{\ast,e}) \\
        & \cong \ld{M}/\ld{G} \times_{\ld{\g}^\ast/\ld{G}} (\psi + \ld{\fr{n}}^{-,\perp})/\ld{N}^- \\
        & \cong (\ld{M} \times_{\ld{\fr{n}}^{-,\ast}} \{\psi\})/\ld{N}^- = \ld{M}/_\psi \ld{N}^-.
    \end{align*}
    Therefore, the fiber product in the statement of the proposition identifies with the Whittaker reduction $\tilde{\ld{\g}}/_\psi \ld{N}^-$. Since $\tilde{\ld{\g}} \cong T^\ast(\ld{G}/\ld{N})/\ld{T}$, we may identify $\tilde{\ld{\g}}/_\psi \ld{N}^-$ with the quotient by $\ld{T}$ of $T^\ast(\ld{N}^- {}_\psi\backslash \ld{G}/\ld{N})$. Since Whittaker functions are supported on the big cell, this twisted cotangent bundle is in turn isomorphic to $T^\ast(\ld{N}^- {}_\psi\backslash (\ld{N}^- \times \ld{T} \times \ld{N})/\ld{N}) \cong \ld{T} \times (\psi + \ld{\fr{t}}^\ast)$. The desired Cartesian square follows.
\end{proof}
Again, $\tilde{\ld{\g}}$ admits a $\GG_m$-action obtained by placing $\ld{\fr{n}}^\perp$ in weight $2$, and the map $\kappa: \ld{\fr{t}}^\ast \to \ld{\fr{n}}^\perp$ is equivariant if $\ld{\fr{t}}^\ast$ is also placed in weight $2$. Therefore, shearing (as in \cite[Section 2.1]{ku-rel-langlands}) defines a map
$$\ld{\fr{t}}^\ast[2] \xar{\kappa} \ld{\fr{n}}^\perp[2] \to \tilde{\ld{\g}}[2].$$
We will sometimes denote this composite also by $\kappa$.

The first part of the below equivalence was proved by Arkhipov-Bezrukavnikov-Ginzburg in \cite{abg-iwahori-satake}; the commutative diagram below follows from \cref{prop: psi + t} and \cref{thm: derived satake}.
\begin{theorem}\label{thm: abg}
    Let $B\subseteq G$ be a Borel subgroup, and let $I = G\pw{t} \times_G B$ denote the associated Iwahori subgroup. Then there is an equivalence
    $$\Shv_{I}(\Gr_G; k)^\lcc \simeq \Perf(\tilde{\ld{\g}}[2]/\ld{G}),$$
    which restricts to a monoidal equivalence
    $$\Shv_{I}(\Gr_G; k)^\omega \simeq \Perf_{{\ld{\cN}}/\ld{G}}(\tilde{\ld{\g}}[2]/\ld{G}).$$
    Furthermore, there is a commutative diagram
    $$\xymatrix{
    \Ind(\Shv_{I}(\Gr_G; k)^\lcc) \ar[r]^-\sim \ar[d]_{p_!} & \QCoh(\tilde{\ld{\g}}[2]/\ld{G}) \ar[d]^-{\kappa^\ast} \\
    \Shv_{I}(\ast; k) \ar[r]_-\sim & \QCoh(\ld{\fr{t}}^\ast[2]),
    }$$
    where $p: \Gr_G \to \ast$ is the canonical map to a point and $\kappa^\ast$ is pullback along the (shifted) Kostant slice.
\end{theorem}
As in \cref{cor: reg locus satake}, we find:
\begin{corollary}\label{cor: reg locus abg}
    Let $\ul{k}_{\Gr_G} \in \Shv_{I}(\Gr_G; k)$ denote the constant sheaf, and let $\Loc_{I}(\Gr_G; k)$ denote the full subcategory generated by $\ul{k}_{\Gr_G}$. Then there is an equivalence
    $$\Loc_{I}(\Gr_G; k) \simeq \QCoh(\tilde{\ld{\g}}^{\reg}[2]/\ld{G}).$$
\end{corollary}
The constant sheaf has singular support given by the zero section. In fact, the $\infty$-categories $\Loc_{G\pw{t}}(\Gr_G; k)$ and $\Loc_{I}(\Gr_G; k)$ are the subcategories of \textit{locally constant} (equivariant) sheaves on $\Gr_G$. As such, they depend only on the underlying homotopy types of $G\pw{t}$, $I$, and $\Gr_G$. 
\begin{notation}
    Let $G_c$ be the maximal compact subgroup of $G(\cc)$, and let $T_c$ be the maximal torus of $G_c$ corresponding to the Borel $B$. It is not difficult to see that there are homotopy equivalences
    \begin{align*}
        G\pw{t} & \simeq G(\cc) \simeq G_c \\
        I & \simeq B(\cc) \simeq T_c.
    \end{align*}
    The homotopy type of $\Gr_G$ follows from the next result, due to Quillen and Garland-Raghunathan:
\end{notation}
\begin{theorem}[Quillen, Garland-Raghunathan, \cite{garland-raghunathan, mitchell-buildings}]\label{thm: quillen}
    There is a homotopy equivalence $\Gr_G \simeq \Omega G_c$ (and a homeomorphism onto the subspace of $\Omega G_c$ on those based loops with \textit{polynomial} Fourier expansion) which is equivariant for the left-action of $G_c \subseteq G(\cc) \subseteq G(\cc\pw{t})$ on the left-hand side and the action of $G_c$ on the right-hand side by conjugation.
\end{theorem}
In our discussion below, we will mostly be concerned with the homology of $\Gr_G$, in which case we may replace $\Gr_G$ by $\Omega G_c$. To this extent, we will implicitly use \cref{thm: quillen} without further mention. We will describe analogues of the equivalences of \cref{cor: reg locus satake} and \cref{cor: reg locus abg} for equivariant K-theory and equivariant elliptic cohomology.