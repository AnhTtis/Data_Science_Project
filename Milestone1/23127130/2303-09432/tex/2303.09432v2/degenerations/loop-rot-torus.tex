In this section, we describe an extension of \cref{thm: torus satake} (or rather, of \cref{ex: graded torus satake}) which includes loop-rotation equivariance. Recall that \cref{thm: torus satake} gives an isomorphism $\cf_{T_c}(\Gr_T)^\vee \cong \co(\ld{T}_k \times_{\spec(k)} \cM_T)$. The action of $T$ on $\Gr_T$ refines to an action of $\tilde{T} = T \times \GG_m^\rot$, where $\GG_m^\rot$ acts by loop rotation;  we may therefore consider the \textit{loop-rotation equivariant} homology $\cf_{\tilde{T}_c}(\Gr_T)^\vee$. There is an equivalence $\cM_{\tilde{T}} \simeq \cM_T \times \GG$, where the second factor is identified as $\cM_{\GG_m^\rot}$. Therefore, $\cf_{\tilde{T}_c}(\Gr_T)^\vee$ is a quasicoherent sheaf over $\cM_T \times \GG$ whose fiber over the zero section of $\GG$ recovers $\cf_{{T}_c}(\Gr_T)^\vee$.
\begin{definition}\label{def: G-diff ops}
    Let $\bH$ be a smooth $1$-dimensional group scheme over a base commutative ring $A$, let $T_c$ be a compact torus, and let $\bH_T = \Hom(\bX^\ast(T), \bH)$. (When $\GG$ is an oriented commutative $k$-group scheme, and $\bH = \GG_0$ is its underlying group scheme over $A = \pi_0(k)$, then $\bH_T$ is precisely $\cM_{T,0}$.)
    Let $\lambda$ be a cocharacter of $T_c$, so that $\lambda$ defines a homomorphism $\bX^\ast(T) \to \Z$, and hence a homomorphism $\lambda^\ast: \bH \to \bH_T$. In turn, this defines a map
    $$f^\lambda: \bH_{\tilde{T}} \simeq \bH_T \times \bH \xar{\pr \times \lambda^\ast} \bH_T.$$
    If $y$ is a local section of $\co_{\bH_T}$, we will write $\lambda^\ast(y)$ to denote the resulting local section of $\co_{\bH_{\tilde{T}}}$.
    
    Let $\cd_{\ld{T}}^{\bH}$ denote the quotient of the associative $\co_{\bH}$-algebra $\co_{\bH_{\tilde{T}}}\pdb{x_\lambda | \lambda\in \bX_\ast(T)}$ by the relations given locally by
    $$x_\lambda \cdot x_\mu = x_{\lambda+\mu}, \  y \cdot x_\lambda = x_\lambda \cdot \lambda^\ast(y).$$
    Here, $\lambda,\mu\in \bX_\ast(T)$, and $y$ is a local section of $\co_{\bH_T}$. We will call $\cd_{\ld{T}}^{\bH}$ the \textit{algebra of $\bH$-differential operators} on $\ld{T}$.
\end{definition}
\begin{remark}\label{rmk: G-mellin}
    The algebra $\cd_{\ld{T}}^{\bH}$ satisfies a Mellin transform: namely, it follows from unwinding the definition that there is an equivalence
    $$\cd_{\ld{T}}^{\bH}\modc \simeq \IndCoh(\bH_{\tilde{T}}/\bX^\ast(\ld{T})),$$
    where $\lambda\in \bX^\ast(\ld{T}) \cong \bX_\ast(T)$ acts on $\bH_{\tilde{T}}$ via $y\mapsto \lambda^\ast y$.
\end{remark}
\begin{notation}
    If $k$ is a complex-oriented $2$-periodic $\Eoo$-ring and $\GG_0$ is the $\pi_0(k)$-group underlying a oriented commutative $A$-group $\GG$, we will write $\cd_{\ld{T}}^\GG$ to denote $\cd_{\ld{T}}^{\GG_0}$, and refer to it as the \textit{algebra of $\GG$-differential operators} on $\ld{T}$. We hope this does not cause any confusion.
\end{notation}
\begin{prop}\label{prop: T homology and quantized diffop}
    There is an isomorphism 
    $$\pi_0 \cf_{\tilde{T}}(\Gr_T)^\vee \cong \cd_{\ld{T}}^\GG$$
    of $\co_{\GG_0}$-algebras. In particular, there is an equivalence
    $$\Loc_{\tilde{T}_c}(\Gr_T; k) \simeq \cd_{\ld{T}}^\GG\modc^{(\ld{T} \times \ld{T}, \weak)},$$
    where the right-hand side denotes the category of left $\cd_{\ld{T}}^\GG$-modules whose underlying quasicoherent sheaf over $\ld{T}$ is equivariant for $\ld{T} \times \ld{T}$-action on $\ld{T}$ given by left and right multiplication.
\end{prop}
\begin{proof}
    Since $\Gr_T \cong \bX_\ast(T)$, it is easy to see that $\pi_0 \cf_{\tilde{T}}(\Gr_T)^\vee \cong \bigoplus_{\lambda\in \bX_\ast(T)} \pi_0 \co_{\cM_{\tilde{T}}}$; let $x_\lambda$ be a generator of the summand indexed by $\lambda\in \bX_\ast(T)$. If $\lambda\in \bX_\ast(T) = \Hom(\bX^\ast(T), \Z)$, the map $\Omega T_c \to \Omega T_c$ given by multiplication-by-$\lambda$ is $T_c\times S^1_\rot$-equivariant for the homomorphism $T_c\times S^1_\rot \to T_c\times S^1_\rot$ given by 
    $$(t, \theta) \mapsto (t \lambda(\theta), \theta),$$
    where $\lambda$ is viewed as a homomorphism $S^1 \to T$. On weight lattices, this homomorphism induces the map $\bX^\ast(T) \times \Z \to \bX^\ast(T) \times \Z$ which sends $(\mu, n) \mapsto (\mu, n+\bX_\ast(T)(\mu))$. In particular, the composite 
    $$\bX^\ast(T) \to \bX^\ast(T) \times \Z \to \bX^\ast(T) \times \Z$$
    sends $\mu \mapsto (\mu, \bX_\ast(T)(\mu))$. Applying $\Hom(-, \GG)$ to this composite precisely produces the map $f^\lambda: \cM_{\tilde{T}} \to \cM_T$ from \cref{def: G-diff ops}. This implies the desired identification of $\pi_0 \cf_{\tilde{T}}(\Gr_T)^\vee$.
\end{proof}
\begin{example}\label{ex: ordinary quantized diffop}
    Let $T \cong S^1$ be a torus of rank $1$ (for simplicity).
    Suppose $k = \QQ[u^{\pm 1}]$ with $u$ in degree $2$, so $\GG = \GG_a$ and $\co_{\GG_0} \cong \QQ[\hbar]$. Then the algebra of \cref{def: G-diff ops} is the quotient of the $\QQ[\hbar]$-algebra $\QQ[\hbar]\pdb{y, x^{\pm 1}}$ by the relation $yx = x(y+\hbar)$. In other words, $y$ acts as the operator $\hbar x\partial_x$, so we simply have that 
    $$\H^{\tilde{T}}_0(\Gr_T; \QQ[u^{\pm 1}]) \cong \H^{\tilde{T}}_\ast(\Gr_T; \QQ) \cong \QQ[\hbar]\pdb{\hbar x\partial_x, x^{\pm 1}}.$$
    This has been stated previously as \cite[Proposition 5.19(2)]{bfn-ii}.
    In particular, the localization $\H^{\tilde{T}}_0(\Gr_T; \QQ[u^{\pm 1}])[\hbar^{-1}]$ is isomorphic to the rescaled Weyl algebra $\cd_{\ld{T}}^\hbar$; this is the motivation behind the terminology in \cref{def: G-diff ops}.
    Note that for a general torus, \cref{rmk: G-mellin} simply reduces to the standard Mellin transform, which gives an equivalence between $\DMod_\hbar(\ld{T})$ and $\IndCoh(\fr{t}_{\QQ[\hbar]}/\bX^\ast(\ld{T}))$; here, $\lambda \in \bX^\ast(\ld{T})$ acts on $\fr{t}_{\QQ[\hbar]}$ by $x \mapsto x + (d\lambda)(\hbar)$.
\end{example}
\begin{example}\label{ex: q quantized diffop}
    Again, let $T \cong S^1$ be a torus of rank $1$ (for simplicity).
    Suppose $k = \KU$, so $\GG = \GG_m$ and $\co_{\GG_0} \cong \Z[q^{\pm 1}]$. Then the algebra of \cref{def: G-diff ops} is the quotient of the $\Z[q^{\pm 1}]$-algebra $\Z[q^{\pm 1}]\pdb{y^{\pm 1}, x^{\pm 1}}$ by the relation $yx = qxy$. (This is also known as the ``quantum torus''.) In other words, $y$ acts as the operator $q^{x\partial_x}$ sending $f(x) \mapsto f(qx)$, so we simply have that 
    $$\KU^{\tilde{T}}_0(\Gr_T) \cong \Z[q^{\pm 1}]\pdb{q^{x\partial_x}, x^{\pm 1}}.$$
    This is closely related to the $q$-Weyl algebra $\cd_q = \Z[q^{\pm 1}]\pdb{\Theta, x^{\pm 1}}/(\Theta x = x(q\Theta+1))$ for $\ld{T} = \GG_m$: indeed, since the logarithmic $q$-derivative $\Theta = x\nabla_q$ is given by the fraction $\frac{q^{x\partial_x}-1}{q-1}$, the pullback of $\cd_{\ld{T}}^\GG$ along $\GG_m-\{1\} \hookrightarrow \GG_m$ is isomorphic to the algebra $\cd_q[\frac{1}{q-1}]$.
    Note that \cref{rmk: G-mellin} gives a ``$q$-Mellin transform'', i.e., an equivalence between $\LMod_{\KU^{\tilde{T}}_0(\Gr_T)}$ and $\IndCoh(T_{\Z[q^{\pm 1}]}/\bX^\ast(\ld{T}))$, where $\lambda \in \bX^\ast(\ld{T}) = \bX_\ast(T)$ acts on $T_{\Z[q^{\pm 1}]}$ by sending $y\mapsto \lambda(q) y$.
\end{example}

Let us briefly outline the relationship between the algebra $\cd_{\ld{T}}^{\bH}$ of \cref{def: G-diff ops} and the $F$-de Rham complex of \cite{generalized-n-series}.
\begin{notation}
    For the purpose of this discussion, we will assume that $T \cong S^1$ is a torus of rank $1$, so that $\ld{T} \cong \GG_m$. We will also fix an invariant differential form on the formal completion $\hat{\bH}$ of $\bH$ at the zero section, so that there is an isomorphism $\hat{\bH} \cong \spf A\pw{t}$ of formal $A$-schemes. Let $F(x,y)$ denote the resulting formal group law over $A$, and define the $n$-series of $F$ by
    $$[n]_F := \overbrace{F(t, F(t, F(t, \cdots F(t, t) \cdots )))}^n.$$
    We will often write $x+_Fy = x+_\GG y$ to denote $F(x,y)$.
    Let $\hat{\cd}_{\ld{T}}^{\bH}$ denote the completion of $\cd_{\ld{T}}^{\bH}$ at the zero section of $\bH$.
\end{notation}
\begin{lemma}[Cartier duality]\label{cartier-duality}
    Let $\hat{\bH}$ be a $1$-dimensional formal group over a commutative ring $A$, and let $\Cart(\hat{\bH})$ denote its Cartier dual (see \cite[Section 3.3]{drinfeld-formal-group} for more on Cartier duals of formal groups). Then there is an equivalence of categories $\QCoh(\hat{\bH}) \simeq \QCoh(B\Cart(\hat{\bH}))$ sending the convolution tensor product on the left-hand side to the usual tensor product on the right-hand side. Under this equivalence, the functor $\QCoh(\hat{\bH}) \to \Mod_A$ given by restriction to the zero section is identified with the functor $\QCoh(B\Cart(\hat{\bH})) \to \Mod_A$ given by pullback along the map $\spec(A) \to B\Cart(\hat{\bH})$.
\end{lemma}
\begin{prop}\label{prop: endomorphism F-de Rham}
    There is a canonical action of $\hat{\cd}_{\ld{T}}^{\bH}$ on $(\GG_m)_{A\pw{t}} = \spf A\pw{t}[x^{\pm 1}]$ such that $A\pw{t}[x^{\pm 1}] \otimes_{\hat{\cd}_{\ld{T}}^{\bH}} A\pw{t}[x^{\pm 1}]$ is isomorphic to the two-term complex
    $$C^\bull = (A\pw{t}[x^{\pm 1}] \to A\pw{t}[x^{\pm 1}]dx), \ x^n \mapsto [n]_F x^n dx$$
    from \cite[Remark 4.3.8]{generalized-n-series}.
\end{prop}
\begin{proof}[Proof sketch]
    Since $T$ is of rank $1$, there is an isomorphism $\bH_T \cong \bH$, and hence an isomorphism $\hat{\bH}_T \cong \hat{\AA}^1$ of formal $A$-schemes, where $\hat{\bH}_T$ denotes the completion of $\bH_T$ at the zero section. Let $y$ be a local coordinate on $\bH_T$.
    Then, $\hat{\cd}_{\ld{T}}^{\bH}$ is isomorphic to the quotient of the associative $\hat{\co}_{\bH}$-algebra $\hat{\co}_{\bH \times \bH_T}\pdb{x^{\pm 1}}$ subject to the relation $yx = x(y +_\GG t)$.
    The $t$-adic filtration on $\hat{\cd}_{\ld{T}}^{\bH}$ therefore has associated graded $\gr(\hat{\cd}_{\ld{T}}^{\bH}) \cong \hat{\co}_{\bH_T}[x^{\pm 1}]\pw{\ol{t}}$, where $\ol{t}$ lives in weight $1$. View $A$ as a $\co_{\bH_T}$-algebra via the zero section, i.e., the augmentation $\co_{\bH_T} \to A$. Then, the action of $\gr(\hat{\cd}_{\ld{T}}^{\bH})$ on $A[x^{\pm 1}]\pw{\ol{t}}$ is induced by the augmentation $\hat{\co}_{\bH_T} \to A$. The isomorphism $\hat{\bH}_T \cong \hat{\AA}^1$ of formal $A$-schemes then implies an isomorphism $A \otimes_{\co_{\bH_T}} A \cong A[\epsilon]/\epsilon^2$ with $\epsilon$ in homological degree $1$.
    It follows that
    $$A\pw{\ol{t}}[x^{\pm 1}] \otimes_{\gr(\hat{\cd}_{\ld{T}}^{\bH})} A\pw{\ol{t}}[x^{\pm 1}] \simeq A\pw{\ol{t}}[x^{\pm 1}][\epsilon]/\epsilon^2,$$
    where $\ol{t}$ is in weight $1$ and degree $0$, and $\epsilon$ is in weight $0$ and degree $1$.
    
    By \cref{cartier-duality}, the $t$-adic filtration on $\hat{\cd}_{\ld{T}}^{\bH}$ is equivalent to the data of a $\Cart(\hat{\bH})$-action on $A\pw{\ol{t}}[x^{\pm 1}] \otimes_{\gr(\hat{\cd}_{\ld{T}}^{\bH})} A\pw{\ol{t}}[x^{\pm 1}] \simeq A\pw{\ol{t}}[x^{\pm 1}][\epsilon]/\epsilon^2$. This in turn is equivalent to the data of a differential 
    $$\nabla: A\pw{\ol{t}}[x^{\pm 1}] \to A\pw{\ol{t}}[x^{\pm 1}]\cdot \epsilon$$
    satisfying an $\hat{\bH}$-analogue of the Leibniz rule: if\footnote{Note that $\nabla$ has to be homogeneous in the degree of the monomial in $x$, as can be seen by keeping track of the $x$-weight.} $\nabla(x^n) = f(n) x^n \epsilon$ for some $f(n)\in A\pw{t}$, then $f(n+m) = f(n) +_\GG f(m)$.
    It therefore suffices to determine $\nabla(x)$; but the relation $yx = x(y +_\GG t)$ forces $\nabla(x) = tx\epsilon$. This implies that 
    $$\nabla(x^n) = (\overbrace{t +_\GG \cdots +_\GG t}^n) x^n \epsilon = [n]_F x^n \epsilon,$$
    as desired.
\end{proof}
\begin{example}
    When $\bH = {\GG}_a$ over\footnote{Of course, one can work over $\Z$ too; we just chose $\QQ$ to continue with \cref{ex: ordinary quantized diffop}.} $\QQ$, the complex $C^\bull$ is
    $$C^\bull = (\QQ\pw{\hbar}[x^{\pm 1}] \to \QQ\pw{\hbar}[x^{\pm 1}]dx), \ x^n \mapsto n\hbar x^n dx.$$
    Indeed, since $yx = x(y+\hbar)$, we have $yx^n = x^n(y+n\hbar)$; since $t = \hbar$ in this case, we have $x^n\mapsto n\hbar x^n \epsilon$. This is evidently a $\hbar$-rescaling of the classical de Rham complex of $\GG_m$.

    When $\bH = \GG_m$ over $\Z$, the complex $C^\bull$ is
    $$C^\bull = (\Z\pw{q-1}[x^{\pm 1}] \to \Z\pw{q-1}[x^{\pm 1}]dx), \ x^n \mapsto (q^n-1) x^n dx.$$
    Indeed, since $yx = x(qy)$, we have $yx^n = x^n (q^n y)$, and hence 
    $$(y-1)x^n = x^n(q^n y - 1) = x^n((y-1) +_F (q^n-1)),$$
    where $F(z,w) = z + w + zw$ is the multiplicative formal group law; since $t = q-1$ in this case, we have $x^n \mapsto (q^n-1) x^n \epsilon$. The complex $C^\bull$ is a $(q-1)$-rescaling of the $q$-de Rham complex of $\GG_m$ from \cite{scholze-q-def}.
\end{example}
\begin{remark}
    The complex of \cref{prop: endomorphism F-de Rham} is not quite the $F$-de Rham complex of \cite[Definition 4.3.6]{generalized-n-series}; rather, if $\eta_t$ denotes the d\'ecalage functor of \cite{berthelot-ogus} with respect to the ideal $(t)\subseteq A\pw{t}$, the $F$-de Rham complex is given by the d\'ecalage $\eta_t C^\bull$. In particular, the complex of \cref{prop: endomorphism F-de Rham} is isomorphic to the $F$-de Rham complex after inverting $t$. One can modify the algebra $\cd_{\ld{T}}^{\bH}$ of \cref{def: G-diff ops} (by performing a noncommutative analogue of an affine blowup/deformation to the normal cone\footnote{For instance, in the case of \cref{ex: ordinary quantized diffop}, this procedure simply adjoins the fraction $\frac{y}{\hbar}$; in the case of \cref{ex: q quantized diffop}, this procedure simply adjoins the fraction $\frac{y-1}{q-1}$.}) such that the relative tensor product as in \cref{prop: endomorphism F-de Rham} is the $F$-de Rham complex itself. Since it is not needed for this article, we will not describe this modification here.
\end{remark}
\begin{remark}\label{rmk: koszul duality LT}
    Suppose $k$ is a complex-oriented $2$-periodic $\Eoo$-ring equipped with an oriented commutative $k$-group scheme $\GG$.
    \cref{prop: endomorphism F-de Rham} says that $\hat{\cd}_{\ld{T}}^{\GG_0}$ is Koszul dual to the complex $C^\bull$. Forthcoming work of Arpon Raksit shows that the d\'ecalage $\eta_t C^\bull$ can be recovered from the ``even filtration'' (in the sense of \cite{even-filtr}) on the periodic cyclic homology $\HP(\tau_{\geq 0} k[x^{\pm 1}]/\tau_{\geq 0} k)$. See also the discussion in \cite[Section 3.3]{thh-xn}.
    Using similar techniques, one can show that $C^\bull$ can be recovered from the even filtration on the negative cyclic homology $\HC^-(k[x^{\pm 1}]/k) = \HH(k[x^{\pm 1}]/k)^{hS^1}$.

    Recalling that $T = S^1$, this $\Eoo$-$k$-algebra is simply $\HC^-(k[\Omega T]/k)$. The Hochschild homology $\HH(k[\Omega T]/k) \simeq k \otimes \THH(S[\Omega T])$ is $S^1$-equivariantly equivalent to the $k$-chains $C_\ast(\cL T; k)$ on the free loop space of $T$. (For a reference, see \cite[Corollary IV.3.3]{nikolaus-scholze}.) The $k$-chains $k[\cL T]$ is $S^1$-equivariantly Koszul dual\footnote{This Koszul duality essentially stems from the (non-$S^1_\rot$-equivariant) decomposition $\cL T \simeq T \times \Omega T$.} to $k[\Omega T]^{hT}$; this can be identified as a completion of $\cf_T(\Omega T)^\vee$ at the zero section of $\cM_T$. In other words, $\HC^-(k[\Omega T]/k)$ is Koszul dual to the completion of $\cf_{T\times S^1_\rot}(\Omega T)^\vee$ at the zero section of $\cM_T \times \GG$. This is the topological source of the Koszul duality of \cref{prop: endomorphism F-de Rham}.
\end{remark}