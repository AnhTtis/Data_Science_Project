We begin this section by immediately amending the goal referred to at the end of the preceding section. Namely, instead of studying the $\infty$-category $\Loc_{T_c}(\Gr_G; k)$ for a connected reductive group $G$ and a maximal torus $T\subseteq G$, we will study a particular \textit{degeneration} of this $\infty$-category. Before discussing the construction of this degeneration, let us motivate \textit{why} it is useful (see also the introduction for some ``philosophy'' regarding this degeneration).

Suppose that there was an equivalence of the form $\Loc_{T_c}(\Gr_G; k) \simeq \QCoh(\fr{X}_k)$ for some spectral $k$-stack $\fr{X}_k$. In order for such an equivalence to be considered related to Langlands duality, the stack $\fr{X}_k$ must have some relationship to the dual group $\ld{G}$; for instance, one can wonder whether the underlying classical $\pi_0(k)$-stack of $\fr{X}_k$ lives over the classifying stack $B\ld{G}_{\pi_0(k)}$. Here, $\ld{G}_{\pi_0(k)}$ is the base-change of the Chevalley split form of $\ld{G}$ along the map $\Z \to \pi_0(k)$. (When $k$ is an $\Eoo$-$\QQ$-algebra, the stack $\fr{X}_k$ is $\tilde{\ld{\g}}[2]/\ld{G}$, which does indeed live over $B\ld{G}$.) The most satisfying description of $\fr{X}_k$ must therefore involve a lift of the dual group $\ld{G}$ to a (flat) spectral group scheme over $k$. Unfortunately, the existence of such a lift is far from clear: giving a flat lift of $\SL_2$ (even just as a \textit{scheme}!) to complex K-theory leads to very subtle questions; see \cref{sec: lifting SL2}.

Instead, let us return to the general situation of a finite $T_c$-space $X$. One can then view $\Loc_{T_c}(X; k)$ as a categorification of the cochains $\cf_T(X)\in \QCoh(\cM_T)$; so for the moment, let us just describe a degeneration of $\cf_T(X)$ and $\cM_T$. There is a natural filtered lift of $\cM_T = \spec k_T$ to a filtered $\tau_{\geq \star}(k)$-scheme, given by $\spec \tau_{\geq \star}(k_T)$. (This construction is, of course, closely related to the even filtration constructed in \cite{even-filtr, piotr-even-filtr}.) In particular, one obtains a corresponding graded $\pi_\ast(k)$-scheme $\spec \pi_\ast(k_T)$. Note that this is now a \textit{classical} scheme, with no spectral algebro-geometric nature.
If $k$ is even-periodic, i.e., is equipped with an isomorphism $\pi_\ast(k) \cong \pi_0(k)[u^{\pm 1}]$ with $u \in \pi_2(k)$, then this is equivalent to the data of the classical $\pi_0(k)$-scheme $\spec \pi_0(k_T)$. (Recall that this is the affinization of the scheme $\cM_{T,0}$; to get to the definition described below, one needs to replace $\spec \pi_0(k_T)$ in the below discussion by $\cM_{T,0}$.)

If the finite $T_c$-space $X$ has even cells, then one can construct a well-behaved filtered lift of $\cf_T(X)$ to a filtered quasicoherent sheaf over $\spec \tau_{\geq \star}(k_T)$, given by $\tau_{\geq \star} \cf_T(X)$. This defines a corresponding graded variant of $\cf_T(X)$, given simply by the quasicoherent sheaf $\pi_0 \cf_T(X)$ over $\spec \pi_0(k_T)$. Again, this is an object in the realm of \textit{classical} algebraic geometry; so when applied to the affine Grassmannian $\Gr_G$, it is something that could, in theory, be described in terms of the usual dual group $\ld{G}$ base-changed to $\pi_0(k)$. 

The idea for constructing the desired degeneration of $\Loc_{T_c}(X; k)$ is very similar; we now turn to its mechanics.  Let us begin with a simple observation. If $Y$ is a {connected} space, the $\infty$-category $\Loc(Y; k) = \Fun(Y, \Mod_k)$ of local systems on $Y$ is equivalent, by Koszul duality, to $\LMod_{C_\ast(\Omega Y; k)}$. This is very useful, since it allows one to reduce the study of local systems to the study of a particular (derived) algebra. A similar property is true for $\Loc_{T_c}(X; k)$:
\begin{prop}\label{prop: equivariant-koszul}
    Let $X$ be a connected finite $T_c$-space. Then there is an equivalence $\Loc_{T_c}(X; k) \simeq \LMod_{\cf_T(\Omega X)^\vee}(\QCoh(\cM_T))$.
\end{prop}
\begin{proof}
    Let $s: \ast \to X$ denote the inclusion of a point. We claim that $s^\ast: \Loc_{T_c}(X; k) \to \QCoh(\cM_T)$ admits a left adjoint $s_!$. Indeed, the statement for general $X$ follows formally from the case of $X = T/T'$ for some closed subgroup $T'\subseteq T$ (so $s$ is the inclusion of the trivial coset). In this case, $s^\ast$ is the functor $\QCoh(\cM_{T'}) \to \QCoh(\cM_T)$ given by pushforward along the associated morphism $q: \cM_{T'} \to \cM_T$, so it has a left adjoint $s_!$ given by $q^\ast$. Note that $s^\ast$ also has a right adjoint; in particular, it preserves small limits and colimits. Observe now that $s_! \co_{\cM_T}$ is a compact generator of $\Loc_{T_c}(X; k)$: indeed, suppose $\cf\in \Loc_{T_c}(X; k)$ such that $\Map_{\Loc_{T_c}(X; k)}(s_! \co_{\cM_T}, \cf) \simeq 0$ as an object of $\QCoh(\cM_T)$. Because $s^\ast \cf \simeq \Map_{\Loc_{T_c}(X; k)}(s_! \co_{\cM_T}, \cf)$ in $\QCoh(\cM_T)$, we see that $s^\ast \cf \simeq 0$. Using the connectivity of $X$, we see that $\cf$ itself must be zero, which implies that $s_! \co_{\cM_T}$ is a compact generator of $\Loc_{T_c}(X; k)$. It follows from the Barr-Beck-Lurie theorem \cite[Theorem 4.7.3.5]{HA} that $\Loc_{T_c}(X; k)$ is equivalent to the $\infty$-category of left $\End_{\Loc_{T_c}(X; k)}(s_! \co_{\cM_T})$-modules in $\QCoh(\cM_T)$. But $\End_{\Loc_{T_c}(X; k)}(s_! \co_{\cM_T}) \simeq s^\ast s_! \co_{\cM_T}$, which identifies with $\cf_T(\Omega X)^\vee$.
\end{proof}
\begin{remark}\label{rmk: loc and comod}
    Modifying the preceding argument shows that if $X$ is a connected finite $T_c$-space, there is an equivalence 
    \begin{equation}\label{eq: loc and comod}
        \Loc_{T_c}(X; k) \simeq \coLMod_{\cf_T(X)^\vee}(\QCoh(\cM_T)).
    \end{equation}
    In particular, if $X$ admits an $\E{n}$-algebra structure (compatible with the $T_c$-action), then $\cf_T(X)^\vee$ admits the structure of an $\E{n}$-algebra\footnote{If $\cC$ is a symmetric monoidal $\infty$-category, \cite[Corollary 3.3.4]{elliptic-i} can be used to show that there is an equivalence $\coCAlg(\Alg_\E{n}(\cC)) \simeq \Alg_\E{n}(\coCAlg(\cC))$.} in $\coCAlg(\QCoh(\cM_T))$, and the equivalence \cref{eq: loc and comod} is $\E{n}$-monoidal for the convolution tensor product on both sides. 
\end{remark}
\cref{prop: equivariant-koszul} and \cref{rmk: loc and comod} continue to hold even when $X$ is a filtered colimit of finite $T_c$-spaces. In order for the claim in \cref{rmk: loc and comod} about $\E{n}$-algebra structures to hold, we need the filtered diagram $\{X_\lambda\}$ presenting $X$ to admit the structure of an $\E{n}$-algebra in filtered $T_c$-spaces. We will need to apply this in the case when $X$ is the affine Grassmannian, in which case we can apply the following observation. 
\begin{lemma}\label{filtered E2}
    The $\bX_\ast(T)^+$-indexed Schubert filtration $\{\Gr_G^{\leq \lambda}(\cc)\}$ naturally admits the structure of an $\E{2}$-algebra in $\Fun(\bX_\ast(T)^+, \Top(T_c))$.
\end{lemma}
\begin{proof}
    This can be proved in essentially the same way as \cite[Theorem 3.10]{hahn-yuan}; let us sketch the argument. We will utilize \cite[Proposition 5.4.5.15]{HA}, which states that if $\cC$ is a symmetric monoidal $\infty$-category, then a nonunital $\E{2}$-algebra object in $\cC$ is equivalent to the datum of a locally constant $\mathrm{N}(\mathrm{Disk}(\cc))_\mathrm{nu}$-algebra object in $\cC$. Concretely, this amounts to specifying an object $A(D)\in \cC$ for every disk $D\subseteq \cc$ and coherent maps $\bigotimes_{i=1}^n A(D_i)\to A(D)$ for every inclusion $\coprod_{i=1}^n D_i\to D$ of disks, such that for every embedding $D\subseteq D'$ of disks, the induced map $A(D)\to A(D')$ is an equivalence.

    In this case, $\cC = \Fun(\bX_\ast(T)^+, \Top(T_c))$, and the object $A(D)\in \Fun(\bX_\ast(T)^+, \Top(T_c))$ assigned to a disk $D\subseteq \cc$ may be defined via the Beilinson-Drinfeld Grassmannian $\Gr_{G,\Ran}$. Namely, the Beilinson-Drinfeld Grassmannian admits (by construction) a morphism $\Gr_{G, \Ran} \to \Ran_{\AA^1}$; upon taking complex points, we obtain a map $\Gr_{G, \Ran}(\cc) \to \Ran(\cc)$. If $S\subseteq \cc$ is a subset, then the preimage of $\Ran(S)\subseteq \Ran(\cc)$ defines a subspace $\Gr_{G, \Ran}(S\subseteq \cc)\subseteq \Gr_{G, \Ran}(\cc)$. The filtration of $\Gr_G$ via the Bruhat decomposition extends to a filtration $\Gr_{G, \Ran, \leq \mu}$ of $\Gr_{G, \Ran}$ by dominant coweights $\mu\in \bX_\ast(T)^+$; see \cite[3.1.11]{zhu-grass}. Finally, the object $A(D)\in \Fun(\bX_\ast(T)^+, \Top(T_c))$ associated to a disk $D\subseteq \cc$ is the functor $\bX_\ast(T)^+\to \Top(T_c)$ sending $\mu\in \bX_\ast(T)^+$ to $\Gr_{G, \Ran, \leq \mu}(D\subseteq \cc)$.

    Suppose $\coprod_{i=1}^n D_i\to D$ is an inclusion of disks. The induced map $\bigotimes_{i=1}^n A(D_i)\to A(D)$ is defined as follows. Let $\mu\in \bX_\ast(T)^+$; for every $n$-tuple $(\mu_1, \cdots, \mu_n)$ with $\sum_{i=1}^n \mu_i\leq \mu$, we need to exhibit maps $\bigotimes_{i=1}^n A(D_i)(\mu_i)\to A(D)(\mu)$ satisfying the obvious coherences. But
    $$\bigotimes_{i=1}^n A(D_i)(\mu_i) = \prod_{i=1}^n \Gr_{G, \Ran, \leq \mu_i}(D_i\subseteq \cc),$$
    so it suffices to show that if $\mu_1 + \mu_2 \leq \mu$, then there are maps $\Gr_{G, \Ran, \leq \mu_1}(D_1\subseteq \cc) \times \Gr_{G, \Ran, \leq \mu_2}(D_2\subseteq \cc)\to \Gr_{G, \Ran, \leq \mu}(D\subseteq \cc)$. The argument for this is exactly as in \cite[Construction 3.15]{hahn-yuan}.

    We next need to show that the $\mathrm{N}(\mathrm{Disk}(\cc))_\mathrm{nu}$-algebra $A$ defined above is locally constant, i.e., that if $D\subseteq D'$ is an embedding of disks, then $A(D)\to A(D')$ is an equivalence of functors $\bX_\ast(T)^+\to \Top(T_c)$. This follows from \cite[Proposition 3.17]{hahn-yuan}. To conclude, it suffices (by \cite[Theorem 5.4.4.5]{HA}) to establish the existence of a quasi-unit for the functor $A:\bX_\ast(T)^+\to \Top(T_c)$, i.e., a map $1_{\Fun(\bX_\ast(T)^+, \Top(T_c))}\to A$ which is both a left and right unit up to homotopy. Since the unit in $\Fun(\bX_\ast(T)^+, \Top(T_c))$ is the functor sending $\mu\in \bX_\ast(T)^+$ to the point $\ast$, a quasi-unit is the datum of a map $\ast \to \Gr_{G, \leq \mu}(\cc)$ for each $\mu\in \bX_\ast(T)^+$. As in the proof of \cite[Theorem 3.10]{hahn-yuan}, this can be taken to be the inclusion of the point corresponding to the trivial $G$-bundle over $\AA^1$ with the canonical trivialization away from the origin.
\end{proof}

Suppose, now, that $A$ is an $\E{1}$-ring with even homotopy. Any left $A$-module $M$ then defines a filtered left $\tau_{\geq 2\star}(A)$-module $\tau_{\geq 2\star}(M)$; we will denote the corresponding associated graded left $\pi_{2\ast}(A)$-module by $\gr_\ev(M)$. If $M,N\in \LMod_A$, there is then a canonical (complete and exhaustive) filtration on the $A$-module $\Map_A(M,N)$ whose associated graded is given by the shearing of $\Map_{\pi_{2\ast}(A)}(\gr_\ev(M), \gr_\ev(N))$. Informally, this means that there is a $1$-parameter degeneration (constructed using the double-speed Postnikov filtration) from $\LMod_A$ to the category $\LMod_{\pi_{2\ast}(A)}^\gr$, given by the filtered category $\LMod_{\tau_{\geq 2\star} A}$.
Motivated by the preceding discussion, we can now define our desired degeneration of $\Loc_{T_c}(X; k)$.
\begin{definition}\label{def: graded Loc}
    Suppose that $X$ is a (ind-)finite $T_c$-space with even cells (such as $\Gr_G$). The $\infty$-category $\Loc_{T_c}^\gr(X; k)$ is defined as
    $$\Loc_{T_c}^\gr(X; k) = \coLMod_{\pi_0(\cf_T(X)^\vee)}(\QCoh(\cM_{T,0})).$$
    The ``constant sheaf'' $\ul{k}^\gr$ in this category is the comodule $\pi_0(\cf_T(X)^\vee)$ itself.
    Similarly, suppose $Y$ is a finite $T_c$-space such that $\Omega Y$ has even cells (such as $G_c$). The $\infty$-category $\Loc_{T_c}^\gr(Y; k)$ is defined as
    $$\Loc_{T_c}^\gr(Y; k) = \LMod_{\pi_0(\cf_T(\Omega Y)^\vee)}(\QCoh(\cM_{T,0})).$$
    The ``constant sheaf'' $\ul{k}^\gr$ in this category is the structure sheaf $\co_{\cM_{T,0}}$ viewed as a $\pi_0(\cf_T(\Omega Y)^\vee)$-module via the augmentation.
\end{definition}
These should be viewed as ``mixed'' (in the sense of \cite{bbdg}) variants of the full $\infty$-categories $\Loc_{T_c}(X; k)$ and $\Loc_{T_c}(Y; k)$.
\begin{remark}\label{rmk: loc gr convolution tensor}
    If $X$ admits an $\E{n}$-algebra structure (compatible with the $T_c$-action), then the $\E{n}$-algebra structure on $\cf_T(X)^\vee$ equips $\pi_0(\cf_T(X)^\vee)$ with the structure of a commutative algebra object in $\QCoh(\cM_{T,0})$. In particular, $\Loc_{T_c}^\gr(X; k)$ acquires a symmetric monoidal structure, which we will refer to as the ``convolution tensor structure'' and denote by $\star$.
\end{remark}
\begin{remark}
    There is an apparent asymmetry in \cref{def: graded Loc}: why could we not have defined $\Loc_{T_c}^\gr(Y; k)$ to be $\coLMod_{\pi_0(\cf_T(Y)^\vee)}(\QCoh(\cM_{T,0}))$? The issue is that since $Y$ contains odd-dimensional cells, taking $\pi_0$ of $\cf_T(Y)^\vee$ is a very destructive process. More generally, as in the discussion at the beginning of this section, $\pi_0 \cf_T(X)^\vee$ for a finite $T_c$-space $X$ should only be regarded as a well-behaved reflection of $\cf_T(X)^\vee$ itself when $X$ has even cells.

    However, if $Y$ was the total space of an iterated fibration of odd-dimensional spheres (which happens when $Y = \U(n)$ or $\Sp(n)$), then one could alternatively consider the category of comodules in $\QCoh(\cM_{T,0})$ over the truncation $\tau_{[-1,0]}(\cf_T(Y)^\vee)$. If the cobar construction on $\tau_{[-1,0]}(\cf_T(Y)^\vee)$ is given by $\pi_0(\cf_T(\Omega Y)^\vee)$, it then follows from Koszul duality that (up to finiteness questions) this new category would be equivalent to the definition of $\Loc_{T_c}^\gr(Y; k)$ from \cref{def: graded Loc}.
\end{remark}
\begin{remark}
    If $k  = \QQ[u^{\pm 1}]$ with $u$ in degree $2$, then (using the results of \cite{abg-iwahori-satake}) $\Loc_{T_c}(\Gr_G; k)$ is equivalent to the shearing of the $2$-periodification of the category $\Loc_{T_c}^\gr(\Gr_G; k)$. This can be understood as a statement about formality. If $k$ is a more general $\Eoo$-ring (like complex K-theory $\KU$), then formality is generally impossible: for instance, a $\KU$-module $M$ is generally not equivalent (even as a spectrum!) to the shearing of $\pi_\ast(M)$, unless $M$ is also a $\QQ$-module.
\end{remark}
\begin{remark}\label{def: graded G equiv loc sys}
    We will not discuss $G_c$-equivariant cohomology much in this article, except for the end of \cref{sec: review Q coeff}. There, we will only consider the case $k = \QQ[u^{\pm 1}]$ with $u$ in degree $2$. In this case, the equivariant cohomology $\H^\ast_{G_c}(\ast; \QQ)$ is concentrated in even weights; in fact, we may identify $\spec \H^0_{G_c}(\ast; k) \cong \fr{t}\mmod W$. It is still reasonable to define $\Loc_{G_c}^\gr(\Gr_G; k)$ to be
    $$\Loc_{G_c}^\gr(\Gr_G; k) = \coLMod_{\H_0^{G_c}(\Gr_G; k)}(\QCoh(\fr{t}\mmod W)).$$
    Similarly, the $\infty$-category $\Loc_{G_c}^\gr(G_c; k)$ can be defined as
    $$\Loc_{G_c}^\gr(G_c; k) = \LMod_{\H_0^{G_c}(\Gr_G; k)}(\QCoh(\fr{t}\mmod W)).$$
\end{remark}
\begin{example}\label{ex: graded torus satake}
    If $G = T$ is a maximal torus, it follows from \cref{thm: torus satake} that there are equivalences of $\pi_0(k)$-linear $\infty$-categories
    \begin{align*}
        \Loc_{T_c}^\gr(\Gr_T; k) & \simeq \QCoh(\cM_{T,0}/\ld{T}), \\
        \Loc_{T_c}^\gr(T_c; k) & \simeq \QCoh(\cM_{T,0} \times_{\spec \pi_0(k)} \ld{T}).
    \end{align*}
\end{example}

Suppose $X$ is a (ind-)finite $T_c$-space with even cells. Since $\Loc_{T_c}^\gr(X; k)$ is a degeneration of $\Loc_{T_c}(X; k)$, one should expect a spectral sequence computing the cohomology $\Gamma_{T_c}(X; \cf)$ for $\cf \in \Loc_{T_c}(X; k)$ from corresponding objects $\cf^\gr\in \Loc_{T_c}^\gr(X; k)$. Similarly, if $Y$ is a finite $T_c$-space such that $\Omega Y$ has even cells, one should expect a spectral sequence computing the cohomology $\Gamma_{T_c}(Y; \cf)$ for $\cf \in \Loc_{T_c}(Y; k)$ from corresponding objects $\cf^\gr\in \Loc_{T_c}^\gr(Y; k)$. This is a special case of the following general setup.
\begin{construction}\label{cstr: sseq degeneration cohomology}
    Recall that if $\fr{X}$ is a spectral stack and $\cf \in \QCoh(\fr{X})$, the truncation $\ul{\tau}_{\geq n}(\cf)$ is the quasicoherent $\co_{\fr{X}}$-module given on an affine open $U$ by $\tau_{\geq n}(\cf(U))$; similarly for $\ul{\tau}_{\leq n}$ and $\ul{\tau}_{[n,m]}$ with $m\geq n$.
    There is a functor $\QCoh(\cM_T) \to \QCoh(\cM_{T,0})$ given by sending a quasicoherent sheaf $\cf$ on $\cM_T$ to the quasicoherent sheaf $\ul{\tau}_{[0,1]}(\cf)$ over $\cM_{T,0}$. This functor can be expressed as the composite of two functors: the first sends the $\co_{\cM_T}$-module $\cf$ to the filtered  $\ul{\tau}_{\geq 2\star} \co_{\cM_T}$-module $\ul{\tau}_{\geq 2\star}(\cf)$; and the second is given by taking associated graded. Note that since the structure sheaf $\co_{\cM_T}$ is $2$-periodic, the data of the graded $\ul{\pi}_{2\ast} \co_{\cM_T}$-module $\gr(\ul{\tau}_{\geq 2\star}(\cf))$ is equivalent to the data of the (ungraded) $\co_{\cM_{T,0}}$-module $\ul{\tau}_{[0,1]}(\cf)$.
    
    Let $\cA$ be an $\Eoo$-coalgebra in $\QCoh(\cM_T)$ whose homotopy sheaves are concentrated in even degrees (such as $\cf_T(X)^\vee$).  If $\cf \in \coMod_\cA(\QCoh(\cM_T))$, the comodule map $\cf \to \cf \otimes_{\co_{\cM_T}} \cA$ induces a comodule map 
    $$\tau_{\leq 2\star} \cf \to \tau_{\leq 2\star} (\cf \otimes_{\co_{\cM_T}} \cA) \to \tau_{\leq 2\star} (\cf) \otimes_{\tau_{\leq 2\star}(\co_{\cM_T})} \tau_{\leq 2\star}(\cA)$$
    due to the oplax symmetric monoidality of the truncation functor. Taking associated graded and using the $2$-periodicity of $\co_{\cM_T}$, we obtain a $\pi_0(\cA)$-comodule structure on the $\co_{\cM_{T,0}}$-module $\ul{\tau}_{[0,1]}(\cf)$. This defines a functor $\coMod_\cA(\QCoh(\cM_T)) \to \coMod_{\pi_0(\cA)}(\QCoh(\cM_{T,0}))$, which we will denote by $\cf \mapsto \cf^\gr$. For instance, if $\cA = \cf_T(X)^\vee$ and $\cf \in \Loc_{T_c}(X; k) = \coMod_\cA(\QCoh(\cM_T))$, then there is a spectral sequence
    \begin{equation}\label{eq: sseq for coh of sheaf from loc gr}
        \pi_\ast(k) \otimes_{\pi_0(k)} \pi_\ast \Map_{\Loc_{T_c}^\gr(X; k)}(\ul{k}^\gr, \cf^\gr) \Rightarrow \pi_\ast \Map_{\Loc_{T_c}(X; k)}(\ul{k}, \cf) = \pi_\ast \Gamma_{T_c}(X; \cf).
    \end{equation}

    Similarly, let $\cB$ be an $\E{1}$-algebra in $\QCoh(\cM_T)$ whose homotopy sheaves are concentrated in even degrees (such as $\cf_T(\Omega Y)^\vee$).  If $\cf \in \LMod_\cB(\QCoh(\cM_T))$, the module map $\cB \otimes_{\co_{\cM_T}} \cf \to \cf$ induces a comodule map 
    $$\tau_{\geq 2\star} (\cB) \otimes_{\tau_{\geq 2\star}(\co_{\cM_T})} \tau_{\geq 2\star}(\cf) \simeq \tau_{\geq 2\star} (\cB \otimes_{\co_{\cM_T}} \cf) \to \tau_{\geq 2\star}(\cf)$$
    due to the lax symmetric monoidality of the cotruncation functor. Taking associated graded and using the $2$-periodicity of $\co_{\cM_T}$, we obtain a left $\pi_0(\cB)$-module structure on the $\co_{\cM_{T,0}}$-module $\ul{\tau}_{[0,1]}(\cf)$. This defines a functor $\LMod_\cB(\QCoh(\cM_T)) \to \LMod_{\pi_0(\cB)}(\QCoh(\cM_{T,0}))$, which we will denote by $\cf \mapsto \cf^\gr$. For instance, if $\cB = \cf_T(\Omega Y)^\vee$ and $\cf \in \Loc_{T_c}(Y; k) = \LMod_\cB(\QCoh(\cM_T))$, then there is a spectral sequence
    $$\pi_\ast(k) \otimes_{\pi_0(k)} \pi_\ast \Map_{\Loc_{T_c}^\gr(Y; k)}(\ul{k}^\gr, \cf^\gr) \Rightarrow \pi_\ast \Map_{\Loc_{T_c}(Y; k)}(\ul{k}, \cf) = \pi_\ast \Gamma_{T_c}(Y; \cf).$$
\end{construction}

Let us now discuss how one might define analogous degenerations if $k$ is not necessarily an even and $2$-periodic $\Eoo$-ring. Although this discussion can be generalized to some other $\Eoo$-rings (such as $\TMF$), we will focus only on the case when $k$ is the $\Eoo$-ring $\KO$ of \textit{real K-theory}. Here is a brief summary of its relevant properties: $\KO$ can be defined from $\KU$ using the $\Z/2$-action on $\KU$ via complex conjugation. Namely, $\KO = \KU^{h\Z/2}$; in fact, as proved in \cite{rognes}, the map $\KO \to \KU$ is a $\Z/2$-Galois extension, meaning that the base-change of any $\KO$-module to $\KU$ acquires the structure of a $\Z/2$-equivariant $\KU$-module. In the discussion below, we will not need to know much about $\KO$, other than the following facts: the generator of $\Z/2$ sends the Bott class $\beta \in \pi_2(\KU)$ to $-\beta$; and the homotopy groups of $\KO$ are \textit{not} even, nor are they $2$-periodic\footnote{In fact, there is an isomorphism
$$\pi_\ast(\KO) \cong \Z[\eta, 2\beta^2, \beta^{\pm 4}]/(2\eta, \eta^3, \eta \cdot (2\beta^2), (2\beta)^2 - 4\beta^4),$$
where $\eta$ is in degree $1$, $2\beta^2$ is in degree $4$, and $\beta^4$ is in degree $8$. The map $\pi_\ast(\KO) \to \pi_\ast(\KU) \cong \Z[\beta^{\pm 1}]$ kills $\eta$, and sends the other classes to their eponyms.}. Therefore, $\KO$ does not quite fit into the setup of \cref{sec: equiv coh} and \cref{sec: degenerations}. Nevertheless, the fact that $\KO$ is the homotopy fixed points $\KU^{h\Z/2}$ does admit a spectral algebro-geometric description: the global sections of the spectral stack $\spec(\KU)/(\Z/2)$ can be identified with $\KO$. Moreover, any $\KO$-module $N$ defines a quasicoherent sheaf over this spectral stack given by the $\Z/2$-action on $\KU \otimes_\KO N$.

Therefore, a more reasonable analogue of the degeneration from a $\KU$-module $M$ to $\pi_\ast(M)$ for a $\KO$-module $N$ is given by considering the graded $\Z/2$-equivariant $\pi_\ast(\KU)$-module $\pi_\ast(\KU \otimes_\KO N)$. If $\KU \otimes_\KO N$ is even, then (since $\pi_\ast(\KU)$ is isomorphic to $\Z[\beta^{\pm 1}]$ with $\beta$ in weight $2$), we may simply view this as the data of the $\Z/2$-equivariant abelian group $\pi_0(\KU \otimes_\KO N)$. That is, studying (spectral) algebraic geometry over $\KO$ amounts simply to keeping track of $\Z/2$-equivariance for (spectral) algebraic geometry over $\KU$. Moreover, the analogue of the degeneration of the spectral scheme $\spec \KU$ to $\spec(\pi_\ast(\KU))/\GG_m \cong \spec(\Z)$ should be understood as a degeneration of the spectral scheme $\spec \KO$ to the $\GG_m$-quotient of $\spec(\pi_\ast(\KU))/(\Z/2) \cong \GG_m/(\Z/2)$, i.e., to the classifying stack $B\Z/2$. Note that if we identify $\Z/2$ with $\spec \Map(\Z/2, \Z) = \spec \Z[a]/(a^2-a)$, where $a$ is the delta function at the non-identity element of $\Z/2$, then the action of $\Z/2$ on $\pi_\ast(\KU)$ is given by the coaction
\begin{equation}\label{eq: Z/2 coaction on KU}
    \Z[\beta^{\pm 1}] \to \Z[\beta^{\pm 1}, a]/(a^2 - a), \ \beta \mapsto (1 - 2a) \beta.
\end{equation}

Motivated by this discussion, we may define $\KO_{T_c}$ for a compact torus $T_c$ as the homotopy $\Z/2$-fixed points of $\KU_{T_c}$ for a particular $\Z/2$-action extending the action of $\Z/2$ on $\KU^{hT_c}$ by complex conjugation. To do so, we need the following simple observation.
\begin{lemma}\label{lem: cplx conj and inversion}
    Under the isomorphism $\pi_0(\KU^{hS^1}) \cong \Z\pw{q-1}$, the action of $\Z/2$ by complex conjugation sends $q\mapsto q^{-1}$. In other words, the action of $\Z/2$ on $\pi_0(\KU^{hS^1})$ is given by the coaction
    $$\Z\pw{q-1} \to \Z\pw{q-1}[a]/(a^2-a), \ q \mapsto q^{1-2a}.$$
\end{lemma}
Motivated by \cref{lem: cplx conj and inversion}, we make the following:
\begin{construction}
    There is an action of $\Z/2$ on the multiplicative group $(\GG_m)_\KU$ over $\KU$ given by inversion. If $T_c$ is a compact torus, this extends to an action of $\Z/2$ on $\cM_T^\KU = T_\KU$. Define $\cM_T^\KO$ to be the spectral stack over $\spec(\KU)/(\Z/2)$ given by $\cM_T^\KU/(\Z/2)$. Observe that the underlying stack of $\cM_T^\KO$ is given by $\cM_{T,0}/(\Z/2)$ over $B\Z/2$ (again, $\Z/2$ acts on $\cM_{T,0} \cong T$ by inversion).
    
    It is clear from \cref{cstr: def-equiv-coh} that the functor $\cf_T(-; \KU): \Top(T_c)^\op \to \QCoh(\cM_T^\KU)$ factors through a functor $\Top(T_c)^\op \to \QCoh(\cM_T^\KO)$. We will denote this new functor by $\cf_T(-; \KO)$. In exactly the same way as in \cref{cstr: def-loc}, one can define a $\QCoh(\cM_T^\KO)$-linear $\infty$-category $\Loc_{T_c}(X; \KO)$ for a finite $T_c$-space $X$. As in \cref{rmk: loc and comod}, there will be an equivalence
    $$\Loc_{T_c}(X; \KO) \simeq \coMod_{\cf_T(X; \KO)^\vee}(\QCoh(\cM_T^\KO));$$
    furthermore, the latter category is equivalent to the $\infty$-category of $\Z/2$-equivariant objects in $\Loc_{T_c}(X; \KU)$.
\end{construction}
Thus, following \cref{def: graded Loc}, we are led to the following.
\begin{definition}\label{def: KO graded Loc}
    Suppose that $X$ is a (ind-)finite $T_c$-space with even cells (such as $\Gr_G$).
    The $\infty$-category $\Loc_{T_c}^\gr(X; \KO)$ is defined as
    $$\Loc_{T_c}^\gr(X; \KO) = \coLMod_{\pi_0(\cf_T(X; \KU)^\vee)}(\QCoh(\cM_{T,0}/(\Z/2))).$$
    Similarly, suppose $Y$ is a finite $T_c$-space such that $\Omega Y$ has even cells (such as $G_c$). The $\infty$-category $\Loc_{T_c}^\gr(Y; \KO)$ is defined as
    $$\Loc_{T_c}^\gr(Y; \KO) = \LMod_{\pi_0(\cf_T(\Omega Y; \KU)^\vee)}(\QCoh(\cM_{T,0}/(\Z/2))).$$
    These categories admit an interesting grading (unlike the analogues with $\KU$-coefficients): the stack $\cM_{T,0}/(\Z/2) = T/(\Z/2)$ lives over $B\GG_m$ via the composite $T/(\Z/2) \to B\Z/2 \to B\GG_m$ where the final map classifies the sign representation of $\Z/2$. We will denote the resulting line bundle over $T/(\Z/2)$ by $\omega$.
\end{definition}
Just as in \cref{eq: sseq for coh of sheaf from loc gr}, if $\cf \in \Loc_{T_c}(X; \KO)$, there is a spectral sequence
\begin{equation}\label{eq: sseq for coh of sheaf from loc gr KO}
    E_2^{\ast,\ast} \cong \pi_\ast \Map_{\Loc_{T_c}^\gr(X; \KO)}(\ul{\KO}^\gr, \cf^\gr \otimes \omega^{\otimes \ast}) \Rightarrow \pi_\ast \Map_{\Loc_{T_c}(X; \KO)}(\ul{k}, \cf) = \pi_\ast \Gamma_{T_c}(X; \cf).
\end{equation}
There is an isomorphism
$$E_2^{\ast,\ast} \cong \H^\ast(B\Z/2, \pi_\ast \Map_{\Loc_{T_c}^\gr(X; \KU)}(\ul{\KU}^\gr, \cf^\gr)[\beta^{\pm 1}]),$$
where $\Z/2$ acts on $\beta$ by negation.
\begin{example}
    It follows from \cref{ex: graded torus satake} that there are equivalences of $\QCoh(B\Z/2)$-linear $\infty$-categories
    \begin{align*}
        \Loc_{T_c}^\gr(\Gr_T; \KO) & \simeq \QCoh(T/(\Z/2) \times B\ld{T}), \\
        \Loc_{T_c}^\gr(T_c; \KO) & \simeq \QCoh(T/(\Z/2) \times \ld{T}).
    \end{align*}
\end{example}


Before proceeding to describing an analogue of the above picture with $\KO$ replaced by the $K(1)$-local sphere, we will describe $\KO_T = \Gamma(\cM_T^\KO; \co)$ for the sake of completeness. There is a spectral sequence 
\begin{equation}\label{eq: sseq KO T}
    E_2^{s,\ast} \cong \H^s(\Z/2; \co_T[\beta^{\pm 1}]) \cong \H^s(\bV(\omega^{-1})^\times; \co) \Rightarrow \pi_{\ast-s}(\KO_T),
\end{equation}
where $\ast$ denotes the grading on $\co_T[\beta^{\pm 1}]$ (so $\beta$ is in weight $2$).
Here, $\bV(\omega^{-1})^\times$ is the complement of the zero section in the total space of the line bundle $\omega^{-1}$ over $T/(\Z/2)$. The action of $\Z/2$ on $\co_T[\beta^{\pm 1}]$ is given by inversion on $T$, and sends $\beta \mapsto -\beta$. One can view \cref{eq: sseq KO T} as the spectral sequence \cref{eq: sseq for coh of sheaf from loc gr KO} computing the cohomology of the constant sheaf on a point. As we will explain below, this spectral sequence has nontrivial differentials, so it does not immediately collapse.

For simplicity, we will focus on the case $T = S^1$, so $\co_T = \Z[x^{\pm 1}]$. Then an elementary calculation in group cohomology shows that the $E_2$-page of \cref{eq: sseq KO T} is given by
$$E_2^{\ast,\ast} \cong \Z[\eta, \beta^{\pm 2}, x + x^{-1}, \tfrac{x^n - x^{-n}}{\beta}]_{n\geq 1}/2\eta,$$
where all classes except for $\eta$ lie in $E_2^{0,\ast}$, and $\eta \in E_2^{1,2}$. A standard calculation in homotopy theory (coming from the analysis of the Adams-Novikov spectral sequence) says that there is a differential $d_3(\beta^2) = \eta^3$. There are no further differentials past this point, and propagating this differential shows:
\begin{prop}\label{prop: htpy KOS1}
    There is an isomorphism
    $$\pi_\ast(\KO_{S^1}) \cong \Z[\eta, 2\beta^2, \beta^{\pm 4}, x + x^{-1}, \tfrac{x^n - x^{-n}}{\beta}]_{n\geq 1}/(2\eta, \eta^3, \eta \cdot (2\beta^2), (2\beta^2)^2 = 4\beta^4),$$
    where the terms $\eta, 2\beta^2, \beta^{\pm 4}$ simply contribute a copy of $\pi_\ast(\KO)$, the term $x + x^{-1}$ contributes a class to $\pi_0(\KO_{S^1})$, and the terms $\tfrac{x^n - x^{-n}}{\beta}$ contribute infinitely many classes to $\pi_2(\KO_{S^1})$.
\end{prop}
It is hard to extract concrete implications\footnote{This is not to say that computing $\KO$-(co)homology groups is a worthless endeavor: in \cite{adams-vector-fields}, Adams famously computed the $\KO$-cohomology of real projective spaces to solve the question of counting linearly independent vector fields on spheres.} for Langlands duality from the structure of $\pi_\ast(\KO_{S^1})$; so we will not compute the homotopy groups of $\pi_0(\cf_T(\Gr_G; \KU)^\vee)$ below, and content ourselves with just describing the $\Z/2$-action on $\pi_0(\cf_T(\Gr_G; \KU)^\vee)$.

\begin{remark}\label{rmk: connective ko def}
    The above story can be extended to include the case of \textit{connective} real K-theory $\ko = \tau_{\geq 0}(\KO)$, too. Since we will only return to this picture occasionally in this article, we will be scant on details. The article \cite{ku-rel-langlands} studied a variant of Langlands duality with coefficients in connective complex K-theory $\ku$, which is an $\Eoo$-ring such that $\pi_\ast(\ku) = \Z[\beta]$ with $\beta$ in degree $2$ (so that $\ku/\beta = \Z$ and $\ku[\beta^{-1}] = \KU$). Its $S^1$-equivariant version $\ku_{S^1}$ has homotopy groups given by $\pi_\ast(\ku_{S^1}) \cong \Z[\beta, x, \tfrac{1}{1+\beta x}]$ with $x$ in weight $-2$. Let $\GG_\beta = \spec \pi_\ast(\ku_{S^1})/\GG_m$, where the group law is given by $x + y + \beta xy$. If $T$ is a torus, let $T_\beta = \Hom(\bX^\ast(T), \GG_\beta)$.

    Since $\ku$ is the connective cover $\tau_{\geq 0}(\KU)$ of $\KU$, the action of $\Z/2$ on $\KU$ by complex conjugation lifts to an action of $\Z/2$ on $\ku$. While there is a map $\ko \to \ku^{h\Z/2}$, this map is \textit{not} an equivalence; rather, it exhibits $\ko$ as the connective cover $\tau_{\geq 0}(\ku^{h\Z/2})$. In particular, while the $\Eoo$-ring $\ku \otimes_\ko \ku$ is not equivalent to $\Map(\Z/2, \ku)$, it is still a finite free $\ku$-algebra with even homotopy.
    Therefore, the appropriate degeneration of the spectral scheme $\spec(\ko)$ is no longer the algebraic stack $(\spec(\pi_\ast(\ku))/\GG_m)/(\Z/2)$, but is rather given by the stack\footnote{The reason for the notation ``$\spev$'' will be explained in a future article, and uses the even filtration of \cite{even-filtr, piotr-even-filtr}.} $\spev(\ko)$ defined as the quotient by $\GG_m$ of the geometric realization of the simplicial stack
    % https://q.uiver.app/#q=WzAsNCxbMiwwLCJcXHNwZWMoXFxwaV9cXGFzdChcXGt1XntcXG90aW1lc19cXGtvIDJ9KSkvXFxHR19tIl0sWzMsMCwiXFxzcGVjKFxccGlfXFxhc3QoXFxrdSkpL1xcR0dfbSJdLFsxLDAsIlxcc3BlYyhcXHBpX1xcYXN0KFxca3Vee1xcb3RpbWVzX1xca28gM30pKS9cXEdHX20iXSxbMCwwLCJcXGNkb3RzIl0sWzAsMSwiIiwwLHsib2Zmc2V0IjotMX1dLFswLDEsIiIsMix7Im9mZnNldCI6MX1dLFsyLDBdLFsyLDAsIiIsMix7Im9mZnNldCI6Mn1dLFsyLDAsIiIsMix7Im9mZnNldCI6LTJ9XSxbMywyLCIiLDIseyJvZmZzZXQiOi0xfV0sWzMsMiwiIiwyLHsib2Zmc2V0IjotM31dLFszLDIsIiIsMix7Im9mZnNldCI6M31dLFszLDIsIiIsMix7Im9mZnNldCI6MX1dXQ==
    $$\begin{tikzcd}
    	\cdots & {\spec(\pi_\ast(\ku^{\otimes_\ko 3}))} & {\spec(\pi_\ast(\ku^{\otimes_\ko 2}))} & {\spec(\pi_\ast(\ku))}.
    	\arrow[shift left, from=1-1, to=1-2]
    	\arrow[shift left=3, from=1-1, to=1-2]
    	\arrow[shift right=3, from=1-1, to=1-2]
    	\arrow[shift right, from=1-1, to=1-2]
    	\arrow[from=1-2, to=1-3]
    	\arrow[shift right=2, from=1-2, to=1-3]
    	\arrow[shift left=2, from=1-2, to=1-3]
    	\arrow[shift left, from=1-3, to=1-4]
    	\arrow[shift right, from=1-3, to=1-4]
    \end{tikzcd}$$
    A standard calculation says that $\pi_\ast(\ku \otimes_\ko \ku) \cong \Z[\beta, r]/(r^2 - r\beta)$ with $r$ in weight $2$, and that the two maps $\eta_L, \eta_R: \ku \rightrightarrows \ku \otimes_\ko \ku$ send $\eta_L: \beta \mapsto \beta$ and $\eta_R: \beta \mapsto \beta - 2r$. Upon inverting $\beta$, we may identify $\pi_\ast(\ku \otimes_\ko \ku)[\beta^{-1}]$ with $\Z[\beta^{\pm 1}, a]/(a^2 - a)$ where $a = r\beta^{-1}$, and then $\eta_R$ is precisely the coaction from \cref{eq: Z/2 coaction on KU}. As described in \cite[Section 9]{tmf}, $\spev(\ko)$ classifies isomorphism classes of possibly singular quadratic curves (which are locally of the form $y = x^2 + \beta x$).
    
    Note that 
    $$\eta_R(\beta^n) = \beta^n + ((-1)^n - 1) r\beta^{n-1},$$
    so $\beta^{2n}$ is a well-defined function on $\spev(\ko)$ for any $n\geq 0$; the complement of its vanishing locus is precisely $B\Z/2$.
    Just as $B\Z/2$ is an open substack in $\spev(\ko)$, the stack $\cM^\KO_T$ is also open in a certain stack $\cM_T^\ko$, which can be defined as the stack of homomorphisms from $\bX^\ast(T)$ to the quotient of $\GG_\beta$ (viewed as a scheme over $\spec(\Z[\beta])/\GG_m$) by the coaction of $\pi_\ast(\ku \otimes_\ko \ku)$ given by
    \begin{equation}\label{eq: coaction connective ku on Gbeta}
        \Z[\beta, x, \tfrac{1}{1+\beta x}] \to \Z[\beta, x, \tfrac{1}{1+\beta x}, r]/(r^2 - \beta r), \ x\mapsto x - \tfrac{rx^2}{1+\beta x}.
    \end{equation}
    This might look a bit strange, but it is a pleasant exercise to verify (using the binomial formula) that upon inverting $\beta$, it identifies with the map
    $$\Z[\beta^{\pm 1}, x, \tfrac{1}{1+\beta x}] \to \Z[\beta^{\pm 1}, x, \tfrac{1}{1+\beta x}, a]/(a^2 - a), \ (1+\beta x)\mapsto (1+\beta x)^{1-2a}$$
    as forced by \cref{lem: cplx conj and inversion}.
    In any case, given the stack $\cM_T^\ko$, one can define $\QCoh(\cM_T^\ko)$-linear $\infty$-categories $\Loc_{T_c}^\gr(X; \ko)$ exactly as in \cref{def: KO graded Loc}. We will return to this below in \cref{sec: KU coeff}.
\end{remark}

Finally, we turn to the $K(1)$-local sphere. To motivate it, note that the action of complex conjugation on $\KU$ is given simply by the action of the Adams operation $\psi^{-1}$. It is therefore natural to wonder about the action of other Adams operations. To this end, we will fix a prime $p$ and contemplate a parallel story with $\KO$ replaced by the ``image of J''/$K(1)$-local sphere spectrum $L_{K(1)} S^0 = (\KU^\wedge_p)^{h\Z_p^\times}$, where $\Z_p^\times$ acts continuously on $\KU^\wedge_p$ by Adams operations: there is a map $\Z_p^\times \to \Aut_\Eoo(\KU^\wedge_p)$ sending $n \in \Z_p^\times$ to the Adams operation $\psi^n: \KU^\wedge_p \to \KU^\wedge_p$. (In fact, this map is an equivalence!)

The homotopy groups of $L_{K(1)} S^0$ are somewhat complicated\footnote{Explicitly, if $p>2$, then $\pi_i L_{K(1)} S^0$ is isomorphic to $\Z_p$ when $i=0,-1$, and is isomorphic to $\Z/p^{v_p(j)+1}$ for $i = 2(p-1)j - 1$. The order of the latter subgroup is precisely the $p$-part of the denominator of $B_{2(i+1)}/(i+1)$, where $B_{2j}$ is the $2j$th Bernoulli number.}, but just as with $\KO$, studying (spectral) algebraic geometry over $L_{K(1)} S^0$ amounts simply to keeping track of $\Z_p^\times$-equivariance for (spectral) algebraic geometry over $\KU^\wedge_p$. That is to say, $L_{K(1)} S^0$ is the global sections of the structure sheaf on the spectral stack $\spf(\KU^\wedge_p)/\Z_p^\times$. Moreover, the analogue of the degeneration of the spectral scheme $\spf \KU^\wedge_p$ to $\spf(\pi_\ast(\KU^\wedge_p))/\GG_m \cong \spf(\Z_p)$ should be understood as a degeneration of the spectral scheme $\spf L_{K(1)} S^0$ to the $\GG_m$-quotient of $\spf(\pi_\ast(\KU^\wedge_p))/\Z_p^\times$, i.e., to the classifying stack $B\Z_p^\times$.

To define an analogue of \cref{def: KO graded Loc} for $L_{K(1)} S^0$, we need to upgrade the $\Z_p^\times$-action on $\KU$ to an action on equivariant K-theory. Recall that if $\cT$ denotes the full subcategory of $\Top$ spanned by those spaces which are homotopy equivalent to $BT_c$ with $T_c$ being a compact abelian Lie group, the data of a preorientation of $\GG = \GG_m$ is equivalent to the data of a functor $\cM: \cT \to \Aff_\KU$ along with compatible equivalences $\cM(BT_c) \simeq \cM_T$. This can be composed with the functor $\Aff_\KU \to \Aff_{\KU^\wedge_p}^{p\cpl}$ of $p$-completion. 

Unfortunately, even at the level of classical algebra, there is no natural action of $\Z_p^\times$ on $\GG_m = \spf \Z_p[x^{\pm 1}]$ where $n\in \Z_p^\times$ sends $x\mapsto x^n$: the power series $x^n = \sum_{i\geq 0} \binom{n}{i} (x-1)^n$ need not converge without a further completion. Nevertheless, such an action of $\Z_p^\times$ does exist if we restrict to the subgroups $\mu_{p^n} = \spf \Z_p[\Z/p^n] \subseteq \GG_m$; in fact, the action factors through the surjection $\Z_p^\times \twoheadrightarrow (\Z/p^n)^\times$. The subgroups $\mu_{p^n}$ naturally lift to $\KU$ (by $\spec \KU[\Z/p^n]$), and each admits a natural $\Z_p^\times$-action. Of course, these $\Z_p^\times$-actions exist even before $p$-completion; but to get a well-behaved operation on $\Z/p^n$-equivariant $\KU$-cohomology, we need the $\Z_p^\times$-action to preserve the preorientation on $\mu_{p^n}$, and this in turn happens once $\KU$ is $p$-completed. 

Suppose, therefore, that we restrict to the full subcategory $\cT_p \subseteq \cT$ spanned by those spaces which are homotopy equivalent to $BA$ with $A$ being a $p$-power torsion compact abelian Lie group. Then the preceding paragraph implies that $\cM|_{\cT_p}: \cT_p \to \Aff_\KU$ refines to a functor $\cT_p \to (\Aff_{\KU^\wedge_p}^{p\cpl})^{h\Z_p^\times}$. Following \cref{cstr: def-equiv-coh} verbatim defines an action of $\Z_p^\times$ on $\cM_A$, and furthermore equips the quasicoherent sheaf $\cf_A(X) \in \QCoh(\cM_A)$ associated to a finite $A$-space $X$ with a $\Z_p^\times$-equivariant structure. We will write $\cM_A^{L_{K(1)} S^0} = \cM_A/\Z_p^\times$, and let $\cf_A(-; L_{K(1)} S^0)$ denote the corresponding functor $\Top(A)^\op \to \QCoh(\cM_A^{L_{K(1)} S^0})$. Again, following \cref{def: graded Loc}, we are led to\footnote{Just as with connective $\ko$, one can also define a variant of \cref{def: J graded Loc} for the \textit{connective} image of J spectrum $j$. We leave this to the interested reader.}:
\begin{definition}\label{def: J graded Loc}
    Suppose that $A$ is a $p$-power torsion abelian group, and $X$ is a (ind-)finite $A$-space with even cells (such as $\Gr_G$).
    The $\infty$-category $\Loc_{A}^\gr(X; L_{K(1)} S^0)$ is defined as
    $$\Loc_{A}^\gr(X; L_{K(1)} S^0) = \coLMod_{\pi_0(\cf_A(X; \KU)^\vee)}(\QCoh(\cM_{A,0}/\Z_p^\times)).$$
    Similarly, suppose $Y$ is a finite $A$-space such that $\Omega Y$ has even cells (such as $G_c$). The $\infty$-category $\Loc_{A}^\gr(Y; L_{K(1)} S^0)$ is defined as
    $$\Loc_{A}^\gr(Y; L_{K(1)} S^0) = \LMod_{\pi_0(\cf_A(\Omega Y; \KU)^\vee)}(\QCoh(\cM_{A,0}/\Z_p^\times)).$$
    These categories admit an interesting grading (just like the analogue with $\KO$-coefficients): the stack $\cM_{A,0}/(\Z/2) = A/\Z_p^\times$ lives over $B\GG_m$ via the composite $A/\Z_p^\times \to B\Z_p^\times \to B\GG_m$ where the final map classifies the standard (cyclotomic) representation of $\Z_p^\times$ on $\Z_p$. We will denote the resulting line bundle over $A/\Z_p^\times$ by $\omega$.
\end{definition}
For the sake of completness (and partly because it is a pleasant calculation), let us describe $(L_{K(1)} S^0)_{T[p^\infty]} = \Gamma(\cM_{T[p^\infty]}^{L_{K(1)} S^0}; \co)$ when $p$ is odd. Since this is built as a limit of the spectra $(L_{K(1)} S^0)_{T[p^n]}$, we will just compute each of these individually. There is a spectral sequence 
\begin{equation}\label{eq: sseq K1-local mod pn}
    E_2^{s,\ast} \cong \H^s_\mathrm{cts}(\Z_p^\times; \co_{T[p^n]}[\beta^{\pm 1}]) \cong \H^s(\bV(\omega^{-1})^\times; \co) \Rightarrow \pi_{\ast-s}((L_{K(1)} S^0)_{T[p^n]}),
\end{equation}
where $\ast$ denotes the grading on $\co_{T[p^n]}[\beta^{\pm 1}]$ (so $\beta$ is in weight $2$).
Here, $\bV(\omega^{-1})^\times$ is the complement of the zero section in the total space of the line bundle $\omega^{-1}$ over $A/\Z_p^\times$. Fix a topological generator $g \in \Z_p^\times$ such that $g^{p-1} = 1 + p$, so that its action (denoted $\psi^g$) on $\co_{T[p^n]}[\beta^{\pm 1}]$ is given by exponentiation on $T[p^n]$, and sends $\beta \mapsto g\beta$. One can view \cref{eq: sseq K1-local mod pn} as the spectral sequence \cref{eq: sseq for coh of sheaf from loc gr} computing the cohomology of the constant sheaf on a point. This spectral sequence has no nontrivial differentials, so it collapses; this, however, is no longer true if $p=2$.

For simplicity, we will focus on the case $T = S^1$, so $\co_{T[p^n]} = \co_{\mu_{p^n}} = \Z_p[x^{\pm 1}]/(x^{p^n}-1)$ (recall that we have $p$-completed!). We then have:
\begin{prop}\label{prop: htpy K1-local S0}
    If $p>2$, there are isomorphisms
    $$\pi_j((L_{K(1)} S^0)_{\mu_{p^n}}) \cong \begin{cases}
        \pi_j(L_{K(1)} S^0) \oplus \Z_p^{\oplus n} & j=0,-1 \\
        \pi_j(L_{K(1)} S^0) \oplus \bigoplus_{i=0}^{n-1} \Z_p/kp^{n-i} & j=2k-1, k \in \Z \\
        0 & \text{else}.
    \end{cases}$$
\end{prop}
\begin{proof}
    The $E_2$-page of \cref{eq: sseq K1-local mod pn} is given by the group cohomology of $\Z_p^\times$ acting on $\Z_p[x^{\pm 1}, \beta^{\pm 1}]/(x^{p^n}-1)$, so $E_2^{\ast,2k}$ is given by the cohomology of the two-term complex
    \begin{align*}
        \Z_p[x^{\pm 1}]/(x^{p^n}-1) & \xrightarrow{\psi^g - 1} \Z_p[x^{\pm 1}]/(x^{p^n}-1), \\
        f(x) & \mapsto g^k f(x^g) - f(x).
    \end{align*}
    Let us sketch the calculation of the cohomology of this complex, which we will denote by $C^\bull$ below. Write $\Z_p[x^{\pm 1}]/(x^{p^n} - 1) = \Z_p[\Z/p^n]$, so it is a free $\Z_p$-module on the classes $\{1,x,\cdots,x^{p^n - 1}\}$. The action of $\Z_p^\times$ on $\Z/p^n$ (which factors through the quotient map $\Z_p^\times \twoheadrightarrow (\Z/p^n)^\times$) has $n+1$ orbits, with representatives given by $\{p^i\}_{0\leq i \leq n-1} \cup \{0\}$. The orbit of $0$ is a singleton, and the orbit of $p^i$ has size $p^{n-i-1}(p-1)$. It follows that the $p^n \times p^n$-matrix $\psi^g - 1$ can be written as the block sum $(g^k - 1) \oplus \bigoplus_{i=0}^{n-1} A_i$, where $A_i$ is an $p^{n-i-1}(p-1) \times p^{n-i-1}(p-1)$-matrix. For consistency, we will write $A_{-1}$ to denote the scalar $g^k - 1$.

    Let $0\leq i \leq n-1$. Then the matrix $A_i$ acts on the submodule $\Z_p^{\oplus p^{n-i-1}(p-1)} = \Z_p\{x^{p^i}, x^{p^i g}, \cdots, x^{p^i g^{p^{n-i-1}(p-1) - 1}}\}$, and each row and column of $A_i$ has exactly two entries (namely, $-1$ on the diagonal entry, and $g^k$ elsewhere). Computing the Smith normal form of this matrix shows that $A_i$ has no kernel unless $k=0$, in which case its kernel is free of rank $1$. If $k=0$, then the cokernel of $A_i$ is also free of rank $1$, and if $k\neq 0$, then the cokernel of $A_i$ is $\Z_p/(g^{k p^{n-i-1}(p-1)} - 1)$. Since $g \in \Z_p^\times$ was chosen to satisfy $g^{p-1} = 1+p$, it follows that $\Z_p/(g^{kp^{n-i-1}(p-1)} - 1) \cong \Z_p/kp^{n-i}$.
    
    We only need to take care of the block $A_{-1}$. If $k = 0$, then $A_{-1}$ is the zero matrix; but if $k$ is nonzero, then $A_{-1}$ has no kernel, and has cokernel given by $\Z_p/(g^k - 1)$. It follows that if $k=0$, then
    $$\H^s(C^\bull) \cong \Z_p^{\oplus n+1} \text{ for }s=0,1.$$
    If $k\neq 0$, then
    $$\H^s(C^\bull) \cong \begin{cases}
        0 & s=0\\
        \Z_p/(g^k - 1) \oplus \bigoplus_{i=0}^{n-1} \Z_p/kp^{n-i} & s=1.
    \end{cases}$$
    The groups $E_2^{s,\ast}$ vanish for $s>1$, so there cannot be any differentials in the spectral sequence \cref{eq: sseq K1-local mod pn}. Using the calculation of the homotopy groups of $K(1)$-local sphere, we obtain the desired answer for $\pi_\ast((L_{K(1)} S^0)_{\mu_{p^n}})$.
\end{proof}
Just as with $\KO_{S^1}$, the groups $\pi_\ast((L_{K(1)} S^0)_{\mu_{p^n}})$ are interesting but form a rather unpleasant ring to do algebraic geometry with; so we will content ourselves with just understanding the category $\Loc_{T_c[p^\infty]}^\gr(\Gr_G; L_{K(1)} S^0)$ below (and not calculate actual homotopy groups).