\subsection{Outline and other results}

We now give an overview of the content of this article. In \cref{sec: regular locus}, we briefly review the derived geometric Satake and the Arkhipov-Bezrukavnikov-Ginzburg equivalences, and show how to deduce the corresponding statements \cref{eq: intro reg locus satake} and \cref{eq: intro reg locus abg} over the regular loci. 

Motivated by the issue of \textit{decompleting} Borel-equivariant cohomology (which appears naturally in studying the derived geometric Satake and the Arkhipov-Bezrukavnikov-Ginzburg equivalences), we recall in \cref{sec: equiv coh} the setup of (genuine) equivariant generalized cohomology following \cite{survey, t-equiv-tmf, gepner-meier}. For a compact abelian group $T_c$ and a finite $T_c$-space $X$, we also introduce the category $\Loc_{T_c}(X; k)$ of equivariant local systems of $k$-modules on $X$ for an $\Eoo$-ring $k$ equipped with an ``oriented'' $1$-dimensional commutative group scheme $\GG$. This section also reviews/generalizes the all-important Atiyah-Bott localization theorem \cite{atiyah-bott-localization}, and the corresponding results of Goresky-Kottwitz-MacPherson \cite{gkm-original}.

In \cref{sec: degenerations}, we introduce the degeneration $\Loc_{T_c}^\gr(\Gr_G; k)$ of the $\infty$-category $\Loc_{T_c}(\Gr_G; k)$ that we discussed in the preceding subsection, and also discuss the definition of this $\infty$-category for the non-complex-oriented $\Eoo$-ring $\KO$. The definition of $\Loc_{T_c}^\gr(\Gr_G; k)$ depends only on the equivariant homology $k^{T_c}_\ast(\Gr_G)$ (denoted by $\pi_\ast \cf_{T_c}(\Gr_G)^\vee$ in the body of this article) as a coalgebra/Hopf algebroid over the equivariant cohomology $\pi_\ast k_{T_c}$ of a point.

The next \cref{sec: torus loop rot} is essentially purely combinatorial: it studies the case when $G$ is a torus $T$, and imposes the additional data of loop-rotation equivariance. In terms of the homotopy equivalence $\Gr_G \cong \Omega G_c$, this comes from the $S^1$-action on $\Omega G_c$ obtained by viewing it as $\Omega^2 (BG_c) = \Map_\ast(S^2, BG_c)$ and using the $S^1$-action by rotation on $S^2$. Namely, we show that if $k$ is an $\Eoo$-ring equipped with an ``oriented'' $1$-dimensional commutative group scheme $\GG$, then $k^{T_c \times S^1_\rot}_\ast(\Gr_T)$ (or really, its sheafification over $\Hom(\bX^\ast(T), \GG)$) can be identified with a \textit{$\GG$-analogue} of the Weyl algebra of the Langlands dual torus $\ld{T}$. For instance, if $k = \QQ[u^{\pm 1}]$ and $\GG = \GG_a$, then $k^{T_c \times S^1_\rot}_\ast(\Gr_T)$ is the rescaled Weyl algebra of $\ld{T}$; similarly, if $k = \KU$ and $\GG = \GG_m$, then $k^{T_c \times S^1_\rot}_\ast(\Gr_T)$ is the $q$-Weyl algebra of the dual torus $\ld{T}$. We also explain the relationship between this {$\GG$-analogue} of the Weyl algebra of $\ld{T}$ and the ``$F$-de Rham complex'' of \cite{generalized-n-series}.

In \cref{sec: review Q coeff}, we review the classical story concerning $\Loc_{T_c}^\gr(\Gr_G; k)$ when $k = \QQ[u^{\pm 1}]$. The purpose of this section is to reprove the results of \cite{bfm, homology-langlands} using only techniques amenable to generalization to other equivariant cohomology theories. In particular, in \cref{cor: reg locus ordinary ABG}, we reprove the equivalence between $\Loc_{T_c}^\gr(\Gr_G; k)$ and $\QCoh(\tilde{\ld{\g}}^\reg/\ld{G})$. The remainder of \cref{sec: review Q coeff} is concerned with the question of loop-rotation equivariance. Using results of \cite{ginzburg-kapranov-vasserot-residue-hecke} and \cite{ginzburg-whittaker}, we prove that $\Loc_{G_c \times S^1_\rot}^\gr(\Gr_G; k)$ can be identified with a certain localization of the Harish-Chandra category $\HC^\hbar_{\ld{G}} = U_\hbar(\ld{\g})\modc^{(\ld{G}, \weak)}$. This line of argument is, in some sense, precisely the opposite of that of \cite{lonergan-fourier}.

We finally turn to the K-theoretic story in \cref{sec: KU coeff}. \cref{cor: ku reg locus ordinary ABG} therein states that $\Loc_{T_c}^\gr(\Gr_G; k)$ is equivalent to $\QCoh(\tilde{\ld{G}}^\reg/\ld{G})$ when $G$ is connected, almost simple, and simply-laced; in this section, unlike in \cref{thm: intro omnibus}, the multiplicative Grothendieck-Springer resolution $\tilde{\ld{G}}$ is defined to be $\ld{G} \times^{\ld{B}} B$ (instead of $\ld{G} \times^{\ld{B}} \ld{B}$). This is to be understood as analogous to the usual Grothendieck-Springer resolution $\tilde{\ld{\g}}$, which is defined to be $\ld{G} \times^{\ld{B}} \ld{\fr{n}}^\perp$ (as opposed to $\ld{G} \times^{\ld{B}} \ld{\fr{b}}$). We also briefly study the question of loop-rotation equivariance in \cref{thm: ku loop-rot flag} using degenerate affine nil-Hecke algebras, and phrase some precise expectations about the relationship to the representation theory of quantum groups; but we do not yet know how to prove these statements. In \cref{prop: cplx conj KU and B mod B^}, we also study the effect under Langlands duality of complex conjugation on equivariant K-theory.

The elliptic story is studied in \cref{sec: ell coeff}, where we use the important results of \cite{davis-elliptic-springer} to show in \cref{cor: ell reg locus ordinary ABG} that $\Loc_{\ld{T}_c}^\gr(\Gr_G; k)$ is equivalent to a localization of $\QCoh(\Bun_{\ld{B}}^0(E))$ when $G$ is connected, almost simple, and simply-laced. We also briefly study the question of loop-rotation equivariance in \cref{thm: ell loop-rot flag}, but do not even know how to describe the expected Langlands dual story. It should, however, be related to the representation theory of elliptic quantum groups \cite{felder-elliptic-quantum}.

The remainder of this article is concerned with comparisons to (by now) classical constructions in equivariant algebraic topology. \cref{sec: power operations} studies the effect of ``power operations'' on $k$ under the Langlands duality of \cref{thm: intro omnibus}. These are additional symmetries of the $\Eoo$-ring $k$ which yield the theory of Steenrod operations in ordinary cohomology and Adams operations in K-theory. (This is closely related to, but distinct from, work \cite{lonergan-frob} of Lonergan.) We review how the theory of isogenies of $\GG_0$ controls power operations for $k$, which is used to show that power operations for $k$ identify with natural ``Artin-Schreier'' type maps on $\ld{\g}/\ld{G}$, $\ld{G}/\ld{G}$, and $\Bun_{\ld{G}}^\ss(E)$. As an amusing application, we reprove that the restricted Lie operation vanishes on the nilpotent cone in characteristics at least the Coxeter number (see \cref{prop: pth power zero on nilcone}).

In \cref{sec: brylinski zhang}, we explain how the degeneration of $\Loc_{T_c}(\Gr_G; k)$ to $\Loc_{T_c}^\gr(\Gr_G; k)$ should be viewed as analogous to the Hochschild-Kostant-Rosenberg degeneration of Hochschild homology to differential forms. (See \cite{raksit} for a modern take on this degeneration.) Using this perspective, we show how (when $G$ has torsion-free fundamental group) \cref{thm: intro omnibus} recovers results of \cite{brylinski-zhang} identifying the conjugation-equivariant cohomologies $\H^\ast_{G_c}(G_c; \QQ)$ (resp. $\KU^\ast_{G_c}(G_c)$) with the algebra of K\"ahler differentials on $\fr{t}\mmod W$ (resp. $T\mmod W$); the same argument also describes the equivariant elliptic cohomology of $G_c$ in terms of the algebra of K\"ahler differentials on the moduli \textit{space} of semistable degree zero $G$-bundles on the elliptic curve.

In \cref{sec: lifting SL2}, we show by explicit calculation that the group scheme $\SL_2$ (as well as other classical geometric objects, like Grassmannians which are not projective spaces) cannot admit a natural lifting from $\Z$ to the sphere spectrum or even to connective complex K-theory. The argument goes by showing that $\SL_2$ cannot admit a reasonable $\delta$-structure (even mod $p$). This supplements a comment made in the beginning of \cref{sec: degenerations}.

The results of this article were motivated by the work of Hopkins-Kuhn-Ravenel \cite{hkr} describing the generalized equivariant cohomology of \textit{finite} groups. In \cref{sec: comparison hkr}, we briefly review their results (and the corresponding categorifications, due to Lurie \cite{elliptic-iii}). Despite the case of finite groups being the diametric opposite to the case of connected compact Lie groups studied in this article, we give a heuristic argument showing that \cref{thm: intro omnibus} can be viewed as an analogue of (some of) the results of \cite{hkr, elliptic-iii}.

Another motivation for the results of this article came from physics. Namely, the equivariant homology of $\Gr_G$ describes the Coulomb branch of $3$d $\cN=4$ pure gauge theory \cite{bfn-ii}, and one expects that the generalized equivariant $\KU$-homology (resp. elliptic homology) of $\Gr_G$ is related to the Coulomb branch of $4$d $\cN=2$ (resp. $5$d $\cN = 1$) pure gauge theory. We briefly review this story in \cref{sec: coulomb}, and give explicit generators and relations for the Coulomb branches of $3$d $\cN = 4$ and $4$d $\cN = 2$ pure gauge theories with gauge groups $\SL_2$ and $\PGL_2$. The $4$-dimensional case is a $q$-analogue of the quantization of the Atiyah-Hitchin manifold \cite{atiyah-hitchin} from \cite[Equation 5.51]{bullimore-dimofte-gaiotto}.