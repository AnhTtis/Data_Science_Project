\subsection{The derived geometric Satake equivalence}

Let $G$ be a semisimple algebraic group over $\cc$, and let $\Gr_G$ denote the affine Grassmannian, defined as the quotient $G(\cc\ls{t})/G(\cc\pw{t}) =: G\ls{t}/G\pw{t}$. The geometric Satake equivalence of Mirkovic-Vilonen states:
\begin{theorem}[{Mirkovic-Vilonen, \cite{mirkovic-vilonen}}]\label{thm: mirkovic-vilonen}
    Let $\ld{G}_\Z$ denote the smooth split reductive group scheme over $\Z$ whose root datum is dual to that of $G$.
    Then there is an equivalence of symmetric  monoidal abelian categories
    $$\mathrm{Perv}_{G\pw{t}}(\Gr_G; \Z) \simeq \Rep(\ld{G}_\Z).$$
\end{theorem}
The abelian category $\mathrm{Perv}_{G\pw{t}}(\Gr_G; \Z)$ arises as the heart of a $t$-structure on the stable $\infty$-category $\Shv^c_{G\pw{t}}(\Gr_G; \Z)$ of $G\pw{t}$-equivariant constructible sheaves on $\Gr_G$. However, \cref{thm: mirkovic-vilonen} does not lift to an equivalence of stable $\infty$-categories between $\Shv^c_{G\pw{t}}(\Gr_G; \Z)$ and the derived $\infty$-category of $\Rep(\ld{G}_\Z)$. Nevertheless, one has:
\begin{theorem}[{Bezrukavnikov-Finkelberg, \cite{bf-derived-satake}}]\label{thm: intro bf derived satake}
    Let $\ld{G} = \ld{G}_\cc$. There is a $\cc$-linear equivalence of monoidal stable $\infty$-categories
    $$\Shv^c_{G\pw{t}}(\Gr_G; \cc) \simeq \Perf(\ld{\g}^\ast[2]/\ld{G}).$$
    Here, $\ld{\g}^\ast[2]$ denotes the derived scheme given by $\ld{\g}^\ast$ placed in (homological) degree $2$. Using Koszul duality, the right-hand side can be rewritten as
    $$\Perf(\ld{\g}^\ast[2]/\ld{G}) \simeq \Coh((\{0\} \times_{\ld{\g}} \{0\})/\ld{G}).$$
\end{theorem}
In \cite[Theorem 6.6.1]{campbell-raskin-satake}, it was shown that the above equivalence can be refined to an equivalence of monoidal factorization stable $\infty$-categories. (We will not need this refinement here, and only mention it for the sake of completeness.) There are several variants of \cref{thm: intro bf derived satake} which have been proved in the literature; of relevance to us is a theorem from \cite{abg-iwahori-satake} concerning the Iwahori subgroup $I$ of $G\pw{t}$. Namely, fix a Borel subgroup $B\subseteq G$ with unipotent radical $N$, and let $I = G\pw{t} \times_G B$. Then:
\begin{theorem}[{Arkhipov-Bezrukavnikov-Ginzburg, \cite{abg-iwahori-satake}}]\label{thm: intro abg}
    There is an equivalence 
    $$\Shv_I^c(\Gr_G; \cc) \simeq \Coh((\tilde{\ld{\cN}} \times_{\ld{\g}} \{0\})/\ld{G}),$$
    where $\tilde{\ld{\cN}} = T^\ast(\ld{G}/\ld{B})$ is the Springer resolution. Using Koszul duality, the right-hand side can be rewritten as
    $$\Coh((\tilde{\ld{\cN}} \times_{\ld{\g}} \{0\})/\ld{G}) \simeq \Perf(\tilde{\ld{\g}}_\cc[2]/\ld{G}_\cc),$$
    where $\tilde{\ld{\g}}_\cc[2] = \ld{G} \times^{\ld{B}} \ld{\fr{n}}^\perp[2]$ is a shifted analogue of the Grothendieck-Springer resolution. 
\end{theorem}
The equivalences of \cref{thm: intro bf derived satake} and \cref{thm: intro abg} admit simpler analogues, where one considers the full subcategories of the ind-completions of $\Shv^c_{G\pw{t}}(\Gr_G; \cc)$ and $\Shv^c_{I}(\Gr_G; \cc)$ generated by the constant sheaf. Let us denote these $\infty$-categories by $\Loc_{G\pw{t}}(\Gr_G; \cc)$ and $\Loc_I(\Gr_G; \cc)$; they are categories of $G\pw{t}$- and $I$-equivariant local systems of complexes of $\cc$-vector spaces on $\Gr_G$. The notion of ``local system'' here must be interpreted in the derived sense, as (equivariant) representations of the fundamental $\infty$-groupoid of $\Gr_G$, and not of its fundamental groupoid (which would be trivial if $G$ is simply-connected). As we will discuss in the body of the article, the above results then restrict to equivalences
\begin{align}
    \Loc_{G\pw{t}}(\Gr_G; \cc) & \simeq \QCoh(\ld{\g}^{\ast,\reg}[2]/\ld{G}), \label{eq: intro reg locus satake} \\
    \Loc_I(\Gr_G; \cc) & \simeq \QCoh(\tilde{\ld{\g}}^{\reg}[2]/\ld{G}). \label{eq: intro reg locus abg}
\end{align}
Here, $\ld{\g}^{\ast,\reg}$ denotes the open subscheme of \textit{regular} elements in $\ld{\g}^\ast$, i.e., those elements whose stabilizer under the coadjoint action of $\ld{G}$ has dimension given by the rank of $\ld{G}$; and similarly for $\tilde{\ld{\g}}^\reg$. We will refer to these equivalences as the derived geometric Satake equivalence (resp. the ABG equivalence) over the regular locus.
In fact, one can can give a proof of \cref{thm: intro bf derived satake} using \cref{eq: intro reg locus satake} combined with the fact that $\ld{\g}^{\ast,\reg}$ has complement of large codimension in $\ld{\g}^\ast$ (see \cite[Section 3.2]{ku-rel-langlands}).

Our goal in this article is to study an analogue of the derived geometric Satake equivalence over the regular locus where $\cc$ is replaced by other ring (spectra) of coefficients. The observation allowing us to do this is that the $\infty$-categories $\Loc_{G\pw{t}}(\Gr_G; \cc)$ and $\Loc_I(\Gr_G; \cc)$ are homotopy-theoretic in nature. Indeed, they depend only on the $G\pw{t}$-equivariant (resp. $I$-equivariant) homotopy type of $\Gr_G$. This, in turn can be understood using a result of Quillen and Garland-Raghunathan (see \cite{garland-raghunathan, mitchell-buildings}), which gives a homotopy equivalence $\Gr_G \simeq \Omega G_c$ (and a homeomorphism onto the subspace of $\Omega G_c$ on those based loops with \textit{polynomial} Fourier expansion); so all computations reduce to homotopy-theoretic statements about $\Omega G_c$. Here, $G_c \subseteq G(\cc)$ is a maximal compact subgroup. Since $G\pw{t}$ is homotopy equivalent to $G_c$ (and $I$ is homotopy equivalent to a maximal torus $T_c \subseteq G_c$), the equivalences \cref{eq: intro reg locus satake} and \cref{eq: intro reg locus abg} can be restated as a pair of equivalences
\begin{align}
    \Loc_{G_c}(\Gr_G; \cc) & \simeq \QCoh(\ld{\g}^{\ast,\reg}[2]/\ld{G}), \label{eq: intro restated reg locus satake} \\
    \Loc_{T_c}(\Gr_G; \cc) & \simeq \QCoh(\tilde{\ld{\g}}^{\reg}[2]/\ld{G}). \label{eq: intro restated reg locus abg}
\end{align}