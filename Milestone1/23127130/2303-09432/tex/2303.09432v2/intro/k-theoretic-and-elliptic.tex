\subsection{A K-theoretic and elliptic variant}

As mentioned above, our goal in this article is to generalize the equivalence \cref{eq: intro reg locus abg} to the case of K-theoretic and elliptic cohomology coefficients. In order to motivate the discussion, we need to review some of the setup of equivariant generalized cohomology. (For the homotopy theorists: it is exactly the \textit{genuine} nature of equivariant generalized cohomology which lends substance to results like \cref{thm: intro omnibus} below.)

If $G_c$ is a compact Lie group, Atiyah and Segal defined $G_c$-equivariant complex K-theory $\KU_{G_c}$ in \cite{segal-equiv-KU, atiyah-segal-original}: this is a generalized cohomology theory, viewed as a spectrum in the sense of homotopy theory, which classifies $G_c$-equivariant vector bundles on finite $G_c$-spaces. Direct sum and tensor products of $G$-equivariant vector bundles equips $\KU_{G_c}$ with the structure of a \textit{ring} spectrum; in fact, it is an $\Eoo$-ring, meaning (for instance) that the multiplication on cohomology can be refined by Adams operations.
Despite its definition, the geometric interpretation of cocycles for equivariant K-theory as equivariant vector bundles will play \textit{no} role below. 

Two important examples are the following. When $G_c$ is the trivial group, $\KU_{G_c}$ is simply periodic complex K-theory $\KU$, and Bott periodicity gives a graded isomorphism $\pi_\ast \KU \cong \Z[\beta^{\pm 1}]$ with the Bott class $\beta$ in weight $2$. On the other hand, when $G_c$ is a connected compact Lie group with complex representation ring $\mathrm{RU}(G_c)$, the coefficient ring $\pi_\ast \KU_{G_c}$ is the tensor product $\mathrm{RU}(G_c) \otimes_\Z \Z[\beta^{\pm 1}]$. In particular, if $G_c$ is a compact torus $T_c$, then $\spec \pi_\ast \KU_{T_c}$ is the corresponding algebraic torus $T_{\Z[\beta^{\pm 1}]}$ over $\Z[\beta^{\pm 1}]$.

In \cref{sec: equiv coh}, we define a $\KU$-linear $\infty$-category $\Loc_{T_c}(\Gr_G; \KU)$ of $T_c$-equivariant local systems of $\KU$-modules on $\Gr_G$; this should be viewed as a $\KU$-theoretic analogue of the $\infty$-category $\Loc_I(\Gr_G; \KU)$. The ideal analogue of \cref{eq: intro reg locus abg} would identify $\Loc_{T_c}(\Gr_G; \KU)$ with the $\infty$-category of perfect complexes on some stack over $\KU$ defined in terms of the Langlands dual group. Unfortunately, we do not know how to define this putative stack over $\KU$; instead, we will study a particular \textit{degeneration} of this $\infty$-category, denoted $\Loc_{T_c}^\gr(\Gr_G; \KU)$. The reason for studying this degeneration is explained in the introduction to \cref{sec: degenerations}; we will also discuss its ``philosophical'' meaning momentarily. For the moment, let us just note the utility of this degeneration: while $\Loc_{T_c}(\Gr_G; \KU)$ is a $\KU$-linear $\infty$-category, $\Loc_{T_c}^\gr(\Gr_G; \KU)$ is instead an ordinary ($\Z$-linear) $\infty$-category; so we do not have to be concerned with questions such as the definition of the dual group over $\KU$.

Exactly the same setup works if $k$ is a complex oriented $2$-periodic $\Eoo$-ring which is an elliptic cohomology theory (in the sense of \cite{survey, t-equiv-tmf, gepner-meier}) with associated elliptic curve $E$ over $\pi_0(k)$. Namely, we construct a $k$-linear $\infty$-category $\Loc_{T_c}(\Gr_G; k)$, as well as a degeneration $\Loc_{T_c}^\gr(\Gr_G; k)$ which is instead an ordinary ($\pi_0(k)$-linear) $\infty$-category. In both this case and the case of $\KU$, an object $\cf \in \Loc_{T_c}(\Gr_G; k)$ defines a corresponding object $\cf^\gr \in \Loc_{T_c}^\gr(\Gr_G; k)$, and there is a spectral sequence
\begin{equation}\label{eq: intro sseq cohomology and Loc gr}
    \pi_\ast(k) \otimes_{\pi_0(k)} \pi_\ast \Map_{\Loc_{T_c}^\gr(\Gr_G; k)}(\ul{k}^\gr, \cf^\gr) \Rightarrow \pi_\ast \Map_{\Loc_{T_c}(\Gr_G; k)}(\ul{k}, \cf) = \pi_\ast \Gamma_{T_c}(\Gr_G; \cf).
\end{equation}
Here, $\ul{k}$ denotes the constant sheaf.

To state our main result, we need a small observation. Suppose $G$ is a connected, almost simple, and \textit{simply-laced} algebraic group over $\cc$, and let $\ld{G}_\cc$ denote the Langlands dual over $\cc$. Then $\ld{G}_\cc$ is centrally isogeneous to $G$. For instance, if $G$ is simply-connected, $\ld{G}_\cc$ is the quotient of $G$ by its center. In general (under the simply-laced hypothesis), the action of $G$ on itself by conjugation descends/ascends to an action of $\ld{G}_\cc$. In particular, the action of $T_\cc$ on $G$ by conjugation descends/ascends to an action of $\ld{T}_\cc$ on $G$. Therefore, if $T_c$ denote the maximal compact subgroup of $T_\cc$ (and similarly for $\ld{T}_c$), then the conjugation action of $T_c$ on $\Gr_G$ descends/ascends to an action of $\ld{T}_c$ on $\Gr_G$.

The main result of this article is the following.
\begin{theorem}[\cref{cor: reg locus ordinary ABG}, \cref{cor: ku reg locus ordinary ABG}, \cref{cor: ell reg locus ordinary ABG}]\label{thm: intro omnibus}
    Suppose $G$ is a connected, almost simple, and \textit{simply-laced} algebraic group over $\cc$. Let $T \subseteq G$ be a maximal torus, and let $\ld{T}_c$ denote a maximal compact subgroup of the Langlands dual torus over $\cc$. Let $k$ denote either $2$-periodified rational cohomology $\QQ[u^{\pm 1}]$, complex K-theory $\KU$, or elliptic cohomology with associated elliptic curve $E$, and let $F$ be an algebraically closed field over $\pi_0(k)$. Then there are equivalences
    \begin{align*}
        \Loc_{\ld{T}_c}^\gr(\Gr_G; \QQ[u^{\pm 1}]) \otimes_\QQ F & \simeq \QCoh(\tilde{\ld{\g}}^{',\reg}/\ld{G}), \\
        \Loc_{\ld{T}_c}^\gr(\Gr_G; \KU) \otimes_\Z F & \simeq \QCoh(\tilde{\ld{G}}^{\reg}/\ld{G}), \\
        \Loc_{\ld{T}_c}^\gr(\Gr_G; k) \otimes_{\pi_0(k)} F & \simeq \QCoh(\Bun_{\ld{B}}^0(E)^\reg).
    \end{align*}
    Here, the dual groups on the right-hand side are defined over $F$; the final equivalence is assuming that $k$ is an elliptic cohomology theory; $\tilde{\ld{\g}}^{'}$ denotes $\ld{G} \times^{\ld{B}} \ld{\fr{b}}$; $\tilde{\ld{G}}$ denotes $\ld{G} \times^{\ld{B}} \ld{B}$ for the conjugation action of $\ld{B}$ on itself\footnote{We warn the reader that the symbol $\tilde{\ld{G}}$ will mean $\ld{G} \times^{\ld{B}} \ld{B}$ \textit{only} in the introduction; it has a slightly different meaning in the body of the article, described in \cref{def: mult kostant slice}.}; $\Bun_{\ld{B}}^0(E)$ denotes the moduli stack of degree zero $\ld{B}$-bundles on $E$; and the adornment $\reg$ denotes an open ``regular'' locus whose complement has codimension $2$.
\end{theorem}
Since $T_c$ and $\ld{T}_c$ are isogeneous by a finite group, there is no difference between $\Loc_{T_c}^\gr(\Gr_G; \QQ[u^{\pm 1}])$ and $\Loc_{\ld{T}_c}^\gr(\Gr_G; \QQ[u^{\pm 1}])$. Therefore, the first part of \cref{thm: intro omnibus} is, of course, just \cref{eq: intro restated reg locus abg} (once one identifies $\tilde{\ld{\g}}^{'} \cong \tilde{\ld{\g}}$). However, we warn the reader that if $k$ is not $\QQ[u^{\pm 1}]$, the categories $\Loc_{T_c}^\gr(\Gr_G; k)$ and $\Loc_{\ld{T}_c}^\gr(\Gr_G; k)$ are generally genuinely different, and do not agree even upon base-change along $\pi_0(k) \to F$.

\begin{remark}\label{rmk: parabolic variant}
    The same argument used to prove \cref{thm: intro omnibus} shows that if $G$ is a connected, almost simple, and \textit{simply-laced} algebraic group over $\cc$ with torsion-free fundamental group, and $k$ denotes either $2$-periodified rational cohomology $\QQ[u^{\pm 1}]$, complex K-theory $\KU$, or elliptic cohomology with associated elliptic curve $E$, then there are equivalences
    \begin{align*}
        \Loc_{\ld{G}_c}^\gr(\Gr_G; \QQ[u^{\pm 1}]) \otimes_\QQ F & \simeq \QCoh(\ld{\g}^{\reg}/\ld{G}), \\
        \Loc_{\ld{G}_c}^\gr(\Gr_G; \KU) \otimes_\Z F & \simeq \QCoh(\ld{G}^{\reg}/\ld{G}), \\
        \Loc_{\ld{G}_c}^\gr(\Gr_G; k) \otimes_{\pi_0(k)} F & \simeq \QCoh(\Bun_{\ld{G}}^\ss(E)^\reg),
    \end{align*}
    where $\Bun_{\ld{G}}^\ss(E)^\reg \subseteq \Bun_{\ld{G}}^\ss(E)$ is a particular open substack of the moduli stack of semistable degree zero $\ld{G}$-bundles on $E$. More generally, our arguments are easily modified to show that if $L$ is the Levi quotient of a parabolic subgroup $P \subseteq G$ and $L$ has torsion-free fundamental group, then there are equivalences
    \begin{align*}
        \Loc_{\ld{L}_c}^\gr(\Gr_G; \QQ[u^{\pm 1}]) \otimes_\QQ F & \simeq \QCoh(\tilde{\ld{\g}}_{\ld{P}}^{',\reg}/\ld{P}), \\
        \Loc_{\ld{L}_c}^\gr(\Gr_G; \KU) \otimes_\Z F & \simeq \QCoh(\tilde{\ld{G}}_{\ld{P}}^{\reg}/\ld{P}), \\
        \Loc_{\ld{L}_c}^\gr(\Gr_G; k) \otimes_{\pi_0(k)} F & \simeq \QCoh(\Bun_{\ld{P}}^\ss(E)^\reg),
    \end{align*}
    where $\tilde{\ld{\g}}_{\ld{P}}^{'}$ denotes $\ld{G} \times^{\ld{P}} \ld{\fr{p}}$; $\tilde{\ld{G}}_{\ld{P}}$ denotes $\ld{G} \times^{\ld{P}} \ld{P}$ for the conjugation action of $\ld{P}$ on itself; and $\Bun_{\ld{P}}^\ss(E)$ denotes the moduli stack of degree zero semistable $\ld{P}$-bundles on $E$. The equivalence concerning $\Loc_{\ld{L}_c}^\gr(\Gr_G; \QQ[u^{\pm 1}])$ above is closely related to the parabolic variant of \cref{thm: intro abg} which can be deduced from \cite{chen-dhillon}.
\end{remark}
\cref{thm: intro omnibus} also gives an analogue of \cite[Theorem 4]{bf-derived-satake}. Namely, the stacks $\ld{\g}/\ld{G}$, $\ld{G}/\ld{G}$, and $\Bun_{\ld{G}}^\ss(E)$ each have a canonical map to $B\ld{G}$; let $\QCoh_\free(\ld{\g}/\ld{G})$, $\QCoh_\free(\ld{G}/\ld{G})$, and $\QCoh_\free(\Bun_{\ld{G}}^\ss(E))$ denote the essential images of the resulting pullback functors from $\Rep(\ld{G})$ to $\QCoh(\ld{\g}/\ld{G})$, $\QCoh(\ld{G}/\ld{G})$, and $\QCoh(\Bun_{\ld{G}}^\ss(E))$.
Then, we show:
\begin{corollary}[\cref{cor: ordinary minuscule equivalence}, \cref{cor: ku minuscule equivalence}, and \cref{cor: ell minuscule equivalence}]
    In the setting of \cref{thm: intro omnibus}, let $F$ be an algebraically closed field of characteristic zero over $\pi_0(k)$.
    Then there is a $t$-structure on $\Loc_{G_c}^\gr(\Gr_G; k)$ such that there are fully faithful embeddings
    \begin{align*}
        \QCoh_\free(\ld{\g}/\ld{G})^\heartsuit & \hookrightarrow \Loc_{\ld{G}_c}^\gr(\Gr_G; \QQ[u^{\pm 1}])^\heartsuit \otimes_{\QQ} F, \\
        \QCoh_\free(\ld{G}/\ld{G})^\heartsuit & \hookrightarrow \Loc_{\ld{G}_c}^\gr(\Gr_G; \KU)^\heartsuit \otimes_{\Z} F, \\
        \QCoh_\free(\Bun_{\ld{G}}^\ss(E))^\heartsuit & \hookrightarrow \Loc_{\ld{G}_c}^\gr(\Gr_G; k)^\heartsuit \otimes_{\pi_0(k)} F.
    \end{align*}
    When $\ld{G}$ is not of type $E_8$, we explicitly identify the essential image of these embeddings in terms of a topologically defined subcategory of $\Loc_{\ld{G}_c}^\gr(\Gr_G; k)^\heartsuit \otimes_{\pi_0(k)} F$.
\end{corollary}
In the case of K-theory, this is similar to the expectations of \cite{cautis-kamnitzer}. Again, just as in \cref{rmk: parabolic variant}, there are also parabolic versions of these embeddings. In the case of $2$-periodic rational cohomology, one obtains an upgrade (as in \cite[Theorem 2]{bf-derived-satake}): namely, if $\HC^{\hbar,\free}_{\ld{G}}$ denotes the category of Harish-Chandra bimodules of the form $U_\hbar(\ld{\g}) \otimes V$ for $V \in \Rep(\ld{G})$, then there is a fully faithful embedding 
$$(\HC^{\hbar,\free}_{\ld{G}})^\heartsuit \hookrightarrow \Loc_{G_c \times S^1_\rot}^\gr(\Gr_G; k)^\heartsuit.$$
See \cref{cor: reg locus quantized satake} and the discussion surrounding it.
At least in the case $G = \SL_2$, one can similarly relate a category of Harish-Chandra bimodules for the quantum group to $\Loc_{G_c \times S^1_\rot}^\gr(\Gr_G; \KU)^\heartsuit$ (see \cref{prop: rep q fully faithful}).
\begin{remark}
    We also study the effect of power operations (like Steenrod and Adams operations) on $k$ under the Langlands duality of \cref{thm: intro omnibus}. The reader is referred to \cref{thm: frobenius and langlands} for a precise statement, but let us just mention here that using the theory of degree $p$ isogenies on $\GG_a$, $\GG_m$, and the elliptic curve $E$, we show that power operations on $k$ correspond under \cref{thm: intro omnibus} to natural ``Artin-Schreier'' maps on $\tilde{\ld{\g}}'/\ld{G}$, $\tilde{\ld{G}}/\ld{G}$, and $\Bun_{\ld{B}}^0(E)$. (This is different, but related to, Lonergan's work in \cite{lonergan-frob}: we only consider power operations on $k$, while Lonergan considers power operations on the entire $\E{3}$-algebra $k^{G_c}[\Gr_G]$; we will address the latter situation in future work.)
\end{remark}
%\begin{remark}
%    Each $\ld{G}$-representation $V$ defines a vector bundle over $\tilde{\ld{\g}}^{',\reg}/\ld{G}$, $\tilde{\ld{G}}^{\reg}/\ld{G}$, and $\Bun_{\ld{B}}^0(E)^\reg$. In general, we do not know whether this vector bundle arises as $\cf^\gr$ for some $\cf \in \Loc_{\ld{T}_c}(\Gr_G; k)$; but in \cref{prop: ordinary realizing minuscule reps}, \cref{prop: KU realizing minuscule reps}, and \cref{prop: ell realizing minuscule reps}, we show (following \cite{gross-minuscule}) that such an object $\cf$ exists if $V$ is an irreducible $\ld{G}$-representation with \textit{minuscule} highest weight. This is used to prove a mild refinement of the fully faithful functors of \cref{rmk: intro full faithful}; see \cref{cor: ordinary minuscule equivalence}, \cref{cor: ku minuscule equivalence}, and \cref{cor: ell minuscule equivalence}.
%\end{remark}

\begin{remark}
    Let us mention some previous work towards analogues of the geometric Satake equivalence with other coefficients. For instance, an early paper in the context of geometric representation theory is \cite{ginzburg-kapranov-vasserot}. A conjecture about complex K-theoretic geometric Satake was proposed in \cite{cautis-kamnitzer}; in a similar vein, a discussion of the K-theoretic case is the content of the talk \cite{lonergan-slides}. In \cite{yang-zhao-e-thy-quantum-group}, Yang and Zhao study a higher chromatic analogue of quantum groups, and it would be interesting to study the relationship between the present article and their work. After the first version of this paper was written, the preprint \cite{zhong-equiv-homology-of-Gr} was posted on the arXiv; it is concerned with ideas similar to the ones studied here. 
\end{remark}

The proofs of the equivalences in \cref{thm: intro omnibus} with $\QQ[u^{\pm 1}]$-coefficients follow from the work of Bezrukavnikov-Finkelberg-Mirkovic \cite{bfm} and Yun-Zhu \cite{homology-langlands}; and the proofs of the equivalences involving $\KU$ follow from the aforementioned work of Bezrukavnikov-Finkelberg-Mirkovic. However, the approaches taken in these references rely either on the geometric Satake equivalence (for which there is no existing analogue with coefficients in $\KU$ or elliptic cohomology), or on a geometric interpretation of cycles in the relevant cohomology theory (for which there is no known analogue in elliptic cohomology). We therefore reprove these results in the present article so as to provide a \textit{uniform} approach to all the equivalences of \cref{thm: intro omnibus} and of \cref{rmk: parabolic variant}.
\begin{remark}
    In \cref{sec: brylinski zhang}, we also study a degeneration $\Loc_{G_c}^\gr(G_c; k)$ of the category $\Loc_{G_c}(G_c; k)$ of \textit{conjugation-equivariant} locally constant sheaves on $G_c$. Namely, we show that (at least if $G_c$ has torsion-free fundamental group) $\Loc_{G_c}^\gr(G_c; k)$ is equivalent to the category of quasicoherent sheaves on the additive (resp. multiplicative; resp. elliptic) regular centralizer group scheme for $\ld{G}$ if $k = \QQ[u^{\pm 1}]$ (resp. $k = \KU$; resp. $k$ is an elliptic cohomology theory). Motivated by \cite{nadler-zaslow} and \cite[Theorem 1.1]{ganatra-pardon-shende}, one can heuristically interpret our discussion as describing a version of mirror symmetry for the wrapped Fukaya category of the symplectic stack $T^\ast(G_c/G_c)$, albeit with coefficients in the complex-oriented $2$-periodic $\Eoo$-ring $k$. Namely, the ``$k$-theoretic'' mirror to $T^\ast(G_c/G_c)$ is the appropriate variant of the regular centralizer group scheme for the Langlands dual group.
\end{remark}

Let us now discuss further the degeneration of the $\KU$-linear $\infty$-category $\Loc_{T_c}(\Gr_G; k)$ to $\Loc_{T_c}^\gr(\Gr_G; k)$. This can be explained in ``two'' ways\footnote{The quotes are to indicate that these two approaches are really the same; since it would be too digressive to do so here, we will explain the meaning of this (admittedly cryptic) statement in a sequel to this article.}:
\begin{enumerate}
    \item One important lesson from chromatic homotopy theory is that the stable homotopy groups of spheres are closely connected to the coherent cohomology of the moduli stack $\Mfg$ of $1$-dimensional formal groups. For instance, the Adams-Novikov spectral sequence can be restated as a spectral sequence
    $$E_2^{\ast,\ast} \cong \H^\ast(\Mfg; \omega^{\otimes \ast}) \Rightarrow \pi_\ast(S^0),$$
    where $S^0$ is the sphere spectrum and $\omega$ is the line bundle of invariant differentials on the universal $1$-dimensional formal group over $\Mfg$.
    This picture can in fact be categorified: the $\infty$-category $\Sp$ of spectra behaves like the $\infty$-category $\QCoh(\Mfg)$ in a very precise sense\footnote{This is not quite correct: namely, one has to instead replace $\QCoh(\Mfg)$ by the ind-completion of the thick subcategory of $\QCoh(\Mfg)$ generated by tensor powers of $\omega$. For brevity, we will simply denote this variant category by $\QCoh(\Mfg)$. In the homotopy theory literature, e.g., \cite{gregoric-synthetic} or \cite[Definition 5.14]{bhs-artin-tate}, this variant subcategory is often instead denoted by $\IndCoh(\Mfg)$. However, the use of the symbol $\IndCoh(\Mfg)$ does not agree with the more established notion of ind-coherent sheaves from \cite{gr-i}.}. Namely, the Adams-Novikov spectral sequence is categorified by a $1$-parameter degeneration of $\Sp$ into $\QCoh(\Mfg)$; see \cite{piotr-synthetic, gwx-special-fiber, gregoric-synthetic}. Moreover, this degeneration can be constructed using the stable motivic category over $\cc$. This gives a precise sense in which ``topology is approximated by algebra''. The degeneration of $\Loc_{T_c}(\Gr_G; k)$ into $\Loc_{T_c}^\gr(\Gr_G; k)$ is of exactly the same type. In fact, both degenerations ($\Sp \leadsto \QCoh(\Mfg)$ and $\Loc_{T_c}(\Gr_G; k) \leadsto \Loc_{T_c}^\gr(\Gr_G; k)$) can be constructed simultaneously using the even filtration of \cite{even-filtr, piotr-even-filtr}, and we will address this point in a future article. (That is to say, the Adams-Novikov spectral sequence and \cref{eq: intro sseq cohomology and Loc gr} should both be regarded as special cases of a more general construction.)
    \item In geometric representation theory, one often considers ``graded lifts'' of categories of ($\cc$-valued, say) sheaves on a scheme/stack $X$: this is generally defined as the category $\Shv^\mix(X; \cc)$ of \textit{mixed} sheaves on $X$. See \cite{bbdg} and the more recent \cite{ho-li-mixed}. It is generally expected, for instance, that there is a mixed variant of \cref{thm: intro bf derived satake}, stating that there is an equivalence $\Shv^\mix_{G\pw{t}}(\Gr_G; \cc) \simeq \Perf(\ld{\g}^\ast(2)/\ld{G})$, where $\ld{\g}^\ast(2)/\ld{G}$ is now a classical (not derived!) stack, except with a grading which places $\ld{\g}^\ast(2)$ in weight $2$. This grading can be ignored if we replace $\cc$ by its $2$-periodification. One can view $\Loc_{T_c}^\gr(\Gr_G; k)$ as a category of ``mixed'' $k$-valued local systems. (So where is the grading? It is not visible on the right-hand sides of the equivalences in \cref{thm: intro omnibus} is that the $\Eoo$-ring $k$ is assumed to be $2$-periodic; but the grading will reappear if we assume $k$ is ordinary (non-periodic) cohomology or connective K-theory as in \cite{ku-rel-langlands}.) A natural question, of course, is to define a full $\infty$-category $\Shv_I^\gr(\Gr_G; k)$ which specializes to the usual category of mixed sheaves when $k = \cc[u^{\pm 1}]$; we hope to address this in the future article referred to in the preceding bullet point.
\end{enumerate}

The perspective of the degeneration $\Loc_{T_c}(\Gr_G; k) \leadsto \Loc_{T_c}^\gr(\Gr_G; k)$ as being analogous to the degeneration $\Sp \leadsto \QCoh(\Mfg)$ -- and furthermore that both are related to the even filtration of \cite{even-filtr, piotr-even-filtr} -- is very helpful, because it gives us an indication of how to define $\Loc_{T_c}^\gr(\Gr_G; k)$ when $k$ is not necessarily complex-oriented and $2$-periodic. In particular, we also study the example of $k$ being \textit{real} K-theory $\KO$. This is an $\Eoo$-ring with somewhat complicated homotopy groups. Despite this, the $\Eoo$-ring $\KO$ is itself easy to describe using $\KU$: namely, there is an action of $\Z/2$ on $\KU$ by complex conjugation, and $\KO = \KU^{h\Z/2}$. Moreover, just like the standard Adams-Novikov spectral sequence for the homotopy of the sphere spectrum, there is a spectral sequence
$$E_2^{\ast,\ast} \cong \H^\ast(B\Z/2; \omega^\ast) \Rightarrow \pi_\ast(\KO),$$
where $\omega$ is the line bundle over $B\Z/2$ given by the sign action of $\Z/2$ on $\Z$; in fact, this can be identified with the Adams-Novikov spectral sequence for the homotopy of $\KO$. That is, the sphere spectrum is to $\Mfg$ as $\KO$ is to $B\Z/2$. There is also a good notion of $G_c$-equivariant real K-theory.

Instead of constructing a degeneration of the $\KO$-linear $\infty$-category $\Loc_{T_c}(\Gr_G; \KO)$ into a graded $\pi_\ast(\KO)$-linear $\infty$-category, one can construct a degeneration of $\Loc_{T_c}(\Gr_G; \KO)$ into a $\QCoh(B\Z/2)$-module $\infty$-category $\Loc_{T_c}^\gr(\Gr_G; \KO)$. The construction of $\Loc_{T_c}^\gr(\Gr_G; \KO)$ is straightforward: the $\Z/2$-action via complex conjugation on $\Loc_{T_c}(\Gr_G; \KU)$ defines a $\Z/2$-action on $\Loc_{T_c}^\gr(\Gr_G; \KU)$, and this defines the $\QCoh(B\Z/2)$-module category $\Loc_{T_c}^\gr(\Gr_G; \KO)$.
\begin{prop}[\cref{prop: cplx conj KU and B mod B^}]
    Let $\theta$ denote the involution on $\tilde{\ld{G}}/\ld{G}$ given on $\tilde{\ld{G}} = \ld{G} \times^{\ld{B}} \ld{B}$ by $(g,x)\mapsto (g, x^{-1})$, so that the quotient $(\tilde{\ld{G}}/\ld{G})/\pdb{\theta}$ defines a stack over $B\Z/2$.
    Then there is a $\QCoh(B\Z/2)$-linear equivalence
    $$\Loc_{\ld{T}_c}^\gr(\Gr_G; \KO) \otimes_\Z F \simeq \QCoh((\tilde{\ld{G}}^{\reg}/\ld{G})/\pdb{\theta}).$$
\end{prop}
For instance, if $G = \PGL_2$, so that $\tilde{\ld{G}}$ is the moduli of pairs $(g, \ell)$ with $g \in \SL_2$ and $\ell = [x:y] \in \PP^1$ is a line preserved by $g$, the involution $\theta$ simply inverts $g$. We also briefly discuss the case of coefficients in \textit{connective} real K-theory $\ko$.

Similarly, if one fixes a prime $p$ and sets $T_c[p^\infty]$ to be the $p$-power torsion subgroup of $T_c$, one can also define a $\QCoh(B\Z_p^\times)$-module category $\Loc_{T_c[p^\infty]}^\gr(\Gr_G; \Lone S^0)$. This $\infty$-category is a degeneration of the $\infty$-category $\Loc_{T_c[p^\infty]}(\Gr_G; \Lone S^0)$ of $T_c[p^\infty]$-equivariant local systems of $\Lone S^0$-modules on $\Gr_G$, where $\Lone S^0 = (\KU^\wedge_p)^{h\Z_p^\times}$ is the \textit{$K(1)$-local sphere} (also known as the ``image of J'' spectrum) \cite{adams-jx-iv, ravenel-loc}. In this case, let $\tilde{\ld{G}}_{p^\infty} \subseteq \tilde{\ld{G}}$ denote the locus of pairs $(g, \ld{B}') \in \tilde{\ld{G}}$ where $\ld{B}' \subseteq \ld{G}$ is a Borel subgroup containing $g$ such that the eigenvalues of $g$ are all $p$-power roots of unity. Then there is a $\Z_p^\times \times \ld{G}$-action on $\tilde{\ld{G}}_{p^\infty}$, where $n \in \Z_p^\times$ acts by $(g, \ld{B}')\mapsto (g^n, \ld{B}')$, and \cref{prop: imJ reg locus ABG} similarly states that a $\QCoh(B\Z_p^\times)$-linear equivalence
$$\Loc_{\ld{T}_c[p^\infty]}^\gr(\Gr_G; \Lone S^0) \otimes_{\Z_p} F \simeq \QCoh((\tilde{\ld{G}}_{p^\infty}^{\reg}/\ld{G})/\Z_p^\times).$$
The stack on the right-hand side can be thought of as an open substack in the moduli stack of $\ld{B}$-bundles on the $p$-adic solenoid, modulo the natural symmetries (by $\Z_p^\times$) of this solenoid. In \cref{rmk: morava e-theory}, we make some speculations about the analogous picture when $\Lone S^0$ is replaced by the $K(n)$-local sphere $L_{K(n)} S^0$: the primary modification is that one must now consider the moduli stack of $\ld{B}$-bundles on the $p$-adic $n$-dimensional solenoid, modulo an action of the units in the division algebra over $\QQ_p$ with Hasse invariant $1/n$.