\begin{landscape}
\begin{table}
\centering
%\hspace*{-2cm}
\vspace*{-1cm}
%\resizebox{18cm}{!}{
{
\begin{tabular}{ |c|c|c|c|c|c|c|c| } 
\hline
Stack over $\Bun_{\ld{G}}^\ss(\GG_0^\vee)'$ & Image under $\LL$ & Citation \\
\hline
$\Bun_{\ld{G}}^\ss(\GG_0^\vee)'$ & $\Shv(\Gr_G; k)$ & \cref{thm: intro omnibus} \\
$\Bun_{\ld{P}}^\ss(\GG_0^\vee)'$ & $\Shv(G\ls{t}/I_P; k)$ & \cref{rmk: parabolic variant} \\
$\Bun_{\ld{P}}^{\ss,\nil}(\GG_0^\vee)'$ & $\Shv(G\ls{t}/I_P^0; k)$ & \cref{rmk: parabolic variant} \\
$\Bun_{\ld{G}}^\ss(\GG_0^\vee)^{',\mathrm{coarse}}$ & $\Shv(G\ls{t}/G\ls{t}; k) = \Mod_k$ & Kostant/Steinberg/elliptic slices \\
$B\ld{G}$ & $\mathrm{Kir}(\Shv(G\ls{t}; k))$ & Conjectural in general; \cite{geometric-casselman-shalika-i, geometric-casselman-shalika-ii} when $k = \QQ[u^{\pm 1}]$ \\
$\ld{G}\backslash \ol{T^\ast_{\GG_0}(\ld{G}/\ld{N})}$ & $\Shv^\sph(G\ls{t}/T\ls{t}; k)$ & \cref{prop: ordinary gelfand-graev}, \cref{prop: ku gelfand-graev}, \cref{prop: ell gelfand-graev} \\
$\GL_n\backslash \cB_\beta(\AA^n, \AA^{n-1})/\GL_{n-1}$ & $\Shv^\sph(\GL\ls{t}_n; k)$ for $k = \QQ[u^{\pm 1}], \KU$ & \cite{mirabolic-satake}, \cite[Remark 4.3.4]{ku-rel-langlands} \\
$\gl_n/\GL_n$ & $\Shv^\sph(\GL\ls{t}_{2n}/\Sp\ls{t}_{2n}; k)$ for $k = \QQ[u^{\pm 1}], \KU$ & \cite{quat-satake}, \cref{ex: symplectic period} \\
%$\GL_n\backslash \cB_\beta(\AA^n, \AA^{n+1})/\GL_{n+1}$ & $\Shv(\GL\ls{t}_{2n+1}/\Sp\ls{t}_{2n}; k)$ for $k = \QQ[u^{\pm 1}], \KU$ & Consequence of preceding line \\
\hline
\end{tabular}
}
\vspace{.5cm}
\caption{A (non-exhaustive) list of some stacks $\fr{X}$ over $\Bun_{\ld{G}}^\ss(\GG_0^\vee)'$, and the image of $\QCoh(\fr{X})$ under the conjectural functor \cref{eq: pseudo local langlands}. The citation in the table is possibly to a statement only concerning a ``regular locus'' in $\fr{X}$; but with the ``faux'' definition of graded sheaves from \cite[Construction 3.7.15]{ku-rel-langlands}, the statement can be extended to one about the entirety of $\QCoh(\fr{X})$. For most of these cases, natural symmetries on $k$ match with corresponding evident symmetries on the stack over $\Bun_{\ld{G}}^\ss(\GG_0^\vee)'$. For instance, power operations on $k$ match with the effect of isogenies on $\GG_0$; see \cref{sec: power operations}. However, the meaning of such symmetries on $k$ under this Langlands duality is sometimes less clear (like in the final line; see \cref{ex: symplectic period}). When $k = \QQ[u^{\pm 1}]$ or $\ku$, there are many other examples one could add to this table coming from relative Langlands duality \cite{bzsv}; see there, as well as some of the calculations in \cite{ku-rel-langlands}, for a more comprehensive list in these cases.
\newline
\newline
In the fifth row, $B\ld{G}$ is viewed as a stack over $\Bun_{\ld{G}}^\ss(\GG_0^\vee)'$ via pullback along the map $\GG_0^\vee \to \spec(F)$. The category $\mathrm{Kir}(\Shv(G\ls{t}; k))$ is the ``Kirillov model'' from \cite{gaitsgory-lysenko-kirillov}; the matching in this line should be a version of the geometric Casselman-Shalika formula \cite{geometric-casselman-shalika-i, geometric-casselman-shalika-ii}. See also \cite[Section 10]{lurie-icm} for a related discussion.
In the following row, the scheme $\ol{T^\ast_{\GG_0}(\ld{G}/\ld{N})}$ is introduced below in \cref{prop: ordinary gelfand-graev}, \cref{prop: ku gelfand-graev}, and \cref{prop: ell gelfand-graev}, and $\Shv^\mathrm{sph}(G\ls{t}/T\ls{t}; k)$ is the full subcategory of $\Shv(G\ls{t}/T\ls{t}; k)$ generated by the $G\pw{t}$-equivariant objects.
\newline
\newline
The next row has $G = \GL_n \times \GL_{n-1}$. In the left column, $\cB_\beta(\AA^n, \AA^{n-1})$ denotes the Hamiltonian $\GL_n \times \GL_{n-1}$-space $T^\ast \Hom(\AA^n, \AA^{n-1})$ if $k = \QQ[u^{\pm 1}]$, and denotes Van den Bergh's quasi-Hamiltonian $\GL_n \times \GL_{n-1}$-space \cite{van-den-bergh-double-poisson} of pairs $(u,v)\in \Hom(\AA^n, \AA^{n-1}) \times \Hom(\AA^{n-1}, \AA^n)$ such that $\id + uv$ is invertible if $k = \KU$. In the next column, $\GL_n$ is viewed as a $\GL_n \times \GL_{n-1}$-space by the left and right actions.
\newline
\newline
The next line concerns the quaternionic Satake equivalence, where $G = \GL_{2n}$. When $k = \QQ[u^{\pm 1}]$, the stack $\gl_n/\GL_n$ lives over $\Bun_{\GL_{2n}}^\ss(\GG_a^\vee) = \gl_{2n}/\GL_{2n}$ via the diagonal embedding $\GL_n \subseteq \GL_{2n}$ and the map $\gl_n \to \gl_{2n}$ sending $x\mapsto \begin{psmallmatrix}
    0 & \id_n \\
    x & 0
\end{psmallmatrix}$ (see \cite{quat-satake}). When $k = \KU$, the stack $\gl_n/\GL_n$ lives over $\Bun_{\GL_{2n}}^\ss(\GG_m^\vee) = \GL_{2n}/\GL_{2n}$ via the diagonal embedding $\GL_n \subseteq \GL_{2n}$ and the map $\gl_n \to \GL_{2n}$ sending $x\mapsto \begin{psmallmatrix}
    x+\id_n & \id_n \\
    x & \id_n
\end{psmallmatrix}$. As explained in \cref{rmk: ku symplectic}, one can interpolate between these two cases; but the situation is genuinely different when $k$ is elliptic cohomology (because the quaternionic affine Grassmannian is not $k$-orientable). 
%The final row is a consequence of the preceding one.
}
\label{table: pseudo local}
\end{table}
\end{landscape}