\subsection{Notation and terminology}

In most of this article, $G$ will denote a connected, almost simple, and \textit{simply-laced} algebraic group over $\cc$, and $B$ will denote a Borel subgroup therein. If $H$ is a reductive algebraic group over $\cc$, we will write $H_c$ to denote the maximal compact subgroup of the complex Lie group $H(\cc)$, so $H_c$ is a compact Lie group. We will use the terminology ``finite $H_c$-space'' to mean a finite $H_c$-CW complex. I have tried to be careful to add the subscript $c$ where necessary, but some omissions have certainly inevitably crept in.

The symbol $k$ will denote an $\Eoo$-ring which will generally be fixed at the beginning of each section/theorem statement. The symbol $F$ will denote an algebraically closed field with a map $\pi_0(k) \to F$. If $H$ is a group scheme acting on a scheme $Y$, the symbol $Y/H$ will mean the stacky quotient, and $Y\mmod H$ will denote the invariant-theoretic quotient. The Langlands dual group $\ld{G}$ will generally be defined over $F$; if we wish to view it as defined over a commutative ring $R$, we will denote it by $\ld{G}_R$. If $T$ is a torus, we will write $\bX^\ast(T)$ and $\bX_\ast(T)$ to denote its lattice of characters and cocharacters. We will also write $\ld{\Lambda}$ to denote the root lattice of $G$ and $\Lambda$ to denote the coroot lattice of $G$. The symbol $\tilde{T}$ will denote the extended torus $T \times \GG_m^\rot$. If $\ld{N}\subseteq \ld{G}$ is the unipotent radical of a Borel subgroup of $\ld{G}$, we will write $\psi: \ld{\fr{n}} \to \GG_a$ to denote a nondegenerate character of $\ld{\fr{n}}$, i.e., an additive character which is nonzero on each simple root space. If $X$ is an $\ld{N}$-scheme, we will write $T^\ast(X/_\psi \ld{N})$ to denote the Hamiltonian reduction of $T^\ast X$ at $\psi$ (that is, if $\mu: T^\ast(X) \to \ld{\fr{n}}^\ast$ is the moment map, then $T^\ast(X/_\psi \ld{N}) \cong \mu^{-1}(\psi)/\ld{N}$).

Finally, we will write $\cS$ to denote the $\infty$-category of spaces (also known as ``anima'', but we will not use this terminology here). The symbols $\QCoh$, $\Mod$, etc. are all to be understood in the \textit{derived} sense, as are all fiber and tensor products; we will make it clear if any of these operations are to be understood in their classical sense. In particular, if $X$ is a scheme or stack, we will write $\QCoh(X)^\heartsuit$ to denote the abelian category of quasicoherent sheaves on $X$. If $A$ is an $\Eoo$-ring, $\cC$ is an $A$-linear $\infty$-category (that is, a $\Mod_A$-module in the $\infty$-category of presentable stable $\infty$-categories), and $A \to B$ is a map of $\Eoo$-rings, then $\cC \otimes_A B$ will denote the $B$-linear $\infty$-category $\cC \otimes_{\Mod_A} \Mod_B$; similarly if $A$ is a classical commutative ring and $\cC$ is an abelian $A$-linear category.

\subsection{Acknowledgements}

I am grateful to Lin Chen, Charles Fu, Tom Gannon, and Kevin Lin for helpful conversations and for entertaining my numerous silly questions. I am also grateful to David Ben-Zvi, Victor Ginzburg, Yiannis Sakellaridis, David Treumann, and Akshay Venkatesh for very enlightening discussions, and Pavel Safronov for a useful email. Thanks to Ben Gammage for discussions which helped shape my understanding of some of the topics in \cref{sec: coulomb}, and to Hiraku Nakajima for a very informative email exchange on the same topic. My interest in this area started after I took a class taught by Roman Bezrukavnikov when I was an undergraduate; I am very grateful to him for suggesting that I read \cite{chriss-ginzburg}, and also for introducing me to \cite{bfm}, which led me down the beautiful road to geometric representation theory. Since the first version of this article, my understanding of the subject has also been influenced by \cite{bzsv}. I would also like to thank some anonymous referees for helpful suggestions on improving this article. Last, but certainly far from least, the influence, support, advice, and encouragement of my advisors Dennis Gaitsgory and Mike Hopkins is evident throughout this project; I cannot thank them enough.