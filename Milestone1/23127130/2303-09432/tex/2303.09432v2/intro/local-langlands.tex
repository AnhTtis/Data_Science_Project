\subsection{A unifying picture?}

In this informal section, we suggest a picture which attempts to unify the various calculations in this article. Throughout, we will again take $G$ to be a connected, almost simple, and \textit{simply-laced} algebraic group over $\cc$ with torsion-free fundamental group, and $k$ will denote either $2$-periodified rational cohomology $\QQ[u^{\pm 1}]$, complex K-theory $\KU$, or elliptic cohomology with associated elliptic curve $E$.  (As long as the discussion below is interpreted correctly, one could even take $k$ to be (connective) real K-theory or the $\Eoo$-ring of topological modular forms \cite{tmf}.)

First, just as there is a degeneration of the $k$-linear $\infty$-category $\Loc_{T_c}(\Gr_G; k)$ to an ordinary ($\pi_0(k)$-linear) $\infty$-category $\Loc_{T_c}^\gr(\Gr_G; k)$, we expect to show in future work that there is a well-behaved $k$-linear $\infty$-category $\Shv_I(\Gr_G; k)$ of Iwahori-equivariant sheaves on $\Gr_G$, along with a degeneration to an ordinary $\pi_0(k)$-linear $\infty$-category $\Shv_I^\gr(\Gr_G; k)$. (This is related to the ``faux'' definition from \cite[Construction 3.7.15]{ku-rel-langlands}.) Just as in \cref{thm: intro omnibus}, we also hope to show that there are equivalences
\begin{align*}
    \Shv_{\ld{I}}^\gr(\Gr_G; \QQ[u^{\pm 1}]) \otimes_\QQ F & \simeq \QCoh(\tilde{\ld{\g}}^{'}/\ld{G}), \\
    \Shv_{\ld{I}}^\gr(\Gr_G; \KU) \otimes_\Z F & \simeq \QCoh(\tilde{\ld{G}}/\ld{G}), \\
    \Shv_{\ld{I}}^\gr(\Gr_G; k) \otimes_{\pi_0(k)} F & \simeq \QCoh(\Bun_{\ld{B}}^0(E));
\end{align*}
there should also be similar equivalences when the Iwahori subgroup is replaced by standard parahorics in $L^+ \ld{G}$. In particular, when $\ld{I}$ is replaced by $L^+ \ld{G}$, we expect to prove equivalences
\begin{align*}
    \Shv_{L^+ \ld{G}}^\gr(\Gr_G; \QQ[u^{\pm 1}]) \otimes_\QQ F & \simeq \QCoh(\ld{\g}/\ld{G}), \\
    \Shv_{L^+ \ld{G}}^\gr(\Gr_G; \KU) \otimes_\Z F & \simeq \QCoh(\ld{G}/\ld{G}), \\
    \Shv_{L^+ \ld{G}}^\gr(\Gr_G; k) \otimes_{\pi_0(k)} F & \simeq \QCoh(\Bun_{\ld{G}}^\ss(E)).
\end{align*}
Let us write $\Bun_{G}^\ss(\GG_0^\vee)$ to denote either of the stacks $\g/G$, $G/G$, and $\Bun_{G}^\ss(E)$, depending on the choice of $k$. See \cref{rmk: 1-shifted cartier} for an explanation of this notation: roughly, $\GG_0^\vee$ is the stack of multiplicative line bundles on $\GG_0 = \GG_a$, $\GG_m$, or $E$, respectively. Note that there is a map $\Bun_G^\ss(\GG_0^\vee) \to BG$; in fact, it naturally lifts/descends to a stack $\Bun_{\ld{G}}^\ss(\GG_0^\vee)'$ defined over $B\ld{G}$. (For instance, if $k$ is $2$-periodic rational cohomology or $\KU$, then $\Bun_{\ld{G}}^\ss(\GG_0^\vee)'$ is isomorphic to $\g/\ld{G} \cong \ld{\g}^\ast/\ld{G}$ or $G/\ld{G}$, respectively.)

Suppose $P \subseteq G$ is a parabolic subgroup with Levi quotient $L$ (with torsion-free fundamental group), and let $I_P$ denote its preimage under the map $G\pw{t} \to G$. Set $I_P^0$ to be the kernel of the composite $I_P \to P \to L$, and $\ld{P} \subseteq \ld{G}$ to be the dual parabolic subgroup to $P$. One can then similarly define a stack $\Bun_{\ld{P}}^\ss(\GG_0^\vee)'$, and we expect to prove equivalences
$$\Shv_{I_P}^\gr(\Gr_G; k) \otimes_{\pi_0(k)} F \simeq \QCoh(\Bun_{\ld{P}}^\ss(\GG_0^\vee)').$$
If we replace $I_P$ on the left-hand side by $I_P^0$, then $\Bun_{\ld{P}}^\ss(\GG_0^\vee)'$ must be replaced by the stack $\Bun_{\ld{P}}^{\ss,\nil}(\GG_0^\vee)'$ defined as the fiber of the map $\Bun_{\ld{P}}^\ss(\GG_0^\vee)' \to \Bun_{\ld{P}}^\ss(\GG_0^\vee)^{',\mathrm{coarse}}$ over the basepoint of the trivial bundle. For instance, if $k$ is $2$-periodic rational cohomology or $\KU$, then $\Bun_{\ld{P}}^{\ss,\nil}(\GG_0^\vee)'$ is isomorphic to $\tilde{\cN}_P/\ld{G}$ or $\tilde{\cU}_P/\ld{G}$, respectively, where $\tilde{\cN}_P$ and $\tilde{\cU}_P$ are the partial resolutions (\`a la \cite{borho-macpherson}) of the nilpotent and unipotent cones in $G$, respectively.

The above discussion suggest the following informal picture, which I hope to make precise in future work. There should be a well-defined $(\infty,2)$-category of ``$\infty$-categories with an action of $\Shv(G\ls{t}; k)$'' (this seems to be rather subtle to make sense of in the genuine equivariant setting), and it should admit a graded analogue such that there is a faithful embedding
\begin{equation}\label{eq: pseudo local langlands}
    \LL: \left\{\begin{tabular}{l}
      Quasi-coherent sheaves of \\
      $\infty$-categories on $\Bun_{\ld{G}}^\ss(\GG_0^\vee)'$ \\
    \end{tabular}\right\} \hookrightarrow \left\{\begin{tabular}{l}
      $\infty$-categories with an \\
      action of $\Shv(G\ls{t}; k)$ \\
    \end{tabular}\right\}^\gr \otimes_{\pi_0(k)} F
\end{equation}
whose essential image consists of those $\Shv(G\ls{t}; k)$-module categories which are generated by their $G\pw{t}$-equivariant objects.
This is similar in spirit to (a small part of) the local geometric Langlands correspondence. In particular, if $\cC$ and $\cd$ are quasi-coherent sheaves of $\infty$-categories on $\Bun_{\ld{G}}^\ss(\GG_0^\vee)'$, then there should be an equivalence
$$\Map_{\mathrm{QCohCat}(\Bun_{\ld{G}}^\ss(\GG_0^\vee)')}(\cC, \cd) \xrightarrow{\sim} \Map_{\Shv(G\ls{t}; k)\modc^\gr}(\LL(\cC), \LL(\cd)) \otimes_{\pi_0(k)} F.$$
%which should sometimes be an equivalence of $\infty$-categories.
(One also expects a tamely-ramified variant of \cref{eq: pseudo local langlands}, where the left-hand side is replaced by the category of modules over $\QCoh(\Bun_{\ld{P}}^\ss(\GG_0^\vee)' \times_{\Bun_{\ld{G}}^\ss(\GG_0^\vee)'} \Bun_{\ld{P}}^\ss(\GG_0^\vee)')$ viewed as a monoidal category under convolution. The essential image of this functor would consist of those $\Shv(G\ls{t}; k)$-module categories which are generated by their $I_P$-equivariant objects.)
Some examples of the value of $\LL$ are given in \cref{table: pseudo local}. The calculations of \cite{ku-rel-langlands} also suggest additional lines in this table, at least when $k$ is connective complex K-theory.

The usual local geometric Langlands correspondence \cite{frenkel-gaitsgory-local-langlands} would relate the $(\infty,2)$-category of $\infty$-categories with an action of $\Shv(G\ls{t})$ to the $(\infty,2)$-category of ind-coherent sheaves of $\infty$-categories on the moduli stack $\Loc_{\ld{G}}(D^\times)$ of $\ld{G}$-local systems on the punctured formal disk $D^\times$. Let us try to informally explain the analogy between this and the generalized correspondence suggested above, where one must instead replace $\Loc_{\ld{G}}(D^\times)$ by $\Bun_{\ld{G}}^\ss(\GG_0^\vee)'$. In the de Rham setting, $\Loc_{\ld{G}}(D^\times)$ can informally be viewed as the stack of maps from the (ill-defined) de Rham stack $D^\times_\dR$ of $D^\times$ to $B\ld{G}$. We are then suggesting that one should view $\GG_0^\vee$ as analogous to $D^\times_\dR$ (or at least to a small part thereof).

In fact, this analogy is not new: if $\GG_0$ was a $1$-dimensional formal group (instead of an actual $1$-dimensional algebraic group), then $\GG_0^\vee$ has been studied in the homotopy theory literature \cite{sibilla-tomasini, toen-hkr, moulinos-loop} as the ``$\GG_0$-circle'' $S^1_{\GG_0}$, marketed as a $\GG_0$-analogue of the homotopy type of the usual circle. For instance, there should be an equivalence $R\Gamma(D^\times_\dR; \co) = R\Gamma_\dR(D^\times)$, so that its cohomology is an exterior algebra over $\cc$ on a single class in cohomological degree $1$. Similarly, the cohomology of $R\Gamma(\GG_0^\vee; \co)$ is also an exterior algebra over $\pi_0(k)$ on a single class in cohomological degree $1$ (but outside of characteristic zero, the derived algebra structures on $R\Gamma(\GG_0^\vee; \co)$ and $R\Gamma_\dR(D^\times)$ generally disagree). Thus, the affinizations of $\GG_0^\vee$ and $D^\times_\dR$, at least, behave similarly. This suggests that $\Bun_{\ld{G}}^\ss(\GG_0^\vee)'$ behaves like a slight variant of (the formal neighborhood of the trivial local system in) $\Loc_{\ld{G}}(D^\times)$.

Informally, \cref{eq: pseudo local langlands} is suggesting that the $(\infty,2)$-category of quasicoherent sheaves of $\infty$-categories on $\Bun_{\ld{G}}^\ss(S^1_{\GG_0})$ should be a full subcategory of a graded analogue of the $(\infty,2)$-category of $\infty$-categories with $\Shv(G\ls{t}; k)$-action. The latter is roughly the $(\infty,2)$-category of $\infty$-categories over the stack $\Bun_{G_k}(S^1_k)$ of $G_k$-bundles on the Betti stack (over $k$) of the topological circle\footnote{Here, if $X$ is a suitably nice topological space, we write $X_k$ to denote the stack such that $\QCoh(X_k) \simeq \Shv(X; k)$.}. Thus the Langlands duality of \cref{eq: pseudo local langlands} ``swaps'' the topological circle $S^1$ (manifested as its Betti stack $S^1_k$) with the $\GG_0$-circle $S^1_{\GG_0}$. The merit of this perspective, especially in the context of topological quantum field theories \cite{kapustin-witten, bzsv}, is completely unclear to me.