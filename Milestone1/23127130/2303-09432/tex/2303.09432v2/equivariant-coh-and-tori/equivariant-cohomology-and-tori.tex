In order to study and prove analogues of the equivalences of \cref{cor: reg locus satake} and \cref{cor: reg locus abg} for other cohomology theories, we need to review some foundational aspects of the theory of equivariant cohomology. I have reviewed some of the basics of equivariant K-theory in \cite[Section 2.2]{ku-rel-langlands}. The theory of equivariant elliptic cohomology is developed similarly, and we will now describe this story (in a somewhat leisurely fashion) following \cite{survey, gepner-meier, t-equiv-tmf}. At the end of this section, we describe the geometric Satake equivalence for tori.

The basic question we will address is giving a definition of the $\infty$-category $\Loc_{T_c}(X; k)$ for a $T_c$-space $X$ for a sufficiently general $\Eoo$-ring $k$. When $k$ is an $\Eoo$-$\QQ$-algebra, \cref{thm: abg} requires that there is an equivalence 
$$\Loc_{T_c}(\ast; k) \simeq \QCoh(\ld{\fr{t}}^\ast[2]).$$
One often defines the $\infty$-category of $k$-modules on a space $X$ as the $\infty$-category $\Fun(X, \Mod_k)$. However, when $X = BT_c$, the $\infty$-category $\Fun(BT_c, \Mod_k)$ does \textit{not} agree with $\QCoh(\ld{\fr{t}}^\ast[2])$; instead, it only agrees with a certain completion of this $\infty$-category, as we will now explain. 
\begin{lemma}\label{lem: Fun BT_c to Perf}
    Let $k$ be an $\Eoo$-algebra. Then there is an equivalence
    $$\Fun(BT_c, \Mod_k) \simeq \IndCoh(\{1\} \times_{\ld{T}} \{1\}).$$
    If, moreover, $k$ is an $\Eoo$-$\QQ$-algebra, this can be rewritten as an equivalence
    $$\Fun(BT_c, \Mod_k) \simeq \QCoh(\widehat{\ld{\fr{t}}}^\ast[2]),$$
    where $\widehat{\ld{\fr{t}}}^\ast$ denotes the completion of $\ld{\fr{t}}^\ast$ at the origin.
\end{lemma}
\begin{proof}
    If $X$ is a finite space, there is an equivalence $\Fun(X, \Mod_k) \simeq \IndCoh_{C_\ast(\Omega X; k)}$, where $C_\ast(\Omega X; k)$ is the $\E{1}$-$k$-algebra of $k$-chains on the based loop space $\Omega X$. When $X = BT_c$, we may identify $\Omega X = T_c$. Recall that $T_c$ is the classifying space of the lattice $\bX_\ast(T)$, so that there is an equivalence
    $$C_\ast(T_c; k) \cong k \otimes_{C_\ast(\bX_\ast(T); k)} k.$$
    Of course, we may identify $C_\ast(\bX_\ast(T); k) \cong k[\bX_\ast(T)]$ with the ring of functions on $\ld{T}$. Therefore, $\spec C_\ast(T_c; k) \cong \{1\} \times_{\ld{T}} \{1\}$, as desired.

    Koszul duality gives an equivalence $\IndCoh(C_\ast(T_c; k)) \to \QCoh(C^\ast(BT_c; k))$ given by $M\mapsto \Hom_{C_\ast(T_c; k)}(k, M)$. If $k$ is an $\Eoo$-$\QQ$-algebra, then $C^\ast(BT_c; k)$ is formal, and so it can be identified with the shearing of $\H^\ast(BT_c; k)$. But
    $$\spf \H^\ast(BT_c; k) \cong \widehat{\fr{t}}(2) \cong \widehat{\ld{\fr{t}}}^\ast(2),$$
    so $\IndCoh(C_\ast(T_c; k))$ is equivalent to $\QCoh(\widehat{\ld{\fr{t}}}^\ast[2])$, as desired.
\end{proof}
\begin{example}
    Suppose $T_c = S^1$. Then \cref{lem: Fun BT_c to Perf} tells us that $\Fun(BS^1, \Mod_k) \simeq \QCoh(\widehat{\AA^1}[2])$; the equivalence sends a functor $BS^1 \to \Mod_k$, regarded as a $k$-module $M$ with $S^1$-action, to its homotopy invariants $M^{hS^1}$. Let $t \in \pi_{-2}(k^{hS^1})$ denote a generator. Observe that if $a_\lambda: k \to k[2]$ denotes the boundary map in the cofiber sequence $k[1] \to C_\ast(S^1; k) \to k$, the homotopy invariants of $k[a_\lambda^{-1}]$ are simply $k^{hS^1}[t^{-1}]$ (i.e., the Tate construction). In particular, $\pi_\ast (k[a_\lambda^{-1}])^{hS^1} \cong \pi_\ast(k)\ls{t}$. However (even if $k$ is an $\Eoo$-$\QQ$-algebra), there is no (ind-)object in $\Fun(BS^1, \Mod_k)$ whose image in $\QCoh(\widehat{\AA^1}[2])$ has homotopy given by $\pi_\ast(k)[t^{\pm 1}]$: any object of $\QCoh(\widehat{\AA^1}[2])$ must have $t$ as a topologically nilpotent element in its homotopy.
\end{example}
We therefore need an alternative definition of $\Loc_{T_c}(\ast; k)$, so that it is equivalent to $\QCoh(\ld{\fr{t}}^\ast[2])$ when $k$ is an $\Eoo$-$\QQ$-algebra. Motivated by methods from equivariant homotopy theory, as well as \cite{survey, elliptic-i, elliptic-ii, elliptic-iii}, we will simply \textit{define} $\Loc_{T_c}(\ast; k)$ to be the category of quasicoherent sheaves on a (spectral) stack $\cM_T$. That this category has any relation to topology will come from the requirement that the category of quasicoherent sheaves on the \textit{completion} of $\cM_T$ at a certain basepoint is equivalent to the ind-completion of $\Fun(BT_c, \Mod_k)$.

For this, we review some constructions from \cite{survey} in a form suitable for our applications.
This review will necessarily be brief, since a detailed exposition may be found in \textit{loc. cit.}; there is also some discussion in the early sections of \cite{ginzburg-kapranov-vasserot} in the setting of ordinary (as opposed to spectral) algebraic geometry.
\begin{setup}
    Fix an $\Eoo$-ring $k$ and a commutative $k$-group $\GG$, so $\GG$ defines a functor $\CAlg_A \to \Mod_{\Z,\geq 0}$ which is representable by a \textit{flat} $k$-algebra; here, $\Mod_{\Z, \geq 0}$ denotes the category of connective $\Z$-module spectra. We will write $\GG_0$ to denote the resulting commutative group scheme over $\pi_0 k$. Note that taking zeroth spaces defines an equivalence between $\Mod_{\Z,\geq 0}$ and topological abelian groups.
\end{setup}
\begin{definition}
    A \textit{preorientation of $\GG$} is a pointed map $S^2 \to \Omega^\infty \GG(k)$ of spaces, i.e., a map $\Sigma^2 \Z \to \GG(k)$ of $\Z$-modules (by adjunction). This induces a map $\CP^\infty = \Omega^\infty \Sigma^2 \Z \to \Omega^\infty \GG(k)$ of topological abelian groups, and hence a map $\spf A^{\CP^\infty} \to \GG$ of $\Eoo$-$k$-group schemes. (Note that $\spf A^{\CP^\infty}$ need not admit the structure of a commutative $k$-group scheme: for instance, $A^{\CP^\infty}$ need not be flat over $k$.)
\end{definition}
\begin{definition}\label{def: orientation}
    Given a preorientation $S^2 \to \Omega^\infty \GG(k)$, we obtain a map $\co_\GG \to C^\ast(S^2; k)$ of $\Eoo$-$k$-algebras. On $\pi_0$, this induces a map $\pi_0 \co_\GG = \co_{\GG_0} \to \pi_0 C^\ast(S^2; k)$. However, the target can be identified with the trivial square-zero extension $\pi_0 k \oplus \pi_{-2} k$, so that the preorientation defines a derivation $\co_{\GG_0} \to \pi_{-2} k$. This defines a map $\beta: \omega = \Omega^1_{\GG_0/\pi_0 k} \to \pi_{-2} k$. The preorientation is called an \textit{orientation} if $\GG_0$ is smooth of relative dimension $1$ over $\pi_0 k$, and the composite
    $$\pi_n k \otimes_{\pi_0 k} \omega \to \pi_n k \otimes_{\pi_0 k} \pi_{-2} k \xar{\beta} \pi_{n-2} k$$
    is an isomorphism for each $n\in \Z$. This forces $k$ to be $2$-periodic (but does not force its homotopy to be concentrated in even degrees).
\end{definition}
\begin{warning}\label{warning: additive-orientation}
    As discussed in \cite[Section 3.2]{survey}, the universal $\Eoo$-$\Z$-algebra over which the additive group scheme $\GG_a$ admits an orientation is given by $\Z[\CP^\infty][\tfrac{1}{\beta}] = \QQ[\beta^{\pm 1}]$. Therefore, we are allowed to let $\GG = \GG_a$ in the story below only when $k$ is a $2$-periodic \textit{$\Eoo$-$\QQ$-algebra}. (If $k$ is not an $\Eoo$-$\Z$-algebra, one cannot in general define $\GG_a = \spec k[t]$ as a commutative $k$-group: the coproduct $k[t] \to k[x,y]$ will in general not be a map of $\Eoo$-$k$-algebras.)
\end{warning}
We can now review the definition of $T_c$-equivariant $k$-cohomology when $T_c$ is a compact torus. We will write $T$ to denote the corresponding  split torus over $\Z$.
\begin{construction}\label{cstr: def-equiv-coh}
    Fix an $\Eoo$-ring $k$ as above and a commutative $k$-group $\GG$. Given a compact abelian Lie group $T_c$, define a $k$-scheme $\cM_T$ by the mapping stack $\Hom(\bX^\ast(T), \GG)$. The underlying $\pi_0(k)$-schemes will be denoted by $\GG_0$ and $\cM_{T,0}$. If we wish to emphasize the dependence on $k$, we will add a superscript (e.g., $\cM_T^k$).
    
    We will be particularly interested in the case when $T_c$ is a torus. Let $\cT$ be the full subcategory of $\Top$ spanned by those spaces which are homotopy equivalent to $BT_c$ with $T_c$ being a compact abelian Lie group. By arguing as in \cite[Theorem 3.5.5]{elliptic-iii}, a preorientation of $\GG$ is equivalent to the data of a functor $\cM: \cT \to \Aff_k$ along with compatible equivalences $\cM(BT_c) \simeq \cM_T$. The $\Eoo$-$k$-algebra $\co_{\cM_T}$ is the $T_c$-equivariant $k$-cochains of a point, and will occasionally be denoted by $k_T$. 

    We can now sketch the construction of the $T_c$-equivariant $k$-cochains of more general $T_c$-spaces; see \cite[Theorem 3.2]{survey}. Let $T_c$ be a torus over $\cc$ for the remainder of this discussion, and let $\GG$ be an \textit{oriented} commutative $k$-group. Let $\Top(T_c)$ denote the $\infty$-category of finite $T_c$-spaces, i.e., the smallest subcategory of $\Fun(BT_c, \Top)$ which contains the quotients $T_c/T_c'$ for closed subgroups $T_c'\subseteq T_c$, and which is closed under finite colimits. There is a functor $\cf_T: \Top(T_c)^\op \to \QCoh(\cM_T)$ which is uniquely characterized by the requirement that it preserve finite limits and sends $T_c/T_c' \mapsto q_\ast \co_{\cM_{T_c'}}$. Here, $q: \cM_{T_c'} \to \cM_T$ is the canonical map induced by the inclusion $T_c'\subseteq T_c$. If $X\in \Top(T_c)$, then the \textit{$T_c$-equivariant $k$-cochains of $X$} is the global sections $\Gamma(\cM_T; \cf_T(X))$; we will denote it by $C^\ast_{T_c}(X; k)$. This can be extended to define $T_c$-equivariant $k$-cochains of filtered colimits of finite $T_c$-spaces. If we wish to emphasize the dependence on $k$, we will denote $\cf_T(X)$ by $\cf_T(X; k)$.
\end{construction}
\begin{remark}\label{rmk: oriented and completion}
    If $k$ is $2$-periodic and $\GG$ is a commutative $k$-group, then \cite[Proposition 4.3.23]{elliptic-ii} shows that the data of an orientation on $k$ (in the sense of \cref{def: orientation}) is equivalent to the formal completion of $\GG$ at the origin being isomorphic to $\spf C^\ast(BS^1; k)$. That is, when $\GG$ is oriented, the formal completion of $\cM_T$ at its basepoint is isomorphic to $\spf C^\ast(BT_c; k)$.
\end{remark}
We will denote the functor $\Gamma(\cM_T; \cf_T(-)): \Top(T_c)^\op \to \Mod(\Gamma(\cM_T; \co_{\cM_T}))$ by $C^\ast_{T_c}(-; k): \Top(T_c)^\op \to \Mod(k_T)$.
\begin{definition}\label{def: equiv-homology}
    If $X\in \Top(T_c)$, then the \textit{$T_c$-equivariant $k$-chains of $X$} is the quasicoherent sheaf on $\cM_T$ given by the $\co_{\cM_T}$-linear dual $\cf_T(X)^\vee$. We will denote its global sections by $C_\ast^{T_c}(X; k)$. Note that if $X$ admits an $\E{n}$-algebra structure (compatible with the $T_c$-action), then $\cf_T(X)^\vee$ admits the structure of an $\E{n}$-algebra\footnote{If $\cC$ is a symmetric monoidal $\infty$-category, \cite[Corollary 3.3.4]{elliptic-i} can be used to show that there is an equivalence $\coCAlg(\Alg_\E{n}(\cC)) \simeq \Alg_\E{n}(\coCAlg(\cC))$.} in $\coCAlg(\QCoh(\cM_T))$. 
    Note that $C_\ast^{T_c}(\ast; k) \simeq k_T$, which completes to the $k$-cochains (\textit{not} $k$-chains) of $BT_c$.
\end{definition}
If $X$ is a filtered colimit $\colim_\alpha X_\alpha$ of finite $T_c$-spaces, we will write $\cf_T(X)^\vee$ to denote $\colim_\alpha (\cf_T(X_\alpha)^\vee)$. Note that if we equip the presentation of $X$ as a filtered colimit $\colim_\alpha X_\alpha$ with the structure of a filtered $\E{n}$-algebra, then $\cf_T(X)^\vee$ acquires the structure of an $\E{n}$-algebra in $\coCAlg(\QCoh(\cM_T))$. 
\begin{notation}\label{notn: ideal-character}
    Let $\lambda: T \to \GG_m$ be a character, and let $T_\lambda = \ker(\lambda)$. Then the map $q: \cM_{T_\lambda} \to \cM_T$ is a closed immersion, and we will denote the ideal in $\co_{\cM_T}$ defined by this closed immersion by $\cI_\lambda$. Equivalently, let $V_\lambda$ denote the $T_c$-representation obtained by the projection $T \to T_\lambda$. Then $\cI_\lambda$ is given by the line bundle $\cf_T(S^{V_\lambda})$.
\end{notation}
It is trickier to extend the definition of equivariant cochains to nonabelian groups, but a construction is sketched in \cite[Section 3.5]{survey}, and a detailed construction is given in \cite{gepner-meier}. However, we will not recall this here, because we will only be concerned with torus-equivariance in the present article. 

We now take a moment to prove some foundational aspects of the theory of generalized equivariant cohomology.
\begin{lemma}[{Atiyah-Bott localization \cite{atiyah-bott-localization}}]\label{lem: atiyah localization}
    Let $X$ be a finite $T_c$-space, and let $\cU_X \subseteq \cM_T$ denote the complement of the union of the closed substacks $\cM_{T'}$ over all stabilizers $T'_c \subseteq T_c$ of points in $X$. Then the map $\cf_T(X) \to \cf_T(X^{T_c})$ is an isomorphism after pulling back to $\cU_X$.
\end{lemma}
\begin{proof}
    This follows from induction on the cell structure of $X$. Namely, the statement is true when the $T$-action on $X$ is trivial, which gives the base case. For the inductive step, note that if $X$ is the cofiber of a map $T/T' \to Y$, then there is a cofiber sequence $\cf_T(X) \to \cf_T(Y) \to \cf_T(T/T')$; but $\cf_T(T/T')$ is isomorphic to the pushforward of the structure sheaf along the map $\cM_{T'} \to \cM_T$, and so it vanishes upon pulling back to $\cU_X$. This implies that the map $\cf_T(X) \to \cf_T(Y)$ is an isomorphism upon pulling back to $\cU_X$, as desired.
\end{proof}
One consequence of \cref{lem: atiyah localization} which is worth restating is the following. Let $\punc{\cM}_T$ denote the complement of the union of the closed subschemes $\cM_{T'}$ ranging over all closed \textit{proper} subgroups $T' \subsetneq T$. Then the map $\cf_T(X) \to \cf_T(X^{T_c})$, and hence the map $\cf_T(X^{T_c})^\vee \to \cf_T(X)^\vee$, is an equivalence upon restriction to $\punc{\cM}_T$.

We will also need a version of the Goresky-Kottwitz-MacPherson approach \cite{gkm-original} to equivariant cohomology; in the setting of generalized equivariant cohomology, it has also been studied in \cite{generalized-gkm, gepner-meier}. As such, we will only give a sketch of the relevant argument.
\begin{definition}\label{def: GKM space}
    Let $X$ be a finite $T_c$-space equipped with a chosen presentation in terms of $T_c$-cells. Say that $X$ is a \textit{GKM space} if the following conditions are satisfied:
    \begin{enumerate}
        \item $\pi_0 \cf_T(X)$ is a vector bundle over $\cM_{T,0}$;
        \item if $X^{(1)}$ denotes the equivariant $1$-skeleton of $X$, then $X^{(1)}$ consists of a finite number of spheres $S^\lambda$ meeting only at the fixed points, where $\lambda$ ranges over characters of $T$.
    \end{enumerate}
    In this setup, let $V$ denote the set $X^{T_c}$ of fixed points, and let $E$ denote the set of characters $\lambda$ such that $S^\lambda \subseteq X^{(1)}$. There are two maps $E \rightrightarrows V$ sending $\lambda$ to the points $0,\infty\in S^\lambda \subseteq X^{(1)}$. The resulting graph with set of vertices $V$ and set of edges $E$ will be referred to as the \textit{GKM graph} of $X$.
\end{definition}
The utility of the first condition in the above definition is due to the following.
\begin{lemma}\label{lem: injective coh}
    Let $X$ be a finite $T_c$-space. If $\pi_0 \cf_T(X)$ is a vector bundle over $\cM_{T,0}$, the map $\pi_0 \cf_T(X) \to \pi_0 \cf_T(X^{T_c})$ is an injection.
\end{lemma}
\begin{proof}
    Since the map $\cf_T(X) \to \cf_T(X^{T_c}) \to \cf_T(X^{T_c})_{\punc{\cM}_T}$ factors as
    $\cf_T(X) \to \cf_T(X)|_{\punc{\cM}_T} \to \cf_T(X^{T_c})|_{\punc{\cM}_T}$,
    and the map $\cf_T(X)|_{\punc{\cM}_T} \to \cf_T(X^{T_c})|_{\punc{\cM}_T}$ is an equivalence by \cref{lem: atiyah localization}, it suffices to show that the map $\cf_T(X) \to \cf_T(X)|_{\punc{\cM}_T}$ induces an injection on $\pi_0$. But $\pi_0 \cf_T(X)$ was assumed to be a vector bundle over $\cM_{T,0}$, so one is reduced to the case $X = \ast$, i.e., to showing that the map $\co_{\cM_T} \to \co_{\cM_T}|_{\punc{\cM}_T}$ induces an injection on $\pi_0$. This, however, is clear, since the closed subscheme $\cM_{T',0} \hookrightarrow \cM_{T,0}$ defined by each closed subgroup $T'\subseteq T$ is cut out by a regular sequence.
\end{proof}
\begin{prop}[Goresky-Kottwitz-MacPherson]\label{prop: gkm}
    Let $X$ be a finite GKM $T_c$-space, and choose a presentation in terms of $T_c$-cells. For each character $\lambda: T \to S^1$, let $T_\lambda$ denote the kernel of $T$, let $q_\lambda: \cM_{T_\lambda} \to \cM_T$ denote the induced map, and let $S(\lambda)$ denote the unit representation sphere. Then there is an equalizer diagram
    $$\pi_0 \cf_T(X) \hookrightarrow \pi_0 \cf_T(X^{T_c}) \cong \Map(V, \co_{\cM_{T,0}}) \rightrightarrows \prod_{\lambda \in E} q_{\lambda, \ast} \co_{\cM_{T_\lambda,0}},$$ 
    where the two maps in the equalizer are defined in the evident manner.
\end{prop}
\begin{proof}[Proof sketch]
    First, we show that the maps $\pi_0 \cf_T(X) \to \pi_0 \cf_T(X^{T_c})$ and $\pi_0 \cf_T(X^{(1)}) \to \pi_0 \cf_T(X^{T_c})$ have the same image. There is an evident map from the image of $\pi_0 \cf_T(X) \to \pi_0 \cf_T(X^{T_c})$ to the image of $\pi_0 \cf_T(X^{(1)}) \to \pi_0 \cf_T(X^{T_c})$, which we will denote by $f$. The map $f$ is an injection by \cref{lem: injective coh}.
    Let $T'$ denote a proper closed subgroup of $T$ of codimension $1$, and let $U' \subseteq \cM_{T',0}$ denote the complement of the union of the closed varieties $\cM_{T'',0}$ ranging over the proper closed subgroups $T''\subseteq T'$.
    By \cref{lem: atiyah localization}, the map $f$ is an isomorphism upon restriction to $U'\subseteq \cM_{T',0} \subseteq \cM_{T,0}$ for each proper closed subgroup $T'\subseteq T$ of codimension $1$. 
    Therefore, the locus $Z \subseteq \cM_{T,0}$ over which $f$ fails to be an isomorphism is contained in the union of closed subvarieties $\cM_{T',0}$ for finitely many $T'\subseteq T$ of codimension at least $2$. However, the map $\pi_0 \cf_T(X) \to \pi_0 \cf_T(X)|_{\cM_{T,0} - Z}$ is an isomorphism (by Hartogs). Since the same is true of the map $\pi_0 \cf_T(X^{T_c}) \to \pi_0 \cf_T(X^{T_c})|_{\cM_{T,0} - Z}$, and the map $\pi_0 \cf_T(X) \to \pi_0 \cf_T(X^{T_c})$ factors through the map $\pi_0 \cf_T(X^{(1)}) \to \pi_0 \cf_T(X^{T_c})$, the desired result follows.

    For the equalizer diagram, an easy induction on the cell structure of $X$ reduces us to the case $X = S^\lambda$ for a character $\lambda: T \to S^1$. In this case, the isomorphism $T/T_\lambda \cong S^\lambda$ defines an isomorphism between $\pi_0 \cf_T(S(\lambda))$ and the pushforward of the structure sheaf along the map $\cM_{T_\lambda,0} \to \cM_{T,0}$. Since $S^\lambda \cong \Sigma S(\lambda)$, we obtain an equalizer diagram
    $$\pi_0 \cf_T(S^\lambda) \to \co_{\cM_{T,0}} \oplus \co_{\cM_{T,0}} \cong \Map(\{0,\infty\}, \co_{\cM_{T,0}}) \rightrightarrows q_{\lambda, \ast} \co_{\cM_{T_\lambda,0}}.$$
    This proves the desired claim.
\end{proof}
The same argument proves the following dual to \cref{prop: gkm} (see also \cite{brion-poincare-dual}).
\begin{prop}\label{prop: hmlgy gkm}
    Let $X$ be a finite GKM $T_c$-space, and choose a presentation in terms of $T_c$-cells. Then $\pi_0 \cf_T(X)^\vee$ is isomorphic to the subset of $\pi_0 \cf_T(X^{T_c})^\vee \cong \co_{\cM_{T,0}}[X^{T_c}]$ of those $\sum_{x \in X^{T_c}} f_x [x] \in \co_{\cM_{T,0}}[X^{T_c}]$ such that:
    \begin{itemize}
        \item For each fixed point $x \in X^{T_c}$, the poles of $f_x$ all have order $\leq 1$, and these are contained in the ideal sheaf of $\co_{\cM_{T_\lambda,0}}$ for each character $\lambda: T_c \to S^1$ such that the $T_c$-orbit $S^\lambda$ meets $x$.
        \item For each character $\lambda: T_c \to S^1$ such that the $T_c$-orbit $S^\lambda$ meets $x_0, x_\infty \in X^{T_c}$, we have
        $$\Res_{\cM_{T_\lambda,0}}(f_{x_0}) + \Res_{\cM_{T_\lambda,0}}(f_{x_\infty}) = 0.$$
    \end{itemize}
\end{prop}
These results can be extended without much trouble to ind-$T_c$-spaces $X$ with isolated fixed points satisfying the conditions of \cref{def: GKM space}. (The first condition therein should be replaced by the condition that $\pi_0 \cf_T(X)$ is an ind-vector bundle over $\cM_{T,0}$.)

The preceding discussion can be categorified, as we now explain. The following categorifies the $T_c$-equivariant $k$-cochains $C_{T_c}^\ast(X; k)$.
\begin{construction}\label{cstr: def-loc}
    Let $\Loc_{T_c}(\ast; k)$ denote the $\infty$-category $\QCoh(\cM_T)$. Let $T_c'\subseteq T_c$ be a closed subgroup, so that there is an associated morphism $q: \cM_{T'} \to \cM_T$. This defines a symmetric monoidal functor $\QCoh(\cM_T) \to \QCoh(\cM_{T'})$, which equips $\QCoh(\cM_{T'})$ with the structure of a $\QCoh(\cM_T)$-module.

    Let $\cLoc_{T_c}(-; k): \Top(T_c)^\op \to \CAlg(\shvcat(\cM_T))$ be the functor uniquely characterized by the requirement that it preserve finite limits and send $T/T' \mapsto \QCoh(\cM_{T'})$. If $X\in \Top(T_c)$, then the $\infty$-category $\Loc_{T_c}(X; k)$ of \emph{$T_c$-equivariant local systems of $k$-modules on $X$} is defined to be the global sections of the quasicoherent stack $\cLoc_{T_c}(X; k)$ on $\cM_T$.  If $X$ is a $T_c$-space which is presented as a filtered colimit of finite $T_c$-spaces $X_\alpha$, we will write $\Loc_{T_c}(X; k)$ to denote $\colim \Loc_{T_c}(X_\alpha; k)$.

    If $f: X \to Y$ is a map in $\Top(T_c)$, the associated symmetric monoidal functor $f^\ast: \Loc_{T_c}(Y; k) \to \Loc_{T_c}(X; k)$ (induced by taking global sections of the morphism $f^\ast: \cLoc_{T_c}(Y; k) \to \cLoc_{T_c}(X; k)$ of $\Eoo$-algebras in quasicoherent stacks over $\cM_T$) will be called the \textit{pullback}. One can show that $\Loc_{T_c}(X; k)$ is a presentable stable $\infty$-category, and that $f^\ast$ preserves small colimits (so it has a right adjoint $f_\ast$, which will be called \textit{pushforward}).
\end{construction}
For instance, if $T_c = \{1\}$, then $\Loc_{T_c}(X; k)$ is equivalent to the $\infty$-category $\Loc(X; k) := \Fun(X, \Mod_k)$ of local systems on $X$.
\begin{remark}
    Let $X$ be a finite $T_c$-space. The \textit{constant local system} $\ul{k}$ is defined to be the image of $\co_{\cM_T}$ under the symmetric monoidal functor $\Loc_{T_c}(\ast; k) \simeq \QCoh(\cM_T) \to \Loc_{T_c}(X; k)$ induced by pullback along $f: X \to \ast$. Observe that if $\ul{k}$ denotes the constant local system, then $\End_{\Loc_{T_c}(X; k)}(\ul{k}) \simeq C_{T_c}^\ast(X; k)$. Indeed, $\End_{\Loc_{T_c}(X; k)}(\ul{k}) \simeq \Gamma(\cM_T; f_\ast f^\ast \co_{\cM_T})$, but it is easy to see that $f_\ast f^\ast \co_{\cM_T} = \cf_T(X)\in \QCoh(\cM_T)$. The desired claim then follows from \cref{cstr: def-equiv-coh}.
\end{remark}
\begin{remark}
    If the complexification of $T_c$ were a \textit{finite} diagonalizable group scheme (such as $\mu_n$), the desired category $\Loc_{T_c}(X; k)$ is closely related to the $\infty$-category of \textit{$\GG$-tempered local systems} on the orbispace $X\mmod T$, as described in \cite{elliptic-iii}. Our understanding is that Lurie is planning to describe an extension of the work in \cite{elliptic-iii} and its connections to equivariant homotopy theory in a future article. We warn the reader that \cref{cstr: def-loc} is somewhat \textit{ad hoc}; so the resulting category of equivariant local systems may or may not agree with the output of forthcoming work of Lurie.
\end{remark}
\begin{remark}
    If $X$ is a finite $T_c$-space, a more straightforward definition of the category of $T_c$-equivariant local systems on $X$ is simply the category $\Fun(X/T_c, \Mod_k)$. Equivalently, it can be described as the functor $\Top(T_c)^\op \to \CAlg(\PrL)$ which is uniquely characterized by the requirement that it preserve finite limits and send $T_c/T_c' \mapsto \Fun(BT_c', \Mod_k)$. It follows from \cref{lem: Fun BT_c to Perf} that $\Fun(BT_c', \Mod_k)$ is equivalent to $\Mod(C^\ast(BT_c'; k))$. As discussed in \cref{rmk: oriented and completion}, if the group scheme $\GG$ is oriented, then this is in turn equivalent to $\QCoh(\widehat{\cM_T})$, where $\widehat{\cM_T}$ is the completion of $\cM_T$ at its basepoint. That is, $\Fun(BT_c', \Mod_k)$ can be viewed as a completion of $\QCoh(\cM_{T'})$. This implies that $\Fun(X/T_c, \Mod_k)$ can be viewed as a completion of the subcategory of compact objects of $\Loc_{T_c}(X; k)$. Motivated by this, we will write $\Loc_{T_c}^\wedge(X; k)$ to denote $\Fun(X/T_c, \Mod_k)$; we will use the same notation to denote the extension of the assignment $X\mapsto \Loc_{T_c}^\wedge(X; k)$ to filtered colimits of finite $T_c$-spaces.
\end{remark}
Using this discussion, let us now discuss geometric Satake with $k$-coefficients in the case of a torus. 
\begin{theorem}\label{thm: torus satake}
    Fix a complex-oriented $2$-periodic $\Eoo$-ring $k$ and an oriented commutative $k$-group scheme $\GG$. Let $\ld{T} = \spec k[\bX^\ast(\ld{T})]$ denote the dual torus over $k$. In the following statements, all actions of $\ld{T}$ are trivial.
    Then there are equivalences
    \begin{align*}
        \Loc_{T_c}^\wedge(\Gr_T; k) & \simeq \IndCoh((\{1\} \times_{\ld{T}} \{1\})/\ld{T}), \\
        \Loc_{T_c}(\Gr_T; k) & \simeq \QCoh(\cM_T/\ld{T}).
    \end{align*}
    Moreover, there is an isomorphism of spectral group $k$-schemes
    $$\spec \cf_T(\Gr_T)^\vee \cong \cM_T \times_{\spec(k)} \ld{T} \cong \cM_T \times_{\cM_T/\ld{T}} \cM_T.$$
\end{theorem}
\begin{proof}
    Since the underlying topological space of $\Gr_T$ is simply the lattice $\bX_\ast(T)$, it follows from \cref{lem: Fun BT_c to Perf} that 
    $$\Loc_{T_c}^\wedge(\Gr_T; k) \simeq \bigoplus_{\bX_\ast(T)} \Loc_{T_c}^\wedge(\ast; k) \simeq \QCoh(B\ld{T}) \otimes_{\Mod_k} \IndCoh(\{1\} \times_{\ld{T}} \{1\}).$$
    For the trivial action of $\ld{T}$ on $\{1\} \times_{\ld{T}} \{1\}$, this is precisely $\IndCoh((\{1\} \times_{\ld{T}} \{1\})/\ld{T})$. Exactly the same discussion proves the second equivalence:
    $$\Loc_{T_c}(\Gr_T; k) \simeq \bigoplus_{\bX_\ast(T)} \Loc_{T_c}(\ast; k) \simeq \QCoh(B\ld{T}) \otimes_{\Mod_k} \QCoh(\cM_T).$$
    The claim about $\cf_T(\Gr_T)^\vee$ can be proved similarly.
\end{proof}
\begin{remark}
    Note that in \cref{thm: torus satake}, the ``spectral''/algebro-geometric description of $\Loc_{T_c}^\wedge(\Gr_T; k)$ does not seem to depend on the choice of coefficient $k$ (in particular, not on $\GG$). This dependence, however, can be made more explicit by noting that $\IndCoh(\{1\} \times_{\ld{T}} \{1\})$ is equivalent to $\Mod(k^{hT_c}) \simeq \QCoh(\widehat{\cM_T})$. That is, there is an equivalence $\Loc_{T_c}^\wedge(\Gr_T; k) \simeq \QCoh(\widehat{\cM_T}/\ld{T})$.
\end{remark}
Our basic goal is to find a replacement of \cref{thm: torus satake} where $\Gr_T$ is replaced by $\Gr_G$ for a general connected reductive group $G$.