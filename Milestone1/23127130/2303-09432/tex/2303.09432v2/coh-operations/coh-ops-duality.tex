We will momentarily review some of the rich theory of power operations in homotopy theory; these force the existence of additional structures on the Langlands dual side of \cref{thm: intro omnibus}. Our goal in this section is to describe these structures explicitly. This section is motivated by a discussion with David Treumann. 

Before proceeding, we warn the reader of a terminological mismatch. In \cite{lonergan-frob}, Lonergan uses ``Steenrod operators'' to construct new structures on Coulomb branches (and in particular, on $\ld{J}$). These operators, as we will explain in future work, are better viewed as \textit{$\E{3}$-power operations} coming from an $\E{3}$-algebra structure on $C^{G_c}_\ast(\Gr_G; \FF_p)$. While these are related to Steenrod operations in the usual sense of the word (as used by algebraic topologists), they are not the same. More generally, $\E{3}$-power operations on $\cf_G(\Gr_G)^\vee$ are closely related to, but distinct from, the power operations we will describe below. These $\E{3}$-power operations will be described in future work; there, we will prove a generalization to other $\Eoo$-rings of the ``Azumaya property'' of crystalline differential operators in characteristic $p$.

Let $k$ be an $\Eoo$-ring; we will momentarily specialize to the case when $k$ is $2$-periodic \textit{integral} cohomology, complex K-theory, or elliptic cohomology. The theory of power operations describes the additional structure acquired by $k$-cohomology from the $\Eoo$-structure on $k$. As we will see below, it is closely related to the structure of isogenies on the associated $1$-dimensional group scheme. This relationship is not new; we refer the reader to \cite{strickland-symmetric-gps, ando-power-operations, rezk-icm} for some sources.
\begin{construction}\label{cstr: power operations}
    Any $\Eoo$-ring $k$ admits a \textit{Tate-valued Frobenius} $k \to k^{t\Cp}$, which is given by the composite of the Tate diagonal $k \to (k^{\otimes p})^{t\Cp}$ with the $\Cp$-Tate construction of the multiplication map $k^{\otimes p} \to k$. See, e.g., \cite[Definition IV.1.1]{nikolaus-scholze} for a modern reference. 
    
    If $k$ admits additional structure, then this structure can be refined: namely, if $k$ admits a refinement to a normed algebra in the $\infty$-category of genuine $\Cp$-spectra (which will be true in the examples we will study), and $\Phi^{\Cp} k$ is its geometric fixed points, then the Tate-valued Frobenius $k \to k^{t\Cp}$ lifts to an $\Eoo$-map $\varphi: k \to \Phi^{\Cp} k$. This map is given by taking geometric fixed points of the $\Cp$-equivariant norm-multiplication map $N^{\Cp} k \to k$, where $N^{\Cp} k$ is the Hill-Hopkins-Ravenel norm from \cite{hhr}.
    
    If $X$ is any (finite) space, let $\cf_k(X)$ denote the $\Eoo$-$k$-algebra of $k$-cochains on $X$, and let $\cf_k(X)^\vee$ denote the $\Eoo$-$k$-coalgebra of $k$-chains on $X$. Then $\varphi$ induces maps
    $$\cf_k(X) \to \cf_{\Phi^{\Cp} k}(X), \ \cf_k(X)^\vee \to \cf_{\Phi^{\Cp} k}(X)^\vee.$$
    We will denote either of these maps by $\varphi_X$, and call them the \textit{decompleted Frobenius}. Sometimes, we will consider the further composites to $\cf_{k^{t\Cp}}(X)$ and $\cf_{k^{t\Cp}}(X)^\vee$; these composites exist for any $\Eoo$-ring $k$, even if it does not lift to a normed algebra in genuine $\Cp$-spectra.
\end{construction}
In the above context, one should view $\Phi^{\Cp} k$ as a decompletion of $k^{t\Cp}$; we will see this in \cref{ex: examples of power operations} below.
\begin{remark}\label{rmk: power op and tate frob}
    Let $I_\tr$ denote the transfer ideal in $\pi_0 \cf_k(X \times B\Cp)$, given by the image of the map $\pi_0 \cf_k(X) \to \pi_0 \cf_k(X \times B\Cp)$ induced by the transfer. On $\pi_0$, the map $\varphi_X: \cf_k(X) \to \cf_{k^{t\Cp}}(X)$ then factors as a composite
    $$\pi_0 \cf_k(X) \to \pi_0 \cf_k(X \times B\Cp)/I_\tr \to \pi_0 \cf_{k^{t\Cp}}(X).$$
    The first map in this composite is often referred to as the \textit{total power operation}. We will denote it by $\varphi^\tr_X$. It will not be used below in any serious way; we have mentioned it only for completeness.
\end{remark}
\begin{remark}\label{rmk: total power op and phiCp}
    \cref{cstr: power operations} might seem somewhat abstract, but it has very concrete consequences. Suppose, for simplicity, that $k$ is even and $2$-periodic, and that $\pi_0 \cf_k(X \times B\Cp) \cong \pi_0 \cf_k(X) \otimes_{\pi_0(k)} \pi_0 \cf_k(B\Cp)$. Under the assumption on $k$, this happens if, for instance, either $X$ is a finite space with even cells, or $\pi_0 \cf_k(B\Cp)$ is flat over $\pi_0(k)$. The total power operation is then a ring map
    $$\varphi_X^\tr: \pi_0 \cf(X) \to \pi_0 \cf_k(X) \otimes_{\pi_0(k)} \pi_0 \cf_k(B\Cp)/I_\tr.$$
    In fact, this can be upgraded to a map
    \begin{equation}\label{eq: Sigma-p power operation}
        \pi_0 \cf(X) \to \pi_0 \cf_k(X) \otimes_{\pi_0(k)} \pi_0 \cf_k(B\Sigma_p)/I_\tr,
    \end{equation}
    where $\Sigma_p$ is the symmetric group on $p$ letters.
    
    Moreover, under the hypothesis on $k$, there is an isomorphism $\pi_0 \cf_k(B\Cp) \cong \pi_0(k)\pw{t}/[p](t)$, where $[p](t)$ is the $p$-series of the formal group law over $\pi_0 \cf_k(\CP^\infty) \cong \pi_0(k)\pw{t}$.\footnote{Unfortunately, this $t$ is common practice in homotopy theory; but it conflicts with the $t$ which is the coordinate of the formal (punctured) disk used to define the affine Grassmannian. We will use the same symbol $t$ to denote both, and the distinction should be clear from context.} The composite of \cref{rmk: power op and tate frob} can be identified with the map
    $$\pi_0 \cf_k(X) \xrightarrow{\varphi_X^\tr} \pi_0 \cf_k(X) \otimes_{\pi_0(k)} \pi_0 \cf_k(B\Cp)/I_\tr \to \pi_0 \cf_k(X) \otimes_{\pi_0(k)} \pi_0 \cf_k(B\Cp)[1/t].$$
    If $k$ admits the structure of a normed algebra in genuine $\Cp$-spectra, then this composite factors through
    $$\pi_0 \cf_k(X) \xrightarrow{\varphi_X} \pi_0 \cf_k(X) \otimes_{\pi_0(k)} \pi_0 \Phi^{\Cp}(k) \to \pi_0 \cf_k(X) \otimes_{\pi_0(k)} \pi_0 \cf_k(B\Cp)[1/t].$$
    It follows, in particular, that $\varphi_X^\tr$ and $\varphi_X$ together define a map
    $$\pi_0 \cf_k(X) \to \pi_0 \cf_k(X) \otimes_{\pi_0(k)} \left(\pi_0 \Phi^{\Cp}(k) \times_{\pi_0 \cf_k(B\Cp)[1/t]} \pi_0 \cf_k(B\Cp)/I_\tr\right).$$
    The fiber product on the right-hand side does not have any denominators in $t$, and we will see this explicitly in the examples below.
\end{remark}
\begin{example}\label{ex: examples of power operations}
    Let us explicate the preceding remark in two examples.
    \begin{itemize}
        \item Suppose $k = \Z[u^{\pm 1}]$ with $u$ in degree $2$. Then $\pi_0 \cf(B\Cp) \cong \Z\pw{t}/pt$, and the transfer ideal is simply generated by $t$. Therefore, $\pi_0 \cf(B\Cp)/I_\tr \cong \FF_p\pw{t}$. If $X$ is a finite space with even cells, then the map of \cref{rmk: power op and tate frob} can be viewed as an (ungraded) map
        $$\H^\ast(X; \Z) \xrightarrow{\varphi_X^\tr} \H^\ast(X; \FF_p\pw{t}) \to \H^\ast(X; \FF_p\ls{t}).$$
        The decompleted Frobenius is given by an (ungraded) map
        $$\varphi_X: \H^\ast(X; \Z) \to \H^\ast(X; \FF_p[t^{\pm 1}]).$$
        Explicitly, these maps are given on a class $\alpha \in \H^\ast(X; \Z)$ by the formula
        $$\alpha \mapsto \sum_{i\geq 0} (-1)^i P^i(\alpha) t^{(p-1)i}.$$
        Here, $P^i$ is the $i$th Steenrod operation. That is to say, $\varphi_X$ encodes the action of the Steenrod operations on $\H^\ast(X; \Z)$.\footnote{This is perhaps bad terminology, because the Steenrod algebra does give an endomorphism of integral cohomology. Here, however, we are viewing the Steenrod algebra as acting on a class $\alpha$ in integral cohomology through its mod $p$ reduction $\ol{\alpha}$. Our $\varphi_X$ will only see the action of $P^i$ on $\ol{\alpha}$, and not the operations $\beta P^i$ (when $p=2$, this is $\Sq^{2i+1}$). In fact, it turns out that the decompleted Frobenius $\varphi: \Z \to \Phi^{\Cp} \Z$ factors as an $\Eoo$-map through the reduction map $\Z \to \FF_p$ (so that $\varphi_X(\alpha)$ depends only on $\ol{\alpha}$ in a coherently multiplicative way), but proving this is out of the scope of the present article.
        
        Let us mention that only tracking $P^i(\ol{\alpha})$ definitely loses some information about the entire Steenrod algebra action. First, since $\ol{\alpha}$ came from the integral class $\alpha$, its Bockstein $\beta(\ol{\alpha})$ vanishes. It is, however, possible that $\beta P^i(\ol{\alpha})$ be nonzero despite $\ol{\alpha}$ lifting to integral cohomology. For instance, if we identify $\H^\ast(\RP^4 \times \RP^4; \FF_2) = \FF_2[x,y]/(x^5, y^5)$, then the class $\ol{\alpha} = xy(x+y)$ lifts to integral cohomology, but $\Sq^3(\ol{\alpha}) = x^2 y^2 (x^2 + y^2) \neq 0$.} As expected by \cref{rmk: total power op and phiCp}, there are no denominators in $t$ in the above formula.
        For instance, if $X = \CP^n$ for any finite $n$, this map sends $x\in \H^2(\CP^n; \Z)$ to $x - t^{p-1} x^p$.
        \item Suppose $k = \KU$. Then $\pi_0 \cf(B\Cp) \cong \Z\pw{t}/((1+t)^p - 1)$, and the transfer ideal is simply generated by $t$. Therefore, 
        $$\pi_0 \cf(B\Cp)/I_\tr \cong \Z\pw{t}/\tfrac{(1+t)^p - 1}{t} \cong \Z[\zeta_p]^\wedge_t.$$
        Here, $\zeta_p$ is a primitive $p$th root of unity and $t = \zeta_p - 1$. Note that since $t^{p-1}$ is a unit multiple of $p$ in $\Z_p[\zeta_p]$, the $t$-completion above is equivalent to $p$-completion.
        The ring $\Z_p[\zeta_p]$ is flat over $\pi_0(k)^\wedge_p = \Z_p$, and so the composite of \cref{rmk: power op and tate frob} can be viewed as a ring map
        \begin{equation}\label{eq: KU completed tate frob}
            \KU^0(X) \xrightarrow{\varphi_X^\tr} \KU^0(X)[\zeta_p]^\wedge_p \to \KU^0(X)[\zeta_p]^\wedge_p[1/p]
        \end{equation}
        The geometric fixed points $\Phi^{\Cp} \KU$, on the other hand, has homotopy groups given by
        $$\pi_\ast \Phi^{\Cp} \KU \cong \Z[q^{\pm 1}, \beta^{\pm 1}][\tfrac{1}{(q-1)\cdots(q^{p-1}-1)}]/(q^p-1) \cong \Z[\zeta_p, \beta^{\pm 1}][1/p];$$
        the final isomorphism comes from noticing that $(\zeta_p-1)\cdots(\zeta_p^{p-1}-1)$ is $(-1)^{p-1} p$. The decompleted Frobenius is given by a ring map
        $$\varphi_X: \KU^0(X) \to \KU^0(X)[\zeta_p][1/p].$$
        Note that this map is, indeed, a de-$p$-adic completion of \cref{eq: KU completed tate frob}.
        Both $\varphi_X^\tr$ and $\varphi_X$ send a vector bundle $V$ to the $p$th Adams operation $\psi^p(V) \in \KU^0(X)$, viewed as a subalgebra of $\KU^0(X)[\zeta_p]^\wedge_p$ and of $\KU^0(X)[\zeta_p][1/p]$. As expected by \cref{rmk: total power op and phiCp}, there are no denominators in $t = \zeta_p - 1$ in this formula.
    \end{itemize}
\end{example}

In order to understand the interaction between these power operations and \cref{thm: intro omnibus}, we will need to port \cref{cstr: power operations} to the setting of genuine equivariant (co)homology. Namely, we need a decompletion of the map 
$$\varphi_{BS^1}: \cf_k(BS^1) \to \cf_{\Phi^{\Cp} k}(BS^1) \simeq \lim_n \cf_{\Phi^{\Cp} k}(\CP^n).$$
First, observe that this map factors through an $\Eoo$-map
$$\varphi_{BS^1}': \cf_k(BS^1) \to \Phi^{\Cp} k \otimes_k \cf_k(BS^1) \simeq \Phi^{\Cp} k \otimes_k \lim_n \cf_k(\CP^n);$$
the map from the target to $\cf_{\Phi^{\Cp} k}(BS^1)$ generally induces a strict inclusion on homotopy\footnote{For instance, take $k = \KU$. Then the map $\varphi_{BS^1}$ is given on homotopy by the map $\Z\pw{t} \to \Z[\zeta_p][1/p]\pw{t}$ which sends $t\mapsto (1+t)^p - 1$. This factors through a map $\Z\pw{t} \to \Z[\zeta_p]\pw{t}[1/p]$; this is the effect of the map $\varphi_{BS^1}'$ on homotopy. Note that there is a strict inclusion $\Z[\zeta_p]\pw{t}[1/p] \subseteq \Z[\zeta_p][1/p]\pw{t}$.}. Note that $\varphi_{BS^1}'$ can be viewed as a homomorphism
$$\hat{\GG} \times_{\spec k} \spec \Phi^{\Cp} k \to \hat{\GG},$$
where $\hat{\GG} = \spf \cf_k(BS^1)$.

We will now specialize to the case when $k$ is $2$-periodic {integral} cohomology, complex K-theory, or elliptic cohomology, and let $\GG$ denote $\GG_a$, $\GG_m$, or the spectral elliptic curve $E$ over $k$ (respectively). The choice of $\GG$ equips $k$ with a lift to the $\infty$-category of normed rings in genuine $\Cp$-spectra. As usual, let $\GG_0$ denote the underlying group scheme over $\pi_0(k)$. Our desired decompletion will then be given by a particular homomorphism
\begin{equation}\label{eq: phi on genuine S1 equiv}
    \varphi: \GG \times_{\spec k} \spec \Phi^{\Cp} k \to \GG.
\end{equation}
To describe it, we need to give a moduli-theoretic interpretation of $\Phi^{\Cp} k$. Let $\GG[p]$ denote the $p$-torsion subgroup of $\GG$, so that $\GG[p] = \Hom(\Z/p, \GG)$. 

There is a natural action of $\FF_p^\times$ on $\GG[p]$ given by sending $i\in \FF_p^\times$ to the multiplication-by-$i$ map $[i]$. Let $U\subseteq \GG[p]$ denote the open subscheme given by the complement of the closed subscheme
$$\bigcup_{i\in \FF_p^\times} \ker(\GG_0[p] \xrightarrow{[i]} \GG_0[p]) \subseteq \GG_0[p].$$
The following is a straightforward consequence of \cite[Proposition 2.25]{hausmann-meier}.
\begin{lemma}\label{lem: phi Cp and moduli problem}
    The spectral scheme $\spec \Phi^{\Cp} k$ is isomorphic to $U$ over $k$.
\end{lemma}
The spectral scheme $U \subseteq \GG[p]$ is specified by its underlying (classical) scheme $U_0 \subseteq \GG_0[p]$ over $\pi_0(k)$. If $Y$ is a $\pi_0(k)$-scheme, a map $Y \to U_0$ is equivalent to the data of a homomorphism $f: \Z/p \to \GG_Y = \GG \times_{\spec \pi_0(k)} Y$ such that $f(i)$ is not the identity section for $i\in \Z/p - \{0\}$. This implies that $f$ exhibits $\Z/p$ as a closed subgroup scheme of $\GG_Y$ which is isomorphic to the Cartier divisor $\sum_{j\in \FF_p} f(j)$.
\begin{construction}\label{cstr: genuine S1 frobenius}
    Over $U_0$, there is a universal isogeny $q_0: \GG_{0,U_0} \to \GG_{0,U_0}$ given by quotienting by the subgroup scheme $\Z/p \cong \sum_{j\in \FF_p} f(j)$. This isogeny defines an \textit{\'etale} morphism $\co_{\GG_{0,U_0}} \to \co_{\GG_{0,U_0}}$; so \cite[Theorem 7.5.0.6]{HA} implies that the isogeny $q_0$ lifts to a map $q: \GG_U \to \GG_U$ over $\spec \Phi^{\Cp} k$. (In general, $q_0$ is to be understood as an analogue for $\GG_0$ of the Artin-Schreier map on $\GG_a$.) The map \cref{eq: phi on genuine S1 equiv} is then given by the composite
    $$\GG_U \xrightarrow{q} \GG_U \simeq \GG \times_{\spec k} \spec \Phi^{\Cp} k \xrightarrow{\pr} \GG.$$
    We will denote its underlying map by
    $$\varphi_0: \GG_0 \times_{\spec \pi_0(k)} \spec \pi_0(\Phi^{\Cp} k) \to \GG_0.$$
\end{construction}
\begin{example}\label{ex: genuine isogeny examples}
    Let us explicate \cref{cstr: genuine S1 frobenius} in two examples.
    \begin{enumerate}
        \item\label{item: Z isogeny} Let $k = \Z[u^{\pm 1}]$ and $\GG = \GG_a$. Then $U_0 = \spec \FF_p[t^{\pm 1}]$, and the isogeny $q: \GG_{0,U_0} \to \GG_{0,U_0}$ is given by the Artin-Schreier map 
        $$x\mapsto x - t^{p-1} x^p.$$
        \item\label{item: KU isogeny} Let $k = \KU$ and $\GG = \GG_m$ with coordinate $y$. Then $U_0 = \spec \Z[\zeta_p][1/p]$, and $q: \GG_{0,U_0} \to \GG_{0,U_0}$ is given by the map 
        $$y\mapsto 1 + \prod_{j\in \FF_p} (y - \zeta_p^j) = y^p.$$
    \end{enumerate}
\end{example}
\begin{remark}\label{rmk: interpolating artin schreier}
    Let us mention for the sake of completeness that one can interpolate between the two cases in \cref{ex: genuine isogeny examples}, using the group scheme $\GG$ over connective complex K-theory $\ku$ studied in \cite{ku-rel-langlands}. (Using this, the results discussed below can be extended to the case $k = \ku$, too, but we will not address this here.) Let $\GG_\beta := \GG_0$ denote its underlying group scheme. Explicitly, $\GG_\beta$ is the group scheme over $\Z[\beta]$ given by $\spec \Z[\beta, v^{\pm 1}][\tfrac{v-1}{\beta}]$, where the group law is determined by $v \mapsto v \otimes v$. In an abuse of notation, we will also write $\GG_{f(\beta)}$ for an element $f(\beta)\in \Z[\beta]$ to denote the group scheme given by $\spec \Z[\beta, v^{\pm 1}][\tfrac{v-1}{f(\beta)}]$; hopefully this will not cause any confusion to the reader. We will (perhaps unexpectedly) define $t^{-1} := \tfrac{v-1}{\beta}$, and also define the scheme
    $$U_0 = \spec \Z[\beta, v^{\pm 1}][\tfrac{v-1}{\beta}, \tfrac{\beta^{p-1}}{(v-1)\cdots(v^{p-1}-1)}]/\tfrac{v^p-1}{\beta}.$$
    Note that $v = \zeta_p$ is a primitive $p$th root of unity, and $\beta = \tfrac{\zeta_p - 1}{t^{-1}}$. The scheme $U_0$ is rather remarkable: its fiber over the locus where $\beta$ is a unit is precisely $\spec \Z[\zeta_p, \beta^{\pm 1}][1/p]$, while its fiber over $\beta = 0$ is given by $\spec \FF_p[t^{\pm 1}]$. (In homotopy theory, $U_0$ arises as $\spec \pi_\ast (\Phi^{\Cp} \ku)$, where $\ku$ is connective complex K-theory.)
    
    Let $y$ denote the invertible coordinate on $\GG_{\beta,U_0}$, and let $x = \tfrac{y-1}{\beta}$. Then the map $q: \GG_{\beta,U_0} \to \GG_{p\beta,U_0}$ is given by the map $y \mapsto y^p$ and $\beta \mapsto p\beta$, so that it sends
    $$q: x = \tfrac{y-1}{\beta} \mapsto \tfrac{y^p - 1}{p\beta} = \tfrac{(1+\beta x)^p - 1}{p\beta}.$$
    We claim that, as a morphism over $\spec \Z[\beta]$, this map interpolates between the isogenies of Cases \ref{item: Z isogeny} and \ref{item: KU isogeny} in \cref{ex: genuine isogeny examples}. First, it is obvious that when $\beta$ is a unit, we simply recover Case \ref{item: KU isogeny}. Next, let us consider the fiber over $\beta = 0$. Recall that $\beta = (\zeta_p - 1)t$, so the binomial theorem gives
    $$q(x) = \tfrac{1}{p} \sum_{i=1}^p \binom{p}{i} \beta^{i-1} x^i = \sum_{i=1}^p \tfrac{(\zeta_p - 1)^{i-1}}{p} \binom{p}{i} t^{i-1} x^i.$$
    Almost all terms vanish modulo $\beta$, except for the terms $i=1,p$; one is left with
    $$q(x) \equiv x + \tfrac{(\zeta_p - 1)^{p-1}}{p} t^{p-1} x^p = x + \tfrac{t^{p-1} x^p}{[1]_{\zeta_p} \cdots [p-1]_{\zeta_p}} = x - t^{p-1} x^p \pmod{\beta},$$
    as desired. (Here, $[j]_q = \tfrac{q^j - 1}{q-1}$ is the $q$-integer corresponding to $j\in \Z$; we are using the fact that $[1]_{\zeta_p} \cdots [p-1]_{\zeta_p} \equiv -1\pmod{\zeta_p - 1}$, which amounts to the fact that $(p-1)! = -1 \in \FF_p$.) In general, $\ku$ gives a degeneration of power operations/the $p$th Adams operation on $\KU$ to power operations/the Steenrod algebra action on ordinary cohomology; this goes back (albeit not in the form presented above) to \cite[Proposition 6.4 and Theorem 6.5]{atiyah-power-operations}.
\end{remark}
For any compact torus $T_c$, we obtain a map 
$$\varphi_T: \cM_T \times_{\spec k} \spec \Phi^{\Cp} k \to \cM_T,$$
whose underlying map on classical $\pi_0(k)$-schemes will be denoted by $\varphi_{T,0}$.
If $X$ is any (ind-)finite $T_c$-space $X$, we then obtain maps
$$\cf_T(X) \to \varphi_{T,\ast} \varphi_T^\ast \cf_T(X), \ \cf_T(X)^\vee \to \varphi_{T,\ast} \varphi_T^\ast (\cf_T(X)^\vee).$$
We will denote these maps by $\varphi_{T,X}$, and call them the \textit{$T_c$-equivariant decompleted Frobenius}. Note that $\varphi_{T,X}$ on $\cf_T(X)$ is a map of $\Eoo$-algebras in $\QCoh(\cM_T)$, and similarly $\varphi_{T,X}$ on $\cf_T(X)^\vee$ is a map of $\Eoo$-coalgebras in $\QCoh(\cM_T)$. The map $\varphi_{T,X}$ in fact comes from a functor
$$\varphi_{T,X}: \Loc_T(X; k) \to \Loc_T(X; \Phi^{\Cp} k).$$
\begin{remark}
    It is easy to see that if $\cH(\GG_0, T, W)$ denotes the nil-Hecke algebra from \cref{def: nil-hecke} associated to a root system with torus $T$ and Weyl group $W$, then the map $\varphi_{T,0}$ induces a map $\cH(\GG_0, T, W) \to \cH(\GG_0, T, W) \otimes_{\pi_0(k)} \pi_0(\Phi^{\Cp} k)$. This map is very interesting, but we will postpone a detailed study of its combinatorial implications to a future article. When $\GG_0 = \GG_a$, for instance, this map describes the total Steenrod operation on the nil-Hecke algebra; similar ideas are explored in \cite{kitchloo-steenrod, beliakova-cooper}.
\end{remark}

We will now study the case $X = \Gr_G$, where $G$ is connected, almost simple, and simply-laced over $\cc$. For notational simplicity, we will write $\Loc^\gr_T(\Gr_G; \Phi^{\Cp} k)$ to denote the tensor product $\Loc^\gr_T(\Gr_G; k) \otimes_{\pi_0 k} \pi_0 (\Phi^{\Cp} k)$. The $T_c$-equivariant decompleted Frobenius on $\cf_T(\Gr_G)^\vee$ induces a functor 
\begin{equation}\label{eq: frob on loc gr}
    (\varphi_{T, \Gr_G})_\ast: \Loc^\gr_{T_c}(\Gr_G; k) \to \Loc^\gr_{T_c}(\Gr_G; \Phi^{\Cp} k).
\end{equation}
Moreover, the homomorphism $\varphi_{\ld{T},0}$ induces a map in the opposite direction on Cartier duals, and hence a morphism
\begin{equation}\label{eq: frob on Bun B^}
    \varphi_{T,0}: \Bun_{\ld{B}}^0(\GG_0^\vee) \times_{\spec \pi_0(k)} \spec \pi_0(\Phi^{\Cp} k) \to \Bun_{\ld{B}}^0(\GG_0^\vee).
\end{equation}
\begin{remark}
    In fact, it follows from \cref{rmk: total power op and phiCp} that one can replace $\pi_0(\Phi^{\Cp} k)$ above by the fiber product $\pi_0(\Phi^{\Cp} k) \times_{\pi_0 \cf_k(B\Cp)[1/t]} \pi_0 (\cf_k(B\Cp))/I_\tr$, which has the effect of working in a ``$t$-lattice'' inside $\pi_0(\Phi^{\Cp} k)$. For simplicity, we will ignore this point below, and just work with $\pi_0(\Phi^{\Cp} k)$.
\end{remark}
The following says that the map \cref{eq: frob on Bun B^} is precisely the effect of the $T_c$-equivariant decompleted Frobenius under Langlands duality.
\begin{theorem}\label{thm: frobenius and langlands}
    Under the equivalence of \cref{thm: intro omnibus} as rephrased in \cref{rmk: 1-shifted cartier} (which continues to hold true in the case $k = \Z[u^{\pm 1}]$, at least upon inverting enough primes), the functor $(\varphi_{\ld{T}, \Gr_G})_\ast$ of \cref{eq: frob on loc gr} identifies with the functor given by pullback along the map \cref{eq: frob on Bun B^}. That is, the following diagram commutes:
    $$\xymatrix{
    F \otimes_{\pi_0(k)} \Loc^\gr_{\ld{T}_c}(\Gr_G; k) \ar[r]^-{(\varphi_{\ld{T}, \Gr_G})_\ast} \ar[d]_-\sim & F \otimes_{\pi_0(k)} \Loc^\gr_{\ld{T}_c}(\Gr_G; \Phi^{\Cp} k) \ar[d]^-\sim \\
    \QCoh(\Bun_{\ld{B}}^0(\GG_0^\vee)^\reg) \ar[r]_-{\varphi_{\ld{T},0}^\ast} & \QCoh(\Bun_{\ld{B}}^0(\GG_0^\vee)^\reg) \otimes_{\pi_0(k)} \pi_0(\Phi^{\Cp} k).
    }$$
\end{theorem}
\begin{proof}
    The argument is essentially that of \cref{prop: cplx conj KU and B mod B^}, so we only give a sketch. Let us begin by observing that if $\kappa: \cM_{\ld{T}, 0} \to \Bun_{\ld{B}}^0(\GG_0^\vee)$ denotes the Kostant section, there is a commutative diagram
    $$\xymatrix{
    \cM_{\ld{T},0} \times_{\spec \pi_0(k)} \spec \pi_0(\Phi^{\Cp} k) \ar[d]_-\kappa \ar[r]^-{\varphi_{\ld{T},0}} & \cM_{\ld{T},0} \ar[d]^-\kappa \\
    \Bun_{\ld{B}}^0(\GG_0^\vee) \times_{\spec \pi_0(k)} \spec \pi_0(\Phi^{\Cp} k) \ar[r]_-{\varphi_{\ld{T},0}} & \Bun_{\ld{B}}^0(\GG_0^\vee).
    }$$
    The proof of \cref{thm: intro omnibus} shows that it suffices to prove that under the isomorphism
    \begin{equation}\label{eq: iso of reg centr for frob}
        \spec_{\cM_{\ld{T},0}}(\pi_0 \cf_{\ld{T}}(\Gr_G)^\vee) \cong \cM_{\ld{T}, 0} \times_{\Bun_{\ld{B}}^0(\GG_0^\vee)} \cM_{\ld{T},0},
    \end{equation}
    the $\ld{T}_c$-equivariant decompleted Frobenius on $\cf_{\ld{T}}(\Gr_G)^\vee$ identifies with the effect of the map $\varphi_{\ld{T},0}$ on the right-hand side. For brevity, we will phrase this condition as the ``Frobenius-equivariance'' of \cref{eq: iso of reg centr for frob}.
    
    Let $\cM_{\ld{T},0}^\gen \subseteq \cM_{\ld{T},0}$ denote the complement of $\bigcup_{\alpha} \cM_{\ld{T}_\alpha,0}$ as $\alpha$ ranges over the roots of $\ld{G}$, and $\ld{T}_\alpha$ denotes the kernel of the map $\alpha: \ld{T} \to \GG_m$. Since both sides of \cref{eq: iso of reg centr for frob} are flat over $\cM_{\ld{T},0}$, their sheaves of functions inject into the corresponding localizations along the map $\cM_{\ld{T},0}^\gen \subseteq \cM_{\ld{T},0}$. It therefore suffices to show that when restricted to $\cM_{\ld{T},0}^\gen$, the isomorphism of \cref{eq: iso of reg centr for frob} is Frobenius-equivariant.

    By \cref{lem: atiyah localization}, there is an isomorphism 
    $$\pi_0 \cf_{\ld{T}}(\Gr_G)^\vee|_{\cM_{\ld{T},0}^\gen} \cong \pi_0 \cf_{\ld{T}}(\Gr_T)^\vee|_{\cM_{\ld{T},0}^\gen} \cong \co_{\cM_{\ld{T},0}^\gen}[\bX_\ast(T)].$$
    Under this isomorphism, the $\ld{T}_c$-equivariant decompleted Frobenius is given simply by the Frobenius on $\cM_{\ld{T},0}^\gen$, and acts trivially on $\bX_\ast(T)$.
    Similarly, there is an isomorphism
    $$(\cM_{\ld{T}, 0} \times_{\Bun_{\ld{B}}^0(\GG_0^\vee)} \cM_{\ld{T},0}) \times_{\cM_{\ld{T},0}} \cM_{\ld{T},0}^\gen \cong \cM_{\ld{T},0}^\gen \times \ld{T}.$$
    Under this isomorphism, the action of $\varphi_{\ld{T},0}$ is given simply by the Frobenius on $\cM_{\ld{T},0}^\gen$, and acts trivially on $\ld{T}$. It is clear that this matches with the Frobenius on $\pi_0 \cf_{\ld{T}}(\Gr_G)^\vee|_{\cM_{\ld{T},0}^\gen}$, as desired.
\end{proof}
The entire discussion above can be adapted without much difficulty to the setting of $G_c$-equivariant local systems. If $G$ is almost simple and simply-laced, and has torsion-free fundamental group, then the analogue of \cref{thm: frobenius and langlands} states the following. Under the equivalence of \cref{rmk: 1-shifted cartier} (which continues to hold true in the case $k = \Z[u^{\pm 1}]$, at least upon inverting enough primes), the following diagram commutes:
$$\xymatrix{
F \otimes_{\pi_0(k)} \Loc^\gr_{\ld{G}_c}(\Gr_G; k) \ar[r]^-{(\varphi_{\ld{G}, \Gr_G})_\ast} \ar[d]_-\sim & F \otimes_{\pi_0(k)} \Loc^\gr_{\ld{G}_c}(\Gr_G; \Phi^{\Cp} k) \ar[d]^-\sim \\
\QCoh(\Bun_{\ld{G}}^\ss(\GG_0^\vee)^\reg) \ar[r]_-{\varphi_{\ld{G},0}^\ast} & \QCoh(\Bun_{\ld{G}}^\ss(\GG_0^\vee)^\reg) \otimes_{\pi_0(k)} \pi_0(\Phi^{\Cp} k).
}$$
The top and bottom maps are defined just as in \cref{eq: frob on loc gr} and \cref{eq: frob on Bun B^}. 
\begin{remark}\label{rmk: steenrod on G^?}
    Recall from the proof of \cref{thm: intro omnibus} that there is a closed immersion $\spec \pi_0 \cf_{\ld{T}}(\Gr_G)^\vee \hookrightarrow \ld{G} \times \cM_{T,0}$. One can try to extend the action of the decompleted Frobenius to $\ld{G} \times \cM_{T,0}$ itself, but such an extension will \textit{not} be canonical (and seems to be essentially useless in studying $\ld{G}$).
\end{remark}
\begin{remark}
    Let $\lambda$ be a dominant minuscule weight for $\ld{G}$, and let $G/P_\lambda$ denote the corresponding flag variety for $G$ as in \cref{table: minuscule varieties}. The decompleted Frobenius acts on $\pi_0 \cf_{\ld{T}}(G/P_\lambda)$, and does so compatibly with its action on $\pi_0 \cf_{\ld{T}}(\Gr_G)^\vee$, in the sense that the action of $\spec \pi_0 \cf_{\ld{T}}(\Gr_G)^\vee$ on $\pi_0 \cf_{\ld{T}}(G/P_\lambda)$ is equivariant for the decompleted Frobenius. It would be interesting to understand, in some uniform manner, the action of the decompleted Frobenius on $\pi_0 \cf_{\ld{T}}(G/P_\lambda)$. In the case of ordinary cohomology, this amounts to understanding Steenrod operations on $\H^\ast(G/P_\lambda; \Z)$. This is already interesting in the case when $G$ is of type $A$ (i.e., in the case of Grassmannians), where it was studied, for instance, in \cite{borel-serre-steenrod, borel-serre-steenrod-ii, lance-steenrod-dyer-lashof-BU}.
\end{remark}

Let us now explicate \cref{thm: frobenius and langlands} in some examples. Since the description in the case of elliptic cohomology is not much more explicit than the statement of \cref{thm: frobenius and langlands} -- that is, that the decompleted Frobenius on $\Bun_{\ld{G}}^\ss(E)$ is induced by the degree $p$ \'etale isogeny $E \to E$ over $\spec \pi_0(\Phi^{\Cp} k)$ -- we will mostly focus on the cases of ordinary cohomology and complex K-theory below for simplicity. We will also briefly discuss the example of ``Tate K-theory'', where one can also make the decompleted Frobenius explicit at the level of isomorphism classes of objects of $\Bun_{\ld{G}}^\ss(E)$.

Before proceeding, we warn the reader that our discussion above only shows that the decompleted Frobenius is canonically defined on the stack $\Bun_{\ld{G}}^\ss(\GG_0^\vee)$, and not necessarily on a uniformization. For instance, when $\GG_0 = \GG_a$, so that $\Bun_{\ld{G}}^\ss(\GG_0^\vee) = \ld{\g}/\ld{G}$, we will often compute the decompleted Frobenius as a map on $\ld{\g}$; but the resulting formulas are only unique up to $\ld{G}$-conjugation.
\begin{example}\label{ex: steenrod operations langlands}
    Let $k = \Z[u^{\pm 1}]$, $\GG = \GG_a$, and invert $N\gg 0$ so that the equivalence of \cref{cor: reg locus ordinary ABG} continues to hold: that is, so that there is an equivalence $\Loc^\gr_{\ld{T}_c}(\Gr_G; k) \simeq \QCoh(\ld{\fr{b}}^\reg/\ld{B})$. This can be proved by showing that the isomorphism of \cref{thm: ordinary hmlgy reg centr} over $\spec \Z[1/N]$ for some $N\gg 0$. In fact, \cite{homology-langlands} shows that one can take $N$ to be the integer $n_G$ from \cite[Remark 5.8]{homology-langlands}.\footnote{In fact, in the setup at hand, one can take $N = 1$.}
    Under the identification $\Bun_{\ld{B}}^0(\GG_0^\vee) \cong \ld{\fr{b}}/\ld{B}$, the map \cref{eq: frob on Bun B^} is given (for $p\nmid N$) by the map
    $$\varphi_{\ld{T},0}: (\ld{\fr{b}} \times_{\spec \Z[1/N]} \spec \FF_p[t^{\pm 1}])/\ld{B} \to \ld{\fr{b}}/\ld{B}$$
    which is the $\ld{B}$-quotient of the map
    $$\ld{\fr{b}} \times_{\spec \Z[1/N]} \spec \FF_p[t^{\pm 1}] \to \ld{\fr{b}}, \ (x,t) \mapsto x - t^{p-1} x^{[p]}.$$
    Here, $x^{[p]}$ denotes the restricted Lie operation on $\ld{\fr{b}}$. It follows from \cref{thm: frobenius and langlands} that this map implements the action of the decompleted Frobenius/Steenrod operations on $\Loc^\gr_{\ld{T}_c}(\Gr_G; \Z[u^{\pm 1}])$ (upon inverting $N \gg 0$).
    
    Similarly, under the identification $\Bun_{\ld{G}}^\ss(\GG_0^\vee) \cong \ld{\g}/\ld{G}$, the analogue of the map \cref{eq: frob on Bun B^} is given (for $p\nmid N$) by the map
    $$\varphi_{\ld{G},0}: (\ld{\g} \times_{\spec \Z[1/N]} \spec \FF_p[t^{\pm 1}])/\ld{G} \to \ld{\g}/\ld{G}$$
    which is the $\ld{G}$-quotient of the map
    \begin{equation}\label{eq: artin-schreier on ld-g}
        \ld{\g} \times_{\spec \Z[1/N]} \spec \FF_p[t^{\pm 1}] \to \ld{\g}, \ (x,t) \mapsto x - t^{p-1} x^{[p]}.
    \end{equation}
    Again, this map implements the action of the decompleted Frobenius/Steenrod operations on $\Loc^\gr_{\ld{G}_c}(\Gr_G; \Z[1/N,u^{\pm 1}])$ under the equivalence between $\Loc^\gr_{\ld{G}_c}(\Gr_G; \Z[1/N,u^{\pm 1}])$ and $\QCoh(\ld{\g}^\reg/\ld{G})$.

    For instance, suppose $G = \SL_2$, and assume $p>2$. When restricted to the Kostant slice $f + \ld{\g}^e = \left\{\begin{psmallmatrix}
        0 & x \\
        1 & 0
    \end{psmallmatrix}\right\} \subseteq \ld{\g} = \pgl_2$, the map $\varphi_{\ld{G},0}$ sends 
    $$\left(\begin{psmallmatrix}
        0 & x \\
        1 & 0
    \end{psmallmatrix}, t\right) \mapsto \begin{psmallmatrix}
        0 & x - t^{p-1} x^{(p+1)/2} \\
        1 - t^{p-1} x^{(p-1)/2} & 0
    \end{psmallmatrix}.$$
    This is conjugate to the matrix $\begin{psmallmatrix}
        0 & x(1 - t^{p-1} x^{(p-1)/2})^2 \\
        1 & 0
    \end{psmallmatrix}$, so we find that $\varphi_{\ld{G},0}$ is given in coordinates by the map 
    $$\varphi_{\ld{G},0}: x \mapsto x - 2 t^{p-1} x^{(p+1)/2} + t^{2(p-1)} x^p = \prod_{j\in \FF_p} (x - j^2 t^2)$$
    on $f + \ld{\g}^e$. Under the isomorphism $f + \ld{\g}^e \cong \spec \H^\ast_{\SU(2)}(\ast; \Z)$, the coordinate $x$ identifies with the first Pontryagin class $p_1$; and $\varphi_{\ld{G},0}(x)$ is exactly the total Steenrod operation on this class, as expected. Alternatively, one could conjugate the above description of $\varphi_{\ld{G},0}$ to find that the decompleted Frobenius acts on a binary quadratic form $q(x,y)\in \pgl_2 = \Sym^2(\AA^2)$ by
    $$\varphi_{\SL_2,0}: q(x,y) \mapsto (1 - t^{p-1} \det(q)^{(p-1)/2}) q(x,y),$$
    where $\det(q)$ is the discriminant of $q$.
\end{example}

\cref{ex: steenrod operations langlands} has the following algebraic consequence. This result is not new, and can be found in the literature as \cite[Section 4.1]{jantzen-kohomologie};
%\cite[Corollary 4.4]{mcninch-order-nilpotence};
it also holds in the non-simply-laced case. (The proof below is a thinly veiled topological analogue of Jantzen's argument.)
\begin{prop}\label{prop: pth power zero on nilcone}
    The map $x \mapsto x^{[p]}$ is zero on the nilpotent cone $\ld{\cN} \subseteq \ld{\g}$ if $p$ is at least the Coxeter number of $\ld{G}$.
\end{prop}
\begin{proof}
    The map $\varphi_{\ld{G},0}$ from \cref{eq: artin-schreier on ld-g} is given by taking affine closures of the map 
    $$\ld{\g}^\reg \times_{\spec \Z[1/N]} \spec \FF_p[t^{\pm 1}] \to \ld{\g}^\reg, \ (x,t) \mapsto x - t^{p-1} x^{[p]}.$$
    It suffices to show that $\varphi_{\ld{G},0}|_{\ld{\cN}^\reg}$ sends $x\mapsto x$. If we identify $\ld{\cN}^\reg = \ld{G}/Z_{\ld{G}}(e)$ and $Z_{\ld{G}}(e) = \spec \H_\ast(\Gr_G; \Z[1/N])$ by \cref{thm: ordinary hmlgy reg centr} (or, \cite[Theorem 6.1]{homology-langlands}), then $\varphi_{\ld{G},0}|_{\ld{\cN}^\reg}$ is induced by the map 
    $$Z_{\ld{G}}(e) \times_{\spec \Z[1/N]} \spec \FF_p[t^{\pm 1}] \to Z_{\ld{G}}(e)$$
    coming from the decompleted Frobenius/total Steenrod operation on $\H_\ast(\Gr_G; \Z[1/N])$. It therefore suffices to show that the decompleted Frobenius acts by the identity on $\H_\ast(\Gr_G; \Z[1/N])$.

    This can be proved using the generating complexes from \cite{bott-space-of-loops}, as elaborated upon in \cite{littig-mitchell}. Namely, recall that if $X$ is a homotopy commutative H-space and $f: Y \to X$ is a map from a CW-complex into $X$, then $f$ is said to exhibit $Y$ as a generating complex for $X$ if $f$ induces a surjection $\Sym(\H_\ast(Y; \Z[1/N])) \twoheadrightarrow \H_\ast(X; \Z[1/N])$. In \cite{littig-mitchell}, it was shown that if $\theta$ denotes the highest (short) coroot of $G$, then the Schubert variety $\ol{\Gr_G^{-\theta}}$ corresponding to the antidominant weight $-\theta$ is a generating complex for $\Gr_G$. Since $\H_\ast(\ol{\Gr_G^{-\theta}}; \Z[1/N])$ generates $\H_\ast(\Gr_G; \Z[1/N])$ as a ring, and the decompleted Frobenius is a ring map, it suffices to show that the decompleted Frobenius/total Steenrod operation on $\H_\ast(\ol{\Gr_G^{-\theta}}; \Z[1/N])$ sends $x\mapsto x$. Equivalently, it suffices to show that all Steenrod operations $P^i$ act trivially on $\H_\ast(\ol{\Gr_G^{-\theta}}; \Z[1/N])$ for $i>0$.
    
    To see this, observe that the dimension of $\ol{\Gr_G^{-\theta}}$ is given by $2(h-1)$, where $h$ is the Coxeter number of $\ld{G}$. The operation $P^i$ sends a class in $\H_\ast(\ol{\Gr_G^{-\theta}}; \Z[1/N])$ in homological degree $j$ to a class in $\H_\ast(\ol{\Gr_G^{-\theta}}; \FF_p)$ in homological degree $j - 2i(p-1)$. Since $\H_\ast(\ol{\Gr_G^{-\theta}}; \FF_p)$ is concentrated in nonnegative degrees and $p\geq h$, we see that $P^i$ could only possibly act nontrivially when $p=h$ and $i=1$, and that too only on classes in $\H_{2(h-1)}(\ol{\Gr_G^{-\theta}}; \Z[1/N])$. However, $P^1$ applied to such a class would land in $\H_0(\ol{\Gr_G^{-\theta}}; \FF_p)$. This implies that it is zero: any Steenrod operation landing in $\H_0(X; \FF_p)$ necessarily vanishes if $X$ is a connected space.
    %(since the $0$-cell stably splits off the suspension spectrum of $\ol{\Gr_G^{-\theta}}$).
\end{proof}
\begin{example}\label{ex: small p and steenrod on SLn}
    Running the argument of \cref{prop: pth power zero on nilcone} backwards tells us that if the map $x \mapsto x^{[p]}$ is not zero on the nilpotent cone $\ld{\cN} \subseteq \ld{\g}$, then the decompleted Frobenius/total Steenrod operation on $\H_\ast(\Gr_G; \Z)$ must be nontrivial. (The following example was shown to me by David Treumann, and was my impetus for more generally exploring the decompleted Frobenius.) Indeed, suppose (for simplicity) that $G = \SL_3$ and $p = 2$. Then the map $x \mapsto x^{[2]}$ is not zero on the nilpotent cone in $\fr{pgl}_3$, and in fact the map $\varphi: x \mapsto x - t^{p-1} x^{[p]}$ sends the principal nilpotent $e = \begin{psmallmatrix}
        0 & 1 & 0\\
        0 & 0 & 1\\
        0 & 0 & 0
    \end{psmallmatrix}$ to the principal nilpotent $\begin{psmallmatrix}
        0 & 1 & t\\
        0 & 0 & 1\\
        0 & 0 & 0
    \end{psmallmatrix}$. This is conjugate to $e$ itself by the matrix $n_e = \begin{psmallmatrix}
        1 & t & 0\\
        0 & 1 & 0\\
        0 & 0 & 0
    \end{psmallmatrix}$. Conjugating the centralizer $Z_{\ld{G}}(e)$ by $n_e$ sends 
    \begin{equation}\label{eq: frob on homology of SL3}
        \begin{psmallmatrix}
        1 & a & b\\
        0 & 1 & a\\
        0 & 0 & 1
        \end{psmallmatrix} \mapsto \begin{psmallmatrix}
        1 & a & b + at\\
        0 & 1 & a\\
        0 & 0 & 1
        \end{psmallmatrix}.
    \end{equation}
    Indeed, this is exactly how the decompleted Frobenius acts on $\H_\ast(\Gr_{\SL_3}; \Z) = \Z[a,b]$. (One can verify this by observing that the generating complex in this case is given by the map $\CP^2 \to \Gr_{\SL_3}$. The $2$- and $4$-cells of $\CP^2$ give the classes $a$ and $b$, respectively, and they are connected by the Steenrod square $\Sq^2$.)
    Note that since the action of the decompleted Frobenius on $Z_{\ld{G}}(e)$ is just conjugation by $n_e$, one can extend it to an action on all of $\ld{G}$. However, the element $n_e$ is not canonical, and a different choice of $n_e$ will act differently on $\ld{G}$.

    Recall from \cref{rmk: steenrod on G^?} that if $\lambda$ denotes a dominant minuscule weight for $\PGL_3$, the action of $\spec \H_\ast(\Gr_G; \Z)$ on $\H^\ast(\PGL_3/P_\lambda; \Z)$ must be equivariant for the decompleted Frobenius. Let us quickly verify this in the case when $\lambda$ is the fundamental weight: in this case, $\PGL_3/P_\lambda = \CP^2$, and if we write $\H^\ast(\PGL_3/P_\lambda; \Z) = \Z\{x,y,z\}$, the decompleted Frobenius sends $y \mapsto y + tz$. (Indeed, $\H^\ast(\CP^2; \Z) \cong \Z[w]/w^3$, and the total Steenrod operation sends $w \mapsto w + tw^2$. Writing $x = w^0$, $y = w$, and $z = w^2$ gives the desired claim.) It is straightforward to see that the action of $Z_{\ld{G}}(e)$ on $\Z\{x,y,z\}$ is equivariant for the decompleted Frobenius as described in \cref{eq: frob on homology of SL3}.
\end{example}

\begin{example}\label{ex: adams operations langlands}
    Let $k = \KU$ and $\GG = \GG_m$. Under the identification $\Bun_{\ld{B}}^0(\GG_0^\vee) \cong \ld{B}/\ld{B}$, the map \cref{eq: frob on Bun B^} is given by the $\ld{B}$-quotient of the $p$th power map on $\ld{B}$. That is, if $F$ is an algebraically closed field, then under the equivalence
    $$\Loc^\gr_{\ld{T}_c}(\Gr_G; \KU) \otimes_\Z F \simeq \QCoh(\ld{B}^\reg/\ld{B})$$
    of \cref{cor: ku reg locus ordinary ABG}, the decompleted Frobenius on the left-hand side (which encodes the $p$th Adams operation on $\KU$) identifies with the $p$th power map on $\ld{B}^\reg$. Similarly, under the equivalence
    $$\Loc^\gr_{\ld{G}_c}(\Gr_G; \KU) \otimes_\Z F \simeq \QCoh(\ld{G}^\reg/\ld{G}),$$
    the decompleted Frobenius on the left-hand side (which encodes the $p$th Adams operation on $\KU$) identifies with the $p$th power map on $\ld{G}^\reg$.

    For instance, suppose $\ld{G} = \SL_2$. When restricted to the Kostant slice inside $\ld{G} = \SL_2$ of matrices of the form $\begin{psmallmatrix}
        x-1 & x-2 \\
        1 & 1
    \end{psmallmatrix}$, the map $\varphi_{\ld{G},0}$ is given by raising to the $p$th power. It turns out that
    $$\begin{psmallmatrix}
        x-1 & x-2 \\
        1 & 0
    \end{psmallmatrix}^p \text{ is conjugate to } \kappa(x) = \begin{psmallmatrix}
        L_p(x)-1 & L_p(x)-2 \\
        1 & 1
    \end{psmallmatrix},$$
    where $L_n(x)$ is the $n$th ``Lucas polynomial'', given by
    $$L_n(x) = \sum_{j=0}^{\lfloor n/2\rfloor} (-1)^j \tfrac{n}{n-j} \binom{n-j}{j} x^{n-2j} = D_n(x,1).$$
    Here, $D_n(x,\alpha)$ is the ``Dickson polynomial'' from \cite{dickson-polynomial}. We therefore find that $\varphi_{\ld{G},0}$ is given on $g\in \SL_2$ by the map 
    $$\varphi_{\ld{G},0}(g) = L_p(g).$$
    Under the isomorphism between the Kostant slice for $\SL_2$ and $\spec \pi_0 \KU_{\SU(2)}$, the coordinate $x$ identifies with the $\KU$-theoretic Pontryagin class; and $\varphi_{\ld{G},0}(x)$ is exactly the $p$th Adams operation on this class. %(This formula for the $p$th Adams operation on $\pi_0 \KU_{\SU(2)}$ does not seem to be as well-known in the homotopy theory literature, but it is reasonably easy to deduce directly.)
\end{example}
\begin{remark}\label{rmk: adams connective ku}
    In fact, one can interpolate between \cref{ex: steenrod operations langlands} and \cref{ex: adams operations langlands} using the results of \cite{ku-rel-langlands} and \cref{rmk: interpolating artin schreier}. To state the result, we will use notation from \cref{rmk: interpolating artin schreier}. Namely, the aforementioned results imply that there are equivalences
    \begin{align}
        \Loc^\gr_{\ld{T}_c}(\Gr_G; \ku) \otimes_\Z F & \simeq \QCoh(\ld{B}_\beta^\reg/\ld{B}) \\
        \Loc^\gr_{\ld{G}_c}(\Gr_G; \ku) \otimes_\Z F & \simeq \QCoh(\ld{G}_\beta^\reg/\ld{G}), \label{eq: G-equiv reg satake connective ku}
    \end{align}
    where, for a group scheme $H$, we define $H_\beta$ to be (the stacky quotient by $\GG_m$ of) the $1$-parameter degeneration of $H$ into its Lie algebra. Explicitly, $H_\beta = \Hom(\DD(\GG_0), H)$, where $\DD(\GG_0)$ is the Cartier dual of the $1$-dimensional group scheme over $\spec(\pi_\ast(\ku))/\GG_m = \spec (\Z[\beta])/\GG_m$ from \cref{rmk: interpolating artin schreier}. For instance, if $H = \SL_n$, then $\SL_{n,\beta}$ consists of (the stacky quotient by $\GG_m$ of) the group scheme of those $n\times n$-matrices $x$ such that $\tfrac{\det(\id + \beta x) - 1}{\beta} = 0$.
    
    For simplicity, let us focus on the equivalence \cref{eq: G-equiv reg satake connective ku} above. The decompleted Frobenius on the topological side of \cref{eq: G-equiv reg satake connective ku} interpolates between the $p$th Adams operation and the total Steenrod operation, and it identifies with pullback along the map $(\ld{G}_\beta \times_{\spec \Z[\beta]} U_0)/\ld{G} \to \ld{G}_{p\beta}/\ld{G}$ given by the $\ld{G}$-quotient of the map on $\ld{G}_\beta$ defined by
    $$x\mapsto \tfrac{(1 + \beta x)^p - 1}{p\beta}.$$
    Using an argument similar to \cref{rmk: interpolating artin schreier}, one finds that when $\beta = 0$, the above map reduces to the Artin-Schreier map on $\ld{\g}$ from \cref{ex: steenrod operations langlands}. In the case $\ld{G} = \SL_2$, for instance, the decompleted Frobenius on the Kostant slice is given by $x\mapsto f_p(x)$, where $f_n(x)$ is the polynomial 
    %defined recursively by the formula\footnote{For instance, $f_2(x) = 4x - \beta^2 x^2$, $f_3(x) = \beta^4 x^3 - 6\beta^2 x^2 + 9x$, and $f_5(x) = \beta^8 x^5 - 10 \beta^6 x^4 + 35 \beta^4 x^3 - 50 \beta^2 x^2 + 25 x$.}
    %$$f_1(x) = x, \ f_n(x) = f_{n-1}(x) + (2n-1)x - \beta^2 x (f_{n-1}(x) + f_{n-2}(x) + \cdots + f_1(x)).$$
    given by\footnote{For instance, $f_2(x) = 4x - \beta^2 x^2$, $f_3(x) = \beta^4 x^3 - 6\beta^2 x^2 + 9x$, and $f_5(x) = \beta^8 x^5 - 10 \beta^6 x^4 + 35 \beta^4 x^3 - 50 \beta^2 x^2 + 25 x$.}
    $$f_n(x) = \sum_{j=0}^{n - 1} (-1)^j \tfrac{2n}{2n-j} \binom{2n-j}{j} \beta^{2(n-j)-2} x^{n-j} = \beta^{-2} (D_{2n}(\beta x^{1/2}, 1) - 2),$$
    where $D_n(x,\alpha)$ is the ``Dickson polynomial'' from \cite{dickson-polynomial}.
    Elementary arithmetic manipulations confirm that the polynomial $f_p(x)$ indeed computes the decompleted Frobenius on $\spec \pi_0 \ku_{\SU(2)}$, and furthermore that upon writing $\beta = (\zeta_p - 1)t$ in $\co_{U_0}$, we have 
    $$\tfrac{f_p(x)}{(\zeta_p - 1)^{2(p-1)}} = \sum_{j=0}^{p - 1} \tfrac{(-1)^j}{(\zeta_p - 1)^{2j}} \tfrac{2p}{2p-j} \binom{2p-j}{j} t^{2(p-1-j)} x^{p-j}.$$
    Upon reducing modulo $\zeta_p-1$, only the terms indexed by $j=0,\tfrac{p-1}{2}$, and $p-1$ survive. When $j=\tfrac{p-1}{2}$, the coefficient of $t^{p-1} x^{(p+1)/2}$ is 
    $$\tfrac{(-1)^{(p-1)/2}}{(\zeta_p - 1)^{p-1}} \tfrac{4p}{3p+1} \binom{(3p+1)/2}{(p-1)/2} \equiv -2\pmod{(\zeta_p-1)},$$
    so that
    $$\tfrac{f_p(x)}{(\zeta_p - 1)^{2(p-1)}} \equiv x - 2t^{p-1} x^{(p+1)/2} + t^{2(p-1)} x^p \pmod{(\zeta_p - 1)}.$$
    This is exactly as expected from \cref{ex: steenrod operations langlands}.
    %closely related to the ``Dickson polynomial'' \cite{dickson-polynomial} of degree $p$ with parameter given by the Bott class $\beta$.
\end{remark}
\begin{example}\label{ex: small p and adams on SLn}
    For the sake of completeness, let us explain the analogue of the calculation in \cref{ex: small p and steenrod on SLn} for $\KU$, so that $G = \SL_3$ and $p = 2$.  The map $\varphi: x \mapsto x^2$ sends $e = \begin{psmallmatrix}
        1 & 1 & 0\\
        0 & 1 & 1\\
        0 & 0 & 1
    \end{psmallmatrix}$ to $e^2 = \begin{psmallmatrix}
        1 & 2 & 1\\
        0 & 1 & 2\\
        0 & 0 & 1
    \end{psmallmatrix}$. This is conjugate to $e$ itself by the matrix $n_e = \begin{psmallmatrix}
        4 & 1 & 0\\
        0 & 2 & 0\\
        0 & 0 & 1
    \end{psmallmatrix}$. Conjugating the centralizer $Z_{\ld{G}}(e)$ by $n_e$ sends 
    \begin{equation}\label{eq: frob on KU homology of SL3}
        \begin{psmallmatrix}
        1 & a & b\\
        0 & 1 & a\\
        0 & 0 & 1
        \end{psmallmatrix} \mapsto \begin{psmallmatrix}
        1 & 2a & a + 4b\\
        0 & 1 & 2a\\
        0 & 0 & 1
        \end{psmallmatrix}.
    \end{equation}
    Indeed, this is exactly how the decompleted Frobenius acts on $\KU_0(\Gr_{\SL_3}) = \Z[a,b]$.
    %\footnote{Similarly, the decompleted Frobenius acts on $\ku_\ast(\Gr_{\SL_3}) \cong \Z[\beta][a',b']$ by sending $a' \mapsto 2a'$ and $b' \mapsto \beta a' + 4b'$. Note that here, $a'$ lives in weight $2$ and $b'$ lives in weight $4$. The classes $a,b \in \KU_0(\Gr_{\SL_3})$ are given by $a = a' \beta^{-1}$ and $b = b' \beta^{-2}$. Note that these formulas imply that $a \mapsto 2a$ and $b \mapsto a + 4b$, as expected.} 
    (One can verify this by observing that the generating complex in this case is given by the map $\CP^2 \to \Gr_{\SL_3}$. The $2$- and $4$-cells of $\CP^2$ give the classes $a$ and $b$, respectively, and the Adams operation $\psi^2$ sends $a\mapsto 2a$ and $b\mapsto a + 4b$.)
    Again, since the action of the decompleted Frobenius on $Z_{\ld{G}}(e)$ is just conjugation by $n_e$, one can extend it to an action on all of $\ld{G}$; but the element $n_e$ is not canonical, and a different choice of $n_e$ will act differently on $\ld{G}$.

    Recall from \cref{rmk: steenrod on G^?} that if $\lambda$ denotes a dominant minuscule weight for $\PGL_3$, the action of $\spec \KU_0(\Gr_G)$ on $\KU^0(\PGL_3/P_\lambda)$ must be equivariant for the decompleted Frobenius. Let us quickly verify this in the case when $\lambda$ is the fundamental weight: in this case, $\PGL_3/P_\lambda = \CP^2$, and if we write $\KU^0(\PGL_3/P_\lambda) = \Z\{x,y,z\}$, the decompleted Frobenius sends $y \mapsto z + 2y$ and $z \mapsto 4z$. (Indeed, $\KU^0(\CP^2) \cong \Z[w]/w^3$, and the Adams operation $\psi^2$ is given by the ring map sending $w \mapsto w^2 + 2w$. Writing $x = w^0$, $y = w$, and $z = w^2$ gives the desired claim.) It is straightforward to see that the action of $Z_{\ld{G}}(e)$ on $\Z\{x,y,z\}$ is equivariant for the decompleted Frobenius as described in \cref{eq: frob on KU homology of SL3}.
\end{example}

\begin{example}
    Let $k$ denote \textit{Tate K-theory} \cite[Section 2.7]{ando-hopkins-strickland}, so that $k = \KU\ls{q}$ and $\GG$ is a lift to $k$ of the Tate elliptic curve $\GG_0 = \tate(q)$ over $\Z\ls{q} = \pi_0(k)$. (See \cite[Section 4.3]{survey} for a sketch of the construction of $\GG$.) As usual, we will identify $\tate(q)^\vee$ with $\tate(q)$. Take $F = \cc$, and let $q$ be a point in the punctured open unit disk, so that it defines a continuous embedding $\Z\ls{q} \hookrightarrow \cc$. Then there are equivalences
    \begin{align*}
        \Loc^\gr_{\ld{T}_c}(\Gr_G; \KU\ls{q}) \otimes_{\Z\ls{q}} \cc & \simeq \QCoh(\Bun_{\ld{B}}^0(\tate(q))^\reg) \\
        \Loc^\gr_{\ld{G}_c}(\Gr_G; \KU\ls{q}) \otimes_{\Z\ls{q}} \cc & \simeq \QCoh(\Bun_{\ld{G}}^\ss(\tate(q))^\reg).
    \end{align*}
    The ring $\pi_0 \Phi^{\Cp} k$ and the Frobenius $\varphi: \pi_0(k) \to \pi_0 \Phi^{\Cp} k$ can be computed explicitly using \cref{lem: phi Cp and moduli problem}.
    %. In this case, one can compute (see, e.g., \cite[Theorem 3.5]{huan-finite-subgroup-tate}) that there is an isomorphism
    %$$\pi_0 \cf_k(B\Sigma_p)/I_\tr \cong \Z\ls{q'}[q]/({q'}^p - q) \times \Z\ls{q'}[q]/(q^p - q').$$
    %The Frobenius map $\varphi: \Z\ls{q} \to \pi_0 \cf_k(B\Sigma_p)/I_\tr$ sends $q\mapsto (q, q)$.
    We will not review the precise description here; instead, we only note that $\varphi$ sends $q\mapsto q^p$ on homotopy, and refer the reader to \cite[Section 6.3]{ando-power-operations} and \cite[Theorem 3.5]{huan-finite-subgroup-tate} for a description of the degree $p$-isogeny $\varphi^\ast \tate(q) \to \varphi^\ast \tate(q)$. This isogeny defines a map $\Bun_{\ld{G}}^\ss(\varphi^\ast \tate(q)) \to \Bun_{\ld{G}}^\ss(\tate(q))$, pullback along which identifies (by \cref{thm: frobenius and langlands}) with the decompleted Frobenius $\Loc^\gr_{\ld{G}_c}(\Gr_G; \KU\ls{q}) \to \Loc^\gr_{\ld{G}_c}(\Gr_G; \Phi^{\Cp} \KU\ls{q})$.

    In \cite{baranovsky-ginzburg}, Baranovsky and Ginzburg explicitly describe the set of $\cc$-points of $\Bun_{\ld{G}}^\ss(\tate(q))$. Namely, define the $q$-twisted conjugation action $G\ls{z}$ on itself as follows:
    $$\Ad^q_{h(z)}(g(z)) := h(qz) g(z) h(z)^{-1}.$$
    Then, there is a natural bijection between $\Bun_{\ld{G}}^\ss(\tate(q))(\cc)$ and the set of those $q$-twisted conjugacy classes in $G\ls{z}$ which contain an element of $G\pw{z}$. Under this bijection, one can show that the decompleted Frobenius on $\Bun_{\ld{G}}^\ss(\tate(q))(\cc)$ can be identified with the effect of the map
    $$g(z) \mapsto g(q^{p-1} z) g(q^{p-2} z) \cdots g(qz) g(z)$$
    on $q$-twisted conjugacy classes in $G\ls{z}$.
    %\footnote{Let us verify that this operation does indeed preserve $q$-twisted conjugacy. Suppose $g_2(z) = h(qz) g_1(z) h(z)^{-1}$. Recall that $\varphi(q) = q$. Then
    %$$\varphi(g_2(z)) = g_2(q^{p-1} z) g_2(q^{p-2} z) \cdots g_2(qz) g_2(z) = h(q^p z) \varphi(g_1(z)) h(z)^{-1}.$$}
\end{example}

The structures imposed by \cref{thm: frobenius and langlands} are quite rigid. For instance, there is an action of $\Loc^\gr_{\ld{G}_c}(\Gr_G; k)$ on $\Loc^\gr_{\ld{T}_c}(\Gr_G; k)$ by convolution, which, under the equivalences of \cref{thm: intro omnibus} as rephrased in \cref{rmk: 1-shifted cartier}, defines an action of $\QCoh(\Bun_{\ld{G}}^\ss(\GG_0^\vee)^\reg)$ on $\QCoh(\Bun_{\ld{B}}^0(\GG_0^\vee)^\reg)$. This action is given by pullback along the map $\Bun_{\ld{B}}^0(\GG_0^\vee) \to \Bun_{\ld{G}}^\ss(\GG_0^\vee)$, and it is compatible with power operations.
\begin{example}\label{ex: steenrod on TA2}
    When $k = \Z[u^{\pm 1}]$ and $\GG = \GG_a$ (where we again invert some $N \gg 0$ so that the equivalence of \cref{cor: reg locus ordinary ABG} holds), the action of $\Loc^\gr_{G_c}(\Gr_G; k)$ on $\Loc^\gr_{T_c}(\Gr_G; k)$ by convolution identifies with the action of $\QCoh(\ld{\g}^{\ast,\reg}/\ld{G})$ on $\QCoh(\ld{\fr{n}}^{\perp,\reg}/\ld{B})$ via pullback along the map $\ld{\fr{n}}^{\perp,\reg}/\ld{B} \to \ld{\g}^{\ast,\reg}/\ld{G}$. It follows from \cref{ex: steenrod operations langlands} that this map is compatible with the decompleted Frobenius/Steenrod operations.
    
    The composite map
    $$\ld{\fr{n}}^{\perp,\reg}/\ld{N} \to \ld{\fr{n}}^{\perp,\reg}/\ld{B} \to \ld{\g}^{\ast,\reg}/\ld{G}$$
    can be realized as the $\ld{G}$-quotient of the restriction to regular loci of the moment map $\mu: T^\ast(\ld{G}/\ld{N}) \to \ld{\g}^\ast$. The action of the decompleted Frobenius/Steenrod operations on the regular locus of $T^\ast(\ld{G}/\ld{N})$ in fact extends to all of $T^\ast(\ld{G}/\ld{N})$ itself (and hence on its affine closure $\ol{T^\ast(\ld{G}/\ld{N})}$), and the moment map $\mu$ is equivariant for this action. The action of the decompleted Frobenius on $\ol{T^\ast(\ld{G}/\ld{N})}$ commutes with the Gelfand-Graev action of the Weyl group from \cref{prop: ordinary gelfand-graev}; this can be seen by reducing to the rank $1$ case described below (with a bit of care in keeping track of the difference between $\AA^2$ and $(\AA^2)^\ast$).
    
    An explicit description of this action when $\ld{G} = \SL_2$ is as follows. If we identify $\ol{T^\ast(\ld{G}/\ld{N})} = T^\ast(\AA^2)$ with coordinates $(u,v) \in \AA^2 \oplus (\AA^2)^\ast$, then the total power operation is given by the map
    $$\varphi: (u,v) \mapsto (u, v - t^{p-1} v\pdb{u,v}^{p-1}).$$
    Since the moment map $T^\ast(\AA^2) \to \sl_2^\ast \cong \pgl_2$ sends $(u,v) \mapsto \begin{psmallmatrix}
        u_1 v_1 & u_1 v_2 \\
        u_2 v_1 & u_2 v_2
    \end{psmallmatrix}$, it is easy to check that this map is compatible with the action of the decompleted Frobenius on $\sl_2^\ast$ as described in \cref{ex: steenrod operations langlands}.
\end{example}
\begin{example}\label{ex: adams on mult quiver}
    When $k = \KU$ and $\GG = \GG_m$, the action of $\Loc^\gr_{G_c}(\Gr_G; k)$ on $\Loc^\gr_{T_c}(\Gr_G; k)$ by convolution identifies with the action of $\QCoh(G^\reg/\ld{G})$ on $\QCoh(B^\reg/\ld{B})$ via pullback along the map $B^\reg/\ld{B} \to G^\reg/\ld{G}$. It follows from \cref{ex: steenrod operations langlands} that this map is compatible with the decompleted Frobenius/$p$th Adams operation.
    The composite map
    $$B^\reg/\ld{N} \to B^\reg/\ld{B} \to G^\reg/\ld{G}$$
    can be realized as the $\ld{G}$-quotient of the restriction to regular loci of the multiplicative moment map $\mu: \ld{G}\times^{\ld{N}} B \to G$. The action of the decompleted Frobenius/$p$th Adams operation on the regular locus of $\ld{G}\times^{\ld{N}} B$ in fact extends to all of $\ld{G}\times^{\ld{N}} B$ itself (and hence on its affine closure $\ol{\ld{G}\times^{\ld{N}} B}$), and the moment map $\mu$ is equivariant for this action. The action of the decompleted Frobenius on $\ol{\ld{G}\times^{\ld{N}} B}$ commutes with the Gelfand-Graev action of the Weyl group from \cref{prop: ku gelfand-graev}; this can be seen by reducing to the rank $1$ case described below (with a bit of care in keeping track of the difference between $\AA^2$ and $(\AA^2)^\ast$).
    
    An explicit description of the action of the decompleted Frobenius when $\ld{G} = \SL_2$ is as follows. As in \cref{ex: Z/2 multiplicative symplectic fourier}, we may identify $\ol{\ld{G}\times^{\ld{N}} B}$ with an open subset of $T^\ast(\AA^2)$ with coordinates $(u,v) \in \AA^2 \oplus (\AA^2)^\ast$. The total power operation is then given by the map
    $$\varphi: (u,v) \mapsto \left(u, v\tfrac{(1 + \pdb{u,v})^p - 1}{\pdb{u,v}}\right).$$
    Since the moment map $\ol{\ld{G}\times^{\ld{N}} B} \to \PGL_2$ sends $(u,v) \mapsto \begin{psmallmatrix}
        1 + u_1 v_1 & u_1 v_2 \\
        u_2 v_1 & 1 + u_2 v_2
    \end{psmallmatrix}$, it is easy to check that this map is compatible with the action of the $p$th power map on $\PGL_2$ as described in \cref{ex: adams operations langlands}. In checking that the total power operation is compatible with the Gelfand-Graev action as described in \cref{ex: Z/2 multiplicative symplectic fourier}, the basic input is the identity $q^{-1} [p]_{q^{-1}} = q^{-p} [p]_q$ applied to $q = 1 + \pdb{u,v}$ (where $[p]_q = \tfrac{q^p - 1}{q-1}$).
\end{example}
More generally, (a mild variant of) the relative Langlands program from \cite{bzsv} predicts that if $X$ is an affine spherical $G$-variety, there exists a graded affine Hamiltonian $\ld{G}$-variety $\ld{M}$ over $\Z$ (possibly with an integer $N \gg 0$ inverted) with moment map $\mu: \ld{M} \to \ld{\g}^\ast$ such that there is an equivalence
$$\Shv_{G\pw{t}}^c(X\ls{t}; \Z) \simeq \Perf^{\sh}(\ld{M}/\ld{G}).$$
Here, $\Perf^{\sh}(\ld{M}/\ld{G})$ denotes the $\infty$-category of perfect complexes on the shearing of $\ld{M}$ with respect to its grading. Moreover, under a $\Z$-linear analogue of the derived geometric Satake equivalence, the natural action of $\Shv_{G\pw{t}}^c(\Gr_G; \Z)$ on the left-hand side by convolution should identify with the action of $\Perf(\ld{\g}^\ast[2]/\ld{G})$ on $\Perf^\sh(\ld{M}/\ld{G})$ via pullback along the moment map. This equivalence will restrict (and degenerate) to an equivalence
$$\Loc_{G\pw{t}}^\gr(X\ls{t}; \Z) \simeq \Perf(\ld{M}^\reg/\ld{G})$$
for some open $\ld{M}^\reg \subseteq \ld{M}$, which again satisfies a form of Hecke compatibility. Following the discussion above, the left-hand side will admit an action of the decompleted Frobenius/Steenrod operations, and so one expects the right-hand side to also admit such a structure. That is to say, $\ld{M}$ should admit an action of the decompleted Frobenius, and the moment map $\mu: \ld{M} \to \ld{\g}^\ast$ should be compatible with this action; here, $\ld{\g}^\ast$ is equipped with the action of the decompleted Frobenius described in \cref{ex: steenrod operations langlands}. It is worth remarking that this picture of relative Langlands duality only predicts that the decompleted Frobenius/Steenrod operations only act canonically on the \textit{stack} $\ld{M}/\ld{G}$, and that any formula one writes on $\ld{M}$ will not be canonical. This will be abundantly clear in the examples below, where it is obvious that the formulas we write are not unique (but any other choice will be an $\ld{G}$-translate of our formulas). In any case, these extra symmetries on $\ld{M}/\ld{G}$ are very interesting, and we expect them to play an important role in positive-characteristic analogues of the relative Langlands program.

In \cite{ku-rel-langlands}, we propose a version of this picture for sheaves with coefficients in $\KU$ (and more generally in $\ku$): the main difference is that $\ld{M}$ must be replaced by a \textit{quasi-Hamiltonian} $\ld{G}$-variety in the sense of \cite{amm-qham}, so that its moment map goes from $\ld{M}$ to $\ld{G}$. Again, $\ld{M}$ (or more canonically, $\ld{M}/\ld{G}$) should admit an action of the decompleted Frobenius/$p$th Adams operation on $\KU$, and the multiplicative moment map $\ld{M} \to \ld{G}$ should be compatible with this action, where the action of the decompleted Frobenius on $\ld{G}$ is as described in \cref{ex: adams operations langlands}. Outside of simple cases like \cref{ex: adams on mult quiver}, the quasi-Hamiltonian varieties can be quite complicated; so we will not discuss this case below. 

Let us present two explicit and nontrivial examples of ``Frobenius compatibility'' in the context of relative Langlands. The simplest is perhaps the following example, which generalizes \cref{ex: steenrod on TA2} and \cref{ex: adams on mult quiver}.
\begin{example}[Mirabolic Satake]\label{ex: mirabolic satake}
    This example is concerned with the relative Langlands dual to $G = \GL_n \times \GL_{n-1}$ acting on $G/\GL_{n-1}^\mathrm{diag}$.
    In \cite{mirabolic-satake}, it was shown that there is an equivalence
    $$\Shv_{\GL_{n-1}\pw{t}}^{c,\mathrm{Sat}}(\Gr_{\GL_n}; \QQ) \simeq \Perf^{\sh}(T^\ast \Hom(\AA^n, \AA^{n-1})/(\GL_n \times \GL_{n-1})),$$
    where, if we identify $T^\ast \Hom(\AA^n, \AA^{n-1})$ with $\Hom(\AA^{n-1}, \AA^n) \oplus \Hom(\AA^n, \AA^{n-1})$, the moment map $\mu: T^\ast \Hom(\AA^n, \AA^{n-1}) \to \gl_n^\ast \times \gl_{n-1}^\ast$ sends 
    $$\mu: (f,g)\mapsto (fg, gf).$$
    The equivalence of categories above will continue to hold over $\Z[1/N]$ for some $N \gg 0$, so we may consider the decompleted Frobenius for $p\nmid N$.
    Unwinding the proof of the above equivalence shows that the decompleted Frobenius/Steenrod algebra acts on $T^\ast \Hom(\AA^n, \AA^{n-1})$ via
    $$\varphi: (f,g) \mapsto (f, g - t^{p-1} g (fg)^{p-1}).$$
    It is easy to check that the moment map is indeed Frobenius-equivariant.

    There is also a multiplicative version of this picture. Namely, it follows from \cite[Remark 4.3.4]{ku-rel-langlands} that there is an equivalence
    $$\Loc_{\GL_{n-1}\pw{t}}^\gr(\Gr_{\GL_n}; \KU) \simeq \Perf(\cB(\AA^n, \AA^{n-1})^\reg/(\GL_n \times \GL_{n-1})),$$
    where $\cB(\AA^n, \AA^{n-1})^\reg$ is a particular open subset inside Van den Bergh's variety from \cite{van-den-bergh-double-poisson}:
    $$\cB(\AA^n, \AA^{n-1}) = \{(f,g) \in \Hom(\AA^{n-1}, \AA^n) \oplus \Hom(\AA^n, \AA^{n-1}) | \id + fg \in \GL_n\}.$$
    There is a multiplicative moment map $\mu: \cB(\AA^n, \AA^{n-1}) \to \GL_n \times \GL_{n-1}$ which sends 
    $$\mu: (f,g)\mapsto (\id + fg, \id + gf).$$
    The decompleted Frobenius/$p$th Adams operation acts on $\cB(\AA^n, \AA^{n-1})$ via
    $$\varphi: (f,g) \mapsto (f, f^{-1} ((\id + fg)^p - \id)),$$
    and again, the multiplicative moment map is Frobenius-equivariant.
\end{example}
\begin{example}
    In \cite{mirabolic-satake}, it was also shown that there is an equivalence
    $$\Shv_{\GL_n\pw{t}}^{c,\mathrm{Sat}}(\Gr_{\GL_n} \times \AA^n\ls{t}; \QQ) \simeq \Perf^{\sh}(T^\ast \gl_n/(\GL_n \times \GL_n)),$$
    where, if we identify $T^\ast \gl_n$ with $\gl_n \oplus \gl_n$, the moment map $\mu: T^\ast \gl_n \to \gl_n^\ast \times \gl_n^\ast$ sends 
    $$\mu: (f,g)\mapsto (fg, gf).$$
    Such an equivalence will continue to hold over $\Z[1/N]$ for some $N \gg 0$, so we may consider the decompleted Frobenius for $p\nmid N$. 
    Unwinding the proof of the above equivalence shows that, just as in \cref{ex: mirabolic satake}, the decompleted Frobenius/Steenrod algebra acts on $T^\ast \gl_n$ via
    $$\varphi: (f,g) \mapsto (f, g - t^{p-1} g (fg)^{p-1}).$$
    Again, it is easy to check that the moment map is indeed Frobenius-equivariant.
\end{example}

\begin{example}[Symplectic period]\label{ex: symplectic period}
    The ``quaternionic'' Satake equivalence is concerned with the relative Langlands dual to $G = \GL_{2n}$ acting on $\GL_{2n}/\Sp_{2n}$.
    The main result of \cite{quat-satake} says that there is an equivalence
    $$\Shv_{\GL_{2n}\pw{t}}^c(\GL_{2n}\ls{t}/\Sp_{2n}\ls{t}; \QQ) \simeq \Perf^{\sh}(\ld{M}/\GL_{2n}),$$
    where $\ld{M} \cong \GL_{2n} \times^{\GL_n} \gl_n^\ast[4]$ is equipped with a particular Hamiltonian structure. (Here, $\GL_n$ sits diagonally inside $\GL_{2n}$.) Such an equivalence will continue to hold over $\Z[1/N]$ for some $N \gg 0$ (in fact, one can take $N=1$), so we may consider the decompleted Frobenius for $p\nmid N$. In particular, we will assume $p>2$. The moment map $\ld{M} \to \gl_{2n}^\ast$ is induced by the inclusion $\gl_n^\ast \to \gl_{2n}^\ast$ sending 
    $$\mu: x \mapsto \begin{psmallmatrix}
        0 & \id_n \\
        x & 0
    \end{psmallmatrix}.$$
    Unwinding the proof of \cite{quat-satake} shows that the decompleted Frobenius/Steenrod algebra acts on $\ld{M}$ via the map 
    $$\varphi: x \mapsto x - 2 t^{p-1} x^{(p+1)/2} + t^{2(p-1)} x^p = \prod_{j\in \FF_p} (x - j^2 t^2 \id_n)$$
    on $\gl_n^\ast$. (Observe that the formula for $\varphi$ is a matrix version of the total Steenrod operation on $\H^\ast_{\SU(2)}(\ast; \FF_p)$.) If $x \in \gl_n^\ast$, it is \textit{not} true that $\varphi(\mu(x)) = \mu(\varphi(x))$; but these two elements of $\gl_{2n}^\ast$ are conjugate, from which it follows that the moment map $\ld{M}/\GL_{2n} \to \gl_{2n}^\ast/\GL_{2n}$ is equivariant for the action of the decompleted Frobenius. 

    There is also a multiplicative version of this picture. Namely, using the methods of \cite[Theorem 3.6.4]{ku-rel-langlands}, one obtains a K-theoretic version of the quaternionic Satake equivalence:
    $$\Loc_{\GL_{2n}\pw{t}}^c(\GL_{2n}\ls{t}/\Sp_{2n}\ls{t}; \KU) \simeq \Perf(\ld{M}_\KU^\reg/\GL_{2n}).$$
    Here, $\ld{M}_\KU^\reg$ is an open subset inside $\ld{M}_\KU \cong \GL_{2n} \times^{\GL_n} \gl_n$ (with $\GL_n$ sitting diagonally inside $\GL_{2n}$). The scheme $\ld{M}_\KU$ is equipped with a particular quasi-Hamiltonian structure, which we will now describe. The multiplicative moment map $\mu: \ld{M}_\KU \to \GL_{2n}$ is induced by the inclusion $\gl_n \to \GL_{2n}$ sending 
    $$\mu: x \mapsto \begin{psmallmatrix}
        x + \id_n & \id_n \\
        x & \id_n
    \end{psmallmatrix}.$$
    %or \begin{psmallmatrix}
        %x - \id_n & \id_n \\
        %x - 2\id_n & \id_n
    %\end{psmallmatrix}, or \begin{psmallmatrix}
        %x & -\id_n \\
        %\id_n & 0
    %\end{psmallmatrix}$.
    In this case, the decompleted Frobenius/$p$th Adams operation acts on $\ld{M}_\KU$ via the map $\varphi(x) = L_p(x)$ on $\gl_n$, with $L_p(x)$ as in \cref{ex: adams operations langlands}. For $x \in \gl_n$, the elements $\varphi(\mu(x))$ and $\mu(\varphi(x))$ of $\GL_{2n}$ are not equal. However, they are conjugate, which implies that the moment map $\ld{M}_\KU/\GL_{2n} \to \GL_{2n}/\GL_{2n}$ is equivariant for the action of the decompleted Frobenius.
    
    %\footnote
    {In fact, there is also an equivalence between $\Loc_{\GL_{2n}\pw{t}}^\gr(\GL_{2n}\ls{t}/\Sp_{2n}\ls{t}; \KO)$ and $\Perf(\ld{M}_\KU^\reg/\GL_{2n} \times B\Z/2)$. In other words, complex conjugation on $\KU$ acts trivially on $\ld{M}_\KU$. This is easy to see algebraically: using \cref{def: cplx conj on B mod B} and \cref{rmk: cplx conj on G equiv KU}, this follows from the observation that if $x\in \gl_n$, then $\mu(x)^{-1}$ is conjugate to $\mu(x)$. However, the triviality of complex conjugation in this case also has a topological explanation. Namely, the quotient $\GL_{2n}\pw{t} \backslash \GL_{2n}\ls{t}/\Sp_{2n}\ls{t}$ is homotopy equivalent to the quaternionic affine Grassmannian $\Gr_{\GL_n(\bH)}$. The map $\HHP^{n-1} \to \Gr_{\GL_n(\bH)}$ exhibits $\HHP^{n-1}$ as a generating complex for $\Gr_{\GL_n(\bH)}$. Since $\HHP^{n-1}$ is a $\Spin$-manifold, it is $\KO$-oriented \cite{atiyah-bott-shapiro}, which implies that complex conjugation on $\KU$ acts trivially on $\KU^\ast(\HHP^{n-1})$ (and hence on $\Loc_{\GL_{2n}\pw{t}}^\gr(\GL_{2n}\ls{t}/\Sp_{2n}\ls{t}; \KU)$).}
\end{example}
\begin{remark}\label{rmk: ku symplectic}
    There is a $\ku$-theoretic variant of \cref{ex: symplectic period}. Just as in \cite{ku-rel-langlands}, there is an equivalence
    $$\Loc_{\GL_{2n}\pw{t}}^\gr(\GL_{2n}\ls{t}/\Sp_{2n}\ls{t}; \ku) \simeq \Perf(\ld{M}_\beta^\reg/\GL_{2n});$$
    in fact, in the notation of \cite{ku-rel-langlands}, there is an equivalence
    $$\Shv_{\GL_{2n}\pw{t}}^{c,\Sat}(\GL_{2n}\ls{t}/\Sp_{2n}\ls{t}; \ku)^\faux \simeq \Perf(\ld{M}_\beta/\GL_{2n});$$
    Here, $\ld{M}_\beta^\reg$ is an open subset inside $\ld{M}_\beta \cong \GL_{2n} \times^{\GL_n} \gl_n$ (with $\GL_n$ sitting diagonally inside $\GL_{2n}$).
    %\footnote{Let us be more specific about the grading on $\ld{M}_\beta$: the space $\gl_n$ is equipped with the grading given by the character $4-\rho_{\GL_n}$, so that the diagonal entries have weight $-4$, and the $k$th subdiagonal (resp. superdiagonal) has weight $-4-k$ (resp. $k-4$).}
    In fact, there is also an equivalence between $\Loc_{\GL_{2n}\pw{t}}^\gr(\GL_{2n}\ls{t}/\Sp_{2n}\ls{t}; \ko)$ and $\Perf(\gl_n^\reg/\GL_{2n} \times \spev(\ko))$, where we are using notation as in \cref{rmk: connective ko def}.
    
    The scheme $\ld{M}_\beta$ is equipped with a particular $\ku$-Hamiltonian structure (in the sense of \cite{ku-rel-langlands}), which we will now describe. The multiplicative moment map $\mu: \ld{M}_\beta \to \GL_{2n,\beta}$ is induced by the inclusion $\gl_n \to \GL_{2n,\beta}$ sending 
    $$\mu: x \mapsto \begin{psmallmatrix}
        \id_n + \beta^2 x & \beta \id_n \\
        \beta x & \id_n
    \end{psmallmatrix}.$$
    Observe that when $\beta$ is inverted (or, more informally, set to $1$), this is the quasi-Hamiltonian moment map from \cref{ex: symplectic period}; similarly, $\left.\tfrac{\mu(x) - \id_{2n}}{\beta}\right|_{\beta=0}$ is well-defined, and reduces to the moment map $x \mapsto \begin{psmallmatrix}
        0 & \id_n\\
        x & 0
    \end{psmallmatrix}$ from the ordinary symplectic period of \cref{ex: symplectic period}. In this case, the decompleted Frobenius/$p$th Adams operation acts on $\ld{M}_\beta$ via the map $\varphi(x) = f_p(x)$ on $\gl_n$, with $f_p(x)$ as in \cref{rmk: adams connective ku}. For $x \in \gl_n$, the elements $\varphi(\mu(x))$ and $\mu(\varphi(x))$ of $\GL_{2n}$ are conjugate, so the moment map $\ld{M}_\beta/\GL_{2n} \to \GL_{2n,\beta}/\GL_{2n}$ is equivariant for the action of the decompleted Frobenius. 
\end{remark}
In the language of \cite{bzsv, ku-rel-langlands}, the preceding discussion says that the stack which is relative Langlands dual to the Hamiltonian $\GL_{2n}$-space $T^\ast(\GL_{2n}/\Sp_{2n})$ is isomorphic to $\gl_n(4)/\GL_n$ with coefficients in both ordinary cohomology \textit{and} complex/real K-theory. However, this will no longer be true for elliptic cohomology. Geometrically, this is because elliptic cohomology is not $\Spin$-oriented, but is only ``$\String$-oriented'' \cite{koandtmf}; and $\HHP^{n-1}$ is a generating complex for the quaternionic affine Grassmannian $\Gr_{\GL_n(\bH)}$, but it is not a $\String$-manifold.\footnote{If $\TMF$ denotes the universal elliptic cohomology theory \cite{tmf}, then the $\TMF$-homology of $\HHP^n$ is described explicitly in \cite[Proposition 7.5]{meier-tmf-modules}.}

There are some examples where the generating complex for the real Grassmannian $\Gr_{G,\RR}$ \textit{is} orientable for elliptic cohomology, such as the case of (the simply-connected form of) $E_6$ equipped with the involution whose fixed subgroup is $F_4$. (This is the Cartan symmetric space EIV, and in the parlance of relative Langlands duality, it corresponds to the ``octonionic'' Satake equivalence of \cite{octonionic-period} and \cite[Remark 3.6.5]{ku-rel-langlands}.) In this case, the generating complex for $\Gr_{E_6,\RR}$ is given by the octonionic projective plane $\OP^2$, which is indeed a $\String$-manifold (and hence is orientable for elliptic cohomology). It is possible to use this observation to compute the relative Langlands dual to the Hamiltonian $E_6$-space $T^\ast(E_6/F_4)$ with coefficients in elliptic cohomology, but the calculations become very intricate (so we will leave it to future work).

Finally, let us discuss the Frobenius for a non-polarized example. The most famous example of this is the Gan-Gross-Prasad period, which, in the parlance of \cite{bzsv}, is concerned with the relative Langlands dual to the homogeneous spherical $G = \SO_{2n-1} \times \SO_{2n}$-variety given by $G/\SO_{2n-1}^\mathrm{diag}$. This dual is given by the Hamiltonian $\ld{G} = \Sp_{2n-2} \times \SO_{2n}$-space $\std_{2n-2} \otimes \std_{2n}$. It was studied geometrically in \cite{orthosymplectic-satake}. The following is one of the simplest nontrivial cases of the Gan-Gross-Prasad period:
\begin{example}[Triple product period]
    The triple product period, studied geometrically in \cite{pgl2-cubes}, is concerned with the relative Langlands dual to $G = \PGL_2^{\times 3}$ acting on $X = G/\PGL_2^\mathrm{diag}$. (This can be regarded as a special case of the Gan-Gross-Prasad period, because $\PGL_2 \cong \SO_3$ and $\PGL_2^{\times 2} \cong \PSO_4$.) The dual Hamiltonian $\ld{G} = \SL_2^{\times 3}$-variety in this case is given by the $8$-dimensional symplectic vector space $\std^{\otimes 3}$, with each factor of $\SL_2$ in $\ld{G}$ acting on the corresponding tensor factor. Let us quickly summarize a few facts from \cite{pgl2-cubes}: if we identify $\sl_2^\ast \cong \pgl_2$ with the space $\Sym^2(\std)$ of binary quadratic forms, the moment map $\mu: \std^{\otimes 3} \to \Sym^2(\std)^{\times 3}$ is given precisely by Bhargava's construction \cite{bhargava-composition-i} of three quadratic forms from a $2 \times 2 \times 2$-cube $\cC$; furthermore, all three of these quadratic forms have the same discriminant $\det(\cC)$, called the \textit{hyperdeterminant} of the cube \cite{cayley-original, gelfand-hyperdet}.
    
    This information can be used to compute the action of the Frobenius on $\std^{\otimes 3}/\SL_2^{\times 3}$, at least for $p>2$. We will only describe the \textit{completed} Frobenius, i.e., the functor
    $$\Shv^\gr_{G\pw{t}}(X\ls{t}; k) \otimes_{\pi_0(k)} F \to \Shv^\gr_{G\pw{t}}(X\ls{t}; k^{t\Cp}) \otimes_{\pi_0(k)} F,$$
    which, when $k = \Z[u^{\pm 1}]$, identifies with the functor given by pullback along a map
    $$\varphi: \std^{\otimes 3}/\SL_2^{\times 3} \times_{\spec(F)} \spec(F\ls{t}) \to \std^{\otimes 3}/\SL_2^{\times 3}.$$
    To describe it, pick a basis $e_1,e_2\in \std$, and equip $\std$ with the $\Z[1/3]$-grading where $e_1$ has weight $2/3$ and $e_2$ has weight $-1/3$. This equips $\std^{\otimes 3}$ with an $\Z$-grading, and one can then show that the Frobenius map is given by scaling the cube by its natural $\GG_m$-action with respect to the scalar $\delta = 1-t^{p-1} \det(\cC)^{(p-1)/2}$. In other words, it is given by  multiplying each coordinate $\cC_{ijk}$ of a cube $\cC$ by $\delta^{|\cC_{ijk}|}$, where $|\cC_{ijk}|$ is the weight of $\cC_{ijk}$.
    Note that some coordinates will have negative weight, and in this case one must interpret $\delta^{-1}$ as $\sum_{n\geq 0} t^{n(p-1)} \det(\cC)^{n(p-1)/2}$; ensuring convergence of this power series is why we elected to work with the completed Frobenius in the present example.
\end{example}
It might be interesting to describe the (de)completed Frobenius explicitly for the general case of the Gan-Gross-Prasad period, as well as for other non-polarized examples.
%ggp period?