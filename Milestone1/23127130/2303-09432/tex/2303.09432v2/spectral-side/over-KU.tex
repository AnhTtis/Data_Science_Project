Our goal in this section is to prove an analogue of \cref{cor: reg locus ordinary ABG}, albeit with coefficients in $k = \KU$. Note that in this case, $\cM_{T,0} \cong T$. To do so, we need an analogue of \cref{def: additive kostant slice} and constructions surrounding it. Recall that the group $G$ (over $\cc$) is connected, almost simple, and simply-laced. We will also fix an algebraically closed field $F$,% of characteristic zero,
over which the Langlands dual group $\ld{G}$ will live. When dealing with the algebraic geometry (as opposed to the topology) of $G$, we will also view it as living over $F$; since $G$ is simply-laced, it is isogenous to $\ld{G}$.
\begin{definition}\label{def: mult kostant slice}
    Let $G^\mathrm{sc}$ denote the simply-connected cover of $G$, and let $f\in G^\mathrm{sc}$ be a principal nilpotent element as defined in \cite[Theorem 4.6]{steinberg-slice}. We will denote its image under the map $G^\mathrm{sc} \to G$ also by $f$. Then the map $\GG_a \to G$ corresponding to $f$ factors through the map $\GG_a = B \to \SL_2$; we will denote the image of the standard generator $\begin{psmallmatrix}
    1 & 0\\
    1 & 1
    \end{psmallmatrix}$ under the map $\SL_2 \to G$ by $e\in G$. Let $Z_G(e)^\circ$ be the connected component of the identity in the centralizer of $e$ in $G$. Define the \textit{multiplicative Kostant slice} $\cS_\mu$ by $f\cdot Z_G(e)^\circ \subseteq G$. Since $G$ is assumed to be simply-connected, the composite 
    $$\cS_\mu \to G \to G\mmod G \cong G\mmod \ld{G} \cong T\mmod W$$
    is an isomorphism. We will often denote the inclusion of the Kostant slice by $\kappa: T\mmod W \to G$.

    The \textit{multiplicative Grothendieck-Springer resolution} $\tilde{\ld{G}}$ is defined as
    $$\tilde{\ld{G}} = B \times^{\ld{B}} \ld{G},$$
    where $\ld{B}$ acts on $B$ by conjugation. (This makes sense thanks to the assumption that $G$ is simply-laced.) There is a natural map $\tilde{\ld{G}} \to G$, given by the conjugation action of $\ld{G}$ on $B$.
    Let $\tilde{\cS}_\mu$ denote the fiber product $\tilde{\cS}_\mu \times_G \tilde{\ld{G}}$, so that the composite 
    $$\tilde{\cS}_\mu \to \tilde{\ld{G}} \to T$$
    is an isomorphism; we will denote the inclusion of $\tilde{\cS}_\mu$ as a map $\kappa: \tilde{\cS}_\mu \cong T \to \tilde{\ld{G}}$.

    As with the additive Kostant slice, we will only care about the composite $T \to \tilde{\ld{G}} \to \tilde{\ld{G}}/\ld{G}$ below, so we will also denote it by $\kappa$. If we identify $\tilde{\ld{G}}/\ld{G} \cong B/\ld{B}$, then the map $\kappa$ admits a simple description: it is the composite $f\cdot T \to B \to B/\ld{B}$. %Just as in \cite[Proposition 19]{kostant-lie-group-reps}, there is a unique map $\mu: T\cdot f \to N$ such that $\Ad_{\mu(x)}(x) \in Z_G(e)^\circ\cdot f$, and the image of any $x\in T$ under the map $T \to T\mmod W \xar{\kappa} G$ can be identified with $\Ad_{\mu(xf)}(xf)$.
\end{definition}
\begin{definition}
    The stabilizer (inside $\ld{G}$) of the multiplicative Kostant slice $\cS_\mu \subseteq G^\reg$ is a closed subgroup scheme of the constant group scheme $\ld{G} \times \cS_\mu$, and will be denoted by $\ld{J}_\mu$. It will be called the \textit{multiplicative regular centralizer group scheme}; if we wish to emphasize the dependence on $G$, we will denote it by $\ld{J}_\mu(G)$.  Note that since the composite $\cS_\mu \to G^\reg \to G\mmod \ld{G}$ is an isomorphism, we may identify
    $$\ld{J}_\mu \cong \cS_\mu \times_{G/\ld{G}} \cS_\mu.$$
    Similarly, the stabilizer (inside $\ld{G}$) of the multiplicative Kostant slice $\tilde{\cS}_\mu \subseteq \tilde{\ld{G}}^\reg$ is a closed subgroup scheme of the constant group scheme $\ld{G} \times \tilde{\cS}_\mu$, and will be denoted by $\tilde{\ld{J}}_\mu$. Since $\tilde{\cS}_\mu \cong \cS_\mu \times_{G} \tilde{\ld{G}}$, we may identify
    $$\tilde{\ld{J}}_\mu \cong \ld{J}_\mu \times_{\cS_\mu} \tilde{\cS}_\mu \cong (f \cdot T) \times_{B/\ld{B}} (f\cdot T).$$
\end{definition}
The following calculation also appears in \cite{bfm}, albeit using different techniques.
\begin{theorem}\label{thm: ku hmlgy reg centr}
    There is an isomorphism of group schemes over $f \cdot T \cong T \cong \cM_{T,0}$:
    $$\spec (\pi_0 \cf_T(\Gr_G)^\vee \otimes_\Z F) \cong (f \cdot T) \times_{B/\ld{B}} (f\cdot T).$$
\end{theorem}
Just as in \cref{thm: ordinary hmlgy reg centr}, the proof of \cref{thm: ku hmlgy reg centr} will rely on two lemmas.
\begin{lemma}\label{lem: ku borel is flat}
    The projection map $\tilde{\ld{J}}_\mu \to f\cdot T$ (onto either factor) is flat.
\end{lemma}
\begin{proof}
    Like in the proof of \cref{lem: kappa for borel is flat}, it suffices, by miracle flatness, to show that the fibers of the map $\tilde{\ld{J}}_\mu \to f\cdot T$ have dimension exactly $\rank(\ld{G})$. The fiber of this map over $f \cdot x\in f \cdot T$ is the scheme 
    $$Y = \{(g,y) \in \ld{B} \times T | \Ad_g(fy) = fx\}.$$
    Observe that the image of $\Ad_g(fy)$ and $fx$ (viewed as elements of $B$) under the map $B \to T$ are $y$ and $x$; so $y=x$ in $T$, which means that $Y$ is isomorphic to the centralizer $Z_{\ld{B}}(fx)$. The dimension estimate is equivalent to the claim that $fx$ is a regular element of $G$, since this means that its centralizer has minimal dimension (namely, the rank of $G$, which is also the rank of $\ld{G}$). The desired regularity of $fx$ follows from the discussion in \cite[Remark 4.7]{steinberg-slice}. (Note that, as mentioned in \textit{loc. cit.}, the specific choice of the regular unipotent $f$ is crucial for the regularity of $fx$.)
\end{proof}
\begin{notation}
    Let $\alpha$ be a root of $\ld{G}$. Say that a point $x \in T$ is \textit{$\alpha$-generic} if $x(h_\beta) \neq 1$ for all roots $\beta\neq \alpha$. This implies that the centralizer $Z_{\ld{G}}(x)$ has semisimple rank at most $1$. Let $T_{\alpha\dreg}$ denote the $\alpha$-regular locus. Observe that $T_\reg = \bigcup_{\alpha\in \Phi} T_{\alpha\dreg} \subseteq T$ is open, with complement of codimension $2$.
\end{notation}
The proof of the next result is exactly as in \cref{lem: localization of ordinary J}.
\begin{lemma}\label{lem: ku localization of ordinary J}
    There is an isomorphism
    \begin{equation}\label{eq: ku J of centralizer}
        \tilde{\ld{J}}_\mu(\ld{G})|_{T_{\alpha\dreg}} \xar{\sim} \tilde{\ld{J}}_\mu(Z_{\ld{G}}(x)^\circ)|_{T_{\alpha\dreg}},
    \end{equation}
    where $Z_{\ld{G}}(x)$ is the centralizer of some $x\in T_{\alpha\dreg}$ which lies on the $\alpha$-hyperplane, and $Z_{\ld{G}}(x)^\circ$ denotes the connected component of the identity. 
\end{lemma}
\begin{proof}[Proof of \cref{thm: ku hmlgy reg centr}]
    The argument of \cref{thm: ordinary hmlgy reg centr} reduces us to checking that the isomorphism of \cref{thm: ku hmlgy reg centr} holds if $G$ has semisimple rank $1$, i.e., is the product of a torus with one of $\GL_2$, $\SL_2$, or $\PGL_2$. Again, it is easy to match up the contributions from the toral factors, so we will assume that $G$ is either $\GL_2$, $\SL_2$, or $\PGL_2$. In this case, we can even replace $F$ by $\Z$. 
    \begin{itemize}
        \item When $G = \GL_2$, we may identify $\tilde{\ld{J}}_\mu$ with the centralizer (in $\ld{B}$) of $\begin{psmallmatrix}
            x & 0\\
            x & y
        \end{psmallmatrix}$. It is easy to compute that $\begin{psmallmatrix}
            a & 0 \\
            c & d
        \end{psmallmatrix}$ stabilizes $\begin{psmallmatrix}
            x & 0\\
            x & y
        \end{psmallmatrix}$ if and only if $c = \frac{a-d}{x-y} \cdot x$, meaning that 
        $$\tilde{\ld{J}}_\mu \cong \spec \Z[x^{\pm 1}, y^{\pm 1}, a^{\pm 1}, d^{\pm 1}, \tfrac{a-d}{x-y}].$$
        The coproduct sends $a\mapsto a \otimes a$ and $d\mapsto d \otimes d$.
        The same argument as in \cref{thm: ordinary hmlgy reg centr} implies that
        $$\KU^{T_c}_\ast(\Omega \GL_2) \cong \Z[u^{\pm 1}, x^{\pm 1}, y^{\pm 1}, a^{\pm 1}, d^{\pm 1}, \tfrac{a-d}{x-y}].$$
        The map induced on $T$-equivariant $\KU$-homology by the inclusion $T^2 \to \GL_2$ is simply given by the inclusion of the subalgebra $\Z[u^{\pm 1}, x^{\pm 1}, y^{\pm 1}, a^{\pm 1}, d^{\pm 1}]$. The coproduct on this subalgebra (and hence, on $\KU^{T_c}_\ast(\Omega \GL_2)$) is determined by the formulas $a\mapsto a \otimes a$ and $d \mapsto d \otimes d$. It follows that $\spec \KU^{T_c}_0(\Omega \GL_2)$ is isomorphic to $\tilde{\ld{J}}_\mu$ as group schemes over $\spec \pi_0 \KU_{T_c} \cong \spec \Z[x^{\pm 1}, y^{\pm 1}]$, as desired.
        \item When $G = \SL_2$, we may identify $\tilde{\ld{J}}_\mu$ with the centralizer (in $\ld{B} \subseteq \PGL_2$) of $\begin{psmallmatrix}
            x & 0\\
            x & x^{-1}
        \end{psmallmatrix}$. An element $\begin{psmallmatrix}
            a & 0 \\
            c & 1
        \end{psmallmatrix} \in \ld{B} \subseteq \PGL_2$ stabilizes $\begin{psmallmatrix}
            x & 0\\
            x & x^{-1}
        \end{psmallmatrix}$ if and only if $c = \frac{a-1}{x-x^{-1}} \cdot x$. Therefore, 
        $$\tilde{\ld{J}}_\mu \cong \spec \Z[x^{\pm 1}, a^{\pm 1}, \tfrac{a-1}{x-x^{-1}}];$$
        the coproduct sends $a\mapsto a \otimes a$.
        
        Next, there is an isomorphism
        $$\KU^{S^1}_\ast(\Omega \SL_2) \cong \Z[u^{\pm 1}, x^{\pm 1}, a^{\pm 1}, \tfrac{a-1}{x^2-1}].$$
        This is proved exactly as in \cref{thm: ordinary hmlgy reg centr}; the role of the class $2x$ is now played by the Chern class $x^2 - 1 \in \pi_0 \KU_{S^1}$ of the weight $2$ representation of $S^1$. (Recall that the action of $S^1$ on $G_c \cong \SU(2) \cong S^3$ exhibits it as the one-point compactification of the trivial $1$-dimensional representation summed with the weight $2$ representation of $S^1$ on $\cc$.)
        The map induced on $T$-equivariant $\KU$-homology by the inclusion $S^1 \to \SU(2)$ of the maximal torus is simply given by the inclusion of the subalgebra $\Z[u^{\pm 1}, x^{\pm 1}, a^{\pm 1}]$. The coproduct on this subalgebra (and hence, on $\KU^{S^1}_\ast(\Omega \SL_2)$) is determined by the formula $a\mapsto a \otimes a$. It follows that $\spec \KU^{S^1}_0(\Omega \SL_2)$ is isomorphic to $\tilde{\ld{J}}_\mu$ as group schemes over $\spec \pi_0 \KU_{S^1} \cong \spec \Z[x^{\pm 1}]$, as desired.
        \item When $G = \PGL_2$, we may identify $\tilde{\ld{J}}_\mu$ with the centralizer (in $\ld{B} \subseteq \SL_2$) of $\begin{psmallmatrix}
            x & 0\\
            x & 1
        \end{psmallmatrix}$. An element $\begin{psmallmatrix}
            a & 0 \\
            c & a^{-1}
        \end{psmallmatrix} \in \ld{B} \subseteq \SL_2$ stabilizes $\begin{psmallmatrix}
            x & 0\\
            x & 1
        \end{psmallmatrix}$ if and only if $c = \frac{a-a^{-1}}{x-1}\cdot x$. Therefore, 
        $$\tilde{\ld{J}}_\mu \cong \spec \Z[x^{\pm 1}, a^{\pm 1}, \tfrac{a-a^{-1}}{x-1}];$$
        the coproduct sends $a\mapsto a \otimes a$.
        Again, as in the preceding cases, there is an isomorphism
        $$\KU^{S^1}_\ast(\Omega \PGL_2) \cong \Z[u^{\pm 1}, x^{\pm 1}, a^{\pm 1}, \tfrac{a-a^{-1}}{x-1}],$$
        where the coproduct sends $a\mapsto a \otimes a$. It follows that $\spec \KU^{S^1}_0(\Omega \PGL_2)$ is isomorphic to $\tilde{\ld{J}}_\mu$ as group schemes over $\spec \pi_0 \KU_{S^1} \cong \spec \Z[x^{\pm 1}]$, as desired.\qedhere
    \end{itemize}
\end{proof}

\begin{remark}\label{rmk: mult GS alternative}
    Just for posterity, let us record a more canonical variant of the calculation above for $\ld{G} = \SL_2$, which does not require picking a Borel subgroup (i.e., which does not involve identifying $\tilde{\ld{G}}/\ld{G} \cong B/\ld{B}$). If $\lambda\in \GG_m$, we denote $\lambda + \lambda^{-1}\in \AA^1$ by $f(\lambda)$. The Kostant slice $\kappa:\ld{T} \cong \GG_m \to \tilde{\SL}_2$ is the map sending $\lambda \in \GG_m$ to the pair $(x, \ell)$ with
    $$x = \begin{pmatrix}
    f(\lambda)-1 & f(\lambda)-2 \\
    1 & 1
    \end{pmatrix}, \ \ell = \left[\lambda-1: 1\right].$$
    Note that this indeed a well-defined point in $\tilde{\SL}_2$, since one can check that $x$ preserves $\ell$: the key point is the conic relation
    $$2\lambda = f(\lambda)-\sqrt{f(\lambda)^2-4}.$$
    Indeed, this calculation of $\kappa(\lambda)$ is essentially immediate from the requirement that the following diagram commutes:
    $$\xymatrix{
    \GG_m \cong \ld{T} \ar[r]^-\kappa \ar[d]_-{\lambda \mapsto f(\lambda)} & \tilde{\SL}_2 \ar[d]\\
    \AA^1 \cong \ld{T}\mmod W \ar[r]^-\kappa_-{\lambda\mapsto \begin{psmallmatrix}
    \lambda-1 & \lambda-2 \\
    1 & 1
    \end{psmallmatrix}} & \SL_2.
    }$$
    Moreover, the $\SL_2$-action on $\tilde{\SL}_2$ sends $g\in \SL_2$ and $(x,\ell)$ to $(\Ad_g(x), g\ell)$. If $g = \begin{psmallmatrix}
    a & b \\
    c & d
    \end{psmallmatrix}$, we directly compute that $\Ad_g(x) = x$ if and only if $b = c(f(\lambda) - 2)$ and $a-d = (f(\lambda) - 2)c$, in which case $g$ also preserves $\ell$. Therefore, $g = \begin{psmallmatrix}
    (f(\lambda) - 2)c + d & (f(\lambda)-2)c \\
    c & d
    \end{psmallmatrix}$ for $c,d\in k$. In order for $\det(g) = 1$, we need 
    $$d^2 + c(f(\lambda)-2)(d-c) = 1.$$
    Both $x$ and $g$ can be simultaneously diagonalized (if $f(\lambda) \neq \pm 2$); note that $\lambda+\lambda^{-1}$ is an eigenvalue of $x$. If $t$ is an eigenvalue of $g$, then we have $c = \tfrac{t-t^{-1}}{\lambda - \lambda^{-1}}$ and $d = \tfrac{t^2\lambda + 1}{t(\lambda+1)}$.
    When $k$ is not of characteristic $2$, this shows that 
    $$\GG_m \times_{\tilde{\SL}_2/\SL_2} \GG_m \cong k[\lambda^{\pm 1}, t^{\pm 1}, \tfrac{t-t^{-1}}{\lambda - \lambda^{-1}}].$$
    This in turn implies that
    $$\GG_m \times_{\tilde{\SL}_2/\PGL_2} \GG_m \cong k[\lambda^{\pm 1}, t^{\pm 2}, \tfrac{t^2-1}{\lambda - \lambda^{-1}}],$$
    as desired.
\end{remark}
There is \textit{another} choice of slice when $G$ is simply-connected; the calculation of \cref{thm: ku hmlgy reg centr} continues to hold for it, too, as we now illustrate in the example of $\SL_2$.
\begin{definition}[Steinberg slice]
    Let $G$ be a simply-connected semisimple algebraic group. Given $w\in W$, let $N_w = N \cap w^{-1} N^- w$, so that $N_w = \prod_{\alpha\in \Phi_w} U_\alpha$, where $\Phi_w$ is the set of roots made negative by $w$. Let $w = \prod_{\alpha\in \Delta} s_\alpha\in W$ be a Coxeter element, and let $\dot{w}$ be a lift of $w$ to $N_G(T)$. Define the Steinberg slice $\Sigma = \dot{w} N_w\subseteq G$. Then \cite{steinberg-slice} proved/stated that the composite $\Sigma \to G \to G\mmod G \cong T\mmod W$ is an isomorphism. 
    Let $\tilde{\Sigma}$ denote the fiber product $\Sigma\times_G \tilde{\ld{G}}$, so that the composite $\tilde{\Sigma} \to \tilde{\ld{G}} \to T$ is an isomorphism. We will denote the inclusion of $\tilde{\Sigma}$ by $\sigma: T \to \tilde{\ld{G}}$.
\end{definition}
\begin{observe}
    We will illustrate the calculation of $T \times_{\tilde{\ld{G}}/\ld{G}} T$ (with $T$ mapping to $\tilde{\ld{G}}$ by $\sigma$) when ${G} = \SL_2$.
    View a point in $\tilde{\ld{G}}$ as a pair $(x\in \SL_2, \ell\subseteq \cc^2)$ such that $x$ preserves $\ell$. The Steinberg slice $\sigma:\GG_m \to \tilde{\SL}_2$ is the map sending $\lambda \in \GG_m$ to the pair $(x, \ell)$ with 
    $$x = \begin{pmatrix}
    \lambda + \lambda^{-1} & -1 \\
    1 & 0
    \end{pmatrix}, \ \ell = \left[\lambda: 1\right].$$
    Note that this indeed a well-defined point in $\tilde{\SL}_2$, since one can check that $x$ preserves $\ell$. This calculation of $\sigma(\lambda)$ is essentially immediate from the requirement that the following diagram commutes:
    $$\xymatrix{
    \GG_m \cong \ld{T} \ar[r]^-\sigma \ar[d]_-{\lambda \mapsto \lambda + \lambda^{-1}} & \tilde{\SL}_2 \ar[d]\\
    \AA^1 \cong \ld{T}\mmod W \ar[r]^-\sigma_-{\lambda\mapsto \begin{psmallmatrix}
    \lambda & -1 \\
    1 & 0
    \end{psmallmatrix}} & \SL_2.
    }$$
    Moreover, the $\SL_2$-action on $\tilde{\SL}_2$ sends $g\in \SL_2$ and $(x,\ell)$ to $(\Ad_g(x), g\ell)$. If $g = \begin{psmallmatrix}
    a & b \\
    c & d
    \end{psmallmatrix}$, one can directly compute that $g$ commutes with $\begin{psmallmatrix}
    \lambda + \lambda^{-1} & -1 \\
    1 & 0
    \end{psmallmatrix}$ if and only if $a = c(\lambda + \lambda^{-1}) + d$ and $b=-c$. Therefore, $g = \begin{psmallmatrix}
    c(\lambda + \lambda^{-1}) + d & -c \\
    c & d
    \end{psmallmatrix}$ for $c,d\in k$. In order for $\det(g) = 1$, we need 
    $$c^2 + d^2 + cd(\lambda + \lambda^{-1}) = 1.$$
    As long as $\lambda\neq \pm 1$, both $x$ and $g$ can be simultaneously diagonalized by $\begin{psmallmatrix}
    \lambda & \lambda^{-1} \\
    1 & 1
    \end{psmallmatrix}$: the diagonalization of $x$ is $\begin{psmallmatrix}
    \lambda & 0 \\
    0 & \lambda^{-1}
    \end{psmallmatrix}$, and the diagonalization of $g$ is $\begin{psmallmatrix}
    c\lambda + d & 0 \\
    0 & c\lambda^{-1} + d
    \end{psmallmatrix}$. If $t = c\lambda+d$, then $c\lambda^{-1}+d = t^{-1}$ by the above determinant relation. We also have that $a = t - \tfrac{\lambda(t-t^{-1})}{\lambda - \lambda^{-1}}$ and $c = \tfrac{t-t^{-1}}{\lambda - \lambda^{-1}}$. This shows that 
    $$\GG_m \times_{\tilde{\SL}_2/\SL_2} \GG_m \cong \spec k[\lambda^{\pm 1}, t^{\pm 1}, \tfrac{t-t^{-1}}{\lambda - \lambda^{-1}}],$$
    and hence that
    $$\GG_m \times_{\tilde{\SL}_2/\PGL_2} \GG_m \cong \spec k[\lambda^{\pm 1}, t^{\pm 2}, \tfrac{t^2-1}{\lambda - \lambda^{-1}}],$$
    as desired.
\end{observe}

\begin{corollary}\label{cor: ku reg locus ordinary ABG}
    There is an $F$-linear equivalence
    $$\Loc_{T_c}^\gr(\Gr_G; \KU) \otimes_\Z F \simeq \QCoh(\tilde{\ld{G}}^\reg/\ld{G}).$$
    Furthermore, the pushforward functor $\Loc_{T_c}^\gr(\Gr_G; \KU) \to \Loc_{T_c}^\gr(\ast; \KU)$ identifies with the pullback functor $\kappa^\ast: \QCoh(\tilde{\ld{G}}^\reg/\ld{G}) \to \QCoh(T)$.
\end{corollary}
\begin{proof}
    By definition, $\Loc_{T_c}^\gr(\Gr_G; \KU)$ is equivalent to the category of comodules over $\pi_0 \cf_T(\Gr_G)^\vee = \KU_0^T(\Gr_G)$ in the category of $\pi_0 \KU_{T_c}$-modules. By \cref{thm: ku hmlgy reg centr}, it can be identified the category of quasicoherent sheaves on the quotient stack $(f \cdot T)/\tilde{\ld{J}}_\mu$. We may view $\tilde{\ld{J}}_\mu$ as a closed subgroup scheme of the constant group scheme $\ld{B} \times (f \cdot T)$. This gives an isomorphism
    $$(f \cdot T)/\tilde{\ld{J}}_\mu \cong \ld{B} \backslash (\ld{B} \times (f \cdot T))/\tilde{\ld{J}}_\mu.$$
    It follows from Steinberg's work in \cite{steinberg-slice} that the $\ld{B}$-orbit of $f \cdot T$ inside $B$ is precisely the regular locus $B^\reg$. Since $\tilde{\ld{J}}_\mu$ is definitionally the stabilizer of $f \cdot T \subseteq B$, the quotient $(\ld{B} \times (f \cdot T))/\tilde{\ld{J}}_\mu$ is isomorphic to $B^\reg$; so there is an isomorphism $(f \cdot T)/\tilde{\ld{J}}_\mu \cong B^\reg/\ld{B}$.
    To finish, note that $\tilde{\ld{G}}^\reg/\ld{G} \cong B^\reg/\ld{B}$.
\end{proof}
Similarly, there is an $F$-linear equivalence
$$\Loc_{\ld{T}_c}^\gr(\Gr_G; \KU) \otimes_\Z F \simeq \QCoh(\tilde{\ld{G}}^{',\reg}/\ld{G}),$$
where $\tilde{\ld{G}}^{'}$ is $\ld{G} \times^{\ld{B}} \ld{B}$, with $\ld{B}$ acting on itself by conjugation. Note that $\tilde{\ld{G}}^{',\reg}/\ld{G} \cong \ld{B}^\reg/\ld{B}$ is an open substack of the stack $\ld{B}/\ld{B} \cong \Map(B\Z, B\ld{B})$ of $\ld{B}$-bundles on the circle $S^1 = B\Z$.

The equivalence of \cref{cor: ku reg locus ordinary ABG} is in fact symmetric monoidal for the convolution tensor structure on $\Loc_{T_c}^\gr(\Gr_G; \KU)$ (described in \cref{rmk: loc gr convolution tensor}) and the standard tensor product on $\QCoh(\tilde{\ld{G}}^\reg/\ld{G})$.

\begin{remark}\label{rmk: Loc G for KU}
    It can be shown that if $G$ has torsion-free fundamental group, there is an $F$-linear equivalence
    $$\Loc_{G_c}^\gr(\Gr_G; \KU) \otimes_\Z F \simeq \QCoh(G^\reg/\ld{G}).$$
    Just as in \cref{sec: degenerations}, the left-hand side is defined as
    $$\Loc_{G_c}^\gr(\Gr_G; \KU) = \coLMod_{\pi_0(\cf_G(\Gr_G)^\vee)}(\QCoh(T\mmod W)).$$
    Note that this is a sensible definition since $\pi_\ast \cf_G(\Gr_G)^\vee$ is concentrated in even degrees.
    Furthermore, the pushforward functor $\Loc_{G_c}^\gr(\Gr_G; \KU) \to \Loc_{G_c}^\gr(\ast; \KU)$ identifies with the pullback functor $\kappa^\ast: \QCoh(G^\reg/\ld{G}) \to \QCoh(T\mmod W)$. The proof of the displayed equivalence is quite similar to that of \cref{cor: ku reg locus ordinary ABG}, and in fact can be deduced from it using the observation that $\pi_0(\cf_G(\Gr_G)^\vee) = \pi_0(\cf_T(\Gr_G)^\vee)^W$ and that the natural map $\tilde{\ld{G}}^\reg \to G^\reg$ is a (ramified) $W$-cover. The first statement uses that $G$ has torsion-free fundamental group, and the second is a multiplicative version of Grothendieck-Springer theory.
\end{remark}
\begin{remark}
    In \cite[Section 3.7]{ku-rel-langlands}, we study a variant of \cref{cor: ku reg locus ordinary ABG}, where $\KU$ is replaced by \textit{connective} complex K-theory $\ku$; that is, $\Loc_{T_c}^\gr(\Gr_G; \KU)$ is replaced by $\Loc_{T_c}^\gr(\Gr_G; \ku)$. On the Langlands dual side, this has the effect of replacing $\tilde{\ld{G}}^\reg/\ld{G}$ by the $1$-parameter family over $\spec(\pi_\ast(\ku))/\GG_m \cong \AA^1/\GG_m$ whose generic fiber is $\tilde{\ld{G}}^\reg/\ld{G}$, and whose special fiber is $\tilde{\ld{\g}}^\reg/\ld{G}$.
\end{remark}
\begin{remark}\label{rmk: non-simply-laced ku}
    There is a variant of \cref{cor: ku reg locus ordinary ABG} if $G$ is not simply-laced, but it is more complicated to state. Let us just give the analogue of \cref{thm: ku hmlgy reg centr}. Suppose $G$ is not simply-laced, and let $T$ be a maximal torus of $G$; then $\ld{\g}$ is the fixed point subalgebra $\ld{\fr{h}}^\tau$ of an finite-order outer automorphism $\tau$ of a simply-laced Lie algebra $\ld{\fr{h}}$. Let $H$ denote the simply-connected simply-laced group corresponding to the Langlands dual $\fr{h}$, and let $T_H$ denote its maximal torus.  Then we may identify the fixed subset $\bX^\ast(T')^\tau$ with $\bX^\ast(T)$. If $n$ denotes the order of $\tau$, there is an action of $\Z/n$ on $T\pw{t}$, $G\pw{t}$, and $G\ls{t}$, given by $\tau$ on $T$ and $G$, and $t\mapsto \zeta_n \tau$ for a primitive $n$th root of unity $\zeta_n$. The appropriate replacement of $\pi_0 \cf_T(\Gr_G)^\vee$ in this case is $\pi_0 \cf_{T\pw{t}^{\Z/n}}(G_\ad\ls{t}^{\Z/n}/G_\ad\pw{t}^{\Z/n})^\vee$. The analogue of \cref{thm: ku hmlgy reg centr} (see \cite[Theorem 3.9]{finkelberg-tsymbaliuk}) states that this algebra is isomorphic to the stabilizer $\cS_\mu \times_{\ld{G}/\ld{G}} \cS_\mu$. 
\end{remark}
The map $\tilde{\ld{G}}^\reg/\ld{G} \to B\ld{G}$ defines a functor
\begin{equation}\label{eq: Rep G^ to KU loc}
    \Rep(\ld{G}) \to \QCoh(\tilde{\ld{G}}^\reg/\ld{G}) \simeq \Loc_{T_c}^\gr(\Gr_G; \KU) \otimes_\Z F.
\end{equation}
More generally, the map $\tilde{\ld{G}}^\reg/\ld{G} \to B\ld{G} \times B\ld{T}$ defines a functor
\begin{equation}\label{eq: Rep T^ x G^ to KU loc}
    \Rep(\ld{G} \times \ld{T}) \to \QCoh(\tilde{\ld{G}}^\reg/\ld{G}) \simeq \Loc_{T_c}^\gr(\Gr_G; \KU) \otimes_\Z F.
\end{equation}
If $V \in \Rep(\ld{G})$, let $\cS_\KU(V)$ denote the corresponding object of $\Loc_{T_c}^\gr(\Gr_G; \KU) \otimes_\Z F$. The same argument as in \cref{prop: ordinary realizing minuscule reps} shows the following, which says that $\cS_\KU(V) \in \Loc_{T_c}^\gr(\Gr_G; \KU)$ is the associated graded of a particular object $\cf_\lambda \in \Loc_{T_c}(\Gr_G; \KU)$ if $V$ is a minuscule $\ld{G}$-representation. 
\begin{prop}\label{prop: KU realizing minuscule reps}
    Let $\lambda_\bull = (\lambda_1, \cdots, \lambda_n)$ be a tuple of dominant minuscule weights of $\ld{G}$, let $|\lambda_\bull| = \sum_i \lambda_i$, and let $\ol{\Gr_G^{\lambda_\bull}}$ denote the corresponding \textit{convolution variety}. Let $\cf_{\lambda_\bull}$ denote the pushforward of the constant sheaf along the canonical map $q: \ol{\Gr_G^{\lambda_\bull}} \to \ol{\Gr_G^{|\lambda|}} \subseteq \Gr_G$. If $V_{\lambda_i}$ denotes the irreducible representation of $\ld{G}$ with highest weight $\lambda_i$, then there is an isomorphism $\cS_\KU(\bigotimes_i V_{\lambda_i}) \cong \cf_{\lambda_\bull}^\gr$.
\end{prop}
It would be very interesting to understand whether \cref{prop: KU realizing minuscule reps} can be extended to other non-minuscule irreducible representations. As in \cref{rmk: ordinary action on minuscule}, if $\lambda$ is a dominant minuscule weight of $\ld{G}$, then the coaction of $\pi_0 \cf_T(\Gr_G)^\vee$ on $\pi_0 \cf_T(G/P_\lambda)$ defines a homomorphism 
\begin{equation}
    \spec \pi_0 \cf_T(\Gr_G)^\vee \to \GL(\pi_0 \cf_T(G/P_\lambda))
\end{equation}
of group schemes over $T$, where $\GL(\pi_0 \cf_T(G/P_\lambda))$ denotes the group scheme of $\co_T$-linear automorphisms of the vector bundle $\pi_0 \cf_T(G/P_\lambda)$. Under the isomorphisms of \cref{thm: ku hmlgy reg centr} and \cref{prop: KU realizing minuscule reps}, this homomorphism factors as the composite
\begin{equation}\label{eq: KU factorization action on minuscule}
    \tilde{\ld{J}}_\mu \to \ld{G} \times T \to \GL(V_\lambda) \times T,
\end{equation}
where the second map describes the $\ld{G}$-action on $V_\lambda$.
Similar statements hold with $\tilde{\ld{J}}_\mu$ replaced by $\ld{J}_\mu$ and $\pi_0 \cf_T(G/P_\lambda)$ replaced by $\pi_0 \cf_G(G/P_\lambda) \cong \pi_0 \KU_{L_\lambda}$ (where $L_\lambda$ is the Levi quotient of $P_\lambda$).

\cref{thm: ku hmlgy reg centr} has several applications. Here is one, following the same proof as in \cref{prop: ordinary gelfand-graev}; it gives a \textit{multiplicative} version of the Gelfand-Graev action on the affine closure $\ol{T^\ast(\ld{G}/\ld{N})}$:
\begin{prop}[Multiplicative Gelfand-Graev action]\label{prop: ku gelfand-graev}
    The natural action of $\ld{G} \times \ld{T}$ on the affine closure $\ol{\ld{G} \times^{\ld{N}}B}$ extends to an action of $\ld{G} \times (W \rtimes \ld{T})$, where $W$ is the Weyl group.
\end{prop}
In the following, we will write $\ol{T^\ast_{\GG_m}(\ld{G}/\ld{N})}$ to denote the affine closure of the ``multiplicative'' cotangent bundle ${\ld{G} \times^{\ld{N}}B}$.
Unlike with \cref{prop: ordinary gelfand-graev}, \cref{prop: ku gelfand-graev} does require $G$ to be simply-laced; otherwise $\ol{T^\ast_{\GG_m}(\ld{G}/\ld{N})}$ would not even be well-defined. The moment map $\ol{T^\ast_{\GG_m}(\ld{G}/\ld{N})} \to G$ is $W$-equivariant for the trivial action on the target. There is a commutative diagram
$$\xymatrix{
\tilde{\ld{G}} \ar@{^(->}[r] \ar[dr] & \ol{T^\ast_{\GG_m}(\ld{G}/\ld{N})}/\ld{T} \ar[d] \\
& G
}$$
which relates $\ol{T^\ast_{\GG_m}(\ld{G}/\ld{N})}$ to the multiplicative Grothendieck-Springer resolution; and via this diagram, the multiplicative Gelfand-Graev action is closely related to the Weyl action in trigonometric/multiplicative Springer theory.
\begin{remark}
    The proof of \cref{prop: ku gelfand-graev} generalizes to show that if $\ld{P} \subseteq \ld{G}$ is a parabolic subgroup with Levi quotient $\ld{L}$ and unipotent radical $U_{\ld{P}}$, then the natural action of $\ld{G} \times \ld{L}$ on the affine closure $\ol{\ld{G} \times^{U_{\ld{P}}} P}$ extends to an action of $\ld{G} \times (W_L \rtimes \ld{L})$, where $W_L = N_{\ld{G}}(\ld{L})/\ld{L}$ is the Weyl group.
\end{remark}
\begin{example}\label{ex: Z/2 multiplicative symplectic fourier}
    Let us make the above action explicit in the example of $\ld{G} = \SL_2$ (so $W = \Z/2$). The group $B$ in this case is contained in $\PGL_2$, and can be chosen to be represented by matrices of the form $\begin{psmallmatrix}
        x & y \\
        0 & 1
    \end{psmallmatrix}$. The action of $\begin{psmallmatrix}
        1 & n \\
        0 & 1
    \end{psmallmatrix} \in \ld{N}$ on $\ld{G} \times B$ sends
    $$\begin{psmallmatrix}
        a & b \\
        c & d
    \end{psmallmatrix} \mapsto \begin{psmallmatrix}
        a & an + b \\
        c & cn + d
    \end{psmallmatrix}, \ \begin{psmallmatrix}
        x & y \\
        0 & 1
    \end{psmallmatrix} \mapsto \begin{psmallmatrix}
        x & y-n(x-1) \\
        0 & 1
    \end{psmallmatrix}.$$
    As explained in \cite[Remark 5.1.19]{ku-rel-langlands}, this means that the $\GG_a$-action fixes $a,c,x$, $B := ay + (x-1) b$, and $D = cy + (x-1) d$. There is a single relation between these classes, given by
    $$aD - cB = x-1.$$
    Let us relabel these variables so that $u = \begin{psmallmatrix}
        u_1 \\
        u_2
    \end{psmallmatrix} = \begin{psmallmatrix}
        a \\
        c
    \end{psmallmatrix}$ and $v = (v_1, v_2) = (D, -B)$. Since $x$ must be invertible, it follows that the affine closure $\ol{\SL_2 \times^{\GG_a} B}$ is given by the complement of the hypersurface $1 + \pdb{u,v}$ in $T^\ast(\AA^2)$. This is Van den Bergh's multiplicative quiver variety $\cB(U, V)$ from \cite{van-den-bergh-double-poisson}, specialized to the case when the vector spaces $U, V$ are $\AA^2, \AA^1$. An elementary analysis as in \cref{rmk: Z/2 symplectic fourier} shows that the $\Z/2$-action of \cref{prop: ku gelfand-graev} is given on $\ol{\SL_2 \times^{\GG_a} B} \subseteq T^\ast(\AA^2)$ by the formula
    $$\left(\begin{psmallmatrix}
        u_1\\
        u_2
    \end{psmallmatrix}, (v_1, v_2)\right) \mapsto \left(\tfrac{1}{1 + \pdb{u,v}}\begin{psmallmatrix}
        -v_2 \\
        v_1
    \end{psmallmatrix}, (u_2, -u_1)\right).$$
    In particular, it can be viewed as a multiplicative version of the symplectic Fourier transform.
\end{example}
\begin{remark}
    The multiplicative symplectic Fourier transform of \cref{ex: Z/2 multiplicative symplectic fourier} is related to another, more geometric, Fourier-type transform, as we now describe. Let $\ell$ be a (complex) line. Recall from \cite{beilinson-glue-perverse} that the ($1$-)category $\Perv(\ell)$ of perverse sheaves on $\ell$ with respect to the stratification by $0\in \ell$ and its complement is equivalent to the category of diagrams of the form
    \begin{equation}\label{eq: perverse on disk}   
    \begin{tikzcd}
	    X & Y
	\arrow["u"', shift right, from=1-1, to=1-2]
	\arrow["v"', shift right, from=1-2, to=1-1]
    \end{tikzcd}
    \end{equation}
    with $X$ and $Y$ being vector spaces, such that $\id_Y+uv$ (and therefore $\id_X + vu$) is invertible. This equivalence sends $\cf \in \Perv(\ell)$ to its spaces of nearby and vanishing cycles at $0\in \ell$ (and the maps $u,v$ arise via monodromy). The Fourier-Sato transform (see \cite[Definition 3.7.8]{kashiwara-schapira}) gives an equivalence $\Perv(\ell) \to \Perv(\ell^\ast)$, and one can check that it sends an object \cref{eq: perverse on disk} to the object
    $$\begin{tikzcd}
	    Y & X.
	\arrow["-v"', shift right, from=1-1, to=1-2]
	\arrow["u(\id+vu)^{-1}"', shift right, from=1-2, to=1-1]
    \end{tikzcd}$$
    \cref{ex: Z/2 multiplicative symplectic fourier} defines a morphism from $\ol{\SL_2 \times^{\GG_a} B}$ to the moduli of isomorphism classes of objects of $\Perv(\ell)$ (where $X = \AA^2$ and $Y = \AA^1$); this morphism intertwines the multiplicative symplectic Fourier transform with the Fourier-Sato transform.
\end{remark}
We also have the following analogue of \cref{prop: ordinary full faithful on gr loc}, whose proof is exactly the same (one only needs to note that $\tilde{\ld{G}}^\reg \hookrightarrow \tilde{\ld{G}}$ has complement of codimension $2$, and similarly for $G^\reg \hookrightarrow G$).
\begin{prop}\label{prop: KU full faithful on gr loc}
    Let $\Loc_{T_c}^\gr(\Gr_G; \KU)^\heart$ denote the heart of the $t$-structure on $\Loc_{T_c}^\gr(\Gr_G; \KU) = \coMod_{\pi_0(\cf_T(\Gr_G))^\vee}(\QCoh(T))$ coming from the standard (homological truncation) $t$-structure on $\QCoh(T)$. 
    Then, the composite functor
    $$\Loc_{T_c}^\gr(\Gr_G; \KU) \otimes_\Z F \simeq \QCoh(\tilde{\ld{G}}^\reg/\ld{G}) \to \QCoh(\ld{G}\backslash \ol{T^\ast_{\GG_m}(\ld{G}/\ld{N})}/\ld{T})$$
    is $t$-exact, and on hearts, it restricts to a fully faithful functor on the essential image of \cref{eq: Rep T^ x G^ to KU loc}. Furthermore, this functor is $W$-equivariant for the natural action of $W = \N_{G_c}(T_c)/T_c$ on the left-hand side and the Gelfand-Graev action of \cref{prop: ku gelfand-graev} on the right-hand side.

    Similarly, suppose $G$ has torsion-free fundamental group, and let $\Loc_{G_c}^\gr(\Gr_G; \KU)^\heart$ denote the heart of the $t$-structure on $\Loc_{G_c}^\gr(\Gr_G; \KU) = \coMod_{\pi_0(\cf_G(\Gr_G))^\vee}(\QCoh(T\mmod W))$ coming from the standard (homological truncation) $t$-structure on $\QCoh(T\mmod W)$. 
    %If $V \in \Rep(\ld{G})$, the object $\cS_\KU(V)$ lies in $\Loc_{G_c}^\gr(\Gr_G; \KU)^\heart \otimes_\Z F$, and there is an isomorphism
    %$$\Map_{\QCoh(G/\ld{G})^\heart}(V \otimes_F \co_G, W \otimes_F \co_G) \xrightarrow{\cong} \Map_{\Loc_{G_c}^\gr(\Gr_G; \KU)^\heart \otimes_\Z F}(\cS_\KU(V), \cS_\KU(W))$$
    %of $F$-vector spaces for any two representations $V,W \in \Rep(\ld{G})$.
    Then, the composite functor
    $$\Loc_{G_c}^\gr(\Gr_G; \KU) \otimes_\Z F \simeq \QCoh(G^\reg/\ld{G}) \to \QCoh(G/\ld{G})$$
    is $t$-exact, and on hearts, it restricts to a fully faithful functor on the essential image of the functor $\Rep(\ld{G}) \to \Loc_{G_c}^\gr(\Gr_G; \KU) \otimes_\Z F$ (analogous to \cref{eq: Rep G^ to KU loc}).
\end{prop}
\cref{prop: KU full faithful on gr loc} gives an analogue of \cite[Theorem 4]{bf-derived-satake}: namely, if $\QCoh_\free(G/\ld{G})$ denotes the essential image of the pullback functor $\Rep(\ld{G}) \to \QCoh(G/\ld{G})$, then there is a fully faithful embedding 
$$\QCoh_\free(G/\ld{G})^\heartsuit \hookrightarrow \Loc_{G_c}^\gr(\Gr_G; \KU)^\heartsuit \otimes_\Z F.$$
Similarly, if $\QCoh_\free(\ld{G}\backslash \ol{T^\ast_{\GG_m}(\ld{G}/\ld{N})}/\ld{T})$ denotes the essential image of the pullback functor $\Rep(\ld{G} \times \ld{T}) \to \QCoh(\ld{G}\backslash \ol{T^\ast_{\GG_m}(\ld{G}/\ld{N})}/\ld{T})$, then there is a fully faithful embedding 
$$\QCoh_\free(\ld{G}\backslash \ol{T^\ast_{\GG_m}(\ld{G}/\ld{N})}/\ld{T})^\heartsuit \hookrightarrow \Loc_{T_c}^\gr(\Gr_G; \KU)^\heartsuit \otimes_\Z F.$$
This implies the following result.
\begin{corollary}\label{cor: ku minuscule equivalence}
    Let $\QCoh_{\free}(G/\ld{G})^{\min,\heartsuit}$ denote the essential image of $\Rep_\min(\ld{G})$ under the pullback functor $\Rep(\ld{G})^\heartsuit \to \QCoh(G/\ld{G})^\heartsuit$ (so it is the entirety of $\QCoh(G/\ld{G})^\heartsuit$ if $F$ has characteristic zero and $\ld{G}$ is not of type $E_8$). Similarly, let $(\Loc_{G_c}^\gr(\Gr_G; \KU)^{\heartsuit} \otimes_\Z F)^\min$ denote the idempotent completion of the subcategory of $\Loc_{G_c}^\gr(\Gr_G; \KU)^\heartsuit \otimes_\Z F$ spanned by $\cf_{\lambda_\bull}^\gr$ ranging over sequences $\lambda_\bull$ of minuscule highest weights. Then there is an equivalence
    $$\QCoh_\free(G/\ld{G})^{\min,\heartsuit} \simeq (\Loc_{G_c}^\gr(\Gr_G; \KU)^{\heartsuit} \otimes_\Z F)^\min.$$
\end{corollary}
There is a similar equivalence 
$$(\Loc_{T_c}^\gr(\Gr_G; \KU)^{\heartsuit} \otimes_\Z F)^\min \simeq \QCoh_\free(\ld{G}\backslash \ol{T^\ast_{\GG_m}(\ld{G}/\ld{N})}/\ld{T})^{\min,\heartsuit},$$
where these categories are defined analogously by idempotent completion. 

Note, again, that the category $(\Loc_{G_c}^\gr(\Gr_G; \KU)^{\heartsuit} \otimes_\Z F)^\min$ is the heart of a degeneration, in the sense of \cref{sec: degenerations}, of the similarly-defined category $(\Loc_{G_c}(\Gr_G; \KU) \otimes_\KU F[u^{\pm 1}])^\min$. (In particular, \cref{cor: ku minuscule equivalence} gives an equivalence between the purely algebraically defined category $\QCoh_\free(G/\ld{G})^{\min,\heartsuit}$ and a degeneration of the purely topologically defined category $(\Loc_{G_c}(\Gr_G; \KU) \otimes_\KU F[u^{\pm 1}])^\min$.) If $\lambda_\bull$ and $\mu_\bull$ are two sequences of dominant minuscule weights of $\ld{G}$, there is an equivalence of $\KU$-modules
$$\Map_{(\Loc_{G_c}(\Gr_G; \KU) \otimes_\KU F[u^{\pm 1}])^\min}(\cf_{\lambda_\bull}, \cf_{\mu_\bull}) \simeq \cf_{G_c}(\ol{\Gr_G^{\lambda_\bull}} \times_{\Gr_G} \ol{\Gr_G^{\mu_\bull}}),$$
so that the category $(\Loc_{G_c}(\Gr_G; \KU) \otimes_\KU F[u^{\pm 1}])^\min$ compares to the category from \cite[Section 3.5]{cautis-kamnitzer}.

As with \cref{prop: ordinary full faithful on gr loc}, the existence of the $t$-structure on $\Loc_{T_c}^\gr(\Gr_G; \KU)$ from \cref{prop: KU full faithful on gr loc} may at first glance perhaps be a bit surprising, since $\KU$ is a \textit{$2$-periodic} $\Eoo$-ring. Again, this periodicity prohibits $\Loc_{T_c}(\Gr_G; \KU)$ itself from having a $t$-structure; but the $\infty$-category $\Loc_{T_c}^\gr(\Gr_G; k)$ itself has both a \textit{homological} shift operation and a (periodic) \textit{weight} shifting operation. The homological shift on $\Loc_{T_c}^\gr(\Gr_G; \KU)$ is no longer periodic, and it is therefore reasonable to equip this $\infty$-category with a $t$-structure.

We now turn to the question of the analogue of \cref{cor: ku reg locus ordinary ABG} if $\KU$ is replaced by \textit{real} K-theory $\KO$. (Recall the definition of $\Loc_{T_c}^\gr(\Gr_G; \KO)$ from \cref{def: KO graded Loc}.) We begin by constructing a $\Z/2$-action on $\tilde{\ld{G}}/\ld{G} \cong B/\ld{B}$.
\begin{lemma}\label{lem: gamma from T to B^}
    There is a map $\gamma: T \to \ld{B}$ such that if $x\in T$, then $\Ad_{\gamma(x)}$ sends $(fx)^{-1}$ to $fx^{-1}$; moreover, $\gamma(x)$ squares to the identity.
\end{lemma}
\begin{proof}
    This follows from the fact that $(fx)^{-1}$ and $fx^{-1}$ in $B$ both have image $x^{-1}$ under the map $B \to B \mmod \ld{B} \cong T$.
\end{proof}
\begin{definition}\label{def: cplx conj on B mod B}
    Denote by $\chi$ the map $B \to B \mmod \ld{B} \cong T$. There is an involution $\theta$ of $B$ sending $x\mapsto \Ad_{\gamma(\chi(x))}(x^{-1})$, and similarly an involution $\theta$ of the constant group scheme $\ld{B} \times T$ over $T$ sending $(g, y) \mapsto (\Ad_{\gamma(y)}(g), y^{-1})$. This defines an involution $\theta$ of $B/\ld{B}$, and hence a $\Z/2$-action on it.
\end{definition}
\begin{example}
    Suppose $G = \GL_2$ or $\SL_2$. Then one can take for $\gamma$ the constant map $T \cong \GG_m^2 \to \ld{B}$ sending $(x,y)\mapsto \begin{psmallmatrix}
        1 & 0\\
        1 & -1
    \end{psmallmatrix}$. If $G = \PGL_2$, one can simply multiply $\gamma$ by a primitive fourth root of unity to get an element of $\ld{G} = \SL_2$. If $G = \GL_3$, then one can take for $\gamma$ the map $T \cong \GG_m^3 \to \ld{B}$ sending $(x,y,z)\mapsto \begin{psmallmatrix}
        1 & 0 & 0\\
        1 & -1 & 0 \\
        0 & 0 & zy^{-1}
    \end{psmallmatrix}$.
\end{example}
It is easy to show:
\begin{lemma}
    The involution $\theta: B/\ld{B} \to B/\ld{B}$ is isomorphic to the map induced by inversion on $B$.
\end{lemma}
\begin{prop}\label{prop: cplx conj KU and B mod B^}
    The $\Z/2$-action by $\theta$ on $\tilde{\ld{G}}/\ld{G} \cong B/\ld{B}$ restricts to an action on $\tilde{\ld{G}}^\reg/\ld{G}$, and under the equivalence of \cref{cor: ku reg locus ordinary ABG}, it identifies with the $\Z/2$-action via complex conjugation on equivariant $\KU$. In particular, there is an equivalence
    $$\Loc_{T_c}^\gr(\Gr_G; \KO) \otimes_\Z F \simeq \QCoh((\tilde{\ld{G}}^\reg/\ld{G})/\pdb{\theta}).$$
\end{prop}
\begin{proof}
    It follows from \cref{def: cplx conj on B mod B} that there is a commutative diagram
    $$\xymatrix{
    T \ar[r]^-{x \mapsto x^{-1}} \ar[d]_-\kappa & T \ar[d]^-\kappa \\
    B \ar[r]_-\theta & B.
    }$$
    Therefore, $\theta$ induces an automorphism of $T \times_{B^\reg/\ld{B}} T$, and it suffices (by the proof of \cref{cor: ku reg locus ordinary ABG}) to show that under the isomorphism 
    \begin{equation}\label{eq: for proof hmlgy cplx conj}
        \spec \pi_0 \cf_T(\Gr_G)^\vee \cong T \times_{B^\reg/\ld{B}} T
    \end{equation}
    of \cref{thm: ku hmlgy reg centr}, the action of $\theta$ corresponds to the action of complex conjugation on equivariant K-theory. Let $T^\gen \subseteq T$ denote the complement of the union of all hypertori cut out by the coroots of $G$. Since both sides of \cref{eq: for proof hmlgy cplx conj} are flat and affine over $T$, their rings of functions inject into the corresponding localizations along the map $T^\gen \to T$. Furthermore, these localizations are $\Z/2$-equivariant (for complex conjugation and $\theta$, respectively), and so it suffices to show that these localizations are $\Z/2$-equivariantly isomorphic.
    
    By \cref{lem: atiyah localization}, there is an isomorphism 
    $$\pi_0 \cf_T(\Gr_G)^\vee|_{T^\gen} \cong \pi_0 \cf_T(\Gr_T)^\vee|_{T^\gen} \cong \co_{T^\gen}[\bX_\ast(T)].$$
    Under this isomorphism, the action via complex conjugation on $\KU$ is given simply by inversion on $T^\gen$, and acts trivially on $\bX_\ast(T)$.
    Similarly, since $fx \in B$ is regular \textit{semisimple} if $x\in T^\gen$, and the centralizers of regular semisimple elements are tori, there is an isomorphism
    $$(T \times_{B^\reg/\ld{B}} T) \times_T T^\gen \cong T^\gen \times \ld{T}.$$
    Under this isomorphism, the action of $\theta$ is given simply by inversion on $T^\gen$, and acts trivially on $\ld{T}$. This clearly matches with the action on $\pi_0 \cf_T(\Gr_G)^\vee|_{T^\gen}$ via complex conjugation on $\KU$, as desired.
\end{proof}
\cref{prop: cplx conj KU and B mod B^} says that, up to replacing $B/\ld{B}$ by $\ld{B}/\ld{B}$ (that is, replacing $\Loc_{T_c}^\gr(\Gr_G; \KU)$ by $\Loc_{\ld{T}_c}^\gr(\Gr_G; \KU)$), the $\Z/2$-action via complex conjugation on equivariant $\KU$ identifies under \cref{cor: ku reg locus ordinary ABG} with the $\Z/2$-action on $\ld{B}/\ld{B} = \Map(B\Z, B\ld{B})$ coming from inversion on $\Z$.
\begin{remark}\label{rmk: cplx conj on G equiv KU}
    Assume $G$ has torsion-free fundamental group. One can similarly compute the effect of complex conjugation for $G_c$-equivariant local systems. Namely, as in \cref{lem: gamma from T to B^}, there is a map $\delta: T\mmod W \to \ld{B}$ such that if $x\in T\mmod W$, then $\Ad_{\delta(x)}$ sends $(fx)^{-1}$ to $fx$. Just as in \cref{def: cplx conj on B mod B}, we obtain an involution $\Theta$ on $G/\ld{G}$ which can be identified with the effect of inversion on $G$, and the resulting $\Z/2$-action on $\QCoh(G^\reg/\ld{G})$ identifies, under the equivalence of \cref{rmk: Loc G for KU}, with the $\Z/2$-action on $\Loc_{G_c}^\gr(\Gr_G; \KU)$ coming from complex conjugation on equivariant $\KU$. This gives an equivalence
    $$\Loc_{G_c}^\gr(\Gr_G; \KO) \otimes_\Z F \simeq \QCoh((G^\reg/\ld{G})/\pdb{\Theta}).$$
    Applied to the constant sheaf, the spectral sequence \cref{eq: sseq for coh of sheaf from loc gr KO} becomes
    $$E_2^{\ast,\ast} \cong \H^\ast(\Z/2; \co_{T\mmod W \times_{G/\ld{G}} T\mmod W}[u^{\pm 1}]) \Rightarrow \KO^{G_c}_\ast(\Gr_G) \otimes_\Z F.$$
\end{remark}
Let us now make a brief comment about the case of \textit{connective} real K-theory $\ko$, discussed in \cref{rmk: connective ko def}. For this, recall from \cite[Section 3.7]{ku-rel-langlands} that if $\GG_\beta$ denotes the group scheme over $\spec(\Z[\beta])/\GG_m$ given by $\spec \Z[\beta, x, \tfrac{1}{1+\beta x}]$ with group law $x + y + \beta xy$, $\DD(\GG_\beta)$ denotes its Cartier dual, and $H_\beta$ denotes $\Hom(\DD(\GG_\beta), H)$ for any group scheme $H$, then there is an equivalence
$$\Loc_{T_c}^\gr(\Gr_G; \ku) \otimes_\Z F \simeq \QCoh(B_\beta^\reg/\ld{B}),$$
where $B_\beta^\reg$ is the regular locus in $B_\beta$. Similarly, if $G$ has torsion-free fundamental group, there is an equivalence
$$\Loc_{G_c}^\gr(\Gr_G; \ku) \otimes_\Z F \simeq \QCoh(G_\beta^\reg/\ld{G}),$$
where $G_\beta^\reg$ is the regular locus in $G_\beta$. To descend these equivalences to $\ko$-coefficients, we need to describe an action of $\spec(\pi_\ast(\ku \otimes_\ko \ku))/\GG_m$ on $B_\beta$ and $G_\beta$. This is provided by \cref{eq: coaction connective ku on Gbeta}: if we write $\pi_\ast(\ku \otimes_\ko \ku) \cong \Z[\beta, r]/(r^2 - \beta r)$, then the action is given by the map
$$\spec(\Z[\beta, r]/(r^2 - \beta r)) \times_{\spec(\Z[\beta])} G_\beta \to G_\beta, (r,x) \mapsto x - \tfrac{rx^2}{1+\beta x}.$$
When $G = \SL_n$, this can be expressed in terms of $g = \id + \beta x$:
$$(r,g) \mapsto 1 + \tfrac{(\beta - 2 r)(g-1)}{\beta} + r \tfrac{g-g^{-1}}{\beta}.$$
The action of $\spec(\pi_\ast(\ku \otimes_\ko \ku))/\GG_m$ on $B_\beta$ and $G_\beta$ defines stacks $B_\beta^\ko$ and $G_\beta^\ko$ over $\spev(\ko)$. For any closed point $\spec(F) \to \spev(\ko)$, we obtain $F$-linear equivalences
\begin{align*}
    \Loc_{T_c}^\gr(\Gr_G; \ko) \otimes_{\spev(\ko)} F & \simeq \QCoh(B_\beta^{\ko,\reg}/\ld{B}), \\
    \Loc_{G_c}^\gr(\Gr_G; \ko) \otimes_{\spev(\ko)} F & \simeq \QCoh(G_\beta^{\ko,\reg}/\ld{G});
\end{align*}
upon inverting $\beta$, these are the equivalences of \cref{prop: cplx conj KU and B mod B^} and \cref{rmk: cplx conj on G equiv KU}.

Let us note that the calculation in \cref{prop: htpy KOS1} tells us that the actual homotopy groups of $\KO^{G_c}_\ast(\Gr_G)$ could differ from a calculation at the level of $\QCoh((G^\reg/\ld{G})/\pdb{\Theta})$, and furthermore that the resulting answer could be somewhat complicated. In general, the groups $\KO^{G_c}_\ast(\Gr_G)$ will not necessarily be concentrated in even degrees, and the differentials in the preceding spectral sequence will capture some of the (equivariant) attaching maps of the (equivariant) cells in $\Gr_G$. Let us now illustrate this by describing $\KO_\ast(\Gr_G)$ for $G = \SL_3$.
\begin{example}\label{ex: KO of Gr SLn}
    If $G = \SL_2$, then the James splitting says that stable homotopy type of $\Gr_G$ splits as the direct sum $\bigoplus_{n\geq 0} S^{2n}$. This implies that $\KO_\ast(\Gr_G) \simeq \bigoplus_{n\geq 0} \KO_{\ast-2n}$, and in fact there is a ring isomorphism $\KO_\ast(\Gr_G) \cong \KO_\ast[a]$ with $a$ in weight $2$. However, already in the case $G = \SL_3$, the analogous ring isomorphism $\KO_\ast(\Gr_G) \cong \KO_\ast[a,b]$ (with $a$ in weight $2$ and $b$ in weight $4$) fails. Let us indicate the topological reason for this failure: there is a map $\CP^2 \to \Gr_{\SL_3}$ which exhibits $\CP^2$ as a generating complex, meaning that the $2$- and $4$-cells of $\CP^2$ hit the classes $a$ and $b$, respectively. The ring $\KO_\ast(\Gr_{\SL_3})$ is therefore controlled by the $\KO_\ast$-module $\KO_\ast(\CP^2)$. The key point is that a classical theorem of Wood \cite{wood-banach} (which we reprove below) gives a $\KO$-module equivalence $\KO[\CP^2] \simeq \KO \oplus \KU$. In particular, the $\KO$-module $\KO[\CP^2]$ is not equivalent to $\KO \oplus \Sigma^2 \KO$ (unlike $\KU[\CP^2]$, which is equivalent to $\KU \oplus \Sigma^2 \KU$). This implies that $\KO_\ast(\Gr_{\SL_3})$ cannot be isomorphic to $\KO_\ast[a,b]$. 
    
    In fact, this can be generalized: there are equivalences $\KO[\CP^{2n}] \simeq \KO \oplus \KU^{\oplus n}$ and $\KO[\CP^{2n+1}] \simeq \KO \oplus \KU^{\oplus n} \oplus \Sigma^{2n+2} \KO$. This implies that $\KO_\ast(\Gr_{\SL_n})$ is not isomorphic to $\KO_\ast[a_1,\cdots, a_{n-1}]$ for $n>2$ (and in fact the behavior of $\KO_\ast(\Gr_{\SL_n})$ will depend on the parity of $n$, in stark contrast to the way one usually thinks about special linear groups!). Geometrically, this arises from the fact that the generating complex $\CP^{2n+1}$ of $\Gr_{\SL_{2n+2}}$ is a spin manifold, while the generating complex $\CP^{2n}$ of $\Gr_{\SL_{2n+1}}$ is not a spin manifold; and $\KO$ is $\Spin$-oriented \cite{atiyah-bott-shapiro} (meaning that spin manifolds admit $\KO$-fundamental classes).

    Calculating $\KO_\ast(\Gr_{\SL_3})$ explicitly is somewhat unpleasant, but let us at least indicate (from the perspective of \cref{sec: degenerations}) why there is a $\KO$-module equivalence $\KO[\CP^2] \simeq \KO \oplus \KU$. To do so, let $\tilde{\KU}_\ast(\CP^2)$ denote the reduced $\KU$-homology of $\CP^2$. One then has the homotopy fixed points spectral sequence
    \begin{equation}\label{eq: hfpss KO CP2}
        E_2^{s,\ast} \cong \H^s(\Z/2; \tilde{\KU}_\ast(\CP^2)) \Rightarrow \tilde{\KO}_{\ast-s}(\CP^2),
    \end{equation}
    which we will now calculate. This can be viewed as a special case of the spectral sequence \cref{eq: sseq for coh of sheaf from loc gr}, applied to $k = \KO$ and $\cf$ being the pushforward of the constant sheaf along the map $\CP^2 \to \Gr_{\SL_3}$. To compute \cref{eq: hfpss KO CP2}, one first observed that the action of complex conjugation on $\KU_\ast(\Gr_{\SL_3}) \cong \Z[u^{\pm 1}, a,b]$ is given by $u\mapsto -u$, $a\mapsto -a$, and $b \mapsto b+a$. 
    The action on $a$ and $b$ can also be seen from the equivalence
    $$\Loc^\gr(\Gr_G; \KO) \otimes_\Z F \simeq \QCoh((\cU^\reg/\ld{G})/\pdb{\theta})$$
    derived from \cref{prop: cplx conj KU and B mod B^}, where $\cU^\reg$ is the regular locus in the unipotent cone of $G$ (and $\theta$ acts by inversion on $\cU^\reg$). The action of complex conjugation on $\KU_0(\Gr_{\SL_3})$ identifies with the action of $\theta$ on the centralizer of a regular unipotent element of $\SL_3$, which consists of matrices of the form $\begin{psmallmatrix}
        1 & a & b \\
        0 & 1 & a \\
        0 & 0 & 1
    \end{psmallmatrix}$.
    
    This specifies the action of $\Z/2$ on $\tilde{\KU}_\ast(\CP^2) \cong \Z[u^{\pm 1}]\{a,b\}$, from which we find that
    $$E_2^{\ast,\ast} \cong E_2^{0,\ast} \cong \Z\{\cdots, (2b+a)u^{-2}, au^{-1}, 2b+a, au, (2b+a)u^2, au^3, \cdots\}.$$
    The entire spectral sequence is concentrated in a single line, so it automatically degenerates; this implies that $\tilde{\KO}_\ast(\CP^2)$ is isomorphic to $\Z$ in each even degree (and is zero otherwise). There are canonical maps $\KO \to \KU$ and $\CP^2 \to \KU$, which define a map $\KO \otimes \CP^2 \to \KU$. The above calculation implies that it induces an isomorphism on homotopy groups, and hence is an equivalence.
\end{example}

It is also possible to describe an analogue of \cref{cor: ku reg locus ordinary ABG} with coefficients in the $K(1)$-local sphere $\Lone S^0$ (for some fixed prime $p$). Recall from \cref{def: J graded Loc} that if $A$ is a $p$-power torsion abelian group and $X$ is a (ind-)finite $A$-space with even cells, then the $\infty$-category $\Loc_{A}^\gr(X; \Lone S^0)$ is obtained from $\Loc_{A}^\gr(X; \KU)$ by taking homotopy $\Z_p^\times$-invariants. 
%In the following discussion, we will take $F$ to be an algebraically closed field of characteristic $0$ containing $\Z_p$.
\begin{definition}\label{def: Zpx and grothendieck springer}
    For $n\geq 0$, let $\tilde{\ld{G}}_{p^n}$ denote the (derived) fiber product
    $$\tilde{\ld{G}}_{p^n} := \tilde{\ld{G}} \times_{T} T[p^n].$$
    That is, $\tilde{\ld{G}}_{p^n}/\ld{G} \cong B_{p^n}/\ld{B}$, where $B_{p^n}$ is the subgroup of those elements of $B$ whose eigenvalues are all $p^n$th roots of unity. Similarly, let $\tilde{\ld{G}}_{p^n}^\reg$ denote the fiber product
    $$\tilde{\ld{G}}_{p^n}^\reg := \tilde{\ld{G}}^\reg \times_{T} T[p^n],$$
    There is an action of $\Z_p^\times$ (which factors through an action of $(\Z/p^n)^\times$) on $B_{p^n}$ given by exponentiation; the $\Z_p^\times$-action commutes with the $\ld{B}$-action by conjugation, and hence defines a $\Z_p^\times$-action on the quotient stack $B_{p^n}/\ld{B} \cong \tilde{\ld{G}}_{p^n}/\ld{G}$.
\end{definition}
\begin{prop}\label{prop: imJ reg locus ABG}
    Let $n\geq 0$. The $\Z_p^\times$-action on $\tilde{\ld{G}}_{p^n}/\ld{G}$ restricts to an action on $\tilde{\ld{G}}_{p^n}^\reg/\ld{G}$, and there is an equivalence
    $$\Loc_{T_c[p^n]}^\gr(\Gr_G; \Lone S^0) \otimes_{\Z_p} F \simeq \QCoh((\tilde{\ld{G}}_{p^n}^\reg/\ld{G})/\Z_p^\times).$$
\end{prop}
\begin{proof}
    Base-changing the $\QCoh(T)$-linear equivalence \cref{cor: ku reg locus ordinary ABG} along $\QCoh(T) \to \QCoh(T[p^n])$ gives an equivalence
    $$\Loc_{T_c[p^n]}^\gr(\Gr_G; \KU^\wedge_p) \otimes_\Z F \simeq \QCoh(\tilde{\ld{G}}_{p^n}^\reg/\ld{G}).$$
    Since the $\Z_p^\times$-action on $T[p^n]$ is given by exponentiation, the strategy of \cref{prop: cplx conj KU and B mod B^} shows that the $\Z_p^\times$-action on the left-hand side of the above equivalence via Adams operations on $p$-completed $\KU$ identifies with the $\Z_p^\times$-action on $\tilde{\ld{G}}_{p^n}^\reg/\ld{G}$ described in \cref{def: Zpx and grothendieck springer}. Taking homotopy $\Z_p^\times$-invariants of the displayed equivalence then yields the desired statement.
\end{proof}

The equivalences of \cref{prop: imJ reg locus ABG} are all compatible in $n$, and one finds that there is an equivalence
$$\Loc_{T_c[p^\infty]}^\gr(\Gr_G; \Lone S^0) \otimes_{\Z_p} F \simeq \QCoh((\tilde{\ld{G}}_{p^\infty}^\reg/\ld{G})/\Z_p^\times).$$
\begin{lemma}\label{lem: p-nilpotence and p power torsion}
    There is an isomorphism $B_{p^\infty} \cong B[p^\infty]$ over a $p$-nilpotent ring.
\end{lemma}
\begin{proof}
    Recall that $B_{p^\infty} = B \times_{T} T[p^\infty]$, so there is a canonical map $B[p^\infty] \to B_{p^\infty}$. It suffices to show that if $N$ denotes the unipotent radical of $B$, then $N \cong N[p^\infty]$. This follows by induction on the central series of $N$ (whose quotients are all isomorphic to $\GG_a$), and the fact that $\GG_a \cong \GG_a[p^\infty]$ since we are working over a $p$-nilpotent base.
\end{proof}
It follows that there is an equivalence
$$\Loc_{T_c[p^\infty]}^\gr(\Gr_G; \Lone S^0) \otimes_{\Z_p} F \simeq \QCoh((B[p^\infty]^\reg/\ld{B})/\Z_p^\times);$$
similarly, there is an equivalence
$$\Loc_{\ld{T}_c[p^\infty]}^\gr(\Gr_G; \Lone S^0) \otimes_{\Z_p} F \simeq \QCoh((\ld{B}[p^\infty]^\reg/\ld{B})/\Z_p^\times).$$
Note that $\ld{B}[p^\infty]^\reg/\ld{B}$ is an open substack of $\ld{B}[p^\infty]/\ld{B} \cong \colim_n \Map(B\Z/p^n, B\ld{B})$; one might heuristically view the latter as the stack of $\ld{B}$-bundles on the $p$-adic solenoid.

Finally, let us discuss the question of loop-rotation equivariance. Recall from \cref{def: nil-hecke} the algebra $\cH(\bH, T, W)$ associated to a $1$-dimensional group scheme $\bH$ over a field $F$ and a root system with torus $T$ and Weyl group $W$. In the following discussion, we will set $\bH = \GG_m$, so that $\bH_T = T$; we will also write $q$ to denote the standard character of $S^1_\rot$, so that $\pi_0 \KU_{S^1_\rot} \cong \Z[q^{\pm 1}]$. Exactly the same argument as in \cref{thm: ordinary loop-rot flag} shows the following result; here, $G$ does not need to be simply-laced.
\begin{theorem}\label{thm: ku loop-rot flag}
    There is an isomorphism of associative $\Z[q^{\pm 1}]$-algebras
    \begin{equation}\label{eq: ku comparison to nil hecke}
        \pi_0 \cf_{\tilde{T}_c}(\Fl_G)^\vee \cong \cH(\GG_m, \tilde{T}, \tilde{W}).
    \end{equation}
    Here, $\pi_0 \cf_{\tilde{T}_c}(\Fl_G)^\vee$ is equipped with the associative algebra structure coming from convolution. Moreover, the above isomorphism is also one of (cocommutative) Hopf $\pi_0 \KU_{\tilde{T}_c} \cong \co_{\bH_{\tilde{T}}}$-algebroids.
\end{theorem}
\begin{remark}
    Recall the quotient $\tilde{T}\mmod \tilde{W}$ from \cref{rmk: relationship to t mmod Waff}. The discussion therein combined with \cref{thm: ku loop-rot flag} gives an equivalence of categories
    $$\pi_0 \cf_{\tilde{T}_c}(\Fl_G)^\vee\modc \simeq \cH(\GG_m, \tilde{T}, \tilde{W})\modc \simeq \IndCoh(\tilde{T}\mmod \tilde{W}).$$
    It follows, via the argument of \cref{cor: reg locus quantized satake}, that $\Loc_{\tilde{T}_c}^\gr(\Fl_G; \KU) \otimes_\Z F$ is equivalent to the quotient of $\QCoh(\tilde{T})$ by the action of $\IndCoh(\tilde{T}\mmod \tilde{W})$.
\end{remark}
Assume, that $G$ is simple, simply-connected, and simply-laced.
Just as in \cref{sec: review Q coeff}, one would like to use \cref{thm: ku loop-rot flag} to prove analogues of \cref{cor: reg locus quantized satake} and \cref{eq: gen quantized ABG}. Namely, we expect that $\Loc_{T_c \times S^1_\rot}^\gr(\Gr_G; \KU)$ can be identified with a certain localization of the \textit{quantum} version $\co^{\univ,q}_\ld{G}$ of $\DMod_\hbar(\ld{G} / \ld{N})^{(\ld{G} \times \ld{T}, \weak)}$; the category $\co^{\univ,q}_\ld{G}$ itself is described in \cite[Definition 4.24]{univ-cat-o}. Similarly, we expect that $\Loc_{G_c \times S^1_\rot}^\gr(\Gr_G; \KU)$ can be identified with a certain localization of the \textit{quantum} version $\HC^{q}_\ld{G}$ of $\HC^\hbar_{\ld{G}}$. Here, $\HC^q_\ld{G}$ denotes a slight variant of the category described in \cite[Definition 2.24]{univ-cat-o}: instead of left $\co_q(\ld{G})$-modules in $\Rep_q(\ld{G})$, one needs to consider left $\co_q(G)$-modules in $\Rep_q(\ld{G})$. There is a fully faithful embedding of $\HC^q_\ld{G}$ into the category of $\co_q(G)$-bimodules in $\Rep_q(\ld{G})$.

The strategy of \cref{cor: reg locus quantized satake} very nearly works to prove these expectations: the only sticking point is that we do not have a $q$-analogue of \cite[Theorem 1.2.1]{ginzburg-whittaker} (which would give a $q$-analogue of \cref{cor: loop-rot Gr and biWhit}); I am currently exploring this direction of research. This is related to \cite[Conjecture 3.17]{finkelberg-tsymbaliuk}. 