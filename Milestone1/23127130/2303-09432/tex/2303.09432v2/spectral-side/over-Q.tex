To prepare ourselves for the calculations of $\pi_0(\cf_T(\Gr_G)^\vee)$ for $k$ being complex K-theory or elliptic cohomology, we begin with the simpler case of $k$ being $\QQ[u^{\pm 1}]$ with $u$ in degree $2$; recall that $\cM_{T,0}$ is then isomorphic to $\fr{t}$. In this case, the discussion in the present section follows from the work of Bezrukavnikov, Finkelberg, and Mirkovic in \cite{bfm}, as well as the work of Yun and Zhu in \cite{homology-langlands}. We will nevertheless go through this calculation (and discuss several applications) since the argument is different from that of the papers mentioned above, and also because it will serve as a useful template later. Our goal is specifically to \textit{not} appeal to the derived geometric Satake equivalence of \cite{bf-derived-satake}, but rather do the calculation in such a way that proof technique generalizes to the K-theoretic or elliptic setting, so as to apply it to prove an analogue of \cite{bf-derived-satake}.

\textit{In the remainder of this article, we will assume the group $G$ is connected, almost simple, and simply-laced.} The assumption that $G$ is simply-laced provides many simplifications; in particular, it implies that the Chevalley split forms of the groups $G$ and $\ld{G}$ are centrally isogenous (so that the adjoint action of $G$ on $\g$ descends to an action of $\ld{G}$ on $\g$), and that there is a $\ld{G}$-equivariant isomorphism $\g \cong \ld{\g}^\ast$ (even over $\Z$). However, we will \textit{never} use an $\ld{G}$-equivariant isomorphism $\ld{\g} \cong \ld{\g}^\ast$! The latter fails over $\Z$ (e.g., $\sl_2 \not\cong \pgl_2$ over $\Z$), and the effect of reliance on such failures becomes amplified in the settings of K-theory and elliptic cohomology.

In the following discussion, all dual groups are to be understood as defined over $\QQ$ (although some of our discussion will work even over $\Z$, perhaps with some small primes inverted). 

\begin{definition}[(Additive) Kostant slice]\label{def: additive kostant slice}
    Fix a nondegenerate character $\psi \in \ld{\fr{n}}^\ast$; under the isomorphism $\ld{\g}^\ast \cong \g$, there is an isomorphism $\ld{\fr{n}}^\ast \cong \fr{n}$, and $\psi$ corresponds to a principal nilpotent element $f\in \fr{n}$. Let $(e,f,h)$ be the associated $\sl_2$-triple in $\g$, and let $\psi_-: \ld{\fr{n}}_- \to \AA^1$ denote the element corresponding to $e$. Let $\ld{\g}^{\ast,\psi_-} \cong \g^e$ denote the centralizer (so $\g = \g^e \oplus [e,\g]$), and let $\cS := f + \g^e\subseteq \g^\reg$ be the Kostant slice. Note that $\cS \cong \psi + \ld{\g}^{\ast,\psi_-} \subseteq \ld{\g}^{\ast,\reg}$. The composite $f + \g^e \to \g \to \g\mmod G \cong \fr{t}\mmod W$ is an isomorphism, by \cite{kostant-lie-group-reps}.
    
    Recall that the Grothendieck-Springer resolution is defined as
    $$\tilde{\ld{\g}} = \ld{\fr{n}}^\perp \times^\ld{B} \ld{G} \cong \fr{b} \times^{\ld{B}} \ld{G},$$
    so that $\tilde{\ld{\g}}/\ld{G} \simeq \fr{b}/\ld{B}$. A point of $\tilde{\ld{\g}}$ can be regarded as a pair $(\ld{\fr{b}}', x\in (\ld{\fr{n}}')^\perp)$; here, $\ld{\fr{b}}'$ denotes a Borel subalgebra of $\ld{\g}$, and $\ld{\fr{n}}'$ denotes its nilpotent radical.
    There is a map $\tilde{\chi}: \tilde{\ld{\g}} \to \fr{t}$ which sends a pair $(\ld{\fr{b}}', x)$ to the image of $x$ modulo $(\ld{\fr{b}}')^\perp$. Let $\tilde{\cS}$ denote the fiber product $\cS\times_{\ld{\g}^\ast} \tilde{\ld{\g}}$, so that 
    $$\tilde{\cS} \subseteq \tilde{\ld{\g}}^\reg = \ld{\g}^{\ast, \reg} \times_{\ld{\g}^\ast} \tilde{\ld{\g}}.$$
    Then, Kostant's result on the Kostant slice implies formally that the composite 
    $$\tilde{\cS} \to \tilde{\ld{\g}} \xar{\tilde{\chi}} \fr{t}$$
    is an isomorphism. We will often abusively write the inclusion of $\tilde{\cS}$ as a map $\kappa: \fr{t} \to \tilde{\ld{\g}}$.

    In fact, we will only care about the composite $\fr{t} \to \tilde{\ld{\g}} \to \tilde{\ld{\g}}/\ld{G}$ below, so we will also denote it by $\kappa$. If we identify $\tilde{\ld{\g}}/\ld{G} \cong \fr{b}/\ld{B}$, then the map $\kappa$ admits a simple description: it is the composite $f + \fr{t} \to \fr{b} \to \fr{b}/\ld{B}$. (See \cref{prop: psi + t}.) In our discussion below, we will often identify $f + \fr{t}$ with $\psi + \ld{\fr{t}}^\ast$. 
\end{definition}
\begin{definition}
    The stabilizer (inside $\ld{G}$) of the Kostant slice $\cS \subseteq \g^\reg$ is a closed subgroup scheme of the constant group scheme $\ld{G} \times \cS$, and will be denoted by $\ld{J}$. It will be called the \textit{regular centralizer group scheme}; if we wish to emphasize the dependence on $G$, we will denote it by $\ld{J}(G)$.  Note that since the composite $\cS \to \g^\reg \to \g\mmod \ld{G}$ is an isomorphism, we may identify
    $$\ld{J} \cong \cS \times_{\g/\ld{G}} \cS.$$
    Similarly, the stabilizer (inside $\ld{G}$) of the Kostant slice $\tilde{\cS} \subseteq \tilde{\ld{\g}}^\reg$ is a closed subgroup scheme of the constant group scheme $\ld{G} \times \tilde{\cS}$, and will be denoted by $\tilde{\ld{J}}$. Since $\tilde{\cS} \cong \cS\times_{\ld{\g}^\ast} \tilde{\ld{\g}}$, we may identify
    $$\tilde{\ld{J}} \cong \ld{J} \times_{\cS} \tilde{\cS} \cong (f + \fr{t}) \times_{\fr{b}/\ld{B}} (f + \fr{t}).$$
\end{definition}
\begin{theorem}\label{thm: ordinary hmlgy reg centr}
    There is an isomorphism of group schemes over $f + \fr{t} \cong \fr{t} \cong \cM_{T,0}$:
    $$\spec \pi_0 \cf_T(\Gr_G)^\vee \cong (f + \fr{t}) \times_{\fr{b}/\ld{B}} (f + \fr{t}).$$
\end{theorem}
\cref{thm: ordinary hmlgy reg centr} can be proved directly using \cref{prop: hmlgy gkm}, but I find the discussion below more enlightening (of course, it is essentially an elaboration of the application of \cref{prop: hmlgy gkm}).
We first need a few lemmas.
\begin{lemma}\label{lem: kappa for borel is flat}
    The projection map $\tilde{\ld{J}} \to \psi + \ld{\fr{t}}^\ast$ (onto either factor) is flat.
\end{lemma}
\begin{proof}
    For this, we will follow \cite[Step II]{homology-langlands}. 
    Consider the morphism $\ld{B} \times \ld{\fr{t}}^\ast \to \ld{\fr{n}}^{\perp}$ sending $(g, x)\mapsto \Ad_g(\psi + x) - \psi$. Unwinding definitions shows that there is a Cartesian square
    $$\xymatrix{
    \tilde{\ld{J}} \ar[r] \ar[d] & \ld{\fr{t}}^\ast \ar[d] \\
    \ld{B} \times \ld{\fr{t}}^\ast \ar[r] & \ld{\fr{n}}^{\perp},
    }$$
    so $\tilde{\ld{J}}$ is a closed subscheme of $\ld{B} \times \ld{\fr{t}}^\ast$ of codimension $\dim(\ld{\fr{b}}^{\perp}) = \dim(\ld{N})$. This means that the fibers of the map $\tilde{\ld{J}} \to \ld{\fr{t}}^\ast$ are have dimension at least $\dim(\ld{B}) - \dim(\ld{N}) = \rank(\ld{G})$. If all fibers had dimension \textit{exactly} $\rank(\ld{G})$, then miracle flatness would imply that the map $\tilde{\ld{J}} \to \ld{\fr{t}}^\ast$ is flat. To show that all fibers have dimension $\rank(\ld{G})$, observe that there is a contracting $\GG_m$-action on the vector space $\ld{\fr{t}}^\ast$ which pushes everything down to the origin; so it suffices to show that the fiber over $0 \in \ld{\fr{t}}^\ast$ is of the correct dimension.
    
    That is, we need to see that the scheme 
    $$Y := \{(g,x)\in \ld{B} \times \ld{\fr{t}}^\ast | \Ad_g(\psi) = \psi + x\}$$
    is $\rank(\ld{G})$-dimensional. First, observe that if $\Ad_g(\psi) = \psi + x \in \ld{\fr{n}}^{\perp}$ with $x\in \ld{\fr{t}}^\ast$, then actually $x = 0$. This is because the image of $x$ under the map 
    $$\ld{\fr{n}}^{\perp} \to (\ld{\fr{n}} \oplus \ld{\fr{n}}^-)^\perp \cong \ld{\fr{t}}^\ast$$
    is the same as the image of $\psi + x$, which is the same as the image of $\Ad_g(\psi)$. But the above map $\ld{\fr{n}}^{\perp} \to \ld{\fr{t}}^\ast$ is $\Ad$-invariant, and so the image of $\Ad_g(\psi)$ is equal to the image of $\psi$, which is zero. This means that the image of $x$ is also zero. But the inclusion $\ld{\fr{t}}^\ast \subseteq \ld{\fr{n}}^{\perp}$ splits the map $\ld{\fr{n}}^{\perp} \to \ld{\fr{t}}^\ast$, and so we see that $x = 0$. Therefore,
    $$Y \cong \{g\in \ld{B} | \Ad_g(\psi) = \psi\} = Z_{\ld{B}}(\psi).$$
    The centralizer of $\psi$ is contained entirely in $\ld{B}$, so $Z_{\ld{B}}(\psi) \cong Z_{\ld{G}}(\psi)$. This, in turn, has dimension given by $\rank(\ld{G})$ since $\psi$ (corresponding to $e \in \g$) is a regular nilpotent.
\end{proof}
Note that
$$\tilde{\ld{J}} \cong \{(x,y,g) \in \ld{\fr{t}}^\ast \times \ld{\fr{t}}^\ast \times \ld{B} | \Ad_g(\psi + x) = \psi + y\}.$$
The argument at the end of \cref{lem: kappa for borel is flat} allows us to identify $x = y\in \ld{\fr{t}}^\ast$, and so
$$\tilde{\ld{J}} \cong \{(x,g) \in \ld{\fr{t}}^\ast \times \ld{B} | \Ad_g(\psi + x) = \psi + x\}.$$

\begin{notation}
    If $\alpha$ is a root of $\ld{G}$, let $\{e_\alpha, h_\alpha\}$ denote a pinning of $\ld{G}$. Say that a point $x \in \ld{\fr{t}}^\ast$ is \textit{$\alpha$-generic} if $x(h_\beta) \neq 0$ for all roots $\beta\neq \alpha$. This implies that the centralizer $Z_{\ld{G}}(x)$ has semisimple rank at most $1$. Let $\ld{\fr{t}}^\ast_{\alpha\dreg}$ denote the $\alpha$-regular locus. Observe that $\ld{\fr{t}}^\ast_\reg = \bigcup_{\alpha\in \Phi} \ld{\fr{t}}^\ast_{\alpha\dreg} \subseteq \ld{\fr{t}}^\ast$ is open, with complement of codimension $2$.
\end{notation}
\begin{lemma}\label{lem: localization of ordinary J}
    There is an isomorphism
    \begin{equation}\label{eq: J of centralizer}
        \tilde{\ld{J}}(\ld{G})|_{\ld{\fr{t}}^\ast_{\alpha\dreg}} \xar{\sim} \tilde{\ld{J}}(Z_{\ld{G}}(x)^\circ)|_{\ld{\fr{t}}^\ast_{\alpha\dreg}},
    \end{equation}
    where $Z_{\ld{G}}(x)$ is the centralizer of some $x\in \ld{\fr{t}}^\ast_{\alpha\dreg}$ which lies on the $\alpha$-hyperplane, and $Z_{\ld{G}}(x)^\circ$ denotes the connected component of the identity. 
\end{lemma}
\begin{proof}[Proof sketch]
    Let us, for simplicity, write $\ld{H}$ to denote $Z_{\ld{G}}(x)^\circ$.
    There is a map from the left-hand side to the right-hand side, which sends 
    $$\ld{\fr{t}}^\ast \times \ld{B} \ni (x, g)\mapsto (x, g)\in \ld{\fr{t}}^\ast \times (\ld{B} \cap \ld{H}).$$
    Note that $\ld{B} \cap \ld{H}$ is a Borel subgroup of $\ld{H}$.
    To see that the above map gives an isomorphism, observe that if $y\in \ld{\fr{t}}^\ast$, we may identify the centralizer in $\ld{G}$ of $\psi + y$ with the centralizer in $Z_{\ld{G}}(y)^\circ$ of $\psi$.  That \cref{eq: J of centralizer} is an isomorphism is now a consequence of the observation that if $y\in \ld{\fr{t}}^\ast_{\alpha\dreg}$, then this centralizer $Z_{\ld{G}}(y)^\circ$ is contained in $\ld{H}$. That is, if $(x,g)\in \tilde{\ld{J}}(\ld{G})|_{\ld{\fr{t}}^\ast_{\alpha\dreg}}$, then $g$ is already contained in $H$, and so $(x,g)\in \tilde{\ld{J}}(\ld{H})|_{\ld{\fr{t}}^\ast_{\alpha\dreg}}$.
\end{proof}
\begin{proof}[Proof of \cref{thm: ordinary hmlgy reg centr}]
    We begin by noting that $\Gr_G$ only has even cells; so $\pi_0 \cf_T(\Gr_G)^\vee = \pi_0 C_\ast^T(\Gr_G; \QQ[u^{\pm 1}])$ can be identified with $\H_\ast^T(\Gr_G; \QQ)$, regarded now as an ungraded $\QQ$-algebra. Similarly, $\pi_0(k_T) \cong \H^\ast_T(\ast; \QQ)$, again regarded as an ungraded $\QQ$-algebra.
    The equivariant formality of $\Gr_G$ implies that $\H_\ast^T(\Gr_G; \QQ)$ is flat over $\H^\ast_T(\ast; \QQ)$. To prove \cref{thm: ordinary hmlgy reg centr}, it therefore suffices to prove an isomorphism 
    $$\tilde{\ld{J}}|_{\ld{\fr{t}}^\ast_{\alpha\dreg}} \cong \spec \H^{T_c}_\ast(\Omega G; \QQ)|_{\ld{\fr{t}}^\ast_{\alpha\dreg}}$$
    for each root $\alpha$. By Atiyah-Bott localization, the right-hand side can be identified with 
    \begin{equation}\label{eq: atiyah bott for gr}
        \spec \H^{T_c}_\ast(\Omega G; \QQ)|_{\ld{\fr{t}}^\ast_{\alpha\dreg}} \cong \spec \H^{T_c}_\ast(\Omega Z_G(x); \QQ)|_{\ld{\fr{t}}^\ast_{\alpha\dreg}},
    \end{equation}
    where $Z_G(x)$ is the centralizer of some $x\in \fr{t}_{\alpha\dreg}$ which lies on the $\alpha$-hyperplane. 
    Note that the right-hand side depends only on the connected component $Z_G(x)^\circ$ of the identity in $Z_G(x)$; so we might as well replace $Z_G(x)$ by $Z_G(x)^\circ$. Using \cref{lem: localization of ordinary J}, we are therefore reduced to showing that there is an isomorphism
    $$\H^{T_c}_\ast(\Omega Z_G(x)^\circ; \QQ)|_{\ld{\fr{t}}^\ast_{\alpha\dreg}} \cong \tilde{\ld{J}}(Z_{\ld{G}}(x)^\circ)|_{\ld{\fr{t}}^\ast_{\alpha\dreg}}.$$
    Since $Z_G(x)^\circ$ has semisimple rank $1$, we are reduced to checking that \cref{thm: ordinary hmlgy reg centr} holds in this case.

    That is, we may assume $G$ is the product of a torus with $\GL_2$, $\SL_2$, or $\PGL_2$. It is easy to match up the contribution from the toral factors, so we will assume that $G$ is $\GL_2$, $\SL_2$, or $\PGL_2$.
    \begin{itemize}
        \item For $\GL_2$, we may identify $\gl_2^\ast \cong \gl_2$. Then, $\tilde{\ld{J}}$ is the centralizer (in $\ld{B}$) of $\begin{psmallmatrix}
            x & 0 \\
            1 & y
        \end{psmallmatrix}$. It is not hard to compute directly that $\begin{psmallmatrix}
            a & 0\\
            c & d
        \end{psmallmatrix}$ stabilizes $\begin{psmallmatrix}
            x & 0 \\
            1 & y
        \end{psmallmatrix}$ if and only if $c = \tfrac{a-d}{x-y}$, meaning that
        $$\tilde{\ld{J}} \cong \spec \QQ[x,y, a^{\pm 1}, d^{\pm 1}, \tfrac{a-d}{x-y}].$$
        The coproduct sends $a\mapsto a \otimes a$ and $d\mapsto d \otimes d$.
        
        Let us now calculate $\H^{T^2}_\ast(\Omega \GL_2; \QQ)$ as an algebra over $\H^\ast_{T^2}(\ast; \QQ) \cong \QQ[x, y]$. There is a simple $T^2$-equivariant cell decomposition of $\Omega \GL_2$ with $\bX_\ast(T^2) = \Z^2$ many $0$-cells, and where there is a $T^2$-equivariant $1$-cell connecting $\mu_1$ to $\mu_2$ if and only if $\mu_1 - \mu_2$ is a multiple of a root of $\GL_2$. (There are higher equivariant cells, but they will not matter.) This implies, by Atiyah-Bott localization, that the fixed points of the $T^2$-action on $\Omega \GL_2$ are simply $\Omega T^2 = \Z^2$, and so
        $$\H^{T^2}_\ast(\Omega \GL_2; \QQ)[\tfrac{1}{x-y}] \cong \H^{T^2}_\ast(\Omega T^2; \QQ)[\tfrac{1}{x-y}] \cong \QQ[x, y, \tfrac{1}{x-y}, a^{\pm 1}, d^{\pm 1}].$$
        On the other hand, the \textit{completion} $\H^{T^2}_\ast(\Omega \GL_2; \QQ)^\wedge_{(x-y)}$ can be determined directly. After completing at $(x-y, y) = (x,y)$, the equivariant homology $\H^{T^2}_\ast(\Omega \GL_2; \QQ)$ simply becomes the \textit{Borel-equivariant} homology, and this can be computed directly via a spectral sequence
        $$E_2 = \H^\ast(BT^2; \QQ) \otimes_k \H_\ast(\Omega \GL_2; \QQ) \Rightarrow \H^{T^2}_\ast(\Omega \GL_2; \QQ)^\wedge_{(x,y)}.$$
        Since $\H_\ast(\Omega \GL_2; \QQ) = \QQ[A^{\pm 1}, b]$ with $A$ in weight $0$ and $b$ in weight $2$, the $E_2$-page of this spectral sequence is concentrated entirely in even degrees, and hence collapses. This means that 
        $$\H^{T^2}_\ast(\Omega \GL_2; \QQ)^\wedge_{(x,y)} \cong k\pw{x,y}[A^{\pm 1}, b].$$
        This in fact comes from an isomorphism
        $$\H^{T^2}_\ast(\Omega \GL_2; \QQ)^\wedge_{(x-y)} \cong \QQ[x, y, A^{\pm 1}, b]^\wedge_{(x-y)}.$$
        We may therefore recover $\H^{T^2}_\ast(\Omega \GL_2; \QQ)$ via the gluing square
        $$\xymatrix{
        \H^{T^2}_\ast(\Omega \GL_2; \QQ) \ar[r] \ar[d] & \H^{T^2}_\ast(\Omega \GL_2; \QQ)[\tfrac{1}{x-y}] \ar[d] \\
        \H^{T^2}_\ast(\Omega \GL_2; \QQ)^\wedge_{(x-y)} \ar[r] & \H^{T^2}_\ast(\Omega \GL_2; \QQ)^\wedge_{(x-y)}[\tfrac{1}{x-y}].
        }$$
        Explicitly:
        $$\xymatrix{
        \H^{T^2}_\ast(\Omega \GL_2; \QQ) \ar[r] \ar[d] & \QQ[x, y, \tfrac{1}{x-y}, a^{\pm 1}, d^{\pm 1}]\ar[d] \\
        \QQ[x, y, A^{\pm 1}, b]^\wedge_{(x-y)} \ar[r] & \QQ[x, y, A^{\pm 1}, b]^\wedge_{(x-y)}[\tfrac{1}{x-y}].
        }$$
        The right vertical map sends $a-d \mapsto b(x-y)$; and $d \mapsto A$. Note that $b(x-y)$ is topologically nilpotent, so $A + b(x-y)$ is a unit, and this is what $a$ maps to. This discussion implies that the fiber product above identifies with
        $$\H^{T^2}_\ast(\Omega \GL_2; \QQ) \cong \QQ[x, y, a^{\pm 1}, d^{\pm 1}, \tfrac{a-d}{x-y}].$$
        We need to determine the coproduct. Since this ring is flat over $\QQ[x,y]$, it suffices to determine the coproduct after inverting $x-y$. As we have seen, $\H^{T^2}_\ast(\Omega \GL_2; \QQ)[\tfrac{1}{x-y}] \cong \H^{T^2}_\ast(\Omega T^2; \QQ)[\tfrac{1}{x-y}]$, and $\Omega T^2 = \Z^2$. The coproduct here simply comes from the \textit{diagonal} on $\Z^2$, which obviously sends $a\mapsto a \otimes a$ and $d\mapsto d \otimes d$. It follows that 
        $$\spec \H^{T^2}_\ast(\Omega \GL_2; \QQ) \cong \tilde{\ld{J}}$$
        as (graded) group schemes over $\QQ[x,y]$, as desired.
        \item For $G = \SL_2$, one can similarly calculate that
        $$\H^{S^1}_\ast(\Gr_{\SL_2}; \QQ) \cong \QQ[x, a^{\pm 1}, b]/(a = 1 + 2xb) \cong \QQ[x, a^{\pm 1}, \tfrac{a-1}{2x}].$$
        The coproduct is determined by the formula $a\mapsto a \otimes a$, so that
        $$b\mapsto b \otimes 1 + 1\otimes b + 2x b \otimes b.$$
        For completeness, let us quickly summarize the argument. The fixed points of $S^1$ acting on $\Omega \SU(2)$ is $\Omega S^1 = \Z$, and the action of $S^1$ on $\SU(2) \cong S^3$ exhibits it as the one-point compactification of $\RR\oplus \cc$, where $\RR$ is the trivial representation and $\cc$ is the \textit{weight $2$} representation. Therefore, inverting the Chern class $2x$ of the weight $2$ representation lets us identify
        $$\H^{S^1}_\ast(\Gr_{\SL_2}; \QQ)[\tfrac{1}{2x}] \cong \H^{S^1}_\ast(\Omega S^1; \QQ)[\tfrac{1}{2x}] \cong \QQ[x^{\pm 1}, a^{\pm 1}].$$
        On the other hand, the completion of $\H^{S^1}_\ast(\Gr_{\SL_2}; \QQ)$ at the class $2x$ is, via the same spectral sequence argument as in the preceding bullet point, given by
        $$\H^{S^1}_\ast(\Gr_{\SL_2}; \QQ)^\wedge_{(2x)} \cong \QQ\pw{x}[b],$$
        with $b$ in weight $2$. The ring $\H^{S^1}_\ast(\Gr_{\SL_2}; \QQ)$ can be recovered via the gluing square
        $$\xymatrix{
        \H^{S^1}_\ast(\Gr_{\SL_2}; \QQ) \ar[r] \ar[d] & \H^{S^1}_\ast(\Gr_{\SL_2}; \QQ)[\tfrac{1}{2x}] \ar[d] \\
        \H^{S^1}_\ast(\Gr_{\SL_2}; \QQ)^\wedge_{(2x)} \ar[r] & \H^{S^1}_\ast(\Gr_{\SL_2}; \QQ)^\wedge_{(2x)}[\tfrac{1}{2x}].
        }$$
        The right vertical map sends $a-1 \mapsto b \cdot 2x$, and so the above Cartesian square gives an isomorphism
        $$\H^{S^1}_\ast(\Gr_{\SL_2}; \QQ) \cong \QQ[x, a^{\pm 1}, b]/(a = 1 + 2xb),$$
        as desired.
        
        On the other hand, $\tilde{\ld{J}}$ is the centralizer in $\ld{B}\subseteq \PGL_2$ of $\begin{psmallmatrix}
            x & 0\\
            1 & -x
        \end{psmallmatrix} \subseteq \fr{sl}_2 \cong \ld{\g}^\ast$. It is easy to compute directly that $\begin{psmallmatrix}
            a & 0 \\
            c & 1
        \end{psmallmatrix} \in \ld{B}$ (where we only care about this as an element of $\PGL_2$!) stabilizes $\begin{psmallmatrix}
            x & 0\\
            1 & -x
        \end{psmallmatrix}$ if and only if $2xc = a-1$. Therefore, 
        $$\tilde{\ld{J}} \cong \spec \QQ[x, a^{\pm 1}, c]/(a = 1 + 2xc),$$
        and again the group law is determined by the formulae
        $$a\mapsto a \otimes a, \ c \mapsto c \otimes 1 + 1\otimes c + 2x c \otimes c.$$
        Therefore, 
        $$\spec \H^{S^1}_\ast(\Gr_{\SL_2}; \QQ) \cong \tilde{\ld{J}}$$
        as (graded) group schemes over $\QQ[x]$, as desired.
        \item In exactly the same way, for $G = \PGL_2$, one can similarly calculate that
        $$\H^{S^1}_\ast(\Omega \PGL_2; \QQ) \cong \QQ[x, a^{\pm 1}, b]/(a^2 = 1 + xb) \cong \QQ[x, a^{\pm 1}, \tfrac{a^2-1}{x}].$$
        This is because the fixed points of $S^1$ acting on $\Omega \PGL_2 \simeq \Z/2 \times \Omega S^3$ is $\Z$, and the action of $S^1$ on $\PGL_2$, which is homotopy equivalent to $\RP^3$, exhibits it as the $\Z/2$-quotient of the one-point compactification of $\RR\oplus \cc$, where $\RR$ is the trivial representation and $\cc$ is the \textit{weight $1$} representation. The coproduct is determined by the formula $a\mapsto a \otimes a$, so that
        $$b\mapsto b \otimes 1 + 1\otimes b + x b \otimes b.$$

        On the other hand, $\tilde{\ld{J}}$ is the centralizer in $\ld{B}\subseteq \SL_2$ of the equivalence class of $\begin{psmallmatrix}
            x & 0\\
            1 & 0
        \end{psmallmatrix}$ in $\pgl_2 \cong \ld{\g}^\ast$. It is easy to compute directly that $\begin{psmallmatrix}
            a & 0 \\
            c & a^{-1}
        \end{psmallmatrix} \in \ld{B}$ stabilizes $\begin{psmallmatrix}
            x & 0\\
            1 & 0
        \end{psmallmatrix}$ if and only if $xc = a-a^{-1}$. Therefore, 
        $$\tilde{\ld{J}} \cong \spec \QQ[x, a^{\pm 1}, c]/(a = a^{-1} + xc) \cong \spec \QQ[x, a^{\pm 1}, \tfrac{a - a^{-1}}{x}].$$
        Replacing $c$ by $b := ca^{-1}$, we see that the group law is determined by the formulae
        $$a\mapsto a \otimes a, \ b\mapsto b \otimes 1 + 1\otimes b + x b \otimes b.$$
        Therefore, 
        $$\spec \H^{S^1}_\ast(\Omega \PGL_2; \QQ) \cong \tilde{\ld{J}}$$
        as (graded) group schemes over $\QQ[x]$, as desired.\qedhere
    \end{itemize}
\end{proof}
\begin{remark}
    Just for posterity, let us record a more canonical variant of the calculation above for $\ld{G} = \SL_2$, which does not require picking a Borel subgroup (i.e., which does not involve identifying $\tilde{\ld{\g}}/\ld{G} \cong \fr{b}/\ld{B}$). For simplicity, we will use the fact that $2$ is invertible in $\QQ$ to identify $\fr{sl}_2 \cong \fr{pgl}_2$. In this case, the Grothendieck-Springer resolution $\tilde{\ld{\g}} = T^\ast(\AA^2-\{0\})/\GG_m$ is the total space of $\co(-1) \oplus \co(-1)$ over $\PP^1$; we will think of a point in $\tilde{\g}$ as a pair $(x\in \sl_2, \ell\subseteq \cc^2)$ such that $x$ preserves $\ell$. The Kostant slice $\kappa:\fr{t} \cong \AA^1 \to \tilde{\ld{\g}}$ is the map sending $\lambda \in \AA^1$ to the pair $(x, \ell)$ with $x = \begin{psmallmatrix}
    0 & \lambda^2 \\
    1 & 0
    \end{psmallmatrix}$ and $\ell = [\lambda: 1]$. Indeed, this is essentially immediate from the requirement that the following diagram commutes:
    $$\xymatrix{
    \AA^1 \cong \fr{t} \ar[r]^-\kappa \ar[d]_-{\lambda \mapsto \lambda^2} & \tilde{\sl}_2 \ar[d]\\
    \AA^1 \cong \fr{t}\mmod W \ar[r]^-\kappa_-{\lambda\mapsto \begin{psmallmatrix}
    0 & \lambda \\
    1 & 0
    \end{psmallmatrix}} & \sl_2.
    }$$
    Moreover, the $\SL_2$-action on $\tilde{\ld{\g}}$ sends $g\in \SL_2$ and $(x,\ell)$ to $(\Ad_g(x), g\ell)$. If $g = \begin{psmallmatrix}
    a & b \\
    c & d
    \end{psmallmatrix}$, we compute that
    $$\Ad_g\begin{pmatrix}
    0 & \lambda^2 \\
    1 & 0
    \end{pmatrix} = \begin{pmatrix}
    bd-ac\lambda^2 & (a\lambda)^2 - b^2 \\
    d^2 - (c\lambda)^2 & ac\lambda^2 - bd
    \end{pmatrix}, \ g\cdot [\lambda: 1] = [a\lambda + b: c\lambda + d].$$
    From this, we see that $\Ad_g(x) = x$ if and only if $a = d$ and $b = c\lambda^2$, in which case $g$ also fixes $[\lambda: 1]$. In other words, $g = \begin{psmallmatrix}
    a & c\lambda^2 \\
    c & a
    \end{psmallmatrix}$ with $a,c\in k$; in order for $\det(g) = 1$, we need $a^2-c^2\lambda^2=1$. When $\lambda \neq 0$, both $x$ and $g$ are diagonalized by the matrix $\tfrac{1}{2}\begin{psmallmatrix}
    1 & -1 \\
    -\lambda^{-1} & -\lambda^{-1}
    \end{psmallmatrix}\in \SL_2$: the diagonalization of $x$ is $\begin{psmallmatrix}
    \lambda & 0 \\
    0 & \lambda^{-1}
    \end{psmallmatrix}$, and the diagonalization of $g$ is $\begin{psmallmatrix}
    t & 0 \\
    0 & w
    \end{psmallmatrix}$ where $2a = t+w$ and $2\lambda c = t-w$. Since we have $\det(g) = a^2 - (c\lambda)^2 = 1$, we have $w = t^{-1}$. This shows that if $k$ is not of characteristic $2$, then $\fr{t} \times_{\tilde{\sl}_2/\SL_2} \fr{t} \cong \spec \QQ[\lambda, t^{\pm 1}, \tfrac{t-t^{-1}}{\lambda}]$.
\end{remark}
\begin{corollary}\label{cor: reg locus ordinary ABG}
    There is an equivalence
    $$\Loc_{T_c}^\gr(\Gr_G; k) \simeq \QCoh(\tilde{\ld{\g}}^\reg/\ld{G}).$$
    Furthermore, the pushforward functor $\Loc_{T_c}^\gr(\Gr_G; k) \to \Loc_{T_c}^\gr(\ast; k)$ identifies with the pullback functor $\kappa^\ast: \QCoh(\tilde{\ld{\g}}^\reg/\ld{G}) \to \QCoh(\fr{t})$.
\end{corollary}
\begin{proof}
    By definition, $\Loc_{T_c}^\gr(\Gr_G;k)$ is equivalent to the category of comodules over $\pi_0 \cf_T(\Gr_G)^\vee = \H_\ast^T(\Gr_G; \QQ)$ in the category of $\pi_0 k_T \cong \H^\ast_T(\ast; \QQ)$-modules. By \cref{thm: ordinary hmlgy reg centr}, it can be identified the category of quasicoherent sheaves on the quotient stack $(f + \fr{t})/\tilde{\ld{J}}$. As discussed after \cref{lem: kappa for borel is flat}, we may view $\tilde{\ld{J}}$ as a closed subgroup scheme of the constant group scheme $\ld{B} \times (f + \fr{t})$. This gives an isomorphism
    $$(f + \fr{t})/\tilde{\ld{J}} \cong \ld{B} \backslash (\ld{B} \times (f + \fr{t}))/\tilde{\ld{J}}.$$
    It follows from Kostant's work in \cite{kostant-lie-group-reps} that the $\ld{B}$-orbit of $f + \fr{t}$ inside $\fr{b}$ is precisely the regular locus $\fr{b}^\reg$. Since $\tilde{\ld{J}}$ is definitionally the stabilizer of $f + \fr{t} \subseteq \fr{b}$, the quotient $(\ld{B} \times (f + \fr{t}))/\tilde{\ld{J}}$ is isomorphic to $\fr{b}^\reg$; so there is an isomorphism $(f + \fr{t})/\tilde{\ld{J}} \cong \fr{b}^\reg/\ld{B}$.
    To finish, note that $\tilde{\ld{\g}}^\reg/\ld{G} \cong \fr{b}^\reg/\ld{B}$.
\end{proof}
The equivalence of \cref{cor: reg locus ordinary ABG} is in fact symmetric monoidal for the convolution tensor structure on $\Loc_{T_c}^\gr(\Gr_G; k)$ (described in \cref{rmk: loc gr convolution tensor}) and the standard tensor product on $\QCoh(\tilde{\ld{\g}}^\reg/\ld{G})$.

\begin{remark}
    Note that the definition of the Kostant slice $f + \fr{t} \subseteq \fr{b}$ involved the choice of a regular nilpotent element $f \in \g$. However, this choice does not materialize in \cref{cor: reg locus ordinary ABG}. This is because two such slices obtained by choosing two different regular nilpotent elements in $\g$ are \textit{conjugate} to each other (by $\ld{B}$). That is, while the specific inclusion $f + \fr{t} \subseteq \fr{b}$ depends on the choice of $f$, the composite $f + \fr{t} \subseteq \fr{b} \to \fr{b}/\ld{B}$ is independent of said choice. 
\end{remark}
\begin{example}
    Suppose $G = \SL_n$. In this case, $\H_\ast(\Gr_{\SL_n}; \QQ)$ is simply isomorphic to a polynomial algebra $\QQ[b_1, \cdots, b_{n-1}]$ on $n-1$ generators. The coproduct is given by $b_j \mapsto \sum_i b_i \otimes b_{j-i}$, where $b_0$ is understood to be $1$. This result  is classical, and can be found, for instance, in \cite{bott-space-of-loops}. The proof there amounts to the following observation. Consider the map $\CP^{n-1} \to \Gr_{\SL_n}$ given by sending $\ell \in \CP^{n-1}$ to (an appropriate rescaling of) the loop sending $\theta \in S^1$ to rotation by angle $\theta$ about the line $\ell$. Then the image of the induced map $\H_\ast(\CP^{n-1}; \QQ) \to \H_\ast(\Gr_{\SL_n}; \QQ)$ generates $\H_\ast(\Gr_{\SL_n}; \QQ)$; that is, $\CP^{n-1}$ is a generating complex for $\Gr_{\SL_n}$. The formula for the coproduct comes from the coproduct on $\H_\ast(\CP^{n-1}; \QQ)$, which is determined easily by the cup product on $\H^\ast(\CP^{n-1}; \QQ)$. The above description of $\H_\ast(\Gr_{\SL_n}; \QQ)$ implies that $\spec \H_\ast(\Gr_{\SL_n}; \QQ)$ is isomorphic to the group scheme $\WW_{n-1}$ of big Witt vectors of length $n-1$.

    On the other hand, \cref{thm: ordinary hmlgy reg centr} implies that $\spec \H_\ast(\Gr_{\SL_n}; \QQ)$ is isomorphic to the centralizer inside $\PGL_n$ of of the regular nilpotent $f \in \sl_n$. Indeed, if $R$ is a $\QQ$-algebra, then an element $g \in \GL_n(R)$ commutes with $f$ if and only if $g$ is an invertible polynomial in $e$. By the Cayley-Hamilton theorem, such a polynomial is divisible by the minimal polynomial $t^n$ of $e$; that is, $g \in (R[t]/t^n)^\times$. For this to live in $\PGL_n(R)$, we need to quotient out by the scalars $R^\times$. The assignment $R \mapsto (1 + tR[t]/t^n)^\times$ is precisely the functor of points of $\WW_{n-1}$. One can therefore understand the isomorphism between $\spec \H_\ast(\Gr_{\SL_n}; \QQ)$ and $Z_{\PGL_n}(e)$ as being a way to identify the two descriptions of the Witt vector group scheme (either via its functor of points, or via the explicit Witt addition law).
\end{example}
\begin{example}\label{ex: equiv homology Omega SUn and Witt}
    Continuing the preceding example (so $G = \SL_n$), it is not hard to add in torus-equivariance (so $T_c = (S^1)^{n-1}$). In this case, we will identify $\H^\ast_{T_c}(\ast; \QQ) \cong \QQ[x_1, \cdots, x_n]$. One can write down an explicit $T_c$-equivariant cell structure on $\Omega \SU(n)$ to find that $\spec \H_\ast^{T_c}(\Gr_{\SL_n}; \QQ)$ is isomorphic to the deformation of $\WW_{n-1}$ over $\spec \H^\ast_{T_c}(\ast; \QQ) \cong \AA^n$ which sends a $\QQ[x_1, \cdots, x_n]$-algebra $R$ to the group of units $(1 + tR[t]/(t-x_1)\cdots(t-x_n))^\times$. On the other hand, by the same argument using Cayley-Hamilton, the centralizer inside $\GL_n(R)$ of $f+x \in \sl_n$ is isomorphic to the group $(R[t]/(t-x_1)\cdots(t-x_n))^\times$, since the characteristic polynomial of $f+x$ is precisely $(t-x_1)\cdots(t-x_n)$. Quotienting by the scalars $R^\times$, we obtain an isomorphism between $\spec \H_\ast^{T_c}(\Gr_{\SL_n}; \QQ)$ and $Z_{\PGL_n}(f + x)$.
\end{example}
\begin{remark}
    For a general reductive group $G$, Kostant proved (in \cite{kostant-whittaker}) an isomorphism $(f + \fr{b})/\ld{N} \cong \fr{t}\mmod W$. In fact, the natural map $(f + \fr{b})/\ld{N} \to \g/\ld{G}$ identifies with the map $\cS \to \g/\ld{G}$ given by the Kostant slice. Since $(f + \fr{b})/\ld{N}$ is isomorphic to the quotient $\ld{G} \backslash T^\ast(\ld{G}/_\psi \ld{N})$ of the Whittaker reduction of $T^\ast(\ld{G})$, it follows that there are isomorphisms
    \begin{align*}
        \ld{J} & \cong (f + \fr{b})/\ld{N} \times_{\g/\ld{G}} (f + \fr{b})/\ld{N} \\
        & \cong \ld{G} \backslash T^\ast(\ld{G}/_\psi \ld{N}) \times_{\ld{\g}^\ast/\ld{G}} T^\ast(\ld{N} {}_\psi \backslash \ld{G})/\ld{G} \\
        & \cong T^\ast(\ld{N} {}_\psi \backslash \ld{G} /_\psi \ld{N}).
    \end{align*}
    That is, $\ld{J}$ can be identified with the \textit{bi-Whittaker reduction} of the cotangent bundle $T^\ast(\ld{G})$. In particular, \cref{thm: ordinary hmlgy reg centr} gives an isomorphism
    $$\spec \H^{T_c}_\ast(\Gr_G; \QQ) \cong \ld{\fr{t}}^\ast \times_{\ld{\fr{t}}^\ast\mmod W} T^\ast(\ld{N} {}_\psi \backslash \ld{G} /_\psi \ld{N}).$$
    In fact, this isomorphism can be checked to be $W$-equivariant (for the action of $W$ on $\H^{T_c}_\ast(\Gr_G; \QQ)$ via the action on $T$, and for the action on the right-hand side coming from the cover $\ld{\fr{t}}^\ast \to \ld{\fr{t}}^\ast\mmod W$). This implies that there is an isomorphism
    $$\spec \H^G_\ast(\Gr_G; \QQ) \cong T^\ast(\ld{N} {}_\psi \backslash \ld{G} /_\psi \ld{N}).$$
    This isomorphism has been exploited heavily in \cite{teleman-icm}, among others.
\end{remark}

The map $\tilde{\ld{\g}}^\reg/\ld{G} \to B\ld{G}$ defines a functor
\begin{equation}\label{eq: Rep G^ to ordinary loc}
    \Rep(\ld{G}) \to \QCoh(\tilde{\ld{\g}}^\reg/\ld{G}) \simeq \Loc_{T_c}^\gr(\Gr_G; k).
\end{equation}
More generally, the map $\tilde{\ld{\g}}^\reg/\ld{G} \to B\ld{T} \times B\ld{G}$ defines a functor
\begin{equation}\label{eq: Rep T^ x G^ to ordinary loc}
    \Rep(\ld{T} \times \ld{G}) \to \QCoh(\tilde{\ld{\g}}^\reg/\ld{G}) \simeq \Loc_{T_c}^\gr(\Gr_G; k).
\end{equation}
If $V \in \Rep(\ld{G})$, let $\cS_k(V)$ denote the corresponding object of $\Loc_{T_c}^\gr(\Gr_G; k)$. It is natural to ask whether $\cS_k(V) \in \Loc_{T_c}^\gr(\Gr_G; k)$ is given by $\cf^\gr$ for some $\cf \in \Loc_{T_c}(\Gr_G; k)$. Of course, \cref{cor: reg locus abg} says that the answer is yes; but it is not clear how to answer this question in a manner that will generalize to other $\Eoo$-rings $k$. However, it is possible to give a positive (and generalizable) answer to this question in the case when $V$ is a direct sum of tensor products of irreducible representations with \textit{minuscule} highest weights.
\begin{prop}\label{prop: ordinary realizing minuscule reps}
    Let $\lambda_\bull = (\lambda_1, \cdots, \lambda_n)$ be a tuple of dominant minuscule weights of $\ld{G}$, let $|\lambda_\bull| = \sum_i \lambda_i$, and let $\ol{\Gr_G^{\lambda_\bull}}$ denote the corresponding \textit{convolution variety} \cite{mirkovic-vilonen, ngo-polo}. Let $\cf_{\lambda_\bull}$ denote the pushforward of the constant sheaf along the canonical map $q: \ol{\Gr_G^{\lambda_\bull}} \to \ol{\Gr_G^{|\lambda|}} \subseteq \Gr_G$. If $V_{\lambda_i}$ denotes the irreducible representation of $\ld{G}$ with highest weight $\lambda_i$, then there is an isomorphism $\cS_k(\bigotimes_i V_{\lambda_i}) \cong \cf_{\lambda_\bull}^\gr$.
\end{prop}
\begin{proof}
    First, suppose that $\lambda_\bull = \lambda$ consists of single element. Let $P_\lambda \subseteq G$ denote the corresponding maximal parabolic subgroup, so that $\ol{\Gr_G^\lambda} \cong G/P_\lambda$, and let $\cf_\lambda \in \Loc_{T_c}(\Gr_G; k)$ denote the pushforward of the constant sheaf along the inclusion $G/P_\lambda \hookrightarrow \Gr_G$. We then need to show that there is an isomorphism $\cS_k(V_\lambda) \cong \cf_\lambda^\gr$.
    
    Since $V_\lambda$ is an $\ld{G}$-representation, the tensor product $V_\lambda \otimes_\QQ \co_{\fr{t}}$ is a comodule over $\co_{\ld{G} \times \fr{t}}$. In particular, it is a comodule over $\co_{\tilde{\ld{J}}}$ via the closed immersion $\tilde{\ld{J}} \hookrightarrow \ld{G} \times \fr{t}$. It follows from \cref{cor: reg locus ordinary ABG} that we need to show that $V_\lambda \otimes_\QQ \co_{\fr{t}}$ is isomorphic to $\pi_0 \cf_T(G/P_\lambda)$ as $\pi_0 \cf_T(\Gr_G)^\vee \cong \co_{\tilde{\ld{J}}}$-comodules.

    Let $\fr{t}^\gen$ denote the complement of $\bigcup_{\alpha} \fr{t}_\alpha$ as $\alpha$ ranges over the roots of $\ld{G}$, and $\fr{t}_\alpha$ denotes the hyperplane cut out by $\alpha$. Since $V_\lambda \otimes_\QQ \co_{\fr{t}}$, $\pi_0 \cf_T(G/P_\lambda)$, and $\pi_0 \cf_T(\Gr_G)^\vee$ are all flat over $\fr{t}$, it suffices to prove that there is an isomorphism $V_\lambda \otimes_\QQ \co_{\fr{t}} \cong \pi_0 \cf_T(G/P_\lambda)$ of quasicoherent sheaves over $\fr{t}$, and further show that they are isomorphic as $\pi_0 \cf_T(\Gr_G)^\vee \cong \co_{\tilde{\ld{J}}}$-comodules when restricted to $\fr{t}^\gen$.
    
    Let $W_\lambda$ denote the Weyl group of $P_\lambda$, so that $W_\lambda$ is the stabilizer of the weight $\lambda$. 
    %Then $\pi_0 \cf_T(G/P_\lambda) \cong \H^\ast_T(G/P_\lambda; \QQ)$ is isomorphic to $\co_{\fr{t} \times_{\fr{t}\mmod W} \fr{t}\mmod W_\lambda}$. It follows from the Chevalley-Shephard-Todd theorem that 
    Since $G/P_\lambda$ has even cells, $\pi_0 \cf_T(G/P_\lambda)$ is a vector bundle over $\co_{\fr{t}}$, and its rank can be determined by its restriction to $\fr{t}^\gen$. By Atiyah-Bott localization, $\pi_0 \cf_T(G/P_\lambda)|_{\fr{t}^\gen} \cong \pi_0 \cf_T((G/P_\lambda)^T)|_{\fr{t}^\gen}$; but $(G/P_\lambda)^T = W/W_\lambda$, so we conclude that
    $\pi_0 \cf_T(G/P_\lambda)$ is a free $\co_{\fr{t}}$-module of rank $|W/W_\lambda|$. Since $\lambda$ is minuscule, there is an isomorphism $V_\lambda \cong \Map(W/W_\lambda, \QQ)$ (see, e.g., \cite[Proposition 5.1]{gross-minuscule}). We therefore conclude that there is an isomorphism $V_\lambda \otimes_\QQ \co_{\fr{t}} \cong \pi_0 \cf_T(G/P_\lambda)$ of quasicoherent sheaves over $\fr{t}$.
    
    To see that they are isomorphic as $\pi_0 \cf_T(\Gr_G)^\vee \cong \co_{\tilde{\ld{J}}}$-comodules when restricted to $\fr{t}^\gen$, note that $\pi_0 \cf_T(\Gr_G)^\vee|_{\fr{t}^\gen} \cong \co_{\ld{T} \times \fr{t}^\gen}$. We therefore need to check that the weights of $\ld{T}$ acting on $V_\lambda$ and $\pi_0 \cf_T(G/P_\lambda)|_{\fr{t}^\gen}$ agree. The $T$-fixed points of the map $G/P_\lambda \to \Gr_G$ is given by the map $W/W_\lambda \to \Gr_T \cong \bX_\ast(T)$ which is the inclusion of the $W$-orbit of $\lambda$; these are the weights of $\ld{T}$ acting on $\pi_0 \cf_T(G/P_\lambda)|_{\fr{t}^\gen}$. The desired isomorphism of $\ld{T}$-representations between $V_\lambda$ and $\pi_0 \cf_T(G/P_\lambda)|_{\fr{t}^\gen}$ now follows from the fact that the weights of $\ld{T}$ on $V_\lambda$ are also precisely the elements in the $W$-orbit of $\lambda$. 

    Suppose that $\lambda_\bull$ has more than one element. Since the equivalence of \cref{cor: reg locus ordinary ABG} is symmetric monoidal, we find that $\cS_k(\bigotimes_i V_{\lambda_i})$ is equivalent to the convolution tensor product $\cf_{\lambda_1}^\gr \star \cdots \star \cf_{\lambda_n}^\gr$. We therefore need to show that there is an isomorphism $\cf_{\lambda_1}^\gr \star \cdots \star \cf_{\lambda_n}^\gr \cong \cf_{\lambda_\bull}^\gr$. For this, note that the following diagram of homotopy types commutes:
    $$\xymatrix{
    \ol{\Gr_G^{\lambda_\bull}} \ar[r] \ar[d] \ar[dr]^-q & \ol{\Gr_G^{|\lambda|}} \ar[d] \\
    \Gr_G^{\times n} \ar[r] & \Gr_G;
    }$$
    here, the bottom horizontal map is the $\E{2}$-multiplication on $\Gr_G \simeq \Omega G_c$. This implies that $\cf_{\lambda_\bull}^\gr$ is isomorphic to the pushforward of $(\cf_{\lambda_1} \boxtimes \cdots \boxtimes \cf_{\lambda_n})^\gr \cong \cf_{\lambda_1}^\gr \boxtimes \cdots \boxtimes \cf_{\lambda_n}^\gr$ along the map $\Gr_G^{\times n} \to \Gr_G$; but this is precisely the definition of $\cf_{\lambda_1}^\gr \star \cdots \star \cf_{\lambda_n}^\gr$, as desired.
\end{proof}

To convince homotopy theorists that the flag varieties appearing in \cref{prop: ordinary realizing minuscule reps} are in fact (relatively) familiar objects, we have recorded the list of such $G/P_\lambda$ for dominant minuscule $\lambda$ (even in the non-simply-laced cases) in \cref{table: minuscule varieties}.\footnote{Those homotopy theorists who have reached this far in the article may not need this table to be convinced!}
%aka cominuscule homogeneous varieties
\begin{table}[]
\hspace*{-.5cm}
\begin{tabular}{l|l|l|l|l}
$G$ & {$G/P_\lambda$} & {$\ld{G} \act V_\lambda$} & {$\dim_\cc(G/P_\lambda)$} & {$|W/W_\lambda|$} \\ \hline
$A_n$ & $\Gr_j(\cc^{n+1})$, $1\leq j \leq n$ & $\wedge^j \std_{n+1}$ & $j(n+1-j)$ & $\binom{n+1}{j}$ \\
$B_n$ & Smooth quadric in $\CP^n$ & $\std_{2n}$ & $2n-1$ & $2n$ \\
$C_n$ & Lagrangian Grassmannian $\mathrm{LGr}_n(\cc^{2n})$ & Spin & $\binom{n+1}{2}$ & $2^n$ \\
$D_n$ & Smooth quadric in $\CP^{n-1}$ & $\std_{2n}$ & $2n-2$ & $2n$ \\
$D_n$ & Orthogonal Grassmannian $\mathrm{OGr}_n(\cc^{2n})$ & Half-spin (both) & $\binom{n}{2}$ & $2^{n-1}$ \\
$E_6$ & $\mathrm{EIII} \simeq (E_6)_c/\Spin(10) \cdot \U(1)$ & $\std_{27}, \std_{27}^\ast$ & $16$ & $27$ \\
$E_7$ & $\mathrm{EVII} \simeq (E_7)_c/(E_6)_c \cdot \U(1)$ & $\std_{56}$ & $27$ & $56$
\end{tabular}
\vspace{1cm}
\caption{Minuscule homogeneous varieties for $G$ of adjoint type. Here, $(E_6)_c$ and $(E_7)_c$ denote the compact forms of $E_6$ and $E_7$, respectively. The labelings $\mathrm{EIII}$ and $\mathrm{EVII}$ denote the labelings of these symmetric spaces in \'E. Cartan's classification. In the example of $D_n$ acting on the orthogonal Grassmannian, there are two realizations as a homogeneous variety, which correspond to the two half-spin representations: namely, $\SO_{2n}/P_{\alpha_{n-1}} \cong \SO_{2n}/P_{\alpha_n}$. Note, also, that $|W/W_\lambda|$ is equal to the dimension of $V_\lambda$ and also to the number of cells in a minimal ($T$-equivariant) cell structure on $G/P_\lambda$, while $\dim_\cc(G/P_\lambda)$ is the highest weight of the restriction of $\ld{G} \to \GL(V_\lambda)$ to the principal $\SL_2$ inside $\ld{G}$.}
\label{table: minuscule varieties}
\end{table}
\begin{remark}\label{rmk: ordinary action on minuscule}
    Let $\lambda$ be a dominant minuscule weight of $\ld{G}$. The coaction of $\pi_0 \cf_T(\Gr_G)^\vee \cong \H^T_\ast(\Gr_G; \QQ)$ on $\pi_0 \cf_T(G/P_\lambda) \cong \H^\ast_T(G/P_\lambda; \QQ)$ defines a homomorphism 
    \begin{equation}\label{eq: map from tilde J to cohomology of minuscule flag}
        \spec \pi_0 \cf_T(\Gr_G)^\vee \to \GL(\H^\ast_T(G/P_\lambda; \QQ))
    \end{equation}
    of group schemes over $\fr{t}$, where $\GL(\H^\ast_T(G/P_\lambda; \QQ))$ denotes the group scheme of $\co_{\fr{t}}$-linear automorphisms of the vector bundle $\H^\ast_T(G/P_\lambda; \QQ)$. By \cref{thm: ordinary hmlgy reg centr} and \cref{prop: ordinary realizing minuscule reps}, this homomorphism factors as the composite
    \begin{equation}\label{eq: factorization action on minuscule}
        \tilde{\ld{J}} \to \ld{G} \times \fr{t} \to \GL(V_\lambda) \times \fr{t},
    \end{equation}
    where the second map describes the $\ld{G}$-action on $V_\lambda$. Similarly, the coaction of $\pi_0 \cf_G(\Gr_G)^\vee \cong \H^G_\ast(\Gr_G; \QQ)$ on $\pi_0 \cf_G(G/P_\lambda) \cong \H^\ast_{P_\lambda}(\ast; \QQ)$ defines a homomorphism 
    \begin{equation}\label{eq: map from J to G-equiv cohomology of minuscule flag}
        \spec \pi_0 \cf_G(\Gr_G)^\vee \to \GL(\H^\ast_{P_\lambda}(\ast; \QQ))
    \end{equation}
    of group schemes over $\spec \H^\ast_G(\ast; \QQ) \cong \fr{t}\mmod W$. As an $\co_{\fr{t}\mmod W}$-module, $\H^\ast_{P_\lambda}(\ast; \QQ)$ is isomorphic to $\co_{\fr{t}\mmod W} \otimes V_\lambda$, and \cref{eq: map from J to G-equiv cohomology of minuscule flag} factors as the composite
    \begin{equation}\label{eq: factorization G-action on minuscule}
        {\ld{J}} \to \ld{G} \times \fr{t}\mmod W \to \GL(V_\lambda) \times \fr{t}\mmod W.
    \end{equation}
    In fact, all of these maps already exist \textit{integrally} (i.e., using cohomology with integral coefficients, and working with group schemes over $\Z$). 

    For instance, suppose $G = \SO_{2n}$ is of type $D_n$, and let us take coefficients in $\Z' = \Z[1/2]$; otherwise, the cohomology of $BG$ is not isomorphic to $\co_{\fr{t}\mmod W}$. (See \cite[Example 3.2.14]{ku-rel-langlands} for other classical types.) Then the Levi quotient of $P_\lambda$ is $\SO_2 \times \SO_{2n-2}$, so that $\H^\ast_{P_\lambda}(\ast; \Z') \cong \Z'[x, p_1', \cdots, p_{n-2}', c'_{n-1}]$ with $x$ in weight $2$, $p_i'$ in weight $-4i$, and $c'_i$ in weight $-2i$. A simple calculation with symmetric polynomials shows that as an algebra over $\H^\ast_G(\ast; \Z') \cong \Z'[p_1, \cdots, p_{n-1}, c_n]$, there is an isomorphism
    $$\H^\ast_{P_\lambda}(\ast; \Z') \cong \H^\ast_G(\ast; \Z')[x, c'_{n-1}]/(xc'_{n-1} = c_n, \ x^{2n-2} - p_1 x^{2n-4} - \cdots - p_{n-1} + {c'_{n-1}}^2).$$
    As an $\H^\ast_G(\ast; \Z')$-module, this is indeed isomorphic to $\H^\ast_G(\ast; \Z') \otimes \std_{2n}$. 
    Building on \cref{ex: equiv homology Omega SUn and Witt} shows that as a group scheme over $\H^\ast_G(\ast; \Z')$, the functor of points of $\ld{J}$ sends an $\H^\ast_G(\ast; \Z')$-algebra $R$ to the subgroup of those units $f(x, c'_{n-1}) \in (\H^\ast_{P_\lambda}(\ast; \Z') \otimes_{\H^\ast_G(\ast; \Z')} R)^\times$ such that $f(x, c'_{n-1})^{-1} = f(-x,-c'_{n-1})$. The action of $\ld{J}$ on $\H^\ast_{P_\lambda}(\ast; \Z')$ preserves the symmetric bilinear form given by
    \begin{align*}
        \H^\ast_{P_\lambda}(\ast; \Z') \otimes_{\H^\ast_G(\ast; \Z')} \H^\ast_{P_\lambda}(\ast; \Z') & \xrightarrow{\pdb{-,-}} \H^\ast_G(\ast; \Z') \\
        f,g & \mapsto \text{coefficient of }x^{2n-2}\text{ in }f(x, c'_{n-1}) g(-x, -c'_{n-1}).
    \end{align*}
    Geometrically, this bilinear form comes from $G$-equivariant Poincar\'e duality on $G/P_\lambda$, twisted by the natural action of $\Z/2$ on $G/P_\lambda$. (This $\Z/2$ acts on $\H^\ast_{P_\lambda}(\ast; \Z')$ by sending $x\mapsto -x$ and $c'_{n-1} \mapsto -c'_{n-1}$.)
    The bilinear form $\pdb{-,-}$ on $\H^\ast_{P_\lambda}(\ast; \Z')$ gives the desired factorization \cref{eq: factorization G-action on minuscule} of the map ${\ld{J}} \to \GL_{2n} \times \fr{t}\mmod W$ through the inclusion $\SO_{2n} \times \fr{t}\mmod W \hookrightarrow \GL_{2n} \times \fr{t}\mmod W$.
\end{remark}

As we have seen, the calculation of \cref{thm: ordinary hmlgy reg centr} is quite powerful. Here is another simple application, motivated by \cite{ginzburg-kazhdan} and \cite{ginzburg-riche}; see also \cite[Example 3.6.13]{ku-rel-langlands}, where the same example is presented.
\begin{prop}[Gelfand-Graev action]\label{prop: ordinary gelfand-graev}
    The natural action of $\ld{G} \times \ld{T}$ on the affine closure $\ol{T^\ast(\ld{G}/\ld{N})}$ extends to an action of $\ld{G} \times (W \rtimes \ld{T})$, where $W$ is the Weyl group.
\end{prop}
\begin{proof}
    Let $T^\ast(\ld{G}/\ld{N})_\reg = \ld{G} \times^{\ld{N}} \ld{\fr{n}}^\perp_\reg$ denote the regular locus in $T^\ast(\ld{G}/\ld{N})$; then $T^\ast(\ld{G}/\ld{N})_\reg \subseteq T^\ast(\ld{G}/\ld{N})$ is open, with complement of codimension $2$, so that $\ol{T^\ast(\ld{G}/\ld{N})} \cong \ol{T^\ast(\ld{G}/\ld{N})_\reg}$. Note that there is an isomorphism
    $$\ld{G} \backslash T^\ast(\ld{G}/\ld{N})_\reg/\ld{T} \cong \ld{\fr{n}}^\perp_\reg/\ld{B},$$
    so (the proof of) \cref{cor: reg locus ordinary ABG} gives isomorphisms
    \begin{equation}\label{eq: reg locus in T*G/N}
        T^\ast(\ld{G}/\ld{N})_\reg \cong (\ld{G} \times \ld{T}) \times^{\ld{B}} \ld{\fr{n}}^\perp_\reg \cong (\ld{G} \times \ld{T} \times (\psi + \ld{\fr{t}}^\ast))/\tilde{\ld{J}}.
    \end{equation}
    There is a canonical $W$-action on $\ld{G} \times \ld{T} \times (\psi + \ld{\fr{t}}^\ast)$, given by the natural $W$-actions on $\ld{T}$ and on $\psi + \ld{\fr{t}}^\ast \cong \ld{\fr{t}}^\ast$. Similarly, $\tilde{\ld{J}}$ also admits a natural $W$-action; it is given via \cref{thm: ordinary hmlgy reg centr} by the natural $W$-action on $\H^{T_c}_\ast(\Gr_G; \QQ)$. Moreover, the closed immersion
    $$\tilde{\ld{J}} \to \ld{G} \times \ld{T} \times (\psi + \ld{\fr{t}}^\ast)$$
    is $W$-equivariant (indeed, the map $\tilde{\ld{J}} \to \ld{T} \times (\psi + \ld{\fr{t}}^\ast)$ is induced by the inclusion $\H^{T_c}_\ast(\Gr_T; \QQ) \to \H^{T_c}_\ast(\Gr_G; \QQ)$ on equivariant homology). This implies that the quotient of \cref{eq: reg locus in T*G/N} admits a $W$-action, which defines a $W$-action on the affine closure of $T^\ast(\ld{G}/\ld{N})_\reg$ as desired.
\end{proof}
Note that we assumed in \cref{cor: reg locus ordinary ABG} that $G$ is simply-laced; but this is not necessary, because we know (by the discussion in \cref{sec: regular locus}) that the main result of \cite{abg-iwahori-satake} implies \cref{cor: reg locus ordinary ABG} is true for any connected reductive $G$. Alternatively, one can observe that the proof of \cref{cor: reg locus ordinary ABG} itself never seriously appeals to $G$ being simply-laced. 
\begin{remark}
    The proof of \cref{prop: ordinary gelfand-graev} generalizes to show that if $\ld{P} \subseteq \ld{G}$ is a parabolic subgroup with Levi quotient $\ld{L}$ and unipotent radical $U_{\ld{P}}$, then the natural action of $\ld{G} \times \ld{L}$ on the affine closure $\ol{T^\ast(\ld{G}/U_{\ld{P}})}$ extends to an action of $\ld{G} \times (W_L \rtimes \ld{L})$, where $W_L = N_{\ld{G}}(\ld{L})/\ld{L}$ is the Weyl group. (Also see \cite{affine-closure-G-UP}.)
\end{remark}

The $W$-action of \cref{prop: ordinary gelfand-graev} is known as the (semiclassical) \textit{Gelfand-Graev action}. The moment map $\ol{T^\ast(\ld{G}/\ld{N})} \to \ld{\g}^\ast$ is $W$-equivariant for the trivial action on the target. There is a commutative diagram
$$\xymatrix{
\tilde{\ld{\g}} \ar@{^(->}[r] \ar[dr] & \ol{T^\ast(\ld{G}/\ld{N})}/\ld{T} \ar[d] \\
& \ld{\g}^\ast
}$$
which relates $\ol{T^\ast(\ld{G}/\ld{N})}$ to the Grothendieck-Springer resolution; and via this diagram, the Gelfand-Graev action is closely related to the Weyl action in Springer theory.
\begin{example}\label{rmk: Z/2 symplectic fourier}
    When $\ld{G} = \SL_2$, the affine closure $\ol{T^\ast(\ld{G}/\ld{N})}$ is simply $T^\ast(\AA^2)$, and the $W = \Z/2$-action on it is given by the symplectic Fourier transform. To see this, let $\ld{J}_X$ denote the kernel of the homomorphism $\tilde{\ld{J}} \to\ld{T} \times (\psi + \ld{\fr{t}}^\ast)$ of group schemes over $\psi + \ld{\fr{t}}^\ast$. (This follows the notation from \cite{ku-rel-langlands}.) Then \cref{eq: reg locus in T*G/N} gives an isomorphism
    $$(\ld{G} \times (\psi + \ld{\fr{t}}^\ast))/\ld{J}_X \xrightarrow{\cong} T^\ast(\ld{G}/\ld{N})_\reg.$$
    In the case at hand, $\psi + \ld{\fr{t}}^\ast \cong \AA^1$ with coordinate $x$, and the group scheme $\ld{J}_X$ is just $\spec \Z[x,b]/bx$ (where the group law sends $b\mapsto b \otimes 1 + 1 \otimes b$). The above isomorphism defines a map
    $$q: \SL_2 \times \AA^1 \to \ol{T^\ast(\ld{G}/\ld{N})} = T^\ast(\AA^2),$$
    and the affine closure of the image is all of $T^\ast(\AA^2)$. The map $q$ can be explicitly described as follows. View a point of $T^\ast(\AA^2)$ as a pair $\left(\begin{psmallmatrix}
        u_1 \\
        u_2
    \end{psmallmatrix}, (v_1, v_2)\right)$ of a vector and a covector. Then $q$ is the natural extension to $\SL_2 \times \AA^1$ of the map $\kappa: \AA^1 \to T^\ast(\AA^2)$ which sends $x\mapsto \left(\begin{psmallmatrix}
        1 \\
        0
    \end{psmallmatrix}, (x,0)\right)$. In other words, $q$ sends
    $$(g, x) = \left(\begin{psmallmatrix}
        a & b \\
        c & d
    \end{psmallmatrix}, x\right) \mapsto \left(g \begin{psmallmatrix}
        1 \\
        0
    \end{psmallmatrix}, (g^T)^{-1} (x,0)\right) = \left(\begin{psmallmatrix}
        a \\
        c
    \end{psmallmatrix}, (dx, -bx)\right).$$ 
    Of course, one could also swap the roles of $\AA^2$ and $(\AA^2)^\ast$ in $T^\ast(\AA^2)$; the map $\kappa$ would then send $x\mapsto \left(\begin{psmallmatrix}
        0 \\
        x
    \end{psmallmatrix}, (0,1)\right)$, and $q$ would send
    $$(g, x) = \left(\begin{psmallmatrix}
        a & b \\
        c & d
    \end{psmallmatrix}, x\right) \mapsto \left(\begin{psmallmatrix}
        0 \\
        x
    \end{psmallmatrix} \cdot g^T, (0,1) \cdot g^{-1}\right) = \left(\begin{psmallmatrix}
        bx \\
        dx
    \end{psmallmatrix}, (-c, a)\right).$$ 
    If we compose with the involution sending $x\mapsto -x$, the resulting involution
    $$\left(\begin{psmallmatrix}
        a \\
        c
    \end{psmallmatrix}, (dx, -bx)\right) \mapsto \left(\begin{psmallmatrix}
        -bx \\
        -dx
    \end{psmallmatrix}, (-c, a)\right).$$
    This, of course, is precisely the symplectic Fourier transform, which sends
    $$\left(\begin{psmallmatrix}
        u_1 \\
        u_2
    \end{psmallmatrix}, (v_1, v_2)\right) \mapsto \left(\begin{psmallmatrix}
        v_2 \\
        -v_1
    \end{psmallmatrix}, (-u_2, u_1)\right).$$
\end{example}
We also have the following, which is an obvious consequence of \cref{cor: reg locus satake} and \cref{cor: reg locus abg} (but can be reproved using \cref{cor: reg locus ordinary ABG}).
\begin{prop}\label{prop: ordinary full faithful on gr loc}
    Let $\Loc_{T_c}^\gr(\Gr_G; k)^\heart$ denote the heart of the $t$-structure on $\Loc_{T_c}^\gr(\Gr_G; k) = \coMod_{\pi_0(\cf_T(\Gr_G))^\vee}(\QCoh(\fr{t}))$ coming from the standard (homological truncation) $t$-structure on $\QCoh(\fr{t})$. 
    Then, the composite functor
    $$\Loc_{T_c}^\gr(\Gr_G; k) \simeq \QCoh(\tilde{\ld{\g}}^\reg/\ld{G}) \to \QCoh(\ld{G}\backslash \ol{T^\ast(\ld{G}/\ld{N})}/\ld{T})$$
    is $t$-exact, and on hearts, it restricts to a fully faithful functor on the essential image of the functor \cref{eq: Rep T^ x G^ to ordinary loc}. Furthermore, this functor is $W$-equivariant for the natural action of $W = \N_{G_c}(T_c)/T_c$ on the left-hand side and the Gelfand-Graev action of \cref{prop: ordinary gelfand-graev} on the right-hand side.

    Similarly, let $\Loc_{G_c}^\gr(\Gr_G; k)^\heart$ denote the heart of the $t$-structure on $\Loc_{G_c}^\gr(\Gr_G; k) = \coMod_{\pi_0(\cf_G(\Gr_G))^\vee}(\QCoh(\fr{t}\mmod W))$ coming from the standard (homological truncation) $t$-structure on $\QCoh(\fr{t}\mmod W)$. 
    %If $V \in \Rep(\ld{G})$, the object $\cS_k(V)$ lies in $\Loc_{G_c}^\gr(\Gr_G; k)^\heart$, and there is an isomorphism
    %$$\Map_{\QCoh(\ld{\g}^\ast/\ld{G})^\heart}(V \otimes_\QQ \co_{\ld{\g}^\ast}, W \otimes_\QQ \co_{\ld{\g}^\ast}) \xrightarrow{\cong} \Map_{\Loc_{G_c}^\gr(\Gr_G; k)^\heart}(\cS_k(V), \cS_k(W))$$
    %of $\QQ$-vector spaces for any two representations $V,W \in \Rep(\ld{G})$. In other words, 
    Then, the composite functor
    $$\Loc_{G_c}^\gr(\Gr_G; k) \simeq \QCoh(\ld{\g}^{\ast,\reg}/\ld{G}) \to \QCoh(\ld{\g}^\ast/\ld{G})$$
    is $t$-exact, and on hearts, it restricts to a fully faithful functor on the essential image of the functor $\Rep(\ld{G}) \to \Loc_{G_c}^\gr(\Gr_G; k)$ (analogous to \cref{eq: Rep G^ to ordinary loc}).
\end{prop}
\begin{proof}
    If $V \in \Rep(\ld{G})$, the object $\cS_k(V)$ lies in $\Loc_{T_c}^\gr(\Gr_G; k)^\heart$, and we need to show that there is an isomorphism
    \begin{multline*}
        \Map_{\QCoh(\ld{G}\backslash \ol{T^\ast(\ld{G}/\ld{N})}/\ld{T})^\heart}(V_1 \otimes_\QQ \co_{\ol{T^\ast(\ld{G}/\ld{N})}}, V_2 \otimes_\QQ \co_{\ol{T^\ast(\ld{G}/\ld{N})}}) \\
        \xrightarrow{\cong} \Map_{\Loc_{T_c}^\gr(\Gr_G; k)^\heart}(\cS_k(V_1), \cS_k(W_1))
    \end{multline*}
    of $\QQ$-vector spaces for any two representations $V_1,V_2 \in \Rep(\ld{G})$. In other words, 
    By \cref{cor: reg locus ordinary ABG}, there is an isomorphism
    $$\Map_{\Loc_{T_c}^\gr(\Gr_G; k)^\heart}(\cS_k(V), \cS_k(W)) \cong \Map_{\QCoh(\tilde{\ld{\g}}^\reg/\ld{G})^\heart}(V \otimes_\QQ \co_{\tilde{\ld{\g}}^\reg}, W \otimes_\QQ \co_{\tilde{\ld{\g}}^\reg}).$$
    Since $\tilde{\ld{\g}}^\reg/\ld{G} \hookrightarrow \ld{G} \backslash \ol{T^\ast(\ld{G}/\ld{N})}/\ld{T}$ has complement of codimension $2$ and $\ol{T^\ast(\ld{G}/\ld{N})}$ is affine and normal, the algebraic Hartogs lemma implies that the restriction map
    \begin{multline*}
        \Map_{\QCoh(\ld{G} \backslash \ol{T^\ast(\ld{G}/\ld{N})}/\ld{T})^\heart}(V \otimes_\QQ \co_{\ol{T^\ast(\ld{G}/\ld{N})}}, W \otimes_\QQ \co_{\ol{T^\ast(\ld{G}/\ld{N})}}) \\
        \to \Map_{\QCoh(\tilde{\ld{\g}}^\reg/\ld{G})^\heart}(V \otimes_\QQ \co_{\tilde{\ld{\g}}^\reg}, W \otimes_\QQ \co_{\tilde{\ld{\g}}^\reg})
    \end{multline*}
    is an isomorphism, as desired. (This is where it is crucial that we work at the level of abelian categories.) The same argument works for $\Loc_{G_c}^\gr(\Gr_G; k)$; in this case, $\ld{\g}^{\ast,\reg} \hookrightarrow \ld{\g}^\ast$ even has complement of codimension $3$.
\end{proof}
\cref{prop: ordinary full faithful on gr loc} gives an analogue of \cite[Theorem 4]{bf-derived-satake}: namely, if $\QCoh_\free(\ld{\g}^\ast/\ld{G})$ denotes the essential image of the pullback functor $\Rep(\ld{G}) \to \QCoh(\ld{\g}^\ast/\ld{G})$, then there is a fully faithful embedding 
$$\QCoh_\free(\ld{\g}^\ast/\ld{G})^\heartsuit \hookrightarrow \Loc_{G_c}^\gr(\Gr_G; k)^\heartsuit.$$
Similarly, if $\QCoh_\free(\ld{G} \backslash \ol{T^\ast(\ld{G}/\ld{N})}/\ld{T})$ denotes the essential image of the pullback functor $\Rep(\ld{G} \times \ld{T}) \to \QCoh(\ld{G} \backslash \ol{T^\ast(\ld{G}/\ld{N})}/\ld{T})$, then there is a fully faithful embedding 
$$\QCoh_\free(\ld{G} \backslash \ol{T^\ast(\ld{G}/\ld{N})}/\ld{T})^\heartsuit \hookrightarrow \Loc_{T_c}^\gr(\Gr_G; k)^\heartsuit.$$

Let $\Rep_\min(\ld{G})$ denote the idempotent completion of the subcategory of $\Rep(\ld{G})$ spanned by tensor products of irreducible $\ld{G}$-representations with minuscule highest weights. In general, if $\ld{G}$ is simple (but not necessarily simply-laced) and not of types $G_2$, $F_4$, or $E_8$, then any representation is a summand of a tensor product of irreducible $\ld{G}$-representations with minuscule highest weights, so that $\Rep_\min(\ld{G}) \simeq \Rep(\ld{G})$.\footnote{When $\ld{G}$ is of types $G_2$, $F_4$, or $E_8$, there are no minuscule weights at all. In general, $\Rep(\ld{G})$ is the idempotent completion of its full subcategory spanned by tensor products of irreducible $\ld{G}$-representations with minuscule and {quasi}-minuscule highest weights.}
\begin{corollary}\label{cor: ordinary minuscule equivalence}
    Let $\QCoh_{\free}(\ld{\g}^\ast/\ld{G})^{\min,\heartsuit}$ denote the essential image of $\Rep_\min(\ld{G})$ under the pullback functor $\Rep(\ld{G})^\heartsuit \to \QCoh(\ld{\g}^\ast/\ld{G})^\heartsuit$ (so it is the entirety of $\QCoh(\ld{\g}^\ast/\ld{G})^\heartsuit$ if $\ld{G}$ is not of type $E_8$). Similarly, let $\Loc_{G_c}^\gr(\Gr_G; k)^{\min,\heartsuit}$ denote the idempotent completion of the subcategory of $\Loc_{G_c}^\gr(\Gr_G; k)^\heartsuit$ spanned by $\cf_{\lambda_\bull}^\gr$ ranging over sequences $\lambda_\bull$ of minuscule highest weights. Then there is an equivalence
    $$\QCoh_\free(\ld{\g}^\ast/\ld{G})^{\min,\heartsuit} \simeq \Loc_{G_c}^\gr(\Gr_G; k)^{\min,\heartsuit}.$$
\end{corollary}
There is a similar equivalence 
$$\Loc_{T_c}^\gr(\Gr_G; k)^{\min,\heartsuit} \simeq \QCoh_\free(\ld{G} \backslash \ol{T^\ast(\ld{G}/\ld{N})}/\ld{T})^{\min,\heartsuit},$$
where these categories are defined analogously by idempotent completion. 

Note that the category $\Loc_{G_c}^\gr(\Gr_G; k)^{\min,\heartsuit}$ is the heart of a degeneration, in the sense of \cref{sec: degenerations}, of the similarly-defined category $\Loc_{G_c}(\Gr_G; k)^\min$. (In particular, \cref{cor: ordinary minuscule equivalence} gives an equivalence between the purely algebraically defined category $\QCoh_\free(\ld{\g}^\ast/\ld{G})^{\min,\heartsuit}$ and a degeneration of the purely topologically defined category $\Loc_{G_c}(\Gr_G; k)^\min$.) If $\lambda_\bull$ and $\mu_\bull$ are two sequences of dominant minuscule weights of $\ld{G}$, there is an equivalence of $k$-modules
\begin{align*}
    \Map_{\Loc_{G_c}(\Gr_G; k)^\min}(\cf_{\lambda_\bull}, \cf_{\mu_\bull}) & \simeq \cf_G(\ol{\Gr_G^{\lambda_\bull}}) \otimes_{\coMod_{\cf_{G}(\Gr_G)^\vee}(\QCoh(\cM_G))} \cf_G(\ol{\Gr_G^{\mu_\bull}}) \\
    & \simeq \cf_{G}(\ol{\Gr_G^{\lambda_\bull}} \times_{\Gr_G} \ol{\Gr_G^{\mu_\bull}}),
\end{align*}
where the final equivalence uses the K\"unneth formula at the level of $k$-cochains. The category $\Loc_{G_c}(\Gr_G; k)^\min$ therefore compares to (the $k$-analogue of) the category from \cite[Section 3.4]{cautis-kamnitzer}.

At first glance, the existence of the $t$-structure on $\Loc_{T_c}^\gr(\Gr_G; k)$ from \cref{prop: ordinary full faithful on gr loc} may perhaps be a bit surprising, since $k$ is a \textit{$2$-periodic} $\Eoo$-ring. In fact, this periodicity prohibits $\Loc_{T_c}(\Gr_G; k)$ itself from having a $t$-structure. However, the $\infty$-category $\Loc_{T_c}^\gr(\Gr_G; k)$ ``flattens'' out the homological periodicity in $\Loc_{T_c}(\Gr_G; k)$ to a weight periodicity, but it is itself also a stable $\infty$-category. In particular, it has a \textit{homological} shift operation, which is distinct from the operation of shifting \textit{weights} (just as with the usual category of mixed sheaves). (The $2$-periodicity of $k$ implies that the weight-shifting operation on $\Loc_{T_c}^\gr(\Gr_G; k)$ is an equivalence, which is why we do not see gradings/weights when discussing $\Loc_{T_c}^\gr(\Gr_G; k)$; but the weight-shifting operation will be nontrivial on, say, $\Loc_{T_c}^\gr(\Gr_G; \QQ)$.) The resulting homological shift on $\Loc_{T_c}^\gr(\Gr_G; k)$ is no longer periodic, and it is therefore reasonable to equip this $\infty$-category with a $t$-structure. (This $t$-structure is unrelated to the perverse $t$-structure on $\Shv_I(\Gr_G; \QQ)$.)

We will now discuss a \textit{deformation quantization} of \cref{cor: reg locus ordinary ABG} by adding loop-rotation equivariance. Write $\tilde{T} = T \times \GG_m^\rot$ to denote the corresponding affine torus. In the case when $G$ is a torus, we have already discussed this in \cref{sec: torus loop rot}. For more general $G$, this turns out to be a bit tricky: while $\H^{T_c}_\ast(\Gr_G; \QQ)$ is a bicommutative Hopf algebra\footnote{To be more precise, the $\E{2}$-space structure on $\Gr_G$ equips $C^{T_c}_\ast(\Gr_G; \QQ)$ with the structure of an $\E{2}$-algebra in $\Eoo$-coalgebras over $C_{T_c}^\ast(\ast; \QQ)$.}, the loop-rotation equivariant homology $\H^{\tilde{T}_c}_\ast(\Gr_G; \QQ)$ is only a cocommutative coalgebroid over $\H^\ast_{\tilde{T}_c}(\ast; \QQ)$. That is, it does not admit an algebra structure. While this is not a mathematical issue, it does make the task of explicitly understanding $\H^{\tilde{T}_c}_\ast(\Gr_G; \QQ)$ in a satisfactory way more complicated. 
Instead, it turns out to be easier to describe $\H^{\tilde{T}_c}_\ast(\Fl_G; \QQ)$, where $\Fl_G$ is the \textit{affine flag variety}, defined as the quotient $G\ls{t}/I$ for the Iwahori subgroup $I \subseteq G\pw{t}$ associated to a Borel subgroup $B \subseteq G$.
To state the result, we need a definition from \cite{ginzburg-kapranov-vasserot-residue-hecke}.
\begin{definition}\label{def: nil-hecke}
    Let $(\Lambda, \Phi, \ld{\Lambda}, \ld{\Phi})$ be a root datum with associated Weyl group $W$ and torus $T = \Hom(\Lambda, \GG_m)$. Let $\Delta$ be a base of simple roots, let $\Phi^+$ denote the corresponding set of positive roots, and let $\Phi'$ denote the subset $W \cdot \Delta \subseteq \Phi$. Let $\bH$ be a $1$-dimensional group scheme (over a commutative ring $R$). As in \cref{def: G-diff ops}, let $\bH_T = \Hom(\Lambda, T)$, and for each character $\lambda \in \Lambda$, let $\bH_{T_\lambda} \hookrightarrow \bH_T$ denote the subgroup corresponding to the subtorus $T_\lambda = \ker(\lambda) \subseteq T$.
    Let $Q(\co_{\bH_T})$ denote the sheaf of functions on the generic point of $\co_{\bH_T}$. The twisted group algebra $Q(\co_{\bH_T})[W]$ is the algebra which is additively given by the tensor product $Q(\co_{\bH_T}) \otimes_F F[W]$, and whose multiplication law is given by 
    $$(f_1 \otimes w_1) \cdot (f_2 \otimes w_2) = (f \cdot w_1 g) \otimes (w_1 w_2).$$
    The algebra $\cH(\bH, T, W)$ is defined to be the subset of $Q(\co_{\bH_T})[W]$ of those elements $\sum_{w \in W} f_w [w]$ such that:
    \begin{itemize}
        \item The poles of $f_x$ all have order $\leq 1$, and these are contained in the divisors $\bH_{T_\alpha}$ for each $\alpha \in \Phi'$.
        \item For each $w \in W$ and $\alpha \in \Phi^+ \cap \Phi'$, we have
        $$\Res_{\bH_{T_\alpha}}(f_w) + \Res_{\bH_{T_\alpha}}(f_{s_\alpha w}) = 0.$$
    \end{itemize}
    In \cite{ginzburg-kapranov-vasserot-residue-hecke}, this algebra is denoted $\tilde{\bH}$.
    It is proved in \cite[Theorem 1.4]{ginzburg-kapranov-vasserot-residue-hecke} that $\cH(\bH, T, W)$ is a \textit{subalgebra} of $Q(\co_{\bH_T})[W]$.
\end{definition}
\begin{remark}
    The pair $(Q(\co_{\bH_T}), Q(\co_{\bH_T})[W])$ admits the structure of a (cocommutative) Hopf algebroid; we will abusively say that $Q(\co_{\bH_T})[W]$ admits the structure of a Hopf $Q(\co_{\bH_T})$-algebroid. The coproduct comes from the diagonal on $W$; the left unit comes from the inclusion $Q(\co_{\bH_T}) \subseteq Q(\co_{\bH_T})[W]$; and the right unit comes from the action of $W$ on $\bH_T$ (which defines a coaction of $W$ on $\co_{\bH_T}$ that extends to a coaction on $Q(\co_{\bH_T})$). The resulting Hopf $\co_{\bH_T}$-algebroid structure on $Q(\co_{\bH_T})[W]$ restricts to $\cH(\bH, T, W)$, so that $\cH(\bH, T, W)$ admits the structure of a (cocommutative) Hopf $\co_{\bH_T}$-algebroid. (See \cite[Theorem 4.11]{hopf-algebroid-nil-hecke} for the case $\bH = \GG_a$.)
    
    When $W$ is finite, \cite[Proposition 2.3]{ginzburg-kapranov-vasserot-residue-hecke} states that {upon rationalization}, the action of $\cH(\bH, T, W)$ on $\co_{\bH_T}$ gives an isomorphism between $\cH(\bH, T, W)$ and $\End_{\co_{\bH_T}^W}(\co_{\bH_T})$. This gives a Morita equivalence between the category of $\cH(\bH, T, W)$-modules and the category of $\co_{\bH_T}^W$-modules. Under this equivalence, the symmetric monoidal structure on the category of $\cH(\bH, T, W)$-modules from the cocommutative Hopf algebroid structure on $\cH(\bH, T, W)$ identifies with the standard symmetric monoidal structure on the category of $\co_{\bH_T}^W$-modules.
\end{remark}
If $\Lambda$ denotes the \textit{co}root lattice of ${G}$, let $W^\aff = \Lambda \rtimes W$ denote the corresponding affine Weyl group, and let $\tilde{W} = \bX_\ast({T}) \rtimes W$ denote the extended affine Weyl group.\footnote{The affine Weyl group $W^\aff$ introduced above is very slightly different from the affine Weyl group studied in \cite[Section 7.2]{ginzburg-whittaker} or \cite{gannon-tmmodw}; the affine Weyl group there is the semidirect product $W'^\aff = \ld{\Lambda} \rtimes W$, where $\ld{\Lambda}$ is the \textit{root} lattice of $G$. When $G$ is simply-laced, these are, of course, isomorphic; but they differ otherwise.} %The action of $W^\aff$ on $\bX^\ast(T)$ defines an action on $\fr{t} = \Hom(\bX^\ast(T), \GG_a)$. There is also an action of $W'^\aff$ on $\fr{t}$: if $(-,-)$ denotes the $W$-invariant inner product on $\fr{t}$, $\ld{\alpha}$ is a root of $G$, and $n \in \Z$, the generator $s_{\ld{\alpha}, n}$ acts by reflection along the affine hyperplane $\{x \in \bX^\ast(T) | (x, \ld{\alpha}) = n\}$. Note that if $\alpha$ is the coroot dual to $\ld{\alpha}$, then $(x, \ld{\alpha}) = \frac{(\ld{\alpha}, \ld{\alpha})}{2} \pdb{x, \alpha}$.
For clarity, note that the action of $\tilde{W}$ on $\bX^\ast(T)$ (and hence on $\bH_T$) is given as follows: if $\alpha$ is a coweight of $T$ and $n \in \Z$, the generator $s_{\alpha,n}$ of $W^\aff$ acts on $\bX^\ast(T)$ by reflection along the affine hyperplane $\{x\in \bX^\ast(T) | \pdb{x, \alpha} = n\}$.
The \textit{degenerate nil-Hecke algebra} $\cH(\bH, \tilde{T}, \tilde{W})$ is defined to be $\bX_\ast(T) \ltimes_\Lambda \cH(\bH, \tilde{T}, W^\aff)$. In the following discussion, we will simply write $Q(\co_{\bH_{\tilde{T}}})[\tilde{W}]$ to denote $\bX_\ast({T}) \ltimes_\Lambda Q(\co_{\bH_{\tilde{T}}})[W^\aff]$, so that $\cH(\bH, \tilde{T}, \tilde{W})$ is contained in $Q(\co_{\bH_{\tilde{T}}})[\tilde{W}]$.

There is a natural inclusion $\cd_{\ld{T}}^\bH[W] \hookrightarrow \cH(\bH, \tilde{T}, \tilde{W})$ of (sheaves of) algebras. The following result can be proved exactly as in \cite[Proposition 7.2.4]{ginzburg-whittaker}; one only has to use \cite[Proposition 2.3]{ginzburg-kapranov-vasserot-residue-hecke} in place of \cite[Lemma 7.1.5]{ginzburg-whittaker}, and also observe that the arguments of \cite{lonergan-descent} generalize to the setting of descent along the map $\bH_T/W \to \bH_T \mmod W$.
\begin{prop}\label{prop: ascending to nil-hecke module}
    Let $\cf$ be $\cd_{\ld{T}}^\bH[W]$-module\footnote{Here, we mean a module in the usual, underived, sense of the word; but it is easy to generalize the statement to the setting of perfect $\cd_{\ld{T}}^\bH[W]$-modules by induction on the length of the bounded complex.}. Then the action of $\cd_{\ld{T}}^\bH[W]$ extends (necessarily uniquely) along the inclusion $\cd_{\ld{T}}^\bH[W] \hookrightarrow \cH(\bH, \tilde{T}, \tilde{W})$ if and only if the natural map $\co_{\bH_T} \otimes_{\co_{\bH_T}^W} \cf^W \to \cf$ is an isomorphism.
\end{prop}
\begin{remark}\label{rmk: relationship to t mmod Waff}
    Let us remark on a relationship to \cite{gannon-tmmodw}. Following \textit{loc. cit.}, let $\Gamma_{W^\aff}$ denote the ind-scheme given by the union of graphs of the affine Weyl group $W^\aff$ acting on $\bH_{\tilde{T}}$, and let $\Gamma_{\tilde{W}}$ denote $\tilde{W} \times^{W^\aff} \bH_{\tilde{T}}$. Then there are two projections $\Gamma_{\tilde{W}} \rightrightarrows \bH_{\tilde{T}}$. This can be extended to a simplicial diagram $\Gamma_\bull$ of ind-schemes. Define the stack $\bH_{\tilde{T}}\mmod \tilde{W}$ to be the geometric realization of $\Gamma_\bull$. (For instance, if $W$ is trivial, this is the quotient $\bH_{\tilde{T}}/\bX_\ast(T)$. Similarly, if $\bH = \GG_a$, so that $\bH_{\tilde{T}} \cong \fr{t} \oplus \AA^1_\hbar$, then the specialization of the quotient $\bH_{\tilde{T}}\mmod \tilde{W}$ to $\hbar=1$ agrees with the quotient $\fr{t}\mmod \tilde{W}$ from \cite{gannon-tmmodw}.) In general, there is a map of stacks $\phi: \bH_{\tilde{T}}/\tilde{W} \to \bH_{\tilde{T}} \mmod \tilde{W}$. By arguing exactly as in \cite[Theorem 4.23]{gannon-tmmodw}, one can show that the pullback functor $\phi^!$ is fully faithful; and furthermore, an object of $\IndCoh(\bH_{\tilde{T}}/\tilde{W})$ descends\footnote{That is, it lies in the essential image of the left adjoint $\phi^!$ to $\phi_\ast^\IndCoh$.} along $\phi$ if and only if the corresponding object of $\IndCoh(\bH_T/W)$ descends to $\bH_T\mmod W$. Since \cref{rmk: G-mellin} gives an equivalence between $\IndCoh(\bH_{\tilde{T}}/\tilde{W})$ and $\cd_{\ld{T}}^\bH[W]\modc$, \cref{prop: ascending to nil-hecke module} can be used to obtain an equivalence between $\cH(\bH, \tilde{T}, \tilde{W})\modc$ and $\IndCoh(\bH_{\tilde{T}} \mmod \tilde{W})$.
\end{remark}

\cref{prop: hmlgy gkm} yields the following result due to Kostant and Kumar \cite{kostant-kumar, kostant-kumar-2, kumar-kac-moody}, which (as we will explain momentarily) could also be seen as a consequence of results from \cite{ginzburg-whittaker, lonergan-fourier, bf-derived-satake}.
In the discussion below, $\bH = \GG_a$. Note that $\H^\ast_{\tilde{T}_c}(\ast; \QQ)$ is isomorphic to $\co_{\tilde{\fr{t}}} \cong \co_{\bH_{\tilde{T}}}$. 
\begin{theorem}\label{thm: ordinary loop-rot flag}
    There is an isomorphism of associative $\QQ[\hbar]$-algebras
    \begin{equation}\label{eq: comparison to nil hecke}
        \H^{\tilde{T}_c}_\ast(\Fl_G; \QQ) \cong \cH(\GG_a, \tilde{T}, \tilde{W}).
    \end{equation}
    Here, $\H^{\tilde{T}_c}_\ast(\Fl_G; \QQ)$ is equipped with the associative algebra structure coming from convolution. Moreover, the above isomorphism is also one of (cocommutative) Hopf $\H^\ast_{\tilde{T}_c}(\ast; \QQ) \cong \co_{\bH_{\tilde{T}}}$-algebroids.
\end{theorem}
\begin{proof}
    The affine flag variety $\Fl_G$ is an ind-finite GKM space in the sense of \cref{def: GKM space}, and so we may use \cref{prop: hmlgy gkm} to describe $\H_\ast^{\tilde{T}_c}(\Fl_G; \QQ)$. The GKM graph of $\Fl_G$ has set of vertices given by $\Fl_G^{\tilde{T}_c} = \bX_\ast(T) \rtimes W \cong \tilde{W}$, and an edge $w \to s_{\alpha,n} w$ for each affine reflection $s_{\alpha,n} \in \tilde{W}$. In particular, if $\punc{\tilde{\fr{t}}}$ denotes the complement of the union of affine hyperplanes in $\tilde{\fr{t}}$, then $\H^{\tilde{T}_c}_\ast(\Fl_G; \QQ)$ is a subalgebra of $\H^{\tilde{T}_c}_\ast(\Fl_G^{\tilde{T}_c}; \QQ)|_{\punc{\tilde{\fr{t}}}}$. The latter is isomorphic to $\H^{\tilde{T}_c}_\ast(\tilde{W}; \QQ)|_{\punc{\tilde{\fr{t}}}}$, which in turn can be identified (using \cref{prop: T homology and quantized diffop}, for instance) with a localization of $\cd^\hbar_{\ld{T}}[W]$. This localization of $\cd^\hbar_{\ld{T}}[W]$ is  isomorphic to $Q(\co_{\bH_{\tilde{T}}})[\tilde{W}]$, so $\H^{\tilde{T}_c}_\ast(\Fl_G; \QQ)$ is a subalgebra of $Q(\co_{\bH_{\tilde{T}}})[\tilde{W}]$. 
    
    \cref{prop: hmlgy gkm} now gives an isomorphism between the two subsets
    $$\H^{\tilde{T}_c}_\ast(\Fl_G; \QQ) \subseteq Q(\co_{\bH_{\tilde{T}}})[\tilde{W}] \supseteq \cH(\GG_a, \tilde{T}, \tilde{W}).$$
    To see that this is an isomorphism of sub\textit{algebras}, simply observe that both $\H^{\tilde{T}_c}_\ast(\Fl_G; \QQ)$ and $\cH(\GG_a, \tilde{T}, \tilde{W})$ inherit their multiplicative structure from $Q(\co_{\bH_{\tilde{T}}})[\tilde{W}]$. That this is an isomorphism of Hopf $\co_{\bH_{\tilde{T}}}$-algebroids is also elementary: for instance, the coproduct on both $\H^{\tilde{T}_c}_\ast(\Fl_G; \QQ)$ and $\cH(\GG_a, \tilde{T}, \tilde{W})$ are inherited from the $\co_{\bH_{\tilde{T}}}$-linear coproduct on $Q(\co_{\bH_{\tilde{T}}})[\tilde{W}]$ coming from the diagonal on $\Fl_G^{\tilde{T}_c} = \tilde{W}$.
\end{proof}
\begin{remark}
    The left-hand side of \cref{thm: ordinary loop-rot flag} admits an obvious grading; on the right-hand side, the resulting grading on $\cH(\GG_a, \tilde{T}, \tilde{W})$ can be identified with that inherited from $Q(\co_{(\GG_a)_{\tilde{T}}})[\tilde{W}]$, where the coordinates of $(\GG_a)_{\tilde{T}}$ are placed in weight $2$.

    Moreover, \cref{thm: ordinary loop-rot flag} holds even if $\QQ$ is replaced by $\Z$ (as long as, on the right-hand side, $\GG_a$ is viewed as defined over $\Z$).
\end{remark}

\begin{remark}
    Suppose $W$ is finite. Then \cite[Proposition 2.3]{ginzburg-kapranov-vasserot-residue-hecke} states that \textit{upon rationalization}, the action of $\cH(\GG_a, T, W)$ on $\co_{(\GG_a)_T} = \co_{\fr{t}}$ gives an isomorphism between $\cH(\GG_a, T, W)$ and $\End_{\co_{\fr{t}}^W}(\co_{\fr{t}})$. (This result is false without rationalization, or at least without inverting enough primes.) Its $\co_{\fr{t}}$-linear dual is therefore $\co_{\fr{t}} \otimes_{\co_{\fr{t}}^W} \co_{\fr{t}} \cong \co_{\fr{t} \times_{\fr{t}\mmod W} \fr{t}}$. Note that this naturally admits the structure a cocommutative Hopf $\co_{\fr{t}}$-algebroid. The analogue of \cref{thm: ordinary loop-rot flag} states that there is an isomorphism $\H^{T_c}_\ast(G_c/T_c; \QQ) \cong \cH(\GG_a, T, W)$ of (cocommutative) Hopf $\H^\ast_{T_c}(\ast; \QQ) \cong \co_{\fr{t}}$-algebroids.
\end{remark}

Let $\mathbf{e} = \frac{1}{|W|} \sum_{w \in W} [w]$ denote the symmetrizer, viewed as an element of $\QQ[W]$. The spherical subalgebra $\cH(\GG_a, \tilde{T}, \tilde{W})^\sph$ is defined to be $\mathbf{e} \cH(\GG_a, \tilde{T}, \tilde{W}) \mathbf{e}$. The following result is now an easy consequence of \cref{thm: ordinary loop-rot flag}.
\begin{corollary}\label{cor: ordinary loop-rot Gr}
    There is an isomorphism of associative $\QQ[\hbar]$-algebras
    $$\H^{G_c \times S^1_\rot}_\ast(\Gr_G; \QQ) \cong \cH(\GG_a, \tilde{T}, \tilde{W})^\sph.$$
    Here, $\H^{G_c \times S^1_\rot}_\ast(\Gr_G; \QQ)$ is equipped with the associative algebra structure coming from convolution. Moreover, the above isomorphism is also one of (cocommutative) Hopf $\H^\ast_{G \times S^1_\rot}(\ast; \QQ) \cong \co_{\fr{t}\mmod W \times \AA^1_\hbar}$-algebroids.
\end{corollary}
Let $\cd^\hbar_{\ld{G}}$ denote the algebra of (rescaled) differential operators on $\ld{G}$, and let $\ld{N} {}_\psi \backslash \cd^\hbar_{\ld{G}}/_\psi \ld{N}$ denote its bi-Whittaker reduction (that is, its two-sided Hamiltonian reduction by the left and right actions of $\ld{N}$ with respect to a nondegenerate character $\psi: \ld{\fr{n}} \to \GG_a$). \cref{cor: ordinary loop-rot Gr} and \cite[Theorem 1.2.1]{ginzburg-whittaker} yield:
\begin{corollary}[{\cite[Theorem 3]{bf-derived-satake}}]\label{cor: loop-rot Gr and biWhit}
    There is an isomorphism of associative $\QQ[\hbar]$-algebras
    $$\H^{G_c \times S^1_\rot}_\ast(\Gr_G; \QQ) \cong \ld{N} {}_\psi \backslash \cd^\hbar_{\ld{G}}/_\psi \ld{N}.$$
\end{corollary}
Note that the diagonal on $\ld{G}$ equips $\cd^\hbar_{\ld{G}}$ with the structure of a coalgebra in the category of $U_\hbar(\ld{\g})$-bimodules. This in turn equips the bi-Whittaker reduction $\ld{N} {}_\psi \backslash \cd^\hbar_{\ld{G}}/_\psi \ld{N}$ with the structure of a (cocommutative) Hopf algebroid over $U_\hbar(\ld{\g})/_\psi \ld{N}$; by \cite{kostant-whittaker}, the latter is isomorphic to $Z(U_\hbar(\ld{\g})) \cong \Sym(\fr{t}^\ast)^W[\hbar]$.
Again, one can verify (by reduction to the case of a torus) that the isomorphism of \cref{cor: loop-rot Gr and biWhit} is one of cocommutative Hopf coalgebroids over $\H_{G_c \times S^1_\rot}^\ast(\ast; \QQ) \cong \Sym(\fr{t}^\ast)^W[\hbar]$.
\begin{remark}
    Since $\H^{G_c \times S^1_\rot}_\ast(\Gr_G; \QQ)$ is Morita equivalent to $\H^{T_c \times S^1_\rot}_\ast(\Fl_G; \QQ)$, \cref{rmk: relationship to t mmod Waff}, \cref{thm: ordinary loop-rot flag}, \cref{cor: ordinary loop-rot Gr}, and \cref{cor: loop-rot Gr and biWhit} together tell us that there are equivalences of categories
    \begin{multline*}
        \H^{T_c \times S^1_\rot}_\ast(\Fl_G; \QQ)\modc \simeq \H^{G_c \times S^1_\rot}_\ast(\Gr_G; \QQ)\modc \\
        \simeq \ld{N} {}_\psi \backslash \cd^\hbar_{\ld{G}}/_\psi \ld{N}\modc
        \simeq \cH(\GG_a, \tilde{T}, \tilde{W})\modc \simeq \IndCoh(\tilde{\fr{t}}\mmod \tilde{W}).
    \end{multline*}
\end{remark}

\begin{definition}\label{def: kappa-hbar}
    Denote by $\HC^\hbar_{\ld{G}}$ the $\infty$-category $\cd^\hbar_{\ld{G}}\modc^{\ld{G} \times \ld{G}, \weak} \simeq U_\hbar(\ld{\g})\modc^{\ld{G}, \weak}$ of Harish-Chandra bimodules. Let $\kappa_\hbar: \HC^\hbar_{\ld{G}} \to U_\hbar(\ld{\g})\modc^{(\ld{N}, \psi)}$ denote the Kostant functor of \cite[Section 2.3]{bf-derived-satake}, so that it is given by the composite
    $$\HC^\hbar_{\ld{G}} \xrightarrow{\mathrm{forget}} U_\hbar(\ld{\g})\modc \xrightarrow{\mathrm{Av}_{\ld{N},\psi}} U_\hbar(\ld{\g})\modc^{(\ld{N}, \psi)}.$$
    Note that by Skryabin's theorem (see the appendix of \cite{premet}), there is an equivalence $U_\hbar(\ld{\g})\modc^{(\ld{N}, \psi)} \simeq \QCoh(\fr{t}\mmod W \times \AA^1_\hbar)$.  Define $(\HC^\hbar_{\ld{G}})_\reg$ to denote the localizing subcategory of $\HC^\hbar_{\ld{G}}$ on which $\kappa_\hbar$ is conservative.
\end{definition}
One can check that upon ``setting $\hbar = 1$'', the category $(\HC^\hbar_{\ld{G}})_\reg$ identifies with the category $\mathcal{HC}_\mathrm{nondeg}$ from \cite[Remark 4.22]{gannon-thesis}.\footnote{Let me note here my aversion to the phrase ``setting $\hbar = 1$''. As we have seen above, $\hbar$ arises naturally as a generator of $\H^2_{S^1_\rot}(\ast; \QQ)$, and as such, it lives in \textit{nonzero grading}. It is therefore not sensible to set $\hbar$ to be equal to a nonzero number. A better -- and in some sense equivalent -- way to ``set $\hbar = 1$'' in a graded $\QQ[\hbar]$-module/category $M_\hbar$ is to extract the weight zero piece of the localization $M_\hbar[\hbar^{-1}]$. Doing this procedure to $(\HC^\hbar_{\ld{G}})_\reg$ will product $\mathcal{HC}_\mathrm{nondeg}$.} 
Before proceeding, we need a category-theoretic result, which follows from \cite[Corollary 4.7.5.3]{HA}.
\begin{prop}\label{prop: cosimplicial full faithful}
    Let $\cC^\bull$ be an augmented cosimplicial presentable stable $\infty$-category. Suppose that:
    \begin{enumerate}
        \item For every $[n]\in \Deltab^+$, the face map $d^0: \cC^i \to \cC^{i+1}$ admits a left adjoint $(d^0)^L$.
        \item The ``Beck-Chevalley conditions'' hold. That is, for every morphism $\alpha: [m] \to [n]$ in $\Deltab^+$, the following diagram commutes:
        $$\xymatrix{
        \cC^{m+1} \ar[d]_{(d^0)^L} \ar[r]^-{([0] \star \alpha)^\ast} & \cC^{n+1} \ar[d]^{(d^0)^L} \\
        \cC^m \ar[r]^-{\alpha^\ast} & \cC^n.
        }$$
    \end{enumerate}
    Then the functor $\cC^{-1} \to \Tot(\cC^\bull|_{N(\Deltab)})$ admits a fully faithful right adjoint; moreover, the essential image of this functor identifies with the full subcategory of $\cC^{-1}$ on which the functor $\cC^{-1} \to \cC^0$ is conservative.
\end{prop}
It is my understanding that the following result is closely related to recent work of Gannon and Ginzburg \cite{gannon-ginzburg} (but I have not made a comparison).
\begin{corollary}\label{cor: reg locus quantized satake}
    Recall the category $\Loc_{G_c \times S^1_\rot}^\gr(\Gr_G; k)$ from \cref{def: graded G equiv loc sys}. There is an equivalence
    $$\Loc_{G_c \times S^1_\rot}^\gr(\Gr_G; k) \simeq (\HC^\hbar_{\ld{G}})_\reg.$$
    Furthermore, the pushforward functor $\Loc_{G_c \times S^1_\rot}^\gr(\Gr_G; k) \to \Loc_{G_c \times S^1_\rot}^\gr(\ast; k)$ identifies with the functor $\kappa_\hbar: (\HC^\hbar_{\ld{G}})_\reg \to \QCoh(\fr{t}\mmod W \times \AA^1_\hbar)$.
\end{corollary}
\begin{proof}
    By definition, $\Loc_{G_c \times S^1_\rot}^\gr(\Gr_G; k)$ is the $\infty$-category of left comodules over $\H^{G_c \times S^1_\rot}_\ast(\Gr_G; k)$ in $\Loc_{G_c \times S^1_\rot}(\ast; k)$. The latter category can be identified with 
    $$\Loc_{G_c \times S^1_\rot}(\ast; k) \simeq \QCoh(\fr{t}\mmod W \times \AA^1_\hbar) \simeq U_\hbar(\ld{\g})\modc^{(\ld{N}, \psi)}.$$
    Let us denote this category by $\cC^0$. Just as in Skryabin's theorem, there is an equivalence 
    $$\ld{N} {}_\psi \backslash \cd^\hbar_{\ld{G}}/_\psi \ld{N}\modc \simeq \cd^\hbar_{\ld{G}}\modc^{(\ld{N} \times \ld{N}, \psi \times \psi)}.$$
    The Hopf algebroid structure on the pair $(U_\hbar(\ld{\g})/_\psi \ld{N}, \ld{N} {}_\psi \backslash \cd^\hbar_{\ld{G}}/_\psi \ld{N})$ defines a cosimplicial diagram
    % https://q.uiver.app/#q=WzAsNCxbMCwwLCJcXGNDXjAiXSxbMSwwLCJcXGNDXjEiXSxbMiwwLCJcXGNDXjEgXFxvdGltZXNfe1xcY0NeMH0gXFxjQ14xIl0sWzMsMCwiXFxjZG90cyJdLFswLDEsIiIsMix7Im9mZnNldCI6MX1dLFswLDEsIiIsMCx7Im9mZnNldCI6LTF9XSxbMSwyLCIiLDAseyJvZmZzZXQiOjJ9XSxbMSwyLCIiLDAseyJvZmZzZXQiOi0yfV0sWzEsMl0sWzIsMywiIiwwLHsib2Zmc2V0IjozfV0sWzIsMywiIiwwLHsib2Zmc2V0IjotM31dLFsyLDMsIiIsMCx7Im9mZnNldCI6MX1dLFsyLDMsIiIsMCx7Im9mZnNldCI6LTF9XV0=
    $$\begin{tikzcd}
	{\cC^0} & {\cC^1} & {\cC^1 \otimes_{\cC^0} \cC^1} & \cdots
	\arrow[shift right, from=1-1, to=1-2]
	\arrow[shift left, from=1-1, to=1-2]
	\arrow[shift right=2, from=1-2, to=1-3]
	\arrow[shift left=2, from=1-2, to=1-3]
	\arrow[from=1-2, to=1-3]
	\arrow[shift right=3, from=1-3, to=1-4]
	\arrow[shift left=3, from=1-3, to=1-4]
	\arrow[shift right, from=1-3, to=1-4]
	\arrow[shift left, from=1-3, to=1-4]
    \end{tikzcd}$$
    The preceding discussion implies that its totalization computes the $\infty$-category of comodules over the cocommutative Hopf algebroid $(U_\hbar(\ld{\g})/_\psi \ld{N}, \ld{N} {}_\psi \backslash \cd^\hbar_{\ld{G}}/_\psi \ld{N})$. \cref{cor: loop-rot Gr and biWhit} gives an isomorphism $\H^{G_c \times S^1_\rot}_\ast(\Gr_G; k) \cong \ld{N} {}_\psi \backslash \cd^\hbar_{\ld{G}}/_\psi \ld{N}$ of cocommutative Hopf algebroids over $U_\hbar(\ld{\g})/_\psi \ld{N} \cong \H^\ast_{G_c \times S^1_\rot}(\ast; k)$, and so the totalization of the above cosimplicial diagram is equivalent to $\Loc_{G_c \times S^1_\rot}^\gr(\Gr_G; k)$.

    There are equivalences
    \begin{align*}
        \cC^0 & = U_\hbar(\ld{\g})\modc^{(\ld{N}, \psi)} \simeq \cd^\hbar_\ld{G}\modc^{(\ld{G}, \weak), (\ld{N}, \psi)}, \\
        \cC^1 & = \cd^\hbar_{\ld{G}}\modc^{(\ld{N} \times \ld{N}, \psi \times \psi)} \simeq \cC^0 \otimes_{\HC^\hbar_{\ld{G}}} \cC^0 \simeq \End_{\HC^\hbar_{\ld{G}}}(\cC^0),
    \end{align*}
    which refine to give an equivalence of cosimplicial $\infty$-categories
    $$\cC^\bull \simeq (\cC^0)^{\otimes_{\HC^\hbar_{\ld{G}}} \bull+1}.$$
    Observe that $\cC^\bull$ extends to an {augmented} cosimplicial $\infty$-category $\tilde{\cC^\bull}$ by setting $\cC^{-1} = \HC^\hbar_{\ld{G}}$, where the functor $\cC^{-1} \to \cC^0$ induced by the unique morphism $[-1] \to [0]$ in $\Deltab^+$ is given by the Kostant functor $\kappa_\hbar$.
    It is straightforward to check that both conditions in \cref{prop: cosimplicial full faithful} hold for $\tilde{\cC^\bull}$, so we find that $\Tot(\cC^\bull)$ is equivalent the localizing subcategory $(\HC^\hbar_{\ld{G}})_\reg$ of $\cC^{-1} = \HC^\hbar_{\ld{G}}$ spanned those objects on which the Kostant functor is conservative.
\end{proof}
\begin{remark}
    One can also deduce \cref{cor: reg locus quantized satake} from \cite{bf-derived-satake}, as discussed in \cite{lonergan-fourier}. This, combined with \cite[Theorem 1.2.1]{ginzburg-whittaker}, gives an alternative proof of \cref{thm: ordinary loop-rot flag} assuming the results of \cite{bf-derived-satake}. However, as mentioned in the introduction to this section, we specifically do \textit{not} want to appeal to \cite{bf-derived-satake}, since it does not have analogues in the K-theoretic or elliptic settings.
\end{remark}
\begin{remark}\label{rmk: S1-rot ordinary full faithful on gr loc}
    Just as with \cref{prop: ordinary full faithful on gr loc}, if $\HC^{\hbar,\free}_{\ld{G}}$ denotes the essential image of the pullback functor $\Rep(\ld{G}) \to \HC^\hbar_{\ld{G}}$, then there is a fully faithful embedding 
    $$(\HC^{\hbar,\free}_{\ld{G}})^\heartsuit \hookrightarrow \Loc_{G_c \times S^1_\rot}^\gr(\Gr_G; k)^\heartsuit.$$
    This can be understood as an analogue of \cite[Theorem 2]{bf-derived-satake}.
\end{remark}
\begin{remark}
    There is a Kostant functor 
    $$\kappa_\hbar: \DMod_\hbar(\ld{G} / \ld{N})^{(\ld{G} \times \ld{T}, \weak)} \to U_\hbar(\ld{\fr{t}})\modc \simeq \QCoh(\fr{t} \times \AA^1_\hbar)$$
    given by the composite
    \begin{align*}
        \DMod_\hbar(\ld{G} / \ld{N})^{(\ld{G} \times \ld{T}, \weak)} & \xrightarrow{\mathrm{forget}} \DMod_\hbar(\ld{G} / \ld{N})^{(\ld{T}, \weak)}\\
        & \xrightarrow{\mathrm{Av}_{\ld{N},\psi}} \DMod_\hbar(\ld{G} / \ld{N})^{(\ld{T}, \weak), (\ld{N}, \psi)} \\
        & \simeq \DMod_\hbar(\ld{T})^{(\ld{T}, \weak)} \simeq U_\hbar(\ld{\fr{t}})\modc.
    \end{align*}
    Using $\kappa_\hbar$, one can define an $\infty$-category $\DMod_\hbar(\ld{G} / \ld{N})^{(\ld{G} \times \ld{T}, \weak)}_\reg$.
    Just as in \cref{cor: reg locus quantized satake}, there is an equivalence
    \begin{equation}\label{eq: gen quantized ABG}
        \Loc_{T_c \times S^1_\rot}^\gr(\Gr_G; k) \simeq \DMod_\hbar(\ld{G} / \ld{N})^{(\ld{G} \times \ld{T}, \weak)}_\reg.
    \end{equation}
    Furthermore, the pushforward functor $\Loc_{T_c \times S^1_\rot}^\gr(\Gr_G; k) \to \Loc_{T_c \times S^1_\rot}^\gr(\ast; k)$ identifies with the Kostant functor $\DMod_\hbar(\ld{G} / \ld{N})^{(\ld{G} \times \ld{T}, \weak)}_\reg \to \QCoh(\fr{t} \times \AA^1_\hbar)$. The arguments in this case are slightly more subtle, though: the equivariant homology $\H^{\tilde{T}_c}_\ast(\Gr_G; \QQ)$ no longer admits an algebra structure, but it still does admit the structure of a cocommutative coalgebra over $\H^\ast_{\tilde{T}_c}(\ast; \QQ)$. In fact, $\H^{\tilde{T}_c}_\ast(\Gr_G; \QQ)$ is isomorphic as a $(\H^{G_c \times S^1_\rot}_\ast(\Gr_G; \QQ), \H^{\tilde{T}_c}_\ast(\Fl_G; \QQ))$-bimodule to the $(\cH(\GG_a, \tilde{T}, \tilde{W}), \mathbf{e} \cH(\GG_a, \tilde{T}, \tilde{W}) \mathbf{e})$-bimodule 
    $$\cH(\GG_a, \tilde{T}, \tilde{W}) \mathbf{e} \cong \ld{N} {}_\psi \backslash \cd^\hbar_{\ld{G}}/_\psi \ld{N} \otimes_{Z(U_\hbar(\ld{\g}))} \Sym(\ld{\fr{t}})[\hbar].$$
    This bimodule is denoted $\mathbb{M}_\hbar$ in \cite[Theorem 8.1.2]{ginzburg-whittaker}.
\end{remark}
\begin{remark}
    Just as \cref{cor: reg locus abg} can be viewed as a ``generic'' version of the Arkhipov-Bezrukavnikov-Ginzburg \cite{abg-iwahori-satake} equivalence
    $$\Shv^c_I(\Gr_G; k) \simeq \QCoh(\tilde{\ld{\g}}/\ld{G}),$$
    the equivalence \cref{cor: reg locus quantized satake} can be viewed as a ``generic'' version of the Bezrukavnikov-Finkelberg \cite{bf-derived-satake} equivalence
    $$\Shv^c_{G(\co) \rtimes \GG_m^\rot}(\Gr_G; k) \simeq \HC^\hbar_{\ld{G}}.$$
    Similarly, the equivalence of \cref{eq: gen quantized ABG} can be viewed as a  ``generic'' version of the quantized Arkhipov-Bezrukavnikov-Ginzburg equivalence
    $$\Shv^c_{I \rtimes \GG_m^\rot}(\Gr_G; k) \simeq \DMod_\hbar(\ld{G} / \ld{N})^{(\ld{G} \times \ld{T}, \weak)}.$$
    Unfortunately, I am not aware of a reference for this final statement, but it can be deduced from the work of Ginzburg-Riche in \cite{ginzburg-riche}.
\end{remark}
%\stodo{frobenius constant}