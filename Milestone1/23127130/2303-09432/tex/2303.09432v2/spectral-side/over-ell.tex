We now turn to the topology of $G$, so it is connected, almost simple, and simply-laced over $\cc$.  In this setting, $k$ will be an even $2$-periodic $\Eoo$-ring equipped with an oriented group scheme $\GG$ whose underlying classical scheme $\GG_0$ over $\pi_0(k)$ is an elliptic curve $E$. We will continue to fix an algebraically closed field $F$ over $\pi_0(k)$, over which the Langlands dual group $\ld{G}$ will live. As usual, when dealing with the algebraic geometry (as opposed to the topology) of $G$, we will also view it as living over $F$; since $G$ is simply-laced, it is isogenous to $\ld{G}$. 
\begin{definition}
    The \textit{elliptic regular centralizer group scheme} $\tilde{\ld{J}}_\elc$ is defined to be the group scheme over $\Bun_{\ld{T}}^0(E)$ given by the fiber product
    $$\tilde{\ld{J}}_\elc \cong \Bun_{\ld{T}}^0(E) \times_{\Bun_{\ld{B}}^0(E)} \Bun_{\ld{T}}^0(E).$$
    Note that this is very slightly (but importantly) different from the definition of $\tilde{\ld{J}}_\mu$ and $\tilde{\ld{J}}$; the analogues of the fiber product above would instead be $(f \cdot \ld{T}) \times_{\ld{B}/\ld{B}} (f\cdot \ld{T})$ and $(f + \ld{\fr{t}}) \times_{\ld{\fr{b}}/\ld{B}} (f + \ld{\fr{t}})$. 
\end{definition}
In the following discussion, we will consider the $\ld{T}$-equivariant elliptic homology of $\Gr_G$ (instead of the $T$-equivariant elliptic homology); this will capture the minor difference between the definitions of $\tilde{\ld{J}}_\elc$ and $\tilde{\ld{J}}$ mentioned above.
\begin{theorem}\label{thm: elliptic hmlgy reg centr}
    There is an isomorphism of group schemes over $\Bun_{\ld{T}}^0(E) \cong \cM_{\ld{T},0}$:
    $$\spec_{\Bun_{\ld{T}}^0(E)}(\pi_0 \cf_{\ld{T}}(\Gr_G)^\vee) \otimes_{\pi_0(k)} F \cong \Bun_{\ld{T}}^0(E) \times_{\Bun_{\ld{B}}^0(E)} \Bun_{\ld{T}}^0(E).$$
    Here, $\spec_{\Bun_{\ld{T}}^0(E)}(\pi_0 \cf_{\ld{T}}(\Gr_G)^\vee)$ denotes the relative $\spec$ of $\pi_0 \cf_{\ld{T}}(\Gr_G)^\vee$ over $\Bun_{\ld{T}}^0(E)$.
\end{theorem}
As with \cref{thm: ordinary hmlgy reg centr} and \cref{thm: ku hmlgy reg centr}, the proof of \cref{thm: elliptic hmlgy reg centr} relies on two lemmas.
\begin{lemma}\label{lem: ell borel is flat}
    The projection map $\tilde{\ld{J}}_\elc \to \Bun_{\ld{T}}^0(E)$ (onto either factor) is flat.
\end{lemma}
\begin{proof}
    Like in the proof of \cref{lem: kappa for borel is flat}, it suffices, by miracle flatness, to show that the fibers of the map $\tilde{\ld{J}}_\elc \to \Bun_{\ld{T}}^0(E)$ have dimension exactly $\rank(\ld{G})$. But this follows from the fact that the map $\Bun_{\ld{T}}^0(E) \to \Bun_{\ld{B}}^0(E)$ lands in $\Bun_{\ld{B}}^0(E)^\reg$ (see \cref{elliptic-kostant}).
\end{proof}
For a root $\alpha$, let $\Bun_{\ld{T}}^0(E)_{\alpha\dreg} \subseteq \Bun_{\ld{T}}^0(E)$ denote the union of the substacks $\Bun_{\ld{T}}^0(E)_{\{\alpha\}}$ and $\Bun_{\ld{T}}^0(E)_{\emptyset}$. 
The next result follows exactly as in \cref{lem: localization of ordinary J} (using \cref{lem: vanishing and B-subgroups}).
\begin{lemma}\label{lem: ell localization of ordinary J}
    There is an isomorphism
    \begin{equation}\label{eq: ell J of centralizer}
        \tilde{\ld{J}}_\elc(\ld{G})|_{\Bun_{\ld{T}}^0(E)_{\alpha\dreg}} \xar{\sim} \tilde{\ld{J}}_\elc(Z_{\ld{G}}(x)^\circ)|_{\Bun_{\ld{T}}^0(E)_{\alpha\dreg}},
    \end{equation}
    where $Z_{\ld{G}}(x)$ is the centralizer of some $x\in \Bun_{\ld{T}}^0(E)_{\alpha\dreg}$ which lies in $\Bun_{\ld{T}}^0(E)_{\{\alpha\}}$, and $Z_{\ld{G}}(x)^\circ$ denotes the connected component of the identity. 
\end{lemma}
Recall that if $X$ is a scheme with subschemes $V = V(\cI) \subseteq D = V(\cJ)$ (so that $\cJ \subseteq \cI$) where $D$ is locally principal, the affine blowup $\Bl_V^D(X)$ is defined to be the complement of $V_+(\cJ)$ in the blowup $\Bl_V(X)$. That is, it is the relative $\spec$ of the algebra $\co_X[\tfrac{\cI}{\cJ}]$ of weight zero elements in $\Bl_\cI(\co_X)[\tfrac{1}{\cJ}]$, where $\Bl_\cI(\co_X) = \co_X \oplus \cI \oplus \cI^2 \oplus \cdots$ is the Rees algebra.
\begin{proof}[Proof of \cref{thm: elliptic hmlgy reg centr}]
    The argument of \cref{thm: ordinary hmlgy reg centr} reduces us to checking that the isomorphism of \cref{thm: elliptic hmlgy reg centr} holds if $G$ has semisimple rank $1$, i.e., is the product of a torus with one of $\GL_2$, $\SL_2$, or $\PGL_2$. Again, it is easy to match up the contributions from the toral factors, so we will assume that $G$ is either $\GL_2$, $\SL_2$, or $\PGL_2$. In this case, we can even replace $F$ by $\pi_0(k)$. The proofs are all rather uniform (as we have seen in \cref{thm: ordinary hmlgy reg centr} and \cref{thm: ku hmlgy reg centr}), so we will simply illustrate the argument when $G = \SL_2$ and $G = \PGL_2$.

    We begin with the case $G = \SL_2$. Since $\ld{T} = \GG_m$, we may identify $\Bun_{\ld{T}}^0(E) \cong E$; to emphasize that it plays the role of the base of $S^1$-equivariant elliptic cohomology, we will denote it by $\cM$. Let $\infty \in \cM = E$ denote the identity section. Consider the closed subschemes 
    $$V = \{(\infty,1)\} \subseteq D = \{\infty\} \times \GG_m \subseteq \cM \times \GG_m.$$
    Then, as in \cref{thm: ordinary hmlgy reg centr} and \cref{thm: ku hmlgy reg centr}, $\spec_{\Bun_{\ld{T}}^0(E)}(\pi_0 \cf_{\ld{T}}(\Gr_G)^\vee)$ identifies with the affine blowup $\Bl_V^D(\cM \times \GG_m)$. 
    
    Since $\ld{G} = \PGL_2$, an $S$-point of the stack $\Bun_{\ld{B}}^0(E)$ is the data of a degree zero rank $2$ vector bundle $\cV$ over $S \times E$ along with a line subbundle $\cL \subseteq \cV$ and an isomorphism $\cV/\cL \cong \co_{S \times E}$.
    In this language, the elliptic Kostant section $\cM = E \to \Bun_{\ld{B}}^0(E)$ classifies the unique indecomposable extension $\cV$ of $\co_{\cM \times E}$ by the Poincar\'e line bundle $\cP$. (Recall that $\cP$ can be identified, for instance, with the line bundle corresponding to the divisor $\Delta - E \times \{\infty\} - \{\infty\} \times E$.) This extension is classified by a nonzero section of $\ul{\Ext}^1_{\cM \times E}(\co_{\cM \times E}, \cP)$. 
    
    Let us now compute $\tilde{\ld{J}}_\elc$. The fiber product $\cM \times_{\Bun_{\ld{B}}^0(E)} \cM$ is isomorphic (as a group scheme over $\cM$) to the subgroup of the constant group scheme $\ul{\ld{B}} := \cM \times \ld{B}$ of those $b \in \ul{\ld{B}}$ such that $b\cdot \cV = \cV$. First, let $U = (\cM - \{\infty\}) \times E$; then $\cV|_U$ splits as $\co_U \oplus \cP|_U$. Indeed, the restriction $\cP|_U$ is a nontrivial line bundle on $U$, so its pushforward to $\cM - \{\infty\}$ has no cohomology (and hence the extension class is trivial).
    % check: if y in M, then P|_{y x E} = O_E(y - infty). so the map O_{M x E} --> P[1] restricts to O_E --> O_E(y-infty)[1]. but if y != infty, then O_E(y-infty) is nontrivial, so any such map is zero
    It follows that $\Aut_{\ld{B}}(\cV)|_U = \cM \times_{\Bun_{\ld{B}}^0(E)} U$ can be identified with $U \times \GG_m$.

    On the other hand, let $Z = \{\infty\} \times E$ denote the complement of $U$, so that the formal neighborhood $\hat{Z}$ of $Z$ is isomorphic to $\cM^\wedge_\infty \times E = \hat{\AA}^1 \times E$. Let $t$ denote a coordinate on $\hat{\AA}^1$. Then, the restriction of $\cP$ to $\hat{Z}$ is given by the $1$-parameter family of line bundles $\co_{\hat{Z}}(t - \infty)$ over $\hat{\AA}^1 \times E$. The restriction of $\cV$ to $\hat{Z}$ is classified by a map $\co_{\hat{Z}} \to \co_{\hat{Z}}(t - \infty)[1]$ which vanishes except at the origin of $\hat{\AA}^1$, where it is given by the unique (up to nonzero scalar) nontrivial map $\co_E \to \co_E[1]$.
    
    For instance, $\cV|_Z$ is isomorphic to the Atiyah bundle over $E$ from \cite{atiyah-bundle-elliptic} (i.e., the unique indecomposable rank 2 extension of the structure sheaf by itself), so that it can be realized away from $\infty \in E$ by pairs $(f_1, f_2)$ of regular functions on $E$; and near $\infty$ by pairs $(f_1, f_2)$ such that $f_1$ and $f_1 - zf_2$ are regular, where $z$ is a local coordinate of $E$. Under this description, $\End(\cV|_Z) = \End(\cV)|_Z$ is spanned by the identity and the map $(f_1, f_2) \mapsto (0, f_1)$. That is, $\End(\cV)|_Z$ is isomorphic to the group of matrices $\begin{psmallmatrix}
        a & b\\
        0 & a
    \end{psmallmatrix}$, and so $\Aut_{\ld{B}}(\cV)|_Z$ is isomorphic to $Z \times \GG_a$. It is easy to extend this description to the formal neighborhood of $Z$, and thereby find that $\Aut_{\ld{B}}(\cV)|_{\hat{Z}}$ is isomorphic to the canonical degeneration of $\GG_m$ into $\GG_a$. In other words, there is an isomorphism
    $$\Aut_{\ld{B}}(\cV)|_{\hat{Z}} \cong \spec \pi_0(k)\pw{t}[a^{\pm 1}, \tfrac{a-1}{t}].$$
    Gluing this with the description of $\Aut_{\ld{B}}(\cV)|_U$ from the preceding paragraph, we find that $\Aut_{\ld{B}}(\cV) \cong \cM \times_{\Bun_{\ld{B}}^0(E)} \cM$ is isomorphic to the affine blowup $\Bl_V^D(\cM \times \GG_m)$. We will leave it to the reader to verify that the resulting sequence of isomorphisms
    $$\Aut_{\ld{B}}(\cV) \cong \cM \times_{\Bun_{\ld{B}}^0(E)} \cM \cong \Bl_V^D(\cM \times \GG_m) \cong \spec_{\Bun_{\ld{T}}^0(E)}(\pi_0 \cf_{\ld{T}}(\Gr_G)^\vee)$$
    is one of group schemes over $\cM$.

    The case when $G = \PGL_2$ is very similar; we only indicate the necessary changes. Let $E[2] \subseteq E$ denote the $2$-torsion subgroup, and consider the closed subschemes 
    $$V = E[2] \times \mu_2 \subseteq D = E[2] \times \GG_m \subseteq \cM \times \GG_m.$$
    By arguing as in \cref{thm: ordinary hmlgy reg centr} and \cref{thm: ku hmlgy reg centr}, we find that $\spec_{\Bun_{\ld{T}}^0(E)}(\pi_0 \cf_{\ld{T}}(\Gr_G)^\vee)$ identifies with the affine blowup $\Bl_V^D(\cM \times \GG_m)$. 
    In this case, $\ld{G} = \SL_2$, and the elliptic Kostant section $\cM = E \to \Bun_{\ld{B}}^0(E)$ sends a line bundle $\cL$ to the trivially filtered $\SL_2$-bundle $\co_E \subseteq \co_E \oplus \cL$ if $\cL^2 \neq \co_E$; and to the Atiyah extension of $\cL$ by itself if $\cL^2 \cong \co_E$. This extension is defined by a nontrivial element of $\Ext^1_E(\cL, \cL^{-1}) \cong \H^1(E; \cL^{-2})$. The calculation of $\cM \times_{\Bun_{\ld{B}}^0(E)} \cM$ follows exactly the same path as in the case $G = \SL_2$ studied above.
\end{proof}
\begin{remark}
    The most classical instantiation of the Atiyah bundle $\cA$ is via the Weierstrass functions. The $\GG_a$-torsor over $E$ associated to $\cA$ is the complement of the section at $\infty$ of the projective line $\PP(\cA)$. If we work complex-analytically, $E^\an$ can be identified as the quotient $\cc/\Lambda$ for some rank $2$ lattice $\Lambda\subseteq \cc$. Associated to $\Lambda$ are two Weierstrass functions defined on $\cc$:
    \begin{align*}
        \wp(z; \Lambda) & = \tfrac{1}{z^2} + \sum_{\lambda \in \Lambda-\{0\}} \left(\tfrac{1}{(z-\lambda)^2} - \tfrac{1}{\lambda^2}\right), \\
        \zeta(z; \Lambda) & = \tfrac{1}{z} + \sum_{\lambda \in \Lambda-\{0\}} \left(\tfrac{1}{z-\lambda} + \tfrac{1}{\lambda} + \tfrac{z}{\lambda^2}\right).
    \end{align*}
    Note that $\wp(z; \Lambda)$ is doubly-periodic, i.e., $\wp(z + \lambda; \Lambda) = \wp(z; \Lambda)$ for any $\lambda \in \Lambda$. Alternatively, $\wp$ defines a map $\cc \to \cc$ which factors through a map $\cc/\Lambda = E^\an \to \cc$.

    Although $\zeta(z; \Lambda)$ is not doubly-periodic, an easy calculation shows that $\wp(z; \Lambda) = -\partial_z \zeta(z; \Lambda)$; so if $\lambda \in \Lambda$, then $\zeta(z+\lambda; \Lambda) - \zeta(z; \Lambda) = c(\lambda)$ for some constant $c(\lambda)$. The function $\lambda \mapsto c(\lambda)$ is evidently additive, and defines a homomorphism $\Lambda \to \cc$, which defines a $\cc$-bundle over $E^\an = \cc/\Lambda$. This $\cc$-bundle is precisely the analytification of the $\GG_a$-torsor associated to the Atiyah bundle. It follows that although $\zeta$ is not defined on $E^\an$, this analytification is the universal space over $E^\an$ on which $\zeta$ is well-defined.

    This discussion also describes the total space of the rank $2$-bundle $\cA^\an$ purely analytically. For instance, if $q\in \cc^\times$ is a unit complex number of modulus $<1$, we can identify $\Tot(\cA^\an)$ over the Tate curve $\cc^\times/q^\Z$ with the quotient
    $$\Tot(\cA^\an) = \left(\cc^\times \times \cc^2\right)/\left((z,x) \sim \left(qz, \begin{psmallmatrix}
    1 & 1\\
    0 & 1
    \end{psmallmatrix} x\right)\right).$$
    The appearance of the Jordan block $\begin{psmallmatrix}
    1 & 1\\
    0 & 1
    \end{psmallmatrix}$ is the basic reason why the Atiyah bundle plays the role of the principal nilpotent element $f$ in the proof of \cref{thm: elliptic hmlgy reg centr}.
\end{remark}

\begin{corollary}\label{cor: ell reg locus ordinary ABG}
    There is an $F$-linear equivalence
    $$\Loc_{\ld{T}_c}^\gr(\Gr_G; k) \otimes_{\pi_0(k)} F \simeq \QCoh(\Bun_{\ld{B}}^0(E)^\reg).$$
    Furthermore, the pushforward functor $\Loc_{\ld{T}_c}^\gr(\Gr_G; k) \to \Loc_{\ld{T}_c}^\gr(\ast; k)$ identifies with the pullback functor $\kappa^\ast: \QCoh(\Bun_{\ld{B}}^0(E)) \to \QCoh(\Bun_{\ld{T}}^0(E))$.
\end{corollary}
\begin{proof}
    By definition, $\Loc_{\ld{T}_c}^\gr(\Gr_G; k)$ is equivalent to the category of comodules over $\pi_0 \cf_\ld{T}(\Gr_G)^\vee$ in $\QCoh(\cM_{\ld{T},0}) = \QCoh(\Bun_{\ld{T}}^0(E))$. By \cref{thm: elliptic hmlgy reg centr}, it can be identified the category of quasicoherent sheaves on the quotient stack $\Bun_{\ld{T}}^0(E)/\tilde{\ld{J}}_\elc$.  We may view $\tilde{\ld{J}}_\elc$ as a closed subgroup scheme of the constant group scheme $\ld{B} \times \Bun_{\ld{T}}^0(E)$. This gives an isomorphism
    $$\Bun_{\ld{T}}^0(E)/\tilde{\ld{J}}_\elc \cong \ld{B} \backslash (\ld{B} \times \Bun_{\ld{T}}^0(E))/\tilde{\ld{J}}_\elc.$$
    Let $\Bun_{\ld{B}}^0(E)_\triv$ denote the scheme whose $S$-points are of $\ld{B}$-bundles over $S \times E$  of degree $0$ equipped with a trivialization at $S \times \{\infty\}$, so that there is a natural map $\Bun_{\ld{B}}^0(E)_\triv \to \Bun_{\ld{B}}^0(E)$. Let $\Bun_{\ld{B}}^0(E)_\triv^\reg$ denote the restriction of $\Bun_{\ld{B}}^0(E)_\triv$ to the regular locus $\Bun_{\ld{B}}^0(E)^\reg \subseteq \Bun_{\ld{B}}^0(E)$.
    It follows from Davis' work in \cite{davis-elliptic-springer} that the $\ld{B}$-orbit of $\Bun_{\ld{T}}^0(E)$ inside $\Bun_{\ld{B}}^0(E)_\triv$ is precisely the regular locus $\Bun_{\ld{B}}^0(E)_\triv^\reg$. Since $\tilde{\ld{J}}_\elc$ is by definition the stabilizer of $\kappa: \Bun_{\ld{T}}^0(E) \to \Bun_{\ld{B}}^0(E)$, the quotient $\ld{B} \backslash (\ld{B} \times \Bun_{\ld{T}}^0(E))/\tilde{\ld{J}}_\elc$ is isomorphic to $\Bun_{\ld{B}}^0(E)^\reg$; so there is an isomorphism $\Bun_{\ld{T}}^0(E)/\tilde{\ld{J}}_\mu \cong \Bun_{\ld{B}}^0(E)^\reg$.
\end{proof}
The equivalence of \cref{cor: ell reg locus ordinary ABG} is in fact symmetric monoidal for the convolution tensor structure on $\Loc_{T_c}^\gr(\Gr_G; k)$ (described in \cref{rmk: loc gr convolution tensor}) and the standard tensor product on $\QCoh(\Bun_{\ld{B}}^0(E)^\reg)$.
\begin{remark}\label{rmk: g-equiv regular satake elliptic}
    The work of Gepner and Meier in \cite{gepner-meier, t-equiv-tmf} sets up the theory of $G_c$-equivariant elliptic cohomology for compact Lie groups $G_c$. In particular, they describe a scheme $\cM_G$ over $k$ with underlying scheme $\cM_{G,0}$ over $\pi_0(k)$, such that the global sections of the structure sheaf of $\cM_G$ computes $G_c$-equivariant $k$-cohomology.
    Using this setup (and assuming a slight extension of the results of \cite{davis-elliptic-springer} replacing the simply-connectedness assumption with the condition of having torsion-free fundamental group), it can be shown that if $G$ is almost simple and simply-laced, and has torsion-free fundamental group, there is an $F$-linear equivalence
    $$\Loc_{\ld{G}_c}^\gr(\Gr_G; k) \otimes_{\pi_0(k)} F \simeq \QCoh(\Bun_{\ld{G}}^\ss(E)^\reg).$$
    Here, the left-hand side is defined to be the $\infty$-category $\coLMod_{\pi_0(\cf_G(\Gr_G)^\vee)}(\QCoh(\cM_{G,0}))$, just as in \cref{sec: degenerations}. The proof of the displayed equivalence is quite similar to that of \cref{cor: ell reg locus ordinary ABG}, and in fact can be deduced from it using the observation that $\pi_0(\cf_G(\Gr_G)^\vee) = \pi_0(\cf_T(\Gr_G)^\vee)^W$ and that the natural map $\Bun_{\ld{B}}^0(E)^\reg \to \Bun_{\ld{G}}^\ss(E)^\reg$ is a (ramified) $W$-cover. The first statement uses that $G$ is simply-connected, and the second is the elliptic version of Grothendieck-Springer theory studied in \cite[Proposition 3.1.14]{davis-elliptic-springer}.
\end{remark}

Restriction of a $\ld{B}$-bundle on $E$ to the zero section defines a map $q: \Bun_{\ld{B}}^0(E) \to B\ld{B} \to B\ld{G}$, which in turn defines a functor
\begin{equation}\label{eq: Rep G^ to ell loc}
    \Rep(\ld{G}) \to \QCoh(\Bun_{\ld{B}}^0(E)^\reg) \simeq \Loc_{\ld{T}_c}^\gr(\Gr_G; k) \otimes_{\pi_0(k)} F.
\end{equation}
More generally, the map $q: \Bun_{\ld{B}}^0(E) \to B\ld{B} \to B\ld{G} \times B\ld{T}$ defines a functor
\begin{equation}\label{eq: Rep T^ x G^ to ell loc}
    \Rep(\ld{G} \times \ld{T}) \to \QCoh(\Bun_{\ld{B}}^0(E)^\reg) \simeq \Loc_{\ld{T}_c}^\gr(\Gr_G; k) \otimes_{\pi_0(k)} F.
\end{equation}
If $V \in \Rep(\ld{G})$, let $\cS_k(V)$ denote the corresponding object of $\Loc_{\ld{T}_c}^\gr(\Gr_G; k) \otimes_{\pi_0(k)} F$. The same argument as in \cref{prop: ordinary realizing minuscule reps} shows the following, which says that $\cS_k(V) \in \Loc_{T_c}^\gr(\Gr_G; k)$ is the associated graded of a particular object $\cf_\lambda \in \Loc_{T_c}(\Gr_G; k)$ if $V$ is a minuscule $\ld{G}$-representation. 
\begin{prop}\label{prop: ell realizing minuscule reps}
    Let $\lambda_\bull = (\lambda_1, \cdots, \lambda_n)$ be a tuple of dominant minuscule weights of $\ld{G}$, let $|\lambda_\bull| = \sum_i \lambda_i$, and let $\ol{\Gr_G^{\lambda_\bull}}$ denote the corresponding \textit{convolution variety}. Let $\cf_{\lambda_\bull}$ denote the pushforward of the constant sheaf along the canonical map $q: \ol{\Gr_G^{\lambda_\bull}} \to \ol{\Gr_G^{|\lambda|}} \subseteq \Gr_G$. If $V_{\lambda_i}$ denotes the irreducible representation of $\ld{G}$ with highest weight $\lambda_i$, then there is an isomorphism $\cS_k(\bigotimes_i V_{\lambda_i}) \cong \cf_{\lambda_\bull}^\gr$.
\end{prop}
It would be very interesting to understand whether \cref{prop: ell realizing minuscule reps} can be extended to other non-minuscule irreducible representations. Again, as in \cref{rmk: ordinary action on minuscule}, if $\lambda$ is a dominant minuscule weight of $\ld{G}$, then the coaction of $\pi_0 \cf_T(\Gr_G)^\vee$ on $\pi_0 \cf_T(G/P_\lambda)$ defines a homomorphism 
\begin{equation}
    \spec \pi_0 \cf_T(\Gr_G)^\vee \to \GL(\pi_0 \cf_T(G/P_\lambda))
\end{equation}
of group schemes over $\Bun_T^0(E)$, where $\GL(\pi_0 \cf_T(G/P_\lambda))$ denotes the group scheme of $\co_{\Bun_T^0(E)}$-linear automorphisms of the vector bundle $\pi_0 \cf_T(G/P_\lambda)$. Under the isomorphisms of \cref{thm: elliptic hmlgy reg centr} and \cref{prop: ell realizing minuscule reps}, this homomorphism factors as the composite
\begin{equation}\label{eq: ell factorization action on minuscule}
    \tilde{\ld{J}}_\elc \to \ld{G} \times \Bun_T^0(E) \to \GL(V_\lambda) \times \Bun_T^0(E),
\end{equation}
where the second map describes the $\ld{G}$-action on $V_\lambda$.

\begin{remark}\label{rmk: 1-shifted cartier}
    The statements of \cref{cor: reg locus ordinary ABG}, \cref{cor: ku reg locus ordinary ABG}, and \cref{cor: ell reg locus ordinary ABG} can be packaged into a single statement as follows. Suppose $k$ is a complex-oriented $2$-periodic $\Eoo$-ring, and let $\GG$ be an oriented commutative $k$-group scheme. Let $\GG_0$ denote the underlying commutative group scheme over $\pi_0(k)$, and let $\GG_0^\vee = \Hom(\GG_0, B\GG_m)$ denote its $1$-shifted Cartier dual. Let $F$ be an algebraically closed field over $\pi_0(k)$; then there is an $F$-linear equivalence
    $$\Loc_{\ld{T}_c}^\gr(\Gr_G; k) \otimes_{\pi_0(k)} F \simeq \QCoh(\Bun_{\ld{B}}^0(\GG_0^\vee)^\reg).$$
    Similarly, there is an $F$-linear equivalence
    $$\Loc_{\ld{G}_c}^\gr(\Gr_G; k) \otimes_{\pi_0(k)} F \simeq \QCoh(\Bun_{\ld{G}}^\ss(\GG_0^\vee)^\reg).$$
    In fact, the arguments of \cref{cor: reg locus ordinary ABG}, \cref{cor: ku reg locus ordinary ABG}, and \cref{cor: ell reg locus ordinary ABG} show that these equivalences are monoidal for the convolution tensor products on $\Loc_{\ld{T}_c}^\gr(\Gr_G; k)$ and $\Loc_{\ld{G}_c}^\gr(\Gr_G; k)$ coming from the $\E{2}$-structure on $\Gr_G$, and the ordinary tensor product of quasicoherent sheaves. Moreover, a simple adaptation of the discussion at the end of \cref{sec: KU coeff} (as well as the discussion in \cref{sec: power operations}) shows that the above equivalences are canonical: they respect natural symmetries of $k$ coming from endomorphisms of $\GG_0$. 
    
    The object $\Bun_{\ld{G}}^\ss(\GG_0^\vee)$ has also appeared previously in the literature in connection to equivariant elliptic cohomology; see, for instance, \cite{sibilla-tomasini, toen-hkr}. One could heuristically view $\QCoh(\Bun_{\ld{B}}^0(\GG_0^\vee))$ and $\QCoh(\Bun_{\ld{G}}^\ss(\GG_0^\vee))$ as the ``$\GG_0$-Hochschild homology'' of $\QCoh(B\ld{B})$ and $\QCoh(B\ld{G})$, respectively.
    
    To see that these equivalences do indeed package \cref{cor: reg locus ordinary ABG}, \cref{cor: ku reg locus ordinary ABG}, and \cref{cor: ell reg locus ordinary ABG}, note that if $k = \QQ[u^{\pm 1}]$ and $\GG = \GG_a$, then the $1$-shifted Cartier dual of $\GG_0$ is $B\hat{\GG}_a$, and $\Map(B\hat{\GG}_a, B\ld{B}) \cong \ld{\fr{b}}/\ld{B}$.\footnote{In fact, this works even if $k$ is an $\Eoo$-$\Z$-algebra. Indeed, the $1$-shifted Cartier dual of $\GG_a$ over $\Z$ is the classifying stack of $\Hom(\GG_a, \GG_m) = \widehat{\GG_a^\sharp}$; here, $\widehat{\GG_a^\sharp}$ denotes the formal scheme $\spf(\Z\pdb{x}/I^{[n]})$ where $\Z\pdb{x}$ is the divided power algebra on a class $x$ and $I^{[n]}$ is the ideal generated by elements of $\Z\pdb{x}$ of degree $\geq n$. Then, $\Map(B\widehat{\GG_a^\sharp}, X)$ is isomorphic to the $1$-shifted tangent bundle $T[-1](X)$, so that $\Map(B\widehat{\GG_a^\sharp}, B\ld{B}) \cong \ld{\fr{b}}/\ld{B}$ even over $\Z$.} Similarly, if $k = \KU$ and $\GG = \GG_m$, then the $1$-shifted Cartier dual of $\GG_0$ is $B\Z$, and $\Map(B\Z, B\ld{B}) \cong \ld{B}/\ld{B}$. Finally, if $\GG_0$ is an elliptic curve $E$, then its $1$-shifted Cartier dual is $\Pic^0(E) = E$, so $\Bun_{\ld{B}}^0(\GG_0^\vee) = \Bun_{\ld{B}}^0(E)$. In fact, in this language, the calculations of \cite{ku-rel-langlands} show that the stated equivalence continues to hold if $k = \ku$ (now one must replace $\pi_0(k)$ by $\Z[\beta]$, and $F$ by $F[\beta]$) and $\GG$ is the group scheme $\spec \Z[\beta, x, \tfrac{1}{1+\beta x}]$ with group law $x + y + \beta xy$.
\end{remark}
Observe that if $\cL$ is a degree zero line bundle on $\GG_0^\vee$, then $\H^\ast(\GG_0^\vee; \cL)$ vanishes unless $\cL$ is trivial, in which case it is isomorphic to an exterior algebra over $k$ on a class in degree $1$. Using this, the Kostant slice is straightforward to describe in the semisimple rank $1$ cases. For instance, if $\ld{G} = \PGL_2$, the map $\kappa: \GG_0 \to \Bun_{\ld{B}}^0(\GG_0^\vee)$ can be understood as follows. Since $\GG_0 = \Hom(\GG_0^\vee, B\GG_m)$, a point of $\GG_0$ can be viewed as a degree zero line bundle on $\GG_0^\vee$. Given such a line bundle $\cL$, the map $\kappa$ sends it to the trivial $\ld{B}$-bundle $\cL \subseteq \cL \oplus \co_{\GG_0^\vee} \twoheadrightarrow \co_{\GG_0^\vee}$ if $\cL$ is nontrivial, and to the unique nontrivial extension $\co_{\GG_0^\vee} \subseteq \cA \twoheadrightarrow \co_{\GG_0^\vee}$ if $\cL$ is trivial. This nontrivial extension comes from a nonzero section of $\H^1(\GG_0^\vee; \co)$.
\begin{remark}\label{rmk: morava e-theory}
    \cref{rmk: 1-shifted cartier} suggests that there might be an analogue of \cref{thm: intro omnibus} for other $\Eoo$-rings $k$ which may not be equipped with a $1$-dimensional $k$-group scheme $\GG$, but which may only be equipped with an \textit{$S$-divisible group} for finite set $S$ of primes. (This is the data of a $p$-divisible group for each prime $p\in S$.) We have already seen one example of such an $\Eoo$-ring in \cref{prop: imJ reg locus ABG}: namely, $k = L_{K(1)} S^0 = (\KU^\wedge_p)^{h\Z_p^\times}$ being the $K(1)$-local sphere at a prime $p$. In this case, one only has the $p$-divisible group $\mu_{p^\infty}$ over $\spf(\KU^\wedge_p)/\Z_p^\times$. This was reflected accordingly in \cref{prop: imJ reg locus ABG}, in the sense that we could only consider the $\infty$-category $\Loc_{T_c[p^\infty]}^\gr(\Gr_G; L_{K(1)} S^0)$ of $T_c[p^\infty]$-equivariant (and not $T_c$-equivariant) sheaves on $\Gr_G$.

    To this end, suppose $k$ is an $\Eoo$-ring equipped with an oriented $S$-divisible spectral group scheme $\GG$ in the sense of \cite{elliptic-ii, elliptic-iii} for some set $S$ of primes. Let $\GG_0$ denote the corresponding (classical) $S$-divisible group over $\pi_0(k)$. Let $T_c[S^\infty]$ denote the subgroup of $T_c$ given by the union of $T_c[p^\infty]$ over all $p\in S$. We then expect that one can define the $\infty$-category $\Loc_{T_c[S^\infty]}^\gr(\Gr_G; k)$, and that just as in \cref{rmk: 1-shifted cartier}, there is an $F$-linear equivalence
    \begin{equation}\label{eq: expected S-divisible equivalence}
        \Loc_{\ld{T}_c[S^\infty]}^\gr(\Gr_G; k) \otimes_{\pi_0(k)} F \simeq \QCoh(\Bun_{\ld{B}}^0(\GG_0^\vee)^\reg).
    \end{equation}
    Here, following the lead of \cref{lem: p-nilpotence and p power torsion} and the surrounding discussion, we define
    $$\Bun_{\ld{B}}^0(\GG_0^\vee) = \bigcup_{p\in S} \colim_n \Bun_{\ld{B}}^0(\GG_0[p^n]^\vee).$$
    Both sides of \cref{eq: expected S-divisible equivalence} naturally admit an action of $\Aut(\GG_0)$, and \cref{eq: expected S-divisible equivalence} should be equivariant for this action.
    
    In fact, it should be possible to take $F$ in \cref{eq: expected S-divisible equivalence} to be the $\pi_0(k)$-algebra classifying isomorphisms of $S$-divisible groups between $\GG_0$ and a constant $S$-divisible group. For instance, if $S = \{p\}$ and $\GG_0$ is of height $h \geq 1$, then $\Bun_{\ld{B}}^0(\GG_0^\vee)$ over such an $F$ would be isomorphic to the stack of commuting $h$-tuples of $p$-power torsion elements in $\ld{B}$ modulo simultaneous $\ld{B}$-conjugation. 

    One particularly interesting instance is the case when $k$ is a Lubin-Tate theory associated to a height $n$ formal group over a perfect field of characteristic $p>0$, and $\GG$ denotes the associated ``Quillen $p$-divisible group'' over $k$ (see \cite{elliptic-ii}). Let $\co_{D_{1/n}}^\times$ denote the Morava stabilizer group of automorphisms of the fiber of $\GG_0$ over the residue field of $\pi_0(k)$, so that $\co_{D_{1/n}}^\times$ is the group of units in the ring of integers of the division algebra over $\QQ_p$ of Hasse invariant $1/n$. Then $\co_{D_{1/n}}^\times$ acts on $k$ through $\Eoo$-ring maps, with homotopy fixed points given by the $K(n)$-local sphere. Assuming the discussion surrounding \cref{eq: expected S-divisible equivalence}, we would find that $\Loc_{\ld{T}_c[p^\infty]}^\gr(\Gr_G; L_{K(n)} S^0)$ is well-defined, and is furthermore equivalent (upon an appopriate base-change) to $\QCoh(\Bun_{\ld{B}}^0(\GG_0^\vee)^\reg/\co_{D_{1/n}}^\times)$. Regardless of whether these statements are true, it seems interesting to investigate the stack $\Bun_{\ld{B}}^0(\GG_0^\vee)$ and the action of $\co_{D_{1/n}}^\times$ on it. (When $n=1$, this was done in \cref{prop: imJ reg locus ABG}.)
\end{remark}

Let us now return to some more concrete consequences of \cref{thm: elliptic hmlgy reg centr}. 
Just as with \cref{prop: ordinary gelfand-graev} and \cref{prop: ku gelfand-graev}, the calculation of \cref{thm: elliptic hmlgy reg centr} gives an \textit{elliptic} version of the Gelfand-Graev action on the affine closure $\ol{T^\ast(\ld{G}/\ld{N})}$. Taking the affine closure in the naive sense is very destructive in the case of elliptic cohomology. Nevertheless, one can define $\ol{T^\ast_E(\ld{G}/\ld{N})}$ to be the relative spectrum over $\Bun_{\ld{T}}^0(E)$ of $\pi_0$ of the (\textit{classical}, not derived!) pushforward of the structure sheaf along the quotient morphism
$$(\ld{G} \times \ld{T} \times \Bun_{\ld{T}}^0(E))/\tilde{\ld{J}}_\elc \to \Bun_{\ld{T}}^0(E).$$
\begin{prop}[Elliptic Gelfand-Graev action]\label{prop: ell gelfand-graev}
    The natural action of $\ld{G} \times \ld{T}$ on $\ol{T^\ast_E(\ld{G}/\ld{N})}$ extends to an action of $\ld{G} \times (W \rtimes \ld{T})$, where $W$ is the Weyl group.
\end{prop}
The moment map $\ol{T^\ast_E(\ld{G}/\ld{N})}/\ld{G} \to \Bun_{\ld{G}}^\ss(E)$ is $W$-equivariant for the trivial action on the target. There is a commutative diagram
$$\xymatrix{
\Bun_{\ld{B}}^0(E) \ar@{^(->}[r] \ar[dr] & \ol{T^\ast_E(\ld{G}/\ld{N})}/\ld{T} \ar[d] \\
& \Bun_{\ld{G}}^\ss(E)
}$$
which relates $\ol{T^\ast_E(\ld{G}/\ld{N})}$ to the elliptic Grothendieck-Springer resolution \cite{bzn-elliptic-springer}; and via this diagram, the elliptic Gelfand-Graev action is closely related to the Weyl action in elliptic Springer theory.
\begin{remark}
    The proof of \cref{prop: ell gelfand-graev} generalizes to show that if $\ld{P} \subseteq \ld{G}$ is a parabolic subgroup with Levi quotient $\ld{L}$ and unipotent radical $U_{\ld{P}}$, then the natural action of $\ld{G} \times \ld{L}$ on the affine closure $\ol{T^\ast_E(\ld{G}/U_{\ld{P}})}$ extends to an action of $\ld{G} \times (W_L \rtimes \ld{L})$, where $W_L = N_{\ld{G}}(\ld{L})/\ld{L}$ is the Weyl group.
\end{remark}
As in \cref{rmk: Z/2 symplectic fourier} and \cref{ex: Z/2 multiplicative symplectic fourier}, it is possible to make the action of \cref{prop: ell gelfand-graev} explicit in the case when $\ld{G} = \SL_2$.
\begin{example}\label{ex: Z/2 ell symplectic fourier}
    Let $\co(\infty)$ denote the inverse of the ideal sheaf cutting out the zero section inside $E$, and let $\cf = (\co \oplus \co(\infty))^{\oplus 2}$. As in \cref{ex: Z/2 multiplicative symplectic fourier}, $\ol{T^\ast_E(\SL_2/\GG_a)}$ can be identified with the space of sections $(u,v) = \left(\begin{psmallmatrix}
        u_1 \\
        u_2
    \end{psmallmatrix}, (v_1, v_2)\right)$ of the bundle $\cf \to E$ such that the resulting section $\pdb{u,v} = u_1 v_1 + u_2 v_2$ of $\co(\infty)$ has vanishing locus given by the zero section of $E$. 
    Modifying the analysis of \cref{rmk: Z/2 symplectic fourier} shows that if $[-1]: E \to E$ denotes the inversion map, the $\Z/2$-action of \cref{prop: ell gelfand-graev} sends 
    $$\left(\begin{psmallmatrix}
        u_1 \\
        u_2
    \end{psmallmatrix}, (v_1, v_2)\right) \mapsto \left(\begin{psmallmatrix}
        -u_2 \\
        u_1
    \end{psmallmatrix}, \alpha(v_2, -v_1)\right),$$
    where $\alpha$ is given locally around $\infty$ by multiplication by $-\tfrac{[-1](\pdb{u,v})}{\pdb{u,v}}$.
    %For instance, suppose $E$ is the universal Weierstrass curve over $\Z[a_1, a_2, a_3, a_4, a_6]$. Then (see page 120 of Silverman)
    (The discussion here makes sense with $E$ replaced by any $1$-dimensional group scheme $\bH$. When $\bH = \GG_a$ or $\GG_m$, the class $-\tfrac{[-1](\pdb{u,v})}{\pdb{u,v}}$ is equal to $1$ or $\tfrac{1}{1 + \pdb{u,v}}$, respectively, as expected from \cref{rmk: Z/2 symplectic fourier} and \cref{ex: Z/2 multiplicative symplectic fourier}.)
\end{example}
One could regard the variety $\ol{T^\ast_E(\SL_2/\GG_a)}$ of \cref{ex: Z/2 ell symplectic fourier} as an elliptic version of Van den Bergh's multiplicative quiver variety $\cB(\AA^1, \AA^2)$ from \cite{van-den-bergh-double-poisson}. Motivated by this observation, we hope to similarly define a notion of ``elliptic quiver varieties'' (generalizing the notion of multiplicative quiver variety from \cite{crawley-boevey-shaw}) in future work.

We also have the following analogue of \cref{prop: ordinary full faithful on gr loc}, whose proof is exactly the same (one only needs to use \cite[Proposition 3.1.16]{davis-elliptic-springer}, which says that $\Bun_{\ld{B}}^0(E)^\reg \hookrightarrow \Bun_{\ld{B}}^0(E)$ has complement of codimension $2$, and similarly for $\Bun_{\ld{G}}^\ss(E)^\reg \hookrightarrow \Bun_{\ld{G}}^\ss(E)$).
\begin{prop}\label{prop: ell full faithful on gr loc}
    Let $\Loc_{\ld{T}_c}^\gr(\Gr_G; k)^\heart$ denote the heart of the $t$-structure on $\Loc_{\ld{T}_c}^\gr(\Gr_G; k) = \coMod_{\pi_0(\cf_\ld{T}(\Gr_G))^\vee}(\QCoh(\Bun_\ld{T}^0(E)))$ coming from the standard (homological truncation) $t$-structure on $\QCoh(\Bun_\ld{T}^0(E))$. 
    Then, the composite functor
    $$\Loc_{\ld{T}_c}^\gr(\Gr_G; k) \otimes_{\pi_0(k)} F \simeq \QCoh(\Bun_{\ld{B}}^0(E)^\reg) \to \QCoh(\ld{G}\backslash \ol{T^\ast_E(\ld{G}/\ld{N})}/\ld{T})$$
    is $t$-exact, and on hearts, it restricts to a fully faithful functor on the essential image of \cref{eq: Rep T^ x G^ to ell loc}. Furthermore, this functor is $W$-equivariant for the natural action of $W = \N_{G_c}(\ld{T}_c)/\ld{T}_c$ on the left-hand side and the Gelfand-Graev action of \cref{prop: ell gelfand-graev} on the right-hand side.

    Similarly, suppose $G$ has torsion-free fundamental group, and let $\Loc_{\ld{G}_c}^\gr(\Gr_G; k)^\heart$ denote the heart of the $t$-structure on $\Loc_{\ld{G}_c}^\gr(\Gr_G; k) = \coMod_{\pi_0(\cf_{\ld{G}}(\Gr_G))^\vee}(\QCoh(\cM_{\ld{G},0}))$ coming from the standard (homological truncation) $t$-structure on $\QCoh(\cM_{\ld{G},0})$. 
    %If $V \in \Rep(\ld{G})$, the object $\cS_k(V)$ lies in $\Loc_{G_c}^\gr(\Gr_G; k)^\heart \otimes_{\pi_0(k)} F$, and there is an isomorphism
    %$$\Map_{\QCoh(\Bun_{\ld{G}}^\ss(E))^\heart}(q^\ast(V), q^\ast(W)) \xrightarrow{\cong} \Map_{\Loc_{\ld{G}_c}^\gr(\Gr_G; k)^\heart \otimes_{\pi_0(k)} F}(\cS_k(V), \cS_k(W))$$
    %of $F$-vector spaces for any two representations $V,W \in \Rep(\ld{G})$.
    Then, the composite functor
    $$\Loc_{\ld{G}_c}^\gr(\Gr_G; k) \otimes_{\pi_0(k)} F \simeq \QCoh(\Bun_{\ld{G}}^\ss(E)^\reg) \to \QCoh(\Bun_{\ld{G}}^\ss(E))$$
    is $t$-exact, and on hearts, it restricts to a fully faithful functor on the essential image of the functor $\Rep(\ld{G}) \to \Loc_{G_c}^\gr(\Gr_G; k) \otimes_{\pi_0(k)} F$ (analogous to \cref{eq: Rep G^ to ell loc}).
\end{prop}
\cref{prop: ell full faithful on gr loc} gives an analogue of \cite[Theorem 4]{bf-derived-satake}: namely, if $\QCoh_\free(\Bun_{\ld{G}}^\ss(E))$ denotes the essential image of the pullback functor $\Rep(\ld{G}) \to \QCoh(\Bun_{\ld{G}}^\ss(E))$, then there is a fully faithful embedding 
$$\QCoh_\free(\Bun_{\ld{G}}^\ss(E))^\heartsuit \hookrightarrow \Loc_{\ld{G}_c}^\gr(\Gr_G; k)^\heartsuit \otimes_{\pi_0(k)} F.$$ 
Similarly, if $\QCoh_\free(\ld{G}\backslash \ol{T^\ast_E(\ld{G}/\ld{N})}/\ld{T})$ denotes the essential image of the pullback functor $\Rep(\ld{G} \times \ld{T}) \to \QCoh(\ld{G}\backslash \ol{T^\ast_E(\ld{G}/\ld{N})}/\ld{T})$, then there is a fully faithful embedding 
$$\QCoh_\free(\ld{G}\backslash \ol{T^\ast_E(\ld{G}/\ld{N})}/\ld{T})^\heartsuit \hookrightarrow \Loc_{\ld{T}_c}^\gr(\Gr_G; k)^\heartsuit \otimes_{\pi_0(k)} F.$$
This implies the following result.
\begin{corollary}\label{cor: ell minuscule equivalence}
    Let $\QCoh_{\free}(\Bun_{\ld{G}}^\ss(E))^{\min,\heartsuit}$ denote the essential image of $\Rep_\min(\ld{G})$ under the pullback functor $\Rep(\ld{G})^\heartsuit \to \QCoh(\Bun_{\ld{G}}^\ss(E))^\heartsuit$ (so it is the entirety of $\QCoh(\Bun_{\ld{G}}^\ss(E))^\heartsuit$ if $F$ has characteristic zero and $\ld{G}$ is not of type $E_8$). Similarly, let $(\Loc_{\ld{G}_c}^\gr(\Gr_G; \KU)^{\heartsuit} \otimes_{\pi_0(k)} F)^\min$ denote the idempotent completion of the subcategory of $\Loc_{\ld{G}_c}^\gr(\Gr_G; \KU)^\heartsuit \otimes_{\pi_0(k)} F$ spanned by $\cf_{\lambda_\bull}^\gr$ ranging over sequences $\lambda_\bull$ of minuscule highest weights. Then there is an equivalence
    $$\QCoh_\free(\Bun_{\ld{G}}^\ss(E))^{\min,\heartsuit} \simeq (\Loc_{\ld{G}_c}^\gr(\Gr_G; k)^{\heartsuit} \otimes_{\pi_0(k)} F)^\min.$$
\end{corollary}
There is a similar equivalence
$$(\Loc_{\ld{T}_c}^\gr(\Gr_G; k)^{\heartsuit} \otimes_{\pi_0(k)} F)^\min \simeq \QCoh_\free(\ld{G}\backslash \ol{T^\ast_E(\ld{G}/\ld{N})}/\ld{T})^{\min,\heartsuit},$$
where these categories are defined analogously by idempotent completion. 

Note, again, that the category $(\Loc_{\ld{G}_c}^\gr(\Gr_G; k)^{\heartsuit} \otimes_{\pi_0(k)} F)^\min$ is the heart of a degeneration, in the sense of \cref{sec: degenerations}, of the similarly-defined category $(\Loc_{\ld{G}_c}(\Gr_G; k) \otimes_k F[u^{\pm 1}])^\min$. Thus \cref{cor: ell minuscule equivalence} gives an equivalence between the purely algebraically defined category $\QCoh_\free(\Bun_{\ld{G}}^\ss(E))^{\min,\heartsuit}$ and a degeneration of the purely topologically defined category $(\Loc_{\ld{G}_c}(\Gr_G; k) \otimes_k F[u^{\pm 1}])^\min$. Observe, again, that if $\lambda_\bull$ and $\mu_\bull$ are two sequences of dominant minuscule weights of $\ld{G}$, there is an equivalence of $k$-modules
$$\Map_{(\Loc_{\ld{G}_c}(\Gr_G; k) \otimes_k F[u^{\pm 1}])^\min}(\cf_{\lambda_\bull}, \cf_{\mu_\bull}) \simeq \cf_{\ld{G}_c}(\ol{\Gr_G^{\lambda_\bull}} \times_{\Gr_G} \ol{\Gr_G^{\mu_\bull}}),$$
so that the category $(\Loc_{\ld{G}_c}(\Gr_G; k) \otimes_k F[u^{\pm 1}])^\min$ compares to the $k$-analogue of the category from \cite[Section 3.4]{cautis-kamnitzer}.

Let us end this section with a brief comment regarding loop-rotation equivariance.
Recall from \cref{def: nil-hecke} the algebra $\cH(\bH, T, W)$ associated to a $1$-dimensional group scheme $\bH$ over a field $F$ and a root system with torus $T$ and Weyl group $W$. In the following discussion, we will set $\bH = E$, so that $\bH_T = \Bun_T^0(E) = \cM_T$. Exactly the same argument as in \cref{thm: ordinary loop-rot flag} shows the following result; here, $G$ does not need to be simply-laced.
\begin{theorem}\label{thm: ell loop-rot flag}
    There is an isomorphism of sheaves of associative algebras over $\bH_{\GG_m^\rot} = E$:
    \begin{equation}\label{eq: ell comparison to nil hecke}
        \pi_0 \cf_{\tilde{T}_c}(\Fl_G)^\vee \cong \cH(E, \tilde{T}, \tilde{W}).
    \end{equation}
    Here, $\pi_0 \cf_{\tilde{T}_c}(\Fl_G)^\vee$ is equipped with the associative algebra structure coming from convolution. Moreover, the above isomorphism is also one of (cocommutative) Hopf $\co_{\cM_{\tilde{T},0}} \cong \co_{\bH_{\tilde{T}}}$-algebroids.
\end{theorem}
\begin{remark}
    Recall the quotient $\Bun_{\tilde{T}}^0(E)\mmod \tilde{W}$ from \cref{rmk: relationship to t mmod Waff}. The discussion therein combined with \cref{thm: ell loop-rot flag} gives an equivalence of categories
    $$\pi_0 \cf_{\tilde{T}_c}(\Fl_G)^\vee\modc \simeq \cH(E, \tilde{T}, \tilde{W})\modc \simeq \IndCoh(\Bun_{\tilde{T}}^0(E)\mmod \tilde{W}).$$
    It follows, via the argument of \cref{cor: reg locus quantized satake}, that $\Loc_{\tilde{T}_c}^\gr(\Fl_G; k) \otimes_{\pi_0(k)} F$ is equivalent to the quotient of $\QCoh(\Bun_{\tilde{T}}^0(E))$ by the action of $\IndCoh(\Bun_{\tilde{T}}^0(E)\mmod \tilde{W})$.
\end{remark}
Assume, again, that $G$ is simply-laced.
Just as in \cref{sec: review Q coeff}, one would like to use \cref{thm: ell loop-rot flag} to prove analogues of \cref{cor: reg locus quantized satake} and \cref{eq: gen quantized ABG}. However, unlike with \cref{thm: ku loop-rot flag}, we do not even have a putative description for the Langlands dual side. By analogy with the K-theoretic case, it is natural to expect that the dual side will be related to elliptic quantum groups \`a la \cite{felder-elliptic-quantum}; I am currently exploring this direction of research.