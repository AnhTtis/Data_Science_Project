In this section, we will work over a given algebraically closed field $F$. For the moment, $\ld{G}$ will be a (split) almost-simple group over $F$ with torsion-free fundamental group. Let $E$ be a (smooth) elliptic curve over $k$, let $\Bun_\ld{B}^0(E)$ denote the moduli stack of $\ld{B}$-bundles on $E$ of degree $0$, and let $\Bun_\ld{T}^0(E)$ denote the scheme of $\ld{T}$-bundles on $E$ of degree $0$. We will also make use of the stack $\Bun_\ld{G}^\ss(E)$ of semistable $\ld{G}$-bundles on $E$. Our main references for the structure of $\Bun_\ld{B}^0(E)$ and $\Bun_\ld{G}^\ss(E)$ will be \cite{davis-elliptic-springer, grojnowski-shepherd-barron}.
\begin{definition}\label{B-bundle regular}
    Say that a $\ld{B}$-bundle $\cP_\ld{B}$ on $E$ is \textit{regular} if $\dim \Aut(\cP_\ld{B}) = \rank(\ld{G})$. Let $\Bun_\ld{B}^0(E)^\reg$ denote the open substack of $\Bun_\ld{B}^0(E)$ defined by the regular $\ld{B}$-bundles. Similarly, if $\cP\in \Bun_\ld{G}^\ss(E)$ is a semistable $\ld{G}$-bundle on $E$, we say that $\cP$ is \textit{regular} if $\dim \Aut(\cP) = \rank(\ld{G})$. Let $\Bun_\ld{G}^\ss(E)^\reg \subseteq \Bun_\ld{G}^\ss(E)$ denote the open substack of regular semistable $\ld{G}$-bundles.
\end{definition}
\begin{prop}\label{elliptic-kostant}
    The map $\Bun_\ld{B}^0(E) \to \Bun_\ld{T}^0(E)$ admits a canonical unique section $\kappa: \Bun_\ld{T}^0(E) \to \Bun_\ld{B}^0(E)$ landing in $\Bun_\ld{B}^0(E)^\reg$.
\end{prop}
\begin{proof}
    Let $\cP$ be a semistable $\ld{G}$-bundle on $E$.
    By \cite[Proposition 4.4.5]{davis-elliptic-springer}, the regularity of $\cP$ is equivalent to the condition that for any (or some) $\ld{B}$-reduction $\cP_\ld{B}$ of $\cP$ of degree $0$, the associated $\ld{N}$-bundle $\cP_\ld{B}/\ld{T}$ is induced from an $\ld{N}_\cP$-bundle with nontrivial associated $\ld{N}_\alpha$-bundle for each simple root $\alpha$ in a particular subset of $\Delta$ determined by $\cP$. Moreover, every geometric fiber of the map $\Bun_\ld{G}^\ss(E) \to \Hom(\bX^\ast(\ld{T}), E)\mmod W$ to the coarse moduli space of $\Bun_\ld{G}^\ss(E)$ contains a unique regular semistable $\ld{G}$-bundle. Also see \cite[Proposition 3.9]{friedman-morgan-witten}, where a similar result is stated.

    Following \cite[Definition 3.1.7]{davis-elliptic-springer}, set
    $$\tilde{\Bun}_\ld{G}^\ss(E)^\reg \cong \Bun_\ld{G}^\ss(E)^\reg \times_{\Hom(\bX^\ast(\ld{T}), E)\mmod W} \Hom(\bX^\ast(\ld{T}), E).$$
    Let $\Bun_\ld{B}^0(E)^\reg$ denote the moduli stack of $\ld{B}$-bundles on $E$ of degree $0$.
    It then follows from the isomorphism $\tilde{\Bun}_\ld{G}^\ss(E) \cong \Bun_\ld{B}^0(E)$ of \cite[Proposition 2.1.11]{davis-elliptic-springer} and the equality $\dim \Aut(\cP) = \dim \Aut(\cP_\ld{B})$ that there is an isomorphism $\tilde{\Bun}_\ld{G}^\ss(E)^\reg \cong \Bun_\ld{B}^0(E)^\reg$. In particular, every geometric fiber of the map $\Bun_\ld{B}^0(E) \to \Hom(\bX^\ast(\ld{T}), E) = \Bun_\ld{T}^0(E)$ contains a unique regular $\ld{B}$-bundle of degree $0$. 
    %This implies that if such a section $\kappa: \Bun_\ld{T}^0(E) \to \Bun_\ld{B}^0(E)$ exists, it must be unique.

    The existence of $\kappa$ is a consequence of \cite[Theorem 4.3.2]{davis-elliptic-springer}, which is a refinement of \cite[Theorem 5.1.1]{friedman-morgan-ii}. Since we will not need the full strength of \cite[Theorem 4.3.2]{davis-elliptic-springer} outside of this proof, we will only briefly recall the necessary notation and statements. In \textit{loc. cit.}, the scheme $\Bun_\ld{T}^0(E)$ is denoted by $Y$. Let $\tilde{\Bun}_\ld{G}(E)$ denote the Kontsevich-Mori compactification of $\tilde{\Bun}_\ld{G}^\ss(E) \cong \Bun_\ld{B}^0(E)$; see \cite[Definition 2.1.2]{davis-elliptic-springer}. Let $\Theta$ denote the theta-line bundle over $\Bun_\ld{T}^0(E)$ of \cite[Corollary 3.2.10]{davis-elliptic-springer}, and let $\tilde{\chi}: \tilde{\Bun}_\ld{G}(E) \to \Theta^{-1}/\GG_m$ denote the map constructed in \cite[Corollary 3.3.2]{davis-elliptic-springer}. Then, \cite[Theorem 4.3.2]{davis-elliptic-springer} shows that there is a map $\Theta^{-1} \to \tilde{\Bun}_\ld{G}^\ss(E)$ landing in $\tilde{\Bun}_\ld{G}^\ss(E)^\reg$ such that the composite 
    $$\Theta^{-1} \to \tilde{\Bun}_\ld{G}^\ss(E) \xar{\tilde{\chi}} \Theta^{-1}/\GG_m$$
    is the canonical map. Composing with the zero section of $\Theta^{-1}$, we obtain a map 
    $$\Bun_\ld{T}^0(E) \cong 0_{\Theta^{-1}} \to \Theta^{-1} \to \tilde{\Bun}_\ld{G}^\ss(E)^\reg \cong \Bun_\ld{B}^0(E).$$
    This is the desired map $\kappa$.
\end{proof}
\begin{definition}
    The map $\kappa: \Bun_\ld{T}^0(E) \to \Bun_\ld{B}^0(E)$ from \cref{elliptic-kostant} will be called the \textit{elliptic Kostant slice}.
\end{definition}
The elliptic Kostant slice builds on work of Friedman-Morgan \cite{friedman-morgan, friedman-morgan-ii, friedman-morgan-iii, friedman-morgan-witten}.

If $E$ is replaced by the constant stack $S^1$ or by $B\GG_a$, the stack $\Bun_\ld{B}^0(E)$ is to be interpreted as $\ld{B}/\ld{B}$ and $\ld{\fr{b}}/\ld{B}$, respectively. The analogue of the elliptic Kostant section is given by the maps $f\cdot \ld{T} \to \ld{B}/\ld{B}$ and $f + \ld{\fr{t}} \to \ld{\fr{b}}/\ld{B}$, respectively.

The following is \cite[Lemma 3.1.11]{davis-elliptic-springer}.
\begin{lemma}\label{lem: vanishing and B-subgroups}
    Let $I\subseteq \Phi^-$ be a subset, and let $\Bun_\ld{T}^0(E)_I$ denote the subscheme of $\Bun_\ld{T}^0(E)$ defined by those bundles $\cP_\ld{T}$ whose $\alpha$-component is trivial precisely for $\alpha\in I$. Let $\ld{N}_I\subseteq \ld{N}$ be the smallest unipotent subgroup which is invariant under $\ld{T}$-conjugation and which contains $\ld{N}_\alpha$ for every $\alpha\in I$. Then the natural map
    $$\Bun_{\ld{T}\ld{N}_I}^0(E) \times_{\Bun_\ld{T}^0(E)} \Bun_\ld{T}^0(E)_I \to \Bun_\ld{B}^0(E) \times_{\Bun_\ld{T}^0(E)} \Bun_\ld{T}^0(E)_I$$
    is an isomorphism.
\end{lemma}
%\begin{proof}
%    Let $\cP_I$ denote the universal $\ld{T}$-bundle over $\Bun_\ld{T}^0(E)_I$, so that $\Bun_\ld{B}^0(E) \times_{\Bun_\ld{T}^0(E)} \Bun_\ld{T}^0(E)_I$ is the stack of $\ld{B}$-bundles $\cP_\ld{B}$ such that $\cP_\ld{B}/\ld{N} \cong \cP_\ld{T}$; therefore, it is isomorphic to the stack $\Bun_\ld{N}^{\cP_I}$ in the notation of \cite[Section 2.1.1]{frenkel-gaitsgory-vilonen}. Similarly, $\Bun_{\ld{T}\ld{N}_I}^0(E) \times_{\Bun_\ld{T}^0(E)} \Bun_\ld{T}^0(E)_I \cong \Bun_{\ld{N}_I}^{\cP_I}$. To show that these stacks are isomorphic, consider the filtration
%    $$\ld{N}_\ell \subseteq \ld{N}_{\ell-1} \subseteq \cdots \subseteq \ld{N}_2 \subseteq \ld{N}_1 = \ld{N}$$
%    by root height (recall that the height of a root is the sum of its simple root components), so that it is invariant under $\ld{T}$-conjugation, and there is an induced filtration
%    $$\ld{N}_{I,\ell} \subseteq \ld{N}_{I, \ell-1} \subseteq \cdots \subseteq \ld{N}_{I,2} \subseteq \ld{N}_{I,1} = \ld{N}_I.$$
%    Then, $\ld{N}_j\subseteq \ld{N}$ is normal and $\ld{N}_{j-1}/\ld{N}_j$ is central in $\ld{N}/\ld{N}_j$ (and similarly for $\ld{N}_{I,j}$); this implies that $\Bun_{\ld{N}/\ld{N}_j}^{\cP_I}$ is a $\Bun_{\ld{N}_{j-1}/\ld{N}_j}^{\cP_I}$-torsor over $\Bun_{\ld{N}/\ld{N}_{j-1}}^{\cP_I}$. Similar statements hold for $\Bun_{\ld{N}_I/\ld{N}_{I,j}}^{\cP_I}$. To show that $\Bun_{\ld{N}_I}^{\cP_I} \to \Bun_\ld{N}^{\cP_I}$ is an isomorphism, it therefore suffices to show that the induced map $\Bun_{\ld{N}_{I,j-1}/\ld{N}_{I,j}}^{\cP_I} \to \Bun_{\ld{N}_{j-1}/\ld{N}_j}^{\cP_I}$ is an isomorphism. Let $\cN = \cP_I \times^\ld{T} N$, $\cN_I = \cP_I \times^\ld{T} N_I$, etc., so that $\cN_{j-1}/\cN_j$ is a direct sum of line bundles of degree zero. By choice of $N_I$, the inclusion of the trivial line bundle summands into $\cN_{j-1}/\cN_j$ factors through the map $\cN_{I,j-1}/\cN_{I,j} \to \cN_{j-1}/\cN_j$. The desired isomorphism then follows from the observation that if $U$ is a vector group with $\GG_m$-action, then $\Bun_U^\cL$ is a point if $\cL$ is a nontrivial line bundle of degree zero (because then $\H^1(E; U(\cL)) = 0$).
%\end{proof}
\begin{example}
    Suppose that $I = \emptyset$, so that $\Bun_\ld{T}^0(E)_\emptyset$ denotes the open subscheme of $\ld{T}$-bundles of degree zero whose $\alpha$-component is nontrivial for every negative root $\alpha$. The isomorphism $\tilde{\Bun}_\ld{G}^\ss(E) \cong \Bun_\ld{B}^0(E)$ implies that the map $\tilde{\Bun}_\ld{G}^\ss(E) \to \Bun_\ld{T}^0(E)$ is an isomorphism over $\Bun_\ld{T}^0(E)_\emptyset$. In particular, every point of $\Bun_\ld{T}^0(E)_\emptyset$ has a canonical associated (regular) semistable $\ld{G}$-bundle.
    
    The above results continue to hold if $E$ is replaced by the constant stack $S^1$ or by $B\GG_a$ (in which case  $\Bun_\ld{B}^0(E)$ is to be interpreted as $\ld{B}/\ld{B}$ and $\ld{\fr{b}}/\ld{B}$, respectively). In the case of $S^1$, for instance, the semistable $\ld{G}$-bundles obtained in this way from $\Bun_\ld{T}^0(E)_\emptyset$ are precisely those which lie in the regular \textit{semisimple} locus $\ld{G}^{\rss}/\ld{G}$; similarly for the case of $B\GG_a$.
\end{example}