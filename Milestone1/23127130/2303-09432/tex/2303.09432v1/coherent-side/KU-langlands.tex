\subsection{Rationalized Langlands duality over $\KU$}

Let us now discuss the $K$-theoretic analogue of \cref{once-looped-satake}. First, we discuss the story where the Kostant slice from \cref{sec: Q intersection} is replaced by the ``Steinberg slice''; below, we will discuss the story where the Kostant slice from \cref{sec: Q intersection} is replaced by a multiplicative version of the Kostant slice.
\begin{definition}[Steinberg slice]
Let $G$ be a simply-connected semisimple algebraic group or a torus. Given $w\in W$, let $N_w = N \cap w^{-1} N^- w$, so that $N_w = \prod_{\alpha\in \Phi_w} U_\alpha$, where $\Phi_w$ is the set of roots made negative by $w$. Let $w = \prod_{\alpha\in \Delta} s_\alpha\in W$ be a Coxeter element, and let $\dot{w}$ be a lift of $w$ to $N_G(T)$. Define the Steinberg slice $\Sigma = \dot{w} N_w\subseteq G$. Then \cite{steinberg-slice} proved/stated that the composite $\Sigma \to G \to G\mmod G \cong T\mmod W$ is an isomorphism. 
Let $\tilde{G} = B \times_{B} G$ be the multiplicative Grothendieck-Springer resolution, so that $\tilde{G}/G = B/B$. There is a map $\tilde{G} \to T$ sending a pair $x\in gBg^{-1}$ to $x\pmod{g[B,B]g^{-1}}$. Let $\tilde{\Sigma}$ denote the fiber product $\Sigma\times_G \tilde{G}$, so that the composite $\tilde{\Sigma} \to \tilde{G} \to T$ is an isomorphism. We will denote the inclusion of $\tilde{\Sigma}$ by $\sigma: T \to \tilde{G}$.
\end{definition}
\begin{prop}\label{steinberg-kthy-once-looped-satake}
Let $G$ be a connected and simply-connected semisimple algebraic group or a torus over $\cc$. Let $A$ be an $\Eoo$-$\KU$-algebra, and let $\GG = \GG_m$ (so $\cM_T$ is the torus $T$ over $A$). View $\tilde{\ld{G}}$ as a scheme over $\QQ$. If $\QCoh(\ld{T})$ is viewed as a module over $\QCoh(\tilde{\ld{G}}/\ld{G})$ via $\sigma^\ast$, then there is an equivalence
$$\End_{\QCoh(\tilde{\ld{G}}/\ld{G})}(\QCoh(\ld{T})) \otimes_\QQ \pi_0 A_\QQ \simeq \LMod_{\pi_0 C_\ast^T(\Gr_G(\cc); A)} \otimes \QQ.$$
\end{prop}
\begin{proof}
We will assume without loss of generality that $A = \KU$. By \cref{t-homology-grg}, there is an equivalence $\pi_0 C_\ast^T(\Gr_G(\cc); A) = \pi_0 \cf_T(\Gr_G(\cc))^\vee \simeq \co_{(T^\ast_{\GG_m} \ld{T})^\bl}$. It follows that $\LMod_{\pi_0 C_\ast^T(\Gr_G(\cc); A)} \simeq \QCoh((T^\ast_{\GG_m} \ld{T})^\bl)$. 
%Note that since $G$ is assumed to be simply-connected, $\ld{G}$ is of adjoint type.
It therefore suffices to show that over a field $k$ of characteristic zero, there is an equivalence $\End_{\QCoh(\tilde{\ld{G}}/\ld{G})}(\QCoh(\ld{T})) \simeq \QCoh((T^\ast_{\GG_m} \ld{T})^\bl)$.

As in \cref{bfm-self-intersect}, there is an equivalence $\End_{\QCoh(\tilde{\ld{G}}/\ld{G})}(\QCoh(\ld{T})) \simeq \QCoh(\ld{T} \times_{\tilde{\ld{G}}/\ld{G}} \ld{T})$, so it suffices to establish the existence of a Cartesian square
\begin{equation}\label{kthy-intersection-kostant}
    \xymatrix{
    (T^\ast_{\GG_m} \ld{T})^\bl \ar[r] \ar[d] & \ld{T} \ar[d]^-\sigma \\
    \ld{T} \ar[r]_-\sigma & \tilde{\ld{G}}/\ld{G}.
    }
\end{equation}
Again, one can reduce to the case when $\ld{G}$ has semisimple rank $1$ by the argument of \cite[Section 4.3]{bfm}.
Every split reductive group of semisimple rank $1$ is isomorphic to the product of a split torus with $\SL_2$, $\PGL_2$, or $\GL_2$.
We will illustrate the calculation when $\ld{G} = \SL_2$, and describe an alternative simpler calculation in the case $\ld{G} = \PGL_2$ later.

%%Let us return to the argument for $\ld{G} = \SL_2$. 
View a point in $\tilde{\ld{G}}$ as a pair $(x\in \SL_2, \ell\subseteq \cc^2)$ such that $x$ preserves $\ell$. The Steinberg slice $\sigma:\ld{T} \cong \GG_m \to \tilde{\SL}_2$ is the map sending $\lambda \in \GG_m$ to the pair $(x, \ell)$ with 
$$x = \begin{pmatrix}
\lambda + \lambda^{-1} & -1 \\
1 & 0
\end{pmatrix}, \ \ell = \left[\lambda: 1\right].$$
Note that this indeed a well-defined point in $\tilde{\SL}_2$, since one can check that $x$ preserves $\ell$. This calculation of $\sigma(\lambda)$ is essentially immediate from the requirement that the following diagram commutes:
$$\xymatrix{
\GG_m \cong \ld{T} \ar[r]^-\sigma \ar[d]_-{\lambda \mapsto \lambda + \lambda^{-1}} & \tilde{\SL}_2 \ar[d]\\
\AA^1 \cong \ld{T}\mmod W \ar[r]^-\sigma_-{\lambda\mapsto \begin{psmallmatrix}
\lambda & -1 \\
1 & 0
\end{psmallmatrix}} & \SL_2.
}$$
Moreover, the $\SL_2$-action on $\tilde{\SL}_2$ sends $g\in \SL_2$ and $(x,\ell)$ to $(\Ad_g(x), g\ell)$. If $g = \begin{psmallmatrix}
a & b \\
c & d
\end{psmallmatrix}$, one can directly compute that $g$ commutes with $\begin{psmallmatrix}
\lambda + \lambda^{-1} & -1 \\
1 & 0
\end{psmallmatrix}$ if and only if $a = c(\lambda + \lambda^{-1}) + d$ and $b=-c$. Therefore, $g = \begin{psmallmatrix}
c(\lambda + \lambda^{-1}) + d & -c \\
c & d
\end{psmallmatrix}$ for $c,d\in k$. In order for $\det(g) = 1$, we need 
$$c^2 + d^2 + cd(\lambda + \lambda^{-1}) = 1.$$
As long as $\lambda\neq \pm 1$, both $x$ and $g$ can be simultaneously diagonalized by $\begin{psmallmatrix}
\lambda & \lambda^{-1} \\
1 & 1
\end{psmallmatrix}$: the diagonalization of $x$ is $\begin{psmallmatrix}
\lambda & 0 \\
0 & \lambda^{-1}
\end{psmallmatrix}$, and the diagonalization of $g$ is $\begin{psmallmatrix}
c\lambda + d & 0 \\
0 & c\lambda^{-1} + d
\end{psmallmatrix}$. If $t = c\lambda+d$, then $c\lambda^{-1}+d = t^{-1}$ by the above determinant relation. We also have that $a = t - \tfrac{\lambda(t-t^{-1})}{\lambda - \lambda^{-1}}$ and $c = \tfrac{t-t^{-1}}{\lambda - \lambda^{-1}}$. This shows that $\GG_m \times_{\tilde{\SL}_2/\SL_2} \GG_m \cong \spec k[\lambda^{\pm 1}, t^{\pm 1}, \tfrac{t-t^{-1}}{\lambda - \lambda^{-1}}]$ (even if if $k$ is of characteristic $2$).
\end{proof}
An alternative argument for the Cartesian square \cref{kthy-intersection-kostant} can be given using the multiplicative Kostant slice, which gives a \textit{different} section of the map $G \to G\mmod G$. The multiplicative Kostant slice is significantly more accessible, and the resulting \cref{kthy-once-looped-satake} is what we will generalizing below to other cohomology theories.
\begin{definition}[Multiplicative Kostant slice]
Let $e\in \fr{n}$ be a principal nilpotent element. Then the map $\GG_a \to G$ corresponding to $e$ factors through the map $\GG_a = B \to \SL_2$; we will denote the image of the standard generator $\begin{psmallmatrix}
1 & 0\\
1 & 1
\end{psmallmatrix}\in B^-$ under the map $\SL_2 \to G$ by $f\in G$. Let $Z_G(e)^\circ$ be the connected component of the identity in the centralizer of $e$ in $G$. Define the \textit{multiplicative Kostant slice} $\cS_\mu$ by $Z_G(e)^\circ \cdot f \subseteq G$. Since $G$ is assumed to be simply-connected, the composite $\cS_\mu \to G \to G\mmod G \cong T\mmod W$ is an isomorphism. We will often denote the inclusion of the Kostant slice by $\kappa: T\mmod W \to G$.
Let $\tilde{\cS}_\mu$ denote the fiber product $\tilde{\cS}_\mu \times_G \tilde{G}$, so that the composite $\tilde{\cS}_\mu \to \tilde{G} \to T$ is an isomorphism; we will denote the inclusion of $\tilde{\cS}_\mu$ as a map $\kappa: \tilde{\cS}_\mu \cong T \to \tilde{G}$.

As with the additive Kostant slice, we will only care about the composite $T \to \tilde{G} \to \tilde{G}/G$ below, so we will also denote it by $\kappa$. If we identify $\tilde{G}/G \cong B/B$, then the map $\kappa$ admits a simple description: it is the composite $T \to B \to B/B$ which sends $x\mapsto xf$. Just as in \cite[Proposition 19]{kostant-lie-group-reps}, there is a unique map $\mu: T\cdot f \to N$ such that $\Ad_{\mu(x)}(x) \in Z_G(e)^\circ\cdot f$, and the image of any $x\in T$ under the map $T \to T\mmod W \xar{\kappa} G$ can be identified with $\Ad_{\mu(xf)}(xf)$.
\end{definition}
\begin{remark}
The main result of \cite{friedman-morgan-sections} states that any two sections of the map $G \to T\mmod W$ are conjugate. For instance, the multiplicative Kostant section ${T}\mmod W \cong \AA^1 \to \SL_2$ sending $\lambda \mapsto \begin{psmallmatrix}
\lambda-1 & \lambda-2 \\
1 & 1
\end{psmallmatrix}$ and the Steinberg section ${T}\mmod W \cong \AA^1 \to \SL_2$ sending $\lambda \mapsto \begin{psmallmatrix}
\lambda & -1 \\
1 & 0
\end{psmallmatrix}$ are conjugated into each other by the matrix $\begin{psmallmatrix}
1 & -1\\
0 & 1
\end{psmallmatrix}$.
\end{remark}
\begin{theorem}\label{kthy-once-looped-satake}
Let $G$ be a connected and simply-connected semisimple algebraic group or a torus over $\cc$. Let $A$ be an $\Eoo$-$\KU$-algebra, and let $\GG = \GG_m$ (so $\cM_T$ is the torus $T$ over $A$). View $\tilde{\ld{G}}$ as a scheme over $\QQ$. If $\QCoh(\ld{T})$ is viewed as a module over $\QCoh(\tilde{\ld{G}}/\ld{G})$ via $\kappa^\ast$, then there is an equivalence
$$\End_{\QCoh(\tilde{\ld{G}}/\ld{G})}(\QCoh(\ld{T})) \otimes_\QQ \pi_0 A_\QQ \simeq \LMod_{\pi_0 C_\ast^T(\Gr_G(\cc); A)} \otimes \QQ.$$
\end{theorem}
\begin{proof}
Following the argument of \cref{steinberg-kthy-once-looped-satake}, we only need to prove the Cartesian-ness of \cref{kthy-intersection-kostant}, where the map $\ld{T} \to \tilde{\ld{G}}/\ld{G}$ is chosen to be the multiplicative Kostant slice instead of the Steinberg slice. Again, we only review the calculation for $\ld{G} = \SL_2$; this was done in \cite{bfm}. For convenience, we will drop the ``check''s. As before, there are ``two'' ways to compute in the case $G = \SL_2$. First, we describe the argument essentially present in \cite{bfm} (which works over a base field of characteristic not $2$). If $\lambda\in \GG_m$, we denote $\lambda + \lambda^{-1}\in \AA^1$ by $f(\lambda)$. The Kostant slice $\kappa:\ld{T} \cong \GG_m \to \tilde{\SL}_2$ is the map sending $\lambda \in \GG_m$ to the pair $(x, \ell)$ with
$$x = \begin{pmatrix}
f(\lambda)-1 & f(\lambda)-2 \\
1 & 1
\end{pmatrix}, \ \ell = \left[\lambda-1: 1\right].$$
Note that this indeed a well-defined point in $\tilde{\SL}_2$, since one can check that $x$ preserves $\ell$: the key point is the conic relation
$$2\lambda = f(\lambda)-\sqrt{f(\lambda)^2-4}.$$
Indeed, this calculation of $\kappa(\lambda)$ is essentially immediate from the requirement that the following diagram commutes:
$$\xymatrix{
\GG_m \cong \ld{T} \ar[r]^-\kappa \ar[d]_-{\lambda \mapsto f(\lambda)} & \tilde{\SL}_2 \ar[d]\\
\AA^1 \cong \ld{T}\mmod W \ar[r]^-\kappa_-{\lambda\mapsto \begin{psmallmatrix}
\lambda-1 & \lambda-2 \\
1 & 1
\end{psmallmatrix}} & \SL_2.
}$$
Moreover, the $\SL_2$-action on $\tilde{\SL}_2$ sends $g\in \SL_2$ and $(x,\ell)$ to $(\Ad_g(x), g\ell)$. If $g = \begin{psmallmatrix}
a & b \\
c & d
\end{psmallmatrix}$, we directly compute that $\Ad_g(x) = x$ if and only if $b = c(f(\lambda) - 2)$ and $a-d = (f(\lambda) - 2)c$, in which case $g$ also preserves $\ell$. Therefore, $g = \begin{psmallmatrix}
(f(\lambda) - 2)c + d & (f(\lambda)-2)c \\
c & d
\end{psmallmatrix}$ for $c,d\in k$. In order for $\det(g) = 1$, we need 
$$d^2 + c(f(\lambda)-2)(d-c) = 1.$$
Both $x$ and $g$ can be simultaneously diagonalized (if $f(\lambda) \neq \pm 2$); note that $\lambda+\lambda^{-1}$ is an eigenvalue of $x$. If $t$ is an eigenvalue of $g$, then we have $c = \tfrac{t-t^{-1}}{\lambda - \lambda^{-1}}$ and $d = \tfrac{t^2\lambda + 1}{t(\lambda+1)}$.
When $k$ is not of characteristic $2$, this shows that $\GG_m \times_{\tilde{\SL}_2/\SL_2} \GG_m \cong k[\lambda^{\pm 1}, t^{\pm 1}, \tfrac{t-t^{-1}}{\lambda - \lambda^{-1}}]$, as desired.

For the ``second'' method of calculation when $G = \SL_2$ (which works in arbitary characteristic), we use the fact that $\kappa: T \to \tilde{G}/G$ can be identified with the composite $T \to B \to B/B$ sending $x\mapsto xf$. Then, $T \times_{B/B} T$ is isomorphic to the subvariety of $T \times B$ consisting of pairs $(x,g)$ with $x\in T$ (identified with the matrix $\begin{psmallmatrix}
x & 0\\
0 & x^{-1}
\end{psmallmatrix}$) and $\Ad_g(xf) = xf$. Note that $xf$ is the matrix $\begin{psmallmatrix}
x & 0 \\
x^{-1} & x^{-1}
\end{psmallmatrix}$. If $g = \begin{psmallmatrix}
a & 0 \\
b & a^{-1}
\end{psmallmatrix}\in B$, then 
$$\Ad_g \begin{pmatrix}
x & 0 \\
x^{-1} & x^{-1}
\end{pmatrix} = \begin{pmatrix}
x & 0 \\
a^{-2} x^{-1} + ba^{-1}(x-x^{-1}) & x^{-1}
\end{pmatrix}.$$
Therefore, $\Ad_g(xf) = xf$ if and only if 
$$a^{-2} x^{-1} + ba^{-1}(x-x^{-1}) = x^{-1},$$
which forces $b = \tfrac{a-a^{-1}}{x^2-1}$. This implies that $T \times_{B/B} T$ is isomorphic to $\spec k[x^{\pm 1}, a^{\pm 1}, \tfrac{a-a^{-1}}{x^2-1}]$, as desired.

We can also run this argument in the case $G = \PGL_2$ (again in arbitary characteristic). Again, $T \times_{B/B} T$ is isomorphic to the subvariety of $T \times B$ consisting of pairs $(x,g)$ with $x\in T$ (identified with the matrix $\begin{psmallmatrix}
x & 0\\
0 & 1
\end{psmallmatrix}$) and $\Ad_g(xf) = xf$. Note that $xf$ is the matrix $\begin{psmallmatrix}
x & 0 \\
1 & 1
\end{psmallmatrix}$. If $g = \begin{psmallmatrix}
a & 0 \\
b & 1
\end{psmallmatrix}\in B$, then
$$\Ad_g \begin{pmatrix}
x & 0 \\
1 & 1
\end{pmatrix} = \begin{pmatrix}
x & 0 \\
ba^{-1}(x-1) + a^{-1} & 1
\end{pmatrix}.$$
Therefore, $\Ad_g(xf) = xf$ if and only if 
$$ba^{-1}(x-1) + a^{-1} = 1,$$
which forces $b = \frac{a-1}{x-1}$. This implies that $T \times_{B/B} T$ is isomorphic to $\spec k[x^{\pm 1}, a^{\pm 1}, \tfrac{a-1}{x-1}]$, as desired.
\end{proof}
\begin{observe}
In the second argument for the Cartesian square \cref{kthy-intersection-kostant}, we may replace the symbol $\lambda$ by the symbol $e^\lambda$; then, $e^\lambda-1$ is the exponential of the multiplicative formal group law. In particular, the defining equation for the line $\ell$ in the cases of $\GG = \GG_a, \GG_m$ precisely describes the exponential for the formal completion $\hat{\GG}$ of $\GG$ at the identity.
\end{observe}

\begin{remark}
In \cite{bfm}, the following analogue of \cref{kthy-intersection-kostant} is established (over $\cc$, but this does not affect the statement): there is a Cartesian square
\begin{equation}\label{kthy-tmmodw-bfm}
    \xymatrix{
    (T^\ast_{\GG_m} \ld{T})^\bl \mmod W \ar[r] \ar[d] & \ld{T}\mmod W \ar[d]^-\kappa\\
    \ld{T}\mmod W \ar[r]_-\kappa & \ld{G}/\ld{G},
    }
\end{equation}
where the top-left corner can be identified with $\spec C^G_0(\Gr_G(\cc); \KU) \otimes \QQ$. We can take the fiber product of \cref{kthy-intersection-kostant} with itself over \cref{kthy-tmmodw-bfm} to obtain a Cartesian square
\begin{equation}\label{kthy-springer-kostant}
    \xymatrix{
    (T^\ast_{\GG_m} \ld{T})^\bl \times_{(T^\ast_{\GG_m} \ld{T})^\bl \mmod W} (T^\ast_{\GG_m} \ld{T})^\bl \ar[r] \ar[d] & \ld{T} \times_{\ld{T}\mmod W} \ld{T} \ar[d]^-\kappa\\
    \ld{T} \times_{\ld{T}\mmod W} \ld{T} \ar[r]_-\kappa & (\tilde{\ld{G}} \times_{\ld{G}} \tilde{\ld{G}})/\ld{G}.
    }
\end{equation}
Using \cref{kthy-once-looped-satake} and the above discussion, one can use \cref{kthy-springer-kostant} to show that $\End_{\QCoh((\tilde{\ld{G}} \times_{\ld{G}} \tilde{\ld{G}})/\ld{G})}(\QCoh(\ld{T} \times_{\ld{T}\mmod W} \ld{T}))$ can be identified with $\LMod_{\pi_0 C^T_\ast(\Fl_G(\cc); \KU)} \otimes \QQ$.
This can be viewed as a ``once-looped'' version of a K-theoretic analogue of Bezrukavnikov's equivalence from \cite{bezrukavnikov-two-geometric}.
%$T\backslash LG/T \simeq T\backslash LG/G \times_{G\backslash LG / G} G\backslash LG/T$. In 
\end{remark}

\begin{remark}\label{mult-nil-hecke}
We expect that most of the steps of \cref{looped-quantized-abg} can be replicated to study $\LMod_{C^{\tilde{T}}_\ast(\Gr_G(\cc); \KU)} \otimes \QQ$. More precisely, let $d\in \Z$, and fix a symmetric bilinear form $(-,-): \Lambda \times \Lambda \to \tfrac{1}{d}\Z$ such that whose Gram matrix is the associated Cartan matrix (i.e., $(\alpha_i, \alpha_j)$ is the $a_{ij}$ entry of the associated Cartan matrix). We then have the quantum group $U_q(\g)$ defined over $\Z[q^{\pm 1}]$  associated to the pairing $\Lambda \times \Lambda \to \Z[q^{\pm 1}]$ sending $\lambda, \mu \mapsto q^{-(\lambda, \mu)}$.
Following \cite[Definition 4.24]{univ-cat-o}, define the \textit{quantum universal category} ${\co}^\univ_q$ as the $\infty$-category of $(U_q(\g), U_q(\fr{t}))$-bimodules whose diagonal $U_q(\fr{b})$-action is integrable.

Let $(W, \Delta)$ be a crystallographic root system, let $\Lambda^\vee = \Z\Phi$ denote the associated root lattice, and let $T = \spec \Z[\Lambda]$ denote the associated torus. Each $\alpha\in W$ defines an operator $s_\alpha$ on $\co_T$. Define the \textit{multiplicative nil-Hecke algebra} $\cH(T, W)$ as the subalgebra of $\mathrm{Frac}(\co_T) \rtimes \QQ[W]$ generated by $\co_T$ and the operators $T_\alpha = \tfrac{1}{e^\alpha-1}(s_\alpha-1)$. (Also see \cite{elias-williamson-kthy} for a study of a multiplicative analogue of Soergel theory.) Then, there are relations
$$T_\alpha^2 = T_\alpha, \ (T_\alpha T_\beta)^{m_{\alpha,\beta}} = (T_\beta T_\alpha)^{m_{\alpha,\beta}} , \ x\cdot T_\alpha = T_\alpha \cdot s_\alpha(x) + T_\alpha(x), \ \alpha\in \Delta.$$
Recall that $m_{\alpha_i\alpha_j}$ is $2$, $3$, $4$, $6$, $\infty$ if $a_{ij} a_{ji}$ is $0$, $1$, $2$, $3$, $\geq 4$ (respectively). This algebra was also studied in \cite[Section 2.2]{k-thy-schubert-grg}.
Note that if $\lambda \in \Lambda$ (corresponding to the function $e^\lambda$ on $T$), we have $T_\alpha(e^\lambda) = [\langle \alpha^\vee, \lambda\rangle]_{e^\alpha} e^\lambda$, where $[\langle \alpha^\vee, \lambda\rangle]_{e^\alpha}$ denotes the $q$-integer $\tfrac{q^{\langle \alpha^\vee, \lambda\rangle} - 1}{q-1}$ with $q = e^\alpha$.

Given the discussion in \cref{sec: quantized homology torus} relating loop-rotation equivariance in $K$-theory to $q$-deformations, as well as \cref{looped-quantized-abg}, we expect:
\begin{conjecture}\label{oq-univ}
There is a Kostant functor $\kappa: \ld{\co}^\univ_q \to \QCoh(\ld{T}_\QQ \times \GG_m^q)$ (where $\GG_m^q = \spec \QQ[q^{\pm 1}]$) such that there is a $\QQ[q^{\pm 1}]$-linear equivalence
\begin{equation}\label{quantum-abg}
    \LMod_{\pi_0 C^{\tilde{T}}_\ast(\Gr_G(\cc); \KU)} \otimes \QQ \simeq \End_{\ld{\co}^\univ_q}(\QCoh(\ld{T}_\QQ \times \GG_m^q)).
\end{equation}
Similarly, if $\HC_q(\ld{G})$ denotes the category of \cite[Definition 2.24]{univ-cat-o}, there is a Kostant functor $\kappa: \HC_q(\ld{G}) \to \QCoh(\ld{T}_\QQ\mmod W \times \GG_m^q)$ and a $\QQ[q^{\pm 1}]$-linear equivalence
\begin{equation}\label{quantum-ginzburg}
    \LMod_{\pi_0 C^{G\times S^1_\rot}_\ast(\Gr_G(\cc); \KU)} \otimes \QQ \simeq \End_{\HC_q(\ld{G})}(\QCoh(\ld{T}_\QQ\mmod W \times \GG_m^q)).
\end{equation}
\end{conjecture}
At the moment, we are only able to describe the left-hand side in terms of combinatorial data. Let $e = \tfrac{1}{\# W}\sum_{w\in W} w$ be the symmetrizer idempotent. Using \cref{cohomology-grg} and \cite[Proposition 2.6]{k-thy-schubert-grg} (see also \cref{basis-cohomology}), one can show that $\pi_0 C^{\tilde{T}}_\ast(\Gr_G(\cc); \KU) \otimes \QQ$ is isomorphic to $\co_{\ld{T}} \otimes_{\co_{\ld{T}\mmod W}} e\cH(\tilde{\ld{T}}, \tilde{W}^\aff)e$, where the parameter $q\in \pi_0 \KU_{\GG_m^\rot} \cong \Z[q^{\pm 1}]$ corresponds to the coordinate on $\GG_m^q\subseteq \tilde{\ld{T}}$ viewed as an element of $\cH(\tilde{\ld{T}}, \tilde{W}^\aff)e$. Similarly, $\pi_0 C^{G \times S^1_\rot}_\ast(\Gr_G(\cc); \KU) \otimes \QQ$ is isomorphic to $e\cH(\tilde{\ld{T}}, \tilde{W}^\aff)e$. The conjectural equivalence \cref{quantum-ginzburg} then reduces to proving an (also conjectural) equivalence
\begin{equation}\label{simpler-quantum-ginzburg}
    %\End_{\ld{\co}_q^\univ}(\QCoh(\ld{T} \times \GG_m^q)) & \simeq \LMod_{\cH(\tilde{\ld{T}}, \tilde{W}^\aff)e}, \\
    \End_{\HC_q(\ld{G})}(\QCoh(\ld{T}\mmod W \times \GG_m^q)) \simeq \LMod_{e\cH(\tilde{\ld{T}}, \tilde{W}^\aff)e}.
\end{equation}
This may be understood as a quantum analogue of \cite[Theorem 8.1.2]{ginzburg-whittaker}. Note that the above equivalences are now statements which are squarely on one side of Langlands duality. In the case $G = \SL_2$, we described $C^{G\times S^1_\rot}_\ast(\Gr_G(\cc); \KU) \otimes \QQ$ (and hence $e\cH(\tilde{\ld{T}}, \tilde{W}^\aff)e$) below in \cref{4d-sl2}; it might be possible to use this calculation to compare with $\End_{\ld{\co}_q^\univ}(\QCoh(\ld{T} \times \GG_m^q))$ for $\ld{G} = \PGL_2$. A positive resolution to \cite[Conjecture 3.17]{finkelberg-tsymbaliuk} should be the key input into proving \cref{simpler-quantum-ginzburg}.

For general $G$, just as $(T\times \ld{T})^\bl$ is birational to $T \times \ld{T}$, the map from the algebra of $q$-difference operators on $\ld{T}$ to $\cH(\tilde{\ld{T}}, \tilde{W}^\aff)e$ is an isomorphism after a particular localization. One therefore expects $\ld{\co}_q^\univ$ and $\HC_q(\ld{T})$ to generically be equivalent. This is indeed true, and can be seen using \cite[Theorem 4.33]{univ-cat-o} (although the functor $\ld{\co}_q^\univ \to \HC_q(\ld{T})$ in \textit{loc. cit.} is not our expected functor $\kappa$).
\end{remark}
\begin{remark}
Since $\ld{G}/\ld{G} = \Map(S^1, B\ld{G})$, the canonical orientation of $S^1$ defines a $1$-shifted symplectic structure on $\ld{G}/\ld{G}$ via \cite[Theorem 2.5]{ptvv}. The quasi-classical limit (i.e., $q\to 1$) of the conjectural equivalence \cref{quantum-ginzburg} gives the following strengthening of \cref{kthy-once-looped-satake}. (This strengthening can be proved independently of \cref{quantum-ginzburg}.) 

Observe that the Kostant slice $\ld{T}\mmod W \to \ld{G}/\ld{G}$ is a Lagrangian morphism. It follows that the self-intersection $\ld{T}\mmod W \times_{\ld{G}/\ld{G}} \ld{T}\mmod W$ admits the structure of a symplectic stack by \cite[Theorem 2.9]{ptvv}. Since this fiber product is isomorphic to $(T^\ast_{\GG_m} \ld{T})^\bl\mmod W$ by \cref{kthy-tmmodw-bfm}, we obtain a Poisson bracket on $\co_{(T^\ast_{\GG_m} \ld{T})^\bl\mmod W} \cong \pi_0 C^G_\ast(\Gr_G(\cc); \KU)$. This structure can be seen topologically, at least after a completion: using one of the main results of \cite{klang}, the Borel-equivariant analogue/completion $C_\ast(\Gr_G(\cc); \KU)^{hG_c}$ of $C^G_\ast(\Gr_G(\cc); \KU)$ can be identified with the $\E{3}$-center of $\pi_0 C_\ast(\Gr_G(\cc); \KU)$. This defines a $2$-shifted Poisson bracket on $\pi_0 C_\ast(\Gr_G(\cc); \KU)^{hG_c}$, which can be identified with the ($0$-shifted, via the $2$-periodicity of $\KU$) Poisson bracket on $\co_{(T^\ast_{\GG_m} \ld{T})^\bl\mmod W}$.
\end{remark}
\begin{remark}
Following \cref{oq-univ}, one can also hope for a result analogous to \cref{quantum-abg} when $q \rightsquigarrow \zeta_p$ is specialized to a primitive $p$th root of unity. Namely, consider the $\infty$-category $\LMod_{C^{T\times \mu_{p,\rot}}_\ast(\Gr_G(\cc); \KU)}$, where $\mu_{p,\rot} \subseteq S^1_\rot$ acts by loop rotation. Note that $C^{T \times \mu_{p,\rot}}_\ast(\Gr_G(\cc); \KU)$ is a module over $\KU^{h\Cp}$, and $\pi_\ast \KU^{h\Cp} \cong \Z\pw{q-1}[\beta^{\pm 1}]/(q^p-1)$. Inverting $q-1$, we find that $C^{T \times \mu_{p,\rot}}_\ast(\Gr_G(\cc); \KU)[\frac{1}{q-1}]$ is a module over $\KU^{h\Cp}[\frac{1}{q-1}] \simeq \KU^{t\Cp} \simeq \QQ(\zeta_p)[\beta^{\pm 1}]$. We then expect the following (likely simpler) analogues of \cref{quantum-abg} and \cref{quantum-ginzburg}:
\begin{conjecture}\label{zetap-oq}
There are Kostant functors $\kappa: \ld{\co}^\univ_{\zeta_p} \to \QCoh(\ld{T}_{\QQ(\zeta_p)})$ and $\kappa: \HC_{\zeta_p}(\ld{G}) \to \QCoh(\ld{T}_{\QQ(\zeta_p)}\mmod W)$ such that there are $\QQ(\zeta_p)$-linear equivalences
\begin{align*}%\label{zetap-abg}
    \LMod_{\pi_0 C^{T \times \mu_{p,\rot}}_\ast(\Gr_G(\cc); \KU)[\frac{1}{q-1}]} & \simeq \End_{\ld{\co}^\univ_{\zeta_p}}(\QCoh(\ld{T}_{\QQ(\zeta_p)})),\\
    \LMod_{\pi_0 C^{G \times \mu_{p,\rot}}_\ast(\Gr_G(\cc); \KU)[\frac{1}{q-1}]} & \simeq \End_{\HC_{\zeta_p}(\ld{G})}(\QCoh(\ld{T}_{\QQ(\zeta_p)}\mmod W)).
\end{align*}
Note that there is no rationalization necessary on the left-hand sides.
\end{conjecture}
As with \cref{oq-univ}, \cref{zetap-oq} reduces to proving the (also conjectural) equivalence
\begin{align*}
    %\End_{\ld{\co}^\univ_{\zeta_p}}(\QCoh(\ld{T}_{\QQ(\zeta_p)})) & \simeq \LMod_{\cH_{\zeta_p}(\tilde{\ld{T}}, \tilde{W}^\aff)e}, \\
    \End_{\HC_{\zeta_p}(\ld{G})}(\QCoh(\ld{T}_{\QQ(\zeta_p)}\mmod W)) & \simeq \LMod_{e\cH_{\zeta_p}(\tilde{\ld{T}}, \tilde{W}^\aff)e},
\end{align*}
where $\cH_{\zeta_p}(\tilde{\ld{T}}, \tilde{W}^\aff)$ denotes the algebra obtained from $\cH(\tilde{\ld{T}}, \tilde{W}^\aff)$ by setting $q$ (arising from the loop rotation torus in $\tilde{\ld{T}}$) to $\zeta_p$.
\end{remark}