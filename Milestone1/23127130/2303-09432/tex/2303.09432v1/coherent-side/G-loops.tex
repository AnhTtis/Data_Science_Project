\subsection{Rationalized Langlands duality over elliptic cohomology}\label{section-G-loops}

\begin{definition}\label{shifted-dual}
Let $\GG_0$ be a commutative group scheme over a ring $A_0$ (even an $\Eoo$-ring, but we will not need this). Let $\GG_0^\vee$ denote the stack $\Hom(\GG_0, B\GG_m)$.
\end{definition}
\begin{example}
If $\GG_0 = \GG_m$, then $\GG_0^\vee = B\Z$, i.e., is $S^1$ viewed as a constant stack. If $\GG_0$ is an abelian variety, then $\GG_0^\vee$ is the dual abelian variety. If $\GG_0 = \Z$, then $\GG_0^\vee$ is $B\GG_m$. Let $W$ denote the commutative group scheme over $\Z_{(p)}$ of $p$-typical Witt vectors. Let $W[F]$ denote the kernel of Frobenius on $W$. If $\hat{\GG}_a$ denotes the formal completion of $\GG_a$ at the origin, then $\hat{\GG}_a^\vee \cong BW[F]$ (over $\Z_{(p)}$). Since $W[F] \cong \GG_a$ over a field of characteristic zero, there is an isomorphism $\hat{\GG}_{a,\QQ}^\vee \cong B\GG_a$.
\end{example}
\begin{remark}\label{cartier-self-duality}
In general, there is a canonical map $\GG_0 \to (\GG_0^\vee)^\vee$, and the above examples imply that it is an isomorphism if $\GG_0$ is a finite product of abelian varieties, classifying stacks of groups of multiplicative type, and finitely generated abelian groups. If this is the case, $\GG_0$ is said to be \textit{dualizable}.
\end{remark}
\begin{remark}\label{poincare-bundle}
Note that the pairing $\GG_0 \times \GG_0^\vee \to B\GG_m$ defines a line bundle over $\GG_0 \times \GG_0^\vee$, which we will denote by $\cP$ and call the \textit{Poincar\'e line bundle}. If $\GG_0$ is an abelian variety, this is the usual Poincar\'e line bundle over $\GG_0 \times \GG_0^\vee$. If $\GG_0 = \GG_m$, the Poincar\'e line bundle gives the equivalence $\Rep(\Z) \simeq \QCoh(\GG_m)$ obtained by viewing $\GG_m$ as the torus associated to the monoid $\Z$.
\end{remark}
\begin{remark}
If $\GG_0$ is a finite flat, diagonal, or constant group scheme (but not an abelian variety!), then $\GG_0^\vee$ can be identified with the classifying stack of the Cartier dual of $\GG_0$. 
If $X$ is an $A_0$-scheme, let $\cL_{\GG_0} X$ denote the \textit{$\GG_0$-loop space} of $X$, given by the mapping stack $\Map(\GG_0^\vee, X)$.
Then, if $\GG_0$ is replaced by its formal completion at the zero section, the $\GG_0$-loop space recovers the loop space of \cite{moulinos-loop}.
\end{remark}

\begin{assume}\label{char-cochar}
Fix an isomorphism $\bX^\bull(T) \cong \bX_\bull(T)$ of lattices, which will be used implicitly below without further mention. (Note that we are not asking for a $W$-equivariant isomorphism, which would not exist in general.) This gives an isomorphism $\cM_T \cong \cM_{\ld{T}}$, which we will use below as an analogue of the identification between $\ld{\fr{t}} = \fr{t}^\ast$ and $\fr{t}$ (ubiquitous in geometric representation theory). Although potentially confusing, we will see below in the proof of \cref{borel-intersection} that this identification does not run the risk of conflating different sides of Langlands duality. 
%In future work, we hope to discuss \cref{general-looped-satake} without using such an isomorphism.
\end{assume}
We will prove the following at the end of the section, after a discussion of some consequences.
\begin{theorem}\label{borel-intersection}
Fix a complex-oriented even-periodic $\Eoo$-ring $A$ and an oriented commutative $A$-group $\GG$, as well as a semisimple algebraic group $\ld{G}$ over $\QQ$.
Assume that the underlying $\pi_0 A$-scheme $\GG_0$ is $\GG_a$, $\GG_m$, or an elliptic curve $E$.
Given a principal nilpotent $f\in \fr{n}$, there is a ``$\GG$-Kostant slice'' $\kappa: (\cM_{T,0})_\QQ \to \Bun_{\ld{B}}(\GG_{0,\QQ}^\vee)$ over $\pi_0 A_\QQ$. If $\Bun_{\ld{B}}^0(\GG_{0,\QQ}^\vee) = \Bun_{\ld{B}}(\GG_{0,\QQ}^\vee) \times_{\Bun_T} \cM_T$, there is a Cartesian square
$$\xymatrix{
(T^\ast_\GG \ld{T})^\bl \otimes \QQ \ar[r] \ar[d] & (\cM_{T,0})_\QQ \ar[d]^-\kappa \\
(\cM_{T,0})_\QQ \ar[r]_-\kappa & \Bun_{\ld{B}}^0(\GG_{0,\QQ}^\vee).
}$$
\end{theorem}
Combining with \cref{t-homology-grg}, we obtain the following:
\begin{corollary}\label{general-looped-satake}
Suppose that $G$ is a connected and simply-connected semisimple algebraic group or a torus over $\cc$.
Assume that the underlying $\pi_0 A$-scheme $\GG_0$ is $\GG_a$, $\GG_m$, or an elliptic curve $E$.
Then there is an equivalence
$$\End_{\QCoh(\Bun_{\ld{B}}^0(\GG_{0,\QQ}^\vee))}(\QCoh((\cM_{T,0})_\QQ))
%\simeq \QCoh((T^\ast_{\GG} \ld{T})^\bl) 
\simeq \Mod_{\pi_0 \cf_T(\Gr_G(\cc))^\vee}(\QCoh(\cM_{T,0})) \otimes \QQ,$$
where $\QCoh((\cM_{T,0})_\QQ)$ is regarded as a $\QCoh(\Bun_{\ld{B}}^0(\GG_{0,\QQ}^\vee))$-module via $\kappa$.
\end{corollary}
\begin{example}\label{examples-dual and reduction to smooth elliptic}
For example, if $\GG = \hat{\GG}_a$, then $\GG_0^\vee = BW[F]$. Therefore, $\GG_{0,\QQ}^\vee = B\GG_a$, and $\Bun_{\ld{B}}^0(\GG_{0,\QQ}^\vee) = \ld{\fr{b}}_\QQ/\ld{B}_\QQ \cong \tilde{\ld{\g}}_\QQ/\ld{G}_\QQ$ by \cite[Theorem 1.2.4]{toen-hkr}. In particular, \cref{borel-intersection} was proved above in this case as \cref{once-looped-satake}.
If $\GG = \GG_m$, then $\GG_{0,\QQ}^\vee = B\Z = S^1$, so that $\Bun_{\ld{B}}^0(\GG_{0,\QQ}^\vee) = \Map(S^1_{\KU_\QQ}, B\ld{B}_{\KU_\QQ})$ is isomorphic to the $2$-periodification of $\ld{B}_\QQ/\ld{B}_\QQ$. In particular, \cref{borel-intersection} was proved above in this case as \cref{kthy-once-looped-satake}.
If $\GG_0$ is an elliptic curve $E$, then $\GG_0^\vee = E^\vee$, so that $\Bun_{\ld{B}}^0(\GG_0^\vee) = \Bun_{\ld{B}}^0(E^\vee)$. \cref{borel-intersection} in this case will be proved below.
\end{example}
We also obtain a proof of \cref{intro-mirror-dual-of-g} (which we restate for convenience):
\begin{corno}[\cref{intro-mirror-dual-of-g}]
Suppose that $G$ is a connected and simply-connected semisimple algebraic group or a torus over $\cc$, and let $T$ act on $G$ by conjugation. Let $G_c$ denote the maximal compact subgroup of $G(\cc)$. Fix a complex-oriented even-periodic $\Eoo$-ring $A$, and let $\GG$ be an oriented group scheme in the sense of \cite{elliptic-ii}. Assume that the underlying $\pi_0 A$-scheme $\GG_0$ is $\GG_a$, $\GG_m$, or an elliptic curve $E$. Then there is an equivalence of $\pi_0 A_\QQ$-linear $\infty$-categories:
$$\Loc_{T_c}^\gr(G_c; A) \otimes \QQ \simeq \QCoh((\cM_{\ld{T},0})_\QQ \times_{\Bun_{\ld{B}}^0(\GG_{0,\QQ}^\vee)} (\cM_{\ld{T},0})_\QQ).$$
\end{corno}
\begin{proof}
Note that $G_c$ is connected. By \cref{graded local systems}, there is an equivalence $\Loc_{T_c}^\gr(G_c; A) \simeq \LMod_{\pi_0 \cf_T(\Omega G_c)^\vee}(\QCoh(\cM_{T,0}))$, so the claim follows from \cref{general-looped-satake}.
\end{proof}
\begin{remark}\label{no-2-periodification}
If $A = \QQ[\beta^{\pm 1}]$, the equivalence resulting from \cref{intro-mirror-dual-of-g} is an equivalence of $2$-periodic $\QQ$-linear $\infty$-categories. However, the equivalence can be de-periodified, and one obtains an equivalence
$$\Loc_{T_c}(G_c; \QQ) \simeq \QCoh(\ld{\fr{t}}[2]_\QQ \times_{\tilde{\ld{\g}}[2]_\QQ/\ld{G}_\QQ} \ld{\fr{t}}[2]_\QQ).$$
There is also a $G_c$-equivariant analogue:
$$\Loc_{G_c}(G_c; \QQ) \simeq \QCoh(\ld{\fr{t}}[2]_\QQ\mmod W \times_{\ld{\g}[2]_\QQ/\ld{G}_\QQ} \ld{\fr{t}}[2]_\QQ\mmod W).$$
This equivalence can be de-equivariantized, to obtain an equivalence
$$\Loc(G_c; \QQ) \simeq \QCoh(Z_f(\ld{B})),$$
where $f\in \ld{\g}$ is the image of the origin in $\ld{\fr{t}}\mmod W$ under the Kostant slice, and $Z_f(\ld{B})$ is a shifted analogue of the centralizer of $f$ in $\ld{B}$.
Note that $T^\ast G_c = G(\cc)$, so that the left-hand side can be interpreted as a relative of the $\QQ$-linearization of the wrapped Fukaya category of $T^\ast G_c$ by \cite[Theorem 1.1]{ganatra-pardon-shende}. In particular, this shifted analogue of $Z_f(\ld{B})$ is a (derived) mirror to $G(\cc)$ viewed as a symplectic manifold.
\end{remark}
\begin{remark}\label{expected G-def}
The proof of \cref{intro-mirror-dual-of-g} above uses the Koszul duality equivalence $\Loc_{T_c}(G_c; A) \simeq \LMod_{\cf_T(\Omega G_c)^\vee}(\QCoh(\cM_{T,0}))$ of \cref{equivariant-koszul}. The category $\LMod_{\cf_T(\Omega G_c)^\vee}(\QCoh(\cM_{T,0}))$ (and hence the right-hand side of \cref{intro-mirror-dual-of-g}) admits a ``quantization'' parametrized by $\GG$, given by $\LMod_{\cf_{\tilde{T}}(\Omega G_c)^\vee}(\QCoh(\cM_{T,0}))$. For instance, if $A = \QQ[\beta^{\pm 1}]$, the right-hand side of \cref{intro-mirror-dual-of-g} quantizes to $\End_{\ld{\co}^\univ_\hbar}(\QCoh(\tilde{\fr{t}}))$; and if $A = \KU$, the right-hand side of \cref{intro-mirror-dual-of-g} quantizes to $\End_{\ld{\co}^\univ_q}(\QCoh(\tilde{T}))$. 
It follows from this discussion that the $\infty$-category $\Loc_{T_c}(G_c; A)$ must itself admits a quantization. We have seen a quantization of this form above in \cref{G-quantization torus}.

In fact, \cref{looped-quantized-abg} and \cref{oq-univ} suggest that $\LMod_{\cf_{\tilde{T}}(\Gr_G(\cc))^\vee}(\QCoh(\cM_{\tilde{T}})) \otimes \QQ$ should be viewed as $\End_{\cC_\GG}(\QCoh(\cM_{\tilde{T}})\otimes \QQ)$ for some $A_\QQ$-linear $\infty$-category $\co_\GG$ which is a $1$-parameter deformation of $\QCoh(\Bun_{\ld{B}}(\GG_{0,\QQ}^\vee))$. The coordinate on the group scheme $\GG$ defines a ``quantization parameter'' (i.e., the analogue of $\hbar$ and $q$). This putative $\infty$-category $\co_\GG$ would be an analogue of the (quantum) universal category $\co$. We do not know how to define such an $\infty$-category $\co_\GG$ at the moment; however, in future work, we plan to use the results of \cite{generalized-n-series} to study an ``$F$-deformation'' of $U(\g)$ for certain formal group laws $F(x,y)$ (at least for $G = \SL_2, \PGL_2$). When $F$ is the multiplicative formal group, this $F$-deformation of $U(\g)$ recovers the quantum enveloping algebra $U_q(\g)$. We hope that further study of such deformations will point to a good definition of the putative $\infty$-category $\co_\GG$.
\end{remark}
\begin{remark}
It is natural to ask for an explicit description of the $1$-parameter deformation of $\Loc_{T_c}(G_c; A)$ over $\GG$ from \cref{expected G-def} (i.e., not in terms of the {framed} $\E{2}$-structure on $\Omega G_c = \Omega^2 BG_c$). To describe this, let us view $\Loc_{T_c}(G_c; A)$ as the $\infty$-category of local systems on the orbifold $G_c/_\ad T_c$. We now need the following:
\begin{lemma}\label{S1-connections level}
The orbifold $G_c/_\ad T_c$ is isomorphic to the the moduli stack $\Conn(S^1; \g)^\lev$ of $\g$-valued smooth connections on $S^1$ equipped with a level structure given by a $T_c$-reduction at $\{1\}\in S^1$, taken modulo gauge transformations.
\end{lemma}
\begin{proof}
Write $G_c/_\ad T_c \simeq \ast/T_c \times_{\ast/G_c} G_c/_\ad G_c$. There is an equivalence
$$G_c/_\ad G_c \simeq \ast/G_c \times_{\ast/G_c \times \ast/G_c} \ast/G_c$$
which exhibits $G_c/_\ad G_c$ as the free loop space $\cL(\ast/G_c) \cong \ast/\cL G_c$ in the $\infty$-category of orbifolds. To see this, note that $G_c/G_c\simeq G_c\backslash (G_c\times G_c)/G_c$, where $G_c\times G_c$ acts on $G_c\times G_c$ via
$$(g_1, g_2): (h_1, h_2) \mapsto (g_1 h_1 g_2^{-1}, g_1 h_2 g_2^{-1}).$$
In any case, the above equivalence implies that $G_c/_\ad G_c$ is isomorphic to the moduli stack $\Conn(S^1; \g)/\cL G_c$, where $\Conn(S^1; \g)$ is the moduli space of smooth connections on $S^1$ valued in $\g$; see \cite[Section 15.1]{fht-iii}. This implies the desired claim.
\end{proof}
One natural way to quantize $\Loc_{T_c}(G_c; A)$ is therefore to consider the $\infty$-category of ``$S^1_\rot \ltimes \cL G_c$-equivariant $A$-valued local systems on $\Conn(S^1; \g)^\lev$''; this is a module over $\Loc_{S^1_\rot}(\ast; A) \simeq \QCoh(\GG)$, and its fiber over the zero section of $\GG$ is $\Loc_{T_c}(G_c; A)$ itself. However, defining this $\infty$-category precisely requires additional effort, since $S^1_\rot \ltimes \cL G_c$ is not a compact group.
\end{remark}

Let us now turn to the proof of \cref{borel-intersection}; by \cref{examples-dual and reduction to smooth elliptic}, we only need to consider the case when $\GG$ is a (smooth) elliptic curve $E$. Since we are working on one side of Langlands duality, we now drop the ``check''.
\begin{proof}[Proof of \cref{borel-intersection}]
We will work over $\QQ$, and omit it from the notation.
Write $X$ to denote the fiber product in \cref{borel-intersection}, so that our goal is to identify $X$ with $(T^\ast_\GG \ld{T})^\bl$. (The reader should keep in mind \cref{char-cochar}.) The argument of \cite[Section 4.3]{bfm} can be used to reduce to the case when $\ld{G}$ has semisimple rank $1$.

Namely, first note that both $X$ and $(T^\ast_\GG \ld{T})^\bl$ are flat over $\cM_T$: the only nontrivial case is $(T^\ast_\GG \ld{T})^\bl$, in which case this follows from \cite[Claim in Lemma 4.1]{bfm}. Let $\cM_T^\circ \hookrightarrow \cM_T$ denote the open immersion given by the complement of the union of the divisors $\cM_{T_\alpha} \hookrightarrow \cM_T$ for $\alpha\in \Phi$.
Upon localizing to $\cM_T^\circ$, both $X$ and $(T^\ast_\GG \ld{T})^\bl$ are isomorphic to $\ld{T} \times \cM_T^\circ$. 
Let $\cM_T^\bull$ denote the complement of the union of all pairwise intersections of the divisors $\cM_{T_\alpha} \hookrightarrow \cM_T$ for $\alpha\in \Phi$. Then $\cM_T - \cM_T^\bull \hookrightarrow \cM_T$ is of codimension $\geq 2$.
It therefore suffices to show (by flatness of both $X$ and $(T^\ast_\GG \ld{T})^\bl$ over the normal irreducible scheme $\cM_T$) that the isomorphism $X|_{\cM_T^\circ} \cong (T^\ast_\GG \ld{T})^\bl|_{\cM_T^\circ}$ extends across the codimension $1$ points of $\cM_T - \cM_T^\circ$ (i.e., points of $\cM_T^\bull - \cM_T^\circ$).

If $y$ is a codimension $1$ point of $\cM_T$ which lies on the divisor $\cM_{T_\alpha} \hookrightarrow \cM_T$ for some $\alpha\in \Phi$, let $Z_\alpha(y)\subseteq \ld{G}$ denote the reductive subgroup of $\ld{G}$ containing $\ld{T}$ and whose nonzero roots are $\pm \alpha$.
%21.11 of milne; it's the centralizer of the torus T_\alpha inside G
This is a connected Levi subgroup of semisimple rank $1$. It is easy to see that the localization $(T^\ast_\GG \ld{T})^\bl_y$ depends only on $Z_\alpha(y)$.
Let $\ld{B}_\alpha^- \subseteq \ld{B}$ denote the Borel subgroup of $Z_\alpha(y)$ determined by $\ld{B}$. \cref{vanishing and B-subgroups} with $I = \{\alpha\}$ implies that the induced map from $(\cM_{T,0})_\QQ \times_{\Bun_{\ld{B}_\alpha^-}^0(E)} (\cM_{T,0})_\QQ$ to $(\cM_{T,0})_\QQ \times_{\Bun_{\ld{B}}^0(E)} (\cM_{T,0})_\QQ$ defines an isomorphism upon localizing at $y$. In particular, the localization $X_y$ also depends only on $Z_\alpha(y)$.

We are now reduced to the case when $\ld{G}$ has semisimple rank $1$. Every split reductive group of semisimple rank $1$ is isomorphic to the product of a split torus with $\SL_2$, $\PGL_2$, or $\GL_2$. Let us illustrate the calculation when $\ld{G} = \PGL_2$. The cases $\ld{G} = \SL_2, \GL_2$, and products of tori with these groups can be addressed similarly.
For notational convenience, we will drop the ``check''s and write $B$ instead of $\ld{B}$, etc.; also note that since $T$ is of rank $1$, we may identify $\cM_T \cong \GG$.
Let $\cV$ denote the unique indecomposable rank $2$ ``Atiyah bundle'' over $E^\vee \times \GG_0$; this is an extension of the structure sheaf by the Poincar\'e line bundle $\cP$, which is specified by a nonzero section of $\H^1(E^\vee \times \GG_0; \cP) \cong k$. The bundle $\cV$ sits in a short exact sequence
$$0 \to \cP \to \cV \to \co_{E^\vee \times \GG_0} \to 0.$$

Any fixed basepoint $p_0\in E^\vee$ defines an isomorphism $E^\vee \cong \GG_0$, and allows us to identify $\cP$ with the line bundle on $E^\vee \times E^\vee$ corresponding to the divisor $\Delta - E^\vee \times \{p_0\} - \{p_0\} \times E^\vee$, where $\Delta$ is the diagonal. In particular, $\cP|_{E^\vee \times \{x\}} \cong \co_{E^\vee}(x-p_0)$, and is therefore only trivial when $x = p_0$. The fiber of $\cV$ over $E^\vee \times \{x\}$ is specified by a nonzero element of $\Ext^1_{E^\vee}(\co, \co(x-p_0))$; but if $\cL$ is a nontrivial line bundle, then $\H^1(E^\vee; \cL) = 0$. This implies that the map $\kappa: \GG_0 \to \Bun_B(E^\vee)$ sends a degree $0$ line bundle $\cL$ on $E^\vee$ to the trivial extension $\co_{E^\vee} \subseteq \co_{E^\vee} \oplus \cL$ if $\cL \not\cong \co_{E^\vee}$, and to the Atiyah extension $\co_{E^\vee}\subseteq \cf_2$ if $\cL \cong \co_{E^\vee}$. 

We need to understand $\Aut_B(\{\cP\subseteq \cV\})$.
If $\cL$ is a nontrivial line bundle on $E^\vee$, then $\cL$ has no sections, so $\Aut_B(\{\co_{E^\vee} \subseteq \co_{E^\vee} \oplus \cL\}) \cong \GG_m$. On the other hand, the algebra $\End(\cf_2)$ of endomorphisms of $\cf_2$ as a rank $2$ vector bundle is isomorphic to $k[\epsilon]/\epsilon^2$ as an algebra; the element $\epsilon$ acts as the composite $\cf_2 \twoheadrightarrow \co_{E^\vee} \hookrightarrow \cf_2$.
In particular, the group scheme $\Aut(\cf_2)$ of automorphisms of $\cf_2$ as a rank $2$ vector bundle is $(k[\epsilon]/\epsilon^2)^\times$.
An automorphism of $\cf_2$ preserving the flag $\co_{E^\vee} \subseteq \cf_2$ is defined by a matrix
$\begin{psmallmatrix}
x & y\\
0 & z
\end{psmallmatrix}$, where $x,y,z\in \Hom(\co_{E^\vee}, \co_{E^\vee})$. In order for two maps $x,z: \co_{E^\vee} \to \co_{E^\vee}$ to define an automorphism of $\cf_2$, we need $x = z$. Since we are only calculating the automorphisms of $\cf_2$ as a $\PGL_2$-bundle, the factor $x = z$ can be scaled out, and we find that $\Aut_B(\{\co_{E^\vee} \subseteq \cf_2\}) \cong \GG_a$.
%(See also \cite[Lemma 1.10]{friedman-morgan-witten-vector-bundles}.)
%also \cite[Lemma 1.14]{friedman-morgan}
The fiber of the map $\GG_0 \times_{\Bun_B(E^\vee)} \GG_0 \to \GG_0$ over $\cL\in \GG_0$ is therefore $\GG_m$ if $\cL \not \cong \co_{E^\vee}$ (i.e., away from the zero section), which degenerates to $\AA^1$ over the zero section corresponding to $\cL = \co_{E^\vee}$.

(In the case $\ld{G} = \SL_2$, the same argument shows that the fiber of the map $\GG_0 \times_{\Bun_B(E^\vee)} \GG_0 \to \GG_0$ is still $\GG_m$ if $\cL^2$ is not trivial, but the fiber over any point of $\cL\in \GG_0[2]$ is instead $\GG_a\times \mu_2$. Indeed, the image of of $\cL\in \GG_0[2]$ under the Kostant slice $\GG_0 \to \Bun_B(E^\vee)$ is the nontrivial extension 
$$0 \to \cL \to \cL \otimes \cf_2 \to \cL^{-1} \to 0.$$
Note that the subgroup of $\ld{B} \subseteq \SL_2$ given by $\Aut_{\ld{B}}(\{\cL \subseteq \cf_2 \otimes \cL\})$ is of the form $\begin{psmallmatrix}
x & y\\
0 & z
\end{psmallmatrix}$, where $x\in \Hom(\cL, \cL)$, $y\in \Hom(\cL^{-1}, \cL)$, and $z\in \Hom(\cL^{-1}, \cL^{-1})$. Not every such matrix defines an automorphism of $\cf_2 \otimes \cL$; for instance, in order for two maps $x: \cL \to \cL$ and $z:\cL^{-1} \to \cL^{-1}$ to define an automorphism of $\cf_2 \otimes \cL$, we need $x = z \otimes \cL^2 = z$. In order for the resulting matrix $\begin{psmallmatrix}
x & y\\
0 & z
\end{psmallmatrix}$ to preserve the trivialization of $\det(\cV \otimes \cL)$, we need $x^2 = 1$; the function $y$ can be arbitrary. This discussion implies that $\Aut_{\ld{B}}(\{\cL \subseteq \cf_2 \otimes \cL\}) \cong \mu_2 \times \GG_a$, where the $\mu_2$ encodes $x$, and $\GG_a$ encodes $y$.)

The intersection $\GG_0 \times_{\Bun_B(E^\vee)} \GG_0$ consists of $\cL, \cL'\in \GG_0$ equipped with an isomorphism $\kappa(\cL)\cong \kappa(\cL')$ of $B$-bundles over $E^\vee$ (which in particular forces $\cL \cong \cL'$). In fact, the discussion above can be used to conclude that $\GG_0 \times_{\Bun_B(E^\vee)} \GG_0$ is isomorphic to an affine blowup of $\GG_0 \times \GG_m$, defined as the complement $U$ of the proper preimage of the zero section of $\GG_0$ inside the blowup $\fr{B}$ of $\GG_0 \times \GG_m$ at the union of the zero sections of $\GG_0$ and $\GG_m$.
(In the case $\ld{G} = \SL_2$, the fiber product $\GG_0 \times_{\Bun_B(E^\vee)} \GG_0$ is isomorphic to an affine blowup of $\GG_0 \times \GG_m$, defined as the complement $U$ of the proper preimage of the $2$-torsion $\GG_0[2]\subseteq \GG_0$ inside the blowup $\fr{B}$ of $\GG_0 \times \GG_m$ at the union of the $2$-torsion sections $\GG_0[2] \subseteq \GG_0$ and $\mu_2 \subseteq \GG_m$.)
%(When $G = \SL_2$, the fiber over any point in the $2$-torsion $\GG[2]$ is instead $\AA^1 \times \mu_2$.)
But $U\subseteq \fr{B}$ is precisely the affine blowup $(T^\ast_\GG T)^\bl$, as desired.
\end{proof}
\begin{remark}
The most classical instantiation of the Atiyah bundle is via the Weierstrass functions. The $\GG_a$-torsor $\cA$ over $E$ associated to $\cV$ is the complement of the section at $\infty$ of the projective line $\PP(\cV)$. If we work complex-analytically, $E^\an$ can be identified as the quotient $\cc/\Lambda$ for some rank $2$ lattice $\Lambda\subseteq \cc$. Associated to $\Lambda$ are two Weierstrass functions defined on $\cc$:
\begin{align*}
    \wp(z; \Lambda) & = \frac{1}{z^2} + \sum_{\lambda \in \Lambda-\{0\}} \left(\frac{1}{(z-\lambda)^2} - \frac{1}{\lambda^2}\right), \\
    \zeta(z; \Lambda) & = \frac{1}{z} + \sum_{\lambda \in \Lambda-\{0\}} \left(\frac{1}{z-\lambda} + \frac{1}{\lambda} + \frac{z}{\lambda^2}\right).
\end{align*}
Note that $\wp(z; \Lambda)$ is doubly-periodic, i.e., $\wp(z + \lambda; \Lambda) = \wp(z; \Lambda)$ for any $\lambda \in \Lambda$. Alternatively, $\wp$ defines a map $\cc \to \cc$ which factors through a map $\cc/\Lambda = E^\an \to \cc$.

Although $\zeta(z; \Lambda)$ is not doubly-periodic, an easy calculation shows that $\wp(z; \Lambda) = -\partial_z \zeta(z; \Lambda)$; so if $\lambda \in \Lambda$, then $\zeta(z+\lambda; \Lambda) - \zeta(z; \Lambda) = c(\lambda)$ for some constant $c(\lambda)$. The function $\lambda \mapsto c(\lambda)$ is evidently additive, and defines a homomorphism $\Lambda \to \cc$, which defines a $\cc$-bundle over $E^\an = \cc/\Lambda$. This $\cc$-bundle is precisely the analytification $\cA^\an$ of the $\GG_a$-torsor $\cA$. It follows that although $\zeta$ is not defined on $E^\an$, the torsor $\cA^\an$ is the universal space over $E^\an$ on which $\zeta$ is defined.

This discussion also describes the total space of the rank $2$-bundle $\cV^\an$ purely analytically. For instance, if $q\in \cc^\times$ is a unit complex number of modulus $<1$, we can identify $\Tot(\cV^\an)$ over the Tate curve $\cc^\times/q^\Z$ with the quotient
$$\Tot(\cV^\an) = \left(\cc^\times \times \cc^2\right)/\left((z,x) \sim \left(qz, \begin{psmallmatrix}
1 & 1\\
0 & 1
\end{psmallmatrix} x\right)\right).$$
\end{remark}