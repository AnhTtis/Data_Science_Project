\subsection{Coefficients in the sphere spectrum?}\label{sphere-coeffs}

In this brief section, we study the natural question of whether there is an analogue of \cref{intro-mirror-dual-of-g} and \cref{regular-satake} with coefficients in a more general $\Eoo$-ring $R$ (e.g., the sphere spectrum). This is closely related to the discussion in \cref{sec: quantized homology torus}, and already turns out to be rather nontrivial for a torus as soon as $R$ is not complex-orientable. As a warmup, let us make the following observation.
\begin{prop}\label{torus-satake}
Fix a complex-oriented even-periodic $\Eoo$-ring $A$, and let $\GG$ be an oriented group scheme in the sense of \cite{elliptic-ii} which is dualizable. Let $T$ be a torus over $\cc$, and let $\ld{T}_A := \spec A[\bX_\ast(T)]$ denote the dual torus over $A$. Then there is an $\E{2}$-monoidal $A$-linear equivalence $\Shv_T(\Gr_T(\cc); A) \simeq \QCoh(\cM_T \times B\ld{T}_A)$. Fixing an isomorphism $\cM_T \cong \cM_{\ld{T}}$ as in \cref{char-cochar} makes this category equivalent to $\QCoh(\cL_\GG B\ld{T}_A)$.
\end{prop}
\begin{proof}
Note that there is an $\E{2}$-monoidal equivalence $\Shv_T(\Gr_T(\cc); A) \simeq \Shv_{T_c}(\Omega T_c; A)$. Since the $T_c$-action on $\Omega T_c$ is trivial and $\Omega T_c \cong \bX_\ast(T)$ as $\Eoo$-spaces, we obtain an $\E{2}$-monoidal equivalence 
$$\Shv_T(\Gr_T(\cc); A) \simeq \Fun(\bX_\ast(T), \Loc_{T_c}(\ast; A)) \simeq \Fun(\bX_\ast(T), \Mod_A) \otimes_{\Mod_A} \QCoh(\cM_T).$$
The first claim now follows from the equivalence $\QCoh(B\ld{T}_A) \simeq \Fun(\bX_\ast(T), \Mod_A)$. Fixing an isomorphism $\cM_T \cong \cM_{\ld{T}}$ and using that $\cL_\GG B\ld{T}_A \cong B\ld{T}_A \times \cM_{\ld{T}}$, we see that $\Shv_T(\Gr_T(\cc); A)$ can be identified with $\QCoh(\cL_\GG B\ld{T}_A)$, as desired.
\end{proof}
Crucial to the argument of \cref{torus-satake} was the equivalence $\Loc_{T_c}(\ast; A) \simeq \QCoh(\cM_T)$. If $R$ is a general $\Eoo$-ring, then such a statement will generally \textit{only} be true (for an appropriate definition of $\Loc_{T_c}(\ast; R)$) when $R$ is close to being complex-oriented. For example:
\begin{example}
The methods of this article show that there is an analogue of \cref{intro-mirror-dual-of-g} for $\KO$:
$$\Loc_{T_c}^\gr(G_c; \KO) \otimes \QQ \simeq \QCoh((\cM_{\ld{T},0})_\QQ \times_{\Bun_{\ld{B}}^0(\GG_0^\vee)} (\cM_{\ld{T},0})_\QQ).$$
Here, $\GG$ is the universal spectral multiplicative group over $B\Z/2$. Similarly, using the definition of genuine $T$-equivariant $\TMF$ from \cite{t-equiv-tmf}, one can also obtain an analogue of \cref{intro-mirror-dual-of-g} (where $\GG$ is replaced by the universal oriented spectral elliptic curve over the moduli stack of oriented spectral elliptic curves from \cite[Proposition 7.2.10]{elliptic-ii}).
\end{example}
See also \cite[Section 8.1]{mnn-nilpotence-equiv} for a variant of the following:
\begin{example}
Let $\Sp_{T_c}$ denote the $\infty$-category of genuine ${T_c}$-equivariant spectra, and let $i_{T_c}^\ast: \Sp_{T_c} \to \Sp$ be the lax symmetric monoidal right adjoint to the unique symmetric monoidal colimit-preserving functor $\Sp \to \Sp_{T_c}$. Suppose $R$ is an $\Eoo$-ring such that there is an $\Eoo$-algebra $R_{T_c}\in \CAlg(\Sp_{T_c})$ given by ``genuine ${T_c}$-equivariant $R$-cohomology''. Then, $\Loc_{T_c}(\ast; R)$ might be understood to mean $\Mod_{R_{T_c}}(\Sp_{T_c})$. We are interested in the following question: when is $\Loc_{T_c}(\ast; R)$ equivalent (as a symmetric monoidal category) to the $\infty$-category of modules over some $\Eoo$-ring $B$? It is not difficult to see that if this happens, then the $\Eoo$-ring $B$ will simply be $i_{T_c}^\ast(R_{T_c})$. (One could more generally ask when $\Loc_{T_c}(\ast; R)$ is equivalent to the $\infty$-category of quasicoherent sheaves on some spectral $R$-stack; but this obscures the key homotopical point.)

Let us suppose for simplicity that ${T_c}$ is of rank $1$, i.e., that $T_c = S^1$. Recall that the $\infty$-category $\Sp_{S^1}$ is compactly generated by $S^0$ (with the trivial $S^1$-action) and $(S^1/\mu_n)_+$ for $n\geq 2$. If $\lambda$ denote the $1$-dimensional complex representation of $\mu_n$, there is a cofiber sequence $(S^1/\mu_n)_+ \to S^0 \to S^{\lambda^n}$; so $\Sp_{S^1}$ is compactly generated by $S^0$ and $S^{\lambda^n}$ for $n\geq 2$. It follows that $\Loc_{S^1}(\ast; R) \simeq \Mod_{R_{S^1}}(\Sp_{S^1})$ is compactly generated by $R_{S^1}$ and $R_{S^1} \otimes S^{\lambda^n}$ for $n\geq 2$. 
If $R$ is complex-oriented, there is an equivalence $R_{S^1} \otimes S^{\lambda^n} \simeq \Sigma^2 R_{S^1}$. This lets us conclude that $\Loc_{S^1}(\ast; R)$ is compactly generated by the \textit{single} unit object $R_{S^1}$, so that \cite[Lemma 4.4]{greenlees-shipley-fixed-pt} implies that $\Loc_{S^1}(\ast; R) \simeq \Mod_{i_{S^1}^\ast(R_{S^1})}$.
\end{example}
\begin{remark}\label{sptc}
In contrast to the above discussion, if $R$ is not complex-oriented (or more generally does not admit a finite flat cover by a complex-oriented ring), then $\Loc_{T_c}(\ast; R)$ stands little chance of being compactly generated by the unit object. For example, if $R$ is the sphere spectrum, then $\Loc_{T_c}(\ast; R) \simeq \Sp_{T_c}$ is \textit{not} compactly generated by the unit object. Note, however, that the Barr-Beck-Lurie theorem (\cite[Theorem 4.7.3.5]{HA}) implies $\Sp_{S^1}$ is equivalent to the $\infty$-category of left modules over the $\E{1}$-ring $\End_{\Sp_{S^1}}\left(S^0 \oplus \bigoplus_{n\geq 2} (S^1/\mu_n)_+\right)$; this is \textit{not} an $\Eoo$-ring.
\end{remark}
In particular, if $\ld{T}_S := \spec S[\bX_\ast(T)]$ denotes the dual torus over the sphere spectrum, then one can run part of the proof of \cref{torus-satake} to conclude that 
$$\Shv_T(\Gr_T(\cc); S) \simeq \Fun(\bX_\ast(T), \Loc_{T_c}(\ast; S)) \simeq \Fun(\bX_\ast(T), \Sp_{T_c}) \simeq \Sp_{\ld{T}_c} \otimes \QCoh(B\ld{T}_S).$$
Here, we have identified $\Sp_{T_c} \simeq \Sp_{\ld{T}_c}$ (see \cref{char-cochar}). The discussion in \cref{sptc} shows that it is not clear how to view the right-hand side in terms of quasicoherent sheaves on some spectral stack. In particular, we see that already in the case of a torus, the coherent side of ``derived geometric Satake with spherical coefficients'' starts to deviate from the standard form of derived geometric Satake. It seems as though the appropriate analogue of the coherent side involves some combination of Hausmann's global group laws \cite{hausmann-global-group} and the spectral moduli stack of oriented formal groups (see \cite{gregoric-synthetic, piotr-synthetic}). We hope to approach this in future work via $T$-equivariant complex cobordism $\MU_T$.

At the moment, derived geometric Satake with spherical coefficients for a general reductive group over $\cc$ seems to require more technical setup than is currently available in the literature (although a version of the geometric Casselman-Shalika equivalence of \cite{geometric-casselman-shalika-ii} was discussed in \cite[Section 10]{lurie-icm}).