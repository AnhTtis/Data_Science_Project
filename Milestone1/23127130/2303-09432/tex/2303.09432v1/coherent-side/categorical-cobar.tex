\subsection{Putting it together}\label{categorical-cobar}
We will now explore one corollary of \cref{general-looped-satake}.
\begin{setup}\label{cobar-setup}
Let $\Prst$ be the $\infty$-category of compactly generated presentable $\infty$-categories and colimit-preserving functors which preserve compact objects.
Let $\cC\in \CAlg(\Prst)$, and let $\cd\in \CAlg(\LMod_\cC(\Prst))$ whose underlying object of $\LMod_\cC(\Prst)$ is dualizable. The unit map $i^\ast: \cC \to \cd$ defines a symmetric monoidal functor ${i'}^\ast: \cd \simeq \cd \otimes_\cC \cC \to \cd \otimes_\cC \cd$, and if $i_\ast: \cd \to \cC$ denotes the right adjoint to $i^\ast$, the following diagram commutes:
$$\xymatrix{
\cd \ar[r]^-{i_\ast} \ar[d]_-{{i'}^\ast} & \cC \ar[d]^-{i^\ast}\\
\cd \otimes_\cC \cd \ar[r]_-{{i'}_\ast} & \cd.
}$$
\end{setup}
\begin{prop}\label{completion-morita}
In \cref{cobar-setup}, there is a fully faithful colimit-preserving functor $\Tot(\cd^{\otimes_\cC \bull+1}) \hookrightarrow \cC$; we will denote its essential image by $\cC^\wedge_\cd$.
\end{prop}
\begin{proof}
The assumptions in \cref{cobar-setup} imply that the augmented cosimplicial diagram $ N(\Deltab_+) \to \Cat_\infty$ given by $\cd^{\otimes_\cC \bull+1}$ satisfies the assumptions of \cite[Corollary 4.7.5.3]{HA}. Therefore, the functor $\cC \to \Tot(\cd^{\otimes_\cC \bull+1})$ has a fully faithful left adjoint, as desired.
\end{proof}
\begin{observe}\label{kost-support}
Regard $\QCoh((\cM_{T,0})_\QQ)$ as a $\QCoh(\Bun_{\ld{B}}^0(\GG_{0,\QQ}^\vee))$-algebra via $\kappa: (\cM_{T,0})_\QQ \to \Bun_{\ld{B}}^0(\GG_{0,\QQ}^\vee)$. Then the completion $\QCoh(\Bun_{\ld{B}}^0(\GG_{0,\QQ}^\vee))^\wedge_{\QCoh((\cM_{T,0})_\QQ)}$ of \cref{completion-morita} with respect to $\QCoh((\cM_{T,0})_\QQ)$ can be identified with $\QCoh(\Bun_{\ld{B}}^0(\GG_{0,\QQ}^\vee)^\reg)$.
\end{observe}
\begin{example}
When $A$ is an $\Eoo$-$\QQ[\beta^{\pm 1}]$-algebra and $\GG = \hat{\GG}_a$, the Koszul duality equivalence of \cref{kd-springer} gives $\QCoh(\Bun_{\ld{B}}^0(\GG_{0,\QQ}^\vee)) \simeq \IndCoh((\tilde{\ld{\cN}} \times_{\ld{\g}} \{0\})/\ld{G})$; we define $\IndCoh((\tilde{\ld{\cN}} \times_{\ld{\g}} \{0\})/\ld{G})_\Kost$ to be the essential image of $\QCoh(\Bun_{\ld{B}}^0(\GG_{0,\QQ}^\vee)^\reg)$ under this equivalence. We remark that in this case, $\QCoh(\Bun_{\ld{B}}^0(\GG_{0,\QQ}^\vee)^\reg) \simeq \QCoh(\tilde{\ld{\g}}^\reg/\ld{G})$. Similarly, if $A = \KU$ and $\GG = \GG_m$, then $\QCoh(\Bun_{\ld{B}}^0(\GG_{0,\QQ}^\vee)^\reg) \simeq \QCoh(\tilde{\ld{G}}^\reg/\ld{G})$.
\end{example}
\begin{corollary}\label{regular-satake}
Fix a complex-oriented even-periodic $\Eoo$-ring $A$ and an oriented commutative $A$-group $\GG$.
Assume that the underlying $\pi_0 A$-scheme $\GG_0$ is $\GG_a$, $\GG_m$, or an elliptic curve $E$.
Suppose $G$ is a connected and simply-connected semisimple algebraic group over $\cc$.
Then there is an $\E{2}$-monoidal equivalence
$$\QCoh(\Bun_{\ld{B}}^0(\GG_{0,\QQ}^\vee)^\reg) \simeq \Loc_{T_c}^\gr(\Omega G_c; A) \otimes \QQ.$$
\end{corollary}
\begin{proof}
The $\Eoo$-coalgebra structure on $\pi_0 \cf_T(\Gr_G(\cc))^\vee$ defines a $\QCoh(\cM_{T,0})$-coalgebra structure on $\LMod_{\pi_0 \cf_T(\Gr_G(\cc))^\vee}(\QCoh(\cM_{T,0}))$.
The right-hand side of the equivalence of \cref{general-looped-satake} also admits a $\QCoh((\cM_{T,0})_\QQ)$-coalgebra structure, being the tensor product of $\QCoh((\cM_{T,0})_\QQ)$ with itself over $\QCoh(\Bun_{\ld{B}}^0(\GG_{0,\QQ}^\vee))$; and it is not difficult to check that the equivalence of \cref{general-looped-satake} is one of $\QCoh((\cM_{T,0})_\QQ)$-coalgebras.
In particular, there is a commutative diagram
$$\xymatrix{
& \QCoh((\cM_{T,0})_\QQ) \ar[d]^-{\ast \to \Gr_G(\cc)} \ar[dl] \\
\QCoh((\cM_{T,0})_\QQ \times_{\Bun_{\ld{B}}^0(\GG_{0,\QQ}^\vee)} (\cM_{T,0})_\QQ) \ar[r]^-\sim & \LMod_{\pi_0 \cf_T(\Gr_G(\cc))^\vee}(\QCoh(\cM_{T,0})) \otimes \QQ\\
}$$
which defines an equivalence of cosimplicial diagrams, and hence of their totalizations. The totalization of the cosimplicial diagram built from the functor $\QCoh(\cM_{T,0}) \to \Mod_{\pi_0 \cf_T(\Gr_G(\cc))^\vee}(\QCoh(\cM_{T,0}))$ defines an equivalence
$$\Tot(\LMod_{(\pi_0 \cf_T(\Gr_G(\cc))^\vee)^{\otimes \bull}}(\QCoh(\cM_{T,0}))) \simeq \coLMod_{\pi_0 \cf_T(\Gr_G(\cc))^\vee}(\QCoh(\cM_{T,0}));$$
note that by \cref{graded local systems}, this is in turn equivalent to $\Loc_{T_c}^\gr(\Omega G_c; A)$.
By \cref{completion-morita}, we also have 
$$\Tot(\QCoh((\cM_{T,0})_\QQ)^{\otimes_{\QCoh(\Bun_{\ld{B}}^0(\GG_{0,\QQ}^\vee))} \bull+1}) \simeq \QCoh(\Bun_{\ld{B}}^0(\GG_{0,\QQ}^\vee)^\reg).$$
This gives the desired equivalence.
\end{proof}
\begin{example}\label{regular-satake-special-cases}
When $A = \QQ[\beta^{\pm 1}]$ and $\GG = {\GG}_a$, we have $\Bun_{\ld{B}}^0(\GG_{0,\QQ}^\vee) = \tilde{\ld{\g}}/\ld{G}$. \cref{regular-satake} gives an $\E{2}$-monoidal equivalence
\begin{align*}
    \IndCoh((\tilde{\ld{\cN}} \times_{\ld{\g}} \{0\})/\ld{G})_\Kost & \simeq \QCoh(\tilde{\ld{\g}}^\reg/\ld{G}) \\
    & \simeq \Loc_{T_c}^\gr(\Omega G_c; \QQ[\beta^{\pm 1}]).
    %\coLMod_{C^T_\ast(\Gr_G(\cc); \QQ)} \otimes_\QQ \QQ[\beta^{\pm 1}].
\end{align*}
Note that $\tilde{\ld{\g}}/\ld{G}$ is isomorphic to the quotient $\ld{G} \backslash (\ld{G} \times^{\ld{N}} \ld{\fr{b}})/\ld{T}$; and \cite[Proposition 3.10]{safronov-implosion} says that $\ld{G} \times^{\ld{N}} \ld{\fr{b}}$ is the universal symplectic implosion (i.e., the symplectic implosion of $T^\ast G$). The relationship of this perspective to Langlands duality is closely related to the program of Ben-Zvi--Sakellaridis--Venkatesh \cite{sakellaridis-icm}: namely, the Hamiltonian $\ld{G} \times \ld{T}$-space $T^\ast(\ld{G}/\ld{N})$ acts as a ``kernel'' for the symplectic implosion functor from Hamiltonian $\ld{G}$-spaces to Hamiltonian $\ld{T}$-spaces.

Similarly, using \cref{looped-quantized-abg}, one can prove an equivalence between $\Loc_{\tilde{T}_c}^\gr(\Omega G_c; \QQ[\beta^{\pm 1}])$ and a localization of $\ld{\co}^\univ_\hbar$.
There is also an $\E{2}$-monoidal equivalence
$$\QCoh(\ld{\g}^\reg/\ld{G}) \simeq
\Loc_{G_c}^\gr(\Omega G_c; \QQ[\beta^{\pm 1}]);
%\coLMod_{C^G_\ast(\Gr_G(\cc); \QQ)} \otimes_\QQ \QQ[\beta^{\pm 1}]
$$
this follows from the analogue of \cref{loc and comod} for $G_c$-local systems and \cite[Proposition 2.2.1]{ngo-ihes}, which says that the classifying stack of the group scheme $\ld{J} = \spec \H^G_\ast(\Gr_G(\cc); \QQ)$ of regular centralizers is isomorphic to $\ld{\g}^\reg/\ld{G}$.
\end{example}
\begin{remark}
In the case of rank $1$, one can use \cref{regular-satake-special-cases} to show that if the circle $\SO(2)$ acts on $\SO(3)/\SO(2) = S^2$ by left multiplication, there is an equivalence
\begin{equation}\label{eq: SO2-equiv Loops S2 periodic}
    \Loc_{\SO(2)}^\gr(\Omega S^2; \QQ[\beta^{\pm 1}]) \simeq \QCoh(T^\ast(\AA^2)^\reg/\SL_2).
\end{equation}
Here, $\AA^2$ is equipped with the standard action of $\SL_2$, and $T^\ast(\AA^2)^\reg$ denotes the preimage of the regular locus of $\sl_2$ under the moment map $T^\ast(\AA^2) \to \sl_2$. This is because there is an equivalence
\begin{align*}
    \Loc_{\SO(2)}^\gr(\Omega S^2; \QQ[\beta^{\pm 1}]) & \simeq \Loc_{\SO(2)}^\gr(\Omega \SO(3); \QQ[\beta^{\pm 1}]) \otimes_{\Loc^\gr(\Omega \SO(2); \QQ[\beta^{\pm 1}])} \Vect_\QQ\\
    & \simeq \QCoh(\tilde{\sl}_2^\reg/\SL_2 \times_{B\GG_m} \spec(\QQ)) \simeq \QCoh(\SL_2\backslash T^\ast(\SL_2/N)^\reg),
\end{align*}
where the second equivalence uses \cref{regular-satake-special-cases} (i.e., the Arkhipov-Bezrukavnikov-Ginzburg equivalence over the regular locus). If $\ld{B}\subseteq \SL_2$ is a fixed Borel subgroup, the map $\tilde{\sl}_2^\reg/\SL_2 \to B\GG_m$ is given by the composite
$$\tilde{\sl}_2^\reg/\SL_2 \cong \ld{\fr{b}}^\reg/\ld{B} \to B\ld{B}\to B\ld{T} = B\GG_m.$$
However, $\SL_2/N \cong \AA^2 - \{0\}$, and there is an $\SL_2$-equivariant isomorphism $T^\ast(\AA^2)^\reg \cong T^\ast(\AA^2-\{0\})^\reg$. Let us remark that \cref{eq: SO2-equiv Loops S2 periodic} can be de-periodified to give an equivalence
$$\Loc_{\SO(2)}(\Omega S^2; \QQ) \simeq \QCoh(T^\ast[2](\AA^2)^\reg/\SL_2).$$
This is in fact related to the program of Ben-Zvi--Sakellaridis--Venkatesh \cite{sakellaridis-icm} applied to the ``Hecke period''; their program predicts a duality between the Hamiltonian $\PGL_2$-variety $T^\ast(\PGL_2/\GG_m)$ and the Hamiltonian $\SL_2$-variety $T^\ast(\AA^2)$.
\end{remark}
\begin{example}\label{KU-regular-satake}
When $A = \KU$ and $\GG = \GG_m$, we have $\Bun_{\ld{B}}^0(\GG_{0,\QQ}^\vee) = \tilde{\ld{G}}/\ld{G}$. Therefore, \cref{regular-satake} gives an $\E{2}$-monoidal equivalence
$$\QCoh(\tilde{\ld{G}}^\reg/\ld{G}) \simeq
\Loc_{T_c}^\gr (\Omega G_c; \KU) \otimes \QQ.
%\coLMod_{C_\ast^T(\Gr_G(\cc); \KU)} \otimes \QQ.
$$
Note that $\tilde{\ld{G}}/\ld{G}$ is isomorphic to the quotient $\ld{G} \backslash (\ld{G} \times^{\ld{N}} \ld{B})/\ld{T}$; and \cite[Discussion following Proposition 3.10]{safronov-implosion} says that $\ld{G} \times^{\ld{N}} \ld{B}$ is the universal group-valued symplectic implosion (i.e., the symplectic implosion of the internal fusion double of $G$). The relationship of this perspective to Langlands duality is closely related to a quasi-Hamiltonian analogue of the program of Ben-Zvi--Sakellaridis--Venkatesh \cite{sakellaridis-icm}, which we will explore in future work.

Similarly, one can show that there is an $\E{2}$-monoidal equivalence
$$\QCoh(\ld{G}^\reg/\ld{G}) \simeq
\Loc_{G_c}^\gr (\Omega G_c; \KU) \otimes \QQ.
$$
Were there a full $\KU$-theoretic geometric Satake equivalence, the above equivalence would be obtained by localization over the (open) regular locus of $\ld{G}$. The above equivalence is presumably related to \cite[Section 1.2]{cautis-kamnitzer}.
\end{example}
\begin{example}
Suppose $A$ is a complex-oriented even-periodic $\Eoo$-ring and $\GG$ is an oriented elliptic curve over $A$ (in the sense of \cite{elliptic-ii}). Let $E$ be the underlying classical scheme of $\GG$ over the classical ring $\pi_0(A)$, so that $E$ is an elliptic curve, and let $E^\vee$ be the dual elliptic curve. Then $\Bun_{\ld{B}}^0(\GG_0^\vee) = \Bun_{\ld{B}}^0(E^\vee)$, and \cref{regular-satake} gives an $\E{2}$-monoidal $\pi_0 A_\QQ$-linear equivalence
$$\QCoh(\Bun_{\ld{B}}^0(E^\vee)^\reg) \simeq 
\Loc_{T_c}^\gr (\Omega G_c; A) \otimes \QQ.
%\coLMod_{\cf_T(\Gr_G(\cc))^\vee}(\QCoh(\cM_T)) \otimes \QQ.
$$
This may be understood as a step towards a full $A$-theoretic analogue of the ABG equivalence.
\end{example}