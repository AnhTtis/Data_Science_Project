\subsection{Langlands duality over $\QQ[\beta^{\pm 1}]$}\label{sec: Q intersection}

We now turn to the coherent side of the geometric Satake equivalence. For general $\GG$, it is not obvious what the Langlands dual algebraic stack should be; we will discuss this in \cref{section-G-loops}. As a warmup, we will focus only on $\QQ[\beta^{\pm 1}]$ in this section (this is more for pedagogical purposes than originality).
\begin{definition}[(Additive) Kostant slice]\label{additive kostant slice}
Let $G$ be a connected reductive group over $\cc$, and fix the rest of notation as in \cref{group-notation}. Fix a principal nilpotent element $e\in \fr{n}$, and let $(e,f,h)$ be the associated $\sl_2$-triple in $\g$. Let $\g^e$ be the centralizer (so $\g = \g^e \oplus [e,\g]$), and let $\cS := f + \g^e\subseteq \g^\reg$ be the Kostant slice. The composite $f + \g^e \to \g \to \g\mmod G \cong \fr{t}\mmod W$ is an isomorphism, by \cite{kostant-lie-group-reps}.

Let $\tilde{\g} = \fr{b} \times_B G$ be the Grothendieck-Springer resolution, so that $\tilde{\g}/G \simeq \fr{b}/B$. We will often work with $\tilde{\g}^\ast$ instead, defined as $\fr{b}^\ast \times_B G$. There is a map $\tilde{\chi}: \tilde{\g} \to \fr{t}$ which sends a pair $(x\in \Ad_g(\fr{b}))$ to the inverse image under the isomorphism $\fr{t} \to \fr{b} \to \fr{b}/\fr{n}$ of the image of $g^{-1} x\in \fr{b}$. Let $\tilde{\cS}$ denote the fiber product $\cS\times_\g \tilde{\g}$, so that $\tilde{\cS} \subseteq \tilde{\g}^\reg = \g^\reg \times_\g \tilde{\g}$. Then, Kostant's result on the Kostant slice implies formally that the composite $\tilde{\cS} \to \tilde{\g} \xar{\tilde{\chi}} \fr{t}$ is an isomorphism. We will often abusively write the inclusion of $\tilde{\cS}$ as a map $\kappa: \fr{t} \to \tilde{\g}$.

In fact, we will only care about the composite $\fr{t} \to \tilde{\g} \to \tilde{\g}/G$ below, so we will also denote it by $\kappa$. If we identify $\tilde{\g}/G \cong \fr{b}/B$, then the map $\kappa$ admits a simple description: it is the composite $\fr{t} \to \fr{b} \to \fr{b}/B$ which sends $x\mapsto f+x$. This is proved, for instance, in \cite[Proposition 19]{kostant-lie-group-reps}, where it is shown that there is a unique map $\mu: f+\fr{t} \to N$ such that $\Ad_{\exp(\mu(x))}(x) \in f + \g^e$; this further implies that the image of any $x\in \fr{t}$ under the map $\fr{t} \to \fr{t}\mmod W \xar{\kappa} \g$ can be identified with $\Ad_{\exp(\mu(x+f))}(x+f)$.
\end{definition}

Fix a nondegenerate invariant bilinear form on $\g$, to identify $\g$ with $\g^\ast$. The first main result of this section is the following; it is essentially equivalent to \cite[Proposition 2.8]{bfm} and the rationalization of \cite[Theorem 6.1]{homology-langlands}.
\begin{theorem}\label{once-looped-satake}
Let $G$ be a connected and simply-connected semisimple algebraic group over $\cc$. Let $A$ be an $\Eoo$-$\QQ[\beta^{\pm 1}]$-algebra, and let $\GG = \GG_a$ (so $\cM_T$ is the affine space $\fr{t}[2]$ over $A$). 
View $\ld{\fr{t}}^\ast$, $\ld{\fr{n}}$, $\ld{\g}$, and $\ld{B}$ as schemes over $\QQ$.
Then $\QCoh(\ld{\fr{t}}^\ast)$ admits the structure of a module over $\IndCoh((\tilde{\ld{\cN}} \times_{\ld{\g}} \{0\})/\ld{G})$, where the fiber product is (always) derived, such that there is an equivalence
$$\End_{\IndCoh((\tilde{\ld{\cN}} \times_{\ld{\g}} \{0\})/\ld{G})}(\QCoh(\ld{\fr{t}}^\ast)) \otimes_\QQ \pi_0 A \simeq \LMod_{\pi_0 C_\ast^T(\Gr_G(\cc); A)} = \Loc^\gr_{T_c}(G_c; A).$$
\end{theorem}
\begin{remark}\label{iwahori-satake}
Recall from \cite{abg-iwahori-satake} that there is an Iwahori-Satake equivalence $\IndCoh((\tilde{\ld{\cN}} \times_{\ld{\g}} \{0\})/\ld{G}) \simeq \Shv(\Gr_G)^I$ over $\cc$, where the right-hand side is normalized appropriately. One should therefore regard \cref{once-looped-satake} as a bar construction of the restriction of this equivalence (lifted from $\cc$ to $\QQ$) to the regular locus, and more optimistically as a first step towards an alternative proof. See also \cref{regular-satake-special-cases} for the equivalence resulting from ``undoing'' the bar construction.
\end{remark}
We now turn to the proof of \cref{once-looped-satake}. For the next two results, we only work on one side of Langlands duality, so we drop the ``check''s for notational simplicity. Note that $(\tilde{\ld{\cN}} \times_{\ld{\g}} \{0\})/\ld{G} \cong (\ld{\fr{n}} \times_{\ld{\g}} \{0\})/\ld{B}$; it will be more convenient to work with the latter description.
\begin{lemma}\label{kd-springer}
There is a Koszul duality equivalence $\QCoh(\tilde{\g}^\ast[2]/G) \simeq \IndCoh((\fr{n} \times_{\g} \{0\})/B)$.
\end{lemma}
%\begin{proof}
%Observe that $\fr{n} \times_{\g} \{0\} \cong \spec \Sym(\fr{n}^\ast \times_{\g^\ast} 0)$. Since $\g = \fr{n} \oplus \fr{b}^-$, we see that $\fr{n}^\ast \times_{\g^\ast} 0$ is equivalent as a $B$-equivariant module to $\fr{b}[-1]$. (Here, we use the invariant bilinear form on $\g$.) It follows that $\fr{n} \times_{\g} \{0\} \cong \spec \Sym(\fr{b}[-1])$, so that $\IndCoh((\fr{n} \times_{\g} \{0\})/B) \simeq \IndCoh(\Sym(\fr{b}[-1]))^B$. By Koszul duality, $\IndCoh(\Sym(\fr{b}^\ast[-1]))^B \simeq \QCoh(\fr{b}^\ast/B)$; this implies the claim, since $\fr{b}^\ast/B \simeq \tilde{\g}^\ast/G$.
%\end{proof}
We will give two proofs of the following fact.
\begin{prop}[{Variant of \cite[Proposition 2.8]{bfm}}]\label{bfm-self-intersect}
Work over a field $k$ of characteristic $0$, and view $\QCoh(\fr{t}^\ast)$ as a $\QCoh(\tilde{\g}^\ast/G)$-module via the Kostant slice $\kappa: \fr{t}^\ast \to \tilde{\g}^\ast$. Then there is an equivalence
$\End_{\QCoh(\tilde{\g}^\ast/G)}(\QCoh(\fr{t}^\ast)) \simeq \QCoh((T^\ast T)^\bl)$.
\end{prop}
\begin{proof}[First proof of \cref{bfm-self-intersect}]
We may identify $\End_{\QCoh(\tilde{\g}^\ast/G)}(\QCoh(\fr{t}^\ast))$ with $\QCoh(\fr{t}^\ast \times_{\tilde{\g}^\ast/G} \fr{t}^\ast)$. We will show, in fact, that there is a Cartesian square
\begin{equation}\label{intersection-kostant}
    \xymatrix{
    (T^\ast T)^\bl \ar[r] \ar[d] & \fr{t}^\ast \ar[d]^-\kappa\\
    \fr{t}^\ast \ar[r]_-\kappa & \tilde{\g}^\ast/G \simeq \fr{b}^\ast/B.
    }
\end{equation}
This is an analogue of \cite[Proposition 2.2.1]{ngo-ihes} and \cite[Proposition 2.8]{bfm}.
(Note that since $\fr{t}^\ast \to \tilde{\g}^\ast$ lands in the open locus $\tilde{\g}^{\ast, \reg}$, it does not matter whether we intersect $\fr{t}^\ast$ with itself in $\tilde{\g}^\ast/G$ or in $\tilde{\g}^{\ast, \reg}/G$; indeed, the intersection $\tilde{\g}^{\ast, \reg} \times_{\tilde{\g}^\ast} \tilde{\g}^{\ast, \reg}$ is just $\tilde{\g}^{\ast, \reg}$.) In what follows, it will be convenient (notationally) to use the chosen nondegenerate invariant bilinear form on $\g$ to identify $\fr{b}^\ast$ with the opposite Borel $\fr{b}^-$ and $N$ with its opposite unipotent, and then to flip the role of $\fr{b}$ and $\fr{b}^-$, etc. 

Recall that the Kostant slice $\cS\subseteq \g$ is transverse to the regular $G$-orbits, and intersects each orbit exactly once; this implies that the image of the map $\kappa: \fr{t} \to \tilde{\g}$ is transverse to the regular $G$-orbits on $\tilde{\g}$, and intersects each orbit exactly once. In particular, if $C$ denotes the locally closed subvariety of $\tilde{\g} \times G$ consisting of pairs $(x,g)$ with $x\in \tilde{\g}^\reg$ and $\Ad_g(x) = x$, then $C\mmod G = \fr{t} \times_{\tilde{\g}/G} \fr{t}$ (so we may assume without loss of generality that $x\in \fr{t}$). To compute $C\mmod G$, one can reduce to the case when $G$ has semisimple rank $1$ by the argument of \cite[Section 4.3]{bfm}. To work out this case, we will assume $G = \SL_2, \PGL_2$.

There are ``two'' ways to compute in these cases; we will describe both, because each has its own conceptual advantages when generalizing to the multiplicative case (for instance). First, we present the argument which is essentially present in \cite{bfm}; for this, we will assume $G = \SL_2$. The Grothendieck-Springer resolution $\tilde{\g} = T^\ast(\AA^2-\{0\})/\GG_m$ is the total space of $\co(-1) \oplus \co(-1)$ over $\PP^1$; we will think of a point in $\tilde{\g}$ as a pair $(x\in \sl_2, \ell\subseteq \cc^2)$ such that $x$ preserves $\ell$. The Kostant slice $\kappa:\fr{t} \cong \AA^1 \to \tilde{\g}$ is the map sending $\lambda \in \AA^1$ to the pair $(x, \ell)$ with $x = \begin{psmallmatrix}
0 & \lambda^2 \\
1 & 0
\end{psmallmatrix}$ and $\ell = [\lambda: 1]$. Indeed, this is essentially immediate from the requirement that the following diagram commutes:
$$\xymatrix{
\AA^1 \cong \fr{t} \ar[r]^-\kappa \ar[d]_-{\lambda \mapsto \lambda^2} & \tilde{\sl}_2 \ar[d]\\
\AA^1 \cong \fr{t}\mmod W \ar[r]^-\kappa_-{\lambda\mapsto \begin{psmallmatrix}
0 & \lambda \\
1 & 0
\end{psmallmatrix}} & \sl_2.
}$$
Moreover, the $\SL_2$-action on $\tilde{\g}$ sends $g\in \SL_2$ and $(x,\ell)$ to $(\Ad_g(x), g\ell)$. If $g = \begin{psmallmatrix}
a & b \\
c & d
\end{psmallmatrix}$, we compute that
$$\Ad_g\begin{pmatrix}
0 & \lambda^2 \\
1 & 0
\end{pmatrix} = \begin{pmatrix}
bd-ac\lambda^2 & (a\lambda)^2 - b^2 \\
d^2 - (c\lambda)^2 & ac\lambda^2 - bd
\end{pmatrix}, \ g\cdot [\lambda: 1] = [a\lambda + b: c\lambda + d].$$
From this, we see that $\Ad_g(x) = x$ if and only if $a = d$ and $b = c\lambda^2$, in which case $g$ also fixes $[\lambda: 1]$. In other words, $g = \begin{psmallmatrix}
a & c\lambda^2 \\
c & a
\end{psmallmatrix}$ with $a,c\in k$; in order for $\det(g) = 1$, we need $a^2-c^2\lambda^2=1$. When $\lambda \neq 0$, both $x$ and $g$ are diagonalized by the matrix $\tfrac{1}{2}\begin{psmallmatrix}
1 & -1 \\
-\lambda^{-1} & -\lambda^{-1}
\end{psmallmatrix}\in \SL_2$: the diagonalization of $x$ is $\begin{psmallmatrix}
\lambda & 0 \\
0 & \lambda^{-1}
\end{psmallmatrix}$, and the diagonalization of $g$ is $\begin{psmallmatrix}
t & 0 \\
0 & w
\end{psmallmatrix}$ where $2a = t+w$ and $2\lambda c = t-w$. Since we have $\det(g) = a^2 - (c\lambda)^2 = 1$, we have $w = t^{-1}$. This shows that if $k$ is not of characteristic $2$, then $\fr{t} \times_{\tilde{\sl}_2/\SL_2} \fr{t} \cong \spec k[\lambda, t^{\pm 1}, \tfrac{t-t^{-1}}{\lambda}]$.

The ``second'' way to reach this calculation (still with $G = \SL_2$) is to use the fact that $\kappa: \fr{t} \to \tilde{\g}/G$ can be identified with the composite $\fr{t} \to \fr{b} \to \fr{b}/B$ sending $x\mapsto f+x$. Then, $\fr{t} \times_{\fr{b}/B} \fr{t}$ is isomorphic to the subvariety of $\fr{t} \times B$ consisting of pairs $(x,g)$ with $x\in \fr{t}$ and $\Ad_g(x+f) = x+f$. If $g = \begin{psmallmatrix}
a & 0 \\
b & a^{-1}
\end{psmallmatrix}\in B$, then
$$\Ad_g \begin{pmatrix}
x & 0 \\
1 & -x
\end{pmatrix} = \begin{pmatrix}
x & 0 \\
2a^{-1}bx + a^{-2} & -x
\end{pmatrix}.$$
Therefore, $\Ad_g(x+f) = x+f$ if and only if 
$$2a^{-1}bx + a^{-2} = 1,$$
which forces $b = \tfrac{a-a^{-1}}{2x}$. This implies that $\fr{t} \times_{\fr{b}/B} \fr{t}$ is isomorphic to $\spec k[x, a^{\pm 1}, \tfrac{a-a^{-1}}{x}]$, as desired.

We will now do the calculation with $G = \PGL_2$ via the second method. Again, $\fr{t} \times_{\fr{b}/B} \fr{t}$ is isomorphic to the subvariety of $\fr{t} \times B$ consisting of pairs $(x,g)$ with $x\in \fr{t}$ (identified with the matrix $\begin{psmallmatrix}
x & 0 \\
0 & 0
\end{psmallmatrix} \in \gl_2$) and $\Ad_g(x+f) = x+f$. If $g = \begin{psmallmatrix}
a & 0 \\
b & 1
\end{psmallmatrix}\in B$, then 
$$\Ad_g \begin{pmatrix}
x & 0 \\
1 & 0
\end{pmatrix} = \begin{pmatrix}
x & 0 \\
(bx + 1) a^{-1} & 0
\end{pmatrix}.$$
Therefore, $\Ad_g(x+f) = x+f$ if and only if 
$$(bx + 1) a^{-1} = 1,$$
which forces $b = \tfrac{a-1}{x}$. This implies that $\fr{t} \times_{\fr{b}/B} \fr{t}$ is isomorphic to $\spec k[x, a^{\pm 1}, \tfrac{1-a}{x}]$, as desired.
\end{proof}
\begin{proof}[Second proof of \cref{bfm-self-intersect}]
As in the first proof of \cref{bfm-self-intersect}, it will be convenient to use the chosen nondegenerate invariant bilinear form on $\g$ to identify $\fr{b}^\ast$ with the opposite Borel $\fr{b}^-$ and $N$ with its opposite unipotent, and then to flip the role of $\fr{b}$ and $\fr{b}^-$, etc. We will prove the following variant of \cref{bfm-self-intersect}, which in turn implies the desired result: view $\QCoh(\fr{t}^\ast\mmod W)$ as a $\QCoh({\g}^\ast/G)$-module via the Kostant slice. Then there is an equivalence
$\End_{\QCoh({\g}^\ast/G)}(\QCoh(\fr{t}^\ast\mmod W)) \simeq \QCoh((T^\ast T)^\bl\mmod W)$.

Let $\chi$ be a nondegenerate character on $\fr{n}^-$. The $N^-$-action on $G$ via conjugation induces a Hamiltonian $N^-$-action on $T^\ast G$; let $N^- {}_\chi\backslash (T^\ast G)/_\chi N^-$ denote the bi-Whittaker reduction of $T^\ast G$ with respect to this $N^-$-action at the character $\chi\in \fr{n}^{-,\ast}$. Then $(T^\ast T)^\bl\mmod W \cong N^- {}_\chi\backslash (T^\ast G)/_\chi N^-$; see \cite[Theorem 6.3]{teleman-icm}, for instance. There is a Morita equivalence between $\QCoh(\g^\ast/G)$ and $\QCoh(T^\ast G)$ (equipped with the convolution monoidal structure); under this equivalence, the $\QCoh(\g^\ast/G)$-module $\QCoh(\g^\ast/_\chi N^-)$ is sent to the $\QCoh(T^\ast G)$-module $\QCoh((T^\ast G)/_\chi N^-)$. We conclude the series of equivalences:
\begin{align*}
    \QCoh((T^\ast T)^\bl\mmod W) & \simeq \QCoh(N^- {}_\chi\backslash (T^\ast G)/_\chi N^-) \\
    & \simeq \End_{\QCoh(T^\ast G)} (\QCoh((T^\ast G)/_\chi N^-))\\
    & \simeq \End_{\QCoh(\g^\ast/G)}(\QCoh(\g^\ast/_\chi N^-)).
\end{align*}
However, Kostant's theorem identifies $\g^\ast/_\chi N^-$ with $\fr{t}^\ast\mmod W$ (viewed as a substack of $\g^\ast/G$ via the Kostant slice), which finishes the proof.
\end{proof}
\begin{proof}[Proof of \cref{once-looped-satake}]
By \cref{t-homology-grg}, we have $\H_0^T(\Gr_G(\cc); A) = \pi_0 \cf_T(\Gr_G(\cc))^\vee \cong \co_{(T^\ast T)^\bl}$. It follows that $\LMod_{\H_\ast^T(\Gr_G(\cc); A)} \simeq \QCoh((T^\ast \ld{T})^\bl_A)$. Since $\End_{\IndCoh((\tilde{\ld{\cN}} \times_{\ld{\g}} \{0\})/\ld{G})}(\QCoh(\ld{\fr{t}}^\ast)) \simeq \QCoh((T^\ast \ld{T})^\bl)$ by \cref{kd-springer} and \cref{bfm-self-intersect}, we conclude the desired result.
\end{proof}
\begin{remark}
So far, we have not emphasized the role of Whittaker reduction in the above story (except for the second proof of \cref{bfm-self-intersect}). However, we take a moment to describe this briefly, since it is a key aspect of Langlands duality. Recall that a theorem of Kostant's gives an isomorphism $(f+\fr{b})/N \cong \cS = f + \g^e$. In terms of Whittaker reduction, this says that $\cS \cong \g/_\chi N^-$. Since \cref{bfm-self-intersect} is concerned with $\tilde{\g}$ instead of $\g$, we need a slight variant of this statement. Namely, recall the map $\pi: \tilde{\g} \to \g$, let $\mu: \g \to \fr{n}$ be the moment map for the adjoint $N$-action on $\g$, and let $\tilde{\mu}$ denote the composite $\tilde{\g} \to \g \to \fr{n}$. Then $\mu^{-1}(f)$ is the variety $f+\fr{b}$, so that $\tilde{\mu}^{-1}(f)$ is the subscheme of $\tilde{\g}$ spanned by those pairs $(\fr{b}', y\in \fr{b}'\cap (f+\fr{b}))$. Kostant's result implies that there is an isomorphism $\tilde{\mu}^{-1}(f)/N^- \xar{\sim} \fr{t}$. 
%(This map is easily seen to be surjective: an inverse map is given by sending $x\in \fr{t}$ to the pair $(\fr{b}^-, f+x\in \fr{b}^-)$; injectivity is equivalent to the claim that the space of pairs $(\fr{b}', y\in \fr{n}'\cap (f+\fr{b}))$ is an $N$-torsor, which is more difficult to prove.) 
Whittaker reduction is a key aspect of the Langlands-dual side of \cref{once-looped-satake}: it is needed to even define the action of $\QCoh(\tilde{\g}^\ast/G)$ on $\QCoh(\fr{t}^\ast)$.
\end{remark}

\begin{example}\label{witt-example}
Note that \cref{once-looped-satake} implies that $\H_0(\Omega G_c; \QQ[\beta^{\pm 1}])$ can be identified with the ring of functions on the centralizer $Z_f(\ld{G})$ of a regular nilpotent element $f\in \ld{\g}$ over $\QQ$. In type $A$ at least, one can directly check that there is such an isomorphism. (Exactly the same argument works in the K-theoretic and elliptic cases, too; in the K-theoretic case, one instead considers the centralizer of a regular \textit{unipotent} element $f\in \ld{G}$.) For instance, if $\ld{G} = \SL_n$, the centralizer $Z_f(\ld{G})$ is the direct product of $\mu_n$ with a connected (commutative) unipotent group $U_n$. If $(x_1, \cdots, x_{n-1})$ is a point in $U_n$ (corresponding to the element in $Z_f(\SL_n)$ given by the $n\times n$-matrix whose $j$th row is $(0, \cdots, 0, 1, x_1, \cdots, x_{n-j})$), the group operation is given by
$$(x_1, \cdots, x_{n-1}) \cdot (y_1, \cdots, y_{n-1}) = (x_1 + y_1, \cdots, x_{n-1} + x_{n-2} y_1 + \cdots + x_1 y_{n-2} + y_{n-1}).$$
The group scheme $U_n$ is isomorphic over $\QQ$ to $\GG_a^{\times n-1}$, via Newton's identities for the transformation law for expressing the power sum symmetric polynomials in terms of the elementary symmetric polynomials. For instance, the isomorphism between $U_6 \subseteq Z_f(\SL_6)$ and $\GG_a^{\times 5}$ is given by the map
\begin{align}
    (x_1, \cdots, x_5) & \mapsto (x_1, x_1^2-2x_2, x_1^3-3x_1 x_2+3x_3, x_1^4 + 2x_2^2 - 4x_4-4x_2x_1^2+4x_1x_3, \nonumber\\
    & \ \ \ x_1^5-5x_1^3 x_2 + 5x_1^2 x_3 - 5x_1 (x_4 - x_2^2) - 5x_2 x_3 + 5x_5). \label{eq: power sum}
\end{align}
In general, the transformation can be determined by extracting the coefficient of $(-t)^n/n$ in the power series $\log\left(\sum_{j\geq 0} x_j (-t)^j\right)$.

On the other hand, $G_c$ is a maximal compact subgroup of $\PGL_n(\cc)$, and there is a homotopy equivalence $\Omega \PGL_n(\cc) \simeq \Z/n \times \Omega \SU(n)$, so that
$$\H_0(\Omega \PGL_n(\cc); \QQ[\beta^{\pm 1}]) \simeq \QQ[x^{\pm 1}]/(x^n-1) \otimes_\Z \H_0(\Omega \SU(n); \Z[\beta^{\pm 1}]).$$
Under Langlands duality, the $\mu_n$ factor in $Z_f(\SL_n)$ comes from the first tensor factor. Similarly, $\spec \H_0(\Omega \SU(n); \Z[\beta^{\pm 1}])$ is a connected unipotent group scheme: for instance, there is a Bott periodicity equivalence $\Omega \SU \simeq \BU$ (where $\SU = \colim_{n\to\infty} \SU(n)$), so $\spec \H_0(\Omega \SU; \Z[\beta^{\pm 1}])$ can be identified with the ring of functions over the big Witt ring scheme $\WW$ over $\Z$. This group scheme is unipotent over $\Z$, and the ghost components define an isomorphism to $\prod_{\Z_{\geq 0}} \GG_a$ upon rationalization (see \cite[Theorem II.6.7]{serre-local-fields} for a textbook reference). The group scheme $\spec \H_0(\Omega \SU(n); \Z[\beta^{\pm 1}])$ is a quotient of $\WW$ (hence is unipotent): in fact, it is isomorphic to the group scheme $\WW_{n-1}$ of big Witt vectors of length $n-1$. Since this is rationally isomorphic to $\GG_a^{\times n-1}$, we see that
$$\spec \H_0(\Omega \PGL_n(\cc); \QQ[\beta^{\pm 1}]) \cong \mu_n \times \WW_{n-1} \cong \mu_n \times \GG_a^{\times n-1} \cong Z_f(\SL_n),$$
as desired. Note, however, that the isomorphism $\WW_{n-1} \cong U_n\subseteq Z_f(\SL_n)$ is somewhat tricky to write down in coordinates. As an example, using the formula for the ghost components in the big Witt vectors, it is easy to see that the formula \cref{eq: power sum} implies that the isomorphism $Z_f(\SL_6) \supseteq U_6 \xar{\sim} \WW_5$ sends $(x_1, \cdots, x_5)$ to the Witt vector
$$(x_1, \cdots, x_5) \mapsto (x_1, -x_2, x_3-x_1x_2, x_1x_3-x_2x_1^2-x_4, x_5-x_1^3 x_2 + x_1^2 x_3 - x_1 (x_4 - x_2^2) - x_2 x_3).$$
\end{example}
\begin{remark}
One special feature of rational homology which sets it apart from K-theory or elliptic cohomology is that it can be de-periodified. On the Langlands-dual side, this equips the relevant geometric objects with a $\GG_m$-action, i.e., with a grading. Continuing \cref{witt-example}, there is still an isomorphism 
$$\H_\ast(\Omega \PGL_n(\cc); \QQ) \simeq \QQ[x^{\pm 1}]/(x^n-1) \otimes_\Z \H_\ast(\Omega \SU(n); \Z),$$
and there is still an isomorphism $\spec \H_\ast(\Omega \SU(n); \Z) \cong \WW_{n-1}$. Here, the grading on $\H_\ast(\Omega \SU(n); \Z)$ by half the homological degree corresponds to the $\GG_m$-action on $\WW_{n-1}$ defined as follows: if we view $\WW_{n-1}(R) = 1+R[t]/t^n \subseteq (R[t]/t^n)^\times$, the coordinate $t$ is given weight $-1$.
This defines a grading on $Z_f(\SL_n)$, which can also be described directly in general as follows (see \cite{kostant-lie-group-reps}). The element $2\rho=\sum_{\alpha\in \Phi^+} \alpha\in \bX^\ast(T) \cong \bX_\ast(\ld{T})$ defines a homomorphism $2\rho: \GG_m \to \ld{T}$, which defines a $\GG_m$-action on $\ld{\g}$. This $\GG_m$-action stabilizes the Kostant section $f + \ld{\g}^e$, and hence defines a $\GG_m$-action on $Z_e(\ld{G})$; this is the grading on $\co_{Z_e(\ld{G})}$ corresponding to half the homological grading on $\H_\ast(\Omega G_c; \QQ)$.
\end{remark}

\begin{remark}
In \cite{bfm}, the following analogue of \cref{intersection-kostant} is established (over $\cc$, but this does not affect the statement): there is a Cartesian square
\begin{equation}\label{tmmodw-bfm}
    \xymatrix{
    (T^\ast \ld{T})^\bl \mmod W \ar[r] \ar[d] & \fr{t}\mmod W \ar[d]^-\kappa\\
    \fr{t}\mmod W \ar[r]_-\kappa & \ld{\g}/\ld{G},
    }
\end{equation}
where the top-left corner can be identified with $\spec \pi_0 C^G_\ast(\Gr_G(\cc); \QQ)$. We can take the fiber product of \cref{intersection-kostant} with itself over \cref{tmmodw-bfm} to obtain a Cartesian square
\begin{equation}\label{springer-kostant}
    \xymatrix{
    (T^\ast \ld{T})^\bl \times_{(T^\ast \ld{T})^\bl \mmod W} (T^\ast \ld{T})^\bl \ar[r] \ar[d] & \fr{t} \times_{\fr{t}\mmod W} \fr{t} \ar[d]^-\kappa\\
    \fr{t} \times_{\fr{t}\mmod W} \fr{t} \ar[r]_-\kappa & (\tilde{\ld{\g}} \times_{\ld{\g}} \tilde{\ld{\g}})/\ld{G}.
    }
\end{equation}
Using \cref{once-looped-satake} and the above discussion, one can use \cref{springer-kostant} to show that $\End_{\QCoh((\tilde{\ld{\g}} \times_{\ld{\g}} \tilde{\ld{\g}})/\ld{G})}(\QCoh(\fr{t} \times_{\fr{t}\mmod W} \fr{t}))$ can be identified with $\LMod_{\pi_0 C^T_\ast(\Fl_G(\cc); \QQ[\beta^{\pm 1}])}$. This can be viewed as a ``once-looped'' version of Bezrukavnikov's equivalence from \cite{bezrukavnikov-two-geometric}.
%$T\backslash LG/T \simeq T\backslash LG/G \times_{G\backslash LG / G} G\backslash LG/T$. In 
\end{remark}

One can quantize \cref{once-looped-satake} as follows.
\begin{definition}
Following \cite{univ-cat-o}, define the (Langlands dual) \textit{universal category} $\ld{\co}_\hbar^\univ$ to be $\DMod_\hbar(\ld{G}/\ld{N})^{(\ld{G}\times \ld{T}, w)} \simeq U_\hbar(\ld{\g})\modc^{\ld{N}, (\ld{T}, w)}$. The $\infty$-category $\ld{\co}_\hbar^\univ$ is a quantization of $\QCoh(\ld{\fr{b}}^-/\ld{B}^-)$, since there are isomorphisms
$$\ld{\fr{b}}^-/\ld{B}^- \cong \tilde{\ld{\g}}/\ld{G} \cong \ld{T}\backslash T^\ast(\ld{G}/\ld{N})/\ld{G}.$$
\end{definition}
\begin{theorem}\label{looped-quantized-abg}
Let $A$ be an $\Eoo$-$\cc[\beta^{\pm 1}]$-algebra, and let $G$ be a connected and simply-connected semisimple algebraic group or a torus over $\cc$. Then there is a Kostant functor $\ld{\co}^\univ_\hbar \to \QCoh(\ld{\fr{t}}^\ast \times \AA^1_\hbar)$ and a left $A\pw{\hbar}$-linear equivalence 
$$\LMod_{\pi_0 C^{\tilde{T}}_\ast(\Gr_G(\cc); A)} \simeq \End_{\ld{\co}^\univ_\hbar}(\QCoh(\ld{\fr{t}}^\ast \times \AA^1_\hbar)).$$
\end{theorem}
\begin{proof}[Proof sketch; compare to the second proof of \cref{bfm-self-intersect}]
We will assume $A = \cc[\beta^{\pm 1}]$, so that $\pi_0 C^{\tilde{T}}_\ast(\Gr_G(\cc); A)$ is a $2$-periodification of $\pi_\ast C^{\tilde{T}}_\ast(\Gr_G(\cc); \cc)$. Let $\cH(\tilde{\ld{\fr{t}}}^\ast, \tilde{W}^\aff)$ be the nil-Hecke algebra associated to $\tilde{\ld{\fr{t}}}^\ast \cong \ld{\fr{t}}^\ast \oplus \cc\alpha_0$, and let $e = \tfrac{1}{\# W}\sum_{w\in W} w\in \QQ[W]$ be the symmetrizer idempotent. Using \cref{cohomology-grg}, one can then show that $\H^{\tilde{T}}_\ast(\Gr_G(\cc); \cc)$ is isomorphic to $\co_{\ld{\fr{t}}^\ast} \otimes_{\co_{\ld{\fr{t}}^\ast \mmod W}} e\cH(\tilde{\ld{\fr{t}}}^\ast, \tilde{W}^\aff)e$, where the loop rotation parameter $\hbar$ corresponds to the affine root $\alpha_0$; see \cite{kostant-kumar}. This implies that $\LMod_{\pi_0 C^{\tilde{T}}_\ast(\Gr_G(\cc); A)}$ can be identified with $\QCoh(\ld{\fr{t}}^\ast) \otimes_{\QCoh(\ld{\fr{t}}^\ast \mmod W)} \LMod_{e\cH(\tilde{\ld{\fr{t}}}^\ast, \tilde{W}^\aff)e}$.

We now construct the Kostant functor $\kappa_\hbar: \ld{\co}^\univ_\hbar \to \QCoh(\ld{\fr{t}}^\ast \times \AA^1_\hbar)$. Recall that the Kostant functor $\HC_\hbar(\ld{G}) \to \QCoh(\ld{\fr{t}}^\ast\mmod W \times \AA^1_\hbar)$ is given by the composite
$$\HC_\hbar(\ld{G}) = \DMod_\hbar(\ld{G})^{(\ld{G}\times \ld{G}, w)} \to \DMod_\hbar(\ld{G})^{(\ld{G}, w)} \to \DMod_\hbar(\ld{N}^-\backslash_\chi \ld{G})^{(\ld{G}, w)}.$$
However, the final term is equivalent to $U_\hbar(\ld{\g})\modc^{(\ld{N}^-, \chi)}$, which in turn can be identified with $\QCoh(\ld{\fr{t}}^\ast\mmod W \times \AA^1_\hbar)$ by the Skryabin equivalence (see the appendix of \cite{premet}).
Similarly, the desired Kostant functor on $\ld{\co}^\univ_\hbar$ is also given by Whittaker averaging: there is a composite
$$\ld{\co}^\univ_\hbar = \DMod_\hbar(\ld{G}/\ld{N})^{(\ld{G}\times \ld{T}, w)} \to \DMod_\hbar(\ld{G}/\ld{N})^{(\ld{T}, w)} \xar{\mathrm{Av}_!^\chi} \DMod_\hbar(\ld{N}^-\backslash_\chi \ld{G}/\ld{N})^{(\ld{T}, w)}.$$
However, the final term is equivalent by a standard argument to $\DMod_\hbar(\ld{T})^{(\ld{T}, w)} \simeq \QCoh(\ld{\fr{t}}^\ast \times \AA^1_\hbar)$. Note that by construction, the following diagram commutes:
$$\xymatrix{
\HC_\hbar(\ld{G}) \ar[d] \ar[r] & \QCoh(\ld{\fr{t}}^\ast\mmod W \times \AA^1_\hbar) \ar[d] \\
\ld{\co}^\univ_\hbar \ar[r] & \QCoh(\ld{\fr{t}}^\ast \times \AA^1_\hbar).
}$$
Here, the horizontal maps are given by the Kostant functors.
%There is an isomorphism $\Gamma(\ld{G}/\ld{N}; \cd_{\ld{G}/\ld{N}})^{\ld{G} \times \ld{T}} \cong U(\ld{\fr{t}}) \cong \co_{\ld{\fr{t}}^\ast}$, where $\cd_{\ld{G}/\ld{N}}$ denotes the sheaf of differential operators on $\ld{G}/\ld{N}$. This is essentially part of the Beilinson-Bernstein theorem; for example, \cite[Theorem 2.6.5]{milicic} shows that there is an $\ld{G}$-equivariant isomorphism $\Gamma(\ld{G}/\ld{N}; \cd^\hbar_{\ld{G}/\ld{N}})^{\ld{T}} \cong U_\hbar(\ld{\g}) \otimes_{Z(\ld{\g})} U(\ld{\fr{t}})$, so the claim follows from the fact that $U_\hbar(\ld{\g})^\ld{G} \cong Z(\ld{\g}) \otimes_\cc \cc\pw{\hbar}$ and that $U(\ld{\fr{t}})$ is a flat $Z(\ld{\g})$-module.

To finish, we need to show that $\QCoh(\ld{\fr{t}}^\ast) \otimes_{\QCoh(\ld{\fr{t}}^\ast \mmod W)} \LMod_{e\cH(\tilde{\ld{\fr{t}}}^\ast, \tilde{W}^\aff)e}$ is equivalent to $\End_{\ld{\co}^\univ_\hbar}(\QCoh(\ld{\fr{t}}^\ast \times \AA^1_\hbar))$. 
%We will denote $\cd_\ld{G} \otimes_{Z(\ld{\g})} U(\ld{\fr{t}})$ by $\tilde{\cd_\ld{G}}$, so that $\tilde{\cd_\ld{G}}$ is a quantization of $\tilde{\ld{\g}} \times \ld{G}$.  Kostant's isomorphism $U(\ld{\g})/_\chi \ld{N}^- \cong Z(\ld{\g})$ from \cite{kostant-whittaker} implies that
%$$\End_{\ld{\co}^\univ_\hbar}(\QCoh(\ld{\fr{t}}^\ast \times \AA^1_\hbar)) \simeq \LMod_{\ld{N}^- {}_\chi\backslash \tilde{\cd^\hbar_\ld{G}} /_\chi \ld{N}^-}.$$
There is an equivalence 
$$\QCoh(\ld{\fr{t}}^\ast \times \AA^1_\hbar) \simeq \ld{\co}^\univ_\hbar \otimes_{\HC_\hbar(\ld{G})} \QCoh(\ld{\fr{t}}^\ast \mmod W \times \AA^1_\hbar),$$
so that
$$\End_{\ld{\co}^\univ_\hbar}(\QCoh(\ld{\fr{t}}^\ast \times \AA^1_\hbar)) \simeq \QCoh(\ld{\fr{t}}^\ast) \otimes_{\QCoh(\ld{\fr{t}}^\ast \mmod W)} \End_{\HC_\hbar(\ld{G})}(\QCoh(\ld{\fr{t}}^\ast\mmod W \times \AA^1_\hbar)).$$
The desired claim now follows from the observation that there is an isomorphism $\ld{N}^- {}_\chi\backslash {\cd_\ld{G}} /_\chi \ld{N}^- \cong e\cH(\tilde{\ld{\fr{t}}}^\ast, \tilde{W}^\aff)e$ given by \cite[Theorem 8.1.2]{ginzburg-whittaker}, which gives an equivalence between $\End_{\HC_\hbar(\ld{G})}(\QCoh(\ld{\fr{t}}^\ast\mmod W \times \AA^1_\hbar))$ and $\LMod_{e\cH(\tilde{\ld{\fr{t}}}^\ast, \tilde{W}^\aff)e}$.
\end{proof}
\begin{remark}\label{idk-reference}
In fact, one can quantize the result of \cite{abg-iwahori-satake}: namely, there is an equivalence
\begin{equation}\label{gm-rot-abg}
    \DMod_{I \rtimes \GG_m^\rot}(\Gr_G)\simeq \ld{\co}_\hbar^\univ.
\end{equation}
We do not have a reference for this fact when $G$ lives over $\cc$, but it can be deduced using the equivalence of \cite[Section 1.6]{ginzburg-riche} and the arguments of \cite{abg-iwahori-satake}. I am grateful to Tom Gannon for discussions about this equivalence. (If $G$ lives over $\ol{\FF_p}$ and $\DMod$ is replaced with $\ol{\QQ_\ell}$-adic sheaves, then \cref{gm-rot-abg} can be deduced from \cite[Theorem 84]{dodd-thesis} and the parabolic-Whittaker duality for the affine Grassmannian from \cite{bezrukavnikov-yun}.) Just as with \cref{once-looped-satake}, \cref{looped-quantized-abg} may be regarded as a ``once-looped'' version of \cref{gm-rot-abg}. 
One can similarly show that there is an equivalence
\begin{equation}\label{gm-rot-bez}
    \DMod_{I \rtimes \GG_m^\rot}(\Fl_G)\simeq \DMod_\hbar(\ld{N} \backslash \ld{G}/\ld{N})^{(\ld{T} \times \ld{T}, \weak)},
\end{equation}
which quantizes Bezrukavnikov's equivalence from \cite{bezrukavnikov-two-geometric}. Note that $\ld{T} \backslash T^\ast(\ld{N} \backslash \ld{G}/\ld{N}) / \ld{T}$ is isomorphic to $(\tilde{\ld{\g}} \times_{\ld{\g}} \tilde{\ld{\g}})/\ld{G}$, so that this equivalence does indeed quantize Bezrukavnikov's equivalence
$$\DMod_I(\Fl_G) \simeq \QCoh((\tilde{\ld{\g}}[2] \times_{\ld{\g}[2]} \tilde{\ld{\g}}[2])/\ld{G}).$$
\end{remark}
\begin{remark}
If $G$ is a connected and simply-connected semisimple algebraic group or a torus over $\cc$, let $\HC_\hbar(\ld{G})$ denote the $\infty$-category $U_\hbar(\ld\g)\modc^{\ld{G},w}$. Then $\Gamma(\ld{G}; \cd_{\ld{G}})^{\ld{G}\times \ld{G}} \cong U(\ld{\g})^{\ld{G}} \cong \Sym(\ld{\fr{t}})^W$. An argument very similar to \cref{looped-quantized-abg} proves that there is a Kostant functor $\HC_\hbar(\ld{G}) \to \QCoh(\ld{\fr{t}}^\ast\mmod W \times \AA^1_\hbar)$ and a left $A\pw{\hbar}$-linear equivalence 
\begin{equation}\label{looped-HC}
    \LMod_{\pi_0 C^{G \times S^1_\rot}_\ast(\Gr_G(\cc); A)} \simeq \End_{\HC_\hbar(\ld{G})}(\QCoh(\ld{\fr{t}}^\ast\mmod W \times \AA^1_\hbar)).
\end{equation}

This is closely related to \cite{ginzburg-whittaker}, \cite{lonergan-fourier}, and \cite[Theorem 1.4]{gannon-thesis}. Let $\ld{\fr{t}}\mmod \tilde{W}^\aff$ be the coarse quotient as defined in \cite{gannon-tmmodw}.
Then, the aforementioned articles provide a monoidal ``Fourier transform'' equivalence $\DMod(\ld{N}^- {}_\chi\backslash \ld{G} /_\chi \ld{N}^-) \simeq \IndCoh(\ld{\fr{t}}\mmod \tilde{W}^\aff)$. Note that combined with the preceding discussion, we obtain an equivalence
\begin{equation}\label{622-a}
    \IndCoh(\ld{\fr{t}}\mmod \tilde{W}^\aff) \simeq \End_{\HC(\ld{G})}(\QCoh(\ld{\fr{t}}^\ast\mmod W)).
\end{equation}
There is also an equivalence (see \cite{lonergan-fourier})
$$\End_{\Shv_{G\times S^1_\rot}(\Gr_G; \cc)}(\QCoh(\ld{\fr{t}}^\ast\mmod W)) \simeq \LMod_{\H^{G \times S^1_\rot}_\ast(\Gr_G(\cc); \cc)} \simeq \IndCoh(\ld{\fr{t}}\mmod \tilde{W}^\aff),$$
and its relationship to \cref{622-a} is explained by the derived loop-rotation equivariant geometric Satake equivalence of \cite{bf-derived-satake}.
\end{remark}
In the same way, we have the following result. We expect that the techniques of \cite{bzgo} can be used to show that this implies the equivalences conjectured in \cite[Remark 6.22]{gannon-thesis}.
\begin{prop}
We have:
\begin{align}
    \IndCoh(\ld{\fr{t}}\mmod \tilde{W}^\aff) & \simeq \End_{\DMod(\ld{N}\backslash \ld{G}/\ld{N})^{(\ld{T}\times \ld{T}, w)}}(\QCoh(\ld{\fr{t}}^\ast)), \label{622-b}
    %\IndCoh(\ld{\fr{t}}\mmod \tilde{W}^\aff) & \simeq \Fun^L_{\ld{\g}\modc^\ld{N}}(\QCoh(\ld{\fr{t}}^\ast/\Lambda^\vee), \QCoh(\ld{\fr{t}}^\ast\mmod W)). \label{622-c}
\end{align}
\end{prop}
\begin{proof}
The equivalence \cref{622-b} is proved via:
\begin{align*}
    \IndCoh(\ld{\fr{t}}\mmod \tilde{W}^\aff) & \simeq \DMod(\ld{N}^- {}_\chi\backslash \ld{G} /_\chi \ld{N}^-)\\
    & \simeq \End_{\DMod(\ld{G})}(\DMod(\ld{G} /_\chi \ld{N}^-))\\
    & \simeq \End_{\DMod(\ld{N} \backslash \ld{G}/\ld{N})^{(\ld{T}\times \ld{T}, w)}}(\DMod(\ld{N} \backslash \ld{G} /_\chi \ld{N}^-)^{\ld{T},w}) \\
    & \simeq \End_{\DMod(\ld{N} \backslash \ld{G}/\ld{N})^{(\ld{T}\times \ld{T}, w)}}(\DMod(\ld{T})^{\ld{T},w}) \\
    & \simeq \End_{\DMod(\ld{N} \backslash \ld{G}/\ld{N})^{(\ld{T}\times \ld{T}, w)}}(\QCoh(\ld{\fr{t}}^\ast)).
\end{align*}
The third equivalence above uses \cite[Corollary 1.2]{bzgo}, and the fourth equivalence above is the well-known fact that restriction to the big cell in $\ld{G}$ defines an equivalence $\DMod(\ld{N} \backslash \ld{G} /_\chi \ld{N}^-) \xar{\sim} \DMod(\ld{N} \backslash \ld{B} /_\chi \ld{N}^-) \simeq \DMod(\ld{T})$; see \cite[Proposition 1.8]{gannon-thesis}, for instance. 
%The proof of \cref{622-c} is similar:
%\begin{align*}
%    \IndCoh(\ld{\fr{t}}\mmod \tilde{W}^\aff) & \simeq \End_{\DMod(\ld{G})}(\DMod(\ld{G} /_\chi \ld{N}^-))\\
%    %& \simeq \DMod(\ld{G} /_\chi \ld{N}^-) \otimes_{\DMod(\ld{G})} \DMod(\ld{G} /_\chi \ld{N}^-)\\
%    %& \simeq \DMod(\ld{G} /_\chi \ld{N}^-)^{\ld{N}} \otimes_{\DMod(\ld{G})^{\ld{N}, (\ld{G}, w)}} \DMod(\ld{G} /_\chi \ld{N}^-)^{\ld{G},w}\\
%    & \simeq \Fun^L_{\DMod(\ld{G})^{\ld{N}, (\ld{G}, w)}} (\DMod(\ld{G} /_\chi \ld{N}^-)^{\ld{N}}, \DMod(\ld{G} /_\chi \ld{N}^-)^{\ld{G},w}) \\
%    & \simeq \Fun^L_{\ld{\g}\modc^{\ld{N}}}(\DMod(\ld{T}), \ld{\g}\modc^{(\ld{N}^-, \chi)})\\
%    %& \simeq \QCoh(\ld{\fr{t}}^\ast/\Lambda^\vee) \otimes_{\ld{\g}\modc^{\ld{N}}} \QCoh(\ld{\fr{t}}^\ast\mmod W)\\
%    & \simeq \Fun^L_{\ld{\g}\modc^\ld{N}}(\QCoh(\ld{\fr{t}}^\ast/\Lambda^\vee), \QCoh(\ld{\fr{t}}^\ast\mmod W)).
%\end{align*}
%The penultimate line uses two observations. The first is that the Mellin transform gives an equivalence $\QCoh(\ld{\fr{t}}^\ast/\Lambda^\vee) \simeq \DMod(\ld{T})$, which can be identified with $\DMod(\ld{N} \backslash \ld{G} /_\chi \ld{N}^-)$. The second observation is that there is an equivalence $\ld{\g}\modc^{(\ld{N}^-, \chi)} \simeq \QCoh(\ld{\fr{t}}^\ast\mmod W)$, given by the Skryabin equivalence (see the appendix of \cite{premet}).
\end{proof}
\begin{remark}
Since $\ld{\g}/\ld{G} = \Map(B\GG_a, B\ld{G})$, the canonical orientation of $B\GG_a$ defines a $1$-shifted symplectic structure on $\ld{\g}/\ld{G}$ via \cite[Theorem 2.5]{ptvv}. The quasi-classical limit (i.e., $\hbar\to 0$) of the quantized equivalence \cref{looped-HC} gives the following strengthening of \cref{once-looped-satake}. The Kostant slice $\ld{\fr{t}}\mmod W \to \ld{\g}/\ld{G}$ is a Lagrangian morphism by \cite[Proposition 4.18]{safronov-geom-quant}, so that the self-intersection $\ld{\fr{t}}\mmod W \times_{\ld{\g}/\ld{G}} \ld{\fr{t}}\mmod W$ admits the structure of a symplectic stack (using \cite[Theorem 2.9]{ptvv}). Since this fiber product is isomorphic to $(T^\ast \ld{T})^\bl\mmod W$ by \cref{tmmodw-bfm}, we obtain a Poisson bracket on $\co_{(T^\ast \ld{T})^\bl\mmod W} \cong \H^G_\ast(\Gr_G(\cc); \cc)$. This structure can be seen topologically, at least after a completion: using one of the main results of \cite{klang}, the Borel-equivariant analogue/completion $C_\ast(\Gr_G(\cc); \cc)^{hG_c}$ of $C^G_\ast(\Gr_G(\cc); \cc)$ can be identified with the $\E{3}$-center of $C_\ast(\Gr_G(\cc); \cc)$. This defines a $2$-shifted Poisson bracket on $\H_\ast(\Gr_G(\cc)^{hG_c}; \cc)$, which can be identified after $2$-periodification with the ($0$-shifted) Poisson bracket on $\co_{(T^\ast \ld{T})^\bl\mmod W}$.
\end{remark}