\subsection{The elliptic Kostant slice}

Fix a (classical) $\QQ$-algebra $k$ for the remainder of this section. Let $E$ be a (smooth) elliptic curve over $k$, let $\Bun_B^0(E)$ denote the moduli stack of $B$-bundles on $E$ of degree $0$, and let $\Bun_T^0(E)$ denote the scheme of $T$-bundles on $E$ of degree $0$. We will also make use of the stack $\Bun_G^\ss(E)$ of semistable $G$-bundles on $E$.
\begin{definition}\label{B-bundle regular}
Say that a $B$-bundle $\cP_B$ on $E$ is \textit{regular} if $\dim \Aut(\cP_B) = \rank(G)$. Let $\Bun_B^0(E)^\reg$ denote the open substack of $\Bun_B^0(E)$ defined by the regular $B$-bundles. Similarly, if $\cP\in \Bun_G^\ss(E)$ is a semistable $G$-bundle on $E$, we say that $\cP$ is \textit{regular} if $\dim \Aut(\cP) = \rank(G)$. Let $\Bun_G^\ss(E)^\reg \subseteq \Bun_G^\ss(E)$ denote the open substack of regular semistable $G$-bundles.
\end{definition}
\begin{notation}
For $\cP_T\in \Bun_T^0(E)$, write $\Delta_\cP$ to denotes the set of those simple roots $\alpha\in \Delta$ such that the $\alpha$-component of $\cP_T$ is trivial. We will also write $N_\cP = \prod_{\alpha\in \Phi^- \cap \Delta_\cP} N_\alpha\subseteq N$.
\end{notation}
\begin{prop}\label{elliptic-kostant}
The map $\Bun_B^0(E) \to \Bun_T^0(E)$ admits a canonical unique section $\kappa: \Bun_T^0(E) \to \Bun_B^0(E)$ landing in $\Bun_B^0(E)^\reg$.
\end{prop}
\begin{proof}
Let $\cP$ be a semistable $G$-bundle on $E$.
By \cite[Proposition 5.5.5]{davis-elliptic-springer}, the regularity of $\cP$ is equivalent to the condition that for any (or some) $B$-reduction $\cP_B$ of $\cP$ of degree $0$, the associated $N$-bundle $\cP_B/T$ is induced from an $N_\cP$-bundle with nontrivial associated $N_\alpha$-bundle for each $\alpha\in \Delta_\cP$. Moreover, every geometric fiber of the map $\Bun_G^\ss(E) \to \Hom(\bX^\ast(T), E)\mmod W$ to the coarse moduli space of $\Bun_G^\ss(E)$ contains a unique regular semistable $G$-bundle. Also see \cite[Proposition 3.9]{friedman-morgan-witten}, where a similar result is stated.

Following \cite[Definition 4.3.7]{davis-elliptic-springer}, set
$$\tilde{\Bun}_G^\ss(E)^\reg \cong \Bun_G^\ss(E)^\reg \times_{\Hom(\bX^\ast(T), E)\mmod W} \Hom(\bX^\ast(T), E).$$
Let $\Bun_B^0(E)^\reg$ denote the moduli stack of $B$-bundles on $E$ of degree $0$.
It then follows from the isomorphism $\tilde{\Bun}_G^\ss(E) \cong \Bun_B^0(E)$ of \cite[Proposition 4.1.2]{davis-elliptic-springer} and the equality $\dim \Aut(\cP) = \dim \Aut(\cP_B)$ that there is an isomorphism $\tilde{\Bun}_G^\ss(E)^\reg \cong \Bun_B^0(E)^\reg$. In particular, every geometric fiber of the map $\Bun_B^0(E) \to \Hom(\bX^\ast(T), E) = \Bun_T^0(E)$ contains a unique regular $B$-bundle of degree $0$. 
%This implies that if such a section $\kappa: \Bun_T^0(E) \to \Bun_B^0(E)$ exists, it must be unique.

The existence of $\kappa$ is a consequence of \cite[Theorem 4.3.2]{davis-elliptic-springer}, which is a refinement of \cite[Theorem 5.1.1]{friedman-morgan-ii}. Since we will not need the full strength of \cite[Theorem 4.3.2]{davis-elliptic-springer} outside of this proof, we will only briefly recall the necessary notation and statements. In \textit{loc. cit.}, the scheme $\Bun_T^0(E)$ is denoted by $Y$. Let $\tilde{\Bun}_G(E)$ denote the Kontsevich-Mori compactification of $\tilde{\Bun}_G^\ss(E) \cong \Bun_B^0(E)$; see \cite[Definition 2.1.2]{davis-elliptic-springer}. Let $\Theta$ denote the theta-line bundle over $\Bun_T^0(E)$ of \cite[Corollary 3.2.10]{davis-elliptic-springer}, and let $\tilde{\chi}: \tilde{\Bun}_G(E) \to \Theta^{-1}/\GG_m$ denote the map constructed in \cite[Corollary 3.3.2]{davis-elliptic-springer}. Then, \cite[Theorem 4.3.2]{davis-elliptic-springer} shows that there is a map $\Theta^{-1} \to \tilde{\Bun}_G^\ss(E)$ landing in $\tilde{\Bun}_G^\ss(E)^\reg$ such that the composite 
$$\Theta^{-1} \to \tilde{\Bun}_G^\ss(E) \xar{\tilde{\chi}} \Theta^{-1}/\GG_m$$
is the canonical map. Composing with the zero section of $\Theta^{-1}$, we obtain a map 
$$\Bun_T^0(E) \cong 0_{\Theta^{-1}} \to \Theta^{-1} \to \tilde{\Bun}_G^\ss(E)^\reg \cong \Bun_B^0(E).$$
This is the desired map $\kappa$.
\end{proof}
\begin{definition}
We will refer to the map $\kappa: \Bun_T^0(E) \to \Bun_B^0(E)$ from \cref{elliptic-kostant} as the \textit{elliptic Kostant slice}.
\end{definition}
\begin{example}
Let $G = \SL_2$, so that a $B$-bundle on $E$ is just a rank $2$ vector bundle $\cV$ with $\det(\cV) = 0$, equipped with a full flag. Then, the map $\kappa: \Pic^0(E) \to \Bun_B^0(E)$ sends a line bundle $\cL$ to the trivial filtration $\co_E \subseteq \co_E \oplus \cL$ if $\cL^2 \neq \co_E$; and to the Atiyah extension $\cL \subseteq \cf_2 \twoheadrightarrow \cL^{-1}$ from \cite{atiyah-bundle-elliptic} if $\cL^2 \cong \co_E$. This extension is defined by a nontrivial element of $\Ext^1_E(\cL, \cL^{-1}) \cong \H^1(E; \cL^{-2})$.
This can either be shown by unwinding the construction of the section $\kappa$ via \cite[Theorem 4.3.2]{davis-elliptic-springer}, or directly by noting that the description above provides the unique regular $B$-bundle lifting $\cL$.
\end{example}
We will need the following lemma below.
\begin{lemma}\label{vanishing and B-subgroups}
Let $I\subseteq \Phi^-$ be a subset, and let $\Bun_T^0(E)_I$ denote the subscheme of $\Bun_T^0(E)$ defined by those bundles $\cP_T$ whose $\alpha$-component is trivial precisely for $\alpha\in I$. Let $N_I\subseteq N$ be the smallest unipotent subgroup which is invariant under $T$-conjugation and which contains $N_\alpha$ for every $\alpha\in I$. Then the natural map
$$\Bun_{TN_I}^0(E) \times_{\Bun_T^0(E)} \Bun_T^0(E)_I \to \Bun_B^0(E) \times_{\Bun_T^0(E)} \Bun_T^0(E)_I$$
is an isomorphism.
\end{lemma}
\begin{proof}
Let $\cP_I$ denote the universal $T$-bundle over $\Bun_T^0(E)_I$, so that $\Bun_B^0(E) \times_{\Bun_T^0(E)} \Bun_T^0(E)_I$ is the stack of $B$-bundles $\cP_B$ such that $\cP_B/N \cong \cP_T$; therefore, it is isomorphic to the stack $\Bun_N^{\cP_I}$ in the notation of \cite[Section 2.1.1]{frenkel-gaitsgory-vilonen}. Similarly, $\Bun_{TN_I}^0(E) \times_{\Bun_T^0(E)} \Bun_T^0(E)_I \cong \Bun_{N_I}^{\cP_I}$. To show that these stacks are isomorphic, consider the filtration
$$N_\ell \subseteq N_{\ell-1} \subseteq \cdots \subseteq N_2 \subseteq N_1 = N$$
by root height (recall that the height of a root is the sum of its simple root components), so that it is invariant under $T$-conjugation, and there is an induced filtration
$$N_{I,\ell} \subseteq N_{I, \ell-1} \subseteq \cdots \subseteq N_{I,2} \subseteq N_{I,1} = N_I.$$
Then, $N_j\subseteq N$ is normal and $N_{j-1}/N_j$ is central in $N/N_j$ (and similarly for $N_{I,j}$); this implies that $\Bun_{N/N_j}^{\cP_I}$ is a $\Bun_{N_{j-1}/N_j}^{\cP_I}$-torsor over $\Bun_{N/N_{j-1}}^{\cP_I}$. Similar statements hold for $\Bun_{N_I/N_{I,j}}^{\cP_I}$. To show that $\Bun_{N_I}^{\cP_I} \to \Bun_N^{\cP_I}$ is an isomorphism, it therefore suffices to show that the induced map $\Bun_{N_{I,j-1}/N_{I,j}}^{\cP_I} \to \Bun_{N_{j-1}/N_j}^{\cP_I}$ is an isomorphism. Let $\cN = \cP_I \times^T N$, $\cN_I = \cP_I \times^T N_I$, etc., so that $\cN_{j-1}/\cN_j$ is a direct sum of line bundles of degree zero. By choice of $N_I$, the inclusion of the trivial line bundle summands into $\cN_{j-1}/\cN_j$ factors through the map $\cN_{I,j-1}/\cN_{I,j} \to \cN_{j-1}/\cN_j$. The desired isomorphism then follows from the observation that if $U$ is a vector group with $\GG_m$-action, then $\Bun_U^\cL$ is a point if $\cL$ is a nontrivial line bundle of degree zero (because then $\H^1(E; U(\cL)) = 0$).
\end{proof}
\begin{example}
For instance, suppose that $I = \emptyset$, so that $\Bun_T^0(E)_\emptyset$ denotes the open subscheme of $T$-bundles of degree zero whose $\alpha$-component is nontrivial for every negative root $\alpha$. The isomorphism $\tilde{\Bun}_G^\ss(E) \cong \Bun_B^0(E)$ implies that the map $\tilde{\Bun}_G^\ss(E) \to \Bun_T^0(E)$ is an isomorphism over $\Bun_T^0(E)_\emptyset$. In particular, every point of $\Bun_T^0(E)_\emptyset$ has a canonical associated (regular) semistable $G$-bundle.
The above results continue to hold if $E$ is replaced by the constant stack $S^1$ or by $B\GG_a$ (in which case $\tilde{\Bun}_G^\ss(E)$ and $\Bun_B^0(E)$ are to be interpreted as $G/G$ and $B/B$, and $\g/G$ and $\fr{b}/B$, respectively). In the case of $S^1$, for instance, the semistable $G$-bundles obtained in this way from $\Bun_T^0(E)_\emptyset$ are precisely those which lie in the regular \textit{semisimple} locus $G^{\rs}/G$; similarly for the case of $B\GG_a$.
\end{example}