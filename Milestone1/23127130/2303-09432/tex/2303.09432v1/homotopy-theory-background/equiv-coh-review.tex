\subsection{Review of generalized equivariant cohomology}\label{review-equiv}

We review the construction of generalized equivariant cohomology via spectral algebraic geometry from \cite{survey}, in a form suitable for our applications. This review will necessarily be brief, since a detailed exposition may be found in \textit{loc. cit.}; there is also some discussion in the early sections of \cite{ginzburg-kapranov-vasserot} in the setting of ordinary (as opposed to spectral) algebraic geometry.
\begin{setup}
Fix an $\Eoo$-ring $A$ and a commutative $A$-group $\GG$, so $\GG$ defines a functor $\CAlg_A \to \Mod_{\Z,\geq 0}$ which is representable by a \textit{flat} $A$-algebra. We will write $\GG_0$ to denote the resulting commutative group scheme over $\pi_0 A$.
\end{setup}
\begin{remark}
The equivalence $\Omega^\infty: \Sp_{\geq 0} \xar{\sim} \CAlg(\Top_\ast)$ extends to an equivalence between $\Mod_{\Z,\geq 0}$ and topological abelian groups. More precisely, by the Dold-Kan correspondence and the Schwede-Shipley theorem, there are equivalences of categories
$$\Mod^{\geq 0}_\Z \simeq \mathrm{Ch}_{\geq 0}(\Z) \simeq \Fun(\Deltab^{op},\Ab) = s\Ab.$$
The image of $\Mod^{\geq 0}_\Z$ under the equivalence $\Omega^\infty: \Sp_{\geq 0} \xar{\sim} \CAlg(\Top_\ast)$ can be characterized as follows.
Let us model grouplike infinite loop spaces $X$ as functors $X:\Fin_\ast\to \Top$ such that $\pi_0 \Map_\Top(Y,X)$ is an abelian group for all spaces $Y$ (i.e., $X$ is grouplike) and such that the map $X([n])\to X([1])^n$ is an equivalence. Such an object should be in the image of $\Mod^{\geq 0}_\Z$ iff it is ``strictly commutative''. One way to make this precise is as follows. Let $\mathrm{Lattice}$ denote the full subcategory of the category of abelian groups spanned by the groups $\Z^n$ with $n\geq 0$, so there is a functor $\Fin_\ast\to \mathrm{Lattice}$. Then an infinite loop space is in the image of $\Mod^{\geq 0}_\Z$ if and only if the functor $\Fin_\ast\to \Top$ classifying it factors through a finite-product-preserving functor $\mathrm{Lattice} \to \Top$. In other words, $\Mod^{\geq 0}_\Z$ is equivalent to the full subcategory spanned by the grouplike objects in the category $\Fun^\pi(\mathrm{Lattice}, \Top)$. This is a very strong condition to impose on an infinite loop space: it forces the infinite loop space to decompose as a product of Eilenberg-Maclane spaces. For example, $\CP^\infty$ admits such a factorization, but $\BU$ (with either the additive or multiplicative infinite loop space structure) does not.
\end{remark}
\begin{definition}
A \textit{preorientation of $\GG$} is a pointed map $S^2 \to \Omega^\infty \GG(A)$ of spaces, i.e., a map $\Sigma^2 \Z \to \GG(A)$ of $\Z$-modules (by adjunction). This induces a map $\CP^\infty = \Omega^\infty \Sigma^2 \Z \to \Omega^\infty \GG(A)$ of topological abelian groups, and hence a map $\spf A^{\CP^\infty} \to \GG$ of $\Eoo$-$A$-group schemes. (Note that $\spf A^{\CP^\infty}$ need not admit the structure of a commutative $A$-group scheme: for instance, $A^{\CP^\infty}$ need not be flat over $A$.)
\end{definition}
\begin{definition}
Given a preorientation $S^2 \to \Omega^\infty \GG(A)$, we obtain a map $\co_\GG \to C^\ast(S^2; A)$ of $\Eoo$-$A$-algebras. On $\pi_0$, this induces a map $\pi_0 \co_\GG = \co_{\GG_0} \to \pi_0 C^\ast(S^2; A)$. However, the target can be identified with the trivial square-zero extension $\pi_0 A \oplus \pi_{-2} A$, so that the preorientation defines a derivation $\co_{\GG_0} \to \pi_{-2} A$. This defines a map $\beta: \omega = \Omega^1_{\GG_0/\pi_0 A} \to \pi_{-2} A$. The preorientation is called an \textit{orientation} if $\GG_0$ is smooth of relative dimension $1$ over $\pi_0 A$, and the composite
$$\pi_n(A) \otimes_{\pi_0 A} \omega \to \pi_n(A) \otimes_{\pi_0 A} \pi_{-2} A \xar{\beta} \pi_{n-2} A$$
is an isomorphism for each $n\in \Z$. This forces $A$ to be $2$-periodic (but does not force its homotopy to be concentrated in even degrees).
\end{definition}
\begin{warning}\label{additive-orientation}
As discussed in \cite[Section 3.2]{survey}, the universal $\Eoo$-$\Z$-algebra over which the additive group scheme $\GG_a$ admits an orientation is given by $\Z[\CP^\infty][\tfrac{1}{\beta}] = \QQ[\beta^{\pm 1}]$. Therefore, we are allowed to let $\GG = \GG_a$ in the story below only when $A$ is a $2$-periodic \textit{$\Eoo$-$\QQ$-algebra}. (If $A$ is not an $\Eoo$-$\Z$-algebra, one cannot in general define $\GG_a = \spec A[t]$ as a commutative $A$-group: the coproduct $A[t] \to A[x,y]$ will in general not be a map of $\Eoo$-$A$-algebras.)
\end{warning}
We can now review the definition of $T$-equivariant $A$-cohomology when $T$ is a torus.
\begin{construction}\label{def-equiv-coh}
Fix an $\Eoo$-ring $A$ as above and a commutative $A$-group $\GG$. Given a compact abelian Lie group $T$, define an $A$-scheme $\cM_T$ by the mapping stack $\Hom(\bX^\ast, \GG)$. We will be particularly interested in the case when $T$ is a torus. Let $\cT$ be the full subcategory of $\Top$ spanned by those spaces which are homotopy equivalent to $BT$ with $T$ being a compact abelian Lie group. By arguing as in \cite[Theorem 3.5.5]{elliptic-iii}, a preorientation of $\GG$ is equivalent to the data of a functor $\cM: \cT \to \Aff_A$ along with compatible equivalences $\cM(BT) \simeq \cM_T$. The $\Eoo$-$A$-algebra $\co_{\cM_T}$ is the $T$-equivariant $A$-cochains of a point, and will occasionally be denoted by $A_T$.

We can now sketch the construction of the $T$-equivariant $A$-cochains of more general $T$-spaces; see \cite[Theorem 3.2]{survey}. Let $T$ be a torus over $\cc$ for the remainder of this discussion, and let $\GG$ be an \textit{oriented} commutative $A$-group. Let $\Top(T)$ denote the $\infty$-category of finite $T$-spaces, i.e., the smallest subcategory of $\Fun(BT, \Top)$ which contains the quotients $T/T'$ for closed subgroups $T'\subseteq T$, and which is closed under finite colimits. There is a functor $\cf_T: \Top(T)^\op \to \QCoh(\cM_T)$ which is uniquely characterized by the requirement that it preserve finite limits and sends $T/T' \mapsto q_\ast \co_{\cM_{T'}}$. Here, $q: \cM_{T'} \to \cM_T$ is the canonical map induced by the inclusion $T'\subseteq T$. If $X\in \Top(T)$, then the \textit{$T$-equivariant $A$-cochains of $X$} is the global sections $\Gamma(\cM_T; \cf_T(X))$; we will denote it by $C^\ast_T(X; A)$.
\end{construction}
\begin{remark}
We will denote the functor $\Gamma(\cM_T; \cf_T(-)): \Top(T)^\op \to \Mod(\Gamma(\cM_T; \co_{\cM_T}))$ by $C^\ast_T(-;A): \Top(T)^\op \to \Mod(A_T)$.
\end{remark}
\begin{definition}\label{equiv-homology}
%The $\Eoo$-$A$-algebra $A_T = \co_{\cM_T}$ admits an $\Eoo$-coalgebra structure (owing to it being the $\Eoo$-ring of functions on the group scheme $\GG$). Its $A$-linear dual $A^T := A_T^\vee$ therefore admits the structure of an $\Eoo$-$A$-algebra (as well as the structure of an $\Eoo$-$A$-coalgebra). If $X\in \Top(T)$, then the \textit{$T$-equivariant $A$-chains of $X$} is the $A^T$-comodule given by $\cf_T(X)^\vee$; we also denote it by $C_\ast^T(X; A)$. Therefore, $A^T$ is the $T$-equivariant $A$-chains of a point. We will write $\cM_T^\vee$ to denote $\spec A^T$.
If $X\in \Top(T)$, then the \textit{$T$-equivariant $A$-chains of $X$} is the quasicoherent sheaf on $\cM_T$ given by the $\co_{\cM_T}$-linear dual $\cf_T(X)^\vee$. We will denote its global sections by $C_\ast^T(X; A)$. Note that $C_\ast^T(\ast;A) \simeq A_T$, which completes to the $A$-cochains (\textit{not} $A$-chains) of $BT$.
\end{definition}
\begin{warning}
Let $A$ be an $\Eoo$-$\Z$-algebra, and let $\GG = \GG_a$; then \cref{additive-orientation} says that $A$ must be an $\Eoo$-$\QQ[\beta^{\pm 1}]$-algebra. Suppose for simplicity that $T = \GG_m$; then $\pi_\ast C_\ast(BT; A)$ may therefore be identified with the divided power algebra $\Gamma_{\pi_\ast(A)}(\hbar^\vee)$ with $|\hbar^\vee|=2$. Since $A$ is rational, this may further be identified with the polynomial ring $\pi_\ast(A)[\hbar^\vee]$. Unfortunately, this can be confused with $\pi_{\ast}(A_T)$, albeit with the reversed grading. Although this identification is technically correct, it is rather abusive: there is no canonical way to identify $A_T$ with $C_\ast(BT; A)$ when $A$ is an $\Eoo$-$\QQ[\beta^{\pm 1}]$-algebra. We will therefore refrain from making this identification, since it is not valid for more general $\Eoo$-rings $A$.
\end{warning}
\begin{notation}\label{ideal-character}
Let $\lambda: T \to \GG_m$ be a character, and let $T_\lambda = \ker(\lambda)$. Then the map $q: \cM_{T_\lambda} \to \cM_T$ is a closed immersion, and we will denote the ideal in $\co_{\cM_T}$ defined by this closed immersion by $\cI_\lambda$. Equivalently, let $V_\lambda$ denote the $T$-representation obtained by the projection $T \to T_\lambda$. Then $\cI_\lambda$ is given by the line bundle $\cf_T(S^{V_\lambda})$.
\end{notation}
It is trickier to extend the definition of equivariant cochains to nonabelian groups, but a construction is sketched in \cite[Section 3.5]{survey}, and a detailed construction is given in \cite{gepner-meier}. We recall this for completeness; in this article, we will only be concerned with torus-equivariance. The methods of this article should work for more general compact Lie groups, but we have not studied this here.
\begin{construction}\label{nonabelian-equiv-cochains}
Let $G$ be a reductive group scheme over $\cc$. Let $\Top(G)$ denote the smallest subcategory of $\Fun(BG, \Top)$ which contains the quotients $G/T'$ for closed \textit{commutative} subgroups $T'\subseteq G$, and which is closed under finite colimits. Then there is a functor $C^\ast_G(-;A): \Top(G)^\op \to \Mod(A)$ which is uniquely characterized by the requirement that it preserve finite limits and sends $G/T' \mapsto A_{T'}$. According to \cite[End of Section 3.5]{survey} and \cite[Section 3]{gepner-meier}, when $G$ is connected, there is a flat $A$-scheme $\cM_G$ and a functor $\cf_G: \Top(G)^\op \to \QCoh(\cM_G)$, such that composition with the forgetful functor $\QCoh(\cM_G) \to \Mod(A)$ is the functor $C^\ast_G(-;A)$. If $X\in \Top(G)$, we will write $\cf_G(X)^\vee$ to denote the linear dual of $\cf_G(X)$ in $\QCoh(\cM_G)$, and refer to it as the \textit{$G$-equivariant $A$-chains} on $X$.
\end{construction}
\begin{remark}\label{ind-finite}
Let $X$ be an ind-finite space with a $G$-action, so that $X$ can be written as the filtered colimit of a diagram $\{X_i\}$ of subspaces, each of which are in $\Top(G)$. Write $C^\ast_G(X;A)$ to denote $\varprojlim_i C^\ast_G(X_i;A)$. Similarly for $\cf_G(X)$.
\end{remark}
\begin{example}
Let $G$ be a connected compact Lie group, and let $T$ be a maximal torus in $G$. The flag variety $G/T$ is a $G$-space whose stabilizers are commutative, and therefore $G/T\in \Top(G)$. Therefore, $C^\ast_G(G/T; A) = A_T$. For the remainder of this text, we will make the following \textit{assumption}: after inverting $|W|$, there is a (homotopy-coherent) $W$-action on $A_T$ by maps of $\Eoo$-$A$-algebras, and $A_G := C^\ast_G(\ast;A)$ is equivalent to $A_T^{hW}$ as an $\Eoo$-$A$-algebra. %We will then also write $\cM_G = \spec A_G$, even though this might not agree with the putative definition of $\cM_G$ from \cite{survey}.
\end{example}