\subsection{Categories of equivariant local systems}\label{categories-of-equiv-loc}

Fix a complex-oriented even-periodic $\Eoo$-ring $A$ and an oriented $A$-group scheme $\GG$. Let $T$ be a compact torus. Let $X\in \Top(T)$ be a finite $T$-space. The following categorifies the $T$-equivariant $A$-cochains $C_T^\ast(X; A)$.
\begin{construction}\label{def-loc}
Let $\Loc_T(\ast; A)$ denote the $\infty$-category $\QCoh(\cM_T)$. Let $T'\subseteq T$ be a closed subgroup, so that there is an associated morphism $q: \cM_{T'} \to \cM_T$. This defines a symmetric monoidal functor $\QCoh(\cM_T) \to \QCoh(\cM_{T'})$, which equips $\QCoh(\cM_{T'})$ with the structure of a $\QCoh(\cM_T)$-module.
Let $\cLoc_T(-; A): \Top(T)^\op \to \CAlg(\shvcat(\cM_T))$ be the functor uniquely characterized by the requirement that it preserve finite limits and send $T/T' \mapsto \QCoh(\cM_{T'})$. If $X\in \Top(T)$, then the $\infty$-category $\Loc_T(X; A)$ of \emph{$T$-equivariant local systems of $A$-modules on $X$} is defined to be the global sections of the quasicoherent stack $\cLoc_T(X; A)$ on $\cM_T$. If $f: X \to Y$ is a map in $\Top(T)$, the associated symmetric monoidal functor $f^\ast: \Loc_T(Y; A) \to \Loc_T(X; A)$ (induced by taking global sections of the morphism $f^\ast: \cLoc_T(Y; A) \to \cLoc_T(X; A)$ of $\Eoo$-algebras in quasicoherent stacks over $\cM_T$) will be called the \textit{pullback}. One can show that $\Loc_T(X; A)$ is a presentable stable $\infty$-category, and that $f^\ast$ preserves small colimits (so it has a right adjoint $f_\ast$, which will be called \textit{pushforward}).
\end{construction}
\begin{example}
If $T = \{1\}$, then $\Loc_T(X; A)$ is equivalent to the $\infty$-category $\Loc(X; A) := \Fun(X, \Mod_A)$ of local systems on $X$.
\end{example}
\begin{remark}
Let $X$ be a finite $T$-space. The \textit{constant local system} $\ul{A}_T$ is defined to be the image of $\co_{\cM_T}$ under the symmetric monoidal functor $\Loc_T(\ast; A) \simeq \QCoh(\cM_T) \to \Loc_T(X; A)$ induced by pullback along $f: X \to \ast$. Observe that if $\ul{A}_T$ denotes the constant local system, then $\End_{\Loc_T(X; A)}(\ul{A}_T) \simeq C_T^\ast(X; A)$. Indeed, $\End_{\Loc_T(X; A)}(\ul{A}_T) \simeq \Gamma(\cM_T; f_\ast f^\ast \co_{\cM_T})$, but it is easy to see that $f_\ast f^\ast \co_{\cM_T} = \cf_T(X)\in \QCoh(\cM_T)$. The desired claim then follows from \cref{def-equiv-coh}.
\end{remark}
\begin{remark}
If $T$ were a \textit{finite} diagonalizable group scheme (such as $\mu_n$), the desired category $\Loc_T(X; A)$ is closely related to the $\infty$-category of \textit{$\GG$-tempered local systems} on the orbispace $X\mmod T$, as described in \cite{elliptic-iii}. Our understanding is that Lurie is planning to describe an extension of the work in \cite{elliptic-iii} and its connections to equivariant homotopy theory in a future article. We warn the reader that \cref{def-loc} is somewhat \textit{ad hoc}; so the resulting category of equivariant local systems may or may not agree with the output of forthcoming work of Lurie.
\end{remark}
\begin{remark}
Let $X$ be a $T$-space with a chosen presentation as a filtered colimit of finite $T$-spaces $X_\alpha$. Then we will write $\Loc_T(X; A)$ to denote $\lim \Loc_T(X_\alpha; A)$.
\end{remark}
If $Y$ is a \textit{connected} space, the $\infty$-category $\Loc(Y; A) = \Fun(Y, \Mod_A)$ of local systems on $Y$ is equivalent by Koszul duality to $\LMod_{C_\ast(\Omega Y; A)}$. This property of local systems is very useful, since it allows one to study of local systems using (derived) algebra. A similar property is true for $\Loc_T(X; A)$:
\begin{prop}\label{equivariant-koszul}
Let $X$ be a connected finite $T$-space. Then there is an equivalence $\Loc_T(X; A) \simeq \LMod_{\cf_T(\Omega X)^\vee}(\QCoh(\cM_T))$.
\end{prop}
\begin{proof}
Let $s: \ast \to X$ denote the inclusion of a point. We claim that $s^\ast: \Loc_T(X; A) \to \QCoh(\cM_T)$ admits a left adjoint $s_!$. Indeed, the statement for general $X$ follows formally from the case of $X = T/T'$ for some closed subgroup $T'\subseteq T$ (so $s$ is the inclusion of the trivial coset). In this case, $s^\ast$ is the functor $\QCoh(\cM_{T'}) \to \QCoh(\cM_T)$ given by pushforward along the associated morphism $q: \cM_{T'} \to \cM_T$, so it has a left adjoint $s_!$ given by $q^\ast$. Note that $s^\ast$ also has a right adjoint; in particular, it preserves small limits and colimits. Observe now that $s_! \co_{\cM_T}$ is a compact generator of $\Loc_T(X; A)$: indeed, suppose $\cf\in \Loc_T(X; A)$ such that $\Map_{\Loc_T(X; A)}(s_! \co_{\cM_T}, \cf) \simeq 0$ as an object of $\QCoh(\cM_T)$. Because $s^\ast \cf \simeq \Map_{\Loc_T(X; A)}(s_! \co_{\cM_T}, \cf)$ in $\QCoh(\cM_T)$, we see that $s^\ast \cf \simeq 0$. Using the connectivity of $X$, we see that $\cf$ itself must be zero, which implies that $s_! \co_{\cM_T}$ is a compact generator of $\Loc_T(X; A)$. It follows from the Barr-Beck-Lurie theorem \cite[Theorem 4.7.3.5]{HA} that $\Loc_T(X; A)$ is equivalent to the $\infty$-category of left $\End_{\Loc_T(X;A)}(s_! \co_{\cM_T})$-modules in $\QCoh(\cM_T)$. But $\End_{\Loc_T(X;A)}(s_! \co_{\cM_T}) \simeq s^\ast s_! \co_{\cM_T}$, which identifies with $\cf_T(\Omega X)^\vee$.
%\todo hmm
\end{proof}
\begin{remark}\label{loc and comod}
Modifying the preceding argument shows that if $X$ is a connected finite $T$-space, there is an equivalence $\Loc_T(X; A) \simeq \coLMod_{\cf_T(X)^\vee}(\QCoh(\cM_T))$. In particular, if $X$ admits an $\E{n}$-algebra structure (compatible with the $T$-action), then $\cf_T(X)^\vee$ admits the structure of an $\E{n}$-algebra\footnote{If $\cC$ is a symmetric monoidal $\infty$-category, \cite[Corollary 3.3.4]{HA} can be used to show that there is an equivalence $\coCAlg(\Alg_\E{n}(\cC)) \simeq \Alg_\E{n}(\coCAlg(\cC))$.} in $\coCAlg(\QCoh(\cM_T))$, and the equivalence $\Loc_T(X; A) \simeq \coLMod_{\cf_T(X)^\vee}(\QCoh(\cM_T))$ is $\E{n}$-monoidal for the convolution tensor product on both sides. More generally, if $X$ is a $T$-space with a chosen presentation as a filtered colimit of finite $T$-spaces $X_\alpha$, there is an equivalence $\Loc_T(X; A) \simeq \coLMod_{\cf_T(X)^\vee}(\QCoh(\cM_T))$.
\end{remark}