\subsection{GKM and complex periodic $\Eoo$-rings}

We review the main result of \cite{generalized-gkm}, which proves a generalization of a result of Goresky-Kottwitz-MacPherson to generalized cohomology theories. This is also studied in the forthcoming work \cite[Section 3]{gepner-meier}.
\begin{setup}\label{gkm-assumption}
Let $A$ be a complex-oriented even-periodic $\Eoo$-ring, and let $\GG$ be an oriented commutative $A$-group. 
%Recall that if $Y$ is a space and $Y \to \BU \times \Z$ classifies a complex vector bundle $\xi$ over $Y$, then $\xi$ is said to be \textit{$A$-orientable} if the composite
%$$Y \xar{\xi} \BU\times \Z\xar{J} \Pic(\Sp) \to \Pic(\Mod_A)$$
%is nullhomotopic. Let $Y^\xi$ denote the Thom space of $Y$, and let $e(\xi)\in A^\ast(Y)$ denote the Euler class of $\xi$ (obtained by pulling back the Thom class along the zero section). 
Fix a compact torus $T$. We will consider (ind-finite; see \cref{ind-finite}) $T$-spaces $X$ such that the following assumptions hold.
\begin{enumerate}
    \item $X$ admits a $T$-invariant stratification $\bigcup_{w\in W} X_x$ with only \textit{even-dimensional} cells, with only finitely many in each dimension.
    \item The $T$-action on each cell $X_w = \AA^{\ell(w)}$ is via a linear action, whose weights are pairwise relatively prime.
    \item For each weight $\lambda$ of the $T$-action on $X_w = \AA^{\ell(w)}$, the closure of $\cc_\lambda\subseteq X_w$ is a sphere $S^\lambda$ such that $0$ and $\infty$ are fixed points of the $T$-action.
    %attaching map for a cell sends the boundary of each $D(\lambda)\subseteq D(\AA^{2\ell(w)})$ to a fixed point of one of the cells in the lower strata. 
\end{enumerate}
\end{setup}
\begin{definition}
The \textit{GKM graph} $\Gamma$ asssociated to an $X$ as in \cref{gkm-assumption} is defined as follows. The vertices are the (isolated) fixed points of the $T$-action, and there is an edge $x \to y$ labeled by a character $\lambda$ if $x = 0$ and $y=\infty$ in the closure $S^\lambda$ of $D(\lambda)\subseteq D(\AA^{2\ell(w)})$. Let $V$ denote the set of vertices of $\Gamma$, and $E$ the set of edges.
\end{definition}
\begin{theorem}[{\cite[Theorem 3.1]{generalized-gkm}, \cite[Section 3]{gepner-meier}}]\label{gkm-main}
In \cref{gkm-assumption}, the map $\cf_T(X) \to \Map(V, \co_{\cM_T}) \simeq \cf_T(X^T)$ induces an injection on homotopy sheaves, and the following diagram is an equalizer on $\pi_0$:
$$\cf_T(X) \to \Map(V, \co_{\cM_T}) \rightrightarrows \prod_{\alpha\in E} \co_{\cM_{T_\alpha}}.$$
Here, the two maps are induced by the inclusion of the source and target of $\alpha: x \to y$.
\end{theorem}
\begin{proof}[Proof sketch]
The argument is exactly as in \cite[Theorem 3.1]{generalized-gkm} (where the spaces denoted $F_i$ are points, corresponding to the origin in $\AA^{\ell(w)}$), so we only give a sketch. We will work locally on $\GG$. In this case, we need to show that the map $\cf_T(X) \to \Map(V, \co_{\cM_T}) \simeq \cf_T(X^T)$ is injective on homotopy sheaves, and the following diagram is an equalizer on $\pi_0$:
$$\cf_T(X) \to \cf_T(X^T) \rightrightarrows \prod_{\alpha\in E} \cf_{T_\alpha}.$$

For the injectivity claim, we first claim that $\cf_T(X)^{tT} \simeq \cf_T(X^T)^{tT}$. (This is a version of Atiyah-Bott localization.) Since $X$ is generated by finite colimits from $T$-orbits $T/T'$, it suffices to prove this claim when $X$ is of that form. Then $\cf_T(T/T') \simeq \cf_{T'}(\ast) = q_\ast \co_{\cM_{T'}}$; this has zero Tate construction if $T'\neq T$. On the other hand, $X^T = \emptyset$ if $T' \neq T$, so $\cf_T(X^T)^{tT} = 0$ as desired. If $T' = T$, then $X^T = \ast$, so that both sides are simply $A^{tT}$.

Note that $\cf_T(X^T)^{tT} \simeq \cf_T(X^T) \otimes_A A^{tT}$. Since $\cf_T(X)^{tT} \simeq \cf_T(X) \otimes_{\co_{\cM_T}} A^{tT}$ is a localization, it suffices to prove that the map $\cf_T(X) \to \cf_T(X)^{tT}$ induces an injection on homotopy. For this, it suffices to prove that $\cf_T(X)$ is a free $\co_{\cM_T}$-module. This is a consequence of the assumptions on $X$.
%There is a spectral sequence
%$$E_2^{\ast,\ast} = \H^\ast(BT; C^\ast(X; A)) \Rightarrow \pi_\ast \cf_T(X)$$
%of sheaves on $\cM_T$, which degenerates at the $E_2$-page by the assumptions on $X$. This implies the desired claim.

To prove the statement about the equalizer diagram, the key case is when $X = S^W$ for a $T$-representation $W$; the general case is obtained by induction on the stratification of $X$. Let $\lambda_1,\cdots,\lambda_n$ be the weights of $W$, so that $X = \bigotimes_{i=1}^n S^{\lambda_i}$. Therefore, $X$ is the quotient of $\prod_{i=1}^n S^{\lambda_i}$ by its $(2n-2)$-skeleton. Using this observation, it is not difficult to reduce to the case when $W = \lambda$ is a character of $T$.
In this case, $X = S^\lambda$ has $T$-fixed points given by $\{0,\infty\}$. There is a cofiber sequence $S(\lambda) \to \ast \to S^\lambda$, which induces a pushout square
$$\xymatrix{
S(\lambda)_+ \ar[r] \ar[d] & S^0 = \{\infty\}_+ \ar[d] \\
S^0 = \{0\}_+ \ar[r] & S^\lambda_+.
}$$
Therefore, we get an equalizer diagram
$$\cf_T(S^\lambda) \to \co_{\cM_T} \rightrightarrows \cf_T(S(\lambda)).$$
However, if $T_\lambda = \ker(\lambda: T \to \GG_m)$, then $S(\lambda) \simeq T/T_\lambda$, so that $\cf_T(S(\lambda)) \simeq q_\ast \co_{\cM_{T_\lambda}}$. It follows that $\cf_T(S^\lambda)$ is the fiber of the map $\co_{\cM_T} \oplus \co_{\cM_T} \to q_\ast \co_{\cM_{T_\lambda}}$ given by the following composite:
$$\co_{\cM_T} \oplus \co_{\cM_T} \xar{(x,y)\mapsto x-y} \co_{\cM_T} \to q_\ast \co_{\cM_{T_\lambda}}.$$
However, the map $\co_{\cM_T} \to q_\ast \co_{\cM_{T_\lambda}}$ is precisely given by quotienting by the ideal $\cI_\lambda$ (by \cref{ideal-character}). Therefore, $\cf_T(S^\lambda)$ is described by the claimed equalizer diagram.
\end{proof}
\begin{remark}
Informally, the image on homotopy sheaves of the map $\cf_T(X) \to \Map(V, \co_{\cM_T}) \simeq \cf_T(X^T)$ consists of those $f\in \pi_\ast \co_{\cM_T}^V$ such that $f(x) \equiv f(y) \pmod{\cI_\alpha}$ for every edge $\alpha: x \to y$ in $\Gamma$. Here, $\cI_\alpha$ is as in \cref{ideal-character}.
\end{remark}