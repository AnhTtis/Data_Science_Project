Let $G$ be a simply-connected semisimple algebraic group or a torus over $\cc$. Many deep results in geometric representation theory are concerned with describing the ``topological''/A-side category of D-modules on algebraic (ind-)schemes associated to $G$ (such as the flag variety, the nilpotent cone, the affine Grassmannian, the affine flag variety, etc.) in terms of representation-theoretic/algebro-geometric B-side data associated to $\ld{G}$, the Langlands dual. These equivalences can be interpreted as refinements of the Fourier/Mellin transform. By the Riemann-Hilbert equivalence, the A-side category of D-modules on $X$ may be interpreted instead as categories of constructible sheaves of $\cc$-vector spaces on $X(\cc)$. The goal of this manuscript is to study analogues of some of these equivalences when we instead consider the category of constructible sheaves of $A$-module spectra on $X(\cc)$, where $A$ is a complex-oriented even-periodic $\Eoo$-ring (such as topological K-theory $\KU$, or an elliptic cohomology theory).

\subsection{Summary of content}

In this article, we take a few steps towards establishing a chromatic homotopy-theoretic analogue of the derived geometric Satake equivalence.
%\footnote{We warn the reader that up to \cref{satake}, all statements rely on results which are present in the literature. In \cref{review-equiv}, we review some constructions from \cite{survey}; some, but not all, of this theory has been written down in \cite{elliptic-i, elliptic-ii, elliptic-iii, t-equiv-tmf, gepner-meier}.}.
Let $B$ be a Borel subgroup of $G$. 
%If $X$ is a $\cc$-scheme, let $X(\cK)$ denote the formal loop space $X\ls{t}$, and let $X(\co)$ denote the formal arc space $X\pw{t}$.
Let $\cK$ denote $\cc\ls{t}$, and let $\co$ denote $\cc\pw{t}$.
The affine Grassmannian $\Gr_G$ is defined as the sheafification of the functor of points $\CAlg_\cc \ni R \mapsto G(R \otimes_\cc \cK)/G(R \otimes_\cc \co)$. It has the property that $\Gr_G(\cc)$ is homotopy equivalent to $\Omega G_c \simeq \Omega^2 BG_c$, where $G_c$ is a maximal compact subgroup of $G(\cc)$; see \cite{mitchell-buildings}. (Note that $G_c$ is homotopy equivalent to $G(\cc)$, so for most of the topological parts of this article, the distinction between them will be irrelevant.) The classical geometric Satake equivalence says: 
\begin{theorem}[{Classical geometric Satake, \cite{mirkovic-vilonen}}]
The abelian category $\mathrm{Perv}_{G(\co)}(\Gr_G; \QQ)$ of $G(\co)$-equivariant perverse sheaves on $\Gr_G$ is equivalent to $\Rep(\ld{G}_\QQ)$, where $\ld{G}_\QQ$ is the Langlands dual group\footnote{This denotes the base-change to $\QQ$ of the Chevalley scheme over $\Z$, i.e., the split reductive group scheme whose root datum coincides with the root datum of $\ld{G}_\cc$.} over $\QQ$.
\end{theorem}
In \cite{bf-derived-satake}, building on work of Ginzburg \cite{ginzburg-langlands}, Bezrukavnikov-Finkelberg proved a \textit{derived} analogue of the geometric Satake equivalence: 
\begin{theorem}[{Derived geometric Satake, \cite{bf-derived-satake}}]
There is an equivalence $\DMod_{G(\co)}(\Gr_G) \simeq \QCoh(\ld{\g}_\cc[2]/\ld{G}_\cc)$ of $\cc$-linear $\infty$-categories, where $\ld{\g}_\cc[2]$ is the derived $\cc$-scheme $\spec \Sym_\cc(\ld{\g}^\ast_\cc[-2])$.
\end{theorem}
\begin{remark}\label{simpler-bf}
The Bezrukavnikov-Finkelberg equivalence leads to a simpler equivalence on the level of local systems: $\Loc_{G_c}(\Omega G_c; \cc) \simeq \QCoh(\ld{\g}_\cc^\reg[2]/\ld{G}_\cc)$. This can be proved using \cite[Proposition 2.2.1]{ngo-ihes} and \cite[Proposition 2.8]{bfm}. This statement over the regular locus in fact plays a key role in proving the derived geometric Satake equivalence.
\end{remark}
Our goal in this article (partly inspired by Adams' quote above, the work \cite{hkr} of Hopkins-Kuhn-Ravenel corresponding to the diametric case of $G$ being a \textit{finite} group, and the 
%Coulomb branch of pure $4$d $\cN=2$ gauge theory on $\RR^3 \times S^1$ and the Coulomb branch of pure $5$d $\cN=1$ gauge theory on $\RR^2 \times E$ for an elliptic curve $E$ over $\cc$; see
discussion in \cite{teleman-icm} and \cref{coulomb}) is to begin exploring the analogous story when $\cc$ is replaced by a generalized cohomology theory. Specifically, we will replace $\cc$ with an even-periodic $\Eoo$-ring equipped with specific additional data. The idea of considering other coefficient cohomology theories in the context of geometric representation theory is not new; see \cite{ginzburg-kapranov-vasserot} for an early discussion of such ideas, as well as \cite{cautis-kamnitzer, lonergan-slides, yang-zhao-e-thy-quantum-group} for more recent work in this direction.

\begin{remark}
Part of the reason the derived contributions are vital to generalizing the geometric Satake equivalence is that when one considers sheaves with coefficients in a $2$-periodic $\Eoo$-ring (or any $\Eoo$-ring with nonzero homotopy in positive degrees), contributions from higher cohomology are circulated to degree $0$. For instance, the result of Bezrukavnikov-Finkelberg implies that $\Shv^c_{G(\co)}(\Gr_G(\cc); \cc[\beta^{\pm 1}]) \simeq \QCoh(\ld{\g}_\cc[2]/\ld{G}_\cc) \otimes_\cc \cc[\beta^{\pm 1}]$ where $|\beta|=2$; but this is in turn equivalent to $\QCoh(\ld{\g}_\cc/\ld{G}_\cc) \otimes_\cc \cc[\beta^{\pm 1}]$, which is \textit{not} the $2$-periodification of $\Rep(\ld{G}_\cc)$. However, let us note that in the setting of relative geometric Langlands (as discussed in \cite{sakellaridis-icm}), $2$-periodification is a rather destructive procedure: the particular shifts involved on the coherent side are extremely important, since they provide a geometric analogue of the point of evaluation of the Langlands dual L-function.
%We do not yet know how to resolve this in the higher chromatic context studied below.
\end{remark}

We will study a variant of a result of Arkhipov-Bezrukavnikov-Ginzburg (ABG) from \cite{abg-iwahori-satake}, which is closely related to the geometric Satake equivalence. Namely, let $I = G(\co) \times_G B$ denote the Iwahori subgroup of $G(\co)$. Then:
\begin{theorem}[Arkhipov-Bezrukavnikov-Ginzburg]
There is an equivalence $\DMod_I(\Gr_G) \simeq \IndCoh((\tilde{\ld{\cN}} \times_{\ld{\g}} \{0\})/\ld{G})$, where $\tilde{\ld{\cN}} = T^\ast(\ld{G}/\ld{B})$ is the Springer resolution. This is in turn equivalent to $\QCoh(\tilde{\ld{\g}}_\cc[2]/\ld{G}_\cc)$ by Koszul duality, where $\tilde{\ld{\g}}_\cc[2] = \ld{G} \times^{\ld{B}} \ld{\fr{b}}[2]$ is a shifted analogue of the Grothendieck-Springer resolution. 
\end{theorem}
\begin{remark}\label{simpler abg}
As in \cref{simpler-bf}, the ABG equivalence leads to a simpler equivalence on the level of local systems: $\Loc_{T_c}(\Omega G_c; \cc) \simeq \QCoh(\tilde{\ld{\g}}_\cc^\reg/\ld{G}_\cc)$. Upon $2$-periodification, we therefore see that $\Loc_{T_c}(\Omega G_c; \cc[\beta^{\pm 1}]) \simeq \QCoh(\tilde{\ld{\g}}_\cc^\reg/\ld{G}_\cc) \otimes_\cc \cc[\beta^{\pm 1}]$. Again, this statement over the regular locus in fact plays a key role in proving the ABG equivalence.
\end{remark}
%Taking global sections defines a functor $\Shv^c_I(\Gr_G(\cc); \cc[\beta^{\pm 1}]) \to \Mod_{\H_I^\ast(\Gr_G(\cc); \cc)} \otimes_\cc \cc[\beta^{\pm 1}]$.
%A first step towards reproving the main result of \cite{abg-iwahori-satake} using techniques similar to those of \cite{bf-derived-satake} is to describe $\Mod_{\H_I^\ast(\Gr_G(\cc); \cc)}$ in Langlands dual terms.
Note that pullback along the inclusion of a point into $\Gr_G(\cc)$ defines a symmetric monoidal functor $\Shv^c_I(\Gr_G(\cc); \cc[\beta^{\pm 1}]) \to \Shv^c_I(\ast; \cc[\beta^{\pm 1}])$, and there is an equivalence $\Loc_{T_c}(G_c; \cc[\beta^{\pm 1}]) \simeq \End_{\Loc_{T_c}(\Omega G_c; \cc[\beta^{\pm 1}])}(\Loc_{T_c}(\ast; \cc[\beta^{\pm 1}]))$. Using the ABG theorem, one can prove an equivalence 
\begin{equation}\label{abg-consequence}
    \Loc_{T_c}(G_c; \cc[\beta^{\pm 1}]) \simeq \QCoh(\ld{\fr{t}} \times_{\tilde{\ld{\g}}/\ld{G}} \ld{\fr{t}}) \otimes_\cc \cc[\beta^{\pm 1}],
\end{equation}
where the map $\ld{\fr{t}} \to \tilde{\ld{\g}}/\ld{G}$ is given by the Kostant slice, and $T_c$ acts on $G_c$ by conjugation.

The goal of this article is to study a generalization of \cref{simpler abg} and \cref{abg-consequence}.
Fix a complex-oriented even-periodic $\Eoo$-ring $A$, and let $\GG$ be an oriented group scheme in the sense of \cite{elliptic-ii}. If $T$ is a torus and $X$ is a sufficiently nice $T$-space, one can define a $\pi_0 A$-linear $\infty$-category $\Loc_{T}^\gr(X; A)$ of ``genuine $T$-equivariant $\Mod_A$-valued local systems\footnote{This is meant in the fully $\infty$-categorical sense, so it depends on the entire homotopy type (i.e., the entire fundamental $\infty$-groupoid) of $X$, and not just on the fundamental groupoid of $X$. For instance, in the nonequivariant setting, $\Loc(X; A)$ is the $\infty$-category $\Fun(X, \Mod_A)$ where $X$ is viewed as a Kan complex.} on $X$''; see \cref{categories-of-equiv-loc} and \cref{graded local systems}.
Let $\cM_T$ denote the Hom-stack $\Hom(\bX^\bull(T), \GG)$, and let $\cM_{T,0}$ denote its underlying stack over $\pi_0 A$. For instance, if $\GG_0$ is an elliptic curve, $\cM_{T,0}$ can be identified with the moduli stack (scheme) of $T$-bundles on $E$ of degree $0$ equipped with a trivialization at the zero section.
Let $\GG_0^\vee$ denote the group scheme $\Hom(\GG_0, B\GG_m)$ (this is a slight variant of the construction studied in \cite{moulinos-loop}). 
%If $X$ is a scheme, let $\cL_\GG X$ denote the ``$\GG$-loop space'' $\Map(\GG_0^\vee, X)$, so that if $X = BG$, then $\cL_\GG BG = \Bun_G(\GG_0^\vee)$.
Then, one of our main results is the following; we will unwind the statement in special cases below.

\begin{thmno}[See \cref{regular-satake} for a precise statement]
Suppose that $G$ is a connected and simply-connected semisimple algebraic group or a torus over $\cc$, and let $T$ act on $G$ by conjugation. Let $G_c$ denote the maximal compact subgroup of $G(\cc)$, and fix a principal nilpotent element of $\ld{\fr{n}}$. Fix a complex-oriented even-periodic $\Eoo$-ring $A$, and let $\GG$ be an oriented group scheme in the sense of \cite{elliptic-ii}. 
Assume that the underlying $\pi_0 A$-scheme $\GG_0$ is $\GG_a$, $\GG_m$, or an elliptic curve $E$.
Let $\Bun_{\ld{B}}^0(\GG_{0,\QQ}^\vee)^\reg$ denote the moduli stack of regular $\ld{B}$-bundles of degree zero on $\GG_{0,\QQ}^\vee$. Then, there is an $\E{2}$-monoidal equivalence of $\pi_0 A_\QQ$-linear $\infty$-categories
$$\Loc_{T_c}^\gr(\Omega G_c; A) \otimes \QQ \simeq \QCoh(\Bun_{\ld{B}}^0(\GG_{0,\QQ}^\vee)^\reg).$$
\end{thmno}
We view the above result as a first step towards describing $\Shv^c_{T_c}(\Omega G_c; A) \otimes \QQ$ in a manner analogous to \cite{abg-iwahori-satake}. We hope to complete this description in a sequel to this article, and further use the above result to revisit (the $2$-periodification of) the ABG equivalence. The basic point in the proof of \cref{regular-satake} is the computation of the $T_c$-equivariant $A$-homology $\pi_0 C^{T_c}_\ast(\Omega G_c; A)$ in terms of the Langlands dual $\ld{G}$. It is likely that the rationalization in \cref{regular-satake} is unnecessary, but we have not attempted to verify this.
\begin{remark}
Essentially the same argument shows that there is an $\E{2}$-monoidal equivalence of $\pi_0 A_\QQ$-linear $\infty$-categories
$$\Loc_{G_c}^\gr(\Omega G_c; A) \otimes \QQ \simeq \QCoh(\Bun_{\ld{G}}^{0,\ss}(\GG_{0,\QQ}^\vee)^\reg),$$
where $\Bun_{\ld{G}}^{0,\ss}(\GG_{0,\QQ}^\vee)^\reg$ denotes the moduli stack of regular semistable $\ld{G}$-bundles of degree zero. For simplicity, we will only focus on $T_c$-equivariant local systems.
\end{remark}
\begin{example}
When $G$ is a torus, it is easy to establish an analogue of the geometric Satake equivalence, even before rationalization: if $T$ is a torus over $\cc$, let $\ld{T}_A := \spec A[\bX_\ast(T)]$ denote the dual torus over $A$. Then there is an $\E{2}$-monoidal $A$-linear equivalence $\Loc_T(\Gr_T(\cc); A) \simeq \QCoh(\cL_\GG B\ld{T}_A)$; see \cref{torus-satake}. One can also ``quantize'' by considering loop-rotation equivariance, which results in a $\GG$-analogue of the algebra of differential operators on $\ld{T}$; see \cref{sec: quantized homology torus} for more. In \cref{sphere-coeffs}, we discuss the story for a torus where $A$ is replaced by the sphere spectrum $S$ --- already in this case, homotopy-theoretic considerations prevent one from describing $\Loc_T(\Gr_T(\cc); S)$ in terms of the algebraic geometry of some spectral stack over the sphere spectrum.
\end{example}

\begin{remark}
The reason that the left-hand side of \cref{regular-satake} is not merely $\Loc_{T_c}(\Omega G_c; \QQ) \otimes_\QQ A_\QQ$ (which could then be described by \cref{abg-consequence}) is that the rationalization of equivariant $A$-(co)homology is essentially never isomorphic to equivariant $A \otimes \QQ$-(co)homology. This is the key reason for why \cref{regular-satake} is not a consequence of the results of Arkhipov-Bezrukavnikov-Ginzburg. This perspective also features in \cite{chriss-ginzburg}. For example, if $X$ is a finite CW-complex equipped with an action of a group $H$, then $\KU^\ast(X) \otimes \QQ \cong \H^\ast(X; \QQ) \otimes_\QQ \QQ[\beta^{\pm 1}]$, but $\KU^\ast_H(X) \otimes \QQ$ is generally not isomorphic to $\H^\ast_H(X; \QQ) \otimes_\QQ \QQ[\beta^{\pm 1}]$. Indeed, they already differ if $X$ is a point: in this case, $\KU^\ast_H(X) \otimes \QQ$ is the rationalization of the representation ring of $H$, which is generally not isomorphic to $\H^\ast_H(X; \QQ) \otimes_\QQ \QQ[\beta^{\pm 1}]$ (for instance, if $H$ is finite, the latter is $\QQ[\beta^{\pm 1}]$).
\end{remark}
\cref{intro-mirror-dual-of-g} is closely related to the following instantiation of Langlands duality:
\begin{theorem}\label{intro-mirror-dual-of-g}
In the above setup (so that the underlying $\pi_0 A$-scheme $\GG_0$ is $\GG_a$, $\GG_m$, or an elliptic curve $E$), there is a ``$\GG$-Kostant slice'' $\kappa: (\cM_{\ld{T},0})_\QQ \to \Bun_{\ld{B}}^0(\GG_{0,\QQ}^\vee)^\reg$ over $\pi_0 A_\QQ$ such that there is an equivalence of $\pi_0 A_\QQ$-linear $\infty$-categories:
$$\Loc_{T_c}^\gr(G_c; A) \otimes \QQ \simeq \QCoh((\cM_{\ld{T},0})_\QQ \times_{\Bun_{\ld{B}}^0(\GG_{0,\QQ}^\vee)} (\cM_{\ld{T},0})_\QQ).$$
Here, $T_c$ acts on $G_c$ by conjugation.
\end{theorem}
\begin{remark}
Let $K_0(\Rep(G_c))$ denote the (complex) representation ring of $G_c$. In \cite{brylinski-zhang}, Brylinski and Zhang proved that there is an isomorphism $\KU_{G_c}^\ast(G_c) \cong \Omega^\ast_{K_0(\Rep(G_c))/\Z} \otimes_\Z \Z[\beta^{\pm 1}]$.
When $A = \KU$, one can use the Hochschild-Kostant-Rosenberg theorem to view the variant of \cref{intro-mirror-dual-of-g} for $\Loc_{G_c}^\gr(G_c; \KU) \otimes \QQ$ as a categorification of the Brylinski-Zhang isomorphism. See \cref{brylinski-zhang-section} for further discussion. In \cref{A-cohomology of Gr}, we also use Hochschild homology to describe a generalization of the $\hbar=0$ case of \cite[Theorem 1]{bf-derived-satake}, which computes the equivariant \textit{co}homology of $\Omega G_c$.
\end{remark}
\begin{remark}
Motivated by \cite[Theorem 1.1]{ganatra-pardon-shende}, one can heuristically interpret \cref{intro-mirror-dual-of-g} as describing a version of mirror symmetry for the wrapped Fukaya category of the symplectic orbifold $T^\ast(G_c/_\ad T_c)$, albeit with coefficients in the complex-oriented even-periodic $\Eoo$-ring $A$.
\end{remark}

%%%%%%%

Let us discuss \cref{regular-satake} individually for each case $\GG_0 = \GG_a, \GG_m$, and an elliptic curve.
\begin{enumerate}[leftmargin=\parindent,align=left,labelwidth=\parindent,labelsep=0pt]
    \item When $A = \QQ[\beta^{\pm 1}]$, \cref{regular-satake} describes an equivalence between $\Loc_{T_c}^\gr(\Omega G_c; \QQ[\beta^{\pm 1})$ and $\QCoh(\tilde{\ld{\g}}^\reg/\ld{G})$. This is a rather formal consequence of the following observation proved in \cref{bfm-self-intersect}:
    \begin{observe}
    There is a ``Kostant section'' $\kappa: \ld{\fr{t}} \to \tilde{\ld{\g}}/\ld{G}$ and a Cartesian square
    $$\xymatrix{
    \spec \H^T_0(\Gr_G(\cc); \QQ[\beta^{\pm 1}]) \cong (T^\ast \ld{T})^\bl \ar[r] \ar[d] & \fr{t} \cong \ld{\fr{t}} \ar[d]^-\kappa\\
    \ld{\fr{t}} \ar[r]_-\kappa & \tilde{\ld{\g}}/\ld{G},
    }$$
    where $(T^\ast \ld{T})^\bl$ is a particular affine blowup of $T^\ast \ld{T} \cong \ld{T} \times \fr{t}$. 
    \end{observe}
    This can be viewed as an analogue of \cite[Proposition 2.2.1]{ngo-ihes} and \cite[Proposition 2.8]{bfm}, and it can be used to reprove the rationalization of \cite[Theorem 6.1]{homology-langlands}. There is an isomorphism $\tilde{\ld{\g}}/\ld{G} \cong \ld{\fr{b}}/\ld{B}$, and in characteristic zero, this can be identified with $\Bun_{\ld{B}}^0(B\GG_a)$, viewed as the shifted tangent bundle of $B\ld{B}$. Moreover, there is an isomorphism $\spec \H^T_0(\Gr_G(\cc); \QQ[\beta^{\pm 1}]) \cong (T^\ast \ld{T})^\bl$, and $(T^\ast \ld{T})^\bl$ admits a $W$-action (via the $W$-action on $\ld{T}$ and $T^\ast_{\{1\}} \ld{T} \cong \fr{t}$) such that $(T^\ast \ld{T})^\bl\mmod W \cong \spec \H^G_0(\Gr_G(\cc); \QQ)$ is isomorphic to the group scheme of regular centralizers in $\ld{\fr{g}}$. See \cite{bfm} for further discussion.
    
    In this case, \cref{intro-mirror-dual-of-g} says that if $T$ acts on $G$ by conjugation, then there is an equivalence
    $$\Loc_{T_c}^\gr(G_c; \QQ[\beta^{\pm 1}]) \simeq \QCoh(\ld{\fr{t}}_\QQ \times_{\tilde{\ld{\g}}_\QQ/\ld{G}_\QQ} \ld{\fr{t}}_\QQ).$$
    Similarly, if $G$ acts on itself by conjugation, one obtains an equivalence
    $$\Loc_{G_c}^\gr(G_c; \QQ[\beta^{\pm 1}]) \simeq \QCoh(\ld{\fr{t}}_\QQ\mmod W \times_{\ld{\g}_\QQ/\ld{G}_\QQ} \ld{\fr{t}}_\QQ\mmod W).$$
    These equivalences can be de-periodified (\cref{no-2-periodification}).
    Motivated by \cite[Theorem 1.1]{ganatra-pardon-shende}, these equivalences suggest viewing $\ld{\fr{t}} \times_{\tilde{\ld{\g}}/\ld{G}} \ld{\fr{t}}$ (resp. $\ld{\fr{t}}\mmod W \times_{\ld{\g}/\ld{G}} \ld{\fr{t}}\mmod W$) as a (derived) mirror to the symplectic orbifold $T^\ast(G_c/_\ad T_c)$ (resp. $T^\ast(G_c/_\ad G_c)$). Concretely, these results show that if $f$ is a regular nilpotent element of $\ld{\g}$ and $Z_f(\ld{B})$ is its centralizer in $\ld{B}$, then there is an equivalence
    $$\Loc^\gr(G_c; \QQ[\beta^{\pm 1}]) \simeq \QCoh(Z_f(\ld{B}_\QQ));$$
    therefore, $Z_f(\ld{B}_\QQ)$ is a mirror to $G(\cc) = T^\ast(G_c)$ viewed as a symplectic manifold. These results are not new, and can easily be deduced from the work of Bezrukavnikov-Finkelberg \cite{bf-derived-satake} and Yun-Zhu \cite{homology-langlands}. Notice that if $G_c = T_c$, then we are simply stating that there is an equivalence $\Loc(T_c; \QQ[\beta^{\pm 1}]) \simeq \QCoh(\ld{T})$, given by taking monodromy.
    
    \begin{remark}
    Upon adding loop rotation equivariance, there is an equivalence between $\Loc_{T_c \times S^1_\rot}^\gr(\Omega G_c; \cc)$ and a particular localization of the universal category $\ld{\co}^\univ = U_\hbar(\ld{\g})\modc^{\ld{N}, (\ld{T},w)}$ from \cite[Section 2.4]{univ-cat-o}; this is a consequence of \cref{looped-quantized-abg} and \cref{completion-morita}.
    \end{remark}
    See \cref{3d-sl2} for an explicit description of $\H^{G\times S^1_\rot}_\ast(\Gr_G(\cc); \cc)$ when $G = \SL_2$. From the homotopical perspective, the action of $S^1$ by loop rotation on $\Gr_G(\cc)$ arises by viewing $\Gr_G(\cc) \simeq \Omega^\lambda BG(\cc)$, where $\lambda$ is the $2$-dimensional rotation representation of $S^1$; in other words, $\Gr_G(\cc)$ admits the structure of a framed $\E{2}$-algebra, and the action of $S^1$ is via change-of-framing.
    
    \item When $A = \KU$, \cref{regular-satake} describes an equivalence between $\Loc_{T_c}^\gr(\Omega G_c; \KU) \otimes \QQ$ and $\QCoh(\tilde{\ld{G}}^\reg_\QQ/\ld{G}_\QQ)$, where $\tilde{\ld{G}}_\QQ^\reg/\ld{G}_\QQ$ is the regular locus in the stacky quotient of the multiplicative Grothendieck-Springer resolution $\tilde{\ld{G}}_\QQ = \ld{G}_\QQ \times_{\ld{B}_\QQ} \ld{B}_\QQ$. As above, this is a rather formal consequence of the following observation, which is a \textit{multiplicative} analogue of \cite[Proposition 2.2.1]{ngo-ihes} and \cite[Proposition 2.8]{bfm}:
    \begin{observe}
    There is a ``Kostant section'' $\kappa: \ld{T} \to \tilde{\ld{G}}/\ld{G}$ and a Cartesian square
    $$\xymatrix{
    \spec \pi_0 C^T_\ast(\Gr_G(\cc); \KU) \otimes \QQ \cong (\ld{T} \times T)^\bl \ar[r] \ar[d] & \ld{T} \ar[d]^-\kappa\\
    \ld{T} \ar[r]_-\kappa & \tilde{\ld{G}}/\ld{G} \simeq \Bun_{\ld{B}}^0(S^1),
    }$$
    where $(\ld{T} \times T)^\bl$ is a particular affine blowup of $\ld{T} \times T$. Moreover, there is an isomorphism $\spec \pi_0 C^T_\ast(\Gr_G(\cc); \KU) \otimes \QQ$ and $(\ld{T}\times T)^\bl$.
    There is also a $W$-action on $(\ld{T} \times T)^\bl$ (by the $W$-action on $T$ and $\ld{T}$) such that $(\ld{T} \times T)^\bl\mmod W \cong \spec \pi_0 C^G_\ast(\Gr_G(\cc); \KU) \otimes \QQ$ is isomorphic to the group scheme of regular centralizers in $\ld{G}$. Again, see \cite{bfm} for further discussion.
    \end{observe}
    
    In this case, \cref{intro-mirror-dual-of-g} says that if $T$ acts on $G$ by conjugation, then there is an equivalence
    $$\Loc_{T_c}^\gr(G_c; \KU) \otimes \QQ \simeq \QCoh(\ld{T}_\QQ \times_{\tilde{\ld{G}}_\QQ/\ld{G}_\QQ} \ld{T}_\QQ).$$
    Similarly, if $G$ acts on itself by conjugation, one obtains an equivalence
    $$\Loc_{G_c}^\gr(G_c; \KU) \otimes \QQ \simeq \QCoh(\ld{T}_\QQ\mmod W \times_{\ld{G}_\QQ/\ld{G}_\QQ} \ld{T}_\QQ\mmod W).$$
    If $\{f\}$ is a regular unipotent element of $\ld{G}_\QQ$ (determined by the image of the origin in $\ld{T}_\QQ\mmod W$ under the multiplicative Kostant slice), and $Z_f^\mu(\ld{B}_\QQ)$ is the centralizer of $f\in \ld{G}_\QQ$, then the preceding equivalence in turn implies an equivalence
    $$\Loc^\gr(G_c; \KU) \otimes \QQ \simeq \QCoh(Z_f^\mu(\ld{B}_\QQ)).$$
    Therefore, $Z_f^\mu(\ld{B}_\QQ)$ can be viewed as a $\KU$-theoretic mirror to $G(\cc) = T^\ast(G_c)$ viewed as a symplectic manifold.
    The main input into these results are not new, and can be deduced from the work of Bezrukavnikov-Finkelberg-Mirkovic \cite{bfm}. Notice that if $G_c = T_c$, then we are simply stating that there is an equivalence $\Loc(T_c; \KU) \simeq \QCoh(\ld{T}_\KU)$, given by taking monodromy.
    
    \begin{remark}
    We expect (see \cref{oq-univ} for a more precise statement) that upon adding loop rotation equivariance, there is an equivalence between $\Loc_{T_c \times S^1_\rot}^\gr(\Omega G_c; \KU) \otimes \QQ$ and a particular localization of the quantum universal category $\ld{\co}^\univ_q$ from \cite[Section 2.4]{univ-cat-o}. Using the calculations in this article, this expected equivalence reduces to proving an analogue of \cite[Theorem 8.1.2]{ginzburg-whittaker} for the quantum group and the multiplicative nil-Hecke algebra; such a conjecture also appears as \cite[Conjecture 3.17]{finkelberg-tsymbaliuk}.
    
    We also expect (see \cref{zetap-oq}) that there is an equivalence between $\Loc_{T_c \times \mu_{p,\rot}}^\gr(\Omega G_c; \KU)[\frac{1}{q-1}]$ and a particular localization of $\ld{\co}^\univ_{\zeta_p}$, i.e., the quantum universal category $\ld{\co}$ at a primitive $p$th root of unity.
    \end{remark}
    The reader is referred to \cref{4d-sl2} for an explicit description of $\pi_0 C^{G\times S^1_\rot}_\ast(\Gr_G(\cc); \KU) \otimes \QQ$ when $G = \SL_2$.
    
    \item Suppose $A$ is a complex-oriented even-periodic $\Eoo$-ring and $\GG$ is an oriented elliptic curve over $A$ (in the sense of \cite{elliptic-ii}). Let $E = \GG_0$ be the underlying classical scheme of $\GG$ over the classical ring $\pi_0(A)$, so that $E$ is an elliptic curve, and let $E^\vee$ be the dual elliptic curve. The Cartesian squares from (a) and (b) above can be generalized to this setting (see \cref{borel-intersection}). For simplicity, let us explain this in the case $G = \SL_2$, i.e., $\ld{G} = \PGL_2$. 
    \begin{observe}
    Then, there is a ``Kostant section'' $\kappa: E = \Pic^0(E^\vee) \to \Bun_{\ld{B}}^0(E^\vee)$ which sends a line bundle $\cL$ to the trivial extension $\co_{E^\vee} \subseteq \co_{E^\vee} \oplus \cL$ if $\cL \not\cong \co_{E^\vee}$, and to the Atiyah extension $\co_{E^\vee}\subseteq \cf_2 \twoheadrightarrow \co_{E^\vee}$ from \cite{atiyah-bundle-elliptic} if $\cL$ is trivial. %(Recall that $\cf_2$ is the rank $2$ bundle on $E^\vee$ defined by a generator of $\Ext^1(\co_{E^\vee}, \co_{E^\vee}) \cong \H^1(E^\vee)$.)
    Note that by construction, the $\ld{G}$-bundle underlying $\kappa(\cL)$ is semistable of degree $0$. Moreover, \cref{borel-intersection} says that there is a Cartesian square
    $$\xymatrix{
    (\GG_m \times E)^\bl \ar[r] \ar[d] & E \ar[d]^-\kappa\\
    E \ar[r]_-\kappa & \Bun_{\ld{B}}^0(E^\vee),
    }$$
    where $(\GG_m \times E)^\bl$ is a particular affine blowup of $\GG_m \times E$. \footnote{The desired affine blowup $(\GG_m \times E)^\bl$ is obtained by blowing up $\GG_m\times E$ at the locus cut out by the zero sections of $\GG_m$ and $E$, and deleting the proper preimage of the zero section of $E$; see also \cite[Lemma 4.1]{bfm}.} 
    
    Notice that $\GG_m\times E$ admits an action of $ W = \Z/2$, via inversion on $\GG_m$ and $E$; this extends to an action of $\Z/2$ on $(\GG_m\times E)^\bl$, and the above diagram suggests viewing $(\GG_m\times E)^\bl\mmod (\Z/2)$ as an \textit{elliptic} analogue of the group scheme of regular centralizers.
    \end{observe}
    
    Furthermore, there is an isomorphism 
    $$\Gamma((\GG_m \times E)^\bl; \co_{(\GG_m \times E)^\bl}) \cong \pi_0 C^T_\ast(\Gr_G(\cc); A) \otimes \QQ$$
    between the coherent cohomology of $(\GG_m \times E)^\bl$ and the rationalization of the $T$-equivariant $A$-homology of $\Gr_G(\cc)$. Using this, \cref{regular-satake} shows that there is an equivalence between a variant of $\Loc_{T_c}^\gr(\Omega G_c; A) \otimes \QQ$ and an explicit full subcategory of $\QCoh(\Bun_{\ld{B}}^0(E^\vee))$.
    
    In this case, \cref{intro-mirror-dual-of-g} says that if $T$ acts on $G$ by conjugation, then there is an equivalence
    $$\Loc_{T_c}^\gr(G_c; A) \otimes \QQ \simeq \QCoh(E \times_{\Bun_{\ld{B}}^0(E^\vee)} E) \otimes_{\pi_0 A} \pi_0 A_\QQ.$$
    If $\{\co_{E^\vee}\subseteq \cf_2\}\in \Bun_{\ld{B}}^0(E^\vee)$ denotes the Atiyah bundle, then let $Z_f^E(\ld{B}) := (\{\co_{E^\vee}\subseteq \cf_2\} \times_{\Bun_{\ld{B}}^0(E^\vee)} E)$ be the ``centralizer in $\ld{B}$ of the regular `elli-potent' element $\{\co_{E^\vee}\subseteq \cf_2\}\in \Bun_{\ld{B}}^0(E^\vee)$''. There is then an equivalence
    $$\Loc^\gr(G_c; A) \otimes \QQ \simeq \QCoh(Z_f^E(\ld{B})) \otimes_{\pi_0 A} \pi_0 A_\QQ.$$
    Therefore, $Z_f^E(\ld{B})$ can be viewed as an $A$-theoretic mirror to $G(\cc) = T^\ast(G_c)$ viewed as a symplectic manifold.
\end{enumerate}

\begin{remark}
One might hope that these results hold without rationalization, but we do not know how to prove such a statement. In the case of $\KU$, for instance, the key obstruction is that we do not know whether the $2$-periodification of $\tilde{\ld{G}}^\reg_\QQ/\ld{G}_\QQ$ can be lifted to a flat stack $(\tilde{\ld{G}}^\reg/\ld{G})_\KU$ over $\KU$. If it does lift, then it seems reasonable to expect a $\KU$-linear equivalence of the form $\Loc_{T_c}(\Omega G_c; A) \simeq \QCoh((\tilde{\ld{G}}^\reg/\ld{G})_\KU)$.
\end{remark}

%In a sequel to this article (Part II), we will use \cref{regular-satake} to study the rationalization of the entire category of $T_c$-equivariant constructible sheaves of $A$-modules on $\Gr_G(\cc)$.
%%One important hurdle is that the $\Eoo$-$\QQ$-algebra of functions on the \textit{derived} stacks $\ld{G}/\ld{G}$ and $\ld{G}^\reg/\ld{G}$ (temporarily omitting the subscript $\QQ$ for simplicity) do not agree, because the algebraic Hartogs lemma fails to hold derivedly\footnote{For instance, if $k$ is a commutative ring, then $\Gamma(\AA^2-\{0\}; \co_{\AA^2 - \{0\}})$ has $\pi_0$ given by $k[x,y]$, but $\pi_{-1}$ is given by the local cohomology $k[x,y]/(x^\infty, y^\infty)$.}. This difficulty is often avoided when studying constructible sheaves with coefficients in $\cc$ by cleverly using tilting sheaves, etc; see, e.g., \cite[Sections 5.5 and 5.6]{riche}. Unfortunately, these techniques are often unavailable when studying constructible sheaves with more exotic coefficients, so we are forced to find a workaround. In Part II, this will be done using the techniques of \cite{soergel-etheory}.

In \cref{coulomb}, we discuss some motivation for this article stemming from the Coulomb branches of $3$d $\cN=4$, $4$d $\cN=2$, and $5$d $\cN=1$ pure gauge theories (i.e., no matter). We also give explicit generators and relations for the Coulomb branches of $3$d $\cN=4$ and $4$d $\cN=2$ pure gauge theories with gauge group $\SL_2$ (i.e., $\pi_0 C_\ast^G(\Gr_G(\cc); \QQ)$ and $\pi_0 C_\ast^G(\Gr_G(\cc); \KU)$ with $G = \SL_2$). The $4$-dimensional case is a $q$-analogue of the quantization of the Atiyah-Hitchin manifold from \cite[Equation 5.51]{bullimore-dimofte-gaiotto}.

We will use the following notation throughout; furthermore, the reader should keep in mind that \textit{everything} in this article will be derived, unless explicitly mentioned otherwise.
\begin{notation}\label{group-notation}
Let $G$ be a connected (often simply-connected) semisimple group over $\cc$ (or a torus). Fix a maximal torus $T\subseteq B$ contained in a Borel subgroup of $G$. Let $U = [B,B]$ denote the unipotent radical of $B$, so that $B/U \cong T$. Let $\Phi$ be the set of roots of $G$, $\Phi^+$ the set of positive roots, and $\Delta$ a set of simple roots. Let $W$ be the Weyl group; if $w\in W$, let $\dot{w}\in N_G(T)$ denote a lift of $w$ to the normalizer of $T$ in $G$. Let 
%$\bX^\ast$ denote the lattice of characters of $T$, $\bX_\ast$ the lattice of cocharacters of $T$,
$\Lambda$ denote the weight lattice, and $\Lambda^+ = \Lambda^\pos$ the set of dominant weights.
We will also follow other standard notation in homotopy theory: for instance, $\Top$ will denote the $\infty$-category of spaces, and $\Sp$ will denote the $\infty$-category of spectra.
\end{notation}

There has been some work done previously towards analogues of the geometric Satake equivalence with other coefficients. For instance, when $A = \KU$, a conjecture was proposed in \cite{cautis-kamnitzer}; in a similar vein, a discussion of the case $A = \KU$ is the content of the talk \cite{lonergan-slides}.
In \cite{yang-zhao-e-thy-quantum-group}, Yang and Zhao study a higher chromatic analogue of quantum groups, and it would be interesting to study the relationship between the present article and their work. After this paper was written, the preprint \cite{zhong-equiv-homology-of-Gr} was posted on the arXiv; it is concerned with ideas similar to the ones studied here.
Our work is closely related to the exciting program of Ben-Zvi--Sakellaridis--Venkatesh (see \cite{sakellaridis-icm, sakellaridis-mit-nt} for an overview); we hope to describe this relationship in future work.

\subsection{Acknowledgements}
I'd like to acknowledge Lin Chen, Charles Fu, Tom Gannon, and Kevin Lin for helpful conversations and for entertaining my numerous silly questions. I'm also grateful to Victor Ginzburg for a very enlightening discussion, and Pavel Safronov for a useful email.
Thanks to Ben Gammage for discussions which helped shape my understanding of some of the topics in \cref{coulomb}, and to Hiraku Nakajima for a very informative email exchange on the same topic.
Part of this work started after I took a class taught by Roman Bezrukavnikov; I'm very grateful to him for introducing me to \cite{bfm}, which led me down the beautiful road to geometric representation theory.
Last, but certainly far from least, the influence, support, advice, and encouragement of my advisors Dennis Gaitsgory and Mike Hopkins is evident throughout this project; I cannot thank them enough.