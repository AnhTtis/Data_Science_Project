\documentclass[10pt]{amsproc}
\usepackage[utf8]{inputenc}
%\usepackage[margin=1.375in]{geometry}
%\usepackage{setspace}
%\setstretch{1.25}
%\usepackage{times}
%\usepackage[english,russian]{babel}
\usepackage{hyperref}       % hyperlinks
\usepackage{url}            % simple URL typesetting
\usepackage{booktabs}       % professional-quality tables
\usepackage{amsfonts}       % blackboard math symbols
\usepackage{tablefootnote}
\usepackage{verbatim}
\usepackage{nicefrac}       % compact symbols for 1/2, etc.
\usepackage{microtype}      % microtypography
\usepackage{lipsum}
\usepackage{mathtools}

\usepackage{amsmath,amssymb}
\usepackage{algorithm,algorithmic}
\usepackage{pifont}
\usepackage{cases}
\usepackage{subcaption,graphicx}
\usepackage{stackengine}    % circled symbols
\usepackage{wrapfig}
\usepackage{enumitem}

%\newtheorem{theorem}{Theorem}[section]
%\newtheorem{corollary}[theorem]{Corollary}
%\newtheorem{lemma}[theorem]{Lemma}
\newtheorem{assumption}[theorem]{Assumption}
%\newtheorem{definition}[theorem]{Definition}
%\newtheorem{remark}[theorem]{Remark}
%\newtheorem{proposition}[theorem]{Proposition}

\newcommand*{\LargerCdot}{\raisebox{-0.25ex}{\scalebox{2.4}{$\cdot$}}}
\newcommand{\Sum}{\displaystyle\sum\limits}
\newcommand{\Max}{\max\limits}
\newcommand{\Min}{\min\limits}
\newcommand{\fromto}[3]{{#1}=\overline{{#2},\,{#3}}}
\newcommand{\floor}[1]{\left\lfloor{#1}\right\rfloor}
\newcommand{\ceil}[1]{\left\lceil{#1}\right\rceil}

\newcommand{\tild}{\widetilde}
\newcommand{\eps}{\varepsilon}
\newcommand{\lam}{\lambda}
\newcommand{\ol}{\overline}
\newcommand{\one}{\mathbf{1}}
\newcommand{\cset}{\mathcal{C}}
%\newcommand{\Breg}{\mathcal{D}_{h}}
%\newcommand{\PBreg}{\mathbb{D}_{h}}

%\newcommand{\EndProof}{\begin{flushright}$\square$\end{flushright}}

\newcommand{\circledOne}{\text{\ding{172}}}
\newcommand{\circledTwo}{\text{\ding{173}}}
\newcommand{\circledThree}{\text{\ding{174}}}
\newcommand{\circledFour}{\text{\ding{175}}}
\newcommand{\circledFive}{\text{\ding{176}}}
\newcommand{\circledSix}{\text{\ding{177}}}
\newcommand{\circledSeven}{\text{\ding{178}}}
\newcommand{\circledEight}{\text{\ding{179}}}
\newcommand{\circledNine}{\text{\ding{180}}}
\newcommand{\circledTen}{\text{\ding{181}}}
\newcommand{\balashstar}{\stackMath\mathbin{\stackinset{c}{0ex}{c}{0ex}{\text{\ding{83}}}{\bigcirc}}}
\renewcommand\balashstar{\stackMath\mathbin{\stackinset{c}{0ex}{c}{0ex}{\ast}{\bigcirc}}}


\renewcommand{\le}{\leqslant}
\renewcommand{\ge}{\geqslant}
\renewcommand{\hat}{\widehat}

\newcommand{\numberthis}{\addtocounter{equation}{1}\tag{\theequation}}


\DeclareMathOperator*{\argmin}{arg\,min}
\DeclareMathOperator*{\argmax}{arg\,max}
\DeclareMathOperator*{\Argmin}{Arg\,min}
\DeclareMathOperator*{\Argmax}{Arg\,max}
\DeclareMathOperator{\spn}{span}
\DeclareMathOperator{\kernel}{Ker}
\DeclareMathOperator{\image}{Im}
\DeclareMathOperator{\prox}{prox}
\DeclareMathOperator{\proj}{Proj}
\DeclareMathOperator{\col}{col}
\DeclareMathOperator{\diag}{diag}

\newcommand{\N}{\mathbb{N}}
\newcommand{\R}{\mathbb{R}}
\newcommand{\Z}{\mathbb{Z}}
\newcommand{\V}{\mathbb{V}}
\newcommand{\E}{\mathbb{E}}
%\newcommand{\P}{\mathbb{P}}
\newcommand{\I}{\mathbb{I}}
\newcommand{\F}{\mathbb{F}}

\newcommand{\mA}{{\bf A}}
\newcommand{\mB}{{\bf B}}
\newcommand{\mC}{{\bf C}}
\newcommand{\mD}{{\bf D}}
\newcommand{\mE}{{\bf E}}
\newcommand{\mF}{{\bf F}}
\newcommand{\mG}{{\bf G}}
\newcommand{\mH}{{\bf H}}
\newcommand{\mI}{{\bf I}}
\newcommand{\mJ}{{\bf J}}
\newcommand{\mK}{{\bf K}}
\newcommand{\mL}{{\bf L}}
\newcommand{\mM}{{\bf M}}
\newcommand{\mN}{{\bf N}}
\newcommand{\mO}{{\bf O}}
\newcommand{\mP}{{\bf P}}
\newcommand{\mQ}{{\bf Q}}
\newcommand{\mR}{{\bf R}}
\newcommand{\mS}{{\bf S}}
\newcommand{\mT}{{\bf T}}
\newcommand{\mU}{{\bf U}}
\newcommand{\mV}{{\bf V}}
\newcommand{\mW}{{\bf W}}
\newcommand{\mX}{{\bf X}}
\newcommand{\mY}{{\bf Y}}
\newcommand{\mZ}{{\bf Z}}

\newcommand{\cA}{{\mathcal{A}}}
\newcommand{\cB}{{\mathcal{B}}}
\newcommand{\cC}{{\mathcal{C}}}
\newcommand{\cD}{{\mathcal{D}}}
\newcommand{\cE}{{\mathcal{E}}}
\newcommand{\cF}{{\mathcal{F}}}
\newcommand{\cG}{{\mathcal{G}}}
\newcommand{\cH}{{\mathcal{H}}}
\newcommand{\cI}{{\mathcal{I}}}
\newcommand{\cJ}{{\mathcal{J}}}
\newcommand{\cK}{{\mathcal{K}}}
\newcommand{\cL}{{\mathcal{L}}}
\newcommand{\cM}{{\mathcal{M}}}
\newcommand{\cN}{{\mathcal{N}}}
\newcommand{\cO}{{\mathcal{O}}}
\newcommand{\cP}{{\mathcal{P}}}
\newcommand{\cQ}{{\mathcal{Q}}}
\newcommand{\cR}{{\mathcal{R}}}
\newcommand{\cS}{{\mathcal{S}}}
\newcommand{\cT}{{\mathcal{T}}}
\newcommand{\cU}{{\mathcal{U}}}
\newcommand{\cV}{{\mathcal{V}}}
\newcommand{\cW}{{\mathcal{W}}}
\newcommand{\cX}{{\mathcal{X}}}
\newcommand{\cY}{{\mathcal{Y}}}
\newcommand{\cZ}{{\mathcal{Z}}}

\newcommand{\ba}{{\bf a}}
\newcommand{\bb}{{\bf b}}
\newcommand{\bc}{{\bf c}}
\newcommand{\bd}{{\bf d}}
\newcommand{\be}{{\bf e}}
%\newcommand{\bf}{{\bf f}}
\newcommand{\bg}{{\bf g}}
\newcommand{\bh}{{\bf h}}
\newcommand{\bi}{{\bf i}}
\newcommand{\bj}{{\bf j}}
\newcommand{\bk}{{\bf k}}
\newcommand{\bl}{{\bf l}}
\newcommand{\bm}{{\bf m}}
\newcommand{\bn}{{\bf n}}
\newcommand{\bo}{{\bf o}}
\newcommand{\bp}{{\bf p}}
\newcommand{\bq}{{\bf q}}
\newcommand{\br}{{\bf r}}
\newcommand{\bs}{{\bf s}}
\newcommand{\bt}{{\bf t}}
\newcommand{\bu}{{\bf u}}
\newcommand{\bv}{{\bf v}}
\newcommand{\bw}{{\bf w}}
\newcommand{\bx}{{\bf x}}
\newcommand{\by}{{\bf y}}
\newcommand{\bz}{{\bf z}}

\newcommand{\ds}{\displaystyle}
\newcommand{\norm}[1]{\left\| #1 \right\|}
\newcommand{\normtwo}[1]{\left\| #1 \right\|_2}
\newcommand{\sqn}[1]{\norm{#1}_2^2}
\newcommand{\angles}[1]{\left\langle #1 \right\rangle}
\newcommand{\cbraces}[1]{\left( #1 \right)}
\newcommand{\sbraces}[1]{\left[ #1 \right]}
\newcommand{\braces}[1]{\left\{ #1 \right\}}
\def\<#1,#2>{\langle #1,#2\rangle}

\newcommand{\sigmamax}{\sigma_{\max}(\cA)}
\newcommand{\sigmamaxsqr}{\sigma_{\max}^2(\cA)}
\newcommand{\sigmaminplus}{\sigma_{\min}^+(\cA)}
\newcommand{\sigmaminplussqr}{(\sigma_{\min}^+(\cA))^2}

\usepackage[colorinlistoftodos,bordercolor=blue,backgroundcolor=blue!20,linecolor=blue,textsize=scriptsize]{todonotes}
\newcommand{\arogozin}[1]{\todo[inline]{{\textbf{Alexander R.:} \emph{#1}}}}
\newcommand{\schezhegov}[1]{\todo[inline]{{\textbf{Savelii C.:} \emph{#1}}}}

%\setlength{\epigraphwidth}{\textwidth}

\title[Chromatic aberrations of geometric Satake]{Chromatic aberrations of geometric Satake over the regular locus}
\author{S. K. Devalapurkar}
\address{1 Oxford St, Cambridge, MA 02139}
\email{sdevalapurkar@math.harvard.edu, \today}
\thanks{Part of this work was done when the author was supported by the PD Soros Fellowship and NSF DGE-2140743. The present article is a preliminary version, so any comments and suggestions for improving it are greatly appreciated! I'll post major updates to the arXiv, but I'll upload minor edits to my website; so please see my website for the most up-to-date version.}

\begin{document}

\maketitle

\epigraph{It is an established practice to take old theorems about ordinary homology, and generalise them so as to obtain theorems about generalised homology theories.}{J. F. Adams, \cite{adams-lectures-coh}}

\begin{abstract}
    Let $G$ be a connected and simply-connected semisimple group over $\cc$, let $G_c$ be a maximal compact subgroup of $G(\cc)$, and let $T$ be a maximal torus. The derived geometric Satake equivalence of Bezrukavnikov-Finkelberg localizes to an equivalence between a full subcategory of $\Loc_{G_c}(\Omega G_c; \cc)$ and $\QCoh(\ld{\g}^{\reg}[2]/\ld{G})$, which can be thought of as a version of the geometric Satake equivalence ``over the regular locus''. In this article, we study the story when $\Loc_{T_c}(\Omega G_c; \cc)$ is replaced by the $\infty$-category of $T$-equivariant local systems of $A$-modules over $\Gr_G(\cc)$, where $A$ is a complex-oriented even-periodic $\Eoo$-ring equipped with an oriented group scheme $\GG$. We show that upon rationalization, $\Loc_{T_c}(\Omega G_c; A)$, which was studied variously by Arkhipov-Bezrukavnikov-Ginzburg and Yun-Zhu when $A = \cc[\beta^{\pm 1}]$, can be described in terms of the spectral geometry of various Langlands-dual stacks associated to $A$ and $\GG$. For example, this implies that if $A$ is an elliptic cohomology theory with elliptic curve $E$, then $\Loc_{T_c}(\Omega G_c; A) \otimes \QQ$ can be described via the moduli stack of $\ld{B}$-bundles of degree $0$ on $E^\vee$.
\end{abstract}

\tableofcontents
\newpage

\section{Introduction}
\section{Introduction}
\label{sec:introduction}
% \begin{itemize}
%     % Diffusion of FL
%     \item {\st{Diffusion of FL}}
%     % Security threats to FL
%     \item {\st{Security threats to FL with particular focus on model poisoning}}
%     % Limitations of existing countermeasures
%     \item {\st{Current countermeasures (e.g., KRUM) and their limitations}}
%     % Proposed method and its advantages
%     \item {\st{Intuitive description of the proposed method and its difference (i.e., advantages) w.r.t. state of the art}}
%     % Main contributions
%     \item {\st{Summary of the main contributions of this work}}
%     % Paper's structure and organization
%     \item {\st{Paper's structure and organization}}
% \end{itemize}

% Diffusion of FL
Recently, {\em federated learning} (FL) has emerged as the leading paradigm for training distributed, large-scale, and privacy-preserving machine learning (ML) systems~\cite{mcmahan2017googleai,mcmahan2017aistats}. 
The core idea of FL is to allow multiple edge clients to collaboratively train a shared, global model without disclosing their local private training data.
%Specifically, an FL system consists of a central server and many edge clients; 
A typical FL round involves the following steps: {\em(i)} the server randomly picks some clients and sends them the current, global model; {\em(ii)} each selected client locally trains its model with its own private data; then, it sends the resulting local model to the server;\footnote{Whenever we refer to global/local model, we mean global/local model {\em parameters}.} {\em(iii)} the server updates the global model by computing an \emph{aggregation function}, usually the average (FedAvg), on the local models received from clients.
% \begin{enumerate}
%     \item[{\em(i)}] the server sends the current, global model to the clients and appoints some of them for training;
%     \item[{\em(ii)}] each selected client locally trains its copy of the global model with its own private data; then, it sends the resulting local model back to the server;\footnote{Whenever we refer to global/local model, we mean global/local model {\em parameters}.}
%     \item[{\em(iii)}] the server updates the global model by computing an \emph{aggregation function} on the local models received from clients (by default, the average, also referred to as FedAvg~\cite{mcmahan2017aistats}).
% \end{enumerate}
This process goes on until the global model converges. %(e.g., after a certain number of rounds or other similar stopping criteria).
%\\
% The advantages of FL over the traditional, centralized learning paradigm are undoubtedly clear in terms of flexibility/scalability (clients can join/disconnect from the FL network dynamically), network communications (only model weights\footnote{We will use \textit{parameters} and \textit{weights} interchangeably.} are exchanged between clients and server), and privacy (each client's private training data is kept local at the client's end and not uploaded to the server).
\\
% Security threats to FL
%However, the growing adoption of FL also raises security concerns~\cite{costa2022covert}, particularly about its confidentiality, integrity, and availability.
Although its advantages over standard ML, FL also raises security concerns~\cite{costa2022covert}. %, particularly about its confidentiality, integrity, and availability~\cite{costa2022covert}.
% OLD, LONG VERSION
% Indeed, some work deals with privacy leakage that may expose the local data of some clients~\cite{melis2019sp}. 
% A large body of work, instead, investigates attacks that usually aim to detriment the predictive accuracy of the learned global model. For instance, \emph{data poisoning} attacks achieve this goal by letting an adversary pollute the training set of some corrupt FL clients with maliciously crafted examples~\cite{jagielski2018sp}.
% Similarly, in \emph{model poisoning} the attacker attempts to tweak the global model weights~\cite{bhagoji2019pmlr} by directly perturbing the local model's weights of some infected FL clients before these are sent to the central server for aggregation, usually via so-called Byzantine attacks. 
% It turns out that Byzantine model poisoning attacks severely impact standard FedAvg; therefore, more robust aggregation functions must be designed to make FL systems secure.
Here, we focus on \emph{untargeted model poisoning} attacks~\cite{bhagoji2019pmlr}, where an adversary attempts to tweak the global model weights %\footnote{We will use the terms \textit{parameters} and \textit{weights} interchangeably.} 
by directly perturbing the local model's parameters of some infected clients before these are sent to the central server for aggregation.
In doing so, the adversary aims to jeopardize the global model \textit{indiscriminately} at inference time.
Such model poisoning attacks severely impact standard FedAvg; therefore, more robust aggregation functions must be designed to secure FL systems.
\\
% In this paper, we focus on designing a novel robust aggregation scheme at the server's end to contrast the effect of Byzantine model poisoning attacks.
%
% Current countermeasures and their limitations
%Several countermeasures have been proposed in the literature to combat model poisoning attacks on FL systems.
% Some methods use simple statistics more robust than plain average to smooth the impact of malicious updates (e.g., Trimmed Mean and FedMedian~\cite{yin2018icml}). 
% Other defenses implement outlier detection techniques to discard malicious updates from the aggregation performed at the server's end. Those are either based on heuristics (e.g., Krum/Multi-Krum~\cite{blanchard2017nips} and Bulyan~\cite{mhamdi2018pmlr}) or data-driven approaches (e.g., K-means clustering~\cite{shen2016acm} or DnC via spectral analysis~\cite{shejwalkar2021ndss}). 
% Finally, some strategies rely on a centralized ``source of trust'' to spot potential malicious updates (e.g., FLTrust~\cite{cao2020fltrust}).
% Several countermeasures have been proposed in the literature to combat model poisoning attacks on FL systems, i.e., to discard possible malicious local updates from the aggregation performed at the server's end. 
% These techniques range from simple statistics more robust than plain average (e.g., Trimmed Mean and FedMedian~\cite{yin2018icml}) to outlier detection heuristics (e.g., Krum/Multi-Krum~\cite{blanchard2017nips} and Bulyan~\cite{mhamdi2018pmlr}) or data-driven approaches (e.g., spectral analysis via K-means clustering~\cite{shen2016acm} or spectral analysis), or methods based on ``source of trust'' (e.g., FLTrust~\cite{cao2020fltrust}).
% OLD, LONG VERSION
%Several countermeasures have been proposed in the literature to combat Byzantine model poisoning attacks on FL systems.
% Descriptive statistics
% For example, Trimmed Mean and FedMedian aggregate local model updates using more robust statistics than standard average~\cite{yin2018icml}.
%
% % Heuristics for outlier detection
% Many existing Byzantine-resilient strategies implement some outlier detection heuristics to discard the model updates sent by potentially malicious clients from the input of the aggregation function.
% One of the most popular heuristics is Krum~\cite{blanchard2017nips}.
% This strategy tries to mitigate the impact of Byzantine attacks by selecting as a global model the local model with the smallest sum of Euclidean distances to {\em all} the other local models.
% Although powerful, Krum requires the server to know (or, at least, estimate) the number of malicious FL clients upfront, which is generally impossible in a realistic attack scenario. %
% Moreover, Krum may become ineffective for complex, high-dimensional model parameter spaces due to the curse of dimensionality.
% Bulyan~\cite{mhamdi2018pmlr} tries to overcome this issue by combining Krum with a variant of Trimmed Mean.
% % Data-driven outlier detection
% Other strategies use data-driven outlier detection techniques -- e.g., via K-means clustering~\cite{shen2016acm} -- to spot potential malicious local model updates. 
% %For instance, Shen et al. propose to cluster local model updates with K-means and thus identify outliers.
%
% % Other techniques
% As far as the server is concerned, any local model received can be from a potential malicious client. 
% FLTrust~\cite{cao2020fltrust} assumes the server acts as a client, i.e., trains a local model on an additional {\em trustworthy} dataset at the server's end and compares it against all the local models from other clients. 
% This way, the server can rely on some ``source of trust'' when discarding potentially malicious clients.
%\\
% Limitations of existing Byzantine-resilient strategies
Unfortunately, existing defense mechanisms either rely on simple heuristics (e.g., Trimmed Mean and FedMedian by~\cite{yin2018icml}) or need strong and unrealistic assumptions to work effectively (e.g., foreknowledge or estimation of the number of malicious clients in the FL system, as for Krum/Multi-Krum~\cite{blanchard2017nips} and Bulyan~\cite{mhamdi2018pmlr}, which, however, cannot exceed a fixed threshold).
Furthermore, outlier detection methods using K-means clustering~\cite{shen2016acm} or spectral analysis like DnC~\cite{shejwalkar2021ndss} do not directly consider the temporal evolution of local model updates received.
Finally, strategies like FLTrust~\cite{cao2020fltrust} require the server to collect its own dataset and act as a proper client, thereby altering the standard FL protocol.
\\
% OLD, LONG VERSION
% Overall, existing Byzantine-resilient strategies are either simple heuristics (e.g., FedMedian) or, if they are more complex, they rely on strong and unrealistic assumptions to work effectively (e.g., knowing the number of malicious clients in the FL system in advance, as for Krum and alike).
% Furthermore, data-driven outlier detection methods do not consider the temporary evolution of local model updates received (e.g., K-means clustering). 
% Finally, strategies like FLTrust requires the server to collect its own dataset and act as a proper client, thereby altering the standard FL protocol.
%
% Description of the proposed method
This work introduces a novel pre-aggregation \textit{filter} robust to untargeted model poisoning attacks. Notably, this filter $(i)$ operates without requiring prior knowledge or constraints on the number of malicious clients and $(ii)$ inherently integrates temporal dependencies. 
The FL server can employ this filter as a preprocessing step before applying \textit{any} aggregation function, be it standard like FedAvg or robust like Krum or Bulyan.
Specifically, we formulate the problem of identifying corrupted updates as a multidimensional (i.e., matrix-valued) time series anomaly detection task. 
The key idea is that legitimate local updates, resulting from well-calibrated iterative procedures like stochastic gradient descent (SGD) with an appropriate learning rate, show \textit{higher predictability} compared to malicious updates. This hypothesis stems from the fact that the sequence of gradients (thus, model parameters) observed during legitimate training exhibit regular patterns, as validated in Section~\ref{subsec:intuition}. %until convergence. 
%This regularity may be more pronounced for smooth convex loss functions, but it can still be captured within an appropriate time window, even for more complex and convoluted loss surfaces. 
%We provide evidence of this claim in Appendix~B, where we show that the average mutual information (i.e., ``predictability''), calculated over pairs of legitimate model updates sent at different FL rounds, is significantly higher than the corresponding computation for a malicious client.
\\
Inspired by the matrix autoregressive (MAR) framework for multidimensional time series forecasting~\cite{chen2021je}, we propose the FLANDERS ({\em \textbf{F}ederated \textbf{L}earning meets \textbf{AN}omaly \textbf{DE}tection for a \textbf{R}obust and \textbf{S}ecure}) filter.
The main advantages of FLANDERS over existing strategies like FLDetector~\cite{zhao2020multivariate} are its resilience to large-scale attacks, where $50\%$ or more FL participants are hostile, and the capability of working under realistic non-iid scenarios.
We attribute such a capability to two key factors: $(i)$ FLANDERS works without knowing a priori the ratio of corrupted clients, and $(ii)$ it embodies temporal dependencies between intra- and inter-client updates, quickly recognizing local model drifts caused by evil players. Below, we summarize our main contributions:

\begin{itemize}
\item[{\em(i)}]
We provide empirical evidence that the sequence of models sent by legitimate clients is more predictable than those of malicious participants performing untargeted model poisoning attacks.
\\
\item[{\em(ii)}] 
We introduce FLANDERS, the first pre-aggregation filter for FL robust to untargeted model poisoning based on multidimensional time series anomaly detection.
\\
\item[{\em(iii)}] 
We integrate FLANDERS into Flower,\footnote{\scriptsize{\url{https://flower.dev/}}} a popular FL simulation framework for reproducibility.
\\
\item[{\em(iv)}] 
We show that FLANDERS improves the robustness of the existing aggregation methods under multiple settings: different datasets, client's data distribution (non-iid), models, and attack scenarios.
\\
\item[{\em(v)}] 
We publicly release all the implementation code of FLANDERS along with our experiments.\footnote{\scriptsize{\url{https://anonymous.4open.science/r/flanders_exp-7EEB}}}
\end{itemize}

% Paper's structure and organization
The remainder of the paper is structured as follows. %some related work and the current state-of-the-art solutions to security issues that FL entails. 
Section~\ref{sec:background} covers background and preliminaries. 
In Section~\ref{sec:related}, we discuss related work.
Section~\ref{sec:problem} and Section~\ref{sec:method} describe the problem formulation and the method proposed. % to tackle it. 
Section~\ref{sec:experiments} gathers experimental results. %, and Section~\ref{sec:limitations} discusses some limitations of this work.
Finally, we conclude in Section~\ref{sec:conclusion}.
 %discusses the limitations of this work and draws future research directions.
%reports conclusions and draws perspectives for future research directions.

%%%%%%% OLD %%%%%%%
%to overcome the resilience of Byzantine failures in distributed Stochastic Gradient Descent computations. 
% The strength of Krum is its time complexity, which is linear in the gradient dimension. 
% However, the robustness of the approach is guaranteed for gradient-based learning applications only when the majority of the clients are not compromised. 
% Besides, the aggregation mechanism of Krum, as well as that of similar methods, is robust from a coarse-grained perspective and does not provide solutions to errors and perturbations that may occur at inference time.
%A related approach to~\cite{blanchard2017nips} is the work of Su et al.~\cite{su2016dc}. Here, the authors propose an iterated approximate agreement to tackle a multi-layer scenario attacked by Byzantine agents. 
%However, the method works efficiently on the sole discrete context and it is inapplicable to continuous state environments.
%\gabri{Maybe, we should just talk about the main limitations of existing countermeasures without digging into their details (or, we can just mention Krum as this is the most popular one). I will move the description of all these methods to the Related Work section.}
\newpage

\section{Homotopy theory background}
\subsection{Review of generalized equivariant cohomology}\label{review-equiv}

We review the construction of generalized equivariant cohomology via spectral algebraic geometry from \cite{survey}, in a form suitable for our applications. This review will necessarily be brief, since a detailed exposition may be found in \textit{loc. cit.}; there is also some discussion in the early sections of \cite{ginzburg-kapranov-vasserot} in the setting of ordinary (as opposed to spectral) algebraic geometry.
\begin{setup}
Fix an $\Eoo$-ring $A$ and a commutative $A$-group $\GG$, so $\GG$ defines a functor $\CAlg_A \to \Mod_{\Z,\geq 0}$ which is representable by a \textit{flat} $A$-algebra. We will write $\GG_0$ to denote the resulting commutative group scheme over $\pi_0 A$.
\end{setup}
\begin{remark}
The equivalence $\Omega^\infty: \Sp_{\geq 0} \xar{\sim} \CAlg(\Top_\ast)$ extends to an equivalence between $\Mod_{\Z,\geq 0}$ and topological abelian groups. More precisely, by the Dold-Kan correspondence and the Schwede-Shipley theorem, there are equivalences of categories
$$\Mod^{\geq 0}_\Z \simeq \mathrm{Ch}_{\geq 0}(\Z) \simeq \Fun(\Deltab^{op},\Ab) = s\Ab.$$
The image of $\Mod^{\geq 0}_\Z$ under the equivalence $\Omega^\infty: \Sp_{\geq 0} \xar{\sim} \CAlg(\Top_\ast)$ can be characterized as follows.
Let us model grouplike infinite loop spaces $X$ as functors $X:\Fin_\ast\to \Top$ such that $\pi_0 \Map_\Top(Y,X)$ is an abelian group for all spaces $Y$ (i.e., $X$ is grouplike) and such that the map $X([n])\to X([1])^n$ is an equivalence. Such an object should be in the image of $\Mod^{\geq 0}_\Z$ iff it is ``strictly commutative''. One way to make this precise is as follows. Let $\mathrm{Lattice}$ denote the full subcategory of the category of abelian groups spanned by the groups $\Z^n$ with $n\geq 0$, so there is a functor $\Fin_\ast\to \mathrm{Lattice}$. Then an infinite loop space is in the image of $\Mod^{\geq 0}_\Z$ if and only if the functor $\Fin_\ast\to \Top$ classifying it factors through a finite-product-preserving functor $\mathrm{Lattice} \to \Top$. In other words, $\Mod^{\geq 0}_\Z$ is equivalent to the full subcategory spanned by the grouplike objects in the category $\Fun^\pi(\mathrm{Lattice}, \Top)$. This is a very strong condition to impose on an infinite loop space: it forces the infinite loop space to decompose as a product of Eilenberg-Maclane spaces. For example, $\CP^\infty$ admits such a factorization, but $\BU$ (with either the additive or multiplicative infinite loop space structure) does not.
\end{remark}
\begin{definition}
A \textit{preorientation of $\GG$} is a pointed map $S^2 \to \Omega^\infty \GG(A)$ of spaces, i.e., a map $\Sigma^2 \Z \to \GG(A)$ of $\Z$-modules (by adjunction). This induces a map $\CP^\infty = \Omega^\infty \Sigma^2 \Z \to \Omega^\infty \GG(A)$ of topological abelian groups, and hence a map $\spf A^{\CP^\infty} \to \GG$ of $\Eoo$-$A$-group schemes. (Note that $\spf A^{\CP^\infty}$ need not admit the structure of a commutative $A$-group scheme: for instance, $A^{\CP^\infty}$ need not be flat over $A$.)
\end{definition}
\begin{definition}
Given a preorientation $S^2 \to \Omega^\infty \GG(A)$, we obtain a map $\co_\GG \to C^\ast(S^2; A)$ of $\Eoo$-$A$-algebras. On $\pi_0$, this induces a map $\pi_0 \co_\GG = \co_{\GG_0} \to \pi_0 C^\ast(S^2; A)$. However, the target can be identified with the trivial square-zero extension $\pi_0 A \oplus \pi_{-2} A$, so that the preorientation defines a derivation $\co_{\GG_0} \to \pi_{-2} A$. This defines a map $\beta: \omega = \Omega^1_{\GG_0/\pi_0 A} \to \pi_{-2} A$. The preorientation is called an \textit{orientation} if $\GG_0$ is smooth of relative dimension $1$ over $\pi_0 A$, and the composite
$$\pi_n(A) \otimes_{\pi_0 A} \omega \to \pi_n(A) \otimes_{\pi_0 A} \pi_{-2} A \xar{\beta} \pi_{n-2} A$$
is an isomorphism for each $n\in \Z$. This forces $A$ to be $2$-periodic (but does not force its homotopy to be concentrated in even degrees).
\end{definition}
\begin{warning}\label{additive-orientation}
As discussed in \cite[Section 3.2]{survey}, the universal $\Eoo$-$\Z$-algebra over which the additive group scheme $\GG_a$ admits an orientation is given by $\Z[\CP^\infty][\tfrac{1}{\beta}] = \QQ[\beta^{\pm 1}]$. Therefore, we are allowed to let $\GG = \GG_a$ in the story below only when $A$ is a $2$-periodic \textit{$\Eoo$-$\QQ$-algebra}. (If $A$ is not an $\Eoo$-$\Z$-algebra, one cannot in general define $\GG_a = \spec A[t]$ as a commutative $A$-group: the coproduct $A[t] \to A[x,y]$ will in general not be a map of $\Eoo$-$A$-algebras.)
\end{warning}
We can now review the definition of $T$-equivariant $A$-cohomology when $T$ is a torus.
\begin{construction}\label{def-equiv-coh}
Fix an $\Eoo$-ring $A$ as above and a commutative $A$-group $\GG$. Given a compact abelian Lie group $T$, define an $A$-scheme $\cM_T$ by the mapping stack $\Hom(\bX^\ast, \GG)$. We will be particularly interested in the case when $T$ is a torus. Let $\cT$ be the full subcategory of $\Top$ spanned by those spaces which are homotopy equivalent to $BT$ with $T$ being a compact abelian Lie group. By arguing as in \cite[Theorem 3.5.5]{elliptic-iii}, a preorientation of $\GG$ is equivalent to the data of a functor $\cM: \cT \to \Aff_A$ along with compatible equivalences $\cM(BT) \simeq \cM_T$. The $\Eoo$-$A$-algebra $\co_{\cM_T}$ is the $T$-equivariant $A$-cochains of a point, and will occasionally be denoted by $A_T$.

We can now sketch the construction of the $T$-equivariant $A$-cochains of more general $T$-spaces; see \cite[Theorem 3.2]{survey}. Let $T$ be a torus over $\cc$ for the remainder of this discussion, and let $\GG$ be an \textit{oriented} commutative $A$-group. Let $\Top(T)$ denote the $\infty$-category of finite $T$-spaces, i.e., the smallest subcategory of $\Fun(BT, \Top)$ which contains the quotients $T/T'$ for closed subgroups $T'\subseteq T$, and which is closed under finite colimits. There is a functor $\cf_T: \Top(T)^\op \to \QCoh(\cM_T)$ which is uniquely characterized by the requirement that it preserve finite limits and sends $T/T' \mapsto q_\ast \co_{\cM_{T'}}$. Here, $q: \cM_{T'} \to \cM_T$ is the canonical map induced by the inclusion $T'\subseteq T$. If $X\in \Top(T)$, then the \textit{$T$-equivariant $A$-cochains of $X$} is the global sections $\Gamma(\cM_T; \cf_T(X))$; we will denote it by $C^\ast_T(X; A)$.
\end{construction}
\begin{remark}
We will denote the functor $\Gamma(\cM_T; \cf_T(-)): \Top(T)^\op \to \Mod(\Gamma(\cM_T; \co_{\cM_T}))$ by $C^\ast_T(-;A): \Top(T)^\op \to \Mod(A_T)$.
\end{remark}
\begin{definition}\label{equiv-homology}
%The $\Eoo$-$A$-algebra $A_T = \co_{\cM_T}$ admits an $\Eoo$-coalgebra structure (owing to it being the $\Eoo$-ring of functions on the group scheme $\GG$). Its $A$-linear dual $A^T := A_T^\vee$ therefore admits the structure of an $\Eoo$-$A$-algebra (as well as the structure of an $\Eoo$-$A$-coalgebra). If $X\in \Top(T)$, then the \textit{$T$-equivariant $A$-chains of $X$} is the $A^T$-comodule given by $\cf_T(X)^\vee$; we also denote it by $C_\ast^T(X; A)$. Therefore, $A^T$ is the $T$-equivariant $A$-chains of a point. We will write $\cM_T^\vee$ to denote $\spec A^T$.
If $X\in \Top(T)$, then the \textit{$T$-equivariant $A$-chains of $X$} is the quasicoherent sheaf on $\cM_T$ given by the $\co_{\cM_T}$-linear dual $\cf_T(X)^\vee$. We will denote its global sections by $C_\ast^T(X; A)$. Note that $C_\ast^T(\ast;A) \simeq A_T$, which completes to the $A$-cochains (\textit{not} $A$-chains) of $BT$.
\end{definition}
\begin{warning}
Let $A$ be an $\Eoo$-$\Z$-algebra, and let $\GG = \GG_a$; then \cref{additive-orientation} says that $A$ must be an $\Eoo$-$\QQ[\beta^{\pm 1}]$-algebra. Suppose for simplicity that $T = \GG_m$; then $\pi_\ast C_\ast(BT; A)$ may therefore be identified with the divided power algebra $\Gamma_{\pi_\ast(A)}(\hbar^\vee)$ with $|\hbar^\vee|=2$. Since $A$ is rational, this may further be identified with the polynomial ring $\pi_\ast(A)[\hbar^\vee]$. Unfortunately, this can be confused with $\pi_{\ast}(A_T)$, albeit with the reversed grading. Although this identification is technically correct, it is rather abusive: there is no canonical way to identify $A_T$ with $C_\ast(BT; A)$ when $A$ is an $\Eoo$-$\QQ[\beta^{\pm 1}]$-algebra. We will therefore refrain from making this identification, since it is not valid for more general $\Eoo$-rings $A$.
\end{warning}
\begin{notation}\label{ideal-character}
Let $\lambda: T \to \GG_m$ be a character, and let $T_\lambda = \ker(\lambda)$. Then the map $q: \cM_{T_\lambda} \to \cM_T$ is a closed immersion, and we will denote the ideal in $\co_{\cM_T}$ defined by this closed immersion by $\cI_\lambda$. Equivalently, let $V_\lambda$ denote the $T$-representation obtained by the projection $T \to T_\lambda$. Then $\cI_\lambda$ is given by the line bundle $\cf_T(S^{V_\lambda})$.
\end{notation}
It is trickier to extend the definition of equivariant cochains to nonabelian groups, but a construction is sketched in \cite[Section 3.5]{survey}, and a detailed construction is given in \cite{gepner-meier}. We recall this for completeness; in this article, we will only be concerned with torus-equivariance. The methods of this article should work for more general compact Lie groups, but we have not studied this here.
\begin{construction}\label{nonabelian-equiv-cochains}
Let $G$ be a reductive group scheme over $\cc$. Let $\Top(G)$ denote the smallest subcategory of $\Fun(BG, \Top)$ which contains the quotients $G/T'$ for closed \textit{commutative} subgroups $T'\subseteq G$, and which is closed under finite colimits. Then there is a functor $C^\ast_G(-;A): \Top(G)^\op \to \Mod(A)$ which is uniquely characterized by the requirement that it preserve finite limits and sends $G/T' \mapsto A_{T'}$. According to \cite[End of Section 3.5]{survey} and \cite[Section 3]{gepner-meier}, when $G$ is connected, there is a flat $A$-scheme $\cM_G$ and a functor $\cf_G: \Top(G)^\op \to \QCoh(\cM_G)$, such that composition with the forgetful functor $\QCoh(\cM_G) \to \Mod(A)$ is the functor $C^\ast_G(-;A)$. If $X\in \Top(G)$, we will write $\cf_G(X)^\vee$ to denote the linear dual of $\cf_G(X)$ in $\QCoh(\cM_G)$, and refer to it as the \textit{$G$-equivariant $A$-chains} on $X$.
\end{construction}
\begin{remark}\label{ind-finite}
Let $X$ be an ind-finite space with a $G$-action, so that $X$ can be written as the filtered colimit of a diagram $\{X_i\}$ of subspaces, each of which are in $\Top(G)$. Write $C^\ast_G(X;A)$ to denote $\varprojlim_i C^\ast_G(X_i;A)$. Similarly for $\cf_G(X)$.
\end{remark}
\begin{example}
Let $G$ be a connected compact Lie group, and let $T$ be a maximal torus in $G$. The flag variety $G/T$ is a $G$-space whose stabilizers are commutative, and therefore $G/T\in \Top(G)$. Therefore, $C^\ast_G(G/T; A) = A_T$. For the remainder of this text, we will make the following \textit{assumption}: after inverting $|W|$, there is a (homotopy-coherent) $W$-action on $A_T$ by maps of $\Eoo$-$A$-algebras, and $A_G := C^\ast_G(\ast;A)$ is equivalent to $A_T^{hW}$ as an $\Eoo$-$A$-algebra. %We will then also write $\cM_G = \spec A_G$, even though this might not agree with the putative definition of $\cM_G$ from \cite{survey}.
\end{example}
\subsection{Categories of equivariant local systems}\label{categories-of-equiv-loc}

Fix a complex-oriented even-periodic $\Eoo$-ring $A$ and an oriented $A$-group scheme $\GG$. Let $T$ be a compact torus. Let $X\in \Top(T)$ be a finite $T$-space. The following categorifies the $T$-equivariant $A$-cochains $C_T^\ast(X; A)$.
\begin{construction}\label{def-loc}
Let $\Loc_T(\ast; A)$ denote the $\infty$-category $\QCoh(\cM_T)$. Let $T'\subseteq T$ be a closed subgroup, so that there is an associated morphism $q: \cM_{T'} \to \cM_T$. This defines a symmetric monoidal functor $\QCoh(\cM_T) \to \QCoh(\cM_{T'})$, which equips $\QCoh(\cM_{T'})$ with the structure of a $\QCoh(\cM_T)$-module.
Let $\cLoc_T(-; A): \Top(T)^\op \to \CAlg(\shvcat(\cM_T))$ be the functor uniquely characterized by the requirement that it preserve finite limits and send $T/T' \mapsto \QCoh(\cM_{T'})$. If $X\in \Top(T)$, then the $\infty$-category $\Loc_T(X; A)$ of \emph{$T$-equivariant local systems of $A$-modules on $X$} is defined to be the global sections of the quasicoherent stack $\cLoc_T(X; A)$ on $\cM_T$. If $f: X \to Y$ is a map in $\Top(T)$, the associated symmetric monoidal functor $f^\ast: \Loc_T(Y; A) \to \Loc_T(X; A)$ (induced by taking global sections of the morphism $f^\ast: \cLoc_T(Y; A) \to \cLoc_T(X; A)$ of $\Eoo$-algebras in quasicoherent stacks over $\cM_T$) will be called the \textit{pullback}. One can show that $\Loc_T(X; A)$ is a presentable stable $\infty$-category, and that $f^\ast$ preserves small colimits (so it has a right adjoint $f_\ast$, which will be called \textit{pushforward}).
\end{construction}
\begin{example}
If $T = \{1\}$, then $\Loc_T(X; A)$ is equivalent to the $\infty$-category $\Loc(X; A) := \Fun(X, \Mod_A)$ of local systems on $X$.
\end{example}
\begin{remark}
Let $X$ be a finite $T$-space. The \textit{constant local system} $\ul{A}_T$ is defined to be the image of $\co_{\cM_T}$ under the symmetric monoidal functor $\Loc_T(\ast; A) \simeq \QCoh(\cM_T) \to \Loc_T(X; A)$ induced by pullback along $f: X \to \ast$. Observe that if $\ul{A}_T$ denotes the constant local system, then $\End_{\Loc_T(X; A)}(\ul{A}_T) \simeq C_T^\ast(X; A)$. Indeed, $\End_{\Loc_T(X; A)}(\ul{A}_T) \simeq \Gamma(\cM_T; f_\ast f^\ast \co_{\cM_T})$, but it is easy to see that $f_\ast f^\ast \co_{\cM_T} = \cf_T(X)\in \QCoh(\cM_T)$. The desired claim then follows from \cref{def-equiv-coh}.
\end{remark}
\begin{remark}
If $T$ were a \textit{finite} diagonalizable group scheme (such as $\mu_n$), the desired category $\Loc_T(X; A)$ is closely related to the $\infty$-category of \textit{$\GG$-tempered local systems} on the orbispace $X\mmod T$, as described in \cite{elliptic-iii}. Our understanding is that Lurie is planning to describe an extension of the work in \cite{elliptic-iii} and its connections to equivariant homotopy theory in a future article. We warn the reader that \cref{def-loc} is somewhat \textit{ad hoc}; so the resulting category of equivariant local systems may or may not agree with the output of forthcoming work of Lurie.
\end{remark}
\begin{remark}
Let $X$ be a $T$-space with a chosen presentation as a filtered colimit of finite $T$-spaces $X_\alpha$. Then we will write $\Loc_T(X; A)$ to denote $\lim \Loc_T(X_\alpha; A)$.
\end{remark}
If $Y$ is a \textit{connected} space, the $\infty$-category $\Loc(Y; A) = \Fun(Y, \Mod_A)$ of local systems on $Y$ is equivalent by Koszul duality to $\LMod_{C_\ast(\Omega Y; A)}$. This property of local systems is very useful, since it allows one to study of local systems using (derived) algebra. A similar property is true for $\Loc_T(X; A)$:
\begin{prop}\label{equivariant-koszul}
Let $X$ be a connected finite $T$-space. Then there is an equivalence $\Loc_T(X; A) \simeq \LMod_{\cf_T(\Omega X)^\vee}(\QCoh(\cM_T))$.
\end{prop}
\begin{proof}
Let $s: \ast \to X$ denote the inclusion of a point. We claim that $s^\ast: \Loc_T(X; A) \to \QCoh(\cM_T)$ admits a left adjoint $s_!$. Indeed, the statement for general $X$ follows formally from the case of $X = T/T'$ for some closed subgroup $T'\subseteq T$ (so $s$ is the inclusion of the trivial coset). In this case, $s^\ast$ is the functor $\QCoh(\cM_{T'}) \to \QCoh(\cM_T)$ given by pushforward along the associated morphism $q: \cM_{T'} \to \cM_T$, so it has a left adjoint $s_!$ given by $q^\ast$. Note that $s^\ast$ also has a right adjoint; in particular, it preserves small limits and colimits. Observe now that $s_! \co_{\cM_T}$ is a compact generator of $\Loc_T(X; A)$: indeed, suppose $\cf\in \Loc_T(X; A)$ such that $\Map_{\Loc_T(X; A)}(s_! \co_{\cM_T}, \cf) \simeq 0$ as an object of $\QCoh(\cM_T)$. Because $s^\ast \cf \simeq \Map_{\Loc_T(X; A)}(s_! \co_{\cM_T}, \cf)$ in $\QCoh(\cM_T)$, we see that $s^\ast \cf \simeq 0$. Using the connectivity of $X$, we see that $\cf$ itself must be zero, which implies that $s_! \co_{\cM_T}$ is a compact generator of $\Loc_T(X; A)$. It follows from the Barr-Beck-Lurie theorem \cite[Theorem 4.7.3.5]{HA} that $\Loc_T(X; A)$ is equivalent to the $\infty$-category of left $\End_{\Loc_T(X;A)}(s_! \co_{\cM_T})$-modules in $\QCoh(\cM_T)$. But $\End_{\Loc_T(X;A)}(s_! \co_{\cM_T}) \simeq s^\ast s_! \co_{\cM_T}$, which identifies with $\cf_T(\Omega X)^\vee$.
%\todo hmm
\end{proof}
\begin{remark}\label{loc and comod}
Modifying the preceding argument shows that if $X$ is a connected finite $T$-space, there is an equivalence $\Loc_T(X; A) \simeq \coLMod_{\cf_T(X)^\vee}(\QCoh(\cM_T))$. In particular, if $X$ admits an $\E{n}$-algebra structure (compatible with the $T$-action), then $\cf_T(X)^\vee$ admits the structure of an $\E{n}$-algebra\footnote{If $\cC$ is a symmetric monoidal $\infty$-category, \cite[Corollary 3.3.4]{HA} can be used to show that there is an equivalence $\coCAlg(\Alg_\E{n}(\cC)) \simeq \Alg_\E{n}(\coCAlg(\cC))$.} in $\coCAlg(\QCoh(\cM_T))$, and the equivalence $\Loc_T(X; A) \simeq \coLMod_{\cf_T(X)^\vee}(\QCoh(\cM_T))$ is $\E{n}$-monoidal for the convolution tensor product on both sides. More generally, if $X$ is a $T$-space with a chosen presentation as a filtered colimit of finite $T$-spaces $X_\alpha$, there is an equivalence $\Loc_T(X; A) \simeq \coLMod_{\cf_T(X)^\vee}(\QCoh(\cM_T))$.
\end{remark}
\subsection{Filtered deformations}

As usual, we will fix a complex-oriented even-periodic $\Eoo$-ring $A$ and an oriented $A$-group scheme $\GG$ throughout this section. The main idea of this section (using the double-speed Postnikov filtration) has been used to great effect in \cite{even-filtr, piotr-synthetic, raksit}, but the focus of this section is rather different from \textit{loc. cit.}.

Write $\Sp^\fil$ to denote the $\infty$-category $\Fun(\Z, \Sp)$ of filtered spectra, where $\Z$ is viewed as a poset via the standard ordering. Similarly, write $\Sp^\gr$ to denote the $\infty$-category $\Fun(\Z^\ds, \Sp)$ of graded spectra, where $\Z^\ds$ denotes the discrete set of integers. There is a functor $\gr: \Sp^\fil \to \Sp^\gr$ given by taking associated graded. See \cite{rotinv, raksit} for further discussion on filtered and graded spectra.
Recall the following equivalence from \cite{filtered-A1-Gm}, which let us view a filtration as equivalent to a one-parameter deformation.
\begin{prop}[Rees construction]\label{A1 Gm}
There is a symmetric monoidal equivalence $\Sp^\fil \simeq \QCoh(\AA^1/\GG_m)$, where $\AA^1/\GG_m$ is the flat spectral stack over the sphere spectrum. Under this equivalence, the functor $\gr: \Sp^\fil \to \Sp^\gr$ is given by pullback along the closed immersion $B\GG_m \hookrightarrow \AA^1/\GG_m$.
In particular, a $\Z$-filtered $\E{n}$-algebra in $\Sp$ defines an $\E{n}$-algebra in $\QCoh(\AA^1/\GG_m)$.
\end{prop}
\begin{notation}
If $R\in \CAlg(\Sp^\fil)$, we will simply write $\Mod_R^\fil$ to denote $\Mod_R(\Sp^\fil)$. Similarly, if $R\in \CAlg(\Sp^\gr)$, we will simply write $\Mod_R^\gr$ to denote $\Mod_R(\Sp^\gr)$. 
If $\cC$ is a $\Sp^\fil$-linear $\infty$-category, write $\cC^\gr$ to denote $\cC \otimes_{\Sp^\fil} \Sp^\gr$.
For $R\in \CAlg(\Sp^\fil)$, the $\infty$-category $\Mod_R^\fil$ is canonically a $\Sp^\fil$-linear $\infty$-category, and there is an equivalence 
$$(\Mod_R^\fil)^\gr = \Mod_R^\fil \otimes_{\Sp^\fil} \Sp^\gr \simeq \Mod_{\gr(R)}^\gr.$$
\end{notation}
\begin{construction}\label{synthetic lifts}
The $\Eoo$-ring $A$ defines a canonical $\Z$-filtered $\Eoo$-algebra in $\Sp$, given by $\tau_{\geq 2\star} A$. Note that since $\tau_{\geq 2\star}: \Sp \to \Fun(\Z, \Sp)$ is a lax symmetric monoidal functor, $\tau_{\geq 2\star} A$ is an $\Eoo$-algebra in filtered spectra.
The discussion in the preceding section in turn admits a canonical one-parameter deformation.
Namely, the spectral $A$-scheme $\cM_T$ admits a filtered deformation $\cM_T^\fil$: its underlying $\pi_0 A$-scheme is just the underlying scheme of $\cM_T$, and its ring of functions is given by the sheaf $\tau_{\geq 2\star} \co_{\cM_T}$ of filtered $\tau_{\geq 2\star} A$-algebras. Motivated by the comparison to synthetic spectra in \cite{even-filtr}, we will write $\QCoh^\syn(\cM_T)$ to denote the $\Mod_{\tau_{\geq 2\star} A}^\fil$-linear $\infty$-category $\QCoh(\cM_T^\fil)$.

Similarly, if $X$ is a $T$-space, one can also consider filtered deformations of the sheaves $\cf_T(X)$ and $\cf_T(X)^\vee$. For simplicity, we will only consider the case when $\cf_T(X)$ (resp. $\cf_T(X)^\vee$) has homotopy sheaves concentrated in even degrees; in this case, the filtered deformation of $\cf_T(X)$ (resp. $\cf_T(X)^\vee$) is simply given by $\tau_{\geq 2\star} \cf_T(X)$ (resp. $\tau_{\geq 2\star} \cf_T(X)^\vee$). These are quasicoherent sheaves on $\cM_T^\fil$; since $\tau_{\geq 2\star}$ is lax symmetric monoidal, $\tau_{\geq 2\star} \cf_T(X)$ is an $\Eoo$-algebra in $\QCoh^\syn(\cM_T)$. Similarly, if $X$ is an $\E{n}$-space (compatible with the $T$-action), then $\tau_{\geq 2\star} \cf_T(X)^\vee$ is an $\E{n}$-algebra in $\QCoh^\syn(\cM_T)$.

Let $X$ be a connected finite $T$-space such that $\cf_T(\Omega X)^\vee$ is concentrated in even degrees. 
Motivated by \cref{equivariant-koszul}, define $\Loc_T^\syn(X; A)$ to denote $\LMod_{\tau_{\geq 2\star} \cf_T(\Omega X)^\vee}(\QCoh^\syn(\cM_T))$.
Similarly, if $Y$ is an $\E{n}$-algebra in connected $T$-spaces such that $\cf_T(Y)^\vee$ is concentrated in even degrees, define $\Loc_T^\syn(Y; A)$ to be $\coLMod_{\tau_{\geq 2\star} \cf_T(Y)^\vee}(\QCoh^\syn(\cM_T))$.
\end{construction}
\begin{remark}
In \cref{synthetic lifts}, the definition of $\Loc_T^\syn(X; A)$ is rather \textit{ad hoc}; we have not attempted to describe a general construction here, because this definition suffices for our purposes.
\end{remark}
The key point of the preceding construction is that it allows us to interpolate between spectral and (derived) algebraic geometry. More precisely:
\begin{lemma}\label{gr of tau 2star A}
There is an equivalence $(\Mod_{\tau_{\geq 2\star} A}^\fil)^\gr \simeq \Mod_{\pi_0 A}$.
\end{lemma}
\begin{proof}
Base-changing the $\Sp^\fil$-linear $\infty$-category $\Mod_{\tau_{\geq 2\star} A}^\fil$ along $\gr: \Sp^\fil \to \Sp^\gr$ produces the $\Sp^\gr$-linear $\infty$-category $\Mod_{\pi_{2\star} A}^\gr$, where $\pi_{2\star} A$ is viewed as a graded $\Eoo$-ring. However, $A$ is even-periodic, so $\pi_{2\star} A \cong \pi_0(A)[\beta^{\pm 1}]$ with $\beta$ in weight $1$. This implies that $\Mod_{\pi_{2\star} A}^\gr \simeq \Mod_{\pi_0 A}$.
\end{proof}
Let $\cM_{T,0}$ denote the underlying $\pi_0 A$-scheme of the $A$-scheme $\cM_T$.
\cref{gr of tau 2star A} identifies $\QCoh^\syn(\cM_T)^\gr = \QCoh^\syn(\cM_T) \otimes_{\Sp^\fil} \Sp^\gr$ with $\QCoh(\cM_{T,0})$ as $\pi_0 A$-linear $\infty$-categories. 
\begin{notation}\label{graded local systems}
Let $X$ be a connected finite $T$-space such that $\cf_T(\Omega X)^\vee$ is concentrated in even degrees. The preceding discussion implies that $\pi_{2\star} \cf_T(\Omega X)^\vee$ defines an $\E{1}$-algebra in $\coCAlg(\QCoh^\syn(\cM_T)^\gr)$.
Let $\Loc_T^\gr(X; A)$ denote $\LMod_{\pi_{2\star} \cf_T(\Omega X)^\vee}(\QCoh^\syn(\cM_T)^\gr)$; note that the $\Eoo$-coalgebra structure on $\cf_T(\Omega X)^\vee$ equips $\Loc_T^\gr(X; A)$ with the structure of a symmetric monoidal $\infty$-category. By $2$-periodicity, we can identify 
$$\Loc_T^\gr(X; A) \simeq \LMod_{\pi_0 \cf_T(\Omega X)^\vee}(\QCoh(\cM_{T,0})).$$
Similarly, if $Y$ is an $\E{n}$-algebra in connected $T$-spaces such that $\cf_T(Y)^\vee$ is concentrated in even degrees, $\pi_{2\star} \cf_T(Y)^\vee$ defines an $\Eoo$-coalgebra in $\Alg_\E{n}(\QCoh^\syn(\cM_T)^\gr)$.
Let $\Loc_T^\gr(Y; A)$ denote $\coLMod_{\pi_{2\star} \cf_T(Y)^\vee}(\QCoh^\syn(\cM_T)^\gr)$; note that the $\E{n}$-algebra structure on $\cf_T(Y)^\vee$ equips $\Loc_T^\gr(Y; A)$ with the structure of an $\E{n}$-monoidal $\infty$-category. By $2$-periodicity, we can identify 
$$\Loc_T^\gr(Y; A) \simeq \coLMod_{\pi_0 \cf_T(Y)^\vee}(\QCoh(\cM_{T,0})).$$
Both $\Loc_T^\gr(X; A)$ and $\Loc_T^\gr(Y; A)$ are $\QCoh(\cM_{T,0})$-linear $\infty$-categories, which arise as $\Loc_T^\syn(X; A)^\gr$ and $\Loc_T^\syn(Y; A)^\gr$, respectively.
\end{notation}
\subsection{GKM and complex periodic $\Eoo$-rings}

We review the main result of \cite{generalized-gkm}, which proves a generalization of a result of Goresky-Kottwitz-MacPherson to generalized cohomology theories. This is also studied in the forthcoming work \cite[Section 3]{gepner-meier}.
\begin{setup}\label{gkm-assumption}
Let $A$ be a complex-oriented even-periodic $\Eoo$-ring, and let $\GG$ be an oriented commutative $A$-group. 
%Recall that if $Y$ is a space and $Y \to \BU \times \Z$ classifies a complex vector bundle $\xi$ over $Y$, then $\xi$ is said to be \textit{$A$-orientable} if the composite
%$$Y \xar{\xi} \BU\times \Z\xar{J} \Pic(\Sp) \to \Pic(\Mod_A)$$
%is nullhomotopic. Let $Y^\xi$ denote the Thom space of $Y$, and let $e(\xi)\in A^\ast(Y)$ denote the Euler class of $\xi$ (obtained by pulling back the Thom class along the zero section). 
Fix a compact torus $T$. We will consider (ind-finite; see \cref{ind-finite}) $T$-spaces $X$ such that the following assumptions hold.
\begin{enumerate}
    \item $X$ admits a $T$-invariant stratification $\bigcup_{w\in W} X_x$ with only \textit{even-dimensional} cells, with only finitely many in each dimension.
    \item The $T$-action on each cell $X_w = \AA^{\ell(w)}$ is via a linear action, whose weights are pairwise relatively prime.
    \item For each weight $\lambda$ of the $T$-action on $X_w = \AA^{\ell(w)}$, the closure of $\cc_\lambda\subseteq X_w$ is a sphere $S^\lambda$ such that $0$ and $\infty$ are fixed points of the $T$-action.
    %attaching map for a cell sends the boundary of each $D(\lambda)\subseteq D(\AA^{2\ell(w)})$ to a fixed point of one of the cells in the lower strata. 
\end{enumerate}
\end{setup}
\begin{definition}
The \textit{GKM graph} $\Gamma$ asssociated to an $X$ as in \cref{gkm-assumption} is defined as follows. The vertices are the (isolated) fixed points of the $T$-action, and there is an edge $x \to y$ labeled by a character $\lambda$ if $x = 0$ and $y=\infty$ in the closure $S^\lambda$ of $D(\lambda)\subseteq D(\AA^{2\ell(w)})$. Let $V$ denote the set of vertices of $\Gamma$, and $E$ the set of edges.
\end{definition}
\begin{theorem}[{\cite[Theorem 3.1]{generalized-gkm}, \cite[Section 3]{gepner-meier}}]\label{gkm-main}
In \cref{gkm-assumption}, the map $\cf_T(X) \to \Map(V, \co_{\cM_T}) \simeq \cf_T(X^T)$ induces an injection on homotopy sheaves, and the following diagram is an equalizer on $\pi_0$:
$$\cf_T(X) \to \Map(V, \co_{\cM_T}) \rightrightarrows \prod_{\alpha\in E} \co_{\cM_{T_\alpha}}.$$
Here, the two maps are induced by the inclusion of the source and target of $\alpha: x \to y$.
\end{theorem}
\begin{proof}[Proof sketch]
The argument is exactly as in \cite[Theorem 3.1]{generalized-gkm} (where the spaces denoted $F_i$ are points, corresponding to the origin in $\AA^{\ell(w)}$), so we only give a sketch. We will work locally on $\GG$. In this case, we need to show that the map $\cf_T(X) \to \Map(V, \co_{\cM_T}) \simeq \cf_T(X^T)$ is injective on homotopy sheaves, and the following diagram is an equalizer on $\pi_0$:
$$\cf_T(X) \to \cf_T(X^T) \rightrightarrows \prod_{\alpha\in E} \cf_{T_\alpha}.$$

For the injectivity claim, we first claim that $\cf_T(X)^{tT} \simeq \cf_T(X^T)^{tT}$. (This is a version of Atiyah-Bott localization.) Since $X$ is generated by finite colimits from $T$-orbits $T/T'$, it suffices to prove this claim when $X$ is of that form. Then $\cf_T(T/T') \simeq \cf_{T'}(\ast) = q_\ast \co_{\cM_{T'}}$; this has zero Tate construction if $T'\neq T$. On the other hand, $X^T = \emptyset$ if $T' \neq T$, so $\cf_T(X^T)^{tT} = 0$ as desired. If $T' = T$, then $X^T = \ast$, so that both sides are simply $A^{tT}$.

Note that $\cf_T(X^T)^{tT} \simeq \cf_T(X^T) \otimes_A A^{tT}$. Since $\cf_T(X)^{tT} \simeq \cf_T(X) \otimes_{\co_{\cM_T}} A^{tT}$ is a localization, it suffices to prove that the map $\cf_T(X) \to \cf_T(X)^{tT}$ induces an injection on homotopy. For this, it suffices to prove that $\cf_T(X)$ is a free $\co_{\cM_T}$-module. This is a consequence of the assumptions on $X$.
%There is a spectral sequence
%$$E_2^{\ast,\ast} = \H^\ast(BT; C^\ast(X; A)) \Rightarrow \pi_\ast \cf_T(X)$$
%of sheaves on $\cM_T$, which degenerates at the $E_2$-page by the assumptions on $X$. This implies the desired claim.

To prove the statement about the equalizer diagram, the key case is when $X = S^W$ for a $T$-representation $W$; the general case is obtained by induction on the stratification of $X$. Let $\lambda_1,\cdots,\lambda_n$ be the weights of $W$, so that $X = \bigotimes_{i=1}^n S^{\lambda_i}$. Therefore, $X$ is the quotient of $\prod_{i=1}^n S^{\lambda_i}$ by its $(2n-2)$-skeleton. Using this observation, it is not difficult to reduce to the case when $W = \lambda$ is a character of $T$.
In this case, $X = S^\lambda$ has $T$-fixed points given by $\{0,\infty\}$. There is a cofiber sequence $S(\lambda) \to \ast \to S^\lambda$, which induces a pushout square
$$\xymatrix{
S(\lambda)_+ \ar[r] \ar[d] & S^0 = \{\infty\}_+ \ar[d] \\
S^0 = \{0\}_+ \ar[r] & S^\lambda_+.
}$$
Therefore, we get an equalizer diagram
$$\cf_T(S^\lambda) \to \co_{\cM_T} \rightrightarrows \cf_T(S(\lambda)).$$
However, if $T_\lambda = \ker(\lambda: T \to \GG_m)$, then $S(\lambda) \simeq T/T_\lambda$, so that $\cf_T(S(\lambda)) \simeq q_\ast \co_{\cM_{T_\lambda}}$. It follows that $\cf_T(S^\lambda)$ is the fiber of the map $\co_{\cM_T} \oplus \co_{\cM_T} \to q_\ast \co_{\cM_{T_\lambda}}$ given by the following composite:
$$\co_{\cM_T} \oplus \co_{\cM_T} \xar{(x,y)\mapsto x-y} \co_{\cM_T} \to q_\ast \co_{\cM_{T_\lambda}}.$$
However, the map $\co_{\cM_T} \to q_\ast \co_{\cM_{T_\lambda}}$ is precisely given by quotienting by the ideal $\cI_\lambda$ (by \cref{ideal-character}). Therefore, $\cf_T(S^\lambda)$ is described by the claimed equalizer diagram.
\end{proof}
\begin{remark}
Informally, the image on homotopy sheaves of the map $\cf_T(X) \to \Map(V, \co_{\cM_T}) \simeq \cf_T(X^T)$ consists of those $f\in \pi_\ast \co_{\cM_T}^V$ such that $f(x) \equiv f(y) \pmod{\cI_\alpha}$ for every edge $\alpha: x \to y$ in $\Gamma$. Here, $\cI_\alpha$ is as in \cref{ideal-character}.
\end{remark}
\newpage

\section{Equivariant topology of the affine Grassmannian}\label{satake}
For a topologically minded reader, we recommend the book \cite{chriss-ginzburg} for a nice introduction to more classical aspects of geometric representation theory.
\subsection{Kac-Moody flag varieties}

Fix a complex-oriented even-periodic $\Eoo$-ring $A$ and an oriented commutative $A$-group $\GG$.
\begin{observe}\label{gkm-kac-moody-assumption}
Let $\cg$ be a Kac-Moody group, and let $\cP\subseteq \cg$ be a parabolic subgroup associated to a subset $J\subseteq \Delta$ of simple roots. Let $T = T_\cg/Z(\cg)$ denote the torus of $\cg/Z(\cg)$, and let $W$ be the Weyl group associated to $\cg$. Let $W_\cP$ denote the subgroup of $W$ generated by $s_{\alpha_j}$ for $\alpha_j\in J$, and let $W^J$ denote the set of minimal-length coset representatives in $W_\cg/W_\cP$.

Then $(\cg/\cP)^T \cong W^\cP$, and the Schubert decomposition $\cg/\cP = \coprod_{w\in W^\cP} \cB\dot{w}\cP/\cP$ is a $T$-invariant stratification, where $\ol{w} = \dot{w}\cP/\cP$ is the unique $T$-fixed point in the cell $\cB\dot{w}\cP/\cP$. We claim that $\cg/\cP$ satisfies the hypotheses of \cref{gkm-assumption}. Clearly, condition (a) is satisfied. For condition (b), observe that the tangent space to $\cB\ol{w}$ at $\ol{w}$ is 
$$T_{\ol{w}} \cB\dot{w} \cP/\cP = \fr{b}/(\fr{b} \cap w\cdot \fr{p}) = \bigoplus_{\alpha\in \Phi^+ - w\Phi^+(\fr{p})} \fr{g}_\alpha,$$
where each $\fr{g}_\alpha$ is $1$-dimensional. The weights are therefore all distinct, so condition (b) in \cref{gkm-assumption} is satisfied. For condition (c), let $\alpha\in \Phi^+ - w\Phi^+(\fr{p})$, and let $i_\alpha: \SL_2 \to \cg$ denote the associated subgroup. The closure of $\cB_\alpha\ol{w}$ is $\SL_2\ol{w} = \PP^1$, where the point at $0$ corresponds to $\ol{w}$, and the point at $\infty$ corresponds to $\ol{s_\alpha w}$. Then the GKM graph $\Gamma$ of $\cg/\cP$ has vertices $W^\cP$ and edges $w \to s_\alpha w$ labeled by $s_\alpha\in W_\cg$. See also \cite[Section 5]{generalized-gkm}.
\end{observe}
\begin{warning}\label{warning homology}
In the following, the reader should replace the symbol ``$\cf_T(\cg/\cP)$'' by $\cf_T(X_{\leq w})$ where $X_{\leq w}$ is a Schubert cell in $\cg/\cP$. In this case, $X_{\leq w}$ is a finite CW-complex, so that $\cf_T(X_{\leq w})$ is a \textit{perfect} $\co_{\cM_T}$-module. This implies that the $T$-equivariant \textit{homology} $\cf_T(X_{\leq w})^\vee$ is the $\co_{\cM_T}$-linear dual of $\cf_T(X_{\leq w})$; note that this is not true of $\cf_T(\cg/\cP)$ when the Kac-Moody group is not of finite type. (In general, homology is a predual of cohomology, but the linear dual of cohomology does not recover homology in the non-finite case.) We \textit{define} $\cf_T(\cg/\cP)^\vee$ as the direct limit of $\cf_T(X_{\leq w})^\vee$. 
\end{warning}
Since $\cg/\cP$ satisfies the hypotheses of \cref{gkm-assumption} by \cref{gkm-kac-moody-assumption}, we may apply \cref{gkm-main} to calculate $\cf_T(\cg/\cP)$. See \cite{k-thy-schubert-grg} for a related discussion.
\begin{theorem}\label{kac-moody-gkm}
The following diagram is an equalizer on $\pi_0$:
$$\cf_T(\cg/\cP) \to \Map(W^\cP, \co_{\cM_T}) \rightrightarrows \prod_{\alpha: w \to s_\alpha w} \co_{\cM_{T_\alpha}}.$$
Here, the two maps are given by restriction and applying $s_\alpha$ to $W^\cP$.
Therefore, $\pi_0 \cf_T(\cg/\cP)$ is the sub-$\pi_0 \co_{\cM_T}$-algebra of $\Map(W^\cP, \pi_0 \co_{\cM_T})$ consisting of those maps $f: W^\cP \to \pi_0 \co_{\cM_T}$ such that 
\begin{equation}\label{gkm-condition}
    f(s_\alpha w) \equiv f(w) \pmod{\cI_\alpha} \text{ for all }w \in W^\cP, \alpha\in \Phi.
\end{equation}
\end{theorem}
Motivated by \cref{kac-moody-gkm}, we may define an algebraic generalization of $\pi_0 \cf_T(\cg/\cP)$ as follows.
\begin{construction}\label{coxeter-system}
Let $(W,S)$ be a Coxeter system, and let $V = \RR^S$ denote the associated geometric representation. For $s\in S$, let $\alpha_s$ denote the associated vector, let $\Phi = \{w(\alpha_s) | s\in S, w\in W\}$ be the set of roots, and let $\Phi^+\subseteq \Phi$ denote the set of positive roots. Let $\Lambda = \Z\Phi \subseteq V$ denote the associated root lattice. Fix a smooth $1$-dimensional affine group scheme $\GG_0$ over a commutative ring $R$, and let $\cM_{T,0} = \Hom(\Lambda^\vee, \GG_0)$. Given a character $\lambda$, let $c_\lambda$ denote a function which cuts out the closed subscheme $\GG_{\ker(\lambda)} \hookrightarrow \cM_{T,0}$.
Define $\KK$ to be the sub-$\co_{\cM_{T,0}}$-algebra of $\Map(W, \co_{\cM_{T,0}})$ consisting of those maps $f: W \to \co_{\cM_{T,0}}$ satisfying \cref{gkm-condition}, i.e., such that $f(s_\alpha w) \equiv f(w) \pmod{c_\alpha}$ for $\alpha\in \Phi$ and $w\in W$.
\end{construction}
\begin{remark}
Note that if $\lambda$ is a character, then the function $c_\lambda$ on $\cM_T$ is given by the $T$-equivariant Thom class of the representation of $T$ given by $\lambda: T \to \GG_m^\rot$. Morever, $c_\lambda$ generates $\cI_\lambda$.
\end{remark}
\begin{lemma}\label{salpha-equiv-1}
Let $s_\alpha\in W$, and let $T_\alpha = \ker(\alpha)\subseteq T$. Then we have the following commutative diagram of $R$-schemes (where the non-vertical arrows are closed immersions):
$$\xymatrix{
\cM_{T_\alpha,0} \ar[r]^-q \ar[dr]_-q & \cM_{T,0} \ar[d]^-{s_\alpha} \\
& \cM_{T,0};
}$$
informally, $s_\alpha \equiv 1\pmod{\cI_\alpha}$.
\end{lemma}
\begin{proof}
This follows from the fact that the character lattice of $T_\alpha$ is the quotient of $\bX^\ast(T)$ by the rank $1$ sublattice generated by $\alpha$; therefore, if $\chi\in \bX^\ast(T)$, then $s_\alpha \chi|_{T_\alpha} = \chi|_{T_\alpha}$.
\end{proof}
\cref{kac-moody-gkm} implies the following:
\begin{corollary}
Suppose $\GG_0$ is affine. Then there is an equivalence $\pi_0 \cf_T(\cg/\cP)^\vee \simeq \co_{\cM_{T,0}}[W^\cP, \tfrac{s_\alpha-1}{c_\alpha}, \alpha\in \Phi]$ of $\pi_0 \co_{\cM_T}$-modules.
\end{corollary}
Recall that if $w\in W$, then $\inv(w)\subseteq \Phi^+$ denotes the set of positive roots $\alpha$ such that $s_\alpha w < w$. The following is then the analogue of \cite[Lemma 2.3, Lemma 2.5, Proposition 2.6]{k-thy-schubert-grg}.
\begin{prop}\label{basis-cohomology}
Suppose that $\GG$ is affine.
In \cref{coxeter-system}, $\KK$ is a free $\co_{\cM_{T,0}}$-module spanned by functions $\psi_w: W \to \co_{\cM_{T,0}}$ for $w\in W$, where $\psi_w$ is uniquely characterized by the property that it satisfies \cref{gkm-condition} and the following two properties:
\begin{align*}
    \psi_w(v) & = 0 \text{ if }v<w,\\
    \psi_w(w) & = \prod_{\alpha\in \inv(w)} c_\alpha.
\end{align*}
\end{prop}
\begin{proof}
The two stated conditions define $\psi_w$ on the interval $[1, w]\subseteq W$. We will now define an extension of $\psi_w$ to the whole of $W$. We will in fact prove the following more general claim by induction on $\ell(w)$:
\begin{enumerate}
    \item[$(\ast)$] Let $w\in W$, and let $[1,w]^\circ = [1,w]-\{w\}$. Then any function $\psi: [1,w]^\circ \to \co_{\cM_{T,0}}$ satisfying \cref{gkm-condition} extends to a function $[1,w] \to \co_{\cM_{T,0}}$ satisfying \cref{gkm-condition}. 
\end{enumerate}

To see this, write $w = s_{i_1} \cdots s_{i_n}$, let $\alpha = \alpha_{i_1}$, and let $w' = s_\alpha w$ (so that $w'<w$). Consider the restriction of $\psi$ to $[1, w']^\circ$, so that $\psi$ itself is an extension to $[1,w']$. Define $\psi': [1,w']^\circ \to \co_{\cM_{T,0}}$ by the formula $\psi'(v) = s_\alpha \psi(s_\alpha v)$. Then $\psi'$ also satisfies \cref{gkm-condition}: indeed, if $\beta$ is another root, then $\psi'(s_\beta v) \equiv \psi'(v) \pmod{\cI_\beta}$ if and only if $\psi(s_\alpha s_\beta v) \equiv \psi(s_\alpha v) \pmod{s_\alpha \cI_\beta}$. However, $s_\alpha \cI_\beta = \cI_{s_\alpha(\beta)}$, while $s_\alpha s_\beta = s_{s_\alpha(\beta)} s_\alpha$. The claim therefore follows from the assumption that $\psi$ satisfies \cref{gkm-condition}.

Since $w'<w$, the inductive hypothesis says that $\psi'$ extends to a function $\psi': [1,w'] \to \co_{\cM_{T,0}}$ which satisfies \cref{gkm-condition}. If $v\in [1,w']^\circ$, then
$$\psi(v) - \psi'(v) = \psi(v) - s_\alpha \psi(s_\alpha v) \equiv (1 - s_\alpha) \psi(v) \pmod{\cI_\alpha}.$$
By \cref{salpha-equiv-1}, we see that $\psi(v) - \psi'(v) \equiv 0\pmod{\cI_\alpha}$, so we may define a function $p_v\in \co_{\cM_{T,0}}$ by the formula $\frac{\psi(v) - \psi'(v)}{c_\alpha}$. If $\beta\in \Phi^+$ is such that $s_\beta w'<w'$, then:
\begin{align*}
    \psi(w') - \psi'(w') & \equiv \psi(s_\beta w') - \psi'(s_\beta w') \pmod{\cI_\beta} \\
    & = c_\alpha p_{s_\beta w'} \pmod{\cI_\beta}.
\end{align*}
In particular, there is a function $p_{w'} \in \co_{\cM_{T,0}}$ such that 
$$\psi(w') - \psi'(w') \equiv c_\alpha p_{w'} \pmod{\cI_\beta}$$
for all $\beta\in \Phi^+$ such that $s_\beta w'<w'$, i.e., $\beta\in \inv(w')$. In particular, 
\begin{equation}\label{pw-prime}
    \psi(w') - \psi'(w') \equiv c_\alpha p_{w'} \pmod{\prod_{\beta\in \inv(w')} \cI_\beta}.
\end{equation}
Note that $s_\alpha \inv(w')$ is the set of $\beta \in \Phi^+-\{\alpha\}$ such that $s_\beta w'<w'$. Define
$$\psi(w) = s_\alpha \psi'(w') + x\prod_{\beta\in s_\alpha \inv(w')} c_\beta$$
for some $x$ that we will determine in a moment. We check that $\psi$ satisfies \cref{gkm-condition}. Let $\alpha'\in \Phi^+$ be such that $s_{\alpha'} w<w$. Then:
\begin{enumerate}
    \item If $\alpha' = \alpha$, then
    \begin{align*}
        \psi(w) - \psi(s_\alpha w) & = s_\alpha \psi'(w') - \psi(w') + x\prod_{\beta\in s_\alpha \inv(w')} c_\beta \\
        & \equiv s_\alpha (\psi'(w') - \psi(w')) + x\prod_{\beta\in s_\alpha\inv(w')} c_\beta \pmod{\cI_\alpha}
    \end{align*}
    However, \cref{pw-prime} implies that 
    $$s_\alpha (\psi(w') - \psi'(w')) \equiv c_{-\alpha} s_\alpha(p_{w'}) \pmod{\prod_{\beta\in s_\alpha \inv(w')} \cI_\beta}$$
    Therefore, taking $x$ to be the negative of the residue of $s_\alpha (\psi(w') - \psi'(w')) - c_{-\alpha} s_\alpha(p_{w'})$ modulo $\prod_{\beta\in s_\alpha \inv(w')} \cI_\beta$, we see that
    \begin{align*}
        \psi(w) - \psi(s_\alpha w) & \equiv c_{-\alpha} s_\alpha(p_{w'}) \equiv 0 \pmod{\cI_\alpha},
    \end{align*}
    as desired.
    \item If $\alpha' \neq \alpha$, then $\alpha'\in s_\alpha \inv(w')$. Then, we have
    \begin{align*}
        \psi'(w') & \equiv \psi'(s_{s_\alpha(\alpha')} w') \pmod{\cI_{s_\alpha(\alpha')}} \\
        & = s_\alpha \psi(s_\alpha s_{s_\alpha(\alpha')} s_\alpha w) \pmod{\cI_{s_\alpha(\alpha')}}\\
        & = s_\alpha \psi(s_{\alpha'} w) \pmod{\cI_{s_\alpha(\alpha')}}.
    \end{align*}
    In particular, $s_\alpha \psi'(w') \equiv \psi(s_{\alpha'} w) \pmod{\cI_{\alpha'}}$. But this implies that
    \begin{align*}
        \psi(w) - \psi(s_{\alpha'} w) & \equiv s_\alpha \psi'(w') - \psi(s_{\alpha'} w) \pmod{\cI_{\alpha'}}\\
        & \equiv 0 \pmod{\cI_{\alpha'}},
    \end{align*}
    as desired.
\end{enumerate}
This finishes the proof of $(\ast)$.

To finish the proof of the proposition, note that the two conditions on $\psi_w$ specify it on $[1,w]$, and hence on the subset of $W$ consisting of elements of length $<\ell(w)$. By $(\ast)$, we may inductively extend $\psi_w$ to the subset of $W$ consisting of elements of length $\geq \ell(w)$, and hence to all of $W$. It remains to show that any $\psi\in \Map(W, \co_{\cM_{T,0}})$ satisfying \cref{gkm-condition} can be written as a $\co_{\cM_{T,0}}$-linear combination of the $\psi_w$; see the second half of \cite[Proposition 2.6]{k-thy-schubert-grg} for the following argument.

Let $\supp(\psi)$ denote the subset of $w\in W$ such that $f(\psi)\neq 0$. Let $v\in \supp(\psi)$ be minimal. If $\alpha\in \inv(v)$ (so $s_\alpha v<v$), then $\psi(v) \equiv \psi(s_\alpha v) = 0\pmod{\cI_\alpha}$. This implies that $\psi(v) \equiv 0 \pmod{\psi_v(v)}$. Define $\psi': W \to \pi_0 \co_{\cM_{T,0}}$ by $\psi'(w) = \psi(w) - \tfrac{\psi(v)}{\psi_v(v)} \psi_v(w)$; then $\psi'$ satisfies \cref{gkm-condition} (since $\psi$ and $\psi_v$ do). By construction, $v\not\in \supp(\psi')$, and $\supp(\psi')-\supp(\psi)$ consists of elements which are strictly larger than $v$. Therefore, we may repeat this argument for $\psi'$, and induct; this yields the desired result.
\end{proof}


\subsection{The affine Grassmannian}
\begin{setup}
Fix notation as in \cref{group-notation}, and assume that $G$ is semisimple. Then we have an associated affine root datum: the affine simple roots are $\Delta_\aff = \Delta \cup\{0\}$, and the affine weight lattice is given by $\Z K \oplus \bigoplus_{\alpha_i\in \Delta_\aff} \Z \alpha_i$. (In particular, we denote the affine root by $\alpha_0$.) Thus the associated Kac-Moody algebra is $\hat{\g} = \g\ls{t} \oplus \cc \alpha_0 \oplus \cc K$, where $K$ is the central class, and $\alpha_0$ is the scaling factor.
Let $\cg$ denote the associated Kac-Moody group, and let $W^\aff = \Lambda^\vee \rtimes W$ denote the associated affine Weyl group. If $\lambda^\vee \in \Lambda^\vee$, we write $t_{\lambda^\vee}$ to denote the associated element of $W^\aff$. If $\alpha + n\alpha_0$ is an affine root and $x\in \fr{t}$, then
$$s_{\alpha+n\alpha_0}(x) = x - (\langle x, \alpha\rangle + n) \alpha^\vee = s_\alpha(x) + n\alpha^\vee.$$
Let $\cB$ denote the Iwahori subgroup, and $T_\aff$ the maximal torus of $\cg$. Then $\cg/\cB$ is the affine flag variety $\Fl_G$; similarly, $\Gr_G$ is the Kac-Moody flag variety associated to the subset $\Delta\subseteq \Delta_\aff$. Up to keeping track of the central torus, we may view $\cg$ as $G\ls{t}$, and $\cB$ as the Iwahori $I$. Thus $T = T^\aff \cap G$ is the maximal torus of $G$. Let $\tilde{T}$ denote the extended torus $T\times \GG_m^\rot$ (where $\GG_m^\rot$ is the loop rotation torus); we may identify its Lie algebra $\tilde{\fr{t}}$ with $\fr{t} \oplus \cc \alpha_0$.
\end{setup}
\begin{remark}
Let $\alpha\in \Phi$ and $n\in \Z$. Then $n\alpha_0$ is the $\GG_m^\rot$-representation of weight $n$. 
%Write $[n](\hbar_A)$ to denote the local section of $\pi_0 \co_{\cM_{\GG_m^\rot}} \cong \pi_0 \co_\GG$ given by the Thom class of the line bundle $\cf_{\GG_m^\rot}(S^{n\alpha_0})$ on $\cM_{\GG_m^\rot} \simeq \GG$, so that it is $\hbar_A +_\GG \cdots +_\GG \hbar_A$.
Note that $\alpha + n\alpha_0$ defines an ideal sheaf $\cI_{\alpha + n\alpha_0} \subseteq \pi_0 \co_{\cM_{\tilde{T}}} = \pi_0 \co_{\cM_T} \otimes_{\pi_0 A} \pi_0 \co_\GG$.
\end{remark}
\cref{kac-moody-gkm} gives an explicit description of $\pi_0 \cf_{T^\aff}(\Fl_G)$ and $\pi_0 \cf_{T^\aff}(\Gr_G)$. Using that 
\begin{align*}
(\Fl_G)^T & = (\Fl_G)^{\tilde{T}} = W^\aff\\
(\Gr_G)^T & = (\Gr_G)^{\tilde{T}} = W^\aff/W \cong \Lambda^\vee,
\end{align*}
this further immediately specializes to the following explicit description of $\pi_0 \cf_{\tilde{T}}(\Fl_G)$ and $\pi_0 \cf_{\tilde{T}}(\Gr_G)$:
\begin{corollary}\label{loop-rot-coh-grg}
The following statements are true:
\begin{enumerate}
    \item We may identify $\pi_0 \cf_{\tilde{T}}(\Fl_G) \cong \pi_0 \cf_{\GG_m^\rot}(\Fl_G/I)$ with $\KK$ from \cref{coxeter-system}, i.e., as the sub-$\pi_0 \co_{\cM_{\tilde{T}}}$-algebra of $\Map(W^\aff, \pi_0 \co_{\cM_{\tilde{T}}})$ consisting of those maps $f: W^\aff \to \pi_0 \co_{\cM_{\tilde{T}}}$ such that 
    \begin{equation}
    f(s_{\alpha+n\alpha_0}(w)) \equiv f(w) \pmod{\cI_{\alpha + n\alpha_0}}
    \end{equation}
    for all $w \in W^\aff, \alpha\in \Phi, n\in \Z$.
    \item We may identify $\pi_0 \cf_{\tilde{T}}(\Gr_G) \cong \pi_0 \cf_{\GG_m^\rot}(\Gr_G/I)$ as the sub-$\pi_0 \co_{\cM_{\tilde{T}}}$-algebra of $\Map(\Lambda^\vee, \pi_0 \co_{\cM_{\tilde{T}}})$ consisting of those maps $f: \Lambda^\vee \to \pi_0 \co_{\cM_{\tilde{T}}}$ such that 
    \begin{equation}
    f(s_{\alpha+n\alpha_0}(\lambda)) \equiv f(\lambda) \pmod{\cI_{\alpha + n\alpha_0}}
    \end{equation}
    for all $\lambda \in \Lambda^\vee, \alpha\in \Phi, n\in \Z$.
\end{enumerate}
\end{corollary}
\begin{corollary}\label{cohomology-grg}
The following statements are true:
\begin{enumerate}
    \item We may identify $\pi_0 \cf_T(\Fl_G)$ as the sub-$\pi_0 \co_{\cM_T}$-algebra of $\Map(W^\aff, \pi_0 \co_{\cM_T})$ consisting of those maps $f: W^\aff \to \pi_0 \co_{\cM_T}$ such that 
    \begin{equation}
    f(s_{\alpha+n\alpha_0}(w)) \equiv f(w) \pmod{\cI_\alpha}
    \end{equation}
    for all $w \in W^\aff, \alpha\in \Phi, n\in \Z$.
    \item We may identify $\pi_0 \cf_T(\Gr_G)$ as the sub-$\pi_0 \co_{\cM_T}$-algebra of $\Map(\Lambda^\vee, \pi_0 \co_{\cM_T})$ consisting of those maps $f: \Lambda^\vee \to \pi_0 \co_{\cM_T}$ such that 
    \begin{equation}\label{gkm-grg}
    f(s_{\alpha+n\alpha_0}(\lambda)) \equiv f(\lambda) \pmod{\cI_\alpha}
    \end{equation}
    for all $\lambda \in \Lambda^\vee, \alpha\in \Phi, n\in \Z$.
\end{enumerate}
\end{corollary}
\begin{observe}
The image of $s_{\alpha+n\alpha_0}$ under the identification $W^\aff/W \cong \Lambda^\vee$ is the right coset $s_{\alpha+n\alpha_0} W$. However, $s_{\alpha+n\alpha_0} s_\alpha$ is translation by $n\alpha^\vee$. If $k$ is a commutative ring, we may view $k[\Lambda^\vee]$ as the $\Eoo$-ring of functions on $\ld{T}_k$; the element $n\alpha^\vee\in \Lambda^\vee$ corresponds to the function $e^{n\alpha^\vee}$. Therefore, \cref{gkm-grg} can be restated as 
$$f((e^{n\alpha^\vee}-1)(\lambda)) \equiv 0 \pmod{\cI_\alpha}.$$
If $\GG$ is affine, then $\pi_0 \cf_T(\Gr_G)$ is the $\pi_0 \co_{\cM_T}$-linear dual of $\pi_0 \co_{\cM_T}[\Lambda^\vee][\tfrac{e^{n\alpha^\vee}-1}{c_\alpha}]_{n\geq 1}$. However, note that for any $n\geq 1$, we may write
%use the multiplicative formal group law to obtain $\tfrac{e^{n\alpha^\vee}-1}{c_\alpha}$ from $\tfrac{e^{\alpha^\vee}-1}{c_\alpha}$.
$$\tfrac{e^{n\alpha^\vee}-1}{c_\alpha} = \tfrac{e^{\alpha^\vee}-1}{c_\alpha} + \tfrac{e^{(n-1)\alpha^\vee}-1}{c_\alpha} + c_\alpha \tfrac{e^{\alpha^\vee}-1}{c_\alpha} \tfrac{e^{(n-1)\alpha^\vee}-1}{c_\alpha}.$$
This implies that
$$\pi_0 \cf_T(\Gr_G) \cong \Map_{\QCoh(\cM_{T,0})}(\pi_0 \co_{\cM_T}[\Lambda^\vee][\tfrac{e^{\alpha^\vee}-1}{c_\alpha}], \pi_0 \co_{\cM_T}).$$
\end{observe}
\begin{remark}
Let $\lambda\in \Lambda^{\vee,\pos}$ be a dominant coweight, and let $\Lambda^{\vee,\pos}_{\leq \lambda}$ denote the subset of $\Lambda^{\vee,\pos}$ consisting of those dominant weights which are at most $\lambda$. Then we may identify
$$(\Gr_G^{\leq \lambda})^T = W\cdot \Lambda^{\vee,\pos}_{\leq \lambda} \subseteq \Lambda^\vee = (\Gr_G)^T,$$
which allows us to calculate that if $\GG$ is affine, then
$$\pi_0 \cf_T(\Gr_G^{\leq \lambda}) \cong \Map_{\QCoh(\cM_{T,0})}(\pi_0 \co_{\cM_T}[W\cdot \Lambda^{\vee,\pos}_{\leq \lambda}][\tfrac{e^{\alpha^\vee}-1}{c_\alpha}], \pi_0 \co_{\cM_T}).$$
In the above expression, $\alpha$ ranges over $\Phi \cap W\cdot \Lambda^{\vee,\pos}_{\leq \lambda}$; in other words, $\alpha$ is of the form $w\alpha_i$ with $\alpha_i\in \Delta$ such that $\alpha_i\leq \lambda$.
\end{remark}
\begin{remark}\label{caveat filtration}
Recall from \cref{warning homology} that $\cf_T(\Gr_G)^\vee$ is defined to be the direct limit of $\cf_T(\Gr_G^{\leq \lambda})^\vee$.
We trust the reader to make the appropriate modifications below as needed (which we have not done to avoid an overbearance of notation), so that the calculation of the $T$-equivariant homology $\cf_T(\Gr_G)^\vee$ in \cref{t-homology-grg} by taking the linear dual of $\cf_T(\Gr_G)$ does not suffer from completion issues. This can be done, for instance, by working with the \textit{$\Lambda^{\vee,\pos}$-filtered} $\co_{\cM_T}$-module $\{\cf_T(\Gr_G^{\leq \lambda})^\vee\}$.
In order for the colimit $\cf_T(\Gr_G)^\vee$ of the $\Lambda^{\vee,\pos}$-filtered module $\{\cf_T(\Gr_G^{\leq \lambda})^\vee\}$ to admit the structure of an $\E{2}$-$\co_{\cM_T}$-algebra, it suffices to show that $\{\cf_T(\Gr_G^{\leq \lambda})^\vee\}$ admits the structure of an $\E{2}$-algebra in $\Lambda^{\vee,\pos}$-filtered module; this is proved in \cref{filtered E2} below.
\end{remark}
\begin{lemma}\label{filtered E2}
The $\Lambda^{\vee,\pos}$-indexed Schubert filtration $\{\Gr_G^{\leq \lambda}(\cc)\}$ naturally admits the structure of an $\E{2}$-algebra in $\Fun(\Lambda^{\vee,\pos}, \Top)$.
\end{lemma}
\begin{proof}
This can be proved in essentially the same way as \cite[Theorem 3.10]{hahn-yuan}; let us sketch the argument. We will utilize \cite[Proposition 5.4.5.15]{HA}, which states that if $\cC$ is a symmetric monoidal $\infty$-category, then a nonunital $\E{2}$-algebra object in $\cC$ is equivalent to the datum of a locally constant $\mathrm{N}(\mathrm{Disk}(\cc))_\mathrm{nu}$-algebra object in $\cC$. Concretely, this amounts to specifying an object $A(D)\in \cC$ for every disk $D\subseteq \cc$ and coherent maps $\bigotimes_{i=1}^n A(D_i)\to A(D)$ for every inclusion $\coprod_{i=1}^n D_i\to D$ of disks, such that for every embedding $D\subseteq D'$ of disks, the induced map $A(D)\to A(D')$ is an equivalence.

In this case, $\cC = \Fun(\Lambda^{\vee,\pos}, \Top)$, and the object $A(D)\in \Fun(\Lambda^{\vee,\pos}, \Top)$ assigned to a disk $D\subseteq \cc$ may be defined via the Beilinson-Drinfeld Grassmannian $\Gr_{G,\Ran}$. Namely, the Beilinson-Drinfeld Grassmannian admits (by construction) a morphism $\Gr_{G, \Ran} \to \Ran_{\AA^1}$; upon taking complex points, we obtain a map $\Gr_{G, \Ran}(\cc) \to \Ran(\cc)$. If $S\subseteq \cc$ is a subset, then the preimage of $\Ran(S)\subseteq \Ran(\cc)$ defines a subspace $\Gr_{G, \Ran}(S\subseteq \cc)\subseteq \Gr_{G, \Ran}(\cc)$. The filtration of $\Gr_G$ via the Bruhat decomposition extends to a filtration $\Gr_{G, \Ran, \leq \mu}$ of $\Gr_{G, \Ran}$ by dominant coweights $\mu\in \Lambda^{\vee,\pos}$; see \cite[3.1.11]{zhu-grass}. Finally, the object $A(D)\in \Fun(\Lambda^{\vee,\pos}, \Top)$ associated to a disk $D\subseteq \cc$ is the functor $\Lambda^{\vee,\pos}\to \Top$ sending $\mu\in \Lambda^{\vee,\pos}$ to $\Gr_{G, \Ran, \leq \mu}(D\subseteq \cc)$.

Suppose $\coprod_{i=1}^n D_i\to D$ is an inclusion of disks. The induced map $\bigotimes_{i=1}^n A(D_i)\to A(D)$ is defined as follows. Let $\mu\in \Lambda^{\vee,\pos}$; for every $n$-tuple $(\mu_1, \cdots, \mu_n)$ with $\sum_{i=1}^n \mu_i\leq \mu$, we need to exhibit maps $\bigotimes_{i=1}^n A(D_i)(\mu_i)\to A(D)(\mu)$ satisfying the obvious coherences. But
$$\bigotimes_{i=1}^n A(D_i)(\mu_i) = \prod_{i=1}^n \Gr_{G, \Ran, \leq \mu_i}(D_i\subseteq \cc),$$
so it suffices to show that if $\mu_1 + \mu_2 \leq \mu$, then there are maps $\Gr_{G, \Ran, \leq \mu_1}(D_1\subseteq \cc) \times \Gr_{G, \Ran, \leq \mu_2}(D_2\subseteq \cc)\to \Gr_{G, \Ran, \leq \mu}(D\subseteq \cc)$. The argument for this is exactly as in \cite[Construction 3.15]{hahn-yuan}.

We next need to show that the $\mathrm{N}(\mathrm{Disk}(\cc))_\mathrm{nu}$-algebra $A$ defined above is locally constant, i.e., that if $D\subseteq D'$ is an embedding of disks, then $A(D)\to A(D')$ is an equivalence of functors $\Lambda^{\vee,\pos}\to \Top$. This follows from \cite[Proposition 3.17]{hahn-yuan}. To conclude, it suffices (by \cite[Theorem 5.4.4.5]{HA}) to establish the existence of a quasi-unit for the functor $A:\Lambda^{\vee,\pos}\to \Top$, i.e., a map $1_{\Fun(\Lambda^{\vee,\pos}, \Top)}\to A$ which is both a left and right unit up to homotopy. Since the unit in $\Fun(\Lambda^{\vee,\pos}, \Top)$ is the functor sending $\mu\in \Lambda^{\vee,\pos}$ to the point $\ast$, a quasi-unit is the datum of a map $\ast \to \Gr_{G, \leq \mu}(\cc)$ for each $\mu\in \Lambda^{\vee,\pos}$. As in the proof of \cite[Theorem 3.10]{hahn-yuan}, this can be taken to be the inclusion of the point corresponding to the trivial $G$-bundle over $\AA^1$ with the canonical trivialization away from the origin.
\end{proof}

With \cref{caveat filtration} in mind, we can now use \cref{cohomology-grg} to compute the $T$-equivariant homology of $\Gr_G$.
\begin{lemma}\label{grt-homology}
There is an equivalence in $\Alg_\E{2}(\coCAlg(\QCoh(\cM_T)))$:
$$\cf_T(\Gr_T(\cc))^\vee \cong \co(\ld{T}_{A} \times_{\spec(A)} \cM_T).$$
\end{lemma}
\begin{proof}
Since the action of $T$ on $\Gr_T(\cc)$ is trivial, we have a canonical equivalence $\cf_T(\Gr_T(\cc))^\vee \simeq \Gr_T(\cc)_+ \otimes \cf_T(\ast)^\vee$. By definition, $\cf_T(\ast)^\vee \simeq \co_{\cM_T}$. We conclude that $\cf_T(\Gr_T(\cc))^\vee$ is equivalent as an $\E{2}$-$A$-algebra to $C_\ast(\Gr_T(\cc); A) \otimes_A \co_{\cM_T}$. Since $BT(\cc) \simeq B^2\Lambda^\vee$, there is an equivalence $\Gr_T(\cc) \simeq \Lambda^\vee$ of $\E{2}$-spaces. Therefore, $C_\ast(\Gr_T(\cc); A) \simeq A[\Lambda^\vee]$ as $\E{2}$-$A$-algebras, which is $\co(\ld{T}_A)$. This implies the desired claim.
\end{proof}
\begin{question}
Can \cref{grt-homology} be upgraded to an equivalence of \textit{$\E{3}$-$A$-algebras} for a geometrically defined $\E{3}$-algebra structure on $\cf_T(\Gr_T(\cc))^\vee$? This additional structure is crucial for a statement of the geometric Satake correspondence which is $\E{3}$-monoidal.
\end{question}

\begin{notation}
Let $T^\ast_\GG \ld{T}_A$ denote $\ld{T}_{A} \times_{\spec(A)} \cM_T$, and let $T^\ast_\GG \ld{T}$ denote its underlying scheme (over $\cM_{T,0}$).
%We will write $(T^\ast_\GG \ld{T})^\bl$ for the affine blowup of $T^\ast_\GG \ld{T}$ from \cref{t-homology-grg}, so that $\ul{\spec}_{\cM_{T,0}}(\pi_0 \cf_T(\Gr_G(\cc))^\vee) \cong (T^\ast_\GG \ld{T})^\bl$ as $\cM_{T,0}$-schemes. 
Note that if $\GG = \GG_a$, then $T^\ast_\GG \ld{T}$ is the cotangent bundle of $\ld{T}$, while if $\GG = \GG_m$, then $T^\ast_\GG \ld{T} = \ld{T} \times T$. If $\fr{B}_\GG$ denotes the blowup of $T^\ast_\GG \ld{T}$ at the closed subscheme given by $\cM_{T_\alpha,0}$ and the zero set of $e^{\alpha^\vee}-1$ for $\alpha\in \Phi$, then define $(T^\ast_\GG \ld{T})^\bl$ as the complement of the proper preimage of $\cM_{T_\alpha,0}$ in $\fr{B}_\GG$ for $\alpha\in \Phi$.
\end{notation}
\begin{theorem}\label{t-homology-grg}
Let $G$ be a connected semisimple algebraic group over $\cc$. Then there is a $W$-equivariant isomorphism 
$\spec \pi_0 \cf_T(\Gr_G(\cc))^\vee \cong (T^\ast_\GG \ld{T})^\bl$ of schemes over $\cM_{T,0}$, where the left-hand side denotes the relative $\spec$.
%$$\pi_0 \cf_T(\Gr_G(\cc))^\vee \cong \pi_0 \co_{\cM_T}[\Lambda^\vee] [\tfrac{e^{\alpha^\vee}-1}{c_\alpha}, \alpha\in \Phi]$$
\end{theorem}
\begin{proof}
There is an $\E{2}$-map $\Gr_T(\cc) \to \Gr_G(\cc)$, which induces an $\E{2}$-map $\cf_T(\Gr_T(\cc))^\vee\to \cf_T(\Gr_G(\cc))^\vee$. This is given by dualizing the map $r: \cf_T(\Gr_G(\cc)) \to \cf_T(\Gr_T(\cc))$ of $\E{2}$-coalgebras in $\QCoh(\cM_T)$. The non-$W$-equivariant claim now follows from \cref{cohomology-grg}, since $r$ induces an injection on $\pi_0$, and the (cocommutative) Hopf algebra structure on $\pi_0 \cf_T(\Gr_T(\cc))$ is given by the dual of the equivalence of \cref{grt-homology}. Proving $W$-equivariance requires a bit more work, but can easily be incorporated by keeping track of the $W$-action throughout the above discussion.
\end{proof}
\begin{remark}
The $T$-equivariant and $G$-equivariant $A$-\textit{co}homologies of $\Gr_G(\cc)$ are significantly easier to compute in terms of the stack $\cM_G$ (without any reference to root data); see \cref{A-cohomology of Gr}. In particular, see \cref{BF coh of gr} for an alternative argument for \cite[Theorem 1]{bf-derived-satake} using Hochschild homology and the Hochschild-Kostant-Rosenberg theorem.
\end{remark}
%\begin{remark}
%If $\GG$ is affine, \cref{t-homology-grg} says that there is an equivalence
%$$\cf_T(\Gr_G(\cc))^\vee \cong \co(\ld{T}_{A} \times_{\spec(A)} \cM_T)[\tfrac{e^{\alpha^\vee}-1}{c_\alpha}, \alpha\in \Phi].$$
%%of $\E{1}$-$\co_{\cM_T}$-algebras; note that \cref{blowup-example}(c) only guarantees an $\E{1}$-algebra structure on the right-hand side.
%\end{remark}
\begin{remark}
Suppose $A = \KU$, so that $\GG = \GG_m$ and $c_\alpha$ is $e^\alpha-1$.
It follows from \cref{t-homology-grg} that replacing $T$ with $\ld{T}$, we get an isomorphism between $\pi_0 \cf_{\ld{T}}(\Gr_\ld{G}(\cc))^\vee$ and $\pi_0 (T_{A} \times_{\spec(A)} \ld{T}_A)[\tfrac{e^{\alpha}-1}{e^{\alpha^\vee}-1}, \alpha\in \Phi]$. Therefore, $\pi_0 \cf_{{T}}(\Gr_{G}(\cc))^\vee$ and $\pi_0 \cf_{\ld{T}}(\Gr_\ld{G}(\cc))^\vee$ are both obtained from the blowup $\fr{B}_{\GG_m}$ of $T^\ast_\GG \ld{T}$ by deleting the proper preimage of two different closed subschemes which are Langlands dual to each other. In particular, the Langlands self-duality of the blowup $\fr{B}_{\GG_m}$ swaps the affine pieces $\spec \pi_0 \cf_{T}(\Gr_{G}(\cc))^\vee$ and $\spec \pi_0 \cf_{\ld{T}}(\Gr_\ld{G}(\cc))^\vee$ in $\fr{B}_{\GG_m}$.
\end{remark}

\begin{remark}
When $G = \SL_2$ or $\PGL_2$, we can explicitly verify \cref{t-homology-grg} at least after base-changing along $C_T^\ast(\ast; A) \to C^\ast(\ast; A)$. We will identify $\PGL_2$ with $\SO_3$ (via the $\PGL_2$-action on $\fr{pgl}_2$ which preserves the quadratic form given by the determinant). If $A = \QQ[\beta^{\pm 1}]$, for instance, \cref{t-homology-grg} says:
\begin{align*}
    \pi_0 C_\ast^{S^1}(\Omega S^3; \QQ[\beta^{\pm 1}]) & \cong \QQ[x,y^{\pm 1}, \tfrac{y-1}{x}], \\
    \pi_0 C_\ast^{S^1}(\Omega \SO(3); \QQ[\beta^{\pm 1}]) & \cong \QQ[x,y^{\pm 1}, \tfrac{y^2-1}{2x}].
\end{align*}
After killing $x$, the fraction $\tfrac{y-1}{x}$ (resp. $\tfrac{y^2-1}{x}$) defines a polynomial generator, and so we have
\begin{align*}
    \pi_0 C_\ast(\Omega S^3; \QQ[\beta^{\pm 1}]) & \cong \QQ[\tfrac{y-1}{x}], \\
    \pi_0 C_\ast(\Omega \SO(3); \QQ[\beta^{\pm 1}]) & \cong \QQ[y^{\pm 1}, \tfrac{y^2-1}{2x}]/(y^2-1).
\end{align*}
The second of these isomorphisms is compatible with the identification $\Omega \SO(3) \simeq \Z/2 \times \Omega S^3$ arising from the isomorphism $S^3/(\Z/2) \cong \SO(3)$ (but note that the equivalence $\Omega \SO(3) \simeq \Z/2 \times \Omega S^3$ is \textit{not} one of $\E{1}$-spaces). Similarly, if $A = \KU$, \cref{t-homology-grg} says:
\begin{align*}
    \pi_0 C_\ast^{S^1}(\Omega S^3; \KU) & \cong \Z[x^{\pm 1},y^{\pm 1}, \tfrac{y-1}{x-1}], \\
    \pi_0 C_\ast^{S^1}(\Omega \SO(3); \KU) & \cong \Z[x^{\pm 1},y^{\pm 1}, \tfrac{y^2-1}{x^2-1}].
\end{align*}
After killing $x-1$, the fraction $\tfrac{y-1}{x-1}$ (resp. $\tfrac{y^2-1}{x^2-1}$) defines a polynomial generator, and so we have
\begin{align*}
    \pi_0 C_\ast(\Omega S^3; \KU) & \cong \Z[\tfrac{y-1}{x-1}], \\
    \pi_0 C_\ast(\Omega \SO(3); \KU) & \cong \Z[y^{\pm 1}, \tfrac{y^2-1}{x^2-1}]/(y^2-1).
\end{align*}
Again, this is compatible with the identification $\Omega \SO(3) \simeq \Z/2 \times \Omega S^3$.

In the case $G = \SL_2$, we refer the reader to \cref{3d-sl2} and \cref{4d-sl2} for an explicit description of $\H^{G\times S^1_\rot}_\ast(\Gr_G(\cc); \cc)$ and $\KU^{G\times S^1_\rot}_0(\Gr_G(\cc)) \otimes \cc$. 
\end{remark}
\subsection{Quantized equivariant homology of $\Gr_T$}\label{sec: quantized homology torus}

We now explore the equivariant homology of $\Gr_T$ in more detail; no GKM theory is required here, but several interesting algebraic structures turn up.
Let us begin by recalling that \cref{grt-homology} gives a $W$-equivariant equivalence $\cf_T(\Gr_T(\cc))^\vee \cong \co(\ld{T}_{A} \times_{\spec(A)} \cM_T)$, which can be thought of as giving an equivalence between $\ld{T}_{A} \times_{\spec(A)} \cM_T$ and the ``$\E{2}$-$\cM_T$-scheme $\spec \cf_T(\Gr_T(\cc))^\vee$''. This admits a natural deformation given by the loop-rotation equivariant homology $\cf_{\tilde{T}}(\Gr_T(\cc))^\vee$. Since $\tilde{T} = T \times \GG_m^\rot$, there is an equivalence $\cM_{\tilde{T}} \simeq \cM_T \times \GG$, where the second factor is identified as $\cM_{\GG_m^\rot}$.
\begin{definition}\label{def: G-diff ops}
Let $\GG_0$ be a smooth $1$-dimensional group scheme over a base commutative ring, let $T$ be a compact torus, let $\Lambda$ (resp. $\Lambda^\vee$) denote the (co)character lattice of $T$, and let $\cM_{0,T} = \Hom(\Lambda, \GG_0)$.
Let $\lambda$ be a cocharacter of $T$, so that $\lambda$ defines a homomorphism $\Lambda \to \Z$, and hence a homomorphism $\lambda^\ast: \GG_0 \to \cM_{0,T}$. In turn, this defines a map
$$f^\lambda: \cM_{0,\tilde{T}} \simeq \cM_{0,T} \times \GG_0 \xar{\pr \times \lambda^\ast} \cM_{0,T}.$$
%There is therefore a map $\co_{\cM_{0,T}} \to f^\lambda_\ast \co_{\cM_{0,\tilde{T}}}$ of $\co_{\cM_{0,T}}$-algebras.
If $y$ is a local section of $\co_{\cM_{0,T}}$, we will write $\lambda^\ast(y)$ to denote the resulting local section of $\co_{\cM_{0,\tilde{T}}}$.
Let $\cd_{\ld{T}}^{\GG_0}$ denote the quotient of the associative $\co_{\GG_0}$-algebra $\co_{\cM_{0,\tilde{T}}}\pdb{x_\lambda | \lambda\in \Lambda}$ by the relations given locally by
$$x_\lambda \cdot x_\mu = x_{\lambda+\mu}, \  y \cdot x_\lambda = x_\lambda \cdot \lambda^\ast(y).$$
Here, $\lambda,\mu\in \Lambda^\vee$, and $y$ is a local section of $\co_{\cM_{0,T}}$. We will call $\cd_{\ld{T}}^{\GG_0}$ the \textit{algebra of $\GG_0$-differential operators}.
\end{definition}
\begin{remark}\label{G-mellin}
The algebra $\cd_{\ld{T}}^{\GG_0}$ satisfies a Mellin transform: namely, it follows from unwinding the definition that there is an equivalence
$$\LMod_{\cd_{\ld{T}}^{\GG_0}}(\QCoh(\GG_0)) \simeq \QCoh(\cM_{0,\tilde{T}}/\Lambda),$$
where $\lambda\in \Lambda$ acts on $\cM_{0,\tilde{T}}$ via $y\mapsto \lambda^\ast y$.
\end{remark}
\begin{notation}
If $A$ is a complex-oriented even-periodic $\Eoo$-ring and $\GG_0$ is the $\pi_0(A)$-group underlying a oriented commutative $A$-group $\GG$, we will write $\cd_{\ld{T}}^\GG$ to denote $\cd_{\ld{T}}^{\GG_0}$, and refer to it as the algebra of $\GG$-differential operators. We hope this does not cause any confusion.
\end{notation}
\begin{prop}[Quantization of \cref{grt-homology}]\label{homology and quantized diffop}
There is an isomorphism $\pi_0 \cf_{\tilde{T}}(\Gr_T(\cc))^\vee \cong \cd_{\ld{T}}^\GG$ of $\pi_0 \co_\GG$-algebras.
\end{prop}
\begin{proof}
Since $\Gr_T(\cc) \simeq \Omega T_c \simeq \Lambda^\vee$, it is easy to see that $\pi_0 \cf_{\tilde{T}}(\Gr_T(\cc))^\vee \cong \bigoplus_{\lambda\in \Lambda^\vee} \pi_0 \co_{\cM_{\tilde{T}}}$; let $x_\lambda$ be a generator of the summand indexed by $\lambda\in \Lambda^\vee$. If $\lambda\in \Lambda^\vee = \Hom(\Lambda, \Z)$, the map $\Omega T_c \to \Omega T_c$ given by multiplication-by-$\lambda$ is $T\times S^1_\rot$-equivariant for the homomorphism $T\times S^1_\rot \to T\times S^1_\rot$ given by $(t, \theta) \mapsto (t \lambda(\theta), \theta)$, where $\lambda$ is viewed as a homomorphism $S^1 \to T$. On weight lattices, this homomorphism induces the map $\Lambda \times \Z \to \Lambda \times \Z$ which sends $(\mu, n) \mapsto (\mu, n+\lambda^\vee(\mu))$. In particular, the composite $\Lambda \to \Lambda \times \Z \to \Lambda \times \Z$ sends $\mu \mapsto (\mu, \lambda^\vee(\mu))$. Applying $\Hom(-, \GG)$ to this composite precisely produces the map $f^\lambda: \cM_{\tilde{T}} \to \cM_T$ from \cref{def: G-diff ops}. This implies the desired identification of $\pi_0 \cf_{\tilde{T}}(\Gr_T(\cc))^\vee$.
\end{proof}
\begin{example}\label{ordinary quantized diffop}
Let $T \cong S^1$ be a torus of rank $1$ (for simplicity).
Suppose $A = \QQ[\beta^{\pm 1}]$, so $\GG = \hat{\GG}_a$ and $\pi_0 \co_{\GG} \cong \QQ\pw{\hbar}$. Then the algebra of \cref{def: G-diff ops} is the quotient of the $\QQ\pw{\hbar}$-algebra $\QQ\pw{\hbar}\pdb{y, x^{\pm 1}}$ by the relation $yx = x(y+\hbar)$. In other words, $y$ acts as the operator $\hbar x\partial_x$, so we simply have that 
$$\H^{\tilde{T}}_0(\Gr_T(\cc); \QQ[\beta^{\pm 1}]) \cong \H^{\tilde{T}}_\ast(\Gr_T(\cc); \QQ) \cong \QQ\pw{\hbar}\pdb{\hbar x\partial_x, x^{\pm 1}}.$$
This has been stated previously as \cite[Proposition 5.19(2)]{bfn-ii}.
In particular, the localization $\H^{\tilde{T}}_0(\Gr_T(\cc); \QQ[\beta^{\pm 1}])[\hbar^{-1}]$ is isomorphic to the rescaled Weyl algebra $\cd_{\ld{T}}^\hbar$; this is the motivation behind the terminology in \cref{def: G-diff ops}.
Note that \cref{G-mellin} simply reduces to the standard Mellin transform, which gives an equivalence between $\DMod_\hbar(\ld{T})$ and $\QCoh(\fr{t}_{\QQ\pw{\hbar}}/\Lambda)$.
\end{example}
\begin{example}\label{q quantized diffop}
Again, let $T \cong S^1$ be a torus of rank $1$ (for simplicity).
Suppose $A = \KU$, so $\GG = \GG_m$ and $\pi_0 \co_\GG \cong \Z[q^{\pm 1}]$. Then the algebra of \cref{def: G-diff ops} is the quotient of the $\Z[q^{\pm 1}]$-algebra $\Z[q^{\pm 1}]\pdb{y^{\pm 1}, x^{\pm 1}}$ by the relation $yx = qxy$. (This is also known as the ``quantum torus''.) In other words, $y$ acts as the operator $q^{x\partial_x}$ sending $f(x) \mapsto f(qx)$, so we simply have that 
$$\KU^{\tilde{T}}_0(\Gr_T(\cc)) \cong \Z[q^{\pm 1}]\pdb{q^{x\partial_x}, x^{\pm 1}}.$$
This is closely related to the $q$-Weyl algebra $\cd_q = \Z[q^{\pm 1}]\pdb{\Theta, x^{\pm 1}}/(\Theta x = x(q\Theta+1))$ for $\ld{T} = \GG_m$: indeed, since the logarithmic $q$-derivative $\Theta = x\nabla_q$ is given by the fraction $\frac{q^{x\partial_x}-1}{q-1}$, the pullback of $\cd_{\ld{T}}^\GG$ along $\GG_m-\{1\} \hookrightarrow \GG_m$ is isomorphic to the algebra $\cd_q[\frac{1}{q-1}]$.
Note that \cref{G-mellin} gives a ``$q$-Mellin transform'', i.e., an equivalence between $\LMod_{\KU^{\tilde{T}}_0(\Gr_T(\cc))}$ and $\QCoh((\GG_m)_{\Z[q^{\pm 1}]}/\Z)$, where $\Z$ acts on $(\GG_m)_{\Z[q^{\pm 1}]}$ by sending $y\mapsto qy$.
\end{example}
\begin{remark}\label{G-quantization torus}
Using \cref{equivariant-koszul}, there is an equivalence $\Loc_{T_c}(T_c; A) \simeq \LMod_{\cf_T(\Gr_T(\cc))^\vee}$. Since $\pi_0 \cf_{\tilde{T}}(\Gr_T(\cc))^\vee \cong \cd_{\ld{T}}^\GG$ is a ``quantization'' of $\pi_0 \cf_T(\Gr_T(\cc))^\vee \cong \co_{T^\ast_\GG \ld{T}}$ (i.e., an associative deformation of $T^\ast_\GG \ld{T}$ along $\GG$), and \cref{homology and quantized diffop} implies an equivalence of $\E{1}$-$A_\QQ$-algebras $\cf_{\tilde{T}}(\Gr_T(\cc))^\vee \otimes \QQ \cong \cd_{\ld{T}}^\GG \otimes_{\pi_0 A} A_\QQ$, we see that $\LMod_{\cd_{\ld{T}}^\GG} \otimes_{\pi_0 A} A_\QQ$ defines a ``quantization'' of $\Loc_{T_c}(T_c; A) \otimes \QQ$.
\end{remark}

Let us briefly outline the relationship between the algebra $\cd_{\ld{T}}^{\GG_0}$ of \cref{def: G-diff ops} and the $F$-de Rham complex of \cite{generalized-n-series}.
\begin{notation}
For the purpose of this discussion, we will assume that $T \cong S^1$ is a torus of rank $1$, so that $\ld{T} \cong \GG_m$. We will also fix an invariant differential form on the formal completion $\hat{\GG}_0$ of $\GG_0$ at the zero section, so that there is an isomorphism $\hat{\GG}_0 \cong \spf R\pw{t}$ of formal $R$-schemes. Let $F(x,y)$ denote the resulting formal group law over $R$, and define the $n$-series of $F$ by
$$[n]_F := \overbrace{F(t, F(t, F(t, \cdots F(t, t) \cdots )))}^n.$$
We will often write $x+_Fy = x+_\GG y$ to denote $F(x,y)$.
Let $\hat{\cd}_{\ld{T}}^{\GG_0}$ denote the completion of $\cd_{\ld{T}}^{\GG_0}$ at the zero section of $\cM_{0,\tilde{T}} \cong \cM_{0,T} \times \GG_0$.
\end{notation}
\begin{lemma}[Cartier duality]\label{cartier-duality}
Let $\hat{\GG}_0$ be a $1$-dimensional formal group over a commutative ring $R$, and let $\Cart(\hat{\GG}_0)$ denote its Cartier dual (see \cite[Section 3.3]{drinfeld-formal-group} for more on Cartier duals of formal groups). Then there is an equivalence of categories $\QCoh(\hat{\GG}_0) \simeq \QCoh(B\Cart(\hat{\GG}_0))$ sending the convolution tensor product on the left-hand side to the usual tensor product on the right-hand side. Under this equivalence, the functor $\QCoh(\hat{\GG}_0) \to \Mod_R$ given by restriction to the zero section is identified with the functor $\QCoh(B\Cart(\hat{\GG}_0)) \to \Mod_R$ given by pullback along the map $\spec(R) \to B\Cart(\hat{\GG}_0)$.
\end{lemma}
\begin{prop}\label{endomorphism F-de Rham}
There is a canonical action of $\hat{\cd}_{\ld{T}}^{\GG_0}$ on $(\GG_m)_{R\pw{t}} = \spf R\pw{t}[x^{\pm 1}]$ such that $R\pw{t}[x^{\pm 1}] \otimes_{\hat{\cd}_{\ld{T}}^{\GG_0}} R\pw{t}[x^{\pm 1}]$ is isomorphic to the two-term complex
$$C^\bull = (R\pw{t}[x^{\pm 1}] \to R\pw{t}[x^{\pm 1}]dx), \ x^n \mapsto [n]_F x^n dx$$
from \cite[Remark 4.3.8]{generalized-n-series}.
\end{prop}
\begin{proof}[Proof sketch]
%The argument is very similar to the proof of the well-known fact that for the standard action of $\cd_{\ld{T}}^\hbar$ on $\QQ\pw{\hbar}[x^{\pm 1}]$, the endomorphism algebra $\End_{\cd_{\ld{T}}^\hbar}(\QQ\pw{\hbar}[x^{\pm 1}])$ is isomorphic to the complex
%$$C^\bull = (\QQ\pw{\hbar}[x^{\pm 1}] \xar{\hbar x\partial_x} \QQ\pw{\hbar}[x^{\pm 1}]dx), \ x^n \mapsto n\hbar x^n dx.$$
Since $T$ is of rank $1$, there is an isomorphism $\cM_{0,T} \cong \GG_0$, and hence an isomorphism $\hat{\cM}_{0,T} \cong \hat{\AA}^1$ of formal $R$-schemes, where $\hat{\cM}_{0,T}$ denotes the completion of $\cM_{0,T}$ at the zero section. Let $y$ be a local coordinate on $\cM_{0,T}$.
Then, $\hat{\cd}_{\ld{T}}^{\GG_0}$ is isomorphic to the quotient of the associative $\hat{\co}_{\GG_0}$-algebra $\hat{\co}_{\GG_0 \times \cM_{0,T}}\pdb{x^{\pm 1}}$ subject to the relation $yx = x(y +_\GG t)$.
The $t$-adic filtration on $\hat{\cd}_{\ld{T}}^{\GG_0}$ therefore has associated graded $\gr(\hat{\cd}_{\ld{T}}^{\GG_0}) \cong \hat{\co}_{\cM_{0,T}}[x^{\pm 1}]\pw{\ol{t}}$, where $\ol{t}$ lives in weight $1$. View $R$ as a $\co_{\cM_{0,T}}$-algebra via the zero section, i.e., the augmentation $\co_{\cM_{0,T}} \to R$. Then, the action of $\gr(\hat{\cd}_{\ld{T}}^{\GG_0})$ on $R[x^{\pm 1}]\pw{\ol{t}}$ is induced by the augmentation $\hat{\co}_{\cM_{0,T}} \to R$. The isomorphism $\hat{\cM}_{0,T} \cong \hat{\AA}^1$ of formal $R$-schemes then implies an isomorphism $R \otimes_{\co_{\cM_{0,T}}} R \cong R[\epsilon]/\epsilon^2$ with $\epsilon$ in homological degree $1$.
It follows that
$$R\pw{\ol{t}}[x^{\pm 1}] \otimes_{\gr(\hat{\cd}_{\ld{T}}^{\GG_0})} R\pw{\ol{t}}[x^{\pm 1}] \simeq R\pw{\ol{t}}[x^{\pm 1}][\epsilon]/\epsilon^2,$$
where $\ol{t}$ is in weight $1$ and degree $0$, and $\epsilon$ is in weight $0$ and degree $1$.

By \cref{cartier-duality}, the $t$-adic filtration on $\hat{\cd}_{\ld{T}}^{\GG_0}$ is equivalent to the data of a $\Cart(\hat{\GG}_0)$-action on $R\pw{\ol{t}}[x^{\pm 1}] \otimes_{\gr(\hat{\cd}_{\ld{T}}^{\GG_0})} R\pw{\ol{t}}[x^{\pm 1}] \simeq R\pw{\ol{t}}[x^{\pm 1}][\epsilon]/\epsilon^2$. This in turn is equivalent to the data of a differential 
$$\nabla: R\pw{\ol{t}}[x^{\pm 1}] \to R\pw{\ol{t}}[x^{\pm 1}]\cdot \epsilon$$
satisfying a $\hat{\GG}_0$-analogue of the Leibniz rule: if\footnote{Note that $\nabla$ has to be homogeneous in the degree of the monomial in $x$, as can be seen by keeping track of the $x$-weight.} $\nabla(x^n) = f(n) x^n \epsilon$ for some $f(n)\in R\pw{t}$, then $f(n+m) = f(n) +_\GG f(m)$.
It therefore suffices to determine $\nabla(x)$; but the relation $yx = x(y +_\GG t)$ forces $\nabla(x) = tx\epsilon$. This implies that 
$$\nabla(x^n) = (\overbrace{t +_\GG \cdots +_\GG t}^n) x^n \epsilon = [n]_F x^n \epsilon,$$
as desired.
%Now, the differential $\nabla$ is determined by the formula:
%$$\nabla: x^n \mapsto (yx^n -_\GG x^n y)\epsilon.$$
%Since $yx = x(y +_\GG t)$, a simple induction shows that 
%$$y x^n = x^n(y+_\GG \overbrace{t +_\GG \cdots +_\GG t}^n) = x^n(y +_\GG [n]_F).$$
%It follows that $yx^n -_\GG x^n y = x^n [n]_F$, so $\nabla(x^n) = [n]_F x^n \epsilon$, as desired. \todo hmm
\end{proof}
\begin{example}
When $\GG_0 = \hat{\GG}_a$ over\footnote{Of course, one can work over $\Z$ too; we just chose $\QQ$ to continue with \cref{ordinary quantized diffop}.} $\QQ$, the complex $C^\bull$ is
$$C^\bull = (\QQ\pw{\hbar}[x^{\pm 1}] \to \QQ\pw{\hbar}[x^{\pm 1}]dx), \ x^n \mapsto n\hbar x^n dx.$$
Indeed, since $yx = x(y+\hbar)$, we have $yx^n = x^n(y+n\hbar)$; since $t = \hbar$ in this case, we have $x^n\mapsto n\hbar x^n \epsilon$. This is evidently a $\hbar$-rescaling of the classical de Rham complex of $\GG_m$.

When $\GG_0 = \GG_m$ over $\Z$, the complex $C^\bull$ is
$$C^\bull = (\Z\pw{q-1}[x^{\pm 1}] \to \Z\pw{q-1}[x^{\pm 1}]dx), \ x^n \mapsto (q^n-1) x^n dx.$$
Indeed, since $yx = x(qy)$, we have $yx^n = x^n (q^n y)$, and hence 
$$(y-1)x^n = x^n(q^n y - 1) = x^n((y-1) +_F (q^n-1)),$$
where $F(z,w) = z + w + zw$ is the multiplicative formal group law; since $t = q-1$ in this case, we have $x^n \mapsto (q^n-1) x^n \epsilon$. The complex $C^\bull$ is a $(q-1)$-rescaling of the $q$-de Rham complex of $\GG_m$ from \cite{scholze-q-def}.
\end{example}
\begin{remark}
The complex of \cref{endomorphism F-de Rham} is not quite the $F$-de Rham complex of \cite[Definition 4.3.6]{generalized-n-series}; rather, if $\eta_t$ denotes the d\'ecalage functor of \cite{berthelot-ogus} with respect to the ideal $(t)\subseteq R\pw{t}$, the $F$-de Rham complex is given by the d\'ecalage $\eta_t C^\bull$. In particular, the complex of \cref{endomorphism F-de Rham} is isomorphic to the $F$-de Rham complex after inverting $t$. One can modify the algebra $\cd_{\ld{T}}^{\GG_0}$ of \cref{def: G-diff ops} (by performing a noncommutative analogue of an affine blowup/deformation to the normal cone\footnote{For instance, in the case of \cref{ordinary quantized diffop}, this procedure simply adjoins the fraction $\frac{y}{\hbar}$; in the case of \cref{q quantized diffop}, this procedure simply adjoins the fraction $\frac{y-1}{q-1}$.}) such that the relative tensor product as in \cref{endomorphism F-de Rham} is the $F$-de Rham complex itself. Since it is not needed for this article, we will not describe this modification here.
\end{remark}
\begin{remark}\label{rmk: koszul duality LT}
\cref{endomorphism F-de Rham} says that $\hat{\cd}_{\ld{T}}^{\GG_0}$ is Koszul dual to the complex $C^\bull$. Forthcoming work of Arpon Raksit shows that the d\'ecalage $\eta_t C^\bull$ can be recovered from the ``even filtration'' (in the sense of \cite{even-filtr}) on the periodic cyclic homology $\HP(\tau_{\geq 0} A[x^{\pm 1}]/\tau_{\geq 0} A)$. See also the discussion in \cite[Section 3.3]{thh-xn}.
Using similar techniques, one can show that $C^\bull$ can be recovered from the even filtration on the negative cyclic homology $\HC^-(A[x^{\pm 1}]/A) = \HH(A[x^{\pm 1}]/A)^{hS^1}$.

Recalling that $T = S^1$, this $\Eoo$-$A$-algebra is simply $\HC^-(A[\Omega T]/A)$. The Hochschild homology $\HH(A[\Omega T]/A) \simeq A \otimes \THH(S[\Omega T])$ is $S^1$-equivariantly equivalent to the $A$-chains $C_\ast(\cL T; A)$ on the free loop space of $T$. (For a reference, see \cite[Corollary IV.3.3]{nikolaus-scholze}.) The $A$-chains $A[\cL T]$ is $S^1$-equivariantly Koszul dual\footnote{This Koszul duality essentially stems from the (nonequivariant) decomposition $\cL T \simeq T \times \Omega T$.} to $A[\Omega T]^{hT}$; this can be identified as a completion of $\cf_T(\Omega T)^\vee$ at the zero section of $\cM_T$. In other words, $\HC^-(A[\Omega T]/A)$ is Koszul dual to the completion of $\cf_{T\times S^1_\rot}(\Omega T)^\vee$ at the zero section of $\cM_T \times \GG$. This is the topological source of the Koszul duality of \cref{endomorphism F-de Rham}.
\end{remark}
\begin{remark}
In \cref{rmk: koszul duality LT}, we mentioned that the Koszul duality between $\GG$-differential operators and the $F$-de Rham complex manifests in topology as the Koszul duality between $\cf_{T \times S^1_\rot}(\Omega T)^\vee$ and $\HC^-(A[\Omega T]/A)$. There is clearly nothing special about $T$ in this Koszul duality: given a sufficiently robust theory of $G$-equivariant $A$-(co)homology (see the discussion surrounding \cref{nonabelian-equiv-cochains}), there is also a Koszul duality between $\cf_{G \times S^1_\rot}(\Omega G)^\vee$ and $\HC^-(A[\Omega G]/A) = A[\cL G]^{hS^1}$. When $A = \cc[\beta^{\pm 1}]$, \cite[Theorem 3]{bf-derived-satake} states that $\cf_{G \times S^1_\rot}(\Omega G)^\vee$ can be identified with (the $2$-periodification of) the bi-Whittaker reduction $\ld{N}^-\backslash_\chi \cd_{\ld{G}}/_\chi \ld{N}^-$. Using the results of this article, it is also possible to compute $A[\cL G]^{hS^1}$ in this manner, at least if we assume that small primes are inverted: the zeroth graded piece of the ``even filtration'' on $A[\cL G]^{hS^1}$ looks like the $2$-periodification of the $F$-de Rham complex of $Z_f(\ld{B})$ for a chosen principal nilpotent element $f\in \ld{\g}$. We plan to explain this in future work.
\end{remark}
\newpage

\section{The coherent side}
\subsection{Langlands duality over $\QQ[\beta^{\pm 1}]$}\label{sec: Q intersection}

We now turn to the coherent side of the geometric Satake equivalence. For general $\GG$, it is not obvious what the Langlands dual algebraic stack should be; we will discuss this in \cref{section-G-loops}. As a warmup, we will focus only on $\QQ[\beta^{\pm 1}]$ in this section (this is more for pedagogical purposes than originality).
\begin{definition}[(Additive) Kostant slice]\label{additive kostant slice}
Let $G$ be a connected reductive group over $\cc$, and fix the rest of notation as in \cref{group-notation}. Fix a principal nilpotent element $e\in \fr{n}$, and let $(e,f,h)$ be the associated $\sl_2$-triple in $\g$. Let $\g^e$ be the centralizer (so $\g = \g^e \oplus [e,\g]$), and let $\cS := f + \g^e\subseteq \g^\reg$ be the Kostant slice. The composite $f + \g^e \to \g \to \g\mmod G \cong \fr{t}\mmod W$ is an isomorphism, by \cite{kostant-lie-group-reps}.

Let $\tilde{\g} = \fr{b} \times_B G$ be the Grothendieck-Springer resolution, so that $\tilde{\g}/G \simeq \fr{b}/B$. We will often work with $\tilde{\g}^\ast$ instead, defined as $\fr{b}^\ast \times_B G$. There is a map $\tilde{\chi}: \tilde{\g} \to \fr{t}$ which sends a pair $(x\in \Ad_g(\fr{b}))$ to the inverse image under the isomorphism $\fr{t} \to \fr{b} \to \fr{b}/\fr{n}$ of the image of $g^{-1} x\in \fr{b}$. Let $\tilde{\cS}$ denote the fiber product $\cS\times_\g \tilde{\g}$, so that $\tilde{\cS} \subseteq \tilde{\g}^\reg = \g^\reg \times_\g \tilde{\g}$. Then, Kostant's result on the Kostant slice implies formally that the composite $\tilde{\cS} \to \tilde{\g} \xar{\tilde{\chi}} \fr{t}$ is an isomorphism. We will often abusively write the inclusion of $\tilde{\cS}$ as a map $\kappa: \fr{t} \to \tilde{\g}$.

In fact, we will only care about the composite $\fr{t} \to \tilde{\g} \to \tilde{\g}/G$ below, so we will also denote it by $\kappa$. If we identify $\tilde{\g}/G \cong \fr{b}/B$, then the map $\kappa$ admits a simple description: it is the composite $\fr{t} \to \fr{b} \to \fr{b}/B$ which sends $x\mapsto f+x$. This is proved, for instance, in \cite[Proposition 19]{kostant-lie-group-reps}, where it is shown that there is a unique map $\mu: f+\fr{t} \to N$ such that $\Ad_{\exp(\mu(x))}(x) \in f + \g^e$; this further implies that the image of any $x\in \fr{t}$ under the map $\fr{t} \to \fr{t}\mmod W \xar{\kappa} \g$ can be identified with $\Ad_{\exp(\mu(x+f))}(x+f)$.
\end{definition}

Fix a nondegenerate invariant bilinear form on $\g$, to identify $\g$ with $\g^\ast$. The first main result of this section is the following; it is essentially equivalent to \cite[Proposition 2.8]{bfm} and the rationalization of \cite[Theorem 6.1]{homology-langlands}.
\begin{theorem}\label{once-looped-satake}
Let $G$ be a connected and simply-connected semisimple algebraic group over $\cc$. Let $A$ be an $\Eoo$-$\QQ[\beta^{\pm 1}]$-algebra, and let $\GG = \GG_a$ (so $\cM_T$ is the affine space $\fr{t}[2]$ over $A$). 
View $\ld{\fr{t}}^\ast$, $\ld{\fr{n}}$, $\ld{\g}$, and $\ld{B}$ as schemes over $\QQ$.
Then $\QCoh(\ld{\fr{t}}^\ast)$ admits the structure of a module over $\IndCoh((\tilde{\ld{\cN}} \times_{\ld{\g}} \{0\})/\ld{G})$, where the fiber product is (always) derived, such that there is an equivalence
$$\End_{\IndCoh((\tilde{\ld{\cN}} \times_{\ld{\g}} \{0\})/\ld{G})}(\QCoh(\ld{\fr{t}}^\ast)) \otimes_\QQ \pi_0 A \simeq \LMod_{\pi_0 C_\ast^T(\Gr_G(\cc); A)} = \Loc^\gr_{T_c}(G_c; A).$$
\end{theorem}
\begin{remark}\label{iwahori-satake}
Recall from \cite{abg-iwahori-satake} that there is an Iwahori-Satake equivalence $\IndCoh((\tilde{\ld{\cN}} \times_{\ld{\g}} \{0\})/\ld{G}) \simeq \Shv(\Gr_G)^I$ over $\cc$, where the right-hand side is normalized appropriately. One should therefore regard \cref{once-looped-satake} as a bar construction of the restriction of this equivalence (lifted from $\cc$ to $\QQ$) to the regular locus, and more optimistically as a first step towards an alternative proof. See also \cref{regular-satake-special-cases} for the equivalence resulting from ``undoing'' the bar construction.
\end{remark}
We now turn to the proof of \cref{once-looped-satake}. For the next two results, we only work on one side of Langlands duality, so we drop the ``check''s for notational simplicity. Note that $(\tilde{\ld{\cN}} \times_{\ld{\g}} \{0\})/\ld{G} \cong (\ld{\fr{n}} \times_{\ld{\g}} \{0\})/\ld{B}$; it will be more convenient to work with the latter description.
\begin{lemma}\label{kd-springer}
There is a Koszul duality equivalence $\QCoh(\tilde{\g}^\ast[2]/G) \simeq \IndCoh((\fr{n} \times_{\g} \{0\})/B)$.
\end{lemma}
%\begin{proof}
%Observe that $\fr{n} \times_{\g} \{0\} \cong \spec \Sym(\fr{n}^\ast \times_{\g^\ast} 0)$. Since $\g = \fr{n} \oplus \fr{b}^-$, we see that $\fr{n}^\ast \times_{\g^\ast} 0$ is equivalent as a $B$-equivariant module to $\fr{b}[-1]$. (Here, we use the invariant bilinear form on $\g$.) It follows that $\fr{n} \times_{\g} \{0\} \cong \spec \Sym(\fr{b}[-1])$, so that $\IndCoh((\fr{n} \times_{\g} \{0\})/B) \simeq \IndCoh(\Sym(\fr{b}[-1]))^B$. By Koszul duality, $\IndCoh(\Sym(\fr{b}^\ast[-1]))^B \simeq \QCoh(\fr{b}^\ast/B)$; this implies the claim, since $\fr{b}^\ast/B \simeq \tilde{\g}^\ast/G$.
%\end{proof}
We will give two proofs of the following fact.
\begin{prop}[{Variant of \cite[Proposition 2.8]{bfm}}]\label{bfm-self-intersect}
Work over a field $k$ of characteristic $0$, and view $\QCoh(\fr{t}^\ast)$ as a $\QCoh(\tilde{\g}^\ast/G)$-module via the Kostant slice $\kappa: \fr{t}^\ast \to \tilde{\g}^\ast$. Then there is an equivalence
$\End_{\QCoh(\tilde{\g}^\ast/G)}(\QCoh(\fr{t}^\ast)) \simeq \QCoh((T^\ast T)^\bl)$.
\end{prop}
\begin{proof}[First proof of \cref{bfm-self-intersect}]
We may identify $\End_{\QCoh(\tilde{\g}^\ast/G)}(\QCoh(\fr{t}^\ast))$ with $\QCoh(\fr{t}^\ast \times_{\tilde{\g}^\ast/G} \fr{t}^\ast)$. We will show, in fact, that there is a Cartesian square
\begin{equation}\label{intersection-kostant}
    \xymatrix{
    (T^\ast T)^\bl \ar[r] \ar[d] & \fr{t}^\ast \ar[d]^-\kappa\\
    \fr{t}^\ast \ar[r]_-\kappa & \tilde{\g}^\ast/G \simeq \fr{b}^\ast/B.
    }
\end{equation}
This is an analogue of \cite[Proposition 2.2.1]{ngo-ihes} and \cite[Proposition 2.8]{bfm}.
(Note that since $\fr{t}^\ast \to \tilde{\g}^\ast$ lands in the open locus $\tilde{\g}^{\ast, \reg}$, it does not matter whether we intersect $\fr{t}^\ast$ with itself in $\tilde{\g}^\ast/G$ or in $\tilde{\g}^{\ast, \reg}/G$; indeed, the intersection $\tilde{\g}^{\ast, \reg} \times_{\tilde{\g}^\ast} \tilde{\g}^{\ast, \reg}$ is just $\tilde{\g}^{\ast, \reg}$.) In what follows, it will be convenient (notationally) to use the chosen nondegenerate invariant bilinear form on $\g$ to identify $\fr{b}^\ast$ with the opposite Borel $\fr{b}^-$ and $N$ with its opposite unipotent, and then to flip the role of $\fr{b}$ and $\fr{b}^-$, etc. 

Recall that the Kostant slice $\cS\subseteq \g$ is transverse to the regular $G$-orbits, and intersects each orbit exactly once; this implies that the image of the map $\kappa: \fr{t} \to \tilde{\g}$ is transverse to the regular $G$-orbits on $\tilde{\g}$, and intersects each orbit exactly once. In particular, if $C$ denotes the locally closed subvariety of $\tilde{\g} \times G$ consisting of pairs $(x,g)$ with $x\in \tilde{\g}^\reg$ and $\Ad_g(x) = x$, then $C\mmod G = \fr{t} \times_{\tilde{\g}/G} \fr{t}$ (so we may assume without loss of generality that $x\in \fr{t}$). To compute $C\mmod G$, one can reduce to the case when $G$ has semisimple rank $1$ by the argument of \cite[Section 4.3]{bfm}. To work out this case, we will assume $G = \SL_2, \PGL_2$.

There are ``two'' ways to compute in these cases; we will describe both, because each has its own conceptual advantages when generalizing to the multiplicative case (for instance). First, we present the argument which is essentially present in \cite{bfm}; for this, we will assume $G = \SL_2$. The Grothendieck-Springer resolution $\tilde{\g} = T^\ast(\AA^2-\{0\})/\GG_m$ is the total space of $\co(-1) \oplus \co(-1)$ over $\PP^1$; we will think of a point in $\tilde{\g}$ as a pair $(x\in \sl_2, \ell\subseteq \cc^2)$ such that $x$ preserves $\ell$. The Kostant slice $\kappa:\fr{t} \cong \AA^1 \to \tilde{\g}$ is the map sending $\lambda \in \AA^1$ to the pair $(x, \ell)$ with $x = \begin{psmallmatrix}
0 & \lambda^2 \\
1 & 0
\end{psmallmatrix}$ and $\ell = [\lambda: 1]$. Indeed, this is essentially immediate from the requirement that the following diagram commutes:
$$\xymatrix{
\AA^1 \cong \fr{t} \ar[r]^-\kappa \ar[d]_-{\lambda \mapsto \lambda^2} & \tilde{\sl}_2 \ar[d]\\
\AA^1 \cong \fr{t}\mmod W \ar[r]^-\kappa_-{\lambda\mapsto \begin{psmallmatrix}
0 & \lambda \\
1 & 0
\end{psmallmatrix}} & \sl_2.
}$$
Moreover, the $\SL_2$-action on $\tilde{\g}$ sends $g\in \SL_2$ and $(x,\ell)$ to $(\Ad_g(x), g\ell)$. If $g = \begin{psmallmatrix}
a & b \\
c & d
\end{psmallmatrix}$, we compute that
$$\Ad_g\begin{pmatrix}
0 & \lambda^2 \\
1 & 0
\end{pmatrix} = \begin{pmatrix}
bd-ac\lambda^2 & (a\lambda)^2 - b^2 \\
d^2 - (c\lambda)^2 & ac\lambda^2 - bd
\end{pmatrix}, \ g\cdot [\lambda: 1] = [a\lambda + b: c\lambda + d].$$
From this, we see that $\Ad_g(x) = x$ if and only if $a = d$ and $b = c\lambda^2$, in which case $g$ also fixes $[\lambda: 1]$. In other words, $g = \begin{psmallmatrix}
a & c\lambda^2 \\
c & a
\end{psmallmatrix}$ with $a,c\in k$; in order for $\det(g) = 1$, we need $a^2-c^2\lambda^2=1$. When $\lambda \neq 0$, both $x$ and $g$ are diagonalized by the matrix $\tfrac{1}{2}\begin{psmallmatrix}
1 & -1 \\
-\lambda^{-1} & -\lambda^{-1}
\end{psmallmatrix}\in \SL_2$: the diagonalization of $x$ is $\begin{psmallmatrix}
\lambda & 0 \\
0 & \lambda^{-1}
\end{psmallmatrix}$, and the diagonalization of $g$ is $\begin{psmallmatrix}
t & 0 \\
0 & w
\end{psmallmatrix}$ where $2a = t+w$ and $2\lambda c = t-w$. Since we have $\det(g) = a^2 - (c\lambda)^2 = 1$, we have $w = t^{-1}$. This shows that if $k$ is not of characteristic $2$, then $\fr{t} \times_{\tilde{\sl}_2/\SL_2} \fr{t} \cong \spec k[\lambda, t^{\pm 1}, \tfrac{t-t^{-1}}{\lambda}]$.

The ``second'' way to reach this calculation (still with $G = \SL_2$) is to use the fact that $\kappa: \fr{t} \to \tilde{\g}/G$ can be identified with the composite $\fr{t} \to \fr{b} \to \fr{b}/B$ sending $x\mapsto f+x$. Then, $\fr{t} \times_{\fr{b}/B} \fr{t}$ is isomorphic to the subvariety of $\fr{t} \times B$ consisting of pairs $(x,g)$ with $x\in \fr{t}$ and $\Ad_g(x+f) = x+f$. If $g = \begin{psmallmatrix}
a & 0 \\
b & a^{-1}
\end{psmallmatrix}\in B$, then
$$\Ad_g \begin{pmatrix}
x & 0 \\
1 & -x
\end{pmatrix} = \begin{pmatrix}
x & 0 \\
2a^{-1}bx + a^{-2} & -x
\end{pmatrix}.$$
Therefore, $\Ad_g(x+f) = x+f$ if and only if 
$$2a^{-1}bx + a^{-2} = 1,$$
which forces $b = \tfrac{a-a^{-1}}{2x}$. This implies that $\fr{t} \times_{\fr{b}/B} \fr{t}$ is isomorphic to $\spec k[x, a^{\pm 1}, \tfrac{a-a^{-1}}{x}]$, as desired.

We will now do the calculation with $G = \PGL_2$ via the second method. Again, $\fr{t} \times_{\fr{b}/B} \fr{t}$ is isomorphic to the subvariety of $\fr{t} \times B$ consisting of pairs $(x,g)$ with $x\in \fr{t}$ (identified with the matrix $\begin{psmallmatrix}
x & 0 \\
0 & 0
\end{psmallmatrix} \in \gl_2$) and $\Ad_g(x+f) = x+f$. If $g = \begin{psmallmatrix}
a & 0 \\
b & 1
\end{psmallmatrix}\in B$, then 
$$\Ad_g \begin{pmatrix}
x & 0 \\
1 & 0
\end{pmatrix} = \begin{pmatrix}
x & 0 \\
(bx + 1) a^{-1} & 0
\end{pmatrix}.$$
Therefore, $\Ad_g(x+f) = x+f$ if and only if 
$$(bx + 1) a^{-1} = 1,$$
which forces $b = \tfrac{a-1}{x}$. This implies that $\fr{t} \times_{\fr{b}/B} \fr{t}$ is isomorphic to $\spec k[x, a^{\pm 1}, \tfrac{1-a}{x}]$, as desired.
\end{proof}
\begin{proof}[Second proof of \cref{bfm-self-intersect}]
As in the first proof of \cref{bfm-self-intersect}, it will be convenient to use the chosen nondegenerate invariant bilinear form on $\g$ to identify $\fr{b}^\ast$ with the opposite Borel $\fr{b}^-$ and $N$ with its opposite unipotent, and then to flip the role of $\fr{b}$ and $\fr{b}^-$, etc. We will prove the following variant of \cref{bfm-self-intersect}, which in turn implies the desired result: view $\QCoh(\fr{t}^\ast\mmod W)$ as a $\QCoh({\g}^\ast/G)$-module via the Kostant slice. Then there is an equivalence
$\End_{\QCoh({\g}^\ast/G)}(\QCoh(\fr{t}^\ast\mmod W)) \simeq \QCoh((T^\ast T)^\bl\mmod W)$.

Let $\chi$ be a nondegenerate character on $\fr{n}^-$. The $N^-$-action on $G$ via conjugation induces a Hamiltonian $N^-$-action on $T^\ast G$; let $N^- {}_\chi\backslash (T^\ast G)/_\chi N^-$ denote the bi-Whittaker reduction of $T^\ast G$ with respect to this $N^-$-action at the character $\chi\in \fr{n}^{-,\ast}$. Then $(T^\ast T)^\bl\mmod W \cong N^- {}_\chi\backslash (T^\ast G)/_\chi N^-$; see \cite[Theorem 6.3]{teleman-icm}, for instance. There is a Morita equivalence between $\QCoh(\g^\ast/G)$ and $\QCoh(T^\ast G)$ (equipped with the convolution monoidal structure); under this equivalence, the $\QCoh(\g^\ast/G)$-module $\QCoh(\g^\ast/_\chi N^-)$ is sent to the $\QCoh(T^\ast G)$-module $\QCoh((T^\ast G)/_\chi N^-)$. We conclude the series of equivalences:
\begin{align*}
    \QCoh((T^\ast T)^\bl\mmod W) & \simeq \QCoh(N^- {}_\chi\backslash (T^\ast G)/_\chi N^-) \\
    & \simeq \End_{\QCoh(T^\ast G)} (\QCoh((T^\ast G)/_\chi N^-))\\
    & \simeq \End_{\QCoh(\g^\ast/G)}(\QCoh(\g^\ast/_\chi N^-)).
\end{align*}
However, Kostant's theorem identifies $\g^\ast/_\chi N^-$ with $\fr{t}^\ast\mmod W$ (viewed as a substack of $\g^\ast/G$ via the Kostant slice), which finishes the proof.
\end{proof}
\begin{proof}[Proof of \cref{once-looped-satake}]
By \cref{t-homology-grg}, we have $\H_0^T(\Gr_G(\cc); A) = \pi_0 \cf_T(\Gr_G(\cc))^\vee \cong \co_{(T^\ast T)^\bl}$. It follows that $\LMod_{\H_\ast^T(\Gr_G(\cc); A)} \simeq \QCoh((T^\ast \ld{T})^\bl_A)$. Since $\End_{\IndCoh((\tilde{\ld{\cN}} \times_{\ld{\g}} \{0\})/\ld{G})}(\QCoh(\ld{\fr{t}}^\ast)) \simeq \QCoh((T^\ast \ld{T})^\bl)$ by \cref{kd-springer} and \cref{bfm-self-intersect}, we conclude the desired result.
\end{proof}
\begin{remark}
So far, we have not emphasized the role of Whittaker reduction in the above story (except for the second proof of \cref{bfm-self-intersect}). However, we take a moment to describe this briefly, since it is a key aspect of Langlands duality. Recall that a theorem of Kostant's gives an isomorphism $(f+\fr{b})/N \cong \cS = f + \g^e$. In terms of Whittaker reduction, this says that $\cS \cong \g/_\chi N^-$. Since \cref{bfm-self-intersect} is concerned with $\tilde{\g}$ instead of $\g$, we need a slight variant of this statement. Namely, recall the map $\pi: \tilde{\g} \to \g$, let $\mu: \g \to \fr{n}$ be the moment map for the adjoint $N$-action on $\g$, and let $\tilde{\mu}$ denote the composite $\tilde{\g} \to \g \to \fr{n}$. Then $\mu^{-1}(f)$ is the variety $f+\fr{b}$, so that $\tilde{\mu}^{-1}(f)$ is the subscheme of $\tilde{\g}$ spanned by those pairs $(\fr{b}', y\in \fr{b}'\cap (f+\fr{b}))$. Kostant's result implies that there is an isomorphism $\tilde{\mu}^{-1}(f)/N^- \xar{\sim} \fr{t}$. 
%(This map is easily seen to be surjective: an inverse map is given by sending $x\in \fr{t}$ to the pair $(\fr{b}^-, f+x\in \fr{b}^-)$; injectivity is equivalent to the claim that the space of pairs $(\fr{b}', y\in \fr{n}'\cap (f+\fr{b}))$ is an $N$-torsor, which is more difficult to prove.) 
Whittaker reduction is a key aspect of the Langlands-dual side of \cref{once-looped-satake}: it is needed to even define the action of $\QCoh(\tilde{\g}^\ast/G)$ on $\QCoh(\fr{t}^\ast)$.
\end{remark}

\begin{example}\label{witt-example}
Note that \cref{once-looped-satake} implies that $\H_0(\Omega G_c; \QQ[\beta^{\pm 1}])$ can be identified with the ring of functions on the centralizer $Z_f(\ld{G})$ of a regular nilpotent element $f\in \ld{\g}$ over $\QQ$. In type $A$ at least, one can directly check that there is such an isomorphism. (Exactly the same argument works in the K-theoretic and elliptic cases, too; in the K-theoretic case, one instead considers the centralizer of a regular \textit{unipotent} element $f\in \ld{G}$.) For instance, if $\ld{G} = \SL_n$, the centralizer $Z_f(\ld{G})$ is the direct product of $\mu_n$ with a connected (commutative) unipotent group $U_n$. If $(x_1, \cdots, x_{n-1})$ is a point in $U_n$ (corresponding to the element in $Z_f(\SL_n)$ given by the $n\times n$-matrix whose $j$th row is $(0, \cdots, 0, 1, x_1, \cdots, x_{n-j})$), the group operation is given by
$$(x_1, \cdots, x_{n-1}) \cdot (y_1, \cdots, y_{n-1}) = (x_1 + y_1, \cdots, x_{n-1} + x_{n-2} y_1 + \cdots + x_1 y_{n-2} + y_{n-1}).$$
The group scheme $U_n$ is isomorphic over $\QQ$ to $\GG_a^{\times n-1}$, via Newton's identities for the transformation law for expressing the power sum symmetric polynomials in terms of the elementary symmetric polynomials. For instance, the isomorphism between $U_6 \subseteq Z_f(\SL_6)$ and $\GG_a^{\times 5}$ is given by the map
\begin{align}
    (x_1, \cdots, x_5) & \mapsto (x_1, x_1^2-2x_2, x_1^3-3x_1 x_2+3x_3, x_1^4 + 2x_2^2 - 4x_4-4x_2x_1^2+4x_1x_3, \nonumber\\
    & \ \ \ x_1^5-5x_1^3 x_2 + 5x_1^2 x_3 - 5x_1 (x_4 - x_2^2) - 5x_2 x_3 + 5x_5). \label{eq: power sum}
\end{align}
In general, the transformation can be determined by extracting the coefficient of $(-t)^n/n$ in the power series $\log\left(\sum_{j\geq 0} x_j (-t)^j\right)$.

On the other hand, $G_c$ is a maximal compact subgroup of $\PGL_n(\cc)$, and there is a homotopy equivalence $\Omega \PGL_n(\cc) \simeq \Z/n \times \Omega \SU(n)$, so that
$$\H_0(\Omega \PGL_n(\cc); \QQ[\beta^{\pm 1}]) \simeq \QQ[x^{\pm 1}]/(x^n-1) \otimes_\Z \H_0(\Omega \SU(n); \Z[\beta^{\pm 1}]).$$
Under Langlands duality, the $\mu_n$ factor in $Z_f(\SL_n)$ comes from the first tensor factor. Similarly, $\spec \H_0(\Omega \SU(n); \Z[\beta^{\pm 1}])$ is a connected unipotent group scheme: for instance, there is a Bott periodicity equivalence $\Omega \SU \simeq \BU$ (where $\SU = \colim_{n\to\infty} \SU(n)$), so $\spec \H_0(\Omega \SU; \Z[\beta^{\pm 1}])$ can be identified with the ring of functions over the big Witt ring scheme $\WW$ over $\Z$. This group scheme is unipotent over $\Z$, and the ghost components define an isomorphism to $\prod_{\Z_{\geq 0}} \GG_a$ upon rationalization (see \cite[Theorem II.6.7]{serre-local-fields} for a textbook reference). The group scheme $\spec \H_0(\Omega \SU(n); \Z[\beta^{\pm 1}])$ is a quotient of $\WW$ (hence is unipotent): in fact, it is isomorphic to the group scheme $\WW_{n-1}$ of big Witt vectors of length $n-1$. Since this is rationally isomorphic to $\GG_a^{\times n-1}$, we see that
$$\spec \H_0(\Omega \PGL_n(\cc); \QQ[\beta^{\pm 1}]) \cong \mu_n \times \WW_{n-1} \cong \mu_n \times \GG_a^{\times n-1} \cong Z_f(\SL_n),$$
as desired. Note, however, that the isomorphism $\WW_{n-1} \cong U_n\subseteq Z_f(\SL_n)$ is somewhat tricky to write down in coordinates. As an example, using the formula for the ghost components in the big Witt vectors, it is easy to see that the formula \cref{eq: power sum} implies that the isomorphism $Z_f(\SL_6) \supseteq U_6 \xar{\sim} \WW_5$ sends $(x_1, \cdots, x_5)$ to the Witt vector
$$(x_1, \cdots, x_5) \mapsto (x_1, -x_2, x_3-x_1x_2, x_1x_3-x_2x_1^2-x_4, x_5-x_1^3 x_2 + x_1^2 x_3 - x_1 (x_4 - x_2^2) - x_2 x_3).$$
\end{example}
\begin{remark}
One special feature of rational homology which sets it apart from K-theory or elliptic cohomology is that it can be de-periodified. On the Langlands-dual side, this equips the relevant geometric objects with a $\GG_m$-action, i.e., with a grading. Continuing \cref{witt-example}, there is still an isomorphism 
$$\H_\ast(\Omega \PGL_n(\cc); \QQ) \simeq \QQ[x^{\pm 1}]/(x^n-1) \otimes_\Z \H_\ast(\Omega \SU(n); \Z),$$
and there is still an isomorphism $\spec \H_\ast(\Omega \SU(n); \Z) \cong \WW_{n-1}$. Here, the grading on $\H_\ast(\Omega \SU(n); \Z)$ by half the homological degree corresponds to the $\GG_m$-action on $\WW_{n-1}$ defined as follows: if we view $\WW_{n-1}(R) = 1+R[t]/t^n \subseteq (R[t]/t^n)^\times$, the coordinate $t$ is given weight $-1$.
This defines a grading on $Z_f(\SL_n)$, which can also be described directly in general as follows (see \cite{kostant-lie-group-reps}). The element $2\rho=\sum_{\alpha\in \Phi^+} \alpha\in \bX^\ast(T) \cong \bX_\ast(\ld{T})$ defines a homomorphism $2\rho: \GG_m \to \ld{T}$, which defines a $\GG_m$-action on $\ld{\g}$. This $\GG_m$-action stabilizes the Kostant section $f + \ld{\g}^e$, and hence defines a $\GG_m$-action on $Z_e(\ld{G})$; this is the grading on $\co_{Z_e(\ld{G})}$ corresponding to half the homological grading on $\H_\ast(\Omega G_c; \QQ)$.
\end{remark}

\begin{remark}
In \cite{bfm}, the following analogue of \cref{intersection-kostant} is established (over $\cc$, but this does not affect the statement): there is a Cartesian square
\begin{equation}\label{tmmodw-bfm}
    \xymatrix{
    (T^\ast \ld{T})^\bl \mmod W \ar[r] \ar[d] & \fr{t}\mmod W \ar[d]^-\kappa\\
    \fr{t}\mmod W \ar[r]_-\kappa & \ld{\g}/\ld{G},
    }
\end{equation}
where the top-left corner can be identified with $\spec \pi_0 C^G_\ast(\Gr_G(\cc); \QQ)$. We can take the fiber product of \cref{intersection-kostant} with itself over \cref{tmmodw-bfm} to obtain a Cartesian square
\begin{equation}\label{springer-kostant}
    \xymatrix{
    (T^\ast \ld{T})^\bl \times_{(T^\ast \ld{T})^\bl \mmod W} (T^\ast \ld{T})^\bl \ar[r] \ar[d] & \fr{t} \times_{\fr{t}\mmod W} \fr{t} \ar[d]^-\kappa\\
    \fr{t} \times_{\fr{t}\mmod W} \fr{t} \ar[r]_-\kappa & (\tilde{\ld{\g}} \times_{\ld{\g}} \tilde{\ld{\g}})/\ld{G}.
    }
\end{equation}
Using \cref{once-looped-satake} and the above discussion, one can use \cref{springer-kostant} to show that $\End_{\QCoh((\tilde{\ld{\g}} \times_{\ld{\g}} \tilde{\ld{\g}})/\ld{G})}(\QCoh(\fr{t} \times_{\fr{t}\mmod W} \fr{t}))$ can be identified with $\LMod_{\pi_0 C^T_\ast(\Fl_G(\cc); \QQ[\beta^{\pm 1}])}$. This can be viewed as a ``once-looped'' version of Bezrukavnikov's equivalence from \cite{bezrukavnikov-two-geometric}.
%$T\backslash LG/T \simeq T\backslash LG/G \times_{G\backslash LG / G} G\backslash LG/T$. In 
\end{remark}

One can quantize \cref{once-looped-satake} as follows.
\begin{definition}
Following \cite{univ-cat-o}, define the (Langlands dual) \textit{universal category} $\ld{\co}_\hbar^\univ$ to be $\DMod_\hbar(\ld{G}/\ld{N})^{(\ld{G}\times \ld{T}, w)} \simeq U_\hbar(\ld{\g})\modc^{\ld{N}, (\ld{T}, w)}$. The $\infty$-category $\ld{\co}_\hbar^\univ$ is a quantization of $\QCoh(\ld{\fr{b}}^-/\ld{B}^-)$, since there are isomorphisms
$$\ld{\fr{b}}^-/\ld{B}^- \cong \tilde{\ld{\g}}/\ld{G} \cong \ld{T}\backslash T^\ast(\ld{G}/\ld{N})/\ld{G}.$$
\end{definition}
\begin{theorem}\label{looped-quantized-abg}
Let $A$ be an $\Eoo$-$\cc[\beta^{\pm 1}]$-algebra, and let $G$ be a connected and simply-connected semisimple algebraic group or a torus over $\cc$. Then there is a Kostant functor $\ld{\co}^\univ_\hbar \to \QCoh(\ld{\fr{t}}^\ast \times \AA^1_\hbar)$ and a left $A\pw{\hbar}$-linear equivalence 
$$\LMod_{\pi_0 C^{\tilde{T}}_\ast(\Gr_G(\cc); A)} \simeq \End_{\ld{\co}^\univ_\hbar}(\QCoh(\ld{\fr{t}}^\ast \times \AA^1_\hbar)).$$
\end{theorem}
\begin{proof}[Proof sketch; compare to the second proof of \cref{bfm-self-intersect}]
We will assume $A = \cc[\beta^{\pm 1}]$, so that $\pi_0 C^{\tilde{T}}_\ast(\Gr_G(\cc); A)$ is a $2$-periodification of $\pi_\ast C^{\tilde{T}}_\ast(\Gr_G(\cc); \cc)$. Let $\cH(\tilde{\ld{\fr{t}}}^\ast, \tilde{W}^\aff)$ be the nil-Hecke algebra associated to $\tilde{\ld{\fr{t}}}^\ast \cong \ld{\fr{t}}^\ast \oplus \cc\alpha_0$, and let $e = \tfrac{1}{\# W}\sum_{w\in W} w\in \QQ[W]$ be the symmetrizer idempotent. Using \cref{cohomology-grg}, one can then show that $\H^{\tilde{T}}_\ast(\Gr_G(\cc); \cc)$ is isomorphic to $\co_{\ld{\fr{t}}^\ast} \otimes_{\co_{\ld{\fr{t}}^\ast \mmod W}} e\cH(\tilde{\ld{\fr{t}}}^\ast, \tilde{W}^\aff)e$, where the loop rotation parameter $\hbar$ corresponds to the affine root $\alpha_0$; see \cite{kostant-kumar}. This implies that $\LMod_{\pi_0 C^{\tilde{T}}_\ast(\Gr_G(\cc); A)}$ can be identified with $\QCoh(\ld{\fr{t}}^\ast) \otimes_{\QCoh(\ld{\fr{t}}^\ast \mmod W)} \LMod_{e\cH(\tilde{\ld{\fr{t}}}^\ast, \tilde{W}^\aff)e}$.

We now construct the Kostant functor $\kappa_\hbar: \ld{\co}^\univ_\hbar \to \QCoh(\ld{\fr{t}}^\ast \times \AA^1_\hbar)$. Recall that the Kostant functor $\HC_\hbar(\ld{G}) \to \QCoh(\ld{\fr{t}}^\ast\mmod W \times \AA^1_\hbar)$ is given by the composite
$$\HC_\hbar(\ld{G}) = \DMod_\hbar(\ld{G})^{(\ld{G}\times \ld{G}, w)} \to \DMod_\hbar(\ld{G})^{(\ld{G}, w)} \to \DMod_\hbar(\ld{N}^-\backslash_\chi \ld{G})^{(\ld{G}, w)}.$$
However, the final term is equivalent to $U_\hbar(\ld{\g})\modc^{(\ld{N}^-, \chi)}$, which in turn can be identified with $\QCoh(\ld{\fr{t}}^\ast\mmod W \times \AA^1_\hbar)$ by the Skryabin equivalence (see the appendix of \cite{premet}).
Similarly, the desired Kostant functor on $\ld{\co}^\univ_\hbar$ is also given by Whittaker averaging: there is a composite
$$\ld{\co}^\univ_\hbar = \DMod_\hbar(\ld{G}/\ld{N})^{(\ld{G}\times \ld{T}, w)} \to \DMod_\hbar(\ld{G}/\ld{N})^{(\ld{T}, w)} \xar{\mathrm{Av}_!^\chi} \DMod_\hbar(\ld{N}^-\backslash_\chi \ld{G}/\ld{N})^{(\ld{T}, w)}.$$
However, the final term is equivalent by a standard argument to $\DMod_\hbar(\ld{T})^{(\ld{T}, w)} \simeq \QCoh(\ld{\fr{t}}^\ast \times \AA^1_\hbar)$. Note that by construction, the following diagram commutes:
$$\xymatrix{
\HC_\hbar(\ld{G}) \ar[d] \ar[r] & \QCoh(\ld{\fr{t}}^\ast\mmod W \times \AA^1_\hbar) \ar[d] \\
\ld{\co}^\univ_\hbar \ar[r] & \QCoh(\ld{\fr{t}}^\ast \times \AA^1_\hbar).
}$$
Here, the horizontal maps are given by the Kostant functors.
%There is an isomorphism $\Gamma(\ld{G}/\ld{N}; \cd_{\ld{G}/\ld{N}})^{\ld{G} \times \ld{T}} \cong U(\ld{\fr{t}}) \cong \co_{\ld{\fr{t}}^\ast}$, where $\cd_{\ld{G}/\ld{N}}$ denotes the sheaf of differential operators on $\ld{G}/\ld{N}$. This is essentially part of the Beilinson-Bernstein theorem; for example, \cite[Theorem 2.6.5]{milicic} shows that there is an $\ld{G}$-equivariant isomorphism $\Gamma(\ld{G}/\ld{N}; \cd^\hbar_{\ld{G}/\ld{N}})^{\ld{T}} \cong U_\hbar(\ld{\g}) \otimes_{Z(\ld{\g})} U(\ld{\fr{t}})$, so the claim follows from the fact that $U_\hbar(\ld{\g})^\ld{G} \cong Z(\ld{\g}) \otimes_\cc \cc\pw{\hbar}$ and that $U(\ld{\fr{t}})$ is a flat $Z(\ld{\g})$-module.

To finish, we need to show that $\QCoh(\ld{\fr{t}}^\ast) \otimes_{\QCoh(\ld{\fr{t}}^\ast \mmod W)} \LMod_{e\cH(\tilde{\ld{\fr{t}}}^\ast, \tilde{W}^\aff)e}$ is equivalent to $\End_{\ld{\co}^\univ_\hbar}(\QCoh(\ld{\fr{t}}^\ast \times \AA^1_\hbar))$. 
%We will denote $\cd_\ld{G} \otimes_{Z(\ld{\g})} U(\ld{\fr{t}})$ by $\tilde{\cd_\ld{G}}$, so that $\tilde{\cd_\ld{G}}$ is a quantization of $\tilde{\ld{\g}} \times \ld{G}$.  Kostant's isomorphism $U(\ld{\g})/_\chi \ld{N}^- \cong Z(\ld{\g})$ from \cite{kostant-whittaker} implies that
%$$\End_{\ld{\co}^\univ_\hbar}(\QCoh(\ld{\fr{t}}^\ast \times \AA^1_\hbar)) \simeq \LMod_{\ld{N}^- {}_\chi\backslash \tilde{\cd^\hbar_\ld{G}} /_\chi \ld{N}^-}.$$
There is an equivalence 
$$\QCoh(\ld{\fr{t}}^\ast \times \AA^1_\hbar) \simeq \ld{\co}^\univ_\hbar \otimes_{\HC_\hbar(\ld{G})} \QCoh(\ld{\fr{t}}^\ast \mmod W \times \AA^1_\hbar),$$
so that
$$\End_{\ld{\co}^\univ_\hbar}(\QCoh(\ld{\fr{t}}^\ast \times \AA^1_\hbar)) \simeq \QCoh(\ld{\fr{t}}^\ast) \otimes_{\QCoh(\ld{\fr{t}}^\ast \mmod W)} \End_{\HC_\hbar(\ld{G})}(\QCoh(\ld{\fr{t}}^\ast\mmod W \times \AA^1_\hbar)).$$
The desired claim now follows from the observation that there is an isomorphism $\ld{N}^- {}_\chi\backslash {\cd_\ld{G}} /_\chi \ld{N}^- \cong e\cH(\tilde{\ld{\fr{t}}}^\ast, \tilde{W}^\aff)e$ given by \cite[Theorem 8.1.2]{ginzburg-whittaker}, which gives an equivalence between $\End_{\HC_\hbar(\ld{G})}(\QCoh(\ld{\fr{t}}^\ast\mmod W \times \AA^1_\hbar))$ and $\LMod_{e\cH(\tilde{\ld{\fr{t}}}^\ast, \tilde{W}^\aff)e}$.
\end{proof}
\begin{remark}\label{idk-reference}
In fact, one can quantize the result of \cite{abg-iwahori-satake}: namely, there is an equivalence
\begin{equation}\label{gm-rot-abg}
    \DMod_{I \rtimes \GG_m^\rot}(\Gr_G)\simeq \ld{\co}_\hbar^\univ.
\end{equation}
We do not have a reference for this fact when $G$ lives over $\cc$, but it can be deduced using the equivalence of \cite[Section 1.6]{ginzburg-riche} and the arguments of \cite{abg-iwahori-satake}. I am grateful to Tom Gannon for discussions about this equivalence. (If $G$ lives over $\ol{\FF_p}$ and $\DMod$ is replaced with $\ol{\QQ_\ell}$-adic sheaves, then \cref{gm-rot-abg} can be deduced from \cite[Theorem 84]{dodd-thesis} and the parabolic-Whittaker duality for the affine Grassmannian from \cite{bezrukavnikov-yun}.) Just as with \cref{once-looped-satake}, \cref{looped-quantized-abg} may be regarded as a ``once-looped'' version of \cref{gm-rot-abg}. 
One can similarly show that there is an equivalence
\begin{equation}\label{gm-rot-bez}
    \DMod_{I \rtimes \GG_m^\rot}(\Fl_G)\simeq \DMod_\hbar(\ld{N} \backslash \ld{G}/\ld{N})^{(\ld{T} \times \ld{T}, \weak)},
\end{equation}
which quantizes Bezrukavnikov's equivalence from \cite{bezrukavnikov-two-geometric}. Note that $\ld{T} \backslash T^\ast(\ld{N} \backslash \ld{G}/\ld{N}) / \ld{T}$ is isomorphic to $(\tilde{\ld{\g}} \times_{\ld{\g}} \tilde{\ld{\g}})/\ld{G}$, so that this equivalence does indeed quantize Bezrukavnikov's equivalence
$$\DMod_I(\Fl_G) \simeq \QCoh((\tilde{\ld{\g}}[2] \times_{\ld{\g}[2]} \tilde{\ld{\g}}[2])/\ld{G}).$$
\end{remark}
\begin{remark}
If $G$ is a connected and simply-connected semisimple algebraic group or a torus over $\cc$, let $\HC_\hbar(\ld{G})$ denote the $\infty$-category $U_\hbar(\ld\g)\modc^{\ld{G},w}$. Then $\Gamma(\ld{G}; \cd_{\ld{G}})^{\ld{G}\times \ld{G}} \cong U(\ld{\g})^{\ld{G}} \cong \Sym(\ld{\fr{t}})^W$. An argument very similar to \cref{looped-quantized-abg} proves that there is a Kostant functor $\HC_\hbar(\ld{G}) \to \QCoh(\ld{\fr{t}}^\ast\mmod W \times \AA^1_\hbar)$ and a left $A\pw{\hbar}$-linear equivalence 
\begin{equation}\label{looped-HC}
    \LMod_{\pi_0 C^{G \times S^1_\rot}_\ast(\Gr_G(\cc); A)} \simeq \End_{\HC_\hbar(\ld{G})}(\QCoh(\ld{\fr{t}}^\ast\mmod W \times \AA^1_\hbar)).
\end{equation}

This is closely related to \cite{ginzburg-whittaker}, \cite{lonergan-fourier}, and \cite[Theorem 1.4]{gannon-thesis}. Let $\ld{\fr{t}}\mmod \tilde{W}^\aff$ be the coarse quotient as defined in \cite{gannon-tmmodw}.
Then, the aforementioned articles provide a monoidal ``Fourier transform'' equivalence $\DMod(\ld{N}^- {}_\chi\backslash \ld{G} /_\chi \ld{N}^-) \simeq \IndCoh(\ld{\fr{t}}\mmod \tilde{W}^\aff)$. Note that combined with the preceding discussion, we obtain an equivalence
\begin{equation}\label{622-a}
    \IndCoh(\ld{\fr{t}}\mmod \tilde{W}^\aff) \simeq \End_{\HC(\ld{G})}(\QCoh(\ld{\fr{t}}^\ast\mmod W)).
\end{equation}
There is also an equivalence (see \cite{lonergan-fourier})
$$\End_{\Shv_{G\times S^1_\rot}(\Gr_G; \cc)}(\QCoh(\ld{\fr{t}}^\ast\mmod W)) \simeq \LMod_{\H^{G \times S^1_\rot}_\ast(\Gr_G(\cc); \cc)} \simeq \IndCoh(\ld{\fr{t}}\mmod \tilde{W}^\aff),$$
and its relationship to \cref{622-a} is explained by the derived loop-rotation equivariant geometric Satake equivalence of \cite{bf-derived-satake}.
\end{remark}
In the same way, we have the following result. We expect that the techniques of \cite{bzgo} can be used to show that this implies the equivalences conjectured in \cite[Remark 6.22]{gannon-thesis}.
\begin{prop}
We have:
\begin{align}
    \IndCoh(\ld{\fr{t}}\mmod \tilde{W}^\aff) & \simeq \End_{\DMod(\ld{N}\backslash \ld{G}/\ld{N})^{(\ld{T}\times \ld{T}, w)}}(\QCoh(\ld{\fr{t}}^\ast)), \label{622-b}
    %\IndCoh(\ld{\fr{t}}\mmod \tilde{W}^\aff) & \simeq \Fun^L_{\ld{\g}\modc^\ld{N}}(\QCoh(\ld{\fr{t}}^\ast/\Lambda^\vee), \QCoh(\ld{\fr{t}}^\ast\mmod W)). \label{622-c}
\end{align}
\end{prop}
\begin{proof}
The equivalence \cref{622-b} is proved via:
\begin{align*}
    \IndCoh(\ld{\fr{t}}\mmod \tilde{W}^\aff) & \simeq \DMod(\ld{N}^- {}_\chi\backslash \ld{G} /_\chi \ld{N}^-)\\
    & \simeq \End_{\DMod(\ld{G})}(\DMod(\ld{G} /_\chi \ld{N}^-))\\
    & \simeq \End_{\DMod(\ld{N} \backslash \ld{G}/\ld{N})^{(\ld{T}\times \ld{T}, w)}}(\DMod(\ld{N} \backslash \ld{G} /_\chi \ld{N}^-)^{\ld{T},w}) \\
    & \simeq \End_{\DMod(\ld{N} \backslash \ld{G}/\ld{N})^{(\ld{T}\times \ld{T}, w)}}(\DMod(\ld{T})^{\ld{T},w}) \\
    & \simeq \End_{\DMod(\ld{N} \backslash \ld{G}/\ld{N})^{(\ld{T}\times \ld{T}, w)}}(\QCoh(\ld{\fr{t}}^\ast)).
\end{align*}
The third equivalence above uses \cite[Corollary 1.2]{bzgo}, and the fourth equivalence above is the well-known fact that restriction to the big cell in $\ld{G}$ defines an equivalence $\DMod(\ld{N} \backslash \ld{G} /_\chi \ld{N}^-) \xar{\sim} \DMod(\ld{N} \backslash \ld{B} /_\chi \ld{N}^-) \simeq \DMod(\ld{T})$; see \cite[Proposition 1.8]{gannon-thesis}, for instance. 
%The proof of \cref{622-c} is similar:
%\begin{align*}
%    \IndCoh(\ld{\fr{t}}\mmod \tilde{W}^\aff) & \simeq \End_{\DMod(\ld{G})}(\DMod(\ld{G} /_\chi \ld{N}^-))\\
%    %& \simeq \DMod(\ld{G} /_\chi \ld{N}^-) \otimes_{\DMod(\ld{G})} \DMod(\ld{G} /_\chi \ld{N}^-)\\
%    %& \simeq \DMod(\ld{G} /_\chi \ld{N}^-)^{\ld{N}} \otimes_{\DMod(\ld{G})^{\ld{N}, (\ld{G}, w)}} \DMod(\ld{G} /_\chi \ld{N}^-)^{\ld{G},w}\\
%    & \simeq \Fun^L_{\DMod(\ld{G})^{\ld{N}, (\ld{G}, w)}} (\DMod(\ld{G} /_\chi \ld{N}^-)^{\ld{N}}, \DMod(\ld{G} /_\chi \ld{N}^-)^{\ld{G},w}) \\
%    & \simeq \Fun^L_{\ld{\g}\modc^{\ld{N}}}(\DMod(\ld{T}), \ld{\g}\modc^{(\ld{N}^-, \chi)})\\
%    %& \simeq \QCoh(\ld{\fr{t}}^\ast/\Lambda^\vee) \otimes_{\ld{\g}\modc^{\ld{N}}} \QCoh(\ld{\fr{t}}^\ast\mmod W)\\
%    & \simeq \Fun^L_{\ld{\g}\modc^\ld{N}}(\QCoh(\ld{\fr{t}}^\ast/\Lambda^\vee), \QCoh(\ld{\fr{t}}^\ast\mmod W)).
%\end{align*}
%The penultimate line uses two observations. The first is that the Mellin transform gives an equivalence $\QCoh(\ld{\fr{t}}^\ast/\Lambda^\vee) \simeq \DMod(\ld{T})$, which can be identified with $\DMod(\ld{N} \backslash \ld{G} /_\chi \ld{N}^-)$. The second observation is that there is an equivalence $\ld{\g}\modc^{(\ld{N}^-, \chi)} \simeq \QCoh(\ld{\fr{t}}^\ast\mmod W)$, given by the Skryabin equivalence (see the appendix of \cite{premet}).
\end{proof}
\begin{remark}
Since $\ld{\g}/\ld{G} = \Map(B\GG_a, B\ld{G})$, the canonical orientation of $B\GG_a$ defines a $1$-shifted symplectic structure on $\ld{\g}/\ld{G}$ via \cite[Theorem 2.5]{ptvv}. The quasi-classical limit (i.e., $\hbar\to 0$) of the quantized equivalence \cref{looped-HC} gives the following strengthening of \cref{once-looped-satake}. The Kostant slice $\ld{\fr{t}}\mmod W \to \ld{\g}/\ld{G}$ is a Lagrangian morphism by \cite[Proposition 4.18]{safronov-geom-quant}, so that the self-intersection $\ld{\fr{t}}\mmod W \times_{\ld{\g}/\ld{G}} \ld{\fr{t}}\mmod W$ admits the structure of a symplectic stack (using \cite[Theorem 2.9]{ptvv}). Since this fiber product is isomorphic to $(T^\ast \ld{T})^\bl\mmod W$ by \cref{tmmodw-bfm}, we obtain a Poisson bracket on $\co_{(T^\ast \ld{T})^\bl\mmod W} \cong \H^G_\ast(\Gr_G(\cc); \cc)$. This structure can be seen topologically, at least after a completion: using one of the main results of \cite{klang}, the Borel-equivariant analogue/completion $C_\ast(\Gr_G(\cc); \cc)^{hG_c}$ of $C^G_\ast(\Gr_G(\cc); \cc)$ can be identified with the $\E{3}$-center of $C_\ast(\Gr_G(\cc); \cc)$. This defines a $2$-shifted Poisson bracket on $\H_\ast(\Gr_G(\cc)^{hG_c}; \cc)$, which can be identified after $2$-periodification with the ($0$-shifted) Poisson bracket on $\co_{(T^\ast \ld{T})^\bl\mmod W}$.
\end{remark}
\subsection{Rationalized Langlands duality over $\KU$}

Let us now discuss the $K$-theoretic analogue of \cref{once-looped-satake}. First, we discuss the story where the Kostant slice from \cref{sec: Q intersection} is replaced by the ``Steinberg slice''; below, we will discuss the story where the Kostant slice from \cref{sec: Q intersection} is replaced by a multiplicative version of the Kostant slice.
\begin{definition}[Steinberg slice]
Let $G$ be a simply-connected semisimple algebraic group or a torus. Given $w\in W$, let $N_w = N \cap w^{-1} N^- w$, so that $N_w = \prod_{\alpha\in \Phi_w} U_\alpha$, where $\Phi_w$ is the set of roots made negative by $w$. Let $w = \prod_{\alpha\in \Delta} s_\alpha\in W$ be a Coxeter element, and let $\dot{w}$ be a lift of $w$ to $N_G(T)$. Define the Steinberg slice $\Sigma = \dot{w} N_w\subseteq G$. Then \cite{steinberg-slice} proved/stated that the composite $\Sigma \to G \to G\mmod G \cong T\mmod W$ is an isomorphism. 
Let $\tilde{G} = B \times_{B} G$ be the multiplicative Grothendieck-Springer resolution, so that $\tilde{G}/G = B/B$. There is a map $\tilde{G} \to T$ sending a pair $x\in gBg^{-1}$ to $x\pmod{g[B,B]g^{-1}}$. Let $\tilde{\Sigma}$ denote the fiber product $\Sigma\times_G \tilde{G}$, so that the composite $\tilde{\Sigma} \to \tilde{G} \to T$ is an isomorphism. We will denote the inclusion of $\tilde{\Sigma}$ by $\sigma: T \to \tilde{G}$.
\end{definition}
\begin{prop}\label{steinberg-kthy-once-looped-satake}
Let $G$ be a connected and simply-connected semisimple algebraic group or a torus over $\cc$. Let $A$ be an $\Eoo$-$\KU$-algebra, and let $\GG = \GG_m$ (so $\cM_T$ is the torus $T$ over $A$). View $\tilde{\ld{G}}$ as a scheme over $\QQ$. If $\QCoh(\ld{T})$ is viewed as a module over $\QCoh(\tilde{\ld{G}}/\ld{G})$ via $\sigma^\ast$, then there is an equivalence
$$\End_{\QCoh(\tilde{\ld{G}}/\ld{G})}(\QCoh(\ld{T})) \otimes_\QQ \pi_0 A_\QQ \simeq \LMod_{\pi_0 C_\ast^T(\Gr_G(\cc); A)} \otimes \QQ.$$
\end{prop}
\begin{proof}
We will assume without loss of generality that $A = \KU$. By \cref{t-homology-grg}, there is an equivalence $\pi_0 C_\ast^T(\Gr_G(\cc); A) = \pi_0 \cf_T(\Gr_G(\cc))^\vee \simeq \co_{(T^\ast_{\GG_m} \ld{T})^\bl}$. It follows that $\LMod_{\pi_0 C_\ast^T(\Gr_G(\cc); A)} \simeq \QCoh((T^\ast_{\GG_m} \ld{T})^\bl)$. 
%Note that since $G$ is assumed to be simply-connected, $\ld{G}$ is of adjoint type.
It therefore suffices to show that over a field $k$ of characteristic zero, there is an equivalence $\End_{\QCoh(\tilde{\ld{G}}/\ld{G})}(\QCoh(\ld{T})) \simeq \QCoh((T^\ast_{\GG_m} \ld{T})^\bl)$.

As in \cref{bfm-self-intersect}, there is an equivalence $\End_{\QCoh(\tilde{\ld{G}}/\ld{G})}(\QCoh(\ld{T})) \simeq \QCoh(\ld{T} \times_{\tilde{\ld{G}}/\ld{G}} \ld{T})$, so it suffices to establish the existence of a Cartesian square
\begin{equation}\label{kthy-intersection-kostant}
    \xymatrix{
    (T^\ast_{\GG_m} \ld{T})^\bl \ar[r] \ar[d] & \ld{T} \ar[d]^-\sigma \\
    \ld{T} \ar[r]_-\sigma & \tilde{\ld{G}}/\ld{G}.
    }
\end{equation}
Again, one can reduce to the case when $\ld{G}$ has semisimple rank $1$ by the argument of \cite[Section 4.3]{bfm}.
Every split reductive group of semisimple rank $1$ is isomorphic to the product of a split torus with $\SL_2$, $\PGL_2$, or $\GL_2$.
We will illustrate the calculation when $\ld{G} = \SL_2$, and describe an alternative simpler calculation in the case $\ld{G} = \PGL_2$ later.

%%Let us return to the argument for $\ld{G} = \SL_2$. 
View a point in $\tilde{\ld{G}}$ as a pair $(x\in \SL_2, \ell\subseteq \cc^2)$ such that $x$ preserves $\ell$. The Steinberg slice $\sigma:\ld{T} \cong \GG_m \to \tilde{\SL}_2$ is the map sending $\lambda \in \GG_m$ to the pair $(x, \ell)$ with 
$$x = \begin{pmatrix}
\lambda + \lambda^{-1} & -1 \\
1 & 0
\end{pmatrix}, \ \ell = \left[\lambda: 1\right].$$
Note that this indeed a well-defined point in $\tilde{\SL}_2$, since one can check that $x$ preserves $\ell$. This calculation of $\sigma(\lambda)$ is essentially immediate from the requirement that the following diagram commutes:
$$\xymatrix{
\GG_m \cong \ld{T} \ar[r]^-\sigma \ar[d]_-{\lambda \mapsto \lambda + \lambda^{-1}} & \tilde{\SL}_2 \ar[d]\\
\AA^1 \cong \ld{T}\mmod W \ar[r]^-\sigma_-{\lambda\mapsto \begin{psmallmatrix}
\lambda & -1 \\
1 & 0
\end{psmallmatrix}} & \SL_2.
}$$
Moreover, the $\SL_2$-action on $\tilde{\SL}_2$ sends $g\in \SL_2$ and $(x,\ell)$ to $(\Ad_g(x), g\ell)$. If $g = \begin{psmallmatrix}
a & b \\
c & d
\end{psmallmatrix}$, one can directly compute that $g$ commutes with $\begin{psmallmatrix}
\lambda + \lambda^{-1} & -1 \\
1 & 0
\end{psmallmatrix}$ if and only if $a = c(\lambda + \lambda^{-1}) + d$ and $b=-c$. Therefore, $g = \begin{psmallmatrix}
c(\lambda + \lambda^{-1}) + d & -c \\
c & d
\end{psmallmatrix}$ for $c,d\in k$. In order for $\det(g) = 1$, we need 
$$c^2 + d^2 + cd(\lambda + \lambda^{-1}) = 1.$$
As long as $\lambda\neq \pm 1$, both $x$ and $g$ can be simultaneously diagonalized by $\begin{psmallmatrix}
\lambda & \lambda^{-1} \\
1 & 1
\end{psmallmatrix}$: the diagonalization of $x$ is $\begin{psmallmatrix}
\lambda & 0 \\
0 & \lambda^{-1}
\end{psmallmatrix}$, and the diagonalization of $g$ is $\begin{psmallmatrix}
c\lambda + d & 0 \\
0 & c\lambda^{-1} + d
\end{psmallmatrix}$. If $t = c\lambda+d$, then $c\lambda^{-1}+d = t^{-1}$ by the above determinant relation. We also have that $a = t - \tfrac{\lambda(t-t^{-1})}{\lambda - \lambda^{-1}}$ and $c = \tfrac{t-t^{-1}}{\lambda - \lambda^{-1}}$. This shows that $\GG_m \times_{\tilde{\SL}_2/\SL_2} \GG_m \cong \spec k[\lambda^{\pm 1}, t^{\pm 1}, \tfrac{t-t^{-1}}{\lambda - \lambda^{-1}}]$ (even if if $k$ is of characteristic $2$).
\end{proof}
An alternative argument for the Cartesian square \cref{kthy-intersection-kostant} can be given using the multiplicative Kostant slice, which gives a \textit{different} section of the map $G \to G\mmod G$. The multiplicative Kostant slice is significantly more accessible, and the resulting \cref{kthy-once-looped-satake} is what we will generalizing below to other cohomology theories.
\begin{definition}[Multiplicative Kostant slice]
Let $e\in \fr{n}$ be a principal nilpotent element. Then the map $\GG_a \to G$ corresponding to $e$ factors through the map $\GG_a = B \to \SL_2$; we will denote the image of the standard generator $\begin{psmallmatrix}
1 & 0\\
1 & 1
\end{psmallmatrix}\in B^-$ under the map $\SL_2 \to G$ by $f\in G$. Let $Z_G(e)^\circ$ be the connected component of the identity in the centralizer of $e$ in $G$. Define the \textit{multiplicative Kostant slice} $\cS_\mu$ by $Z_G(e)^\circ \cdot f \subseteq G$. Since $G$ is assumed to be simply-connected, the composite $\cS_\mu \to G \to G\mmod G \cong T\mmod W$ is an isomorphism. We will often denote the inclusion of the Kostant slice by $\kappa: T\mmod W \to G$.
Let $\tilde{\cS}_\mu$ denote the fiber product $\tilde{\cS}_\mu \times_G \tilde{G}$, so that the composite $\tilde{\cS}_\mu \to \tilde{G} \to T$ is an isomorphism; we will denote the inclusion of $\tilde{\cS}_\mu$ as a map $\kappa: \tilde{\cS}_\mu \cong T \to \tilde{G}$.

As with the additive Kostant slice, we will only care about the composite $T \to \tilde{G} \to \tilde{G}/G$ below, so we will also denote it by $\kappa$. If we identify $\tilde{G}/G \cong B/B$, then the map $\kappa$ admits a simple description: it is the composite $T \to B \to B/B$ which sends $x\mapsto xf$. Just as in \cite[Proposition 19]{kostant-lie-group-reps}, there is a unique map $\mu: T\cdot f \to N$ such that $\Ad_{\mu(x)}(x) \in Z_G(e)^\circ\cdot f$, and the image of any $x\in T$ under the map $T \to T\mmod W \xar{\kappa} G$ can be identified with $\Ad_{\mu(xf)}(xf)$.
\end{definition}
\begin{remark}
The main result of \cite{friedman-morgan-sections} states that any two sections of the map $G \to T\mmod W$ are conjugate. For instance, the multiplicative Kostant section ${T}\mmod W \cong \AA^1 \to \SL_2$ sending $\lambda \mapsto \begin{psmallmatrix}
\lambda-1 & \lambda-2 \\
1 & 1
\end{psmallmatrix}$ and the Steinberg section ${T}\mmod W \cong \AA^1 \to \SL_2$ sending $\lambda \mapsto \begin{psmallmatrix}
\lambda & -1 \\
1 & 0
\end{psmallmatrix}$ are conjugated into each other by the matrix $\begin{psmallmatrix}
1 & -1\\
0 & 1
\end{psmallmatrix}$.
\end{remark}
\begin{theorem}\label{kthy-once-looped-satake}
Let $G$ be a connected and simply-connected semisimple algebraic group or a torus over $\cc$. Let $A$ be an $\Eoo$-$\KU$-algebra, and let $\GG = \GG_m$ (so $\cM_T$ is the torus $T$ over $A$). View $\tilde{\ld{G}}$ as a scheme over $\QQ$. If $\QCoh(\ld{T})$ is viewed as a module over $\QCoh(\tilde{\ld{G}}/\ld{G})$ via $\kappa^\ast$, then there is an equivalence
$$\End_{\QCoh(\tilde{\ld{G}}/\ld{G})}(\QCoh(\ld{T})) \otimes_\QQ \pi_0 A_\QQ \simeq \LMod_{\pi_0 C_\ast^T(\Gr_G(\cc); A)} \otimes \QQ.$$
\end{theorem}
\begin{proof}
Following the argument of \cref{steinberg-kthy-once-looped-satake}, we only need to prove the Cartesian-ness of \cref{kthy-intersection-kostant}, where the map $\ld{T} \to \tilde{\ld{G}}/\ld{G}$ is chosen to be the multiplicative Kostant slice instead of the Steinberg slice. Again, we only review the calculation for $\ld{G} = \SL_2$; this was done in \cite{bfm}. For convenience, we will drop the ``check''s. As before, there are ``two'' ways to compute in the case $G = \SL_2$. First, we describe the argument essentially present in \cite{bfm} (which works over a base field of characteristic not $2$). If $\lambda\in \GG_m$, we denote $\lambda + \lambda^{-1}\in \AA^1$ by $f(\lambda)$. The Kostant slice $\kappa:\ld{T} \cong \GG_m \to \tilde{\SL}_2$ is the map sending $\lambda \in \GG_m$ to the pair $(x, \ell)$ with
$$x = \begin{pmatrix}
f(\lambda)-1 & f(\lambda)-2 \\
1 & 1
\end{pmatrix}, \ \ell = \left[\lambda-1: 1\right].$$
Note that this indeed a well-defined point in $\tilde{\SL}_2$, since one can check that $x$ preserves $\ell$: the key point is the conic relation
$$2\lambda = f(\lambda)-\sqrt{f(\lambda)^2-4}.$$
Indeed, this calculation of $\kappa(\lambda)$ is essentially immediate from the requirement that the following diagram commutes:
$$\xymatrix{
\GG_m \cong \ld{T} \ar[r]^-\kappa \ar[d]_-{\lambda \mapsto f(\lambda)} & \tilde{\SL}_2 \ar[d]\\
\AA^1 \cong \ld{T}\mmod W \ar[r]^-\kappa_-{\lambda\mapsto \begin{psmallmatrix}
\lambda-1 & \lambda-2 \\
1 & 1
\end{psmallmatrix}} & \SL_2.
}$$
Moreover, the $\SL_2$-action on $\tilde{\SL}_2$ sends $g\in \SL_2$ and $(x,\ell)$ to $(\Ad_g(x), g\ell)$. If $g = \begin{psmallmatrix}
a & b \\
c & d
\end{psmallmatrix}$, we directly compute that $\Ad_g(x) = x$ if and only if $b = c(f(\lambda) - 2)$ and $a-d = (f(\lambda) - 2)c$, in which case $g$ also preserves $\ell$. Therefore, $g = \begin{psmallmatrix}
(f(\lambda) - 2)c + d & (f(\lambda)-2)c \\
c & d
\end{psmallmatrix}$ for $c,d\in k$. In order for $\det(g) = 1$, we need 
$$d^2 + c(f(\lambda)-2)(d-c) = 1.$$
Both $x$ and $g$ can be simultaneously diagonalized (if $f(\lambda) \neq \pm 2$); note that $\lambda+\lambda^{-1}$ is an eigenvalue of $x$. If $t$ is an eigenvalue of $g$, then we have $c = \tfrac{t-t^{-1}}{\lambda - \lambda^{-1}}$ and $d = \tfrac{t^2\lambda + 1}{t(\lambda+1)}$.
When $k$ is not of characteristic $2$, this shows that $\GG_m \times_{\tilde{\SL}_2/\SL_2} \GG_m \cong k[\lambda^{\pm 1}, t^{\pm 1}, \tfrac{t-t^{-1}}{\lambda - \lambda^{-1}}]$, as desired.

For the ``second'' method of calculation when $G = \SL_2$ (which works in arbitary characteristic), we use the fact that $\kappa: T \to \tilde{G}/G$ can be identified with the composite $T \to B \to B/B$ sending $x\mapsto xf$. Then, $T \times_{B/B} T$ is isomorphic to the subvariety of $T \times B$ consisting of pairs $(x,g)$ with $x\in T$ (identified with the matrix $\begin{psmallmatrix}
x & 0\\
0 & x^{-1}
\end{psmallmatrix}$) and $\Ad_g(xf) = xf$. Note that $xf$ is the matrix $\begin{psmallmatrix}
x & 0 \\
x^{-1} & x^{-1}
\end{psmallmatrix}$. If $g = \begin{psmallmatrix}
a & 0 \\
b & a^{-1}
\end{psmallmatrix}\in B$, then 
$$\Ad_g \begin{pmatrix}
x & 0 \\
x^{-1} & x^{-1}
\end{pmatrix} = \begin{pmatrix}
x & 0 \\
a^{-2} x^{-1} + ba^{-1}(x-x^{-1}) & x^{-1}
\end{pmatrix}.$$
Therefore, $\Ad_g(xf) = xf$ if and only if 
$$a^{-2} x^{-1} + ba^{-1}(x-x^{-1}) = x^{-1},$$
which forces $b = \tfrac{a-a^{-1}}{x^2-1}$. This implies that $T \times_{B/B} T$ is isomorphic to $\spec k[x^{\pm 1}, a^{\pm 1}, \tfrac{a-a^{-1}}{x^2-1}]$, as desired.

We can also run this argument in the case $G = \PGL_2$ (again in arbitary characteristic). Again, $T \times_{B/B} T$ is isomorphic to the subvariety of $T \times B$ consisting of pairs $(x,g)$ with $x\in T$ (identified with the matrix $\begin{psmallmatrix}
x & 0\\
0 & 1
\end{psmallmatrix}$) and $\Ad_g(xf) = xf$. Note that $xf$ is the matrix $\begin{psmallmatrix}
x & 0 \\
1 & 1
\end{psmallmatrix}$. If $g = \begin{psmallmatrix}
a & 0 \\
b & 1
\end{psmallmatrix}\in B$, then
$$\Ad_g \begin{pmatrix}
x & 0 \\
1 & 1
\end{pmatrix} = \begin{pmatrix}
x & 0 \\
ba^{-1}(x-1) + a^{-1} & 1
\end{pmatrix}.$$
Therefore, $\Ad_g(xf) = xf$ if and only if 
$$ba^{-1}(x-1) + a^{-1} = 1,$$
which forces $b = \frac{a-1}{x-1}$. This implies that $T \times_{B/B} T$ is isomorphic to $\spec k[x^{\pm 1}, a^{\pm 1}, \tfrac{a-1}{x-1}]$, as desired.
\end{proof}
\begin{observe}
In the second argument for the Cartesian square \cref{kthy-intersection-kostant}, we may replace the symbol $\lambda$ by the symbol $e^\lambda$; then, $e^\lambda-1$ is the exponential of the multiplicative formal group law. In particular, the defining equation for the line $\ell$ in the cases of $\GG = \GG_a, \GG_m$ precisely describes the exponential for the formal completion $\hat{\GG}$ of $\GG$ at the identity.
\end{observe}

\begin{remark}
In \cite{bfm}, the following analogue of \cref{kthy-intersection-kostant} is established (over $\cc$, but this does not affect the statement): there is a Cartesian square
\begin{equation}\label{kthy-tmmodw-bfm}
    \xymatrix{
    (T^\ast_{\GG_m} \ld{T})^\bl \mmod W \ar[r] \ar[d] & \ld{T}\mmod W \ar[d]^-\kappa\\
    \ld{T}\mmod W \ar[r]_-\kappa & \ld{G}/\ld{G},
    }
\end{equation}
where the top-left corner can be identified with $\spec C^G_0(\Gr_G(\cc); \KU) \otimes \QQ$. We can take the fiber product of \cref{kthy-intersection-kostant} with itself over \cref{kthy-tmmodw-bfm} to obtain a Cartesian square
\begin{equation}\label{kthy-springer-kostant}
    \xymatrix{
    (T^\ast_{\GG_m} \ld{T})^\bl \times_{(T^\ast_{\GG_m} \ld{T})^\bl \mmod W} (T^\ast_{\GG_m} \ld{T})^\bl \ar[r] \ar[d] & \ld{T} \times_{\ld{T}\mmod W} \ld{T} \ar[d]^-\kappa\\
    \ld{T} \times_{\ld{T}\mmod W} \ld{T} \ar[r]_-\kappa & (\tilde{\ld{G}} \times_{\ld{G}} \tilde{\ld{G}})/\ld{G}.
    }
\end{equation}
Using \cref{kthy-once-looped-satake} and the above discussion, one can use \cref{kthy-springer-kostant} to show that $\End_{\QCoh((\tilde{\ld{G}} \times_{\ld{G}} \tilde{\ld{G}})/\ld{G})}(\QCoh(\ld{T} \times_{\ld{T}\mmod W} \ld{T}))$ can be identified with $\LMod_{\pi_0 C^T_\ast(\Fl_G(\cc); \KU)} \otimes \QQ$.
This can be viewed as a ``once-looped'' version of a K-theoretic analogue of Bezrukavnikov's equivalence from \cite{bezrukavnikov-two-geometric}.
%$T\backslash LG/T \simeq T\backslash LG/G \times_{G\backslash LG / G} G\backslash LG/T$. In 
\end{remark}

\begin{remark}\label{mult-nil-hecke}
We expect that most of the steps of \cref{looped-quantized-abg} can be replicated to study $\LMod_{C^{\tilde{T}}_\ast(\Gr_G(\cc); \KU)} \otimes \QQ$. More precisely, let $d\in \Z$, and fix a symmetric bilinear form $(-,-): \Lambda \times \Lambda \to \tfrac{1}{d}\Z$ such that whose Gram matrix is the associated Cartan matrix (i.e., $(\alpha_i, \alpha_j)$ is the $a_{ij}$ entry of the associated Cartan matrix). We then have the quantum group $U_q(\g)$ defined over $\Z[q^{\pm 1}]$  associated to the pairing $\Lambda \times \Lambda \to \Z[q^{\pm 1}]$ sending $\lambda, \mu \mapsto q^{-(\lambda, \mu)}$.
Following \cite[Definition 4.24]{univ-cat-o}, define the \textit{quantum universal category} ${\co}^\univ_q$ as the $\infty$-category of $(U_q(\g), U_q(\fr{t}))$-bimodules whose diagonal $U_q(\fr{b})$-action is integrable.

Let $(W, \Delta)$ be a crystallographic root system, let $\Lambda^\vee = \Z\Phi$ denote the associated root lattice, and let $T = \spec \Z[\Lambda]$ denote the associated torus. Each $\alpha\in W$ defines an operator $s_\alpha$ on $\co_T$. Define the \textit{multiplicative nil-Hecke algebra} $\cH(T, W)$ as the subalgebra of $\mathrm{Frac}(\co_T) \rtimes \QQ[W]$ generated by $\co_T$ and the operators $T_\alpha = \tfrac{1}{e^\alpha-1}(s_\alpha-1)$. (Also see \cite{elias-williamson-kthy} for a study of a multiplicative analogue of Soergel theory.) Then, there are relations
$$T_\alpha^2 = T_\alpha, \ (T_\alpha T_\beta)^{m_{\alpha,\beta}} = (T_\beta T_\alpha)^{m_{\alpha,\beta}} , \ x\cdot T_\alpha = T_\alpha \cdot s_\alpha(x) + T_\alpha(x), \ \alpha\in \Delta.$$
Recall that $m_{\alpha_i\alpha_j}$ is $2$, $3$, $4$, $6$, $\infty$ if $a_{ij} a_{ji}$ is $0$, $1$, $2$, $3$, $\geq 4$ (respectively). This algebra was also studied in \cite[Section 2.2]{k-thy-schubert-grg}.
Note that if $\lambda \in \Lambda$ (corresponding to the function $e^\lambda$ on $T$), we have $T_\alpha(e^\lambda) = [\langle \alpha^\vee, \lambda\rangle]_{e^\alpha} e^\lambda$, where $[\langle \alpha^\vee, \lambda\rangle]_{e^\alpha}$ denotes the $q$-integer $\tfrac{q^{\langle \alpha^\vee, \lambda\rangle} - 1}{q-1}$ with $q = e^\alpha$.

Given the discussion in \cref{sec: quantized homology torus} relating loop-rotation equivariance in $K$-theory to $q$-deformations, as well as \cref{looped-quantized-abg}, we expect:
\begin{conjecture}\label{oq-univ}
There is a Kostant functor $\kappa: \ld{\co}^\univ_q \to \QCoh(\ld{T}_\QQ \times \GG_m^q)$ (where $\GG_m^q = \spec \QQ[q^{\pm 1}]$) such that there is a $\QQ[q^{\pm 1}]$-linear equivalence
\begin{equation}\label{quantum-abg}
    \LMod_{\pi_0 C^{\tilde{T}}_\ast(\Gr_G(\cc); \KU)} \otimes \QQ \simeq \End_{\ld{\co}^\univ_q}(\QCoh(\ld{T}_\QQ \times \GG_m^q)).
\end{equation}
Similarly, if $\HC_q(\ld{G})$ denotes the category of \cite[Definition 2.24]{univ-cat-o}, there is a Kostant functor $\kappa: \HC_q(\ld{G}) \to \QCoh(\ld{T}_\QQ\mmod W \times \GG_m^q)$ and a $\QQ[q^{\pm 1}]$-linear equivalence
\begin{equation}\label{quantum-ginzburg}
    \LMod_{\pi_0 C^{G\times S^1_\rot}_\ast(\Gr_G(\cc); \KU)} \otimes \QQ \simeq \End_{\HC_q(\ld{G})}(\QCoh(\ld{T}_\QQ\mmod W \times \GG_m^q)).
\end{equation}
\end{conjecture}
At the moment, we are only able to describe the left-hand side in terms of combinatorial data. Let $e = \tfrac{1}{\# W}\sum_{w\in W} w$ be the symmetrizer idempotent. Using \cref{cohomology-grg} and \cite[Proposition 2.6]{k-thy-schubert-grg} (see also \cref{basis-cohomology}), one can show that $\pi_0 C^{\tilde{T}}_\ast(\Gr_G(\cc); \KU) \otimes \QQ$ is isomorphic to $\co_{\ld{T}} \otimes_{\co_{\ld{T}\mmod W}} e\cH(\tilde{\ld{T}}, \tilde{W}^\aff)e$, where the parameter $q\in \pi_0 \KU_{\GG_m^\rot} \cong \Z[q^{\pm 1}]$ corresponds to the coordinate on $\GG_m^q\subseteq \tilde{\ld{T}}$ viewed as an element of $\cH(\tilde{\ld{T}}, \tilde{W}^\aff)e$. Similarly, $\pi_0 C^{G \times S^1_\rot}_\ast(\Gr_G(\cc); \KU) \otimes \QQ$ is isomorphic to $e\cH(\tilde{\ld{T}}, \tilde{W}^\aff)e$. The conjectural equivalence \cref{quantum-ginzburg} then reduces to proving an (also conjectural) equivalence
\begin{equation}\label{simpler-quantum-ginzburg}
    %\End_{\ld{\co}_q^\univ}(\QCoh(\ld{T} \times \GG_m^q)) & \simeq \LMod_{\cH(\tilde{\ld{T}}, \tilde{W}^\aff)e}, \\
    \End_{\HC_q(\ld{G})}(\QCoh(\ld{T}\mmod W \times \GG_m^q)) \simeq \LMod_{e\cH(\tilde{\ld{T}}, \tilde{W}^\aff)e}.
\end{equation}
This may be understood as a quantum analogue of \cite[Theorem 8.1.2]{ginzburg-whittaker}. Note that the above equivalences are now statements which are squarely on one side of Langlands duality. In the case $G = \SL_2$, we described $C^{G\times S^1_\rot}_\ast(\Gr_G(\cc); \KU) \otimes \QQ$ (and hence $e\cH(\tilde{\ld{T}}, \tilde{W}^\aff)e$) below in \cref{4d-sl2}; it might be possible to use this calculation to compare with $\End_{\ld{\co}_q^\univ}(\QCoh(\ld{T} \times \GG_m^q))$ for $\ld{G} = \PGL_2$. A positive resolution to \cite[Conjecture 3.17]{finkelberg-tsymbaliuk} should be the key input into proving \cref{simpler-quantum-ginzburg}.

For general $G$, just as $(T\times \ld{T})^\bl$ is birational to $T \times \ld{T}$, the map from the algebra of $q$-difference operators on $\ld{T}$ to $\cH(\tilde{\ld{T}}, \tilde{W}^\aff)e$ is an isomorphism after a particular localization. One therefore expects $\ld{\co}_q^\univ$ and $\HC_q(\ld{T})$ to generically be equivalent. This is indeed true, and can be seen using \cite[Theorem 4.33]{univ-cat-o} (although the functor $\ld{\co}_q^\univ \to \HC_q(\ld{T})$ in \textit{loc. cit.} is not our expected functor $\kappa$).
\end{remark}
\begin{remark}
Since $\ld{G}/\ld{G} = \Map(S^1, B\ld{G})$, the canonical orientation of $S^1$ defines a $1$-shifted symplectic structure on $\ld{G}/\ld{G}$ via \cite[Theorem 2.5]{ptvv}. The quasi-classical limit (i.e., $q\to 1$) of the conjectural equivalence \cref{quantum-ginzburg} gives the following strengthening of \cref{kthy-once-looped-satake}. (This strengthening can be proved independently of \cref{quantum-ginzburg}.) 

Observe that the Kostant slice $\ld{T}\mmod W \to \ld{G}/\ld{G}$ is a Lagrangian morphism. It follows that the self-intersection $\ld{T}\mmod W \times_{\ld{G}/\ld{G}} \ld{T}\mmod W$ admits the structure of a symplectic stack by \cite[Theorem 2.9]{ptvv}. Since this fiber product is isomorphic to $(T^\ast_{\GG_m} \ld{T})^\bl\mmod W$ by \cref{kthy-tmmodw-bfm}, we obtain a Poisson bracket on $\co_{(T^\ast_{\GG_m} \ld{T})^\bl\mmod W} \cong \pi_0 C^G_\ast(\Gr_G(\cc); \KU)$. This structure can be seen topologically, at least after a completion: using one of the main results of \cite{klang}, the Borel-equivariant analogue/completion $C_\ast(\Gr_G(\cc); \KU)^{hG_c}$ of $C^G_\ast(\Gr_G(\cc); \KU)$ can be identified with the $\E{3}$-center of $\pi_0 C_\ast(\Gr_G(\cc); \KU)$. This defines a $2$-shifted Poisson bracket on $\pi_0 C_\ast(\Gr_G(\cc); \KU)^{hG_c}$, which can be identified with the ($0$-shifted, via the $2$-periodicity of $\KU$) Poisson bracket on $\co_{(T^\ast_{\GG_m} \ld{T})^\bl\mmod W}$.
\end{remark}
\begin{remark}
Following \cref{oq-univ}, one can also hope for a result analogous to \cref{quantum-abg} when $q \rightsquigarrow \zeta_p$ is specialized to a primitive $p$th root of unity. Namely, consider the $\infty$-category $\LMod_{C^{T\times \mu_{p,\rot}}_\ast(\Gr_G(\cc); \KU)}$, where $\mu_{p,\rot} \subseteq S^1_\rot$ acts by loop rotation. Note that $C^{T \times \mu_{p,\rot}}_\ast(\Gr_G(\cc); \KU)$ is a module over $\KU^{h\Cp}$, and $\pi_\ast \KU^{h\Cp} \cong \Z\pw{q-1}[\beta^{\pm 1}]/(q^p-1)$. Inverting $q-1$, we find that $C^{T \times \mu_{p,\rot}}_\ast(\Gr_G(\cc); \KU)[\frac{1}{q-1}]$ is a module over $\KU^{h\Cp}[\frac{1}{q-1}] \simeq \KU^{t\Cp} \simeq \QQ(\zeta_p)[\beta^{\pm 1}]$. We then expect the following (likely simpler) analogues of \cref{quantum-abg} and \cref{quantum-ginzburg}:
\begin{conjecture}\label{zetap-oq}
There are Kostant functors $\kappa: \ld{\co}^\univ_{\zeta_p} \to \QCoh(\ld{T}_{\QQ(\zeta_p)})$ and $\kappa: \HC_{\zeta_p}(\ld{G}) \to \QCoh(\ld{T}_{\QQ(\zeta_p)}\mmod W)$ such that there are $\QQ(\zeta_p)$-linear equivalences
\begin{align*}%\label{zetap-abg}
    \LMod_{\pi_0 C^{T \times \mu_{p,\rot}}_\ast(\Gr_G(\cc); \KU)[\frac{1}{q-1}]} & \simeq \End_{\ld{\co}^\univ_{\zeta_p}}(\QCoh(\ld{T}_{\QQ(\zeta_p)})),\\
    \LMod_{\pi_0 C^{G \times \mu_{p,\rot}}_\ast(\Gr_G(\cc); \KU)[\frac{1}{q-1}]} & \simeq \End_{\HC_{\zeta_p}(\ld{G})}(\QCoh(\ld{T}_{\QQ(\zeta_p)}\mmod W)).
\end{align*}
Note that there is no rationalization necessary on the left-hand sides.
\end{conjecture}
As with \cref{oq-univ}, \cref{zetap-oq} reduces to proving the (also conjectural) equivalence
\begin{align*}
    %\End_{\ld{\co}^\univ_{\zeta_p}}(\QCoh(\ld{T}_{\QQ(\zeta_p)})) & \simeq \LMod_{\cH_{\zeta_p}(\tilde{\ld{T}}, \tilde{W}^\aff)e}, \\
    \End_{\HC_{\zeta_p}(\ld{G})}(\QCoh(\ld{T}_{\QQ(\zeta_p)}\mmod W)) & \simeq \LMod_{e\cH_{\zeta_p}(\tilde{\ld{T}}, \tilde{W}^\aff)e},
\end{align*}
where $\cH_{\zeta_p}(\tilde{\ld{T}}, \tilde{W}^\aff)$ denotes the algebra obtained from $\cH(\tilde{\ld{T}}, \tilde{W}^\aff)$ by setting $q$ (arising from the loop rotation torus in $\tilde{\ld{T}}$) to $\zeta_p$.
\end{remark}
\subsection{The elliptic Kostant slice}

Fix a (classical) $\QQ$-algebra $k$ for the remainder of this section. Let $E$ be a (smooth) elliptic curve over $k$, let $\Bun_B^0(E)$ denote the moduli stack of $B$-bundles on $E$ of degree $0$, and let $\Bun_T^0(E)$ denote the scheme of $T$-bundles on $E$ of degree $0$. We will also make use of the stack $\Bun_G^\ss(E)$ of semistable $G$-bundles on $E$.
\begin{definition}\label{B-bundle regular}
Say that a $B$-bundle $\cP_B$ on $E$ is \textit{regular} if $\dim \Aut(\cP_B) = \rank(G)$. Let $\Bun_B^0(E)^\reg$ denote the open substack of $\Bun_B^0(E)$ defined by the regular $B$-bundles. Similarly, if $\cP\in \Bun_G^\ss(E)$ is a semistable $G$-bundle on $E$, we say that $\cP$ is \textit{regular} if $\dim \Aut(\cP) = \rank(G)$. Let $\Bun_G^\ss(E)^\reg \subseteq \Bun_G^\ss(E)$ denote the open substack of regular semistable $G$-bundles.
\end{definition}
\begin{notation}
For $\cP_T\in \Bun_T^0(E)$, write $\Delta_\cP$ to denotes the set of those simple roots $\alpha\in \Delta$ such that the $\alpha$-component of $\cP_T$ is trivial. We will also write $N_\cP = \prod_{\alpha\in \Phi^- \cap \Delta_\cP} N_\alpha\subseteq N$.
\end{notation}
\begin{prop}\label{elliptic-kostant}
The map $\Bun_B^0(E) \to \Bun_T^0(E)$ admits a canonical unique section $\kappa: \Bun_T^0(E) \to \Bun_B^0(E)$ landing in $\Bun_B^0(E)^\reg$.
\end{prop}
\begin{proof}
Let $\cP$ be a semistable $G$-bundle on $E$.
By \cite[Proposition 5.5.5]{davis-elliptic-springer}, the regularity of $\cP$ is equivalent to the condition that for any (or some) $B$-reduction $\cP_B$ of $\cP$ of degree $0$, the associated $N$-bundle $\cP_B/T$ is induced from an $N_\cP$-bundle with nontrivial associated $N_\alpha$-bundle for each $\alpha\in \Delta_\cP$. Moreover, every geometric fiber of the map $\Bun_G^\ss(E) \to \Hom(\bX^\ast(T), E)\mmod W$ to the coarse moduli space of $\Bun_G^\ss(E)$ contains a unique regular semistable $G$-bundle. Also see \cite[Proposition 3.9]{friedman-morgan-witten}, where a similar result is stated.

Following \cite[Definition 4.3.7]{davis-elliptic-springer}, set
$$\tilde{\Bun}_G^\ss(E)^\reg \cong \Bun_G^\ss(E)^\reg \times_{\Hom(\bX^\ast(T), E)\mmod W} \Hom(\bX^\ast(T), E).$$
Let $\Bun_B^0(E)^\reg$ denote the moduli stack of $B$-bundles on $E$ of degree $0$.
It then follows from the isomorphism $\tilde{\Bun}_G^\ss(E) \cong \Bun_B^0(E)$ of \cite[Proposition 4.1.2]{davis-elliptic-springer} and the equality $\dim \Aut(\cP) = \dim \Aut(\cP_B)$ that there is an isomorphism $\tilde{\Bun}_G^\ss(E)^\reg \cong \Bun_B^0(E)^\reg$. In particular, every geometric fiber of the map $\Bun_B^0(E) \to \Hom(\bX^\ast(T), E) = \Bun_T^0(E)$ contains a unique regular $B$-bundle of degree $0$. 
%This implies that if such a section $\kappa: \Bun_T^0(E) \to \Bun_B^0(E)$ exists, it must be unique.

The existence of $\kappa$ is a consequence of \cite[Theorem 4.3.2]{davis-elliptic-springer}, which is a refinement of \cite[Theorem 5.1.1]{friedman-morgan-ii}. Since we will not need the full strength of \cite[Theorem 4.3.2]{davis-elliptic-springer} outside of this proof, we will only briefly recall the necessary notation and statements. In \textit{loc. cit.}, the scheme $\Bun_T^0(E)$ is denoted by $Y$. Let $\tilde{\Bun}_G(E)$ denote the Kontsevich-Mori compactification of $\tilde{\Bun}_G^\ss(E) \cong \Bun_B^0(E)$; see \cite[Definition 2.1.2]{davis-elliptic-springer}. Let $\Theta$ denote the theta-line bundle over $\Bun_T^0(E)$ of \cite[Corollary 3.2.10]{davis-elliptic-springer}, and let $\tilde{\chi}: \tilde{\Bun}_G(E) \to \Theta^{-1}/\GG_m$ denote the map constructed in \cite[Corollary 3.3.2]{davis-elliptic-springer}. Then, \cite[Theorem 4.3.2]{davis-elliptic-springer} shows that there is a map $\Theta^{-1} \to \tilde{\Bun}_G^\ss(E)$ landing in $\tilde{\Bun}_G^\ss(E)^\reg$ such that the composite 
$$\Theta^{-1} \to \tilde{\Bun}_G^\ss(E) \xar{\tilde{\chi}} \Theta^{-1}/\GG_m$$
is the canonical map. Composing with the zero section of $\Theta^{-1}$, we obtain a map 
$$\Bun_T^0(E) \cong 0_{\Theta^{-1}} \to \Theta^{-1} \to \tilde{\Bun}_G^\ss(E)^\reg \cong \Bun_B^0(E).$$
This is the desired map $\kappa$.
\end{proof}
\begin{definition}
We will refer to the map $\kappa: \Bun_T^0(E) \to \Bun_B^0(E)$ from \cref{elliptic-kostant} as the \textit{elliptic Kostant slice}.
\end{definition}
\begin{example}
Let $G = \SL_2$, so that a $B$-bundle on $E$ is just a rank $2$ vector bundle $\cV$ with $\det(\cV) = 0$, equipped with a full flag. Then, the map $\kappa: \Pic^0(E) \to \Bun_B^0(E)$ sends a line bundle $\cL$ to the trivial filtration $\co_E \subseteq \co_E \oplus \cL$ if $\cL^2 \neq \co_E$; and to the Atiyah extension $\cL \subseteq \cf_2 \twoheadrightarrow \cL^{-1}$ from \cite{atiyah-bundle-elliptic} if $\cL^2 \cong \co_E$. This extension is defined by a nontrivial element of $\Ext^1_E(\cL, \cL^{-1}) \cong \H^1(E; \cL^{-2})$.
This can either be shown by unwinding the construction of the section $\kappa$ via \cite[Theorem 4.3.2]{davis-elliptic-springer}, or directly by noting that the description above provides the unique regular $B$-bundle lifting $\cL$.
\end{example}
We will need the following lemma below.
\begin{lemma}\label{vanishing and B-subgroups}
Let $I\subseteq \Phi^-$ be a subset, and let $\Bun_T^0(E)_I$ denote the subscheme of $\Bun_T^0(E)$ defined by those bundles $\cP_T$ whose $\alpha$-component is trivial precisely for $\alpha\in I$. Let $N_I\subseteq N$ be the smallest unipotent subgroup which is invariant under $T$-conjugation and which contains $N_\alpha$ for every $\alpha\in I$. Then the natural map
$$\Bun_{TN_I}^0(E) \times_{\Bun_T^0(E)} \Bun_T^0(E)_I \to \Bun_B^0(E) \times_{\Bun_T^0(E)} \Bun_T^0(E)_I$$
is an isomorphism.
\end{lemma}
\begin{proof}
Let $\cP_I$ denote the universal $T$-bundle over $\Bun_T^0(E)_I$, so that $\Bun_B^0(E) \times_{\Bun_T^0(E)} \Bun_T^0(E)_I$ is the stack of $B$-bundles $\cP_B$ such that $\cP_B/N \cong \cP_T$; therefore, it is isomorphic to the stack $\Bun_N^{\cP_I}$ in the notation of \cite[Section 2.1.1]{frenkel-gaitsgory-vilonen}. Similarly, $\Bun_{TN_I}^0(E) \times_{\Bun_T^0(E)} \Bun_T^0(E)_I \cong \Bun_{N_I}^{\cP_I}$. To show that these stacks are isomorphic, consider the filtration
$$N_\ell \subseteq N_{\ell-1} \subseteq \cdots \subseteq N_2 \subseteq N_1 = N$$
by root height (recall that the height of a root is the sum of its simple root components), so that it is invariant under $T$-conjugation, and there is an induced filtration
$$N_{I,\ell} \subseteq N_{I, \ell-1} \subseteq \cdots \subseteq N_{I,2} \subseteq N_{I,1} = N_I.$$
Then, $N_j\subseteq N$ is normal and $N_{j-1}/N_j$ is central in $N/N_j$ (and similarly for $N_{I,j}$); this implies that $\Bun_{N/N_j}^{\cP_I}$ is a $\Bun_{N_{j-1}/N_j}^{\cP_I}$-torsor over $\Bun_{N/N_{j-1}}^{\cP_I}$. Similar statements hold for $\Bun_{N_I/N_{I,j}}^{\cP_I}$. To show that $\Bun_{N_I}^{\cP_I} \to \Bun_N^{\cP_I}$ is an isomorphism, it therefore suffices to show that the induced map $\Bun_{N_{I,j-1}/N_{I,j}}^{\cP_I} \to \Bun_{N_{j-1}/N_j}^{\cP_I}$ is an isomorphism. Let $\cN = \cP_I \times^T N$, $\cN_I = \cP_I \times^T N_I$, etc., so that $\cN_{j-1}/\cN_j$ is a direct sum of line bundles of degree zero. By choice of $N_I$, the inclusion of the trivial line bundle summands into $\cN_{j-1}/\cN_j$ factors through the map $\cN_{I,j-1}/\cN_{I,j} \to \cN_{j-1}/\cN_j$. The desired isomorphism then follows from the observation that if $U$ is a vector group with $\GG_m$-action, then $\Bun_U^\cL$ is a point if $\cL$ is a nontrivial line bundle of degree zero (because then $\H^1(E; U(\cL)) = 0$).
\end{proof}
\begin{example}
For instance, suppose that $I = \emptyset$, so that $\Bun_T^0(E)_\emptyset$ denotes the open subscheme of $T$-bundles of degree zero whose $\alpha$-component is nontrivial for every negative root $\alpha$. The isomorphism $\tilde{\Bun}_G^\ss(E) \cong \Bun_B^0(E)$ implies that the map $\tilde{\Bun}_G^\ss(E) \to \Bun_T^0(E)$ is an isomorphism over $\Bun_T^0(E)_\emptyset$. In particular, every point of $\Bun_T^0(E)_\emptyset$ has a canonical associated (regular) semistable $G$-bundle.
The above results continue to hold if $E$ is replaced by the constant stack $S^1$ or by $B\GG_a$ (in which case $\tilde{\Bun}_G^\ss(E)$ and $\Bun_B^0(E)$ are to be interpreted as $G/G$ and $B/B$, and $\g/G$ and $\fr{b}/B$, respectively). In the case of $S^1$, for instance, the semistable $G$-bundles obtained in this way from $\Bun_T^0(E)_\emptyset$ are precisely those which lie in the regular \textit{semisimple} locus $G^{\rs}/G$; similarly for the case of $B\GG_a$.
\end{example}
\subsection{Rationalized Langlands duality over elliptic cohomology}\label{section-G-loops}

\begin{definition}\label{shifted-dual}
Let $\GG_0$ be a commutative group scheme over a ring $A_0$ (even an $\Eoo$-ring, but we will not need this). Let $\GG_0^\vee$ denote the stack $\Hom(\GG_0, B\GG_m)$.
\end{definition}
\begin{example}
If $\GG_0 = \GG_m$, then $\GG_0^\vee = B\Z$, i.e., is $S^1$ viewed as a constant stack. If $\GG_0$ is an abelian variety, then $\GG_0^\vee$ is the dual abelian variety. If $\GG_0 = \Z$, then $\GG_0^\vee$ is $B\GG_m$. Let $W$ denote the commutative group scheme over $\Z_{(p)}$ of $p$-typical Witt vectors. Let $W[F]$ denote the kernel of Frobenius on $W$. If $\hat{\GG}_a$ denotes the formal completion of $\GG_a$ at the origin, then $\hat{\GG}_a^\vee \cong BW[F]$ (over $\Z_{(p)}$). Since $W[F] \cong \GG_a$ over a field of characteristic zero, there is an isomorphism $\hat{\GG}_{a,\QQ}^\vee \cong B\GG_a$.
\end{example}
\begin{remark}\label{cartier-self-duality}
In general, there is a canonical map $\GG_0 \to (\GG_0^\vee)^\vee$, and the above examples imply that it is an isomorphism if $\GG_0$ is a finite product of abelian varieties, classifying stacks of groups of multiplicative type, and finitely generated abelian groups. If this is the case, $\GG_0$ is said to be \textit{dualizable}.
\end{remark}
\begin{remark}\label{poincare-bundle}
Note that the pairing $\GG_0 \times \GG_0^\vee \to B\GG_m$ defines a line bundle over $\GG_0 \times \GG_0^\vee$, which we will denote by $\cP$ and call the \textit{Poincar\'e line bundle}. If $\GG_0$ is an abelian variety, this is the usual Poincar\'e line bundle over $\GG_0 \times \GG_0^\vee$. If $\GG_0 = \GG_m$, the Poincar\'e line bundle gives the equivalence $\Rep(\Z) \simeq \QCoh(\GG_m)$ obtained by viewing $\GG_m$ as the torus associated to the monoid $\Z$.
\end{remark}
\begin{remark}
If $\GG_0$ is a finite flat, diagonal, or constant group scheme (but not an abelian variety!), then $\GG_0^\vee$ can be identified with the classifying stack of the Cartier dual of $\GG_0$. 
If $X$ is an $A_0$-scheme, let $\cL_{\GG_0} X$ denote the \textit{$\GG_0$-loop space} of $X$, given by the mapping stack $\Map(\GG_0^\vee, X)$.
Then, if $\GG_0$ is replaced by its formal completion at the zero section, the $\GG_0$-loop space recovers the loop space of \cite{moulinos-loop}.
\end{remark}

\begin{assume}\label{char-cochar}
Fix an isomorphism $\bX^\bull(T) \cong \bX_\bull(T)$ of lattices, which will be used implicitly below without further mention. (Note that we are not asking for a $W$-equivariant isomorphism, which would not exist in general.) This gives an isomorphism $\cM_T \cong \cM_{\ld{T}}$, which we will use below as an analogue of the identification between $\ld{\fr{t}} = \fr{t}^\ast$ and $\fr{t}$ (ubiquitous in geometric representation theory). Although potentially confusing, we will see below in the proof of \cref{borel-intersection} that this identification does not run the risk of conflating different sides of Langlands duality. 
%In future work, we hope to discuss \cref{general-looped-satake} without using such an isomorphism.
\end{assume}
We will prove the following at the end of the section, after a discussion of some consequences.
\begin{theorem}\label{borel-intersection}
Fix a complex-oriented even-periodic $\Eoo$-ring $A$ and an oriented commutative $A$-group $\GG$, as well as a semisimple algebraic group $\ld{G}$ over $\QQ$.
Assume that the underlying $\pi_0 A$-scheme $\GG_0$ is $\GG_a$, $\GG_m$, or an elliptic curve $E$.
Given a principal nilpotent $f\in \fr{n}$, there is a ``$\GG$-Kostant slice'' $\kappa: (\cM_{T,0})_\QQ \to \Bun_{\ld{B}}(\GG_{0,\QQ}^\vee)$ over $\pi_0 A_\QQ$. If $\Bun_{\ld{B}}^0(\GG_{0,\QQ}^\vee) = \Bun_{\ld{B}}(\GG_{0,\QQ}^\vee) \times_{\Bun_T} \cM_T$, there is a Cartesian square
$$\xymatrix{
(T^\ast_\GG \ld{T})^\bl \otimes \QQ \ar[r] \ar[d] & (\cM_{T,0})_\QQ \ar[d]^-\kappa \\
(\cM_{T,0})_\QQ \ar[r]_-\kappa & \Bun_{\ld{B}}^0(\GG_{0,\QQ}^\vee).
}$$
\end{theorem}
Combining with \cref{t-homology-grg}, we obtain the following:
\begin{corollary}\label{general-looped-satake}
Suppose that $G$ is a connected and simply-connected semisimple algebraic group or a torus over $\cc$.
Assume that the underlying $\pi_0 A$-scheme $\GG_0$ is $\GG_a$, $\GG_m$, or an elliptic curve $E$.
Then there is an equivalence
$$\End_{\QCoh(\Bun_{\ld{B}}^0(\GG_{0,\QQ}^\vee))}(\QCoh((\cM_{T,0})_\QQ))
%\simeq \QCoh((T^\ast_{\GG} \ld{T})^\bl) 
\simeq \Mod_{\pi_0 \cf_T(\Gr_G(\cc))^\vee}(\QCoh(\cM_{T,0})) \otimes \QQ,$$
where $\QCoh((\cM_{T,0})_\QQ)$ is regarded as a $\QCoh(\Bun_{\ld{B}}^0(\GG_{0,\QQ}^\vee))$-module via $\kappa$.
\end{corollary}
\begin{example}\label{examples-dual and reduction to smooth elliptic}
For example, if $\GG = \hat{\GG}_a$, then $\GG_0^\vee = BW[F]$. Therefore, $\GG_{0,\QQ}^\vee = B\GG_a$, and $\Bun_{\ld{B}}^0(\GG_{0,\QQ}^\vee) = \ld{\fr{b}}_\QQ/\ld{B}_\QQ \cong \tilde{\ld{\g}}_\QQ/\ld{G}_\QQ$ by \cite[Theorem 1.2.4]{toen-hkr}. In particular, \cref{borel-intersection} was proved above in this case as \cref{once-looped-satake}.
If $\GG = \GG_m$, then $\GG_{0,\QQ}^\vee = B\Z = S^1$, so that $\Bun_{\ld{B}}^0(\GG_{0,\QQ}^\vee) = \Map(S^1_{\KU_\QQ}, B\ld{B}_{\KU_\QQ})$ is isomorphic to the $2$-periodification of $\ld{B}_\QQ/\ld{B}_\QQ$. In particular, \cref{borel-intersection} was proved above in this case as \cref{kthy-once-looped-satake}.
If $\GG_0$ is an elliptic curve $E$, then $\GG_0^\vee = E^\vee$, so that $\Bun_{\ld{B}}^0(\GG_0^\vee) = \Bun_{\ld{B}}^0(E^\vee)$. \cref{borel-intersection} in this case will be proved below.
\end{example}
We also obtain a proof of \cref{intro-mirror-dual-of-g} (which we restate for convenience):
\begin{corno}[\cref{intro-mirror-dual-of-g}]
Suppose that $G$ is a connected and simply-connected semisimple algebraic group or a torus over $\cc$, and let $T$ act on $G$ by conjugation. Let $G_c$ denote the maximal compact subgroup of $G(\cc)$. Fix a complex-oriented even-periodic $\Eoo$-ring $A$, and let $\GG$ be an oriented group scheme in the sense of \cite{elliptic-ii}. Assume that the underlying $\pi_0 A$-scheme $\GG_0$ is $\GG_a$, $\GG_m$, or an elliptic curve $E$. Then there is an equivalence of $\pi_0 A_\QQ$-linear $\infty$-categories:
$$\Loc_{T_c}^\gr(G_c; A) \otimes \QQ \simeq \QCoh((\cM_{\ld{T},0})_\QQ \times_{\Bun_{\ld{B}}^0(\GG_{0,\QQ}^\vee)} (\cM_{\ld{T},0})_\QQ).$$
\end{corno}
\begin{proof}
Note that $G_c$ is connected. By \cref{graded local systems}, there is an equivalence $\Loc_{T_c}^\gr(G_c; A) \simeq \LMod_{\pi_0 \cf_T(\Omega G_c)^\vee}(\QCoh(\cM_{T,0}))$, so the claim follows from \cref{general-looped-satake}.
\end{proof}
\begin{remark}\label{no-2-periodification}
If $A = \QQ[\beta^{\pm 1}]$, the equivalence resulting from \cref{intro-mirror-dual-of-g} is an equivalence of $2$-periodic $\QQ$-linear $\infty$-categories. However, the equivalence can be de-periodified, and one obtains an equivalence
$$\Loc_{T_c}(G_c; \QQ) \simeq \QCoh(\ld{\fr{t}}[2]_\QQ \times_{\tilde{\ld{\g}}[2]_\QQ/\ld{G}_\QQ} \ld{\fr{t}}[2]_\QQ).$$
There is also a $G_c$-equivariant analogue:
$$\Loc_{G_c}(G_c; \QQ) \simeq \QCoh(\ld{\fr{t}}[2]_\QQ\mmod W \times_{\ld{\g}[2]_\QQ/\ld{G}_\QQ} \ld{\fr{t}}[2]_\QQ\mmod W).$$
This equivalence can be de-equivariantized, to obtain an equivalence
$$\Loc(G_c; \QQ) \simeq \QCoh(Z_f(\ld{B})),$$
where $f\in \ld{\g}$ is the image of the origin in $\ld{\fr{t}}\mmod W$ under the Kostant slice, and $Z_f(\ld{B})$ is a shifted analogue of the centralizer of $f$ in $\ld{B}$.
Note that $T^\ast G_c = G(\cc)$, so that the left-hand side can be interpreted as a relative of the $\QQ$-linearization of the wrapped Fukaya category of $T^\ast G_c$ by \cite[Theorem 1.1]{ganatra-pardon-shende}. In particular, this shifted analogue of $Z_f(\ld{B})$ is a (derived) mirror to $G(\cc)$ viewed as a symplectic manifold.
\end{remark}
\begin{remark}\label{expected G-def}
The proof of \cref{intro-mirror-dual-of-g} above uses the Koszul duality equivalence $\Loc_{T_c}(G_c; A) \simeq \LMod_{\cf_T(\Omega G_c)^\vee}(\QCoh(\cM_{T,0}))$ of \cref{equivariant-koszul}. The category $\LMod_{\cf_T(\Omega G_c)^\vee}(\QCoh(\cM_{T,0}))$ (and hence the right-hand side of \cref{intro-mirror-dual-of-g}) admits a ``quantization'' parametrized by $\GG$, given by $\LMod_{\cf_{\tilde{T}}(\Omega G_c)^\vee}(\QCoh(\cM_{T,0}))$. For instance, if $A = \QQ[\beta^{\pm 1}]$, the right-hand side of \cref{intro-mirror-dual-of-g} quantizes to $\End_{\ld{\co}^\univ_\hbar}(\QCoh(\tilde{\fr{t}}))$; and if $A = \KU$, the right-hand side of \cref{intro-mirror-dual-of-g} quantizes to $\End_{\ld{\co}^\univ_q}(\QCoh(\tilde{T}))$. 
It follows from this discussion that the $\infty$-category $\Loc_{T_c}(G_c; A)$ must itself admits a quantization. We have seen a quantization of this form above in \cref{G-quantization torus}.

In fact, \cref{looped-quantized-abg} and \cref{oq-univ} suggest that $\LMod_{\cf_{\tilde{T}}(\Gr_G(\cc))^\vee}(\QCoh(\cM_{\tilde{T}})) \otimes \QQ$ should be viewed as $\End_{\cC_\GG}(\QCoh(\cM_{\tilde{T}})\otimes \QQ)$ for some $A_\QQ$-linear $\infty$-category $\co_\GG$ which is a $1$-parameter deformation of $\QCoh(\Bun_{\ld{B}}(\GG_{0,\QQ}^\vee))$. The coordinate on the group scheme $\GG$ defines a ``quantization parameter'' (i.e., the analogue of $\hbar$ and $q$). This putative $\infty$-category $\co_\GG$ would be an analogue of the (quantum) universal category $\co$. We do not know how to define such an $\infty$-category $\co_\GG$ at the moment; however, in future work, we plan to use the results of \cite{generalized-n-series} to study an ``$F$-deformation'' of $U(\g)$ for certain formal group laws $F(x,y)$ (at least for $G = \SL_2, \PGL_2$). When $F$ is the multiplicative formal group, this $F$-deformation of $U(\g)$ recovers the quantum enveloping algebra $U_q(\g)$. We hope that further study of such deformations will point to a good definition of the putative $\infty$-category $\co_\GG$.
\end{remark}
\begin{remark}
It is natural to ask for an explicit description of the $1$-parameter deformation of $\Loc_{T_c}(G_c; A)$ over $\GG$ from \cref{expected G-def} (i.e., not in terms of the {framed} $\E{2}$-structure on $\Omega G_c = \Omega^2 BG_c$). To describe this, let us view $\Loc_{T_c}(G_c; A)$ as the $\infty$-category of local systems on the orbifold $G_c/_\ad T_c$. We now need the following:
\begin{lemma}\label{S1-connections level}
The orbifold $G_c/_\ad T_c$ is isomorphic to the the moduli stack $\Conn(S^1; \g)^\lev$ of $\g$-valued smooth connections on $S^1$ equipped with a level structure given by a $T_c$-reduction at $\{1\}\in S^1$, taken modulo gauge transformations.
\end{lemma}
\begin{proof}
Write $G_c/_\ad T_c \simeq \ast/T_c \times_{\ast/G_c} G_c/_\ad G_c$. There is an equivalence
$$G_c/_\ad G_c \simeq \ast/G_c \times_{\ast/G_c \times \ast/G_c} \ast/G_c$$
which exhibits $G_c/_\ad G_c$ as the free loop space $\cL(\ast/G_c) \cong \ast/\cL G_c$ in the $\infty$-category of orbifolds. To see this, note that $G_c/G_c\simeq G_c\backslash (G_c\times G_c)/G_c$, where $G_c\times G_c$ acts on $G_c\times G_c$ via
$$(g_1, g_2): (h_1, h_2) \mapsto (g_1 h_1 g_2^{-1}, g_1 h_2 g_2^{-1}).$$
In any case, the above equivalence implies that $G_c/_\ad G_c$ is isomorphic to the moduli stack $\Conn(S^1; \g)/\cL G_c$, where $\Conn(S^1; \g)$ is the moduli space of smooth connections on $S^1$ valued in $\g$; see \cite[Section 15.1]{fht-iii}. This implies the desired claim.
\end{proof}
One natural way to quantize $\Loc_{T_c}(G_c; A)$ is therefore to consider the $\infty$-category of ``$S^1_\rot \ltimes \cL G_c$-equivariant $A$-valued local systems on $\Conn(S^1; \g)^\lev$''; this is a module over $\Loc_{S^1_\rot}(\ast; A) \simeq \QCoh(\GG)$, and its fiber over the zero section of $\GG$ is $\Loc_{T_c}(G_c; A)$ itself. However, defining this $\infty$-category precisely requires additional effort, since $S^1_\rot \ltimes \cL G_c$ is not a compact group.
\end{remark}

Let us now turn to the proof of \cref{borel-intersection}; by \cref{examples-dual and reduction to smooth elliptic}, we only need to consider the case when $\GG$ is a (smooth) elliptic curve $E$. Since we are working on one side of Langlands duality, we now drop the ``check''.
\begin{proof}[Proof of \cref{borel-intersection}]
We will work over $\QQ$, and omit it from the notation.
Write $X$ to denote the fiber product in \cref{borel-intersection}, so that our goal is to identify $X$ with $(T^\ast_\GG \ld{T})^\bl$. (The reader should keep in mind \cref{char-cochar}.) The argument of \cite[Section 4.3]{bfm} can be used to reduce to the case when $\ld{G}$ has semisimple rank $1$.

Namely, first note that both $X$ and $(T^\ast_\GG \ld{T})^\bl$ are flat over $\cM_T$: the only nontrivial case is $(T^\ast_\GG \ld{T})^\bl$, in which case this follows from \cite[Claim in Lemma 4.1]{bfm}. Let $\cM_T^\circ \hookrightarrow \cM_T$ denote the open immersion given by the complement of the union of the divisors $\cM_{T_\alpha} \hookrightarrow \cM_T$ for $\alpha\in \Phi$.
Upon localizing to $\cM_T^\circ$, both $X$ and $(T^\ast_\GG \ld{T})^\bl$ are isomorphic to $\ld{T} \times \cM_T^\circ$. 
Let $\cM_T^\bull$ denote the complement of the union of all pairwise intersections of the divisors $\cM_{T_\alpha} \hookrightarrow \cM_T$ for $\alpha\in \Phi$. Then $\cM_T - \cM_T^\bull \hookrightarrow \cM_T$ is of codimension $\geq 2$.
It therefore suffices to show (by flatness of both $X$ and $(T^\ast_\GG \ld{T})^\bl$ over the normal irreducible scheme $\cM_T$) that the isomorphism $X|_{\cM_T^\circ} \cong (T^\ast_\GG \ld{T})^\bl|_{\cM_T^\circ}$ extends across the codimension $1$ points of $\cM_T - \cM_T^\circ$ (i.e., points of $\cM_T^\bull - \cM_T^\circ$).

If $y$ is a codimension $1$ point of $\cM_T$ which lies on the divisor $\cM_{T_\alpha} \hookrightarrow \cM_T$ for some $\alpha\in \Phi$, let $Z_\alpha(y)\subseteq \ld{G}$ denote the reductive subgroup of $\ld{G}$ containing $\ld{T}$ and whose nonzero roots are $\pm \alpha$.
%21.11 of milne; it's the centralizer of the torus T_\alpha inside G
This is a connected Levi subgroup of semisimple rank $1$. It is easy to see that the localization $(T^\ast_\GG \ld{T})^\bl_y$ depends only on $Z_\alpha(y)$.
Let $\ld{B}_\alpha^- \subseteq \ld{B}$ denote the Borel subgroup of $Z_\alpha(y)$ determined by $\ld{B}$. \cref{vanishing and B-subgroups} with $I = \{\alpha\}$ implies that the induced map from $(\cM_{T,0})_\QQ \times_{\Bun_{\ld{B}_\alpha^-}^0(E)} (\cM_{T,0})_\QQ$ to $(\cM_{T,0})_\QQ \times_{\Bun_{\ld{B}}^0(E)} (\cM_{T,0})_\QQ$ defines an isomorphism upon localizing at $y$. In particular, the localization $X_y$ also depends only on $Z_\alpha(y)$.

We are now reduced to the case when $\ld{G}$ has semisimple rank $1$. Every split reductive group of semisimple rank $1$ is isomorphic to the product of a split torus with $\SL_2$, $\PGL_2$, or $\GL_2$. Let us illustrate the calculation when $\ld{G} = \PGL_2$. The cases $\ld{G} = \SL_2, \GL_2$, and products of tori with these groups can be addressed similarly.
For notational convenience, we will drop the ``check''s and write $B$ instead of $\ld{B}$, etc.; also note that since $T$ is of rank $1$, we may identify $\cM_T \cong \GG$.
Let $\cV$ denote the unique indecomposable rank $2$ ``Atiyah bundle'' over $E^\vee \times \GG_0$; this is an extension of the structure sheaf by the Poincar\'e line bundle $\cP$, which is specified by a nonzero section of $\H^1(E^\vee \times \GG_0; \cP) \cong k$. The bundle $\cV$ sits in a short exact sequence
$$0 \to \cP \to \cV \to \co_{E^\vee \times \GG_0} \to 0.$$

Any fixed basepoint $p_0\in E^\vee$ defines an isomorphism $E^\vee \cong \GG_0$, and allows us to identify $\cP$ with the line bundle on $E^\vee \times E^\vee$ corresponding to the divisor $\Delta - E^\vee \times \{p_0\} - \{p_0\} \times E^\vee$, where $\Delta$ is the diagonal. In particular, $\cP|_{E^\vee \times \{x\}} \cong \co_{E^\vee}(x-p_0)$, and is therefore only trivial when $x = p_0$. The fiber of $\cV$ over $E^\vee \times \{x\}$ is specified by a nonzero element of $\Ext^1_{E^\vee}(\co, \co(x-p_0))$; but if $\cL$ is a nontrivial line bundle, then $\H^1(E^\vee; \cL) = 0$. This implies that the map $\kappa: \GG_0 \to \Bun_B(E^\vee)$ sends a degree $0$ line bundle $\cL$ on $E^\vee$ to the trivial extension $\co_{E^\vee} \subseteq \co_{E^\vee} \oplus \cL$ if $\cL \not\cong \co_{E^\vee}$, and to the Atiyah extension $\co_{E^\vee}\subseteq \cf_2$ if $\cL \cong \co_{E^\vee}$. 

We need to understand $\Aut_B(\{\cP\subseteq \cV\})$.
If $\cL$ is a nontrivial line bundle on $E^\vee$, then $\cL$ has no sections, so $\Aut_B(\{\co_{E^\vee} \subseteq \co_{E^\vee} \oplus \cL\}) \cong \GG_m$. On the other hand, the algebra $\End(\cf_2)$ of endomorphisms of $\cf_2$ as a rank $2$ vector bundle is isomorphic to $k[\epsilon]/\epsilon^2$ as an algebra; the element $\epsilon$ acts as the composite $\cf_2 \twoheadrightarrow \co_{E^\vee} \hookrightarrow \cf_2$.
In particular, the group scheme $\Aut(\cf_2)$ of automorphisms of $\cf_2$ as a rank $2$ vector bundle is $(k[\epsilon]/\epsilon^2)^\times$.
An automorphism of $\cf_2$ preserving the flag $\co_{E^\vee} \subseteq \cf_2$ is defined by a matrix
$\begin{psmallmatrix}
x & y\\
0 & z
\end{psmallmatrix}$, where $x,y,z\in \Hom(\co_{E^\vee}, \co_{E^\vee})$. In order for two maps $x,z: \co_{E^\vee} \to \co_{E^\vee}$ to define an automorphism of $\cf_2$, we need $x = z$. Since we are only calculating the automorphisms of $\cf_2$ as a $\PGL_2$-bundle, the factor $x = z$ can be scaled out, and we find that $\Aut_B(\{\co_{E^\vee} \subseteq \cf_2\}) \cong \GG_a$.
%(See also \cite[Lemma 1.10]{friedman-morgan-witten-vector-bundles}.)
%also \cite[Lemma 1.14]{friedman-morgan}
The fiber of the map $\GG_0 \times_{\Bun_B(E^\vee)} \GG_0 \to \GG_0$ over $\cL\in \GG_0$ is therefore $\GG_m$ if $\cL \not \cong \co_{E^\vee}$ (i.e., away from the zero section), which degenerates to $\AA^1$ over the zero section corresponding to $\cL = \co_{E^\vee}$.

(In the case $\ld{G} = \SL_2$, the same argument shows that the fiber of the map $\GG_0 \times_{\Bun_B(E^\vee)} \GG_0 \to \GG_0$ is still $\GG_m$ if $\cL^2$ is not trivial, but the fiber over any point of $\cL\in \GG_0[2]$ is instead $\GG_a\times \mu_2$. Indeed, the image of of $\cL\in \GG_0[2]$ under the Kostant slice $\GG_0 \to \Bun_B(E^\vee)$ is the nontrivial extension 
$$0 \to \cL \to \cL \otimes \cf_2 \to \cL^{-1} \to 0.$$
Note that the subgroup of $\ld{B} \subseteq \SL_2$ given by $\Aut_{\ld{B}}(\{\cL \subseteq \cf_2 \otimes \cL\})$ is of the form $\begin{psmallmatrix}
x & y\\
0 & z
\end{psmallmatrix}$, where $x\in \Hom(\cL, \cL)$, $y\in \Hom(\cL^{-1}, \cL)$, and $z\in \Hom(\cL^{-1}, \cL^{-1})$. Not every such matrix defines an automorphism of $\cf_2 \otimes \cL$; for instance, in order for two maps $x: \cL \to \cL$ and $z:\cL^{-1} \to \cL^{-1}$ to define an automorphism of $\cf_2 \otimes \cL$, we need $x = z \otimes \cL^2 = z$. In order for the resulting matrix $\begin{psmallmatrix}
x & y\\
0 & z
\end{psmallmatrix}$ to preserve the trivialization of $\det(\cV \otimes \cL)$, we need $x^2 = 1$; the function $y$ can be arbitrary. This discussion implies that $\Aut_{\ld{B}}(\{\cL \subseteq \cf_2 \otimes \cL\}) \cong \mu_2 \times \GG_a$, where the $\mu_2$ encodes $x$, and $\GG_a$ encodes $y$.)

The intersection $\GG_0 \times_{\Bun_B(E^\vee)} \GG_0$ consists of $\cL, \cL'\in \GG_0$ equipped with an isomorphism $\kappa(\cL)\cong \kappa(\cL')$ of $B$-bundles over $E^\vee$ (which in particular forces $\cL \cong \cL'$). In fact, the discussion above can be used to conclude that $\GG_0 \times_{\Bun_B(E^\vee)} \GG_0$ is isomorphic to an affine blowup of $\GG_0 \times \GG_m$, defined as the complement $U$ of the proper preimage of the zero section of $\GG_0$ inside the blowup $\fr{B}$ of $\GG_0 \times \GG_m$ at the union of the zero sections of $\GG_0$ and $\GG_m$.
(In the case $\ld{G} = \SL_2$, the fiber product $\GG_0 \times_{\Bun_B(E^\vee)} \GG_0$ is isomorphic to an affine blowup of $\GG_0 \times \GG_m$, defined as the complement $U$ of the proper preimage of the $2$-torsion $\GG_0[2]\subseteq \GG_0$ inside the blowup $\fr{B}$ of $\GG_0 \times \GG_m$ at the union of the $2$-torsion sections $\GG_0[2] \subseteq \GG_0$ and $\mu_2 \subseteq \GG_m$.)
%(When $G = \SL_2$, the fiber over any point in the $2$-torsion $\GG[2]$ is instead $\AA^1 \times \mu_2$.)
But $U\subseteq \fr{B}$ is precisely the affine blowup $(T^\ast_\GG T)^\bl$, as desired.
\end{proof}
\begin{remark}
The most classical instantiation of the Atiyah bundle is via the Weierstrass functions. The $\GG_a$-torsor $\cA$ over $E$ associated to $\cV$ is the complement of the section at $\infty$ of the projective line $\PP(\cV)$. If we work complex-analytically, $E^\an$ can be identified as the quotient $\cc/\Lambda$ for some rank $2$ lattice $\Lambda\subseteq \cc$. Associated to $\Lambda$ are two Weierstrass functions defined on $\cc$:
\begin{align*}
    \wp(z; \Lambda) & = \frac{1}{z^2} + \sum_{\lambda \in \Lambda-\{0\}} \left(\frac{1}{(z-\lambda)^2} - \frac{1}{\lambda^2}\right), \\
    \zeta(z; \Lambda) & = \frac{1}{z} + \sum_{\lambda \in \Lambda-\{0\}} \left(\frac{1}{z-\lambda} + \frac{1}{\lambda} + \frac{z}{\lambda^2}\right).
\end{align*}
Note that $\wp(z; \Lambda)$ is doubly-periodic, i.e., $\wp(z + \lambda; \Lambda) = \wp(z; \Lambda)$ for any $\lambda \in \Lambda$. Alternatively, $\wp$ defines a map $\cc \to \cc$ which factors through a map $\cc/\Lambda = E^\an \to \cc$.

Although $\zeta(z; \Lambda)$ is not doubly-periodic, an easy calculation shows that $\wp(z; \Lambda) = -\partial_z \zeta(z; \Lambda)$; so if $\lambda \in \Lambda$, then $\zeta(z+\lambda; \Lambda) - \zeta(z; \Lambda) = c(\lambda)$ for some constant $c(\lambda)$. The function $\lambda \mapsto c(\lambda)$ is evidently additive, and defines a homomorphism $\Lambda \to \cc$, which defines a $\cc$-bundle over $E^\an = \cc/\Lambda$. This $\cc$-bundle is precisely the analytification $\cA^\an$ of the $\GG_a$-torsor $\cA$. It follows that although $\zeta$ is not defined on $E^\an$, the torsor $\cA^\an$ is the universal space over $E^\an$ on which $\zeta$ is defined.

This discussion also describes the total space of the rank $2$-bundle $\cV^\an$ purely analytically. For instance, if $q\in \cc^\times$ is a unit complex number of modulus $<1$, we can identify $\Tot(\cV^\an)$ over the Tate curve $\cc^\times/q^\Z$ with the quotient
$$\Tot(\cV^\an) = \left(\cc^\times \times \cc^2\right)/\left((z,x) \sim \left(qz, \begin{psmallmatrix}
1 & 1\\
0 & 1
\end{psmallmatrix} x\right)\right).$$
\end{remark}
\subsection{Putting it together}\label{categorical-cobar}
We will now explore one corollary of \cref{general-looped-satake}.
\begin{setup}\label{cobar-setup}
Let $\Prst$ be the $\infty$-category of compactly generated presentable $\infty$-categories and colimit-preserving functors which preserve compact objects.
Let $\cC\in \CAlg(\Prst)$, and let $\cd\in \CAlg(\LMod_\cC(\Prst))$ whose underlying object of $\LMod_\cC(\Prst)$ is dualizable. The unit map $i^\ast: \cC \to \cd$ defines a symmetric monoidal functor ${i'}^\ast: \cd \simeq \cd \otimes_\cC \cC \to \cd \otimes_\cC \cd$, and if $i_\ast: \cd \to \cC$ denotes the right adjoint to $i^\ast$, the following diagram commutes:
$$\xymatrix{
\cd \ar[r]^-{i_\ast} \ar[d]_-{{i'}^\ast} & \cC \ar[d]^-{i^\ast}\\
\cd \otimes_\cC \cd \ar[r]_-{{i'}_\ast} & \cd.
}$$
\end{setup}
\begin{prop}\label{completion-morita}
In \cref{cobar-setup}, there is a fully faithful colimit-preserving functor $\Tot(\cd^{\otimes_\cC \bull+1}) \hookrightarrow \cC$; we will denote its essential image by $\cC^\wedge_\cd$.
\end{prop}
\begin{proof}
The assumptions in \cref{cobar-setup} imply that the augmented cosimplicial diagram $ N(\Deltab_+) \to \Cat_\infty$ given by $\cd^{\otimes_\cC \bull+1}$ satisfies the assumptions of \cite[Corollary 4.7.5.3]{HA}. Therefore, the functor $\cC \to \Tot(\cd^{\otimes_\cC \bull+1})$ has a fully faithful left adjoint, as desired.
\end{proof}
\begin{observe}\label{kost-support}
Regard $\QCoh((\cM_{T,0})_\QQ)$ as a $\QCoh(\Bun_{\ld{B}}^0(\GG_{0,\QQ}^\vee))$-algebra via $\kappa: (\cM_{T,0})_\QQ \to \Bun_{\ld{B}}^0(\GG_{0,\QQ}^\vee)$. Then the completion $\QCoh(\Bun_{\ld{B}}^0(\GG_{0,\QQ}^\vee))^\wedge_{\QCoh((\cM_{T,0})_\QQ)}$ of \cref{completion-morita} with respect to $\QCoh((\cM_{T,0})_\QQ)$ can be identified with $\QCoh(\Bun_{\ld{B}}^0(\GG_{0,\QQ}^\vee)^\reg)$.
\end{observe}
\begin{example}
When $A$ is an $\Eoo$-$\QQ[\beta^{\pm 1}]$-algebra and $\GG = \hat{\GG}_a$, the Koszul duality equivalence of \cref{kd-springer} gives $\QCoh(\Bun_{\ld{B}}^0(\GG_{0,\QQ}^\vee)) \simeq \IndCoh((\tilde{\ld{\cN}} \times_{\ld{\g}} \{0\})/\ld{G})$; we define $\IndCoh((\tilde{\ld{\cN}} \times_{\ld{\g}} \{0\})/\ld{G})_\Kost$ to be the essential image of $\QCoh(\Bun_{\ld{B}}^0(\GG_{0,\QQ}^\vee)^\reg)$ under this equivalence. We remark that in this case, $\QCoh(\Bun_{\ld{B}}^0(\GG_{0,\QQ}^\vee)^\reg) \simeq \QCoh(\tilde{\ld{\g}}^\reg/\ld{G})$. Similarly, if $A = \KU$ and $\GG = \GG_m$, then $\QCoh(\Bun_{\ld{B}}^0(\GG_{0,\QQ}^\vee)^\reg) \simeq \QCoh(\tilde{\ld{G}}^\reg/\ld{G})$.
\end{example}
\begin{corollary}\label{regular-satake}
Fix a complex-oriented even-periodic $\Eoo$-ring $A$ and an oriented commutative $A$-group $\GG$.
Assume that the underlying $\pi_0 A$-scheme $\GG_0$ is $\GG_a$, $\GG_m$, or an elliptic curve $E$.
Suppose $G$ is a connected and simply-connected semisimple algebraic group over $\cc$.
Then there is an $\E{2}$-monoidal equivalence
$$\QCoh(\Bun_{\ld{B}}^0(\GG_{0,\QQ}^\vee)^\reg) \simeq \Loc_{T_c}^\gr(\Omega G_c; A) \otimes \QQ.$$
\end{corollary}
\begin{proof}
The $\Eoo$-coalgebra structure on $\pi_0 \cf_T(\Gr_G(\cc))^\vee$ defines a $\QCoh(\cM_{T,0})$-coalgebra structure on $\LMod_{\pi_0 \cf_T(\Gr_G(\cc))^\vee}(\QCoh(\cM_{T,0}))$.
The right-hand side of the equivalence of \cref{general-looped-satake} also admits a $\QCoh((\cM_{T,0})_\QQ)$-coalgebra structure, being the tensor product of $\QCoh((\cM_{T,0})_\QQ)$ with itself over $\QCoh(\Bun_{\ld{B}}^0(\GG_{0,\QQ}^\vee))$; and it is not difficult to check that the equivalence of \cref{general-looped-satake} is one of $\QCoh((\cM_{T,0})_\QQ)$-coalgebras.
In particular, there is a commutative diagram
$$\xymatrix{
& \QCoh((\cM_{T,0})_\QQ) \ar[d]^-{\ast \to \Gr_G(\cc)} \ar[dl] \\
\QCoh((\cM_{T,0})_\QQ \times_{\Bun_{\ld{B}}^0(\GG_{0,\QQ}^\vee)} (\cM_{T,0})_\QQ) \ar[r]^-\sim & \LMod_{\pi_0 \cf_T(\Gr_G(\cc))^\vee}(\QCoh(\cM_{T,0})) \otimes \QQ\\
}$$
which defines an equivalence of cosimplicial diagrams, and hence of their totalizations. The totalization of the cosimplicial diagram built from the functor $\QCoh(\cM_{T,0}) \to \Mod_{\pi_0 \cf_T(\Gr_G(\cc))^\vee}(\QCoh(\cM_{T,0}))$ defines an equivalence
$$\Tot(\LMod_{(\pi_0 \cf_T(\Gr_G(\cc))^\vee)^{\otimes \bull}}(\QCoh(\cM_{T,0}))) \simeq \coLMod_{\pi_0 \cf_T(\Gr_G(\cc))^\vee}(\QCoh(\cM_{T,0}));$$
note that by \cref{graded local systems}, this is in turn equivalent to $\Loc_{T_c}^\gr(\Omega G_c; A)$.
By \cref{completion-morita}, we also have 
$$\Tot(\QCoh((\cM_{T,0})_\QQ)^{\otimes_{\QCoh(\Bun_{\ld{B}}^0(\GG_{0,\QQ}^\vee))} \bull+1}) \simeq \QCoh(\Bun_{\ld{B}}^0(\GG_{0,\QQ}^\vee)^\reg).$$
This gives the desired equivalence.
\end{proof}
\begin{example}\label{regular-satake-special-cases}
When $A = \QQ[\beta^{\pm 1}]$ and $\GG = {\GG}_a$, we have $\Bun_{\ld{B}}^0(\GG_{0,\QQ}^\vee) = \tilde{\ld{\g}}/\ld{G}$. \cref{regular-satake} gives an $\E{2}$-monoidal equivalence
\begin{align*}
    \IndCoh((\tilde{\ld{\cN}} \times_{\ld{\g}} \{0\})/\ld{G})_\Kost & \simeq \QCoh(\tilde{\ld{\g}}^\reg/\ld{G}) \\
    & \simeq \Loc_{T_c}^\gr(\Omega G_c; \QQ[\beta^{\pm 1}]).
    %\coLMod_{C^T_\ast(\Gr_G(\cc); \QQ)} \otimes_\QQ \QQ[\beta^{\pm 1}].
\end{align*}
Note that $\tilde{\ld{\g}}/\ld{G}$ is isomorphic to the quotient $\ld{G} \backslash (\ld{G} \times^{\ld{N}} \ld{\fr{b}})/\ld{T}$; and \cite[Proposition 3.10]{safronov-implosion} says that $\ld{G} \times^{\ld{N}} \ld{\fr{b}}$ is the universal symplectic implosion (i.e., the symplectic implosion of $T^\ast G$). The relationship of this perspective to Langlands duality is closely related to the program of Ben-Zvi--Sakellaridis--Venkatesh \cite{sakellaridis-icm}: namely, the Hamiltonian $\ld{G} \times \ld{T}$-space $T^\ast(\ld{G}/\ld{N})$ acts as a ``kernel'' for the symplectic implosion functor from Hamiltonian $\ld{G}$-spaces to Hamiltonian $\ld{T}$-spaces.

Similarly, using \cref{looped-quantized-abg}, one can prove an equivalence between $\Loc_{\tilde{T}_c}^\gr(\Omega G_c; \QQ[\beta^{\pm 1}])$ and a localization of $\ld{\co}^\univ_\hbar$.
There is also an $\E{2}$-monoidal equivalence
$$\QCoh(\ld{\g}^\reg/\ld{G}) \simeq
\Loc_{G_c}^\gr(\Omega G_c; \QQ[\beta^{\pm 1}]);
%\coLMod_{C^G_\ast(\Gr_G(\cc); \QQ)} \otimes_\QQ \QQ[\beta^{\pm 1}]
$$
this follows from the analogue of \cref{loc and comod} for $G_c$-local systems and \cite[Proposition 2.2.1]{ngo-ihes}, which says that the classifying stack of the group scheme $\ld{J} = \spec \H^G_\ast(\Gr_G(\cc); \QQ)$ of regular centralizers is isomorphic to $\ld{\g}^\reg/\ld{G}$.
\end{example}
\begin{remark}
In the case of rank $1$, one can use \cref{regular-satake-special-cases} to show that if the circle $\SO(2)$ acts on $\SO(3)/\SO(2) = S^2$ by left multiplication, there is an equivalence
\begin{equation}\label{eq: SO2-equiv Loops S2 periodic}
    \Loc_{\SO(2)}^\gr(\Omega S^2; \QQ[\beta^{\pm 1}]) \simeq \QCoh(T^\ast(\AA^2)^\reg/\SL_2).
\end{equation}
Here, $\AA^2$ is equipped with the standard action of $\SL_2$, and $T^\ast(\AA^2)^\reg$ denotes the preimage of the regular locus of $\sl_2$ under the moment map $T^\ast(\AA^2) \to \sl_2$. This is because there is an equivalence
\begin{align*}
    \Loc_{\SO(2)}^\gr(\Omega S^2; \QQ[\beta^{\pm 1}]) & \simeq \Loc_{\SO(2)}^\gr(\Omega \SO(3); \QQ[\beta^{\pm 1}]) \otimes_{\Loc^\gr(\Omega \SO(2); \QQ[\beta^{\pm 1}])} \Vect_\QQ\\
    & \simeq \QCoh(\tilde{\sl}_2^\reg/\SL_2 \times_{B\GG_m} \spec(\QQ)) \simeq \QCoh(\SL_2\backslash T^\ast(\SL_2/N)^\reg),
\end{align*}
where the second equivalence uses \cref{regular-satake-special-cases} (i.e., the Arkhipov-Bezrukavnikov-Ginzburg equivalence over the regular locus). If $\ld{B}\subseteq \SL_2$ is a fixed Borel subgroup, the map $\tilde{\sl}_2^\reg/\SL_2 \to B\GG_m$ is given by the composite
$$\tilde{\sl}_2^\reg/\SL_2 \cong \ld{\fr{b}}^\reg/\ld{B} \to B\ld{B}\to B\ld{T} = B\GG_m.$$
However, $\SL_2/N \cong \AA^2 - \{0\}$, and there is an $\SL_2$-equivariant isomorphism $T^\ast(\AA^2)^\reg \cong T^\ast(\AA^2-\{0\})^\reg$. Let us remark that \cref{eq: SO2-equiv Loops S2 periodic} can be de-periodified to give an equivalence
$$\Loc_{\SO(2)}(\Omega S^2; \QQ) \simeq \QCoh(T^\ast[2](\AA^2)^\reg/\SL_2).$$
This is in fact related to the program of Ben-Zvi--Sakellaridis--Venkatesh \cite{sakellaridis-icm} applied to the ``Hecke period''; their program predicts a duality between the Hamiltonian $\PGL_2$-variety $T^\ast(\PGL_2/\GG_m)$ and the Hamiltonian $\SL_2$-variety $T^\ast(\AA^2)$.
\end{remark}
\begin{example}\label{KU-regular-satake}
When $A = \KU$ and $\GG = \GG_m$, we have $\Bun_{\ld{B}}^0(\GG_{0,\QQ}^\vee) = \tilde{\ld{G}}/\ld{G}$. Therefore, \cref{regular-satake} gives an $\E{2}$-monoidal equivalence
$$\QCoh(\tilde{\ld{G}}^\reg/\ld{G}) \simeq
\Loc_{T_c}^\gr (\Omega G_c; \KU) \otimes \QQ.
%\coLMod_{C_\ast^T(\Gr_G(\cc); \KU)} \otimes \QQ.
$$
Note that $\tilde{\ld{G}}/\ld{G}$ is isomorphic to the quotient $\ld{G} \backslash (\ld{G} \times^{\ld{N}} \ld{B})/\ld{T}$; and \cite[Discussion following Proposition 3.10]{safronov-implosion} says that $\ld{G} \times^{\ld{N}} \ld{B}$ is the universal group-valued symplectic implosion (i.e., the symplectic implosion of the internal fusion double of $G$). The relationship of this perspective to Langlands duality is closely related to a quasi-Hamiltonian analogue of the program of Ben-Zvi--Sakellaridis--Venkatesh \cite{sakellaridis-icm}, which we will explore in future work.

Similarly, one can show that there is an $\E{2}$-monoidal equivalence
$$\QCoh(\ld{G}^\reg/\ld{G}) \simeq
\Loc_{G_c}^\gr (\Omega G_c; \KU) \otimes \QQ.
$$
Were there a full $\KU$-theoretic geometric Satake equivalence, the above equivalence would be obtained by localization over the (open) regular locus of $\ld{G}$. The above equivalence is presumably related to \cite[Section 1.2]{cautis-kamnitzer}.
\end{example}
\begin{example}
Suppose $A$ is a complex-oriented even-periodic $\Eoo$-ring and $\GG$ is an oriented elliptic curve over $A$ (in the sense of \cite{elliptic-ii}). Let $E$ be the underlying classical scheme of $\GG$ over the classical ring $\pi_0(A)$, so that $E$ is an elliptic curve, and let $E^\vee$ be the dual elliptic curve. Then $\Bun_{\ld{B}}^0(\GG_0^\vee) = \Bun_{\ld{B}}^0(E^\vee)$, and \cref{regular-satake} gives an $\E{2}$-monoidal $\pi_0 A_\QQ$-linear equivalence
$$\QCoh(\Bun_{\ld{B}}^0(E^\vee)^\reg) \simeq 
\Loc_{T_c}^\gr (\Omega G_c; A) \otimes \QQ.
%\coLMod_{\cf_T(\Gr_G(\cc))^\vee}(\QCoh(\cM_T)) \otimes \QQ.
$$
This may be understood as a step towards a full $A$-theoretic analogue of the ABG equivalence.
\end{example}
\subsection{Coefficients in the sphere spectrum?}\label{sphere-coeffs}

In this brief section, we study the natural question of whether there is an analogue of \cref{intro-mirror-dual-of-g} and \cref{regular-satake} with coefficients in a more general $\Eoo$-ring $R$ (e.g., the sphere spectrum). This is closely related to the discussion in \cref{sec: quantized homology torus}, and already turns out to be rather nontrivial for a torus as soon as $R$ is not complex-orientable. As a warmup, let us make the following observation.
\begin{prop}\label{torus-satake}
Fix a complex-oriented even-periodic $\Eoo$-ring $A$, and let $\GG$ be an oriented group scheme in the sense of \cite{elliptic-ii} which is dualizable. Let $T$ be a torus over $\cc$, and let $\ld{T}_A := \spec A[\bX_\ast(T)]$ denote the dual torus over $A$. Then there is an $\E{2}$-monoidal $A$-linear equivalence $\Shv_T(\Gr_T(\cc); A) \simeq \QCoh(\cM_T \times B\ld{T}_A)$. Fixing an isomorphism $\cM_T \cong \cM_{\ld{T}}$ as in \cref{char-cochar} makes this category equivalent to $\QCoh(\cL_\GG B\ld{T}_A)$.
\end{prop}
\begin{proof}
Note that there is an $\E{2}$-monoidal equivalence $\Shv_T(\Gr_T(\cc); A) \simeq \Shv_{T_c}(\Omega T_c; A)$. Since the $T_c$-action on $\Omega T_c$ is trivial and $\Omega T_c \cong \bX_\ast(T)$ as $\Eoo$-spaces, we obtain an $\E{2}$-monoidal equivalence 
$$\Shv_T(\Gr_T(\cc); A) \simeq \Fun(\bX_\ast(T), \Loc_{T_c}(\ast; A)) \simeq \Fun(\bX_\ast(T), \Mod_A) \otimes_{\Mod_A} \QCoh(\cM_T).$$
The first claim now follows from the equivalence $\QCoh(B\ld{T}_A) \simeq \Fun(\bX_\ast(T), \Mod_A)$. Fixing an isomorphism $\cM_T \cong \cM_{\ld{T}}$ and using that $\cL_\GG B\ld{T}_A \cong B\ld{T}_A \times \cM_{\ld{T}}$, we see that $\Shv_T(\Gr_T(\cc); A)$ can be identified with $\QCoh(\cL_\GG B\ld{T}_A)$, as desired.
\end{proof}
Crucial to the argument of \cref{torus-satake} was the equivalence $\Loc_{T_c}(\ast; A) \simeq \QCoh(\cM_T)$. If $R$ is a general $\Eoo$-ring, then such a statement will generally \textit{only} be true (for an appropriate definition of $\Loc_{T_c}(\ast; R)$) when $R$ is close to being complex-oriented. For example:
\begin{example}
The methods of this article show that there is an analogue of \cref{intro-mirror-dual-of-g} for $\KO$:
$$\Loc_{T_c}^\gr(G_c; \KO) \otimes \QQ \simeq \QCoh((\cM_{\ld{T},0})_\QQ \times_{\Bun_{\ld{B}}^0(\GG_0^\vee)} (\cM_{\ld{T},0})_\QQ).$$
Here, $\GG$ is the universal spectral multiplicative group over $B\Z/2$. Similarly, using the definition of genuine $T$-equivariant $\TMF$ from \cite{t-equiv-tmf}, one can also obtain an analogue of \cref{intro-mirror-dual-of-g} (where $\GG$ is replaced by the universal oriented spectral elliptic curve over the moduli stack of oriented spectral elliptic curves from \cite[Proposition 7.2.10]{elliptic-ii}).
\end{example}
See also \cite[Section 8.1]{mnn-nilpotence-equiv} for a variant of the following:
\begin{example}
Let $\Sp_{T_c}$ denote the $\infty$-category of genuine ${T_c}$-equivariant spectra, and let $i_{T_c}^\ast: \Sp_{T_c} \to \Sp$ be the lax symmetric monoidal right adjoint to the unique symmetric monoidal colimit-preserving functor $\Sp \to \Sp_{T_c}$. Suppose $R$ is an $\Eoo$-ring such that there is an $\Eoo$-algebra $R_{T_c}\in \CAlg(\Sp_{T_c})$ given by ``genuine ${T_c}$-equivariant $R$-cohomology''. Then, $\Loc_{T_c}(\ast; R)$ might be understood to mean $\Mod_{R_{T_c}}(\Sp_{T_c})$. We are interested in the following question: when is $\Loc_{T_c}(\ast; R)$ equivalent (as a symmetric monoidal category) to the $\infty$-category of modules over some $\Eoo$-ring $B$? It is not difficult to see that if this happens, then the $\Eoo$-ring $B$ will simply be $i_{T_c}^\ast(R_{T_c})$. (One could more generally ask when $\Loc_{T_c}(\ast; R)$ is equivalent to the $\infty$-category of quasicoherent sheaves on some spectral $R$-stack; but this obscures the key homotopical point.)

Let us suppose for simplicity that ${T_c}$ is of rank $1$, i.e., that $T_c = S^1$. Recall that the $\infty$-category $\Sp_{S^1}$ is compactly generated by $S^0$ (with the trivial $S^1$-action) and $(S^1/\mu_n)_+$ for $n\geq 2$. If $\lambda$ denote the $1$-dimensional complex representation of $\mu_n$, there is a cofiber sequence $(S^1/\mu_n)_+ \to S^0 \to S^{\lambda^n}$; so $\Sp_{S^1}$ is compactly generated by $S^0$ and $S^{\lambda^n}$ for $n\geq 2$. It follows that $\Loc_{S^1}(\ast; R) \simeq \Mod_{R_{S^1}}(\Sp_{S^1})$ is compactly generated by $R_{S^1}$ and $R_{S^1} \otimes S^{\lambda^n}$ for $n\geq 2$. 
If $R$ is complex-oriented, there is an equivalence $R_{S^1} \otimes S^{\lambda^n} \simeq \Sigma^2 R_{S^1}$. This lets us conclude that $\Loc_{S^1}(\ast; R)$ is compactly generated by the \textit{single} unit object $R_{S^1}$, so that \cite[Lemma 4.4]{greenlees-shipley-fixed-pt} implies that $\Loc_{S^1}(\ast; R) \simeq \Mod_{i_{S^1}^\ast(R_{S^1})}$.
\end{example}
\begin{remark}\label{sptc}
In contrast to the above discussion, if $R$ is not complex-oriented (or more generally does not admit a finite flat cover by a complex-oriented ring), then $\Loc_{T_c}(\ast; R)$ stands little chance of being compactly generated by the unit object. For example, if $R$ is the sphere spectrum, then $\Loc_{T_c}(\ast; R) \simeq \Sp_{T_c}$ is \textit{not} compactly generated by the unit object. Note, however, that the Barr-Beck-Lurie theorem (\cite[Theorem 4.7.3.5]{HA}) implies $\Sp_{S^1}$ is equivalent to the $\infty$-category of left modules over the $\E{1}$-ring $\End_{\Sp_{S^1}}\left(S^0 \oplus \bigoplus_{n\geq 2} (S^1/\mu_n)_+\right)$; this is \textit{not} an $\Eoo$-ring.
\end{remark}
In particular, if $\ld{T}_S := \spec S[\bX_\ast(T)]$ denotes the dual torus over the sphere spectrum, then one can run part of the proof of \cref{torus-satake} to conclude that 
$$\Shv_T(\Gr_T(\cc); S) \simeq \Fun(\bX_\ast(T), \Loc_{T_c}(\ast; S)) \simeq \Fun(\bX_\ast(T), \Sp_{T_c}) \simeq \Sp_{\ld{T}_c} \otimes \QCoh(B\ld{T}_S).$$
Here, we have identified $\Sp_{T_c} \simeq \Sp_{\ld{T}_c}$ (see \cref{char-cochar}). The discussion in \cref{sptc} shows that it is not clear how to view the right-hand side in terms of quasicoherent sheaves on some spectral stack. In particular, we see that already in the case of a torus, the coherent side of ``derived geometric Satake with spherical coefficients'' starts to deviate from the standard form of derived geometric Satake. It seems as though the appropriate analogue of the coherent side involves some combination of Hausmann's global group laws \cite{hausmann-global-group} and the spectral moduli stack of oriented formal groups (see \cite{gregoric-synthetic, piotr-synthetic}). We hope to approach this in future work via $T$-equivariant complex cobordism $\MU_T$.

At the moment, derived geometric Satake with spherical coefficients for a general reductive group over $\cc$ seems to require more technical setup than is currently available in the literature (although a version of the geometric Casselman-Shalika equivalence of \cite{geometric-casselman-shalika-ii} was discussed in \cite[Section 10]{lurie-icm}).
\newpage

%\section{Quantization and loop rotation}
%\input{quantization/classical loop rotation}
%\input{quantization/quantized kostant}

\appendix
\section{Relationship to Brylinski-Zhang}\label{brylinski-zhang-section}
\counterwithin{lemma}{section}
\cref{intro-mirror-dual-of-g} is closely related to the results of Brylinski-Zhang (see \cite{brylinski-zhang}). To explain this, we begin by recasting the results of \cite{brylinski-zhang} in the language of \cref{review-equiv}.
\begin{recall}
Let $G$ be a simply-connected compact Lie group. Then the main result of \cite{brylinski-zhang} says that there is an isomorphism $\KU_G^\ast(G) \cong \Omega^\ast_{K_0(\Rep(G))/\Z} \otimes_\Z \Z[\beta^{\pm 1}]$, where $K_0(\Rep(G))$ is the (complex) representation ring of $G$. If $G$ is not necessarily simply-connected, there is also an isomorphism $\H_G^\ast(G; \QQ) \cong \Omega^\ast_{\H^\ast(BG; \QQ)/\QQ}$.
\end{recall}
These can be simultaneously generalized by the following:
\begin{prop}\label{hh-mg}
Let $A$ be a complex-oriented even-periodic $\Eoo$-ring, and let $\GG$ be an oriented commutative $A$-group. Let $G$ be a simply-connected compact Lie group, and suppose that the functor $\cf_G: \Top(G)_\conn^\op \to \QCoh(\cM_G)$ of $G$-equivariant $A$-cochains
on connected finite $G$-spaces
%following \cite[End of Section 3.5]{survey} (see \cref{nonabelian-equiv-cochains}).
is symmetric monoidal\footnote{Note that this assumption will fail if $G$ is not connected!}.
Then there is an equivalence
$$\Gamma(\cM_G; \cf_G(G)) \simeq \HH(\cM_G/A).$$
\end{prop}
\begin{proof}
Indeed, since $G$ is connected, we have $G \simeq \Omega(BG)$. 
%Therefore, if $G/G$ and $\ast/G$ denote the orbifold quotients, then there is an equivalence 
%$$G/G \simeq \ast/G \times_{\ast/G \times \ast/G} \ast/G$$
%of orbifolds. Here, the maps $\ast/G \rightrightarrows \ast/G \times \ast/G \simeq \ast/(G\times G)$ are both given by the diagonal embedding $G\subseteq G \times G$. (To see this, note that $G/G\simeq G\backslash (G\times G)/G$, where $G\times G$ acts on $G\times G$ via $(g_1, g_2): (h_1, h_2) \mapsto (g_1 h_1 g_2^{-1}, g_1 h_2 g_2^{-1})$.)
Recall from \cref{S1-connections level} that $G/G$ is the free loop space of $\ast/G$ in the category of orbifolds.
The assumption on $\cf_G$ now implies that
$$\cf_G(G) \simeq \cf_G(\ast) \otimes_{\cf_{G\times G}(\ast)} \cf_G(\ast) \simeq \co_{\cM_G} \otimes_{\co_{\cM_G} \otimes \co_{\cM_G}} \co_{\cM_G}.$$
Therefore, $\Gamma(\cM_G; \cf_G(G))$ is precisely the Hochschild homology of $\cM_G$.
\end{proof}
\begin{remark}
One can view $\Gamma(\cM_G; \cf_G(G))$ as endomorphisms of the unit object in $\Loc_G(G; A)$, so that $\Mod_{\Gamma(\cM_G; \cf_G(G))}$ behaves as a completion of $\Loc_G(G; A)$.
\end{remark}
\begin{remark}
In some cases, the Hochschild-Kostant-Rosenberg spectral sequence degenerates integrally. Then, $\pi_\ast \HH(\cM_G/A)$ can be identified with the $2$-periodification of the (derived) Hodge cohomology of the underlying stack of $\cM_G$ over $\pi_0(A)$. This applies, for instance, when $A = \KU$; in this case, $\cM_G$ is a lift to $\KU$ of $\spec K_0(\Rep(G)) \cong T\mmod W$, and \cref{hh-mg} is precisely the calculation of \cite{brylinski-zhang}.
\end{remark}
\begin{remark}\label{A-cohomology of Gr}
\cref{hh-mg} can be continued further to study the $G$-equivariant $A$-\textit{co}homology of $\Omega G$, if we additionally assume that the functor $\cf_G: \Ind(\Top(G))_\conn^\op \to \QCoh(\cM_G)$ on connected \textit{ind-finite} $G$-spaces is symmetric monoidal. Indeed, observe that there is an equivalence
$$G\backslash \Omega G \simeq G\backslash \cL G/G \simeq \ast/G \times_{\ast/\cL G} \ast/G$$
of orbifolds. But $\ast/\cL G \simeq \cL(\ast/G) \simeq \ast/G \times_{\ast/G \times \ast/G} \ast/G$, so that $G\backslash \Omega G \simeq \Map(S^2, \ast/G)$, i.e., the cotensoring of $\ast/G$ by $S^2$ (in \textit{unpointed} orbifolds).
Using the assumption on $\cf_G$, we therefore conclude that the $G$-equivariant $A$-cohomology of $\Omega G$ can be identified with the factorization homology
$$\Gamma(\cM_G; \cf_G(\Omega G)) \simeq \int_{S^2} \cM_G\in \CAlg_A$$
taken internally to $A$-modules.

The preceding discussion also computes the $T$-equivariant $A$-cohomology of $\Omega G$. To explain this, write $p: \cM_T \to \cM_G$ to denote the canonical map. The above discussion shows that
there is an equivalence
$$T\backslash \Omega G \simeq \ast/T \times_{\ast/G \times_{\ast/G \times \ast/G} \ast/G} \ast/G$$
of orbifolds, so that $p_\ast \cf_T(\Omega G)$ can be identified with the factorization homology over $S^2$ of $\cM_G$ with coefficients in the $\E{2}$-module $p_\ast \co_{\cM_T}$. In other words, there is an equivalence
$$\Gamma(\cM_T; \cf_T(\Omega G)) \simeq \int_{S^2} (\cM_G; p_\ast \co_{\cM_T})\in \CAlg_A.$$
\end{remark}
\begin{remark}
This approach is rather robust: for instance, if $K\subseteq G$ is a closed subgroup such that $G/K$ is a finite space, there are equivalences of orbifolds
$$G\backslash \cL(G/K) \simeq \Omega(G/K)/K \simeq (\ast \times_{\ast \times_{\ast/G} \ast/K} \ast)/K \simeq \ast/K \times_{\ast/K \times_{\ast/G} \ast/K} \ast/K.$$
Under the same hypotheses as \cref{A-cohomology of Gr}, this implies that $\Gamma(\cM_G; \cf_G(\cL(G/K)))$ is isomorphic to the relative Hochschild homology $\HH(\cM_K/\cM_G)$. One can recover \cref{A-cohomology of Gr} by noting that if $H$ is a simply-connected compact Lie group and $K = H\subseteq H \times H = G$, the Hochschild homology $\HH(\cM_H/\cM_H \times \cM_H)$ of the diagonal embedding $\Delta: \cM_H \hookrightarrow \cM_H \times \cM_H$ is precisely the factorization homology $\int_{S^2} \cM_H$.
\end{remark}
The relationship of the Brylinski-Zhang isomorphism to \cref{intro-mirror-dual-of-g} can now be explained as follows.
\begin{example}
Continue to assume that $G$ is a simply-connected compact Lie group.
If $A = \QQ[\beta^{\pm 1}]$, then there is an equivalence
$$\Loc_G^\gr(G; \QQ[\beta^{\pm 1}]) \simeq \QCoh(\ld{\fr{t}}\mmod W \times_{\ld{\g}/\ld{G}} \ld{\fr{t}}\mmod W),$$
where all objects on the coherent side are defined over $\QQ$. Since $\ld{\fr{t}}\mmod W \times_{\ld{\g}/\ld{G}} \ld{\fr{t}}\mmod W \cong (T^\ast \ld{T})^\bl\mmod W$ is isomorphic to the group scheme of regular centralizers in $\ld{\g}$, we will write write $\ld{J}_{\GG_a}$ to denote $\ld{\fr{t}}\mmod W \times_{\ld{\g}/\ld{G}} \ld{\fr{t}}\mmod W$. The above equivalence therefore states that
\begin{equation}\label{loc-g-q}
    \Loc_G^\gr(G; \QQ[\beta^{\pm 1}]) \simeq \QCoh(\ld{J}_{\GG_a}).
\end{equation}
On the other hand, by \cite[Theorem 3.4.2]{riche}, the Lie algebra of $\ld{J}_{\GG_a}$ over $\ld{\fr{t}}\mmod W$ is isomorphic to $T^\ast(\ld{\fr{t}}\mmod W)$. Therefore, \cref{hh-mg} and the Hochschild-Kostant-Rosenberg theorem gives an isomorphism
$$\H^\ast_G(G; \QQ[\beta^{\pm 1}]) \cong \pi_\ast \HH(\ld{\fr{t}}\mmod W / \QQ) \otimes_\QQ \QQ[\beta^{\pm 1}] \cong \co_{T[-1](\ld{\fr{t}}\mmod W)} \otimes_\QQ \QQ[\beta^{\pm 1}].$$
In particular, there is an equivalence
\begin{equation}\label{coherent sheaves Q-coh}
    \Mod_{\H_G^0(G; \QQ[\beta^{\pm 1}])} \simeq \QCoh(T[-1](\ld{\fr{t}}\mmod W)) \otimes_\QQ \QQ[\beta^{\pm 1}].
\end{equation}
By Koszul duality, the right-hand side is equivalent to the $2$-periodification of the $\infty$-category of ind-coherent sheaves over the formal completion of $\ld{J}_{\GG_a}$ at the zero section.
One can view the resulting description of $\Mod_{\H_G^0(G; \QQ[\beta^{\pm 1}])}$ as a infinitesimal version of the equivalence \cref{loc-g-q}. By construction, the equivalence \cref{coherent sheaves Q-coh} is just a restatement of the Brylinski-Zhang isomorphism $\H_G^\ast(G; \QQ) \cong \Omega^\ast_{\H^\ast(BG; \QQ)/\QQ}$.
\end{example}

\begin{example}\label{BF coh of gr}
We can also specialize \cref{A-cohomology of Gr} to this case: we have
\begin{equation}\label{fact homology S2}
    \H^\ast_G(\Omega G; \QQ) \cong \pi_\ast \left(\int_{S^2} \cM_G\right).
\end{equation}
Here, $\cM_G = \spec C^\ast(BG; \QQ)$ is the derived $\QQ$-scheme whose underlying graded $\QQ$-scheme is $\ld{\fr{t}}[2]\mmod W = \spec \H^\ast(BG; \QQ)$.
Since $\QQ$ is a field of characteristic zero and $G$ is assumed to be connected, $\H^\ast(BG; \QQ)$ is a polynomial algebra on generators in even degrees; this implies that $C^\ast(BG; \QQ)$ is formal as an $\Eoo$-$\QQ$-algebra\footnote{This follows from the fact that the free $\Eoo$-$\QQ$-algebra on classes in even degrees can be identified with the polynomial $\QQ$-algebra, i.e., is itself formal.}. In particular, we may identify $\cM_G = \ld{\fr{t}}[2]\mmod W$.
Just as the Hochschild homology of $\ld{\fr{t}}[2]\mmod W$ can be identified with the ring of functions on $T[-1](\ld{\fr{t}}[2]\mmod W)$, a version of the Hochschild-Kostant-Rosenberg theorem implies that the factorization homology over $S^2$ can be identified with the ring of functions on the $(-2)$-shifted tangent bundle
$$T[-2](\ld{\fr{t}}[2]\mmod W) = \spec \Sym_{\ld{\fr{t}}[2]\mmod W} (\Omega^1_{\ld{\fr{t}}[2]\mmod W}[2]).$$
Now\footnote{We will not need such a general statement, but we recall it since it is very useful in many other contexts, too.}, if $R$ is a (simplicial) commutative ring and $M$ is a connective $R$-module, there is a d\'ecalage isomorphism\footnote{The d\'ecalage isomorphism only applies to simplicial commutative algebras $R$, and \textit{not} general $\Eoo$-$\Z$-algebras (in part because of issues in defining the derived functors of $\Sym$ and $\Gamma$). This is the reason why we conspicuously shifted from working with coefficients in $\QQ[\beta^{\pm 1}]$ to working with coefficients in $\QQ$. For instance, observe that if $R$ was instead a $\QQ[\beta^{\pm 1}]$-algebra (hence not a simplicial commutative ring) and $M$ is an $R$-module, it is \textit{not possible} to distinguish between $M[2]$ and $M$. Although this might seem like a useless point, the observation that working with $2$-periodic coefficients is inherently destructive is important to clarifying why divided power structures appear in the $G$-equivariant cohomology of $\Omega G$ when one considers more general coefficients.} (see \cite[Sec. I.4.3.2]{illusie-decalage}) $\Sym^j_R(M[2]) \cong \Gamma^j_R(M)[2j]$, where $\Gamma^j$ denotes (the left derived functor of) the $j$th divided power construction.
Therefore, we see that $\Sym_{\ld{\fr{t}}[2]\mmod W} (\Omega^1_{\ld{\fr{t}}[2]\mmod W}[2])$ can be identified with a shearing (which we will simply denote by $[2\bull]$) of the divided power algebra $\Gamma_{\ld{\fr{t}}[2]\mmod W} (\Omega^1_{\ld{\fr{t}}[2]\mmod W})$. In other words, there is an isomorphism
$$\pi_\ast \left(\int_{S^2} \ld{\fr{t}}[2]\mmod W\right) \cong \Gamma_{\ld{\fr{t}}[2]\mmod W} (\Omega^1_{\ld{\fr{t}}[2]\mmod W})[2\bull];$$
the shearing on the right-hand side is undone by $2$-periodifying the left-hand side. Therefore, we obtain an isomorphism
$$\H^\ast_G(\Omega G; \QQ) \otimes_\QQ \QQ[\beta^{\pm 1}] \cong \Gamma_{\ld{\fr{t}}\mmod W} (\Omega^1_{\ld{\fr{t}}\mmod W}).$$
Up to this point, the fact that the coefficients are $\QQ$ (as opposed to a general $\Z$-algebra with some small primes inverted) has not been used outside of the formality of $C^\ast(BG; \QQ)$. Using it now, we see that the divided power algebra can be identified with a symmetric algebra, in which case the above formula implies that $\H^\ast_G(\Omega G; \QQ) \otimes_\QQ \QQ[\beta^{\pm 1}]$ can be identified with the ring of functions on the tangent bundle $T(\ld{\fr{t}}\mmod W)$. This should be compared to \cite[Theorem 1]{bf-derived-satake} with $\hbar = 0$; see \cite[Section 2.6]{bf-derived-satake} and \cite[Section 1.7]{ginzburg-langlands}. A similar argument using the $S^1$-action on $S^2$ by rotation can be used to recover (the $2$-periodification of) the full quantized statement of \cite[Theorem 1]{bf-derived-satake}.
\end{example}
\begin{remark}
The above discussion implies a more general statement. Namely, suppose that $R$ is a (classical) commutative ring such that \cref{A-cohomology of Gr} applies to $G$-equivariant $R$-cohomology --- in particular, such that there is an isomorphism
\begin{equation}\label{general fact homology S2}
    \H^\ast_G(\Omega G; R) \cong \pi_\ast \left(\int_{S^2} \cM_G\right) \in \CAlg_{\pi_\ast R}
\end{equation}
as in \cref{fact homology S2}. (This assumption is likely to hold for rather general rings $R$.) As usual, $\cM_G = \spec C^\ast(BG; R)$ is an $\Eoo$-$R$-scheme with underlying graded $R$-scheme $\ld{\fr{t}}_R[2]\mmod W$; here, $\ld{\fr{t}}_R$ denotes the base-change of $\ld{\fr{t}}$ from $\Z$ to $R$. Suppose that $C^\ast(BG; R)$ is formal as an $\E{n}$-$R$-algebra (i.e., there is an equivalence $C^\ast(BG; R) \simeq \H^\ast(BG; R)$ as $\E{n}$-$R$-algebras); by obstruction theory, this can always be guaranteed if $n=2$ and $\H^\ast(BG; R)$ is a polynomial algebra on generators in even degrees. Then \cref{general fact homology S2} implies that $\H^\ast_G(\Omega G; R)$ is equivalent to $\pi_\ast \left(\int_{S^2} \ld{\fr{t}}_R[2]\mmod W\right)$ as $\E{n-2}$-$R$-algebras. In particular, since $C^\ast(BG; R)$ is formal as an $\E{2}$-$R$-algebra, we see that $\H^\ast_G(\Omega G; R)$ is equivalent to $\pi_\ast \left(\int_{S^2} \ld{\fr{t}}_R[2]\mmod W\right)$ as unital $R$-modules. If $C^\ast(BG; R)$ is formal as an $\E{3}$-$R$-algebra, then we can also identify $\H^\ast_G(\Omega G; R)$ as an \textit{$R$-algebra}.

In any case, since $R$ is not necessarily a $\QQ$-algebra, the Hochschild-Kostant-Rosenberg theorem need not give an isomorphism between $\pi_\ast \left(\int_{S^2} \ld{\fr{t}}_R[2]\mmod W\right)$ and $\Sym_{\ld{\fr{t}}[2]\mmod W} (\Omega^1_{\ld{\fr{t}}[2]\mmod W}[2])$; rather, there will always be a ``HKR'' filtration on $\pi_\ast \left(\int_{S^2} \ld{\fr{t}}_R[2]\mmod W\right)$ whose associated graded is given by $\Sym_{\ld{\fr{t}}[2]\mmod W} (\Omega^1_{\ld{\fr{t}}[2]\mmod W}[2])$. 
If this filtration splits, we conclude that the cohomology ring $\H^\ast_G(\Omega G; R)$ will admit divided powers on the $\co_{\ld{\fr{t}}_R[2]\mmod W}$-algebra generators $\Omega^1_{\ld{\fr{t}}_R[2]\mmod W}$. 
The assumption that the HKR filtration splits seems likely to hold if some primes are assumed to be units in $R$ (e.g., if $\dim(\fr{t})!\in R^\times$). Note that by virtue of the argument establishing \cref{general fact homology S2}, the divided power structure on $\H^\ast_G(\Omega G; R)$ is closely related to the $\E{3}$-algebra structure on the derived Satake category.

The preceding discussion is directly connected with a question asked by Bezrukavnikov about divided powers in the cohomology of the affine Grassmannian (see \cite{roman-MO}). It would be interesting to determine the exact conditions under which the above assumptions on $R$ hold true (namely, $C^\ast(BG; R)$ being formal as an $\E{3}$-$R$-algebra, \cref{general fact homology S2}, and the splitting of the HKR filtration for $\int_{S^2} \ld{\fr{t}}_R[2]\mmod W$). The formality of $C^\ast(BG; R)$ seems to be the thorniest of these conditions, but we nevertheless hope that \cref{general fact homology S2} could be useful in approaching Bezrukavnikov's question.
\end{remark}

\begin{example}
Recall that there is an equivalence
$$\Loc_{G}^\gr(G; \KU) \otimes \QQ \simeq \QCoh(\ld{T}\mmod W \times_{\ld{G}/\ld{G}} \ld{T}\mmod W),$$
where all objects on the coherent side are defined over $\QQ$. Since $\ld{T}\mmod W \times_{\ld{G}/\ld{G}} \ld{T}\mmod W \cong (T \times \ld{T})^\bl\mmod W$ is isomorphic to the group scheme of regular centralizers in $\ld{G}$, we will write write $\ld{J}_{\GG_m}$ to denote $\ld{T}\mmod W \times_{\ld{G}/\ld{G}} \ld{T}\mmod W$.
The above equivalence therefore states that
\begin{equation}\label{loc-g-ku}
    \Loc_G^\gr(G; \KU) \otimes \QQ \simeq \QCoh(\ld{J}_{\GG_m}).
\end{equation}
There is a multiplicative analogue of \cite[Theorem 3.4.2]{riche}, which says that when $\ld{G}$ is adjoint, the Lie algebra of $\ld{J}_{\GG_m}$ over $\ld{T}\mmod W$ is isomorphic to $T^\ast(\ld{T}\mmod W)$. Therefore, \cref{hh-mg} and the Hochschild-Kostant-Rosenberg theorem gives an isomorphism
$$\KU^\ast_G(G) \otimes \QQ \cong \pi_\ast \HH(\ld{T}\mmod W / \Z) \otimes_\Z \QQ[\beta^{\pm 1}] \cong \co_{T[-1](\ld{T}\mmod W)} \otimes_\Z \QQ[\beta^{\pm 1}].$$
In particular, there is an equivalence
\begin{equation}\label{coherent sheaves KU-coh}
    \Mod_{\KU_G^0(G)} \otimes \QQ \simeq \QCoh(T[-1](\ld{T}\mmod W)) \otimes_\Z \QQ.
\end{equation}
By Koszul duality, the right-hand side is equivalent to the $2$-periodification of the $\infty$-category of ind-coherent sheaves over the formal completion of $\ld{J}_{\GG_m}$ at the zero section.
One can view the resulting description of $\Mod_{\KU_G^0(G)} \otimes \QQ$ as a infinitesimal version of the equivalence \cref{loc-g-ku}. By construction, the equivalence \cref{coherent sheaves KU-coh} is just a restatement of the Brylinski-Zhang isomorphism $\KU_G^0(G) \otimes \QQ \cong \Omega^\ast_{K_0(\Rep(G))/\Z} \otimes_\Z \QQ$.
\end{example}
\begin{remark}
Just as in \cref{BF coh of gr}, we can also specialize \cref{A-cohomology of Gr} to the case of K-theory. Then, we have
\begin{equation}\label{Kthy fact homology S2}
    \KU^\ast_G(\Omega G) \cong \pi_\ast \left(\int_{S^2} \cM_G\right).
\end{equation}
Here, $\cM_G = \spec \KU_G$ as a $\KU$-scheme, and the factorization homology is taken over $\KU$. 
%The right-hand side above has a filtration (induced by the double-speed Postnikov filtration on $\co_{\cM_G} = \KU_G$) whose graded pieces are given by the $2$-periodification of $\int_{S^2} \ld{T}\mmod W$, where the factorization homology is taken over $\Z$. As in \cref{BF coh of gr}, $\int_{S^2} \ld{T}\mmod W$ can be identified with the $2$-periodification of the divided power algebra $\Gamma_{\ld{T}\mmod W} (\Omega^1_{\ld{T}\mmod W})$. Upon rationalization, we conclude that $\KU^\ast_G(\Omega G) \otimes \QQ$ has a filtration whose graded pieces are given by the $2$-periodification of the ring of functions on the tangent bundle $T(\ld{T}\mmod W)$.
\end{remark}
In general, $\Mod_{\pi_0 C_{G}^\ast(G; A)} \otimes \QQ$ is an ``infinitesimal analogue'' of $\Loc_{G}^\gr(G; A)$. The equivalence of \cref{hh-mg} can therefore be viewed as a infinitesimal version of the analogue of the equivalence of \cref{intro-mirror-dual-of-g} for $\Loc_{G}^\gr(G; A) \otimes \QQ$.
\newpage
\section{Coulomb branches of pure supersymmetric gauge theories}\label{coulomb}
In this brief appendix, we explain some motivation for the results of this article from the perspective of Coulomb branches of $4$d $\cN=2$ and $5$d $\cN=1$ gauge theories with a generic choice of complex structure. The goal here is not to be precise, but instead explain some motivation for the ideas in this article. While reading this appendix, the reader should keep in mind that I know very little physics!
In \cite{bfn-ii, nakajima-coulomb} (see also \cite{nakajima-intro}), it is argued that the Coulomb branch of $3$d $\cN=4$ pure gauge theory on $\RR^3$ can be modeled by the algebraic symplectic variety $\cM_C := \spec \H^{G_c}_\ast(\Gr_G; \cc)$ over $\cc$. This is in turn isomorphic by \cite[Theorem 3]{bf-derived-satake} (reproved here as \cref{cor: loop-rot Gr and biWhit}) to the phase space of the Toda lattice for $\ld{G}$, as well as (by \cite[Theorem A.1]{bfn-ii}) to the moduli space of solutions of Nahm's equations on $[-1,1]$ for a compact form of $\ld{G}$ with an appropriate boundary condition.
The \textit{quantized} Coulomb branch of $3$d $\cN=4$ pure gauge theory on $\RR^3$ is then modeled by $\cA_\epsilon := \H^{G_c\times S^1_\rot}_\ast(\Gr_G; \cc)$. Note that $\cA_\epsilon$ is isomorphic to the algebra of operators of the quantized Toda lattice for $\ld{G}$.

The physical reason for the definition of $\cA_\epsilon$ is the ``$\Omega$-background'' (introduced in \cite{nek-shat}); we refer the reader to \cite{ben-zvi-susy, teleman-icm} for helpful expositions on this topic. The essential idea is as follows: the equivariant homology $C^G_\ast(\Gr_G; \cc)$ admits the structure of an $\Efr{3}$-algebra. In particular, the $\E{3}$-algebra structure on $C^G_\ast(\Gr_G; \cc)$ is equivariant for the action of $S^1$ on $C^G_\ast(\Gr_G; \cc)$ via loop rotation, and the action of $S^1$ on $\E{3}$ via rotation about a line $\ell\subseteq \RR^3$. Using the fact that the fixed points of the $S^1$-action on $\RR^3$ are given by the line $\ell$, it is argued in \cite{ben-zvi-susy} that the homotopy fixed points of $C^G_\ast(\Gr_G; \cc)$ admits the structure of an $\E{1}$-$C_{S^1}^\ast(\ast; \cc)$-algebra. Furthermore, the associative multiplication on $C^{G_c\times S^1_\rot}_\ast(\Gr_G; \cc)$ degenerates to the $2$-shifted Poisson bracket on $\H^{G_c}_\ast(\Gr_G; \cc)$ obtained from the $\E{3}$-algebra structure. The ``$\Omega$-background'' is supposed to refer to the compatibility of the $S^1$-action on $C^G_\ast(\Gr_G; \cc)$ with the $S^1$-action on the $\E{3}$-operad.

From the mathematical perspective, the idea that $S^1$-actions can be viewed as deformation quantizations has been made precise by \cite{preygel, toen-icm}, and more recently in \cite{butson-i, butson-ii}, at least in characteristic zero.  Although often not said explicitly, the idea has been a cornerstone of the development of Hochschild homology and its relatives. (The reader can skip the following discussion, since it will not be necessary in the remainder of this section; we only include it for completeness.) 

Consider a smooth $\cc$-scheme $X$, so that the Hochschild-Kostant-Rosenberg theorem gives an isomorphism $\HH(X/\cc) \simeq \Sym(\Omega^1_{X/\cc}[1])$. There is an isomorphism $\Sym(\Omega^1_{X/\cc}[1]) \simeq \bigoplus_{n\geq 0} (\wedge^n \Omega^1_{X/\cc})[n]$, so $\Sym(\Omega^1_{X/\cc}[1])$ can be understood as a shearing of the algebra $\Omega^\ast_{X/\cc} = \bigoplus_{n\geq 0} (\wedge^n \Omega^1_{X/\cc})[-n]$ of differential forms. The Hochschild-Kostant-Rosenberg theorem further states that the $S^1$-action on $\HH(X/\cc)$ is a shearing of the de Rham differential on $\Omega^\ast_{X/\cc}$. 
    
The Koszul dual of the algebra $\HH(X/\cc) \simeq \Sym(\Omega^1_{X/\cc}[1])$ is $\Sym(T_{X/\cc}[-2]) \simeq \co_{T^\ast[2] X}$; in the same way, the sheaf of differential operators on $X$ is Koszul dual to the de Rham complex of $X$. This can be drawn pictorially as follows:
$$\xymatrix{
\Sym(T_{X/\cc}[-2]) \simeq \co_{T^\ast[2] X} \ar@{~>}[r]^-{\text{def. quant}} \ar@{~>}[d]_-{\text{Koszul dual}} & \cd^\hbar_{X/\cc} \ar@{~>}[d]^-{\text{Koszul dual}} \\
\Sym_{\co_X}(\Omega^1_{X/\cc}[1]) \simeq \HH(X/\cc) \ar@{~>}[r]_-{S^1\text{-action}} & \text{shearing of }(\Omega^\ast_{X/k}, d_\dR).
}$$
Since the algebra $\cd_X^\hbar$ of differential operators is a quantization of $T^\ast[2] X$, this diagram illustrates the idea that the $S^1$-action on Hochschild homology plays the role of a Koszul dual to deformation quantization.

\begin{example}\label{ex: 3d-sl2}
    We will keep $G = \PGL_2$ as a running example in discussing Coulomb branches (see also \cite[Section 2]{seiberg-witten-coulomb}), so that $\ld{G} = \SL_2$. In this case,
    $$\cM_C \cong \spec \cc[x, a^{\pm 1}, \tfrac{a-a^{-1}}{x}]^{\Z/2} \cong \spec \cc[x^2, a+a^{-1}, \tfrac{a-a^{-1}}{x}]$$
    by \cref{thm: ordinary hmlgy reg centr}, where $\Z/2$ acts on $\cc[x, a^{\pm 1}, \tfrac{a-a^{-1}}{x}]$ by $x\mapsto -x$ and $a\mapsto a^{-1}$. 
    This is the regular centralizer group scheme of $\SL_2$. %atiyah hitchin 
    Let us denote by 
    \begin{align*}
        \Phi & = x^2,  \\
        U & = a + a^{-1}, \\
        V & = \tfrac{a-a^{-1}}{x}.
    \end{align*}
    Then we have the single relation
    $$U^2 - \Phi V^2 = (a + a^{-1})^2 - (a - a^{-1})^2 = 4,$$
    so $\cM_C$ is isomorphic to the subvariety of $\AA^3_\cc$ cut out by the above equation. 
    %Alternatively, and perhaps more suggestively, this equation can be rewritten as follows:
    %$$(U+2)(U-2) = \Phi V^2.$$
    This is known as the \textit{Atiyah-Hitchin manifold}, and was studied in great detail in \cite{atiyah-hitchin} (see \cite[Page 20]{atiyah-hitchin} for the definition). In \cite[Theorem A.1]{bfn-ii}, it was shown that the Atiyah-Hitchin manifold is isomorphic to the moduli space of solutions of Nahm's equations on $[-1,1]$ for $\mathrm{SU}(2)$ with an appropriate boundary condition.

    Since a normal vector to the defining equation of $\cM_C$ is $2U\partial_U - V^2 \partial_\phi - 2V\Phi \partial_V$, the standard holomorphic $3$-form $dU \wedge d\Phi \wedge dV$ on $\AA^3_\cc$ induces a holomorphic symplectic form $\tfrac{d\Phi \wedge dV}{2U}$ on $\cM_C$. (This can also be written as $\tfrac{dU \wedge dV}{V^2}$ or as $\tfrac{d\Phi \wedge dU}{2\Phi V}$.) The associated Poisson bracket on $\co_{\cM_C} \cong \H^{G_c}_\ast(\Gr_G; \cc)$ agrees with the $2$-shifted Poisson bracket arising from the $\E{3}$-structure on $C^{G_c}_\ast(\Gr_G; \cc)$.

    The quantized algebra $\cA_\epsilon$ can be described explicitly as follows. Let us write $\theta = \tfrac{1}{x}(s-1)$, where $s$ is the simple reflection generating the Weyl group of $\SL_2$. Then $\cA_\epsilon$ is generated as an algebra over $\cc\pw{\hbar}$ by $\Z/2$-invariant polynomials in $x$, $a^{\pm 1}$, and $\theta$, where $x$ is to be viewed as $a\partial_a$. Moreover, under the isomorphism $\cA_\epsilon/\hbar \cong \co_{\cM_C}$, the class $x$ is sent to $x$, and $\theta$ is sent to $\tfrac{a-1}{x}$. We then have the commutation relation $[x,a^{\pm 1}] = \pm \hbar a^{-1}$, induced by $[\partial_a, a] = \hbar$; see \cref{ex: ordinary quantized diffop}. This implies that 
    $$[x^2, a^{\pm 1}] = \hbar^2 a^{\pm 1} \pm 2\hbar a^{\pm 1} x,$$
    which in turn implies that $\cA_\epsilon$ is the quotient of the free associative $\cc\pw{\hbar}$-algebra on $\Phi$, $U$, and $V = \tfrac{1}{x}(a-a^{-1})$ subject to the relations
    \begin{align*}
        [\Phi,V] & = 2\hbar U - \hbar^2 V,\\
        [\Phi, U] & = 2\hbar \Phi V - \hbar^2 U, \\
        [U,V] & = \hbar V^2,\\
        U^2 - 4 & = \Phi V^2 - \hbar UV.
    \end{align*}
    Note that the commutation relations for $[\Phi, U]$ and $[U,V]$ in \cite[Equation B.3]{dimofte-garner} have typos, but it is stated correctly in \cite[Equation 5.51]{bullimore-dimofte-gaiotto}. 
\end{example}
\begin{example}
    When $G = \SL_2$, we can identify $\cM_C$ with the quotient of the scheme of \cref{ex: 3d-sl2} by the free $\Z/2$-action sending $U\mapsto -U$ and $V\mapsto -V$; so
    $$\cM_C \cong \spec \cc[x^2, (a+a^{-1})^2, \left(\tfrac{a-a^{-1}}{2x}\right)^2, \tfrac{(a+a^{-1})(a-a^{-1})}{2x}].$$
    This is the regular centralizer group scheme for $\PGL_2$. Note that if we denote
    \begin{align*}
        \Phi & = x^2,\\
        A & = (a+a^{-1})^2,\\
        B & = 4\left(\tfrac{a-a^{-1}}{2x}\right)^2 = \tfrac{(a-a^{-1})^2}{x^2},\\
        C & = 2\tfrac{(a+a^{-1})(a-a^{-1})}{2x} = \tfrac{(a+a^{-1})(a-a^{-1})}{x},
    \end{align*}
    then we have relations
    \begin{align*}
        AB & = C^2,\\
        A - \Phi B & = 4.
    \end{align*}
    In particular, $\cM_C$ is cut out in $\AA^3_\cc$ (with coordinates $\Phi$, $B$, and $C$) via the equation
    $$C^2 - \Phi B^2 = 4B.$$
    Note the similarity to the manifold from \cref{ex: 3d-sl2}: in fact, it is the quotient of the aforementioned manifold by the free $\Z/2$-action sending $U\mapsto -U$ and $V\mapsto -V$. In terms of these coordinates, $B = V^2$ and $C = UV$. (Sometimes, this quotient is also referred to as the Atiyah-Hitchin manifold.) It is also possible to describe $\cA_\epsilon$; we leave this to the reader, since it is rather tedious.
\end{example}

\begin{heuristic}\label{heuristic: 4d-n2}
    An unpublished conjecture of Gaiotto (which I learned about from Nakajima) says that the Coulomb branch of $4$d $\cN=2$ pure gauge theory over $\RR^3 \times S^1$ with a generic choice of complex structure can be modeled by $\cM_C^\fourd := \spec \KU^{G_c}_0(\Gr_G) \otimes_\Z \cc$. Although I do not know Gaiotto's motivation for this conjecture (it is probably inspired by \cite{seiberg-witten-coulomb}), my attempt at heuristically justifying it goes as follows. (In \cite[Appendix C(b)]{ku-rel-langlands}, I suggest that it might be slightly better to consider $\spec \ku^{G_c}_\ast(\Gr_G) \otimes_\Z \cc$ instead, where $\ku$ denotes \textit{connective} complex K-theory. The Bott class generating $\pi_2(\ku)$ plays the role of the radius of the circle $S^1$.)
    
    Recall that $\Gr_G/G\pw{t}$ can be viewed as $\Bun_G(S^2)$. It is reasonable to view $\KU_0(\Bun_G(S^2)) \otimes \cc$ as closely related to $\H_\ast(L\Bun_G(S^2); \cc)$, where $L\Bun_G(S^2)$ denotes the topological free loop space of $\Bun_G(S^2)$. Since $LBG \simeq BLG$, we have $L\Bun_G(S^2) \simeq \Bun_{LG}(S^2)$, so one might view $\H_\ast(L\Bun_G(S^2); \cc)$ as the ring of functions on the ``Coulomb branch of $3$d $\cN=4$ pure gauge theory on $\RR^3$ with gauge group $LG$''.

    Making precise sense of this phrase seems difficult, but one possible workaround could be the following. It is often useful to view gauge theory with gauge group $LG$ as ``finite temperature'' gauge theory with gauge group $G$. Recall that Wick rotation relates $(3+1)$-dimensional quantum field theory at a finite temperature $T$ to statistical mechanics over $\RR^3 \times S^1$ where the circle has radius $\tfrac{1}{2\pi T}$. This suggests that $\H_\ast(L\Bun_G(S^2); \cc)$ (which is more precisely to be replaced by $\KU^{G_c}_0(\Gr_G) \otimes \cc$) can be viewed as the ring of functions on the ``Coulomb branch of $4$d $\cN=2$ pure gauge theory on $\RR^3 \times S^1$ with gauge group $G$''. See \cite[Remark 3.14]{bfn-ii}. In \cite{bfm}, $\spec \KU^{G_c}_0(\Gr_G) \otimes \cc$ was identified with the phase space of the relativistic Toda lattice for $\ld{G}$.

    One can also define a quantization of $\cM_C^\fourd$ via $\cA_\epsilon^\fourd := \KU^{G_c \times S^1_\rot}_0(\Gr_G) \otimes \cc$; this can be viewed as a model for the quantized Coulomb branch of $4$d $\cN=2$ pure gauge theory on $\RR^3\times S^1$. The algebra $\cA_\epsilon^\fourd$ can be identified with the algebra of operators of the quantized relativistic Toda lattice for $\ld{G}$.
\end{heuristic}
\begin{example}\label{ex: 4d-sl2}
    When $G = \PGL_2$, \cref{thm: ku hmlgy reg centr} tells us that 
    $$\cM_C^\fourd \cong \spec \cc[x^{\pm 1}, a^{\pm 1}, \tfrac{a-a^{-1}}{x-1}]^{\Z/2} \cong \spec \cc[x+x^{-1}, a+a^{-1}, \tfrac{(a-a^{-1})(x+1)}{x-1}],$$
    where $\Z/2$ acts by $x\mapsto x^{-1}$ and $a\mapsto a^{-1}$.
    For simplicity, let us consider instead a slight variant of $\cM_C^\fourd = \spec \KU^{\mathrm{PSU}(2)}_0(\Gr_{\PGL_2}) \otimes_\Z \cc$, given by ${\cM'}_C^\fourd = \spec \KU^{\SU(2)}_0(\Gr_{\PGL_2}) \otimes_\Z \cc$. Then
    $${\cM'}_C^\fourd \cong \spec \cc[x^{\pm 1}, a^{\pm 1}, \tfrac{a-a^{-1}}{x-x^{-1}}]^{\Z/2} \cong \spec \cc[x+x^{-1}, a+a^{-1}, \tfrac{a-a^{-1}}{x-x^{-1}}].$$
    Let us write $\Psi = x + x^{-1}$, $W = a+a^{-1}$, and $Z = \tfrac{a-a^{-1}}{x-x^{-1}}$. Then, one easily verifies that ${\cM'}_C^\fourd$ is the subvariety of $\AA^3_\cc$ cut out by the equation
    $$W^2 - (\Psi^2 - 4)Z^2 = 4.$$
    %Alternatively, and perhaps more suggestively:
    %$$(W+2)(W-2) = (\Psi+2)(\Psi-2)Z^2.$$
    This may be regarded as a multiplicative analogue of the Atiyah-Hitchin manifold. It would be very interesting to understand a relationship between this manifold and the moduli space of solutions to some analogue of Nahm's equations for $\mathrm{PSU}(2)$ with an appropriate boundary condition. The complex manifold ${\cM'}_C^\fourd$ has a holomorphic symplectic form given by $\tfrac{d\Psi \wedge dZ}{W}$, which can also be written as $\tfrac{d\Psi \wedge dW}{(\Psi^2-4)Z}$ or as $\tfrac{dZ \wedge dW}{\Psi Z^2}$.

    It is also possible to explicitly describe the quantized algebra $\cA_\epsilon^\fourd$. The resulting description is not very enlightening, so we will only indicate how one reaches the answer. In this case, instead of the relation $[\partial_a, a] = \hbar$ which appeared in \cref{ex: 3d-sl2}, we have the relation $xa = qax$ (i.e., $xax^{-1} a^{-1} = q$); see \cref{ex: q quantized diffop}. In particular, $xa^{-1} = q^{-1} a^{-1} x$, $x^{-1} a = q^{-1} a x^{-1}$, and $x^{-1} a^{-1} = qa^{-1}x^{-1}$. It follows after some tedious calculation that $\cA_\epsilon^\fourd$ is the quotient of the free associative $\cc[q^{\pm 1}]$-algebra on $\Psi$, $W$, and $\tfrac{x+1}{x-1} (a-a^{-1})$ subject to four relations.

    Suppose we consider instead the variant of $\cA_\epsilon^\fourd$ defined by ${\cA'}_\epsilon^\fourd = \KU^{\SU(2) \times S^1_\rot}_0(\Gr_{\PGL_2}) \otimes \cc$.  Then ${\cA'}_\epsilon^\fourd$ is the quotient of the free associative $\cc[q^{\pm 1}]$-algebra on $\Psi$, $W$, and $Z = \tfrac{1}{x-x^{-1}} (a-a^{-1})$ subject to the relations
    \begin{align*}
        [\Psi, W] & = (q-1)(\Psi^2-4)Z - \tfrac{(q-1)^2}{2q} ((\Psi^2-4)Z + \Psi W), \\
        [\Psi, Z] & = (q-1)W - \tfrac{(q-1)^2}{2q}(\Psi Z + W),\\
        [Z,W] & = (q-1) \Psi Z^2 - \tfrac{(q-1)^2}{2q} (\Psi Z + W)Z, \\
        W^2 - 4 & = (\Psi^2-4)Z^2 - \tfrac{(q-1)^2}{2q} (\Psi^2-4)Z^2 + \tfrac{q^2-1}{2q} \Psi WZ.
    \end{align*}
\end{example}
\begin{remark}
    As always, one can also replace $\KU$ by connective complex K-theory $\ku$. This introduces a new ``Bott'' parameter $\beta$ which recovers $\KU$ when $\beta$ is set to $1$ (informally speaking), and recovers ordinary cohomology when $\beta$ is killed. In \cite[Appendix C(b)]{ku-rel-langlands}, I have suggested that working with $\ku$ instead of $\KU$ in \cref{heuristic: 4d-n2}  should produce the ``Coulomb branch of $4$d $\cN=2$ pure gauge theory on $\RR^3 \times S^1$ with gauge group $G$'' where the Bott parameter $\beta$ identifies with the radius of the circle $S^1$.
    
    In the case of \cref{ex: 4d-sl2}, the resulting manifold ${\cM'}_C^\beta = \spec \ku^{\SU(2)}_\ast(\Gr_{\PGL_2}) \otimes_{\Z[\beta]} \cc[\beta]$ is given by
    $${\cM'}_C^\beta \cong \spec \cc[\beta, x, \tfrac{1}{1+\beta x}, a^{\pm 1}, \tfrac{a-a^{-1}}{x-\ol{x}}]^{\Z/2} \cong \spec \cc[x+\ol{x}, a+a^{-1}, \tfrac{a-a^{-1}}{x-\ol{x}}].$$
    Here, $\ol{x}$ is the inverse of $x$ in the group law $x + y + \beta xy$, so that $\ol{x} = -\tfrac{x}{1+\beta x}$. 
    Let us write $\Psi' = x + \ol{x}$, $W = a+a^{-1}$, and $Z = \tfrac{a-a^{-1}}{x-\ol{x}}$; then ${\cM'}_C^\beta$ is the subvariety of $\AA^3_\cc \times \AA^1_\beta$ cut out by the equation
    $$W^2 - (\beta^2 \Psi' - 4)\Psi' Z^2 = 4.$$
    The complex manifold ${\cM'}_C^\beta$ has a holomorphic symplectic form given by $\tfrac{d\Psi' \wedge dZ}{W}$, which can also be written as $\tfrac{d\Psi' \wedge dW}{2Z\Psi' (\beta^2 \Psi'-4)}$ or as $\tfrac{dZ \wedge dW}{2Z^2(\beta^2 \Psi' - 2)}$.
    When $\beta$ is set to $1$, one can identify $\Psi'$ with $\Psi + 2$ with $\Psi$ as in \cref{ex: 4d-sl2}; then ${\cM'}_C^\beta$ recovers the multiplicative Atiyah-Hitchin manifold. When $\beta$ is killed, ${\cM'}_C^\beta$ is the usual Atiyah-Hitchin manifold. Again, it would be very interesting to understand a relationship between ${\cM'}_C^\beta$ and the moduli space of solutions to some analogue of Nahm's equations for $\mathrm{PSU}(2)$ with an appropriate boundary condition.  It is also possible to compute the loop-rotation equivariant version of ${\cM'}_C^\beta$, but we leave this to the reader.
\end{remark}
\begin{example}
    When $G = \SL_2$, one can view $\cM_C^\fourd$ with the quotient of the scheme ${\cM'}_C^\fourd$ of \cref{ex: 4d-sl2} by the free $\Z/2$-action sending $W\mapsto -W$ and $Z\mapsto -Z$; so
    $$\cM_C^\fourd \cong \spec \cc[x + x^{-1}, (a + a^{-1})^2, \left(\tfrac{a - a^{-1}}{x - x^{-1}}\right)^2, \tfrac{(a - a^{-1})(a + a^{-1})}{x - x^{-1}}].$$
    This is the regular centralizer group scheme for $\PGL_2$. Note that if we denote
    \begin{align*}
        \Psi & = x + x^{-1},\\
        A & = (a+a^{-1})^2 = a^2 + a^{-2} + 2,\\
        B & = \left(\tfrac{a-a^{-1}}{x-x^{-1}}\right)^2 = \tfrac{a^2+a^{-2}-2}{x^2 + x^{-2} - 2},\\
        C & = \tfrac{(a+a^{-1})(a-a^{-1})}{x - x^{-1}} = \tfrac{a^2-a^{-2}}{x-x^{-1}},
    \end{align*}
    then we have relations
    \begin{align*}
        AB & = C^2,\\
        A - (\Psi^2 - 4) B & = 4.
    \end{align*}
    In particular, $\cM_C^\fourd$ is cut out in $\AA^3_\cc$ (with coordinates $\Psi$, $B$, and $C$) via the equation
    $$C^2 - (\Phi^2 - 4) B^2 = 4B.$$
    Note the similarity to \cref{ex: 4d-sl2}. It is also possible to describe $\cA_\epsilon^\fourd$; again, we leave this to the reader, since it is rather tedious.

    Let us again note the variant involving connective complex K-theory $\ku$: in this case, 
    $$\cM_C^\beta \cong \spec \cc[\beta, x + \ol{x}, (a + a^{-1})^2, \left(\tfrac{a - a^{-1}}{x - \ol{x}}\right)^2, \tfrac{(a - a^{-1})(a + a^{-1})}{x - \ol{x}}].$$
    If we denote
    \begin{align*}
        \Psi' & = x + \ol{x},\\
        A & = (a+a^{-1})^2 = a^2 + a^{-2} + 2,\\
        B' & = \left(\tfrac{a-a^{-1}}{x-\ol{x}}\right)^2 = \tfrac{a^2+a^{-2}-2}{x^2 + \ol{x}^2 - 2x\ol{x}},\\
        C' & = \tfrac{(a+a^{-1})(a-a^{-1})}{x - \ol{x}} = \tfrac{a^2-a^{-2}}{x-\ol{x}},
    \end{align*}
    then we have relations
    \begin{align*}
        AB' & = {C'}^2,\\
        A - (\beta^2 \Psi' - 4)\Psi' B & = 4.
    \end{align*}
    In particular, $\cM_C^\beta$ is cut out in $\AA^3_\cc \times \AA^1_\beta$ (with coordinates $\Psi$, $B$, $C$, and $\beta$) via the equation
    $$C^2 - (\beta^2 \Psi' - 4)\Psi' B^2 = 4B.$$
\end{example}

Consider an elliptic curve $E(\cc)$ over $\cc$. Motivated by \cref{heuristic: 4d-n2} and \cite{nakajima-yoshioka}, one might expect that (in some specific complex structure) the Coulomb branch of $5$d $\cN=1$ pure gauge theory over $\RR^3 \times E(\cc)$ can be modeled by the complexification of the $G_c$-equivariant $k$-homology of $\Gr_G$, where $k$ is an elliptic cohomology theory associated to a putative integral lift of $E$. Unfortunately, a classical result of Tate says that there are no smooth elliptic curves over $\Z$ (see \cite{ogg-ell-curve-over-Z} for an elementary proof); so $E(\cc)$ cannot literally lift to $\Z$ (i.e., $\pi_0(k)$ cannot be $\Z$).

As a fix, one can more generally simultaneously consider all possible ``Coulomb branches'' $\cM_C^\fived := \spec \pi_0 \cf_G(\Gr_G)^\vee \otimes \cc$ associated to every complex-oriented $2$-periodic $\Eoo$-ring $k$ equipped with an oriented elliptic curve (this is very nearly the same as considering the universal example $\spec \tmf^{G_c}_0(\Gr_G) \otimes \cc$). We have described $\spec \pi_0 \cf_T(\Gr_G)^\vee \otimes \cc$ in \cref{thm: elliptic hmlgy reg centr}, from which one can calculate $\cM_C^\fived$. Similarly, one can even use \cref{thm: ell loop-rot flag} to calculate $\pi_0 \cf_{T\times \GG_m^\rot}(\Gr_G)^\vee \otimes \cc$ and $\cA_\epsilon^\fived := \pi_0 \cf_{G\times \GG_m^\rot}(\Gr_G)^\vee \otimes \cc$, but this is already incredibly complicated for $G = \SL_2$.

It would be very interesting to give a physical interpretation to $\pi_0 \cf_G(\Gr_G)^\vee \otimes \cc$ and $\pi_0 \cf_{G\times \GG_m^\rot}(\Gr_G)^\vee \otimes \cc$ for other $2$-periodic $\Eoo$-rings $k$, although we expect this to be very difficult. Indeed, most other chromatically interesting generalized equivariant cohomology theories only exist after profinite or $p$-adic completion, and do not admit transcendental analogues; but see \cref{rmk: morava e-theory}. It would also be very interesting to describe the analogue of our calculations for the ind-schemes $\cR_{G,\mathbf{N}}$ introduced in \cite{bfn-ii}. By adapting the methods of \cite[Section 4]{bfn-ii}, this is easy when $G$ is a torus. We expect it to lead to interesting geometry for nonabelian $G$. In \cite{ku-rel-langlands}, we extend the discussion of this paper (at least, the parts concerning ordinary cohomology and K-theory) to connective K-theory, and suggest an analogue of the relative Langlands program of \cite{bzsv} in this setting; as mentioned in the introduction of \cite{ku-rel-langlands}, the story therein also admits an elliptic variant.
\newpage
%\section{Algebraic obstructions to mapping from polynomial rings}\label{k1-local}
%\input{appendix/k1-local stuff}
%\newpage
%\section{Examples of $\DD(\GG)$}\label{cartier-fourier}
%\input{appendix/cartier fourier mukai}
%\newpage

\bibliographystyle{alpha}
%\bibliographystyle{plain}
\bibliography{main}
\end{document}