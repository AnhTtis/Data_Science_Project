\subsection{Kac-Moody flag varieties}

Fix a complex-oriented even-periodic $\Eoo$-ring $A$ and an oriented commutative $A$-group $\GG$.
\begin{observe}\label{gkm-kac-moody-assumption}
Let $\cg$ be a Kac-Moody group, and let $\cP\subseteq \cg$ be a parabolic subgroup associated to a subset $J\subseteq \Delta$ of simple roots. Let $T = T_\cg/Z(\cg)$ denote the torus of $\cg/Z(\cg)$, and let $W$ be the Weyl group associated to $\cg$. Let $W_\cP$ denote the subgroup of $W$ generated by $s_{\alpha_j}$ for $\alpha_j\in J$, and let $W^J$ denote the set of minimal-length coset representatives in $W_\cg/W_\cP$.

Then $(\cg/\cP)^T \cong W^\cP$, and the Schubert decomposition $\cg/\cP = \coprod_{w\in W^\cP} \cB\dot{w}\cP/\cP$ is a $T$-invariant stratification, where $\ol{w} = \dot{w}\cP/\cP$ is the unique $T$-fixed point in the cell $\cB\dot{w}\cP/\cP$. We claim that $\cg/\cP$ satisfies the hypotheses of \cref{gkm-assumption}. Clearly, condition (a) is satisfied. For condition (b), observe that the tangent space to $\cB\ol{w}$ at $\ol{w}$ is 
$$T_{\ol{w}} \cB\dot{w} \cP/\cP = \fr{b}/(\fr{b} \cap w\cdot \fr{p}) = \bigoplus_{\alpha\in \Phi^+ - w\Phi^+(\fr{p})} \fr{g}_\alpha,$$
where each $\fr{g}_\alpha$ is $1$-dimensional. The weights are therefore all distinct, so condition (b) in \cref{gkm-assumption} is satisfied. For condition (c), let $\alpha\in \Phi^+ - w\Phi^+(\fr{p})$, and let $i_\alpha: \SL_2 \to \cg$ denote the associated subgroup. The closure of $\cB_\alpha\ol{w}$ is $\SL_2\ol{w} = \PP^1$, where the point at $0$ corresponds to $\ol{w}$, and the point at $\infty$ corresponds to $\ol{s_\alpha w}$. Then the GKM graph $\Gamma$ of $\cg/\cP$ has vertices $W^\cP$ and edges $w \to s_\alpha w$ labeled by $s_\alpha\in W_\cg$. See also \cite[Section 5]{generalized-gkm}.
\end{observe}
\begin{warning}\label{warning homology}
In the following, the reader should replace the symbol ``$\cf_T(\cg/\cP)$'' by $\cf_T(X_{\leq w})$ where $X_{\leq w}$ is a Schubert cell in $\cg/\cP$. In this case, $X_{\leq w}$ is a finite CW-complex, so that $\cf_T(X_{\leq w})$ is a \textit{perfect} $\co_{\cM_T}$-module. This implies that the $T$-equivariant \textit{homology} $\cf_T(X_{\leq w})^\vee$ is the $\co_{\cM_T}$-linear dual of $\cf_T(X_{\leq w})$; note that this is not true of $\cf_T(\cg/\cP)$ when the Kac-Moody group is not of finite type. (In general, homology is a predual of cohomology, but the linear dual of cohomology does not recover homology in the non-finite case.) We \textit{define} $\cf_T(\cg/\cP)^\vee$ as the direct limit of $\cf_T(X_{\leq w})^\vee$. 
\end{warning}
Since $\cg/\cP$ satisfies the hypotheses of \cref{gkm-assumption} by \cref{gkm-kac-moody-assumption}, we may apply \cref{gkm-main} to calculate $\cf_T(\cg/\cP)$. See \cite{k-thy-schubert-grg} for a related discussion.
\begin{theorem}\label{kac-moody-gkm}
The following diagram is an equalizer on $\pi_0$:
$$\cf_T(\cg/\cP) \to \Map(W^\cP, \co_{\cM_T}) \rightrightarrows \prod_{\alpha: w \to s_\alpha w} \co_{\cM_{T_\alpha}}.$$
Here, the two maps are given by restriction and applying $s_\alpha$ to $W^\cP$.
Therefore, $\pi_0 \cf_T(\cg/\cP)$ is the sub-$\pi_0 \co_{\cM_T}$-algebra of $\Map(W^\cP, \pi_0 \co_{\cM_T})$ consisting of those maps $f: W^\cP \to \pi_0 \co_{\cM_T}$ such that 
\begin{equation}\label{gkm-condition}
    f(s_\alpha w) \equiv f(w) \pmod{\cI_\alpha} \text{ for all }w \in W^\cP, \alpha\in \Phi.
\end{equation}
\end{theorem}
Motivated by \cref{kac-moody-gkm}, we may define an algebraic generalization of $\pi_0 \cf_T(\cg/\cP)$ as follows.
\begin{construction}\label{coxeter-system}
Let $(W,S)$ be a Coxeter system, and let $V = \RR^S$ denote the associated geometric representation. For $s\in S$, let $\alpha_s$ denote the associated vector, let $\Phi = \{w(\alpha_s) | s\in S, w\in W\}$ be the set of roots, and let $\Phi^+\subseteq \Phi$ denote the set of positive roots. Let $\Lambda = \Z\Phi \subseteq V$ denote the associated root lattice. Fix a smooth $1$-dimensional affine group scheme $\GG_0$ over a commutative ring $R$, and let $\cM_{T,0} = \Hom(\Lambda^\vee, \GG_0)$. Given a character $\lambda$, let $c_\lambda$ denote a function which cuts out the closed subscheme $\GG_{\ker(\lambda)} \hookrightarrow \cM_{T,0}$.
Define $\KK$ to be the sub-$\co_{\cM_{T,0}}$-algebra of $\Map(W, \co_{\cM_{T,0}})$ consisting of those maps $f: W \to \co_{\cM_{T,0}}$ satisfying \cref{gkm-condition}, i.e., such that $f(s_\alpha w) \equiv f(w) \pmod{c_\alpha}$ for $\alpha\in \Phi$ and $w\in W$.
\end{construction}
\begin{remark}
Note that if $\lambda$ is a character, then the function $c_\lambda$ on $\cM_T$ is given by the $T$-equivariant Thom class of the representation of $T$ given by $\lambda: T \to \GG_m^\rot$. Morever, $c_\lambda$ generates $\cI_\lambda$.
\end{remark}
\begin{lemma}\label{salpha-equiv-1}
Let $s_\alpha\in W$, and let $T_\alpha = \ker(\alpha)\subseteq T$. Then we have the following commutative diagram of $R$-schemes (where the non-vertical arrows are closed immersions):
$$\xymatrix{
\cM_{T_\alpha,0} \ar[r]^-q \ar[dr]_-q & \cM_{T,0} \ar[d]^-{s_\alpha} \\
& \cM_{T,0};
}$$
informally, $s_\alpha \equiv 1\pmod{\cI_\alpha}$.
\end{lemma}
\begin{proof}
This follows from the fact that the character lattice of $T_\alpha$ is the quotient of $\bX^\ast(T)$ by the rank $1$ sublattice generated by $\alpha$; therefore, if $\chi\in \bX^\ast(T)$, then $s_\alpha \chi|_{T_\alpha} = \chi|_{T_\alpha}$.
\end{proof}
\cref{kac-moody-gkm} implies the following:
\begin{corollary}
Suppose $\GG_0$ is affine. Then there is an equivalence $\pi_0 \cf_T(\cg/\cP)^\vee \simeq \co_{\cM_{T,0}}[W^\cP, \tfrac{s_\alpha-1}{c_\alpha}, \alpha\in \Phi]$ of $\pi_0 \co_{\cM_T}$-modules.
\end{corollary}
Recall that if $w\in W$, then $\inv(w)\subseteq \Phi^+$ denotes the set of positive roots $\alpha$ such that $s_\alpha w < w$. The following is then the analogue of \cite[Lemma 2.3, Lemma 2.5, Proposition 2.6]{k-thy-schubert-grg}.
\begin{prop}\label{basis-cohomology}
Suppose that $\GG$ is affine.
In \cref{coxeter-system}, $\KK$ is a free $\co_{\cM_{T,0}}$-module spanned by functions $\psi_w: W \to \co_{\cM_{T,0}}$ for $w\in W$, where $\psi_w$ is uniquely characterized by the property that it satisfies \cref{gkm-condition} and the following two properties:
\begin{align*}
    \psi_w(v) & = 0 \text{ if }v<w,\\
    \psi_w(w) & = \prod_{\alpha\in \inv(w)} c_\alpha.
\end{align*}
\end{prop}
\begin{proof}
The two stated conditions define $\psi_w$ on the interval $[1, w]\subseteq W$. We will now define an extension of $\psi_w$ to the whole of $W$. We will in fact prove the following more general claim by induction on $\ell(w)$:
\begin{enumerate}
    \item[$(\ast)$] Let $w\in W$, and let $[1,w]^\circ = [1,w]-\{w\}$. Then any function $\psi: [1,w]^\circ \to \co_{\cM_{T,0}}$ satisfying \cref{gkm-condition} extends to a function $[1,w] \to \co_{\cM_{T,0}}$ satisfying \cref{gkm-condition}. 
\end{enumerate}

To see this, write $w = s_{i_1} \cdots s_{i_n}$, let $\alpha = \alpha_{i_1}$, and let $w' = s_\alpha w$ (so that $w'<w$). Consider the restriction of $\psi$ to $[1, w']^\circ$, so that $\psi$ itself is an extension to $[1,w']$. Define $\psi': [1,w']^\circ \to \co_{\cM_{T,0}}$ by the formula $\psi'(v) = s_\alpha \psi(s_\alpha v)$. Then $\psi'$ also satisfies \cref{gkm-condition}: indeed, if $\beta$ is another root, then $\psi'(s_\beta v) \equiv \psi'(v) \pmod{\cI_\beta}$ if and only if $\psi(s_\alpha s_\beta v) \equiv \psi(s_\alpha v) \pmod{s_\alpha \cI_\beta}$. However, $s_\alpha \cI_\beta = \cI_{s_\alpha(\beta)}$, while $s_\alpha s_\beta = s_{s_\alpha(\beta)} s_\alpha$. The claim therefore follows from the assumption that $\psi$ satisfies \cref{gkm-condition}.

Since $w'<w$, the inductive hypothesis says that $\psi'$ extends to a function $\psi': [1,w'] \to \co_{\cM_{T,0}}$ which satisfies \cref{gkm-condition}. If $v\in [1,w']^\circ$, then
$$\psi(v) - \psi'(v) = \psi(v) - s_\alpha \psi(s_\alpha v) \equiv (1 - s_\alpha) \psi(v) \pmod{\cI_\alpha}.$$
By \cref{salpha-equiv-1}, we see that $\psi(v) - \psi'(v) \equiv 0\pmod{\cI_\alpha}$, so we may define a function $p_v\in \co_{\cM_{T,0}}$ by the formula $\frac{\psi(v) - \psi'(v)}{c_\alpha}$. If $\beta\in \Phi^+$ is such that $s_\beta w'<w'$, then:
\begin{align*}
    \psi(w') - \psi'(w') & \equiv \psi(s_\beta w') - \psi'(s_\beta w') \pmod{\cI_\beta} \\
    & = c_\alpha p_{s_\beta w'} \pmod{\cI_\beta}.
\end{align*}
In particular, there is a function $p_{w'} \in \co_{\cM_{T,0}}$ such that 
$$\psi(w') - \psi'(w') \equiv c_\alpha p_{w'} \pmod{\cI_\beta}$$
for all $\beta\in \Phi^+$ such that $s_\beta w'<w'$, i.e., $\beta\in \inv(w')$. In particular, 
\begin{equation}\label{pw-prime}
    \psi(w') - \psi'(w') \equiv c_\alpha p_{w'} \pmod{\prod_{\beta\in \inv(w')} \cI_\beta}.
\end{equation}
Note that $s_\alpha \inv(w')$ is the set of $\beta \in \Phi^+-\{\alpha\}$ such that $s_\beta w'<w'$. Define
$$\psi(w) = s_\alpha \psi'(w') + x\prod_{\beta\in s_\alpha \inv(w')} c_\beta$$
for some $x$ that we will determine in a moment. We check that $\psi$ satisfies \cref{gkm-condition}. Let $\alpha'\in \Phi^+$ be such that $s_{\alpha'} w<w$. Then:
\begin{enumerate}
    \item If $\alpha' = \alpha$, then
    \begin{align*}
        \psi(w) - \psi(s_\alpha w) & = s_\alpha \psi'(w') - \psi(w') + x\prod_{\beta\in s_\alpha \inv(w')} c_\beta \\
        & \equiv s_\alpha (\psi'(w') - \psi(w')) + x\prod_{\beta\in s_\alpha\inv(w')} c_\beta \pmod{\cI_\alpha}
    \end{align*}
    However, \cref{pw-prime} implies that 
    $$s_\alpha (\psi(w') - \psi'(w')) \equiv c_{-\alpha} s_\alpha(p_{w'}) \pmod{\prod_{\beta\in s_\alpha \inv(w')} \cI_\beta}$$
    Therefore, taking $x$ to be the negative of the residue of $s_\alpha (\psi(w') - \psi'(w')) - c_{-\alpha} s_\alpha(p_{w'})$ modulo $\prod_{\beta\in s_\alpha \inv(w')} \cI_\beta$, we see that
    \begin{align*}
        \psi(w) - \psi(s_\alpha w) & \equiv c_{-\alpha} s_\alpha(p_{w'}) \equiv 0 \pmod{\cI_\alpha},
    \end{align*}
    as desired.
    \item If $\alpha' \neq \alpha$, then $\alpha'\in s_\alpha \inv(w')$. Then, we have
    \begin{align*}
        \psi'(w') & \equiv \psi'(s_{s_\alpha(\alpha')} w') \pmod{\cI_{s_\alpha(\alpha')}} \\
        & = s_\alpha \psi(s_\alpha s_{s_\alpha(\alpha')} s_\alpha w) \pmod{\cI_{s_\alpha(\alpha')}}\\
        & = s_\alpha \psi(s_{\alpha'} w) \pmod{\cI_{s_\alpha(\alpha')}}.
    \end{align*}
    In particular, $s_\alpha \psi'(w') \equiv \psi(s_{\alpha'} w) \pmod{\cI_{\alpha'}}$. But this implies that
    \begin{align*}
        \psi(w) - \psi(s_{\alpha'} w) & \equiv s_\alpha \psi'(w') - \psi(s_{\alpha'} w) \pmod{\cI_{\alpha'}}\\
        & \equiv 0 \pmod{\cI_{\alpha'}},
    \end{align*}
    as desired.
\end{enumerate}
This finishes the proof of $(\ast)$.

To finish the proof of the proposition, note that the two conditions on $\psi_w$ specify it on $[1,w]$, and hence on the subset of $W$ consisting of elements of length $<\ell(w)$. By $(\ast)$, we may inductively extend $\psi_w$ to the subset of $W$ consisting of elements of length $\geq \ell(w)$, and hence to all of $W$. It remains to show that any $\psi\in \Map(W, \co_{\cM_{T,0}})$ satisfying \cref{gkm-condition} can be written as a $\co_{\cM_{T,0}}$-linear combination of the $\psi_w$; see the second half of \cite[Proposition 2.6]{k-thy-schubert-grg} for the following argument.

Let $\supp(\psi)$ denote the subset of $w\in W$ such that $f(\psi)\neq 0$. Let $v\in \supp(\psi)$ be minimal. If $\alpha\in \inv(v)$ (so $s_\alpha v<v$), then $\psi(v) \equiv \psi(s_\alpha v) = 0\pmod{\cI_\alpha}$. This implies that $\psi(v) \equiv 0 \pmod{\psi_v(v)}$. Define $\psi': W \to \pi_0 \co_{\cM_{T,0}}$ by $\psi'(w) = \psi(w) - \tfrac{\psi(v)}{\psi_v(v)} \psi_v(w)$; then $\psi'$ satisfies \cref{gkm-condition} (since $\psi$ and $\psi_v$ do). By construction, $v\not\in \supp(\psi')$, and $\supp(\psi')-\supp(\psi)$ consists of elements which are strictly larger than $v$. Therefore, we may repeat this argument for $\psi'$, and induct; this yields the desired result.
\end{proof}

