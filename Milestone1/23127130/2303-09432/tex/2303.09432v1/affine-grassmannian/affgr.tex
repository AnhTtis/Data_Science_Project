\subsection{The affine Grassmannian}
\begin{setup}
Fix notation as in \cref{group-notation}, and assume that $G$ is semisimple. Then we have an associated affine root datum: the affine simple roots are $\Delta_\aff = \Delta \cup\{0\}$, and the affine weight lattice is given by $\Z K \oplus \bigoplus_{\alpha_i\in \Delta_\aff} \Z \alpha_i$. (In particular, we denote the affine root by $\alpha_0$.) Thus the associated Kac-Moody algebra is $\hat{\g} = \g\ls{t} \oplus \cc \alpha_0 \oplus \cc K$, where $K$ is the central class, and $\alpha_0$ is the scaling factor.
Let $\cg$ denote the associated Kac-Moody group, and let $W^\aff = \Lambda^\vee \rtimes W$ denote the associated affine Weyl group. If $\lambda^\vee \in \Lambda^\vee$, we write $t_{\lambda^\vee}$ to denote the associated element of $W^\aff$. If $\alpha + n\alpha_0$ is an affine root and $x\in \fr{t}$, then
$$s_{\alpha+n\alpha_0}(x) = x - (\langle x, \alpha\rangle + n) \alpha^\vee = s_\alpha(x) + n\alpha^\vee.$$
Let $\cB$ denote the Iwahori subgroup, and $T_\aff$ the maximal torus of $\cg$. Then $\cg/\cB$ is the affine flag variety $\Fl_G$; similarly, $\Gr_G$ is the Kac-Moody flag variety associated to the subset $\Delta\subseteq \Delta_\aff$. Up to keeping track of the central torus, we may view $\cg$ as $G\ls{t}$, and $\cB$ as the Iwahori $I$. Thus $T = T^\aff \cap G$ is the maximal torus of $G$. Let $\tilde{T}$ denote the extended torus $T\times \GG_m^\rot$ (where $\GG_m^\rot$ is the loop rotation torus); we may identify its Lie algebra $\tilde{\fr{t}}$ with $\fr{t} \oplus \cc \alpha_0$.
\end{setup}
\begin{remark}
Let $\alpha\in \Phi$ and $n\in \Z$. Then $n\alpha_0$ is the $\GG_m^\rot$-representation of weight $n$. 
%Write $[n](\hbar_A)$ to denote the local section of $\pi_0 \co_{\cM_{\GG_m^\rot}} \cong \pi_0 \co_\GG$ given by the Thom class of the line bundle $\cf_{\GG_m^\rot}(S^{n\alpha_0})$ on $\cM_{\GG_m^\rot} \simeq \GG$, so that it is $\hbar_A +_\GG \cdots +_\GG \hbar_A$.
Note that $\alpha + n\alpha_0$ defines an ideal sheaf $\cI_{\alpha + n\alpha_0} \subseteq \pi_0 \co_{\cM_{\tilde{T}}} = \pi_0 \co_{\cM_T} \otimes_{\pi_0 A} \pi_0 \co_\GG$.
\end{remark}
\cref{kac-moody-gkm} gives an explicit description of $\pi_0 \cf_{T^\aff}(\Fl_G)$ and $\pi_0 \cf_{T^\aff}(\Gr_G)$. Using that 
\begin{align*}
(\Fl_G)^T & = (\Fl_G)^{\tilde{T}} = W^\aff\\
(\Gr_G)^T & = (\Gr_G)^{\tilde{T}} = W^\aff/W \cong \Lambda^\vee,
\end{align*}
this further immediately specializes to the following explicit description of $\pi_0 \cf_{\tilde{T}}(\Fl_G)$ and $\pi_0 \cf_{\tilde{T}}(\Gr_G)$:
\begin{corollary}\label{loop-rot-coh-grg}
The following statements are true:
\begin{enumerate}
    \item We may identify $\pi_0 \cf_{\tilde{T}}(\Fl_G) \cong \pi_0 \cf_{\GG_m^\rot}(\Fl_G/I)$ with $\KK$ from \cref{coxeter-system}, i.e., as the sub-$\pi_0 \co_{\cM_{\tilde{T}}}$-algebra of $\Map(W^\aff, \pi_0 \co_{\cM_{\tilde{T}}})$ consisting of those maps $f: W^\aff \to \pi_0 \co_{\cM_{\tilde{T}}}$ such that 
    \begin{equation}
    f(s_{\alpha+n\alpha_0}(w)) \equiv f(w) \pmod{\cI_{\alpha + n\alpha_0}}
    \end{equation}
    for all $w \in W^\aff, \alpha\in \Phi, n\in \Z$.
    \item We may identify $\pi_0 \cf_{\tilde{T}}(\Gr_G) \cong \pi_0 \cf_{\GG_m^\rot}(\Gr_G/I)$ as the sub-$\pi_0 \co_{\cM_{\tilde{T}}}$-algebra of $\Map(\Lambda^\vee, \pi_0 \co_{\cM_{\tilde{T}}})$ consisting of those maps $f: \Lambda^\vee \to \pi_0 \co_{\cM_{\tilde{T}}}$ such that 
    \begin{equation}
    f(s_{\alpha+n\alpha_0}(\lambda)) \equiv f(\lambda) \pmod{\cI_{\alpha + n\alpha_0}}
    \end{equation}
    for all $\lambda \in \Lambda^\vee, \alpha\in \Phi, n\in \Z$.
\end{enumerate}
\end{corollary}
\begin{corollary}\label{cohomology-grg}
The following statements are true:
\begin{enumerate}
    \item We may identify $\pi_0 \cf_T(\Fl_G)$ as the sub-$\pi_0 \co_{\cM_T}$-algebra of $\Map(W^\aff, \pi_0 \co_{\cM_T})$ consisting of those maps $f: W^\aff \to \pi_0 \co_{\cM_T}$ such that 
    \begin{equation}
    f(s_{\alpha+n\alpha_0}(w)) \equiv f(w) \pmod{\cI_\alpha}
    \end{equation}
    for all $w \in W^\aff, \alpha\in \Phi, n\in \Z$.
    \item We may identify $\pi_0 \cf_T(\Gr_G)$ as the sub-$\pi_0 \co_{\cM_T}$-algebra of $\Map(\Lambda^\vee, \pi_0 \co_{\cM_T})$ consisting of those maps $f: \Lambda^\vee \to \pi_0 \co_{\cM_T}$ such that 
    \begin{equation}\label{gkm-grg}
    f(s_{\alpha+n\alpha_0}(\lambda)) \equiv f(\lambda) \pmod{\cI_\alpha}
    \end{equation}
    for all $\lambda \in \Lambda^\vee, \alpha\in \Phi, n\in \Z$.
\end{enumerate}
\end{corollary}
\begin{observe}
The image of $s_{\alpha+n\alpha_0}$ under the identification $W^\aff/W \cong \Lambda^\vee$ is the right coset $s_{\alpha+n\alpha_0} W$. However, $s_{\alpha+n\alpha_0} s_\alpha$ is translation by $n\alpha^\vee$. If $k$ is a commutative ring, we may view $k[\Lambda^\vee]$ as the $\Eoo$-ring of functions on $\ld{T}_k$; the element $n\alpha^\vee\in \Lambda^\vee$ corresponds to the function $e^{n\alpha^\vee}$. Therefore, \cref{gkm-grg} can be restated as 
$$f((e^{n\alpha^\vee}-1)(\lambda)) \equiv 0 \pmod{\cI_\alpha}.$$
If $\GG$ is affine, then $\pi_0 \cf_T(\Gr_G)$ is the $\pi_0 \co_{\cM_T}$-linear dual of $\pi_0 \co_{\cM_T}[\Lambda^\vee][\tfrac{e^{n\alpha^\vee}-1}{c_\alpha}]_{n\geq 1}$. However, note that for any $n\geq 1$, we may write
%use the multiplicative formal group law to obtain $\tfrac{e^{n\alpha^\vee}-1}{c_\alpha}$ from $\tfrac{e^{\alpha^\vee}-1}{c_\alpha}$.
$$\tfrac{e^{n\alpha^\vee}-1}{c_\alpha} = \tfrac{e^{\alpha^\vee}-1}{c_\alpha} + \tfrac{e^{(n-1)\alpha^\vee}-1}{c_\alpha} + c_\alpha \tfrac{e^{\alpha^\vee}-1}{c_\alpha} \tfrac{e^{(n-1)\alpha^\vee}-1}{c_\alpha}.$$
This implies that
$$\pi_0 \cf_T(\Gr_G) \cong \Map_{\QCoh(\cM_{T,0})}(\pi_0 \co_{\cM_T}[\Lambda^\vee][\tfrac{e^{\alpha^\vee}-1}{c_\alpha}], \pi_0 \co_{\cM_T}).$$
\end{observe}
\begin{remark}
Let $\lambda\in \Lambda^{\vee,\pos}$ be a dominant coweight, and let $\Lambda^{\vee,\pos}_{\leq \lambda}$ denote the subset of $\Lambda^{\vee,\pos}$ consisting of those dominant weights which are at most $\lambda$. Then we may identify
$$(\Gr_G^{\leq \lambda})^T = W\cdot \Lambda^{\vee,\pos}_{\leq \lambda} \subseteq \Lambda^\vee = (\Gr_G)^T,$$
which allows us to calculate that if $\GG$ is affine, then
$$\pi_0 \cf_T(\Gr_G^{\leq \lambda}) \cong \Map_{\QCoh(\cM_{T,0})}(\pi_0 \co_{\cM_T}[W\cdot \Lambda^{\vee,\pos}_{\leq \lambda}][\tfrac{e^{\alpha^\vee}-1}{c_\alpha}], \pi_0 \co_{\cM_T}).$$
In the above expression, $\alpha$ ranges over $\Phi \cap W\cdot \Lambda^{\vee,\pos}_{\leq \lambda}$; in other words, $\alpha$ is of the form $w\alpha_i$ with $\alpha_i\in \Delta$ such that $\alpha_i\leq \lambda$.
\end{remark}
\begin{remark}\label{caveat filtration}
Recall from \cref{warning homology} that $\cf_T(\Gr_G)^\vee$ is defined to be the direct limit of $\cf_T(\Gr_G^{\leq \lambda})^\vee$.
We trust the reader to make the appropriate modifications below as needed (which we have not done to avoid an overbearance of notation), so that the calculation of the $T$-equivariant homology $\cf_T(\Gr_G)^\vee$ in \cref{t-homology-grg} by taking the linear dual of $\cf_T(\Gr_G)$ does not suffer from completion issues. This can be done, for instance, by working with the \textit{$\Lambda^{\vee,\pos}$-filtered} $\co_{\cM_T}$-module $\{\cf_T(\Gr_G^{\leq \lambda})^\vee\}$.
In order for the colimit $\cf_T(\Gr_G)^\vee$ of the $\Lambda^{\vee,\pos}$-filtered module $\{\cf_T(\Gr_G^{\leq \lambda})^\vee\}$ to admit the structure of an $\E{2}$-$\co_{\cM_T}$-algebra, it suffices to show that $\{\cf_T(\Gr_G^{\leq \lambda})^\vee\}$ admits the structure of an $\E{2}$-algebra in $\Lambda^{\vee,\pos}$-filtered module; this is proved in \cref{filtered E2} below.
\end{remark}
\begin{lemma}\label{filtered E2}
The $\Lambda^{\vee,\pos}$-indexed Schubert filtration $\{\Gr_G^{\leq \lambda}(\cc)\}$ naturally admits the structure of an $\E{2}$-algebra in $\Fun(\Lambda^{\vee,\pos}, \Top)$.
\end{lemma}
\begin{proof}
This can be proved in essentially the same way as \cite[Theorem 3.10]{hahn-yuan}; let us sketch the argument. We will utilize \cite[Proposition 5.4.5.15]{HA}, which states that if $\cC$ is a symmetric monoidal $\infty$-category, then a nonunital $\E{2}$-algebra object in $\cC$ is equivalent to the datum of a locally constant $\mathrm{N}(\mathrm{Disk}(\cc))_\mathrm{nu}$-algebra object in $\cC$. Concretely, this amounts to specifying an object $A(D)\in \cC$ for every disk $D\subseteq \cc$ and coherent maps $\bigotimes_{i=1}^n A(D_i)\to A(D)$ for every inclusion $\coprod_{i=1}^n D_i\to D$ of disks, such that for every embedding $D\subseteq D'$ of disks, the induced map $A(D)\to A(D')$ is an equivalence.

In this case, $\cC = \Fun(\Lambda^{\vee,\pos}, \Top)$, and the object $A(D)\in \Fun(\Lambda^{\vee,\pos}, \Top)$ assigned to a disk $D\subseteq \cc$ may be defined via the Beilinson-Drinfeld Grassmannian $\Gr_{G,\Ran}$. Namely, the Beilinson-Drinfeld Grassmannian admits (by construction) a morphism $\Gr_{G, \Ran} \to \Ran_{\AA^1}$; upon taking complex points, we obtain a map $\Gr_{G, \Ran}(\cc) \to \Ran(\cc)$. If $S\subseteq \cc$ is a subset, then the preimage of $\Ran(S)\subseteq \Ran(\cc)$ defines a subspace $\Gr_{G, \Ran}(S\subseteq \cc)\subseteq \Gr_{G, \Ran}(\cc)$. The filtration of $\Gr_G$ via the Bruhat decomposition extends to a filtration $\Gr_{G, \Ran, \leq \mu}$ of $\Gr_{G, \Ran}$ by dominant coweights $\mu\in \Lambda^{\vee,\pos}$; see \cite[3.1.11]{zhu-grass}. Finally, the object $A(D)\in \Fun(\Lambda^{\vee,\pos}, \Top)$ associated to a disk $D\subseteq \cc$ is the functor $\Lambda^{\vee,\pos}\to \Top$ sending $\mu\in \Lambda^{\vee,\pos}$ to $\Gr_{G, \Ran, \leq \mu}(D\subseteq \cc)$.

Suppose $\coprod_{i=1}^n D_i\to D$ is an inclusion of disks. The induced map $\bigotimes_{i=1}^n A(D_i)\to A(D)$ is defined as follows. Let $\mu\in \Lambda^{\vee,\pos}$; for every $n$-tuple $(\mu_1, \cdots, \mu_n)$ with $\sum_{i=1}^n \mu_i\leq \mu$, we need to exhibit maps $\bigotimes_{i=1}^n A(D_i)(\mu_i)\to A(D)(\mu)$ satisfying the obvious coherences. But
$$\bigotimes_{i=1}^n A(D_i)(\mu_i) = \prod_{i=1}^n \Gr_{G, \Ran, \leq \mu_i}(D_i\subseteq \cc),$$
so it suffices to show that if $\mu_1 + \mu_2 \leq \mu$, then there are maps $\Gr_{G, \Ran, \leq \mu_1}(D_1\subseteq \cc) \times \Gr_{G, \Ran, \leq \mu_2}(D_2\subseteq \cc)\to \Gr_{G, \Ran, \leq \mu}(D\subseteq \cc)$. The argument for this is exactly as in \cite[Construction 3.15]{hahn-yuan}.

We next need to show that the $\mathrm{N}(\mathrm{Disk}(\cc))_\mathrm{nu}$-algebra $A$ defined above is locally constant, i.e., that if $D\subseteq D'$ is an embedding of disks, then $A(D)\to A(D')$ is an equivalence of functors $\Lambda^{\vee,\pos}\to \Top$. This follows from \cite[Proposition 3.17]{hahn-yuan}. To conclude, it suffices (by \cite[Theorem 5.4.4.5]{HA}) to establish the existence of a quasi-unit for the functor $A:\Lambda^{\vee,\pos}\to \Top$, i.e., a map $1_{\Fun(\Lambda^{\vee,\pos}, \Top)}\to A$ which is both a left and right unit up to homotopy. Since the unit in $\Fun(\Lambda^{\vee,\pos}, \Top)$ is the functor sending $\mu\in \Lambda^{\vee,\pos}$ to the point $\ast$, a quasi-unit is the datum of a map $\ast \to \Gr_{G, \leq \mu}(\cc)$ for each $\mu\in \Lambda^{\vee,\pos}$. As in the proof of \cite[Theorem 3.10]{hahn-yuan}, this can be taken to be the inclusion of the point corresponding to the trivial $G$-bundle over $\AA^1$ with the canonical trivialization away from the origin.
\end{proof}

With \cref{caveat filtration} in mind, we can now use \cref{cohomology-grg} to compute the $T$-equivariant homology of $\Gr_G$.
\begin{lemma}\label{grt-homology}
There is an equivalence in $\Alg_\E{2}(\coCAlg(\QCoh(\cM_T)))$:
$$\cf_T(\Gr_T(\cc))^\vee \cong \co(\ld{T}_{A} \times_{\spec(A)} \cM_T).$$
\end{lemma}
\begin{proof}
Since the action of $T$ on $\Gr_T(\cc)$ is trivial, we have a canonical equivalence $\cf_T(\Gr_T(\cc))^\vee \simeq \Gr_T(\cc)_+ \otimes \cf_T(\ast)^\vee$. By definition, $\cf_T(\ast)^\vee \simeq \co_{\cM_T}$. We conclude that $\cf_T(\Gr_T(\cc))^\vee$ is equivalent as an $\E{2}$-$A$-algebra to $C_\ast(\Gr_T(\cc); A) \otimes_A \co_{\cM_T}$. Since $BT(\cc) \simeq B^2\Lambda^\vee$, there is an equivalence $\Gr_T(\cc) \simeq \Lambda^\vee$ of $\E{2}$-spaces. Therefore, $C_\ast(\Gr_T(\cc); A) \simeq A[\Lambda^\vee]$ as $\E{2}$-$A$-algebras, which is $\co(\ld{T}_A)$. This implies the desired claim.
\end{proof}
\begin{question}
Can \cref{grt-homology} be upgraded to an equivalence of \textit{$\E{3}$-$A$-algebras} for a geometrically defined $\E{3}$-algebra structure on $\cf_T(\Gr_T(\cc))^\vee$? This additional structure is crucial for a statement of the geometric Satake correspondence which is $\E{3}$-monoidal.
\end{question}

\begin{notation}
Let $T^\ast_\GG \ld{T}_A$ denote $\ld{T}_{A} \times_{\spec(A)} \cM_T$, and let $T^\ast_\GG \ld{T}$ denote its underlying scheme (over $\cM_{T,0}$).
%We will write $(T^\ast_\GG \ld{T})^\bl$ for the affine blowup of $T^\ast_\GG \ld{T}$ from \cref{t-homology-grg}, so that $\ul{\spec}_{\cM_{T,0}}(\pi_0 \cf_T(\Gr_G(\cc))^\vee) \cong (T^\ast_\GG \ld{T})^\bl$ as $\cM_{T,0}$-schemes. 
Note that if $\GG = \GG_a$, then $T^\ast_\GG \ld{T}$ is the cotangent bundle of $\ld{T}$, while if $\GG = \GG_m$, then $T^\ast_\GG \ld{T} = \ld{T} \times T$. If $\fr{B}_\GG$ denotes the blowup of $T^\ast_\GG \ld{T}$ at the closed subscheme given by $\cM_{T_\alpha,0}$ and the zero set of $e^{\alpha^\vee}-1$ for $\alpha\in \Phi$, then define $(T^\ast_\GG \ld{T})^\bl$ as the complement of the proper preimage of $\cM_{T_\alpha,0}$ in $\fr{B}_\GG$ for $\alpha\in \Phi$.
\end{notation}
\begin{theorem}\label{t-homology-grg}
Let $G$ be a connected semisimple algebraic group over $\cc$. Then there is a $W$-equivariant isomorphism 
$\spec \pi_0 \cf_T(\Gr_G(\cc))^\vee \cong (T^\ast_\GG \ld{T})^\bl$ of schemes over $\cM_{T,0}$, where the left-hand side denotes the relative $\spec$.
%$$\pi_0 \cf_T(\Gr_G(\cc))^\vee \cong \pi_0 \co_{\cM_T}[\Lambda^\vee] [\tfrac{e^{\alpha^\vee}-1}{c_\alpha}, \alpha\in \Phi]$$
\end{theorem}
\begin{proof}
There is an $\E{2}$-map $\Gr_T(\cc) \to \Gr_G(\cc)$, which induces an $\E{2}$-map $\cf_T(\Gr_T(\cc))^\vee\to \cf_T(\Gr_G(\cc))^\vee$. This is given by dualizing the map $r: \cf_T(\Gr_G(\cc)) \to \cf_T(\Gr_T(\cc))$ of $\E{2}$-coalgebras in $\QCoh(\cM_T)$. The non-$W$-equivariant claim now follows from \cref{cohomology-grg}, since $r$ induces an injection on $\pi_0$, and the (cocommutative) Hopf algebra structure on $\pi_0 \cf_T(\Gr_T(\cc))$ is given by the dual of the equivalence of \cref{grt-homology}. Proving $W$-equivariance requires a bit more work, but can easily be incorporated by keeping track of the $W$-action throughout the above discussion.
\end{proof}
\begin{remark}
The $T$-equivariant and $G$-equivariant $A$-\textit{co}homologies of $\Gr_G(\cc)$ are significantly easier to compute in terms of the stack $\cM_G$ (without any reference to root data); see \cref{A-cohomology of Gr}. In particular, see \cref{BF coh of gr} for an alternative argument for \cite[Theorem 1]{bf-derived-satake} using Hochschild homology and the Hochschild-Kostant-Rosenberg theorem.
\end{remark}
%\begin{remark}
%If $\GG$ is affine, \cref{t-homology-grg} says that there is an equivalence
%$$\cf_T(\Gr_G(\cc))^\vee \cong \co(\ld{T}_{A} \times_{\spec(A)} \cM_T)[\tfrac{e^{\alpha^\vee}-1}{c_\alpha}, \alpha\in \Phi].$$
%%of $\E{1}$-$\co_{\cM_T}$-algebras; note that \cref{blowup-example}(c) only guarantees an $\E{1}$-algebra structure on the right-hand side.
%\end{remark}
\begin{remark}
Suppose $A = \KU$, so that $\GG = \GG_m$ and $c_\alpha$ is $e^\alpha-1$.
It follows from \cref{t-homology-grg} that replacing $T$ with $\ld{T}$, we get an isomorphism between $\pi_0 \cf_{\ld{T}}(\Gr_\ld{G}(\cc))^\vee$ and $\pi_0 (T_{A} \times_{\spec(A)} \ld{T}_A)[\tfrac{e^{\alpha}-1}{e^{\alpha^\vee}-1}, \alpha\in \Phi]$. Therefore, $\pi_0 \cf_{{T}}(\Gr_{G}(\cc))^\vee$ and $\pi_0 \cf_{\ld{T}}(\Gr_\ld{G}(\cc))^\vee$ are both obtained from the blowup $\fr{B}_{\GG_m}$ of $T^\ast_\GG \ld{T}$ by deleting the proper preimage of two different closed subschemes which are Langlands dual to each other. In particular, the Langlands self-duality of the blowup $\fr{B}_{\GG_m}$ swaps the affine pieces $\spec \pi_0 \cf_{T}(\Gr_{G}(\cc))^\vee$ and $\spec \pi_0 \cf_{\ld{T}}(\Gr_\ld{G}(\cc))^\vee$ in $\fr{B}_{\GG_m}$.
\end{remark}

\begin{remark}
When $G = \SL_2$ or $\PGL_2$, we can explicitly verify \cref{t-homology-grg} at least after base-changing along $C_T^\ast(\ast; A) \to C^\ast(\ast; A)$. We will identify $\PGL_2$ with $\SO_3$ (via the $\PGL_2$-action on $\fr{pgl}_2$ which preserves the quadratic form given by the determinant). If $A = \QQ[\beta^{\pm 1}]$, for instance, \cref{t-homology-grg} says:
\begin{align*}
    \pi_0 C_\ast^{S^1}(\Omega S^3; \QQ[\beta^{\pm 1}]) & \cong \QQ[x,y^{\pm 1}, \tfrac{y-1}{x}], \\
    \pi_0 C_\ast^{S^1}(\Omega \SO(3); \QQ[\beta^{\pm 1}]) & \cong \QQ[x,y^{\pm 1}, \tfrac{y^2-1}{2x}].
\end{align*}
After killing $x$, the fraction $\tfrac{y-1}{x}$ (resp. $\tfrac{y^2-1}{x}$) defines a polynomial generator, and so we have
\begin{align*}
    \pi_0 C_\ast(\Omega S^3; \QQ[\beta^{\pm 1}]) & \cong \QQ[\tfrac{y-1}{x}], \\
    \pi_0 C_\ast(\Omega \SO(3); \QQ[\beta^{\pm 1}]) & \cong \QQ[y^{\pm 1}, \tfrac{y^2-1}{2x}]/(y^2-1).
\end{align*}
The second of these isomorphisms is compatible with the identification $\Omega \SO(3) \simeq \Z/2 \times \Omega S^3$ arising from the isomorphism $S^3/(\Z/2) \cong \SO(3)$ (but note that the equivalence $\Omega \SO(3) \simeq \Z/2 \times \Omega S^3$ is \textit{not} one of $\E{1}$-spaces). Similarly, if $A = \KU$, \cref{t-homology-grg} says:
\begin{align*}
    \pi_0 C_\ast^{S^1}(\Omega S^3; \KU) & \cong \Z[x^{\pm 1},y^{\pm 1}, \tfrac{y-1}{x-1}], \\
    \pi_0 C_\ast^{S^1}(\Omega \SO(3); \KU) & \cong \Z[x^{\pm 1},y^{\pm 1}, \tfrac{y^2-1}{x^2-1}].
\end{align*}
After killing $x-1$, the fraction $\tfrac{y-1}{x-1}$ (resp. $\tfrac{y^2-1}{x^2-1}$) defines a polynomial generator, and so we have
\begin{align*}
    \pi_0 C_\ast(\Omega S^3; \KU) & \cong \Z[\tfrac{y-1}{x-1}], \\
    \pi_0 C_\ast(\Omega \SO(3); \KU) & \cong \Z[y^{\pm 1}, \tfrac{y^2-1}{x^2-1}]/(y^2-1).
\end{align*}
Again, this is compatible with the identification $\Omega \SO(3) \simeq \Z/2 \times \Omega S^3$.

In the case $G = \SL_2$, we refer the reader to \cref{3d-sl2} and \cref{4d-sl2} for an explicit description of $\H^{G\times S^1_\rot}_\ast(\Gr_G(\cc); \cc)$ and $\KU^{G\times S^1_\rot}_0(\Gr_G(\cc)) \otimes \cc$. 
\end{remark}