\subsection{Quantized equivariant homology of $\Gr_T$}\label{sec: quantized homology torus}

We now explore the equivariant homology of $\Gr_T$ in more detail; no GKM theory is required here, but several interesting algebraic structures turn up.
Let us begin by recalling that \cref{grt-homology} gives a $W$-equivariant equivalence $\cf_T(\Gr_T(\cc))^\vee \cong \co(\ld{T}_{A} \times_{\spec(A)} \cM_T)$, which can be thought of as giving an equivalence between $\ld{T}_{A} \times_{\spec(A)} \cM_T$ and the ``$\E{2}$-$\cM_T$-scheme $\spec \cf_T(\Gr_T(\cc))^\vee$''. This admits a natural deformation given by the loop-rotation equivariant homology $\cf_{\tilde{T}}(\Gr_T(\cc))^\vee$. Since $\tilde{T} = T \times \GG_m^\rot$, there is an equivalence $\cM_{\tilde{T}} \simeq \cM_T \times \GG$, where the second factor is identified as $\cM_{\GG_m^\rot}$.
\begin{definition}\label{def: G-diff ops}
Let $\GG_0$ be a smooth $1$-dimensional group scheme over a base commutative ring, let $T$ be a compact torus, let $\Lambda$ (resp. $\Lambda^\vee$) denote the (co)character lattice of $T$, and let $\cM_{0,T} = \Hom(\Lambda, \GG_0)$.
Let $\lambda$ be a cocharacter of $T$, so that $\lambda$ defines a homomorphism $\Lambda \to \Z$, and hence a homomorphism $\lambda^\ast: \GG_0 \to \cM_{0,T}$. In turn, this defines a map
$$f^\lambda: \cM_{0,\tilde{T}} \simeq \cM_{0,T} \times \GG_0 \xar{\pr \times \lambda^\ast} \cM_{0,T}.$$
%There is therefore a map $\co_{\cM_{0,T}} \to f^\lambda_\ast \co_{\cM_{0,\tilde{T}}}$ of $\co_{\cM_{0,T}}$-algebras.
If $y$ is a local section of $\co_{\cM_{0,T}}$, we will write $\lambda^\ast(y)$ to denote the resulting local section of $\co_{\cM_{0,\tilde{T}}}$.
Let $\cd_{\ld{T}}^{\GG_0}$ denote the quotient of the associative $\co_{\GG_0}$-algebra $\co_{\cM_{0,\tilde{T}}}\pdb{x_\lambda | \lambda\in \Lambda}$ by the relations given locally by
$$x_\lambda \cdot x_\mu = x_{\lambda+\mu}, \  y \cdot x_\lambda = x_\lambda \cdot \lambda^\ast(y).$$
Here, $\lambda,\mu\in \Lambda^\vee$, and $y$ is a local section of $\co_{\cM_{0,T}}$. We will call $\cd_{\ld{T}}^{\GG_0}$ the \textit{algebra of $\GG_0$-differential operators}.
\end{definition}
\begin{remark}\label{G-mellin}
The algebra $\cd_{\ld{T}}^{\GG_0}$ satisfies a Mellin transform: namely, it follows from unwinding the definition that there is an equivalence
$$\LMod_{\cd_{\ld{T}}^{\GG_0}}(\QCoh(\GG_0)) \simeq \QCoh(\cM_{0,\tilde{T}}/\Lambda),$$
where $\lambda\in \Lambda$ acts on $\cM_{0,\tilde{T}}$ via $y\mapsto \lambda^\ast y$.
\end{remark}
\begin{notation}
If $A$ is a complex-oriented even-periodic $\Eoo$-ring and $\GG_0$ is the $\pi_0(A)$-group underlying a oriented commutative $A$-group $\GG$, we will write $\cd_{\ld{T}}^\GG$ to denote $\cd_{\ld{T}}^{\GG_0}$, and refer to it as the algebra of $\GG$-differential operators. We hope this does not cause any confusion.
\end{notation}
\begin{prop}[Quantization of \cref{grt-homology}]\label{homology and quantized diffop}
There is an isomorphism $\pi_0 \cf_{\tilde{T}}(\Gr_T(\cc))^\vee \cong \cd_{\ld{T}}^\GG$ of $\pi_0 \co_\GG$-algebras.
\end{prop}
\begin{proof}
Since $\Gr_T(\cc) \simeq \Omega T_c \simeq \Lambda^\vee$, it is easy to see that $\pi_0 \cf_{\tilde{T}}(\Gr_T(\cc))^\vee \cong \bigoplus_{\lambda\in \Lambda^\vee} \pi_0 \co_{\cM_{\tilde{T}}}$; let $x_\lambda$ be a generator of the summand indexed by $\lambda\in \Lambda^\vee$. If $\lambda\in \Lambda^\vee = \Hom(\Lambda, \Z)$, the map $\Omega T_c \to \Omega T_c$ given by multiplication-by-$\lambda$ is $T\times S^1_\rot$-equivariant for the homomorphism $T\times S^1_\rot \to T\times S^1_\rot$ given by $(t, \theta) \mapsto (t \lambda(\theta), \theta)$, where $\lambda$ is viewed as a homomorphism $S^1 \to T$. On weight lattices, this homomorphism induces the map $\Lambda \times \Z \to \Lambda \times \Z$ which sends $(\mu, n) \mapsto (\mu, n+\lambda^\vee(\mu))$. In particular, the composite $\Lambda \to \Lambda \times \Z \to \Lambda \times \Z$ sends $\mu \mapsto (\mu, \lambda^\vee(\mu))$. Applying $\Hom(-, \GG)$ to this composite precisely produces the map $f^\lambda: \cM_{\tilde{T}} \to \cM_T$ from \cref{def: G-diff ops}. This implies the desired identification of $\pi_0 \cf_{\tilde{T}}(\Gr_T(\cc))^\vee$.
\end{proof}
\begin{example}\label{ordinary quantized diffop}
Let $T \cong S^1$ be a torus of rank $1$ (for simplicity).
Suppose $A = \QQ[\beta^{\pm 1}]$, so $\GG = \hat{\GG}_a$ and $\pi_0 \co_{\GG} \cong \QQ\pw{\hbar}$. Then the algebra of \cref{def: G-diff ops} is the quotient of the $\QQ\pw{\hbar}$-algebra $\QQ\pw{\hbar}\pdb{y, x^{\pm 1}}$ by the relation $yx = x(y+\hbar)$. In other words, $y$ acts as the operator $\hbar x\partial_x$, so we simply have that 
$$\H^{\tilde{T}}_0(\Gr_T(\cc); \QQ[\beta^{\pm 1}]) \cong \H^{\tilde{T}}_\ast(\Gr_T(\cc); \QQ) \cong \QQ\pw{\hbar}\pdb{\hbar x\partial_x, x^{\pm 1}}.$$
This has been stated previously as \cite[Proposition 5.19(2)]{bfn-ii}.
In particular, the localization $\H^{\tilde{T}}_0(\Gr_T(\cc); \QQ[\beta^{\pm 1}])[\hbar^{-1}]$ is isomorphic to the rescaled Weyl algebra $\cd_{\ld{T}}^\hbar$; this is the motivation behind the terminology in \cref{def: G-diff ops}.
Note that \cref{G-mellin} simply reduces to the standard Mellin transform, which gives an equivalence between $\DMod_\hbar(\ld{T})$ and $\QCoh(\fr{t}_{\QQ\pw{\hbar}}/\Lambda)$.
\end{example}
\begin{example}\label{q quantized diffop}
Again, let $T \cong S^1$ be a torus of rank $1$ (for simplicity).
Suppose $A = \KU$, so $\GG = \GG_m$ and $\pi_0 \co_\GG \cong \Z[q^{\pm 1}]$. Then the algebra of \cref{def: G-diff ops} is the quotient of the $\Z[q^{\pm 1}]$-algebra $\Z[q^{\pm 1}]\pdb{y^{\pm 1}, x^{\pm 1}}$ by the relation $yx = qxy$. (This is also known as the ``quantum torus''.) In other words, $y$ acts as the operator $q^{x\partial_x}$ sending $f(x) \mapsto f(qx)$, so we simply have that 
$$\KU^{\tilde{T}}_0(\Gr_T(\cc)) \cong \Z[q^{\pm 1}]\pdb{q^{x\partial_x}, x^{\pm 1}}.$$
This is closely related to the $q$-Weyl algebra $\cd_q = \Z[q^{\pm 1}]\pdb{\Theta, x^{\pm 1}}/(\Theta x = x(q\Theta+1))$ for $\ld{T} = \GG_m$: indeed, since the logarithmic $q$-derivative $\Theta = x\nabla_q$ is given by the fraction $\frac{q^{x\partial_x}-1}{q-1}$, the pullback of $\cd_{\ld{T}}^\GG$ along $\GG_m-\{1\} \hookrightarrow \GG_m$ is isomorphic to the algebra $\cd_q[\frac{1}{q-1}]$.
Note that \cref{G-mellin} gives a ``$q$-Mellin transform'', i.e., an equivalence between $\LMod_{\KU^{\tilde{T}}_0(\Gr_T(\cc))}$ and $\QCoh((\GG_m)_{\Z[q^{\pm 1}]}/\Z)$, where $\Z$ acts on $(\GG_m)_{\Z[q^{\pm 1}]}$ by sending $y\mapsto qy$.
\end{example}
\begin{remark}\label{G-quantization torus}
Using \cref{equivariant-koszul}, there is an equivalence $\Loc_{T_c}(T_c; A) \simeq \LMod_{\cf_T(\Gr_T(\cc))^\vee}$. Since $\pi_0 \cf_{\tilde{T}}(\Gr_T(\cc))^\vee \cong \cd_{\ld{T}}^\GG$ is a ``quantization'' of $\pi_0 \cf_T(\Gr_T(\cc))^\vee \cong \co_{T^\ast_\GG \ld{T}}$ (i.e., an associative deformation of $T^\ast_\GG \ld{T}$ along $\GG$), and \cref{homology and quantized diffop} implies an equivalence of $\E{1}$-$A_\QQ$-algebras $\cf_{\tilde{T}}(\Gr_T(\cc))^\vee \otimes \QQ \cong \cd_{\ld{T}}^\GG \otimes_{\pi_0 A} A_\QQ$, we see that $\LMod_{\cd_{\ld{T}}^\GG} \otimes_{\pi_0 A} A_\QQ$ defines a ``quantization'' of $\Loc_{T_c}(T_c; A) \otimes \QQ$.
\end{remark}

Let us briefly outline the relationship between the algebra $\cd_{\ld{T}}^{\GG_0}$ of \cref{def: G-diff ops} and the $F$-de Rham complex of \cite{generalized-n-series}.
\begin{notation}
For the purpose of this discussion, we will assume that $T \cong S^1$ is a torus of rank $1$, so that $\ld{T} \cong \GG_m$. We will also fix an invariant differential form on the formal completion $\hat{\GG}_0$ of $\GG_0$ at the zero section, so that there is an isomorphism $\hat{\GG}_0 \cong \spf R\pw{t}$ of formal $R$-schemes. Let $F(x,y)$ denote the resulting formal group law over $R$, and define the $n$-series of $F$ by
$$[n]_F := \overbrace{F(t, F(t, F(t, \cdots F(t, t) \cdots )))}^n.$$
We will often write $x+_Fy = x+_\GG y$ to denote $F(x,y)$.
Let $\hat{\cd}_{\ld{T}}^{\GG_0}$ denote the completion of $\cd_{\ld{T}}^{\GG_0}$ at the zero section of $\cM_{0,\tilde{T}} \cong \cM_{0,T} \times \GG_0$.
\end{notation}
\begin{lemma}[Cartier duality]\label{cartier-duality}
Let $\hat{\GG}_0$ be a $1$-dimensional formal group over a commutative ring $R$, and let $\Cart(\hat{\GG}_0)$ denote its Cartier dual (see \cite[Section 3.3]{drinfeld-formal-group} for more on Cartier duals of formal groups). Then there is an equivalence of categories $\QCoh(\hat{\GG}_0) \simeq \QCoh(B\Cart(\hat{\GG}_0))$ sending the convolution tensor product on the left-hand side to the usual tensor product on the right-hand side. Under this equivalence, the functor $\QCoh(\hat{\GG}_0) \to \Mod_R$ given by restriction to the zero section is identified with the functor $\QCoh(B\Cart(\hat{\GG}_0)) \to \Mod_R$ given by pullback along the map $\spec(R) \to B\Cart(\hat{\GG}_0)$.
\end{lemma}
\begin{prop}\label{endomorphism F-de Rham}
There is a canonical action of $\hat{\cd}_{\ld{T}}^{\GG_0}$ on $(\GG_m)_{R\pw{t}} = \spf R\pw{t}[x^{\pm 1}]$ such that $R\pw{t}[x^{\pm 1}] \otimes_{\hat{\cd}_{\ld{T}}^{\GG_0}} R\pw{t}[x^{\pm 1}]$ is isomorphic to the two-term complex
$$C^\bull = (R\pw{t}[x^{\pm 1}] \to R\pw{t}[x^{\pm 1}]dx), \ x^n \mapsto [n]_F x^n dx$$
from \cite[Remark 4.3.8]{generalized-n-series}.
\end{prop}
\begin{proof}[Proof sketch]
%The argument is very similar to the proof of the well-known fact that for the standard action of $\cd_{\ld{T}}^\hbar$ on $\QQ\pw{\hbar}[x^{\pm 1}]$, the endomorphism algebra $\End_{\cd_{\ld{T}}^\hbar}(\QQ\pw{\hbar}[x^{\pm 1}])$ is isomorphic to the complex
%$$C^\bull = (\QQ\pw{\hbar}[x^{\pm 1}] \xar{\hbar x\partial_x} \QQ\pw{\hbar}[x^{\pm 1}]dx), \ x^n \mapsto n\hbar x^n dx.$$
Since $T$ is of rank $1$, there is an isomorphism $\cM_{0,T} \cong \GG_0$, and hence an isomorphism $\hat{\cM}_{0,T} \cong \hat{\AA}^1$ of formal $R$-schemes, where $\hat{\cM}_{0,T}$ denotes the completion of $\cM_{0,T}$ at the zero section. Let $y$ be a local coordinate on $\cM_{0,T}$.
Then, $\hat{\cd}_{\ld{T}}^{\GG_0}$ is isomorphic to the quotient of the associative $\hat{\co}_{\GG_0}$-algebra $\hat{\co}_{\GG_0 \times \cM_{0,T}}\pdb{x^{\pm 1}}$ subject to the relation $yx = x(y +_\GG t)$.
The $t$-adic filtration on $\hat{\cd}_{\ld{T}}^{\GG_0}$ therefore has associated graded $\gr(\hat{\cd}_{\ld{T}}^{\GG_0}) \cong \hat{\co}_{\cM_{0,T}}[x^{\pm 1}]\pw{\ol{t}}$, where $\ol{t}$ lives in weight $1$. View $R$ as a $\co_{\cM_{0,T}}$-algebra via the zero section, i.e., the augmentation $\co_{\cM_{0,T}} \to R$. Then, the action of $\gr(\hat{\cd}_{\ld{T}}^{\GG_0})$ on $R[x^{\pm 1}]\pw{\ol{t}}$ is induced by the augmentation $\hat{\co}_{\cM_{0,T}} \to R$. The isomorphism $\hat{\cM}_{0,T} \cong \hat{\AA}^1$ of formal $R$-schemes then implies an isomorphism $R \otimes_{\co_{\cM_{0,T}}} R \cong R[\epsilon]/\epsilon^2$ with $\epsilon$ in homological degree $1$.
It follows that
$$R\pw{\ol{t}}[x^{\pm 1}] \otimes_{\gr(\hat{\cd}_{\ld{T}}^{\GG_0})} R\pw{\ol{t}}[x^{\pm 1}] \simeq R\pw{\ol{t}}[x^{\pm 1}][\epsilon]/\epsilon^2,$$
where $\ol{t}$ is in weight $1$ and degree $0$, and $\epsilon$ is in weight $0$ and degree $1$.

By \cref{cartier-duality}, the $t$-adic filtration on $\hat{\cd}_{\ld{T}}^{\GG_0}$ is equivalent to the data of a $\Cart(\hat{\GG}_0)$-action on $R\pw{\ol{t}}[x^{\pm 1}] \otimes_{\gr(\hat{\cd}_{\ld{T}}^{\GG_0})} R\pw{\ol{t}}[x^{\pm 1}] \simeq R\pw{\ol{t}}[x^{\pm 1}][\epsilon]/\epsilon^2$. This in turn is equivalent to the data of a differential 
$$\nabla: R\pw{\ol{t}}[x^{\pm 1}] \to R\pw{\ol{t}}[x^{\pm 1}]\cdot \epsilon$$
satisfying a $\hat{\GG}_0$-analogue of the Leibniz rule: if\footnote{Note that $\nabla$ has to be homogeneous in the degree of the monomial in $x$, as can be seen by keeping track of the $x$-weight.} $\nabla(x^n) = f(n) x^n \epsilon$ for some $f(n)\in R\pw{t}$, then $f(n+m) = f(n) +_\GG f(m)$.
It therefore suffices to determine $\nabla(x)$; but the relation $yx = x(y +_\GG t)$ forces $\nabla(x) = tx\epsilon$. This implies that 
$$\nabla(x^n) = (\overbrace{t +_\GG \cdots +_\GG t}^n) x^n \epsilon = [n]_F x^n \epsilon,$$
as desired.
%Now, the differential $\nabla$ is determined by the formula:
%$$\nabla: x^n \mapsto (yx^n -_\GG x^n y)\epsilon.$$
%Since $yx = x(y +_\GG t)$, a simple induction shows that 
%$$y x^n = x^n(y+_\GG \overbrace{t +_\GG \cdots +_\GG t}^n) = x^n(y +_\GG [n]_F).$$
%It follows that $yx^n -_\GG x^n y = x^n [n]_F$, so $\nabla(x^n) = [n]_F x^n \epsilon$, as desired. \todo hmm
\end{proof}
\begin{example}
When $\GG_0 = \hat{\GG}_a$ over\footnote{Of course, one can work over $\Z$ too; we just chose $\QQ$ to continue with \cref{ordinary quantized diffop}.} $\QQ$, the complex $C^\bull$ is
$$C^\bull = (\QQ\pw{\hbar}[x^{\pm 1}] \to \QQ\pw{\hbar}[x^{\pm 1}]dx), \ x^n \mapsto n\hbar x^n dx.$$
Indeed, since $yx = x(y+\hbar)$, we have $yx^n = x^n(y+n\hbar)$; since $t = \hbar$ in this case, we have $x^n\mapsto n\hbar x^n \epsilon$. This is evidently a $\hbar$-rescaling of the classical de Rham complex of $\GG_m$.

When $\GG_0 = \GG_m$ over $\Z$, the complex $C^\bull$ is
$$C^\bull = (\Z\pw{q-1}[x^{\pm 1}] \to \Z\pw{q-1}[x^{\pm 1}]dx), \ x^n \mapsto (q^n-1) x^n dx.$$
Indeed, since $yx = x(qy)$, we have $yx^n = x^n (q^n y)$, and hence 
$$(y-1)x^n = x^n(q^n y - 1) = x^n((y-1) +_F (q^n-1)),$$
where $F(z,w) = z + w + zw$ is the multiplicative formal group law; since $t = q-1$ in this case, we have $x^n \mapsto (q^n-1) x^n \epsilon$. The complex $C^\bull$ is a $(q-1)$-rescaling of the $q$-de Rham complex of $\GG_m$ from \cite{scholze-q-def}.
\end{example}
\begin{remark}
The complex of \cref{endomorphism F-de Rham} is not quite the $F$-de Rham complex of \cite[Definition 4.3.6]{generalized-n-series}; rather, if $\eta_t$ denotes the d\'ecalage functor of \cite{berthelot-ogus} with respect to the ideal $(t)\subseteq R\pw{t}$, the $F$-de Rham complex is given by the d\'ecalage $\eta_t C^\bull$. In particular, the complex of \cref{endomorphism F-de Rham} is isomorphic to the $F$-de Rham complex after inverting $t$. One can modify the algebra $\cd_{\ld{T}}^{\GG_0}$ of \cref{def: G-diff ops} (by performing a noncommutative analogue of an affine blowup/deformation to the normal cone\footnote{For instance, in the case of \cref{ordinary quantized diffop}, this procedure simply adjoins the fraction $\frac{y}{\hbar}$; in the case of \cref{q quantized diffop}, this procedure simply adjoins the fraction $\frac{y-1}{q-1}$.}) such that the relative tensor product as in \cref{endomorphism F-de Rham} is the $F$-de Rham complex itself. Since it is not needed for this article, we will not describe this modification here.
\end{remark}
\begin{remark}\label{rmk: koszul duality LT}
\cref{endomorphism F-de Rham} says that $\hat{\cd}_{\ld{T}}^{\GG_0}$ is Koszul dual to the complex $C^\bull$. Forthcoming work of Arpon Raksit shows that the d\'ecalage $\eta_t C^\bull$ can be recovered from the ``even filtration'' (in the sense of \cite{even-filtr}) on the periodic cyclic homology $\HP(\tau_{\geq 0} A[x^{\pm 1}]/\tau_{\geq 0} A)$. See also the discussion in \cite[Section 3.3]{thh-xn}.
Using similar techniques, one can show that $C^\bull$ can be recovered from the even filtration on the negative cyclic homology $\HC^-(A[x^{\pm 1}]/A) = \HH(A[x^{\pm 1}]/A)^{hS^1}$.

Recalling that $T = S^1$, this $\Eoo$-$A$-algebra is simply $\HC^-(A[\Omega T]/A)$. The Hochschild homology $\HH(A[\Omega T]/A) \simeq A \otimes \THH(S[\Omega T])$ is $S^1$-equivariantly equivalent to the $A$-chains $C_\ast(\cL T; A)$ on the free loop space of $T$. (For a reference, see \cite[Corollary IV.3.3]{nikolaus-scholze}.) The $A$-chains $A[\cL T]$ is $S^1$-equivariantly Koszul dual\footnote{This Koszul duality essentially stems from the (nonequivariant) decomposition $\cL T \simeq T \times \Omega T$.} to $A[\Omega T]^{hT}$; this can be identified as a completion of $\cf_T(\Omega T)^\vee$ at the zero section of $\cM_T$. In other words, $\HC^-(A[\Omega T]/A)$ is Koszul dual to the completion of $\cf_{T\times S^1_\rot}(\Omega T)^\vee$ at the zero section of $\cM_T \times \GG$. This is the topological source of the Koszul duality of \cref{endomorphism F-de Rham}.
\end{remark}
\begin{remark}
In \cref{rmk: koszul duality LT}, we mentioned that the Koszul duality between $\GG$-differential operators and the $F$-de Rham complex manifests in topology as the Koszul duality between $\cf_{T \times S^1_\rot}(\Omega T)^\vee$ and $\HC^-(A[\Omega T]/A)$. There is clearly nothing special about $T$ in this Koszul duality: given a sufficiently robust theory of $G$-equivariant $A$-(co)homology (see the discussion surrounding \cref{nonabelian-equiv-cochains}), there is also a Koszul duality between $\cf_{G \times S^1_\rot}(\Omega G)^\vee$ and $\HC^-(A[\Omega G]/A) = A[\cL G]^{hS^1}$. When $A = \cc[\beta^{\pm 1}]$, \cite[Theorem 3]{bf-derived-satake} states that $\cf_{G \times S^1_\rot}(\Omega G)^\vee$ can be identified with (the $2$-periodification of) the bi-Whittaker reduction $\ld{N}^-\backslash_\chi \cd_{\ld{G}}/_\chi \ld{N}^-$. Using the results of this article, it is also possible to compute $A[\cL G]^{hS^1}$ in this manner, at least if we assume that small primes are inverted: the zeroth graded piece of the ``even filtration'' on $A[\cL G]^{hS^1}$ looks like the $2$-periodification of the $F$-de Rham complex of $Z_f(\ld{B})$ for a chosen principal nilpotent element $f\in \ld{\g}$. We plan to explain this in future work.
\end{remark}