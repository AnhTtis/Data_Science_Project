\cref{intro-mirror-dual-of-g} is closely related to the results of Brylinski-Zhang (see \cite{brylinski-zhang}). To explain this, we begin by recasting the results of \cite{brylinski-zhang} in the language of \cref{review-equiv}.
\begin{recall}
Let $G$ be a simply-connected compact Lie group. Then the main result of \cite{brylinski-zhang} says that there is an isomorphism $\KU_G^\ast(G) \cong \Omega^\ast_{K_0(\Rep(G))/\Z} \otimes_\Z \Z[\beta^{\pm 1}]$, where $K_0(\Rep(G))$ is the (complex) representation ring of $G$. If $G$ is not necessarily simply-connected, there is also an isomorphism $\H_G^\ast(G; \QQ) \cong \Omega^\ast_{\H^\ast(BG; \QQ)/\QQ}$.
\end{recall}
These can be simultaneously generalized by the following:
\begin{prop}\label{hh-mg}
Let $A$ be a complex-oriented even-periodic $\Eoo$-ring, and let $\GG$ be an oriented commutative $A$-group. Let $G$ be a simply-connected compact Lie group, and suppose that the functor $\cf_G: \Top(G)_\conn^\op \to \QCoh(\cM_G)$ of $G$-equivariant $A$-cochains
on connected finite $G$-spaces
%following \cite[End of Section 3.5]{survey} (see \cref{nonabelian-equiv-cochains}).
is symmetric monoidal\footnote{Note that this assumption will fail if $G$ is not connected!}.
Then there is an equivalence
$$\Gamma(\cM_G; \cf_G(G)) \simeq \HH(\cM_G/A).$$
\end{prop}
\begin{proof}
Indeed, since $G$ is connected, we have $G \simeq \Omega(BG)$. 
%Therefore, if $G/G$ and $\ast/G$ denote the orbifold quotients, then there is an equivalence 
%$$G/G \simeq \ast/G \times_{\ast/G \times \ast/G} \ast/G$$
%of orbifolds. Here, the maps $\ast/G \rightrightarrows \ast/G \times \ast/G \simeq \ast/(G\times G)$ are both given by the diagonal embedding $G\subseteq G \times G$. (To see this, note that $G/G\simeq G\backslash (G\times G)/G$, where $G\times G$ acts on $G\times G$ via $(g_1, g_2): (h_1, h_2) \mapsto (g_1 h_1 g_2^{-1}, g_1 h_2 g_2^{-1})$.)
Recall from \cref{S1-connections level} that $G/G$ is the free loop space of $\ast/G$ in the category of orbifolds.
The assumption on $\cf_G$ now implies that
$$\cf_G(G) \simeq \cf_G(\ast) \otimes_{\cf_{G\times G}(\ast)} \cf_G(\ast) \simeq \co_{\cM_G} \otimes_{\co_{\cM_G} \otimes \co_{\cM_G}} \co_{\cM_G}.$$
Therefore, $\Gamma(\cM_G; \cf_G(G))$ is precisely the Hochschild homology of $\cM_G$.
\end{proof}
\begin{remark}
One can view $\Gamma(\cM_G; \cf_G(G))$ as endomorphisms of the unit object in $\Loc_G(G; A)$, so that $\Mod_{\Gamma(\cM_G; \cf_G(G))}$ behaves as a completion of $\Loc_G(G; A)$.
\end{remark}
\begin{remark}
In some cases, the Hochschild-Kostant-Rosenberg spectral sequence degenerates integrally. Then, $\pi_\ast \HH(\cM_G/A)$ can be identified with the $2$-periodification of the (derived) Hodge cohomology of the underlying stack of $\cM_G$ over $\pi_0(A)$. This applies, for instance, when $A = \KU$; in this case, $\cM_G$ is a lift to $\KU$ of $\spec K_0(\Rep(G)) \cong T\mmod W$, and \cref{hh-mg} is precisely the calculation of \cite{brylinski-zhang}.
\end{remark}
\begin{remark}\label{A-cohomology of Gr}
\cref{hh-mg} can be continued further to study the $G$-equivariant $A$-\textit{co}homology of $\Omega G$, if we additionally assume that the functor $\cf_G: \Ind(\Top(G))_\conn^\op \to \QCoh(\cM_G)$ on connected \textit{ind-finite} $G$-spaces is symmetric monoidal. Indeed, observe that there is an equivalence
$$G\backslash \Omega G \simeq G\backslash \cL G/G \simeq \ast/G \times_{\ast/\cL G} \ast/G$$
of orbifolds. But $\ast/\cL G \simeq \cL(\ast/G) \simeq \ast/G \times_{\ast/G \times \ast/G} \ast/G$, so that $G\backslash \Omega G \simeq \Map(S^2, \ast/G)$, i.e., the cotensoring of $\ast/G$ by $S^2$ (in \textit{unpointed} orbifolds).
Using the assumption on $\cf_G$, we therefore conclude that the $G$-equivariant $A$-cohomology of $\Omega G$ can be identified with the factorization homology
$$\Gamma(\cM_G; \cf_G(\Omega G)) \simeq \int_{S^2} \cM_G\in \CAlg_A$$
taken internally to $A$-modules.

The preceding discussion also computes the $T$-equivariant $A$-cohomology of $\Omega G$. To explain this, write $p: \cM_T \to \cM_G$ to denote the canonical map. The above discussion shows that
there is an equivalence
$$T\backslash \Omega G \simeq \ast/T \times_{\ast/G \times_{\ast/G \times \ast/G} \ast/G} \ast/G$$
of orbifolds, so that $p_\ast \cf_T(\Omega G)$ can be identified with the factorization homology over $S^2$ of $\cM_G$ with coefficients in the $\E{2}$-module $p_\ast \co_{\cM_T}$. In other words, there is an equivalence
$$\Gamma(\cM_T; \cf_T(\Omega G)) \simeq \int_{S^2} (\cM_G; p_\ast \co_{\cM_T})\in \CAlg_A.$$
\end{remark}
\begin{remark}
This approach is rather robust: for instance, if $K\subseteq G$ is a closed subgroup such that $G/K$ is a finite space, there are equivalences of orbifolds
$$G\backslash \cL(G/K) \simeq \Omega(G/K)/K \simeq (\ast \times_{\ast \times_{\ast/G} \ast/K} \ast)/K \simeq \ast/K \times_{\ast/K \times_{\ast/G} \ast/K} \ast/K.$$
Under the same hypotheses as \cref{A-cohomology of Gr}, this implies that $\Gamma(\cM_G; \cf_G(\cL(G/K)))$ is isomorphic to the relative Hochschild homology $\HH(\cM_K/\cM_G)$. One can recover \cref{A-cohomology of Gr} by noting that if $H$ is a simply-connected compact Lie group and $K = H\subseteq H \times H = G$, the Hochschild homology $\HH(\cM_H/\cM_H \times \cM_H)$ of the diagonal embedding $\Delta: \cM_H \hookrightarrow \cM_H \times \cM_H$ is precisely the factorization homology $\int_{S^2} \cM_H$.
\end{remark}
The relationship of the Brylinski-Zhang isomorphism to \cref{intro-mirror-dual-of-g} can now be explained as follows.
\begin{example}
Continue to assume that $G$ is a simply-connected compact Lie group.
If $A = \QQ[\beta^{\pm 1}]$, then there is an equivalence
$$\Loc_G^\gr(G; \QQ[\beta^{\pm 1}]) \simeq \QCoh(\ld{\fr{t}}\mmod W \times_{\ld{\g}/\ld{G}} \ld{\fr{t}}\mmod W),$$
where all objects on the coherent side are defined over $\QQ$. Since $\ld{\fr{t}}\mmod W \times_{\ld{\g}/\ld{G}} \ld{\fr{t}}\mmod W \cong (T^\ast \ld{T})^\bl\mmod W$ is isomorphic to the group scheme of regular centralizers in $\ld{\g}$, we will write write $\ld{J}_{\GG_a}$ to denote $\ld{\fr{t}}\mmod W \times_{\ld{\g}/\ld{G}} \ld{\fr{t}}\mmod W$. The above equivalence therefore states that
\begin{equation}\label{loc-g-q}
    \Loc_G^\gr(G; \QQ[\beta^{\pm 1}]) \simeq \QCoh(\ld{J}_{\GG_a}).
\end{equation}
On the other hand, by \cite[Theorem 3.4.2]{riche}, the Lie algebra of $\ld{J}_{\GG_a}$ over $\ld{\fr{t}}\mmod W$ is isomorphic to $T^\ast(\ld{\fr{t}}\mmod W)$. Therefore, \cref{hh-mg} and the Hochschild-Kostant-Rosenberg theorem gives an isomorphism
$$\H^\ast_G(G; \QQ[\beta^{\pm 1}]) \cong \pi_\ast \HH(\ld{\fr{t}}\mmod W / \QQ) \otimes_\QQ \QQ[\beta^{\pm 1}] \cong \co_{T[-1](\ld{\fr{t}}\mmod W)} \otimes_\QQ \QQ[\beta^{\pm 1}].$$
In particular, there is an equivalence
\begin{equation}\label{coherent sheaves Q-coh}
    \Mod_{\H_G^0(G; \QQ[\beta^{\pm 1}])} \simeq \QCoh(T[-1](\ld{\fr{t}}\mmod W)) \otimes_\QQ \QQ[\beta^{\pm 1}].
\end{equation}
By Koszul duality, the right-hand side is equivalent to the $2$-periodification of the $\infty$-category of ind-coherent sheaves over the formal completion of $\ld{J}_{\GG_a}$ at the zero section.
One can view the resulting description of $\Mod_{\H_G^0(G; \QQ[\beta^{\pm 1}])}$ as a infinitesimal version of the equivalence \cref{loc-g-q}. By construction, the equivalence \cref{coherent sheaves Q-coh} is just a restatement of the Brylinski-Zhang isomorphism $\H_G^\ast(G; \QQ) \cong \Omega^\ast_{\H^\ast(BG; \QQ)/\QQ}$.
\end{example}

\begin{example}\label{BF coh of gr}
We can also specialize \cref{A-cohomology of Gr} to this case: we have
\begin{equation}\label{fact homology S2}
    \H^\ast_G(\Omega G; \QQ) \cong \pi_\ast \left(\int_{S^2} \cM_G\right).
\end{equation}
Here, $\cM_G = \spec C^\ast(BG; \QQ)$ is the derived $\QQ$-scheme whose underlying graded $\QQ$-scheme is $\ld{\fr{t}}[2]\mmod W = \spec \H^\ast(BG; \QQ)$.
Since $\QQ$ is a field of characteristic zero and $G$ is assumed to be connected, $\H^\ast(BG; \QQ)$ is a polynomial algebra on generators in even degrees; this implies that $C^\ast(BG; \QQ)$ is formal as an $\Eoo$-$\QQ$-algebra\footnote{This follows from the fact that the free $\Eoo$-$\QQ$-algebra on classes in even degrees can be identified with the polynomial $\QQ$-algebra, i.e., is itself formal.}. In particular, we may identify $\cM_G = \ld{\fr{t}}[2]\mmod W$.
Just as the Hochschild homology of $\ld{\fr{t}}[2]\mmod W$ can be identified with the ring of functions on $T[-1](\ld{\fr{t}}[2]\mmod W)$, a version of the Hochschild-Kostant-Rosenberg theorem implies that the factorization homology over $S^2$ can be identified with the ring of functions on the $(-2)$-shifted tangent bundle
$$T[-2](\ld{\fr{t}}[2]\mmod W) = \spec \Sym_{\ld{\fr{t}}[2]\mmod W} (\Omega^1_{\ld{\fr{t}}[2]\mmod W}[2]).$$
Now\footnote{We will not need such a general statement, but we recall it since it is very useful in many other contexts, too.}, if $R$ is a (simplicial) commutative ring and $M$ is a connective $R$-module, there is a d\'ecalage isomorphism\footnote{The d\'ecalage isomorphism only applies to simplicial commutative algebras $R$, and \textit{not} general $\Eoo$-$\Z$-algebras (in part because of issues in defining the derived functors of $\Sym$ and $\Gamma$). This is the reason why we conspicuously shifted from working with coefficients in $\QQ[\beta^{\pm 1}]$ to working with coefficients in $\QQ$. For instance, observe that if $R$ was instead a $\QQ[\beta^{\pm 1}]$-algebra (hence not a simplicial commutative ring) and $M$ is an $R$-module, it is \textit{not possible} to distinguish between $M[2]$ and $M$. Although this might seem like a useless point, the observation that working with $2$-periodic coefficients is inherently destructive is important to clarifying why divided power structures appear in the $G$-equivariant cohomology of $\Omega G$ when one considers more general coefficients.} (see \cite[Sec. I.4.3.2]{illusie-decalage}) $\Sym^j_R(M[2]) \cong \Gamma^j_R(M)[2j]$, where $\Gamma^j$ denotes (the left derived functor of) the $j$th divided power construction.
Therefore, we see that $\Sym_{\ld{\fr{t}}[2]\mmod W} (\Omega^1_{\ld{\fr{t}}[2]\mmod W}[2])$ can be identified with a shearing (which we will simply denote by $[2\bull]$) of the divided power algebra $\Gamma_{\ld{\fr{t}}[2]\mmod W} (\Omega^1_{\ld{\fr{t}}[2]\mmod W})$. In other words, there is an isomorphism
$$\pi_\ast \left(\int_{S^2} \ld{\fr{t}}[2]\mmod W\right) \cong \Gamma_{\ld{\fr{t}}[2]\mmod W} (\Omega^1_{\ld{\fr{t}}[2]\mmod W})[2\bull];$$
the shearing on the right-hand side is undone by $2$-periodifying the left-hand side. Therefore, we obtain an isomorphism
$$\H^\ast_G(\Omega G; \QQ) \otimes_\QQ \QQ[\beta^{\pm 1}] \cong \Gamma_{\ld{\fr{t}}\mmod W} (\Omega^1_{\ld{\fr{t}}\mmod W}).$$
Up to this point, the fact that the coefficients are $\QQ$ (as opposed to a general $\Z$-algebra with some small primes inverted) has not been used outside of the formality of $C^\ast(BG; \QQ)$. Using it now, we see that the divided power algebra can be identified with a symmetric algebra, in which case the above formula implies that $\H^\ast_G(\Omega G; \QQ) \otimes_\QQ \QQ[\beta^{\pm 1}]$ can be identified with the ring of functions on the tangent bundle $T(\ld{\fr{t}}\mmod W)$. This should be compared to \cite[Theorem 1]{bf-derived-satake} with $\hbar = 0$; see \cite[Section 2.6]{bf-derived-satake} and \cite[Section 1.7]{ginzburg-langlands}. A similar argument using the $S^1$-action on $S^2$ by rotation can be used to recover (the $2$-periodification of) the full quantized statement of \cite[Theorem 1]{bf-derived-satake}.
\end{example}
\begin{remark}
The above discussion implies a more general statement. Namely, suppose that $R$ is a (classical) commutative ring such that \cref{A-cohomology of Gr} applies to $G$-equivariant $R$-cohomology --- in particular, such that there is an isomorphism
\begin{equation}\label{general fact homology S2}
    \H^\ast_G(\Omega G; R) \cong \pi_\ast \left(\int_{S^2} \cM_G\right) \in \CAlg_{\pi_\ast R}
\end{equation}
as in \cref{fact homology S2}. (This assumption is likely to hold for rather general rings $R$.) As usual, $\cM_G = \spec C^\ast(BG; R)$ is an $\Eoo$-$R$-scheme with underlying graded $R$-scheme $\ld{\fr{t}}_R[2]\mmod W$; here, $\ld{\fr{t}}_R$ denotes the base-change of $\ld{\fr{t}}$ from $\Z$ to $R$. Suppose that $C^\ast(BG; R)$ is formal as an $\E{n}$-$R$-algebra (i.e., there is an equivalence $C^\ast(BG; R) \simeq \H^\ast(BG; R)$ as $\E{n}$-$R$-algebras); by obstruction theory, this can always be guaranteed if $n=2$ and $\H^\ast(BG; R)$ is a polynomial algebra on generators in even degrees. Then \cref{general fact homology S2} implies that $\H^\ast_G(\Omega G; R)$ is equivalent to $\pi_\ast \left(\int_{S^2} \ld{\fr{t}}_R[2]\mmod W\right)$ as $\E{n-2}$-$R$-algebras. In particular, since $C^\ast(BG; R)$ is formal as an $\E{2}$-$R$-algebra, we see that $\H^\ast_G(\Omega G; R)$ is equivalent to $\pi_\ast \left(\int_{S^2} \ld{\fr{t}}_R[2]\mmod W\right)$ as unital $R$-modules. If $C^\ast(BG; R)$ is formal as an $\E{3}$-$R$-algebra, then we can also identify $\H^\ast_G(\Omega G; R)$ as an \textit{$R$-algebra}.

In any case, since $R$ is not necessarily a $\QQ$-algebra, the Hochschild-Kostant-Rosenberg theorem need not give an isomorphism between $\pi_\ast \left(\int_{S^2} \ld{\fr{t}}_R[2]\mmod W\right)$ and $\Sym_{\ld{\fr{t}}[2]\mmod W} (\Omega^1_{\ld{\fr{t}}[2]\mmod W}[2])$; rather, there will always be a ``HKR'' filtration on $\pi_\ast \left(\int_{S^2} \ld{\fr{t}}_R[2]\mmod W\right)$ whose associated graded is given by $\Sym_{\ld{\fr{t}}[2]\mmod W} (\Omega^1_{\ld{\fr{t}}[2]\mmod W}[2])$. 
If this filtration splits, we conclude that the cohomology ring $\H^\ast_G(\Omega G; R)$ will admit divided powers on the $\co_{\ld{\fr{t}}_R[2]\mmod W}$-algebra generators $\Omega^1_{\ld{\fr{t}}_R[2]\mmod W}$. 
The assumption that the HKR filtration splits seems likely to hold if some primes are assumed to be units in $R$ (e.g., if $\dim(\fr{t})!\in R^\times$). Note that by virtue of the argument establishing \cref{general fact homology S2}, the divided power structure on $\H^\ast_G(\Omega G; R)$ is closely related to the $\E{3}$-algebra structure on the derived Satake category.

The preceding discussion is directly connected with a question asked by Bezrukavnikov about divided powers in the cohomology of the affine Grassmannian (see \cite{roman-MO}). It would be interesting to determine the exact conditions under which the above assumptions on $R$ hold true (namely, $C^\ast(BG; R)$ being formal as an $\E{3}$-$R$-algebra, \cref{general fact homology S2}, and the splitting of the HKR filtration for $\int_{S^2} \ld{\fr{t}}_R[2]\mmod W$). The formality of $C^\ast(BG; R)$ seems to be the thorniest of these conditions, but we nevertheless hope that \cref{general fact homology S2} could be useful in approaching Bezrukavnikov's question.
\end{remark}

\begin{example}
Recall that there is an equivalence
$$\Loc_{G}^\gr(G; \KU) \otimes \QQ \simeq \QCoh(\ld{T}\mmod W \times_{\ld{G}/\ld{G}} \ld{T}\mmod W),$$
where all objects on the coherent side are defined over $\QQ$. Since $\ld{T}\mmod W \times_{\ld{G}/\ld{G}} \ld{T}\mmod W \cong (T \times \ld{T})^\bl\mmod W$ is isomorphic to the group scheme of regular centralizers in $\ld{G}$, we will write write $\ld{J}_{\GG_m}$ to denote $\ld{T}\mmod W \times_{\ld{G}/\ld{G}} \ld{T}\mmod W$.
The above equivalence therefore states that
\begin{equation}\label{loc-g-ku}
    \Loc_G^\gr(G; \KU) \otimes \QQ \simeq \QCoh(\ld{J}_{\GG_m}).
\end{equation}
There is a multiplicative analogue of \cite[Theorem 3.4.2]{riche}, which says that when $\ld{G}$ is adjoint, the Lie algebra of $\ld{J}_{\GG_m}$ over $\ld{T}\mmod W$ is isomorphic to $T^\ast(\ld{T}\mmod W)$. Therefore, \cref{hh-mg} and the Hochschild-Kostant-Rosenberg theorem gives an isomorphism
$$\KU^\ast_G(G) \otimes \QQ \cong \pi_\ast \HH(\ld{T}\mmod W / \Z) \otimes_\Z \QQ[\beta^{\pm 1}] \cong \co_{T[-1](\ld{T}\mmod W)} \otimes_\Z \QQ[\beta^{\pm 1}].$$
In particular, there is an equivalence
\begin{equation}\label{coherent sheaves KU-coh}
    \Mod_{\KU_G^0(G)} \otimes \QQ \simeq \QCoh(T[-1](\ld{T}\mmod W)) \otimes_\Z \QQ.
\end{equation}
By Koszul duality, the right-hand side is equivalent to the $2$-periodification of the $\infty$-category of ind-coherent sheaves over the formal completion of $\ld{J}_{\GG_m}$ at the zero section.
One can view the resulting description of $\Mod_{\KU_G^0(G)} \otimes \QQ$ as a infinitesimal version of the equivalence \cref{loc-g-ku}. By construction, the equivalence \cref{coherent sheaves KU-coh} is just a restatement of the Brylinski-Zhang isomorphism $\KU_G^0(G) \otimes \QQ \cong \Omega^\ast_{K_0(\Rep(G))/\Z} \otimes_\Z \QQ$.
\end{example}
\begin{remark}
Just as in \cref{BF coh of gr}, we can also specialize \cref{A-cohomology of Gr} to the case of K-theory. Then, we have
\begin{equation}\label{Kthy fact homology S2}
    \KU^\ast_G(\Omega G) \cong \pi_\ast \left(\int_{S^2} \cM_G\right).
\end{equation}
Here, $\cM_G = \spec \KU_G$ as a $\KU$-scheme, and the factorization homology is taken over $\KU$. 
%The right-hand side above has a filtration (induced by the double-speed Postnikov filtration on $\co_{\cM_G} = \KU_G$) whose graded pieces are given by the $2$-periodification of $\int_{S^2} \ld{T}\mmod W$, where the factorization homology is taken over $\Z$. As in \cref{BF coh of gr}, $\int_{S^2} \ld{T}\mmod W$ can be identified with the $2$-periodification of the divided power algebra $\Gamma_{\ld{T}\mmod W} (\Omega^1_{\ld{T}\mmod W})$. Upon rationalization, we conclude that $\KU^\ast_G(\Omega G) \otimes \QQ$ has a filtration whose graded pieces are given by the $2$-periodification of the ring of functions on the tangent bundle $T(\ld{T}\mmod W)$.
\end{remark}
In general, $\Mod_{\pi_0 C_{G}^\ast(G; A)} \otimes \QQ$ is an ``infinitesimal analogue'' of $\Loc_{G}^\gr(G; A)$. The equivalence of \cref{hh-mg} can therefore be viewed as a infinitesimal version of the analogue of the equivalence of \cref{intro-mirror-dual-of-g} for $\Loc_{G}^\gr(G; A) \otimes \QQ$.