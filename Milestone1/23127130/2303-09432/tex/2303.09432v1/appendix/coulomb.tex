In this brief appendix, we explain some motivation for the results of this article from the perspective of Coulomb branches of $4$d $\cN=2$ and $5$d $\cN=1$ gauge theories with a generic choice of complex structure. Our goal here is not to be precise, but instead explain some motivation for the ideas in this article. While reading this appendix, the reader should keep in mind that I know very little physics!
\begin{recall}
In \cite{bfn-ii, nakajima-coulomb} (see also \cite{nakajima-intro}), it is argued that the Coulomb branch of $3$d $\cN=4$ pure gauge theory on $\RR^3$ can be modeled by the algebraic symplectic variety $\cM_C := \spec \H^G_\ast(\Gr_G(\cc); \cc)$ over $\cc$. The calculations of \cite{bfm} say that $\cM_C$ is isomorphic to $(T^\ast \ld{T})^\bl\mmod W$. This is in turn isomorphic by \cite[Theorem 3]{bf-derived-satake} to the phase space of the Toda lattice for $\ld{G}$, as well as to the moduli space of solutions of Nahm's equations on $[-1,1]$ for a compact form of $\ld{G}$ by \cite[Theorem A.1]{bfn-ii} with an appropriate boundary condition.
The \textit{quantized} Coulomb branch of $3$d $\cN=4$ pure gauge theory on $\RR^3$ is then modeled by $\cA_\epsilon := \H^{G\times S^1_\rot}_\ast(\Gr_G(\cc); \cc)$. In \cite{bfm}, $\cA_\epsilon$ was identified with the algebra of operators of the quantized Toda lattice for $\ld{G}$.
\end{recall}
\begin{remark}
The physical reason for the definition of $\cA_\epsilon$ is the ``$\Omega$-background'' (introduced in \cite{nek-shat}); we refer the reader to \cite{ben-zvi-susy, teleman-icm} for helpful expositions on this topic. The essential idea is as follows: $C^G_\ast(\Gr_G(\cc); \cc)$ admits the structure of an $\Efr{3}$-algebra. In particular, the $\E{3}$-algebra structure on $C^G_\ast(\Gr_G(\cc); \cc)$ is equivariant for the action of $S^1$ on $C^G_\ast(\Gr_G(\cc); \cc)$ via loop rotation, and the action of $S^1$ on $\E{3}$ via rotation about a line $\ell\subseteq \RR^3$. Using the fact that the fixed points of the $S^1$-action on $\RR^3$ are given by the line $\ell$, it is argued in \cite{ben-zvi-susy} that the homotopy fixed points of $C^G_\ast(\Gr_G(\cc); \cc)$ admits the structure of an $\E{1}$-$C_{S^1}^\ast(\ast; \cc)$-algebra. Furthermore, the associative multiplication on $C^{G\times S^1_\rot}_\ast(\Gr_G(\cc); \cc)$ is argued to degenerate to the $2$-shifted Poisson bracket on $\H^G_\ast(\Gr_G(\cc); \cc)$ obtained from the $\E{3}$-algebra structure. The ``$\Omega$-background'' is supposed to refer to the compatibility of the $S^1$-action on $C^G_\ast(\Gr_G(\cc); \cc)$ with the $S^1$-action on the $\E{3}$-operad.

From the mathematical perspective, the idea that $S^1$-actions can be viewed as a deformation quantizations has been made precise by \cite{preygel, toen-icm}, and more recently in \cite{butson-i, butson-ii} (at least in characteristic zero). Although often not said explicitly, the idea has been a cornerstone of Hochschild homology. (The reader can skip the following discussion, since it will not be necessary in the remainder of this section; we only include it for completeness.) 

Consider a smooth $\cc$-scheme $X$, so that the HKR theorem gives an isomorphism $\HH(X/\cc) \simeq \Sym(\Omega^1_{X/\cc}[1])$. There is an isomorphism $\Sym(\Omega^1_{X/\cc}[1]) \simeq \bigoplus_{n\geq 0} (\wedge^n \Omega^1_{X/\cc})[n]$, so $\Sym(\Omega^1_{X/\cc}[1])$ can be understood as a shearing of the algebra $\Omega^\ast_{X/\cc} = \bigoplus_{n\geq 0} (\wedge^n \Omega^1_{X/\cc})[-n]$ of differential forms. The HKR theorem further states that the $S^1$-action on $\HH(X/\cc)$ is a shearing of the de Rham differential on $\Omega^\ast_{X/\cc}$. The Koszul dual of the algebra $\HH(X/\cc) \simeq \Sym(\Omega^1_{X/\cc}[1])$ is $\Sym(T_{X/\cc}[-2]) \simeq \co_{T^\ast[2] X}$; in the same way, the sheaf of differential operators on $X$ is Koszul dual to the de Rham complex of $X$. This can be drawn pictorially as follows:
$$\xymatrix{
\Sym(T_{X/\cc}[-2]) \simeq \co_{T^\ast[2] X} \ar@{~>}[r]^-{\text{def. quant}} \ar@{~>}[d]_-{\text{Koszul dual}} & \cd^\hbar_{X/\cc} \ar@{~>}[d]^-{\text{Koszul dual}} \\
\Sym_{\co_X}(\Omega^1_{X/\cc}[1]) \simeq \HH(X/\cc) \ar@{~>}[r]_-{S^1\text{-action}} & \text{shearing of }(\Omega^\ast_{X/k}, d_\dR).
}$$
Since the algebra $\cd_X^\hbar$ of differential operators is a quantization of $T^\ast[2] X$, this drawing illustrates that the $S^1$-action on Hochschild homology plays the role of a Koszul dual to deformation quantization.
\end{remark}
\begin{example}\label{3d-sl2}
We will keep $G = \SL_2$ as a running example in discussing Coulomb branches (see also \cite[Section 2]{seiberg-witten-coulomb}), so that $\ld{G} = \PGL_2$. In this case, $\cM_C \cong \spec \cc[x, t^{\pm 1}, \frac{t-1}{x}]^{\Z/2} \cong \spec \cc[x^2, t+t^{-1}, \frac{t-t^{-1}}{x}]$ by \cref{t-homology-grg} (and \cite{bfm}), where $\Z/2$ acts on $\cc[x, t^{\pm 1}, \frac{t-1}{x}]$ by $x\mapsto -x$ and $t\mapsto t^{-1}$. %atiyah hitchin 
Let us denote $\Phi = x^2$, $U = t + t^{-1}$, and $V = \frac{t-t^{-1}}{x}$. Then
$$U^2 - \Phi V^2 = (t+t^{-1})^2 - (t-t^{-1})^2 = 4,$$
so $\cM_C$ is isomorphic to the subvariety of $\AA^3_\cc$ cut out by the equation 
$$U^2 - \Phi V^2 = 4.$$
Alternatively, and perhaps more suggestively:
$$(U+2)(U-2) = \Phi V^2.$$
This is known as the \textit{Atiyah-Hitchin manifold}, and was studied in great detail in \cite{atiyah-hitchin} (see \cite[Page 20]{atiyah-hitchin} for the definition). In \cite[Theorem A.1]{bfn-ii}, it was shown that the Atiyah-Hitchin manifold is isomorphic to the moduli space of solutions of Nahm's equations on $[-1,1]$ for $\mathrm{PSU}(2)$ with an appropriate boundary condition.
Since a normal vector to the defining equation of $\cM_C$ is $2U\partial_U - V^2 \partial_\phi - 2V\Phi \partial_V$, the standard holomorphic $3$-form $dU \wedge d\Phi \wedge dV$ on $\AA^3_\cc$ induces a holomorphic symplectic form $\frac{d\Phi \wedge dV}{2U}$ on $\cM_C$. (This can also be written as $\frac{dU \wedge dV}{V^2}$ or as $\frac{d\Phi \wedge dU}{2\Phi V}$.) The associated Poisson bracket on $\co_{\cM_C} \cong \H^G_\ast(\Gr_G(\cc); \cc)$ agrees with the $2$-shifted Poisson bracket arising from the $\E{3}$-structure on $C^G_\ast(\Gr_G(\cc); \cc)$.

The quantized algebra $\cA_\epsilon$ was described explicitly in \cite{bf-derived-satake}. Let us write $\theta = \frac{1}{x}(s-1)$, where $s$ is the simple reflection generating the Weyl group of $\SL_2$. Then $\cA_\epsilon$ is generated as an algebra over $\cc\pw{\hbar}$ by $\Z/2$-invariant polynomials in $x$, $t^{\pm 1}$, and $\theta$, where $x$ is to be viewed as $t\partial_t$. Moreover, under the isomorphism $\cA_\epsilon/\hbar \cong \co_{\cM_C}$, the class $x$ is sent to $x$, and $\theta$ is sent to $\frac{t-1}{x}$. We then have the commutation relation $[x,t^{\pm 1}] = \pm \hbar t^{-1}$, induced by $[\partial_t, t] = \hbar$; see \cref{ordinary quantized diffop}. This implies that $[x^2, t^{\pm 1}] = \hbar^2 t^{\pm 1} \pm 2\hbar t^{\pm 1} x$, which in turn implies that $\cA_\epsilon$ is the quotient of the free associative $\cc\pw{\hbar}$-algebra on $\Phi$, $U$, and $V = \frac{1}{x}(t-t^{-1})$ subject to the relations
%\begin{align*}
%    tx & = xt-\hbar t, \\
%    t^{-1} x & = xt^{-1} + \hbar t^{-1}, \\
%    x(t-t^{-1}) & = (t-t^{-1})x + \hbar(t+t^{-1}), \\
%    \left[\frac{1}{x}, t\right] & = -\hbar \frac{1}{x} t \frac{1}{x}, \\
%    % since 0 = [x*1/x,t] = x[1/x, t] + [x,t] * 1/x = x[1/x, t] + hbar * t * 1/x
%    \left[\frac{1}{x}, t^{-1}\right] & = \hbar \frac{1}{x} t^{-1} \frac{1}{x}, \\
%    \left[\frac{1}{x}, t+t^{-1}\right] & = -\hbar \frac{1}{x} (t-t^{-1}) \frac{1}{x}.
%\end{align*}
\begin{align*}
    [\Phi,V] 
    %& = \left[x^2, \frac{1}{x}(t-t^{-1})\right] \\
    %& = x(t-t^{-1}) - \frac{1}{x}(tx^2 - t^{-1}x^2)\\
    %& = x(t-t^{-1}) - \frac{1}{x} ([x^2 t - \hbar^2 t - 2\hbar tx] - [x^2 t^{-1} - \hbar^2 t^{-1} + 2\hbar t^{-1} x])\\
    %& = \frac{\hbar^2}{x}(t-t^{-1}) + \frac{2\hbar}{x}(t+t^{-1})x \\
    %& = \frac{\hbar^2}{x}(t-t^{-1}) + \frac{2\hbar}{x}(xt - \hbar t + xt^{-1} + \hbar t^{-1})\\
    %& = \frac{\hbar^2}{x}(t-t^{-1}) + 2\hbar \left(t+t^{-1} - \frac{\hbar}{x}(t-t^{-1})\right)\\
    & = 2\hbar U - \hbar^2 V,\\
    [\Phi, U]
    %& = [x^2, t] + [x^2, t^{-1}]\\
    %& = \hbar^2 t + 2\hbar tx + \hbar^2 t^{-1} - 2\hbar t^{-1} x\\
    %& = \hbar^2(t+t^{-1}) - 2\hbar(t-t^{-1})x \\
    %& = \hbar^2(t+t^{-1}) - 2\hbar([xt-\hbar t]-[xt^{-1}+\hbar t^{-1}]) \\
    %& = 2\hbar x(t-t^{-1}) - \hbar^2(t+t^{-1})\\
    & = 2\hbar \Phi V - \hbar^2 U, \\
    [U,V]
    %& = \left[t+t^{-1}, \frac{1}{x} (t-t^{-1})\right] \\
    %& = \left((t+t^{-1}) \frac{1}{x} (t-t^{-1}) - \frac{1}{x} (t-t^{-1}) (t+t^{-1}) \right) - \left(\frac{1}{x} (t+t^{-1}) (t-t^{-1}) - \frac{1}{x} (t+t^{-1}) (t-t^{-1})\right) \\
    %& = (t+t^{-1}) \frac{1}{x} (t-t^{-1}) - \frac{1}{x} (t+t^{-1}) (t-t^{-1})\\
    %& = \left[t+t^{-1}, \frac{1}{x}\right] (t-t^{-1})\\
    %& = \hbar \frac{1}{x} (t-t^{-1}) \frac{1}{x} (t-t^{-1}) \\
    & = \hbar V^2,\\
    %U^2 - \Phi V^2
    %& = (t+t^{-1})^2 - x(t-t^{-1}) \frac{1}{x}(t-t^{-1})\\
    %& = (t+t^{-1})^2 - \left((t-t^{-1})x + \hbar (t+t^{-1})\right) \frac{1}{x}(t-t^{-1})\\
    %& = (t+t^{-1})^2 - (t-t^{-1})^2 + \hbar (t+t^{-1}) \frac{1}{x}(t-t^{-1})\\
    %& = (t+t^{-1})^2 - (t-t^{-1})^2 - \hbar (t+t^{-1}) \frac{1}{x} (t-t^{-1})\\
    %& = 4 - \hbar UV.
    (U+2)(U-2) & = \Phi V^2 - \hbar UV.
\end{align*}
Note that the commutation relations for $[\Phi, U]$ and $[U,V]$ in \cite[Equation B.3]{dimofte-garner} have typos, but it is stated correctly in \cite[Equation 5.51]{bullimore-dimofte-gaiotto}. The above is an explicit description of the nil-Hecke algebra $e\cH(\tilde{\fr{t}}, W^\aff)e$ for $\PGL_2$.
\end{example}
\begin{heuristic}\label{4d-n2}
An unpublished conjecture of Gaiotto (which I learned about from Nakajima) says that the Coulomb branch of $4$d $\cN=2$ pure gauge theory over $\RR^3 \times S^1$ with a generic choice of complex structure can be modeled by $\cM_C^\fourd := \spec \KU^G_0(\Gr_G(\cc)) \otimes \cc$. Although I do not know Gaiotto's motivation for this conjecture (it is probably inspired by \cite{seiberg-witten-coulomb}), my attempt at heuristically justifying it goes as follows. Recall that $\Gr_G(\cc)/G(\cc\pw{t})$ can be viewed as $\Bun_G(S^2)$. It is reasonable to view $\KU_0(\Bun_G(S^2)) \otimes \cc$ as closely related to $\H_\ast(\cL \Bun_G(S^2); \cc)$, where $\cL \Bun_G(S^2)$ denotes the free loop space. Since $\cL BG \simeq B\cL G$, we have $\cL \Bun_G(S^2) \simeq \Bun_{\cL G}(S^2)$, so one might view $\H_\ast(\cL \Bun_G(S^2); \cc)$ as the ring of functions on the ``Coulomb branch of $3$d $\cN=4$ pure gauge theory on $\RR^3$ with gauge group $\cL G$''.

Making precise sense of this phrase seems difficult, but one possible workaround could be the following. It is often useful to view gauge theory with gauge group $\cL G$ as ``finite temperature'' gauge theory with gauge group $G$. Recall that Wick rotation relates $(3+1)$-dimensional quantum field theory at a finite temperature $T$ to statistical mechanics over $\RR^3 \times S^1$ where the circle has radius $\frac{1}{2\pi T}$. This suggests that $\H_\ast(\cL \Bun_G(S^2); \cc)$ (which is more precisely to be understood as $\KU^G_0(\Gr_G(\cc)) \otimes \cc$) can be viewed as the ring of functions on the ``Coulomb branch of $4$d $\cN=2$ pure gauge theory on $\RR^3 \times S^1$ with gauge group $G$''. See \cite[Remark 3.14]{bfn-ii}. In \cite{bfm}, $\spec \KU^G_0(\Gr_G(\cc)) \otimes \cc$ was identified with the phase space of the relativistic Toda lattice for $\ld{G}$.

One can also define a quantization of $\cM_C^\fourd$ via $\cA_\epsilon^\fourd := \KU^{G\times S^1_\rot}_0(\Gr_G(\cc)) \otimes \cc$; this can be viewed as a model for the quantized Coulomb branch of $4$d $\cN=2$ pure gauge theory on $\RR^3\times S^1$. In \cite{bfm}, $\cA_\epsilon^\fourd$ was be identified with the algebra of operators of the quantized relativistic Toda lattice for $\ld{G}$.
\end{heuristic}
\begin{example}\label{4d-sl2}
When $G = \SL_2$, the calculations of \cref{t-homology-grg} and \cite{bfm} tell us that $\cM_C^\fourd \cong \spec \cc[x^{\pm 1}, t^{\pm 1}, \frac{t-1}{x-1}]^{\Z/2} \cong \spec \cc[x+x^{-1}, t+t^{-1}, \frac{t-t^{-1}}{x-x^{-1}}]$, where $\Z/2$ acts on $\cc[x^{\pm 1}, t^{\pm 1}, \frac{t-1}{x-1}]$ by $x\mapsto x^{-1}$ and $t\mapsto t^{-1}$. Let us write $\Psi = x + x^{-1}$, $W = t+t^{-1}$, and $Z = \frac{t-t^{-1}}{x-x^{-1}}$. Then, one easily verifies that $\cM_C^\fourd$ is the subvariety of $\AA^3_\cc$ cut out by the equation
%a = Psi, b = -i/2 * (W+Z*Psi), c = -i*Z, then abc - b^2 - c^2 = 1
$$W^2 - (\Psi^2 - 4)Z^2 = 4.$$
Alternatively, and perhaps more suggestively:
$$(W+2)(W-2) = (\Psi+2)(\Psi-2)Z^2.$$
This may be regarded as an a multiplicative analogue of the Atiyah-Hitchin manifold. It would be very interesting to understand a relationship between this manifold and the moduli space of solutions to some analogue of Nahm's equations for $\mathrm{PSU}(2)$ with an appropriate boundary condition. The complex manifold $\cM_C^\fourd$ has a holomorphic symplectic form given by $\frac{d\Psi \wedge dZ}{W}$, which can also be written as $\frac{d\Psi \wedge dW}{(\Psi^2-4)Z}$ or as $\frac{dZ \wedge dW}{\Psi Z^2}$.

We can also describe the quantized algebra $\cA_\epsilon^\fourd$ explicitly. In this case, instead of the relation $[\partial_t, t] = \hbar$ which appeared in \cref{3d-sl2}, we have the relation $xt = qtx$ (i.e., $xtx^{-1} t^{-1} = q$); see \cref{q quantized diffop}. In particular, $xt^{-1} = q^{-1} t^{-1} x$, $x^{-1} t = q^{-1} tx^{-1}$, and $x^{-1} t^{-1} = qt^{-1}x^{-1}$. It follows after some tedious calculation that $\cA_\epsilon^\fourd$ is the quotient of the free associative $\cc\pw{q-1}$-algebra (in fact, $\cc[q^{\pm 1}]$-algebra) on $\Psi$, $W$, and $Z = \frac{1}{x-x^{-1}} (t-t^{-1})$ subject to the relations
%\begin{align*}
%    q^{-1}-1 & = -q^{-1}(q-1),\\
%    q-q^{-1} & = [2]_q(q-1)/q,\\
%    (q-1)(q^{-1}-1) & = -\frac{(q-1)^2}{q},\\
%    (q-1)(q^{-1}+1) & = q-q^{-1},\\
%    \Psi^2-4 & = (x-x^{-1})^2,\\
%    (\Psi^2-4)Z & = (x-x^{-1})(t-t^{-1}),\\
%    (\Psi^2-4)Z + \Psi W & = (x-x^{-1})(t-t^{-1}) + (x+x^{-1})(t+t^{-1})\\
%    & = 2(xt + x^{-1} t^{-1})\\
%    \Psi W - (\Psi^2-4)Z & = (x+x^{-1})(t+t^{-1}) - (x-x^{-1})(t-t^{-1})\\
%    & = 2(xt^{-1} + x^{-1}t)\\
%    [t-t^{-1}, x-x^{-1}] & = (tx- tx^{-1} - t^{-1} x + t^{-1} x^{-1} - xt + xt^{-1} + x^{-1} t - x^{-1} t^{-1})\\
%    & = (q-1)(q^{-1}(tx^{-1} + t^{-1} x) - (tx + t^{-1} x^{-1}))\\
%    & = \frac{(q-1)}{2} (q^{-1}(\Psi W - (\Psi^2-4)Z) - ((\Psi^2-4)Z + \Psi W))\\
%    & = \frac{(q-1)}{2} ((q^{-1}-1) \Psi W - (q^{-1}+1) (\Psi^2-4) Z)\\
%    & = \frac{(q-1)}{2} ((q^{-1}-1) \Psi W - (q^{-1}+1) (\Psi^2-4) Z)\\
%    (t-t^{-1}) \frac{1}{x-x^{-1}} & = \frac{1}{x-x^{-1}} (t-t^{-1}) + \frac{1}{x-x^{-1}} [t-t^{-1}, x-x^{-1}] \frac{1}{x-x^{-1}}\\
%    (x+x^{-1})(t-t^{-1}) + (x-x^{-1})(t+t^{-1}) & = 2(xt - x^{-1} t^{-1})\\
%    (x+x^{-1}) \cdot \frac{1}{x-x^{-1}} (t-t^{-1}) + (t+t^{-1}) & = 2\frac{1}{x-x^{-1}}(xt - x^{-1} t^{-1})\\
%    \frac{1}{2}(\Psi Z + W) & = \frac{1}{x-x^{-1}}(xt - x^{-1} t^{-1})\\
%%% for ANY f
%    [\frac{1}{x-x^{-1}}, f] & = \frac{1}{x-x^{-1}} [x-x^{-1}, f] \frac{1}{x-x^{-1}}\\
%    [x-x^{-1}, t+t^{-1}] & = xt - tx + xt^{-1} - t^{-1} x - x^{-1} t + tx^{-1} - x^{-1} t^{-1} + t^{-1} x^{-1}\\
%    [x-x^{-1}, t+t^{-1}] & = (q-1)(q^{-1} xt - xt^{-1} + x^{-1} t - q^{-1} x^{-1} t^{-1})\\
%    [x-x^{-1}, t+t^{-1}] & = (q-1)((q^{-1}-1) (xt - x^{-1} t^{-1}) + (x+x^{-1})(t-t^{-1}))\\
%    \frac{1}{x-x^{-1}} [x-x^{-1}, t+t^{-1}] & = (q-1)((q^{-1}-1) \frac{1}{x-x^{-1}}  (xt - x^{-1} t^{-1}) + (x+x^{-1}) \frac{1}{x-x^{-1}} (t-t^{-1}))\\
%    \frac{1}{x-x^{-1}} [x-x^{-1}, t+t^{-1}] & = (q-1)((q^{-1}-1) \frac{\Psi Z+W}{2} + \Psi Z)\\
%\end{align*}
\begin{align*}
    [\Psi, W] 
    %& = [x+x^{-1}, t+t^{-1}]\\
    %& = (xt + xt^{-1} + x^{-1} t + x^{-1} t^{-1}) - (q^{-1} xt + q xt^{-1} + q x^{-1} t + q^{-1} x^{-1} t^{-1})\\
    %& = (1-q^{-1}) xt + (1-q) xt^{-1} + (1-q) x^{-1} t + (1-q^{-1}) x^{-1} t^{-1}\\
    %& = (q-1)(q^{-1} xt - xt^{-1} - x^{-1} t + q^{-1} x^{-1} t^{-1})\\
    %& = (q-1)(q^{-1} xt - xt + xt - xt^{-1} - x^{-1} t + x^{-1} t^{-1} - x^{-1} t^{-1} + q^{-1} x^{-1} t^{-1}) \\
    %& = (q-1)((x-x^{-1}) (t-t^{-1}) + (q^{-1}-1) (xt+x^{-1} t^{-1})) \\
    %& = (q-1)(\Psi^2-4)Z + \frac{(q-1)(q^{-1}-1)}{2} ((\Psi^2-4)Z + \Psi W) \\
    & = (q-1)(\Psi^2-4)Z - \frac{(q-1)^2}{2q} ((\Psi^2-4)Z + \Psi W), \\
    %& = (q-1)(\Psi^2-4)Z - \frac{1}{2} \sum_{n\geq 2} (-1)^n (q-1)^n ((\Psi^2-4)Z + \Psi W), \\
    %%%%%%%%%%%%%%%
    [\Psi, Z] 
    %& = [x+x^{-1}, \frac{1}{x-x^{-1}} (t-t^{-1})]\\
    %& = \frac{1}{x-x^{-1}} [x+x^{-1}, t-t^{-1}]\\
    %& = \frac{1}{x-x^{-1}} ((1-q^{-1}) xt - (1-q) xt + (1-q) x^{-1} t - (1-q^{-1}) x^{-1} t^{-1})\\
    %& = \frac{q-1}{x-x^{-1}} (q^{-1} xt + xt^{-1} - x^{-1} t - q^{-1} x^{-1} t^{-1})\\
    %& = \frac{q-1}{x-x^{-1}} ((q^{-1}-1) xt + x(t+t^{-1}) - x^{-1} (t+t^{-1}) + (1-q^{-1}) x^{-1} t^{-1})\\
    %& = \frac{q-1}{x-x^{-1}} ((x-x^{-1})(t+t^{-1}) + (q^{-1}-1)(xt - x^{-1} t^{-1}))\\
    %& = (q-1)W + \frac{(q-1)(q^{-1}-1)}{x-x^{-1}} (xt - x^{-1} t^{-1})\\
    %& = (q-1)W + \frac{(q-1)(q^{-1}-1)}{2}(\Psi Z + W),\\
    & = (q-1)W - \frac{(q-1)^2}{2q}(\Psi Z + W),\\
    %& = (q-1)W - \frac{1}{2} \sum_{n\geq 2} (-1)^n (q-1)^n (\Psi Z + W) ,\\
    %%%%%%%%%%%%%%%
    [Z,W] 
    %& = \frac{1}{x-x^{-1}} (t-t^{-1}) (t+t^{-1}) - (t+t^{-1}) \frac{1}{x-x^{-1}} (t-t^{-1})\\
    %& = [\frac{1}{x-x^{-1}}, t+t^{-1}] (t-t^{-1})\\
    %& = \frac{1}{x-x^{-1}} [x-x^{-1}, t+t^{-1}] \frac{1}{x-x^{-1}} (t-t^{-1})\\
    %& = (q-1)((q^{-1}-1) \frac{\Psi Z+W}{2} + \Psi Z) Z\\
    %& = (q-1) \Psi Z^2 + \frac{(q-1)(q^{-1}-1)}{2} (\Psi Z + W)Z, \\
    & = (q-1) \Psi Z^2 - \frac{(q-1)^2}{2q} (\Psi Z + W)Z, \\
    %& = (q-1) \Psi Z^2 - \frac{1}{2} \sum_{n\geq 2} (-1)^n (q-1)^n (\Psi Z + W)Z , \\
    %%%%%%%%%%%
    %(\Psi^2-4) Z^2 -W^2 
    %& = (x-x^{-1})(t-t^{-1}) \frac{1}{x-x^{-1}} (t-t^{-1}) -(t-t^{-1})^2 - 4 \\
    %& = (x-x^{-1})(t-t^{-1}) \frac{1}{x-x^{-1}} (t-t^{-1}) -(t-t^{-1})^2 - 4\\
    %& = (t-t^{-1})^2 + [t-t^{-1}, x-x^{-1}] \frac{1}{x-x^{-1}} (t-t^{-1}) -(t-t^{-1})^2 - 4\\
    %& = [t-t^{-1}, x-x^{-1}] Z - 4\\
    %& = \frac{(q-1)}{2} ((q^{-1}-1) \Psi W - (q^{-1}+1) (\Psi^2-4) Z) Z - 4\\
    %& = \frac{(q-1)^2}{2q} \Psi W Z - \frac{q-q^{-1}}{2} (\Psi^2-4)Z^2 - 4\\
    %%%%%%% or instead of the final two lines, note:
    %[t-t^{-1}, x-x^{-1}] & = (tx-xt - tx^{-1} + x^{-1} t - t^{-1} x + xt^{-1} + t^{-1} x^{-1} - x^{-1} t^{-1})\\
    %& = ((q^{-1}-1) xt + (1-q) xt^{-1} + (1-q) x^{-1} t - (1-q^{-1}) x^{-1} t^{-1})\\
    %& = (q-1)(-q^{-1} xt - xt^{-1} - x^{-1} t - q^{-1} x^{-1} t^{-1})\\
    %& = (q-1)((x-x^{-1})(t-t^{-1}) - (q^{-1}+1)(xt + x^{-1} t^{-1}))\\
    %& = (q-1)(\Psi^2-4) Z - \frac{(q-1)(q^{-1}+1)}{2} ((\Psi^2-4)Z + \Psi W)\\
    %& = (q-1)(\Psi^2-4) Z - \frac{q^2-1}{2q} ((\Psi^2-4)Z + \Psi W)\\
    %& = \frac{(q-1)^2}{2q} (\Psi^2-4) Z - \frac{q^2-1}{2q} \Psi W\\
    %% negate the final answer from the previous line
    %%%%
    %W^2 - (\Psi^2-4)Z^2 & = 4 - \frac{(q-1)^2}{2q} (\Psi^2-4)Z^2 + \frac{q^2-1}{2q} \Psi WZ.
    (W+2)(W-2) & = (\Psi+2)(\Psi-2)Z^2 - \frac{(q-1)^2}{2q} (\Psi^2-4)Z^2 + \frac{q^2-1}{2q} \Psi WZ.
\end{align*}
This algebra is an explicit description of the multiplicative nil-Hecke algebra $e\cH(\tilde{T}, W^\aff)e$ from \cref{oq-univ} for $\PGL_2$.
\end{example}

Now consider an elliptic curve $E(\cc)$ over $\cc$. Motivated by \cref{4d-n2} and \cite{nakajima-yoshioka}, one might expect that the Coulomb branch of $5$d $\cN=1$ pure gauge theory over $\RR^3 \times E(\cc)$ (with some specific complex structure) can be modeled by the complexification of the $G$-equivariant $A$-homology of $\Gr_G(\cc)$, where $A$ is an elliptic cohomology theory associated to a putative integral lift of $E$. A classical result of Tate says that there are no smooth elliptic curves over $\Z$, so $E(\cc)$ cannot literally lift to $\Z$ (i.e., $\pi_0(A)$ cannot be $\Z$). As a fix, one can more generally simultaneously consider all possible ``Coulomb branches'' $\cM_C^\fived := \spec A^G_0(\Gr_G(\cc)) \otimes \cc$ associated to every complex-oriented even-periodic $\Eoo$-ring $A$ equipped with an oriented elliptic curve (this is almost equivalent to considering the universal example $\spec \tmf^G_0(\Gr_G(\cc)) \otimes \cc$). We have described $\spec A^T_0(\Gr_G(\cc)) \otimes \cc$ in \cref{t-homology-grg}, from which one can calculate $\cM_C^\fived$. Similarly, one can even use \cref{loop-rot-coh-grg} to calculate $A^{T\times S^1_\rot}_0(\Gr_G(\cc)) \otimes \cc$ and $\cA_\epsilon^\fived := A^{G\times S^1_\rot}_0(\Gr_G(\cc)) \otimes \cc$, but this is already incredibly complicated for $G = \SL_2$.
\begin{example}
Let $A$ be a complex-oriented even-periodic $\Eoo$-ring equipped with an oriented elliptic curve $\tilde{E}$, and let $E$ denote the associated elliptic curve over $\pi_0(A)\otimes \cc$. Let $(\GG_m\times E)^\bl$ denote the complement of the proper preimage of the zero section of $E$ inside the blowup of $\GG_m\times E$ at the locus cut out by the zero sections of $\GG_m$ and $E$. There is an action of $\Z/2$ on $(\GG_m\times E)^\bl$, induced by the inversion on the group structures on $\GG_m$ and $E$. If $G = \SL_2$, then \cref{t-homology-grg} can be used to show that $\cM_C^\fived = \spec A^G_0(\Gr_G(\cc)) \otimes \cc$ is isomorphic to $(\GG_m\times E)^\bl\mmod(\Z/2)$; this can be viewed as an elliptic analogue of the Atiyah-Hitchin manifold. We do not have a simple description for $\cA_\epsilon^\fived$ analogous to \cref{3d-sl2} and \cref{4d-sl2}.
\end{example}
It would be very interesting to give a physical interpretation to $A^G_0(\Gr_G(\cc)) \otimes \cc$ and $A^{G\times S^1_\rot}_0(\Gr_G(\cc)) \otimes \cc$ for other even-periodic $\Eoo$-rings $A$, although we expect this to be very difficult (since most other chromatically interesting generalized cohomology theories only exist after profinite or $p$-adic completion, and do not admit transcendental analogues). It would also be very interesting to describe the analogue of our calculations for the ind-schemes $\cR_{G,\mathbf{N}}$ introduced in \cite{bfn-ii}. By adapting the methods of \cite[Section 4]{bfn-ii}, this is approachable when $G$ is a torus. We expect it to lead to interesting geometry for nonabelian $G$.