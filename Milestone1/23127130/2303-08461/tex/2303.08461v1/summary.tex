\section{Summary and outlook}
\label{sec:summary}

We have proposed the prethermal expectation value problem as a way to study thermal observables on noisy, intermediate-scale quantum devices. Our approach relies on the observation that relatively large Trotter steps, which do not permit a rigorous bound on the Trotter error, can give rise to prethermalization. We showed that in the prethermal regime, the equilibration of observables is similar to the expected dynamics under the original Hamiltonian. It may be possible to approximate evolution under the original Hamiltonian even better by cancelling higher-order terms of the Magnus expansion at the cost of more complex circuits. The range of energies at which the observables can be probed is set by the range of energies of the used intial states. We restricted ourselves to product states for this work, but the protocol can straightforwardly be extended to different initial states, which may increase the range of accessible energies.

We further demonstrated that the prethermal regime is experimentally accessible with noise rates of near-term devices using an error-mitigation scheme based on measuring and rescaling survival probabilities. This scheme is not limited to the PEVP but can be applied much more broadly in the context of quantum simulation. Our work provides all necessary ingredients to also study the approach to equilibrium and to extract, for instance, diffusion constants. Alternatively, one could consider the quantum dynamics of models which do not thermalize, such as quantum scars~\cite{Turner2018, Lin2019} or many-body localized systems~\cite{Pal2010, Abanin2018}.

Our work creates a new avenue to demonstrating useful quantum advantage on noisy devices. Although the XY model studied here can be efficiently simulated on classical computers with quantum Monte Carlo methods~\cite{Ding1992}, our approach can be readily adapted to more complex Hamiltonians. As a simple modification of the XY model, one might consider adding a site-dependent sign to the interaction strength $J$. This renders classical simulation of this model much harder since it causes a sign problem in quantum Monte Carlo methods~\cite{Loh1990, Takasu1986, Hatano1992}. The complexity of our proposed approach to quantum simulation however remains unaffected by this modification. Hence, quantum advantage may be within reach for studying the equilibrium properties of Hamiltonians with a sign problem.


\section*{Acknowledgements}
TEO and VS thank Yaroslav Herasymenko, Robin Kothari and Rolando Somma for useful discussions. We acknowledge the support from the German Federal Ministry of Education and Research (BMBF) through FermiQP (Grant No. 13N15890) and EQUAHUMO (Grant No. 13N16066) within the funding program quantum technologies - from basic research to market. This research is part of the Munich Quantum Valley (MQV), which is supported by the Bavarian state government with funds from the Hightech Agenda Bayern Plus. YY was funded by a grant from Google Quantum AI. DSW has received funding from the European Union’s Horizon 2020 research and innovation programme under the Marie Sk{\l}odowska-Curie Grant Agreement No. 101023276.
The work was partially supported by the Deutsche Forschungsgemeinschaft (DFG, German Research Foundation) under Germany's Excellence Strategy -- EXC-2111 -- 390814868.
