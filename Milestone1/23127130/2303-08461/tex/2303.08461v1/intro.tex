\section{Introduction}

Quantum computers promise to have a great impact on scientific research. A particular example is the study of thermalization of quantum many-body systems. The problem is computationally challenging with classical methods~\cite{Vidal2004, Ido2015, Carleo2017} as it requires simulating the long-time dynamics of large systems. A fault-tolerant quantum computer would render this problem tractable by enabling quantum simulation~\cite{Feynman1982, Lloyd1996, Georgescu2014, Altman2021}. 

Despite impressive recent progress, present day quantum devices are still far from the regime of fault tolerance. Any current quantum simulation is therefore affected by noise and imperfections. In circuit-based quantum computers, continuous-time dynamics can be approximated using, for example, Trotterization~\cite{childs2021}. With state-of-the-art gate errors~\cite{Yoneda2018a, Arute2019, Huang2019a, Wu2021, Mi2022b, Wei2022a}, it is however only possible to run simulations with a controlled Trotter error up to short times, which are insufficient to explore thermalization in classically intractable systems (50 or so qubits in two or more dimensions). 

\begin{figure}[b]
    \centering
    \includegraphics[width=0.45\textwidth]{figs/01_regime.pdf}
    \caption{
        Different regimes of the dynamics of local observables depending on the Trotter step $\tau$ and the evolution time $T$. The black lines separate three regimes: bounded Trotter error (bottom), Floquet prethermalization (middle), and chaotic dynamics (top). The lower line scales as $\mathcal{O}\left( \max\left\{ \tau^{- p/(d+1)}, N^{-1}\tau^{-p} \right\} \right)$, following error bounds for the $p^\mathrm{th}$-order Trotter decomposition with system size $N$. The upper line is determined by the Floquet heating time and scales as $e^{\mathcal{O}\left( 1/\tau \right)}$. The blue shaded area indicates constant maximum circuit depth, relevant for noisy quantum computers. The grey area is excluded due to the constraint that $T \ge \tau$. The red shading highlights where the total time $T$ exceeds a system-dependent (pre)thermalization time scale $T_{\mathrm{th}}$. The prethermalized expectation value problem is experimentally accessible in the purple intersection.
    }
\label{fig:scheme}
\end{figure}

In this work, we demonstrate that thermalization can already be observed for much larger Trotter steps than needed to guarantee a bounded Trotter error, making it feasible to study this phenomenon on near-term quantum devices. In this regime, the system may be viewed as subject to a periodic Floquet drive \cite{Goldman2014,Bukov2015,Moessner2017}, where one Trotter step corresponds to one period. The fate of Floquet systems at late times has been a subject of recent interest~\cite{Sieberer2019, Kargi2021,Morningstar2022}. Even though the system generally heats up to infinite temperatures ~\cite{lazarides2014,dalessio2014}, the heating time may be very long if the driving frequency is large compared to all local energy scales \cite{abanin2015}. The system then \emph{prethermalizes} ~\cite{kuwahara2016,Mori2016, Else2017, Mori2018, Pizzi2021, Ye2021}: Before it heats up, its dynamics mirror the equilibration of a closed system. 
The prethermal regime is relatively easy to access in practice because the Floquet heating time increases exponentially with the driving frequency or, equivalently, the inverse Trotter step size (see Fig.~\ref{fig:scheme}).


With this in mind, we define the \emph{prethermalized expectation value problem} (PEVP): Given a Floquet unitary and a product initial state,  what value does a local observable reach in the prethermal plateau? We find that this problem can be solved even in presence of realistic noise. Following a small circuit adjustment, the PEVP turns out to be amenable to a simple but highly effective error-mitigation scheme based on rescaling survival probabilities. Using this strategy, the error-mitigated PEVP reproduces the equilibrium properties of a model that is closely related to the Hamiltonian underlying the Trotterization. More precisely, the prethermal expectation values describe the diagonal ensemble of this model, which is equivalent to the microcanonical ensemble assuming that the eigenstate thermalization hypothesis (ETH)~\cite{Srednicki1999, Rigol2008a, DAlessio2016, Deutsch2018} is valid. Besides its application to the study of thermalization, the PEVP may be viewed as a problem of independent computational interest in the context of demonstrating quantum advantage.

The paper is structured as follows. In Sec.~\ref{sec:pre}, we discuss thermalization in Floquet systems and present simulation results for the two-dimensional XY model as an example. We introduce our error mitigation strategy based on the rescaling of survival probabilities in Sec.~\ref{sec:mitigation}, where we also provide a thorough numerical analysis of its performance. Equipped with that, we demonstrate the suitability of the PEVP for near-term devices by simulating it with realistic noise parameters of superconducting quantum computers. We conclude in Sec.~\ref{sec:summary}.