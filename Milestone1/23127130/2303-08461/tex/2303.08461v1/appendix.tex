\section{Definition of problems}
\label{sec:def}

\subsection{Setup}
In this section we consider 
\begin{itemize}
    \item a local Hamiltonian $H$ as considered in \ref{sec:equilibration}, with spectral decomposition
        \begin{eqnarray}
            H = \sum_k E_k \ket{k}\bra{k},
        \end{eqnarray}
    \item the Trotterized time-evolution unitary $U_{\mathrm {Trotter}}(\tau)$ (see Eq.\ \ref{def:Utrot}) with time step $\tau$,
    \item an observable $A$ with operator norm $\Vert A \Vert$,
    \item and an initial state $\ket{\psi}$.
\end{itemize}
When also given a Trotter step $\tau$, any time appearing in text will be stroboscopic, i.e., an integer multiple of $\tau$.


\subsection{Definition of thermal ensembles} Here we provide definitions of the microcanonical and diagonal ensembles in Fig.~\ref{fig:prethermal}.
\begin{definition}[The microcanonical ensemble]
    \label{def:micro}
    Given an energy $E$ and energy interval $\delta$, the value of an observable $A$ in the corresponding microcanonical ensemble is defined as
    \begin{equation}
        \langle A \rangle_{\mathrm{micro}, E} = \sum_{k \in I_{E,\delta}} \braket{k | A | k}  / |I_{E, \delta}|,
    \end{equation}
    where $I_{E,\delta} = \left\{ k | |E_k - E| < \delta / 2 \right\}$.
\end{definition}
Alternatively, for the convenience of computation, the energy cutoff may be replaced by a Gaussian filter:
\begin{definition}[The broadened microcanonical ensemble]
    With the same setup as Definition~\ref{def:micro}, the broadened microcanonical ensemble is defined as
    \begin{equation}
        \langle A \rangle_{\mathrm{micro}^{\prime}, E} = \sum_k \braket{k | A | k} e^{-\frac{(E - E_k)^2}{2\delta ^2}} \big/ \sum_k e^{-\frac{(E - E_k)^2}{2\delta ^2}}.
    \end{equation}
\end{definition}
 The two definitions are equivalent in the thermodynamic limit under the eigenstate thermalization hypothesis~\cite{Lu2020, Yang2022}. In Fig.~\ref{fig:prethermal} we take the latter definition, which can be efficiently computed in 1D systems with classical computers using filtering algorithms for $\delta$ being a constant~\cite{Yang2022}.

\begin{definition}[The diagonal ensemble]
    Given a state $\ket{\psi}$, the value of an observable $A$ in the the diagonal ensemble is defined as
    \begin{equation}
        A_{d,\psi} = \sum_k |\langle \psi | k \rangle|^2 \langle k | A | k \rangle.
    \end{equation}
\end{definition}
The diagonal ensemble values are equivalent to the long time average of the initial state $\ket{\psi}$ and observable $A$ for non-degenerate Hamiltonians. It can be approximated again by filtering out the off-diagonal elements of an initial density matrix~\cite{Cakan2021}. The entanglement entropy of the diagonal ensemble in operator space however obeys a volume law scaling, which limits the system size reachable in classical simulations.


\subsection{Definition of PEVP} To define the PEVP, we first need give a precise definition of a prethermal plateau. 
There is not a single accepted definition for a prethermal plateau in the literature. 
Here we formulate the practical definition we use. First we define what we consider to be a plateau.

\begin{definition} [The plateau]
    \label{def:plateau}
    Given a tolerance $\epsilon \ll 1$, a plateau is a time interval $[t_1, t_2)$ with $t_1 < t_2 \le \infty$ such that
    \begin{enumerate}
        \item $\max\limits_{t_1 \leq t < t_2} \braket{A}_t - \min\limits_{t_1 \leq t < t_2} \braket{A}_t \leq \epsilon \Vert A\Vert$,
        where $\langle A \rangle_t$ is defined in Eq.~(\ref{eq:AFloq}).
        \item there exists no overlapping interval $[t_1^{\prime}, t_2^{\prime})$ also satisfying 1 for which $t_2^{\prime} / t_1^{\prime} > t_2 / t_1$.
    \end{enumerate}
\end{definition}
The second criterion ensures the plateau we find is locally the longest. Here we take the ratio $t_2/t_1$ as the measure of the length of the plateau to be more consistent with the ideas of prethermalization.  A plateau can be identified as a prethermal plateau, if 
\begin{itemize}
\item it is not connected to the final Floquet thermalization plateau at infinite time and temperature~\cite{Mori2018},
\item the ratio $t_2 / t_1$ grows exponentially with $1 / \tau$ and
\item in the small $\tau$ limit, $t_1$ converges to a positive number.
\end{itemize}

It is in general hard to identify a prethermal plateau, due to the difficulty of reaching the exponentially growing $t_2$ in simulations. Nevertheless, assuming its existence, it is relatively easy to find the plateau and compute the plateau value. Now let us restate Problem~\ref{prob:PEVP} in the main text:
\begin{definition}[The prethermalized expectation value problem]
    Given a unitary $U_{\mathrm{Trotter}}(\tau)$, a state $\ket{\psi}$, and a local observable $A$, assume that a prethermal plateau exists between times $t_1$ to $t_2$, such that $\max_{t \in [t_1, t_2)} \langle A \rangle_t - \min_{t \in [t_1, t_2)} \langle A \rangle_t \leq \epsilon \Vert A\Vert$. Find the value of $\braket{A}_t$ to within additive error $2 \epsilon \Vert A\Vert$ for any $t \in [t_1, t_2)$ .
\end{definition}


\section{The Magnus expansion}
\label{sec:magnus}

The Magnus expansion serves as a series expansion for the effective Hamiltonian of a Floquet driving $H(t)$ with period $\tau$:
\begin{eqnarray}
	\begin{aligned}
		{U}_{F}(\tau) = & \mathcal{T}\left( e^{-i \int_{0}^{\tau}  {H}(t) \dInt t} \right) = e^{-i \tau \sum_{k=1}^{\infty} \Omega_k}                       \\
		\Omega_0 =     & \frac{1}{\tau} \int_{0}^{\tau} \dInt t_1 {H}(t_1)                                                                                                 \\
		\Omega_1 =                   & \frac{1}{2i \tau} \int_{0}^{\tau} \dInt t_1 \int_{0}^{t_1} \dInt t_2 \left[{H}(t_1), {H}(t_2)\right]                                 \\
		\Omega_2 =                   & - \frac{1}{6\tau} \int_{0}^{\tau} \dInt t_1 \int_{0}^{t_1} \dInt t_2 \int_{0}^{t_2} \dInt t_3 \\
        ([{H} & (t_1), [{H}(t_2), {H}(t_3)]]  + [{H}(t_3), [{H}(t_2), {H}(t_1)]])                                                                                        \\
		                                   & \vdots
	\end{aligned}
\end{eqnarray}

In general, the Magnus expansion is not convergent~\cite{Blanes2009, Bukov2015} and thus higher order contributions are not negligible for finite driving frequencies. Nevertheless, its finite truncation is still expected to approximate the quasi-stationary prethermal plateau~\cite{kuwahara2016}. To be more precise, let ${H}_{\mathrm{Magnus}}^{(n)} = \sum_{j = 0}^{n} \Omega_j$ denote the $n$-th order truncated effective Hamiltonian, then there exists $n_0 = \varO(\omega / kJ)$ such that 
\begin{eqnarray}
    \Vert U_F(\tau)^m - e^{-i{H}_{\mathrm{Magnus}}^{(n_0)} m\tau} \Vert \lesssim N m \tau 2^{-n_0}.
    \label{eq:floq_magnus1}
\end{eqnarray}

The general estimation Eq.~(\ref{eq:floq_magnus1}) for the unitary evolution operators has a linear dependence on system size, which does not imply prethermalization for $N \gtrsim \exp(\varO(\omega / kJ))$. When considering local observables acting on a subsystem $L$ and short-range interacting Hamiltonians, however, the bound can be tightened for the reduced density matrix $\rho_L$:
\begin{eqnarray}
    \begin{aligned}
        \Vert (\rho_L)_F(m\tau) -  (\rho_L)_{\mathrm{Magnus}}^{(n_0)}(m\tau) \Vert_1
        \lesssim |L|m\tau e^{- \varO(\omega)}
    \end{aligned}
\end{eqnarray}
for the same $n_0$, where the system size dependence is erased~\cite{kuwahara2016}. 

For the proof of this relation to hold rigorously, the required driving frequency is $\omega \ge 16\pi kJ \approx 100 J$ for nearest neighbour interacting Hamiltonians, while in our numerical simulation in Fig.~\ref{fig:prethermal}, prethermalization has occurred for $\omega \sim 8 J $. For all of our numerical simulations of the XY-model, we use the Trotterization shown in Fig.~\ref{fig:magnus}a. In Fig.~\ref{fig:magnus}b-c the differences between Floquet evolution and its Magnus expansions up to the third order are plotted. Note that the zeroth order Magnus expansion is just the original non-Floquet Hamiltonian. For $\omega = 8J$, it turns out that the $n = 1$ case already gives a good approximation of the Floquet Hamiltonian.

\begin{figure}
    \centering
    \xincludegraphics[width=0.38\textwidth,trim={0.6cm 0.5cm 0.6cm 1cm},clip, label=(a)]{figs/06a_trotter.pdf}
    
    \xincludegraphics[width=0.235\textwidth, label=(b)]{figs/06b_Floquet_TimeEvol_Lx4_Ly3_Jz0.00_p0_X+_alphaPi4}
    \xincludegraphics[width=0.235\textwidth, label=(c)]{figs/06c_Floquet_TimeEvol_Lx4_Ly3_Jz0.00_p0_X+_alphaPi8}
    \caption{\textbf{(a):} Trotterization of 2D XY Hamiltonian. The circles represent qubits and the rectangles on the bounds represent two-qubit evolution gates on neighbouring qubits. 
    \textbf{(b-c):} Simulation of evolving Floquet XY model with the Magnus expansion truncated to $n$-th order. Their differences from Floquet evolution are plotted. The initial state is $\ket{X+}$ and ${A} = m_x^2 + m_y^2$. \textbf{(b):} $\omega = 4J$, \textbf{(c):} $\omega = 8J$. Note the difference in scale on the y-axis. The system size is $N = 4 \times 3$.}
    \label{fig:magnus}
\end{figure}


\section{Difficulty of error mitigation in time evolution}
\label{sec:diff_TE}
The difficulty of error mitigation of observables by measuring them directly can be explained in the following two ways. 

First, if we take the formalism as in Eq.~(\ref{eq:dm_noisy}), the aim will be to obtain $A_{\psi}(t) = \braket{\psi_t | A | \psi_t}$ from
\begin{eqnarray}
    A_{\psi}^{\mathcal{N}_p}(t) = \tr \left( A \rho_{\psi}^{\mathcal{N}_p} \right) = qA_{\psi}(t) + (1-q) \tr\left( 
A \tilde{\rho} \right).
\label{eq:direct_te}
\end{eqnarray}
Although the second term vanishes for global depolarizing channel and traceless $A$, one can not use the same trick as Eq.~(\ref{eq:rescale}) to directly estimate $q$, since setting $A = \id$ would not give any meaningful output. Of course, it is in principle still possible to measure the survival probability with backward evolution that approximates $q^2$ and take its square root. In the latter circuit, however, any coherent noise will partially cancel in forward and backward evolutions, which gives a different value of $q$ from the one we need in Eq.~(\ref{eq:direct_te}).

Alternatively, we can think about the problem using a random walk picture, where an initial state will be quickly heated during time evolution on noisy digial simulators, because of the strong energy dependence of the density of states (DOS). Let us consider the quantum trajectory simulation process of a noisy circuit. Assume the absolute average energy change per error to be a constant $g > 0$ and denote the expectation value of the energy of the simulated state after $n$ errors by $E_n$. The probability of increasing or decreasing energy after each gate of noise will be
\begin{eqnarray}
  \frac{\mathbb{P}(E_{n+1} = E_n + g)}{\mathbb{P}(E_{n+1} = E_n -g)} = \frac{\mathrm{DOS}(E_n+g)}{\mathrm{DOS}(E_n-g)}.
\end{eqnarray}
For short-range interacting and locally bounded Hamiltonians, the DOS converges weakly to a Gaussian in the thermodynamic limit~\cite{Hartmann2005}:
\begin{eqnarray}
    \mathrm{DOS}(E) \propto \exp\left( - E^2 / 2N\sigma^2 \right),
\end{eqnarray}
where $\sigma$ is a constant depending on local energy scale. Inserting ${\mathbb{P}(E_{n+1} = E_n + g)} + {\mathbb{P}(E_{n+1} = E_n -g)} = 1$, it can be concluded that
\begin{eqnarray}
    \begin{aligned}
      \Delta E =  & E_{n+1} - E_n\\
      = & g \left[\mathbb{P}(E_{n+1} = E_n + g) - \mathbb{P}(E_{n+1} = E_n -g) \right]                  \\
      = & -g \tanh \left(\frac{gE_n}{N\sigma^2}\right)
    \end{aligned}
\end{eqnarray}
The circuit depth $D$ required for a single noise to occur is $\Delta D = 1 / p N$, where $p$ is the noise rate. Therefore
\begin{eqnarray}
    \frac{\Delta E}{\Delta D} = - p g N \tanh \frac{gE}{N\sigma^2},
\end{eqnarray}
whose solution in the continuous limit is
\begin{eqnarray}
    \sinh \left( \frac{g}{N\sigma^2}E \right) = \sinh \left( \frac{g}{N\sigma^2}E_0 \right) e^{-pg^2D / \sigma^2}.
\end{eqnarray}
It gives rise to an exponential decay in energy with regard to the circuit depth. In other words, the initial state will be heated to infinite temperature, and this process is much faster than the heating caused by Floquet driving in the prethermal regime. Post-selection error mitigation strategies for direct time evolution would then imply that it is possible to extract low temperature properties from higher temperatures. There is no reason to assume that this would be the case, especially in the case when phase transitions exist.


\begin{figure}[b]
    \centering
    (a) Phase damping noise\\
    \xincludegraphics[width=0.25\textwidth, label=(a1)]{figs/07a1_TELoschmidt_alphaPi8.0_angle8_phaDamp0.003_scaling.pdf}
    \xincludegraphics[width=0.2\textwidth, label=(a2)]{figs/07a2_TELoschmidt_alphaPi8.0_angle8_phaDamp0.003_XX_scaling.pdf}
    \xincludegraphics[width=0.235\textwidth, label=(a3)]{figs/07a3_TELoschmidt_alphaPi8.0_angle8_phaDamp0.003_XX_scaling_s_runs2000_cutoff0.010.pdf}
    \xincludegraphics[width=0.235\textwidth, label=(a4)]{figs/07a4_TELoschmidt_alphaPi8.0_angle8_cutoff0.010_new_Phase_damping.pdf}
    (b) Amplitude damping noise\\
    \xincludegraphics[width=0.25\textwidth, label=(b1)]{figs/07b1_TELoschmidt_alphaPi8.0_angle8_ampDamp0.003_scaling.pdf}
    \xincludegraphics[width=0.2\textwidth, label=(b2)]{figs/07b2_TELoschmidt_alphaPi8.0_angle8_ampDamp0.003_XX_scaling.pdf}
    \xincludegraphics[width=0.235\textwidth, label=(b3)]{figs/07b3_TELoschmidt_alphaPi8.0_angle8_ampDamp0.003_XX_scaling_s_runs2000_cutoff0.010.pdf}
    \xincludegraphics[width=0.235\textwidth, label=(b4)]{figs/07b4_TELoschmidt_alphaPi8.0_angle8_cutoff0.010_new_Amplitude_damping.pdf}
    \caption{Simulation results for phase damping (a1-a4) and amplitude damping (b1-b4) noise and $p=0.3\%$. The plotted quantities are the same as shown in Fig.~\ref{fig:scaling} and Fig.~\ref{fig:error_scaling}.}
    \label{fig:supp_plot}
\end{figure}

\section{Phase and amplitude damping noises}
\label{sec:supp_plot}

In the main text, we focused on depolarizing noise. In this appendix, we show that the effects of phase damping and amplitude damping noise are qualitatively similar. The relevant noise channels are given by~\cite{Nielsen2012}:
\begin{itemize}
    \item the phase damping channel
    \begin{eqnarray}
        \mathcal{N}_{p}^{P}(\rho) = (1 - p) \rho + p \sigma_i^z \rho \sigma_i^z,
    \end{eqnarray}
    \item and the amplitude damping channel
    \begin{eqnarray}
        \mathcal{N}_{p}^{A}(\rho) = M_0 \rho M_0^{\dagger} + M_1 \rho M_1^{\dagger},
    \end{eqnarray}
    where $M_0 = \begin{pmatrix}
        1 & 0 \\ 0 & \sqrt{1 - p}
    \end{pmatrix}$ and $M_1 = \begin{pmatrix}
        0 & \sqrt{p} \\ 0 & 0
    \end{pmatrix}$.
\end{itemize}

In Fig.~\ref{fig:supp_plot} we plot the simulation results for these two types of noises in the same fashion as in Fig.~\ref{fig:scaling} and Fig.~\ref{fig:error_scaling}. From top to bottom, they are the scaling of survival probability without (left) and with (right) applying the observable, the error $s$ of the mitigation strategy and the moving quadratic average of $s$. The scalings are also fit well with Eq.(\ref{eq:scaling}), while the error after rescaling is much smaller for phase damping error than for the other two. Note that for amplitude damping noise, the effective survival probability is $q^2 = (1 - p / 2)^{ND}$. This is likely due to the balanced distribution of our initial states in the $z$ direction, which reduces the probability of seeing a single state jumping to $p/2$.

In Fig.~\ref{fig:convergence}, we show the convergence of the Monte Carlo simulations of $L_{\id}^{\mathcal{N}_p}(t)$ . We observe that phase and amplitude damping noises require a much smaller number of trajectories than depolarizing noise to reach the same estimation error. For circuit depth $D=80$ and $n=2000$ trajectories, which are the parameters used in Fig.~\ref{fig:full_exp}, the error can be read off from Fig.~\ref{fig:convergence}a) to be about $15\%$.

\begin{figure}
    \centering
    \xincludegraphics[width=0.25\textwidth, label=(a)]{figs/08a_TELoschmidt_alphaPi8.0_angle8_dep0.003_runs.pdf}
    \\
    \xincludegraphics[width=0.235\textwidth, label=(b)]{figs/08b_TELoschmidt_alphaPi8.0_angle8_phaDamp0.003_runs.pdf}
    \xincludegraphics[width=0.235\textwidth, label=(c)]{figs/08c_TELoschmidt_alphaPi8.0_angle8_ampDamp0.003_runs.pdf}
    \caption{The relative estimation error of $L_{\id}^{\mathcal{N}_p}(t)$ as a function of the number $n$ of Monte Carlo samples for circuit depth $D=40$ and $80$. The estimation error is defined as the standard deviation of the ensemble of expectation values from the trajectories divided by $\sqrt{n}$. The figures show the relative estimation error, i.e.~the ratio of the estimation error to the estimated value (mean) of $L_{\id}^{\mathcal{N}_p}(t)$. The system size is $4\times 4$ and the initial state is $\ket{X+}$.}
    \label{fig:convergence}
\end{figure}

\section{Phase and amplitude damping noises}
\label{sec:supp_plot}


\section{Proof of Eq.~(\ref{eq:scaling_math})\label{app:proof}}
The trace of the product of two matrices $\tr \left( A^{\dagger} B \right)$ can be viewed as an inner product, and thus the Cauchy-Schwarz inequality applies:
\begin{eqnarray}
    \left|\tr \left( A^{\dagger} B \right) \right| \le \sqrt{\tr \left( A^{\dagger} A\right) \cdot \tr \left( B^{\dagger} B \right)}.
\end{eqnarray}
Since ${A}$ is hermitian and  unitary, ${A}^2 = \id$ and the first perturbation term in Eq.~(\ref{eq:expansion_sp}) can be bounded by
\begin{eqnarray}
    \begin{aligned}
        \left|\tr \left[ (\tilde{\rho} {A})^2 \right]\right|          
        \le  \sqrt{\tr \left[({A} \tilde{\rho} {A})^2\right] \cdot \tr \left[\tilde{\rho} ^2\right]}
        =   r^2.
    \end{aligned}    
\end{eqnarray}
Similarly, for the other term,
\begin{eqnarray}
    \begin{aligned}
        & |\braket{\psi_t | {A} \tilde{\rho} {A} | \psi_t}|  \\
        = & \left|\tr \left( {A} \tilde{\rho} {A} \ket{\psi_t}\bra{\psi_t} \right) \right| \le \sqrt{\tr \left[({A} \tilde{\rho} {A})^2\right]} = r.
    \end{aligned}
\end{eqnarray}
Combining these inequalities, we get Eq.~(\ref{eq:scaling_math}).