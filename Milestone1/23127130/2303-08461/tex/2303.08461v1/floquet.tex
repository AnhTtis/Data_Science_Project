\section{The prethermalized expectation value problem}
\label{sec:pre}


\subsection{Time evolution on digital quantum computers\label{sec:equilibration}}
The time evolution under a Hamiltonian $H$ can be reproduced on a digital quantum computer using the Suzuki--Trotter decomposition. In its simplest, first-order form, the decomposition approximates the time-evolution unitary $U(\tau) = e^{- i H \tau}$ by
\begin{eqnarray}\label{def:Utrot}
	U_{\mathrm{Trotter}}(\tau) = \prod_{j=1}^{\Gamma} e^{-i H_j\tau}.
\end{eqnarray}
where $H = \sum_{j=1}^{\Gamma} H_j$. Each $H_j$ is a sum of mutually commuting local terms, such that $e^{-i H_j \tau}$ can be efficiently implemented using local gates. The smaller the Trotter step $\tau$, the more accurate the Trotter decomposition. For the $p$-th order Trotter decomposition \cite{Hatano2005}, which generalizes the previous simple formula, the error of $U_\mathrm{Trotter}(\tau)$ with respect to the desired unitary $U(\tau)$ is bounded from above by $\varO(N\tau^{p+1})$, where $N$ is the system size~\cite{childs2021}. The dependence on $N$ can be eliminated if all quantities of interest are local observables. According to the Lieb--Robinson bound, only a light cone with a radius proportional to the total evolution time $T$ is relevant~\cite{Lieb1972}. Therefore, the system size $N$ can be replaced with the size of the light cone $\sim T^d$ before it reaches the edges of the system, where $d$ is the spatial dimension. We hence require that the Trotter step $\tau$ be less than $\varO(\max\left\{T^{-(d+1)/p}, (NT)^{-1/p}\right\})$ for the Trotterized time evolution of local observables to converge to the continuous evolution under $H$.

We can now define the following computational problem.
\begin{problem}[The Trotter time-average problem]
    \label{def:time_average}
    Given a unitary $U_{\mathrm{Trotter}}(\tau)$, a state $\ket{\psi}$, a local observable $A$ and a time $t=m \tau$ for positive integer $m$, and a small positive constant $\epsilon$, 
    compute the time-averaged observable
\begin{eqnarray} \label{eq:AFloq}
    \braket{A}_t = \frac{1}{m + 1} \sum_{n = 0}^{m}  \braket{ \psi |  U^{\dagger}_{\mathrm{Trotter}}(\tau)^{n} A
    U_{\mathrm{Trotter}}(\tau)^{n} | \psi}
\end{eqnarray}
within additive error $\epsilon \Vert A\Vert$, where $\Vert \cdot \Vert$ is the operator norm.
\end{problem}
Note that the Trotterization is not uniquely defined by the Hamiltonian and $U_\mathrm{Trotter}$ must be specified explicitly. The cost of solving this problem on a classical computer generically scales exponentially with either the number of Trotter steps $m$ or the system size $N$~\footnote{For example, a state vector simulation scales linearly in the number of Trotter steps but exponentially with the system size. While a tensor network simulation scales polynomially in system size but exponentially with the number of Trotter steps.}, whereas on a fault-tolerant quantum computer, the effort increases at most polynomially with both. The hardness of the problem is further supported by the fact that it becomes BQP-complete at times $t = \mathrm{poly}(n)$ if the Trotter error is negligible~\cite{janzing2005}. In section~\ref{sec:mitigation}, we present evidence that the problem is solvable on noisy quantum computers up to a maximum number of Trotter steps, which is independent of system size. We then show in section~\ref{sec:implementation} that noisy quantum devices may reach a classically intractable regime with realistic noise parameters, even when taken into account the overhead of our error mitigation strategy.

\subsection{Prethermalization}

Problem~\ref{def:time_average} is not only interesting from the perspective of dynamics but it can also yield insight into equilibrium properties. In condensed matter or statistical physics, one would typically describe a system in equilibrium in terms of its temperature, or in case of the microcanonical ensemble, its internal energy. Under ETH, %Problem~\ref{def:time_average} gives a way to probe 
the microcanonical ensemble at the mean energy of the state $\ket{\psi}$ can be approximated by solving Problem~\ref{def:time_average}.

More precisely, in the limit of continuous time evolution, the long-time average of an observable is described by the diagonal ensemble. For a given initial state $\ket{\psi}$ and an observable $A$,
\begin{eqnarray}
    \begin{aligned}
        \lim_{T\to\infty}\frac{1}{T} \int_0^{T}  \braket{\psi (t)| A | \psi (t)} \dInt t
        = \sum_{k} |\braket{k | \psi}|^2 \braket{k | A | k},
    \end{aligned}
\end{eqnarray}
where $H = \sum_{k} E_k \ket{k}\bra{k}$ is the spectral decomposition of a non-degenerate Hamiltonian~\footnote{In the case of degenerate Hamiltonian spectrum, one can still diagonalize the observable projected onto each subspace of Hamiltonian eigenvalue to define the diagonal ensemble as long time average}. Assuming ETH, the expectation value $\braket{k | A | k}$ is a smooth function of the energy $E_k$ up to a small, state-dependent correction~\cite{Srednicki1999}. The diagonal ensemble is then equivalent to the microcanonical ensemble at energy $\braket{\psi | H | \psi}$ provided the energy variance of $\ket{\psi}$ is sufficiently small. For observables that are an average of an extensive number of local terms, e.g., the total magnetization per site, we expect the microcanonical ensemble to vary significantly only on an extensive energy scale. It is thus possible to estimate expectation values in the microcanonical ensemble from the diagonal ensemble of states whose width in energy is subextensive. Product states satisfy this condition as their widths in energy are (under weak assumptions) proportional to $\sqrt{N}$~\cite{Hartmann2004}.

The above discussion shows that it is possible to probe the microcanonical ensemble by solving problem~\ref{def:time_average} with product initial states at different mean energies. This is, however, challenging with current quantum devices for two reasons. First, the maximum number of Trotter steps $T/\tau$ is limited by the maximum circuit depth in the presence of noise, while the total time $T$ required to reach equilibrium may be large. Therefore, noisy quantum devices are usually unable to reach long enough times with bounded Trotter error.  Secondly, the finite calibration precision renders it challenging to get high relative precision in the angle of rotation for gates that are very close to the identity, bounding from below the size of $\tau$. 

We will now argue that it is nevertheless possible to study equilibrium phenomena. Using larger, experimentally feasible Trotter steps can be viewed as applying a periodic Floquet drive. The system can be described by the Floquet Hamiltonian $H_F$, which is implicitly defined by
\begin{eqnarray}
	U_{\mathrm{Trotter}}(\tau) = e^{-i H_F \tau}.
\end{eqnarray}
The Floquet Hamiltonian is not unique as its eigenvalues are only defined modulo $\omega=2 \pi / \tau$, the effective driving frequency. For large $\tau$, (small $\omega$), i.e., outside the Trotter limit, the Floquet Hamiltonian is highly non-local and will cause a generic initial state to heat up to infinite temperature~\cite{lazarides2014,dalessio2014}. Despite this, it is possible to observe (approximate) equilibration if the heating time scale is much greater than the equilibration time scale. This is known as Floquet prethermalization~\cite{kuwahara2016, Fleckenstein2020, Morningstar2021}. Fortunately for our purposes, Floquet prethermalization is relatively easy to access because Floquet heating occurs on a time scale $t_F \propto e^{\varO(\omega / kJ)}$, where $k$ is the interaction range and $J$ is the local energy scale, assuming $\omega \gtrsim kJ $. We highlight the favorable exponential dependence of $t_F$ on $\omega / k J$ and the fact that $k J$ is independent of the system size.

For times much less than $t_F$, the system evolves approximately according to an effective Hamiltonian which is close to, but not the same as, the original Hamiltonian $H$. More precisely, the effective Hamiltonian is local and it is given by the $n_0$-th order Magnus expansion~\cite{Magnus1954, Blanes2009} of the Floquet Hamiltonian, where $n_0 = \varO(\omega / kJ)$ (see Appendix~\ref{sec:magnus} for details). Observables start to equilibrate under the effective Hamiltonian before eventually heating up. If the equilibration time $t_0$ is much shorter than $t_F$, then there exists a prethermal plateau $t_0 \le t \ll t_F$, during which the expectation value of the observable is approximately constant. We provide a formal definition of a plateau in Appendix \ref{sec:def}. 

The above observations motivate the definition of the PEVP: 
\begin{problem}[Prethermalized expectation value problem]
    \label{prob:PEVP}
    Given a unitary $U_{\mathrm{Trotter}}(\tau)$, a state $\ket{\psi}$, and a local observable $A$, assume that a prethermal plateau exists between times $t_1$ to $t_2$, such that $\max_{t \in [t_1, t_2)} \langle A \rangle_t - \min_{t \in [t_1, t_2)} \langle A \rangle_t \leq \epsilon \Vert A\Vert$ for some positive constant $\epsilon$. Find the value of $\braket{A}_t$ to within additive error $2 \epsilon \Vert A\Vert$ for any $t \in [t_1, t_2)$ .
\end{problem}
This problem reduces to solving Problem~\ref{def:time_average} at time $t = t_1$. In the following sections, we show using the example of the two-dimensional XY model that the prethermal plateau is indeed accessible and that the properties of the effective Hamiltonian closely resemble those of the initial Hamiltonian. We further demonstrate that the PEVP can be solved on a noisy quantum device with realistic parameters up to system sizes for which classical simulation of the dynamics is intractable.


\subsection{PEVP with the XY model\label{sec:xy}}
We focus on the two-dimensional quantum XY model on a square lattice for the remainder of this work. We emphasize, however, that the approach can be readily applied to many other models. The Hamiltonian of the XY model is given by
\begin{eqnarray}
	H_{\mathrm{XY}} = - J \sum_{\braket{ij}} \left( S^x_i S^x_j + S^y_i S^y_j \right),
\end{eqnarray} 
where $J$ is the interaction strenth, $S_i^\alpha$ ($\alpha \in \{ x, y, z\}$) are spin-1/2 operators on site $i$, and the sum runs over all pairs of nearest neighbors. The model is convenient for digital quantum computers as its two-site interaction generates a partial iSWAP gate,
\begin{eqnarray}
    e^{-i J\left( S^x_i S^x_j + S^y_i S^y_j \right) \tau} = \text{iSWAP}^{ - J \tau / \pi}_{ij}.
\end{eqnarray}
A single Trotter step in a first-order decomposition consists of applying a partial iSWAP gate to each nearest-neighbor pair of qubits. As non-overlapping gates can be performed in parallel, these operations can be carried out in a circuit whose depth is equal to the number of nearest neighbors (4 in the case of the square lattice).

The XY model in two dimensions can be solved with quantum Monte Carlo algorithms~\cite{Loh1985, Ding1992} and thus serves as a good benchmark to our method. It is known to undergo the Kosterlitz--Thouless (KT) transition~\cite{Kosterlitz1973, Ding1992} at nonzero temperature. This phase transition can be characterized by the mean-squared in-plane magnetization per site,
\begin{eqnarray}
	m_{x}^2 + m_{y}^2 = 4\cdot \frac{ \left( \sum_i S^x_i\right)^2 + \left( \sum_i S^y_i\right)^2 }{N^2},
\end{eqnarray}
which is an approximation to the in-plane susceptibility~\cite{Ding1992}. The mean-squared magnetization can be written as the sum of two-site correlators, which decay exponentially with the distance between the two sites at high temperature. Hence, $m_x^2 + m_y^2$ decreases with the system size as $1/N$ in the thermodynamic limit. Below the critical temperature, the system exhibits quasi long-range order. The mean-squared magnetization decays only as $1/N^{1/8}$ and its value remains non-negligible for moderately large systems ~\cite{Ding1992}.

\begin{figure}
    \centering
    \xincludegraphics[width=0.45\textwidth, label=(a)]{figs/02a_longTimeEvol_XY_Lx4_Ly4_hx0.0_theta8.0.pdf}
    \xincludegraphics[width=0.48\textwidth, label=(b)]    {figs/02b_Lx4_Ly4_Jz0.00_p0_r20_alphaPi8_delta0.5.pdf}
    \caption{\textbf{(a)}~Prethermal plateau of the 2D XY model for system size $N = 4 \times 4$. The initial state is $\ket{X+}$. The colored lines show the time averages of the mean-squared in-plane magnetization for different Trotter step sizes $\tau$, corresponding to different driving frequencies $\omega = 2 \pi / \tau$. The large circles stand for the starting and end points of the plateaus according to Definition~\ref{def:plateau} with tolerance $\epsilon = 0.05$ and a maximum value of $t_2 J$ of $10^3$. The black dashed line represents the value in the diagonal ensemble of the initial Hamiltonian. \textbf{(b)}~Comparison of the value at the prethermal plateau with the values in the microcanonical and diagonal ensemble values of the initial XY Hamiltonian and in the diagonal ensemble of the first-order Magnus expansion. The system size is $4 \times 4$. The driving frequency is as $\omega = 8J$ and the plateau value is taken from the time average at $t = 20 / J$, which is on the prethermal plateau for all computed initial states with tolerance $\epsilon = 0.05$. For the microcanonical ensemble, we average over an energy window of width $\delta = 0.5J$ in the $m_z = 0$ subspace (see Appendix~\ref{sec:def}).}
    \label{fig:prethermal}
\end{figure}

In analogy to the long-time average that gives rise to the diagonal ensemble, we probe the prethermal plateaus using the Floquet time average as in Definition~\ref{def:time_average}, where the Trotterization is shown in the appendix in Fig.~\ref{fig:magnus}a. We explore this quantity using exact diagonalization on a square lattice with $N = 4 \times 4$ spins and open boundary conditions. Figure~\ref{fig:prethermal}a shows the values of the mean-squared in-plane magnetization for the initial state $\ket{\psi} = \ket{X+} = \left[ \frac{1}{\sqrt{2}}\left( \ket{0} + \ket{1} \right)\right]^{\otimes N}$. The different colors indicate the Trotter step size $\tau$ or, equivalently, the driving frequency $\omega = 2\pi / \tau$. The initial state is close to the ground state of the XY Hamiltonian. We therefore expect the in-plane magnetization to remain high in the prethermal plateau, provided the effective Hamiltonian does not differ too much from the XY model.

We indeed observe prethermal plateaus for large driving frequencies ($\omega \ge 8J$), and these last for $t > 10^3 / J$ when $\omega \ge 9J$. The plateau values approach the diagonal ensemble value (black dashed line) with increasing driving frequencies. They deviate only slightly due to the correction in the Magnus expansion, which will be discussed later in this subsection. This confirms that the dynamics with fast Floquet drive are similar to the dynamics of the original Hamiltonian in this prethermal regime. By contrast, no plateaus are observed at low driving frequencies, where the time average of the mean-squared magnetization quickly drops to expected value at infinite temperature, $2 / N$. 

We may perform the same analysis for different initial states. We choose product states in which the spins on the two sublattices of the square lattice are in the respective states $\ket{\theta, 0}$ and $\ket{\pi - \theta, \phi}$, where $\ket{\theta, \phi} = \cos(\theta / 2) \ket{0} + \sin(\theta /2 ) e^{i\phi} \ket{1}$ parametrizes an arbitrary state of a qubit (spin-1/2). This choice of states allows us to cover a wide range of the spectrum while ensuring that the total magnetization in the $z$ direction vanishes. The latter constraint is convenient because the Hamiltonian conserves the total $z$-magnetization, $m_z = \sum_{i = 1} ^{N} \sigma^z_i / N$. Thermalization therefore occurs in the eigenspaces of $m_z$. Low-energy product states however are not eigenstates of $m_z$. By choosing the expectation value of $m_z$ to be zero, we maximize the overlap of the product state with the sectors of low $z$-magnetization, for which we expect similar equilibration dynamics.


We find that all product states of the above form exhibit prethermal plateaus at similar driving frequencies and evolution times. We evaluate the prethermal values of the in-plane magnetization by performing the Floquet time average up to time $t = 20/J$ with driving frequency $\omega = 8 J$. The result is shown for various initial states as a function of their mean energy in Fig.~\ref{fig:prethermal}b. For comparison, we also show the diagonal and microcanonical ensemble values of the initial XY model, as well as the diagonal ensemble one of the first-order Magnus expansion of Floquet Hamiltonian, given by
\begin{eqnarray}
	\begin{aligned}
		H_{\mathrm{Magnus}}^{(1)} =  & \frac{1}{\tau} \int_{0}^{\tau} \dInt t_1 H(t_1)            \\
		& + \frac{1}{2i \tau} \int_{0}^{\tau} \dInt t_1 \int_{0}^{t_1} \dInt t_2 \left[H(t_1), H(t_2)\right].
	\end{aligned}
\end{eqnarray}
Here, $H(t)$ is the piecewise constant Hamiltonian corresponding to the different terms of the Trotter expansion Eq.~(\ref{def:Utrot}): 
\begin{eqnarray}
    H(t) = \Gamma H_j \text{ for } (j-1)\tau / \Gamma \le t < j \tau / \Gamma,
\end{eqnarray}
where $1 \le j \le \Gamma$. Definitions of the different ensembles and higher orders of the Magnus expansion can be found in App.~\ref{sec:def} and App.~\ref{sec:magnus}, respectively.

The values at the prethermal plateau are close to those of the diagonal ensemble $H_\mathrm{Magnus}^{(1)}$, indicating that the first-order truncation already serves as a good approximation for Floquet Hamiltonian in the prethermal regime. In Appendix~\ref{sec:magnus}, we show that the higher orders lead to no significant improvement for $\omega = 8J$. The thermal equilibrium values of the initial XY Hamiltonian, in both the diagonal and the microcanonical ensemble, deviate slightly from the Floquet values. Nevertheless, the comparison indicates that the prethermal properties of the Floquet system can reveal nontrivial thermal properties of the XY Hamitlonian.
