\documentclass[aps,prx,twocolumn,superscriptaddress,floatfix,citeautoscript,preprintnumbers]{revtex4-2}
\usepackage{amsfonts, amssymb, amsmath, amsthm}
\usepackage{array}
\usepackage{bm}
\usepackage{braket}
\usepackage{dcolumn}
\usepackage{graphicx}
\usepackage{hyperref}
\hypersetup{
  colorlinks,%
  citecolor=blue,%
  filecolor=blue,%
  linkcolor=blue,%
  urlcolor=blue
}
\usepackage[normalem]{ulem}
\usepackage{dsfont}
\usepackage{indentfirst}
\usepackage{listings}
\usepackage{newtxtext,newtxmath}
\usepackage{subfigure}
\usepackage{float}
\usepackage{titlesec}
\usepackage{color}
\usepackage{xcolor}
\usepackage{enumitem}
\usepackage[title]{appendix}

\newtheorem{definition}{Definition}
\newtheorem{problem}{Problem}
\newtheorem{assumption}{Assumption}
\newtheorem{prop}[assumption]{Proposition}


\newcommand{\dInt}{{\mathrm{d}}}
\newcommand{\id}{{\mathds{1}}}
\newcommand{\tr}{{\mathrm{Tr}~}}
\newcommand{\floor}[1]{\lfloor #1 \rfloor}
\newcommand{\ceil}[1]{\lceil #1 \rceil}
\newcommand{\varO}{\mathcal{O}}  %O


% For subfigure label
\usepackage{xparse,xcoffins}
\ExplSyntaxOn
\NewCoffin\imagecoffin
\NewCoffin\labelcoffin
\keys_define:nn { figs/label }
 {
  label   .tl_set:N = \l_label_tl,
  labelbox .bool_set:N = \l_label_box_bool,
  labelbox .default:n = true,
  fontsize .tl_set:N = \l_label_size_tl,
  fontsize .initial:n = \footnotesize,
  pos .choice:,
  pos/nw .code:n = \tl_set:Nn \l_label_pos_tl { left,up },
  pos/ne .code:n = \tl_set:Nn \l_label_pos_tl { right,up },
  pos/sw .code:n = \tl_set:Nn \l_label_pos_tl { left,down },
  pos/se .code:n = \tl_set:Nn \l_label_pos_tl { right,down },
  pos/n .code:n = \tl_set:Nn \l_label_pos_tl { hc,up },
  pos/w .code:n = \tl_set:Nn \l_label_pos_tl { left,vc },
  pos/s .code:n = \tl_set:Nn \l_label_pos_tl { hc,down },
  pos/e .code:n = \tl_set:Nn \l_label_pos_tl { right,vc },
  pos .initial:n = nw,
  unknown .code:n   = \clist_put_right:Nx \l_label_clist
                       { \l_keys_key_tl = \exp_not:n { #1 } }
 }
\clist_new:N \l_label_clist
\box_new:N \l_label_box
\box_new:N \l_label_image_box

\NewDocumentCommand{\xincludegraphics}{O{}m}
 {
  \group_begin:
  \tl_clear:N \l_label_tl
  \clist_clear:N \l_label_clist
  \keys_set:nn { figs/label } { #1 }
  \tl_if_empty:NTF \l_label_tl
   {
    \includegraphics:Vn \l_label_clist { #2 }
   }
   {
    \SetHorizontalCoffin\imagecoffin
     {
      \includegraphics:Vn \l_label_clist { #2 }
     }
    \SetHorizontalCoffin\labelcoffin
     {
      \raisebox{\depth}
       {
        \bool_if:NTF \l_label_box_bool
         { \fcolorbox{white}{white}{\l_label_size_tl\l_label_tl} }
         { \l_label_size_tl\l_label_tl }
       }
     }
    \SetVerticalPole\imagecoffin{left}{3pt+\CoffinWidth\labelcoffin/2}
    \SetVerticalPole\imagecoffin{right}{\Width-3pt-\CoffinWidth\labelcoffin/2}
    \SetHorizontalPole\imagecoffin{up}{\Height-3pt-\CoffinHeight\labelcoffin/2}
    \SetHorizontalPole\imagecoffin{down}{3pt+\CoffinHeight\labelcoffin/2}
    \use:x{\JoinCoffins\imagecoffin[\l_label_pos_tl]\labelcoffin[vc,hc]} 
    \TypesetCoffin\imagecoffin
   }
   \group_end:
 }
\NewDocumentCommand{\setlabel}{m}
 {
  \keys_set:nn { figs/label } { #1 }
 }
\cs_new_protected:Nn \includegraphics:nn
 {
  \includegraphics[#1]{#2}
 }
\cs_generate_variant:Nn \includegraphics:nn { V }
\ExplSyntaxOff
% For subfigure label



\begin{document}

\title{Simulating prethermalization using near-term quantum computers}

\begin{abstract}
    Quantum simulation is one of the most promising scientific applications of quantum computers. Due to decoherence and noise in current devices, it is however challenging to perform digital quantum simulation in a regime that is intractable with classical computers. In this work, we propose an experimental protocol for probing dynamics and equilibrium properties on near-term digital quantum computers. As a key ingredient of our work, we show that it is possible to study thermalization even with a relatively coarse Trotter decomposition of the Hamiltonian evolution of interest. Even though the step size is too large to permit a rigorous bound on the Trotter error, we observe that the system prethermalizes in accordance with previous results for Floquet systems. The dynamics closely resemble the thermalization of the model underlying the Trotterization up to long times. We extend the reach of our approach by developing an error mitigation scheme based on measurement and rescaling of survival probabilities. To demonstrate the effectiveness of the entire protocol, we apply it to the two-dimensional XY model and numerically verify its performance with realistic noise parameters for superconducting quantum devices. Our proposal thus provides a route to achieving quantum advantage for relevant problems in condensed matter physics.
\end{abstract}

\author{Yilun Yang}
\author{Arthur Christianen}
\author{Sandra Coll-Vinent}
\affiliation{Max-Planck-Institut f\"{u}r Quantenoptik, 85748 Garching, Germany}
\affiliation{Munich Center for Quantum Science and Technology, 80799 M\"{u}nchen, Germany}

\author{Vadim Smelyanskiy}
\affiliation{Google Quantum AI, Venice, California 90291, USA}

\author{Mari Carmen Bañuls}
\affiliation{Max-Planck-Institut f\"{u}r Quantenoptik, 85748 Garching, Germany}
\affiliation{Munich Center for Quantum Science and Technology, 80799 M\"{u}nchen, Germany}

\author{Thomas E.~O'Brien}
\affiliation{Google Quantum AI, 80636 M\"{u}nchen, Germany}

\author{Dominik S.~Wild}
\author{J.~Ignacio Cirac}
\affiliation{Max-Planck-Institut f\"{u}r Quantenoptik, 85748 Garching, Germany}
\affiliation{Munich Center for Quantum Science and Technology, 80799 M\"{u}nchen, Germany}


\date{\today}							
\maketitle



\newpage
\section{Introduction}

The increasing complexity of source code poses a key challenge to the reliability of large-scale software systems. Software bugs in these systems can lead to safety issues~\cite{bug_safety} for users around the world as well as cause non-negligible financial losses~\cite{bug_loss}. As such, developers have to spend a large amount of time and effort on bug fixing. Consequently, \aprfull (\apr), designed to automatically generate patches to fix software bugs, has attracted wide attention from both academia and industry~\cite{long2016prophet, legoues2012genprog, long2015spr, lou2020can, tufano2018empstudy}. 


To achieve \apr, one popular approach is known as Generate-and-Validate (G\&V)~\cite{qi2015gv, ghanbari2019prapr, lou2020can, le2016hdrepair, legoues2012genprog, wen2018capgen, hua2018sketchfix, martinez2016astor, koyuncu2020fixminder, liu2019tbar, liu2019avatar}, which is typically based on the following pipeline: First, fault localization techniques~\cite{wong2016fl, abreu2007ochiai, zhang2013injecting, papadakis2015metallaxis, li2019deepfl, li2017transforming} are applied to determine the suspicious locations in programs where bugs are likely to exist. Then, the buggy locations are used by the \apr tools to generate a list of patches that replace buggy lines with correct lines. Afterward, each patch is validated against the original test suite to identify any \emph{plausible patches} (i.e., passing all tests in the test suite). Finally, to determine the \emph{correct patches}, developers examine the list of plausible patches to see if any of them can correctly fix the bug. 

Traditional \apr tools can mainly be categorized into heuristic-based~\cite{legoues2012genprog, le2016hdrepair, wen2018capgen}, constraint-based~\cite{mechtaev2016angelix, le2017s3, demacro2014nopol, long2015spr} and \template~\cite{ghanbari2019prapr, hua2018sketchfix, martinez2016astor, liu2019tbar, liu2019avatar}. Among these traditional tools, \template \apr tools~\cite{ghanbari2019prapr, liu2019tbar, benton2020effectiveness} have been able to achieve state-of-the-art results. \Template \apr tools typically leverage pre-defined templates (e.g., adding a nullness check) for bug fixing. However, since these fix templates are typically handcrafted, the number and types of bugs they are able to fix can be limited. 



To address the limitations of traditional \apr, researchers have proposed various \learning \apr tools~\cite{li2020dlfix, chen2018sequencer, jiang2021cure, lutellier2020coconut, zhu2021recoder, ye2022rewardrepair} based on the \nmtfull (\nmt) architecture~\cite{sutskever2014mt} where the input is the buggy code snippets and the goal is to translate the buggy code snippets into a fixed version. To accomplish this, \learning \apr tools require supervised training datasets with pairs of both buggy and fixed code snippets in order to learn how to perform this translation step. These training data are usually obtained by mining historical bug fixes using heuristics/keywords~\cite{dallmeier2007benchmark}, which can be imprecise for identifying bug-fixing commits; even the actual bug-fixing commits can include irrelevant code changes, leading to further pollution in the dataset~\cite{xia2022alpharepair}.
% 
Moreover, it can be hard for such \apr tools to generalize and fix bug types unseen during training. 



To better leverage recent advances in \plmfull{s} (\plm{s}), researchers~\cite{xia2022alpharepair, xia2023repairstudy, kolak2022patch, prenner2021codexws} have directly applied \plm{s} to generate patches without bug-fixing datasets. These \llm-based \apr tools work by either directly generating a complete code function~\cite{prenner2021codexws, xia2023repairstudy} or predict/infill the correct code snippet given its surrounding context~\cite{xia2022alpharepair, xia2023repairstudy}. By directly using \llm{s} that are pre-trained on billions of open-source code snippets, \llm-based \apr tools can achieve state-of-the-art performance on many repair datasets~\cite{xia2022alpharepair}. 


% 
%
%

Traditional \apr tools have long used the insight of the \emph{plastic surgery hypothesis}~\cite{barr2014plastic} where it states that the code ingredients to fix a bug already exist within the same project. Traditional \apr tools have manually designed pattern-~\cite{ghanbari2019prapr, saha2017elixir} or heuristic-based~\cite{jiang2018simfix, legoues2012genprog} approaches to finding and using such relevant code ingredients to generate fixes for bugs. However, the plastic surgery hypothesis has been largely ignored in \llm-based \apr. In fact, \llm provides a unique opportunity to fully automate the plastic surgery hypothesis idea via fine-tuning (learning project-specific information via model updates from the buggy project) and prompting (directly providing relevant code ingredients to the model), and make it directly applicable to different languages (since the \llm{s} are typically multi-lingual).%
Moreover, despite the intensive manual efforts involved, traditional \apr tools still cannot fully leverage project-specific information due to large search space for leveraging/composing existing code ingredients. In contrast, the project-specific information can effectively leveraged by \llm{s} due to their power in code understanding/vectorization, e.g., even partial/imprecise information may still guide \llm{s} in correct patch generation!
 To this end, we ask the question: \emph{How useful is the plastic surgery hypothesis in the era of \plm{s}}?








\mypara{Our Work.} To answer the question, we present \ourtech{\xspace} -- a \llm-based approach that automatically utilizes the plastic surgery hypothesis by systematically combining multiple fine-tuning and prompting strategies for \apr. \ourtech fine-tunes \plm{s} using two novel domain-specific training strategies: \textbf{\epfinetune} -- we fine-tune using the original buggy project by aggressively masking out a high percentage of tokens, which allows \plm to learn project-specific code tokens and programming styles; and \textbf{\rofinetune} -- which only masks out a single continuous code sequence per training sample, allowing the model to get used to the final \csapr task of predicting a single continuous code sequence. Furthermore, we directly leverage the ability for \plm{s} to understand natural language instructions and introduce a novel prompting strategy, \textbf{\idprompting}, which uses information retrieval and static analysis to obtain a list of relevant identifiers for the buggy lines. While such relevant identifiers are critical for fixing some difficult bugs, they may not be seen by the \llm during inference due to limited context window size. Through the use of prompting, we directly tell the model to use these extracted identifiers (relevant code ingredients) to generate the correct code. Finally, to perform repair, we combine all four model variants (including the base model, both fine-tuned models and the base model with prompting) for the final repair.





While our insight of leveraging the plastic surgery hypothesis for \llm-based \apr is generalizable across different types of \plm{s}, to implement \ourtech, we choose a recent \plm{\xspace}, \ctfive~\cite{wang2021codet5}, which is pre-trained on millions of open-source code snippets. \ctfive is an encoder-decoder model trained using \mspfull (\msp) objective where a percentage of tokens are masked out and each continuous masked token sequence is referred to as a masked span. Also, although we only extract relevant identifiers from the current buggy project (since this paper focuses on the plastic surgery hypothesis), our work can be easily extended to obtain other code information (such as relevant statements or functions) from other sources, such as  the massive pre-training corpora~\cite{husain2020codesearchnet} or historical bug-fixing datasets~\cite{jiang2019infer}, which can provide more coding knowledge for \llm{s}. Besides, although we mainly focus on using traditional string comparison algorithms for information retrieval in this paper, these techniques can be easily replaced by other frequency-based retrieval~\cite{robertson2009probabilistic} and neural search (or embedding-based search)~\cite{reimers2019sentence}.
  In summary, this paper makes the following contributions:


%


\begin{itemize}[noitemsep, leftmargin=*, topsep=0pt]
    \item \textbf{Dimension.} This paper is the first to revisit the important plastic surgery hypothesis in the era of \llm{s}. It opens up a new dimension for \llm-based \apr to incorporate previously neglected information from the buggy project itself to boost \apr performance. Furthermore, it demonstrates the promising future of retrieval-based prompting for modern \llm-based \apr.
    \item \textbf{Implementation.} We implement \ourtech based on the recent \ctfive model. We augment the model using two novel fine-tuning strategies: \epfinetune and \rofinetune, along with a novel prompting strategy based on information retrieval and static analysis: \idprompting. We combine the patches generated by all four models together and perform patch ranking to speed up \apr.% 
    \item \textbf{Evaluation Study.} We conduct an extensive evaluation against state-of-the-art \apr tools. On the widely studied \dfj 1.2 and 2.0 datasets~\cite{just2014dfj}, \ourtech is able to achieve the new state-of-the-art results of 89 and 44 correct bug fixes (15 and 8 more than best baseline) respectively.  Furthermore, we perform a broad ablation study to justify our design. \ourtech demonstrates for the first time that the plastic surgery hypothesis can substantially boost \llm-based \apr and advance state-of-the-art \apr, while being fully automated and general. Moreover, even partial/imprecise code ingredients may still effectively guide \llm{s} for \apr!
\end{itemize}


\section{The prethermalized expectation value problem}
\label{sec:pre}


\subsection{Time evolution on digital quantum computers\label{sec:equilibration}}
The time evolution under a Hamiltonian $H$ can be reproduced on a digital quantum computer using the Suzuki--Trotter decomposition. In its simplest, first-order form, the decomposition approximates the time-evolution unitary $U(\tau) = e^{- i H \tau}$ by
\begin{eqnarray}\label{def:Utrot}
	U_{\mathrm{Trotter}}(\tau) = \prod_{j=1}^{\Gamma} e^{-i H_j\tau}.
\end{eqnarray}
where $H = \sum_{j=1}^{\Gamma} H_j$. Each $H_j$ is a sum of mutually commuting local terms, such that $e^{-i H_j \tau}$ can be efficiently implemented using local gates. The smaller the Trotter step $\tau$, the more accurate the Trotter decomposition. For the $p$-th order Trotter decomposition \cite{Hatano2005}, which generalizes the previous simple formula, the error of $U_\mathrm{Trotter}(\tau)$ with respect to the desired unitary $U(\tau)$ is bounded from above by $\varO(N\tau^{p+1})$, where $N$ is the system size~\cite{childs2021}. The dependence on $N$ can be eliminated if all quantities of interest are local observables. According to the Lieb--Robinson bound, only a light cone with a radius proportional to the total evolution time $T$ is relevant~\cite{Lieb1972}. Therefore, the system size $N$ can be replaced with the size of the light cone $\sim T^d$ before it reaches the edges of the system, where $d$ is the spatial dimension. We hence require that the Trotter step $\tau$ be less than $\varO(\max\left\{T^{-(d+1)/p}, (NT)^{-1/p}\right\})$ for the Trotterized time evolution of local observables to converge to the continuous evolution under $H$.

We can now define the following computational problem.
\begin{problem}[The Trotter time-average problem]
    \label{def:time_average}
    Given a unitary $U_{\mathrm{Trotter}}(\tau)$, a state $\ket{\psi}$, a local observable $A$ and a time $t=m \tau$ for positive integer $m$, and a small positive constant $\epsilon$, 
    compute the time-averaged observable
\begin{eqnarray} \label{eq:AFloq}
    \braket{A}_t = \frac{1}{m + 1} \sum_{n = 0}^{m}  \braket{ \psi |  U^{\dagger}_{\mathrm{Trotter}}(\tau)^{n} A
    U_{\mathrm{Trotter}}(\tau)^{n} | \psi}
\end{eqnarray}
within additive error $\epsilon \Vert A\Vert$, where $\Vert \cdot \Vert$ is the operator norm.
\end{problem}
Note that the Trotterization is not uniquely defined by the Hamiltonian and $U_\mathrm{Trotter}$ must be specified explicitly. The cost of solving this problem on a classical computer generically scales exponentially with either the number of Trotter steps $m$ or the system size $N$~\footnote{For example, a state vector simulation scales linearly in the number of Trotter steps but exponentially with the system size. While a tensor network simulation scales polynomially in system size but exponentially with the number of Trotter steps.}, whereas on a fault-tolerant quantum computer, the effort increases at most polynomially with both. The hardness of the problem is further supported by the fact that it becomes BQP-complete at times $t = \mathrm{poly}(n)$ if the Trotter error is negligible~\cite{janzing2005}. In section~\ref{sec:mitigation}, we present evidence that the problem is solvable on noisy quantum computers up to a maximum number of Trotter steps, which is independent of system size. We then show in section~\ref{sec:implementation} that noisy quantum devices may reach a classically intractable regime with realistic noise parameters, even when taken into account the overhead of our error mitigation strategy.

\subsection{Prethermalization}

Problem~\ref{def:time_average} is not only interesting from the perspective of dynamics but it can also yield insight into equilibrium properties. In condensed matter or statistical physics, one would typically describe a system in equilibrium in terms of its temperature, or in case of the microcanonical ensemble, its internal energy. Under ETH, %Problem~\ref{def:time_average} gives a way to probe 
the microcanonical ensemble at the mean energy of the state $\ket{\psi}$ can be approximated by solving Problem~\ref{def:time_average}.

More precisely, in the limit of continuous time evolution, the long-time average of an observable is described by the diagonal ensemble. For a given initial state $\ket{\psi}$ and an observable $A$,
\begin{eqnarray}
    \begin{aligned}
        \lim_{T\to\infty}\frac{1}{T} \int_0^{T}  \braket{\psi (t)| A | \psi (t)} \dInt t
        = \sum_{k} |\braket{k | \psi}|^2 \braket{k | A | k},
    \end{aligned}
\end{eqnarray}
where $H = \sum_{k} E_k \ket{k}\bra{k}$ is the spectral decomposition of a non-degenerate Hamiltonian~\footnote{In the case of degenerate Hamiltonian spectrum, one can still diagonalize the observable projected onto each subspace of Hamiltonian eigenvalue to define the diagonal ensemble as long time average}. Assuming ETH, the expectation value $\braket{k | A | k}$ is a smooth function of the energy $E_k$ up to a small, state-dependent correction~\cite{Srednicki1999}. The diagonal ensemble is then equivalent to the microcanonical ensemble at energy $\braket{\psi | H | \psi}$ provided the energy variance of $\ket{\psi}$ is sufficiently small. For observables that are an average of an extensive number of local terms, e.g., the total magnetization per site, we expect the microcanonical ensemble to vary significantly only on an extensive energy scale. It is thus possible to estimate expectation values in the microcanonical ensemble from the diagonal ensemble of states whose width in energy is subextensive. Product states satisfy this condition as their widths in energy are (under weak assumptions) proportional to $\sqrt{N}$~\cite{Hartmann2004}.

The above discussion shows that it is possible to probe the microcanonical ensemble by solving problem~\ref{def:time_average} with product initial states at different mean energies. This is, however, challenging with current quantum devices for two reasons. First, the maximum number of Trotter steps $T/\tau$ is limited by the maximum circuit depth in the presence of noise, while the total time $T$ required to reach equilibrium may be large. Therefore, noisy quantum devices are usually unable to reach long enough times with bounded Trotter error.  Secondly, the finite calibration precision renders it challenging to get high relative precision in the angle of rotation for gates that are very close to the identity, bounding from below the size of $\tau$. 

We will now argue that it is nevertheless possible to study equilibrium phenomena. Using larger, experimentally feasible Trotter steps can be viewed as applying a periodic Floquet drive. The system can be described by the Floquet Hamiltonian $H_F$, which is implicitly defined by
\begin{eqnarray}
	U_{\mathrm{Trotter}}(\tau) = e^{-i H_F \tau}.
\end{eqnarray}
The Floquet Hamiltonian is not unique as its eigenvalues are only defined modulo $\omega=2 \pi / \tau$, the effective driving frequency. For large $\tau$, (small $\omega$), i.e., outside the Trotter limit, the Floquet Hamiltonian is highly non-local and will cause a generic initial state to heat up to infinite temperature~\cite{lazarides2014,dalessio2014}. Despite this, it is possible to observe (approximate) equilibration if the heating time scale is much greater than the equilibration time scale. This is known as Floquet prethermalization~\cite{kuwahara2016, Fleckenstein2020, Morningstar2021}. Fortunately for our purposes, Floquet prethermalization is relatively easy to access because Floquet heating occurs on a time scale $t_F \propto e^{\varO(\omega / kJ)}$, where $k$ is the interaction range and $J$ is the local energy scale, assuming $\omega \gtrsim kJ $. We highlight the favorable exponential dependence of $t_F$ on $\omega / k J$ and the fact that $k J$ is independent of the system size.

For times much less than $t_F$, the system evolves approximately according to an effective Hamiltonian which is close to, but not the same as, the original Hamiltonian $H$. More precisely, the effective Hamiltonian is local and it is given by the $n_0$-th order Magnus expansion~\cite{Magnus1954, Blanes2009} of the Floquet Hamiltonian, where $n_0 = \varO(\omega / kJ)$ (see Appendix~\ref{sec:magnus} for details). Observables start to equilibrate under the effective Hamiltonian before eventually heating up. If the equilibration time $t_0$ is much shorter than $t_F$, then there exists a prethermal plateau $t_0 \le t \ll t_F$, during which the expectation value of the observable is approximately constant. We provide a formal definition of a plateau in Appendix \ref{sec:def}. 

The above observations motivate the definition of the PEVP: 
\begin{problem}[Prethermalized expectation value problem]
    \label{prob:PEVP}
    Given a unitary $U_{\mathrm{Trotter}}(\tau)$, a state $\ket{\psi}$, and a local observable $A$, assume that a prethermal plateau exists between times $t_1$ to $t_2$, such that $\max_{t \in [t_1, t_2)} \langle A \rangle_t - \min_{t \in [t_1, t_2)} \langle A \rangle_t \leq \epsilon \Vert A\Vert$ for some positive constant $\epsilon$. Find the value of $\braket{A}_t$ to within additive error $2 \epsilon \Vert A\Vert$ for any $t \in [t_1, t_2)$ .
\end{problem}
This problem reduces to solving Problem~\ref{def:time_average} at time $t = t_1$. In the following sections, we show using the example of the two-dimensional XY model that the prethermal plateau is indeed accessible and that the properties of the effective Hamiltonian closely resemble those of the initial Hamiltonian. We further demonstrate that the PEVP can be solved on a noisy quantum device with realistic parameters up to system sizes for which classical simulation of the dynamics is intractable.


\subsection{PEVP with the XY model\label{sec:xy}}
We focus on the two-dimensional quantum XY model on a square lattice for the remainder of this work. We emphasize, however, that the approach can be readily applied to many other models. The Hamiltonian of the XY model is given by
\begin{eqnarray}
	H_{\mathrm{XY}} = - J \sum_{\braket{ij}} \left( S^x_i S^x_j + S^y_i S^y_j \right),
\end{eqnarray} 
where $J$ is the interaction strenth, $S_i^\alpha$ ($\alpha \in \{ x, y, z\}$) are spin-1/2 operators on site $i$, and the sum runs over all pairs of nearest neighbors. The model is convenient for digital quantum computers as its two-site interaction generates a partial iSWAP gate,
\begin{eqnarray}
    e^{-i J\left( S^x_i S^x_j + S^y_i S^y_j \right) \tau} = \text{iSWAP}^{ - J \tau / \pi}_{ij}.
\end{eqnarray}
A single Trotter step in a first-order decomposition consists of applying a partial iSWAP gate to each nearest-neighbor pair of qubits. As non-overlapping gates can be performed in parallel, these operations can be carried out in a circuit whose depth is equal to the number of nearest neighbors (4 in the case of the square lattice).

The XY model in two dimensions can be solved with quantum Monte Carlo algorithms~\cite{Loh1985, Ding1992} and thus serves as a good benchmark to our method. It is known to undergo the Kosterlitz--Thouless (KT) transition~\cite{Kosterlitz1973, Ding1992} at nonzero temperature. This phase transition can be characterized by the mean-squared in-plane magnetization per site,
\begin{eqnarray}
	m_{x}^2 + m_{y}^2 = 4\cdot \frac{ \left( \sum_i S^x_i\right)^2 + \left( \sum_i S^y_i\right)^2 }{N^2},
\end{eqnarray}
which is an approximation to the in-plane susceptibility~\cite{Ding1992}. The mean-squared magnetization can be written as the sum of two-site correlators, which decay exponentially with the distance between the two sites at high temperature. Hence, $m_x^2 + m_y^2$ decreases with the system size as $1/N$ in the thermodynamic limit. Below the critical temperature, the system exhibits quasi long-range order. The mean-squared magnetization decays only as $1/N^{1/8}$ and its value remains non-negligible for moderately large systems ~\cite{Ding1992}.

\begin{figure}
    \centering
    \xincludegraphics[width=0.45\textwidth, label=(a)]{figs/02a_longTimeEvol_XY_Lx4_Ly4_hx0.0_theta8.0.pdf}
    \xincludegraphics[width=0.48\textwidth, label=(b)]    {figs/02b_Lx4_Ly4_Jz0.00_p0_r20_alphaPi8_delta0.5.pdf}
    \caption{\textbf{(a)}~Prethermal plateau of the 2D XY model for system size $N = 4 \times 4$. The initial state is $\ket{X+}$. The colored lines show the time averages of the mean-squared in-plane magnetization for different Trotter step sizes $\tau$, corresponding to different driving frequencies $\omega = 2 \pi / \tau$. The large circles stand for the starting and end points of the plateaus according to Definition~\ref{def:plateau} with tolerance $\epsilon = 0.05$ and a maximum value of $t_2 J$ of $10^3$. The black dashed line represents the value in the diagonal ensemble of the initial Hamiltonian. \textbf{(b)}~Comparison of the value at the prethermal plateau with the values in the microcanonical and diagonal ensemble values of the initial XY Hamiltonian and in the diagonal ensemble of the first-order Magnus expansion. The system size is $4 \times 4$. The driving frequency is as $\omega = 8J$ and the plateau value is taken from the time average at $t = 20 / J$, which is on the prethermal plateau for all computed initial states with tolerance $\epsilon = 0.05$. For the microcanonical ensemble, we average over an energy window of width $\delta = 0.5J$ in the $m_z = 0$ subspace (see Appendix~\ref{sec:def}).}
    \label{fig:prethermal}
\end{figure}

In analogy to the long-time average that gives rise to the diagonal ensemble, we probe the prethermal plateaus using the Floquet time average as in Definition~\ref{def:time_average}, where the Trotterization is shown in the appendix in Fig.~\ref{fig:magnus}a. We explore this quantity using exact diagonalization on a square lattice with $N = 4 \times 4$ spins and open boundary conditions. Figure~\ref{fig:prethermal}a shows the values of the mean-squared in-plane magnetization for the initial state $\ket{\psi} = \ket{X+} = \left[ \frac{1}{\sqrt{2}}\left( \ket{0} + \ket{1} \right)\right]^{\otimes N}$. The different colors indicate the Trotter step size $\tau$ or, equivalently, the driving frequency $\omega = 2\pi / \tau$. The initial state is close to the ground state of the XY Hamiltonian. We therefore expect the in-plane magnetization to remain high in the prethermal plateau, provided the effective Hamiltonian does not differ too much from the XY model.

We indeed observe prethermal plateaus for large driving frequencies ($\omega \ge 8J$), and these last for $t > 10^3 / J$ when $\omega \ge 9J$. The plateau values approach the diagonal ensemble value (black dashed line) with increasing driving frequencies. They deviate only slightly due to the correction in the Magnus expansion, which will be discussed later in this subsection. This confirms that the dynamics with fast Floquet drive are similar to the dynamics of the original Hamiltonian in this prethermal regime. By contrast, no plateaus are observed at low driving frequencies, where the time average of the mean-squared magnetization quickly drops to expected value at infinite temperature, $2 / N$. 

We may perform the same analysis for different initial states. We choose product states in which the spins on the two sublattices of the square lattice are in the respective states $\ket{\theta, 0}$ and $\ket{\pi - \theta, \phi}$, where $\ket{\theta, \phi} = \cos(\theta / 2) \ket{0} + \sin(\theta /2 ) e^{i\phi} \ket{1}$ parametrizes an arbitrary state of a qubit (spin-1/2). This choice of states allows us to cover a wide range of the spectrum while ensuring that the total magnetization in the $z$ direction vanishes. The latter constraint is convenient because the Hamiltonian conserves the total $z$-magnetization, $m_z = \sum_{i = 1} ^{N} \sigma^z_i / N$. Thermalization therefore occurs in the eigenspaces of $m_z$. Low-energy product states however are not eigenstates of $m_z$. By choosing the expectation value of $m_z$ to be zero, we maximize the overlap of the product state with the sectors of low $z$-magnetization, for which we expect similar equilibration dynamics.


We find that all product states of the above form exhibit prethermal plateaus at similar driving frequencies and evolution times. We evaluate the prethermal values of the in-plane magnetization by performing the Floquet time average up to time $t = 20/J$ with driving frequency $\omega = 8 J$. The result is shown for various initial states as a function of their mean energy in Fig.~\ref{fig:prethermal}b. For comparison, we also show the diagonal and microcanonical ensemble values of the initial XY model, as well as the diagonal ensemble one of the first-order Magnus expansion of Floquet Hamiltonian, given by
\begin{eqnarray}
	\begin{aligned}
		H_{\mathrm{Magnus}}^{(1)} =  & \frac{1}{\tau} \int_{0}^{\tau} \dInt t_1 H(t_1)            \\
		& + \frac{1}{2i \tau} \int_{0}^{\tau} \dInt t_1 \int_{0}^{t_1} \dInt t_2 \left[H(t_1), H(t_2)\right].
	\end{aligned}
\end{eqnarray}
Here, $H(t)$ is the piecewise constant Hamiltonian corresponding to the different terms of the Trotter expansion Eq.~(\ref{def:Utrot}): 
\begin{eqnarray}
    H(t) = \Gamma H_j \text{ for } (j-1)\tau / \Gamma \le t < j \tau / \Gamma,
\end{eqnarray}
where $1 \le j \le \Gamma$. Definitions of the different ensembles and higher orders of the Magnus expansion can be found in App.~\ref{sec:def} and App.~\ref{sec:magnus}, respectively.

The values at the prethermal plateau are close to those of the diagonal ensemble $H_\mathrm{Magnus}^{(1)}$, indicating that the first-order truncation already serves as a good approximation for Floquet Hamiltonian in the prethermal regime. In Appendix~\ref{sec:magnus}, we show that the higher orders lead to no significant improvement for $\omega = 8J$. The thermal equilibrium values of the initial XY Hamiltonian, in both the diagonal and the microcanonical ensemble, deviate slightly from the Floquet values. Nevertheless, the comparison indicates that the prethermal properties of the Floquet system can reveal nontrivial thermal properties of the XY Hamitlonian.

\section{Error mitigation}
\label{sec:mitigation}


\subsection{Rescaling of survival probabilities}

Without mitigation, noise will frustrate any naive attempts to observe prethermal plateaus on current quantum hardware. As we show in Appendix~\ref{sec:diff_TE}, noise provides an additional heating source to the Floquet driving already discussed; one that we expect to be far stronger with today's error rates, and one without favourable scaling in the system size. It is therefore desirable to develop an error mitigation technique to estimate the result of a noiseless quantum circuit from multiple measurements in a noisy circuit~\cite{Temme2017, Endo2018a, Cai2022}. However, we do not see a reliable method for extracting the desired noiseless results from measurements of the noisy state as this would imply the ability of inferring low-temperature results from high-temperature ones.

To circumvent this issue, we avoid direct tomography of the time-evolved observables on the noisy state. Instead, we convert observable estimation into a survival probability circuit, in a manner similar to that used in out-of-time-order correlators (OTOC)~\cite{mi2021} or echo verification circuits~\cite{obrien2021,huo2022}. Following forward evolution, we \emph{apply} the observable and then evolve backwards in time, followed by a projection onto the initial state (see Fig.~\ref{fig:scaling}a). This yields a survival probability of the form
\begin{eqnarray}
    L_{A, \psi}(t) = \left| \braket{\psi | e^{iHt} A e^{-i H t} | \psi} \right|^2 = \braket{\psi | A(t) | \psi}^2.
\end{eqnarray}
In the following, we drop the label $\psi$ for notational simplicity. For this procedure to work, $A$ must be a (local) unitary. For spin systems, it is possible to write any observable as a sum of products of unitary Pauli operators and to measure each Pauli operator separately. Although $L_{A}(t)$ only gives the expectation value of an observable up to a sign, one can infer the sign by tracking it from the known initial value, assuming $\braket{\psi | A(t) | \psi }$ is a smooth function~\cite{Lu2020}.
This simplifies previous Loschmidt-echo style methods for learning $\braket{\psi | A(t) |\psi}$, which required ancilla qubits, the preparation of large Greenberger-Horne-Zeilinger (GHZ) states~\cite{obrien2021} or intermediate re-preparation and measurement of qubits~\cite{huo2022}.

As we will now demonstrate, a simple rescaling is remarkably effective at mitigating errors in the estimation of the survival probability. The strategy is based on the observation that the survival probability is approximately proportional to the probability of no error occurring. The reason is that the state becomes highly entangled during the evolution, at which point a single-qubit error results in an orthogonal state with high probability. To be more concrete, consider a single Pauli error $\sigma^{\mu}_i$ occurring at time $t^{\prime} < t$ at site $i$ and set the observable $A$ to be identity. The survival probability is then given by $[\tr (\rho_i(t^{\prime}) \sigma^{\mu}_i)]^2$, where $\rho_i(t^{\prime})$ is the reduced density matrix of $\ket{\psi(t^{\prime})}$ at site $i$. If this site is entangled with the other parts of the system, the reduced density matrix will be close to the identity (completely mixed) and the survival probability will be close to zero. 

The above discussion suggests that the survival probability with noise is related to the noiseless value, times the probability that no error has occurred. For concreteness, we consider error models in which a single-qubit noise channel $\mathcal{N}_p$ is applied to each qubit after every layer of unitary gates. Here, $p$ is the probability that the channel causes an error on the qubit. The state of art gate error rate is around $0.5\%$ for two-qubit gates~\cite{Mi2022b, Wei2022a}, motivating our choice of $p = 0.3\%$ per qubit per gate as the reference value in our model~\footnote{In experiments, XY rotations are sometimes compiled into more than one two-qubit gate. The value of $p$ should then be increased accordingly.}.

Denoting the survival probability in the presence of noise by $L_A^{\mathcal{N}_p}(t)$, we then expect that
\begin{eqnarray}
    L_{A}^{\mathcal{N}_p}(t) / L_{A}(t) \approx (1 - p)^{ND},
    \label{eq:scaling}
\end{eqnarray}
where $N$ is the number of qubits and $D$ is the circuit depth including both forward and backward evolutions. Crucially, no independent knowledge of the noise channel is required to estimate $L_A(t)$. By setting $A = \id$, we obtain $L_\id^{\mathcal{N}_p}(t) \approx (1 - p)^{ND}$ since the noiseless survival probability satisfies $L_\id(t) = 1$. Hence,
\begin{eqnarray}
    L_A(t) \approx L_A^{\mathcal{N}_p}(t) / L_\id^{\mathcal{N}_p}(t),
    \label{eq:rescale}
\end{eqnarray}
where the right-hand side can be obtained from measurements on the noisy quantum device.

We can make this argument more rigorous for channels that can be represented in terms of unitary Kraus operators. For such channels, the probability that a particular error occurs is independent of the state. This class of channels includes depolarizing and dephasing noise as well as all other Pauli channels~\footnote{Even though amplitude damping error is not included in this class of channels, we find that the conclusions of this section nevertheless hold to a good approximation. See Appendix~\ref{sec:supp_plot} for numerical results.}. The survival probability after the noisy circuit can be expressed as
\begin{eqnarray}
    L_{A}^{\mathcal{N}_p}(t) = \tr \left[ \left( A \rho^{\mathcal{N}_p}_{\psi}(t) \right)^2\right],
    \label{eq:noisy_sp}
\end{eqnarray}
where $\rho^{\mathcal{N}_p}_{\psi}(t)$ is the mixed state after the noisy forward evolution~\footnote{To obtain this equation, the circuit in Fig.~\ref{fig:scaling}a has to be slightly modified: during backward evolution, the error gates occur before each evolution unitary gate instead of after it.}. We write the state $\rho^{\mathcal{N}_p}_{\psi}(t)$ as
\begin{eqnarray}
    \rho^{\mathcal{N}_p}_{\psi}(t) = q \ket{\psi_t}\bra{\psi_t} + (1-q)\tilde{\rho},
    \label{eq:dm_noisy}
\end{eqnarray}
where $\ket{\psi_t} = U_{\mathrm{Trotter}}^{t/\tau}(\tau) \ket{\psi}$ is the state after noiseless forward evolution and $q = (1 - p)^{ND / 2}$ is the probability that no error occurred during the forward evolution. The density matrix $\tilde{\rho}$ is the state conditioned on at least one error having occurred. The survival probability in noisy simulation then becomes
\begin{eqnarray}
    \begin{aligned}
        L_{A}^{\mathcal{N}_p}(t)
        = & q^2 |\braket{\psi_t | A | \psi_t }|^2 + (1-q)^2\tr \left[ (\tilde{\rho} A)^2 \right]\\
        &  + 2q(1-q) \braket{\psi_t | A \tilde{\rho} A | \psi_t}.
    \end{aligned}
    \label{eq:expansion_sp}
\end{eqnarray}
Defining $r = \sqrt{ \tr \left[\tilde{\rho}^2 \right] }$, we can use Cauchy-Schwarz inequality to obtain (see Appendix~\ref{app:proof})
\begin{eqnarray}
    \left | \frac{L_{A}^{\mathcal{N}_p}(t)}{q^2} - L_{A}(t) \right| \le (1-q)^2\left(\frac{r}{q}\right)^2 + 2(1-q)\frac{r}{q}.
    \label{eq:scaling_math}
\end{eqnarray}
Since $0 < q, r \le 1$, $L_{A}^{\mathcal{N}_p}(t) / q^2$ serves as a good approximation of $L_{A}(t)$ when $q \gg r$. This condition can be satisfied over a broad range of parameters because $r$ typically decays with the system size. In the most extreme case of global depolarizing noise, $\tilde{\rho}$ is a completely mixed state, for which $r^2 = 2^{-N}$. The condition $q \gg r$ then gives rise to
\begin{eqnarray}
    (1-p)^{ND} > \frac{C}{2^N} \Rightarrow  ND < \frac{N\log{2} + \log (1 / C)}{\log[1 / (1-p)]}
    \label{eq:limit_depth}
\end{eqnarray}
for some constant $C$. For $p=0.3\%$, this evaluates to $D < 230$ in the thermodynamic limit. For more general types of noise, we similarly expect the scaling with $q^2$ to hold up to some constant circuit depth in the thermodynamic limit. The noisy survival probability at this constant circuit depth will, however, decay exponentially when increasing the system size such that exponentially many measurements are required to resolve the signal. Nevertheless, we will show below that the number of measurements remains experimentally feasible in superconducting quantum devices for moderately sized systems with realistic error rates.

Two situations where Eq.~(\ref{eq:rescale}) fails directly follow from our argument. One is the case when $q$ approaches $r$, as already discussed. The other is when the initial state does not thermalize. For example, the product state $\ket{Z+} = \ket{0}^{\otimes N}$ is invariant under the (Floquet) XY Hamiltonian and thus will not get entangled. However, even in this case Eq.~(\ref{eq:rescale}) works well for many practical channels because two independent errors are unlikely to cancel each other.

    
\subsection{Numerical results}
\begin{figure}
    \centering
    \xincludegraphics[width=0.499\textwidth,trim={1.2cm 0.2cm 1.cm 0.cm},clip,label=(a)]{figs/03a_circuit.pdf}
    \\\ \\
    \xincludegraphics[width=0.25\textwidth, label=(b)]{figs/03b_TELoschmidt_alphaPi8.0_angle8_dep0.003_scaling.pdf}
    \xincludegraphics[width=0.2\textwidth, label=(c)]{figs/03c_TELoschmidt_alphaPi8.0_angle8_dep0.003_XX_scaling.pdf}
    \caption{\textbf{(a)} Quantum circuit to map the expectation value of a (unitary) observable onto a survival probability. The initial state is prepared with $V$, $U= U_1, U_2, U_3$ or $ U_4$ is a single step in the Trotter decomposition, and $\mathcal{N}$ denotes a local noise channel.
    \textbf{(b)}, \textbf{(c)} Dependence of $L_{\id}^{\mathcal{N}_p}(t)$ and $L_{A}^{\mathcal{N}_p}(t)$ on the circuit depth $D$ and system size $N$ in the presence of depolarizing noise with error probability $p = 0.3\%$. The initial state is $\ket{\psi} = \ket{X+}$. The observable $A = 4S^x_{i}S_{i+1}^x$ is a correlator in the center of the lattice. The black dashed lines represent the scaling predicted by Eq.~(\ref{eq:scaling}).
    }
    \label{fig:scaling}
\end{figure}


We now numerically verify these considerations for the Floquet evolution of the XY model described in Sec.~\ref{sec:xy} in the presence of local depolarizing noise. For each qubit, the noise channel is given by
\begin{eqnarray}
    \mathcal{N}_{p}(\rho) = (1 - p)\rho + \sum_{\mu = 1}^{3} \frac{p}{3} \sigma^{\mu} \rho \sigma^{\mu}.
\end{eqnarray}
Other types of noise are discussed in the Appendix~\ref{sec:supp_plot}. In Fig.~\ref{fig:scaling}b and c, we respectively show $L_{\id}^{\mathcal{N}_p}(t)$ and  $L_{A}^{\mathcal{N}_p}(t)$ for the initial state $\ket{\psi} = \ket{X+}$ for different system sizes. The computations were performed using the Monte Carlo wavefunction method with the Cirq library~\cite{cirq_developers_2022_6599601}. Each data point in the figure corresponds to an average over 2000 quantum trajectories. This number of trajectories is sufficient to observe convergence of the mean value in the region of our interest. The results agree well with Eq.~(\ref{eq:scaling}). This also holds for different types of noise as we show in Appendix~\ref{sec:supp_plot}. We note that the data points start to deviate from the estimated black dashed lines at $ND$ approximately linear in $N$, in line with the expectation from Eq.~(\ref{eq:limit_depth}).


\begin{figure}
    \centering
    \xincludegraphics[width=0.235\textwidth, label=(a)]{figs/04a_TELoschmidt_alphaPi8.0_angle8_dep0.003_XX_scaling_s_runs2000_cutoff0.010.pdf}
    \xincludegraphics[width=0.235\textwidth, label=(b)]{figs/04b_TELoschmidt_alphaPi8.0_angle8_cutoff0.010_new_depolarizing.pdf}
    \caption{\textbf{(a)} The mitigation error $s_A^{\mathcal{N}_p}$ for the range of data in Fig.~\ref{fig:scaling} where $L_{\id}^{\mathcal{N}_p}(t) > 0.01$. We choose this cutoff due to the limited number of trajectories in the simulation, which limits the significant digits. \textbf{(b)} The root-mean-square of $s$, $\sqrt{\sum_{\mathrm{data}} s^2 / \sum_{\mathrm{data}}}$, evaluated over the window of circuit depth $[D - 16, D + 16]$ in \textbf{(a)}.}
    \label{fig:error_scaling}
\end{figure}

To quantify the error of the mitigation strategy, we define 
\begin{eqnarray}
    s_{A}^{\mathcal{N}_p}(t) = L_{A}^{\mathcal{N}_p}(t) \big/  L_{\id}^{\mathcal{N}_p}(t) - L_{A}(t).
\end{eqnarray}
Figure~\ref{fig:error_scaling}a shows the distribution of $s$ of the mitigated data from Fig.~\ref{fig:scaling}. The error remains small for depths up to $D \approx 100$. To compare different noise rates, we plot in Fig.~\ref{fig:error_scaling}b the square root of the moving average of $s^2$ for different values of $p$. Similar plots for types of noise other than depolarizing noise are presented in Appendix~\ref{sec:supp_plot}. For reference, the typical value of $L_{A}(t)$ in the simulation is around $0.3$, which indicates that for circuit depth $D=80$, the relative error is around 10\% for $p = 0.3\%$.

Although these results confirm the effectiveness of our error mitigation strategy, we also observe a systematic shift of $s$ towards positive values. This can be explained by the error terms in Eq.~(\ref{eq:expansion_sp}). Let us assume for simplicity that $\tilde{\rho} = \id / 2^N$, from which it follows that 
\begin{eqnarray}
    \begin{aligned}
    \frac{L_{A}^{\mathcal{N}_p}(t)}{L_{\id}^{\mathcal{N}_p}(t)} & = \frac{q^2 L_{A}(t) + (1 - q^2)/2^N}{q^2 + (1 - q^2)/2^N},
    \end{aligned}
\end{eqnarray}
where we used the fact that $A^2 = \id$ since $A$ is hermitian and unitary. Hence,
\begin{eqnarray}
    s_{A}^{\mathcal{N}_p}(t) = \left[1 - L_{A}(t)\right] \frac{(1 - q^2)}{q^2\cdot 2^N + (1 - q^2)} > 0.
    \label{eq:error_evaluation}
\end{eqnarray}
For certain error models, it may be possible to remove this systematic error by using a more complicated rescaling formula instead of \eqref{eq:rescale}. Nevertheless, the systematic error remains small as long as $q^2 \gg \tr(\tilde \rho^2)$.

\begin{figure}[t]
    \centering
    \includegraphics[width=0.38\textwidth]{figs/05_Spec_XX_Lx4_Ly4_alphaPi8.0_new_p0.003.pdf}
    \caption{The time average of $L_{\sigma_i^x \sigma_{i+1}^x, \psi}(t)$ at $J t \approx 7.85$ with $\omega = 8J$, corresponding to 10 Trotter steps. The black crosses show the noiseless result. The red points were obtained by applying our error mitigation strategy to noisy simulations with a single-qubit depolarization rate $p = 0.3\%$. Error bars indicate the statistic errors due to fluctuations of different Monte Carlo trajectories, propagated from the standard deviations of $L_{A}^{\mathcal{N}_p}(t)$ and $L_{\id}^{\mathcal{N}_p}(t)$. The system size $N = 4 \times 4$.}
    \label{fig:full_exp} 
\end{figure}

We will now argue that our mitigation strategy enables the observation of prethermalization on current and near-term quantum devices. After Trotterization, the total required circuit depth $D$ to simulate time evolution of the two-dimensional XY model up to time $t_{\max}$ is
\begin{eqnarray}
	D = 4 \cdot 2 \cdot t_{\max} / \tau,
\end{eqnarray}
which, from left to right, represents the number of layers per Trotter step, back and forward evolution, and the number of Trotter steps. To see prethermalization of the Floquet XY model, Fig.~\ref{fig:prethermal} indicates that $t_{\max}$ should be at least $8 / J$ for $\omega = 8J$, which yields $D \approx 80$. The estimation is within the limit of the maximum circuit depth from Eq.~(\ref{eq:limit_depth}) and Fig.~\ref{fig:error_scaling} for $p =0.3\%$, showing that our proposal is suitable for current and near-term quantum devices.

We have now gathered all the ingredients for the full simulation of the PEVP on a noisy quantum device. We consider the two-dimensional XY model on a $4 \times 4$ square lattice in the presence of depolarizing noise with noise rate $p = 0.3 \%$. For the observable, we focus on the correlator $A = 4S_i^x S_{i+1}^x$ of a pair of neighboring sites at the center of the lattice.  In Fig.~\ref{fig:full_exp}, we plot the time averages of $\langle A(t) \rangle^2$ at driving frequency $\omega = 8J$ as a function of the initial state energy $E$ up to $t = 10 \tau$, corresponding to circuit depth $D=80$. The initial states were chosen from the same set as in Fig.~\ref{fig:prethermal}b. The black crosses represent the noise-free results, whereas for the red points the experiment was simulated including noise and error mitigation. The error bars show statistical errors due to fluctuations of different Monte Carlo trajectories, propagated from the standard deviations of $L_{A}^{\mathcal{N}_p}(t)$ and $L_{\id}^{\mathcal{N}_p}(t)$. Note that the sign of $\braket{\psi | A (t) | \psi }$ turns out to be constant during the Floquet time evolution in our range of simulations. In the long-time limit, the time average of the square is therefore equivalent to the square of the time average, given that they converge to a constant.

We find that the noise-free results lie within the error bars for all initial states and that the trend of the observable is well reproduced. This shows that our error mitigation procedure is viable to solve the PEVP. We note that the deviation between the noisy and noise-free results is biased since the red points are systematically above the black crosses, consistent with the expectation from Eq.~(\ref{eq:error_evaluation}).


\subsection{Implementation\label{sec:implementation}}

The results of the previous section show that our error mitigation strategy enables the solution of the PEVP for the XY model at a depolarizing noise rate of $p = 0.3 \%$. One more step remains to assess the experimental viability: an estimate of the number of required measurements.

In experiments, the survival probabilities are estimated from binary outcomes (success / failure). This gives rise to shot noise, which in turn sets a lower bound on the necessary number of samples. To achieve a statistical uncertainty of $\epsilon$, roughly $1/\epsilon^2$ samples are needed. For the error mitigation scheme to work, the shot noise must be smaller than the survival probability. As the noisy survival probability is suppressed by the factor $(1-p)^{ND}$, it follows that the number of needed measurements scales as $(1-p)^{-2ND}$. We note that this number of samples is typically orders of magnitude larger than the number needed to suppress the fluctuations in Monte Carlo trajectories due to noisy dynamics.

Since the sample complexity scales exponentially with the number of qubits, this is an important limitation to the system size that can realistically be reached. Nevertheless, classically hard regimes are accessible with realistic parameters. For instance, setting $N = 50$ while keeping $p = 0.3 \%$ and $D = 80$, we find that $(1-p)^{-2ND} \approx 3 \times 10^{10}$ samples are needed. This is inconveniently large as current superconducting quantum devices can collect millions of samples on the time scale of minutes. However, a modest improvement in the error rate to $p = 0.2\%$ reduces the number of samples to a much more realistic value of $9 \times 10^6$.

We have so far neglected the role of measurement errors, which occur with probability $p_{m} \approx 1 \% - 2\%$ for each single qubit measurement in current devices~\cite{Satzinger2021, Wei2022a}. Fortunately, these errors are automatically remedied by our error mitigation strategy. The measurement errors simply suppress the survival probability by another factor $(1-p_{m})^N$, which is independent of the circuit depth. For system sizes up to $N=50$, this increases the required number of measurements by at most an order of magnitude.


This paper is organized as follows: Section \ref{sec:preliminaries} formalizes notation and summarizes concepts in measure theory, time-delay, occupation measures, and \ac{ODE} peak estimation. Section \ref{sec:peak_lp} defines an \ac{MV}-solution for free-terminal-time \ac{DDE} solutions to create a measure-\ac{LP} that upper-bounds \eqref{eq:peak_delay_traj}.
Section \ref{sec:moment} reviews the Moment-\ac{SOS} hierarchy and applies it to finding \acp{SDP} to upper-bound the peak-estimation measure \ac{LP}.
Section \ref{sec:examples} provides two examples of \ac{DDE} peak estimation. Section \ref{sec:conclusion} concludes the paper.

\bibliography{main}

\clearpage
\appendix
\begin{appendices}
    \section{Appendix for Proofs}

\paragraph{Proof of Theorem \ref{thm:main}.}

\begin{proof}
\label{proof:main}
Our proof has two steps. In Step 1, we will show that SimCLR is equivalent to minimizing the cross entropy loss defined in Eqn.~(\ref{eqn:cross-entropy}). 
In Step 2, we will show  that minimizing the cross-entropy loss 
is equivalent to spectral clustering on $\bfpi$. 
Combining the two steps together, we have proved our theorem. 

\textbf{Step 1: } SimCLR is equivalent to minimizing the cross entropy loss.

The cross-entropy loss takes expectation over 
$\bfW_\bfX\sim \mathbb{P}(\cdot ; \bfpi)$, 
which means $\bfW_\bfX$ has exactly one non-zero entry in each row $i$. By Lemma~\ref{lem:multinomial}, we know every row $i$ of $\bfW_\bfX$ is independent of other rows. Moreover, 
$\bfW_{\bfX,i}\sim \mathcal{M}(1, \bfpi_i/\sum_j \bfpi_{i,j})=\mathcal{M}(1, \bfpi_i)$, because $\bfpi_i$ itself is a probability distribution.
Similarly, we know $\bfW_\bfZ$ also has the row-independent property by sampling over $\mathbb{P}(\cdot;\bfK_\bfZ)$.
Therefore, by Lemma~\ref{lem:cross_split}, we know Eqn.~(\ref{eqn:cross-entropy}) is equivalent to:
\[
 -\sum_{i=1}^n \mathbb{E}_{\bfW_{\bfX,i}}[\log \mathbb{P}(\bfW_{\bfZ,i}=\bfW_{\bfX,i};\bfK_\bfZ)],
\]

This expression takes expectation over $\bfW_{\bfX,i}$ for the given row $i$. Notice that 
$\bfW_{\bfX,i}$ has exactly one non-zero entry, which equals $1$ (same for $\bfW_{\bfZ,i}$). 
As a result
we expand the above expression to be:
\begin{equation}
 -\sum_{i=1}^n \sum_{j\neq i} \Pr(\bfW_{\bfX,i,j}=1)\log \Pr(\bfW_{\bfZ,i,j}=1).
\label{eqn:detailed-expansion}    
\end{equation}


By Lemma~\ref{lem:multinomial}, $\Pr(\bfW_{\bfZ,i,j}=1)=\bfK_{\bfZ,i,j}/\|\bfK_{\bfZ,i}\|_1$ for $j\neq i$. Recall that $\bfK_\bfZ=(k(\bfZ_i-\bfZ_j))_{(i,j)\in[n]^2}$, which means 
$\bfK_{\bfZ,i,j}/\|\bfK_{\bfZ,i}\|_1=\frac{\exp(-\|\bfZ_i-\bfZ_j\|^2/{2\tau})}{\sum_{k\neq i}
\exp(-\|\bfZ_i-\bfZ_k\|^2/{2\tau})
}$ for $j\neq i$, when $k$ is the Gaussian kernel with variance $\tau$. 

Notice that $\bfZ_i=f(\bfX_i)$, so we know
\begin{equation}
-\log \Pr(\bfW_{\bfZ,i,j}=1)=
-\log \frac{\exp(-\|f(\bfX_i)-f(\bfX_j)\|^2/{2\tau})}{\sum_{k\neq i}
\exp(-\|f(\bfX_i)-f(\bfX_k)\|^2/{2\tau}),
}
\label{eqn:infonce-equivalence}    
\end{equation}


The right hand side is exactly the InfoNCE loss defined in Eqn.~(\ref{eqn:infonce}).
Inserting Eqn.~(\ref{eqn:infonce-equivalence}) into Eqn.~(\ref{eqn:detailed-expansion}), we get the SimCLR algorithm, which first samples augmentation pairs $(i,j)$ with $\Pr(\bfW_{\bfX,i,j}=1)$ for each row $i$, and then optimize the InfoNCE loss. 

\textbf{Step 2: } minimizing the cross entropy loss 
is equivalent to spectral clustering on $\bfpi$.


By Lemma~\ref{lem:convert_to_spectral}, we may further convert the loss to 
\begin{equation}
\label{eqn:main-theorem-repul-attr}
\min_{\bfZ}
-\sum_{(i,j)\in [n]^2} \mathbf{P}_{i,j}
\log k (\bfZ_i-\bfZ_j)+\log \mathbf{R}(\bfZ).
\end{equation}
Since $k$ is the Gaussian kernel, this reduces to \[
\min_\bfZ \mathrm{tr}(\bfZ^\top \mathbf{L}(\bfpi) \bfZ)
+\log \mathbf{R}(\bfZ),
\]

where we use the fact that $\mathbb{E}_{\bfW_\bfX\sim \mathbb{P}(\cdot; \bfpi)}[\mathbf{L}(\bfW_\bfX)]
=\mathbf{L}(\bfpi)
$, because the Laplacian operator is linear and $
\mathbb{E}_{\bfW_\bfX\sim \mathbb{P}(\cdot; \bfpi)}(\bfW_\bfX)=\bfpi
$.
\end{proof}

\paragraph{Proof of Theorem \ref{thm:clip}.}
\begin{proof}
Since $\bfW_\bfX\sim \mathbb{P}(\cdot;\bfpi_{\mathbf{A}, \mathbf{B}})$, we know 
$\bfW_\bfX$ has exactly one non-zero entry in each row, denoting the pair that got sampled. 
A notable difference compared to the previous proof is we now have $n_\mathcal{A}+n_\mathcal{B}$ objects in our graph. CLIP deals with this by taking a mini-batch of size $2N$, 
such that $n_\mathcal{A}=n_\mathcal{B}=N$, and adding the $2N$ InfoNCE losses together. We label the objects in $\mathcal{A}$ as $[n_\mathcal{A}]$, and the objects in $\mathcal{B}$ as $\{n_\mathcal{A}+1, \cdots, n_\mathcal{A}+n_\mathcal{B}\}$. 

Notice that $\bfpi_{\mathbf{A}, \mathbf{B}}$ is a bipartite graph, so the edges of objects in $\mathcal{A}$ will only connect to object in $\mathcal{B}$ and vice versa. We can define the similarity matrix in $\cZ$ as $\bfK_\bfZ$, 
where $\bfK_\bfZ(i, j+n_\mathcal{A})=\bfK_\bfZ(j+n_\mathcal{A},i)= k(\bfZ_i-\bfZ_j)$ for $i\in [n_\mathcal{A}], j\in [n_\mathcal{B}]$, and otherwise we set $\bfK_\bfZ(i,j)=0$. 
The rest is same as the previous proof. 
\end{proof}

\paragraph{Proof of Theorem \ref{thm:exponential}.}

\begin{proof}
\label{proof:exponential}
Since the objective function consists of a linear term combined with an entropy regularization, which is a strongly concave function, the maximization problem is a convex optimization problem. Owing to the implicit constraints provided by the entropy function, the problem is equivalent to having only the equality constraint. We then introduce the Lagrangian multiplier $\lambda$ and obtain the following relaxed problem:

$$
\widetilde{E}(\boldsymbol{\alpha})=\psi_{1}-\sum_{i=1}^n \alpha_{i} \psi_{i}+\tau \sum_{i=1}^n \alpha_{i}\log \alpha_{i}+\lambda\left(\boldsymbol{\alpha}^{\top} \mathbf{1}_n-1\right).
$$

As the relaxed problem is unconstrained, taking the derivative with respect to $\alpha_{i}$ yields

$$
\frac{\partial \widetilde{E}(\boldsymbol{\alpha})}{\partial \alpha_{i}}=-\psi_{i}+\tau\left(\log \alpha_{i}+\alpha_{i} \frac{1}{\alpha_{i}}\right)+\lambda=0.
$$

Solving the above equation implies that $\alpha_{i}$ takes the form
$
\alpha_{i}=\exp \left(\frac{1}{\tau} \psi_{i}\right) \exp \left(\frac{-\lambda}{\tau}-1\right).
$ Since $\alpha_{i}$ lies on the probability simplex, the optimal $\alpha_{i}$ is explicitly given by
$
\alpha^{*}_{i}=\frac{\exp \left(\frac{1}{\tau} \psi_{i}\right)}{\sum_{i^{\prime}=1}^n \exp \left(\frac{1}{\tau} \psi_{i^{\prime}}\right)} .
$ Substituting the optimal point into the objective function, we obtain
$$
\begin{aligned}
E\left(\boldsymbol{\alpha}^*\right)  &=\psi_1-\sum_{i=1}^n \frac{\exp \left(\frac{1}{\tau} \psi_{i}\right)}{\sum_{i^{\prime}=1}^n \exp \left(\frac{1}{\tau} \psi_{i^{\prime}}\right)} \psi_{i}+\tau \sum_{i=1}^n \frac{\exp \left(\frac{1}{\tau} \psi_{i}\right)}{\sum_{i^{\prime}=1}^n \exp \left(\frac{1}{\tau} \psi_{i^{\prime}}\right)}\log \frac{\exp \left(\frac{1}{\tau} \psi_{i}\right)}{\sum_{i^{\prime}=1}^n \exp \left(\frac{1}{\tau} \psi_{i^{\prime}}\right)} \\
& =\psi_1 - \tau \log \left(\sum_{i=1}^n \exp \left(\frac{1}{\tau} \psi_{i}\right)\right).
\end{aligned}
$$
Thus, the Lagrangian dual function is given by
\begin{equation*}
-E\left(\boldsymbol{\alpha}^*\right)= -\tau \log \frac{\exp \left(\frac{1}{\tau} \psi_{1}\right)}{\sum_{i=1}^n \exp \left(\frac{1}{\tau} \psi_{i}\right)}.\qedhere
\end{equation*}
\end{proof}



\section{More on Experiments} \label{section: experiment_details}

\paragraph{CIFAR-10 and CIFAR-100} CIFAR-10 ~\citep{krizhevsky2009learning} and CIFAR-100 ~\citep{krizhevsky2009learning} are well-known classic image classification datasets. Both CIFAR-10 and CIFAR-100 contain a total of 60k $32 \times 32$ labeled images of different classes, with 50k for training and 10k for testing. CIFAR-10 is similar to CIFAR-100, except there are 10 different classes in CIFAR-10 and 100 classes in CIFAR-100.

\paragraph{TinyImageNet} TinyImageNet ~\citep{le2015tiny} is a subset of ImageNet ~\citep{deng2009imagenet}. There are 200 different object classes in TinyImageNet, with 500 training images, 50 validation images, and 50 test images for each class. All the images in TinyImageNet are colored and labeled with a size of $64 \times 64$.

\textbf{Pseudo-code.} Algorithm \ref{alg:Training Procedure} presents the pseudo-code for our empirical training procedure.

\begin{algorithm}[!htbp]
\caption{Training Procedure}
\label{alg:Training Procedure}
\begin{algorithmic}[1]
\REQUIRE trainable encoder network $f$, batch size $N$, augmentation strategy \textit{aug}, loss function $L$ with hyperparameters \textit{args}
\FOR {sampled minibatch ${x_i}_{i=1}^N$}
\FORALL{$i \in { 1, ..., N }$}
\STATE draw two augmentations $t_i = \textit{aug}\left(x_i\right) $, $t_i' = \textit{aug}\left(x_i\right) $
\STATE $z_i = f\left(t_i\right)$, $z_i' = f\left(t_i'\right)$
\ENDFOR
\STATE compute loss $\mathcal{L} = L(N, z, z', \textit{args})$
\STATE update encoder network $f$ to minimize $\mathcal{L}$
\ENDFOR
\STATE \textbf{Return} encoder network $f$
\end{algorithmic}
\end{algorithm}

We also provide the pseudo-code for our core loss function used in the training procedure in Algorithm \ref{alg:Core loss}. The pseudo-code is almost identical to SimCLR's loss function, with the exception of an extra parameter $\gamma$.

\begin{algorithm}[!htbp]
\caption{Core loss function $\mathcal{C}$}
\label{alg:Core loss}
\begin{algorithmic}[1]
\REQUIRE batch size $N$, two encoded minibatches $z_1, z_2$, $\gamma$, temperature $\tau$
\STATE $z = \textit{concat}\left(z_1, z_2\right)$
\FOR {$i \in {1, ..., 2N }, j \in {1, ..., 2N}$ }
\STATE $s_{i,j} = \Vert z_i - z_j \Vert_2^{\gamma}$
\ENDFOR
\STATE \textbf{define} $l(i, j)$ \textbf{as} $l(i, j) = - \log \frac{exp\left(s_{i,j}/\tau \right)}{\sum_{k=1}^{2N} \mathbf{1}{[k \ne i]} exp\left(s{i, j} / \tau \right)} $
\STATE \textbf{Return} $\frac{1}{2N} \sum_{k=1}^N\left[l(i, i+N) + l(i+N, i)\right]$
\end{algorithmic}
\end{algorithm}

Utilizing the core loss function $\mathcal{C}$, we can define all kernel loss functions used in our experiments in Table \ref{table: loss definition}. For all $z_i \in z$ with even dimensions $n$, we define $z_{L_i} = z_i\left[0:n/2\right]$ and $z_{R_i} = z_i\left[n/2:n\right]$.

\begin{table}[ht]
\centering
\begin{tabular}{{@{}l|l@{}}}
Kernel  &  Loss function \\ \midrule
Laplacian & $\mathcal{C}\left(N, z, z', \gamma=1, \tau\right)$\\ \midrule
Sum       & $\lambda * \mathcal{C}\left(N, z, z', \gamma=1, \tau_1\right) + (1-\lambda) * \mathcal{C}\left(N, z, z', \gamma=2, \tau_2\right)$  \\ \midrule
Concatenation Sum&$\lambda * \mathcal{C}\left(N, z_L, z'_L, \gamma=1, \tau_1\right) + (1-\lambda) * \mathcal{C}\left(N, z_R, z'_R, \gamma=2, \tau_2\right)$\\ \midrule
$\gamma = 0.5$ & $\mathcal{C}\left(N, z, z', \gamma=0.5, \tau\right)$          \\ 

\end{tabular}

\caption{Definition of kernel loss functions in our experiments}
\label {table: loss definition}
\end{table}

\textbf{Baselines.} We reproduce the SimCLR algorithm using PyTorch Lightning~\citep{PytorchLightning}.

\textbf{Encoder details.}
The encoder $f$ consists of a backbone network and a projection network. We employ ResNet50~\citep{ResNet} as the backbone and a 2-layer MLP (connected by a batch normalization~\citep{ioffe2015batch} layer and a ReLU \cite{nair2010rectified} layer) with hidden dimensions 2048 and output dimensions 128 (or 256 in the concatenation kernel case).

\textbf{Encoder hyperparameter tuning.}
For each encoder training case, we randomly sample 500 hyperparameter groups (sample details are shown in Table \ref{table: Hyperparameter sample}) and train these samples simultaneously using Ray Tune ~\citep{RayTune}, with the ASHA scheduler~\citep{li2018massively}. Ultimately, the hyperparameter group that maximizes the online validation accuracy (integrated in PyTorch Lightning) within 5000 validation steps is chosen for the given encoder training case.

\begin{table}[ht]
\centering

\begin{tabular}{@{}l|l|l@{}}
\midrule
Hyperparameter  & Sample Range & Sample Strategy \\ \midrule
start learning rate & $\left[10^{-2}, 10\right]$ & log uniform \\ \midrule
$\lambda$       & $\left[0, 1\right]$ & uniform \\ \midrule
$\tau$, $\tau_1$, $\tau_2$ & $\left[0, 1\right]$ & log uniform \\ \midrule
\end{tabular}

\caption{Hyperparameters sample strategy}
\label {table: Hyperparameter sample}
\end{table}

\textbf{Encoder training.} 
We train each encoder using the LARS optimizer~\citep{LARSOptimizer}, LambdaLR Scheduler in PyTorch, momentum 0.9, weight decay $10^{-6}$, batch size 256, and the aforementioned hyperparameters for 400 epochs on a single A-100 GPU.

\textbf{Image transformation.} The image transformation strategy, including augmentation, is identical to the default transformation strategy provided by PyTorch Lightning.

\textbf{Linear evaluation.}
The linear head is trained using the SGD optimizer with a cosine learning rate scheduler, batch size 64, and weight decay $10^{-6}$ for 100 epochs. The learning rate starts at $0.3$ and ends at $0$.

\textbf{Moco Experiments.} We also tested our method based on MoCo~\citep{he2019moco}. The results are summarized in Table \ref{tab:results-moco}. Here we choose ResNet18~\citep{ResNet} as the backbone and set a temperature of $0.1$ as default. For our simple sum kernel, we set $\lambda=0.8$. The results show that our method outperforms the original MoCo method.

\begin{table}[thb]
\centering
\caption{MoCo Experiment Results on CIFAR-10 and CIFAR-100.}
\label{tab:results-moco}
\resizebox{\textwidth}{!}{%
\begin{tabular}{@{}c|ccc|ccc@{}}
\toprule
\multirow{3}{*}{Method} & \multicolumn{3}{c|}{CIFAR-10} & \multicolumn{3}{c}{CIFAR-100} \\ \cmidrule(lr){2-4} \cmidrule(lr){5-7} 
                        & 200 epochs & 400 epochs    & 1000 epochs   & 200 epochs & 400 epochs & 1000 epochs         \\ \midrule
MoCo (repro.)         & $76.41 \pm 0.12$    & $80.01 \pm 0.15$          & $84.45 \pm 0.08$    & $\mathbf{47.02 \pm 0.11}$ & $52.50 \pm 0.07$ & $57.62 \pm 0.15$            \\
\midrule
Laplacian Kernel        & ${78.09 \pm 0.10}$    & $\mathbf{83.85 \pm 0.09}$          & $\mathbf{88.34 \pm 0.16}$    & $46.12 \pm 0.22$   & $53.44 \pm 0.17$ & $59.10 \pm 0.14$        \\
Simple Sum Kernel & $\mathbf{78.12 \pm 0.15}$   & $83.23 \pm 0.18$ & $87.50 \pm 0.20$ & $46.65 \pm 0.06$ & $\mathbf{53.62 \pm 0.19}$ & $\mathbf{59.83 \pm 0.12}$\\
\bottomrule
\end{tabular}
}
\end{table}



\section{More Experiments on Synthetic Data}


Consider a scenario with $n$ clusters, each containing $k$ vertices. Let the probability of vertices $u$ and $v$ from the same cluster belonging to $\bfpi$ be $p$. Conversely, for vertices $u$ and $v$ from different clusters, let the probability of belonging to $\pi$ be $q$. We generate the graph $\bfpi$ randomly, based on $p$ and $q$. We experiment with values of $k=100$ and $n=6$ for ease of visualization, embedding all points in a two-dimensional space. Each vertex's initial position originates from a normal distribution. In each iteration, we sample a subgraph of $\bfpi$ uniformly, ensuring each vertex has an out-degree of $1$. We then optimize the corresponding vectors using InfoNCE loss with an SGD optimizer and iterate until convergence. Our experimental setup consists of an SGD learning rate of $1$, an InfoNCE loss temperature of $0.5$, and a batch size of $50$. We evaluate two scenarios with different $p$ and $q$ values: $p=1$, $q=0$, and $p=0.75$, $q=0.2$. The results of these experiments are visualized in Figure \ref{fig:vis-spectral-cluster}. The obtained embeddings exhibit the hallmark pattern of spectral clustering of graph $\bfpi$.

\begin{figure}[!tb]
\centering
\subfigure{
\includegraphics[width=1\textwidth]{Figures/cluster_pi.png}
\label{fig:vis-cluster}
}
\subfigure{
\includegraphics[width=1\textwidth]{Figures/noised_cluster_pi.png}
\label{fig:vis-noised-cluster}
}
\caption{Visualizations of the optimization process using InfoNCE Loss on the vectors corresponding to $\bfpi$. Points of identical color belong to the same cluster within $\bfpi$. To showcase the internal structure of $\bfpi$, we randomly select 10 vertices from each cluster to display the edge distribution of $\bfpi$.}
\label{fig:vis-spectral-cluster}
\end{figure}


\end{appendices}


\end{document}