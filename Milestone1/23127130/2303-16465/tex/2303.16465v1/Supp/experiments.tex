\section{More Details of Experiments}
\label{sec:supp_exp}
In this section, we give more details about the experiments, such as baseline settings in the comparison and a detailed explanation of cube point choice in the ablation study. Finally, we show more results of our method in Fig.~\ref{fig:more_results}. 

\paragraph{Baseline Settings.} 
In our experiments, we adopt VCM~\cite{merigot2010voronoi}, EC-NET~\cite{yu2018ec}, and PIE-NET~\cite{wang2020pie} as baselines to evaluate our proposed method. Specifically, we use the implementation of VCM in the CGAL library~\cite{fabri2009cgal}. Given a point cloud for testing, we compute the Voronoi covariance at each point, where the offset radius is 0.2, the convolution radius is 0.25, and the pareto-optimal threshold is 0.24. When a Voronoi covariance value is larger than the threshold, we consider the corresponding point to belong to an edge. For EC-NET and PIE-NET, we utilize the released source codes and pre-trained models to test the input point cloud with their specified normalization, and then transform the outputs to align with the ground truth for a fair evaluation.

\paragraph{Time Consumption.}
The average inference times of VCM, EC-Net, PIE-Net, and Ours are 2.06, 0.84, 0.52, and 0.15 seconds, respectively. For post-processing, our method takes 0.02s on average, which is more efficient than the post-processing of PIE-Net (3.01s). It can be explained by using masks, which can only choose surface cubes for the calculation to reduce consumption and make it more efficient than other approaches. 

\begin{table}[htb]
    \centering
    \resizebox{\columnwidth}{!}{
        \begin{tabular}{c|c|c|c|c|c}
        \toprule
            & & VCM~\cite{merigot2010voronoi} & EC-NET~\cite{yu2018ec} & PIE-NET\cite{wang2020pie} & Ours \\
            \hline
            Clean & & 0.194 & 0.128 & 0.132 & \textbf{0.071} \\ 
            \hline
            \multirow{2}{2em}{Noise} & $\sigma=l/4$ & 0.200 & 0.222 & 0.289 & \textbf{0.097} \\
            & $\sigma=l/2$ & 0.270 & 0.278 & 0.301 & \textbf{0.110} \\
            \hline
            \multirow{3}{3em}{\#Sample Points} & 16384 & 0.185 & 0.135 & 0.179 & \textbf{0.095}  \\
            & 8192 & 0.192 & 0.164 & 0.238 & \textbf{0.112} \\
            & 4096 & 0.203 & 0.212 & 0.246 & \textbf{0.146} \\
        \bottomrule
        \end{tabular}
    }
    \caption{Edge estimation errors (HD, the smaller the better) of four methods on noisy or resampled inputs. $l=2/32$ is the edge length of a cube in grid.}
    \label{tab:Supp_stresstest_edgepoints}
    \vspace{-4mm}
\end{table}

\paragraph{HD Results of Stress Tests.} Table~\ref{tab:Supp_stresstest_edgepoints} shows additional quantitative results of edge estimation in stress tests. Here the metric is HD instead of CD used in the main paper.

\paragraph{Ablation Study for Cube Point Choice .} We show details for the discussion of cube point choice. In Dual Contouring~\cite{ju2002dual}, the point position in the cube is calculated by minimizing a quadratic error function (QEF) on Hermite data of the surface, which are intersection points of the surface with the cube edges and their corresponding normal vectors. Let $p$ be the point position, it can be calculated by the following minimization:
\begin{align}
    p = \arg \min_x \sum_{i} (n_i \cdot (x - p_i))^2.
\end{align}
where $p_i$ is one of the intersection points with the cube edges and $n_i$ is its normal vector. The surface is approximated as a plane at each intersection point, and the formulation minimizes all the distances from $p$ to all planes in the sense of least squares. One natural counterpart for a 3D curve is to consider the intersection points with cube faces and their tangent vectors. Similarly, we approximate the curve as a line at each intersection point $p_i$ with the cube face, and minimize all distances from $p$ to all lines in the sense of least squares. Let $t_i$ be the tangent vector of $p_i$ (the direction does not need to be specified), QEF for the curve can be formulated as:
\begin{align}
    \min_{x, \forall \alpha_i} \sum_{i} \left \| x - p_i - \alpha_i t_i\right \|_2^2 + \lambda \sum_{i} \alpha_i^2.
\end{align}
where $\alpha_i$ is proposed to enable the problem to be solved by a linear system and $\lambda$ is a weight to balance the two terms. Notice that we only need to solve for $x$, and the system can be easily reformed as a linear system of order 3 to solve for $x$. However, such a type of point position definition does not perform well in the curve restoration as shown in our main paper. Therefore, we finally choose a simple and accurate definition of the point position, which adopts the midpoint of the truncated curve inside the cube. See Fig.~\ref{fig:cube_point_choice} for a visual comparison of these two definitions.

\begin{figure}[tb]
	\centering
	\includegraphics[width=\linewidth]{figures/images/cube_point_choice_v1.png}
	\caption{2D illustration for two types of cube point definitions. (a) $p$ is obtained by minimizing a QEF. (b) $p$ is obtained by taking the midpoint of the truncated curve inside the cube.
	}
	\label{fig:cube_point_choice}
	\vspace{-1mm}
\end{figure}

\paragraph{Performance in Higher Resolution.}
To evaluate the performance of \modelName{} in a higher resolution, we perform experiments using resolution $128^3$ and compare the results with lower resolutions, as shown in Table~\ref{tab:performance_reso128}. Using resolution $128^3$ brings more accurate PWL outputs (CD, HD), with slightly fewer scores of \RecallCube{}, \PrecisionCube{}, and \CorrectFace{}. The average inference times of Ours-$32^3$, Ours-$64^3$, and Ours-$128^3$ are 0.15, 0.21, and 0.57 seconds, respectively. On the whole, the performance of \modelName{} can scale well with the increase of voxel grid resolutions. 

\begin{table}[htb]
    \centering
    \resizebox{0.91\columnwidth}{!}{
        \begin{tabular}{l|c|c|c|c|c|c}
        \toprule
                   & \RecallCube{}$\uparrow$ & \PrecisionCube{}$\uparrow$ & \CorrectFace{}$\uparrow$ & \DistancePoint{}$\downarrow$ & CD$\downarrow$ & HD$\downarrow$ \\
             \hline
            Reso $32^3$ & \textbf{0.965}  & \textbf{0.965}  & \textbf{0.940}  & 0.003  & 0.0012  & 0.0714 \\
            \hline
            Reso $64^3$ & 0.958  & 0.960  & \textbf{0.940}  & 0.001  & 0.0009  & 0.0523 \\
            \hline
            Reso $128^3$ & 0.945  & 0.947  & 0.914  & \textbf{0.0005}  & \textbf{0.0008}  & \textbf{0.0484} \\
        \bottomrule
        \end{tabular}
    }
    \caption{\modelName{} performance in different resolutions.}
    \label{tab:performance_reso128}
    \vspace{-4mm}
\end{table}

\paragraph{Ablation on Choice of $k$ in Point Encoder.}
We choose $k=8$ empirically based on the experiments of resolution $32^3$. For the higher resolution $64^3$, using a larger $k$ may improve the accuracy but incur much more calculation cost, as shown in Table~\ref{tab:supp_ablation_KNN}. It is a trade-off of accuracy and efficiency to apply $k=8$ in our experiments.

\begin{table}[htb]
    \centering
    \resizebox{\columnwidth}{!}{
        \begin{tabular}{l|c|c|c|c|c|c}
        \toprule
              & $32^3,k=4$ & $32^3,k=8$ & $32^3,k=16$ & $64^3,k=4$ & $64^3,k=8$  & $64^3,k=16$ \\
            \hline
            CD$\downarrow$ & 0.0017 & 0.0012 & 0.0012 & 0.0015 & 0.0009 & 0.0008 \\
            \hline
            HD$\downarrow$ & 0.0833 & 0.0714 & 0.0669 & 0.0680 & 0.0523 & 0.0491 \\
        \bottomrule
        \end{tabular}
    }
    \caption{Ablation study on choices of k in the point encoder. }
    \label{tab:supp_ablation_KNN}
    \vspace{-5mm}
\end{table}

\paragraph{Ablation on Using PointNet++ or 3DCNN Features.}
We conducted an ablation study with or without using PointNet++ and 3DCNN features. The results of the network predictions and the PWL curves are reported in Table~\ref{tab:supp_PN_CNN}. As shown, both the PointNet++ and 3DCNN features can promote the performance of \modelName{}.

\begin{table}[htb]
    \centering
    \resizebox{1.\columnwidth}{!}{
        \begin{tabular}{l|c|c|c|c|c|c}
        \toprule
                   & \RecallCube{}$\uparrow$ & \PrecisionCube{}$\uparrow$ & \CorrectFace{}$\uparrow$ & \DistancePoint{}$\downarrow$ & CD$\downarrow$ & HD$\downarrow$ \\
            \hline
            wo PointNet++ & 0.9358 & 0.9494 & 0.8517 & 0.0041 & 0.0025 & 0.0982 \\
            \hline
            wo 3DCNN & 0.9383 & 0.9460 & 0.9127 & 0.0040 & 0.0020 & 0.0942 \\
            \hline
            Ours & \textbf{0.9649} & \textbf{0.9650} & \textbf{0.9437} & \textbf{0.0030} & \textbf{0.0012} & \textbf{0.0714} \\
        \bottomrule
        \end{tabular}
    }
    \caption{Ablation study on using PointNet++ and 3DCNN blocks.}
    \label{tab:supp_PN_CNN}
    \vspace{-4mm}
\end{table}

\paragraph{Ablation Study on Post-processing.}
The quantitative results of parametric curves with/without post-processing are CD: 0.008/0.067, HD: 0.224/0.225. Thus, the post-processing is necessary numerically for better parametric extraction.


\subsection{Correlation distribution}

\paragraph{Generalization across $\alpha$'s.} In \Cref{fig:jointplot} left, we compare the linear datamodeling scores (LDS) evaluated on $\alpha=0.5$ sub-sampled training sets to those evaluated on $\alpha=0.75$.
(The numbers are overall lower as these are evaluated on data where only one model was trained on each subset,instead of averaging over 5 models; hence, there is more noise in the data.) As we observe, the LDS scores on different $\alpha$'s are highly correlated, suggesting that \trak scores computed on a single $\alpha$ generalize well.

\paragraph{LDS correlation between \trak and datamodels.} In \Cref{fig:jointplot} right, we compare the LDS correlations of datamodels to that of \trak and find that they are correlated across examples; in general, \trak also performs better on examples on which datamodels perform better.

\begin{figure}[!htbp]
    \centering
    \includegraphics[width=0.45\linewidth]{figures/cifar2_off_dist.pdf}
    \includegraphics[width=0.45\linewidth]{figures/cifar2_dm_vs_trak.pdf}
\caption{{\bf (Left)} The LDS of \cifartwo \trak scores computed with $\alpha=0.5$ models then evaluated on either models trained with either $\alpha=0.5$ or $\alpha=0.75$. Each point corresponds to a validation example. {\bf (Right)} The LDS of \cifartwo datamodel scores compared with that of \trak. Here, the LDS is measured on two different estimators.}
\label{fig:jointplot}
\end{figure}



\clearpage
\subsection{Table for LDS evaluation}

\begin{table}[h]
    \centering
    \begin{tabular}{llrrrrrrrrr}
        \toprule
        Dataset & & TRAK & TracIn \citep{pruthi2020estimating} & Infl. \citep{koh2017understanding} & Datamodels \citep{ilyas2022datamodels} \\
        \midrule
        \cifartwo & \# models & 5 & 100 & - & 1,000 \\
        & Time (min.) & 3 & 100 & - & 500  \\
        & LDS & {\bf 0.203(3)} & 0.056(2) & - & 0.162(5)  \\
        \midrule
        \cifarten & \# models & 20 & 20 & 1 & 5,000 \\
        & Time (min.) & 20 & 60 & 20,000 & 2,500 \\
        & LDS & {\bf 0.271(4)} & 0.056(7) & 0.037(13) & 0.199(4) \\
        \midrule
        \qnli & \# models & 10 & 1 &  1 & 20,000 \\
        & Time (min.) & 640 & 284 & 18,000 & 176,000 \\
        & LDS & {\bf 0.416(10)} & 0.077(29) & 0.114(43) & 0.344(32) \\
        \midrule
        ImageNet & \# models & 100 & 1 &  20 & 30,000 \\
        & Time (min.) & 2920 & 76 & $>$100,000 &   525,000  \\
        & LDS & {\bf 0.188(6)} & 0.008(6) &   0.037(6) & 0.1445(6) \\
        \bottomrule
        \end{tabular}
        \caption{{\em Comparison of different data attribution methods.} We quantify various data attribution methods in terms of both their {\em predictiveness}---as
        measured by the linear datamodeling score---as well as their {\em
        computational efficiency}---as measured by either the total computation
        time (wall-time measured in minutes on a single A100 GPU; see
        \Cref{app:wall_time} for details) or the number of trained models used
        to compute the attribution scores. The errors indicate 95\%
        bootstrap confidence intervals.
        Sampling-based methods (datamodels and
        empirical influences) can outperform \trak when allowed to use more
        computation, but this leads to a significant
        increase in computational cost.
        }
        \label{tab:all_best}
\end{table}





\clearpage
\subsection{\trak examples}
\label{app:more_examples}
We display more examples identified with \trak scores in \Cref{fig:imagenet_nns_extra} (ImageNet), \Cref{tab:qnli_more} (\qnli), and \Cref{fig:clip_examples_extra} (\clip on \mscoco).

\begin{figure}[!b]
    \centering
    \includegraphics[width=.9\linewidth,trim={0 0 0 0},clip]{figures/imagenet_nns_extra.pdf}
\caption{
     {\em \trak attributions for ResNets trained on ImageNet.}
    We display random test examples and their corresponding
    most helpful (highest-scoring) and most detracting (lowest-scoring)
    training examples according to \trak.
}
\label{fig:imagenet_nns_extra}
\end{figure}


\clearpage
\begin{figure}
    \centering
    \begin{tabular}{p{0.33\textwidth}p{0.30\textwidth}p{0.30\textwidth}}
    \toprule
    \textbf{Example} & \textbf{Highest \trak score (+)} & \textbf{Lowest \trak score (-)} \\
    \midrule
    \scriptsize {\bf Q:} What was a major success, especially in rebuilding Warsaw? {\bf A:} Like many cities in Central and Eastern Europe, infrastructure in Warsaw suffered considerably during its time as an Eastern Bloc economy – though it is worth mentioning that the initial Three-Year Plan to rebuild Poland (especially Warsaw) was a major success, but what followed was very much the opposite. {\bf (Yes)} & \scriptsize {\bf Q:} In 1998, the deal was renewed for what amount over four years? {\bf A:} Television money had also become much more important; the Football League received £6.3 million for a two-year agreement in 1986, but when that deal was renewed in 1988, the price rose to £44 million over four years. {\bf (Yes)} & \scriptsize {\bf Q:} Who was a controversial figure due to a corked-bat incident? {\bf A:} Already a controversial figure in the clubhouse after his corked-bat incident, Sammy's actions alienated much of his once strong fan base as well as the few teammates still on good terms with him, (many teammates grew tired of Sosa playing loud salsa music in the locker room) and possibly tarnished his place in Cubs' lore for years to come. {\bf (No)} \\
    \midrule
    \scriptsize {\bf Q:} What is the name associated with the eight areas that make up a part of southern California? {\bf A:} Southern California consists of one Combined Statistical Area, eight Metropolitan Statistical Areas, one international metropolitan area, and multiple metropolitan divisions. {\bf (Yes)} & \scriptsize {\bf Q:} Was was the name given to the Alsace provincinal court? {\bf A:} The province had a single provincial court (Landgericht) and a central administration with its seat at Hagenau. {\bf (Yes)} & \scriptsize {\bf Q:} What do six of the questions asses? {\bf A:} For each question on the scale that measures homosexuality there is a corresponding question that measures heterosexuality giving six matching pairs of questions. {\bf (No)} \\
    \midrule
    \scriptsize {\bf Q:} What words are inscribed on the mace of parliament? {\bf A:} The words There shall be a Scottish Parliament, which are the first words of the Scotland Act, are inscribed around the head of the mace, which has a formal ceremonial role in the meetings of Parliament, reinforcing the authority of the Parliament in its ability to make laws. {\bf (No)} & \scriptsize {\bf Q:} Whose name is on the gate-house fronting School Yard? {\bf A:} His name is borne by the big gate-house in the west range of the cloisters, fronting School Yard, perhaps the most famous image of the school. {\bf (No)} & \scriptsize {\bf Q:} What kind of signs were removed form club Barcelona? {\bf A:} All signs of regional nationalism, including language, flag and other signs of separatism were banned throughout Spain. {\bf (Yes)} \\
    \midrule
    \scriptsize {\bf Q:} What was the percentage of a female householder with no husband present? {\bf A:} There were 158,349 households, of which 68,511 (43.3\%) had children under the age of 18 living in them, 69,284 (43.8\%) were opposite-sex married couples living together, 30,547 (19.3\%) had a female householder with no husband present, 11,698 (7.4\%) had a male householder with no wife present. {\bf (Yes)} & \scriptsize {\bf Q:} What percent of household have children under 18? {\bf A:} There were 46,917 households, out of which 7,835 (16.7\%) had children under the age of 18 living in them, 13,092 (27.9\%) were opposite-sex married couples living together, 3,510 (7.5\%) had a female householder with no husband present, 1,327 (2.8\%) had a male householder with no wife present. {\bf (Yes)} & \scriptsize {\bf Q:} Roughly how many same-sex couples were there? {\bf A:} There were 46,917 households, out of which 7,835 (16.7\%) had children under the age of 18 living in them, 13,092 (27.9\%) were opposite-sex married couples living together, 3,510 (7.5\%) had a female householder with no husband present, 1,327 (2.8\%) had a male householder with no wife present. {\bf (No)} \\
        \midrule
        \scriptsize {\bf Q:} What did Warsz own? {\bf A:} In actuality, Warsz was a 12th/13th-century nobleman who owned a village located at the modern-day site of Mariensztat neighbourhood. {\bf (Yes)} & \scriptsize {\bf Q:} What company did Ray Kroc own? {\bf A:} It was founded in 1986 through the donations of Joan B. Kroc, the widow of McDonald's owner Ray Kroc. {\bf (Yes)} & \scriptsize {\bf Q:} What did Cerberus guard? {\bf A:} In Norse mythology, a bloody, four-eyed dog called Garmr guards Helheim. {\bf (No)} \\
        \midrule
        \scriptsize {\bf Q:} What words are inscribed on the mace of parliament? {\bf A:} The words There shall be a Scottish Parliament, which are the first words of the Scotland Act, are inscribed around the head of the mace, which has a formal ceremonial role in the meetings of Parliament, reinforcing the authority of the Parliament in its ability to make laws. {\bf (No)} & \scriptsize {\bf Q:} Whose name is on the gate-house fronting School Yard? {\bf A:} His name is borne by the big gate-house in the west range of the cloisters, fronting School Yard, perhaps the most famous image of the school. {\bf (No)} & \scriptsize {\bf Q:} What kind of signs were removed form club Barcelona? {\bf A:} All signs of regional nationalism, including language, flag and other signs of separatism were banned throughout Spain. {\bf (Yes)} \\
    \bottomrule
\end{tabular}
\caption{{\em Top \trak attributions for \qnli examples.} Yes/No indicates the label (entailment vs. no entailment).}
\label{tab:qnli_more}
\end{figure}

\clearpage
\begin{figure}[!t]
    \centering
    \includegraphics[width=\linewidth,trim={0 0 0 0},clip]{figures/CLIP_examples/clip_examples_0.pdf}
    \includegraphics[width=\linewidth,trim={0 0 0 0},clip]{figures/CLIP_examples/clip_examples_1.pdf}
    \includegraphics[width=\linewidth,trim={0 0 0 0},clip]{figures/CLIP_examples/clip_examples_2.pdf}
\caption{
     {\em Top attributions for \clip models trained on \mscoco.}
    We display random test examples and their corresponding
    most helpful (highest-scoring) and most detracting (lowest-scoring)
    training examples according to \trak, \clip similarity distance, and \tracin.
    }
\label{fig:clip_examples_extra}
\end{figure}






\clearpage
\subsection{\modeldiff with \trak}
\Cref{fig:modeldiff} shows how we apply \trak to dramatically accelerate the
\modeldiff algorithm.
\begin{figure}[h]
    \centering
    \includegraphics[width=\linewidth,trim={0 0 0 0},clip]{figures/modeldiff_pipeline.pdf}
\caption{
{\em Accelerating learning algorithm comparisons with \trak.}
The \modeldiff framework from \citep{shah2022modeldiff} uses datamodel
representations to surface features that distinguish two learning algorithms. In
the case study here, we compare models trained on the \textsc{Living17} dataset {\em with} and {\em without} data
augmentation. Applying \modeldiff involves three stages: (1) computing datamodel
representations; (2) applying the \modeldiff algorithm to extract {\em
distinguishing subpopulations} of inputs on which two model classes behave
differently; (3) counterfactually testing the inferred feature associated
with the subpopulation. \citet{shah2022modeldiff} find that models trained with
data augmentation latch onto the presence of spider webs as a spurious
correlation to predict the class spider. Here, we recover their result by using
\trak scores instead of datamodel scores in step (1); doing so reduces the
computational cost of \modeldiff by 100x.
}
\label{fig:modeldiff}
\end{figure}

\paragraph{Ablation Study on data splits.}
Our experiments are all based on the same random split. Here, we test on other two different random splits with resolution $32^3$, the CD results are 0.0013 and 0.0016 (the number in our main paper is 0.0012). As shown, the differences are insignificant.

\paragraph{More Results.} More results of our method are shown in Fig.~\ref{fig:more_results}. Our method can produce reasonable results even for complicated cases, as shown in the last row of Fig.~\ref{fig:more_results}. 