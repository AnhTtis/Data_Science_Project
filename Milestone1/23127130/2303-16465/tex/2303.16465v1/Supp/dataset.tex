\section{Details of Dataset}
\label{sec:supp_dataset}
In this section, we provide more details about data processing, including data cleaning of raw ABC dataset~\cite{koch2019abc}, ground truth data preparation of NerVE attributes, and pre-processing for input point clouds.
\begin{figure}[ht]
	\centering
	\includegraphics[width=0.9\linewidth]{figures/images/ABC_Sharp_v1_crop.png}
	\caption{Differences between sharp edge and non-sharp edge. We highlight the sharp edges on the left figure, while highlight both sharp and non-sharp edges on the right figure.
	}
	\label{fig:abc_sharp_edges}
	\vspace{-5mm}
\end{figure}
\paragraph{Data Cleaning.}As discussed in our paper, the dataset has to be cleaned due to data missing, shape repetition, and lack of sharp edges. For those shapes with little difference in geometry or structure, we regard them as repeated. As for the lack of sharp edges, there are cases where a shape does not possess any sharp edges (e.g. a sphere) or only has a few sharp edges. Here we use the sharp edges originally defined in the ABC dataset~\cite{koch2019abc} and Fig.~\ref{fig:abc_sharp_edges} shows the difference. To filter out the mentioned shapes, we manually examined all shapes from the first chunk and obtained 2,364 shapes for network learning. 

\paragraph{Ground Truth Preparation.}To obtain the ground truth of cube attributes in NerVE, we first obtain the PWL curves from the GT parametric curves by uniform sampling and denote the PWL curves as the GT PWL curves. In a grid of $32^3$ cubes (here we assume the grid resolution is $32$), all cubes intersected with the GT PWL curves are labeled True, which are occupied cubes. 
Similarly, all faces intersected with the GT PWL curves are labeled True. Intersections of GT PWL curves with cubes can be easily calculated by considering each line segment in GT PWL curves. In the occupied cubes, we take the midpoint of the truncated GT PWL curve inside the cube as the point position.

\paragraph{Input Point Cloud Pre-processing.} To benefit network training, we normalize the input point cloud and GT point positions of cubes. For the input point cloud, we first subtract the point position from its $k$ nearest neighbors for each point, then multiply by a factor of $r$ ($r=32$ for resolution $32^3$). For the GT global point positions, we convert it to the local coordinates of its cube, where the center of a cube is the new origin, and the axes are scaled by $r$ ($r=32$ for resolution $32^3$). In this way, the range of the GT point position becomes $[-1,1]^3$.