\section{Method}
\label{sec:method}
In this work, we proposed a simple and effective curve representation which can represent all types of curves in unified form, and can be directly predicted by neural networks. Following starts with the concept of our representation. 

\subsection{Neural Edge Representation}
% ----------------------
% Definition
% ----------------------
Consider a cubic tessellation on space $[-1,1]^3 \subset \mathbb{R}^3$ which is uniformly divided into $r$ segments along each axis. The formed cube grid can be denoted as $\mathcal{G}=(\mathcal{C}, \mathcal{F}_{in})$, where $\mathcal{C}$ is all cubes of $\mathcal{G}$ and $\mathcal{F}_{in}$ is all inner cube faces of the tessellation. Each cube in $\mathcal{C}$ indexed by $\mathcal{I}=(i,j,k)$ is defined to locate at $\frac{2}{r}[i,j,k]$ with its min corner, see~\ref{fig:cube_grid}. Now we give a formal definition of our proposed grid cube curve representation. 

\noindent\textit{Definition} The general grid cube curve is a mapping $\pi$:
\begin{align*}
    \pi: &f \in \mathcal{F}_{in} \mapsto \pi(f) \in \{0,1\} \\
        &c \in \mathcal{C} \mapsto \point=\pi(c) \in \mathbb{R}^3.
\end{align*}
For given set of curves in $[-1,1]^3$, e.g. curves of a CAD model, it can be transformed into grid cube representation by specifying mapped values of face and cube in grid. For face, it is set as 1 if there exists one curve through this face, and set as 0 otherwise. For the cube passed through by curves, the mapped point can be obtained by simply taking the midpoint of one truncated curve inside the cube. If the curve stops at its endpoint inside the cube, we choose the endpoint. In case where multiple curves in one cube, we simply use the average of midpoints of these curves. Another possible way is to minimize a quadratic error function as in Dual Contouring(DC)~\cite{ju2002dual}, and we will discuss it in ablation study.

% ----------------------
% PWL curve extraction
% ----------------------
As stated, our representation can be the bridge between parametric curves and one general piece-wise linear(PWL) curve. In fact, there is a simple and efficient method to extract the PWL curve from a general grid cube curve $\pi$ and let us first regard the PWL curve as a graph $G =(V, E)$. Denote a cube as $c_{\mathcal{I}}$ with value $p_{\mathcal{I}}=\pi(c_{\mathcal{I}})$. To obtain $E$ of $G$, we can loop for all faces $f \in \mathcal{F}_{in}$ with value 1, and connect vertices in two cubes($c_{\mathcal{I}_1},c_{\mathcal{I}_2}$) sharing the face $f$, i.e.  $(p_{\mathcal{I}_1}, p_{\mathcal{I}_2}) \in E$, just similar to the meshing process in DC.Then $V$ of $G$ is the set of all vertices involved in $E$. The final PWL curve can be extracted from graph $G$ by drawing line on every edge in $E$, see~\ref{fig:grid_cube_curve} for illustration.

% ----------------------
% advantage and cost of discretization
% ----------------------
We have introduced a simple curve representation which can compact continuous parametric curves into discrete and regular grid of cubes, and it benefits in operation and data representation in neural networks. Meanwhile, it can be efficiently extracted as a PWL curve, which results in the capability of representing multiple curves of different types(line, Bspline, closed curve...) in unified form. The following procedure for parametric curve restoration from PWL curve is introduced in 3.3. Notice there may exist imperfection in restoration due to the discretization nature of our method, and we will discuss ways to handle those issues in later chapter. 

\subsection{Network and Training}
Our curve representation can be directly predicted by neural networks and be easily converted to structured PWL curve. In contrast, previous methods are based on unstructured point cloud near the curves, which are predicted by network or handcrafted feature. In this section, we show details related to the learning process.

% ----------------------
% data form and network
% ----------------------
Given input point cloud $\mathcal{P}$ of 3D shape with its ground truth(GT) grid cube curve $\pi$, network function $f(\theta | \mathcal{P})$ with net parameters $\theta$ is used to fit the mapping $\pi$. It suffices to minimize the value error on cube grid $\mathcal{G}$ defined in 3.1, which means our network should output two value tensors for face and cube point to fit two GT value tensors of $\pi$. For resolution $r$, both value tensors are in shape $(r,r,r,3)$\footnote{In cube grid, each cube possesses three faces to avoid face repeating. Here we simply choose three faces around the min corner of the cube as its faces.}. In another word, one 3D point and three boolean values will be predicted for each cube.

To avoid meaningless calculation on cubes away from curves, we build a network to generate a mask for cube grid to find cubes with curve inside. In inference, the face and point networks only evaluate on masked cubes instead of the whole $r^3$ grid cubes. 

% ----------------------
% Net Architecture
% ----------------------
In this work, we use network of encoder-decoder structure to predict cube mask, face and point from input $\mathcal{P}$. The encoder can encode $\mathcal{P}$ into a feature grid of shape $(r,r,r, N_{\text{feat}})$ corresponding to all cubes in the grid, and we adopt multilayer perceptron(MLP) as decoder. See~\ref{fig:network}.

In encoder network~\ref{fig:network}, first per-point feature of $\mathcal{P}$ is obtained by a dense and simplified PointNet++~\cite{qi2017pointnet++}. Next the per-point features are dumped into grid to become per-cube features by using average pooling. Features of empty cubes are initially set as zero. Finally we apply three 3D convolution layers to obtain final feature grid. More details are available in the supplementary material.

% ----------------------
% Training Process
% ----------------------
% point cloud mask
During training, we can further devise different cube masks based on prior knowledge and only supervise the prediction within the mask to improve efficiency of training. For cube mask network, prediction can be constrained in a narrow band cubes of input surface, which can be obtained by a sufficiently sampled point cloud. For face and point networks, it is only necessary to evaluate the cubes near the curves. Thus we use a narrow band cubes of GT curves as the mask. More details are shown in the supplementary material. 

Denote the masks for cube mask, face and point networks by $\mathcal{M}_C, \mathcal{M}_F, \mathcal{M}_P$. Cube mask network $f_{C}$essentially does binary classification on each cube thus we minimize it by Binary Cross Entropy(BCE) loss. Face network $f_{F}$ outputs like three binary classification so we repeatedly use BCE loss for it. For point network $f_{P}$, we use $L_1$ loss to penalize the point prediction. In summary:
\begin{align}
    L_C &= 1/|\mathcal{M}_C| \sum_{c \in \mathcal{M}_C} \text{BCE}(f_{C}(c), C_{gt})\\
    L_F &= 1/|\mathcal{M}_F| \sum_{c \in \mathcal{M}_F} \text{BCE}(f_{F}(c), F_{gt})\\
    L_P &= 1/|\mathcal{M}_P| \sum_{c \in \mathcal{M}_P} \left \| f_{P}(c)- P_{gt} \right \|_1
\end{align}
Note that three networks can be separately trained, so no balancing weights need to be given.

\subsection{Parametric Curve Extraction}
% ----------------------
% Top down& Bottom up difference
% ----------------------
For further parametric curve extraction, also known as 3D curve vectorization, our method enables the algorithm to start with PWL curve instead of unstructured point cloud used in previous methods. In fact, our algorithm does not need complicated endpoints(corner points) detection and graph structure analysis like in previous methods~\cite{matveev2022def,wang2020pie,matveev20213d} by using PWL curve. 

% ----------------------
% algorithm outline
% ----------------------
To see this, let us first assume we start with the PWL curve from GT grid cube curve, as shown in~\ref{fig:curve_from_PWL}. For CAD models, known as boundary representation(B-Rep), one curve has exact two endpoints(a closed curve has zero endpoints) and two curves can only be intersected at one endpoint. 

Therefore, it suffices to find out all vertices with degree larger than 2 in PWL curve as a graph. Given these 'degree$>$2' endpoints, the connection between two endpoints can be discovered by checking if there exists a path(no endpoints except start and end) in PWL curve graph between two endpoints. The closed curve can also be easily detected as single connected component with only degree 2 vertices in PWL curve graph. 

%%% if space is enough
% In contrast, previous methods struggle in endpoints and their connection analysis. Network based analysis has error in endpoints prediction and connection proposal~\cite{wang2020pie}, leading to loss of curves. Manual feature based analysis involves in tuning of many parameters~\cite{matveev2022def}. And they both must develop extra modules or network to deal with closed curves. 

With endpoints and their connection obtained, we use similar spline fitting procedure described in previous methods~\cite{matveev2022def,wang2020pie}. Points of PWL curves are used for fitting. 

% ----------------------
% Network refinement and Post processing 
% ----------------------
However, network can not make perfect prediction for all inputs. To alleviate the influence of error and outlier of network prediction, we propose a refinement network and a post processing method. See~\ref{fig:curve_refine}. The refinement network takes predicted PWL curve and input point cloud as inputs, and aims to filter out outlier lines which are in cubes away from curves. Then our post processing method can leverage prior knowledge from B-Rep models, producing topological refinement on predicted PWL curve. For instance, two degree 1 vertices will be connected if their positions are closed and tangent vectors are consistent; if two paths are closed in geometry sense and share the same two endpoints, we will merge them into one path. Limited to space, more details+ are put in the supplementary material. 

% ----------------------
% cost of discretization
% ----------------------
Cost of discretization. TBD(face pair?)