% ----------------------
% Problem and background
% ----------------------
Extracting parametric curves from 3D point cloud is most likely related to two areas. One is to detect and extract sharp edge or sharp feature from point cloud, which can be used to improve algorithms in geometry processing like normal estimation, surface reconstruction and shape classification. The other is for computer aided design(CAD) model reconstruction from point cloud, also known as reverse engineering, which is a fundamental problem with extensive literature and industrial applications~\cite{buonamici2018reverse}. 

% ----------------------
% drawbacks of previous methods, Top down
% ----------------------
Current methods commonly focus on improving the quality of edge point detection, with hand-designed feature~\cite{merigot2010voronoi} or neural network~\cite{yu2018ec, matveev2022def,wang2020pie,bazazian2021edc}. Predicted edge points are like narrow band near the curves. Typical following process for parametric extraction is to detect endpoints(corner points) from edge points and discover connections in endpoints by heuristic graph structure analysis, then spline fitting is applied on points near the path between two connected endpoints. However, these Top-Down approaches would encounter multiple lost or extra curves even if one wrong prediction happened in endpoints and their connection. In practice such errors usually arise from inaccuracy of network prediction~\cite{wang2020pie} or improper setting of hyper parameters~\cite{matveev2022def,merigot2010voronoi}. Another drawback of previous methods is they constrain type of curve in training~\cite{wang2020pie,guo2022complexgen} or separately process closed curves in extraction~\cite{wang2020pie, guo2022complexgen,matveev2022def}.

% ----------------------
% PWL curve can solve problem but hard for network
% ----------------------
The key observation is it can solve all mentioned problems if we directly predict a piece-wise linear(PWL) curve(or polyline) of edge instead of unstructured edge points from input point cloud. To see this, notice that all vertices with degree larger than 2 are endpoints in CAD models, so we can filter out endpoints with degree in PWL curve as graph. 
Connection in endpoints can also be discovered by search from one endpoint on PWL curve graph. For issue of curve type, analogous to triangle mesh for surface, PWL curve can represent general curves with different types(Line,BSpline,Circle...), whether closed or not. The remaining problem is general PWL curve is not regular structure that neural network can directly predict. 

% ----------------------
% Our method,bottom up, multi-type curve
% ----------------------
In this work, we propose a simple and effective curve representation which can be the bridge between parametric curve and PWL curve.
, inspired by the idea of Dual Contouring~\cite{ju2002dual} which is the algorithm for shape reconstruction. 

The representation has voxel-like regular structure that can be directly predicted by network. By converting to our representation, curves of different types from multiple CAD models can be represented in a unified form and it supports complicated topological relationship in curves as long as PWL curve can support. In inference, conversion to PWL curve is like a Bottom-Up process where each line segments are extracted and assembled as whole PWL curve. As discussed, the endpoints and their connection can then be easily discovered by PWL curve and standard spline fitting is applied to obtain final parametric curves. 

% ----------------------
% learning framework and post processing
% ----------------------
Our learning framework adopts a simplified PointNet++~\cite{qi2017pointnet++} and 3D convolution for encoding the input point cloud into feature grid, similar to structure in ~\cite{chen2022neural}. We use multilayer perceptron(MLP) as decoder to predict the curve in our proposed representation. In training and inference, several masks can be specified to reduce computation in 3D grid, which makes our method fast and efficient. Unlike those Top-Down approaches, prediction errors of network in our method make local and limited influence to the whole curve structure and can be fixed by lightweight post processing. In fact, we develop neural method and topological method to deal with prediction errors of network.

% ----------------------
% ABC dataset and experiments
% ----------------------
We train and evaluate our learning model on ABC dataset~\cite{koch2019abc}, which is a large scale CAD model dataset with ground truth parametric curves. The dataset contains a million of models and we only use part of it. Quantitative and qualitative evaluations and comparisons on our proposed representation show the effectiveness of our method. Further experiments on noisy or varying sampling density input point cloud prove robustness of our method. 
