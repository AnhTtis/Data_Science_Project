\section{Related Works}
\label{sec:related}
% Sharp edge detection has been studied for a long time. In this work, we focus on edge feature detection and parametric curve extraction from unstructured point clouds. We also review the advanced neural representation learning works in 3D vision, which inspire us to propose \modelName{} for free-form curve learning.

\noindent \textbf{Edge Feature Detection.}
Traditional edge feature detection from a point cloud is based mainly on local geometric features, such as the eigenstructure of the covariance matrix~\cite{gumhold2001feature,pauly2003multi,bazazian2015fast,xia2017fast}, normals~\cite{fleishman2005robust,demarsin2006detection,demarsin2007detection,weber2010sharp}, curvatures~\cite{lin2015line,hackel2016contour}, or other statistical metrics~\cite{weber2011methods,nie2016extracting}.
%A classic workflow is to process each point with a geometric descriptor using the eigenstructure of the covariance matrix~\cite{gumhold2001feature,pauly2003multi,bazazian2015fast,xia2017fast}, normals~\cite{fleishman2005robust,demarsin2006detection,demarsin2007detection,weber2010sharp}, curvatures~\cite{lin2015line,hackel2016contour}, or statistical metrics~\cite{weber2011methods,nie2016extracting} to distinguish whether a point belongs to a sharp edge. 
To improve the robustness of edge detection for noisy point clouds, Daniels et al.~\cite{daniels2007robust} use a multistep refinement method with robust moving least squares to fit the surface to potential features. Ni et al.~\cite{ni2016edge} combine RANSAC and the angular map metric to detect edges. VCM~\cite{merigot2010voronoi} is presented by measuring Voronoi covariance and applying the Monte-Carlo algorithm to compute feature boundaries, which is widely used in geometry processing. With the bloom of deep learning, PIE-Net~\cite{wang2020pie}, EDC-Net~\cite{bazazian2021edc}, and PCEDNet~\cite{himeur2021pcednet} are proposed to formulate the edge detection as a classification task and utilize neural networks to learn it. On the other hand, EC-Net~\cite{yu2018ec} reformulates it as a regression problem, then learns residual point coordinates and point-to-edge distances to identify edge points. All of these learning-based methods significantly improve the accuracy of edge detection. In this paper, we also utilize advanced neural networks but avoid explicit edge detection. We propose a novel \modelName{} representation to directly learn the positions and topology of edge points. 
% In this paper, we also utilize advanced neural networks but decompose edge detection into two subtasks of edge cube classification and edge point regression, and learn them separately, which achieves the state of the art.

\begin{figure*}[t]
    \centering
    \includegraphics[width=\textwidth]{figures/Overview_STG.pdf}
    %\includegraphics[width=\textwidth]{figures/pipeline2.pdf}
    \vspace{-20pt}
    \caption{\textbf{Spatio-temporal grounding approach.} 
    % We incorporate both spatial and temporal information in the training process including three modalities. 
    (a)~We want to select frames with possible groundable objects and tasks. To this end, projected word features are matched with respective frame features. (b)~Sinkhorn-knopp optimal transport is then leveraged to ensure the variety of our selected frames. (c)~Based on the selected frames, a global representation is learned to allow for temporal localization as well as (d)~a local representation to ground the action description to the spatial region. 
    %Local contrastive loss on video spatio-temporal and text features to learn multimodal interactions between finer-grained features. Global pairwise contrastive loss on video and text features to pull the features close across modalities in a high-level semantic space. 
    }
    \label{fig:pipeline}
    %\vspace{-10pt}
\end{figure*}

\vspace{1mm}
\noindent \textbf{Parametric Curve Extraction.}
Parametric curves have been widely used in CAD modeling to design complex 3D shapes with sharp geometries. However, extracting parametric curves from a point cloud is challenging due to the various types and complex connections of curves. Pioneering work~\cite{gumhold2001feature} attempts to detect feature points and fits them with splines to recover feature lines. Recent deep learning methods~\cite{wang2020pie,matveev20213d,liu2021pc2wf,tan2022coarse,matveev2022def,guo2022complexgen} leverage a keypoint fitting strategy that first detects a sparse set of keypoints (e.g., corner points and edge points), then groups and connects them, followed by a determination of the target parametric curve type and performs curve fitting. Specifically, PIE-Net~\cite{wang2020pie} utilizes a PointNet++~\cite{qi2017pointnet++}-like network to extract point features for the classification of edge points, corners, and others, then generates curve proposals for parametric curve extraction. ComplexGen~\cite{guo2022complexgen} formulates the prediction of validness and primitive types as classification tasks, and recovers corners, curves, and patches simultaneously along with their mutual topology constraints. In contrast, DEF~\cite{matveev2022def} regresses a continuous distance field to represent the distance from the input points to the closest feature lines, and then extracts parametric feature curves from the inferred field. Other works~\cite{matveev20213d,liu2021pc2wf,tan2022coarse} simplify the curves into lines only, and focus on 3D wireframe reconstruction. However, complicated corner detection and connection estimation make these keypoint fitting methods prone to produce artifacts (e.g., missing curves). To this end, we propose \modelName{}, a novel neural-based edge representation that supports the prediction of structured edges in the form of PWL curves, making the following parametric curve extraction easy and efficient.

% \vspace{1mm}
\noindent \textbf{Neural Representation Learning.}
Many well-known 3D representations, such as voxels~\cite{choy20163d}, point clouds~\cite{fan2017point}, meshes~\cite{wang2018pixel2mesh,groueix2018papier}, occupancy fields~\cite{mescheder2019occupancy}, and signed distance functions~\cite{park2019deepsdf,chen2019learning}, have been introduced into deep learning to resolve the problem of 3D reconstruction and achieve impressive results. Liao et al.~\cite{liao2018deep} propose a differentiable learning architecture to represent the classical Marching Cubes~\cite{lorensen1987marching} algorithm for shape reconstruction. Chen et al.~\cite{chen2021neural,chen2022neural} further extend the algorithms of Marching Cubes~\cite{lorensen1987marching} and Dual Contouring~\cite{ju2002dual} with data-driven approaches. Specifically, they implicitly represent triangle meshes in compact per-cube parameterizations that are compatible with neural learning. With the learned implicit field, a high-quality triangle mesh with sharp features can be directly extracted. Inspired by the tendency, we introduce the traditional PWL curve representation into deep learning for parametric curve estimation. To make it easy to learn PWL curves, we propose \modelName{} to parameterize PWL curves in a uniform volumetric field and apply advanced neural networks for learning. % Our \modelName{} outperforms existing methods for the extraction of parametric curves from point clouds. 

