\section{Experiment}
\label{sec:exp}

In this section, we evaluate the effectiveness of our method with learning-based methods~\cite{yu2018ec,wang2020pie,matveev2022def} and a traditional method~\cite{merigot2010voronoi}. Then we test the robustness by adding noise or changing the sampling density of a point cloud, and provide an ablation study to explore the effects of certain choices in our method.

% ------------ figure ----------------
\begin{figure*}[!t]
	\centering
	\includegraphics[width=1.\linewidth]{figures/images/comp_edgepoints_v2_crop.jpg}
	\caption{Qualitative comparisons on the edge point estimation with VCM~\cite{merigot2010voronoi}, EC-NET~\cite{yu2018ec} and PIE-NET~\cite{wang2020pie}.}
	\label{fig:comp_edgepoints_vis}
	\vspace{-1mm}
\end{figure*}
% ------------ figure ----------------

% ------------ figure ----------------
\begin{figure*}[!t]
	\centering
	\includegraphics[width=1.\linewidth]{figures/images/comp_param_visualize_v1_crop.png}
	\caption{Qualitative comparisons with DEF on parametric curve extraction.}
	\label{fig:comp_paramcurve}
	\vspace{-5mm}
\end{figure*}
% ------------ figure ----------------

\subsection{Experiment Setup}
\noindent \textbf{Dataset.} We train and test our model on the ABC dataset~\cite{koch2019abc}, which has one million CAD models in total created by human users.  We use the first chunk for our experiments, which is already enough for use. Following~\cite{wang2020pie,matveev2022def,chen2022neural}, we consider \textit{sharp} curves in CAD models. However, some models need to be filtered due to data missing, shape repetition, or lack of sharp edges. The final filtered dataset contains 2364 models and is randomly split into a train set (80\%) and a test set (20\%). We provide additional details on dataset cleaning and data preparation in the supplemental.
% Due to the high repetition rate of shapes in~\cite{koch2019abc},  No repetition rate found 

\vspace{1mm}
\noindent \textbf{Implementation.} In our experiments, we set the resolution of NerVE grid at $32^3$. The effects of using higher resolutions are discussed in our ablation study (see Sec.~\ref{exp:ablation}). All input point clouds are normalized into $[-1,1]^3$. We train our networks on a NVIDIA RTX 3090 for 60 epochs. The Adam optimizer is used with an initial learning rate of 5e-4. We follow NDC~\cite{chen2022neural} and set the batch size as 1. The input point number varies for different shapes. On average, it is around 22,000. For more details like layer specifications and time cost, we provide them in the supplemental.
%It turns out training can converge at the end with batch size 1 and it makes it flexible to use input points of varying numbers and masks of varying shapes.

\vspace{1mm}
\noindent \textbf{Metrics.} We quantitatively evaluate the prediction of our networks using average recall \RecallCube{} and precision \PrecisionCube{} for edge occupancy, average correct rate \CorrectFace{} for edge orientations and average $L_2$ distance \DistancePoint{} for edge point positions. To define \CorrectFace{}, in the evaluation of edge orientation, a cube is regarded correct if all of its predictions of three faces are identical to GT, otherwise the cube is wrong. \CorrectFace{} is defined as the ratio of correct cubes in all considered cubes in an input. Note that the metrics are calculated under the cube masks defined in Sec.~\ref{sec:training}. 

To compare the quality of both PWL curves and parametric curves with other methods, we adopt the typical Chamfer distance (CD) and average Hausdorff distance (HD), %\xiangyu{average version HD means divide by 2} 
which assess the similarity between two point sets. Assume $X,Y$ are two finite point sets, CD and HD can be calculated as:
\begin{align*}
    \text{CD} &= \frac{1}{|X|} \sum_{p \in X} \min_{q \in Y} \lVert p-q \rVert_2^2 + \frac{1}{|Y|} \sum_{q \in Y} \min_{p \in X} \lVert q-p \rVert_2^2,
\end{align*}
\begin{align*}
    \text{HD} &= \frac{1}{2} (\max_{p \in X} \min_{q \in Y} \lVert p-q \rVert_2 + \max_{q \in Y} \min_{p \in X} \lVert p-q \rVert_2).
\end{align*}

\subsection{Comparisons}
\noindent \textbf{Edge Estimation.} We evaluate our predicted PWL curves on our test set by comparing them with the baseline methods: VCM~\cite{merigot2010voronoi}, EC-NET~\cite{yu2018ec} and PIE-NET~\cite{wang2020pie}.
%\xiangyu{In fact not SOTA in terms of edge estimation} 
To measure the error between predicted edge points and the ground-truth, we convert PWL curves into point sets by sampling the midpoints of all edges. Numerical results are reported in Table~\ref{tab:comp_edgepoints} and visual comparisons are shown in Fig.~\ref{fig:comp_edgepoints_vis}. We provide the settings of baseline methods in the supplemental.

\begin{table}[htb]
    \centering
    \resizebox{0.86\columnwidth}{!}{
    \begin{tabular}{l|c|c|c|c}
    \toprule
           & VCM~\cite{merigot2010voronoi} & EC-NET~\cite{yu2018ec} & PIE-NET~\cite{wang2020pie} & Ours \\
        \hline
        CD$\downarrow$ & 0.0226 & 0.0037 & 0.0074 & \textbf{0.0012} \\
        \hline
        HD$\downarrow$ & 0.1941 & 0.1284 & 0.1318 & \textbf{0.0714} \\
    \bottomrule
    \end{tabular}}
    \caption{Quantitative comparisons on edge points estimation.}
    \label{tab:comp_edgepoints}
    \vspace{-4mm}
\end{table}
\vspace{1mm}

The quantitative and qualitative results show that our method significantly outperforms the baselines, where we produce accurate and uniformly distributed points visually. VCM and EC-NET are prone to generate redundant points around the ground-truth edges, thus producing higher CD and HD values. PIE-NET has a thinner band of outputs, but it tends to miss some edge points. 
% \dong{As our method adopts masks and only chooses surface cubes for calculation to reduce consumption, it is more efficient than other approaches. The average inference times of VCM, EC-Net, PIE-Net, and Ours are 2.06, 0.84, 0.52, 0.15 seconds, respectively. For post-processing, our method takes 0.02s on average, which is more efficient than the post-processing of PIE-Net (3.01s). }


\begin{figure*}[!t]
	\centering
	\includegraphics[width=1.\linewidth]{figures/images/stress_test_v3_crop.jpg}
	\caption{Qualitative comparisons with EC-NET~\cite{yu2018ec} and PIE-NET\cite{wang2020pie} on noisy input points or input points with varying sampling density. $l=2/32$ is the edge length of a cube in the grid.
	}
	\label{fig:stress_test}
	\vspace{-5mm}
\end{figure*}

% Xiangyu: If space is enough
% VCM~\cite{merigot2010voronoi} is a non-learning method for sharp point detection. EC-NET~\cite{yu2018ec} predicts point positions on sharp edges for shape reconstruction, using deep networks. PIE-NET~\cite{wang2020pie}, also a learning based method, detect sharp curves in its first stage for next parametric inference.

\vspace{1mm}
\noindent \textbf{Parametric Curves.} With the predicted PWL curves as input, our method fit them to generate parametric curves.
We compare our parametric curves with the state-of-the-art method DEF~\cite{matveev2022def} both quantitatively and qualitatively on the DEF-Sim~\cite{matveev2022def} dataset, which has 68 carefully selected shapes from ABC dataset. Since the official code of DEF is not available, we use the input data in DEF-Sim and their results of parametric curves provided by DEF's authors. We directly use the reported results of PIE-Net and DEF from DEF's paper~\cite{matveev2022def} for quantitative comparisons (shown in Table~\ref{tab:comp_paramcurve}). Since PIE-NET does not provide the official implementation for its parametric curve inference, we only compare with DEF qualitatively in Fig.~\ref{fig:comp_paramcurve}.

\begin{table}[htb]
    \centering
    \resizebox{0.845\columnwidth}{!}{
    \begin{tabular}{l|c|c|c|c}
    \toprule
          & PIE-NET~\cite{wang2020pie} & DEF~\cite{matveev2022def} & Ours($32^3$) & Ours($64^3$) \\
        \hline
        CD$\downarrow$  & 0.97 & 0.04  & 0.008 & \textbf{0.005} \\
        \hline
        HD$\downarrow$   & 2.19 & 0.55  & 0.224 & \textbf{0.184}\\
    \bottomrule
    \end{tabular}}
    \caption{Quantitative comparisons on parametric curve extraction.}
    \label{tab:comp_paramcurve}
    \vspace{-4mm}
\end{table}
\vspace{1mm}


% \footnote{DEF does not publish their codes by CVPR23 deadline. The author of DEF provides us with input data in DEF-Sim and their results of parametric curves. Values of PIE-NET and DEF in Table~\ref{tab:comp_paramcurve} are from Table 5 in DEF~\cite{matveev2022def}. PIE-NET does not provide codes for its parametric curve inference, so we only compare with DEF in Figure~\ref{fig:comp_paramcurve}. },

The results in Table~\ref{tab:comp_paramcurve} show that our method presents significantly better performance than DEF and PIE-NET, where the CD and HD errors of PIE-NET primarily arise from missing curve instances, which usually happen when keypoints (e.g., corner points or edge points) are wrongly predicted or simply missed. As shown in Fig.~\ref{fig:comp_paramcurve}, DEF performs well around sharp corners but struggles in processing structures with circle curves. In contrast, our method can handle closed curves and sharp corners consistently and produce convincing parametric curves in visual. 

\subsection{Stress Tests}
We investigate the robustness of our method using point clouds with varying noise or sampled point numbers. We train and test networks on our dataset with augmentation by adding Gaussian noise and random resampling. 

\vspace{1mm}
\noindent \textbf{NerVE Prediction.} Table~\ref{tab:stresstest_network} shows the effects of noise intensity and sampled point number on the prediction of our NerVE representation. We observe that even with a large noise intensity ($\sigma=l/2$, where $l$ is the edge length of a cube in the grid) or much fewer sampled points (4,096), our method still can produce reasonable results of PWL curves with low-level CD and HD errors.
% Xiangyu: hard to explain
% except the case of 4096 resampling points. In this undersampling case, our network could not obtain good enough per cube feature from input 
\begin{table}[htb]
    \centering
    \resizebox{\columnwidth}{!}{
        \begin{tabular}{c|c|c|c|c|c|c|c}
        \toprule
            & & \RecallCube{}$\uparrow$ & \PrecisionCube{}$\uparrow$ & \CorrectFace{}$\uparrow$ & \DistancePoint{}$\downarrow$ & CD$\downarrow$ & HD$\downarrow$ \\
            \hline
            Clean & & \textbf{0.965} & \textbf{0.965} & \textbf{0.940} & \textbf{0.0030} & \textbf{0.0012} & \textbf{0.071} \\ 
            \hline
            \multirow{2}{2em}{Noise} & $\sigma=l/4$ & 0.919 & 0.937 & 0.887 & 0.0053 & 0.0019 & 0.097 \\
            & $\sigma=l/2$ & 0.892 & 0.923 & 0.854 & 0.0079 & 0.0027 & 0.110 \\
            \hline
            \multirow{3}{3em}{\#Sample Points} & 16384 & 0.952 & 0.955 & 0.919 & 0.0044 & 0.0019 & 0.095 \\
            & 8192 & 0.938 & 0.951 & 0.902 & 0.0048 & 0.0028 & 0.112 \\
            & 4096 & 0.923 & 0.941 & 0.876 & 0.0053 & 0.0057 & 0.146 \\
        \bottomrule
        \end{tabular}
    }
    \caption{Influence of input points with varying noise or sampled point numbers. $l=2/32$ is the edge length of a cube in the grid.}
    \label{tab:stresstest_network}
    \vspace{-4mm}
\end{table}
\vspace{1mm}

\noindent \textbf{Edge Estimation.} We also compare with VCM~\cite{merigot2010voronoi}, EC-NET~\cite{yu2018ec} and PIE-NET\cite{wang2020pie} on noisy or resampled inputs. Table~\ref{tab:stresstest_edgepoints} shows that our method achieves the best numerical performance. Fig.~\ref{fig:stress_test} further demonstrates that our method outperforms baseline methods and presents robustness against noisy or resampled inputs.

\begin{table}[htb]
    \centering
    \resizebox{\columnwidth}{!}{
        \begin{tabular}{c|c|c|c|c|c}
        \toprule
            & & VCM~\cite{merigot2010voronoi} & EC-NET~\cite{yu2018ec} & PIE-NET\cite{wang2020pie} & Ours \\
            \hline
            Clean & & 0.0226 & 0.0037 & 0.0074 & \textbf{0.0012} \\ 
            \hline
            \multirow{2}{2em}{Noise} & $\sigma=l/4$ & 0.0253 & 0.0055 & 0.0323 & \textbf{0.0019} \\
            & $\sigma=l/2$ & 0.0250 & 0.0132 & 0.0420 & \textbf{0.0027} \\
            \hline
            \multirow{3}{3em}{\#Sample Points} & 16384 & 0.0211 & 0.0039 & 0.0060 & \textbf{0.0019} \\
            & 8192 & 0.0214 & 0.0060 & 0.0145 & \textbf{0.0028} \\
            & 4096 & 0.0223 & 0.0130 & 0.0216 & \textbf{0.0057} \\
        \bottomrule
        \end{tabular}
    }
    \caption{Edge estimation errors (CD, the smaller the better) of four methods on noisy or resampled inputs. $l=2/32$ is the edge length of a cube in grid. Results of HD are provided in the supplemental. }
    \label{tab:stresstest_edgepoints}
    \vspace{-4mm}
\end{table}

\subsection{Ablation Study}
\label{exp:ablation}
\noindent \textbf{Resolution.} Table~\ref{tab:ablation_reso} shows the performance of our method under different NerVE grid resolution, where we observe that using resolution $32^3$ achieves slightly better performance on occupancy prediction of edge cubes. As a binary classification problem, data imbalance of edge occupancy is aggravated as resolution increases since the number of non-edge points grows faster than edge points number. Therefore, using resolution $64^3$ meets a more challenging classification problem, producing slightly worse \RecallCube{} and \PrecisionCube{} than using resolution $32^3$. Nevertheless, we notice that using resolution $64^3$ can achieve better CD and HD. Meanwhile, the parametric curves under resolution $64^3$ present better performance as shown in Table~\ref{tab:comp_paramcurve}. Fig.~\ref{fig:ablation_reso} shows the visualizations and provides some insights to explain the phenomenon.
% We also notice that using resolution $64^3$ predicts better point position (\DistancePoint{}), and this is natural since higher resolution has smaller cubes delivering more precise per-point position.

\begin{figure}[ht]
	\centering
	\includegraphics[width=1.\linewidth]{figures/images/ablation_reso_v3_crop.png}
	\caption{Qualitative comparisons on the resolution of $32^3$ and $64^3$ for \modelName{}. The resolution $64^3$ can represent complicated curves but also have imperfections in simple instances. }
	\label{fig:ablation_reso}
	\vspace{-3mm}
\end{figure}

\begin{figure}[ht]
	\centering
	\includegraphics[width=1.\linewidth]{figures/images/ablation_reso_v3_crop.png}
	\caption{Qualitative comparisons on the resolution of $32^3$ and $64^3$ for \modelName{}. The resolution $64^3$ can represent complicated curves but also have imperfections in simple instances. }
	\label{fig:ablation_reso}
	\vspace{-3mm}
\end{figure}

\vspace{1mm}

% Moreover, we also observe that using resolution $64^3$ produces larger CD error in PWL curve prediction. Meanwhile, the parametric curves under resolution $64^3$ present better results as shown in Table~\ref{tab:comp_paramcurve}. Fig.~\ref{fig:ablation_reso} shows the visualizations and provides some insight to explain the contradiction. 

% For simple shapes, e.g., the left side of Fig.~\ref{fig:ablation_reso}, using higher resolution (e.g., $64^3$) could disconnect at several positions on curves (although it can be fixed by simple post- processing), which makes the CD error of its PWL curves larger.
% However, for parametric curve extraction, using resolution $32^3$ suffers from representing complicated shapes (right side in Fig.~\ref{fig:ablation_reso}) where using resolution $64^3$ performs better. 
% In summary, this is a trade-off between resolution and representation difficulty in NerVE, where a higher resolution usually brings more details to process complicated shapes, but also indicates higher learning difficulty. In our experiments, we observe that using resolution $32^3$ generally yields better results for normal shapes in the dataset.

For simple shapes, e.g., the left side of Fig.~\ref{fig:ablation_reso}, using a higher resolution (e.g., $64^3$) could disconnect at several positions on curves, which will degrade the edge occupancy accuracy of its PWL curves, but these artifacts (e.g., disconnection) can be well addressed by a simple post-processing in the following parametric curve extraction. On the contrary, using resolution $32^3$ suffers from representing complicated shapes (right side in Fig.~\ref{fig:ablation_reso}) while using resolution $64^3$ performs better and brings more geometric details. More experiments and analysis are provided in the supplemental. %\dongc{I put the results of $128^3$ and other ablation study of our rebuttal in the supplemental.}

% \dong{In our experiments, we find the ABC dataset~\cite{koch2019abc} is biased toward simpler shapes. Thus, we use resolution $32^3$ for default, which already performs well for our \modelName{}.}

\noindent \textbf{Cube Point Choice.} As stated in Sec.~\ref{sec:method}, edge point position in a cube is defined as the midpoint of the truncated curve. Another option is to use a point which minimizes a quadratic error function (QEF), similar to the point definition in Dual Contouring~\cite{ju2002dual}. We provide the details in our supplemental. We validate our choice by restoring curves from ground-truth NerVE with two different definitions of point position. As shown in Table~\ref{tab:ablation_cubepoint_choice}, current definition of the point position clearly has better performance on curve restoration. 

\begin{table}[htb]
    \centering
    \resizebox{0.5\columnwidth}{!}{
        \begin{tabular}{l|c|c}
        \toprule
             & DC QEF & Our Choice \\
            \hline
            CD $\downarrow$ & 0.002799 & \textbf{0.000136} \\
            \hline
            HD $\downarrow$ & 0.052306 & \textbf{0.024263} \\
        \bottomrule
        \end{tabular}
    }
    \caption{Ablation study of edge point estimation on ground-truth NerVE. We measure the errors of generated PWL curves using two types of point positions in cubes, i.e, DC QEF~\cite{ju2002dual} and ours. }
    \label{tab:ablation_cubepoint_choice}
    \vspace{-4mm}
\end{table}

\vspace{1mm}
\section{Generalization, Limitation and Future Work}
The Matcha framework exhibits a high degree of generalizability thanks to the commonsense knowledge inside LLMs.
Without LLMs, a control algorithm, e.g. one trained with reinforcement learning \cite{Li23InternallyRewarded, Singh20COGConnecting}, may require massive datasets/interactions to learn
the common sense \cite{Singh20COGConnecting} of collaborating different modalities, yet being less efficient and generalizable.

However, interpreting the real world with language can be limited to the complexity of the task and the environment dynamics.
For example, advanced reasoning techniques such as decomposing may be required to deal with a complicated task,
where the task is decomposed into several sub-tasks to tackle separately. 
This automatic operation highlights the flexibility of LLMs but also poses challenges to the static language expression of a complex world
--- The vision-to-language module should be called multiple times with flexible queries.
This brings the requirement of vision-enabled LLMs \cite{Zhu23MiniGPT4Enhancing, Brohan23RT2Visionlanguageaction}, 
built on which the reasoning can be malleable. But multimodal LLMs are yet less controllable and accurate in terms of describing the scene
compared with a templated module.

Despite current limitations, multimodal LLMs gain increasing attention due to their great potential and flexibility.
Future work will explore the multimodal models \cite{Tong22VideoMAEMasked, Brohan23RT2Visionlanguageaction} to leverage unified features.

% 
\subsection{Correlation distribution}

\paragraph{Generalization across $\alpha$'s.} In \Cref{fig:jointplot} left, we compare the linear datamodeling scores (LDS) evaluated on $\alpha=0.5$ sub-sampled training sets to those evaluated on $\alpha=0.75$.
(The numbers are overall lower as these are evaluated on data where only one model was trained on each subset,instead of averaging over 5 models; hence, there is more noise in the data.) As we observe, the LDS scores on different $\alpha$'s are highly correlated, suggesting that \trak scores computed on a single $\alpha$ generalize well.

\paragraph{LDS correlation between \trak and datamodels.} In \Cref{fig:jointplot} right, we compare the LDS correlations of datamodels to that of \trak and find that they are correlated across examples; in general, \trak also performs better on examples on which datamodels perform better.

\begin{figure}[!htbp]
    \centering
    \includegraphics[width=0.45\linewidth]{figures/cifar2_off_dist.pdf}
    \includegraphics[width=0.45\linewidth]{figures/cifar2_dm_vs_trak.pdf}
\caption{{\bf (Left)} The LDS of \cifartwo \trak scores computed with $\alpha=0.5$ models then evaluated on either models trained with either $\alpha=0.5$ or $\alpha=0.75$. Each point corresponds to a validation example. {\bf (Right)} The LDS of \cifartwo datamodel scores compared with that of \trak. Here, the LDS is measured on two different estimators.}
\label{fig:jointplot}
\end{figure}



\clearpage
\subsection{Table for LDS evaluation}

\begin{table}[h]
    \centering
    \begin{tabular}{llrrrrrrrrr}
        \toprule
        Dataset & & TRAK & TracIn \citep{pruthi2020estimating} & Infl. \citep{koh2017understanding} & Datamodels \citep{ilyas2022datamodels} \\
        \midrule
        \cifartwo & \# models & 5 & 100 & - & 1,000 \\
        & Time (min.) & 3 & 100 & - & 500  \\
        & LDS & {\bf 0.203(3)} & 0.056(2) & - & 0.162(5)  \\
        \midrule
        \cifarten & \# models & 20 & 20 & 1 & 5,000 \\
        & Time (min.) & 20 & 60 & 20,000 & 2,500 \\
        & LDS & {\bf 0.271(4)} & 0.056(7) & 0.037(13) & 0.199(4) \\
        \midrule
        \qnli & \# models & 10 & 1 &  1 & 20,000 \\
        & Time (min.) & 640 & 284 & 18,000 & 176,000 \\
        & LDS & {\bf 0.416(10)} & 0.077(29) & 0.114(43) & 0.344(32) \\
        \midrule
        ImageNet & \# models & 100 & 1 &  20 & 30,000 \\
        & Time (min.) & 2920 & 76 & $>$100,000 &   525,000  \\
        & LDS & {\bf 0.188(6)} & 0.008(6) &   0.037(6) & 0.1445(6) \\
        \bottomrule
        \end{tabular}
        \caption{{\em Comparison of different data attribution methods.} We quantify various data attribution methods in terms of both their {\em predictiveness}---as
        measured by the linear datamodeling score---as well as their {\em
        computational efficiency}---as measured by either the total computation
        time (wall-time measured in minutes on a single A100 GPU; see
        \Cref{app:wall_time} for details) or the number of trained models used
        to compute the attribution scores. The errors indicate 95\%
        bootstrap confidence intervals.
        Sampling-based methods (datamodels and
        empirical influences) can outperform \trak when allowed to use more
        computation, but this leads to a significant
        increase in computational cost.
        }
        \label{tab:all_best}
\end{table}





\clearpage
\subsection{\trak examples}
\label{app:more_examples}
We display more examples identified with \trak scores in \Cref{fig:imagenet_nns_extra} (ImageNet), \Cref{tab:qnli_more} (\qnli), and \Cref{fig:clip_examples_extra} (\clip on \mscoco).

\begin{figure}[!b]
    \centering
    \includegraphics[width=.9\linewidth,trim={0 0 0 0},clip]{figures/imagenet_nns_extra.pdf}
\caption{
     {\em \trak attributions for ResNets trained on ImageNet.}
    We display random test examples and their corresponding
    most helpful (highest-scoring) and most detracting (lowest-scoring)
    training examples according to \trak.
}
\label{fig:imagenet_nns_extra}
\end{figure}


\clearpage
\begin{figure}
    \centering
    \begin{tabular}{p{0.33\textwidth}p{0.30\textwidth}p{0.30\textwidth}}
    \toprule
    \textbf{Example} & \textbf{Highest \trak score (+)} & \textbf{Lowest \trak score (-)} \\
    \midrule
    \scriptsize {\bf Q:} What was a major success, especially in rebuilding Warsaw? {\bf A:} Like many cities in Central and Eastern Europe, infrastructure in Warsaw suffered considerably during its time as an Eastern Bloc economy – though it is worth mentioning that the initial Three-Year Plan to rebuild Poland (especially Warsaw) was a major success, but what followed was very much the opposite. {\bf (Yes)} & \scriptsize {\bf Q:} In 1998, the deal was renewed for what amount over four years? {\bf A:} Television money had also become much more important; the Football League received £6.3 million for a two-year agreement in 1986, but when that deal was renewed in 1988, the price rose to £44 million over four years. {\bf (Yes)} & \scriptsize {\bf Q:} Who was a controversial figure due to a corked-bat incident? {\bf A:} Already a controversial figure in the clubhouse after his corked-bat incident, Sammy's actions alienated much of his once strong fan base as well as the few teammates still on good terms with him, (many teammates grew tired of Sosa playing loud salsa music in the locker room) and possibly tarnished his place in Cubs' lore for years to come. {\bf (No)} \\
    \midrule
    \scriptsize {\bf Q:} What is the name associated with the eight areas that make up a part of southern California? {\bf A:} Southern California consists of one Combined Statistical Area, eight Metropolitan Statistical Areas, one international metropolitan area, and multiple metropolitan divisions. {\bf (Yes)} & \scriptsize {\bf Q:} Was was the name given to the Alsace provincinal court? {\bf A:} The province had a single provincial court (Landgericht) and a central administration with its seat at Hagenau. {\bf (Yes)} & \scriptsize {\bf Q:} What do six of the questions asses? {\bf A:} For each question on the scale that measures homosexuality there is a corresponding question that measures heterosexuality giving six matching pairs of questions. {\bf (No)} \\
    \midrule
    \scriptsize {\bf Q:} What words are inscribed on the mace of parliament? {\bf A:} The words There shall be a Scottish Parliament, which are the first words of the Scotland Act, are inscribed around the head of the mace, which has a formal ceremonial role in the meetings of Parliament, reinforcing the authority of the Parliament in its ability to make laws. {\bf (No)} & \scriptsize {\bf Q:} Whose name is on the gate-house fronting School Yard? {\bf A:} His name is borne by the big gate-house in the west range of the cloisters, fronting School Yard, perhaps the most famous image of the school. {\bf (No)} & \scriptsize {\bf Q:} What kind of signs were removed form club Barcelona? {\bf A:} All signs of regional nationalism, including language, flag and other signs of separatism were banned throughout Spain. {\bf (Yes)} \\
    \midrule
    \scriptsize {\bf Q:} What was the percentage of a female householder with no husband present? {\bf A:} There were 158,349 households, of which 68,511 (43.3\%) had children under the age of 18 living in them, 69,284 (43.8\%) were opposite-sex married couples living together, 30,547 (19.3\%) had a female householder with no husband present, 11,698 (7.4\%) had a male householder with no wife present. {\bf (Yes)} & \scriptsize {\bf Q:} What percent of household have children under 18? {\bf A:} There were 46,917 households, out of which 7,835 (16.7\%) had children under the age of 18 living in them, 13,092 (27.9\%) were opposite-sex married couples living together, 3,510 (7.5\%) had a female householder with no husband present, 1,327 (2.8\%) had a male householder with no wife present. {\bf (Yes)} & \scriptsize {\bf Q:} Roughly how many same-sex couples were there? {\bf A:} There were 46,917 households, out of which 7,835 (16.7\%) had children under the age of 18 living in them, 13,092 (27.9\%) were opposite-sex married couples living together, 3,510 (7.5\%) had a female householder with no husband present, 1,327 (2.8\%) had a male householder with no wife present. {\bf (No)} \\
        \midrule
        \scriptsize {\bf Q:} What did Warsz own? {\bf A:} In actuality, Warsz was a 12th/13th-century nobleman who owned a village located at the modern-day site of Mariensztat neighbourhood. {\bf (Yes)} & \scriptsize {\bf Q:} What company did Ray Kroc own? {\bf A:} It was founded in 1986 through the donations of Joan B. Kroc, the widow of McDonald's owner Ray Kroc. {\bf (Yes)} & \scriptsize {\bf Q:} What did Cerberus guard? {\bf A:} In Norse mythology, a bloody, four-eyed dog called Garmr guards Helheim. {\bf (No)} \\
        \midrule
        \scriptsize {\bf Q:} What words are inscribed on the mace of parliament? {\bf A:} The words There shall be a Scottish Parliament, which are the first words of the Scotland Act, are inscribed around the head of the mace, which has a formal ceremonial role in the meetings of Parliament, reinforcing the authority of the Parliament in its ability to make laws. {\bf (No)} & \scriptsize {\bf Q:} Whose name is on the gate-house fronting School Yard? {\bf A:} His name is borne by the big gate-house in the west range of the cloisters, fronting School Yard, perhaps the most famous image of the school. {\bf (No)} & \scriptsize {\bf Q:} What kind of signs were removed form club Barcelona? {\bf A:} All signs of regional nationalism, including language, flag and other signs of separatism were banned throughout Spain. {\bf (Yes)} \\
    \bottomrule
\end{tabular}
\caption{{\em Top \trak attributions for \qnli examples.} Yes/No indicates the label (entailment vs. no entailment).}
\label{tab:qnli_more}
\end{figure}

\clearpage
\begin{figure}[!t]
    \centering
    \includegraphics[width=\linewidth,trim={0 0 0 0},clip]{figures/CLIP_examples/clip_examples_0.pdf}
    \includegraphics[width=\linewidth,trim={0 0 0 0},clip]{figures/CLIP_examples/clip_examples_1.pdf}
    \includegraphics[width=\linewidth,trim={0 0 0 0},clip]{figures/CLIP_examples/clip_examples_2.pdf}
\caption{
     {\em Top attributions for \clip models trained on \mscoco.}
    We display random test examples and their corresponding
    most helpful (highest-scoring) and most detracting (lowest-scoring)
    training examples according to \trak, \clip similarity distance, and \tracin.
    }
\label{fig:clip_examples_extra}
\end{figure}






\clearpage
\subsection{\modeldiff with \trak}
\Cref{fig:modeldiff} shows how we apply \trak to dramatically accelerate the
\modeldiff algorithm.
\begin{figure}[h]
    \centering
    \includegraphics[width=\linewidth,trim={0 0 0 0},clip]{figures/modeldiff_pipeline.pdf}
\caption{
{\em Accelerating learning algorithm comparisons with \trak.}
The \modeldiff framework from \citep{shah2022modeldiff} uses datamodel
representations to surface features that distinguish two learning algorithms. In
the case study here, we compare models trained on the \textsc{Living17} dataset {\em with} and {\em without} data
augmentation. Applying \modeldiff involves three stages: (1) computing datamodel
representations; (2) applying the \modeldiff algorithm to extract {\em
distinguishing subpopulations} of inputs on which two model classes behave
differently; (3) counterfactually testing the inferred feature associated
with the subpopulation. \citet{shah2022modeldiff} find that models trained with
data augmentation latch onto the presence of spider webs as a spurious
correlation to predict the class spider. Here, we recover their result by using
\trak scores instead of datamodel scores in step (1); doing so reduces the
computational cost of \modeldiff by 100x.
}
\label{fig:modeldiff}
\end{figure}
% \subsection{More Results} For more instances of our extracted parametric curves, refer to Fig.~\ref{fig:more_results}.