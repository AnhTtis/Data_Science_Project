\begin{abstract}
Extracting parametric edge curves from point clouds is a fundamental problem in 3D vision and geometry processing.
Existing approaches mainly rely on keypoint detection, a challenging procedure that tends to generate noisy output, making the subsequent edge extraction error-prone.
To address this issue, we propose to directly detect structured edges to circumvent the limitations of the previous point-wise methods.
We achieve this goal by presenting \modelName{}, a novel neural volumetric edge representation that can be easily learned through a volumetric learning framework.
\modelName{} can be seamlessly converted to a versatile piece-wise linear (PWL) curve representation, enabling a unified strategy for learning all types of free-form curves. Furthermore, as \modelName{} encodes rich structural information, we show that edge extraction based on \modelName{} can be reduced to a simple graph search problem.
After converting \modelName{} to the PWL representation, parametric curves can be obtained via off-the-shelf spline fitting algorithms.
We evaluate our method on the challenging ABC dataset~\cite{koch2019abc}.
We show that a simple network based on \modelName{} can already outperform the previous state-of-the-art methods by a great margin. Project page: \href{https://dongdu3.github.io/projects/2023/NerVE/}{https://dongdu3.github.io/projects/2023/NerVE/.}
\end{abstract}


% \begin{abstract}
%   We propose a simple and effective curve representation for extracting parametric curves from 3D point cloud. Existing learning based methods commonly estimate endpoints and their connections after neural detection of edge points in point cloud, where erroneous predictions or improper parameters could result in missing curves\yinyu{previous sentence is vague, every method could have these problems}. Our curve representation can be directly predicted by network and easily converted to piece-wise linear (PWL) curve. We will show that finding endpoints and their connection can be reduced to simple search on PWL curve graph. Another advantage is that we do not constrain the curve type since curves are represented in unified form. Extensive experiments are done to evaluate our method on ABC dataset(large scale CAD model dataset). Evaluation results evidently show the superiority of our method against others, in both visual and numerical aspects. 
% \end{abstract}

