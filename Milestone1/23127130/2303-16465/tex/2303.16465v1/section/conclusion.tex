\section{Conclusions}
\label{sec:conclusion}
In this paper, we propose NerVE, a neural volumetric edge representation, for the extraction of parametric curves from a point cloud. This edge structure representation is fully compatible with the volumetric learning framework and can be easily converted to explicit PWL curves, which greatly reduces the complexity of following parametric curve fitting. The quantitative and qualitative results in the experiments evidently show the superiority of our method. 
% More extracted parametric curves can be found in Figure~\ref{fig:more_results}.

\noindent \textbf{Limitations and Future Works.} One limitation is that our method may produce unexpected junction points. As illustrated in Figure~\ref{fig:limitation}, when two curves are so close and pass through the same cube, it will produce a junction point since we only predict one point in one cube by definition. This issue is essentially the cost of discretization, but it is insignificant in a statistical sense. In fact, there are only 1.62\% cubes with described junction points in all edge-occupied cubes in our dataset at resolution $32^3$, and the number decreases to 0.71\% as the resolution increases from $32^3$ to $64^3$. Another limitation is the vertex in one cube can only have at most 6 connections, due to the definition of \modelName{} and the structure of cube grids. In the future, we would like to devise a better representation that can completely solve these problems.

% \dong{Another limitation is our \modelName{} only allows for the detection of edges that have less than 6 connections.}
Designing a differentiable architecture for the extraction of PWL curves or parametric curves is also an interesting future work. Furthermore, notice \modelName{} can essentially represent general edges, we plan to test its ability boundary. For example, we wish to apply it not only to CAD models but also to some other shapes that have thin structures, such as the neural nerves of humans. 
%since NerVE can essentially represent general edges, we hope to validate its effectiveness on other types of edge data, like curve skeleton, thin structure shapes or the neural nerves of human. However, these types of edges do not have large-scale dataset as far as we know. Hence, we want to develop a differentiable version of NerVE and enable unsupervised learning on general edge data.

