%  LaTeX support: latex@mdpi.com
%  For support, please attach all files needed for compiling as well as the log file, and specify your operating system, LaTeX version, and LaTeX editor.

%=================================================================
\documentclass[symmetry,review,accept,pdftex,moreauthors]{Definitions/mdpi}
% For posting an early version of this manuscript as a preprint, you may use "preprints" as the journal and change "submit" to "accept". The document class line would be, e.g., \documentclass[preprints,article,accept,moreauthors,pdftex]{mdpi}. This is especially recommended for submission to arXiv, where line numbers should be removed before posting. For preprints.org, the editorial staff will make this change immediately prior to posting.

\graphicspath{{figures/}} % Directory in which figures are stored

%--------------------
% Class Options:
%--------------------
%----------
% journal
%----------
% Choose between the following MDPI journals:
% acoustics, actuators, addictions, admsci, adolescents, aerospace, agriculture, agriengineering, agronomy, ai, algorithms, allergies, alloys, analytica, animals, antibiotics, antibodies, antioxidants, applbiosci, appliedchem, appliedmath, applmech, applmicrobiol, applnano, applsci, aquacj, architecture, arts, asc, asi, astronomy, atmosphere, atoms, audiolres, automation, axioms, bacteria, batteries, bdcc, behavsci, beverages, biochem, bioengineering, biologics, biology, biomass, biomechanics, biomed, biomedicines, biomedinformatics, biomimetics, biomolecules, biophysica, biosensors, biotech, birds, bloods, blsf, brainsci, breath, buildings, businesses, cancers, carbon, cardiogenetics, catalysts, cells, ceramics, challenges, chemengineering, chemistry, chemosensors, chemproc, children, chips, cimb, civileng, cleantechnol, climate, clinpract, clockssleep, cmd, coasts, coatings, colloids, colorants, commodities, compounds, computation, computers, condensedmatter, conservation, constrmater, cosmetics, covid, crops, cryptography, crystals, csmf, ctn, curroncol, currophthalmol, cyber, dairy, data, dentistry, dermato, dermatopathology, designs, diabetology, diagnostics, dietetics, digital, disabilities, diseases, diversity, dna, drones, dynamics, earth, ebj, ecologies, econometrics, economies, education, ejihpe, electricity, electrochem, electronicmat, electronics, encyclopedia, endocrines, energies, eng, engproc, ent, entomology, entropy, environments, environsciproc, epidemiologia, epigenomes, est, fermentation, fibers, fintech, fire, fishes, fluids, foods, forecasting, forensicsci, forests, foundations, fractalfract, fuels, futureinternet, futureparasites, futurepharmacol, futurephys, futuretransp, galaxies, games, gases, gastroent, gastrointestdisord, gels, genealogy, genes, geographies, geohazards, geomatics, geosciences, geotechnics, geriatrics, hazardousmatters, healthcare, hearts, hemato, heritage, highthroughput, histories, horticulturae, humanities, humans, hydrobiology, hydrogen, hydrology, hygiene, idr, ijerph, ijfs, ijgi, ijms, ijns, ijtm, ijtpp, immuno, informatics, information, infrastructures, inorganics, insects, instruments, inventions, iot, j, jal, jcdd, jcm, jcp, jcs, jdb, jeta, jfb, jfmk, jimaging, jintelligence, jlpea, jmmp, jmp, jmse, jne, jnt, jof, joitmc, jor, journalmedia, jox, jpm, jrfm, jsan, jtaer, jzbg, kidney, kidneydial, knowledge, land, languages, laws, life, liquids, literature, livers, logics, logistics, lubricants, lymphatics, machines, macromol, magnetism, magnetochemistry, make, marinedrugs, materials, materproc, mathematics, mca, measurements, medicina, medicines, medsci, membranes, merits, metabolites, metals, meteorology, methane, metrology, micro, microarrays, microbiolres, micromachines, microorganisms, microplastics, minerals, mining, modelling, molbank, molecules, mps, msf, mti, muscles, nanoenergyadv, nanomanufacturing, nanomaterials, ncrna, network, neuroglia, neurolint, neurosci, nitrogen, notspecified, nri, nursrep, nutraceuticals, nutrients, obesities, oceans, ohbm, onco, oncopathology, optics, oral, organics, organoids, osteology, oxygen, parasites, parasitologia, particles, pathogens, pathophysiology, pediatrrep, pharmaceuticals, pharmaceutics, pharmacoepidemiology, pharmacy, philosophies, photochem, photonics, phycology, physchem, physics, physiologia, plants, plasma, pollutants, polymers, polysaccharides, poultry, powders, preprints, proceedings, processes, prosthesis, proteomes, psf, psych, psychiatryint, psychoactives, publications, quantumrep, quaternary, qubs, radiation, reactions, recycling, regeneration, religions, remotesensing, reports, reprodmed, resources, rheumato, risks, robotics, ruminants, safety, sci, scipharm, seeds, sensors, separations, sexes, signals, sinusitis, skins, smartcities, sna, societies, socsci, software, soilsystems, solar, solids, sports, standards, stats, stresses, surfaces, surgeries, suschem, sustainability, symmetry, synbio, systems, taxonomy, technologies, telecom, test, textiles, thalassrep, thermo, tomography, tourismhosp, toxics, toxins, transplantology, transportation, traumacare, traumas, tropicalmed, universe, urbansci, uro, vaccines, vehicles, venereology, vetsci, vibration, viruses, vision, waste, water, wem, wevj, wind, women, world, youth, zoonoticdis

%---------
% article
%---------
% The default type of manuscript is "article", but can be replaced by:
% abstract, addendum, article, book, bookreview, briefreport, casereport, comment, commentary, communication, conferenceproceedings, correction, conferencereport, entry, expressionofconcern, extendedabstract, datadescriptor, editorial, essay, erratum, hypothesis, interestingimage, obituary, opinion, projectreport, reply, retraction, review, perspective, protocol, shortnote, studyprotocol, systematicreview, supfile, technicalnote, viewpoint, guidelines, registeredreport, tutorial
% supfile = supplementary materials

%----------
% submit
%----------
% The class option "submit" will be changed to "accept" by the Editorial Office when the paper is accepted. This will only make changes to the frontpage (e.g., the logo of the journal will get visible), the headings, and the copyright information. Furthermore, line numbering will be removed. Journal info and pagination for accepted papers will also be assigned by the Editorial Office.

%------------------
% moreauthors
%------------------
% If there is only one author the class option oneauthor should be used. Otherwise use the class option moreauthors.

%---------
% pdftex
%---------
% The option pdftex is for use with pdfLaTeX. If eps figures are used, remove the option pdftex and use LaTeX and dvi2pdf.

%=================================================================
% MDPI internal commands
\firstpage{1}
\makeatletter
\setcounter{page}{\@firstpage}
\makeatother
\pubvolume{15}
\issuenum{3}
\articlenumber{727}
\pubyear{2023}
\copyrightyear{2023}
\externaleditor{{Academic Editors: Vasilis K. Oikonomou and Silvio Pardi}}%MDPI: Please add academic editor if available.
\datereceived{31 December 2022}
\daterevised{17 February 2023}
\dateaccepted{6 March 2023}
\datepublished{15 March 2023}
%\datecorrected{} % Corrected papers include a "Corrected: XXX" date in the original paper.
%\dateretracted{} % Corrected papers include a "Retracted: XXX" date in the original paper.
\hreflink{https://doi.org/10.3390/\linebreak sym15030727} % If needed use \linebreak
%\doinum{}
%------------------------------------------------------------------
% The following line should be uncommented if the LaTeX file is uploaded to arXiv.org
%\pdfoutput=1

%=================================================================
% Add packages and commands here. The following packages are loaded in our class file: fontenc, inputenc, calc, indentfirst, fancyhdr, graphicx, epstopdf, lastpage, ifthen, lineno, float, amsmath, amssymb, setspace, enumitem, mathpazo, booktabs, titlesec, etoolbox, tabto, xcolor, colortbl, soul, multirow, microtype, tikz, totcount, changepage, attrib, upgreek, array, tabularx, pbox, ragged2e, tocloft, marginnote, marginfix, enotez, amsthm, natbib, hyperref, scrextend, url, geometry, newfloat, caption, draftwatermark, seqsplit

%=================================================================
%% Please use the following mathematics environments: Theorem, Lemma, Corollary, Proposition, Characterization, Property, Problem, Example, ExamplesandDefinitions, Hypothesis, Remark, Definition, Notation, Assumption
%% For proofs, please use the proof environment (the amsthm package is loaded by the MDPI class).

%=================================================================
% Full title of the paper (Capitalized)
\Title{Heavy-Flavour Jets in High-Energy Nuclear Collisions}
%Urgent: title change please confirm.
% MDPI internal command: Title for citation in the left column
\TitleCitation{Heavy-Flavour Jets in High-Energy Nuclear Collisions}

% Author Orchid ID: enter ID or remove command
\newcommand{\orcidauthorA}{0000-0001-6372-7352} % Add \orcidA{} behind the author's name
%\newcommand{\orcidauthorB}{0000-0000-0000-000X} % Add \orcidB{} behind the author's name

% Authors, for the paper (add full first names)
\Author{{Sa Wang} $^{1,2}$\orcidA{}, Wei Dai $^{3}$, Enke Wang $^{1,2}$, {Xin-Nian Wang} $^{4}$ and {Ben-Wei Zhang} $^{2}$*}
%MDPI: 1. Please carefully check the accuracy of ALL names and affiliations.
%2. we changed the names (Xin-Nian Wang, Ben-Wei Zhang) to the recommended format (Xinnian Wang, Benwei Zhang). please confirm


%\longauthorlist{yes}

% MDPI internal command: Authors, for metadata in PDF
\AuthorNames{Sa Wang, Wei Dai, Enke Wang, Xin-Nian Wang, Ben-Wei Zhang}

% MDPI internal command: Authors, for citation in the left column
\AuthorCitation{{Wang, S.; Dai, W.; Wang, E.; Wang, X.-N.; Zhang, B.-W.}}
%MDPI: Please check all author names carefully.

% If this is a Chicago style journal: Lastname, Firstname, Firstname Lastname, and Firstname Lastname.

% Affiliations / Addresses (Add [1] after \address if there is only one affiliation.)
\address{%
$^{1}$ \quad Guangdong Provincial Key Laboratory of Nuclear Science, Institute of Quantum Matter,\linebreak South China Normal University, Guangzhou 510006, China\\
%MDPI: We added the email addresses here according to those submitted online at susy.mdpi.com. Please confirm.
$^{2}$ \quad {Key Laboratory of Quark \& Lepton Physics (MOE)} and {Institute of Particle Physics},\linebreak Central China Normal University, Wuhan 430079, China\\
%MDPI: If the information provided presents more than one address, please separate the addresses into different affiliations; if not, please check if ``and'' can be changed to a comma

$^{3}$ \quad School of Mathematics and Physics, China University of Geosciences, Wuhan 430074, China\\
$^{4}$ \quad {Nuclear Science Division MS 70R0319, Lawrence Berkeley National Laboratory}, Berkeley,  CA 94720, USA\\}%MDPI: Please ensure the address information is  arranged from subordinate to superior

% Contact information of the corresponding author
\corres{Correspondence: bwzhang@mail.ccnu.edu.cn}

% Current address and/or shared authorship
%\firstnote{Current address: Affiliation 3.}
%\secondnote{These authors contributed equally to this work.}
% The commands \thirdnote{} till \eighthnote{} are available for further notes

%\simplesumm{} % Simple summary

%\conference{} % An extended version of a conference paper

% Abstract (Do not insert blank lines, i.e., \\)
\abstract{Reconstructed jets initiated from heavy quarks provide a powerful tool to probe the properties of the quark-gluon plasma (QGP) and to explore the mass hierarchy of jet quenching. In this article, we review the recent theoretical progresses on heavy-flavour jets in high-energy nuclear collisions at the RHIC and LHC. We focus on the yields and substructures of charm and bottom quark jets with jet quenching effect, such as the nuclear modification factors, transverse momentum imbalance, angular correlation, radial profiles, fragmentation functions, the ``dead-cone'' effect, etc.}

% Keywords
\keyword{quark--gluon plasma; jet quenching; high-energy nuclear collisions; heavy-flavour jet}

% The fields PACS, MSC, and JEL may be left empty or commented out if not applicable
%\PACS{J0101}
%\MSC{}
%\JEL{}

%%%%%%%%%%%%%%%%%%%%%%%%%%%%%%%%%%%%%%%%%%
% Only for the journal Diversity
%\LSID{\url{http://}}

%%%%%%%%%%%%%%%%%%%%%%%%%%%%%%%%%%%%%%%%%%
% Only for the journal Applied Sciences
%\featuredapplication{Authors are encouraged to provide a concise description of the specific application or a potential application of the work. This section is not mandatory.}
%%%%%%%%%%%%%%%%%%%%%%%%%%%%%%%%%%%%%%%%%%

%%%%%%%%%%%%%%%%%%%%%%%%%%%%%%%%%%%%%%%%%%
% Only for the journal Data
%\dataset{DOI number or link to the deposited data set if the data set is published separately. If the data set shall be published as a supplement to this paper, this field will be filled by the journal editors. In this case, please submit the data set as a supplement.}
%\datasetlicense{License under which the data set is made available (CC0, CC-BY, CC-BY-SA, CC-BY-NC, etc.)}

%%%%%%%%%%%%%%%%%%%%%%%%%%%%%%%%%%%%%%%%%%
% Only for the journal Toxins
%\keycontribution{The breakthroughs or highlights of the manuscript. Authors can write one or two sentences to describe the most important part of the paper.}

%%%%%%%%%%%%%%%%%%%%%%%%%%%%%%%%%%%%%%%%%%
% Only for the journal Encyclopedia
%\encyclopediadef{For entry manuscripts only: please provide a brief overview of the entry title instead of an abstract.}

%%%%%%%%%%%%%%%%%%%%%%%%%%%%%%%%%%%%%%%%%%
% Only for the journal Advances in Respiratory Medicine
%\addhighlights{yes}
%\renewcommand{\addhighlights}{%

%\noindent This is an obligatory section in ???Advances in Respiratory Medicine??????????????, whose goal is to increase the discoverability and readability of the article via search engines and other scholars. Highlights should not be a copy of the abstract, but a simple text allowing the reader to quickly and simplified find out what the article is about and what can be cited from it. Each of these parts should be devoted up to 2~bullet points.\vspace{3pt}\\
%\textbf{What are the main findings?}
% \begin{itemize}[labelsep=2.5mm,topsep=-3pt]
% \item First bullet.
% \item Second bullet.
% \end{itemize}\vspace{3pt}
%\textbf{What is the implication of the main finding?}
% \begin{itemize}[labelsep=2.5mm,topsep=-3pt]
% \item First bullet.
% \item Second bullet.
% \end{itemize}
%}

\begin{document}

%%%%%%%%%%%%%%%%%%%%%%%%%%%%%%%%%%%%%%%%%%

%%%introduction%%%
\section{Introduction}
\label{sec:intro}


High-energy nuclear collisions at the Relativistic Heavy Ion Collider~(RHIC) and Large Hadron Collider~(LHC) have opened up new avenues for the search for strongly interacting nuclear matter, the quark--gluon plasma~(QGP)~\cite{Wang:1992qdg, Gyulassy:2003mc, Mehtar-Tani:2013pia, Qin:2015srf}. Investigating the formation of the QGP deepens our understanding of quantum chromodynamics (QCD) under extreme conditions at high temperature and density~\cite{Cunqueiro:2021wls, Cao:2020wlm} and the evolution of the Universe at the first microsecond~\cite{Collins:1974ky}. The jet-quenching phenomena, the energy attenuation of fast partons due to their strong interactions with the QCD medium, provide an army of powerful tools to study the properties of the QGP, such as the yield suppression of high-$p_T$ hadron/jet, the $p_T$ asymmetry of dijets, $\gamma/Z^0$+ jets as well as jet substructures~\cite{Wang:1998bha,Wang:2001cs,Wang:2002ri,Majumder:2004pt,Zhang:2007ja,Vitev:2009rd,He:2020iow,Wang:1996yh,Neufeld:2010fj,Dai:2012am,Wang:2013cia,Li:2010ts,Vitev:2008rz,Caucal:2021cfb}.

In elementary proton--proton reactions, the productions of charm and bottom quarks are perturbatively calculable, since their large masses~($M_c\sim 1.5$~GeV, $M_b\sim 4.8$~GeV) act as a natural cut-off above the $\Lambda_{QCD}$~\cite{Andronic:2015wma}. Heavy quarks are produced in the initial hard scattering at a very early stage due to their large masses, therefore witnessing the whole QGP evolution. Meanwhile, while their thermal production is almost negligible with the initial conditions so far accessible in heavy-ion programs at the RHIC and LHC~\cite{Zhang:2007yoa}, the productions of charm and bottom hadron/jets make a very promising hard probe to the transport properties of hot and dense quark matter. During the past decade, the experimental measurements including the nuclear modification factor $R_{AA}$~\cite{Adamczyk:2014uip,Adam:2015sza,Sirunyan:2017xss,PHENIX:2011img,STAR:2013eve,ALICE:2014wnc,CMS:2011all} and the collective flow (the direct flow $v_1$~\cite{STAR:2019clv,ALICE:2019sgg} and elliptical flow $v_2$~\cite{Abelev:2014ipa,Adamczyk:2017xur,Acharya:2017qps,Sirunyan:2017plt}) of heavy-flavour hadrons both at the RHIC and LHC have attracted much attention from the community of high-energy nuclear physics.


A lot of theoretical studies have been performed to confront the experimental data obtained in the high-energy heavy-ion collisions, which greatly improve our understanding of the in-medium evolution~\cite{Zhang:2003wk,Djordjevic:2003zk,Zhang:2004qm,vanHees:2007me,CaronHuot:2008uh,Djordjevic:2015hra,He:2014cla,Kang:2016ofv,Svetitsky:1987gq,Moore:2004tg,
Cao:2013ita,Alberico:2013bza,Xu:2015bbz,Cao:2016gvr,Das:2016cwd,Ke:2018tsh,Li:2020kax,Liu:2009nb,Yan:2006ve,JETSCAPE:2022hcb,Liu:2021dpm,Ding:2021ajz} and hadronization mechanisms~\cite{Plumari:2017ntm,Cao:2019iqs,He:2019vgs} of heavy quarks (for detailed reviews see~\cite{Rapp:2009my,Prino:2016cni,Rapp:2018qla,Dong:2019unq,Dong:2019byy,Zhao:2020jqu,Apolinario:2022vzg,He:2022ywp,Tang:2020ame}). Specifically, the current models treat the elastic and inelastic interactions between heavy quarks and the QGP medium with multiple methods, consisting mainly the perturbative or non-perturbative analytic calculations (SCET~\cite{Kang:2016ofv,Li:2018xuv}, \textls[-25]{CUJET~\cite{Xu:2015bbz,Xu:2014tda}, DREENA~\cite{Djordjevic:2009cr,Djordjevic:2008iz,Djordjevic:2013xoa,Zigic:2021rku,Zigic:2018ovr,Zigic:2018smz}, WHDG~\cite{Wicks:2007am,Wicks:2005gt}, AdS/CFT (HG)~\cite{Horowitz:2007su,Horowitz:2011wm}), and the Monte Carlo transport approaches based on the Boltzmann (\mbox{BAMPS~\cite{Uphoff:2010sh,Uphoff:2011ad,Uphoff:2013rka,Uphoff:2014hza}}, \textls[-5]{MC$@_s$HQ~\cite{Gossiaux:2008jv,Gossiaux:2009mk,Nahrgang:2013saa}, \linebreak (Q)LBT~\cite{Cao:2016gvr,Liu:2021dpm}, LIDO~\cite{Ke:2018tsh,Ke:2020clc}, Catania-pQCD/QPM~\cite{Plumari:2012ep,Das:2015ana,Plumari:2011mk,Scardina:2017ipo}), the Langevin} (POWL \linebreak ANG~\cite{Alberico:2013bza,Alberico:2011zy,Beraudo:2014boa},} Duke~\cite{Cao:2013ita,Cao:2011et}, UrQMD~\cite{Lang:2012nqy,Lang:2013cca,Lang:2013wya}, TAMU~\cite{He:2012df,He:2011yi,He:2014cla}, \mbox{SHELL~\cite{Wang:2019xey,Dai:2018mhw,Wang:2021jgm}}) and the Kadanoff--Baym (PHSD~\cite{Cassing:2008sv,Cassing:2009vt,Bratkovskaya:2011wp}) equations. These phenomenological studies reveal a fact that the elastic scattering of heavy quarks in the hot/dense nuclear matter is important, especially at the lower $p_T$ region ($p_T^Q<5m_Q$), different from our experience of treating light quarks or gluons. One of the central issues of investigating the heavy-flavour production in the heavy-ion program is extracting the diffusion coefficient $D_s$, which is directly related to the transport properties of the hot QCD matter. Additionally, different from the fragmentation hadronization of heavy quarks in a vacuum, within the hot and dense nuclear matter, the heavy-flavour hadrons can be produced by a combination of heavy quarks and thermal partons. Such a coalescence hadronization mechanism plays an important role in the collective flow~\cite{Abelev:2014ipa,Adamczyk:2017xur} and baryon-to-meson ratio~\cite{STAR:2019ank,Vermunt:2019ecg} of charmed hadron in nucleus--nucleus collisions at the RHIC and LHC.


In recent years, the experimental measurements on heavy-flavour jet (a reconstructed jet containing a heavy quark or a heavy-flavour hadron) have made great strides in \linebreak p+p~\cite{CMS:2016wma,ALICE:2019cbr,CMS:2012pgw,ATLAS:2011chi,ATLAS:2021agf,STAR:2009kkp,ATLAS:2016anw,CMS:2020geg}, p+A~\cite{CMS:2016wma,Khachatryan:2015sva,ALICE:2021wct} and A+A collisions~\cite{Chatrchyan:2013exa,Sirunyan:2018jju,Sirunyan:2019dow,ATLAS:2022fgb,Roy:2022yrw,CMS:2021puf,ALICE:2018lyv,CMS:2022btc}. A wealth of information carried by heavy-flavour jets not only offers a new topic of jet physics and the application of the perturbative QCD, but also their medium modifications in heavy-ion collisions are also of great significance to reveal the in-medium energy loss mechanism of heavy quarks, to address the mass effect of jet quenching, and to extract the transport properties of the~QGP.

%The rest of this paper is organized as follows: in Sec.~\ref{sec:theory}, we review the latest theoretical %developments of the heavy flavor jet in heavy-ion collisions, including the full-jet observables and jet %substructures. We will summarize this review in Sec.~\ref{sec:summary}.
\section{Recent Advances of Heavy-Flavour Phenomenology in Heavy-Ion Collisions}
\label{sec:overview}

Generally speaking, as we discussed in the last section, the reason for treating the heavy flavours as powerful hard probes to the transport properties of the QGP consists of at least three aspects. Firstly, the large mass ($M_Q \gg \Lambda_{QCD}$) makes it available to compute the differential cross-section of heavy quarks in the binary nucleon--nucleon collisions based on the perturbative QCD (pQCD) scheme within the next-to-next-to-leading order (NNLO) precision~\cite{Cacciari:2005rk}. Secondly, due to the large mass ($M_Q \gg T_{\rm med}$), the total yield of heavy quarks in nucleus--nucleus collisions only depends on their initial production at hard scattering. Since the momentum transfer of the in-medium collisions $q^2\sim g^2T^2$\linebreak ($T\sim 0.4$--$0.5$ GeV) is much smaller than the creation energy of heavy quark pairs at the current collision energy, both at the RHIC and LHC, the subsequent contribution from the thermal creation during the QGP evolution is negligible~\cite{Zhang:2007yoa}. Apart from this, according to the Heisenberg uncertainty principle, the formation time of heavy quarks\linebreak ($\tau_0\sim \frac{1}{2m_Q}<0.1$~fm/c) is shorter than the formation time of the quark--gluon plasma ($\tau_f\sim 0.6$ fm/c). Therefore heavy quarks witness the entire evolution of the hot/dense nuclear matter until the freeze-out. In this section, we will briefly introduce the recent theoretical advances that help us understand the heavy-flavour production in heavy-ion collisions, including mainly the following several aspects, the initial production, the transport approaches, the hadronization mechanisms, and the extraction of diffusion~coefficient.


\subsection{Production of Heavy Quarks in p+p Collisions}
\label{sec:ppbaseline}

The production of heavy quarks in proton--proton collisions establishes a baseline to investigate the nuclear modification in high-energy nuclear collisions both at the RHIC and LHC. The yield of heavy flavours in nucleus--nucleus collisions generally is viewed as the sum of that in $N_{\rm coll}$ binary nucleon--nucleon collisions while taking into account the initial cold nuclear matter effect (usually considered by using the nuclear-modified parton distribution function~\cite{Eskola:2009uj,Eskola:2016oht,NNPDF:2014otw}). In the fixed-flavour-number scheme (FFNS)~\cite{Andronic:2015wma}, the cross-section of heavy quarks in p+p collisions can be expressed based on the factorization~theorem,
\vspace{-10pt}
\begin{adjustwidth}{-\extralength}{0cm}
\centering %% If there is a figure in wide page, please release command \centering
\begin{eqnarray}
d{\sigma}_Q[s,p_T,y,m_Q]\simeq\sum_{i,j}\int_0^1 dx_i\int_0^1 dx_j f_i^A(x_i,\mu_F)f_j^A(x_j,\mu_F)d\tilde{{\sigma}}_{ij\rightarrow Q+X}[x_i,x_j,s,p_T,y,m_Q,\mu_F,\mu_R]
\label{eq:sigma}
\end{eqnarray}
\end{adjustwidth}
where $s$ is the square of the centre-of-mass energy of the incoming proton, $p_T$ is the transverse momentum of the produced heavy quark, and $y$ is the rapidity. $f_i^A$ ($f_j^B$) is the parton distribution function (PDF) quantifying the probability to find a parton with flavour {$i$($j$)} %%MDPI: Please confirm if i, j should be italic.
and carrying momentum fraction $x_{i(j)}$ in the colliding proton $A(B)$, which relies on the factorization scale $\mu_F$. {$\tilde{{\sigma}}_{ij\rightarrow Q+X}$} represents the cross-section of the partonic hard process $i+j\rightarrow Q+X$ that can be calculated relying on the pQCD. %please check intended meaning is retained.
The partonic cross-section $\tilde{{\sigma}}_{ij\rightarrow Q+X}$ also relies on the strong coupling constant $\alpha_s$ determined at the renormalization scale $\mu_R$. Note that Equation~(\ref{eq:sigma}) sums all partonic hard processes $i+j\rightarrow Q+X$, where $i,j$ are the active flavours including ($u,\bar{u},d,\bar{d},s,\bar{s},g$) but not heavy quarks. Only at the factorization scale $\mu_F>m_c$, can charm be viewed as an active flavour, often used for beauty production. The differential cross-section $d{\sigma}_Q$ can be convolved with a scale-independent fragmentation function $D_Q^H(z)$, such as the Peterson~\cite{Peterson:1982ak} or Lund~\cite{Andersson:1983ia} forms, to obtain the cross-section of the heavy-flavour hadron,
\begin{eqnarray}
d{\sigma}_H=d{\sigma}_Q\otimes D_Q^H(z)
\label{eq:sigmaH1}
\end{eqnarray}
where $H$ denotes the heavy-flavour hadron and $z$ the momentum fraction carried by $H$.

Since the FFNS is usually applicable at the low $p_T$ region ($0<p_T<5m_Q$), for the higher kinematic region ($p_T\gg m_Q$), the logarithmic terms ($\frac{\alpha_s}{2\pi}ln(p_T^2/m_Q^2)$) in the perturbative expansion of the cross-section become large, and should be resummed to all orders. To implement such a resummation, one has to absorb the large logarithmic terms into the parton distribution function and fragmentation function. This treatment requires that heavy quarks are active flavours when the factorization scale is $\mu_F>m_Q$. In other words, such a scheme has a variable number of active flavours when $\mu_F$ crosses the heavy quark mass, hence named the variable-flavour-number scheme (VFNS). In particular, when the heavy quark mass can be neglected in the evaluation of the short-distance cross-section, the VFNS scheme is called the zero-mass VFNS (ZM-VFNS). In the ZM-VFNS, the differential cross-section of a heavy-flavour hadron based on the factorization theorem can be expressed~as:
\begin{eqnarray}
d{\sigma}_{H+X} \simeq \sum_{i,j}\int_0^1 dx_i\int_0^1 dx_j f_i^A(x_i,\mu_F)f_j^A(x_j,\mu_F)d\tilde{{\sigma}}_{ij\rightarrow k+X}D_k^H(z,{\mu}_F^{\prime})
\label{eq:sigmaH2}
\end{eqnarray}
where $D_k^H(z,{\mu}_F^{\prime})$ is given by a convolution of a perturbative-fragmentation function (PFF) $D_k^Q(z,{\mu}_F^{\prime})$ describing a parton k fragmentation into heavy quark Q, with a scale-independent one $D_Q^H(z)$ for the hadronization of a heavy quark. Note that in Equation~(\ref{eq:sigmaH2}) the sum covers all possible partonic-hard processes ($i+j\rightarrow k+X$) where $i,j,k $ can be light quarks, gluons, and heavy quarks~\cite{Aversa:1988vb}. Since heavy quark mass is neglected in the computation of the cross-section, the ZM-VFNS is expected to be reliable only at very high~$p_T$.

To find a unified theoretical framework that combines the advantages of the FFNS at low $p_T$ and the ZM-VFNS at high $p_T$, in recent years the interpolation schemes have been established, such as the general-mass VFNS (GM-VFNS)~\cite{Kniehl:2004fy,Kniehl:2005mk} and the fixed-order plus next-to-leading logarithms (FONLL)~\cite{Cacciari:2005rk,Cacciari:2012ny}. For instance, by using an interpolating function $G(m_Q,p_T)=p_T^2/(p_T^2+c^2m_Q^2)$ where $c$ is set to $c=5$, the FONLL scheme can well describe the heavy-flavour production in the entire kinematic region. For more details of the interpolation schemes see~\cite{Andronic:2015wma} and the references therein.

Compared to the analytic calculation schemes discussed above, the general-purpose Monte Carlo event generator, such as PYTHIA~\cite{Sjostrand:2006za}, HERWIG~\cite{,Corcella:2000bw}, POWHEG~\cite{Frixione:2007vw} and SHERPA~\cite{Gleisberg:2008ta}, can provide a more complete description of all the final-state particles at the parton or hadron level. Especially for the studies of jet physics, the Monte Carlo event generators can give more precise descriptions of the observations relating to the jet substructure than analytic calculations.

\subsection{Transport of Heavy Quarks in the QGP}
\label{sec:transport}

Transport approaches are wildly used in the current theoretical studies of heavy-flavour production in high-energy nuclear collisions. At the lower $p_T$ region, the elastic scattering of heavy quarks with the thermal parton (light quark or gluon) has been proven to be the dominant mechanism of energy loss. Generally,  the kinetic theory based on the Boltzmann transport equation is a popular treatment for in-medium heavy quark evolution. The Boltzmann equation for the distribution function of heavy quarks can be written in a compact form,
\begin{eqnarray}
p_{\mu}\partial_{\mu}f_Q(x,p)=\mathit{C}[f_q,f_{\bar{q}},f_g,f_Q](x,p)
\label{eq:Boltzmann}
\end{eqnarray}
where $f_Q(x,p)$ is the phase-space distribution of heavy quarks. In the QGP, the phase-space distributions of light quark $f_q$ and gluon $f_g$ can be solved by the Boltzmann equation~\mbox{\cite{Ferini:2008he,Ruggieri:2013ova}}. Subsequently, the relativistic Boltzmann-like collision integral $\mathit{C}[f_Q](x,p)$ has a simplified form~\cite{Svetitsky:1987gq,Beraudo:2014iva},
\begin{eqnarray}
\mathit{C}[f_Q]=\int d^3q[\omega(\mathbf{p}+\mathbf{q},\mathbf{q})f_Q(\mathbf{x},\mathbf{p}+\mathbf{q},t)-\omega(\mathbf{p},\mathbf{q})f_Q(\mathbf{x},\mathbf{p},t)]
\label{eq:integ}
\end{eqnarray}
%MDPI: Please confirm if the bold should be retained. same for all bold in equations
{where} $\omega(\mathbf{p}+\mathbf{q},\mathbf{q})$ represents the transition rate of a heavy quark from the momentum $\mathbf{p}+\mathbf{q}$ to $\mathbf{p}$ by collisions with quasiparticles. This rate is usually determined by the matrix elements of the $2\rightarrow 2$ QCD scattering. With the assumption that the momentum transfer $|\mathbf{q}|$ is small compared to the momentum of a heavy quark, we can expand $\omega(\mathbf{p}+\mathbf{q},\mathbf{q})f_Q(\mathbf{x},\mathbf{p}+\mathbf{q},t)$ around $\mathbf{q}$ by utilizing the Taylor formula to obtain the Fokker--Planck equation,
\begin{eqnarray}
\frac{\partial f_Q}{\partial t}=\frac{\partial}{\partial p_i}\left [ A_i(\mathbf{p})f_Q+\frac{\partial}{\partial p_j}[B_{ij}(\mathbf{p})f_Q] \right ]
\label{eq:FP}
\end{eqnarray}
where two coefficients $A_i(\mathbf{p})=\int d^3q\omega(\mathbf{p},\mathbf{q})q_i$ and $B_{ij}(\mathbf{p})=\int d^3q\omega(\mathbf{p},\mathbf{q})q_iq_j$ are directly related to the drag coefficient ($\eta_D$) and the momentum diffusion coefficient ($\kappa$), which control rate of the energy loss and the momentum broadening of heavy quarks in the hot medium, respectively. Indeed, the Fokker--Planck equation is equivalent to another more well-known equation, the Langevin equation,
\begin{eqnarray}
\frac{d\vec{x}}{dt}&=&\frac{\vec{p}}{E}\\
\frac{d\vec{p}}{dt}&=&-\eta_D(p) \vec{p}+\vec{\xi}(t)
\label{eq:Fokker}
\end{eqnarray}
where the stochastic term $\vec{\xi}(t)$ describes the random kicks suffered in heavy quarks from the medium constituents, which obeys a Gaussian distribution with a mean value $0$ and variance $\kappa$. The drag coefficient $\eta_D$ and the diffusion coefficient $\kappa$ are related by the fluctuation--dissipation theorem (FDT) $\kappa=2\eta_DET$. Note that at higher kinematic regions ($p_T^Q>5m_Q$), the medium-induced gluon radiation plays an increasingly important role in the energy loss of heavy quarks. The radiative energy loss of heavy quarks is treated with various formalisms and at different approximations~\cite{Zhang:2003wk,Arnold:2002ja,Abir:2011jb,Abir:2012pu,Zapp:2008gi,Armesto:2003jh}, which usually provide the radiated gluon spectra as a function of momentum fraction $x$ and transverse momentum $k_{\perp}$. In the Langevin equation, the radiative energy loss of heavy quarks can be coupled with the collisional one by adding a recoil term $-\vec{p}_g$ caused by the radiated gluon~\cite{Cao:2013ita}. The four-momentum of the radiated gluon can be easily sampled based on the radiation spectra $d{\rm N}_g/d{\rm x}d{\rm k}_{\perp}^2$.

In many of the recently developed theoretical frameworks modelling the production of heavy flavour in heavy-ion collisions, the Boltzmann and Langevin equations are the two most popular choices, especially for Monte Carlo simulations. Concerning the performance of these two approaches, detailed comparisons have been discussed in~\cite{Das:2013kea, Li:2019wri}. In general, the implementation of the Boltzmann equation implies that the medium consists of well-defined quasiparticles, while the Fokker--Planck (Langevin) equation is realized in a more general way without the quasiparticle assumption. However, the advantage of the Boltzmann equation is that it can naturally describe the heavy quark evolution even under off-equilibrium conditions, which may be the case of the early pre-equilibrium stage in heavy-ion collisions~\cite{Das:2017dsh}.

\subsection{Hadronization: Fragmentation and Coalescence}
\label{sec:hadronization}
Studying the yield suppression and collective flow of heavy-flavour hadrons also deepens our understanding of heavy quark hadronization in nucleus--nucleus collisions, which shows different mechanisms with that in a vacuum. As discussed in Section~\ref{sec:ppbaseline}, fragmentation functions describe the non-perturbative hadronization process of heavy quarks into heavy-flavour hadron in a vacuum. The most commonly used fragmentation function is the Peterson form~\cite{Peterson:1982ak},
\begin{eqnarray}
D_{H/Q}(z)=\frac{N}{z[1-\frac{1}{z}-\frac{\epsilon_Q}{1-z}]}
\label{eq:frag}
\end{eqnarray}
where $z$ denotes the momentum fraction carried by the heavy hadron from the heavy quark in the fragmentation process ($0<z<1$), which implies that the heavy hadron must have smaller energy than the heavy quark. The only tunable parameter in Equation~(\ref{eq:frag}) is $\epsilon_Q$ that can be determined by fitting to the measured spectra of the heavy-flavour hadrons. $N$ is the normalization factor to guarantee $\int_0^1 dz D_{H/Q}(z)=1$.

Measurements on the collective flow~\cite{Abelev:2014ipa, Adamczyk:2017xur} and baryon-to-meson ratio~\cite{STAR:2019ank, Vermunt:2019ecg} of charmed hadron A+A collisions suggest the existence of a new hadronization mechanism, coalescent of heavy quarks. %please check intended meaning is retained.
The basic idea behind the coalescence mechanism is that a heavy quark can combine with a light anti-quark from the medium when they have enough small distance in the coordinate-momentum space. It means that the heavy-flavour meson has larger energy than the parent heavy quark, differing from the mechanism of fragmentation. The distribution function of the formed heavy-flavour meson usually can be obtained by a convolution with the following schematic form.
\begin{eqnarray}
f_M \sim g_M f_{Q(\bar{Q})} \otimes f_{\bar{q}(q)} \otimes \phi_M
\label{eq:coal}
\end{eqnarray}
where $g_M$ denotes the degeneracy of the heavy-flavour meson in spin and isospin, $f_{Q(\bar{Q})}$ and $f_{\bar{q}(q)}$ are the distribution functions of the heavy and light quarks in the coordinate-momentum space, respectively. $\phi_M$ represents the Wigner transform of the wave function of the heavy-flavour meson, commonly approximated by the ground state one of the simple harmonic oscillators~\cite{Cao:2016gvr}.

In the realistic implementation of the heavy quark hadronization in nuclear collisions, the first step is to determine the probability of coalescence by integrating the distribution function of Equation~(\ref{eq:coal}). If coalescence occurs, one can sample a light anti-quark based on the thermal equilibrium distribution, otherwise Equation~(\ref{eq:frag}) is used to fragment the heavy quark into a hadron. At least in the lower $p_T$ region, the experimental results favour the coalescence mechanism~\cite{Scardina:2017ipo}. The coalescence of heavy quarks seems to decrease the suppression factor and enhance the collective flow of heavy-flavour hadrons, especially at $p_T<6$ GeV. The recent studies~\cite{Cao:2019iqs,He:2019vgs} show that the coalescence mechanism is important in the description of the $\Lambda_c/D^0$ ratio measured by the STAR~\cite{STAR:2019ank} and ALICE~\cite{Vermunt:2019ecg} collaborations. Additionally, the hadronic scattering between the D meson and light-flavour hadrons ($D-\pi$, $D-\rho$) has also been studied in~\cite{Lin:2000jp}, but its influence on the D meson $R_{AA}$ was found to be very limited~\cite{Cao:2015hia}.

\subsection{Extraction of the Diffusion Coefficient of Heavy Quarks}
\label{sec:Extraction}


One of the most important goals of the heavy-ion collision experiment is to investigate the transport properties of the QCD matter under extremely hot and dense conditions. As discussed above, due to the large mass ($m_Q \gg T_{\rm med}$), heavy quarks are believed to be powerful tools for exploring the transport properties of the QGP. Phenomenological studies of heavy-flavour production in high-energy nuclear collisions provide a unique opportunity to extract the transport coefficient of the QGP, such as the momentum diffusion coefficient $\kappa$ of heavy quarks, whose longitudinal and transverse components can be convenient to define as,
\begin{eqnarray}
\kappa_{||}\equiv-\frac{d\left\langle (\Delta p_{||})^2 \right\rangle}{dt}\\
\kappa_{\perp}\equiv\frac{1}{2}\frac{d\left\langle (\Delta p_{\perp})^2 \right\rangle}{dt}
\label{eq:kappa}
\end{eqnarray}
where $\Delta p_{||}$ and $\Delta p_{\perp}$ momentum changes parallel and perpendicular to the heavy quark formulation. %please check intended meaning is retained.
By definition, $\kappa_{\perp}$ can be directly related to the jet transport coefficient $\hat{q}$ which quantifies all the transverse momentum broadening of hard partons as traversing the QGP medium. Assuming that the $\kappa$ is isotropic, namely, $\kappa_{\perp}=\kappa_{||}=\kappa$, one can obtain a simplified relation $\hat{q}=2\kappa$. This relation has been employed in the modified Langevin equation to balance the two parts of the contribution from the collisional and radiative energy loss of heavy quarks~\cite{Cao:2013ita,Li:2019lex}. Here we only overview the recent advances of the $\kappa$ extraction by different model calculations. A more detailed and profound discussion about this topic can be found in~\cite{Rapp:2018qla}.

The momentum diffusion coefficient $\kappa$ can be easily converted to the spatial one $D_s$ with the relation $\kappa=2T^2/D_s$. In recent years, the temperature dependence of the dimensionless quantity $2\pi TD_s$ has been estimated by a lot of theoretical frameworks, such as the lattice QCD (lQCD)~\cite{Ding:2015ona,Banerjee:2011ra,Kaczmarek:2014jga}, LO pQCD~\cite{Moore:2004tg,vanHees:2004gq}, QPM calculations~\cite{Das:2015ana}, \textit{T}-matrix~\cite{He:2014cla}, PHSD~\cite{Song:2015sfa}, MC$@_s$HQ~\cite{Andronic:2015wma}, Ads/CFT~\cite{Horowitz:2015dta}, duke (Bayesian analysis)~\cite{Xu:2017obm}, and hadronic matter~\cite{He:2011yi,Tolos:2013kva}, as shown in Figure \ref{fig:Ds}.




The estimates by the lQCD from the first principles provide a valuable reference for the model extractions of $2\pi TD_s$. As one can see, with relatively large uncertainties, the lQCD calculations in the quenched approximation give $D_s\sim$ 3.7--7.0~\cite{Kaczmarek:2014jga} over the temperature range from $T_{pc}$ to 2$T_{pc}$. However, it is difficult to extract meaningful information about the temperature dependence of $2\pi TD_s$ from the current lQCD results. Furthermore, except for the pQCD calculations at the leading-order which show obvious larger values than others, these extractions of $2\pi TD_s$ based on the recently developed models are consistent with the lQCD data, as well as previous studies presented in~\cite{Rapp:2018qla} which give $2\pi TD_s\sim$ 2--4 near the critical temperature. Although these calculations give different values of $2\pi TD_s$ versus $T/T_{pc}$, most estimations show that $D_s$ slightly increases with $T$. It implies that the interactions between a charm quark and the QCD medium have the strongest strength near the critical temperature. However, no direct evidence has been found in the experiment to verify this upward trend of $D_s$ so far because it is hard to find an observation only sensitive to the in-medium interactions at the late stage of the QGP evolution. Fortunately, the data-driven analysis utilizing Bayesian inferences seems to shed new light on this issue. The temperature and momentum dependence of $D_s$ has been extracted from the available experimental data ($R_{AA}$ and $v_2$ of a D meson both at the RHIC and LHC)~\cite{Xu:2017obm} based on the Duke--Langevin transport model, which indeed shows an upward trend of $2\pi TD_s$. More recently, this approach of Bayesian inference has been improved with the help of information field theory~\cite{Bialek:1996kd,Lemm:1999kd} in~\cite{Xie:2022fak}. Therefore, one can now extract model parameters without relying on an explicit form of parametrization, leading to a robust determination by such a model--data fit. %please check intended meanining is retained.
\begin{figure}[H]
%\begin{center}
%%\vspace{0.1in}
\includegraphics[width=4.5in,height=3.7in,angle=0]{Ds.pdf}
\caption{Spatial diffusion coefficient ($2\pi TD_s$) of charm quark in the quark--gluon plasma calculated by different approaches versus the reduced temperature ($T/T_{pc}$). The lattice QCD calculations in the quenched approximation~\cite{Ding:2015ona,Banerjee:2011ra,Kaczmarek:2014jga} are compared with the estimations based on different models~\cite{Moore:2004tg,vanHees:2004gq,Das:2015ana,He:2014cla,Song:2015sfa,Andronic:2015wma,
Horowitz:2015dta,Xu:2017obm,He:2011yi,Tolos:2013kva}. The figure is from~\cite{Dong:2019byy}.}
\label{fig:Ds}
%\end{center}
\end{figure}

\section{Production of Heavy-Flavour Jets in Heavy-Ion Collisions}
\label{sec:theory}

\subsection{Nuclear Modification Factors of Production Yields}
\label{sec:RAA}



To address the nuclear effect in relativistic heavy-ion collisions, the nuclear modification factor $R_{AA}$ is conventionally utilized to quantify the yield suppression of hadron/jet in A+A collisions per binary nucleon--nucleon collision relative to p+p~\cite{STAR:2003pjh},
\begin{eqnarray}
R_{AA}=\frac{1}{\left\langle N_{\rm bin}^{\rm AA} \right\rangle}\frac{d\sigma^{\rm AA}/\rm dydp_T}{d\sigma^{\rm pp}/\rm dydp_T}
\label{eq:raa}
\end{eqnarray}
where the scaling factor $\left\langle N_{\rm bin}^{\rm AA} \right\rangle$ denotes the number of binary nucleon--nucleon collisions in A+A~\cite{Miller:2007ri}. It has been observed that the values of $R_{AA}$ of hadrons and jets are smaller than one in nucleus--nucleus collisions both at the RHIC~\cite{STAR:2003wqp,PHENIX:2005nhb,PHENIX:2010bqp} and LHC~\cite{ATLAS:2018gwx,ATLAS:2014ipv}, and these measurements could be explained by the mechanism of partonic energy loss, which in turns serve as convincing evidence for the formation of QGP in such extremely hot and dense conditions. Meanwhile, the jet transport parameter $\hat{q}\equiv d\left\langle p^2_{\perp}\right\rangle/dL$~\cite{Baier:2002tc} representing the strength of in-medium partonic interactions could be extracted from the available $R_{AA}$ data by various theoretical models~\cite{JET:2013cls,Xie:2019oxg,Ru:2019qvz,JETSCAPE:2021ehl,Xie:2022fak}.

Additionally, to test the mass dependence of jet quenching, the $R_{AA}$ has also been used in a comparison of the yield suppression between heavy-flavour jets and inclusive jets. Benefiting from the fact that heavy-flavour jets are produced abundantly as the centre-mass energy increases in hadronic collisions at the LHC, the exploration of a heavy quark-tagged jet produced in heavy-ion collisions has gradually attracted much attention. The first experimental effort focused on the production of a b-jet was implemented by the CMS collaboration~\cite{Chatrchyan:2013exa} in 2013, as shown in the left plot of Figure~\ref{fig:raa}, where a b-jet is defined as jets containing at least one B hadron inside the jet-cone. The red points are the CMS data and the coloured bands are the theoretical calculations. This measurement accounts for the b-jet samples in minimum bias collisions (0--100\%). We note that even with large experimental uncertainties, the b-jet $R_{AA}$ slightly increases with jet $p_T$ and varies from 0.4 to 0.8. Significant suppression of the b-jet yield in Pb+Pb collisions at \mbox{$\sqrt{s_{NN}}$ = 2.76 TeV} relative to the p+p baseline was observed for the first time, which indicates that bottom quarks strongly interact with the hot/dense nuclear matter. Furthermore, within experiment uncertainties, the results were found to be consistent with the pQCD-based calculations conducted in~\cite{Huang:2013vaa} when the coupling factor $g^{med}$ varied from 1.8 to 2.2.
\vspace{-12pt}
\begin{figure}[H]
%\begin{center}
%%\vspace{0.1in}
\includegraphics[width=3.0in,height=2.7in,angle=0]{raa-cms.pdf}
\includegraphics[width=2.4in,height=2.5in,angle=0]{raa-b-inclusive.pdf}
\caption{\textbf{Left}: the measured nuclear modification factor $R_{AA}$ of the inclusive b-jet versus b-jet $p_T$ by the CMS collaboration in Pb+Pb at $\sqrt{s_{NN}}$~=~2.76 TeV for minimum bias collisions~\cite{Chatrchyan:2013exa}. \textbf{Right}: the comparison of $R_{AA}$ between the inclusive jets and b-jets versus jet $p_T$ in central Pb+Pb collisions at \mbox{$\sqrt{s_{NN}}$ = 2.76 TeV~\cite{Dai:2018mhw}}. The figures are from~\cite{Chatrchyan:2013exa,Dai:2018mhw}.}
\label{fig:raa}
%\end{center}
\end{figure}



To address the difference of the yield suppression between the b-jet and inclusive jet (mainly initiated by a massless light quark or gluon), a direct comparison of their $R_{AA}$ in the right plot of Figure~\ref{fig:raa} was presented by the SHELL approach, which applies a Langevin transport model to describe heavy quark propagation in the QGP~\cite{Dai:2018mhw} in central Pb+Pb collisions at $\sqrt{s_{NN}}$ = 2.76 TeV, as well as the next-to-leading order pQCD calculations matched with the parton shower effect for the p+p baseline~\cite{Gleisberg:2008ta,Frixione:2002ik}. In the model, the jet transport parameter $\hat{q}$ was extracted by the production of an identified hadron in A+A collisions~\cite{Ma:2018swx}, and then the spatial diffusion coefficient $D_s$ of heavy quarks can be determined by the D meson $R_{AA}$ data~\cite{Sirunyan:2017xss,ALICE:2018lyv}. The measured $R_{AA}$ of the inclusive jet with the centrality of \mbox{0--5\%~\cite{CMS:2016uxf}} and b-jet with 0--10\%~\cite{Chatrchyan:2013exa} are also illustrated in the plot of Figure~\ref{fig:raa}. Although the $R_{AA}$ of the b-jet seems to be slightly smaller than that of the inclusive jet, the CMS collaboration claims that no clear difference of $R_{AA}$ between the inclusive jet and b-jet was found, because the current uncertainties of the b-jet data are too large. However, the theoretical calculations in~\cite{Dai:2018mhw} suggest that b-jet $R_{AA}$ may be larger than inclusive jet $R_{AA}$, due to the ``dead-cone'' effect of the bottom quarks, which suppresses the medium-induced gluon radiation of massive heavy quarks within a cone $\theta\sim M/E$~\cite{Dokshitzer:2001zm}. A more precise measurement is necessary to resolve the tension between the experimental data and theoretical calculations. It is very exciting the fact that recently, the ATLAS collaboration reported preliminary results by simultaneously measuring the $R_{AA}$ of the inclusive jet and b-jet in 0--20\% Pb+Pb collisions at \mbox{$\sqrt{s_{NN}}$ = 5.02 TeV~\cite{ATLAS:2022fgb}}, which shows a clear weaker suppression of the b-jet, and the features can be described by the theoretical calculations~\mbox{\cite{Dai:2018mhw,Ke:2018tsh}}. Although the mass hierarchy of jet quenching at the particle level has been confirmed by a lot of experimental data~\cite{PHENIX:2022wim,STAR:2021uzu}, it is indisputable that the ATLAS measurement makes a crucial step towards finding the mass effect at the jet level. The comparison of the c- and b-jet $R_{AA}$ has been presented in~\cite{Li:2018xuv} with the SCET model~\mbox{\cite{Ovanesyan:2011kn,Sievert:2018imd}}, which shows no significant difference at $p_T>50$ GeV. More recently, some exploratory estimates indicate that the $R_{AA}$ of the c-jet may be stronger than that of the inclusive jet at higher jet $p_T$ due to their different constituents~\cite{Ke:2020nsm,wang:2022aaa}, an interesting finding to be investigated further in detail.


Beyond $R_{AA}$, another observation $I_{AA}$~\cite{Neufeld:2010fj} has also been utilized to study the yield suppression of b-jets tagged by $Z^0$ bosons in high-energy nuclear collisions~\cite{Wang:2020qwe}. Similar to $R_{AA}$, $I_{AA}$ is defined as follows,
\begin{eqnarray}
I_{AA}=\frac{1}{\left\langle N^{\rm AA}_{\rm bin} \right\rangle}\frac{\frac{d\sigma^{\rm AA}}{dp^{\rm jet}_{\rm T}}|_{p_{\rm T}^{\rm min}<p_{\rm T}^Z<p_{\rm T}^{\rm max}}}{\frac{d\sigma^{\rm pp}}{dp^{\rm jet}_{\rm T}}|_{p_{\rm T}^{\rm min}<p_{\rm T}^Z<p_{\rm T}^{\rm max}}}
\label{eq:IAA}
\end{eqnarray}

One can see that $I_{AA}$ quantifies the yield variation of the jet in A+A collisions per binary nucleon--nucleon collision relative to the p+p baseline, after integrating the $Z^0$ boson $p_T$. It has been proposed that the associated production $Z^0\,+\,$b-jet may be helpful in addressing the mass dependence of the jet-quenching effect, since the $Z^0\,+\,$jet processes significantly exclude the contamination of gluon-initiated jets~\cite{Kartvelishvili:1995fr}. Thereby the comparison of $I_{AA}$ between the $Z^0\,+\,$jet and $Z^0\,+\,$b-jet can provide direct features of the mass effect of heavy quark jets compared to light-quark jets. Figure~\ref{fig:IAA} shows the comparisons of the calculated $I_{AA}$ of the $Z^0\,+\,$b-jet and $Z^0\,+\,$jet in central 0--10\% Pb+Pb collisions at $\sqrt{s_{NN}}$~=~5.02 TeV, where the jets are reconstructed by the anti-$k_T$ algorithm with a cone size R = 0.3 and pseudorapidity \mbox{$|\eta^{\rm jet}|<$ 1.6}. The calculations are presented within different $p_T$ windows of $Z^0$ bosons in the three panels, namely 40--60, 60--80, and 80--100 GeV. We note that the shapes of $I_{AA}$ are flat at the panel of $40<p_T^Z<60$ GeV but have downward trends at that of $80<p_T^Z<100$, and the $I_{AA}$ in the right panel even shows enhancement at\linebreak  $p_T^{\rm jet}$. This is because if one constrains the $Z^0$ meson momentum in the event selection, such as $80<p_T^Z<120$~GeV, then the selected jet distribution with $p_T$ would fall steeper at\linebreak $p_T^{\rm jet}<$ 80 GeV, which naturally leads to a relatively large value of the nuclear modification factors at \mbox{$p_T^{\rm jet}<$ 80 GeV}, even larger than one. Additionally, one observes that at each panel the model calculations show that $I_{AA}$ of the $Z^0\,+\,$b-jet is visibly higher than that of the $Z^0\,+\,$jet in nucleus--nucleus collisions, which indicates that the $Z^0$-tagged light-quark jets lose more energy than the $Z^0$-tagged b-jets traversing the QGP. These comparisons would be helpful to directly test the mass effect of jet energy loss in heavy-ion collisions at the LHC from a new perspective.

\begin{figure}[H]
%\begin{center}
%%\vspace{0.1in}
\includegraphics[width=5.4in,height=2.2in,angle=0]{raa-Iaa.pdf}
\caption{{Nuclear modification factor} $I_{AA}$ as a function of the transverse momentum of the tagged jet within three $p_T^Z$ windows: 40--60, 60--80, 80--120~GeV in central 0--10\% Pb+Pb collisions at $\sqrt{s_{NN}}~=~5.02$~TeV~\cite{Wang:2020qwe}.}
\label{fig:IAA}%MDPI: Figure 3  has not been mentioned in the text. Please cite it in the text and ensure the first citation of each figure appears in numerical order.
%\end{center}
\end{figure}
\newpage
\subsection{Transverse Momentum Imbalance}
\label{sec:XJ}



The transverse momentum imbalance ($x_J=p_{\rm T,2}/p_{\rm T,1}$) is another useful observation, describing the momentum asymmetry of the dijet system in the transverse plane, where $p_{\rm T,1}$ and $p_{\rm T,2}$ denote the leading and sub-leading jet $p_T$. It is noted that in the fixed-leading-order QCD calculations the two outgoing hard partons should be strictly back-to-back in the transverse ($x_J$~=~1), but the higher-order corrections and vacuum shower may break the symmetry which leads to $x_J<$ 1. In heavy-ion collisions, the smaller $x_J$ of the $\gamma$+jet~\cite{CMS:2017ehl} and $Z^0$+jet~\cite{CMS:2017eqd} systems have been observed in Pb+Pb collisions compared to p+p, which results from the energy loss of the tagged jet. The CMS collaboration reports the measurement on $x_J$ of the inclusive and $b\bar{b}$ dijets in Pb+Pb collisions at \mbox{$\sqrt{s_{NN}}$~=~5.02 TeV~\cite{Sirunyan:2018jju}}. In their measurements, the biggest challenge was how to select the $b\bar{b}$ dijet events initiated by the hard heavy-quark pairs, because it is crucial to address the mass effect by directly comparing such heavy-quark dijets with inclusive dijets. On the theoretical side, the production mechanisms of heavy quarks can be categorized into three classes: flavour creation (FCR), flavour excitation (FEX), and gluon splitting (GSP)~\cite{Nason:1989zy,Beenakker:1990maa,Mangano:1991jk,Norrbin:2000zc}, only FCR represents the dijets initiated by heavy-quark pairs originating from the hard process. The CMS collaboration suggests a strategy to separate the FCR processes by selecting $b\bar{b}$ dijets that have a large opening angle ($\vert \Delta \phi \vert > 2\pi/3$) in azimuth, which could significantly suppress the contributions of the other two. This method has also been used in theoretical studies~\cite{Huang:2015mva,Dai:2018mhw,Kang:2018wrs}. Figure~\ref{fig:avexj} shows a comparison of the averaged $x_J$ of the inclusive and $b\bar{b}$ dijets in both p+p and Pb+Pb collisions with different centrality bins at $\sqrt{s_{NN}}$~=~5.02 TeV, as well as the experimental data~\cite{Sirunyan:2018jju}, where $\left\langle x_J \right\rangle$ was estimated as follows.
\begin{figure}[H]
%\begin{center}
\includegraphics[width=2.5in,height=2.4in,angle=0]{xj-dijet.pdf}
\includegraphics[width=2.5in,height=2.4in,angle=0]{xj-bjet.pdf}
\caption{Averaged $x_J$ value as a function of the number of participants calculated in p+p and Pb+Pb collisions $\sqrt{s_{NN}}$~=~5.02 TeV within different centrality bins compared with the experimental data, both for inclusive (\textbf{left}) and $b\bar{b}$ (\textbf{right}) dijets. Figures are from~\cite{Dai:2018mhw}.}
\label{fig:avexj}
\vspace{-0.2in}
%\end{center}
\end{figure}
\vspace{-6pt}
\begin{eqnarray}
\left\langle x_J \right\rangle = \frac{1}{\sigma}\int_{0}^{1}\frac{d\sigma}{dx_J}dx_J
\end{eqnarray}

{The} black triangle points are the CMS data in Pb+Pb collisions, and the black star points are the p+p reference used in their measurements. The blue and red rhombus points are the theoretical calculations, while the p+p reference is provided by the Monte Carlo event generator SHERPA~\cite{Gleisberg:2008ta} which matches the next-to-leading order QCD matrix elements and the parton shower effect in a vacuum~\cite{Gleisberg:2008fv,Schumann:2007mg}. The $\left\langle x_J \right\rangle$ points of the inclusive (left panel) and $b\bar{b}$ (right panel) dijets are listed within three centrality bins which correspond to the different numbers of the participant in Pb+Pb collisions. In Figure~\ref{fig:avexj}, the theoretical calculations based on the SHELL model~\cite{Dai:2018mhw} show an overall decrease in $\left\langle x_J \right\rangle$ in Pb+Pb collisions relative to the p+p baseline both for the inclusive and $b\bar{b}$ dijets, consistent with the CMS data and indicates that the asymmetry between these two leading jets is amplified in A+A collisions. The reduction in $\left\langle x_J \right\rangle$ is centrality-dependent since the in-medium interaction is sensitive to the temperature and size of the QGP. It is even more important that the calculations show that the decrease in $\left\langle x_J \right\rangle$ of the $b\bar{b}$ dijets is slightly smaller than that of the inclusive dijets within the same centrality bins. These results suggest that dijets initiated by bottom quarks may suffer smaller energy loss compared to those initiated by light quarks or gluons. Furthermore, another study on the $b\bar{b}$ dijet in heavy-ion collisions~\cite{Kang:2018wrs} proposed that the invariant mass $m_{jj}$ of the dijet system could be a novel observation sensitive to mass effects of jet quenching.




In addition to the dijet system, the transverse momentum imbalances of the $Z^0\,+\,$jet ($x_{jZ}=p_T^{\textit{\rm jet}}/p_T^Z$) and $Z^0\,+\,$b-jet ($x_{bZ}=p_T^{\textit{\rm b-jet}}/p_T^Z$) have also been investigated~\cite{Wang:2020qwe}. It was found that the $Z^0$-tagging requirement considerably decreased the contribution of gluon-jets by 40\% in $Z^0\,+\,$jets compared to the dijet sample, especially at a lower jet $p_T$. The comparison of the medium modification on the $x_J$ of $Z^0\,+\,$jet and $Z^0\,+\,$b-jet may be suitable to address the mass effect of jet quenching. Figure~\ref{fig:xjib} shows the distributions of the $x_{jZ}$ (left) and $x_{bZ}$ (right) both in p+p and 0--10\% Pb+Pb collisions at $\sqrt{s_{NN}}$~=~5.02~TeV. In the calculations, the selected $Z^0$ bosons are required to have $p_T^Z>$ 60 GeV. The tagged jets (b-jets) are reconstructed with the anti-$k_T$ algorithm with a cone-size R = 0.3 and pseudorapidity $|\eta^{\rm jet}|<$ 1.6, required to have \mbox{$p_T^{\rm jet}>$ 30 GeV}. In particular, to guarantee that the $Z^0$ bosons and the tagged jets are back-to-back in the transverse plane, the $Z^0\,+\,$jet or $Z^0\,+\,$b-jet pairs are required to have a large opening angle in azimuth, $\Delta\phi_{jZ}(\Delta\phi_{bZ})>7\pi/8$. The differences of $x_{jZ}$~($x_{bZ}$) distributions in p+p and Pb+Pb collisions are also shown in the lower panels. Due to the jet energy loss, the $x_{jZ}$ and $x_{bZ}$ distributions shift towards smaller $x_J$ values in Pb+Pb collisions relative to p+p. Furthermore, one can find in the lower panel that the variations of $x_{bZ}$ are slightly smaller than that of $x_{jZ}$. More intuitive comparisons between the averaged $x_{jZ}$ and $x_{bZ}$ are listed in Table \ref{tab:axjz}. Within the statistical errors, the results show that $\Delta \left\langle x_{jz} \right\rangle \sim$ 0.136 is considerably larger than $\Delta \left\langle x_{bz} \right\rangle \sim$ 0.092, consistent with the expectation that bottom jets lose less energy than light-quark jets.
\begin{figure}[H]
%\begin{center}
%%\vspace{0.1in}
\includegraphics[width=2.6in,height=3.2in,angle=0]{xj-jz.pdf}
\includegraphics[width=2.6in,height=3.2in,angle=0]{xj-bz.pdf}
\caption{{Distributions} of $x_{jZ}$~($x_{bZ}$) of $Z^0\,+\,$jet (\textbf{left}) and $Z^0\,+\,$b-jet (\textbf{right}) both in p+p and 0--10\% Pb+Pb  collisions at $\sqrt{s_{NN}}$~=~5.02~TeV. The differences of $x_{jZ}$~($x_{bZ}$) distributions in p+p and Pb+Pb collisions (green line) are also shown in the lower panels. Figures are from Ref.~\cite{Wang:2020qwe}.}%MDPI: 1. Please change the hyphen (-) into minus sign (?) in the figure, e.g., ???????-1??????? should be ????????1???????.
%2. please add explanation for the green line
  \label{fig:xjib}
%\end{center}
\end{figure}
\vspace{-6pt}
\begin{table}[H]
\caption{The averaged $x_J$ of $Z^0$ + jet and $Z^0$ + b-jet both in p+p  and Pb+Pb collisions at \mbox{$\sqrt{s_{NN}}=5.02$~TeV}, as well as their variations $\Delta x_{J}={\left\langle x_{J} \right\rangle }_{\rm pp}-{\left\langle x_{J} \right\rangle }_{\rm PbPb}$. The statistical errors of $x_J$ in the simulations are also presented. Table is from Ref.~\cite{Wang:2020qwe}.}\label{tab:axjz}
\begin{tabular}{p{4.0cm}<{\centering}p{4.3cm}<{\centering}p{4.3cm}<{\centering}}
  \toprule
  %\hspace*{0.1in}
  & \boldmath{$Z^0$} \textbf{+ jet} & \boldmath{$Z^0$} \textbf{+ b-jet} \\
  \midrule
  ${\left\langle x_{J} \right\rangle }_{pp}$ & 0.987 $\pm$ 0.0047 & 0.941 $\pm$ 0.0056 \\
  \midrule
  ${\left\langle x_{J} \right\rangle }_{PbPb}$ & 0.851 $\pm$ 0.0061 & 0.849 $\pm$ 0.0064 \\
  \midrule
  $\Delta \left\langle x_{J} \right\rangle $ & 0.136 $\pm$ 0.0108 & 0.092 $\pm$ 0.012 \\
  \bottomrule
\end{tabular}
\end{table}



\subsection{Angular Correlation}
\label{sec:corre}



Jet angular correlations, such as $\Delta \phi$ distribution of dijets~\cite{Mueller:2016gko,Jia:2019qbl} and $\gamma/Z^0$ +\linebreak jet~\mbox{\cite{Luo:2018pto,Zhang:2018urd}}, are useful observable to address the medium-induced transverse momentum effect. In this context, estimating the medium modification on the angular distribution of heavy quark dijets in nucleus--nucleus collisions may also be of interest from the theoretical point of view. As shown in the left plot of Figure~\ref{fig:angular}, medium modification of the azimuthal angular correlations~($\Delta\phi=|\phi_{b1}-\phi_{b2}|$) of the $b\bar{b}$ dijet system in Pb+Pb collisions with different centralities at $\sqrt{s_{NN}}$~=~5.02 TeV are calculated~\cite{Wang:2018gxz}. One can observe suppression at $\Delta\phi\sim$0 and enhancement at $\Delta\phi\sim\pi$ in Pb+Pb collisions compared to the p+p, and the modifications are centrality dependent. Since the distributions are self-normalized, it implies that $b\bar{b}$ dijets with a larger opening angle (back-to-back) suffer relatively weaker yield suppression compared to that with a smaller one (collinear). It can be noted that the main contribution of $b\bar{b}$ dijet production at smaller $\Delta\phi$ is from the GSP process while larger $\Delta\phi_{bb}$ from the FCR process. The two b-jets from the former process share the energy of the gluon and then usually have lower $p_T$ than that from the latter process. As a result, the yield at the smaller $\Delta\phi$ region is more sensitive to the selection cut $p_T^{\rm jet}>$ 20 GeV. Actually, in another study on the angular correlations of $Z^0\,+\,$b-jet~\cite{Wang:2020qwe}, it's found that initial average b-jet $p_T$ distribution versus $\Delta\phi$ play a critical role, as shown in the right plot of Figure~\ref{fig:angular}. We see that the ratio of PbPb/pp in the middle panel is flat, and the average b-jet $p_T$ distribution is also flat. It's reasonable to guess that in Pb+Pb the azimuthal angle between b-jet and $Z^0$ has not been modified compared to p+p, and the overall suppression occurs at whole $\Delta\phi_{\rm bZ}$ region. Of course, we can imagine that it is more difficult for high-$p_T$ ($>$30 GeV) jets to be significantly deflected by the scattering with thermal parton.
\begin{figure}[H]
%\begin{center}
%\vspace{0.1in}
\includegraphics[width=2.2in,height=2.4in,angle=0]{phibb.pdf}
\includegraphics[width=2.2in,height=2.45in,angle=0]{phizb.pdf}
\caption{\textbf{Left}: normalized azimuthal angular distributions of $b\bar{b}$ dijet system in p+p and Pb+Pb collisions at $\sqrt{s_{NN}}$~=~5.02 TeV. Results for different centrality bins, 0--10\%, 10--30\%, 30--100\%, are presented. \textbf{Right}: {the azimuthal} angular distribution of $Z^0\,+\,$b-jet in p+p and 0--10\% Pb+Pb collisions at $\sqrt{s_{NN}}$~=~5.02 TeV in the upper panel, while the ratio of PbPb/pp (green solid line) was shown in the middle panel and the averaged b-jet $p_T$ (blue band) in the lower panel. Figures are from Refs.~\cite{Wang:2018gxz,Wang:2020qwe}.}%MDPI:  please add explanation for the green line and blue area in the right panel
\label{fig:angular}
%\end{center}
\end{figure}



To probe the angular deflection caused by the in-medium $p_T$-broadening,  observables accessible to lower $p_T$ region are needed. For this reason, it's proposed in Ref.~\cite{Wang:2021jgm} that the heavy-flavour meson tagged by direct photon ($\gamma+$HF) may provide a promising channel, with several advantages: (1) the transverse momentum resolution of $D^0$ meson can be low down to $\sim$1 GeV~\cite{ALICE:2018lyv} where the angular deflection is significant, (2) the photon gauges the initial momentum of heavy quarks, therefore, it's easy to quantify the direction change, (3) the selection bias effect can be suppressed by constraining the photon energy \ to cite{Cunqueiro:2021wls}. In this way, the considerable angular de-correlations between the heavy quarks and photons are predicted both in central Au+Au collisions at the RHIC and Pb+Pb collisions at the LHC. Furthermore, by constructing the 2-dimensional ($\Delta\phi, x_J$) correlation diagram of $\gamma+$HF, it's argued that the two aspects of jet quenching, energy loss, and $p_T$-broadening, can be well displayed simultaneously. Additionally, it's noted that another measurement on the angular correlations of $D^0$+hadron in Au+Au collisions at $\sqrt{s_{NN}}$~=~200 GeV may reflect the medium modification of the charm+jet correlation in the $\eta-\phi$ plane~\cite{STAR:2019qbf}, that awaits further detailed investigations.

\subsection{Radial Profile}
\label{sec:radial}



The radial profile of the heavy-flavour jet represents the distribution of the angular distance $r=\sqrt{(\phi_{\rm Q}-\phi_{\rm jet})^2+(\eta_{\rm Q}-\eta_{\rm jet})^2}$ between the heavy-flavour meson and the jet-axis in the $\eta-\phi$ plane. Systematic studies with a focus on the radial profiles of D-jet and B-jet in heavy-ion collisions are performed in Refs.~\cite{Wang:2019xey,Wang:2020bqz}. As shown in the left panel of Figure~\ref{fig:radial}, the model calculated radial profiles of D-jets both in p+p and 0--100\% Pb+Pb collisions at $\sqrt{s_{NN}}=$ 5.02 TeV compared to the CMS measurements~\cite{Sirunyan:2019dow}. The black and red triangle points represent the measured data. The D-jets are reconstructed with anti-$k_T$ algorithm with R = 0.3 and $|\eta^{\rm jet}|<$ 1.6. All selected D-jets must satisfy $p_T^{\rm jet}>$ 60 GeV and contain at least one $D^0$ meson in jet-cone with $4<p_T^D<20$ GeV. The blue solid line is the p+p baseline provided by SHERPA~\cite{Gleisberg:2008ta}, and the red dashed line denotes the calculations based on the SHELL model. One can observe that the model calculations show the radial profile of D-jets in Pb+Pb collisions shifts towards larger radii relative to that of p+p, which is consistent with the diffusion trend observed by the CMS collaboration. These results show a clear physics picture, that charm quarks change their moving direction when scattering with the thermal partons in the hot and dense QCD matter. The studies argue that the diffusion behavior of D meson is closely related to the $p_T$-broadening when charm quarks scatter with the thermal partons in the medium. It should be noted that in such an estimate the jets are required to have $p_T>60$ GeV while D meson $p_T<20$ GeV, which makes that the higher $p_T$ jets can be viewed as a reference to probe the moving direction changes of charm quarks. It is found that the angular deviation $\Delta r=\sqrt{(\phi_c^f-\phi_c^i)^2+(\eta_c^f-\eta_c^i)^2}$ of charm quarks from their initial position in the $\eta-\phi$ plane is $p_T$ dependent, as shown in the right plot of Figure~\ref{fig:radial}. The charm quarks with lower $p_T$ are more likely to change their traveling direction via the in-medium scattering, and this feature also explains why no visible modification is observed in the CMS data for $p_T^D>20$ GeV~\cite{Sirunyan:2019dow}. The angular deviation at lower $p_T$ (\mbox{$<$5 GeV}) is dominated by elastic scattering, whereas at higher $p_T$ by inelastic reactions. These investigations may cast light on the in-medium energy loss mechanisms and constrain the transport coefficients of heavy quarks from a new perspective. We notice that a preliminary result of the D-jet radial profile in Au+Au collisions at \mbox{$\sqrt{s_{NN}}=$ 200 GeV} has been reported by the STAR collaboration in Ref.~\cite{Roy:2022yrw}. This result shows a similar diffusion effect of charm quark in jets in mid-central 10--40\% collisions.



To test the mass effect reflected in the radial profile, an additional comparison of the medium modification between D-jet and B-jet has been presented in Refs.~\cite{Wang:2020ffj,Wang:2020ukj}, where an inverse modification pattern on the radial profile of B-jets compared to D-jets is observed. The jet quenching effect seems to narrow the jet radial profiles of B-jets while broadening those of D-jets. It's demonstrated that the selection bias effect~\cite{Renk:2012ve} in A+A collisions may play a pivotal role. Heavy quark jets with higher $p_T$ have narrower initial radial distributions, and would naturally lead to narrower modifications when they fall into the lower $p_T$ domain due to jet energy loss. This reveals the fact that the final-state modification of the jet is not only influenced by the pure medium effect, but also by the other factors, such as the initial spectra and the selection bias~\cite{Cunqueiro:2021wls}.
\begin{figure}[H]
%\begin{center}
%\vspace{0.1in}
\includegraphics[width=2.5in,height=2.7in,angle=0]{radial1.pdf}
\includegraphics[width=2.5in,height=2.7in,angle=0]{radial-pt.pdf}
%\vspace*{-.2in}
\caption{\textbf{Left}: radial profile of D-jet in p+p and Pb+Pb collisions at $\sqrt{s_{NN}}$~=~5.02 TeV. \textbf{Right}: angular deviation of charm quark as a function of initial $p_T$. Figures are from Refs.~\cite{Wang:2019xey,Wang:2020bqz}.}
\label{fig:radial}
%\end{center}
\end{figure}

\subsection{Fragmentation Function}



The jet fragmentation function $D(z)=(1/N_{\rm jet})dN_{\rm ch}(z)/dz$ is one of the most well-explored jet substructure observable~\cite{Sjostrand:1986hx,Webber:1983if,Procura:2009vm}, which usually refers to the longitudinal momentum distribution of charged hadrons inside the jet-cone~\cite{CMS:2014jjt,ATLAS:2014dtd,ATLAS:2017nre,ATLAS:2019dsv}. For heavy-flavour jets, the corresponding observable is the longitudinal momentum distribution of heavy-flavour mesons in jets, defined as in~\cite{ALICE:2019cbr}.
\begin{eqnarray}
D(z_{||})=\frac{1}{N_{\rm jet}}\frac{dN_{\rm HQ}(z_{||})}{dz_{||}}, \quad {\rm where} \
z_{||}=\frac{\vec{p}_{\rm HQ} \cdot \vec{p}_{\rm jet}}{\vec{p}_{\rm jet} \cdot \vec{p}_{\rm jet}} \,\,  .
\label{eq:dz}
\end{eqnarray}

On the one hand, the $D(z_{||})$ distribution may provide useful information to reveal the production mechanisms and substructure of heavy quark jets~\cite{ALICE:2022mur}. On the other hand, since $z_{||}$ denotes the momentum projection of the heavy-flavour hadron on the jet axis, the medium modification of the $D(z_{||})$ distribution in nucleus--nucleus collisions is closely related to the interplay of the partonic energy loss between the massive heavy quarks and the massless light partons~\cite{Li:2022tcr}. Figure~\ref{fig:zll} shows the first theoretical investigation of the medium modification on the $D(z_{||})$ distributions of both D-jets and B-jets in Pb+Pb collisions at $\sqrt{s_{NN}}$~=~5.02 TeV. In these calculations, two jet $p_T$ ranges are chosen, \linebreak $5<p_T^{\rm jet}<15$ GeV and $15<p_T^{\rm jet}<50$ GeV. Respectively the selected $D^0$ ($B^0$) mesons are also required to have $p_{T,D^0(B^0)}>$ 2 GeV and $p_{T,D^0(B^0)}>$ 5 GeV. The black solid lines represent the p+p baseline of $D(z_{||})$ distributions calculated by the POWHEG+PYTHIA8 event generator~\cite{Nason:2004rx,Frixione:2007vw,Alioli:2010xd,Sjostrand:2007gs}, and the orange dash lines are the theoretical calculations in Pb+Pb collisions based on the SHELL model. The upper and middle panels correspond to the $D(z_{||})$ distributions of $D$-jets and $B$-jets, while the lower panels are their nuclear modification $D(z_{||})_{PbPb}/D(z_{||})_{pp}$ (green is D-jet and yellow B-jet). One can observe that the initial $D(z_{||})$ distributions in p+p are sensitive to the kinematic region of jet and heavy-flavour hadron, especially for D-jets. Moreover, even within the same kinematic region, a B-jet has an evident harder fragmentation pattern compared to a D-jet. The difference could be relevant to the fact that the stronger ``dead-cone'' effect suffered in heavier bottom quarks, in other words, the bottom quarks radiate less gluon and carry more energy fraction of jets than charm quarks. Besides, the contribution of the GSP process may also play different roles in the production of B-jets and D-jets, which may lead to additional differences in their $z_{||}$ distributions~\cite{Li:2022tcr}. In nuclear collisions, the main finding is that the jet quenching effect results in softer fragmentation patterns of heavy-flavour jets in the QGP compared to that in a vacuum. It's different from what one could naively argue, that is, the energy fraction of heavy quarks in jets may increase because heavy quarks lose less energy than light partons. The modification of $D(z_{||})$ reveals the different energy loss mechanisms between the single parton and the full jet. Critically, the lost energy from the jet constituents may be partially brought back to the jet energy by the reconstruction procedures. This is an essential difference in energy loss mechanisms between the full-jet and the single parton, which leads to less energy loss of full-jet compared to heavy quarks. Therefore, stronger medium modification of $D(z_{||})$ can be obtained with larger R, which may be related to the R-dependence of jet energy loss~\cite{ATLAS:2012tjt,Bossi:2022fpc}. Furthermore, stronger medium modification of $D(z_{||})$ is observed for B-jets compared to D-jets, due to their different initial spectra.
\vspace{-10pt}
\begin{figure}[H]
%\begin{center}
%\vspace*{0.1in}
\includegraphics[width=5.2in,height=4.8in,angle=0]{zll.pdf}
\vspace*{-.1in}
\caption{$D(z_{||})$ distributions of D-jet and B-jet within two $p_T$ windows both in p+p and 0--10\% Pb+Pb collisions, as well as the medium modifications (PbPb/pp). Figure is from Ref.~\cite{Li:2022tcr}.}
\label{fig:zll}
%\end{center}
\end{figure}


\subsection{The ``Dead-Cone'' Effect and Other Observables}

Until now, there are a few other heavy-flavour jet observables accessible in the current experimental measurements at the LHC, which have also attracted attention from the high-energy nuclear physics community. We briefly discuss them in the following.

\begin{itemize}
    \item The Cambridge-Aachen (CA) declustering techniques~\cite{Dokshitzer:1997in} which can help to obtain the angular-ordered pairwise tree of subjets~\cite{Larkoski:2014wba} and the Soft Drop condition mentioned above enable us to expose the most basic heavy quark splitting structure by measuring the splitting-angle distributions in D$^0$ meson jets in p+p collisions at $\sqrt{s_{NN}}=13$~TeV~\cite{ALICE:2021aqk}. It has been measured in three different energy intervals of the radiators: $5 \leq E_{\rm Radiator}  \leq 10$~GeV, $10 \leq E_{\rm Radiator}  \leq 20$~GeV and $20 \leq E_{\rm Radiator}  \leq 30$~GeV and constrain the transverse momentum of the D$^0$ meson in jet to be $2 < p^{\rm D^0}_{\rm T}< 36$~GeV/c. The ALICE collaboration directly observed for the first time a clear distribution suppression at the splitting angle smaller than the ratio of quark mass and the energy of such quark radiator: $\theta \leq M_{\rm charm}/E_{\rm radiator}$, known as the ``dead-cone'' effect~\cite{Dokshitzer:1991fd,Cunqueiro:2018jbh}. Such a heavy quark jet and its substructure measurement reveal and confirm this most basic property of a fast quark interacting with the vacuum described by the QCD theory.




A subsequent phenomenology study exposed the ``dead-cone'' effect of the medium-induced gluon radiation of jet queching~\cite{,Dokshitzer:2001zm,Zhang:2003wk,Armesto:2003jh}, by calculating the emission angle distribution of the heavy-flavour quark initiated splittings in a $\rm D^0$ meson tagged jet and that of the light parton initiated splittings with the existence of the QGP in Pb+Pb collisions at $\sqrt{s_{NN}}=5.02$~TeV ~\cite{Dai:2022sjk}, as demonstrated in Figure~\ref{fig:dcinAA}. Very interestingly, they find the collisional energy loss mechanism will not obscure the observation of the ``dead-cone'' effect in the medium-induced radiation. Such a proposal has also been verified by an analytical study that proposes a new jet substructure groomer that selects the most collinear splitting in a QCD jet above a certain transverse momentum cutoff~\cite{Cunqueiro:2022svx}. It's also found in another study that the ``dead-cone'' domain would be partially filled by the medium-induced emission as heavy quarks traversing QGP~\cite{Armesto:2003jh}.
\begin{figure}[H]
	%\begin{center}
    \hspace{-15pt}    \includegraphics[scale=0.745]{figure2.pdf}
	%\end{center}
	%\vspace{-0.05\textwidth}	
	\caption{The splitting-angle distributions for $\rm D^0$ meson tagged jets, inclusive jets and also light-quark jets normalized to the number of jets in Pb+Pb collisions at $\rm \sqrt{s}=5.02$~TeV (upper plots) and also the $D^0$ meson tagged jets/inclusive jets (light-quark jets/inclusive jets) ratios (bottom plots) calculated for three energy intervals of the radiators: $\rm 5<E_\text{Radiator}<10$~GeV (left panel),  $\rm 10<E_\text{Radiator}<20$~GeV (middle panel) and  $\rm 20<E_\text{Radiator}<30$~GeV (right panel). The shaded areas correspond to the angles at which the radiation is suppressed due to the ``dead-cone'' effect. Figure is from Ref.~\cite{Dai:2022sjk}.}
	\label{fig:dcinAA}
\end{figure}



 \item The jet shape $\rho(r)$ describes the transverse energy profile of charged hadrons as a function of the angular distance from the jet axis. This observable has been well-studied for light flavor jets~\cite{Chang:2019sae,Luo:2018pto} to search the medium response effect as energetic parton dissipating energy to the medium~\cite{Cao:2020wlm}. The measurement of the medium modification on the b-jet shape has been reposted in Refs.~\cite{CMS:2020geg,CMS:2022btc} by the CMS collaboration. On the ond hand, the comparison of jet shapes of b-jets in Pb+Pb and p+p collisions shows the presence of the QGP modifies the energy distributions around the jet axis of b-jets. On the other hand, their measurements indicate a stronger jet energy redistribution of b-jets at larger radii compared to that of inclusive jets. Generally speaking, the bottom quarks are expected to dissipate less energy in nuclear collisions compared to light quarks and gluons due to the ``dead-cone'' effect. However, at larger jet radii, the medium response effect plays the dominant role in the enhancement of jet energy distribution in Pb+Pb collisions compared to the p+p baseline. Therefore, these interesting results may suggest that the heavier quark, like the bottom,  may drive a stronger medium response effect than a massless parton. In this context, the heavy-flavour jets can serve as promising sensitive probes to the quasi-particle excitation of the quark~soup.

\item The Soft Drop (SD) grooming procedures reveal the two-prong structure of a jet, described by the momentum sharing $z_g$ and opening angle $R_g$~\cite{Larkoski:2014wba}, which establishes the connection between the final state observable to the parton splitting function. The splitting history could be helpful to identify the production mechanisms of heavy-flavour jets~\cite{Goncalves:2015prv,Ilten:2017rbd}. Heavy quark jets from the gluon splitting process usually tend to have more balanced $z_g$ and larger $R_g$ compared to that from the FEX and FCR. The first measurement of the D-jet splitting function is performed by ALICE~\cite{ALICE:2022phr}, and some theoretical efforts which focus on the medium modifications of $z_g$ and $R_g$ of c- and b-jets are presented in Refs. \cite{Li:2017wwc,Zhang:2023jpe}. The medium effects result in more imbalanced $z_g$ distribution and larger opening angles between the two subjets in the heavy quark jets, similar to the medium modification of inclusive jets observed by the CMS~\cite{Sirunyan:2017bsd} and ALICE~\cite{Acharya:2019djg} collaboration.

\end{itemize}

\section{Summary and Conclusions}
\label{sec:summary}

This review covers the current development of theoretical studies on heavy-flavour jets in ultra-relativistic heavy-ion collisions. We introduce the recent theoretical advances of heavy-flavour production in heavy-ion collisions and then give a comprehensive discussion of several recent investigations relating to the heavy-flavour jet observables.

\begin{itemize}

\item We briefly overview the recent theoretical advances that help us understand the heavy-flavour production in heavy-ion collisions, mainly focusing on the initial production,  transport approaches, hadronization mechanism, and diffusion coefficient extraction. These phenomenological studies based on the transport models reveal a fact that the elastic scattering of heavy quarks is dominant at lower $p_T$ region ($p_T^Q<5m_Q$), while the inelastic one dominate the high $p_T$ regions. Besides, different from the fragmentation hadronization of heavy quarks in a vacuum, within the hot and dense nuclear matter, the coalescence mechanism plays an important role in explaining the large collective flow and the enhancement of baryon-to-meson ratio of a charmed hadron in nucleus--nucleus collisions at the RHIC and the LHC. The diffusion coefficient of heavy quarks in the QGP has been extracted by various theoretical frameworks, which implies that $2\pi TD_s$ slightly increases with temperature. The newly developed Bayesian inference approach may be promising to implement a robust determination of the transport coefficient of heavy quarks by a model-data fit.

\item The studies on yield suppression and momentum imbalance of heavy-flavour jets are dedicated to addressing the mass effect of jet energy loss. Theoretical investigations predict stronger yield suppression of light quark jets compared to heavy-flavour jets, which is preliminarily proven by the recent ATLAS measurement of b-jet $R_{AA}$. However, the dijet asymmetry shows a reduced sensitivity to the jet quenching effect, therefore the difference of the medium modification on $x_J$ between inclusive and $b\bar{b}$ dijets seems to be moderate. We have to say the nuclear modification factor is still an effective and powerful observable to test the mass effect of energy loss in QGP. On the other hand, the strategy to isolate the jets initiated by heavy quarks is also crucial to address the mass effect, since GSP processes indeed have a large contribution to the production of heavy quark jets but suffer stronger suppression in nucleus--nucleus~collisions.

\item An observable related to angular correlation aims at the deflection of the jet axis caused by the medium-induced $p_T$-broadening of jet quenching. It's found that the angular deviation caused by the in-medium scattering is hard to be observed for high-$p_T$ jets, both for $b\bar{b}$ dijets and $Z^0\,+\,$(b-)jet. That makes sense because higher $p_T$ jets are more difficult to be changed by the in-medium scattering with the thermal parton in QGP. Meanwhile, medium modification on the radial profiles of jets containing lower-$p_T$ D meson can well capture the angular de-correlation of the charm quark and the jet axis. This suggests that heavy flavors may be more suitable to address the medium-induced $p_T$-broadening of jet quenching since they are experimentally accessible to the low-$p_T$ domain where the angular deviation is visible.

\item The substructure observable can reveal a wealth of information about the inner configuration of heavy-flavour jets. In the vacuum case, declustering techniques provide an inventive way to reestablish the splitting history of hard partons which helps us unlock the ``dead-cone'' effect of charm quark in the experiment. For heavy-flavour jets, the substructure observable also provides a unique opportunity to identify their production mechanisms. Furthermore, jet substructure, such as jet shape, seems more sensitive to the induced medium excitation in nucleus--nucleus collisions than full-jet observables. Much theoretical effort should be made to address the interplay of the ``dead-cone'' effect of medium-induced radiation and the medium response of heavy quarks. From the current perspective, the studies of substructures of heavy-flavour jets could play an increasingly important role in high-energy nuclear physics.

\item The initial jet spectra and the ``selection bias'' play important roles in the medium modifications of jet substructure in nuclear collisions. Normally when we focus on the mass effect of the yield or substructure modification of heavy quark jets, it is apriori to believe that bottom jets should have a weaker medium modification in heavy-ion collisions compared to charm jets under the same conditions. However, in the studies of radial profile and fragmentation function of heavy-flavour jets, it's found that b-jets have very different initial substructure compared to that of c-jets event within the same kinematic constraints, which eventually leads to stronger medium modification of b-jets at the final-state compared to c-jets. On the other hand, the ``selection bias'' poses a challenge to the theoretical studies that aim at the nuclear modification mechanism of heavy-flavour jets in the hot and dense QCD medium. It brings additional ``modifications'' to the ratio PbPb/pp of jet substructure distributions, nevertheless, these ``modifications'' do not exactly reflect the change of jet substructure but only the decrease of jet energy from the higher kinematic region in Pb+Pb collisions.
\end{itemize}


%%%%%%%%%%%%%%%%%%%%%%%%%%%%%%%%%%%%%%%%%%
%\vspace{6pt}

%%%%%%%%%%%%%%%%%%%%%%%%%%%%%%%%%%%%%%%%%%
%% optional
%\supplementary{The following supporting information can be downloaded at:  \linksupplementary{s1}, Figure S1: title; Table S1: title; Video S1: title.}

% Only for the journal Methods and Protocols:
% If you wish to submit a video article, please do so with any other supplementary material.
% \supplementary{The following supporting information can be downloaded at: \linksupplementary{s1}, Figure S1: title; Table S1: title; Video S1: title. A supporting video article is available at doi: link.}

%%%%%%%%%%%%%%%%%%%%%%%%%%%%%%%%%%%%%%%%%%
\authorcontributions{{Conceptualization, S.W. and B.-W.Z.; methodology, W.D. and B.-W.Z.; investigation, S.W. and W.D.; writing---original draft preparation, S.W. and W.D.; writing---review and editing, B.-W.Z., E.W. and X.-N.W.; supervision, B.-W.Z., E.W. and X.-N.W.} All authors have read and agreed to the published version of the manuscript.}%MDPI: For research articles with several authors, a short paragraph specifying their individual contributions must be provided. The following statements should be used ``Conceptualization, X.X. and Y.Y.; methodology, X.X.; software, X.X.; validation, X.X., Y.Y. and Z.Z.; formal analysis, X.X.; investigation, X.X.; resources, X.X.; data curation, X.X.; writing---original draft preparation, X.X.; writing---review and editing, X.X.; visualization, X.X.; supervision, X.X.; project administration, X.X.; funding acquisition, Y.Y. All authors have read and agreed to the published version of the manuscript.'', please turn to the  \href{http://img.mdpi.org/data/contributor-role-instruction.pdf}{CRediT taxonomy} for the term explanation. Authorship must be limited to those who have contributed substantially to the work~reported.

\funding{{This research} is supported by the Guangdong Major Project of Basic and Applied Basic Research No.~2020B0301030008, and the Natural Science Foundation of China with Project Nos.~11935007, 12035007, 12247127. S. Wang is also supported by China Postdoctoral Science Foundation under project No.~2021M701279.}%MDPI: Information regarding the funder and the funding number should be provided. Please check the accuracy of funding data and any other information carefully.

%\institutionalreview{In this section, you should add the Institutional Review Board Statement and approval number, if relevant to your study. You might choose to exclude this statement if the study did not require ethical approval. Please note that the Editorial Office might ask you for further information. Please add ???The study was conducted in accordance with the Declaration of Helsinki, and approved by the Institutional Review Board (or Ethics Committee) of NAME OF INSTITUTE (protocol code XXX and date of approval).?????????????? for studies involving humans. OR ???The animal study protocol was approved by the Institutional Review Board (or Ethics Committee) of NAME OF INSTITUTE (protocol code XXX and date of approval).?????????????? for studies involving animals. OR ???Ethical review and approval were waived for this study due to REASON (please provide a detailed justification).?????????????? OR ???Not applicable?????????????? for studies not involving humans or animals.}

%\informedconsent{Any research article describing a study involving humans should contain this statement. Please add ``Informed consent was obtained from all subjects involved in the study.'' OR ``Patient consent was waived due to REASON (please provide a detailed justification).'' OR ``Not applicable'' for studies not involving humans. You might also choose to exclude this statement if the study did not involve humans.

%Written informed consent for publication must be obtained from participating patients who can be identified (including by the patients themselves). Please state ``Written informed consent has been obtained from the patient(s) to publish this paper'' if applicable.}

\dataavailability{Not applicable.} %MDPI: We encourage all authors of articles published in MDPI journals to share their research data. In this section, please provide details regarding where data supporting reported results can be found, including links to publicly archived datasets analyzed or generated during the study. Where no new data were created, or where data are unavailable due to privacy or ethical re-strictions, a statement is still required. Suggested Data Availability Statements are available in section ???MDPI Research Data Policies?????????????? at \url{https://www.mdpi.com/ethics}.

%\acknowledgments{}   %MDPI: Titles (e.g., Dr., Mr., and Prof.) should NOT be used in the Acknowledgments section. We removed them. Please confirm.
%Please include the full first names if possible here


\conflictsofinterest{The authors declare no conflict of interest.}

%%%%%%%%%%%%%%%%%%%%%%%%%%%%%%%%%%%%%%%%%%
%% Optional
%\sampleavailability{Samples of the compounds ... are available from the authors.}

%% Only for journal Encyclopedia
%\entrylink{The Link to this entry published on the encyclopedia platform.}

%%%%%%%%%%%%%%%%%%%%%%%%%%%%%%%%%%%%%%%%%%
%% Optional
%%%%%%%%%%%%%%%%%%%%%%%%%%%%%%%%%%%%%%%%%%
\begin{adjustwidth}{-\extralength}{0cm}
%\printendnotes[custom] % Un-comment to print a list of endnotes

\reftitle{References}

\begin{thebibliography}{999}

\bibitem{Wang:1992qdg}
X.~N.~Wang and M.~Gyulassy,
%``Gluon shadowing and jet quenching in A + A collisions at s**(1/2) = 200-GeV,''
Phys. Rev. Lett. \textbf{68}, 1480-1483 (1992).
%doi:10.1103/PhysRevLett.68.1480
%935 citations counted in INSPIRE as of 31 Dec 2022

%\cite{Gyulassy:2003mc}
\bibitem{Gyulassy:2003mc}
  M.~Gyulassy, I.~Vitev, X.~N.~Wang and B.~W.~Zhang,
  %``Jet quenching and radiative energy loss in dense nuclear matter,''
  In *Hwa, R.C. (ed.) et al.: Quark gluon plasma* 123-191
 % doi:10.1142/9789812795533_0003
  %[nucl-th/0302077].
  %%CITATION = doi:10.1142/9789812795533_0003;%%

%\cite{Mehtar-Tani:2013pia}
\bibitem{Mehtar-Tani:2013pia}
Y.~Mehtar-Tani, J.~G.~Milhano and K.~Tywoniuk,
%``Jet physics in heavy-ion collisions,''
Int. J. Mod. Phys. A \textbf{28}, 1340013 (2013)
%doi:10.1142/S0217751X13400137
[arXiv:1302.2579 [hep-ph]].
%180 citations counted in INSPIRE as of 28 Dec 2022

%\cite{Qin:2015srf}
\bibitem{Qin:2015srf}
  G.~Y.~Qin and X.~N.~Wang,
  %``Jet quenching in high-energy heavy-ion collisions,''
  Int.\ J.\ Mod.\ Phys.\ E {\bf 24}, no. 11, 1530014 (2015).
 % doi:10.1142/S0218301315300143, 10.1142/9789814663717_0007
%  [arXiv:1511.00790 [hep-ph]].
  %%CITATION = doi:10.1142/S0218301315300143, 10.1142/9789814663717_0007;%%

%\cite{Cunqueiro:2021wls}
\bibitem{Cunqueiro:2021wls}
L.~Cunqueiro and A.~M.~Sickles,
%``Studying the QGP with Jets at the LHC and RHIC,''
Prog. Part. Nucl. Phys. \textbf{124}, 103940 (2022)
%doi:10.1016/j.ppnp.2022.103940
[arXiv:2110.14490 [nucl-ex]].
%27 citations counted in INSPIRE as of 29 Nov 2022

%\cite{Cao:2020wlm}
\bibitem{Cao:2020wlm}
S.~Cao and X.~N.~Wang,
%``Jet quenching and medium response in high-energy heavy-ion collisions: a review,''
Rept. Prog. Phys. \textbf{84} (2021) no.2, 024301
%doi:10.1088/1361-6633/abc22b
[arXiv:2002.04028 [hep-ph]].
%24 citations counted in INSPIRE as of 14 Jul 2021

%\cite{Collins:1974ky}
\bibitem{Collins:1974ky}
J.~C.~Collins and M.~J.~Perry,
%``Superdense Matter: Neutrons Or Asymptotically Free Quarks?,''
Phys. Rev. Lett. \textbf{34}, 1353 (1975)
%doi:10.1103/PhysRevLett.34.1353
%1326 citations counted in INSPIRE as of 08 Dec 2022


%\cite{Wang:1998bha}
\bibitem{Wang:1998bha}
X.~N.~Wang,
%``Effect of jet quenching on high $p_{T}$ hadron spectra in high-energy nuclear collisions,''
Phys. Rev. C \textbf{58}, 2321 (1998)
%doi:10.1103/PhysRevC.58.2321
[arXiv:hep-ph/9804357 [hep-ph]].
%238 citations counted in INSPIRE as of 12 Dec 2022

%\cite{Wang:2001cs}
\bibitem{Wang:2001cs}
E.~Wang and X.~N.~Wang,
%``Parton energy loss with detailed balance,''
Phys. Rev. Lett. \textbf{87}, 142301 (2001)
%doi:10.1103/PhysRevLett.87.142301
[arXiv:nucl-th/0106043 [nucl-th]].
%120 citations counted in INSPIRE as of 12 Dec 2022

%\cite{Majumder:2004pt}
\bibitem{Majumder:2004pt}
A.~Majumder, E.~Wang and X.~N.~Wang,
%``Modified dihadron fragmentation functions in hot and nuclear matter,''
Phys. Rev. Lett. \textbf{99}, 152301 (2007)
%doi:10.1103/PhysRevLett.99.152301
[arXiv:nucl-th/0412061 [nucl-th]].
%103 citations counted in INSPIRE as of 12 Dec 2022

%\cite{Wang:2002ri}
\bibitem{Wang:2002ri}
E.~Wang and X.~N.~Wang,
%``Jet tomography of dense and nuclear matter,''
Phys. Rev. Lett. \textbf{89}, 162301 (2002)
%doi:10.1103/PhysRevLett.89.162301
[arXiv:hep-ph/0202105 [hep-ph]].
%345 citations counted in INSPIRE as of 12 Dec 2022

%\cite{Zhang:2007ja}
\bibitem{Zhang:2007ja}
H.~Zhang, J.~F.~Owens, E.~Wang and X.~N.~Wang,
%``Dihadron tomography of high-energy nuclear collisions in NLO pQCD,''
Phys. Rev. Lett. \textbf{98}, 212301 (2007)
%doi:10.1103/PhysRevLett.98.212301
[arXiv:nucl-th/0701045 [nucl-th]].
%203 citations counted in INSPIRE as of 12 Dec 2022



%\cite{Vitev:2009rd}
\bibitem{Vitev:2009rd}
  I.~Vitev and B.~W.~Zhang,
  %``Jet tomography of high-energy nucleus-nucleus collisions at next-to-leading order,''
  Phys.\ Rev.\ Lett.\  {\bf 104}, 132001 (2010).
%  doi:10.1103/PhysRevLett.104.132001
%  [arXiv:0910.1090 [hep-ph]].
  %%CITATION = doi:10.1103/PhysRevLett.104.132001;%%

%\cite{He:2020iow}
\bibitem{He:2020iow}
Y.~He, L.~G.~Pang and X.~N.~Wang,
%``Gradient Tomography of Jet Quenching in Heavy-Ion Collisions,''
Phys. Rev. Lett. \textbf{125}, no.12, 122301 (2020)
%doi:10.1103/PhysRevLett.125.122301
[arXiv:2001.08273 [hep-ph]].
%17 citations counted in INSPIRE as of 12 Dec 2022

%\cite{Wang:1996yh}
\bibitem{Wang:1996yh}
X.~N.~Wang, Z.~Huang and I.~Sarcevic,
%``Jet quenching in the opposite direction of a tagged photon in high-energy heavy ion collisions,''
Phys. Rev. Lett. \textbf{77}, 231-234 (1996)
%doi:10.1103/PhysRevLett.77.231
[arXiv:hep-ph/9605213 [hep-ph]].
%320 citations counted in INSPIRE as of 12 Dec 2022

  %\cite{Neufeld:2010fj}
\bibitem{Neufeld:2010fj}
  R.~B.~Neufeld, I.~Vitev and B.-W.~Zhang,
  %``The Physics of $Z^0/\gamma^*$-tagged jets at the LHC,''
  Phys.\ Rev.\ C {\bf 83}, 034902 (2011).
%  doi:10.1103/PhysRevC.83.034902
 % [arXiv:1006.2389 [hep-ph]].
  %%CITATION = doi:10.1103/PhysRevC.83.034902;%%


%\cite{Dai:2012am}
\bibitem{Dai:2012am}
  W.~Dai, I.~Vitev and B.~W.~Zhang,
  %``Momentum imbalance of isolated photon-tagged jet production at RHIC and LHC,''
  Phys.\ Rev.\ Lett.\  {\bf 110} (2013) no.14,  142001
  %doi:10.1103/PhysRevLett.110.142001
  [arXiv:1207.5177 [hep-ph]].
  %%CITATION = doi:10.1103/PhysRevLett.110.142001;%%
  %75 citations counted in INSPIRE as of 15 Oct 2018

  %\cite{Wang:2013cia}
\bibitem{Wang:2013cia}
X.~N.~Wang and Y.~Zhu,
%``Medium Modification of $\gamma$-jets in High-energy Heavy-ion Collisions,''
Phys. Rev. Lett. \textbf{111}, no.6, 062301 (2013)
%doi:10.1103/PhysRevLett.111.062301
[arXiv:1302.5874 [hep-ph]].
%139 citations counted in INSPIRE as of 12 Dec 2022

%\cite{Li:2010ts}
\bibitem{Li:2010ts}
H.~Li, F.~Liu, G.~l.~Ma, X.~N.~Wang and Y.~Zhu,
%``Mach cone induced by $\gamma$-triggered jets in high-energy heavy-ion collisions,''
Phys. Rev. Lett. \textbf{106}, 012301 (2011)
%doi:10.1103/PhysRevLett.106.012301
[arXiv:1006.2893 [nucl-th]].
%75 citations counted in INSPIRE as of 12 Dec 2022

%\cite{Vitev:2008rz}
\bibitem{Vitev:2008rz}
  I.~Vitev, S.~Wicks and B.~W.~Zhang,
  %``A Theory of jet shapes and cross-sections: From hadrons to nuclei,''
  JHEP {\bf 0811}, 093 (2008).
%  doi:10.1088/1126-6708/2008/11/093
 % [arXiv:0810.2807 [hep-ph]].
  %%CITATION = doi:10.1088/1126-6708/2008/11/093;%%

%\cite{Caucal:2021cfb}
\bibitem{Caucal:2021cfb}
P.~Caucal, A.~Soto-Ontoso and A.~Takacs,
%``Dynamically groomed jet radius in heavy-ion collisions,''
Phys. Rev. D \textbf{105}, no.11, 114046 (2022)
%doi:10.1103/PhysRevD.105.114046
[arXiv:2111.14768 [hep-ph]].
%11 citations counted in INSPIRE as of 28 Dec 2022

%\cite{Andronic:2015wma}
\bibitem{Andronic:2015wma}
  A.~Andronic {\it et al.},
  %``Heavy-flavour and quarkonium production in the LHC era: from proton?��Cproton to heavy-ion collisions,''
  Eur.\ Phys.\ J.\ C {\bf 76}, no. 3, 107 (2016).
  %doi:10.1140/epjc/s10052-015-3819-5
 % [arXiv:1506.03981 [nucl-ex]].
  %%CITATION = doi:10.1140/epjc/s10052-015-3819-5;%%
  %287 citations counted in INSPIRE as of 04 May 2019

%\cite{Zhang:2007yoa}
\bibitem{Zhang:2007yoa}
B.~W.~Zhang, C.~M.~Ko and W.~Liu,
%``Thermal charm production in a quark-gluon plasma in Pb-Pb collisions at s**(1/2)(NN) = 5.5-TeV,''
Phys. Rev. C \textbf{77}, 024901 (2008)
% doi:10.1103/PhysRevC.77.024901
[arXiv:0709.1684 [nucl-th]].
%50 citations counted in INSPIRE as of 29 Dec 2022

%%%%%%%%%%%%%%%%%%%%%%%%%%%%%%%% End jet quenching%%%%%


%%%%%%%%%%%%%%%%%%%%%%%%%%%%%%%%%heavy flavor%%%%%%%%%%%%%%%%%%%%%
%%%%%%%%%%%%%%%%experiment%%%%%%%%%%%%%
%%%%%D meson Raa%%%%%%%
%\cite{Adamczyk:2014uip}
\bibitem{Adamczyk:2014uip}
  L.~Adamczyk {\it et al.} [STAR Collaboration],
  %``Observation of $D^0$ Meson Nuclear Modifications in Au+Au Collisions at $\sqrt{s_{NN}}=200$  GeV,''
  Phys.\ Rev.\ Lett.\  {\bf 113}, no. 14, 142301 (2014).
 % Erratum: [Phys.\ Rev.\ Lett.\  {\bf 121}, no. 22, 229901 (2018)]
  %doi:10.1103/PhysRevLett.121.229901, 10.1103/PhysRevLett.113.142301
%  [arXiv:1404.6185 [nucl-ex], arXiv:1809.08737 [nucl-ex]].
  %%CITATION = doi:10.1103/PhysRevLett.121.229901, 10.1103/PhysRevLett.113.142301;%%
  %190 citations counted in INSPIRE as of 18 Apr 2019

%\cite{Adam:2015sza}
\bibitem{Adam:2015sza}
  J.~Adam {\it et al.} [ALICE Collaboration],
  %``Transverse momentum dependence of D-meson production in Pb-Pb collisions at $ \sqrt{{\mathrm{s}}_{\mathrm{NN}}}=$ 2.76  TeV,''
  JHEP {\bf 1603}, 081 (2016).
  %doi:10.1007/JHEP03(2016)081
 % [arXiv:1509.06888 [nucl-ex]].
  %%CITATION = doi:10.1007/JHEP03(2016)081;%%
  %126 citations counted in INSPIRE as of 18 Apr 2019

%\cite{Sirunyan:2017xss}
\bibitem{Sirunyan:2017xss}
  A.~M.~Sirunyan {\it et al.} [CMS Collaboration],
  %``Nuclear modification factor of D$^0$ mesons in PbPb collisions at  $\sqrt{s_\mathrm{NN}} = 5.02$ TeV,''
  Phys.\ Lett.\ B {\bf 782}, 474 (2018)
  %doi:10.1016/j.physletb.2018.05.074
  [arXiv:1708.04962 [nucl-ex]].
  %%CITATION = doi:10.1016/j.physletb.2018.05.074;%%
  %52 citations counted in INSPIRE as of 18 Apr 2019

%\cite{PHENIX:2011img}
\bibitem{PHENIX:2011img}
A.~Adare \textit{et al.} [PHENIX],
%``$J/\psi$ suppression at forward rapidity in Au+Au collisions at $\sqrt{s_{NN}}=200$ GeV,''
Phys. Rev. C \textbf{84}, 054912 (2011)
%doi:10.1103/PhysRevC.84.054912
[arXiv:1103.6269 [nucl-ex]].
%270 citations counted in INSPIRE as of 30 Nov 2022

%\cite{STAR:2013eve}
\bibitem{STAR:2013eve}
L.~Adamczyk \textit{et al.} [STAR],
%``$J/\psi$ production at low $p_T$ in Au + Au and Cu + Cu collisions at $\sqrt{s_{NN}}=200$ GeV with the STAR detector,''
Phys. Rev. C \textbf{90}, no.2, 024906 (2014)
%doi:10.1103/PhysRevC.90.024906
[arXiv:1310.3563 [nucl-ex]].
%95 citations counted in INSPIRE as of 30 Nov 2022

%\cite{ALICE:2014wnc}
\bibitem{ALICE:2014wnc}
B.~B.~Abelev \textit{et al.} [ALICE],
%``Suppression of $\Upsilon (1S)$ at forward rapidity in Pb-Pb collisions at $\sqrt{s_{\rm NN}} = 2.76$ TeV,''
Phys. Lett. B \textbf{738}, 361-372 (2014)
%doi:10.1016/j.physletb.2014.10.001
[arXiv:1405.4493 [nucl-ex]].
%110 citations counted in INSPIRE as of 30 Nov 2022

%\cite{CMS:2011all}
\bibitem{CMS:2011all}
S.~Chatrchyan \textit{et al.} [CMS],
%``Indications of suppression of excited $\Upsilon$ states in PbPb collisions at $\sqrt{S_{NN}}$ = 2.76 TeV,''
Phys. Rev. Lett. \textbf{107}, 052302 (2011)
%doi:10.1103/PhysRevLett.107.052302
[arXiv:1105.4894 [nucl-ex]].
%261 citations counted in INSPIRE as of 30 Nov 2022

%%%%%%%%%%%%%%%%%%%%%v1,v2%%%%%%%%%%%%%%%%%%%%%%%

%\cite{STAR:2019clv}
\bibitem{STAR:2019clv}
J.~Adam \textit{et al.} [STAR],
%``First Observation of the Directed Flow of $D^{0}$ and $\overline{D^0}$ in Au+Au Collisions at $\sqrt{s_{\rm NN}}$ = 200 GeV,''
Phys. Rev. Lett. \textbf{123}, no.16, 162301 (2019)
%doi:10.1103/PhysRevLett.123.162301
[arXiv:1905.02052 [nucl-ex]].
%73 citations counted in INSPIRE as of 30 Nov 2022

%\cite{ALICE:2019sgg}
\bibitem{ALICE:2019sgg}
S.~Acharya \textit{et al.} [ALICE],
%``Probing the effects of strong electromagnetic fields with charge-dependent directed flow in Pb-Pb collisions at the LHC,''
Phys. Rev. Lett. \textbf{125}, no.2, 022301 (2020)
%doi:10.1103/PhysRevLett.125.022301
[arXiv:1910.14406 [nucl-ex]].
%62 citations counted in INSPIRE as of 30 Nov 2022

%\cite{Abelev:2014ipa}
\bibitem{Abelev:2014ipa}
  B.~B.~Abelev {\it et al.} [ALICE Collaboration],
  %``Azimuthal anisotropy of D meson production in Pb-Pb collisions at $\sqrt{s_{\rm NN}} = 2.76$ TeV,''
  Phys.\ Rev.\ C {\bf 90}, no. 3, 034904 (2014)
  %doi:10.1103/PhysRevC.90.034904
  [arXiv:1405.2001 [nucl-ex]].
  %%CITATION = doi:10.1103/PhysRevC.90.034904;%%
  %142 citations counted in INSPIRE as of 18 Apr 2019

%\cite{Adamczyk:2017xur}
\bibitem{Adamczyk:2017xur}
  L.~Adamczyk {\it et al.} [STAR Collaboration],
  %``Measurement of $D^0$ Azimuthal Anisotropy at Midrapidity in Au+Au Collisions at $\sqrt{s_{NN}}$=200  GeV,''
  Phys.\ Rev.\ Lett.\  {\bf 118}, no. 21, 212301 (2017)
  %doi:10.1103/PhysRevLett.118.212301
  %[arXiv:1701.06060 [nucl-ex]].
  %%CITATION = doi:10.1103/PhysRevLett.118.212301;%%
  %57 citations counted in INSPIRE as of 18 Apr 2019

%\cite{Acharya:2017qps}
\bibitem{Acharya:2017qps}
  S.~Acharya {\it et al.} [ALICE Collaboration],
  %``$D$-meson azimuthal anisotropy in midcentral Pb-Pb collisions at $\mathbf{\sqrt{s_{\rm NN}}=5.02}$ TeV,''
  Phys.\ Rev.\ Lett.\  {\bf 120}, no. 10, 102301 (2018)
  %doi:10.1103/PhysRevLett.120.102301
  %[arXiv:1707.01005 [nucl-ex]].
  %%CITATION = doi:10.1103/PhysRevLett.120.102301;%%
  %42 citations counted in INSPIRE as of 18 Apr 2019

%\cite{Sirunyan:2017plt}
\bibitem{Sirunyan:2017plt}
  A.~M.~Sirunyan {\it et al.} [CMS Collaboration],
  %``Measurement of prompt $D^0$ meson azimuthal anisotropy in Pb-Pb collisions at $\sqrt{{s}_{NN}}$ = 5.02 TeV,''
  Phys.\ Rev.\ Lett.\  {\bf 120}, no. 20, 202301 (2018)
  %doi:10.1103/PhysRevLett.120.202301
  %[arXiv:1708.03497 [nucl-ex]].
  %%CITATION = doi:10.1103/PhysRevLett.120.202301;%%
  %40 citations counted in INSPIRE as of 18 Apr 2019


%%%%%%%%%%%%%%%%End experiment%%%%%%%%%%%%%



%%%%%%%%%%%%%%%% models %%%%%%%%%%%%%%%%%%%
%\cite{vanHees:2007me}
\bibitem{vanHees:2007me}
  H.~van Hees, M.~Mannarelli, V.~Greco and R.~Rapp,
  %``Nonperturbative heavy-quark diffusion in the quark-gluon plasma,''
  Phys.\ Rev.\ Lett.\  {\bf 100}, 192301 (2008)
  %doi:10.1103/PhysRevLett.100.192301
  %[arXiv:0709.2884 [hep-ph]].
  %%CITATION = doi:10.1103/PhysRevLett.100.192301;%%
  %205 citations counted in INSPIRE as of 18 Apr 2019


%\cite{CaronHuot:2008uh}
\bibitem{CaronHuot:2008uh}
  S.~Caron-Huot and G.~D.~Moore,
  %``Heavy quark diffusion in QCD and N=4 SYM at next-to-leading order,''
  JHEP {\bf 0802}, 081 (2008)
  %doi:10.1088/1126-6708/2008/02/081
  [arXiv:0801.2173 [hep-ph]].
  %%CITATION = doi:10.1088/1126-6708/2008/02/081;%%
  %97 citations counted in INSPIRE as of 18 Apr 2019

%\cite{Djordjevic:2015hra}
\bibitem{Djordjevic:2015hra}
  M.~Djordjevic and M.~Djordjevic,
  %``Predictions of heavy-flavor suppression at 5.1 TeV Pb + Pb collisions at the CERN Large Hadron Collider,''
  Phys.\ Rev.\ C {\bf 92}, no. 2, 024918 (2015)
  %doi:10.1103/PhysRevC.92.024918
  [arXiv:1505.04316 [nucl-th]].
  %%CITATION = doi:10.1103/PhysRevC.92.024918;%%
  %36 citations counted in INSPIRE as of 18 Apr 2019

%\cite{He:2014cla}
\bibitem{He:2014cla}
  M.~He, R.~J.~Fries and R.~Rapp,
  %``Heavy Flavor at the Large Hadron Collider in a Strong Coupling Approach,''
  Phys.\ Lett.\ B {\bf 735}, 445 (2014)
  %doi:10.1016/j.physletb.2014.05.050
  [arXiv:1401.3817 [nucl-th]].
  %%CITATION = doi:10.1016/j.physletb.2014.05.050;%%
  %106 citations counted in INSPIRE as of 18 Apr 2019

%\cite{Kang:2016ofv}
\bibitem{Kang:2016ofv}
  Z.~B.~Kang, F.~Ringer and I.~Vitev,
  %``Effective field theory approach to open heavy flavor production in heavy-ion collisions,''
  JHEP {\bf 1703}, 146 (2017)
  %doi:10.1007/JHEP03(2017)146
  [arXiv:1610.02043 [hep-ph]].
  %%CITATION = doi:10.1007/JHEP03(2017)146;%%
  %32 citations counted in INSPIRE as of 18 Apr 2019


%%%%%%%%%%%%%%%%%%MC model%%%%%%%%%%%%%%%%%%%%%%

\bibitem{Svetitsky:1987gq}
  B.~Svetitsky,
  %``Diffusion of charmed quarks in the quark-gluon plasma,''
  Phys.\ Rev.\ D {\bf 37} (1988) 2484.
  %doi:10.1103/PhysRevD.37.2484
  %%CITATION = doi:10.1103/PhysRevD.37.2484;%%

%\cite{Moore:2004tg}
\bibitem{Moore:2004tg}
  G.~D.~Moore and D.~Teaney,
  %``How much do heavy quarks thermalize in a heavy ion collision?,''
  Phys.\ Rev.\ C {\bf 71}, 064904 (2005)
  %doi:10.1103/PhysRevC.71.064904
  [hep-ph/0412346].
  %%CITATION = doi:10.1103/PhysRevC.71.064904;%%
  %551 citations counted in INSPIRE as of 27 Aug 2019

%\cite{Cao:2013ita}
\bibitem{Cao:2013ita}
  S.~Cao, G.~Y.~Qin and S.~A.~Bass,
  %``Heavy-quark dynamics and hadronization in ultrarelativistic heavy-ion collisions: Collisional versus radiative energy loss,''
  Phys.\ Rev.\ C {\bf 88}, 044907 (2013)
  %doi:10.1103/PhysRevC.88.044907
  [arXiv:1308.0617 [nucl-th]].
  %%CITATION = doi:10.1103/PhysRevC.88.044907;%%
  %154 citations counted in INSPIRE as of 18 Apr 2019

%\cite{Alberico:2013bza}
\bibitem{Alberico:2013bza}
  W.~M.~Alberico, A.~Beraudo, A.~De Pace, A.~Molinari, M.~Monteno, M.~Nardi, F.~Prino and M.~Sitta,
  %``Heavy flavors in $AA$ collisions: production, transport and final spectra,''
  Eur.\ Phys.\ J.\ C {\bf 73}, 2481 (2013)
  %doi:10.1140/epjc/s10052-013-2481-z
  [arXiv:1305.7421 [hep-ph]].
  %%CITATION = doi:10.1140/epjc/s10052-013-2481-z;%%
  %80 citations counted in INSPIRE as of 04 May 2019

%\cite{Xu:2015bbz}
\bibitem{Xu:2015bbz}
  J.~Xu, J.~Liao and M.~Gyulassy,
  %``Bridging Soft-Hard Transport Properties of Quark-Gluon Plasmas with CUJET3.0,''
  JHEP {\bf 1602}, 169 (2016)
  %doi:10.1007/JHEP02(2016)169
  [arXiv:1508.00552 [hep-ph]].
  %%CITATION = doi:10.1007/JHEP02(2016)169;%%
  %69 citations counted in INSPIRE as of 18 Apr 2019

%\cite{Cao:2016gvr}
\bibitem{Cao:2016gvr}
  S.~Cao, T.~Luo, G.~Y.~Qin and X.~N.~Wang,
  %``Linearized Boltzmann transport model for jet propagation in the quark-gluon plasma: Heavy quark evolution,''
  Phys.\ Rev.\ C {\bf 94}, no. 1, 014909 (2016).
  %doi:10.1103/PhysRevC.94.014909
%  [arXiv:1605.06447 [nucl-th]].
  %%CITATION = doi:10.1103/PhysRevC.94.014909;%%
  %67 citations counted in INSPIRE as of 18 Apr 2019

%\cite{Das:2016cwd}
\bibitem{Das:2016cwd}
  S.~K.~Das, S.~Plumari, S.~Chatterjee, J.~Alam, F.~Scardina and V.~Greco,
  %``Directed Flow of Charm Quarks as a Witness of the Initial Strong Magnetic Field in Ultra-Relativistic Heavy Ion Collisions,''
  Phys.\ Lett.\ B {\bf 768}, 260 (2017)
  %doi:10.1016/j.physletb.2017.02.046
  %[arXiv:1608.02231 [nucl-th]].
  %%CITATION = doi:10.1016/j.physletb.2017.02.046;%%
  %46 citations counted in INSPIRE as of 18 Apr 2019

%\cite{Ke:2018tsh}
\bibitem{Ke:2018tsh}
  W.~Ke, Y.~Xu and S.~A.~Bass,
  %``Linearized Boltzmann-Langevin model for heavy quark transport in hot and dense QCD matter,''
  Phys.\ Rev.\ C {\bf 98}, no. 6, 064901 (2018).
  %doi:10.1103/PhysRevC.98.064901
%  [arXiv:1806.08848 [nucl-th]].
  %%CITATION = doi:10.1103/PhysRevC.98.064901;%%
  %12 citations counted in INSPIRE as of 04 Jun 2019


  %\cite{Li:2020kax}
\bibitem{Li:2020kax}
S.~Q.~Li, W.~J.~Xing, F.~L.~Liu, S.~Cao and G.~Y.~Qin,
%``Heavy flavor quenching and flow: the roles of initial condition, pre-equilibrium evolution, and in-medium interaction,''
Chin. Phys. C \textbf{44}, no.11, 114101 (2020)
%doi:10.1088/1674-1137/abadee
[arXiv:2005.03330 [nucl-th]].
%16 citations counted in INSPIRE as of 10 Dec 2022

  %\cite{Yan:2006ve}
\bibitem{Yan:2006ve}
L.~Yan, P.~Zhuang and N.~Xu,
%``Competition between J / psi suppression and regeneration in quark-gluon plasma,''
Phys. Rev. Lett. \textbf{97}, 232301 (2006)
%doi:10.1103/PhysRevLett.97.232301
[arXiv:nucl-th/0608010 [nucl-th]].
%196 citations counted in INSPIRE as of 29 Nov 2022

%\cite{Liu:2009nb}
\bibitem{Liu:2009nb}
Y.~p.~Liu, Z.~Qu, N.~Xu and P.~f.~Zhuang,
%``J/psi Transverse Momentum Distribution in High Energy Nuclear Collisions at RHIC,''
Phys. Lett. B \textbf{678}, 72-76 (2009)
%doi:10.1016/j.physletb.2009.06.006
[arXiv:0901.2757 [nucl-th]].
%207 citations counted in INSPIRE as of 29 Nov 2022

%\cite{Zhang:2003wk}
\bibitem{Zhang:2003wk}
  B.~W.~Zhang, E.~Wang and X.~N.~Wang,
  %``Heavy quark energy loss in nuclear medium,''
  Phys.\ Rev.\ Lett.\  {\bf 93} (2004) 072301
  %doi:10.1103/PhysRevLett.93.072301
  [nucl-th/0309040].
  %%CITATION = doi:10.1103/PhysRevLett.93.072301;%%
  %184 citations counted in INSPIRE as of 17 Oct 2018

%\cite{Djordjevic:2003zk}
\bibitem{Djordjevic:2003zk}
M.~Djordjevic and M.~Gyulassy,
%``Heavy quark radiative energy loss in QCD matter,''
Nucl. Phys. A \textbf{733}, 265-298 (2004)
%doi:10.1016/j.nuclphysa.2003.12.020
[arXiv:nucl-th/0310076 [nucl-th]].
%327 citations counted in INSPIRE as of 28 Dec 2022

%\cite{Zhang:2004qm}
\bibitem{Zhang:2004qm}
B.~W.~Zhang, E.~k.~Wang and X.~N.~Wang,
%``Multiple parton scattering in nuclei: Heavy quark energy loss and modified fragmentation functions,''
Nucl. Phys. A \textbf{757}, 493-524 (2005)
%doi:10.1016/j.nuclphysa.2005.04.022
[arXiv:hep-ph/0412060 [hep-ph]].
%41 citations counted in INSPIRE as of 28 Dec 2022



%\cite{JETSCAPE:2022hcb}
\bibitem{JETSCAPE:2022hcb}
W.~Fan \textit{et al.} [JETSCAPE],
%``Multi-scale evolution of charmed particles in a nuclear medium,''
[arXiv:2208.00983 [nucl-th]].
%1 citations counted in INSPIRE as of 08 Dec 2022

%\cite{Liu:2021dpm}
\bibitem{Liu:2021dpm}
F.~L.~Liu, W.~J.~Xing, X.~Y.~Wu, G.~Y.~Qin, S.~Cao and X.~N.~Wang,
%``QLBT: a linear Boltzmann transport model for heavy quarks in a quark-gluon plasma of quasi-particles,''
Eur. Phys. J. C \textbf{82}, no.4, 350 (2022)
%doi:10.1140/epjc/s10052-022-10308-x
[arXiv:2107.11713 [hep-ph]].
%9 citations counted in INSPIRE as of 08 Dec 2022

%\cite{Ding:2021ajz}
\bibitem{Ding:2021ajz}
C.~Ding, W.~Y.~Ke, L.~G.~Pang and X.~N.~Wang,
%``Hydrodynamic description of D meson production in high-energy heavy-ion collisions,''
Chin. Phys. C \textbf{45}, no.7, 074102 (2021)
%doi:10.1088/1674-1137/abf645
[arXiv:2101.02356 [nucl-th]].
%5 citations counted in INSPIRE as of 08 Dec 2022

 %\cite{Cao:2019iqs}
\bibitem{Cao:2019iqs}
  S.~Cao, K.~J.~Sun, S.~Q.~Li, S.~Y.~F.~Liu, W.~J.~Xing, G.~Y.~Qin and C.~M.~Ko,
  %``Charmed hadron chemistry in relativistic heavy-ion collisions,''
  Phys.\ Lett.\ B {\bf 807}, 135561 (2020)
  %doi:10.1016/j.physletb.2020.135561
  [arXiv:1911.00456 [nucl-th]].
  %%CITATION = doi:10.1016/j.physletb.2020.135561;%%
  %4 citations counted in INSPIRE as of 25 Jul 2020

%\cite{Plumari:2017ntm}
\bibitem{Plumari:2017ntm}
S.~Plumari, V.~Minissale, S.~K.~Das, G.~Coci and V.~Greco,
%``Charmed Hadrons from Coalescence plus Fragmentation in relativistic nucleus-nucleus collisions at RHIC and LHC,''
Eur. Phys. J. C \textbf{78}, no.4, 348 (2018)
%doi:10.1140/epjc/s10052-018-5828-7
[arXiv:1712.00730 [hep-ph]].
%131 citations counted in INSPIRE as of 29 Nov 2022

%\cite{He:2019vgs}
\bibitem{He:2019vgs}
M.~He and R.~Rapp,
%``Hadronization and Charm-Hadron Ratios in Heavy-Ion Collisions,''
Phys. Rev. Lett. \textbf{124}, no.4, 042301 (2020)
%doi:10.1103/PhysRevLett.124.042301
[arXiv:1905.09216 [nucl-th]].
%67 citations counted in INSPIRE as of 29 Nov 2022
%%%%%%%%%%%%%%%% End models %%%%%%%%%%%%%%%%%%%

%%%%%%%%%%%%%%%%%%%%review%%%%%%%%%%%%%%%%
%\cite{Rapp:2009my}
\bibitem{Rapp:2009my}
R.~Rapp and H.~van Hees,
%``Heavy Quarks in the Quark-Gluon Plasma,''
%doi:10.1142/9789814293297\_0003
[arXiv:0903.1096 [hep-ph]].
%162 citations counted in INSPIRE as of 30 Nov 2022

%\cite{Prino:2016cni}
\bibitem{Prino:2016cni}
F.~Prino and R.~Rapp,
%``Open Heavy Flavor in QCD Matter and in Nuclear Collisions,''
J. Phys. G \textbf{43}, no.9, 093002 (2016)
%doi:10.1088/0954-3899/43/9/093002
[arXiv:1603.00529 [nucl-ex]].
%145 citations counted in INSPIRE as of 30 Nov 2022


%\cite{Rapp:2018qla}
\bibitem{Rapp:2018qla}
R.~Rapp, \textit{et al.}
%``Extraction of Heavy-Flavor Transport Coefficients in QCD Matter,''
Nucl. Phys. A \textbf{979} (2018), 21-86
%doi:10.1016/j.nuclphysa.2018.09.002
[arXiv:1803.03824 [nucl-th]].
%99 citations counted in INSPIRE as of 22 Jul 2021

%\cite{Dong:2019unq}
\bibitem{Dong:2019unq}
  X.~Dong and V.~Greco,
  %``Heavy quark production and properties of Quark?��CGluon Plasma,''
  Prog.\ Part.\ Nucl.\ Phys.\  {\bf 104}, 97 (2019).
  %doi:10.1016/j.ppnp.2018.08.001
  %%CITATION = doi:10.1016/j.ppnp.2018.08.001;%%
  %5 citations counted in INSPIRE as of 04 May 2019

%\cite{Dong:2019byy}
\bibitem{Dong:2019byy}
X.~Dong, Y.~J.~Lee and R.~Rapp,
%``Open Heavy-Flavor Production in Heavy-Ion Collisions,''
Ann. Rev. Nucl. Part. Sci. \textbf{69}, 417-445 (2019)
%doi:10.1146/annurev-nucl-101918-023806
[arXiv:1903.07709 [nucl-ex]].
%87 citations counted in INSPIRE as of 30 Nov 2022

%\cite{Zhao:2020jqu}
\bibitem{Zhao:2020jqu}
J.~Zhao, K.~Zhou, S.~Chen and P.~Zhuang,
%``Heavy flavors under extreme conditions in high energy nuclear collisions,''
Prog. Part. Nucl. Phys. \textbf{114}, 103801 (2020)
%doi:10.1016/j.ppnp.2020.103801
[arXiv:2005.08277 [nucl-th]].
%51 citations counted in INSPIRE as of 30 Nov 2022

%\cite{Apolinario:2022vzg}
\bibitem{Apolinario:2022vzg}
L.~Apolin\'ario, Y.~J.~Lee and M.~Winn,
%``Heavy quarks and jets as probes of the QGP,''
Prog. Part. Nucl. Phys. \textbf{127}, 103990 (2022)
%doi:10.1016/j.ppnp.2022.103990
[arXiv:2203.16352 [hep-ph]].
%8 citations counted in INSPIRE as of 30 Nov 2022

%\cite{He:2022ywp}
\bibitem{He:2022ywp}
M.~He, H.~van Hees and R.~Rapp,
%``Heavy-Quark Diffusion in the Quark-Gluon Plasma,''
[arXiv:2204.09299 [hep-ph]].
%3 citations counted in INSPIRE as of 30 Nov 2022

%\cite{Tang:2020ame}
\bibitem{Tang:2020ame}
Z.~Tang, Z.~B.~Tang, W.~Zha, W.~M.~Zha, Y.~Zhang and Y.~F.~Zhang,
%``An experimental review of open heavy flavor and quarkonium production at RHIC,''
Nucl. Sci. Tech. \textbf{31}, no.8, 81 (2020)
%doi:10.1007/s41365-020-00785-8
[arXiv:2105.11656 [nucl-ex]].
%23 citations counted in INSPIRE as of 01 Dec 2022

%\cite{Li:2018xuv}
\bibitem{Li:2018xuv}
H.~T.~Li and I.~Vitev,
%``Inclusive heavy flavor jet production with semi-inclusive jet functions: from proton to heavy-ion collisions,''
JHEP \textbf{07} (2019), 148
%doi:10.1007/JHEP07(2019)148
[arXiv:1811.07905 [hep-ph]].
%37 citations counted in INSPIRE as of 10 Feb 2023

%\cite{Xu:2014tda}
\bibitem{Xu:2014tda}
J.~Xu, J.~Liao and M.~Gyulassy,
%``Consistency of Perfect Fluidity and Jet Quenching in semi-Quark-Gluon Monopole Plasmas,''
Chin. Phys. Lett. \textbf{32} (2015) no.9, 092501
%doi:10.1088/0256-307X/32/9/092501
[arXiv:1411.3673 [hep-ph]].
%121 citations counted in INSPIRE as of 10 Feb 2023

%\cite{Djordjevic:2009cr}
\bibitem{Djordjevic:2009cr}
M.~Djordjevic,
%``Theoretical formalism of radiative jet energy loss in a finite size dynamical QCD medium,''
Phys. Rev. C \textbf{80} (2009), 064909
%doi:10.1103/PhysRevC.80.064909
[arXiv:0903.4591 [nucl-th]].
%86 citations counted in INSPIRE as of 04 Feb 2023

%\cite{Djordjevic:2008iz}
\bibitem{Djordjevic:2008iz}
M.~Djordjevic and U.~W.~Heinz,
%``Radiative energy loss in a finite dynamical QCD medium,''
Phys. Rev. Lett. \textbf{101} (2008), 022302
%doi:10.1103/PhysRevLett.101.022302
[arXiv:0802.1230 [nucl-th]].
%126 citations counted in INSPIRE as of 04 Feb 2023

%\cite{Djordjevic:2013xoa}
\bibitem{Djordjevic:2013xoa}
M.~Djordjevic and M.~Djordjevic,
%``LHC jet suppression of light and heavy flavor observables,''
Phys. Lett. B \textbf{734} (2014), 286-289
%doi:10.1016/j.physletb.2014.05.053
[arXiv:1307.4098 [hep-ph]].
%89 citations counted in INSPIRE as of 04 Feb 2023

%\cite{Zigic:2021rku}
\bibitem{Zigic:2021rku}
D.~Zigic, I.~Salom, J.~Auvinen, P.~Huovinen and M.~Djordjevic,
%``DREENA-A framework as a QGP tomography tool,''
Front. in Phys. \textbf{10} (2022), 957019
%doi:10.3389/fphy.2022.957019
[arXiv:2110.01544 [nucl-th]].
%7 citations counted in INSPIRE as of 04 Feb 2023

%\cite{Zigic:2018ovr}
\bibitem{Zigic:2018ovr}
D.~Zigic, I.~Salom, J.~Auvinen, M.~Djordjevic and M.~Djordjevic,
%``DREENA-B framework: first predictions of $R_{AA}$ and $v_2$ within dynamical energy loss formalism in evolving QCD medium,''
Phys. Lett. B \textbf{791} (2019), 236-241
%doi:10.1016/j.physletb.2019.02.020
[arXiv:1805.04786 [nucl-th]].
%51 citations counted in INSPIRE as of 04 Feb 2023

%\cite{Zigic:2018smz}
\bibitem{Zigic:2018smz}
D.~Zigic, I.~Salom, J.~Auvinen, M.~Djordjevic and M.~Djordjevic,
%``DREENA-C framework: joint $R_{AA}$ and $v_2$ predictions and implications to QGP tomography,''
J. Phys. G \textbf{46} (2019) no.8, 085101
%doi:10.1088/1361-6471/ab2356
[arXiv:1805.03494 [nucl-th]].
%22 citations counted in INSPIRE as of 04 Feb 2023

%\cite{Wicks:2007am}
\bibitem{Wicks:2007am}
S.~Wicks, W.~Horowitz, M.~Djordjevic and M.~Gyulassy,
%``Heavy quark jet quenching with collisional plus radiative energy loss and path length fluctuations,''
Nucl. Phys. A \textbf{783} (2007), 493-496
%doi:10.1016/j.nuclphysa.2006.11.102
[arXiv:nucl-th/0701063 [nucl-th]].
%96 citations counted in INSPIRE as of 04 Feb 2023

%\cite{Wicks:2005gt}
\bibitem{Wicks:2005gt}
S.~Wicks, W.~Horowitz, M.~Djordjevic and M.~Gyulassy,
%``Elastic, inelastic, and path length fluctuations in jet tomography,''
Nucl. Phys. A \textbf{784} (2007), 426-442
%doi:10.1016/j.nuclphysa.2006.12.048
[arXiv:nucl-th/0512076 [nucl-th]].
%588 citations counted in INSPIRE as of 04 Feb 2023

%\cite{Horowitz:2007su}
\bibitem{Horowitz:2007su}
W.~A.~Horowitz and M.~Gyulassy,
%``Heavy quark jet tomography of Pb + Pb at LHC: AdS/CFT drag or pQCD energy loss?,''
Phys. Lett. B \textbf{666} (2008), 320-323
%doi:10.1016/j.physletb.2008.04.065
[arXiv:0706.2336 [nucl-th]].
%114 citations counted in INSPIRE as of 04 Feb 2023

%\cite{Horowitz:2011wm}
\bibitem{Horowitz:2011wm}
W.~A.~Horowitz,
%``Testing pQCD and AdS/CFT Energy Loss at RHIC and LHC,''
AIP Conf. Proc. \textbf{1441} (2012) no.1, 889-891
%doi:10.1063/1.3700710
[arXiv:1108.5876 [hep-ph]].
%59 citations counted in INSPIRE as of 04 Feb 2023

%\cite{Uphoff:2010sh}
\bibitem{Uphoff:2010sh}
J.~Uphoff, O.~Fochler, Z.~Xu and C.~Greiner,
%``Heavy quark production at RHIC and LHC within a partonic transport model,''
Phys. Rev. C \textbf{82} (2010), 044906
%doi:10.1103/PhysRevC.82.044906
[arXiv:1003.4200 [hep-ph]].
%90 citations counted in INSPIRE as of 04 Feb 2023

%\cite{Uphoff:2011ad}
\bibitem{Uphoff:2011ad}
J.~Uphoff, O.~Fochler, Z.~Xu and C.~Greiner,
%``Elliptic Flow and Energy Loss of Heavy Quarks in Ultra-Relativistic heavy Ion Collisions,''
Phys. Rev. C \textbf{84} (2011), 024908
%doi:10.1103/PhysRevC.84.024908
[arXiv:1104.2295 [hep-ph]].
%174 citations counted in INSPIRE as of 04 Feb 2023

%\cite{Uphoff:2013rka}
\bibitem{Uphoff:2013rka}
J.~Uphoff, F.~Senzel, Z.~Xu and C.~Greiner,
%``Momentum imbalance of D mesons in ultra-relativistic heavy-ion collisions at LHC,''
Phys. Rev. C \textbf{89} (2014) no.6, 064906
%doi:10.1103/PhysRevC.89.064906
[arXiv:1310.1340 [hep-ph]].
%19 citations counted in INSPIRE as of 04 Feb 2023

%\cite{Uphoff:2014hza}
\bibitem{Uphoff:2014hza}
J.~Uphoff, O.~Fochler, Z.~Xu and C.~Greiner,
%``Elastic and radiative heavy quark interactions in ultra-relativistic heavy-ion collisions,''
J. Phys. G \textbf{42} (2015) no.11, 115106
%doi:10.1088/0954-3899/42/11/115106
[arXiv:1408.2964 [hep-ph]].
%72 citations counted in INSPIRE as of 04 Feb 2023

%\cite{Gossiaux:2008jv}
\bibitem{Gossiaux:2008jv}
P.~B.~Gossiaux and J.~Aichelin,
%``Towards an understanding of the RHIC single electron data,''
Phys. Rev. C \textbf{78} (2008), 014904
%doi:10.1103/PhysRevC.78.014904
[arXiv:0802.2525 [hep-ph]].
%247 citations counted in INSPIRE as of 04 Feb 2023

%\cite{Gossiaux:2009mk}
\bibitem{Gossiaux:2009mk}
P.~B.~Gossiaux, R.~Bierkandt and J.~Aichelin,
%``Tomography of a quark gluon plasma at RHIC and LHC energies,''
Phys. Rev. C \textbf{79} (2009), 044906
%doi:10.1103/PhysRevC.79.044906
[arXiv:0901.0946 [hep-ph]].
%134 citations counted in INSPIRE as of 04 Feb 2023

%\cite{Nahrgang:2013saa}
\bibitem{Nahrgang:2013saa}
M.~Nahrgang, J.~Aichelin, P.~B.~Gossiaux and K.~Werner,
%``Azimuthal correlations of heavy quarks in Pb + Pb collisions at $\sqrt{s}=2.76$ TeV at the CERN Large Hadron Collider,''
Phys. Rev. C \textbf{90} (2014) no.2, 024907
%doi:10.1103/PhysRevC.90.024907
[arXiv:1305.3823 [hep-ph]].
%119 citations counted in INSPIRE as of 04 Feb 2023

%\cite{Ke:2020clc}
\bibitem{Ke:2020clc}
W.~Ke and X.~N.~Wang,
%``QGP modification to single inclusive jets in a calibrated transport model,''
JHEP \textbf{05} (2021), 041
%doi:10.1007/JHEP05(2021)041
[arXiv:2010.13680 [hep-ph]].
%24 citations counted in INSPIRE as of 10 Feb 2023

%\cite{Plumari:2012ep}
\bibitem{Plumari:2012ep}
S.~Plumari, A.~Puglisi, F.~Scardina and V.~Greco,
%``Shear Viscosity of a strongly interacting system: Green-Kubo vs. Chapman-Enskog and Relaxation Time Approximation,''
Phys. Rev. C \textbf{86} (2012), 054902
%doi:10.1103/PhysRevC.86.054902
[arXiv:1208.0481 [nucl-th]].
%135 citations counted in INSPIRE as of 04 Feb 2023

%\cite{Das:2015ana}
\bibitem{Das:2015ana}
S.~K.~Das, F.~Scardina, S.~Plumari and V.~Greco,
%``Toward a solution to the $R_{AA}$ and $v_2$ puzzle for heavy quarks,''
Phys. Lett. B \textbf{747} (2015), 260-264
%doi:10.1016/j.physletb.2015.06.003
[arXiv:1502.03757 [nucl-th]].
%156 citations counted in INSPIRE as of 04 Feb 2023

%\cite{Plumari:2011mk}
\bibitem{Plumari:2011mk}
S.~Plumari, W.~M.~Alberico, V.~Greco and C.~Ratti,
%``Recent thermodynamic results from lattice QCD analyzed within a quasi-particle model,''
Phys. Rev. D \textbf{84} (2011), 094004
%doi:10.1103/PhysRevD.84.094004
[arXiv:1103.5611 [hep-ph]].
%135 citations counted in INSPIRE as of 04 Feb 2023

%\cite{Scardina:2017ipo}
\bibitem{Scardina:2017ipo}
F.~Scardina, S.~K.~Das, V.~Minissale, S.~Plumari and V.~Greco,
%``Estimating the charm quark diffusion coefficient and thermalization time from D meson spectra at energies available at the BNL Relativistic Heavy Ion Collider and the CERN Large Hadron Collider,''
Phys. Rev. C \textbf{96} (2017) no.4, 044905
%doi:10.1103/PhysRevC.96.044905
[arXiv:1707.05452 [nucl-th]].
%107 citations counted in INSPIRE as of 04 Feb 2023

%\cite{Alberico:2011zy}
\bibitem{Alberico:2011zy}
W.~M.~Alberico, A.~Beraudo, A.~De Pace, A.~Molinari, M.~Monteno, M.~Nardi and F.~Prino,
%``Heavy-flavour spectra in high energy nucleus-nucleus collisions,''
Eur. Phys. J. C \textbf{71} (2011), 1666
%doi:10.1140/epjc/s10052-011-1666-6
[arXiv:1101.6008 [hep-ph]].
%196 citations counted in INSPIRE as of 04 Feb 2023

%\cite{Beraudo:2014boa}
\bibitem{Beraudo:2014boa}
A.~Beraudo, A.~De Pace, M.~Monteno, M.~Nardi and F.~Prino,
%``Heavy flavors in heavy-ion collisions: quenching, flow and correlations,''
Eur. Phys. J. C \textbf{75} (2015) no.3, 121
%doi:10.1140/epjc/s10052-015-3336-6
[arXiv:1410.6082 [hep-ph]].
%92 citations counted in INSPIRE as of 04 Feb 2023

%\cite{Cao:2011et}
\bibitem{Cao:2011et}
S.~Cao and S.~A.~Bass,
%``Thermalization of charm quarks in infinite and finite QGP matter,''
Phys. Rev. C \textbf{84} (2011), 064902
%doi:10.1103/PhysRevC.84.064902
[arXiv:1108.5101 [nucl-th]].
%76 citations counted in INSPIRE as of 04 Feb 2023

%\cite{Lang:2012nqy}
\bibitem{Lang:2012nqy}
T.~Lang, H.~van Hees, J.~Steinheimer, G.~Inghirami and M.~Bleicher,
%``Heavy quark transport in heavy ion collisions at energies available at the BNL Relativistic Heavy Ion Collider and at the CERN Large Hadron Collider within the UrQMD hybrid model,''
Phys. Rev. C \textbf{93} (2016) no.1, 014901
%doi:10.1103/PhysRevC.93.014901
[arXiv:1211.6912 [hep-ph]].
%77 citations counted in INSPIRE as of 04 Feb 2023

%\cite{Lang:2013cca}
\bibitem{Lang:2013cca}
T.~Lang, H.~van Hees, J.~Steinheimer and M.~Bleicher,
%``Elliptic flow and nuclear modification factors of $D$-mesons at FAIR in a Hybrid-Langevin approach,''
[arXiv:1305.1797 [hep-ph]].
%12 citations counted in INSPIRE as of 04 Feb 2023

%\cite{Lang:2013wya}
\bibitem{Lang:2013wya}
T.~Lang, H.~van Hees, J.~Steinheimer and M.~Bleicher,
%``Dileptons from correlated D- and $\bar{\text{D}}$-meson decays in the invariant mass range of the QGP thermal radiation using the UrQMD hybrid model,''
[arXiv:1305.7377 [hep-ph]].
%9 citations counted in INSPIRE as of 04 Feb 2023

%\cite{He:2012df}
\bibitem{He:2012df}
M.~He, R.~J.~Fries and R.~Rapp,
%``$\mathbf{D_s}$-Meson as Quantitative Probe of Diffusion and Hadronization in Nuclear Collisions,''
Phys. Rev. Lett. \textbf{110} (2013) no.11, 112301
%doi:10.1103/PhysRevLett.110.112301
[arXiv:1204.4442 [nucl-th]].
%236 citations counted in INSPIRE as of 04 Feb 2023

%\cite{He:2011yi}
\bibitem{He:2011yi}
M.~He, R.~J.~Fries and R.~Rapp,
%``Thermal Relaxation of Charm in Hadronic Matter,''
Phys. Lett. B \textbf{701} (2011), 445-450
%doi:10.1016/j.physletb.2011.06.019
[arXiv:1103.6279 [nucl-th]].
%88 citations counted in INSPIRE as of 04 Feb 2023

%\cite{Wang:2019xey}
\bibitem{Wang:2019xey}
  S.~Wang, W.~Dai, B.~W.~Zhang and E.~Wang,
  %``Diffusion of charm quarks in jets in high-energy heavy-ion collisions,''
  Eur.\ Phys.\ J.\ C {\bf 79} (2019) no.9,  789
  %doi:10.1140/epjc/s10052-019-7312-4
  [arXiv:1906.01499 [nucl-th]].
  %%CITATION = doi:10.1140/epjc/s10052-019-7312-4;%%
  %6 citations counted in INSPIRE as of 06 Jul 2020

%\cite{Dai:2018mhw}
\bibitem{Dai:2018mhw}
  W.~Dai, S.~Wang, S.~L.~Zhang, B.~W.~Zhang and E.~Wang,
  %``Transverse Momentum Balance and Angular Distribution of $b\bar{b}$ Dijets in Pb+Pb collisions,''
  Chin.\ Phys.\ C {\bf 44} (2020) no.10,  104105
  arXiv:1806.06332 [nucl-th];
  %%CITATION = ARXIV:1806.06332;%%

%\cite{Wang:2021jgm}
\bibitem{Wang:2021jgm}
S.~Wang, J.~W.~Kang, W.~Dai, B.~W.~Zhang and E.~Wang,
%``Probing the in-medium $P_T$-broadening by $\gamma $+HF angular de-correlations,''
Eur. Phys. J. A \textbf{58}, no.7, 135 (2022)
%doi:10.1140/epja/s10050-022-00785-9
[arXiv:2107.12000 [nucl-th]].
%2 citations counted in INSPIRE as of 30 Nov 2022


%\cite{Cassing:2008sv}
\bibitem{Cassing:2008sv}
W.~Cassing and E.~L.~Bratkovskaya,
%``Parton transport and hadronization from the dynamical quasiparticle point of view,''
Phys. Rev. C \textbf{78} (2008), 034919
%doi:10.1103/PhysRevC.78.034919
[arXiv:0808.0022 [hep-ph]].
%183 citations counted in INSPIRE as of 04 Feb 2023

%\cite{Cassing:2009vt}
\bibitem{Cassing:2009vt}
W.~Cassing and E.~L.~Bratkovskaya,
%``Parton-Hadron-String Dynamics: an off-shell transport approach for relativistic energies,''
Nucl. Phys. A \textbf{831} (2009), 215-242
%doi:10.1016/j.nuclphysa.2009.09.007
[arXiv:0907.5331 [nucl-th]].
%312 citations counted in INSPIRE as of 04 Feb 2023

%\cite{Bratkovskaya:2011wp}
\bibitem{Bratkovskaya:2011wp}
E.~L.~Bratkovskaya, W.~Cassing, V.~P.~Konchakovski and O.~Linnyk,
%``Parton-Hadron-String Dynamics at Relativistic Collider Energies,''
Nucl. Phys. A \textbf{856} (2011), 162-182
%doi:10.1016/j.nuclphysa.2011.03.003
[arXiv:1101.5793 [nucl-th]].
%185 citations counted in INSPIRE as of 04 Feb 2023

%\cite{STAR:2019ank}
\bibitem{STAR:2019ank}
J.~Adam \textit{et al.} [STAR],
%``First measurement of $\Lambda_c$ baryon production in Au+Au collisions at $\sqrt{s_{\rm NN}}$ = 200 GeV,''
Phys. Rev. Lett. \textbf{124} (2020) no.17, 172301
%doi:10.1103/PhysRevLett.124.172301
[arXiv:1910.14628 [nucl-ex]].
%61 citations counted in INSPIRE as of 04 Feb 2023

%\cite{Vermunt:2019ecg}
\bibitem{Vermunt:2019ecg}
L.~Vermunt [ALICE],
%``Measurement of $\Lambda_{\rm c}$ baryons and D$_{\rm s}^{+}$ mesons in Pb\textendash{}Pb collisions with ALICE,''
PoS \textbf{EPS-HEP2019} (2020), 297
%doi:10.22323/1.364.0297
[arXiv:1910.11738 [nucl-ex]].
%6 citations counted in INSPIRE as of 04 Feb 2023

%%%%%%%%%%%%%%%%%%%%review%%%%%%%%%%%%%%%%

%%%%%%%%%%%%%%%% HQ jet in expriment%%%%%%%%%%%
%\cite{CMS:2016wma}
\bibitem{CMS:2016wma}
A.~M.~Sirunyan \textit{et al.} [CMS],
%``Measurements of the charm jet cross-section and nuclear modification factor in pPb collisions at $\sqrt{{s}_{NN}}$ = 5.02 TeV,''
Phys. Lett. B \textbf{772}, 306-329 (2017)
%doi:10.1016/j.physletb.2017.06.053
[arXiv:1612.08972 [nucl-ex]].
%30 citations counted in INSPIRE as of 29 Nov 2022

%\cite{ALICE:2019cbr}
\bibitem{ALICE:2019cbr}
S.~Acharya \textit{et al.} [ALICE],
%``Measurement of the production of charm jets tagged with D$^{0}$ mesons in pp collisions at $ \sqrt{\mathrm{s}}=7 $ TeV,''
JHEP \textbf{08}, 133 (2019)
%doi:10.1007/JHEP08(2019)133
[arXiv:1905.02510 [nucl-ex]].
%37 citations counted in INSPIRE as of 29 Nov 2022

%\cite{CMS:2012pgw}
\bibitem{CMS:2012pgw}
S.~Chatrchyan \textit{et al.} [CMS],
%``Inclusive $b$-jet production in $pp$ collisions at $\sqrt{s}=7$ TeV,''
JHEP \textbf{04}, 084 (2012)
%doi:10.1007/JHEP04(2012)084
[arXiv:1202.4617 [hep-ex]].
%103 citations counted in INSPIRE as of 29 Nov 2022

%\cite{ATLAS:2011chi}
\bibitem{ATLAS:2011chi}
G.~Aad \textit{et al.} [ATLAS],
%``Measurement of $D^{*+/-}$ meson production in jets from pp collisions at sqrt(s) = 7 TeV with the ATLAS detector,''
Phys. Rev. D \textbf{85}, 052005 (2012)
%doi:10.1103/PhysRevD.85.052005
[arXiv:1112.4432 [hep-ex]].
%64 citations counted in INSPIRE as of 29 Nov 2022

%\cite{ATLAS:2021agf}
\bibitem{ATLAS:2021agf}
G.~Aad \textit{et al.} [ATLAS],
%``Measurement of $b$-quark fragmentation properties in jets using the decay $B^{\pm} \to J/\psi K^{\pm}$ in $pp$ collisions at $ \sqrt{s} $ = 13 TeV with the ATLAS detector,''
JHEP \textbf{12}, 131 (2021)
%doi:10.1007/JHEP12(2021)131
[arXiv:2108.11650 [hep-ex]].
%10 citations counted in INSPIRE as of 29 Nov 2022

%\cite{STAR:2009kkp}
\bibitem{STAR:2009kkp}
B.~I.~Abelev \textit{et al.} [STAR],
%``Measurement of D* Mesons in Jets from p+p Collisions at s**(1/2) = 200-GeV,''
Phys. Rev. D \textbf{79}, 112006 (2009)
%doi:10.1103/PhysRevD.79.112006
[arXiv:0901.0740 [hep-ex]].
%27 citations counted in INSPIRE as of 29 Nov 2022

%\cite{ATLAS:2016anw}
\bibitem{ATLAS:2016anw}
M.~Aaboud \textit{et al.} [ATLAS],
%``Measurement of the $b\overline{b}$ dijet cross-section in pp collisions at $\sqrt{s} = 7$  TeV with the ATLAS detector,''
Eur. Phys. J. C \textbf{76}, no.12, 670 (2016)
%doi:10.1140/epjc/s10052-016-4521-y
[arXiv:1607.08430 [hep-ex]].
%23 citations counted in INSPIRE as of 29 Nov 2022

%\cite{CMS:2020geg}
\bibitem{CMS:2020geg}
A.~M.~Sirunyan \textit{et al.} [CMS],
%``Measurement of b jet shapes in proton-proton collisions at $\sqrt{s} =$ 5.02 TeV,''
JHEP \textbf{05}, 054 (2021)
%doi:10.1007/JHEP05(2021)054
[arXiv:2005.14219 [hep-ex]].
%6 citations counted in INSPIRE as of 29 Nov 2022


%\cite{Khachatryan:2015sva}
\bibitem{Khachatryan:2015sva}
  V.~Khachatryan {\it et al.} [CMS Collaboration],
  %``Transverse momentum spectra of inclusive b jets in pPb collisions at $\sqrt{s_{NN}} = $ 5.02 TeV,''
  Phys.\ Lett.\ B {\bf 754} (2016) 59
  %doi:10.1016/j.physletb.2016.01.010
  [arXiv:1510.03373 [nucl-ex]].
  %%CITATION = doi:10.1016/j.physletb.2016.01.010;%%
  %48 citations counted in INSPIRE as of 24 Jul 2020

%\cite{ALICE:2021wct}
\bibitem{ALICE:2021wct}
S.~Acharya \textit{et al.} [ALICE],
%``Measurement of inclusive charged-particle b-jet production in pp and p-Pb collisions at $ \sqrt{s_{\mathrm{NN}}} $ = 5.02 TeV,''
JHEP \textbf{01}, 178 (2022)
%doi:10.1007/JHEP01(2022)178
[arXiv:2110.06104 [nucl-ex]].
%9 citations counted in INSPIRE as of 29 Nov 2022

\bibitem{Chatrchyan:2013exa}
  S.~Chatrchyan {\it et al.} [CMS Collaboration],
  %``Evidence of b-Jet Quenching in PbPb Collisions at $\sqrt{s_{NN}}=2.76$  TeV,''
  Phys.\ Rev.\ Lett.\  {\bf 113}, no. 13, 132301 (2014)
  Erratum: [Phys.\ Rev.\ Lett.\  {\bf 115}, no. 2, 029903 (2015)]
  %doi:10.1103/PhysRevLett.115.029903, 10.1103/PhysRevLett.113.132301
  [arXiv:1312.4198 [nucl-ex]].
  %%CITATION = doi:10.1103/PhysRevLett.115.029903, 10.1103/PhysRevLett.113.132301;%%
  %103 citations counted in INSPIRE as of 18 Apr 2019


%\cite{Sirunyan:2018jju}
\bibitem{Sirunyan:2018jju}
  A.~M.~Sirunyan {\it et al.} [CMS Collaboration],
  %``Comparing transverse momentum balance of b jet pairs in pp and PbPb collisions at $ \sqrt{s_{\mathrm{NN}}}=5.02 $ TeV,''
  JHEP {\bf 1803}, 181 (2018)
  %doi:10.1007/JHEP03(2018)181
  [arXiv:1802.00707 [hep-ex]].
  %%CITATION = doi:10.1007/JHEP03(2018)181;%%
  %10 citations counted in INSPIRE as of 18 Apr 2019

%\cite{Sirunyan:2019dow}
\bibitem{Sirunyan:2019dow}
  A.~M.~Sirunyan {\it et al.} [CMS Collaboration],
  %``Studies of charm quark diffusion inside jets using PbPb and pp collisions at $\sqrt{s_\mathrm{NN}} =$ 5.02 TeV,''
  Phys.\ Rev.\ Lett.\  {\bf 125} (2020) no.10,  102001
  %doi:10.1103/PhysRevLett.125.102001
  [arXiv:1911.01461 [hep-ex]].
  %%CITATION = doi:10.1103/PhysRevLett.125.102001;%%
  %5 citations counted in INSPIRE as of 25 Nov 2020

  %\cite{ATLAS:2022fgb}
\bibitem{ATLAS:2022fgb}
 [ATLAS],
%``Measurement of the nuclear modification factor of $b$-jets in 5.02 TeV Pb+Pb collisions with the ATLAS detector,''
[arXiv:2204.13530 [nucl-ex]].
%8 citations counted in INSPIRE as of 26 Nov 2022

%\cite{Roy:2022yrw}
\bibitem{Roy:2022yrw}
D.~Roy [STAR],
%``An Investigation of Charm Quark Jet Spectrum and Shape Modifications in Au+Au Collisions at $\sqrt{s_{\text{NN}}} = 200 \text{ GeV}$,''
[arXiv:2207.14434 [nucl-ex]].
%0 citations counted in INSPIRE as of 29 Nov 2022

%\cite{CMS:2021puf}
\bibitem{CMS:2021puf}
A.~Tumasyan \textit{et al.} [CMS],
%``Fragmentation of jets containing a prompt J$/\psi$ meson in PbPb and pp collisions at $\sqrt{s_\mathrm{NN}} =$ 5.02 TeV,''
Phys. Lett. B \textbf{825}, 136842 (2022)
%doi:10.1016/j.physletb.2021.136842
[arXiv:2106.13235 [hep-ex]].
%10 citations counted in INSPIRE as of 29 Nov 2022

%\cite{ALICE:2018lyv}
\bibitem{ALICE:2018lyv}
S.~Acharya \textit{et al.} [ALICE],
%``Measurement of D$^{0}$, D$^{+}$, D$^{*+}$ and D$_{s}^{+}$ production in Pb-Pb collisions at $ \sqrt{{\mathrm{s}}_{\mathrm{NN}}}=5.02 $ TeV,''
JHEP \textbf{10} (2018), 174
%doi:10.1007/JHEP10(2018)174
[arXiv:1804.09083 [nucl-ex]].
%162 citations counted in INSPIRE as of 26 Nov 2022

%\cite{CMS:2022btc}
\bibitem{CMS:2022btc}
 CMS Collaboration,
%``Search for medium effects using jets from bottom quarks in PbPb collisions at $\sqrt{s_\mathrm{NN}}$ = 5.02 TeV,''
[arXiv:2210.08547 [hep-ex]].
%1 citations counted in INSPIRE as of 28 Nov 2022
%%%%%%%%%%%%%%%% HQ jet in expriment%%%%%%%%%%%

%\cite{Cacciari:2005rk}
\bibitem{Cacciari:2005rk}
M.~Cacciari, P.~Nason and R.~Vogt,
%``QCD predictions for charm and bottom production at RHIC,''
Phys. Rev. Lett. \textbf{95} (2005), 122001
%doi:10.1103/PhysRevLett.95.122001
[arXiv:hep-ph/0502203 [hep-ph]].
%517 citations counted in INSPIRE as of 07 Feb 2023

%\cite{Eskola:2009uj}
\bibitem{Eskola:2009uj}
K.~J.~Eskola, H.~Paukkunen and C.~A.~Salgado,
%``EPS09: A New Generation of NLO and LO Nuclear Parton Distribution Functions,''
JHEP \textbf{04} (2009), 065
%doi:10.1088/1126-6708/2009/04/065
[arXiv:0902.4154 [hep-ph]].
%1214 citations counted in INSPIRE as of 07 Feb 2023

%\cite{Eskola:2016oht}
\bibitem{Eskola:2016oht}
K.~J.~Eskola, P.~Paakkinen, H.~Paukkunen and C.~A.~Salgado,
%``EPPS16: Nuclear parton distributions with LHC data,''
Eur. Phys. J. C \textbf{77} (2017) no.3, 163
%doi:10.1140/epjc/s10052-017-4725-9
[arXiv:1612.05741 [hep-ph]].
%530 citations counted in INSPIRE as of 07 Feb 2023

%\cite{NNPDF:2014otw}
\bibitem{NNPDF:2014otw}
R.~D.~Ball \textit{et al.} [NNPDF],
%``Parton distributions for the LHC Run II,''
JHEP \textbf{04} (2015), 040
%doi:10.1007/JHEP04(2015)040
[arXiv:1410.8849 [hep-ph]].
%3352 citations counted in INSPIRE as of 07 Feb 2023

%\cite{Peterson:1982ak}
\bibitem{Peterson:1982ak}
  C.~Peterson, D.~Schlatter, I.~Schmitt and P.~M.~Zerwas,
  %``Scaling Violations in Inclusive e+ e- Annihilation Spectra,''
  Phys.\ Rev.\ D {\bf 27}, 105 (1983).
  %doi:10.1103/PhysRevD.27.105
  %%CITATION = doi:10.1103/PhysRevD.27.105;%%
  %2215 citations counted in INSPIRE as of 20 Apr 2019

%\cite{Andersson:1983ia}
\bibitem{Andersson:1983ia}
B.~Andersson, G.~Gustafson, G.~Ingelman and T.~Sjostrand,
%``Parton Fragmentation and String Dynamics,''
Phys. Rept. \textbf{97} (1983), 31-145
%doi:10.1016/0370-1573(83)90080-7
%3908 citations counted in INSPIRE as of 07 Feb 2023

%\cite{Aversa:1988vb}
\bibitem{Aversa:1988vb}
F.~Aversa, P.~Chiappetta, M.~Greco and J.~P.~Guillet,
%``QCD Corrections to Parton-Parton Scattering Processes,''
Nucl. Phys. B \textbf{327} (1989), 105
%doi:10.1016/0550-3213(89)90288-5
%435 citations counted in INSPIRE as of 07 Feb 2023

%\cite{Kniehl:2004fy}
\bibitem{Kniehl:2004fy}
B.~A.~Kniehl, G.~Kramer, I.~Schienbein and H.~Spiesberger,
%``Inclusive D*+- production in p anti-p collisions with massive charm quarks,''
Phys. Rev. D \textbf{71} (2005), 014018
%doi:10.1103/PhysRevD.71.014018
[arXiv:hep-ph/0410289 [hep-ph]].
%172 citations counted in INSPIRE as of 07 Feb 2023

%\cite{Kniehl:2005mk}
\bibitem{Kniehl:2005mk}
B.~A.~Kniehl, G.~Kramer, I.~Schienbein and H.~Spiesberger,
%``Collinear subtractions in hadroproduction of heavy quarks,''
Eur. Phys. J. C \textbf{41} (2005), 199-212
%doi:10.1140/epjc/s2005-02200-7
[arXiv:hep-ph/0502194 [hep-ph]].
%147 citations counted in INSPIRE as of 07 Feb 2023

%\cite{Cacciari:2012ny}
\bibitem{Cacciari:2012ny}
  M.~Cacciari, S.~Frixione, N.~Houdeau, M.~L.~Mangano, P.~Nason and G.~Ridolfi,
  %``Theoretical predictions for charm and bottom production at the LHC,''
  JHEP {\bf 1210}, 137 (2012)
  %doi:10.1007/JHEP10(2012)137
  [arXiv:1205.6344 [hep-ph]].
  %%CITATION = doi:10.1007/JHEP10(2012)137;%%
  %392 citations counted in INSPIRE as of 17 Jul 2019

%\cite{Sjostrand:2006za}
\bibitem{Sjostrand:2006za}
T.~Sjostrand, S.~Mrenna and P.~Z.~Skands,
%``PYTHIA 6.4 Physics and Manual,''
JHEP \textbf{05} (2006), 026
%doi:10.1088/1126-6708/2006/05/026
[arXiv:hep-ph/0603175 [hep-ph]].
%12624 citations counted in INSPIRE as of 07 Feb 2023


%\cite{Corcella:2000bw}
\bibitem{Corcella:2000bw}
G.~Corcella, I.~G.~Knowles, G.~Marchesini, S.~Moretti, K.~Odagiri, P.~Richardson, M.~H.~Seymour and B.~R.~Webber,
%``HERWIG 6: An Event generator for hadron emission reactions with interfering gluons (including supersymmetric processes),''
JHEP \textbf{01} (2001), 010
%doi:10.1088/1126-6708/2001/01/010
[arXiv:hep-ph/0011363 [hep-ph]].
%3632 citations counted in INSPIRE as of 07 Feb 2023

%\cite{Frixione:2007vw}
\bibitem{Frixione:2007vw}
S.~Frixione, P.~Nason and C.~Oleari,
%``Matching NLO QCD computations with Parton Shower simulations: the POWHEG method,''
JHEP \textbf{11} (2007), 070
%doi:10.1088/1126-6708/2007/11/070
[arXiv:0709.2092 [hep-ph]].
%4218 citations counted in INSPIRE as of 07 Feb 2023

%\cite{Gleisberg:2008ta}
\bibitem{Gleisberg:2008ta}
T.~Gleisberg, S.~Hoeche, F.~Krauss, M.~Schonherr, S.~Schumann, F.~Siegert and J.~Winter,
%``Event generation with SHERPA 1.1,''
JHEP \textbf{02} (2009), 007
%doi:10.1088/1126-6708/2009/02/007
[arXiv:0811.4622 [hep-ph]].
%3578 citations counted in INSPIRE as of 02 Feb 2023

%\cite{Ferini:2008he}
\bibitem{Ferini:2008he}
G.~Ferini, M.~Colonna, M.~Di Toro and V.~Greco,
%``Scalings of Elliptic Flow for a Fluid at Finite Shear Viscosity,''
Phys. Lett. B \textbf{670} (2009), 325-329
%doi:10.1016/j.physletb.2008.10.062
[arXiv:0805.4814 [nucl-th]].
%101 citations counted in INSPIRE as of 08 Feb 2023

%\cite{Ruggieri:2013ova}
\bibitem{Ruggieri:2013ova}
M.~Ruggieri, F.~Scardina, S.~Plumari and V.~Greco,
%``Thermalization, Isotropization and Elliptic Flow from Nonequilibrium Initial Conditions with a Saturation Scale,''
Phys. Rev. C \textbf{89} (2014) no.5, 054914
%doi:10.1103/PhysRevC.89.054914
[arXiv:1312.6060 [nucl-th]].
%69 citations counted in INSPIRE as of 08 Feb 2023

%\cite{Beraudo:2014iva}
\bibitem{Beraudo:2014iva}
A.~Beraudo,
%``Dynamics of heavy flavor quarks in high energy nuclear collisions,''
Nucl. Phys. A \textbf{931} (2014), 145-154
%doi:10.1016/j.nuclphysa.2014.07.038
[arXiv:1407.5918 [hep-ph]].
%15 citations counted in INSPIRE as of 08 Feb 2023

%\cite{Arnold:2002ja}
\bibitem{Arnold:2002ja}
P.~B.~Arnold, G.~D.~Moore and L.~G.~Yaffe,
%``Photon and gluon emission in relativistic plasmas,''
JHEP \textbf{06} (2002), 030
%doi:10.1088/1126-6708/2002/06/030
[arXiv:hep-ph/0204343 [hep-ph]].
%527 citations counted in INSPIRE as of 08 Feb 2023

%\cite{Abir:2011jb}
\bibitem{Abir:2011jb}
R.~Abir, C.~Greiner, M.~Martinez, M.~G.~Mustafa and J.~Uphoff,
%``Soft gluon emission off a heavy quark revisited,''
Phys. Rev. D \textbf{85} (2012), 054012
%doi:10.1103/PhysRevD.85.054012
[arXiv:1109.5539 [hep-ph]].
%54 citations counted in INSPIRE as of 08 Feb 2023

%\cite{Abir:2012pu}
\bibitem{Abir:2012pu}
R.~Abir, U.~Jamil, M.~G.~Mustafa and D.~K.~Srivastava,
%``Heavy quark energy loss and D-mesons in RHIC and LHC energies,''
Phys. Lett. B \textbf{715} (2012), 183-189
%doi:10.1016/j.physletb.2012.07.044
[arXiv:1203.5221 [hep-ph]].
%70 citations counted in INSPIRE as of 08 Feb 2023

%\cite{Zapp:2008gi}
\bibitem{Zapp:2008gi}
K.~Zapp, G.~Ingelman, J.~Rathsman, J.~Stachel and U.~A.~Wiedemann,
%``A Monte Carlo Model for 'Jet Quenching',''
Eur. Phys. J. C \textbf{60} (2009), 617-632
%doi:10.1140/epjc/s10052-009-0941-2
[arXiv:0804.3568 [hep-ph]].
%184 citations counted in INSPIRE as of 08 Feb 2023


%\cite{Armesto:2003jh}
\bibitem{Armesto:2003jh}
N.~Armesto, C.~A.~Salgado and U.~A.~Wiedemann,
%``Medium induced gluon radiation off massive quarks fills the dead cone,''
Phys. Rev. D \textbf{69} (2004), 114003
%doi:10.1103/PhysRevD.69.114003
[arXiv:hep-ph/0312106 [hep-ph]].
%359 citations counted in INSPIRE as of 08 Feb 2023

%\cite{Das:2013kea}
\bibitem{Das:2013kea}
S.~K.~Das, F.~Scardina, S.~Plumari and V.~Greco,
%``Heavy-flavor in-medium momentum evolution: Langevin versus Boltzmann approach,''
Phys. Rev. C \textbf{90} (2014), 044901
%doi:10.1103/PhysRevC.90.044901
[arXiv:1312.6857 [nucl-th]].
%98 citations counted in INSPIRE as of 08 Feb 2023

%\cite{Li:2019wri}
\bibitem{Li:2019wri}
S.~Li, C.~Wang, R.~Wan and J.~Liao,
%``Probing the transport properties of Quark-Gluon Plasma via heavy-flavor Boltzmann and Langevin dynamics,''
Phys. Rev. C \textbf{99} (2019) no.5, 054909
%doi:10.1103/PhysRevC.99.054909
[arXiv:1901.04600 [hep-ph]].
%16 citations counted in INSPIRE as of 08 Feb 2023

%\cite{Das:2017dsh}
\bibitem{Das:2017dsh}
S.~K.~Das, M.~Ruggieri, F.~Scardina, S.~Plumari and V.~Greco,
%``Effect of pre-equilibrium phase on $R_{AA}$ and $v_2$ of heavy quarks in heavy ion collisions,''
J. Phys. G \textbf{44} (2017) no.9, 095102
%doi:10.1088/1361-6471/aa815a
[arXiv:1701.05123 [nucl-th]].
%32 citations counted in INSPIRE as of 08 Feb 2023

%\cite{Lin:2000jp}
\bibitem{Lin:2000jp}
Z.~w.~Lin, T.~G.~Di and C.~M.~Ko,
%``Charm meson scattering cross-sections by pion and rho meson,''
Nucl. Phys. A \textbf{689} (2001), 965-979
%doi:10.1016/S0375-9474(00)00611-4
[arXiv:nucl-th/0006086 [nucl-th]].
%42 citations counted in INSPIRE as of 10 Feb 2023

%\cite{Cao:2015hia}
\bibitem{Cao:2015hia}
S.~Cao, G.~Y.~Qin and S.~A.~Bass,
%``Energy loss, hadronization and hadronic interactions of heavy flavors in relativistic heavy-ion collisions,''
Phys. Rev. C \textbf{92} (2015) no.2, 024907
%doi:10.1103/PhysRevC.92.024907
[arXiv:1505.01413 [nucl-th]].
%133 citations counted in INSPIRE as of 10 Feb 2023

%\cite{Li:2019lex}
\bibitem{Li:2019lex}
S.~Li and J.~Liao,
%``Data-driven extraction of heavy quark diffusion in quark-gluon plasma,''
Eur. Phys. J. C \textbf{80} (2020) no.7, 671
%doi:10.1140/epjc/s10052-020-8243-9
[arXiv:1912.08965 [hep-ph]].
%21 citations counted in INSPIRE as of 11 Feb 2023

%\cite{Ding:2015ona}
\bibitem{Ding:2015ona}
H.~T.~Ding, F.~Karsch and S.~Mukherjee,
%``Thermodynamics of strong-interaction matter from Lattice QCD,''
Int. J. Mod. Phys. E \textbf{24} (2015) no.10, 1530007
%doi:10.1142/S0218301315300076
[arXiv:1504.05274 [hep-lat]].
%310 citations counted in INSPIRE as of 08 Feb 2023

%\cite{Banerjee:2011ra}
\bibitem{Banerjee:2011ra}
D.~Banerjee, S.~Datta, R.~Gavai and P.~Majumdar,
%``Heavy Quark Momentum Diffusion Coefficient from Lattice QCD,''
Phys. Rev. D \textbf{85} (2012), 014510
%doi:10.1103/PhysRevD.85.014510
[arXiv:1109.5738 [hep-lat]].
%175 citations counted in INSPIRE as of 08 Feb 2023

%\cite{Kaczmarek:2014jga}
\bibitem{Kaczmarek:2014jga}
O.~Kaczmarek,
%``Continuum estimate of the heavy quark momentum diffusion coefficient $\kappa$,''
Nucl. Phys. A \textbf{931} (2014), 633-637
%doi:10.1016/j.nuclphysa.2014.09.031
[arXiv:1409.3724 [hep-lat]].
%42 citations counted in INSPIRE as of 08 Feb 2023

%\cite{vanHees:2004gq}
\bibitem{vanHees:2004gq}
H.~van Hees and R.~Rapp,
%``Thermalization of heavy quarks in the quark-gluon plasma,''
Phys. Rev. C \textbf{71} (2005), 034907
%doi:10.1103/PhysRevC.71.034907
[arXiv:nucl-th/0412015 [nucl-th]].
%264 citations counted in INSPIRE as of 08 Feb 2023

%\cite{Song:2015sfa}
\bibitem{Song:2015sfa}
T.~Song, H.~Berrehrah, D.~Cabrera, J.~M.~Torres-Rincon, L.~Tolos, W.~Cassing and E.~Bratkovskaya,
%``Tomography of the Quark-Gluon-Plasma by Charm Quarks,''
Phys. Rev. C \textbf{92} (2015) no.1, 014910
%doi:10.1103/PhysRevC.92.014910
[arXiv:1503.03039 [nucl-th]].
%153 citations counted in INSPIRE as of 08 Feb 2023

%\cite{Horowitz:2015dta}
\bibitem{Horowitz:2015dta}
W.~A.~Horowitz,
%``Fluctuating heavy quark energy loss in a strongly coupled quark-gluon plasma,''
Phys. Rev. D \textbf{91} (2015) no.8, 085019
%doi:10.1103/PhysRevD.91.085019
[arXiv:1501.04693 [hep-ph]].
%49 citations counted in INSPIRE as of 08 Feb 2023

%\cite{Xu:2017obm}
\bibitem{Xu:2017obm}
Y.~Xu, J.~E.~Bernhard, S.~A.~Bass, M.~Nahrgang and S.~Cao,
%``Data-driven analysis for the temperature and momentum dependence of the heavy-quark diffusion coefficient in relativistic heavy-ion collisions,''
Phys. Rev. C \textbf{97} (2018) no.1, 014907
%doi:10.1103/PhysRevC.97.014907
[arXiv:1710.00807 [nucl-th]].
%89 citations counted in INSPIRE as of 08 Feb 2023

%\cite{Tolos:2013kva}
\bibitem{Tolos:2013kva}
L.~Tolos and J.~M.~Torres-Rincon,
%``D-meson propagation in hot dense matter,''
Phys. Rev. D \textbf{88} (2013), 074019
%doi:10.1103/PhysRevD.88.074019
[arXiv:1306.5426 [hep-ph]].
%62 citations counted in INSPIRE as of 08 Feb 2023

%\cite{Bialek:1996kd}
\bibitem{Bialek:1996kd}
W.~Bialek, C.~G.~Callan, Jr. and S.~P.~Strong,
%``Field theories for learning probability distributions,''
Phys. Rev. Lett. \textbf{77} (1996), 4693-4697
%doi:10.1103/PhysRevLett.77.4693
[arXiv:cond-mat/9607180 [cond-mat]].
%22 citations counted in INSPIRE as of 11 Feb 2023

%\cite{Bialek:1996kd}
\bibitem{Lemm:1999kd}
J. C. Lemm,
%``Field theories for learning probability distributions,''
%Bayesian field theory: Nonparametric approaches to density estimation, regression, classification, and inverse quantum problems (1999).
[arXiv:physics/9912005 [cond-mat]].

%\cite{Xie:2022fak}
\bibitem{Xie:2022fak}
M.~Xie, W.~Ke, H.~Zhang and X.~N.~Wang,
%``Global constraint on the jet transport coefficient from single hadron, dihadron and $\gamma$-hadron spectra in high-energy heavy-ion collisions,''
[arXiv:2208.14419 [hep-ph]].
%0 citations counted in INSPIRE as of 11 Feb 2023

%%%%%%%%%%%%%%%%%%%%%%%%%%%%%%%%%%%%%% HQ jet in experiment %%%%%%%%%%%%%%%%%%%%%%%%%%%%%%%%%

%\cite{STAR:2003pjh}
\bibitem{STAR:2003pjh}
J.~Adams \textit{et al.} [STAR],
%``Evidence from d + Au measurements for final state suppression of high p(T) hadrons in Au+Au collisions at RHIC,''
Phys. Rev. Lett. \textbf{91} (2003), 072304
%doi:10.1103/PhysRevLett.91.072304
[arXiv:nucl-ex/0306024 [nucl-ex]].
%926 citations counted in INSPIRE as of 26 Nov 2022

%\cite{Miller:2007ri}
\bibitem{Miller:2007ri}
M.~L.~Miller, K.~Reygers, S.~J.~Sanders and P.~Steinberg,
%``Glauber modeling in high energy nuclear collisions,''
Ann. Rev. Nucl. Part. Sci. \textbf{57} (2007), 205-243
%doi:10.1146/annurev.nucl.57.090506.123020
[arXiv:nucl-ex/0701025 [nucl-ex]].
%1684 citations counted in INSPIRE as of 26 Nov 2022

%\cite{STAR:2003wqp}
\bibitem{STAR:2003wqp}
J.~Adams \textit{et al.} [STAR],
%``Particle type dependence of azimuthal anisotropy and nuclear modification of particle production in Au + Au collisions at s(NN)**(1/2) = 200-GeV,''
Phys. Rev. Lett. \textbf{92} (2004), 052302
%doi:10.1103/PhysRevLett.92.052302
[arXiv:nucl-ex/0306007 [nucl-ex]].
%696 citations counted in INSPIRE as of 26 Nov 2022

%\cite{PHENIX:2010bqp}
\bibitem{PHENIX:2010bqp}
A.~Adare \textit{et al.} [PHENIX],
%``Nuclear modification factors of $\phi$ mesons in $d+$Au, Cu+Cu and Au+Au collisions at sqrt(S\_NN) =200 GeV,''
Phys. Rev. C \textbf{83} (2011), 024909
%doi:10.1103/PhysRevC.83.024909
[arXiv:1004.3532 [nucl-ex]].
%115 citations counted in INSPIRE as of 26 Nov 2022

%\cite{PHENIX:2005nhb}
\bibitem{PHENIX:2005nhb}
S.~S.~Adler \textit{et al.} [PHENIX],
%``Nuclear modification of electron spectra and implications for heavy quark energy loss in Au+Au collisions at s(NN)**(1/2) - 200-GeV,''
Phys. Rev. Lett. \textbf{96} (2006), 032301
%doi:10.1103/PhysRevLett.96.032301
[arXiv:nucl-ex/0510047 [nucl-ex]].
%371 citations counted in INSPIRE as of 26 Nov 2022

%\cite{ATLAS:2014ipv}
\bibitem{ATLAS:2014ipv}
G.~Aad \textit{et al.} [ATLAS],
%``Measurements of the Nuclear Modification Factor for Jets in Pb+Pb Collisions at $\sqrt{s_{\mathrm{NN}}}=2.76$ TeV with the ATLAS Detector,''
Phys. Rev. Lett. \textbf{114} (2015) no.7, 072302
%doi:10.1103/PhysRevLett.114.072302
[arXiv:1411.2357 [hep-ex]].
%244 citations counted in INSPIRE as of 26 Nov 2022

%\cite{ATLAS:2018gwx}
\bibitem{ATLAS:2018gwx}
M.~Aaboud \textit{et al.} [ATLAS],
%``Measurement of the nuclear modification factor for inclusive jets in Pb+Pb collisions at $\sqrt{s_\mathrm{NN}}=5.02$ TeV with the ATLAS detector,''
Phys. Lett. B \textbf{790} (2019), 108-128
%doi:10.1016/j.physletb.2018.10.076
[arXiv:1805.05635 [nucl-ex]].
%155 citations counted in INSPIRE as of 26 Nov 2022

%\cite{Baier:2002tc}
\bibitem{Baier:2002tc}
R.~Baier,
%``Jet quenching,''
Nucl. Phys. A \textbf{715} (2003), 209-218
%doi:10.1016/S0375-9474(02)01429-X
[arXiv:hep-ph/0209038 [hep-ph]].
%200 citations counted in INSPIRE as of 26 Nov 2022

%\cite{JET:2013cls}
\bibitem{JET:2013cls}
K.~M.~Burke \textit{et al.} [JET],
%``Extracting the jet transport coefficient from jet quenching in high-energy heavy-ion collisions,''
Phys. Rev. C \textbf{90} (2014) no.1, 014909
%doi:10.1103/PhysRevC.90.014909
[arXiv:1312.5003 [nucl-th]].
%406 citations counted in INSPIRE as of 26 Nov 2022

%\cite{Xie:2019oxg}
\bibitem{Xie:2019oxg}
M.~Xie, S.~Y.~Wei, G.~Y.~Qin and H.~Z.~Zhang,
%``Extracting jet transport coefficient via single hadron and dihadron productions in high-energy heavy-ion collisions,''
Eur. Phys. J. C \textbf{79} (2019) no.7, 589
%doi:10.1140/epjc/s10052-019-7100-1
[arXiv:1901.04155 [hep-ph]].
%26 citations counted in INSPIRE as of 26 Nov 2022

%\cite{Ru:2019qvz}
\bibitem{Ru:2019qvz}
P.~Ru, Z.~B.~Kang, E.~Wang, H.~Xing and B.~W.~Zhang,
%``Global extraction of the jet transport coefficient in cold nuclear matter,''
Phys. Rev. D \textbf{103} (2021) no.3, L031901
%doi:10.1103/PhysRevD.103.L031901
[arXiv:1907.11808 [hep-ph]].
%25 citations counted in INSPIRE as of 26 Nov 2022

%\cite{JETSCAPE:2021ehl}
\bibitem{JETSCAPE:2021ehl}
S.~Cao \textit{et al.} [JETSCAPE],
%``Determining the jet transport coefficient q? from inclusive hadron suppression measurements using Bayesian parameter estimation,''
Phys. Rev. C \textbf{104} (2021) no.2, 024905
%doi:10.1103/PhysRevC.104.024905
[arXiv:2102.11337 [nucl-th]].
%58 citations counted in INSPIRE as of 26 Nov 2022

%\cite{Huang:2013vaa}
\bibitem{Huang:2013vaa}
  J.~Huang, Z.~B.~Kang and I.~Vitev,
  %``Inclusive b-jet production in heavy ion collisions at the LHC,''
  Phys.\ Lett.\ B {\bf 726} (2013) 251
  %doi:10.1016/j.physletb.2013.08.009
  [arXiv:1306.0909 [hep-ph]].
  %%CITATION = doi:10.1016/j.physletb.2013.08.009;%%
  %60 citations counted in INSPIRE as of 06 Jul 2020

%\cite{Frixione:2002ik}
\bibitem{Frixione:2002ik}
  S.~Frixione and B.~R.~Webber,
  %``Matching NLO QCD computations and parton shower simulations,''
  JHEP {\bf 0206}, 029 (2002)
  %doi:10.1088/1126-6708/2002/06/029
  [hep-ph/0204244].
  %%CITATION = doi:10.1088/1126-6708/2002/06/029;%%
  %2498 citations counted in INSPIRE as of 04 Jun 2018

%\cite{Ma:2018swx}
\bibitem{Ma:2018swx}
  G.~Y.~Ma, W.~Dai, B.~W.~Zhang and E.~K.~Wang,
  %``NLO Productions of $\omega $ and $K^0_{\mathrm{S}}$ with a global extraction of the jet transport parameter in heavy-ion collisions,''
  Eur.\ Phys.\ J.\ C {\bf 79} (2019) no.6,  518
  %doi:10.1140/epjc/s10052-019-7005-z
  [arXiv:1812.02033 [nucl-th]].
  %%CITATION = doi:10.1140/epjc/s10052-019-7005-z;%%
  %1 citations counted in INSPIRE as of 29 Jul 2019

%\cite{CMS:2016uxf}
\bibitem{CMS:2016uxf}
V.~Khachatryan \textit{et al.} [CMS],
%``Measurement of inclusive jet cross-sections in $pp$ and PbPb collisions at $\sqrt{s_{NN}}=$ 2.76 TeV,''
Phys. Rev. C \textbf{96}, no.1, 015202 (2017)
%doi:10.1103/PhysRevC.96.015202
[arXiv:1609.05383 [nucl-ex]].
%143 citations counted in INSPIRE as of 29 Nov 2022

%\cite{Dokshitzer:2001zm}
\bibitem{Dokshitzer:2001zm}
Y.~L.~Dokshitzer and D.~E.~Kharzeev,
%``Heavy quark colorimetry of QCD matter,''
Phys. Lett. B \textbf{519}, 199-206 (2001)
%doi:10.1016/S0370-2693(01)01130-3
[arXiv:hep-ph/0106202 [hep-ph]].
%966 citations counted in INSPIRE as of 29 Nov 2022

%\cite{PHENIX:2022wim}
\bibitem{PHENIX:2022wim}
U.~A.~Acharya \textit{et al.} [PHENIX],
%``Charm- and Bottom-Quark Production in Au$+$Au Collisions at $\sqrt{s_{_{NN}}}$ = 200 GeV,''
[arXiv:2203.17058 [nucl-ex]].
%3 citations counted in INSPIRE as of 28 Oct 2022

%\cite{STAR:2021uzu}
\bibitem{STAR:2021uzu}
S.~Collaboration \textit{et al.} [STAR],
%``Evidence of Mass Ordering of Charm and Bottom Quark Energy Loss in Au+Au Collisions at RHIC,''
[arXiv:2111.14615 [nucl-ex]].
%7 citations counted in INSPIRE as of 28 Oct 2022


%\cite{Ovanesyan:2011kn}
\bibitem{Ovanesyan:2011kn}
  G.~Ovanesyan and I.~Vitev,
  %``Medium-induced parton splitting kernels from Soft Collinear Effective Theory with Glauber gluons,''
  Phys.\ Lett.\ B {\bf 706} (2012) 371
  %doi:10.1016/j.physletb.2011.11.040
  [arXiv:1109.5619 [hep-ph]].
  %%CITATION = doi:10.1016/j.physletb.2011.11.040;%%
  %74 citations counted in INSPIRE as of 24 Jul 2020

  %\cite{Sievert:2018imd}
\bibitem{Sievert:2018imd}
  M.~D.~Sievert and I.~Vitev,
  %``Quark branching in QCD matter to any order in opacity beyond the soft gluon emission limit,''
  Phys.\ Rev.\ D {\bf 98} (2018) no.9,  094010
  %doi:10.1103/PhysRevD.98.094010
  [arXiv:1807.03799 [hep-ph]].
  %%CITATION = doi:10.1103/PhysRevD.98.094010;%%
  %21 citations counted in INSPIRE as of 24 Jul 2020


%\cite{Ke:2020nsm}
\bibitem{Ke:2020nsm}
W.~Ke, X.~N.~Wang, W.~Fan and S.~A.~Bass,
%``Study of heavy-flavor jets in a transport approach,''
PoS \textbf{HardProbes2020}, 060 (2021)
%doi:10.22323/1.387.0060
[arXiv:2008.07622 [nucl-th]].
%3 citations counted in INSPIRE as of 28 Nov 2022

\bibitem{wang:2022aaa}
S.~Wang, B.~W.~Zhang and E.~Wang,
``Probing the mass effect of jet quenching by heavy-flavor jets in heavy-ion collisions,''
in preparation.

%\cite{Wang:2020qwe}
\bibitem{Wang:2020qwe}
  S.~Wang, W.~Dai, B.~W.~Zhang and E.~Wang,
  %``$Z^0$ Boson Associated b-jet Production in High-Energy Nuclear Collisions,''
  arXiv:2005.07018 [hep-ph].
  %%CITATION = ARXIV:2005.07018;%%

%\cite{Kartvelishvili:1995fr}
\bibitem{Kartvelishvili:1995fr}
  V.~Kartvelishvili, R.~Kvatadze and R.~Shanidze,
  %``On Z and Z + jet production in heavy ion collisions,''
  Phys.\ Lett.\ B {\bf 356}, 589 (1995)
  %doi:10.1016/0370-2693(95)00865-I
  [hep-ph/9505418].
  %%CITATION = doi:10.1016/0370-2693(95)00865-I;%%
  %50 citations counted in INSPIRE as of 22 May 2019

  %\cite{CMS:2017ehl}
\bibitem{CMS:2017ehl}
A.~M.~Sirunyan \textit{et al.} [CMS],
%``Study of jet quenching with isolated-photon+jet correlations in PbPb and pp collisions at $\sqrt{s_{_{\mathrm{NN}}}} =$ 5.02 TeV,''
Phys. Lett. B \textbf{785} (2018), 14-39
%doi:10.1016/j.physletb.2018.07.061
[arXiv:1711.09738 [nucl-ex]].
%67 citations counted in INSPIRE as of 26 Nov 2022


%\cite{CMS:2017eqd}
\bibitem{CMS:2017eqd}
A.~M.~Sirunyan \textit{et al.} [CMS],
%``Study of Jet Quenching with $Z+\text{jet}$ Correlations in Pb-Pb and $pp$ Collisions at ${\sqrt{s}}_{NN}=5.02\text{ }\text{ }\mathrm{TeV}$,''
Phys. Rev. Lett. \textbf{119} (2017) no.8, 082301
%doi:10.1103/PhysRevLett.119.082301
[arXiv:1702.01060 [nucl-ex]].
%90 citations counted in INSPIRE as of 26 Nov 2022

%\cite{Nason:1989zy}
\bibitem{Nason:1989zy}
P.~Nason, S.~Dawson and R.~K.~Ellis,
%``The One Particle Inclusive Differential Cross-Section for Heavy Quark Production in Hadronic Collisions,''
Nucl. Phys. B \textbf{327}, 49-92 (1989)
[erratum: Nucl. Phys. B \textbf{335}, 260-260 (1990)]
%doi:10.1016/0550-3213(89)90286-1
%1207 citations counted in INSPIRE as of 30 Nov 2022

%\cite{Beenakker:1990maa}
\bibitem{Beenakker:1990maa}
W.~Beenakker, W.~L.~van Neerven, R.~Meng, G.~A.~Schuler and J.~Smith,
%``QCD corrections to heavy quark production in hadron hadron collisions,''
Nucl. Phys. B \textbf{351}, 507-560 (1991)
%doi:10.1016/S0550-3213(05)80032-X
%661 citations counted in INSPIRE as of 30 Nov 2022

%\cite{Mangano:1991jk}
\bibitem{Mangano:1991jk}
M.~L.~Mangano, P.~Nason and G.~Ridolfi,
%``Heavy quark correlations in hadron collisions at next-to-leading order,''
Nucl. Phys. B \textbf{373}, 295-345 (1992)
%doi:10.1016/0550-3213(92)90435-E
%990 citations counted in INSPIRE as of 30 Nov 2022

%\cite{Norrbin:2000zc}
\bibitem{Norrbin:2000zc}
E.~Norrbin and T.~Sjostrand,
%``Production and hadronization of heavy quarks,''
Eur. Phys. J. C \textbf{17} (2000), 137-161
%doi:10.1007/s100520000460
[arXiv:hep-ph/0005110 [hep-ph]].
%253 citations counted in INSPIRE as of 26 Nov 2022

 %\cite{Kang:2018wrs}
\bibitem{Kang:2018wrs}
  Z.~B.~Kang, J.~Reiten, I.~Vitev and B.~Yoon,
  %``Light and heavy flavor dijets production and dijets mass modification in heavy ion collisions,''
  Phys.\ Rev.\ D {\bf 99}, no. 3, 034006 (2019)
  %doi:10.1103/PhysRevD.99.034006
  [arXiv:1810.10007 [hep-ph]].
  %%CITATION = doi:10.1103/PhysRevD.99.034006;%%
  %3 citations counted in INSPIRE as of 04 Jun 2019

  %\cite{Huang:2015mva}
\bibitem{Huang:2015mva}
  J.~Huang, Z.~B.~Kang, I.~Vitev and H.~Xing,
  %``Photon-tagged and B-meson-tagged b-jet production at the LHC,''
  Phys.\ Lett.\ B {\bf 750} (2015) 287
  %doi:10.1016/j.physletb.2015.09.029
  [arXiv:1505.03517 [hep-ph]].
  %%CITATION = doi:10.1016/j.physletb.2015.09.029;%%
  %24 citations counted in INSPIRE as of 06 Jul 2020

%\cite{Gleisberg:2008fv}
\bibitem{Gleisberg:2008fv}
T.~Gleisberg and S.~Hoeche,
%``Comix, a new matrix element generator,''
JHEP \textbf{12} (2008), 039
%doi:10.1088/1126-6708/2008/12/039
[arXiv:0808.3674 [hep-ph]].
%915 citations counted in INSPIRE as of 08 Mar 2022

%\cite{Schumann:2007mg}
\bibitem{Schumann:2007mg}
S.~Schumann and F.~Krauss,
%``A Parton shower algorithm based on Catani-Seymour dipole factorisation,''
JHEP \textbf{03} (2008), 038
%doi:10.1088/1126-6708/2008/03/038
[arXiv:0709.1027 [hep-ph]].
%932 citations counted in INSPIRE as of 08 Mar 2022

%\cite{Mueller:2016gko}
\bibitem{Mueller:2016gko}
A.~H.~Mueller, B.~Wu, B.~W.~Xiao and F.~Yuan,
%``Probing Transverse Momentum Broadening in Heavy Ion Collisions,''
Phys. Lett. B \textbf{763}, 208-212 (2016)
%doi:10.1016/j.physletb.2016.10.037
[arXiv:1604.04250 [hep-ph]].
%57 citations counted in INSPIRE as of 30 Nov 2022

%\cite{Jia:2019qbl}
\bibitem{Jia:2019qbl}
J.~Jia, S.~Y.~Wei, B.~W.~Xiao and F.~Yuan,
%``Medium-Induced Transverse Momentum Broadening via Forward Dijet Correlations,''
Phys. Rev. D \textbf{101}, no.9, 094008 (2020)
%doi:10.1103/PhysRevD.101.094008
[arXiv:1910.05290 [hep-ph]].
%5 citations counted in INSPIRE as of 30 Nov 2022

%\cite{Luo:2018pto}
\bibitem{Luo:2018pto}
T.~Luo, S.~Cao, Y.~He and X.~N.~Wang,
%``Multiple jets and $\gamma$-jet correlation in high-energy heavy-ion collisions,''
Phys. Lett. B \textbf{782}, 707-716 (2018)
%doi:10.1016/j.physletb.2018.06.025
[arXiv:1803.06785 [hep-ph]].
%67 citations counted in INSPIRE as of 29 Nov 2022

%\cite{Zhang:2018urd}
\bibitem{Zhang:2018urd}
S.~L.~Zhang, T.~Luo, X.~N.~Wang and B.~W.~Zhang,
%``Z+jet correlation with NLO-matched parton-shower and jet-medium interaction in high-energy nuclear collisions,''
Phys. Rev. C \textbf{98}, 021901 (2018)
%doi:10.1103/PhysRevC.98.021901
[arXiv:1804.11041 [nucl-th]].
%41 citations counted in INSPIRE as of 30 Nov 2022

%\cite{Wang:2018gxz}
\bibitem{Wang:2018gxz}
  S.~Wang, W.~Dai, S.~L.~Zhang, B.~W.~Zhang and E.~Wang,
  %``The production of $b\bar{b}$ dijets in heavy-ion collisions at the LHC,''
  PoS HardProbes {\bf 2018} (2018) 091
  arXiv:1812.00391 [nucl-th].
  %%CITATION = ARXIV:1812.00391;%%

  %\cite{STAR:2019qbf}
\bibitem{STAR:2019qbf}
J.~Adam \textit{et al.} [STAR],
%``Measurement of D$^0$-meson + hadron two-dimensional angular correlations in Au+Au collisions at $\sqrt{s_{\rm NN}}$ = 200 GeV,''
Phys. Rev. C \textbf{102} (2020) no.1, 014905
%doi:10.1103/PhysRevC.102.014905
[arXiv:1911.12168 [nucl-ex]].

  %\cite{Wang:2020bqz}
\bibitem{Wang:2020bqz}
S.~Wang, W.~Dai, J.~Yan, B.~W.~Zhang and E.~Wang,
%``Radial distribution of charm quarks in jets in high-energy heavy-ion collisions,''
Nucl. Phys. A \textbf{1005} (2021), 121787
%doi:10.1016/j.nuclphysa.2020.121787
[arXiv:2001.11660 [nucl-th]].
%3 citations counted in INSPIRE as of 14 Jul 2021


%\cite{Wang:2020ffj}
\bibitem{Wang:2020ffj}
  S.~Wang, W.~Dai, B.~W.~Zhang and E.~Wang,
  %``Radial profile of heavy quarks in jets in high-energy nuclear collisions,''
  PoS HardProbes {\bf 2020} (2020) 097
  arXiv:2009.13959 [nucl-th].
  %%CITATION = ARXIV:2009.13959;%%

%\cite{Wang:2020ukj}
\bibitem{Wang:2020ukj}
S.~Wang, W.~Dai, B.~W.~Zhang and E.~Wang,
%``Radial profile of bottom quarks in jets in high-energy nuclear collisions,''
Chin. Phys. C \textbf{45} (2021) no.6, 064105
%doi:10.1088/1674-1137/abf4f5
[arXiv:2012.13935 [nucl-th]].

%\cite{Renk:2012ve}
\bibitem{Renk:2012ve}
T.~Renk,
%``Biased showers: A common conceptual framework for the interpretation of high-$P_T$ observables in heavy-ion collisions,''
Phys. Rev. C \textbf{88}, no.5, 054902 (2013)
%doi:10.1103/PhysRevC.88.054902
[arXiv:1212.0646 [hep-ph]].
%28 citations counted in INSPIRE as of 29 Nov 2022

%\cite{Sjostrand:1986hx}
\bibitem{Sjostrand:1986hx}
T.~Sjostrand and M.~Bengtsson,
%``The Lund Monte Carlo for Jet Fragmentation and e+ e- Physics. Jetset Version 6.3: An Update,''
Comput. Phys. Commun. \textbf{43} (1987), 367
%doi:10.1016/0010-4655(87)90054-3
%2229 citations counted in INSPIRE as of 25 Nov 2022

%\cite{Webber:1983if}
\bibitem{Webber:1983if}
B.~R.~Webber,
%``A QCD Model for Jet Fragmentation Including Soft Gluon Interference,''
Nucl. Phys. B \textbf{238} (1984), 492-528
%doi:10.1016/0550-3213(84)90333-X
%1410 citations counted in INSPIRE as of 25 Nov 2022

%\cite{Procura:2009vm}
\bibitem{Procura:2009vm}
M.~Procura and I.~W.~Stewart,
%``Quark Fragmentation within an Identified Jet,''
Phys. Rev. D \textbf{81} (2010), 074009

%\cite{CMS:2014jjt}
\bibitem{CMS:2014jjt}
S.~Chatrchyan \textit{et al.} [CMS],
%``Measurement of Jet Fragmentation in PbPb and pp Collisions at $\sqrt{s_{NN}}= 2.76$ TeV,''
Phys. Rev. C \textbf{90} (2014) no.2, 024908
%doi:10.1103/PhysRevC.90.024908
[arXiv:1406.0932 [nucl-ex]].
%239 citations counted in INSPIRE as of 25 Nov 2022

%\cite{ATLAS:2014dtd}
\bibitem{ATLAS:2014dtd}
G.~Aad \textit{et al.} [ATLAS],
%``Measurement of inclusive jet charged-particle fragmentation functions in Pb+Pb collisions at $\sqrt{s_{NN}}=2.76$ TeV with the ATLAS detector,''
Phys. Lett. B \textbf{739} (2014), 320-342
%doi:10.1016/j.physletb.2014.10.065
[arXiv:1406.2979 [hep-ex]].

%\cite{ATLAS:2017nre}
\bibitem{ATLAS:2017nre}
M.~Aaboud \textit{et al.} [ATLAS],
%``Measurement of jet fragmentation in Pb+Pb and $pp$ collisions at $\sqrt{{s_\mathrm{NN}}} = 2.76$ TeV with the ATLAS detector at the LHC,''
Eur. Phys. J. C \textbf{77} (2017) no.6, 379
%doi:10.1140/epjc/s10052-017-4915-5
[arXiv:1702.00674 [hep-ex]].
%88 citations counted in INSPIRE as of 25 Nov 2022

%\cite{ATLAS:2019dsv}
\bibitem{ATLAS:2019dsv}
M.~Aaboud \textit{et al.} [ATLAS],
%``Comparison of Fragmentation Functions for Jets Dominated by Light Quarks and Gluons from $pp$ and Pb+Pb Collisions in ATLAS,''
Phys. Rev. Lett. \textbf{123} (2019) no.4, 042001
%doi:10.1103/PhysRevLett.123.042001
[arXiv:1902.10007 [nucl-ex]].
%50 citations counted in INSPIRE as of 25 Nov 2022

%\cite{ALICE:2022mur}
\bibitem{ALICE:2022mur}
 [ALICE],
%``Measurement of the production of charm jets tagged with ${\rm D^0}$ mesons in pp collisions at $\sqrt{s}$ = 5.02 and 13 TeV,''
[arXiv:2204.10167 [nucl-ex]].
%3 citations counted in INSPIRE as of 25 Nov 2022

%\cite{Li:2022tcr}
\bibitem{Li:2022tcr}
Y.~Li, S.~Wang and B.~W.~Zhang,
%``Longitudinal momentum fraction of $\rm{D^0}$ in jets in high-energy nuclear collisions,''
[arXiv:2209.00548 [hep-ph]].
%0 citations counted in INSPIRE as of 29 Nov 2022

%\cite{Nason:2004rx}
\bibitem{Nason:2004rx}
  P.~Nason,
  %``A New method for combining NLO QCD with shower Monte Carlo algorithms,''
  JHEP \textbf{11}, 040 (2004)
  %doi:10.1088/1126-6708/2004/11/040
  [arXiv:hep-ph/0409146 [hep-ph]].
  %3435 citations counted in INSPIRE as of 25 Aug 2022

%\cite{Alioli:2010xd}
\bibitem{Alioli:2010xd}
  S.~Alioli, P.~Nason, C.~Oleari and E.~Re,
  %``A general framework for implementing NLO calculations in shower Monte Carlo programs: the POWHEG BOX,''
  JHEP \textbf{06}, 043 (2010)
  %doi:10.1007/JHEP06(2010)043
  [arXiv:1002.2581 [hep-ph]].
  %3764 citations counted in INSPIRE as of 25 Aug 2022

%\cite{Sjostrand:2007gs}
\bibitem{Sjostrand:2007gs}
T.~Sjostrand, S.~Mrenna and P.~Z.~Skands,
``A Brief Introduction to PYTHIA 8.1,''
Comput. Phys. Commun. \textbf{178} (2008), 852-867
%doi:10.1016/j.cpc.2008.01.036
[arXiv:0710.3820 [hep-ph]].
% 6504 citations counted in INSPIRE as of 27 Jun 2022

%\cite{ATLAS:2012tjt}
\bibitem{ATLAS:2012tjt}
G.~Aad \textit{et al.} [ATLAS],
%``Measurement of the jet radius and transverse momentum dependence of inclusive jet suppression in lead-lead collisions at $\sqrt{s_{NN}}$= 2.76 TeV with the ATLAS detector,''
Phys. Lett. B \textbf{719}, 220-241 (2013)
%doi:10.1016/j.physletb.2013.01.024
[arXiv:1208.1967 [hep-ex]].
%337 citations counted in INSPIRE as of 29 Nov 2022

%\cite{Bossi:2022fpc}
\bibitem{Bossi:2022fpc}
H.~Bossi [ALICE],
%``$R$-dependence of inclusive jet suppression and groomed jet splittings in heavy-ion collisions with ALICE,''
[arXiv:2208.14492 [nucl-ex]].
%2 citations counted in INSPIRE as of 25 Nov 2022

%\cite{Dokshitzer:1997in}
\bibitem{Dokshitzer:1997in}
Y.~L.~Dokshitzer, G.~D.~Leder, S.~Moretti and B.~R.~Webber,
%``Better jet clustering algorithms,''
JHEP \textbf{08}, 001 (1997)
%doi:10.1088/1126-6708/1997/08/001
[arXiv:hep-ph/9707323 [hep-ph]].
%1415 citations counted in INSPIRE as of 28 Nov 2022

%\cite{Larkoski:2014wba}
\bibitem{Larkoski:2014wba}
A.~J.~Larkoski, S.~Marzani, G.~Soyez and J.~Thaler,
%``Soft Drop,''
JHEP \textbf{05}, 146 (2014)
%doi:10.1007/JHEP05(2014)146
[arXiv:1402.2657 [hep-ph]].
%830 citations counted in INSPIRE as of 28 Nov 2022

%\cite{ALICE:2021aqk}
\bibitem{ALICE:2021aqk}
S.~Acharya \textit{et al.} [ALICE],
%``Direct observation of the dead-cone effect in QCD,''
[arXiv:2106.05713 [nucl-ex]].
%0 citations counted in INSPIRE as of 14 Jul 2021



%\cite{Dokshitzer:1991fd}
\bibitem{Dokshitzer:1991fd}
Y.~L.~Dokshitzer, V.~A.~Khoze and S.~I.~Troian,
%``On specific QCD properties of heavy quark fragmentation ('dead cone'),''
J. Phys. G \textbf{17}, 1602-1604 (1991).
%doi:10.1088/0954-3899/17/10/023

%\cite{Cunqueiro:2018jbh}
\bibitem{Cunqueiro:2018jbh}
  L.~Cunqueiro and M.~P.
  %``Searching for the dead cone effects with iterative declustering of heavy-flavor jets,''
  Phys.\ Rev.\ D {\bf 99} (2019) no.7,  074027
  %doi:10.1103/PhysRevD.99.074027
  [arXiv:1812.00102 [hep-ph]].
  %%CITATION = doi:10.1103/PhysRevD.99.074027;%%
  %9 citations counted in INSPIRE as of 25 Jul 2020

%\cite{Dai:2022sjk}
\bibitem{Dai:2022sjk}
W.~Dai, M.~Z.~Li, B.~W.~Zhang and E.~Wang,
%``Exposing the dead-cone effect of jet quenching in QCD medium,''
[arXiv:2205.14668 [hep-ph]].
%3 citations counted in INSPIRE as of 28 Nov 2022


%\cite{Cunqueiro:2022svx}
\bibitem{Cunqueiro:2022svx}
L.~Cunqueiro, D.~Napoletano and A.~Soto-Ontoso,
%``Dead-cone searches in heavy-ion collisions using the jet tree,''
[arXiv:2211.11789 [hep-ph]].
%0 citations counted in INSPIRE as of 28 Nov 2022


%\cite{Chang:2019sae}
\bibitem{Chang:2019sae}
N.~B.~Chang, Y.~Tachibana and G.~Y.~Qin,
%``Nuclear modification of jet shape for inclusive jets and $\gamma$-jets at the LHC energies,''
Phys. Lett. B \textbf{801}, 135181 (2020)
%doi:10.1016/j.physletb.2019.135181
[arXiv:1906.09562 [nucl-th]].
%29 citations counted in INSPIRE as of 28 Nov 2022


%\cite{Goncalves:2015prv}
\bibitem{Goncalves:2015prv}
D.~Goncalves, F.~Krauss and R.~Linten,
%``Distinguishing b-quark and gluon jets with a tagged b-hadron,''
Phys. Rev. D \textbf{93}, no.5, 053013 (2016)
%doi:10.1103/PhysRevD.93.053013
[arXiv:1512.05265 [hep-ph]].
%12 citations counted in INSPIRE as of 07 Dec 2022

%\cite{Ilten:2017rbd}
\bibitem{Ilten:2017rbd}
P.~Ilten, N.~L.~Rodd, J.~Thaler and M.~Williams,
%``Disentangling Heavy Flavor at Colliders,''
Phys. Rev. D \textbf{96}, no.5, 054019 (2017)
%doi:10.1103/PhysRevD.96.054019
[arXiv:1702.02947 [hep-ph]].
%31 citations counted in INSPIRE as of 07 Dec 2022

%\cite{ALICE:2022phr}
\bibitem{ALICE:2022phr}
 [ALICE],
%``Measurements of groomed-jet substructure of charm jets tagged by ${\rm D}^0$ mesons in proton-proton collisions at $\sqrt{s}$ = 13 TeV,''
[arXiv:2208.04857 [nucl-ex]].
%1 citations counted in INSPIRE as of 28 Nov 2022

%\cite{Zhang:2023jpe}
\bibitem{Zhang:2023jpe}
Q.~Zhang, Z.~X.~Xu, W.~Dai, B.~W.~Zhang and E.~Wang,
%``Substructures of heavy flavor jets in $pp$ and PbPb collisions at $\sqrt{s}$ = 5.02 TeV,''
[arXiv:2303.08620 [nucl-th]].
%0 citations counted in INSPIRE as of 30 Mar 2023

\bibitem{Li:2017wwc}
  H.~T.~Li and I.~Vitev,
  %``Inverting the mass hierarchy of jet quenching effects with prompt $b$-jet substructure,''
  Phys.\ Lett.\ B {\bf 793}, 259 (2019)
  %doi:10.1016/j.physletb.2019.04.052
  [arXiv:1801.00008 [hep-ph]].
  %%CITATION = doi:10.1016/j.physletb.2019.04.052;%%
  %14 citations counted in INSPIRE as of 04 Jun 2019

%\cite{Sirunyan:2017bsd}
\bibitem{Sirunyan:2017bsd}
  A.~M.~Sirunyan {\it et al.} [CMS Collaboration],
  %``Measurement of the Splitting Function in $pp$ and Pb-Pb Collisions at $\sqrt{s_{_{\mathrm{NN}}}} =$ 5.02 TeV,''
  Phys.\ Rev.\ Lett.\  {\bf 120} (2018) no.14,  142302
  %doi:10.1103/PhysRevLett.120.142302
  [arXiv:1708.09429 [nucl-ex]].
  %%CITATION = doi:10.1103/PhysRevLett.120.142302;%%
  %70 citations counted in INSPIRE as of 13 May 2020

%\cite{Acharya:2019djg}
\bibitem{Acharya:2019djg}
  S.~Acharya {\it et al.} [ALICE Collaboration],
  %``Exploration of jet substructure using iterative declustering in pp and Pb?��CPb collisions at LHC energies,''
  Phys.\ Lett.\ B {\bf 802} (2020) 135227
  %doi:10.1016/j.physletb.2020.135227
  [arXiv:1905.02512 [nucl-ex]].
  %%CITATION = doi:10.1016/j.physletb.2020.135227;%%
  %28 citations counted in INSPIRE as of 25 Jul 2020


\end{thebibliography}

%%%%%%%%%%%%%%%%%%%%%%%%%%%%%%%%%%%%%%%%%%
\PublishersNote{}
\end{adjustwidth}

\end{document}
