While the results obtained are an encouraging first step towards open-set 3D semantic segmentation there are still many open questions to improve such approaches, some of which we discuss in the following.

Currently, the largest factor limiting segmentation performance is the quality of the vision-language features. While LSeg uses natural language features from CLIP trained on a very large dataset, the visual encoder is trained on the small closed-set ADE20K dataset. If we were able to compute dense pixel-aligned visual-language features from open-set web scraped data without requiring any human annotations, we believe that results could eventually surpass supervised learning methods. \cite{ranasinghe2022perceptual} presented some promising initial results on learning pixel aligned features without using segmentation masks or other expert annotations. 

In real-time experiments, our system relied on poses coming from a SLAM system. If many bad poses are computed by the SLAM system, the 3D representation could become corrupted by bad updates. Possible solutions include treating the sparse SLAM poses as initial guesses and optimizing the poses jointly with scene geometry, as in \cite{sucar2021imap, zhu2022nice}, or bad poses could be filtered out by analyzing the photometric or geometric error across frames. 

In robotics, downstream modules, such as motion planners and high-level planning systems, might benefit from a more explicit and principled representation of geometry than what we presented in this paper. For example, signed distance function based approaches \cite{wang2021neus} might provide better surface and occupancy reconstruction and have other favorable properties, such as the ability to compute the normal of a surface by differentiating through the distance function. For the time being, our method is limited to static scenes. Dealing with moving objects within scenes remains an open problem, but promising recent research \cite{kong2023vmap} suggests that extending neural implicit representations to dynamic scenes might be feasible.
