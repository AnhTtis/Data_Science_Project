%%%%%%%%%%%%%%%%%%%%%%%%%%%%%%%%%%%%%%%%%%%%%%%%%%%%%%%%%%%%%%%%%%%%%%%%%%%%%%%%
%2345678901234567890123456789012345678901234567890123456789012345678901234567890
%        1         2         3         4         5         6         7         8

\documentclass[letterpaper, 10 pt, conference]{ieeeconf}  % Comment this line out if you need a4paper

%\documentclass[a4paper, 10pt, conference]{ieeeconf}      % Use this line for a4 paper

\IEEEoverridecommandlockouts                              % This command is only needed if 
                                                          % you want to use the \thanks command

\overrideIEEEmargins                                      % Needed to meet printer requirements.

%In case you encounter the following error:
%Error 1010 The PDF file may be corrupt (unable to open PDF file) OR
%Error 1000 An error occurred while parsing a contents stream. Unable to analyze the PDF file.
%This is a known problem with pdfLaTeX conversion filter. The file cannot be opened with acrobat reader
%Please use one of the alternatives below to circumvent this error by uncommenting one or the other
%\pdfobjcompresslevel=0
%\pdfminorversion=4

% See the \addtolength command later in the file to balance the column lengths
% on the last page of the document

% The following packages can be found on http:\\www.ctan.org
\usepackage{graphics} % for pdf, bitmapped graphics files
\usepackage{graphicx}
\usepackage{epsfig} % for postscript graphics files
% \usepackage{mathptmx} % assumes new font selection scheme installed
\usepackage{times} % assumes new font selection scheme installed
\usepackage{amsmath} % assumes amsmath package installed
\usepackage{amssymb}  % assumes amsmath package installed
\usepackage{bbm}
\usepackage{hyperref}
\usepackage{csquotes}
\usepackage{cite}
\usepackage{booktabs}
\usepackage{siunitx}

\usepackage[utf8]{inputenc}
\usepackage[english]{babel}
% \usepackage[backend=bibtex,bibstyle=ieee,citestyle=numeric-comp]{biblatex}
% \addbibresource{references.bib}

\newcommand{\norm}[1]{\left\lVert#1\right\rVert}

% \title{\LARGE \bf
% Zero-Shot Open-Vocabulary Volumetric Segmentation through Vision-Language Feature Fields
% }

\title{\LARGE \bf
Neural Implicit Vision-Language Feature Fields
}

\author{Kenneth Blomqvist$^{1}$, Francesco Milano$^{1}$, Jen Jen Chung$^{2}$, Lionel Ott$^{1}$ and Roland Siegwart$^{1}$% <-this % stops a space
\thanks{$^{1}$Autonomous Systems Lab, Swiss Federal Institute of Technology in Z\"urich, Switzerland. 
    {\tt\small kblomqvist@mavt.ethz.ch}}%
\thanks{$^{2}$School of ITEE, The University of Queensland, Australia.}%
\thanks{This project has received funding from EU Horizon 2020 program, project PILOTING H2020-ICT-2019-2 871542.}
}

\begin{document}

\maketitle
\thispagestyle{empty}
\pagestyle{empty}

%%%%%%%%%%%%%%%%%%%%%%%%%%%%%%%%%%%%%%%%%%%%%%%%%%%%%%%%%%%%%%%%%%%%%%%%%%%%%%%%
\begin{abstract}

Recently, groundbreaking results have been presented on open-vocabulary semantic image segmentation. Such methods segment each pixel in an image into arbitrary categories provided at run-time in the form of text prompts, as opposed to a fixed set of classes defined at training time. In this work, we present a zero-shot \emph{volumetric} open-vocabulary semantic scene segmentation method. Our method builds on the insight that we can fuse image features from a vision-language model into a neural implicit representation. We show that the resulting feature field can be segmented into different classes by assigning points to natural language text prompts. The implicit volumetric representation enables us to segment the scene both in 3D and 2D by rendering feature maps from any given viewpoint of the scene. We show that our method works on noisy real-world data and can run in real-time on live sensor data dynamically adjusting to text prompts. We also present quantitative comparisons on the ScanNet dataset.
\end{abstract}

%%%%%%%%%%%%%%%%%%%%%%%%%%%%%%%%%%%%%%%%%%%%%%%%%%%%%%%%%%%%%%%%%%%%%%%%%%%%%%%%
\section{INTRODUCTION}

A key component of building intelligent robots capable of operating in unstructured and cluttered human environments is the representation used to model the robot's surroundings.
Often times representations have to trade-off properties which depend on the usage scenario. These properties include the quality of the reconstruction, the ability to integrate sensor data continuously, and the computational complexity to query the representation. The importance of these aspects differs based on what components of a robotic system needs to use the representation, dictating the requirements for available capabilities, sensor data throughput, or query latency. For instance, an obstacle avoidance system needs to query for occupancy at high frequency, while a high-level planning system needs access to semantic knowledge, and finally a grasp planning system requires fine-grained segmentation information.

While in the past occupancy was the main information of interest, robotics has moved towards richer representations using semantics in recent years. A challenge is that most semantic approaches use a fixed, closed set, of pre-determined semantic labels. However, real environments contain more than a few dozen classes, and thus methods capable of handling arbitrary semantic classes, i.e. open set, are desirable. Additionally, objects in an environment do not necessarily belong to distinct, mutually exclusive classes. Certain objects might belong to several classes. A bookshelf is also a piece of furniture, for example. For high-level planning purposes, being able to reason about relations between their semantics might also be useful.

An environment representation that has wide applicability has several desirable properties, including: (1) can be built incrementally as the robot explores the environment, (2) enables real-time integration of new measurements, (3) has a compact memory footprint, (4) represents geometry at a high-level of detail, (5) is differentiable, (6) supports open set semantic queries, and (7) allows fast querying by downstream modules. Previously introduced 3D semantic scene representations are either built from global scene information \cite{peng2022openscene}, use closed set semantics \cite{grinvald2019volumetric, rosinol2020kimera, zhi2021ilabel, mazur2022feature}, operate on a fixed level of detail \cite{grinvald2019volumetric, rosinol2020kimera}, or are not differentiable \cite{grinvald2019volumetric, rosinol2020kimera}. In this paper, we take a step towards a representation which has the above-mentioned properties.

\begin{figure}[t]
    \centering
    \includegraphics[width=0.8\linewidth]{images/hero_image.jpg}
    \caption{Our method enables real-time segmentation of scenes into arbitrary text classes provided at run-time.}
    \label{fig:main}
    \vspace{-0.2cm}
\end{figure}

Vision-language models (VLM) have shown remarkable performance on open vocabulary object detection \cite{zareian2021open, gu2021open}. Recently, these results have been extended to dense semantic segmentation \cite{ghiasi2022scaling, li2022language, zhang2022glipv2, zou2022generalized}. Some of these methods \cite{ghiasi2022scaling, li2022language} associate each pixel with a semantically meaningful vector, which is embedded in the same high-dimensional vector space as natural language prompts through a text encoder. This allows direct computation of the similarity between text prompts and image features at run-time.

As vision-language models can be trained on massive web-scale datasets that can be collected automatically without human supervision, they often show better generalization capabilities than models trained on smaller closed-set manually curated datasets. Additionally, VLMs can capture the long tail of scenarios and classes that are so rare that they are unlikely to be included in curated datasets. These properties offer great promise for applications in robotics, where we might want our robots to be able to perform new tasks in never-before-seen environments. 

In this paper, we present a method for grounding dense vision-language features into a 3D implicit neural representation that can be built up incrementally, in real-time, as new observations come in. We jointly model radiance, vision-language model features, and density in the scene using an implicit neural representation. Our representation can be incrementally built up given posed images of the scene and a pre-trained language model. We can directly compute the similarity between natural language text prompts and either 3D points or 2D image coordinates for any given viewpoint of the scene through volumetric rendering. This enables semantically segmenting a scene zero-shot into text categories provided at run-time, without having to fine-tune the system on any domain specific semantics.

In experiments, we showcase results in real-world experiments where we build up our scene representation in real-time on a real system, and demonstrate the ability to segment the scene into different classes provided as natural language prompts at run-time. We additionally present quantitative segmentation results on the large and diverse ScanNet dataset. To the best of our knowledge, our method is the first real-time capable 3D vision-language neural implicit representation. Our implementation will be made available through the Autolabel project \footnote{https://github.com/ethz-asl/autolabel}.

% Need to check if the fused features are actually better than the raw 2D features. If that is the case, this would be a nice result to showcase.


\section{RELATED WORK}

\subsection*{Open Vocabulary Semantic Segmentation and Vision-Language Models}

CLIP \cite{radford2021learning} introduced a visual-language model capable of mapping images into the same vector space as natural language queries by correlating images to their text descriptions mined from the open web.
Open vocabulary segmentation methods typically learn dense features which are compared to text queries given at run-time \cite{li2022language, ghiasi2022scaling}. Others take a multi-task learning approach, fusing a task prompt with the architecture \cite{zhang2022glipv2, zou2022generalized}.
Other methods such as Clippy \cite{ranasinghe2022perceptual} explored learning pixel-aligned visual-language models from large scale web datasets without requiring segmentation labels, potentially enabling large-scale open set training, if the results can be extended to full semantic segmentation. 

\subsection*{Language Models in Robotics}

Large language models have been explored as an approach to high-level planning \cite{ahn2022can, huang2022visual, song2022llm, chen2022open, raman2022planning} and scene understanding \cite{chen2022leveraging, ha2022semantic}. Vision-language models embedding image features into the same space as text have been applied to open vocabulary object detection \cite{song2022llm, chen2022open}, natural language maps \cite{blukis2021few, chen2022open, shafiullah2022clip, huang2022visual, tan2022self}, and for language-informed navigation \cite{shah2022lm, wang2022find, majumdar2022zson}. 

Recent methods have explored fusing global CLIP features \cite{shafiullah2022clip}, image caption embeddings \cite{ding2022language}, or dense pixel-aligned \cite{peng2022openscene} visual-language model features into a point cloud representation for scene understanding. Concurrent work ConceptFusion \cite{jatavallabhula2023conceptfusion} explores building multi-modal semantic maps by fusing features from vision-language models as well as audio into a reconstructed 3D point cloud. Similar to these, we also fuse VLM features into a 3D representation. Unlike \cite{shafiullah2022clip, peng2022openscene, jatavallabhula2023conceptfusion}, we use a continuous neural representation of geometry and semantics which we learn jointly through volumetric rendering. \cite{peng2022openscene, ding2022language} fuse image features from a pre-built point cloud using a multi-view fusion method and learn a 3D convolutional network to map scene points to dense features. Our representation can be built incrementally as measurements are collected and does not require global scene geometry upfront. 


% note: Should we somehow specify what we mean by map. I feel it's quite vague and in this case I consider both NLmaps (2D), Clip fields (sparse 3D point cloud), and our method (dense volumetric 3D) to be "maps", even though they are quite different. A SLAM person might also have a heavy preconception of a map being a representation of sparse 3D landmarks.  

% Clip-Fields \cite{shafiullah2022clip} introduced the first system capable of associating natural language text queries to coarse spatial locations within a scene. They do this by fusing CLIP features into a neural representation which maps points from a point cloud to CLIP feature vectors computed from images observing each point. While this results in a natural language queriable 3D representation, the representation is not fine enough to enable dense scene understanding, as only a single global feature vector is extracted from each image. 

% \cite{huang2022visual} 

% \cite{peng2022openscene} presented a system enabling dense 3D scene understanding by fusing dense, pixel-aligned LSeg or OpenSeg features using multi-view fusion and a 3D convolutinal neural network. 

% Our approach similarly leverages dense pixel-aligned visual-language model features, but we directly fuse extracted features into a neural field, outputting features, color and density for any point in a volume. As our approach uses a continuous feature field, we are able to render fine-grained feature maps for any given viewpoint in the scene. Our approach also doesn't require a point cloud of the scene, as our neural field representation implicitly and jointly learns the geometry, radiance and feature field from the posed input images, enabling the use of regular RGB cameras. % Let's see if BitF feature fusion actually works better than their approach.

\subsection*{Semantic Scene Representations}

Voxel-based map representations have been proposed to store semantic information about a scene \cite{strecke2019fusion, grinvald2019volumetric, narita2019panopticfusion, rosinol2020kimera, schmid2022panoptic}. These methods assign a semantic class to each individual voxel in the scene. Voxel-based dense semantic representations typically operate on static scenes, but some have explored modeling dynamic objects \cite{xu2019mid, grinvald2021tsdf++}. 

Scene graphs \cite{armeni20193d, hughes2022hydra, wu2021scenegraphfusion} have also been proposed as a candidate for a semantic scene representation that can be built-up online. Such methods decompose the scene into a graph where edges model relations between parts of the scene. The geometry of the parts are typically represented as a signed distance functions stored in a voxel grid \cite{huang2022visual}.

Neural implicit representations infer scene semantics \cite{zhi2021place, zhi2021ilabel, blomqvist2022baking, mazur2022feature, fu2022panoptic, siddiqui2022panoptic, liu2022unsupervised, kundu2022panoptic} jointly with geometry using a multi-layer perceptron or similar parametric model. These have been extended to dynamic scenes \cite{kong2023vmap}. Neural feature fields \cite{kobayashi2022d3f, tschernezkineural, blomqvist2022baking, mazur2022feature} are neural implicit representations which map continuous 3D coordinates to vector-valued features. Such representations have shown remarkable ability at scene segmentation and editing. \cite{kobayashi2022d3f} also presented some initial results on combining feature fields with vision-language features, motivating their use for language driven semantic segmentation and scene composition. 



\section{METHOD}
\section{
Automatically Countering Essentialism
} \label{sec:method}
We operationalize our counterstatement generation by focusing on the expression of stereotypes through generics (\S\ref{ssec:link-generics-stereotypes}). 
Inspired by work in psychology and philosophy, we construct five types of counterstatements to a stereotype (\S\ref{sec:countertypes}).

\begin{table*}[t]
    \centering
    \scalebox{0.85}{
    \begin{tabular}{ll}
        \hline
        \textbf{TEXT:} \textit{RT @Vbomb20: Got these hoes on my dick like brad pitt} & \textbf{GENERIC:} Women are sex objects.\\
        \hdashline
        \multicolumn{2}{l}{
            \begin{tabular}[t]{@{}l@{}}Actually this is a generalization about women. +\\
            \quad\quad \textbf{(\ref{template1}-\textsc{Grp})}\ The following women are not sex objects: businesswomen, female atheletes, and female movie stars.\\
            \quad\quad \textbf{(\ref{template1}-\textsc{Ind})}\ The following women are not sex objects: ellen degeneres, sarah palin, and rachel maddow.\\
            \quad\quad \textbf{(\ref{template2})}\ Men can also be sex objects.\\
            \quad\quad \textbf{(\ref{template3})}\ Lots of people can be sex objects.\\
            \quad\quad \textbf{(\textsc{Tol})}\ All groups of people deserve tolerance.\end{tabular}
        }\\
        \hline
        % \begin{tabular}[t]{@{}l@{}}\textbf{TEXT:} \textit{They say science disproves Islam But they also say}\\\quad \textit{the world started with a BANG}\end{tabular} & \textbf{GENERIC.:}  muslims blow things up\\
        % \hdashline
        % \multicolumn{2}{l}{
        % \begin{tabular}[t]{@{}l@{}} Actually, this is a generalization about muslims. +\\
        % \quad\quad \textbf{(1a)}\ The following muslims do not blow things: male muslim businessmen, muslims who attend mosque,\\\quad\quad\quad\quad and male muslim movie stars.\\
        % \quad\quad \textbf{(1b)}\ The following muslims do not blow things: adult muslim men, salman rushdie, and mohammad amir.\\
        % \quad\quad \textbf{(2)}\ Christian folks may also blow things.\\
        % \quad\quad \textbf{(3)}\ Lots of people may blow things.\\
        % \quad\quad \textbf{(4)}\ All groups of people deserve tolerance.\end{tabular}
        % }\\
        % \hline
        \textbf{TEXT:} \textit{What's black and doesn't work? Half of London} & \textbf{GENERIC:}  Black people don't work\\
        \hdashline
        \multicolumn{2}{l}{
        \begin{tabular}[t]{@{}l@{}}Actually, this is a generalization about black people. + \\ \quad\quad \textbf{(\ref{template1}-\textsc{Grp})}\ The following black people work: black businessmen, famous black people, and black movie stars.\\
        \quad\quad \textbf{(\ref{template1}-\textsc{Ind})}\ The following black people work: barack obama, misty copeland, and usain bolt.\\
        \quad\quad \textbf{(\ref{template2})}\ White folks may also not work.\\
        \quad\quad \textbf{(\ref{template3})}\ Lots of people don't work.\\
        \quad\quad \textbf{(\textsc{Tol})}\ All groups of people deserve tolerance. \end{tabular}
        }
        \\
        \hline
        \hline
        \begin{tabular}[t]{@{}l@{}}\textbf{TEXT:} \textit{How do you kill a thousand flies in one hit?}\\ \textit{Slap an Ethiopian in the face.}\end{tabular} & \textbf{GENERIC:}  Ethiopian people are dirty.\\
        \hdashline
        \multicolumn{2}{l}{
        \begin{tabular}[t]{@{}l@{}}Actually, this is a generalization about ethiopian people. + \\ \quad\quad \textbf{(\ref{template1}-\textsc{Grp})}\ The following ethiopian people are not dirty: male atheletes, female movie stars, \\\quad\quad\quad\quad\quad\quad\quad and people who practice judaism.\\
        \quad\quad \textbf{(\ref{template1}-\textsc{Ind})}\  The following ethiopian people are not dirty: kenyan marathon runners, michael jackson,\\\quad\quad\quad\quad\quad\quad\quad and ryan reynolds.\\
        % \quad\quad \textbf{(\ref{template2})}\ American folks can also be dirty.\\
        % \quad\quad \textbf{(\ref{template3})}\ Lots of people can be dirty.\\
        % \quad\quad \textbf{(\textsc{Tol})}\ All groups of people deserve tolerance. 
        \end{tabular}
        }
        \\
        \hline
        \begin{tabular}[t]{@{}l@{}}\textbf{TEXT:} \textit{A muslim enters a building..}\\ \textit{With 500 passengers and a plane}\end{tabular} & \textbf{GENERIC:}  Muslims are terrorists.\\
        \hdashline
        \multicolumn{2}{l}{
        \begin{tabular}[t]{@{}l@{}}Actually, this is a generalization about muslims. + \\ \quad\quad \textbf{(\ref{template1}-\textsc{Grp})}\ The following muslims are not terrorists: male muslim businessmen, muslims businessmen, \\\quad\quad\quad\quad\quad\quad\quad  and male muslim movie stars.\\
        \quad\quad \textbf{(\ref{template1}-\textsc{Ind})}\  The following muslims are not terrorists: adult muslim men, all muslims, and malala yousafzai.\\
        \quad\quad\quad\quad\quad\quad\quad $\bm{\hdots}$
        % \quad\quad \textbf{(\ref{template2})}\ Christian folks can also be terrorists.\\
        % \quad\quad \textbf{(\ref{template3})}\ Lots of people can be terrorists.\\
        % \quad\quad \textbf{(\textsc{Tol})}\ All groups of people deserve tolerance. 
        \end{tabular}
        }
        \\
        \hline
    \end{tabular}
    }
    \caption{Automatically generated counterstatements (\S\ref{sec:countertypes}) from our system. The bottom two examples illustrate challenges with factuality in the \textsc{Dir} counterstatements.}
    \label{tab:statementexs}
\end{table*}
\subsection{Stereotypes as Generics}\label{ssec:link-generics-stereotypes}
Many negative stereotypes are expressed as generics; they generalize a dangerous or harmful quality (e.g., being a drunkard) to an entire group (e.g., Scots) based on the behavior of only a few individuals. ~\citet{leslie2008generics,leslie2017original} termed such generics \textbf{striking} and argued that such generalizations are based upon an assumption that all members of the group in question (e.g., Scots) are \textit{disposed} to possess the dangerous or harmful quality. We argue that many stereotypes can also be interpreted as asserting a \textbf{quasi-unique} association between the group and quality. For example, ``Scots are drunkards''  also implies that Scots are distinctly more likely than other groups (e.g., the English) to exhibit drunkenness.
In our work, we assume that all stereotypes under consideration are generics and have both interpretations. 

Since generics are unquantified, they naturally allow for \textbf{exceptions} (i.e., counterexamples to the generic).
While these exceptions may provide a relevant source of counter-statements for a stereotype, some evidence from psychology suggests that people are adept at maintaining their stereotyped beliefs in the face of such specific exceptions \cite[e.g.,][]{kunda1995maintaining}. Therefore, we experiment with a variety of different counter-statements.

\subsection{Generating Counter-Speech}
\label{sec:countertypes}
To generate counter-speech to stereotypes, we produce five types of outputs in three broad categories (see Table~\ref{tab:statementexs}). Since the stereotypes we consider are expressed as generics (e.g., ``Scots are drunkards''), they can be separated into three components: a \textit{group} (e.g., Scots), a \textit{relation} (e.g., are), and a \textit{quality} (e.g., ``drunkards''), which we use to construct the counter-speech. Additionally, we prepend the sentence ``Actually, this is a generalization about \textsc{GROUP}'' to each type of statement we generate, in order to contextualize the statements as counter-speech.


\paragraph{Direct Exceptions (\textsc{Dir})}
Direct exceptions present subgroups or individuals that do not have the quality specified in the generic, and thereby counter the striking or extrapolating implications of the stereotype. For example, for ``Scots are drunkards'', the extrapolating implication is that ``\textit{All} Scots are drunkards''; thus, direct exceptions would be either individual Scots (e.g., Ewan McGregor\footnote{\url{https://fherehab.com/learning/celebrities-who-dont-drink}}) or sub-groups of Scots (e.g., Scottish babies) who are not drunkards. We follow~\citet{allaway2022penguins} who propose that these exceptions can be constructed with the following template:
\setlength{\abovedisplayskip}{3pt}
\setlength{\belowdisplayskip}{3pt}
\begin{align}
    &\textsc{Group}(x) + \text{not } \textit{relation} + \textsc{Quality}. \label{template1}\tag{\textsc{Dir}}
\end{align}
We say that $\textsc{Group}(x)$ is satisfied if $x$ is either a specific member of the group or a subgroup. 
We generate subtypes (i.e., subgroups and specific group members) using GPT-3~\citep{brown2020language}. In particular, we prompt GPT-3 with a list of subtypes for an example group not in our data and query the model to produce subtypes for \textsc{Group} as the prompts completion. We choose as our example group ``men'' (see Appendix~\ref{appsec:gpt3} for prompts).
We then construct exceptions following template~\ref{template1} using each generated subtype. 
In order to select the most truthful and relevant subtypes, we apply a truth discriminator from \citet{allaway2022penguins} 
to each exception, and rank the subtypes by the probability of being true and relevant. 
We construct the final statements by combining the top three ranked subgroups into a single exception ((\ref{template1}-\textsc{Grp}) in Table~\ref{tab:statementexs}) and combining the top three individuals into a single exception ((\ref{template1}-\textsc{Ind}) in Table~\ref{tab:statementexs}).

\paragraph{Broadening Exceptions (\textsc{Alts})}
Broadening exceptions challenge the quasi-unique implication of the generic by attributing the quality in question to a different social group (e.g., ``Americans can also be drunkards''). ~\citet{allaway2022penguins} propose that these exceptions follow the template:
\begin{align}
    &{\nsim}\textsc{Group}(x) + \textit{relation} + \textsc{Quality}.\label{template2}\tag{\textsc{Alt}}
\end{align}
where ${\nsim}\textsc{Group}$ indicates a contextually relevant alternative group. For example, if \textsc{Group} = \textsc{Scots}, then a contextually relevant alternative would be ${\nsim}\textsc{Group}$ = \textsc{Americans}. In our work, we define the relevant alternative group ${\nsim}\textsc{Group}$ to be the perceived oppressing group. For example, if the generic is ``women are vain'', then ``men'' would be the relevant alternative group ${\nsim}\textsc{Women}$ (i.e., the oppressing group). To avoid generating stereotypes about the oppressing group, we convert the relation into a hedged form (see Appendix~\ref{appsec:dataprocess}). For example, if the relation is ``are'', the hedged form of the relation would be ``can be''. 

\begin{figure*}[t!]
    \centering
    \begin{subfigure}[b]{0.5\textwidth}
        \centering
        \includegraphics[width=\textwidth]{graphs/images/first-choice.pdf}
        \caption{First choice.}
        \label{subfig:firstchoice}
    \end{subfigure}%
    \begin{subfigure}[b]{0.5\textwidth}
        \centering
        \includegraphics[width=\textwidth]{graphs/images/second-choice.pdf}
        \caption{Second choice.}
        \label{subfig:secondchoice}
    \end{subfigure}
    \caption{Percentage of annotators that selected each counterstatement type (\S\ref{sec:method}) across all three settings.
    % \maarten{Can we add the raw percentage number (e.g., `15') in the bars?}
    }
    \label{fig:choicecharts}
\end{figure*}


\paragraph{Broadening Universals (\textsc{Lots})}
In addition to broadening exceptions, we generate \textit{broadening universals}, which maximize the scope of the quality so that it includes people in general, rather than any specific social group. That is, we generate statements following:
\begin{align}
    \text{Lots of people} + \textit{relation} + \textsc{Quality}.\label{template3}\tag{\textsc{Lots}}
\end{align}
For example, ``Lots of people are drunkards" is a broadening universal for the stereotype ``Scots are drunkards''. See (\ref{template3}) in Table~\ref{tab:statementexs}. Similarly to the statements following template~\ref{template2}, we also hedge the relation in template~\ref{template3}. 



\paragraph{Tolerance (\textsc{Tol})}
Finally, we include the denouncing statement, ``All groups of people deserve tolerance'', since denouncing is a common strategy in countering hate-speech~\cite[e.g.,][]{mathew2019thou,qian2019benchmark,Ziegele2018JournalisticCI}. This form of counter-speech does not depend on the details of the generic in question and so is the same for all stereotypes. See (\textsc{Tol}) in Table~\ref{tab:statementexs}. 


\section{EXPERIMENTAL RESULTS}
\section{Experiments}
We use the CelebA \cite{liu2015faceattributes} dataset as our ``base'' dataset, and compare the performance of our method against several variants and semantically related datasets both qualitatively and quantitatively. To evaluate our method we consider the following set of data splits and train StyleGAN models on all of them: one model trained with the full dataset, 5 models trained missing one attribute (hats, glasses, males, females, and beards), and two models trained missing more than one attribute (beards/hats and smiles/glasses/ties). This setup allows us to control for missing attributes in a systematic manner, while also testing the effectiveness of our approach on practical datasets.

When available, we utilize pre-trained StyleGAN \cite{karras2019style} models, like for Met Faces and the AFHQ animal faces dataset \cite{choi2018stargan}. We also include non-human style-transfered GANs of cartoons \cite{cartoonStyleGan22}, Disney \cite{cartoonStyleGan22}, and Anime GAN \cite{danbooru2021}. Lastly, we investigate using non-StyleGAN based architectures in the supplement (StyleGAN3-t vs. StyleGAN3-r \cite{karras2021alias} and PGGAN \cite{karras2017progressive} vs. GANformer \cite{hudson2021ganformer2}). 

\subsection{Experimental Setup}
\label{sec:setup}
For all GANs we train, we use StyleGAN-ADA \cite{Karras2020ada} trained over 15 million images at a resolution of 128. We train 8 models in total, one for each of the 8 CelebA splits we used.   

For training the direction models, we randomly sample points for $10,000$ iterations. For all learnable directions, we set $\alpha=3$, how far a model must walk along a particular direction as in equation \eqref{eq:direction}. For the feature extractor, $\mathcal{F}$, we experiment with four ResNet-50 models and three ViT models: original ResNet trained on ImageNet \cite{he2016deep}, a ResNet trained to predict CelebA attributes, a ResNet trained to be robust to style using advBN\cite{shu2021encoding}, and a ResNet trained with CLIP \cite{radford2021learning}; we use the original ViT trained on ImageNet \cite{dosovitskiy2020vit}, a ViT trained with CLIP, and the recently introduced ViT Masked Autoencoder (MAE) \cite{MaskedAutoencoders2021}.  Our batch-size is set to $10$, where $2$ samples are needed for positive pairs and  $5$ directions have gradients enabled (randomly selected for each iteration). This is due to a memory constraint on a single experiment for a 15GB Tesla T4 GPU. We use an Adam \cite{kingma2014adam} optimizer with learning rate $0.001$ and default parameters. We only update the direction models, and keep all other networks fixed. As with some earlier observations, we find better performance operating in the $\mathcal{W}$ space of StyleGAN, so we restrict all our direction models to operate in it. Larger generator's outputs (512 resolution) are cropped to fit a given feature extractor, other generator's outputs are resized to fit.

We fix the total number of directions $N = 16$, with $12$ common and $4$ novel/missing directions. This is a hyper-parameter that can be adjusted depending on the application, but in general we found using a larger $N$ resulted in the model learning multiple similar directions with minor variations between (e.g. slightly changing hair color and adding eyeglasses). This is the case even when using existing approaches like LatentCLR \cite{yuksel2021latentclr}, where larger $N$ values lead to attributes that are very similar. Finally, we set the $L_{xent}$ overlap direction trade-off $\lambda_a = 0.1$. We found higher values to make directions not self-consistent and lower values made directions ignore the other GAN.

\myparagraph{DRE Training} For the $DRE$ models, we use 2-layer MLPs trained to minimize the objective specified in \eqref{eq:DRE_loss}. These models are trained for $1,000$ iterations with a default Adam optimizer. At each step, we draw $32$ samples from each GAN and project it into the feature space defined by $\mathcal{F}$. The $DRE$ models are pre-trained prior to direction model training.

\subsection{Evaluation metrics}
As the problem of interest in this work is new, there are no existing metrics that allow us to evaluate performance. Here, we introduce three new metrics for evaluating the quality of learned attributes. The first metric measures the quality of a single GAN and its corresponding feature extractor, while the other two metrics require multiple GANs. CelebA is our testbed dataset as we can leverage the multi-attribute labels it provides.
Let $A(\bm{x})$ be the attribute vector predicted from a pre-trained CelebA classifier $A$. We can then compute the attribute difference produced by using a given direction modification, $\bm{a}_n(G(z,\delta_n)) = A(G(z)) - A(G(D(z,\delta_n))$.

\myparagraph{I. Single GAN attributes.}
First, we want to explore to what extent the choice of feature space affects the attribute identification process for a single GAN, such that they can effectively be leveraged for multiple models. To study this, we use a pre-trained CelebA attribute classifier to measure the degree to which the directions are focused on a single attribute.
Our metric is an entropy-based metric computed over $B$ samples from the generator and $N$ directions:

\begin{equation}
\text{Score}_{ent} = C * \frac{1}{B} \frac{1}{N} \sum^B_{b=1} \sum^N_{n=1} \mathcal{H}(\sigma( \bm{a}_n(G(z,\delta_n)_b)))
\end{equation}

where $C$ is a constant that scales the score (we use $C$=100), $\mathcal{H}(\bm{p}) = \sum_i - p_i \text{log}(p_i)$ is the entropy function, $\sigma$ is the softmax function and $G(z,\delta_n)_b$ is the $b$th sample. 


\myparagraph{II. Common attributes}
To compute a score for the common directions, we rely on the intuition that images perturbed along the same attribute will result in similar prediction changes through an ``oracle" attribute classifier. To compute this, we first randomly sample several images from each GAN. Then, we apply a common direction (inferred by our model) $n$ to these samples and compute the average difference vectors $\bm{\bar{a}}_{(n,1)}$ for GAN 1 and $\bm{\bar{a}}_{(n,2)}$ for GAN 2, where the bar indicates that they are normalized to have unit norm. We then compute the cosine similarity between the attributes difference vectors to measure the similarity of the learned common directions from the two distributions, where a score of 1 indicates they are identical.
\begin{equation}
\text{Score}_{cos} = \frac{1}{N} \sum_n^N  \cos(\bm{\bar{a}}_{(n,1)}, \bm{\bar{a}}_{(n,2)}) 
\end{equation}

\myparagraph{III. Novel/Missing attributes}
To measure attributes that are exclusive to a particular GAN, we train different leave-k-attributes-out GANs, on artificial splits of the CelebA dataset as described earlier.  Here, we have a ground truth set of attributes which are missing $\bm{\mathcal{M}}$. In order to measure how well the model finds the missing attributes in $\bm{\mathcal{M}}$, we derive a metric from mean reciprocal rank (MRR) \cite{radev2002evaluating,voorhees1999trec}. Let $\bm{a} = (\bm{a}_1, \ldots, \bm{a}_N)$ be a collection of all difference vectors as defined earlier. Let $rank(m,\bm{a}_n)$ be the rank of missing attribute $m$ in difference vector $\bm{a}_n$ and let $\bm{\mathcal{I}}_u$ be the indices of unique directions. The unique direction score is computed as shown in Eqn \ref{eq:score_unique}, where higher is better.


\begin{equation}
\text{Score}_{unique}(\bm{a}) = \frac{1}{|\bm{\mathcal{M}}|} \sum_{m \in \bm{\mathcal{M}}} \underset{n \in \bm{\mathcal{I}}_u} {\max} \left( \frac{1}{rank(m,\bm{a}_n)} \right)
\label{eq:score_unique}
\end{equation}

\subsection{Results}
Here we present the quantitative and qualitative results that illustrate the performance of our proposed xGA method.  

\myparagraph{Single GAN attribute discovery}
Even though our focus is not on the single GAN attribute discovery tasks, our experiments show that the utilization of an external latent space can improve the overall task on several important metrics.
Table \ref{tbl:entropy_results} depicts the impact of different types of pre-trained features on the disentanglement properties in attribute discovery over all eight of our CelebA test-bed GANs. The original LatentCLR does poorly as it heavily focuses on non-dataset attributes (e.g. rotation/zoom/etc). As expected, the CelebA attribute classifier (which is an oracle) is the best at finding low entropy directions as it is the only extractor model trained with additional information (i.e. labels) about the datasets of interest. Overall, most features perform adequately at the task especially compared to operating in the latent space of the GAN itself.
We also show qualitatively the power of using a feature extractor in figure \ref{fig:1gan_examples}, which shows the top four most changed attributes (according to the classifier) for a few methods: xGA finds more diverse dirdctions. In the supplement, we include a detailed example for all methods starting with a randomly selected initial point. 

\begin{table}[!tb]
\centering
\resizebox{1.0 \columnwidth}{!}{
\begin{tabular}{rll}
Method            & $\mathcal{H}_{\text{score}}$ ($\downarrow$) & $\mathcal{D}_{\text{score}}$ ($\uparrow$) \\ \hline
SeFa  \cite{shen2021closed}                          & $4.006 \pm{0.259}$ & $1.031 \pm{0.077}$  \\
	
% SeFa (n=100) \cite{shen2021closed}                 & $5.811 \pm{0.327}$ &  $12345 \pm{0.1234}$   \\
LatentCLR \cite{yuksel2021latentclr}                  & $2.348 \pm{0.203}$ &   $0.749 \pm{0.929}$   \\ 
% LatentCLR (n=100)           & $3.818 \pm{0.084}$  & $12345 \pm{0.1234}$    \\ 
Voynov \cite{voynov2020unsupervised}                    & $2.508 \pm{0.069}$ &  $0.585 \pm{0.725}$    \\ 
Hessian \cite{peebles2020hessian}                    & $2.707 \pm{0.145}$ &  $0.642 \pm{0.795}$    \\ 
Jacobian \cite{wei2021jacobian}                    & $2.675 \pm{0.070}$ &  $0.661 \pm{0.826}$    \\ \midrule
\ours~(ViT)           &   $1.988 \pm{0.068}$                        &   $3.072 \pm{3.845}$   \\ 
\ours~(MAE ViT)     &   $2.102 \pm{0.035}$                          &   $3.103 \pm{3.814}$   \\ 
\ours~(CLIP ViT)     &   $2.091 \pm{0.041}$                         &   $3.135 \pm{3.901}$   \\
\ours~(ResNet-50)                  & $1.901 \pm{0.060}$             &   $3.111 \pm{3.852}$  \\ 
\ours~(Clip ResNet-50)               & $2.033 \pm{0.038}$           &   $3.121 \pm{3.863}$  \\ 
\ours~(advBN ResNet-50)           & $\mathbf{1.881} \pm{0.057}$     &  $\mathbf{3.153} \pm{3.904}$    \\ \hline
% \midrule
% \ours + Attr. Cls. (oracle)           & $1.858 \pm{0.054}$ &   $2.824	 \pm{3.497}$   \\
\end{tabular}
}
\caption{ 
\textbf{Choice of the feature space for attribute discovery}. Using an external feature space is superior to GAN's native style space, in terms of both entropy ($\times 100$) and deviation metrics. In this experiment, we set $\mathcal{G}_r = \mathcal{G}_c$, and aggregate the metrics from the set of controlled CelebA StyleGANs.
%The experiment uses the optimization described by LatentCLR, where we additionally use a variety of encoders rather than the GAN's own feature space. 
%While each encoder does relatively well, the CelebA attribute classifier, unsurprisingly, enables finding the most disentangled attributes, as its feature space is directly tuned for CelebA attributes.
}
%\vspace{-5mm}
\label{tbl:entropy_results}
\end{table}


\begin{figure}[!htbp]
    \centering
    \includegraphics[width=0.66\linewidth]{figures/1gan_example.pdf}
    \caption{An example of single GAN method's most changed attribute direction using random seed 0. Using \ours, more diverse and novel attributes are found. Complete examples for all methods are provided in the supplement.}
    \label{fig:1gan_examples}
\end{figure}

\myparagraph{Cross-GAN attribute auditing with ground truth}
In order to verify the effectiveness of the proposed method, it is crucial to provide evaluation against a straightforward baseline method on a well-understood task, where we know the ground truth.
%%%%%
An intuitive idea for multi-GAN attribute alignment and comparison is to use a single GAN attribute discovery method and then align the discovered attributes after applying them on separate GANs. 
Here, we run 4 recent single GAN attribute discovery methods (Voynov \cite{voynov2020unsupervised}, latentCLR \cite{yuksel2021latentclr}, Jacobian \cite{wei2021jacobian}, and Hessian \cite{peebles2020hessian}) on each of our celebA GANs (i.e., controlled setup with ground truth discussed in Section \ref{sec:setup}, in which we know what attributes are excluded in each model), then align the models after the fact. 
Once each of the attribute discovery models is trained, we align each direction for all pairwise combinations. We use our pretrained face attribute classifier to greedily identify the directions that maximize cosine similarity between the predicted attributes $a$ (e.g., the pair of directions between two GANs that achieve the highest cosine similarity are selected as the first aligned attribute). We then select the top 12 most similar directions for each pair of GANs, along with the top 4 most unique directions, and compute our previously described metrics. This setup is designed to be the most generous as possible for the single GAN methods. We show in table  \ref{tbl:celeba_metrics}, however,  the single GAN methods still under-perform our \ours{} method. 

\input{tables/CelebA_scores}

In Table \ref{tbl:celeba_metrics}, we also show the results of the average common and unique scores for all pairwise combinations (28 in total) of our CelebA GANs, with full experimental provided in the supplement. The robust variant of ResNet finds the most interesting attributes consistently. Surprisingly, variants of the ImageNet-pretrained ResNet models outperform the attribute classifier, particularly on finding common directions indicating that these models are better suited for this optimization likely because of having a more expressive feature space. Overall, we find that the robust ResNet variant performs the best across all our metrics. Additional experiments (on ViT-based feature extractors) are included in the supplement.

\begin{figure}[!tb]
    \centering
    \includegraphics[width=0.99\linewidth]{figures/leave_n_out_CelebA.pdf}
    \caption{The top missing attribute according to the pretrained classifier from full CelebA GAN against three different GANs trained on various attribute splits. }
    \label{fig:leave_out_celeba}
\end{figure}

Alignment between CelebA GANs is relatively easy as the two models already share much of the same data distribution. We show a modestly challenging version in Figure \ref{fig:leave_out_celeba} (a) with the alignment between two CelebA models. Lastly, we show the unique attributes when comparing a CelebA GAN versus missing attribute(s) CelebA GANs. 
Figure \ref{fig:leave_out_celeba} (b) shows examples from a few different setups, where missing attributes were accurately identified.
We do note that many of the common attributes show  attribute entanglement between multiple attributes. However, our goal for GAN auditing is not to discover disentangled attributes, but to characterize the GAN.



\myparagraph{Unique attribute ablation study} 
Here we investigate the effect of the KLIEP loss on unique attribute discovery. In addition to KLIEP loss presented above, we analyze a model trained with simple log loss. 
Full details for the log-loss model are provided in the supplement, but essentially we use the same setup as the KLIEP and DRE loss, but minimize/maximize log probabilities (rather than ratios) to accurately distinguish between $G_1$ and $G_2$.  
Table \ref{tbl:dre_lambda_abl} illustrates the unique attribute discovery score for each CelebA split versus full CelebA. With $\lambda=0$ (i.e. ignoring the DRE loss), the unique direction discovery process has difficulty capturing some missing attributes, like gender or hats (see supplement for detailed results). When using a regularization model trained with Log-loss, the results are consistently worse than DRE, sometimes even worse than with $\lambda=0$. The KLIEP loss model, on the other hand, performs consistently better for all lambda values $>0$.

\begin{table}[!tb]
\centering
\small{
\begin{tabular}{ccc}
$\lambda$ & $\mathcal{R}_{\text{Score}}$ (DRE loss) ($\uparrow$) & $\mathcal{R}_{\text{Score}}$ (Log loss )($\uparrow$)\\ \hline
0                                          & $0.42  \pm{0.38}$           & $0.42 \pm{0.38}$     \\
0.1                                       & $\bm{0.61} \pm{0.35}$       & $0.37 \pm{0.41}$     \\
0.2                                       & $0.54 \pm{0.33}$             & $0.44 \pm{0.39}$     \\
0.5                                       & $0.57 \pm{0.40}$            & $0.45 \pm{0.38}$     \\
1                                         & $\bm{0.61} \pm{0.33}$       & $0.40 \pm{0.40}$     \\
5                                         & $0.57 \pm{0.39}$            & $0.34 \pm{0.32}$    \\                     \hline
\end{tabular}
\caption{ The effect on the unique direction score when modifying the regularization $\lambda$ on the average $\mathcal{R}_{\text{Score}}$ ($\pm{\text{ std}}$) for the the 7 CelebA pairwise leave-attribute-out experiments using a Robust ResNet-50 encoder.} 
%\vspace{-2.5mm}
\label{tbl:dre_lambda_abl}
}
\end{table}



\myparagraph{Cross-GAN attribute auditing in the wild}
Figure \ref{fig:cats_vs_dogs} shows alignments between non-human faces of cats and dogs.


\begin{figure}[!tb]
    \centering
    \includegraphics[width=1.0\linewidth]{figures/cats_vs_dogs.pdf}
    \caption{
    \textbf{(Top): }  The common attributes from AFHQ Cats (client) and Dogs (reference) \textbf{(Bottom):} The novel/missing attributes from these non-human GANs. 
    }
    \label{fig:cats_vs_dogs}
\end{figure}

Figure \ref{fig:metface} shows interesting examples of aligned attributes between a GAN trained on Metface and another trained on CelebA. We demonstrate the general applicability of these identified attributes by showing multiple directions from a variety of starting points. We see an expected attribute of formal wear between both GANs, but we also highlight the fact that our CelebA GAN, despite being trained on human faces, appears to discover an unexpected attribute direction corresponding to sketch-like images as well as a direction for transforming an adult face into a child.

\begin{figure}[!htbp]
    \centering
    \includegraphics[width=1.0\linewidth]{figures/metface.pdf}
    \caption{ Examples of common attributes discovered between Metface (Left) and CelebA (Right).} 
    \label{fig:metface}
\end{figure}

\begin{figure}[!tb]
    \centering
    \includegraphics[width=1.0\linewidth]{figures/3gan.pdf }
    %\vspace{-5mm}
    \caption{Common attributes from training three attribute models over three different GANs (Eqn \ref{eq:xent_1}). 
    } 
%    \vspace{-2.5mm}
     \label{fig:3gan}
\end{figure}

In theory, our proposed method is not limited to working on only two generative models and can be scaled beyond two models, with the only constraint being the memory on the GPU since we need to load all the generators into memory to perform optimization. Here, we run an experiment with 3 different CelebA StyleGANs to verify similar directions can appropriately be found. Figure \ref{fig:3gan} shows the results of aligning 3 models at the same time.

Finally, a qualitative investigation into the task of applying \ours{} to different GAN architectures can be found in the supplement (StyleGAN3-t vs. StyleGAN3-r and PGGAN vs. GANformer).


\myparagraph{Limitations} The proposed method has the following limitations.
First, similar to other optimization-based attribute discovery approaches \cite{voynov2020unsupervised}, \cite{yuksel2021latentclr}, there is no guarantee that all prevalent factors are captured, but we do show that all the distinct attributes are identified in our experiments when ground truth is available. Second, compared to existing methods for a single GAN, additional choices with respect to the feature space need to be made for attribute discovery. 
As shown by our result with the robust pre-trained ResNet producing the best result, the choice of feature space for attribute recognition not only has a large impact on the overall performance of the model but also determines prior knowledge brought into the attribute discovery process.  

\section{DISCUSSION AND CONCLUSIONS}
While the results obtained are an encouraging first step towards open-set 3D semantic segmentation there are still many open questions to improve such approaches, some of which we discuss in the following.

Currently, the largest factor limiting segmentation performance is the quality of the vision-language features. While LSeg uses natural language features from CLIP trained on a very large dataset, the visual encoder is trained on the small closed-set ADE20K dataset. If we were able to compute dense pixel-aligned visual-language features from open-set web scraped data without requiring any human annotations, we believe that results could eventually surpass supervised learning methods. \cite{ranasinghe2022perceptual} presented some promising initial results on learning pixel aligned features without using segmentation masks or other expert annotations. 

In real-time experiments, our system relied on poses coming from a SLAM system. If many bad poses are computed by the SLAM system, the 3D representation could become corrupted by bad updates. Possible solutions include treating the sparse SLAM poses as initial guesses and optimizing the poses jointly with scene geometry, as in \cite{sucar2021imap, zhu2022nice}, or bad poses could be filtered out by analyzing the photometric or geometric error across frames. 

In robotics, downstream modules, such as motion planners and high-level planning systems, might benefit from a more explicit and principled representation of geometry than what we presented in this paper. For example, signed distance function based approaches \cite{wang2021neus} might provide better surface and occupancy reconstruction and have other favorable properties, such as the ability to compute the normal of a surface by differentiating through the distance function. For the time being, our method is limited to static scenes. Dealing with moving objects within scenes remains an open problem, but promising recent research \cite{kong2023vmap} suggests that extending neural implicit representations to dynamic scenes might be feasible.


% \section{CONCLUSIONS}
To conclude, we proposed a volumetric neural representation which is able to jointly learn geometry, radiance, and semantic feature information of a scene. We have shown that by using dense pixel-aligned vision-language features, our resulting learned representation can be used to volumetrically segment scenes into, at run-time, user defined categories. We have also shown how the representation can be used to produce dense 2D segmentation maps for queried viewpoints. Experiments on the ScanNet dataset showed competitive performance and our real-world experiments demonstrate that the method could be run onboard a robotic system.

%\addtolength{\textheight}{-1cm}   % This command serves to balance the column lengths
                                  % on the last page of the document manually. It shortens
                                  % the textheight of the last page by a suitable amount.
                                  % This command does not take effect until the next page
                                  % so it should come on the page before the last. Make
                                  % sure that you do not shorten the textheight too much.

%%%%%%%%%%%%%%%%%%%%%%%%%%%%%%%%%%%%%%%%%%%%%%%%%%%%%%%%%%%%%%%%%%%%%%%%%%%%%%%%

%%%%%%%%%%%%%%%%%%%%%%%%%%%%%%%%%%%%%%%%%%%%%%%%%%%%%%%%%%%%%%%%%%%%%%%%%%%%%%%%

%%%%%%%%%%%%%%%%%%%%%%%%%%%%%%%%%%%%%%%%%%%%%%%%%%%%%%%%%%%%%%%%%%%%%%%%%%%%%%%%

\bibliography{references}
\bibliographystyle{ieeetr}
% \printbibliography

\end{document}
