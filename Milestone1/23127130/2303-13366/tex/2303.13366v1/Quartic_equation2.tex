\documentclass[review]{elsarticle}

\usepackage{lineno,hyperref}
%\usepackage[mathscr]{euscript}
\usepackage{amsmath}
\usepackage{amssymb}
\usepackage{amsfonts}

\modulolinenumbers[5]

\journal{Indagationes Mathematicae}

%%%%%%%%%%%%%%%%%%%%%%%
%% Elsevier bibliography styles
%%%%%%%%%%%%%%%%%%%%%%%
%% To change the style, put a % in front of the second line of the current style and
%% remove the % from the second line of the style you would like to use.
%%%%%%%%%%%%%%%%%%%%%%%

%% Numbered
%\bibliographystyle{model1-num-names}

%% Numbered without titles
%\bibliographystyle{model1a-num-names}

%% Harvard
%\bibliographystyle{model2-names.bst}\biboptions{authoryear}

%% Vancouver numbered
%\usepackage{numcompress}\bibliographystyle{model3-num-names}

%% Vancouver name/year
%\usepackage{numcompress}\bibliographystyle{model4-names}\biboptions{authoryear}

%% APA style
%\bibliographystyle{model5-names}\biboptions{authoryear}

%% AMA style
%\usepackage{numcompress}\bibliographystyle{model6-num-names}

%% `Elsevier LaTeX' style
\bibliographystyle{elsarticle-num}
%%%%%%%%%%%%%%%%%%%%%%%

\begin{document}

\begin{frontmatter}

\title{Quartic diophantine equation $X^4 - Y^4 = R^2 - S^2 $
%\tnotetext[mytitlenote]{Fully documented templates are available in the elsarticle package on \href{http://www.ctan.org/tex-archive/macros/latex/contrib/elsarticle}{CTAN}.
}

%% Group authors per affiliation:
%\author{S. Muthuvel\fnref{myfootnote1}}
%\author{R. Venkatraman\fnref{footnote2}}
%\address{Department of Mathematics, \\
%College of Engineering and Technology, \\
%%SRM Institute of Science and Technology, Vadapalani Campus,\\
%No.1 Jawaharlal Nehru Salai, Vadapalani, Chennai-600026, Tamilnadu, India.}
%\fntext[myfootnote1]{Since 1880.}
%\fntext[]{venkatrr1@srmist.edu.in}

%% or include affiliations in footnotes:
\author{S.Muthuvel}
\ead{muthushan15@gmail.com, ms3081@srmist.edu.in}

\author{R.Venkatraman
\corref{mycorrespondingauthor}}
\cortext[mycorrespondingauthor]{Corresponding author}
\ead{venkatrr1@srmist.edu.in}

\address{Department of Mathematics, College of Engineering and Technology, SRM Institute of Science and Technology, Vadapalani Campus,
No.1 Jawaharlal Nehru Salai, Vadapalani, Chennai-600026, Tamilnadu, India.}

%\address[mysecondaryaddress]{Assistant Professor, Department of Mathematics, College of Engineering and Technology, SRM Institute of Science and Technology, Vadapalani Campus, No.1 Jawaharlal Nehru Salai, Vadapalani, Chennai-600026, Tamilnadu, India.}

%\address[mysecondaryaddress]{360 Park Avenue South, New York}

\begin{abstract}
In this paper, we deal with the quartic diophantine equation $X^4 - Y^4 = R^2 - S^2$ to present its infinitely many integer solutions.
\end{abstract}

\begin{keyword}
Quartic diophantine equation \sep Elementary method
\MSC[2020] 11D25 \sep 11D45
\end{keyword}

\end{frontmatter}

\section{Introduction}\label{sec1}

In the general form of Diophantine equation
$$x^n + y^n = u^n +v^n, \qquad \qquad n \in \mathbb{N}$$
The case $n=2$ has been expressed in \cite{Davies,Kersey book,Pasternak}. For $n=4$, the parametric solutions to the aforementioned equation are described in \cite{Choudhry-1,Euler ppr, Hardy book}. More general diophantine equations with more variable or with integer coefficients that are not all equal to one were taken into consideration by several researchers \cite{Choudhry-h,Elkies,Izadi & Nabardi,Janfada & Nabardi}. The authors provided an infinite number of positive integral solutions for various powers as illustrated in \cite{Babic & Nabardi}.
The objective of this work is to obtain infinitely many integral solutions of
\begin{eqnarray}
X^4 - Y^4=R^2 - S^2 \label{1}
\end{eqnarray}
for each one of the parametric method.
In \cite{Baghlaghdam & Izadi form}, the equations
\begin{eqnarray}
a\left(X_{1}^{'5}+X_{2}^{'5}\right)+\sum_{i=0}^{m}a_{i}X_{i}^{5}=b\left(Y_{1}^{'3}+Y_{2}^{'3}\right)+\sum_{i=0}^{n} b_{i}Y_{i}^{3} \label{s1}
\end{eqnarray}
where $m,n \in \mathbb{N} \cup \{0\}$ and $a,b \neq 0, \ a_{i},b_{i}$ are fixed arbitrary rational numbers are
examined.  The solution to \eqref{s1}, which is converted into a cubic or a quartic elliptic curve with a positive rank, is found using the theory of elliptic curves.
 Authors demonstrate that in [\cite{Baghlaghdam & Izadi high}, Main Theorem 2]

$$\sum_{i=1}^{n}p_{i} x_{i}^{a_i}= \sum_{j=1}^{m}q_{j} y_{j}^{b_j}$$

$m,n,a_{i},b_{j} \in \mathbb{N}, \ p_{i},q_{j} \in \mathbb{Z}, \ i=1,2,\dots,n, \ j=1,2,\dots,m$ 
 has a parametric solution and
infinitely many solutions in nonzero integers if there exists an $i$ such that $p_i = 1$
and $(a_{i},a_{1}a_{2} \dots a_{i-1}a_{i+1}\dots a_{n}b_{1}b_{2}\dots b_{m}) = 1$ or there exists a $j$ such that $q_j = 1$ and
$(b_{j},a_{1}\dots a_{n}b_{1}\dots b_{j-1}b_{j+1}\dots b_{m}) = 1$. Although linear transformations are also employed in this article, we propose a different strategy and some different conditions for the integer coefficients in order to solve \eqref{1}.
\section{Solving the Diophantine equation $X^4 - Y^4 = R^2 - S^2$}
The trivial solution of the equation \eqref{1} is $(X,Y,R,S)=(m,n,m^2 , n^2)$ for $m, n \in \mathbb{Z}$.
Four different linear transformations are considered and for each one of them, we give a different class of infinitely many integer solutions of equation \eqref{1}.
\subsection{Method-1}
Consider the linear transformations,
\begin{eqnarray}
% \nonumber % Remove numbering (before each equation)
  X=px+u, \qquad Y=qx-u, \qquad R=x+v, \qquad S=px-v \label{M1LT}
\end{eqnarray}
\noindent $p,x,u,v \in \mathbb{Z}$. Introducing \eqref{M1LT} in \eqref{1}, we get
\begin{eqnarray}
   \alpha x^4 + \beta x^3 + \gamma x^2 + \delta x &=& 0 \label{M1_1}
\end{eqnarray}
\noindent where
\begin{align}
  \begin{aligned}
   \alpha &= p^4 - q^4, \\       \gamma &= 6p^2 u^2 - 6p^2 u^2 + p^2 - 1,
  \end{aligned}
  &&
  \begin{aligned}
   \beta &= 4p^3 u + 4q^3 u, \\       \delta &= 4pu^3 + 4qu^3 - 2v + 2pv \label{M1_2}
  \end{aligned}
 \end{align}
\noindent For $\delta=0$ in \eqref{M1_2}, we obtain
$$(2p+2q)u^3 = v(1-p)$$
Further, we put $u=t, v=t^3$ and get $p=\frac{1-2q}{3}$. \\ In \eqref{M1_2}, equating the like terms $\gamma = 0$
   $$\left(q + 1\right)\left[\left(-15t^2 +2\right)q + \left(3t^2 -4\right)\right]= 0$$
\noindent Simplifying the above expression, we have $q=\frac{4 - 3t^2}{2 - 15t^2}$ and therefore \eqref{M1_1} becomes
\begin{eqnarray*}
  \alpha x^4 + \beta x^3 &=& 0 \\
   x&=& \frac{-t\left(405t^8 + 459t^6 - 1404t^4 + 600t^2 - 56\right)}{3\left(27t^6 - 27t^4 + 36t^2 - 10\right)}
  \end{eqnarray*}
\noindent Plugging the values $p,q,u,v$ in \eqref{M1LT}, we acquire
\begin{eqnarray}
\begin{aligned}
% \nonumber % Remove numbering (before each equation)
  X &= \frac{t\left(1215t^{10} - 1782t^8 + 4671t^6 - 774t^4 - 366t^2 + 52\right)}{3(2-15t^2)\left(27t^6 - 27t^4 + 36t^2 - 10\right)} \\
  Y &= \frac{t\left(1215t^{10} - 1782t^8 + 4671t^6 - 5634t^4 + 1902t^2 - 164\right)}{3(2-15t^2)\left(27t^6 - 27t^4 + 36t^2 - 10\right)} \\
  R &= \frac{-t\left(324t^8 - 378t^6 + 1296t^4 - 570t^2 + 56\right)}{3\left(27t^6 - 27t^4 + 36t^2 - 10\right)} \\
  S &= \frac{t\left(810t^8 + 1512t^6 + 1674t^4 - 1092t^2 + 112\right)}{3(2-15t^2)\left(27t^6 - 27t^4 + 36t^2 - 10\right)}
\end{aligned}
\end{eqnarray}
\noindent Eliminating the denominators from the above equations,
\begin{eqnarray*}
\begin{aligned}
% \nonumber % Remove numbering (before each equation)
  X &= 1215t^{11} - 1782t^9 + 4671t^7 - 774t^5 - 366t^3 + 52t \\
  Y &= 1215t^{11} - 1782t^9 + 4671t^7 - 5634t^5 + 1902t^3 - 164t \\
  %R &= 3t(2-15t^2)^2 \left(27t^6 - 27t^4 + 36t^2 - 10\right)\left(324t^8 - 378t^6 + 1296t^4 - 570t^2 + 56\right) \\
R&=5904900t^{19}-14368590t^{17}+41898546t^{15}-55842858t^{13}+58236894t^{11} \\ 
& \qquad -36547200t^9+12314916t^7-2186784t^5+193392t^3-6720t \\
  %S &= 3t(15t^2-2)\left(27t^6 - 27t^4 + 36t^2 - 10\right)\left(810t^8 + 1512t^6 + 1674t^4 - 1092t^2 + 112\right)
S&=984150t^{17}+721710t^{15}+1395306t^{13}-1476954t^{11}+3664440t^9-3124332t^7 \\
& \qquad +1027296t^5-140112t^3+6720t
\end{aligned}
\end{eqnarray*}
We get a integer solution $(X,Y,R,S)$ of equation \eqref{1} for every $t \in \mathbb{Z}$. So,
the presented method generates infinitely many integer solutions of the initial
equation \eqref{1}.
\subsection{Method-2}
In this method, we deal with different transformation in \eqref{1}. Let
\begin{eqnarray}
% \nonumber % Remove numbering (before each equation)
  X=px+u, \qquad  Y=qx-u, \qquad  R=x+v, \qquad  S=px-v \label{M2LT}
\end{eqnarray}
\noindent $p,x,u,v \in \mathbb{Z}$. In previous subsection, by introducing these linear transformations in \eqref{1}, leads us to the equation of the form
\begin{eqnarray}
  Ax^4 + Bx^3 + Cx^2 + Dx &=& 0 \label{M2_1}
\end{eqnarray}
\noindent where
\begin{align}
  \begin{aligned}
   A &= p^4-q^4, \\       C &= 6p^2 u^2 - 6p^2 u^2 + p^2 - 1,
  \end{aligned}
  &&
  \begin{aligned}
   B &= 4p^3 u + 4q^3 u, \\       D &= 4pu^3 + 4qu^3 - 2v - 2pv \label{M2_2}
  \end{aligned}
 \end{align}
\noindent For $D=0$ in \eqref{M2_2}, we get
$$(2p+2q)u^3 = v(1+p)$$
Further, we set $u=t, v=t^3$ and attain $p=1-2q$. In \eqref{M2_2}, equating the like terms $C = 0$
$$\left(q - 1\right)\left[\left(9t^2 +2\right)q - 3t^2\right]= 0$$
\noindent Using the above equation, we obtain $q=\frac{3t^2}{9t^2 + 2}$ and therefore \eqref{M2_1} becomes
\begin{eqnarray*}
% \nonumber % Remove numbering (before each equation)
  A x^4 + B x^3 &=& 0 \\
   x&=& \frac{-t\left(243t^8 + 297t^6 + 216t^4 + 72t^2 +8\right)}{\left(27t^6 + 27t^4 + 12t^2 + 2\right)}
  \end{eqnarray*}
\noindent Applying the values $p,q,u,v$ in \eqref{M2LT}, we acquire
\begin{eqnarray}
\begin{aligned}
% \nonumber % Remove numbering (before each equation)
  X &= \frac{-3t\left(27t^8 + 36t^6 + 27t^4 + 12t^2 + 2\right)}{\left(27t^6 + 27t^4 + 12t^2 + 2\right)} \\
  Y &= \frac{-t\left(81t^8 + 108t^6 + 81t^4 + 24t^2 + 2\right)}{\left(27t^6 + 27t^4 + 12t^2 + 2\right)} \\
  R &= \frac{-2t\left(108t^8 + 135t^6 + 102t^4 + 35t^2 + 4\right)}{\left(27t^6 + 27t^4 + 12t^2 + 2\right)} \\ \label{M2_F}
  S &= \frac{-2t\left(54t^8 + 81t^6 + 60t^4 + 25t^2 + 4\right)}{\left(27t^6 + 27t^4 + 12t^2 + 2\right)}
\end{aligned}
\end{eqnarray}
\noindent By cancelling the denominators in \eqref{M2_F},
\begin{eqnarray*}
\begin{aligned}
% \nonumber % Remove numbering (before each equation)
  %X &= 3t\left(27t^6 + 27t^4 + 12t^2 + 2\right)\left(27t^8 + 36t^6 + 27t^4 + 12t^2 + 2\right) \\
X&=2187t^{15}+5103t^{13}+6075t^{11}+4617t^9+2322t^7+756t^5+144t^3+12t \\
  %Y &=  t\left(27t^6 + 27t^4 + 12t^2 + 2\right)\left(81t^8 + 108t^6 + 81t^4 + 24t^2 + 2\right)\\
Y&=2187t^{15}+5103t^{13}+6075t^{11}+4293t^9+1890t^7+504t^5+72t^3+4t \\
  %R &= 2t \left(27t^6 + 27t^4 + 12t^2 + 2\right)^3 \left(108t^8 + 135t^6 + 102t^4 + 35t^2 + 4\right)\\
R&=4251528t^{27}+18068994t^{25}+38381850t^{23}+52986636t^{21}+52435512t^{19}\\
& \qquad+38959218t^{17}+22221378t^{15}+9803592t^{13}+3331692t^{11}+858168t^9 \\
& \qquad +162216t^7+21216t^5+1712t^3+64t \\
  %S &= 2t \left(27t^6 + 27t^4 + 12t^2 + 2\right)^3 \left(54t^8 + 81t^6 + 60t^4 + 25t^2 + 4\right)
S&=2125764t^{27}+9565938t^{25}+21139542t^{23}+30154356t^{21}+30784212t^{19}\\
& \qquad +23632722t^{17}+13977846t^{15}+6426864t^{13}+2290356t^{11}+623160t^9 \\
& \qquad +125496t^7+17664t^5+1552t^3+64t
\end{aligned}
\end{eqnarray*}
For any $t \in \mathbb{Z}$, we obtain an integer solution $(X,Y,R,S)$ to equation \eqref{1}. Consequently, the proposed method yields an infinite number of integer solutions to the starting equation \eqref{1}.
\subsection{Method-3}
In this method, we deal with different transformation in \eqref{1}. Let
\begin{eqnarray}
% \nonumber % Remove numbering (before each equation)
  X=v, \qquad  Y=px+v, \qquad  R=qx+u, \qquad  S=x+u \label{M3LT}
\end{eqnarray}
\noindent $p,x,u,v \in \mathbb{Z}$. In subsection-1, by introducing these linear transformations in \eqref{1}, leads us to the equation of the form
\begin{eqnarray}
  ax^4 + bx^3 + cx^2 + dx &=& 0 \label{M3_1}
\end{eqnarray}
\noindent where
\begin{align}
  \begin{aligned}
   a &= p^4, \\       c &= 6p^2 v^2 + q^2 - 1,
  \end{aligned}
  &&
  \begin{aligned}
   b &= 4p^3 v, \\       d &= 4pv^3 - 2u - 2qu  \label{M3_2}
  \end{aligned}
 \end{align}
\noindent For $d=0$ in \eqref{M3_2}, we obtain
$$(2p)v^3 = u(1 - q)$$
Additionally, we put $u=t^3, v=t$ and get $p=\frac{1 - q}{2}$. In \eqref{M3_2}, equating the like terms $c = 0$
$$\left(q - 1\right)\left[\left(3t^2 +2\right)q - \left(3t^2 - 2\right)\right]= 0$$
\noindent Thus, we get $q=\frac{3t^2-2}{3t^2 + 2}$ and therefore \eqref{M3_1} becomes
\begin{eqnarray*}
% \nonumber % Remove numbering (before each equation)
  a x^4 + b x^3 &=& 0 \\
    x&=& \frac{-2t\left(81t^8 + 216t^6 + 216t^4 + 96t^2 + 16\right)}{\left(27t^6 + 54t^4 + 36t^2 + 8\right)}
  \end{eqnarray*}
\noindent Taking the values $p,q,u,v$ and applying it in \eqref{M3LT}, we get

\begin{eqnarray}
\begin{aligned}
% \nonumber % Remove numbering (before each equation)
  X &= t \\
  Y &= \frac{-3t\left(81t^8 + 216t^6 + 216t^4 + 96t^2 + 16\right)}{(3t^2 + 2)\left(27t^6 + 54t^4 + 36t^2 + 8\right)} \\
  R &= \frac{-t\left(405t^{10} + 756t^8 + 216t^6 - 384t^4 - 304t^2 - 64\right)}{(3t^2 + 2)\left(27t^6 + 54t^4 + 36t^2 + 8\right)} \\
  S &= \frac{-t\left(135t^8 + 358t^6 + 396t^4 + 184t^2 + 32\right)}{\left(27t^6 + 54t^4 + 36t^2 + 8\right)}
\end{aligned}
\end{eqnarray}

\noindent Neglecting the denominators from the above equation,

\begin{eqnarray*}
\begin{aligned}
% \nonumber % Remove numbering (before each equation)
  %X &= t^2(3t^2 + 2)\left(27t^6 + 54t^4 + 36t^2 + 8\right) \\
X&= 81t^{10}+216t^8+216t^6+96t^4+16t^2 \\
  %Y &= 3t^2 \left(81t^8 + 216t^6 + 216t^4 + 96t^2 + 16\right)\\
Y&=243t^{10}+648t^8+648t^6+288t^4+48t^2 \\
  %R &= t^3(3t^2 + 2)\left(27t^6 + 54t^4 + 36t^2 + 8\right) \left(405t^{10} + 756t^8 + 216t^6 - 384t^4 - 304t^2 - 64\right) \\
R&=32805t^{21}+148716t^{19}+268272t^{17}+217728t^{15}+18144t^{13}-120960t^{11} \\
& \qquad -112896t^9-49152t^7-11008t^5-1024t^3 \\
  %S &= t^3(3t^2 + 2)\left(27t^6 + 54t^4 + 36t^2 + 8\right) \left(135t^8 + 358t^6 + 396t^4 + 184t^2 + 32\right)
 S&=32805t^{21}+196344t^{19}+532008t^{17}+849312t^{15}+874656t^{13}+600000t^{11} \\ 
&\qquad +273536t^9+79872t^7+13568t^5+1024t^3
\end{aligned}
\end{eqnarray*}

For each $t \in \mathbb{Z}$, equation \eqref{1} has an integer solution $(X,Y,R,S)$. The resulting method produces an infinite number of integer solutions to the initial equation \eqref{1}.

 \subsection{Method-4}

In this method, we deal with different transformation in \eqref{1}. Let
\begin{eqnarray}
% \nonumber % Remove numbering (before each equation)
  X=-v, \qquad  Y=px-v, \qquad  R=qx-u, \qquad  S=x+u \label{M4LT}
\end{eqnarray}

\noindent $p,x,u,v \in \mathbb{Z}$. In subsection-1, by introducing these linear transformations in \eqref{1}, leads us to the equation of the form

\begin{eqnarray}
  Mx^4 + Nx^3 + Px^2 + Qx &=& 0 \label{M4_1}
\end{eqnarray}

\noindent where

\begin{align}
  \begin{aligned}
   M &= p^4, \\       P &= 6p^2 v^2 + q^2 - 1,
  \end{aligned}
  &&
  \begin{aligned}
   N &= -4p^3 v, \\       Q &= -4pv^3 - 2u - 2qu  \label{M4_2}
  \end{aligned}
 \end{align}

\noindent For $Q=0$ in \eqref{M4_2}, we obtain

$$(2p)v^3 = u(-1 - q)$$
Further, we put $u=t^3, v=t$ and get $q=-1-2p$. In \eqref{M4_2} equating the like terms $Q = 0$

$$p = \frac{-2}{3t^2 + 2}$$

\noindent Therefore \eqref{M4_1} becomes

\begin{eqnarray*}
% \nonumber % Remove numbering (before each equation)
  M x^4 + N x^3 &=& 0 \\
     x&=& -2t\left(3t^2 + 2\right)
  \end{eqnarray*}

\noindent Substituting the values $p,q,u,v$ in \eqref{M4LT}, we obtain

\begin{eqnarray*}
\begin{aligned}
% \nonumber % Remove numbering (before each equation)
  X &= -t \\
  Y &= 3t \\
  R &= 5t^3 -4t \\
  S &= -5t^3 -4t
\end{aligned}
\end{eqnarray*}

We get a integer solution $(X,Y,R,S)$ of equation \eqref{1} for every $t \in \mathbb{Z}$. So,
the presented method generates infinitely many integer solutions of the initial
equation \eqref{1}.

\begin{thebibliography}{9}

\bibitem{Baghlaghdam & Izadi high} M. Baghlaghdam, F. Izadi, A note on the high power diophantine equations, Proc. Math. Sci., 129(14), (2019).

\bibitem{Baghlaghdam & Izadi form} M. Baghlaghdam, F. Izadi, On the diophantine equation in the form that a sum of cubes equals a sum of quantics, Math. J. Okaama Univ., 61, (2019) 75-84.

\bibitem{Babic & Nabardi} S. Buja\v{c}i\'{c} Babi\'{c}, K. Nabardi, On some Diophantine equations, Miskolc Mathematical Notes, 22(1), (2021), 65-75.

\bibitem{Choudhry-1} A. Choudhry, The diophantine equation $A^4+B^4=C^4+D^4$, Indian J. Pure Appl. Math., 22(1), (1991) 9-11. 

\bibitem{Choudhry-h} A. Choudhry, On the Diophantine equation $A^4 +hB^4=C^4+hD^4$, Indian J. Pure Appl. Math., 26(11), (1995) 1057-1061.

\bibitem{Davies} H.B. Davies, On generating solutions of the Diophantine equation $x^2+y^2=u^2+v^2$, Int. J. Math. Educ. Sci. Technol., 15(1), (1984) 43-46. 

\bibitem{Elkies} N. Elkies, On $A^4+B^4+C^4=D^4$, Math. Comput., 51(184), (1988) 825-835. 

\bibitem{Euler ppr} L. Euler, Novi Comm. Acad. Petrop. v(17), (1772).

\bibitem{Hardy book} G. H. Hardy,E. M. Wright, An Introduction to the Theory of Numbers, Oxford University Press, London (1960).

\bibitem{Izadi & Nabardi} F. Izadi, K. Nabardi, Diophantine equation $X^4+Y^4=2(U^4+V^4)$, Math. Slovaca., 66(3), (2016) 557-560.

\bibitem{Janfada & Nabardi} A. S. Janfada, N. Nabardi, On Diophantine equation $x^4+y^4=n(u^4+v^4)$, Math. Slovaca.,  69(6), (2019) 1245-1248.

\bibitem{Kersey book} J. Kersey, The Elements of Algebra, London (1674).

\bibitem{Pasternak} P. Pasternak, Zeitschr, Math. Naturw. Unterrieht., 37, (1906) 33-35.

\end{thebibliography}
\end{document}


%\section*{References}

%\bibliography{mybibfile}

%\end{document}