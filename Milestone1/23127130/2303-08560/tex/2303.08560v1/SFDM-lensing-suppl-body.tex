\section{Mass-dependence of deviations from MOND}
\label{sec:suppl:estar}

In Fig.~\refstar{fig:SFDM-illustrate-kinematic} of the main text, we saw that, for kinematic observables, smaller galaxies follow the MOND prediction more easily than larger galaxies.
To see why, first note that deviations from MOND become relevant when $\varepsilon_*$ violates the condition $\varepsilon_* \ll 1$ (see Eq.~(\refstar{eq:mondlimit}) of the main text).
Consider then the rough analytical lower bound on $\varepsilon_*$ derived in earlier work \cite{Mistele2022},
\begin{align}
 \label{eq:estarlowerbound}
 \varepsilon_* \gtrsim \frac{2 m^2 r}{\alpha M_{\mathrm{Pl}}} + \varepsilon_{*\mathrm{min}} \,,
\end{align}
where $\varepsilon_{*\mathrm{min}}$ is the value of $\varepsilon_*$ that gives $\rho_{\mathrm{SF}} = 0$ (see Eq.~(\refstar{eq:rhoSF}) of the main text, $\varepsilon_{*\mathrm{min}} \approx -0.3$ for the fiducial parameter value $\beta = 2$).
Eq.~\eqref{eq:estarlowerbound} suggests that $\varepsilon_*$ becomes larger than $1$, at the latest, at a universal radius that is independent of baryonic mass $M_b$.
That is, Eq.~\eqref{eq:estarlowerbound} suggests that deviations from MOND set in, at the latest, at this universal radius.
This, in turn, suggests that the maximum radius $r_{\mathrm{max}}$ is roughly independent of $M_b$.
This maximum radius corresponds to a minimum acceleration $a_{b\mathrm{min}} = G M_b/r_{\mathrm{max}}^2$ which is then proportional to $M_b$.
This qualitatively fits with the mass-dependence of deviations from MOND in Fig.~\refstar{fig:SFDM-illustrate-kinematic} of the main text.

\section{Dependence on parameter values}
\label{sec:appendix:parameters}

The phenomenology of SFDM around galaxies depends on the values of the model parameters $a_0$, $\beta$, and $m^2/\alpha$ \citep{Mistele2022}.
In the main text, we assumed the numerical parameter values proposed by the original authors \cite{Berezhiani2018}.
Here, we discuss the effects of changing these values.

Consider first $a_0$.
In the MOND limit $\varepsilon_* \ll 1$, the kinematically-inferred acceleration $a_{\mathrm{obs}}$ has the form $a_b + \sqrt{a_0 a_b}$.
This roughly reproduces MOND as long as $a_0$ has roughly the value it usually has in MOND, i.e. $a_0 \approx 10^{-10}\,\mathrm{m}/\mathrm{s}^2$.
Thus, we cannot change this parameter much without giving up one of the main ideas behind SFDM, namely to reproduce MOND in this regime.
Thus, for any reasonable value of $a_0$, our conclusions remain the same.

The most important parameter for our purposes is $m^2/\alpha$.
It multiplies both the superfluid energy density $\rho_{\mathrm{SF}}$ as well as the quantity $\varepsilon_*$ that controls whether or not the kinematically-inferred $a_{\mathrm{obs}}$ is in the MOND limit.
With the original fiducial parameter values \cite{Berezhiani2018} and at large $a_b$,
the SFDM prediction for the
lensing-inferred $a_{\mathrm{obs}}$ is much smaller than observations
(see Sec.~\refstar{sec:weaklensing} of the main text).
To counter this, we could choose a larger value of $m^2/\alpha$ which gives a larger superfluid mass and thus a larger lensing-inferred $a_{\mathrm{obs}}$.
However, this also makes $\varepsilon_*$ larger.
As a result, the kinematically-observed $a_{\mathrm{obs}}$ starts to deviate from the MOND prediction already at larger $a_b$.

\begin{figure}
\centering
\includegraphics[width=\hsize]{plots/SFDM/illustrate-RAR-BC-Mb-dependence-no-NFW-tail-kinematic-beta-2.00-fm2-10.00.pdf}
\caption{
    Same as Fig.~\refstar{fig:SFDM-illustrate-kinematic} of the main text but with $m^2/\alpha$ increased by a factor $10$.
}
\label{fig:SFDM-illustrate-kinematic-fm2-10}
\end{figure}

\begin{figure}
\centering
\includegraphics[width=\hsize]{plots/SFDM/illustrate-RAR-rNFW-Mb-dependence-beta-2.00-fm2-10.00.pdf}
\caption{
    Same as Fig.~\refstar{fig:SFDM-illustrate-kinematic-vs-lensing-NFWtail} of the main text but with $m^2/\alpha$ increased by a factor $10$.
}
\label{fig:SFDM-illustrate-kinematic-vs-lensing-NFWtail-fm2-10}
\end{figure}

That is, by adjusting the parameter $m^2/\alpha$ we can change where the problems discussed in Sec.~\refstar{sec:weaklensing} of the main text occur.
But we cannot avoid these fundamental problems of SFDM.
This is illustrated in Fig.~\ref{fig:SFDM-illustrate-kinematic-fm2-10} and Fig.~\ref{fig:SFDM-illustrate-kinematic-vs-lensing-NFWtail-fm2-10}.
These are the same as Fig.~\refstar{fig:SFDM-illustrate-kinematic} and Fig.~\refstar{fig:SFDM-illustrate-kinematic-vs-lensing-NFWtail} of the main text but with $m^2/\alpha$ increased by a factor $10$.
We see that there is still a tension between the kinematic and the lensing RAR.
And reproducing a MOND-like lensing RAR is, at best, possible when tuning the boundary condition and NFW matching radius.

\begin{figure}
\centering
\includegraphics[width=\hsize]{plots/SFDM/illustrate-RAR-BC-Mb-dependence-no-NFW-tail-kinematic-beta-1.55-fm2-1.00.pdf}
\caption{
    Same as Fig.~\refstar{fig:SFDM-illustrate-kinematic} of the main text but with $\beta = 1.55$ instead of $\beta = 2$.
}
\label{fig:SFDM-illustrate-kinematic-beta-1.55}
\end{figure}

\begin{figure}
\centering
\includegraphics[width=\hsize]{plots/SFDM/illustrate-RAR-rNFW-Mb-dependence-beta-1.55-fm2-1.00.pdf}
\caption{
    Same as Fig.~\refstar{fig:SFDM-illustrate-kinematic-vs-lensing-NFWtail} of the main text but with $\beta = 1.55$ instead of $\beta = 2$.
}
\label{fig:SFDM-illustrate-kinematic-vs-lensing-NFWtail-beta-1.55}
\end{figure}

This leaves $\beta$.
This parameter controls the precise shape of $\rho_{\mathrm{SF}}$ and the phonon force $a_\theta$ outside the MOND limit, i.e. for $\varepsilon_* \gtrsim 1$ (see for example Appendix A of \cite{Mistele2022}).
To avoid instabilities and to ensure a positive superfluid density, $\beta$ must be between $3/2$ and $3$ \citep{Berezhiani2015}.
In Fig.~\ref{fig:SFDM-illustrate-kinematic-beta-1.55} and Fig.~\ref{fig:SFDM-illustrate-kinematic-vs-lensing-NFWtail-beta-1.55}, we show the effect of using $\beta = 1.55$ instead of the original fiducial value $\beta = 2.0$ \cite{Berezhiani2015}.
We see that there are minor quantitative changes but our qualitative conclusions remain unchanged.

Even if changing $\beta$ did lead to significant changes, it would be best to not rely on them too much.
This is because both the value of $\beta$ and the form of the finite-temperature corrections that $\beta$ is supposed to represent are completely ad-hoc \citep{Mistele2020}.
So any significant changes related to $\beta$ might easily turn out to be unphysical.
