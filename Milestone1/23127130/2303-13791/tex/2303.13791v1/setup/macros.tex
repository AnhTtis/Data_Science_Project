% Macros - Jia-Bin Huang (jbhuang0604@gmail.com)

\renewcommand{\baselinestretch}{0.991}       % for squeezing the draft into the page limit, do not use

\newcommand{\centeredtab}[1]{\setlength{\tabcolsep}{0pt}\begin{tabular}{c} #1 \end{tabular}}

% For the unsure command.
\usepackage{soul}
\usepackage{svg}

\usepackage[accsupp]{axessibility} % Improves PDF readability for those with disabilities.

% \usepackage[final]{changes}

% =========================================
% Useful macros
% =========================================

% Latin abbreviations
\def\etal{et~al.\_}			  % and others, and co-workers
\def\eg{e.g.,~}               % for example
\def\ie{i.e.,~}               % that is, in other words
\def\etc{etc}                 % and other things, and so forth
\def\cf{cf.~}                 % compare
\def\viz{viz.~}               % namely, precisely
\def\vs{vs.~}                 % against

\def\naive{na{\"i}ve\xspace}
\def\Naive{Na{\"i}ve\xspace}
\def\naively{na{\"i}vely\xspace}
\def\Naively{Na{\"i}vely\xspace}

% Math related
\DeclareMathOperator*{\argmin}{\arg\!\min} 
\DeclareMathOperator*{\argmax}{\arg\!\max}
\DeclareMathOperator*{\minimize}{minimize}

\DeclareMathAlphabet{\altmathcal}{OMS}{cmsy}{m}{n}
\DeclareMathAlphabet{\mathbfit}{OT1}{ptm}{bx}{it}

% Consistent margin adjustment for paragraphs, figures, and sections
\newlength\paramargin
\newlength\figmargin

\newlength\secmargin
\newlength\figcapmargin
\newlength\tabcapmargin

\setlength{\secmargin}{0.0mm}
\setlength{\paramargin}{0.0mm}
\setlength{\figmargin}{0.0mm}
\setlength{\figcapmargin}{-0mm}
\setlength{\tabcapmargin}{0.0mm}

\setlength{\fboxsep}{0pt}

\newcommand{\red}{\textcolor{red}}
\newcommand{\blue}{\textcolor{blue}}
\newcommand{\changed}{\textcolor{black}} 

\newcommand {\first}[1]{{\color{red}\textbf{#1}}}
\newcommand {\second}[1]{{\color{blue}\underline{#1}}}


\newenvironment{tightitemize}{
	\vspace{-1.5mm}
	\begin{itemize}
		\setlength{\itemsep}{1pt}
		\setlength{\parskip}{2pt}
		\setlength{\parsep}{0pt}}{\end{itemize}
	
}

% minipage
\newcommand{\mpage}[2]
{
\begin{minipage}{#1\linewidth}\centering
#2
\end{minipage}
}

\newcommand{\mfigure}[2]
{
\includegraphics[width=#1\linewidth]{#2}
}

\newcommand{\topic}[1]
{
\vspace{1mm}\noindent\textbf{#1}
}

% References for figures, tables, equations, and sections
\newcommand{\secref}[1]{Section~\ref{sec:#1}}
\newcommand{\figref}[1]{Figure~\ref{fig:#1}} 
\newcommand{\tabref}[1]{Table~\ref{tab:#1}}
\newcommand{\eqnref}[1]{\eqref{eq:#1}}
\newcommand{\thmref}[1]{Theorem~\ref{#1}}
\newcommand{\prgref}[1]{Program~\ref{#1}}
\newcommand{\algref}[1]{Algorithm~\ref{#1}}
\newcommand{\clmref}[1]{Claim~\ref{#1}}
\newcommand{\lemref}[1]{Lemma~\ref{#1}}
\newcommand{\ptyref}[1]{Property~\ref{#1}}

% Comments
\long\def\ignorethis#1{}
\newcommand {\chen}[1]{{\color{red}\textbf{Chen: }#1}\normalfont}
\newcommand {\ANDREAS}[1]{{\color{purple}\textbf{Andreas: }#1}\normalfont}
\newcommand {\jiabin}[1]{{\color{green}\textbf{Jia-Bin: }#1}\normalfont}
\newcommand {\MK}[1]{{\color{cyan}\textbf{MK: }#1}\normalfont}
\newcommand {\johannes}[1]{{\color{magenta}\textbf{Johannes: }#1}\normalfont}

%\newcommand {\todo}{{\textbf{\color{red}[TO-DO]\_}}}
\def\newtext#1{\textcolor{blue}{#1}}
\def\modtext#1{\textcolor{red}{#1}}
%\newcommand{\note}[1]{{\it\color{blue} #1}}

%\newcommand{\unsure}[1]{{\sethlcolor{yellow}\hl{#1}}}

\newcommand{\tb}[1]{\textbf{#1}}
\newcommand{\mb}[1]{\mathbf{#1}}

\newcommand{\jbox}[2]{
  \fbox{%
  	\begin{minipage}{#1}%
  		\hfill\vspace{#2}%
  	\end{minipage}%
  }}

\newcommand{\jblock}[2]{%
	\begin{minipage}[t]{#1}\vspace{0cm}\centering%
	#2%
	\end{minipage}%
}

% subfigures with automatic width
\newbox\jsavebox%
\newcommand{\jsubfig}[2]{%
	\sbox\jsavebox{#1}%
	\parbox[t]{\wd\jsavebox}{\centering\usebox\jsavebox\\#2}%
	}

%%%%%%%%%%%%%%%%	
%% \providelength command that will define a new length if not already defined, but
%% also checks whether the command passed as argument has been defined with
%% \newlength, in order to issue an error message if you try to use, say,
%% \providelength{\textit}
\makeatletter
\newcommand{\providelength}[1]{%
  \@ifundefined{\expandafter\@gobble\string#1}
   {% if #1 is undefined, do \newlength
    \typeout{\string\providelength: making new length \string#1}%
    \newlength{#1}%
   }
   {% else check whether #1 is actually a length
    \sdaau@checkforlength{#1}%
   }%
}

\newcommand{\sdaau@checkforlength}[1]{%
  % get the first five characters from \meaning#1
  \edef\sdaau@temp{\expandafter\sdaau@getfive\meaning#1TTTTT$}%
  % compare with the string "\skip"
  \ifx\sdaau@temp\sdaau@skipstring
    \typeout{\string\providelength: \string#1 already a length}%
  \else
    \@latex@error
      {\string#1 illegal in \string\providelength}
      {\string#1 is defined, but not with \string\newlength}%
  \fi
}
\def\sdaau@getfive#1#2#3#4#5#6${#1#2#3#4#5}
\edef\sdaau@skipstring{\string\skip}
\makeatother
%%%%%%%%%%%%%%%%%%%%%

% Support for easy cross-encing
\usepackage[capitalize]{cleveref}
\crefname{section}{Sec.}{Secs.}
\Crefname{section}{Section}{Sections}
\Crefname{table}{Table}{Tables}
\crefname{table}{Tab.}{Tabs.}

% Images
\def\D{\altmathcal{D}}
\def\I{\altmathcal{I}}
\def\O{\altmathcal{O}}
\def\res{\altmathcal{R}}

% Vectors\mathbf{x}
\def\b{\mathbf{b}}
\def\c{\mathbf{c}}
\def\d{\mathbf{d}}
\def\o{\mathbf{o}}
\def\p{\mathbf{p}}
\def\t{\mathbf{t}}
\def\x{\mathbf{x}}
\def\z{\mathbf{z}}

% Matrices
\def\K{\mathbfit{K}}
\def\R{\mathbfit{R}}

% Functions
\def\ang{\phi}
\def\dehom{\mu}
\def\proj{\pi}
\def\sigmoid{S}
\def\vis{\nu}
\def\r{\mathbfit{r}}

% Bracketed
\def\bp{(\p\!)} % (p)
\def\bt{(t\!)} % (p)
\def\bx{(\x\neg)} % (x)

% Subsets
\def\ok{\o_{\neg k}}
\def\tk{\t_{\neg k}}
\def\wk{\w_{\neg k}}
\def\xi{\x_{\neg i}}
\def\zk{\z_{\neg k}}
\def\Kk{\K_{\neg k}}
\def\Rk{\R_{\neg k}}

% Helpers
\def\ng{\hspace{-0.1mm}}
\def\neg{\hspace{-0.2mm}}
\def\pos{\hspace{0.2mm}}

\newcommand{\norm}[1]{\left\lVert#1\right\rVert}

% Create \overrightharpoon, as a replacement for \overrightvector.
\makeatletter
\newcommand*\MY@rightharpoonupfill@{%
    \arrowfill@\relbar\relbar\rightharpoonup
}
\newcommand*\overrightharpoon{%
    \mathpalette{\overarrow@\MY@rightharpoonupfill@}%
}
\makeatother

% A scalable sum symbol. Use like this: \nsum[0.8]
\newlength{\depthofsumsign}
\setlength{\depthofsumsign}{\depthof{$\sum$}}
\newcommand{\nsum}[1][1.4]{% only for \displaystyle
    \mathop{%
        \raisebox
            {-#1\depthofsumsign+1\depthofsumsign}
            {\scalebox
                {#1}
                {$\displaystyle\sum$}%
            }
    }
}
