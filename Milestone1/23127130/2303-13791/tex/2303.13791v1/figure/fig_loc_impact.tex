\begin{figure}[t]
\centering
\small
\setlength{\tabcolsep}{1pt}
\renewcommand{\arraystretch}{0.6}
\begin{tabular}{cc}
\centeredtab{\includegraphics[width=0.49\columnwidth]{fig/motivations/ballroom_global.jpg}} &
\centeredtab{\includegraphics[width=0.49\columnwidth]{fig/motivations/ballroom_local.jpg}} \\
\centeredtab{\includegraphics[width=0.49\columnwidth]{fig/motivations/uw2_global.jpg}} &
\centeredtab{\includegraphics[width=0.49\columnwidth]{fig/motivations/uw2_local.jpg}} \\
Global & Local \\
\end{tabular}%
\vspace{-2mm}
\caption{\textbf{Importance of locality.} When using a single global radiance field, the whole scene reconstruction fails if a few camera poses are not estimated correctly (first row). Local radiance fields also allow us to dynamically allocate the representation's capacity around the camera trajectory, producing sharper results (second row). }
\vspace{-4mm}
\label{fig:loc_impact}
\end{figure}
