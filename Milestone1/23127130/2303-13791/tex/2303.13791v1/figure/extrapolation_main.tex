\begin{figure}[ht]
\centering
\footnotesize
\setlength{\tabcolsep}{1pt}
\renewcommand{\arraystretch}{0.3}
%%
\begin{tabular}{cc}
\centeredtab{\animategraphics[width=40px]{30}{fig/extra/in3/}{0}{49} \\ Training \\ views} & 
\begin{tabular}{cc}
\centeredtab{\rotatebox[origin=c]{90}{Rotation}} & 
\centeredtab{\animategraphics[loop,height=47px]{10}{fig/extra/r3/}{0}{15}} \\
\centeredtab{\rotatebox[origin=c]{90}{Translation}} &
\centeredtab{\animategraphics[loop,height=47px]{10}{fig/extra/t3/}{0}{39}} \\
\end{tabular}
%%
% \begin{tabular}{cc}
% \centeredtab{\includegraphics[width=40px]{fig/extra/in3/0.jpg} \\ Training \\ view} & 
% \begin{tabular}{cc}
% \centeredtab{\rotatebox[origin=c]{90}{Rotation}} & 
% \centeredtab{\includegraphics[height=47px]{fig/extra/r3/0.jpg}} \\
% \centeredtab{\rotatebox[origin=c]{90}{Translation}} &
% \centeredtab{\includegraphics[height=47px]{fig/extra/t3/0.jpg}} \\
% \end{tabular}
%%
\\
 & \hspace{8mm} Mip-NeRF360 \hfill Ours \hspace{11mm} 
\end{tabular}%
\vspace{-3mm}
\caption{\small\textbf{Input path deviation.} We can render novel views that deviate from input path.
Please use Adobe Reader and click on images to see the embedded \emph{animations}.
}
\label{fig:extra}
\vspace{-1.5mm}
\end{figure}
