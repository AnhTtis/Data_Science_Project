\begin{figure}[t]
\footnotesize
\centering
\begin{overpic}[width=\columnwidth]{fig/parameterization.pdf}
\put (0, 42) {\renewcommand{\arraystretch}{0}\centeredtab{Represented \\ space}}
\put (50, 41) {\renewcommand{\arraystretch}{0.5}\centeredtab{Uncontracted \\ space}}
\put (6, -1) {(a) NDC}
\put (31.5, -1) {(b) Mip-NeRF360}
\put (79, -1) {(c) Ours}
\end{overpic}
\captionof{figure}{
\textbf{Space parameterization.}
(a) NDC, used by NeRF~\cite{mildenhall2020nerf} for forward-facing scenes, maps a frustum to a unit cube volume. 
While a sensible approach for forward-facing cameras, it is only able to represent a small portion of a scene as the frustum cannot be extended beyond a field of view of $120^\circ$ or so without significant distortion. 
(b) Mip-NeRF360's~\cite{barron2022mipnerf360} space contraction squeezes the background and fits the entire space into a sphere of radius 2. 
It is designed for inward-facing 360 scenes and cannot scale to long trajectories. 
(c) Our approach allocates several radiance fields along the camera trajectory. 
Each radiance field maps the entire space to a $[-2, 2]$ cube (Equation~\eqref{eq:contract}) and, each having its own center for contraction (Equatione~\eqref{eq:shift}), the high-resolution uncontracted space follows the camera trajectory and our approach can adapt to any camera path. 
}
\label{fig:mip360vsloc}
\vspace{-3mm}
\end{figure}
