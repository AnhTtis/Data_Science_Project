\documentclass[pmlr]{jmlr}% new name PMLR (Proceedings of Machine Learning Research)

 % The following packages will be automatically loaded:
 % amsmath, amssymb, natbib, graphicx, url, algorithm2e

 %\usepackage{rotating}% for sideways figures and tables
\usepackage{longtable}% for long tables
\usepackage{cleveref}
\usepackage{xcolor}
\usepackage{comment}
\definecolor{rulescolor}{RGB}{200,200,200}
 % The booktabs package is used by this sample document
 % (it provides \toprule, \midrule and \bottomrule).
 % Remove the next line if you don't require it.
\usepackage{booktabs}
\usepackage{wrapfig,lipsum}
\usepackage{multirow}
 % The siunitx package is used by this sample document
 % to align numbers in a column by their decimal point.
 % Remove the next line if you don't require it.
\usepackage[load-configurations=version-1]{siunitx} % newer version
 %\usepackage{siunitx}

 % The following command is just for this sample document:
\newcommand{\cs}[1]{\texttt{\char`\\#1}}

 % Define an unnumbered theorem just for this sample document:
\theorembodyfont{\upshape}
\theoremheaderfont{\scshape}
\theorempostheader{:}
\theoremsep{\newline}
\newtheorem*{note}{Note}

 % change the arguments, as appropriate, in the following:
\jmlrvolume{1}
\jmlryear{2023}
\jmlrworkshop{}

\title[A Retrospective of the MineRL BASALT 2022 Competition]{Towards Solving Fuzzy Tasks with Human Feedback: \titlebreak A Retrospective of the MineRL BASALT 2022 Competition}%\titletag{\thanks{sample footnote}}}

 % Use \Name{Author Name} to specify the name.

 % Spaces are used to separate forenames from the surname so that
 % the surnames can be picked up for the page header and copyright footer.
 
 % If the surname contains spaces, enclose the surname
 % in braces, e.g. \Name{John {Smith Jones}} similarly
 % if the name has a "von" part, e.g \Name{Jane {de Winter}}.
 % If the first letter in the forenames is a diacritic
 % enclose the diacritic in braces, e.g. \Name{{\'E}louise Smith}

 % *** Make sure there's no spurious space before \nametag ***

 % Two authors with the same address
%  \author{\Name{Author Name1\nametag{\thanks{with a note}}} \Email{abc@sample.com}\and
%   \Name{Author Name2} \Email{xyz@sample.com}\\
%   \addr Address}

 % Three or more authors with the same address:
 % \author{\Name{Author Name1} \Email{an1@sample.com}\\
 %  \Name{Author Name2} \Email{an2@sample.com}\\
 %  \Name{Author Name3} \Email{an3@sample.com}\\
 %  \Name{Author Name4} \Email{an4@sample.com}\\
 %  \Name{Author Name5} \Email{an5@sample.com}\\
 %  \Name{Author Name6} \Email{an6@sample.com}\\
 %  \Name{Author Name7} \Email{an7@sample.com}\\
 %  \Name{Author Name8} \Email{an8@sample.com}\\
 %  \Name{Author Name9} \Email{an9@sample.com}\\
 %  \Name{Author Name10} \Email{an10@sample.com}\\
 %  \Name{Author Name11} \Email{an11@sample.com}\\
 %  \Name{Author Name12} \Email{an12@sample.com}\\
 %  \Name{Author Name13} \Email{an13@sample.com}\\
 %  \Name{Author Name14} \Email{an14@sample.com}\\
 %  \addr Address}


 % Authors with different addresses:
  \author{\Name{Stephanie Milani} \Email{smilani@cs.cmu.edu}\\
  \addr Carnegie Mellon University
  \AND
  \Name{Anssi Kanervisto}\thanks{Equal contribution} \Email{anssi.kanervisto@microsoft.com}\\
  \addr Microsoft Research, Cambridge 
  \AND 
  \Name{Karolis Ramanauskas$^*$} 
  \Email{kr711@bath.ac.uk}\\
  \addr University of Bath
  \AND 
  \Name{Sander Schulhoff} 
  \Email{sschulho@umd.edu}\\ 
  \addr University of Maryland
  \AND 
  \Name{Brandon Houghton} 
  \Email{brandon@openai.com}\\ 
  \addr OpenAI
  \AND 
  \Name{Sharada Mohanty} 
  \Email{mohanty@aicrowd.com}\\
  \addr AIcrowd
  \AND 
  \Name{Byron Galbraith} 
  \Email{}
  \addr 
  \AND 
  %% 1st place team 
  \Name{Ke Chen} 
  \Email{chenke3@corp.netease.com}\\
  \Name{Yan Song}
  \Email{songyan@corp.netease.com}\\
  \Name{Tianze Zhou}  
  \Email{zhoutianze@corp.netease.com}\\
  \Name{Bingquan Yu}
  \Email{yubingquan@corp.netease.com}\\
  \Name{He Liu}
  \Email{liuhe1@corp.netease.com}\\
  \Name{Kai Guan} 
  \Email{guankai1@corp.netease.com|}\\
  \Name{Yujing Hu} 
  \Email{huyujing@corp.netease.com}\\
  \Name{Tangjie Lv}
  \Email{hzlvtangjie@corp.netease.com}\\ 
  \addr NetEase Fuxi AI Lab 
  \AND 
  %% second place team 
  \Name{Federico Malato}
  \Email{fmalato@uef.fi}\\
  \addr University of Eastern Finland
  \AND 
  \Name{Florian Leopold} 
  \Email{fleopold@techfak.uni-bielefeld.de}\\
  \addr University of Bielefeld
  \AND 
  \Name{Amogh Raut}
  \Email{}\\
  \addr Indian Institute of Technology 
  \AND 
  \Name{Ville Hautamäki}
  \Email{ville.hautamaki@uef.fi}\\
  \addr University of Eastern Finland 
  \AND 
  \Name{Andrew Melnik}
  \Email{andrew.melnik@uni-bielefeld.de}\\
  \addr University of Bielefeld
  \AND 
  %% third place team 
  \Name{Shu Ishida}
  \Email{ishida@robots.ox.ac.uk}\\
  \Name{João F. Henriques} 
  \Email{joao@robots.ox.ac.uk}\\
  \addr Visual Geometry Group, University of Oxford
  \AND 
  %% KABasalt (tied first research prize) 
  \Name{Robert Klassert} 
  \Email{robertklassert@pm.me}\\
  \addr Forschungszentrum Informatik Karlsruhe, Berkeley Existential Risk Initiative
  \AND 
  \Name{Walter Laurito} 
  \Email{lauritowal@yahoo.com} \\
  \addr Forschungszentrum Informatik Karlsruhe
  \AND 
  %% KAIROS (3 place research prize)
  \Name{Ellen Novoseller}
  \Email{ellen.r.novoseller.ctr@army.mil}\\
  \Name{Vinicius G. Goecks} 
  \Email{vinicius.goecks@gmail.com}\\
  \Name{Nicholas Waytowich}
  \Email{nicholas.r.waytowich.civ@army.mil}\\
  \addr DEVCOM Army Research Laboratory 
  \AND 
  \Name{David Watkins}
  \Email{davidwatkins@cs.columbia.edu}\\
  \addr Boston Dynamics AI Institute, Columbia University
  \AND 
  \Name{Josh Miller}
  \Email{joshua.r.miller138.civ@army.mil}\\
  \addr DEVCOM Army Research Laboratory
  \AND 
  \Name{Rohin Shah} 
  \Email{rohinmshah@deepmind.com}\\ 
  \addr DeepMind
 }

 \newcommand{\kb}[1]{{\color{purple}{[{\bf Kiante}: #1]}}}

% \editor{Editor's name}
 % \editors{List of editors' names}

%%% 10 pages + appendix
\begin{document}

\maketitle
\DeclareMathOperator*{\veccat}{%
    \mathchoice%
        {\Bigg\Vert}%
        {\Big\Vert}%
        {\Vert}%
        {\Vert}%
}%


\begin{abstract}
\smchange{To facilitate research in the direction of fine-tuning foundation models from human feedback, we held the MineRL BASALT Competition on Fine-Tuning from Human Feedback at NeurIPS 2022. 
The BASALT challenge asks teams to compete to develop algorithms to solve tasks with hard-to-specify reward functions in Minecraft. 
Through this competition, we aimed to promote the development of algorithms that use human feedback as channels to learn the desired behavior. 
We describe the competition and provide an overview of the top solutions. 
We conclude by discussing the impact of the competition and future directions for improvement.}
\end{abstract}

\begin{keywords}
Learning from humans, fine-tuning, reward modeling, imitation learning, preference learning, reinforcement learning from human feedback
\end{keywords}

\section{Introduction}\label{sec:intro}

Collaborative grounded natural language interactions involve multiple agents, either human or machine, working together to complete tasks while coordinating using natural language.
A key obstacle in studying such scenarios is building the research interaction platform,  a significant design and engineering undertaking. 
This requires building and designing the interaction environment, the task the agents collaborate on, an interface for both machine learning models and human agents, and a process to onboard human agents. 
Each aspect dramatically influences the interaction and language elicited, and is critical to get right. 






We introduce \gamename, a platform for the study of collaborative grounded natural language interaction, and demonstrate its use through the deployment of a learned collaborative natural language agent. 
\gamename largely instantiates the \cerealbar scenario~\citep{Suhr2019:cerealbar},\footnote{\gamename introduces several optional modifications to \cerealbar aimed at richer language and tighter collaboration.} but is implemented from scratch to emphasize research accessibility. 
\gamename is a customizable, scalable, and complete research platform, including server and clients for multi-agent human-machine interactions, tools for real-time data management, and processes to onboard crowdsourcing workers. 








The \gamename scenario poses learning and reasoning challenges, as well as opportunities. 
Comprehending and producing instructions in \gamename requires addressing the symbol grounding problem~\cite{Harnad1990:symbol-grounding-problem}, which is studied extensively in the instruction following~\cite[e.g.,][]{Chen:11, Artzi:13,Misra:17instructions,Fried:17pragmatic-models} and generation~\cite[e.g.,][]{Mei:16generation,Wang2021:generatingInstructions} literature. 
However, the collaborative scenario remains relatively understudied. 
Collaboration is not simply an added complication, but dramatically alters both interaction and learning through joint presence and action. 
It allows the instructor to ad-hoc modify the tasks they delegate based on the follower behavior, potentially recovering from system failures. 
At the same time, this adaptation creates constant distribution shift, a significant generalization challenge. 
Learning is also drastically transformed through collaboration.
The constant engagement of other agents (including humans), the ability to modify delegation strategies, and the shared task-based incentives  bring about within-interaction signals that can be used for continual learning, reducing the dependency on annotated data and enabling model adaptation.



We deploy a demonstration of \gamename with a learned baseline instruction following agent (\autoref{sec:deployment}). 
Players can connect to \gamename and collaborate with our agent or other human agents at \url{https://cb2.ai/}.\footnote{Our deployment has received IRB exemption. All recorded data is anonymized.}
The \gamename platform is available at \url{https://github.com/lil-lab/cb2}.
A video demonstration of \gamename is available at \url{https://youtu.be/tALpX_KKmIw}. 







\section{Competition Overview}
\label{sec:overview}

\smchange{We now present important details about the competition (more in \Cref{apd:comp_details}).} % , including the tasks, provided resources, rules, validation technique, evaluation, and prizes.}
%\kr{Figure~\ref{fig:overview} shows the overall competition structure.}
%\smchange{For more information, please see \Cref{apd:comp_details}.}

\paragraph{Tasks}
\smchange{
We provided a set of tasks in Minecraft that consist of a simple English-language task description and a Gym~\citep{brockman2016openai} environment. 
They were: \cavetasknospace, \waterfalltasknospace, \pentaskfull (\pentasknospace), and \housetaskfull (\housetasknospace).}
\smchange{
%We describe the tasks in more detail in \Cref{apd:comp_details}.
Crucially, the Gym environments lacked any associated reward functions.}
For each task, we provided a dataset of human demonstrations consisting of a sequence of state-action pairs.
\smchange{To help participants familiarize themselves with the code pipeline, we included an introductory track with a task with an explicit reward function, \texttt{ObtainDiamondShovel}.}

\paragraph{Resources}
\smchange{Through our partnership with AIcrowd, we provided competitors with a unified interface to register for the competition, submit trained agents, ask questions, and monitor their progress on a public leaderboard.}\footnote{\url{https://www.aicrowd.com/challenges/neurips-2022-minerl-basalt-competition}}
We partnered with OpenAI to release over 600h of human demonstrations in the four tasks and baselines built on the VPT model~\citep{baker2022video}. 
The baseline solution fine-tunes VPT model with behavioral cloning using the collected dataset.\footnote{The pre-trained models and human demonstration data are available from the following GitHub pages: \url{https://github.com/openai/Video-Pre-Training} and \url{https://github.com/minerllabs/basalt-2022-behavioural-cloning-baseline}.}
\ak{To provide mentorship and foster an active community, we continued maintaining the MineRL Discord server.} %, as we previously found it to increase engagement and provide support to participants.}
%\smnote{mention Aicrowd website, baselines, mentorship, computing and eval resources, and tutorial/documentation}

\paragraph{Rules and Validation}
\smchange{We required methods to use only the specified Gym API. 
For reproducibility, we mandated that participants submit their \textit{training} code.
We limited the size of data and public models (30MB) that could be included in their submissions. 
The provided resources did not count toward this limit. 
The AIcrowd page contains the full rules.\footnote{\url{https://www.aicrowd.com/challenges/neurips-2022-minerl-basalt-competition/challenge_rules}}}
%\paragraph{Validation}
\ak{
To ensure reproducibility, we retrained the finalists' submissions with their provided training code for up to 4 days (on all four tasks) using at most 10 hours of human feedback. 
We fixed the compute to 12 CPU cores, 56GB of RAM, and an NVIDIA Tesla K80 GPU. The training code could query humans for input with traditional desktop UIs. 
We provided human contractors who connected to the training instances to provide input. We also manually inspected the code to ensure compliance with the 30MB upload limit.
}

\paragraph{Evaluation}
\smchange{
%\ak{We describe evaluation in more detail in \Cref{apd:comp_details} but explain it at a high level here.}
\ak{Upon submission,} we deployed the agents on fixed world seeds of each task to generate multiple example videos.
\ak{After the submission deadline,} we asked human judges recruited through Amazon Mechanical Turk (MTurk) to choose which agent better completed the task through pairwise comparisons of the agents.
Given a dataset of these comparisons, we computed each agent's scores using the TrueSkill system \citep{herbrich2006trueskill}.
We determined the winners by normalizing and aggregating these scores across tasks.}
%We describe this metric in more detail in \Cref{apd:comp_details} but explain it at a high level here.
%We first normalize the TrueSkill scores across submissions to the tasks to ensure that no one task dominates the final score.
%We then average the normalized score to obtain an ovethe rall score.
%}
%\kb{Could you shed a little light on how the MTurk participants were preped? What information was given to them? How were the told to judge the results?}

\paragraph{Prizes}
\ak{We awarded the top three solutions as ranked by the human evaluation 7000, 4000, and 3000 USD, respectively. 
\smchange{To encourage the exploration of creative solutions, we asked each advisor to select a single team to award a research prize of 1000 USD.} 
To drive community engagement, we gave a total of 1000 USD to participants who helped others or otherwise contributed to the competition.}


\paragraph{Related Competitions}
\smchange{
Minecraft is a popular platform for various research benchmarks~\citep{grbic2021evocraft,gray2019craftassist,johnson2016malmo,hafner2021benchmarking,fan2022minedojo} and competitions. 
%These benchmarks include searching for in-game artifacts in an open-ended manner~\citep{grbic2021evocraft}, developing dialogue-enabled interactive agents~\citep{gray2019craftassist}, and building generally-capable embodied agents~\citep{johnson2016malmo,hafner2021benchmarking,fan2022minedojo}. 
Their diversity demonstrates Minecraft's flexibility to both instantiate and evaluate a variety of interesting problems. 
Research competitions have focused on multi-agent learning of cooperative and competitive tasks~\citep{perez2019multi}, generating functional and believable settlements~\citep{salge2018generative}, and learning to build structures based on natural language descriptions~\citep{kiseleva2021neurips}.
However, none of these competitions are particularly relevant to \fthf{} for hard-to-specify sequential decision-making tasks. 
The most related competitions are the previous MineRL Diamond competitions \smchange{that focused on} learning from a reward function and demonstrations to solve a crisply-defined task~\citep{guss2019neurips}. 
In contrast, we emphasize the use of fine-tuning techniques and utilize tasks that do not have an easy-to-define reward function.
}
\section{Solutions}
\label{sec:solutions}
\begin{table}[t]
    \centering
    \begin{tabular}{lcc}
    \toprule 
         & 2021 & 2022  \\
    \midrule
     Number of Teams & 37 & \textbf{63} \\
     Number of Individuals & 358 & \textbf{446} \\ 
     Number of Submissions & 271 & \textbf{504} \\
     Number of Teams that Scored Higher than Baselines & 10 & \textbf{11} \\
    \bottomrule
    \end{tabular}
    \caption{\textbf{Participation statistics.} The 2022 competition saw more teams, individuals, and submissions than 2021, demonstrating an increased interest.\vspace{-8pt}}
    \label{tab:basalt_participation}
\end{table}

\smchange{
Compared to last year, this year we saw a larger number of individuals (and teams) competing and nearly double the submissions (see \Cref{tab:basalt_participation}).
%This data is summarized in \Cref{tab:basalt_participation}.
We believe that the number of submissions increased in part due to the introductory track. 
Interestingly, the scores corresponding to submissions that achieved higher than the lowest scores (e.g. approaches that did not just submit the baselines) remained the same as last year.
This similarity indicates that the increased popularity likely stemmed from less experienced teams.
}

\smchange{
In the rest of this section, we describe approaches taken by the competition winners and the teams who were selected for research prizes.\footnote{To see example videos of each of the agents trained using the algorithms described performing each of the tasks, please see: \url{https://www.youtube.com/playlist?list=PL7H6ODSaA0Y-yyJDXLOJQQcThg7_SBoqU}.}
}
%\kr{The higher interest can also be seen in the MineRL Discord server statistics in Table \ref{tab:discord_stats}. Despite there being two competitions (Diamond and BASALT) last year, and only the less popular one being held this year, there was only a modest reduction in the community activity and new members.}

\begin{table}[t]
\centering
\floatconts
  {tab:leaderboard}
  {\begin{tabular}{lSSSSS} \label{table:winners}
  \\ \toprule
  \bfseries Team & {\bfseries \cavetasknospace} & {\bfseries \waterfalltasknospace} & {\bfseries \pentasknospace} & {\bfseries \housetasknospace} & {\bfseries Average}\\
  \midrule
   GoUp & 0.31 & {\bfseries 1.21} & {\bfseries 0.28} & {\bfseries 1.11} & {\bfseries 0.73} \\
   UniTeam & {\bfseries 0.56} & -0.10 & 0.02 & 0.04 & {\bfseries 0.13} \\
   voggite & 0.21 & 0.43 & -0.20 & -0.18 & {\bfseries 0.06} \\
   JustATry & -0.31 & -0.02 & -0.15 & -0.14 & {\bfseries -0.15} \\
   TheRealMiners & 0.07 & -0.03 & -0.28 & -0.38 & {\bfseries -0.16} \\
   yamato.kataoka & -0.33 & -0.20 & -0.27 & -0.18 & {\bfseries -0.25} \\
   corianas & -0.05 & -0.26 & -0.45 & -0.24 & {\bfseries -0.25} \\
   Li\_and\_Ivan & -0.15 & -0.72 & -0.14 & -0.22 & {\bfseries -0.31} \\
   KAIROS & -0.35 & -0.32 & -0.41 & -0.36 & {\bfseries -0.36} \\
   Miner007 & -0.07 & -0.76 & -0.12 & -0.52 & {\bfseries -0.37} \\
   KABasalt & -0.57 & -0.23 & -0.41 & -0.31 & {\bfseries -0.38} \\
   %pm & -0.52 & -0.98 & -0.45 & -0.41 & {\bfseries -0.59} \\
   %Dopamind & -1.03 & -0.84 & -1.09 & -1.25 & {\bfseries -1.05} \\
   \midrule
   Human2 & 2.52 & 2.42 & 2.46 & 2.34 & {\bfseries 2.43} \\
   Human1 & 1.94 & 1.94 & 2.52 & 2.28 & {\bfseries 2.17} \\
   BC-Baseline & -0.43 & -0.23 & -0.19 & -0.42 & {\bfseries -0.32} \\
   Random & -1.80 & -1.29 & -1.14 & -1.16 & {\bfseries -1.35} \\
  \bottomrule
  \end{tabular}}
  {\caption{{\bfseries Leaderboard: normalized TrueSkill scores.} \smchange{The top three teams were GoUp, UniTeam, and voggite. GoUp achieved higher performance on all tasks but \cavetasknospace. We include scores for BC-Baseline (organizer-provided baseline), two expert humans, and a random agent.}\vspace{-8pt}}\label{tab:leaderboard}}
\end{table}

\subsection{Approaches of Competition Winners}
\label{subsec:winner_approaches}
%\smchange{We now turn our attention to the approaches taken by the competition winners.}
\smchange{The scores of the top teams are captured in \Cref{tab:leaderboard}.}
\smchange{GoUp achieved the highest overall score of $2.09$, with UniTeam and voggite taking second- and third-place, respectively.}
\smchange{GoUp achieved the highest scores on all tasks except for \cavetasknospace. 
In this case, the UniTeam scored higher.
We now describe each team's solution in turn (more details in \Cref{sec:solutions}).} 

\paragraph{First Place: GoUp}
\smchange{This solution utilized the power of machine learning and human knowledge by dividing each task into two parts: one that can be solved by transforming human knowledge into code (i.e., scripts) and the other that requires machine learning to solve.\footnote{Open-source code for GoUp: \url{https://github.com/gomiss/neurips-2022-minerl-basalt-competition}.} % for completing which machine learning approaches should be adopted. 
The team found that all of the four tasks consist of the same flow.
The agent walks around, searches for a target (e.g., a cave), then solves the task. 
They identified the targets in each task by training several classifiers and object detection models. 
To source data for training their models, they manually labeled images collected from the expert videos provided by the competition with the corresponding task. 
\Cref{Fig:Solution} shows the framework of the solution. 
Taking the AnimalPen task as an example, the solution for this task contains a fine-tuned VPT model for moving the agent, a fine-tuned YOLOv5 \citep{yolov5} detector for detecting the types and locations of animals, a fine-tuned MobileNet~\citep{howard2019searching} detector for identifying the location of fence placement, and a finite state machine that controls the executing flow of all the components. 
}

\paragraph{Second Place: UniTeam} %, \textit{Search-Based Behavioral Cloning}}
\begin{figure*}[t]
\centering
  \includegraphics[scale=0.4]{media/go-up-solution.pdf}
  \caption{\smchange{\textbf{Decomposition of subtasks by GoUp.} To solve tasks that cannot be precisely described by a person, they leverage a variety of machine learning techniques. To solve the remaining tasks, they use human knowledge through scripting. \vspace{-8pt}} \label{Fig:Solution} }
\end{figure*}

%\kb{Is this search-based BC or just K-NN on trajectories using BC features?}
\smchange{UniTeam proposed a search-based behavioral cloning approach, which aims to reproduce an expert's behavior by copying relevant actions from relevant situations in the demonstration dataset.\footnote{Open-source code for UniTeam: \url{https://github.com/fmalato/basalt_2022_submission}.} 
\Cref{fig:uniteam_approach} shows their approach.
They defined a situation as a subset of consecutive frames and actions within a recorded trajectory, which they encoded using the VPT network.
To find the most similar situations to the current one, they used a pre-trained VPT model to produce latent representations.
They searched for the nearest embedding point in the VPT latent space to find the reference situation.
They assumed that retrieved situations represent an optimal solution to a specific past situation. 
As a result, they copy the corresponding actions.
%Such solutions can be re-used to address similar situations happening in the present~\citep{beohar2022solving}.
}
\smchange{ 
After each timestep, they updated both the current and reference situations. 
Because the reference and current situations diverge over time, they measured the L1 similarity between the situations at each timestep. 
Once the distance between trajectories was greater than some fixed threshold (or after 128 timesteps), they performed a new search.
See~\cite{malato2022behavioral} for more details.}

\begin{figure}[t]
    \centering
    \includegraphics[width=\textwidth]{media/uniteam_figure.png}
    \caption{\smchange{\textbf{Generation of past-situation points in the VPT latent space by UniTeam.} (A) They encode situations from the BC dataset through a pre-trained VPT model. (B) They compute the L1 distance between their embedded current situation and the embedded situations from the expert’s dataset. Then, they copy actions from the best-matching reference situation. (C) Examples of visual (left) and numerical (right) similarity between current and reference situations. In the right figures, the blue vertical lines denote new time-based searches and red lines correspond to new threshold-based searches.
}\vspace{-8pt}}
    \label{fig:uniteam_approach}
\end{figure}

\paragraph{Third Place: voggite} 
\smchange{This team focused on improving the behavioral cloning baseline.\footnote{Open-source code for voggite: \url{https://github.com/shuishida/minerl_2022}.}
To enable quick iterations of solutions, they precomputed the VPT state embeddings for all of the expert demonstrations, then trained a lightweight policy model with PyTorch Lightning. 
To focus on rarer but significant actions, they introduced action reweighting based on how frequently they were encountered. 
They observed that certain actions serve as signals that trigger higher-level changes in state. 
For example, the use of a bucket to pour water to complete the waterfall task signals that the agent should begin to climb down the mountain to take a scenic picture. 
They manually encoded these trigger actions along with the change in action distribution (e.g., decreasing the probability of moving forward once the agent starts building something); however, they plan to incorporate this idea into a hierarchical reinforcement learning framework, such as option-critic~\citep{bacon2017option}. 
}

\subsection{Approaches of Research Prize Winners}
\smchange{In addition to evaluating teams on the performance of their algorithms using human evaluations, we awarded additional prizes for research contributions.
To select the winners, each advisor read through a description of the approach and watched a video demonstration of the agent behavior. 
They had the option to view the agent's score. 
Each advisor then awarded a prize to the approach for their research contribution.
In general, the advisors preferred elegant, intuitive approaches that were ambitious, even if the final scores were relatively low. 
Advisors independently chose the following teams for the research prize: UniTeam (2 votes), KABasalt (2 votes), and KAIROS (1 vote). 
We now describe the remaining approaches in turn.
} 

\paragraph{KABasalt} 
\smchange{This team aimed for a reward modeling approach based on preference feedback after a pre-training phase through imitation and preference learning with demonstrations.\footnote{Open-source code for KABasalt: \url{https://github.com/BASALT-2022-Karlsruhe/ka-basalt-2022}.}
The main contribution was integrating the VPT models into the \textit{Imitation} library~\citep{gleave2022imitation} and {Stable-Baselines3}~\citep{stable-baselines3}, which required the ability to create compatible policy and reward model objects from the VPT backbone.
After tackling the myriad challenges of making reinforcement learning work with foundation models, such as incompatible interfaces, they are now further developing their solution and adding support for more features, like off-policy reinforcement learning algorithms.
}
%Although we have produced a working training pipeline, we couldn't carry out and optimize the full training including human preferences because of time constraints.
%We underestimated the difficulty of making RL work with foundation models like the provided VPT models. For example, integrating them into existing libraries was not straightforward because of initially incompatible interfaces and missing features like support for recurrent policies.
%We are in the process of polishing our solution and to contribute our code, packaged into new features of the libraries we utilized. In the future we plan to add support for off-policy RL algorithms like PEBBLE.
%Overall, we believe that our work represents an important step towards large scale reward modeling.
%\smnote{todo for kabasalt: add missing refs, link to open source, and update language/description}

\paragraph{KAIROS}
\begin{figure}[t]
    \centering
    \includegraphics[width=0.95\columnwidth]{media/PIQL_no_state_classifier.png}
    \caption{\smchange{\textbf{Preference-Based IQ-Learning (PIQL) algorithm proposed by KAIROS.} Their algorithm incorporates imitation learning with reinforcement learning using a reward model trained using pairwise human preferences over videos agents completing the tasks.}\vspace{-8pt}}
    \label{fig:piql_diagram}
\end{figure}

\smchange{This team proposed Preference-Based IQ-Learning (PIQL), a novel algorithm that extends a state-of-the-art actor-critic imitation learning algorithm (IQ-Learn~\citep{garg2021iq}) to additionally leverage the VPT model and online pairwise preferences over trajectories~\citep{christiano2017deep}.\footnote{Open-source code for KAIROS: \url{https://github.com/nwayt001/preference-IQL.git}.}
\Cref{fig:piql_diagram} shows their approach. 
PIQL uses pairwise preferences over videos of agents completing the task to learn a reward function via the Bradley-Terry model~\citep{christiano2017deep} through a reward head attached to the VPT network.
They define a novel critic loss that includes an IQ-learn term that prioritizes imitating the demonstrations and a reinforcement learning term~\citep{haarnoja2018soft} that aims to maximize the reward learned from human preferences. 
They include regularization terms that penalize the behavioral cloning loss on the demonstrations and the KL divergence between policies in successive learning iterations.
Thus, while performing imitation learning via IQ-Learn, PIQL obtains human pairwise preferences and learns a reward model that assists Q-function learning alongside the demonstrations.
}

%We propose Preference-Based IQ-Learning (PIQL, pronounced “pickle”), a novel algorithmic approach that extends IQ-Learn~\citep{garg2021iq}---a state-of-the-art actor-critic imitation learning algorithm---to not only learn from expert demonstrations, but to also leverage both the VPT model and online human pairwise preferences over trajectories of agent-environment interactions~\citep{christiano2017deep}, as seen in Figure \ref{fig:piql_diagram}. Each preference is collected by a human comparing two side-by-side video segments of the Minecraft agent interacting with the BASALT environment. PIQL utilizes these pairwise preferences to learn a reward function via the Bradley-Terry model~\citep{christiano2017deep} through a reward head attached to the VPT network. We define a novel Q loss that sums a) an IQ-learn term that prioritizes imitating the demonstrations and b) a reinforcement learning (RL) term as in the Soft Actor Critic (SAC) RL algorithm~\citep{haarnoja2018soft}, which aims to maximize the reward learned from human preferences. Simultaneously, we train the policy as in SAC to maximize the learned Q-values, with regularization terms penalizing the behavioral cloning loss on the demonstrations and the KL divergence between policies in successive learning iterations. 
%Thus, while performing imitation learning via IQ-Learn, PIQL obtains human pairwise preferences and learns a reward model that assists Q-function learning alongside the demonstrations. 
%Algorithm~\ref{alg:PIQL} describes the complete PIQL algorithm.

\begin{comment}
\begin{algorithm}[h]
    \label{alg:PIQL}
    Load pre-trained VPT weights \\
    \For{each iteration}{
    Optimize policy via SAC given current Q weights \\
    Collect pairwise preferences \\
    Train reward to maximize pairwise preference likelihood \\
    Optimize Q-function via: a) IQ-learn loss (from demonstrations) and b) SAC RL loss to encourage high human preference reward
    }
    \caption{PIQL algorithm.}
\end{algorithm}
\end{comment}

\section{Discussion}
\label{sec:discussion}
\smchange{The results and outcomes of the competition indicate that there are parts that worked well and many opportunities for improvement.}
%\smnote{lessons learned, future challenges, and improvement.}
%\smnote{my general comment here is that it would be good to have concrete takeaways. for example, if i want to make a competition next year, what should i concretely consider when designing tasks? what should i do to evaluate submissions?}

\paragraph{Task Design}
\kr{
While the four tasks stayed the same this year, the intended way to solve them changed. Fine-tuning the VPT model from human feedback meant that participants started with an agent already capable of traversing and exploring the Minecraft world. Compared to the last year, the solutions of the 2022 edition were indeed all able to navigate around the world, and they were mainly separable by the ability to place blocks down and coherency in their actions.
}
\smchange{We were again pleased to see that teams used significantly different approaches in tackling the tasks, which serves the research goals of the competition.}

\paragraph{Evaluation Methodology and Results}
This year, we used Amazon Mechanical MTurk to crowdsource the evaluation across multiple workers (see \Cref{app:evaluation_results} for details).
By using a scalable platform, we increased the number of human evaluators over the 2021 edition, which relied on a small pool of hired contractors.
To ensure high-quality responses, we set high qualification limits. 
We also asked the human judges to justify their choice of higher-performing agent, which we manually vetted to filter out low-quality answers. 
We found that asking for these responses helped substantially with filtering out low-quality answers.
We recommend utilizing this setup in future competitions that include human evaluations.

Compared to last year, we saw an increase in the performance of the submitted algorithms. 
Meanwhile, we also saw an increase in informative answers, where evaluators chose one of the two pairwise videos as better instead of answering ``draw". 
In 2021, we had a ``draw" rate of 80\%~\citep{shah2022retrospective}, while this year we had a 27-44\% ``draw" rate, depending on the task (see \Cref{app:evaluation_results}). 
\smchange{However, we note that this difference may be attributed to the change in crowd-sourcing platform from contract workers to MTurk. 
Regardless, this finding affirms the utility of this setup for assessing agents.
}



%\paragraph{Organizational Details}
%\kr{(1. Evaluation late again (AK: could you give some color here pls?). 2. (maybe) Young Minecrafters excited about AI flooding discord - more signposting, beginner content/videos would help save organizer time.)}
%\smnote{we could consider cutting this part. depending on the benchmark solution, could cite this discussion as motivation for automated eval and reward modeling.}

%\subsection{Participation}
%\kr{(possibly move to Winning Solutions part?) The competition this year had significantly more teams competing, a higher number of individuals and nearly double the submissions. Non-lowest score submissions stayed roughly the same however. This could indicate that most of the increased interest came from less experienced teams, that had a lower conversion rate to full participation. The number of submissions was increased by the addition of the introductory track and possibly due to the lower reliability of the submission system which led to more re-submissions (AK: was it less reliable?). Full participation details can be seen in Table \ref{tab:basalt_participation}.}

%\kr{The higher interest can also be seen in the MineRL Discord server statistics in Table \ref{tab:discord_stats}. Despite there being two competitions (Diamond and BASALT) last year, and only the less popular one being held this year, there was only a modest reduction in the community activity and new members.}

\paragraph{Intent of the Competition Compared with Evaluation Metrics}
\kr{When organizing an AI competition, one of the biggest questions is how to align the intent of the competition with the evaluation metrics. Especially in reinforcement or imitation learning competitions, there is usually a way to place high on the leaderboard without meaningfully following the intent. One way to solve this is through stringent rules on what methods are allowed, like in past MineRL Diamond competitions \citep{guss2019neurips}. The downside is the extra work required to check rule compliance and to continuously clarify the fine line between allowed and disallowed methods. The way we attempted to solve this tension was by allowing any methods but awarding extra prizes for interesting research contributions. The balance between the more objective evaluation metrics and the more subjective interestingness is one knob that could be tuned to achieve the goals of such competitions.} 

{An illustrative example of this tension is how team KAIROS decided to follow the intent of BASALT 2022.} %However, using these methods they couldn't place high on the leaderboard.)}
\smchange{Although they did not score highly on the leaderboard, they received a research reward by closely following the intent of the competition.}
\smchange{In the future, to encourage creativity of solutions that adhere more closely to the intent of the competition, we recommend providing a diverse set of baselines for participants to build off of. 
Then, participants have examples of what the competition organizers are looking for and preliminary working code to further refine. }
%\smnote{karolis, that's a great point. i'd like to think of a tactful way to discuss this without implying that the top solution is "less interesting" than others. } 
%\smnote{i think a concrete suggestion would be good here. do we think having a separate research prize is sufficient? what if instead of a single pmlr submission comp tracks were treated more like actual research conf tracks? what is the point of competitions in ml? }
%\kb{This is a good point. There is a tradeoff between solutions and creativity. The ideal solution should be creative and  potential shed light on new ideas but also impactful in the sense that it solves the problem. Maybe in the future the best way to get around this is to train simple baselines like: BC, KNN, LWL, etc  and provide these as methods on the leaderboard.}
\section{Conclusion}
\label{sec:conclusion}
\smchange{
We ran the MineRL BASALT Competition on Fine-Tuning from Human Feedback at NeurIPS 2022 to promote research on fine-tuning techniques that enable agents to accomplish tasks without crisply-defined reward functions. 
We described the competition, summarized the top solutions, and investigated the performance of the competitors. 
We believe that this competition achieved its main research goals of encouraging the development of algorithms to solve hard-to-specify tasks. 
However, we identified and provided concrete avenues for improvement in the future.
}


\acks{\smchange{
Running this competition was only possible with the help of many people and organizations.
FTX Future Fund, Microsoft, Encultured AI, and AI Journal provided financial support.
We thank our amazing advisory board: Fei Fang, Kiant\'e Brantley, Andrew Critch, Sam Devlin, and Oriol Vinyals for their advice and guidance. We thank Skylar Anastasia Ekamper and Martin Andrews for supporting other participants of the competition. 
We thank Matthew Rahtz for providing detailed feedback on a draft of this paper.
Finally, we thank AIcrowd for their help and the MTurk workers for their efforts in evaluating submissions.}}

\bibliography{bib}

\newpage
\section{Appendix for Proofs}

\paragraph{Proof of Theorem \ref{thm:main}.}

\begin{proof}
\label{proof:main}
Our proof has two steps. In Step 1, we will show that SimCLR is equivalent to minimizing the cross entropy loss defined in Eqn.~(\ref{eqn:cross-entropy}). 
In Step 2, we will show  that minimizing the cross-entropy loss 
is equivalent to spectral clustering on $\bfpi$. 
Combining the two steps together, we have proved our theorem. 

\textbf{Step 1: } SimCLR is equivalent to minimizing the cross entropy loss.

The cross-entropy loss takes expectation over 
$\bfW_\bfX\sim \mathbb{P}(\cdot ; \bfpi)$, 
which means $\bfW_\bfX$ has exactly one non-zero entry in each row $i$. By Lemma~\ref{lem:multinomial}, we know every row $i$ of $\bfW_\bfX$ is independent of other rows. Moreover, 
$\bfW_{\bfX,i}\sim \mathcal{M}(1, \bfpi_i/\sum_j \bfpi_{i,j})=\mathcal{M}(1, \bfpi_i)$, because $\bfpi_i$ itself is a probability distribution.
Similarly, we know $\bfW_\bfZ$ also has the row-independent property by sampling over $\mathbb{P}(\cdot;\bfK_\bfZ)$.
Therefore, by Lemma~\ref{lem:cross_split}, we know Eqn.~(\ref{eqn:cross-entropy}) is equivalent to:
\[
 -\sum_{i=1}^n \mathbb{E}_{\bfW_{\bfX,i}}[\log \mathbb{P}(\bfW_{\bfZ,i}=\bfW_{\bfX,i};\bfK_\bfZ)],
\]

This expression takes expectation over $\bfW_{\bfX,i}$ for the given row $i$. Notice that 
$\bfW_{\bfX,i}$ has exactly one non-zero entry, which equals $1$ (same for $\bfW_{\bfZ,i}$). 
As a result
we expand the above expression to be:
\begin{equation}
 -\sum_{i=1}^n \sum_{j\neq i} \Pr(\bfW_{\bfX,i,j}=1)\log \Pr(\bfW_{\bfZ,i,j}=1).
\label{eqn:detailed-expansion}    
\end{equation}


By Lemma~\ref{lem:multinomial}, $\Pr(\bfW_{\bfZ,i,j}=1)=\bfK_{\bfZ,i,j}/\|\bfK_{\bfZ,i}\|_1$ for $j\neq i$. Recall that $\bfK_\bfZ=(k(\bfZ_i-\bfZ_j))_{(i,j)\in[n]^2}$, which means 
$\bfK_{\bfZ,i,j}/\|\bfK_{\bfZ,i}\|_1=\frac{\exp(-\|\bfZ_i-\bfZ_j\|^2/{2\tau})}{\sum_{k\neq i}
\exp(-\|\bfZ_i-\bfZ_k\|^2/{2\tau})
}$ for $j\neq i$, when $k$ is the Gaussian kernel with variance $\tau$. 

Notice that $\bfZ_i=f(\bfX_i)$, so we know
\begin{equation}
-\log \Pr(\bfW_{\bfZ,i,j}=1)=
-\log \frac{\exp(-\|f(\bfX_i)-f(\bfX_j)\|^2/{2\tau})}{\sum_{k\neq i}
\exp(-\|f(\bfX_i)-f(\bfX_k)\|^2/{2\tau}),
}
\label{eqn:infonce-equivalence}    
\end{equation}


The right hand side is exactly the InfoNCE loss defined in Eqn.~(\ref{eqn:infonce}).
Inserting Eqn.~(\ref{eqn:infonce-equivalence}) into Eqn.~(\ref{eqn:detailed-expansion}), we get the SimCLR algorithm, which first samples augmentation pairs $(i,j)$ with $\Pr(\bfW_{\bfX,i,j}=1)$ for each row $i$, and then optimize the InfoNCE loss. 

\textbf{Step 2: } minimizing the cross entropy loss 
is equivalent to spectral clustering on $\bfpi$.


By Lemma~\ref{lem:convert_to_spectral}, we may further convert the loss to 
\begin{equation}
\label{eqn:main-theorem-repul-attr}
\min_{\bfZ}
-\sum_{(i,j)\in [n]^2} \mathbf{P}_{i,j}
\log k (\bfZ_i-\bfZ_j)+\log \mathbf{R}(\bfZ).
\end{equation}
Since $k$ is the Gaussian kernel, this reduces to \[
\min_\bfZ \mathrm{tr}(\bfZ^\top \mathbf{L}(\bfpi) \bfZ)
+\log \mathbf{R}(\bfZ),
\]

where we use the fact that $\mathbb{E}_{\bfW_\bfX\sim \mathbb{P}(\cdot; \bfpi)}[\mathbf{L}(\bfW_\bfX)]
=\mathbf{L}(\bfpi)
$, because the Laplacian operator is linear and $
\mathbb{E}_{\bfW_\bfX\sim \mathbb{P}(\cdot; \bfpi)}(\bfW_\bfX)=\bfpi
$.
\end{proof}

\paragraph{Proof of Theorem \ref{thm:clip}.}
\begin{proof}
Since $\bfW_\bfX\sim \mathbb{P}(\cdot;\bfpi_{\mathbf{A}, \mathbf{B}})$, we know 
$\bfW_\bfX$ has exactly one non-zero entry in each row, denoting the pair that got sampled. 
A notable difference compared to the previous proof is we now have $n_\mathcal{A}+n_\mathcal{B}$ objects in our graph. CLIP deals with this by taking a mini-batch of size $2N$, 
such that $n_\mathcal{A}=n_\mathcal{B}=N$, and adding the $2N$ InfoNCE losses together. We label the objects in $\mathcal{A}$ as $[n_\mathcal{A}]$, and the objects in $\mathcal{B}$ as $\{n_\mathcal{A}+1, \cdots, n_\mathcal{A}+n_\mathcal{B}\}$. 

Notice that $\bfpi_{\mathbf{A}, \mathbf{B}}$ is a bipartite graph, so the edges of objects in $\mathcal{A}$ will only connect to object in $\mathcal{B}$ and vice versa. We can define the similarity matrix in $\cZ$ as $\bfK_\bfZ$, 
where $\bfK_\bfZ(i, j+n_\mathcal{A})=\bfK_\bfZ(j+n_\mathcal{A},i)= k(\bfZ_i-\bfZ_j)$ for $i\in [n_\mathcal{A}], j\in [n_\mathcal{B}]$, and otherwise we set $\bfK_\bfZ(i,j)=0$. 
The rest is same as the previous proof. 
\end{proof}

\paragraph{Proof of Theorem \ref{thm:exponential}.}

\begin{proof}
\label{proof:exponential}
Since the objective function consists of a linear term combined with an entropy regularization, which is a strongly concave function, the maximization problem is a convex optimization problem. Owing to the implicit constraints provided by the entropy function, the problem is equivalent to having only the equality constraint. We then introduce the Lagrangian multiplier $\lambda$ and obtain the following relaxed problem:

$$
\widetilde{E}(\boldsymbol{\alpha})=\psi_{1}-\sum_{i=1}^n \alpha_{i} \psi_{i}+\tau \sum_{i=1}^n \alpha_{i}\log \alpha_{i}+\lambda\left(\boldsymbol{\alpha}^{\top} \mathbf{1}_n-1\right).
$$

As the relaxed problem is unconstrained, taking the derivative with respect to $\alpha_{i}$ yields

$$
\frac{\partial \widetilde{E}(\boldsymbol{\alpha})}{\partial \alpha_{i}}=-\psi_{i}+\tau\left(\log \alpha_{i}+\alpha_{i} \frac{1}{\alpha_{i}}\right)+\lambda=0.
$$

Solving the above equation implies that $\alpha_{i}$ takes the form
$
\alpha_{i}=\exp \left(\frac{1}{\tau} \psi_{i}\right) \exp \left(\frac{-\lambda}{\tau}-1\right).
$ Since $\alpha_{i}$ lies on the probability simplex, the optimal $\alpha_{i}$ is explicitly given by
$
\alpha^{*}_{i}=\frac{\exp \left(\frac{1}{\tau} \psi_{i}\right)}{\sum_{i^{\prime}=1}^n \exp \left(\frac{1}{\tau} \psi_{i^{\prime}}\right)} .
$ Substituting the optimal point into the objective function, we obtain
$$
\begin{aligned}
E\left(\boldsymbol{\alpha}^*\right)  &=\psi_1-\sum_{i=1}^n \frac{\exp \left(\frac{1}{\tau} \psi_{i}\right)}{\sum_{i^{\prime}=1}^n \exp \left(\frac{1}{\tau} \psi_{i^{\prime}}\right)} \psi_{i}+\tau \sum_{i=1}^n \frac{\exp \left(\frac{1}{\tau} \psi_{i}\right)}{\sum_{i^{\prime}=1}^n \exp \left(\frac{1}{\tau} \psi_{i^{\prime}}\right)}\log \frac{\exp \left(\frac{1}{\tau} \psi_{i}\right)}{\sum_{i^{\prime}=1}^n \exp \left(\frac{1}{\tau} \psi_{i^{\prime}}\right)} \\
& =\psi_1 - \tau \log \left(\sum_{i=1}^n \exp \left(\frac{1}{\tau} \psi_{i}\right)\right).
\end{aligned}
$$
Thus, the Lagrangian dual function is given by
\begin{equation*}
-E\left(\boldsymbol{\alpha}^*\right)= -\tau \log \frac{\exp \left(\frac{1}{\tau} \psi_{1}\right)}{\sum_{i=1}^n \exp \left(\frac{1}{\tau} \psi_{i}\right)}.\qedhere
\end{equation*}
\end{proof}



\section{More on Experiments} \label{section: experiment_details}

\paragraph{CIFAR-10 and CIFAR-100} CIFAR-10 ~\citep{krizhevsky2009learning} and CIFAR-100 ~\citep{krizhevsky2009learning} are well-known classic image classification datasets. Both CIFAR-10 and CIFAR-100 contain a total of 60k $32 \times 32$ labeled images of different classes, with 50k for training and 10k for testing. CIFAR-10 is similar to CIFAR-100, except there are 10 different classes in CIFAR-10 and 100 classes in CIFAR-100.

\paragraph{TinyImageNet} TinyImageNet ~\citep{le2015tiny} is a subset of ImageNet ~\citep{deng2009imagenet}. There are 200 different object classes in TinyImageNet, with 500 training images, 50 validation images, and 50 test images for each class. All the images in TinyImageNet are colored and labeled with a size of $64 \times 64$.

\textbf{Pseudo-code.} Algorithm \ref{alg:Training Procedure} presents the pseudo-code for our empirical training procedure.

\begin{algorithm}[!htbp]
\caption{Training Procedure}
\label{alg:Training Procedure}
\begin{algorithmic}[1]
\REQUIRE trainable encoder network $f$, batch size $N$, augmentation strategy \textit{aug}, loss function $L$ with hyperparameters \textit{args}
\FOR {sampled minibatch ${x_i}_{i=1}^N$}
\FORALL{$i \in { 1, ..., N }$}
\STATE draw two augmentations $t_i = \textit{aug}\left(x_i\right) $, $t_i' = \textit{aug}\left(x_i\right) $
\STATE $z_i = f\left(t_i\right)$, $z_i' = f\left(t_i'\right)$
\ENDFOR
\STATE compute loss $\mathcal{L} = L(N, z, z', \textit{args})$
\STATE update encoder network $f$ to minimize $\mathcal{L}$
\ENDFOR
\STATE \textbf{Return} encoder network $f$
\end{algorithmic}
\end{algorithm}

We also provide the pseudo-code for our core loss function used in the training procedure in Algorithm \ref{alg:Core loss}. The pseudo-code is almost identical to SimCLR's loss function, with the exception of an extra parameter $\gamma$.

\begin{algorithm}[!htbp]
\caption{Core loss function $\mathcal{C}$}
\label{alg:Core loss}
\begin{algorithmic}[1]
\REQUIRE batch size $N$, two encoded minibatches $z_1, z_2$, $\gamma$, temperature $\tau$
\STATE $z = \textit{concat}\left(z_1, z_2\right)$
\FOR {$i \in {1, ..., 2N }, j \in {1, ..., 2N}$ }
\STATE $s_{i,j} = \Vert z_i - z_j \Vert_2^{\gamma}$
\ENDFOR
\STATE \textbf{define} $l(i, j)$ \textbf{as} $l(i, j) = - \log \frac{exp\left(s_{i,j}/\tau \right)}{\sum_{k=1}^{2N} \mathbf{1}{[k \ne i]} exp\left(s{i, j} / \tau \right)} $
\STATE \textbf{Return} $\frac{1}{2N} \sum_{k=1}^N\left[l(i, i+N) + l(i+N, i)\right]$
\end{algorithmic}
\end{algorithm}

Utilizing the core loss function $\mathcal{C}$, we can define all kernel loss functions used in our experiments in Table \ref{table: loss definition}. For all $z_i \in z$ with even dimensions $n$, we define $z_{L_i} = z_i\left[0:n/2\right]$ and $z_{R_i} = z_i\left[n/2:n\right]$.

\begin{table}[ht]
\centering
\begin{tabular}{{@{}l|l@{}}}
Kernel  &  Loss function \\ \midrule
Laplacian & $\mathcal{C}\left(N, z, z', \gamma=1, \tau\right)$\\ \midrule
Sum       & $\lambda * \mathcal{C}\left(N, z, z', \gamma=1, \tau_1\right) + (1-\lambda) * \mathcal{C}\left(N, z, z', \gamma=2, \tau_2\right)$  \\ \midrule
Concatenation Sum&$\lambda * \mathcal{C}\left(N, z_L, z'_L, \gamma=1, \tau_1\right) + (1-\lambda) * \mathcal{C}\left(N, z_R, z'_R, \gamma=2, \tau_2\right)$\\ \midrule
$\gamma = 0.5$ & $\mathcal{C}\left(N, z, z', \gamma=0.5, \tau\right)$          \\ 

\end{tabular}

\caption{Definition of kernel loss functions in our experiments}
\label {table: loss definition}
\end{table}

\textbf{Baselines.} We reproduce the SimCLR algorithm using PyTorch Lightning~\citep{PytorchLightning}.

\textbf{Encoder details.}
The encoder $f$ consists of a backbone network and a projection network. We employ ResNet50~\citep{ResNet} as the backbone and a 2-layer MLP (connected by a batch normalization~\citep{ioffe2015batch} layer and a ReLU \cite{nair2010rectified} layer) with hidden dimensions 2048 and output dimensions 128 (or 256 in the concatenation kernel case).

\textbf{Encoder hyperparameter tuning.}
For each encoder training case, we randomly sample 500 hyperparameter groups (sample details are shown in Table \ref{table: Hyperparameter sample}) and train these samples simultaneously using Ray Tune ~\citep{RayTune}, with the ASHA scheduler~\citep{li2018massively}. Ultimately, the hyperparameter group that maximizes the online validation accuracy (integrated in PyTorch Lightning) within 5000 validation steps is chosen for the given encoder training case.

\begin{table}[ht]
\centering

\begin{tabular}{@{}l|l|l@{}}
\midrule
Hyperparameter  & Sample Range & Sample Strategy \\ \midrule
start learning rate & $\left[10^{-2}, 10\right]$ & log uniform \\ \midrule
$\lambda$       & $\left[0, 1\right]$ & uniform \\ \midrule
$\tau$, $\tau_1$, $\tau_2$ & $\left[0, 1\right]$ & log uniform \\ \midrule
\end{tabular}

\caption{Hyperparameters sample strategy}
\label {table: Hyperparameter sample}
\end{table}

\textbf{Encoder training.} 
We train each encoder using the LARS optimizer~\citep{LARSOptimizer}, LambdaLR Scheduler in PyTorch, momentum 0.9, weight decay $10^{-6}$, batch size 256, and the aforementioned hyperparameters for 400 epochs on a single A-100 GPU.

\textbf{Image transformation.} The image transformation strategy, including augmentation, is identical to the default transformation strategy provided by PyTorch Lightning.

\textbf{Linear evaluation.}
The linear head is trained using the SGD optimizer with a cosine learning rate scheduler, batch size 64, and weight decay $10^{-6}$ for 100 epochs. The learning rate starts at $0.3$ and ends at $0$.

\textbf{Moco Experiments.} We also tested our method based on MoCo~\citep{he2019moco}. The results are summarized in Table \ref{tab:results-moco}. Here we choose ResNet18~\citep{ResNet} as the backbone and set a temperature of $0.1$ as default. For our simple sum kernel, we set $\lambda=0.8$. The results show that our method outperforms the original MoCo method.

\begin{table}[thb]
\centering
\caption{MoCo Experiment Results on CIFAR-10 and CIFAR-100.}
\label{tab:results-moco}
\resizebox{\textwidth}{!}{%
\begin{tabular}{@{}c|ccc|ccc@{}}
\toprule
\multirow{3}{*}{Method} & \multicolumn{3}{c|}{CIFAR-10} & \multicolumn{3}{c}{CIFAR-100} \\ \cmidrule(lr){2-4} \cmidrule(lr){5-7} 
                        & 200 epochs & 400 epochs    & 1000 epochs   & 200 epochs & 400 epochs & 1000 epochs         \\ \midrule
MoCo (repro.)         & $76.41 \pm 0.12$    & $80.01 \pm 0.15$          & $84.45 \pm 0.08$    & $\mathbf{47.02 \pm 0.11}$ & $52.50 \pm 0.07$ & $57.62 \pm 0.15$            \\
\midrule
Laplacian Kernel        & ${78.09 \pm 0.10}$    & $\mathbf{83.85 \pm 0.09}$          & $\mathbf{88.34 \pm 0.16}$    & $46.12 \pm 0.22$   & $53.44 \pm 0.17$ & $59.10 \pm 0.14$        \\
Simple Sum Kernel & $\mathbf{78.12 \pm 0.15}$   & $83.23 \pm 0.18$ & $87.50 \pm 0.20$ & $46.65 \pm 0.06$ & $\mathbf{53.62 \pm 0.19}$ & $\mathbf{59.83 \pm 0.12}$\\
\bottomrule
\end{tabular}
}
\end{table}



\section{More Experiments on Synthetic Data}


Consider a scenario with $n$ clusters, each containing $k$ vertices. Let the probability of vertices $u$ and $v$ from the same cluster belonging to $\bfpi$ be $p$. Conversely, for vertices $u$ and $v$ from different clusters, let the probability of belonging to $\pi$ be $q$. We generate the graph $\bfpi$ randomly, based on $p$ and $q$. We experiment with values of $k=100$ and $n=6$ for ease of visualization, embedding all points in a two-dimensional space. Each vertex's initial position originates from a normal distribution. In each iteration, we sample a subgraph of $\bfpi$ uniformly, ensuring each vertex has an out-degree of $1$. We then optimize the corresponding vectors using InfoNCE loss with an SGD optimizer and iterate until convergence. Our experimental setup consists of an SGD learning rate of $1$, an InfoNCE loss temperature of $0.5$, and a batch size of $50$. We evaluate two scenarios with different $p$ and $q$ values: $p=1$, $q=0$, and $p=0.75$, $q=0.2$. The results of these experiments are visualized in Figure \ref{fig:vis-spectral-cluster}. The obtained embeddings exhibit the hallmark pattern of spectral clustering of graph $\bfpi$.

\begin{figure}[!tb]
\centering
\subfigure{
\includegraphics[width=1\textwidth]{Figures/cluster_pi.png}
\label{fig:vis-cluster}
}
\subfigure{
\includegraphics[width=1\textwidth]{Figures/noised_cluster_pi.png}
\label{fig:vis-noised-cluster}
}
\caption{Visualizations of the optimization process using InfoNCE Loss on the vectors corresponding to $\bfpi$. Points of identical color belong to the same cluster within $\bfpi$. To showcase the internal structure of $\bfpi$, we randomly select 10 vertices from each cluster to display the edge distribution of $\bfpi$.}
\label{fig:vis-spectral-cluster}
\end{figure}



\end{document}