We now turn to the upper bound on correlations for $H$.

\begin{prop}
\label{prop:correlationsHupperBound}
Correlations $C_n(\varphi,\psi,H,\mu)$ for $H$ satisfy 
\begin{equation}
\label{eq:HcorrelationUpperBound}
|C_n(\varphi,\psi,H,\mu) | \leq c \, n^{-1} 
\end{equation}
for some constant $c$ and H\"older observables $\varphi,\psi$.
\end{prop}

We follow the approach outlined in \cite{chernov_improved_2008} to infer the polynomial mixing rate of $H$ from the exponential mixing rate of $H_\sigma$. For reference, their $\mathcal{M}$ and $M \subset \mathcal{M}$ are our $\tor$ and $\sigma \subset \tor$, their $\mathcal{F}: \mathcal{M} \to \mathcal{M}$ and $F:M \to M$ are our $H$ and $H_\sigma$ respectively. With $\Delta_0$ above, define
\[ A_n = \{ z \in \tor \, | \, R(z,H,\Delta_0 > n) \}. \]
We will show
\begin{prop}
\label{prop:AnPolynomial}
$\mu(A_n) = \mathcal{O}(n^{-1})$. 
\end{prop}
Proposition \ref{prop:correlationsHupperBound} then follows from the work of \cite{young_recurrence_1999}. Proving Proposition \ref{prop:AnPolynomial} involves treating separately a set of infrequently returning points, a method due to \cite{markarian_billiards_2004}. For each $z \in \tor$ and $n \geq 1$ define
\[ r(z;n,\sigma) = \# \{ 1 \leq i \leq n \, | \, H^i(z) \in \sigma  \},  \]
counting the number of times the orbit of $z$ hits $\sigma$ over $n$ iterates of $H$. Define 
\[B_{n,b} = \{ z \in \tor \, | \, r(z;n,\sigma) > b \ln n \} \]
where $b$ is a constant to be chosen shortly.
\begin{lemma}
\label{lemma:AncapBn}
$\mu(A_n \cap B_{n,b}) = \mathcal{O}(n^{-1})$.
\end{lemma}
\begin{proof}
This follows from (\ref{eq:delta0tailbound}), choosing $b$ large enough so that $n \, \theta^{b \ln n} < n^{-1}$. See \cite{chernov_improved_2008}, or \cite{springham_polynomial_2014} for a detailed proof.
\end{proof}
Proposition \ref{prop:AnPolynomial} then follows from similarly establishing
\begin{lemma}
\label{lemma:AnlessBn}
$\mu(A_n \setminus B_{n,b}) = \mathcal{O}(n^{-1})$.
\end{lemma}
Analysis of the set $A_n \setminus B_{n,b}$ is the focus of \cite{chernov_improved_2008}. It consists, for large $n$, of points which take many iterates to hit $\Delta_0$ and hit $\sigma$ infrequently during these iterates. \citeauthor{chernov_improved_2008} define $m$-cells
\[ M_m = \{ z \in \sigma \, | \, R(z;H,\sigma) = m+1 \}  \]
for $m \geq 0$. For the OTM the coloured regions of Figure \ref{fig:returnTime1} form $M_0$ and for $m >0$ each set $M_m$ is the union $\cup_{i,j} A_{i,j}^m$. For these latter sets, the authors assume that their measures decrease polynomially
\begin{equation}
\label{eq:muMm}
    \mu(M_m) \leq C_1 / m^r,
\end{equation}
where $r \geq 3$. Further they assume that if $z \in M_m$ then $F(z) \in M_k$ with
\begin{equation}
    \label{eq:betaCondition}
    \beta^{-1} m -C_2 \leq k \leq \beta m +C_3
\end{equation}
for some $\beta >1$ and unimportant constants $C_i>0$. It is straightforward to verify that (\ref{eq:muMm}) holds for $r=3$: each $A_{i,j}^m$ has length similar to $|\mathcal{L}_m| = \mathcal{O}(m^{-1})$, see (\ref{eq:Lklength}), and width similar to $|W_m| = \mathcal{O}(m^{-2})$, see (\ref{eq:Wklength}). Recalling the bounds $l_0$ and $l_1$ found in the proof of Proposition \ref{prop:upperTwoStep}, we see that for our map $H_\sigma$, condition (\ref{eq:betaCondition}) holds with $\beta=7$. In §4 of \cite{chernov_improved_2008} the authors describe an `ideal situation' under which the action of $F$ on the cells $M_m$ is equivalent to a discrete Markov chain. This requires:
\begin{enumerate}[label={(I\arabic*)}]
    \item The components of each $M_m$ and their images under $F$ are exact trapezoids which shrink homotetically as $m$ grows,
    \item The measure $\mu$ has constant density,
    \item $F$ is linear over each component,
    \item Condition (\ref{eq:betaCondition}) holds with $C_2 = C_3 =0$ (no irregular intersections).
\end{enumerate}
These conditions, together with:
\begin{equation}
\label{eq:conditionalProp}
   \frac{ \mu(F(M_m) \cap M_k)}{\mu(F(M_m))} = C_4 \frac{m}{k^2} + \mathcal{O}\left(\frac{1}{m^2}\right)
\end{equation}
for some $C_4>0$ and $k$ satisfying (\ref{eq:betaCondition}), are sufficient to establish the lemma. The authors show that their cells admit good linear approximations and the irregular intersections are of relative measure $\mathcal{O}(1/k)$ so that (I1) and (I4) are essentially satisfied, removing some portion of negligible measure from each cell. They then go on to estimate the effect of nonlinearity and nonuniform density of $\mu$ to address (I2) and (I3), requiring a more sophisticated approach. For our system (I2) and (I3) are already satisfied by $H_\sigma$, so it remains to verify (\ref{eq:conditionalProp}) for $H_\sigma$, show that our $A_{i,j}^k$ are well approximated by exact trapezoids, and calculate the relative measure of the irregular intersections. For (\ref{eq:conditionalProp}), areas of the regular intersections can be calculated using the shoelace formula on the corner coordinates $p_{k,l}$, $\overline{p}_{k,l}$, given explicitly in the appendix, section \ref{sec:upperCalculationsAppendix} and Proposition \ref{prop:lowerTwoStep}. Our cells are near exact trapezoids; unlike the billiards systems considered in \cite{chernov_improved_2008} whose cell sides are curvilinear, our cell boundaries are linear with the sides (e.g. $\mathcal{L}_m$, $\mathcal{L}_{m-1}$) near parallel for large $m$. For the irregular intersections, (\ref{eq:conditionalProp}) still gives an upper bound on their measure and there are some constant $C_2+C_3$ of them. The total number of intersections scales with $m$ by (\ref{eq:betaCondition}) so they have negligible measure compared to $\mu(H_\sigma(M_m))$.