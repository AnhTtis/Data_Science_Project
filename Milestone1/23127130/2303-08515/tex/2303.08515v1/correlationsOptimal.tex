In this section we show that the bound (\ref{eq:HcorrelationUpperBound}) is optimal. That is, a $\mathcal{O}(1/n)$ decay rate of correlations is met for a typical choice of H\"older observables $\varphi,\psi$, proving our main theorem.

\begin{proof}[Proof of Theorem \ref{thm:mainTheoremMixingRate}]
Without loss of generality we assume $\psi$ to have zero average and estimate
\[  C_n =  \int \left(\varphi \circ H^n \right) \psi \d \mu . \]
We first follow \cite{chernov_decay_1999}, Proposition 10.1. Letting $n \in \mathbb{N}$ we have
\begin{equation*}
    \begin{split}
        \int \left(\varphi \circ H^{2n+1} \right) \psi \d \mu &= \int \left(\varphi \circ H^{2n+1} - \varphi \circ H^{2n} \right) \psi \d \mu + \int \left(\varphi \circ H^{2n} \right) \psi \d \mu \\
        & = \int \left(\phi \circ H^{2n} \right) \psi \d \mu + \int \left(\varphi \circ H^{2n} \right) \psi \d \mu
    \end{split}
\end{equation*}
where $\phi = \varphi \circ H - \varphi$. The observable $\phi$ is piecewise H\"older continuous (H\"older over each of the four connected components $A_j$) so correlations of $H$ follow from those of $H^2$.

Recall the partition of points $z \in A_3$ into sets $A^k$ of constant escape time $k = \min \{ i \geq 1 \, | \, H^i(z) \notin A_3 \}$, sketched in Figure \ref{fig:escapeTimes}. For $k\geq 2$, each $A^k$ consists of four connected components, pairs of which limit onto the left and right sides of the invariant segments $l_1,l_2$ as $k \to \infty$. We restrict our attention to the collection $\mathrm{A}^k \subset A^k$ which limits onto the left hand side of the segment $l_2$. For reference, intersecting with $\sigma$ gives the sets $A_{4,3}^k = \mathrm{A}^k \cap \sigma$ studied in Figure \ref{fig:sigma1b}. The lines $\mathcal{L}_k$ and $\mathcal{L}_{k-1}$ from ($\ddag$) similarly bound each $\mathrm{A}^k$. We restrict to the annulus $Q = \left\{ (x,y) \in \tor \, |  \, 1/4 \leq x \leq 1/2 \right\}$ and take unions
\[ \tau_k = \bigcup_{i \geq k} \mathrm{A}^i \cap Q. \]
The sets $\tau_k$ are triangular wedges bounded by the lines $x= 1/4$, $l_2$, and $\mathcal{L}_{k-1}$: $y= \frac{(4k-2)x + k+ 1}{4k}$.

We claim that for even $k$,
\begin{equation}
\label{eq:taukcontribution}
    \int_{\tau_k} \left( \varphi \circ H^k \right) \psi \d \mu \sim \frac{C}{k}
\end{equation}
for some constant $C$. Our argument is roughly analogous to the argument presented in \cite{sturman_rate_2013}. For the linked twist map example studied there, the map leaves one side of a wedge invariant and simply shears the rest horizontally or vertically. For the OTM, we similarly have invariance of the side $l_2$ but $H^k$ is instead only conjugate to a horizontal shear. In particular, for even $k$ and $z \in \tau_k$ we have that $H^k(z) = \mathcal{G} \circ \mathcal{F}^{2k} \circ \mathcal{G}^{-1}(z)$, where $\mathcal{F}(x,y) = (x+y,y)$ mod 1 and $\mathcal{G}(x,y) = (x,y+x)$ mod 1 define the horizontal and vertical shears of the Arnold Cat Map \cite{arnold_ergodic_1968}. Hence
\begin{equation*}
    \begin{split}
    \int_{\tau_k} \left( \varphi \circ H^k \right) \psi \d \mu &= \int_{\tau_k} \left( \varphi \circ \mathcal{G} \circ \mathcal{F}^{2k} \circ \mathcal{G}^{-1} \right) \psi \d \mu \\
    & = \int_{\mathcal{G}^{-1}\tau_k} \left( \varphi \circ \mathcal{G} \circ \mathcal{F}^{2k} \right) (\psi \circ \mathcal{G}) \d \mu \\
    & = \int_{\mathcal{G}^{-1}\tau_k} \left( \tilde{\varphi} \circ \mathcal{F}^{2k} \right) \tilde{\psi} \d \mu \\
    \end{split}
\end{equation*}
where we have used the fact that $\mathcal{G}$ preserves $\mu$ and defined $\tilde{\varphi} := \varphi \circ \mathcal{G}$, $\tilde{\psi} := \psi \circ \mathcal{G}$. Noting that for even $k$ the map $\mathcal{F}^{2k}$ commutes with the phase shift $V(x,y) = (x,y+1/4)$ mod 1, a further substitution yields
\[ \int_{\tau_k} \left( \varphi \circ H^k \right) \psi \d \mu = \int_{V^{-1}\mathcal{G}^{-1}\tau_k} \left( \hat{\varphi} \circ \mathcal{F}^{2k} \right) \hat{\psi} \d \mu  \]
where $\hat{\varphi} := \tilde{\varphi} \circ V$, $\hat{\psi} := \tilde{\psi} \circ V$. The domain of integration $R_k = V^{-1}\mathcal{G}^{-1}\tau_k$ is the right triangle with sides on $x=1/4$, $y=0$, and $V^{-1}\mathcal{G}^{-1} \mathcal{L}_{k-1}: y= 1/(4k) - x/(2k)$. Hence (\ref{eq:taukcontribution}) is given by
\[ I_k := \int_{\frac{1}{4}}^{\frac{1}{2}} \int_0^{\frac{1}{4k} - \frac{x}{2k}} \hat{\varphi} (x+2ky,y) \, \hat{\psi}(x,y) \d y \d x.\]
Mirroring \cite{sturman_rate_2013}, we define a similar integral
\[ J_k := \int_{\frac{1}{4}}^{\frac{1}{2}} \int_0^{\frac{1}{4k} - \frac{x}{2k}} \hat{\varphi} (x+2ky,0) \, \hat{\psi}(x,0) \d y \d x\] 
and note that by a change in coordinates $t = x + 2ky$ we have
\[ J_k =  \frac{1}{2k} \int_{\frac{1}{4}}^{\frac{1}{2}} \int_x^{\frac{1}{2}} \hat{\varphi} (t,0) \, \hat{\psi}(x,0) \d t \d x. \]
Clearly $J_k \sim c_1/k$ for some constant $c_1$, noting that the double integral is just some constant depending on $\varphi$ and $\psi$, non-zero for any typical choice of these observables. Relation (\ref{eq:taukcontribution}) then follows from establishing $\lim_{k \to \infty} k | I_k - J_k | = 0$, i.e. these sequences are asymptotically close. Indeed, noting that $\mu(R_k) = \mu(\tau_k) \leq c_2/k$, we have
\begin{equation*}
    \begin{split}
        k | I_k - J_k | &= k \left| \int_{R_k} \hat{\varphi} (x+2ky,y) \, \hat{\psi}(x,y) - \hat{\varphi} (x+2ky,0) \, \hat{\psi}(x,0) \d y \d x \right| \\
        & \leq c_2 \sup_{(x,y) \in R_k} \left| \hat{\varphi} (x+2ky,y) \, \hat{\psi}(x,y) - \hat{\varphi} (x+2ky,0) \, \hat{\psi}(x,0) \right| \to 0
    \end{split}
\end{equation*}
since
\begin{equation*}
    \begin{split}
        | \hat{\varphi} (x+2ky,y) \, \hat{\psi}(x,y) - \hat{\varphi} (x+2ky,0) \, \hat{\psi}(x,0) | &= | \hat{\varphi} (t,y) \, \hat{\psi}(x,y) - \hat{\varphi} (t,y) \, \hat{\psi}(x,0) +  \hat{\varphi} (t,y) \, \hat{\psi}(x,0)   - \hat{\varphi} (t,0) \, \hat{\psi}(x,0) | \\
        & = | \hat{\varphi} (t,y)| |\hat{\psi}(x,y) - \hat{\psi}(x,0)| + |\hat{\psi}(x,0)| | \hat{\varphi} (t,y) -  \hat{\varphi} (t,0)| \\
        & \leq \hat{\varphi}_{\mathrm{max}} |y|^a + \hat{\psi}_{\mathrm{max}}|y|^a \\
        & \leq c_3 k^{-a} \to 0,
    \end{split}
\end{equation*}
where we have used the the H\"older property of $\hat{\varphi}, \hat{\psi}$ (inherited from $\varphi,\psi$ as the mappings $\mathcal{G},V$ are diffeomorphisms) and the fact that $|y|\leq 1/(8k)$ on $R_k$. Now by (\ref{eq:taukcontribution}) we have
\[ C_k(\varphi,\psi,H,\mu) \sim \frac{C}{k} +  \int_{\tor \setminus \tau_k} \left( \varphi \circ H^k \right) \psi \d \mu. \]
By Proposition \ref{prop:correlationsHupperBound} the contribution over $\tor \setminus \tau_k$ decays no slower than $\mathcal{O}(1/k)$ and (for a typical choice of observables $\varphi,\psi$) does not precisely cancel out the $C/k$ contribution over $\tau_k$. The $\mathcal{O}(1/k)$ law for $|C_k(\varphi,\psi,H,\mu)|$ then follows for typical H\"older observables $\varphi,\psi$.

\end{proof}