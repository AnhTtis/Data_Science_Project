\subsection{Calculations for Proposition \ref{prop:upperTwoStep}}
\label{sec:upperCalculationsAppendix}
We begin by showing the bounds (\ref{eq:Ustarj}). Note that $U_{\star,j}$ is a curve traversing $A_{4,3}^j$ near $(1/4,1)$ with tangent vectors $(v_1,v_2)^T$ in the cone $M_1 \mathcal{C}_1$ satisfying $41/17 \leq |v_2|/|v_1| \leq 17/7$. Noting that the geometry of $A_{4,3}^j$ near $(1/4,1)$ is a $180^\circ$ rotation of $A_{4,3}^j$ near $(1/4,1/2)$ and the cone is invariant under this rotation, we can follow an analogous argument to (\ref{eq:Wdiamond}) to calculate $a_\star$. In particular $|U_{\star,j}|$ is bounded below by the length of the segment passing through $r_4$ with gradient $17/7$, which gives $a_\star = 13\sqrt{2}/80$ as required. For the upper bound, the height of $|U_{\star,j}|$ is bounded above by the height of the line segment with endpoints on $r_3(j)$ and $\mathcal{L}_{j-1}$ with gradient $41/17$. In particular
\[ \ell_v(U_{\star,j}) \leq \frac{48j^2 + 41j+29}{(2j+1)(48j+17)} - \frac{1}{2} = \frac{41}{2(96j^2+82j+17)} \sim \frac{41}{192\, j^2} \]
so that, by Lemma \ref{lemma:componentLength} and $|v_2/v_1| \geq 41/17$, $|U_{\star,j}| \geq b_\star/j^2$ with $b_\star = \frac{41}{192} \sqrt{1+17^2/41^2}$ as required.

We move onto calculating $b$ such that $|U_{k,l}| \lesssim b/l^2$. Define a $(k,l)$-cell as the intersection $H_\sigma\left(A_{4,3}^k \right) \cap A_{1,3}^l$ near the accumulation point $(1/2,3/4)$, shown as the magnified region in Figure \ref{fig:twoStepB}, the quadrilateral bounded by the lines $\mathcal{L}_l$, $\mathcal{L}_{l-1}$ (as defined in equation \ref{eq:linesLl}) on $\partial A_{1,3}^l$ and $\mathscr{L}_k$, $\mathscr{L}_{k-1}$ on $\partial H_\sigma \left(A_{4,3}^k \right)$. The explicit equation for $\mathscr{L}_k$ is given in (\ref{eq:cellLinesk}), letting us calculate the corner coordinates $p_{k,l} \in \mathscr{L}_k \cap \mathcal{L}_l$ as
\begin{equation}
    \label{eq:Pklcoords}
     p_{k,l} = \left( x_{k,l},y_{k,l} \right) =  \left(     \frac{16 k l + 7 k + 23 l + 10}{32 k l + 12 k + 44 l + 16} , \frac{12 k l + 3 k + 17 l + 4}{16 k l + 6 k + 22 l + 8}    \right).
\end{equation}
The curve $U_{k,l}$ traverses the $(k,l)$-cell with endpoints on the segments $p_{k,l}p_{k-1,l}$ and $p_{k,l-1}p_{k-1,l-1}$ and has tangent vectors in the cone $M_4M_3^k \mathcal{C}_1$. Roughly speaking, for large $k$ the vectors in this cone are essentially parallel to the cell boundaries $\mathscr{L}_k$, $\mathscr{L}_{k-1}$ with gradient approaching -3, so that $\ell_v(U_{k,l})$ is given to leading order by $y_{k,l} - y_{k,l-1} \sim \frac{3}{32} l^{-2} $. Noting that $M_4M_3^k \mathcal{C}_1 \subset \mathcal{C}_4$ for any $k$, we can bound the gradient of vectors as $|v_2/v_1| \geq 7/3$ so that by Lemma \ref{lemma:componentLength} we have $|U_{k,l}| \lesssim b/l^2$ with $b = \frac{3}{32}\sqrt{1+\frac{9}{49}}$.
A more careful calculation similar to that of $b_\star$ above gives the same bound to leading order.

\subsection{Two-step expansion near $P_1$}
\label{sec:lowerCalculationsAppendix}
We will follow similar analysis to the proof of Proposition \ref{prop:upperTwoStep} to show:
\begin{prop}
\label{prop:lowerTwoStep}
Condition (\ref{eq:oneStep}) holds for $H_\sigma^2$ when $W \cap B_\varepsilon(P_1) \neq \varnothing$ for all $\varepsilon>0$.
\end{prop}

\begin{proof}

We consider the case where $W$ lies near the accumulation point $(0,1/4)$, split by $\mathcal{S}$ into subcurves $\overline{W}_\star = W \cap A_1$ and $\overline{W}_k = W \cap A_{4,2}^k$. The image of the lower subcurve $\overline{U}_\star = H_\sigma \left(\overline{W}_\star \right)$ lies near the accumulation point $(1/2,1/4) = H(0,1/4)$ and is split by $\mathcal{S}$ into curves $\overline{U}_{\star,j} \subset A_{4,2}^j$. The image of each upper subcurve $\overline{U}_k = \overline{W}_k$ maps close to $(3/4,1/2)$ for $k$ odd, $(3/4,0)$ for $k$ even. Analysis for both of these cases is analogous, as before we take $k$ to be odd and consider the geometry of $\mathcal{S}$ near the accumulation point $(3/4,1/2)$. We calculate the corners of $A_{4,2}^k$ near $(0,1/4)$ as
\[  \overline{r}_1 = \left( 0, \frac{k}{4k-2}  \right), \quad \overline{r}_2 = \left(\frac{1}{4k-6}, \frac{k-2}{4k-6}  \right), \quad \overline{r}_3 = \left(\frac{1}{4k-2}, \frac{k-1}{4k-2}  \right), \quad \overline{r}_4 = \left(0, \frac{k+1}{4k+2}  \right).  \]
so that, using the integer valued matrix $M_4M_2^k = (-1)^k \big(\begin{smallmatrix}
  1-6k & -6k-2\\
  14k-2 & 14k+5
\end{smallmatrix}\big)$, its image $H_\sigma \left(A_{4,2}^k \right)$ is the quadrilateral with corners
\[  \overline{r}_1' = \left( \frac{3k+1}{4k-2}, \frac{2k-7}{4k-1}  \right), \quad \overline{r}_2' = \left(\frac{3k-2}{4k-6}, \frac{2k-9}{4k-6}  \right),\] \[ \overline{r}_3' = \left(\frac{3k-2}{4k-2}, \frac{k}{2k-1}  \right), \quad \overline{r}_4' = \left(\frac{3k+1}{4k+2}, \frac{k+1}{2k+1}  \right).  \]
The curve $\overline{U}_{k}$ has endpoints on the segments $\overline{r}_1'\overline{r}_2'$ and $\overline{r}_3'\overline{r}_4'$ and is split by $\mathcal{S}$ into an upper portion $\overline{U}_{k,\star}$ in $A_4$ above $y=1/2$ and subcurves $\overline{U}_{k,l} \subset A_{1,2}^l$ where $l_0 \leq l \leq l_1$. Comparison of the point $\overline{r}_2'$ with the lines $\overline{\mathcal{L}}_l: y -1/4 = -\frac{2l+1}{2l+2} (x-1)$ and $\overline{\mathcal{L}}_{l-1}$ on $\partial A_{1,2}^l$ yields $l_0(k) \geq \lfloor \frac{k-4}{7} \rfloor $, intersecting $\overline{r}_1'\overline{r}_4'$ with $y=1/2$ yields $l_1(k) \leq 7k+2$. Let $W_i = H_\sigma^{-1}\left(\overline{U}_i \right)$ then $W$ splits in an analogous fashion to (\ref{eq:Wsplitting}) with $DH_\sigma^2$ constant on each component. It follows that for $q=1/2$, 
\[ \liminf_{\delta \to 0} \sup_{W: |W|< \delta}  \sum_i \left( \frac{|W|}{|V_i|}\right)^q \frac{|W_i|}{|W|} \leq \sup_{0 \leq p \leq 1} \left( 2\sqrt \frac{(1-p)\overline{b}_\star \Lambda_1^+}{\overline{c}_\star \overline{a}_\star\Lambda_1^-} +    2\sqrt{\frac{p\overline{b}_\diamond}{\overline{c}_\diamond \overline{a} \overline{\gamma}}} +  4 \sqrt{\frac{p \overline{b} h}{\overline{c}\overline{a} \overline{\gamma}}} \right) \]
where the new constants satisfy (letting $K(M)$ denote the minimum expansion of $M$ over $\mathcal{C}_1$)
\begin{itemize}
    \item $K\left(M_4M_2^jM_1\right) \sim \overline{c}_\star j$
    \item $K\left(M_4^2M_2^k\right) \sim \overline{c}_\diamond k$
    \item $K\left(M_1M_2^lM_4M_2^k\right) \sim \overline{c} kl$
    \item $K\left(M_4M_2^k\right) \sim \overline{\gamma} k$
    \item $\overline{a}_\star /j^2 \lesssim |\overline{U}_{\star,j}| \leq \overline{b}_\star /j^2$
    \item $|\overline{U}_{k,\star}| \lesssim \overline{b}_\diamond / k$
    \item $|\overline{W}_k| \gtrsim \overline{a}/k^2$
    \item $|\overline{U}_{k,l}| \lesssim \overline{b}/l^2$
\end{itemize}
and $\Lambda_1^{\pm}$, $h$ are unchanged from (\ref{eq:justConstants}). The expansion factors can be calculated in the same fashion as (\ref{eq:Lambdastarj}), in particular
\[ \overline{c}_\star = \frac{48\sqrt{145}}{5}, \quad \overline{c}_\diamond = 8\sqrt{197},  \quad \overline{c} =64, \quad \overline{\gamma} = \frac{8\sqrt{145}}{5}.   \]
The constant $\overline{a}_\star$ is obtained by considering the shortest path across $A_{4,2}^j$ with tangent vectors aligned in the cone $M_1\mathcal{C}_1$, bounded by the length of the segment with endpoints on $r_4(j)$ and $\mathcal{L}_{j-1}$ (as defined in ($\dagger$), proof of Lemma \ref{lemma:unstableGrowth}) with gradient 41/17. The constant $\overline{b}_\star$ is obtained by considering the maximum height of a segment joining $\mathcal{L}_{j-1}$ to $\mathcal{L}_j$, given by the segment passing through $r_3(j) \in \mathcal{L}_j$ with gradient 17/17, and applying Lemma \ref{lemma:componentLength}. In particular $\overline{a}_\star = \sqrt{1970}/464$ and $\overline{b}_\star = \frac{17}{192} \sqrt{1+\frac{17^2}{41^2}}$. Similar analysis to the calculation of $\overline{a}_\star$ but using the wider cone $\mathcal{C}_1$ yields $\overline{a} = \sqrt{55}/80$. We again apply Lemma \ref{lemma:componentLength} to find $\overline{b}_\diamond$, with $\ell_v \left(\overline{U}_{k,\star} \right)$ bounded above by the height $1/(4k-2) \sim 1/(4k)$ of $r_3'(k)$ above $y=1/2$. Tangent vectors of $\overline{U}_{k,\star}$ lie in the cone $M_4M_2^k \mathcal{C}_1 \subset \mathcal{C}_4$ so that $\overline{b}_\diamond = \frac{1}{4}\sqrt{1 + 9/49}$ provides the upper bound. Finally we calculate $\overline{b}$, following a similar approach to section \ref{sec:upperCalculationsAppendix}. For each $k$ the segments $\overline{r}_1'\overline{r}_4'$ and $\overline{r}_2'\overline{r}_3'$ lie on the lines $\overline{\mathscr{L}}_k$ and $\overline{\mathscr{L}}_{k-1}$ respectively, with 
\[ \overline{\mathscr{L}}_k: y - \frac{k+1}{2k+1} = -\frac{14k+5}{6k+2} \left( x- \frac{3k+1}{4k+2} \right).\]
Define a $\overline{(k,l)}$ cell as the intersection of $H_\sigma(A_{4,2}^k) \cap A_{1,2}^l$, given the by quadrilateral bounded by the lines $\overline{\mathscr{L}}_k$, $\overline{\mathscr{L}}_{k-1}$, $\overline{\mathcal{L}}_l$, $\overline{\mathcal{L}}_{l-1}$. Its corners $\overline{p}_{k,l},\dots,\overline{p}_{k-1,l-1}$ are given by
\[ \overline{p}_{k,l} = \left( \frac{(3k+1)(2l+3)}{8kl+11k+3l+4}  ,\frac{16kl+15k+7l+6}{4(8kl+11k+3l+4)} \right)  \]
with (as before, to leading order terms for $k$ large) $\ell_v\left(\overline{U}_{k,l}\right)$ bounded above by the height of the segment joining $\overline{p}_{k-1,l}$ to $\overline{p}_{k-1,l-1}$, $\ell_v \left(\overline{p}_{k-1,l}\overline{p}_{k-1,l-1} \right) \sim 7/32 l^{-2}$. Again, tangent vectors to $\overline{U}_{k,l}$ lie in $\mathcal{C}_4$ so that $\overline{b} = \frac{7}{32}\sqrt{1 + 9/49}$ gives an upper bound $|\overline{U}_{k,l}| \lesssim \overline{b}/l^2$ by Lemma \ref{lemma:componentLength}.

As before we take
\[ \overline{s} =  2\sqrt \frac{\overline{b}_\star \Lambda_1^+}{\overline{c}_\star \overline{a}_\star\Lambda_1^-} \approx 0.186, \quad \overline{t} = 2\sqrt{\frac{\overline{b}_\diamond}{\overline{c}_\diamond \overline{a} \overline{\gamma}}} +  4 \sqrt{\frac{ \overline{b} h}{\overline{c}\overline{a} \overline{\gamma}}} \approx 0.488, \]
giving
\[ \liminf_{\delta \to 0} \sup_{W: |W|< \delta}  \sum_i \left( \frac{|W|}{|V_i|}\right)^q \frac{|W_i|}{|W|} \leq \overline{s}\sqrt{\frac{\overline{s}^2}{\overline{s}^2+\overline{t}^2}} + \overline{t}\sqrt{\frac{\overline{t}^2}{\overline{s}^2+\overline{t}^2}} \approx 0.522 < 1  \]
as required. 


\end{proof}