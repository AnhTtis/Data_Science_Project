A necessary prerequisite for applying the machinery of \cite{chernov_billiards_2005} and similar is establishing mixing with respect to the invariant measure. In hyperbolic systems possessing singularities, the following scheme of \cite{katok_invariant_1986} is useful, giving conditions for the (stronger) Bernoulli property. We paraphrase from \cite{sturman_mathematical_2006}:

\begin{thm}[\citeauthor{katok_invariant_1986}]
\label{thm:katok-strelcyn}
Let $(X,\mathcal{F},\mu,f)$ be a measure preserving dynamical system such that $f$ is $C^2$ smooth outside of a singularity set $S$. Suppose that the Katok-Strelcyn conditions hold:
\begin{enumerate}[label={\bfseries (KS\arabic*):}]
    \item There exist $a,C_1>0$ such that for all $\epsilon>0$, $\mu(B_\varepsilon(S)) \leq C_1 \varepsilon^a$.
    \item There exist $b,C_2>0$ such that for all $x \in X \setminus S$, $||D^2_xf|| \leq C_2 \,d(x,S)^{-b}$.
    \item Lyapunov exponents exist and are non-zero almost everywhere.
\end{enumerate}
Then at almost every $x$ we can define local unstable and stable manifolds $\gamma_u(x)$ and $\gamma_s(x)$. Suppose that the manifold intersection property holds:
\begin{enumerate}[label={\bfseries (M):}]
    \item For almost any $x,x'\in X$, there exist $m,n$ such that $f^m(\gamma_u(x)) \cap f^{-n}(\gamma_s(x')) \neq \varnothing$.
\end{enumerate}
Then $f$ is ergodic. Provided the repeated manifold intersection property holds:
\begin{enumerate}[label={\bfseries (MR):}]
    \item For almost any $x,x'\in X$, there exist $M,N$ such that for all $m>M$ and $n>N$, $f^m(\gamma_u(x)) \cap f^{-n}(\gamma_s(x')) \neq \varnothing$,
\end{enumerate}
the Bernoulli property follows.
\end{thm}

The nature of the constant $a$ giving \textbf{(KS1)} plays an important role in showing expansion conditions such as (\ref{eq:oldOneStepExpansion}). In systems possessing a finite number of singularity curves, see for example \cite{przytycki_ergodicity_1983,myers_hill_exponential_2022,myers_hill_family_2022}, a covering by $\varepsilon$-balls immediately gives \textbf{(KS1)} with $a=1$. Showing (\ref{eq:oldOneStepExpansion}) is then quite straightforward; the singularity set splits an unstable manifold $W$ of vanishing length $|W| \to 0$ into at most $K$ components $W_k$, where $K$ is the maximum number of singularity curves which meet at a given point. This reduces (\ref{eq:oldOneStepExpansion}) to calculating the expansion factors $\lambda_k = |f(W_k)|/|W_k|$ and verifying the finite summation $\sum_k \lambda_k^{-1} <1 $. In many systems, in particular those driven by a return map where recurrence follows a law such as (\ref{eq:E1}), singularity curves instead form a \emph{countable} family. Expansion factors $\lambda_k \sim c\,k$ are typical so that bounding the above sum is challenging, indeed it may even diverge. Such systems satisfy \textbf{(KS1)}, but only with some $a<1$. In certain $a<1$ scenarios, precise mapping behaviour may reduce (\ref{eq:oldOneStepExpansion}) to a finite summation; see for example the return map considered in \cite{springham_polynomial_2014}. Such a scenario is not typical however, with (\ref{eq:oldOneStepExpansion}) failing in many examples \cite{chernov_billiards_2005}. More recent schemes for bounds on correlations have revised (\ref{eq:oldOneStepExpansion}) to suit these more general $a < 1$ systems. We quote the first of these, given in \cite{chernov_statistical_2009}, which is sufficient for our purposes.

Let $\Omega$ denote a two dimensional connected compact Riemannian manifold, $f: \Omega \to \Omega$ preserving a measure $\mu$. Let $d$ denote the distance in $\Omega$ induced by the Riemannian metric $\rho$. For any smooth curve $W$ in $\Omega$, denote by $|W|$ its length, and by $m_W$ the Lebesgue measure on $W$ induced by the Riemannian metric $\rho_W$ restricted to $W$. Also let $\nu_W$ = $m_W /|W|$ be the normalised (probability) measure on W.

\textbf{(H1):} Hyperbolicity of $f$ (with uniform expansion and contraction). There exist two families of cones $C_x^u$ (unstable) and $C_x^s$ (stable) in the tangent spaces $\mathcal{T}_x\Omega$, for all $x \in \Omega$, and there exists a constant $\Lambda$ > 1, with the following properties:
\begin{enumerate}
    \item $Df(C_x^u) \subset C_{fx}^u$ and $Df(C_x^s) \supset C_{fx}^s$ whenever $Df$ exists.
    \item $\| D_xf(v) \| \geq \Lambda \| v \|$ for all  $v \in C_x^u$ and $\| D_xf^{-1}(v) \| \geq \Lambda \| v \|$ for all $v \in C_x^s$.
    \item These families of cones are continuous on $\Omega$ and the angle between $C_x^u$ and $C_x^s$ is uniformly bounded away from zero.
\end{enumerate}

We say that a smooth curve $W \subset \Omega$ is an unstable (stable) curve if at every point $x \in W$
the tangent line $\mathcal{T}_x W$ belongs in the unstable (stable) cone $C_x^u$ ($C_x^s$).

\textbf{(H2):} Singularities and smoothness. Let $\mathcal{S}_0$ be a closed subset in $\Omega$, such that $M := \Omega \setminus \mathcal{S}_0$ is a dense set in $\Omega$. We put $\mathcal{S}_{\pm 1} = f^\mp \mathcal{S}_0$.
\begin{enumerate}
    \item $f:M\setminus \mathcal{S}_1 \to M \setminus \mathcal{S}_{-1}$ is a $C^2$ diffeomorphism.
    \item $\mathcal{S}_0 \cup \mathcal{S}_1$ is a finite or countable union of smooth, compact curves in $\Omega$.
    \item Curves in $\mathcal{S}_0$ are transversal to stable and unstable cones. Every smooth curve in $\mathcal{S}_1$ (resp. $\mathcal{S}_{-1}$) is a stable (resp. unstable) curve. Every curve in $\mathcal{S}_1$ terminates either inside another curve of $\mathcal{S}_1$ or on $\mathcal{S}_0$.
    \item There exists $b \in (0,1)$ and $c>0$ such that for any $x\in M\setminus \mathcal{S}_1$
    \begin{equation}
        \label{eq:DxbCondition}
        \| D_xf  \| \leq c\, d(x,\mathcal{S}_1)^{-b}.
    \end{equation}
\end{enumerate}

\textbf{(H3):} Regularity of smooth unstable curves. We assume that there is a $f$-invariant class of unstable curves $W \subset M$ that are \emph{regular} (see \citealp{chernov_statistical_2009}).

\textbf{(H4):} SRB measure. $\mu$ is a Sinai-Ruelle-Bowen (SRB) measure which is mixing.

\textbf{(H5):} One-step expansion. There exists $q \in (0,1]$ such that
\begin{equation}
    \label{eq:oneStep}
    \liminf_{\delta \to 0} \sup_{W: |W|< \delta} \sum_i \left( \frac{|W|}{|f(W_i)|}\right)^q \frac{|W_i|}{|W|} < 1,
\end{equation}
where the supremum is taken over all unstable curves, $W_i$ are the components of $W$ split by the singularity set for $f$.

\begin{thm}[\citeauthor{chernov_statistical_2009}]
\label{thm:chernovZhang}
Under the conditions \textbf{(H1)}–\textbf{(H5)}, the system $(f, \mu)$ enjoys exponential decay of correlations.
\end{thm}

Note that the new one-step expansion condition (\ref{eq:oneStep}) may be reduced to the old (\ref{eq:oldOneStepExpansion}) by taking $q=1$. The new condition ensures that the images of unstable curves grow `on average'. Choosing a $q<1$ essentially permits summing over countably many components, broadening the potential applications of the scheme to a wider class of $a<1$ systems. The image coupling methods (\cite{young_recurrence_1999}, see also \cite{chernov_chaotic_2006} and the references therein) used to establish Theorem \ref{thm:chernovZhang} differ substantially from those employed in \cite{chernov_billiards_2005}. The key `magnet' construction \cite{chernov_advanced_2006,chernov_chaotic_2006}, however, further serves as the base $\Delta_0$ of a Young tower satisfying the exponential tail bound (\ref{eq:expRtailYoung}) \cite{chernov_statistical_2009}. As such the scheme may similarly be applied to some return map $f_M$ as a step towards proving polynomial decay of correlations for $f$. We conclude this section with two technical adjustments we will refer back to later in section \ref{sec:Hsigma}.

\begin{remark}
\label{remark:firstRemark}
Condition \textbf{(H1.3)} has been relaxed in subsequent schemes \cite{demers_spectral_2014,wang_decay_2021} and can be replaced by
\begin{enumerate}[label=3'.]
    \item These families of cones are continuous on components of $\Omega \setminus \mathcal{S}_0$ and the angle between $C_x^u$ and $C_x^s$ is uniformly bounded away from zero.
\end{enumerate}
Theorem \ref{thm:chernovZhang} still follows under this relaxed assumption by applying (for example) Theorem 1 of \cite{wang_decay_2021}. Despite the improvement over older growth conditions, condition \textbf{(H5)} still fails for many systems over one iterate. See, for example, the modified stadia considered in \cite{chernov_statistical_2009}. It can be replaced by a multi-step expansion condition, establishing \textbf{(H5)} for some higher power $f^n$ of the map and its enlarged singularity set.
\end{remark}



