The statistics of chaotic dynamics driven by an area-preserving map are often described by its mixing properties. Given such a map $f:X \to X$, preserving a measure $\mu$, we say that $f$ is \emph{mixing} if
its correlations $C_n(\varphi,\psi)$ decay to 0 for $L^2$ observables $\varphi,\psi:X \to \mathbb{R}$, where
\[  C_n(\varphi,\psi) =  \int \left(\varphi \circ f^n \right) \psi \d \mu - \int \varphi \d \mu \int \psi \d \mu \]
denotes the \emph{correlation function}.

Given $C_n \to 0$, the speed at which these correlations decay (the \emph{mixing rate}) further characterises a map's dynamics. We say that $f$ enjoys exponential decay of correlations if there exists constants $0<\theta<1$ and $c(\varphi,\psi)>0$ such that
\begin{equation}
    \label{eq:expMixing}
    |C_n(\varphi,\psi)| \leq c \, \theta^n.
\end{equation}
Similarly we say that $f$ enjoys polynomial decay of correlations if there exists $\alpha>0$ and $c(\varphi,\psi)>0$ such that
\begin{equation}
    \label{eq:polyMixing}
    |C_n(\varphi,\psi)| \leq c \, n^{-\alpha}.
\end{equation}
Some regularity on the observables is typically assumed; we assume H\"older continuity throughout this article. Other statistical properties (e.g. the central limit theorem) are also intimately linked this rate of decay.

Building on \cite{young_statistical_1998,young_recurrence_1999,chernov_decay_1999,markarian_billiards_2004}, \cite{chernov_billiards_2005} gives conditions under which a uniformly hyperbolic map with singularities satisfies (\ref{eq:expMixing}). These include mild restrictions on the nature of the singularities and regularity of local manifolds, alongside a \emph{one-step expansion} estimate which ensures expansion by hyperbolicity dominates the cutting by singularities:
\begin{equation}
    \label{eq:oldOneStepExpansion}
     \liminf_{\delta \to 0} \sup_{W: |W|<\delta} \sum_i \frac{|W_i|}{|f(W_i)|} < 1 
\end{equation}
where the supremum is taken over unstable manifolds $W$, of length $|W|$, split into components $W_i$ by the singularities. Key to this analysis is construction of a \emph{Young tower} \cite{young_statistical_1998}. Given a subset $A \subset X$ and $x \in A$, define the 
\emph{return time} of $x$ to $A$ under $f$ as $R(x;f,A) = \inf \{ i>0 \, | \, f^i(x) \in A  \}$. Young considers returns\footnote{In particular `good' returns which satisfy additional technical constrains, see \cite{young_statistical_1998}. A precise definition of \emph{hyperbolic product structure} is also found therein.} to some subset $\Delta_0$ (the tower base) with hyperbolic product structure, showing that if returns satisfy an exponential tail bound:
\begin{equation}
    \label{eq:expRtailYoung}
    \mu( \{ x \in \Delta_0  \, | \, R(x;f,\Delta_0) > n\}) < C \theta^n,
\end{equation}
then (\ref{eq:expMixing}) holds. Explicitly constructing $\Delta_0$ is challenging in many systems, as is estimating its recurrence, requiring all the iterates of $f$ to be considered. The scheme of \cite{chernov_billiards_2005} both avoids the explicit construction of $\Delta_0$ and reduces the analysis down to conditions such as (\ref{eq:oldOneStepExpansion}) concerning a single iterate of the map $f$.

The scheme has utility beyond uniformly hyperbolic examples. Following \cite{young_recurrence_1999}, if there exists $C>0$ and $\alpha>0$ such that
\begin{equation}
    \label{eq:polyRtailYoung}
    \mu( \{ x \in \Delta_0  \, | \, R(x;f,\Delta_0) > n\}) < C n^{-\alpha},
\end{equation}
then $f$ satisfies (\ref{eq:polyMixing}). Suppose $f:X \to X$ has suspected polynomial decay of correlations, non-uniformly hyperbolic and possessing some region $N$ where $f$ is non-hyperbolic with escape times $E(x;f,N) = \inf \{i>0 \, | \, f^i(x) \notin N \}$ satisfying
\begin{equation}
\label{eq:escapeDist}
     \mu( \{ x \in N  \, | \, E(x;f,N) > n\}) < C n^{-\alpha}.
\end{equation}
By non-uniform hyperbolicity, a.e. $x \in N$ eventually escapes and hits some region of `strong' hyperbolicity, precisely a subset $M \subset X$ with uniformly hyperbolic \emph{return map} $f_M(x) = f^R(x)$, $R=R(x;f,M)$. Using its strong hyperbolic properties to satisfy the conditions of \cite{chernov_billiards_2005}, $f_M$ then admits a Young tower with base $\Delta_0 \subset M$, satisfying
\begin{equation}
\label{eq:expTailReturnMap}
    \mu( \{ x \in M  \, | \, R(x;f_M,\Delta_0) > n\}) < C \theta^n.
\end{equation}
Extending the domain of $f_M$ to $X$ in the obvious fashion, the bound (\ref{eq:escapeDist}) suggests
\begin{equation}
\label{eq:E1}
     \mu( \{ x \in X  \, | \, R(x;f,M) > n\}) < C n^{-\alpha}, 
\end{equation}
which can be extended, making use of (\ref{eq:expTailReturnMap}), to give (\ref{eq:polyRtailYoung}). This final step is non-trivial and typically relies on utilising precise mapping behaviour of $f_M$. The above scheme has been used to establish polynomial decay of correlations for various billiards maps including certain stadia and tables with cusps \cite{chernov_billiards_2005,chernov_improved_2008}. Beyond billiards, in \cite{springham_polynomial_2014} $\mathcal{O}(1/n)$ correlation decay was shown for a family of \emph{linked twist maps} (hereafter LTMs). These are Lebesgue measure preserving continuous maps on the 2-torus $\tor$, composing monotonic shears restricted to horizontal and vertical annuli $P,Q \subsetneq \tor$. Here, mixing is slowed by orbits remaining trapped in $P \triangle Q$\footnote{The symmetric difference $P \triangle Q = (P \setminus Q) \cup (Q \setminus P)$.} for arbitrarily long periods, with recurrence to $M = P \cap Q$ satisfying the tail bound (\ref{eq:E1}). Monotonicity of the shears was important in the analysis, allowing for a straightforward proof of the mixing property.

More recently the scheme was directly applied to a family of non-monotonic toral maps \cite{myers_hill_exponential_2022}. Parameterising $\tor$ by $(x,y) \in \mathbb{R}^2 / \mathbb{Z}^2$, these maps similarly compose horizontal and vertical shears $H_{(\xi,\eta)} = G \circ F$ where
\[ F(x,y) =
\begin{cases}
\left(  x + \frac{y}{1-\eta}, y  \right) \text{ mod 1 } & \text{ for } y \leq 1-\eta, \\

\left(  x + \frac{1-y}{\eta}, y  \right) \text{ mod 1 } & \text{ for } y \geq 1-\eta, \\
\end{cases}
\quad
G(x,y) =
\begin{cases}
\left(  x  , y + \frac{x}{1-\xi}  \right) \text{ mod 1 } & \text{ for } x \leq 1-\xi, \\

\left(  x  , y + \frac{1-x}{\xi}  \right) \text{ mod 1 } & \text{ for } x \geq 1-\xi, \\
\end{cases}\]
and $0<\xi,\eta<1$. Exponential mixing rates were established over a wide neighbourhood of $(\xi,\eta) = (0,0)$, with boundary determined by (\ref{eq:oldOneStepExpansion}), and are expected over $1-\frac{1}{4 \xi} < \eta < \frac{1}{4-4 \xi}$ where $H_{(\xi,\eta)}$ is uniformly hyperbolic (with singularities). This includes the parameter subspace $\eta=\xi$ corresponding to matching $F$ and $G$, excluding the cusp $\xi = \eta = 1/2$ where $F$ and $G$ are symmetric tent maps. This cusp is notable in the transverse subspace $\eta = 1- \xi$ also, being the only parameters for which elliptic islands do not form. Following \cite{myers_hill_exponential_2022}, we refer to $H_{(\xi,\eta)}$ at these precise parameters as the \emph{orthogonal tents map} (OTM) and denote it simply by $H$. 

\begin{figure}
    \centering
    \begin{tikzpicture}
    \definecolor{white}{RGB}{255,255,255}
    \definecolor{tomato}{RGB}{255, 99, 71}
    \definecolor{teal}{RGB}{95, 158, 160} 
    \node at (-5,0) {
    \begin{tikzpicture}[scale=0.4]
    
    
    \draw[ultra thick, tomato] (0,7.5) -- (2.5,10);
    \node[text=tomato] at (1.5,8.2) {$l_1$};
    \draw[ultra thick, gray] (2.5,5) -- (5,7.5);
    \node[text=gray] at (3.5,6.8) {$l_2$};
    \draw[ultra thick, teal] (5,2.5) -- (7.5,0);
    \node[text=teal] at (7,1.5) {$l_3$};
    \draw[ultra thick, black] (7.5,5) -- (10,2.5);
    \node at (8,3.5) {$l_4$};
    
    \draw (0,0) rectangle (10,10);
    \end{tikzpicture}
    };
    \node at (0,0) {
    \begin{tikzpicture}[scale=0.4]
    
    \draw[ultra thick, tomato] (5,7.5) -- (2.5,10);
    \draw[ultra thick, gray] (2.5,5) -- (0,7.5);
    \draw[ultra thick, teal] (10,2.5) -- (7.5,0);
    \draw[ultra thick, black] (7.5,5) -- (5,2.5);
    
    
    \draw (0,0) rectangle (10,10);
    
    
    \end{tikzpicture}
    };
    \node at (5,0) {
    \begin{tikzpicture}[scale=0.4]
    \draw[ultra thick, gray] (0,7.5) -- (2.5,10);
    \draw[ultra thick, tomato] (2.5,5) -- (5,7.5);
    \draw[ultra thick, black] (5,2.5) -- (7.5,0);
    \draw[ultra thick, teal] (7.5,5) -- (10,2.5);
    
    
    \draw (0,0) rectangle (10,10);
    \end{tikzpicture}
    };
    
\draw[->, thick] (-2.7,0) -- (-2.2,0); 
    \node[scale=1.5] at (-2.45,0.4) {$F$};
    \draw[->, thick] (2.2,0) -- (2.7,0); 
    \node[scale=1.5] at (2.45,0.4) {$G$};
    
    \end{tikzpicture}
    \caption{Line segments $l_j$ satisfying $H:l_1 \leftrightarrow l_2$, $l_3  \leftrightarrow l_4$. Each are periodic with period 2, their union is invariant under $H$.}
    \label{fig:periodic}
\end{figure}

Other authors \cite{cheng_numerical_2023} have recognised the interest of this map, including it as part of a wider fundamental class of alternating wedge flows. We claim in particular it serves as a prototype example of piecewise linear maps which sit on the boundary of ergodicity. Limiting onto $H$ from its non-ergodic perturbations, the nature of the periodic orbits seeding the islands changes from elliptic to parabolic. Provided such an orbit does not limit onto a singularity line, its surrounding islands shrink, leaving behind periodic line segments ($H$ possesses four, sketched in Figure \ref{fig:periodic}) of null measure. This permits mixing with respect to Lebesgue but, as observed in \cite{cheng_numerical_2023}, only at a reduced polynomial rate for we can find orbits which `stick' to the segment for arbitrarily long periods. Here we show that $H$ mixes no slower than this. Our main theorem is the following:

\begin{thm}
    \label{thm:mainTheoremMixingRate}
    Correlations for $H$ decay as $|C_n(\varphi,\psi) | = \mathcal{O}(1/n)$ for H\"older observables $\varphi,\psi$.
\end{thm}

We expect a similar law to hold for piecewise linear systems obeying the limiting behaviour described above, for example the pointwise limit of $H_{(\xi,\eta)}$ as $\xi \to 0$ at $\eta=1/3$ (see \citealp{myers_hill_family_2022}). We focus on $H$ in particular for two key reasons. Firstly, as a fundamental piecewise linear model of alternating shear flows where no-slip boundary conditions force non-monotonic shear profiles, it is of interest to (laminar) fluid mixing applications \cite{cerbelli_continuous_2005}. Indeed, it is the logical extension to \citeauthor{cerbelli_continuous_2005}'s map, incorporating non-monotonicity into both the horizontal and vertical shears. Questions surrounding the mechanism by which $H$ is mixing, but at a reduced rate, are natural in this setting and are answered conclusively by a proof of Theorem \ref{thm:mainTheoremMixingRate}. Secondly, the map possesses certain properties which speed up its analysis. Since both $1/\eta$ and $1/(1-\eta)$ are integer valued over $0<\eta<1$ if and only if $\eta=1/2$, $H$ is the only map in the $0<\xi,\eta<1$ parameter space with all integer valued Jacobians and can be expressed as $H(x,y) = DH \cdot (x,y)^T$ mod 1\footnote{This also implies that periodic orbits are dense on $\tor$, as the cardinality of any orbit containing a rational point $(s/q,p/q) \in \tor$ with $s,p,q \in \mathbb{N}$ is bounded above by $q^2$ \cite{cerbelli_continuous_2005}}. This will prove useful for tracking the orbits of certain points under large powers of $H$. In addition $H$ can be related to its inverse by a conjugacy and behaves symmetrically on certain regions, reducing the calculations required to establish growth conditions by a factor of four.

The following sections are organised as follows. In section \ref{sec:outline} we state two theorems from the billiards literature that we rely upon to establish Theorem \ref{thm:mainTheoremMixingRate}. We next prove hyperbolicity for $H$ in section \ref{sec:OTMhyp} and the mixing property in section \ref{sec:Hmixing}. Central to this analysis is recurrence to a set $\sigma$ with the return map $H_\sigma$ exhibiting strong hyperbolic properties. We establish more formal properties of the map $H_\sigma$ in section \ref{sec:Hsigma}, sufficient to establish exponential decay of correlations. We use this to infer a polynomial bound on correlations for $H$ in section \ref{sec:polyMixingRate}, proving Theorem \ref{thm:mainTheoremMixingRate}. Finally in section \ref{sec:discussion} we comment on the relevance of our work to similar systems and suggest possible extensions.




