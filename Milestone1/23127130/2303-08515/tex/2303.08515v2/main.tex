\documentclass{article}
\usepackage[utf8]{inputenc}
%\usepackage[round, sort]{natbib}

\newcommand{\rob}[1]{\textcolor{blue}{\bf rob: #1}}
\newcommand{\joe}[1]{\textcolor{red}{\bf joe: #1}}

\usepackage[style=alphabetic,natbib,giveninits=true,doi=false,isbn=false,url=false,eprint=false]{biblatex}
\AtEveryBibitem{\clearlist{language}}
\AtEveryBibitem{\clearfield{pages}} 
\addbibresource{refs.bib}


\usepackage{soul}
\usepackage{amsmath}
\usepackage{amssymb}
\usepackage{amsthm}
\usepackage{mathtools}

\usepackage{todonotes}


\usepackage{pgf,tikz,pgfplots}

\usepackage{tikz-cd}
\usetikzlibrary{patterns}
\usetikzlibrary{spy}
\usetikzlibrary{arrows,decorations.markings}
\usetikzlibrary{math}

\usepackage{dirtytalk}


\usepackage[mathscr]{euscript}
\usepackage{enumerate}
\usepackage[shortlabels]{enumitem}

\usepackage{subfigure}

\usepackage[margin=1in]{geometry}

\usepackage{xcolor}

\usepackage{setspace}
\doublespacing

\theoremstyle{plain}
\newtheorem{thm}{Theorem}

\theoremstyle{plain}
\newtheorem{lemma}{Lemma}

\theoremstyle{plain}
\newtheorem{prop}{Proposition}

\theoremstyle{plain}
\newtheorem{cor}{Corollary}

\theoremstyle{definition}
\newtheorem{definition}{Definition}

\theoremstyle{definition}
\newtheorem{remark}{Remark}

\usepackage[T2A,T1]{fontenc}
\newcommand{\CD}{\mbox{\usefont{T2A}{\rmdefault}{m}{n}\CYRD}}

\usepackage[symbol]{footmisc}
\renewcommand{\thefootnote}{\fnsymbol{footnote}}

%\usepackage{fourier} 
\usepackage{array}
\usepackage{makecell}

\renewcommand\theadalign{bl}
\renewcommand\theadfont{\bfseries}
\renewcommand\theadgape{\Gape[4pt]}
\renewcommand\cellgape{\Gape[4pt]}
\renewcommand{\cellalign}{l}

\renewcommand{\d}{\,\mathrm{d}}
\newcommand{\limn}{\lim_{n\rightarrow \infty}}
\newcommand{\limsupn}{\limsup_{n\rightarrow \infty}}
\newcommand{\liminfn}{\liminf_{n\rightarrow \infty}}
\newcommand{\tor}{\mathbb{T}^2}


\usepackage{subfiles}

\usepackage{multicol}
\setlength{\columnsep}{-5.8cm}

\providecommand{\keywords}[1]
{
  \small	
  \textbf{\textit{Keywords---}} #1
}

\providecommand{\Acknowledgements}[1]
{
  \small	
  \textbf{\textit{Acknowledgements---}} #1
}

\providecommand{\Availability}[1]
{
  \small	
  \textbf{\textit{Availability of Data and Materials---}} #1
}

\title{Loss of Exponential Mixing in a Non-Monotonic Toral Map}

\author{J. Myers Hill$^{1\,\star}$, R. Sturman$^{1}$, M. C. T. Wilson$^{2}$  \\
        \small $^{1}$School of Mathematics, University of Leeds, Leeds LS2 9JT, United Kingdom \\
        \small $^{2}$School of Mechanical Engineering, University of Leeds, Leeds LS2 9JT, United Kingdom \\
        \small $^\star$ E: j.d.myershill@leeds.ac.uk
}

\pgfplotsset{compat=1.17}

\begin{document}

\date{}

\maketitle

\begin{abstract}

We consider a Lebesgue measure preserving map of the 2-torus, given by the composition of orthogonal tent shaped shears. We establish strong mixing properties with respect to the invariant measure and polynomial decay of correlations for H\"older observables, making use of results from the chaotic billiards literature. The system serves as a prototype example of piecewise linear maps which sit on the boundary of ergodicity, possessing null measure sets around which mixing is slowed and which birth elliptic islands under certain perturbations. 
 
\end{abstract}

\Acknowledgements{JMH supported by EPSRC under Grant Refs. EP/L01615X/1 and EP/W524372/1.}

%\tableofcontents

\section{Introduction}
\subfile{intro}

\section{Some results from the billiards literature}
\label{sec:outline}
\subfile{outline}

\section{Hyperbolicity}
\label{sec:OTMhyp}
\subfile{hyperbolicity}

\section{The mixing property}
\label{sec:Hmixing}
\subfile{mixing}

\section{Decay of correlations for the return map}
\label{sec:Hsigma}
\subfile{mixingRate}

\section{Decay of correlations for the OTM}
\label{sec:polyMixingRate}
\subfile{correlationsUpper}

\section{Final remarks}
\label{sec:discussion}
\subfile{discussion}

\section{Appendix}
\subfile{appendix}


\printbibliography

%\bibliographystyle{apalike}
%\bibliography{refs.bib}

\end{document}
