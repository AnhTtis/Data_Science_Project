\vspace{-10pt}
\section{Discussion}
\label{sec:discussion}

\subsection{System utility} 
\textbf{Usability of components.} The feedback from study participants and experts was generally positive. Most of them found that the use of \name informed them of the otherwise hidden associations between concepts and classes. In the user study, \contrastiveview, and \instancespace were most often used by participants to iteratively make hypotheses. Among multiple components, \conceptassociationplot was the most appreciated by 5 participants. They mentioned that it helped them to make sense of the two interrelated aspects of associations that practitioners are typically unable to quantify. Also, two participants mentioned about the informativeness of \debiasview , ``\textit{it helps to understand how much data cleaning and training process should be done right, which is very critical to know.}’’ In the expert interview, two practitioners found \name to be useful in performing the proposed workflow by successfully identifying the intentionally biased concept-class pairs. The qualitative evaluations showed our statistical methods were effective in identifying and mitigating spurious associations compared to baseline settings.

\textbf{Concept-level systematic error debugging tool.} There are other tools available in the area of concept-based interpretability and systematic error detection. Here, we summarize when and how \name is better capable of supporting the following tasks:
\begin{itemize}
    \item Debugging-focused workflow and visualization: Compared to other visual analytic tools for concept-level interpretability such as ConceptExtract \cite{zhao2021humanintheloopextraction} or ConceptExplainer \cite{ConceptExplainerUnderstandingMentalModelDeep}, ESCAPE visually highlights and validates spurious associations between concepts and misclassifications.
    \item Human-in-the-loop workflow: While many automatic methods support the discovery of sub-regions as a part of required tasks, we highlight the comprehensive workflow to showcase how practitioners can leverage their capability of identifying spurious associations in systematic errors to generate and testing their hypotheses and precisely debug systematic errors.
    \item Quantifying the global influence of concepts: Our system is capable of quantifying the association and attribution of concepts to their impact on misclassification at the \textit{global} level.
    \item No need for auxiliary data: Compared to some studies leveraging additional data to capture the associated concepts such as text embedding \cite{DominoExtractingComparingManipulating} or user-generated report \cite{DiscoveringValidatingAIErrorsCrowdsourced}, our tool does not require auxiliary data or segments necessary to define concepts. 
\end{itemize}

\subsection{Human behavior in concept discovery}
We summarize our findings in the user study on human behavior and capability in discovering concepts. 

\textbf{Concept diversity and heuristic thinking.} During the user study, we found that participants are able to discover diverse concepts using heuristic thinking and common sense. Throughout the sessions, 26 unique concepts encompassing 11 different types such as object, pattern, size and pose, background, and location were discovered. We observed that participants often found concepts with thinking aloud in leveraging common sense (e.g., ``Dogs are usually inside than outside, so dogs may be more associated with grass.'' or ``Cats are inside and likely to be in a cage or with stuff at home.''). Such heuristic-driven analysis can be leveraged to filter out false concepts that are too deviated from common sense.

\textbf{Trade-off between precision and recall.} From the user study, we found that humans are better at precisely capturing concepts (precision: 0.75) despite achieving lower recall (recall: 0.39). For some concepts that rarely appear (e.g., red/pink or jean/skyblue in Table \ref{table:user-study-result-1}), participants were not able to hardly identify them. While automatic discovery is better at collecting significant pieces of underperformed data, but these methods still fail to recover over ~60\% of coherent slices \cite{DominoExtractingComparingManipulating} and also lack precisely capturing concepts \cite{DominoExtractingComparingManipulating, SpotlightGeneralMethodDiscoveringSystematic}. This sheds light on collaboration in automatic and human-driven methods to capture spurious associations better. For example, it is desirable to introduce some automatic methods in Identification stage in our system and accommodate human supervision in guiding subspace discovery. \\

\subsection{Generalizability}
In this section, we discuss the generalizability of our workflow and system toward data types, tasks, and applications.

\textbf{Data type}: The system interface was built to support the comprehension and mining of concepts in image data with image segmentation and contrasting images, but the framework can serve a wider range of data representations such as texts where fine-grained units as concepts are human-interpretable (vs. time-series or graphs whose subpatterns do not have semantic meanings).

\begin{figure}[!ht]
    \centering
    \includegraphics[width=.95\columnwidth]{figures/case-study-3.pdf}
    \caption{\label{fig:case-study-3}
    \textbf{Examples of females misclassified as males with certain attributes in gender classification from CelebA-HQ dataset.}}
    \vspace{-5pt}
\end{figure}

\textbf{Applicability to high-stakes scenario.} As illustrated in Section \ref{sec:introduction}, there are increasing number of AI applications impacting individuals to a greater extent. Our system can be also applicable to detecting spurious associations in the applications such as medical imaging (i.e., are there any missing cases of symptoms that can lead to diagnosis?). This was partially demonstrated with CelebA dataset whose classification can impact subgroups that have certain attributes as illustrated in Fig. \ref{fig:case-study-3} such as female with eyeglasses, hats, or no smiling.

\textbf{System and workflow}: While our system and workflow were demonstrated in discovering systematic errors and mitigating biases, these can generalize to other tasks and methods. For example, \name can be leveraged to expore concept interpretability especially for subspaces of interest in general. In addition, relative concept association can be leveraged to identifying concept-associated instances from in-the-wild cases for extending the dataset. In the mitigation stage of the workflow, active learning approach can mitigate the bias as widely demonstrated in existing work.

\subsection{Limitations and future work} In this section, we summarize the limitation and the possible future work to deliver insights and takeaways in five aspects.

\textbf{Scalability.}
Several components in \name (e.g., \instancespace and \segmentview) allows users to identify and contrast instance-level patterns in a test set and define coherent concepts with a group of segments. Nevertheless, the current design has a limited scalability for a larger number of instances and segments. According to our experiments, the system may experience degradation while rendering more than 15,000 visual elements. To get around this issue, the system may extend its ability to randomly filter out some correctly predicted instances to reduce the total number of instances while still loading all misclassification cases.
    
\textbf{Complex interaction between concepts.} During the user study, we also found that some users hypothesized combinations of concepts as a trigger of systematic errors (e.g., blackness but no person). In this case, a model may learn a composite of concepts as patterns of some classes rather than individual concepts separately. While we provided a perspective that multiple concepts in individual instances may influence predictions in a competitive manner, the complex interactions between concepts can be further investigated to see if the influence of concepts over systematic errors are amplified or have canceling-out effects when concepts are interacting with each other.

\textbf{Slice(subclass)-based vs. class(confusion-matrix)-based summary.} Among existing work in discovering systematic errors different in units of summary, namely slice-based (others) and class-based summary (ours), we find several studies with slice-based approach, which aim to summarize systematic errors from the entire instance space rather than per-class subspaces, are beneficial in identifying groups of instances that are semantically coherent but distributed across several classes. On the other hand, class-based summary gives a clear overview of which confusion cases are vulnerable with spurious associations between classes, which are better aligned with classification goal (e.g., male and female). These approaches, therefore, need to be carefully examined depending on the goal of given tasks.

\textbf{Extendability to visualize multi-classes.} The workflow of \name is designed to let users focus on the analysis of a binary case that is particularly vulnerable to systematic errors for two purposes (details in Section \ref{sec:misclassification-diagnosis-view}). The system can be extended to visualize multiple classes at once. An automatic support to detect vulnerable cases will be of help to lesson users’ cognitive support without overwhelming components presented in the system.

\textbf{Better inform the need for mitigation phase.} While our system was mostly appreciated by participants throughout the user study, three of them found \debiasview to be less useful than other components, or were not sure about whether mitigation was necessary. For the future work, the system can enhance the debias module to intuitively guide what it serves, and add visual illustrations to better educate users the consequences of biases in various scenarios when they are not mitigated.


\section{Conclusion}
In this work, we proposed \name, an interactive visualization tool for inspecting systematic errors and spurious concept associations. Our work aimed to promote a pragmatic workflow that help better inform users of the underlying mechanism behind systematic errors, namely spurious associations between undesirable patterns and target classes. To support the workflow beyond the challenges in existing tools, visual components in \name highlight the regions of systematic errors and affords human users to keep track of what undesirable associations were learned in the training process. A suite of comprehensive evaluations showed the effectiveness and utility of our system and methods. Especially, the user study with ML practitioners suggests that \name provides suitable aid for users to identify and hypothesize concepts and their associations with target classes, which was also demonstrated in the analysis of their behavioral patterns.