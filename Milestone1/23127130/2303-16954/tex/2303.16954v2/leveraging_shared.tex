% SIAM Shared Information Template
% This is information that is shared between the main document and any
% supplement. If no supplement is required, then this information can
% be included directly in the main document.


% Packages and macros go here
\usepackage{soul}
\usepackage{lipsum}
\usepackage{amsfonts}
\usepackage{epstopdf}
\ifpdf
  \DeclareGraphicsExtensions{.eps,.pdf,.png,.jpg}
\else
  \DeclareGraphicsExtensions{.eps}
\fi

% date 
\usepackage{datetime}
\newdateformat{monthyeardate}{%
	\monthname[\THEMONTH] \THEDAY, \THEYEAR}

% ORCID 
\usepackage{academicons}
\usepackage{xcolor}
\renewcommand{\orcid}[1]{\href{https://orcid.org/#1}{\textcolor[HTML]{A6CE39}{orcid.org/#1}}}

% math packages
\usepackage{amsmath}
\allowdisplaybreaks
\usepackage{amssymb}
\usepackage{commath}
\usepackage{mathtools}
\usepackage{bbm}
\usepackage{bm}

% figures
\usepackage{color}
\usepackage{graphicx}
\usepackage[small]{caption}
\usepackage{subcaption}

% Pseudocode, tables, and tikz 
\usepackage{relsize}
\usepackage{adjustbox}
\usepackage{algorithmic}
\usepackage{booktabs}
\usepackage{tikz}
\usepackage{pifont}% http://ctan.org/pkg/pifont for cmark
\usetikzlibrary{bayesnet} % for Bayesian network (graphical models)

% Itemize with squares 
\renewcommand{\labelitemi}{\tiny$\blacksquare$}

% Prevent itemized lists from running into the left margin inside theorems and proofs
\usepackage{enumitem}
\setlist[enumerate]{leftmargin=.5in}
\setlist[itemize]{leftmargin=.5in}

% Add a serial/Oxford comma by default.
\newcommand{\creflastconjunction}{, and~}

% Used for creating new theorem and remark environments
\newsiamremark{remark}{Remark}
\newsiamremark{example}{Example}
\newsiamremark{hypothesis}{Hypothesis}
\crefname{hypothesis}{Hypothesis}{Hypotheses}
\newsiamthm{claim}{Claim}
%\newsiamthm{definition}{Definition}

% Sets running headers as well as PDF title and authors
\headers{Leveraging joint sparsity in hierarchical Bayesian learning}{J.\ Glaubitz and A.\ Gelb}

% Title. If the supplement option is on, then "Supplementary Material"
% is automatically inserted before the title.
\title{Leveraging joint sparsity in hierarchical Bayesian learning
\thanks{\monthyeardate\today 
\corresponding{Jan Glaubitz} 
}}
% Authors: full names plus addresses.
\author{ 
Jan Glaubitz\thanks{Department of Aeronautics and Astronautics, MIT, Cambridge, MA 02139, USA (\email{glaubitz@mit.edu}, \orcid{0000-0002-3434-5563})}
\and 
Anne Gelb\thanks{Department of Mathematics, Dartmouth College, Hanover, NH 03755, USA (\email{Anne.E.Gelb@Dartmouth.edu}, \orcid{0000-0002-9219-4572})}
}

\usepackage{amsopn}

% commands 
\usepackage{comment}
\newcommand{\todo}[1]{{\Large \textcolor{pink}{#1}}}
\newcommand{\revA}[1]{{\color{red}#1}}
\newcommand{\revB}[1]{{\color{blue}#1}}
\newcommand{\revC}[1]{{\color{cyan}#1}}
\newcommand{\AG}[1]{{\color{red}#1}}
\newcommand{\JG}[1]{{\color{blue}#1}} 
\newcommand{\HELP}[1]{{\color{green}#1}}
\usepackage[normalem]{ulem}

% editing packages
\usepackage[normalem]{ulem}

% definitions 
\newcommand{\MAP}{\mathrm{MAP}}
\newenvironment{eq}{\begin{equation}}{\end{equation}}
\DeclareMathOperator{\diag}{diag} 
\DeclareMathOperator{\trace}{trace} 
\DeclareMathOperator{\sign}{sign}
\DeclareMathOperator*{\argmin}{arg\,min} 
\DeclareMathOperator*{\argmax}{arg\,max} 
\DeclareMathOperator*{\cond}{cond}
\DeclareMathOperator*{\kernel}{kernel} 
\newcommand{\scp}[2]{\left\langle{#1,\, #2}\right\rangle} 
\renewcommand{\d}{\mathrm{d}} 
\newcommand{\intd}{\, \mathrm{d}} 
\renewcommand{\vec}{\, \mathrm{vec}}
\newcommand{\N}{\mathbb{N}}
\newcommand{\Z}{\mathbb{Z}}
\newcommand{\R}{\mathbb{R}} 
\newcommand{\C}{\mathbb{C}}
\renewcommand{\Re}{\mathrm{Re}}
\renewcommand{\Im}{\mathrm{Im}}
\newcommand{\cmark}{\ding{51}}%
\newcommand{\xmark}{\ding{55}}%
\newcommand{\widesmall}[2]{#1}

%%% Local Variables: 
%%% mode:latex
%%% TeX-master: "ex_article"
%%% End: 
