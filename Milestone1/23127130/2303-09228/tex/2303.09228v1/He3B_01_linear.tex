%%%%%%%%%%%%%%%%%%%%%%%%%%%%%%%%%%%%%%%%%%%%%%%%%%%%%%%%%%%%%%%%%%%%%%
\section*{Part 1. Linear spin waves in $^3$He-B}

At energy scales much less then energy gap~$\Delta$ the state of
superfluid $^3$He-B is described by the order parameter:
\begin{equation}
A_{aj}  = \frac{1}{\sqrt{3}}\ \Delta\ e^{i\varphi} R_{aj}
\end{equation}
where~$\varphi$ is phase, and $R_{aj}$
is a rotation matrix which can be written in terms of rotation axis~$\bf n$
and rotation angle~$\vartheta$ as
\begin{equation}\label{eq1:r_nt}
R_{a j} = \ct\ \delta_{a j} + (1-\ct)\ n_a n_j - \st\ e_{ajk} n_k.
\end{equation}
In small magnetic fields,~$\hbar\gamma H \ll \Delta$ the gap~$\Delta$ and
magnetic susceptibility~$\chi_B$ are isotropic. There are four degrees of
freedom in the system, the phase~$\varphi$, the unit vector~$\bf n$ and
the angle~$\vartheta$. Oscillations of the phase is sound (here we do not
consider normal component and thus have only one sound mode), and
oscillations of the $R_{aj}({\bf n},\vartheta)$ matrix are spin waves. In
the following discussion we are interested only in these three spin-wave
modes. Hamiltonian of the system can be written as a combination of three
components: magnetic energy, energy of spin-orbit interaction and
gradient energy (see VW 6.103, VW 7.17).
\begin{equation}\label{eq1:ham}
\mathcal{H} = F_M + F_{SO} + F_\nabla,
\end{equation}
\begin{eqnarray}
\label{eq1:en_m0}
F_M &=& - ({\bf S} \cdot \gamma {\bf H})
+ \frac{\gamma^2}{2\chi_B}{\bf S}^2,\\
\label{eq1:en_d0}
F_{SO}
&=& g_D\Delta^2 \Big[
           R_{jj}R_{kk}
         + R_{jk}R_{kj}
 - \frac23 R_{jk}R_{jk}\Big]
= g_D\Delta^2 \Big[ R_{jj}R_{kk} + R_{jk}R_{kj}] + \mbox{const.}
,\\
\label{eq1:en_g0}
F_\nabla
&=& \frac12 \Delta^2 \Big[
  K_1 (\nabla_j R_{ak})(\nabla_j R_{ak})
+ K_2 (\nabla_j R_{ak})(\nabla_k R_{aj})
+ K_3 (\nabla_j R_{aj})(\nabla_k R_{ak}) \Big],
\end{eqnarray}
where~$\bf S$ is spin and $\bf H$ is magnetic field.

We are going to write equations of motion for this system. Spin is a not
a canonical variable, its components do not commute with each other and
thus we have to use equations with Poisson brackets. In this approach
evolution of any parameter~$a$ can be written as~$\dot a =
\{\mathcal{H},a\}$. Poisson brackets can be found from microscopic
considerations, from commutation rules in quantum mechanics, or from
symmetry~\cite{1980_pbrackets,2004_bunk_dyn}.

The matrix~$R_{aj}$ for any value of index $j$ can be treated as a
vector in spin space. For such a vector commutation rules can be
written as:
\begin{equation}\label{eq1:brackets_R}
\{S_a, S_b\} = -e_{abc} S_c, \quad
\{R_{aj}, S_b\} = \{S_a, R_{bj}\} = -e_{abc} R_{cj}, \quad
\{R_{aj}, R_{bk}\} = 0.
\end{equation}

It is possible to write equations of motion for the spin~$\bf S$ and
matrix $R_{aj}$ (see later, in the second part of this text), but here we
will use small rotations~$\ts$ of the spin space as coordinates.

%%%%%%%%%%%%%%%%%%%%%%%%%%%%%%%%%%%%%%%%%%%%%%%%%%%%%%%%%%%%%%%%%%%%%%
\subsection*{Using small rotations $\ts_a$ as coordinates}

We follow derivation of Theodorakis and Fetter
\cite{1983_theodorakis}, see also VW~8.4.6.
Consider some equilibrium distribution of the order parameter matrix
$R^0$. We are going to study small oscillations around the equilibrium.
Arbitrary change of the rotation matrix can be represented as an
additional rotation:
\begin{equation}
R_{a j} = R_{ab}({\bf \ts}) R^0_{bj}.
\end{equation}
Here $\ts$ is a vector and $R_{ab}(\ts)$ is a rotation matrix around the
direction of $\ts$ by angle $|\ts|$. We will use small rotations,
$|\ts|\ll1$. Using formula~(\ref{eq1:r_nt}) for $R_{ab}(\ts)$ one can write
up to second order terms in~$|\ts|$:
\begin{equation}\label{eq1:srot}
R_{a j}
\quad=\quad
\left(
\delta_{ab}
- e_{abc}\ts_c
+ \frac12 \ts_a \ts_b
- \frac12 \delta_{ab}\ts_c \ts_c + o({\ts}^2)
\right) R^0_{bj}
\end{equation}
The Hamiltonian can be written as a function of the spin and angles~$\ts$.
Poisson brackets in the equations of motion can be expanded as
\begin{eqnarray}
\dot S_a
&=& \{\mathcal{H}, S_a\}
\quad=\quad
  \frac{\delta \mathcal{H}}{\delta S_b} \{S_b,S_a\}
+ \frac{\delta \mathcal{H}}{\delta \ts_b} \{\ts_b,S_a\}\\
\dot \ts_a
&=& \{\mathcal{H}, \ts_a\}
\quad=\quad
  \frac{\delta \mathcal{H}}{\delta S_b} \{S_b,\ts_a\}
+ \frac{\delta \mathcal{H}}{\delta \ts_b} \{\ts_b,\ts_a\}
\end{eqnarray}
We are going to obtain linear equations for small~$\ts$. Derivatives of
the Hamiltonian are zero in the equilibrium (energy is in the minimum
at~$\ts=0$) and thus we have only first-order terms there. Thus in the
Poisson brackets we need only zero-order terms. As the matrix $R_{aj}$ is a
function of $\ts$ one can write
\begin{equation}
\{R_{aj}, S_b\} = \frac{d R_{aj}}{d\ts_c}\{\ts_c, S_b\}.
\end{equation}
Using~(\ref{eq1:brackets_R}) and~(\ref{eq1:srot}) one can find commutation
rules for our set of coordinates:
\begin{equation}\label{eq1:brackets_ts}
\{S_a, S_b\} = -e_{abc} S_c, \quad
\{\ts_a, S_b\} = -\{S_a, \ts_b\} = -\delta_{ab}, \quad
\{\ts_a, \ts_b\} = 0.
\end{equation}
It is easy to find the derivative~$\delta \mathcal{H}/\delta S_b$. Using it
and the commutation rules we have:
\begin{eqnarray}
\label{eq1:hameq1_ts}
\dot {\bf S} &=&
  [ {\bf S} \times \gamma {\bf H} ]
- \frac{\delta \mathcal{H}}{\delta {\bf \ts}}\\
\label{eq1:hameq2_ts}
{\bf \dot\ts} &=& \gamma\left(\frac{\gamma}{\chi_B} {\bf S} - {\bf H} \right)
  \quad=\quad \frac{\gamma^2}{\chi_B} {\bf \delta S}
\end{eqnarray}
where we introduced~$\delta {\bf S}$, deviation of the spin from its
equilibrium value ${\bf S^0} = \chi_B{\bf H}/\gamma$. These equations
describe a simple Larmor precession with the additional term
$-\delta\mathcal{H}/\delta\ts$~--- a torque acting on the spin because
of gradient and spin-orbit interactions. This derivative is not trivial,
and will be calculated below.

%%%%%%%%%%%%%%%%%%%%%%%%%%%%%%%%%%%%%%%%%%%%%%%%%%%%%%%%%%%%%%%%%%%%%%
\subsection*{Variational derivative for small rotations}
Since the energy depends on both angles $\ts_a$ and their gradients
$\nabla_j\ts_a$ we have to use a so-called variational derivative
$\delta\mathcal{H}/\delta \ts_a$~\cite{haar_mech}. An additional
difficulty appears because rotations~$\ts$ do not commute. Let us take
this into account. Using formula~(\ref{eq1:srot}) one can check that two
small successive rotations, $\ta$ and then $\tb$ are equivalent to the
rotation by the angle $\ta + \tb + \frac12 \ta\times\tb$ up to second
order terms in the angles:
\begin{equation}\label{eq1:srot_double}
R_{ac}(\tb)R_{cb}(\ta) \quad=\quad
R_{ab}\left(\ta + \tb + \frac12 [\ta\times\tb] + o(\ta^2,\tb^2)\right).
\end{equation}
Two rotations done in a different order produce a difference $[\ta\times\tb]$.
This can be written as a commutation rule for differentials (see VW~9.14):
\begin{equation}\label{eq1:srot_comm}
\delta\nabla\ts - \nabla\delta\ts = [\delta\ts\times\nabla\ts].
\end{equation}
Consider a small non-uniform rotation, $\delta\ts({\bf r})$,
with boundary condition~$\delta\ts=0$ which changes gradients by $\delta\nabla_j\ts$.
The functional derivative~$\delta\mathcal{H}/\delta\ts_a$ is given by
\begin{equation}
\int_V \frac{\delta\mathcal{H}}{\delta\ts_a} \delta\ts_a\ dr =
\int_V \left(
  \frac{\partial F}{\partial\ts_a}\ \delta\ts_a
+ \frac{\partial F}{\partial\nabla_j\ts_a}\ \delta\nabla_j\ts_a
\right)\ dr
\end{equation}
From mechanical point of view this means that work produced against a
torque~$T_a=-\delta\mathcal{H}/\delta\ts_a$ equals to the total change
of energy. Now we swap differentials~$\delta\nabla\ts$ using the
formula~(\ref{eq1:srot_comm}) and integrate this term by parts using the
zero boundary condition.
\begin{equation}
\int_V \frac{\delta\mathcal{H}}{\delta\ts_a} \delta\ts_a\ dr = 
\int_V \left(
  \frac{\partial F}{\partial\ts_a}\ \delta\ts_a
- \nabla_j\frac{\partial F}{\partial\nabla_j\ts_a}\ \delta\ts_a
+ \frac{\partial F}{\partial\nabla_j\ts_a} e_{abc}\ \delta\ts_b\ \nabla_j\ts_c
\right) dr
\end{equation}
This equation is true for any possible variations~$\delta\ts$. This means that
the integrands are equal at any point and the derivative is:
\begin{equation}\label{eq1:torque}
\frac{\delta\mathcal{H}}{\delta\ts_a} =
 \frac{\partial F}{\partial\ts_a}
- \nabla_j \frac{\partial F}{\partial \nabla_j\ts_a}
+ \frac{\partial F}{\partial\nabla_j\ts_c} e_{abc}\ \nabla_j\ts_b
\end{equation}

%%%%%%%%%%%%%%%%%%%%%%%%%%%%%%%%%%%%%%%%%%%%%%%%%%%%%%%%%%%%%%%%%%%%%%
\subsection*{Gradient energy}
The gradient energy in $^3$He-B is given by~(\ref{eq1:en_g0}).
Now put the expression for the distorted matrix~(\ref{eq1:srot})
into it. Both original matrix $R^0$ and rotation angles
$\ts$ can be non-uniform here:
\begin{eqnarray}\label{eq1:fgrad}
\frac{2}{\Delta^2}(F_\nabla - F_\nabla^0) 
&=&(2K_1+K_2+K_3)
\ (\nabla_j \ts_a) (\nabla_j \ts_a)
- [K_2 R^0_{aj} R^0_{bk} + K_3 R^0_{ak} R^0_{bj}]
\ (\nabla_j \ts_a) (\nabla_k \ts_b)\\\nonumber
&+& [K_1 R^0_{ak} (\nabla_j R^0_{bk}) + K_2 R^0_{ak} (\nabla_k R^0_{bj}) + K_3 R^0_{aj} (\nabla_k R^0_{bk})]
\ (2e_{abc} + \delta_{ac}\ts_b - \delta_{bc}\ts_a)\ (\nabla_j \ts_c),
\end{eqnarray}
where $F_\nabla^0$~is an energy calculated for the undistorted matrix
$R^0$. Substituting this into~(\ref{eq1:torque}) we have the torque $T_c^\nabla = -\delta F_\nabla/\delta\ts$:
\begin{eqnarray}\label{eq1:tgrad}
\frac{T_c^\nabla}{\Delta^2}
&=& (2K_1+K_2+K_3)\ (\nabla_j \nabla_j \ts_c)
- \nabla_j \left[ (K_2 R^0_{ak} R^0_{cj} + K_3 R^0_{aj} R^0_{ck})\ (\nabla_k \ts_a)\right] \\\nonumber
%
&+& \frac12 R^0_{ak}\ \left[K_1 (\nabla_j\nabla_j R^0_{bk}) + (K_2 + K_3)( \nabla_j\nabla_k R^0_{bj})\right]
\ (2e_{abc} + \delta_{ac}\ts_b - \delta_{bc}\ts_a)
\end{eqnarray}

If gradients of equilibrium texture $R^0$ can be neglected ("uniform texture") then
\begin{equation}\label{eq1:tgrad_uniform}
T^\nabla_c
=  \Delta^2\ [K\ \delta_{ac}\ \nabla_j\nabla_j - K'\ R^0_{ak} R^0_{cj} \nabla_j\nabla_k]\ \ts_a
\end{equation}
where $K=2K_1+K_2+K_3$ and $K'=K_2+K_3$.


%%%%%%%%%%%%%%%%%%%%%%%%%%%%%%%%%%%%%%%%%%%%%%%%%%%%%%%%%%%%%%%%%%%%%%
\subsection*{Dipolar energy}

Dipolar energy is given by~(\ref{eq1:en_g0}).
Substituting the small rotation~(\ref{eq1:srot}) we get:
\begin{eqnarray}
R_{jj}R_{kk} + R_{jk}R_{kj}
&=& [ R^0_{jj} R^0_{kk} + R^0_{jk}R^0_{kj} ]\ (1-|\ts|^2)\\ \nonumber
&+& [R^0_{a j} R^0_{kk} + R^0_{a k} R^0_{kj}]\ (\ts_j \ts_a - 2e_{ja b}\ \ts_b) \\ \nonumber
&+& [R^0_{a j} R^0_{a' k} + R^0_{a k} R^0_{a' j}]\ e_{jab}\ \ts_{b}\ e_{ka'b'}\ \ts_{b'}
\end{eqnarray}

Or in terms of axis $\bf n$ and rotation angle $\vartheta$ of matrix $R^0$:
\begin{eqnarray}\label{eq1:edip2}
R_{jj}R_{kk} + R_{jk}R_{kj}
&=& \frac12(4\ct+1)^2  -\frac12\\
&-& 4\st (4 \ct + 1)\ ({\bf n} \cdot {\bf \theta}^s)\nonumber\\ \nonumber
&-& (4\ct+1)(\ct+1)\ |\ts|^2 \\ \nonumber
&+& (9 + 3\ct - 12\cts)\ ({\bf n} \cdot {\bf \theta}^s)^2 \nonumber
\end{eqnarray}

The torque $T_c^D = -\delta F_D/\delta\ts$ is
\begin{eqnarray}\label{eq1:tdip}
\frac{{\bf T}^D}{\Delta^2} = 
&=& 2 g_D (4 \ct + 1) \left[
  2\st\ {\bf n} + (\ct+1)\ {\bf\ts}
  \right] \\ \nonumber
&-& 2 g_D (9 + 3\ct - 12\cts)\ ({\bf n} \cdot {\bf \theta}^s) {\bf n}
\end{eqnarray}


%% other way: theta -> theta+dt, dt=(n ts)
%% not so simple!

Equilibrium texture stays in the minimum of the dipolar energy
where $\ct=-1/4$. In this case the dipolar torque is
\begin{equation}\label{eq1:tdip2}
{\bf T}^D = - 15 g_D\Delta^2\ ({\bf n\cdot\ts}) {\bf n}
\end{equation}

%\begin{equation}\label{eq1:He3_omegab}
%g_D = \frac{\chi_B\Omega_B^2}{15\gamma^2\Delta^2},
%\end{equation}
%$\Omega_B$, $\chi_B$ and $\gamma$ are Leggett frequency, susceptibility and
%gyromagnetic ratio of $^3$He.

% If $\vartheta$ angle in the equilibrium texture differs from~$\mbox{acos}(-1/4)$,
% for example near vortex cores, and the difference $\delta\vartheta$ is small,
% additional torque can be written as
% \begin{equation}\label{eq1:tdip3}
% {\bf T}^D_{\delta\vartheta} =
%   - \frac32 \sqrt{15} g_D\Delta^2\ [{\bf\ts} - 3 ({\bf n\cdot\ts}) {\bf n} ]\ \delta\vartheta.
% \end{equation}


%%%%%%%%%%%%%%%%%%%%%%%%%%%%%%%%%%%%%%%%%%%%%%%%%%%%%%%%%%%%%%%%%%%%%%
\subsection*{Spin waves in a uniform texture}

Put the dipolar torque~(\ref{eq1:tdip2}) and the gradient
torque~(\ref{eq1:tgrad_uniform}) into the equation~(\ref{eq1:hameq1_ts}),
differentiate it over time and exclude $\ts$ using the
equation~(\ref{eq1:hameq2_ts}). Here we work only with first order terms
in~$\ts$ and thus do not care about commutation of
derivatives~(\ref{eq1:srot_comm}).
\begin{eqnarray}\label{eq1:ham_eq3}
\delta{\bf \ddot S} &=& [\delta{\bf \dot S}\times \gamma {\bf H}] 
  + {\bf\hat\Lambda}\ \delta{\bf S}, \\\nonumber
&& \mbox{where}\quad  \Lambda_{ab} = \frac{\Delta^2\gamma^2}{\chi_B} \left[
K\ \delta_{ab}\delta_{kj}
- K'\ R^0_{ak}R^0_{bj}\right] \nabla_j\nabla_k
- \Omega_B^2\ n_a n_b.
\end{eqnarray}
Here we introduced Leggett frequency:
\begin{equation}\label{eq1:omega_b}
\Omega_B^2 = \frac{\gamma^2\Delta^2}{\chi_B} 15g_D.
\end{equation}

For a flat wave $\delta {\bf S} = {\bf s} \exp(i\omega t + i{\bf k}{\bf x})$
with frequency~$\omega$ and wave vector~${\bf k}$ in the magnetic field directed along $z$~axis

\begin{equation}
\left[
\left(
\begin{array}{ccc}
\omega^2 & -i\omega\omega_L & 0 \\
i\omega\omega_L & \omega^2 & 0 \\
0 & 0 & \omega^2
\end{array}
\right)
+ {\bf\hat\Lambda}
\right]
\left(
\begin{array}{c}
s_x \\ s_y \\ s_z
\end{array}
\right)
=0,
\end{equation}
\begin{equation}
\qquad
\Lambda_{ab} = \frac{\Delta^2\gamma^2}{\chi_B} \left[
K\ \delta_{ab}{\bf k}^2
- K'\ R^0_{ak}k_k R^0_{bj}k_j\right]
- \Omega_B^2\ n_a n_b.
\end{equation}
where~$\omega_L=\gamma H$. Solution corresponds to zero determinant of the matrix.
%\begin{equation}
%\left|
%\begin{array}{lll}
%\Lambda_{xx} + \omega^2        & \Lambda_{xy} - i\omega\omega_L & \Lambda_{xz} \\
%\Lambda_{yx} + i\omega\omega_L & \Lambda_{yy} + \omega^2        & \Lambda_{yz} \\
%\Lambda_{zx}                   & \Lambda_{zy}                   & \Lambda_{zz} + \omega^2
%\end{array}
%\right| = 0
%\end{equation}
Using the fact that ${\bf\hat\Lambda}$ is symmetric this can be written as:
\begin{eqnarray}
[ (\Lambda_{xx} + \omega^2)(\Lambda_{yy} + \omega^2)
  - \Lambda_{xy}^2 - \omega^2\omega_L^2](\Lambda_{zz} + \omega^2)
&+&\\\nonumber
2\Lambda_{xy}\Lambda_{yz}\Lambda_{xz}
 - (\Lambda_{xx} + \omega^2)\Lambda_{yz}^2
 - (\Lambda_{yy} + \omega^2)\Lambda_{xz}^2
&=& 0,
\end{eqnarray}
or
\begin{eqnarray}
\omega^6
&+& \omega^4[ \Lambda_{xx} + \Lambda_{yy} + \Lambda_{zz} - \omega_L^2 ]\\\nonumber
%
&+& \omega^2[
  \Lambda_{xx}\Lambda_{yy}
+ \Lambda_{yy}\Lambda_{zz}
+ \Lambda_{zz}\Lambda_{xx}
- \Lambda_{xy}^2
- \Lambda_{yz}^2
- \Lambda_{xz}^2
- \omega_L^2\Lambda_{zz}
]\\\nonumber
%
&+& \Lambda_{xx}\Lambda_{yy}\Lambda_{zz}
+ 2\Lambda_{xy}\Lambda_{yz}\Lambda_{xz}
- \Lambda_{xx}\Lambda_{yz}^2
- \Lambda_{yy}\Lambda_{zx}^2
- \Lambda_{zz}\Lambda_{xy}^2
= 0,
\end{eqnarray}

We have a third-order equation for $\omega^2$. It gives three doubly
degenerate spin-wave modes with positive and negative frequencies. The
equation can be analytically solved for arbitrary wave-vector~${\bf k}$,~${\bf n}$
and~$\omega_L$ in the uniform texture ($\nabla R^0 \ll k$). Calculation
is implemented in {\tt he3lib}~\cite{he3lib}.


%%%%%%%%%%%%%%%%%%%%%%%%%%%%%%%%%%%%%%%%%%%%%%%%%%%%%%%%%%%%%%%%%%%%%%
\subsection*{Uniform precession}

If we neglect all gradient terms then $\Lambda_{ab} = -\Omega_B^2 n_a n_b$ and
equation~(\ref{eq1:ham_eq3}) is (compare with~\cite{2006_jetpl_catrel}):
\begin{equation}
\omega^6 - \omega^4\big(\omega_L^2 + \Omega_B^2\big) + \omega^2\omega_L^2\Omega_B^2 n_z^2 = 0.
\end{equation}
NMR experiments are often done at a fixed frequency $\omega$. We can solve the equation
for $\gamma H = \omega_L$ and find magnetic field where the resonance is observed:
\begin{equation}
\gamma H  =  \omega \sqrt{ \frac{\omega^2 - \Omega_B^2}{\omega^2 - \Omega_B^2 n_z^2} }.
\end{equation}
At a fixed magnetic field we solve the equation for~$\omega$ and find three spin-wave modes:
\begin{equation}
\omega^2 = \frac12(\omega_L^2 + \Omega_B^2) \pm \sqrt{\frac14(\omega_L^2 + \Omega_B^2)^2 - \omega_L^2\Omega_B^2 n_z^2},
\qquad \omega^2 = 0
\end{equation}
For $\gamma H \gg \Omega_B$ we have a well-known expressions for transverse and longitudinal NMR frequency:
\begin{equation}
\omega = \gamma H + \frac{\Omega_B^2}{2\gamma H}\ (1-n_z^2),
\qquad
\omega = \Omega_B\ n_z,
\qquad
\omega = 0,
\end{equation}
Here meaning of $\Omega_B$ becomes clear, it is frequency of the
longitudinal NMR in a texture with~$\bf n || H$.

%%%%%%%%%%%%%%%%%%%%%%%%%%%%%%%%%%%%%%%%%%%%%%%%%%%%%%%%%%%%%%%%%%%%%%
\subsection*{Separating equations for transverse and longitudinal magnons}

Let's use complex {\it spherical coordinates}.
For an arbitrary vector $A$:
\begin{equation}
A_{\pm} = \frac{1}{\sqrt2} (A_x\pm i A_y),\quad
A_0 = A_z,\qquad
A_i B_i = A_q B^*_{q},\quad\mbox{where $q=0,+,-$}
\end{equation}
In terms of polar and azimuthal angles $\alpha$ and $\beta$:
\begin{equation}
A_\pm = \frac{|A|}{\sqrt2}\sin\beta\exp(\pm i\alpha),\qquad
A_0 = |A|\cos\beta
\end{equation}

If field $\bf H$ is directed alone the $z$ axis, then $[A \times H]_q = -q i A_q H$.
Now one can write the equation in spherical coordinates for a harmonic
oscillation $\delta S_q = s_q e^{i\omega t}$:
\begin{equation}
\left[
\frac{\gamma^2\Delta^2}{\chi_B}
  \big[-K\ \delta_{qp} \nabla_j\nabla_j
 + K'\   R^0_{qj}R^0_{p^*k} \nabla_j\nabla_k\big]
+ \Omega_B^2\ n_q n_p^*
\right] s_p =
\omega(\omega-q\gamma H) s_q
\end{equation}

Now let's remove non-diagonal coupling between different modes. According
to~\cite{1983_theodorakis} this coupling is small in the high-frequency
limit $\omega\gg\Omega_B$ (as $\Omega_B^2/\omega^2$). Then
we can write (without summation of the $q$ index):
\begin{equation}\label{eq1:spinwaves1}
\left[
\frac{\gamma^2\Delta^2}{\chi_B}
  \big[-K\ \nabla_j\nabla_j
 + K'\  R^0_{qj}R^0_{q^*k} \nabla_j\nabla_k\big]
+ \Omega_B^2\ |n_q|^2
\right] s_q =
\omega(\omega-q\gamma H) s_q
\end{equation}

Now we can write separate equations for transverse ($q={+},{-}$) and
longitudinal ($q=0$) modes. We use the fact that
\begin{equation}
R^0_{+j} R^0_{-k}\label{eq1:rr}
= \frac12 (R^0_{xj}R_{xk}+R^0_{yj}R_{yk})
= \frac12(\delta_{jk} - R^0_{zj}R_{zk}),
\end{equation}
%\qquad
%|n_q|^2 = \frac12 \sin^2\beta_N
and $R^0_{zj}$ is an equilibrium orbital anisotropy
axis: $\hat L^0_j = R^0_{aj} \hat S^0_a = R^0_{zj}$.

\begin{eqnarray}\label{eq1:tr_spinwaves}
\Big[
 - c_\perp^2\ \nabla^2
 - (c_\parallel^2-c_\perp^2) \hat L^0_j \hat L^0_k \nabla_j\nabla_k
+ \frac12 \Omega_B^2 \sin^2\beta_N
\Big] s_+ &=&
\omega(\omega-\gamma H)\ s_+\\
%%
\Big[\label{eq1:lo_spinwaves}
 - C_\perp^2 \nabla^2
 - (C_\parallel^2-C_\perp^2) \hat L^0_j \hat L^0_k \nabla_j\nabla_k
+ \Omega_B^2 \cos^2\beta_N
\Big] s_0 &=&
\omega^2\ s_0
\end{eqnarray}
here we introduce parameters:
\begin{equation}
c_\parallel^2 = C_\perp^2 = \frac{\gamma^2\Delta^2}{\chi_B} K,\quad
c_\perp^2 = \frac{\gamma^2\Delta^2}{\chi_B}(K-K'/2),\quad
C_\parallel^2 = \frac{\gamma^2\Delta^2}{\chi_B}(K-K'),\qquad
\end{equation}

We have skipped the equation for~$s_{-}$. It can be obtained from one for~$s_{+}$ by changing
$\omega$ by $-\omega$ and thus have same solutions but with an opposite sign. As expected we
have three doubly degenerate modes with positive and negative~$\omega$.
Note that these equations have been obtained in the assumption $\omega\gg\Omega_B$.

%%%%%%%%%%%%%%%%%%%%%%%%%%%%%%%%%%%%%%%%%%%%%%%%%%%%%%%%%%%%%%%%%%%%%%
\subsection*{Spin-wave velocities in a weak coupling approximation}

Gradient energy coefficients
have been obtained in~\cite{1975_cross} (see also VW~7.25).
\begin{equation}
2 K_1 + K_2 + K_3 = -\frac{2}{\Delta^2}\left(\frac{\hbar}{2m}\right)^2 (4+\delta)c,
\qquad
K_2 = -\frac{2}{\Delta^2}\left(\frac{\hbar}{2m}\right)^2 c,
\qquad
K_3 = -\frac{2}{\Delta^2}\left(\frac{\hbar}{2m}\right)^2 (1+\delta)c
\end{equation}
where
\begin{equation}
c=-\frac{\rho_s}{10}
\ \frac{3+F_1^a}{3+F_1^s}
\ \frac{1}{1+F_1^a(5-3\rho_s/\rho)/15}
,\qquad
\delta = \frac{F_1^a \rho_s/\rho}{3+F_1^s(1-\rho_s/\rho)}
\end{equation}
Then
\begin{equation}\label{eq1:swvel_theor}
c_\parallel^2 = C_\perp^2 =
-\frac{2\gamma^2}{\chi_B}
\left(\frac{\hbar}{2m}\right)^2
(4+\delta)c
,\quad
c_\perp^2 =
-\frac{2\gamma^2}{\chi_B}\left(\frac{\hbar}{2m}\right)^2
(3+\delta/2)c
,\quad
C_\parallel^2 =
-\frac{2\gamma^2}{\chi_B}\left(\frac{\hbar}{2m}\right)^2
2c
\end{equation}

Without Fermi liquid corrections $K_1=K_2=K_3$ and
\begin{equation}
c_\perp/c_\parallel = \sqrt{3/4},\qquad
C_\perp/C_\parallel = \sqrt{2}.
\end{equation}
In many papers one of these two sets of spin-wave velocities is usually
used. For example, in early Leggett's papers it is $C_\perp/C_\parallel$~\cite{1975_he3_teor_leggett},
in Fomin's papers it is $c_\perp/c_\parallel$~\cite{1980_jetp_fomin_spinwaves}.



%%%%%%%%%%%%%%%%%%%%%%%%%%%%%%%%%%%%%%%%%%%%%%%%%%%%%%%%%%%%%%%%%%%%%%
\subsection*{Motion of $\bf L$, $\bf n$ and $\vartheta$ in the spin wave}

\def\dt{\delta\vartheta}
\def\dn{\delta n}

A common question is how $\bf L$, $\bf n$ and $\vartheta$ move in the
transverse NMR. Here we calculate this explicitly. First let us find
deviations of rotation angle and axis of the matrix $R_{aj}$, ~$\bf\dn$
and~$\dt$ caused by rotation $\bf\ts$. In a linear approximation:
\begin{equation}
R_{aj} =
R^0_{aj}
+ \frac{\partial R^0_{aj}}{\partial\vartheta}\dt
+ \frac{\partial R^0_{aj}}{\partial n_k}\dn_k
= (\delta_{ab} - e_{abc}\ts_c) R^0_{bj}
\end{equation}
Using expression~(\ref{eq1:r_nt}) for~$R^0_{aj}$ and finding convolutions
with~$\delta_{aj}$, $n_a$ and $n_j$ one can find:

% \begin{eqnarray}
% & & [\st\ \delta_{a j} - \st\ n_a n_j + \ct\ e_{ajk} n_k] \dt\\
% &-& (1-\ct)\ (n_a \dn_j + n_j \dn_a)
%     + \st\ e_{ajk} \dn_k\\
% &=&
%   \ct\ e_{ajc}\ts_c
% + (1-\ct)\ e_{abc}\ts_c n_b n_j
% - \st\ \ts_j n_a
% + \delta_{aj} \st\ \ts_c n_c
% \end{eqnarray}

% * n_a
% \begin{eqnarray}
% && - (1-\ct)\ \dn_j
% + \st\ e_{ajk} n_a \dn_k\\
% &=&
%   \ct\ e_{ajc} n_a \ts_c
% - \st\ \ts_j
% + \st\ \ts_c n_c n_j
% \end{eqnarray}
% % * n_j
% \begin{eqnarray}
% &-& (1-\ct)\ \dn_a
% + \st\ e_{ajk} n_j \dn_k\\
% &=&
%   e_{ajc} n_j \ts_c
% \end{eqnarray}

\begin{equation}\label{eq1:dtdn}
\dt = {(\bf n \cdot \ts)}
,\qquad
{\bf \dn} =
- \frac12\ {[\bf n \times \ts]}
+ \frac{{\bf \ts} - {(\bf \ts\cdot n)}\ {\bf n}}{2 \tan\vartheta/2}
\end{equation}

Orbital and spin anisotropy axes are connected by the order parameter:
$L_j = S_a R_{aj}$. Using~(\ref{eq1:srot}) and~(\ref{eq1:hameq2_ts}) one can write
up to the first order of $\ts$:
\begin{equation}\label{eq1:dl}
L_j =
\frac{\chi_B}{\gamma^2}
(\gamma {\bf H} + \dot \ts - [\gamma {\bf H} \times \ts] )_a R^0_{aj}
\end{equation}

Consider a harmonic transverse spin wave with frequency~$\omega$ in the field
$\bf H\parallel \hat z$. Then
\begin{equation}
\dot\ts = \frac{\gamma^2}{\chi_B} {\bf \delta S}
,\qquad
\ts = - \frac{\gamma^2}{\omega \chi_B} [{\hat z} \times {\bf\delta S}]
\end{equation}
and substituting this to~(\ref{eq1:dtdn}) and~(\ref{eq1:dl}) we write~$\dt$ and~$\bf L$:
\begin{equation}\label{eq1:dtdn_harm}
\dt = - \frac{\gamma^2}{\omega \chi_B} [{\bf n} \times {\bf\delta S}]_z
=  \frac{\gamma^2}{\omega \chi_B} (n_x \delta S_y - n_y \delta S_x)
\end{equation}
\begin{equation}\label{eq1:dl_harm}
L_j =
\frac{\chi_B H}{\gamma} R^0_{zj}
 + \frac{\omega - \gamma H}{\omega}\ \delta S_a R^0_{aj}
\end{equation}

One can see, that the motion of the orbital anisotropy axis $\bf L$ is small if
the precession frequency is close to $\gamma H$, and motion of~$\vartheta$
is is small if the texture is close to the Leggett configuration
($\bf n\parallel \hat z$).

In the equilibrium dipolar energy stays at the minimum, where $\ct=-1/4$
(so-called Leggett angle). In the spin wave it goes out of the minimum
and thus some additional energy appear. It depends on the orientation of~$\bf n$
and thus can modify equilibrium texture. This effect causes self-localization
of magnons~\cite{2012_PPD_Helsinki}. From the expression~(\ref{eq1:edip2}) for the dipolar energy:
\begin{equation}
F_D - F_D^0 = \frac{15}{2}g_D\Delta^2\ {(\bf n \cdot \ts)}^2
 = \frac{15}{2}g_D\Delta^2\ \delta\vartheta^2
\end{equation}
Average change in the energy, caused by the transverse spin wave can
be calculated using~(\ref{eq1:dtdn_harm}):
\begin{equation}
\langle F_D - F_D^0 \rangle =
\frac{15}{4} g_D\Delta^2
\left( \frac{\gamma H}{\omega}\right)^2
\sin^2\beta_N\ \beta_M^2
\end{equation}

Another useful formula is a relation between equilibrium textural angles
$\beta_N$ and $\beta_L$. In the equilibrium $\bf \hat S^0=\hat z$ and
\begin{equation}
\cos\beta_L = \hat L^0_z = R^0_{zz} = \ct + (1-\ct) \cos^2\beta_N.
\end{equation}
In the minimum of the dipolar energy
\begin{equation}\label{eq1:beta_l_n}
\sin^2\frac{\beta_L}{2} = \frac58 \sin^2\beta_N.
\end{equation}

%%%%%%%%%%%%%%%%%%%%%%%%%%%%%%%%%%%%%%%%%%%%%%%%%%%%%%%%%%%%%%%%%%%%%%
\subsection*{Quasiclassical equation for magnons}

Consider case of short spin waves, in which spin changes on much
smaller distances then the texture. We can represent such a wave as a set
of flat waves $s_q \propto \cos({\bf k}\cdot {\bf r})$ where wave
vector~$k$ also changes slowly. Substituting the flat wave
into~(\ref{eq1:tr_spinwaves}) and~(\ref{eq1:lo_spinwaves}) we get:
\begin{eqnarray}
\label{eq1:tr_spinwaves_qc}
 c_\perp^2\ {\bf k}^2 + (c_\parallel^2-c_\perp^2) ({\bf k}\cdot\hat {\bf L}^0)^2
+ \frac12 \Omega_B^2 \sin^2\beta_N
&=& \omega(\omega-\gamma H)\\
%%
\label{eq1:lo_spinwaves_qc}
 C_\perp^2 {\bf k}^2 + (C_\parallel^2-C_\perp^2) ({\bf k}\cdot\hat {\bf L}^0)^2
+ \Omega_B^2 \cos^2\beta_N &=& \omega^2
\end{eqnarray}

Here meaning of~$c_{\perp,\parallel}$ and~$C_{\perp,\parallel}$ becomes
clear. In a short-wave limit where magnetic field and spin-orbit
interaction are not important ($\omega\gg\Omega_B,\gamma H$) we have a
linear dispersion laws where~$c_{\perp,\parallel}$ are velocities of
transverse waves, propagating perpendicular and parallel to $\hat
{\bf L}^0$ direction; $C_{\perp,\parallel}$ are same velocities for
longitudinal waves.

The first equation describes transverse spin waves, which are similar to
that in other magnetic systems. In the presence of magnetic field it has two
solutions $\omega(k)$, which are called acoustic
(low $\omega$) and optical (high $\omega$) magnons. The second
equation for longitudinal waves is unique for $^3$He.

%%%%%%%%%%%%%%%%%%%%%%%%%%%%%%%%%%%%%%%%%%%%%%%%%%%%%%%%%%%%%%%%%%%%%%
\subsection*{Schr\"odinger equation for magnons}

Consider a long-wave optical magnons with frequencies
$\omega\approx\gamma H$ in the texture where~$\bf n$ is close to
vertical. In this case we can write equation for transverse spin
waves~(\ref{eq1:tr_spinwaves}) in a form of a Schr\"odinger equation for
magnon quasiparticles, where complex transverse spin $s_{+}$ plays role
of a wave function and precessing frequency $\omega$ plays role of an
energy. Effect of texture on the gradient terms is neglected because
it adds only a small correction to the total gradient energy. The dipolar
energy, which also depends on texture, can be of the same order as the total
gradient energy.
\begin{equation}\label{eq1:schred}
\left[
- \frac{c_\perp^2}{\gamma H}\ (\nabla_x^2+\nabla_y^2)
- \frac{c_\parallel^2}{\gamma H}\ \nabla_z^2
+ \frac{\Omega_B^2}{2\gamma H} \sin^2\beta_N + \gamma H
\right] s_{+} =
\omega\ s_{+}
\end{equation}
\begin{equation}
s_{+}
 = \frac1{\sqrt2}\ \frac{\chi_B H}{\gamma}\sin\beta_M\ e^{i(\alpha_M + \omega t)}
\end{equation}
Both texture (spatial distribution of $\beta_N$) and magnetic field form
a potential for magnons. It our experiment we used a combined effect of
a flare-out texture in the cylindrical cell and non-uniform field of
a small longitudinal coil to create a harmonic energy trap for magnons.

%Similar Schr\"odinger-like equations can be written also for the
%acoustic magnons and for the longitudinal mode, but in our experiments
%they are not useful.


%%%%%%%%%%%%%%%%%%%%%%%%%%%%%%%%%%%%%%%%%%%%%%%%%%%%%%%%%%%%%%%%%%%%%%
\subsection*{Magnons in a harmonic trap}

Schr\"odinger equation in a cylindrical harmonic potential can be solved analytically.
Let us write it in the form:
\begin{equation}\label{eq1:schred3d}
\left[
- \frac{\hbar^2}{2m_\perp}\ (\nabla_x^2 + \nabla_y^2)
- \frac{\hbar^2}{2m_\parallel}\ \nabla_z^2
+ \hbar\omega_0
+  \frac{m_\perp\omega_r^2}{2}\ r^2 +  \frac{m_\parallel\omega_z^2}{2}\ z^2
\right] s_{+} =
\hbar \omega\ s_{+}
\end{equation}
where parameters~$\omega_0$, $\omega_r$ and~~$\omega_z$ describes the potential and
masses~$m_{\perp,\parallel}$ in the case of magnons are:
\begin{equation}\label{eq1:magnon_mass}
m_{\perp,\parallel} = \frac{\hbar}{2 c_{\perp,\parallel}^2}\gamma H.
\end{equation}

Then eigenvalues are
\begin{equation}
\omega = \omega_0 + \omega_r(2n_r + |n_\phi| + 1) + \omega_z(n_z + 1/2),
\end{equation}
for $n_r,n_z=0,1,2\ldots$, $n_\phi=0,\pm1,\pm2\ldots$,
and  normalized solutions are:
\begin{equation}
s_{+} = s_+^r(r)\ s_+^\phi(\phi)\ s_+^z(z),
\end{equation}
where
\begin{eqnarray}
s_+^r(r) &=&
\frac{1}{a_r} \sqrt{ \frac{2\ n_r!}{(n_r + |n_\phi|)!} }
\ \left(\frac{r}{a_r}\right)^{|n_\phi|}
\exp\left(-\frac{r^2}{2a_r^2}\right)
L_{n_r}^{|n_\phi|}\left(\frac{r^2}{a_r^2}\right)\\
s_+^\phi(\phi) &=&
\ \frac{1}{\sqrt{2\pi}} \exp(i n_\phi \phi)\\
\nonumber
s_+^z(z) &=& 
\ \sqrt{\frac{1}{a_z \sqrt{\pi}\ 2^{n_z}\ n_z!}}
\ \exp\left(-\frac{z^2}{2a_z^2}\right)
H_{n_z}\left(\frac{z}{a_z}\right),
\nonumber
\end{eqnarray}
$L_n^m(x)$ and $H_n(x)$ are Laguerre and Hermite polynomials:
\begin{equation}
L_n^m(x) = \frac{1}{n!}\ x^{-m}\ e^x\ \frac{d^n}{dx^n} (x^{n+m}\ e^{-x}),\qquad
H_n(x) = (-1)^n\ e^{x^2}\ \frac{d^n}{dx^n}\ e^{-x^2},
\end{equation}
and $a_r$ and $a_z$ are sizes of the wave:
\begin{equation}\label{eq1:wave_size}
a_r=\sqrt{\frac{\hbar}{m_\perp\omega_r}} = c_\perp \sqrt{\frac{2}{\omega_r \gamma H}},\qquad
a_z=\sqrt{\frac{\hbar}{m_\parallel\omega_z}} = c_\parallel \sqrt{\frac{2}{\omega_z \gamma H}}.
\end{equation}
Usually we are interested only in non-antisymmetric states which
have non-zero integral transverse magnetization and can be excited and
observed in NMR experiments. For these states $n_\phi = 0$, $n_z =
0,2,4\ldots$, and
\begin{equation}\label{eq1:wave_form}
s_+(r,z) =
\ \left(\pi^{3/2} a_r^2 a_z \ 2^{n_z}\ n_z!\right)^{-1/2}
\ \exp\left(-\frac{r^2}{2a_r^2}-\frac{z^2}{2a_z^2}\right)
L_{n_r}^0\left(\frac{r^2}{a_r^2}\right)
H_{n_z}\left(\frac{z}{a_z}\right),
\end{equation}
\begin{equation}
\omega = \omega_0 + \omega_r(2n_r+1) + \omega_z(n_z+1/2),\qquad
n_r=0,1,2\ldots, n_z=0,2,4\ldots
\end{equation}

%%%%%%%%%%%%%%%%%%%%%%%%%%%%%%%%%%%%%%%%%%%%%%%%%%%%%%%%%%%%%%%%%%%%%%
%\subsection*{Effect of a non-uniform flare-out texture}
%
%In the center of harmonic trap for magnons gradient term in the equilibrium
%texture is not small compared to gradients in the long-wave magnons. But it
%is possible to show that it does not affect magnon frequency and quasiclassical
%approach can be used even for lowest magnon states in the trap.
%
%Near the center of the flare-out texture orbital vector $\bf L^0$ is tilted by a
%small angle. Using cylindrical coordinates ($z$, $r$, $\phi$), and
%introducing radial derivative of the tilting angle $\beta_L'$ one can
%write radial component of $\bf L^0$ as $L^0_r \approx \beta_L' r$.
%
% and assuming that
%$r\beta_L' \ll 1$ Then the gradient term
%from~(\ref{eq1:tr_spinwaves}) is
%\begin{equation}
%\nabla_j \hat L^0_j \hat L^0_k \nabla_k = \nabla_z^2 + \beta_L' (1 + 2 r\nabla_r)\nabla_z
%\end{equation}
%This term is not small in our geometry, but it does not shift
%eigenfrequencies of the original magnon equation. To see this one can do
%change of coordinates:
%\begin{equation}
%z \to z - \beta_L'\frac{c_\parallel^2 - c_\perp^2}{2c_\perp^2} r^2
%\end{equation}
%
%
%
%If size of the spin wave is much smaller then the size of the texture
%($r\beta_L' \ll 1$) we can throw away second order terms in $\beta_L'$ and
%write the magnon equation in the cylindrical coordinates as
%\begin{equation}\label{eq1:schred_ls}
%\left[
%- c_1\ \left(\frac1r \nabla_r + \nabla_r^2\right)
%- c_2\ \nabla_z^2
%- a c_1\ (1 + r\nabla_r)\nabla_z
%+ U(r,z)
%\right] s_{+} =
%\omega\ s_{+}
%\end{equation}
%with parameters
%\begin{equation}
%c_1 = \frac{c_\perp^2}{\gamma H}, \quad
%c_2 = \frac{c_\parallel^2}{\gamma H}, \quad
%a = 2\beta_L'\frac{c_\parallel^2 - c_\perp^2}{c_\perp^2},\quad
%U(r,z) = \frac{\Omega_B^2}{2\gamma H}\sin^2\beta_N + \gamma H
%\end{equation}

%Change of coordinates: $u  = z - a r^2/4, \quad v=r$:
%\begin{eqnarray*}
%\nabla_z   &=& \nabla_u\\
%\nabla^2_z &=& \nabla^2_u\\
%\nabla_r   &=& \nabla_v - av/2\ \nabla_u\\
%\nabla^2_r &=& \nabla^2_v - av\ \nabla_u\nabla_v - a/2\ \nabla_u + (av)^2/4\ \nabla^2_u\\
%\nabla_r\nabla_z &=& \nabla_v\nabla_u - av/2\ \nabla^2_u
%\end{eqnarray*}
%
%\begin{equation}\label{eq1:schred_ls}
%\left[
%- c_1\ \left( \frac1v\nabla_v + \nabla^2_v \right)
%- \left(c_2 - c_1 \frac{(av)^2}{4} \right)\ \nabla_u^2
%+ U(u,v)
%\right] s_{+} =
%\omega\ s_{+}
%\end{equation}
%
%
%According with the perturbation theory first order corrections
%to the magnon levels are:
%\begin{equation}
%\delta\omega  = \int_V s_{+}\ D\ s_{+}
%\end{equation}
%where $s_{+}$ is a normalized solution of the equation without the
%additional term.
%
%In harmonic approximation for first levels
%\begin{eqnarray}
%\delta\omega_{00} &=& \left(\frac{a_r^2}{2a_z^2} - 2\right)  \frac{c_\parallel^2 - c_\perp^2}{\gamma H}(\beta'_L)^2\\
%\delta\omega_{10} &=& \left(\frac{3a_r^2}{2a_z^2} - 6\right)  \frac{c_\parallel^2 - c_\perp^2}{\gamma H}(\beta'_L)^2\\
%\delta\omega_{02} &=& \left(\frac{5a_r^2}{a_z^2} - 2\right)  \frac{c_\parallel^2 - c_\perp^2}{\gamma H}(\beta'_L)^2
%\end{eqnarray}

%%%%%%%%%%%%%

% Let's look for a solution in a form:
% \begin{equation}
% s_+ = s_z(z - ar^2/4)\ s_r(r),
% \end{equation}
% 
% % \begin{eqnarray*}
% \nabla_z s &=& s_z' s_r\\
% \nabla_r s &=&  s_z s_r' - ar/2 s_z' s_r\\
%
% 1/r\ \nabla_r (r \nabla_r) s &=& 
% 1/r\ s_z s_r'
% - a\ s_z' s_r
% - ar\ s_z' s_r'
% + s_z s_r''
% + (ar)^2/4 s_z'' s_r\\
%
% \nabla_z^2 s &=& s_z'' s_r\\
% %%
% (1 +  r\nabla_r)\nabla_z s &=&
% s_z' s_r - ar^2/2\ s_z''s_r
% +r\ s_z's_r'
% \end{eqnarray*}
% \begin{equation}
% \frac{1}{\gamma H}\left[
% - c_\perp^2 [s_r'' + s_r'/r] s_z
% - [c_\parallel^2 - c_\perp^2 (ar)^2/4] s_z'' s_r
% + (\frac12(\Omega_B\beta_N')^2 r^2 + (\gamma H)^2) s_r s_z
% \right] =
% \omega\ s_{+}
% \end{equation}

%%%%%%%%%%%%%


%$$
%c_x \nabla_x^2 u + c_z \nabla_z^2 u + b\nabla_z u + a u = 0
%$$
%$$
%u' = u \exp\left(-\frac{b z}{2 c_z}\right)
%$$
%$$
%c_x \nabla_x^2 u' + c_z \nabla_z^2 u' + \left(a-\frac{b^2}{2 c_z}\right) u = 0
%$$
