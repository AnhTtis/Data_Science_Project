\documentclass{article}
\usepackage[a4paper]{geometry}
\usepackage{amssymb}
\usepackage{euscript}
\usepackage{graphicx}
\usepackage{wrapfig}
\usepackage{color}
\usepackage{epsfig}
\usepackage{fullpage}
\pagestyle{empty}


\newcommand{\image}[3]{
\begin{figure}[#1]
\begin{center}
\includegraphics{img_#2.pdf}
\caption{\small#3}
\label{image:#2}
\end{center}
\end{figure}
}

\newcommand{\wrapimage}[4]{
\begin{wrapfigure}{#1}{#2}
\begin{center}
\includegraphics{img_#3.pdf}
\caption{\small#4}
\label{image:#3}
\end{center}
\end{wrapfigure}
}

\def\ts{\theta^s}
\def\ta{\theta_A}
\def\tb{\theta_B}
\def\ct{\cos\vartheta}
\def\st{\sin\vartheta}
\def\cts{\cos^2\vartheta}
\def\sts{\sin^2\vartheta}
\def\cpa{c_{\parallel}}
\def\cpe{c_{\perp}}

\usepackage[colorlinks=true,hypertexnames=true]{hyperref}
\usepackage[style=nature,backend=bibtex,doi=false,isbn=false,url=true,alldates=year]{biblatex}

%%%%%%%%%%%%%%%%%%%%%%%%%%%%%%
\addbibresource{bib/wv.bib} %
\addbibresource{bib/haar_mech.bib} %
\addbibresource{bib/he3lib.bib} %
\addbibresource{bib/cadabra.bib} %
\addbibresource{bib/1975_he3_teor_leggett.bib} %
\addbibresource{bib/1975_cross.bib} %
\addbibresource{bib/1976_maki_solitons.bib} %
\addbibresource{bib/1977_leggett_takagi.bib} %
\addbibresource{bib/1980_jetp_fomin_spinwaves.bib} %
\addbibresource{bib/1980_pbrackets.bib} %
\addbibresource{bib/1983_theodorakis.bib} %
\addbibresource{bib/1985_hpd_e.bib} %
\addbibresource{bib/1985_hpd_f_e.bib} %
\addbibresource{bib/1986_rozhkov_solitons.bib} %
\addbibresource{bib/1992_mis_hpd_topol.bib} %
\addbibresource{bib/2004_bunk_dyn.bib} %
\addbibresource{bib/2006_jetpl_catrel.bib} %
\addbibresource{bib/2012_PPD_Helsinki.bib} %
\addbibresource{bib/2022_zav_theta_hpd.bib} %

%%%%%%%%%%%%%%%%%%%%%%%%%%%%%%

\begin{document}
\title{NMR in $^3$He-B}
\author{V.~V.~Zavjalov\/\thanks{e-mail: v.zavjalov@lancaster.ac.uk}}

\date{\today}
\maketitle

\begin{abstract}
This text contains a collection of equations useful for understanding
Nuclear Magnetic Resonance (NMR) experiments in superfluid $^3$He-B. This
is a part of my notebook where I try to describe some parts of this
sophisticated system.
\end{abstract}

\section*{Introduction}

Nuclear magnetic resonance (NMR) in superfluid $^3$He is a very powerful
tool used for studying this system since its discovery in 1972.
Superfluidity in fermionic $^3$He is formed via Cooper pairing of atoms, with
pairs having spin 1 and orbital momentum 1. As a result the order
parameter of the system includes both spin and orbital degrees of freedom
and can be written as 3x3 complex matrix. A few different superfluid
phases with different broken symmetries are possible. For the B-phase
degenerate space of the order parameter includes an arbitrary 3D rotation
matrix which describes fixed mutual orientation of spin and orbital
spaces. In NMR experiments we observe motion of this matrix, which means
three degrees of freedom and three spin-wave modes. This is different
from many other magnetic materials where we study motion of magnetization
vector with only two degrees of freedom. Another important feature of
this system is possibility of non-uniform spatial distribution and
topological defects in the order-parameter field (so-called textures).
This can be clearly observed in linear NMR where small oscillations of
the order parameter are happening around the equilibrium texture. One
more feature of this system is spin-orbit interaction. It introduces an
additional non-linear force acting on the order parameter and leads to
important effects such as longitudinal NMR or Homogeneously Precessing
Domain (HPD).

This text contains two almost independent parts. The first one is about
linear NMR. It was written during my work in Vladimir Eltsov's group in
Aalto University (Finland) in 2012-2015. We had a really nice
experimental system for measuring optical magnons trapped in a harmonic
potential formed by the order parameter texture and magnetic field. It was
possible to control shape of the trap and population of individual levels
in it, to observe Bose condensation of magnons, parametric excitation of
other spin-wave modes, interaction of the magnon condensate with texture,
quantized vortices, and free surface of helium. I would like to thank
Petri Heikkinen, Samuli Autti, and Jere M\"{a}kinen who also worked on this
project.

The second part contains Leggett equations for non-linear
NMR and many things related to Homogeneously
Precessing Domain (HPD), a unique coherent
state which is also a very useful tool
for various $^3$He studies. I worked with HPD in Vladimir Dmitriev's
group in Kapitza Institute (Russia) in 2000-2006, and in Pertti Hakonen's
group in Aalto University in 2016-2019.
%cite{1975_he3_teor_leggett}
%cite{1985_hpd_f_e,1985_hpd_e}


The text contains many references to Vollhardt and W\"olfle
book~\cite{WV}, usually written as (WV \textless equation
number\textgreater). Many derivations require straightforward but huge
tensor manipulations. A useful tool for this kind of calculations is {\tt
cadabra}~\cite{cadabra}. Some of $^3$He-related parameters and equations
can be found in my {\tt he3lib} library~\cite{he3lib}.

I hope this work will be useful for understanding NMR experiments in $^3$He-B.


\eject
\input He3B_01_linear

\eject
\input He3B_02_hpd

\eject
\printbibliography

\end{document}
