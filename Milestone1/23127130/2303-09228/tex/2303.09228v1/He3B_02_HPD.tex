%%%%%%%%%%%%%%%%%%%%%%%%%%%%%%%%%%%%%%%%%%%%%%%%%%%%%%%%%%%%%%%%%%%%%%
\section*{Part 2. Non-linear spin dynamics}

\subsection*{Leggett equations in {}$R_{aj}$ coordinates}

In this part non-linear Leggett equations of spin dynamics in
$^3$He-B~\cite{1975_he3_teor_leggett} are derived. We start from same
Hamiltonian as in the first part and use components of matrix~$R_{aj}$ as
coordinates.

Order parameter of $^3$He-B is
\begin{equation}
A_{aj}  = \frac{1}{\sqrt{3}}\ \Delta\ e^{i\varphi} R_{aj},
\end{equation}
where~$\varphi$ is the phase, and $R_{aj}$
is a rotation matrix with axis~$\bf n$ and rotation angle~$\vartheta$:
\begin{equation}\label{eq:r_nt}
R^0_{a j} = \ct\ \delta_{a j} + (1-\ct)\ n_a n_j - \st\ e_{ajk} n_k.
\end{equation}
Hamiltonian of this system is:
\begin{equation}\label{eq:ham}
\mathcal{H} = F_M + F_{SO} + F_\nabla,
\end{equation}
\begin{eqnarray}
\label{eq:en_m0}
F_M &=& - ({\bf S} \cdot \gamma {\bf H})
+ \frac{\gamma^2}{2\chi_B}{\bf S}^2,\\
\label{eq:en_d0}
F_{SO}
&=& g_D\Delta^2 \Big[
           R_{jj}R_{kk}
         + R_{jk}R_{kj}
 - \frac23 R_{jk}R_{jk}\Big]
= g_D\Delta^2 \Big[ R_{jj}R_{kk} + R_{jk}R_{kj}] + \mbox{const.}
,\\
\label{eq:en_g0}
F_\nabla
&=& \frac12 \Delta^2 \Big[
  K_1 (\nabla_j R_{ak})(\nabla_j R_{ak})
+ K_2 (\nabla_j R_{ak})(\nabla_k R_{aj})
+ K_3 (\nabla_j R_{aj})(\nabla_k R_{ak}) \Big].
\end{eqnarray}
Equations of motion are
\begin{eqnarray}\label{eq:shs}
\dot S_a
&=& \{\mathcal{H}, S_a\}
\quad=\quad
  \frac{\delta \mathcal{H}}{\delta S_b} \{S_b,S_a\}
+ \frac{\delta \mathcal{H}}{\delta R_{bj}} \{R_{bj},S_a\}
\quad=\quad
  \frac{\delta \mathcal{H}}{\delta S_b} e_{abc} S_c
+ \frac{\delta \mathcal{H}}{\delta R_{bj}} e_{abc} R_{cj},
\\\label{eq:shr}
\dot R_{aj}
&=& \{\mathcal{H}, R_{aj}\}
\quad=\quad
  \frac{\delta \mathcal{H}}{\delta S_b} \{S_b,R_{aj}\}
+ \frac{\delta \mathcal{H}}{\delta R_{bk}} \{R_{bk},R_{aj}\}
\quad=\quad
  \frac{\delta \mathcal{H}}{\delta S_b} e_{abc} R_{cj}.
\end{eqnarray}
The derivatives of the Hamiltonian are much simpler then in the case
of~$\ts$ angles. There are no commutation problem in the variational
derivative. The result is:
\begin{eqnarray}
\frac{\delta \mathcal{H}}{\delta {\bf S}} &=&
- \gamma {\bf H} + \frac{\gamma^2}{\chi_B}{\bf S},
\\
\frac{\delta \mathcal{H}}{\delta R_{bj}}
 &=& \Delta^2 \big[
  2 g_D(\delta_{bj}R_{kk} + R_{jb} - 2/3 R_{bj})
- K_1 (\nabla_k \nabla_k R_{bj})
- (K_2+K_3) (\nabla_k \nabla_j R_{bk})
\big].
\end{eqnarray}
Substituting this into equations of motion~(\ref{eq:shs},\ref{eq:shr}) we obtain
Leggett equations:
\begin{eqnarray}\label{eq:leggett_r}
\dot S_a &=&
  [{\bf S}\times \gamma {\bf H}]_a + T^D_a + T^\nabla_a,
\\\nonumber
\dot R_{aj} &=&
  e_{abc} R_{cj} \Big(\frac{\gamma^2}{\chi_B} {\bf S} - \gamma {\bf H} \Big)_b,
\end{eqnarray}
with dipolar and gradient torques:
\begin{eqnarray}\label{eq:torqueR}
T^D_a
 &=& e_{abc} \frac{\delta F_D}{\delta R_{bj}} R_{cj}
 \quad=\quad
  2 \Delta^2 g_D\ e_{abc} R_{cj} (\delta_{bj}R_{kk} + R_{jb}),
\\\nonumber
T^\nabla_a
 &=& e_{abc} \frac{\delta F_\nabla}{\delta R_{bj}} R_{cj}
 \quad=\quad
 - \Delta^2 e_{abc} R_{cj} \big[
     K_1 (\nabla_k \nabla_k R_{bj})
    + (K_2+K_3) (\nabla_k \nabla_j R_{bk}) \big].
\end{eqnarray}
The gradient torque can be rewritten in terms of spin current~$J_{ak}$ which
carries $a$ component of the spin in the direction $k$:
\begin{equation}\label{eq:spin_curr}
T^\nabla_a = - \nabla_k J_{ak},
\qquad
J_{ak} = \Delta^2  e_{abc} R_{cj} \big[ K_1 (\nabla_k R_{bj}) + (K_2+K_3) (\nabla_j R_{bk}) \big].
\end{equation}
Here we used equality~$e_{abc}\nabla_k R_{bj}\nabla_k R_{cj} =
e_{abc}\nabla_k R_{bj}\nabla_j R_{ck} = 0$.

It is also interesting to study motion of orbital anisotropy
vector~$L_j = S_aR_{aj}$:
\begin{equation}
\dot L_k
\quad=\quad
\{\mathcal{H},  S_aR_{ak}\}
\quad=\quad
\{\mathcal{H}, S_a\} R_{ak} +
\{\mathcal{H}, R_{ak}\} S_a 
\quad=\quad
\dot S_a R_{aj} + \dot R_{aj} S_a.
\end{equation}
%There is an idea that in hydrodynamic approximation~$\bf L$ is
%an independent variable and additional equations for ``orbital waves'' are
%needed. There is a discussion in Bunkov's paper about numerical simulation,
%but there he neglected all additional terms in the Hamiltonian and came
%(as I understand) to the ``simple'' case of ${\bf L}=R {\bf S}$.
Substituting equations for $\dot S_a$ and $\dot R_{aj}$ we have:
\begin{eqnarray}
\dot L_k &=& (T^D_a + T^\nabla_a) R_{ak}.
\end{eqnarray}


%%%%%%%%%%%%%%%%%%%%%%%%%%%%%%%%%%%%%%%%%%%%%%%%%%%%%%%%%%%%%%%%%%%%%%
\subsection*{Leggett equations in ${\bf n}$ and $\vartheta$ coordinates} %$

Let's rewrite Leggett equations in ${\bf n}$ and $\vartheta$ coordinates
(axis and angle or rotation of the matrix $R_{aj}$).

Time derivative of matrix~$R({\bf n}, \vartheta)$ is:
\begin{equation}
\dot R_{aj} = \big[ \st\ (n_a n_j - \delta_{aj}) - \ct\ e_{ajk} n_k \big]\,\dot \vartheta
 + (1-\ct) [ n_j \dot n_a + \ n_a \dot n_j ] - \st\ e_{ajk} \dot n_k.
\end{equation}
Let's calculate following combinations:
\begin{eqnarray}
n_a \dot R_{aj} + n_a \dot R_{ja} &=& 2 (1-\ct)\ \dot n_j,
\\\nonumber
R_{ja} \dot R_{aj} &=& -4\ct\st\ \dot\vartheta.
\end{eqnarray}
By substituting $R$ from~(\ref{eq:r_nt}) and
$\dot R$ from~(\ref{eq:leggett_r}) we can write Leggett equations in $\dot n$ and $\dot\theta$ coordinates:
\begin{eqnarray}\label{eq:leggett_nt}
\dot {\bf S} &=& {\bf S}\times \gamma {\bf H}  + {\bf T}^D - \nabla_k J_{ak},
\\\nonumber
{\bf\dot n} &=&
-\frac12\ {\bf n}\times \Big[
\Big(\frac{\gamma^2}{\chi_B} {\bf S} - \gamma {\bf H} \Big)
+ \frac{\st}{1-\ct}
\ {\bf n}\times \Big(\frac{\gamma^2}{\chi_B} {\bf S} - \gamma {\bf H} \Big)
\Big],
\\\nonumber
\dot\vartheta &=&  {\bf n} \cdot \Big(\frac{\gamma^2}{\chi_B} {\bf S} - \gamma {\bf H} \Big),
\end{eqnarray}
Expression for dipolar torque via ${\bf n}$ and
$\vartheta$:
\begin{equation}
{\bf T}^D = 4 \Delta^2 g_D\ \st(4\ct + 1)\ {\bf n}.
\end{equation}
%%%%%%%%%%%%%%%%%%%%%%%%%%%%%%%%%%%%%%%%%%%%%%%%%%%%
% J_{ak} =  \Delta^2 K1         e_{abc} R_{cj} (\nabla_k R_{bj})
% J_{ak} =  \Delta^2 (K_2+K_3)  e_{abc} R_{cj} (\nabla_j R_{bk})
% 
% \nabla_k R_{bj} =
%  + \st n_b n_j \nabla_k\vartheta
%  - \st \delta_{bj} \nabla_k\vartheta
%  - \ct e_{bjl} n_l \nabla_k\vartheta
%  + (1-\ct) n_j \nabla_k n_b
%  + (1-\ct) n_b \nabla_k n_j
%  - \st e_{bjl} \nabla_k n_l
% 
% \nabla_j R_{bk} =
%  + \st n_b n_k \nabla_j\vartheta
%  - \st \delta_{bk} \nabla_j\vartheta
%  - \ct e_{bkl} n_l \nabla_j\vartheta
%  + (1-\ct) n_k \nabla_j n_b
%  + (1-\ct) n_b \nabla_j n_k
%  - \st e_{bkl} \nabla_j n_l
% 
% R_{cj} = \ct \delta_{cj} + (1-\ct) n_c n_j - \st e_{cjm}n_m
%%%%%%%%%%%%%%%%%%%%
% ! checked 2 times!
Spin current:
\begin{eqnarray}
J_{ak} &=&  2\Delta^2 K1 \big[
  - n_a \nabla_k\vartheta
  - \st\ \nabla_k n_a
  + (1-\ct)\ e_{abc} n_c \nabla_k n_b
%
%
\big]\\\nonumber
&+& \Delta^2 (K_2+K_3) \big[
 -             n_a             \nabla_k\vartheta
 + \ct\        \delta_{ak} n_j \nabla_j\vartheta
 + (1-\ct)\    n_k n_j n_a     \nabla_j\vartheta
\\\nonumber
&&
+ \ct\st\     e_{abj} n_b n_k \nabla_j\vartheta
+ (1-\ct)\st\ e_{abk} n_b n_j \nabla_j\vartheta
\\\nonumber
&&
 - \ct\st\     e_{kbj} n_b n_a \nabla_j\vartheta
 - \ct\st\     e_{akj}         \nabla_j\vartheta
\\\nonumber
&&
 + \st\ct         \delta_{ak} \nabla_j n_j
 + (1-\ct)\st     n_a n_k     \nabla_j n_j
 + (1-\ct)\st     n_a n_j     \nabla_j n_k
\\\nonumber
&&
 - (1-\ct)\st     n_k n_j     \nabla_j n_a
 - \sts           \delta_{ak} e_{cjb} n_b \nabla_j n_c
 + (1-\ct)^2      e_{abc} n_c n_j n_k \nabla_j n_b
\\\nonumber
&&
 + (1-\ct)\ct     e_{abj} n_k \nabla_j n_b
 + (1-\ct)\ct     e_{abj} n_b \nabla_j n_k
 - \ct\st                     \nabla_k n_a
\\\nonumber
&&
 - (1-\ct)\st                 \nabla_a n_k
 + \sts           e_{kjb} n_b  \nabla_a n_j
\big].
\end{eqnarray}
Expression for gradient torque can be obtain by finding spin current derivative~(\ref{eq:spin_curr}).

%%%%%%%%%%%%%%%%%%%%%%%%%%%%%%%%%%%%%%%%%%%%%%%%%%%%
%% e_cba n_b J_{ak}  =  \Delta^2 K1
%  - 2 \st\  e_cba n_b \nabla_k n_a
%  + 2(1-\ct)\ \nabla_k n_c
%
%%        +  \Delta^2 (K_2+K_3)
%+ \ct        e_cbk n_b n_j \nabla_j\vartheta
%+            n_k n_c n_j \nabla_j\vartheta
%-            \delta_{ck} n_j \nabla_j\vartheta
%
%+ \st\ct         e_cbk n_b \nabla_j n_j
%- (1-\ct)\st     e_cba n_b n_k n_j     \nabla_j n_a
%+ (1-\ct)(1-\ct) e_cba n_b e_{ade} n_e n_j n_k \nabla_j n_d
%+ (1-\ct)\ct     e_cba n_b e_{adj} n_k \nabla_j n_d
%+ (1-\ct)\ct     e_cba n_b e_{adj} n_d \nabla_j n_k
%- \ct\st         e_cba n_b             \nabla_k n_a
%- \st(1-\ct)     e_cba n_b             \nabla_a n_k
%+ \st\st         e_cba n_b e_{dke} n_d \nabla_a n_e
%% J_{ak} n_a =  \Delta^2 K1
%  - 2 \nabla_k\vartheta
%
%%        +  \Delta^2 (K_2+K_3)
%+ n_k n_j \nabla_j\vartheta
%-         \nabla_k\vartheta
%
%+ \st            n_k     \nabla_j n_j
%- \st\st         e_{jkc} \nabla_j n_c
%+ (1-\ct)\ct     e_{abj} n_a n_k \nabla_j n_b
%+ \st\st         e_{bkc} n_a n_b \nabla_a n_c           +


%% J_{ak} n_a n_k =  \Delta^2 K1
%  - 2 n_k \nabla_k\vartheta
%
%%        +  \Delta^2 (K_2+K_3)
%+ \st       \nabla_j n_j
%- (1-\ct)   e_{jkc} n_k \nabla_j n_c


%%%%%%%%%%%%%%%%%%%%
%% n_j = \delta_{j3}, \nabla_j = \delta_{j3} \nabla_3, n'=0
%% J_{a3} = - 2 \Delta^2 K1 \delta_{a3} \vartheta'

%%%%%%%%%%%%%%%%%%%%
%% theta'=0, \nabla_j = \delta_{j3}, k=3

%% J_{a3} =  \Delta^2 K1
%  - 2 \st\ n_a'
%  + 2(1-\ct)\ e_{abc} n_c n_b'
%
%%        +  \Delta^2 (K_2+K_3)
%- (1-2\ct)\st    \delta_{a3} n_3'
%+ 2(1-\ct)\st     n_a n_3     n_3'
%- (1-\ct)\st     n_3^2       n_a'
%+ (1-\ct)(1-\ct) n_3^2 e_{abc} n_c  n_b'
%+ (1-\ct)\ct     e_{ab3} n_3 n_b'
%+ (1-\ct)\ct     e_{ab3} n_b n_3'
%- \ct\st                     n_a'
%+ \st\st         \delta_{a3} e_{cb3} n_b n_c'

%%%%%%%%%%%%%%%%%%%%%%%%%%%%%%%%%%%%%%%%%%%%%%%%%%%%%%%%%%%%%%%%%%%%%%
\subsection*{
Spin created by motion of ${\bf n}$ and $\vartheta$
}

It could be useful to solve second and third Leggett equations~(\ref{eq:leggett_nt}) for spin. We use
temporary notation
\begin{equation}
{\bf s} = \frac{\gamma^2}{\chi_B} {\bf S} - \gamma
{\bf H},
\qquad
C = \frac{\st}{1-\ct}.
\end{equation}
Then the equations are:
\begin{equation}
{\bf\dot n} =
-\frac12\ {\bf n}\times \Big[
{\bf s}
+ C\ {\bf n}\times {\bf s}
\Big],
\qquad
\dot\vartheta =  {\bf n} \cdot {\bf s}
\end{equation}
Let's solve them for {\bf s}. Solution can be written as ${\bf s} = {\bf
n}\dot\vartheta + {\bf n}\times{\bf A}$ with ${\bf A}\cdot{\bf n} = 0$.
This satisfies the equation for~$\dot\vartheta$ and splits vector~$\bf s$ into two
parts, parallel and perpendicular to~$\bf n$.
Substitute~$\bf s$ into the equation for~$\bf\dot n$:
\begin{equation}
{\bf A} + C {\bf n}\times{\bf A} = 2 {\bf\dot n},
\end{equation}
find vector product of this equation with $\bf n$, multiply it by $C$ and
subtract from original equation. From this we can find
${\bf A} = (1-\ct)\ {\bf\dot n} - \st\ {\bf n}\times{\bf\dot n}$
and get the result:
\begin{equation}\label{eq:S_dndt}
\frac{\gamma^2}{\chi_B} {\bf S} =
\gamma {\bf H} + {\bf n}\ \dot\vartheta + (1-\ct)\ {\bf n}\times{\bf\dot n} + \st\ {\bf\dot n}.
\end{equation}
Here one can see how spin is created by magnetic field and motion of ${\bf n}$ and
$\vartheta$.

There is an interesting observation: if we have a soliton moving with a
constant velocity, we can have a uniform spin distribution in the moving
frame, but non-uniform in the static one.


%%%%%%%%%%%%%%%%%%%%%%%%%%%%%%%%%%%%%%%%%%%%%%%%%%%%%%%%%%%%%%%%%%%%%%
\subsection*{
Leggett equations for ${\bf n}$ and $\vartheta$ in the rotating frame
}

For numerical integration of Leggett equations~(\ref{eq:leggett_nt}) it is
convenient to switch to a ``rotating frame'' by doing substitution
\begin{equation}
S^r_{a'} = U_{a'a} S_a,\quad
T^r_{a'} = U_{a'a} T_a,\quad
R^r_{a'k'} = U_{k'k} U_{a'a} R_{ak},\quad\ldots
\end{equation}
with rotation matrix
$U_{jk} = R_{jk}(\omega t, {\bf\hat z}) = \cos\omega t\ \delta_{j k} + (1-\cos\omega t)\ {\bf\hat z}_j {\bf\hat z}_k - \sin\omega t\ e_{jkl}{\bf\hat z}_l$.

For me it is still not clear is this ``rotating frame'' has a physical
meaning (for example in 3D space we do not specify any axis of rotation).
Anyway, we can always do the substitution, solve equations and return
back to original vectors if needed.

By making the substitution in~(\ref{eq:leggett_r}) and using facts that
$U{\bf a}\cdot U{\bf b} = {\bf a\cdot b}$, $[U{\bf a}\times U{\bf b}] = U
[{\bf a}\times {\bf b}]$ and $U \dot U^{-1} {\bf a} = \omega [{\bf a}
\times {\bf\hat z}]$ we have equations in the rotating frame:
\begin{eqnarray}\label{eq:r_motion_rot}
 \dot {\bf S}^r + \omega [{\bf S}^r \times {\bf\hat z}]  &=&
  [{\bf S}^r\times \gamma {\bf H}^r] + {\bf T}^{Dr} + {\bf T}^{\nabla r}
\\\nonumber
{\bf\dot n}^r + \omega [{\bf n}^r \times {\bf\hat z}] &=&
-\frac12\ {\bf n}^r\times \Big [ \Big(\frac{\gamma^2}{\chi_B} {\bf S}^r - \gamma {\bf H}^r \Big)
+ \frac{\st}{1-\ct}\ {\bf n}^r\times \Big(\frac{\gamma^2}{\chi_B} {\bf S}^r - \gamma {\bf H}^r \Big)
\Big]
\\\nonumber
\dot\vartheta &=& {\bf n}^r \cdot \Big(\frac{\gamma^2}{\chi_B} {\bf S}^r - \gamma {\bf H}^r \Big).
\end{eqnarray}
with dipolar torque
\begin{eqnarray}\label{eq:td_nt}
{\bf T}^{Dr} = 4 \Delta^2 g_D\ \st(4\ct + 1)\ {\bf n}^r
\end{eqnarray}
Writing the gradient torque in the rotating frame is not trivial, because
matrix $U$ will act on gradient operators. To avoid this difficulty we
will be solving only 1D problems with all gradients directed along
${\bf\hat z}$ axis. Then spin current is not affected by rotation and we
can simply write:
\begin{equation}
{\bf T}^{\nabla r} = -\nabla_3 J_{a3} = -{\bf J}',
\qquad
J_{a} =
 \Delta^2  e_{abc} R_{cj} \big[ K_1 R_{bj}' + (K_2+K_3) R_{b3}'\delta_{j3} \big],
\end{equation}
using primes for $\nabla_3$. Usually we assume that magnetic field in the rotating frame~${\bf H^r}$
is constant and choose axis~$\hat {\bf x}$ in such a way
that field is in $x-z$ plane:
\begin{equation}
\label{eq:def_first}
  \gamma{\bf H^r} = {\bf\hat z}\,\omega_z + \hat {\bf x}\,\omega_x
\end{equation}
This is a good description of a usual continuous-wave NMR experiments with
static field $\gamma H_z = \omega_z$ and rotating radio-frequency
field $\gamma H_x = \omega_x$.

%%%%%%%%%%%%%%%%%%%%%%%%%%%%%%%%%%%%%%%%%%%%%%%%%%%%%%%%%%%%%%%%%%%%%%
\subsection*{Spin in the rotating frame}

In the following calculations we will work only with vectors in the
rotating frame and omit indices~``$r$''.
It's convenient to introduce a dimensionless vector~$\bf w$:
\begin{equation}\label{eq:w}
{\bf w}
\ =\ %
\frac{1}{\omega}\Big(\frac{\gamma^2}{\chi_B} {\bf S} - \gamma {\bf H} \Big) + {\bf\hat z}
\end{equation}
Then Leggett equations will be written as
\begin{eqnarray}\label{eq:r_motion_wnt_eq}
  \dot{\bf w}  &=&
  {\bf w}\times ({\bf\hat z}\,(\omega_z-\omega) + \hat {\bf x}\,\omega_x)
 + \frac{\gamma^2}{\omega\chi_B}\left({\bf T}^{D} + {\bf T}^{\nabla}\right)
\\\nonumber
{\bf\dot n} / \omega &=&
-\frac12\ {\bf n}\times \Big[
( {\bf w} + {\bf\hat z})
+ \frac{\st}{1-\ct}
\ {\bf n}\times ({\bf w} - {\bf\hat z})
\Big]
\\\nonumber
\dot\vartheta / \omega &=& {\bf n} \cdot ({\bf w} - {\bf\hat  z}).
\end{eqnarray}
We can solve second and third equations for~$\bf w$ in the same way as~(\ref{eq:S_dndt}):
\begin{eqnarray}
{\bf w} &=&
  \ct\ {\bf\hat z}
+ (1-\ct)\ {\bf n}\ n_z
+ \st\ {\bf n}\times {\bf\hat z}
\\\nonumber
&+&
  \frac1{\omega}\Big[ {\bf n}\ \dot\vartheta
+ (1-\ct)\ {\bf n}\times{\bf\dot n}
+ \st\ {\bf\dot n}
\Big]
\end{eqnarray}
This can be also written as:
\begin{equation}\label{eq:wr_dndt}
{\bf w} = R{\bf\hat z}
+ \frac1{\omega}\left[ {\bf n}\ \dot\vartheta
+ \frac{1-\ct}{\st}\ (R{\bf\dot n} + {\bf\dot n})
\right]
\end{equation}
Or in original notation:
\begin{equation}
\frac{\gamma^2}{\chi_B} {\bf S}
=
\gamma {\bf H}
+ {\bf n}\ \dot\vartheta
+ (1-\ct)\ {\bf n}\times[{\bf\dot n} + \omega {\bf n}\times{\bf\hat z}]
+ \st\ [{\bf\dot n} + \omega {\bf n}\times {\bf\hat z}]
\end{equation}
\begin{equation}\label{eq:Sr_dndt}
\frac{\gamma^2}{\chi_B} {\bf S}
=
\gamma {\bf H}
+ \omega\ (R{\bf\hat z}-{\bf\hat z})
+ {\bf n}\ \dot\vartheta
+ \frac{1-\ct}{\st}\ (R{\bf\dot n} + {\bf\dot n})
\end{equation}


\subsection*{Non-uniform equilibrium in the rotating frame}

Equilibrium state can be found by setting all time derivatives to zero.
From~(\ref{eq:wr_dndt}) we have
\begin{equation}\label{eq:w_n}
w_a = R_{a3},
\end{equation}
it means that in the equilibrium $|{\bf w}| = 1$.

Using this equation we can exclude $\bf w$ from the first equation and
write equilibrium problem using only $R_{aj}$ (or $\bf n$ and $\vartheta$) coordinates:
\begin{equation}\label{eq:hpd_texture1}
e_{abc} R_{bj} \Big[
\frac{\omega\chi_B}{\gamma^2 \Delta^2}\left(\delta_{c3}\,(\omega_z-\omega) + \delta_{c1}\,\omega_x\right) \delta_{j3}
-  2g_D
\ (\delta_{cj}R_{kk} + R_{jc})
+ K_1 R_{cj}'' + (K_2+K_3)\delta_{j3} R_{c3}''
\Big] = 0.
\end{equation}
Here we use expressions~(\ref{eq:torqueR}) for dipolar and gradient
torques and keep only gradients in $\bf\hat z$ direction, represented by
primes.

This can be written in term of Leggett frequency and spin-wave velocities:
\begin{equation}\label{eq:hpd_texture2}
e_{abc} R_{bj} \Big[
\omega(\left(\omega_z-\omega)\ \delta_{c3} + \omega_x\ \delta_{c1}\right)\delta_{j3}
-  \frac{2}{15}\Omega_B^2 \ (\delta_{cj}R_{kk} + R_{jc})
+ \frac12 C_{\parallel}^2 R_{cj}''
+ (C_{\perp}^2 - C_{\parallel}^2)\delta_{j3} R_{c3}''
\Big] = 0.
\end{equation}
This is a set of three equations for three parameters~($\bf n$ and $\vartheta$). It
describes non-uniform texture of a precessing state and
can be used for calculating HPD boundaries or solitons.

%

% T^M_a = e_{abc} R_{b3} (\gamma H_c - \omega z_c)

% T^D_a = 4 \Delta^2 g_D \st(4\ct+1) n_a

% T^{\nabla}_a = -\delta^2 e_{abc} [K1 R_{bj}'' + (K_2+K_3)R_{b3}'']






%=================================

% (e_{abc} R_{bj} X_{cj})^2 =
% = e_{abc} R_{bj} X_{cj} e_{ade} R_{dk} X_{ek}
% =  R_{bj} X_{cj} ( R_{bk} X_{ck} - R_{ck} X_{bk})

% R_{bj} X_{cj} = 

%A = \frac{\omega\chi_B}{\gamma^2 \Delta^2}(\omega_z-\omega)
%B = \frac{\omega\chi_B}{\gamma^2 \Delta^2} \omega_x
%C = 2g_D
%D = K_1
%E = (K_2+K_3)

%(\ct \delta_{bj} + (1-\ct) n_b n_j - \st e_{bjl} n_l)
%(A \delta_{c3}\delta_{j3} + B \delta_{c1}\delta_{j3} - C (\delta_{cj}R_{kk} + R_{jc}) +  D R_{cj}'' + E \delta_{j3} R_{c3}'')

% R_{bj} X_{cj} = 
%+ \ct \delta_{b3} A \delta_{c3}
%+ \ct \delta_{b3} B \delta_{c1}
%- \ct C \delta_{cb}
%- \ct C R_{bc}
%+ \ct D R_{cb}''
%+ \ct \delta_{b3} E R_{c3}''
%+ (1-\ct) n_b n_3 A \delta_{c3}
%+ (1-\ct) n_b n_3 B \delta_{c1}
%- (1-\ct) n_b n_c C R_{kk}
%- (1-\ct) n_b n_j C R_{jc}
%+ (1-\ct) n_b n_j D R_{cj}''
%+ (1-\ct) n_b n_3 E R_{c3}''
%- \st e_{b3l} n_l A \delta_{c3}
%- \st e_{b3l} n_l B \delta_{c1}
%+ \st e_{bcl} n_l C R_{kk}
%+ \st e_{bjl} n_l C R_{jc}
%- \st e_{bjl} n_l D R_{cj}''
%- \st e_{b3l} n_l R_{c3}''
%
%
%Rbk Xck - Rck Xbk = 
%+ \ct \delta_{b3} B \delta_{c1}
%- \ct C R_{bc}
%+ \ct D R_{cb}''
%+ \ct \delta_{b3} E R_{c3}''
%+ (1-\ct) n_b n_3 A \delta_{c3}
%+ (1-\ct) n_b n_3 B \delta_{c1}
%- (1-\ct) n_b n_j C R_{jc})
%+ (1-\ct) n_b n_j D R_{cj}''
%+ (1-\ct) n_b n_3 E R_{c3}''
%- \st e_{b3l} n_l A \delta_{c3}
%- \st e_{b3l} n_l B \delta_{c1}
%+ \st e_{bcl} n_l C R_{kk}
%+ \st e_{bjl} n_l C R_{jc}
%- \st e_{bjl} n_l D R_{cj}''
%- \st e_{b3l} n_l R_{c3}''
%

%%%%%%%%%%%%%%%%%%%%%%%%%%%%%%%%%%%%%%%%%%%%%%%%%%%%%%%%%%%%%%%%%%%%%%
\subsection*{Uniform equilibrium}

Let's remove gradient terms in the first equation
in~(\ref{eq:r_motion_wnt_eq}) and use expression~(\ref{eq:td_nt}) for
dipolar torque:
\begin{equation}\label{eq:eq_wn}
{\bf w}\times({\bf\hat z}\,(\omega_z-\omega) + {\bf\hat x}\,\omega_x)
= \frac{4}{15}\frac{\Omega_B^2}{\omega}\ \st(4\ct + 1)\ {\bf n},
\end{equation}
By multiplying it by $\bf w$ we have left-hand side equals zero and three cases:
${\bf n}\cdot{\bf w} = 0$, or $\ct = -1/4$, or $\st=0$.
The last one is unstable equilibrium, it corresponds to maximum of
dipolar energy. Two other cases are called HPD and NPD, now we can
find them.

\medskip
{\bf HPD (Homogeneously-precessing domain):}

We already showed in~(\ref{eq:w_n}) that
${\bf w} = R {\bf\hat z}$. Using this,~(\ref{eq:eq_wn}), and HPD condition ${\bf n}\cdot{\bf w} = 0$
we can find
\begin{equation}\label{eq:hpd_eq1}
{\bf n} = \pm{\bf\hat y},
\qquad
{\bf w} = \ct\ {\bf\hat z} +  n_y\st\ {\bf\hat x}.
\end{equation}
\begin{equation}\label{eq:hpd_eq2}
\ct =
- \frac14 + \frac{15}{16}\frac{\omega}{\Omega_B^2}
\ \left((\omega-\omega_z) + n_y \frac{\ct}{\st}\,\omega_x\right).
\end{equation}
We can use only positive sign of $\vartheta$ because matrix $R({\bf n},
\vartheta)$ is same as $R(-{\bf n}, -\vartheta)$. Later it will be shown
that state with $\vartheta>0$ and ${\bf n} = -{\bf\hat y}$ is unstable.
To check it we will keep parameter $n_y = \pm1$, which can be always set to 1
for HPD equilibrium state.

Usually RF-pumping $\omega_x$ is much smaller then frequency shift $\omega-\omega_z$,
$\ct$ is close to -1/4 and we can use $\ct/\st = -1/\sqrt{15}$ in the right part
of the last equation:
\begin{equation}\label{eq:hpd_eq2a}
\ct =
- \frac14 + \frac{15}{16}\frac{\omega}{\Omega_B^2}
\ \left(\omega-\omega_z - \frac{n_y}{\sqrt{15}}\,\omega_x\right).
\end{equation}
In arbitrary case to find $\vartheta$ we can do substitutions
\begin{equation}
t = \tan\vartheta/2,
\qquad
A = \frac{15}{16}\frac{\omega(\omega-\omega_z)}{\Omega_B^2},
\qquad
B = \frac{15}{16}\frac{\omega}{\Omega_B^2} n_y\,\omega_x,
\end{equation}
and obtaining 4-th order equation for $t$:
\begin{equation}
Bt^4 - \frac12(3 + 4A)t^3 + \frac12(5 - 4A)t - B = 0.
\end{equation}

\medskip
{\bf NPD (Non-precessing domain), Brinkman-Smith mode:}

Using the NPD condition $\ct = -1/4$ we have:
%\begin{equation}
%{\bf w}\times({\bf\hat z}\,(\omega_z-\omega) + {\bf\hat x}\,\omega_x) = 0,
%\end{equation}
\begin{equation}
w_1 = \frac{-\omega_x}{\sqrt{\omega_x^2 + (\omega_z-\omega)^2}},
\qquad
w_2 = 0,
\qquad
w_3 = \frac{-(\omega_z-\omega)}{\sqrt{\omega_x^2 + (\omega_z-\omega)^2}},
\end{equation}
\begin{equation}
n_1 =  \frac{w_1 n_3}{1+w_3},
\qquad
n_2 = \ \frac{w_1\sqrt{3/5}}{1+w_3},
\qquad
n_3 = \pm(4w_3+1)/5.
\end{equation}
(There is also solution with opposite signs of $w_1$ and $w_3$ which
corresponds to unstable equilibrium).

Here $\bf n$ is precessing around~$\bf\hat z$ axis. This is
case of a ``vertical'' texture. In real NMR experiments if RF pumping is
small enough then texture is determined by boundary conditions in the
experimental cell and possible topological defects such as $\bf
n$-solitons. It can have any equilibrium orientation of~$\bf n$ vector
and small oscillations around this equilibrium. These solutions can not
be found as an equilibrium in our rotating frame.

%TODO: Positive and negative shifts in both cases. More accurate study of
%stable and unstable equilibrium (this can be done in numerical simulation).

%%%%%%%%%%%%%%%%%%%%%%%%%%%%%%%%%%%%%%%%%%%%%%%%%%%%%%%%%%%%%%%%%%%%%%
\subsection*{Small oscillations of HPD in the rotating frame}

This section follows calculations in my PhD thesis (2012, available only
in Russian).

Let's again consider a uniform case with zero gradient terms. Introduce
dimensionless parameters:
\begin{equation}\label{eq:defs1}
d = \frac{\omega-\omega_z}{\omega},
\qquad
h = \frac{\omega_x}{\omega},
\qquad
b = \frac{15\gamma^2\Delta^2 g_D}{\chi_B\ \omega^2} = \frac{\Omega_B^2}{\omega^2}
\end{equation}

Then equations~(\ref{eq:r_motion_wnt_eq}) will be
\begin{eqnarray}\label{eq:r_motion_wnt}
\dot {\bf w}/\omega  &=&
 {\bf w} \times (h\,\hat {\bf x} - d\,{\bf\hat z})
  + \frac{4b}{15}\ \st(4\ct + 1)\ {\bf n}
\\\nonumber
{\bf\dot n}/\omega &=&
-\frac12\ {\bf n}\times ({\bf w} + {\bf\hat z})
-\frac12\ \frac{\st}{1-\ct}\ \Big[ {\bf n}\ ({\bf n}\cdot({\bf w} - {\bf\hat z}))
- ({\bf w} - {\bf\hat z})\Big],
\\\nonumber
\dot\vartheta/\omega &=& {\bf n} \cdot ({\bf w} - {\bf\hat z}),
\end{eqnarray}
(In my PhD thesis slightly different definition of dimensionless parameters were
used. Also there was an error in equilibrium magnetization values, but it
does not change the answer for small oscillation frequencies.)

The equilibrium which corresponds to HPD~(\ref{eq:hpd_eq1},\ref{eq:hpd_eq2a}) is
\begin{equation}
{\bf n}^0 = \pm{\bf\hat y},
\qquad
{\bf w}^0 = \ct\ {\bf\hat z} +  n_y\st\ {\bf\hat x}
\qquad
\ct^0 = -\frac{1}{4} - \frac{15}{16} \Big(d - \frac{h\,n_y}{\sqrt{15}}\Big)\frac{1}{b}
\end{equation}

Let's rewrite equation~(\ref{eq:r_motion_wnt}) in coordinates:
\begin{eqnarray}
\dot w_x/\omega &=&
F\,b\ n_x - w_y d
\\\nonumber
\dot w_y/\omega &=&
F\,b\ n_y + w_z h + w_x d
\\\nonumber
\dot w_z/\omega &=&
F\,b\ n_z - w_y h
\\\nonumber
2 \dot n_x/\omega &=&
- n_y (w_z + 1) + n_z w_y
-C\ n_x\ ({\bf n}\cdot({\bf w} - {\bf\hat z}))
+ C\ w_x
\\\nonumber
2 \dot n_y/\omega &=& - n_z w_x + n_x (w_z + \omega)
-C\ n_y\ ({\bf n}\cdot({\bf w} - {\bf\hat z}))
+ C\ w_y
\\\nonumber
2 \dot n_z/\omega &=& - n_x w_y + n_y w_x
- C\ n_z\ ({\bf n}\cdot({\bf w} - {\bf\hat z}))
+ C\ (w_z - 1)
\\\nonumber
\dot\vartheta/\omega &=& {\bf n} \cdot ({\bf w} - {\bf\hat z}).
\end{eqnarray}
where
\begin{equation}\label{eq:CF}
C=\frac{\st}{1-\ct},\qquad
F=\frac{4}{15}\ \st(4\ct + 1).
\end{equation}

Three equations for $\dot{\bf n}$, are dependent because $|{\bf n}|=1$. This
can  be checked by calculating~$n_x\dot n_x + n_y\dot n_y + n_z\dot n_z$.

Lets write the equations for small deviations $d{\bf n}$, $d{\bf w}$, $d\vartheta$
from the equilibrium. Here we keep only first-order terms with the deviations.
Also we put $dn_y = 0$ and omit equation for~$n_y$ because $\bf n$ is moving near $\hat{\bf y}$:
\begin{eqnarray}
d \dot w_x/\omega &=& F^0\,b\ dn_x - d\,dw_y,
\\\nonumber
d \dot w_y/\omega &=& dF\,b\ n^0_y + h\,dw_z + d\,dw_x,
\\\nonumber
d \dot w_z/\omega &=& F^0\,b\ dn_z - h\,dw_y,
\\\nonumber
2 d\dot n_x/\omega &=& - n^0_y\ dw_z + C^0\ dw_x + w^0_x\ dC,
\\\nonumber
2 d\dot n_z/\omega &=&   n^0_y\ dw_x + C^0\ dw_z + (w_z-1)\ dC,
\\\nonumber
d\dot\vartheta/\omega &=& w^0_x\ dn_x  + (w^0_z-1)\ dn_z + n^0_y\ dw_y.
\end{eqnarray}
For periodic motion with dimensionless frequency~$x = \Omega/\omega$ this
gives an equation
\begin{equation}
\left|
\begin{array}{rrrrrr}
-ix &  -d & 0   & b\,F^0 & 0 & 0\\
 d  & -ix & h   & 0 & 0 & n^0_y b\,dF/d\vartheta\\
0   & -h  & -ix & 0 & b\,F^0 & 0\\
%
C   &  0  & -n_y & -2ix & 0    & n^0_y \st\ dC/d\vartheta\\
n_y &  0  &    C & 0    & -2ix & (\ct^0 - 1)\ dC/d\vartheta\\
    0&  n^0_y&  0& w^0_x & (\ct^0 - 1) & -ix
\end{array}
\right| = 0
\end{equation}

If~$d,h \ll b \ll 1$, then the equation is
\begin{equation}
-x^6 + (1 + b)\,x^4 - \frac{1}{\sqrt{15}}
\left(\frac38 (\sqrt{15}d - n_y h) + b (\sqrt{15}d + 3 n_y h)\right)\,x^2
+ \frac{1}{10}(\sqrt{15}d -n_y h)\,b\,n_y h = 0;
\end{equation}
where~$n^0_y=\pm1$ and~$\vartheta^0$ is positive (changing sign of~$\vartheta$ is
equivalent to changing sign of~$\bf n$).

Solutions are
\begin{eqnarray}
x_1^2 &=& 1+b, \\\nonumber
x_2^2 &=& \frac{4}{\sqrt{15}}\ \frac{n^0_y h\,b}{1+8/3\,b}, \\\nonumber
x_3^2 &=& \frac{\sqrt{15}\,d - n^0_y h}{\sqrt{15}}
          \ \frac{3/8 + b}{1+b}.
\end{eqnarray}
One can see that $n^0_y = -1$ corresponds to unstable state ($x_2^2 < 0$) and we can use $n^0_y=+1$.

In original notations
\begin{eqnarray}
\Omega_1^2 &=& \omega^2 + \Omega_B^2, \\\nonumber
\Omega_2^2 &=& \frac{4}{\sqrt{15}}\ \frac{\omega_x\omega\,\Omega_B^2}{\omega^2+8/3\,\Omega_B^2}, \\\nonumber
\Omega_3^2 &=& \frac{\sqrt{15}\,(\omega-\omega_z)\omega - \omega_x\omega}{\sqrt{15}}
          \ \frac{3/8\,\omega^2 + \Omega_B^2}{\omega^2 + \Omega_B^2}.
\end{eqnarray}
The second mode is a useful tool for calibrating RF pumping~$\omega_x$ or
measuring Leggett frequency~$\Omega_B$. The third mode as fas as I know
have not been observed in experiments.

%%%%%%%%%%%%%%%%%%%%%%%%%%%%%%%%%%%%%%%%%%%%%%%%%%%%%%%%%%%%%%%%%%%%%%
\subsection*{Leggett-Takagi relaxation} %$

Leggett-Takagi equations~\cite{1977_leggett_takagi} can be made from Leggett equations~(\ref{eq:leggett_nt})
by adding a relaxation term with parameter~$\tau$ to the equation for~$\dot\vartheta$.
\begin{eqnarray}
\dot\vartheta &=&  {\bf n} \cdot \Big(\frac{\gamma^2}{\chi_B} {\bf S} - \gamma {\bf H} \Big)
+ \frac4{15}\ \st(4\ct + 1)\ \frac1{\tau}.
\end{eqnarray}

Switching to the rotating frame and using dimensionless parameters ${\bf
w}, d, h, b$~(\ref{eq:w}, \ref{eq:defs1}) we can write equations of
motion~(\ref{eq:r_motion_wnt}) with the relaxation term:
\begin{eqnarray}\label{eq:r_motion_lt}
\dot {\bf w}/\omega  &=&
 {\bf w} \times (-d\,\hat {\bf z} + h\,\hat {\bf x})
  + \frac{4}{15}b\ \st(4\ct + 1)\ {\bf n}
\\\nonumber
\dot {\bf n}/\omega &=&
-\frac12\ {\bf n}\times ({\bf w} + \hat {\bf z})
-\frac12\ \frac{\st}{1-\ct}\ \Big[ {\bf n}\ ({\bf n}\cdot({\bf w} - \hat {\bf z}))
- ({\bf w} - \hat {\bf z})\Big]
\\\nonumber
\dot\vartheta/\omega &=& {\bf n} \cdot ({\bf w} - \hat {\bf z})
+ \frac4{15}\ \st(4\ct + 1)\ \frac1{\omega\tau},
\end{eqnarray}
or in coordinates with $C$ and $F$ defined in~(\ref{eq:CF}):
\begin{eqnarray}\label{eq:r_motion_lt}
\dot w_x/\omega  &=& -d\ w_y + bF\ n_x \\\nonumber
\dot w_y/\omega  &=&  h\ w_z + d\ w_x + bF\ n_y \\\nonumber
\dot w_z/\omega  &=& -h\ w_y  + bF\ n_z\\\nonumber
%
2\dot n_x/\omega &=& -n_y (w_z + 1) + n_z w_y
-C\ n_x\ ({\bf n}\cdot({\bf w} - \hat {\bf z})) + C w_x
\\\nonumber
2\dot n_y/\omega &=& -n_z w_x + n_x (w_z + 1)
-C\ n_y\ ({\bf n}\cdot({\bf w} - \hat {\bf z})) + C w_y
\\\nonumber
2\dot n_z/\omega &=& -n_x w_y +n_y w_x
-C\ n_z\ ({\bf n}\cdot({\bf w} - \hat {\bf z})) + C (w_z - 1)
\\\nonumber
\dot\vartheta/\omega &=& {\bf n} \cdot ({\bf w} - \hat {\bf z})
+ \frac{F}{\omega\tau}.
\end{eqnarray}

For the equilibrium state we have:
\begin{equation}
{\bf n}^0\cdot{\bf w}^0 = 0,
\qquad
n^0_z = \frac{F^0}{\omega\tau},
\qquad
n^0_x = \frac{d}{h}\ \frac{F^0}{\omega\tau},
\qquad
w^0_y  = \frac{b}{h}\ \frac{(F^0)^2}{\omega\tau}.
\end{equation}
One can see that equilibrium position of vector $\bf n$ is rotated from
${\bf\hat y}$ direction by Leggett-Takagi relaxation, and vector~${\bf
w}^0$ stays perpendicular to ${\bf n}^0$. Note that even at low
relaxation~$n^0_x$ can be large because of~$d/h$ factor. This probably
determines maximum frequency shift where HPD can exist for a given RF
pumping and Leggett-Takagi relaxation. It is very interesting to measure
this effect in experiment.

%Typical experimental conditions are $h \ll d < b \ll 1$. At zero pressure
%and $T=0.5\,T_c$ Leggett frequency $\Omega_B = 111$~kHz, for NMR frequency 1~MHz we have
%$b\approx 10^{-2}$. Typical frequency shift is 1~kHz and radio-frequency pumping is 10~Hz which gives
%$d\approx 10^{-3}$ and $h\approx 10^{-5}$.

%% equilibrium for wx,wz, theta:
% \begin{eqnarray}\label{eq:r_motion_lt}
%0 &=& -(h\ w_z + d\ w_x)(w_z + 1)
%      + C \frac{b^2}{h} \frac{F^4}{(\omega\tau)^2}
%      + C \frac{db}{h}\frac{F^3}{(\omega\tau)^2} + C b F\ w_x
%\\\nonumber
%0 &=& - b \frac{F^2}{\omega\tau}\ w_x
%     + \frac{db}{h}\frac{F^2}{\omega\tau}\ (w_z + 1)
%     + C\ (h\ w_z + d\ w_x)\ \frac{F}{\omega\tau}
%     + C \frac{b^2}{h}\frac{F^3}{\omega\tau}
%\\\nonumber
%0 &=& -\frac{db^2}{h^2} \frac{F^4}{(\omega\tau)^2}
%      + (h\ w_z + d\ w_x) w_x
%      + Cb\ \frac{F^3}{(\omega\tau)^2}
%      + CbF\ (w_z - 1)
%\\\nonumber
%\end{eqnarray}
%
%
%Linearized equations of motion near the equilibrium:
%\begin{eqnarray}\label{eq:r_motion_lt}
%\dot dw_x/\omega  &=& -d\ dw_y + bF^0\ dn_x + b\ n^0_x\ dF \\\nonumber
%\dot dw_y/\omega  &=&  h\ dw_z + d\ dw_x + b\ n^0_y\ dF + b\ F^0\ dn_y \\\nonumber
%\dot dw_z/\omega  &=& -h\ dw_y  + bF^0\ dn_z + bn^0_z dF\\\nonumber
%%
%2\dot dn_x/\omega &=& -n_y dw_z + dn_z w_y
%+ dC w_x
%-C\ n_x\ d({\bf n}\cdot({\bf w} - \hat {\bf z})) + C dw_x
%-C\ dn_x\ \frac{F}{\omega\tau}
%\\\nonumber
%2\dot n_y/\omega &=& -dn_z w_x + dn_x (w_z + 1) + n_x dw_z
%-dC\ n_y\ \frac{F}{\omega\tau} + dC w_y
%-C\ n_y\ d({\bf n}\cdot({\bf w} - \hat {\bf z})) + C dw_y
%\\\nonumber
%2\dot n_z/\omega &=& -dn_x w_y -n_x dw_y + n_y dw_x
%-C\ dn_z\ \frac{F}{\omega\tau} + C dw_z
%\\\nonumber
%\dot\vartheta/\omega &=&
%dn_x w_x + dn_z w_z + n_x dw_x + n_y dw_y - dn_z
%+ \frac{dF}{\omega\tau}.
%\end{eqnarray}


%%%%%%%%%%%%%%%%%%%%%%%%%%%%%%%%%%%%%%%%%%%%%%%%%%%%%%%%%%%%%%%%%%%%%%
\subsection*{$\vartheta$-solitons} %$

Minimum of spin-orbit interaction corresponds to $\ct = -1/4$. It is
possible to have a so-called theta-soliton~\cite{1976_maki_solitons}
there $\vartheta$ changes between two minima, $\vartheta_L =
\cos^{-1}(-1/4)$ and $2\pi-\vartheta_L$, as a solution of equilibrium
Leggett equations.

Let's assume ${\bf n} = \hat{\bf z}$, ${\bf H} = H_z\hat{\bf z}$, and ${\bf
S} = \chi_B {\bf H}/\gamma$. Only $\vartheta$ is changing along $\hat{\bf
z}$ axis.

Using usual expression $R_{a j} = \ct\ \delta_{a j} + (1-\ct)\ n_a n_j - \st\ e_{ajk} n_k$,
and $n_j = \delta_{j3}$ one can find that only $J_{33}$ component of spin current is non-zero:
\begin{eqnarray}
J_{33} &=& - 2 \Delta^2 K_1 \vartheta'
\end{eqnarray}
Second and third of Leggett equations are satisfied by chosen
values of ${\bf n}$ and ${\bf S}$. The first equation gives
non-trivial expression for distribution of $\vartheta$:
\begin{equation}
T^D_3 = J_{33}'
\end{equation}
By substituting spin current and dipolar torque we obtain a second-order
differential equation for $\vartheta$:
\begin{equation}\label{eq:th_sol0}
\vartheta'' = -\frac{2 g_D}{K1}\ \st(4\ct + 1)
\end{equation}
or with usual definition of dipolar length~$\xi_D^2 = \frac{13}{24}\frac{K_1}{g_D}$.
\begin{equation}
\frac{12}{13}\xi_D^2\ \vartheta'' = -\st(4\ct + 1)
\end{equation}

To solve this equation we multiply both sides by $\vartheta'$ and integrate using
boundary conditions at $z=\pm\infty$: $\cos\vartheta = -1/4$, $\vartheta'=0$:
\begin{equation}
\sqrt{\frac{12}{13}} \xi_D\ \vartheta' = \pm 2 (\cos\vartheta + 1/4)
\end{equation}

Second integration will give us formula of the $\vartheta$-soliton.
Integration constant determines position of the soliton, we choose it to
localize the soliton at $z=0$. There are two solutions, one goes from
$-\theta_L$ to $+\theta_L$, another from $+\theta_L$ to $2\pi-\theta_L$.
To find the second solution one can make substitution $\bar\theta =
\theta - \pi$ and do similar integration. Result is:
\begin{equation}\label{eq:th_sol}
\vartheta = 2 \tan^{-1}\left[\sqrt{\frac{5}{3}} \tanh \left(\sqrt{\frac{65}{64}}\ \frac{\pm z}{\xi_D}\right)\right],
\qquad
\vartheta = \pi + 2 \tan^{-1}\left[\sqrt{\frac{3}{5}} \tanh \left(\sqrt{\frac{65}{64}}\ \frac{\pm z}{\xi_D}\right)\right].
\end{equation}
The first solution is not topologically protected because it is possible
to move from $-\theta_L$ to $+\theta_L$ by rotating $\bf n$ vector,
always staying in the minimum of spin-orbit interaction. If vector $\bf
n$ is oriented along $\bf\hat z$ for example by magneto-dipolar energy
(not discussed in this text yet) or by spin precession, it will form a
so-called $\bf n$-soliton which has much smaller energy and bigger size.
The second solution is the $\vartheta$-soliton.

$\vartheta$-solitons can exist in HPD state. In this case they have small
core where $\vartheta$ is changing and large tails where $\bf n$ vector
is rotated. This have been discussed in~\cite{1992_mis_hpd_topol} and
numerically calculated in~\cite{2022_zav_theta_hpd}.

%%%%%%%%%%%%%%%%%%%%%%%%%%%%%%%%%%%%%%%%%%%%%%%%%%%%%%%%%%%%%%%%%%%%%%

\subsection*{Motion of arbitrary soliton (in the rotating frame)} %$

Consider a soliton with some distribution $n(z)$ and $\vartheta(z)$ which
are equilibrium solution of Leggett equations.

Consider a soliton which has the same distribution of $n$ and~$\vartheta$ and moves with
velocity~$v$. For the moving soliton we can write~$n(z,dt) = n(z + v\,dt)$ and $ \dot n(z) =
n'\,v$, and same for theta. Now put this into~(\ref{eq:Sr_dndt}) and find
spin $\bf S$ distribution of the moving soliton:
\begin{equation}
\frac{\gamma^2}{\chi_B} {\bf S}(v)
=
\Big[ {\bf n}\ \vartheta' + (1-\ct)\ {\bf n}\times{\bf n'} + \st\ {\bf n'} \Big] v
+ \gamma {\bf H}
+ \omega\ (R{\bf\hat z}-{\bf\hat z})
\end{equation}
Calculate kinetic energy of the soliton using~(\ref{eq:en_m0}), as an
energy difference between moving and static soliton. Note that
expression for energy is same in the rotating frame:
\begin{equation}
E_K = \int_{-\infty}^{\infty} \left[ - ({\bf S}(v) \cdot \gamma {\bf H})
+ \frac{\gamma^2}{2\chi_B}{\bf S}(v)^2
\right] dz
- \int_{-\infty}^{\infty} \left[ - ({\bf S}(0) \cdot \gamma {\bf H})
+ \frac{\gamma^2}{2\chi_B}{\bf S}(0)^2
\right] dz
\end{equation}
$$
= - \frac{\chi_B}{\gamma^2}\ \omega v \int_{-\infty}^{\infty} 2(1-\ct)[{\bf n}\times{\bf n'}]_z\ dz
+ \frac{\chi_B}{\gamma^2}\ \frac{v^2}{2} \int_{-\infty}^{\infty}
\left[
{\bf n}\ \vartheta' + (1-\ct)\ {\bf n}\times{\bf n'} + \st\ {\bf n'}
\right]^2\ dz
$$
The first term is proportional to $\omega\,v$, which is interesting: it
says that if the soliton in the rotating frame has a helical structure
where $\bf n$ rotates around $\bf\hat z$ axis then energy minimum
corresponds to a non-zero velocity. The second term is proportional to
$v^2$, from this term we can get effective mass of the soliton:
\begin{equation}\label{eq:mass}
m = \frac{\chi_B}{\gamma^2}\ \int_z \left[
(\vartheta')^2 + 2(1-\ct)\ ({\bf n'})^2
\right]  dz
\end{equation}
This calculation is valid for small velocities. Arbitrary motion of
$\vartheta$-solitons have been calculated in \cite{1986_rozhkov_solitons}.

\subsection*{Motion of $\vartheta$-solitons} %$

Let's look at a 3D case: a soliton perpendicular to $\bf\hat z$ direction
in an experimental cell. It can be considered as a membrane attached to
walls because of some pinning effects. Let's calculate effective mass and
tension of this membrane using profile of the soliton $\vartheta(z)$
from~(\ref{eq:th_sol}).

Tension of the soliton membrane is its total potential energy (spin-orbit + gradient)
per unit area.  If $\bf n$ is uniform the gradient energy~(\ref{eq:en_g0}) is
\begin{equation}
E_{\nabla} = \frac12 \Delta^2 
  \big[(2K_1 + K_2 + K_3) (\nabla\vartheta)^2 - (K_2 + K_3) ({\bf n}\cdot \nabla\vartheta)^2\big].
\end{equation}
For the soliton~(\ref{eq:th_sol0}-\ref{eq:th_sol}) it is
\begin{equation}
F_{\nabla} = \Delta^2 K_1 (\vartheta')^2 = 8 \Delta^2 g_D (\ct + 1/4)^2
\end{equation}
Energy of spin-orbit interaction~(\ref{eq:en_d0}) is also
\begin{equation}
F_{SO} = 8 \Delta^2 g_D (\ct + 1/4)^2
\end{equation}
And tension is integral of total energy density over~$z$:
\begin{equation}
T = 16 \Delta^2 g_D\ \int_{-\infty}^{+\infty}(\ct(z) + 1/4)^2\, dz
\end{equation}
Mass of the soliton per unit area can be found by integrating~(\ref{eq:mass}):
\begin{equation}
M = 8 \frac{\chi_B}{\gamma^2}\frac{g_D}{K_1}
\ \int_{-\infty}^{+\infty} (\ct + 1/4)^2  dz
\end{equation}
Speed of sound is square root of tension divided by mass:
\begin{equation}
C^2 =
2\frac{\Delta^2\gamma^2}{\chi_B} K_1 = 2\cpe^2-\cpa^2
\end{equation}

The first oscillation mode of a circular membrane with radius $R$ is $2.405\,C/R$.


%%%%%%%%%%%%%%%%%%%%%%%%%%%%%%%%%%%%%%%%%%%%%%%%%%%%%%%%%%%%%%%%%%%%%%%
%\subsection*{Solving non-uniform texture} %$
%
%Consider a 1D problem, with no RF field $({\bf H}\parallel{\bf\hat z})$. In this system one can have
%HPD boundary, theta-soliton, or some other non-uniform texture described by equation ??.
%We have two non-uniform scalars ($\vartheta$ and~$n_z = ({\bf n}\times{\bf\hat z})$) and vector~${\bf n}$
%which can move with respect to ${\bf\hat z}$.
%
%Let's look for a solution in the form
%\begin{eqnarray}
%\vartheta' &=& G,\\\nonumber
%{\bf n}'   &=& F_1\ ({\bf n}\times{\bf\hat z}) + F_2\ {\bf n}\times({\bf n}\times{\bf\hat z}),
%\end{eqnarray}
%were $G, F_1, F_2$ are functions of $\vartheta$ and~$n_z$.
%Here direction of ${\bf n}'$ is perpendicular to {\bf n} and depends only on
%${\bf n}$ and ${\bf\hat z}$ mutual orientation.
%Note that in order to have uniform HPD and NPD solutions these derivatives
%should be zero at~$n_z = 0$ (HPD) and~$\ct = -1/4, {\bf n}\parallel{\bf\hat z}$ (NPD).
%
%By comparing result of numerical simulation of theta-soliton with this equation we
%can guess forms of $G$ and $F$ functions:
%\begin{eqnarray}
%G   &\propto& n_z (1 + c_0/(n_z + b_0))(\ct + 1/4) ,\\\nonumber
%F_1 &\propto& n_z (a_1\ct + b_1 + c_1/(n_z + d_1)),\\\nonumber
%F_2 &\propto& n_z (a_2\st(1-2\ct) + b_2)
%\end{eqnarray}
%a1*x*(-sin(y)*(1-2*cos(y)) - e1)
%
%
%
% Now differentiate this expressions. Here $F_i'$
%and $G_i'$ means derivatives with respect to $\vartheta$, and ${\bf n}'$
%and $\vartheta'$ are derivatives with respect to $z$, as before:
%\begin{eqnarray}
%\vartheta'' &=&  G' \vartheta'
%\\\nonumber
%{\bf n}'' &=&
%  [F_1'\ ({\bf n}\times{\bf H}) + F_2'\ {\bf n}\times({\bf n}\times{\bf H})]\ \vartheta'
%+ F_1\ ({\bf n}'\times{\bf H})
%+ F_2\ {\bf n}' ({\bf n}\cdot{\bf H})
%+ F_2\ {\bf n} ({\bf n}'\cdot{\bf H})
%\end{eqnarray}
%
%
%Now substitute~$\vartheta'$ and ${\bf n}'$:
%\begin{eqnarray}
%\vartheta'' &=& G' G
%\\\nonumber
%{\bf n}'' &=&
%  G [F_1'\ ({\bf n}\times{\bf H}) + F_2'\ {\bf n}\times({\bf n}\times{\bf H})]\\
%&& + 2F_1 F_2\ ({\bf n}\cdot {\bf H})\ ({\bf n}\times{\bf H}) \\\nonumber
%&& + 2F_2^2\ ({\bf n}\cdot {\bf H})^2\ {\bf n} \\\nonumber
%&& + (F_1^2 - F_2^2)\ ({\bf n}\cdot{\bf H})\ {\bf H} \\\nonumber
%&& - (F_1^2 + F_2^2)\ {\bf H}^2\ {\bf n} \\\nonumber
%\\\nonumber
%{\bf n}'' &=&
%+ F_1\ ({\bf n}'\times{\bf H})
%+ F_2\ {\bf n}' ({\bf n}\cdot{\bf H})
%+ F_2\ {\bf n} ({\bf n}'\cdot{\bf H})
%\end{eqnarray}
%
%We do not want to have second-order terms on ${\bf H}$ in the equation for texture.
%This means~$G_2=0$ or $G_2=\mbox{const}, F_2=0$.
%
%$$
%J^1_a =
%- 2(G_1 + G_2 n_{3}) n_{a}
%+ 2F_1 \big( (1-\ct)(\delta_{a3} - n_{3} n_{a}) - \st e_{a b 3} n_{b} \big)
%+ 2F_2 \big( \st(\delta_{a3} - n_{3} n_{a}) + (1-\ct) e_{a b 3} n_{b} \big)
%$$
%
%$$
%J^1_a =
%- 2(G_1 + G_2 n_{3} + F_1 (1-\ct) n_{3} + F_2 \st n_{3}) n_{a}
%+ 2(F_1 (1-\ct) + F_2 \st)\delta_{a3}
%- 2(F_1 \st - F_2 (1-\ct))e_{a b 3} n_{b}
%$$

