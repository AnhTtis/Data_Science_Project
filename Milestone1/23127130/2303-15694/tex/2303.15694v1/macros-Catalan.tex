%% not putting in most usepackage commands since FPSAC is picky
%% are in separate packages file
%%% is Fpsac picky about theorem styles?

%%%%%%%%%%%%%%%%% DIAGRAM PACKAGES %%%%%%%%%%%%%%%%%%%%%%%%%%%%%%%%%%%%%%%%%%%%%%

%%%%%%%%%%%%%%%%%%%%%%
% xy pic hackcenter and tikz
%%%%%%%%%%%%%%%%%%%%%%

% xypic
\input xy
\usepackage[all]{xy}
\xyoption{line}
\xyoption{arrow}
\xyoption{color}
\SelectTips{cm}{}
%%

%%-----------------------------------------------------
%%% tikz
\usepackage{tikz}
\usetikzlibrary{decorations.markings}
\usetikzlibrary{decorations.pathreplacing}
\newcommand{\hackcenter}[1]{
 \xy (0,0)*{#1}; \endxy}

\tikzstyle directed=[postaction={decorate,decoration={markings,
    mark=at position #1 with {\arrow{>}}}}]
\tikzstyle rdirected=[postaction={decorate,decoration={markings,
    mark=at position #1 with {\arrow{<}}}}]

\usetikzlibrary{calc}
\usepackage{relsize}

\tikzset{fontscale/.style = {font=\relsize{#1}}
    }
\newcommand{\bbullet}{
\begin{tikzpicture}
  \draw[fill=black] circle (0.55ex);
\end{tikzpicture}
}
\usetikzlibrary{decorations.pathmorphing}
\usetikzlibrary{decorations.text}

\tikzset{snake it/.style={decorate, decoration=snake}}
%%-------------------------------------


% including eps files
\usepackage{graphicx}
\usepackage{color}
%%%%%%%%%%%%%%%%%%%%%%%%%%%%%%%%%


\usepackage{bbm}
\def\C{{\mathbb{C}}}
%%\def\N{{\mathds N}}
\def\R{{\mathds R}}
%%\def\Z{{\mathds Z}}
\def\Q{{\mathsf Q}}
\def\P{{\mathsf P}}
%% \def\B{{\mathsf B}} %% not used in either doc 
\def\H{{\mathcal{H}}}
\def\k{{\mathds{k}}}
\def\He{{\mathbb{H}}}
\def\K{{\mathcal{K}}}
\def\cA{\mathcal{A}}
\def\h{{\mathfrak{h}}}
\def\d{{\mathsf{d}}}
\def\1{{\mathbbm{1}}}
\def\D{\mathsf{D}}

\newcommand{\B}{\mathcal{B}}
\newcommand{\I}{\mathcal{I}}
\newcommand{\W}{\mathcal{W}}
%%\newcommand{\D}{\mathfrak{D}}
\newcommand{\Z}{\mathbb{Z}}
\newcommand{\N}{\mathbb{N}}
\newcommand{\Hik}{\mathcal{H}} %hikita
\newcommand{\E}{\mathcal{E}}
\newcommand{\F}{\mathcal{F}}

\DeclareMathAlphabet{\mathpzc}{OT1}{pzc}{m}{it}
\newcommand{\Amap}{\mathpzc{A}}    %%% changed \A to \Amap as there were two \A's
\newcommand{\s}{\mathcal{S}}


\newcommand{\action}{\mathbf{a}}



%%%%%%%%%%%%%%%%%%%%%%
% fancy comments
%%%%%%%%%%%%%%%%%%%%%%
\usepackage[normalem]{ulem}
\usepackage[colorinlistoftodos]{todonotes}
% 
\newcommand{\jose}[1]{\todo[size=\small,inline,color=violet!30]{#1 \\ \hfill --- Jose}}
\newcommand{\monica}[1]{\todo[size=\small,inline,color=green!20]{#1 \\ \hfill --- Monica}}
\newcommand{\nicolle}[1]{\todo[size=\small,color=orange!30]{#1 \\  \hfill--- Nicolle}}
% 
\newcommand{\Jose}[1]{\todo[size=\tiny,inline,color=violet!30]{#1 \\ \hfill --- Jose}}
\newcommand{\Monica}[1]{\todo[size=\tiny,inline,color=blue!20]{#1 \\ \hfill --- Monica}}
\newcommand{\Nicolle}[1]{\todo[size=\tiny,inline,color=orange!30]{#1 \\ \hfill --- Nicolle}}
\newcommand{\TODO}[1]{\todo[size=\small,inline,color=red!60]{#1 \\ \hfill --- anyone}}
%
\newcommand{\JS}[1]{{\color{violet}{\bf Jos\'e:} #1}}
\newcommand{\MV}[1]{{\color{green}{\bf MV:} #1}}
\newcommand{\NGG}[1]{{\color{orange}{\bf NG:} #1}}
\newcommand{\ToDo}[1]{{\color{red}{\bf: TO DO:} #1}}
%
%%%%%%%%%


\DeclareMathOperator{\sgn}{sgn}
\DeclareMathOperator{\Inv}{Inv}
\DeclareMathOperator{\std}{\mathsf{std}}
%% \newcommand{\std}{\operatorname{\mathsf{std}}}
\newcommand{\sstd}[1]{\operatorname{sstd}^{#1}}


\newcommand{\SSYT}{\ensuremath\mathrm{SSYT}}
\newcommand{\SSKD}{\ensuremath\mathrm{SSKD}}
\newcommand{\SSKT}{\ensuremath\mathrm{SSKT}}
\newcommand{\wt}{\ensuremath\mathrm{wt}}

\newcommand{\Exp}{\ensuremath\mathrm{exp}}
\newcommand{\Span}{\ensuremath\mathrm{Span}}

\newcommand{\SBim}{\mathbb{S}\operatorname{Bim}}
\newcommand{\ep}{\epsilon}
\newcommand{\CO}{\mathcal{O}}
\newcommand{\ba}{\mathbf{a}}
\newcommand{\Fl}{\mathcal{F}\ell}

\DeclareMathOperator{\GL}{GL}
\DeclareMathOperator{\SL}{SL}
\DeclareMathOperator{\spann}{span}
\DeclareMathOperator{\gr}{gr}

\newcommand{\GLr}{\GL_r}
\newcommand{\glr}{{\mathfrak{gl}_r}}
\newcommand{\bw}{\mathbf{w}}
\newcommand\End{\operatorname{End}}
%\newcommand\ker{\operatorname{ker}}
\newcommand\Hom{\operatorname{Hom}}
\newcommand{\dyckP}[1]{\operatorname{DP}(#1)}
\newcommand{\dyck}[1]{\operatorname{D}(#1)}
\newcommand{\area}{\operatorname{\mathtt{area}}}
\newcommand{\coarea}{\operatorname{\mathtt{co-area}}}
\newcommand{\dinv}{\operatorname{\mathtt{dinv}}}
\newcommand{\codinv}{\operatorname{\mathtt{co-dinv}}}
\newcommand{\qbinom}[2]{\left[\begin{matrix} #1 \\ #2 \end{matrix}\right]_{q}}
\newcommand{\ver}{\mathcal{V}}
\newcommand{\park}[1]{\operatorname{PF}(#1)}
\newcommand{\sspf}[2]{\operatorname{SSPF}_{#1}(#2)}
\newcommand{\we}{\operatorname{\mathtt{wt}}}
\newcommand{\A}{\mathcal{A}}
\newcommand{\affsym}[2]{\widetilde{S}^{#2}_{#1}}

\newcommand{\des}{\operatorname{Des}} %Des for set, des for cardinality?
\newcommand{\hik}{\mathcal{H}}
\newcommand{\affc}[1]{\operatorname{AC}_{#1}}
\newcommand{\affcz}[1]{\overline{\operatorname{AC}}_{#1}}
\newcommand{\strip}{\operatorname{St}}
\newcommand{\minlength}[3]{S_{#1}\backslash\widetilde{S}^{#3}_{#2}}
\newcommand{\EHA}{\mathcal{E}^{++}}
\newcommand{\DAHA}[1]{\mathbb{SH}(#1)^{++}}
\newcommand{\georep}{V^{\mathrm{geom}}}
\newcommand{\algrep}{V^{\mathrm{alg}}}
\newcommand{\pol}[1]{\mathrm{Pol}_{#1}}

\newcommand{\Sym}{\Lambda}
\newcommand{\Symr}{\Lambda_r}
\newcommand{\bx}{\mathbf{x}}

\DeclareMathOperator{\inv}{inv}
\DeclareMathOperator{\INV}{Inv}
\DeclareMathOperator{\Hilb}{Hilb}
\newcommand{\AS}{\widetilde{S}}

\newcommand{\g}{\mathfrak{g}}

\newcommand{\gam}{\delta}

\DeclareMathOperator{\Pol}{Pol}
\DeclareMathOperator{\Geom}{Geo}

\newcommand{\newword}[1]{\emph{\textbf{#1}}}
%\newcommand{\bf}[1]{\textbf{#1}}

\newcommand{\KD}{\mathrm{KD}}
%%\newcommand\Hom{\mathrm{Hom}} %%Hom is best done via DeclareMathOperator
\newcommand\Des{\mathrm{Des}}

\newcommand{\dyckmn}{\dyck{(m,n)}} %% might be better done as \dyck{m,n}


%\newcommand{\PF}{\operatorname{\mathsf{PF}}}
%\newcommand{\SSPF}{\operatorname{\mathsf{SSPF}}}
%\newcommand{\AC}{\operatorname{\mathcal{AC}}}

%%%% colors for labels 
\newcommand{\redone}{{\color{red} 1}}
\newcommand{\redzero}{{\color{red} 0}}
\newcommand{\redtwo}{{\color{red} 2}}
\newcommand{\bluetwo}{{\color{blue} 2}}
\newcommand{\blueone}{{\color{blue} 1}}
\newcommand{\bluezero}{{\color{blue} 0}}

