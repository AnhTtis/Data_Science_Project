







Our work described thus far aims to provide productivity enhancements for people who can already read scientific papers.
Unfortunately, blind and low-vision individuals and other people with disabilities who rely on assistive reading tools like screen readers face more basic challenges accessing paper contents, such as reading the text in the proper order.
Assistive reading tools do not function properly on most scientific papers due to the widespread usage of the PDF document format, which requires manual postprocessing to ensure accessibility~\cite{bigham-uninteresting-tour}.
However, this work is costly (requiring manual annotation using proprietary software) and rarely performed, leaving most scientific documents inaccessible~\cite{wang-2021-accessibility}.
This lack of access is a major problem that makes participation in STEM careers difficult for people with disabilities, creating equity issues and slowing the pace of scientific progress.

Semantic Scholar's document processing pipeline, which automatically extracts the contents of scientific PDFs along with semantic annotations in order to power tools like the Semantic Reader, is a promising avenue to addressing these access barriers.
In addition to supporting automated analysis of documents, these extractions can enable rerendering of PDFs in alternative formats like HTML that offer better accessibility for people with disabilities.
To test this idea, we built SciA11y~\cite{wang-assets21}, a prototype reading tool that rerendered 1.5 million open-access PDFs as more accessible HTML using this approach.
While these extractions are still imperfect, the potential of this approach was recognized by a Best Artifact Award at ASSETS, the top accessibility conference, and user study participants such as a blind scientist, who said, ``for unaccessible PDFs, this is life-changing."
This prototype has since been released for broader use and study as PaperToHTML,\footnote{\url{https://papertohtml.org/}} a website that allows users to upload their own PDFs on-demand.

Several design considerations arise when performing automatic document rerendering~\cite{wang-2021-accessibility}.
First, rendering errors can be harmful, so the system should handle missing data and potential errors carefully. To make users aware of missing figures, SciA11y still renders figures that have not been extracted properly, with placeholder images and text indicating the potential failure. SciA11y also does not seek to improve accessibility where automated approaches are likely to be poor quality, such as alternative text generation for figure descriptions.
Second, the system must make new decisions about where to place content elements, and should attempt to place elements where they are expected and most useful. For example, when converting a two-column layout with floating figures into a one-column HTML layout, the system must decide where to insert figures in the text. SciA11y places such floating elements following the paragraph that first references the element.
Third, the system must take care in how it renders elements such as in-line citations, to ensure that they are not overly verbose when read by a screen reader.

We plan to incorporate this functionality in the Semantic Reader product, to enable an HTML view of any scientific paper.
Such a view has many potential benefits for a wide range of readers that is not limited to people with disabilities.
As stated earlier, HTML can allow more flexibility for adding additional document affordances without creating clutter or occluding the original paper content.
Moreover, people without disabilities have situational impairments that can benefit from HTML renders.
For example, people using devices with small form factors, like smartphones, often find it challenging to read PDFs and may prefer a rerendered view as single-column HTML with larger font size.
As another example, people performing activities that require their visual attention, like driving, may still wish to consume papers via assistive reading tools like text-to-speech (TTS).
Like other changes to the original PDF formatted by the author, care must be taken to do so in an ethical manner (see Section~\ref{sec:ethical}).

Beyond rerendering documents in a more accessible HTML format, we are designing augmented reading functionality that addresses the needs of blind and low-vision readers, who experience increased costs navigating within a paper~\cite{maschulla-bvi-skimming}.
While the augmented reading functionality described thus far can be applied to HTML renders, it must be designed for and evaluated with blind and low-vision readers.
For example, blind SciA11y users appreciated the system's bibliographic references augmented with hyperlinks that point back to the referencing text; it is unclear that more ``in situ'' content akin to citation cards would be preferred by these users.
Low-vision readers have their own distinct navigation needs and preferences, which we explored through formative studies with four navigation interfaces for consuming hyperlinked content~\cite{park-cscw22}.
Finally, blind and low-vision readers may benefit from different types or amounts of hyperlinked content, which is difficult to provide automatically in all cases.
In response, we designed an interface that helps any reader create additional hyperlinks, which low-vision readers found helpful in a field deployment study with mixed-ability teams of low-vision and sighted readers~\cite{park-cscw22}.
Much more remains to be done, but we are excited about the potential of Semantic Reader not only to make papers newly accessible, but also to provide productivity-enhancing reading affordances that transform the paper reading experience for blind and low-vision readers.










\kyle{TBD -- rewriting this stuff about AI opportunities in a way that's more cognizant of GPT4 n recent stuff}











