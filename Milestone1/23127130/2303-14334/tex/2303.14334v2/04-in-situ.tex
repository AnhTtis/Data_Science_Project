\section{In-Situ Explanations for Better Comprehension}
\label{sec:in-situ-explanations}


Could an augmented reading application help readers understand a paper by reducing the  cognitive load associated with reading a paper? 
In this section, we discuss several ways in which interactive reading aids can help a reader understand a paper with less work through three systems: ScholarPhi~\cite{Head2021AugmentingSP}, PaperPlain~\cite{August2022PaperPM} and Papeo~\cite{papeo}. 
In particular, papers can be augmented with  definitions of terms and symbols,  provide plain-language summaries of paper passages, and connect readers with alternative forms of expression (for instance, video clips of research talks) that offer more approachable explanations of the paper's content.
\subsection{Defining Terms and Symbols with ScholarPhi}
\label{s:individual-terms}

\begin{figure}[b] 
     \centering
     \includegraphics[width=1\columnwidth]{figs/CACM-paper-figures-01-31-23-01-crop1.png}        \caption{ScholarPhi~\cite{Head2021AugmentingSP} shows definitions of terms and symbols in pop-up tooltips. When a reader selects a formula, all known definitions of symbols are shown simultaneously.
     To let readers select nested symbols (e.g., ``$h$'' in ``$V^{(j)}_{h}$''), ScholarPhi supports ``drill-down'' subsymbol selection.
     }
     \label{fig:scholarphi}
\end{figure}

Understanding a paper requires understanding its vocabulary. However, this is by no means an easy task---a typical paper may contain dozens of acronyms, symbols, and invented terms. And often, these terms appear without accompanying definitions~\cite{murthy2022accord}. 
How can we design interactive aids that present definitions of terms when and where readers most need them? ScholarPhi~\cite{Head2021AugmentingSP} takes as its basis the term gloss---an extension to a reading interface that shows a reader an explanation of a phrase when they click it. Glosses appeared in early research interfaces for reading hypertext~\cite{ref:zellweger1998fluid} and have since become part of widely-used reading interfaces including Wikipedia and Kindle.

That said, familiar gloss designs do not work well for scientific papers, where glosses run the risk of distracting readers, terms have multiple meanings, and phrases (specifically math symbols) are difficult to unambiguously select. 
The ScholarPhi design addresses these challenges. First, it aims to reduce distraction by showing definitions with high economy: glosses show multiple definitions and
 and in-context usages within a compact tooltip.
 Second, it provides position-sensitive definitions, revealing definitions that appears most recently prior to the selected usages of terms. 
 Terms and definitions are automatically identified using a pretrained language model~\cite{kang-etal-2020-document}.
 Finally, it provides easier access to definitions of mathematical symbols. Readers can access definitions of both a symbol and the subsymbols it is made of through a multi-click, ``drill-down'' selection mechanism. Furthermore, when a reader selects a formula, they can see definitions for all symbols at once, automatically placed adjacent to the symbols in the formula's margins (see Figure~\ref{fig:scholarphi}).
 
 In a usability study, the above interactions reduced the time it took readers to find answers to questions involving the understanding of terminology. All readers reported they would use the definition tooltips and formula diagrams often or always if available in their PDF reader tools. 





\subsection{Simplifying Complex Passages with Paper Plain}
\label{sec:paper-plain}





\begin{figure}[t]
     \centering
     \includegraphics[width=1\columnwidth]{figs/CACM-paper-figures-01-31-23-02.png}
     \caption{
     Paper Plain~\cite{August2022PaperPM} 
     provides in-situ plain language summaries of passages called ``gists'' to help readers who are overwhelmed by complex textual passages. Readers access gists by clicking a flag next to a section header. These gists are generated by large language models.  
     }
     \label{fig:paper-plain}
\end{figure}

Helping a reader understand individual terms and phrases only addresses part of the problem. Papers often contain passages so dense and complex that individual definitions are not enough to help someone read the passages, especially if they are a novice or non-expert in a field \cite{Britt2014ScientificLT}. 
Can we make complex texts more approachable by incorporating plain language summaries in the margins of the text?
With Paper Plain~\cite{August2022PaperPM}, when a reader encounters a section they find difficult to read, they can access a plain language summary of that section by clicking a button adjacent to the section header (see Figure~\ref{fig:paper-plain}). 
These summaries are generated by prompting a large language model with section text~\cite{gpt3-brown-2020}. 

Furthermore, Paper Plain helps guide readers using these summaries as an ``index'' into the text. A sidebar containing questions a reader may have about the text (e.g., \emph{What did the paper find?} or \emph{What were the limitations?}) provides links into answering passages identified using a question-answering system~\cite{Yoon2019PretrainedLM} alongside their associated plain language summaries.
These features were designed to help readers understand the ``gist'' of passages that contain unfamiliar vocabulary, providing support beyond that of individual term definitions.
Drawing inspiration from prior interactive reading affordances for term definitions~\cite{Jain2018ContentDE}, in-situ question answering~\cite{Zhao2020TalkTP, Chaudhri2013InquireBA}, and guiding reading~\cite{Dzara2019MedicalEJ}, Paper Plain seeks to bring these features together into a holistic system capable of supporting reading of a paper by a non-expert readership. In a usability study, readers made more frequent use of passage summaries than definition tooltips when both were available, suggesting the potential value of plain language summaries as allowing readers to bypass definitions of individual terms when acquiring a broad understanding of a paper.





\vspace{-2mm}


\subsection{Fusing Papers and Videos with Papeo}
\label{sec:papeo}


Sometimes, the best explanation of an idea is non-textual.
Videos can enhance understanding~\cite{Mayer1998ACT} while also requiring less mental load~\cite{Mayer1998ASE}, and various tools have been designed to facilitate searching and browsing for explanations in informational videos such as lectures~\cite{Kim2014DatadrivenIT, Liu2018ConceptScapeCC, Pavel2014VideoDA, Krosnick2015VideoDocC} and tutorials~\cite{Kim2014CrowdsourcingSI, Truong2021AutomaticGO, Khandwala2018CodemotionET}.
Similarly, for research papers, an algorithm might be better explained through an animation, a user interface might be better showcased through an screen recording, compared to the proses of a paper \cite{Hffler2007InstructionalAV}.
Instead of consuming the two formats independently, could interactive reading interfaces offer readers access to these alternative, more powerful descriptive forms as they read? 
For this, Papeo~\cite{papeo} was developed as a tool that supplements papers with more engaging, concise, dynamic presentations of information by linking excerpts of talk videos to corresponding paper passages.
 To grant authors more control over how their work is presented, we developed an AI-supported authoring interface for linking paper passages and videos efficiently: candidate passages are linked to excerpts of videos as suggestions using a pretrained language model~\cite{wang-2020-minilm-v1}, and an author interactively confirms or refines them. 
 
 Unlike text-skimming with Scim (\S\ref{sec:scim}) and Paper Plain (\S\ref{sec:paper-plain}), video-skimming in Papeo combines multiple modalities to explain complex information. For example, instead of reading a long text description of an interactive system, readers could see the system's behavior in a screen recording video with the author's commentary, and switch to corresponding passages to see implementation details or design motivations if desired.
Our early-stage evaluations of Papeo suggest that readers can use these interactions to fluidly transition between watching video and reading text, using video to quickly understand, and then selectively descending into the text when they desire a detailed understanding of the paper. 



















































