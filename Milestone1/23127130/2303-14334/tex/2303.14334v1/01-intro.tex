
\section{Introduction}




The exponential growth of scientific publication~\cite{Bornmann2020GrowthRO,brainard2020scientists} and increasing interdisciplinary nature of scientific progress \cite{van2015interdisciplinary, okamura2019interdisciplinarity} makes it increasingly hard for scholars to keep up with the latest developments. 
Academic search engines, such as Google Scholar and Semantic Scholar
help scholars discover research papers. 
Automated summarization for research papers~\cite{Cachola2020TLDRES} helps scholars triage between {\papers}.  
But when it comes to actually {\em reading} {\papers}, the process, based on a static PDF format, has remained largely unchanged for many decades.
This is a problem because digesting technical research papers is difficult~\cite{Bazerman1985-la,bem1995writing}.%

In contrast, interactive and personalized documents have seen significant adoption in domains outside of academic research. For example, news websites such as the \textit{New York Times} often present interactive articles with explorable visualizations that allow readers to understand complex data in a personalized way. E-readers, such as the Kindle, provide in-situ context to help readers better comprehend complex documents, showing inline term definitions and tracking occurrence of characters in a long novel. 
While prior work has focused on authoring support tools~\cite{conlen2021idyll,Latif2021KoriIS,Conlen2022FidyllAC} that can reduce effort in creating interactive scientific documents~\cite{Hohman2020CommunicatingWI,head2022math}, they have not seen widespread adoption due to a lack of incentive structure \cite{Distill_Editorial_Team2021-ix}. 
Furthermore, millions of {\papers} are locked in the rigid and static PDF format, whose low-level syntax makes it extremely difficult for systems to access semantic content, augment interactivity, or even provide basic reading functionality for assistive tools like screen readers~\cite{bigham-uninteresting-tour}.

\begin{figure*}[ht!]
    \centering
    \includegraphics[width=0.8\textwidth]{figs/product-overview-03-22-23.png}
    \caption{The Semantic Reader Project consists of research, product, and open science resources. The Semantic Reader product\footref{product} is a free interactive interface for research papers. It supports standard reading features (e.g., (A) table of contents), integration with Semantic Scholar (e.g., (B) save to library), useful augmentations atop the existing PDF (e.g., (C) in-situ Paper Cards when clicking inline citations), and integration with third-party features (e.g. (D) Hypothes.is\footref{hypothesis} for user highlights).
    We continues to integrate research features into this product as they mature (e.g., (E) Scim automated highlights \S\ref{sec:scim}).
    } 

    \label{fig:product}
\end{figure*}

Fortunately, recent work on layout-aware document parsing~\cite{Xu2019LayoutLMPO,Huang2022LayoutLMv3PF,shen-etal-2022-vila}
and large language models~\cite{Beltagy2019SciBERTAP,Raffel2019ExploringTL,gpt3-brown-2020}
show promise for accessing the content of PDF documents, and building systems that can better understand their semantics. 
This raises an exciting challenge: \emph{Can we create intelligent, interactive, and accessible reading interfaces for research papers, even atop existing PDFs?}

To explore this question, we present the \textbf{Semantic Reader Project}, a broad collaborative effort across multiple non-profit, industry, and academic institutions to create interactive, intelligent reading interfaces for research papers. 
This project consists of three pillars: research, product, and open science resources.
On the research front, the Semantic Reader Project combines AI and HCI research to design novel, AI-powered interactive reading interfaces that address a variety of user challenges faced by today's scholars. We developed research prototypes and conducted usability studies that clarify their benefits.
On the product front, we are developing the Semantic Reader~(Figure~\ref{fig:product}),\footnote{\label{product}Semantic Reader: \url{ https://www.semanticscholar.org/product/semantic-reader}} 
a freely available reading interface that integrates features from research prototypes as they mature.\footnote{Available for over 369K papers as of February 2023.}
Finally, we are developing and releasing open science resources that drive both the research and the product. These resources together open-source software,\footnote{\label{library}For UI development: \url{https://github.com/allenai/pdf-component-library}}\footnote{\label{papermage}For processing PDFs: \url{https://github.com/allenai/papermage}} AI models \cite{cohan2020specter,Cachola2020TLDRES,shen-etal-2022-vila,kang-etal-2020-document}, and open datasets \cite{kinney2023semantic,lo-etal-2020-s2orc} to support continued work in this area.%

In this paper, we focus on summarizing our efforts under the \emph{research pillar} of the Semantic Reader Project. We structure our discussion around five broad challenges faced by readers of research papers:
\addtocounter{footnote}{1} %
\footnotetext{\label{hypothesis}Hypothes.is: https://web.hypothes.is\url{https://web.hypothes.is}}

\begin{itemize}
  \setlength\itemsep{0.25em}

    \item \textbf{Discovery:} Following paper citations is one of the main strategies that scholars employ to discover additional relevant papers, but keeping track of the large numbers of citations can be overwhelming. In \S\ref{sec:citation-discovery}, we explore ways to visually augment research papers to help readers prioritize their paper exploration during literature reviews.
    \item \textbf{Efficiency:} The exponential growth of publication makes it difficult for scholars to keep up-to-date with the literature---scholars need to skim and read many papers while making sure they capture enough details in each. In \S\ref{sec:guiding}, we explore how support for non-linear reading can help readers consume research papers more efficiently.
    \item \textbf{Comprehension:} Research papers can be dense and contain terms that are unfamiliar either because the author newly introduces them or assumes readers have prerequisite domain knowledge. In \S\ref{sec:in-situ-explanations}, we explore how providing in-situ definitions and summaries can benefit readers especially when reading outside of their domains.
    \item \textbf{Synthesis:} The sensemaking \cite{Russell_Sensemaking_1993} process of synthesizing knowledge scattered across multiple papers
    is effortful but important. It allows scholars to make connections between prior work and identify opportunities for future research. In \S\ref{sec:bootstrap}, we explore how to help readers collect information from and make sense of many papers to gain better understanding of broad research topics.%
    \item \textbf{Accessibility:} Static PDFs are an ill-suited format for many reading interfaces. For example, PDFs are notoriously incompatible with screen readers, and represent a significant barrier for blind and low vision readers~\cite{bigham-uninteresting-tour}. Furthermore, an increasing number of scholars access content on mobile devices, on which PDFs of papers are difficult to read. In \S\ref{sec:accessibility}, we explore methods for converting legacy papers to more accessible representations.

\end{itemize}
\noindent Specifically, we present ten research prototypes developed in the Semantic Reader Project---CiteSee~\cite{Chang2022CiteSee}, CiteRead~\cite{Rachatasumrit2022CiteReadIL}, 
Scim~\cite{Fok2023Scim}, 
Ocean~\cite{park-cscw22}, ScholarPhi~\cite{Head2021AugmentingSP}, Paper Plain~\cite{August2022PaperPM}, Papeo~\cite{papeo}, Threddy~\cite{Kang2022Threddy}, Relatedly~\cite{relatedly}, and SciA11y~\cite{wang-2021-scia11y,paper2html}---and explain how they address these reading challenges. 
We conclude by discussing ongoing research opportunities in both AI and HCI for developing the future of scholarly reading interfaces. We provide pointers to our production reading interface and associated open resources to invite the broader research community to join our effort.






