
\section{Navigation and Efficient Reading}
\label{sec:guiding}



An important part of reading a paper is knowing what and where to read. Scholars often read papers non-linearly; they might return to a previously-read passage to recall some information, or jump forward to a different section of the paper (or to another paper) to satisfy an information need before jumping back. While jumping can help scholars orient their reading to sections of interest, it can also be a distraction by causing readers to constantly switch contexts. Non-linear navigation can be especially burdensome when the reader is interested in a particular \emph{type} of information (e.g., skimming a paper for the main results), but doesn't know precisely where to find it within the paper.
In this section we discuss two systems, Scim~\cite{Fok2023Scim} and Ocean~\cite{park-cscw22}, which demonstrate different approaches to helping readers navigate efficiently through a paper toward high-value, relevant information. 






\subsection{Guided Reading with Scim}
\label{sec:scim}



\begin{figure}[h]
    \centering
    \includegraphics[width=1\columnwidth]{figs/CACM-paper-figures-01-31-23-03-crop2.png}
    \caption{The Scim~\cite{Fok2023Scim} interface guides reader attention using color highlights corresponding to discourse facets. A sidebar allows users to toggle facets on/off. Clicking a color-coded snippet scrolls the reader to the relevant passage.
    }
    \label{fig:guided-reading}
\end{figure}






Scholarly reading can be considered a type of sensemaking represented as a continuous interplay between two processes: \textit{information foraging} in which readers identify relevant paper content, and \textit{comprehension} in which readers attempt to integrate the new information into their working model of the paper and with relevant prior knowledge~\cite{Pirolli_InformationForaging_1999, Russell_Sensemaking_1993}. Distinguishing between relevant and irrelevant content could help facilitate efficient reading. Paper abstracts offer one such separation, in essence an author-crafted determination of relevant content. However, static paper abstracts can leave readers to desire additional details that then require them to skim the paper itself. 






Scim~\cite{Fok2023Scim} addresses this problem via an augmented reading interface designed to guide readers' attention using automatically-created in-situ faceted highlights (Figure \ref{fig:guided-reading}). Though prior work has explored highlighting as a visual cue for guiding reader attention~\cite{wecker_semantize_2014, chi_scenthighlights_2005, yang_hitext_2017}, the efficacy for reading of scholarly text is less well-understood. Scim investigated the following design goals for intelligent highlights in scholarly reading:  highlights should be (1) evenly-distributed throughout a paper, (2) have just the right density (too few highlights will present the guise of an inept tool, and too many will slow a reader down), and (3) highlight several key categories of information in the paper. Because readers often skim for common types of information, Scim uses a pretrained language model~\cite{wang-etal-2021-minilmv2} to classify salient sentences within papers into one of four information facets: research objectives, novel aspects of the research, methodology, and results, coupled with heuristics that ensure an even distribution of highlights. Usability studies of Scim have shown these highlights can reduce the time it takes readers to find specific information within a paper.
Readers found Scim particularly useful when skimming text-dense papers, or for papers that fell outside their area of expertise. Moreover, readers learned to use both Scim's inline highlights and a sidebar summary of highlights to augment their existing reading strategies.









\subsection{Low-Vision Navigation Support and Reader-Sourced Hyperlinks with Ocean}
\label{sec:ocean}
The task of navigating between sections and retrieving content can be particularly challenging for blind and low-vision readers due to limitations in auditory information access or small viewports under high magnification~\cite{Szpiro2016HowPW}.
Even when related content is linked, a small viewport can make navigation difficult and necessitate scrolling~\cite{park-cscw22}.
Most existing tools such as for auditory skimming~\cite{Khan2020DesigningAE} do not address such challenges associated with low-vision and magnification.

Ocean~\cite{park-cscw22} minimizes scrolling requirements for low-vision readers by providing bi-directional, viewport-preserving hyperlinks that enable navigating to and from associated content without disrupting the viewport.
Based on reported findings from interviews with low-vision readers, Ocean also allows for easily revisiting portions of the paper with tabbed reading.
Since papers do not always provide hyperlinks and automated link creation is imperfect, Ocean includes an authoring interface that allows readers to create and share paper links during reading. An exploratory field deployment study with mixed-ability groups of low-vision and sighted readers revealed that readers found value in creating and consuming these links, and that reader-created links can increase trust. %



