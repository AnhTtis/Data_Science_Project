


\section{Unlocking Citations for Discovery}
\label{sec:citation-discovery}

Scholars use many methods to discover relevant {\papers} to read, including search engines, word of mouth, and browsing familiar venues. However, once they find one {\paper}, it's especially common for scholars to use its references and citations  to further expand their knowledge of a research area. This behavior, sometimes referred to as \emph{forward/backward chaining} or \emph{footnote chasing}, is ubiquitous and has been observed across many scholarly disciplines \cite{palmer2009scholarly}.
Supporting this, one popular feature in the Semantic Reader\footnoteref{product} is in-situ Paper Cards that pop up when {\users} click on an inline citation, dramatically reducing the interaction cost caused by 
jumping back-and-forth between inline citations and their corresponding references at the end of a {\paper} (Figure~\ref{fig:product}).
Despite this affordance, during literature reviews, {\users} may still be overwhelmed trying to make sense of the tens to hundreds of inline citations in each paper \cite{Denney2013HowTW,Peroni2015SettingOB,Chang2022CiteSee}.
Conversely, when reading a given paper, a reader cannot see  relevant follow-on {\papers} that cited the current paper.
Here we discuss how interactive {\systems} can help scholars more effectively explore citations to important relevant work in both directions with two systems called CiteSee \cite{Chang2022CiteSee} and CiteRead \cite{Rachatasumrit2022CiteReadIL}.







\subsection{Augmenting Citations with CiteSee}
\label{sec:citesee}



\begin{figure}[t]
    \centering
    \includegraphics[width=1\columnwidth]{figs/cite-see-03-22-23.png}
    \caption{
CiteSee \cite{Chang2022CiteSee} highlights citations to familiar papers (e.g., recently read or saved in their libraries) 
as well as unfamiliar papers to help readers avoid overlooking important citations when conducting literature reviews.
    Clicking on \emph{Expand} surfaces additional context, such as citing sentences from recently read papers.
    }
    \label{fig:citation-discovery}
\end{figure}





While most prior work on supporting {\paper} discovery has focused on developing bespoke interfaces of recommender systems or visualizations based on paper contents \cite{Sugiyama2010ScholarlyPR,Philip2014ApplicationOC}, the citation graph \cite{Huang2002AGR,Gori2006ResearchPR,Xia2016ScientificAR,Mackinlay1995AnOU,chau2011apolo,He2019PaperPolesFA,Ponsard2016PaperQuestAV}, or a combination of the two \cite{Wang2011CollaborativeTM,cohan2020specter}, {\paper} discovery via inline citations in a reading interface is important but under-explored. 
One study estimates that reading and exploring inline citations accounts for around one in five  {\paper} discoveries during active research \cite{King2009ScholarlyJI}. However, while all inline citations are relevant to the current {\paper}, it is likely that some are more relevant to the current {\user} than others.
For example, a {\user} reading papers about \emph{aspect extraction of online product reviews} to learn more about \emph{natural language processing techniques} would be less interested in citations to {\papers} around \emph{e-commerce and marketing}.
In addition,  citations to the same {\papers} often have different surface forms across papers (i.e., reference numbers), making it all the more difficult for {\users} to keep track of all the inline citations they should explore or have already explored during literature reviews.

To address this, CiteSee provides a personalized {\paper} reading experience by automatically identifying and resolving inline citations in PDFs to research paper entities in our academic graph \cite{kinney2023semantic}, and visually augmenting inline citations based on their connections to the current {\user}.
First, CiteSee leverages a {\user}'s reading behavior and history as a way to capture their short-term and fluid interests during literature reviews. Using this signal, CiteSee scores and highlights inline citations to help the {\user} triage them and discover prior work that are likely relevant to their literature review topics (Figure~\ref{fig:citation-discovery}). Second, CiteSee leverages {\papers} saved in the {\user}'s Semantic Scholar paper library and the {\user}'s publication record \cite{kinney2023semantic} to understand their longer-term research interests. Using this signal, CiteSee changes the colors of the inline citations to familiar papers so that the {\user} can both better contextualize the current paper and keep track of citations to papers they have already explored.
In addition, CiteSee also helps {\users} better make sense of the cited papers by showing how they connect to a {\user}'s previous activities; for example, showing which library folders they were saved under or the citing sentences from a familiar {\paper} (Figure~\ref{fig:citation-discovery}).
Based on lab and field studies, CiteSee showed promise that providing visual augmentation and personalized context around inline citations in an interactive reading environment can allow {\users} to more effectively discover relevant prior work and keep track of their exploration during real-world literature review tasks.



\begin{figure}[h]
    \centering
    \includegraphics[width=1\columnwidth]{figs/CACM-paper-figures-01-31-23-05-crop2.png}
    \caption{
   CiteRead \cite{Rachatasumrit2022CiteReadIL} finds subsequently published citing {\papers}, extracts the citation context, and localizes it to relevant parts of the current {\paper} as margin notes. This allows {\users} to become aware of important follow on work and explore them in-situ.
    }
    \label{fig:citation-citeread}
\end{figure}




\subsection{Exploring Future Work with CiteRead}
\label{sec:cite-read}





While augmenting inline citations helps readers to triage them, many relevant {\papers} are not cited in a {\paper} in the first place, for example, because they were published afterwards.
CiteRead is a novel {\system} that helps {\users} discover how follow-on work has built on or engaged with the {\paper} \cite{Rachatasumrit2022CiteReadIL}.
Much like social document annotation systems~\cite{zyto2012successful}, CiteRead annotates text in the paper with margin notes containing relevant commentary from citing papers~\cite{nakov2004citances}, thereby helping the reader to become aware of the citing paper and its connection.
In order to produce these annotations automatically, CiteRead first filters citing {\papers} for ones that are most relevant to the reader using a trained model atop a number of features representing citational discourse and textual similarity, i.e. from scientific paper embeddings~\cite{cohan2020specter}.
CiteRead then localizes citing papers to particular spans of text in the paper being read, and extracts relevant information from the citing paper.
Figure~\ref{fig:citation-citeread} shows a {\paper} annotated with this information from citing papers.
Localization is a technical challenge because while inline citations reference cited papers, they do not typically reference specific locations in the cited paper; CiteRead determines location by looking for overlapping spans of text (e.g., a number in common in the citing paper and the cited paper) or localizes to the relevant section when this overlap is unavailable. With CiteRead, a {\user} can directly examine follow-on work while keeping the citation contexts of both the current paper and the citing paper.
In a lab study, CiteRead helped readers better understand a {\paper} and its follow-on work compared to providing readers with a separate interface for faceted browsing of follow-on work. %





