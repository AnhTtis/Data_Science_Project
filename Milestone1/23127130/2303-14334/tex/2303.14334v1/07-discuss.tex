
















    
    
    



\section{Discussion and Future Work}

There are additional directions to explore to better support scholarly activities through the Semantic Reader Project. 

\paragraph{Towards a full-featured reading experience.} One question is how to integrate the different kinds of functionality across these projects into one coherent user interface, especially as we migrate research features into the production interface. 
Another question is how to develop support for the oftentimes social and collaborative nature of scholarly reading. Scholars frequently leverage their social networks and other social signals for paper discovery~\cite{kang_from_who_you_know}, work in groups to conduct literature review triage and synthesis, or engage in reading group discussions to aid comprehension. Existing augmentations within the Semantic Reader product could imbue social information, such as providing signals from one's co-author network (e.g., in CiteSee \S\ref{sec:citesee}) or aggregate navigation traces (e.g., in Scim \S\ref{sec:scim}). The publicly-available Semantic Reader tool could also scaffold the creation of novel crowd- or community-sourced content, such as author- or reader-provided explanations, commentary, or verification of paper content.
Finally is the question of how we can allow the scholarly community to step in where current AI systems fall short, such as by fixing improperly-extracted content or incorrect generated text which are especially problematic for interfaces  such as SciA11y (\S\ref{sec:accessibility}).


\paragraph{Advancing AI for scholarly documents.}
The Semantic Reader Project presents an opportunity for further AI research in scholarly document processing, especially when paired with human-centered research grounded in user-validated systems and scenarios. 
The bar for deploying AI models to support real-world reading is high; we often found during iterative design and usability studies that even slight errors in these models can have detrimental effects on the readers.
Until recently, interface design could require months of development of bespoke AI models which creates a barrier for quickly iterate different system designs.
Recent advancements in scaling large language models (LLMs) has altered this landscape by enabling researchers to experiment with a wide range of new NLP capabilities at relatively low cost~\cite{gpt3-brown-2020}.
This has the potential of significantly lowering the cost of human-centered AI design by incorporating user feedback in earlier stages of system development to create AI systems that work in symphony with the users beyond pure automation \cite{shneiderman2022human}.
For example, when developing Paper Plain (\S\ref{sec:paper-plain}), LLMs enabled us to quickly test different granularities and complexity-levels of plain language summaries with participants, eschewing the need for expensive changes to data requirements and model retraining.
In the near-term, we will revisit interface designs relying on bespoke AI models to evaluate whether LLMs can close the gap between research prototype and ready-for-production (e.g., more accurate definition identification for ScholarPhi \S\ref{s:individual-terms}).
Longer-term, we will explore whether LLMs can power new interactions (e.g., user-provided natural language queries while reading~\cite{dasigi-etal-2021-dataset,wadden-etal-2020-fact}).
While recent work has shown that these models can occasionally make critical errors or generate factually incorrect text when processing scientific text~\cite{otmakhova-etal-2022-patient},
we remain cautiously optimistic about developing ways to address their limitations \cite{dove2017ux,Lee2010GracefullyMB}. 







\paragraph{Ethics of augmented papers}
Finally, all the new interfaces for reading that we propose pose a number of important ethical considerations that will require further research and discussion. One aspect that arises with any system for elevating certain papers or certain content over others is bias.
For instance, using signals such as citation counts faces the risk of a ``rich get richer'' bias, which can reflect other kinds of documented biases \cite{Beel2009GoogleSR,Maliniak2013TheGC,way2019productivity}. As a result, systems such as CiteSee (\S\ref{sec:citesee}) or Relatedly (\ref{sec:relatedly}) should carefully consider additional signals of relevance such as semantic similarity to surface newer and overlooked papers. 
Another tension that we have encountered is the potential discrepancy between author desires and reader desires for how a work is presented and how much control to provide authors. For instance, in our work on Papeo (\S\ref{sec:papeo}), we found that authors desired control over placement of their talk video snippets, even as they found automated mapping support to be helpful. In other cases, authors might not have the requisite expertise (e.g., they may not have a good sense of reader needs or what non-experts are confused by) or may have the wrong incentives.
Future work should consider author perspectives on these augmented experiences.
A related issue is around systems for more efficient reading or synthesis, which may  encourage readers to take shortcuts that lead to incorrect understanding, sloppy research, or even outright plagiarism. Instead of simply seeking to increase reading throughput uniformly, our systems should enable \textit{triage}, so that readers can dedicate time for thoughtful and careful reading when the content is important.
For instance, our systems could design pathways that, while they may be more efficient, do not obfuscate the full context (e.g., Scim \S\ref{sec:scim}), and that encourage good practices such as verification and provenance tracing.
A final consideration is around what is ethical reuse of a paper's contents to support reader experiences outside of that paper and its licensing implications. For instance, CiteRead (\S\ref{sec:cite-read}) extracts paper citances and places them in the cited paper, and Relatedly (\S\ref{sec:relatedly}) extracts related work sections from different papers for users to explore. 
Recent trends in \emph{open science and datasets} \cite{mckiernan2016open,mckiernan2000arxiv,ginsparg2011arxiv,lo-etal-2020-s2orc} point to a promising future where we could continue to explore different ways to \emph{remix and reuse} scholarly content across context so that future scientists can take fuller advantage of prior research.










\section{Conclusion}

This paper describes the Semantic Reader Project, which currently consists of ten research prototypes focusing on supporting scientists around Discovery \cite{Chang2022CiteSee,Rachatasumrit2022CiteReadIL}, Efficiency \cite{Fok2023Scim,park-cscw22}, Comprehension \cite{Head2021AugmentingSP,August2022PaperPM,papeo}, Synthesis \cite{relatedly, Kang2022Threddy}, and Accessibility \cite{wang-2021-scia11y,paper2html} when reading {\papers}.
Validating our approach of augmenting existing PDFs of {\papers}, we have seen tremendous adoption of the freely-available Semantic Reader product\footref{product} which has grown to 10k weekly users.\footnote{As of late February, 2023} 
While we focused on augmenting PDF documents to support common scholar reading practices, all of our {\systems} are built with web technologies---allowing these novel interactions to extend to future publication formats which can be rendered in web browsers. 
We plan to continue experimenting with novel AI-powered intelligent {\systems}, as well as migrating successful interactive features 
into the product. Finally, we offer a collection of freely-available resources to the larger research community, including datasets of open-access research papers \cite{lo-etal-2020-s2orc}, APIs for accessing the academic citation graph \cite{kinney2023semantic}, machine learning models for processing and understanding {\papers} \cite{cohan2020specter,Cachola2020TLDRES,shen-etal-2022-vila,kang-etal-2020-document},\footref{papermage} and open-source software for rendering and augmenting PDF documents for developing reading interfaces.\footref{library} We hope by providing these resources we can enable and encourage the broader research community to work on exciting novel intelligent {\systems} for {\papers} with us.













