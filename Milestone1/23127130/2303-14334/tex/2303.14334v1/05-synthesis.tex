
\section{Scaffolding Synthesis with Related Work Sections}
\label{sec:bootstrap}











Scientific breakthroughs often rely upon scholars synthesizing multiple published works into broad overviews  to identify gaps in the current literature \cite{portenoy2022bursting}. For this, scholars periodically compile survey articles to help other scholars gain a comprehensive overview of important research topics. For example, some fields have dedicated outlet for such articles (e.g., the \textit{Psychological Bulletin} \cite{bem1995writing}). However, survey articles require significant time and effort to synthesize, and can quickly become outdated with the exponential growth of scientific publication \cite{Bornmann2020GrowthRO}. 

Instead, scholars in fast-paced disciplines often rely on the related work section when they need to better understand the broader background when reading a paper. 
While related work sections also summarize multiple prior works, unlike comprehensive survey articles, they typically provide partial views of the larger research topic most relevant to a single paper. There is an opportunity to build better tooling for scholars to consume and synthesize multiple related work sections across many papers to gain richer and more comprehensive overviews of fast-paced domains. 
The Threddy \cite{Kang2022Threddy} and Related \cite{relatedly} projects explored this opportunity using two different approaches: clipping and organizing research threads mentioned across papers \cite{Kang2022Threddy}, and directly exploring and reading related work sections extracted across many papers \cite{relatedly}.






\begin{figure}[t] 
     \centering
     \includegraphics[width=1\columnwidth]{figs/papeo3.png}
     \caption{Papeo \cite{papeo} enables authors to map segments of talk videos to relevant passages in the paper, allowing readers to fluidly switch between the two formats. Color-coded bars show the mapping between the two formats, and allow readers to scrub through video segments for quick previews.
     }
     \label{fig:papeo}
\end{figure}


\subsection{Clipping and Synthesizing across Papers with Threddy}
\label{sec:threddy}

Clipping and note-taking is one common approach to supporting synthesis across multiple documents.
Prior work has pointed to the importance of tightly integrating clipping and synthesis support in the reading process, and how incurring significant context-switching costs can be detrimental to sensemaking~\cite{kittur_chi13_cost_benefit,Russell_Sensemaking_1993,Pirolli_InformationForaging_1999}.
Therefore, recent work has developed tools aimed at reducing the cognitive and interaction costs of clipping~
\cite{liu_wigglite,chang_uist16_uncertain_highlighting} and structuring~\cite{chang_mesh,kuznetsov_fuse,liu_crystalline,liu2019unakite,texSketch} to support everyday online researchers \cite{chang_mesh}, programmers \cite{liu2019unakite}, and students \cite{texSketch}. However, designing clipping and synthesis support tools for research papers is relatively under-explored and introduces exciting new research opportunities. For example, additional organizational structures for literature reviews (\eg threads of prior work instead of tables \cite{liu2019unakite,chang_mesh}), and research paper discovery (\eg based on inline citations in clipped text).

For this, Threddy \cite{Kang2022Threddy} is a thread-focused clipping tool integrated into scholars' paper reading process to support literature review and discovery. Using Threddy, readers can select and save sentences into a sidebar from the related work sections of a paper. The system maintains rich context for each clip, including its provenance and inline citations. This allows readers to navigate back to the clipped paper and cited papers afterward. In the sidebar, readers can further organize clips collected across papers into a hierarchy of threads to form their view of the research landscape. The content of the sidebar is preserved across papers that were read over time, and provides valuable context for subsequent reading based on the emerging threads of research the reader have curated.
Finally, readers can further expand their coverage by exploring paper recommendations for each thread, based on the referenced papers in the corresponding clips. 
A lab study showed that Threddy was able to lower the interaction costs of saving clips while maintaining context, allowed participants to curate research threads without breaking reading flows, and discover interesting new papers to further grow their understanding of the research fields.



\subsection{Reading and Exploring Related Work Sections across Papers with Relatedly}
\label{sec:relatedly}

In contrast to Threddy, which aims to improve readers' existing literature review process through enhanced in-situ clipping and synthesis \cite{Kang2022Threddy}, the Relatedly system introduced a novel workflow that allows readers to explore many related work sections across papers in an interactive search and reading interface to quickly gain a comprehensive overview of rich research topics \cite{relatedly}. While prior work have explored providing overview structure of multiple documents based on citations \cite{Ponsard2016PaperQuestAV,chau2011apolo}, semantic similarity \cite{hearst2006clustering,shahaf2012metro}, or human computation \cite{hahn2016knowledge,chang2016alloy,luther2015crowdlines}, they could still lead to complex structures that are hard to interpret \cite{hearst1999use} or require significant crowdsourcing efforts. Relatedly sidesteps these issues by \emph{reusing} existing related work paragraphs in published papers which already cite sets of related references with descriptions connecting them \cite{relatedly}. As an example, consider a scholar trying to better understand the space of \emph{online misinformation}. With \emph{online misinformation} as the query term, Relatedly shows the reader a list of paragraphs that describe and cite multiple relevant prior work. Using a pretrained language model for summarization~\cite{lewis-etal-2020-bart}, Relatedly generates short and descriptive titles for each paragraph, and uses a diversity-based ranking algorithm so that the reader can quickly see and explore paragraphs describing different research threads, such as \emph{Fact Checking Datasets}, \emph{Social Media and Misinformation}, and \emph{Fake News Detection Techniques}. 

One challenge here is that paragraphs of the same threads often cite overlapping prior work, making them hard to explore and read while keeping track of which papers were new versus already explored. For this, Relatedly provides reading and cross-referencing support by keeping track of paragraphs and references explored by the readers. This allows Relatedly to help readers prioritize their reading for both breadth and depth.
Specifically, Relatedly dynamically re-ranks paragraphs and highlights sentences to spotlight unexplored and dissimilar references for breadth, but also allow readers to explore clusters of paragraphs that cited similar references for depth.
A usability study comparing Relatedly to a strong document-centric baseline showed that Relatedly led to participants writing summaries that were rated significantly more coherent, insightful, and detailed after 20 minutes of literature review.





