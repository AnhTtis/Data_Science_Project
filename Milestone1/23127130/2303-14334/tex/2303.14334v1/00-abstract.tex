\begin{abstract}
Scholarly publications are key to the transfer of knowledge from scholars to others. However, research papers are information-dense, and as the volume of the scientific literature grows, the need for new technology to support the reading process grows.
In contrast to the process of {\em finding} papers, which has been transformed by Internet technology, the experience of {\em reading} {\papers} has changed little in decades. 
The PDF format for sharing {\papers} is widely used due to its portability, but it has significant downsides including: static content, poor accessibility for low-vision readers, and difficulty reading on mobile devices.
This paper explores the question ``Can recent advances in AI and HCI power intelligent, interactive, and accessible reading interfaces---even for legacy PDFs?'' 
We describe the Semantic Reader Project, a collaborative effort across multiple institutions to explore automatic creation of dynamic reading interfaces for research papers. Through this project, we've developed ten research prototype interfaces and conducted usability studies with more than 300 participants and real-world users showing improved reading experiences for scholars. We've also released a production reading interface for {\papers} that will incorporate the best features as they mature. We structure this paper around challenges scholars and the public face when reading {\papers}---Discovery, Efficiency, Comprehension, Synthesis, and Accessibility---and present an overview of our progress and remaining open challenges.
\end{abstract}





