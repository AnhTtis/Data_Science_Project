\begin{table*}[ht]
    \centering
    \scriptsize
    \def\arraystretch{0.65}
    \setlength{\tabcolsep}{0.01em}
    \begin{tabular}{p{0.13\linewidth}@{\hskip 1em} p{0.3\linewidth}@{\hskip 1em} p{0.25\linewidth}@{\hskip 1em} p{0.25\linewidth}}
        Type & OntoNotes & PreCo & Phrase Detectives \\
        \midrule
        \textbf{Generic Mentions} \newline [dogs] can bark & Generics mentions are only annotated when they corefer with a pronoun or determinate noun phrase, or when they occur in a news headline. & All generic noun phrases and modifiers can be annotated as coreferring. & All generic noun phrases can be annotated as coreferring. \\
        \midrule
        \textbf{Verb Phrases} \newline it will [grow] & The head of a verb phrase can be annotated as coreferring with a determinant noun phrase. & Not annotated. & Not annotated. \\
        \midrule
        \textbf{Appositives} \newline [[Abe] , [the chef]] & Annotated in the dataset, but not considered coreference. & Annotated as three mentions: both noun phrases and the larger span. & Annotated in the dataset, but not considered coreference. \\
        \midrule
        \textbf{Copular Predicates} \newline [he] is [the teacher] & Not annotated. & In a copular structure, the referent and attribute are annotated as coreferring. & Annotated in the dataset, but not considered coreference. \\
        \midrule
        \midrule
        \textbf{Nesting} \newline [he [himself]] & When two nested mentions share a head, only the dominant mention is annotated. Proper nouns cannot contain nested mentions. & Appositives and mentions with shared heads are annotated as nested mentions. & The right-most mention in an appositive is considered referring to a distinct entity and can therefore be annotated as a nested mention. No restrictions on nesting of proper nouns. \\
        \midrule
        \textbf{Compound \newline Modifiers} \newline [Taiwan] authorities & Compound modifiers are annotated if non-adjective proper nouns that are not a nationality acronym. & All compound modifiers can be annotated as coreferring. & No explicit restrictions on the annotation of compound modifiers. \\
        \bottomrule
    \end{tabular}
    \caption{Noted differences in how coreference is defined and operationalized in the training datasets.}
    \label{tab:general-cr-differences}
\end{table*}
