\documentclass[12pt]{article}

\usepackage{amssymb,amsmath,amsthm,bbm,enumitem,xcolor}
\usepackage{graphicx,caption}
\usepackage{subcaption}
\usepackage[a4paper, total={6.3in, 9.2in}]{geometry}
\usepackage{setspace}
\usepackage{hyperref}
\usepackage[title]{appendix}
%\usepackage[%
%    pdfborder={0 0 0},%  <=== make some edits toward AGU style links (on the web page mentioned)
 %   colorlinks=true,%
 %   linkcolor=blue, %
 %   citecolor=blue, % 
 %   urlcolor=blue,  % 
 %   breaklinks=true,%
%            ]{hyperref}
%\usepackage{url}    % <== linebreaking in url       

\usepackage{apacite} 
\AtBeginDocument{%
   \renewcommand{\BRetrievedFrom}{}  
  % \renewcommand{\BAstyle}{\itshape} % <=== made some other edits toward AGU style...
   \urlstyle{APACsame}%
   \renewcommand{\BBAA}{and}%
  % \renewcommand{\BBOP}{[}%
   %\renewcommand{\BBCP}{}%
}
\usepackage[round]{natbib}    
%\usepackage[style=apa, backend=biber]{biblatex}


\newtheorem{theorem}{Theorem}[section]
\theoremstyle{definition}
\newtheorem{definition}{Definition}[section]
\theoremstyle{definition}
\newtheorem{example}{Example}[section]
\theoremstyle{definition}
\newtheorem{assumption}{Assumption}[section]
\theoremstyle{definition}
\newtheorem{proposition}{Proposition}[section]
\theoremstyle{definition}
\newtheorem{corollary}{Corollary}[section]
\theoremstyle{remark}
\newtheorem{remark}{Remark}[section]
\theoremstyle{definition}
\newtheorem{lemma}{Lemma}[section]
\DeclareMathOperator{\naturalnumber}{\mathbb{N}}
\DeclareMathOperator{\real}{\mathbb{R}}
\DeclareMathOperator{\integer}{\mathbb{Z}}
\DeclareMathOperator{\p}{\mathbb{P}}
\DeclareMathOperator{\E}{\mathbb{E}}
\DeclareMathOperator{\borel}{\mathcal{B}}
\DeclareMathOperator{\1}{\mathbbm{1}}
\newcommand{\filtration}[2]{\mathcal{F}^{#1}_{#2}}
\newcommand{\density}[3]{\frac{d\mu_{ {#1} }}{ d\mu_{#2} }\left({#3} \right)}
\newcommand{\spacefunctionalnewnew}{C\left([0,T];\real \right)}
\newcommand{\D}{\mathbb{D} }
\newcommand{\ltwo}{L^2(\Omega,\mathcal{F},\p)}
\newcommand{\cp}{C^\infty_p(\real^n)}
\newcommand{\dom}{\text{Dom}(\delta)}
\allowdisplaybreaks

\setlist[itemize]{leftmargin=0.4cm,labelindent=\parindent}

\title{Optimal Investment in Defined Contribution Pension Schemes with Forward Utility Preferences}
\author{Kenneth Tsz Hin Ng\thanks{Department of Mathematics, University of Illinois at Urbana-Champaign, Urbana, Illinois, United States. Email: \href{mailto:tszhinn2@illinois.edu}{tszhinn2@illinois.edu}.}    \;and   Wing Fung Chong\thanks{Maxwell Institute for Mathematical Sciences and Department of Actuarial Mathematics and Statistics, Heriot-Watt University, Edinburgh, Scotland, United Kindgom. Email: \href{mailto:alfred.chong@hw.ac.uk}{alfred.chong@hw.ac.uk}.}}
%\parindent 0pt
\date{\today}

\begin{document}
\maketitle

\begin{abstract}
Optimal investment strategies of an individual worker during the accumulation phase in the defined contribution pension scheme have been well studied in the literature. Most of them adopted the classical backward model and approach, but any pre-specifications of retirement time, preferences, and market environment models do not often hold in such a prolonged horizon of the pension scheme. Pre-commitment to ensure the time-consistency of an optimal investment strategy derived from the backward model and approach leads the supposedly optimal strategy to be sub-optimal in the actual realizations. This paper revisits the optimal investment problem for the worker during the accumulation phase in the defined contribution pension scheme, via the forward preferences which resolve the pre-specification issues in the backward model and approach. Stochastic partial differential equation representation for the worker's forward preferences is illustrated. This paper  constructs two of the forward utility preferences and solves the corresponding optimal investment strategies, in the cases of initial power and exponential utility functions.
% We consider the problem of optimal dynamic asset allocations for defined contribution pension funds. Instead of assuming a fixed terminal utility, we construct forward utility preferences of a pension member, which suit better the long term feature of a pension fund that evolves in response to the dynamically changing market and salary conditions. Herein, the homothetic forward processes of power and exponential utilities are considered. The resulting utilities and optimal investment strategies can be represented in terms of a pseudo fund under an exogenously chosen investment strategy, which connects with the relative performance criteria in fund management. 
\end{abstract}

\textit{Keywords}: Optimal investment, defined contribution pension scheme, forward utility preferences, pre-commitment resolution, exogenous baseline strategy.\\

% Forward utility, CRRA utility, CARA utility, DC pension, Stochastic partial differential equations, baseline strategy, martingales, CPPI, optimal investment, habit formation  \\

\textit{JEL Classifications}: G22, G11, C61.


\section{Introduction}
% \subsection{Literature}
%%%% DC Pension and model
Pension schemes are classified into two categories, namely the defined benefit (DB) and the defined contribution (DC), which compose the hybrid and the collective defined contribution schemes. On one hand, the DB scheme pre-defines a retirement benefit, while its plan members are required to contribute in accordance with the performance of the fund in order to maintain their future retirement benefits. On the other hand, the DC scheme asks its plan members to contribute a pre-determined proportion of their salaries to the fund, while their retirement benefits are based on, not only their portions of contribution from compensation, but also future investment returns by their customized portfolios. Shifts from the DB scheme to the DC scheme are evident in many countries. As the investment risk is fully borne by plan members in the DC scheme, seeking for the optimal investment strategies of its members has been a frontier research since the change of paradigm.\\
% Members of DC plans are required to contribute a predefined proportion of their salaries to the plan such that the retirement benefits are determined by their contributions along with the returns from their customized investments. DB plans, on the other hand, has a predefined retirement benefit and members are required to contribute in accordance with the performance of the fund in order to maintain the retirement benefit. Due to the increased workforce mobility and flexibility in selecting a preferred investment plan, DC pension plans are becoming dominant in the past few decades and widely used in the private sectors in most countries.    \\
% The shift from defined benefit to defined contribution pension

In the literature, optimal investment strategies of a worker (she/her) during her accumulation phase in the DC pension scheme were investigated with various objectives and quantities of interest. To name a few, \cite{BOULIER2001173}, \cite{HAN2012172:DC:inflation}, and \cite{GUAN201458:stochastic:interest:volatiltiy} maximized her expected utility on the difference between her pension fund value and an exogenous guarantee amount at the time of retirement. Instead of evaluating her pension fund value in absolute term, \cite{CAIRNS2006843:stochastic:lifestyle} suggested habit-formation in life-cycle models, that the worker would maximize her expected utility on her pension fund value with respect to her salary, as a num{\'e}raire, at her retirement time. Instead of assuming the classical expected utility as her preference, \cite{GUAN2016224:var:constraint} and \cite{CHEN2017137:minimum:performance} solved her optimal investment strategy by maximizing her expected $S$-shaped utility, on her absolute pension fund value in \cite{GUAN2016224:var:constraint}, and on the difference between her absolute fund value and an exogenous guarantee amount in \cite{CHEN2017137:minimum:performance}, both at her time of retirement. \cite{YAO201484:mean:variance} and \cite{GUAN201599:mean:variance:2} assumed the mean-variance objective for the worker, evaluating on her absolute pension fund value at her retirement time.\\

All of the aforementioned works, among others in the literature, adopted a classical backward model, and hence employed the classical backward approach to solve the optimal investment strategies. Any of these backward models sequentially pre-specify, at the current time when the optimal investment strategy is determined, (i) the worker's future time of retirement, (ii) her future preferences to evaluate her pension fund's performance, and (iii) models to govern the future dynamics between the current time and her retirement time. The worker's optimal investment strategy is then solved backwardly, via, such as, dynamic programming principle (DPP), backward stochastic differential equation (BSDE) approach, or martingale method. Though it is solved backwardly, her optimal investment strategy being planned is implemented forwardly in time.\\

%evidence 
However, the revealed retirement time of the worker shall most likely be deviated from her assumption when the optimal investment strategy is planned. For example, in the United Kingdom, the State Pension age, which is the earliest age when a state pension could be claimed and acts as a proxy of retirement age, has been gradually increasing among generations. By the Pensions Act 2007, the State Pension age for women was increased from 60 to 65, over the period from April 2010 to April 2020. The Act also planned to increase the State Pension age for both men and women, from 65 to 66 phasing in between April 2024 and April 2026, from 66 to 67 between April 2034 and April 2036, and from 67 to 68 between April 2044 and April 2046. Also, her future preferences, assumed when the optimal investment strategy is planned, are most likely not able to capture the worker's change of risk appetite due to the actual realizations in the market environment. In addition, the assumed models for the market environment are certainly not going to imitate its true dynamics. For instance, it could be, microscopically, that a drift or a volatility of a dynamics, be it calibrated, does not behave as being revealed in real time; or, a systemic risk, such as pandemic or climate, is unfolded but was not taken into consideration; or, a state-of-the-art investment opportunity, such as green bond, and environmental, social, and governance (ESG) stock, emerges but was not available at the inception time of the worker's pension scheme.\\

%fail to mode-update in backward
These together drive the sub-optimality of the worker's planned investment strategy, which is supposed to be optimal. Given a long horizon, typically at least 40 years, this is particularly significant when determining investment strategies of pension plans. In order to forwardly execute the backwardly solved strategy, she must actually pre-commit the assumed backward model to ensure time-consistency of the strategy with respect to the model, but not necessarily to the actual realizations in the market environment. One might then incorporate updates on the backward model period-by-period, based on the latest experiences in the market environment, and, after each model-update, revise the investment strategy, which is again solved backwardly, going forward. However, not only does such an approach violate the time-consistency of the strategies with respect to the models, as well as require substantial computational resources, but also, due to the classical backward approach to solve the strategies, an earlier portfolio adjustment action has already been contaminated by the mis-specified backward models of the market environment, albeit models being recurrently updated later.\\

Properly solving the optimal investment strategies during the accumulation phase of the DC pension scheme thus demands a call for forward model and approach. These should not pre-specify the worker's retirement time as well as her future preferences, and, while it is inevitable to assume a model for future dynamics, if model-update is regularly carried out, any portfolio adjustment actions before an update shall still be optimal. In other words, the optimal investment strategy should be time-consistent with respect to, not only the model assumed at the current time, but also the recurrently revised models, if any, using the actual realizations in the market environment. This would be feasible by a forward approach to solve the optimal investment strategy.\\






% {\color{red}
% In this paper, we conciple osider the problem of optimal asset allocations in an equity market under a DC pension plan. The management of a pension fund is customized to achieve different objectives and preferences.  In classical settings where one seeks to maximize her expected utility at the time of retirement, \cite{BOULIER2001173,HAN2012172:DC:inflation,GUAN201458:stochastic:interest:volatiltiy} considered maximizing the difference of the terminal fund value with an exogenous guarantee amount. Instead of using the absolute fund value as a measure of the level of satisfaction, \cite{CAIRNS2006843:stochastic:lifestyle} suggested a measurement based on a worker's ability to maintain her current living standard  by using her salary as a numeraire. Deviating from the classical utility,  \cite{GUAN2016224:var:constraint,CHEN2017137:minimum:performance} considered using an expected S-shaped utility on the absolute fund value; and \cite{YAO201484:mean:variance,GUAN201599:mean:variance:2} employed a mean-variance approach on the absolute fund value.  Despite the abundance of studies, the utilities or preferences in the aforementioned works are predefined at a terminal time. However, due to the long-lasting nature of a pension plan, especially with the shift of retirement age under longer life expectancy, assuming a predetermined terminal utility over the entire life of a pension fund may not best describe the members' changing risk of aversion and their utility preferences.  Therefore, the objective herein is to employ forward utilities for valuing the performance of a DC pension fund, using her (stochastic) salary as a numeraire, which can address the aforementioned issues of precommitment in backward settings.}  \\     

%group by objective, e.g. S-shaped utility, classical utility, etc. => why forward utility , since the above are all backward , motivation!

%\cite{dynamic:preference:pension}
%%%% Forward preference 
Forward preferences transpire to be a right notion for this. They were pioneered by \cite{Musiela2007,Musiela2008,Musiela2009,Musiela2010a,Musiela2010b,Musiela2011}, in which the forward preferences, as well as the corresponding optimal investment strategies, are constructed and solved for an agent (she/her) investing in a financial market, but without any considerations of her salary as well as its associated non-hedgeable risk which are crucial components during the accumulation phase of the DC pension scheme. First, the forward model pre-specifies, at the current time when the optimal investment strategy is determined, (i) the current preference of the agent, and (ii) a model of future dynamics in the market environment from the current time onward. The forward approach then, from the current time onward, constructs her future preferences, and hence the name as forward preferences, and solves her optimal investment strategy, via a forward induction on the Bellman optimality equation (which would be the backward induction on the equation, for DPP in the classical backward model and approach), or equivalently the forward version of the (super-)martingale (sub-)optimality principle. Therefore, in the forward model and approach, the agent's optimal investment strategy is solved and implemented forwardly in the same direction of time, and hence any portfolio adjustment actions before a model-update, if any, are still optimal without any time-inconsistent issues, even though the agent does not fully commit to the assumed and recurrently updated, if any, models.\\
 

Since the above series of works by Musiela and Zariphopoulou, forward preferences have been extensively advanced in the literature on, for instance, dual characterization by \cite{zitkovi:2009}, stochastic partial differential equation (SPDE) representation by \cite{karoui:2013}, \cite{shkolnikov:2016}, and \cite{El:2018:FBSPDE}, homothetic processes by \cite{Nadtochiy2014}, ergodic BSDE and infinite horizon BSDE representations by \cite{liang:bsde}, model uncertainty using dual characterization by \cite{kallblad2018dynamically}, model uncertainty via direct construction by \cite{chong2019optimal}, mean-field games by \cite{lacker2019mean}, \cite{dosreisL:mean:field}, and \cite{dosreisL:mean:field2}, discrete-time binomial model by \cite{angoshtari:2020}, inverse investment problem by \cite{sigrid:inverse:investment}, rank-dependent preference by \cite{He:2021:rank:dependent}, Arrow-Pratt measure by \cite{strub2021evolution}, as well as investment and reinsurance by \cite{colaneri:katia:reinsurance:2021,colaneri:katia:reinsurance:2022}. Numerous applications of forward preferences can be observed in the literature on, maturity-independent risk measures by \cite{zitkovi:2010}, valuation of American options by \cite{leung2012forward}, indifference valuation in discrete-time binomial model by \cite{musiela2010indifference}, fund management by \cite{dynamic:preference:pension}, \cite{anthropelos:relative}, and \cite{hillairet:social}, forward entropic risk measures by \cite{chong2019ergodic}, pricing and hedging equity-linked life insurance by \cite{CHONG201993}, as well as robo-advising by \cite{robo:2021} and \cite{capponi2022personalized}.\\

% With the aforementioned advantages over its backward counterpart, forward utility preferences have been developed rapidly with applications in financial markets. Notably,  forward utility preferences have been incorporated into rank-dependent utilities  \cite{He:2021:rank:dependent}; mean-field games with relative performance considerations in \cite{dosreisL:mean:field,lalacker2019mean}; robo-advising in \cite{capponi2022personalized,robo:2021}; evolution of the Arrow-Pratt measure in \cite{strub2021evolution}; discrete-time setting with predictable performance process in \cite{angoshtari:2020}; inverse investment problem in \cite{sigrid:inverse:investment}; relative performance in fund management \cite{anthropelos:relative}; optimal investment-consumption problems with model ambiguity \cite{chong2019optimal}; forward entropic risk measures in \cite{chong2019ergodic}; model uncertainty via dual characterization in \cite{kallblad2018dynamically}. See also some earlier works  \cite{shkolnikov:2016,Nadtochiy2014,liang:bsde,karoui:2013,leung2012forward,zitkovi:2010,musiela2010indifference,anthropelos:2010,zitkovi:2009} and the references therein.\\
% \nocite{zitkovi:2009}



% Forward utilities were first introduced in a series of works . Unlike classical backward utility optimizations where the length of the investment horizon $T$ and the terminal utility are pre-committed, forward utility preferences are constructed by employing the dynamic programming principle (DPP) forward in time given an initial utility function $u_0(x)$. Hence, not only could the forward nature addresses the shortcomings of pre-commitment, but it also allows adaptations to the changing market and labour information, as well as the risk preferences of workers while preserving time-consistency. These properties are accordant with the prolonged investment period of a pension fund.





% {\color{red}  
% Forward utilities had not been  major measures of one's preference in the actuarial community until recently, where their nature of non-precommitment and ability of incorporating model updates have started to gain attentions, and making them compelling tools in modeling prolonged  investments and funds, such as a pension plan. To name a few recent studies, \cite{dynamic:preference:pension} considered an insurance-reinsurance problem using a forward exponential utility; \cite{CHONG201993} studied the pricing and hedging equity-linked life insurance contracts by the principle of indifference pricing; \cite{colaneri:katia:reinsurance:2021,colaneri:katia:reinsurance:2022} studied the optimal investment and proportional reinsurance problem  using a forward exponential utility; \cite{hillairet:social} considered a pension fund management  for a pay-as-you-go pension scheme using a time-consistent dynamic utility in the social planner's perspective. Using a relaxed type of forward preference known as stochastic utilities, which do not require the underlying random field to satisfy the time-consistent (super-)martingale condition, \cite{YE2019193} considered the problem of optimal investment, consumption and insurance purchase.  }\\  
%Recently, forward preferences have emerged as a compelling tool in modeling insurance products and pension funds, thanks to its flexibility of model update, which is particularly suitable for prolonged funds and investments.

This paper solves the optimal investment strategy of the worker enrolled to the DC pension scheme during the accumulation phase, using the forward utility preferences. Inspired by \cite{CAIRNS2006843:stochastic:lifestyle}, the worker evaluates her pension fund performance during the accumulation phase by the ratio of the fund value to her salary, ensuring that her pre-retirement habit could be consistent after any of her possible future retirement times. Given her current preference as her initial utility function on the ratio, her future utility preferences on the ratio are constructed, instead of being pre-specified, and her optimal investment strategy is correspondingly solved (see Definition \ref{def:forward} below). Again, by the forward model and approach, the planned investment strategy applies to any time moving forward, without the need of committing to any pre-specified retirement times; technically speaking, the solved optimal investment strategy is independent of any terminal times, and thus contributes the flexibility for the worker to decide her actual retirement time in due course. While one of the advantages for adopting the forward model and approach is, as mentioned above, the compatibility of model-update for future dynamics and time-consistency for regularly revised strategy, this paper solves the optimal investment strategy at the current time, with the focus on addressing the pre-specification issues, of (i) the worker's future retirement time and (ii) her future preferences, under the classical backward model and approach. Incorporating recurrent updates on the models for the market environment and regular revisions on the optimal time-consistent investment strategy using the forward preferences, as in \cite{robo:2021}, is very relevant to pension fund management, typically with a prolonged horizon; this shall be studied as one of the future directions.\\

A long-standing technical challenge of the forward preferences is their non-uniqueness, and thus any associated problems are ill-posed; positively, this explains why individual preferences vary among agents. To illustrate the non-uniqueness of the worker's forward preferences and the corresponding optimal investment strategy, her (translated) forward preferences are characterized by an SPDE (see Equation \eqref{eq:SPDE} below), which particularly highlights the volatility processes of the (translated) forward preferences with respect to the financial market and the labour market. Unlike the endogenously implied volatility processes of the backward preferences derived from the backward model and approach, the volatility processes of the (translated) forward preferences are exogenously chosen by the worker herself at the current time, when she plans the optimal investment strategy going forward. A pair of volatility processes maps an SPDE for her forward preferences, which also then induces her optimal investment strategy (see Equation \eqref{eq:SPDE_strategy} below).\\

Following the SPDE representation, two forward preferences of the worker are constructed which solve the SPDE (see Equations \eqref{eq:U:power} and \eqref{eq:U:exp} below), and her corresponding optimal investment strategies are explicitly solved (see Equations \eqref{eq:pi*:power:1} and  \eqref{eq:pi*:exp} below). One case is when the worker's current preference on the ratio is given by the constant relative risk aversion (CRRA) power utility function, while the other case is specified by the constant absolute risk aversion (CARA) exponential utility function. In both cases, an exogenous baseline investment strategy emerges to drive the worker's forward preferences and her optimal strategy. Her forward preferences evaluate the comparison between, the ratio of her pension fund value to her salary, and the exogenous ratio generated by the baseline strategy. Such constructions are similar to relative performance criteria using the forward utility preferences in fund management of a financial market by \cite{anthropelos:relative}; see also \cite{dosreisL:mean:field} for the mean-field game setting. They constructed the forward utility preferences for one of the agents by comparing her fund performance to the other agent's (respectively, agents', in the mean-field game setting) performance, while the forward utility preferences of the worker constructed in this paper compare her fund performance to the exogenous performance, such as generated by a target-date fund. In the CRRA power utility case, any of the worker's admissible investment strategies leads to her pension fund value to salary ratio being bounded below by the exogenous ratio as a floor, which resembles the requirement in constructing the constant proportion portfolio insurance (CPPI) strategy in \cite{perold:1986:cppi}, \cite{black1987simplifying}, \cite{CPPI:1989}, and \cite{CPPI:1992}. In the CARA exponential utility case, the reciprocal of a similar exogenous ratio, also generated by the baseline strategy but without salary contribution, serves as a dynamic risk aversion process of the worker.  \\

As for the worker's optimal investment strategies, in both cases, they are in a form of (convex) combination of the baseline investment strategy and a myopic strategy, in which the myopic component resembles the optimal investment strategy in the respective cases under the classical backward model and approach. In the CRRA power utility case, her optimal investment strategy can also be viewed as a generalization of the CPPI strategy, with (i) the stochastic floor given by the exogenous baseline fund performance, (ii) a stochastic multiple of the cushion given by the myopic strategy, and (iii) an additive factor given by the amount of the exogenous baseline fund investing into a risky asset. Such a perspective does not hold for the CARA exponential utility case, as the other similar exogenous ratio, but without salary contribution, does not attribute as a floor of the worker's pension fund value to salary ratio. \\

%similar to relative performance criteria A (2022) and dos R (2022)
%power: CPPI relation
%exponential: dynamic risk aversion
%optimal investment strategy as combinations of myopic and baseline
%myopic resembles classical backward setting one

%technical contributions
%non-uniqueness and stochasticity; unlike backward being implied
%constructed depends on the comparison between exogeneous baseline strategy
%stochasticity show by SPDE with volatility processes
%power and exp
%emphasize again that our construction is not the only possible one

To the best of knowledge, only \cite{dynamic:preference:pension} and \cite{hillairet:social} investigated pension management problems using the forward utility preferences. \cite{dynamic:preference:pension} constructed both CRRA power forward utility preferences, and symmetric asymptotic hyperbolic absolute risk aversion (SAHARA) forward utility preferences, of the worker, as well as solved her corresponding optimal CPPI and life-cycle investment strategies respectively. \cite{hillairet:social} proposed an investment-pension management problem for a social planner, who is endowed with forward utility preferences and solves the optimal investment-pension pair for her society, which consists of, workers who contribute their portion of compensation to a social pension system, and pensioners who receive their fixed amount of pension income from the system. This paper is different from \cite{dynamic:preference:pension} and \cite{hillairet:social}. Comparing to the former one, this paper takes the worker's salary contribution as well as its associated non-hedgeable risk into considerations, the worker herein evaluates her DC pension scheme fund performance by her fund value to salary ratio following the accumulation phase habit-formation advocate in \cite{CAIRNS2006843:stochastic:lifestyle}, and the worker's forward utility preference orders are caused by the comparison of her pension fund performance to the exogenous stochastic baseline. Comparing to the latter one, this paper solves an individual worker's pension fund management problem microscopically, while \cite{hillairet:social} concerns the social pension system at the macroscopic level to ensure sustainability and actuarial
fairness. \\



%related literature (the two) and differences from ours

%organization



%%%% Methodology and results
% To construct a forward utility, one seeks for a random field which satisfies the super-martingale property for any admissible investment strategy, and becomes a true martingale under the optimal strategy. The major challenge of a general construction of forward utility preferences emerges from their non-uniqueness, which is a consequence of a specified initial preference instead of a terminal one.  Indeed, by applying the It\^o-Wentzell formula,  it was shown in \cite{Musiela2010b} that a forward utility process is characterized by an ill-posed non-linear stochastic partial differential equation (SPDE) with an exogenously chosen volatility process. Nevertheless, solving a fully non-linear SPDE is highly non-trivial, which makes constructing a forward utility preference directly from the SPDE almost implausible. Alternatively, one can bypass the SPDE by considering a deterministic function of suitable stochastic input, whose dynamics shall be determined so that the (super-)martingale property is satisfied. One is then able to retrieve the solution to the original SPDE by identifying the corresponding volatility processes.  Notably in \cite{liang:bsde},  the constructions of forward utility preferences were directly done by the Markovian solutions of ergodic backward stochastic differential equations (BSDEs).   \\
%  This difficulty is amplified by the presence of a non-hedgeable risk in the salary herein.

%optimal investment and reinsurance problem of an insurance company

% Since an initial instead of a terminal utility is specified, forward utility preferences are often ill-posed, i.e., non-unique.  This nevertheless allows  flexibility  for one to choose their own appetite for risk.  

% \subsection{Problem and Contributions}
% In this paper, inspired by \cite{CAIRNS2006843:stochastic:lifestyle}, we consider a worker enrolled in a DC pension plan who wishes to maximize her fund value to salary ratio in order to maintain her living standard, where her salary is modeled by a geometric Brownian motion type process. Her preferences are measured by forward utilities on the fund value to salary ratio. In particular, we construct forward utilities under an initial exponential utility (CARA) and a power utility (CRRA). It is not surprising that the stochastic salary and its contribution to the fund would alter the form of the worker-specific stochastic inputs compared with the aforementioned works, since the investment strategy is no longer self-financing. To compensate the effect, it is natural to consider additive stochastic input to the fund value to salary ratio.  For power utility, the additive stochastic input which emerges as a floor level to the  fund value to salary ratio.   {\color{blue} This requirement thus resembles the constant proportion portfolio insurance (CPPI) strategy in managing portfolio insurance or pension funds \cite{CPPI:1992,CPPI:1989}.  The essential difference herein with \cite{dynamic:preference:pension} is the random nature of the floor level. Nevertheless, the set of admissible strategies can be characterized by  employing a simple transformation. For the exponential case, our construction is similar to the one in \cite{Musiela2008,zitkovi:2009}, where the stochastic input emerges as a dynamic risk of aversion which varies according to the conditions of the market and salary.}\\
%{\color{blue} From the construction in Proposition \ref{pp:exp}, we see that while the dynamic risk of aversion  always aligns with the non-hedgeable risk, its movement with respect to the market risk is subject to the employee's own appetite.}  } \\ 

% stochastic other than determinsitic 
% In both constructions, we observe that the additive stochastic input emerged in the utilities can indeed be interpreted as the fund value to salary ratio under an exogenously chosen investment strategy. The problem then amounts to finding the optimal investment strategy that performs the best relative to the exogenous baseline strategy. This echoes the study of fund management with relative performance criteria under forward utility preferences in usual financial markets \cite{anthropelos:relative,dosreisL:mean:field}, despite only a single agent is considered herein. In particular, we find that the optimal strategy is the composition of a myopic strategy and the exogenously chosen baseline strategy, {\color{blue}where the myopic part resembles  the optimal strategy under classical backward setting.} Workers then change their positions by varying the amount being put into the two components according to the relative performance of the baseline fund and worker's own fund. With an initial power utility, the optimal investment strategy is exactly a convex combination of the myopic strategy and the baseline strategy.  With an initial exponential utility, the utility penalizes for the deviation of the myopic strategy from the baseline strategy, and asymptotically, the optimal strategy can converge to or diverge from the baseline strategy according to worker's preference. The comparison with a baseline strategy also improves the original paradigm of using the fund value to salary ratio as a measure of satisfaction in \cite{CAIRNS2006843:stochastic:lifestyle}. In particular, without comparing with a baseline strategy, a worker's level of satisfaction will immediately increase due to a drop of salary, since the fund value to salary ratio will increase accordingly. However, this hypothetical rise will be mitigated when comparing with the increased baseline fund value to salary ratio due to the drop of salary. \\

%the amount invested in the myopic strategy is indifferent with the current fund value, while the amount in the baseline strategy increases in magnitude as the fund grows. For power utility, the composition emerges as a weighted average of the two components, where the weight on the baseline component is given by the ratio of the fund values between the baseline strategy and the optimal strategy. In other words, workers are putting more on the baseline strategy when it is well-performing, and vice versa. In particular, we observe from a scenario analysis that one tends to adhere to the myopic strategy when the equity prices are driven up by the market's risk, and stick to the baseline strategy otherwise. \\%This observation is further elaborated in Section \ref{sec:relation}, where we consider constructions of forward utility preferences over the relative performance.      \\

% In \cite{anthropelos:relative}, Nash equilibrium of two competing agents are considered, where each of them is concerned with the relative wealth ratio with the others; \cite{dosreisL:mean:field} considered a multi-agent mean field Nash equilibrium where the relative performance is measured in terms of the difference of individual wealth with the population average.

%%%% Detailed descriptions of the two papers. 
% {\color{red} 
% In the absence of contributions of income, \cite{dynamic:preference:pension} considered two types of pension fund management strategies, the constant proportion portfolio insurance (CPPI) strategy and the life-cycle strategy. A CPPI strategy maintains the fund value in a way that it has to exceed a  prescribed target, which thus provides a minimum guarantee to pensioners. Therein, a CARA forward utility and the associated CPPI optimal investment strategy are constructed, where the target is set to be a saving account which earns at the risk-free rate. On the other hand, a life-cycle strategy describes the behaviour where pensioners tend to invest almost entirely on the risky assets at early stages, and subsequently reduce the composition of the risky assets in the portfolio at later stages. By the time of retirement, the portfolio consists of entirely the riskless asset.  They showed that a life-cycle strategy is not optimal for any forward utilities. In addition, they constructed a SAHARA forward utility and showed that the associated optimal investment strategy exhibits life-cycle type feature. (Therein, they do not consider incomes and thus not fund value to salary ratio) \\

% \cite{hillairet:social} considered the optimal investment and insurance policy of a pay-as-you-go pension scheme in a social planner's perspective. Therein, the society consists of workers (who contribute their income to the social pension system) and pensioners (who receive a fixed amount of pension income). The social planner decides the optimal investment-pension pair for the society, which is required to exceed a 
% buffer fund (which they later show that it is $\hat{X}$, a fund with an exogenous policy for power utility), using a power forward utility. (so their dynamics is for the society as a whole has contribution, withdrawal (pension) and investment simultaneously, whereas we are considering a fundamental problem from the workers' perspective, and of course, they are considering the absolute fund value for the society, whereas we consider the ratio to the salary). \\

% }

%%%% organization
% \subsection{Organization}
This paper is organized as follows. Section \ref{sec:problem} formulates the optimal investment problem of the worker enrolled to the DC pension scheme during the accumulation phase, and revisits the forward preferences. Section \ref{sec:SPDE} illustrates the SPDE representation of her (translated) forward preferences and highlights their volatility processes. Sections \ref{sec:power} and \ref{sec:exp} construct her forward utility preferences and solve her optimal investment strategies, in the cases of CRRA power and CARA exponential initial utility functions respectively. Section \ref{sec:conclusion} concludes and discusses future directions.

% The paper is organized as follows. In Section \ref{sec:problem}, the DC pension fund model and the notion of a forward utility are introduced. In Sections  \ref{sec:power} and \ref{sec:exp} , we construct forward utilities  with the associated optimal investment strategies under initial  power and exponential utility,   respectively. Numerical scenario analysis is also included.  The paper is concluded in Section \ref{sec:conclusion}.

%In Section \ref{sec:relation}, we discuss a general relationship of the constructions with relative performance criteria.

\section{Problem Formulation and Preliminaries}
\label{sec:problem}

Let $(\Omega,\mathcal{F},\mathbb{P})$ be a probability space and ${\bf B^1}=\{{\bf B}^1_t = (B^{1,j}_t)_{j=1}^n\}_{t\geq 0}$, ${\bf B^2}=\{{\bf B}^2_t=(B^{2,j}_t)_{j=1}^m\}_{t\geq 0}$ be independent Brownian motions of dimensions $n$ and $m$, respectively.  Let $\mathbb{F}=\{\mathcal{F}_t\}_{t\geq 0}$ be the filtration generated by the Brownian motions $({\bf B}^1,{\bf B}^2)$ satisfying the usual conditions, and $\mathcal{P}_n(\mathbb{F})$ be the collection of all $\mathbb{F}$-progressively measurable processes taking values in $\mathbb{R}^n$. We also define
\begin{equation*}
\mathcal{L}^2_n := \Big\{ \pi=\{\pi_t = (\pi_t^i)_{i=1}^{n} \}_{t\geq 0} \in \mathcal{P}_n(\mathbb{F}):      \mathbb{E}\left[ \int_0^t \|\pi_s\|^2 ds   \right]<\infty, \text{ for all } t>0  \Big\},
\end{equation*}
where $\|\cdot\|$ is the Euclidean norm, and
\begin{equation*}
\mathcal{L}^2_{n,\text{BMO}} := \left\{ \pi\in \mathcal{P}_n(\mathbb{F}):\mathop{\text{esssup}}_{\tau\in\mathcal{T}[0,t]} \mathbb{E}\left[ \int_\tau^t \|\pi_s\|^2 ds \Big\vert \mathcal{F}_\tau   \right]<\infty, \text{ for all } t>0  \right\},
\end{equation*}
\begin{equation*}
\mathcal{T}[0,t]:=\left\{\tau\in \mathcal{P}_1(\mathbb{F}):\tau\text{ is an $\mathbb{F}$-stopping time in $\left[0,t\right]$}\right\},\text{ for all }t>0.
\end{equation*}


%    \begin{equation*}
%        \mathcal{A}_t := \{ \{\pi_s\}_{s\in[0,t]} \in L^2_{\text{BMO}}[0,t] : X^\pi_s \in \mathcal{D}_s, \ s \in[0,t]\},
%    \end{equation*}
%where $\{\mathcal{D}_t\}_{t\geq 0}$ is $\mathbb{F}$-progressively measurable with the property that, for each $t\geq 0$ and $\omega\in\Omega$,  $\mathcal{D}_t(\omega)\subset \mathbb{R}$ is a domain. We also denote the Euclidean norm of a vector by  $\|\cdot\|$. 

\subsection{Financial Market, Salary, and Pension Fund}
We consider a financial market with a riskless asset $S^0=\{S_t^0\}_{t\geq0}$, which earns a risk-free interest rate $r\in\mathbb{R}$, i.e., $S^0_t = S^0_0 e^{rt}$, and with $n$ risky assets ${\bf S} = \{{\bf S}_t = (S^i_t)_{i=1}^{n}\}_{t\geq 0}$, whose dynamics is given by, for any $i=1,2,\dots,n$, and $t\geq 0$,
\begin{equation*}
dS^i_t = S^i_t\left( (r  + \mu^i_t ) dt + \sum_{j=1}^n \sigma^{ij}_t dB^{1,j}_t\right). 
\end{equation*}
Herein, for any $i,j=1,2,\dots,n$, $\mu^i_{\cdot} : [0,\infty) \to \mathbb{R}$ represents the risk premium (instead of the mean in conventional notations) of the $i$-th risky asset, and $\sigma^{ij}_{\cdot} : [0,\infty) \to \mathbb{R}_+$ is the volatility of the $i$-th risky asset with respect to the $j$-th Brownian motion of ${\bf B^1}$. For simplicity, both are assumed to be deterministic functions.\\

During the accumulation period of a pension fund, a worker receives salary, with a process $Y=\{Y_t\}_{t\geq 0}$ whose dynamics is given by, for any $t\geq 0$,
\begin{equation*}
dY_t = Y_t\left( (r+ \mu^Y_t)dt + (\sigma^{Y,1}_t)^\top d{\bf B}^1_t + (\sigma^{Y,2}_t)^\top d{\bf B}^2_t \right),\ Y_0=y>0,
\end{equation*}
where  $\mu^Y_{\cdot} :[0,\infty) \to \mathbb{R}$, $\sigma^{Y,1}_{\cdot} : [0,\infty) \to \mathbb{R}^{n}_+$, and $\sigma^{Y,2}_{\cdot} : [0,\infty) \to \mathbb{R}^{m}_+$ are, also for simplicity, assumed to be deterministic functions, which represent, respectively, the risk premium of the salary, and the volatility of the salary with respect to the Brownian motions ${\bf B}^1$ and ${\bf B}^2$. When $\sigma_{\cdot}^{Y,2}\not\equiv{\bf 0}$, the worker's salary is subject to risk(s) which could not be hedged using the risky assets, whence inducing market incompleteness. Despite its simplicity, this diffusion model serves as a reasonable approximation of a more realistic compound jump process for her salary dynamics. Such an approximation has also been applied in other actuarial contexts, such as in ruin theory; see, for instance, \cite{hanspeter1994diffusion,LUO2011123,COHEN2020333} and the references therein.\\
% is also ubiquitous and crucial in other actuarial models, for example, diffusion approximations of Cram\'{e}r-Lundberg models  in ruin theory (see  ). }  \\ % Note that the path Y serve as an approximation of a realistic compound jump process, such approximation has also been playing a crucial role in ruin theroy  diffusion approximation ; Cramer Lundberg 

The worker is enrolled in a DC pension plan. Let $W=\{W_t\}_{t\geq0}$ be the process of her pension fund value. During the accumulation period and at any time $t\geq 0$, on one hand, the worker contributes part of her salary into the fund with an instantaneous rate $p_tY_t$, for some deterministic proportion of contribution $p_{\cdot}:\left[0,\infty\right)\rightarrow\mathbb{R}_+$; on the other hand, she self-manages her pension fund by actively investing in the financial market, in which $\pi_t^{i}$, for $i=1,2,\dots,n$, denotes the proportion of the fund investing into the $i$-th risky asset. 
%At time $t$, the employee has to  contribute to the fund at an instantaneous rate $p_tY_t$, where the proportion of contribution $p_t$ is predefined. On the other hand, he is able to choose between the risk-free and risky assets to invest in. Let $\pi_t = (\pi^1_t,\dots,\pi^n_t)$ be the proportion of the fund that is invested into the risky assets at time $t$. 
The dynamics of $W$ is thus given by, for any $t\geq 0$,
\begin{equation*}
dW_t = p_t Y_t dt  + W_t\left(   (r + \pi_t^\top \mu_t )dt + \pi^\top_t\Sigma_t d{\bf B}^1_t \right), \ W_0=w>0,
\end{equation*}
where $\pi_t = (\pi_t^i)_{i=1}^{n}$, $\mu_t = (\mu^i_t)_{i=1}^{n}$, and $\Sigma_t = ( \sigma^{ij}_t)_{i,j=1}^{n}$. The following assumptions are imposed in the remaining of this paper.
\begin{assumption}
\label{ass}
\quad
\begin{enumerate}
\item[(i)] The volatility matrix $\Sigma_t$ has a full rank for all $t\geq 0$;
\item[(ii)] The coefficients $\mu_{\cdot},\Sigma_{\cdot},\Sigma_{\cdot}^{-1}, \mu_{\cdot}^Y,\sigma_{\cdot}^{Y,1},\sigma_{\cdot}^{Y,2}$, the proportion of contribution $p_{\cdot}$, and the market price of risk $\lambda_{\cdot} := \Sigma_{\cdot}^{-1}\mu_{\cdot} $ are bounded in $t\geq 0$.
\end{enumerate}
\end{assumption}

In the context of retirement planning, the worker should perceive her living standard by the relative pension fund value to her pre-retirement salary, instead of by the absolute fund value; see also \cite{CAIRNS2006843:stochastic:lifestyle}, particularly the life-cycle and habit-formation discussions therein. To this end, define the pension fund value to salary ratio by $X=\{X_t := W_t/Y_t\}_{t\geq 0}$, which satisfies that, for any $t\geq 0$,
\begin{equation}
\begin{aligned}
dX_t =&\;p_tdt + X_t\left( \left( \pi^\top_t\Sigma_t (\lambda_t-\sigma^{Y,1}_t) - \mu^Y_t + \|\sigma^{Y,1}_t\|^2 + \|\sigma^{Y,2}_t\|^2 \right) dt  \right.\\&\quad\quad\quad\quad\quad\left.+ (\pi_t^\top\Sigma_t-(\sigma^{Y,1}_t)^\top) d{\bf B}^1_t - (\sigma^{Y,2}_t)^\top d{\bf B}^2_t \right),\\X_0=&\;x_0:= \frac{w}{y}>0.
\end{aligned}
\label{eq:X}
\end{equation}
% To quantify the performance of the pension fund, we consider the fund to salary ratio, $X=\{X_t := W_t/Y_t\}_{t\geq 0}$, which measures the ability of employees to maintain their living standard at the time of retirement \cite{CAIRNS2006843:stochastic:lifestyle}. This perspective echoes the feature of {\color{red}consumption smoothing on life-cycle hypothesis of savings} \cite{lifecycle}. By a simple application of It\^o's lemma, we see that $X_t$ satisfies
%     \begin{align}
%     \label{eq:X}
%         dX_t &= Y_t^{-1}dW_t -  W_t Y_t^{-2} dY_t  + W_t Y_t^{-3} d\langle Y\rangle_t - Y_t^{-2} d\langle W,Y\rangle_t  \nonumber \\ 
%         &=\left(p_t + X_t \left( \pi^\top_t\Sigma_t (\lambda_t-\sigma^{Y,1}_t) - \mu^Y_t + \|\sigma^{Y,1}_t\|^2 + \|\sigma^{Y,2}_t\|^2 \right) \right)dt \nonumber  + \\ & \hspace{2cm} X_t\left( (\pi_t^\top\Sigma_t-(\sigma^{Y,1}_t)^\top) d{\bf B}^1_t - (\sigma^{Y,2}_t)^\top d{\bf B}^2_t \right), \ X_0 =x:= \frac{w}{y}>0.
%     \end{align}
Whenever necessary, we shall write $X^\pi=\{X^{\pi}_t\}_{t\geq 0}$ to emphasize its dependence on an investment strategy $\pi=\{\pi_t \}_{t\geq 0}$.\\

% In the sequel, we shall also write $X^\pi=\{X^{\pi}_t\}_{t\geq 0}$ to emphasize the solution to \eqref{eq:X} under the investment strategy $\pi=\{\pi_t\}_{t\geq 0}$ whenever necessary.

Let $\mathcal{A}$ be the admissible set of investment strategies in $\mathcal{P}_n(\mathbb{F})$, which shall minimally include an integrability condition, as well as a condition to ensure a preference of the worker being well-defined. For the former one, it shall be defined respectively in Sections \ref{sec:power_admissibility} and \ref{sec:admis_exponential}. For the latter one, in general, an admissible investment strategy $\pi$ should satisfy that
\begin{equation*}
X^\pi_t(\omega) \in \mathcal{D}_t(\omega),    \text{ for a.a. }(t,\omega)\in [0,\infty) \times\Omega,
% \label{eq:A}
\end{equation*}
where the stochastic domain $\mathcal{D} = \{\mathcal{D}_t\}_{t\geq 0}$ is an $\mathbb{F}$-progressively measurable set-valued process; in particular, the $\mathcal{F}_t$-measurable $\mathcal{D}_t\left(\cdot\right)$ is assumed throughout the paper to be an open random interval\footnote{Fix a time $t\geq 0$. The map $\omega \mapsto \mathcal{D}_t(\omega)$ is an  $\mathcal{F}_t$-measurable random open set in $\mathbb{R}$ if, for any compact set $F \subset \real$, it holds that  $  \{\omega \in\Omega : \left(\mathbb{R}\backslash \mathcal{D}_t(\omega)\right)\cap F \neq \varnothing\}\in \mathcal{F}_t$. Moreover, $\mathcal{D}_t\left(\cdot\right)$ is an open random interval, if $\mathcal{D}_t\left(\cdot\right)$ is a random open set and $\mathcal{D}_t(\omega)$ is an interval for almost all $\omega\in \Omega$. For details of the theory of random sets, see, such as, \cite{random:set}.} of the form $(\inf \mathcal{D}_t(\cdot),\infty)\subseteq\mathbb{R}$, for any time $t\geq 0$,  where $\{ \inf \mathcal{D}_t \}_{t\geq 0} \in \mathcal{P}_1(\mathbb{F})$.

% {\color{red} Let $\mathcal{A}$ be the admissible set of investment strategies, where the exact specification shall be stated separately in the power and the exponential case. Herein, it is required that }
%     \begin{equation}

%         \{ {\color{red}\pi  } : {\color{red}   \ X^\pi_t(\omega) \in \mathcal{D}_t(\omega)    \text{ for a.e. }(t,\omega)\in [0,\infty) \times\Omega  \}}\subset  \mathcal{A},
%     \end{equation}
% where $\mathcal{D} = \{\mathcal{D}_t\}_{t\geq 0}$ is an $\mathbb{F}$-progressively measurable set-valued process, where $\mathcal{D}_t$ is an $\mathcal{F}_t$-measurable {\color{red} open} random interval for each $t\geq0$. {\color{red}To be precise, the map $\omega \mapsto \mathcal{D}_t(\omega)$ is an  $\mathcal{F}_t$-measurable random open set if for any compact set $F \subset \real$, it holds that  $  \{\omega \in\Omega : \left(\mathbb{R}\backslash \mathcal{D}_t(\omega)\right)\cap F \neq \varnothing\}\in \mathcal{F}_t$; and $\mathcal{D}_t$ is an open random interval if $\mathcal{D}_t$ is a random open set and $\mathcal{D}_t(\omega)$ is an interval for almost all $\omega\in \Omega$. We refer the readers to \cite{random:set} for a detailed introduction to the theory of random sets.} As we shall see in Section \ref{sec:power}, the stochastic domain $\mathcal{D}$ emerges from the dynamic floor level for $X^\pi$, which ensures the utility process to be well-defined. 

%%%%%%%%%%%%%%%%%%%%%%%%%%%%%%%%%%%%%%%%%%%%%%%%%%%%%%%%%%%%%%%%%%%%%%%%%%%%%%%%%%%%%%%%%%%%%%%%%%%%%%%%%%%%%%%%%%%%%%%%%%%%%%%%%%%%%%%%%%


\subsection{Forward Preferences}
The worker aims to solve an optimal investment strategy, using her forward preference on the fund value to salary ratio $X$ as the optimality criterion. The notion of forward preference is essentially a stochastic process satisfying a martingale (resp. super-martingale) property when it evaluates an optimal (resp. a sub-optimal) fund value to salary ratio. The preference models the worker's future utilities on the ratio moving forward in the accumulation period, which is subject to both financial market and salary performances and thus explains its nature of stochasticity. The following definition recalls such notion, with a slight modification from the original definition in \cite{Musiela2007} to incorporate the restriction for the fund value to salary ratio lying in $\mathcal{D}$.

% The objective of this paper is to optimize the strategy of asset allocations based on forward utility preferences on $X$. We first review the notion of a forward utility preference, which is essentially a stochastic process satisfying a super-martingale/martingale property when evaluated at $X$. To be precise, we have the following definition. 

% should we also omit x \not\in D?
% should admissible set be included in definition?
\begin{definition}
\label{def:forward}
Let $\mathcal{D}=\{\mathcal{D}_t\}_{t\geq 0}$ be an open set-valued process  with $\mathcal{D}_t=(\inf \mathcal{D}_t,\infty)$ for any $t\geq 0$, where $\{ \inf \mathcal{D}_t \}_{t\geq 0} \in \mathcal{P}_1(\mathbb{F})$. A random field $U=\{U(x,t;\omega) :  \omega\in\Omega, \ t \geq 0,\ x\in  \mathbb{R} \}$ is called a forward preference on the fund value to salary ratio defined in  $\mathcal{D}$, if  it satisfies all of the following properties:
\begin{enumerate}[label=(\roman*)]
\item for each $x\in\mathbb{R}$, $U(x,\cdot;\cdot)$ is $\mathbb{F}$-adapted;
\item for each $t \geq 0$ and $\omega\in\Omega$, $x \in \mathcal{D}_t(\omega) \mapsto U(x,t;\omega)$ is strictly increasing and strictly concave, while, for $x\leq   \inf \mathcal{D}_t(\omega)$, $U(x,t;\omega) = -\infty$.
%\begin{equation*}
%U(x,t;\omega) = \begin{cases}
%U(\sup \mathcal{D}_t(\omega),t;\omega) &\text{if } x \geq \sup \mathcal{D}_t(\omega)\\
%-\infty &\text{if } x \leq \inf \mathcal{D}_t(\omega)
%\end{cases};
%\end{equation*}
\item for any $\pi \in \mathcal{A}$, and $0\leq s\leq t$,
\begin{equation*}
U(X_s^\pi,s) \geq \E[U(X_t^\pi,t) \vert\mathcal{F}_s], \  \mathbb{P}\text{-a.s.},
\end{equation*}
and there exists an optimal investment strategy $\pi^* \in \mathcal{A}$ such that, for any $0\leq s\leq t$, 
\begin{equation*}
U(X_s^{\pi^*},s) = \E[U(X_t^{\pi^*},t) |\mathcal{F}_s], \  \mathbb{P}\text{-a.s.}.
\end{equation*}
% The strategy $\pi^*$ is then the optimal investment strategy for the utility process $U$.
\end{enumerate}
\end{definition} 
% Definition \ref{def:forward} is a slight modification to the original definition of forward utility preferences in \cite{Musiela2007}. The  distinction herein is the restriction to $\mathcal{D}$.

\section{Volatility Processes of Forward Preferences} 
\label{sec:SPDE}
The stochastic nature of the worker's forward preference on her fund value to salary ratio $X$ can be revealed by a SPDE representation. Let $U$ be a random field satisfying (i) and (ii) in Definition \ref{def:forward} with stochastic domain $\mathcal{D}$. Consider the following two cases:
    \begin{enumerate}
        \item[(i)]  if $\inf \mathcal{D}_0 > -\infty$, $\mathbb{P}$-a.s., let the process $Z:=\{Z_t:= \inf \mathcal{D}_t\}_{t\geq 0}$ which admits the It\^o's diffusion form: for any $t\geq 0$,
         \begin{equation}
         \label{eq:Z:SPDE}
        dZ_t = \nu_t dt + (\kappa^1_t)^\top d{\bf B}^1_t + (\kappa^2_t)^\top d{\bf B}^2_t.
    \end{equation}
    where $\nu=\{\nu_t \}_{t\geq 0}\in \mathcal{P}_1(\mathbb{F})$, $\kappa^1=\{\kappa^1_t \}_{t\geq 0}\in \mathcal{P}_n(\mathbb{F})$ and $\kappa^2=\{\kappa^2_t \}_{t\geq 0}\in \mathcal{P}_m(\mathbb{F})$, and let $\mathcal{\tilde{D}} := \mathbb{R}_{++}$, which is $\mathbb{R}_+\backslash\{0\}$.
    \item[(ii)] if $\inf \mathcal{D}_0 = -\infty$, $\mathbb{P}$-a.s., let $\mathcal{D} \equiv \mathbb{R}$ and the process $Z:\equiv 0$ (which is equivalent to $Z$ satisfying \eqref{eq:Z:SPDE} with $Z_0=0$ and coefficients $\nu$, $\kappa^1$ and $\kappa^2$ all being identical to zero), and let $\mathcal{\tilde{D}} := \mathbb{R}$.
    \end{enumerate}
To unify the subsequent discussions, define the translated random field $\tilde{U}=\{\tilde{U}(\tilde{x},t;\omega) : \omega \in\Omega, \ t\geq 0, \ \tilde{x} \in \mathcal{\tilde{D}}\}$ by $\tilde{U}(\tilde{x},t):= U(\tilde{x}+Z_t,t)$, which obviously satisfies (i) and (ii) in an equivalent form of Definition \ref{def:forward}. Notice that  $U$ is a forward preference on $X$ with stochastic domain $\mathcal{D}$ if and only if $\tilde{U}$ is a forward preference on $\tilde{X}:=\{\tilde{X}_t:=X_t-Z_t\}_{t\geq 0}$ with deterministic domain $\mathcal{\tilde{D}}$.  Assume that $\tilde{U}$ takes the following It\^o's diffusion form: for any $\tilde{x}\in\tilde{\mathcal{D}}$ and $t\geq 0$,
\begin{equation*}
d\tilde{U}(\tilde{x},t) =  b(\tilde{x},t) dt + a_1(\tilde{x},t)^\top d{\bf B}^1_t + a_2(\tilde{x},t)^\top d{\bf B}^2_t, \ 
\tilde{U}(\tilde{x},0) =  u_0(\tilde{x}+Z_0), 
\label{eq:U}
\end{equation*}
where, for each $\tilde{x}\in\tilde{\mathcal{D}}$, $b\left(\tilde{x},\cdot;\cdot\right)\in\mathcal{P}_1(\mathbb{F})$, $a_1\left(\tilde{x},\cdot;\cdot\right)\in\mathcal{P}_n(\mathbb{F})$, and $a_2\left(\tilde{x},\cdot;\cdot\right)\in\mathcal{P}_m(\mathbb{F})$, and where  $u_0\left(\cdot\right)$ is a deterministic function mapping from $\mathcal{D}_0$ to $\mathbb{R}$. Suppose further that, for each $t\geq 0$ and $\omega\in\Omega$, $a_1\left(\cdot,t;\omega\right)$ and $a_2\left(\cdot,t;\omega\right)$ are differentiable in $\mathcal{\tilde{D}}$. \\

At the current time $0$, the worker's utility on the ratio is given by the deterministic $u_0\left(\cdot\right)$. The random fields $a_1=\left\{a_1\left(\tilde{x},t;\omega\right)\right\}_{\omega\in\Omega,\tilde{x}\in\mathcal{\tilde{D}},t\geq 0}$ and $a_2=\left\{a_2\left(\tilde{x},t;\omega\right)\right\}_{\omega\in\Omega,\tilde{x}\in\mathcal{\tilde{D}},t\geq 0}$ represent the volatility processes of the random field $\tilde{U}$ with respect to the hedgeable component of the financial market and the non-hedgeable component of the labour market respectively, which are chosen by the worker at the current time $0$. The random field $b=\left\{b\left(\tilde{x},t;\omega\right)\right\}_{\omega\in\Omega,\tilde{x}\in\mathcal{\tilde{D}},t\geq 0}$, which is the drift process of the random field $\tilde{U}$, is then determined to ensure that the equivalent form of Definition \ref{def:forward}, particularly (iii), is satisfied moving forward, and hence $\tilde{U}$ is a forward preference. With proper transformation, $U$ is then a forward preference.\\

% for each $x\in\real$, $b(x,\cdot),a_1(x,\cdot),a_2(x,.\cdot) \in \mathcal{P}(\mathbb{F})$ are  respectively $\mathbb{R},\mathbb{R}^n$ and $\mathbb{R}^m$-valued processes. In addition, for each $t\geq 0$ and $\omega\in\Omega$,  $a_1(\cdot,t;\omega)$ and $a_2(\cdot,t;\omega)$ are differentiable in $\mathcal{D}_t(\omega)$.

% The volatility processes $a_1$ and $a_2$ are to be chosen by the worker, which then determines $b$ via a SPDE so that $U(x,t)$ shall satisfy Definition \ref{def:forward}.
For any $\pi\in\mathcal{A}$, by the It\^o-Wentzell formula, for any $t\geq 0$,
\begin{equation}
\begin{aligned}
&\;d\tilde{U}(\tilde{X}_t,t)\\=&\;b(\tilde{X}_t,t) dt + a_1(\tilde{X}_t,t)^\top d{\bf B}^1_t + a_2(\tilde{X}_t,t)^\top d{\bf B}^2_t + \tilde{U}_{\tilde{x}}(\tilde{X}_t,t)d\tilde{X}_t  \\ & \ + \frac{1}{2} \tilde{U}_{\tilde{x}\tilde{x}}(\tilde{X}_t,t) d\langle \tilde{X}_{\cdot}\rangle_t +   \left( X_t(\pi_t^\top\Sigma_t-(\sigma^{Y,1}_t)^\top) - (\kappa^1_t)^\top\right) \nabla_{\tilde{x}}a_1(\tilde{X}_t,t)dt  \\ & \ - \left(X_t(\sigma^{Y,2}_t)^\top + (\kappa_t^2)^\top  \right)\nabla_{\tilde{x}}a_2(\tilde{X}_t,t)  dt\\=&\;\Big( b(\tilde{X}_t,t) + \tilde{U}_{\tilde{x}}(\tilde{X}_t,t)\left(p_t -\nu_t + X_t\left( \pi_t^\top\Sigma_t(\lambda_t-\sigma^{Y,1}_t) - \mu^Y_t +  \|\sigma^{Y,1}_t\|^2 + \|\sigma^{Y,2}_t\|^2 \right)\right)\\&\;\quad+ \frac{\tilde{U}_{\tilde{x}\tilde{x}}(\tilde{X}_t,t)}{2} \left(\| X_t(\Sigma^\top_t\pi_t-\sigma^{Y,1}_t)-\kappa_t^1\|^2 + \|X_t\sigma^{Y,2}_t + \kappa_t^2\|^2\right)  \\
& \;\quad +\left( X_t(\pi_t^\top\Sigma_t-(\sigma^{Y,1}_t)^\top) - (\kappa^1_t)^\top\right) \nabla_{\tilde{x}}a_1(\tilde{X}_t,t)  - \left(X_t(\sigma^{Y,2}_t)^\top + (\kappa_t^2)^\top  \right)\nabla_{\tilde{x}}a_2(\tilde{X}_t,t) \Big)  dt\\&\; + (a_1(\tilde{X}_t,t)^\top + \tilde{U}_{\tilde{x}}(\tilde{X}_t,t)X_t(\pi^\top_t \Sigma_t-(\sigma^{Y,1}_t)^\top ) -(\kappa_t^1)^\top  ) d{\bf B}^1_t\\&\;  +( a_2(\tilde{X}_t,t)^\top -\tilde{U}_{\tilde{x}}(\tilde{X}_t,t)X_t (\sigma^{Y,2}_t)^\top - (\kappa^2_t)^\top  ) d{\bf B}^2_t.
\end{aligned}
\label{eq:SPDE1}
\end{equation}


% \begin{align}
% & \ \ \ dU(X_t,t) \nonumber \\
% &= b(X_t,t) dt + a_1^\top(X_t,t) d{\bf B}^1_t + a_2^\top(X_t,t)d{\bf B}^2_t + U_x(X_t,t)dX_t+ \nonumber\\ &\hspace{0.5cm} + \frac{1}{2} U_{xx}(X_t,t) d\langle X\rangle_t +  X_t ( (\pi_t^\top\Sigma_t-(\sigma^{Y,1}_t)^\top)\nabla_{\tilde{x}}a_1(X_t,t) - (\sigma^{Y,2}_t)^\top\nabla_{\tilde{x}}a_2(X_t,t) )  dt  \nonumber\\
% &= \Big( b(X_t,t) + U_x(X_t,t)\left(p_t + X_t\left( \pi_t^\top\Sigma_t(\lambda_t-\sigma^{Y,1}_t) - \mu^Y_t +  \|\sigma^{Y,1}_t\|^2 + \|\sigma^{Y,2}_t\|^2 \right)\right) \nonumber \\ &\hspace{0.5cm} + \frac{U_{xx}(X_t,t)X_t^2}{2}\left( \|\Sigma^\top_t\pi_t-\sigma^{Y,1}_t\|^2 + \|\sigma^{Y,2}_t\|^2 \right) + X_t \Big( (\pi_t^\top\Sigma_t-(\sigma^{Y,1}_t)^\top)\nabla_{\tilde{x}}a_1(X_t,t) - \nonumber \\ &\hspace{0.5cm} (\sigma^{Y,2}_t)^\top\nabla_{\tilde{x}}a_2(X_t,t) \Big)   \Big)dt + (a_1^\top(X_t,t) + U_x(X_t,t)X_t(\pi^\top_t \Sigma_t-(\sigma^{Y,1}_t)^\top )) d{\bf B}^1_t+ \nonumber \\ &\hspace{0.5cm}  ( a_2^\top(X_t,t) -U_x(X_t,t)X_t (\sigma^{Y,2}_t)^\top  ) d{\bf B}^2_t.
% \end{align}

Notice that the drift term of \eqref{eq:SPDE1} can be written as, for any $t\geq 0$,
\begin{align*}
%&  \frac{  U_{xx}(X_t,t)X_t^2 }{2}\|\Sigma^\top_t\pi_t \|^2 + \pi_t^\top\Sigma_t\Big( \nabla_{\tilde{x}}a_1(X_t,t)X_t -\sigma^{Y,1}_tU_{xx}(X_t,t)X_t^2  \\ &\hspace{1cm} +(\lambda_t-\sigma^{Y,1}_t)U_x(X_t,t)X_t  \Big) +  A(X_t,t) \\
 & \frac{  \tilde{U}_{\tilde{x}\tilde{x}}(\tilde{X}_t,t) X_t^2}{2} \left\| \Sigma^\top_t \pi_t - \sigma^{Y,1}_t+ \frac{\nabla_{\tilde{x}} a_1(\tilde{X}_t,t)+ (\lambda_t-\sigma^{Y,1}_t)\tilde{U}_{\tilde{x}}(\tilde{X}_t,t) -\kappa^1_t \tilde{U}_{\tilde{x}\tilde{x}}(\tilde{X}_t,t) }{ X_t \tilde{U}_{\tilde{x}\tilde{x}}(\tilde{X}_t,t) } \right\|^2  + A_t,
\end{align*}
where, 
\begin{align*}
A_t :=&\; b(\tilde{X}_t,t) +  \frac{(\|X_t\sigma^{Y,1}_t + \kappa^1_t\|^2+\|X_t\sigma^{Y,2}_t+\kappa_t^2\|^2)\tilde{U}_{\tilde{x}\tilde{x}}(\tilde{X}_t,t)}{2} \\
& - \left( X_t(\sigma^{Y,1}_t)^\top + (\kappa^1_t)^\top \right) \nabla_{\tilde{x}}a_1(\tilde{X}_t,t) - \left(X_t(\sigma^{Y,2}_t)^\top + (\kappa^2_t)^\top \right)\nabla_{\tilde{x}}a_2(\tilde{X}_t,t)   \\ & +\tilde{U}_{\tilde{x}}(\tilde{X}_t,t) (p_t-\nu_t + X_t( \|\sigma^{Y,1}_t\|^2+\|\sigma^{Y,2}_t\|^2 - \mu^Y_t))   \\ & - \frac{  \tilde{U}_{\tilde{x}\tilde{x}}(\tilde{X}_t,t) X_t^2}{2} \left\|   \sigma^{Y,1}_t- \frac{\nabla_{\tilde{x}} a_1(\tilde{X}_t,t)+ (\lambda_t-\sigma^{Y,1}_t)\tilde{U}_{\tilde{x}}(\tilde{X}_t,t) -\kappa^1_t \tilde{U}_{\tilde{x}\tilde{x}}(\tilde{X}_t,t) }{ X_t \tilde{U}_{\tilde{x}\tilde{x}}(\tilde{X}_t,t) } \right\|^2.\nonumber
\end{align*}
Now, let the drift process of the random field $\tilde{U}$ by, for any $\tilde{x}\in\mathcal{\tilde{D}}$ and $t\geq 0$,
\begin{align*}
b(\tilde{x},t):=&\;-  \frac{(\|(\tilde{x}+Z_t)\sigma^{Y,1}_t + \kappa^1_t\|^2+\|(\tilde{x}+Z_t)\sigma^{Y,2}_t+\kappa_t^2\|^2)\tilde{U}_{\tilde{x}\tilde{x}}(\tilde{x},t)}{2}   \\  &\;+
\left( (\tilde{x}+Z_t)(\sigma^{Y,1}_t)^\top + (\kappa^1_t)^\top \right) \nabla_{\tilde{x}}a_1(\tilde{x},t) + \left((\tilde{x}+Z_t)(\sigma^{Y,2}_t)^\top + (\kappa^2_t)^\top \right)\nabla_{\tilde{x}}a_2(\tilde{x},t) \\ &\ 
-  \tilde{U}_{\tilde{x}}(\tilde{x},t) (p_t-\nu_t + (\tilde{x}+Z_t)( \|\sigma^{Y,1}_t\|^2+\|\sigma^{Y,2}_t\|^2 - \mu^Y_t))   \\ &\; +\frac{  \tilde{U}_{\tilde{x}\tilde{x}}(\tilde{x},t)(\tilde{x}+Z_t)^2}{2} \left\|   \sigma^{Y,1}_t- \frac{\nabla_{\tilde{x}} a_1(\tilde{x},t)+ (\lambda_t-\sigma^{Y,1}_t)\tilde{U}_{\tilde{x}}(\tilde{x},t) -\kappa^1_t \tilde{U}_{\tilde{x}\tilde{x}}(\tilde{x},t) }{ (\tilde{x}+Z_t) \tilde{U}_{\tilde{x}\tilde{x}}(\tilde{x},t) } \right\|^2\\
=&\;- \frac{ \|(\tilde{x}+Z_t)\sigma^{Y,2}_t+\kappa_t^2\|^2\tilde{U}_{\tilde{x}\tilde{x}}(\tilde{x},t)}{2}    + \left((\tilde{x}+Z_t)\sigma^{Y,2}_t  + \kappa_t^2 \right)^\top\nabla_{\tilde{x}}a_2(\tilde{x},t) \\ &\ 
-  \tilde{U}_{\tilde{x}}(\tilde{x},t) \left(p_t-\nu_t + (\lambda_t-\sigma^{Y,1}_t)^\top \kappa^1_t+ (\tilde{x}+Z_t)( \lambda_t^\top\sigma^{Y,1}_t+\|\sigma^{Y,2}_t\|^2 - \mu^Y_t)\right)   \\ &\; +\frac{ \left\|     \nabla_{\tilde{x}} a_1(\tilde{x},t)+ (\lambda_t-\sigma^{Y,1}_t)\tilde{U}_{\tilde{x}}(\tilde{x},t)    \right\|^2 }{2 \tilde{U}_{\tilde{x}\tilde{x}}(\tilde{x},t)}  ,
\end{align*}
which leads to $A_\cdot\equiv 0$, and thus the drift term of \eqref{eq:SPDE1} is simply, for any $t\geq 0$,
\begin{equation*}
\frac{  \tilde{U}_{\tilde{x}\tilde{x}}(\tilde{X}_t,t) X_t^2}{2} \left\| \Sigma^\top_t \pi_t - \sigma^{Y,1}_t+ \frac{\nabla_{\tilde{x}} a_1(\tilde{X}_t,t)+ (\lambda_t-\sigma^{Y,1}_t)\tilde{U}_{\tilde{x}}(\tilde{X}_t,t) -\kappa^1_t \tilde{U}_{\tilde{x}\tilde{x}}(\tilde{X}_t,t) }{ X_t \tilde{U}_{\tilde{x}\tilde{x}}(\tilde{X}_t,t) } \right\|^2.
\end{equation*}
% By setting $A(x,t)=0$, i.e., 
% \begin{align}
% \label{eq:drift_U}
% b(x,t) &:=   x((\sigma^{Y,1}_t)^\top\nabla_{\tilde{x}}a_1(x,t) +(\sigma^{Y,2}_t)^\top\nabla_{\tilde{x}}a_2(x,t))  
% - U_x(x,t) \big(p_t + x( \|\sigma^{Y,1}_t\|^2+\|\sigma^{Y,2}_t\|^2 - \nonumber \\ &\hspace{1cm} \mu^Y_t)\big)  
% +  \frac{ U_{xx}(x,t) x^2}{2} \left\| \sigma^{Y,1}_t- \frac{\nabla_{\tilde{x}} a_1(x,t)+ (\lambda_t-\sigma^{Y,1}_t)U_x(x,t)}{ x U_{xx}(x,t) }   \right\|^2 - \nonumber  \\ &\hspace{1cm} \frac{(\|\sigma^{Y,1}_t\|^2+\|\sigma^{Y,2}_t\|^2)U_{xx}(x,t)x^2}{2} ,
% \end{align}
% the drift term of \eqref{eq:SPDE1} is reduced to 
% \begin{equation*}
% \frac{  U_{xx}(X_t,t) X_t^2}{2} \left\| \Sigma^\top_t \pi_t - \sigma^{Y,1}_t+ \frac{\nabla_{\tilde{x}} a_1(X_t,t)+ (\lambda_t-\sigma^{Y,1}_t)U_x(X_t,t)}{ X_t U_{xx}(X_t,t) } \right\|^2 .
% \end{equation*}
By the strict concavity of the random field $\tilde{U}$ in $\tilde{x}$, the process $\{\tilde{U}(\tilde{X}_t,t)\}_{t\geq 0}$ is indeed an $\mathbb{F}$-super-martingale under some integrability conditions. Furthermore, with, for any $t\geq 0$,
\begin{equation}
\begin{aligned}
\pi_t^* :=&\; (\Sigma^\top_t)^{-1}\left(\sigma^{Y,1}_t-\frac{\nabla_{\tilde{x}} a_1(\tilde{X}_t,t)+ (\lambda_t-\sigma^{Y,1}_t)\tilde{U}_{\tilde{x}}(\tilde{X}_t,t) -\kappa^1_t \tilde{U}_{\tilde{x}\tilde{x}}(\tilde{X}_t,t) }{ X_t \tilde{U}_{\tilde{x}\tilde{x}}(\tilde{X}_t,t) }\right)
\\=&\;  (\Sigma^\top_t)^{-1} \left(\sigma^{Y,1}_t- \frac{\nabla_{\tilde{x}} a_1(X^{\pi^*}_t-Z_t,t)+ (\lambda_t-\sigma^{Y,1}_t)U_x(X^{\pi^*}_t,t)-\kappa^1_t U_{xx}(X^{\pi^*}_t,t)}{ X^{\pi^*}_t U_{xx}(X^{\pi^*}_t,t) }\right),
\end{aligned}
\label{eq:SPDE_strategy}
\end{equation}
$\{\tilde{U}(\tilde{X}_t,t)\}_{t\geq 0}$ is an $\mathbb{F}$-martingale under some integrability conditions. Therefore, $\tilde{U}$ is a forward preference on $\tilde{X}$, and hence $U$ is a forward preference on $X$.\\

To summarize, the translated forward preference $\tilde{U}$ satisfies the SPDE: for any $\tilde{x}\in\mathcal{\tilde{D}}$ and $t\geq 0$,
% Therefore, with $U_{xx}(X_t,t) <0$,  $\{U(X_t,t)\}_{t\geq 0}$ is indeed a super-martingale under some mild integrability conditions. Let
% \begin{equation*}
% \pi_t^* :=  (\Sigma^\top_t)^{-1} \left(\sigma^{Y,1}_t- \frac{\nabla_{\tilde{x}} a_1(X_t,t)+ (\lambda_t-\sigma^{Y,1}_t)U_x(X_t,t)}{ X_t U_{xx}(X_t,t) }\right),
% \end{equation*}
% we see that $\{U(X^{\pi^*}_t,t)\}_{t\geq 0}$ is a martingale under some integrability conditions. Therefore, $\pi^*$ is the optimal investment strategy to the pension fund. Substituting \eqref{eq:drift_U} back into \eqref{eq:U}, we obtain the SPDE 
\begin{equation}
\begin{aligned}
d\tilde{U}(\tilde{x},t) =&\; %\Big(  x((\sigma^{Y,1}_t)^\top\nabla_{\tilde{x}}a_1(x,t) +(\sigma^{Y,2}_t)^\top\nabla_{\tilde{x}}a_2(x,t))    
%- U_x(x,t) (p_t + x( \sigma_{Y_1}^2+\sigma_{Y_2}^2 - \mu_Y))  \nonumber\\ &\hspace{0.5cm}  
%+   \frac{\sigma_1^2 U_{xx}(x,t) x^2}{2} \left(  \frac{\sigma_{Y_1}}{\sigma_1}  - \frac{a_x^1(x,t)+ (\lambda-\sigma_{Y_1})U_x(x,t)}{\sigma_1 x U_{xx}(x,t)  }   \right)^2 \nonumber \\ &\hspace{0.5cm} -\frac{(\sigma_{Y_1}^2+\sigma_{Y_2}^2)U_{xx}(x,t)x^2}{2} \Big) dt + a^1(x,t)dB^1_t + a^2(x,t)dB^2_t \nonumber \\
\Bigg(\left((\tilde{x}+Z_t)\sigma^{Y,2}_t+ \kappa_t^2 \right)^\top \nabla_{\tilde{x}}a_2(\tilde{x},t) - \frac{ \|(\tilde{x}+Z_t)\sigma^{Y,2}_t+\kappa_t^2\|^2\tilde{U}_{\tilde{x}\tilde{x}}(\tilde{x},t)}{2} \\ & \ \quad -  \tilde{U}_{\tilde{x}}(\tilde{x},t) (p_t-\nu_t + (\lambda_t-\sigma^{Y,1}_t)^\top \kappa^1_t+ (\tilde{x}+Z_t)(\lambda_t^\top\sigma^{Y,1}_t+\|\sigma^{Y,2}_t\|^2 - \mu^Y_t))    \\ &\ \quad  + \frac{ \left\|\nabla_{\tilde{x}} a_1(\tilde{x},t)+ (\lambda_t-\sigma^{Y,1}_t)\tilde{U}_{\tilde{x}}(\tilde{x},t)   \right\|^2   }{2\tilde{U}_{\tilde{x}\tilde{x}}(\tilde{x},t)}  \Bigg) dt \\ &\; + a_1^\top(\tilde{x},t) d{\bf B}^1_t + a^\top_2(\tilde{x},t) d{\bf B}^2_t,
\end{aligned} 
\label{eq:SPDE}
\end{equation}
and the forward preference $U$ is given by, for any $t\geq 0$,
    \begin{equation*}
        U(x,t) = \begin{cases}
            \Tilde{U}(x-Z_t,t), &\text{if }x\in\mathcal{D}_t;\\
            -\infty, &\text{otherwise}.
        \end{cases}
    \end{equation*}
% {\color{red}Compared to the corresponding SPDE for investment problems in the usual financial markets (see, e.g., \cite{Musiela2010b}): for any $x\in\mathbb{R}$ and $t\geq 0$,
% \begin{equation*}
% dU(x,t) = \frac{\|\nabla_xa_1(x,t) + \lambda_tU_x(x,t)\|^2}{2U_{xx}(x,t)} + a_1^\top(x,t)d{\bf B}^1_t,
% \end{equation*}
% the presence of stochastic salary and thus contribution herein has brought extra non-linearity to the drift of \eqref{eq:SPDE}.}


% \iffalse %%%%% hide previous SPDE, 27/11
% For any $\pi\in\mathcal{A}$, by the It\^o-Wentzell formula, for any $t\geq 0$,
% \begin{equation}
% \begin{aligned}
% &\;dU(X_t,t)\\=&\;b(X_t,t) dt + a_1(X_t,t)^\top d{\bf B}^1_t + a_2(X_t,t)^\top d{\bf B}^2_t + U_x(X_t,t)dX_t\\&\;+ \frac{1}{2} U_{xx}(X_t,t) d\langle X_{\cdot}\rangle_t +  X_t ( (\pi_t^\top\Sigma_t-(\sigma^{Y,1}_t)^\top)\nabla_xa_1(X_t,t) - (\sigma^{Y,2}_t)^\top\nabla_xa_2(X_t,t) )  dt\\=&\;\Big( b(X_t,t) + U_x(X_t,t)\left(p_t + X_t\left( \pi_t^\top\Sigma_t(\lambda_t-\sigma^{Y,1}_t) - \mu^Y_t +  \|\sigma^{Y,1}_t\|^2 + \|\sigma^{Y,2}_t\|^2 \right)\right)\\&\;\quad+ \frac{U_{xx}(X_t,t)X_t^2}{2}\left( \|\Sigma^\top_t\pi_t-\sigma^{Y,1}_t\|^2 + \|\sigma^{Y,2}_t\|^2 \right) + X_t \Big( (\pi_t^\top\Sigma_t-(\sigma^{Y,1}_t)^\top)\nabla_xa_1(X_t,t)\\&\;\quad-(\sigma^{Y,2}_t)^\top\nabla_xa_2(X_t,t) \Big)   \Big)dt + (a_1(X_t,t)^\top + U_x(X_t,t)X_t(\pi^\top_t \Sigma_t-(\sigma^{Y,1}_t)^\top )) d{\bf B}^1_t\\&\;+( a_2(X_t,t)^\top -U_x(X_t,t)X_t (\sigma^{Y,2}_t)^\top  ) d{\bf B}^2_t.
% \end{aligned}
% \label{eq:SPDE1}
% \end{equation}


% % \begin{align}
% % & \ \ \ dU(X_t,t) \nonumber \\
% % &= b(X_t,t) dt + a_1^\top(X_t,t) d{\bf B}^1_t + a_2^\top(X_t,t)d{\bf B}^2_t + U_x(X_t,t)dX_t+ \nonumber\\ &\hspace{0.5cm} + \frac{1}{2} U_{xx}(X_t,t) d\langle X\rangle_t +  X_t ( (\pi_t^\top\Sigma_t-(\sigma^{Y,1}_t)^\top)\nabla_xa_1(X_t,t) - (\sigma^{Y,2}_t)^\top\nabla_xa_2(X_t,t) )  dt  \nonumber\\
% % &= \Big( b(X_t,t) + U_x(X_t,t)\left(p_t + X_t\left( \pi_t^\top\Sigma_t(\lambda_t-\sigma^{Y,1}_t) - \mu^Y_t +  \|\sigma^{Y,1}_t\|^2 + \|\sigma^{Y,2}_t\|^2 \right)\right) \nonumber \\ &\hspace{0.5cm} + \frac{U_{xx}(X_t,t)X_t^2}{2}\left( \|\Sigma^\top_t\pi_t-\sigma^{Y,1}_t\|^2 + \|\sigma^{Y,2}_t\|^2 \right) + X_t \Big( (\pi_t^\top\Sigma_t-(\sigma^{Y,1}_t)^\top)\nabla_xa_1(X_t,t) - \nonumber \\ &\hspace{0.5cm} (\sigma^{Y,2}_t)^\top\nabla_xa_2(X_t,t) \Big)   \Big)dt + (a_1^\top(X_t,t) + U_x(X_t,t)X_t(\pi^\top_t \Sigma_t-(\sigma^{Y,1}_t)^\top )) d{\bf B}^1_t+ \nonumber \\ &\hspace{0.5cm}  ( a_2^\top(X_t,t) -U_x(X_t,t)X_t (\sigma^{Y,2}_t)^\top  ) d{\bf B}^2_t.
% % \end{align}

% Notice that the drift term of \eqref{eq:SPDE1} can be written as, for any $t\geq 0$,
% \begin{align*}
% &  \frac{  U_{xx}(X_t,t)X_t^2 }{2}\|\Sigma^\top_t\pi_t \|^2 + \pi_t^\top\Sigma_t\Big( \nabla_xa_1(X_t,t)X_t -\sigma^{Y,1}_tU_{xx}(X_t,t)X_t^2  \\ &\hspace{1cm} +(\lambda_t-\sigma^{Y,1}_t)U_x(X_t,t)X_t  \Big) +  A(X_t,t) \\
% =&\;\frac{  U_{xx}(X_t,t) X_t^2}{2} \left\| \Sigma^\top_t \pi_t - \sigma^{Y,1}_t+ \frac{\nabla_x a_1(X_t,t)+ (\lambda_t-\sigma^{Y,1}_t)U_x(X_t,t)}{ X_t U_{xx}(X_t,t) } \right\|^2 + A(X_t,t),
% \end{align*}
% where, {\color{red}for any $x\in\mathbb{R}$,}
% \begin{align*}
% A(x,t) :=&\; b(x,t) +  \frac{(\|\sigma^{Y,1}_t\|^2+\|\sigma^{Y,2}_t\|^2)U_{xx}(x,t)x^2}{2} - x\big((\sigma^{Y,1}_t)^\top\nabla_xa_1(x,t)  \\  &\;+(\sigma^{Y,2}_t)^\top\nabla_xa_2(x,t)\big)+  U_x(x,t) (p_t + x( \|\sigma^{Y,1}_t\|^2+\|\sigma^{Y,2}_t\|^2 - \mu^Y_t))   \\ &\; -\frac{ U_{xx}(x,t) x^2}{2} \left\| \sigma^{Y,1}_t- \frac{\nabla_x a_1(x,t)+ (\lambda_t-\sigma^{Y,1}_t)U_x(x,t)}{ x U_{xx}(x,t) }   \right\|^2. \nonumber
% \end{align*}
% Now, let the drift process of the random field $U$ by, {\color{red}for any $x\in\mathbb{R}$} and $t\geq 0$,
% \begin{align*}
% b(x,t):=&\;- \frac{(\|\sigma^{Y,1}_t\|^2+\|\sigma^{Y,2}_t\|^2)U_{xx}(x,t)x^2}{2} + x\big((\sigma^{Y,1}_t)^\top\nabla_xa_1(x,t)  \\  &\;+(\sigma^{Y,2}_t)^\top\nabla_xa_2(x,t)\big)-  U_x(x,t) (p_t + x( \|\sigma^{Y,1}_t\|^2+\|\sigma^{Y,2}_t\|^2 - \mu^Y_t))   \\ &\; +\frac{ U_{xx}(x,t) x^2}{2} \left\| \sigma^{Y,1}_t- \frac{\nabla_x a_1(x,t)+ (\lambda_t-\sigma^{Y,1}_t)U_x(x,t)}{ x U_{xx}(x,t) }   \right\|^2\\=&\;- \frac{\|\sigma^{Y,2}_t\|^2U_{xx}(x,t)x^2}{2} + x(\sigma^{Y,2}_t)^\top\nabla_xa_2(x,t)\\&\;-  U_x(x,t) (p_t + x( \lambda^\top_t\sigma^{Y,1}_t+\|\sigma^{Y,2}_t\|^2 - \mu^Y_t))   \\ &\; +\frac{ U_{xx}(x,t) }{2} \left\| \frac{\nabla_x a_1(x,t)+ (\lambda_t-\sigma^{Y,1}_t)U_x(x,t)}{ U_{xx}(x,t) }   \right\|^2,
% \end{align*}
% which leads to $A\left(\cdot,\cdot\right)\equiv 0$, and thus the drift term of \eqref{eq:SPDE1} is simply, for any $t\geq 0$,
% \begin{equation*}
% \frac{  U_{xx}(X_t,t) X_t^2}{2} \left\| \Sigma^\top_t \pi_t - \sigma^{Y,1}_t+ \frac{\nabla_x a_1(X_t,t)+ (\lambda_t-\sigma^{Y,1}_t)U_x(X_t,t)}{ X_t U_{xx}(X_t,t) } \right\|^2.
% \end{equation*}
% % By setting $A(x,t)=0$, i.e., 
% % \begin{align}
% % \label{eq:drift_U}
% % b(x,t) &:=   x((\sigma^{Y,1}_t)^\top\nabla_xa_1(x,t) +(\sigma^{Y,2}_t)^\top\nabla_xa_2(x,t))  
% % - U_x(x,t) \big(p_t + x( \|\sigma^{Y,1}_t\|^2+\|\sigma^{Y,2}_t\|^2 - \nonumber \\ &\hspace{1cm} \mu^Y_t)\big)  
% % +  \frac{ U_{xx}(x,t) x^2}{2} \left\| \sigma^{Y,1}_t- \frac{\nabla_x a_1(x,t)+ (\lambda_t-\sigma^{Y,1}_t)U_x(x,t)}{ x U_{xx}(x,t) }   \right\|^2 - \nonumber  \\ &\hspace{1cm} \frac{(\|\sigma^{Y,1}_t\|^2+\|\sigma^{Y,2}_t\|^2)U_{xx}(x,t)x^2}{2} ,
% % \end{align}
% % the drift term of \eqref{eq:SPDE1} is reduced to 
% % \begin{equation*}
% % \frac{  U_{xx}(X_t,t) X_t^2}{2} \left\| \Sigma^\top_t \pi_t - \sigma^{Y,1}_t+ \frac{\nabla_x a_1(X_t,t)+ (\lambda_t-\sigma^{Y,1}_t)U_x(X_t,t)}{ X_t U_{xx}(X_t,t) } \right\|^2 .
% % \end{equation*}
% By the strict concavity of the random field $U$ in $x$, $\{U(X^{\pi}_t,t)\}_{t\geq 0}$ is indeed an $\mathbb{F}$-super-martingale under some mild integrability conditions. Furthermore, with, for any $t\geq 0$,
% \begin{equation*}
% \pi_t^* :=  (\Sigma^\top_t)^{-1} \left(\sigma^{Y,1}_t- \frac{\nabla_x a_1(X_t,t)+ (\lambda_t-\sigma^{Y,1}_t)U_x(X_t,t)}{ X_t U_{xx}(X_t,t) }\right),
% \end{equation*}
% $\{U(X^{\pi^*}_t,t)\}_{t\geq 0}$ is an $\mathbb{F}$-martingale under some integrability conditions.\\

% Therefore, with the defined drift process $b$, the random field $U$ is indeed a forward preference of the worker on the fund value to salary ratio. Her forward preference then satisfies the SPDE: {\color{red}for any $x\in\mathbb{R}$ and $t\geq 0$,}
% % Therefore, with $U_{xx}(X_t,t) <0$,  $\{U(X_t,t)\}_{t\geq 0}$ is indeed a super-martingale under some mild integrability conditions. Let
% % \begin{equation*}
% % \pi_t^* :=  (\Sigma^\top_t)^{-1} \left(\sigma^{Y,1}_t- \frac{\nabla_x a_1(X_t,t)+ (\lambda_t-\sigma^{Y,1}_t)U_x(X_t,t)}{ X_t U_{xx}(X_t,t) }\right),
% % \end{equation*}
% % we see that $\{U(X^{\pi^*}_t,t)\}_{t\geq 0}$ is a martingale under some integrability conditions. Therefore, $\pi^*$ is the optimal investment strategy to the pension fund. Substituting \eqref{eq:drift_U} back into \eqref{eq:U}, we obtain the SPDE 
% \begin{equation}
% \begin{aligned}
% dU(x,t) =&\; %\Big(  x((\sigma^{Y,1}_t)^\top\nabla_xa_1(x,t) +(\sigma^{Y,2}_t)^\top\nabla_xa_2(x,t))    
% %- U_x(x,t) (p_t + x( \sigma_{Y_1}^2+\sigma_{Y_2}^2 - \mu_Y))  \nonumber\\ &\hspace{0.5cm}  
% %+   \frac{\sigma_1^2 U_{xx}(x,t) x^2}{2} \left(  \frac{\sigma_{Y_1}}{\sigma_1}  - \frac{a_x^1(x,t)+ (\lambda-\sigma_{Y_1})U_x(x,t)}{\sigma_1 x U_{xx}(x,t)  }   \right)^2 \nonumber \\ &\hspace{0.5cm} -\frac{(\sigma_{Y_1}^2+\sigma_{Y_2}^2)U_{xx}(x,t)x^2}{2} \Big) dt + a^1(x,t)dB^1_t + a^2(x,t)dB^2_t \nonumber \\
% \Big( x (\sigma^{Y,2}_t)^\top \nabla_x a_2(x,t) - U_x(x,t)\left(p_t + x\left(\lambda^\top_t\sigma^{Y,1}_t +\|\sigma^{Y,2}_t\|^2 -\mu^Y_t \right) \right)  \\ &\;\quad- \frac{\|\sigma^{Y,2}_t\|^2 x^2 U_{xx}(x,t)}{2} + \frac{1}{2U_{xx}(x,t)} \| \nabla_xa_1(x,t) + (\lambda_t-\sigma^{Y,1}_t) U_x(x,t)\|^2   \Big) dt \\ &\;+ a_1^\top(x,t) d{\bf B}^1_t + a^\top_2(x,t) d{\bf B}^2_t.
% \end{aligned}
% \label{eq:SPDE}
% \end{equation}
% % {\color{red}Compared to the corresponding SPDE for investment problems in the usual financial markets (see, e.g., \cite{Musiela2010b}): for any $x\in\mathbb{R}$ and $t\geq 0$,
% % \begin{equation*}
% % dU(x,t) = \frac{\|\nabla_xa_1(x,t) + \lambda_tU_x(x,t)\|^2}{2U_{xx}(x,t)} + a_1^\top(x,t)d{\bf B}^1_t,
% % \end{equation*}
% % the presence of stochastic salary and thus contribution herein has brought extra non-linearity to the drift of \eqref{eq:SPDE}.}

% \fi %%%%% end hide 
After motivating the volatility processes of forward preferences, we shall construct forward preferences by incorporating exogenous stochastic input factors into an \textit{ansatz} and identify their corresponding volatility processes $a_1$ and $a_2$. The following worker's initial utility are considered respectively in Sections \ref{sec:power} and \ref{sec:exp}.
\begin{itemize}
\item Power utility:
\begin{equation*}
u_0\left(x\right)=
\begin{cases}
\frac{1}{\gamma}x^{\gamma} & \text{for }x\in\mathcal{D}_0\\
-\infty & \text{for }x\in\mathbb{R}\backslash\mathcal{D}_0\\
\end{cases}
\end{equation*}
where $0<\gamma<1$ and $\mathcal{D}_0=\left(0,\infty\right)$.
\item Exponential utility:
\begin{equation*}
u_0\left(x\right)=-\exp\left(-\gamma x\right)\quad\text{for }x\in\mathcal{D}_0=\mathbb{R},
\end{equation*}
where $\gamma>0$.
\end{itemize}


\section{Power Forward Utility Preferences} 
\label{sec:power}
To construct power forward utility preference, the stochastic domain $\mathcal{D}$ is bounded from below, with $\inf\mathcal{D}_0=0$. Let $Z=\{Z_t:= \inf \mathcal{D}_t\}_{t\geq 0}$ be the greatest lower bound of the domain $\mathcal{D}$. In particular, assume that $Z$ admits the following It\^o's diffusion form: for any $t\geq 0$,
\begin{equation}
\label{eq:Z:power}
dZ_t = p_t dt + Z_t\left(\alpha_t dt + \beta^\top_t d{\bf B}^1_t -(\sigma^{Y,2}_t)^\top d{\bf B}^2_t\right),
\end{equation}
where $\alpha=\{\alpha_t\}_{t\geq 0}\in\mathcal{P}_1\left(\mathbb{F}\right)$ and $\beta=\{\beta_t\}_{t\geq 0}\in\mathcal{P}_n\left(\mathbb{F}\right)$ are uniformly bounded; that is, $Z$ satisfies \eqref{eq:Z:SPDE} with $\nu_t = p_t + Z_t\alpha_t$, $\kappa^1_t = Z_t\beta_t$, and $\kappa^2_t = -Z_t\sigma^{Y,2}_t$, for any $t\geq 0$. Furthermore, assume that, for any $t\geq 0$,
\begin{equation}
\label{eq:power:alpha:beta}
\alpha_t - (\lambda_t-\sigma^{Y,1}_t)^\top\beta_t =\lambda^\top_t\sigma^{Y,1}_t +\|\sigma^{Y,2}_t\|^2 -\mu^Y_t.
\end{equation}

While the dynamics in \eqref{eq:Z:power} is technically motivated by the need of balancing both the proportion of contribution $p_{\cdot}$ and the non-hedgeable risk ${\bf B}^2$ in the dynamics of $X$, together with \eqref{eq:power:alpha:beta}, $Z$ bears an economic interpretation. To this end, 
% The domain $\mathcal{D}$ of a power forward utility preference is bounded from below. Let $Z=\{Z_t:= \inf \mathcal{D}_t\}_{t\geq 0}$ be the lower bound of the domain $\mathcal{D}$. In particular, we let $Z$ satisfy the following It\^o's diffusion form: for any $t\geq 0$,
% \begin{equation}
% \label{eq:Z:power}
% dZ_t = p_t dt + Z_t(\alpha_t dt + \beta^\top_t d{\bf B}^1_t -(\sigma^{Y,2}_t)^\top d{\bf B}^2_t), \ Z_0=0,
% \end{equation}
% where $\alpha=\{\alpha_t\}_{t\geq 0}\in\mathcal{P}_1\left(\mathbb{F}\right)$ and $\beta=\{\beta_t\}_{t\geq 0}\in\mathcal{P}_n\left(\mathbb{F}\right)$ are uniformly bounded.
% Hence, $Z$ satisfies \eqref{eq:Z:SPDE} with $\nu_t = p_t + Z_t\alpha_t$, $\kappa^1_t = Z_t\beta_t$ and $\kappa^2_t = -Z_t\sigma^{Y,2}_t$, for any $t\geq 0$.
% Mathematically, the form \eqref{eq:Z:power} is motivated by the need of balancing the contribution $p$ and the non-hedgeable risk ${\bf B}^2$ in the dynamics of $X$. As we shall see in the sequel, to construct a power forward utility preference, it is required that
define $\hat{\pi} = \{\hat{\pi}_t\}_{t\geq 0} \in \mathcal{P}_n(\mathbb{F})$ by,  for any $t\geq 0$,
$$\hat{\pi}_t:=(\Sigma_t^\top)^{-1}(\sigma^{Y,1}_t+\beta_t).$$
Then, using \eqref{eq:Z:power} and \eqref{eq:power:alpha:beta} together, the dynamics of $Z$ can be rewritten as, for any $t\geq 0$,
    \begin{equation}
    \begin{aligned}
    dZ_t =&\;p_tdt + Z_t\left( \left( \hat{\pi}^\top_t\Sigma_t (\lambda_t-\sigma^{Y,1}_t) - \mu^Y_t + \|\sigma^{Y,1}_t\|^2 + \|\sigma^{Y,2}_t\|^2 \right) dt  \right.\\&\quad\quad\quad\quad\quad\left.+ (\hat{\pi}_t^\top\Sigma_t-(\sigma^{Y,1}_t)^\top) d{\bf B}^1_t - (\sigma^{Y,2}_t)^\top d{\bf B}^2_t \right).
    \end{aligned}
    \label{eq:Z:power:rewrite}
\end{equation}
Comparing \eqref{eq:Z:power:rewrite} with \eqref{eq:X}, $Z$ is thus a fund value to salary ratio under an exogenous baseline strategy $\hat{\pi}$ and starting from a zero initial fund value (recall that $Z_0=\inf\mathcal{D}_0=0$). In the sequel, we shall use the notation $X^{\hat{\pi},0}=\{ X^{\hat{\pi},0}_t:=Z_t\}_{t\geq 0} $ to emphasize this economic meaning, where the superscript $0$ indicates that the fund value starts at zero. 
%This auxiliary process resembles the dynamics of the worker's fund value to salary ratio in \eqref{eq:X}, with a more general drift and volatility with respect to the Brownian motions ${\bf B}^1$. 

\subsection{Admissibility}\label{sec:power_admissibility}
With the stochastic domain given by, for any $t\geq 0$, $\mathcal{D}_t=\left(X^{\hat{\pi},0}_t,\infty\right)$, together with an integrability condition for constructing power forward utility preference, define the admissible set of investment strategies by
\begin{equation*}
\label{eq:A:power}
\mathcal{A} =  \{ \pi\in\mathcal{P}_n(\mathbb{F})  : X^{\pi}\pi \in \mathcal{L}^2_n,\  X^\pi_t>X^{\hat{\pi},0}_t, \   \text{for a.a. } (t,\omega) \in [0,\infty)\times \Omega   \}.
\end{equation*}
The admissibility of an investment strategy $\pi$ ensures that the corresponding fund value to salary ratio must be strictly better than the baseline performance; that is, for any $t\geq 0$, $X^{\pi}_t>X_t^{\hat{\pi},0}$. Hence, the auxiliary process $X^{\hat{\pi},0}$ also renders a subsistence level which the worker's fund value to salary ratio $X$ must exceed; see, for example, \cite{method:financial:math}.\\

The following lemma first recalls a standard result for linear SDEs, which shall be used to show that $\hat{\pi}\in\mathcal{A}$ in the next proposition.
% {\color{red} give a class of admissible strategies.}
\begin{lemma} % m dimensional -> d dimensional; m is taken by dimension of B^2
\label{lemma:sde}
Let $\{\tilde{W}_t\}_{t\geq 0} \in \mathcal{P}_d(\mathbb{F})$ be an $d$-dimensional Brownian motion. Let $\eta^1=\{\eta^1_t\}_{t\geq 0}\in \mathcal{P}_1(\mathbb{F})$, $\eta^2=\{\eta^2_t\}_{t\geq 0}\in \mathcal{P}_d(\mathbb{F})$, $\varphi^1=\{\varphi^1_t\}_{t\geq 0}\in \mathcal{P}_1(\mathbb{F})$, and $\varphi^2=\{\varphi^2_t\}_{t\geq 0}\in \mathcal{P}_d(\mathbb{F})$. Consider the following linear SDE: for any $t\geq 0$,
\begin{equation}
\label{eq:app:sde}
d\mathcal{X}_t = \eta^1_t dt + (\eta^2_t)^\top d\tilde{W}_t + \mathcal{X}_t\left(\varphi^1_t dt + (\varphi^2_t)^\top d\tilde{W}_t \right).
\end{equation}
If $\eta^1\in\mathcal{L}^2_1$, $\eta^2\in\mathcal{L}^2_d$, as well as $\varphi^1$ and $\varphi^2$ are both uniformly bounded, then there exists a unique solution $\mathcal{X}=\{\mathcal{X}_t\}_{t\geq 0}\in\mathcal{P}_1(\mathbb{F})$ of \eqref{eq:app:sde} such that $\mathcal{X}\in\mathcal{L}^2_1$. Moreover, $\mathcal{X}$ admits a continuous version.
% Suppose that $\{\tilde{W}_t\}_{t\geq 0} \in \mathcal{P}(\mathbb{F})$ is an $m$-dimensional Brownian motion. Let  $\xi^1=\{\xi^1_t\}_{t\geq 0}\in \mathcal{L}^2_1, \xi_2= \{\xi^2_t\}_{t\geq 0} \in \mathcal{L}^2_m$ and that $a=\{a_t\}_{t\geq 0}, b=\{b_t\}_{t\geq 0} \in \mathcal{P}(\mathbb{F})$ be uniform bounded $\mathbb{R}$ and $\mathbb{R}^m$-valued processes respectively. Then the linear SDE 
% has a unique solution $X=\{X_t\}_{t\geq 0}$ with $X \in \mathcal{L}^2_1$. {\color{blue} In particular, $X$ admits a continuous version. } 
\end{lemma}
% {\color{red} PP3.1 can be absorbed in PP3.2, by taking $\xi=\beta$.}
\begin{proposition}\label{prop:non-empty}
We have $\hat{\pi}\in\mathcal{A}$, which is then non-empty.
\end{proposition}
\begin{proof}
By \eqref{eq:X} and \eqref{eq:power:alpha:beta}, $X^{\hat{\pi}}$ satisfies
\begin{equation*}
dX^{\hat{\pi}}_t = p_t dt +X^{\hat{\pi}}_t(\alpha_t dt + \beta^\top_t d{\bf B}^1_t -(\sigma^{Y,2}_t)^\top d{\bf B}^2_t).
\end{equation*}
Since $p_{\cdot}$, $\alpha$, $\beta$, and $\sigma^{Y,2}$ are all (uniformly) bounded, by Lemma \ref{lemma:sde}, $X^{\hat{\pi}}\in\mathcal{L}^2_1$, and hence $X^{\hat{\pi}}\hat{\pi}\in\mathcal{L}^2_n$ due to the uniform boundedness of $\hat{\pi}$. Since $X^{\hat{\pi}}$ and $X^{\hat{\pi},0}$ solve the same linear SDE (see \eqref{eq:Z:power}), while $X^{\hat{\pi}}_0=x_0>0=X^{\hat{\pi},0}_0$, by the comparison principle, $X^{\hat{\pi}}_t>X^{\hat{\pi},0}_t$ for a.a. $(t,\omega) \in [0,\infty)\times \Omega $. These show that $\hat{\pi}\in\mathcal{A}$.
\end{proof}
% green color: included above
%{\color{red} Included above:}
%{\color{green}The fact that $X^{\hat{\pi}}$ and $Z$ solve the same linear SDE grants a practical interpretation for the auxiliary process $Z$. It is a fund value to salary ratio under an exogenous baseline strategy $\hat{\pi}$ starting from a zero initial fund value. Therefore, the admissibility of an investment strategy $\pi$ ensures that the corresponding fund value to salary ratio must be strictly better than the baseline performance; that is, for any $t\geq 0$, $X_t^{\pi}>X_t^{\hat{\pi},0}=Z_t$. More generally, the auxiliary process $Z$ renders a subsistence level which the worker's fund value to salary ratio $X$ must exceed; see, for example, \cite{method:financial:math}}.\\

Define the set of investment strategies: \begin{align*}
\tilde{\mathcal{A}}=\Bigg\{\pi\in\mathcal{P}_n\left(\mathbb{F}\right):&\;\pi_t=\frac{X^{\hat{\pi},0}_t}{X^\pi_t} \hat{\pi}_t + \left(1-\frac{X^{\hat{\pi},0}_t}{X^\pi_t} \right) (\Sigma^\top_t)^{-1} (\sigma^{Y,1}_t+\xi_t),\;t\geq 0,\\&\;\text{for some uniformly bounded }\xi=\{\xi_t\}_{t\geq 0} \in \mathcal{P}_n(\mathbb{F})\Bigg\},
\end{align*}
in which each investment strategy is essentially a modification around the baseline strategy $\hat{\pi}$ with a uniformly bounded process $\xi$. The following proposition shows that the admissible set of investment strategies $\mathcal{A}$ is not only non-empty, but also rich enough.
\begin{proposition}
\label{pp:admissible}
We have $\tilde{\mathcal{A}}\subseteq\mathcal{A}$.
\end{proposition}

\begin{proof}
Let $\pi\in\mathcal{\tilde{A}}$, which can be simplified as, for any $t\geq 0$,
\begin{equation*}
\pi_t=(\Sigma^\top_t)^{-1}\left(\sigma^{Y,1}_t+\frac{X^{\hat{\pi},0}_t}{X^\pi_t} \beta_t + \left(1-\frac{X^{\hat{\pi},0}_t}{X^\pi_t} \right)\xi_t\right).
\end{equation*}
Also, let $\Tilde{X}_t := X^\pi_t-X^{\hat{\pi},0}_t$, for $t\geq 0$. We have, for any $t\geq 0$,
% \begin{equation}
% \begin{aligned}
% dX^\pi_t =& \ \Big( p_t + (X_t^{\hat{\pi},0}\beta_t+\Tilde{X}_t\xi_t)^\top(\lambda_t-\sigma_t^{Y,1})+X^{\pi}_t(\lambda^\top_t\sigma^{Y,1}_t +\|\sigma^{Y,2}_t\|^2 -\mu^Y_t)\Big)dt  \\&\ \quad+ (X_t^{\hat{\pi},0}\beta_t+\Tilde{X}_t\xi_t)^\top d{\bf B}^1_t- X^\pi_t(\sigma^{Y,2}_t)^\top d{\bf B}_t^2,
% \end{aligned}
% \label{eq:X:lemma}
% \end{equation}
% and
\begin{equation}
\begin{aligned}
d\Tilde{X}_t = & \ \Tilde{X}_t \Big(  \left(  \xi^\top_t(\lambda_t-\sigma^{Y,1}_t) + \lambda^\top_t\sigma^{Y,1}_t+\|\sigma^{Y,2}_t\|^2- \mu^Y_t  \right)  dt  \\ & \ \quad\quad+ \xi_t^\top d{\bf B}^1_t  -  (\sigma^{Y,2}_t)^\top d{\bf B}^2_t\Big).
\end{aligned}
\label{eq:tilde:X}
\end{equation}
By the (uniform) boundedness and Lemma \ref{lemma:sde}, $\Tilde{X}$ satisfying \eqref{eq:tilde:X} implies that $\Tilde{X}\in\mathcal{L}^2_1$. As $X^{\hat{\pi},0}\in\mathcal{L}^2_1$ (see the proof of Proposition \ref{prop:non-empty}), $X^\pi\equiv\Tilde{X}+X^{\hat{\pi},0}\in\mathcal{L}^2_1$.\\

Moreover, since $\tilde{X}_0 = x_0>0$, for any $t\geq 0$,
\begin{equation*}
\tilde{X}_t = x_0e^{\int_0^t \left(  \xi^\top_s(\lambda_s-\sigma^{Y,1}_s) + \lambda^\top_s\sigma^{Y,1}_s+\|\sigma^{Y,2}_s\|^2- \mu^Y_s  \right)ds   } \mathcal{E}_t,
\end{equation*}
where the Dol{\'e}ans-Dade exponential $\mathcal{E}=\{\mathcal{E}_t\}_{t\geq 0}$ satisfies that $\mathcal{E}_0=1$ and, for any $t\geq 0$,
\begin{equation*}
d\mathcal{E}_t = \mathcal{E}_t\left( \xi_t^\top d{\bf B}^1_t - (\sigma_t^{Y,2})^\top d{\bf B}^2_t \right).
\end{equation*}
Obviously, $\tilde{X}_t>0$, and thus $X^\pi_t>X^{\hat{\pi},0}_t$, $\text{for a.a. } (t,\omega) \in [0,\infty)\times \Omega$.\\

Finally, since $p_{\cdot}\in\mathbb{R}_+$ and $X^{\hat{\pi},0}$ solves the linear SDE \eqref{eq:Z:power}, by the comparison principle, $X^{\hat{\pi},0}_t>0$ for $t\geq 0$. Again, by the (uniform) boundedness, together with $X^{\hat{\pi},0}_t/X^\pi_t\in[0,1)$ for $t\geq 0$, $\pi$ is also uniformly bounded. Therefore, $X^\pi\pi\in\mathcal{L}^2_n$. % t=0, \hat{X} = 0.

%   Let $\pi\in\mathcal{\tilde{A}}$, so that for any $t\geq 0$, 
% \begin{equation}
% \label{eq:pi:transform}
% \pi_t := \frac{X^{\hat{\pi},0}_t}{X^\pi_t} \hat{\pi}_t + \left(1-\frac{X^{\hat{\pi},0}_t}{X^\pi_t} \right) (\Sigma^\top_t)^{-1} (\sigma^{Y,1}_t+\xi_t),
% \end{equation}
% where $\xi \in \mathcal{P}_n(\mathbb{F})$ is uniformly bounded. For $t\geq 0$, let $\Tilde{X}_t := X^\pi_t-X^{\hat{\pi},0}_t$. Using \eqref{eq:pi:transform}, we have

% whence

% By Lemma \ref{lemma:sde}, we infer that $\tilde{X}\in \mathcal{L}^2_1$. In addition, 
%     \begin{equation*}
%         \tilde{X}_t^{-1} = x_0^{-1} e^{ -\int_0^t \left( \xi^\top_s(\lambda_s-\sigma^{Y,1}_s) - \mu^Y_s + \lambda^\top_s\sigma^{Y,1}_s+\|\sigma^{Y,2}_s\|^2  \right)ds   } \mathcal{E}^{-1}_t,
%     \end{equation*}
% where $\mathcal{E}=\{\mathcal{E}_t\}_{\{t\geq 0\}}$ satisfies, for $t\geq 0$, 
%     \begin{equation*}
%         d\mathcal{E}_t = \mathcal{E}_t\left( \xi_t^\top d{\bf B}^1_t - (\sigma_t^{Y,2})^\top d{\bf B}^2_t \right), \ \mathcal{E}_0=1.
%     \end{equation*}
% By Doob's maximal inequality, we have $\sup_{s\in [0,t]}\mathcal{E}^{-1}_s <\infty$ $\mathbb{P}$-a.s., whence $\tilde{X}_t = X^\pi_t- X^{\hat{\pi},0}_t >0$ for a.a. $(t,\omega) \in \mathbb{R}_+\times \Omega$. This also implies that $\pi$ is uniformly bounded  by \eqref{eq:pi:transform}. To complete the proof, it remains to show that $X\in \mathcal{L}^2_n$. Since we have shown that $X^{\hat{\pi},0},\tilde{X} \in \mathcal{L}^2_1$, the fact that $X\in \mathcal{L}^2_n$ follows from \eqref{eq:X:lemma} and Lemma \ref{lemma:sde}. 
\end{proof}

%Doob's maximal inequality: 
% \mathcal{E}^{-1} is a submartingale, whence P(\sup \mathcal{E}^{-1}  \geq N) \leq C/N  for some C%

\subsection{Non-Zero Volatility Forward Preferences}\label{sec:admis_power} 
% Owing to the contribution of salary, along with the non-hedgeable risk ${\bf B}^2$, the investment strategy $\pi$ fails to be self-financing. To compensate the effect, it is natural to consider an additive factor $Z$ on $X$, whose dynamics is given by


% where $\alpha=\{\alpha_t\}_{t\geq 0}$ and $\beta=\{\beta_t\}_{t\geq 0}$ are uniformly bounded  $\mathbb{R}$ and $\mathbb{R}^n$-valued $\mathbb{F}$-progressively measurable  processes respectively which satisfies a certain relation. The additional process herein allows freedom for investors to choose their appetite for risk. Nevertheless, this flexibility cannot be carried to ${\bf B}^2$ since the associated risk is non-hedgeable. To ensure the power function is well-defined, it is necessary that the process $X-Z$ is bounded away from zero, whence the admissible set is given by  

% i.e., the domain $\mathcal{D}=\{ \mathcal{D}_t\}_{t\geq 0}$ is taken to be $\mathcal{D}_t := (Z_t,\infty)$. {\color{red} As we shall see from the relation of $\alpha$ and $\beta$ in \eqref{eq:power:alpha:beta}, the admissible set $\mathcal{A}$ is non-empty. In particular, the strategy $\hat{\pi}:= (\Sigma)^{-1}(\sigma^{Y,1} +\beta)$ lies in $\mathcal{A}$. The admissibility and the interpretation of $Z$ will be further discussed in Section \ref{sec:power:discussion}.} In an actuarial point of view, the process $Z$ resembles the floor level of a CPPI insurance. As we shall show, $Z$ corresponds to the fund to salary ratio under an exogenously chosen investment strategy, and the abundance of admissible strategies can be justified by a simple transformation.


% {\color{blue}In particular, for the case of power utility, we have that $\mathcal{D}_t(\omega) = [Z_t(\omega),\infty)$ for some process $Z\in \mathcal{P}(\mathbb{F})$. The process $Z$ renders the subsistence level where $X$ must exceed \cite{method:financial:math}, which can also be interpreted as the floor level of a CPPI type insurance or fund.}
Inspired by Section \ref{sec:SPDE}, the worker's forward utility preference is driven by the comparison on the performance between her investment strategy and the baseline strategy. Moreover, the preference depends on the worker's appetite with respect to the hedgeable and non-hedgeable risks. The following theorem constructs such a (non-)zero volatility power forward utility preference of the worker, together with her corresponding optimal investment strategy.
\begin{theorem}\label{pp:power}
%Let $X^{\hat{\pi},0}$ be the process in \eqref{eq:Z:power}, where $\alpha$ and $\beta$ are uniformly bounded and satisfy \eqref{eq:power:alpha:beta}. 
%\begin{equation}
%\label{eq:power:alpha:beta}
%\alpha_t - (\lambda^\top_t-(\sigma^{Y,1}_t)^\top)\beta_t =\lambda^\top_t\sigma^{Y,1}_t +\|\sigma^{Y,2}_t\|^2 -\mu^Y_t.
%\end{equation}
Let $V=\{V_t\}_{ t\geq 0} \in \mathcal{P}_1(\mathbb{F})$ be a process, given by $V_0=0$ and, for any $t\geq 0$, 
    \begin{equation*}
    \label{eq:V:power:1}
        dV_t = v_t dt + (\theta_t^1)^\top d{\bf B}^1_t + (\theta_t^2)^\top d{\bf B}^2_t,  
    \end{equation*}
where $v = \{v_t\}_{t\geq 0}\in \mathcal{P}_1(\mathbb{F})$, $\theta^1 = \{\theta^1_t\}_{t\geq 0}\in \mathcal{P}_n(\mathbb{F})$, and $\theta^2 =\{\theta^2_t\}_{t\geq 0} \in \mathcal{P}_m(\mathbb{F})$ are uniformly bounded processes, such that, for any $t\geq 0$, 
\begin{equation}
\begin{aligned}
v_t = &  -\frac{\gamma(1+\gamma)}{2}( \|\sigma^{Y,1}_t\|^2 +  \|\sigma^{Y,2}_t\|^2)  - \frac{\gamma\| \lambda_t - \gamma \sigma^{Y,1}_t + \theta^1_t  \|^2}{2(1-\gamma)} \\
& + \gamma\left( (\theta_t^1)^\top \sigma^{Y,1}_t +  (\theta_t^2)^\top \sigma^{Y,2}_t +\mu^Y_t \right) - \frac{\|\theta_t^1\|^2+\|\theta_t^2\|^2}{2}.
\end{aligned}
\label{eq:V:power:2}
%V'_t =  \frac{\gamma\|\lambda_t-\gamma \sigma^{Y,1}_t\|^2}{2(\gamma-1)} -\frac{\gamma(\gamma+1)}{2}(\|\sigma^{Y,1}_t\|^2+\|\sigma^{Y,2}_t\|^2) + \gamma\mu^Y_t , \ V_0=0.
\end{equation}

The random field, for any $t\geq 0$ and $x\in\mathbb{R}$,
\begin{equation}
\label{eq:U:power}
U(x,t) = \begin{cases}
\frac{1}{\gamma} (x-X^{\hat{\pi},0}_t)^\gamma  e^{V_t} &\text{if } x> X^{\hat{\pi},0}_t\\
-\infty &\text{otherwise}
\end{cases},
\end{equation}
is a power forward utility preference on the fund value to salary ratio. In addition, the volatility processes of its translated random field $\{\tilde{U}(\tilde{x},t)=U(\tilde{x}+X^{\hat{\pi},0}_t,t)\}_{ \tilde{x}\in \mathbb{R}_{++} ,  t\geq 0 }$ are given by, for any $\Tilde{x}\in \mathbb{R}_{++}$ and $t\geq 0$,
\begin{equation}
\label{eq:vol:power}
a_1(\tilde{x},t) = \frac{\Tilde{x}^\gamma}{\gamma}e^{V_t}\theta^1_t\quad \text{and} \quad a_2(\tilde{x},t) = \frac{\Tilde{x}^\gamma}{\gamma}e^{V_t}\theta^2_t.
\end{equation}
Moreover, the optimal investment strategy is given by, for any $t\geq 0$, 
\begin{equation}
\label{eq:pi*:power:1}
\pi_t^* = \frac{X^{\hat{\pi},0}_t}{X^*_t} \hat{\pi}_t + \left( 1-\frac{X^{\hat{\pi},0}_t}{X^*_t}\right) (\Sigma^\top_t)^{-1}\frac{\lambda_t-\gamma\sigma^{Y,1}_t + \theta^1_t}{1-\gamma},
\end{equation}
where $X^*:\equiv X^{\pi^*}$  satisfies, for any $t\geq 0$, 
\begin{align*}
    % \begin{aligned}
    dX^*_t =& % \Bigg(  p_t + X^{\hat{\pi},0}_t(\lambda_t-\sigma^{Y,1}_t)^\top (\beta_t + \sigma^{Y,1}_t) + \frac{(X_t^*-X^{\hat{\pi},0}_t)(\lambda_t-\sigma^{Y,1}_t)^\top(\lambda_t-\gamma\sigma^{Y,1}_t+\theta^1_t)}{1-\gamma}   %\\ & \ \quad + X^*_t\left(    \|\sigma^{Y,1}_t\|^2+\|\sigma^{Y,2}_t\|^2 -\mu^Y_t \right)      \Bigg) dt  \\
%& \ \quad + \Bigg( X^{\hat{\pi},0}_t \beta_t  +   (X^*_t-X^{\hat{\pi},0}_t)\left( \frac{\lambda_t- \sigma^{Y,1}_t + \theta^1_t }{1-\gamma}  \right)         \Bigg)^\top       d{\bf B}_t^1 
%- X_t^*(\sigma^{Y,2}_t)^\top d{\bf B}^2_t.
 \  \Bigg(  p_t + X^{\hat{\pi},0}_t\left( (\lambda_t-\sigma^{Y,1}_t)^\top\beta_t + \lambda^\top_t \sigma^{Y,1}_t + \|\sigma^{Y,2}_t\|^2 -\mu^Y_t \right)   \\
 & \quad + (X_t^*-X^{\hat{\pi},0}_t) \left( \frac{(\lambda_t-\sigma^{Y,1}_t)^\top(\lambda_t-\sigma^{Y,1}_t+\theta^1_t)}{1-\gamma} +  \lambda^\top_t \sigma^{Y,1}_t + \|\sigma^{Y,2}_t\|^2 -\mu^Y_t  \right)\Bigg)dt  
 \\ & \ + \Bigg( X^{\hat{\pi},0}_t \beta_t  +   (X^*_t-X^{\hat{\pi},0}_t)\left( \frac{\lambda_t- \sigma^{Y,1}_t + \theta^1_t }{1-\gamma}  \right)         \Bigg)^\top       d{\bf B}_t^1 
- X_t^*(\sigma^{Y,2}_t)^\top d{\bf B}^2_t.
    % \end{aligned}
% \label{eq:power:X}
\end{align*} 
% In addition, the random field $\{\tilde{U}(\tilde{x},t)=U(\tilde{x}+X^{\hat{\pi},0}_t,t)\}_{ \tilde{x}\in \mathbb{R}_{++} ,  t\geq 0 }$ is the solution to the SPDE \eqref{eq:SPDE} with   volatility processes  
\end{theorem}
\begin{proof} 
First, notice that $\pi^*$ can be rewritten as, for any $t\geq 0$,
\begin{equation*}
\pi_t^* = \frac{X^{\hat{\pi},0}_t}{X^*_t} \hat{\pi}_t + \left( 1-\frac{X^{\hat{\pi},0}_t}{X^*_t}\right) (\Sigma^\top_t)^{-1}\left(\sigma^{Y,1}_t+\frac{\lambda_t-\sigma^{Y,1}_t + \theta^1_t}{1-\gamma}\right),
\end{equation*}
and thus, by the (uniform) boundedness and Proposition \ref{pp:admissible}, $\pi^*\in\Tilde{\mathcal{A}}\subseteq\mathcal{A}$. Next, it is clear that $U$ in \eqref{eq:U:power} satisfies (i) and (ii) in Definition \ref{def:forward}. For verifying (iii) of Definition \ref{def:forward}, it suffices to show that $\{U(X^\pi_t,t)\}_{t\geq 0}$ is an $\mathbb{F}$-super-martingale for any $\pi\in\mathcal{A}$,  and that $\{U(X^*_t,t)\}_{t\geq 0}$ is an $\mathbb{F}$-martingale.\\

For any $\pi\in\mathcal{A}$, define the processes $\Tilde{X}=\{\Tilde{X}_t:=X^\pi_t-X^{\hat{\pi},0}_t\}_{t\geq 0}$, and $\tilde{\xi}=\{\tilde{\xi}_t\}_{t\geq 0}$ which is given by, for any $t\geq 0$, 
\begin{equation*}
\tilde{\xi}_t := \frac{X^\pi_t(\Sigma^\top_t \pi_t-\sigma^{Y,1}_t)-X^{\hat{\pi},0}_t\beta_t }{\Tilde{X}_t}.
\end{equation*}   %%% this \xi is not necessarily boundeded, because \pi \in A (but not in \mathcal{A}).
Then, $\Tilde{X}_0=x_0$ and $\Tilde{X}$ satisfies, for any $t\geq 0$,
\begin{equation*}
\begin{aligned}
d\Tilde{X}_t = & \ \Tilde{X}_t \Big(  \left(  \tilde{\xi}^\top_t(\lambda_t-\sigma^{Y,1}_t) + \lambda^\top_t\sigma^{Y,1}_t+\|\sigma^{Y,2}_t\|^2- \mu^Y_t  \right)  dt  \\ & \ \quad\quad+ \tilde{\xi}_t^\top d{\bf B}^1_t  -  (\sigma^{Y,2}_t)^\top d{\bf B}^2_t\Big).
\end{aligned}
%\label{eq:tilde:X}
\end{equation*}
Also, define $R_t^{\pi}:=\frac{1}{\gamma} (X^\pi_t-X^{\hat{\pi},0}_t)^\gamma e^{V_t}=\frac{1}{\gamma} \Tilde{X}_t^\gamma e^{V_t}$ for $t\geq 0$.\\

By It\^o's lemma, we obtain, for any $t\geq 0$,
    \begin{align*}
         \frac{dR_t^{\pi}}{R_t^{\pi}}  = & \ \frac{\gamma}{\tilde{X}_t} d\Tilde{X}_t + dV_t + \frac{\gamma(\gamma-1)}{2\tilde{X}_t^2} d\langle \tilde{X}_\cdot \rangle_t + \frac{1}{2}d\langle V_\cdot \rangle_t + \frac{\gamma}{\tilde{X}_t} d\langle  \Tilde{X}_\cdot ,V_\cdot\rangle_t \\ 
         = & \ \bigg( \gamma\left(  \tilde{\xi}^\top_t(\lambda_t-\sigma^{Y,1}_t) + \lambda^\top_t\sigma^{Y,1}_t+\|\sigma^{Y,2}_t\|^2- \mu^Y_t \right) + v_t  - \frac{\gamma(1-\gamma)}{2}(\|\tilde{\xi}_t\|^2+\|\sigma^{Y,2}_t\|^2)  \\ 
         & \ \quad + \frac{\|\theta^1_t\|^2+\|\theta_t^2\|^2}{2} + \gamma\left(\tilde{\xi}_t^\top \theta^1_t - (\sigma^{Y,2}_t)^\top \theta^2_t \right) \bigg) dt \\
         & \ + \left( \gamma \tilde{\xi}_t + \theta^1_t \right)^\top d{\bf B}^1_t + \left( \theta^2_t - \gamma\sigma^{Y,2}_t \right)^\top d{\bf B}^2_t \\
        %   = & \ \bigg( -\frac{\gamma(1-\gamma)}{2}\left\| \tilde{\xi}_t - \frac{\lambda_t-\sigma^{Y,1}_t + \theta^1_t}{1-\gamma} \right\|^2 + v_t + \gamma\left(\lambda_t^\top \sigma^{Y,1}_t-\mu^Y_t - (\sigma^{Y,2}_t)^\top \theta^2_t\right) \\ 
        % & \ \quad +\frac{\gamma(1+\gamma)\|\sigma^{Y,2}_t\|^2}{2}       +   \frac{\|\theta^1_t\|^2+\|\theta_t^2\|^2}{2}+\frac{\gamma\|\lambda_t-\sigma^{Y,1}_t+\theta^1_t\|^2}{2(1-\gamma)} \bigg)dt \\
        % & \ +   \left( \gamma \tilde{\xi}_t + \theta^1_t \right)^\top d{\bf B}^1_t + \left( \theta^2_t - \gamma\sigma^{Y,2}_t \right)^\top d{\bf B}^2_t \\
        = &  \ \bigg( -\frac{\gamma(1-\gamma)}{2}\left\| \tilde{\xi}_t - \frac{\lambda_t-\sigma^{Y,1}_t + \theta^1_t}{1-\gamma} \right\|^2 + v_t - \gamma\left((\theta^1_t)^\top\sigma^{Y,1}_t + (\theta^2_t)^\top \sigma^{Y,2}_t+\mu^Y_t \right) \\ 
        & \ \quad +\frac{\gamma(1+\gamma)\left(\|\sigma^{Y,1}_t\|^2 + \|\sigma^{Y,2}_t\|^2 \right) }{2}      +   \frac{\|\theta^1_t\|^2+\|\theta_t^2\|^2}{2}+\frac{\gamma\|\lambda_t-\gamma\sigma^{Y,1}_t+\theta^1_t\|^2}{2(1-\gamma)} \bigg)dt \\
        & \ +   \left( \gamma \tilde{\xi}_t + \theta^1_t \right)^\top d{\bf B}^1_t + \left( \theta^2_t - \gamma\sigma^{Y,2}_t \right)^\top d{\bf B}^2_t \\
        = &  \  -\frac{\gamma(1-\gamma)}{2}\left\| \tilde{\xi}_t - \frac{\lambda_t-\sigma^{Y,1}_t + \theta^1_t}{1-\gamma} \right\|^2  dt +  \left( \gamma \tilde{\xi}_t + \theta^1_t \right)^\top d{\bf B}^1_t + \left( \theta^2_t - \gamma\sigma^{Y,2}_t \right)^\top d{\bf B}^2_t ,
    \end{align*}
in which the last equality is due to \eqref{eq:V:power:2}. Hence, since $\pi\in\mathcal{A}$, for any $0\leq s\leq t$,
    \begin{equation}
    \label{eq:U:power:supermartingale}
        U(X_t^\pi,t) = U(X_s^\pi,s)\exp\left(-\int_s^t  \frac{\gamma(1-\gamma)}{2}\left\| \tilde{\xi}_l - \frac{\lambda_l-\sigma^{Y,1}_l + \theta^1_l}{1-\gamma} \right\|^2 dl\right)\mathcal{E}_{s,t},
    \end{equation}
where $\{\mathcal{E}_{s,t}\}_{t\geq s}$ satisfies $\mathcal{E}_{s,s}=1$ and, for any $0\leq s\leq t$,
\begin{equation}
d\mathcal{E}_{s,t} = \mathcal{E}_{s,t}\left( (\gamma\tilde{\xi}_t+\theta^1_t)^\top d{\bf B}^1_t + (\theta^2_t - \gamma \sigma^{Y,2}_t)^\top d{\bf B}^2_t \right).
\label{eq:dolean_exp}
\end{equation}

We shall show that, for any $t\geq 0$, 
    \begin{equation}
\label{eq:power:P}
\int_0^t \|\gamma\tilde{\xi}_s+\theta^1_s\|^2 ds = \int_0^t \left\| \frac{\gamma X^\pi_s(\Sigma^\top_s \pi_s-\sigma^{Y,1}_s)-X^{\hat{\pi},0}_s\beta_s }{\Tilde{X}_s} + \theta^1_s \right\|^2 ds < \infty,  \  \mathbb{P}\text{-a.s.}.
\end{equation}
This then implies that $\{\mathcal{E}_{s,t}\}_{t\geq s}$ in \eqref{eq:dolean_exp} is an $\mathbb{F}$-local martingale. Since $\pi \in \mathcal{A}$, we have $X^\pi\pi\in \mathcal{L}^2_n$. Consider the process $\bar{X}=\{\bar{X}_t\}_{t\geq 0}$, which satisfies $\bar{X}_0=x_0$ and, for any $t\geq 0$,
 \begin{align*}
d\bar{X}_t =& \left(p_t + (X^\pi_t\pi_t)^\top \Sigma_t(\lambda_t - \sigma^{Y,1}_t)\right)dt  + (X^\pi_t \pi_t)^\top\Sigma_t d{\bf B}^1_t   \\ &\ +\bar{X}_t \left( \left(   - \mu^Y_t + \|\sigma^{Y,1}_t\|^2 + \|\sigma^{Y,2}_t\|^2 \right)  dt     -(\sigma^{Y,1}_t)^\top d{\bf B}^1_t - (\sigma^{Y,2}_t)^\top d{\bf B}^2_t \right).
\end{align*}
Using the fact that $X^\pi\pi\in \mathcal{L}^2_n$ and Lemma \ref{lemma:sde}, we infer that $\bar{X}\in \mathcal{L}^2_1$. Notice that the difference $X^{\pi}-\bar{X}$ satisfies $X^{\pi}_0-\bar{X}_0 = 0$, and, for any $t\geq 0$, 
 \begin{equation}
 \label{eq:xbar-x}
d(\bar{X}_t-X^\pi_t) = (\bar{X}_t-X^\pi_t) \left( \left(   - \mu^Y_t + \|\sigma^{Y,1}_t\|^2 + \|\sigma^{Y,2}_t\|^2 \right)  dt     -(\sigma^{Y,1}_t)^\top d{\bf B}^1_t - (\sigma^{Y,2}_t)^\top d{\bf B}^2_t \right).
\end{equation}
By the uniqueness of the solution to \eqref{eq:xbar-x}, we infer that $\bar{X}$ and $X^\pi$ are indistinguishable, and hence $X^\pi \in \mathcal{L}^2_1$. Also, recall that $X^{\hat{\pi},0}\in \mathcal{L}^2_1$ (see the proof of Proposition \ref{prop:non-empty}). Finally, also by Lemma \ref{lemma:sde}, $X^\pi$ and $X^{\hat{\pi},0}$ admit continuous version, and thus $\Tilde{X}$ also admits continuous sample paths $\mathbb{P}$-a.s.. Together with the admissibility condition $\tilde{X}_t=X^\pi_t-X^{\hat{\pi},0}_t>0$ for a.a. $(t,\omega)\in \mathbb{R}_{+}\times \Omega$, we have $\sup_{s\in [0,t]}|\tilde{X}_s|^{-1}<\infty$, $\mathbb{P}$-a.s., for any $t\geq 0$. Therefore, these together imply that
\begin{equation*}
\begin{aligned}
& \int_0^t \left\| \frac{\gamma X^\pi_s(\Sigma^\top_s \pi_s-\sigma^{Y,1}_s)-X^{\hat{\pi},0}_s\beta_s }{\Tilde{X}_s} + \theta^1_s \right\|^2 ds \\
\leq &\ 2\int_0^t \left(2\gamma^2\left(\sup_{s\in[0,t]}|\Tilde{X}_s|^{-1}\right)^2\left( 2\left( \|\Sigma_s\|^2\|X^\pi_s\pi\|^2 + \|\sigma^{Y,1}_s\|^2|X^\pi_s|^2 \right) + \|\beta_s\|^2|X^{\hat{\pi},0}_s|^2 \right) + \|\theta^1_s\|^2\right) ds \\
<&\ \infty , \  \mathbb{P}\text{-a.s.,}
\end{aligned}
\end{equation*}
which proves \eqref{eq:power:P}. 
% To proceed, we first recall that $X^{\hat{\pi},0}\in \mathcal{L}^2_1$ (see the proof of Proposition \ref{prop:non-empty}).
% Since $\pi \in \mathcal{A}$, we also have $X^\pi\pi\in \mathcal{L}^2_n$. Using this, we can show that $X^\pi\in \mathcal{L}^2_1$. To this end, consider the process $\bar{X}=\{\bar{X}_t\}_{t\geq 0}$, which satisfies $\bar{X}_0=x_0$ and for any $t\geq 0$,
%  \begin{align*}
% d\bar{X}_t =& \left(p_t + (X^\pi_t\pi^\top_t) \Sigma_t(\lambda_t - \sigma^{Y,1}_t)\right)dt  + (X^\pi_t \pi^\top_t)\Sigma_t d{\bf B}^1_t   \\ &\ +\bar{X}_t \left( \left(   - \mu^Y_t + \|\sigma^{Y,1}_t\|^2 + \|\sigma^{Y,2}_t\|^2 \right)  dt     -(\sigma^{Y,1}_t)^\top d{\bf B}^1_t - (\sigma^{Y,2}_t)^\top d{\bf B}^2_t \right).
% \end{align*}
% Using the fact that $X^\pi\pi\in \mathcal{L}^2_n$ and Lemma \ref{lemma:sde}, we infer that $\bar{X}\in \mathcal{L}^2_1$. Notice that the difference $X^{\pi}-\bar{X}$ satisfies $X^{\pi}_0-\bar{X}_0 = 0$, and for any $t\geq 0$, 
%  \begin{equation}
%  \label{eq:xbar-x}
% d(\bar{X}_t-X^\pi_t) = (\bar{X}_t-X^\pi_t) \left( \left(   - \mu^Y_t + \|\sigma^{Y,1}_t\|^2 + \|\sigma^{Y,2}_t\|^2 \right)  dt     -(\sigma^{Y,1}_t)^\top d{\bf B}^1_t - (\sigma^{Y,2}_t)^\top d{\bf B}^2_t \right).
% \end{equation}
% By the uniqueness of the solution to \eqref{eq:xbar-x}, we infer that $\bar{X}$ and $X^\pi$ are indistinguishable, whence $X^\pi \in \mathcal{L}^2_1$ as desired.
% Using the continuity of the sample paths of $X^\pi$ and $X^{\hat{\pi},0}$, we see that $\Tilde{X}$ also admits continuous sample paths $\mathbb{P}$-a.s.. Along with the admissibility condition $\tilde{X}_t=X^\pi_t-X^{\hat{\pi},0}_t>0$ for a.a. $(t,\omega)\in \mathbb{R}_{+}\times \Omega$, we have $\sup_{s\in [0,t]}|\tilde{X}_s|^{-1}<\infty$ $\mathbb{P}$-a.s., for any $t\geq 0$. Using this, alongside with the proven facts $X^\pi\pi \in \mathcal{L}^2_n$, $X^{\hat{\pi},0},X^\pi\in \mathcal{L}^2_1$ and the uniform boundedness of $\Sigma, \theta^1,\beta$ and $\sigma^{Y,1}$, we obtain 
%     \begin{equation*}
%         \begin{aligned}
%            & \ \ \   \int_0^t \left\| \frac{\gamma X^\pi_s(\Sigma^\top_s \pi_s-\sigma^{Y,1}_s)-X^{\hat{\pi},0}_s\beta_s }{\Tilde{X}_s} + \theta^1_s \right\|^2 ds \\
%            &\leq  2\int_0^t \left(2\gamma^2\sup_{s\in[0,t]}|\Tilde{X}_s^{-1}|^2\left( 2\left( \|\Sigma_s\|^2\|X^\pi_s\pi\|^2 + \|\sigma^{Y,1}_s\|^2|X^\pi_s|^2 \right) + \|\beta_s\|^2|X^{\hat{\pi},0}_s|^2 \right) + \|\theta^1_s\|^2\right) ds \\
%            &<\infty , \  \mathbb{P}\text{-a.s.,}
%         \end{aligned}
%     \end{equation*} 
% which proves \eqref{eq:power:P}.
Since $\{\mathcal{E}_{s,t}\}_{t\geq s}$ in \eqref{eq:dolean_exp} is bounded below by zero, it is also an $\mathbb{F}$-super-martingale. 
% Using this, and the boundedness of $\theta^2$ and $\sigma^{Y,2}$, we infer that $\{\mathcal{E}_{s,t}\}_{t\geq s}$ is a local martingale. Since $\{\mathcal{E}_{s,t}\}_{t\geq s}$ is bounded from below by zero, it is indeed a super-martingale.
With $0<\gamma<1$, by \eqref{eq:U:power:supermartingale}, for any $0\leq s\leq t$,
\begin{equation*}
\mathbb{E}[U(X^\pi_t,t)|\mathcal{F}_s] \leq  U(X^\pi_s,s) \mathbb{E}[\mathcal{E}_{s,t}|\mathcal{F}_s] \leq U(X^\pi_s,s),
\end{equation*}
and hence $\{U(X^\pi_t,t)\}_{t\geq 0}$ is also an $\mathbb{F}$-super-martingale.\\

% Now, consider the strategy $\pi^*$ given in \eqref{eq:pi*:power:1}, which is admissible since $\pi^*\in \mathcal{\Tilde{A}}\subset \mathcal{A}$, thanks to Proposition \ref{pp:admissible}. Substituting $\pi^*$ into \eqref{eq:X}, we obtain \eqref{eq:power:X}.
In particular, with $\pi\in\mathcal{A}$ given by $\pi^*$, we have, for any $t\geq 0$,
\begin{equation*}
\xi^*_t := \tilde{\xi}^{\pi^*}_t = \frac{X^*_t(\Sigma^\top_t \pi^*_t-\sigma^{Y,1}_t)-X^{\hat{\pi},0}_t\beta_t }{X^*_t-X^{\hat{\pi},0}_t} =\frac{\lambda_t-\sigma^{Y,1}_t+\theta^1_t}{1-\gamma},
\end{equation*}
and correspondingly, from \eqref{eq:U:power:supermartingale}, for any $0\leq s\leq t$,
\begin{equation*}
U(X_t^*,t) = U(X_s^*,s) \mathcal{E}_{s,t}^*,
\end{equation*}
% Substituting $\tilde{\xi}=\xi^*$ into \eqref{eq:U:power:supermartingale}, we obtain, for any $t\geq s\geq 0$, 
% \begin{equation*}
% U(X_t^*,t) = U(X_s^*,s) \mathcal{E}_{s,t}^*,
% \end{equation*}
where  $\{\mathcal{E}_{s,t}^*\}_{t\geq s}$ satisfies $\mathcal{E}_{s,s}^*=1$ and, for any $0\leq s\leq t$, 
\begin{equation*}
d\mathcal{E}_{s,t}^* = \mathcal{E}_{s,t}^*\left(  \frac{\left(\gamma(\lambda_t - \sigma^{Y,1}_t)+  \theta_t^1  \right)^\top}{1-\gamma} d{\bf B}^1_t + \left( \theta_t^2 - \gamma \sigma^{Y,2}_t   \right)^\top d{\bf B}^2_t\right),
\end{equation*}
which is clearly an $\mathbb{F}$-martingale, by the (uniform) boundedness of $\lambda_{\cdot}$, $\sigma^{Y,1}_{\cdot}$, $\sigma^{Y,2}_{\cdot}$, $\theta^1$, and $\theta^2$. Therefore, $\{U(X_t^*,t)\}_{t\geq 0}$ is also an $\mathbb{F}$-martingale.\\

Finally, by It\^o's lemma on the translated random field $\Tilde{U}$, for any $\tilde{x}\in\mathbb{R}_{++}$ and $t\geq 0$, 
    \begin{equation*}
        d\tilde{U}(\tilde{x},t) = \tilde{U}(\tilde{x},t) \left(  \left( v_t + \frac{\|\theta^1_t\|^2+\|\theta^2_t\|^2}{2}\right)dt + (\theta^1_t)^\top d{\bf B}^1_t +  (\theta^2_t)^\top d{\bf B}^2_t \right),
    \end{equation*}
which coincides with \eqref{eq:SPDE}, with $\nu_t = p_t + X^{\hat{\pi},0}_t\alpha_t$, $\kappa^1_t = X^{\hat{\pi},0}_t\beta_t$, and $\kappa^2_t = -X^{\hat{\pi},0}_t\sigma^{Y,2}_t$, for any $t\geq 0$, as well as $a_1$ and $a_2$ given in \eqref{eq:vol:power}.

% checked, it agrees with the SPDE with $\kappa^1 = \beta\hat{X}$, $\kappa_2 = -\sigma^{Y,2}\hat{X}$, $a_1 = \theta^1 \tilde{U}$ and $a_2=\theta^2 \tilde{U}$. 
%Alfred: I checked as well. It's correct.

%For $\pi\in\mathcal{A}$ and $X := X^\pi$, applying It\^o's lemma yields
%\begin{align*}
%&  \ \   \frac{dU(X_t,t)}{U(X_t,t)} \\
%=& \  \frac{\gamma}{X_t-Z_t} d(X_t-Z_t) + dV_t + \frac{\gamma(\gamma-1)}{2(X_t-Z_t)^2} d\langle X_\cdot - Z_\cdot \rangle_t + \frac{1}{2}d\langle V_\cdot \rangle_t + \frac{\gamma}{X_t-Z_t} d\langle X_\cdot-Z_\cdot,V_\cdot \rangle_t \\
%=& \  \frac{1}{(X_t-Z_t)^2}\Big( \gamma X_t(X_t-Z_t)  \left( \pi_t^\top\Sigma_t(\lambda_t-\sigma^{Y,1}_t) - \mu^Y_t + \|\sigma^{Y,1}_t\|^2 + \|\sigma^{Y,2}_t\|^2 \right) \\
%& \ + {\color{red} v_t(X_t-Z_t)^2}   -\gamma\alpha_t Z_t(X_t-Z_t) + \frac{\gamma(\gamma-1)}{2}\Big( X_t^2\left(\|\pi_t^\top\Sigma_t-(\sigma^{Y,1}_t)^\top\|^2 + \|\sigma^{Y,2}_t\|^2 \right) \\ 
%& \   +  Z_t^2 (\|\beta_t\|^2 + \|\sigma^{Y,2}_t\|^2)   - 2X_tZ_t((\pi_t^\top \Sigma_t-(\sigma^{Y,1}_t)^\top)\beta_t +\|\sigma^{Y,2}_t\|^2)  \Big) \\
%& \ +   {\color{red} \frac{(\|\theta_t^1\|^2 + \|\theta_t^2\|^2)(X_t-Z_t)^2 }{2} + \gamma(X_t-Z_t)(\theta_t^1)^\top\left(  X(\Sigma^\top_t \pi_t - \sigma^{Y,1}_t) - \beta_tZ_t  \right) }  \\
%& \ -{\color{red}\gamma (\theta_t^2)^\top \sigma^{Y,2}_t (X_t-Z_t)^2} \Big) dt  + \left( {\color{red}(\theta_t^1)^\top} + \frac{\gamma}{(X_t-Z_t)}\left(  X_t(\pi_t^\top\Sigma_t-(\sigma^{Y,1}_t)^\top) - \beta^\top_tZ_t\right) \right)d{\bf B}^1_t     \\
%& \ + \left({\color{red} (\theta_t^2)^\top  } -\gamma(\sigma^{Y,2}_t)^\top\right) d{\bf B}^2_t \\
%= & \ \frac{1}{(X_t-Z_t)^2}\Bigg( \frac{\gamma(\gamma-1)X_t^2}{2}\left\| \Sigma^\top_t\pi_t - \frac{\lambda_t-\gamma\sigma^{Y,1}_t + {\color{red} \theta^1_t}}{1-\gamma} + \frac{  Z_t(\lambda_t-\sigma^{Y,1}_t-(1-\gamma)\beta_t + {\color{red}  \theta_t^1})}{(1-\gamma)X_t} \right\|^2 \\
%&\   + X_t^2 \Big(  \gamma(\|\sigma^{Y,1}_t\|^2 + \|\sigma^{Y,2}_t\|^2 -\mu^Y_t) + \frac{\gamma(\gamma-1)}{2}(\|\sigma^{Y,1}_t\|^2+\|\sigma^{Y,2}_t\|^2) +{\color{red}v_t - \gamma (\theta_t^1)^\top \sigma^{Y,1}_t } \\ & \ 
% -{\color{red} \gamma (\theta_t^2)^\top \sigma^{Y,2}_t + \frac{\|\theta_t^1\|^2 + \|\theta_t^2\|^2}{2} }         + \frac{\gamma\|\lambda_t-\gamma \sigma^{Y,1}_t + {\color{red} \theta_t^1} \|^2}{2(1-\gamma)}     \Big) \\
% & \ - X_tZ_t\Big(  \gamma (\|\sigma^{Y,1}_t\|^2 + \|\sigma^{Y,2}_t\|^2 -\mu^Y_t) + \gamma \alpha_t + {\color{red}2v_t + \|\theta_t^1\|^2 + \|\theta_t^2\|^2 + \gamma \beta_t^\top \theta^1_t}   \\ 
%& \ - {\color{red}   \gamma (\theta_t^1)^\top \sigma^{Y,1}_t - 2\gamma (\theta_t^2)^\top \sigma^{Y,2}_t } + \gamma(1-\gamma)(\beta^\top_t\sigma^{Y,1}_t-\|\sigma^{Y,2}_t\|^2)   \\ 
%& \  +\frac{\gamma (\lambda_t-\gamma \sigma^{Y,1}_t + {\color{red} \theta^1_t })^\top( \lambda_t-\sigma^{Y,1}_t- (1-\gamma)\beta_t + {\color{red} \theta^1_t } )}{1-\gamma}       \Big)  \\
%& \ +   Z_t^2\Big( \gamma\alpha_t+ \frac{\gamma(\gamma-1)(\|\beta_t\|^2+\|\sigma^{Y,2}_t\|^2)}{2} + \frac{\gamma\|\lambda_t-\sigma^{Y,1}_t - (1-\gamma)\beta_t + {\color{red} \theta^1_t }\|^2}{2(1-\gamma)} + {\color{red}v_t} \\ 
%& \ + {\color{red}  \gamma  \beta_t^\top \theta^1_t - \gamma (\theta_t^2)^\top \sigma^{Y,2}_t+ \frac{\|\theta^1_t\|^2+\|\theta_t^2\|^2}{2} } \Big)   \Bigg)  dt \\ 
%& \ +  \left( {\color{red}(\theta_t^1)^\top} + \frac{\gamma}{(X_t-Z_t)}\left(  X_t(\pi_t^\top\Sigma_t-(\sigma^{Y,1}_t)^\top) - \beta^\top_tZ_t\right) \right)d{\bf B}^1_t     + \left({\color{red} (\theta_t^2)^\top  } -\gamma(\sigma^{Y,2}_t)^\top\right) d{\bf B}^2_t.
%\end{align*}



%\begin{align*}
%&  \ \   \frac{dU(X_t,t)}{U(X_t,t)} \\
%=& \  \frac{\gamma}{X_t-Z_t} d(X_t-Z_t) + V'_t dt + \frac{\gamma(\gamma-1)}{2(X_t-Z_t)^2}\left( d\langle X_\cdot \rangle_t + d\langle Z_\cdot \rangle_t - 2 d\langle X_\cdot,Z_\cdot \rangle_t  \right) \\
%=& \  \frac{1}{(X_t-Z_t)^2}\Big( \gamma X_t(X_t-Z_t)  \left( \pi_t^\top\Sigma_t(\lambda_t-\sigma^{Y,1}_t) - \mu^Y_t + \|\sigma^{Y,1}_t\|^2 + \|\sigma^{Y,2}_t\|^2 \right) \\   
%& \  + (X_t-Z_t)^2V'_t   -\gamma\alpha_t Z_t(X_t-Z_t) + \frac{\gamma(\gamma-1)}{2}\Big( X_t^2\left(\|\pi_t^\top\Sigma_t-(\sigma^{Y,1}_t)^\top\|^2 + \|\sigma^{Y,2}_t\|^2 \right) \\ 
%& \  +  Z_t^2 (\|\beta_t\|^2 + \|\sigma^{Y,2}_t\|^2)   - 2X_tZ_t((\pi_t^\top \Sigma_t-(\sigma^{Y,1}_t)^\top)\beta_t {\color{red} +}\|\sigma^{Y,2}_t\|^2)  \Big)      \Big) dt \\
%& \  + \frac{\gamma}{(X_t-Z_t)}\left(  X_t(\pi_t^\top\Sigma_t-(\sigma^{Y,1}_t)^\top) - \beta^\top_tZ_t\right)d{\bf B}^1_t      - \gamma(\sigma^{Y,2}_t)^\top d{\bf B}^2_t \\
%= & \ \frac{1}{(X_t-Z_t)^2}\Bigg( \frac{\gamma(\gamma-1)X_t^2}{2}\left\| \Sigma^\top_t\pi_t - \frac{(\lambda_t-\gamma\sigma^{Y,1}_t)X_t - Z_t(\lambda_t-\sigma^{Y,1}_t-(1-\gamma)\beta_t)}{(1-\gamma)X_t} \right\|^2 \\
%&\   + X_t^2 \Big(  \gamma(\|\sigma^{Y,1}_t\|^2 + \|\sigma^{Y,2}_t\|^2 -\mu^Y_t) + \frac{\gamma(\gamma-1)}{2}(\|\sigma^{Y,1}_t\|^2+\|\sigma^{Y,2}_t\|^2) +V'_t \\ & \          + \frac{\gamma\|\lambda_t-\gamma \sigma^{Y,1}_t\|^2}{2(1-\gamma)}     \Big)  - X_tZ_t\Big(  \gamma (\|\sigma^{Y,1}_t\|^2 + \|\sigma^{Y,2}_t\|^2 -\mu^Y_t) + \gamma \alpha_t + 2V'_t   \\ 
%& \ + \gamma(1-\gamma)(\beta^\top_t\sigma^{Y,1}_t-\|\sigma^{Y,2}_t\|^2)   +\frac{\gamma (\lambda_t-\gamma \sigma^{Y,1}_t)^\top( \lambda_t-\sigma^{Y,1}_t- (1-\gamma)\beta_t )}{1-\gamma}       \Big)  \\ &  \ +   Z_t^2\Big( \gamma\alpha_t+ \frac{\gamma(\gamma-1)(\|\beta_t\|^2+\|\sigma^{Y,2}_t\|^2)}{2} + \frac{\gamma\|\lambda_t-\sigma^{Y,1}_t - (1-\gamma)\beta_t\|^2}{2(1-\gamma)} + V'_t \Big)   \Bigg)  dt \\ 
%& \   + \frac{\gamma}{(X_t-Z_t)}\left(  X_t(\pi_t^\top\Sigma_t-(\sigma^{Y,1}_t)^\top) - \beta^\top_tZ_t\right)d{\bf B}^1_t      - \gamma(\sigma^{Y,2}_t)^\top d{\bf B}^2_t.
%\end{align*}
%Using \eqref{eq:power:alpha:beta} and \eqref{eq:V:power:2}, we arrive at, for any $t\geq 0$,  
%\begin{equation*}
%    \begin{aligned}
%  \frac{dU(X_t,t)}{U(X_t,t)} 
%= &  -  \frac{\gamma(1-\gamma)X_t^2}{2(X_t-Z_t)^2}\left\| \Sigma^\top_t\pi_t - \frac{\lambda_t-\gamma\sigma^{Y,1}_t + {\color{red} \theta^1_t}}{1-\gamma} + \frac{  Z_t(\lambda_t-\sigma^{Y,1}_t-(1-\gamma)\beta_t + {\color{red}  \theta_t^1})}{(1-\gamma)X_t} \right\|^2dt \\ & \ +  \left( {\color{red}(\theta_t^1)^\top} + \frac{\gamma}{(X_t-Z_t)}\left(  X_t(\pi_t^\top\Sigma_t-(\sigma^{Y,1}_t)^\top) - \beta^\top_tZ_t\right) \right)d{\bf B}^1_t   \\
%& \ + \left({\color{red} (\theta_t^2)^\top  } -\gamma(\sigma^{Y,2}_t)^\top\right) d{\bf B}^2_t.
%    \end{aligned}
%\end{equation*}
%\begin{equation*}
%    \begin{aligned}
%\frac{dU(X_t,t)}{U(X_t,t)} = & -\frac{\gamma(1-\gamma)X_t^2}{2(X_t-Z_t)^2}  \left\| \Sigma^\top_t\pi_t - \frac{(\lambda_t-\gamma\sigma^{Y,1}_t)X_t - Z_t(\lambda_t-\sigma^{Y,1}_t-\beta_t(1-\gamma))}{(1-\gamma)X_t} \right\|^2 dt  \\ & \  + \frac{\gamma}{(X_t-Z_t)}\left(  X_t(\pi_t^\top\Sigma_t-(\sigma^{Y,1}_t)^\top) - \beta^\top_tZ_t\right)d{\bf B}^1_t      -\gamma (\sigma^{Y,2}_t)^\top d{\bf B}^2_t.
%    \end{aligned}
%\end{equation*}
%Hence, for any $0\leq s\leq t$, we have
%\begin{equation*}
%\begin{aligned}
%U(X_t,t) =& U(X_s,s) \exp\Bigg(   \int_s^t \frac{\gamma(\gamma-1)X_l^2}{2(X_l-Z_l)^2}  \bigg\| \Sigma^\top_l\pi_l- \frac{\lambda_l-\gamma\sigma^{Y,1}_l + {\color{red} \theta^1_l}}{1-\gamma}  \\ & \ +  \frac{Z_l(\lambda_l-\sigma^{Y,1}_l-(1-\gamma)\beta_l + {\color{red} \theta^1_l})}{(1-\gamma)X_l} \bigg\|^2dl   \Bigg) \mathcal{E}_{s,t},
%\end{aligned}
%\end{equation*}
%where  the process $\{\mathcal{E}_{s,t}\}_{t\geq s}$ is governed by, for any $t\geq s$,
%\begin{align*}
%d\mathcal{E}_{s,t} = & \  \mathcal{E}_{s,t}\left( {\color{red}(\theta_t^1)^\top} + \frac{\gamma }{(X_t-Z_t)}\left(  X_t(\pi_t^\top\Sigma_t-(\sigma^{Y,1}_t)^\top)  - \beta^\top_tZ_t\right) \right)d{\bf B}^1_t \\      & \ \quad  +\mathcal{E}_{s,t}\left( (\theta_t^2)^\top -\gamma (\sigma^{Y,2}_t)^\top\right) d{\bf B}^2_t, \ \mathcal{E}_{s,s}=1.
%\end{align*}
%We show that for $t\geq 0$,
%\begin{equation}
%\label{eq:power:P}
%\int_0^t \left\| {\color{red}(\theta_s^1)^\top } + \frac{X_s(\pi_s^\top \Sigma_s - (\sigma^{Y,1}_s)^\top)-\beta^\top_sZ_s  )}{X_s-Z_s} \right\|^2 ds < \infty  \  \mathbb{P}\text{-a.s.}.
%\end{equation}
%To proceed, we first notice that by  \eqref{eq:Z:power} and Lemma \ref{lemma:sde}, $Z \in \mathcal{L}^2_1$.  Next, we show that $X\in \mathcal{L}^2_1$.  Given $X\pi \in \mathcal{L}^2_n$, consider the process $\bar{X}=\{ \bar{X}_t \}_{t\geq 0}$ which satisfies the linear SDE
%\begin{align*}
%d\bar{X}_t =& \left(p_t + X_t\pi^\top_t \Sigma_t(\lambda_t - \sigma^{Y,1}_t)\right)dt  + X_t \pi^\top_t\Sigma_t d{\bf B}^1_t   \\ &\ +\bar{X}_t \left( \left(   - \mu^Y_t + \|\sigma^{Y,1}_t\|^2 + \|\sigma^{Y,2}_t\|^2 \right)  dt     -(\sigma^{Y,1}_t)^\top d{\bf B}^1_t - (\sigma^{Y,2}_t)^\top d{\bf B}^2_t \right), \bar{X}_0 = x_0.
%\end{align*}
%By Lemma \ref{lemma:sde}, we have $\bar{X} \in \mathcal{L}^2_1$. Notice from \eqref{eq:X} that $\bar{X}-X$ satisfies $\bar{X}_0-X_0=0$ and
%\begin{equation*}
%d(\bar{X}_t-X_t) = (\bar{X}_t-X_t) \left( \left(   - \mu^Y_t + \|\sigma^{Y,1}_t\|^2 + \|\sigma^{Y,2}_t\|^2 \right)  dt   +  -(\sigma^{Y,1}_t)^\top d{\bf B}^1_t - (\sigma^{Y,2}_t)^\top d{\bf B}^2_t \right).
%\end{equation*}
%By the uniqueness of solution, thanks to Lemma \ref{lemma:sde}, we infer that $X =\bar{X} \in \mathcal{L}^2_1$. Now, since $\pi \in \mathcal{A}$ and the sample paths of $X-Z$ is $\mathbb{P}$-a.s. continuous, we have $\sup_{s\in[0,t]} |X_s-Z_s|^{-1} < \infty$ for a.a. $(t,\omega) \in [0,\infty)\times\Omega$, whence 
%\begin{align*}
%& \ \ \ \int_0^t \left\| \frac{X_s(\pi_s^\top \Sigma_s - (\sigma^{Y,1}_s)^\top)-\beta^\top_sZ_s  )}{X_s-Z_s} \right\|^2 ds \\
%&\leq 2\sup_{s\in[0,t]} |X_s-Z_s|^{-2}  \int_0^t \big( \|\theta^1_s\|^2 + \|\beta_s\|^2 |Z_s|^2 +  2(\|X_s \pi\|^2 + \|\sigma^{Y,1}_s\|^2 |X_s|^2) \big)ds.
%\end{align*}
%Using the boundedness of $\theta^1$, $\beta$ and $\sigma^{Y,1}$, alongside with the fact that $X\pi\in \mathcal{L}^2_n$, $X,Z\in \mathcal{L}^2_1$,
%we arrive at \eqref{eq:power:P}. Consequently, the Dol\'eans-Dade exponential $\{\mathcal{E}_{s,t}\}_{t\geq s}$ is a local martingale bounded from below by 0, which is therefore a super-martingale. With $\gamma\in (0,1)$, we infer that     \begin{equation*}
%\mathbb{E}[U(X_t,t)|\mathcal{F}_s] \leq  U(X_s,s) \mathbb{E}[\mathcal{E}_{s,t}|\mathcal{F}_s] \leq U(X_s,s),
%\end{equation*}
%whence $\{U(X_t,t)\}_{t\geq 0}$ is a super-martingale. Substituting $\pi^*$ into \eqref{eq:X}, we arrive at \eqref{eq:power:X}. In addition, for any $0\leq s\leq t$,
%\begin{equation*}
%U(X_t^*,t) = U(X_s^*,s) \mathcal{E}_{s,t}^*,
%\end{equation*}
%where for $t\geq s\geq 0$, $\{\mathcal{E}_{s,t}^*\}_{t\geq s}$ is the solution to 
%\begin{equation*}
%d\mathcal{E}_{s,t}^* = \mathcal{E}_{s,t}^*\left(  \frac{\left(\gamma(\lambda_t - \sigma^{Y,1}_t ) +  \theta_t^1\right)^\top}{1-\gamma} d{\bf B}^1_t + \left( \theta_t^2 - \gamma \sigma^{Y,2}_t   \right)^\top d{\bf B}^2_t\right), \ \mathcal{E}_{s,s}=1,
%\end{equation*}
%which is clearly a martingale by the boundedness of $\lambda$, $\sigma^{Y,1}$, $\sigma^{Y,2}$, $\theta^1$ and $\theta^2$. Therefore, $\{U(X_t^*,t)\}_{t\geq 0}$ is  a martingale. To show that $\pi^*\in \mathcal{A}$, notice that 
 %   \begin{align*}
  %      \pi_t^*  %& \ (\Sigma^\top_t)^{-1}\left( \frac{\lambda_t-\gamma \sigma^{Y,1}_t + \theta^1_t}{1-\gamma} - \frac{Z_t(\lambda_t-\sigma^{Y,1}_t -  (1-\gamma)\beta_t + \theta^1_t) }{(1-\gamma)X^*_t}\right) \\
   %     =& \  \frac{X^{\hat{\pi},0 }_t}{X^*_t} \hat{\pi}_t + \left( 1- \frac{X^{\hat{\pi},0 }_t}{X^*_t}\right)\left( \frac{(\Sigma_t^\top)^{-1}\left(\lambda_t-\gamma\sigma^{Y,1}_t + \theta^1_t\right)}{1-\gamma} \right).
%    \end{align*}
%Hence, with $\xi := \frac{\lambda-\sigma^{Y,1}+\theta^1}{1-\gamma}$, we see that $\pi^*\in \tilde{\mathcal{A}} \subset \mathcal{A}$ by Proposition \ref{pp:admissible}. 
%%%% the original proof for \pi \in \mathcal{A} is hidden
%\iffalse 
%It remains to show that $\pi^*\in\mathcal{A}$.  By \eqref{eq:Z:power}, \eqref{eq:pi*:power:1} and \eqref{eq:power:X}, one has
%\begin{align}
%\label{eq:power:X-Z}
%d(X_t^* - Z_t) %&= \Bigg(   \left(  -\frac{(\lambda_t-\sigma^{Y,1}_t)^\top(\lambda_t-\gamma\sigma^{Y,1}_t)}{\gamma-1} + \|\sigma^{Y,1}_t\|^2+\|\sigma^{Y,2}_t\|^2 -\mu^Y_t \right)    X_t^* + \\
%&  \left(  \alpha_t - \frac{(\lambda_t-\sigma^{Y,1}_t)^\top(\lambda_t-\sigma^{Y,1}_t+(\gamma-1)\beta_t }{\gamma-1} \right)Z_t  \Bigg) dt  - (X_t^*+Z_t)\left( \frac{\lambda^\top_t-(\sigma^{Y,1}_t)^\top}{\gamma-1}\right) d{\bf B}^1_t\\ 
%&=  ( X_t^* -Z_t) \Bigg(   \Big(  -\frac{(\lambda_t-\sigma^{Y,1}_t)^\top(\lambda_t-\gamma\sigma^{Y,1}_t)}{\gamma-1} + \|\sigma^{Y,1}_t\|^2+\|\sigma^{Y,2}_t\|^2   -\mu^Y_t \Big) dt  -  \nonumber \\ &\hspace{1cm}  \left( \frac{\lambda^\top_t-(\sigma^{Y,1}_t)^\top}{\gamma-1}\right) d{\bf B}^1_t - (\sigma^{Y,2}_t)^\top d{\bf B}^2_t\Bigg), \ X_0^*-Z_0=x_0.
%\end{align}
%It is clear from Lemma \ref{lemma:sde} that $X^*-Z \in \mathcal{L}^2_1$. In addition, upon solving \eqref{eq:power:X-Z}, one can easily obtain  
%\begin{equation*}
%(X_t^*-Z_t)^{-1} = x_0^{-1} e^{ \int_0^s \left( \frac{(\lambda_s-\sigma^{Y,1}_s)^\top(\lambda_s-\gamma\sigma^{Y,1}_s)}{\gamma-1} + \|\sigma^{Y,1}_s\|^2+\|\sigma^{Y,2}_s\|^2   -\mu^Y_s\right)  ds}(\mathcal{E}^*_{0,t})^{-1}.
%\end{equation*}
%By Doob's maximal inequality, for every $t\geq 0$, one has $\sup_{s\in[0,t]} (\mathcal{E}_{0,t}^*)^{-1} < \infty$ $\mathbb{P}$-a.s., whence $\sup_{s\in[0,t]}|X^*_s-Z_s|^{-1} < \infty$ $\mathbb{P}$-a.s. In particular, we have  $X^*-Z>0$  for a.a. $(t,\omega)\in [0,\infty)\times \Omega$. On the other hand, for any $t\geq 0$, 
%\begin{equation*}
%X_t^*\pi^*_t =  - (\Sigma^\top_t)^{-1}\left( \frac{\lambda_t-\gamma \sigma^{Y,1}_t}{\gamma-1}(X_t^*-Z_t) -  (\sigma^{Y,1}_t+\beta_t)Z_t  \right).
%\end{equation*}
%Therefore, by the proven facts $X^*-Z,Z\in \mathcal{L}^2_1$, we have $X^*\pi^*\in \mathcal{L}^2_n$. 
%\fi 
%By a plain application of It\^o's lemma on $\tilde{U}(\tilde{x},t):=U(\tilde{x}+Z_t,t)$, we see that for any $\tilde{x}>0$ and $t\geq 0$, 
 %   \begin{equation*}
 %       d\tilde{U}(\tilde{x},t) = \tilde{U}(\tilde{x},t) \left(  \left( v_t + \frac{\|\theta^1_t\|^2+\|\theta^2_t\|^2}{2}\right)dt + (\theta^1_t)^\top d{\bf B}^1_t +  (\theta^2_t)^\top d{\bf B}^1_t \right),
 %   \end{equation*}
%whence $\tilde{U}$ is the solution to the SPDE \eqref{eq:SPDE} with volatility processes specified in \eqref{eq:vol:power}. 


\end{proof} 

The worker's power forward utility preference in Theorem \ref{pp:power} depends on her exogenous (stochastic) inputs. The process $\beta$ models the sensitivity of the baseline fund value to salary ratio with respect to the hedgeable risk ${\bf B}^1$ in the financial market. The worker observes by how much more her fund value to salary ratio exceeding the chosen baseline ratio, or equivalently, by how much more her absolute fund value exceeding the baseline absolute fund value, using her salary as the num{\'e}raire. The worker's forward preference is then given by this surplus being evaluated by the classical power utility function, together with multiplying a homothetic factor. The worker also exogenously chooses the volatility processes $\theta^1$ and $\theta^2$, with respect to the hedgeable risk ${\bf B}^1$ in the financial market and the non-hedgeable risk ${\bf B}^2$ in the labour market, governing the evolution of the homothetic factor to her preference. In particular, if the volatility processes are both exogenously chosen to be zero, the worker's translated forward preference is also of zero volatility; in this case, the time monotonicity of her translated forward preference depends on the sign of the deterministic $v_{\cdot}$ (with $\theta^1\equiv\theta^2\equiv 0$).\\

The worker's optimal investment strategy, based on her power forward utility preference, as outlined in Theorem \ref{pp:power}, is clearly a convex combination of the baseline investment strategy $\hat{\pi}$ and a myopic strategy $(\Sigma^\top_t)^{-1}\frac{\lambda_t-\gamma\sigma^{Y,1}_t + \theta^1_t}{1-\gamma}$, for $t\geq 0$. This myopic strategy is in fact consistent with the optimal investment strategy in the classical backward setting of \cite{CAIRNS2006843:stochastic:lifestyle}, in the case of $p_{\cdot}\equiv 0$ therein (see Cases 1 and 2), as well as $\theta^1\equiv 0$ endogenously from the backward value function. Unlike the backward approach in \cite{CAIRNS2006843:stochastic:lifestyle}, where the worker's optimal investment strategy at a particular time depends on the projection of her salary contribution afterwards (see Cases 3 and 4 therein), her optimal investment strategy herein at that time, based on the forward approach, depends on her past salary contribution. 
% {\color{red} Unlike the backward approach in \cite{CAIRNS2006843:stochastic:lifestyle}, where the worker's optimal investment strategy depends on the ratio of the present value of the future salary contributed with the current fund value to salary ratio (see Cases 3 and 4 therein), the optimal investment strategy herein, based on the forward preference, depends on the convex combination coefficient $X^{\hat{\pi},0}_t/X^*_t$ for any $t\geq 0$, where $X^{\hat{\pi},0}_t$ consists of the accumulated contribution of salary from the past, along with the return based on the investment strategy $\hat{\pi}$.}
%the worker's optimal investment strategy based on her forward preference depends on the proportion of contribution $p_{\cdot}$ only indirectly via the fund value to salary ratios, of both baseline and optimality; in \cite{CAIRNS2006843:stochastic:lifestyle}, the optimal investment strategy based on the worker's backward preference also directly depends on the proportion of contribution $p_{\cdot}$ (see Cases 3 and 4 therein).}
Moreover, observe that the convex combination coefficient is the quotient of the two ratios for the fund values (of baseline and optimality) to salary, which is actually the ratio of the baseline and optimal absolute fund values. Therefore, the worker's optimal investment strategy does not directly depend on the salary value, but it depends implicitly from the absolute fund value dynamics and the coefficient $\sigma^{Y,1}$ of the salary value's dynamics. However, the worker's forward preference indeed evaluates her fund value to salary ratio, in order for the (super-)martingale conditions to be satisfied. Since the ratio of the baseline and optimal absolute fund values is the convex combination coefficient, her optimal investment strategy tends to last in either one of the baseline and optimal investment strategies, unless the other one outperforms the lasting one substantially at a future time. For example, when the baseline and optimal absolute fund values are close, the worker's optimal investment strategy leans towards the baseline strategy, and vice versa; see also the numerical example in the next section.





% \subsection{Discussion of Results}
% \label{sec:power:discussion}
% % {\color{red}[ Included above } {\color{green}
% % The exogenous random factor $Z$ can be interpreted as the fund to salary ratio under an exogenous baseline strategy $\hat{\pi}=\{\hat{\pi}_t\}_{t\geq 0}$, which is given by   
% % \begin{equation}
% % \label{eq:base:power}
% % \hat{\pi}_t :=   (\Sigma^\top_t)^{-1}( \sigma^{Y,1}_t+\beta_t).
% % \end{equation}
% % Indeed, by \eqref{eq:power:alpha:beta} and \eqref{eq:base:power} , one can rewrite  the dynamics of  process $\hat{X}:=Z$ as
% % \begin{align}
% % \label{eq:hat:X}
% % d\hat{X}_t &= \left(p_t + \hat{X}_t \left( \hat{\pi}^\top_t\Sigma_t (\lambda_t-\sigma^{Y,1}_t) - \mu^Y_t + \|\sigma^{Y,1}_t\|^2 + \|\sigma^{Y,2}_t\|^2 \right) \right)dt+ \nonumber   \\ & \hspace{2cm} \hat{X}_t\left( (\hat{\pi}_t^\top\Sigma_t-(\sigma^{Y,1}_t)^\top) d{\bf B}^1_t - (\sigma^{Y,2}_t)^\top d{\bf B}^2_t \right), \ \hat{X}_0 = 0,
% % \end{align}
% % which coincides with \eqref{eq:X} with $\pi=\hat{\pi} $ and $x=0$. Hence, a strategy is admissible if it always outperforms $\hat{\pi}$ with zero initial wealth:
% % \begin{equation*}
% % \mathcal{A} =  \{ \pi  : X\pi\in \mathcal{L}^2_n, \   X^\pi_t> \hat{X}_t  \text{ for a.a. } (t,\omega)\in[0,\infty)\times \Omega  \}.
% % \end{equation*}
% % To see that the set of admissible strategies $\mathcal{A}$ is sufficiently rich, we let $\xi=\{\xi_t\}_{t\geq 0} \in \mathcal{P}(\mathbb{F})$ and consider a strategy of the form
% % \begin{equation}
% % \label{eq:pi:transform}
% % \pi_t := \frac{\hat{X}_t}{X_t} \hat{\pi}_t + \left(1-\frac{\hat{X}_t}{X_t} \right) (\Sigma^\top_t)^{-1} (\sigma^{Y,1}_t+\xi_t).
% % \end{equation}
% % Let $\Tilde{X}_t := X_t-\hat{X}_t$. Using \eqref{eq:pi:transform}, we have
% % \begin{align*}
% % dX_t =& \ \Big( p_t + X_t(-\mu^Y_t+\|\sigma^{Y,1}_t\|^2+\|\sigma^{Y,2}_t\|^2) + \Tilde{X}_t(\lambda_t-\sigma^{Y,1}_t)^\top( \xi_t + \sigma^{Y,1}_t) \\ 
% % & \ \quad + \hat{X}_t\hat{\pi}_t^\top\Sigma_t(\lambda_t-\sigma^{Y,1}_t) \Big)dt  + \left( \hat{X}_t(\hat{\pi}_t^\top\Sigma_t-(\sigma^{Y,1}_t)^\top) + \Tilde{X}_t\xi^\top_t \right)d{\bf B}^1_t- X_t(\sigma^{Y,2}_t)^\top d{\bf B}_t^2,
% % \end{align*}
% % whence
% % \begin{equation}
% % \begin{aligned}
% % d\Tilde{X}_t = & \ \Tilde{X}_t \Big(  \left(  \xi^\top_t(\lambda_t-\sigma^{Y,1}_t) - \mu^Y_t + \lambda^\top_t\sigma^{Y,1}_t+\|\sigma^{Y,2}_t\|^2  \right)  dt  \\ & \ \quad+ \xi_t^\top d{\bf B}^1_t  -  (\sigma^{Y,2}_t)^\top d{\bf B}^2_t\Big), \ \tilde{X}_0 = x_0.
% % \end{aligned}
% % \label{eq:tilde:X}
% % \end{equation}
% % The solution $\tilde{X}$ is clearly positive under Assumption
% % \ref{ass} and mild boundedness condition on $\xi$.   In particular, we have that $\tilde{A} \subset \mathcal{A} $, where 
% % \begin{equation*}
% % \tilde{A} := \{ \pi : \pi \text{ takes the form of \eqref{eq:pi:transform} with } \xi\in\mathcal{L}^2_n  \}.
% % \end{equation*}

% % Next, we rewrite the optimal investment strategy as 
% % \begin{align}
% % \label{eq:power:pi*:decompose}
% % \pi_t^* &= -\frac{(\Sigma^\top_t)^{-1}( \lambda_t-\gamma\sigma^{Y,1}_t)}{\gamma-1} - \frac{\hat{X}_t}{X_t^*}\left( (\Sigma^\top_t)^{-1}\left( \sigma^{Y,1}_t - \frac{\lambda_t-\sigma^{Y,1}_t}{\gamma-1}\right) - \hat{\pi}_t \right) \nonumber \\
% % %&= -\frac{(\Sigma^\top_t)^{-1}( \lambda_t-\gamma\sigma^{Y,1}_t)}{\gamma-1}  + \frac{\hat{X}_t}{X_t^*}\left( \hat{\pi}_t + \frac{\lambda-\gamma\sigma_{Y_1}}{\sigma_1(\gamma-1)}    \right) \nonumber \\
% % &= \left(1 -\frac{ \hat{X}_t}{X_t^*}   \right)\left(\frac{(\Sigma^\top_t)^{-1}( \lambda_t-\gamma\sigma^{Y,1}_t)}{1-\gamma}  \right)  + \frac{ \hat{X}_t}{X_t^*}  \hat{\pi}_t,
% % \end{align}}
% % {\color{red} End of Included above] }




% From \eqref{eq:pi*:power:1}, we see that $\pi^*\in\mathcal{\tilde{A}}$   with $\xi =\frac{\lambda-\sigma^{Y,1}+\theta^1}{1-\gamma}$. Hence, the optimal investment strategy at time $t$ is the weighted sum of the myopic strategy $\frac{(\Sigma^\top_t)^{-1}( \lambda_t-\gamma\sigma^{Y,1}_t + \theta^1_t)}{1-\gamma} $ and the baseline strategy $\hat{\pi}_t$,  where in the case $\theta^1\equiv {\bf0}$, $\sigma^{Y,2}\equiv {\bf0}$ and $p\equiv 0$, the myopic strategy is essentially the optimal investment strategy in the backward setting \cite{CAIRNS2006843:stochastic:lifestyle}.  Intutively, the objective of the worker is to maximize the surplus over the exogenous fund under the strategy $\hat{\pi}$, which is therefore natural to consider a strategy which is composed of both $\hat{\pi}$ and the classical optimal strategy $\frac{(\Sigma^\top_t)^{-1}( \lambda_t-\gamma\sigma^{Y,1}_t+\theta^1_t)}{1-\gamma}$.  Evidently, a worker would adhere to $\hat{\pi}$ if it is performing well, i.e., the ratio $X^{\hat{\pi},0}/X^*$ is high; otherwise one should invest mostly according to the myopic strategy. Notice that the optimal investment strategy can be rewritten as, for any $t\geq 0$,  
%     \begin{equation*}
%         \pi_t^*  =   \frac{W^{\hat{\pi},0 }_t}{W^*_t} \hat{\pi}_t + \left( 1- \frac{W^{\hat{\pi},0 }_t}{W^*_t}\right)\left( \frac{(\Sigma_t^\top)^{-1}\left(\lambda_t-\gamma\sigma^{Y,1}_t + \theta^1_t\right)}{1-\gamma} \right),
%     \end{equation*}
% where $W^*=\{W^*_t\}_{t\geq 0}$ and $W^{\hat{\pi},0}=\{W^{\hat{\pi},0}_t\}_{t\geq 0}$ are respectively, the optimal absolute fund value under $\pi^*$ and the absolute fund value under $\hat{\pi}$ with zero initial wealth. Therefore, despite that workers' utility are measured in terms of the ratio $X$, the performance of the absolute fund value is also taken into account when making decisions. This balances the utility of maintaining the living standard as weighted by $Y$, along with the wealth $W$ a worker actually has.    \\

%  The functions $\theta^1$, $\theta^2$ and $\beta$ are exogenously chosen by the worker which reflects her appetite to the changing market information and  labour conditions. Due to the presence of the stochastic inputs $V$ and $X^{\hat{\pi},0}$, one is not able to discuss the time monotonicity of the utility $\{U(x,t;\omega)\}_{t\geq 0}$, where $x > X^{\hat{\pi},0}_t(\omega)$ for all $t>0$. Nevertheless, in the case where $\theta^1$ and $\theta^2$ are taken to be zero functions, the translated utility $\tilde{U}$ for the difference $\tilde{X}$ defined in \eqref{eq:tilde:X} satisfies the SPDE \eqref{eq:SPDE} with zero volatility processes. The time monotonicity is then governed by the sign of $v_t$. For instance, for large values of $\mu^Y$  and small values of $\|\sigma^{Y,1}\|$ and $\|\sigma^{Y,2}\|$ , the utility process tends to increases with time; and  decreases with time otherwise. Therefore, the forward utility herein can capture workers' aversion of risk in the labour market. 

\subsection{Numerical Example}
\label{sec:power:numeric}
This section further illustrates the worker's power forward utility preference and optimal investment strategy in Theorem \ref{pp:power} via a numerical example. Consider that $n=m=1$, and denote the independent one-dimensional Brownian motions ${\bf B^1}$ and ${\bf B^2}$ simply as $B^1$ and $B^2$. Assume that all financial market and salary parameters are constant over time, with $r=0.03$, $\mu^{1}_{\cdot}=0.12$, $\Sigma_{\cdot}=\sigma^{11}_{\cdot}=0.2$ (and thus, $\lambda_{\cdot}=0.6$), $\mu^Y_{\cdot}=0.1$, $\sigma^{Y,1}_{\cdot}=0.15$, and $\sigma^{Y,2}_{\cdot}=0.2$. The worker's initial fund value to salary ratio is set as $x_0=0.5$, while she contributes a constant proportion $p_{\cdot}=0.1$ throughout the accumulation period. Her constant relative risk aversion parameter is given by $\gamma=0.6$, and she chooses $\theta^1_{\cdot}=0$ and $\theta^2_{\cdot}=0.1$ as the constant parameters for the volatility processes of her translated forward utility preferences.\\

This example particularly sheds light on the stochasticity of her forward utility preference and optimal investment strategy. To this end, consider four scenarios $\omega_1,\omega_2,\omega_3,\omega_4$, which yield the following sample paths of the Brownian motions: for any $t\geq 0$,
\begin{align*}
\left(B^1_t,B^2_t\right)\left(\omega_1\right)=&\;\left(b^{1,U}_t,b^0_t\right),\\
\left(B^1_t,B^2_t\right)\left(\omega_2\right)=&\;\left(b^{1,S}_t,b^0_t\right),\\
\left(B^1_t,B^2_t\right)\left(\omega_3\right)=&\;\left(b^0_t,b^{2,U}_t\right),\\
\left(B^1_t,B^2_t\right)\left(\omega_4\right)=&\;\left(b^0_t,b^{2,D}_t\right),
\end{align*}
where $b^{1,U}_{\cdot},b^{1,S}_{\cdot},b^{2,U}_{\cdot},b^{2,D}_{\cdot},b^{0}_{\cdot}$ are depicted in Figure \ref{fig:B} (in which $t=10$ is a time before the accumulation period ends). Under scenarios $\omega_1$ and $\omega_2$, the non-hedgeable risk $B^2$ evolves moderately around zero as in Figure \ref{fig:B3}, and the hedgeable risk $B^1$ moves up in the first scenario while it remains relatively stable in the second scenario (see Figure \ref{fig:B1}). Similar interpretations under the scenarios $\omega_3$ and $\omega_4$, with $B^1$ and $B^2$ as in Figures \ref{fig:B3} and \ref{fig:B2} respectively.\\

\begin{figure}[!h]
\centering
\begin{subfigure}{.33\textwidth}
\centering
\includegraphics[scale=0.3]{B1.png}
\caption{Two realizations of $B^1$}
\label{fig:B1}
\end{subfigure}%
\begin{subfigure}{.33\textwidth}
\centering
\includegraphics[scale=0.3]{B2.png}
\caption{Two realizations of $B^2$}
\label{fig:B2}
\end{subfigure}%
\begin{subfigure}{.34\textwidth}
\centering
\includegraphics[scale=0.3]{B0.png}
\caption{Sample path $b^0_{\cdot}$}
\label{fig:B3}
\end{subfigure}%
\caption{Realizations of Brownian motions}
\label{fig:B}
\end{figure}

This example also considers two baseline fund value to salary ratios, with $\beta\equiv 0.5$ and $\beta\equiv -0.5$ respectively for the sensitivity of the subsistence level with respect to the hedgeable risk $B^1$, and studies their influences on the worker's forward utility preference and optimal investment strategy.




%%%% B^1_t(\omega_1), control path -> sth else; 4 scenarios, 4 ?omegas? 
%sth else? B^0 to another color. 
% \beta choice put to the parameter choice paragraph/ 
% recall the meaning of v in Theorem (13)
% utility: B^1: \beta=-0.5, \hat{\pi} < 0=>short selling => reduced in B^{1,U} and vice versa. 
% level of utility delete. 
% compare x intercepts, which represents \hat{X}
% B^2, not directly comparable with backward, rather it's on the method of construction of the utility. 
% Figure 4, change title, difference of U(5,t) under scenario 1 and 2
% Figure 4, flip the difference U(5,t;\omega_U) - U(5,t,\omega_D). <- add legend stating which - which, delete explanation -> change color 
% plot also difference of U1? 
% \pi under scenario \omega_1....... (title), change legends U-> S; \pi^*(\omega_1) \pi^*(\omega_2).  Change styling of lines (not repeat for myopic and baseline). e.g. deeper green 

% We consider $n=m=1$ so that $B^1=\{B^1_t\}_{t\geq 0}$ and $B^2=\{B^2_t\}_{t\geq 0}$ are both one-dimensional Brownian motions, and the parameters are assumed to be constant over time.  Unless otherwise specified, we shall use the following parameters : $r=0.03$, $p = 0.1$, $\mu^Y=0.1$, $\sigma_1:=\Sigma=0.2$, $\sigma^{Y,1}=0.15$, $\sigma^{Y,2}=0.2$, $\lambda=0.6$, $x_0=0.5$, $\gamma=0.6$, $\theta^1=0$ and $\theta^2=0.1$. We study the optimal investment strategy and the evolution of the utility process for $0\leq t\leq 10$. We also consider two different choices $\beta=0.5$ and $\beta=-0.5$. The choices correspond to different sensitivity of the baseline fund value to salary ratio $X^{\hat{\pi},0}$ with strategy $\hat{\pi}$ towards the risk $B^1$.   In particular, $\hat{\pi}<0$ when $\beta=-0.5$; and $\hat{\pi}>0$ when $\beta=0.5$.
  % \\



% To consider separately the effect of the hedgeable risk from the financial market and the non-hedgeable risk from the labour market, we simulate two paths for each of the Brownian motions $B^1$ and $B^2$ (see Figure \ref{fig:B}) and consider the realizations of the associated utility  $U$ and  optimal investment strategies.  In Figure \ref{fig:B1}, the orange path $B^{1,U}$ represents a scenario with an upward moving $B^1$, while the blue path $B^{1,S}$ represents a relatively stable $B^1$ with an ultimate downward movement. In Figure \ref{fig:B2}, the two paths $B^{2,U}$ and $B^{2,D}$ represent an upward (orange) and downward (blue) movements on the salary bought by the non-hedgeable risk $B^2$, respectively.  We also simulate an additional path $B^0$ as a control in Figure \ref{fig:B3}. To compare the difference between the two market (resp. salary) scenarios,  we simulate the utility processes and optimal investment strategies by taking $B^1=B^{1,U},B^{1,S}$ (resp. $B^2=B^{2,U},B^{2,D}$) while fixing $B^2=B^0$ (resp. $B^1=B^0$).

% $\omega_1,\omega_2,\omega_3,\omega_4$.

% $\left(B^1_t,B^2_t\right)\left(\omega_1\right)=\left(b^{1,U}_t,b^{0}_t\right)$

% $\left(B^1_t,B^2_t\right)\left(\omega_2\right)=\left(b^{1,S}_t,b^{0}_t\right)$

% $\left(B^1_t,B^2_t\right)\left(\omega_3\right)=\left(b^{0}_t,b^{2,U}_t\right)$

% $\left(B^1_t,B^2_t\right)\left(\omega_4\right)=\left(b^{0}_t,b^{2,D}_t\right)$\\

\subsubsection{Forward Utility Preferences}
The worker's forward utility preferences are summarized in Figures \ref{fig:power:utility:market} and \ref{fig:power:utility:salary}. Her preferences $U\left(\cdot,t;\omega\right)$ are pictured as functions of the ratio argument $x\in\mathbb{R}$, for any fixed scenario $\omega\in\Omega$ and time $t\geq 0$. Their time dependence are illustrated by various colors highlighting the respective functions. Figure \ref{fig:power:utility:market} depicts the scenarios $\omega_1$ and $\omega_2$, while Figure \ref{fig:power:utility:salary} illustrates the scenarios $\omega_3$ and $\omega_4$.\\

\begin{figure}[!h]
\centering
\begin{subfigure}{.5\textwidth}
\centering
\includegraphics[scale=0.4]{Power_U_UM_bm05.png}
\caption{$\omega=\omega_1$, $\beta\equiv -0.5$}
\label{fig:power:U:UM:B-05}
\end{subfigure}%
\begin{subfigure}{.5\textwidth}
\centering
\includegraphics[scale=0.4]{Power_U_DM_bm05.png}
\caption{$\omega=\omega_2$, $\beta\equiv -0.5$}
\label{fig:power:U:DM:B-05}
\end{subfigure}%


\begin{subfigure}{.5\textwidth}
\centering
\includegraphics[scale=0.4]{Power_U_UM_b05.png}
\caption{$\omega=\omega_1$, $\beta\equiv 0.5$}
\label{fig:power:U:UM:B05}
\end{subfigure}%
\begin{subfigure}{.5\textwidth}
\centering
\includegraphics[scale=0.4]{Power_U_DM_b05.png}
\caption{$\omega=\omega_2$, $\beta\equiv 0.5$}
\label{fig:power:U:DM:B05}
\end{subfigure}%


\caption{Realizations of worker's forward utility preferences under scenarios $\omega_1$ and $\omega_2$ and with respect to two different baseline performances}
\label{fig:power:utility:market}
\end{figure}

\begin{figure}[!h]
\centering
\begin{subfigure}{.5\textwidth}
\centering
\includegraphics[scale=0.4]{Power_U_US_bm05.png}
\caption{$\omega=\omega_3$, $\beta\equiv -0.5$}
\label{fig:power:U:US:B-05}
\end{subfigure}%
\begin{subfigure}{.5\textwidth}
\centering
\includegraphics[scale=0.4]{Power_U_DS_bm05.png}
\caption{$\omega=\omega_4$, $\beta\equiv -0.5$}
\label{fig:power:U:SS:B-05}
\end{subfigure}%

\begin{subfigure}{.5\textwidth}
\centering
\includegraphics[scale=0.4]{Power_U_US_b05.png}
\caption{$\omega=\omega_3$, $\beta\equiv 0.5$}
\label{fig:power:U:US:B05}
\end{subfigure}%
\begin{subfigure}{.5\textwidth}
\centering
\includegraphics[scale=0.4]{Power_U_DS_b05.png}
\caption{$\omega=\omega_4$, $\beta\equiv 0.5$}
\label{fig:power:U:SS:B05}
\end{subfigure}%



\caption{Realizations of worker's forward utility preferences under scenarios $\omega_3$ and $\omega_4$ and with respect to two different baseline performances}
\label{fig:power:utility:salary}
\end{figure} 

% The realized utility $U$ as a function of $(x,t)$ are plotted in Figures \ref{fig:power:utility:market} and \ref{fig:power:utility:salary} under different realizations of $B^1$ and $B^2$, respectively. The utility function $U(\cdot,t)$ at different time $t$ are indicated by the colors of the lines.  With the current choices of parameters, we have $v< 0$ whence the utility processes tend to decrease with time for large values of $x$. This monotonicity, however, could be opposed by the change of the baseline fund value to salary ratio $X^{\hat{\pi},0}$ under different market and salary conditions for smaller values of $x$.     \\
% large values of x, since the effect of V can outweigh the effect of \hat{X} in (x-\hat{X})^\gamma, if \hat{X} is moving in opposite direction. 

If the worker compares the baseline performance with $\beta\equiv -0.5$, contrasting between Figures \ref{fig:power:U:UM:B-05} and \ref{fig:power:U:DM:B-05}, her preference under the scenario $\omega_1$ is in general larger than that under the scenario $\omega_2$; see also Figure \ref{fig:power:DU:M:Bm05} for the preference value difference between the two scenarios fixing the ratio argument $x=5$. With the sensitivity of the baseline ratio with respect to the hedgeable risk $B^1$ being negative, when $B^1$ moves up in the first scenario, it draws down the baseline ratio. Therefore, the x-intercepts, which are the baseline ratio values, in Figure \ref{fig:power:U:UM:B-05} are smaller than those in Figure \ref{fig:power:U:DM:B-05}, and thus leading to her generally larger preference values in the first scenario. If the worker compares the baseline performance with $\beta\equiv 0.5$, the opposite holds between Figures \ref{fig:power:U:UM:B05} and \ref{fig:power:U:DM:B05} (see also Figure \ref{fig:power:DU:M:B05} for their difference), since the sensitivity of the baseline ratio with respect to the hedgeable risk $B^1$ is positive.\\

\begin{figure}[!h]
    \centering
    \begin{subfigure}{.5\textwidth}
    \centering
    \includegraphics[scale=0.4]{Power_DU_M_bm05.png}
    \caption{$\beta\equiv -0.5$}
    \label{fig:power:DU:M:Bm05}
    \end{subfigure}%
    \begin{subfigure}{.5\textwidth}
    \centering
    \includegraphics[scale=0.4]{Power_DU_M_b05.png}
    \caption{$\beta\equiv 0.5$}
    \label{fig:power:DU:M:B05}
    \end{subfigure}%
    \caption{Difference of realized worker's forward utility preferences, between scenarios $\omega_1$ and $\omega_2$, with respect to two different baseline performances, and at ratio argument $x=5$}
    \label{fig:power:du:M}
\end{figure}

From Figure \ref{fig:power:utility:salary}, regardless of which baseline performances the worker compares to, as the sensitivity of the baseline ratios with respect to the non-hedgeable risk $B^2$ is negative (which is given by $-\sigma^{Y,2}_{\cdot}=-0.2$), the baseline ratio values as the x-intercepts in the scenario $\omega_4$ (when the non-hedgeable risk $B^2$ moves down, c.f. Figures \ref{fig:power:U:SS:B-05} and \ref{fig:power:U:SS:B05}) are larger than those in the scenario $\omega_3$ (when the non-hedgeable risk $B^2$ moves up, c.f. Figures \ref{fig:power:U:US:B-05} and \ref{fig:power:U:US:B05}). Hence, the worker's preference values are in general larger in the third scenario; see also Figure \ref{fig:power:du} for the preference value differences when $x=5$.

% \textbf{Effect of Hedgeable Risk $B^1$}\\
% When $\beta=-0.5<0$, the growth of the process $X^{\hat{\pi},0}$ is depleted by an upward movement of $B^1$, whence $X^{\hat{\pi},0}$  tends to be smaller under $B^{1,U}$ than that of $B^{1,S}$. Thus, the utility process of the later case tends to be smaller. As a function of $x$, the utilities are shifted towards right more in Figure \ref{fig:power:U:DM:B-05} than in Figure \ref{fig:power:U:UM:B-05}; see also Figure \ref{fig:power:XZ:M}. Heuristically, the baseline strategy is betting in opposition to favourable market conditions. An upward trend of $B^1$ leads to a decline of the baseline fund value to salary ratio $X^{\hat{\pi},0}$ and workers are able to maintain the same level of utility with a smaller $X$.\\

% The opposite is observed for $\beta=0.5>0$, where $X^{\hat{\pi},0}$ experiences a significant growth under $B^{1,U}$. This leads to a lower level of utility compared to the case of $B^{1,S}$ and the curves are shifted  more to the right in Figure \ref{fig:power:U:DM:B05} than in Figure \ref{fig:power:U:DM:B05}. Therefore, workers are required to have a higher $X$ in order to maintain the same level of utility.   \\ 

%The realized optimal fund to salary ratio $X^*$ for both $\beta=-0.5$ and $\beta=0.5$ are remarkably greater under $B^{1,U}$. Indeed, as we shall see in the analysis of the optimal investment strategies, one can exploit the upward market and  invest myopically to boost up the fund value substantially.  \\

% \textbf{Effect of Non-hegeable Risk $B^2$}\\
% For both choices of $\beta$, the downward trend $B^{2,D}$  tends to increase $X^{\hat{\pi},0}$ , which in turn reduces the utility (c.f. the $x$-intercepts of Figure \ref{fig:power:U:SS:B-05} with Figure \ref{fig:power:U:US:B-05}; Figure \ref{fig:power:U:SS:B05} with Figure \ref{fig:power:U:US:B05}; and also Figure \ref{fig:power:du}). In the classical backward approach, in terms of maintaining the living standard with respect to the current salary, workers are more satisfactory as measured by the utility level with unfavourable labour conditions given a fixed fund value $W$, since a reduction of salary $Y$ directly boosts up the ratio $X=W/Y$. In the forward setting, however, a downward trend in $B^{2,D}$ leads to a rise of the subsistence level $X^{\hat{\pi},0}$. The choice $\theta^2>0$ also reflects a reduction of utility when unfavourable information in the labour market emerges. Hence, higher $X$ is required to maintain the same level of utility. In reality, workers are generally less satisfactory with unfavourable labour conditions. For instance, a worker who used to have a high fund value $W$ and receives a high salary $Y$, is less satisfactory after a decline of salary despite being able to maintain the same ratio $X$. Therefore, the forward utility herein mitigates the immediate rise of the level of utility due to a drop of salary, which  provides a more realistic model by balancing the ability of maintaining the living standard as measured by $X$, and the actual salary $Y$ they are receiving.   %Nevertheless, the difference of effect of $B^2$ on $X^*$ is less significant compared to that of $B^1$, for it being non-hedgeable using $\pi$.

% %\begin{figure}[!h]
% %   \centering
% %     \begin{subfigure}{.5\textwidth}
% %       \centering
% %       \includegraphics[scale=0.4]{Power_Z_M_bm05.png}
% %       \caption{$\hat{X}$, $\beta=-0.5$.}
% %       \label{fig:power:Z:M:B-05}
% %   \end{subfigure}%
% % \begin{subfigure}{.5\textwidth}
% %       \centering
% %       \includegraphics[scale=0.4]{Power_Z_M_b05.png}
% %       \caption{$\hat{X}$, $\beta=0.5$.}
% %       \label{fig:power:Z:M:B05}
% %   \end{subfigure}%
% %    \caption{Realizations of the exogenous fund $\hat{X}$ subject to different market conditions.   }
% %    \label{fig:power:z:market}
% %\end{figure} 




\begin{figure}[!h]
    \centering
    \begin{subfigure}{.5\textwidth}
    \centering
    \includegraphics[scale=0.4]{Power_DU_bm05.png}
    \caption{$\beta\equiv -0.5$}
    \label{fig:power:DU:Bm05}
    \end{subfigure}%
    \begin{subfigure}{.5\textwidth}
    \centering
    \includegraphics[scale=0.4]{Power_DU_b05.png}
    \caption{$\beta\equiv 0.5$}
    \label{fig:power:DU:B05}
    \end{subfigure}%
    \caption{Difference of realized worker's forward utility preferences, between scenarios $\omega_3$ and $\omega_4$, with respect to two different baseline performances, and at ratio argument $x=5$}
    \label{fig:power:du}
\end{figure}

%\begin{figure}[!h]
%   \centering
%     \begin{subfigure}{.5\textwidth}
%       \centering
%       \includegraphics[scale=0.4]{Power_Z_S_bm05.png}
%       \caption{$\hat{X}$, $\beta=-0.5$.}
%       \label{fig:power:Z:S:B-05}
%   \end{subfigure}%
% \begin{subfigure}{.5\textwidth}
%       \centering
%       \includegraphics[scale=0.4]{Power_Z_S_b05.png}
%       \caption{$\hat{X}$, $\beta=0.5$.}
%       \label{fig:power:Z:S:B05}
%   \end{subfigure}%
%    \caption{Realizations of the exogenous fund $\hat{X}$ subject to different salary conditions.   }
%   \label{fig:power:z:salary}
%\end{figure}  



\subsubsection{Optimal Investment Strategy}
The worker's optimal investment strategies are summarized in Figures \ref{fig:power:Pi:M} and \ref{fig:power:Pi:S}; Figure \ref{fig:power:Pi:M} illustrates the scenarios $\omega_1$ and $\omega_2$, while Figure \ref{fig:power:Pi:S} depicts the scenarios $\omega_3$ and $\omega_4$. Under the given parameters, the myopic strategy $(\Sigma^\top_{\cdot})^{-1}\frac{\lambda_{\cdot}-\gamma\sigma^{Y,1}_{\cdot} + \theta^1_{\cdot}}{1-\gamma}=6.375$; the exogenous baseline strategy $\hat{\pi}_{\cdot}\equiv -1.75<0$ when $\beta\equiv -0.5$, while $\hat{\pi}_{\cdot}\equiv 3.25>0$ when $\beta\equiv 0.5$. Recall that her optimal investment strategy is a convex combination of the baseline and myopic strategies, with the coefficient for the baseline strategy given by the quotient of two ratios for the fund values, respectively of baseline and of optimality, to salary. Since the initial baseline fund value is zero, her optimal strategy starts from the myopic one.\\

From Figure \ref{fig:power:Pi:M}, the worker's optimal investment strategy follows the myopic strategy closely when the hedgeable risk $B^1$ moves up in the first scenario, while she tends to follow the baseline investment strategy as time propagates when $B^1$ remains relatively stable in the second scenario, no matter which baseline performances she compares to. Under the scenario $\omega_1$, the upward movement of $B^1$ drives up the risky asset value as well as the salary. On one hand, by following the positive myopic strategy, the worker's pension fund value grows substantially more than her own salary, leading to a large ratio value; on the other hand, by the exogenous baseline strategy, the performance in terms of the fund value to salary ratio stays relatively moderate. Therefore, the difference between the two ratios widens as time progresses (see Figures \ref{fig:power:XZ:UM:b-05} and \ref{fig:power:XZ:UM:b05}), and thus the coefficient for the baseline strategy is negligible; this explains why the worker follows the myopic strategy in this case. Under the scenario $\omega_2$, relatively stable hedgeable risk $B^1$ does not enlarge the difference between the two ratios much, but instead, the ratios gradually evolve adjacent to each other moving forward in time (see Figures \ref{fig:power:XZ:DM:b-05} and \ref{fig:power:XZ:DM:b05}), and thus the coefficient for the myopic strategy eventually becomes negligible; these illustrate why the worker's optimal investment strategies diverge from the myopic one while converge to the baseline strategies in this case.

\begin{figure}[!h]
\centering
\begin{subfigure}{.5\textwidth}
\centering
\includegraphics[scale=0.4]{Power_Pi_M_bm05.png}
\caption{$\beta\equiv -0.5$}
\label{fig:power:Pi:M:b-05}
\end{subfigure}%
\begin{subfigure}{.5\textwidth}
\centering
\includegraphics[scale=0.4]{Power_Pi_M_b05.png}
\caption{$\beta\equiv 0.5$}
\label{fig:power:Pi:M:b05}
\end{subfigure}%

\caption{Realizations of worker's optimal investment strategies under scenarios $\omega_1$ and $\omega_2$ and with respect to two different baseline performances}
\label{fig:power:Pi:M}
\end{figure}

\begin{figure}[!h]
\centering
\begin{subfigure}{.5\textwidth}
\centering
\includegraphics[scale=0.4]{Power_Pi_S_bm05.png}
\caption{$\beta\equiv -0.5$}
\label{fig:power:Pi:S:b-05}
\end{subfigure}%
\begin{subfigure}{.5\textwidth}
\centering
\includegraphics[scale=0.4]{Power_Pi_S_b05.png}
\caption{$\beta\equiv 0.5$}
\label{fig:power:Pi:S:b05}
\end{subfigure}%

\caption{Realizations of worker's optimal investment strategies under scenarios $\omega_3$ and $\omega_4$ and with respect to two different baseline performances}
\label{fig:power:Pi:S}
\end{figure}

% The optimal investment strategy $\pi^*$ in \eqref{eq:pi*:power:1} can be decomposed into a weighed sum of a myopic component with fraction $1-X^{\hat{\pi},0}/X^*$, and a baseline component with fraction $X^{\hat{\pi},0}/X^*$. Figures \ref{fig:power:Pi:M} and \ref{fig:power:Pi:S} illustrate $\pi^*$ under different realizations of $B^1$ and $B^2$, respectively. \\



% \textbf{Effect of Hedgeable Risk $B^1$}\\ 
% For both choices of $\beta$, under $B^{1,S}$, the optimal investment strategy $\pi^*$ tends to follow the baseline strategy $\hat{\pi}$ as time propagates; and under the favourable market condition $B^{1,U}$, it follows closely to the myopic strategy. Intutively, an upward movement of $B^1$ has driven up both the price of the risky asset and the salary. By following the myopic strategy, the growth of the fund amount $W$ would outweigh the growth of $Y$,  which leads to a substantial growth of $X^*$. Along with a relatively moderate movement of $X^{\hat{\pi},0}$ under $\hat{\pi}$ (see Figures \ref{fig:power:XZ:UM:b-05} and \ref{fig:power:XZ:UM:b05}), the difference between $X^*$ and the level $X^{\hat{\pi},0}$ is further widened. This reduces the investment proportion $X^{\hat{\pi},0}/X^*$ on $\hat{\pi}$ and thus leaning the strategy further towards the myopic component. However, under $B^{1,S}$, the ratio $X^{\hat{\pi},0}/X^*$ tends to decrease, since the change of the two funds are dominated by the contribution of the salary. Moving forward in time, the difference of performance between $X^*$ and $X^{\hat{\pi},0}$ is shrunken. Thus, the overall strategy will lean towards $\hat{\pi}$. Hence, the baseline strategy $\hat{\pi}$ provides a guarantee to workers under non-bullish market which does not favour the use of the myopic strategy.\\  %In this scenario, by maintaining a suitable fraction of investment under $\hat{\pi}$, excessive loss due to indiscriminately following the myopic strategy could be avoided, whence introduces a hedging-like effect.   \\
%avoid excessive loss by blindly flowing the myopic component 

Similar interpretation, in terms of the worker's optimal investment strategy being a convex combination of the baseline and myopic strategies, is also clearly shown in Figure \ref{fig:power:Pi:S}; see also Figure \ref{fig:power:XZ:S} for the corrseponding fund value to salary ratios, of optimality and of baseline. However, note that her optimal investment strategies, when the non-hedgeable risk $B^2$ moves up under the scenario $\omega_3$, and when $B^2$ moves down under the scenario $\omega_4$, are almost identical. This is because the diffusion of the fund value to salary ratio with respect to the non-hedgeable risk $B^2$ is independent of her investment strategy; the influence on the ratio by different $B^2$ is not amplified by non-identical investment strategies.


% \textbf{Effect of Non-Hedgeable Risk $B^2$}\\ 
% %Since $B^2$ is non-hedgeable, the effect of different realizations of $B^2$ on the optimal investment strategy is less distinctive  compared to that of $B^1$. Indeed, changes of $B^2$ only affect the salary but not the market condition. 
% From Figure \ref{fig:power:Pi:S}, we see that the optimal investment strategy tends to lean towards the baseline strategy $\hat{\pi}$ as time propagates for both choices of $\beta$ and movements of $B^2$.  Given $B^1=B^0$ in Figure \ref{fig:B3}, the same moderate movement of the market, the accumulation of the pension fund will mainly due to the contribution of the salary.  Consequently, the investment fraction $X^{\hat{\pi},0}/X^*$ on the baseline strategy tends to increase and one therefore chooses to lean towards $\hat{\pi}$. The fraction $X^{\hat{\pi},0}/X^*=W^{\hat{\pi},0}/W^*$ is smaller under the downward movement $B^{2,D}$, whence workers would be less biased towards the baseline strategy $\hat{\pi}$. As mentioned above, although a drop of salary $Y$ immediately increases the ratio $X^*=W^*/Y$, it also further reduces the utility. Hence, workers desire an even higher fund amount $W^*$ in order to maintain the same level of utility when facing the uncertainty in the labour market. This encourages them to invest slightly more on the myopic strategy. 

%In other words, the difference of performance between choosing $\pi^*$ and $\hat{\pi}$ is further shrunken when the salary is moving upwards substantially.% Indeed, from Figures \ref{fig:power:Pi:M}-\ref{fig:power:Pi:S}, the only situation that one may like to follow closely to the myopic strategy is when $B^1$ is experiencing an upward trend. 

% consider that $y/(y+1) \to 0 as y\to \infty$



\begin{figure}[!h]
\centering
\begin{subfigure}{.5\textwidth}
\centering
\includegraphics[scale=0.4]{Power_XZ_UM_bm05.png}
\caption{$\beta\equiv -0.5$}
\label{fig:power:XZ:UM:b-05}
\end{subfigure}%
\begin{subfigure}{.5\textwidth}
\centering
\includegraphics[scale=0.4]{Power_XZ_DM_bm05.png}
\caption{$\beta\equiv -0.5$}
\label{fig:power:XZ:DM:b-05} 
\end{subfigure}%

\begin{subfigure}{.5\textwidth}
\centering
\includegraphics[scale=0.4]{Power_XZ_UM_b05.png}
\caption{$\beta\equiv 0.5$}
\label{fig:power:XZ:UM:b05}
\end{subfigure}%
\begin{subfigure}{.5\textwidth}
\centering
\includegraphics[scale=0.4]{Power_XZ_DM_b05.png}
\caption{$\beta\equiv 0.5$}
\label{fig:power:XZ:DM:b05} 
\end{subfigure}%

\caption{Realizations of worker's pension fund value to salary ratios under scenarios $\omega_1$ and $\omega_2$ (be mindful for the scale difference) and with respect to two different baseline performances}
\label{fig:power:XZ:M}
\end{figure}



%\begin{figure}[!h]
%    \centering
%     \begin{subfigure}{.5\textwidth}
%        \centering
%        \includegraphics[scale=0.4]{Power_X_M_bm05.png}
%        \caption{$X^*$, $\beta=-0.5$.}
%        \label{fig:power:X:M:b-05}
%    \end{subfigure}%
%     \begin{subfigure}{.5\textwidth}
%        \centering
%        \includegraphics[scale=0.4]{Power_X_M_b05.png}
%        \caption{$X^*$, $\beta=0.5$.}
%        \label{fig:power:X:M:b05}
%    \end{subfigure}%

%  \caption{$X^*$  under the realizations of $B^1$ in Figure \ref{fig:B1}.   }
%   \label{fig:power:X:M}
%\end{figure}





\begin{figure}[!h]
\centering
\begin{subfigure}{.5\textwidth}
\centering
\includegraphics[scale=0.4]{Power_XZ_US_bm05.png}
\caption{$\beta\equiv -0.5$}
\label{fig:power:XZ:US:b-05}
\end{subfigure}%
\begin{subfigure}{.5\textwidth}
\centering
\includegraphics[scale=0.4]{Power_XZ_DS_bm05.png}
\caption{$\beta\equiv -0.5$}
\label{fig:power:XZ:DS:b-05} 
\end{subfigure}%

\begin{subfigure}{.5\textwidth}
\centering
\includegraphics[scale=0.4]{Power_XZ_US_b05.png}
\caption{$\beta\equiv 0.5$}
\label{fig:power:XZ:US:b05}
\end{subfigure}%
\begin{subfigure}{.5\textwidth}
\centering
\includegraphics[scale=0.4]{Power_XZ_DS_b05.png}
\caption{$\beta\equiv 0.5$}
\label{fig:power:XZ:DS:b05} 
\end{subfigure}%

\caption{Realizations of worker's pension fund value to salary ratios under scenarios $\omega_3$ and $\omega_4$ and with respect to two different baseline performances}
\label{fig:power:XZ:S}
\end{figure}



%\begin{figure}[!h]
%    \centering
%     \begin{subfigure}{.5\textwidth}
%        \centering
%        \includegraphics[scale=0.4]{Power_X_S_bm05.png}
%        \caption{$X^*$, $\beta=-0.5$.}
%        \label{fig:power:X:S:b-05}
%    \end{subfigure}%
%     \begin{subfigure}{.5\textwidth}
%        \centering
%        \includegraphics[scale=0.4]{Power_X_S_b05.png}
%        \caption{$X^*$, $\beta=0.5$.}
%        \label{fig:power:X:S:b05}
%    \end{subfigure}%

%    \caption{$X^*$  under the realizations of $B^2$ in Figure \ref{fig:B2}.   }
%    \label{fig:power:X:S}
%\end{figure}

\section{Exponential Forward Utility Preferences}
\label{sec:exp}
This section constructs an exponential forward utility preference on the worker's pension fund value to salary ratio. Unlike the constructed power forward utility preferences in Section \ref{sec:power}, the stochastic domain $\mathcal{D}\equiv\mathbb{R}$ in this case, as $\inf\mathcal{D}_0=-\infty$; also, the process $Z\equiv 0$, and thus the exponential forward preference is the translated random field itself with $\tilde{\mathcal{D}}=\mathbb{R}$. In the following, though, the same baseline investment strategy shall still serve a critical role.\\

% We construct an exponential forward preference as follows.
Let $\Gamma:=\{\Gamma_t\}_{t\geq 0}\in\mathcal{P}_1(\mathbb{F})$ be a process, such that (i) it satisfies that $\Gamma_0^{-1}=\gamma>0$, which is the worker's risk aversion parameter at the current time $t=0$, and (ii) it admits the following It\^o's diffusion form: for any $t\geq 0$,
\begin{equation*}
\label{eq:Gamma:exp}
d\Gamma_t = \Gamma_t\left(\alpha_tdt + \beta_t^\top d{\bf B}^1_t -(\sigma^{Y,2}_t)^\top d{\bf B}^2_t \right),
\end{equation*}
where $\alpha=\{\alpha_t\}_{t\geq 0}\in\mathcal{P}_1\left(\mathbb{F}\right)$ and $\beta=\{\beta_t\}_{t\geq 0}\in\mathcal{P}_n\left(\mathbb{F}\right)$ are uniformly bounded and also satisfy the same relation \eqref{eq:power:alpha:beta}. Using the same arguments as in Section \ref{sec:power}, the dynamics of $\Gamma$ can be rewritten as, for any $t\geq 0$,
    \begin{align*}
    % \begin{aligned}
    d\Gamma_t =&\;\Gamma_t\left( \left( \hat{\pi}^\top_t\Sigma_t (\lambda_t-\sigma^{Y,1}_t) - \mu^Y_t + \|\sigma^{Y,1}_t\|^2 + \|\sigma^{Y,2}_t\|^2 \right) dt  \right.\\&\quad\quad\left.+ (\hat{\pi}_t^\top\Sigma_t-(\sigma^{Y,1}_t)^\top) d{\bf B}^1_t - (\sigma^{Y,2}_t)^\top d{\bf B}^2_t \right),
    % \end{aligned}
    % \label{eq:Gamma:exp:rewrite}
\end{align*}
where $\hat{\pi}_t:=(\Sigma_t^\top)^{-1}(\sigma^{Y,1}_t+\beta_t)$, for $t\geq 0$, and thus $\Gamma$ is also a fund value to salary ratio, under the same exogenous baseline investment strategy $\hat{\pi}$, but with zero contribution from the worker as well as the ratio starting at $\gamma^{-1}$. Assigning the reciprocal of the worker's current risk aversion as the initial baseline fund value to salary ratio could seem to be artificial at the first glance; yet, the constructed exponential forward utility preference below shall designate another economic interpretation to $\Gamma$, as its reciprocal $\Gamma^{-1}:=\{\Gamma^{-1}_t\}_{t\geq 0}\in\mathcal{P}_1(\mathbb{F})$ is a dynamic risk aversion process of the worker.
% satisfy the relation \eqref{eq:power:alpha:beta}. Consider the following  \textit{ansatz}: for any $x\in \mathbb{R}$ and $t\geq 0$, 
% \begin{equation}
% \label{eq:U:exp}
% U(x,t) := -\exp\left( - \frac{x-X^{\hat{\pi},0}_t}{\Gamma_t} + V_t \right),
% \end{equation}
% where $X^{\hat{\pi},0}$ is the fund value to salary ratio defined in \eqref{eq:Z:power:rewrite} under an exogenous baseline strategy $\hat{\pi}=(\Sigma^\top)^{-1}(\sigma^{Y,1}+\beta)$, and the dynamics of the processes $V=\{V_t\}_{t\geq 0} \in\mathcal{P}_1(\mathbb{F})$ is to be determined so that the conditions in Definition \ref{def:forward} are satisfied. \\

% The process $\Gamma$ has the following interpretations. First, its reciprocal $\Gamma^{-1}:=\{\Gamma^{-1}_t\}_{t\geq 0}$ can be considered as a dynamic risk of aversion with initial value $\Gamma_0^{-1}=\gamma$. Second, using the relation \eqref{eq:power:alpha:beta}, the dynamics \eqref{eq:Gamma:exp} can be rewritten as 
%   \begin{equation}
%     \begin{aligned}
%     d\Gamma^{-1}_t =&\;  \Gamma^{-1}_t\left( \left( \hat{\pi}^\top_t\Sigma_t (\lambda_t-\sigma^{Y,1}_t) - \mu^Y_t + \|\sigma^{Y,1}_t\|^2 + \|\sigma^{Y,2}_t\|^2 \right) dt  \right.\\&\quad\quad\quad\quad\quad\left.+ (\hat{\pi}_t^\top\Sigma_t-(\sigma^{Y,1}_t)^\top) d{\bf B}^1_t - (\sigma^{Y,2}_t)^\top d{\bf B}^2_t \right).
%     \end{aligned}
%     \label{eq:Gamma:exp:rewrite}
% \end{equation}
% Hence, $\Gamma$ is indeed a fund value to salary ratio under the strategy $\hat{\pi}$ with zero contribution and initial ratio $\gamma^{-1}$. 

\subsection{Admissibility}\label{sec:admis_exponential}
%  The process $\Gamma$ serves as a dynamic risk of aversion which adapts to the changing market and labour conditions. In particular, we assume that $\Gamma$ satisfies $\Gamma_0=\gamma$ and, for any $t\geq 0$, 
%\begin{equation}
%\label{eq:exp:Z}
%d\Gamma_t = \Gamma_t(\alpha_tdt + \beta_t^\top d{\bf B}^1_t + (\sigma^{Y,2}_t)^\top d{\bf B}^2_t),
%\end{equation}
%where $\alpha_\cdot : [0,\infty) \to \mathbb{R}$ and $\beta_\cdot :[0,\infty)\to \mathbb{R}^n$ are uniformly bounded deterministic functions that satisfy a certain relation.
To define the admissible set of investment strategies in this case, only an integrability condition for constructing the preference is imposed:
\begin{equation*}
\label{eq:admissible:exp}
\mathcal{A}= \left\{ \pi \in \mathcal{P}_n(\mathbb{F}) : \frac{X^\pi}{\Gamma}(\pi- \hat{\pi}) \in \mathcal{L}^2_{n,\text{BMO}}  \right\}.
\end{equation*}
Loosely speaking, an investment strategy is admissible if it does not deviate too much from the baseline strategy, in terms of the $\mathcal{L}^2_{n,\text{BMO}}$-norm, with the deviation weighed by the quotient of the two ratios $X^{\pi}$ and $\Gamma$. This integrability condition is in line with the construction of forward utility preferences in the literature, where the worker's risk aversion is a constant and she does not compare her investment strategy to an exogenous baseline strategy, in which the condition would be $X^{\pi}\pi\in\mathcal{L}^2_{n,\text{BMO}}$; see, for example, \cite{liang:bsde}. The following equivalent form of the admissible set shall be more handy:
% The admissible set of investment strategies $\mathcal{A}$ is rich enough; indeed, define the set of investment strategies:
    \begin{equation*}
        \label{eq:A:tile:exp}
        \tilde{\mathcal{A}} = \left\{ \pi \in \mathcal{P}_n(\mathbb{F}) : \pi_t  = \hat{\pi}_t + \frac{\Gamma_t}{X^\pi_t}\xi_t,\;t\geq 0,\text{ for some }\xi = \{\xi_t\}_{t\geq 0}\in \mathcal{L}^2_{n,\text{BMO}} \right\}.
    \end{equation*}
% The following is immediate.
\begin{proposition}
\label{pp:admissible:exp}
% We have $\mathcal{\Tilde{A}}\subseteq \mathcal{A}$. 
We have $\mathcal{\Tilde{A}}=\mathcal{A}$.
\end{proposition}

\begin{proof}
Let $\pi\in\mathcal{\Tilde{A}}$. For any $t\geq 0$, $\frac{X^\pi_t}{\Gamma_t}(\pi_t- \hat{\pi}_t)=\xi_t$. Therefore, $\frac{X^\pi}{\Gamma}(\pi- \hat{\pi})\equiv\xi \in \mathcal{L}^2_{n,\text{BMO}}$, and thus $\pi\in\mathcal{A}$. This shows $\mathcal{\Tilde{A}}\subseteq \mathcal{A}$. The other inclusion $\mathcal{A}\subseteq\mathcal{\Tilde{A}}$ is clear.
\end{proof}
% Comparable constructions of forward exponential utility with a dynamic risk of aversion under usual complete financial market can be found in   \cite{Musiela2008,zitkovi:2009,zitkovi:2010}. {\color{red}Therein, $\Gamma$ is required  to be uniformly bounded from below.} This assumption appears to be infeasible herein, due to the lack of freedom on the non-hedgeable risk ${\bf B}^2$, since the coefficient for the stochastic integral with respect to ${\bf B}^2$ in \eqref{eq:Gamma:exp} is necessarily $-\sigma^{Y,2}$.  To address this,  we shall confine ourselves to the admissible set % strict boundedness assumption 
% % \begin{equation}
% % \label{eq:admissible:exp}
% % \mathcal{A}:= \left\{ \pi \in \mathcal{P}(\mathbb{F}) : X^\pi\Gamma^{-1}(\pi- \hat{\pi}) \in \mathcal{L}^2_{\text{BMO}}  \right\},
% % \end{equation}
% where the domain is taken to be $\mathcal{D}\equiv \mathbb{R}$. Hence, a strategy is admissible if its deviation with the baseline strategy $\hat{\pi}$, weighed by the ratio $X/\Gamma$ of the two fund values to salary ratio $X^\pi$ and $\Gamma^{-1}$, is sufficiently bounded. In particular, if we choose $\beta\equiv -\sigma^{Y,1}$, the admissibility is reduced to $X\Gamma\pi\in \mathcal{L}^2_{\text{BMO}}$, which is a standard admissibility condition for forward utility preferences when $\Gamma$ is assumed to be uniformly bounded away from zero and from above. To give a class of admissible strategies, consider the collection 

% For any $\pi\in \mathcal{\tilde{A}}$, we have $X^\pi\Gamma^{-1}(\pi-\hat{\pi}) = \xi\in \mathcal{L}^2_{\text{BMO}}$. Hence, the following is immediate. 


\subsection{Non-Zero Volatility Forward Preferences}\label{sec:exp:main}
The following theorem constructs a (non-)zero volatility exponential forward utility preference of the worker, together with her corresponding optimal investment strategy. Their economic insights shall be highlighted after the theorem and its proof.
\begin{theorem}
\label{pp:exp}
%Consider the process $\{\Gamma_t\}_{t\geq 0}$ in \eqref{eq:exp:Z}, where  $\alpha$ and $\beta$ satisfy, for any $t\geq 0$, 
%\begin{equation}
%\label{eq:alpha:beta}
%\alpha_t  -   \|\beta_t\|^2 + (\sigma^{Y,1}_t-\lambda_t)^\top\beta_t = \mu^Y_t - \lambda^\top_t\sigma^{Y,1}_t.
%\end{equation}
Let $V=\left\{V_t\right\}_{t\geq 0}\in\mathcal{P}_1\left(\mathbb{F}\right)$ be a process, given by $V_0=0$ and, for any $t\geq 0$,
\begin{equation*}
dV_t=\frac{1}{2}\left(\|\lambda_t-\sigma^{Y,1}_t+\theta^1_t-\beta_t\|^2-\|\theta^1_t\|^2-\|\theta^2_t\|^2\right)dt +(\theta^1_t)^\top d{\bf B}^1_t + (\theta^2_t)^\top d{\bf B}^2_t,
\end{equation*}
where $\theta^1 = \{\theta^1_t\}_{t\geq 0}\in \mathcal{P}_n(\mathbb{F})$ and $\theta^2 =\{\theta^2_t\}_{t\geq 0} \in \mathcal{P}_m(\mathbb{F})$ are uniformly bounded. The random field, for any $t\geq 0$ and $x\in\mathbb{R}$,
\begin{equation}
\label{eq:U:exp}
U\left(x,t\right)=-\exp\left( - \frac{x-X^{\hat{\pi},0}_t}{\Gamma_t} + V_t \right),
\end{equation}
where $X^{\hat{\pi},0}:=\left\{X^{\hat{\pi},0}_t\right\}_{t\geq 0}$ solves \eqref{eq:Z:power:rewrite} with $X^{\hat{\pi},0}_0=0$, is an exponential forward utility preference on the fund value to salary ratio. In addition, its volatility processes are given by, for any $x\in\mathbb{R}$ and $t\geq 0$,
\begin{equation}
\label{eq:vol:exp}
\begin{aligned}
      a_1(x,t) =&-\exp\left(-\frac{x-X^{\hat{\pi},0}_t}{\Gamma_t}+V_t \right)\left(\theta^1_t+\frac{x}{\Gamma_t}\beta_t \right),\text{ and} \\
      a_2(x,t) =&-\exp\left(-\frac{x-X^{\hat{\pi},0}_t}{\Gamma_t}+V_t \right)\left(\theta^2_t-\frac{x}{\Gamma_t} \sigma^{Y,2}_t\right).
\end{aligned}
\end{equation}
Moreover, the optimal investment strategy is given by, for any $t\geq 0$,
\begin{equation}
\label{eq:pi*:exp}
\pi_t^* = \frac{\Gamma_t}{X^*_t}\left(\Sigma^{\top}_t\right)^{-1}\left(\lambda_t+\theta^1_t\right)+\left(1-\frac{\Gamma_t}{X^*_t}\right)\hat{\pi}_t,
\end{equation}
where $X^*:\equiv X^{\pi^*}$ satisfies, for any $t\geq 0$,
\begin{align*}
% \begin{aligned}
dX_t^* =& \  \Big(p_t + \Gamma_t(\lambda_t-\sigma^{Y,1}_t)^\top(\lambda_t-\sigma^{Y,1}_t+\theta^1_t-\beta_t)   \\ & \quad +  X^*_t\left( (\lambda_t-\sigma^{Y,1}_t)^\top\beta_t+\lambda^\top_t\sigma^{Y,1}_t + \|\sigma^{Y,2}_t\|^2 - \mu^Y_t \right)\Big)dt   \\ & \
+   \left(\Gamma_t(\lambda_t-\sigma^{Y,1}_t+ \theta^1_t-\beta_t)+X^*_t \beta_t  \right)^\top d{\bf B}^1_t   - X^*_t(\sigma^{Y,2}_t)^\top d{\bf B}_t^2 .
% \end{aligned}
% \label{eq:exp:X^*}
\end{align*}
% Then $U=\{U(x,t):t\geq 0, \ x\in\mathbb{R} \}$ in \eqref{eq:U:exp} is a forward utility preference for $X^\pi$ with admissible set \eqref{eq:admissible:exp}, and $U(x,0) = -e^{-\gamma x}$. The optimal investment strategy $\pi^*=\{\pi^*_t\}_{t\geq 0}$ under \eqref{eq:U:exp} is given by, for any $t\geq 0$, 

% where $X^*:=X^{\pi^*}$ satisfies, for $t\geq 0$,

% In addition, the random field $U=\tilde{U}$ is the solution to the SPDE \eqref{eq:SPDE} with volatility processes 

% where $x\in\mathbb{R}$ and $t\geq 0$. 
\end{theorem}
%Intutively, the process $Z_t$ is a dynamic risk of aversion in the contest of exponential utility, which varies according to the market.


\begin{proof}
As this proof follows similarly to that of Theorem \ref{pp:power}, only the essential steps are outlined. First, $\pi^*$ can be rewritten as, for any $t\geq 0$,
\begin{equation*}
\pi_t^* = \hat{\pi}_t+\frac{\Gamma_t}{X^*_t}\left(\left(\Sigma^{\top}_t\right)^{-1}\left(\lambda_t+\theta^1_t\right)-\hat{\pi}_t\right),
\end{equation*}
and thus $\pi^*\in\tilde{\mathcal{A}}=\mathcal{A}$, with $\xi\equiv\left(\Sigma^{\top}_\cdot\right)^{-1}\left(\lambda_\cdot+\theta^1\right)-\hat{\pi}\in\mathcal{L}^{2}_{n,\text{BMO}}$, by the (uniform) boundedness and Proposition \ref{pp:admissible:exp}. Next, since $\Gamma$ is uniformly positive, $U$ in \eqref{eq:U:exp} satisfies (i) and (ii) in Definition \ref{def:forward}. To verify (iii) of Definition \ref{def:forward}, it suffices to show that $\{U(X^\pi_t,t)\}_{t\geq 0}$ is an $\mathbb{F}$-super-martingale for any $\pi\in\mathcal{A}$,  and that $\{U(X^*_t,t)\}_{t\geq 0}$ is an $\mathbb{F}$-martingale.\\

For any $\pi\in\mathcal{A}$, define $R^{\pi}_t:=-\exp\left( - \frac{X^{\pi}_t-X^{\hat{\pi},0}_t}{\Gamma_t} + V_t \right)$ for $t\geq 0$. By It\^o's lemma, for any $t\geq 0$,
\begin{align*}
\frac{dR^{\pi}_t}{R^{\pi}_t}=&\;\frac{1}{2}\left(\frac{X^{\pi}_t}{\Gamma_t}\right)^2\left\|\Sigma^\top_t\pi_t - \sigma^{Y,1}_t- \beta_t  -  \frac{\Gamma_t}{X^{\pi}_t}\left(\lambda_t-\sigma^{Y,1}_t+\theta^1_t - \beta_t\right) \right\|^2 dt \\
        & -\left(\frac{X^{\pi}_t}{\Gamma_t} (\Sigma^\top_t\pi_t-\sigma^{Y,1}_t -\beta_t ) -\theta^1_t \right)^{\top}  d{\bf B}^1_t + \left(\theta^2_t\right)^\top d{\bf B}^2_t.
\end{align*}
Hence, for any $0\leq s\leq t$,
\begin{align*}
U\left(X^{\pi}_t,t\right)=&\;U\left(X^{\pi}_s,s\right)\\&\times\exp\left(\int_{s}^{t}\frac{1}{2}\left(\frac{X^{\pi}_l}{\Gamma_l}\right)^2\left\|\Sigma^\top_l\pi_l - \sigma^{Y,1}_l- \beta_l  -  \frac{\Gamma_l}{X^{\pi}_l}\left(\lambda_l-\sigma^{Y,1}_l+\theta^1_l - \beta_l\right) \right\|^2dl\right)\mathcal{E}_{s,t},
\end{align*}
where $\left\{\mathcal{E}_{s,t}\right\}_{t\geq s}$ satisfies $\mathcal{E}_{s,s}=1$ and, for any $0\leq s\leq t$,
\begin{equation*}
d\mathcal{E}_{s,t}=\mathcal{E}_{s,t}\left(-\left(\frac{X^{\pi}_t}{\Gamma_t} (\Sigma^\top_t\pi_t-\sigma^{Y,1}_t -\beta_t ) -\theta^1_t \right)^{\top}  d{\bf B}^1_t + \left(\theta^2_t\right)^\top d{\bf B}^2_t\right).
\end{equation*}
Since $\pi\in\mathcal{A}=\tilde{\mathcal{A}}$, $\frac{X^{\pi}}{\Gamma} (\Sigma^\top_\cdot\pi-\sigma^{Y,1}_\cdot -\beta )\equiv\Sigma_\cdot^\top\xi\in\mathcal{L}^{2}_{n,\text{BMO}}$, and hence, together with the uniform boundedness, the process $\left\{\mathcal{E}_{s,t}\right\}_{t\geq s}$ is an $\mathbb{F}$-martingale.\\

Therefore, for any $\pi\in\mathcal{A}$, and for any $0\leq s\leq t$,
\begin{align*}
&\;\mathbb{E}\left[\exp\left(\int_{s}^{t}\frac{1}{2}\left(\frac{X^{\pi}_l}{\Gamma_l}\right)^2\left\|\Sigma^\top_l\pi_l - \sigma^{Y,1}_l- \beta_l  -  \frac{\Gamma_l}{X^{\pi}_l}\left(\lambda_l-\sigma^{Y,1}_l+\theta^1_l - \beta_l\right) \right\|^2dl\right)\mathcal{E}_{s,t}\Big\vert\mathcal{F}_s\right]\\\geq&\;\mathbb{E}\left[\mathcal{E}_{s,t}\vert\mathcal{F}_s\right]=\mathcal{E}_{s,s}=1,
\end{align*}
where the equality holds when $\pi\equiv\pi^*$. By the uniform negativity of $U$ in \eqref{eq:U:exp}, for any $\pi\in\mathcal{A}$, and for any $0\leq s\leq t$,
\begin{equation*}
\mathbb{E}\left[U\left(X^{\pi}_t,t\right)\vert\mathcal{F}_s\right]\leq U\left(X^{\pi}_s,s\right),
\end{equation*}
and in particular, $\mathbb{E}\left[U\left(X^*_t,t\right)\vert\mathcal{F}_s\right]=U\left(X^*_s,s\right)$. These show that $\{U(X^\pi_t,t)\}_{t\geq 0}$ is an $\mathbb{F}$-super-martingale for any $\pi\in\mathcal{A}$,  and that $\{U(X^*_t,t)\}_{t\geq 0}$ is an $\mathbb{F}$-martingale.\\

Finally, by It\^o's lemma on $U$ in \eqref{eq:U:exp}, for any $x\in\mathbb{R}$ and $t\geq 0$,
\begin{align*}
dU\left(x,t\right)=&\;U\left(x,t\right)\left(\left(\frac{1}{2}\|\lambda_t-\sigma^{Y,1}_t+\theta^1_t-\beta_t\|^2+\frac{p_t}{\Gamma_t}+x\left(\beta_t^\top\theta^1_t-\left(\sigma^{Y,2}_t\right)^\top\theta^2_t\right)\right.\right.\\&\;\left.\left.\quad\quad\quad\quad\;\;+\frac{x}{\Gamma_t}\left(\alpha_t-\|\beta_t\|^2-\|\sigma^{Y,2}_t\|^2\right)+\frac{1}{2}\left(\frac{x}{\Gamma_t}\right)^2\left(\|\beta_t\|^2+\|\sigma^{Y,2}_t\|^2\right)\right)dt\right.\\&\;\left.\quad\quad\quad\quad+\left(\theta^1_t+\frac{x}{\Gamma_t}\beta_t\right)^\top d{\bf B}^1_t+\left(\theta^2_t-\frac{x}{\Gamma_t} \sigma^{Y,2}_t\right)^\top d{\bf B}^2_t\right),
\end{align*}
which shows that the volatility processes of $U$ in \eqref{eq:U:exp} are indeed given by \eqref{eq:vol:exp}.


% Notice that, by It\^o's lemma, $d\left(\frac{X^{\hat{\pi},0}_t}{\Gamma_t}\right)=\frac{p_t}{\Gamma_t}dt$, and hence, for any $t\geq 0$,
% \begin{equation*}
% R^{\pi}_t=-\exp\left( - \frac{X^{\pi}_t}{\Gamma_t}+\int_{0}^{t}\frac{p_s}{\Gamma_s}ds + V_t \right).
% \end{equation*}

% % First, notice that by \eqref{eq:pi*:exp}, $\pi^*\in \mathcal{\tilde{A}}$, whence $\pi^*\in \mathcal{A}$. Next, we show that $U$ in \eqref{eq:U:exp} is indeed a forward preference.
% % For any $\pi\in\mathcal{A}$, let $X=X^\pi$ and define the process $R=\{R_t\}_{t\geq 0}$ by, for any $t\geq 0$. $R_t := -e^{-\Gamma^{-1}_t(X_t-X^{\hat{\pi},0}_t)+V_t}=U(X_t,t)$. Notice that, by It\^o's lemma, for any $t\geq0$, we have
% %     \begin{equation*}
% %         d\Gamma_t^{-1} = \Gamma_t^{-1}\left( \left(\|\beta_t\|^2+\|\sigma^{Y,2}_t\|^2 - \alpha_t \right)dt - \beta^\top_t d{\bf B}^1_t + (\sigma^{Y,2}_t)^\top d{\bf B}^2_t   \right),
% %     \end{equation*}
% % and 
% %     \begin{equation*}
% %         d(X^{\hat{\pi},0}_t\Gamma_t^{-1}) = p_t\Gamma_t^{-1}dt.
% %     \end{equation*}
% % Hence, for any $t\geq 0$, 
% %     \begin{equation*}
% %         R_t = -\exp\left(-\Gamma_t^{-1}X_t + \int_0^t p_s\Gamma_s^{-1} ds + V_t\right).
% %     \end{equation*}




%  We proceed to show that $R$ is a super-martingale. By applying It\^o's lemma on $R$, we have, for any $t\geq 0$,
% \begin{align*}
% \frac{dR_t}{R_t} = & \  (-\Gamma^{-1}_t dX_t -X_td\Gamma^{-1}_t + \Gamma_t^{-1}p_tdt +dV_t) + \frac{1}{2}\bigg( \Gamma^{-2}_t d\langle X_\cdot\rangle_t + X_t^2 d\langle \Gamma^{-1}_\cdot\rangle_t\\
% &\ \quad  +   2(X_t\Gamma^{-1}_t-1)d\langle X_\cdot,\Gamma^{-1}_\cdot\rangle_t   {\color{red} +d\langle V_\cdot\rangle_\cdot -2X_td\langle \Gamma^{-1}_\cdot,V_\cdot\rangle_\cdot - 2\Gamma^{-1}_t d\langle X_\cdot,V_\cdot\rangle_\cdot}
% \bigg) \\
% =&  -X_t\Gamma_t^{-1} \Big( \pi^\top_t\Sigma_t(\lambda_t-\sigma^{Y,1}_t) - \mu^Y_t + \|\sigma^{Y,1}_t\|^2  +  (\pi_t^\top\Sigma_t-(\sigma^{Y,1}_t)^\top)({\color{red}\theta^1_t}-\beta_t )\\
% &\ \quad  -{\color{red}\beta_t^\top\theta^1_t } +\|\beta_t\|^2+\|\sigma^{Y,2}_t\|^2 - \alpha_t  \Big)dt +   \frac{\|\lambda_t-\sigma^{Y,1}_t-\beta_t+{\color{red} \theta^1_t}\|^2}{2}  dt \\
% &\ + \frac{X_t^2\Gamma_t^{-2}}{2}   \|\pi_t^\top\Sigma_t-\sigma^{Y,1}_t-\beta_t\|^2 dt\\
% &\ -\left(X_t\Gamma^{-1}_t (\pi_t^\top \Sigma_t-(\sigma^{Y,1}_t)^\top -\beta^\top_t ) -{\color{red}(\theta^1_t)^\top} \right)  d{\bf B}^1_t + {\color{red}(\theta^2_t)^\top d{\bf B}^2_t  } \\
%  = &  \Bigg( \frac{X_t^2\Gamma_t^{-2}}{2}\|\Sigma^\top_t\pi_t\|^2 -\pi_t^\top\Sigma_tX_t\Gamma^{-1}_t \Big(   X_t\Gamma^{-1}_t (\beta_t +\sigma^{Y,1}_t ) +  (\lambda_t-\sigma^{Y,1}_t-\beta_t+{\color{red} \theta^1_t})   \Big)     \\
% & \ \quad  +\frac{\|\lambda_t-\sigma^{Y,1}_t-\beta_t+{\color{red} \theta^1_t}\|^2}{2}+ \frac{X_t^2\Gamma_t^{-2}}{2}\|\beta_t + \sigma^{Y,1}_t\|^2   \\ 
% & \ \quad + X_t\Gamma^{-1}_t\left( \mu^Y_t - \|\sigma^{Y,1}_t\|^2 +\alpha_t -\beta^\top_t\sigma^{Y,1}_t +{\color{red}(\sigma^{Y,1}_t+\beta_t)^\top\theta^1_t } -\|\beta_t\|^2-\|\sigma^{Y,2}_t\|^2 \right) \Bigg)dt \\
% & \ -\left(X_t\Gamma^{-1}_t (\pi_t^\top \Sigma_t-(\sigma^{Y,1}_t)^\top -\beta^\top_t ) -{\color{red}(\theta^1_t)^\top} \right)  d{\bf B}^1_t + {\color{red}(\theta^2_t)^\top d{\bf B}^2_t  } \\
% =&\ \Bigg( \frac{X_t^2\Gamma_t^{-2}}{2}\left\|\Sigma^\top_t\pi_t -  \beta_t-\sigma^{Y,1}_t  -  \frac{\lambda_t-\sigma^{Y,1}_t - \beta_t +{\color{red} \theta^1_t}}{X_t\Gamma^{-1}_t} \right\|^2  + X_t\Gamma^{-1}_t \Big( \mu^Y_t -  \|\sigma^{Y,1}_t\|^2 \\  &\ \quad +\alpha_t  - \beta^\top_t\sigma^{Y,1}_t    -   (\beta_t+\sigma^{Y,1}_t )^\top(\lambda_t-\sigma^{Y,1}_t-  \beta_t) -\|\beta_t\|^2-\|\sigma^{Y,2}_t\|^2    \Big) \Bigg) dt \\
% & \ -\left(X_t\Gamma^{-1}_t (\pi_t^\top \Sigma_t-(\sigma^{Y,1}_t)^\top -\beta^\top_t ) -{\color{red}(\theta^1_t)^\top} \right)  d{\bf B}^1_t + {\color{red}(\theta^2_t)^\top d{\bf B}^2_t  }.
% \end{align*}
% By \eqref{eq:power:alpha:beta}, we have 
%     \begin{equation*}
%         \begin{aligned}
%         \frac{dU(X_t,t)}{U(X_t,t)} = &  \frac{X_t^2\Gamma_t^{-2}}{2}\left\|\Sigma^\top_t\pi_t -  \beta_t-\sigma^{Y,1}_t  -  \frac{\lambda_t-\sigma^{Y,1}_t - \beta_t +{\color{red} \theta^1_t}  }{X_t\Gamma_t^{-1}} \right\|^2 dt \\
%         & \ -\left(X_t\Gamma^{-1}_t (\pi_t^\top \Sigma_t-(\sigma^{Y,1}_t)^\top -\beta^\top_t ) -{\color{red}(\theta^1_t)^\top} \right)  d{\bf B}^1_t + {\color{red}(\theta^2_t)^\top d{\bf B}^2_t  },
%         \end{aligned}
%     \end{equation*}

% i.e., for $0\leq s\leq t$,  
% \begin{equation*}
% U(X_t,t) = U(X_s,s)e^{\int_s^t\frac{X_l^2\Gamma^{-2}_l}{2}\left\|\Sigma^\top_l\pi_l -  \beta_l-\sigma^{Y,1}_l  -  \frac{\lambda_l-\sigma^{Y,1}_l- \beta_l {\color{red}+\theta^1_l} }{X_l\Gamma_l^{-1}} \right\|^2 dl } \mathcal{E}_{s,t},
% \end{equation*}
% where $\{\mathcal{E}_{s,t}\}_{t\geq s}$ satisfies $\mathcal{E}_{s,s}=1$ and, for $t\geq s$, 
% \begin{equation*}
% d\mathcal{E}_{s,t} = -\left(X_t\Gamma^{-1}_t (\pi_t^\top \Sigma_t-(\sigma^{Y,1}_t)^\top -\beta^\top_t ) -{\color{red}(\theta^1_t)^\top} \right)  d{\bf B}^1_t + {\color{red}(\theta^2_t)^\top d{\bf B}^2_t  }.
% \end{equation*}
% Since $\pi\in \mathcal{A}$, the process  $\{\mathcal{E}_{s,t}\}_{t\geq s}$ is a martingale. Along with the negativity of $U$, we have 
% \begin{equation*}
% \mathbb{E}[U(X_t,t)|\mathcal{F}_s] = U(X_s,s) \mathbb{E}\left[e^{\int_s^t\frac{X_l^2\Gamma_l^{-2}}{2}\left\|\Sigma^\top_l\pi_l -  \beta_l-\sigma^{Y,1}_l  -  \frac{\lambda_l-\sigma^{Y,1}_l- \beta_l +\theta^1_l  }{X_l\Gamma_l^{-1}} \right\|^2 dl } \mathcal{E}_{s,t} \mid \mathcal{F}_s\right]\leq U(X_s,s).
% \end{equation*}
% Thus, $\{U(X_t,t)\}_{t\geq 0}$ is a super-martingale. Substituting \eqref{eq:pi*:exp} into \eqref{eq:X}, we obtain \eqref{eq:exp:X^*}, and 
% \begin{equation}
% \label{eq:exp:U*}
% dU(X_t^*,t) = -U(X_t^*,t)(\lambda_t -\sigma^{Y,1}_t-\beta_t+{\color{red} \theta^1_t })^\top d{\bf B}^1_t +{\color{red} (\theta^2_t)}^\top d{\bf B^2}_t, 
% \end{equation}
% whence $\{U(X_t^*,t)\}_{t\geq 0}$ is a martingale by the boundedness of $\lambda,\sigma^{Y,1}$, $\beta$ and $\theta^1$. By applying It\^o's lemma on \eqref{eq:U:exp}, we obtain
% \begin{align*}
% dU(x,t) =& \ U(x,t)\Bigg( \frac{\|\lambda_t-\sigma^{Y,1}_t-\beta_t+{\color{red} \theta^1_t}\|^2}{2} + \Gamma^{-1}_t\left(p_t+x\left(\alpha_t -\|\beta_t\|^2 -\|\sigma^{Y,2}_t\|^2\right)\right) +  \\ &\ \ 
%  \frac{x^2\Gamma_t^{-2}}{2}\left( \|\beta_t\|^2 + \|\sigma^{Y,2}_t\|^2 \right)+ x\left(\beta_t^\top\theta^1_t - (\sigma^{Y,2}_t)^\top\theta^2_t  \right) \Bigg)dt    \\
% &\ +U(x,t)\left( \left({\color{red} \theta^1_t}+x\Gamma^{-1}_t\beta_t \right)^\top d {\bf B}^1_t + \left({\color{red}\theta^2_t }-x\Gamma^{-1}_t\sigma^{Y,2}_t \right)^\top d{\bf B}^2_t\right),
% \end{align*}
% which solves the SPDE \eqref{eq:SPDE} with the volatility processes given in \eqref{eq:vol:exp}.

\end{proof}

% \xi = \hat{\pi} + \frac{\Gamma}{X-\hat{X}}((\lambda+\theta^1)/\sigma-\hat{\pi})
%\subsection{Discussion of Results}
%Similar to the forward power utility preference, the relation \eqref{eq:alpha:beta} provides a physical interpretation to the risk of aversion $\Gamma$. By It\^o's lemma, one sees that the process $\Gamma^{-1}=\{\Gamma_t^{-1}\}_{t\geq 0}$ satisfies $\Gamma_0^{-1}=\gamma^{-1}$ and
%\begin{align*}
%d\Gamma_t^{-1} &= \Gamma_t^{-1}\Big(  \left( \|\beta_t\|^2 +\|\sigma^{Y,2}_t\|^2 -\alpha_t \right) dt -  \beta^\top_td{\bf B}^1_t -(\sigma^{Y,2}_t)^\top d{\bf B}^2_t  \Big), \nonumber \\
%&= \Gamma_t^{-1}\left( \left(\lambda_t^\top \sigma^{Y,1}_t + \|\sigma^{Y,2}_t\|^2-\mu^Y_t - \beta_t^\top(\sigma^{Y,1}_t-\lambda_t) \right)dt -\beta^\top_td{\bf B}^1_t -(\sigma^{Y,2}_t)^\top d{\bf B}^2_t  \right).
%\end{align*}
%Recall that $\hat{\pi} = (\Sigma^\top)^{-1}(\sigma^{Y,1}-\beta)$, we see that, for any $t\geq 0$, 
%\begin{align*}
%d\Gamma_t^{-1} =& \  \Gamma_t^{-1}\bigg(  \left( \hat{\pi}^\top_t\Sigma_t (\lambda_t-\sigma^{Y,1}_t) - \mu^Y_t + \|\sigma^{Y,1}_t\|^2 + \|\sigma^{Y,2}_t\|^2 \right) dt   \\ & \ +  (\hat{\pi}_t^\top\Sigma_t-(\sigma^{Y,1}_t)^\top) d{\bf B}^1_t - (\sigma^{Y,2}_t)^\top d{\bf B}^2_t \bigg).
%\end{align*}
%Hence, the process $\Gamma^{-1}$ is indeed a fund value to salary ratio with initial wealth $\gamma^{-1}y$ and zero contribution. In addition, the process $X^{\hat{\pi},0}=\{X^{\hat{\pi},0}_t:= \Gamma^{-1}_t \int_0^t p_s \Gamma_s ds\}_{t\geq 0}$  satisfies $X^{\hat{\pi},0}_0=0$, and for any $t\geq 0$,  
%\begin{equation}
%\begin{aligned}
%dX^{\hat{\pi},0}_t = & \  \left(p_t + X^{\hat{\pi},0}_t \left( \hat{\pi}^\top_t\Sigma_t (\lambda_t-\sigma^{Y,1}_t) - \mu^Y_t + \|\sigma^{Y,1}_t\|^2 + \|\sigma^{Y,2}_t\|^2 \right) \right)dt+ \nonumber   \\ & \ X^{\hat{\pi},0}_t\left( (\hat{\pi}_t^\top\Sigma_t-(\sigma^{Y,1}_t)^\top) d{\bf B}^1_t - (\sigma^{Y,2}_t)^\top d{\bf B}^2_t \right),
%\end{aligned}
%\label{eq:hat:X}
%\end{equation}
%which therefore corresponds to the fund value to salary ratio with zero initial wealth under an exogenous baseline investment strategy $\hat{\pi}$.
%The utility process can then be written as 
%\begin{equation}
%\label{eq:exp:U:rewrite}
%\begin{aligned}
%U(X_t,t) =&\ -\exp\Bigg( -\Gamma_t(X_t - %X^{\hat{\pi},0}_t) + \frac{1}{2}\int_0^t  \left( \|  %\lambda_s+{\color{red} \theta^1_s}- %\Sigma^\top_s\hat{\pi}_s \|^2-{\color{red} %\|\theta^1_s\|^2-\|\theta^2_s\|^2}\right) ds \\
%&\ \quad +{\color{red} \int_0^t(\theta^1_s)^\top d{\bf B}^1_s + \int_0^t (\theta^2_s)^\top d{\bf B}^2_s }\Bigg).
%\end{aligned}
%\end{equation}
The worker's forward utility preference depends on the two baseline fund value to salary ratios in this case. One is $X^{\hat{\pi},0}$ with the salary contribution and starting from zero fund value, while another one is $\Gamma$ with zero salary contribution and the initial ratio as the reciprocal of the worker’s current risk aversion. Again, the worker determines how much more her fund value to salary ratio exceeding the baseline ratio $X^{\hat{\pi},0}$; however, unlike the power forward utility preference therein, this surplus $x-X^{\hat{\pi},0}$ could be negative, that her ratio could be less than the baseline ratio. The surplus first measures the worker's fund performance comparing to the baseline, in terms of the ratio, without taking the explicit effect by the salary contribution into account. This absolute difference is then contrasted to the other baseline ratio $\Gamma$, which is without any explicit or implicit considerations of the salary contribution. Therefore, $\frac{x-X^{\hat{\pi},0}}{\Gamma}$ quantifies the relative difference of the worker's fund performance, with respect to the baseline investment strategy.\\
% , purely based on the developments from the hedgeable and non-hedgeable risks.\\

Recall that $U\left(x,0\right)=-\exp\left(-\gamma x\right)=-\exp\left(-\Gamma^{-1}_0 x\right)$, for $x\in\mathbb{R}$, and thus the reciprocal of the initial baseline ratio $\Gamma^{-1}_0$ measures the worker's current risk aversion, by definition. This relationship can be generalized to any future times. That is, define the process $\Gamma^{-1}=\{\Gamma^{-1}_t\}_{t\geq 0}$ as the reciprocal of the baseline ratio; then, for any $t\geq 0$, $U\left(x,t\right)=-\exp\left(-\Gamma^{-1}_t\left(x-X^{\hat{\pi},0}_t\right)+ V_t \right)$, which resembles standard exponential preferences with constant risk aversion parameter in the place of $\Gamma^{-1}$, and thus $\Gamma^{-1}$ can be perceived as the dynamic risk aversion process of the worker. In such manner, if the baseline ratio $\Gamma$ performs well with a large value, the worker will be less risk averse, and vice versa; this is consistent with the fact that any admissible investment strategies are not deviating too much from the baseline strategy.\\

% When evaluated at $X^*$, the forward preference can be written as, for any $t\geq 0$,
% \begin{equation}
% \label{eq:U:exp:X*}
% \begin{aligned}
%     U(X^*_t,t) =&\  -\exp\Bigg( -\Gamma^{-1}_t(X^*_t - X^{\hat{\pi},0}_t) + \frac{1}{2}\int_0^t \left( \frac{ \| X^*_s \Gamma^{-1}_s(\pi^*_s-\hat{\pi}_s)  \|^2}{\| \Sigma_s \|^2}-{\color{red} \|\theta^1_s\|^2-\|\theta^2_s\|^2} \right) ds\\
%     &\ \quad +{\color{red} \int_0^t(\theta^1_s)^\top d{\bf B}^1_s + \int_0^t (\theta^2_s)^\top d{\bf B}^2_s }\Bigg).
% \end{aligned}
% \end{equation}
% Therefore, the utility rewards for the surplus of $X^*$ over the pseudo fund $X^{\hat{\pi},0}$, while penalizing for the difference between $\pi^*$ and $\hat{\pi}$. The later reconciles with the admissibility required in \eqref{eq:admissible:exp}.  \\

%The optimal investment to salary ratio at time $t$, $\pi_t^*X_t^*$, can be expressed as % The optimal strategy $\pi_t^*$ can now be expressed as 
%\begin{equation}
%\label{eq:pi*:exp:decompose}
%\pi_t^*X_t^* = \frac{1}{\Gamma_t} (\Sigma^\top_t)^{-1}\lambda_t + (X_t^* - \Gamma_t^{-1}) \hat{\pi}_t,
%\pi^*_t = \frac{1}{X^*_tZ_t} (\Sigma^\top_t)^{-1}\lambda_t+ \left(1-\frac{1}{X_t^*Z_t} \right) \hat{\pi}_t = {\color{red} \frac{(\Sigma^\top_t)^{-1}( \lambda_t-\sigma^{Y,1}_t+\beta_t)}{X_t^*Z_t}  + \hat{\pi}_t },
%\end{equation}    
The worker’s optimal investment strategy in \eqref{eq:pi*:exp}, based on her exponential forward utility preference, is a combination of the baseline strategy $\hat{\pi}$ and a myopic strategy $\left(\Sigma^{\top}_t\right)^{-1}\left(\lambda_t+\theta^1_t\right)$, for $t\geq 0$. In the case of $\theta^1\equiv 0$, this myopic component resembles the optimal investment strategy under the Merton's classical investment problem. Different from the optimal investment strategy in \eqref{eq:pi*:power:1} derived from power forward utility preference, the optimal investment strategy in \eqref{eq:pi*:exp} is not necessarily being confined between the baseline strategy and the myopic strategy. \\
%{\color{blue} However, the trajectory of the quotient $\frac{\Gamma}{X^*}$ still guides if this optimal investment strategy converges to the baseline strategy.  To this end, by It\^o's lemma,
%\begin{align*}
%d\left(\frac{X^*_t}{\Gamma_t}\right)=&\;\left(\frac{p_t}{\Gamma_t}+\|\lambda_t-\sigma^{Y,1}_t-\beta_t\|^2+\left(\theta^1_t\right)^\top\left(\lambda_t-\sigma^{Y,1}_t-\beta_t\right)\right)dt\\&\;+\left(\lambda_t-\sigma^{Y,1}_t+\theta^1_t-\beta_t\right)^\top d{\bf B}^1_t.
%\end{align*}
 

%Though $\frac{X^*}{\Gamma}$ varies by the realization of the hedgeable risk ${\bf B}^1$, its average rate of change is governed by the drift. In particular, if, at any future time $t\geq 0$,
%\begin{equation}
%\left(\theta^1_t\right)^\top\left(\lambda_t-\sigma^{Y,1}_t-\beta_t\right)\geq-\frac{p_t}{\Gamma_t}-\|\lambda_t-\sigma^{Y,1}_t-\beta_t\|^2,
%\label{eq:convergent_condition}
%\end{equation}
%$\frac{\Gamma}{X^*}$ will tend to move towards zero, and thus the optimal investment strategy will tend to converge to the baseline strategy, in the long run. However, if \eqref{eq:convergent_condition} does not hold at some future time $t$, this momentum will be mitigated momentarily.}\\

% {\color{red} The exogenously chosen process $\theta^1$ also behaves as a regulation factor which modulates worker's preference on the baseline strategy $\hat{\pi}$. In particular, if $(\theta^1_t)^\top(\lambda_t-\Sigma^\top_t\hat{\pi}_t) \geq 0$ for all $t\geq 0$, the optimal strategy $\pi^*$ will converge to $\hat{\pi}$ asymptotically (in probability), and the convergence is bolstered by a high contribution rate $p$, a high deviation between the myopic and the baseline strategy, and the regulation $\theta^1$ with high magnitude. However, when $\theta^1$ is such that $(\theta^1_t)^\top(\lambda_t-\Sigma^\top_t\hat{\pi}_t) < 0$  for some $t\geq 0$, worker's preference for $\hat{\pi}$  at that moment will be mitigated, and her strategy will deviate from $\hat{\pi}$ momentarily when the magnitude of $\theta^1$ is high.}\\

% {\color{red}Covergent or divergent to $\hat{\pi}$? (diverge from $\hat{\pi}$ not necessarily mean converging to the myopic.  }\\

% {\color{red}The optimal strategy \eqref{eq:pi*:exp} consists of a myopic component $(\Sigma^\top_t)^{-1}(\lambda_t+\theta^1_t)/(X_t^*\Gamma^{-1}_t)$, and a baseline component $\hat{\pi}_t$ with weight $1-X^{-1}\Gamma$. {\color{red} When $\theta^1\equiv 0$,} the myopic component resembles the classical optimal investment strategy in usual financial markets under CARA utility when $\Gamma^{-1}\equiv \gamma$. The weight on the baseline strategy can be written as $1-\Gamma/X^*$, where the ratio $\Gamma/X^*$ measures the ratio between the optimal fund value and the one with zero contribution and initial wealth $\gamma^{-1}y$ under $\hat{\pi}$. If the initial risk of aversion $\gamma$ is high so that $\gamma^{-1}$ is small compared with $x_0$, the ratio $\Gamma/X^*$ will start off at a small value. This causes $\pi^*$ to lean towards the baseline strategy $\hat{\pi}$. Since there is no contribution on the pseudo fund $\Gamma$, the deviation between $X^*$ and $\Gamma$ will further be widen. This eventually leads to adherence of $\pi^*$ to $\hat{\pi}$. If, on the other hand, $\gamma$ is small so that $\gamma^{-1}$ is large compared with $x_0$, the ratio $\Gamma/X^*$ starts off high, the optimal strategy will first deviate from $\hat{\pi}$. With no contributions being made in the pseudo fund $\Gamma$, the difference of performance of $\Gamma$ and $X^*$ tends to be shrunken.  Eventually, this shrinkage will cause the optimal strategy to lean towards $\hat{\pi}$.  Indeed, by an application of It\^o's lemma, we see that for any $t\geq 0$, 
%     \begin{equation}
%     \label{eq:exp:XGamma}
%         d(X^*_t\Gamma^{-1}_t) = \left(p_t\Gamma^{-1}_t + (\lambda_t+{\color{red} \theta^1_t} - \Sigma^\top_t\hat{\pi}_t)^\top (\lambda_t-\Sigma^\top_t \hat{\pi}_t)  \right)dt + (\lambda_t+{\color{red} \theta^1_t}-\Sigma^\top_t \hat{\pi}_t)^\top d{\bf B}^1_t.
%     \end{equation}
% Hence, {\color{red} when $(\theta^1)^\top(\lambda-\Sigma^\top\hat{\pi})\geq 0$,} the weight $1-\Gamma/X$ tends to decrease with time and $\pi^*$ converges to $\hat{\pi}$ {\color{red} in probability} as $t\to\infty$. The speed of convergence depends on the  contribution rate $p$, and the deviations between the myopic strategy and $\hat{\pi}$.  In particular, a higher contribution rate $p$ will tend to lead to a faster convergence of $\pi^*$ to $\hat{\pi}$. {\color{red} This convergent behaviour will be opposed when  $(\theta^1)^\top(\lambda-\Sigma^\top\hat{\pi})\leq 0$. In other words, $\theta^1$ is a factor which mitigates or bolsters the convergence of $\pi^*$ to $\hat{\pi}$.     }}\\

A numerical example for exponential preferences shall be omitted, as the illustration of the stochasticity of the preference and the worker's optimal investment strategy are similar to that in Section \ref{sec:power:numeric}.



% the strategy depends on $B^2$ via Gamma!
% convergence is at least L^1. E|X\Gamma| \geq EX\Gamma -> \infty.

%The instantaneous proportion of salary invested at time $t$ according to the baseline strategy $\hat{\pi}_t$ is $X_t^*-\Gamma_t^{-1}$, which can be interpreted as the surplus of the optimal fund over the one under $\hat{\pi}_t$ with zero contribution rate ($p\equiv0$). Suppose that the contribution rate $p$ is sufficiently small. When the difference between the performance of $\pi^*$ and $\pi$ is small, the absolute difference $|X^*-\Gamma^{-1}|$ tends to be small, in which case $\pi^*$ mostly follows the myopic component. Subsequently, this will widen the difference $X^*-\Gamma^{-1}$, which eventually causes $\pi^*$ to lean towards $\hat{\pi}$. Therefore, under exponential forward utility, the optimal strategy oscillates about the baseline strategy $\hat{\pi}$.  This is also reflected in the utility process \eqref{eq:U:exp:X*}, which strikes a trade-off between seeking for a higher surplus of performance of $\pi^*$ over $\hat{\pi}$ while minimizing the deviation between the two.      %The dynamics of the difference is given by 
%which is composed of  a myopic strategy $\frac{(\Sigma^\top_t)^{-1}(\lambda_t-\sigma^{Y,1}_t+\beta_t)}{X_t^*Z_t}$ and the baseline strategy $\hat{\pi_t}$.
%\begin{multline}
%\label{eq:exp:x-z}
%d(X_t^* - \Gamma_t^{-1}) = p_t dt + (X_t^*-\Gamma_t^{-1})   \bigg( \big(\lambda^\top_t\sigma^{Y,1}_t + \|\sigma^{Y,2}_t\|^2 - \mu^Y_t -  \beta^\top_t(\lambda_t-\sigma^{Y,1}_t) \big) dt -\\ \beta^\top_t d{\bf B}^1_t - (\sigma^{Y,2}_t)^\top d{\bf B}_t^2)\bigg) +  \Gamma_t^{-1}(\lambda_t-\sigma^{Y,1}_t+\beta_t)^\top\left( (\lambda_t-\sigma^{Y,1}_t)dt - d{\bf B}^1_t \right).
%\end{multline}

%This findings echoes with the result in e.g. \cite{angoshtari:2020} when the power utility is considered, except for now the strategy depends dynamically on $Z_t$ instead of a constant risk of aversion. Notice that the strategy does not depend on the non-hedgeable risk. 

% \subsection{Numerical Examples} 
% \label{sec:exp:numeric}
% We use the parameters in Section \ref{sec:power:numeric} with $\gamma=0.5$ and consider the cases $\beta=0.5$ and $-0.5$. We also set $\theta^1=0=\theta^2$. As before, we consider two different realizations of both $B^1$ and $B^2$ (see Figure \ref{fig:B}) and study the effect of each of the Brownian motions on the utility process and the optimal investment strategy.


% \subsubsection{Forward Preferences}
% In Figure \ref{fig:exp:utility:market}, we  plot $-\log(-U(x,t))$ as a function of $x$ for a spectrum of $t$ under the two realizations of $B^1$ (see Figure \ref{fig:B1}), and similarly for the two realizations of $B^2$ (see Figure \ref{fig:B2}) in  Figure \ref{fig:exp:utility:salary}. The utilities at different time are distinguished by the colors of the lines. In the figures, the slope of the line which corresponds to time $t$ represents the value of the realization of $\Gamma_t$, while the $y$-intercept indicates the realization of $-V_t$. Driven by the stochastic factors, the utility could vary significantly from its initial value (the deep blue lines in Figure \ref{fig:exp:utility:market}-\ref{fig:exp:utility:salary}). As a benchmark, the utility at $x=0$ is strictly decreasing with time since $V$ is strictly increasing. Herein, we analyze the variations per market and salary conditions.      \\

% \textbf{Effect of Hedgeable Risk $B^1$}\\ 
% The parameter $\beta$ determines the worker's appetite for the risk in the financial market. 
% When $\beta = -0.5<0$, the worker tends to be less risk averse when favourable conditions to the market $B^{1,U}$ emerge. This can be reflected by the reduction of the slopes of lines (i.e., the realizations of $\Gamma$) in Figure \ref{fig:exp:U:UM:B-05}.    In contrast,  values of $\Gamma$ tend to be relatively stable under $B^1=B^{1,S}$. This also results in greater values of $V$ compared to the case of $B^{1,U}$, which is illustrated by the $y$-intercepts of the lines in Figures \ref{fig:exp:U:UM:B-05}-\ref{fig:exp:U:DM:B-05}.  Comparing the two paths $B^{1,U}$ and $B^{1,S}$, we see that the reduction of $\Gamma$ will lead to a smaller growth rate of the process $V$, whence mitigating the decline of the utility level. However, unless the contribution rate $p$ is exceptionally high, the drop of $\Gamma$ tends to dominate the effect of $V$, whence workers will desire a higher $X$ to maintain the same level of utility.    \\  

% %beta = -0.5, deviation between \hat{\pi} and mypopic is small => small reduction in the utility effect of Gamma and deviation of policies. 

% When $\beta=0.5>0$, the worker's risk of aversion is in sync with the movement of $B^1$. Therefore, favourable market condition $B^{1,U}$ will lead to a substantial increase of $\Gamma$, which subsequently reduces the investment proportion on the myopic component in the optimal investment strategy. Comparing the two choices of $\beta$, we see that the deviations between the myopic strategy and the baseline strategy, $(\Sigma^\top)^{-1}\lambda-\hat{\pi}$, are respectively $-0.25$ for $\beta=-0.5$, and $4.75$ for $\beta=0.5$.  Therefore, a significantly higher penalty is incurred in the utility process for $\beta=0.5$, which can be partly reflected by the magnitude of the $y$-intercepts in Figures \ref{fig:exp:U:UM:B05} and \ref{fig:exp:U:DM:B05}. \\

% %This can be reflected by the increase in slopes in Figure \ref{fig:exp:U:UM:B05} under the favourable market condition $B^{1,U}$. This also significantly reduces the optimal utility process at $X^*$ as compared to the case of $B^{1,S}$ in Figure \ref{fig:exp:U:DM:B05}. Notice that the utility processes are significantly lower than those for $\beta=-0.5$. This can be seen by the explicit solution to $Z$, 
% %\begin{equation}
% %Z_t = \gamma \exp\left( \left(\mu^Y -\lambda\sigma^{Y,1} - \frac{(\sigma^{Y,2})^2}{2}+( \lambda-\sigma^{Y,1})\beta + \frac{\beta^2}{2} \right) t + \beta B^1_t + \sigma^{Y,2} B^2_t  \right), 
% %\end{equation}
% %where under the choice of the parameters, $ \mu^Y -\lambda\sigma^{Y,1} - \frac{(\sigma^{Y,2})^2}{2}+( \lambda-\sigma^{Y,1})\beta + \frac{\beta^2}{2}$ is negative when $\beta=-0.5$; and is positive when $\beta=0.5$. Therefore, $Z$ experiences an exponential growth for $\beta=0.5$. \\



% \begin{figure}[!h]
% \centering
% \begin{subfigure}{.5\textwidth}
% \centering
% \includegraphics[scale=0.4]{Exp_U_UM_bm05.png}
% \caption{$B^1=B^{1,U}$, $\beta=-0.5$.}
% \label{fig:exp:U:UM:B-05}
% \end{subfigure}%
% \begin{subfigure}{.5\textwidth}
% \centering
% \includegraphics[scale=0.4]{Exp_U_DM_bm05.png}
% \caption{$B^1=B^{1,S}$, $\beta=-0.5$.}
% \label{fig:exp:U:DM:B-05}
% \end{subfigure}%


% \begin{subfigure}{.5\textwidth}
% \centering
% \includegraphics[scale=0.4]{Exp_U_UM_b05.png}
% \caption{$B^1=B^{1,U}$, $\beta=0.5$.}
% \label{fig:exp:U:UM:B05}
% \end{subfigure}%
% \begin{subfigure}{.5\textwidth}
% \centering
% \includegraphics[scale=0.4]{Exp_U_DM_b05.png}
% \caption{$B^1=B^{1,S}$, $\beta=0.5$.}
% \label{fig:exp:U:DM:B05}
% \end{subfigure}%


% \caption{Realizations of utility process $U(x,t)$ as function of wealth and time subject to different   conditions of $B^1$.    }
% \label{fig:exp:utility:market}
% \end{figure}

% \textbf{Effect of Non-hedgeable Risk $B^2$}\\ 
% Unlike the hedgeable risk $B^1$, the worker's risk of aversion $\Gamma$ is always in sync with the non-hedgeable risk $B^2$. Hence, the values of $\Gamma$ tend to be greater under the favourable condition $B^{2,U}$ than that of $B^{2,D}$; see Figure \ref{fig:exp:utility:salary}. A decreasing salary $Y$ due to $B^{2,D}$ will boost up $X^*=W^*/Y$, since (i) a reduction of $Y$ immediately increases $X^*$; and (ii) a reduction of $\Gamma$ leads to a drop of the utility, whence workers will desire a higher $X^*$ to maintain the same level of utility. 
% Similarly to the power forward utility, although an instantaneous drop of salary leads to an increase workers' ability to maintain the current living standard, their level of satisfaction is negatively impacted by the drop of the salary. {\color{red}  This  reduces the risk aversion and encourages workers to invest more on the myopic component as compared to the case $B^{2,U}$.} % drop in salary means a drop of workers' expected living standard => the pension fund is sufficient to maintain the current living standard => drop in risk aversion and encourages investment. 


% \begin{figure}[!h]
% \centering
% \begin{subfigure}{.5\textwidth}
% \centering
% \includegraphics[scale=0.4]{Exp_U_US_bm05.png}
% \caption{$B^2=B^{2,U}$, $\beta=-0.5$.}
% \label{fig:exp:U:US:B-05}
% \end{subfigure}%
% \begin{subfigure}{.5\textwidth}
% \centering
% \includegraphics[scale=0.4]{Exp_U_DS_bm05.png}
% \caption{$B^2=B^{2,D}$, $\beta=-0.5$.}
% \label{fig:exp:U:SS:B-05}
% \end{subfigure}%

% \begin{subfigure}{.5\textwidth}
% \centering
% \includegraphics[scale=0.4]{Exp_U_US_b05.png}
% \caption{$B^2=B^{2,U}$, $\beta=0.5$.}
% \label{fig:exp:U:US:B05}
% \end{subfigure}%
% \begin{subfigure}{.5\textwidth}
% \centering
% \includegraphics[scale=0.4]{Exp_U_DS_b05.png}
% \caption{$B^2=B^{2,D}$, $\beta=0.5$.}
% \label{fig:exp:U:SS:B05}
% \end{subfigure}%



% \caption{Realizations of utility process $-\log(-U(x,t))$ as function of wealth and time subject to different conditions of $B^2$.   }
% \label{fig:exp:utility:salary}
% \end{figure} 


% \subsubsection{Optimal Investment Strategy}
% The optimal investment strategies under different realizations of $B^1$ and $B^2$ are shown in Figures \ref{fig:exp:pi:market} and \ref{fig:exp:pi:salary}, respectively. \\
% %We consider the optimal investment amount to salary ratio \eqref{eq:pi*:exp:decompose} in terms of the decomposition 
% %\begin{equation}
% %\pi_t^*X_t^* = \underbrace{\frac{\lambda }{\sigma_1\Gamma_t}}_{\text{Myopic component}} + \underbrace{(X_t^*-\Gamma_t^{-1})\hat{\pi}_t}_{\text{Baseline component}},
% %\end{equation} 
% %where $\Gamma^{-1}$ and $X^*-\Gamma^{-1}$ can be considered as the amount to salary ratio invested according to the myopic strategy $\frac{\lambda}{\sigma_1}$ and the baseline strategy $\hat{\pi}$, respectively. \\ 

% \textbf{Effect of Hedgeable Risk $B^1$}\\ 
% As mentioned in Section \ref{sec:exp}, the optimal investment strategy $\pi^*$ tends to converge to $\hat{\pi}$ asymptotically. When $\beta=0.5$, we have $(\Sigma^\top)^{-1}\lambda-\hat{\pi}=4.75>0$. Hence, by referring to the equation \eqref{eq:exp:XGamma}, we see that $\pi^*$ will converge to $\hat{\pi}$ faster under favour market condition $B^{1,U}$ than its counterpart $B^{1,S}$; see Figure \ref{fig:exp:Pi:M:B05}. The opposite is observed when $\beta=-0.5$, where  $(\Sigma^\top)^{-1}\lambda-\hat{\pi}=-0.25<0$. Hence, the convergence of $\pi^*$ to $\hat{\pi}$ is opposed by $B^{1,U}$. Comparing the two values of $\beta$, we see that $\pi^*$ converges more rapidly to $\hat{\pi}$ when $\beta=0.5$, as a result of a greater absolute deviation $|(\Sigma^\top)^{-1}\lambda-\hat{\pi}|$  than that of the case $\beta=-0.5$.  \\

% %When $\beta=-0.5$, the worker's appetite for risk is opposite to the movement of $B^1$. Hence, $Z$ tends to decrease under the upward force $B^{1,U}$, which subsequently increases the myopic component.  {\color{red}Loosely speaking}, based on \eqref{eq:exp:x-z}, a negative $\beta$ also tends to boost up drastically the amount to salary $X^*-Z^{-1}$ invested on the baseline strategy under $B^{1,U}$. Since $\hat{\pi}>0$ when $\beta=-0.5$, we see that both the myopic component and baseline component   are more substantial under $B^{1,U}$ than its relatively stable counterpart $B^{1,S}$ (compare Figures \ref{fig:exp:XZ:UM:B-05} and \ref{fig:exp:XZ:DM:B-05}). \\
% %For $\beta=-0.5<0$, we have $\lambda-\sigma_{Y_1}+\beta <0$ whence the myopic component is always negative. The upward driving force $B^{1,U}$ of the risky asset tend to reduce $Z_t$, thus the myopic component is more negative that that of $B^{1,S}$. This effect is outweighed by the rise of $X_t^*$ (see \eqref{eq:exp:X^*} and Figures \ref{fig:exp:U:UM:B-05}-\ref{fig:exp:U:DM:B-05}). Indeed, by considering \eqref{eq:exp:X^*}, a negative $\beta$ along with an upward moving $B^1$ tend to increase $X_t^*$. Since $\hat{\pi}_t =(\sigma_{Y_1}-\beta)/\sigma_1>0$, $B^{1,U}$ ultimately boosts up the baseline component $\pi_t^*X_t^*$. In both market conditions, the overall investments are positive since the risk of aversion is opposite to that of the $B^1$, which encourages one to take long position under stable or upward market movement.   \\

% %The optimal $\pi^*X^*$ for $\beta = 0.5>0$ is illustrated in Figure \ref{fig:exp:Pi:M:B05}. Notice that the upward force $B^{1,U}$  increases $Z$ and leads to a smaller the myopic component compared with the case of $B^{1,S}$. On the contrary, since a positive $\beta$ with $B^{1,U}$ tends to reduce the difference $X^*-Z^{-1}$, along with the fact that $\hat{\pi}<0$ for $\beta=0.5$,  the magnitude of the baseline component is less than that of the case of $B^{1,S}$ (compare Figures \ref{fig:exp:XZ:UM:B05} and \ref{fig:exp:XZ:DM:B05}). Overall, the effect of the baseline components outweighs that of the myopic components, resulting negative investments in both market conditions with a higher short-selling ratio for the case of $B^{1,S}$. \\

% %Regardless the choice of $\beta$, the investment to salary ratio tends to be higher under $B^{1,U}$ (i.e., the higher price of the risky asset) for a clear rationale. \\

% \begin{figure}[!h]
% \centering
% \begin{subfigure}{.5\textwidth}
% \centering
% \includegraphics[scale=0.4]{Exp_Pi_M_bm05.png}
% \caption{$\beta=-0.5$.}
% \label{fig:exp:Pi:M:B-05}
% \end{subfigure}%
% \begin{subfigure}{.5\textwidth}
% \centering
% \includegraphics[scale=0.4]{Exp_Pi_M_b05.png}
% \caption{$\beta=0.5$.}
% \label{fig:exp:Pi:M:B05}
% \end{subfigure}%


% \caption{Realizations of  $\pi^*$ under different conditions of $B^1$ and $\beta$.  }
% \label{fig:exp:pi:market}
% \end{figure}

% %\begin{figure}[!h]
% %\centering
% %\begin{subfigure}{.5\textwidth}
% %\centering
% %\includegraphics[scale=0.4]{Exp_X-Z_UM_bm05.png}
% %\caption{$B^1=B^{1,U}$, $\beta=-0.5$.}
% %\label{fig:exp:XZ:UM:B-05}
% %\end{subfigure}%
% %\begin{subfigure}{.5\textwidth}
% %\centering
% %\includegraphics[scale=0.4]{Exp_X-Z_DM_bm05.png}
% %\caption{$B^1=B^{1,S}$, $\beta=-0.5$.}
% %\label{fig:exp:XZ:DM:B-05}
% %\end{subfigure}%


% %\begin{subfigure}{.5\textwidth}
% %\centering
% %\includegraphics[scale=0.4]{Exp_X-Z_UM_b05.png}
% %\caption{$B^1=B^{1,U}$, $\beta=0.5$.}
% %\label{fig:exp:XZ:UM:B05}
% %\end{subfigure}%
% %\begin{subfigure}{.5\textwidth}
% %\centering
% %\includegraphics[scale=0.4]{Exp_X-Z_DM_b05.png}
% %\caption{$B^1=B^{1,S}$, $\beta=0.5$.}
% %\label{fig:exp:XZ:DM:B05}
% %\end{subfigure}%


% %\caption{Amount to salary $X^*-Z^{-1}$ and $Z^{-1}$ invested according to the myopic and baseline strategy respectively, under different conditions of $B^1$.    }
% \label{fig:exp:xz:market}
% %\end{figure}





% \textbf{Effect of Non-hedgeable Risk $B^2$}\\ 
% By \eqref{eq:exp:XGamma}, the process $B^2$ affects the optimal investment strategy via $\Gamma$. Since $\Gamma$ is positively correlated with $B^2$, favourable labour condition $B^{2,U}$ will result in a higher risk of aversion $\Gamma$ than that of $B^{2,D}$. Hence, the strategy $\pi^*$ will converge faster to $\hat{\pi}$ under $B^{2,U}$. Although workers are more able to maintain the current living standard under $B^{2,D}$ than that of $B^{2,U}$, due to a lower salary in the former case,  unfavourable labour condition $B^{2,D}$ lead to a decline of the level of utility. Workers then become less risk averse and are more willing to invest more based on the myopic strategy $(\Sigma^\top)^{-1}\lambda$.  This also agrees with the findings for forward power utility.  


% %The optimal investment amount to salary ratios are demonstrated in Figure \ref{fig:exp:pi:salary}. Since the sensitivity of $Z$ towards $B^2$ is always positive, the worker's risk of aversion $Z$ is greater under the upward force $B^{2,U}$ on salary, which results in a smaller myopic component as compared to the case of $B^{2,D}$. In addition, the upward force $B^{2,U}$ tends to reduce the investment amount to salary $X^*-Z^{-1}$ on $\hat{\pi}$, which results in a smaller magnitude of the baseline component (compare Figure \ref{fig:exp:XZ:US:B-05} with Figure \ref{fig:exp:XZ:DS:B-05}; and Figure \ref{fig:exp:XZ:US:B05} and \ref{fig:exp:XZ:DS:B05}). When $\beta=-0.5<0$, $\hat{\pi}>0$ whence both components and the overall ratio $\pi^*X^*$ is smaller under $B^{2,U}$. When $\beta=0.5$, $\hat{\pi}<0$ and the downward force $B^{2,D}$ results in a higher short selling ratio compared with $B^{2,U}$. This outweighs the effect of the myopic component and leading to a high short selling ratio under $B^{2,U}$.     \\

% %In both choices of $\beta$, we see that the investment to salary ratio tends to have a smaller magnitude when $B^2$ is moving upwards. This is clear since it tends to give a lower salary whence increases the ratio to salary. It also encourages one to invest more as reflected by the smaller realization values of the risk of aversion $Z$ under $B^{2,D}$. \\%Therefore, $Z$ grows exponentially for $\beta=0.5$ under the given simulated paths of $B^1$ and $B^2$.  






% %\begin{itemize}
% %    \item When $\beta = -0.5$, the drift of $Z^{-1}$ is positive, hence on average $1/Z^{-1}$ is increasing (i.e., the dynamic risk of aversion is decreasing). Therefore, the investment to salary ratio tends to increase with time. The opposite holds for $\beta=0.5$. 

% %    \item $\lambda$: for $\beta=-0.5$, $\sigma_{Y_1}-\beta>0$, hence $Z_t^{-1}$ tends to increase with $\lambda$, whence both the myopic and baseline components increase with $\lambda$, i.e., a good market condition encourages one to invest more. When $\beta=0.5$, $\sigma_{Y_1}-\beta<0$ and $Z_t^{-1}$ tends to decrease with $\lambda$. In this case, the employee's attitude is going against the market with an increasing risk of aversion with good market condition. Hence, the myopic component eventually decreases with $\lambda$. This eventually reduces $X_t$, which reduces the magnitude of the baseline component, boosting up the investment to salary ratio. 

% %   \item $\mu_Y$: since our goal is to maintain a good fund to salary ratio, when $\mu_Y$ increases, the ratio $X_t$ tends to decrease, and at the same time our risk of aversion to the fund increases. Therefore, the investment amount to salary ratio tends decrease with $\mu_Y$ for the myopic component. The difference $X_t-Z_t^{-1}$ tends to decrease with $\mu_Y$, whence the magnitude of the baseline component is also decreasing with $\mu_Y$. Note that the baseline component is negative when $\beta=0.5$.


% %    \item $\sigma_{Y_2}$: Again since we are considering the risk aversion of the fund to salary ratio. An increase of uncertainty on the salary tends to lower the risk aversion on average, whence increasing the magnitude of investment to both the myopic and baseline components, since it encourages one to invest more on the fund to prepare for the uncertainty of the future salary. 

% %    \item $\sigma_{Y_1}$: The myopic component is independent of $\sigma_{Y_1}$. Following the same reasoning for $\sigma_{Y_2}$, we see that the myopic component tends to increase with $\sigma_{Y_1}$. If we just fix $\beta$, one sees that the baseline component increases with $\sigma_{Y_1}$ initially, and the differences reduce and eventually the order of investment to salary ratio is reversed, due to the increases of $X_t-Z_t^{-1}$ for greater $\sigma_{Y_1}$. 

% %\end{itemize}



% \begin{figure}[!h]
% \centering
% \begin{subfigure}{.5\textwidth}
% \centering
% \includegraphics[scale=0.4]{Exp_Pi_S_bm05.png}
% \caption{ $\beta=-0.5$.}
% \label{fig:exp:S:Pi:B05}
% \end{subfigure}%
% \begin{subfigure}{.5\textwidth}
% \centering
% \includegraphics[scale=0.4]{Exp_Pi_S_b05.png}
% \caption{$\beta=0.5$.}
% \label{fig:exp:S:Pi:B-05}
% \end{subfigure}%


% \caption{Realizations   $\pi^*$ under different conditions of $B^2$ and $\beta$. }
% \label{fig:exp:pi:salary}
% \end{figure}

% %\begin{figure}[!h]
% %\centering
% %\begin{subfigure}{.5\textwidth}
% %\centering
% %\includegraphics[scale=0.4]{Exp_X-Z_US_bm05.png}
% %\caption{$B^2=B^{2,U}$, $\beta=-0.5$.}
% %\label{fig:exp:XZ:US:B-05}
% %\end{subfigure}%
% %\begin{subfigure}{.5\textwidth}
% %\centering
% %\includegraphics[scale=0.4]{Exp_X-Z_DS_bm05.png}
% %\caption{$B^2=B^{2,D}$, $\beta=-0.5$.}
% %\label{fig:exp:XZ:DS:B-05}
% %\end{subfigure}%


% %\begin{subfigure}{.5\textwidth}
% %\centering
% %\includegraphics[scale=0.4]{Exp_X-Z_US_b05.png}
% %\caption{$B^2=B^{2,U}$, $\beta=0.5$.}
% %\label{fig:exp:XZ:US:B05}
% %\end{subfigure}%
% %\begin{subfigure}{.5\textwidth}
% %\centering
% %\includegraphics[scale=0.4]{Exp_X-Z_DS_b05.png}
% %\caption{$B^2=B^{2,D}$, $\beta=0.5$.}
% %\label{fig:exp:XZ:DS:B05}
% %\end{subfigure}%


% %\caption{Amount to salary $X^*-Z^{-1}$ and $Z^{-1}$ invested according to the myopic and baseline strategy respectively, under different conditions of $B^2$.    }
% %\label{fig:exp:xz:salary}
% %\end{figure}


 


% %\subsection{Relative Performance}
% %Next we consider 
% %    \begin{equation}
% %        U(x,t) = -e^{-\gamma (x+Z_t) + V_t  },
% %    \end{equation}
% %where 
% %    \begin{align*}
% %        dZ_t &= -p(t)dt + Z_t(\alpha_t dt + \beta^1_t dB^1_t + \beta^2_t dB^2_t). 
% %    \end{align*}
% %{\color{red} Not solvable}


% \iffalse %%%%%%% previous calculations for exp
% Now if we consider $\sigma_{Y_1} > 0$ and $\lambda \neq \sigma_{Y_1}$, set 
% \begin{align*}
% \beta^1_t := \frac{\sigma_1}{(\sigma_1-\sigma_{Y_1})(\lambda-\sigma{Y_1})}\left( \sigma_{Y_2}^2  + \alpha_t - \mu_Y + \lambda \sigma_{Y_1} \right), \ \beta^2_t := \sigma_{Y_2}, \ v_t := \frac{(\lambda-\sigma_{Y_1})^2}{2}
% \end{align*}
% and $\alpha_t$ is (a) solution to the quadratic equation $A\alpha^2 + B\alpha + C=0$ ({\color{red} Given that exists}), where 
% \begin{align*}
% A &= \sigma_{Y_1}(\sigma_{Y1}-2\sigma_1), \\
% B &= (\sigma_1-\sigma_{Y_1})^2 (\sigma_{Y_1}^2+\sigma_{Y_2}^2-\mu_Y) - \sigma_1^2(\sigma_{Y_2}^2-\mu_Y + \lambda\sigma_{Y_1}),\\
% C&= (\sigma_1-\sigma_{Y_1})^2(\sigma_{Y_1}^2+\sigma_{Y_2}^2-\mu_Y)^2 - \sigma_1^2(\sigma_{Y_2}^2-\mu_Y + \lambda\sigma_{Y_1})^2 - \sigma_{Y_1}^2(\lambda-\sigma_{Y_1})^2.
% \end{align*}
% Then the drift term becomes 
% \begin{equation}
% \frac{(\lambda-\sigma_{Y_1})^2}{2}\left(  \frac{X_tZ_t}{\lambda-\sigma_{Y_1}}\left(   \pi_t \lambda\sigma_1 - \mu_Y + \sigma_{Y_1}^2 + \sigma_{Y_2}^2 +\alpha_t  \right)-1 \right)^2,
% \end{equation}
% whence the optimal investment strategy reads
% \begin{equation}
% \pi^*_t := \frac{\lambda-\sigma_{Y_1}}{\lambda X_tZ_t} + \frac{\mu_Y-\sigma_{Y_1}^2 -\sigma_{Y_2}^2-\alpha_t }{\lambda\sigma_1}. 
% \end{equation}
% \fi 

% \iffalse %%% trial on BSDE
% \subsection{Construction by BSDE}
% Consider an \textit{ansatz} 
% \begin{equation}
% U(x,t) = -e^{-\gamma x + V_t + A_t },
% \end{equation}
% where 
% \begin{align*}
% dV_t &= - \alpha_t dt + \beta^1_t dB^1_t + \beta^2_t dB^2_t, \\
% dA_t &= ??? dt .
% \end{align*}
% We show that $U(X_t,t)$ is a forward utility performance criterion. For $0\leq s <t$, consider 
% \begin{align*}
% \E_s\left[ U(X_t,t)  \right] &= U(X_s,s) \E_s\left[ e^{\int_s^t F(X_u,u)du + \int_s^t G_1(X_u,u)    dB^1_u + G_2(X_u,u) dB^2_u  } \right] \\
% &= U(X_s,s) \E_s\left[ e^{  \int_s^t \left( F(X_u,u) + \frac{G^2_1(X_u,u) + G^2_2(X_u,u) }{2} \right) du}  \rho_{s,t}      \right]
% \end{align*}
% where 
% \begin{align*}
% F(X_t,t) &:=   -\gamma \left(p_t + X_t \left( \pi_t\sigma_1(\lambda-\sigma_{Y_1}) - \mu_Y + \sigma_{Y_1}^2 + \sigma_{Y_2}^2 \right) \right) +\alpha_t, \\
% G_1(X_t,t) &:= \beta^1_t - \gamma X_t(\pi_t\sigma_1-\sigma_{Y_1}),  \\
% G_2(X_t,t) &:=\beta^2_t + \gamma\sigma_{Y_2} X_t ,\\
% \rho_{s,t} &:= \exp\left( -\frac{1}{2} \int_s^t (G^2_1(X_u,u) + G^2_2(X_u,u) ) du + \int_s^t G_1(X_u,u)    dB^1_u + G_2(X_u,u) dB^2_u  \right).
% \end{align*}
% Under some regularity conditions, $\rho_{s,t}$ defines a Radon-Nikodym derivative, whence
% \begin{equation*}
% \E_s\left[ U(X_t,t)  \right] = U(X_s,s) \E_s^{\mathcal{Q}}\left[  e^{  \int_s^t \left( F(X_u,u) + \frac{G^2_1(X_u,u) + G^2_2(X_u,u) }{2} \right) du}  \right].
% \end{equation*}
% Consider that 
% \begin{align*}
% F(X_t,t) + \frac{G^2_1(X_t,t) + G^2_2(X_t,t) }{2}  
% = \frac{\gamma^2X_t^2\sigma_1^2}{2}\left( \pi - \left( \frac{\sigma_{Y_1}}{\sigma_1} + \frac{\beta_1 + \lambda - \sigma_{Y_1} }{\gamma\sigma_1 X } \right) \right)^2 + H(X_t,t),
% \end{align*}
% where 
% \begin{align*}
% H(X_t,t) &= \gamma X_t(\sigma_{Y_1}-\lambda) - \frac{(\beta_1 + \lambda-\sigma_{Y_1})^2}{2} + \frac{\beta_1^2+(\beta_2+ \gamma \sigma_{Y_2}X_t)^2}{2}.
% \end{align*}
% \fi %%%%%


% \iffalse %%%%%%%%%%%%% hidden section 5
% \section{Relationship With Relative Performance}
% \label{sec:relation}
% In the above constructions, we see that the additive stochastic factor always appears as the fund to salary ratio under an exogenously given baseline strategy $\hat{\pi}$. Motivated by this observation, given $\hat{\pi}$, we consider the constructions of forward utility performance processes on the relative performance $\Tilde{X}^\xi = X - \hat{X}$ with dynamics \eqref{eq:tilde:X}, and find the optimal control $\xi^*$ which leads to the corresponding optimal investment strategy $\pi^*$ via \eqref{eq:pi:transform}.  We shall mainly consider the case for power utility, where the admissible set for $\xi$ is given by \eqref{eq:admissible:xi}. 

% \subsection{A SPDE Based on $\Tilde{X}_t$}
% As in Section \ref{sec:SPDE}, we first consider a generic process 
% \begin{equation}
% dU(x,t) = b(x,t) dt + a_1^\top(x,t) d{\bf B}^1_t + a_2^\top(x,t) d{\bf B}^2_t,
% \end{equation}
% where $b,a_1,a_2$ are respectively $\mathbb{R},\mathbb{R}^n$ and $\mathbb{R}^m$-valued $\mathbb{F}$-adapted processes. By It\^o-Wentzell formula, 
% \begin{align}
% dU(\Tilde{X}_t,t) &= b(\Tilde{X}_t,t) dt + a_1^\top(\Tilde{X}_t,t) d{\bf B}^1_t + a_2^\top(\Tilde{X}_t,t) d{\bf B}^2_t + U_x(\Tilde{X}_t,t)d\Tilde{X}_t \nonumber \\
% &\ \ + \frac{1}{2}U_{xx}(\Tilde{X}_t,t) d\langle \Tilde{X}\rangle_t + \Tilde{X}_t(\xi_t^\top \nabla_xa_1(\Tilde{X}_t,t) - (\sigma^{Y,2}_t)^\top\nabla_xa_2(\Tilde{X}_t,t)) dt \nonumber \\
% &= \Big( b(\Tilde{X}_t,t)  +  U_x(\Tilde{X}_t,t)\Tilde{X}_t\left(\xi_t^\top   (\lambda_t-\sigma^{Y,1}_t) - \mu^Y_t + \lambda^\top_t\sigma^{Y,1}_t+\|\sigma^{Y,2}_t\|^2  \right) \nonumber \\
% & \ \ + \frac{\Tilde{X}_t^2}{2}U_{xx}(\Tilde{X}_t,t)(\|\xi_t\|^2 + \|\sigma^{Y,2}_t\|^2) + \Tilde{X}_t(\xi_t^\top \nabla_xa_1(\Tilde{X}_t,t) - (\sigma^{Y,2}_t)^\top\nabla_xa_2(\Tilde{X}_t,t))  \Big) dt \nonumber \\
% &\ \  + \left( a_1^\top(\Tilde{X}_t,t) - U_x(\Tilde{X}_t,t)\Tilde{X}_t\tilde{\xi}^\top_t \right) d{\bf B}^1_t + \left( a_2^\top(\Tilde{X}_t,t) - (\sigma^{Y,2}_t)^\top U_x(\Tilde{X}_t,t) \Tilde{X}_t \right)d{\bf B}^2_t \nonumber \\
% &= \Bigg( \frac{\Tilde{X}_t^2 \|\xi_t\|^2}{2}U_{xx}(\Tilde{X}_t,t) + \Tilde{X}_t\xi_t^\top\left( (\lambda_t-\sigma^{Y,1}_t)  U_x(\Tilde{X}_t,t)  + \nabla_xa_1(\Tilde{X}_t,t) \right) + \nonumber \\ & \ \   \frac{\Tilde{X}_t^2 \|\sigma^{Y,2}_t\|^2}{2}U_{xx}(\Tilde{X}_t,t) +  \Tilde{X}_t\Big( U_x(\Tilde{X}_t,t)\big(\lambda^\top_t\sigma^{Y,1}_t +\|\sigma^{Y,2}_t\|^2 - \mu^Y_t) - \nonumber \\ & \ \ (\sigma^{Y,2}_t)^\top \nabla_x a_2(\Tilde{X}_t,t)  \Big) + b(\Tilde{X}_t,t)   \Bigg) dt  
% + \left( a_1^\top(\Tilde{X}_t,t) - U_x(\Tilde{X}_t,t)\Tilde{X}_t\xi_t \right) d{\bf B}^1_t \nonumber + \\ & \ \   \left( a_2^\top(\Tilde{X}_t,t) - (\sigma^{Y,2}_t)^\top U_x(\Tilde{X}_t,t) \Tilde{X}_t \right)d{\bf B}^2_t \nonumber \\
% &= \Bigg(  \frac{U_{xx}(\Tilde{X}_t,t)}{2}\left\| \Tilde{X}_t\xi_t + \frac{\nabla_xa_1(\Tilde{X}_t,t) + (\lambda_t-\sigma^{Y,1}_t)U_x(\Tilde{X}_t,t)}{U_{xx}(\Tilde{X}_t,t)  }  \right\|^2 - \nonumber \\ &\ \   \frac{\left\|\nabla_xa_1(\Tilde{X}_t,t) + (\lambda_t-\sigma^{Y,1}_t)U_x(\Tilde{X}_t,t) \right)^2 }{2}\  + \frac{\Tilde{X}_t^2 \|\sigma^{Y,2}_t\|^2}{2}U_{xx}(\Tilde{X}_t,t) +\nonumber \\& \ \  \Tilde{X}_t\left( U_x(\Tilde{X}_t,t)(\lambda^\top_t\sigma^{Y,1}_t +\|\sigma^{Y,2}_t\|^2 - \mu^Y_t) - (\sigma^{Y,2}_t)^\top\nabla_xa_2(\Tilde{X}_t,t)  \right) + b(\Tilde{X}_t,t)   \Bigg) dt \nonumber \\
% &\ \  + \left( a_1^\top(\Tilde{X}_t,t) - U_x(\Tilde{X}_t,t)\Tilde{X}_t\xi_t \right) d{\bf B}^1_t+ \left( a_2^\top(\Tilde{X}_t,t) - (\sigma^{Y,2}_t)^\top U_x(\Tilde{X}_t,t) \Tilde{X}_t \right)d{\bf B}^2_t   .
% \end{align}
% Hence, by setting 
% \begin{multline}
% b(x,t) := \frac{\left\|\nabla_xa_1(x,t) + (\lambda_t-\sigma^{Y,1}_t)U_x(x,t) \right\|^2}{2U_{xx}(x,t)} - x\Big( U_x(x,t)(\lambda^\top_t\sigma^{Y,1}_t+\|\sigma^{Y,2}_t\|^2 - \\ \mu^Y_t )  - (\sigma^{Y,2}_t)^\top\nabla_xa_2(x,t)\Big)  -\frac{x^2\|\sigma^{Y,2}_t\|^2U_{xx}(x,t)}{2},
% \end{multline}
% we arrive at 
% \begin{multline}
% dU(\Tilde{X}_t,t) =    \frac{U_{xx}(\Tilde{X}_t,t)}{2}\left\| \Tilde{X}_t\xi_t + \frac{\nabla_xa_1(\Tilde{X}_t,t) + (\lambda_t-\sigma^{Y,1}_t)U_x(\Tilde{X}_t,t)}{U_{xx}(\Tilde{X}_t,t)  }  \right\|^2dt + \\
% \left( a_1^\top(\Tilde{X}_t,t) - U_x(\Tilde{X}_t,t)\Tilde{X}_t\xi_t \right) d{\bf B}^1_t+ \left( a_2^\top(\Tilde{X}_t,t) - (\sigma^{Y,2}_t)^\top U_x(\Tilde{X}_t,t) \Tilde{X}_t \right)d{\bf B}^2_t   .
% \end{multline}
% Since $U_{xx}(x,t) <0$, $U(\Tilde{X}_t,t)$ is indeed a super-martingale, which becomes a martingale when 
% \begin{equation}
% \label{eq:xi:SPDE}
% \xi^*_t := -\frac{\nabla_xa_1(\Tilde{X}_t,t) + (\lambda_t-\sigma^{Y,1}_t)U_x(\Tilde{X}_t,t)}{U_{xx}(\Tilde{X}_t,t) \Tilde{X}_t  }.
% \end{equation}
% Substituting \eqref{eq:xi:SPDE}, we arrive at the SPDE
% \begin{multline}
% \label{eq:SPDE:2}
% dU(x,t) = \Bigg(-\frac{x^2\|\sigma^{Y,2}_t\|^2U_{xx}(x,t)}{2} -x\Big( U_x(x,t)(\lambda^\top_t\sigma^{Y,1}_t+\|\sigma^{Y,2}_t\|^2 -\mu^Y_t )   - \\ (\sigma^{Y,2}_t)^\top\nabla_xa_2(x,t)\Big)  +  \frac{\left(\nabla_xa_1(x,t) + (\lambda_t-\sigma^{Y,1}_t)U_x(x,t) \right)^2}{2U_{xx}(x,t)} \Bigg) dt +\\ a_1^\top(x,t)d{\bf B}^1_t + a_2^\top(x,t) d{\bf B}^2_t.
% \end{multline}



% \subsection{Zero Volatility}
% For the case of power utility, the result in the Section \ref{sec:power} corresponds to the solution to the \eqref{eq:SPDE:2} with $a_1,a_2\equiv 0$. Indeed, the SPDE is reduced to the PDE
% \begin{align*}
% U_t(x,t) &=  -\frac{x^2\|\sigma^{Y,2}_t\|^2U_{xx}(x,t)}{2} -x\left( U_x(x,t)(\lambda^\top_t\sigma^{Y,1}_t+\|\sigma^{Y,2}_t\|^2 -\mu^Y_t )   \right)  +  \\ &\hspace{2cm} \frac{  \|\lambda_t-\sigma^{Y,1}_t\|^2U_x^2(x,t) }{2U_{xx}(x,t)}  ,
% \end{align*}
% which can be readily solved by the method of separation of variables. Consider the \textit{ansatz} $U(x,t)=u(x)v(t)$, where $u(x) := x^\gamma/\gamma$. Then the PDE can be further reduced to solving 
% \begin{equation}
% \frac{V'_t}{v(t)} = -\frac{\gamma(\gamma-1)\|\sigma^{Y,2}_t\|^2}{2} -\gamma(\lambda^\top_t\sigma^{Y,1}_t + \|\sigma^{Y,2}_t\|^2-\mu^Y_t) + \frac{\gamma}{2(\gamma-1)}\|\lambda_t-\sigma^{Y,1}_t\|^2.
% \end{equation}
% With the initial condition $v(0)=1$, we arrive at 
% \begin{align}
% v(t) &= \exp\Bigg( \int_0^t \Big(-\frac{\gamma(\gamma-1)\|\sigma^{Y,2}_s\|^2}{2} -\gamma(\lambda_s^\top\sigma^{Y,1}_s + \|\sigma^{Y,2}_s\|^2-\mu^Y_s) +\nonumber   \\ &\hspace{2cm} \frac{\gamma}{2(\gamma-1)}\|\lambda_s-\sigma^{Y,1}_s\|^2\Big) ds   \Bigg),
% \end{align}
% and $U(x,t) = x^\gamma v(t)/\gamma$ is a forward utility preference, which coincides with the construction in Proposition \ref{pp:power}. However, this construction is not applicable for exponential utility.  

% \subsection{Non-zero Volatility}
% {\color{red} May Remove this subsection}\\
% The construction in Proposition \ref{pp:exp} corresponds to a forward utility preference with respect to the relative performance $\tilde{X}_t$ in \eqref{eq:tilde:X} with non-zero utility. Indeed, as given in \eqref{eq:exp:U:rewrite}, consider
% \begin{equation}
% U(x,t) := -\exp\left( -Z_t x + \frac{1}{2} \int_0^t \|\Sigma_s\|^2\| (\Sigma^\top_s)^{-1} \lambda_s- \hat{\pi}_s \|^2 ds  \right),
% \end{equation} 
% where $Z_t$ is given in Proposition \ref{pp:exp}. By It\^o's lemma, one can show that
% \begin{equation*}
% \frac{dU(\tilde{X}_t,t)}{U(\tilde{X}_t,t)} = \frac{1}{2}\left( (\beta_t + \xi_t)\tilde{X}_tZ_t - (\lambda_t-\sigma^{Y,1}_t + \beta_t) \right)^2 dt + (\tilde{X}_t+Z_t) (\beta_t+\xi_t)^\top d{\bf B}^1_t,
% \end{equation*}
% which is indeed a super-martingale for any admissible $\xi$, with 
% \begin{equation*}
% \xi_t^* = \frac{\lambda_t-\sigma^{Y,1}_t+\beta_t}{\tilde{X}^*_tZ_t} -\beta_t = \frac{\lambda_t}{\tilde{X}^*_tZ_t} + \left(1-\frac{1}{\tilde{X}^*_tZ_t} \right)\hat{\pi}_t ,
% \end{equation*}
% By substituting $\xi^*_t$ into \eqref{eq:pi:transform}, we obtain the optimal investment strategy $\pi_t^*$  as in \eqref{eq:pi*:exp}. Therefore, $U(x,t)$ is a forward utility process with volatilities $a_1(x,t)=xZ_tU(x,t)\beta^\top_t$ and $a_2(x,t) = xZ_tU(x,t)(\sigma^{Y,2}_t)^\top$. 

% \subsection{Ergodic BSDE Construction}
% {\color{red} The section is not to be included. }

% For $a^1,a^2\not\equiv 0$, one way to construct a forward utility preference is by the Marokvian solution of an ergodic BSDE, which was introduced in \cite{liang:bsde}. 
% \begin{proposition}
% Let $U(x,t)$ be the process 
% \begin{equation}
% U(x,t) = \frac{1}{\gamma} x^\gamma e^{V_t-\rho t},
% \end{equation}
% where $(V_t,\{Z^1_t,Z^2_t\},\rho)$ is the solution to the ergodic BSDE
% \begin{equation}
% \label{eq:power:bsde}
% dV_t = (- F(t,Z^1_t,Z^2_t) +\rho) dt + Z^1_t dB^1_t+ Z^2_t dB^2_t
% \end{equation}
% with driver
% \begin{multline}
% F(t,z_1,z_2) := \gamma( \lambda\sigma_{Y_1} + \sigma_{Y_2}^2 - \mu_Y) + \frac{\gamma(\gamma+1)\sigma_{Y_2}^2}{2}  - \frac{\gamma(\lambda-\sigma_{Y_1})^2}{2(\gamma-1)} \\   - \frac{z_1^2}{2(\gamma-1)} + \frac{z_2^2}{2}  - \frac{\gamma(\lambda-\sigma_{Y_1})z_1}{\gamma-1} -\gamma\sigma_{Y_2}z_2.
% \end{multline}
% Then $U(x,t)$ is a forward utility preference of $\Tilde{X}_t$. The optimal investment strategy is given by 
% \begin{equation}
% \label{eq:pi*:power:bsde}
% \pi_t^* = \frac{\hat{X}_t}{X^*_t}\hat{\pi}_t + \left( 1-\frac{\hat{X}_t}{X^*_t}\right) \left(- \frac{\lambda-\gamma\sigma_{Y_1}+Z_t^1}{\sigma_1(\gamma-1)} \right)  .
% \end{equation}
% \end{proposition}
% \begin{remark}
% The unique existence of solution to \eqref{eq:power:bsde} has been established in \cite{liang:bsde}. We refer the readers to it and the reference therein for details. 
% \end{remark}

% \begin{proof}
% By It\^o's lemma, 
% \begin{align*}
% \frac{dU(\Tilde{X}_t,t)}{U(\Tilde{X}_t,t)} &= \frac{\gamma}{\Tilde{X}_t}d\Tilde{X}_t + dV_t + \frac{1}{2}\left( \frac{\gamma(\gamma-1)}{\Tilde{X}_t^2}d\langle \Tilde{X} \rangle_t + d\langle V\rangle_t + \frac{2\gamma}{\Tilde{X}_t} d\langle \Tilde{X},V\rangle_t \right) \\
% &= \Bigg( \gamma\left((\lambda-\sigma_{Y_1})\xi_t - \mu_Y + \lambda\sigma_{Y_1}+\sigma_{Y_2}^2\right)  + \frac{\gamma(\gamma-1)}{2}(\xi_t^2 + \sigma_{Y_2}^2) + \frac{(Z_t^1)^2 + (Z_t^2)^2}{2} \\ &\hspace{1cm} + \gamma \xi_t Z^1_t - \gamma \sigma_{Y_2}Z^2_t -F(t,Z^1_t,Z^2_t)  \Bigg) dt  + (\gamma \xi_t +Z^1_t) dB^1_t +(Z^2_t - \gamma \sigma_{Y_2})dB^2_t \\
% &= \Bigg(   \frac{\gamma(\gamma-1)}{2}\xi_t^2 + \gamma\xi_t \left(  \lambda-\sigma_{Y_1} +  Z^1_t \right) +\gamma( \lambda\sigma_{Y_1}+\sigma_{Y_2}^2 - \mu_Y) + \frac{(Z_t^1)^2 + (Z_t^2)^2  }{2} \\ &\hspace{1cm} + \frac{\gamma(\gamma-1)\sigma_{Y_2}^2}{2} -\gamma\sigma_{Y_2}Z^2_t-F(t,Z^1_t,Z^2_t) \Bigg) +  (\gamma \xi_t +Z^1_t) dB^1_t +(Z^2_t - \gamma \sigma_{Y_2})dB^2_t \\
% &= \Bigg( \frac{\gamma(\gamma-1)}{2}\left(\xi_t + \frac{\lambda-\sigma_{Y_1}+  Z^1_t}{\gamma-1} \right)^2 - \frac{\gamma(\lambda-\sigma_{Y_1}+  Z^1_t)^2}{2(\gamma-1)} + \gamma(\lambda\sigma_{Y_1}+\sigma_{Y_2}^2 - \mu_Y )\\ &\hspace{1cm} + \frac{(Z_t^1)^2 + (Z_t^2)^2  }{2} + \frac{\gamma(\gamma-1)\sigma_{Y_2}^2}{2} -\gamma\sigma_{Y_2}Z^2_t-F(t,Z^1_t,Z^2_t) \Bigg)dt+ \\ &\hspace{1cm}   (\gamma \xi_t +Z^1_t) dB^1_t +(Z^2_t - \gamma \sigma_{Y_2})dB^2_t  \\
% &= \frac{\gamma(\gamma-1)}{2}\left(\xi_t + \frac{\lambda-\sigma_{Y_1}+  Z^1_t}{ \gamma-1} \right)^2 dt + (\gamma \xi_t +Z^1_t) dB^1_t +(Z^2_t - \gamma \sigma_{Y_2})dB^2_t  .
% \end{align*}
% Hence, $U(\Tilde{X}_t,t)$ is a super-martingale, which becomes a martingale when 
% \begin{equation}
% \xi_t^* = -\frac{\lambda-\sigma_{Y_1}+  Z^1_t}{\gamma-1}. 
% \end{equation}
% By \eqref{eq:pi:transform}, the optimal investment strategy is thus given by \eqref{eq:pi*:power:bsde}.
% \end{proof}

% Substituting $\pi^*_t$ into \eqref{eq:X}, we have 
% \begin{align}
% dX^*_t &=  \left(p_t +\hat{X}_t \hat{\pi}_t\sigma_1(\lambda-\sigma_{Y_1}) + X^*_t\left( \frac{\Tilde{X}_t(\lambda-\gamma\sigma_{Y_1}+Z^1_t)(\lambda-\sigma_{Y_1})  }{1-\gamma}  -\mu_Y+ \sigma_{Y_1}^2 +\sigma_{Y_2}^2 \right) \right) dt   \nonumber \\
% & \hspace{1cm} + \left( \hat{X}_t\hat{\pi}_t\sigma_1 + \frac{\Tilde{X}_t(\lambda-\gamma\sigma_{Y_1}Z^1_t)}{1-\gamma} - X^*_t\sigma_{Y_1}dB^1_t  \right) - X^*_t\sigma_{Y_2} dB^2_t.
% \end{align}
% \fi %%%%%%%%%%%% end of hidden section 5


% \iffalse %not to be included
% \section{A General Model}
% Consider the factor model: $\mu : \mathbb{R}^{d_1} \to \mathbb{R}^n$, $\Sigma_1 :  \mathbb{R}^{d_1} \to \mathbb{R}^{n\times n} $,
% \begin{equation*}
% \frac{d {\bf S}^1_t }{{\bf S}^1_t} = (\mu(\theta_t) +r {\bf 1}_{n})dt + \Sigma_1(\theta_t) d{\bf B}^1_t,
% \end{equation*}
% and salary: $\mu_Y : \mathbb{R}^{d_2} \to \mathbb{R}$, $\sigma_{Y_1} : \mathbb{R}^{d_2} \to \mathbb{R}^n$, $\sigma_{Y_2} : \mathbb{R}^{d_2}\to \mathbb{R}^m$
% \begin{equation}
% dY_t = Y_t\left( \mu_Y(\kappa_t) + r{\bf 1}_m \right) dt + \sigma_{Y_1}^\top(\kappa_t)d{\bf B}^1_t + \sigma_{Y_2}^\top(\kappa_t) d{\bf B}^2_t,
% \end{equation}
% where the stochastic factors follow
% \begin{align}
% d\theta_t &= \eta(\theta_t)dt + \Sigma_{\theta}(\theta_t) d{\bf B}^1_t, \\
% d\kappa_t &= \varphi(\kappa_t) dt + \sigma_{\kappa_1}^\top(\kappa_t)d{\bf B}^1_t+ \sigma_{\kappa_2}^\top(\kappa_t)d{\bf B}^2_t.
% \end{align}
% Here, $\eta:  \mathbb{R}^{d_1} \to \mathbb{R}^{d_1}$, $\Sigma_\theta: \mathbb{R}^{d_1} \to \mathbb{R}^{d_1\times d_1}$, $\varphi: \mathbb{R}^{d_2} \to \mathbb{R}^{d_2}$, $\sigma_{\kappa_1}:  \mathbb{R}^{d_2} \to \mathbb{R}^{n\times d_2}$, $\sigma_{\kappa_2}:  \mathbb{R}^{d_2} \to \mathbb{R}^{m\times d_2}$. Then, the account value follows
% \begin{equation}
% dW_t =p_tY_t dt + W_t\left( \left( r + \pi_t^\top \mu(\theta_t) \right)dt  + \pi_t^\top \Sigma_1(\theta_t) d{\bf B}^1_t      \right)
% \end{equation}
% By It\^o's lemma, the fund to salary ratio now reads
% \begin{align}
% dX_t &= Y_t^{-1}dW_t - W_tY_t^{-2} dY_t + W_tY_t^{-3}d\langle Y\rangle_t - Y_t^{-2} d\langle W,Y\rangle_t \nonumber \\
% &= \left(p_t + X_t\left( \pi_t^\top ( \mu(\theta_t)- \sigma_{Y_1}(\kappa_t) ) - \mu_Y(\kappa_t) + \|\sigma_{Y_1}(\kappa_t)\|^2 +  \|\sigma_{Y_2}(\kappa_t)\|^2  \right) \right) dt + \nonumber \\ & \hspace{1cm} + X_t\left( (\pi_t^\top \Sigma_1(\theta_t) - \sigma_{Y_1}^\top(\kappa_t))d{\bf B}^1_t - \sigma^{Y,2}^\top(\kappa_t) d{\bf B}^2_t \right). 
% \end{align}
% \fi 

\section{Concluding Remarks and Future Directions}
\label{sec:conclusion}
This paper solved the optimal investment strategy of the worker, who is enrolled to the DC pension scheme, during the accumulation phase. Instead of following the mainstream in the literature using the backward model and approach, this paper proposed to decide her optimal strategy based on the forward model and approach, via her forward utility preferences. Such a forward methodology allowed the worker to, flexibly decide her actual retirement time in the future, and derive her future preferences due to the actual realizations in the market environment. This paper discussed the SPDE representation for the worker's forward preferences, featuring their non-uniqueness and volatility processes. Two of her forward utility preferences were then constructed, with the corresponding optimal investment strategies being solved, respectively in the cases of power and exponential utility functions as her current preference. In both cases, her forward utility preferences were based on the comparison of her fund value to salary ratio to another ratio generated by the exogenous baseline investment strategy. In the power utility case, the worker's optimal investment strategy generalized the CPPI strategy with the stochastic floor, the stochastic multiple of the cushion, and the additive factor. In the exponential case, the reciprocal of the similar exogenous ratio, without the consideration of salary contribution, was shown to be the dynamic risk aversion process of the worker.\\

Adopting the forward utility preferences should allow regular model-updates for future dynamics together with strategy-revisions satisfying the time-consistency. This is especially applicable to solving the optimal investment strategy of the worker in the accumulation phase of the DC pension plan, which is often with a long horizon, and shall be revisited as one of the future directions.\\

Another future direction derived from this paper is to adopt the forward model and approach to plan for both accumulation and decumulation phases of the DC pension scheme. Only the backward model and approach have been developed in these combined problems. Yet, they are all subject to the same shortcoming that any pre-commitments would not hold in an even longer horizon together with the decumulation phase. The forward model and approach shall provide consistent and streamlining treatments for the plannings in both accumulation and decumulation phases.





% We studied the construction of forward utility preferences for DC pension fund for the cases with initial exponential and power utility. Unlike its backward counterpart, the utility process is adaptive to the market and salary conditions, which better reflects one's changing appetite for risk for a prolonged investment period. Due to the contribution to the fund and presence non-hedgeable risk in the stochastic salary, additional stochastic factors are introduced to the utility processes for preserving the super-martingale/martingale requirement, which emerges as the fund value to salary ratio under an exogenously chosen baseline strategy. The utility process is therefore measuring the surplus of the worker's fund over that of the exogenous fund, rendering an optimal strategy which is composed of a myopic component, which resembles the optimal strategy in the classical backward setting; and the baseline component consisting of the aforementioned baseline strategy.  \\  

% Owing to its distinctive nature for modeling long-term evolution of utilities, forward utility preferences could find further actuarial applications. One possible direction is the generalizations of optimal asset allocations in DC pension fund where the risky assets and salary follow a stochastic factor model \cite{liang:bsde}, where methods of BSDEs can be employed to construct forward utility preferences. 

% \begin{appendix}
% \section{Results in Linear SDEs}
% \begin{lemma}
% \label{lemma:sde}
% Suppose that $\{W_t\}_{t\geq 0} \in \mathcal{P}(\mathbb{F})$ is an $m$-dimensional Brownian motion. Let  $\xi^1=\{\xi^1_t\}_{t\geq 0}\in \mathcal{L}^2_1, \xi_2= \{\xi^2_t\}_{t\geq 0} \in \mathcal{L}^2_m$ and that $a=\{a_t\}_{t\geq 0}, b=\{b_t\}_{t\geq 0} \in \mathcal{P}(\mathbb{F})$ be uniform bounded $\mathbb{R}$ and $\mathbb{R}^m$-valued processes respectively. Then the linear SDE \begin{equation}
% \label{eq:app:sde}
% dX_t = \xi^1_t dt + (\xi^2_t)^\top dW_t + X_t\left(a_t dt + b^\top_t dW_t \right), \ X_0=x \in \mathbb{R}
% \end{equation}
% has a unique solution $X=\{X_t\}_{t\geq 0}$ with $X \in \mathcal{L}^2_1$. {\color{blue} In particular, $X$ admits a continuous version. } 
% \end{lemma}
% \begin{proof}
% For the uniqueness, suppose $\bar{X}=\{\bar{X}_t\}_{t\geq 0}\in \mathcal{L}^2_1$ is a solution to \eqref{eq:app:sde}. Then for any $t>0$,
% \begin{align*}
% \mathbb{E}[|X_t-\bar{X}_t|^2] &\leq 2 \mathbb{E}\left[\int_0^t |a_s(X_s-\bar{X}_s)|^2 ds\right] +\mathbb{E}\left[\left| \int_0^t (X_s-\bar{X}_s)b_s^\top dW_s\right|^2 \right] \\
% &\leq C\mathbb{E}\left[ \int_0^t |X_s-\bar{X}_s|^2 ds \right],
% \end{align*}
% where the last line follows from It\^o's isometry and the fact that $a,b$ are uniformly bounded. Hence uniqueness follows from Gr\"onwall's inequality. The existence is a simple application of Picard's iteration. Indeed, let $X^{(0)}:=x$ and define recursively for $n\geq 1$:
% \begin{equation*}
% X^{(n)}_t := x+\int_0^t (\xi^1_s+ a_s  X^{(n-1)}_s) ds + \int_0^t (\xi^2_s + b_s X^{(n-1)}_s)^\top dW_s .
% \end{equation*}
% Then it holds that 
% \begin{align*}
% \mathbb{E}\left[\left| X^{(n+1)}_t- X^{(n)}_t \right|^2\right] &\leq 2\mathbb{E}\left[ \int_0^t\left|a_s (X^{(n)}_s-X^{(n-1)}_s) \right|^2 ds \right] + 2\mathbb{E}\left[ \left|\int_0^t b_s^\top(X^{(n)}_s-X^{(n-1)}_s) dW_s \right|^2\right] \\
% &\leq C\mathbb{E}\left[ \int_0^t\left|  X^{(n)}_s-X^{(n-1)}_s \right|^2 ds \right],
% \end{align*}
% by revoking It\^o's isometry and the boundedness of $a$ and $b$. Since 
% \begin{align*}
% \mathbb{E}\left[\left| X^{(1)}_t- X^{(0)}_t \right|^2\right] &= \mathbb{E}\left[\left|\int_0^t (\xi^1_s+ a_s  x) ds + \int_0^t (\xi^2_s + b_sx)^\top dW_s \right|^2  \right] <\infty,
% \end{align*}
% the $L^2$ convergence of $X^{(n)}$ to $X$ follows from Borel-Cantelli lemma. The fact that $X$ admits a continuous version is now a consequence of $X\in \mathcal{L}^2_1$ and \cite[Theorem 3.2.5]{oksendal}.    \end{proof}    

% \end{appendix}

\nocite{dosreisL:mean:field}
\nocite{BLAKE2014105}
\nocite{Nadtochiy2014}
\bibliographystyle{apacite}
\bibliography{ref}

\end{document}