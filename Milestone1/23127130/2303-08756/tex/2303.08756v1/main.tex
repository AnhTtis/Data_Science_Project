%\pdfoutput=1
\documentclass{amsart}
\usepackage{xcolor}
\usepackage{hyperref}
\hypersetup{
	colorlinks=true,
	citecolor=blue,
	linkcolor=blue,
	filecolor=magenta,      
	urlcolor=cyan,
}
\usepackage[capitalise,noabbrev]{cleveref}



\usepackage{amsmath,amscd}
\usepackage{amssymb}
\usepackage{mathtools}
\usepackage{stmaryrd}
\usepackage{url}
\usepackage{tikz-cd}
\usepackage{enumitem}


\usepackage{fullpage}




%\usepackage[margin=1.1in]{geometry}
\setlength{\parindent}{0cm}
\setlength{\parskip}{0.7ex}
\usepackage{setspace}
\setstretch{1.1}
\usepackage{microtype}

\usepackage{times}



\usepackage{tikz}

\usepackage{enumitem,kantlipsum}





% ** top matter **
\address{Department of Algebra, Faculty of Mathematics and Physics, Charles University in Prague, Sokolovsk\'a 83, 186 75 Praha, Czech Republic}


\email{shaul@karlin.mff.cuni.cz}

%
%
%



%  ** new environments **

\newtheorem{thm}[equation]{Theorem}
\newtheorem*{thm*}{Theorem}
\newtheorem*{cor*}{Corollary}
\newtheorem*{dfn*}{Definition}


\newtheorem{cthm}{Theorem}
\renewcommand{\thecthm}{\Roman{cthm}} % "letter-numbered" theorems
\newtheorem{ccor}[cthm]{Corollary}
\newtheorem{que}[cthm]{Question}

\newtheorem{cor}[equation]{Corollary}
\newtheorem{prop}[equation]{Proposition}
\newtheorem{lem}[equation]{Lemma}
\newtheorem{conj}[equation]{Conjecture}
\theoremstyle{definition}
\newtheorem{dfn}[equation]{Definition}
\newtheorem{rem}[equation]{Remark}
\newtheorem{exa}[equation]{Example}
\newtheorem{fact}[equation]{}
\newtheorem{claim}{Claim}




%  ** new commands **
\newcommand{\iso}{\xrightarrow{\simeq}}
\newcommand{\inj}{\hookrightarrow}
\newcommand{\surj}{\twoheadrightarrow}
\newcommand{\xar}{\xrightarrow}
\newcommand{\opn}{\operatorname}
\newcommand{\opnt}[1]{\mathrm{#1}} % less space
\newcommand{\cat}[1]{\operatorname{\mathsf{#1}}}
\newcommand{\bdot}{\bsym{\cdot}}
\newcommand{\ul}{\underline}
\newcommand{\blnk}[1]{\mbox{\hspace{#1}}}
\newcommand{\rmitem}[1]{\item[\text{\textup{(#1)}}]}
\newcommand{\mfrak}[1]{\mathfrak{#1}}
\newcommand{\mcal}[1]{\mathcal{#1}}
\newcommand{\msf}[1]{\mathsf{#1}}
\newcommand{\mbf}[1]{\mathbf{#1}}
\newcommand{\mrm}[1]{\mathrm{#1}}
\newcommand{\mbb}[1]{\mathbb{#1}}
\newcommand{\gfrac}[2]{\left[ \genfrac{}{}{0pt}{}{#1}{#2} \right]}
\newcommand{\smgfrac}[2]{\left[
\genfrac{}{}{0pt}{1}{#1}{#2} \right]}
\newcommand{\tup}[1]{\textup{#1}}
\newcommand{\bsym}[1]{\boldsymbol{#1}}
\newcommand{\boplus}{\bigoplus\nolimits}
\newcommand{\wtil}[1]{\widetilde{#1}}
\newcommand{\til}[1]{\tilde{#1}}
\newcommand{\what}[1]{\widehat{#1}}
\newcommand{\bra}[1]{\langle #1 \rangle}
\renewcommand{\k}{\Bbbk}
\renewcommand{\l}{\ell}
\newcommand{\K}{\mbb{K} \hspace{0.05em}}
\renewcommand{\d}{\mathrm{d}}
\newcommand{\bwedge}{\bigwedge\nolimits}
\newcommand{\smfrac}[2]{{\textstyle \frac{#1}{#2}}}
\newcommand{\cmnt}[1]{\mbox{} \newline 
\marginpar{\quad $\Lleftarrow$ \textsf{!!!}} 
\textsf{[#1]} \newline}
\renewcommand{\a}{\mfrak{a}}
\renewcommand{\b}{\mfrak{b}}
\renewcommand{\c}{\mfrak{c}}
\renewcommand{\d}{\mfrak{d}}
\newcommand{\m}{\mfrak{m}}
\newcommand{\n}{\mfrak{n}}
\newcommand{\p}{\mfrak{p}}
\newcommand{\q}{\mfrak{q}}
\newcommand{\injdim}{\operatorname{inj\,dim}}
\newcommand{\projdim}{\operatorname{proj\,dim}}
\newcommand{\flatdim}{\operatorname{flat\,dim}}
\newcommand{\seqdepth}{\operatorname{seq.depth}}
\newcommand{\depth}{\operatorname{depth}}
\newcommand{\amp}{\operatorname{amp}}
\newcommand{\Hom}{\operatorname{Hom}}
\newcommand{\RHom}{\msf{R}\mrm{Hom}}
\newcommand{\Lotimes}{\otimes^\msf{L}}
\def\skewtimes{\ltimes\!}
\newcommand{\D}{\msf{D}}
\newcommand{\ti}{\textit}
\newcommand{\op}{\opn{op}}

\title{Gorenstein symmetry and acyclic complexes of injectives}
\author{Liran Shaul}



\begin{document}
\begin{abstract}
By generalizing a recent result of Rickard about relations between unbounded derived categories and finitistic dimension,
we obtain several new characterizations of noetherian rings which satisfy the Gorenstein symmetry conjecture. 
As a byproduct of our methods, we obtain the following foundational homological result: over a commutative noetherian ring $A$ with a dualizing complex,
any bounded above cochain complex of injectives $A$-modules which is acyclic is null-homotopic. For noncommutative noetherian rings which have finite injective dimension on one side, this property on that side turns out to be equivalent to Gorenstein symmetry.
\end{abstract}


\numberwithin{equation}{section}
\maketitle


\setcounter{section}{-1}
\section{Introduction}



In this paper rings are associative and unital,
modules are by default left modules,
and complexes are indexed cohomologically.
The Gorenstein symmetry conjecture asks, 
given a left and right noetherian ring $A$,
with the property that $\injdim_A(A) < \infty$,
whether it holds that $\injdim_{A^{\op}}(A^{\op})<\infty$.
In that case, by \cite[Lemma A]{Zaks}, it holds that $\injdim_A(A) = \injdim_{A^{\op}}(A^{\op})$.
Most of the existing literature on this problem focuses on the case where $A$ is an artin algebra,
but the problem makes sense (and is open)
in the much more general noetherian case considered here.
As far as we know, 
no counterexamples are known for the problem over noetherian rings.
The aim of this paper is to provide new criteria for a noetherian ring $A$ to satisfy the Gorenstein symmetry conjecture. An important feature of our main result is that we are able to, assuming $A$ has finite injective dimension as a left module over itself, obtain several necessary and sufficient for $A$ to have finite injective dimension as a right module over itself in terms of the behavior of left $A$-modules and complexes of left $A$-modules.

The key technical concept used in this paper is the notion of a dualizing complex, first introduced by Grothendieck \cite{RD} in algebraic geometry, and by Yekutieli \cite{Yek} in non-commutative algebra. Most noetherian rings that arise in nature,
and in particular, finite dimensional algebra over a field posses dualizing complexes. Moreover,
under a very mild technical assumption that will be assumed in our main result, Gorenstein rings also have dualizing complexes.

To state our main results, we first recall the notion of a localizing (respectively colocalizing) subcategory of the unbounded derived category $\cat{D}(A)$,
where $A$ is a ring.
Given a non-empty set $\mathbf{S}$ of objects in $\cat{D}(A)$,
the localizing (resp. colocalizing) subcategory generated by $\mathbf{S}$,
denoted by $\opn{Loc}(\mathbf{S})$ (resp. $\opn{Coloc}(\mathbf{S})$),
is by the definition the smallest full triangulated subcategory of $\cat{D}(A)$ which contains $\mathbf{S}$ and is closed under infinite coproducts (resp. products). We will only be interested in the cases where $\mathbf{S}$ is either $\opn{Proj}(A)$, the collection of all projective $A$-modules, $\opn{Inj}(A)$, the collection of all injective $A$-modules, or $\opn{Flat}(A)$, the collection of all flat $A$-modules.

As our first main result, we show in \cref{thm:acyclic}:
\begin{cthm}\label{cthmA}
Let $A$ be a left and right noetherian ring which has a dualizing complex.
Assume that $\opn{Coloc}(\opn{Proj}(A)) = \cat{D}(A)$.
Then every bounded above cochain complex of injective $A$-modules which is acyclic is null-homotopic.
\end{cthm}

A foundational fact in homological algebra is that over any ring,
every \textbf{bounded below} cochain complex of injective modules which is acyclic is null-homotopic. This basic fact is in the heart of the construction of derived functors. The existence of a dualizing complex over a ring implies that various dualities hold for it, and the fact that a ring is commutative implies the further duality $A = A^{\op}$.
Using this, and \cref{cthmA}, we obtain in \cref{cor:commutative} the following surprising opposite result:

\begin{ccor}
Let $A$ be a commutative noetherian ring which has a dualizing complex.
Then every \textbf{bounded above} cochain complex of injective $A$-modules which is acyclic is null-homotopic.
\end{ccor}

The dual result about bounded below complexes of projective $A$-modules also holds. See \cref{rem:dual} for details.

Our next results concern the big projective finitistic dimension. Recall that it is given by 
\[
\opn{FPD}(A) = \sup\{\projdim_A(M) \mid M \in \opn{Mod}(A), \projdim_A(M) < \infty\}.
\]
The question whether $\opn{FPD}(A)<\infty$ for $A$ being an artin algebra is known as the big finitistic dimension conjecture, 
and is one of the most important open problems in homological algebra.
Our result concerning the finitistic dimension is given in \cref{thm:FPD} and it states:
\begin{cthm}\label{cthmB}
Let $A$ be a left and right noetherian ring with a dualizing complex,
and suppose that every bounded above complex of injective $A$-modules which is acyclic is null-homotopic.
Then $\opn{FPD}(A^{\op}) < \infty$.
\end{cthm}

Combining \cref{cthmA} and \cref{cthmB}, 
we obtain the following, which generalizes \cite[Proposition 5.2]{Rickard} where Rickard proved the same result for finite dimensional algebras over a field:

\begin{ccor}
Let $A$ be a left and right noetherian ring with a dualizing complex.
If $\opn{Coloc}(\opn{Proj}(A)) = \cat{D}(A)$ then $\opn{FPD}(A^{\op}) < \infty$.
\end{ccor}

We then use the above to study Gorenstein symmetry.
We say that an $A$-$A$ bimodule $\phantom{ }_A M_A$ has a \textbf{bi-injective resolution} if there is a bounded below complex of bimodules $\phantom{ }_A I_A$ which is quasi-isomorphic to $M$, and such that forgetting its bimodule structure, 
it is a complex of injectives both on the left and on the right. For example, if there exist a commutative ring $\K$ such that $A$ is a flat central $\K$-algebra,
then every $A$-$A$ bimodule has a bi-injective resolution.

Our main result, given in \cref{thm:Gorenstein} states:
\begin{cthm}
Let $A$ be a ring which is left and right noetherian,
and such that the regular bimodule $\phantom{ }_A A_A$ has a bi-injective resolution.
Assume that $\injdim_A(A) < \infty$.
Then the following are equivalent:
\begin{enumerate}
\item $A$ satisfies the Gorenstein symmetry conjecture: It holds that $\injdim_{A^{\op}}(A^{\op}) < \infty$.
\item $A$ has a dualizing complex and $\opn{Coloc}(\opn{Proj}(A)) = \cat{D}(A)$.
\item $A$ has a dualizing complex and $\opn{Coloc}(\opn{Flat}(A)) = \cat{D}(A)$.
\item $A$ has a dualizing complex, and every bounded above complex of injective $A$-modules which is acyclic is null-homotopic.
\item $A$ has a dualizing complex, and every injective $A$-module has finite flat dimension.
\item $A$ has a dualizing complex $R$ which has finite flat dimension as a complex of left $A$-modules.
\end{enumerate}
\end{cthm}

The fact that conditions (5) and (6) are equivalent to (1) is not new, and is of course well known.
It can be easily seen that for any ring, (5)$\implies$(3),
but condition (3) is in general much weaker than (5).
For instance, any commutative noetherian ring satisfies that 
$\opn{Coloc}(\opn{Flat}(A)) = \cat{D}(A)$,
but this is far from true for (5):
the only commutative noetherian rings that satisfy (5) are the Gorenstein rings,
so that one may view (3) as a substantial weakening of the classical condition for Gorenstein symmetry.
For rings with dualizing complexes, condition (4) is even weaker. As discussed in \cref{rem:flat} below,
for any left and right noetherian ring with a dualizing complex,
it holds that (3)$\implies$(4),
so our main result identifies much weaker necessary and sufficient conditions for Gorenstein symmetry to hold.

\textbf{Acknowledgments.}

The author thanks Jan \v S\v tov\'\i\v cek and Amnon Yekutieli for helpful discussions.
This work has been supported by the grant GA~\v{C}R 20-02760Y from the Czech Science Foundation.


\section{Preliminaries}

In this section we gather various preliminaries used throughout the paper.

\subsection{Derived categories and DG-rings}

We will freely use the language of derived categories over rings and DG-rings. 
We refer the reader to \cite{Kel,YeBook} for background about these.
By a DG-ring $A$ we will always mean a non-positive DG-ring $A = \bigoplus_{n=-\infty}^0 A^n$, equipped with differential of degree $+1$.
For a ring or a DG-ring $A$, 
we denote by $A^{\op}$ the opposite ring or DG-ring.
As modules and DG-modules are always left DG-modules,
right DG-modules will be considered as DG-modules over $A^{\op}$.
The derived category of complexes of left $A$-modules (or left DG-modules in cases $A$ is a DG-ring) will be denoted by $\cat{D}(A)$.
The bounded derived category $\cat{D}^{\mrm{b}}(A)$ is the full triangulated subcategory consisting of all $M$ such that $\mrm{H}^n(M) = 0$ for all $|n|\gg 0$.
Given $M \in \cat{D}(A)$, 
we let $\inf(M) = \inf\{n \in \mathbb{Z} \mid \mrm{H}^n(M) \ne 0\}$,
and similarly $\sup(M) = \sup\{n \in \mathbb{Z} \mid \mrm{H}^n(M) \ne 0\}$.

For a DG-ring $A$, we have that $\mrm{H}^0(A)$ is a ring,
and there is a natural map of DG-rings $A \to \mrm{H}^0(A)$.
This induces two functors 
\[
\mrm{H}^0(A)\otimes^{\mrm{L}}_A -, \mrm{R}\opn{Hom}_A(\mrm{H}^0(A),-) :\cat{D}(A) \to \cat{D}(\mrm{H}^0(A)),
\]
known as the reduction and coreduction functors,
which are extremely useful in the study of DG-rings.
They were studied in detail in \cite{ShINJ,ShRed,YeDual},
and we will frequently use their properties.


\subsection{Homological and finitistic dimensions}

If $A$ is either a ring or a DG-ring, 
we will consider various homological dimensions of complexes or DG-modules over $A$.
These are defined for a given $M \in \cat{D}^{\mrm{b}}(A)$ as follows:
The projective dimension of $M$ is given by
\[
\projdim_A(M) = \inf\{n \in \mathbb{Z} \mid \opn{Ext}_A^i(M,N) = 0 \text{ for all $N \in \cat{D}^\mrm{b}(A)$ and any $i > n + \sup(N)$}\},
\]
the injective dimension of $M$ is defined as
\[
\injdim_A(M) = \inf\{n \in \mathbb{Z} \mid \opn{Ext}_A^i(N,M) = 0 \text{ for all $N \in \cat{D}^\mrm{b}(A)$ and any $i > n - \inf(N)$}
\},
\]
and the flat dimension of $M$ is
\[
\flatdim_A(M) = \inf\{n \in \mathbb{Z} \mid \opn{Tor}^A_i(N,M) = 0\text{ for all $N \in \cat{D}^\mrm{b}(A^{\op})$ and any $i > n - \inf(N)$}\}.
\]

We shall further require the notions of finitistic dimensions over rings and DG-rings.
Over a ring $A$, we define the (big) projective finitistic dimension by 
$\opn{FPD}(A) = \sup\{\projdim_A(M) \mid M \in \opn{Mod}(A), \projdim_A(M) < \infty\}$
Similarly, the injective finitistic dimension of $A$ is 
$\opn{FID}(A) = \sup\{\injdim_A(M) \mid M \in \opn{Mod}(A), \injdim_A(M) < \infty\}$
and the flat finitistic dimension of $A$ is
$\opn{FFD}(A) = \sup\{\flatdim_A(M) \mid M \in \opn{Mod}(A), \flatdim_A(M) < \infty\}$
For our purposes however, 
these are not sufficient, 
and we shall also require the corresponding notions over DG-rings with bounded cohomology, following \cite{DGFinite,ShFinDim}.
It should be noted that in that context, 
where modules are replaced by DG-modules, 
we must normalize the definitions.
Thus, the projective finitistic dimension of a DG-ring $A$ is given by
\[
\opn{FPD}(A) = \sup\{\projdim_A(M) + \inf(M) \mid M \in \cat{D}^{\mrm{b}}(A), \projdim_A(M) < \infty\}.
\]
The injective finitistic dimension is
\[
\opn{FID}(A) = \sup\{\injdim_A(M) - \sup(M) \mid M \in \cat{D}^{\mrm{b}}(A), \injdim_A(M) < \infty\},
\]
and the flat finitistic dimension is given by the formula
\[
\opn{FFD}(A) = \sup\{\flatdim_A(M) + \inf(M) \mid M \in \cat{D}^{\mrm{b}}(A), \flatdim_A(M) < \infty\}.
\]

\subsection{Homotopy categories}

For a ring $A$, recall that $\cat{K}(A)$ is the homotopy category of complexes of $A$-modules.
We let $\opn{Proj}(A)$, $\opn{Inj}(A)$ and $\opn{Flat}(A)$ denote the collections of all projective, injective and flat $A$-modules.
We shall need the following three full triangulated subcategories of the homotopy category: $\cat{K}(\opn{Inj}(A))$, $\cat{K}(\opn{Proj}(A))$ and $\cat{K}(\opn{Flat}(A))$.
By definition, $\cat{K}(\opn{Inj}(A))$ is the homotopy category of all complexes $I$ such that $I^n$ is an injective $A$-module for all $n$. The categories $\cat{K}(\opn{Proj}(A))$ and $\cat{K}(\opn{Flat}(A))$ are defined similarly.
For more information about these triangulated categories,
see \cite{JorCom,Kra05,Neeman2008}.
One important fact we will frequently use is that by \cite[Proposition 8.1]{Neeman2008},
for any ring $A$, the inclusion functor $\cat{K}(\opn{Proj}(A)) \to \cat{K}(\opn{Flat}(A))$ has a right adjoint.
Following \cite{InKr}, we will denote this right adjoint by $q:\cat{K}(\opn{Flat}(A)) \to \cat{K}(\opn{Proj}(A))$.

\subsection{Complexes of bimodules and bi-injective resolutions}

Given a ring $A$, 
we will often work with complexes $\phantom{ }_A M_A$ of $A$-$A$-bimodules. 
For such a complex $M$,
we define its restriction $\opn{Rest}_A(M) \in \cat{D}(A)$ (respectively $\opn{Rest}_{A^{\op}}(M) \in \cat{D}(A^{\op})$) to be the complex obtained from $M$ by forgetting its right (resp. left) structure.

\begin{dfn}
Let $A$ be a ring, 
and let $\phantom{ }_A M_A$ be a bounded below complex of $A$-$A$-bimodule. 
A bi-injective resolution of $M$ is an $A$-$A$-linear quasi-isomorphism $\phantom{ }_A M_A \to \phantom{ }_A I_A$,
such that $\phantom{ }_A I_A$ is a complex of $A$-$A$-bimodules which satisfies:
\begin{enumerate}
\item $I^n = 0$ for all $n \ll 0$.
\item For each $n$
the left $A$-module $\left(\opn{Rest}_A(M)\right)^n$ is an injective left $A$-module.
\item For each $n$,
the right $A$-module $\left(\opn{Rest}_{A^{\op}}(M)\right)^n$ is an injective right $A$-module.
\end{enumerate}
\end{dfn}

Often any bounded below complex of bimodules has a bi-injective resolution,
as in the next result, 
which is essentially contained in \cite[Proposition 2.4]{Yek}:
\begin{prop}
Let $A$ be a ring.
Suppose that there exist a commutative ring $\K$,
and a map $\K \to A$, 
whose image lies in the center of $A$,
and such that $A$ is flat over $\K$.
Then every bounded below complex $M$ of $A$-$A$-bimodules has a bi-injective resolution.
\end{prop}
\begin{proof}
The point is that the commutative ring $\K$ allows us to form the enveloping algebra $A\otimes_{\K} A^{\op}$,
and a complex of $A$-$A$-bimodules is simply a complex over $A\otimes_{\K} A^{\op}$.
Flatness then ensures that for any injective $A\otimes_{\K} A^{\op}$-module $I$,
the restrictions $\opn{Rest}_A(I)$ and $\opn{Rest}_{A^{\op}}(I)$ are injective over $A$ and $A^{\op}$,
which implies that any injective resolution of $M$ over $A\otimes_{\K} A^{\op}$ is a bi-injective resolution of $M$ over $A$.
\end{proof}

\subsection{Dualizing complexes and the covariant Grothendieck duality}

One of the key tools used in this paper is the theory of dualizing complexes over noncommutative rings,
first introduced by Yekutieli in \cite{Yek},
and studied in detail in \cite{InKr,Pos,YZ,YZ2}.

For our purposes, following \cite{Pos},
we wish to distinguish between weak dualizing complexes and strong dualizing complexes.
We start with recalling the former:

\begin{dfn}\label{dfn:weak}
Let $A$ be a left and right noetherian ring.
A complex of bimodules $\phantom{ }_A R_A$ is called a \textbf{weak dualizing complex} over $A$ if the following holds:
\begin{enumerate}
\item It holds that  $\injdim_A(\opn{Rest}_A(R)) < \infty$ and $\injdim_{A^{\op}}(\opn{Rest}_{A^{\op}}(R)) < \infty$.
\item The cohomology of $R$ is bounded and finitely generated on both sides: for each $n$, $\mrm{H}^n(\opn{Rest}_A(R))$ is a finitely generated $A$-module and $\mrm{H}^n(\opn{Rest}_{A^{\op}}(R))$ is a finitely generated $A^{\op}$-module.
\item The natural homothety maps $A \to \mrm{R}\opn{Hom}_A(R,R)$ and $A \to \mrm{R}\opn{Hom}_{A^{\op}}(R,R)$ are isomorphisms.
\end{enumerate}
\end{dfn}

As a key example, 
if $A$ is a left and right noetherian ring such that 
$\injdim_A(A) = \injdim_{A^{\op}}(A^{\op}) < \infty$,
then by definition, the regular bimodule $\phantom{ }_A A_A$ is a weak dualizing complex over itself.

We will however be focusing mostly on strong dualizing complexes, which we will simply call dualizing complexes.

\begin{dfn}
Let $A$ be a left and right noetherian ring.
A complex of bimodules $\phantom{ }_A R_A$ is called a \textbf{dualizing complex} over $A$ if the following holds:
\begin{enumerate}
\item $R$ is a weak dualizing complex.
\item $R$ is bounded, so $R^n = 0$ for all $|n| \gg 0$.
\item For all $n$, $\left(\opn{Rest}_A(R)\right)^n$ is an injective $A$-module and $\left(\opn{Rest}_{A^{\op}}(R)\right)^n$ is an injective $A^{\op}$-module.
\end{enumerate}
\end{dfn}

Thus, a dualizing complex is simply a bounded complex of bimodules which is simultaneously a complex of injectives on the left and on the right, satisfying the axioms of a weak dualizing complex.

\begin{prop}\label{prop:bi-injective}
Let $A$ be a left and right noetherian ring,
and let $R$ be a weak dualizing complex over $A$.
If $R$ has a bi-injective resolution,
then $A$ has a dualizing complex.
\end{prop}
\begin{proof}
Let $S$ be a bi-injective resolution of $R$.
It is possible that $S$ is not bounded,
but the fact that $\injdim_A(\opn{Rest}_A(S)) < \infty$ and $\injdim_{A^{\op}}(\opn{Rest}_{A^{\op}}(S)) < \infty$,
implies, 
as in the proof of \cite[Proposition I.7.6]{RD},
that for $n$ large enough, 
the smart truncation $\sigma_{\le n}(S)$ of $S$ below $n$ will be a bounded complex of $A$-$A$-bimodules which is injective on both sides, and which is quasi-isomorphic to $S$.
Hence, it is a dualizing complex over $A$.
\end{proof}

The reason we must work with dualizing complexes rather than weak dualizing complexes, 
is that they induce the covariant Grothendieck duality:
by \cite[Theorem 4.8]{InKr} (see also \cite[Theorem 4.5]{Pos}),
if $R$ is a dualizing complex over a left and right noetherian ring $A$, the the functor $R\otimes_A -:\cat{K}(\opn{Proj}(A)) \to \cat{K}(\opn{Inj}(A))$ is an equivalence of categories,
with the inverse equivalence given by $q\circ \opn{Hom}_A(R,-)$,
where $q:\cat{K}(\opn{Flat}(A)) \to \cat{K}(\opn{Proj}(A))$ is the right adjoint to the inclusion.

\subsection{Localizing and colocalizing subcategories}

Given a triangulated category $\mathcal{T}$,
a localizing subcategory of $\mathcal{T}$ is a full triangulated subcategory of $\mathcal{T}$ which is closed under infinite coproducts. For a non-empty subset $\mathbf{S}$ of objects of $\mathcal{T}$,
we denote by $\opn{Loc}(\mathbf{S})$ the localizing subcategory of $\mathcal{T}$ generated by $\mathbf{S}$,
which is, by definition, the smallest localizing subcategory of $\mathcal{T}$ which contains $\mathbf{S}$.
If $\opn{Loc}(\mathbf{S}) = \mathcal{T}$,
we will say that $\mathbf{S}$ generates $\mathcal{T}$.

Similarly, a colocalizing subcategory of $\mathcal{T}$ is a full triangulated subcategory of $\mathcal{T}$ which is closed under infinite products. 
Again, given a non-empty subset $\mathbf{S}$ of objects of $\mathcal{T}$,
we let $\opn{Coloc}(\mathbf{S})$ be the colocalizing subcategory of $\mathcal{T}$ generated by $\mathbf{S}$,
that is, the smallest colocalizing subcategory of $\mathcal{T}$ which contains $\mathbf{S}$. 
If $\opn{Coloc}(\mathbf{S}) = \mathcal{T}$,
we say that $\mathbf{S}$ cogenerates $\mathcal{T}$.

For any ring $A$, by \cite[Proposition 2.2]{Rickard},
it holds that $\opn{Loc}(\opn{Proj}(A)) = \cat{D}(A)$.
Similarly, one shows that $\opn{Coloc}(\opn{Inj}(A)) = \cat{D}(A)$. A bit more generally, it is not hard to see that if $I$ is any injective cogenerator of $\opn{Mod}(A)$,
then the singleton $\{I\}$ cogenerates $\cat{D}(A)$.

Following ideas of Keller, it was demonstrated in \cite{Rickard} that it is interesting and fruitful to also consider the categories $\opn{Loc}(\opn{Inj}(A))$ and $\opn{Coloc}(\opn{Proj}(A))$. In this paper we will also consider $\opn{Coloc}(\opn{Flat}(A))$.

The next result is taken from our upcoming joint paper with Peder Thompson \cite{ShThompson}.
We thank Thompson for allowing us to include its proof here.

\begin{thm}\label{thm:injImplyFlat}
Let $A$ be a left coherent ring.
If $\mrm{Loc}(\opn{Inj}(A)) = \cat{D}(A)$ then
\[
\mrm{Coloc}(\opn{Flat}(A^{\opn{op}})) = \cat{D}(A^{\opn{op}}).
\]
\end{thm}
\begin{proof}
Define the following full subcategory of $\cat{D}(A)$:
\[
\mathcal{S} := \{M \in \cat{D}(A) \mid \opn{Hom}_{\mathbb{Z}}(M,\mathbb{Q}/\mathbb{Z}) \in \mrm{Coloc}(\opn{Flat}(A^{\opn{op}}))\}.
\]
It follows from the definition that $\mathcal{S}$ is a localizing subcategory of $\cat{D}(A)$.
As $A$ is left coherent,
according to \cite[Lemma 3.1.4]{Xu},
if $I$ is a left injective $A$-module,
it holds that $\opn{Hom}_{\mathbb{Z}}(I,\mathbb{Q}/\mathbb{Z})$ is a flat $A^{\opn{op}}$-module,
which implies that $I \in \mathcal{S}$.
Hence, the fact that $\mrm{Loc}(\opn{Inj}(A)) = \cat{D}(A)$ implies that $\mathcal{S} = \cat{D}(A)$.
This implies that $\opn{Hom}_{\mathbb{Z}}(A,\mathbb{Q}/\mathbb{Z}) \in \mrm{Coloc}(\opn{Flat}(A^{\opn{op}}))$,
and since it is an injective cogenerator of $\cat{D}(A^{\opn{op}})$,
we deduce that 
$\mrm{Coloc}(\opn{Flat}(A^{\opn{op}})) = \cat{D}(A^{\opn{op}})$.
\end{proof}


\section{Projective cogeneration and acyclic complexes of injectives}


In this section we relate the property that the projective modules cogenerate the derived category over a noetherian ring with a dualizing complex to the contractiblity of bounded above acyclic complexes of injectives.
Our main result is the following:

\begin{thm}\label{thm:acyclic}
Let $A$ be a left and right noetherian ring which has a dualizing complex. 
Assume that $\opn{Coloc}(\opn{Proj}(A)) = \cat{D}(A)$.
Then every bounded above cochain complex of injective $A$-modules which is acyclic is null-homotopic.
\end{thm}
\begin{proof}
Let $R$ be a dualizing complex over $A$,
and suppose that there exist a bounded above complex of injective $A$-modules $D$ which is acyclic but not null-homotopic. As it is a complex of injectives,
we may view $D$ as an element of $\cat{K}(\opn{Inj}(A))$.
Take some projective left $A$-module $P$,
and note that by the Bass-Papp theorem,
the complex $R\otimes_A P$ is a bounded complex of injectives left $A$-modules.
Since $D$ is acyclic,
when considered as an element of $\cat{D}(A)$,
we have that $D \cong 0$,
which implies that
\[
\opn{Hom}_{\cat{D}(A)}(D,R\otimes_A P[n]) = 0
\]
for all $n \in \mathbb{Z}$.
Since $R\otimes_A P[n]$ is a bounded complex of injectives,
it is K-injective, 
so by \cite[Theorem 10.1.13]{YeBook},
it follows that
\[
0 = \opn{Hom}_{\cat{D}(A)}(D,R\otimes_A P[n]) = 
\opn{Hom}_{\cat{K}(A)}(D,R\otimes_A P[n]).
\]
As both sides are complexes of injectives,
we see that
\[
0 = \opn{Hom}_{\cat{K}(A)}(D,R\otimes_A P[n]) = 
\opn{Hom}_{\cat{K}(\opn{Inj}(A))}(D,R\otimes_A P[n]).
\]
According to \cite[Theorem 4.8]{InKr} (or \cite[Theorem 4.5]{Pos}), the functor 
\[
q\circ\opn{Hom}_{A}(R,-):\cat{K}(\opn{Inj}(A)) \to \cat{K}(\opn{Proj}(A))
\]
is an equivalence of categories. 
Applying this equivalence gives us 
\begin{equation}\label{eqn:keyEq}
0 = \opn{Hom}_{\cat{K}(\opn{Proj}(A))}(q\circ \opn{Hom}_{A}(R,D),q\circ \opn{Hom}_{A}(R,R\otimes_A P[n])).
\end{equation}
By definition, $\opn{Hom}_{A}(R,D)$ is a bounded above complex of flat $A$-modules.
According to \cite[Theorem 2.7(2)]{InKr},
this implies that
the complex $q\circ \opn{Hom}_{A}(R,D)$ is a bounded above complex of projective $A$-modules.
Next, let us analyze the right hand size, namely
$q\circ\opn{Hom}_{A}(R,R\otimes_A P[n])$.
To do this, we first consider its image in $\cat{D}(A)$.
There, we may calculate that
\begin{gather*}
q\circ \opn{Hom}_{A}(R,R\otimes_A P[n]) \cong 
\mrm{R}\opn{Hom}_{A}(R,R\otimes^{\mrm{L}}_A P[n]) \cong \\
\mrm{R}\opn{Hom}_{A}(R,R) \otimes^{\mrm{L}}_A P[n] \cong P[n]
\end{gather*}
where the first isomorphism follows from \cite[Theorem 2.7(1)]{InKr}, the second isomorphism follows from \cite[Theorem 12.9.10]{YeBook} and the last isomorphism follows from the fact that $R$ is a dualizing complex.
An alternative way to see this is to observe that we just applied the two mutually inverse equivalences $R\otimes_A -$ and $q\circ \opn{Hom}_{A}(R,-)$ to the element $P[n]\in \cat{K}(\opn{Proj}(A))$, 
which explains why the result is $P[n]$.

Going back to (\ref{eqn:keyEq}),
we have seen that $q\circ \opn{Hom}_{A}(R,D)$ is a bounded above complex of projective $A$-modules,
which implies that it is K-projective.
Hence, we deduce using \cite[Theorem 10.2.9]{YeBook} that
\begin{gather*}
0 = \opn{Hom}_{\cat{K}(\opn{Proj}(A))}(q\circ \opn{Hom}_{A}(R,D),q\circ \opn{Hom}_{A}(R,R\otimes_A P[n])) =\\
\opn{Hom}_{\cat{K}(A)}(q\circ \opn{Hom}_{A}(R,D),q\circ \opn{Hom}_{A}(R,R\otimes_A P[n])) =\\
\opn{Hom}_{\cat{D}(A)}(q\circ \opn{Hom}_{A}(R,D),q\circ \opn{Hom}_{A}(R,R\otimes_A P[n])) \cong\\
\opn{Hom}_{\cat{D}(A)}(q\circ \opn{Hom}_{A}(R,D),P[n]).
\end{gather*}
We may now finish the proof as follows.
Consider the full subcategory of $\cat{D}(A)$ given by
\[
\mathcal{S} = \{X \in \cat{D}(A) \mid \forall n, \opn{Hom}_{\cat{D}(A)}(q\circ \opn{Hom}_{A}(R,D),X[n]) = 0\}.
\]
Clearly $\mathcal{S}$ is closed under shifts, taking cones and taking products,
so it is a colocalizing subcategory.
Moreover, we have seen that $\mathcal{S}$ contains all projective $A$-modules,
so we deduce that 
 $\opn{Coloc}(\opn{Proj}(A)) \subseteq \mathcal{S}$.
 Since $q\circ \opn{Hom}_{A}(R,-)$ is an equivalence of categories, 
and since $D \ncong 0$ in $\cat{K}(\opn{Inj}(A))$,
we deduce that $q\circ \opn{Hom}_{A}(R,D) \ncong 0$ in  $\cat{K}(\opn{Proj}(A))$.
Since it is K-projective, it follows that 
$q\circ \opn{Hom}_{A}(R,D) \ncong 0$ in $\cat{D}(A)$.
 We deduce that the identity map 
 \[
 q\circ \opn{Hom}_{A}(R,D) \to q\circ \opn{Hom}_{A}(R,D)
 \]
 is not the zero map.
 This shows that
 \[
 \opn{Hom}_{\cat{D}(A)}(q\circ \opn{Hom}_{A}(R,D),q\circ \opn{Hom}_{A}(R,D)) \ne 0,
 \]
 so that $q\circ \opn{Hom}_{A}(R,D) \notin \mathcal{S}$.
 Hence, $\opn{Coloc}(\opn{Proj}(A)) \subseteq \mathcal{S} \subsetneq \cat{D}(A)$
as claimed.
\end{proof}

\begin{rem}\label{rem:flat}
Over a left and right noetherian ring with a dualizing complex, it was shown in \cite[Theorem]{Jor05} that every flat module has finite projective dimension,
which implies that over such a ring there is an equality
\[
\opn{Coloc}(\opn{Proj}(A)) = \opn{Coloc}(\opn{Flat}(A)),
\]
so that one may replace projective modules by flat modules in the statement of the above result.
\end{rem}

\begin{rem}\label{rem:dual}
The dual result to \cref{thm:acyclic}, namely, that if $A$ is a left and right noetherian ring with a dualizing complex, 
such that $\opn{Loc}(\opn{Inj}(A)) = \cat{D}(A)$,
then every bounded below cochain complex of projective $A$-modules which is acyclic is null-homotopic also holds.
The proof of this fact is essentially contained in the proof of \cite[Theorem 5.1]{ShFinDim},
but we did not realize this at the time of writing that proof.
\end{rem}

We now restrict our attention to commutative rings.
In that case we obtain the following stronger unconditional result:
\begin{cor}\label{cor:commutative}
Let $A$ be a commutative noetherian ring which has a dualizing complex.
Then every bounded above cochain complex of injective $A$-modules which is acyclic is null-homotopic.    
\end{cor}
\begin{proof}
In view of \cref{thm:acyclic} we only need to show that $\opn{Coloc}(\opn{Proj}(A)) = \cat{D}(A)$,
or equivalently, by \cref{rem:flat},
that $\opn{Coloc}(\opn{Flat}(A)) = \cat{D}(A)$.
According to \cite[Theorem 3.3]{Rickard},
for any commutative noetherian ring $A$ it holds that $\opn{Loc}(\opn{Inj}(A)) = \cat{D}(A)$,
so the result follows from \cref{thm:injImplyFlat} using the fact that $A = A^{\op}$.
Alternatively, one may see directly that for any commutative noetherian ring it holds that $\opn{Coloc}(\opn{Flat}(A)) = \cat{D}(A)$, 
by using Neeman's classification \cite{Neeman2011} of colocalizing subcategories over commutative noetherian rings,
noticing that the flat $A$-module given by the $\p$-adic completion of the localization $A_{\p}$ has cosupport, 
in the sense of \cite{BIK},
equal to $\{\p\}$ for any $\p \in \opn{Spec}(A)$.
\end{proof}

\begin{cor}
Let $A$ be a commutative noetherian ring which has a dualizing complex,
and let $I,J$ be two bounded above cochain complexes of injective $A$-modules.
Then every quasi-isomorphism $\varphi:I \to J$ is a homotopy equivalence.
\end{cor}
\begin{proof}
Letting $D$ be the cone of $\varphi$,
the fact that $I$ and $J$ are both bounded above cochain complexes of injective $A$-modules implies that $D$ is also a bounded above cochain complex of injective $A$-modules.
The fact that $\varphi$ is a quasi-isomorphism implies that $D$ is acyclic, 
so by \cref{cor:commutative}, it is null-homotopic.
This in turn implies that $\varphi$ is a homotopy equivalence.
\end{proof}


\section{Injective finitistic dimension of Gorenstein DG-rings}

This section is dedicated to establishing the finiteness of the injective finitistic dimension of a Gorenstein DG-ring, 
under a further mild conditions that will hold in all of our applications.
This result will be used in the next section to connect the acyclicity of bounded above complexes of injectives with the projective finitistic dimension of a noetherian ring with a dualizing complex.



\begin{prop}\label{prop:flatProjdim}
Let $A$ be a non-positive DG-ring with bounded cohomology.
Suppose that every flat $\mrm{H}^0(A)$-module has finite projective dimension over $\mrm{H}^0(A)$.
Then every bounded DG-module $M \in \cat{D}^{\mrm{b}}(A)$ such that $\flatdim_A(M) < \infty$ satisfies $\projdim_A(M) < \infty$.
\end{prop}
\begin{proof}
By shifting if necessary, 
we may assume that $\sup(M) = 0$.
We prove the result by induction on $\flatdim_A(M)$. 
The fact that $\sup(M) = 0$ implies by the definition of flat dimension of DG-modules that $\flatdim_A(M) \ge 0$.
Assume first that $\flatdim_A(M) = 0$.
Let $\bar{M} = \mrm{H}^0(A)\otimes^{\mrm{L}}_A M$.
By \cite[proof of Proposition 3.1]{YeDual}, we have that $\sup(\bar{M}) = 0$,
and by \cite[Corollary 1.5(ii)]{DGFinite} it holds that $\flatdim_{\mrm{H}^0(A)}(\bar{M}) = 0$.
Moreover, the definition of flat dimension and the fact that $\inf(\mrm{H}^0(A)) = \sup(\mrm{H}^0(A)) = 0$ implies that $\inf(\bar{M}) = 0$.
We deduce that $\bar{M}$ is a flat $\mrm{H}^0(A)$-module,
so by assumption it has finite projective dimension over $\mrm{H}^0(A)$.
According to \cite[Corollary 1.5(i)]{DGFinite}, we know that $\projdim_A(M) = \projdim_{\mrm{H}^0(A)}(\bar{M})$,
establishing the claim.
Next, suppose that $\flatdim_A(M) > 0$.
By using \cite[Lemma 2.8]{Min},
we can find a DG-module of the form $F = \bigoplus_{j \in J} A$,
and a map of DG-modules $\varphi:F \to M$ such that $\mrm{H}^0(\varphi)$ is surjective.
Such a construction makes $\varphi$ the first step of a spft resolution of $M$,
in the sense of \cite[Definition 4.9]{Min}.
Embedding $\varphi$ inside a distinguished triangle
\begin{equation}\label{eqn:distForFlat}
N \to F \xrightarrow{\varphi} M \to N[1]
\end{equation}
in $\cat{D}(A)$,
since $\sup(F) = \sup(M) = 0$,
the fact that $\mrm{H}^0(\varphi)$ is surjective implies that $\sup(N) \le 0$.
By \cite[Lemma 4.10]{Min}, we have that $\flatdim_A(N) < \flatdim_A(M)$.
Hence, by the induction hypothesis, we know that $\projdim_A(N) < \infty$.
As $\projdim_A(F) = 0 <\infty$,
the distinguished triangle (\ref{eqn:distForFlat}) implies that $\projdim_A(M) < \infty$.
\end{proof}

The following fact is obvious for rings, 
but for DG-rings it might require a short proof:
\begin{prop}\label{prop:ineqflatproj}
Let $A$ be a non-positive DG-ring with bounded cohomology,
and let $M \in \cat{D}^{\mrm{b}}(A)$.
Then 
\[
\flatdim_A(M) \le \projdim_A(M).
\]
\end{prop}
\begin{proof}
We may assume that $\projdim_A(M) < \infty$,
for otherwise there is nothing to prove.
Let $\bar{M} = \mrm{H}^0(A)\otimes^{\mrm{L}}_A M \in \cat{D}(\mrm{H}^0(A))$.
We know that $\sup(\bar{M}) = \sup(M) < \infty$,
and by \cite[Corollary 1.5(i)]{DGFinite} we have that $\projdim_{\mrm{H}^0(A)}(\bar{M}) = \projdim_A(M) < \infty$,
so in particular $\bar{M}$ is a bounded complex.
Since flat and projective dimension over rings can be detected from the length of flat and projective resolutions, clearly $\flatdim_{\mrm{H}^0(A)}(\bar{M}) \le \projdim_{\mrm{H}^0(A)}(\bar{M})$.
Finally, by \cite[Corollary 1.5(ii)]{DGFinite} we have that $\flatdim_A(M) = \flatdim_{\mrm{H}^0(A)}(\bar{M})$,
so we obtain
\[
\flatdim_A(M) = \flatdim_{\mrm{H}^0(A)}(\bar{M}) \le \projdim_{\mrm{H}^0(A)}(\bar{M}) = \projdim_A(M).
\]
\end{proof}

\begin{prop}\label{prop:FFDifFPD}
Let $A$ be a non-positive DG-ring with bounded cohomology.
Suppose that every flat $\mrm{H}^0(A)$-module has finite projective dimension over $\mrm{H}^0(A)$.
If $\opn{FPD}(A) < \infty$ then $\opn{FFD}(A) \le \opn{FPD}(A) < \infty$.
\end{prop}
\begin{proof}
Set $n = \opn{FPD}(A) < \infty$, and let $M \in \cat{D}^{\mrm{b}}(A)$ be a DG-module with $\flatdim_A(M) < \infty$.   
By \cref{prop:flatProjdim}, we know that $\projdim_A(M) < \infty$.
Hence, by the definition of $\opn{FPD}(A)$, we deduce that $\projdim_A(M) + \inf(M) \le n$.
From \cref{prop:ineqflatproj} we obtain the inequality
\[
\flatdim_A(M) + \inf(M) \le \projdim_A(M) + \inf(M) \le n.
\]
The definition of $\opn{FFD}(A)$ now implies that 
\[
\opn{FFD}(A) \le n = \opn{FPD}(A) < \infty.
\]
\end{proof}

We will say that a DG-ring $A$ is left noetherian if the ring $\mrm{H}^0(A)$ is left noetherian and for all $i<0$,
the left $\mrm{H}^0(A)$-module $\mrm{H}^i(A)$ is finitely generated. Similarly for right noetherian. 
This definition is justified by \cite[Theorem 6.6]{ShINJ}.

\begin{prop}\label{prop:ffdFID}
Let $A$ be a left and right noetherian DG-ring with bounded cohomology.
Then $\opn{FFD}(A) = \opn{FID}(A^{\op})$.
\end{prop}
\begin{proof}
For rings, this is a well known result due to Matlis \cite[Theorem 1]{Matlis}.
For commutative DG-rings, this was shown in \cite[Corollary 6.16]{DGFinite},
and the same proof carries over to this noncommutative situation.
\end{proof}

Here is the main result of this section.
\begin{thm}\label{thm:FID}
Let $A$ be a left and right noetherian DG-ring with bounded cohomology,
and suppose that that every flat $\mrm{H}^0(A)$-module has finite projective dimension over $\mrm{H}^0(A)$.
If $\injdim_A(A) < \infty$ then $\opn{FID}(A^{\op}) < \infty$.
\end{thm}
\begin{proof}
According to \cite[Theorem 3.5]{ShFinDim},
the fact that $\injdim_A(A) < \infty$ implies that $\opn{FPD}(A) < \infty$.
By \cref{prop:FFDifFPD} this implies that $\opn{FFD}(A) < \infty$,
and from \cref{prop:ffdFID} we get that $\opn{FID}(A^{\op}) < \infty$.
\end{proof}

Given a ring $A$, and a complex of $A$-$A$-bimodules $\phantom{ }_A M_A$ with the property that $\sup(M) < 0$,
the trivial extension DG-ring $A\skewtimes M$ was defined in \cite[Definition 4.1]{ShFinDim}.
It is a non-positive DG-ring equipped with a map of DG-rings $\tau_{A,M}: A\to A\skewtimes M$,
such that as a complex of bimodules over $A$ it is isomorphic to $A\oplus M$,
such that $\mrm{H}^0(A\skewtimes M) = A$, and the composition $A \xrightarrow{\tau_{A,M}} A\skewtimes M \to \mrm{H}^0(A\skewtimes M) = A$ is equal to $1_A$.

\begin{cor}\label{cor:InfFID}
Let $A$ be a left and right noetherian ring,
and let $R$ be a dualizing complex over $A$ such that $\sup(R) < 0$.
Then the DG-ring $B = A\skewtimes R$ satisfies $\opn{FID}(B) < \infty$ and $\opn{FID}(B^{\op}) < \infty$.
\end{cor}
\begin{proof}
By \cite[Theorem]{Jor05}, 
the fact that $A$ has a dualizing complex implies that every flat $A$-module has finite projective dimension.
According to \cite[Theorem 4.5]{ShFinDim}, 
the DG-ring $B$ satisfies $\injdim_B(B) < \infty$ and $\injdim_{B^{\op}}(B^{\op}) < \infty$.
Since $\mrm{H}^0(B) = A$,
the result now follows from \cref{thm:FID}.
\end{proof}







\section{Acyclic complexes of injectives and finitistic dimension}

The aim of this section is to relate properties of acyclic complexes of injective modules to the finitistic dimension of a noetherian ring with a dualizing complex.

We first prove the following implication of a ring with a dualizing complex having infinite finitistic dimension,
which is a dual of \cite[Theorem 4.12]{ShFinDim}. 
This will then be used to construct a certain acyclic complex of injectives in such a situation.

\begin{thm}\label{thm:seperation}
Let $A$ be a left and right noetherian ring with a dualizing complex $S$.
Suppose that $\opn{FPD}(A^{\op}) = +\infty$.
Then there is an integer $j \in \mathbb{Z}$,
an increasing sequence of natural numbers $a_n$,
with $\lim_{n\to +\infty} a_n = +\infty$,
and for each $n$,
a left $A$-module $M_n$,
with $\injdim_A(M_n) = a_n$,
and $\opn{Ext}^{a_n}_A(R,M_n) \ne 0$,
where $R = S[j]$.
\end{thm}
\begin{proof}
The fact that $\opn{FPD}(A^{\op}) = +\infty$ implies by \cite[Corollary 4.10]{ShFinDim} that $\opn{FFD}(A^{\op}) = +\infty$,
so by \cite[Theorem 1]{Matlis}, 
we deduce that $\opn{FID}(A) = +\infty$.
This implies the existence of an increasing sequence of natural numbers $b_n$,
with $\lim_{n\to +\infty} b_n = +\infty$,
and for each $n$,
a left $A$-module $M_n$,
with $\injdim_{A}(M_n) = b_n$.
By shifting if necessary, 
we may assume that $\sup(S) < 0$.
Let  $B = A\skewtimes S$ be the trivial extension DG-ring of $A$ by $S$.
According to \cref{cor:InfFID}, 
it holds that $K = \opn{FID}(B) < \infty$.
For each $n$, let $N_n = \mrm{R}\opn{Hom}_A(B,M_n) \in \cat{D}(B)$.
The fact that the composition $A \to B \to \mrm{H}^0(B) = A$ is equal to the identity implies by adjunction that
\[
\mrm{R}\opn{Hom}_B(\mrm{H}^0(B),N_n) = \mrm{R}\opn{Hom}_B(A,N_n) \cong M_n.
\]
According to \cite[Theorem 2.5]{ShINJ},
there is an equality 
\[
\injdim_B(N_n) = \injdim_{\mrm{H}^0(B)}\left(\mrm{R}\opn{Hom}_B(\mrm{H}^0(B),N_n)\right) = \injdim_A(M_n) = b_n < \infty.
\]
Hence, by the definition of the injective finitistic dimension,
we have an inequality
\[
K = \opn{FID}(B) \ge \injdim_B(N_n) - \sup(N_n) = b_n - \sup(N_n).
\]
The definition of injective dimension over $A$ implies that
\[
\sup(N_n) = \sup\left(\mrm{R}\opn{Hom}_A(B,M_n)\right) \le b_n - \inf(B) = b_n - \inf(S).
\]
Combining the above, 
we see that for each $n$ there are inequalities
\[
-K \le \sup(N_n) - b_n \le -\inf(S).
\]
As $K,\inf(S)$ are fixed constants,
this implies the existence of some $j \in \mathbb{Z}$,
such that $\sup(N_n) - b_n = j$ for infinitely many $n$.
Focusing on these $n$, we obtain an increasing sequence $a_n$ of natural numbers, a subsequence of $b_n$,
such that $\injdim_A(M_n) = a_n$, 
and such that $\sup(N_n) = a_n + j$.
The fact that over $A$ it holds that $B \cong A\oplus S$
implies that
\[
\sup(N_n) = \sup\left(\mrm{R}\opn{Hom}_A(B,M_n)\right) = 
\sup\left(\mrm{R}\opn{Hom}_A(A\oplus S,M_n)\right) = 
\sup\left(\mrm{R}\opn{Hom}_A(S,M_n)\right).
\]
As $\mrm{R}\opn{Hom}_A(-,M_n)$ is a contravariant triangulated functor,
this implies that
\[
\sup\left(\mrm{R}\opn{Hom}_A(S[j],M_n)\right) = 
\sup\left(\mrm{R}\opn{Hom}_A(S,M_n)\right) - j =
\sup(N_n) - j = a_n,
\]
which shows that
\[
0 \ne \mrm{H}^{a_n}\left(\mrm{R}\opn{Hom}_A(S[j],M_n)\right) = \opn{Ext}^{a_n}_A(S[j],M_n) 
\]
for all $n$, as claimed.
\end{proof}

We now use this non-vanishing result to prove the main result of this section which relates acyclic complexes of injectives to the finitistic dimension.

\begin{thm}\label{thm:FPD}
Let $A$ be a left and right noetherian ring with a dualizing complex.
Assume that every bounded above complex of injective left $A$-modules which is acyclic is null-homotopic.
Then $\opn{FPD}(A^{\op}) < \infty$.
\end{thm}
\begin{proof}
Suppose that the conclusion is false,
so that $\opn{FPD}(A^{\op}) = +\infty$.
Then \cref{thm:seperation} holds,
which implies the existence of $a_n, M_n$ and $R$ as in the conclusion of that result.
Note that $R$, 
being a shift of a dualizing complex is also a dualizing complex.
For each $n$,
let $I_n$ be an injective resolution of $M_n$ of length $a_n$.
Consider the natural map
\[
\varphi: \bigoplus_{n=1}^{\infty} I_n[a_n] \to \prod_{n=1}^{\infty} I_n[a_n]
\]
Since $\mrm{H}^{-a_n}(I_n[a_n]) = M_n$,
and $\mrm{H}^i(I_n[a_n]) = 0$ for all $i\ne -a_n$,
it follows that $\varphi$ is a quasi-isomorphism.

Next, we claim that $\varphi$ is not a homotopy equivalence.
To see this, 
recall that the restriction of $R$ to $\cat{D}(A)$, 
the complex complex of left $A$-modules $\opn{Rest}_A(R)$ has bounded and finitely generated cohomology.
Hence, by \cite[Proposition 7.4.9]{YeBook},
there is an isomorphism $\opn{Rest}_A(R) \cong S$ in $\cat{D}(A)$, 
where $S$ is a bounded complex of finitely generated $A$-modules. 
This in turn implies that the functor $\opn{Hom}_A(S,-)$ commutes with both products and coproducts.
Applying this functor to $\varphi$,
and using the fact that 
\[
\mrm{H}^0\left(\opn{Hom}_A(S,I_n[a_n])\right) = \opn{Ext}^{a_n}_A(S,M_n) \cong \opn{Ext}^{a_n}_A(R,M_n),
\]
we deduce that $\mrm{H}^0\left(\opn{Hom}_A(S,\varphi)\right)$ is isomorphic to the natural map
\[
\bigoplus_{n=1}^{\infty}\opn{Ext}^{a_n}_A(R,M_n) \to \prod_{n=1}^{\infty} \opn{Ext}^{a_n}_A(R,M_n).
\]
Since for all $n$, 
it holds that $\opn{Ext}^{a_n}_A(R,M_n) \ne 0$,
we deduce that the map $\mrm{H}^0\left(\opn{Hom}_A(S,\varphi)\right)$ is not an isomorphism.
As $\opn{Hom}_A(S,-)$ is an additive functor,
it preserves homotopy equivalences,
which shows that $\varphi$ is not a homotopy equivalence.

The Bass-Papp theorem implies that $\varphi$ is a map between bounded above complexes of injective $A$-modules.
This shows that its cone $D = \opn{cone}(\varphi)$ is also bounded above complex of injective $A$-modules.
Since $\varphi$ is a quasi-isomorphism,
it follows that $D$ is acyclic,
and since $\varphi$ is not homotopy equivalence,
it follows that $D$ is not null-homotopic,
contradicting our assumption on $A$.
\end{proof}

Combining \cref{thm:acyclic} and \cref{thm:FPD}, 
we have the following result:
\begin{cor}
Let $A$ be a left and right noetherian ring with a dualizing complex.
If $\opn{Coloc}(\opn{Proj}(A)) = \cat{D}(A)$,
then $\opn{FPD}(A^{\op}) < \infty$.
\end{cor}

\begin{rem}
In \cite[Theorem 4.3]{Rickard},
Rickard showed that if $A$ is a finite dimensional algebra over a field and $\opn{Loc}(\opn{Inj}(A)) = \cat{D}(A)$ then $\opn{FPD}(A) < \infty$.
That result was generalized by us in \cite[Corollary 5.3]{ShFinDim}, where the same result was shown to hold for every left and right noetherian ring with a dualizing complex.
It should be noted that every finite dimensional algebra over a field has a dualizing complex.
Similarly, Rickard showed in \cite[Proposition 5.2]{Rickard} that if $A$ is a finite dimensional algebra over a field and $\opn{Coloc}(\opn{Proj}(A)) = \cat{D}(A)$,
then $\opn{FPD}(A^{\op}) < \infty$,
and so the above result completes the picture and generalizes to the same setting Rickard's dual result.
Some ideas of Rickard's proof of \cite[Proposition 5.2]{Rickard} are contained in the proofs of \cref{thm:acyclic} and \cref{thm:FPD},
and one should view his result as a predecessor of both of these theorems.
\end{rem}


\section{Gorenstein symmetry}

In this section we return to the Gorenstein symmetry problem discussed in the introduction.
We start with the following fact,
which is probably known,
and give a proof of it using the covariant Grothendieck duality.

\begin{prop}\label{prop:flatdimOfDual}
Let $A$ be a left and right noetherian ring with a dualizing complex.
Then the following are equivalent:
\begin{enumerate}
\item Every injective $A$-module has finite projective dimension.
\item Every injective $A$-module has finite flat dimension.
\item There is a dualizing complex $R$ over $A$ such that $\flatdim_A(R) < \infty$.
\end{enumerate}
\end{prop}
\begin{proof}
Since $A$ has a dualizing complex,
by \cite[Theorem]{Jor05}, every $A$-module of finite flat dimension has finite projective dimension, 
which shows that (1)$\iff$(2).
Let $R$ be a dualizing complex over $A$.
Since by definition it has finite injective dimension over $A$,
it is clear that (2)$\implies$(3).
To finish the proof, we must show that (3)$\implies$(2),
so suppose that $\flatdim_A(R) < \infty$,
and let $I$ be an injective $A$-module.
Applying the covariant Grothendieck duality,
let $P = q(\opn{Hom}_A(R,I)) \in \cat{K}(\opn{Proj}(A))$.
It follows from \cite[Lemma 4.1(b)]{Pos} that $\opn{Hom}_A(R,I)$ is a bounded complex of flat $A$-modules,
so using \cite[Theorem]{Jor05} again,
we deduce from \cite[Theorem 2.7]{InKr}  that $P$ is a bounded complex of projective $A$-modules.
In particular, $P$ is K-flat, 
so there is an isomorphism
\[
R\otimes^{\mrm{L}}_A P \cong R\otimes_A P,
\]
in $\cat{D}(A)$
and since the left hand side has finite flat dimension over $A$,
we deduce that $R\otimes_A P$ has finite flat dimension over $A$.
According to \cite[Theorem 4.8]{InKr},
there is an isomorphism $R\otimes_A P \cong I$ in $\cat{K}(\opn{Inj}(A))$,
so a fortiori there is such an isomorphism in $\cat{D}(A)$,
showing that $\flatdim_A(I) < \infty$.
\end{proof}

The relation between Gorensteinness and dualizing complexes is expressed in the following elementary fact:
\begin{prop}\label{prop:gorDual}
Let $A$ be a left and right noetherian ring,
such that $\injdim_A(A) = \injdim_{A^{\op}}(A^{\op}) < \infty$.
 If the regular bimodule $\phantom{ }_A A_A$ has a bi-injective resolution,
 then $A$ has a dualizing complex.
\end{prop}
\begin{proof}
Since $\injdim_A(A) = \injdim_{A^{\op}}(A^{\op}) < \infty$,
the regular bimodule $\phantom{ }_A A_A$ is a weak dualizing complex (in the sense of \cref{dfn:weak}) over itself,
so by \cref{prop:bi-injective}, 
there is a dualizing complex over $A$.
\end{proof}

We now turn to give various necessary and sufficient conditions for Gorenstein symmetry to hold.

\begin{thm}\label{thm:Gorenstein}
Let $A$ be a ring which is left and right noetherian,
and such that the regular bimodule $\phantom{ }_A A_A$ has a bi-injective resolution.
Assume that $\injdim_A(A) < \infty$.
Then the following are equivalent:
\begin{enumerate}
\item $A$ satisfies the Gorenstein symmetry conjecture: It holds that $\injdim_{A^{\op}}(A^{\op}) < \infty$.
\item $A$ has a dualizing complex and $\opn{Coloc}(\opn{Proj}(A)) = \cat{D}(A)$.
\item $A$ has a dualizing complex and $\opn{Coloc}(\opn{Flat}(A)) = \cat{D}(A)$.
\item $A$ has a dualizing complex, and every bounded above complex of injective $A$-modules which is acyclic is null-homotopic.
\item $A$ has a dualizing complex, and every injective $A$-module has finite flat dimension.
\item $A$ has a dualizing complex $R$ which has finite flat dimension as a complex of left $A$-modules.
\end{enumerate}
\end{thm}
\begin{proof}
(1) $\implies$ (3):  If $\injdim_{A^{\op}}(A^{\op}) < \infty$,
then by \cite[Theorem 3.2]{Rickard}, 
it holds that $\opn{Loc}(\opn{Inj}(A^{\op})) = \cat{D}(A^{\op})$,
so by \cref{thm:injImplyFlat} it follows that $\opn{Coloc}(\opn{Flat}(A)) = \cat{D}(A)$. Moreover,
by \cref{prop:gorDual}, $A$ has a dualizing complex.
(2) $\iff$ (3): this follows from \cite[Theorem]{Jor05},
since over a ring with a dualizing complex, a module has finite flat dimension if and only if it has finite projective dimension.
(2) $\implies$ (4): this is \cref{thm:acyclic}.
(4) $\implies$ (1): according to \cref{thm:FPD},
the assumption implies that $\opn{FPD}(A^{\op}) < \infty$,
and by \cite[Proposition 6.10]{AusRei}, 
this implies that $\injdim_{A^{\op}}(A^{\op}) < \infty$.
(5) $\iff$ (6): this is contained in \cref{prop:flatdimOfDual}.
(5) $\implies$ (3): If every injective $A$-module has finite flat dimension, then every injective $A$-module is contained in $\opn{Coloc}(\opn{Flat}(A))$,
so in particular $\opn{Coloc}(\opn{Inj}(A)) \subseteq \opn{Coloc}(\opn{Flat}(A))$, 
but for any ring it holds that $\opn{Coloc}(\opn{Inj}(A)) = \cat{D}(A)$, 
establishing the claim.
Finally, we show that (1) $\implies$ (5): Let $I$ be a left injective $A$-module,
and let $F = \opn{Hom}_{\mathbb{Z}}(I,\mathbb{Q}/\mathbb{Z})$.
By \cite[Theorem 3.2.16]{EnJen}, the right $A$-module $F$ is flat, so in particular it has finite projective dimension,
and hence, by assumption, also finite injective dimension over $A^{\op}$.
Hence, by \cite[Theorem 3.2.19]{EnJen}, 
it follows that $I$ has finite flat dimension over $A$.
\end{proof}

\begin{rem}
There is some resemblance between the fact that (4)$\iff$(5) and \cite[Corollary 5.12]{InKr},
where it is shown that for a left and right noetherian ring $A$ with a dualizing complex, the dualizing complex has finite projective dimension over $A$ if and only if every acyclic complex of injective $A$-modules is totally acyclic.
There are however several differences between the results.
First, our result is concerned only with bounded above complexes of injectives, while there the condition is on all complexes of injectives.
Secondly, our conclusion that such complexes are null-homotopic is stronger than being totally acyclic.
\end{rem}

\bibliographystyle{abbrv}
\bibliography{main}


\end{document}
