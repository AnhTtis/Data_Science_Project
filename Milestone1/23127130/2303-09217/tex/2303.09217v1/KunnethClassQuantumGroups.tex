\documentclass[USenglish, a4paper, 10pt]{article}
\usepackage[utf8]{inputenc} %pour avoir un bon codage d'entrée (caractères accentués). utf-8 c'est le système de codage
\usepackage{babel} %pour respecter les usages de la langue du document (specifiée dans documentclass)
\usepackage{lmodern} %police : famille de fonts LatinModern
\usepackage[T1]{fontenc} %codage latin
%\bibliographystyle{abbrv}
\usepackage{amssymb, amsmath, mathrsfs, mathabx, bbm} %inclusion de différentes types de fonts
\newcommand{\bbfamily}{\fontencoding{U}\fontfamily{bbold}\selectfont} %pour utiliser le style \mathbb{} avec les lettres greques. Il faut utiliser la commande \mathbbold{}
\DeclareMathAlphabet{\mathbbold}{U}{bbold}{m}{n}%pour utiliser le style \mathbb{} avec les lettres greques. Il faut utiliser la commande \mathbbold{}
\usepackage{stmaryrd}
\usepackage[a4paper]{geometry} %pour controler les dimensions et l'orientation des pages du document
	\geometry{tmargin=16mm,bmargin=16mm, lmargin=20mm,rmargin=20mm}
	%\geometry{tmargin=1.7cm,bmargin=5.3cm}


%\setlength{\topmargin}{5pt}
%\setlength{\textheight}{655pt}
%\setlength{\textwidth}{475pt}
%\setlength{\headsep}{11pt}
%\setlength{\parindent}{0pt}
%\setlength{\oddsidemargin}{0pt}
%\setlength{\evensidemargin}{28mm}
\setlength{\parskip}{1ex plus 0.5ex minus 0.2ex}
\usepackage{titling} %inclusion du titre, auteur et resume comme dans les articles scientifiques des revues
\usepackage{titlesec} %pour modifier les parametres des sections, chapitres, etc.
		\titlelabel{\thetitle.\quad} %ajouter un point apres le compteur des sections
		\titleformat*{\section}{\center\large} %on met les titres des section dans le centre de la page
		\titleformat*{\subsection}{\sf\large} %style de la police pour les subsections
		\titleformat*{\subsubsection}{\sf\it} %style de la police pour les subsubsections
		
\usepackage[colorlinks=true,linkcolor=black, citecolor=GreenCite,urlcolor=GreenCite]{hyperref} %inclusion de liens sur internet ou des liens dans le document lui-même. Il faut dire que l'inclusion de ce paquet implique automatiquement la hiperliens dedans le document, en particulier on peut cliquer sur les titres de la table des matières ou dans les reférences aux formules ou aux livres de la bibliographie. On écrit \url{direction web} ou \href{direction web}{NOM alternative pour le lien} Les paramètres indiqués signifient que :
		%``colorlinks=false'', on aurait un cadre dans le lien
		%``colorlinks=true'', on n'aurait un cadre, mais des couleurs
		%``linkcolor=...'', on définit la couleur des liens internes dans le document
		%``citecolor=...'', on définit la couleurs des liens aux reférences
		%``urlcolor:...'', on définit la couleurs des liens à internet
\usepackage[usenames,dvipsnames]{color} %inclusion de plusieurs couleurs à part des standars (blue, yellow, black, white, magenta, cyan, green, red)
\definecolor{ToDo}{RGB}{30,144,255}
\definecolor{Provisional}{RGB}{218,165,32}
\definecolor{Question}{RGB}{220,20,60}
\definecolor{GreenCite}{RGB}{47, 79, 79}

\usepackage{verbatim} % commentaires
\usepackage{array} %inclusion de tableaux dans le mode mathématicien
\usepackage[all]{xy} %inclusion de diagrammes (fleches, etc.)
\usepackage[usenames,dvipsnames]{color} %inclusion de plusieurs couleurs à part des standars (blue, yellow, black, white, magenta, cyan, green, red)
\usepackage{enumerate} %configuration des listes
\usepackage{leftidx}
\usepackage{paralist}


\numberwithin{equation}{section} %on numérote les équations selon la section où on se trouve
\usepackage{amsthm} %inclusion des formats pour les théorèmes, remarques, etc.
\swapnumbers %on numérote les énoncés de la forme "2.0 Theorem" au lieu de "Theorem 2.0"
\renewcommand{\qedsymbol}{$\blacksquare$} %on met un carré noir à la fin des démonstrations
\theoremstyle{plain}
\newtheorem{theoSec}{Theorem}[section] %enumération à niveau de section
\newtheorem{theodefiSec}[theoSec]{Theorem-Definition}
\newtheorem{lemSec}[theoSec]{Lemma}
\newtheorem{proSec}[theoSec]{Proposition}
\newtheorem{corSec}[theoSec]{Corollary}
\newtheorem{defiSec}[theoSec]{Definition}
\newtheorem{remSec}[theoSec]{Remark}
\newtheorem{remsSec}[theoSec]{Remarks}
\newtheorem{exSec}[theoSec]{Example}
\newtheorem{exsSec}[theoSec]{Examples}
\theoremstyle{remark}
\newtheorem{noteSec}[theoSec]{Note}
\newtheorem{notesSec}[theoSec]{Notes}
\theoremstyle{plain}
\newtheorem{theo}{Theorem}[subsection] %enumération à niveau de sous-section
\newtheorem{theodefi}[theo]{Theorem-Definition}
\newtheorem{lem}[theo]{Lemma}
\newtheorem{pro}[theo]{Proposition}
\newtheorem{cor}[theo]{Corollary}
\newtheorem{defi}[theo]{Definition}
\newtheorem{rem}[theo]{Remark}
\newtheorem{rems}[theo]{Remarks}
\newtheorem{ex}[theo]{Example}
\newtheorem{exs}[theo]{Examples}
\theoremstyle{remark}
\newtheorem{note}[theo]{Note}
\newtheorem{notes}[theo]{Notes}
\theoremstyle{plain}
\newtheorem{theoSubSub}{Theorem}[subsubsection] %enumération à niveau de sous-sous-section
\newtheorem{theodefiSubSub}[theo]{Theorem-Definition}
\newtheorem{lemSubSub}[theoSubSub]{Lemma}
\newtheorem{proSubSub}[theoSubSub]{Proposition}
\newtheorem{corSubSub}[theoSubSub]{Corollary}
\newtheorem{defiSubSub}[theoSubSub]{Definition}
\newtheorem{remSubSub}[theoSubSub]{Remark}
\newtheorem{remsSubSub}[theoSubSub]{Remarks}
\newtheorem{exSubSub}[theoSubSub]{Example}
\newtheorem{exsSubSub}[theoSubSub]{Examples}
\theoremstyle{remark}
\newtheorem{noteSubSub}[theoSubSub]{Note}
\newtheorem{notesSubSub}[theoSubSub]{Notes}



%Titre, auteur, date du document:
\title{\textbf{Torsion for quantum direct products \\and the Künneth class}}
\author{\textsc{Rubén Martos}\thanks{Department of Mathematical Sciences, University of Copenhagen (Denmark). R.M. is supported by the European Union's Horizon 2020 research and innovation programme under the Marie Skłodowska-Curie grant agreement No 895141.}}
\date{}
\begin{document}
\maketitle
\renewcommand{\abstractname}{}
\vspace{-2.5cm}
\begin{abstract}
\textsc{Abstract}. We classify torsion actions of quantum direct products. On the one hand, this allows to improve several results appearing already in a previous work of the author around permanence properties of the (resp. strong) quantum Baum-Connes property with respect to quantum direct products. On the other hand, we introduce a Künneth class in the quantum equivariant setting inspired by the pioneer work by J. Chabert, H. Oyono-Oyono and S. Echterhoff, which allows to relate the quantum Baum-Connes property with the Künneth formula by generalising some key results of Chabert-Oyono-Oyono-Echterhoff to discrete quantum groups. Finally, we make the observation that the C$^*$-algebra defining a compact quantum group with dual satisfying the strong quantum Baum-Connes property belongs to the Künneth class. This allows to obtain some K-theory computations for quantum direct products based on earlier work by Voigt and Vergnioux-Voigt.
	
\bigskip
\textsc{Keywords.} Baum-Connes conjecture, compact/Discrete quantum groups, Deligne's tensor product, K-theory, Künneth formula, (module) C$^*$-categories, quantum direct product, tensor product of C$^*$-algebras, torsion, triangulated categories, universal coefficient theorem.
\end{abstract}

\tableofcontents

\section{\textsc{Introduction}}
	
	Universal Coefficient Theorems in algebraic topology establish a connection between ordinary homology (resp. cohomology) with homology (resp. cohomology) with coefficients. In noncommutative geometry, C$^*$-algebras are viewed as noncommutative analogues of topological spaces and as such it is reasonable to extend ideas from algebraic topology to a noncommutative framework. In this sense, K-theory of C$^*$-algebras is viewed as a homology theory for C$^*$-algebras, which provides important invariants for C$^*$-algebras (e.g. AF-algebras are completely classified through K-theory \cite{ElliotAF}). Dually, K-homology for C$^*$-algebras is a cohomology theory for C$^*$-algebras, which agree with the $\text{Ext}$-functor in the commutative unital case. K-theory and K-homology can be related to each other by means of the index theory of elliptic pseudo-differential operators (e.g. the Atiyah-Singer's theorem). Kasparov KK-theory is in turn a bivariant K-theory, which gathers together both K-theory and K-homology of C$^*$-algebras. KK-theory plays a central role in the classification program of C$^*$-algebras, but it is also relevant to approach diverse problems outside noncommutative geometry, e.g. the Novikov conjecture.
	
	The Universal Coefficient Theorem (\emph{UCT} for short) of J. Rosenberg and C. Schochet \cite{RosenbergSchochet} approximates the bivariant K-theory of Kasparov in terms of ordinary K-theory. More precisely, given two (separable) C$^*$-algebras $A$ and $B$, there exists a short exact sequence:
	$$\text{Ext}^1_{\mathbb{Z}}(K_*(A), K_*(B))\rightarrowtail KK(A, B)\twoheadrightarrow \text{Hom}(K_*(A), K_*(B)),$$
	provided that $A$ belongs to certain bootstrap class. A remarkable consequence of this is that K-equivalences between C$^*$-algebras satisfying the UCT lift to KK-equivalences. A classical companion to the UCT is the \emph{Künneth theorem} (cf. \cite{RosenbergSchochet}), which asserts that given two (separable) C$^*$-algebras $A$ and $B$, there exists a short exact sequence:
	$$K_*(A)\otimes K_*(B)\rightarrowtail K_*(A\otimes B)\twoheadrightarrow \text{Tor}^{\mathbb{Z}_1}(K_*(A), K_*(B)),$$
	provided that $A$ belongs to certain bootstrap class. Roughly speaking, Künneth theorem allows to compute the K-theory of a tensor product of two C$^*$-algebras in terms of the K-theory of the corresponding factors (e.g. when $K(A)$ or $K(B)$ is torsion-free). When the the first arrow in the above diagram is an isomorphism, we refer to $K_*(A)\otimes K_*(B)\cong K_*(A\otimes B)$ as the \emph{Künneth formula}. Classically, the Künneth formula identifies the (co-)homology of a product of topological spaces with the tensor product of (co-)homologies.
	
	In \cite{ChabertEchterhoffOyono}, J. Chabert, S. Echterhoff and H. Oyono-Oyono establish a connection between the Baum-Connes conjecture of a locally compact group $G$ with coefficients in a C$^*$-algebra $A$ and the Künneth formula for the K-theory of tensor products by the corresponding crossed product $A\underset{r}{\rtimes} G$. One remarkable application of the techniques developed by J. Chabert, S. Echterhoff and H. Oyono-Oyono is a permanence property of the Baum-Connes conjecture for a direct product of locally compact groups with trivial coefficients.
	
	In this article, we establish such a connection in analogy to \cite{ChabertEchterhoffOyono} in the realm of discrete quantum groups and prove a similar permanence property for the quantum counterpart of the Baum-Connes conjecture. It is important to say that such a study was partially initiated by the author in \cite{RubenSemiDirect}, but only under torsion-freeness assumption. This was due mainly to two technical difficulties: the classification of torsion actions for quantum direct products and the formulation of a quantum assembly map for arbitrary discrete quantum groups. Let us give more details about these issues.
	
	The Baum-Connes conjecture has been formulated in 1982 by P. Baum and A. Connes. We still do not know any counter example to the original conjecture but it is known that the one with coefficients is false. For this reason we refer to the Baum-Connes conjecture with coefficients as the \emph{Baum-Connes property} (\emph{BC property} for short). The conjecture aims to understand the link between two operator K-groups of different nature, which would establish a strong connexion between geometry and topology in a more abstract and general index-theory context. More precisely, if $G$ is a (second countable) locally compact group and $A$ is a (separable) $G$-$C^*$-algebra, then the BC property for $G$ with coefficients in $A$ claims that the assembly map $\mu^G_A: K^{top}_{*}(G; A)\longrightarrow K_{*}(A\underset{r}{\rtimes}G)$ is an isomorphism, where $K^{top}_{*}(G; A)$ is the equivariant K-homology with compact support of $G$ with coefficients in $A$ and $K_{*}(A\underset{r}{\rtimes}G)$  is the K-theory of the reduced crossed product $A\underset{r}{\rtimes}G$. This property has been proved for a large class of groups (e.g. a-T-menable groups \cite{HigsonKasparovHaagerup} or hyperbolic groups \cite{Lafforgue}).
	
	On the one hand, a major problem when studying the quantum counterpart of the BC property for a discrete quantum group $\widehat{\mathbb{G}}$ is the torsion structure of $\widehat{\mathbb{G}}$. Indeed, if $G$ is a discrete group, its torsion phenomena is completely described in terms of the finite subgroups of $G$ and encoded in the localizing subcategory of $\mathscr{K}\mathscr{K}^{G}$ of \emph{compactly induced $G$-$C^*$-algebras}, denoted by $\mathscr{L}_G$, according to the Meyer-Nest reformulation \cite{MeyerNest}. We say that $G$ satisfies the \emph{strong} BC property if $\mathscr{L}_G=\mathscr{K}\mathscr{K}^{G}$, which corresponds, in usual terms, to the existence of a $\gamma$-element that equals $\mathbbold{1}_{\mathbb{C}}$ (cf. \cite{MeyerNest}). This approach yields as well a characterization of the BC property of a discrete group $G$ \emph{only} in terms of finite subgroups, K-theory and crossed products. The notion of torsion for a genuine discrete quantum group, $\widehat{\mathbb{G}}$, has been introduced firstly by R. Meyer and R. Nest \cite{MeyerNestTorsion}, \cite{MeyerNestHomological2} in terms of ergodic actions of $\mathbb{G}$. It has been re-interpreted later by Y. Arano and K. De Commer in terms of fusion rings and module $C^*$-categories \cite{YukiKenny}. More details about the notion of torsion can be found in Section \ref{sec.TorsionQG}.
	
	Given a compact quantum group $\mathbb{G}$, an important question in this context is to classify all torsion actions of $\mathbb{G}$ (at least up to equivariant Morita equivalence). Such a classification is known for the most common examples of compact quantum groups, e.g for the q-deformation of $SU(2)$, the free unitary quantum group $U^+(n)$, the free orthogonal quantum group $O^+(n)$ or the quantum permutation group $S^+_N$ with $n, N\in\mathbb{N}$ (cf. \cite{VoigtBaumConnesFree, VoigtBaumConnesAutomorphisms, YukiKenny}). In fact, $\widehat{SU_q(2)}$, $\widehat{U^+(n)}$ and $\widehat{O^+(n)}$ are all torsion-free and the only, up to equivariant Morita equivalence, non-trivial torsion action of $S^+_N$ is its structural action as quantum automorphism group of $\mathbb{C}^N$. However, little is known about the classification of torsion actions for general constructions of quantum groups. In this direction, the only, to the best knowledge to the author, such a general result was given in \cite{RubenAmauryTorsion} by the author in collaboration with A. Freslon. It is shown in \cite{RubenAmauryTorsion} that torsion actions of a free product of compact quantum groups are in one-to-one correspondence, up to equivariant Morita equivalence, with torsion actions of each of the factors. Here we make progress in this direction by classifying torsion actions of a direct product of compact quantum groups (cf. Theorem \ref{theo.TorsionDirectProd} and Theorem \ref{theo.TorsionDirectProd2}). 
	
	It is important to mention that by virtue of the work \cite{YukiKenny} by Y. Arano and K. De Commer, we know that if both $\widehat{\mathbb{G}}$ and $\widehat{\mathbb{H}}$ are torsion-free, then $\widehat{\mathbb{G}\times \mathbb{H}}$ is torsion-free too. The converse is also true because both $\widehat{\mathbb{G}}$ and $\widehat{\mathbb{H}}$ can be viewed as \emph{divisible} discrete quantum subgroups of $\widehat{\mathbb{G}\times \mathbb{H}}$ and torsion-freeness is preserved under divisible discrete quantum subgroups as shown in \cite{RubenTorsionDivisibles}. However, it is an open problem to classify all torsion actions of $\mathbb{G}\times\mathbb{H}$. In Section \ref{sec.TorsionQuantumDirProd} we show that any torsion action of $\mathbb{G}\times\mathbb{H}$ is a tensor product of a torsion action of $\mathbb{G}$ with a torsion action of $\mathbb{H}$. This is done by using two different methods. The one presented in Section \ref{sec.ClassTorActDirectProd} is based on the approach to torsion in terms of module C$^*$-categories by \cite{YukiKenny}. More precisely, we relate torsion actions of $\mathbb{G}\times \mathbb{H}$ to a Deligne's tensor product of certain module categories which allows to apply general results by V. Ostrik in module categories. The method presented in Section \ref{sec.ClassTorActDirectProd} is more operator-algebraic oriented based on the spectral theory for compact quantum groups. Notice that this classification result yields as a corollary the torsion-freeness result for $\widehat{\mathbb{G}\times\mathbb{H}}$ by Y. Arano and K. De Commer.
	
	On the other hand, in order to apply the Meyer-Nest strategy in the quantum setting, one needs a \emph{complementary pair} of localizing subcategories, $(\mathscr{L}_{\widehat{\mathbb{G}}},\mathscr{N}_{\widehat{\mathbb{G}}})$, where $\mathscr{L}_{\widehat{\mathbb{G}}}$ must encode the torsion phenomena of $\widehat{\mathbb{G}}$. A candidate was proposed in \cite{MeyerNestTorsion} and \cite{VoigtBaumConnesAutomorphisms}, but it has been an open question whether the corresponding pair is complementary in $\mathscr{K}\mathscr{K}^{\widehat{\mathbb{G}}}$, which prevented from having a definition of a quantum assembly map whenever $\widehat{\mathbb{G}}$ is not torsion-free. Recently, Y. Arano and A. Skalski \cite{YukiBCTorsion} have observed that the candidates for $\mathscr{L}_{\widehat{\mathbb{G}}}$ and $\mathscr{N}_{\widehat{\mathbb{G}}}$ form indeed a complementary pair of subcategories in $\mathscr{K}\mathscr{K}^{\widehat{\mathbb{G}}}$, which allows to define a quantum assembly map for every discrete quantum group $\widehat{\mathbb{G}}$ (torsion-free or not). Moreover, following a different approach by studying the projective representation theory of a compact quantum group, the same conclusion is reached for permutation torsion-free discrete quantum groups by the author in collaboration with K. De Commer and R. Nest \cite{KennyNestRubenBCProjective}. More details about this formulation can be found in Section \ref{sec.QuantumBC}.
	
	The classification of torsion actions of a quantum direct product together with the formulation of a quantum assembly map for any discrete quantum group allow to improve several results in this matter appearing already in \cite{RubenSemiDirect}. For instance, the triangulated functor $\mathcal{Z}:\mathscr{K}\mathscr{K}^{\widehat{\mathbb{G}}}\times \mathscr{K}\mathscr{K}^{\widehat{\mathbb{H}}} \rightarrow \mathscr{K}\mathscr{K}^{\widehat{\mathbb{G}\times \mathbb{H}}}$ given by the exterior tensor product of Kasparov triples allows to describe appropriately, \emph{and without any torsion-freeness assumption}, the quantum BC property for $\widehat{\mathbb{G}\times \mathbb{H}}$ in terms of the quantum BC property for $\widehat{\mathbb{G}}$ and $\widehat{\mathbb{H}}$. Namely, if $\widehat{\mathbb{G}}$ and $\widehat{\mathbb{H}}$ satisfy the strong BC property, then we show that $\widehat{\mathbb{G}\times \mathbb{H}}$ satisfies the strong BC property with coefficients in $A\otimes B$, for all $\widehat{\mathbb{G}}$-C$^*$-algebra $A$ and all $\widehat{\mathbb{H}}$-C$^*$-algebra $B$ (cf. Theorem \ref{theo.StrongBCDirectProd}).  Accordingly, an analogous assertion about the \emph{usual} quantum BC property needs further hypothesis related to the Künneth formula in order to compute the K-theory of a tensor product. Therefore, we are lead to consider a (quantum) equivariant analogue of the Künneth class $\mathcal{N}$, say $\mathcal{N}_{\widehat{\mathbb{G}}}$, containing those $\widehat{\mathbb{G}}$-C$^*$-algebras which make possible such a K-theory computation. This is done in analogy to the work by J. Chabert, S. Echterhoff and H. Oyono-Oyono in \cite{ChabertEchterhoffOyono}. If $\widehat{\mathbb{G}}$ is a classical locally compact group $G$, then $\mathcal{N}_{\widehat{\mathbb{G}}}=\mathcal{N}_G$ as defined in \cite{ChabertEchterhoffOyono}. Furthermore, our approach, as based in the Meyer-Nest categorical framework, yields a characterisation of the objects in the equivariant Künneth class in terms of the non-equivariant one, up to replacing $A$ by a $\mathscr{L}_{G}$-simplicial approximation of $A$.  This study is contained in Section \ref{sec.KunnethFunctors}.
	
	In Section \ref{sec.BCKunneth} we generalise some key results appearing in \cite{ChabertEchterhoffOyono} about the connection between the BC property with the Künneth formula. Namely, let $A$ be a $\widehat{\mathbb{G}}$-C$^*$-algebra and $B$ a C$^*$-algebra. Then we show that $A\in \mathcal{N}_{\widehat{\mathbb{G}}}$ $\Leftrightarrow$ $A\underset{r}{\rtimes}\widehat{\mathbb{G}}\in\mathcal{N}$ provided that $\widehat{\mathbb{G}}$ satisfies the BC property with coefficients in $A\otimes B$ (cf. Proposition \ref{pro.BCKunneth}). One remarkable result in \cite{ChabertEchterhoffOyono} is the following permanence property of the BC property for a direct product of locally compact groups. Let $G$ and $H$ be two locally compact groups satisfying the BC property with trivial coefficients. If $C^*(G)$ or $C^*(H)$ belongs to $\mathcal{N}$, then $G\times H$ satisfies the BC property with trivial coefficients (cf. \cite[Theorem 5.3]{ChabertEchterhoffOyono}). The analogue statement for quantum groups is stated and proven in Theorem \ref{theo.BCDirectProducts}. In the quantum setting further hypotheses are needed concerning the behaviour of $\mathbb{C}$ with respect to the \emph{equivariant} Künneth formula. Namely, we have to assume that $\mathbb{C}\in\mathcal{N}_{\widehat{\mathbb{G}}}$. This supplementary condition is automatically fulfilled in the classical setting because $\mathbb{C}$ is a type I C$^*$-algebra and $\mathcal{N}_G$ contains all type I $G$-C$^*$-algebras by virtue of \cite[Theorem 0.1]{ChabertEchterhoffOyono}. In the quantum setting, a similar related result is Theorem \ref{theo.QuantumGroupCalgebraKunneth}. However, to the best knowledge of the author, it is not known for instance whether $\mathbb{C}\in\mathcal{N}_{\widehat{\mathbb{G}}}$ for every discrete quantum group $\widehat{\mathbb{G}}$. One reason for this is that in our approach the objects in $\mathcal{N}_{\widehat{\mathbb{G}}}$ are characterised in terms of objects in $\mathcal{N}$ up to a \emph{$\mathscr{L}_{\widehat{\mathbb{G}}}$-simplicial approximation}, which entails to study the localisation functor $L$ in relation with crossed products and the \emph{equivariant} Künneth class (cf. Remark \ref{rem.AnalogoyThm01} for an extended discussion). One possibility to do so might be to adapt the \emph{Going-Down technique} from \cite{ChabertEchterhoffOyono} based on Theorem \ref{theo.QuantumGroupCalgebraKunneth}.
	
	Finally, we make the observation that \cite[Theorem 5.2]{YukiBCTorsion} and \cite[Corollary 5.5]{YukiBCTorsion} can be also obtained for the Künneth class instead of the bootstrap class (cf. Theorem \ref{theo.QuantumGroupCalgebraKunneth}). In particular, one obtains that $C(\mathbb{G})\in\mathcal{N}$ as soon as $\widehat{\mathbb{G}}$ satisfies the \emph{strong} BC property. In the classical setting, one can argue as follows. If $G$ satisfies the BC property with coefficients (\emph{a fortriori} when $G$ satisfies the \emph{strong} BC property), then we can apply \cite[Proposition 4.9]{ChabertEchterhoffOyono} (cf. Proposition \ref{pro.BCKunneth} for the quantum counterpart). In particular, since we always have $\mathbb{C}\in\mathcal{N}_G$ as explained above, then \cite[Proposition 4.9]{ChabertEchterhoffOyono} implies that $C^*(G)=\mathbb{C}\underset{r}{\rtimes} G\in\mathcal{N}$. This observation allows to put the Künneth formula to work by computing K-theory groups of the C$^*$-algebras defining quantum direct products in relevant examples based, for instance, on works by Voigt and Vergnioux-Voigt (cf. \cite{VoigtBaumConnesFree}, \cite{VoigtBaumConnesUnitaryFree}, \cite{VoigtBaumConnesAutomorphisms}). See Section \ref{sec.KTheoryComp}.

%\bigskip
%\textsc{Acknowledgements.}

\section{\textsc{Preliminaries}}
	\subsection{Notations and conventions}\label{sec.NotationsConventions}
	Let us fix the notations and the conventions that we use throughout the whole article.
	
	Whenever $\mathscr{C}$ denotes a category, we shall assume that $\mathscr{C}$ is essentially small, so morphisms $Hom_{\mathscr{C}}(\cdot, \cdot)$ form sets. Given a category $\mathscr{C}$, we denote by $\mathscr{C}^{op}$ its opposite category. 
	We say that $\mathscr{C}$ is \emph{countable additive} if it is additive and it admits \emph{countable direct sums}. If $F$ is an \emph{additive} functor on an additive category, it is, by definition, compatible with \emph{finite} direct sums. The categories considered in the present paper are assumed to be countable additive. Whenever we require an additive functor to be compatible with \emph{infinite (countable)} direct sums, it will be explicitly indicated. We denote by $\mathscr{A}b$ the abelian category of abelian groups and by $\mathscr{A}b^{\mathbb{Z}/2}$ the abelian category of $\mathbb{Z}/2$-graded abelian groups. We denote by $\text{C}^*\text{-Alg}$ the category of separable C$^*$-algebras with $*$-homomorphisms as morphisms. We write $\mathscr{D}\subset \mathscr{C}$ whenever $\mathscr{D}$ is a (full) subcategory of $\mathscr{C}$. We denote by $\mathbb{C}\text{-Vec}$ the category of $\mathbb{C}$-vector spaces.
	
	If $E$ is a $\mathbb{C}$-vector space and $\mathcal{S}$ is a subset of vectors of $E$, then we write $span\, \mathcal{S}$ for the corresponding $\mathbb{C}$-vector subspace generated by $\mathcal{S}$. If $(E,||\cdot||)$ is a normed $\mathbb{C}$-vector space and $F\subset E$ is a vector subspace, we write $[F]:=\overline{F}^{||\cdot||}$ for the $||\cdot||$-closure of $F$ in $E$. We then also write $\overline{span}\,\mathcal{S} = [span\, \mathcal{S}]$ for $\mathcal{S}\subset E$.
	
	Let $H$ be a Hilbert space. We denote by $\mathcal{B}(H)$ (resp.\ $\mathcal{K}(H)$) the space of all linear bounded (resp.\ compact) operators on $H$. %We denote by $\mathcal{B}(H)_*$ the space of normal functionals on $\mathcal{B}(H)$, and for $\xi, \eta\in H$ we denote by $\omega_{\xi, \eta}\in\mathcal{B}(H)_*$ the linear form defined by $\omega_{\xi, \eta}(T):=\langle \xi, T(\eta)\rangle$, for all $T\in \mathcal{B}(H)$. 
	If $A$ is a C$^*$-algebra and $H$ a  Hilbert $A$-module, we denote by $\mathcal{L}_{A}(H)$ (resp.\ $\mathcal{K}_A(H)$) the space of all (resp.\ compact) adjointable operators on $H$. Hilbert $A$-modules are considered to be \emph{right $A$-modules}, so that the corresponding inner products are considered to be conjugate-linear on the left and linear on the right.
	
	All our C$^*$-algebras (except for obvious exceptions such as multiplier C$^*$-algebras and von Neumann algebras) are supposed to be \emph{separable} and all our Hilbert modules are supposed to be \emph{countably generated}. If $A$ is a C$^*$-algebra and $\mathcal{S}$ is a subset of elements in $A$, we write $C^*\langle \mathcal{S}\rangle := C^*\langle \mathcal{S}\cup \mathcal{S}^*\rangle $ for the corresponding C$^*$-subalgebra of $A$ generated by $\mathcal{S}$, that is, the intersection of all C$^*$-subalgebras of $A$ containing $\mathcal{S}$. The symbol $\otimes$ stands for the minimal tensor product of C$^*$-algebras and the exterior tensor product of Hilbert modules depending on the context. The symbol $\underset{\max}{\otimes}$ stands for the maximal tensor product of C$^*$-algebras. In any of the previous cases, the \emph{elementary tensors} in the corresponding tensor product are denoted simply by $\otimes$ and the context will distinguish the specific situation. If $A$ and $B$ are two C$^*$-algebras, $\Sigma:A\otimes B\longrightarrow B\otimes A$ denotes the flip map. The symbol $\Sigma$ is used as well for the suspension functor in the framework of triangulated categories. The context will distinguish the specific situation. We use systematically the leg numbering, so if $H$ is a Hilbert space then $X_{12} = X\otimes 1 \in \mathcal{B}(H^{\otimes 3})$ for $X\in \mathcal{B}(H^{\otimes 2})$, etc.
	
	If $S, A$ are C$^*$-algebras, we denote by $M(A)=\mathcal{L}_A(A)$ the multiplier algebra of $A$ and we put $\widetilde{M}(A\otimes S):=\{x\in M(A\otimes S)\ |\ x(id_A\otimes S)\subset A\otimes S\mbox{ and } (id_A\otimes S)x\subset A\otimes S\}$, which contains $M(A)\otimes S$. If $H$ is a Hilbert $A$-module, we put $M(H):=\mathcal{L}_A(A, H)$, which contains canonically $H\cong \mathcal{K}_A(A, H)$. We put $\widetilde{M}(H\otimes S):=\{X\in M(H\otimes S)\ |\ X(id_A\otimes S)\subset H\otimes S\mbox{ and } (id_H\otimes S)X\subset H\otimes S\}$, which contains $H\otimes M(S)$.
	
	If $\mathbb{G}=(C(\mathbb{G}), \Delta)$ is a compact quantum group, the set of all unitary equivalence classes of irreducible unitary finite dimensional representations of $\mathbb{G}$ is denoted by $\text{Irr}(\mathbb{G})$. The trivial representation of $\mathbb{G}$ is denoted by $\epsilon$. If $x\in \text{Irr}(\mathbb{G})$ is such a class, we write $u^x\in\mathcal{B}(H_x)\otimes C(\mathbb{G})$ for a representative of $x$ and $H_x$ for the finite dimensional Hilbert space on which $u^x$ acts (we write $dim(x):= n_x$ for the dimension of $H_x$). The matrix coefficients of $u^x$ with respect to an orthonormal basis $\{\xi^x_1,\ldots, \xi^x_{n_x}\}$ of $H_x$ are defined by $u^x_{ij}:=(\omega_{ij}\otimes id)(u^x)$ for all $i,j=1,\ldots, n_x$. The linear span of matrix coefficients of all finite dimensional representations of $\mathbb{G}$ is denoted by $Pol(\mathbb{G})$, which is a Hopf $*$-algebra with co-multipliction $\Delta$ and co-unit and antipode denoted by $\varepsilon_\mathbb{G}$ and $S_\mathbb{G}$, respectively. Given $x,y\in \text{Irr}(\mathbb{G})$, the tensor product of $x$ and $y$ is denoted by $x\otimes y$.
	
	The Haar state of $\mathbb{G}$ is denoted by $h_{\mathbb{G}}$. The GNS construction corresponding to $h_{\mathbb{G}}$ is denoted by $(L^2(\mathbb{G}), \lambda, \xi_{\mathbb{G}})$. We also write $\Lambda(x) = \lambda(x)\xi_{\mathbb{G}}$ for $x\in C(\mathbb{G})$. We adopt the standard convention for the inner product on $L^2(\mathbb{G})$, which means that $\langle \Lambda(x), \Lambda(y)\rangle:=h_{\mathbb{G}}(x^*y)$ for all $x,y\in C(\mathbb{G})$. We suppress the notation $\lambda$ in computations so that we simply write $x\Lambda(y)=\Lambda(xy)$ for all $x,y\in C(\mathbb{G})$. We will make the standing assumption that $h_{\mathbb{G}}$ is faithful, so we only work with the reduced form $C(\mathbb{G})$ of a compact quantum group. 
	
	The Haar state extends uniquely to a normal faithful state on $L^{\infty}(\mathbb{G})$, and we denote by $J_{\mathbb{G}}$ the associated modular conjugation on $L^2(\mathbb{G})$. Let $I_0$ be the anti-linear involutive map $\Lambda(Pol(\mathbb{G})) \rightarrow L^2(\mathbb{G})$ defined by $\Lambda(x) \mapsto \Lambda(S(x)^*)$ for $x\in Pol(\mathbb{G})$. Then $I_0$ is closeable, and we denote $I = \widehat{J}_{\mathbb{G}}|I|$ for the polar decomposition of its closure. The map $R(x)=\widehat{J}_{\mathbb{G}}x^* \widehat{J}_{\mathbb{G}}$, for all $x\in C(\mathbb{G})$, is a well-defined anti-multiplicative and anti-co-multiplicative map on $C(\mathbb{G})$ preserving $Pol(\mathbb{G})$, called \emph{unitary antipode}. We put $U_{\mathbb{G}} = J_{\mathbb{G}}\widehat{J}_{\mathbb{G}} = \widehat{J}_{\mathbb{G}}J_{\mathbb{G}}\in \mathcal{B}(L^2(\mathbb{G}))$ for the symmetry of the standard Kac system associated to $\mathbb{G}$. The we put $\rho(a) := U_{\mathbb{G}}\lambda(a)U_{\mathbb{G}}$, for all $a\in C(\mathbb{G})$.
	
	Given a compact quantum group $\mathbb{G}$, we put $c_0(\widehat{\mathbb{G}}):= [\{(id\otimes \eta)(V_{\mathbb{G}})\ |\ \eta\in \mathcal{B}(L^2(\mathbb{G}))_*\}]\subset \mathcal{B}(L^2(\mathbb{G}))$, where $V_\mathbb{G}$ denotes the right regular representation of $\mathbb{G}$ on $L^2(\mathbb{G})$. Recall that $c_0(\widehat{\mathbb{G}})$ is a C$^*$-algebra which defines a locally compact quantum group with co-multiplication $\widehat{\Delta}(x) :=\Sigma V_{\mathbb{G}}^*(1\otimes x)V_{\mathbb{G}}\Sigma $, for all $x\in c_0(\widehat{\mathbb{G}})$. There exists a natural isomorphism $c_0(\widehat{\mathbb{G}})\cong \underset{x\in Irr(\mathbb{G})}{\bigoplus^{c_0}} \mathcal{B}(H_x)$. We denote the identity map by $\widehat{\lambda}: c_0(\widehat{\mathbb{G}}) \rightarrow \mathcal{B}(L^2(\mathbb{G}))$.
	
	If $\mathbb{H}$ is another compact quantum group, we say that $\widehat{\mathbb{H}}$ is a discrete quantum subgroup of $\widehat{\mathbb{G}}$ if one (hence all) of the following conditions hold: $i)$ $Pol(\mathbb{H})$ is a Hopf $*$-subalgebra of $Pol(\mathbb{G})$, $ii)$ $C_r(\mathbb{H})\overset{\iota}{\subset} C_r(\mathbb{G})$ such that $\iota$ intertwines the co-multiplications, $iii)$ $C_m(\mathbb{H})\overset{\iota}{\subset} C_m(\mathbb{G})$ such that $\iota$ intertwines the co-multiplications; $iv)$ $\mathscr{R}\text{ep}(\mathbb{H})$ is a full subcategory of $\mathscr{R}\text{ep}(\mathbb{G})$ containing the trivial representation and stable by direct sums, tensor product and adjoint operations. See \cite{SoltanSubgroups} for more details. In this case we write $\widehat{\mathbb{H}}<\widehat{\mathbb{G}}$. Note that in this case we have $\epsilon:=\epsilon_\mathbb{G}=\epsilon_{\mathbb{H}}$. The trivial quantum subgroup of $\widehat{\mathbb{G}}$ is denoted by $\mathbb{E}$.
	
	Finally, let us recall the \emph{two-sided crossed product} construction. It appears already in \cite[Section 2.6]{NikshychVainerman} in the context of quantum groupoids. It is used to formulate a quantum Baum-Connes assembly map in \cite{YukiBCTorsion} and \cite{KennyNestRubenBCProjective} and it will be useful for the purpose of the present paper. Let $\mathbb{G}$ be a compact quantum group. If $(B, \beta)$ is a right $\mathbb{G}$-C$^*$-algebra and $(A, \alpha)$ is a left $\mathbb{G}$-C$^*$-algebra, then the two-sided crossed product of $B$ and $A$ by $\mathbb{G}$, denoted by $B\underset{r, \beta}{\rtimes}\mathbb{G}\underset{r, \alpha}{\ltimes}A$, is the C$^*$-algebra defined by:
		$$B\underset{r, \beta}{\rtimes}\mathbb{G}\underset{r, \alpha}{\ltimes}A:=C^*\langle ((id\otimes \lambda)\beta(B)\otimes 1)(1\otimes \widehat{\lambda}(c_0(\widehat{\mathbb{G}}))\otimes 1)(1\otimes (\rho\otimes id)(\alpha(A)))\rangle\subset \mathcal{L}_{B\otimes A}(B\otimes L^2(\mathbb{G})\otimes A).$$
	
	To lighten the notations we will omit the representations $\lambda$, $\widehat{\lambda}$ and $\rho$ in the definition of $B\underset{r, \beta}{\rtimes}\mathbb{G}\underset{r, \alpha}{\ltimes}A$, and note that then $\rho(x) = U_{\mathbb{G}}xU_{\mathbb{G}}$ for $x\in C(\mathbb{G})$. We also write $\alpha_U(x) = (U_{\mathbb{G}}\otimes id)\alpha(x)(U_{\mathbb{G}}\otimes id)$ for $x\in A$. It is easy to show that $B\underset{r, \beta}{\rtimes}\mathbb{G}\underset{r, \alpha}{\ltimes}A=\overline{span}\{(\beta(B)\otimes 1)(1\otimes c_0(\widehat{\mathbb{G}})\otimes 1)(1\otimes \alpha_U(A))\}$. We use these two descriptions of $B\underset{r, \beta}{\rtimes}\mathbb{G}\underset{r, \alpha}{\ltimes}A$ interchangeably.
	
	\subsection{C$^*$-categories and tensor products}\label{sec.ModCTensorCat}
	We refer to \cite{Kassel}, \cite{Sergey} or \cite{Gelaki} for details about C$^*$-tensor categories and module C$^*$-categories. The data defining a C$^*$-tensor category is denoted by $(\mathscr{C}, *, \otimes, \mathbbold{1}, \alpha, l, r)$. Then, the data defining a $\mathscr{C}$-module C$^*$-category is denoted by $(\mathscr{M}, \bullet, \mu, e)$, where $\bullet$ denotes the (left or right) action of $\mathscr{C}$ on $\mathscr{M}$. A linear functor between two C$^*$-categories that preserves the $*$-operation is called \emph{C$^*$-functor}. A C$^*$-functor between two module C$^*$-categories that preserves the module action is called \emph{module functor}.
	
	\begin{note}\label{note.AssumptionsCtensorcat}
		If $(\mathscr{C}, *, \otimes, \mathbbold{1}, \alpha, l, r)$ is a C$^*$-tensor category, we assume the following properties.
	\begin{enumerate}[i)]
		%\item $\mathscr{C}$ is small, that is, the class of objects is a set (\textcolor{ToDo}{include this in the preliminaries section})
		\item The unit object $\mathbbold{1}$ is simple (or irreducible), that is, $End_{\mathscr{C}}(\mathbbold{1})=\mathbb{C}$.
		\item	It is well-known that every C$^*$-tensor category is unitarily monoidally equivalent to a strict C$^*$-tensor category (result due to Mac Lane and we refer to \cite[Theorem XI.5.3]{Kassel} for a proof). Hence, from now on we assume that $\mathscr{C}$ is strict meaning that the natural equivalences $\alpha$, $l$ and $r$ are identities.
		%\item $\mathscr{C}$ is rigid, that is, every object admits a conjugate object.
		\item\label{assump.DirectSum} $\mathscr{C}$ has orthogonal finite direct sums. More precisely, given objects $U_1,\ldots, U_n\in Obj(\mathscr{C})$, there exists an object $S\in Obj(\mathscr{C})$ and isometries $u_i\in Hom_{\mathscr{C}}(U_i, S)$ for each $i=1,\ldots, n$ such that $\overset{n}{\underset{i=1}{\sum}}u_iu^*_i=id_S$ and $u_iu^*_j=\delta_{ij}$, for all $i,j=1,\ldots, n$.
		\item\label{assump.Subobjects} $\mathscr{C}$ has subobjects or retracts. More precisely, for any object $U\in Obj(\mathscr{C})$ and for any projection $p\in End_{\mathscr{C}}(U)$, there exists an object $V\in Obj(\mathscr{C})$ and an isometry $u\in Hom_{\mathscr{C}}(V, U)$ such that $p=uu^*$. In particular, $\mathscr{C}$ has a zero object. Namely, the object defined by the zero projection.
		
		\emph{Observe that different terminologies are used in the category theory literature for this property. For instance, we also say that $\mathscr{C}$ is subobject/idempotent/Cauchy/Karoubi complete}.
		%\item $\mathscr{M}$ is strict meaning that the natural equivalences $\mu$ and $e$ are identities.
		%\item $\mathscr{M}$ is semi-simple.
	\end{enumerate}
	\end{note}
	\begin{rem}\label{rem.SemiSimpleAssump}				
		Observe that assumptions (\ref{assump.DirectSum}) and (\ref{assump.Subobjects}) in Note \ref{note.AssumptionsCtensorcat} are equivalent to say that all homomorphism spaces are finite-dimensional and that every object in $\mathscr{C}$ is isomorphic to a finite direct sum of simple objects. In other words, we assume from now on that $\mathscr{C}$ is semi-simple. A module C$^*$-category is called \emph{semi-simple} if the underlying C$^*$-category is semi-simple. The (module) C$^*$-categories used for our purpose are semi-simple. So we also make the assumption that the module C$^*$-categories are semi-simple in this paper.
	\end{rem}
	
	The following are relevant examples of C$^*$-tensor categories.
	\begin{exs}\label{exs.CTensorCat}
		\begin{enumerate}
			\item The cateogory of all Hilbert spaces and bounded linear maps is a C$^*$-tensor category for the ordinary tensor product of Hilbert spaces, $\mathbb{C}$ being the unit object. It is denoted by $\mathscr{H}ilb$. The corresponding rigid subcategory consists of all finite dimensional Hilberts spaces, denoted by $\mathscr{H}ilb_f$.
			
			%Given a finite dimensional Hilbert space, $H\in Obj(\mathscr{H}ilb_f)$, fix an orthonormal basis, say $\{\xi_1,\ldots, \xi_n\}$ where $n:=dim(H)$. The conjugate object of $H$ is simply its complex conjugate space, $\overline{H}$. One possible pair of solutions to the corresponding conjugate equations is given by
			%\begin{center}
 			%	\begin{tabular}{ccc}
 			%	$
 			%		\begin{array}{rccl}
 			%		R_H :&\mathbb{C}& \longrightarrow & \overline{H}\otimes H\\
 			%		&1 & \longmapsto & R(1):=\overset{n}{\underset{i=1}{\sum}}\overline{\xi}_i\otimes\xi_i
 			%		\end{array}
 			%	$
 			%	&
 			%	$
 			%		\begin{array}{rccl}
 			%		\overline{R}_H :&\mathbb{C}& \longrightarrow & H\otimes \overline{H}\\
 			%		&1& \longmapsto & \overline{R}(1):=\overset{n}{\underset{i=1}{\sum}}\xi_i\otimes \overline{\xi}_i
 			%		\end{array}
 			%	$
 			%	\end{tabular}
 			%\end{center}
			\item If $\mathbb{G}$ is a compact quantum group, the category of all its finite dimensional unitary representations and intertwiners is a C$^*$-tensor category for the usual tensor product of representations of $\mathbb{G}$, $\epsilon$ being the unit object. It is denoted by $\mathscr{R}\text{ep}(\mathbb{G})$ and it is rigid (cf. \cite{Sergey}). The category $\mathscr{R}\text{ep}(\mathbb{G})$ satisfies all assumptions considered in Note \ref{note.AssumptionsCtensorcat}; in particular, it is semi-simple.
			\begin{note}\label{note.DiscQSubTannakaKrein}
				Given a subset $\mathcal{S}\subset \text{Irr}(\mathbb{G})$, we denote by $\mathscr{C}:=\langle\mathcal{S}\rangle$ the smallest full subcategory of $\mathscr{R}\text{ep}(\mathbb{G})$ containing $\mathcal{S}$. If, in addition, $\mathscr{C}$ contains the trivial representation and it is closed under taking tensor product and contragredient representations, by Tannaka-Krein-Woronowicz duality, there is an associated C$^*$-subalgebra $C(\mathbb{H})$ such that restricting the coproduct to $C(\mathbb{H})$ endows it with the structure of compact quantum group $\mathbb{H}$. Moreover, $\mathscr{R}\text{ep}(\mathbb{H})$ naturally identifies with $\mathscr{C}$ and we say that \emph{$\widehat{\mathbb{H}}$ is the quantum subgroup of $\widehat{\mathbb{G}}$ generated by $\mathcal{S}$}.
			\end{note}
		\end{enumerate}
	\end{exs}
	
	The notions of algebra and module objects in a tensor category are important for our purpose. We refer to \cite{OstrikModCat} for the details. In the realm of C$^*$-tensor categories these notions become those of C$^*$-algebra and C$^*$-module objects, respectively. A C$^*$-algebra object in a C$^*$-tensor category as defined below is also called a Q-system in the literature (cf. \cite{LongoQSystemsSubfactors} for the details).
	
	\begin{defi}\label{defi.AlgObj}
		Let $\mathscr{C}$ be a rigid C$^*$-tensor category. A C$^*$-algebra object in $\mathscr{C}$ is an algebra object $(\mathcal{A}, m, e)$ in $\mathscr{C}$ such that the multiplication morphism $m$ is a co-isometry, i.e. $mm^*=\mathbb{C}\cdot id_{\mathcal{A}}$. We say that the C$^*$-algebra object $\mathcal{A}$ is ergodic if $\dim (\text{Hom}_{\mathscr{C}}(\mathbbold{1}, \mathcal{A}))=1$.
	\end{defi}
	\begin{exs}\label{exs.AlgObj}
		\begin{enumerate}
			\item If $\mathscr{C}$ is a rigid C$^*$-tensor category, then $\mathbbold{1}$ is a C$^*$-algebra object in $\mathscr{C}$.
			\item Let $\mathscr{C}$ be a rigid C$^*$-tensor category. If $U\in\text{Obj}(\mathscr{C})$ is an object, we denote by $\overline{U}$ its conjugate. Let $R: \mathbbold{1}\rightarrow \overline{U}\otimes U$ and $\overline{R}:\mathbbold{1}\rightarrow U\otimes\overline{U}$ be solutions to the conjugate equations (cf. \cite{Sergey}). Then $\mathcal{A}_U:=U\otimes\overline{U}$ is a C$^*$-algebra object in $\mathscr{C}$ with multiplication and unit morphisms given by $m_U:=\frac{1}{||R||}(id_U\otimes R^*\otimes id_{\overline{U}})$ and $e_U:=||R||\overline{R}$, respectively. Moreover, $U\otimes\overline{U}$ is ergodic if and only if $U$ is irreducible.
		\end{enumerate}
	\end{exs}
	
	\begin{defi}
		Let $\mathscr{C}$ be a rigid C$^*$-tensor category and $(\mathcal{A}, m, e)$ a C$^*$-algebra object in $\mathscr{C}$. A (right) $\mathcal{A}$-C$^*$-module object in $\mathscr{C}$ is a (right) $\mathcal{A}$-module object $(\mathcal{M}, a)$ in $\mathscr{C}$ such that the action morphism $a$ is a co-isometry, i.e. $aa^*=\mathbb{C}\cdot id_{\mathcal{M}}$.
	\end{defi}
	\begin{ex}
		Let $\mathscr{C}$ be a rigid C$^*$-tensor category. Given an object $U\in\text{Obj}(\mathscr{C})$, consider the C$^*$-algebra object $\mathcal{A}_U$ in $\mathscr{C}$ defined in Examples \ref{exs.AlgObj}. Given any object $V\in\text{Obj}(\mathscr{C})$, the object $\mathcal{M}_V:=V\otimes \overline{U}$ is a $\mathcal{A}_U$-C$^*$-module object in $\mathscr{C}$ with action morphism given by $a_V:=id_V\otimes R\otimes id_{\overline{U}}$.
	\end{ex}
	
	Similarly, we can define a \emph{left} C$^*$-module object. In the present paper, a C$^*$-module object is supposed to be a \emph{right} one unless the contrary is explicitly indicated. Hence, we omit the predicate \emph{right} when referring to C$^*$-module objects.
	
	\begin{defi}
		Let $\mathscr{C}$ be a rigid C$^*$-tensor category and $(\mathcal{A}, m, e)$ a C$^*$-algebra object in $\mathscr{C}$. Given two C$^*$-module objects $(\mathcal{M}, a)$ and $(\mathcal{N}, b)$ in $\mathscr{C}$, an $\mathcal{A}$-C$^*$-module morphism between $\mathcal{M}$ and $\mathcal{N}$ is a morphism $\varphi \in\text{Hom}_{\mathscr{C}}(\mathcal{M}, \mathcal{N})$ such that $b\circ (\varphi\otimes id_{\mathcal{A}})=\varphi\circ a$. We denote by $\text{Hom}_{\mathscr{A}}(\mathcal{M}, \mathcal{N})$ the $\mathbb{C}$-vector space of $\mathcal{A}$-C$^*$-module morphisms between $\mathcal{M}$ and $\mathcal{N}$.
	\end{defi}
	
	The following are relevant examples of module C$^*$-categories.
	\begin{exs}\label{exs.ModuleCategories}
		\begin{enumerate}
			\item If $\mathscr{C}$ is a C$^*$-tensor category, then it is a $\mathscr{C}$-bimodule C$^*$category with left and right actions given simply by its own tensor product $\underset{l}{\bullet}:=\otimes=:\underset{r}{\bullet}$.
			\item If $\mathscr{C}$ is a rigid C$^*$-tensor category and $\mathscr{M}$ is a left $\mathscr{C}$-module C$^*$-category, then $\mathscr{M}^{op}$ is a right $\mathscr{C}^{op}$-module category with action given by $X\bullet U:=\overline{U}\bullet X\mbox{,}$ for all $X\in Obj(\mathscr{M}^{op})$ and all $U\in Obj(\mathscr{C})$ and analogously defined on homomorphisms.
			\item\label{ex.ModCatQG} Let $\mathscr{C}$ and $\mathscr{D}$ be C$^*$-tensor categories. If $J:\mathscr{C}\longrightarrow \mathscr{D}$ is a C$^*$-tensor functor, then $\mathscr{D}$ is a right $\mathscr{C}$-module C$^*$-category with the following action $P\bullet U:=P\otimes J(U)\mbox{,}$ for all objects $U\in\mathscr{C}$ and $P\in\mathscr{D}$ and analogously defined on homomorphisms. In particular, let $\mathbb{G}$ and $\mathbb{H}$ be compact quantum groups.
			\begin{enumerate}[a)]
				\item If $\mathbb{H}\leq \mathbb{G}$ with canonical surjection $\rho: C_m(\mathbb{G})\twoheadrightarrow C_m(\mathbb{H})$, then we have a restriction functor between C$^*$-tensor categories, $J:\mathscr{R}\text{ep}(\mathbb{G})\longrightarrow\mathscr{R}\text{ep}(\mathbb{H})$. In this way, $\mathscr{R}\text{ep}(\mathbb{H})$ is a right $\mathscr{R}\text{ep}(\mathbb{G})$-module C$^*$-category.
				\item If $\widehat{\mathbb{H}}\leq \widehat{\mathbb{G}}$, then we have a fully faithful functor between C$^*$-tensor categories, $J:\mathscr{R}\text{ep}(\mathbb{H})\longrightarrow\mathscr{R}\text{ep}(\mathbb{G})$ given by the natural inclusion of $\mathscr{R}\text{ep}(\mathbb{H})$ inside $\mathscr{R}\text{ep}(\mathbb{G})$ as a full subcategory. In this way, $\mathscr{R}\text{ep}(\mathbb{G})$ is a right $\mathscr{R}\text{ep}(\mathbb{H})$-module C$^*$-category.
			\end{enumerate}
		\item Let $\mathscr{C}$ be a rigid C$^*$-tensor category and $(\mathcal{A}, m, e)$ a C$^*$-algebra object in $\mathscr{C}$. Given two C$^*$-module objects $(\mathcal{M}, a)$ and $(\mathcal{N}, b)$ in $\mathscr{C}$, it is clear that $\text{Hom}_{\mathscr{A}}(\mathcal{M}, \mathcal{N})$ is a $\mathbb{C}$-vector subspace of $\text{Hom}_{\mathscr{C}}(\mathcal{M}, \mathcal{N})$. In this way, it is clear that the collection of all $\mathcal{A}$-C$^*$-module objects in $\mathscr{C}$ form an additive subcategory of $\mathscr{C}$. We denote this category by $\mathscr{M}\text{od}_{\mathcal{A}}(\mathscr{C})$. Moreover, it easy to see that $\mathscr{M}\text{od}_{\mathcal{A}}(\mathscr{C})$ has a natural structure of (left) $\mathscr{C}$-module category. Namely, given $U\in\text{Ob}(\mathscr{C})$ and $(\mathcal{M},a)\in\text{Obj}\big(\mathscr{M}\text{od}_{\mathcal{A}}(\mathscr{C})\big)$, then we put $U\bullet \mathcal{M}:=U\otimes \mathcal{M}$, which is a new $\mathcal{A}$-module object in $\mathscr{C}$ with action morphism given by $id_{U}\otimes a$.
		
		Note that it is also true that $\text{Hom}_{\mathscr{A}}(\mathcal{M}, \mathcal{N})$ is a \emph{closed} $\mathbb{C}$-vector subspace of $\text{Hom}_{\mathscr{C}}(\mathcal{M}, \mathcal{N})$. However, it is not clear that the adjoint of a $\mathcal{A}$-C$^*$-module morphism is again a $\mathcal{A}$-C$^*$-module morphism and it requires a proof. This can be found in \cite[Lemma 3.5]{YukiKenny} for instance. In conclusion, one obtains that $\mathscr{M}\text{od}_{\mathcal{A}}(\mathscr{C})$ is in fact a $\mathscr{C}$-module C$^*$-category.
		\end{enumerate}
	\end{exs}
	
	An important notion to study module categories is that of \emph{internal Hom}. We refer to \cite{OstrikModCat} for the details. Let $\mathscr{C}$ be a rigid C$^*$-tensor category and $\mathscr{M}$ a $\mathscr{C}$-module C$^*$-category. Roughly, given two objects $X, Y\in\text{Obj}(\mathscr{M})$, we want to upgrade the homomorphism space $\text{Hom}_{\mathscr{M}}(X, Y)$ to an object in $\mathscr{C}$. More precisely, consider the following functor:
	$$\mathscr{C}\rightarrow \mathbb{C}\text{-Vec}\mbox{, } U\mapsto \text{Hom}_{\mathscr{M}}(U\bullet X, Y).$$
	
	This functor is representable (because it is left exact), i.e. there exists an object $\underline{\text{Hom}}(X, Y)\in\text{Obj}(\mathscr{C})$ such that we have a natural isomorphism $\text{Hom}_{\mathscr{M}}(U\bullet X, Y)\cong \text{Hom}_{\mathscr{C}}(U, \underline{\text{Hom}}(X, Y))$, for all $U\in\text{Obj}(\mathscr{C})$. If there are only finitely many irreducible objects in $\mathscr{C}$, then the functor above is representable in $\mathscr{C}$. If $\mathscr{C}$ has \emph{infinitely} many irreducible objects, then the functor above is still representable but in the ind-completion of $\mathscr{C}$. For the purpose of the paper, $\mathscr{C}$ will be the representation category of a compact quantum group (cf. Examples \ref{exs.CTensorCat}) and $\mathscr{M}$ a \emph{co-finite} $\mathscr{C}$-module C$^*$-category, i.e. $\mathscr{M}$ is such that given any objects $X, Y\in\text{Obj}(\mathscr{M})$, there is only a finite number of objects $U\in\text{Obj}(\mathscr{C})$ such that $\text{Hom}_{\mathscr{M}}(U\bullet X, Y)\neq 0$ (cf. Remark \ref{rem.RephrasedCoFinConnect}). In this situation, the functor above is still representable in $\mathscr{C}$.
	
	\begin{defi}
		Let $\mathscr{C}$ be a rigid C$^*$-tensor category and $\mathscr{M}$ a co-finite $\mathscr{C}$-module C$^*$-category. Given two objects $X, Y\in\text{Obj}(\mathscr{M})$, the internal Hom for $(X, Y)$ is the object $\underline{\text{Hom}}(X, Y)\in\text{Obj}(\mathscr{C})$ representing the functor $\mathscr{C}\rightarrow \mathbb{C}\text{-Vec}\mbox{, } U\mapsto \text{Hom}_{\mathscr{M}}(U\bullet X, Y)$. If $X=Y$, then we write $\underline{\text{End}}(X):=\underline{\text{Hom}}(X, X)$.
	\end{defi}
	
	\begin{pro}\label{pro.AlgFromInternal}
		Let $\mathscr{C}$ be a rigid C$^*$-tensor category and $\mathscr{M}$ a co-finite $\mathscr{C}$-module C$^*$-category. Given an object $X\in\text{Obj}(\mathscr{M})$, the internal Hom object $\underline{\text{End}}(X)\in\text{Obj}(\mathscr{C})$ is an algebra object in $\mathscr{C}$. If $Y\in\text{Obj}(\mathscr{M})$ is another object, then the internal Hom object $\underline{\text{Hom}}(X, Y)\in\text{Obj}(\mathscr{C})$ is a $\underline{\text{End}}(X)$-module object in $\mathscr{C}$.
		
		If moreover $X$ is irreducible in $\mathscr{M}$, then $\underline{\text{End}}(X)$ is a C$^*$-algebra object in $\mathscr{C}$ and $\underline{\text{Hom}}(X, Y)$ is an $\underline{\text{End}}(X)$-C$^*$-module object in $\mathscr{C}$.
	\end{pro}
	\begin{proof}
		The first part of the statement if purely algebraic and it is well known, see for instance \cite{OstrikModCat} for a proof. The second part of the statement involving the involution on the categories needs a further proof using $2$-C$^*$-categories, which can be found in \cite[Lemma 2.18]{GrossmanSnyderQsystems} and the references therein.
	\end{proof}
	
	\bigskip
	In the purely algebraic setting, the notion of relative or balanced tensor product of module categories over tensor categories goes back to constructions by P. Deligne \cite{Deligne} and D. Tambara \cite{Tambara} (see as well \cite{MugerSubfactorsCat}, \cite{EtingofFusCatHomTh}, \cite{IgnacioTensorProdCat}, \cite{DouglasSchommerPriesSnyder} for further related developments). The corresponding construction for module C$^*$-categories has been treated by many authors, e.g. \cite[Section 4.1]{SergeyMakoto}, \cite{AmbrogioLimitsCCat} or \cite{MeyerColimitsCCorresp}; and more recently by J. Antoun and C. Voigt \cite{VoigtBalancedTensorProductCategories}, where they construct a balanced tensor product of module categories over C$^*$-tensor categories without any semisimplicity or rigidity assumption. For the purpose of the present paper we do not need such an advanced technology in view of Remark \ref{rem.SemiSimpleAssump}. Namely, the Deligne's tensor product of $\mathbb{C}$-linear abelian categories \cite{Deligne} (e.g. $\mathscr{R}\text{ep}(\mathbb{G})$ provides an example of such a category) is enough for our purpose. If $\mathscr{C}$ and $\mathscr{D}$ are two such categories, we denote by $\mathscr{C}\boxtimes\mathscr{D}$ their Deligne's tensor product.
	
	It is convenient for our purpose to recall some further properties of tensor products of C$^*$-categories. Let $\mathscr{C}$ and $\mathscr{D}$ be two countable additive C$^*$-categories. One defines their \emph{minimal} and \emph{maximal} tensor product, denoted by $\mathscr{C}\underset{min}{\boxtimes} \mathscr{D}$ and $\mathscr{C}\underset{max}{\boxtimes} \mathscr{D}$, respectively as the subobject completion of the corresponding algebraic minimal and maximal tensor products, respectively (cf. \cite[Section 3]{VoigtBalancedTensorProductCategories}). It is shown in \cite[Proposition 3.4]{VoigtBalancedTensorProductCategories} that $\mathscr{C}\underset{min}{\boxtimes} \mathscr{D}$ and $\mathscr{C}\underset{max}{\boxtimes} \mathscr{D}$ are again countably additive. These definitions make use of the notion of multiplier C$^*$-category. But, as we have already mentioned in Remark \ref{rem.SemiSimpleAssump}, assumptions of Note \ref{note.AssumptionsCtensorcat} allow us to disregard such general considerations. More precisely, if $\mathscr{C}$ and $\mathscr{D}$ are semi-simple C$^*$-categories, then there is no completions involved in the definition of their minimal or maximal tensor products, so that $\mathscr{C}\underset{min}{\boxtimes} \mathscr{D}\cong \mathscr{C}\underset{max}{\boxtimes} \mathscr{D}$. Moreover, one shows  that this tensor product coincides with the Deligne tensor product \cite[Proposition 3.10]{VoigtBalancedTensorProductCategories}. The proof of \cite[Proposition 3.10]{VoigtBalancedTensorProductCategories} shows that $\mathscr{C}\boxtimes \mathscr{D}$ is again semi-simple with a complete set of irreducible objects given by $\{U_i\boxtimes V_j\}_{i\in I, j\in J}$, where $\{U_i\}_{i\in I}$ and $\{V_j\}_{j\in J}$ are complete sets of irreducible objects in $\mathscr{C}$ and $\mathscr{D}$, respectively.
	
	Finally, it is a general fact that the Deligne's tensor product of two tensor categories is again a tensor category. Namely, if $\mathscr{C}$ and $\mathscr{D}$ are two C$^*$-tensor categories as above, then $\mathscr{C}\boxtimes \mathscr{D}$ is a C$^*$-tensor category with tensor product denoted by $\otimes_{\boxtimes}$ and defined as follows:
	\begin{equation}\label{eq.TensorProdDeligneTensor}
	\begin{split}
		(U_1\boxtimes V_1)\otimes_{\boxtimes} (U_2\boxtimes V_2):=(U_1\otimes U_2)\boxtimes (V_1\otimes V_2),
	\end{split}
	\end{equation}
	for all $U_1, U_2\in\text{Obj}(\mathscr{C})$ and $V_1, V_2\in\text{Obj}(\mathscr{D})$. We refer to \cite[Proposition 4.6.1]{Gelaki} for the details. In relation to this construction, the following observation is useful for our purpose.
	\begin{pro}\label{pro.AModTensorProd}
		Let $\mathscr{C}$, $\mathscr{D}_1$ and $\mathscr{D}_2$ be C$^*$-tensor categories. Assume that $\mathscr{D}_1\boxtimes\mathscr{D}_2\subset \mathscr{C}$. If $\mathscr{C}\cong \mathscr{D}_1\boxtimes\mathscr{D}_2$ as C$^*$-tensor categories and as $\mathscr{D}_1\boxtimes\mathscr{D}_2$-module C$^*$-categories (resp. $\mathscr{C}$-module C$^*$-categories), then for any C$^*$-algebra object $\mathcal{A}\in\text{Obj}(\mathscr{C})$ we have:
		$$\mathscr{M}\text{od}_{\mathcal{A}}(\mathscr{C})\cong \mathscr{M}\text{od}_{\mathcal{B}_1\boxtimes\mathcal{B}_2}(\mathscr{D}_1\boxtimes\mathscr{D}_2),$$
		as $\mathscr{D}_1\boxtimes\mathscr{D}_2$-module C$^*$-categories (resp. $\mathscr{C}$-module C$^*$-categories); where $\mathcal{B}_1\in\text{Obj}(\mathscr{D}_1)$ and $\mathcal{B}_2\in\text{Obj}(\mathscr{D}_2)$ are (unique up to isomorphism) C$^*$-algebra objects such that $\mathcal{A}\cong \mathcal{B}_1\boxtimes \mathcal{B}_2$ through the equivalence $\mathscr{C}\cong \mathscr{D}_1\boxtimes\mathscr{D}_2$.
	\end{pro}
	\begin{proof}
		Given the C$^*$-algebra object $\mathcal{A}$ in $\mathscr{C}$, let $(\mathcal{M}, a)$ be a $\mathcal{A}$-C$^*$-module object in $\mathscr{C}$. By means of the equivalence $\mathscr{C}\cong \mathscr{D}_1\boxtimes\mathscr{D}_2$, we write $\mathcal{M}\cong \mathcal{N}_1\boxtimes\mathcal{N}_2$ for (unique up to isomorphism) $\mathcal{B}_1$-C$^*$-module object $(\mathcal{N}_1, b_1)$ in $\mathscr{D}_1$ and $\mathcal{B}_2$-C$^*$-module object $(\mathcal{N}_2, b_2)$ in $\mathscr{D}_2$. The tensor structure on $\mathscr{D}_1\boxtimes\mathscr{D}_2$ together with the assumption that $\mathscr{C}\cong \mathscr{D}_1\boxtimes\mathscr{D}_2$ as C$^*$-tensor categories allow to write the following:
		\begin{equation*}
		\begin{split}
			\text{Hom}_{\mathscr{C}}(\mathcal{M}\otimes \mathcal{A}, \mathcal{M})&\cong \text{Hom}_{ \mathscr{D}_1\boxtimes\mathscr{D}_2}\big((\mathcal{N}_1\boxtimes \mathcal{N}_2)\otimes_{\boxtimes} (\mathcal{B}_1\boxtimes\mathcal{B}_2), \mathcal{N}_1\boxtimes \mathcal{N}_2\big)\\
			&=\text{Hom}_{ \mathscr{D}_1\boxtimes\mathscr{D}_2}\big((\mathcal{N}_1\otimes \mathcal{B}_1)\boxtimes (\mathcal{N}_2\otimes\mathcal{B}_2), \mathcal{N}_1\boxtimes \mathcal{N}_2\big)\\
			&=\text{Hom}_{ \mathscr{D}_1}\big(\mathcal{N}_1\otimes \mathcal{B}_1, \mathcal{N}_1\big)\otimes \text{Hom}_{ \mathscr{D}_2}\big(\mathcal{N}_2\otimes \mathcal{B}_2, \mathcal{N}_2\big).
		\end{split}
		\end{equation*}
		
		Therefore $a\cong b_1\otimes b_2$ and the equivalence $\mathscr{M}\text{od}_{\mathcal{A}}(\mathscr{C})\cong \mathscr{M}\text{od}_{\mathcal{B}_1\boxtimes\mathcal{B}_2}(\mathscr{D}_1\boxtimes\mathscr{D}_2)$ of the statement follows. Namely, since $\mathscr{D}_1\boxtimes\mathscr{D}_2\subset \mathscr{C}$ by assumption, then we can regard $\mathscr{C}$ as a $\mathscr{D}_1\boxtimes\mathscr{D}_2$-module C$^*$-category, the module structure being the tensor structure on $\mathscr{C}$. Similarly, $\mathscr{D}_1\boxtimes\mathscr{D}_2$ can be viewed as a $\mathscr{C}$-module C$^*$-category. The fact that $\mathscr{C}\cong \mathscr{D}_1\boxtimes\mathscr{D}_2$ as C$^*$-tensor categories by assumption and the above computation translate into the equivalence $\mathscr{M}\text{od}_{\mathcal{A}}(\mathscr{C})\cong \mathscr{M}\text{od}_{\mathcal{B}_1\boxtimes\mathcal{B}_2}(\mathscr{D}_1\boxtimes\mathscr{D}_2)$ as both $\mathscr{D}_1\boxtimes\mathscr{D}_2$-module C$^*$-categories and $\mathscr{C}$-module C$^*$-categories.
	\end{proof}
	
	\subsection{Actions of quantum groups}\label{sec.ActionsQG}
	In this section we recall some terminology about actions of quantum groups. We refer to \cite{KennyActions} for more details. %and the corresponding spectral theory. We refer to \cite{KennyActions}, \cite{BichonVaesRijdt} for more details.
	\begin{defi}
			Let $\mathbb{G}=(C(\mathbb{G}),\Delta)$ be a compact quantum group. A right $\mathbb{G}$-C$^*$-algebra is a C$^*$-algebra $A$ together with an injective non-degenerate $*$-homomorphism $\alpha: A\longrightarrow A\otimes C(\mathbb{G})$ such that: $i)$ $(\alpha\otimes id_{C(\mathbb{G})})\circ\alpha = (id_A\otimes \Delta)\circ\alpha$ and $ii)$ $[\alpha(A)(1\otimes C(\mathbb{G}))]=A\otimes C(\mathbb{G})$. Such homomorphism is called a \emph{right action of $\mathbb{G}$ on $A$} or a \emph{right co-action of $C(\mathbb{G})$ on $A$}.	
	\end{defi}
	
	Similarly, we can define a \emph{left} action of $\mathbb{G}$ on $A$ (or a \emph{left} co-action of $C(\mathbb{G})$ on $A$) as a non-degenerate $*$-homomorphism $\alpha: A\longrightarrow C(\mathbb{G})\otimes A$ satisfying the analogous properties of the preceding definition. In the present article, an action of a compact quantum group $\mathbb{G}$ is supposed to be a \emph{right} one unless the contrary is explicitly indicated. Hence, we refer to such actions simply as \emph{actions of $\mathbb{G}$}. Observe however that if $(A, \alpha)$ is a \emph{right} $\mathbb{G}$-C$^*$-algebra, then $(A^{op}, \overline{\alpha})$ is a \emph{left} $\mathbb{G}$-C$^*$-algebra where $A^{op}$ denotes the opposite C$^*$-algebra of $A$ and 
$\overline{\alpha}: A^{op}\longrightarrow C(\mathbb{G})\otimes A^{op}$ is defined by $\overline{\alpha}:=(R\otimes id)\circ\Sigma\circ\alpha$, where $R$ denotes the unitary antipode of $\mathbb{G}$.

	We also recall the notion of action for discrete quantum groups for the sake of completeness. 
	\begin{defi}
		Let $\mathbb{G}=(C(\mathbb{G}),\Delta)$ be a compact quantum group. A right $\widehat{\mathbb{G}}$-C$^*$-algebra is a C$^*$-algebra $A$ together with an injective non-degenerate $*$-homomorphism $\alpha: A\longrightarrow \widetilde{M}(A\otimes c_0(\widehat{\mathbb{G}}))$ such that: $i)$ $(\alpha\otimes id_{c_0(\widehat{\mathbb{G}})})\circ\alpha = (id_A\otimes \widehat{\Delta})\circ\alpha$ and $ii)$ $[\alpha(A)(1\otimes c_0(\widehat{\mathbb{G}}))]=A\otimes c_0(\widehat{\mathbb{G}})$. Such homomorphism is called a \emph{right action of $\widehat{\mathbb{G}}$ on $A$} or a \emph{right co-action of $c_0(\widehat{\mathbb{G}})$ on $A$}.
		\end{defi}
	Again, one has the analogous notion of a left action of $\widehat{\mathbb{G}}$.  In the following, an action of a discrete quantum group $\widehat{\mathbb{G}}$ is supposed to be a \emph{right} one unless the contrary is explicitly indicated.
	\bigskip

	We also recall the notion of equivariant Hilbert module with respect to a compact quantum group for the sake of completeness.
	\begin{defi}\label{defi.EquivariantModule}
		Let $\mathbb{G}$ be a compact quantum group and $(A, \delta)$ a $\mathbb{G}$-C$^*$-algebra. A right $\mathbb{G}$-equivariant Hilbert $A$-module is a right $A$-module $E$ together with an injective linear map $\delta_E: E\longrightarrow E\otimes C(\mathbb{G})$ such that: $i)$ $\delta_E(\xi\cdot a)=\delta_E(\xi)\circ\delta(a)$ for all $\xi\in E$ and $a\in A$; $ii)$ $\delta\big(\langle \xi, \eta\rangle\big)=\langle \delta_E(\xi), \delta_E(\eta) \rangle$ for all $\xi, \eta\in E$; $iii)$ $(\delta_E\otimes id)\circ \delta_E=(id\otimes \Delta)\circ \delta_E$; $iv)$ $[\delta_E(E)(A\otimes C(\mathbb{G}))]=E\otimes C(\mathbb{G})$. Such map is called a \emph{right action of $\mathbb{G}$ on $E$} or a \emph{right co-action of $C(\mathbb{G})$ on $E$}.
	\end{defi}
	\begin{rem}\label{rem.RepGEquivHilbSpace}
		The notion of representation of a compact quantum group can be viewed alternatively as follows. If $u\in\mathcal{B}(H)\otimes C(\mathbb{G})$ is a (finite dimensional) unitary representation of $\mathbb{G}$ on a (finite dimensional) Hilbert space $H$, then the map $\delta_H: H\rightarrow H\otimes C(\mathbb{H})$, $\xi\mapsto \delta_H(\xi):=u(\xi\otimes id)$, turns $H$ into a $\mathbb{G}$-equivariant Hilbert space. Conversely, if $(H, \delta_H)$ is a (finite dimensional) $\mathbb{G}$-equivariant Hilbert space, then the unitary $u\in\mathcal{B}(H)\otimes C(\mathbb{G})$ defined by $u(\xi\otimes 1)=\delta_H(\xi)$, for all $\xi\in H$ is a (finite dimensional) unitary representation of $\mathbb{G}$ on $H$.
	\end{rem}
	\begin{defi}\label{defi.EquivariantModule}
		Let $\mathbb{G}$ be a compact quantum group and $(A, \delta)$ a $\mathbb{G}$-C$^*$-algebra. Let $(E, \delta_E)$ be a $\mathbb{G}$-equivariant Hilbert $A$-module. We say that $E$ is irreducible if the space of equivariant adjointable operators of $E$, $\mathcal{L}_{\mathbb{G}}(E):=\{T\in\mathcal{L}_A(E)\ |\ \delta_E(T(\xi))=(T\otimes 1)\delta_E(\xi)\mbox{, for all $\xi\in E$}\},$ is one-dimensional.
	\end{defi}
	\begin{rem}\label{rem.AdmissibleUnitary}
		If $(E, \delta_E)$ is a $\mathbb{G}$-equivariant Hilbert $A$-module as above, then $\mathcal{K}_A(E)$ is a $\mathbb{G}$-C$^*$-algebra with action $\delta_{\mathcal{K}_A(E)}$ defined by $\delta_{\mathcal{K}_A(E)}(\theta_{\xi, \eta})=\delta_E(\xi)\delta_E(\eta)^*\in \mathcal{K}_A(E)\otimes C(\mathbb{G})$, for all $\xi, \eta\in E$ where $\theta_{\xi, \eta}$ denotes the corresponding rank one operator in $E$. By abuse of notation, we still denote by $\delta_{\mathcal{K}_A(E)}$ the extension of this homomorphism to $\mathcal{L}_A(E) = M(\mathcal{K}_A(E)) \rightarrow M(\mathcal{K}_{A}(E) \otimes C(\mathbb{G}))$. The latter is however not in general an action of $\mathbb{G}$ on $\mathcal{L}_A(E)$. Recall further that giving an action $\delta_E$ is equivalent to giving a unitary operator $V_E\in\mathcal{L}_{A\otimes C(\mathbb{G})}\big(E\underset{\delta}{\otimes}(A\otimes C(\mathbb{G})), E\otimes C(\mathbb{G})\big)$ such that $\delta_E(\xi)=V_E\circ T_\xi$ for all $\xi\in E$ where $T_\xi\in\mathcal{L}_{A\otimes C(\mathbb{G})}(A\otimes C(\mathbb{G}), E\underset{\delta}{\otimes}(A\otimes C(\mathbb{G})))$ is such that $T_\xi(x)=\xi\underset{\delta}{\otimes} x$, for all $x\in A\otimes C(\mathbb{G})$. One calls $V_E$ the \emph{admissible operator for $(E, \delta_E)$}. Moreover, we have $\delta_{\mathcal{K}_A(E)}=Ad_{V_E}$. We refer to \cite{BaajSkandalisQuantumKK} for more details. Note that $\mathcal{L}_{\mathbb{G}}(E)=\mathcal{L}_A(E)^{Ad_{V_E}}$. So, if $(E, \delta_E)$ is irreducible, then $\mathcal{L}_A(E)=\mathcal{K}_A(E)$ together with $Ad_{V_E}$ defines an ergodic action of $\mathbb{G}$.
	\end{rem}
	
	\begin{defi}
		Let $\mathbb{G}$ be a compact quantum group. Let $(A, \alpha)$ and $(B, \beta)$ be two $\mathbb{G}$-C$^*$-algebras. We say that $A$ and $B$ are $\mathbb{G}$-equivariantly Morita equivalent if there exists a $\mathbb{G}$-equivariant Hilbert $A$-module $(E, \delta_E)$ such that $B\cong \mathcal{K}_A(E)$ as $\mathbb{G}$-C$^*$-algebras. In this case we write $A\underset{\mathbb{G}}{\sim} B$.
	\end{defi}
	
	If $(A, \alpha)$ is a $\mathbb{G}$-C$^*$-algebra, we put $A^\alpha:=\{a\in A\ |\ \alpha(a)=a\otimes 1_{C(\mathbb{G})}\}$, which is a sub-C$^*$-algebra of $A$. If $A$ is unital, we say that $\alpha$ is ergodic if $A^\alpha=\mathbb{C}\cdot 1_A$. We denote by $E_\alpha: A\longrightarrow A^\alpha$ the $\alpha$-invariant conditional expectation given by $E_{\alpha}(a)=(id_A\otimes h_{\mathbb{G}})\delta(a)$, for all $a\in A$. Recall that we assume $\mathbb{G}$ to be a reduced compact quantum group, so $E_{\alpha}$ is automatically faithful. Given a $\mathbb{G}$-C$^*$-algebra $A$, we can equip $A$ with the pre-Hilbert $A^\alpha$-module structure given by $\langle a, b\rangle_{E_\alpha}:=E_\alpha(a^*b)$, for all $a,b\in A$. We denote by $L^2(A, E_\alpha)$ the completion of $A$ with respect to the inner product $\langle\cdot,\cdot\rangle_{E_\alpha}$. When $\alpha$ is ergodic, we have $E_\alpha(a)=\varphi_{\alpha}(a)1_A$ for $\varphi_{\alpha}$ a (unique) $\alpha$-invariant state on $A$. We then write $L^2(A)=L^2(A, \varphi)$ for the Hilbert space completion of $A$ with respect to the inner product $\langle a, b\rangle_{\varphi}:=\varphi(a^*b)$, for all $a,b\in A$.
	
	\begin{ex}\label{ex.AlgObjFromGAlg}
		 Let $\mathbb{G}$ be a compact quantum group. If $(A, \alpha)$ is a finite dimensional $\mathbb{G}$-C$^*$-algebra, then $\mathcal{A}_{\alpha}:=(L^2(A, \varphi_A), m, e)$ defines a C$^*$-algebra object in $\mathscr{R}\text{ep}(\mathbb{G})$ (cf. Definition \ref{defi.AlgObj}). Here, $m$ and $e$ are the multiplication and unit morphisms defining the algebra $A$. And $\varphi_A$ is a positive functional on $A$ such that $\langle a, b\rangle_{\varphi_A}:=\varphi_A(a^*b)$ turns $A$ into a unitary representation of $\mathbb{G}$ (cf. Remark \ref{rem.RepGEquivHilbSpace}) for which $m$ is a co-isometry. Then we denote by $L^2(A, \varphi_A)$ the corresponding Hilbert space constructed out of $A$ in this way. We refer to \cite[Lemma 2.2]{YukiKenny} for a proof. Moreover, $\mathcal{A}_{\alpha}$ is ergodic if and only if $\alpha$ is ergodic. Conversely, given a (resp. ergodic) C$^*$-algebra object $\mathcal{A}$ in $\mathscr{R}\text{ep}(\mathbb{G})$ there exists a (resp. ergodic) finite dimensional $\mathbb{G}$-C$^*$-algebra $(A, \alpha)$ such that $\mathcal{A}_\alpha=\mathcal{A}$ (cf. \cite[Proposition 3.4]{YukiKenny} for a proof). This establish a one-to-one correspondence between (resp. ergodic) finite dimensional $\mathbb{G}$-C$^*$-algebras and (resp. ergodic) C$^*$-algebra objects in $\mathscr{R}\text{ep}(\mathbb{G})$, up to $*$-isomorphism.
	\end{ex}


	A useful notion in the realm of compact quantum group actions is that of \emph{spectral subspace}. We recall briefly the basics about spectral theory for compact quantum groups and we refer to \cite{KennyActions} for further details.
	
	Let $\mathbb{G}$ be a compact quantum group and let $(A, \alpha)$ be a $\mathbb{G}$-C$^*$-algebra. We assume that $A$ is unital for simplicity, but the results hold in the non-unital case too. Given any unitary representation $u\in R(\mathbb{G})$, the \emph{$u$-spectral space of $\alpha$} is the following vector space $K_u:=\{X\in H_u\otimes A\ |\ (id\otimes \alpha)(X)=X_{12}u_{13}\}$. We also define the following vector subspace of $A$, $\mathcal{A}_u:=span\{X(\xi^*\otimes 1)\ |\ \xi^*\in \overline{H_u}, X\in K_u\}$, which is still called \emph{$u$-spectral subspace}.  In particular, we can do these constructions for irreducible representations of $\mathbb{G}$ by putting $K_x:=K_{u^x}$ and $\mathcal{A}_x:=\mathcal{A}_{u^x}$ whenever $x\in \text{Irr}(\mathbb{G})$. Then we define the \emph{Podleś subalgebra of $A$} to be $\mathcal{A}_{\mathbb{G}}:=span\{\mathcal{A}_x\ |\ x\in \text{Irr}(\mathbb{G})\}$. One shows that $\mathcal{A}_x$ is a closed subspace of $A$, for all $x\in \text{Irr}(\mathbb{G})$. Moreover, $\mathcal{A}_{\mathbb{G}}$ is a dense $*$-subalgebra of $A$. The action $\alpha$ restricts to an algebraic action of $\mathbb{G}$ on $\mathcal{A}_\mathbb{G}$, which is still denoted by $\alpha$. Finally, the canonical $\alpha$-invariant conditional expectation $E_\alpha:A \longrightarrow A^\alpha$ is faithful on $\mathcal{A}_\mathbb{G}$. Accordingly, $\mathcal{A}_\mathbb{G}=\underset{x\in \text{Irr}(\mathbb{G})}{\bigoplus} \mathcal{A}_x$. In addition, we gather together some observations about spectral subspaces in the following proposition, which are useful for our purpose.
	\begin{pro}\label{pro.FactsSpect}
		Let $\mathbb{G}$ be a compact quantum group and let $(A, \alpha)$ be a $\mathbb{G}$-C$^*$-algebra.
		\begin{enumerate}
			\item For every $x\in\text{Irr}(\mathbb{G})$, $\mathcal{K}_x$ is a Hilbert $A^\alpha$-module with $\langle X, Y\rangle_{K_x}:=X^*Y$, for all $X, Y\in K_x$. Accordingly, $\mathcal{A}_x$ is a Hilbert $A^\alpha$-module for every $x\in\text{Irr}(\mathbb{G})$.
			\item Assume that $\alpha$ is ergodic. For every $x\in\text{Irr}(\mathbb{G})$, $\mathcal{A}_x$ defines a representation of $\mathbb{G}$, i.e. a $\mathbb{G}$-equivariant Hilbert space (cf. Remark \ref{rem.RepGEquivHilbSpace}). %$\alpha(\mathcal{A}_x)\subset \mathcal{A}_x\otimes C(\mathbb{G})$ such that $(id_{\mathcal{A}_x}\otimes \Delta)\alpha=(\alpha\otimes id)\alpha$ and $(id_{\mathcal{A}_x}\otimes \varepsilon)\alpha=id_{\mathcal{A}_x}$. 
			In particular, if $A$ is finite dimensional, then $\mathcal{A}_x$ is equivalent to a (finite dimensional) unitary representation of $\mathbb{G}$.%\textcolor{red}{In this way, $\mathcal{A}_x$ is identified to the representation space of $u^x$. THIS IS NOT TRUE!!!}
			%\item Assume that $\alpha$ is ergodic. Then $\mathcal{A}_\mathbb{G}$ is dense in $L^2(A)$.
		\end{enumerate}
	\end{pro}

	\subsection{Torsion phenomenon for quantum groups}\label{sec.TorsionQG}
	
	\begin{defi}
		Let $\mathbb{G}$ be a compact quantum group. A torsion action of $\mathbb{G}$ is a $\mathbb{G}$-C$^*$-algebra $(T, \tau)$ where $T$ is finite dimensional and $\tau$ is ergodic. We say that $\widehat{\mathbb{G}}$ is torsion-free if any torsion action of $\mathbb{G}$ is $\mathbb{G}$-equivariantly Morita equivalent to the trivial $\mathbb{G}$-C$^*$-algebra $\mathbb{C}$. The set of all equivalence classes under equivariant Morita equivalence of non-trivial torsion actions of $\mathbb{G}$ is denoted by $\mathbb{T}\text{or}(\widehat{\mathbb{G}})$.
	\end{defi}
	
	Y. Arano and K. De Commer \cite{YukiKenny} have re-interpreted the notion of torsion-freeness for discrete quantum groups in terms of fusion rings giving a stronger version of torsion-freeness. Let us recall briefly elementary notions about fusion rings theory (we refer to \cite{YukiKenny} or \cite{Gelaki} for further details and properties).
		
	Let $(I, \mathbbm{1})$ be an involutive pointed set and $J$ any set. Denote by $\mathbb{Z}_I$ the \emph{free $\mathbb{Z}$-module with basis $I$}, that is, every element in $\mathbb{Z}_I$ is a unique finite $\mathbb{Z}$-linear combination of elements of $I$. The addition operation in $\mathbb{Z}_I$ is denoted by $\oplus$. A \emph{ring structure $\otimes$} on $\mathbb{Z}_I$ is given by constants $N_{\alpha, i'}^{i}\in \mathbb{N}\cup \{0\}$ for all $\alpha, i, i'\in I$, called \emph{fusion rules}, such that $\alpha\otimes i'=\underset{i\in I}{\sum}N_{\alpha, i'}^{i}\cdot i\mbox{,}$ where all but finitely many terms vanish. This rule extends obviously to any element of $\mathbb{Z}_I$ and it can be regarded as an action of $\mathbb{Z}_I$ on itself; we call it \emph{regular action of $\mathbb{Z}_I$}. We write $i\subset \alpha\otimes i'$ whenever $N_{\alpha, i'}^{i}\neq 0$. Denote by $\mathbb{Z}_J$ the \emph{free $\mathbb{Z}$-module with basis $J$}. The addition operation in $\mathbb{Z}_J$ is still denoted by $\oplus$. A \emph{(left) $\mathbb{Z}_I$-module structure} on $\mathbb{Z}_J$, still denoted by $\otimes$, is given by constants $N^{j}_{\alpha, j'}\in\mathbb{N}\cup \{0\}$ for all $\alpha\in I$, $j, j'\in J$ such that $\alpha\otimes j'=\underset{j\in J}{\sum}N_{\alpha, j'}^{j}\cdot j\mbox{,}$ where all but finitely many terms vanish. This rule extends obviously to any element of $\mathbb{Z}_I$ and $\mathbb{Z}_J$ and it can be regarded as an action of $\mathbb{Z}_I$ on $\mathbb{Z}_j$; we say that $\mathbb{Z}_J$ is a $\mathbb{Z}_I$-module. We write $j\subset \alpha\otimes j'$ whenever $N_{\alpha, j'}^{j}\neq 0$.
		
	\begin{defi}\label{defi.FusionModules}
		Let $(I, \mathbbm{1})$ be an involutive pointed set and $J$ any set. Let $R:=(\mathbb{Z}_I, \oplus, \otimes)$ be the free $\mathbb{Z}$-module with basis $I$ endowed with a ring structure and let $M:=(\mathbb{Z}_J, \oplus, \otimes)$ be the free $\mathbb{Z}$-module with basis $J$ endowed with a $\mathbb{Z}_I$-module structure.
		\begin{itemize}
			\item[-] We say that $R$ is a $I$-based ring if $\overline{\alpha\otimes\alpha'}=\overline{\alpha'}\otimes\overline{\alpha}$, for all $\alpha, \alpha'\in I$ and $\mathbbold{1}\subset \overline{\alpha}\otimes \alpha'$ if and only if $\alpha=\alpha'$, for all $\alpha, \alpha'\in I$.

			\item[-] We say that $R$ is a fusion ring if it is a $I$-based ring equipped with a dimension function, that is, a unital ring homomorphism $d:\mathbb{Z}_I\longrightarrow\mathbb{R}$ such that $d(\alpha)>0$, for all $\alpha\in I$ and $d(\overline{\alpha})=d(\alpha)$, for all $\alpha\in I$.

			\item[-] We say that $M$ is a $J$-based module if $j\subset \alpha\otimes j' \Leftrightarrow j'\subset \overline{\alpha}\otimes j$, for all $\alpha\in I$, $j, j'\in J$.
			\item[-] A $J$-based module $M$ is said to be co-finite if for all $j, j'\in J$, the set $\{\alpha\in I\ |\ j\subset\alpha\otimes j' \}$ is finite.
			\item[-] A $J$-based module $M$ is said to be connected if for all $j, j'\in J$, there exists $\alpha\in I$ such that $j\subset \alpha\otimes j'$.
			\item[-] A $J$-based module is said to be a torsion module if it is co-finite and connected.
			\item[-] We say that $M$ is a fusion $R$-module if it is $J$-based and it is equipped with a compatible dimension function, that is, a linear map $d_J:\mathbb{Z}_J\longrightarrow\mathbb{R}$ such that $d_J(j)>0$, for all $j\in J$ and $d_J(\alpha\otimes j')=d_I(\alpha)d_J(j')$, for all $\alpha\in I$ and all $j'\in J$.

		\end{itemize}
	\end{defi}
	\begin{note}
		An isomorphism of based modules is assumed to take basis elements to basis elements.
	\end{note}

	\begin{exs}\label{exs.FusionRings}
		\begin{enumerate}
			\item The trivial fusion ring is the fusion ring $\mathbb{Z}_I$ with $I=\{\mathbbold{1}\}$.
			\item Any fusion ring $R$ is a fusion $R$-module with its regular action. It is automatically co-finite and connected by definition. The corresponding $R$-valued bilinear form on $R$ is given simply by $\langle \alpha, \alpha'\rangle =\alpha\otimes \overline{\alpha'}\mbox{,}$ for all $\alpha, \alpha'\in I$. In this way, we say that $R$ is equipped with the \emph{standard fusion module structure}.
				
				A fusion $R$-module is said to be \emph{standard} if it is isomorphic to $R$ with its standard fusion module structure.
			\item Let $R:=(\mathbb{Z}_I, \oplus, \otimes, d)$ be a fusion ring. If $L\subset I$ is subset such that $(L, \mathbbold{1})$ is an involutive pointed set such that $N^{i}_{\alpha, i'}=0$, for all $\alpha, i'\in L$ and all $i\in I\backslash L$, then we obtain by restriction of $\otimes$ and $d$ a fusion ring $S:=(\mathbb{Z}_L, \oplus, \otimes_{|}, d_{|})$. It is called \emph{fusion subring of $R$} and we write $S\subset R$.
				
				For instance, given any basis element $\alpha\in I$ we can consider the \emph{fusion ring generated by $\alpha$}, which is the smallest fusion subring of $R$ containing $\alpha$.
			\item Let $R_1:=(\mathbb{Z}_{I_1}, \oplus, \otimes, d_1)$ and $R_2:=(\mathbb{Z}_{I_2}, \oplus, \otimes, d_2)$ be two fusion rings. We define the tensor product of $R_1$ and $R_2$, denoted by $R_1\boxtimes R_2$, as the free $\mathbb{Z}$-module $\mathbb{Z}_{I_1}\underset{\mathbb{Z}}{\odot} \mathbb{Z}_{I_2}$ equipped with basis $I_{1}\underset{\mathbb{Z}}{\odot}I _{2}$, unit $\mathbbold{1}_1\odot \mathbbold{1}_2$, involution $\overline{x\odot y}=\overline{x}\odot \overline{y}$, for all $x\in\mathbb{Z}_{I_1}$ and all $y\in\mathbb{Z}_{I_2}$ and dimension function $d(i_1\odot i_2)=d_1(i_1)d_2(i_2)$, for all $i_1\in I_1$ and all $i_2\in I_2$. Elementary tensors will be denoted by $r_1\boxtimes r_2$, for every $r_1\in R_1$ and $r_2\in R_2$.
			\item Let $\widehat{\mathbb{G}}$ be a discrete quantum group. Define $(I,\mathbbold{1}):=(\text{Irr}(\mathbb{G}), \epsilon)$ as the pointed set with involution given by the adjoint representation. We define the \emph{fusion ring of $\widehat{\mathbb{G}}$} as the $\text{Irr}(\mathbb{G})$-based ring $\mathbb{Z}_{\text{Irr}(\mathbb{G})}$ with fusion rules and dimension function given by $N_{x, y}^{z}:=dim\Big(Mor(z, x\otimes y)\Big)\mbox{ and } d(x):=dim(H_x)\mbox{ ,}$ for all $x,y,z\in \text{Irr}(\mathbb{G})$. In other words, the ring structure is given simply by the tensor product of representations and so by the corresponding fusion rules. This ring is denoted by $\text{R}(\mathbb{G})$ and we refer to it as the \emph{representation ring of $\mathbb{G}$}. If $\widehat{\mathbb{H}}<\widehat{\mathbb{G}}$, then $\mathscr{R}\text{ep}(\mathbb{H})$ is a full subcategory of $\mathscr{R}\text{ep}(\mathbb{G})$, so that $\text{R}(\mathbb{H})$ is a fusion subring of $\text{R}(\mathbb{G})$.
		\end{enumerate}
	\end{exs}
		
	\begin{defi}
		A fusion ring $R$ is called torsion-free if any non-trivial torsion $R$-module is standard. In particular, given a compact quantum group $\mathbb{G}$, we say that $\widehat{\mathbb{G}}$ is strong torsion-free if $\text{R}(\mathbb{G})$ is torsion-free.
	\end{defi}
		
	The approach of Y. Arano and K. De Commer meets the notion of torsion-freeness in the sense of R. Meyer and R. Nest when we work in the context of module C$^*$-categories. Let us recall the main definitions and results that make possible this connection. We refer to \cite{Longo}, \cite{YukiKenny} and \cite{SergeyMakoto} for more precisions and details. Let $\mathscr{C}$ be a rigid C$^*$-tensor category and $\mathscr{M}$ a $\mathscr{C}$-module C$^*$-category. We associate to $\mathscr{C}$ a fusion ring, denoted by $\text{Fus}(\mathscr{C})$, with basis given by irreducible objects of $\mathscr{C}$, fusion rules analogous to the fusion rules of a discrete quantum group and dimension function defined in \cite{Longo} (called \emph{intrinsic dimension}). We associate to $\mathscr{M}$ a based $\text{Fus}(\mathscr{C})$-module, denoted by $\text{Fus}(\mathscr{M})$, with basis given by irreducible objects in $\mathscr{M}$.

		\begin{defi}
			Let $\mathscr{C}$ be a rigid C$^*$-tensor category and $\mathscr{M}$ a $\mathscr{C}$-module C$^*$-category. We say that $\mathscr{M}$ is \emph{co-finite} (resp. connected, resp. torsion) if $\text{Fus}(\mathscr{M})$ is co-finite (resp. connected, resp. torsion) as $\text{Irr}(\mathscr{M})$-based $\text{Fus}(\mathscr{C})$-module.
		\end{defi}
	\begin{rem}\label{rem.RephrasedCoFinConnect}
		More precisely, the previous definitions give the following. Recall that we always assume that our C$^*$-categories are semi-simple. $\mathscr{M}$ is co-finite if and only if for all non-zero objects $X, Y\in Obj(\mathscr{M})$, the set $\{U\in Obj(\mathscr{C})\ |\ Hom_{\mathscr{M}}(U\bullet Y, X)\neq (0)\}$ is finite. $\mathscr{M}$ is connected if and only if for all non-zero objects $X, Y\in Obj(\mathscr{M})$, there exists an object $U\in Obj(\mathscr{C})$ such that $\text{Hom}_{\mathscr{M}}(U\bullet Y, X)\neq (0)$.
	\end{rem}
	
	The following observation is useful for our purpose.
	\begin{pro}\label{pro.ConnectedIndecomposable}
		Let $\mathscr{C}$ be a rigid C$^*$-tensor category. If $\mathscr{M}$ is a connected $\mathscr{C}$-module C$^*$-category, then $\mathscr{M}$ is indecomposable, i.e. $\mathscr{M}$ is not equivalent to a direct sum of two non-trivial $\mathscr{C}$-module C$^*$-categories. In particular, if $\mathscr{N}$ is another connected $\mathscr{C}$-module C$^*$-category such that $\mathscr{N}\subset \mathscr{M}$, then either $\mathscr{N}\cong (0)$ or $\mathscr{N}\cong\mathscr{M}$.
	\end{pro}
	\begin{proof}
		Assume that $\mathscr{M}$ is indecomposable, i.e. there exist two non-trivial $\mathscr{C}$-module C$^*$-categories $\mathscr{N}_1$ and $\mathscr{N}_2$ such that $\mathscr{M}\cong \mathscr{N}_1\oplus \mathscr{N}_2$. Observe that in this case both $\mathscr{N}_1$ and $\mathscr{N}_2$ can be regarded as $\mathscr{C}$-module C$^*$-subcategories of $\mathscr{M}$. Let $X_1\in\text{Obj}(\mathscr{N}_1)\backslash \text{Obj}(\mathscr{N}_2)$ and $X_2\in\text{Obj}(\mathscr{N}_2)\backslash \text{Obj}(\mathscr{N}_1)$ be non-zero objects. Then, as objects in $\mathscr{M}$, there exists an object $U\in\text{Obj}(\mathscr{C})$ such that $\text{Hom}_{\mathscr{M}}(U\bullet X_1, X_2)\neq 0$ thanks to connectedness of $\mathscr{M}$ (cf. Remark \ref{rem.RephrasedCoFinConnect}). Observe that $U\bullet X_1\in\text{Obj}(\mathscr{N}_1)$ because $\mathscr{N}_1$ is a $\mathscr{C}$-module C$^*$-subcategory of $\mathscr{M}$.  Since, $\mathscr{N}_1\cap \mathscr{N}_2=(0)$ and both $X_1$ and $X_2$ are non-zero objects, the condition $\text{Hom}_{\mathscr{M}}(U\bullet X_1, X_2)\neq 0$ cannot hold true contradicting connectedness of $\mathscr{M}$. We conclude that $\mathscr{M}$ must be indecomposable.
	\end{proof}
	
	\begin{theodefi}[{\cite[Lemma 3.10 \& Lemma 3.11]{YukiKenny}}]\label{theo.TorsionFreeYukiKenny}
		Let $\mathscr{C}$ be a rigid C$^*$-tensor category. We say that $\mathscr{C}$ is torsion-free if one (hence all) of the following equivalent condition holds:
		\begin{enumerate}[i)]
			\item For every torsion $\mathscr{C}$-module C$^*$-category $\mathscr{M}$, $\text{Fus}(\mathscr{M})\cong \text{Fus}(\mathscr{C})$ as based modules.
			\item Every non-trivial torsion $\mathscr{C}$-module C$^*$-category is equivalent to $\mathscr{C}$ as $\mathscr{C}$-module C$^*$-categories.
		\end{enumerate}
		
		In particular, a discrete quantum group $\widehat{\mathbb{G}}$ is torsion-free if and only if for every torsion $\mathscr{R}\text{ep}(\mathbb{G})$-module C$^*$-category $\mathscr{M}$, $\text{Fus}(\mathscr{M})\cong \text{R}(\mathbb{G})$ as based modules.
	\end{theodefi}
	\begin{rem}
		Observe that the fusion ring associated to the C$^*$-tensor category $\mathscr{R}\text{ep}(\mathbb{G})$ is simply the representation ring of $\mathbb{G}$, $\text{R}(\mathbb{G})$. Hence, the above characterization of torsion-freeness of $\widehat{\mathbb{G}}$ is weaker than the strong torsion-freeness of $\widehat{\mathbb{G}}$ since here we work with a more restricted class of $\text{R}(\mathbb{G})$-modules. Namely, those arising from $\mathscr{R}\text{ep}(\mathbb{G})$-module C$^*$-categories.
	\end{rem}
	
	Finally, K. De Commer and M. Yamashita \cite[Theorem 6.4]{KennyMakoto} have obtained a one-to-one correspondence between ergodic actions of $\mathbb{G}$ and connected $\mathscr{R}\text{ep}(\mathbb{G})$-module C$^*$-categories. Taking into account Theorem-Definition \ref{theo.TorsionFreeYukiKenny}, this allows in practice to replace torsion actions of a compact quantum group by torsion $\text{R}(\mathbb{G})$-modules. This correspondence is central to classify torsion actions of a given compact quantum group, so let us describe it more precisely. 
	
	If $(A,\alpha)$ is a $\mathbb{G}$-C$^*$-algebra, then the category $\mathscr{M}_\alpha$ of all $\mathbb{G}$-equivariant Hilbert $A$-modules is a left $\mathscr{R}\text{ep}(\mathbb{G})$-module C$^*$-category with left action of $\mathscr{R}\text{ep}(\mathbb{G})$ given by $u\bullet E:=H_u\otimes E,$ for all $u\in Obj(\mathscr{R}\text{ep}(\mathbb{G}))$ and all $(E,\alpha_E)\in Obj(\mathscr{M}_\alpha)$ where $\alpha_{u\bullet E}: u\bullet E \longrightarrow u\bullet E\otimes C(\mathbb{G})$ is such that $\alpha_{u\bullet E}(\xi\otimes\eta)=u_{13}(\xi\otimes \alpha_E(\eta)),$ for all $\xi\in H_u$ and all $\eta\in E$. Moreover, $\mathscr{M}_\alpha$ is semi-simple and connected whenever $\alpha$ is ergodic. In this case, it is also known \cite{KennyMakoto} that every irreducible $\mathbb{G}$-equivariant Hilbert $A$-module arises as a $\mathbb{G}$-equivariant Hilbert submodule of $H_x\otimes A$ for some irreducible representation $x$ of $\mathbb{G}$.
	
	Conversely, if $\mathscr{M}$ is a connected $\mathscr{R}\text{ep}(\mathbb{G})$-module C$^*$-category, for every object $X\in Obj(\mathscr{M})$ the vector space \\$\mathcal{B}^X_X:=\underset{x\in \text{Irr}(\mathbb{G})}{\bigoplus} Hom_{\mathscr{M}}(u^x\bullet X, X)\otimes H_x$ is a unital $*$-algebra with an algebraic action of $\mathbb{G}$ denoted by $\alpha_{X, \mathscr{M}}$ (see \cite[Section 5]{KennyMakoto} for the explicit formulas). If $B^X_X$ denotes its C$^*$-completion, then $(B^X_X, \alpha_{X, \mathscr{M}})$ is a $\mathbb{G}$-C$^*$-algebra. Moreover, $\alpha_{X, \mathscr{M}}$ is ergodic whenever $X$ is irreducible in $\mathscr{M}$. 
		
	These associations give rise to a one-to-one correspondence between ergodic actions of $\mathbb{G}$ (up to equivariant Morita equivalence) and connected $\mathscr{R}\text{ep}(\mathbb{G})$-module C$^*$-categories (up to equivalence of $\mathscr{R}\text{ep}(\mathbb{G})$-module C$^*$-categories). More precisely, if $(A, \alpha)$ is an ergodic $\mathbb{G}$-C$^*$-algebra, then it can be viewed as an irreducible object in $\mathscr{M}_\alpha$ and one has \cite[Proposition 6.2]{KennyMakoto} $(B^A_A, \alpha_{A, \mathscr{M}_\alpha})\underset{\mathbb{G}}{\sim} (A, \alpha)$. Conversely, if $\mathscr{M}$ is a connected $\mathscr{R}\text{ep}(\mathbb{G})$-module C$^*$-category and $X\in \text{Irr}(\mathscr{\mathscr{M}})$, then one has \cite[Proposition 6.3]{KennyMakoto} $\mathscr{M}_{\alpha_{X, \mathscr{M}}}\cong \mathscr{M}$. In particular, $\mathscr{M}_\alpha$ is a torsion $\mathscr{R}\text{ep}(\mathbb{G})$-module C$^*$-category if and only if $\alpha$ is a torsion action of $\mathbb{G}$. In this correspondence we have that $\mathscr{M}_\alpha\cong \mathscr{R}\text{ep}(\mathbb{G})$ if and only if $(A,\alpha)\underset{\mathbb{G}}{\sim}(\mathbb{C}, trv.)$.
	

\begin{comment}	
	\subsection{Two-sided crossed products}\label{sec.TwosidedCrossProd}
	The following construction appears already in \cite[Section 2.6]{NikshychVainerman} in the context of quantum groupoids. It is used to formulate a quantum Baum-Connes assembly map in \cite{YukiBCTorsion} and \cite{KennyNestRubenBCProjective} and it will be useful for the purpose of the present paper.
	\begin{defi}
		Let $\mathbb{G}$ be a compact quantum group. If $(B, \beta)$ is a right $\mathbb{G}$-C$^*$-algebra and $(A, \alpha)$ is a left $\mathbb{G}$-C$^*$-algebra, then the two-sided crossed product of $B$ and $A$ by $\mathbb{G}$, denoted by $B\underset{r, \beta}{\rtimes}\mathbb{G}\underset{r, \alpha}{\ltimes}A$, is the C$^*$-algebra defined by:
		$$B\underset{r, \beta}{\rtimes}\mathbb{G}\underset{r, \alpha}{\ltimes}A:=C^*\langle ((id\otimes \lambda)\beta(B)\otimes 1)(1\otimes \widehat{\lambda}(c_0(\widehat{\mathbb{G}}))\otimes 1)(1\otimes (\rho\otimes id)(\alpha(A)))\rangle\subset \mathcal{L}_{B\otimes A}(B\otimes L^2(\mathbb{G})\otimes A).$$
	\end{defi}
	
	First, to lighten the notations we will omit the representations $\lambda$, $\widehat{\lambda}$ and $\rho$ in the definition of $B\underset{r, \beta}{\rtimes}\mathbb{G}\underset{r, \alpha}{\ltimes}A$, and note that then $\rho(x) = U_{\mathbb{G}}xU_{\mathbb{G}}$ for $x\in C(\mathbb{G})$. We also write $\alpha_U(x) = (U_{\mathbb{G}}\otimes id)\alpha(x)(U_{\mathbb{G}}\otimes id)$ for $x\in A$. Next, it is easy to show that $B\underset{r, \beta}{\rtimes}\mathbb{G}\underset{r, \alpha}{\ltimes}A=\overline{span}\{(\beta(B)\otimes 1)(1\otimes c_0(\widehat{\mathbb{G}})\otimes 1)(1\otimes \alpha_U(A))\}$. From now on we will use these two descriptions of $B\underset{r, \beta}{\rtimes}\mathbb{G}\underset{r, \alpha}{\ltimes}A$ interchangeably. As a consequence, we see that the maps $B\longrightarrow\mathcal{L}_{B\otimes A}(B\otimes L^2(\mathbb{G})\otimes A)$, $A\longrightarrow\mathcal{L}_{B\otimes A}(B\otimes L^2(\mathbb{G})\otimes A)$ and $c_0(\widehat{\mathbb{G}})\longrightarrow\mathcal{L}_{B\otimes A}(B\otimes L^2(\mathbb{G})\otimes A)$ given by $b\mapsto \beta(b)\otimes 1$, $a\mapsto 1\otimes \alpha_U(a)$ and $x\mapsto 1\otimes \widehat{\lambda}(x)\otimes 1$ respectively, send $B$, $A$ and $c_0(\widehat{\mathbb{G}})$ respectively onto non-degenerate C$^*$-subalgebras of $M(B\underset{r, \beta}{\rtimes}\mathbb{G}\underset{r, \alpha}{\ltimes}A)$.
		
	As for usual crossed products, we can show that the two-sided crossed product construction is functorial. More precisely, we have the following:
	\begin{pro}
		Let $\mathbb{G}$ be a compact quantum group. Let $(B, \beta)$, $(B', \beta')$ be right $\mathbb{G}$-C$^*$-algebras and $(A, \alpha)$, $(A', \alpha')$ left $\mathbb{G}$-C$^*$-algebras. 
		\begin{enumerate}[i)]
			\item If $\phi: B\longrightarrow M(B')$ is a non-degenerate $\mathbb{G}$-equivariant $*$-homomorphism, then there exists a non-degenerate $*$-homomorphism:
		$$\phi\rtimes id\ltimes id: B\underset{r, \beta}{\rtimes}\mathbb{G}\underset{r, \alpha}{\ltimes}A\longrightarrow M(B'\underset{r, \beta'}{\rtimes}\mathbb{G}\underset{r, \alpha}{\ltimes}A)$$
		such that $\phi\rtimes id\ltimes id\big((\beta(b)\otimes 1)(1\otimes x\otimes 1)(1\otimes\alpha_U(a))\big)=(\beta'(\phi(b))\otimes 1)(1\otimes x\otimes 1)(1\otimes\alpha_U(a))$, for all $b\in B$, $a\in A$ and $x\in c_0(\widehat{\mathbb{G}})$.
			\item If $\psi: A\longrightarrow M(A')$ is a non-degenerate $\mathbb{G}$-equivariant $*$-homomorphism, then there exists a non-degenerate $*$-homomorphism:
		$$id\rtimes id\ltimes \psi: B\underset{r, \beta}{\rtimes}\mathbb{G}\underset{r, \alpha}{\ltimes}A\longrightarrow M(B\underset{r, \beta}{\rtimes}\mathbb{G}\underset{r, \alpha'}{\ltimes}A')$$
		such that $id\rtimes id\ltimes \psi\big((\beta(b)\otimes 1)(1\otimes x\otimes 1)(1\otimes\alpha_U(a))\big)=(\beta(b)\otimes 1)(1\otimes x\otimes 1)(1\otimes\alpha_U'(\psi(a)))$, for all $b\in B$, $a\in A$ and $x\in c_0(\widehat{\mathbb{G}})$.
		\end{enumerate}
	\end{pro}
	
	The following result (which is a generalisation of Proposition \ref{pro.DirectProductTensorProduct}) is straightforward after a routine computation by applying the definitions and the fact that $c_0(\widehat{\mathbb{F}})=c_0(\widehat{\mathbb{G}})\otimes c_0(\widehat{\mathbb{H}})$, where $\mathbb{F}:=\mathbb{G}\times\mathbb{H}$ is a quantum direct product; as recalled in Section \ref{sec.NotationsConventions}.
	\begin{proSec}\label{pro.TwoSidedDirectProd}
	Let $\mathbb{G}$ and $\mathbb{H}$ be two compact quantum groups and let $\mathbb{F}:=\mathbb{G}\times\mathbb{H}$ be the corresponding quantum direct product of $\mathbb{G}$ and $\mathbb{H}$. Let $(A,\alpha)$ and $(T, \tau)$ be two right $\mathbb{G}$-C$^*$-algebras. Let $(B,\beta)$ and $(S, \sigma)$ be two left $\mathbb{H}$-C$^*$-algebras. Then $(A\otimes B)\underset{r, \alpha\otimes\beta}{\rtimes}\mathbb{F}\underset{r, \tau\otimes \sigma}{\ltimes}(T\otimes S)\cong (A\underset{r, \alpha}{\rtimes}\mathbb{G}\underset{r, \tau}{\ltimes}T)\otimes (B\underset{r, \beta}{\rtimes}\mathbb{H}\underset{r, \sigma}{\ltimes}S)$.
	\end{proSec}
\end{comment}











	\subsection{The equivariant Kasparov category and the quantum Baum-Connes assembly map}\label{sec.QuantumBC}
	We refer to \cite{Blackadar}, \cite{BaajSkandalisQuantumKK} for more details about (equivariant) $KK$-theory and to  \cite{Jorgensen}, \cite{MeyerNest} for more details about triangulated categories and the Meyer-Nest approach to the BC property, which is fundamental for a quantum counterpart of the conjecture. 
	
	Let $\widehat{\mathbb{G}}$ be a discrete quantum group and consider the corresponding equivariant Kasparov category, $\mathscr{K}\mathscr{K}^{\widehat{\mathbb{G}}}$, with canonical suspension functor denoted by $\Sigma$. $\mathscr{K}\mathscr{K}^{\widehat{\mathbb{G}}}$ is a triangulated category whose distinguished triangles are given by mapping cone triangles (see \cite{MeyerNest} for more details). The word \emph{homomorphism (resp.\ isomorphism)} will mean \emph{homomorphism (resp.\ isomorphism) in the corresponding Kasparov category}; it will be a true homomorphism (resp.\ isomorphism) between $\widehat{\mathbb{G}}$-C$^*$-algebras or any Kasparov triple between $\widehat{\mathbb{G}}$-C$^*$-algebras (resp.\ any equivariant $KK$-equivalence between $\widehat{\mathbb{G}}$-C$^*$-algebras). Analogously, we can consider the equivariant Kasparov category $\mathscr{K}\mathscr{K}^{\mathbb{G}}$.
	
	Exterior tensor product of Kasparov triples is important for the purpose of the present paper. Namely, given C$^*$-algebras $A$, $A'$, $B$, $B'$ and $D$, the map:
						$$
						\begin{array}{rccl}
							\widetilde{\tau}_D:&KK(A,A'\otimes D)\times KK(D\otimes B, B') & \longrightarrow &KK(A\otimes B, A'\otimes B')\\
							&(\mathcal{X}, \mathcal{Y}) & \longmapsto &\widetilde{\tau}_D(\mathcal{X}, \mathcal{Y}):=\tau_{B}(\mathcal{X})\underset{A'\otimes D\otimes B}{\otimes}{_{A'}}\tau(\mathcal{Y})=:\mathcal{X}\underset{D}{\widehat{\otimes}}\mathcal{Y}
						\end{array}
						$$
						is bilinear, contravariantly functorial in $A$ and $B$ and covariantly functorial in $A'$ and $B'$. Here $\underset{A'\otimes D\otimes B}{\otimes}$ denotes the usual Kasparov product over $A'\otimes D\otimes B$; and $\tau_B$ and ${_{A'}}\tau$ denote the right and left exterior tensor products by $B$ and $A'$, respectively (cf. \cite[Chapter VIII, Section 18]{Blackadar} for details). Notice however that $\tau_A\cong{_{A}}\tau$, for all C$^*$-algebra $A$ because the tensor product of C$^*$-algebras is a commutative operation. In particular, taking $D:=\mathbb{C}$, $\widetilde{\tau}_{\mathbb{C}}$ defines a \emph{tensor product of Kasparov triples}. Namely, by virtue of \cite[Proposition 18.9.1]{Blackadar}, the operation $\widetilde{\tau}_{\mathbb{C}}(\ \cdot\ )= (\ \cdot\ ) \underset{\mathbb{C}}{\widehat{\otimes}} (\ \cdot\ )$ is associative. For the sake of completeness, let us check the latter claim by translating the formulas from \cite[Proposition 18.9.1]{Blackadar} to the case $\widetilde{\tau}_{\mathbb{C}}$. Let $A$, $A'$, $B$, $B'$, $C$ and $C'$ be C$^*$-algebras and $\mathcal{X}\in KK(A, A') $, $\mathcal{Y}\in KK(B, B')$, $\mathcal{Z}\in KK(C, C')$. Then:
	\begin{equation*}
	\begin{split}
		(\mathcal{X}\underset{\mathbb{C}}{\widehat{\otimes}}\mathcal{Y})\underset{\mathbb{C}}{\widehat{\otimes}}\mathcal{Z}&=\tau_{C}\Big(\mathcal{X}\underset{\mathbb{C}}{\widehat{\otimes}}\mathcal{Y}\Big)\underset{A'\otimes B'\otimes C}{\otimes} {_{A'\otimes B'}}\tau(\mathcal{Z})\\
		&=\Big(\tau_C(\mathcal{X})\underset{C}{\widehat{\otimes}}{_{C}}\tau(\mathcal{Y})\Big)\underset{A'\otimes B'\otimes C}{\otimes} {_{A'\otimes B'}}\tau(\mathcal{Z})\\
		&=\Big(\tau_B\big(\tau_C(\mathcal{X})\big)\underset{A'\otimes B\otimes C}{\otimes} {_{A'}}\tau\big({_C}\tau(\mathcal{Y})\big)\Big)\underset{A'\otimes B'\otimes C}{\otimes} {_{A'\otimes B'}}\tau(\mathcal{Z})\\
		&=\Big(\tau_{B\otimes C}(\mathcal{X})\underset{A'\otimes B\otimes C}{\otimes} {_{C\otimes A'}}\tau(\mathcal{Y})\Big)\underset{A'\otimes B'\otimes C}{\otimes} {_{A'\otimes B'}}\tau(\mathcal{Z})\\
		&=\tau_{B\otimes C}(\mathcal{X})\underset{A'\otimes B\otimes C}{\otimes} \Big({_{C\otimes A'}}\tau(\mathcal{Y})\underset{A'\otimes B'\otimes C}{\otimes} {_{A'\otimes B'}}\tau(\mathcal{Z})\Big)\\
		&=\tau_{B\otimes C}(\mathcal{X})\underset{A'\otimes B\otimes C}{\otimes} \Big(\tau_C\big(\tau_{A'}(\mathcal{Y})\big)\underset{A'\otimes B'\otimes C}{\otimes} {_{B'}}\tau\big({_{A'}}\tau(\mathcal{Z})\big)\Big)\\
		&=\tau_{B\otimes C}(\mathcal{X})\underset{A'\otimes B\otimes C}{\otimes} \Big(\tau_{A'}(\mathcal{Y})\underset{A'}{\widehat{\otimes}}{_{A'}}\tau(\mathcal{Z})\Big)\\
		&=\tau_{B\otimes C}(\mathcal{X})\underset{A'\otimes B\otimes C}{\otimes} {_{A'}}\tau\Big(\mathcal{Y}\underset{\mathbb{C}}{\widehat{\otimes}}\mathcal{Z}\Big)=\mathcal{X}\underset{\mathbb{C}}{\widehat{\otimes}}(\mathcal{Y}\underset{\mathbb{C}}{\widehat{\otimes}}\mathcal{Z}).
	\end{split}
	\end{equation*}
	
	Moreover, given a C$^*$-algebra $A$, the functor $A\otimes (\ \cdot\ ):\text{C}^*\text{-Alg}\rightarrow \mathscr{K}\mathscr{K}$, $B\mapsto A\otimes B$, is a split-exact, stable and homotopy invariant. By the universal property of the Kasparov category (cf. \cite{HigsonUnivProp}), it extends to a functor $A\otimes (\ \cdot\ ): \mathscr{K}\mathscr{K}\rightarrow \mathscr{K}\mathscr{K}$. Similarly, we have a functor $(\ \cdot\ )\otimes A: \mathscr{K}\mathscr{K}\rightarrow \mathscr{K}\mathscr{K}$. Since these extensions are natural, we obtain a bifunctor $\widetilde{\tau}_{\mathbb{C}}(\ \cdot\ )= (\ \cdot\ ) \underset{\mathbb{C}}{\widehat{\otimes}} (\ \cdot\ ): \mathscr{K}\mathscr{K}\times  \mathscr{K}\mathscr{K}\rightarrow \mathscr{K}\mathscr{K}$. In other words, the bifunctor $\widetilde{\tau}_{\mathbb{C}}$ confer a monoidal structure on $\mathscr{K}\mathscr{K}$. Finally, since the Kasparov product plays the role of composition of morphisms within the category $\mathscr{K}\mathscr{K}$, functoriality of $\widetilde{\tau}$ yields that:
	\begin{equation}\label{eq.TensorProdKasparovProd}
	\begin{split}
		(y \underset{\mathbb{C}}{\widehat{\otimes}} x)\underset{A\otimes B}{\otimes} (z  \underset{\mathbb{C}}{\widehat{\otimes}} w)=(y\underset{A}{\otimes} z)  \underset{\mathbb{C}}{\widehat{\otimes}} (x\underset{B}{\otimes} w),
	\end{split}
	\end{equation}
	for all $y\in KK(D, A)$, $x\in KK(D', B)$, $z\in KK(A, A')$, $w\in KK(B, B')$ and all C$^*$-algebras $A$, $A'$, $B$, $B'$, $D$, $D'$.
	\begin{rem}
		If $G$ is a locally compact group, then the equivariant Kasparov category $\mathscr{K}\mathscr{K}^{G}$ is also characterised by a universal property in terms of split-exactness, stability and homotopy invariance (cf. \cite{ThomsenUnivProp}). The exterior tensor product can be also defined in the equivariant setting, say $\widetilde{\tau}^G_{\mathbb{C}}$; the action of $G$ on a tensor product of C$^*$-algebras being the diagonal action. Again, $\mathscr{K}\mathscr{K}^{G}$ is equipped in this way with a monoidal structure and the analogue to Equation (\ref{eq.TensorProdKasparovProd}) holds. Furthermore, if $\mathbb{G}$ is a locally compact group, then $\mathscr{K}\mathscr{K}^{\mathbb{G}}$ is also characterised by a universal property as in the classical setting (cf. \cite{VoigtPoincareDuality}). If $D(\mathbb{G})$ denotes the Drinfeld double of $\mathbb{G}$, then $\mathscr{K}\mathscr{K}^{D(\mathbb{G})}$ is equipped with a monoidal structure by means of the \emph{braided tensor product} (cf. \cite{VoigtPoincareDuality}). Again, the analogue to Equation (\ref{eq.TensorProdKasparovProd}) holds.
	\end{rem}
	
	
	Assume for the moment that $\widehat{\mathbb{G}}$ is \emph{torsion-free}. In that case, consider the usual complementary pair of localizing subcategories in $\mathscr{K}\mathscr{K}^{\widehat{\mathbb{G}}}$, $(\mathscr{L}_{\widehat{\mathbb{G}}}, \mathscr{N}_{\widehat{\mathbb{G}}})$. Denote by $(L,N)$ the canonical triangulated functors associated to this complementary pair. More precisely we have that $\mathscr{L}_{\widehat{\mathbb{G}}}$ is defined as the \emph{localizing subcategory of $\mathscr{K}\mathscr{K}^{\widehat{\mathbb{G}}}$ generated by the objects of the form $Ind^{\widehat{\mathbb{G}}}_{\mathbb{E}}(C)=C\otimes c_0(\widehat{\mathbb{G}})$ with $C$ any C$^*$-algebra in the Kasparov category $\mathscr{K}\mathscr{K}$} and $\mathscr{N}_{\widehat{\mathbb{G}}}$ is defined as the \emph{localizing subcategory of objects which are isomorphic to $0$ in $\mathscr{K}\mathscr{K}$}: $\mathscr{L}_{\widehat{\mathbb{G}}}:=\langle\{Ind^{\widehat{\mathbb{G}}}_{\mathbb{E}}(C)=C\otimes c_0(\widehat{\mathbb{G}})\ |\ C\in Obj.(\mathscr{K}\mathscr{K})\}\rangle$ and $\mathscr{N}_{\widehat{\mathbb{G}}}=\{A\in Obj.(\mathscr{K}\mathscr{K}^{\widehat{\mathbb{G}}})\ |\ Res^{\widehat{\mathbb{G}}}_{\mathbb{E}}(A)=0\}$.
	%$$\mathscr{L}_{\widehat{\mathbb{G}}}:=\langle\{Ind^{\widehat{\mathbb{G}}}_{\mathbb{E}}(C)=c_0(\widehat{\mathbb{G}})\otimes C\ |\ C\in Obj.(\mathscr{K}\mathscr{K})\}\rangle,$$
	%$$\mathscr{N}_{\widehat{\mathbb{G}}}=\{A\in Obj.(\mathscr{K}\mathscr{K}^{\widehat{\mathbb{G}}})\ |\ Res^{\widehat{\mathbb{G}}}_{\mathbb{E}}(A)=0\}.$$%=\{A\in Obj.(\mathscr{K}\mathscr{K}^{\widehat{\mathbb{G}}})\ |\ L(A)=0\}.$$
	
	If $\widehat{\mathbb{G}}$ is \emph{not} torsion-free, then a technical property lacked in the literature in order to define a suitable complementary pair. The natural candidate used in the related works (see for instance \cite{MeyerNestTorsion} and \cite{VoigtBaumConnesAutomorphisms}) is given by the following localizing subcategories of $\mathscr{K}\mathscr{K}^{\widehat{\mathbb{G}}}$:
	$$\mathscr{L}_{\widehat{\mathbb{G}}}:=\langle\{C\otimes T\underset{r}{\rtimes}\mathbb{G}\ |\  C\in Obj.(\mathscr{K}\mathscr{K}),\ T\in\mathbb{T}\mbox{or}(\widehat{\mathbb{G}})\}\rangle,$$
	$$\mathscr{N}_{\widehat{\mathbb{G}}}:=\mathscr{L}^{\dashv}_{\widehat{\mathbb{G}}}=\{A\in Obj(\mathscr{K}\mathscr{K}^{\widehat{\mathbb{G}}})\ |\ KK^{\widehat{\mathbb{G}}}(L, A)=(0)\mbox{, $\forall$ $L\in Obj(\mathscr{L}_{\widehat{\mathbb{G}}})$}\}.$$
	
	\begin{rem}
			We put $\widehat{\mathscr{L}}_{\widehat{\mathbb{G}}}:=\langle\{C\otimes T\ |\  C\in Obj.(\mathscr{K}\mathscr{K}),\ T\in\mathbb{T}\mbox{or}(\widehat{\mathbb{G}})\}\rangle$, so that we have $\widehat{\mathscr{L}}_{\widehat{\mathbb{G}}} \rtimes \mathbb{G}=\mathscr{L}_{\widehat{\mathbb{G}}}$ by definition. Similarly we put $\widehat{\mathscr{N}}_{\widehat{\mathbb{G}}}:=\mathscr{N}_{\widehat{\mathbb{G}}}\rtimes \widehat{\mathbb{G}}$. The pair $(\widehat{\mathscr{L}}_{\widehat{\mathbb{G}}}, \widehat{\mathscr{N}}_{\widehat{\mathbb{G}}})$ is still complementary. We denote by $(\widehat{L}, \widehat{N})$ the corresponding triangulated functors defining the $(\widehat{\mathscr{L}}_{\widehat{\mathbb{G}}}, \widehat{\mathscr{N}}_{\widehat{\mathbb{G}}})$-triangles in $\mathscr{K}\mathscr{K}^{\mathbb{G}}$.
	\end{rem}
	
	Recently, Y. Arano and A. Skalski \cite{YukiBCTorsion} have showed that these two subcategories form indeed a complementary pair of localizing subcategories in $\mathscr{K}\mathscr{K}^{\widehat{\mathbb{G}}}$. Also, the author in collaboration with K. De Commer and R. Nest has obtained the same conclusion in \cite{KennyNestRubenBCProjective} for permutation torsion-free discrete quantum groups as an application of the study of the projective representation theory for compact quantum groups. In either case, the complementarity of the pair $(\mathscr{L}_{\widehat{\mathbb{G}}}, \mathscr{N}_{\widehat{\mathbb{G}}})$ is based in a generalisation of the \emph{Green-Julg isomorphism}. More precisely, given such a torsion action of $\mathbb{G}$, say $(T, \delta)$, we define the following triangulated functors:
	$$
		\begin{array}{rcclccl}
			j_{\mathbb{G}, T}:&\mathscr{K}\mathscr{K}^\mathbb{G}& \longrightarrow & \mathscr{K}\mathscr{K}, &(B, \beta)& \longmapsto & j_{\mathbb{G}, T}(B,\beta):= B\underset{r, \beta}{\rtimes}\mathbb{G}\underset{r, \overline{\delta}}{\ltimes}T^{op},
		\end{array}
	$$
	$$
		\begin{array}{rcclccl}
			\tau_{ T}:&\mathscr{K}\mathscr{K}& \longrightarrow & \mathscr{K}\mathscr{K}^\mathbb{G}, &C& \longmapsto & \tau_{T}(C):= (C\otimes T, id\otimes \delta).
		\end{array}
	$$
	
	The proof of the following result can be found in \cite{YukiBCTorsion} (see also \cite{KennyNestRubenBCProjective} for a different approach).
	\begin{theoSec}
		Let $\mathbb{G}$ be a compact quantum group. Let $(T, \delta)$ be a torsion action of $\mathbb{G}$. Then $\tau_T: \mathscr{K}\mathscr{K}\longrightarrow \mathscr{K}\mathscr{K}^{\mathbb{G}}$ is a left adjoint of $j_{\mathbb{G}, T}: \mathscr{K}\mathscr{K}^{\mathbb{G}}\longrightarrow \mathscr{K}\mathscr{K}$ as triangulated functors. More precisely,
		$$\Psi_T: KK^\mathbb{G}(C\otimes T, B)\overset{\sim}{\longrightarrow} KK\big(C, B\underset{r, \beta}{\rtimes}\mathbb{G}\underset{r, \overline{\delta}}{\ltimes}T^{op}\big),$$
		for all $C\in Obj(\mathscr{K}\mathscr{K})$ and $(B, \beta)\in Obj(\mathscr{K}\mathscr{K}^\mathbb{G})$.
	\end{theoSec}
	
	Then the general Meyer-Nest's machinery yields in particular that $\widehat{\mathscr{N}}_{\widehat{\mathbb{G}}}=\ker_{\text{Obj}}\big((j_{\mathbb{G}, T})_{T\in \mathbb{T}\text{or}(\widehat{\mathbb{G}})}\big)$. This allows to define a quantum assembly map for arbitrary discrete quantum groups (not necessarily torsion-free) and thus a quantum version of the (strong) BC property. More precisely, consider the following homological functor:
	$$
		\begin{array}{rcclccl}
			F_*:&\mathscr{K}\mathscr{K}^{\widehat{\mathbb{G}}}& \longrightarrow &\mathscr{A}b^{\mathbb{Z}/2}, &(A,\alpha) & \longmapsto &F_*(A):=K_{*}(A\underset{\alpha, r}{\rtimes} \widehat{\mathbb{G}}).
		\end{array}
	$$
	
	The quantum assembly map for $\widehat{\mathbb{G}}$ is given by the natural transformation $\eta^{\widehat{\mathbb{G}}}: \mathbb{L}F_*\longrightarrow F_*$, where $\mathbb{L}F_*$ denotes the localisation of $F_*$ with respect to the complementary pair $(\mathscr{L}_{\widehat{\mathbb{G}}}, \mathscr{N}_{\widehat{\mathbb{G}}})$. More precisely, given any $\widehat{\mathbb{G}}$-C$^*$-algebra $(A, \alpha)$, we consider a $(\mathscr{L}_{\widehat{\mathbb{G}}}, \mathscr{N}_{\widehat{\mathbb{G}}})$-triangle associated to $A$, say $\Sigma N(A)\longrightarrow L(A)\overset{u}{\longrightarrow} A\longrightarrow N(A)$. Then $\eta^{\widehat{\mathbb{G}}}_A=F_*(u)$. Let us point out a more precise picture for $\eta^{\widehat{\mathbb{G}}}_A$. On the one hand, given the $(\mathscr{L}_{\widehat{\mathbb{G}}}, \mathscr{N}_{\widehat{\mathbb{G}}})$-triangle $\Sigma N(A)\longrightarrow L(A)\overset{u}{\longrightarrow} A\longrightarrow N(A)$, the map $L(A)\underset{r}{\rtimes} \widehat{\mathbb{G}}\overset{u}{\rightarrow} A\underset{r}{\rtimes} \widehat{\mathbb{G}}$ must be viewed as a Kaparov triple $u\underset{r}{\rtimes} \widehat{\mathbb{G}}\in KK(L(A)\underset{r}{\rtimes} \widehat{\mathbb{G}}, A\underset{r}{\rtimes} \widehat{\mathbb{G}})$, i.e. a homomorphism in $\mathscr{K}\mathscr{K}$. On the other hand, the functor $F_*(\ \cdot\ )=KK_*(\mathbb{C}, (\ \cdot\ )\underset{r}{\rtimes} \widehat{\mathbb{G}})$ can be viewed as the covariant homomorphism functor $KK(\mathbb{C}, \ \cdot\ )$ in $\mathscr{K}\mathscr{K}$, which consists in composing, i.e. making a Kasparov product. In other words, when we apply this to crossed product C$^*$-algebras we have:
	\begin{equation}\label{eq.PictureQAssemblyMap}
	\begin{split}
	\eta^{\widehat{\mathbb{G}}}_A=F_*(u)=KK_*(\mathbb{C}, u\underset{r}{\rtimes} \widehat{\mathbb{G}})=(\ \cdot\ )\underset{L(A)\underset{r}{\rtimes} \widehat{\mathbb{G}}}{\otimes} u\underset{r}{\rtimes} \widehat{\mathbb{G}}: KK(\mathbb{C}, L(A)\underset{r}{\rtimes} \widehat{\mathbb{G}})\rightarrow KK(\mathbb{C}, A\underset{r}{\rtimes} \widehat{\mathbb{G}}).
	\end{split}
	\end{equation}
	\begin{defi}
		Let $\widehat{\mathbb{G}}$ be a discrete quantum group. We say that $\widehat{\mathbb{G}}$ satisfies the quantum Baum-Connes property (with coefficients) if the natural transformation $\eta^{\widehat{\mathbb{G}}}: \mathbb{L}F_*\longrightarrow F_*$ is a natural equivalence. We say that $\widehat{\mathbb{G}}$ satisfies the \emph{strong} Baum-Connes property if $\mathscr{K}\mathscr{K}^{\widehat{\mathbb{G}}}=\mathscr{L}_{\widehat{\mathbb{G}}}$. We abbreviate the predicate \emph{Baum-Connes property} by \emph{BC property}.
	\end{defi}
	
	\begin{note}\label{note.DiracHom}
		The following nomenclature is useful. Given $A\in Obj.(\mathscr{K}\mathscr{K}^{\widehat{\mathbb{G}}})$ consider a $(\mathscr{L}_{\widehat{\mathbb{G}}}, \mathscr{N}_{\widehat{\mathbb{G}}})$-triangle associated to $A$, say $\Sigma N(A)\longrightarrow L(A)\overset{u}{\longrightarrow} A\longrightarrow N(A)$. We know that such triangles are distinguished and unique up to isomorphism (cf. \cite{MeyerNest} for a proof). The homomorphism $u:L(A)\longrightarrow A$ is called \emph{Dirac homomorphism for $A$}. Sometimes we write $D:=u$. In particular, we consider the Dirac homomorphism for $\mathbb{C}$ (as trivial $\widehat{\mathbb{G}}$-C$^*$-algebra), $D_{\mathbb{C}}:L(\mathbb{C})\longrightarrow \mathbb{C}$. We refer to $D_{\mathbb{C}}$ simply as \emph{Dirac homomorphism}.
	\end{note}
	
	
	
	
\section{\textsc{Torsion for quantum direct products}}\label{sec.TorsionQuantumDirProd}
		First, let us recall briefly  the construction of quantum direct product in the sense of S. Wang \cite{WangSemidirect}. Let $\mathbb{G}$ and $\mathbb{H}$ be two compact quantum groups. The quantum direct product of $\mathbb{G}$ and $\mathbb{H}$ is a compact quantum group denoted by $\mathbb{F}:=\mathbb{G}\times\mathbb{H}$ with $C(\mathbb{F})=C(\mathbb{G})\otimes C(\mathbb{H})$. The representation theory of $\mathbb{F}$ is particularly interesting for the purpose of the present paper. We refer to \cite{WangSemidirect} for more details. For every irreducible representation $z\in \text{Irr}(\mathbb{F})$, take a representative $u^z\in \mathcal{B}(H_z)\otimes C(\mathbb{F})$. Then there exist unique irreducible representations $x\in \text{Irr}(\mathbb{G})$ and $y\in \text{Irr}(\mathbb{H})$ such that if $u^x\in \mathcal{B}(H_x)\otimes C(\mathbb{G})$ and $u^y\in \mathcal{B}(H_y)\otimes C(\mathbb{H})$ are respective representatives of $x$ and $y$, then we have $u^z\cong u^x_{13}u^y_{24}\in\mathcal{B}(H_x\otimes H_y)\otimes C(\mathbb{F})\mbox{,}$ where $u^x_{13}$ and $u^y_{24}$ are the corresponding legs of $u^x$ and $u^y$, respectively inside $\mathcal{B}(H_x)\otimes\mathcal{B}(H_y)\otimes C(\mathbb{G})\otimes C(\mathbb{H})$. In this case we write $u^{(x,y)}:=u^{z}$.
		
		\begin{remSec}\label{rem.TensorProdRepsF}
			Let $x,x'\in\text{Irr}(\mathbb{G})$ and $y,y'\in\text{Irr}(\mathbb{H})$. We consider the following (irreducible) representations of $\mathbb{F}$, $z:=(x,y)$ and $z':=(x', y')$. It is useful to compute their tensor product as representations of $\mathbb{F}$. A straightforward computation yields that $u^{z\otimes z'}:=u^{z}\otimes u^{z'}=(u^{x}\otimes u^{x'})_{13}(u^{y}\otimes u^{y'})_{24}$.
		\end{remSec}
		
		By virtue of the work \cite{YukiKenny} by Y. Arano and K. De Commer, we know that if both $\widehat{\mathbb{G}}$ and $\widehat{\mathbb{H}}$ are torsion-free, then $\widehat{\mathbb{F}}$ is torsion-free too. The converse is also true because both $\widehat{\mathbb{G}}$ and $\widehat{\mathbb{H}}$ can be viewed as \emph{divisible} discrete quantum subgroups of $\widehat{\mathbb{F}}$ and torsion-freeness is preserved under divisible discrete quantum subgroups as shown in \cite{RubenTorsionDivisibles}. However, it is an open problem to classify all torsion actions of $\mathbb{F}:=\mathbb{G}\times\mathbb{H}$. In this section we are going to show that any torsion action of $\mathbb{F}$ is a tensor product of a torsion action of $\mathbb{G}$ with a torsion action of $\mathbb{H}$.
		
	\subsection{Preparatory observations}
	\begin{lem}\label{lem.TorsionTensorProd}
		Let $R:=(\mathbb{Z}_I, \oplus, \otimes)$ be a $I$-based ring and let $S:=(\mathbb{Z}_L, \oplus, \otimes)$ be an $L$-based ring. Let $M:=(\mathbb{Z}_J, \oplus, \otimes)$ be a based $R$-module with basis $J$ and let $N:=(\mathbb{Z}_K, \oplus, \otimes)$ be a based $S$-module with basis $K$. We denote by $M\boxtimes N$ the free $\mathbb{Z}$-module $\mathbb{Z}_J\underset{\mathbb{Z}}{\odot} \mathbb{Z}_K$ with basis $J\underset{\mathbb{Z}}{\odot} K$. Then $M\boxtimes N$ is a based $R\boxtimes S$-module with (left) $\mathbb{Z}_I\underset{\mathbb{Z}}{\odot} \mathbb{Z}_L$-module structure denoted by $\bullet$ and such that: $$(\alpha\boxtimes\beta)\bullet (j\boxtimes k)=(\alpha\otimes j)\boxtimes (\beta\otimes k),$$
		for all $\alpha\in I$, $\beta\in L$, $j\in J$ and $k\in K$. In particular, given $\alpha\in I$, $\beta\in L$, $j, j'\in J$ and $k,k'\in K$ the structural constants for $\bullet$ are $N^{j\boxtimes k}_{\alpha\boxtimes \beta, j'\boxtimes k'}:=N^{j}_{\alpha, j'}\cdot N^{k}_{\beta, k'}$. Moreover, $M\boxtimes N$ is a torsion $R\boxtimes S$-module as soon as $M$ is a torsion $R$-module and $N$ is a torsion $S$-module.
	\end{lem}
	\begin{proof}
		First of all, observe that $\bullet$ is nothing but the (left) $\mathbb{Z}_I\underset{\mathbb{Z}}{\odot} \mathbb{Z}_L$-module structure on $\mathbb{Z}_J\underset{\mathbb{Z}}{\odot} \mathbb{Z}_K$ as exterior tensor product of $\mathbb{Z}$-modules. It is a general fact then that $\bullet$ defines indeed a (left) module structure on $M\boxtimes N$. In this way, it is clear that the structural constants for $\bullet$ are as explained in the statement. Indeed, given $\alpha\in I$, $\beta\in L$, $j, j'\in J$ and $k,k'\in K$ we write:
		\begin{equation*}
		\begin{split}
			(\alpha\boxtimes \beta)\bullet (j'\boxtimes k')&=(\alpha\otimes j')\boxtimes (\beta\otimes k')=\big(\underset{j\in J}{\sum} N^{j}_{\alpha, j'}\cdot j\big)\boxtimes \big(\underset{k\in K}{\sum} N^{k}_{\beta, k'}\cdot k\big)\\
			&=\underset{j\in J,\ k\in K}{\sum} N^{j}_{\alpha, j'}\cdot N^{k}_{\beta, k'}\cdot (j\boxtimes k),
		\end{split}
		\end{equation*}
		which suggests to put $N^{j\boxtimes k}_{\alpha\boxtimes \beta, j'\boxtimes k'}:=N^{j}_{\alpha, j'}\cdot N^{k}_{\beta, k'}$. Moreover, $M\boxtimes N$ is a $J\underset{\mathbb{Z}}{\odot} K$-based $R\boxtimes S$-module, i.e. $j\boxtimes k\subset (\alpha\boxtimes \beta)\bullet (j'\boxtimes k')$ $\Leftrightarrow$ $j'\boxtimes k'\subset (\overline{\alpha\boxtimes \beta})\bullet (j\boxtimes k)$, for all $\alpha\in I$, $\beta\in L$, $j,j'\in J$ and $k, k'\in K$. This is true because $M$ is a $J$-based $R$-module and $N$ is a $K$-based $S$-module, so that:
		\begin{equation*}
		\begin{split}
			j\boxtimes k\subset (\alpha\boxtimes \beta)\bullet (j'\boxtimes k')&\Leftrightarrow N^{j\boxtimes k}_{\alpha\boxtimes \beta, j'\boxtimes k'}\neq 0 \Leftrightarrow N^{j}_{\alpha, j'}\neq 0\mbox{ and }N^{k}_{\beta, k'}\neq 0\\
			&\Leftrightarrow j\subset \alpha\otimes j' \mbox{ and } k\subset \beta\otimes k' \Leftrightarrow j'\subset \overline{\alpha}\otimes j \mbox{ and } k'\subset \overline{\beta}\otimes k\\
			&\Leftrightarrow N^{j'}_{\overline{\alpha}, j}\neq 0\mbox{ and }N^{k'}_{\overline{\beta}, k}\neq 0 \Leftrightarrow N^{j'\boxtimes k'}_{\overline{\alpha}\boxtimes \overline{\beta}, j\boxtimes k}=N^{j'\boxtimes k'}_{\overline{\alpha\boxtimes \beta}, j\boxtimes k}\neq 0\\
			&\Leftrightarrow j'\boxtimes k'\subset (\overline{\alpha\boxtimes \beta})\bullet (j\boxtimes k).
		\end{split}
		\end{equation*}
		
		Let us show that $M\boxtimes N$ is torsion as soon as $M$ and $N$ are torsion. We start by showing that $M\boxtimes N$ is co-finite. Given $j, j'\in J$ and $k, k'\in K$, then:
		\begin{equation*}
		\begin{split}
			\{\alpha\boxtimes \beta\in I\underset{\mathbb{Z}}{\odot} L\ |\ j\boxtimes k\subset (\alpha\boxtimes \beta)\bullet (j'\boxtimes k')\}&=\{\alpha\boxtimes \beta\in I\underset{\mathbb{Z}}{\odot} L\ |\ N^{j\boxtimes k}_{\alpha\boxtimes \beta, j'\boxtimes k'}:=N^{j}_{\alpha, j'}\cdot N^{k}_{\beta, k'}\neq 0\}\\
			&=\{\alpha\boxtimes \beta\in I\underset{\mathbb{Z}}{\odot} L\ |\ N^{j}_{\alpha, j'}\neq 0 \mbox{ and } N^{k}_{\beta, k'}\neq 0\}\\
			&=\{\alpha\in I\ |\ j\subset \alpha \otimes j'\} \cap \{\beta\in L\ |\ k\subset \beta \otimes k'\},
		\end{split}
		\end{equation*}
		which is finite because both $M$ and $N$ are co-finite. Finally, we show that $M\boxtimes S$ is connected. Given $j,j'\in J$ and $k,k'\in K$, there exist $\alpha\in I$ and $\beta\in L$ such that $j\subset \alpha\otimes j'$ and $k\subset \beta\otimes k'$ because both $M$ and $N$ are connected. This means that $N^{j}_{\alpha, j'}\neq 0$ and $N^{k}_{\beta, k'}\neq 0$, i.e. $N^{j\boxtimes k}_{\alpha\boxtimes \beta, j'\boxtimes k'}\neq 0$. Therefore, $\alpha\boxtimes \beta\in I\underset{\mathbb{Z}}{\odot} L$ is such that $j\boxtimes k\subset (\alpha\boxtimes \beta)\bullet (j'\boxtimes k')$.	
	\end{proof}
	\begin{rem}\label{rem.TorsionTensorProd}
		Note that the previous proof shows that $M\boxtimes N$ is co-finite (resp. connected) as soon as $M$ and $N$ are co-finite (resp. connected). It also follows from the proof that if $M\boxtimes N$ is co-finite (resp. connected), then both $M$ and $N$ are co-finite (resp. connected).
	\end{rem}
	
	\begin{pro}\label{pro.TensorProdRepRingDirProd}
		Let $\mathbb{G}$ and $\mathbb{H}$ be two compact quantum groups and let $\mathbb{F}:=\mathbb{G}\times \mathbb{H}$ be the corresponding quantum direct product of $\mathbb{G}$ and $\mathbb{H}$. Then $\text{R}(\mathbb{F})\cong \text{R}(\mathbb{G})\boxtimes \text{R}(\mathbb{H})$ as based modules.
	\end{pro}
	\begin{proof}
		This is an immediate consequence of the representation theory of $\mathbb{F}$. As explained in the beginning of this section, every irreducible representation of $\mathbb{F}$ is represented by the exterior tensor product of an irreducible representation of $\mathbb{G}$ with an irreducible representation of $\mathbb{H}$. This allows to regard $\text{R}(\mathbb{F})$ as the free $\mathbb{Z}$-module with basis $\text{Irr}(\mathbb{G})\underset{\mathbb{Z}}{\odot}\text{Irr}(\mathbb{H})$ so that $\text{R}(\mathbb{F})\cong \text{R}(\mathbb{G})\boxtimes \text{R}(\mathbb{H})$ as defined in Examples \ref{exs.FusionRings}.
	\end{proof}
	
	This proposition together with Lemma \ref{lem.TorsionTensorProd} yields the following:
	\begin{cor}\label{cor.TensorProdTorsMod}
		Let $\mathbb{G}$ and $\mathbb{H}$ be two compact quantum groups and let $\mathbb{F}:=\mathbb{G}\times \mathbb{H}$ be the corresponding quantum direct product of $\mathbb{G}$ and $\mathbb{H}$. If $M$ is a torsion $\text{R}(\mathbb{G})$-module and $N$ is a torsion $\text{R}(\mathbb{H})$-module, then $M\boxtimes N$ is a torsion $\text{R}(\mathbb{F})$-module.
	\end{cor}
	
	Recall from the discussion at the end of Section \ref{sec.ModCTensorCat} that if $\mathscr{C}$ and $\mathscr{D}$ are semi-simple C$^*$-categories, then their Deligne's tensor product, $\mathscr{C}\boxtimes \mathscr{D}$, is again semi-simple with a complete set of irreducible objects given by $\{U_i\boxtimes V_j\}_{i\in I, j\in J}$, where $\{U_i\}_{i\in I}$ and $\{V_j\}_{j\in J}$ are complete sets of irreducible objects in $\mathscr{C}$ and $\mathscr{D}$, respectively. 
	
	This applies to $\mathscr{C}:=\mathscr{R}\text{ep}(\mathbb{G})$ and $\mathscr{D}:=\mathscr{R}\text{ep}(\mathbb{H})$ by virtue of the representation theory of $\mathbb{F}=\mathbb{G}\times \mathbb{H}$; so that $\mathscr{R}\text{ep}(\mathbb{G})\boxtimes \mathscr{R}\text{ep}(\mathbb{H})$ is again semi-simple with a complete set of irreducible objects given by $\{x\boxtimes y\}_{x\in\text{Irr}(\mathbb{G}), y\in\text{Irr}(\mathbb{H})}$, where $u^{x\boxtimes y}:=u^{(x,y)}=u^x_{13}u^y_{24}$. In other words, the objects in $\mathscr{R}\text{ep}(\mathbb{G})\boxtimes \mathscr{R}\text{ep}(\mathbb{H})$ are simply the exterior tensor products of representations of $\mathbb{G}$ with representations of $\mathbb{H}$. In particular, we have $\text{Obj}\Big(\mathscr{R}\text{ep}(\mathbb{G})\boxtimes \mathscr{R}\text{ep}(\mathbb{H})\Big)\subset \text{Obj}(\mathscr{R}\text{ep}(\mathbb{F}))$, so that $\mathscr{R}\text{ep}(\mathbb{F})$ is also a $\mathscr{R}\text{ep}(\mathbb{G})\boxtimes \mathscr{R}\text{ep}(\mathbb{H})$-module C$^*$-category (the module structure being simply the tensor product of representations of $\mathbb{F}$). Similarly, $\mathscr{R}\text{ep}(\mathbb{G})\boxtimes \mathscr{R}\text{ep}(\mathbb{H})$ is a $\mathscr{R}\text{ep}(\mathbb{F})$-module C$^*$-category. Observe as well that the tensor product defined on the Deligne's tensor product $\mathscr{R}\text{ep}(\mathbb{G})\boxtimes \mathscr{R}\text{ep}(\mathbb{H})$ (cf. Equation (\ref{eq.TensorProdDeligneTensor})) is precisely the tensor product of representations of $\mathbb{F}$ as computed in Remark \ref{rem.TensorProdRepsF}. Namely, if $x, x'\in\text{Irr}(\mathbb{G})$ and $y,y'\in\text{Irr}(\mathbb{H})$, then $u^{x\boxtimes y}\otimes_{\boxtimes}u^{x'\boxtimes y'}=(u^x\otimes u^{x'})\boxtimes (u^{y}\otimes u^{y'})=(u^x\otimes u^{x'})_{13} (u^{y}\otimes u^{y'})_{24}$. 
	 
	 Accordingly, another remarkable consequence of Proposition \ref{pro.TensorProdRepRingDirProd} is the following:
	\begin{cor}\label{cor.IdentificationRepCat}
		Let $\mathbb{G}$ and $\mathbb{H}$ be two compact quantum groups and let $\mathbb{F}:=\mathbb{G}\times \mathbb{H}$ be the corresponding quantum direct product of $\mathbb{G}$ and $\mathbb{H}$. Then  $\mathscr{R}\text{ep}(\mathbb{F})\cong \mathscr{R}\text{ep}(\mathbb{G})\boxtimes \mathscr{R}\text{ep}(\mathbb{H})$ as module C$^*$-categories. Moreover, $\mathscr{R}\text{ep}(\mathbb{F})\cong \mathscr{R}\text{ep}(\mathbb{G})\boxtimes \mathscr{R}\text{ep}(\mathbb{H})$ as C$^*$-tensor categories.
	\end{cor}
	\begin{proof}
		On the one hand, Proposition \ref{pro.TensorProdRepRingDirProd} yields an identification between $\text{R}(\mathbb{F})$ and $\text{R}(\mathbb{G})\boxtimes \text{R}(\mathbb{H})$ as based modules. On the other hand, it follows from the discussion prior to the statement that $\text{Fus}\big(\mathscr{R}\text{ep}(\mathbb{G})\boxtimes \mathscr{R}\text{ep}(\mathbb{H})\big)=\text{R}(\mathbb{G})\boxtimes \text{R}(\mathbb{H})$. Hence, \cite[Lemma 3.10]{YukiKenny} yields that $\mathscr{R}\text{ep}(\mathbb{F})\cong \mathscr{R}\text{ep}(\mathbb{G})\boxtimes \mathscr{R}\text{ep}(\mathbb{H})$ as module C$^*$-categories. Moreover, this equivalence also holds as C$^*$-tensor categories because the tensor structure on the Deligne's tensor product $\mathscr{R}\text{ep}(\mathbb{G})\boxtimes \mathscr{R}\text{ep}(\mathbb{H})$ is compatible with the tensor structure of $\mathscr{\mathscr{R}\text{ep}(\mathbb{F})}$ as observed above.
	\end{proof}
	
	\begin{rem}\label{rem.TorsionTensorProdCat}
	More generally, let $\mathscr{C}$ and $\mathscr{D}$ be semi-simple (rigid) C$^*$-categories. If $\mathscr{M}$ is a semi-simple $\mathscr{C}$-module C$^*$-category and $\mathscr{N}$ is a semi-simple $\mathscr{D}$-module C$^*$-category, then the Deligne's tensor product of the underlying C$^*$-categories, $\mathscr{M}\boxtimes \mathscr{N}$ is equipped with a structure of $\mathscr{C}\boxtimes \mathscr{D}$-module C$^*$-category denoted by $\bullet$ and such that:
	$$(U\boxtimes V)\bullet (X\boxtimes Y)=(U\bullet X)\boxtimes (V\bullet Y),$$
	for all $U\in\text{Obj}(\mathscr{C})$, $V\in\text{Obj}(\mathscr{D})$, $X\in \text{Obj}(\mathscr{M})$ and $Y\in\text{Obj}(\mathscr{N})$. As explained previously, $\mathscr{M}\boxtimes \mathscr{N}$ is again semi-simple with a complete set of irreducible objects given by $\{X_i\boxtimes Y_j\}_{i\in I, j\in J}$, where $\{X_i\}_{i\in I}$ and $\{Y_j\}_{j\in J}$ are complete sets of irreducible objects in $\mathscr{M}$ and $\mathscr{N}$, respectively. Accordingly, $\text{Fus}\big(\mathscr{M}\boxtimes \mathscr{N}\big)=\text{Fus}(\mathscr{M})\boxtimes \text{Fus}(\mathscr{N})$, which is a based $\text{Fus}\big(\mathscr{C}\boxtimes \mathscr{D}\big)=\text{Fus}(\mathscr{C})\boxtimes \text{Fus}(\mathscr{D})$-module. It follows from Corollary \ref{cor.TensorProdTorsMod} that if $\mathscr{M}$ and $\mathscr{N}$ are torsion (resp. co-finite/connected), then $\mathscr{M}\boxtimes \mathscr{N}$ is again torsion (resp. co-finite/connected). Also according to Remark \ref{rem.TorsionTensorProd}, if $\mathscr{M}\boxtimes \mathscr{N}$ is torsion (resp. co-finite/connected), then both $\mathscr{M}$ and $\mathscr{N}$ are torsion (resp. co-finite/connected).
	\end{rem}
	
	\subsection{Classification of torsion actions through module C$^*$-categories}\label{sec.ClassTorActDirectProd}
	Let $\mathbb{G}$ and $\mathbb{H}$ be two compact quantum groups and let $\mathbb{F}:=\mathbb{G}\times \mathbb{H}$ be the corresponding quantum direct product of $\mathbb{G}$ and $\mathbb{H}$. On the one hand, the description of $\text{Irr}(\mathbb{F})$ as recalled in the beginning of this section yields that $c_0(\widehat{\mathbb{F}})\cong c_0(\widehat{\mathbb{G}})\otimes c_0(\widehat{\mathbb{H}})$. On the other hand, given a $\widehat{\mathbb{G}}$-C$^*$-algebra $(A, \alpha)$ and a $\widehat{\mathbb{H}}$-C$^*$-algebra $(B, \beta)$, the tensor product $A\otimes B$ is a C$^*$-algebra equipped with the following action of $\widehat{\mathbb{F}}$:
	\begin{equation*}
			\begin{split}
				A\otimes B\overset{\alpha\otimes\beta}{\longrightarrow}&\widetilde{M}(c_0(\widehat{\mathbb{G}})\otimes A)\otimes \widetilde{M}(c_0(\widehat{\mathbb{H}})\otimes B)\subset \widetilde{M}(c_0(\widehat{\mathbb{G}})\otimes A\otimes c_0(\widehat{\mathbb{H}})\otimes B)\\
				&\overset{\Sigma}{\cong} \widetilde{M}(c_0(\widehat{\mathbb{G}})\otimes c_0(\widehat{\mathbb{H}})\otimes A\otimes B)\cong \widetilde{M}(c_0(\widehat{\mathbb{F}})\otimes A\otimes B).	
			\end{split}
		\end{equation*}
		
		By abuse of notation, we denote this composition simply by $\alpha\otimes\beta$. In a similar way, given a $\mathbb{G}$-C$^*$-algebra $(T, \tau)$ and a $\mathbb{H}$-C$^*$-algebra $(S, \sigma)$ we can still define the $\mathbb{F}$-C$^*$-algebra $(T\otimes S, \tau\otimes \sigma)$. In particular, it is clear that if $(T,\tau)$ is a torsion action of $\mathbb{G}$ and $(S, \sigma)$ is a torsion action of $\mathbb{H}$, then $(R, \rho)$ is a torsion action of $\mathbb{F}$. We aim to to show that every torsion action of $\mathbb{F}$ is of this form. In order to do so, we use the picture of module categories for torsion actions as recalled at the end of Section \ref{sec.TorsionQG}. Namely, we need to study tensor products of module C$^*$-categories.
		
		Let $(T, \tau)$ be a torsion action of $\mathbb{G}$ and let $(S, \sigma)$ be a torsion action of $\mathbb{H}$. Consider, the corresponding module C$^*$-categories, say $\mathscr{M}_{\tau}$ and $\mathscr{N}_\sigma$, which are torsion semi-simple $\mathscr{R}\text{ep}(\mathbb{G})$-module and $\mathscr{R}\text{ep}(\mathbb{H})$-module C$^*$-categories, respectively. We also have the torsion action $(T\otimes S, \tau\otimes \sigma)$ of $\mathbb{F}$. We denote by $\mathscr{P}_{\tau\otimes\sigma}$ the corresponding (torsion semi-simple) $\mathscr{R}\text{ep}(\mathbb{F})$-module C$^*$-category.
		\begin{rem}
			Note that if $(R, \rho)$ is any torsion action of $\mathbb{F}$, then the corresponding $\mathscr{R}\text{ep}(\mathbb{F})$-module C$^*$-category is also a $\mathscr{R}\text{ep}(\mathbb{G})\boxtimes \mathscr{R}\text{ep}(\mathbb{H})$-module C$^*$-category by means of the identification $\mathscr{R}\text{ep}(\mathbb{F})\cong \mathscr{R}\text{ep}(\mathbb{G})\boxtimes \mathscr{R}\text{ep}(\mathbb{H}))$ from Corollary \ref{cor.IdentificationRepCat}. Namely, if $(E, \delta_E)$ is any $\mathbb{F}$-equivariant Hilbert $R$-module, $u\in\text{Obj}\big(\mathscr{R}\text{ep}(\mathbb{G})\big)$ and $v\in\text{Obj}\big(\mathscr{R}\text{ep}(\mathbb{H})\big)$, then $(u\boxtimes v)\bullet E:= H_u\otimes H_v\otimes E$. 
		\end{rem}
		
		Recall that the objects of $\mathscr{M}_{\tau}$ are the $\mathbb{G}$-equivariant Hilbert $T$-modules, $(E, \delta_E)$. The $\mathscr{R}\text{ep}(\mathbb{G})$-module structure on $\mathscr{M}_{\tau}$ is given by $u\bullet E:= H_u\otimes E$, for all $u\in\text{Obj}\big(\mathscr{R}\text{ep}(\mathbb{G})\big)$. Similarly, the objects of $\mathscr{N}_\sigma$ are the $\mathbb{H}$-equivariant Hilbert $S$-modules, $(H, \delta_H)$; and the $\mathscr{R}\text{ep}(\mathbb{H})$-module structure on $\mathscr{N}_{\sigma}$ is given by $v\bullet H:= H_v\otimes H$, for all $v\in\text{Obj}\big(\mathscr{R}\text{ep}(\mathbb{H})\big)$. Given $(E, \delta_E)\in\text{Obj}(\mathscr{M}_\tau)$ and $(H, \delta_H)\in\text{Obj}(\mathscr{N}_{\sigma})$, the exterior tensor product of Hilbert modules $E\otimes H$ defines a Hilbert $T\otimes S$-module. Moreover, it is equipped with the action of $\mathbb{F}$ given by the following composition:
		\begin{equation*}
			\begin{split}
				E\otimes H\overset{\delta_E\otimes\delta_H}{\longrightarrow}E\otimes C(\mathbb{G})\otimes H\otimes C(\mathbb{H})\cong E\otimes F\otimes C(\mathbb{F}),
			\end{split}
		\end{equation*}
		which, by abuse of notation, we still denote by $\delta_E\otimes \delta_H$. Recall that $\text{Irr}(\mathscr{M}_\tau)$ is formed by $\mathbb{G}$-equivariant Hilbert $T$-submodules of $H_x\otimes T$ with $x\in\text{Irr}(\mathbb{G})$. Similarly, $\text{Irr}(\mathscr{N}_\sigma)$ is formed by $\mathbb{H}$-equivariant Hilbert $S$-submodules of $H_y\otimes S$ with $y\in\text{Irr}(\mathbb{H})$. So, if $(E, \delta_E)\in\text{Irr}(\mathscr{M}_\tau)$ and $(H, \delta_H)\in\text{Irr}(\mathscr{N}_\sigma)$, then there exist $x\in\text{Irr}(\mathbb{G})$ and $y\in\text{Irr}(\mathbb{H})$ such that $E\subset H_x\otimes T$ and $H\subset H_y\otimes S$. Accordingly, $E\otimes H\subset H_x\otimes H_y\otimes T\otimes S$. In particular, we have that $\mathcal{L}_{T\otimes S}(E\otimes H)=\mathcal{K}_{T\otimes S}(E\otimes H)\cong \mathcal{K}_T(E)\otimes \mathcal{K}_S(H)=\mathcal{L}_T(E)\otimes \mathcal{L}_S(H)$. As a consequence, $E\otimes H$ is again an irreducible $\mathbb{F}$-equivariant Hilbert $T\otimes S$-module arising as a $T\otimes S$-submodule of $H_z\otimes T\otimes S$ with $z:=(x,y)\in\text{Irr}(\mathbb{F})$. Finally, recall that both $\mathscr{M}_{\tau}$ and $\mathscr{N}_\sigma$ are semi-simple C$^*$-categories.
		
		All in all, the previous discussion shows that $\mathscr{M}_\tau\boxtimes\mathscr{N}_\sigma$ is the C$^*$-category whose objects are the $\mathbb{F}$-equivariant Hilbert $T\otimes S$-modules of the form $E\otimes H$ with $(E, \delta_E)\in\text{Obj}(\mathscr{M}_\tau)$ and $(H, \delta_H)\in\text{Obj}(\mathscr{N}_\sigma)$. Moreover, it is again semi-simple with $\{E\otimes H\}_{E\in\text{Irr}(\mathscr{M}_\tau), H\in\text{Irr}(\mathscr{N}_{\sigma})}$ as a complete set of irreducible objects (cf. Remark \ref{rem.TorsionTensorProdCat}). By the general theory, $\mathscr{M}_\tau\boxtimes\mathscr{N}_\sigma$ is equipped with a structure of $\mathscr{R}\text{ep}(\mathbb{G})\boxtimes \mathscr{R}\text{ep}(\mathbb{H})$-module C$^*$-category structure. But using the identification $\mathscr{R}\text{ep}(\mathbb{G})\boxtimes \mathscr{R}\text{ep}(\mathbb{H})\cong \mathscr{R}\text{ep}(\mathbb{F})$ (cf. Corollary \ref{cor.IdentificationRepCat}), we can regard $\mathscr{M}_\tau\boxtimes\mathscr{N}_\sigma$ as a $\mathscr{R}\text{ep}(\mathbb{F})$-module C$^*$-category by putting $w\bullet (E\otimes H):=H_w\otimes E\otimes H$, for all $w\in\text{Irr}(\mathbb{F})$, $(E, \delta_E)\in\text{Obj}(\mathscr{M}_\tau)$ and $(H, \delta_H)\in\text{Obj}(\mathscr{N}_\sigma)$. In other words, the identity functor is a fully faithful functor between $\mathscr{M}_\tau\boxtimes\mathscr{N}_\sigma$ and $\mathscr{P}_{\tau\otimes\sigma}$, so that we view $\mathscr{M}_\tau\boxtimes\mathscr{N}_\sigma$ as a (full) $\mathscr{R}\text{ep}(\mathbb{F})$-module C$^*$-subcategory of $\mathscr{P}_{\tau\otimes\sigma}$. Note that we can also regard $\mathscr{M}_\tau\boxtimes\mathscr{N}_\sigma$ as a (full) $\mathscr{R}\text{ep}(\mathbb{G})\boxtimes \mathscr{R}\text{ep}(\mathbb{H})$-module C$^*$-subcategory of $\mathscr{P}_{\tau\otimes\sigma}$. Actually, it is easy to see that the categories $\mathscr{M}_\tau\boxtimes\mathscr{N}_\sigma$ and $\mathscr{P}_{\tau\otimes\sigma}$ are equivalent as $\mathscr{R}\text{ep}(\mathbb{G})\boxtimes \mathscr{R}\text{ep}(\mathbb{H})$-module C$^*$-categories and as $\mathscr{R}\text{ep}(\mathbb{F})$-module C$^*$-categories. Indeed, observe that $\mathscr{M}_\tau\boxtimes\mathscr{N}_\sigma$ is a torsion $\mathscr{R}\text{ep}(\mathbb{G})\boxtimes \mathscr{R}\text{ep}(\mathbb{H})$-module C$^*$-category as a tensor product of torsion module C$^*$-categories (cf. Remark \ref{rem.TorsionTensorProdCat}). As such, the De Commer-Yamashita correspondence yields the existence of a (non-zero) torsion action $(R, \rho)$ of $\mathbb{F}$ such that $\mathscr{P}_{\rho}\cong \mathscr{M}_\tau\boxtimes\mathscr{N}_\sigma$. Therefore, by the previous discussion we can regard $\mathscr{P}_{\rho}$ as a (full) $\mathscr{R}\text{ep}(\mathbb{G})\boxtimes \mathscr{R}\text{ep}(\mathbb{H})$-module C$^*$-subcategory of $\mathscr{P}_{\tau\otimes\sigma}$. Since both $\mathscr{P}_{\rho}$ and $\mathscr{P}_{\tau\otimes\sigma}$ are in particular \emph{connected} $\mathscr{R}\text{ep}(\mathbb{G})\boxtimes \mathscr{R}\text{ep}(\mathbb{H})$-module C$^*$-subcategories and $\mathscr{P}_{\rho}\ncong (0)$, this inclusion must be an equivalence by virtue of Proposition \ref{pro.ConnectedIndecomposable}. The analogous argument applies also when the categories are viewed as $\mathscr{R}\text{ep}(\mathbb{F})$-module C$^*$-categories. In other words, $\mathscr{M}_\tau\boxtimes\mathscr{N}_\sigma\cong \mathscr{P}_{\tau\otimes\sigma}$ as claimed. We have thus proven the following:
		\begin{lem}\label{lem.TensorProdActionsModCat}
		Let $\mathbb{G}$, $\mathbb{H}$ be two compact quantum groups and let $\mathbb{F}:=\mathbb{G}\times \mathbb{H}$ be the corresponding quantum direct product of $\mathbb{G}$ and $\mathbb{H}$. Let $(T, \tau)$ be a torsion action of $\mathbb{G}$ and let $(S, \sigma)$ be a torsion action of $\mathbb{H}$. %Put $(R, \rho):=( T\otimes S, \tau\otimes \sigma)$. 
		Then $\mathscr{M}_\tau\boxtimes\mathscr{N}_\sigma\cong \mathscr{P}_{\tau\otimes\sigma}$, as $\mathscr{R}\text{ep}(\mathbb{G})\boxtimes \mathscr{R}\text{ep}(\mathbb{H})$-module C$^*$-categories and as $\mathscr{R}\text{ep}(\mathbb{F})$-module C$^*$-categories.
	\end{lem}
		
		
		Next, the finiteness feature of our framework (cf. Remark \ref{rem.SemiSimpleAssump}) allows to apply general results by V. Ostrik in module categories. More precisely, \cite[Theorem 1]{OstrikModCat} can be stated in the realm of module C$^*$-categories in the following way:
		\begin{theo}\label{theo.ModCatAlg}
			Let $\mathscr{C}$ be a rigid semi-simple C$^*$-tensor category. If $\mathscr{M}$ is a connected co-finite semi-simple $\mathscr{C}$-module C$^*$-category, then there exists a C$^*$-algebra object $\mathcal{A}\in\text{Obj}(\mathscr{C})$ such that $\mathscr{M}\cong \mathscr{M}\text{od}_{\mathcal{A}}(\mathscr{C})$ as $\mathscr{C}$-module C$^*$-categories.
		\end{theo}
		\begin{proof}
			Fix a non-zero irreducible object $X_0\in\text{Obj}(\mathscr{M})$. Then $\mathcal{A}:=\underline{\text{End}}(X_0)$ is the desired C$^*$-algebra object in $\mathscr{C}$. The same proof as in \cite[Theorem 1]{OstrikModCat} applies. Namely, the functor $\mathscr{M}\rightarrow \mathscr{M}\text{od}_{\mathcal{A}}(\mathscr{C})$, $Y\mapsto \underline{\text{Hom}}(X_0, Y)$, defines the equivalence of the statement (cf. Proposition \ref{pro.AlgFromInternal}).
		\end{proof}
		
		This theorem can be applied to our module categories $\mathscr{M}_{\tau}$ and $\mathscr{N}_\sigma$. More precisely, there exist representing C$^*$-algebra objects $\mathcal{A}$ in $\mathscr{R}\text{ep}(\mathbb{G})$ and $\mathcal{B}$ in $\mathscr{R}\text{ep}(\mathbb{H})$ such that $\mathscr{M}_\tau\cong \mathscr{M}\text{od}_{\mathcal{A}}(\mathscr{R}\text{ep}(\mathbb{G}))$ and $\mathscr{N}_\sigma\cong \mathscr{M}\text{od}_{\mathcal{B}}(\mathscr{R}\text{ep}(\mathbb{H}))$ as module C$^*$-categories. Moreover, by virtue of \cite[Proposition 3.9]{DouglasSchommerPriesSnyder} we know that: 
	$$\mathscr{M}\text{od}_{\mathcal{A}}(\mathscr{R}\text{ep}(\mathbb{G}))\boxtimes\mathscr{M}\text{od}_{\mathcal{B}}(\mathscr{R}\text{ep}(\mathbb{H}))\cong \mathscr{M}\text{od}_{\mathcal{A}\boxtimes \mathcal{B}}(\mathscr{R}\text{ep}(\mathbb{G})\boxtimes \mathscr{R}\text{ep}(\mathbb{H})),$$
	as (left) $(\mathscr{R}\text{ep}(\mathbb{G})\boxtimes \mathscr{R}\text{ep}(\mathbb{H}))$-module C$^*$-categories. Notice that this equivalence also holds as $\mathscr{R}\text{ep}(\mathbb{F})$-module C$^*$-categories by virtue of Corollary \ref{cor.IdentificationRepCat} and Proposition \ref{pro.AModTensorProd}. This observation, Lemma \ref{lem.TensorProdActionsModCat} and the De Commer-Yamashita correspondence between connected module C$^*$-categories and ergodic actions yield the classification of torsion actions for a quantum direct product in the following way.
		
		\begin{theo}\label{theo.TorsionDirectProd}
			Let $\mathbb{G}$, $\mathbb{H}$ be two compact quantum groups and let $\mathbb{F}:=\mathbb{G}\times \mathbb{H}$ be the corresponding quantum direct product of $\mathbb{G}$ and $\mathbb{H}$. If $(R, \rho)$ is a torsion action of $\mathbb{F}$, then there exist unique, up to equivariant Morita equivalence, torsion action of $\mathbb{G}$, say $(T, \tau)$, and torsion action of $\mathbb{H}$, say $(S, \sigma)$, such that $(R, \rho)\cong (T\otimes S, \tau\otimes \sigma)$ in $\mathscr{K}\mathscr{K}^{\mathbb{F}}$.
		\end{theo}
		\begin{proof}
			Given the torsion action $(R, \rho)$ of $\mathbb{F}$, consider the corresponding $\mathscr{R}\text{ep}(\mathbb{F})$-module C$^*$-category $\mathscr{P}_{\rho}$ of $\mathbb{F}$-equivariant Hilbert $R$-modules, which is torsion. Theorem \ref{theo.ModCatAlg} yields the existence of a C$^*$-algebra object $\mathcal{C}\in\text{Obj}\big(\mathscr{R}\text{ep}(\mathbb{F})\big)$ such that $\mathscr{P}_{\rho}\cong \mathscr{M}\text{od}_{\mathcal{C}}(\mathscr{R}\text{ep}(\mathbb{F}))$ as $\mathscr{R}\text{ep}(\mathbb{F})$-module C$^*$-categories. But using the identification $\mathscr{R}\text{ep}(\mathbb{F})\cong \mathscr{R}\text{ep}(\mathbb{G})\boxtimes \mathscr{R}\text{ep}(\mathbb{H})$ from Corollary \ref{cor.IdentificationRepCat}, $\mathscr{P}_{\rho}\cong \mathscr{M}\text{od}_{\mathcal{C}}(\mathscr{R}\text{ep}(\mathbb{F}))$ also holds as $\mathscr{R}\text{ep}(\mathbb{G})\boxtimes \mathscr{R}\text{ep}(\mathbb{H})$-module C$^*$-categories. Moreover, we deduce that there exist two C$^*$-algebra objects $\mathcal{A}\in\text{Obj}\big(\mathscr{R}\text{ep}(\mathbb{G})\big)$ and $\mathcal{B}\in\text{Obj}\big(\mathscr{R}\text{ep}(\mathbb{H})\big)$ such that $\mathcal{C}\cong \mathcal{A}\boxtimes \mathcal{B}$. Accordingly, we have: 
			\begin{equation}\label{eq.IdentficationModCat}
			\begin{split}
			\mathscr{M}\text{od}_{\mathcal{C}}(\mathscr{R}\text{ep}(\mathbb{F}))\cong \mathscr{M}\text{od}_{\mathcal{A}\boxtimes \mathcal{B}}(\mathscr{R}\text{ep}(\mathbb{G})\boxtimes \mathscr{R}\text{ep}(\mathbb{H}))\cong \mathscr{M}\text{od}_{\mathcal{A}}(\mathscr{R}\text{ep}(\mathbb{G}))\boxtimes\mathscr{M}\text{od}_{\mathcal{B}}(\mathscr{R}\text{ep}(\mathbb{H}))
			\end{split}
			\end{equation}
			as $\mathscr{R}\text{ep}(\mathbb{G})\boxtimes \mathscr{R}\text{ep}(\mathbb{H})$-module C$^*$-categories, where the first equivalence follows from Proposition \ref{pro.AModTensorProd} and the second one from \cite[Proposition 3.9]{DouglasSchommerPriesSnyder}. Since $\mathscr{P}_{\rho}\cong \mathscr{M}\text{od}_{\mathcal{C}}(\mathscr{R}\text{ep}(\mathbb{F}))$ and $\mathscr{P}_{\rho}$ is in particular a torsion $\mathscr{R}\text{ep}(\mathbb{G})\boxtimes \mathscr{R}\text{ep}(\mathbb{H})$-module C$^*$-category, then identification in Equation (\ref{eq.IdentficationModCat}) yields that $\mathscr{M}\text{od}_{\mathcal{A}}(\mathscr{R}\text{ep}(\mathbb{G}))$ is a torsion $\mathscr{R}\text{ep}(\mathbb{G})$-module C$^*$-category and that $\mathscr{M}\text{od}_{\mathcal{B}}(\mathscr{R}\text{ep}(\mathbb{H}))$ is a torsion $\mathscr{R}\text{ep}(\mathbb{H})$-module C$^*$-category (cf. Remark \ref{rem.TorsionTensorProdCat}). As such, the De Commer-Yamashita correspondence yields the existence of a torsion action $(T, \tau)$ of $\mathbb{G}$ and a torsion action $(S, \sigma)$ of $\mathbb{H}$ such that $\mathscr{M}_{\tau}\cong \mathscr{M}\text{od}_{\mathcal{A}}(\mathscr{R}\text{ep}(\mathbb{G}))$ and $\mathscr{N}_\sigma\cong \mathscr{M}\text{od}_{\mathcal{B}}(\mathscr{R}\text{ep}(\mathbb{H}))$ as module C$^*$-categories. These torsion actions are unique up to equivariant Morita equivalence.
			
			All in all, we have obtained that $\mathscr{P}_{\rho}\cong \mathscr{M}_{\tau}\boxtimes \mathscr{N}_{\sigma}$ as $\mathscr{R}\text{ep}(\mathbb{G})\boxtimes \mathscr{R}\text{ep}(\mathbb{H})$-module C$^*$-categories. Again by virtue of Corollary \ref{cor.IdentificationRepCat} and Proposition \ref{pro.AModTensorProd}, the equivalences in Equation (\ref{eq.IdentficationModCat}) also hold as $\mathscr{R}\text{ep}(\mathbb{F})$-module C$^*$-categories. Therefore, we also obtain that $\mathscr{P}_{\rho}\cong \mathscr{M}_{\tau}\boxtimes \mathscr{N}_{\sigma}$ as $\mathscr{R}\text{ep}(\mathbb{F})$-module C$^*$-categories. By Lemma \ref{lem.TensorProdActionsModCat} we also know that $\mathscr{M}_{\tau}\boxtimes \mathscr{N}_{\sigma}\cong\mathscr{P}_{\tau\otimes\sigma}$ as $\mathscr{R}\text{ep}(\mathbb{F})$-module C$^*$-categories. Hence, the De Commer-Yamashita correspondence yields that $(R, \rho)\underset{\mathbb{F}}{\sim} (T\otimes S, \tau\otimes\sigma)$, which concludes the proof.
		\end{proof}

\begin{comment}
	Theorem \ref{theo.TorsionDirectProd} above says that any torsion action of $\mathbb{F}$ is equivariantly Morita equivalent to a tensor product of a torsion action of $\mathbb{G}$ by a torsion action of $\mathbb{H}$. This result is enough for our purpose in Section \ref{sec.KunnethFunctors} because we will work within equivariant Kasparov categories, where an equivariant Morita equivalence becomes  an isomorphism. However, we would like to obtain the classification of Theorem \ref{theo.TorsionDirectProd} at the level of C$^*$-algebras, i.e. our goal is now to show that any torsion action of $\mathbb{F}$ is $*$-isomorphic to a tensor product of a torsion action of $\mathbb{G}$ by a torsion action of $\mathbb{H}$.
	
	Nevertheless, the discussion carried out in this section and Theorem \ref{theo.TorsionDirectProd} in particular, yield a classification result at the level of module C$^*$-categories, which is of independent interest. Namely, it follows that any torsion $\mathscr{R}\text{ep}(\mathbb{F})$-module C$^*$-category is equivalent to $\mathscr{M}\boxtimes\mathscr{N}$, where $\mathscr{M}$ is a torsion $\mathscr{R}\text{ep}(\mathbb{G})$-module C$^*$-category and $\mathscr{N}$ is a torsion $\mathscr{R}\text{ep}(\mathbb{H})$-module C$^*$-category. Equivalently (cf. \cite[Lemma 3.10]{YukiKenny}), every torsion $\text{R}(\mathbb{F})$-module is isomorphic to $M\boxtimes N$, where $M$ is a torsion $\text{R}(\mathbb{G})$-module and $N$ is a torsion $\text{R}(\mathbb{H})$-module. The latter provides the converse to Corollary \ref{cor.TensorProdTorsMod}. \textcolor{red}{what about general $\mathscr{C}$-module C$^*$-categories?}.
	
	\begin{theo}
		Let $\mathbb{G}$, $\mathbb{H}$ be two compact quantum groups and let $\mathbb{F}:=\mathbb{G}\times \mathbb{H}$ be the corresponding quantum direct product of $\mathbb{G}$ and $\mathbb{H}$. If $(R, \rho)$ is a torsion action of $\mathbb{F}$, then there exist unique, up to $*$-isomorphism, torsion action of $\mathbb{G}$, say $(T, \tau)$, and torsion action of $\mathbb{H}$, say $(S, \sigma)$, such that $(R, \rho)\cong (T\otimes S, \tau\otimes \sigma)$.
	\end{theo}
	\begin{proof}
	Let $(R, \rho)$ be a torsion action of $\mathbb{F}$, which means that $(R, \rho)$ is a finite dimensional ergodic $\mathbb{F}$-C$^*$-algebra. By Example \ref{ex.AlgObjFromGAlg} we know that $L^2(R)$ defines an (ergodic) C$^*$-algebra object in $\mathscr{R}\text{ep}(\mathbb{F})$. In particular, $L^2(R)$ is viewed as a finite dimensional unitary representation of $\mathbb{F}$ (cf. Remark \ref{rem.RepGEquivHilbSpace}). As such, it can be written as an exterior tensor product of a finite dimensional unitary representation of $\mathbb{G}$ by a finite dimensional unitary representation of $\mathbb{H}$, say $\mathcal{T}$ and $\mathcal{S}$, respectively. So $L^2(R)\cong \mathcal{T}\boxtimes \mathcal{S}$. By virtue of Corollary \ref{cor.IdentificationRepCat} this identification also holds as (ergodic) C$^*$-algebra objects in $\mathscr{R}\text{ep}(\mathbb{F})\cong \mathscr{R}\text{ep}(\mathbb{G})\boxtimes \mathscr{R}\text{ep}(\mathbb{H})$. Using again Example \ref{ex.AlgObjFromGAlg}, $\mathcal{T}$ and $\mathcal{S}$ define torsion actions of $\mathbb{G}$ and $\mathbb{H}$, say $(T, \tau)$ and $(S, \sigma)$, respectively such that $(R, \rho)\cong (T\otimes S, \tau\otimes\sigma)$.
	\end{proof}
\end{comment}
	
%\begin{comment}	
%Still with this approach, we cannot identify the spectral subspace R_z with R_x\otimes R_y!! And we run into the same troubles as originally tried. But......... it works by finite dimensionality!
\subsection{Classification of torsion actions through spectral theory}\label{sec.ClassTorActDirectProd2}
	In this section, we give an alternative proof of the classification of torsion actions of a quantum direct product discussed previously. For this we will make use of the spectral theory for compact quantum groups (cf. Section \ref{sec.ActionsQG}). 
	
	Let $\mathbb{G}$ and $\mathbb{H}$ be two compact quantum groups and let $\mathbb{F}:=\mathbb{G}\times \mathbb{H}$ be the corresponding quantum direct product of $\mathbb{G}$ and $\mathbb{H}$. First of all, Theorem \ref{theo.TorsionDirectProd} says that any torsion action of $\mathbb{F}$ is equivariantly Morita equivalent to a tensor product of a torsion action of $\mathbb{G}$ by a torsion action of $\mathbb{H}$. This result is enough for our purpose in Section \ref{sec.KunnethFunctors} because we will work within equivariant Kasparov categories, where an equivariant Morita equivalence becomes  an isomorphism. However, we would like to obtain the classification of Theorem \ref{theo.TorsionDirectProd} at the level of C$^*$-algebras, i.e. our goal is now to show that any torsion action of $\mathbb{F}$ is $*$-isomorphic to a tensor product of a torsion action of $\mathbb{G}$ by a torsion action of $\mathbb{H}$.
	
	Nevertheless, the discussion carried out in Section \ref{sec.ClassTorActDirectProd} and Theorem \ref{theo.TorsionDirectProd} in particular, yield a classification result at the level of module C$^*$-categories, which is of independent interest. Namely, it follows that any torsion $\mathscr{R}\text{ep}(\mathbb{F})$-module C$^*$-category is equivalent to $\mathscr{M}\boxtimes\mathscr{N}$, where $\mathscr{M}$ is a torsion $\mathscr{R}\text{ep}(\mathbb{G})$-module C$^*$-category and $\mathscr{N}$ is a torsion $\mathscr{R}\text{ep}(\mathbb{H})$-module C$^*$-category. Equivalently (cf. \cite[Lemma 3.10]{YukiKenny}), every torsion $\text{R}(\mathbb{F})$-module is isomorphic to $M\boxtimes N$, where $M$ is a torsion $\text{R}(\mathbb{G})$-module and $N$ is a torsion $\text{R}(\mathbb{H})$-module. The latter provides the converse of Corollary \ref{cor.TensorProdTorsMod}. %\textcolor{red}{what about general $\mathscr{C}$-module C$^*$-categories?}.

	\bigskip
	Let $(R, \rho)$ be a torsion action of $\mathbb{F}$, which means that $(R, \rho)$ is a finite dimensional ergodic $\mathbb{F}$-C$^*$-algebra. Consider the corresponding spectral decomposition (cf. Section \ref{sec.ActionsQG}): 
	\begin{equation}\label{eq.SpectDecomR}
	\begin{split}
		R=\mathcal{R}_{\mathbb{F}}=\underset{z\in \text{Irr}(\mathbb{F})}{\bigoplus} \mathcal{R}_z=\underset{(x, y)\in \text{Irr}(\mathbb{G})\times \text{Irr}(\mathbb{H})}{\bigoplus} \mathcal{R}_{(x, y)}.
	\end{split}
	\end{equation}
	
	Note that $\mathcal{R}_{(\epsilon_{\mathbb{G}}, \epsilon_{\mathbb{H}})}=R^\rho=\mathbb{C}$. By viewing $\text{Irr}(\mathbb{G}) \subset \text{Irr}(\mathbb{F})$ and $\text{Irr}(\mathbb{H}) \subset \text{Irr}(\mathbb{F})$, we consider as well the following C$^*$-subalgebras of $R$: 
	$$T:=\underset{x\in \text{Irr}(\mathbb{G})}{\bigoplus} \mathcal{R}_{(x, \epsilon_{\mathbb{H}})}\subset R \mbox{ and } S:=\underset{y\in \text{Irr}(\mathbb{H})}{\bigoplus}\ \mathcal{R}_{(\epsilon_{\mathbb{G}}, y)}\subset R.$$
	
	It is clear that $\rho(T)\subset T\otimes C(\mathbb{G})$ and $\rho(S)\subset S\otimes C(\mathbb{H})$. In other words, we obtain actions of $\mathbb{G}$ and $\mathbb{H}$, respectively. Moreover, since $\rho$ is ergodic, $\rho_{|T}$ and $\rho_{|S}$ are ergodic too. Hence, we actually obtain torsion actions of $\mathbb{G}$ and $\mathbb{H}$, respectively. We denote them by $(T, \tau)$ and $(S, \sigma)$, respectively; where $\tau:=\rho_{|T}$ and $\sigma:=\rho_{|S}$. We are going to show that $(R, \rho)\cong (T\otimes S, \tau\otimes \sigma)$.
	
	Given the torsion action $(R, \rho)$, we know from Example \ref{ex.AlgObjFromGAlg} that $L^2(R)$ defines an (ergodic) C$^*$-algebra object in $\mathscr{R}\text{ep}(\mathbb{F})$. In particular, $L^2(R)$ is viewed as a finite dimensional unitary representation of $\mathbb{F}$ (cf. Remark \ref{rem.RepGEquivHilbSpace}). As such, it can be written as an exterior tensor product of a finite dimensional unitary representation of $\mathbb{G}$ by a finite dimensional unitary representation of $\mathbb{H}$, say $\mathcal{T}'$ and $\mathcal{S}'$, respectively. So $L^2(R)\cong \mathcal{T}'\boxtimes \mathcal{S}'$. By virtue of Corollary \ref{cor.IdentificationRepCat} this identification also holds as (ergodic) C$^*$-algebra objects in $\mathscr{R}\text{ep}(\mathbb{F})\cong \mathscr{R}\text{ep}(\mathbb{G})\boxtimes \mathscr{R}\text{ep}(\mathbb{H})$. Using again Example \ref{ex.AlgObjFromGAlg}, $\mathcal{T}'$ and $\mathcal{S}'$ define torsion actions of $\mathbb{G}$ and $\mathbb{H}$, say $(T', \tau')$ and $(S', \sigma')$, respectively such that $(R, \rho)\cong (T'\otimes S', \tau'\otimes\sigma')$. In this situation, the spectral decomposition of $R$ takes the form: 
	\begin{equation}\label{eq.SpectDecomR2}
	\begin{split}
		R=\mathcal{R}_{\mathbb{F}}\cong\mathcal{T}'_{\mathbb{G}}\otimes \mathcal{S}'_{\mathbb{H}}=\Big(\underset{x\in \text{Irr}(\mathbb{G})}{\bigoplus} \mathcal{T}'_x\Big)\otimes \Big(\underset{y\in \text{Irr}(\mathbb{H})}{\bigoplus} \mathcal{S}'_y\Big)\cong\underset{y\in \text{Irr}(\mathbb{H})}{\underset{x\in \text{Irr}(\mathbb{G})}{\bigoplus}} \mathcal{T}'_x\otimes \mathcal{S}'_y.
	\end{split}
	\end{equation}
	
	Comparing Equation (\ref{eq.SpectDecomR}) with Equation (\ref{eq.SpectDecomR2}), we deduce that $\mathcal{R}_z=\mathcal{R}_{(x,y)}\cong \mathcal{T}'_x\otimes \mathcal{S}'_y$, for all $x\in \text{Irr}(\mathbb{G})$ and $y\in \text{Irr}(\mathbb{H})$ such that $z=(x,y)$. In particular, we have $\mathcal{R}_{(x,\epsilon_{\mathbb{H}})}\cong \mathcal{T}'_x$ and $\mathcal{R}_{(\epsilon_{\mathbb{G}},y)}\cong \mathcal{S}'_y$, for all $x\in \text{Irr}(\mathbb{G})$ and $y\in \text{Irr}(\mathbb{H})$. It follows that $(T,\tau)\cong (T', \tau')$ and $(S, \sigma)\cong (S', \sigma')$. Hence, we have proven the following:
	\begin{theo}\label{theo.TorsionDirectProd2}
		Let $\mathbb{G}$, $\mathbb{H}$ be two compact quantum groups and let $\mathbb{F}:=\mathbb{G}\times \mathbb{H}$ be the corresponding quantum direct product of $\mathbb{G}$ and $\mathbb{H}$. If $(R, \rho)$ is a torsion action of $\mathbb{F}$, then there exist unique, up to $*$-isomorphism, torsion action of $\mathbb{G}$, say $(T, \tau)$, and torsion action of $\mathbb{H}$, say $(S, \sigma)$, such that $(R, \rho)\cong (T\otimes S, \tau\otimes \sigma)$.
	\end{theo}
	
	\begin{rem}
		Notice that the proof above can be simplified by avoiding the theory of C$^*$-algebra objects. Instead, we use Proposition \ref{pro.FactsSpect}. Specifically, by Proposition \ref{pro.FactsSpect} we know that the spectral subspaces $\mathcal{R}_z= \mathcal{R}_{(x, y)}$ are (equivalent to) unitary representations of $\mathbb{F}$, for all $z=(x,y)\in\text{Irr}(\mathbb{F})$. As such, there exist a unitary representation $\mathcal{T}'_x$ for $\mathbb{G}$ and a unitary representation $\mathcal{S}'_y$ for $\mathbb{H}$ such that $\mathcal{R}_{(x, y)}\cong\mathcal{T}'_x\otimes \mathcal{S}'_y$, for all $x\in \text{Irr}(\mathbb{G})$ and $y\in \text{Irr}(\mathbb{H})$. Then the same argument as above applies.
	\end{rem}

	%By construction, it is enough to show this at the level of spectral subspaces, i.e. we are going to show that $\mathcal{R}_y\cong \mathcal{R}_{(x, \epsilon_{\mathbb{H}})}\otimes \mathcal{R}_{(\epsilon_{\mathbb{G}}, z)}$, for all $y=(x,z)\in \text{Irr}(\mathbb{F})$. Given $x\in\text{Irr}(\mathbb{G})$ and $z\in\text{Irr}(\mathbb{H})$, consider the map $(H_x\otimes T)\times (H_z\otimes S)\longrightarrow H_{x}\otimes H_{z}\otimes R$, $X\otimes Z\mapsto X_{13}Z_{24}$. A straigtforward computation (e.g. using a coordinate expression as previously) shows that this map is $R^\rho$-bilinear (note that $R^\rho=K_{(\epsilon_\mathbb{G}, \epsilon_{\mathbb{H}})}\subset T\cap S$). Moreover, this maps sends $K_x\times K_z$ to $K_{(x,z)}$ and it is isometric.
%\end{comment}	

		
	\subsection{Quantum direct products and crossed products}
		Observe that the quantum direct product construction can be done more generally for locally compact quantum groups too. In this respect, the following is an easy observation (see \cite[Corollary 2.3.3]{RubenSemiDirect} for a proof in the case of discrete quantum groups).
	\begin{proSec}\label{pro.DirectProductTensorProduct}
		Let $\mathbb{G}$, $\mathbb{H}$ be two locally compact quantum groups and let $\mathbb{F}:=\mathbb{G}\times \mathbb{H}$ be the corresponding quantum direct product of $\mathbb{G}$ and $\mathbb{H}$. If $(A,\alpha)$ is a $\mathbb{G}$-C$^*$-algebra and $(B,\beta)$ is a $\mathbb{H}$-C$^*$-algebra, then there exists a canonical $*$-isomorphism $C \underset{\delta,r}{\rtimes}\mathbb{F} \cong A\underset{\alpha,r}{\rtimes}\mathbb{G} \otimes B\underset{\beta,r}{\rtimes} \mathbb{H}\mbox{,}$ where $C:=A\otimes B$ is the $\mathbb{F}$-C$^*$-algebra with action $\delta:=\alpha\otimes \beta$.
	\end{proSec}
	\begin{proof}
		The isomorphism of the statement is simply induced by the canonical map $\mathcal{L}_A(L^2(\mathbb{G})\otimes A)\otimes \mathcal{L}_B(L^2(\mathbb{H})\otimes B)\rightarrow \mathcal{L}_{A\otimes B}\big(L^2(\mathbb{G})\otimes L^2(\mathbb{H})\otimes A\otimes B\big)=\mathcal{L}_{A\otimes B}(L^2(\mathbb{F})\otimes C)$.
	\end{proof}
	
	In relation with the two-sided crossed product construction recalled in Section \ref{sec.NotationsConventions}, the following result (which is a generalisation of Proposition \ref{pro.DirectProductTensorProduct}) is straightforward after a routine computation by applying the definitions and the fact that $c_0(\widehat{\mathbb{F}})=c_0(\widehat{\mathbb{G}})\otimes c_0(\widehat{\mathbb{H}})$, where $\mathbb{F}:=\mathbb{G}\times\mathbb{H}$.
	\begin{proSec}\label{pro.TwoSidedDirectProd}
	Let $\mathbb{G}$ and $\mathbb{H}$ be two compact quantum groups and let $\mathbb{F}:=\mathbb{G}\times\mathbb{H}$ be the corresponding quantum direct product of $\mathbb{G}$ and $\mathbb{H}$. Let $(A,\alpha)$ and $(B,\beta)$ be a right $\mathbb{G}$-C$^*$-algebra and a right $\mathbb{H}$-C$^*$-algebra, respectively. Let $(T, \tau)$ and $(S, \sigma)$ be a left $\mathbb{G}$-C$^*$-algebra and a left $\mathbb{H}$-C$^*$-algebra. Then $(A\otimes B)\underset{r, \alpha\otimes\beta}{\rtimes}\mathbb{F}\underset{r, \tau\otimes \sigma}{\ltimes}(T\otimes S)\cong (A\underset{r, \alpha}{\rtimes}\mathbb{G}\underset{r, \tau}{\ltimes}T)\otimes (B\underset{r, \beta}{\rtimes}\mathbb{H}\underset{r, \sigma}{\ltimes}S)$.
	\end{proSec}	
	
	
	
	
	
	
	
	
	
	
\section{\textsc{The Künneth classes}}\label{sec.KunnethFunctors}
	We start by recalling some terminology introduced in \cite{ChabertEchterhoffOyono} in the context of the (abstract) Künneth theorem.
	\begin{defiSec}\label{defi.KunnethCategory}
		A Künneth category is a subcategory $\mathcal{S}$ of $\text{C}^*\text{-Alg}$ such that:
		\begin{enumerate}[i)]
			\item $\mathcal{S}$ contains all separable commutative C$^*$-algebras.
			\item $\mathcal{S}$ is closed under stabilization, i.e. if $B\in \text{Obj}(\mathcal{S})$, then $\mathcal{K}\otimes B\in \text{Obj}(\mathcal{S})$.
			\item $\mathcal{S}$ is closed under suspension, i.e. if $B\in \text{Obj}(\mathcal{S})$, then $\Sigma(B)\in \text{Obj}(\mathcal{S})$ are objects in $\mathcal{S}$.
			\item If $0\rightarrow J\rightarrow B\rightarrow B/J\rightarrow 0$ is a semi-split short exact sequence in $\mathcal{S}$ such that two of the C$^*$-algebras are in $\mathcal{S}$, the so is the third.
		\end{enumerate}
	\end{defiSec}
	
	\begin{defiSec}\label{defi.KunnethFunctor}
		Let $\mathcal{S}$ be a Künneth category. A Künneth functor on $\mathcal{S}$ is an additive convariant functor $\kappa_*: \mathcal{S}\longrightarrow \mathscr{A}\text{b}^{\mathbb{Z}/2}$ such that:
		\begin{enumerate}[i)]
			\item $\kappa_*$ is invariant under stabilization and under homotopy.
			\item If $0\rightarrow J\rightarrow B\rightarrow B/J\rightarrow 0$ is a semi-split short exact sequence in $\mathcal{S}$, then the sequence $\kappa_*(J)\rightarrow \kappa_*(B)\rightarrow \kappa_*(B/J)$ is exact.
			\item $\kappa_*$ is stable, i.e. $\kappa_*(\Sigma(B))\cong \kappa_{*+1}(B)$, for all $B\in \text{Obj}(\mathcal{S})$.
			\item There exists a natural zero-graded homomorphism $\alpha: \kappa_*(\mathbb{C})\otimes K_*(B)\rightarrow \kappa_*(B)$, for all $B\in \text{Obj}(\mathcal{S})$ such that $\alpha$ is an isomorphism whenever $K_*(B)$ is free abelian.
		\end{enumerate}
	\end{defiSec}
	
	\begin{theoSec}\label{theo.AbstractKunnethTheorem}
		Let $\mathcal{S}$ be a Künneth category. If $\kappa_*$ is a Künneth functor on $\mathcal{S}$, then there exists a natural $1$-graded homomorphism $\beta: \kappa_*(B)\rightarrow \text{Tor}(\kappa_*(\mathbb{C}), K_*(B))$, for all $B\in \text{Obj}(\mathcal{S})$ such that the sequence:
		$$0\longrightarrow\kappa_*(\mathbb{C})\otimes K_*(B)\overset{\alpha}{\longrightarrow} \kappa_*(B)\overset{\beta}{\longrightarrow} \text{Tor}(\kappa_*(\mathbb{C}), K_*(B))\longrightarrow 0$$
		is exact. This sequence is called \emph{Künneth sequence for $\kappa_*$}.
	\end{theoSec}
	
	\begin{exsSec}\label{exs.KunnethClasses}
		\begin{enumerate}
			\item Let $A$ and $B$ be two C$^*$-algebras. The external Kasparov product: $$KK(\mathbb{C}, A)\times KK(\mathbb{C}, B)\overset{\widetilde{\tau}_{\mathbb{C}}}{\longrightarrow} KK(\mathbb{C}, A\otimes B)$$
			yields a group homomorphism $\alpha: K_*(A)\otimes K_*(B)\rightarrow K_*(A\otimes B)$, which is natural on $A$ and $B$. We denote by $\mathcal{N}$ the class of all C$^*$-algebras $A$ in $\text{C}^*\text{-Alg}$ such that $\alpha$ is an isomorphism for all separable C$^*$-algebra $B$ with $K_*(B)$ free abelian. By abuse of notation, we still denote by $\mathcal{N}$ the full subcategory of $\text{C}^*\text{-Alg}$ generated by the class $\mathcal{N}$. It is well-known that $\mathcal{N}$ is a Künneth category in the sense of Definition \ref{defi.KunnethCategory} \cite{ChabertEchterhoffOyono}. Moreover, if $A\in\mathcal{N}$, it is clear that the map $B\mapsto K_*(A\otimes B)$ defines a functor on the whole category $\text{C}^*\text{-Alg}$ satisfying properties $(i)-(iv)$ from Definition \ref{defi.KunnethFunctor} \cite{ChabertEchterhoffOyono}. In other words, this map defines a Künneth functor. Consequently, Theorem \ref{theo.AbstractKunnethTheorem} yields that the Künneth sequence:
			$$0\longrightarrow K_*(A)\otimes K_*(B)\overset{\alpha}{\longrightarrow} K_*(A\otimes B)\overset{\beta}{\longrightarrow} \text{Tor}(K_*(A), K_*(B))\longrightarrow 0$$
			is exact for all $A\in\mathcal{N}$ and all $B\in \text{Obj}(\text{C}^*\text{-Alg})$. It is important to mention that the class $\mathcal{N}$ contains the bootstrap class and it satisfies the following remarkable stability properties (cf. \cite[Lemma 4.4]{ChabertEchterhoffOyono}):
			\begin{lemSec}\label{lem.StabilityClassN}
				$\mathcal{N}$ contains the bootstrap class and the following properties hold.
				\begin{enumerate}[i)]
					\item $\mathcal{N}$ is stable under $KK$-equivalence hence it is stable under Morita equivalence. 
					\item If $0\rightarrow I\rightarrow A\rightarrow A/I\rightarrow$ is a semi-split short exact sequence of C$^*$-algebras such that two of them are in $\mathcal{N}$, then so is the third.
					\item If $A, B\in\mathcal{N}$, then $A\otimes B\in\mathcal{N}$.
					\item If $A=\underset{\rightarrow}{\lim}\ A_i$ such that all structure maps are injective and $A_i\in\mathcal{N}$ for all $i$, then $A\in\mathcal{N}$.
				\end{enumerate}
			\end{lemSec}
			\item Let $G$ be a locally compact group. Let $A$ be a $G$-C$^*$-algebra and $B$ a C$^*$-algebra. Observe that one has an obvious map $KK(\mathbb{C}, B)\rightarrow KK^G(\mathbb{C}, B)$, which consists in equipping with the trivial action of $G$. Then, for all $G$-compact subspace $X\subset \underline{E}G$, the external Kasparov product: 
			$$KK^G(C_0(X), A)\times KK(\mathbb{C}, B)\longrightarrow KK^G(C_0(X), A)\times KK^G(\mathbb{C}, B)\overset{\widetilde{\tau}_{\mathbb{C}}}{\longrightarrow} KK(C_0(X), A\otimes B)$$
			yields a group homomorphism $\alpha_X: KK^G(C_0(X), A)\otimes K_*(B)\rightarrow KK^G(C_0(X), A\otimes B)$, for all $G$-compact subspace $X\subset \underline{E}G$, which is natural on $A$ and $B$. One checks that the maps $\{\alpha_X\}_{\underset{\text{$G$-compact}}{X\subset \underline{E}G}}$ are compatible with the inclusion of $G$-compact spaces. Hence, one obtains a group homomorphism $\alpha_G: K^{\text{top}}_*(G; A)\otimes K_*(B)\rightarrow K^{\text{top}}_*(G; A\otimes B)$, which is natural on $A$ and $B$. We denote by $\mathcal{N}_G$ the class of all $G$-C$^*$-algebras $A$ in $G$-$\text{C}^*\text{-Alg}$ such that $\alpha_G$ is an isomorphism for all separable C$^*$-algebra $B$ with $K_*(B)$ free abelian. By abuse of notation, we still denote by $\mathcal{N}_G$ the full subcategory of $G$-$\text{C}^*\text{-Alg}$ generated by the class $\mathcal{N}_G$. It is well-known that $\mathcal{N}_G$ is a Künneth category in the sense of Definition \ref{defi.KunnethCategory} \cite{ChabertEchterhoffOyono}. Moreover, if $A\in\mathcal{N}_G$, it is clear that the map $B\mapsto K^{\text{top}}_*(G; A\otimes B)$ defines a functor on the whole category $\text{C}^*\text{-Alg}$ satisfying properties $(i)-(iv)$ from Definition \ref{defi.KunnethFunctor} \cite{ChabertEchterhoffOyono}. In other words, this map defines a Künneth functor. Consequently, Theorem \ref{theo.AbstractKunnethTheorem} yields that the Künneth sequence:
			$$0\longrightarrow K^{\text{top}}_*(G; A)\otimes K_*(B)\overset{\alpha_G}{\longrightarrow} K^{\text{top}}_*(G; A\otimes B)\overset{\beta_G}{\longrightarrow} \text{Tor}(K^{\text{top}}_*(G; A), K_*(B))\longrightarrow 0$$
			is exact for all $A\in\mathcal{N}_G$ and all $B\in \text{Obj}(\text{C}^*\text{-Alg})$.
		\end{enumerate}
	\end{exsSec}
	
	
	\subsection{Preparatory observations}\label{sec.PrepKunneth}
	In order to provide examples of Künneth functors in the framework of quantum groups, let us recall some constructions appearing in \cite{RubenSemiDirect}. Let $\mathbb{G}$ and $\mathbb{H}$ be two compact quantum groups. We put $\mathbb{F}:=\mathbb{G}\times\mathbb{H}$ for the corresponding quantum direct product \cite{WangSemidirect}. Consider the complementary pair of localizing subcategories in $\mathscr{K}\mathscr{K}^{\widehat{\mathbb{F}}}$, $\mathscr{K}\mathscr{K}^{\widehat{\mathbb{G}}}$ and $\mathscr{K}\mathscr{K}^{\widehat{\mathbb{H}}}$: $(\mathscr{L}_{\widehat{\mathbb{F}}}\mathscr{N}_{\widehat{\mathbb{F}}})$, $(\mathscr{L}_{\widehat{\mathbb{G}}}\mathscr{N}_{\widehat{\mathbb{G}}})$ and $(\mathscr{L}_{\widehat{\mathbb{H}}}\mathscr{N}_{\widehat{\mathbb{H}}})$, respectively; as explained in Section \ref{sec.QuantumBC}. Accordingly, the canonical triangulated functors associated to these complementary pairs are denoted by $(L,N)$, $(L', N')$ and $(L'', N'')$, respectively. Consider the homological functors defining the \emph{quantum} Baum-Connes assembly maps for $\widehat{\mathbb{F}}$, $\widehat{\mathbb{G}}$ and $\widehat{\mathbb{H}}$:
	$$F:\mathscr{K}\mathscr{K}^{\widehat{\mathbb{F}}} \rightarrow \mathscr{A}b^{\mathbb{Z}/2}\mbox{, } (C,\delta) \mapsto F(C):=K_{*}(C\underset{\delta, r}{\rtimes} \widehat{\mathbb{F}}),$$
	$$F':\mathscr{K}\mathscr{K}^{\widehat{\mathbb{G}}} \rightarrow \mathscr{A}b^{\mathbb{Z}/2}\mbox{, } (A,\alpha) \mapsto F'(A):=K_{*}(A\underset{\alpha,r}{\rtimes} \widehat{\mathbb{G}}),$$
	$$F'':\mathscr{K}\mathscr{K}^{\widehat{\mathbb{H}}} \rightarrow \mathscr{A}b^{\mathbb{Z}/2}\mbox{, } (B,\beta) \mapsto F''(B):= K_{*}(B \underset{\beta,r}{\rtimes} \widehat{\mathbb{H}}).$$
	
	Therefore, the quantum assembly maps for $\widehat{\mathbb{F}}$, $\widehat{\mathbb{G}}$ and $\widehat{\mathbb{H}}$ are given by the natural transformations $\mathbb{L}F\overset{\eta^{\widehat{\mathbb{F}}}}{\longrightarrow} F$, $\mathbb{L}F'\overset{\eta^{\widehat{\mathbb{G}}}}{\longrightarrow} F'$ and $\mathbb{L}F''\overset{\eta^{\widehat{\mathbb{H}}}}{\longrightarrow} F''$, respectively. Next, consider the functor: 
	$$\mathcal{Z}:\mathscr{K}\mathscr{K}^{\widehat{\mathbb{G}}}\times \mathscr{K}\mathscr{K}^{\widehat{\mathbb{H}}} \rightarrow \mathscr{K}\mathscr{K}^{\widehat{\mathbb{F}}}$$ 
	defined on objects by $(A,\alpha)\times (B,\beta) \mapsto \mathcal{Z}(A, B):= (C:=A\otimes B, \delta:=\alpha\otimes \beta)$ and on homomorphisms by $\mathcal{Z}(\mathcal{X},\mathcal{Y}):=\mathcal{X}\otimes \mathcal{Y}:=\tau_{B}(\mathcal{X})\underset{A'\otimes B}{\otimes}{_{A'}}\tau(\mathcal{Y})$, for all $\mathcal{X}\in KK^{\widehat{\mathbb{G}}}(A,A')$, $\mathcal{Y}\in KK^{\widehat{\mathbb{H}}}(B,B')$ with $(A, \alpha)\times (B, \beta), (A',\alpha')\times (B',\beta')\in \text{Obj}(\mathscr{K}\mathscr{K}^{\widehat{\mathbb{G}}})\times \text{Obj}(\mathscr{K}\mathscr{K}^{\widehat{\mathbb{H}}})$. %As observed in Section \ref{sec.TorsionQuantumDirProd}, given an object $(A,\alpha)\times (B,\beta)\in \text{Obj}(\mathscr{K}\mathscr{K}^{\widehat{\mathbb{G}}})\times \text{Obj}(\mathscr{K}\mathscr{K}^{\widehat{\mathbb{H}}})$, the tensor product $A\otimes B$ is a C$^*$-algebra equipped with the action $\alpha\otimes\beta$.
	\begin{rem}
		By virtue of the classification of torsion actions of a quantum direct product obtained in Theorem \ref{theo.TorsionDirectProd}, we have that $\text{Obj}(\widehat{\mathscr{L}}_{\widehat{\mathbb{F}}})=\text{Obj}(\widehat{\mathscr{L}}_{\widehat{\mathbb{G}}})\otimes \text{Obj}(\widehat{\mathscr{L}}_{\widehat{\mathbb{H}}})$, hence $\text{Obj}(\mathscr{L}_{\widehat{\mathbb{F}}})=\text{Obj}(\mathscr{L}_{\widehat{\mathbb{G}}})\otimes \text{Obj}(\mathscr{L}_{\widehat{\mathbb{H}}})$.
	\end{rem}
	
	\begin{lem}\label{lem.FunctorPreservingLDirectProduct}
		Let $\mathbb{G}$, $\mathbb{H}$ be two compact quantum groups and let $\mathbb{F}:=\mathbb{G}\times \mathbb{H}$ be the corresponding quantum direct product of $\mathbb{G}$ and $\mathbb{H}$. The functor $\mathcal{Z}:\mathscr{K}\mathscr{K}^{\widehat{\mathbb{G}}}\times \mathscr{K}\mathscr{K}^{\widehat{\mathbb{H}}} \rightarrow \mathscr{K}\mathscr{K}^{\widehat{\mathbb{F}}}$ is such that $\mathcal{Z}(\mathscr{L}_{\widehat{\mathbb{G}}}\times \mathscr{L}_{\widehat{\mathbb{H}}})\subset \mathscr{L}_{\widehat{\mathbb{F}}}$ and $\mathcal{Z}(\mathscr{N}_{\widehat{\mathbb{G}}}\times \mathscr{N}_{\widehat{\mathbb{H}}})\subset \mathscr{N}_{\widehat{\mathbb{F}}}$. Moreover, we have the following.
		\begin{enumerate}[i)]
		
		\item If $(A_0, \alpha_0)\in \text{Obj}(\mathscr{K}\mathscr{K}^{\widehat{\mathbb{G}}})$ is a given $\widehat{\mathbb{G}}$-C$^*$-algebra, the functor ${_{A_0}}\mathcal{Z}:\mathscr{K}\mathscr{K}^{\widehat{\mathbb{H}}} \rightarrow \mathscr{K}\mathscr{K}^{\widehat{\mathbb{F}}}\mbox{, }(B,\beta)  \mapsto {_{A_0}}\mathcal{Z}(B):= \mathcal{Z}(A_0, B)$ is triangulated such that ${_{A_0}}\mathcal{Z}(\mathscr{N}_{\widehat{\mathbb{H}}})\subset \mathscr{N}_{\widehat{\mathbb{F}}}$.
		
		\item If  $(B_0, \beta_0)\in \text{Obj}(\mathscr{K}\mathscr{K}^{\widehat{\mathbb{H}}})$ is a given $\widehat{\mathbb{H}}$-C$^*$-algebra, the functor $\mathcal{Z}_{B_0}:\mathscr{K}\mathscr{K}^{\widehat{\mathbb{G}}} \rightarrow\mathscr{K}\mathscr{K}^{\widehat{\mathbb{F}}}\mbox{, }(A,\alpha) \mapsto \mathcal{Z}_{B_0}(A):= \mathcal{Z}(A, B_0)$ is triangulated such that $\mathcal{Z}_{B_0}(\mathscr{N}_{\widehat{\mathbb{G}}})\subset \mathscr{N}_{\widehat{\mathbb{F}}}$.
		\end{enumerate}
	\end{lem}
	\begin{proof}
		Most part of the argument used in \cite[Lemma 5.2.1]{RubenSemiDirect} applies, but here we have removed the torsion-freeness assumption. So, we need to modify some steps of the proof of \cite[Lemma 5.2.1]{RubenSemiDirect}. First, given and object $(T,\tau)\times (S,\sigma)\in \text{Obj}(\mathscr{K}\mathscr{K}^{\mathbb{G}})\times \text{Obj}(\mathscr{K}\mathscr{K}^{\mathbb{H}})$ with $(T,\tau)$ a torsion action of $\mathbb{G}$ and $(S, \sigma)$ a torsion action of $\mathbb{H}$, then $(R:=T\otimes S, \rho:=\tau\otimes \sigma)$ is a torsion action of $\mathbb{F}$ as observed in Section \ref{sec.TorsionQuantumDirProd}. Next, using Proposition \ref{pro.DirectProductTensorProduct} we write:
		\begin{equation*}
			\begin{split}
				\mathcal{Z}\big(T\underset{\tau, r}{\rtimes}\mathbb{G}\otimes C_1, S\underset{\sigma, r}{\rtimes}\mathbb{H}\otimes C_2\big)&=T\underset{\tau, r}{\rtimes}\mathbb{G}\otimes C_1\otimes S\underset{\sigma, r}{\rtimes}\mathbb{H}\otimes C_2\\
				&\cong T\underset{\tau, r}{\rtimes}\mathbb{G}\otimes S\underset{\sigma, r}{\rtimes}\mathbb{H}\otimes C_1\otimes C_2=R\underset{\rho, r}{\rtimes} \mathbb{F}\otimes C_3\mbox{,}
			\end{split}
		\end{equation*}
		where $C_3:=C_1\otimes C_2\in \text{Obj}(\mathscr{K}\mathscr{K})$. Consequently, $\mathcal{Z}(\mathscr{L}_{\widehat{\mathbb{G}}}\times \mathscr{L}_{\widehat{\mathbb{H}}})\subset \mathscr{L}_{\widehat{\mathbb{F}}}$ as in \cite[Lemma 5.2.1]{RubenSemiDirect}. To establish the property $\mathcal{Z}(\mathscr{N}_{\widehat{\mathbb{G}}}\times \mathscr{N}_{\widehat{\mathbb{H}}})\subset \mathscr{N}_{\widehat{\mathbb{F}}}$ in the non torsion-free case, we need an alternative argument to the one in \cite[Lemma 5.2.1]{RubenSemiDirect}. Note that, by virtue of Baaj-Skandalis duality and Proposition \ref{pro.DirectProductTensorProduct}, it is enough to show that $\mathcal{Z}(\widehat{\mathscr{N}}_{\widehat{\mathbb{G}}}\times \widehat{\mathscr{N}}_{\widehat{\mathbb{H}}})\subset \widehat{\mathscr{N}}_{\widehat{\mathbb{F}}}$. Consider $(A, \alpha)\in \text{Obj}(\widehat{\mathscr{N}}_{\widehat{\mathbb{G}}})$ and $(B, \beta)\in \text{Obj}(\widehat{\mathscr{N}}_{\widehat{\mathbb{H}}})$. This means, as explained in Section \ref{sec.QuantumBC}, that $j_{\mathbb{G}, T}(A)=A\underset{r, \alpha}{\rtimes}\mathbb{G}\underset{r, \overline{\tau}}{\ltimes}T^{op}\cong 0$ in $\mathscr{K}\mathscr{K}$, for every torsion action $(T, \tau)$ of $\mathbb{G}$ ; and that $j_{\mathbb{H}, S}(B)=B\underset{r, \beta}{\rtimes}\mathbb{H}\underset{r, \overline{\sigma}}{\ltimes}S^{op}\cong 0$ in $\mathscr{K}\mathscr{K}$, for every torsion action $(S, \sigma)$ of $\mathbb{H}$. Put $(R, \rho):=(T\otimes S, \tau\otimes\sigma)$, which is a torsion action of $\mathbb{F}$. Next, by virtue of Proposition \ref{pro.TwoSidedDirectProd}, we have:
		$$j_{\mathbb{F}, R}(A\otimes B)=(A\otimes B)\underset{r, \alpha\otimes \beta}{\rtimes}\mathbb{F}\underset{r, \overline{\rho}}{\ltimes}R^{op}\cong \big(A\underset{r, \alpha}{\rtimes}\mathbb{G}\underset{r, \overline{\tau}}{\ltimes}T^{op}\big)\otimes \big(B\underset{r, \beta}{\rtimes}\mathbb{H}\underset{r, \overline{\sigma}}{\ltimes}S^{op}\big).$$
		
		Therefore, $j_{\mathbb{F}, R}(A\otimes B)\cong 0$ $\mathscr{K}\mathscr{K}$, for every torsion action $(R, \rho)$ of $\mathbb{F}$ as soon as $(A, \alpha)\in \text{Obj}(\widehat{\mathscr{N}}_{\widehat{\mathbb{G}}})$ and $(B, \beta)\in \text{Obj}(\widehat{\mathscr{N}}_{\widehat{\mathbb{H}}})$ (cf. Theorem \ref{theo.TorsionDirectProd}). In other words, $\mathcal{Z}(\widehat{\mathscr{N}}_{\widehat{\mathbb{G}}}\times \widehat{\mathscr{N}}_{\widehat{\mathbb{H}}})\subset \widehat{\mathscr{N}}_{\widehat{\mathbb{F}}}$ as desired.
		
		To conclude, let us show property $(i)$ of the statement (the proof of property $(ii)$ is similar). It remains to show that, given a $\widehat{\mathbb{G}}$-C$^*$-algebra $(A_0, \alpha_0)\in \text{Obj}(\mathscr{K}\mathscr{K}^{\widehat{\mathbb{G}}})$, then the functor ${_{A_0}}\mathcal{Z}$ is triangulated. For this, the analogous argument as in \cite[Lemma 5.2.1]{RubenSemiDirect} applies.
	\end{proof}
	
	The previous lemma allows to improve \cite[Lemma 5.2.2]{RubenSemiDirect} by removing the torsion-freeness assumption following the analogous argument as in \cite[Lemma 5.2.1]{RubenSemiDirect}. More precisely, we have the following:
	\begin{lem}\label{lem.InvertibleElementDirectProduct}
		Let $\mathbb{G}$, $\mathbb{H}$ be two compact quantum groups and let $\mathbb{F}:=\mathbb{G}\times \mathbb{H}$ be the corresponding quantum direct product of $\mathbb{G}$ and $\mathbb{H}$.
		
		\begin{enumerate}[i)]
			\item For all $\widehat{\mathbb{G}}$-C$^*$-algebra $(A,\alpha)$ and all $\widehat{\mathbb{H}}$-C$^*$-algebra $(B,\beta)$ there exists a Kasparov triple $\psi\in KK^{\widehat{\mathbb{F}}}\big(L'(A)\otimes L''(B), L(A\otimes B)\big)$ such that the diagram:
		\begin{equation*}\label{eq.CommutativeDiagramDirectProduct}
		\begin{gathered}
			\xymatrix@C=20mm@!R=15mm{
				\mbox{$L'(A)\underset{r}{\rtimes}\widehat{\mathbb{G}}\otimes L''(B) \underset{r}{\rtimes}\widehat{\mathbb{H}}$}\ar[d]_{\mbox{$ u'\underset{r}{\rtimes}\widehat{\mathbb{G}}\otimes u''\underset{r}{\rtimes}\widehat{\mathbb{H}}$}}\ar[r]^-{\mbox{$\Psi$}}&\mbox{$L(A\otimes B) \underset{r}{\rtimes}\widehat{\mathbb{F}}$}\ar[d]^{\mbox{$u \underset{r}{\rtimes}\widehat{\mathbb{F}}$}}\\
				\mbox{$A\underset{r}{\rtimes}\widehat{\mathbb{G}}\otimes B\underset{r}{\rtimes}\widehat{\mathbb{H}}$}\ar[r]^{\mbox{$\cong$}}&\mbox{$(A\otimes B)\underset{r}{\rtimes}\widehat{\mathbb{F}}$}}
		\end{gathered}
		\end{equation*}
		is commutative where $\Psi:=\psi\underset{r}{\rtimes}\widehat{\mathbb{F}}$ and $u'$, $u''$, $u$ are the Dirac homomorphisms for $A$, $B$, $A\otimes B$, respectively.
			\item For all $\widehat{\mathbb{G}}$-C$^*$-algebra $(A_0,\alpha_0)\in \mathscr{L}_{\widehat{\mathbb{G}}}$ and all $\widehat{\mathbb{H}}$-C$^*$-algebra $(B,\beta)$ there exists an \emph{invertible} Kasparov triple ${_{A_0}}\psi\in KK^{\widehat{\mathbb{F}}}\big(A_0\otimes L''(B), L(A_0\otimes B)\big)$ such that the diagram: 
		\begin{equation*}\label{eq.CommutativeDiagramDirectProductbisH}
		\begin{gathered}
			\xymatrix@C=20mm@!R=15mm{
				\mbox{$A_0\underset{r}{\rtimes}\widehat{\mathbb{G}}\otimes L''(B)\underset{r}{\rtimes}\widehat{\mathbb{H}}$}\ar[d]_{\mbox{${_{A_0}}\mathcal{Z}(u'')\underset{r}{\rtimes}\widehat{\mathbb{F}}$}}\ar[r]^-{\mbox{$\underset{\sim}{{_{A_0}}\Psi}$}}&\mbox{$L(A_0\otimes B)\underset{r}{\rtimes}\widehat{\mathbb{F}}$}\ar[d]^{\mbox{$u\underset{r}{\rtimes}\widehat{\mathbb{F}}$}}\\
				\mbox{$A_0\underset{r}{\rtimes}\widehat{\mathbb{G}}\otimes B\underset{r}{\rtimes}\widehat{\mathbb{H}}$}\ar[r]^{\mbox{$\cong$}}&\mbox{$(A_0\otimes B)\underset{r}{\rtimes}\widehat{\mathbb{F}}$}}
		\end{gathered}
		\end{equation*}
		is commutative where ${_{A_0}}\Psi:={_{A_0}}\psi\underset{r}{\rtimes}\widehat{\mathbb{F}}$ and $u''$, $u$ are the Dirac homomorphism for $B$, $A_0\otimes B$, respectively.
		\end{enumerate}
	\end{lem}
	
	Of course, using $(ii)$ from Lemma \ref{lem.FunctorPreservingLDirectProduct}, we also obtain that for all $\widehat{\mathbb{H}}$-C$^*$-algebra $(B_0,\beta_0)\in \mathscr{L}_{\widehat{\mathbb{H}}}$ and all $\widehat{\mathbb{G}}$-C$^*$-algebra $(A,\alpha)$ there exists an \emph{invertible} Kasparov triple $\psi_{B_0}\in KK^{\widehat{\mathbb{F}}}\big(L'(A)\otimes B_0, L(A\otimes B_0)\big)$ such that the diagram: 
		\begin{equation*}\label{eq.CommutativeDiagramDirectProductbisG}
		\begin{gathered}
			\xymatrix@C=20mm@!R=15mm{
				\mbox{$L(A)\underset{r}{\rtimes}\widehat{\mathbb{G}}\otimes B_0\underset{r}{\rtimes}\widehat{\mathbb{H}}$}\ar[d]_{\mbox{$\mathcal{Z}_{B_0}(u')\underset{r}{\rtimes}\widehat{\mathbb{F}}$}}\ar[r]^-{\mbox{$\underset{\sim}{\Psi_{B_0}}$}}&\mbox{$L(A\otimes B_0)\underset{r}{\rtimes}\widehat{\mathbb{F}}$}\ar[d]^{\mbox{$u\underset{r}{\rtimes} \widehat{\mathbb{F}}$}}\\
				\mbox{$A\underset{r}{\rtimes}\widehat{\mathbb{G}}\otimes B_0 \underset{r}{\rtimes}\widehat{\mathbb{H}}$}\ar[r]^{\mbox{$\cong$}}&\mbox{$(A\otimes B_0)\underset{r}{\rtimes}\widehat{\mathbb{F}}$}}
		\end{gathered}
		\end{equation*}
		is commutative where $\Psi_{B_0}:=\psi_{B_0}\underset{r}{\rtimes}\widehat{\mathbb{F}}$ and $u'$, $u$ are the Dirac homomorphism for $A$, $A\otimes B_0$, respectively. In particular, taking $\mathbb{H}:=\mathbb{E}$ the trivial quantum group, we have $\mathbb{F}=\mathbb{G}$ and so $(L, N)=(L', N')$. Moreover, $\mathscr{K}\mathscr{K}^{\mathbb{E}}=\mathscr{L}_{\mathbb{E}}$ (because the trivial quantum group satisfies the strong BC property). In other words, given any C$^*$-algebra $B\in \text{Obj}(\mathscr{K}\mathscr{K})$, we have $B=L''(B)\in\mathscr{L}_{\mathbb{E}}$ and the previous yields an isomorphism $\psi_B: L(A)\otimes B\rightarrow L(A\otimes B)$ in $\mathscr{K}\mathscr{K}^{\widehat{\mathbb{G}}}$, for all $\widehat{\mathbb{G}}$-C$^*$-algebra $(A,\alpha)$. More precisely, we have the following:
		
		
		
		
		\begin{comment}As already observed in \cite[Remark 3.3.2.4]{RubenThesis}, the element $\psi_{B_0}$ constructed above is such that the diagram:
		\begin{equation*}
		\begin{gathered}
			\xymatrix@C=20mm@!R=15mm{
				\mbox{$L'(A)\otimes B_0$}\ar[d]_{\mbox{$u'\otimes id$}}\ar[r]^{\mbox{$\psi_{B_0}$}}&\mbox{$L(A\otimes B_0)$}\ar[d]^{\mbox{$u$}}\\
				\mbox{$A\otimes B_0$}\ar[r]^{\mbox{$=$}}&\mbox{$A\otimes B_0$}}
		\end{gathered}
		\end{equation*}
		
	is commutative for all $\widehat{\mathbb{G}}$-C$^*$-algebra $(A,\alpha)\in \text{Obj}(\mathscr{K}\mathscr{K}^{\widehat{\mathbb{G}}})$ and all $\widehat{\mathbb{H}}$-C$^*$-algebra $(B_0,\beta_0)\in \mathscr{L}_{\widehat{\mathbb{H}}}$.\end{comment}
	
	
	
	
	
	
	\begin{cor}\label{cor.InvertibleElementDirectProduct}
		Let $\mathbb{G}$ be a compact quantum group. For all $\widehat{\mathbb{G}}$-C$^*$-algebra $(A,\alpha)\in \text{Obj}(\mathscr{K}\mathscr{K}^{\widehat{\mathbb{G}}})$ and all C$^*$-algebra $B\in \text{Obj}(\mathscr{K}\mathscr{K})$ there exists an \emph{invertible} Kasparov triple $\psi_B\in KK^{\widehat{\mathbb{G}}}\big(L(A)\otimes B, L(A\otimes B)\big)$ such that the diagram: 
		\begin{equation}\label{eq.CommutativeDiagramDirectProductbisGNoH}
		\begin{gathered}
			\xymatrix@C=20mm@!R=15mm{
				\mbox{$ L(A)\underset{r}{\rtimes}\widehat{\mathbb{G}}\otimes B$}\ar[d]_{\mbox{$\mathcal{Z}_{B_0}(u')\underset{r}{\rtimes}\widehat{\mathbb{G}} $}}\ar[r]^-{\mbox{$\underset{\sim}{\Psi_{B}}$}}&\mbox{$ L(A\otimes B)\underset{r}{\rtimes}\widehat{\mathbb{G}}$}\ar[d]^{\mbox{$u\underset{r}{\rtimes} \widehat{\mathbb{G}}$}}\\
				\mbox{$ A\underset{r}{\rtimes}\widehat{\mathbb{G}}\otimes B$}\ar[r]^{\mbox{$\cong$}}&\mbox{$(A\otimes B)\underset{r}{\rtimes}\widehat{\mathbb{G}}$}}
		\end{gathered}
		\end{equation}
		is commutative where $\Psi_{B}:=\psi_{B}\underset{r}{\rtimes}\widehat{\mathbb{G}}$ and $u'$ and $u$ are the Dirac homomorphism for $A$ and $A\otimes B$, respectively.
	\end{cor}
	
	Another consequence of Lemma \ref{lem.InvertibleElementDirectProduct} is that the strong quantum BC property is preserved by quantum direct products. This result is the content of \cite[Theorem 5.2.3-(i)]{RubenSemiDirect} under torsion-freeness assumption. Here we are able to improve it by removing this assumption. The argument, as based on Lemma \ref{lem.InvertibleElementDirectProduct}, remains the same as the one in \cite[Theorem 5.2.3-(i)]{RubenSemiDirect} (see also  \cite[Remark 5.2.4]{RubenSemiDirect}). So, we have:
	\begin{theo}\label{theo.StrongBCDirectProd}
		Let $\mathbb{G}$, $\mathbb{H}$ be two compact quantum groups and let $\mathbb{F}:=\mathbb{G}\times \mathbb{H}$ be the corresponding quantum direct product of $\mathbb{G}$ and $\mathbb{H}$. If $\widehat{\mathbb{G}}$ and $\widehat{\mathbb{H}}$ satisfy the strong quantum BC property, then $\widehat{\mathbb{F}}$ satisfies the quantum BC property with coefficients in $A\otimes B$, for every $A\in Obj(\mathscr{K}\mathscr{K}^{\widehat{\mathbb{G}}})$ and $B\in Obj(\mathscr{K}\mathscr{K}^{\widehat{\mathbb{H}}})$. Moreover, $L(A\otimes B)\cong A\otimes B$, for every $A\in Obj(\mathscr{K}\mathscr{K}^{\widehat{\mathbb{G}}})$ and $B\in Obj(\mathscr{K}\mathscr{K}^{\widehat{\mathbb{H}}})$.%$\widehat{\mathbb{F}}$ satisfies the strong quantum BC property. 
	\end{theo}
%actually, $\widehat{\mathbb{F}}$ satisfies the strong quantum BC property?? NO! because Z is not a triangulated functor. Not every F-C*-algebra is a tensor product of G-C*-algebras and H-C*-algebras.
	\begin{rem}
		In the course of the proof of \cite[Theorem 5.2.3-(i)]{RubenSemiDirect} it is shown that $\eta^{\widehat{\mathbb{G}}}_A\otimes \eta^{\widehat{\mathbb{H}}}_B=\eta^{\widehat{\mathbb{F}}}_{A\otimes B}$ as soon as both $\widehat{\mathbb{G}}$ and $\widehat{\mathbb{H}}$ satisfy the strong quantum BC property.
	\end{rem}

	
	Note that the analogous assertion for the usual quantum BC property needs further hypothesis, which are related to the Künneth formula in order to compute the K-theory of a tensor product. This relation will be studied in Section \ref{sec.BCKunneth}.
	
	
	
	
	\subsection{The Künneth class $\mathcal{N}_{\widehat{\mathbb{G}}}$}
	The previous preliminary observations allow us to formulate an equivariant Künneth class in the spirit of Examples \ref{exs.KunnethClasses} and \cite{ChabertEchterhoffOyono} in the context of quantum groups. More precisely, let $\mathbb{G}$ be a compact quantum group, $(A, \alpha)$ a $\widehat{\mathbb{G}}$-C$^*$-algebra and $B$ a C$^*$algebra. The external Kasparov product $\widetilde{\tau}_\mathbb{C}$ together with the invertible Kasparov triple $\psi_B$ obtained in Corollary \ref{cor.InvertibleElementDirectProduct} give:
	$$KK(\mathbb{C},  L(A)\underset{r}{\rtimes} \widehat{\mathbb{G}})\times KK(\mathbb{C}, B)\overset{\widetilde{\tau}_{\mathbb{C}}}{\longrightarrow} KK(\mathbb{C}, L(A)\underset{r}{\rtimes} \widehat{\mathbb{G}}\otimes B)\cong KK(\mathbb{C}, L(A\otimes B)\underset{r}{\rtimes} \widehat{\mathbb{G}}),$$
	which yields a group homomorphism $\alpha_{\widehat{\mathbb{G}}}: K_*(L(A)\underset{r}{\rtimes} \widehat{\mathbb{G}})\otimes K_*(B)\rightarrow K_*(L(A\otimes B)\underset{r}{\rtimes} \widehat{\mathbb{G}})$, which is natural on $A$ and $B$.
	
	\begin{defi}
		Let $\mathbb{G}$ be a compact quantum group. We denote by $\mathcal{N}_{\widehat{\mathbb{G}}}$ the calss of all $\widehat{\mathbb{G}}$-C$^*$-algebras $(A,\alpha)$ in $\text{C}^*\text{-Alg}$ such that $\alpha_{\widehat{\mathbb{G}}}$ is an isomorphism for all separable C$^*$-algebra $B$ with $K_*(B)$ free abelian. By abuse of notation, we still denote by $\mathcal{N}_{\widehat{\mathbb{G}}}$ the full subcategory of $\text{C}^*\text{-Alg}$ generated by the class $\mathcal{N}_{\widehat{\mathbb{G}}}$.
	\end{defi}
	
	Note that the homomorphism $\alpha_{\widehat{\mathbb{G}}}$ is obtained, up to the identification given by Corollary \ref{cor.InvertibleElementDirectProduct}, as it was obtained the group homomorphism $\alpha$ from Examples \ref{exs.KunnethClasses}. Here we use $L(A)\underset{r}{\rtimes} \widehat{\mathbb{G}}$ instead of an arbitrary C$^*$-algebra $A$. In this sense, we have that $(A,\alpha)\in \mathcal{N}_{\widehat{\mathbb{G}}}$ $\Leftrightarrow$ $L(A)\underset{r}{\rtimes} \widehat{\mathbb{G}}\in\mathcal{N}$, by definition. Therefore, $\mathcal{N}_{\widehat{\mathbb{G}}}$ is a Künneth category in the sense of Definition \ref{defi.KunnethCategory} and, given $(A,\alpha)\in \mathcal{N}_{\widehat{\mathbb{G}}}$, the map $B\mapsto K_*(L(A\otimes B)\underset{r}{\rtimes} \widehat{\mathbb{G}})$ defines a functor on the whole category $\text{C}^*\text{-Alg}$ satisfying properties $(i)-(iv)$ from Definition \ref{defi.KunnethFunctor}. In other words, this map defines a Künneth functor. Consequently, Theorem \ref{theo.AbstractKunnethTheorem} yields that the Künneth sequence:
	$$0\longrightarrow K_*(L(A)\underset{r}{\rtimes} \widehat{\mathbb{G}})\otimes K_*(B)\overset{\alpha_{\widehat{\mathbb{G}}}}{\longrightarrow} K_*(L(A\otimes B)\underset{r}{\rtimes} \widehat{\mathbb{G}})\overset{\beta_{\widehat{\mathbb{G}}}}{\longrightarrow} \text{Tor}(K_*(L(A)\underset{r}{\rtimes} \widehat{\mathbb{G}}), K_*(B))\longrightarrow 0$$
	is exact for all $(A,\alpha)\in\mathcal{N}_{\widehat{\mathbb{G}}}$ and all $B\in \text{Obj}(\text{C}^*\text{-Alg})$.
	
	\begin{ex}
		If $\widehat{\mathbb{G}}$ is a classical locally compact group $G$, then $\mathcal{N}_{\widehat{\mathbb{G}}}=\mathcal{N}_G$ as defined in Examples \ref{exs.KunnethClasses}. Indeed, from the Meyer-Nest reformulation of the BC property (see \cite[Theorem 5.2]{MeyerNest}) one knows that $K^{\text{top}}_*(G, A)\cong K_*(L(A)\underset{r}{\rtimes} G)$ naturally, for all $G$-C$^*$-algebra $(A,\alpha)$. Moreover, this approach to $\mathcal{N}_G$ yields a characterisation of the objects in the equivariant Künneth class in terms of the non-equivariant one, up to replacing $A$ by a $\mathscr{L}$-simplicial approximation of $A$. Namely, $(A,\alpha)\in \mathcal{N}_G$ $\Leftrightarrow$ $L(A)\underset{r}{\rtimes} G\in\mathcal{N}$.
	\end{ex}
	
	
\section{\textsc{Relating the quantum Baum-Connes property with the Künneth formula}}\label{sec.BCKunneth}
	In this section we generalise to the case of discrete quantum groups some key results appearing in \cite{ChabertEchterhoffOyono}. In particular, we are going to improve the stability of the quantum BC property for quantum direct products appearing already in \cite{RubenSemiDirect}.
	\begin{proSec}\label{pro.BCKunneth}
		Let $\mathbb{G}$ be a compact quantum group. The diagram:
		\begin{equation}\label{eq.CommutativeDiagramBCKunneth}
		\begin{gathered}
			\xymatrix@C=20mm@!R=15mm{
				\mbox{$ K_*(L(A)\underset{r}{\rtimes}\widehat{\mathbb{G}})\otimes K_*(B)$}\ar[d]_{\mbox{$\alpha_{\widehat{\mathbb{G}}}$}}\ar[r]^-{\mbox{$\eta^{\widehat{\mathbb{G}}}_{A}\otimes id$}}&\mbox{$ K_*(A\underset{r}{\rtimes}\widehat{\mathbb{G}})\otimes K_*(B)$}\ar[d]^{\mbox{$\alpha$}}\\
				\mbox{$ K_*(L(A\otimes B)\underset{r}{\rtimes}\widehat{\mathbb{G}})$}\ar[r]^{\mbox{$\eta^{\widehat{\mathbb{G}}}_{A\otimes B}$}}&\mbox{$K_*((A\otimes B)\underset{r}{\rtimes}\widehat{\mathbb{G}})$}}
		\end{gathered}
		\end{equation}
		commutes for all $\widehat{\mathbb{G}}$-C$^*$-algebra $A$ and all C$^*$-algebra $B$. In particular, if $\widehat{\mathbb{G}}$ satisfies the BC property with coefficients in $A\otimes B$, for all C$^*$-algebra $B$ equipped with the trivial action of $\widehat{\mathbb{G}}$; then $A\in\mathcal{N}_{\widehat{\mathbb{G}}}$ $\Leftrightarrow$ $A\underset{r}{\rtimes} \widehat{\mathbb{G}}\in\mathcal{N}$.
	\end{proSec}
	\begin{proof}
		Recall that the quantum assembly map $\eta^{\widehat{\mathbb{G}}}$ can be pictured as a certain Kasparov product (cf. Equation (\ref{eq.PictureQAssemblyMap})). Let $y\in K_*(L(A)\underset{r}{\rtimes}\widehat{\mathbb{G}})=KK(\mathbb{C}, L(A)\underset{r}{\rtimes}\widehat{\mathbb{G}})$ and $x\in K_*(B)=KK(\mathbb{C}, B)$. On the one hand, we have:
		\begin{equation*}
		\begin{split}
			y\otimes x&\overset{\alpha_{\widehat{\mathbb{G}}}}{\mapsto} y\ \underset{\mathbb{C}}{\widehat{\otimes}} x \overset{\eta^{\widehat{\mathbb{G}}}_{A\otimes B}}{\mapsto} (y\ \underset{\mathbb{C}}{\widehat{\otimes}} x)\underset{L(A\otimes B)\underset{r}{\rtimes}\widehat{\mathbb{G}}}{\otimes} u\underset{r}{\rtimes} \widehat{\mathbb{G}}\cong (y\ \underset{\mathbb{C}}{\widehat{\otimes}} x)\underset{L(A)\underset{r}{\rtimes}\widehat{\mathbb{G}}\otimes B}{\otimes} \big(u'\underset{r}{\rtimes} \widehat{\mathbb{G}}\otimes id\big)=\big(y \underset{L(A)\underset{r}{\rtimes}\widehat{\mathbb{G}}}{\otimes} u'\underset{r}{\rtimes} \widehat{\mathbb{G}}\big)\ \underset{\mathbb{C}}{\widehat{\otimes}} \big(x\underset{B}{\otimes} id),
		\end{split}
		\end{equation*}
		where the last equality follows from Equation (\ref{eq.TensorProdKasparovProd}). On the other hand, we have:
		\begin{equation*}
		\begin{split}
			y\otimes x&\overset{\eta^{\widehat{\mathbb{G}}}_A\otimes id}{\mapsto} \big(y \underset{L(A)\underset{r}{\rtimes}\widehat{\mathbb{G}}}{\otimes} u'\underset{r}{\rtimes} \widehat{\mathbb{G}}\big)\otimes x\overset{\alpha}{\mapsto} \big(y \underset{L(A)\underset{r}{\rtimes}\widehat{\mathbb{G}}}{\otimes} u'\underset{r}{\rtimes} \widehat{\mathbb{G}}\big)\ \underset{\mathbb{C}}{\widehat{\otimes}} x.
		\end{split}
		\end{equation*}
		
		These computations show that, indeed, Diagram (\ref{eq.CommutativeDiagramBCKunneth}) commutes. To conclude, notice that if $\widehat{\mathbb{G}}$ satisfies the BC property with coefficients in $A\otimes B$, for all C$^*$-algebra $B$ equipped with the trivial action of $\widehat{\mathbb{G}}$; then both $\eta_A$ and $\eta_{A\otimes B}$ are isomorphisms, for all C$^*$-algebra $B$. Therefore, commutativity of Diagram (\ref{eq.CommutativeDiagramBCKunneth}) yields that $\alpha_{\widehat{\mathbb{G}}}$ is an isomorphism if and only if $\alpha$ is an isomorphism.
	\end{proof}
	
	\begin{proSec}\label{pro.KunnethBC}
		Let $\mathbb{G}$ be a compact quantum group. Assume that $A$ is a $\widehat{\mathbb{G}}$-C$^*$-algebra such that $\widehat{\mathbb{G}}$ satisfies the quantum BC property with coefficients in $A$. 
		\begin{enumerate}[i)]
			\item If $A\in\mathcal{N}_{\widehat{\mathbb{G}}}$ and $A\underset{r}{\rtimes} \widehat{\mathbb{G}}\in\mathcal{N}$, then $\widehat{\mathbb{G}}$ satisfies the quantum BC property for $A\otimes B$, for all C$^*$-algebra $B$.
			\item If $A\in\mathcal{N}_{\widehat{\mathbb{G}}}$, then $\widehat{\mathbb{G}}$ satisfies the quantum BC property for $A\otimes B$, for all C$^*$-algebra $B\in\mathcal{N}$.
		\end{enumerate}
	\end{proSec}
	\begin{proof}
	Firstly, recall that given a $\widehat{\mathbb{G}}$-C$^*$-algebra $A$, we have $A\in \mathcal{N}_{\widehat{\mathbb{G}}}$ $\Leftrightarrow$ $L(A)\underset{r}{\rtimes} \widehat{\mathbb{G}}\in\mathcal{N}$ by definition. Next, consider the following diagram:
	$$
	\xymatrix@C=15mm@!R=18mm{
		\mbox{$0\longrightarrow K_*(L(A)\underset{r}{\rtimes}\widehat{\mathbb{G}})\otimes K_*(B)$}\ar[r]^-{\mbox{$\alpha_{\widehat{\mathbb{G}}}$}}\ar[d]_{\mbox{$\eta^{\widehat{\mathbb{G}}}_{A}\otimes id$}}
		&\mbox{$K_*(L(A\otimes B)\underset{r}{\rtimes}\widehat{\mathbb{G}})$}\ar[r]^-{\mbox{$\beta_{\widehat{\mathbb{G}}}$}}\ar[d]^{\mbox{$\eta^{\widehat{\mathbb{G}}}_{A\otimes B}$}}
		&\mbox{$\text{Tor}(K_*(L(A)\underset{r}{\rtimes} \widehat{\mathbb{G}}), K_*(B))\longrightarrow 0$}\ar[d]^{\mbox{$\text{Tor}(\eta^{\widehat{\mathbb{G}}}_{A}\otimes id)$}}\\
		\mbox{$0\longrightarrow K_*(A\underset{r}{\rtimes}\widehat{\mathbb{G}})\otimes K_*(B)$}\ar[r]_-{\mbox{$\alpha$}}
		&\mbox{$K_*((A\otimes B)\underset{r}{\rtimes}\widehat{\mathbb{G}})$}\ar[r]_-{\mbox{$\beta$}}
		&\mbox{$\text{Tor}(K_*(A\underset{r}{\rtimes}\widehat{\mathbb{G}}), K_*(B)) \longrightarrow 0$}}
	$$
	
	The argument follows then the same lines as the one in \cite[Proposition 4.10]{ChabertEchterhoffOyono}.
	\end{proof}
	
	\begin{corSec}\label{cor.KunnethBC2}
		Let $\mathbb{G}$ be a compact quantum group. Assume that $\widehat{\mathbb{G}}$ satisfies the quantum BC property with coefficients in $\mathbb{C}$.%$\mathbb{C}\in\mathscr{L}_{\widehat{\mathbb{G}}}$.
		\begin{enumerate}[i)]
			\item If $C(\mathbb{G})\in\mathcal{N}$ and $\mathbb{C}\in\mathcal{N}_{\widehat{\mathbb{G}}}$, then $\widehat{\mathbb{G}}$ satisfies the quantum BC property for all C$^*$-algebra $B$ equipped with the trivial action of $\widehat{\mathbb{G}}$.
			\item If $\mathbb{C}\in\mathcal{N}_{\widehat{\mathbb{G}}}$, then $\widehat{\mathbb{G}}$ satisfies the quantum BC property for all C$^*$-algebra $B\in\mathcal{N}$.
		\end{enumerate}
	\end{corSec}
	\begin{proof}
		Recall that $C(\mathbb{G})=\mathbb{C}\underset{r}{\rtimes} \widehat{\mathbb{G}}$. So, $(i)$ and $(ii)$ of statement follow from $(i)$ and $(ii)$ of Proposition \ref{pro.KunnethBC}, respectively.	
	\end{proof}
	
		\begin{remSec}\label{rem.FurtherHypoth}
	In the classical setting when $\widehat{\mathbb{G}}$ is a locally compact group $G$, the assumption $\mathbb{C}\in\mathcal{N}_G$ is automatically satisfied. Indeed, $\mathbb{C}$ is a type I C$^*$-algebra and $\mathcal{N}_G$ contains all type I $G$-C$^*$-algebras by virtue of \cite[Theorem 0.1]{ChabertEchterhoffOyono}. In the quantum setting, a similar related result to \cite[Theorem 0.1]{ChabertEchterhoffOyono} is Theorem \ref{theo.QuantumGroupCalgebraKunneth} in the next section (cf. Remark \ref{rem.AnalogoyThm01} for an explanation to this analogy). However, to the best knowledge of the author, it is not known for instance whether $\mathbb{C}\in\mathcal{N}_{\widehat{\mathbb{G}}}$ for every discrete quantum group $\widehat{\mathbb{G}}$. One reason for this is that in our approach the objects in $\mathcal{N}_{\widehat{\mathbb{G}}}$ are characterised in terms of objects in $\mathcal{N}$ up to a \emph{$\mathscr{L}_{\widehat{\mathbb{G}}}$-simplicial approximation}, which entails to study the localisation functor $L$ in relation with crossed products and the \emph{equivariant} Künneth class. One possibility to do so might be to adapt the \emph{Going-Down technique} from \cite{ChabertEchterhoffOyono} based on Theorem \ref{theo.QuantumGroupCalgebraKunneth}. But this is out of the scope of the present article.
	\end{remSec}
	
	The following theorem is an improved version of \cite[Corollary 5.2.5]{RubenSemiDirect}.
	\begin{theoSec}\label{theo.BCDirectProducts}
		Let $\mathbb{G}$, $\mathbb{H}$ be two compact quantum groups and let $\mathbb{F}:=\mathbb{G}\times \mathbb{H}$ be the corresponding quantum direct product of $\mathbb{G}$ and $\mathbb{H}$.
		\begin{enumerate}[i)]
			\item Let $A$ be a $\widehat{\mathbb{G}}$-C$^*$-algebra and $B$ a $\widehat{\mathbb{H}}$-C$^*$-algebra. Assume that $\widehat{\mathbb{G}}$ satisfies the quantum BC property with coefficients in $A$ and that $B\in\mathscr{L}_{\widehat{\mathbb{H}}}$. If $A\underset{r}{\rtimes}\widehat{\mathbb{G}}\in\mathcal{N}$ and $A\in\mathcal{N}_{\widehat{\mathbb{G}}}$, then $\widehat{\mathbb{F}}$ satisfies the quantum BC property with coefficients in $A\otimes B$.
			
			In particular, if $\widehat{\mathbb{G}}$ satisfies the quantum BC property with coefficients in $\mathbb{C}$, $\mathbb{C}\in\mathscr{L}_{\widehat{\mathbb{H}}}$, $C(\mathbb{G})\in\mathcal{N}$ and $\mathbb{C}\in\mathcal{N}_{\widehat{\mathbb{G}}}$; then $\widehat{\mathbb{F}}$ satisfies the quantum BC property with coefficients in $\mathbb{C}$.
			
			\item Assume that $\widehat{\mathbb{G}}$ and $\widehat{\mathbb{H}}$ satisfy the quantum BC property with coefficients in $\mathbb{C}$. If $C(\mathbb{G}), C(\mathbb{H})\in\mathcal{N}$ and $\mathbb{C}\in\mathcal{N}_{\widehat{\mathbb{F}}}$, then $\widehat{\mathbb{F}}$ satisfies the quantum BC property for all C$^*$-algebra $B$ equipped with the trivial action of $\widehat{\mathbb{F}}$.
		\end{enumerate}
	\end{theoSec}
	\begin{proof}
		\begin{enumerate}[i)]
			\item Consider the following diagram:
			$$
	\xymatrix@C=5mm@!R=15mm{
		\mbox{$0$}\ar[r]&\mbox{$ K_*(L'(A)\underset{r}{\rtimes}\widehat{\mathbb{G}})\otimes K_*(L''(B)\underset{r}{\rtimes}\widehat{\mathbb{H}})$}\ar[r]^-{\mbox{$\alpha_{\widehat{\mathbb{G}}}$}}\ar[d]_{\mbox{$\eta^{\widehat{\mathbb{G}}}_{A}\otimes \eta^{\widehat{\mathbb{H}}}_{B}$}}
		&\mbox{$K_*(L(A\otimes B)\underset{r}{\rtimes}\widehat{\mathbb{F}})$}\ar[r]^-{\mbox{$\beta_{\widehat{\mathbb{G}}}$}}\ar[d]^{\mbox{$\eta^{\widehat{\mathbb{F}}}_{A\otimes B}$}}
		&\mbox{$\text{Tor}(K_*(L'(A)\underset{r}{\rtimes} \widehat{\mathbb{G}}), K_*(L''(B)\underset{r}{\rtimes}\widehat{\mathbb{H}}))\longrightarrow 0$}\ar[d]^{\mbox{$\text{Tor}(\eta^{\widehat{\mathbb{G}}}_{A}\otimes \eta^{\widehat{\mathbb{H}}}_{B}$)}}\\
		\mbox{$0$}\ar[r]&\mbox{$ K_*(A\underset{r}{\rtimes}\widehat{\mathbb{G}})\otimes K_*(B\underset{r}{\rtimes}\widehat{\mathbb{H}})$}\ar[r]_-{\mbox{$\alpha$}}
		&\mbox{$K_*((A\otimes B)\underset{r}{\rtimes}\widehat{\mathbb{F}})$}\ar[r]_-{\mbox{$\beta$}}
		&\mbox{$\text{Tor}(K_*(A\underset{r}{\rtimes}\widehat{\mathbb{G}}), K_*(B\underset{r}{\rtimes}\widehat{\mathbb{H}})) \longrightarrow 0$}}
	$$
			
			Since $B\in\mathscr{L}_{\widehat{\mathbb{H}}}$, then $B\cong L''(B)$ and we can apply Lemma \ref{lem.InvertibleElementDirectProduct}. In particular, $L'(A)\otimes L''(B)\cong L(A\otimes B)$ and the middle arrow is indeed $\eta^{\widehat{\mathbb{F}}}_{A\otimes B}$. Moreover, $\eta^{\widehat{\mathbb{H}}}_{B}$ is an isomorphism. Since $\eta^{\widehat{\mathbb{G}}}_A$ is an isomorphism by assumption, then the left arrow is an isomorphism hence so is the right arrow. Finally, the assumptions $A\underset{r}{\rtimes}\widehat{\mathbb{G}}\in\mathcal{N}$ and $A\in\mathcal{N}_{\widehat{\mathbb{G}}}$ say that the bottom and top lines of the diagram are exact sequences, respectively. In conclusion, the Five lemma yields that the middle arrow, i.e. $\eta^{\widehat{\mathbb{F}}}_{A\otimes B}$, is an isomorphism.
			
			
			\item Observe that $C(\mathbb{F})=C(\mathbb{G})\otimes C(\mathbb{H})\in\mathcal{N}$ because $\mathcal{N}$ is closed under taking tensor products (cf. Lemma \ref{lem.StabilityClassN}). Then the conclusion follows from $(i)$ of Corollary \ref{cor.KunnethBC2}.
		\end{enumerate}
	\end{proof}
	
	\begin{remSec}
		The item $(i)$ in Theorem \ref{theo.BCDirectProducts} above is analogue to the item $(i)$ in \cite[Theorem 5.3]{ChabertEchterhoffOyono}. In the quantum setting we need however further hypothesis. Namely, we need $\mathbb{C}\in\mathcal{N}_{\widehat{\mathbb{G}}}$ as explained in Remark \ref{rem.FurtherHypoth} and $\mathbb{C}\in\mathscr{L}_{\widehat{\mathbb{H}}}$. The reason for the latter is again that we need some control for the $\mathscr{L}_{\widehat{\mathbb{H}}}$-simplicial approximatins in relation with tensor products. This is made precise by Lemma \ref{lem.InvertibleElementDirectProduct}, which we use in the proof of Theorem \ref{theo.BCDirectProducts}. In the classical setting, this control is guaranteed by \cite[Corollary 2.10]{ChabertEchterhoffOyono} based on their \emph{Going-Down technique}.
	\end{remSec}
	
\section{\textsc{Some K-theory computations}}\label{sec.KTheoryComp}

On the one hand, by Theorem \ref{theo.StrongBCDirectProd}, we have that if $\widehat{\mathbb{G}}$ and $\widehat{\mathbb{H}}$ satisfy the strong quantum BC property, then $\widehat{\mathbb{G}\times \mathbb{H}}$ satisfies the strong quantum BC property with coefficients in $A\otimes B$, for all $A\in Obj(\mathscr{K}\mathscr{K}^{\widehat{\mathbb{G}}})$ and $B\in Obj(\mathscr{K}\mathscr{K}^{\widehat{\mathbb{H}}})$. On the other hand, by Theorem \ref{theo.TorsionDirectProd}, we have classified all torsion actions of $\mathbb{G}\times \mathbb{H}$ as tensor products of torsion actions of $\mathbb{G}$ by torsion actions of $\mathbb{H}$. These are the main ingredients to apply the homological techniques from the Meyer-Nest work in order to compute the K-theory groups of $C(\mathbb{G}\times\mathbb{H})$. In this sense, it is plausible to construct explicit projective resolutions for $\mathbb{C}$ in $\mathscr{K}\mathscr{K}^{\widehat{\mathbb{G}\times \mathbb{H}}}$ as tensor products of projective resolutions for $\mathbb{C}$ in $\mathscr{K}\mathscr{K}^{\widehat{\mathbb{G}}}$ and in $\mathscr{K}\mathscr{K}^{\widehat{\mathbb{H}}}$. Note that, since $C(\mathbb{G}\times\mathbb{H})=C(\mathbb{G})\otimes C(\mathbb{H})$, we need either $C(\mathbb{G})$ or $C(\mathbb{H})$ to satisfy the Künneth formula to succeed in such a construction.

However, instead of doing that, we can compute $K_*(C(\mathbb{G}\times\mathbb{H}))$ using simply the Künneth formula. In order to do so, let us point out that \cite[Theorem 5.2]{YukiBCTorsion} and \cite[Corollary 5.5]{YukiBCTorsion} can be also obtained for the Künneth class instead of the bootstrap class. This is true because even if these two classes are not the same, they satisfy similar stabilising properties in the sense of Lemma \ref{lem.StabilityClassN}. Namely, $\mathcal{N}$ contains all finite dimensional C$^*$-algebras, it is closed under tensor products, it is closed under semi-split extensions (hence under mapping cones and homotopy limits) and it contains all type I C$^*$-algebras. In other words, the arguments in \cite[Theorem 5.2]{YukiBCTorsion} and \cite[Corollary 5.5]{YukiBCTorsion} can be applied \emph{verbatim} by replacing the bootstrap class by $\mathcal{N}$ and we obtain the following:
\begin{theoSec}\label{theo.QuantumGroupCalgebraKunneth}
	Let $\mathbb{G}$ be a compact quantum group. 
	\begin{enumerate}[i)]
		\item If $(B, \beta)$ is a $\mathbb{G}$-C$^*$-algebra such that $B\underset{r, \beta}{\rtimes}\mathbb{G}\underset{r, \overline{\delta}}{\ltimes}T^{op}\in\mathcal{N}$, for all torsion action $(T, \delta)\in\mathbb{T}\text{or}(\widehat{\mathbb{G}})$; then $\widehat{L}(B)\in\mathcal{N}$.
		\item Assume that $\widehat{\mathbb{G}}$ satisfies the strong quantum BC property. If $(A, \alpha)$ is a type $I$ $\widehat{\mathbb{G}}$-C$^*$-algebra, then $A\underset{r, \alpha}{\rtimes}\widehat{\mathbb{G}}\in\mathcal{N}$. In particular, $C(\mathbb{G})\in\mathcal{N}$.
	\end{enumerate}
\end{theoSec}

\begin{remSec}\label{rem.AnalogoyThm01}
	On the one hand, as we have explained in the introduction, the family of finite subgroups represents the torsion phenomenon for a classical discrete group. In the quantum setting it must be replaced by the family of torsion actions of a compact quantum group. This makes a substantial difference when it comes to define the quantum assembly map for a discrete quantum group. In particular, the induction functor must be replaced by the two-sided crossed product functor as explained in Section \ref{sec.QuantumBC}. In particular, we work in the category $\mathscr{K}\mathscr{K}^{\mathbb{G}}$ and not in $\mathscr{K}\mathscr{K}^{\widehat{\mathbb{G}}}$ itself. If $G$ is a discrete group, then \cite[Theorem 0.1]{ChabertEchterhoffOyono} gives a sufficient condition for a $G$-C$^*$-algebra to be in $\mathcal{N}_G$ in terms of the torsion of $G$. In this sense, item $(i)$ of Theorem \ref{theo.QuantumGroupCalgebraKunneth} gives a sufficient condition for the \emph{$\widehat{\mathscr{L}}_{\widehat{\mathbb{G}}}$-simplicial approximation} of a $\mathbb{G}$-C$^*$-algebra to be in $\mathcal{N}$ in terms of the torsion actions of $\mathbb{G}$. The fact that we work in $\mathscr{K}\mathscr{K}^{\mathbb{G}}$ leads to consider \emph{dual} (in the sense of the Baaj-Skandalis duality) simplicial approximations, which would lead to a \emph{dual} equivariant Künneth class. However, the Künneth class $\mathcal{N}$ is not stable under general crossed products hence it is not clear whether item $(i)$ of Theorem \ref{theo.QuantumGroupCalgebraKunneth} translates into a statement about $\mathcal{N}_{\widehat{\mathbb{G}}}$.
	
	On the other hand, the conclusion of item $(ii)$ of Theorem \ref{theo.QuantumGroupCalgebraKunneth}, i.e. that $C(\mathbb{G})\in\mathcal{N}$ as soons as $\widehat{\mathbb{G}}$ satisfies the strong quantum BC property is also true for classical locally compact groups. One can argue as follows. Assume that $G$ is a locally compact group satisfying the BC property with coefficients (\emph{a fortriori} when $G$ satisfies the \emph{strong} BC property). As explained in Remark \ref{rem.FurtherHypoth}, we always have $\mathbb{C}\in\mathcal{N}_G$. Therefore, \cite[Proposition 4.9]{ChabertEchterhoffOyono} implies that $C^*(G)=\mathbb{C}\underset{r}{\rtimes} G\in\mathcal{N}$.
\end{remSec}

By the work of Voigt and Vergnioux-Voigt (e.g. \cite{VoigtBaumConnesFree}, \cite{VoigtBaumConnesUnitaryFree}, \cite{VoigtBaumConnesAutomorphisms}) we have a number of examples of compact quantum groups with duals satisfying the strong quantum BC property. Hence the C$^*$-algebras defining these compact quantum groups lay in $\mathcal{N}$. Namely:
\begin{corSec}\label{cor.QGKunnetClass}
	Let $\mathbb{G}$ be any of the following compact quantum groups: $SU_q(2)$, $O^+(F)$, $U^+(Q)$, $S^+_N$, where $N\in\mathbb{N}$, $Q$ is a complex invertible matrix and $F$ is a complex invertible matrix such that $F\overline{F}\in \mathbb{R} id$. Then $C(\mathbb{G})\in\mathcal{N}$.
\end{corSec}

Moreover, a computation of the K-theory groups of the C$^*$-algebras $C(\mathbb{G})$ is carried out too in the works \cite{VoigtBaumConnesFree}, \cite{VoigtBaumConnesUnitaryFree}, \cite{VoigtBaumConnesAutomorphisms}. Namely:
\begin{theoSec}\label{theo.KtheoryComp}
	Let $N\in\mathbb{N}$, let $Q$ be a complex invertible matrix and let $F$ be a complex invertible matrix such that $F\overline{F}\in \mathbb{R} id$. Then:
	\begin{itemize}[-]
		\item $K_0(C(SU_q(2)))=\mathbb{Z}$ and $K_1(C(SU_q(2)))=\mathbb{Z}$.
		\item $K_0(C(O^+(F)))=\mathbb{Z}$ and $K_1(C(O^+(F)))=\mathbb{Z}$.
		\item $K_0(C(U^+(Q)))=\mathbb{Z}$ and $K_1(C(U^+(Q)))=\mathbb{Z}\oplus\mathbb{Z}$.
		\item $K_0(C(S^+_N))=\mathbb{Z}^{N^2-2N+2}$ and $K_1(C(S^+_N))=\mathbb{Z}$.
	\end{itemize}
\end{theoSec}

These computations together with Theorem \ref{theo.QuantumGroupCalgebraKunneth} allow to compute the K-theory groups for the corresponding quantum direct products by means of the Künneth formula. More precisely, if $\mathbb{G}$ and $\mathbb{H}$ are any of the compact quantum groups $SU_q(2)$, $O^+(F)$, $U^+(Q)$ or $S^+_N$ as above, then it follows from Theorem \ref{theo.KtheoryComp} that $K_*(C(\mathbb{H}))$ is free abelian. Moreover, $C(\mathbb{G})\in\mathcal{N}$ by Corollary \ref{cor.QGKunnetClass}. Therefore, we have $K_*(C(\mathbb{F}))=K_*(C(\mathbb{G})\otimes C(\mathbb{H}))=K_*(C(\mathbb{G}))\otimes K_*(C(\mathbb{H}))$, where $\mathbb{F}=\mathbb{G}\times \mathbb{H}$. Namely, we have the following:
\begin{theoSec}
	Let $N\in\mathbb{N}$, let $Q$ be a complex invertible matrix and let $F$ be a complex invertible matrix such that $F\overline{F}\in \mathbb{R} id$. Then:
	\begin{itemize}[-]
		\item For $\mathbb{F}:=O^+(F)\times O^+(F)$, we have $K_0(C(\mathbb{F}))=\mathbb{Z}$ and $K_1(C(\mathbb{F}))=\mathbb{Z}$.
		\item For $\mathbb{F}:=O^+(F)\times U^+(Q)$, we have $K_0(C(\mathbb{F}))=\mathbb{Z}$ and $K_1(C(\mathbb{F}))=\mathbb{Z}\oplus\mathbb{Z}$.
		\item For $\mathbb{F}:=O^+(F)\times S^+_N$, we have $K_0(C(\mathbb{F}))=\mathbb{Z}^{N^2-2N+2}$ and $K_1(C(\mathbb{F}))=\mathbb{Z}$.
		\item For $\mathbb{F}:=U^+(F)\times U^+(Q)$, we have $K_0(C(\mathbb{F}))=\mathbb{Z}$ and $K_1(C(\mathbb{F}))=\mathbb{Z}^4$.
		\item For $\mathbb{F}:=U^+(F)\times S^+_N$, we have $K_0(C(\mathbb{F}))=\mathbb{Z}^{N^2-2N+2}$ and $K_1(C(\mathbb{F}))=\mathbb{Z}\oplus\mathbb{Z}$.
		\item For $\mathbb{F}:=S^+_N\times S^+_N$, we have $K_0(C(\mathbb{F}))=\mathbb{Z}^{(N^2-2N+2)^2}$ and $K_1(C(\mathbb{F}))=\mathbb{Z}$.
		
	\end{itemize}
\end{theoSec}

	Of course, a similar list of K-theory groups can be obtained with other combinations of compact quantum groups into a quantum direct product as soon as the strong quantum BC property is satisfied and the K-groups of the C$^*$-algebras defining the quantum groups involved are free abelian. Notice that, in order to apply the Künneth formula, the computation of the K-groups of the C$^*$-algebra defining $\mathbb{G}\times \mathbb{H}$ requires at least $K_*(C(\mathbb{H}))$ to be free abelian.
	
	For instance, let $\mathbb{G}$ be a compact quantum group, $N\geq 4$ and $\mathbb{G}\wr_*S^+_N$ the corresponding free wreath product of $\mathbb{G}$ by $S^+_N$. A result by F. Lemeux and P. Tarrago (cf. \cite{TarragoWreath}) shows that there exists a parameter $q\in [-1,1]$ such that the compact quantum group $\mathbb{H}_q$ is monoidal equivalent to $\mathbb{G}\wr_*S^+_N$, where $\mathbb{H}_q$ is such that $\widehat{\mathbb{H}}_q$ is the discrete quantum subgroup of $\widehat{\mathbb{G} * SU_q(2)}$ generated, in the sense of the Tannaka-Krein duality (cf. Note \ref{note.DiscQSubTannakaKrein}), by the representations $xux$ (as a word in $\text{Irr}(\mathbb{G} * SU_q(2))$) with $x\in\text{Irr}(\mathbb{G})$ and $u$ being the fundamental representation of $SU_q(2)$. It is shown in \cite{RubenAmauryTorsion} that $\widehat{\mathbb{G}\wr_*S^+_N}$ (hence $\widehat{\mathbb{H}}_q$) satisfies the strong quantum BC property as soon as $\widehat{\mathbb{G}}$ is torsion-free and satisfies the strong quantum BC property. 
	
	The K-theory of $C(\mathbb{H}_q)$ is computed in \cite{RubenAmauryTorsion} for concrete and relevant instances of $\mathbb{G}$. For example, when $\widehat{\mathbb{G}}:=\mathbb{F}_n$ is the classical free group with $n\in\mathbb{N}$ generators, then $K_0(C(\mathbb{H}_q))=\mathbb{Z}\oplus \mathbb{Z}_2$ and $K_1(C(\mathbb{H}_q))=\mathbb{Z}^{n+1}$. In this case, we see that the K-groups are \emph{not} free abelian, so that the Künneth formula cannot be applied to compute the K-theory of quantum direct products of the form $\mathbb{X}\times \mathbb{H}_q$, but it still applies for quantum direct products of the form $\mathbb{H}_q \times \mathbb{X}$ for $\mathbb{X}$ any compact quantum group satisfying the strong quantum BC property and such that $K_*(C(\mathbb{X}))$ is free abelian.
	
	The K-theory of $C(\mathbb{G}\wr_*S^+_N)$ is computed in the recent paper \cite{FimaTroupelCouronne} by P. Fima and A. Troupel for concrete and relevant instances of $\mathbb{G}$. For example, when $\widehat{\mathbb{G}}:=\mathbb{F}_n$ is the classical free group with $n\in\mathbb{N}$ generators, then $K_0(C(\widehat{\mathbb{F}}_n\wr_*S^+_N))=\mathbb{Z}^{N^2-2N+2}$ and $K_1(C(\widehat{\mathbb{F}}_n\wr_*S^+_N))=\mathbb{Z}^{N^2n+1}$ (cf. \cite[Corollary 7.2]{FimaTroupelCouronne}). In this case, the K-groups are free abelian.
 






\bibliographystyle{acm}
\bibliography{KunnethClassQuantumGroups}

\vspace{1cm}
\textsc{R. Martos, Department of Mathematical Sciences, University of Copenhagen, Denmark.} 

\textit{E-mail address:} \textbf{\texttt{ruben.martos@math.ku.dk}}














 \end{document}
