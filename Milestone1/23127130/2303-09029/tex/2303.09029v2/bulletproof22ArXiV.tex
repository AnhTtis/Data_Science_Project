\documentclass[reprint,aps,prl,groupedaddress,amsmath,amssymb]{revtex4-2}
%\documentclass[aip,amsmath,amssymb,reprint]{revtex4-1}


\usepackage[dvipsnames]{xcolor}
\usepackage{graphicx}
\usepackage{dcolumn}
\usepackage{bm}

\usepackage[utf8]{inputenc}
\usepackage[T1]{fontenc}
\usepackage{mathptmx}
\usepackage{etoolbox}





\makeatletter
\def\@email#1#2{
 \endgroup
 \patchcmd{\titleblock@produce}
  {\frontmatter@RRAPformat}
  {\frontmatter@RRAPformat{\produce@RRAP{*#1\href{mailto:#2}{#2}}}\frontmatter@RRAPformat}
  {}{}
}%




\makeatother
\begin{document}

\preprint{AIP/123-QED}

\title{Tuning Higher Order Structure in Colloidal Fluids}


\author{Xiaoyue Wu}
\affiliation{School of Chemistry, University of Leeds, Woodhouse Lane, Leeds, LS2 9JT, United Kingdom}

 
\author{Katherine Skipper}%
\affiliation{H. H. Wills Physics Laboratory, University of Bristol, Bristol BS8 1TL, United Kingdom}


\author{Yushi Yang}
\affiliation{H. H. Wills Physics Laboratory, University of Bristol, Bristol BS8 1TL, United Kingdom}


\author{Fergus J. Moore}
\affiliation{H. H. Wills Physics Laboratory, University of Bristol, Bristol BS8 1TL, United Kingdom}


\author{Fiona C. Meldrum}
\affiliation{School of Chemistry, University of Leeds, Woodhouse Lane, Leeds, LS2 9JT, United Kingdom}


\author{C. Patrick Royall}
\email{paddy.royall@espci.psl.eu}
\affiliation{Gulliver UMR CNRS 7083, ESPCI Paris, Universit\'{e} PSL, 75005 Paris, France}




\date{\today}

\begin{abstract}
Colloidal particles self assemble into a wide range of structures under external AC electric fields due to induced dipolar interactions [Yethiraj and Van Blaaderen \emph{Nature} \textbf{421} 513 (2003)]. As a result of these dipolar interactions, at low volume fraction the system is modulated between a hard--sphere like state (in the case of zero applied field) and a ``string fluid'' upon application of the field. Using both particle--resolved experiments and Brownian dynamics simulations, we investigate the emergence of the string fluid with a variety of structural measures including two-body and higher--order correlations. The higher--order structure we probe using three-body spatial correlation functions and a many--body approach based on minimum energy clusters of a dipolar--Lennard--Jones system. This yields a series of geometrically distinct minimum energy clusters upon increasing the strength of the dipolar interaction, which are echoed in the higher--order structure of the colloidal fluids we study here. We find good agreement between experiment and simulation at the two-body level, although some discrepancies are found at higher field strength, where the system falls out of equilibrium. Higher--order correlations exhibit reasonable agreement between experiment and simulation, again with more discrepancy at higher field strength for three--body correlation functions. At higher field strength, the cluster population in our experiments and simulations is dominated by the minimum energy clusters for all sizes $8 \leq m \leq 12$. The agreement that we find here is notable considering that there is no fit parameter in our mapping between experiment and simulation.
\end{abstract}


\maketitle


\section{Introduction}
\label{sectionIntroduction}


Particles with a dipolar interaction are of great fundamental importance in the study of fluids and disordered materials. They are among the simplest models which describe long range directional interactions, which are exhibited by molecules~\cite{israelachvili,gebbie2017}. Colloidal dispersions provide suitable models of atomistic and molecular systems as they exhibit phase behavior following the same rules of statistical mechanics, yet are amenable to real space observation using optical microscopy~\cite{evans2019,ivlev,bharti2015}. 


Colloidal dipolar systems fall broadly into two categories. Some, for example ferromagnetic nanoparticle systems, like atoms and molecules have an intrinsic dipole moment~\cite{boles2016} and can be modeled with the Stockmayer model which combines a dipolar interaction with a Lennard--Jones interaction~\cite{novak2019}. These systems exhibit intriguing string--like structures~\cite{klokkenburg2006,pyanzina2007}, with branching ~\cite{kantorovich2015} coiling ~\cite{spataforasalazar2023} and clustering ~\cite{novak2019} behavior, not to mention a ferromagnetic transition~\cite{weis2006}. Rather than spontaneous dipolar interactions, in other colloidal systems, dipoles may be induced by an external electric or magnetic field~\cite{ivlev,bharti2015,donaldson2017}. This has the consequence that the dipolar interactions are aligned in the direction of the applied field. Using ferromagnetic and superparamagnetic nanoparticles in an external field then opens further possibilities such as a very strong response to the field~\cite{mostarac2023}. Other, more exotic possibilities include the use of a biaxial field, leading to phenomena such as in--plane condensation in (quasi) 2d systems~\cite{ivlev,elsner2009} and direct observation of cluster growth~\cite{bensalah2023}. In addition to their fundamental interest, such dipolar colloidal systems may find application as electrorheological fluids~\cite{winslow2000,dassanayake2000}, hydraulic valves and photonic materials ~\cite{vanblaaderen2004}. 


Here we shall focus on dipolar systems with an external field. A particular attraction of these systems is the ease with which the dipolar interactions can be tuned with the external field. Indeed a combination of real~\cite{yethiraj2003} and reciprocal space~\cite{klokkenburg2007,wiedenmann2008} studies of such systems enables us to probe many crystal structures including fcc, hcp, bcc, body-centered tetragonal (bct) and body-centered orthorhombic (bco) structures~\cite{yethiraj2003,hynninen2005,bharti2015}, along with a transient labyrinthine structure~\cite{dassanayake2000}.  Tuning the electric (or magnetic) field \emph{in--situ} enables the control of phenomena such as a martensitic transition~\cite{yethiraj2004}. Adding softness~\cite{colla2018} or attractions~\cite{semwal2022} to the interaction potential further increases the range of structures into which the system may self--assemble. 


In addition to the rich crystalline phase behavior, dipolar colloids feature a fluid phase at lower colloid volume fraction and electric field strength than those at which the crystals are found. Interestingly, the symmetry--breaking dipolar interactions cause this fluid to assemble into string--like structures which are aligned in the direction of the electric field~\cite{dassanayake2000,yethiraj2003,li2010}. This ``string fluid'' has been investigated analytically~\cite{elfimova2012} and can form the basis for producing ``colloidal polymers''~\cite{vutukuri2012,wei2016}. Meanwhile, it is possible to take the system out of equilibrium which enables investigation of string growth mechanisms~\cite{bensalah2023}, and aggregation phenomena between the strings~\cite{messina2023}. This suggests that the structure of the ``string fluid'' may be rather interesting and it has been investigated analytically~\cite{elfimova2012} and reciprocal space~\cite{pyanzina2007} in addition to real space~\cite{yethiraj2003,semwal2022}. 


In atomic and molecular systems, characterizing structure in the fluid state beyond pair correlations is challenging, although not impossible~\cite{royall2015physrep}. With colloidal systems, particle--resolved studies~\cite{ivlev} which deliver coordinates in real space are amenable to the measurement of higher--order correlation functions, relevant to a variety of phenomena such as dynamical arrest~\cite{vanblaaderen1995,royall2008,leocmach2013,royall2015physrep} and polymorph selection~\cite{taffs2016,leoni2021,gispen2023} and crystal precursors ~\cite{russo2012,leoni2021}. Furthermore, theoretical treatments have been developed to describe the higher--order structure of hard sphere colloids~\cite{hansen,robinson2019prl,robinson2019pre}. 


Methods to characterize higher--order structure include three--body correlation functions such as $g_3$~\cite{hansen,russ2003,royall2015prl}, and higher--order correlations such as common neighbor analysis (CNA)~\cite{honeycutt1987} and Voronoi face analysis~\cite{tanemura1977}. These have been shown to be successful in studying the structure of fluids and glasses~\cite{jonsson1988,bailey2004}. Another strategy, the bond orientation order (BOO) parameters developed by Steinhardt \emph{et al.} \author{steinhardt1983}~\cite{steinhardt1983}, focuses on the local symmetry around a central particle. This method has been shown to be very useful in the study of crystallization, especially in identification of small crystalline clusters in a supercooled liquid~\cite{auer2004,kawasaki2010} and also in the characterization of fivefold symmetric order in amorphous systems~\cite{vanblaaderen1995}.


With the popularity of machine learning rising, it has been applied effectively to local structure in amorphous materials, for example by combining many structural metrics such as the pair correlation function~\cite{cubuk2015,leoni2021}. Other examples include combining it with local descriptors such as CNA and BOO to better characterize the local environment around a single particle in disordered materials~\cite{boattini2020,boattini2019jcp,leoni2021}. Both supervised~\cite{gerardo2021} and unsupervised learning~\cite{joris2020} have been used to further our understanding of supercooled liquid and glass forming systems. 


The methods discussed above are geometric in nature. An alternative approach, which takes into account the interactions between the constituent particles of the system, has its roots in the work of Sir Charles Frank~\cite{frank1952}, who postulated that since the minimum potential energy configuration of 13 Lennard-Jones atoms corresponds to an icosahedron, that this would be a common geometric motif in (supercooled) liquids. With the advent of energy landscape calculations~\cite{wales}, it has become possible to determine the structure of minimum potential energy clusters for a wide range and size of systems including the Lennard-Jones~\cite{wales1997} and Stockmayer~\cite{miller2005} models. Since the dipolar interaction of the latter is not constrained to lie in any particular direction it thus corresponds to a molecular (or nanoparticle~\cite{novak2019}) system, rather than a colloidal dipolar system in an external field in the context of the discussion above. 


Identifying local arrangements of particles in bulk systems whose bond network is identical to such clusters can be carried out using the \emph{topological cluster classification} (TCC)~\cite{malins2013tcc,skipper2023}. The TCC has been used to identify locally favored structures or minimum energy clusters in systems undergoing dynamic arrest~\cite{royall2008}, colloid-polymer mixtures interacting via Morse potential~\cite{taffs2010,klix2013}, colloidal suspensions with attractive interactions~\cite{taffs2010jcp}, colloidal gels~\cite{royall2015prl,malins2010} and the liquid--gas interface~\cite{godonoga2010}.
%The colloidal dipolar system is interesting from the perspective of using minimum energy clusters to characterize the local structure because, unlike the systems mentioned above, the interaction potential is anisotropic and changes as a function of electric field strength. 
Some of us have recently determined the minimum energy clusters for a Lennard--Jones--dipolar system where the dipoles are induced in a particular direction (Fig.~\ref{figTCCClusters}) ~\cite{skipper2023}, which opens the possibility to use this method to probe the higher--order structure of this system.


Herein, we report a combined experimental and computer simulation study of dipolar (nearly) hard sphere colloids in the string fluid phase with a range of methods of quantifying structure. Since the electric field can be tuned at will, it is possible to vary the state point of the system \emph{in situ}. This is somewhat unusual for colloidal systems, where the state point is often fixed by the composition of the system. Here we explore the equilibrium string fluid phase, but we can also increase the field such that the system becomes metastable to fluid-body-centered tetragonal phase coexistence~\cite{hynninen2005}. We consider pair correlations in the form of radial distribution functions $g_2(r)$ and three-body correlations in the form of order parameters to determine ``string-like'' configurations and also the triplet correlation function $g_3(r,r',\eta)$. Finally we use the topological cluster classification~\cite{malins2013tcc,skipper2023} to explore higher--order spatial correlations in the form of minimum energy clusters of the dipolar--Lennard--Jones interaction~\cite{skipper2023}.




\begin{figure}
\includegraphics[width=85mm]{figTCCClusters.pdf}
\caption{Rigid minimum energy clusters of the dipolar--Lennard-Jones system for various sizes $m$.
8B, 9B, 10B, 11C and 12B are minimum energy clusters for the Lennard--Jones system ($\gamma=0$). 
Different geometries correspond to minimum energy clusters at different values of the dipolar strength $\gamma$. Here we consider rigid clusters only. The clusters are formed from rings of three, four or five particles. These are coloured grey. In the axis perpendicular to the rings are so--called spindle particles, colored yellow. Other particles are colored red~\cite{malins2013tcc}. ~\cite{skipper2023}.}
\label{figTCCClusters}
\end{figure}




\section{Dipolar Interactions in Colloidal Systems}


The colloids in our experiments are suspended in an index-matching solvent with added salt, as described in Sec:  \ref{sectionExperimental}. The interaction between such colloids can be approximated by a combination of a hard core Yukawa interaction~\cite{royall2003,royall2006} and dipolar interaction~\cite{yethiraj2003}. The Yukawa term takes the from:


\begin{align}
\label{eqYuk}
\beta u_\mathrm{yuk}(r) & =
\begin{cases}
\infty & \mathrm{for}\ r \le \sigma \\[2px]
\beta \varepsilon_\mathrm{yuk} \frac{\exp\left(-\kappa(r_{ij}-\sigma \right)}{r/\sigma}  & \mathrm{for}\ r > \sigma
\end{cases}
\end{align}


\noindent
where $\beta$ is the inverse of the thermal energy $k_BT$ with $k_B$ the Boltzmann constant, $T$ temperature, $u_\mathrm{yuk}$ the Yukawa interaction potential, $r$ the separation between two particles, $\sigma$ the hard-core particle diameter and $\kappa$ the inverse of the Debye screening length ($\kappa^{-1}=1/\sqrt{8\pi\lambda_{B}c}$ where $c$ is the number density of monovalent ions). $\beta \varepsilon_\mathrm{yuk}$ is the potential at contact and can be expressed as:


\begin{equation}
\beta \varepsilon_\mathrm{yuk} =\frac{Z^2}{(1+\kappa\sigma/2)}\frac{\lambda_B}{\sigma}
\end{equation}


\noindent
where $Z$ is the particle charge and $\lambda_B = e^2/4\pi \varepsilon_0 \varepsilon_m k_B T$, the Bjerrum length of the suspending medium with dielectric constant $\varepsilon_m$. Here $\varepsilon_0$ is the permittivity of the vacuum.


When colloids are subjected to an external electric field, a dipole-dipole interaction is induced which takes the form 


\begin{equation}
\beta u_\mathrm{dip}(r,\theta)= \frac{\gamma}{2} \left(\frac{\sigma}{r}\right)^3 \left(1-3 \cos^2 \theta \right)
\label{eqUDipole}
\end{equation}


\noindent
where $u_\mathrm{dip}$ is the dipolar interaction, $\theta$ is the angle made by $\mathbf{r}$ and the $z$-axis. In our experimental system, $\gamma=\gamma_\mathrm{exp}$ is a dimensionless prefactor that depends on the strength of the external field and material properties of the system. Here $\mathbf{r}$ is the vector connecting the centers of the two particles. 


\begin{equation}
\gamma_\mathrm{exp}=\frac{\mathbf{p}^2}{2\pi \varepsilon_m \varepsilon_0 \sigma^3 k_B T}
\label{eqGamma}
\end{equation} 


\noindent
where 


\begin{equation}
\mathbf{p}=\frac{\pi}{2} \beta \varepsilon_m \varepsilon_0 \sigma^3 \mathbf{E}
\label{eqDipoleMoment}
\end{equation} 


\noindent
is the dipole moment. $\varepsilon_m$ is the dielectric constant of the suspending medium and $\mathbf{E}$ is the electric field. 

Combining with the Yukawa interaction Eq. \ref{eqYuk} the total interaction between two colloids under the electric field becomes


\begin{equation}
\beta u_\mathrm{total}= \beta u_\mathrm{dip} + \beta u_\mathrm{yuk}.
\label{eqUTotal}
\end{equation}




\section{Methods}


\subsection{Experimental}
\label{sectionExperimental}


The colloidal suspension used in this experiment was prepared by adding sterically-stabilised polymethyl methacrylate (PMMA) spheres (synthesized following reference~\cite{campbell2002,hollingsworth2006}) ($\rho$ = 1.196 gcm$^{-3}$)~\cite{royall2005s} of diameter $\sigma=1.73$ $\mu$m {(polydispersity around 5\%)~\cite{donovan} ~\footnote{The polydispersity determined for these particles was 3.8\% with static light scattering and 9.1\% from TEM~\cite{donovan}. That the particles crystallize readily suggests that their polydispersity was around 5\% or less.}
to a mixture of density and refractive index matched solvents. The solvent is a mixture of \emph{cis}-decalin ($\rho \approx $%= 
0.897 gcm$^{-3}$) and cyclohexyl-Bromide (CHB) ($\rho$ = 1.32 gcm$^{-3}$). Tetrabutylammonium bromide (TBAB) salt was dissolved in the solvent to make up a solution with TBAB concentration of 260 $\mu$M. This corresponds to a Debye length $\kappa^{-1}$ of around 100 nm~\cite{taffs2013}.  Since the Debye length is much less than the particle diameter, in the absence of an electric field, the colloids behave as nearly hard spheres~\cite{taffs2013,royall2013myth}. While more sophisticated treatments may be carried out to match the interaction potential~\cite{taffs2013,royall2013myth}, here we use a slightly soft potential in the computer simulations and presume this to be sufficient to match the experimental system, noting that the effects we seek to study are dominated by the dipolar interactions, rather than the hard core or precise colloid volume fraction (which we determine by weighing out the samples) ~\cite{poon2012}.


We determine the dipolar contribution to the interaction potential between the particles by evaluating Eqs.~\ref{eqGamma} and ~\ref{eqDipoleMoment} with the particle diameter $\sigma$, the solvent dielectric constant $\epsilon_m=5.6$~\cite{leunissenThesis} and the measured value of the local electric field $\mathbf{E}$. We emphasize that the resulting values of $\gamma_\mathrm{exp}$ have no fit parameters and are purely dependent on the material properties of the system. See Sec.~\ref{sectionDiscussion} for further discussion as to the importance of the absence of fit parameters.


In order to construct the electric cell to hold the colloidal suspension, two indium tin oxide glass slides were separated with spacer silica particles of approximately 60 $\mu$m in diameter to create a transparent, electrically conductive cell. The electrodes were connected to a signal generator that supplies alternating currents across the electrical cell. 


A Leica SP8 confocal microscope was used to monitor the system under the applied electric field. During each measurement, a stack of 3d confocal images of at least 200 ``slices'' of $xy$ images along the $z$ direction were taken with 256$\times$256 pixels with at least ten pixels per particle diameter in all directions. Following related work with dipolar colloids~\cite{yethiraj2003,leunissenThesis}, only particles at least ten diameters from the wall were analyzed, to ensure that there were no significant wall effects. We saw no influence from the wall proximity in any of our measurements and we conclude that we can treat the system as bulk, as is typical for such particle--resolved studies. It is worth noting that the system sizes used in this experimental technique are not huge~\cite{ivlev}. The shortest dimension of the system (here 60 $\mu$m) is two orders of magnitude larger than that of the colloidal particles.


In each measurement, the applied voltage and the thickness of the electrical cell was measured in order to allow electric field strength comparison across different experiments. Before each measurement, the system is allowed to stabilise for at least 20min. The Brownian time $\tau_B=(3 \pi \nu \sigma^3)/(4 k_B T) \approx 6.09$ s for our system, so we consider this time quite sufficient to relax equilibrium states. Here $\nu$ is the solvent viscosity. Out of equilibrium, for high field strengths, there will likely be some dependence on the history of the system, to which we return below in Secs.~\ref{sectionResults} and ~\ref{sectionDiscussion}. To obtain sufficient statistics at least 50 3d images each separated by 30s were taken for each state point.
 



\subsection{Particle Tracking}
 
 
Here we use a slight modification to enhance the accuracy of the coordinates that we detect~\cite{statt2016}. We begin by carrying out a conventional centroid location~\cite{crocker1995,leocmach2013sm}. This seeks the brightest pixels and weights the brightnesses of the surrounding pixels to obtain an estimate of the centre of the colloidal particle. Overlaps corresponding to multiple pixels within a single particle being identified are removed. This method works well in (quasi) 2d studies~\cite{royall2023}, but in the case of the 3d confocal microscopy that we carry out here, overlaps between blurred images of particles in the $z$-direction can be a problem.


To mitigate such blurring in the $z$--direction, we refine the first set of coordinates determined as described above as follows. Knowing the size of the particles, the algorithm predicts an image based on the set of particle coordinates. This predicted image is then iteratively compared with the original measured image and the coordinates moved following a Monte Carlo method using the difference in pixel values between the predicted and measured image to minimize the differences between them. Further details of our method, including the source code may be found in ~\citet{yang2021}.


Colloid tracking is subject to errors in the location of the coordinates of the particles. Combined with polydispersity in the particle size distribution, this can influence structural measurements as we carry out here. In the case of 2-body correlation functions, the effect of polydispersity and tracking errors can be similar to a convolution~\cite{royall2007jcp}. In the case of amorphous systems, the effect of (mild) polydispersity has been investigated and this was found to have only a minor effect on the higher-order structure~\cite{royall2012}.





\begin{figure}
\includegraphics[width=\linewidth]{figBopString.pdf}
\caption{The colloidal dipolar system. 
(a-d) Representative images using confocal microscopy in the horizontal $xy$ (a,b) and vertical $xz$ (c,d). Here volume fraction $\phi$=0.1.
(a,c) No field, $\mathbf{E}=0$. 
(b,d) Confocal microscopy images of a system at the same volume fraction at the maximum electric field strength 200 Vm$^{-1}$ ($\gamma=46$).  
(e) String fluid order parameter ($\langle {\cos^2{\theta}} \rangle$) as a function of external electric field strength $E$. The inset indicates the angle $\theta$.
Data are shown for experiments (data points) and simulations (lines) for volume fractions $\phi=0.1$ and $0.3$ as indicated.
Scale bars in (a-d) are 20 $\mu$m.
}
\label{figBopString}
\end{figure}





\subsection{Computer Simulation}


We employed computer simulations using the LAMMPS package at constant volume, in the NVT ensemble~\cite{plimpton1995}. Simulations are performed with periodic boundary conditions and the system was evolved using Brownian dynamics~\cite{moore2021,moore2023}. To reproduce the (nearly) hard sphere behavior of the colloids, we use the Weeks-Chandler-Anderson (WCA)~\cite{weeks1971} potential. This takes the form: 
\begin{align}
\label{eqWCA}
\beta u_\mathrm{wca}(r_{ij}) &=
\begin{cases}
\beta 4 \varepsilon_\mathrm{wca}[(\frac{\sigma}{r})^{12} - (\frac{\sigma}{r})^6] + \varepsilon_\mathrm{wca}  & \ r \le 2^{\frac{1}{6}}\sigma \\[2px]
0 &\ r > 2^{\frac{1}{6}}\sigma
\end{cases}
\end{align}
where $\beta \varepsilon_\mathrm{wca}=10$ is the interaction energy. 


We added the dipole-dipole interaction shown in Eq. \ref{eqUDipole}, the Ewald sum for which is implemented with the KSpace package in LAMMPS~\cite{plimpton1995}. Here $\gamma=\gamma_\mathrm{sim}$ controls the strength of  the dipolar interaction. The interaction potential for the simulations then reads 


\begin{equation}
\beta u_\mathrm{sim}= \beta u_\mathrm{wca} + \beta u_\mathrm{dip}.
\label{eqUSimulation}
\end{equation}



Here we quote simulation results in reduced Lennard-Jones units, that is to say the unit of length is the diameter $\sigma$, and we set $k_BT=1$,  and time is in units of $m \sigma^2/\epsilon_\mathrm{wca}$. We use the Barker-Henderson effective hard sphere diameter of the WCA component of the interaction to determine the volume fraction in order to match the experiments. Each simulation run includes at least 3000 particles and was run for 1000 Lennard--Jones time units before being sampled for a further 100 time units. When comparing simulation data and experimental data, please note that we do not add polydispersity nor tracking errors to the simulations. While these might improve agreement between experiment and simulation}, particle tracking errors can be hard to quantify, as these are sample-dependent~\cite{royall2023} and the value obtained for the polydispersity can also depend on the method used~\cite{poon2012,donovan}.






\subsection{Bond order parameters for dipolar colloids}
\label{sectionBond}


One method to quantify the angular correlation between particles as a function of external electric field strength is the string fluid order parameter~\cite{li2010}. This has already been shown to be sensitive to variation in field strength by Li \emph{et al.}~\cite{li2010}. These are calculated by finding the angle $\theta$ made by a reference particle with its two nearest neighbors [see Fig. \ref{figBopString}(c) inset]. Here we take $\langle \cos^2\theta \rangle$ as the order parameter. For a system consisting of perfect chains,  $\langle \cos^2\theta \rangle$ = 1.



\subsection{Two--  and three--body correlation functions}
\label{section23}


We also calculated the two--body spatial correlation function, the radial distribution function $g_2(r)$. 
In addition to the isotropic $g_2(r)$, we consider $g_{2xy}(r)$ which measures correlations in the $xy$ plane perpendicular to the field and $g_{2z}(z)$, which measures correlations in the $z$ direction along the field. For $g_{2xy}(r)$, only pairs of particles that are perpendicular to the $z$-axis with a tolerance of $\pm 5^\circ$ were used. In the case of $g_{2z}(z)$, particle pairs that are parallel to the $z$-axis were chosen with a $\pm 5^\circ$ tolerance. 


We also consider the 3--body spatial correlation function $g_3$. Now this depends on the positions of three particles $1,2,3$, i.e. three vectors, eg $g_3(\mathbf{r}_{12},\mathbf{r}_{23},\mathbf{r}_{31})$ with the numbers reflecting the three particles. Here we elect to simplify our representation to the case where we fix two of the distances (to the particle diameter $\sigma$) so that the 3--body correlation function is plotted as a function of angle between them $g_3(r_{12}=\sigma,{r}_{23}=\sigma,\eta)$ where $\eta$ is the angle between $\mathbf{r}_{12}$ and $ \mathbf{r}_{23}$.







\subsection{Topological Cluster Classification}
\label{sectionTCC}

  

As discussed in the introduction, the topological cluster classification identifies local geometric motifs whose bond topology (defined here through a modified Voronoi decomposition) is identical to that of minimum energy clusters of a specific size~\cite{malins2013tcc}. These clusters are identified using the energy optimization algorithm GMIN which uses \emph{basin-hopping} to find the local energy minimum that corresponds to a specific configuration for a number of particles in isolation~\cite{wales1997}.  Now such minimum energy clusters require an attractive interaction, and therefore to investigate the effect of the dipolar interaction, clusters were determined for a dipole added to a Lennard--Jones interaction~\cite{skipper2023}. That is to say, the interaction potential for which the minimum energy clusters were determined was
\begin{equation}
\beta u_\mathrm{tcc}= \beta u_\mathrm{lj} + \beta u_\mathrm{dip}.
\label{eqUTCC}
\end{equation}
\noindent
where $ u_\mathrm{lj}$ is the Lennard--Jones (LJ) interaction. Although the attractive Lennard--Jones contribution is not part of the experimental (or simulated) system we consider here, the importance of packing effects on the structure of liquids and dense fluids has a long history~\cite{barker1976}, and it has recently been shown that the higher--order structure of dense hard spheres is closely related to those of attractive systems~\cite{robinson2019prl,robinson2019pre}. In fact, the higher--order structure of the Lennard--Jones and WCA systems, along with hard spheres, as determined through the topological cluster classification are rather similar~\cite{taffs2010jcp}. We therefore expect that the use of dipolar--LJ clusters will likewise be reasonable here and in any case will provide a suitable measure of the change in the fluid structure under the electric field. 


Those resulting clusters which are rigid~\cite{skipper2023} are shown in Fig. \ref{figTCCClusters}. In the case of zero field, we have the minimum energy Lennard--Jones clusters 8B, 9B, 10B, 11C and 12B~\cite{wales,doye1995}. The clusters ending with PAA consist of elongated polytetrahedral clusters (9PAA and 10PAA). Meanwhile 
the S-clusters are based on five--membered rings (9S, 10S, 11S and 12S). The dipolar rigid clusters 8O, 9S, 9PAA, 10S, 10PAA, 11S, 11SB, 11O, 12S, 12SB and finally12O are the minimum energy rigid clusters of particles interacting according to Eq. \ref{eqUTCC}. In our analysis of the TCC clusters, we set the Voronoi parameter $f_c=0.82$~\cite{malins2013tcc}. Further details are available in Ref.~\cite{skipper2023}.




\subsection{Orientation of anisotropic clusters}
\label{sectionOrientation}


Since the dipolar interaction is anisotropic, the clusters found by the TCC may have a preferred orientation with respect to the direction  of the electric field. We consider clusters found in both experiments and simulations. We calculated the principal axis of each cluster and determine the angle made by its principal axis with respect to the direction of the electric field. To quantify this angle distribution, we use an order parameter commonly used for liquid crystals,


\begin{equation}
\langle P_2(\cos\alpha) \rangle = \left\langle \frac{3}{2}\cos^2\alpha- \frac{1}{2} \right\rangle.
\label{eqString}
\end{equation}



\noindent
where $\alpha$ is the angle made by the principal axis of each cluster with the direction of the electric field~\cite{selinger2016}. In the isotropic case when the field is switched off, there should be no preferred orientation. In the perfectly aligned state, all the clusters should lie parallel to the electric field so this order parameter is $1$. 




\section{Results}
\label{sectionResults}





\begin{figure*}
\includegraphics[width=140mm]{figRadial.pdf}
\caption{Pair correlations  in the colloidal dipolar system. Data are shown for both experiment and computer simulations at volume fraction (a), (c), (e) $\phi = 0.1$ and (b), (d), (f) $\phi =0.3$. Lines show different field strengths expressed through the parameter $\gamma$ as indicted. Higher field strengths at which the system falls out of equilibrium are indicated by pink points (experiment) and grey lines (simulation), while lower field strengths are shown as red points and black lines. Data are offset for clarity. (a) and (b) show $g_2(r)$, (c) and (d) show $g_{2xy}(r)$ by considering correlations in the $xy$ plane, (e) and (f) show $g_{2z}(r)$ 
where correlations are taken along the $z$ axis.}
\label{figRadial}
\end{figure*}




We now present both experimental and simulation results for colloidal dipolar fluids at volume fractions $\phi=0.1$ and $\phi=0.3$. We investigate the bond order parameters, two and three body correlation functions, and populations of minimum energy clusters and the orientations of these clusters. These quantities are considered as a function of dipole strength $\gamma$. We convert external field strength to the dipole strength $\gamma$ using Eq. \ref{eqGamma}, which we use across experiments (Eq. \ref{eqDipoleMoment}), simulations (Eq. \ref{eqUSimulation}) and minimum energy clusters (Eq. \ref{eqUTCC}).





\subsection{Bond order parameter analysis of the string fluid}
\label{sectionBopAnalysis}


In our study, with the bond order parameter $\langle \cos^2 \theta \rangle$ (see Sec. \ref{sectionBond}), we explore slightly higher colloid volume fraction (0.1 and 0.3) than some previous work~\cite{li2010}. Our results are shown in Fig. \ref{figBopString} and for the string order parameter, we find reasonable agreement between experiment and simulation, considering that there is no fit parameter. Both are small in the case of zero field, and then show a significant increase with field strength to a value of  $\langle \cos^2{\theta}\rangle \sim 0.8$ for $\phi=0.3$ and $\sim 0.65$ for $\phi=0.1$. 


Upon increasing the field strength such that $\gamma\approx12$ and $\approx10$ for volume fraction $\phi=0.1$ and $0.3$ respectively, the string fluid becomes metastable to fluid-body-centered tetragonal phase coexistence~\cite{hynninen2005}. Since the system is now out of equilibrium, it is possible that structural differences between the experiments and simulations may emerge. Here we see that some evidence of a plateau in the experimental data, while in the case of the simulation $\langle \cos^2{\theta}\rangle$,  continues to increase with field strength. 




\subsection{Pair Correlations: the Radial Distribution Function}
\label{sectionPair}


We continue our analysis by considering pair correlations in the form of the radial distribution function $g_2(r)$ in Fig.~\ref{figRadial}. The orientationally averaged $g_2(r)$ is shown in Figs.~\ref{figRadial}(a) and (b) for volume fraction $\phi=0.1$ and 0.3 respectively for both experimental (data points) and simulation (lines). At zero field strength, in the hard sphere limit, we see reasonable agreement between experiment and simulation (as has been noted previously)~\cite{royall2023}. The slightly higher first peaks of the simulation data may be attributed to the particle tracking errors and polydispersity in the experiments~\cite{royall2007jcp}. For weak field strengths ($\gamma=7$), again we see comparable agreement between experiment and simulation to that of the hard sphere case.



As above in Sec.~\ref{sectionBopAnalysis}, at high field strengths when the system falls out of equilibrium, we see some discrepancy between experiment and simulation. In particular, we see stronger peaks, ie stronger ordering, in the simulation data for $\gamma=16$ and $45$ for $\phi=0.1$. We believe that this difference is too large to be attributed to tracking error and polydispersity. Interestingly, the difference between experiment and simulation is rather less significant in the case of $\phi=0.3$. We return to consider possible influences in the structure out of equlibrium in Sec. \ref{sectionDiscussion} below.


Since our system is anisotropic due to the external field, we expect to see some corresponding differences in structure between the plane perpendicular $xy$ and the direction parallel $z$ to the field. To explore this we now consider the pair correlation function in the $xy$ plane $g_{2xy}(r)$ and $z$ direction $g_{2z}(z)$. We expect that the formation of the string fluid may lead to significant structuring in the field direction, while in the perpendicular $xy$ plane, there is repulsion between particles exactly in plane, but out--of--plane attractions lead to aggregation of the strings~\cite{messina2023} and ultimately the condensation of the bct crystal. In Fig.~\ref{figRadial}(c,d), we show the pair correlations in the perpendicular plane for volume fraction $\phi=0.1$ and $0.3$ respectively. In the case of $\phi=0.1$ [Fig.~\ref{figRadial}(c)], at zero field strength we encounter some discrepancy between experiment and simulation in that the $g_{2xy}(r)$ rises from zero at a smaller value of $r$ in the experiment than in the simulation. Such a discrepancy is often associated with particle tracking errors.


In the string fluid at low field strength ($\gamma=7$), the agreement between experiment and simulation is quite good. Upon increasing the field strength, such that the system falls out of equilibrium, for $\gamma=16$, a broad first peak is seen in both experiment and simulation, which may reflect some condensation of the ``strings'' of particles in the $xy$ plane. This peak grows upon increasing the field strength and in the simulations, a split peak is seen, which is absent in the experiments, indicating increased ordering, which may be indicative of some bct crystal-like regions forming.


For the higher volume fraction case [Fig.~\ref{figRadial}(d)], at zero field, a similar discrepancy to the  $\phi=0.1$ case is found. However at all other field strengths, a rather good agreement between experiment and simulation is found. For the non-equilibrium state points, $\gamma=16,45$, the agreement between experiment and simulation is rather better than in the case of the lower volume fraction, with the former capturing the split peak of the latter. The small discrepancy in heights of the peaks we believe may be attributed to particle tracking errors.


Turning to the case of the correlations along the field direction, $g_{2z}(z)$  [Figs.~\ref{figRadial}(e) and (f)], for zero field strength, we recall that we may expect to see evidence of significant ordering upon application of the electric field, as the string fluid develops. At zero and low field strength, correlations are comparatively weak for both $\phi=0.1$ and $0.3$ [though note the scale on the $y$-axis in Figs.~\ref{figRadial}(e) and (f)]. At higher field strength we find rather good agreement between the experiments and simulations. The slightly higher peaks in the case of the latter we attribute to particle tracking errors and polydispersity. It is possible that particle tracking errors account for the difference in position of the peaks between experiment and simulation for [Figs.~\ref{figRadial} (f)] for $\gamma=7$.


Summarizing the behavior we have uncovered through our analysis of the pair correlations, we find overall reasonably good agreement between experiment and simulation when the system is in equilibrium. We suggest that discrepancies between experiment and simulation may be attributed to particle tracking errors and polydispersity~\cite{royall2007jcp,royall2023}.  As anticipated, strong ordering is seen in the field direction as revealed by $g_{2z}(z)$. When the field strength is increased such that the system becomes metastable to fluid-bct coexistence~\cite{hynninen2005}, we see more significant discrepancies, for example a split peak in the simulations which is not found in experiment.





\subsection{Three-body Correlations}
\label{sectionThree}



\begin{figure*}
\includegraphics[width=170mm]{figG3.pdf}
\caption{The three--body correlation function $g_3(\eta)$. 
(a) Schematic indications of geometries of interest, with values of the bond angle $\eta=60^{\circ}$.120$^{\circ}$ and $180^{\circ}$
(b) $g_3(\eta)$ is plotted for volume fraction $\phi=0.1$.
(c) $g_3(\eta)$ for volume fraction $\phi=0.1$.
In (b) and (c), data points are experimental data and lines are simulation data.}
\label{figG3}
\end{figure*}



As noted above, particle--resolved studies lends itself to analysis of higher--order correlations~\cite{vanblaaderen1995,royall2008,leocmach2013,royall2015physrep,royall2023} and here we begin with the three--body correlation function $g_3$. This may be represented in a variety of ways and here we chose to show the dependence of $g_3$ upon the angle $\eta$ between two particles with respect to a particle of interest as shown in Fig.~\ref{figG3}(a). We set the distance between the two particles and the particle of interest to be the diameter, so that the quantity plotted is $g_3(\eta)$. 


For both $\phi=0.1$ and $0.3$, at zero and low field strength, a large and broad peak appears at 
$\eta\approx60^{\circ}$ which is consistent with an isotropic system where the interaction is angle independent~\cite{coslovich2013jcp}. However, at larger angles approaching $\eta\approx180^{\circ}$ we find a broad peak in the simulation data which is not found in experiments. One would expect a strong signal from particles aligned in strings for $\eta\approx180^{\circ}$ indicating that these are less well-defined in the experiments for $\gamma=7.$
We return to this discrepancy below in Sec.~\ref{sectionDiscussion}.


As the field strength increases, peaks emerge at {angles $\eta\approx60^{\circ}$, $\eta\approx120^{\circ}$ and $\eta\approx180^{\circ}$ for both $\phi=0.1$ and $0.3$ [Figs.~\ref{figG3}(b) and (c) respectively]. As noted, $\eta\approx180^{\circ}$ corresponds to string formation. While this could in principle be directly investigated by constraining $g_3(\eta)$ to lie along the direction of the electric field, we argue that such a measure would be similar to Fig.~\ref{figBopString} and in any case such an analysis becomes statistically challenging with our experimental data. The peaks at angles $\eta\approx60^{\circ}$ and $\eta\approx120^{\circ}$ are in any case indicative of further ordering. We see that there is considerably more structure to be found in the simulation data with stronger peaks at a field strength corresponding to $\gamma=16$ and also additional peaks at an angle $\eta\approx90^{\circ}$ which are not seen in the experiments. At even higher field strength ($\gamma=45$), we see a split peak at $\eta\approx60^{\circ}$ in the simulations which is absent in the experiments. As we noted above, for $\gamma=16$ and $45$, the system is out of equilibrium, which likely underlies the increased discrepancy that we see. Overall, the more detailed probing of the system with the three--body correlations highlights discrepancies between the experiments and simulations that were somewhat less obvious in the case of the two--body data presented in Sec.~\ref{sectionPair}.}









\subsection{Populations of minimum energy clusters}
\label{sectionTCCAnalysis}


\begin{figure*}
\includegraphics[width=150mm]{figTCCPopulationA.pdf}
\caption{Populations of smaller minimum energy clusters detected by the topological cluster classification as a function of dipole strength. The number of particles detected in a given cluster $N_c$ is scaled by the number of particles in a cluster of that size, $N_m$. Data points denote experiment and lines are computer simulation.
Shading denotes the change in cluster topology which minimizes the energy at different values of the dipole strength $\gamma$ as indicated~\cite{skipper2023}. White regions of graphs denote values of $\gamma$ where the minimum energy clusters are not rigid. Data are shown for different cluster sizes $m$ and volume fractions as follows. 
(a) Cluster size $m=8$, volume fraction $\phi=0.1$.
(b) $m=8$,  $\phi=0.3$.
(c) $m=9$,  $\phi=0.1$.
(d) $m=9$,  $\phi=0.3$.
(e) $m=10$,  $\phi=0.1$.
(f) $m=10$,  $\phi=0.3$.
}
\label{figTCCPopulationA}
\end{figure*}





We now move to still higher--order spatial correlations and consider the dipolar--Lennard--Jones minimum energy clusters identified by the topological cluster classification (Fig.~\ref{figTCCClusters})~\cite{skipper2023}. To facilitate comparison between experiment and simulation, in each figure we fix the number of particles in the cluster in question. Now the topology of the minimum energy cluster changes upon increasing the dipolar contribution~\cite{skipper2023}. In Fig.~\ref{figTCCPopulationA}, we consider 8, 9 and 10--membered clusters for volume fraction $\phi=0.1$ and $0.3$.  For larger clusters, there are fewer statistics at lower volume fraction and therefore, we focus on $\phi=0.3$ in Fig.~\ref{figTCCPopulationB}. 



Naively, one might expect that zero and low field strength would correspond to minimum energy clusters for the Lennard--Jones interaction as shown in Fig. \ref{figTCCClusters} and known from studies with hard spheres~\cite{taffs2013,royall2018jcp,royall2023,robinson2019pre}, and that increasing the field strength might lead to a cascade of clusters of increasing elongation as indicated in Fig. \ref{figTCCClusters}(b). This turns out to be the case.


For the smallest size of cluster we consider, $m=8$, indeed we see this trend with the Lennard--Jones minimum energy cluster 8B giving way to the 8O which minimises the potential energy for the dipolar system for both volume fractions [Fig.~\ref{figTCCPopulationA}(a,b)]. Notably, experiment and simulation appear reasonably well--matched in the lower field case that the system is in equilibrium ($\gamma \leq 12, \phi=0.1$ and $\gamma \leq 10, \phi=0.3$). By this we mean that discrepancies are typically within an order of magnitude (note that, for an 8-membered cluster to be successfully identified, failure to identify only one of the 8 particles will lead to the cluster not being identified, so ``agreement'' between experiment here is inevitably rather less stringent than in the case of pair correlations, say). The more significant discrepancy between the experiments and simulations emerges at higher field strength, $\gamma=16$ and $45$ in both volume fractions $\phi=0.1$ and $0.3$. We see an increase in the 8B modified pentagonal bipyramid population in the experiments compared to the simulations. Now this structure exhibits fivefold symmetry, and as such, is associated with non--crystalline ordering~\cite{frank1952}. We have noted above that, in the non-equilibrium conditions at higher field strength, the simulations appear more ordered. If the ordering in the simulations is crystalline, then a lack of five-fold symmetry with respect to the experiments would seem to be reasonable.


In the case of 9-particle clusters, a rather different trend is found [Fig.~\ref{figTCCPopulationA}(c,d)]. We find it instructive to start our analysis with $\phi=0.3$ [Fig.~\ref{figTCCPopulationA}(d)]. At low field strength $\gamma=0$ and $7$, there is a rather high population of 9S and 9PAA in both experiment and simulation. This dominates over the minimum energy cluster for the Lennard-Jones system, 9B which has two five--membered rings~\cite{malins2013tcc}. At higher field strengths, the population of 9B drops rather precipitously in the simulations, but less so in the experiments. This is consistent with the case for $m=8$ above, where the 8B cluster which, like the 9B, has a degree a fivefold symmetry is preferred. Notably though, the 9PAA dominates at all field strengths, which is not expected from energy considerations as it is the minimum energy cluster only for $12  \lesssim \gamma \lesssim 21$. Now the 9PAA cluster is polytetrahedral in structure, and is enlongated with respect to the 9B. This is consistent with it being intermediate between the compact 9B in the case of zero field and strings of particles in the case of a strong field. For $\phi=0.1$ [Fig.~\ref{figTCCPopulationA}(c)], there are rather fewer clusters identified. It is often the case that there are fewer (larger) clusters at lower volume fraction~\cite{malins2009,malins2010,klix2013,royall2023}, so this in itself is reasonable. Like the case of $\phi=0.3$, the 9PAA dominates at all field strengths. At high field strength, we see more of the clusters with five--membered rings, 9B and 9S in the experiments than in the simulations.


Turning to $10$-membered clusters [Fig.~\ref{figTCCPopulationA}(e,f)], we see a somewhat similar behavior to the $m=9$ case. For volume fraction $\phi=0.3$, in  Fig.~\ref{figTCCPopulationA}(f) for both experiments and simulations, we see some of the Lennard--Jones minimum energy cluster, the defective icosahedron 10B at low field strength as one might expect. However, at higher field strength $\gamma \gtrsim 12$, its population vanishes. Even at low field strength the 10B population is much smaller than the 10S and 10PAA. At high field strength, the 10PAA dominates with the 10S found in the experimental data (and absent in the simulations). Consistent with the smaller cluster sizes discussed above, the 10S features a five-membered ring, and we have noted that these are preferred in the experiments at higher field strength. For lower volume fraction $\phi=0.1$ [Fig.~\ref{figTCCPopulationA}(e)], we see mainly 10PAA and that only at high field strength. This latter observation is consistent with expectations, as the 10PAA is the minimum energy cluster for the dipolar--Lennard--Jones system for $12 \leq \gamma \leq 25$. 


We now consider larger clusters in Fig. \ref{figTCCPopulationB}(a,b). The statistics for larger clusters at the lower volume fraction are rather poor and therefore here we focus on a volume fraction  $\phi=0.3$ respectively. At low field strength, in experiment, we find 11S, 11SB and 11O dominating at zero field strength with rather less 11C, the latter being the minimum energy Lennard--Jones cluster. There is more 11C at $\gamma=7$, and none at higher field strength where 11O dominates. In simulations, 11S and 11SB are popular at low field strength, while like the experiments, 11O dominates at higher field strength. 11O is the minimum energy cluster at the highest field strength we consider, so its dominance for large $\gamma$ seems reasonable.


Finally, the 12-membered clusters.are shown in Fig. \ref{figTCCPopulationB}(b). Here in the experimental data, 12O, which is the minimum energy cluster for $19 < \gamma < 31$ is the most popular cluster at all field strengths. We see small quantities of 12B and 12SB at weak field strengths, but these vanish for field strengths greater than $\gamma = 7$.  The simulations exhibit the same qualitative trend, of increasing 12O as a function of field strength. However, for weak fields $ \gamma < 10$, 12SB is the most popular cluster. 12S is found in small quantities at weak field strength (although this cluster is not found at all in experiment).





\begin{figure*}
\includegraphics[width=150mm]{figTCCPopulationB.pdf}
\caption{Populations of larger minimum energy clusters detected by the topological cluster classification as a function of dipole strength. As in Fig. \ref{figTCCPopulationB}, the number of particles detected in a given cluster $N_c$ is scaled by the number of particles in a cluster of that size, $N_m$. Data points denote experiment and lines are computer simulation. Shading denotes the change in cluster topology which minimizes the energy at different values of the dipole strength $\gamma$ as indicated~\cite{skipper2023}. White regions of graphs denote values of $\gamma$ where the minimum energy clusters are not rigid. Data are shown for different cluster sizes $m$. 
(a) Cluster size $m=11$, 
(b) $m=12$. 
Here volume fraction $\phi=0.3$.}
\label{figTCCPopulationB}
\end{figure*}






\subsection{Cluster Orientation}
\label{sectionOrientationResults}


We have observed that even at zero dipole strength, some of the minimum energy dipolar clusters (such as 10PAA at $\phi =0.3$) are present in our system [Fig.~\ref{figTCCPopulationA}(f)]. Now the dipolar--Lennard--Jones clusters that we consider (Fig.~\ref{figTCCClusters}) are aligned with the dipolar interactions, and thus with the electric field. It is therefore reasonable to suppose that the clusters might exhibit some alignment with the field, and that this would increase as a function of field strength.


We therefore probe the orientation of some dipolar--Lennard--Jones clusters using our method described above in Sec.~\ref{sectionOrientation}. 
The principle is illustrated in Fig.~\ref{figOrientation}(a) and (b). Here, renderings of experimental data where clusters identified as 10PAA for volume fraction $\phi=0.3$ are shown. In the case of zero field strength [Fig.~\ref{figOrientation}(a)], no preference in cluster orientation is seen. For $\gamma=45$ [Fig.~\ref{figOrientation}(b)], we see that the clusters have oriented with the field. We now explore this phenomenon quantitatively. In Fig.~\ref{figOrientation}(c), for volume $\phi=0.1$ fraction we plot the degree of alignment with the field for experimental data (data points) and simulation (lines). The 9PAA and 10PAA clusters exhibit a higher degree of alignment than do the 8O and 12O in the case of the experiments. Also, the increase in alignment with field strength is not particularly marked, with an increase of around 20\%. In the case of the simulations, the trend is similar, although the alignment of the 12O cluster is the highest at high field strengths.


Turning to the higher volume fraction $\phi=0.3$ in Fig.~~\ref{figOrientation}(d), we see a stronger increase in alignment with the field as a function of field strength. Again the 9PAA and 10PAA show the highest degree of alignment, but (unlike the lower volume fraction), 12O shows a comparable alignment at higher field strengths. The simulation data show a somewhat sharper rise in the cluster alignment as a function of field strength than do the experiments. And overall, the degree of alignment with the field is higher than for the lower volume fraction.




\begin{figure*}
\includegraphics[width=\linewidth]{figOrientation.pdf}
\caption{Orientation of anisotropic clusters with the electric field.
(a) shows a 3d plot of the clusters (10PAA) found in the experiments at $\phi=0.3$ at zero field, where the principal axis (indicated as the black line) of each cluster does not align with the electric field. Whereas in (b) these 10PAA clusters showed a higher degree of alignment along the %E-
field ($z$-axis) as the field takes a value of $\gamma=40$.
(c,d) $\langle P_2(\cos\alpha) \rangle $ of the principal axis is plotted as a function of reduced field strength, for the anisotropic clusters 8O, 9PAA, 9S , 10PAA, 11O and 12O. for the two volume fractions under consideration.
(c) $\phi =0.1$ and (d) $\phi =0.3$. Data points are experimental data and lies are computer simulation.
}
\label{figOrientation}
\end{figure*}



\section{Discussion}
\label{sectionDiscussion}


We now discuss our findings figure by figure.


(\emph{i})
Figure \ref{figBopString} shows confocal microscopy images of our system at volume fraction $\phi$=0.1, taken along the $xy$ plane (perpendicular to the direction of the electric field) and $yz$ plane (along the direction of field) for both zero dipole strength and at maximum dipole strength at $\gamma$=45. We see string formation along the direction of the field at $\gamma$=45.


Now our system becomes metastable to fluid-bct crystal phase coexistence at $\gamma \geq 10$ and $12$ for volume fraction $\phi=0.1$ and $\phi=0.3$ respectively. Under these conditions, given that we do not attempt to treat the time--evolution of the system (see the discussion below), discrepancies between the experiments and simulations are possible due to each taking a different route through the energy landscape.


Figure \ref{figBopString}(c) shows a plot of bond order parameters of the string fluid from both our experiment and simulation results, similar to the study published by Li \emph{et al.}~\cite{li2010}. Again, our simulations (line) and experiments (data points) show quite good agreement. As the field strength increases, the bond angle $\theta$  
tends towards 180$^{\circ}$. As indicated in Fig. \ref{figBopString}(c), the degree of string formation increased gradually as opposed to a sharp increase like a step function. 


(\emph{ii}) 
In Fig. \ref{figRadial} we plot $g_2(r)$. This is generally in reasonable agreement between computer simulations and experiments at for both $\phi$=0.1 and $\phi$=0.3 (with the only exception at $\phi$=0.1, $\gamma$=45). We can therefore be fairly confident that the simulation model used in our work is a reasonable reflection of our experimental system. We see the emergence of long range order as field strength is increased. This is expected since the confocal images show that as the fluid becomes more structured as the colloids aligned along the field when it is switched on. The pair correlation function $g_2$ does however show significant discrepancies emerge at high field strength. The increased ordering in the simulations may be due to the system being further down the path to forming a bct crystal than is the case in experiment.


The results of $g_{2xy}(r)$ and $g_{2z}(z)$ are both consistent with that of $g_2(r)$ where the height of the first peak increases as field strength is switched on. The significant differences in peak positions and shapes between $g_{2xy}(r)$ and $g_{2z}(z)$ show the fluid structure across the $xy$-plane differs from fluid structure along the $z$-axis. We also observe peak splitting ($g_{2xy}(r)$) from a broad peak into two distinct peaks as field strength increases from $\gamma$=16 to $\gamma$=45, showing increase in ordering of fluids across xy-plane above $\gamma$=16. Whereas ordering in fluid along z-axis occurs above  $\gamma$=7. However, fluid structures still show little difference between $\gamma$=0 and $\gamma$=7. 


At higher field strengths ($\gamma=16,45$), as the system falls out of equilibrium and (presumably) starts to order, larger discrepancies emerge. In $g_{2xy}(r)$, the simulations seem to exhibit more ordering, consistent with the suggestion above that they may be further down the path to crystallization. 


(\emph{iii}) 
We now consider three--body correlations in Fig. \ref{figG3}. As the dipolar strength increases, the peak at 180$^{\circ}$ increases as expected since more dipolar colloids form strings. However, we also observe peaks at 60$^{\circ}$ and 120$^{\circ}$ increasing with respect to field strength. Like $g_2$, in simulations, the triplet correlations show increasing structure at high field strength with respect to the experiments.


(\emph{iv}) 
The plots in Figs. \ref{figTCCPopulationA} and \ref{figTCCPopulationB} show the population of dipolar clusters of different geometries and sizes analyzed with the TCC. 
Overall, we find that the clusters we observe in our system follow reasonably those of the dipolar--Lennard--Jones clusters (Fig.~\ref{figTCCClusters}). That is to say, we see
more elongated clusters at higher field strengths. In all cases, at high field strength it is the LJ--dipolar cluster that corresponds to the highest dipolar interaction that we find in both experiment and simulation. We find this to be a significant outcome of this work, providing strong evidence in support of modeling colloids in an AC electric field with dipolar interactions. We reiterate that no fit parameters have been used here.


Minimum energy dipolar clusters by definition imply zero temperature, and are determined by the interaction energy. However, at finite temperature, entropy plays a role. The fact that the cluster population trends largely follow the minimum energy clusters indicates that the behavior of dipolar colloids is influenced by energetics. This is in stark -- and surprising -- contrast to earlier work which suggested that energy plays only a very limited role in observed cluster populations~\cite{taffs2010}. That work investigated Frank's well--known conjecture that icosahedra ``will be a very common grouping in liquids''~\cite{frank1952}. In fact, at the triple point of the Lennard--Jones system, only one particle in 1000 was found to be in an icosahedron and other 13--membered clusters dominate~\cite{taffs2010}, quite unlike the findings here in which the minimum energy structure dominates at high field strengths. Presumably the strength of the dipolar interactions (which are much larger than eg interactions in the Lennard--Jones system when it is in the liquid state~\cite{taffs2010}) is important here.


At high field strength, for $m=8,9,10$ clusters (Fig.~\ref{figTCCPopulationA}), we find more clusters with five--membered rings in the experiments than in the simulations. This we attribute to geometric frustration as the system falls out of equilibrium, with experimental and simulated realizations of the system taking different paths in the energy landscape (see below).



(\emph{v}) Figure \ref{figOrientation} shows that the anisotropic dipolar clusters tend to align along the $z$-axis (parallel with direction of the field). % when it switched on. 
The plots of $P_2(\cos{\alpha})$ in Figs. \ref{figOrientation}(c) and (d) show that as the field strength increases, so does the orientational order parameter $P_{2}$. $P_2$ has a maximum value of 1 which indicates when principal axes of clusters are parallel to the electric field. Our result also shows that more anisotropic, elongated clusters (e.g. 11O) have higher orientational order parameter $P_{2})$ than shorter clusters (e.g. 9S) at the same field strength. 


We now wish to discuss the relevance of our results with higher order structure. A small variation in dipolar strength between $\gamma = 0$ and $\gamma = 7$ does not show up in $g_2(r)$ or even $g_3(r)$. Whereas, even at low field strength, populations of dipolar clusters such as 8B and 8O vary drastically between different field strength (see Fig. \ref{figTCCPopulationA}). We can therefore conclude that higher order analysis such as that presented here is much more sensitive to small variation in structure and interactions. Comparing our study with other previously published work which used a TCC analysis of gels and glasses~\cite{royall2008,royall2015jnonxtalsol,leoni2023,jenkinson2017,royall2015jnonxtalsol,royall2012,thijssen2023}, we can conclude that such higher order structural analysis is better at capturing the onset of structural changes in amorphous systems than pair correlations $g_2(r)$, as may be inferred from other work~\cite{vanblaaderen1995,coslovich2011,leocmach2012,leocmach2013}. What is new here is that we have considered a system with anisotropic interactions.


The major discrepancies that we find are in the regime in which the system departs from equilibrium. That is to say, at high field strengths, the system becomes metastable to fluid-bct crystal phase coexistence. Now the early stages of this transition have been investigated recently~\cite{messina2023}. However, in related phenomena, such as the condensation of colloids with an effective attraction to form a gel network, the role of hydrodynamic interactions was found to be very important. Hydrodynamics control not only the timescale for the condensation~\cite{furukawa2010}, but also the higher--order structure of the resulting non-equilibrium gel network~\cite{royall2015prl,degraaf2019}. In (non--equilibrium) gelation of particles with spherically symmetric attractions, experiments exhibit many fewer clusters with fivefold symmetry than do Brownian dynamics computer simulations~\cite{royall2015prl}, quite the opposite trend if what is observed here.
We suggest that careful study, using simulations with hydrodynamic interactions and time--resolved experimental observation along the lines of Ref.~\cite{royall2015prl} may enable a more quantitative analysis of the time--evolution of this system than we have been able to perform here.


Finally, we have mentioned that here, no fit parameter is used in the mapping of our system between experiment and simulation. Sometimes, SI units are taken for experimental data and reduced units for simulation~\cite{ivlev} rather than a detailed mapping as we employ here. The agreement that we find between experiment and simulation (limited though it may be as the system falls out of equilibrium) is in our view quite remarkable given that a number related studies which do use reduced units for experiments \emph{have} used a \emph{de facto} fit parameter. In studies using colloid--polymer mixtures~\cite{royall2007jcp,royall2015prl,royall2018jcp,royall2008} for example, the radius of gyration of the polymer is known only to a relative error of perhaps~20\%. This amounts to a potential error in the interaction strength of approaching a factor of 2! Often the radius of gyration of the polymer is then taken as a fit parameter (fitted for example to simulation data~\cite{royall2007jcp} or to the phase behavior~\cite{royall2015prl,royall2018jcp,royall2008}). Our procedure here is quite different: no fitting is carried out, we simply use the material properties of the system and measured field strength. This underlines the quality of the agreement we see between experiment and simulation.





\section{Conclusion}


We have performed a detailed analysis of the string fluid structure in an anisotropic system of dipolar colloids and found reasonable agreement between our experiments and computer simulation data across a wide range of interactions tuned with the electric field. We found both bond--order parameter analysis of strings and the three--body correlation function $g_3$ to be suitable to quantify the degree of string formation in dipolar colloids but with $g_3$ offering more detailed information and can be used as a form of ``colloidal finger-print''. Using the topological cluster classification, we find that our experiments and simulations agree with expectations from minimum energy clusters of a dipolar-Lennard-Jones system~\cite{skipper2023}. That is to say, structural transformations predicted at zero temperature for a Lennard--Jones--Dipolar system are rather effective in their prediction of higher--order structure in the nearly--hard sphere--dipolar experiments and simulations. At high field strength, the cluster population in both our experiments and simulations is dominated by the minimum energy clusters for all sizes $8 \leq m \leq 12$. Finally, not only can we identify clusters relevant to the dipolar system but also to investigate their orientation with respect to field strength.


\begin{acknowledgments}
XW and FCM thank European Research Council (ERC) for funding an Advanced Grant for the DYNAMIN project, grant agreement No. 788968.
CPR acknowledges the Agence Nationale de Recherche for grant DiViNew.
YY gratefully acknowledges the China Scholarship Council. 
FJM was supported by a studentship provided by the Bristol Centre for Functional Nanomaterials (EPSRC Grant No. EP/L016648/1). The authors would like to thank Mark Miller, Josh Robinson and Peter Crowther for the help in building the TCC software. The authors would also like to thank Mike Allen and Didi Derks for  helpful discussions.  %The authors are grateful to two anonymous reviewers who helped the clarity of the manuscript very considerably.
\end{acknowledgments}


\section*{Data Availability Statement}

Data and code are available upon reasonable request. 


%apsrev4-2.bst 2019-01-14 (MD) hand-edited version of apsrev4-1.bst
%Control: key (0)
%Control: author (8) initials jnrlst
%Control: editor formatted (1) identically to author
%Control: production of article title (0) allowed
%Control: page (0) single
%Control: year (1) truncated
%Control: production of eprint (0) enabled
\begin{thebibliography}{103}%
\makeatletter
\providecommand \@ifxundefined [1]{%
 \@ifx{#1\undefined}
}%
\providecommand \@ifnum [1]{%
 \ifnum #1\expandafter \@firstoftwo
 \else \expandafter \@secondoftwo
 \fi
}%
\providecommand \@ifx [1]{%
 \ifx #1\expandafter \@firstoftwo
 \else \expandafter \@secondoftwo
 \fi
}%
\providecommand \natexlab [1]{#1}%
\providecommand \enquote  [1]{``#1''}%
\providecommand \bibnamefont  [1]{#1}%
\providecommand \bibfnamefont [1]{#1}%
\providecommand \citenamefont [1]{#1}%
\providecommand \href@noop [0]{\@secondoftwo}%
\providecommand \href [0]{\begingroup \@sanitize@url \@href}%
\providecommand \@href[1]{\@@startlink{#1}\@@href}%
\providecommand \@@href[1]{\endgroup#1\@@endlink}%
\providecommand \@sanitize@url [0]{\catcode `\\12\catcode `\$12\catcode
  `\&12\catcode `\#12\catcode `\^12\catcode `\_12\catcode `\%12\relax}%
\providecommand \@@startlink[1]{}%
\providecommand \@@endlink[0]{}%
\providecommand \url  [0]{\begingroup\@sanitize@url \@url }%
\providecommand \@url [1]{\endgroup\@href {#1}{\urlprefix }}%
\providecommand \urlprefix  [0]{URL }%
\providecommand \Eprint [0]{\href }%
\providecommand \doibase [0]{https://doi.org/}%
\providecommand \selectlanguage [0]{\@gobble}%
\providecommand \bibinfo  [0]{\@secondoftwo}%
\providecommand \bibfield  [0]{\@secondoftwo}%
\providecommand \translation [1]{[#1]}%
\providecommand \BibitemOpen [0]{}%
\providecommand \bibitemStop [0]{}%
\providecommand \bibitemNoStop [0]{.\EOS\space}%
\providecommand \EOS [0]{\spacefactor3000\relax}%
\providecommand \BibitemShut  [1]{\csname bibitem#1\endcsname}%
\let\auto@bib@innerbib\@empty
%</preamble>
\bibitem [{\citenamefont {Israelachvili}(2011)}]{israelachvili}%
  \BibitemOpen
  \bibfield  {author} {\bibinfo {author} {\bibfnamefont {J.~N.}\ \bibnamefont
  {Israelachvili}},\ }\href@noop {} {\emph {\bibinfo {title} {Intermolecular
  and Surface Forces}}},\ \bibinfo {edition} {3rd}\ ed.\ (\bibinfo  {publisher}
  {Academic Press},\ \bibinfo {year} {2011})\BibitemShut {NoStop}%
\bibitem [{\citenamefont {Gebbie}\ \emph {et~al.}(2017)\citenamefont {Gebbie},
  \citenamefont {Smith}, \citenamefont {Dobbs}, \citenamefont {Lee},
  \citenamefont {Warr}, \citenamefont {Banquy}, \citenamefont {Valtiner},
  \citenamefont {Rutland}, \citenamefont {Israelachvili}, \citenamefont
  {Perkin},\ and\ \citenamefont {Atkin}}]{gebbie2017}%
  \BibitemOpen
  \bibfield  {author} {\bibinfo {author} {\bibfnamefont {M.~A.}\ \bibnamefont
  {Gebbie}}, \bibinfo {author} {\bibfnamefont {A.~M.}\ \bibnamefont {Smith}},
  \bibinfo {author} {\bibfnamefont {H.~A.}\ \bibnamefont {Dobbs}}, \bibinfo
  {author} {\bibfnamefont {A.~A.}\ \bibnamefont {Lee}}, \bibinfo {author}
  {\bibfnamefont {G.~G.}\ \bibnamefont {Warr}}, \bibinfo {author}
  {\bibfnamefont {X.}~\bibnamefont {Banquy}}, \bibinfo {author} {\bibfnamefont
  {M.}~\bibnamefont {Valtiner}}, \bibinfo {author} {\bibfnamefont {M.~W.}\
  \bibnamefont {Rutland}}, \bibinfo {author} {\bibfnamefont {J.~N.}\
  \bibnamefont {Israelachvili}}, \bibinfo {author} {\bibfnamefont
  {S.}~\bibnamefont {Perkin}},\ and\ \bibinfo {author} {\bibfnamefont
  {R.}~\bibnamefont {Atkin}},\ }\bibfield  {title} {\bibinfo {title} {Long
  range electrostatic forces in ionic liquids},\ }\href@noop {} {\bibfield
  {journal} {\bibinfo  {journal} {Chemical Communication}\ }\textbf {\bibinfo
  {volume} {54}},\ \bibinfo {pages} {1214} (\bibinfo {year}
  {2017})}\BibitemShut {NoStop}%
\bibitem [{\citenamefont {Evans}\ \emph {et~al.}(2019)\citenamefont {Evans},
  \citenamefont {Frenkel},\ and\ \citenamefont {Dijkstra~M.}}]{evans2019}%
  \BibitemOpen
  \bibfield  {author} {\bibinfo {author} {\bibfnamefont {R.}~\bibnamefont
  {Evans}}, \bibinfo {author} {\bibfnamefont {D.}~\bibnamefont {Frenkel}},\
  and\ \bibinfo {author} {\bibfnamefont {M.}~\bibnamefont {Dijkstra~M.}},\
  }\bibfield  {title} {\bibinfo {title} {From simple liquids to colloids and
  soft matter},\ }\href {https://doi.org/10.1063/PT.3.4135} {\bibfield
  {journal} {\bibinfo  {journal} {Physics Today}\ }\textbf {\bibinfo {volume}
  {72}},\ \bibinfo {pages} {38} (\bibinfo {year} {2019})}\BibitemShut {NoStop}%
\bibitem [{\citenamefont {Ivlev}\ \emph {et~al.}(2012)\citenamefont {Ivlev},
  \citenamefont {L\"{o}wen}, \citenamefont {Morfill},\ and\ \citenamefont
  {Royall}}]{ivlev}%
  \BibitemOpen
  \bibfield  {author} {\bibinfo {author} {\bibfnamefont {A.}~\bibnamefont
  {Ivlev}}, \bibinfo {author} {\bibfnamefont {H.}~\bibnamefont {L\"{o}wen}},
  \bibinfo {author} {\bibfnamefont {G.~E.}\ \bibnamefont {Morfill}},\ and\
  \bibinfo {author} {\bibfnamefont {C.~P.}\ \bibnamefont {Royall}},\
  }\href@noop {} {\emph {\bibinfo {title} {Complex Plasmas and Colloidal
  Dispersions: Particle-resolved Studies of Classical Liquids and Solids}}}\
  (\bibinfo  {publisher} {World Scientific Publishing Co., Singapore
  Scientific},\ \bibinfo {year} {2012})\BibitemShut {NoStop}%
\bibitem [{\citenamefont {Bharti}\ and\ \citenamefont
  {Velev}(2015)}]{bharti2015}%
  \BibitemOpen
  \bibfield  {author} {\bibinfo {author} {\bibfnamefont {B.}~\bibnamefont
  {Bharti}}\ and\ \bibinfo {author} {\bibfnamefont {O.~D.}\ \bibnamefont
  {Velev}},\ }\bibfield  {title} {\bibinfo {title} {Assembly of reconfigurable
  colloidal structures by multidirectional field-induced interactions},\
  }\href@noop {} {\bibfield  {journal} {\bibinfo  {journal} {Langmuir}\
  }\textbf {\bibinfo {volume} {31}},\ \bibinfo {pages} {7897} (\bibinfo {year}
  {2015})}\BibitemShut {NoStop}%
\bibitem [{\citenamefont {Boles}\ \emph {et~al.}(2016)\citenamefont {Boles},
  \citenamefont {Engel},\ and\ \citenamefont {Talapin}}]{boles2016}%
  \BibitemOpen
  \bibfield  {author} {\bibinfo {author} {\bibfnamefont {M.~A.}\ \bibnamefont
  {Boles}}, \bibinfo {author} {\bibfnamefont {M.}~\bibnamefont {Engel}},\ and\
  \bibinfo {author} {\bibfnamefont {D.~V.}\ \bibnamefont {Talapin}},\
  }\bibfield  {title} {\bibinfo {title} {Self-assembly of colloidal
  nanocrystals: From intricate structures to functional materials},\ }\href
  {https://doi.org/10.1021/acs.chemrev.6b00196} {\bibfield  {journal} {\bibinfo
   {journal} {Chem. Rev.}\ }\textbf {\bibinfo {volume} {116}},\ \bibinfo
  {pages} {11220} (\bibinfo {year} {2016})}\BibitemShut {NoStop}%
\bibitem [{\citenamefont {Novak}\ \emph {et~al.}(2019)\citenamefont {Novak},
  \citenamefont {Pyanzina}, \citenamefont {S\'{a}nchez},\ and\ \citenamefont
  {Kantorovich}}]{novak2019}%
  \BibitemOpen
  \bibfield  {author} {\bibinfo {author} {\bibfnamefont {E.~V.}\ \bibnamefont
  {Novak}}, \bibinfo {author} {\bibfnamefont {E.~S.}\ \bibnamefont {Pyanzina}},
  \bibinfo {author} {\bibfnamefont {P.~A.}\ \bibnamefont {S\'{a}nchez}},\ and\
  \bibinfo {author} {\bibfnamefont {S.~S.}\ \bibnamefont {Kantorovich}},\
  }\bibfield  {title} {\bibinfo {title} {The structure of clusters formed by
  stockmayer supracolloidal magnetic polymers},\ }\href@noop {} {\bibfield
  {journal} {\bibinfo  {journal} {Eur. Phys. J. E}\ } (\bibinfo {year}
  {2019})}\BibitemShut {NoStop}%
\bibitem [{\citenamefont {Klokkenburg}\ \emph {et~al.}(2006)\citenamefont
  {Klokkenburg}, \citenamefont {Dullens}, \citenamefont {Kegel}, \citenamefont
  {B.},\ and\ \citenamefont {Philipse}}]{klokkenburg2006}%
  \BibitemOpen
  \bibfield  {author} {\bibinfo {author} {\bibfnamefont {M.}~\bibnamefont
  {Klokkenburg}}, \bibinfo {author} {\bibfnamefont {R.~P.~A.}\ \bibnamefont
  {Dullens}}, \bibinfo {author} {\bibfnamefont {W.~K.}\ \bibnamefont {Kegel}},
  \bibinfo {author} {\bibfnamefont {E.}~\bibnamefont {B.}},\ and\ \bibinfo
  {author} {\bibfnamefont {A.~P.}\ \bibnamefont {Philipse}},\ }\bibfield
  {title} {\bibinfo {title} {Quantitative real-space analysis of self-assembled
  structures of magnetic dipolar colloids},\ }\href@noop {} {\bibfield
  {journal} {\bibinfo  {journal} {Phys. Rev. Lett.}\ }\textbf {\bibinfo
  {volume} {96}},\ \bibinfo {pages} {037203} (\bibinfo {year}
  {2006})}\BibitemShut {NoStop}%
\bibitem [{\citenamefont {Pyanzina}\ \emph {et~al.}(107)\citenamefont
  {Pyanzina}, \citenamefont {Kantorovich}, \citenamefont {Cerd\`{a}},
  \citenamefont {Ivanov},\ and\ \citenamefont {Holm}}]{pyanzina2007}%
  \BibitemOpen
  \bibfield  {author} {\bibinfo {author} {\bibfnamefont {E.}~\bibnamefont
  {Pyanzina}}, \bibinfo {author} {\bibfnamefont {S.~S.}\ \bibnamefont
  {Kantorovich}}, \bibinfo {author} {\bibfnamefont {J.~J.}\ \bibnamefont
  {Cerd\`{a}}}, \bibinfo {author} {\bibfnamefont {A.~O.}\ \bibnamefont
  {Ivanov}},\ and\ \bibinfo {author} {\bibfnamefont {C.}~\bibnamefont {Holm}},\
  }\bibfield  {title} {\bibinfo {title} {How to analyse the structure factor in
  ferrofluids with strong magnetic interactions: a combined analytic and
  simulation approach},\ }\href@noop {} {\bibfield  {journal} {\bibinfo
  {journal} {Mol. Phys.}\ }\textbf {\bibinfo {volume} {107}},\ \bibinfo {pages}
  {571} (\bibinfo {year} {107})}\BibitemShut {NoStop}%
\bibitem [{\citenamefont {Kantorovich}\ \emph {et~al.}(2015)\citenamefont
  {Kantorovich}, \citenamefont {Ivanov}, \citenamefont {Rovigatti},
  \citenamefont {Tavares},\ and\ \citenamefont {Sciortino}}]{kantorovich2015}%
  \BibitemOpen
  \bibfield  {author} {\bibinfo {author} {\bibfnamefont {S.~S.}\ \bibnamefont
  {Kantorovich}}, \bibinfo {author} {\bibfnamefont {A.~O.}\ \bibnamefont
  {Ivanov}}, \bibinfo {author} {\bibfnamefont {L.}~\bibnamefont {Rovigatti}},
  \bibinfo {author} {\bibfnamefont {J.~M.}\ \bibnamefont {Tavares}},\ and\
  \bibinfo {author} {\bibfnamefont {F.}~\bibnamefont {Sciortino}},\ }\bibfield
  {title} {\bibinfo {title} {Temperature-induced structural transitions in
  self-assembling magnetic nanocolloids},\ }\href@noop {} {\bibfield  {journal}
  {\bibinfo  {journal} {Phys. Chem. Chem. Phys.}\ }\textbf {\bibinfo {volume}
  {17}},\ \bibinfo {pages} {16601} (\bibinfo {year} {2015})}\BibitemShut
  {NoStop}%
\bibitem [{\citenamefont {Spatafora-Salazar}\ \emph {et~al.}(2023)\citenamefont
  {Spatafora-Salazar}, \citenamefont {Kuei}, \citenamefont {Cunha},\ and\
  \citenamefont {Biswal}}]{spataforasalazar2023}%
  \BibitemOpen
  \bibfield  {author} {\bibinfo {author} {\bibfnamefont {A.}~\bibnamefont
  {Spatafora-Salazar}}, \bibinfo {author} {\bibfnamefont {S.}~\bibnamefont
  {Kuei}}, \bibinfo {author} {\bibfnamefont {L.~H.~P.}\ \bibnamefont {Cunha}},\
  and\ \bibinfo {author} {\bibfnamefont {S.~L.}\ \bibnamefont {Biswal}},\
  }\bibfield  {title} {\bibinfo {title} {Coiling of semiflexible paramagnetic
  colloidal chains},\ }\href@noop {} {\bibfield  {journal} {\bibinfo  {journal}
  {Soft Matter}\ }\textbf {\bibinfo {volume} {19}},\ \bibinfo {pages} {2385}
  (\bibinfo {year} {2023})}\BibitemShut {NoStop}%
\bibitem [{\citenamefont {Weis}\ and\ \citenamefont
  {Levesque}(2006)}]{weis2006}%
  \BibitemOpen
  \bibfield  {author} {\bibinfo {author} {\bibfnamefont {J.}~\bibnamefont
  {Weis}}\ and\ \bibinfo {author} {\bibfnamefont {D.}~\bibnamefont
  {Levesque}},\ }\bibfield  {title} {\bibinfo {title} {Orientational order in
  high density dipolar hard sphere fluids},\ }\href@noop {} {\bibfield
  {journal} {\bibinfo  {journal} {J. Chem. Phys.}\ }\textbf {\bibinfo {volume}
  {125}},\ \bibinfo {pages} {034504} (\bibinfo {year} {2006})}\BibitemShut
  {NoStop}%
\bibitem [{\citenamefont {Donaldson}\ \emph {et~al.}(2017)\citenamefont
  {Donaldson}, \citenamefont {Linse},\ and\ \citenamefont
  {Kantorovich}}]{donaldson2017}%
  \BibitemOpen
  \bibfield  {author} {\bibinfo {author} {\bibfnamefont {J.~G.}\ \bibnamefont
  {Donaldson}}, \bibinfo {author} {\bibfnamefont {P.}~\bibnamefont {Linse}},\
  and\ \bibinfo {author} {\bibfnamefont {S.~S.}\ \bibnamefont {Kantorovich}},\
  }\bibfield  {title} {\bibinfo {title} {How cube-like must magnetic
  nanoparticles be to modify their self-assembly?},\ }\href@noop {} {\bibfield
  {journal} {\bibinfo  {journal} {Nanoscale}\ }\textbf {\bibinfo {volume}
  {9}},\ \bibinfo {pages} {6448} (\bibinfo {year} {2017})}\BibitemShut
  {NoStop}%
\bibitem [{\citenamefont {Mostarac}\ \emph {et~al.}(2023)\citenamefont
  {Mostarac}, \citenamefont {Novak},\ and\ \citenamefont
  {Kantorovich}}]{mostarac2023}%
  \BibitemOpen
  \bibfield  {author} {\bibinfo {author} {\bibfnamefont {D.}~\bibnamefont
  {Mostarac}}, \bibinfo {author} {\bibfnamefont {E.~V.}\ \bibnamefont
  {Novak}},\ and\ \bibinfo {author} {\bibfnamefont {S.~S.}\ \bibnamefont
  {Kantorovich}},\ }\bibfield  {title} {\bibinfo {title} {Relating the length
  of a magnetic filament with solvophobic, superparamagnetic colloids to its
  properties in applied magnetic fields},\ }\href@noop {} {\bibfield  {journal}
  {\bibinfo  {journal} {Phys. Rev. E}\ }\textbf {\bibinfo {volume} {108}},\
  \bibinfo {pages} {054601} (\bibinfo {year} {2023})}\BibitemShut {NoStop}%
\bibitem [{\citenamefont {Elsner}\ \emph {et~al.}(2009)\citenamefont {Elsner},
  \citenamefont {Snoswell}, \citenamefont {Royall},\ and\ \citenamefont
  {Vincent}}]{elsner2009}%
  \BibitemOpen
  \bibfield  {author} {\bibinfo {author} {\bibfnamefont {N.}~\bibnamefont
  {Elsner}}, \bibinfo {author} {\bibfnamefont {D.~R.~E.}\ \bibnamefont
  {Snoswell}}, \bibinfo {author} {\bibfnamefont {C.~P.}\ \bibnamefont
  {Royall}},\ and\ \bibinfo {author} {\bibfnamefont {B.~V.}\ \bibnamefont
  {Vincent}},\ }\bibfield  {title} {\bibinfo {title} {Simple models for 2d
  tunable colloidal crystals in rotating ac electric fields},\ }\href@noop {}
  {\bibfield  {journal} {\bibinfo  {journal} {J. Chem. Phys.}\ }\textbf
  {\bibinfo {volume} {130}},\ \bibinfo {pages} {154901} (\bibinfo {year}
  {2009})}\BibitemShut {NoStop}%
\bibitem [{\citenamefont {Ben~Salah}\ \emph {et~al.}(2023)\citenamefont
  {Ben~Salah}, \citenamefont {Ukai}, \citenamefont {Mingyuan}, \citenamefont
  {Morimoto},\ and\ \citenamefont {Maekawa}}]{bensalah2023}%
  \BibitemOpen
  \bibfield  {author} {\bibinfo {author} {\bibfnamefont {A.}~\bibnamefont
  {Ben~Salah}}, \bibinfo {author} {\bibfnamefont {T.}~\bibnamefont {Ukai}},
  \bibinfo {author} {\bibfnamefont {L.}~\bibnamefont {Mingyuan}}, \bibinfo
  {author} {\bibfnamefont {H.}~\bibnamefont {Morimoto}},\ and\ \bibinfo
  {author} {\bibfnamefont {T.}~\bibnamefont {Maekawa}},\ }\bibfield  {title}
  {\bibinfo {title} {Growth process of clusters formed by paramagnetic
  microparticles in an ac/dc combined magnetic field},\ }\href@noop {}
  {\bibfield  {journal} {\bibinfo  {journal} {AIP Advances}\ }\textbf {\bibinfo
  {volume} {13}},\ \bibinfo {pages} {055010} (\bibinfo {year}
  {2023})}\BibitemShut {NoStop}%
\bibitem [{\citenamefont {Winslow}(1949)}]{winslow2000}%
  \BibitemOpen
  \bibfield  {author} {\bibinfo {author} {\bibnamefont {Winslow}},\ }\bibfield
  {title} {\bibinfo {title} {Induced fibration of suspensions},\ }\href@noop {}
  {\bibfield  {journal} {\bibinfo  {journal} {Journal of Applied Physics}\
  }\textbf {\bibinfo {volume} {20}},\ \bibinfo {pages} {1137} (\bibinfo {year}
  {1949})}\BibitemShut {NoStop}%
\bibitem [{\citenamefont {Dassanayake}\ \emph {et~al.}(2000)\citenamefont
  {Dassanayake}, \citenamefont {Fraden},\ and\ \citenamefont {van
  Blaaderen}}]{dassanayake2000}%
  \BibitemOpen
  \bibfield  {author} {\bibinfo {author} {\bibfnamefont {U.}~\bibnamefont
  {Dassanayake}}, \bibinfo {author} {\bibfnamefont {S.}~\bibnamefont
  {Fraden}},\ and\ \bibinfo {author} {\bibfnamefont {A.}~\bibnamefont {van
  Blaaderen}},\ }\bibfield  {title} {\bibinfo {title} {Structure of
  electrorheological fluids},\ }\href@noop {} {\bibfield  {journal} {\bibinfo
  {journal} {J. Chem. Phys.}\ }\textbf {\bibinfo {volume} {112}},\ \bibinfo
  {pages} {3851} (\bibinfo {year} {2000})}\BibitemShut {NoStop}%
\bibitem [{\citenamefont {van Blaaderen}(2004)}]{vanblaaderen2004}%
  \BibitemOpen
  \bibfield  {author} {\bibinfo {author} {\bibfnamefont {A.}~\bibnamefont {van
  Blaaderen}},\ }\bibfield  {title} {\bibinfo {title} {Colloids under external
  control},\ }\href@noop {} {\bibfield  {journal} {\bibinfo  {journal} {MRS
  Bulletin}\ }\textbf {\bibinfo {volume} {29}},\ \bibinfo {pages} {85?90}
  (\bibinfo {year} {2004})}\BibitemShut {NoStop}%
\bibitem [{\citenamefont {Yethiraj}\ and\ \citenamefont {van
  Blaaderen}(2003)}]{yethiraj2003}%
  \BibitemOpen
  \bibfield  {author} {\bibinfo {author} {\bibfnamefont {A.}~\bibnamefont
  {Yethiraj}}\ and\ \bibinfo {author} {\bibfnamefont {A.}~\bibnamefont {van
  Blaaderen}},\ }\bibfield  {title} {\bibinfo {title} {A colloidal model system
  with an interaction tunable from hard sphere to soft and dipolar},\
  }\href@noop {} {\bibfield  {journal} {\bibinfo  {journal} {Nature}\ }\textbf
  {\bibinfo {volume} {421}},\ \bibinfo {pages} {513} (\bibinfo {year}
  {2003})}\BibitemShut {NoStop}%
\bibitem [{\citenamefont {Klokkenburg}\ \emph {et~al.}(2007)\citenamefont
  {Klokkenburg}, \citenamefont {Ern\'{e}}, \citenamefont {Weidman},
  \citenamefont {V.},\ and\ \citenamefont {Philipse}}]{klokkenburg2007}%
  \BibitemOpen
  \bibfield  {author} {\bibinfo {author} {\bibfnamefont {M.}~\bibnamefont
  {Klokkenburg}}, \bibinfo {author} {\bibfnamefont {B.}~\bibnamefont
  {Ern\'{e}}}, \bibinfo {author} {\bibfnamefont {A.}~\bibnamefont {Weidman}},
  \bibinfo {author} {\bibfnamefont {P.~A.}\ \bibnamefont {V.}},\ and\ \bibinfo
  {author} {\bibfnamefont {A.~P.}\ \bibnamefont {Philipse}},\ }\bibfield
  {title} {\bibinfo {title} {Dipolar structures in magnetite ferrofluids
  studied with small-angle neutron scattering with and without applied magnetic
  field},\ }\href@noop {} {\bibfield  {journal} {\bibinfo  {journal} {Phys.
  Rev. E}\ }\textbf {\bibinfo {volume} {75}},\ \bibinfo {pages} {051408}
  (\bibinfo {year} {2007})}\BibitemShut {NoStop}%
\bibitem [{\citenamefont {A.~Wiedenmann}\ \emph {et~al.}(2008)\citenamefont
  {A.~Wiedenmann}, \citenamefont {U.~Keiderling}, \citenamefont {Meissner},
  \citenamefont {D.~Wallacher}, \citenamefont {G\"{a}hler}, \citenamefont
  {May}, \citenamefont {Pr\'{e}vost}, \citenamefont {Klokkenburg},
  \citenamefont {B.},\ and\ \citenamefont {Kohlbrecher}}]{wiedenmann2008}%
  \BibitemOpen
  \bibfield  {author} {\bibinfo {author} {\bibfnamefont {A.}~\bibnamefont
  {A.~Wiedenmann}}, \bibinfo {author} {\bibfnamefont {U.}~\bibnamefont
  {U.~Keiderling}}, \bibinfo {author} {\bibfnamefont {M.}~\bibnamefont
  {Meissner}}, \bibinfo {author} {\bibfnamefont {D.}~\bibnamefont
  {D.~Wallacher}}, \bibinfo {author} {\bibfnamefont {R.}~\bibnamefont
  {G\"{a}hler}}, \bibinfo {author} {\bibfnamefont {R.~P.}\ \bibnamefont {May}},
  \bibinfo {author} {\bibfnamefont {S.}~\bibnamefont {Pr\'{e}vost}}, \bibinfo
  {author} {\bibfnamefont {M.}~\bibnamefont {Klokkenburg}}, \bibinfo {author}
  {\bibfnamefont {E.}~\bibnamefont {B.}},\ and\ \bibinfo {author}
  {\bibfnamefont {J.}~\bibnamefont {Kohlbrecher}},\ }\bibfield  {title}
  {\bibinfo {title} {Low-temperature dynamics of magnetic colloids studied by
  time-resolved small-angle neutron scattering},\ }\href@noop {} {\bibfield
  {journal} {\bibinfo  {journal} {Phys. Rev. B}\ }\textbf {\bibinfo {volume}
  {77}},\ \bibinfo {pages} {184417} (\bibinfo {year} {2008})}\BibitemShut
  {NoStop}%
\bibitem [{\citenamefont {Hynninen}\ and\ \citenamefont
  {Dijkstra}(2005)}]{hynninen2005}%
  \BibitemOpen
  \bibfield  {author} {\bibinfo {author} {\bibfnamefont {A.-P.}\ \bibnamefont
  {Hynninen}}\ and\ \bibinfo {author} {\bibfnamefont {M.}~\bibnamefont
  {Dijkstra}},\ }\bibfield  {title} {\bibinfo {title} {Phase diagram of dipolar
  hard and soft spheres: Manipulation of colloidal crystal structures by an
  external field},\ }\href {https://doi.org/10.1103/PhysRevLett.94.138303}
  {\bibfield  {journal} {\bibinfo  {journal} {Physical review letters}\
  }\textbf {\bibinfo {volume} {94}},\ \bibinfo {pages} {138303} (\bibinfo
  {year} {2005})}\BibitemShut {NoStop}%
\bibitem [{\citenamefont {Yethiraj}\ \emph {et~al.}(2004)\citenamefont
  {Yethiraj}, \citenamefont {Wouterse}, \citenamefont {Groh},\ and\
  \citenamefont {van Blaaderen~Alfons}}]{yethiraj2004}%
  \BibitemOpen
  \bibfield  {author} {\bibinfo {author} {\bibfnamefont {A.}~\bibnamefont
  {Yethiraj}}, \bibinfo {author} {\bibfnamefont {A.}~\bibnamefont {Wouterse}},
  \bibinfo {author} {\bibfnamefont {B.}~\bibnamefont {Groh}},\ and\ \bibinfo
  {author} {\bibnamefont {van Blaaderen~Alfons}},\ }\bibfield  {title}
  {\bibinfo {title} {Nature of an electric-field-induced colloidal martensitic
  transition},\ }\href@noop {} {\bibfield  {journal} {\bibinfo  {journal}
  {Phys. Rev. Lett.}\ }\textbf {\bibinfo {volume} {92}},\ \bibinfo {pages}
  {058301} (\bibinfo {year} {2004})}\BibitemShut {NoStop}%
\bibitem [{\citenamefont {Colla}\ \emph {et~al.}(2018)\citenamefont {Colla},
  \citenamefont {N\"{o}jid}, \citenamefont {Riede}, \citenamefont
  {Schurtenberger},\ and\ \citenamefont {Likos}}]{colla2018}%
  \BibitemOpen
  \bibfield  {author} {\bibinfo {author} {\bibfnamefont {P.~S.}\ \bibnamefont
  {Colla}, \bibfnamefont {T.~Mohanty}}, \bibinfo {author} {\bibfnamefont
  {S.}~\bibnamefont {N\"{o}jid}}, \bibinfo {author} {\bibfnamefont
  {A.}~\bibnamefont {Riede}}, \bibinfo {author} {\bibfnamefont
  {P.}~\bibnamefont {Schurtenberger}},\ and\ \bibinfo {author} {\bibfnamefont
  {C.}~\bibnamefont {Likos}},\ }\bibfield  {title} {\bibinfo {title}
  {Self-assembly of ionic microgels driven by an alternating electric field:
  Theory, simulations, and experiments},\ }\href@noop {} {\bibfield  {journal}
  {\bibinfo  {journal} {ACS Nano}\ } (\bibinfo {year} {2018})}\BibitemShut
  {NoStop}%
\bibitem [{\citenamefont {Semwal}\ \emph {et~al.}(2022)\citenamefont {Semwal},
  \citenamefont {Clowe-Coish}, \citenamefont {Saika-Voivod},\ and\
  \citenamefont {Yethiraj}}]{semwal2022}%
  \BibitemOpen
  \bibfield  {author} {\bibinfo {author} {\bibfnamefont {S.}~\bibnamefont
  {Semwal}}, \bibinfo {author} {\bibfnamefont {C.}~\bibnamefont {Clowe-Coish}},
  \bibinfo {author} {\bibfnamefont {I.}~\bibnamefont {Saika-Voivod}},\ and\
  \bibinfo {author} {\bibfnamefont {A.}~\bibnamefont {Yethiraj}},\ }\bibfield
  {title} {\bibinfo {title} {Tunable colloids with dipolar and depletion
  interactions: Toward field-switchable crystals and gels},\ }\href@noop {}
  {\bibfield  {journal} {\bibinfo  {journal} {Phys. Rev. X}\ }\textbf {\bibinfo
  {volume} {12}},\ \bibinfo {pages} {041021} (\bibinfo {year}
  {2022})}\BibitemShut {NoStop}%
\bibitem [{\citenamefont {Li}\ \emph {et~al.}(2010)\citenamefont {Li},
  \citenamefont {Newman}, \citenamefont {Valera}, \citenamefont
  {Saika-Voivod},\ and\ \citenamefont {Yethiraj}}]{li2010}%
  \BibitemOpen
  \bibfield  {author} {\bibinfo {author} {\bibfnamefont {N.}~\bibnamefont
  {Li}}, \bibinfo {author} {\bibfnamefont {H.}~\bibnamefont {Newman}}, \bibinfo
  {author} {\bibfnamefont {M.}~\bibnamefont {Valera}}, \bibinfo {author}
  {\bibfnamefont {I.}~\bibnamefont {Saika-Voivod}},\ and\ \bibinfo {author}
  {\bibfnamefont {A.}~\bibnamefont {Yethiraj}},\ }\bibfield  {title} {\bibinfo
  {title} {Colloids with a tunable dipolar interaction: Equations of state and
  order parameters via confocal microscopy},\ }\bibfield  {journal} {\bibinfo
  {journal} {Soft Matter}\ }\textbf {\bibinfo {volume} {6}},\ \href
  {https://doi.org/10.1039/B909953K} {10.1039/B909953K} (\bibinfo {year}
  {2010})\BibitemShut {NoStop}%
\bibitem [{\citenamefont {Elfimova}\ \emph {et~al.}(2012)\citenamefont
  {Elfimova}, \citenamefont {Ivanov},\ and\ \citenamefont
  {Camp}}]{elfimova2012}%
  \BibitemOpen
  \bibfield  {author} {\bibinfo {author} {\bibfnamefont {E.~A.}\ \bibnamefont
  {Elfimova}}, \bibinfo {author} {\bibfnamefont {A.~O.}\ \bibnamefont
  {Ivanov}},\ and\ \bibinfo {author} {\bibfnamefont {P.~J.}\ \bibnamefont
  {Camp}},\ }\bibfield  {title} {\bibinfo {title} {Theory and simulation of
  anisotropic pair correlations in ferrofluids in magnetic fields},\
  }\href@noop {} {\bibfield  {journal} {\bibinfo  {journal} {J. Chem. Phys.}\
  }\textbf {\bibinfo {volume} {136}},\ \bibinfo {pages} {194502} (\bibinfo
  {year} {2012})}\BibitemShut {NoStop}%
\bibitem [{\citenamefont {Vutukuri}\ \emph {et~al.}(2012)\citenamefont
  {Vutukuri}, \citenamefont {Demir\"{o}rs}, \citenamefont {Peng}, \citenamefont
  {van Oostrum}, \citenamefont {Imhof},\ and\ \citenamefont {van
  Blaaderen}}]{vutukuri2012}%
  \BibitemOpen
  \bibfield  {author} {\bibinfo {author} {\bibfnamefont {H.~R.}\ \bibnamefont
  {Vutukuri}}, \bibinfo {author} {\bibfnamefont {A.~F.}\ \bibnamefont
  {Demir\"{o}rs}}, \bibinfo {author} {\bibfnamefont {B.}~\bibnamefont {Peng}},
  \bibinfo {author} {\bibfnamefont {P.~D.~J.}\ \bibnamefont {van Oostrum}},
  \bibinfo {author} {\bibfnamefont {A.}~\bibnamefont {Imhof}},\ and\ \bibinfo
  {author} {\bibfnamefont {A.}~\bibnamefont {van Blaaderen}},\ }\bibfield
  {title} {\bibinfo {title} {Colloidal analogues of charged and uncharged
  polymer chains with tunable stiffness},\ }\href@noop {} {\bibfield  {journal}
  {\bibinfo  {journal} {Angewandte Chemie International Edition}\ }\textbf
  {\bibinfo {volume} {51}},\ \bibinfo {pages} {11249} (\bibinfo {year}
  {2012})}\BibitemShut {NoStop}%
\bibitem [{\citenamefont {Wei}\ \emph {et~al.}(2016)\citenamefont {Wei},
  \citenamefont {Sonmg},\ and\ \citenamefont {Dobiknar}}]{wei2016}%
  \BibitemOpen
  \bibfield  {author} {\bibinfo {author} {\bibfnamefont {J.}~\bibnamefont
  {Wei}}, \bibinfo {author} {\bibfnamefont {F.}~\bibnamefont {Sonmg}},\ and\
  \bibinfo {author} {\bibfnamefont {J.}~\bibnamefont {Dobiknar}},\ }\bibfield
  {title} {\bibinfo {title} {Assembly of superparamagnetic filaments in
  external field},\ }\href@noop {} {\bibfield  {journal} {\bibinfo  {journal}
  {Langmuir}\ }\textbf {\bibinfo {volume} {32}},\ \bibinfo {pages} {9321}
  (\bibinfo {year} {2016})}\BibitemShut {NoStop}%
\bibitem [{\citenamefont {Messina}\ and\ \citenamefont
  {Kemgang}(2023)}]{messina2023}%
  \BibitemOpen
  \bibfield  {author} {\bibinfo {author} {\bibfnamefont {R.}~\bibnamefont
  {Messina}}\ and\ \bibinfo {author} {\bibfnamefont {E.}~\bibnamefont
  {Kemgang}},\ }\bibfield  {title} {\bibinfo {title} {The relevance of
  curvature-induced quadrupolar interactions in dipolar chain aggregation},\
  }\href@noop {} {\bibfield  {journal} {\bibinfo  {journal} {J. Chem. Phys.}\
  }\textbf {\bibinfo {volume} {159}},\ \bibinfo {pages} {174903} (\bibinfo
  {year} {2023})}\BibitemShut {NoStop}%
\bibitem [{\citenamefont {Royall}\ and\ \citenamefont
  {Williams}(2015)}]{royall2015physrep}%
  \BibitemOpen
  \bibfield  {author} {\bibinfo {author} {\bibfnamefont {C.~P.}\ \bibnamefont
  {Royall}}\ and\ \bibinfo {author} {\bibfnamefont {S.~R.}\ \bibnamefont
  {Williams}},\ }\bibfield  {title} {\bibinfo {title} {The role of local
  structure in dynamical arrest},\ }\href
  {https://doi.org/http://dx.doi.org/10.1016/j.physrep.2014.11.004} {\bibfield
  {journal} {\bibinfo  {journal} {Phys. Rep.}\ }\textbf {\bibinfo {volume}
  {560}},\ \bibinfo {pages} {1} (\bibinfo {year} {2015})}\BibitemShut {NoStop}%
\bibitem [{\citenamefont {van Blaaderen}\ and\ \citenamefont
  {Wiltzius}(1995)}]{vanblaaderen1995}%
  \BibitemOpen
  \bibfield  {author} {\bibinfo {author} {\bibfnamefont {A.}~\bibnamefont {van
  Blaaderen}}\ and\ \bibinfo {author} {\bibfnamefont {P.}~\bibnamefont
  {Wiltzius}},\ }\bibfield  {title} {\bibinfo {title} {Real-space structure of
  colloidal hard-sphere glasses},\ }\href@noop {} {\bibfield  {journal}
  {\bibinfo  {journal} {Science}\ }\textbf {\bibinfo {volume} {270}},\ \bibinfo
  {pages} {1177} (\bibinfo {year} {1995})}\BibitemShut {NoStop}%
\bibitem [{\citenamefont {Royall}\ \emph {et~al.}(2008)\citenamefont {Royall},
  \citenamefont {Williams}, \citenamefont {Ohtsuka},\ and\ \citenamefont
  {Tanaka}}]{royall2008}%
  \BibitemOpen
  \bibfield  {author} {\bibinfo {author} {\bibfnamefont {C.~P.}\ \bibnamefont
  {Royall}}, \bibinfo {author} {\bibfnamefont {S.~R.}\ \bibnamefont
  {Williams}}, \bibinfo {author} {\bibfnamefont {T.}~\bibnamefont {Ohtsuka}},\
  and\ \bibinfo {author} {\bibfnamefont {H.}~\bibnamefont {Tanaka}},\
  }\bibfield  {title} {\bibinfo {title} {Direct observation of a local
  structural mechanism for dynamic arrest},\ }\href@noop {} {\bibfield
  {journal} {\bibinfo  {journal} {Nature Mater.}\ }\textbf {\bibinfo {volume}
  {7}},\ \bibinfo {pages} {556} (\bibinfo {year} {2008})}\BibitemShut {NoStop}%
\bibitem [{\citenamefont {Leocmach}\ \emph {et~al.}(2013)\citenamefont
  {Leocmach}, \citenamefont {Russo},\ and\ \citenamefont
  {Tanaka}}]{leocmach2013}%
  \BibitemOpen
  \bibfield  {author} {\bibinfo {author} {\bibfnamefont {M.}~\bibnamefont
  {Leocmach}}, \bibinfo {author} {\bibfnamefont {J.}~\bibnamefont {Russo}},\
  and\ \bibinfo {author} {\bibfnamefont {H.}~\bibnamefont {Tanaka}},\
  }\bibfield  {title} {\bibinfo {title} {Importance of many-body correlations
  in glass transition: An example from polydispere hard spheres},\ }\href@noop
  {} {\bibfield  {journal} {\bibinfo  {journal} {J. Chem. Phys.}\ }\textbf
  {\bibinfo {volume} {138}},\ \bibinfo {pages} {12A515} (\bibinfo {year}
  {2013})}\BibitemShut {NoStop}%
\bibitem [{\citenamefont {Taffs}\ and\ \citenamefont
  {Royall}(2016)}]{taffs2016}%
  \BibitemOpen
  \bibfield  {author} {\bibinfo {author} {\bibfnamefont {J.}~\bibnamefont
  {Taffs}}\ and\ \bibinfo {author} {\bibfnamefont {C.~P.}\ \bibnamefont
  {Royall}},\ }\bibfield  {title} {\bibinfo {title} {The role of fivefold
  symmetry in suppressing crystallization},\ }\href
  {https://doi.org/https://doi.org/10.1038/ncomms13225} {\bibfield  {journal}
  {\bibinfo  {journal} {Nature Communication}\ }\textbf {\bibinfo {volume}
  {7}},\ \bibinfo {pages} {13225} (\bibinfo {year} {2016})}\BibitemShut
  {NoStop}%
\bibitem [{\citenamefont {Leoni}\ and\ \citenamefont
  {Russo}(2021)}]{leoni2021}%
  \BibitemOpen
  \bibfield  {author} {\bibinfo {author} {\bibfnamefont {F.}~\bibnamefont
  {Leoni}}\ and\ \bibinfo {author} {\bibfnamefont {J.}~\bibnamefont {Russo}},\
  }\bibfield  {title} {\bibinfo {title} {Nonclassical nucleation pathways in
  stacking-disordered crystals},\ }\href@noop {} {\bibfield  {journal}
  {\bibinfo  {journal} {Phys. Rev. X}\ }\textbf {\bibinfo {volume} {11}},\
  \bibinfo {pages} {031006} (\bibinfo {year} {2021})}\BibitemShut {NoStop}%
\bibitem [{\citenamefont {Gispen}\ \emph {et~al.}(2023)\citenamefont {Gispen},
  \citenamefont {Coli}, \citenamefont {van Damme}, \citenamefont {Royall},\
  and\ \citenamefont {Dijkstra}}]{gispen2023}%
  \BibitemOpen
  \bibfield  {author} {\bibinfo {author} {\bibfnamefont {W.}~\bibnamefont
  {Gispen}}, \bibinfo {author} {\bibfnamefont {G.~M.}\ \bibnamefont {Coli}},
  \bibinfo {author} {\bibfnamefont {R.}~\bibnamefont {van Damme}}, \bibinfo
  {author} {\bibfnamefont {C.~P.}\ \bibnamefont {Royall}},\ and\ \bibinfo
  {author} {\bibfnamefont {M.}~\bibnamefont {Dijkstra}},\ }\bibfield  {title}
  {\bibinfo {title} {Crystal polymorph selection mechanism of hard spheres
  hidden in the fluid},\ }\href@noop {} {\bibfield  {journal} {\bibinfo
  {journal} {accepted by ACS Nano}\ } (\bibinfo {year} {2023})}\BibitemShut
  {NoStop}%
\bibitem [{\citenamefont {Russo}\ and\ \citenamefont
  {Tanaka}(2012)}]{russo2012}%
  \BibitemOpen
  \bibfield  {author} {\bibinfo {author} {\bibfnamefont {J.}~\bibnamefont
  {Russo}}\ and\ \bibinfo {author} {\bibfnamefont {H.}~\bibnamefont {Tanaka}},\
  }\bibfield  {title} {\bibinfo {title} {The microscopic pathway to
  crystallization in supercooled liquids},\ }\href@noop {} {\bibfield
  {journal} {\bibinfo  {journal} {Sci. Rep.}\ }\textbf {\bibinfo {volume}
  {2}},\ \bibinfo {pages} {505} (\bibinfo {year} {2012})}\BibitemShut {NoStop}%
\bibitem [{\citenamefont {Hansen}\ and\ \citenamefont
  {McDonald}(2006)}]{hansen}%
  \BibitemOpen
  \bibfield  {author} {\bibinfo {author} {\bibfnamefont {J.-P.}\ \bibnamefont
  {Hansen}}\ and\ \bibinfo {author} {\bibfnamefont {I.~R.}\ \bibnamefont
  {McDonald}},\ }\href@noop {} {\emph {\bibinfo {title} {Theory of Simple
  Liquids}}},\ \bibinfo {edition} {3rd}\ ed.\ (\bibinfo  {publisher}
  {Chichester : Wiley},\ \bibinfo {year} {2006})\BibitemShut {NoStop}%
\bibitem [{\citenamefont {Robinson}\ \emph
  {et~al.}(2019{\natexlab{a}})\citenamefont {Robinson}, \citenamefont {Turci},
  \citenamefont {Roth},\ and\ \citenamefont {Royall}}]{robinson2019prl}%
  \BibitemOpen
  \bibfield  {author} {\bibinfo {author} {\bibfnamefont {J.~F.}\ \bibnamefont
  {Robinson}}, \bibinfo {author} {\bibfnamefont {F.}~\bibnamefont {Turci}},
  \bibinfo {author} {\bibfnamefont {R.}~\bibnamefont {Roth}},\ and\ \bibinfo
  {author} {\bibfnamefont {C.~P.}\ \bibnamefont {Royall}},\ }\bibfield  {title}
  {\bibinfo {title} {Morphometric approach to many-body correlations in hard
  spheres},\ }\href@noop {} {\bibfield  {journal} {\bibinfo  {journal} {Phys.
  Rev. Lett.}\ }\textbf {\bibinfo {volume} {122}},\ \bibinfo {pages} {068004}
  (\bibinfo {year} {2019}{\natexlab{a}})}\BibitemShut {NoStop}%
\bibitem [{\citenamefont {Robinson}\ \emph
  {et~al.}(2019{\natexlab{b}})\citenamefont {Robinson}, \citenamefont {Turci},
  \citenamefont {Roth},\ and\ \citenamefont {Royall}}]{robinson2019pre}%
  \BibitemOpen
  \bibfield  {author} {\bibinfo {author} {\bibfnamefont {J.~F.}\ \bibnamefont
  {Robinson}}, \bibinfo {author} {\bibfnamefont {F.}~\bibnamefont {Turci}},
  \bibinfo {author} {\bibfnamefont {R.}~\bibnamefont {Roth}},\ and\ \bibinfo
  {author} {\bibfnamefont {C.~P.}\ \bibnamefont {Royall}},\ }\bibfield  {title}
  {\bibinfo {title} {Many-body correlations from integral geometry},\
  }\href@noop {} {\bibfield  {journal} {\bibinfo  {journal} {Phys. Rev. E}\
  }\textbf {\bibinfo {volume} {100}},\ \bibinfo {pages} {062126} (\bibinfo
  {year} {2019}{\natexlab{b}})}\BibitemShut {NoStop}%
\bibitem [{\citenamefont {Russ}\ \emph {et~al.}(2003)\citenamefont {Russ},
  \citenamefont {Zahn},\ and\ \citenamefont {von Gruenberg}}]{russ2003}%
  \BibitemOpen
  \bibfield  {author} {\bibinfo {author} {\bibfnamefont {C.}~\bibnamefont
  {Russ}}, \bibinfo {author} {\bibfnamefont {K.}~\bibnamefont {Zahn}},\ and\
  \bibinfo {author} {\bibfnamefont {H.}~\bibnamefont {von Gruenberg}},\
  }\bibfield  {title} {\bibinfo {title} {Triplet correlations in
  two-dimensional colloidal model liquids},\ }\href@noop {} {\bibfield
  {journal} {\bibinfo  {journal} {Journal of Physics:Condensed Matter}\
  }\textbf {\bibinfo {volume} {15}} (\bibinfo {year} {2003})}\BibitemShut
  {NoStop}%
\bibitem [{\citenamefont {Royall}\ \emph
  {et~al.}(2015{\natexlab{a}})\citenamefont {Royall}, \citenamefont {Eggers},
  \citenamefont {Furukawa},\ and\ \citenamefont {Tanaka}}]{royall2015prl}%
  \BibitemOpen
  \bibfield  {author} {\bibinfo {author} {\bibfnamefont {C.~P.}\ \bibnamefont
  {Royall}}, \bibinfo {author} {\bibfnamefont {J.}~\bibnamefont {Eggers}},
  \bibinfo {author} {\bibfnamefont {A.}~\bibnamefont {Furukawa}},\ and\
  \bibinfo {author} {\bibfnamefont {H.}~\bibnamefont {Tanaka}},\ }\bibfield
  {title} {\bibinfo {title} {Probing colloidal gels at multiple length scales:
  The role of hydrodynamics},\ }\href@noop {} {\bibfield  {journal} {\bibinfo
  {journal} {Phys. Rev. Lett.}\ }\textbf {\bibinfo {volume} {114}},\ \bibinfo
  {pages} {258302} (\bibinfo {year} {2015}{\natexlab{a}})}\BibitemShut
  {NoStop}%
\bibitem [{\citenamefont {Honeycutt}\ and\ \citenamefont
  {Andersen}(1987)}]{honeycutt1987}%
  \BibitemOpen
  \bibfield  {author} {\bibinfo {author} {\bibfnamefont {J.~D.}\ \bibnamefont
  {Honeycutt}}\ and\ \bibinfo {author} {\bibfnamefont {H.~C.}\ \bibnamefont
  {Andersen}},\ }\bibfield  {title} {\bibinfo {title} {Molecular dynamics study
  of melting and freezing of small lennard-jones clusters},\ }\href@noop {}
  {\bibfield  {journal} {\bibinfo  {journal} {J. Phys. Chem.}\ }\textbf
  {\bibinfo {volume} {91}},\ \bibinfo {pages} {4950} (\bibinfo {year}
  {1987})}\BibitemShut {NoStop}%
\bibitem [{\citenamefont {Tanemura}\ \emph {et~al.}(1977)\citenamefont
  {Tanemura}, \citenamefont {Hiwatari}, \citenamefont {Matsuda}, \citenamefont
  {Ogawa}, \citenamefont {Ogita},\ and\ \citenamefont {Ueda}}]{tanemura1977}%
  \BibitemOpen
  \bibfield  {author} {\bibinfo {author} {\bibfnamefont {M.}~\bibnamefont
  {Tanemura}}, \bibinfo {author} {\bibfnamefont {Y.}~\bibnamefont {Hiwatari}},
  \bibinfo {author} {\bibfnamefont {H.}~\bibnamefont {Matsuda}}, \bibinfo
  {author} {\bibfnamefont {T.}~\bibnamefont {Ogawa}}, \bibinfo {author}
  {\bibfnamefont {N.}~\bibnamefont {Ogita}},\ and\ \bibinfo {author}
  {\bibfnamefont {A.}~\bibnamefont {Ueda}},\ }\bibfield  {title} {\bibinfo
  {title} {Geometrical analysis of crystallization of the soft-core model},\
  }\href@noop {} {\bibfield  {journal} {\bibinfo  {journal} {Progress of
  Theoretical Physics}\ }\textbf {\bibinfo {volume} {58}},\ \bibinfo {pages}
  {1079?1095} (\bibinfo {year} {1977})}\BibitemShut {NoStop}%
\bibitem [{\citenamefont {J\'onsson}\ and\ \citenamefont
  {Andersen}(1988)}]{jonsson1988}%
  \BibitemOpen
  \bibfield  {author} {\bibinfo {author} {\bibfnamefont {H.}~\bibnamefont
  {J\'onsson}}\ and\ \bibinfo {author} {\bibfnamefont {H.~C.}\ \bibnamefont
  {Andersen}},\ }\bibfield  {title} {\bibinfo {title} {Icosahedral ordering in
  the lennard-jones liquid and glass},\ }\href
  {https://doi.org/10.1103/PhysRevLett.60.2295} {\bibfield  {journal} {\bibinfo
   {journal} {Physical Review Letters}\ }\textbf {\bibinfo {volume} {60}},\
  \bibinfo {pages} {2295} (\bibinfo {year} {1988})}\BibitemShut {NoStop}%
\bibitem [{\citenamefont {Bailey}\ \emph {et~al.}(2004)\citenamefont {Bailey},
  \citenamefont {Schi\o{}tz},\ and\ \citenamefont {Jacobsen}}]{bailey2004}%
  \BibitemOpen
  \bibfield  {author} {\bibinfo {author} {\bibfnamefont {N.~P.}\ \bibnamefont
  {Bailey}}, \bibinfo {author} {\bibfnamefont {J.}~\bibnamefont {Schi\o{}tz}},\
  and\ \bibinfo {author} {\bibfnamefont {K.~W.}\ \bibnamefont {Jacobsen}},\
  }\bibfield  {title} {\bibinfo {title} {Simulation of cu-mg metallic glass:
  Thermodynamics and structure},\ }\href
  {https://doi.org/10.1103/PhysRevB.69.144205} {\bibfield  {journal} {\bibinfo
  {journal} {Phys. Rev. B}\ }\textbf {\bibinfo {volume} {69}},\ \bibinfo
  {pages} {144205} (\bibinfo {year} {2004})}\BibitemShut {NoStop}%
\bibitem [{\citenamefont {Steinhardt}\ \emph {et~al.}(1983)\citenamefont
  {Steinhardt}, \citenamefont {Nelson},\ and\ \citenamefont
  {Ronchetti}}]{steinhardt1983}%
  \BibitemOpen
  \bibfield  {author} {\bibinfo {author} {\bibfnamefont {P.~J.}\ \bibnamefont
  {Steinhardt}}, \bibinfo {author} {\bibfnamefont {D.~R.}\ \bibnamefont
  {Nelson}},\ and\ \bibinfo {author} {\bibfnamefont {M.}~\bibnamefont
  {Ronchetti}},\ }\bibfield  {title} {\bibinfo {title} {Bond-orientational
  order in liquids and glasses},\ }\href
  {https://doi.org/10.1103/PhysRevB.28.784} {\bibfield  {journal} {\bibinfo
  {journal} {Phys. Rev. B}\ }\textbf {\bibinfo {volume} {28}},\ \bibinfo
  {pages} {784} (\bibinfo {year} {1983})}\BibitemShut {NoStop}%
\bibitem [{\citenamefont {Auer}\ and\ \citenamefont
  {Frenkel}(2004)}]{auer2004}%
  \BibitemOpen
  \bibfield  {author} {\bibinfo {author} {\bibfnamefont {S.}~\bibnamefont
  {Auer}}\ and\ \bibinfo {author} {\bibfnamefont {D.}~\bibnamefont {Frenkel}},\
  }\bibfield  {title} {\bibinfo {title} {Numerical prediction of absolute
  crystallization rates in hard-sphere colloids},\ }\href
  {https://doi.org/10.1063/1.1638740} {\bibfield  {journal} {\bibinfo
  {journal} {The Journal of Chemical Physics}\ }\textbf {\bibinfo {volume}
  {120}},\ \bibinfo {pages} {3015} (\bibinfo {year} {2004})},\ \Eprint
  {https://arxiv.org/abs/https://doi.org/10.1063/1.1638740}
  {https://doi.org/10.1063/1.1638740} \BibitemShut {NoStop}%
\bibitem [{\citenamefont {Kawasaki}\ and\ \citenamefont
  {Tanaka}(2010)}]{kawasaki2010}%
  \BibitemOpen
  \bibfield  {author} {\bibinfo {author} {\bibfnamefont {T.}~\bibnamefont
  {Kawasaki}}\ and\ \bibinfo {author} {\bibfnamefont {H.}~\bibnamefont
  {Tanaka}},\ }\bibfield  {title} {\bibinfo {title} {Formation of a crystal
  nucleus from liquid},\ }\href {https://doi.org/10.1073/pnas.1001040107}
  {\bibfield  {journal} {\bibinfo  {journal} {Proceedings of the National
  Academy of Sciences}\ }\textbf {\bibinfo {volume} {107}},\ \bibinfo {pages}
  {14036} (\bibinfo {year} {2010})}\BibitemShut {NoStop}%
\bibitem [{\citenamefont {Cubuk}\ \emph {et~al.}(2015)\citenamefont {Cubuk},
  \citenamefont {Schoenholz}, \citenamefont {Rieser}, \citenamefont {Malone},
  \citenamefont {Rottler}, \citenamefont {Durian}, \citenamefont {Kaxiras},\
  and\ \citenamefont {Liu}}]{cubuk2015}%
  \BibitemOpen
  \bibfield  {author} {\bibinfo {author} {\bibfnamefont {E.~D.}\ \bibnamefont
  {Cubuk}}, \bibinfo {author} {\bibfnamefont {S.~S.}\ \bibnamefont
  {Schoenholz}}, \bibinfo {author} {\bibfnamefont {J.~M.}\ \bibnamefont
  {Rieser}}, \bibinfo {author} {\bibfnamefont {B.~D.}\ \bibnamefont {Malone}},
  \bibinfo {author} {\bibfnamefont {J.}~\bibnamefont {Rottler}}, \bibinfo
  {author} {\bibfnamefont {D.~J.}\ \bibnamefont {Durian}}, \bibinfo {author}
  {\bibfnamefont {E.}~\bibnamefont {Kaxiras}},\ and\ \bibinfo {author}
  {\bibfnamefont {A.~J.}\ \bibnamefont {Liu}},\ }\bibfield  {title} {\bibinfo
  {title} {Identifying structural flow defects in disordered solids using
  machine-learning methods},\ }\href@noop {} {\bibfield  {journal} {\bibinfo
  {journal} {Phys. Rev. Lett.}\ }\textbf {\bibinfo {volume} {114}},\ \bibinfo
  {pages} {108001} (\bibinfo {year} {2015})}\BibitemShut {NoStop}%
\bibitem [{\citenamefont {Boattini}\ \emph {et~al.}(2020)\citenamefont
  {Boattini}, \citenamefont {Marin-Aguilar}, \citenamefont {Mitra},
  \citenamefont {Foffi}, \citenamefont {Smallenburg},\ and\ \citenamefont
  {Filion}}]{boattini2020}%
  \BibitemOpen
  \bibfield  {author} {\bibinfo {author} {\bibfnamefont {E.}~\bibnamefont
  {Boattini}}, \bibinfo {author} {\bibfnamefont {S.}~\bibnamefont
  {Marin-Aguilar}}, \bibinfo {author} {\bibfnamefont {S.}~\bibnamefont
  {Mitra}}, \bibinfo {author} {\bibfnamefont {G.}~\bibnamefont {Foffi}},
  \bibinfo {author} {\bibfnamefont {F.}~\bibnamefont {Smallenburg}},\ and\
  \bibinfo {author} {\bibfnamefont {L.}~\bibnamefont {Filion}},\ }\bibfield
  {title} {\bibinfo {title} {Autonomously revealing hidden local structures in
  supercooled liquids},\ }\href@noop {} {\bibfield  {journal} {\bibinfo
  {journal} {Nature Communication}\ }\textbf {\bibinfo {volume} {11}},\
  \bibinfo {pages} {5479} (\bibinfo {year} {2020})}\BibitemShut {NoStop}%
\bibitem [{\citenamefont {Boattini}\ \emph {et~al.}(2019)\citenamefont
  {Boattini}, \citenamefont {Dijkstra},\ and\ \citenamefont
  {Filion}}]{boattini2019jcp}%
  \BibitemOpen
  \bibfield  {author} {\bibinfo {author} {\bibfnamefont {E.}~\bibnamefont
  {Boattini}}, \bibinfo {author} {\bibfnamefont {M.}~\bibnamefont {Dijkstra}},\
  and\ \bibinfo {author} {\bibfnamefont {L.}~\bibnamefont {Filion}},\
  }\bibfield  {title} {\bibinfo {title} {Unsupervised learning for local
  structure detection in colloidal systems},\ }\href@noop {} {\bibfield
  {journal} {\bibinfo  {journal} {Journal of Chemical Physics}\ }\textbf
  {\bibinfo {volume} {151}},\ \bibinfo {pages} {154901} (\bibinfo {year}
  {2019})}\BibitemShut {NoStop}%
\bibitem [{\citenamefont {Campos-Villalobos}\ \emph {et~al.}(2021)\citenamefont
  {Campos-Villalobos}, \citenamefont {Boattini}, \citenamefont {Filion},\ and\
  \citenamefont {Dijkstra}}]{gerardo2021}%
  \BibitemOpen
  \bibfield  {author} {\bibinfo {author} {\bibfnamefont {G.}~\bibnamefont
  {Campos-Villalobos}}, \bibinfo {author} {\bibfnamefont {E.}~\bibnamefont
  {Boattini}}, \bibinfo {author} {\bibfnamefont {L.}~\bibnamefont {Filion}},\
  and\ \bibinfo {author} {\bibfnamefont {M.}~\bibnamefont {Dijkstra}},\
  }\bibfield  {title} {\bibinfo {title} {Machine learning many-body potentials
  for colloidal systems},\ }\href@noop {} {\bibfield  {journal} {\bibinfo
  {journal} {The Journal of Chemical Physics}\ }\textbf {\bibinfo {volume}
  {155}},\ \bibinfo {pages} {174902} (\bibinfo {year} {2021})}\BibitemShut
  {NoStop}%
\bibitem [{\citenamefont {Paret}\ \emph {et~al.}(2020)\citenamefont {Paret},
  \citenamefont {Jack},\ and\ \citenamefont {Coslovich}}]{joris2020}%
  \BibitemOpen
  \bibfield  {author} {\bibinfo {author} {\bibfnamefont {J.}~\bibnamefont
  {Paret}}, \bibinfo {author} {\bibfnamefont {R.~L.}\ \bibnamefont {Jack}},\
  and\ \bibinfo {author} {\bibfnamefont {D.}~\bibnamefont {Coslovich}},\
  }\bibfield  {title} {\bibinfo {title} {Assembly of reconfigurable colloidal
  structures by multidirectional field-induced interactions},\ }\href@noop {}
  {\bibfield  {journal} {\bibinfo  {journal} {The Journal of Chemical Physics}\
  }\textbf {\bibinfo {volume} {152}},\ \bibinfo {pages} {144502} (\bibinfo
  {year} {2020})}\BibitemShut {NoStop}%
\bibitem [{\citenamefont {Frank}(1952)}]{frank1952}%
  \BibitemOpen
  \bibfield  {author} {\bibinfo {author} {\bibfnamefont {F.~C.}\ \bibnamefont
  {Frank}},\ }\bibfield  {title} {\bibinfo {title} {Supercooling of liquids},\
  }\href@noop {} {\bibfield  {journal} {\bibinfo  {journal} {Proc. R. Soc. A.}\
  }\textbf {\bibinfo {volume} {215}},\ \bibinfo {pages} {43} (\bibinfo {year}
  {1952})}\BibitemShut {NoStop}%
\bibitem [{\citenamefont {Wales}(2004)}]{wales}%
  \BibitemOpen
  \bibfield  {author} {\bibinfo {author} {\bibfnamefont {D.~J.}\ \bibnamefont
  {Wales}},\ }\href@noop {} {\emph {\bibinfo {title} {Energy Landscapes:
  Applications to Clusters, Biomolecules and Glasses}}}\ (\bibinfo  {publisher}
  {Cambridge University Press},\ \bibinfo {address} {Cambridge},\ \bibinfo
  {year} {2004})\BibitemShut {NoStop}%
\bibitem [{\citenamefont {Wales}\ and\ \citenamefont {Doye}(1997)}]{wales1997}%
  \BibitemOpen
  \bibfield  {author} {\bibinfo {author} {\bibfnamefont {D.~J.}\ \bibnamefont
  {Wales}}\ and\ \bibinfo {author} {\bibfnamefont {J.~P.~K.}\ \bibnamefont
  {Doye}},\ }\bibfield  {title} {\bibinfo {title} {Global optimization by
  basin-hopping and the lowest energy structures of lennard-jones clusters
  containing up to 110 atoms},\ }\href@noop {} {\bibfield  {journal} {\bibinfo
  {journal} {Journal of Physical Chemistry A}\ }\textbf {\bibinfo {volume}
  {101}},\ \bibinfo {pages} {5111?5116} (\bibinfo {year} {1997})}\BibitemShut
  {NoStop}%
\bibitem [{\citenamefont {Miller}\ and\ \citenamefont
  {Wales}(2005)}]{miller2005}%
  \BibitemOpen
  \bibfield  {author} {\bibinfo {author} {\bibfnamefont {M.~A.}\ \bibnamefont
  {Miller}}\ and\ \bibinfo {author} {\bibfnamefont {D.}~\bibnamefont {Wales}},\
  }\bibfield  {title} {\bibinfo {title} {Novel structural motifs in clusters of
  dipolar spheres: Knots, links, and coils},\ }\href@noop {} {\bibfield
  {journal} {\bibinfo  {journal} {J. Phys. Chem. B}\ }\textbf {\bibinfo
  {volume} {109}},\ \bibinfo {pages} {23109} (\bibinfo {year}
  {2005})}\BibitemShut {NoStop}%
\bibitem [{\citenamefont {Malins}\ \emph {et~al.}(2013)\citenamefont {Malins},
  \citenamefont {Williams}, \citenamefont {Eggers},\ and\ \citenamefont
  {Royall}}]{malins2013tcc}%
  \BibitemOpen
  \bibfield  {author} {\bibinfo {author} {\bibfnamefont {A.}~\bibnamefont
  {Malins}}, \bibinfo {author} {\bibfnamefont {S.~R.}\ \bibnamefont
  {Williams}}, \bibinfo {author} {\bibfnamefont {J.}~\bibnamefont {Eggers}},\
  and\ \bibinfo {author} {\bibfnamefont {C.~P.}\ \bibnamefont {Royall}},\
  }\bibfield  {title} {\bibinfo {title} {Identification of structure in
  condensed matter with the topological cluster classification},\ }\href@noop
  {} {\bibfield  {journal} {\bibinfo  {journal} {Journal of Chemical Physics}\
  }\textbf {\bibinfo {volume} {139}},\ \bibinfo {pages} {234506} (\bibinfo
  {year} {2013})}\BibitemShut {NoStop}%
\bibitem [{\citenamefont {Skipper}\ \emph {et~al.}(2023)\citenamefont
  {Skipper}, \citenamefont {Moore},\ and\ \citenamefont
  {Royall}}]{skipper2023}%
  \BibitemOpen
  \bibfield  {author} {\bibinfo {author} {\bibfnamefont {K.}~\bibnamefont
  {Skipper}}, \bibinfo {author} {\bibfnamefont {F.~J.}\ \bibnamefont {Moore}},\
  and\ \bibinfo {author} {\bibfnamefont {C.~P.}\ \bibnamefont {Royall}},\
  }\bibfield  {title} {\bibinfo {title} {Induced dipolar colloids beyond 2d
  strings: Identification and classification of novel clusters at low dipole
  strengths},\ }\href@noop {} {\bibfield  {journal} {\bibinfo  {journal} {in
  preparation}\ } (\bibinfo {year} {2023})}\BibitemShut {NoStop}%
\bibitem [{\citenamefont {Taffs}\ \emph
  {et~al.}(2010{\natexlab{a}})\citenamefont {Taffs}, \citenamefont {Malins},
  \citenamefont {Williams},\ and\ \citenamefont {Royall}}]{taffs2010}%
  \BibitemOpen
  \bibfield  {author} {\bibinfo {author} {\bibfnamefont {J.}~\bibnamefont
  {Taffs}}, \bibinfo {author} {\bibfnamefont {A.}~\bibnamefont {Malins}},
  \bibinfo {author} {\bibfnamefont {S.~R.}\ \bibnamefont {Williams}},\ and\
  \bibinfo {author} {\bibfnamefont {C.~P.}\ \bibnamefont {Royall}},\ }\bibfield
   {title} {\bibinfo {title} {A structural comparison of models of
  colloid-polymer mixtures},\ }\href@noop {} {\bibfield  {journal} {\bibinfo
  {journal} {J. Phys.: Condens. Matter}\ }\textbf {\bibinfo {volume} {22}},\
  \bibinfo {pages} {104119} (\bibinfo {year} {2010}{\natexlab{a}})}\BibitemShut
  {NoStop}%
\bibitem [{\citenamefont {Klix}\ \emph {et~al.}(2013)\citenamefont {Klix},
  \citenamefont {Murata}, \citenamefont {Tanaka}, \citenamefont {Williams},
  \citenamefont {Malins},\ and\ \citenamefont {Royall}}]{klix2013}%
  \BibitemOpen
  \bibfield  {author} {\bibinfo {author} {\bibfnamefont {C.~L.}\ \bibnamefont
  {Klix}}, \bibinfo {author} {\bibfnamefont {K.}~\bibnamefont {Murata}},
  \bibinfo {author} {\bibfnamefont {H.}~\bibnamefont {Tanaka}}, \bibinfo
  {author} {\bibfnamefont {S.~R.}\ \bibnamefont {Williams}}, \bibinfo {author}
  {\bibfnamefont {A.}~\bibnamefont {Malins}},\ and\ \bibinfo {author}
  {\bibfnamefont {C.~P.}\ \bibnamefont {Royall}},\ }\bibfield  {title}
  {\bibinfo {title} {Novel kinetic trapping in charged colloidal clusters due
  to self-induced surface charge organization},\ }\href
  {https://doi.org/10.1038/srep02072} {\bibfield  {journal} {\bibinfo
  {journal} {Sci. Rep.}\ }\textbf {\bibinfo {volume} {3}},\ \bibinfo {pages}
  {2072} (\bibinfo {year} {2013})}\BibitemShut {NoStop}%
\bibitem [{\citenamefont {Taffs}\ \emph
  {et~al.}(2010{\natexlab{b}})\citenamefont {Taffs}, \citenamefont {Malins},
  \citenamefont {Williams},\ and\ \citenamefont {Royall}}]{taffs2010jcp}%
  \BibitemOpen
  \bibfield  {author} {\bibinfo {author} {\bibfnamefont {J.}~\bibnamefont
  {Taffs}}, \bibinfo {author} {\bibfnamefont {A.}~\bibnamefont {Malins}},
  \bibinfo {author} {\bibfnamefont {S.~R.}\ \bibnamefont {Williams}},\ and\
  \bibinfo {author} {\bibfnamefont {C.~P.}\ \bibnamefont {Royall}},\ }\bibfield
   {title} {\bibinfo {title} {The effect of attractions on the local structure
  of liquids and colloidal fluids},\ }\href@noop {} {\bibfield  {journal}
  {\bibinfo  {journal} {J. Chem. Phys.}\ }\textbf {\bibinfo {volume} {133}},\
  \bibinfo {pages} {244901} (\bibinfo {year} {2010}{\natexlab{b}})}\BibitemShut
  {NoStop}%
\bibitem [{\citenamefont {Malins}\ \emph {et~al.}(2010)\citenamefont {Malins},
  \citenamefont {Williams}, \citenamefont {Eggers}, \citenamefont {Tanaka},\
  and\ \citenamefont {Royall}}]{malins2010}%
  \BibitemOpen
  \bibfield  {author} {\bibinfo {author} {\bibfnamefont {A.}~\bibnamefont
  {Malins}}, \bibinfo {author} {\bibfnamefont {S.~R.}\ \bibnamefont
  {Williams}}, \bibinfo {author} {\bibfnamefont {J.}~\bibnamefont {Eggers}},
  \bibinfo {author} {\bibfnamefont {H.}~\bibnamefont {Tanaka}},\ and\ \bibinfo
  {author} {\bibfnamefont {C.~P.}\ \bibnamefont {Royall}},\ }\bibfield  {title}
  {\bibinfo {title} {The effect of inter-cluster interactions on the structure
  of colloidal clusters},\ }\href@noop {} {\bibfield  {journal} {\bibinfo
  {journal} {J. Non-Cryst. Solids}\ }\textbf {\bibinfo {volume} {357}},\
  \bibinfo {pages} {760} (\bibinfo {year} {2010})}\BibitemShut {NoStop}%
\bibitem [{\citenamefont {Godogna}\ \emph {et~al.}(2010)\citenamefont
  {Godogna}, \citenamefont {Malins}, \citenamefont {Williams},\ and\
  \citenamefont {Royall}}]{godonoga2010}%
  \BibitemOpen
  \bibfield  {author} {\bibinfo {author} {\bibfnamefont {M.}~\bibnamefont
  {Godogna}}, \bibinfo {author} {\bibfnamefont {A.}~\bibnamefont {Malins}},
  \bibinfo {author} {\bibfnamefont {S.~R.}\ \bibnamefont {Williams}},\ and\
  \bibinfo {author} {\bibfnamefont {C.~P.}\ \bibnamefont {Royall}},\ }\bibfield
   {title} {\bibinfo {title} {The local structure of the gas-liquid
  interfaces},\ }\href@noop {} {\bibfield  {journal} {\bibinfo  {journal} {Mol.
  Phys.}\ }\textbf {\bibinfo {volume} {109}},\ \bibinfo {pages} {1393}
  (\bibinfo {year} {2010})}\BibitemShut {NoStop}%
\bibitem [{\citenamefont {Royall}\ \emph {et~al.}(2003)\citenamefont {Royall},
  \citenamefont {Leunissen},\ and\ \citenamefont {van Blaaderen}}]{royall2003}%
  \BibitemOpen
  \bibfield  {author} {\bibinfo {author} {\bibfnamefont {C.~P.}\ \bibnamefont
  {Royall}}, \bibinfo {author} {\bibfnamefont {M.~E.}\ \bibnamefont
  {Leunissen}},\ and\ \bibinfo {author} {\bibfnamefont {A.}~\bibnamefont {van
  Blaaderen}},\ }\bibfield  {title} {\bibinfo {title} {A new colloidal model
  system to study long-range interactions quantitatively in real space},\
  }\href@noop {} {\bibfield  {journal} {\bibinfo  {journal} {J. Phys.: Condens.
  Matter}\ }\textbf {\bibinfo {volume} {15}},\ \bibinfo {pages} {S3581}
  (\bibinfo {year} {2003})}\BibitemShut {NoStop}%
\bibitem [{\citenamefont {Royall}\ \emph {et~al.}(2006)\citenamefont {Royall},
  \citenamefont {Leunissen}, \citenamefont {Hyninnen}, \citenamefont
  {Dijkstra},\ and\ \citenamefont {van Blaaderen}}]{royall2006}%
  \BibitemOpen
  \bibfield  {author} {\bibinfo {author} {\bibfnamefont {C.~P.}\ \bibnamefont
  {Royall}}, \bibinfo {author} {\bibfnamefont {M.~E.}\ \bibnamefont
  {Leunissen}}, \bibinfo {author} {\bibfnamefont {A.-P.}\ \bibnamefont
  {Hyninnen}}, \bibinfo {author} {\bibfnamefont {M.}~\bibnamefont {Dijkstra}},\
  and\ \bibinfo {author} {\bibfnamefont {A.}~\bibnamefont {van Blaaderen}},\
  }\bibfield  {title} {\bibinfo {title} {Re-entrant melting and freezing in a
  model system of charged colloids},\ }\href@noop {} {\bibfield  {journal}
  {\bibinfo  {journal} {J. Chem. Phys.}\ }\textbf {\bibinfo {volume} {124}},\
  \bibinfo {pages} {244706} (\bibinfo {year} {2006})}\BibitemShut {NoStop}%
\bibitem [{\citenamefont {Campbell}\ and\ \citenamefont
  {Bartlett}(2002)}]{campbell2002}%
  \BibitemOpen
  \bibfield  {author} {\bibinfo {author} {\bibfnamefont {A.~I.}\ \bibnamefont
  {Campbell}}\ and\ \bibinfo {author} {\bibfnamefont {P.}~\bibnamefont
  {Bartlett}},\ }\bibfield  {title} {\bibinfo {title} {Fluorescent hard-sphere
  colloids for confocal microscopy.},\ }\href@noop {} {\bibfield  {journal}
  {\bibinfo  {journal} {J. Coll. Interf. Sci.}\ }\textbf {\bibinfo {volume}
  {256}},\ \bibinfo {pages} {325} (\bibinfo {year} {2002})}\BibitemShut
  {NoStop}%
\bibitem [{\citenamefont {{Hollingsworth}}\ \emph {et~al.}(2006)\citenamefont
  {{Hollingsworth}}, \citenamefont {{Russel}}, \citenamefont {{van Kats}},\
  and\ \citenamefont {{van Blaaderen}}}]{hollingsworth2006}%
  \BibitemOpen
  \bibfield  {author} {\bibinfo {author} {\bibfnamefont {A.}~\bibnamefont
  {{Hollingsworth}}}, \bibinfo {author} {\bibfnamefont {W.}~\bibnamefont
  {{Russel}}}, \bibinfo {author} {\bibfnamefont {C.}~\bibnamefont {{van
  Kats}}},\ and\ \bibinfo {author} {\bibfnamefont {A.}~\bibnamefont {{van
  Blaaderen}}},\ }\bibfield  {title} {\bibinfo {title} {{Preparation of
  PHSA-PMMA stabilizer for model hard sphere systems}},\ }in\ \href@noop {}
  {\emph {\bibinfo {booktitle} {APS March Meeting Abstracts}}},\ \bibinfo
  {series and number} {APS Meeting Abstracts}\ (\bibinfo {year} {2006})\ p.\
  \bibinfo {pages} {G21.005}\BibitemShut {NoStop}%
\bibitem [{\citenamefont {Royall}\ \emph {et~al.}(2005)\citenamefont {Royall},
  \citenamefont {van Roij},\ and\ \citenamefont {van Blaaderen}}]{royall2005s}%
  \BibitemOpen
  \bibfield  {author} {\bibinfo {author} {\bibfnamefont {C.~P.}\ \bibnamefont
  {Royall}}, \bibinfo {author} {\bibfnamefont {R.}~\bibnamefont {van Roij}},\
  and\ \bibinfo {author} {\bibfnamefont {A.}~\bibnamefont {van Blaaderen}},\
  }\bibfield  {title} {\bibinfo {title} {Extended sedimentation profiles in
  charged colloids: the gravitational length, entropy, and electrostatics},\
  }\href@noop {} {\bibfield  {journal} {\bibinfo  {journal} {J. Phys.: Condens.
  Matter}\ }\textbf {\bibinfo {volume} {17}},\ \bibinfo {pages} {2315}
  (\bibinfo {year} {2005})}\BibitemShut {NoStop}%
\bibitem [{\citenamefont {Donovan}(2006)}]{donovan}%
  \BibitemOpen
  \bibfield  {author} {\bibinfo {author} {\bibfnamefont {A.}~\bibnamefont
  {Donovan}},\ }\emph {\bibinfo {title} {Arrest in Colloids with Competing
  Attractive and Repulsive Interactions}},\ \href@noop {} {Ph.D. thesis},\
  \bibinfo  {school} {University of Bristol} (\bibinfo {year}
  {2006})\BibitemShut {NoStop}%
\bibitem [{Note1()}]{Note1}%
  \BibitemOpen
  \bibinfo {note} {The polydispersity determined for these particles was 3.8\%
  with static light scattering and 9.1\% from TEM~\cite {donovan}. That the
  particles crystallize readily suggests that their polydispersity was around
  5\% or less.}\BibitemShut {Stop}%
\bibitem [{\citenamefont {Taffs}\ \emph {et~al.}(2012)\citenamefont {Taffs},
  \citenamefont {Williams}, \citenamefont {Tanaka},\ and\ \citenamefont
  {Royall}}]{taffs2013}%
  \BibitemOpen
  \bibfield  {author} {\bibinfo {author} {\bibfnamefont {J.}~\bibnamefont
  {Taffs}}, \bibinfo {author} {\bibfnamefont {S.~R.}\ \bibnamefont {Williams}},
  \bibinfo {author} {\bibfnamefont {H.}~\bibnamefont {Tanaka}},\ and\ \bibinfo
  {author} {\bibfnamefont {C.~P.}\ \bibnamefont {Royall}},\ }\bibfield  {title}
  {\bibinfo {title} {Structure and kinetics in the freezing of nearly hard
  spheres},\ }\href@noop {} {\bibfield  {journal} {\bibinfo  {journal} {Soft
  Matter}\ }\textbf {\bibinfo {volume} {9}},\ \bibinfo {pages} {297} (\bibinfo
  {year} {2012})}\BibitemShut {NoStop}%
\bibitem [{\citenamefont {Royall}\ \emph {et~al.}(2013)\citenamefont {Royall},
  \citenamefont {Poon},\ and\ \citenamefont {Weeks}}]{royall2013myth}%
  \BibitemOpen
  \bibfield  {author} {\bibinfo {author} {\bibfnamefont {C.~P.}\ \bibnamefont
  {Royall}}, \bibinfo {author} {\bibfnamefont {W.~C.~K.}\ \bibnamefont
  {Poon}},\ and\ \bibinfo {author} {\bibfnamefont {E.~R.}\ \bibnamefont
  {Weeks}},\ }\bibfield  {title} {\bibinfo {title} {In search of colloidal hard
  spheres},\ }\href@noop {} {\bibfield  {journal} {\bibinfo  {journal} {Soft
  Matter}\ }\textbf {\bibinfo {volume} {9}},\ \bibinfo {pages} {17} (\bibinfo
  {year} {2013})}\BibitemShut {NoStop}%
\bibitem [{\citenamefont {Poon}\ \emph {et~al.}(2012)\citenamefont {Poon},
  \citenamefont {Weeks},\ and\ \citenamefont {Royall}}]{poon2012}%
  \BibitemOpen
  \bibfield  {author} {\bibinfo {author} {\bibfnamefont {W.~C.~K.}\
  \bibnamefont {Poon}}, \bibinfo {author} {\bibfnamefont {E.~R.}\ \bibnamefont
  {Weeks}},\ and\ \bibinfo {author} {\bibfnamefont {C.~P.}\ \bibnamefont
  {Royall}},\ }\bibfield  {title} {\bibinfo {title} {On measuring colloidal
  volume fractions},\ }\href {https://doi.org/10.1039/CISM06083J} {\bibfield
  {journal} {\bibinfo  {journal} {Soft Matter}\ }\textbf {\bibinfo {volume}
  {8}},\ \bibinfo {pages} {21} (\bibinfo {year} {2012})}\BibitemShut {NoStop}%
\bibitem [{\citenamefont {Leunissen}(2006)}]{leunissenThesis}%
  \BibitemOpen
  \bibfield  {author} {\bibinfo {author} {\bibfnamefont {M.}~\bibnamefont
  {Leunissen}},\ }\emph {\bibinfo {title} {Manipulating Colloids with Charge
  and Electric Fields}},\ \href@noop {} {Ph.D. thesis},\ \bibinfo  {school}
  {Utrecht Universiteit} (\bibinfo {year} {2006})\BibitemShut {NoStop}%
\bibitem [{\citenamefont {Statt}\ \emph {et~al.}(2016)\citenamefont {Statt},
  \citenamefont {Pinchaipat}, \citenamefont {Turci}, \citenamefont {Evans},\
  and\ \citenamefont {Royall}}]{statt2016}%
  \BibitemOpen
  \bibfield  {author} {\bibinfo {author} {\bibfnamefont {A.}~\bibnamefont
  {Statt}}, \bibinfo {author} {\bibfnamefont {R.}~\bibnamefont {Pinchaipat}},
  \bibinfo {author} {\bibfnamefont {F.}~\bibnamefont {Turci}}, \bibinfo
  {author} {\bibfnamefont {R.}~\bibnamefont {Evans}},\ and\ \bibinfo {author}
  {\bibfnamefont {C.~P.}\ \bibnamefont {Royall}},\ }\bibfield  {title}
  {\bibinfo {title} {Direct observation in 3d of structural crossover in binary
  hard sphere mixtures},\ }\href@noop {} {\bibfield  {journal} {\bibinfo
  {journal} {J. Chem. Phys.}\ }\textbf {\bibinfo {volume} {144}},\ \bibinfo
  {pages} {144506} (\bibinfo {year} {2016})}\BibitemShut {NoStop}%
\bibitem [{\citenamefont {Crocker}\ and\ \citenamefont
  {Grier}(1995)}]{crocker1995}%
  \BibitemOpen
  \bibfield  {author} {\bibinfo {author} {\bibfnamefont {J.~C.}\ \bibnamefont
  {Crocker}}\ and\ \bibinfo {author} {\bibfnamefont {D.~G.}\ \bibnamefont
  {Grier}},\ }\bibfield  {title} {\bibinfo {title} {Methods of digital video
  microscopy for colloidal studies},\ }\href@noop {} {\bibfield  {journal}
  {\bibinfo  {journal} {J. Coll. Interf. Sci.}\ }\textbf {\bibinfo {volume}
  {179}},\ \bibinfo {pages} {298} (\bibinfo {year} {1995})}\BibitemShut
  {NoStop}%
\bibitem [{\citenamefont {Leocmach}\ and\ \citenamefont
  {Tanaka}(2013)}]{leocmach2013sm}%
  \BibitemOpen
  \bibfield  {author} {\bibinfo {author} {\bibfnamefont {M.}~\bibnamefont
  {Leocmach}}\ and\ \bibinfo {author} {\bibfnamefont {H.}~\bibnamefont
  {Tanaka}},\ }\bibfield  {title} {\bibinfo {title} {A novel particle tracking
  method with individual particle size measurement and its application to
  ordering in glassy hard sphere colloids},\ }\href@noop {} {\bibfield
  {journal} {\bibinfo  {journal} {Soft Matter}\ }\textbf {\bibinfo {volume}
  {9}},\ \bibinfo {pages} {1447} (\bibinfo {year} {2013})}\BibitemShut
  {NoStop}%
\bibitem [{\citenamefont {Royall}\ \emph {et~al.}(2023)\citenamefont {Royall},
  \citenamefont {Charbonneau}, \citenamefont {Dijkstra}, \citenamefont {Russo},
  \citenamefont {Smallenburg}, \citenamefont {Speck},\ and\ \citenamefont
  {Valeriani}}]{royall2023}%
  \BibitemOpen
  \bibfield  {author} {\bibinfo {author} {\bibfnamefont {C.~P.}\ \bibnamefont
  {Royall}}, \bibinfo {author} {\bibfnamefont {P.}~\bibnamefont {Charbonneau}},
  \bibinfo {author} {\bibfnamefont {M.}~\bibnamefont {Dijkstra}}, \bibinfo
  {author} {\bibfnamefont {J.}~\bibnamefont {Russo}}, \bibinfo {author}
  {\bibfnamefont {F.}~\bibnamefont {Smallenburg}}, \bibinfo {author}
  {\bibfnamefont {T.}~\bibnamefont {Speck}},\ and\ \bibinfo {author}
  {\bibfnamefont {C.}~\bibnamefont {Valeriani}},\ }\bibfield  {title} {\bibinfo
  {title} {Colloidal hard spheres: Triumphs, challenges and mysteries},\
  }\href@noop {} {\bibfield  {journal} {\bibinfo  {journal} {submitted to Rev.
  Mod. Phys. online at ArXiV}\ ,\ \bibinfo {pages} {2305.02452}} (\bibinfo
  {year} {2023})}\BibitemShut {NoStop}%
\bibitem [{\citenamefont {Yang}(2021)}]{yang2021}%
  \BibitemOpen
  \bibfield  {author} {\bibinfo {author} {\bibfnamefont {Y.}~\bibnamefont
  {Yang}},\ }\href@noop {} {\bibinfo {title} {nplocate}},\ \bibinfo
  {howpublished} {\url{https://github.com/yangyushi/nplocate}} (\bibinfo {year}
  {2021})\BibitemShut {NoStop}%
\bibitem [{\citenamefont {Royall}\ \emph {et~al.}(2007)\citenamefont {Royall},
  \citenamefont {Louis},\ and\ \citenamefont {Tanaka}}]{royall2007jcp}%
  \BibitemOpen
  \bibfield  {author} {\bibinfo {author} {\bibfnamefont {C.~P.}\ \bibnamefont
  {Royall}}, \bibinfo {author} {\bibfnamefont {A.~A.}\ \bibnamefont {Louis}},\
  and\ \bibinfo {author} {\bibfnamefont {H.}~\bibnamefont {Tanaka}},\
  }\bibfield  {title} {\bibinfo {title} {Measuring colloidal interactions with
  confocal microscopy},\ }\href {https://doi.org/10.1063/1.2755962} {\bibfield
  {journal} {\bibinfo  {journal} {J. Chem. Phys.}\ }\textbf {\bibinfo {volume}
  {127}},\ \bibinfo {eid} {044507} (\bibinfo {year} {2007})}\BibitemShut
  {NoStop}%
\bibitem [{\citenamefont {Royall}\ and\ \citenamefont
  {Malins}(2012)}]{royall2012}%
  \BibitemOpen
  \bibfield  {author} {\bibinfo {author} {\bibfnamefont {C.~P.}\ \bibnamefont
  {Royall}}\ and\ \bibinfo {author} {\bibfnamefont {A.}~\bibnamefont
  {Malins}},\ }\bibfield  {title} {\bibinfo {title} {The role of quench rate in
  colloidal gels},\ }\href@noop {} {\bibfield  {journal} {\bibinfo  {journal}
  {Faraday Discussion}\ }\textbf {\bibinfo {volume} {158}},\ \bibinfo {pages}
  {301} (\bibinfo {year} {2012})}\BibitemShut {NoStop}%
\bibitem [{\citenamefont {Plimpton}(1995)}]{plimpton1995}%
  \BibitemOpen
  \bibfield  {author} {\bibinfo {author} {\bibfnamefont {S.}~\bibnamefont
  {Plimpton}},\ }\bibfield  {title} {\bibinfo {title} {Fast parallel algorithms
  for short-range molecular dynamics},\ }\href@noop {} {\bibfield  {journal}
  {\bibinfo  {journal} {Journal of Computational Physics}\ }\textbf {\bibinfo
  {volume} {117}},\ \bibinfo {pages} {1} (\bibinfo {year} {1995})}\BibitemShut
  {NoStop}%
\bibitem [{\citenamefont {Moore}\ \emph {et~al.}(2021)\citenamefont {Moore},
  \citenamefont {Royall}, \citenamefont {Liverpool},\ and\ \citenamefont
  {Russo}}]{moore2021}%
  \BibitemOpen
  \bibfield  {author} {\bibinfo {author} {\bibfnamefont {F.~J.}\ \bibnamefont
  {Moore}}, \bibinfo {author} {\bibfnamefont {C.~P.}\ \bibnamefont {Royall}},
  \bibinfo {author} {\bibfnamefont {T.~B.}\ \bibnamefont {Liverpool}},\ and\
  \bibinfo {author} {\bibfnamefont {J.}~\bibnamefont {Russo}},\ }\bibfield
  {title} {\bibinfo {title} {Crystallisation and polymorph selection in active
  brownian particles},\ }\href@noop {} {\bibfield  {journal} {\bibinfo
  {journal} {The European Physical Journal E}\ }\textbf {\bibinfo {volume}
  {44}},\ \bibinfo {pages} {121} (\bibinfo {year} {2021})}\BibitemShut
  {NoStop}%
\bibitem [{\citenamefont {Moore}\ \emph {et~al.}(2023)\citenamefont {Moore},
  \citenamefont {J.}, \citenamefont {Liverpool},\ and\ \citenamefont
  {Royall}}]{moore2023}%
  \BibitemOpen
  \bibfield  {author} {\bibinfo {author} {\bibfnamefont {F.~J.}\ \bibnamefont
  {Moore}}, \bibinfo {author} {\bibfnamefont {R.}~\bibnamefont {J.}}, \bibinfo
  {author} {\bibfnamefont {T.~B.}\ \bibnamefont {Liverpool}},\ and\ \bibinfo
  {author} {\bibfnamefont {C.~P.}\ \bibnamefont {Royall}},\ }\bibfield  {title}
  {\bibinfo {title} {Active brownian particles in random and porous
  environments},\ }\href {https://doi.org/https://doi.org/10.1063/5.0131340}
  {\bibfield  {journal} {\bibinfo  {journal} {J. Chem. Phys.}\ }\textbf
  {\bibinfo {volume} {158}},\ \bibinfo {pages} {104907} (\bibinfo {year}
  {2023})}\BibitemShut {NoStop}%
\bibitem [{\citenamefont {Weeks}\ \emph {et~al.}(1971)\citenamefont {Weeks},
  \citenamefont {Chandler},\ and\ \citenamefont {Andersen}}]{weeks1971}%
  \BibitemOpen
  \bibfield  {author} {\bibinfo {author} {\bibfnamefont {J.~D.}\ \bibnamefont
  {Weeks}}, \bibinfo {author} {\bibfnamefont {D.}~\bibnamefont {Chandler}},\
  and\ \bibinfo {author} {\bibfnamefont {H.~C.}\ \bibnamefont {Andersen}},\
  }\bibfield  {title} {\bibinfo {title} {{Role of Repulsive Forces in
  Determining the Equilibrium Structure of Simple Liquids}},\ }\href
  {https://doi.org/10.1063/1.1674820} {\bibfield  {journal} {\bibinfo
  {journal} {J. Chem. Phys.}\ }\textbf {\bibinfo {volume} {54}},\ \bibinfo
  {pages} {5237} (\bibinfo {year} {1971})}\BibitemShut {NoStop}%
\bibitem [{\citenamefont {Barker}\ and\ \citenamefont
  {Henderson}(1976)}]{barker1976}%
  \BibitemOpen
  \bibfield  {author} {\bibinfo {author} {\bibfnamefont {J.~A.}\ \bibnamefont
  {Barker}}\ and\ \bibinfo {author} {\bibfnamefont {D.}~\bibnamefont
  {Henderson}},\ }\bibfield  {title} {\bibinfo {title} {What is "liquid"?
  understanding the states of matter},\ }\href@noop {} {\bibfield  {journal}
  {\bibinfo  {journal} {Rev. Mod. Phys.}\ }\textbf {\bibinfo {volume} {48}},\
  \bibinfo {pages} {587} (\bibinfo {year} {1976})}\BibitemShut {NoStop}%
\bibitem [{\citenamefont {Doye}\ \emph {et~al.}(1995)\citenamefont {Doye},
  \citenamefont {Wales},\ and\ \citenamefont {Berry}}]{doye1995}%
  \BibitemOpen
  \bibfield  {author} {\bibinfo {author} {\bibfnamefont {J.~P.~K.}\
  \bibnamefont {Doye}}, \bibinfo {author} {\bibfnamefont {D.~J.}\ \bibnamefont
  {Wales}},\ and\ \bibinfo {author} {\bibfnamefont {R.~S.}\ \bibnamefont
  {Berry}},\ }\bibfield  {title} {\bibinfo {title} {The effect of the range of
  the potential on the structures of clusters},\ }\href@noop {} {\bibfield
  {journal} {\bibinfo  {journal} {J. Chem. Phys.}\ }\textbf {\bibinfo {volume}
  {103}},\ \bibinfo {pages} {4234} (\bibinfo {year} {1995})}\BibitemShut
  {NoStop}%
\bibitem [{\citenamefont {Selinger}(2016)}]{selinger2016}%
  \BibitemOpen
  \bibfield  {author} {\bibinfo {author} {\bibfnamefont {J.~V.}\ \bibnamefont
  {Selinger}},\ }\bibinfo {title} {Liquid crystals},\ in\ \href
  {https://doi.org/10.1007/978-3-319-21054-4_10} {\emph {\bibinfo {booktitle}
  {Introduction to the Theory of Soft Matter: From Ideal Gases to Liquid
  Crystals}}}\ (\bibinfo  {publisher} {Springer International Publishing},\
  \bibinfo {address} {Cham},\ \bibinfo {year} {2016})\ pp.\ \bibinfo {pages}
  {131--182}\BibitemShut {NoStop}%
\bibitem [{\citenamefont {Coslovich}(2013)}]{coslovich2013jcp}%
  \BibitemOpen
  \bibfield  {author} {\bibinfo {author} {\bibfnamefont {D.}~\bibnamefont
  {Coslovich}},\ }\bibfield  {title} {\bibinfo {title} {Static triplet
  correlations in glass-forming liquids: A molecular dynamics study},\ }\href
  {https://doi.org/10.1063/1.4773355} {\bibfield  {journal} {\bibinfo
  {journal} {The Journal of Chemical Physics}\ }\textbf {\bibinfo {volume}
  {138}},\ \bibinfo {pages} {12A539} (\bibinfo {year} {2013})},\ \Eprint
  {https://arxiv.org/abs/https://doi.org/10.1063/1.4773355}
  {https://doi.org/10.1063/1.4773355} \BibitemShut {NoStop}%
\bibitem [{\citenamefont {Royall}\ \emph {et~al.}(2018)\citenamefont {Royall},
  \citenamefont {Williams},\ and\ \citenamefont {Tanaka}}]{royall2018jcp}%
  \BibitemOpen
  \bibfield  {author} {\bibinfo {author} {\bibfnamefont {C.~P.}\ \bibnamefont
  {Royall}}, \bibinfo {author} {\bibfnamefont {S.~R.}\ \bibnamefont
  {Williams}},\ and\ \bibinfo {author} {\bibfnamefont {H.}~\bibnamefont
  {Tanaka}},\ }\bibfield  {title} {\bibinfo {title} {Vitrification and gelation
  in sticky spheres},\ }\href@noop {} {\bibfield  {journal} {\bibinfo
  {journal} {J. Chem. Phys.}\ }\textbf {\bibinfo {volume} {148}},\ \bibinfo
  {pages} {044501} (\bibinfo {year} {2018})}\BibitemShut {NoStop}%
\bibitem [{\citenamefont {Malins}\ \emph {et~al.}(2009)\citenamefont {Malins},
  \citenamefont {Williams}, \citenamefont {Eggers}, \citenamefont {Tanaka},\
  and\ \citenamefont {Royall}}]{malins2009}%
  \BibitemOpen
  \bibfield  {author} {\bibinfo {author} {\bibfnamefont {A.}~\bibnamefont
  {Malins}}, \bibinfo {author} {\bibfnamefont {S.~R.}\ \bibnamefont
  {Williams}}, \bibinfo {author} {\bibfnamefont {J.}~\bibnamefont {Eggers}},
  \bibinfo {author} {\bibfnamefont {H.}~\bibnamefont {Tanaka}},\ and\ \bibinfo
  {author} {\bibfnamefont {C.~P.}\ \bibnamefont {Royall}},\ }\bibfield  {title}
  {\bibinfo {title} {Geometric frustration in small colloidal clusters},\
  }\href@noop {} {\bibfield  {journal} {\bibinfo  {journal} {J. Phys.: Condens.
  Matter}\ }\textbf {\bibinfo {volume} {21}},\ \bibinfo {pages} {425103}
  (\bibinfo {year} {2009})}\BibitemShut {NoStop}%
\bibitem [{\citenamefont {Royall}\ \emph
  {et~al.}(2015{\natexlab{b}})\citenamefont {Royall}, \citenamefont {Malins},
  \citenamefont {Dunleavy},\ and\ \citenamefont
  {Pinney}}]{royall2015jnonxtalsol}%
  \BibitemOpen
  \bibfield  {author} {\bibinfo {author} {\bibfnamefont {C.~P.}\ \bibnamefont
  {Royall}}, \bibinfo {author} {\bibfnamefont {A.}~\bibnamefont {Malins}},
  \bibinfo {author} {\bibfnamefont {A.~J.}\ \bibnamefont {Dunleavy}},\ and\
  \bibinfo {author} {\bibfnamefont {R.}~\bibnamefont {Pinney}},\ }\bibfield
  {title} {\bibinfo {title} {Strong geometric frustration in model
  glassformers},\ }\href@noop {} {\bibfield  {journal} {\bibinfo  {journal}
  {Journal of Non-Crystalline Solids}\ }\textbf {\bibinfo {volume} {407}},\
  \bibinfo {pages} {34} (\bibinfo {year} {2015}{\natexlab{b}})}\BibitemShut
  {NoStop}%
\bibitem [{\citenamefont {Leoni}\ \emph {et~al.}(2023)\citenamefont {Leoni},
  \citenamefont {Martelli},\ and\ \citenamefont {Royall}}]{leoni2023}%
  \BibitemOpen
  \bibfield  {author} {\bibinfo {author} {\bibfnamefont {F.}~\bibnamefont
  {Leoni}}, \bibinfo {author} {\bibfnamefont {F.}~\bibnamefont {Martelli}},\
  and\ \bibinfo {author} {\bibfnamefont {C.~P.}\ \bibnamefont {Royall}},\
  }\bibfield  {title} {\bibinfo {title} {Structural signatures of
  ultrastability in a deposited glassformer},\ }\href@noop {} {\bibfield
  {journal} {\bibinfo  {journal} {Phys. Rev. Lett.}\ }\textbf {\bibinfo
  {volume} {130}},\ \bibinfo {pages} {198201} (\bibinfo {year}
  {2023})}\BibitemShut {NoStop}%
\bibitem [{\citenamefont {Jenkinson}\ \emph {et~al.}(2017)\citenamefont
  {Jenkinson}, \citenamefont {Crowther}, \citenamefont {Turci},\ and\
  \citenamefont {P.}}]{jenkinson2017}%
  \BibitemOpen
  \bibfield  {author} {\bibinfo {author} {\bibfnamefont {T.}~\bibnamefont
  {Jenkinson}}, \bibinfo {author} {\bibfnamefont {P.}~\bibnamefont {Crowther}},
  \bibinfo {author} {\bibfnamefont {F.}~\bibnamefont {Turci}},\ and\ \bibinfo
  {author} {\bibfnamefont {R.~C.}\ \bibnamefont {P.}},\ }\bibfield  {title}
  {\bibinfo {title} {Weak temperature dependence of ageing of structural
  properties in atomistic model glassformers},\ }\href@noop {} {\bibfield
  {journal} {\bibinfo  {journal} {J. Chem. Phys.}\ }\textbf {\bibinfo {volume}
  {147}},\ \bibinfo {pages} {054501} (\bibinfo {year} {2017})}\BibitemShut
  {NoStop}%
\bibitem [{\citenamefont {Thijssen}\ \emph {et~al.}(2023)\citenamefont
  {Thijssen}, \citenamefont {Liverpool}, \citenamefont {Royall},\ and\
  \citenamefont {Jack}}]{thijssen2023}%
  \BibitemOpen
  \bibfield  {author} {\bibinfo {author} {\bibfnamefont {K.}~\bibnamefont
  {Thijssen}}, \bibinfo {author} {\bibfnamefont {T.~B.}\ \bibnamefont
  {Liverpool}}, \bibinfo {author} {\bibfnamefont {C.~P.}\ \bibnamefont
  {Royall}},\ and\ \bibinfo {author} {\bibfnamefont {R.~L.}\ \bibnamefont
  {Jack}},\ }\bibfield  {title} {\bibinfo {title} {Necking and failure of a
  particulate gel strand: signatures of yielding on different length scales},\
  }\href@noop {} {\bibfield  {journal} {\bibinfo  {journal} {Soft Matter}\
  }\textbf {\bibinfo {volume} {19}},\ \bibinfo {pages} {7412} (\bibinfo {year}
  {2023})}\BibitemShut {NoStop}%
\bibitem [{\citenamefont {Coslovich}(2011)}]{coslovich2011}%
  \BibitemOpen
  \bibfield  {author} {\bibinfo {author} {\bibfnamefont {D.}~\bibnamefont
  {Coslovich}},\ }\bibfield  {title} {\bibinfo {title} {Locally preferred
  structures and many-body static correlations in viscous liquids},\ }\href
  {https://doi.org/10.1103/PhysRevE.83.051505} {\bibfield  {journal} {\bibinfo
  {journal} {Phys. Rev. E}\ }\textbf {\bibinfo {volume} {83}},\ \bibinfo
  {pages} {051505} (\bibinfo {year} {2011})}\BibitemShut {NoStop}%
\bibitem [{\citenamefont {Leocmach}\ and\ \citenamefont
  {Tanaka}(2012)}]{leocmach2012}%
  \BibitemOpen
  \bibfield  {author} {\bibinfo {author} {\bibfnamefont {M.}~\bibnamefont
  {Leocmach}}\ and\ \bibinfo {author} {\bibfnamefont {H.}~\bibnamefont
  {Tanaka}},\ }\bibfield  {title} {\bibinfo {title} {Roles of icosahedral and
  crystal-like order in the hard spheres glass transition},\ }\href
  {https://doi.org/doi:10.1038/ncomms1974} {\bibfield  {journal} {\bibinfo
  {journal} {Nature Comm.}\ }\textbf {\bibinfo {volume} {3}},\ \bibinfo {pages}
  {974} (\bibinfo {year} {2012})}\BibitemShut {NoStop}%
\bibitem [{\citenamefont {Furukawa}\ and\ \citenamefont
  {Tanaka}(2010)}]{furukawa2010}%
  \BibitemOpen
  \bibfield  {author} {\bibinfo {author} {\bibfnamefont {A.}~\bibnamefont
  {Furukawa}}\ and\ \bibinfo {author} {\bibfnamefont {H.}~\bibnamefont
  {Tanaka}},\ }\bibfield  {title} {\bibinfo {title} {Key role of hydrodynamic
  interactions in colloidal gelation},\ }\href@noop {} {\bibfield  {journal}
  {\bibinfo  {journal} {Phys. Rev. Lett.}\ }\textbf {\bibinfo {volume} {104}},\
  \bibinfo {pages} {245702} (\bibinfo {year} {2010})}\BibitemShut {NoStop}%
\bibitem [{\citenamefont {de~Graaf}\ \emph {et~al.}(2019)\citenamefont
  {de~Graaf}, \citenamefont {Poon}, \citenamefont {Haughey},\ and\
  \citenamefont {Hermes}}]{degraaf2019}%
  \BibitemOpen
  \bibfield  {author} {\bibinfo {author} {\bibfnamefont {J.}~\bibnamefont
  {de~Graaf}}, \bibinfo {author} {\bibfnamefont {W.~C.~K.}\ \bibnamefont
  {Poon}}, \bibinfo {author} {\bibfnamefont {M.~J.}\ \bibnamefont {Haughey}},\
  and\ \bibinfo {author} {\bibfnamefont {M.}~\bibnamefont {Hermes}},\
  }\bibfield  {title} {\bibinfo {title} {Hydrodynamics strongly affect the
  dynamics of colloidal gelation but not gel structure},\ }\href
  {https://doi.org/DOI: 10.1039/C8SM01611A} {\bibfield  {journal} {\bibinfo
  {journal} {Soft Matter}\ }\textbf {\bibinfo {volume} {15}},\ \bibinfo {pages}
  {10} (\bibinfo {year} {2019})}\BibitemShut {NoStop}%
\end{thebibliography}%



\end{document}



