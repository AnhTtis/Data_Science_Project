% !TeX encoding = UTF-8
%\preprint{APS/123-QED}

% \title{Supplemental Material for Probing Complex-energy Topology via Non-Hermitian Absorption Spectroscopy in a Trapped Ion Simulator}% Force line breaks with \\
% \author{M.-M. Cao}
% \thanks{These authors contribute equally to this work}%
% \affiliation{Center for Quantum Information, Institute for Interdisciplinary Information Sciences, Tsinghua University, Beijing 100084, PR China}
% %\affiliation{Hefei National Laboratory, Hefei 230088, PR China}

% \author{K. Li}
% \thanks{These authors contribute equally to this work}%
% \affiliation{Center for Quantum Information, Institute for Interdisciplinary Information Sciences, Tsinghua University, Beijing 100084, PR China}
% %\affiliation{Hefei National Laboratory, Hefei 230088, PR China}

% \author{W.-D. Zhao}
% \affiliation{HYQ Co., Ltd., Beijing 100176, PR China}

% \author{W.-X. Guo}
% \affiliation{Center for Quantum Information, Institute for Interdisciplinary Information Sciences, Tsinghua University, Beijing 100084, PR China}

% \author{B.-X. Qi}
% \affiliation{Center for Quantum Information, Institute for Interdisciplinary Information Sciences, Tsinghua University, Beijing 100084, PR China}
% %\affiliation{Hefei National Laboratory, Hefei 230088, PR China}

% \author{X.-Y. Chang}
% \affiliation{Center for Quantum Information, Institute for Interdisciplinary Information Sciences, Tsinghua University, Beijing 100084, PR China}
% %\author{L. Yao}
% %\affiliation{HYQ Co., Ltd., Beijing 100176, PR China}

% \author{Z.-C. Zhou}
% \affiliation{Center for Quantum Information, Institute for Interdisciplinary Information Sciences, Tsinghua University, Beijing 100084, PR China}
% \affiliation{Hefei National Laboratory, Hefei 230088, PR China}

% \author{Y. Xu}
% \email{yongxuphy@tsinghua.edu.cn}
% \affiliation{Center for Quantum Information, Institute for Interdisciplinary Information Sciences, Tsinghua University, Beijing 100084, PR China}
% \affiliation{Hefei National Laboratory, Hefei 230088, PR China}

% \author{L.-M. Duan}
% \email{lmduan@tsinghua.edu.cn}
% \affiliation{Center for Quantum Information, Institute for Interdisciplinary Information Sciences, Tsinghua University, Beijing 100084, PR China}
% \affiliation{Hefei National Laboratory, Hefei 230088, PR China}

% \maketitle


In the Supplemental Material, we will
derive the effective non-Hermitian Hamiltonian based on the master equation in Section S-1,
show the experimental details about our system in Section S-2,
provide detailed analysis about errors in our experiments in Section S-3,
and finally demonstrate the feasibility of detecting the complex energy spectra using a shorter evolution time in Section S-4.

\section{S-1. Derivation of the Effective Hamiltonian}
In this section, we will derive the effective Hamiltonian $H_f$ in the main text based on the master equation.
We start with the entire system including the system levels $\ket{0}$ and $\ket{1}$, the other two Zeeman states $\ket{2}$ and $\ket{3}$, the excited state $\ket{e}$ and the auxiliary level $\ket{a}$. 
The dynamics of the entire system is governed by the master equation ($\hbar = 1$)
\begin{equation}
	\label{fullME}
	\begin{aligned}
		\frac{d \rho}{dt} &= -i [H,\rho] + \sum_{\mu=1}^{3} (2 L_\mu \rho L_\mu^\dagger - \{ L_\mu^\dagger L_\mu, \rho \} )
		\\& = -i H_\text{eff} \rho + i \rho H_\text{eff}^\dagger + \sum_{\mu=1}^{3} 2 L_\mu \rho L_\mu^\dagger.
	\end{aligned}
\end{equation}
Here $H = H_{h} + H_{c} + J_L (\ket{1} \bra{e} + \ket{e} \bra{1})$ with $H_{h} = [(J_x - i J_y) \ket{0}\bra{1} + \text{H.c.}] - 2 J_z \ket{1} \bra{1}$ being the Hermitian part of the system Hamiltonian (for generality, we here also consider the $\sigma_y$ term), $H_c = \frac{\Omega}{2} (\ket{0} \bra{a} + \ket{a} \bra{0}) - \delta \ket{a} \bra{a}$ denoting the coupling between system and auxiliary levels, and $J_L$ denoting the coupling strength between the $\ket{1}$ level and an excited $P_{1/2}$ state $\ket{e}$, 
the Lindblad operators are defined as $L_\mu = \sqrt{\Gamma_\mu} \ket{\mu} \bra{e}$ for $\mu = 1,2,3$,
and the effective Hamiltonian is given by $H_\text{eff} = H - i \sum_{\mu=1}^{3} L_\mu^\dagger L_\mu = H - i \Gamma_e \ket{e} \bra{e}$ with $\Gamma_e = \sum_{\mu=1}^{3} \Gamma_\mu = 1/\tau_e$, $\tau_e$ being the lifetime of the $\ket{e}$ level.
For the $^{171} \text{Yb}^{+}$ ion, we have $\tau_e \approx 8.12$ ns and $\Gamma_e \approx 123$ MHz~\cite{yb171plusData}.
We assume that $\Gamma_e$ is much larger than $J_L$ so that $\rho_{ei} = \bra{e} \rho \ket{i} \sim 0$ and $\partial_t \rho_{ei} \sim 0$ for any $\ket{i}$, thus we may adiabatically eliminate the $\ket{e}$ level to obtain the dynamics for the rest of the system.


Specifically, we first write down all the entries of $\rho$ that involve the $\ket{e}$ level, given by
\begin{align}
	\partial_t \rho_{e0} &= -i J_L \rho_{10} - \Gamma_e \rho_{e0} + i (J_x + i J_y) \rho_{e1} + \frac{i\Omega}{2} \rho_{ea},
	\\ \partial_t \rho_{e1} &= -i J_L \rho_{11} + i(J_x - i J_y) \rho_{e0} - (\Gamma_e + 2 i J_z) \rho_{e1} + i J_z \rho_{ee},
	\\ \partial_t \rho_{e2} &= 0, \label{rhoe2}
	\\ \partial_t \rho_{e3} &= 0, \label{rhoe3}
	\\ \partial_t \rho_{ea} &= -i J_L \rho_{1a} + \frac{i\Omega}{2} \rho_{e0} - (\Gamma_e + i \delta) \rho_{ea},
	\\ \partial_t \rho_{ee} &= -2 \Gamma_e \rho_{ee} - i J_L \rho_{1e} + i J_L \rho_{e1}.
\end{align}
Since $\rho(t=0) = \ket{a} \bra{a}$, by Eqs.~(\ref{rhoe2}) and (\ref{rhoe3}) we have $\rho_{e2}(t) = 0$ and $\rho_{e3}(t) = 0$.
Then we apply the adiabatic elimination by assuming $\partial_t \rho_{ei} \approx 0$ for $i=0,1,a,e$, resulting in a system of linear equations
\begin{equation}
	\left\{ 
	\begin{aligned}
		& - \Gamma_e \rho_{e0} + i (J_x + i J_y) \rho_{e1} + \frac{i\Omega}{2} \rho_{ea} \approx  i J_L \rho_{10},
		\\&  i(J_x - i J_y) \rho_{e0} - (\Gamma_e + 2 i J_z) \rho_{e1} + i J_z \rho_{ee} \approx  i J_L \rho_{11},
		\\&  \frac{i\Omega}{2} \rho_{e0} - (\Gamma_e + i \delta) \rho_{ea} \approx   i J_L \rho_{1a},
		\\&  -2 \Gamma_e \rho_{ee} - i J_L \rho_{1e} + i J_L \rho_{e1} \approx 0,
	\end{aligned}
	\right.
\end{equation}
with four variables $\rho_{e0}$, $\rho_{e1}$, $\rho_{ea}$ and $\rho_{ee}$. 
Given that $\Gamma_e \gg J_x, J_y, J_z, J_L, \Omega$, the above linear system can be further approximated by
\begin{equation}
	\left\{ 
	\begin{aligned}
		&    - \Gamma_e \rho_{e0} \approx  i J_L \rho_{10},
		\\&  - \Gamma_e \rho_{e1} \approx  i J_L \rho_{11},
		\\&  - \Gamma_e \rho_{ea} \approx  i J_L \rho_{1a},
		\\&  -2 \Gamma_e \rho_{ee} - i J_L \rho_{1e} + i J_L \rho_{e1} \approx 0,
	\end{aligned}
	\right.
\end{equation}
whose solution is given by
\begin{equation}
	\label{AdiabaticEliminationResults}
	\left\{ 
	\begin{aligned}
		&      \rho_{e0} \approx - i (J_L/\Gamma_e) \rho_{10},
		\\&    \rho_{e1} \approx - i (J_L/\Gamma_e) \rho_{11},
		\\&    \rho_{ea} \approx - i (J_L/\Gamma_e) \rho_{1a},
		\\&  \rho_{ee}  \approx  (J_L^2/\Gamma_e^2) \rho_{11}.
		%&      \rho_{e0} \approx - i \frac{J_L}{\Gamma_e} \rho_{10},
		%\\&    \rho_{e1} \approx - i \frac{J_L}{\Gamma_e} \rho_{11},
		%\\&    \rho_{ea} \approx - i \frac{J_L}{\Gamma_e} \rho_{1a},
		%\\&  \rho_{ee}  \approx  \frac{J_L^2}{\Gamma_e} \rho_{11}.
	\end{aligned}
	\right.
\end{equation}
Eq.~(\ref{AdiabaticEliminationResults}) can be used to eliminate the $\ket{e}$ level in Eq.~(\ref{fullME}).

\begin{figure}
	\centering
	\includegraphics[width=0.4\textwidth]{FigS1.pdf}
	\caption{The spectral lines calculated using the master equation (\ref{fullME}) (solid blue line) and the non-Hermitian Hamiltonian in Eq.~(\ref{EffHam}) (dashed red line).
		Two lines coincide with each other, implying that the dynamics of the master equation is very well characterized by the dynamics of the non-Hermitian Hamiltonian.
		The system parameters we take here are the same as in Fig. 2(a1) in the main text.
		Here, $\Gamma_1=\Gamma_2=\Gamma_3=\Gamma_e/3$ and $J_L = 4.76$ MHz.
	} 
	\label{fig1}
\end{figure}

Now we project the entire system to a subsystem consisting of $\ket{0}$, $\ket{1}$ and $\ket{a}$ levels using the projection operator $P_f = \ket{0}\bra{0} + \ket{1} \bra{1} + \ket{a} \bra{a}$.
We apply the projection operator $P_f$ on Eq.~(\ref{fullME}) and get
\begin{equation}
	\begin{aligned}
		\frac{d \rho_f}{dt} &= -i P_f H_\text{eff} \rho P_f + i P_f \rho H_\text{eff}^\dagger P_f + P_f \sum_{\mu=1}^{3} 2 L_\mu \rho L_\mu^\dagger P_f
		\\ &= -i [H_{f0}, \rho_f]  + [2 \Gamma_1 \rho_{ee} + i J_L (\rho_{1e} - \rho_{e1})]  \ket{1}\bra{1} + i J_L (\rho_{0e} \ket{0}\bra{1} - \rho_{e0} \ket{1} \bra{0} + \rho_{ae} \ket{a}\bra{1} - \rho_{ea} \ket{1}\bra{a}),
	\end{aligned}
\end{equation}
with $\rho_f = P_f \rho P_f$ and $H_{f0} = P_f H P_f = H_{h} + H_c$.
Using Eq.~(\ref{AdiabaticEliminationResults}), we obtain
\begin{equation}
	\begin{aligned}
		\frac{d \rho_f}{dt} & \approx  -i [H_{f0}, \rho_f]  - 2 (\Gamma_e - \Gamma_1) \frac{J_L^2}{\Gamma_e^2} \rho_{11}  \ket{1}\bra{1} - \frac{J_L^2}{\Gamma_e} (\rho_{01} \ket{0}\bra{1} + \rho_{10} \ket{1} \bra{0} + \rho_{a1} \ket{a}\bra{1} + \rho_{1a} \ket{1}\bra{a})
		\\& \begin{aligned}
			\  = &-i [H_{f0}, \rho_f] - 2 (\Gamma_e - \Gamma_1) \frac{J_L^2}{\Gamma_e^2} \ket{1}\bra{1} \rho_f \ket{1}\bra{1} 
			\\& - \frac{J_L^2}{\Gamma_e} (\ket{0}\bra{0}\rho_f\ket{1}\bra{1} +  \ket{1} \bra{1}\rho_f \ket{0} \bra{0} +  \ket{a} \bra{a}\rho_f \ket{1} \bra{1} +  \ket{1} \bra{1} \rho_f \ket{a} \bra{a})
		\end{aligned}
		\\&= -i [H_{f0}, \rho_f] - 2 (\Gamma_e - \Gamma_1) \frac{J_L^2}{\Gamma_e^2} \ket{1}\bra{1} \rho_f \ket{1}\bra{1} + \frac{J_L^2}{\Gamma_e} ( 2 \ket{1} \bra{1} \rho_f\ket{1}\bra{1} - \rho_f\ket{1}\bra{1}  -  \ket{1} \bra{1} \rho_f )
		\\&= -i [H_{f0}, \rho_f] - \frac{J_L^2}{\Gamma_e} ( \rho_f\ket{1}\bra{1}  +  \ket{1} \bra{1} \rho_f ) + 2  \frac{J_L^2 \Gamma_1}{\Gamma_e^2} \ket{1}\bra{1} \rho_f \ket{1}\bra{1} .
	\end{aligned}
\end{equation}
Let $H_f = H_{f0} - (i J_L^2/ \Gamma_e) \ket{1} \bra{1} $, one can find that
\begin{equation}
	\label{withM}
	\frac{d \rho_f}{dt} = -i H_{f} \rho_f + i \rho_f H_f^\dagger   + 2  \frac{J_L^2 \Gamma_1}{\Gamma_e^2} \rho_{11} \ket{1}\bra{1} .
\end{equation}
Furthermore, the dynamics of $\rho_{11}$,
\begin{equation}
	\frac{d \rho_{11}}{dt} = -i \bra{1} [H_{f0},\rho_f] \ket{1} - 2 (\Gamma_2 + \Gamma_3) \frac{ J_L^2}{\Gamma_e^2}  \rho_{11},
\end{equation}
suggesting that $\rho_{11} \approx 0$ because of the dissipation term $- 2 (\Gamma_2 + \Gamma_3) \frac{ J_L^2}{\Gamma_e^2}  \rho_{11}$ on the right-hand side.
Thus, Eq.~(\ref{withM}) can be further approximated by
\begin{equation}
	\label{withoutM}
	\frac{d \rho_f}{dt} \approx -i H_{f} \rho_f + i \rho_f H_f^\dagger.
\end{equation}
Based on Eq.~(\ref{withoutM}), we conclude that the dynamics of a system including $\ket{0}$, $\ket{1}$ and $\ket{a}$ levels is described by an effective non-Hermitian Hamiltonian, 
\begin{equation}
	\label{EffHam}
	H_f =  J_x \sigma_x + J_y \sigma_y - 2 (J_z + i \gamma) \ket{1} \bra{1} + \frac{\Omega}{2} (\ket{0} \bra{a} + \ket{a} \bra{0}) - \delta \ket{a} \bra{a},
\end{equation}
with the decay rate being $\gamma = J_L^2 / (2 \Gamma_e)$ which can be tuned by varying the laser power controlling the value of $J_L$.
We have also numerically verified that the dynamics of the master equation (\ref{fullME}) is very well characterized by the dynamics based on the effective 
non-Hermitian Hamiltonian in Eq.~(\ref{EffHam}), as illustrated in Fig.~\ref{fig1}.


\section{S-2. Experimental setup} \label{sec1}

In this section, we will show the experimental details about our system.
In our experiment, the $^{171} \text{Yb}^{+}$ ion is confined by electrical fields of a linear Paul trap with segmented blade electrodes. 
We apply a $2\pi \times 22.68$ MHz RF field on two blades to generate pseudo potential in the radial direction, and DC voltages with proper gradient on $5$ pairs of segments in the other two blades to provide confinement in the axial direction. 
The qubits are encoded by $\ket{0} \equiv |^{2}{S}_{1/2}, F=0, m_F=0 \rangle$ and $\ket{1} \equiv |^{2}{S}_{1/2}, F=1, m_F=0 \rangle$ in the $^{171} \text{Yb}^{+}$  ground state manifold with hyperfine splitting $\omega_\text{HF}=12.642812$ GHz. 
One microwave with frequency $\omega_\text{MW1}=\omega_\text{HF}+\Delta_B -\Delta_\text{MW1}$ drives the transition between the two qubit states, where $\Delta_B=310.8 \ B^2$ Hz, $B$ is the magnetic field in unit of Gauss (Gs) and $\Delta_\text{MW1}$ is the detuning. 
Another microwave with frequency $\omega_\text{MW2} = \omega_\text{HF}+\Delta_B-\Delta_\text{MW2}$ is used to compensate the AC Stark shift. 
Each microwave frequency is generated by mixing one channel of a $1$ GS/s arbitrary wave generator (AWG) with a stable $12.4$ GHz signal source. 
All signal generators are carefully synchronized to a Rubidium clock with $10$ MHz reference signal using equal length of wires.
We use a pair of Helmholtz coils to create a magnetic field which is aligned perpendicular to the light path and the surface. 
As a result, we get a magnetic field around $8.5$ Gs and the corresponding Zeeman splitting of $\ket{^2 {S}_{1/2}, F=1}$ levels is approximately $12$ MHz. 
%Zeeman splitting of $12MHz$ between $|^2S_{1/2},F=1,m_F=0\rangle$ and $|^2S_{1/2},F=1,m_F=\pm1\rangle$ by a 8.5$Gs$ magnetic field.

A $369$ nm laser beam is used to cool the ion with the aid of $14.7$ GHz electro-optic modulator (EOM) sideband, optically pump the ion to the ground state $\ket{0}$ using a $2.11$ GHz EOM sideband, and detect the state of the ion after turning off all sidebands~\cite{Bruzewicz2019}. 
We also use a $935$ nm laser beam with a $3.07$ GHz EOM sideband to repump the leakage to $^{2}D_{3/2}$ metastable levels back to the Doppler cooling cycle. 
For detection, the $369$ nm laser beam containing both $\sigma_{\pm}$ and $\pi$ polarized components is shined on the ion to excite all Zeeman levels in $|^2S_{1/2}, F=1\rangle$ to the excited state $\ket{e} = |^2P_{1/2}, F=0, m_F = 0\rangle$, which will decay in several nanoseconds. 
Fluorescence photons generated by spontaneous emission from the $|e\rangle$ level will be collected by a homemade object with NA=$0.13$, 
and then imaged by a photon-multiplier tube (PMT) controlled by a field-programmable gate array (FPGA). 
We use a threshold method to identify the state of an ion~\cite{Wineland1998,Zhao2022CP}. 
For a fixed detection time, if the measured photon count exceeds the detection threshold, the ion is identified as occupying the bright states in the $|^2S_{1/2},F=1\rangle$ manifold; otherwise, the ion is identified as occupying the dark state $|0\rangle$. 
Experimentally, we set the detection time to $400$ $\mu$s and the detection threshold to $1$, which leads to a detection fidelity of $98.0\%$ for the bright state.
%the average photon counts are $12.7$ and $0.09$ for the bright and dark state respectively.
Similiar to the detection of ion states, population on the auxiliary level $|a\rangle$ is detected based on this protocol with different definition of bright and dark states.
During the detection of $N_a$, the $14.7$ GHz EOM of the $369$ nm laser is turned on, which enables all transitions from $^2S_{1/2}$ to $^2P_{1/2}$, so that all states on $^2S_{1/2}$ are bright states. At the same time, the $3.07$ GHz EOM of 935nm repumping beam is turned off so that the state $\ket{a}$ cannot be repumped back, making it a dark state. 
As a result, we achieve a bright state detection fidelity up to $99.5\%$ using the same detection time and threshold.


The dissipation on the state $\ket{1}$ is realized by another 369 nm laser beam that only excites the $|1\rangle$ level to the excited state $|e\rangle$. 
Filtered by a Glan-Taylor polarizer, this beam contains only $\pi$-polarized components, 
% This beam is filtered by a Glan-Taylor polarizer and thus contains only the $\pi$-polarization components in the vertical magnetic field.
such that excitations from other Zeeman states ($|^2S_{1/2}, F=1, m_F = \pm1\rangle$) are blocked by selection rules.
The unstable excited state $|e\rangle$ will spontaneously decay to the Zeeman states on $|^2S_{1/2},F=1\rangle$ with equal probabilities. Consequently, there is a net loss of population on $|1\rangle$ whose rate is determined by the laser intensity.

A tilted beam of $435$ nm laser is used to couple the system level  $|0\rangle$ with the auxiliary level $|a\rangle$ by a quadrupole transition and also used for sideband cooling. 
To effectively drive this transition with a narrow linewidth ($3.02$ Hz), we generate the $435$ nm laser by doubling the 871 nm seed laser locked to a high-finesse ULE stable cavity by the PDH scheme~\cite{drever1983}. 
% We tilt this beam slightly with respect to the surface
As a result, the linewidth of the $435$ nm laser is $1.27$ kHz estimated from the full width at half maximum (FWHM) of the clock resonance 
($\ket{^2S_{1/2}, F=0, m_F=0} \rightarrow \ket{^{2} D_{3/2},F=2,m_F=0}$) spectrum. This is the upper bound of the actual linewidth owing to the stray AC magnetic field and drift of the cavity. 
In addition,  other lasers are stabilized to wavelength meter by a homemade program which can limit the long time frequency drift within $2$ MHz.
%The measured laser linewidth is $1.27kHz$ which is enough to drive the quadrupole transition and the measured laser linewidth of $435$ nm is $1.27kHz$.

In experiments, we first prepare the ion on the motional ground state by sideband cooling using the $435$ nm laser~\cite{osti_1595883}. The cooling beam is modulated by an acousto-optic modulator (AOM) using double-pass configuration, and thus the intensity and frequency components can be fine-tuned by changing the input signal. 
After $2$ ms Doppler cooling, we optically pump the ion to the ground state $|0\rangle$, and then tune the $435$ nm laser frequency to the red sideband in radial direction $\omega_x= 2.82$  MHz and $\omega_y=3.07$ MHz in turn. 
As a result, we reach the phonon number $\bar{n}_x<0.11$ and $\bar{n}_y < 0.15$ after $50$ rounds ($2$ ms) of sideband cooling.

The experimental sequence is controlled by FPGA running at $50$ MHz. 
All AOMs and EOMs can be switched in microseconds by the TTL signal generated from FPGA.
For one round of experiment, a typical experimental sequence is shown in Fig. 1(c) in the main text. 
We first cool the ion to the ground state by $2$ ms Doppler cooling followed by $2$ ms sideband cooling, and then prepare the ion on the auxiliary level $|a\rangle$ by a 10 $\mu$s $\pi$ flip using the $435$ nm beam. 
Next, we turn on the system Hamiltonian and $435$ nm weak coupling beam with detuning simultaneously.
After holding for $200$ $\mu$s of time evolution, we take $400$ $\mu$s to detect the state of the ion, which corresponds to one shot in the detection of the population $N_a$ on the auxiliary level.




\begin{figure}
	\centering
	\includegraphics[width=14cm]{FigS2.pdf}
	\caption{Numerical simulations about error sources. 
		(a) The effect of fluctuations in the intensity of the dissipation laser on the spectral lines.
		The boundaries of the filled region correspond to dissipation rate $\gamma_1 = 0.8\gamma$ and $\gamma_2 = 1.2\gamma$ with $\gamma=0.092$ MHz. 
		%The black dashed line marks the shift of the dip center as $\gamma$ changes.
		(b) Errors induced by the dephasing of the $435$ nm laser beam.
		%The blue solid line corresponds to the ideal case where $t_2=\infty$, and 
		The filled region is obtained by scanning $t_2$ from $200$ $\mu$s to $800$ $\mu$s. 
		%Consequently, the dip location keeps unchanged as represented by the black dashed line, which means that dephasing will not induce error in the real part of eigenenergy.
		In (a) and (b), the insets represent the spectral lines in a wider range of detuning, the blue solid lines are the theoretical spectral lines under the same parameters as Fig.~2(a1) in the main text, and the black dashed lines corresponds to locations of the dip center.
		%For a detailed study, we magnify the pink filled regions around the first absorption dip, and the bounds are obtained by scanning relative parameters.
	}
	\label{fig2}
\end{figure}

\section{S-3. Error Analysis}
In this section, we will provide more analysis about errors in our experiments. 
In one round of experiment, the parameters are relatively stable in our system, and thus we mainly focus on the state preparation and measurement errors and quantum projection noises.
The initial preparation of the ion on the auxiliary level is realized by transferring the state from $\ket{0}$ to $\ket{a}$ with a $\pi$-pulse using the $435$ nm laser. 
However, this initialization is not perfect due to the residual phonon and fluctuations of the laser frequency. 
In average, the preparation fidelity is $99.2\%$ which can be further improved by better sideband cooling and laser locking. 
% The detection of population $N_a$ on the auxiliary level that utilizes the threshold method has an average fidelity of $99.5\%$ when detection time is set to $400\mu s$,
%in sec\ref{sec1}
The detection of the population $N_a$ on the auxiliary level utilizes the threshold method, and reaches an average fidelity of $99.5\%$ when the detection time is set to $400$ $\mu$s;
such a fidelity can be improved using a higher NA object.
And the quantum projection noises can be suppressed by increasing the number of measurements. 
Therefore, we repeat the measurement of the final state for $1000$ times to acquire the expectation value.


However, during many rounds of experiments, the fluctuations of experimental parameters such as microwave or laser intensity, especially the intensity of the dissipation light, are the dominant errors, which cannot be simply eliminated by increasing the number of measurements. 
From Eq.~(\ref{EffHam}), one can find that fluctuations of the dissipation light will lead to the variation of $\gamma$, which introduces errors to the measurement of complex eigenenergies.
To demonstrate the effect of intensity fluctuations on the measurement results, we have numerically simulated the evolution of $N_a$ of the {modified} 
non-Hermitian Rice-Mele model under different dissipation rates.  
From the results shown in Fig.~\ref{fig2}(a), we find that as $\gamma$ increases, the absorption dip gets wider and shallower and the position of the dip center also moves slightly, indicating that fluctuations of the dissipation light will lead to errors in both the real and the imaginary parts of eigenenergies. 

Besides, the decoherence between the system level $\ket{0}$ and the auxiliary level $\ket{a}$ caused by the phase noise of the $435$ nm laser cannot be ignored. 
In our experiment, the typical dephasing time $t_2$ of a well locked $435$ nm laser ranges from $400$ $\mu$s to $800$ $\mu$s measured by the Ramsey fringes.
We have done a numerical simulation by considering the effect of laser dephasing in the master equation.
From the simulation results in Fig.~\ref{fig2}(b), one can find that when dephasing happens, the dip becomes shallower and wider while its location is unchanged, 
indicating that errors caused by the laser dephasing mainly reside in the imaginary parts of eigenenergies. 




\section{S-4. Detecting the complex energy spectra with a shorter evolution time}

\begin{figure}
	\centering
	\includegraphics[width=8cm]{FigS3.pdf}
	\caption{Numerically simulated spectral lines under different evolution time $t$, with other parameters being the same as Fig.~2(a1) in the main text.
		%As evolution time grows, the spectral line get sharper making it easier to locate the location and half width of the absorption dip.
		The position of the dip centers are marked by black dashed lines.
		%We expect the location of absorption dip unchanged as shown by the black dashed line.
	}
	\label{fig3}
\end{figure}


In this section, we will demonstrate the feasibility of detecting complex energies using a shorter evolution time.
To start with, we numerically simulate the spectral lines under different evolution times as shown in Fig.~\ref{fig3}.
We see that as evolution time increases, the spectral line gets sharper, making it easier to locate the position and half width of the absorption dip.
In contrast, for the shortest evolution time $t=100$ $\mu$s (blue solid line), a shallow absorption dip makes the information about the eigenenergy within the dip hard to be extracted, especially in the presence of experimental noises.
The results suggest that a longer evolution time may be beneficial for extracting the complex eigenenergies. 
In the main text, we have set the evolution time to $200$ $\mu$s considering the laser dephasing and fluctuations of system parameters.



However, it is yet a challenge for some systems to keep coherent and stable for a long evolution time. 
To demonstrate the feasibility of detecting complex energies in those systems, we adopt a shorter evolution time in the detection of complex eigenenergies.
%further shorten the evolution time $t$ to $80$ $\mu$s to s.
We plot the experimentally extracted complex spectra of the {modified} non-Hermitian Rice-Mele model in Fig.~\ref{fig4}, with the same system parameters as in Fig.~2(a1-d1) in the main text except that the evolution time is $t=80$ $\mu$s.
Compared with results in Fig.~2(a1-d1) in the main text, 
%the real parts of energies are robust against the evolution time as expected in Fig.~\ref{fig4}(c), while 
the errors in the imaginary parts, especially for those with greater $|\mathrm{Im}(E)|$, get larger as shown in Fig. \ref{fig4}(d).
This can be attributed to the fact that a smaller evolution time makes the dip shallower, and thus the information about imaginary parts become more sensitive to experimental noises.
In spite of larger errors in the imaginary parts, the experimental results still agree with theoretical ones well and clearly captures the loop structure of the energy spectra in the complex-plane, which demonstrates the feasibility of detecting the complex energy spectra in a short time. 
%increased sensitivity to experimental noises for a smaller evolution time.
%which can be accounted by the analysis we have done.

\begin{figure}
	\centering
	\includegraphics[width=\textwidth]{FigS4.pdf}
	\caption{Experimentally measured complex energies of the {modified} non-Hermitian Rice-Mele model. 
		The parameters are taken the same as in Fig. 2(a1-d1) in the main text except that the evolution time $t$ is $80$ $\mu$s. 
		(a) The spectral line obtained by scanning the detunning when the momentum $k=2\pi/5$, which is further used to extract the complex eigenenergies [the corresponding extracted energies are highlighted by red circles in (b)–(d)].
		(b) Complex eigenenergies in the complex-energy plane. 
		The real (c) and imaginary (d) parts of experimentally measured complex energies. 
	}
	\label{fig4}
\end{figure}



\begin{thebibliography}{99}

\bibitem{yb171plusData}
S. Olmschenk, D. Hayes, D. N. Matsukevich, P. Maunz, D. L. Moehring, K. C. Younge, and C. Monroe,
%Measurement of the lifetime of the $6p\text{ }{^{2}P}_{1/2}^{o}$ level of ${\text{Yb}}^{+}$,
Phys. Rev. A \textbf{80}, 022502 (2009).

\bibitem{Bruzewicz2019}
C. D. Bruzewicz, J. Chiaverini, R. McConnell, and J. M. Sage, 
%Trapped-ion quantum computing: Progress and challenges, 
Appl. Phys. Rev. \textbf{6}, 021314 (2019).

\bibitem{Wineland1998}
Wineland, D. J. et al.,
%Experimental Issues in Coherent Quantum-State Manipulation of Trapped Atomic Ions,
%\href{https://www.ncbi.nlm.nih.gov/pmc/articles/PMC4898965/}
{J. Res. Natl. Inst. Stand. Technol. \textbf{103,} 259 (1998).}

\bibitem{Zhao2022CP}
Zhao, W., Yang, Y.-B., Jiang, Y. et al. 
Commun Phys \textbf{5,} 223 (2022).

\bibitem{drever1983}
R. W. P. Drever, J. L. Hall, F. V. Kowalski, J. Hough, G. M. Ford, A. J. Munley, and H. Ward, 
%Laser phase and frequency stabilization using an optical resonator, 
Appl. Phys. B \textbf{31}, 97 (1983).

\bibitem{osti_1595883}
M. Revelle, C. W. Hogle, B. Ruzic, P. L. W. Maunz, K. Young, and D. Lobser,
%Demonstration of Sideband Cooling on the $^{171}\text{Yb}^{+}$ Quadrupole Transition,
Sandia National Laboratories Report No. SAND2019-0668C671715, 1 (2019).


\end{thebibliography}




% \end{document}
































