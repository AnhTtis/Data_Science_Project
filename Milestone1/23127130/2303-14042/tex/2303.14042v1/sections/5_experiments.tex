\section{Experiments}
\label{sec_experiments}

%
We incorporate CIM into two baseline CIL methods (i.e., DER~\cite{yan2021dynamically} and FOSTER~\cite{wang2022foster}) and boost their model performances consistently on three datasets.
%
Below, we introduce datasets and experiment settings (Section~\ref{subsection: experimental_settings}), followed by results and analyses (Section~\ref{subsection: results_and_analyses})

\subsection{Experimental Settings}
\label{subsection: experimental_settings}

\myparagraph{Datasets.}
We conduct experiments on three standard CIL benchmarks with high-resolution images.
%
1)~\textbf{Food-101}~\cite{bossard14food101} consists of $101$ food categories with $750$ training and $250$ test samples per category.
%
All images have a maximum side length of $512$ pixels.
%
2)~\textbf{ImageNet-1000}~\cite{deng2009imagenet} is a large-scale dataset with $1,\!000$ classes and each class has around $1,\!300$ training and $50$ test samples.
%
3)~\textbf{ImageNet-100} is a 100-class subset randomly sampled from ImageNet-1000 with a fixed NumPy~\cite{harris2020array} random seed ($1993$), following~\cite{hou2019lucir}.
%
We provide other details of these datasets, e.g., image sizes and pre-processing methods, in the supplementary materials.

\myparagraph{Protocols.}
We use two protocols: \textit{learning from scratch} (LFS) and \textit{learning from half} (LFH), following recent CIL works~\cite{yan2021dynamically,wang2022foster}.
%
In LFS, the model observes the same number of classes in all $N$ phases, where $N$ is optionally $5$, $10$, and $20$. 
%
In LFH, the model is trained on half of the classes (e.g., $500$ classes for ImageNet-1000) in the $1$-th phase. Then, it learns the remaining classes evenly in the subsequent $N$ phases, where $N$ can be $5$, $10$, and $25$.
%
In both protocols, after the training of each phase we evaluate the resultant model on the test data of all seen classes.
%
Our final report includes the average accuracy over all phases and the last-phase accuracy which indicates the degree of model forgetting.
%
We run each experiment three times and report the average results.

\myparagraph{Memory Budget.}
There are two memory budget\footnote{Please note that we measure the memory by the number of original images. Each compressed exemplar in CIM takes less memory than the original image (Eq.~\ref{eq_consumed_memory}), resulting in more exemplars in the same memory.} settings. 1)~In the ``fixed'' setting, we remove some old-class exemplars when new exemplars from the current phase are added in the memory to maintain the ``fixed memory budget''. In this setting, we set the total memory to be $2,\!020$ samples for Food-101 and $2,\!000$ samples for ImageNet-100. For ImageNet-1000, we have two options---$5,\!000$ samples and $20,\!000$ samples. 2)~In the ``growing'' setting, a constant memory budget is allocated for each class across all phases and hence extra memory is appended when new classes come. In this setting, we set the budget to be $20$ samples per class for all datasets. Following~\cite{yan2021dynamically,wang2022foster}, we apply the ``fixed'' setting in LFS experiments and the ``growing'' setting in LFH experiments.

\begin{table*}[t]
\normalsize
\begin{center}
\renewcommand\arraystretch{1}
\setlength{\tabcolsep}{1.6mm}{
\begin{threeparttable}
\begin{tabular}{lccccccccccccccc}
\toprule
\multirow{3.5}{*}{\textbf{Method}}
& \multicolumn{7}{c}{\textit{Learning from Scratch (LFS)}} && \multicolumn{7}{c}{\textit{Learning from Half (LFH)}} \\
\cmidrule{2-8} \cmidrule{10-16}
& \multicolumn{3}{c}{\textit{Food-101}} && \multicolumn{3}{c}{\textit{ImageNet-100}} && \multicolumn{3}{c}{\textit{Food-101}} && \multicolumn{3}{c}{\textit{ImageNet-100}} \\
\cmidrule{2-4} \cmidrule{6-8} \cmidrule{10-12} \cmidrule{14-16}
& $N$=5 & 10 & 20 && 5 & 10 & 20 && 5 & 10 & 25 && 5 & 10 & 25 \\
\hline
iCaRL~\cite{rebuffi2017icarl}               & 69.66 & 62.18 & 56.70 && 73.90 & 67.06 & 62.36 && 60.13 & 53.42 & 46.87 && 62.53 & 59.88 & 52.97  \\
WA~\cite{zhao2020maintaining}               & 70.94 & 63.69 & 58.45 && 74.64 & 68.62 & 63.20 && 63.55 & 57.60 & 52.48 && 65.75 & 63.71 & 58.34  \\
PODNet~\cite{douillard2020podnet}           & 68.03 & 61.24 & 47.38 && 72.14 & 63.96 & 53.69 && 75.37 & 70.01 & 65.32 && 75.54 & 74.33 & 68.33  \\
AANets~\cite{liu2021adaptive}               & 69.46 & 61.59 & 48.83 && 72.98 & 65.77 & 55.36 && 76.07 & 71.22 & 66.93 && 76.96 & 75.58 & 71.78  \\
\cdashline{1-16}[4pt/2pt]
DER~\cite{yan2021dynamically}               & 73.88 & 70.76 & 64.39 && 78.50 & 76.12 & 73.79 && 78.13 & 73.45 &   -   && 79.08 & 77.73 &   -    \\
DER~\textit{w}/ ours                        & 75.63 & 73.09 & 69.17 && 79.63 & 77.57 & \textbf{75.36} && 79.25 & 75.76 &   -   && 80.30 & \textbf{79.05} &   -    \\
\cdashline{1-16}[4pt/2pt]
FOSTER~\cite{wang2022foster}   & 75.03 & 72.72 & 66.73 && {79.93}\tnote{\dag} & {76.55}\tnote{\dag} & 74.49 && 79.08 & 75.07 & 68.08 && {80.07}\tnote{\dag} & 77.54 & 72.40\tnote{*}  \\
FOSTER~\textit{w}/ ours                     & \textbf{76.44} & \textbf{74.85} & \textbf{70.20} && \textbf{80.58} & \textbf{77.94} & 75.23 && \textbf{79.76} & \textbf{76.86} &\textbf{70.50} && \textbf{80.93} & 78.66 & \textbf{75.74}  \\
\bottomrule
\end{tabular}
{\small
\begin{tablenotes}
\item[\dag] The paper of FOSTER~\cite{wang2022foster} did not report the numerical results for $N$=$5$/$10$ (LFS) and $N$=$5$ (LFH) on ImageNet-100. We run these experiments using the public code (released by authors) and report the reproduced results.
\item[*] Our reproduced result ($72.40$) is significantly higher than the original result ($69.34$) reported in the paper of FOSTER.
\end{tablenotes}
}
\end{threeparttable}}
\end{center}
\vspace{-5mm}
\caption{Average accuracies (\%) of two top-performing CIL methods~\cite{yan2021dynamically,wang2022foster} with and without our CIM-CIL plugged-in, and other four baselines~\cite{rebuffi2017icarl,zhao2020maintaining,douillard2020podnet,liu2021adaptive}, on two datasets (Food-101 and ImageNet-100) and using two protocols (learning from scratch (LFS) and learning from half (LFH)). Due to the space limits, we report the $95\%$ confidence intervals for these results in the supplementary materials.
}
\vspace{-0.4cm}
\label{table: sota_food101_and_imagenet100}
\end{table*}

\myparagraph{Implementation Details.}
Our implementation is based on the standard deep learning library PyTorch~\cite{paszke2019pytorch} and image processing library OpenCV~\cite{mordvintsev2014opencv}. Following~\cite{yan2021dynamically,wang2022memory,wang2022foster}, we use an $18$-layer ResNet~\cite{he2016deep} as the network backbone $\theta$ and a fully-connected layer as the classifier $\omega$ in all experiments. We use the same CIL training hyperparameters as in related works~\cite{hou2019lucir,yan2021dynamically,wang2022foster} for fair comparison: 1)~there are $200$ epochs in $1$-st phase and $170$ epochs in the subsequent phases; 2)~the learning rate $\lambda$ is initialized as $0.1$ and decreases to zero with a cosine annealing scheduler~\cite{loshchilov2016sgdr}; 3)~the SGD optimizer is deployed, with momentum factor set to $0.9$ and weight decay set to $0.0005$. For compression-related hyperparameters, we set the masking threshold $\tau$ as $0.6$ and the downsampling ratio $\eta$ as $4.0$. To build the CIM model, we apply PAUs with degrees $m=5$ and $n=4$ as learnable activation layers. For the optimization of the CIM model $\phi$ (i.e., the mask-level optimization), we initially set $\beta_1$ as $0.1$ and $\beta_2$ as $0.01$ and reduce them to zero following the scheduler of $\lambda$. $\mu$ and $\mu^\prime$ is set to $0.1$ and $0.2$, respectively. To smooth the training, we clip the gradient norm of $\phi$ to be no more than $1$. \emph{We report the result of hyperparameter sensitivity analysis in the supplementary materials.}

\subsection{Results and Analyses}
\label{subsection: results_and_analyses}

\begin{table}[t]
\normalsize
\begin{center}
\renewcommand\arraystretch{1}
\setlength{\tabcolsep}{1.0mm}{
\begin{tabular}{llccccccc}
\toprule
\multirow{2.3}{*}{\makecell{\textbf{Memory}\\\textbf{Budget}}} & \multirow{2.5}{*}{\textbf{Method}} & \multicolumn{2}{c}{$N$=5} && \multicolumn{2}{c}{$N$=10} \\
\cmidrule{3-4} \cmidrule{6-7}
&& Avg. & Last && Avg. & Last \\
\hline
\multirow{5}{*}{$M\!=\!20k$}
& iCaRL~\cite{rebuffi2017icarl}                                     & 44.36 & 27.78 && 38.40 & 22.70  \\
& WA~\cite{zhao2020maintaining}                                     & 58.37 & 50.62 && 54.10 & 45.66  \\
& DER~\cite{yan2021dynamically}                                     & 67.49 & 59.75 && 66.73 & 58.62  \\
\cdashline{2-9}[4pt/2pt]
& FOSTER~\cite{wang2022foster}                                      & 69.21 & 64.88 && 68.34 & 60.14  \\
& FOSTER~\textit{w}/ ours                                           & \textbf{69.93} & \textbf{66.05} && \textbf{69.53} & \textbf{62.07}  \\
\cdashline{1-9}[4pt/2pt]
\multirow{2}{*}{$M\!=\!5k$}
& FOSTER                                                            & 57.19 & 49.42 && 54.72 & 44.96  \\
& FOSTER~\textit{w}/ ours                                           & \textbf{61.37} & \textbf{54.46} && \textbf{59.48} & \textbf{50.83}  \\
\bottomrule
\end{tabular}}
\end{center}
\vspace{-4mm}
\caption{Average and last accuracies (\%) on ImageNet-1000 of FOSTER with and without our method plugged-in, and other three baselines~\cite{rebuffi2017icarl,zhao2020maintaining,yan2021dynamically}. We show two memory budgets, $M\!=\!20k$ (upper block) and $M\!=\!5k$ (lower block), in the LFS setting.}
\vspace{-4mm}
\label{table: sota_imagenet1000}
\end{table}
\myparagraph{Comparing with the State-of-the-art.}
In Table~\ref{table: sota_food101_and_imagenet100}, we summarize the experimental results on two datasets (Food-101 and ImageNet-100) and in two CIL protocols (LFS and LFH). From the table, we have the following observations. 1)~Our CIM-based CIL consistently improves the state-of-the-art method FOSTER~\cite{wang2022foster} with clear margins in all settings. E.g., our method surpasses it by an average of $1.4$ percentage points on ImageNet-100, and $2.0$ percentage points on Food-101. 
%
%
2) Our CIM-based CIL achieves more significant improvements when $N$ becomes larger, e.g., on ImageNet-100 (LFH), our method improves FOSTER by $0.9$ and $3.3$ percentage points when $N$=$5$ and $N$=$25$, respectively. 3) Our CIM-based CIL achieves greater improvements consistently on Food-101 (than ImageNet-100). It improves baselines by $2.1$ percentage points on Food-101, while the improvement is $1.4$ on ImageNet-100 ($N$=$10$, LFS). 
%
It shows that our method is particularly effective when the representative visual cues of a class are from some of its components, e.g., the ``cream'' of the class ``cake''.

Table~\ref{table: sota_imagenet1000} shows the results on the large-scale dataset ImageNet-1000 in different memory settings ($M=20k$ and $M=5k$).
%
We can see that our CIM-based CIL improves FOSTER consistently. It is impressive that it achieves more improvements in the more strict memory setting ($M=5k$). Specifically, it boosts the average accuracy of FOSTER by $4.5$ percentage points when $M=5k$, significantly higher than that of $M=20k$ ($1.0$).

\begin{figure}[t]
\centering
\includegraphics[width=1.0\linewidth]{ablation.pdf}
\caption{Ablation study of our method on the tst-COMMON set of MuST-C EnDe dataset. The observed points in the plots represent wait-$k$ policy with $k=\{1,3,5,7,9,12,15,20,30\}$.}
\label{fig:ablation}
\end{figure}


\myparagraph{Ablation Study.}
Table~\ref{table_ablation_study} shows the ablation results. \emph{First block: baselines.}  Row 1 is for the baseline FOSTER~\cite{wang2022foster}. Row 2 shows the results of adding artifact augmentation (see Section~\ref{subsection: cam_based_compression_pipeline}). It shows directly apply this augmentation does not improve and even impair the model. 
\emph{Please note the models in below blocks all use this augmentation}.
%
\emph{Second block: activation methods.} 
%
Rows 3-6 show the results of using different activation methods to compress exemplars.
%
Row 3 is to downsample all pixels (i.e., no region is activated). Row 4 is to randomly select activation regions. Row 5 is to activate only the center region ($\frac{1}{4}$ of the original image), while row 6 is to use naive CAM.
%
Comparing them to row 1, we can see that using naively compressed exemplars can improve CIL models.
%
Row 4 outperforms rows 3-5, validating that it is more reliable to use the model's activation to generate compressed exemplars.
%
\emph{Third block: optimization methods.} 
Rows 7-9 are on top of Row 6 and are the results of applying different optimization strategies.
%
Row 7 is to manually select $\tau$ using a held-out set ($10\%$ of the dataset). Row 8 is to jointly train CIL and CIM models (for each input batch). Row 9 is the proposed method of using a global BOP.
%
\emph{Fourth block: two variants of CIM-based CIL.}
Rows 10-11 are two variants of row 9. In Row 10, only the activation layers in the last block of CIM are learnable, and previous blocks use ReLU. Compared to row 9, row 10 shows slightly worse performance. Row 11 shows the version of adding a weak downsampling ($\eta^\prime=2.0$) on discriminative regions, based on which more compressed exemplars are saved. It results in comparable performance to row 9 but increases costs.

\begin{table}
\normalsize
\begin{center}
\renewcommand\arraystretch{1}
\setlength{\tabcolsep}{0.7mm}{
\begin{tabular}{lcccccccc}
\toprule
\multirow{2.5}{*}{\textbf{Method}} & \multicolumn{3}{c}{\textit{ImageNet-100}} && \multicolumn{3}{c}{\textit{ImageNet-1000}} \\
\cmidrule{2-4} \cmidrule{6-8}
& $N$=6 & 11 & 26 && 6 & 11 & 26\\
\midrule
LUCIR \emph{\small{baseline}}                                      & 71.22 & 69.67 & 67.45 && 65.23 & 62.43 & 59.88 \\
\textit{w/} Mnemonics            & 73.30 & 72.17 & 71.50 && 66.15 & 63.12 & 63.08 \\
\textit{w/} MRDC                   & 73.62 & 72.81 & 70.44 && 67.67 & 65.60 & 62.74 \\
\textit{w/} ours                                         & \textbf{74.05} & \textbf{73.76} & \textbf{72.84} && \textbf{68.03} & \textbf{66.54} & \textbf{63.77} \\
\bottomrule
\end{tabular}}
\end{center}
\vspace{-5mm}
\caption{Comparing with Mnemonics~\cite{liu2020mnemonics} and MRDC~\cite{wang2022memory}. 
We plug each of them in baseline LUCIR~\cite{hou2019lucir} for fair comparison.
}
\label{table: comparing_with_mnemonics_and_jpeg}
\vspace{-4mm}
\end{table}
\myparagraph{Comparing with Other Compression-based Methods.} Table~\ref{table: comparing_with_mnemonics_and_jpeg} shows our results comparing to two compression-based methods: Mnemonics~\cite{liu2020mnemonics} and MRDC~\cite{wang2022memory}. We can see that our method consistently outperforms them in all settings. 
%
This is because our method does not sacrifice the discriminativeness of exemplars while improving the number (variance) of exemplars in a phase-adaptive manner.
%
However, the two related methods either keep a fixed number of exemplars in the memory~\cite{liu2020mnemonics} or use uniform image compression without considering the properties of specific classes in different incremental phases~\cite{wang2022memory}.

\myparagraph{Visualizations (CAM vs. CIM).} Figure~\ref{fig_visualization} gives two visualization examples, ``Afghan hound'' and ``indigo bird'', each with the activation map as well as the bounding boxes.
%
The first column shows their respective confusing classes appearing in earlier phases.
CIM learns to focus on the discriminative (i.e., dissimilar to confusing classes) regions.
%
\section{Visualization On Demand} %Visualization Elements
\label{sec:visrisk}
Based on environment data and trajectory evaluation, we now present ways of communicating the situation and risks on a visual display to achieve an ADAS.
In this context, we employ a renderer that visualizes all the information in a joint Cartesian coordinate system (see section \ref{subsec:sim}). 
Once driving risks are detected, design elements are overlayed on the display with section \ref{subsec:active} and section \ref{subsec:warning}. 

\subsection{Simulator Environment}
\label{subsec:sim}
Nodes of the R-LDM have a range of potential attributes, such as the 3D position or geometrical shape of objects. 
% For instance, the road centerline is a polyline with bounderies to the left and right. Crosswalks have a defined width and buildings a polygonal outline description. 
In the renderer, we always visualize static and quasi-static data that lie in the field of view from the ego vehicle. 
For this, a local 3D model is generated by converting geographic points with (lat, lon, alt) into Cartesian coordinates of (x, y, z). 
% and project the positonal relations from a view perspective with a transformation matrix. 
Fig. \ref{fig:3Dsimulator} depicts an exemplary map section having several intersections in bird's-eye view.
% with several intersections, stop lines and crosswalks. 
On the top right, the first person view of a vehicle approaching a crosswalk is shown. 

The dynamic data is then added to this static view. A zoomed-in excerpt from the map is given at the bottom of Fig. \ref{fig:3Dsimulator} that includes a recorded GNSS trace (red).
We project the trace onto the connected lane center, which is pictured in green. 
% Because we project the ego position on the closest lane segment, on the bottom right the measured trace is changed in red and the aligned trace is marked in green.
Consequently, the virtual horizon and its possible paths are retrieved as described in section \ref{subsec:ldm}. 
We can lastly update and move the excerpt with the current position from the GNSS to obtain a live simulation.

\subsection{Proactive Support}
\label{subsec:active}
Communication of spatial as well as spatio-temporal relations is crucial for risk-averse driver support. 
% This has the reason that humans can estimate the time better than positions (especially for risks). 
% Velocity contains implicitly the time as well. 
Further sources of information are cause, likelihood and severity of a potential risks.  
% if a collision happens. 
The next step for RNS is the choice of suitable design elements. 
In this process, we suppose that we know where the ego vehicle is driving (i.e., the ego path) from its navigation route. 
Yet, for surrounding vehicles, all paths are considered.

\subsubsection{Hazard Route Element}
The so-called hazard route in Fig. \ref{fig:charts} is a concept that consists of a scale portraying distances to an upcoming risk element.
Furthermore, the geometrical area or length of risks is considered.
Risk is thus measured with respect to the ego path, ranging from the current position  $\Delta l \hspace{-0.03cm}=\hspace{-0.03cm} \unit[0]{m}$ to the end of the path $\Delta l_{h}$.
Here, the length $\Delta l_{h}$ can be chosen according to own preferences. 

At an upcoming intersection, risk is defined by the section of the path that lies within the junction.
Since risk corresponds to exposition time, we encode the path part from the intersection $I_z$ with a color, ranging from green for short intersections to red for long ones. 
%allgemein risiko entlang des pfades zu intersection zone
%share of junction segment to navigation route + 
%one case with large intersection far and one case with small intersection close
Fig. \ref{fig:charts}~a) gives two examples of the hazard route.
The left bar shows a large intersection (e.g. multi-lane four-way stop) in vicinity and the right bar has a small and consecutive medium junction. 
% In the case of collision risk, the intersection zone $I_z$ can be used.
% Depending on the value of $I_z$ (low, medium and large), the area is marked from green, to yellow until red for conveying the criticality. 
This emphasizes that we may include more than one intersection in our warnings.

\begin{figure}[t]
  \centering
  \includegraphics[width=0.95\linewidth]{./img/simulator.png}
  \caption{Rendered road network from two perspectives with the ego position being projected on the navigation route. \vspace{0.45cm}}
  \label{fig:3Dsimulator}
\end{figure}

\begin{figure}[t]
  \centering
  \resizebox{\linewidth}{!}{
  \import{img/}{velocity_scale_new.pdf_tex}}  
  \caption{Chart elements for proactive support. Hazard route (left) and velocity scale (right).} %\vspace{-0.3cm}}
  \label{fig:charts} 
\end{figure} 

\subsubsection{Velocity Scale Element}
The velocity scale, Fig. \ref{fig:charts}~b), is a second chart element which qualifies the difference between the current velocity of the vehicle $v_0$ and the target velocity $v_{\text{tar}}$ from the trajectory evaluation of section \ref{subsec:trajeval}. 
The scale shows possible velocity values, from standstill $v\hspace{-0.05cm}=\hspace{-0.05cm}\unit[0]{m/s}$ to a maximal velocity $v_{\text{max}}$. Depending on the difference $|v_0 \hspace{0.05cm} - \hspace{0.05cm} v_{\text{tar}}|$, the situation is rated as safe with $v_0 \hspace{-0.042cm} \approx \hspace{-0.042cm} v_{\text{tar}}$ (green, left), as dangerous with e.g. $v_0 \hspace{-0.05cm} < \hspace{-0.05cm} v_{\text{tar}}$ (yellow, middle) to critical with $v_0 \hspace{-0.07cm} \ll \hspace{-0.07cm} v_{\text{tar}}$ (red, right). The same cases hold true for the opposite circumstances, i.e., $v_0 \hspace{-0.032cm} > \hspace{-0.032cm} v_{\text{tar}}$. 
This velocity scale can be employed for curve or regulatory risks. 
Moreover, we may set an enforced speed limit as the target velocity $v_{\text{tar}}$ for proactive behavior, once there is no risk ahead. 
%\noindent -Warning vs behavior support \\
%-Ghost vehicle as in game \\

\subsection{Short-Term Warning Elements}
\label{subsec:warning}
In order to emphasize the criticality of the situation, we propose to add further intuitive warning elements as e.g. pop-up signs and lane colorings. 
The following elements augment the proactive elements.

\subsubsection{Pop-up Signs}
Explicit symbols indicate the risk cause accompanied with the event time for collisions ($s_E$), distances to the risk spot for turns (i.e., right curve with $d_r$ and left curve with $d_l$) or stopping distance for crosswalks ($d_c$). In Fig. \ref{fig:popups}~a), the pop-up signs are pictured. 
% Besides the velocity difference, the risk type is an indication for the severity of the situation.
%Examples for collision risk are car-to-car crash., curve risk can be  as a single-car accident and regulatory risks will be a car-to-object collision. 
We want to stress that this is just a selection and more risk causes can be added. 
The purpose is also to clarify the reason for the warning and give more human-understandable information.

\subsubsection{Colored Events}
Finally, we highlight lane parts or positions according to the corresponding risks.  
% the determined color rating from the hazard route and velocity scale and relate the risks to the simulator environment. 
In the instance of curve and regulatory risk, the lane is colored from the ego position up to the point of maximal risk. 
For collision risk, we mark the point of the closest encounter as a red cube.
An illustration for regulatory risk induced from a stop line is depicted in Fig. \ref{fig:popups}~b). Again, the color is defined by the deviation $|v_0-v_{\text{tar}}|$. It also shows the therein considered navigation route with length $\Delta l_h$ and another unlikely path. 

It should be noted that the visualization of warnings only occurs if the risks are actually present. 
%\textcolor{red}{improve language, repeat intersection zone and navigation route}
%eingrauen unlikely paths and navigation path and describe in text, maybe delete Iz -> put line from unlikely path to green arrow
Altogether, the RNS provides a variety of tools to analyze and circumvent critical situations in intersection scenarios, while not overloading the driver's awareness.

\begin{figure}[t]
  \centering
  \resizebox{\linewidth}{!}{
  \import{img/}{colored_lane_new.pdf_tex}}  
  \vspace{-0.53cm}
  \caption{Short-term warning elements. Selected pop-up warnings (left) and colored lane (right).}
  \label{fig:popups} 
\end{figure} 




\begin{table}
    \centering
    \vspace{-0.2cm}
    \setlength{\tabcolsep}{2.75mm}{
    \begin{tabular}{lccc}
    \toprule
    \textbf{Metric} & Small & Middle & Large \\
    \midrule
    Mean of \#Exemplars     & 39.40 & 38.30 & 34.77 \\
    \cdashline{1-4}[4pt/2pt]
    Last Acc. (\%, baseline)       & 66.13 & 68.40 & 69.93 \\
    Last Acc. (\%, ours)       & 70.00 & 71.10 & 72.26 \\
    Improvement (\%)         & +3.87 & +3.65 & +2.33 \\
    \bottomrule
    \end{tabular}
    }
    \vspace{-0.2cm}
    \caption{Results for small, middle and large objects in the setting of $N$=$10$ (LFS) on ImageNet-100.
    ``Mean of \#Exemplars'' denotes the average number of saved exemplars by our method. 
    The baseline (FOSTER~\cite{wang2022foster}) has this number as $20$ for all classes.}
    \label{table_hierarchical_compression_strategy}
    \vspace{-0.4cm}
\end{table}
\myparagraph{Results of Different-Size Objects.}
Table~\ref{table_hierarchical_compression_strategy} shows the results for small, middle, and large objects. These size categorization is according to ImageNet Object Localization Challenge~\cite{imagenetobjectlocalizationchallenge}. We calculate the bbox coverage for each class and take top $30$ classes with highest coverages as ``large'', rear $30$ classes with lowest coverages as ``small'' and the rest $40$ classes as ``middle''.
%
It is intriguing that our method achieves the highest improvement (over baseline) for small objects.
%
Our explanation is that small objects benefit more from image compression (than large ones), as their images contain more background pixels to downsample.
%