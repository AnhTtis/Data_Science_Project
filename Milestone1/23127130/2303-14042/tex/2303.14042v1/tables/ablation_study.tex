

\begin{table}
\normalsize
\begin{center}
\renewcommand\arraystretch{1.0}
\setlength{\tabcolsep}{1.5mm}{
\begin{tabular}{lcccccc}
\toprule
\multirow{2.5}{*}{\textbf{Ablation Method}} & \multicolumn{2}{c}{\textit{Food-101}} && \multicolumn{2}{c}{\textit{ImageNet-100}} \\
\cmidrule{2-3} \cmidrule{5-6}
& $N$=10 & 20 && 10 & 20 \\
\midrule
1 Baseline                             & 72.72 & 66.73 && 76.55 & 72.37 \\
2 Artifact Aug.                                   & 71.38 & 66.03 && 75.63 & 71.45 \\
\cdashline{1-6}[4pt/2pt]
3 Full Comp.                                   & 73.03 & 67.38 && 76.92 & 73.26 \\
4 Random Acti.                                    & 73.10 & 67.54 && 76.88 & 73.54 \\
5 Center Acti.                                  & 73.29 & 67.88 && 76.78 & 73.82 \\
6 Class Acti.                                     & 73.76 & 68.65 && 77.21 & 74.67 \\
\cdashline{1-6}[4pt/2pt]
7 Phase-wise $\tau$                             & 73.83 & 69.17 && 77.06 & 74.78 \\
8 Joint Train                                         & 73.44 & 69.01 && 77.34 & 74.59 \\
9 BOP (ours)                                      & 74.85 & \textbf{70.20} && \textbf{77.94} & 75.23 \\
\cdashline{1-6}[4pt/2pt]
10 LastBlock Only                                   & 74.55 & 69.87 && 77.72 & 74.86 \\
11 Fg Compressed                                     & \textbf{75.02} & 70.13 && 77.87 & \textbf{75.46} \\
\bottomrule
\end{tabular}}
\end{center}
\vspace{-5mm}
\caption{Average accuracies (\%) of different ablation methods. The experiments are conducted in the LFS setting.}
\label{table_ablation_study}
\vspace{-4mm}
\end{table}