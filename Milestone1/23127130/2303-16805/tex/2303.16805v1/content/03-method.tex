\section{Concept}
\label{sec:concept}

Within the scope of our experiment, three variants were developed to map vibrotactile 3D directional cues. For the 2D direction, the vibrotactile illusions of the \emph{Cutaneous Rabbit} and \emph{Apparent Tactile Motion} were used. We extended these by a pulse- and intensity-based approach to communicating the gradient of the 3D directional cue (see Figure~\ref{fig:variants}).


\subsection{\conA: Cutaneous Rabbit with Pulse-based Approach}
\label{sec:var-a}
This condition is based on the \emph{Cutaneous Rabbit} for communicating 2D direction, a tactical illusion that can influence the design of vibrotactile patterns. This illusion was discovered in 1972 by \citeauthor{Geldard.1977}~\cite{Geldard.1972}. 
The sequence of taps on different vibrotactile actuators is perceived as a continuous movement between the different points. 
Each directional cue is abstracted using three control points \change{for the} actuators  
\change{(illustrated as dashed lines in \autoref{fig:variants})}. 
Depending on the distance resulting from the gradient of the directional cue, the number of pulses triggered at each actuator is determined in a range of 1 -- 7 with a \ac{BD} of 125ms, an \ac{ISI} between pulses of 50ms, and an \ac{IBI} between actuators of 100ms. 
The closer the control point of the direction cue is to the hand, the higher the number of vibration pulses (see Figure~\ref{fig:rabbitSingle}).


\subsection{\conB: Cutaneous Rabbit with a Pulse- and Intensity-based Combined Approach}
\label{sec:var-b}
\conB is based on \conA but includes a second additional encoding for the gradient of the 3D directional cue. In addition to the number of pulses, we mapped three different intensity levels on the distance of the directional cue to the palm (see Figure~\ref{fig:rabbitDual}). We based the distinct intensity levels on prior work by \citeauthor{Gescheider1990VibrotactileID}, who measured a just noticeable relative difference threshold -- \ac{JND} -- with values of 0.26 at 4 dB above the perceptual threshold~\cite{Gescheider1990VibrotactileID}. To communicate and distinguish between up- and downward gradients, three distinct intensity levels were selected - a baseline level in the middle and one low- as well as one high-intensity level. The anticipated benefit of this condition was that gradient comprehension would be improved due to the dual encoding.


\begin{figure*}
\centering
\captionsetup{justification=centering}
\hfill
    \subfloat[Actuator Placement \change{as provided by \emph{SensorialXR}}]{\includegraphics[width=0.17\linewidth]{figures/Tactors.png}\label{fig:tactors}}
    \hfill
    \subfloat[2D Directions]{\includegraphics[width=0.28\linewidth]{figures/Richtungshinweise_once_compact.png}\label{fig:directions}}
    \hfill
    \subfloat[Study Setup]{\includegraphics[width=0.32\linewidth]{figures/Setup.png}\label{fig:setup}}
    \hfill
\captionsetup{justification=justified}
  \caption{For the 2D directional cues, we used (\textbf{a}) the placement of all actuators across the hand to (\textbf{b}) communicate five different directions. Here, (\textbf{c}) illustrates the study setup, with the arm resting on the armrest while the hand is in the air.}
  \Description{This figures illustrate the actuator placement, 2D directions, and study setup (three figures in a row). Figure 2a: A sketch of a hand with the placement of ten actuators -- at each finger and thumb tip (five; V1 -- V5), between index and middle finger, middle and ring finger, and ring and pinky finger (three; V6 -- V8), and at the left and right bottom of the palm (two; V9 and V10). Figure 2b: Five sketches of the 2D directions with the chronologically activation of three actuators T1 -- T3: left to right (V6 -> V7 -> V8), right to left (V8 -> V7 -> V6), diagonal rear-right to front-left (V10 -> V7 -> V2), straight forward (V9 -> V6 -> V2), and diagonal rear-left to front-right (V9 -> V7 -> V5). Figure 2c: Study setup: Participant is sitting on a chair in front of a table, wearing the glove on the right hand, resting the arm on the chair's armrest and facing the palm downward.}
  \label{fig:hand-setting}
\end{figure*}

\subsection{\conC: Apparent Tactile Motion with Intensity-based Approach}
\label{sec:var-c}
This condition applied the same intensity mapping for the gradient as \conB, but without the pulses. In contrast to the \emph{Cutaneous Rabbit} sensation with distinct pulses as in \conA and \conB, here we applied the vibrotactile illusion of \emph{Apparent Tactile Motion}. This was first studied in the early 20th century by \citeauthor{Burtt.1917}~\cite{Burtt.1917} and is commonly referred to as the \emph{Phi Phenomenon}. The illusion is created by an overlap in the start times of two actuators -- \ac{SOA}, calculated as $SOA = 0.32d + 47.3ms$, where $d$ is the vibration period of an actuator -- 450ms. Instead of two individual actuators, a single stimulus is perceived as moving from the position of the first triggered actuator T1 to the second actuator T2 -- or from actuator T2 to T3 (see Figure~\ref{fig:motionIntensity}). A potential benefit of this illusion is that it may feel more like a natural movement, as it disguises the limited number of actuators. 
\change{After pilot tests, a starting intensity value of 0.22 and a JND value of 0.3 were chosen, which made the intensity levels easily distinguishable. Thus, a total of seven possible intensity levels were defined.}


\subsection{Implementation}
\label{sec:implementation}

To develop our approach, we use the 3D game engine \emph{Unreal Engine 4} optimized for usage with a \emph{Meta Quest~2} \ac{VR} \ac{HMD}. This allows for the use of a virtual environment in which the participants can concentrate purely on the haptic feedback without being visually distracted. It also provides a simple way to visually explain the directional cues to the participants and record their responses for rating scales. As a haptic display, we chose the \emph{SensorialXR} glove as a commercially available device with \ac{SDK} interface to the \emph{Unreal Engine 4}. With ten actuators -- \ac{LRA} vibration motors\change{, fixed in place} -- (see Figure~\ref{fig:tactors}), \emph{SensorialXR} gloves are among the models with the most vibration motors per hand. Thus, they offer the potential to map the 3D directional cues with the highest possible vibrotactile resolution~\cite{Rakkolainen2021}. 






