\section{Discussion}
\label{sec:discussion}
Overall, responses during the interview and the quantitative data are in agreement. They show significant and substantial advantages of \conA and \conB compared to \conC, which we did not expect in such clarity. The parity between these two then is again visible from all angles, with preference being nearly balanced (8 vs. 6). Still, the interviews showed that for \conC, participants did like the smooth transition between the individual feedback factors, which however failed to have a measurable advantage (RQ2). A main reason for this may be that we found that the mechanisms to communicate 2D direction and gradient can interfere with one another. In particular, the intensity gradient mapping had a negative effect on the 2D direction mapping in \conC, as the minimum vibration sometimes \enquote{got lost} (P5), when users did not pay close attention. While pre-tests suggested otherwise, individual differences among the perception of our participants as well as potential fitting issues with the glove (see \emph{limitations} below) may have resulted in this issue. From the comments of participants, we have to assume that \conB was affected by this problem as well, although to a lesser degree; the simultaneous pulse mapping implicitly included repeated vibrations of the same actuator at least twice.

Our analysis also showed that the type of feedback may require more training for participants to get accustomed to. P13 summarized this point nicely: \enquote{Maybe if you market that [...] and someone develops a game for it [...], then I might like it, and in a year, no one can imagine a world without it.} Others noted the effort involved, with P10 saying they \enquote{found that it was really exhausting since you are not used to it.} 
 Still, the overall experience of using vibrotactile feedback to interpret 3D directional cues was described as \enquote{surprisingly good} (P7) and prompted many ideas for use cases, such as medical scenarios (operating table with limited visuals), people with visual impairments in daily life as well as when driving a bike or motorcycle.

\textbf{Limitations:} As an exploratory study, our results should be perceived as preliminary and require further testing and confirmation. In particular, our results may be limited due to the number of participants (14), which also led to the design not being fully balanced. \change{In addition, all participants in our study were right-handed, which could affect our results.} We also found that more training may be required to compensate for initial learning effects, as the type of feedback is so unusual and novel for participants. In addition, the \emph{SensorialXR} glove only \change{provides a fixed setting of the actuators and} offers a \enquote{one-size-fits-all} size, which showed to be problematic for some users with smaller hands, where the actuators were not always in tight contact with the skin. \change{For future research prototypes adding additional Velcro around the actuators might help.} 