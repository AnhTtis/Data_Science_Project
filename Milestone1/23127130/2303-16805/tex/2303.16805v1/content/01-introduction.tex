\section{Introduction}
\label{sec:Introduction}

People perceive objects in their environment primarily through their sense of sight. \change{However, this ability can be reduced or not possible at all in certain situations}. \change{Objects might be covered by other things / \ac{UI} elements (visual clutter) or be outside the human field of view}.
In addition, visual perception may be limited or \change{impossible} due to visual impairments.
Previous research has shown that the haptic modality can, to some extent, compensate for the lack of visual information and outperform audio-based cues~\cite{metaanalysis}. It can also be applied in combination with \change{other modalities} and can offer an additional information channel if, for example, the visual channel is overloaded due to distracting information~\cite{Chen2018, Kaul2016HapticHead3G}. 

Directing attention, guiding, and transmitting patterns via vibrotactile signals have already been researched and found to be useful feedback modalities~\cite{Weber.2011,Gunther.2018,Grushko.2021}. \citeauthor{Barralon.2009} studied pattern recognition using a vibrotactile belt with eight actuators and tasked participants to select the corresponding correct visual representation~\cite{Barralon.2009}. \citeauthor{Lee_Starner_2010} proposed \emph{BuzzWear}, a wearable tactile display with three vibration actuators for notification purposes that function by modulating intensity, pattern, direction, and starting point~\cite{Lee_Starner_2010}. After 40 minutes of training, subjects could distinguish between the 24 patterns with up to 99\% accuracy. Vibrotactile feedback is also used in the context of guidance. Here, a study by \citeauthor{Lehtinen2012}, used a vibrotactile glove to support a visual search task on a flat plane on a wall~\cite{Lehtinen2012}.

However, a common challenge is that tactile displays have a limited resolution. Therefore, researchers have simulated smooth movement patterns with the help of tactile illusions~\cite{Cholewiak.2000}, such as \emph{Phantom Sensations}~\cite{4081935,10.1145/3173574.3173832}, \emph{Apparent Tactile Motion}~\cite{Burtt.1917,Kirman.1974,Sherrick.1966}, and \emph{Cutaneous Rabbit}~\cite{Geldard.1977,McDaniel2011,Raisamo2009,10.1145/1518701.1519044}. \citeauthor{Tan2003} conducted a study using a 3 x 3 tactile display and applied the \emph{Cutaneous Rabbit} sensation to explore the communication of eight 2D directional cues (north, northeast, east, southeast, south, southwest, west, and northwest) and the successful recognition of these cues~\cite{Tan2003}.

While previous work focused on 2D directional cues (e.g., \cite{Tan2003}) or allowed users to feel directions upon approach with their hand (e.g., \cite{Grushko.2021}), we are not aware of any work that aims to communicate 3D directional cues. \change{In particular, our work differs from approaches such as~\cite{Gunther.2018}, who aim to push or pull the hand toward a known target in 3D space but who therefore do not actually need to encode 3D information for the vibration pattern itself. It also differs from work such as~\cite{Wu.2013} which used a \ac{TVSS} to communicate 3D shapes of a static object by directly mapping image features such as contours on a 20 x 20 tactile display.} 

Our approach builds on the idea of \citeauthor{Tan2003} \cite{Tan2003} to communicate 2D directions. We combine their base with pulse or intensity mapping to simultaneously communicate the gradient. Furthermore, we explore the influences of different haptic illusions (i.e., \emph{Cutaneous Rabbit} and \emph{Apparent Tactile Motion}) on the comprehension of directional cues. Our work contributes three specific design proposals for communicating 3D directional cues as well as a study on the effectiveness and subjective experience of this non-visual approach to direction mapping.