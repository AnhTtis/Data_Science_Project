In Human-Computer-Interaction, vibrotactile haptic feedback offers the advantage of being independent of \change{any} visual perception of the environment. \change{Most importantly}, the user's field of view is not obscured by user interface elements, and the visual sense is not unnecessarily strained. This is especially advantageous when the visual channel is already busy, or the visual sense is limited. We developed three design variants based on different vibrotactile illusions to communicate 3D directional cues. In particular, we explored two variants based on the vibrotactile illusion of the cutaneous rabbit and one based on apparent vibrotactile  motion. To communicate gradient information, we combined these with pulse-based and intensity-based mapping. A subsequent study showed that the pulse-based variants based on the vibrotactile illusion of the cutaneous rabbit are suitable for communicating both directional and gradient characteristics. The results further show that a representation of 3D directions via vibrations can be effective and \change{beneficial}.