\section{Results}
\label{sec:results}

For our applied inferential statistics, we distinguished between ratio and ordinal data. 
The estimation percentages for 2D direction and gradient are ratio data, while the Likert items -- including task load -- are ordinal data.
For ratio data only, we first applied a Shapiro-Wilk test to check for normality.
We found that none of our ratio data is normally distributed.
Thus, we treated all our data in the same way and directly applied non-parametric tests, specifically Friedman tests.
Thereafter, we conducted Wilcoxon Signed-rank tests with Bonferroni correction for our post-hoc analysis.
The effect sizes of the Wilcoxon tests are reported as r (r: $>$0.1 small, $>$0.3 medium, and $>$0.5 large effect).

\subsection{Estimation of 2D Direction}
We asked participants to estimate the two-dimensional direction on a ground plane.
The median (interquartile range) percentages of correct 2D direction estimations for each condition are (in descending order): \conB=93.3\% (IQR=12.5\%), \conA=91.7\% (IQR=13.3\%), and \conC=78.3\% (IQR=16.7\%). 
All percentages are compared in Figure~\ref{fig:boxplots:direction}.
Since our data is not normally distributed (p$<$0.01), we directly ran a Friedman test that revealed a significant effect of condition on 2D direction estimation ($\chi^2$(2)=17.70, p$<$0.001, N=14). 
Post-hoc tests showed significant differences between \conA~and \conC~(W=83, Z=2.62, p$=$0.018, r=0.50) as well as \conB~and \conC~(W=0, Z=-3.30, p$<$0.001, r=0.62).
However, we did not find a significant difference between \conA~and \conB~(W=15, Z=-1.88, p$=$0.182).
Here, \textbf{we can conclude that both \conA and \conB~result in better estimation performance for 2D direction than \conC.}

\begin{table*}
\caption{Pairwise comparisons for individual statements, Bonferroni-adjusted, p-values: $<$0.05 (*), $<$0.01 (**), and $<$0.001 (***).}
\label{tab:pairwise_statements}
\begin{tabular}{l|ll|lll|lll}
 & \multicolumn{2}{l|}{Rabbit Single vs. Dual} & \multicolumn{3}{l|}{Rabbit Single vs. Motion Intensity} & \multicolumn{3}{l}{Rabbit Dual vs. Motion Intensity} \\
Statement & test statistic & p-value & test statistic & p-value & effect size & test statistic & p-value & effect size \\ \hline
S1 & Z=~0.00 & p=1.000 & Z=2.89 & p\textless{}.001\textbf{***} & r=0.55 & Z=3.04 & p=.004\textbf{**} & r=0.57 \\
S2 & Z=-0.07 & p=1.000 & Z=2.58 & p=.029\textbf{*} & r=0.49 & Z=2.80 & p=.013\textbf{*} & r=0.53 \\
S3 & Z=-1.17 & p=0.838 & Z=3.06 & p=.003\textbf{**} & r=0.58 & Z=2.97 & p=.003\textbf{**} & r=0.56 \\
S4 & Z=-0.17 & p=1.000 & Z=2.84 & p=.006\textbf{**} & r=0.54 & Z=2.69 & p=.018\textbf{*} & r=0.51
\end{tabular}
\end{table*}



\subsection{Estimation of Gradient}
We asked participants to estimate the gradient behavior of the communicated cue.
The median (interquartile range) percentages of correct gradient estimations for each condition are (in descending order): \conB=93.3\% (IQR=5.8\%), \conA=91.7\% (IQR=8.3\%), and \conC=56.7\% (IQR=10.0\%). 
All percentages are compared in Figure~\ref{fig:boxplots:gradient}.
Since our data is not normally distributed (p$<$0.001), we ran a Friedman test that revealed a significant effect of condition on gradient estimation ($\chi^2$(2)=19.00, p$<$0.001, N=14). 
Post-hoc tests showed significant differences between \conA~and \conC~(W=102, Z=3.11, p$=$0.002, r=0.59) as well as \conB~and \conC~(W=0, Z=-3.30, p$<$0.001, r=0.62).
However, we did not find a significant difference between \conA~and \conB~(W=30, Z=-1.42, p$=$0.501).
Here, \textbf{we can conclude that both \conA and \conB~result in better gradient estimation performance than \conC.}


\begin{figure*}
    \centering
    \includegraphics[width=\linewidth]{figures/plot_likert.pdf}
    \captionsetup{justification=justified}
    \vspace{-1em}
    \caption{Participant responses to the four rated statements (Likert items ranging from 1: strongly disagree to 7: strongly agree).}
    \Description{The results (four figures in two rows with two figures per row) of the participant' statements "Directions Easy to Distinguish", "Directions Easy to Derive From Vibration", "Change of Gradient Easy to Perceive", and "Easier to Perceive Over Time", visualized in a stacked bar plot for Likert items.}
    \label{fig:likert}
\end{figure*}

\subsection{Task Load}
The results of task load ratings as measured by the \ac{RTLX}~\cite{hart1988} are shown in Figure~\ref{fig:boxplots:taskload}. 
The median (interquartile range) task load scores for each condition are (in ascending order): \conA=22.5 (IQR=12.7), \conB=24.5 (IQR=7.9), and \conC=28.3 (IQR=20.0).
We ran a Friedman test that revealed a significant effect of condition on task load ($\chi^2$(2)=13.50, p$=$0.001, N=14).
Post-hoc tests showed a significant difference between \conB~and \conC~(W=105, Z=3.30, p$<$0.001, r=0.62).
However, we did not find any significant differences between \conA~and \conB~(W=39, Z=0.00, p$=$1.000) or between \conA~and \conC~(W=20, Z=-2.04, p=0.120).
Here, \textbf{we can conclude that \conB~induces a lower task load than \conC.}

\subsection{Individual Statements and Preferences}
After each condition, we asked participants to rate four statements, each on a 7-point Likert scale (1=strongly disagree, 7=strongly agree). The results and statements are shown in \autoref{fig:likert}. We found significant main effects for all four statements (N=14; S1: $\chi^2$(2)=14.09, p$<$0.001; S2: $\chi^2$(2)=11.35, p$=$0.003; S3: $\chi^2$(2)=17.08, p$<$0.001; S4:$\chi^2$(2)=12.79, p$=$0.002). Pairwise comparisons are shown in \autoref{tab:pairwise_statements}. Here, \textbf{we can conclude that \conA and \conB~are rated significantly more positively than \conC~for all four statements}. No difference was found between \conA and \conB. Regarding \emph{overall preference}, \textbf{eight participants preferred \conB}, while \textbf{six voted for \conA}~as their favorite. None of the participants preferred \conC.


\subsection{Interviews}
During the interviews, participants were explicitly asked to comment on the duration of the vibration as well as what may have eased or hindered their comprehension. They also had to explain their overall preference and comment on the overall experience and sensation of interpreting 3D directional cues via vibrotactile feedback.
For the analysis, the verbal data was first transcribed by one author and then summarized. The statements were then counted for each question. In addition, across all questions, we applied open coding to identify hidden themes. Data from one interview (P2) was not recorded due to a technical issue. Therefore, only the data from 13 participants was included.

Regarding the duration of the vibration, both \emph{Rabbit} conditions were perceived as having adequate duration (\conA: 10 vs 3 who thought it could have been longer; \conB: 13:0), while 10 participants would have preferred a longer duration for \conC. For the latter, participants struggled to feel the gradient correctly, as mentioned by five participants (e.g., P7 said that the \enquote{[duration was] a little bit short, enough for [2D] direction, but for intensity [gradient] it was really bad.}) The varying strength of the vibration was also an issue, as the most distant control point was criticized as having a too weak vibration, which meant that \enquote{some vibrations got lost} (P5). This also interfered with the comprehension of 2D direction. The smooth transition of movement in \conC was still found to be a pleasant experience, but the mentioned drawbacks regarding the gradient detection prevailed, according to P4 (RQ2). When comparing pulse with intensity for the mapping of gradient, P12 noted an interesting further advantage of pulse, as \enquote{One could decide about the gradient in retrospect even if one wasn't sure before. When the last actuator vibrated many times, then it must have been an upwards gradient.} This also implicitly highlights the problem of immediacy, which required attention and did not allow repetition of the feedback. As P10 put it, \enquote{in case you did not fully pay attention, there wasn't a repeat to make sure.} This sentiment was echoed by P12. 
Consequently, the dual encoding of a gradient in the \conB~condition was cited by most as the main reason for preferring that condition (RQ1). P11 noted, \enquote{I did not just have the number of pulses, but in addition the intensity and that somehow better stuck in my head.}