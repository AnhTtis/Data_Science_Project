\section{Study}
\label{sec:study}
We conducted a within-subjects experiment with 14 participants to explore and understand the differences and similarities between the three presented designs for vibrotactile feedback (independent variable) regarding their effectiveness in communicating 3D directional cues. As participants were supposed to feel and comprehend directional cues without any additional visual feedback, we conducted the study in person and within a neutral VR environment, which allowed participants to focus entirely on the vibrotactile feedback. The age of participants ranged from 21 to 31 years, with a mean age of 25.71 years ($M = 25.71, SD = 2.972$). Four were female, ten were male, and all were university  students of various subjects. None of the participants reported any visual impairment, and all were right-handed.

\subsection{Procedure}
The study was conducted in multiple comparable physical localities. Before commencing, participants were fully informed about the project objective and the various tasks they had to complete. 
Each participant gave their full and informed consent to partake in the study, have video and audio recordings taken, and have all the relevant data documented. 
Participants wore a \ac{HMD} on their head and a vibrotactile glove on the right hand while \change{being asked to keep} their right arm rested on an armrest with the palm facing down (see Figure~\ref{fig:setup}) \change{to avoid any external factors}. 
In the left hand, participants held a controller to control the \ac{VR} environment. 

For each condition, each participant performed six training trials. For each trial, the vibrotactile feedback was repeated three times, and a corresponding visualization was shown to indicate the direction in 3D \change{supporting the participant's mental model}. 
For the actual task, \change{participants were shown a neutral-colored background in VR without any visual representations of the 3D direction. P}articipants were able to trigger the start of the trial with the \ac{VR} motion controller. In total, they completed 30 measured trials per condition, resulting in 90 measured trials per participant and 1,260 measured trials in total. The 30 trials consisted of 2 (blocks) x 5 (2D direction) x 3 (gradient). The variable \emph{2D direction} represented a typical set of five possible mappings of straight horizontal, vertical and diagonal directions, which were physically located on the surface of the hand (see Figure~\ref{fig:directions}). They represented the direction in x-z-coordinates of the overall 3D directional vector. The \change{\emph{gradient}} encoded the direction in y-coordinates: either up, down, or neither any gradient. To counter learning and fatigue effects, we applied a \emph{Balanced Latin Square} design for the order of the three conditions. The order of trials was randomized within each block. \change{Between each condition, participants were able to rest their hand for five minutes.} The average session lasted for 45 minutes and concluded with a debriefing. \change{Non of the participants mentioned any sensory or muscle fatigue.} Participants received 15 EUR in compensation.

\begin{figure*}
    \centering
    \captionsetup{justification=centering}
        \subfloat[2D Direction Estimation.]{\includegraphics[width=0.32\linewidth]{figures/plot_estimation_2ddirection.pdf}
        \label{fig:boxplots:direction}}
            \hfill
        \subfloat[Gradient Estimation.]{\includegraphics[width=0.32\linewidth]{figures/plot_estimation_gradient.pdf}
        \label{fig:boxplots:gradient}}
            \hfill
        \subfloat[Task Load.]{\includegraphics[width=0.32\linewidth]{figures/plot_taskload.pdf}
        \label{fig:boxplots:taskload}}
            \hfill
    \captionsetup{justification=justified}
    \vspace{-0.5em}
    \caption{Measured performance for 2D direction and gradient estimation as well as task load measured with the NASA Raw-TLX (lower score is better). For the task load subscale \enquote{frustration} no bars are visible because all three conditions have a median score of 0.}
    \Description{Three figures in a row, illustrating the results of "Comparison of 2D Direction Estimation" (boxplot), "Comparison of Gradient Estimation" (boxplot), and "Comparison of NASA Raw-TLX" (bar graph) for each condition. Figure 3a: Comparison of 2D Direction Estimation (boxplot): Rabbit Dual=93.3\% (IQR=12.5\%), Rabbit Single=91.7\% (IQR=13.3\%), and Motion Intensity=78.3\% (IQR=16.7\%). Figure 3b: Comparison of Gradient Estimation (boxplot): Rabbit Dual=93.3\% (IQR=5.8\%), Rabbit Single=91.7\% (IQR=8.3\%), and Motion Intensity=56.7\% (IQR=10.0\%). Figure 3c: Comparison of NASA Raw-TLX for the sub-scores Mental, Physical, Temporal, Effort, Performance, and Frustration (bar graph).}
    \label{fig:boxplots}
\end{figure*}

\subsection{Variables and Research Questions}
For dependent variables, we measured the accuracy of the comprehension of the \emph{2D direction} (x-axis, z-axis) and the \emph{gradient} (y-axis). \change{We are measuring the two variables (\emph{2D direction} and \emph{gradient}) separately, as commonly done within the research community (e.g., estimation of direction and distance for \ac{HMD}s~\cite{Gruenefeld2017}). The main reasoning here is that orientation in 3D space and especially describing directions in 3D can be challenging for participants and could negatively affect the validity of the measurements.} To do so, we presented participants with a \ac{UI} panel in VR after each trial. The panel showed five pictures with all 2D directions in a top-down view and, subsequently, three pictures of all gradients in a lateral view. Participants used the \ac{VR} controller to select the fitting representation for each. These two variables were measured with a binary outcome (correct, incorrect) and summarized as percentages of correctly identified outcomes across all trials per person and condition (ratio scale). In addition, we measured mental workload after each condition via the \ac{RTLX}~\cite{hart1988} and additional Likert-scale statements regarding the comprehensibility of the directional cue. We also collected qualitative feedback in a semi-structured interview after each condition as well as at the end of the study, which is when we also asked participants to rank the three conditions. 

As the study is exploratory in nature, we were interested in finding out more about the specific features of our three feedback conditions. In particular, we were interested in the following research questions:

\textbf{RQ1}: Do multiple encodings of gradient, as in the condition \conB, improve the comprehension of gradient information and reduce mental workload? 

\textbf{RQ2}: Does apparent movement, as in the condition \conC, improve comprehension of 2D direction\change{? We assume that to be the case} as the transition \change{by overlapping} of vibration between the actuators \change{may be easier to comprehend and interpret as a path} compared to the sensation of \change{isolated pulses as for the \emph{Rabbit} conditions.} 

\textbf{RQ3}: How would participants experience and rate vibrotactile communication of 3D direction overall and with regard to each individual condition?


