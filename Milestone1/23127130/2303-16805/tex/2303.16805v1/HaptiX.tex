%%
%% This is file `sample-manuscript.tex',
%% generated with the docstrip utility.
%%
%% The original source files were:
%%
%% samples.dtx  (with options: `manuscript')
%% 
%% IMPORTANT NOTICE:
%% 
%% For the copyright see the source file.
%% 
%% Any modified versions of this file must be renamed
%% with new filenames distinct from sample-manuscript.tex.
%% 
%% For distribution of the original source see the terms
%% for copying and modification in the file samples.dtx.
%% 
%% This generated file may be distributed as long as the
%% original source files, as listed above, are part of the
%% same distribution. (The sources need not necessarily be
%% in the same archive or directory.)
%%
%% Commands for TeXCount
%TC:macro \cite [option:text,text]
%TC:macro \citep [option:text,text]
%TC:macro \citet [option:text,text]
%TC:envir table 0 1
%TC:envir table* 0 1
%TC:envir tabular [ignore] word
%TC:envir displaymath 0 word
%TC:envir math 0 word
%TC:envir comment 0 0
%%
%%
%% The first command in your LaTeX source must be the \documentclass command.
%%%% Small single column format, used for CIE, CSUR, DTRAP, JACM, JDIQ, JEA, JERIC, JETC, PACMCGIT, TAAS, TACCESS, TACO, TALG, TALLIP (formerly TALIP), TCPS, TDSCI, TEAC, TECS, TELO, THRI, TIIS, TIOT, TISSEC, TIST, TKDD, TMIS, TOCE, TOCHI, TOCL, TOCS, TOCT, TODAES, TODS, TOIS, TOIT, TOMACS, TOMM (formerly TOMCCAP), TOMPECS, TOMS, TOPC, TOPLAS, TOPS, TOS, TOSEM, TOSN, TQC, TRETS, TSAS, TSC, TSLP, TWEB.
% \documentclass[acmsmall]{acmart}

%%%% Large single column format, used for IMWUT, JOCCH, PACMPL, POMACS, TAP, PACMHCI
% \documentclass[acmlarge,screen]{acmart}

%%%% Large double column format, used for TOG
% \documentclass[acmtog, authorversion]{acmart}

%%%% Generic manuscript mode, required for submission
%\documentclass[manuscript,review,anonymous]{acmart}
%\documentclass[sigconf,prologue,dvipsnames,screen]{acmart}
\documentclass[sigconf,screen]{acmart}

\usepackage{acronym}
\usepackage{booktabs} % For formal tables
\usepackage{ccicons}  % For Creative Commons citation icons
\setlength{\marginparwidth}{2cm} 
\usepackage{todonotes}
\usepackage{subfig}
\graphicspath{{figures/}}
\usepackage[english]{babel}
\usepackage[autostyle=true,english=american]{csquotes}
\usepackage{gensymb}
\usepackage{balance}
%\usepackage[most]{tcolorbox}
\usepackage{xspace}

\def\sectionautorefname{Section}
\let\subsectionautorefname\sectionautorefname
\let\subsubsectionautorefname\sectionautorefname

\DeclareCaptionType{equ}[Equation][]

%\newtcolorbox{tbox}{colbacktitle=black!30!white,coltitle=black, breakable, enhanced}
%\newtcolorbox[auto counter]{tboxNum}{colbacktitle=black!30!white,coltitle=black,title={Classification \thetcbcounter}, breakable, enhanced}

\newacro{AI}[AI]{Artificial Intelligence}
\newacro{UI}[UI]{user interface}
\newacro{GUI}[GUI]{graphical user interface}
\newacro{TLX}[TLX]{NASA-Task Load Index}
\newacro{RTLX}[NASA RTLX]{NASA Raw-Task Load Index}
\newacro{ER}[ER]{error rate}
\newacro{TCT}[TCT]{task completion time}
\newacro{HCI}[HCI]{Human-Computer Interaction}
\newacro{UX}[UX]{user experience}
\newacro{HFE}[HFE]{Human Factors and Ergonomics}
\newacro{cuDNN}[cuDNN]{ CUDA Deep Neural Network library}
\newacro{RMSE}[RMSE]{root mean squared error}
\newacro{HMD}[HMD]{Head-Mounted Display}
\newacro{RF}[RF]{Random Forest}
\newacro{GP}[GP]{Gaussian process, long-plural = Gaussian processes}
\newacro{KNN}[\textit{k}NN]{\textit{k}-nearest neighbor}
\newacro{NN}[NN]{Neural Network}
\newacro{DNN}[DNN]{ Deep Neural Network}
\newacro{CNN}[CNN]{Convolutional Neural Network}
\newacro{FCL}[FCL]{fully connected layer}
\newacro{BoD}[BoD]{Back-of-Device}
\newacro{FOV}[FoV]{field of view}
\newacro{RW}[RW]{Real World}
\newacro{IFRC}[IFRC]{index finger ray cast}
\newacro{FRC}[FRC]{forearm ray cast}
\newacro{EFRC}[EFRC]{eye-finger ray cast}
\newacro{HRC}[HRC]{Human-Robot Collaboration}
\newacro{HRI}[HRI]{Human-Robot Interaction}
\newacro{6DOF}[6DOF]{six-degree-of-freedom}
\newacro{LOOCV}[LOOCV]{leave-one-out cross-validation}
\newacro{CV}[CV]{cross-validation}
\newacro{RM}[RM]{repeated measure}
\newacro{ANOVA}[ANOVA]{analysis of variance}
\newacro{RMANOVA}[RM-ANOVA]{repeated measures analysis of variance}
\newacro{AGATe}[AGATe]{AGreement Analysis Toolkit}
\newacro{GHoST}[GHoST]{Gesture Heatmap Toolkit Gesture Heatmaps Toolkit}
\newacro{GREAT}[GREAT]{Gesture Relative Accuracy Toolkit}
\newacro{GRT}[GRT]{Gesture Recognition Toolkit}
\newacro{DTW}[DTW]{Dynamic Time Warping}
\newacro{LHRD}[LHRD]{large high resolution display}
\newacro{GEQ}[GEQ]{Game Experience Questionnaire}
\newacro{SPGQ}[SPGQ]{Social Presence Gaming Questionnaire}
\newacro{JND}[JND]{Just-Noticeable Difference}
\newacro{SUS}[SUS]{system usability scale}
\newacro{CSCW}[CSCW]{computer-supported cooperative work}
\newacro{CAD}[CAD]{computer-aided design}
\newacro{MR}[MR]{Mixed Reality}
\newacro{CVE}[CVE]{Collaborative Virtual Environment}
\newacro{AR}[AR]{Augmented Reality}
\newacro{AV}[AV]{Augmented Virtuality}
\newacro{VR}[VR]{Virtual Reality}
\newacro{PRISMA}[PRISMA]{Preferred Reporting Items for Systematic Reviews}
\newacro{PRISMA-Scope}[PRISMA-ScR]{Meta-Analyses Extension for Scoping Reviews}
\newacro{TF-IDF}[TF-IDF]{Term Frequency-Inverse Document Frequency}
\newacro{TF}[TF]{Term Frequency}
\newacro{AVs}[AVs]{Automated Vehicles}
\newacro{eHMIs}[eHMIs]{external Human-machine interfaces}
\newacro{SAR}[SAR]{Spatial Augmented Reality}
\newacro{IFR}[IFR]{International Federation of Robotics}
\newacro{ADLs}[ADLs]{Activities of Daily Living}
\newacro{LED}[LED]{Light-Emitting Diode}
\newacro{DoF}[DoF]{Degrees-of-Freedom}
\newacro{HHC}[HHC]{Human-Human Collaboration}
\newacro{IDF}[IDF]{Inverse Document Frequency}
\newacro{BD}[BD]{Burst Duration}
\newacro{IBI}[IBI]{Inter-Burst Interval}
\newacro{SOA}[SOA]{Inter-Stimulus Onset Asynchrony}
\newacro{ISI}[ISI]{Inter-Stimulus Interval}
\newacro{LRA}[LRA]{Linear Resonant Actuator}
\newacro{TVSS}[TVSS]{Tactile Vision Substitution System}
\newacro{SDK}[SDK]{Software Development Kit}

%% Fonts used in the template cannot be substituted; margin 
%% adjustments are not allowed.
%%
%% \BibTeX command to typeset BibTeX logo in the docs
\AtBeginDocument{%
  \providecommand\BibTeX{{%
    \normalfont B\kern-0.5em{\scshape i\kern-0.25em b}\kern-0.8em\TeX}}}

%% Rights management information.  This information is sent to you
%% when you complete the rights form.  These commands have SAMPLE
%% values in them; it is your responsibility as an author to replace
%% the commands and values with those provided to you when you
%% complete the rights form.

\copyrightyear{2023}
\acmYear{2023}
\setcopyright{rightsretained}
\acmConference[CHI EA '23]{Extended Abstracts of the 2023 CHI Conference on Human Factors in Computing Systems}{April 23--28, 2023}{Hamburg, Germany}
\acmBooktitle{Extended Abstracts of the 2023 CHI Conference on Human Factors in Computing Systems (CHI EA '23), April 23--28, 2023, Hamburg, Germany}
\acmDOI{10.1145/3544549.3585601}
\acmISBN{978-1-4503-9422-2/23/04}

\acmPrice{15.00}


%%
%% Submission ID.
%% Use this when submitting an article to a sponsored event. You'll
%% receive a unique submission ID from the organizers
%% of the event, and this ID should be used as the parameter to this command.
%%\acmSubmissionID{123-A56-BU3}

%%
%% For managing citations, it is recommended to use bibliography
%% files in BibTeX format.
%%
%% You can then either use BibTeX with the ACM-Reference-Format style,
%% or BibLaTeX with the acmnumeric or acmauthoryear sytles, that include
%% support for advanced citation of software artefact from the
%% biblatex-software package, also separately available on CTAN.
%%
%% Look at the sample-*-biblatex.tex files for templates showcasing
%% the biblatex styles.
%%

%%
%% The majority of ACM publications use numbered citations and
%% references.  The command \citestyle{authoryear} switches to the
%% "author year" style.
%%
%% If you are preparing content for an event
%% sponsored by ACM SIGGRAPH, you must use the "author year" style of
%% citations and references.
%% Uncommenting
%% the next command will enable that style.
%%\citestyle{acmauthoryear}
\newcommand{\conA}{\emph{Rabbit Single}\xspace}
\newcommand{\conB}{\emph{Rabbit Dual}\xspace}
\newcommand{\conC}{\emph{Motion Intensity}\xspace}
%\newcommand\change[1]{{\color{RoyalBlue}#1}}
\newcommand\change[1]{{#1}}
%%
%% end of the preamble, start of the body of the document source.
\begin{document}

%%
%% The "title" command has an optional parameter,
%% allowing the author to define a "short title" to be used in page headers.
%\title{HaptiX: Communicating Cobot's Motion Intention in 3D Space by Extending Visualizations with Haptic Feedback}
\title[HaptiX: Vibrotactile Haptic Feedback for Communication of 3D Directional Cues]{HaptiX: Vibrotactile Haptic Feedback for Communication of \\3D Directional Cues}

%%
%% The "author" command and its associated commands are used to define
%% the authors and their affiliations.
%% Of note is the shared affiliation of the first two authors, and the
%% "authornote" and "authornotemark" commands
%% used to denote shared contribution to the research.
\author{Max Pascher}
\orcid{0000-0002-6847-0696}
\email{max.pascher@w-hs.de}
\affiliation{
    \institution{Westphalian University of Applied Sciences}
    \city{Gelsenkirchen}
    \country{Germany}
}
\affiliation{
    \institution{University of Duisburg-Essen}
    \city{Essen}
    \country{Germany}
}

\author{Til Franzen}
\orcid{0000-0003-0203-7512}
\email{til.franzen@studmail.w-hs.de}
\affiliation{
    \institution{Westphalian University of Applied Sciences}
    \city{Gelsenkirchen}
    \country{Germany}
}

\author{Kirill Kronhardt}
\orcid{0000-0002-0460-3787}
\email{kirill.kronhardt@.w-hs.de}
\affiliation{
    \institution{Westphalian University of Applied Sciences}
    \city{Gelsenkirchen}
    \country{Germany}
}

\author{Uwe Gruenefeld}
\orcid{0000-0002-5671-1640}
\email{uwe.gruenefeld@uni-due.de}
\affiliation{
    \institution{University of Duisburg-Essen}
    \city{Essen}
    \country{Germany}
}

\author{Stefan Schneegass}
\orcid{0000-0002-0132-4934}
\email{stefan.schneegass@uni-due.de}
\affiliation{
    \institution{University of Duisburg-Essen}
    \city{Essen}
    \country{Germany}
}

\author{Jens Gerken}
\orcid{0000-0002-0634-3931}
\email{jens.gerken@w-hs.de}
\affiliation{
    \institution{Westphalian University of Applied Sciences}
    \city{Gelsenkirchen}
    \country{Germany}
}

%%
%% By default, the full list of authors will be used in the page
%% headers. Often, this list is too long, and will overlap
%% other information printed in the page headers. This command allows
%% the author to define a more concise list
%% of authors' names for this purpose.
\renewcommand{\shortauthors}{Max Pascher et al.}

%%
%% The abstract is a short summary of the work to be presented in the
%% article.
\begin{abstract}
\begin{abstract}

Large language models (LLMs) can enhance writing by automating or supporting specific tasks in writers' workflows (e.g., paraphrasing, creating analogies).
Leveraging this capability, a collection of interfaces have been developed that provide LLM-powered tools for specific writing tasks. However, these interfaces provide limited support for writers to create personal tools for their own unique tasks, and may not comprehensively fulfill a writer’s needs---requiring them to continuously switch between interfaces during writing.
In this work, we envision LMCanvas, an interface that enables writers to create their own LLM-powered writing tools and arrange their personal writing environment by interacting with ``blocks’’ in a canvas. 
In this interface, users can create text blocks to encapsulate writing and LLM prompts, model blocks for model parameter configurations, and connect these to create pipeline blocks that output generations.
In this workshop paper, we discuss the design for LMCanvas and our plans to develop this concept.

\end{abstract}
\end{abstract}

%%
%% The code below is generated by the tool at http://dl.acm.org/ccs.cfm.
%% Please copy and paste the code instead of the example below.
%%
\begin{CCSXML}
<ccs2012>
   <concept>
       <concept_id>10003120.10003123.10010860.10010859</concept_id>
       <concept_desc>Human-centered computing~User centered design</concept_desc>
       <concept_significance>300</concept_significance>
       </concept>
   <concept>
       <concept_id>10003120.10003121.10003125.10011752</concept_id>
       <concept_desc>Human-centered computing~Haptic devices</concept_desc>
       <concept_significance>500</concept_significance>
       </concept>
   <concept>
       <concept_id>10010583.10010588.10010598.10011752</concept_id>
       <concept_desc>Hardware~Haptic devices</concept_desc>
       <concept_significance>500</concept_significance>
       </concept>
   <concept>
       <concept_id>10010147.10010371.10010387.10010866</concept_id>
       <concept_desc>Computing methodologies~Virtual reality</concept_desc>
       <concept_significance>100</concept_significance>
       </concept>
 </ccs2012>
\end{CCSXML}

\ccsdesc[300]{Human-centered computing~User centered design}
\ccsdesc[500]{Human-centered computing~Haptic devices}
\ccsdesc[500]{Hardware~Haptic devices}
\ccsdesc[100]{Computing methodologies~Virtual reality}

%%
%% Keywords. The author(s) should pick words that accurately describe
%% the work being presented. Separate the keywords with commas.
\keywords{directional cues, haptic feedback, vibrotactile feedback}

%% A "teaser" image appears between the author and affiliation
%% information and the body of the document, and typically spans the
%% page.

\begin{teaserfigure}
\centering
\captionsetup{justification=centering}
    \subfloat[Rabbit Single]{\includegraphics[width=0.32\linewidth]{figures/RabbitSingle.png}\label{fig:rabbitSingle}}
    \hfill
    \subfloat[Rabbit Dual]{\includegraphics[width=0.32\linewidth]{figures/RabbitDual.png}\label{fig:rabbitDual}}
    \hfill
    \subfloat[Motion Intensity]{\includegraphics[width=0.32\linewidth]{figures/MotionIntensity.png}\label{fig:motionIntensity}}
    \hfill
    \captionsetup{justification=justified}
\caption{\change{Coding of the gradient for \textbf{a}) \conA, (\textbf{b}) \conB, and (\textbf{c}) \conC. The orange arrow represents the gradient of the directional cue with an intended increase over time. The timing, duration, number of pulses, and intensity (S1 -- S7) of the three actuators (T1 -- T3) are illustrated for each condition.}}
\Description{An overview of the gradient coding for the three conditions (three figures in a row). All figures consist of a hand, palm facing downward in a diagram t (Pulses over the duration t) over y (Intensity level S1 -- S7). A orange arrow from (0,0) toward the tip of the finger represents the gradient of the directional cue with an intended increase over time. Figure 1a: Rabbit Single: The actuator T1 has two, T2 four, and T3 six pulses, all with the same intensity. Figure 1b: Rabbit Dual: The actuator T1 has two pulses with the intensity S1, T2 four pulses with the intensity S3, and T3 six pulses with the intensity S6. Figure 1c: Motion Intensity: The actuator T1 starts at t = 0 with the intensity S1 and the duration is a third of the whole diagram, overlapping with the actuator T2. The actuator T2 starts with the intensity S4 after a third of the whole time axis in the diagram and the duration is a third of the whole diagram, overlapping with the actuator T3. The actuator T3 starts with the intensity S6 in the last third of the whole time axis in the diagram and the duration is a third of the whole diagram.}
\label{fig:variants}
\end{teaserfigure}

%%
%% This command processes the author and affiliation and title
%% information and builds the first part of the formatted document.
\maketitle

\section{introduction}

% 1. importance of TKGs and reasoning on TKGs. 
% 2. low resource languages, main main idea.
% 3. relations and limitations of current works.
% 4. summarize our solutions and contributions.

Temporal Knowledge Graphs (TKGs)~\cite{YAGO,ICEWS18,WIKI,acekg} characterize temporally evolving events, where each event, represented as ({\em subject}, {\em relation}, {\em object}), is associated with temporal information ({\em time}), e.g., ({\em Macron}, {\em reelected}, {\em French president}, {\em 2022}). TKGs has facilitated various knowledge-intensive Web applications with timeliness, such as question answering~\cite{KBQA}, product recommendation~\cite{RippleNet,TKG4Rec,TKG4Rec2,RETE}, and social event forecasting~\cite{KG4Social,DyDiff-VAE,andgan,belief,misinfo,polarization}. 

As new events are continually emerging, modern TKGs are still far from being complete. Conventionally, the TKG construction process relies primarily on information extraction from unstructured corpus~\cite{WIKI,YAGO, EventKG}, which necessitates extensive manual annotations to keep up with changing events. For instance, the recent transition from Trump to Biden as the President of the United States has not been reflected in many TKGs, highlighting the need for timely updates. This spurs research on temporal knowledge graph reasoning to automate evolving events prediction over time~\cite{TA-DistMult,Know-Evolve,Renet,RE-GCN}. Unfortunately, the problem of TKG incompleteness is particularly pronounced in low-resource languages, where it is unable to collect enough corpus and annotations to support robust TKG construction. This results in suboptimal reasoning performance and distinctly unsatisfying accuracy in predicting recent and future events.

% whose performance can degrade significantly in low-resource language TKGs that suffer from severe incompleteness over time. 
% \jingfeng{why don't people  study cross-lingual TKG previously, (i.e. use language alignment to improve TKG). Is it really helpful intuitively to use high resource language to help TKGC? For instance, is it enough to use static langauge-alignment to help KGC, ignoring the temporal information? Are those langauge-alignment changing across time?}



\begin{figure}
    \centering
    \includegraphics[width = 1.0\linewidth]{fig/task.pdf}
    \caption{An illustrative example of cross-lingual reasoning on TKGs. 1) We aim to transfer knowledge from English TKG to Japanese TKG, where the English version provides more complete information; 2) Cross-lingual alignments only cover a small ratio of entities, e.g., Apple Inc; 3) Cross-lingual alignments can be noisy and misleading, e.g., A city called Ventura is linked to new macOS Ventura at $t_2$, introducing noise for reasoning in Japanese.}
    \label{fig:illustration}
    %\vspace{-6mm}
\end{figure}

Inspired by the incompleteness issue facing low-resource languages in constructing TKGs, we introduce a novel task named Cross-Lingual Temporal Knowledge Graph Reasoning (as shown in Figure~\ref{fig:illustration}). This task aims to alleviate the reliance on supervision for TKGs in low-resource languages (referred to as the target language) by transferring temporal knowledge from high-resource languages (referred to as the source language)~\footnote{In this paper, for the sake of brevity, we interchangeably use the terms high-resource/low-resource and source/target.}. In contrast, all the existing efforts are either limited to reasoning in monolingual TKGs (usually high-resource languages, e.g., English)~\cite{TA-DistMult,Know-Evolve,Renet,RE-GCN}, or multilingual static KGs~\cite{KEnS,AlignKGC,SS-AGA}. To the best of our knowledge, cross-lingual TKG reasoning that transfers temporal knowledge between TKGs has not been investigated. 

%Motivated by this, we study a new task named {\em cross-lingual temporal knowledge graph reasoning} as shown in Figure~\ref{fig:illustration}, to alleviate the heavy dependence on supervision for any resource-poor language TKGs by distilling the temporal knowledge from resource-rich ones. Differently, all the existing efforts are either limited to reasoning in monolingual (usually high-resource languages, e.g., English) temporal KGs~\cite{TA-DistMult,Know-Evolve,Renet,RE-GCN}, or multilingual static KG~\cite{KEnS,AlignKGC,SS-AGA}, but neglecting the reasoning in a both temporal and cross-lingual manner that highly requires capturing time-evolving patterns and language discrepancy. To the best of our knowledge, this problem, regarding how to transfer cross-lingual knowledge between TKGs, has still not been formally investigated. 

% Unlike conventional TKG reasoning, 
The fulfillment of this task poses tremendous challenges in two aspects: 1) \textbf{Scarcity of cross-lingual alignment}: as the informative bridge of two separate TKGs, cross-lingual alignment is imperative for cross-lingual knowledge transfer~\cite{AlignKGC,KEnS,SS-AGA}. However, obtaining alignments between languages is a time-consuming and resource-intensive process that heavily relies on human annotations. The transfer of knowledge through a limited number of alignments is often insufficient to fully enhance the TKG in the target language. 2) \textbf{Temporal knowledge discrepancy}: the information associated with two aligned entities is not necessarily identical, especially with regards to temporal patterns. Utilizing a rough approach to equate the aligned entities at all times can result in the transfer of misleading knowledge and negatively impact performance. This becomes more pronounced when the alignments are noisy and unreliable. For example, at the time step $t_2$, a new event about operating system ``{\it Ventura}'' from Apple company occurs in the source English TKG, and meanwhile there is a noisy aligned entity ``{\it Ventura city}'' in the target Japanese TKG. Directly pulling those two entities at this point, can inevitably introduce  noise and fail to predict a set of related events in the target TKG. Therefore, it is crucial to dynamically regulate the alignment strength of each local graph structure over time in order to maximize the effectiveness of cross-lingual knowledge distillation.

% Pulling those entities together cannot augment information in target languages. Small alignment strength is beneficial in the unreliable alignment cases, otherwise the misleading knowledge transferring can even hurt the performance.

% Moreover, in a case that the alignments are not fully reliable, directly pulling the two aligned entities together 


% optimally dynamic alignment strength
% {\em Optimal alignment strength to maximize the benefits of knowledge distillation is difficult to obtain, especially in the temporal manner.} 
% In practical, although the aligned entities can share similar information, they may still differ in other perspectives, including but not limited to frequency, interactions, and temporal patterns. How to adjust the alignment strength (i.e., the distance constrains of the aligned entities in the uni-space) accordingly for different entities at different time is unclear. \zheng{Ruijie TODO: add Ventura case}Moreover, in a case that the alignments are not fully reliable, directly pulling the two aligned entities together can even hurt the performance.



% scarcity of hinders the efficient
% knowledge transfer across languages. 
% {\em Transferring knowledge through a small set of alignments is hard to augment information for all entities.} 

% Aligning the same entities across languages rely heavily on manual labeling or rule-based inference~\cite{EA1,EA2,EA3,selfKG}, which is too time-consuming and impractical to obtain the alignments covering most of the entities in target language. 

% In this paper, we study how to boost the TKG reasoning performance in low-resource languages by explicitly increasing the completeness of those TKGs in history. Instead of improving the underlying information extraction techniques in low-data regime, we propose a new task called {\em Cross-lingual Temporal Knowledge Graph Reasoning}, motivated by the facts that there exists common or complementary knowledge shared by the TKGs in different languages under similar topics. The new task aims to facilitate TKG reasoning in low-resource languages (target languages) by distilling knowledge from a corresponding TKG in high-resource language (source language)  through a small set of entity alignments as bridges~\footnote{In this paper, we interchangeably use the terminology high-resource/low-resource and source/target for briety.}. Figure~\ref{fig:illustration} provides an illustrative example of the proposed task.


% Unfortunately, recent breakthroughs in temporal knowledge graph reasoning model~\cite{TA-DistMult,Know-Evolve,Renet,RE-GCN} highly rely on the completeness of the TKGs, especially for the most recent events. 

% However, the completeness of TKGs varies a lot across different languages, even under similar topics. Conventionally, the TKG construction process relies primarily on information extraction techniques built on the unstructured corpus~\cite{WIKI,YAGO, EventKG}. Therefore, the amount of corpus and human annotations in different languages significantly influence the quality of the corresponding TKGs . 
% Therefore, automatically completing/updating TKGs has been attracting enormous interests in recently years, which aims to predict recent/future events on TKGs based on historical events~\cite{TA-DistMult,Know-Evolve,Renet,RE-GCN}, namely temporal knowledge graph reasoning~\footnote{Broadly speaking, TKG reasoning includes interpolation to predict historical events and extrapolation to predict future events. In this paper, we refer to extrapolation task as TKG reasoning, since it is more vital for time-sensitive downstream tasks.}.


% For languages with large-scale and carefully labeled corpus (we refer to as high-resource languages, e.g., English), the constructed TKGs are more comprehensive than TKGs in other languages that lack the high-quality corpus (we refer to as low-resource languages, e.g., Spanish, Slovene, Danish, etc). Such completeness discrepancy leads to distinctly uneven TKG reasoning performances in different languages, which in turn affects the quality of service of the downstream applications. 


% Compared with the traditional TKG reasoning task, the new task imposes non-trivial challenges. An intuitive solution is to construct a unified graph including two TKGs in both source and target languages, and the knowledge distillation can be fulfilled by pulling the aligned entities from two languages close to each other in the uni-space~\cite{AlignKGC,KEnS}. However, there are still two challenges to be addressed. 

% \zheng{Ruijie TODO, Place this part to related works.}
% Existing works in related areas fail to address the aforementioned challenges. Monolingual reasoning methods on static/temporal knowledge graphs~\cite{TransE,TranR,ComplEX,RotatE,TA-DistMult,Know-Evolve,Renet,RE-GCN} is incapable of the desired knowledge transferring due to the insufficient alignment modeling. Although they can be extended on the cross-lingual scenario by viewing the alignments as a new relation on the merged TKGs, the limited amount of alignments prevent them from augmenting information for most of the entities. Entity alignment methods on KGs~\cite{EA1,EA2,EA3,EA4,EA5,selfKG} can automatically enlarge the alignments by  predicting the correspondence between the two TGs. But most of them, if not all, require the relatively even completeness of two TGs to capture the structural similarities, which can not be satisfied in our case, as target TKGs are far from complete. Some recent works start to study the multilingual TK reasoning on static graphs~\cite{AlignKGC,KEnS,SS-AGA}, which similarly aim to extract knowledge from several source KGs to boost the reasoning performance in the target KG, while they still require a sufficient amount of cross-lingual alignments and totally ignore the temporal perspective in our task.

% to facilitate temporal knowledge graph reasoning in low-resource languages. 
% increase the TKG connection and target TKG capacity
% In light of the mutual benefits, we iteratively generate pseudo alignment pairs and pseudo temporal events to address the co-existing scarcity issue in both cross-lingual alignment and target TKGs. 


In this paper, we propose a novel Mutually-paced Knowledge Distillation (\model) framework, where a teacher network learns more enriched temporal knowledge and reasoning skills from the source TKG to facilitate the learning of a student network in the low-data target one. The knowledge transfer is enabled via an alignment module, which estimates entity correspondence across languages based on temporal patterns. Firstly, to alleviate the limited language alignments (\textbf{Challenge \#1}), such a knowledge distillation process is mutually paced over time. This means, on one hand, we encourage the mutually interactive learning between the teacher and student. Concretely, the alignment module between the teacher and the student learns to generate pseudo alignment between TKGs to maximally expand the upper bound of knowledge transfer. And subsequently, it empowers the student to encode more informative knowledge in target TKG, which can in turn boost the alignment module to explore more reasonable alignments as the bridge across TKGs. One the other hand, inspired by self-paced learning~\cite{spl-1,spl-2}, we make the generations as a progressively easy-to-hard process over time. We start from generating reliable pseudo data with high confidence. As time goes by, we then gradually increase the generation amount by relieving the restriction over time. Secondly, to inhibit the temporal knowledge mismatch (\textbf{Challenge \#2}), the attention module can estimate the graph alignment strength distribution over time. This is achieved by a temporal cross-lingual attention in terms of the local graph structure and temporal-evolving patterns of aligned entities. As such, it can dynamically control the negative effect and suppress noise  propagation from the source TKG. Moreover, we provide a theoretical convergence guarantee for the training objective on both initial ground-truth data and pseudo data. To evaluate \model, we conduct extensive experiments of 12 cross-lingual TKG transfer tasks in multilingual EventKG dataset~\cite{EventKG}. Our empirical results show that the \model method outperforms state-of-the-art baselines in both with and without alignment noise settings, where only $20\%$ of temporal events in the target KG and $10\%$ of cross-lingual alignments are preserved.

% To validate the effectiveness of \model, we conduct extensive experiments of 12 cross-lingual TKG transfer tasks in multilingual EventKG benchmark dataset~\cite{EventKG} . Our experimental results empirically demonstrate the superiority of the \model method over state-of-the-art baselines, ranging from static KG embedding~\cite{TransE,TransR,DistMult,RotatE}, temporal KG reasoning~\cite{TA-DistMult,Renet,RE-GCN} to multilingual KG completion~\cite{KEnS,AlignKGC,SS-AGA}, in both with and without alignment noise settings. We further conduct comprehensive ablation and hyperparameter studies to validate the effectiveness of each design choices. Moreover, we provide theoretical analysis of convergence guarantee for the training objective on both initial groundtruth data and pseudo generative data.



To sum up, our contributions are three-fold:

\begin{itemize}[leftmargin = 15pt]
    \item \textbf{Problem formulation}: We propose the cross-lingual temporal knowledge graph reasoning task, to boost the temporal reasoning performance in target TKG by transferring knowledge from source TKG;
    \item \textbf{Novel framework}: We propose a novel \model framework, which enables the mutually-paced learning between the teacher and student networks, to promote both pseudo alignments and knowledge transfer reliability. Besides, \model involves a dynamic alignment estimation across TKGs that inhibits the influence of temporal knowledge discrepancy.
    \item \textbf{Extensive evaluations}: Empirically, extensive experiments on 12 cross-lingual TKG transfer tasks in multilingual EventKG benchmark dataset demonstrate the effectiveness of \model.
\end{itemize}
% pseudo data generation technique to progressively enhance the training data. The generated pseudo alignments can help the training of the representation modules by the knowledge distillation, and in turn adding pseudo events in the target TKG can improves alignment module by providing high-quality representations. 




% interactively
% TKGs in a source language and a target language are represented by a teacher representation module and a student one into a uni-space, respectively. 
% The knowledge distillation is enabled by a cross-lingual alignment module which pulls the aligned entities close to each other and push other entities far away. 
% To address the challenge caused by the scarcity of cross-lingual alignment, 


%\section{Related Work}

\mypara{2D instance segmentation.}
Instance segmentation in 2D is well-studied.
Before the popularity of vision transformers, prior work adopted region proposal-based methods~\cite{girshick2015fast,ren2015faster,he2017mask}.
\citet{carion2020end} used a transformer decoder to convert the instance segmentation task into a set prediction task with the Hungarian algorithm for a one-to-one matching loss.
MaskFormer~\cite{cheng2021per} further converted the problem into a mask classification problem to unify all 2D segmentation tasks (i.e. semantic segmentation, instance segmentation and panoptic segmentation) and achieved better results.
Recently, Mask2Former~\cite{cheng2021masked} achieved state-of-the-art results in 2D instance segmentation.
Our work builds on recent progress from instance segmentation, taking inspiration from transformer architectures that achieve state-of-the-art instance detection and segmentation performance.

\mypara{Articulated object understanding.}
With the increasing interest in embodied AI, understanding articulated objects is an important research direction.
A number of datasets of articulated objects have been recently introduced, including both synthetic \cite{xiang2020sapien,wang2019shape2motion} and scanned datasets \cite{jiang2022opd,qian2022understanding,liu2022akb,mao2022multiscan}.
These datasets have annotations of part segmentation and corresponding motion parameters.
Such data has enabled data-driven methods for predicting motion parameters from 3D meshes~\cite{hu2017learning} and points clouds~\cite{wang2019shape2motion,yan2019rpm}.
More recent work has focused on detecting articulated parts and their motion parameters from single-view point-clouds~\cite{li2020category}, images~\cite{zeng2021visual,jiang2022opd} and  videos~\cite{qian2022understanding,heppert2022category}, which are closer to real applications.
Researchers have also started to investigate how to use predicted segmentation and motion parameters to automatically create articulated objects~\cite{jiang2022ditto,collins20232}, including in scenes~\cite{hsu2023ditto}.



\mypara{Openable part detection.}
\citet{jiang2022opd} introduced the openable part detection (OPD) task to address the articulated object motion prediction problem for single-view image inputs.  In their work, they focused on images with a single main object and predicting the openable parts for that one object.
Our work generalizes the OPD task to more realistic images with multiple objects.
We also develop a part-informed transformer architecture that leverages object poses to predict more consistent and accurate part motions.


\section{Preliminaries}
\noindent\textbf{Denoising Diffusion Probabilistic Models with classifier-free guidance:}
Diffusion models are probabilistic models that approximate the data distribution by iteratively adding noise and denoising through a forward/reverse Gaussian Diffusion Process~\citep{ddpm,song2020score}. The forward process applies noise at each time step $t\in{0,...,T}$ to the data distribution $\mathbf{x}_{0}$, creating a noisy sample $\mathbf{x}_t$ where $\mathbf{x}_t = \sqrt{\bar{\alpha}_t}\mathbf{x}_0+\sqrt{1-\bar{\alpha}_t}\bm{\epsilon}$ ($\bm{\epsilon}\sim\mathcal{N}(\boldsymbol{0},\boldsymbol{I})$), and $\bar{\alpha}_t$ is the accumulation of the noise schedule $\alpha_{0:T}$ defined by $\bar{\alpha}_t=\prod^t_{s=1}\alpha_s$. To denoise images, the diffusion process uses a reparameterized variant of Gaussian noise prediction $\bm{\epsilon}_\theta(\mathbf{x}_t,t)$ targeting Gaussian noise $\bm{\epsilon}$. The reverse process $p(\mathbf{x}_{t-1}|\mathbf{x}_{t})$ of the Markov Chain generates new samples from Gaussian noise, which is approximated by Bayes' theorem as $q(\mathbf{x}_{t-1}|\mathbf{x}_t,\mathbf{x}_0)$, where $\mathbf{x}_0$ is derived from the forward process as $\mathbf{x}_0 = \frac{1}{\sqrt{\bar{\alpha}_t}}(\mathbf{x}_t-\sqrt{1-\bar{\alpha}_t\bm{\epsilon}_\theta(\mathbf{x}_t,t)})$.

Classifier-free guidance~\citep{clsfree} is introduced for conditional diffusion models to generate images without requiring an extra image classifier. A conditional model with a parameterized reverse process $p(\mathbf{x}_{t-1}|\mathbf{x}_t,\mathbf{c})$ uses a conditional identifier $\mathbf{c}$ through $\bm{\epsilon}_{\theta}(\mathbf{x}_t,t,\mathbf{c})$. To predict an unconditional score, the conditional identifier is replaced with a null token $\O$ and denoted as $\bm{\epsilon}_{\theta}(\mathbf{x}_t,t,\mathbf{c}=\O)$. Classifier-free guidance can then be approximated as a linear combination of conditional and unconditional predictions:
\vspace{-3pt}
\begin{equation}
   \bm{\tilde{\epsilon}}_{\theta}(\mathbf{x}_t,t,\mathbf{c}) = (1+w)\bm{\epsilon}_{\theta}(\mathbf{x}_t,t,\mathbf{c})-w\bm{\epsilon}_{\theta}(\mathbf{x}_t,t,\mathbf{c}=\O),
   \vspace{-3pt}
\end{equation}
where $w$ is the guidance scale. Text-video and text-image-based diffusion models~\citep{ldm,imagen,glide,vdm,makeavideo} use DDPM with classifier-free guidance. This diffusion method can be adapted to various tasks with flexibility.

\noindent\textbf{Latent Diffusion Models:} 
Compared with image diffusion, video diffusion has significantly higher computation costs because it needs to process multiple frames.
Recent works have explored the computation-efficient version of diffusion modeling, such as latent diffusion model (LDM)~\citep{ldm}. LDM proposes the VAE-based latent diffusion, including a KL-regularized autoencoder for encoding/decoding latent representation $\bm{\varepsilon}(\mathbf{x})$, and a diffusion model to operate on the latent space $\mathbf{z}_t$.
For the conditional generation, LDM introduces a domain-specific encoder $\bm{\tau}_\theta$ to the projection of condition $\mathbf{y}$ for various modality generations. Thus, the objective of LDM is: 
\vspace{-5pt}
\begin{equation}
    \vspace{-10pt}
    L_{\mathrm{LDM}} = \mathbb{E}_{t,\bm{\varepsilon}(\mathbf{x}),\mathbf{y},\bm{\epsilon}\sim\mathcal{N}(\boldsymbol{0},\boldsymbol{I})}\Bigr[\|\bm{\epsilon} - \bm{\epsilon}_\theta(\bm{z}_t,t,\bm{\tau}_\theta(\mathbf{y}))\|^2\Bigr]
\end{equation}

\section{Methodology}\label{sec:method}
In this paper, we aim to explore an efficient diffusion method to predict coherent video frames guided by language instructions, which requires learning to parse natural language, understand the scene, and ground the language and scene together. However, it is challenging to directly apply conventional video diffusion models for TVP due to the following problems: (1) The limited labeled text-video data and computational resources. (2) Low fidelity of frame generation. (3) Lack of fine-grained instruction for each frame in the task-level videos.
 
%\gu{Specifically, inheriting from I3D~\cite{i3d} technique, we build an inflated 3D U-Net to extend the prior knowledge contained in Stable Diffusion across the frames to generate high-quality and coherent frames by inserting computation-efficient spatial-temporal attention layers (Sec.~\ref{sec:efficientnet}).} As for the language conditioning model, we propose a novel Frame Sequential Text (FSText) Decomposer to adaptively decompose the text instruction into sub-conditions for each frame (Sec.~\ref{sec:fstext}).

\subsection{Overview of Seer}
\label{sec:inflate}
Motivated by the robust generative capabilities of text-to-image (T2I) diffusion models, we leverage the prior knowledge implied in pretrained T2I models by inflating the 2D U-Net~\citep{ldm} and incorporating temporally consistent layers. However, the inflated video diffusion model guided solely by coarse global language instruction tends to generate irrelevant T2I outcomes and fails to maintain temporal coherency between video frames. To address this limitation and provide precise and controllable guidance for our inflated model, we introduce a novel temporal decomposition component for language instruction, this component decomposes global instruction as temporally aligned sub-instruction for delicate task-level guidance, which significantly enhances the fidelity and coherency of predicted video.

Our Seer method comprises two main components: the video diffusion and the language conditioning modules. We propose to enhance these two components to facilitate high-fidelity frame synthesis and the temporal alignment of text instructions, respectively. Specifically, as shown in Figure~\ref{fig:pipeline} (a), we utilize two pathways to implement the conditional diffusion process guided by reference frames and language: \textbf{1)} We incorporate the spatial-temporal module discussed in Section~\ref{sec:efficientnet} into the Inflated 3D U-Net. This integration enables the propagation of contextual information from reference frames to future frames within the spatial-temporal space, allowing for coherent motion prediction based on the reference frames. \textbf{2)} To plan continuous motion with fine-grained language guidance, we introduce a Frame Sequential Text (FSText) Decomposer in Section~\ref{sec:fstext}. This module transforms global language instructions into multi-timestep sub-instructions that are synchronized with video. Subsequently, we inject these frame-wise subinstruction tokens into the intermediate latent space of the video frames at each time step.
With this design, we merely train the spatial-temporal layers and FSText module from scratch while freezing the remaining pretrained modules within our 3D inflated U-Net. These two modules are jointly trained by the diffusion objective, where $f_\theta$ is our FSText decomposer, $\bm{\tau}$ is the frozen CLIP text encoder, and $\mathbf{y}$ is the input text:
\begin{equation}
    L_{\mathrm{diffusion}} = \mathbb{E}_{t,\bm{\varepsilon}(\mathbf{x}),\mathbf{y},\bm{\epsilon}\sim\mathcal{N}(\boldsymbol{0},\boldsymbol{I})}\Bigr[\|\bm{\epsilon} - \bm{\epsilon}_\theta(\mathbf{z}_t,t,f_\theta(\bm{\tau}(\mathbf{y})))\|^2\Bigr],
\end{equation}


%\chuan{I find this paragraph is similar to the last paragraph of page 4. Just keep one to reduce redundancy.}
%Since the T2I latent diffusion models consist of two main components: the image diffusion module and the language conditioning module~\cite{ldm}. We propose to inflate these two parts to perform the synthesis of high-fidelity frames and the temporal decomposition of text instructions, respectively. \gu{Specifically, as shown in Figure~\ref{fig:pipeline} (a), we incorporate the computation-efficient spatial-temporal module into the Inflated 3D U-Net for optimizing temporal consistency in Section~\ref{sec:efficientnet}, and we propose a Frame Sequential Text (FSText) Decomposer for text conditioning module in Section~\ref{sec:fstext}.} Overall, we adopt two pathways to implement the conditional diffusion process of language guidance and reference frames. During training, we stack the latent space of the reference frames with the noisy latent space of the remaining frames along the temporal dimension. During inference, we predict future frames by propagating the prior reference frames and Gaussian noise through the Inflated 3D U-Net. For text conditioning, we employ FSText Decomposer to incorporate the text condition into the diffusion model.





\subsection{Data \& Computation-efficient 3D Network}\label{sec:efficientnet}
To design a computation-efficient visual backbone for our video diffusion model,  we refer to some relevant works on lifting 2D to 3D video modeling~\citep{i3d} and efficient attention computation~\citep{swin, videoswin}. In general, we leverage the latent diffusion model (LDMs) pretrained on T2I tasks to build a text-video model. Our inflated 3D U-Net consists of two principal components as illustrated in Figure~\ref{fig:pipeline} (b): \textbf{1)} The 3D spatial layers, where we draw inspiration from I3D~\citep{i3d} and enhance the 2D convolution kernel from ($3 \times 3$) to a 3D counterpart ($1 \times 3 \times 3$)  with an added video frames axis from the pre-trained 2D modules, consisting of a series of 2D ResNet blocks and Spatial Attention Blocks.
\begin{wrapfigure}[17]{r}{0.42\textwidth}
\centering
\vspace{-10pt}
\includegraphics[width=0.8\linewidth]{fig/wintempatten.pdf}
\vspace{-10pt}
\caption{Variants of temporal attention, only the blue tokens attend to the current token in the red box. Red dashed arrows indicate the direction of attention. And the orange boxes indicate the local window region ($2\times 2$ window in this case).}
\label{fig:tempattn}
\end{wrapfigure}
\textbf{2)} The temporal layers, play a crucial role in our visual backbone for propagating contextual information from the reference frame's image prior across the temporal sequence. We investigated various temporal attention and incorporated them into our 3D U-Net architecture. Our empirical observations indicate that bi-directional temporal attention tends to disregard guidance from reference frames, and both bi-directional and directed temporal attention struggle to capture dependencies among spatial regions, as discussed in Section~\ref{sec:ablate:temp}. To address these limitations while reducing complexity, we employ an efficient approach that builds upon the concept of window attention~\citep{swin} in 3D space: the implementation of local window attention in an autoregressive manner across spatial-temporal dimensions. As illustrated in Figure~\ref{fig:tempattn}, we establish fixed local windows for each spatial region with a window size of $m \times m$ relative to the global frame sequence $n$. Within this framework, we compute self-attention using a causal mask, considering both local spatial and global temporal dimensions within the 3D space. This effectively constrains pixel propagation from the future temporal-spatial sequence.

Finally, We maintain the acquired knowledge from the 2D modules by freezing all pretrained weights and exclusively training the spatiotemporal attention layers during fine-tuning. Overall, through a combination of frozen pre-trained spatial layers and lightweight spatiotemporal layers, our inflated 3D U-Net not only retains crucial knowledge but also enhances fine-tuning efficiency.

%Our empirical findings indicate that bi-directional temporal attention tends to disregard visual guidance from reference frames in time sequence (as discussed in Section~\ref{sec:ablate:temp}), and both bi-directional and directed temporal attention also miss out on capturing dependencies among nearby spatial regions, resulting in suboptimal frame quality. To address these limitations and enhance generation while reducing complexity, we employ an efficient approach that builds upon the concept of window attention~\cite{swin} in 3D space: the implementation of local window attention in an autoregressive manner across spatial-temporal dimensions. As illustrated in Figure~\ref{fig:tempattn}, we establish fixed local windows for each spatial region with a window size of $m \times m$ relative to the global frame sequence $n$. Within this framework, we compute self-attention with a causal mask across both local spatial and global temporal space within the 3D space, effectively constraining the pixel propagation from the future temporal-spatial sequence.
%The utilization of text-to-image (T2I) priors enhances the generative and imaginative capabilities of video generative models~\citep{makeavideo}. In this spirit, we leverage the latent diffusion model (LDMs) pretrained on T2I tasks, inflating it along the temporal axis. Our proposed latent diffusion model consists of two principal components as illustrated in Figure~\ref{fig:pipeline} (b): \textbf{1)} The 3D spatial layers inflated from the pre-trained image diffusion module, consisting of a series of 2D ResNet blocks and Spatial Attention Blocks. To adapt these 2D modules for 3D processing, we draw inspiration from I3D~\cite{i3d} to enhance the 2D convolution kernel from ($3 \times 3$) to a 3D counterpart ($1 \times 3 \times 3$)  with an added video frames axis. \textbf{2)} The temporal layers propagate contextual information from the image prior across the temporal sequence. We maintain the learned text-to-image knowledge from 2D modules by freezing all pre-trained weights during fine-tuning. This strategy not only retains crucial knowledge but also enhances fine-tuning efficiency.
%In comparison to plain spatial-temporal attention, the application of local windows to spatial regions significantly reduces computational overhead while delivering high-fidelity generation results. Notably, we observe that adopting SAWT-Atten has only a marginal $2.13\%$ computation speed lag compared to directed and bidirectional temporal attention, as shown in Appendix Table~\ref{table:speed_temp}.
%To address these limitations and enhance generation while reducing complexity, we employ an efficient approach: implementing local window attention in an autoregressive manner on both temporal and spatial spaces. Specifically, in this paper, extending from window attention~\cite{swin} in 3D space, we adopt a fixed window strategy with window sizes of $8\times8$, $4\times4$, $4\times4$, and $4\times4$ at stages 1, 2, 3, and 4 of U-Net encoder, respectively, within the U-Net network which utilizes multi-scale features. This strategy replaces the shifted window technique in Swin-Attention~\cite{swin}. Within the Scaled Autoregressive Window Temporal Attention (SAWT-Attn) layer (illustrated in Figure~\ref{fig:tempattn}), we extend vanilla temporal attention into spatial space through window attention for each spatial area with the local window of size $m \times m$, alongside the global video frame sequence $n$ across this window. The SAWT-Attn layer conducts self-attention on this extended sequence with a causal mask, integrating both spatial and temporal dimensions and restricting the model from learning future temporal-spatial tokens.
%As it operates in both spatial and temporal spaces, frame generation attends not only to prior frames but also to adjacent spatial regions, resulting in high-fidelity generation performance.
%In this context, for a video clip with $n$ frames, each is projected into $n\times H/K \times W/K \times 4$ latent vectors. Here $n$ signifies the frame count, $K$ indicates downsample ratio of VAE encoder, $(H/K, W/K)$ represents spatial dimensions, and $4$ corresponds to the number of latent channels.

\subsection{Frame Sequential Text Decomposer}\label{sec:fstext}
For the language conditioning module, since our 3D inflated U-Net is built upon a pretrained text-to-image model, we noticed that using a text-to-image prior alongside a global instruction tends to provide strong semantic guidance, which can override the scene in reference frames, deviating from the intended guidance for prediction based on the existing scenes.
To address the above limitation and better capture long-term dependencies from both text and reference frames, we introduce the Frame Sequential Text (FSText) Decomposer. This novel approach decomposes the global instruction into fine-grained sub-instructions, aligning with each frame. We further explore the interpretability of sub-instruction embeddings in Section~\ref{sec:results:subins}.
%the existing methods~\citep{magicvideo,makeavideo,tuneavideo} simply encode a single text embedding for the whole video with a CLIP text encoder~\cite{radford2021clip}.However, since text instructions often pertain to the overall task, understanding progress at each time step becomes challenging with a global instruction embedding
\begin{figure*}
\centering
\vspace{-30pt}
\includegraphics[width=0.9\linewidth]{fig/seq_text_transformer.pdf}
\vspace{-10pt}
\caption{Frame Sequential Text Decomposer is shown in (a). We start by initializing the weight of the network to project identity vectors from CLIP text tokens. We then optimize the generated text tokens via the diffusion process (b), where frame-individual cross-attention is denoted by ``fic-attn."}
\label{fig:fseq}
\vspace{-10pt}
\end{figure*}
To derive a sequence of temporally aligned sub-instruction embeddings from the global instruction generated by the CLIP text encoder~\cite{radford2021clip}, we employ a transformer-based temporal network designed to fulfill three essential properties for meaningful sub-instructions: \textbf{1)} Contextual aggregation, which ensures that the inner tokens of each sub-instruction aggregate contextual information within the sentence. \textbf{2)} Semantic inheritance, the semantic information of these sub-instructions is inherited directly from the global instruction \textbf{3)} Temporal consistency ensures alignment between the sub-instructions and the time sequence, thereby facilitating the generation of temporally consistent video. Based on these properties,  our network consists of  three key components: \textbf{a)} To achieve the property of contextual aggregation, we employ Text-Sequential Attention, akin to BERT, a bidirectional self-attention layer~\citep{bert} to capture global dependencies among different positions within text sentences. \textbf{b)} To ensure semantic inheritance, we use Cross-Attention, responsible for projecting the global instruction's textual sequence onto the inner tokens of each sub-instruction, this component ensures that all sub-instructions contain essential global instruction signals for guiding video frame generation. \textbf{c)} To maintain temporal consistency, we adopt temporal Attention, a directed attention layer to capture temporal dependencies along the frame axis, which enhances temporal consistency among the generated sub-instructions throughout the video.
\begin{wrapfigure}[10]{r}{0.4\textwidth}
\centering
\vspace{-10pt}
\includegraphics[width=0.9\linewidth]{fig/fstext_module.pdf}
\vspace{-10pt}
\caption{The FSText attention of sub-instruction tokens.}
\label{fig:fstextpipline}
\end{wrapfigure}
Specifically, as shown in Figure~\ref{fig:fstextpipline}, we start with a global CLIP text embedding, denoted as $(l, C)$, where $l$ signifies the text sentence length and $C$ is the channel size, we initialize learnable tokens with shape $(n, l, C)$ where $n$ denotes the number of frames. The tokens are fed into the text sequential attention layer to perform self-attention along the $l$ axis. Subsequently, the cross-attention layer employs these learnable tokens as queries and the global text embedding as keys and values, resulting in a one-to-multiple projection from the global text into $n$ time steps. This yields $(n, l, C)$ tokens for $n$ frame containing task instruction information. Finally, the temporal attention layer conducts directed attention along the $n$ axis for each token in the textual sequence, transforming the macro-instruction progress into frame-specific guidance.

After getting $n$ sub-instruction embeddings corresponding to each frame, the next step is to inject this guidance into the diffusion process, which is commonly completed by a cross-attention layer. As shown in Figure~\ref{fig:fseq} (b), different from the existing works~\citep{magicvideo,tuneavideo} that calculate the cross-attention between the global instruction embedding and $n$ frames. In our cross-attention layer, where cross-attention is calculated separately between visual latent vectors and sub-instruction embeddings for each frame, and the results from all frames are then concatenated, an attention mechanism we refer to as frame-individual cross-attention (\textit{fic-attn}).

\paragraph{Initialization~~} We find initialization is critical to FSText decomposer. Especially, the random initialization fails to approximate the distribution of text embeddings in the pretrained T2I model and results in poor performance. To guarantee the sub-instruction embeddings become a close approximation of the CLIP text embedding, we employ an initialization strategy by enforcing the FSText decomposer to be an identity function (Note that this initialization step is completed before the diffusion process. We ablate this design in Section~\ref{sec:ablate:fstext}). It can be achieved by this objective:
\begin{equation}
    L_{\mathrm{identity}} = \|f_\theta(\bm{\tau}(\mathbf{y})) - \bm{\tau}(\mathbf{y})\|^2
\end{equation}

%\subsection{Inflated 3D U-Net with Autoregressive Spatial-Temporal Attention}\label{sec:tempoal}
%We inflate the Text-to-Image (T2I) pre-trained 2D U-Net to our Inflated 3D U-Net as illustrated in Figure~\ref{fig:pipeline} (b). A standard 2D U-Net block of LDMs consists of a series of 2D ResNet blocks and Spatial Attention Blocks including spatial self-attention and cross-attention modules. Similar to~\cite{vdm}, we replace the $3 \times 3$ 2D convolution kernel with a $1 \times 3 \times 3$ 3D convolution kernel with an additional axis of video frames. Additionally, to further boost the performance of capturing the inter-frame dependency, we incorporate temporal attention after every spatial cross-attention layer. In Figure~\ref{fig:tempattn}, we explore various types of temporal attention, including: (1) bi-directional temporal attention~\cite{vdm,makeavideo,imagenvideo}, which employs a full self-attention across all tokens along the temporal dimension; (2) directed temporal attention~\cite{magicvideo}, which uses a masked attention mechanism that follows the direction of the video sequence along the temporal dimension; and (3) autoregressive spatial-temporal attention: a novel technique proposed by us, which uses causal attention to autoregressively generates the frames on both spatial and temporal dimensions by flattening the tokens into a long sequence.

%We empirically observe that the two existing temporal attention layers cannot achieve promising performance on the TVP task. Bi-directional temporal attention tends to neglect the visual content guidance of the reference frames during the generation process (see Section~\ref{sec:ablate:temp}). 
%And the directed temporal attention fails to capture the dependency of nearby spatial regions and thus generates low-quality frames, while it adheres to the temporal sequence constraint.

%To handle the limitations of bi-directional and directed temporal attention, we introduce the Autoregressive Spatial-Temporal Attention (AST-Attn) mechanism shown in Figure~\ref{fig:tempattn}.
%Given $n$ frames video, a video clip is projected into $n\times s$ latent vectors (where $s$ is the length of a latent vector in each frame) by the pre-trained VAE encoder. We flatten the latent vectors of both temporal and spatial dimensions ($n\times s$) into one dimension. 
%Then, AST-Attn performs self-attention on this long sequence with a causal mask that prevents the model from learning from future temporal-spatial tokens. Because it performs in both spatial and temporal spaces, the frame generation will attend to not only the previous frames but also the nearby spatial regions, which results in high-fidelity generation performance.
%While the calculation of Autoregressive Spatial-Temporal Attention (AST-Attn) involves both temporal and spatial dimensions, its computational complexity remains manageable due to the design of the Inflated 3D U-Net, which maintains the complexity of spatial compression rates and channel depths within a controllable range. Specifically, in the AST-Attn layers of Inflated 3D U-Net, higher-resolution features have more spatial tokens but are computed in lower embedding dimensions, while lower-resolution features have fewer spatial tokens but are computed in higher embedding dimensions. In practice, we observe that adopting AST-Attn has only a $0.4\%$ computation speed lag compared to directed and bidirectional temporal attention.

%By incorporating Autoregressive Spatial-Temporal Attention in the Inflated 3D U-Net, we can generate high-fidelity and coherent video frames with minimal fine-tuning. Specifically, we merely fine-tune the proposed autoregressive spatial-temporal attention layers and freeze the rest of the pre-trained layers in our Inflated 3D U-Net. 
\iffalse
\begin{figure}
\centering
\includegraphics[width=1.0\linewidth]{fig/inflate_unet.pdf}
\caption{The overview of inflated 3D U-Net, we inflate the pre-trained T2I latent diffusion model (LDM) by expanding the 2D Conv kernel to 3D kernels and connecting the cross-attention layer with the trainable causal temporal attention layer.}
\label{fig:inflate3d}
\end{figure}
\fi




\section{Study}
\label{sec:study}
We conducted a within-subjects experiment with 14 participants to explore and understand the differences and similarities between the three presented designs for vibrotactile feedback (independent variable) regarding their effectiveness in communicating 3D directional cues. As participants were supposed to feel and comprehend directional cues without any additional visual feedback, we conducted the study in person and within a neutral VR environment, which allowed participants to focus entirely on the vibrotactile feedback. The age of participants ranged from 21 to 31 years, with a mean age of 25.71 years ($M = 25.71, SD = 2.972$). Four were female, ten were male, and all were university  students of various subjects. None of the participants reported any visual impairment, and all were right-handed.

\subsection{Procedure}
The study was conducted in multiple comparable physical localities. Before commencing, participants were fully informed about the project objective and the various tasks they had to complete. 
Each participant gave their full and informed consent to partake in the study, have video and audio recordings taken, and have all the relevant data documented. 
Participants wore a \ac{HMD} on their head and a vibrotactile glove on the right hand while \change{being asked to keep} their right arm rested on an armrest with the palm facing down (see Figure~\ref{fig:setup}) \change{to avoid any external factors}. 
In the left hand, participants held a controller to control the \ac{VR} environment. 

For each condition, each participant performed six training trials. For each trial, the vibrotactile feedback was repeated three times, and a corresponding visualization was shown to indicate the direction in 3D \change{supporting the participant's mental model}. 
For the actual task, \change{participants were shown a neutral-colored background in VR without any visual representations of the 3D direction. P}articipants were able to trigger the start of the trial with the \ac{VR} motion controller. In total, they completed 30 measured trials per condition, resulting in 90 measured trials per participant and 1,260 measured trials in total. The 30 trials consisted of 2 (blocks) x 5 (2D direction) x 3 (gradient). The variable \emph{2D direction} represented a typical set of five possible mappings of straight horizontal, vertical and diagonal directions, which were physically located on the surface of the hand (see Figure~\ref{fig:directions}). They represented the direction in x-z-coordinates of the overall 3D directional vector. The \change{\emph{gradient}} encoded the direction in y-coordinates: either up, down, or neither any gradient. To counter learning and fatigue effects, we applied a \emph{Balanced Latin Square} design for the order of the three conditions. The order of trials was randomized within each block. \change{Between each condition, participants were able to rest their hand for five minutes.} The average session lasted for 45 minutes and concluded with a debriefing. \change{Non of the participants mentioned any sensory or muscle fatigue.} Participants received 15 EUR in compensation.

\begin{figure*}
    \centering
    \captionsetup{justification=centering}
        \subfloat[2D Direction Estimation.]{\includegraphics[width=0.32\linewidth]{figures/plot_estimation_2ddirection.pdf}
        \label{fig:boxplots:direction}}
            \hfill
        \subfloat[Gradient Estimation.]{\includegraphics[width=0.32\linewidth]{figures/plot_estimation_gradient.pdf}
        \label{fig:boxplots:gradient}}
            \hfill
        \subfloat[Task Load.]{\includegraphics[width=0.32\linewidth]{figures/plot_taskload.pdf}
        \label{fig:boxplots:taskload}}
            \hfill
    \captionsetup{justification=justified}
    \vspace{-0.5em}
    \caption{Measured performance for 2D direction and gradient estimation as well as task load measured with the NASA Raw-TLX (lower score is better). For the task load subscale \enquote{frustration} no bars are visible because all three conditions have a median score of 0.}
    \Description{Three figures in a row, illustrating the results of "Comparison of 2D Direction Estimation" (boxplot), "Comparison of Gradient Estimation" (boxplot), and "Comparison of NASA Raw-TLX" (bar graph) for each condition. Figure 3a: Comparison of 2D Direction Estimation (boxplot): Rabbit Dual=93.3\% (IQR=12.5\%), Rabbit Single=91.7\% (IQR=13.3\%), and Motion Intensity=78.3\% (IQR=16.7\%). Figure 3b: Comparison of Gradient Estimation (boxplot): Rabbit Dual=93.3\% (IQR=5.8\%), Rabbit Single=91.7\% (IQR=8.3\%), and Motion Intensity=56.7\% (IQR=10.0\%). Figure 3c: Comparison of NASA Raw-TLX for the sub-scores Mental, Physical, Temporal, Effort, Performance, and Frustration (bar graph).}
    \label{fig:boxplots}
\end{figure*}

\subsection{Variables and Research Questions}
For dependent variables, we measured the accuracy of the comprehension of the \emph{2D direction} (x-axis, z-axis) and the \emph{gradient} (y-axis). \change{We are measuring the two variables (\emph{2D direction} and \emph{gradient}) separately, as commonly done within the research community (e.g., estimation of direction and distance for \ac{HMD}s~\cite{Gruenefeld2017}). The main reasoning here is that orientation in 3D space and especially describing directions in 3D can be challenging for participants and could negatively affect the validity of the measurements.} To do so, we presented participants with a \ac{UI} panel in VR after each trial. The panel showed five pictures with all 2D directions in a top-down view and, subsequently, three pictures of all gradients in a lateral view. Participants used the \ac{VR} controller to select the fitting representation for each. These two variables were measured with a binary outcome (correct, incorrect) and summarized as percentages of correctly identified outcomes across all trials per person and condition (ratio scale). In addition, we measured mental workload after each condition via the \ac{RTLX}~\cite{hart1988} and additional Likert-scale statements regarding the comprehensibility of the directional cue. We also collected qualitative feedback in a semi-structured interview after each condition as well as at the end of the study, which is when we also asked participants to rank the three conditions. 

As the study is exploratory in nature, we were interested in finding out more about the specific features of our three feedback conditions. In particular, we were interested in the following research questions:

\textbf{RQ1}: Do multiple encodings of gradient, as in the condition \conB, improve the comprehension of gradient information and reduce mental workload? 

\textbf{RQ2}: Does apparent movement, as in the condition \conC, improve comprehension of 2D direction\change{? We assume that to be the case} as the transition \change{by overlapping} of vibration between the actuators \change{may be easier to comprehend and interpret as a path} compared to the sensation of \change{isolated pulses as for the \emph{Rabbit} conditions.} 

\textbf{RQ3}: How would participants experience and rate vibrotactile communication of 3D direction overall and with regard to each individual condition?



\section{Results}
\label{sec:results}

For our applied inferential statistics, we distinguished between ratio and ordinal data. 
The estimation percentages for 2D direction and gradient are ratio data, while the Likert items -- including task load -- are ordinal data.
For ratio data only, we first applied a Shapiro-Wilk test to check for normality.
We found that none of our ratio data is normally distributed.
Thus, we treated all our data in the same way and directly applied non-parametric tests, specifically Friedman tests.
Thereafter, we conducted Wilcoxon Signed-rank tests with Bonferroni correction for our post-hoc analysis.
The effect sizes of the Wilcoxon tests are reported as r (r: $>$0.1 small, $>$0.3 medium, and $>$0.5 large effect).

\subsection{Estimation of 2D Direction}
We asked participants to estimate the two-dimensional direction on a ground plane.
The median (interquartile range) percentages of correct 2D direction estimations for each condition are (in descending order): \conB=93.3\% (IQR=12.5\%), \conA=91.7\% (IQR=13.3\%), and \conC=78.3\% (IQR=16.7\%). 
All percentages are compared in Figure~\ref{fig:boxplots:direction}.
Since our data is not normally distributed (p$<$0.01), we directly ran a Friedman test that revealed a significant effect of condition on 2D direction estimation ($\chi^2$(2)=17.70, p$<$0.001, N=14). 
Post-hoc tests showed significant differences between \conA~and \conC~(W=83, Z=2.62, p$=$0.018, r=0.50) as well as \conB~and \conC~(W=0, Z=-3.30, p$<$0.001, r=0.62).
However, we did not find a significant difference between \conA~and \conB~(W=15, Z=-1.88, p$=$0.182).
Here, \textbf{we can conclude that both \conA and \conB~result in better estimation performance for 2D direction than \conC.}

\begin{table*}
\caption{Pairwise comparisons for individual statements, Bonferroni-adjusted, p-values: $<$0.05 (*), $<$0.01 (**), and $<$0.001 (***).}
\label{tab:pairwise_statements}
\begin{tabular}{l|ll|lll|lll}
 & \multicolumn{2}{l|}{Rabbit Single vs. Dual} & \multicolumn{3}{l|}{Rabbit Single vs. Motion Intensity} & \multicolumn{3}{l}{Rabbit Dual vs. Motion Intensity} \\
Statement & test statistic & p-value & test statistic & p-value & effect size & test statistic & p-value & effect size \\ \hline
S1 & Z=~0.00 & p=1.000 & Z=2.89 & p\textless{}.001\textbf{***} & r=0.55 & Z=3.04 & p=.004\textbf{**} & r=0.57 \\
S2 & Z=-0.07 & p=1.000 & Z=2.58 & p=.029\textbf{*} & r=0.49 & Z=2.80 & p=.013\textbf{*} & r=0.53 \\
S3 & Z=-1.17 & p=0.838 & Z=3.06 & p=.003\textbf{**} & r=0.58 & Z=2.97 & p=.003\textbf{**} & r=0.56 \\
S4 & Z=-0.17 & p=1.000 & Z=2.84 & p=.006\textbf{**} & r=0.54 & Z=2.69 & p=.018\textbf{*} & r=0.51
\end{tabular}
\end{table*}



\subsection{Estimation of Gradient}
We asked participants to estimate the gradient behavior of the communicated cue.
The median (interquartile range) percentages of correct gradient estimations for each condition are (in descending order): \conB=93.3\% (IQR=5.8\%), \conA=91.7\% (IQR=8.3\%), and \conC=56.7\% (IQR=10.0\%). 
All percentages are compared in Figure~\ref{fig:boxplots:gradient}.
Since our data is not normally distributed (p$<$0.001), we ran a Friedman test that revealed a significant effect of condition on gradient estimation ($\chi^2$(2)=19.00, p$<$0.001, N=14). 
Post-hoc tests showed significant differences between \conA~and \conC~(W=102, Z=3.11, p$=$0.002, r=0.59) as well as \conB~and \conC~(W=0, Z=-3.30, p$<$0.001, r=0.62).
However, we did not find a significant difference between \conA~and \conB~(W=30, Z=-1.42, p$=$0.501).
Here, \textbf{we can conclude that both \conA and \conB~result in better gradient estimation performance than \conC.}


\begin{figure*}
    \centering
    \includegraphics[width=\linewidth]{figures/plot_likert.pdf}
    \captionsetup{justification=justified}
    \vspace{-1em}
    \caption{Participant responses to the four rated statements (Likert items ranging from 1: strongly disagree to 7: strongly agree).}
    \Description{The results (four figures in two rows with two figures per row) of the participant' statements "Directions Easy to Distinguish", "Directions Easy to Derive From Vibration", "Change of Gradient Easy to Perceive", and "Easier to Perceive Over Time", visualized in a stacked bar plot for Likert items.}
    \label{fig:likert}
\end{figure*}

\subsection{Task Load}
The results of task load ratings as measured by the \ac{RTLX}~\cite{hart1988} are shown in Figure~\ref{fig:boxplots:taskload}. 
The median (interquartile range) task load scores for each condition are (in ascending order): \conA=22.5 (IQR=12.7), \conB=24.5 (IQR=7.9), and \conC=28.3 (IQR=20.0).
We ran a Friedman test that revealed a significant effect of condition on task load ($\chi^2$(2)=13.50, p$=$0.001, N=14).
Post-hoc tests showed a significant difference between \conB~and \conC~(W=105, Z=3.30, p$<$0.001, r=0.62).
However, we did not find any significant differences between \conA~and \conB~(W=39, Z=0.00, p$=$1.000) or between \conA~and \conC~(W=20, Z=-2.04, p=0.120).
Here, \textbf{we can conclude that \conB~induces a lower task load than \conC.}

\subsection{Individual Statements and Preferences}
After each condition, we asked participants to rate four statements, each on a 7-point Likert scale (1=strongly disagree, 7=strongly agree). The results and statements are shown in \autoref{fig:likert}. We found significant main effects for all four statements (N=14; S1: $\chi^2$(2)=14.09, p$<$0.001; S2: $\chi^2$(2)=11.35, p$=$0.003; S3: $\chi^2$(2)=17.08, p$<$0.001; S4:$\chi^2$(2)=12.79, p$=$0.002). Pairwise comparisons are shown in \autoref{tab:pairwise_statements}. Here, \textbf{we can conclude that \conA and \conB~are rated significantly more positively than \conC~for all four statements}. No difference was found between \conA and \conB. Regarding \emph{overall preference}, \textbf{eight participants preferred \conB}, while \textbf{six voted for \conA}~as their favorite. None of the participants preferred \conC.


\subsection{Interviews}
During the interviews, participants were explicitly asked to comment on the duration of the vibration as well as what may have eased or hindered their comprehension. They also had to explain their overall preference and comment on the overall experience and sensation of interpreting 3D directional cues via vibrotactile feedback.
For the analysis, the verbal data was first transcribed by one author and then summarized. The statements were then counted for each question. In addition, across all questions, we applied open coding to identify hidden themes. Data from one interview (P2) was not recorded due to a technical issue. Therefore, only the data from 13 participants was included.

Regarding the duration of the vibration, both \emph{Rabbit} conditions were perceived as having adequate duration (\conA: 10 vs 3 who thought it could have been longer; \conB: 13:0), while 10 participants would have preferred a longer duration for \conC. For the latter, participants struggled to feel the gradient correctly, as mentioned by five participants (e.g., P7 said that the \enquote{[duration was] a little bit short, enough for [2D] direction, but for intensity [gradient] it was really bad.}) The varying strength of the vibration was also an issue, as the most distant control point was criticized as having a too weak vibration, which meant that \enquote{some vibrations got lost} (P5). This also interfered with the comprehension of 2D direction. The smooth transition of movement in \conC was still found to be a pleasant experience, but the mentioned drawbacks regarding the gradient detection prevailed, according to P4 (RQ2). When comparing pulse with intensity for the mapping of gradient, P12 noted an interesting further advantage of pulse, as \enquote{One could decide about the gradient in retrospect even if one wasn't sure before. When the last actuator vibrated many times, then it must have been an upwards gradient.} This also implicitly highlights the problem of immediacy, which required attention and did not allow repetition of the feedback. As P10 put it, \enquote{in case you did not fully pay attention, there wasn't a repeat to make sure.} This sentiment was echoed by P12. 
Consequently, the dual encoding of a gradient in the \conB~condition was cited by most as the main reason for preferring that condition (RQ1). P11 noted, \enquote{I did not just have the number of pulses, but in addition the intensity and that somehow better stuck in my head.}
\section{Discussion}
\label{sec:discussion}
Overall, responses during the interview and the quantitative data are in agreement. They show significant and substantial advantages of \conA and \conB compared to \conC, which we did not expect in such clarity. The parity between these two then is again visible from all angles, with preference being nearly balanced (8 vs. 6). Still, the interviews showed that for \conC, participants did like the smooth transition between the individual feedback factors, which however failed to have a measurable advantage (RQ2). A main reason for this may be that we found that the mechanisms to communicate 2D direction and gradient can interfere with one another. In particular, the intensity gradient mapping had a negative effect on the 2D direction mapping in \conC, as the minimum vibration sometimes \enquote{got lost} (P5), when users did not pay close attention. While pre-tests suggested otherwise, individual differences among the perception of our participants as well as potential fitting issues with the glove (see \emph{limitations} below) may have resulted in this issue. From the comments of participants, we have to assume that \conB was affected by this problem as well, although to a lesser degree; the simultaneous pulse mapping implicitly included repeated vibrations of the same actuator at least twice.

Our analysis also showed that the type of feedback may require more training for participants to get accustomed to. P13 summarized this point nicely: \enquote{Maybe if you market that [...] and someone develops a game for it [...], then I might like it, and in a year, no one can imagine a world without it.} Others noted the effort involved, with P10 saying they \enquote{found that it was really exhausting since you are not used to it.} 
 Still, the overall experience of using vibrotactile feedback to interpret 3D directional cues was described as \enquote{surprisingly good} (P7) and prompted many ideas for use cases, such as medical scenarios (operating table with limited visuals), people with visual impairments in daily life as well as when driving a bike or motorcycle.

\textbf{Limitations:} As an exploratory study, our results should be perceived as preliminary and require further testing and confirmation. In particular, our results may be limited due to the number of participants (14), which also led to the design not being fully balanced. \change{In addition, all participants in our study were right-handed, which could affect our results.} We also found that more training may be required to compensate for initial learning effects, as the type of feedback is so unusual and novel for participants. In addition, the \emph{SensorialXR} glove only \change{provides a fixed setting of the actuators and} offers a \enquote{one-size-fits-all} size, which showed to be problematic for some users with smaller hands, where the actuators were not always in tight contact with the skin. \change{For future research prototypes adding additional Velcro around the actuators might help.} 
\section{Conclusion}
\label{sec:conclusion}

This work aimed to explore different design approaches to communicate three-dimensional directional cues with vibrotactile feedback. We developed two conditions based on the \emph{Cutaneous Rabbit} illusion and one based on \emph{Apparent Tactile Motion} to communicate 2D direction. The gradient of the overall 3D direction was then encoded by the number of discrete vibration pulses, the vibration intensity, or a combination of both.
Our study showed that three-dimensional directional cues can be communicated by \conA and \conB with a high success rate for both the 2D direction and gradient (median for \conA: 91.7\%, \conB: 93.3\%) -- significantly better compared to \conC. With respect to our research questions, we found partial evidence for RQ1, as multiple participants specifically mentioned the dual mapping for gradient as a benefit. Still, for the quantitative data, both \emph{Rabbit} conditions performed more or less identical. RQ2 has to be dismissed at this point. However, as revealed by our qualitative analysis, we believe that the \emph{Apparent Tactile Motion} illusion can also be a viable option for future designs, as the smooth transition between actuators was appreciated by participants. The challenge will lie in overcoming the inferences we found between 2D directional and gradient intensity mapping.

\begin{figure}
    \centering
    \includegraphics[width=\linewidth]{figures/intendedMovement_both.png}
    \caption{\change{An assistive robotic arm with AI-created directional movement recommendations. The cyan arrow indicate the current movement direction of the arm, while the blue arrow shows the recommendation, which would be mapped as 3D directional cues on the glove. Note: The cyan and blue arrows are only for presentation purposes.}}
    \Description{An overview of a human-robot interaction scenario for future work. A robotic arm (Kinova Jaco) is mounted on a table. The robot is trying to grasp a blue object, and a visual cue (blue arrow) illustrates the intended movement direction toward the object. A cyan arrow points more down toward the table surface, missing the object.}
    \label{fig:future}
\end{figure}

\textbf{Future Research:}  In our work, we aim to apply this approach to communicate the intended movements~\cite{Pascher.2023robotMotionIntent} of a semi-autonomous robot in collaborative scenarios, where vision alone may not be sufficient to successfully predict robot motion. \change{In \autoref{fig:future} an assistive robot arm is illustrated, which is manually controlled by the user but is supported through an \ac{AI} which provides real time directional movement recommendations. Here, our approach could be used to map these directional movement recommendations as vibration input on the hand. Changes in the intensity of the actuators indicate the amount of directional change, thus enabling the user to better imagine the generated trajectory.}
We also encourage researchers to both replicate our design and study and apply it to different use cases. 
Future research should also investigate variables such as the effect of higher-resolution tactile displays\change{, different setting of actuators,} or other approaches to encode gradient (e.g. through different vibration frequencies, \change{varying linear and non-linear intensity levels}), which were not possible with the \emph{SensorialXR} technology.
\change{Furthermore, results of our study should also be evaluated with participants with a dominant left hand or their non-dominant hand.}

%%
%% The next two lines define the bibliography style to be used, and
%% the bibliography file.
\bibliographystyle{ACM-Reference-Format}
\bibliography{bibliography}

\end{document}
\endinput
%%
%% End of file `sample-authordraft.tex'.
