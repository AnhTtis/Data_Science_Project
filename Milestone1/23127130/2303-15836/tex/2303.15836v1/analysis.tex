\section{Real-world Data Analysis}
Fig.~\ref{fig:arch} shows the deployment of our framework in a vehicular cloud context, where cars provide their onboard resources and UEs request \mec{} service execution. As mentioned in section \af{Add reference to design section}, the \mechost{} subscribes to an AoI (e.g., parking lot) to be aware of available resources in that area.  In this context, parked cars may leave the parking lot once their resources have been allocated, thus generating service interruption.To deal with this problem, the MEC-H starts the migration procedure by transferring all the information needed in the new location to restore the execution. However, in the case of stateful \mec~apps, migrations generate a downtime period in which the app involved becomes unavailable. Hence, to measure this metric we created a scenario where vehicles enter and leave a parking lot, and UEs request edge applications during day hours. To make this scenario as close as possible to reality, we based our study on two real-world datasets. The dataset in~\cite{arnhemdataset} provides car transactions of three parking garages in the city of Arnhem. It reports information about cars entering and leaving times of each parking lot. On the other hand, the dataset in~\cite{bolognadataset} describes the usage of public WiFi networks in the city of Bologna. It records the number of users joining the WiFi network each day. We analyzed each parking garage and each network available to find a relationship between these datasets. Thus, we selected one parking garage context and location and filtered Bologna's dataset using this information. The location chosen is the \textit{Ceentral garage}, a parking area close to the train station of Arnhem. Therefore, we extracted data related to WiFi network usage within the train station of Bologna. Once we gathered these data, we studied the distributions of the occupancy time, i.e., the time a car remains parked, the average number of cars entering per hour, and the average number of UEs joining the network per hour. Based on the fitted distribution found for the occupancy time, we decided to approximate it with a normal distribution with mean $\mu=202.80$ minutes and standard deviation $\sigma=135.07$. Other metrics, i.e., the number of cars and UEs joining the area per hour, are described by Poisson distributions as these events are independent of each other and occur randomly within a hour but with a known mean rate.

We used the distributions obtained to feed an event-based simulator that generates cars and UEs in a 5G network. The deployment of this scenario allows us to analyze the trend of migrations needed during the day, to evaluate the performance of our framework during migration caused by cars leaving a parking lot. \af{We give evaluation details in the next section}
% In this environment, each UE generates a \mec{} compliant request for \mec{} app execution. Thus, 

% \af{Cite this \cite{6143927}}