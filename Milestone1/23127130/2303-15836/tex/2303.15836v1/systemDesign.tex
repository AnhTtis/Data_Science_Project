
\section{Our Proposed MEC Extension for the Dynamic Exploitation of Neighbor Vehicular Resources}\label{sec:sysdes}

%Armir(0) broker centralizato vs. distribuito (cloud-edge)
Our work stems from the observation that \mechost{}s (i.e. edge nodes) have limited capabilities when compared to cloud-backed ones. %and it is the entity that provides resources to run \mecapp{}s on top of the \vi{}. 
The proposal turns the \mechost{} into a logical entity that can leverage multiple \vi{}s, dynamically adding and removing localized computational resources.


% VIM exaplanation
Referring to Fig.~\ref{fig:extension_architecture}, our proposal allows the inclusion of the far-edge (vehicular) layer in cloud continuum deployment environments. Thus, in addition to locally defined hosts (edge nodes), it involves also remote host resources that are reachable and added dynamically via the RAN (e.g., 5G RAN). The approach entails some changes in the \mec{} traditional architecture in terms of structure and interactions (steps \textcircled{1}-\textcircled{6} in Fig.~\ref{fig:extension_architecture}). The main component to be affected is the \vim{} that is in charge of administering the host resources. In our design, it handles a heterogeneous pool of distributed resources and is aware of the single contributions that each host brings in terms of capacity. In this context, it is desirable to differentiate between infrastructure resources and the more transient ones, dynamically joining the \mechost{} thanks to vehicle availability in the neighborhood.

% Protocol description - external entity
%Armir (1): service discovery | network or vehicule-initiated (assisted).
% FLOW OF THE SECTION
%   1) Description of the main element: a mechanism to handle the join and leave of cars
%   2) How we designed it? Introduction of AoI as high-level abstraction - we use base station converage
%   3) Introduction of a new entity: the broker
%   4) How the broker is involved? Car discovery vs vehicle initiated - explain
%   5) what we decided (?)
%   6) Description of the joining protocol
%%%%
% writing...
%%%%
% Broker description - a mechanism to handle resources joining and leaving
Our architecture proposal implies the creation of a mechanism to handle far-edge resources joining and leaving the \mechost{} resource pool. Thus, each \mechost{} defines an \aoi{} within which far-edge nodes (hosts) can decide whether to provide their onboard resources. To model this mechanism, we decided to involve a new external component in the core network, named \server{}, which handles the resource pooling of several \mechost{}s. To this end, the \server{} relies on a publish-subscribe model to collect \mechost{} \aoi{} subscriptions and manage notifications whenever new devices enter the area. The \aoi{} may depend on where \mechost{}s are located, i.e., at the network edge or network aggregation points~\cite{kekki2018mec}. 
%Armir: what does decoupled model mean? - pub/sub model
% However, thanks to the decoupled model employed, our architecture supports the opportunity to dynamically aggregate far-edge resources and create \mec{} compliant node in any area within network coverage.
% PROCEDURE DESCRIPTION:
% MEC-HOST SUB
% RESOURCE SUB
% RESOURCE RELEASE
% MEC-host subscription
Hence, each \mechost{}, at bootstrap time, subscribes to an \aoi{} that might coincide with one or more zones, which typically correspond to the coverage of the associated gNBs.

%The joining procedure may be initiated in two ways: \server{} asking each entering device its available resources by encouraging them throug; devices entering the \aoi{}, notifying its presence, and asks the \server{} for a potential reward.


% Joining of new resources
% Finally, a device entering an \aoi{} and deciding to provide its resources triggers the notification chain that allows increasing \mechost{} computational capacity.
A reward system encourages far-edge nodes to lease their computational capacity and join the resource pool. Two different schemes of the procedure are envisioned: network- or vehicle-initiated. The former requires the \server{} to provide each far-edge node in the \aoi{} a set of rewards to incentivise resource leasing. The latter (steps \textcircled{1}-\textcircled{3} ) expects far-edge nodes to manifest their intention to join the pool, by asking for available rewards contextualized to the \aoi{}. In both cases, if the device finds acceptable terms, it can confirm the intent to participate by communicating to the \server{} the set of leased resources. The current design adopts the second approach, whereby vehicles obtain available rewards and can autonomously decide whether completing resource registration. Similarly, when one of them leaves the \aoi{}, it notifies the \server{}, which forwards the release request to the \vim{} managing that area. The latter in turn removes the concerned resources from those available in the pool. 

% Resource release
The resource release procedure (steps \textcircled{4}-\textcircled{6}) requires more attention as the departing host may have applications running on it. In such a case, after receiving the release notification, the \vim{} triggers the migration procedure (step \textcircled{6}) to move running apps from one host to another and thus maintaining service continuity with very low service interruptions. Both registered hosts in the \aoi{} and the local infrastructure of the \mechost{} are eligible to support the migration operation and embrace the new application(s).
The AMS, defined in the ETSI standard, currently supports app migration in environments encompassing multiple edge nodes. %This service is designed to support app mobility in environments with multiple edge nodes. 
Our extension has been designed to work in this perspective by extending the service to further support intra-host migration in a standard way. Furthermore, 
%in such extension, 
it is \mecpm{} that is identified as the main component that, during the procedure, acts as an intermediary node between the AMS and the \vim{} for new app allocations.
% A device (host) that enters the \aoi{} can request available rewards in order to decide whether to join the pool or not. 
 

%can decide whether to join the pool or not based on a reward system, which encourages the leasing of resources for devices that manifest their intention to join the pool.
%Armir(Opt.): se si da anche un idea vaga delle componenti coinvolte e' meglio..


%Armir(Opt): In the current implementation, we provide a basic monetary reward which is always accepted by the party, leasing in return all available resources. As a future extension, we envision to generalize the approach by introducing a mechanism implementing pluggable reward schemes.  - done
%A pub/sub messaging system is used to forward each request to the correct \vim{} and complete the resource registration.
% Similarly, when a far-edge device to leave the \aoi{}, it notifies the \server{}, which then forwards the release request to the \vim{} managing that area. 
%Armir(2): IMPORTANTE DIRE CHE FAR-EDGE RESOURCES ARE SHARED. OTHER APPS MIGHT PREEMPT THE EXECUTION OF OTHERS ---> HENCE MIGRATION NEEDED FOR THE FORMER APP.

Finally, concerning the resource allocation and scheduling approach, the \vim{}s initial selection results in a set of hosts eligible for app deployment; the scheduling phase, leading to the identification of a single host, can be done by ordering nodes depending on strategies that might favor certain aspects over others. Some metric examples are the average latency time between a host and the central infrastructure or the probability of a node to further contribute to the resource pool (i.e., based on historical data).

The proposed design approach paves the way for innovative application scenarios, which go beyond the state-of-the-art far-edge computing ones targeted nowadays, such as task offloading and content caching. For example, our proposal could be a key enabling element for the hosting of Federated Learning \cite{MLSYS2019} enabled environments at the far-edge layer. Specifically, a \mecapp{} (federated server) running on \mechost{} local resource infrastructure can coordinate other \mecapp s (federated clients) deployed on remote resources during the training on local data. The former can choose federated clients by using the model descriptors and information collected through the \mec{} standard API, while the latter, after receiving a federated model, can start the training procedure by relying on their status, local data, and received rewards.
%Just a note: our approach neglects security issues on how malicious nodes can affect mechost operational aspects}%
%Our extension moves in that direction, extending the service to also support intra-host migration in a standard way. \af{may be this first part can be cut} While in the general case it is the \orchestrator{} that handles the messages exchanged during the procedure, in our scenario the \mecpm{}  acts as an intermediary between the AMS and the VIM for new app allocations.%

%In the next section, we validated the proposed architecture by developing a simulation model that relies on parked vehicle resources to distribute \mec{} applications and service requests.
% The next section shows the validation of the architecture proposedby creating in a simulated environment and performance evaluation related to a concrete use case.
 \begin{figure}[b]
    \centering
    \includegraphics[width=\linewidth]{images/rtt_1.pdf}
    \caption{RTT variation with varying number of UE requests in the considered service delivery scenarios. The x-axis denotes the number of UEs requesting an app execution, while the y-axis denotes the average Round-Trip Time between UEs and the \mecapp{}. 
    In the far-edge delivery mode, \mecapp{}s requested by the UEs are deployed onboard the vehicle VI.}
    \label{fig:rtt}
    \vspace{-0.4cm}
\end{figure}
