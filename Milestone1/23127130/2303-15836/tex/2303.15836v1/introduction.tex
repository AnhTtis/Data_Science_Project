\section{Introduction}

%ETSI

% OLD INTRODUCTION
%During the last decade, rapid advancements in software, hardware, and communication technologies are enabling the design and implementation of several types of networks, by driving the dramatic growth of the number of Internet of Things (IoT) devices, which is expected to reach more than 29 billion in 2030~\cite{iotstatistics}. IoT devices enable a wide range of innovative applications and services, changing many aspects of our everyday living. Similarly, smart vehicles promote the deployment of next generation applications in road safety, traffic efficiency, and transportation industry. As the number of intelligent cars and sensors is constantly growing~\cite{sabellafullyconnected}, the amount of data generated may represent a serious problem for the backbone network. Furthermore, cloud-based solutions may not satisfy the strict requirements of time-sensitive applications. By shifting resources at the edge of the network while providing services and data preprocessing functions, the edge computing represents a valuable solution to deal with these challenges.
% On this front,
% the European Telecommunications Standards Institute (ETSI) initiated the standardization of a cloud platform located at the edge of Radio Access Network (RAN), named Multi-Access Edge Computing (\mec)~\cite{etsiwebsiterefarch}. 
% The standard architecture includes components that handle and orchestrate edge resources allocation and virtualization, manage \mec~applications' life-cycle, and provide standard \mec~services. It improves legacy radio units with cloud computing capabilities, thus representing a pillar technology for emerging radio networks (e.g., 5G)~\cite{kekki2018mec}.  Nevertheless, challenges depending on resource availability and service provisioning persist in contexts lacking \mec~infrastructure. Moreover, the computational power of edge nodes is limited. 

Thanks to the vast improvements in computing technologies and the pervasive deployment of next-generation communication networks, it is estimated that every new vehicle will be connected in the near future, embodying the potential of a fully-fledged mobile computing platform where vehicles serve as computation nodes for a diverse range of services. Different from vehicular networking~\cite{7513432}, which serves as a communication enabler for applications associated with transportation, vehicle computing focuses on the computation function and emphasizes the promising role it embodies towards the implementation of a pervasive context-aware computing environment. This computing environment can play a major role in future ICT systems for supporting applications like Intelligent Transportation Systems (ITS) and full-scale smart cities~\cite{WHAIDUZZAMAN2014325}.

Mobility-as-a-service and high-definition (HD) map generation are examples of such services, provisioned spanning cloud-to-vehicle resources where computationally heavy tasks are offloaded to resource-hungry cloud-based nodes. However, as the number of embedded sensors and smart vehicles grows, the amount of in-vehicle generated data may represent a serious problem for the infrastructure~\cite{aecc}. At the same time, cloud-based solutions are generally unfit to serve delay-sensitive application scenarios, which are currently served by shifting computation at the edge of the network while providing services and data preprocessing functions.

On this front, the European Telecommunications Standards Institute (ETSI) has worked on the standardization of a cloud platform co-located at the edge of the network, including the Radio Access Network (RAN), named Multi-Access Edge Computing (\mec)~\cite{etsiwebsiterefarch}. The standardized architecture includes functional components tasked with the management and orchestration of edge resources, managing \mec~applications' life-cycle, and providing standardized reference points to access the services. ETSI \mec{} augments legacy radio units with cloud-like computing capabilities, thus representing a pillar technology for (beyond) 5G cellular networks, allowing application deployment spanning cloud/edge resources~\cite{kekki2018mec}.  

New opportunities and challenges arise as a growing number of businesses start to exploit the shared edge-cloud environment. In contrast to traditional cloud deployment environments in data centers, the edge has limited resources and may not always be able to satisfy application demands for resources and associated QoS. Moreover, the technical challenges associated with advanced edge infrastructures are exacerbated by the convergence trends meant to make sure an end-user can access the whole range of subscribed services. % whatever device technology, wherever the user is connected to the network, and whether the user is in motion or not. %Recently, several research efforts highlighted as present-day vehicles include powerful resources that can be deployed as small data center~\cite{olariu2011taking, 7018198, 8936985,7415983}.
% For this reason, the need to identify new resources that can support the infrastructure dynamically when overloads happen has arisen. Today vehicles are equipped with a rich set of computing, data storage, communication, and sensor resources in their onboard computer unit. Based on current vehicular networking standards, cellular vehicle-to-everything (C-V2X) communication enables vehicles to wirelessly connect and cooperate with each other and with their surroundings, including road infrastructure, pedestrians, and so on. Thus, it is believed that vehicles will play a major role in future Information and Communication Systems (ICT) systems for supporting applications like Intelligent Transportation Systems (ITS) and full-scale smart cities. Because of these characteristics, vehicles can be considered true computational nodes capable of supporting the computing infrastructure.
For these reasons, the need has emerged to identify new resources that can support the edge infrastructure dynamically, thus enabling service availability in dense and congested deployment scenarios. Toward this end, some examples of applications that may benefit from such a dynamic scenario involve decentralized learning contexts, where opportunistic resources can improve the overall learning process in terms of data quality and training/prediction speed. Additionally, in a smart city context, the availability of more resources could enable the creation of edge-enabled digital twins to keep track of assets present in the city and execute computationally demanding simulation tasks.

%\af{To address the above problems, the Vehicular Cloud (VC)~\cite{olariu2011taking, 6257116} paradigm has emerged. It leverages the underutilized vehicular resources that vehicles can share and aggregate to be pooled as real edge nodes. The paradigm has a great potential to enhance edge nodes computational power and improve the overall service continuity and availability.}
% \ac{Introduction of vehicles at the beginning of the section}. 

% VEHICULAR CLOUD PROBLEMS
% In this sense, several research proposed models that exploit resources from parked and moving vehicles \cite{8936985}. For example, some authors~\cite{8522034, 9344808, 9366768, 9709120} proposed their usage for task-offloading, while others \cite{5935198,9475490} to improve connectivity in Vehicular Ad-hoc NETworks (VANETs).
% However, vehicular computing deployment is not trivial for many reasons. The number of vehicle resource providers can change over time, for instance, a vehicle may leave the vehicle-based cluster while running applications. Thus, there are cases where app migrations become unavoidable, and, although most of the works deal with this problem by trying to predict the time of vehicle cluster membership, it is not enough when a vehicle decides to no further continue with task execution. 
% Moreover, state-of-the-art deployments require ad-hoc architectures adopting paradigms that usually does not cope with challenges arising from a multi-vendor and multi-network operators environment.

%Moreover, it becomes hard to follow a model for vehicle resource deployment and easy service distribution\cite{8315203}, if authors deploy their solution in an ad-hoc architecture.

%  In this paper, we deal with the above challenges by proposing a novel ETSI \mec-compliant architecture that leverages far-edge device resources within an Area of Interest (\aoi). To handle node mobility, the architecture assists application migration by triggering the user context transfer in \mec~applications running on nodes leaving the \aoi.
Taking a step towards the implementation of the aforementioned pervasive computing environment, our proposal extends the MEC resource pool so to leverage far-edge device resources, exposed and made available in a standardized way. To this end, we propose a novel ETSI \mec-compliant architecture that can tap into far-edge resources within an Area of Interest (\aoi). Node resources are registered at the edge resource pool, exposed and accessed via standardized interfaces. To handle node mobility, the architecture assists the application migration by triggering the user context transfer to \mec~applications running on nodes leaving the \aoi. To demonstrate the viability of our proposal, we have developed a simulation model, readily available for researchers in \cite{githubrepo}, that extends the \mec~infrastructure with vehicular onboard resources. Our simulation model supports \mec~application deployment on vehicle resources and manages their volatility as vehicles enter and leave the AoI. We evaluate the proposed approach by benefiting from two real-world datasets (about user mobility and parking lot occupancy) where there is the need to manage service migrations triggered by vehicle mobility. Finally, let us highlight that this work represents the first effort at enabling the deployment of vehicular resources in a multi-vendor and multi-operator scenario, through the exploitation of (the original extension of) an established ETSI standard.

% To demonstrate the viability of our proposal in 5G scenarios, we developed a simulation model that extends the \mec~infrastructure through parked vehicles onboard resources. The model supports \mec~application deployment on vehicle resources, and manages their volatility as vehicles enter and leave the parking area. In this sense, the work analyzed two real-world datasets to provide an in-depth study of service migrations generated by vehicle mobility. Moreover, this work represents the first effort at enabling the deployment of vehicular resources in a multi-vendor and multi-operator scenario, as it relies on an extension of an established standard. 

%Additionally, this solution is one of the few proposals that perform app migration and the only one that makes use of standard APIs. 
% \af{Add the fact that we are providing an analysis}
% The paper provides extensive simulations to demonstrate the feasibility and the performance of our framework in 5G contexts. Moreover, we analyzed two real-world datasets to provide an in-depth study of cars and User Equipment (UE) behaviors. This study allowed us to build the first environment tracing service migration generated by cars mobility and to test system performance during migration.
