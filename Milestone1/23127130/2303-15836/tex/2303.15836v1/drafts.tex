\subsection{OMNET ++}
\omnetpp{}\cite{omnetwebsite} is a discrete event simulator framework based on message exchange and used primarily to model and build simulations for generic purposes. However, its highly modular architecture has allowed it to gain popularity within the scientific community as a network simulation tool and, to date, it is supported by several third-party projects that continues to extend it with new models.
The modularity of \omnetpp{} comes from the presence of a programming system based on components called modules that, if properly designed, can be reused in an easy and effective way. The basic modules are known as simple modules; they represent the active elements underlying a simulation and, with each one, there is an associated user-written C++ script that defines its behavior. Simple modules can be assembled to create new modules, called compounds, up to form a hierarchical structure that represents the final model or network of the simulation. Unlike the former, compound modules are defined only in terms of structure through a proprietary declarative language called Network Description (\ned{}) that describes the submodules, parameters, and interconnections. Each module can communicate with others through the exchange of messages that can contain arbitrarily complex data structures. Messages always have their origin and destination in a simple module, where processing takes place, and travel within the hierarchical structure through entry points called gates. Gates can represent either an entry point (input gate) or an exit point (output gate) for a module's messages and are connected respecting the network hierarchy, whereby a module's gates can only be connected to modules at the same level of the hierarchy or to its direct children. The discrete event simulator feature derives from the fact that the state of the simulation changes only as events occur, which in this case corresponds to the arrival of messages. It follows that the simulation time advances only at the reception of a message, and therefore \omnetpp{}+ neglects any processing time, i.e., the time to execute the code of the simple modules, allowing it to collect metrics and rely solely on the communication time between modules.
Finally, the \omnetpp{} framework also exploits events to implement a statistics logging system, which is useful for defining custom metrics that can be collected during the execution of a simulation, making the job of results assessment easier.

\subsection{Simu5G}
Simu5G \cite{simu5g_1} is a simulation library for the \omnetpp{} framework containing a collection of models and components useful for creating arbitrarily complex end-to-end scenarios involving 5G technology. Simu5G models both the core network and the RAN of a 5G network through the implementation of 3GPP-compliant protocols and a physical transmission system based on a set of customizable channels. The simulations created with this library can also leverage heterogeneous models related to gNBs antennas that use the X2 interface to support handover and inter-cell interference coordination. However, Simu5G can be used to analyze transition scenarios from 4G to 5G networks due to the ability to use dual connectivity between eNB and gNB antennas. Among the various implemented models Simu5G provides an implementation of the ETSI MEC standard with its main components (MEC Orchestrator, MEC host, etc.) and the ability to create applications that communicate via ETSI-compliant interfaces with other elements of the MEC ecosystem. Running applications can either be self-contained or take advantage of the presence of standard MEC services; currently, the framework implements the Location Service and the Radio Network Interface Service. A significant feature of the tool is that Simu5G can also be employed as a real-time emulator. It makes possible to arbitrarily replace simulation elements (5G network, MEC elements, etc.) with real devices and thus use the same code for simulation and prototyping. In addition, the use of physical components also allows for more realistic and reliable data collection.