%\begin{table}[htbp]
%\caption{Network settings \af{maybe we should remove the table to have more space. if we want to use it we should add UE parameters as well}}
%\begin{center}
%
%\begin{tabular}{|c|c|}
%\hline
%Parameter name            & Value  \\ \hline
%\#gNB                     & 1      \\ \hline
%gNB numerology index ($\mu$) & 2      \\ \hline
%Carrier frequency         & 2 GHz  \\ \hline
%gNB antenna Gain          & 18dBi  \\ \hline
%gNB tx Power              & 20 dBm \\ \hline
%\end{tabular}
%\end{center}
%\end{table}

\begin{figure*}[ht]
    \begin{subfigure}[b]{0.49\textwidth}
        \centering
        \includegraphics[width=\linewidth, height=0.55\linewidth]{images/migrations_final.pdf}
        \caption{Migration frequency}
        \label{fig:migrations}
    \end{subfigure}
    \begin{subfigure}[b]{0.49\textwidth}
        \centering
        \includegraphics[width=\linewidth, height=0.55\linewidth]{images/downtime_final.pdf}
        \caption{Downtime}
        \label{fig:downtime}
    \end{subfigure}
    \caption{Migration related metrics}
    \label{fig:evaluation}
    \vspace{-0.4cm}
\end{figure*}

\section{Evaluation Results}
We built a simulation model of the proposed ETSI MEC extension in the \omnetpp{} simulation tool, using Simu5G as a communication library, so to demonstrate the viability of our approach in (beyond) 5G scenarios. In the current release, available in \cite{githubrepo}, vehicular nodes are modeled and identified as 5G-enabled User Equipment (UE). The exposed vehicular resources consider the CPU in terms of instructions/second, RAM, and disk space available for the lease. Furthermore, the vehicle model includes a \mec{} \vi{} managing its onboard resources and applications' life-cycle and a module that manages rewards, resource registration, and releasing.

%Armir: 1-3 implementazioni delle design choices. 
% dobbiamo per ora parlare di parking area, perche' non sono stati fatti test in movimento.
At model startup, each \mechost{} identifies its \aoi{} corresponding to an area within the associated gNB coverage. Thus, when a vehicle enters the area, it starts interacting with the \server{}, according to the protocol described in Sec. \ref{sec:sysdes}. As a result, the vehicle joins the \mechost{} resource pool and can be considered by the VIM scheduling logic, which in the current implementation of our model, uses a simple Round Robin algorithm. Note that when vehicles leave the \aoi{}, corresponding app migrations might be triggered: we have decided to leverage only local MEC-H infrastructure as a target platform where to migrate \mecapp{}s.
%This behavior is parametric, and the control logic could be extended to implement more refined mechanisms. 
The rationale behind this choice is to avoid expensive and inefficient domino effects where a vehicle receiving a migrated app leaves the \aoi{}, thus triggering a new migration.

%Our simulation model considers only the local infrastructure as a backup for applications to migrate. In future versions, the model will support other methodologies (e.g., scheduling among other resources in the pool).
%Finally, our model uses the Round Robin algorithm to distribute \mecapp{} on parked cars.

% Il cloud-server non hosta nulla, e' solo usato per effettuare i test. La parte control della core network in Simu5G non e' implementata come componenti distribuite, ma e' centralizzata in un unico componente che ne simula comportamenti basilari. Simu5G permette di agire solo su user-plane
% The adopted 5G network model is the 
In our environment, we adopted a 5G standalone network, encompassing a single \mechost{} and a cloud datacenter to enable comparisons among multiple service delivery modes. The simulation scenario includes a parking area, where vehicles can enter and leave as they want. We currently provide a basic reward scheme, which is always accepted by vehicles partaking the resource acquisition procedure. 
%As a future extension, we envision to generalize the approach by introducing a mechanism supporting pluggable reward schemes.

% cooperative learning scenario?
% We first propose a simulation analysis used to validate the proposed design, comparing delay profiles of different service delivery modes. Next, we propose an innovative scenario, involving vehicular nodes partaking in a cooperative learning scenario, evidencing and profiling the application/state migration feature.  

%All the aforementioned experiments are conducted on 
We have conducted an extensive set of experiments on a Linux VM running \omnetpp{} having 16 CPUs and 64 GB of RAM. It is noteworthy to point out that according to the definition of discrete event simulator, \omnetpp{} does not consider the processing time spent to run the code of any module; hence, all the evaluations concern network-related delays. In the following, we evaluate our model and provide an experimental analysis to assess model performance under dynamic conditions.

\subsection{Model validation}
To validate the effectiveness of our model, we evaluate the network-induced delays in three service delivery schemes, namely at the cloud, edge of the network, and onboard the vehicle. In the cloud scheme, the \mecapp{}s are hosted on a simulated cloud datacenter, thus involving data transmissions spanning the 5G RAN, edge, and core network. The second service delivery scheme embodies delays introduced by running \mecapp{} directly on the \mechost{}, thus those due to 5G RAN and MEC local User Plane Function (UPF) co-located with the gNB \cite{kekki2018mec}. The last scheme involves \mecapp{} running on onboard vehicle resources, i.e., remote host in Fig. \ref{fig:extension_architecture}. In the latter case, network delays are affected by data transmission between the network (i.e., gNBs) and devices (i.e., vehicle running \mecapp{}s and UE requesting their execution). 

%Armir(new): la legenda del grafico mi confondo un'po... perche' evidenziato il 50-100-150-200? Nell'asse X il numero veicoli varie 25-350????  - Spiegato dopo
Figure \ref{fig:rtt} shows the Round Trip Time (RTT) of the above-mentioned schemes. On the x-axis, we considered the number of clients requesting the execution of a \mecapp{} to assess the behavior of the system under different load scenarios. In this experimental setup, we assume that each UE requests precisely the execution of a single \mecapp{} as soon as it enters the gNB coverage. In this scenario, UEs are spawned at the same time, simulating a flash-crowd phenomenon. Furthermore, to avoid any bias in the results, each experiment is repeated 5 times.

% Parked vehicle case - Legato al periodo di prima
%Armir(new): cosa significa "the third scheme has been split?"
The analysis of the third delivery scheme has been conducted by considering different vehicle quantities participating in the resource pool of the \mechost{} namely, 50-100-150-200 vehicles. In this scenario, the \mecapp{} execution triggered by the UEs requests is executed onboard the vehicle. This setting allows us to get further insight into network tolerance and relationships between delays and the number of apps deployed on a single vehicle. The figure shows how the latency times are steady even with more than 500 devices within the coverage of the \mechost{} associated gNB (note that computing-related delays are not considered in the reported simulations).
% , each service delivery mode RTT illustrated remains unchanged when increasing the number of UEs requesting for \mecapp{} execution.}

% Vantaggi parked vehicles
Deploying \mecapp{}s on vehicles brings the advantage of reducing the RTT by around 50\% as opposed to cloud deployment. On the other hand, the far-edge scheme increases the RTT by 3ms if compared with the edge mode, as it exploits fully wireless communications via the 5G network infrastructure. However, it should be noted that Simu5G handles device-to-device (D2D) communications through the gNB base station, thus employing a network-mediated communication model also in D2D scenarios. We expect that the adoption of \emph{sidelink} mode of 5G network would further reduce the RTT, as this enables direct communication between two devices without the participation of a gNB in data transmission and reception.

%\begin{figure}[h]
%    \centering
%    \includegraphics[width=\linewidth]{images/migrations_1.pdf}
%    \caption{Migration}
%    \label{fig:migrations}
%\end{figure}

\subsection{Migration Study}
As mentioned in Sec. \ref{sec:bg}, far-edge nodes may leave the resource pool once their resources have been allocated, resulting in a service disruption. To cope with this problem, the MEC-H initiates a migration procedure, moving running apps from the host leaving the resource pool into another one. However, migration of stateful \mecapp{}s may generate a downtime period in which the involved app becomes unavailable. 
% For instance, a \mecapp{} may involve the decentralized learning, where it coordinated the distribution of a model among several nodes. In such a scenario, when nodes leave the pool the partial results should be transferred to the coordinator \mecapp{}, who started the procedure. 

To measure the associated performance indicators, we build a realistic and innovative scenario involving vehicle volatility and UE activities in a parking area context.
% in the context of a parking area
%where vehicles enter and leave a parking lot, making themselves available for resource sharing, and UEs request for MEC-Application execution. 
To make this scenario as close as possible to reality, we need information regarding vehicle activity (i.e., vehicles joining and leaving an \aoi{}) and UEs activity exploiting a network connection in the surrounding area.
To the best of our knowledge, there are no datasets related to a specific area and collecting information similar to the scenario highlighted above. Thus, in order to recreate such an environment, 
we ground our study and make use of two real-world datasets~\cite{arnhemdataset,bolognadataset}. The first dataset captures vehicle join and leave times in three parking garages in the city of Arnhem, while the other one describes the usage of public WiFi networks in the city of Bologna, recording the number of users joining the WiFi network for each day hour. 

% To link the two dataset and create the described scenario...

To synthetize the aforementioned dynamics, we looked for a relationship between these datasets, analyzing each parking garage and each network available. Therefore, we selected one parking garage context and location, and filtered Bologna's dataset using this information. The location chosen is the \textit{Central garage}, a parking area able to host more than 1000 vehicles and close to the train station of Arnhem. Successively, we extracted data from WiFi networks within the train station of Bologna. After the pre-processing step, we studied the distributions of the occupancy time, i.e., the time a vehicle stays parked, the average number of vehicles entering per hour, and the average number of UEs joining the network per hour. Based on the fitted distribution found for the occupancy time, we decided to approximate it with a normal distribution with mean $\mu=202.80$ minutes and standard deviation $\sigma=135.07$. Other metrics, i.e., the number of entering vehicles and UEs per hour, are better captured by a Poisson distribution as they are independent events that occur randomly within an hour, but with a known average rate. 
% It is noteworthy, that the Poisson distribution describing the vehicle entrance in the parking lot, has been modeled by defining the mean rate based on the ratio between average number of entrance per hour and the park capacity.
%\begin{figure}[h]
%    \centering
%    \includegraphics[width=\linewidth]{images/downtime_1.pdf}
%    \caption{Downtime}
%    \label{fig:downtime}
%\end{figure}

We use the obtained distributions in our simulation model to spawn parked vehicles, hence the requesting UEs. This way, we build a dynamic scenario and simulate vehicles joining and leaving the AoI, and UE requests for \mecapp{} execution. The simulation settings are the same as in the prior scenario, i.e., each vehicle lends its onboard resources, and each UE issues a request to run a \mecapp{}. 
%Armir: sarebbe questa la parte dove si descrive la funct. app? e.g., federated learning?
The employed stateful \mecapp{} logic, provided by Simu5G library, allows the user to define a circular geographic area as a warning zone. The app is responsible for notifying the user whenever it enters and leaves that zone. More complex and meaningful location-based applications could be built, considering innovative scenarios such as decentralized (federated) learning etc.  

%Armir(new): spiegare la "park capacity". generiamo noi sintenticamente queste vehicle vs. abbiamo 4 parcheggi diversi che sn catturati nel dataset? Ce ambiguita. - DONE
Figure~\ref{fig:migrations} illustrates the number of migrations generated adopting the distributions described above. We considered four arbitrary parking lot capacities (i.e., 50, 100, 150, and 200) used to scale quantities generated by the Poisson distribution of the original dataset describing vehicles' entrances in the parking lot. As \mecapp{} s are equally distributed among the parked cars using Round Robin algorithm, lower park capacities may correspond to a greater number of migrations, for instance, during day time at 15:00 and 20:00. Note that the number of migrations also depends on the user activities and the amount of parking cars in that hour of the day, both generated through probability distributions. Thus, the relocations occurred may vary according to the number of users requesting for app execution and available car resources, which are not illustrated in this paper. Furthermore, we analyzed the interval from 13:00 to 21:00, as it corresponds to day hours with more network activity. 

Finally, the downtime has been measured by considering the elapsed time between the shutdown of the app running on the leaving host and the end-user receiving the new \mecapp{} location.
% the resuming of the migrated app on \mechost{}. 
Generally speaking, the downtime may be affected by latency between the two involved entities, i.e., remote and local \vi{}, and bandwidth~\cite{7399400}. In our experiments, the former includes the delays due to \mechost{} distance from the gNB and 5G radio delays, while the latter does not affect the service interruption time, as the state sent by \mecapp s is smaller than 30B.
%The scenario allows us to evaluate the \mec{}-App downtime in real-world conditions. 
%Armir: espandere di piu' il trend grafico! Bisogna essere piu' argomentativi... dando cifre, delta etc.
% Il trend del grafico e' stabile come gia' detto nelle righe qui sotto. Quello che ne emerge da questa analisi e' specificato enlle ultime due righe: il valore del migration time rimane costante nonostante entrambi numero di mitrazioni e carico della rete aumentino.
Hence, as shown in Figure~\ref{fig:downtime}, the downtime remains stable around 7 ms.%, with some peaks near 4 ms reached with lower park capacities.
Overall, despite the number of \mecapp{} relocations (Fig.\ref{fig:migrations}) and the network activity increase, the downtime remains stable when migrating \mecapp{}s from one host to another. %Hence, the results demonstrate the reliability of our solution in a dynamic and high-loaded real-world scenario.

%Armir: parte sotto e' piu' da conclusion. Commentiamo per ora.
%The survey provided in this section allowed us to replicate in our simulation model dynamic real-world conditions, where vehicles join and leave \mechost{} remote resource pool. Consequently, we managed to give a realistic evaluation of our simulation model downtime under these conditions. Furthermore, the presented survey represents the first effort in providing the community vehicular cloud behaviors in a real-world scenario. 
% \af{may be change the last sentence}


% In this environment, each UE generates a \mec{} compliant request for \mecapp{} execution. Thus, 

% \af{note that simulator does not simulate computational delays}

% \af{give some details about the mec app used}

% \af{Cite this \cite{6143927}}

