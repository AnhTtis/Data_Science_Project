\begin{figure*}[h]
  \centering
  \includegraphics[width=\textwidth]{images/MECExtendedArchitecture.pdf}
  \caption{\mec{} extension architecture diagram}
  \label{fig:extension_architecture}
    \vspace{-0.4cm}
\end{figure*}
\section{Background}\label{sec:bg}
% \label{section:backround}

% IMPORTANT NOTE:
% Here we refer to the standard architecture fig 6-1 {https://www.etsi.org/deliver/etsi_gs/MEC/001_099/003/03.01.01_60/gs_MEC003v030101p.pdf}, which does not introduce the concept of virtual network function as in its variant (section 6.2 linked document)
This section provides a brief description of the main functional elements of the ETSI \mec{} architecture used as the standard reference basis in our design, along with details about the simulation tools adopted to develop the extended model.

\subsection{ETSI MEC}
The \mec{} standard by ETSI proposes an architecture \cite{etsiwebsiterefarch} introducing cloud computing
capabilities at the very edge of the network, offering the ability to run so-called MEC Applications (\mecapp{}) within a virtualized and multi-tenant environment. The \mec{} architecture comprises two main parts: the system and host levels.
The system level represents the entry point for users to request service execution and it is usually deployed in the core network. The central element, the \mec{} Orchestrator (\orchestrator{}), has a view of the \mec{} Hosts (\mechost s) present in a particular area and is responsible for choosing the best one according to the requirements of the requested service, triggering both the instantiation and termination of applications. In this context, the User Application LifeCycle Management Proxy (\ualcmp{}) acts as an intermediary for the users by supporting instantiation and termination requests forwarded to the \orchestrator{}.

The host level lies at the edge of the network and realizes and manages the virtualization platform where applications are deployed. At this tier of the network, the \mechost{} is the main entity providing computational, network, and storage resources for the \mecapp{}s by exploiting the \mec{} Platform (\mecplatform{}) and the Virtualization Infrastructure (\vi{}). The \mecplatform{} exposes a service registry in which applications can discover, offer and consume standard \mec{} services defined by ETSI; at the moment of writing, the standard supports four types of services that each \mechost{} might maintain: Radio Network Information Service, Location Service, Traffic Management Service, and  Application Mobility Service (AMS)~\cite{etsiamsapi} conceived to handle \mecapp{} migrations between edge nodes. In this context, each application runs as a virtual machine or container 
on top of the \vi, and it optionally interacts with the \mecplatform{} and its registry to take advantage of the standard \mec{} services.
Finally, alongside the \mechost{}, the Virtualization Infrastructure Manager (\vim{}) and the \mec{} Platform Manager (\mecpm{}) serve as access points for the \orchestrator{} to the lower layer. The \vim{} has the task of managing and releasing the virtualized resources and appropriately configuring the \vi{} to run software images, while the \mecpm{} handles all the components functions related to a specific \mechost{}.

\begin{table}
\centering
\caption{Table of acronyms for MEC elements}
\resizebox{\linewidth}{!}{{\renewcommand{\arraystretch}{1.3}%Vertical padding factor
\begin{tabular}{c c}
\hline
 \textbf{Abbreviation} & \textbf{Definition} \\
 \hline
 AMS & Application Mobility Service \\
 % \hline
 MEC & Multi-access Edge Computing \\
 % \hline
 MEC-App & MEC Application \\
 % \hline
 MEC-H & MEC Host \\
 % \hline
 MEC-O & MEC Orchestrator \\
 % \hline
 MEC-P & MEC Platform \\
 % \hline
 MEC-PM & MEC Platform Manager \\
 % \hline
 UALCMP & User Application LifeCycle Management Proxy \\
 % \hline
 VI & Virtualisation Infrastructure \\
 % \hline
 VIM & Virtualisation Infrastructure Manager \\
 \hline
\end{tabular}}}
\label{table:mec_acr}
\end{table}

\subsection{\omnetpp{} and Simu5G}\label{subsec:bgsimtool}
\omnetpp{}\cite{omnetwebsite} is a discrete event simulator framework used to model and build general-purpose simulations. It proposes a modular architecture based on components that can be arranged to create simulation models easily and effectively. 
%Armir: Thanks to this feature, \omnetpp{} has allowed it to gain popularity within the scientific community as a network simulation tool and, to date, it is supported by several third-party projects that continue to extend it with new models.

%It proposes a programming system based on parameterized components called modules that, if properly designed, can be reused easily and effectively. The modules are divided into simple and compound and are arranged, with a declarative approach, in a hierarchical structure that represents the final model of the simulation. Moreover, the communication between components relies on message exchange that leverages a system of gates used to define the input and output entry points for modules. The highly modular architecture of \omnetpp{} has allowed it to gain popularity within the scientific community as a network simulation tool and, to date, it is supported by several third-party projects that continue to extend it with new models.

Build on top of \omnetpp{} is Simu5G~\cite{simu5g_1}, a simulation library containing a collection of models and components useful for creating arbitrarily complex end-to-end scenarios involving 5G radio networks. Simu5G models both the core network and the RAN of a 5G network through the implementation of 3GPP-compliant protocols and a physical transmission system based on a set of customizable channels. The simulations created with this library can also leverage heterogeneous models related to gNBs base stations 
% that communicate via X2 interface 
while supporting handover and inter-cell interference coordination. Indeed, Simu5G can be used to analyze transition scenarios, from 4G to 5G networks, thanks to the ability to use dual connectivity (X2) between eNB and gNB access types. Among the various implemented models, Simu5G provides an implementation of the ETSI MEC standard with its main components (i.e., \orchestrator, \mechost, etc.) and the ability to create applications that communicate via ETSI-compliant interfaces with other elements of the MEC ecosystem. Running applications can either be self-contained or take advantage of the presence of standard MEC services; currently, the framework implements the Location Service and the Radio Network Interface Service. A significant feature of the tool is that Simu5G can also be employed as a real-time emulator to replace simulation elements 
%(5G network, MEC elements, etc.)
with real devices and thus use the same code for simulation and prototyping. Additionally, this also allows for more realistic and reliable data collection. 

Our simulation model works within the \omnetpp{} simulation tool and relies on the Simu5G communication library. It provides a new interpretation of the ETSI \mec{} functional elements enabling dynamic resource acquisition at \mec{} Host level. 

While preserving the general aspect of our study and without loss of generality, in the following section, we survey prior research effort on key design features our solution embodies.

%Armir: descrivere in modo sintetico il nostro contributo - costruito sopra Simu5G - e anticipare la sec. successiva.
% {\color{red}@Angelo: c'era un comento mio anche prima. 1 frase di delta w.r.t Simu, e.g., broker}
