\section{Related Work}
Considerable research effort has been devoted in the past, advocating for the use of vehicular resources to improve service delivery at the edge~\cite{8936985}. %and/or the support for alternate service delivery models without strict infrastructure reliance~\cite{8936985}. 
The concept of Vehicular Cloud Computing was initially proposed in~\cite{abuelela2010} and~\cite{6257116}. Abuelela and Olariu~\cite{abuelela2010} introduced the concept of Vehicular Cloud (VC), where vehicular resources are exploited as a mean to provide diverse community services. Similarly, Gerla in~\cite{6257116} outlined two applications of VC in which vehicles not only act as datacentres but also as observers of the environment. The service delivery model discussed in these works embodies new challenges when compared to the traditional, infrastructure-based one, such as task scheduling and distribution, resource volatility and acquisition, to be attributed to the unpredictability of the environment, i.e., node mobility. 

%Armir: To deal with the volatility of topological information, Arif et al. in~\cite{6143927} modeled the residency time of vehicles in the parking lot of an airport to create an accurate picture of the number of available resources.
Counteracting the mobility phenomenon, the authors in~\cite{5935198, 9475490} proposed to use parked vehicles as relay nodes, which can help improve connectivity and augment the chances of message delivery. 
Conversely, other works focused on exploiting vehicle resources for edge application execution~\cite{8522034, 9344808, 9366768, 9709120}. On this front, Huang \emph{et al}.\cite{8522034} proposed a centralized architecture, where the central node (i.e., nodes at the edge of the network) receive task request which are then distributed as sub-tasks on selected parked vehicles.
Similar in its objective, the work in~\cite{9344808} proposes a decentralized approach, offloading task execution to vehicle resources available nearby. Addressing a practical challenge to service delivery, Li \emph{et al}.~\cite{8463481} defined a contract-based incentive mechanism to persuade vehicle owners to rent out their resources. Similarly, the authors in~\cite{9475490} propose an auction-based model where participating nodes compete to lend their resources in an extended vehicular resource pool. %to use vehicle resources as network traffic forwarding. 

%Due to mobility, the vehicular resource pool might be subject to continuous changes in terms of capacity, and in task offloading-based algorithm it is a necessary system primitive which needs to be carefully considered.
Due to mobility, the vehicular resource pool might be subject to continuous changes in terms of capacity and task offloading decisions require carefully consideration. The necessity for this might also arise due to an inaccurate estimate of residual resource availability. In this context, most of the works propose algorithmic strategies used to evaluate the probability of nodes to complete task execution~\cite{8522034, 9123902, 9344808, 9366768, 9709120}. However, these approaches are probabilistic, neglecting also practical considerations such as nodes refusing to partake in the resource pool. In this direction, the authors in ~\cite{ge2020two} present a two-stage algorithm that handles service migration from one service provider to another in a vehicular network context. The algorithm relies on several metrics, such as average latency and energy consumption, to properly select the next service provider.
%Armir: dire 1/2 frasi ++ sul lavoro sopra, e.g., come fanno la evaluation. Inoltre rendere chiaro se l'offloading a solo infrastructure->VANET o cosa...
Although there has been some research effort addressing task scheduling and resource allocation problems in the vehicular network tier, these proposals neglect and make no consideration of the challenges arising in multi-vendor and multi-domain environments. Furthermore, the migration problem does not seem to concern many authors. 

In this paper, we propose an ETSI \mec-compliant architecture and an accompanying simulation model, extending the edge resource pool to contemplate far-edge (vehicular) resource infrastructures. Vehicular resources can be transparently accessed and made readily available through standardized interfaces. Our proposal has built-in mechanisms capable of deploying and distributing applications on the available resource pool, while addressing resource volatility issues (i.e., nodes joining/leaving) via a transparent migration mechanism that exploits already available constructs. 
As an extension of the ETSI \mec~standard, the model allows us to deal with some of the aforementioned challenges, providing better integration with cloud resources and enabling coexistence among heterogeneous technologies.