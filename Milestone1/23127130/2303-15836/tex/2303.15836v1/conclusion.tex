\section{Conclusions and Future Work}
In this paper, we propose a novel architecture extending the ETSI MEC standard to enable the dynamic negotiation and acquisition of far-edge resources at \mechost{} level. The presented scheme allows device owners to access a reward system and addresses resource volatility problems as devices join and leave the \mechost{} resource pool. To demonstrate the viability of our approach, we built a simulation model that extends ETSI \mec{} towards scenarios with dynamic availability of vehicular resources. The model is validated with three service delivery schemes, thus demonstrating how our approach introduces good performance in \mec{}-enabled environments. Finally, we also proved its robustness in real-world dynamic conditions.

Although the nature of our architecture considers both physical and mobile nodes, in this discussion we have limited the simulation-based evaluation to parked vehicles and local migrations within the \mechost{}. As future work, we envision expanding our simulation model to support resources provided by stationary and moving vehicles in more complex scenarios, e.g. smart-city ones. Consequently, we plan to generalize the reward mechanism to support pluggable reward schemes, e.g., relying on user behaviors. Moreover, we plan to extend the support for pluggable scheduling modules so to study the effect of scheduling strategies that distribute and migrate \mec{} applications among far-edge nodes according to device parameters (e.g., energy consumption) and app requirements (e.g., max latency).
