\documentclass[a4paper,11pt]{article}
\pdfoutput=1 % if your are submitting a pdflatex (i.e. if you have
             % images in pdf, png or jpg format)

\usepackage{jinstpub} % for details on the use of the package, please
                     % see the JINST-author-manual
\usepackage{siunitx}
\usepackage{lineno}
\usepackage{subcaption}
%\linenumbers

\title{SiPM array of Xenoscope, a full-scale DARWIN vertical demonstrator}

\collaboration[c]{on behalf of the
Xenoscope* team\note[*]{www.physik.uzh.ch/en/groups/baudis/Research/Xenoscope}}

%\author{L.~Baudis,}
%\author{Y.~Biondi,}
%\author{A.~Bismark,}
%\author{A.~P.~Cimental~Chávez,}
%\author{J.~J.~Cuenca~García,}
%\author{M.~Galloway,}
%\author{F.~Girard,}
\author{R.~Peres,}
%\author{D.~Ramírez~García,}
%\author{C.~Wittweg}

\affiliation{Department of Physics, University of Zurich, Winterthurerstrasse 190, 8057 Zurich, Switzerland }

\emailAdd{rperes@physik.uzh.ch}


\abstract{The DARWIN project aims to build and operate a next-generation observatory for dark matter and neutrino physics, featuring a time projection chamber with a proposed active target of \SI{40}{t} of liquid xenon. As an R\&D facility to test fundamental components of the future detector, Xenoscope, a full-scale vertical demonstrator with $\sim$\SI{400}{kg} of liquid xenon and up to \SI{2.6}{m} electron drift length, was built at the University of Zurich. Its main objective is to demonstrate electron drift over unprecedented distances in liquid xenon --- first in a purity monitor setup with charge readout, followed by a dual-phase time projection chamber. In this second phase, an array of 48 VUV4 MMPCs from Hamamatsu (model S13371-6050CQ-02) with a 12-channel readout will be placed above the liquid xenon column and operated as a light readout for the secondary proportional scintillation signals coming from extracted electrons in the time projection chamber.
This work presents the design and development of the silicon photomultiplier array of Xenoscope, covering the structural and electronic design, sensor characterisation at cryogenic temperature and signal simulation.} 


\keywords{Photon detectors for UV, visible and IR photons (solid-state) , Charge transport, multiplication and electroluminescence in rare gases and liquids, Cryogenic detectors, Noble liquid detectors (scintillation, ionization, double-phase), Time projection Chambers (TPC), Simulation methods and programs}


%\arxivnumber{1234.56789} % only if you have one


% if you write for a special issue this may be useful
\proceeding{LIDINE 2022: Light Detection In Noble Elements\\
  21 -- 23 September, 2022\\
  Warsaw, Poland}


\begin{document}
\maketitle
\flushbottom

The advance of Pre-trained Language Models (PLMs) like GPT-3 \cite{brown2020language} and LLaMA \cite{DBLP:journals/corr/abs-2302-13971} has substantially improved the performance of deep neural networks across a variety of Natural Language Processing (NLP) tasks. Various language models, based on the Transformer \cite{vaswani2017attention} architecture,  have been proposed, leading to state-of-the-art (SOTA) performance on the fundamental discrimination tasks. These models are first trained with self-supervised training objectives (e.g., predicting masked tokens according to surrounding tokens) on massive unlabeled text data, then fine-tuned on annotated data to adapt to downstream tasks of interest.  However, annotated data is usually limited to a wide range of downstream tasks, which results in overfitting and a lack of generalization to unseen data.

One straightforward way to deal with this data scarcity problem is data augmentation , and incorporating generative models to perform data augmentation has been widely adopted recently . Despite its popularity, the generated text can easily deviate from the real data distribution without exploiting any of the signals passed back from the discrimination task. In previous studies, generative data augmentation and discrimination have been well studied as separate problems, but it is less clear how these two can be leveraged in one framework and how their performances can be improved simultaneously. \looseness=-1

Generative Adversarial Networks (GANs) \cite{https://doi.org/10.48550/arxiv.1406.2661} are good attempts to couple generative and discriminative models in an adversarial manner, where a two-player minimax game between learners is carefully crafted. GANs have achieved tremendous success in domains such as image generation , and related studies have also shown their effectiveness in semi-supervised learning. However,  in the text field, GANs are difficult to train, most training objectives work well for only one model, either the discriminator or the generator, so rarely both learners can be optimal at the same time. This essentially arises from the adversarial nature of GANs, that during the process, optimizing one learner can easily destroy the learning ability of the other, making GANs fail to converge.

Another limitation of simultaneously optimizing the generator and the discriminator comes from the discrete nature of text in NLP, as no gradient propagation can be done from discriminators to generators. One theoretically sound attempt is to use reinforcement learning (RL), but the sparsity and the high variance of the rewards in NLP make the training particularly unstable \cite{caccia2019language}. 

To address these shortcomings, we novelly introduce a self-consistent learning framework based on one generator and one discriminator: the generator and the discriminator are alternately trained by way of cooperation instead of competition, and the selected samples are used as the medium to pass the feedback signal from the discriminator. Specifically, in each round of training, the samples generated by the generator are synthetically labeled by the discriminator, and then only part of them would be selected based on dynamic thresholds and used for the training of the discriminator and the generator in the next round. Several benefits can be discovered from this cooperative training process. First, a closed-loop form of cooperation can be established so that we can get the optimal generator and discriminator at the same time. Second, this framework helps improve the generation quality while ensuring the domain specificity of generator, which in turn contributes to training. Third, a steady stream of diverse synthetic samples can be added to the training in each round and lead to continuous improvement of the performance of all learners. Finally, we can start the training with only domain-related corpus and obtain strong results, while these data can be easily sampled with little cost or supervision. Also, the performance on labeled datasets can be further boosted based on the strong baselines. As an example to demonstrate the effectiveness of our framework in the text field, we examine it on four downstream text generation benchmarks, including AFQMC, CHIP-STS, QQP, and MRPC. The experiments show that our method significantly improves over standalone state-of-the-art discriminative models on zero-shot and full-data settings.

Our contributions are summarized as follows,

$\bullet$ We propose a self-consistent learning framework in the text field that incorporates the generator and the discriminator, in which both achieve remarkable performance gains simultaneously.

$\bullet$ We propose a dynamic selection mechanism such that cooperation between the generator and the discriminator drives the convergence to reach their scoring consensus.

$\bullet$ Experimental results show that the generator in our framework can continuously adjust its generation samples based on the performance of downstream tasks, while the discriminator can outperform the strong baselines.


%\input{Sections/XenoscopeFacility.tex}
\section{The top array of Xenoscope}
\label{sec:sipmarray}

%Xenoscope will run with SiPMs in its top array as its only light sensors, focusing on recording the S2 signal of the electrons drifted from the photocathode, which are produced via photoelectric effect from the UV light of a xenon flash lamp. The 
The array consists of 192 \mbox{$6\times\SI{6}{mm^2}$} multi-pixel photon counters (MPPCs). These are packaged in $2 \times 2$ quad modules (S13371-6050CQ-02 MPPCs from Hamamatsu) with a total sensitive area of \mbox{$12\times\SI{12}{mm^2}$}. The four quads are loaded onto printed circuit boards (PCBs) with push-pin connectors, named "tiles" (figures~\ref{fig:tiles} left and centre) and There are 12 tiles in the array. In total, the array has ~$36$\% of its area covered by active sensors. The collection of tiles is screwed to a stainless steel plate for mechanical stability with Polytetrafluoroethylene (PTFE) standoffs to ensure the protection of the wiring on the backside of the PCB. When assembled in Xenoscope's TPC, the array is secured by the stainless steel plate fitting into grooves in Polyamid-imide (Torlon) pillars with the photosensors facing downward. The distance from the anode to the SiPM plane is \SI{14.65}{mm}, protecting against the risk of direct discharges through the array. To ensure that the MPPC modules are safely in place and prevent the dislodging of any units, a perforated PTFE cover of \SI{3}{mm} is placed in front of the modules (figure~\ref{fig:tiles}, right). The PTFE mask is the only reflector in Xenoscope and its goal is structural integrity. Because the primary goal is to detect O($10^3$-$10^4$) PE S2 signals from a triggered pulse, light loss due to a lack of reflectors is not a concern.

\begin{figure}[b]
\centering
\begin{subfigure}{0.3\textwidth}
    \includegraphics[width=\textwidth]{Figures/tile_unloaded_2ith_ref.jpg}
    %\caption{}
    %\label{fig:unloadedtile}
\end{subfigure}
\hfill
\begin{subfigure}{0.3\textwidth}
    \includegraphics[width=\textwidth]{Figures/tile_loaded_edit.png}
    %\caption{}
    %\label{fig:loadedtile}
\end{subfigure}
\hfill
\begin{subfigure}{0.3\textwidth}
    \includegraphics[width=\textwidth]{Figures/loaded_array_nov.png}
    %\caption{}
    %\label{fig:loadedarray}
\end{subfigure}

\caption{(Left): Tile module without any MPPC modules loaded. (Centre): Tile module with four 12x12 mm$^2$ quad MPPCs. (Right) Fully loaded SiPM array with the PTFE cover.}
\label{fig:tiles}
\end{figure}

The tiles serve at the same time as holders for the SiPMs, voltage distributors and pre-amplifiers for the signals. The readout scheme is based on the design proposed in~\cite{Arneodo:2017ozr}, where the amplified output is the analogue summed signal, optimised to exclude contributions from non-triggered SiPMs. The pre-amplifier circuit is loaded with an OPA847 operational amplifier from Texas Instruments and provides a $\times$20 amplification factor to the summed signal. 


\section{MPPC characterisation}
\label{sec:charcampaign}

To verify the integrity and properties of the MPPCs purchased for the array, a characterisation campaign of all the photosensor modules was conducted. The main objectives were to measure the breakdown voltage, gain, single photoelectron (SPE) resolution, dark count rate (DCR) and cross-talk probability (CTP) for each of the 48 quad sensors. %The results obtained provide the distribution of the mentioned characteristics in a large set of samples, as needed for a full scale-up of LXe TPCs photosensor arrays, and, in parallel, inform how to organise the quads in each tile in a gain matched configuration. 
This work reports the initial results from the characterisation campaign, where a fully loaded tile was used to verify the performance of the tile electronics and sensors in cold.% The complete campaign analysis and results of all 48 MPPC modules and the fully loaded array will be the subject of a future publication.

The characterisation of the photosensors was performed in the Liquid Argon Setup (LArS) at the University of Zurich using the actual array to later be installed in Xenoscope. The setup consists of a double-walled cylindrical vessel with \SI{250}{mm} diameter sealed to a top flange by an o-ring, where several feedthroughs serve the inner space with both connections to electronics, external gas bottles or vacuum pumps. The array is suspended by stainless steel rods from a PTFE holder, which, in turn, is suspended from the top flange with rods as well. In the centre of the PTFE holder, a blue LED \mbox{($\lambda = \sim~\SI{420}{nm}$)} can be used to illuminate the array. The power transmission wires of the array (kapton-insulated, stranded copper wires), which supply the bias voltage of the SiPMs and the operational amplifier, are connected to a breakout box on the inside of the vessel. This breakout box merges all the wires of the same intended purpose together. %, i.e. all the positive voltage connecting to the pre-amplifiers together, all the negative voltage connecting to the pre-amplifiers together, and so on. 
After the merge, only five wires are needed to be routed to the outside via a potted feedthrough connected on the top flange. In turn, the coax signal cables from each tile are connected directly to another potted feedthrough and are routed to the data acquisition system (DAQ).

The inner volume of the setup is first pumped-out to avoid any residual water vapour being left in the volume and then filled with helium, which is used as a coolant gas. %, which, in the particular case of this characterisation campaign, was Helium. 
The volume is then cooled via the supply of liquid nitrogen through a copper pipe coiled around the upper part of the chamber, where it undergoes liquid-to-gas transition, cooling down the system. As the pressure of the He gas decreases during cooling, more gas is supplied to keep the pressure between 1.9 and \SI{2}{bar}. The temperature inside the vessel is regulated by the flow of liquid nitrogen through the copper coil, controlled by a flow valve connected to a proportional–integral-derivative controller.

Data is acquired with two 8-channel v1724 digitizers from CAEN, read out through a v2718 VME-PCI Optical link bridge module to an on-site computer. The ADC is programmed using a custom C++ software based on CAEN libraries, while the pulse finding, data management, and processing are done with the in-house developed PyLArS package~\cite{pylars}. The software defines a signal pulse when ADC counts exceed five times the RMS of the baseline value, tuning the integration window on a per-pulse basis.

In figure \ref{fig:results}, left, the integrated ADC-count spectrum of a dataset in dark environment is shown after basic noise cuts on the width of the identified pulses. The first peak, result of pure dark count events, is fitted with a Gaussian function to determine the value of the SPE area and SPE resolution. Based on the SPE value, the 1, 2, and 3 PE areas are shown in blue dashed lines, aligning with the correlated cross-talk peaks. In all charge-peaks a shoulder-like region is observed on the higher-area side. The shape of such pulses mimics the shape of the main peak population, albeit with higher integrated area. These events could be due to unresolved fast correlated avalanches and were similarly observed in~\cite{Gallina:2019fxt} when studying the same family of photosensors.

At temperatures between \SI{170}{} and \SI{200}{K}, a voltage scan was performed to study its effect on the measured gain and determine the breakdown voltage. A gain of $3\times10^6$ was observed at \SI{4.5}{V} above breakdown voltage (over-voltage), corresponding to an applied voltage of \SI{52.1}{V} at \SI{170}{K} and \SI{53.2}{V} at \SI{190}{K}. As part of the gain computation, the SPE resolution is also determined. It was observed that the SPE resolution mainly depends on the over-voltage (or gain), with a small dependence on temperature. For a measured gain of $3\times10^6$, the measured SPE resolution was $\sim5\%$ at \SI{170}{K} and $\sim5.5\%$ at \SI{190}{K}.

At the temperatures for which the array is expected to be operated in Xenoscope's TPC gas-phase, i.e. between \SI{190}{K} and \SI{200}{K}, long datasets in a dark environment were taken. Defining CTP as the ratio of pulses above a 1.5 PE threshold to the pulses above 0.5 PE, the DCR and CTP were computed. The results can be found in figure~\ref{fig:results} centre and right, respectively. As expected, both quantities increase exponentially with gain. The DCR is $\mathcal{O}{(1)~\si{Hz/mm^2}}$, comparable to current~(\cite{Gallina:2019fxt}) and earlier~(\cite{Baudis:2018pdv}) versions of this  photosensor type. 

While for higher temperatures the measured DCR is higher, the CTP is observed to be independent of temperature. This is a considerable improvement from previous versions of the VUV SiPM from Hamamatsu~\cite{Baudis:2018pdv}, now $<15$\% CTP at a gain of $3\times10^6$.


\begin{figure}[t]
\centering
\begin{subfigure}{0.315\textwidth}
    \centering
    \includegraphics[width=\textwidth]{Figures/new_plot_area_spectrum_small.pdf}
    %\caption{}
    %\label{fig:spectrum}
\end{subfigure}
\hfill
\begin{subfigure}{0.315\textwidth}
    \centering
    \includegraphics[width=\textwidth]{Figures/NEW_dcr_gain.pdf}
    %\caption{}
    %\label{fig:dcr_gain}
\end{subfigure}
\hfill
\begin{subfigure}{0.315\textwidth}
    \includegraphics[width=\textwidth]{Figures/NEW_ctp_gain.pdf}
    %\caption{}
    %\label{fig:ctp_gain}
\end{subfigure}

\caption{(Left): Area spectrum of a dark dataset at \SI{190}{K}, \SI{4}{V} over-voltage, where the 1, 2, and 3 photoelectron areas are shown in blue dashed lines. DCR (centre) and CTP (right) as a function of gain for \SI{190}{} and \SI{195}{K}.}
\label{fig:results}
\end{figure}
\section{Signal simulation framework development}
\label{sec:signalsim}

To predict the expected signals at the SiPM array arising from ejected electrons at the photocathode with a flash of the xenon lamp, a simulation framework, XenoDiffusionScope~\cite{xenodiffusionscope}, was developed. The framework provides the base for the phenomenological study of electron longitudinal and transversal diffusion properties in the context of Xenoscope. The basic principles and steps of the simulation tool and its first results are described below.

\begin{figure}[t]
\centering

\begin{subfigure}{0.32\textwidth}
    \includegraphics[width=\textwidth]{Figures/results_in_tiles.pdf}
    %\caption{}
    \label{fig:simtiles}
\end{subfigure}
\hfill
\begin{subfigure}{0.32\textwidth}
    \includegraphics[width=\textwidth]{Figures/results_in_quads.pdf}
    %\caption{}
    \label{fig:simquads}
\end{subfigure}
\hfill
\begin{subfigure}{0.32\textwidth}
    \includegraphics[width=\textwidth]{Figures/results_in_hybrid1.pdf}
    %\caption{}
    \label{fig:simhybrid}
\end{subfigure}

\caption{Simulated hit patterns on the top array from a flash of the xenon lamp on the photocathode for different photosensor granularity: current configuration of 12 tiles (left), individual readout channel for each quad module (centre) and an hybrid configuration with both quad modules and individual \mbox{$6\times\SI{6}{mm^2}$} units (right).}
\label{fig:simpatterns}
\end{figure}

The code is organised in a modular fashion to simplify the change of parameters and further development of new or improved sections of the toy-Monte Carlo simulator. The xenon lamp pulse is modelled to determine the initial electron cloud distribution in space and time. Once created in the LXe volume, electrons are drifted from cathode to gate, experiencing longitudinal and transversal diffusion, whose constants are defined by the user. At this stage, the boundaries of the TPC are taken into account to not propagate any electrons outside of the field cage, as well as to include electron lifetime and extraction efficiency effects. Close to the gate, the electrons are focused on the closest centre of an element of the hexagonal mesh, a known effect based on data from Xurich II~\cite{Baudis:2020nwe}. From there, the distribution of photoelectrons in the array area is computed by considering the electron-to-photoelectron gain (also known as single electron gain) and the light collection efficiency map from each of the extracted electron positions. The final pattern is achieved by summing the light patterns of all the extracted electrons and by integrating the resulting pattern in the photosensors' sensitive area.

This simulation framework will inform future design choices with respect to tiling and granularity of the SiPMs  (figure \ref{fig:simpatterns}). With constraints on correlated parameters, it can be used to study electron diffusion properties by comparing observed with simulated hit patterns. 
\section{Conclusion and outlook}
\label{outlook}

A dual-phase xenon TPC is being prepared for installation in Xenoscope, where the proportional scintillation light will be collected by an array of SiPMs in the gas phase. This work reports on the design and first results of the characterisation of the VUV4 MPPCs from Hamamatsu chosen as photosensors, their summed readout, and the developed toy-MC based signal simulation framework to inform future design choices and physics reach.

The measured gain dependence on the bias voltage is similar to the previous generation of VUV sensitive photosensors from Hamamatsu and comparable to other available photosensors. The SPE resolution was measured to be $\sim5\%$ for a gain of $3\times10^6$ in a fully loaded tile. The DCR and CTP measured are on par and improve upon the previous generation of VUV-sensitive photosensors from Hamamatsu, respectively. A shoulder-like structure was observed on the peaks of the charge spectrum, populated by events with similar shape and duration as SPE or 2 PE signals but with higher integrated charge. The results of the characterisation of the full set of 48 MPPCs and fully loaded array are in preparation and will be reported in a future publication.


%\section{Appendix for Proofs}

\paragraph{Proof of Theorem \ref{thm:main}.}

\begin{proof}
\label{proof:main}
Our proof has two steps. In Step 1, we will show that SimCLR is equivalent to minimizing the cross entropy loss defined in Eqn.~(\ref{eqn:cross-entropy}). 
In Step 2, we will show  that minimizing the cross-entropy loss 
is equivalent to spectral clustering on $\bfpi$. 
Combining the two steps together, we have proved our theorem. 

\textbf{Step 1: } SimCLR is equivalent to minimizing the cross entropy loss.

The cross-entropy loss takes expectation over 
$\bfW_\bfX\sim \mathbb{P}(\cdot ; \bfpi)$, 
which means $\bfW_\bfX$ has exactly one non-zero entry in each row $i$. By Lemma~\ref{lem:multinomial}, we know every row $i$ of $\bfW_\bfX$ is independent of other rows. Moreover, 
$\bfW_{\bfX,i}\sim \mathcal{M}(1, \bfpi_i/\sum_j \bfpi_{i,j})=\mathcal{M}(1, \bfpi_i)$, because $\bfpi_i$ itself is a probability distribution.
Similarly, we know $\bfW_\bfZ$ also has the row-independent property by sampling over $\mathbb{P}(\cdot;\bfK_\bfZ)$.
Therefore, by Lemma~\ref{lem:cross_split}, we know Eqn.~(\ref{eqn:cross-entropy}) is equivalent to:
\[
 -\sum_{i=1}^n \mathbb{E}_{\bfW_{\bfX,i}}[\log \mathbb{P}(\bfW_{\bfZ,i}=\bfW_{\bfX,i};\bfK_\bfZ)],
\]

This expression takes expectation over $\bfW_{\bfX,i}$ for the given row $i$. Notice that 
$\bfW_{\bfX,i}$ has exactly one non-zero entry, which equals $1$ (same for $\bfW_{\bfZ,i}$). 
As a result
we expand the above expression to be:
\begin{equation}
 -\sum_{i=1}^n \sum_{j\neq i} \Pr(\bfW_{\bfX,i,j}=1)\log \Pr(\bfW_{\bfZ,i,j}=1).
\label{eqn:detailed-expansion}    
\end{equation}


By Lemma~\ref{lem:multinomial}, $\Pr(\bfW_{\bfZ,i,j}=1)=\bfK_{\bfZ,i,j}/\|\bfK_{\bfZ,i}\|_1$ for $j\neq i$. Recall that $\bfK_\bfZ=(k(\bfZ_i-\bfZ_j))_{(i,j)\in[n]^2}$, which means 
$\bfK_{\bfZ,i,j}/\|\bfK_{\bfZ,i}\|_1=\frac{\exp(-\|\bfZ_i-\bfZ_j\|^2/{2\tau})}{\sum_{k\neq i}
\exp(-\|\bfZ_i-\bfZ_k\|^2/{2\tau})
}$ for $j\neq i$, when $k$ is the Gaussian kernel with variance $\tau$. 

Notice that $\bfZ_i=f(\bfX_i)$, so we know
\begin{equation}
-\log \Pr(\bfW_{\bfZ,i,j}=1)=
-\log \frac{\exp(-\|f(\bfX_i)-f(\bfX_j)\|^2/{2\tau})}{\sum_{k\neq i}
\exp(-\|f(\bfX_i)-f(\bfX_k)\|^2/{2\tau}),
}
\label{eqn:infonce-equivalence}    
\end{equation}


The right hand side is exactly the InfoNCE loss defined in Eqn.~(\ref{eqn:infonce}).
Inserting Eqn.~(\ref{eqn:infonce-equivalence}) into Eqn.~(\ref{eqn:detailed-expansion}), we get the SimCLR algorithm, which first samples augmentation pairs $(i,j)$ with $\Pr(\bfW_{\bfX,i,j}=1)$ for each row $i$, and then optimize the InfoNCE loss. 

\textbf{Step 2: } minimizing the cross entropy loss 
is equivalent to spectral clustering on $\bfpi$.


By Lemma~\ref{lem:convert_to_spectral}, we may further convert the loss to 
\begin{equation}
\label{eqn:main-theorem-repul-attr}
\min_{\bfZ}
-\sum_{(i,j)\in [n]^2} \mathbf{P}_{i,j}
\log k (\bfZ_i-\bfZ_j)+\log \mathbf{R}(\bfZ).
\end{equation}
Since $k$ is the Gaussian kernel, this reduces to \[
\min_\bfZ \mathrm{tr}(\bfZ^\top \mathbf{L}(\bfpi) \bfZ)
+\log \mathbf{R}(\bfZ),
\]

where we use the fact that $\mathbb{E}_{\bfW_\bfX\sim \mathbb{P}(\cdot; \bfpi)}[\mathbf{L}(\bfW_\bfX)]
=\mathbf{L}(\bfpi)
$, because the Laplacian operator is linear and $
\mathbb{E}_{\bfW_\bfX\sim \mathbb{P}(\cdot; \bfpi)}(\bfW_\bfX)=\bfpi
$.
\end{proof}

\paragraph{Proof of Theorem \ref{thm:clip}.}
\begin{proof}
Since $\bfW_\bfX\sim \mathbb{P}(\cdot;\bfpi_{\mathbf{A}, \mathbf{B}})$, we know 
$\bfW_\bfX$ has exactly one non-zero entry in each row, denoting the pair that got sampled. 
A notable difference compared to the previous proof is we now have $n_\mathcal{A}+n_\mathcal{B}$ objects in our graph. CLIP deals with this by taking a mini-batch of size $2N$, 
such that $n_\mathcal{A}=n_\mathcal{B}=N$, and adding the $2N$ InfoNCE losses together. We label the objects in $\mathcal{A}$ as $[n_\mathcal{A}]$, and the objects in $\mathcal{B}$ as $\{n_\mathcal{A}+1, \cdots, n_\mathcal{A}+n_\mathcal{B}\}$. 

Notice that $\bfpi_{\mathbf{A}, \mathbf{B}}$ is a bipartite graph, so the edges of objects in $\mathcal{A}$ will only connect to object in $\mathcal{B}$ and vice versa. We can define the similarity matrix in $\cZ$ as $\bfK_\bfZ$, 
where $\bfK_\bfZ(i, j+n_\mathcal{A})=\bfK_\bfZ(j+n_\mathcal{A},i)= k(\bfZ_i-\bfZ_j)$ for $i\in [n_\mathcal{A}], j\in [n_\mathcal{B}]$, and otherwise we set $\bfK_\bfZ(i,j)=0$. 
The rest is same as the previous proof. 
\end{proof}

\paragraph{Proof of Theorem \ref{thm:exponential}.}

\begin{proof}
\label{proof:exponential}
Since the objective function consists of a linear term combined with an entropy regularization, which is a strongly concave function, the maximization problem is a convex optimization problem. Owing to the implicit constraints provided by the entropy function, the problem is equivalent to having only the equality constraint. We then introduce the Lagrangian multiplier $\lambda$ and obtain the following relaxed problem:

$$
\widetilde{E}(\boldsymbol{\alpha})=\psi_{1}-\sum_{i=1}^n \alpha_{i} \psi_{i}+\tau \sum_{i=1}^n \alpha_{i}\log \alpha_{i}+\lambda\left(\boldsymbol{\alpha}^{\top} \mathbf{1}_n-1\right).
$$

As the relaxed problem is unconstrained, taking the derivative with respect to $\alpha_{i}$ yields

$$
\frac{\partial \widetilde{E}(\boldsymbol{\alpha})}{\partial \alpha_{i}}=-\psi_{i}+\tau\left(\log \alpha_{i}+\alpha_{i} \frac{1}{\alpha_{i}}\right)+\lambda=0.
$$

Solving the above equation implies that $\alpha_{i}$ takes the form
$
\alpha_{i}=\exp \left(\frac{1}{\tau} \psi_{i}\right) \exp \left(\frac{-\lambda}{\tau}-1\right).
$ Since $\alpha_{i}$ lies on the probability simplex, the optimal $\alpha_{i}$ is explicitly given by
$
\alpha^{*}_{i}=\frac{\exp \left(\frac{1}{\tau} \psi_{i}\right)}{\sum_{i^{\prime}=1}^n \exp \left(\frac{1}{\tau} \psi_{i^{\prime}}\right)} .
$ Substituting the optimal point into the objective function, we obtain
$$
\begin{aligned}
E\left(\boldsymbol{\alpha}^*\right)  &=\psi_1-\sum_{i=1}^n \frac{\exp \left(\frac{1}{\tau} \psi_{i}\right)}{\sum_{i^{\prime}=1}^n \exp \left(\frac{1}{\tau} \psi_{i^{\prime}}\right)} \psi_{i}+\tau \sum_{i=1}^n \frac{\exp \left(\frac{1}{\tau} \psi_{i}\right)}{\sum_{i^{\prime}=1}^n \exp \left(\frac{1}{\tau} \psi_{i^{\prime}}\right)}\log \frac{\exp \left(\frac{1}{\tau} \psi_{i}\right)}{\sum_{i^{\prime}=1}^n \exp \left(\frac{1}{\tau} \psi_{i^{\prime}}\right)} \\
& =\psi_1 - \tau \log \left(\sum_{i=1}^n \exp \left(\frac{1}{\tau} \psi_{i}\right)\right).
\end{aligned}
$$
Thus, the Lagrangian dual function is given by
\begin{equation*}
-E\left(\boldsymbol{\alpha}^*\right)= -\tau \log \frac{\exp \left(\frac{1}{\tau} \psi_{1}\right)}{\sum_{i=1}^n \exp \left(\frac{1}{\tau} \psi_{i}\right)}.\qedhere
\end{equation*}
\end{proof}



\section{More on Experiments} \label{section: experiment_details}

\paragraph{CIFAR-10 and CIFAR-100} CIFAR-10 ~\citep{krizhevsky2009learning} and CIFAR-100 ~\citep{krizhevsky2009learning} are well-known classic image classification datasets. Both CIFAR-10 and CIFAR-100 contain a total of 60k $32 \times 32$ labeled images of different classes, with 50k for training and 10k for testing. CIFAR-10 is similar to CIFAR-100, except there are 10 different classes in CIFAR-10 and 100 classes in CIFAR-100.

\paragraph{TinyImageNet} TinyImageNet ~\citep{le2015tiny} is a subset of ImageNet ~\citep{deng2009imagenet}. There are 200 different object classes in TinyImageNet, with 500 training images, 50 validation images, and 50 test images for each class. All the images in TinyImageNet are colored and labeled with a size of $64 \times 64$.

\textbf{Pseudo-code.} Algorithm \ref{alg:Training Procedure} presents the pseudo-code for our empirical training procedure.

\begin{algorithm}[!htbp]
\caption{Training Procedure}
\label{alg:Training Procedure}
\begin{algorithmic}[1]
\REQUIRE trainable encoder network $f$, batch size $N$, augmentation strategy \textit{aug}, loss function $L$ with hyperparameters \textit{args}
\FOR {sampled minibatch ${x_i}_{i=1}^N$}
\FORALL{$i \in { 1, ..., N }$}
\STATE draw two augmentations $t_i = \textit{aug}\left(x_i\right) $, $t_i' = \textit{aug}\left(x_i\right) $
\STATE $z_i = f\left(t_i\right)$, $z_i' = f\left(t_i'\right)$
\ENDFOR
\STATE compute loss $\mathcal{L} = L(N, z, z', \textit{args})$
\STATE update encoder network $f$ to minimize $\mathcal{L}$
\ENDFOR
\STATE \textbf{Return} encoder network $f$
\end{algorithmic}
\end{algorithm}

We also provide the pseudo-code for our core loss function used in the training procedure in Algorithm \ref{alg:Core loss}. The pseudo-code is almost identical to SimCLR's loss function, with the exception of an extra parameter $\gamma$.

\begin{algorithm}[!htbp]
\caption{Core loss function $\mathcal{C}$}
\label{alg:Core loss}
\begin{algorithmic}[1]
\REQUIRE batch size $N$, two encoded minibatches $z_1, z_2$, $\gamma$, temperature $\tau$
\STATE $z = \textit{concat}\left(z_1, z_2\right)$
\FOR {$i \in {1, ..., 2N }, j \in {1, ..., 2N}$ }
\STATE $s_{i,j} = \Vert z_i - z_j \Vert_2^{\gamma}$
\ENDFOR
\STATE \textbf{define} $l(i, j)$ \textbf{as} $l(i, j) = - \log \frac{exp\left(s_{i,j}/\tau \right)}{\sum_{k=1}^{2N} \mathbf{1}{[k \ne i]} exp\left(s{i, j} / \tau \right)} $
\STATE \textbf{Return} $\frac{1}{2N} \sum_{k=1}^N\left[l(i, i+N) + l(i+N, i)\right]$
\end{algorithmic}
\end{algorithm}

Utilizing the core loss function $\mathcal{C}$, we can define all kernel loss functions used in our experiments in Table \ref{table: loss definition}. For all $z_i \in z$ with even dimensions $n$, we define $z_{L_i} = z_i\left[0:n/2\right]$ and $z_{R_i} = z_i\left[n/2:n\right]$.

\begin{table}[ht]
\centering
\begin{tabular}{{@{}l|l@{}}}
Kernel  &  Loss function \\ \midrule
Laplacian & $\mathcal{C}\left(N, z, z', \gamma=1, \tau\right)$\\ \midrule
Sum       & $\lambda * \mathcal{C}\left(N, z, z', \gamma=1, \tau_1\right) + (1-\lambda) * \mathcal{C}\left(N, z, z', \gamma=2, \tau_2\right)$  \\ \midrule
Concatenation Sum&$\lambda * \mathcal{C}\left(N, z_L, z'_L, \gamma=1, \tau_1\right) + (1-\lambda) * \mathcal{C}\left(N, z_R, z'_R, \gamma=2, \tau_2\right)$\\ \midrule
$\gamma = 0.5$ & $\mathcal{C}\left(N, z, z', \gamma=0.5, \tau\right)$          \\ 

\end{tabular}

\caption{Definition of kernel loss functions in our experiments}
\label {table: loss definition}
\end{table}

\textbf{Baselines.} We reproduce the SimCLR algorithm using PyTorch Lightning~\citep{PytorchLightning}.

\textbf{Encoder details.}
The encoder $f$ consists of a backbone network and a projection network. We employ ResNet50~\citep{ResNet} as the backbone and a 2-layer MLP (connected by a batch normalization~\citep{ioffe2015batch} layer and a ReLU \cite{nair2010rectified} layer) with hidden dimensions 2048 and output dimensions 128 (or 256 in the concatenation kernel case).

\textbf{Encoder hyperparameter tuning.}
For each encoder training case, we randomly sample 500 hyperparameter groups (sample details are shown in Table \ref{table: Hyperparameter sample}) and train these samples simultaneously using Ray Tune ~\citep{RayTune}, with the ASHA scheduler~\citep{li2018massively}. Ultimately, the hyperparameter group that maximizes the online validation accuracy (integrated in PyTorch Lightning) within 5000 validation steps is chosen for the given encoder training case.

\begin{table}[ht]
\centering

\begin{tabular}{@{}l|l|l@{}}
\midrule
Hyperparameter  & Sample Range & Sample Strategy \\ \midrule
start learning rate & $\left[10^{-2}, 10\right]$ & log uniform \\ \midrule
$\lambda$       & $\left[0, 1\right]$ & uniform \\ \midrule
$\tau$, $\tau_1$, $\tau_2$ & $\left[0, 1\right]$ & log uniform \\ \midrule
\end{tabular}

\caption{Hyperparameters sample strategy}
\label {table: Hyperparameter sample}
\end{table}

\textbf{Encoder training.} 
We train each encoder using the LARS optimizer~\citep{LARSOptimizer}, LambdaLR Scheduler in PyTorch, momentum 0.9, weight decay $10^{-6}$, batch size 256, and the aforementioned hyperparameters for 400 epochs on a single A-100 GPU.

\textbf{Image transformation.} The image transformation strategy, including augmentation, is identical to the default transformation strategy provided by PyTorch Lightning.

\textbf{Linear evaluation.}
The linear head is trained using the SGD optimizer with a cosine learning rate scheduler, batch size 64, and weight decay $10^{-6}$ for 100 epochs. The learning rate starts at $0.3$ and ends at $0$.

\textbf{Moco Experiments.} We also tested our method based on MoCo~\citep{he2019moco}. The results are summarized in Table \ref{tab:results-moco}. Here we choose ResNet18~\citep{ResNet} as the backbone and set a temperature of $0.1$ as default. For our simple sum kernel, we set $\lambda=0.8$. The results show that our method outperforms the original MoCo method.

\begin{table}[thb]
\centering
\caption{MoCo Experiment Results on CIFAR-10 and CIFAR-100.}
\label{tab:results-moco}
\resizebox{\textwidth}{!}{%
\begin{tabular}{@{}c|ccc|ccc@{}}
\toprule
\multirow{3}{*}{Method} & \multicolumn{3}{c|}{CIFAR-10} & \multicolumn{3}{c}{CIFAR-100} \\ \cmidrule(lr){2-4} \cmidrule(lr){5-7} 
                        & 200 epochs & 400 epochs    & 1000 epochs   & 200 epochs & 400 epochs & 1000 epochs         \\ \midrule
MoCo (repro.)         & $76.41 \pm 0.12$    & $80.01 \pm 0.15$          & $84.45 \pm 0.08$    & $\mathbf{47.02 \pm 0.11}$ & $52.50 \pm 0.07$ & $57.62 \pm 0.15$            \\
\midrule
Laplacian Kernel        & ${78.09 \pm 0.10}$    & $\mathbf{83.85 \pm 0.09}$          & $\mathbf{88.34 \pm 0.16}$    & $46.12 \pm 0.22$   & $53.44 \pm 0.17$ & $59.10 \pm 0.14$        \\
Simple Sum Kernel & $\mathbf{78.12 \pm 0.15}$   & $83.23 \pm 0.18$ & $87.50 \pm 0.20$ & $46.65 \pm 0.06$ & $\mathbf{53.62 \pm 0.19}$ & $\mathbf{59.83 \pm 0.12}$\\
\bottomrule
\end{tabular}
}
\end{table}



\section{More Experiments on Synthetic Data}


Consider a scenario with $n$ clusters, each containing $k$ vertices. Let the probability of vertices $u$ and $v$ from the same cluster belonging to $\bfpi$ be $p$. Conversely, for vertices $u$ and $v$ from different clusters, let the probability of belonging to $\pi$ be $q$. We generate the graph $\bfpi$ randomly, based on $p$ and $q$. We experiment with values of $k=100$ and $n=6$ for ease of visualization, embedding all points in a two-dimensional space. Each vertex's initial position originates from a normal distribution. In each iteration, we sample a subgraph of $\bfpi$ uniformly, ensuring each vertex has an out-degree of $1$. We then optimize the corresponding vectors using InfoNCE loss with an SGD optimizer and iterate until convergence. Our experimental setup consists of an SGD learning rate of $1$, an InfoNCE loss temperature of $0.5$, and a batch size of $50$. We evaluate two scenarios with different $p$ and $q$ values: $p=1$, $q=0$, and $p=0.75$, $q=0.2$. The results of these experiments are visualized in Figure \ref{fig:vis-spectral-cluster}. The obtained embeddings exhibit the hallmark pattern of spectral clustering of graph $\bfpi$.

\begin{figure}[!tb]
\centering
\subfigure{
\includegraphics[width=1\textwidth]{Figures/cluster_pi.png}
\label{fig:vis-cluster}
}
\subfigure{
\includegraphics[width=1\textwidth]{Figures/noised_cluster_pi.png}
\label{fig:vis-noised-cluster}
}
\caption{Visualizations of the optimization process using InfoNCE Loss on the vectors corresponding to $\bfpi$. Points of identical color belong to the same cluster within $\bfpi$. To showcase the internal structure of $\bfpi$, we randomly select 10 vertices from each cluster to display the edge distribution of $\bfpi$.}
\label{fig:vis-spectral-cluster}
\end{figure}



\acknowledgments
This work was supported by the European Research Council (ERC) under the European Union’s Horizon 2020 research and innovation programme, grant agreement No. 742789 (Xenoscope), by the SNF Grant 200020-188716 and by the University of Zurich.

\newpage
\bibliographystyle{jhep}
\bibliography{bibliography}

\end{document}
