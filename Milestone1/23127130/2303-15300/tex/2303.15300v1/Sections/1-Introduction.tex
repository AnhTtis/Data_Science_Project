\section{Introduction}
\label{sec:intro}
The DARWIN~\cite{DARWIN:2016hyl} project aims to be the next-generation liquid xenon (LXe) observatory for direct dark matter detection. With a proposed active target of \SI{40}{t} of LXe and using the dual-phase Time Projection Chamber (TPC) concept for particle detection and identification, the DARWIN experiment aims to probe WIMP-nucleon cross section down to the neutrino fog~\cite{OHare:2021utq}. %At this stage, solar neutrino interactions through CE$\nu$NS and electron scattering will  have a significant impact to the overall background and in the case of the first, be indistinguishable from WIMP-nucleon interactions. 
Apart from the search for dark matter, DARWIN will also conduct other rare-events searches such as for the neutrinoless double beta decay of $^{136}$Xe, neutrino signals from core collapse supernovae and other measurements related to solar neutrinos~\cite{DARWIN:2020jme,DARWIN:2020bnc,Aalbers:2022dzr}.

In its baseline design of a cylindrical dual-phase TPC, the target medium is the liquid phase and a set of electrodes define, from bottom to top, the drift region (cathode to gate) and the extraction region (gate to anode). In the extraction region there is a liquid-gas interface, roughly equidistant from both electrodes. Upon an interaction of a particle in the LXe target, the energy transfer is split between scintillation, ionisation and heat. The prompt \SI{175}{nm} scintillation photons are detected by photosensor arrays, with the signal denoted as S1. The free electrons from ionisation are transported towards the gate due to the existing vertical electric field, $E_{\mathrm{drift}}$. Electrons reaching the gate are extracted from the liquid to the gas phase given the stronger applied field between the gate and the anode, $E_{\mathrm{extraction}}$. As a result, a process of proportional scintillation is induced, giving origin to the second, delayed, scintillation signal or S2. For a strong enough field ($\sim\SI{10}{kV/cm}$), the charge extraction efficiency from liquid to gas is almost 100\%~\cite{Xu:2019dqb}.
 
%In its baseline design of a cylindrical dual-phase TPC, the target medium is the liquid phase and a set of electrodes define, from bottom to top, the drift region (cathode to gate) and the extraction region (gate to anode). In the extraction region stands the liquid-gas interface, roughly equidistant from both electrodes. Upon an interaction of a particle in the LXe target, the energy transfer is split between scintillation, ionisation and heat. From the first, an excited xenon atom merges with another xenon atom forming a diatomic excited molecule, which then de-excites into two xenon atoms and emits a 178 nm scintillation photon. The photon is not absorbed by the xenon atoms and can, therefore, be detected by photosensor arrays, giving origin to a primary scintillation signal or S1. On the other hand, electron-ion pairs, result of ionisation of xenon atoms, are split due to an existing vertical electric field, $E_{drift}$, which is also responsible for the drifting of the electron cloud up to the grounded gate and the positive ions to the cathode. There are two sets of photosensor arrays, located on the bottom of the TPC, bellow the cathode, submerged in LXe, and on the top of the TPC, above the anode, in the gas phase. Upon reaching the gate, the electrons are extracted from the liquid phase to the gas phase due to a much greater applied filed, applied between the gate and the anode, $E_{extraction}$. As a result of this field region, a process of proportional scintillation is induced, giving origin to the second scintillation signal, charge signal or S2. For a strong enough field ($\sim\SI{10}{kV/cm}$), the charge to light conversion is almost 100\%~\cite{PhysRevD.99.103024}. In LXe TPCs, only these two channels are observed, while the fraction of recoil energy transferred to heat is lost. Nonetheless, from the two distinct light signals (S1 and S2), one can determine about the interaction its initial position in the 3d space, the transferred recoil energy and the type of recoil, be it an electronic recoil (ER) or nuclear recoil (NR). 

LXe TPCs have been leading the field of direct dark matter searches in the last decade~\cite{XENON:2018voc, LUX:2016ggv, LZ:2022ufs, XENONCollaboration:2022kmb}. However, numerous challenges are yet to be tackled to successfully scale-up the proven detector concept. The Xenoscope facility developed at the University of Zurich directly targets the challenges arising from a \SI{2.6}{m} high TPC and testing the unprecedented long electron drift length.  A full description of the facility and results from its commissioning run can be found in~\cite{Baudis:2021ipf}.

The first phase of the project was the construction, commissioning and operation of a \SI{52}{cm} high, \SI{15}{cm} diameter, purity monitor with charge readout only. LXe properties such as electron drift velocity and electron cloud longitudinal diffusion were measured and a manuscript detailing these results is under preparation. For the next stage of the project, the full-height TPC is currently being installed. In this new configuration, several sub-systems are changed or introduced: liquid level monitoring and control are added, the high-voltage (HV) supply to the cathode is redesigned to allow considerably higher voltages, and the charge readout is replaced by an array of SiPMs to detect the proportional scintillation signals produced in the extraction region of the TPC.

The following sections describe %the main features of the Xenoscope facility and TPC (section \ref{sec:xenoscope}), 
the SiPM array design, its components and characteristics (section \ref{sec:sipmarray}), the first results of the characterisation of the photosensors used in the array (section \ref{sec:charcampaign}), and the signal simulation framework developed to inform future design, operation and \mbox{physics-studies} choices (section \ref{sec:signalsim}).