\section{MPPC characterisation}
\label{sec:charcampaign}

To verify the integrity and properties of the MPPCs purchased for the array, a characterisation campaign of all the photosensor modules was conducted. The main objectives were to measure the breakdown voltage, gain, single photoelectron (SPE) resolution, dark count rate (DCR) and cross-talk probability (CTP) for each of the 48 quad sensors. %The results obtained provide the distribution of the mentioned characteristics in a large set of samples, as needed for a full scale-up of LXe TPCs photosensor arrays, and, in parallel, inform how to organise the quads in each tile in a gain matched configuration. 
This work reports the initial results from the characterisation campaign, where a fully loaded tile was used to verify the performance of the tile electronics and sensors in cold.% The complete campaign analysis and results of all 48 MPPC modules and the fully loaded array will be the subject of a future publication.

The characterisation of the photosensors was performed in the Liquid Argon Setup (LArS) at the University of Zurich using the actual array to later be installed in Xenoscope. The setup consists of a double-walled cylindrical vessel with \SI{250}{mm} diameter sealed to a top flange by an o-ring, where several feedthroughs serve the inner space with both connections to electronics, external gas bottles or vacuum pumps. The array is suspended by stainless steel rods from a PTFE holder, which, in turn, is suspended from the top flange with rods as well. In the centre of the PTFE holder, a blue LED \mbox{($\lambda = \sim~\SI{420}{nm}$)} can be used to illuminate the array. The power transmission wires of the array (kapton-insulated, stranded copper wires), which supply the bias voltage of the SiPMs and the operational amplifier, are connected to a breakout box on the inside of the vessel. This breakout box merges all the wires of the same intended purpose together. %, i.e. all the positive voltage connecting to the pre-amplifiers together, all the negative voltage connecting to the pre-amplifiers together, and so on. 
After the merge, only five wires are needed to be routed to the outside via a potted feedthrough connected on the top flange. In turn, the coax signal cables from each tile are connected directly to another potted feedthrough and are routed to the data acquisition system (DAQ).

The inner volume of the setup is first pumped-out to avoid any residual water vapour being left in the volume and then filled with helium, which is used as a coolant gas. %, which, in the particular case of this characterisation campaign, was Helium. 
The volume is then cooled via the supply of liquid nitrogen through a copper pipe coiled around the upper part of the chamber, where it undergoes liquid-to-gas transition, cooling down the system. As the pressure of the He gas decreases during cooling, more gas is supplied to keep the pressure between 1.9 and \SI{2}{bar}. The temperature inside the vessel is regulated by the flow of liquid nitrogen through the copper coil, controlled by a flow valve connected to a proportional–integral-derivative controller.

Data is acquired with two 8-channel v1724 digitizers from CAEN, read out through a v2718 VME-PCI Optical link bridge module to an on-site computer. The ADC is programmed using a custom C++ software based on CAEN libraries, while the pulse finding, data management, and processing are done with the in-house developed PyLArS package~\cite{pylars}. The software defines a signal pulse when ADC counts exceed five times the RMS of the baseline value, tuning the integration window on a per-pulse basis.

In figure \ref{fig:results}, left, the integrated ADC-count spectrum of a dataset in dark environment is shown after basic noise cuts on the width of the identified pulses. The first peak, result of pure dark count events, is fitted with a Gaussian function to determine the value of the SPE area and SPE resolution. Based on the SPE value, the 1, 2, and 3 PE areas are shown in blue dashed lines, aligning with the correlated cross-talk peaks. In all charge-peaks a shoulder-like region is observed on the higher-area side. The shape of such pulses mimics the shape of the main peak population, albeit with higher integrated area. These events could be due to unresolved fast correlated avalanches and were similarly observed in~\cite{Gallina:2019fxt} when studying the same family of photosensors.

At temperatures between \SI{170}{} and \SI{200}{K}, a voltage scan was performed to study its effect on the measured gain and determine the breakdown voltage. A gain of $3\times10^6$ was observed at \SI{4.5}{V} above breakdown voltage (over-voltage), corresponding to an applied voltage of \SI{52.1}{V} at \SI{170}{K} and \SI{53.2}{V} at \SI{190}{K}. As part of the gain computation, the SPE resolution is also determined. It was observed that the SPE resolution mainly depends on the over-voltage (or gain), with a small dependence on temperature. For a measured gain of $3\times10^6$, the measured SPE resolution was $\sim5\%$ at \SI{170}{K} and $\sim5.5\%$ at \SI{190}{K}.

At the temperatures for which the array is expected to be operated in Xenoscope's TPC gas-phase, i.e. between \SI{190}{K} and \SI{200}{K}, long datasets in a dark environment were taken. Defining CTP as the ratio of pulses above a 1.5 PE threshold to the pulses above 0.5 PE, the DCR and CTP were computed. The results can be found in figure~\ref{fig:results} centre and right, respectively. As expected, both quantities increase exponentially with gain. The DCR is $\mathcal{O}{(1)~\si{Hz/mm^2}}$, comparable to current~(\cite{Gallina:2019fxt}) and earlier~(\cite{Baudis:2018pdv}) versions of this  photosensor type. 

While for higher temperatures the measured DCR is higher, the CTP is observed to be independent of temperature. This is a considerable improvement from previous versions of the VUV SiPM from Hamamatsu~\cite{Baudis:2018pdv}, now $<15$\% CTP at a gain of $3\times10^6$.


\begin{figure}[t]
\centering
\begin{subfigure}{0.315\textwidth}
    \centering
    \includegraphics[width=\textwidth]{Figures/new_plot_area_spectrum_small.pdf}
    %\caption{}
    %\label{fig:spectrum}
\end{subfigure}
\hfill
\begin{subfigure}{0.315\textwidth}
    \centering
    \includegraphics[width=\textwidth]{Figures/NEW_dcr_gain.pdf}
    %\caption{}
    %\label{fig:dcr_gain}
\end{subfigure}
\hfill
\begin{subfigure}{0.315\textwidth}
    \includegraphics[width=\textwidth]{Figures/NEW_ctp_gain.pdf}
    %\caption{}
    %\label{fig:ctp_gain}
\end{subfigure}

\caption{(Left): Area spectrum of a dark dataset at \SI{190}{K}, \SI{4}{V} over-voltage, where the 1, 2, and 3 photoelectron areas are shown in blue dashed lines. DCR (centre) and CTP (right) as a function of gain for \SI{190}{} and \SI{195}{K}.}
\label{fig:results}
\end{figure}