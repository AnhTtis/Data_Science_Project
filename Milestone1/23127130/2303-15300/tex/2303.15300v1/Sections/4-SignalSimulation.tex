\section{Signal simulation framework development}
\label{sec:signalsim}

To predict the expected signals at the SiPM array arising from ejected electrons at the photocathode with a flash of the xenon lamp, a simulation framework, XenoDiffusionScope~\cite{xenodiffusionscope}, was developed. The framework provides the base for the phenomenological study of electron longitudinal and transversal diffusion properties in the context of Xenoscope. The basic principles and steps of the simulation tool and its first results are described below.

\begin{figure}[t]
\centering

\begin{subfigure}{0.32\textwidth}
    \includegraphics[width=\textwidth]{Figures/results_in_tiles.pdf}
    %\caption{}
    \label{fig:simtiles}
\end{subfigure}
\hfill
\begin{subfigure}{0.32\textwidth}
    \includegraphics[width=\textwidth]{Figures/results_in_quads.pdf}
    %\caption{}
    \label{fig:simquads}
\end{subfigure}
\hfill
\begin{subfigure}{0.32\textwidth}
    \includegraphics[width=\textwidth]{Figures/results_in_hybrid1.pdf}
    %\caption{}
    \label{fig:simhybrid}
\end{subfigure}

\caption{Simulated hit patterns on the top array from a flash of the xenon lamp on the photocathode for different photosensor granularity: current configuration of 12 tiles (left), individual readout channel for each quad module (centre) and an hybrid configuration with both quad modules and individual \mbox{$6\times\SI{6}{mm^2}$} units (right).}
\label{fig:simpatterns}
\end{figure}

The code is organised in a modular fashion to simplify the change of parameters and further development of new or improved sections of the toy-Monte Carlo simulator. The xenon lamp pulse is modelled to determine the initial electron cloud distribution in space and time. Once created in the LXe volume, electrons are drifted from cathode to gate, experiencing longitudinal and transversal diffusion, whose constants are defined by the user. At this stage, the boundaries of the TPC are taken into account to not propagate any electrons outside of the field cage, as well as to include electron lifetime and extraction efficiency effects. Close to the gate, the electrons are focused on the closest centre of an element of the hexagonal mesh, a known effect based on data from Xurich II~\cite{Baudis:2020nwe}. From there, the distribution of photoelectrons in the array area is computed by considering the electron-to-photoelectron gain (also known as single electron gain) and the light collection efficiency map from each of the extracted electron positions. The final pattern is achieved by summing the light patterns of all the extracted electrons and by integrating the resulting pattern in the photosensors' sensitive area.

This simulation framework will inform future design choices with respect to tiling and granularity of the SiPMs  (figure \ref{fig:simpatterns}). With constraints on correlated parameters, it can be used to study electron diffusion properties by comparing observed with simulated hit patterns. 