\section{The top array of Xenoscope}
\label{sec:sipmarray}

%Xenoscope will run with SiPMs in its top array as its only light sensors, focusing on recording the S2 signal of the electrons drifted from the photocathode, which are produced via photoelectric effect from the UV light of a xenon flash lamp. The 
The array consists of 192 \mbox{$6\times\SI{6}{mm^2}$} multi-pixel photon counters (MPPCs). These are packaged in $2 \times 2$ quad modules (S13371-6050CQ-02 MPPCs from Hamamatsu) with a total sensitive area of \mbox{$12\times\SI{12}{mm^2}$}. The four quads are loaded onto printed circuit boards (PCBs) with push-pin connectors, named "tiles" (figures~\ref{fig:tiles} left and centre) and There are 12 tiles in the array. In total, the array has ~$36$\% of its area covered by active sensors. The collection of tiles is screwed to a stainless steel plate for mechanical stability with Polytetrafluoroethylene (PTFE) standoffs to ensure the protection of the wiring on the backside of the PCB. When assembled in Xenoscope's TPC, the array is secured by the stainless steel plate fitting into grooves in Polyamid-imide (Torlon) pillars with the photosensors facing downward. The distance from the anode to the SiPM plane is \SI{14.65}{mm}, protecting against the risk of direct discharges through the array. To ensure that the MPPC modules are safely in place and prevent the dislodging of any units, a perforated PTFE cover of \SI{3}{mm} is placed in front of the modules (figure~\ref{fig:tiles}, right). The PTFE mask is the only reflector in Xenoscope and its goal is structural integrity. Because the primary goal is to detect O($10^3$-$10^4$) PE S2 signals from a triggered pulse, light loss due to a lack of reflectors is not a concern.

\begin{figure}[b]
\centering
\begin{subfigure}{0.3\textwidth}
    \includegraphics[width=\textwidth]{Figures/tile_unloaded_2ith_ref.jpg}
    %\caption{}
    %\label{fig:unloadedtile}
\end{subfigure}
\hfill
\begin{subfigure}{0.3\textwidth}
    \includegraphics[width=\textwidth]{Figures/tile_loaded_edit.png}
    %\caption{}
    %\label{fig:loadedtile}
\end{subfigure}
\hfill
\begin{subfigure}{0.3\textwidth}
    \includegraphics[width=\textwidth]{Figures/loaded_array_nov.png}
    %\caption{}
    %\label{fig:loadedarray}
\end{subfigure}

\caption{(Left): Tile module without any MPPC modules loaded. (Centre): Tile module with four 12x12 mm$^2$ quad MPPCs. (Right) Fully loaded SiPM array with the PTFE cover.}
\label{fig:tiles}
\end{figure}

The tiles serve at the same time as holders for the SiPMs, voltage distributors and pre-amplifiers for the signals. The readout scheme is based on the design proposed in~\cite{Arneodo:2017ozr}, where the amplified output is the analogue summed signal, optimised to exclude contributions from non-triggered SiPMs. The pre-amplifier circuit is loaded with an OPA847 operational amplifier from Texas Instruments and provides a $\times$20 amplification factor to the summed signal. 

