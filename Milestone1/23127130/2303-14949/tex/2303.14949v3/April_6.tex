\documentclass[aps,pre,nofootinbib, onecolumn,showpacs,superscriptaddress,groupedaddress,rmp]{revtex4-1}
%\documentclass[doublecol]{epl2}
\usepackage{scrextend}
%\usepackage{lipsum}
%\usepackage{mathtools}
%\usepackage{cuted}
\usepackage{verbatim}
\usepackage{graphicx}
\usepackage{hyperref}
 %\usepackage{doi}
\usepackage{mathtools}% Loads amsmath
\usepackage{bm}
% This package is just for the example
\usepackage{lipsum}
\allowdisplaybreaks
\usepackage{amsmath}
\newenvironment{braced}
{\par\smallskip\hbox to\columnwidth\bgroup
	\hss$\left\{\begin{minipage}{\columnwidth}}
{\end{minipage}\right\}$\hss\egroup\smallskip}
\usepackage{xcolor}
\usepackage[makeroom]{cancel}
\usepackage{subcaption}
%\usepackage{sectsty}
\usepackage{titlesec}
\usepackage{amsmath}
\newcommand\myText[1]{\text{\scriptsize\tabular[t]{@{}l@{}}#1\endtabular}}


  \begin{document}
	\title{ Steady state vs non-Markovian dynamics in systems connected to multiple thermal reservoirs  }
	 
	\author{Vaibhav Wasnik}
	\email{wasnik@iitgoa.ac.in}
	
	\affiliation{Indian Institute of Technology, Goa}
	
	\begin{abstract}
		
	 In literature on stochastic thermodynamics it is stated that for a system connected to multiple thermal reservoirs, the transition rates between two energy levels equals the sum of transition rates corresponding to each thermal bath the system is connected to. We show that this assumption leads to an impossibility of the system attaining a steady state by considering a system connected to two thermal baths. As an alternative we  suggest splitting up the system into subsystems  with one subsystem connected to one thermal bath and another subsystem connected to a second   bath, and the rest exchanging energy with these two.  We show that the collective evolution of the system as a whole is non-Markovian. 
	 	
	  
	\end{abstract}
	
	\maketitle
 
 
 
 	 \section{ Introduction } 
 	 Traditionally  thermodynamics has  concerned itself with the behaviour of equilibrium and close to equilibrium    macroscopic systems. The microscopic origin of these  thermodynamic laws has been achieved using the tools of equilibrium statistical mechanics. 
 	 Over the past few decades  progress has been made towards extending the concepts of traditional thermodynamics to mesoscopic systems, which has led to understanding the far from equilibrium behavior of these systems \cite{stochastic1}-\cite{stochastic8}. The second law of thermodynamics  finds  itself extended to fluctuation theorems that are concerned with  individual trajectories traversed by the systems as they exchange energy with the surroundings. 
 	 
 	 
 	 Majority of works in the field are concerned with systems that evolve in a Markovian fashion. The master equation for probabilistic evolution of such a system that can take energy values labelled by $\epsilon_j$ is given by  
 	 \begin{eqnarray}
		\frac{dP(\epsilon_i)}{dt} = \sum_{\epsilon_j} W_{\epsilon_i,\epsilon_j}P(\epsilon_j) - \sum_j  W_{\epsilon_j,\epsilon_i}P(\epsilon_i) 
		\label{master_equation}
 	 \end{eqnarray}
 where $P(\epsilon_i)$ is the probability of the system to be in a state of energy $\epsilon_i$ and $W_{\epsilon_i,\epsilon_j}$ is the rate at which transitions happen from state of energy $\epsilon_j$ to $\epsilon_i$.  For systems connected to a single thermal reservoir at temperature $T$, the transition rates obey detailed balance condition
  \begin{eqnarray}
 W_{\epsilon_i,\epsilon_j} e^{-\beta \epsilon_j} = W_{\epsilon_j,\epsilon_i} e^{-\beta \epsilon_i}
 \end{eqnarray}
 where $\beta = \frac{1}{k_B T}$.	Systems in contact with multiple thermal reservoirs are not only important theoretically where they evolve to non-equilibrium steady states, but are also important from a practical viewpoint since most applications such as engines, pumps etc involve systems connected to multiple reservoirs.  Many works  on the subject, \cite{stochastic1},  \cite{demon1},   \cite{demon2}, \cite{sum_1},  \cite{multiple1}, \cite{multiple2},  of    a system in contact with $n$ thermal reservoirs at temperatures $T_k$ with $k \in [1,n]$ make an assumption   that  
 	 \begin{eqnarray}
 W_{\epsilon_i,\epsilon_j} = \sum_{k=1,n} W^k_{\epsilon_i,\epsilon_j}
 \label{sum}
 	 \end{eqnarray}
    where $ W^k_{\epsilon_i,\epsilon_j}$ is the rate at which   transitions  happen between states of energy $\epsilon_j$ to $\epsilon_i$ if the system was only in contact with a reservoir at temperature $T_k$.	
    
%     The $W^k$'s satisfy the detailed balance condition
%    \begin{eqnarray}
%     W^k_{\epsilon_i,\epsilon_j} e^{-\beta^k \epsilon_j} = W^k_{\epsilon_j,\epsilon_i} e^{-\beta^k \epsilon_i}
%    \end{eqnarray}
% 	  
 	   Eq.\ref{sum} seems to work well as an equation because  it even to lead to the second law of thermodynamics as is jotted in Appendix B.  However  is this  reason enough for validity of Eq.\ref{sum}?  
The way such a formula is justified is following. Imagine we have a system  as shown in Fig.\ref{fig:fig11}, where one part of the system is connected to a bath temperature $T_1$ and another to a bath at temperature $T_2$. If one assumes the system can have energy levels $\epsilon_i$ where $i \in [1,N]$, then in time $dt$ the   system can make a jump starting from energy level $\epsilon_i$ to $\epsilon_j$, if it exchanges    energy either with reservoir at temperature $T_1$ or reservoir $T_2$.  The probability    this happens is $W^1_{\epsilon_j, \epsilon_i}dt$ and $W^2_{\epsilon_j, \epsilon_i}dt$ respectively. Note that the probability of a  simultaneous exchange of energy from both reservoirs goes as $dt^2$ and can be neglected. Hence the probability to make a jump $P(\epsilon_i \rightarrow \epsilon_j)dt$ is
\begin{eqnarray}
P(\epsilon_i \rightarrow \epsilon_j)dt =(W^1_{\epsilon_j, \epsilon_i}+W^2_{\epsilon_j, \epsilon_i})P(\epsilon_i)dt = W_{\epsilon_j, \epsilon_i}P(\epsilon_i)dt
\end{eqnarray}
where $P(\epsilon_i)$ is the probability the system is in energy $\epsilon_i$. This leads to  the equation  

\begin{eqnarray}
 W_{\epsilon_i, \epsilon_j} = W^{1}_{\epsilon_i, \epsilon_j} + W^{2}_{\epsilon_i, \epsilon_j}
 \label{proof}
\end{eqnarray}

which is of  the form of Eq.\ref{sum}. The $W$'s are supposed to obey a local detailed balance equation
 \begin{eqnarray}
	\frac{W^{1,2}_{\epsilon_i, \epsilon_j} }{W^{1,2}_{\epsilon_j, \epsilon_i} } &&= e^{-\beta^{1,2} (\epsilon_i-\epsilon_j)} \nonumber\\
	%\sum_{i=1,3} W_{\epsilon_i, \epsilon_j} &&=0, \quad each \; j
\end{eqnarray} 
%It is this local detailed balance that justifies ${=\sum_{k=1,n}\frac{\dot{Q}^k}{T^k} = -\dot{S}_{environment}}$ in Eq.\ref{2ndlaw}
\begin{figure}
	\centering
	\includegraphics[width=0.7\linewidth]{Proof1}
	\caption{}
	\label{fig:fig11}
\end{figure}
%
%One could guess that   if $\epsilon_i> \epsilon_j$,   the rate  $W^1_{\epsilon_i, \epsilon_j}$ would represent the system taking energy from the bath. Since the bath is simply at temperature $T_1$ there should be no issue in the rate being dependent on temperature $T_1$. However, if $\epsilon_i< \epsilon_j$, then we have $W^1_{\epsilon_i, \epsilon_j}$  represents the system losing energy to the bath at temperature $T_1$.

However, now consider the limit $T_1 = T_2 = T$.  In this limit, you have the system is in contact with a thermal reservoir at temperature $T$ and the L.H.S in Eq.\ref{proof}  is then simply the transition rate for a system in contact with a reservoir at temperature $T$. But because $T_1 = T_2 = T$, we should also have that 	$W_{\epsilon_i, \epsilon_j} = W^1_{\epsilon_i, \epsilon_j} = W^2_{\epsilon_i, \epsilon_j}$. This leads to a contradiction in Eq.\ref{proof}. What has gone wrong here?   A question to ask is whether the fact that the system is also in contact with the bath at temperature $T_2$, not affect the rate at which the system loses energy to bath at temperature $T_1$? Eq.\ref{proof} can only be justified if the system being in contact with reservoir at temperature $T_2$ not affect the rate at which system loses energy to reservoir at temperature $T_1$ and vice versa. We note, that bath at temperature $T_1$ would be connected to a part of the system and the bath at temperature $T_2$ would be connected to another part of the system, since both baths cannot be connected at the same point in the system. So, the change in energy because of interaction with bath $T_1$, will affect the region attached to bath $T_1$ in time $dt$, which would then exchange energy with other parts of the system including the part connected to the bath at temperature $T_2$, during further intervals of time. Under such a case, saying the system as a whole changes energy when interacting with the bath at temperature $T_1$, which was used to derive Eq.\ref{proof}leads to a sort of coarse graining, while ignoring the details of how energy is actually absorbed by the system through exchanges between subsystems making up the system. This could also be a reason behind why we noticed a contradiction in the limit $T_1 = T_2 = T$ above. In the section II,   we show more evidence of something not being right. We  show  that  assuming Eq.\ref{proof} and assuming a steady state exists, leads to extra constraints on the transition rates, implying a steady state generically is not possible.   In section III, we try to understand  energy exchange of the system with the reservoirs, having one subsystem connected to a  the thermal bath at temperature $T_1$ and another subsystem connected to  bath at temperature $T_2$, and the rest of subsystems exchange energies with these two subsystems.  With such a construction we show that the evolution of the system as a whole is not Markovian in section III, implying possibilities for new research, such as understanding of fluctuation theorems etc for these systems.  


%
%
%We  prove this fact in section II, by first considering a simple proof by setting all thermal reservoirs to be at the same temperature and illustrating an obvious inconsistency.In the same section we then show that even when the thermal reservoirs are not at the same temperature  Eq.\ref{sum} leads to extra constraints on the $W^k_{\epsilon_i, \epsilon_j}$ which is unduly restrictive.  In section III, we try to understand such systems connected to multiple reservoirs, by considering the simplest realization as  a system made up of two points connected to two different thermal reservoirs, with each point evolving in a Markovian fashion. Considering the two points as a single system, we show that the coarse grained evolution of the system generically cannot be Markovian. This result finds a natural extension to generic systems connected to two thermal reservoirs.   We   conclude by highlighting the difficulty in framing a second law of thermodynamics, if we  only use the information contained in $P(\epsilon)$. 

\section{      $  W_{\epsilon_i, \epsilon_j} = W^1_{\epsilon_i, \epsilon_j} + W^2_{\epsilon_i, \epsilon_j}$ and existence of  steady state leads to a contradiction. } 
%
%In this section we show that Eq.\ref{sum} is inconsistent. We illustrate this through two proofs below
%
%\subsection{Proof I}
%
%
%	Let us assume that the system is in contact with two thermal reservoirs as in Fig.\ref{fig:proof1}, with temperatures $T_1$ and $T_2$ and as quoted in literature let us assume that   the transition rate obeys
%
%\begin{eqnarray}
%	W_{\epsilon_i, \epsilon_j} = W^1_{\epsilon_i, \epsilon_j} + W^2_{\epsilon_i, \epsilon_j}.
%	\label{proof1}
%\end{eqnarray}  
%Since the above should be true for any values of $T_1, T_2$, consider the limit when  $T_1 = T_2 = T$. In this limit, you have the system is in contact with a thermal reservoir at temperature $T$ and the L.H.S quantity $ W_{\epsilon_i, \epsilon_j}$ is then simply the transition rate for a system in contact with temperature $T$. But because $T_1 = T_2 = T$, we should also have that 	$W_{\epsilon_i, \epsilon_j} = W^1_{\epsilon_i, \epsilon_j} = W^2_{\epsilon_i, \epsilon_j}$. This leads to a contradiction in Eq.\ref{proof1}, implying the equation cannot be valid. 
%
%\begin{figure}
%	\centering
%	\includegraphics[width=0.7\linewidth]{proof1}
%	\caption{Graphical representation of the argument in Proof I}
%	\label{fig:proof1}
%\end{figure}
%
% 
%
%This proof  considered the limiting case   $T_1 = T_2 = T$, to show inconsistency in Eq.\ref{proof1}. Despite the fact that Eq.\ref{proof1}, cannot be considered a proper relationship if it fails for any combinations of thermal reservoirs tempertature, someone could still say that Eq.\ref{proof1} may be true atleast in certain regions of parameter values $(T_1,T_2)$ away from the limiting case  $T_1 = T_2 = T$. Below we present a proof which illustrates an inconsistency in Eq.\ref{proof1}, where this limit  $T_1 = T_2 = T$ is not assumed.  
%
%\subsection{Proof II}

	Consider a system that can have three possible energies labelled as $\epsilon_i$ with $i \in [1,2,3]$. We also assume that system   follows a Markovian evolution given by Eq.\ref{master_equation} 	Let us assume that the system is in contact with two thermal reservoirs with temperatures $T_1$ and $T_2$ and as quoted in literature let us assume that   the transition rate obeys
 	
 	\begin{eqnarray}
 W_{\epsilon_i, \epsilon_j} = W^1_{\epsilon_i, \epsilon_j} + W^2_{\epsilon_i, \epsilon_j}.
 \label{sum_2}
 	\end{eqnarray}  
 	
 	 Next, assume the local detailed balance below, along with  the   fact that the steady state exists, i.e. Eq.\ref{steady-state} is true for any $T_1$, $T_2$. Then doing some algebra after Eq.\ref{steady-state} we arrive at  equation Eq.\ref{Eq17}, where the   LHS depends on temperature $T_1$ and the RHS on temperature $T_2$, implying these have to be constant. This implies an extra constraint on the $W$'s other than a local detailed balance, which is inconsistent.  
 	
After summarizing how the proof goes in paragraph above, let us go through the actual proof.  The local detailed balance relationships are 
  \begin{eqnarray}
  \frac{W^{1,2}_{\epsilon_i, \epsilon_j} }{W^{1,2}_{\epsilon_j, \epsilon_i} } &&= e^{-\beta^{1,2} (\epsilon_i-\epsilon_j)} \nonumber\\
  %\sum_{i=1,3} W_{\epsilon_i, \epsilon_j} &&=0, \quad each \; j
  \label{constraints}
  \end{eqnarray} 
 	  we can hence write in the steady state
 	\begin{eqnarray}
  &&\sum_{ \epsilon_j, \epsilon_j \neq \epsilon_i  } [ W^1_{\epsilon_j, \epsilon_i} e^{-\beta^{1 } (\epsilon_i-\epsilon_j)}  + W^2_{\epsilon_j, \epsilon_i} e^{-\beta^{ 2} (\epsilon_i-\epsilon_j)} ] P_{steady}(\epsilon_j) \nonumber\\
  &&=   \sum_{ \epsilon_j, \epsilon_j  \neq \epsilon_i} [ W^1_{\epsilon_j, \epsilon_i} + W^2_{\epsilon_j, \epsilon_i}] P_{steady}(\epsilon_i). \nonumber\\
 \label{steady-state}
 	\end{eqnarray}
  
 %	Because of the symmetry   we see that $P(\epsilon_i)$ should be a symmetric function of  $T_1$ and $T_2$.  
 If we write  $W^{1,2}$ as a column matrix
 \begin{align*} W^{1,2}= 
 \begin{bmatrix}
  	W^{1,2}_{\epsilon_1, \epsilon_2} \\ 
   	W^{1,2}_{\epsilon_2, \epsilon_3}  \\
 	 	W^{1,2}_{\epsilon_3, \epsilon_1}   \\ 
  \end{bmatrix},  
  \end{align*}	
   the above equation   in the matrix form becomes
  
  \begin{eqnarray}
M(\beta_1, \beta_2)W^1 + M(\beta_2, \beta_1)W^2 = 0.
\label{eq8}
  \end{eqnarray}	
 	Here $M(\beta_1, \beta_2) $ is a $3 \times 3$ matrix and $M(\beta_2, \beta_1)$ is obtained from $M(\beta_1, \beta_2) $ by interchanging $\beta_1$ and $\beta_2$. We note that the  dependence of matrix $M(\beta_1,\beta_2)$ on both $\beta_1, \beta_2$ is because the $P_{steady}$'s are dependent on both $\beta_1, \beta_2$.
 	
 	 We next note that the matrix $M(\beta_1, \beta_2)$ is a continous function of $\beta_1, \beta_2$ and in the limit $\beta_1 = \beta_2 = \beta$
 	\begin{eqnarray}
 		 	M(\beta  , \beta  )   =   0.
 		 	\label{equation}
 	\end{eqnarray}	
 	This is because when  $\beta_1 = \beta_2 = \beta$, the steady state is the thermal equilibrium state and $P(\epsilon_i) \propto e^{-\beta \epsilon_i}$. 
 	
 	
 	For $\beta_1 \neq \beta_2$ we get from Eq.\ref{eq8}
 	\begin{eqnarray}
 W^1  &=& A(\beta_1, \beta_2) W^2\nonumber\\
 \label{eq7}
 	\end{eqnarray}
   where 
 	\begin{eqnarray}
A(\beta_1, \beta_2)= 	M(\beta_1, \beta_2)^{-1}M(\beta_2,\beta_1).
\label{limit}
 	\end{eqnarray}
 	
 \begin{comment}	
 	Note   despite  Eq.\ref{equation}, we see from Eq.\ref{eq7} that $A(\beta_1, \beta_1) $ is the unit matrix. We could say that this is how the limit $\beta_1 \rightarrow \beta_2$ is taken in Eq.\ref{limit}.
 	\end{comment}
The above equation assumes existence of the inverse of matrix $ M(\beta_1, \beta_2)$. This requires the matrix to be   of full rank. To show this, we will assume the matrix  $ M(\beta_1, \beta_2)$ is not of full rank and show it leads to a contradiction.  Since,
$\sum_i P_{steady}(\epsilon_i)=1$,   one of the 
probabilities can be expressed in terms of other two. Assume that we express $ P_{steady}(\epsilon_1)$   in terms of  $ P_{steady}(\epsilon_2)$ and $ P_{steady}(\epsilon_3)$. If the matrix $M(\beta_1, \beta_2)$ is not full rank, it implies that the  determinant of $M(\beta_1, \beta_2)$ equals zero. However since $M(\beta_1, \beta_2)$ depends on both $ P_{steady}(\epsilon_2)$ and $ P_{steady}(\epsilon_3)$ by construction, it would imply  a constraint between $ P_{steady}(\epsilon_2)$ and $ P_{steady}(\epsilon_3)$. This constraint would be  independent of the $W$'s, as the matrix $ M(\beta_1, \beta_2)$ is itself independent of the $W$'s. This      would then imply that $ P_{steady}(\epsilon_2)$ and $ P_{steady}(\epsilon_3)$ are related to each other through a relationship independent of the $W$'s ( and hence the system under consideration)   which is not possible for generic values of $W$'s. We have hence proven that the matrix $ M(\beta_1, \beta_2)$  is of full rank and hence Eq.\ref{limit} is consistent. 

	Now Eq.\ref{eq7} is an algebraic relationship, true for any values of the $\beta$'s. Hence if a system was in contact with two thermal reservoirs at temperatures $T_2, T_3$ we would  have 
	\begin{eqnarray}
	W^2  &=& A(\beta_2, \beta_3) W^3.\nonumber\\
	\end{eqnarray}
	%for $\beta_1 \neq \beta_2$ 	  and $\beta_2 \neq \beta_3$
Hence,
 	\begin{eqnarray}
 A(\beta_1, \beta_2) A(\beta_2, \beta_3) = A(\beta_1, \beta_3).
 	\end{eqnarray}	
 	Only way to get such a relationship for generic values of $\beta_1$, $\beta_2$, $\beta_3$ is if 
 	
 	\begin{eqnarray}
 A(\beta_1, \beta_2) = e^{F(\beta_1)}e^{-F(\beta_2)}
 	\end{eqnarray}
where $F(\beta)$ is a $3 \times 3$ matrix, or substituting in Eq.\ref{eq7} we get
	\begin{eqnarray}
W^1  &=& e^{F(\beta_1)}e^{-F(\beta_2)}  W^2\nonumber\\
\end{eqnarray}
or
\begin{eqnarray}
e^{-F(\beta_1)} W^1  &=&  e^{-F(\beta_2)}  W^2\nonumber\\
\label{Eq17}
\end{eqnarray}
Since, $\beta_1, \beta_2$ can be chosen to be arbitarily,  we get that    $ e^{-F(\beta_1 )}W^1 $ is a constant matrix independent of the value of  $\beta_1$, implying an extra constraint on the $W^k_{\epsilon_i, \epsilon_j}$'s in addition to detailed balance, which is inconsistent. This hence proves the inconsistency of Eq.\ref{sum} in case of a system with three energy levels. 

If we instead consider a system with more than three energy levels with $m > 3$, we can extend the above proof to show the inconsistency of Eq.\ref{sum}.   Using
 \begin{eqnarray}
 \frac{W^{1,2}_{\epsilon_i, \epsilon_j} }{W^{1,2}_{\epsilon_j, \epsilon_i} } &&= e^{-\beta^{1,2} (\epsilon_i-\epsilon_j)}, \; i,j \in [1,m] \nonumber\\
% \sum_{i=1,m} W_{\epsilon_i, \epsilon_j} &&=0, \quad each \; j
 \end{eqnarray} 
  we get the dimension of the corresponding vector $W^{1,2}$ would  $\frac{m^2-m}{2}$. At the steady state we similarly have 
 
 
 \begin{eqnarray}
 M(\beta_1, \beta_2)W^1 + M(\beta_2, \beta_1)W^2 = 0
 \end{eqnarray}
 	where $M(\beta_1, \beta_2) $ is a $m \times \frac{m^2-m}{2}$ matrix and $M(\beta_2, \beta_1)$ is obtained from $M(\beta_1, \beta_2) $ by interchanging $\beta_1$ and $\beta_2$. If and only if the square matrix $ M(\beta_1, \beta_2)^T  M(\beta_1, \beta_2)$ is full rank then multiplying both sides by the right inverse $M(\beta_1, \beta_2)^T (M(\beta_1, \beta_2) M(\beta_1, \beta_2)^T)^{-1}$, we get
 	
 	\begin{eqnarray}
 W^1 &&= A(\beta_1,\beta_2)W^2
 	\end{eqnarray} 	
 	where
 	
 	\begin{eqnarray}
 		&&A(\beta_1, \beta_2) \nonumber\\
 		&&= - [M(\beta_1, \beta_2)^T (M(\beta_1, \beta_2) M(\beta_1, \beta_2)^T)^{-1}]M(\beta_2, \beta_1)\nonumber\\
 	\end{eqnarray}
 	Now, following the arguments after Eq.\ref{eq7} we see that a contradiction is reached. This observation is only true if $ M(\beta_1, \beta_2)^T  M(\beta_1, \beta_2)$ is full rank. This is dependent on $ M(\beta_1, \beta_2)$ being full rank and as we have stated above, if this was not the case then we would have the $P(\epsilon)$'s would be related to each through a constraint that is  independent of the $W$'s (and hence the system under consideration)  which is inconsistent.  
 	
 	
 	The reason we reach a contradiction if steady state is assumed to exist,  is because of reaching an equation like  Eq.\ref{Eq17} in our calculations, where L.H.S depends on one temperature and R.H.S on another temperature. This happens because rate of energy exchanges between the system and a particular reservoir   at temperature $T_1/T_2$ is assumed to only depend on temperature $T_1/T_2$ respectively.  Instead of saying the full system exchanges energy with the baths, a better  way out would be to instead consider the system to be made up of multiple subsystems, with one subsystem  in contact with thermal reservoir at temperature $T_1$ and another subsystem in contact with the the thermal reservoir at temperature $T_2$ and rest of the subsystems exchanging energy with these two subsystems. We illustrate this in the section below and find that this results in the evolution of the entire system as a whole being non Markovian. 
   
  
  
  \vspace{20pt}
  
 \section{  Two thermal reservoirs connected to an extended body } 
 In this section we will show that for a system connected to multiple thermal reservoirs the evolution of the system as a whole is non-Markovian. We note that if we have two thermal reservoirs at different temperatures in contact with the system, they cannot be in contact with the system at the same point. At the bare minimum there should exist   two separate points in the system which are in contact with   the two thermal reservoirs. Consider the simplest case where our system is made up of just two points, where in our paper a point is an object connected to a thermal reservoir at a particular temperature.  One point is in contact with a thermal reservoir at temperature $T_1$, the other at temperature $T_2$. The two systems are non-interacting but can exchange energies through a thermal contact.   Each of these two points evolve in a  Markovian fashion and the question is whether generically, the system as a whole evolves in a Markovian fashion or not?  To address this question, consider an illustive example system shown in Fig.\ref{example}, where the energy at points $1$ and $2$ take only the following values   $ 0, \epsilon, 2\epsilon, 3 \epsilon,  4 \epsilon$.   The systems can exchange energy but are non-interacting   \footnote{It should be noted that this assumption is done in statistical mechanics to prove that two systems exchanging energy without interaction, reach a thermal equilibrium when both are at constant temperature. See for example Sec 4.1  in \cite{kardar}}. In calculations below the probabilities are assumed to be dependent on time.
 \begin{figure}
 	\centering
 	\includegraphics[width=0.4\linewidth]{proof3}
 	\caption{The two point system made up of two points at different temperatures, which is studied in the 'Illustrative Example' section below.}
 	\label{example}
 \end{figure}
 
 To move forward let us first assume that the evolution as a whole is Markovian for all values of $T_1, T_2$.  All probabilities are assumed to be time dependent in the presentation below. The probability for the whole system to make a jump from total energy $5 \epsilon$ to $3\epsilon$ in time $dt$  is
 \begin{eqnarray}
 P(5\epsilon \rightarrow 3 \epsilon)dt &&= [w^1_{\epsilon, 3 \epsilon} + w^2_{0, 2 \epsilon}]dt p^1(3\epsilon)p^2(2\epsilon)  \nonumber\\
 &&  + [w^2_{\epsilon, 3 \epsilon} + w^1_{0, 2 \epsilon}] dt p^1(2\epsilon)p^2(3\epsilon)    \nonumber\\
 &&+ w^1_{2\epsilon, 4 \epsilon}dt p^1(4\epsilon)p^2(\epsilon)+ w^2_{2\epsilon, 4 \epsilon}dt p^1(\epsilon)p^2(4\epsilon)\nonumber\\
  &&+ w^1_{3\epsilon, 5 \epsilon}dt p^1(5\epsilon)p^2(0)+ w^2_{3\epsilon, 5 \epsilon}dt p^1(0)p^2(5\epsilon)\nonumber\\
 &&= WW^{1,2}_{3\epsilon, 5\epsilon}P(5\epsilon) dt
 \label{3e}
 \end{eqnarray}
 
  
 
 Also the probability to make the reverse jump from total energy $ 3 \epsilon$ to $5 \epsilon$ assuming a Markovian evolution is 
 \begin{eqnarray}
 P(3\epsilon \rightarrow 5 \epsilon) dt &&=    [w^2_{3\epsilon, \epsilon} +w^1_{4\epsilon, 2\epsilon}]dt p^1(2\epsilon)p^2( \epsilon)\nonumber\\
 &&+     [w^1_{3\epsilon,\epsilon} + w^2_{4\epsilon,2\epsilon}]dt  p^1(\epsilon)p^2(2\epsilon) \nonumber\\
 &&+    [ w^2_{2\epsilon,0}  + w^1_{5\epsilon,3\epsilon}] dt p^1(3\epsilon)p^2(0)\nonumber\\      &&+[w^1_{2\epsilon,0}    
  + w^2_{5\epsilon,3\epsilon}]dt p^1(0)p^2(3\epsilon)\nonumber\\
 &&=WW^{1,2}_{5\epsilon, 3\epsilon}P(3\epsilon) dt
 \label{5e}
 \end{eqnarray}
%In the above there is no requirement that the $W$'s have to be time independent. 

%  $ = WW^{1,2}_{3\epsilon, 5\epsilon}dt[ p^1(2\epsilon)p^2( 3\epsilon)+ p^1(3\epsilon)p^2(2\epsilon) +p^1(5\epsilon)p^2(0)+p^1(0)p^2(5\epsilon) +p^1(4\epsilon)p^2( \epsilon)+ p^1(\epsilon)p^2(4\epsilon)  ]$
 \subsubsection{ $\mathbf{ T_1 = T_2 = T }$}
 
 
Now consider the case where  both points are at the same temperature, i.e. $T_1 =T_2 = T$ . We note that, since, the system as a whole is at constant temperature $T$, the evolution has to be Markovian, so that an thermal equilibrium is reached. We have that  $w^{1 }_{\epsilon,\epsilon'}= w^{  2}_{\epsilon,\epsilon'}= w_{\epsilon,\epsilon'}$, $WW^{1,2}_{\epsilon,\epsilon'} = WW_{\epsilon,\epsilon'}$, both dependent on $T$ and   $p^1(\epsilon) = p^2(\epsilon) = p(\epsilon)$. We note that the $p^1(\epsilon), p^2(\epsilon)$
are generically dependent on time and their equality is guaranteed on grounds of symmetry. 
 Then substituting $P(3\epsilon) = 2[p(3\epsilon)p(0) + p(2\epsilon)p(\epsilon)]$ and $P(5\epsilon) = 2[p(5\epsilon)p(0) + p(4\epsilon)p(\epsilon)+ p(3\epsilon)p(2\epsilon)]$ in above two equations  and demanding     L.H.S = R.H.S  for the full   course of time evolution, implies we  should have that 

\begin{itemize}
	\item $w_{\epsilon,\epsilon'}$ has to be  a function of $ \epsilon-\epsilon'$, so that Eq.\ref{5e} is satisfied, or
	\begin{eqnarray}
	w_{\epsilon,\epsilon'}&&= f(\epsilon-\epsilon') \nonumber\\
&&	\; if \; \epsilon > \epsilon'
	\end{eqnarray}
 
\end{itemize}
 But in order to get Eq.\ref{3e} to be satisfied we need the additional relationship that 

 \begin{itemize}
 	\item %If   $ \epsilon < \epsilon'$, then value of  $w_{\epsilon,\epsilon'}  $ in case  $ \epsilon'-\epsilon > E-\epsilon'$,  is twice the value of $w_{\epsilon,\epsilon'}  $  in case  $ \epsilon'-\epsilon < E-\epsilon'$. Here, $E$ is the total   energy of the system before making a transition.
 \end{itemize}     
 
 \begin{eqnarray}
%w_{\epsilon,\epsilon'} &&= f(\epsilon-\epsilon')\nonumber\\
%&& \; if \; \epsilon>\epsilon' \nonumber\\
w_{\epsilon,\epsilon'} &&= f(\epsilon-\epsilon')\nonumber\\
&& \; if \; \epsilon'>\epsilon\; \nonumber\\
&& \; \epsilon' -\epsilon < E - \epsilon'   \nonumber\\
w_{\epsilon,\epsilon'} &&= 2f(\epsilon-\epsilon')\nonumber\\
&& \; if \; \epsilon'>\epsilon \;\nonumber\\
&& \; \epsilon' -\epsilon > E - \epsilon'.   \nonumber\\
\label{fs}
 \end{eqnarray}
 Here, $E$ is the total   energy of the system before making a transition.
 
 The above two conditions give  us 
 \begin{eqnarray}
WW_{E,E'} = 2f(E-E')
 \end{eqnarray}
 
Since the entire system is exchanging energy with a a single thermal reservoir, detailed balance has to be obeyed. We get  
 \begin{eqnarray}
 \frac{f(E-E')}{f(E'-E)} = \frac{P_{eq}(E)}{P_{eq}(E')} =e^{-\beta (E-E')}.
 \label{detailed_balance}
 \end{eqnarray}
% We also find from Eq.\ref{fs} that $\frac{w_{\epsilon,\epsilon'}}{w_{\epsilon',\epsilon}}$ need not equal $  e^{-\beta (\epsilon-\epsilon')}$. 

\subsubsection{$\mathbf{T_1 \neq T_2}$}
Now, let us use what was derived above to analyze the case when $T_1 \neq T_2$. Using   Eq.\ref{5e} we get,
 \begin{eqnarray}
P(3\epsilon \rightarrow 5\epsilon)dt&&=  [f^1(2\epsilon) + f^2(2\epsilon)]dt P(3\epsilon),
 \end{eqnarray}
 however using Eq.\ref{3e} we have 
  \begin{eqnarray}
&& P(5\epsilon \rightarrow 3 \epsilon) dt = [f^1(-2\epsilon) + f^2(-2\epsilon)]dt p^1(3\epsilon)p^2(2\epsilon) \nonumber\\
 &&   +[f^1(-2\epsilon) + f^2(-2\epsilon)]dt p^1(2\epsilon)p^2(3\epsilon)    \nonumber\\
 &&+ 2 f^1(-2\epsilon)dt p^1(4\epsilon)p^2(\epsilon)+  2 f^2(-2\epsilon)dt p^1(\epsilon)p^2(4\epsilon)\nonumber\\
 &&+ 2 f^1(-2\epsilon)dt p^1(5\epsilon)p^2(0)+  2 f^2(-2\epsilon)dt p^1(0)p^2(5\epsilon)\nonumber\\
% &&= [f^1(-2\epsilon) + f^2(-2\epsilon)]dt p^1(3\epsilon)p^2(2\epsilon)    +[f^1(-2\epsilon) + f^2(-2\epsilon)]dt p^1(2\epsilon)p^2(3\epsilon)    \nonumber\\
% &&+  [f^1(-2\epsilon) + f^2(-2\epsilon)] p^1(4\epsilon)p^2(\epsilon)+  [f^1(-2\epsilon) + f^2(-2\epsilon)] p^1(\epsilon)p^2(4\epsilon)\nonumber\\
% &&+  [f^1(-2\epsilon) - f^2(-2\epsilon)] dt p^1(4\epsilon)p^2(\epsilon)+  [f^2(-2\epsilon) - f^1(-2\epsilon)]dt p^1(\epsilon)p^2(4\epsilon)\nonumber\\
 &&=[f^1(-2\epsilon) + f^2(-2\epsilon)] dt P(5\epsilon)\nonumber\\
 &&+  [f^1(-2\epsilon) - f^2(-2\epsilon)]dt[ p^1(4\epsilon)p^2(\epsilon) -p^1(\epsilon)p^2(4\epsilon)]\nonumber\\
 &&+  [f^1(-2\epsilon) - f^2(-2\epsilon)]dt[ p^1(5\epsilon)p^2(0) -p^1(0)p^2(5\epsilon)]\nonumber\\
 &&< 2 [f^1(-2\epsilon) + f^2(-2\epsilon)] dt P(5\epsilon)\nonumber\\
 \label{inequality1}
 \end{eqnarray}
 implying $ P(5\epsilon \rightarrow 3 \epsilon) dt$ cannot be written in a form $ WW^{1,2}(3\epsilon, 5\epsilon)dt P(5\epsilon)$, implying the system as a whole cannot evolve in a Markovian fashion.  We hence see that rightly assuming Markovian evolution when $T_1 = T_2$, directly leads to non Markovian dynamics for $T_1 \neq T_2$. 
 
 
 We   generalize   this observation and claim  that if the minimum value of energy at a particular site is $\epsilon_{min}$, then if $E > E'$ and $E-E'> \epsilon_{min}$ then 
\begin{eqnarray}
 P(E \rightarrow E')dt&&< 2 [f^1(E'-E) + f^2(E'-E)] dt P(E)   \nonumber\\
 P(E'\rightarrow E) dt&&= [f^1(E-E') + f^2(E-E')] dt P(E')   \nonumber\\
 \label{inequality_1}
\end{eqnarray}


\subsection{Extended body}
We can model an extended body by considering points $1$ and $2$ in contact with other points we label as $a, b, c..$ etc that are themselves not in contact with a thermal reservoir. Then the probability of the system having energy $E$ is 
\begin{eqnarray}
	P(E)=\sum_{\epsilon_1, \epsilon_2, \epsilon_a...} p^1(\epsilon_1) p^2(\epsilon_2)p^a(\epsilon_a)... \delta_{\epsilon_1+ \epsilon_2+ \epsilon_a + ... , E}.\nonumber\\
\end{eqnarray}
Since energy exchanges with the environment still happen at points $1$ and $2$,    the evaluations of $P(E\rightarrow E')$ in this case also follow everything said up until  Eq.\ref{inequality_1}, implying a non-Markovian  evolution.



\subsection{Issues with deriving Second Law.}
We now begin to understand the difficulty in deriving a second law of thermodynamics using $S_{system}=-k_B \sum_E P(E)\ln P(E)$ for this problem. For any $E,E'$ such that $E>E'$ we could atleast write 
\begin{eqnarray}
P(E' \rightarrow E)dt&&= [w^1_{E,E'} + w^2_{E,E'}] dt P(E'), \quad  \nonumber\\
\end{eqnarray}
while for $E<E'$ we get an inequality  Eq.\ref{inequality_1}. However, the only   inequality that would have guaranteed the  second law  is  instead
\begin{eqnarray}
P(E' \rightarrow E)dt&&\geq [w^1_{E,E'} + w^2_{E,E'}] dt P(E'), \quad if \; E < E' \nonumber\\
\label{inequality}
\end{eqnarray}
along with $\frac{w_{E,E'}}{w_{E',E}}= e^{-\beta (E-E')}$ as should be obvious by looking at derivation in Eq.\ref{2ndlaw}.  However  Eq.\ref{fs} and Eq.\ref{detailed_balance} tells us that  $\frac{w_{E,E'}}{w_{E',E}}$ need not equal $  e^{-\beta (E-E')}$. 


%This is because in such a case we have
%\begin{eqnarray}
%\dot{S }_{system} &&= - k_B \sum_E d_t P(E)\ln P(E) = k_B  \sum_{E,E'}[P(E\rightarrow E')- P(E'\rightarrow E)]\ln P(E) \nonumber\\
%&&= \frac{k_B}{2 }\sum_{E,E'} [P(E\rightarrow E') - P(E'\rightarrow E)]  \ln\frac{P(E)}{P(E')}\nonumber\\
%&&> \frac{k_B}{2}\sum_{E,E',\nu\in \{1,2\}} [W^\nu_{E',E}  P(E)  - W^\nu_{E,E'} P(E') ]\ln\frac{P(E)}{P(E')}\nonumber\\
%&&=\underbrace{\frac{k_B}{2}\sum_{E,E',\nu\in \{1,2\}}[W^\nu_{E',E}  P(E)  - W^\nu_{E,E'} P(E') ]\ln \frac{W^\nu_{E',E}  P(E)}{ W^\nu_{E,E'} P(E') }}_{>0}  \nonumber\\
%&&+\underbrace{\frac{k_B}{2} \sum_{E,E',\nu\in \{1,2\}}[W^\nu_{E',E}  P(E)  - W^\nu_{E,E'} P(E') ] \ln \frac{W^\nu_{E,E'}   }{ W^\nu_{E',E} } }_{=\sum_{\nu=1,2}\frac{dQ^\nu}{T^\nu} = -\dot{S}_{environment}}  \nonumber\\
%\label{2ndlaw}
%\end{eqnarray}
    To understand why $S_{system}=-k_B \sum_E P(E)\ln P(E)$ is not the 'right' entropy for a second law calculation,   note that at first 

\begin{eqnarray}
 {S}_{system} &&= -k_B \sum_E   P(E)\ln P(E)\nonumber\\
  &&=-k_B \sum_E   \sum_\epsilon p^1(E-\epsilon)p^2(\epsilon)\ln  \sum_{\epsilon'} p^1(E-\epsilon')p^2(\epsilon') \nonumber\\
&&< -k_B     \sum_{E,\epsilon} p^1(E-\epsilon)p^2(\epsilon) \ln    p^1(E-\epsilon  )p^2(\epsilon  )\nonumber\\
&& = {S}_{point \; 1 } +{S}_{point \; 2 } \nonumber\\
\end{eqnarray}
 where ${S}_{point \; 1 },{S}_{point \; 2 }$ are entropies of points $1$ and $2$ respectively. 
Looking from the lens of information theory, this implies that  , $S_{system}$ doesn't contain  the total information content of points $1$ and $2$.

 Now, consider a system of $N$ particles in contact with a thermal bath at temperature $T$. Let us say the number of particles having energy   $\epsilon_i$ is $n_i$. Then the total number of ways of realizing this arrangement is given by 
$\frac{N!}{\Pi_i n_i!}$ which in limit of large $N$  is $ e^{-N\sum_i \frac{n_i}{N}\ln \frac{n_i}{N}} = e^{-N\sum_i p_i \ln p_i}$, where $p_i = \frac{n_i}{N}$ is the probability of  finding particles with energy $\epsilon_i$. Since $-k_B\sum_i p_i \ln p_i$ is the entropy per particle we have that $S_{particles} = - k_B N\sum_i p_i \ln p_i$ is the total entropy of the particles.  If this changes in time evolution by $\triangle S_{particles}$  and since  $\triangle S_{environment} =  \frac{\triangle Q}{T}$ is the change in the entropy of the environment, where $\triangle Q $ is heat gained by the enivonment, then the second law stating that $\triangle S_{particles} + \triangle S_{environment} > 0$ is the statement that the system plus the environment evolves in to a macrostate that has maximum possible realizations. Hence, since all particles are independent of each other the relation $\frac{ \triangle S_{particles} + \triangle S_{environment}  }{N}>0 $, says that total change in entropy per particle and the entropy change in environment because of heat released by this particle on an average should be greater than zero. For Markovian systems this is mathematically realized in the derivation in Eq.\ref{2ndlaw}.  Second law  hence argues that the environment plus the system moves towards the macrostate which has maximum possible realizations. It is because the  number of ways of realizing a macrostate   is of relevance to the second law, that  the relevant entropy of composite system of points $1$ and $2 $ should be ${S}_{point \; 1 } +{S}_{point \; 2 }  $ and not $S_{system}$. $  {S}_{point \; 1 } + {S}_{point \; 2 } +  S_{environment} $   will be increasing with time as each point is in contact with a thermal bath and the  entropy change of each point plus the energy it releases into the environment increases as per arguments in this paragraph and by the derivation in Eq. \ref{2ndlaw}  considering $k$ to take only one value.  Hence, $S_{system}=-k_B \sum_E P(E)\ln P(E)$ is not the 'right' entropy for a second law calculation.

 
%The rate of change of entropy of the system is 
%
%
%\begin{eqnarray}
% \dot{S}_{system} &&= -k_B \sum_E d_t P(E)\ln P(E) =-k_B \sum_E d_t \sum_\epsilon p^1(E-\epsilon)p^2(\epsilon)\ln  \sum_{\epsilon'} p^1(E-\epsilon')p^2(\epsilon') \nonumber\\
% &&= -k_B     \sum_{E,\epsilon, \epsilon''} [W^1_{E-\epsilon, \epsilon''} p^1(\epsilon'') -W^1_{ \epsilon'',E-\epsilon} p^1(E-\epsilon ) ]p^2(\epsilon)\ln \sum_{\epsilon'}   p^1(E-\epsilon' )p^2(\epsilon' )\nonumber\\
% &&-k_B     \sum_{E,\epsilon, \epsilon''}   p^1(E-\epsilon)[W^2_{ \epsilon, \epsilon''} p^2(\epsilon'') -W^2_{ \epsilon'', \epsilon} p^2( \epsilon ) ]\ln   \sum_{\epsilon'} p^1(E-\epsilon' )p^2(\epsilon' ) \nonumber\\
%\end{eqnarray} 
%% If $T_1 = T_2$ the above can be written as 
%% \begin{eqnarray}
%% P(5\epsilon \rightarrow 3\epsilon) &&= W( - 2 \epsilon) dt [2p^1(3\epsilon) p^2(2\epsilon) +   2 p^1(2\epsilon)p^2(3\epsilon)+p^1(4\epsilon) p^2(1\epsilon) +    p^1(\epsilon)p^2(4\epsilon)  ]  >W( - 2 \epsilon)dt P(5\epsilon) \nonumber\\
% P(2\epsilon \rightarrow 5 \epsilon) &&=  W(  2 \epsilon)dt[ 2 p^1(\epsilon)p^2(\epsilon) + 2p^1(2\epsilon)p^2(0)+2 p^1(0)p^2(2\epsilon)] =2 W(3\epsilon)dt P(2\epsilon)
% \end{eqnarray}
% 
% At equilibrium detailed balance tells us that
% \begin{eqnarray}
% W^1(3\epsilon,0)p^1(0)&&= W^1(0,3\epsilon)p^1(3\epsilon)\nonumber\\
% \end{eqnarray}
% which gives
% \begin{eqnarray}
% \frac{W(3\epsilon)}{W(-3\epsilon)}&& = e^{-3\beta\epsilon}
% \end{eqnarray}
% We also have
% \begin{eqnarray}
% P(5\epsilon \rightarrow 2 \epsilon) &&= P(2\epsilon \rightarrow 5 \epsilon) \nonumber\\
% W( - 3 \epsilon) P(5\epsilon)  > W(3\epsilon)P(2\epsilon)\nonumber\\
% \end{eqnarray}
% or
% \begin{eqnarray}
% \frac{W(3\epsilon)}{W(-3\epsilon)}&& < e^{-3\beta\epsilon}
% \end{eqnarray}
% We note that despite the fact that the total probability of making the jump is the sum of rates corresponding to reservoir $1$ and reservoir $2$, the evolution of the system as we had discussed is non-Markovian if $T_1 \neq T_2$. We see that,
% \begin{eqnarray}
%\frac{ P(E'\rightarrow E)}{ P(E\rightarrow E')} &&= \frac{ \sum_{ \substack{\epsilon'' < E-\epsilon_{min} , E'-\epsilon_{min} }}  [ W^1_{E-\epsilon'', E'-\epsilon''}p^1(E'-\epsilon'')p^2(\epsilon'') +  W^2_{E-\epsilon'',E'- \epsilon''} p^2(E'-\epsilon'')p^1(\epsilon'')]}{ \sum_{ \substack{\epsilon'' < E-\epsilon_{min}  , E'-\epsilon_{min} }}  [ W^1_{E'-\epsilon'', E -\epsilon''}p^1(E -\epsilon'')p^2(\epsilon'') +  W^2_{E'-\epsilon'',E - \epsilon''} p^2(E -\epsilon'')p^1(\epsilon'')]}\nonumber\\
% &&= \frac{ \sum_{ \substack{\epsilon'' < E-\epsilon_{min} , E'-\epsilon_{min} }}  [ W^1_{E'-\epsilon'', E-\epsilon''}p^1(E'-\epsilon'')p^2(\epsilon'')e^{-\beta_1 (E-E')} +  e^{-\beta_2 (E-E')}W^2_{E'-\epsilon'',E- \epsilon''}p^2(E'-\epsilon'')p^1(\epsilon'')]}{ \sum_{ \substack{\epsilon'' < E-\epsilon_{min}  , E'-\epsilon_{min} }}  [ W^1_{E'-\epsilon'', E -\epsilon''}p^1(E -\epsilon'')p^2(\epsilon'') +  W^2_{E'-\epsilon'',E - \epsilon''} p^2(E -\epsilon'')p^1(\epsilon'')]}\nonumber\\
% &&= \alpha^1_{E',E} e^{-\beta_1 (E -E')} +  \alpha^2_{E',E}  e^{-\beta_2 (E-E')}\nonumber\\
%  &&= \alpha^1_{E',E} \frac{ P^1(E'\rightarrow E)}{ P^1(E\rightarrow E')} +  \alpha^2_{E',E}  \frac{ P^2(E'\rightarrow E)}{ P^2(E\rightarrow E')}\nonumber\\
%%  &&> 2 \sqrt{ \alpha^1_{E',E} \beta_{E',E}}e^{-\beta_1 (E'-E)/2-\beta_2 (E'-E)/2 }\nonumber\\
%% &&< e^{- \alpha_{E',E}\beta_1 (E'-E)} +    e^{- \beta_{E',E}\beta_2 (E'-E)}; \quad if \; E'>E \nonumber\\
%%  &&\gtrless    e^{-[\alpha_{E',E}\beta_1 (E'-E)  +\beta_{E',E}\beta_2 (E'-E)]}
% \end{eqnarray}
% 
% 
%% 
%% Let us try to understand the above closely. If the system has energy $E$, then every possible way of points $1$ and $2$    having energies $E_1$ and $E_2$ respectively, such that $E_1+E_2 = E$ is equally probable to any other possible way.
% 
%  Hence   $\alpha^1_{E',E}$ represents the probability of energy of the system changing from $E'$ to $E$ because point $1$ gained energy $E-E'$ and $\alpha^2_{E',E}  $ represents the probability the probability of energy of the system changing from $E'$ to $E$ because point $2$ gained energy $E-E'$. $P^{1 }(E',E)$ and $P^{2 }(E',E)$ correspond to the probabilities of the system making jumps from $E'$ to $E$ if the system is either  at  temperature $T_1$ or $T_2$ respectively.   
%
%This implies the probability of realizing a  trajectory taken by the system going through energy levels $E_1\rightarrow E_2 \rightarrow ...E_n$ to the reverse trajectory  
%\begin{eqnarray}
%\frac{P(E_1\rightarrow E_2 \rightarrow ...E_n) }{P(E_n\rightarrow E_{n-1} \rightarrow ...E_1)} = \Pi_{i=1,n-1} [ \alpha^1_{E_i,E_{i+1}}  \frac{ P^1(E_i \rightarrow E_{i+1})}{ P^1(E_{i+1}\rightarrow E_i)} +  \alpha^2_{E_i,E_{i+1}} \frac{ P^2(E_i \rightarrow E_{i+1})}{ P^2(E_{i+1}\rightarrow E_i)}]
%\label{fluctuation_theorem}
%\end{eqnarray}
%
%If $\beta_1 = \beta_2$, we get back the standard fluctuation theorem that the logarithm of the ratio of   probability of witnessing a trajectory  to the reversed trajectory equals the entropy change of the system plus surroundings along the trajectory. However we find that this changes completely once we consider a set up involving our system made up of two points connected to two different heat baths.  
%
%
%
% If we consider an extended system of three points labeled as $1$, $2$, $3$ on a line, with points $1$ and $3$ in contact with thermal reservoirs at  temperatures $T_1$ and $T_3$ respectively and point $2$ only in contact with points $1$ and $3$ , we have the probability of system having energy $E$ is 
%\begin{eqnarray}
%P(E)&&= \sum_{\substack{\epsilon_2,\epsilon_3:\\
%		\epsilon_2+\epsilon_3< E,\\
%		\epsilon_{min}<\epsilon_2,\epsilon_3}}p^1(E-\epsilon_2-\epsilon_3)p^2(\epsilon_2)p^3(\epsilon_3)= \sum_{\substack{\epsilon_1,\epsilon_2:\\
%		\epsilon_1+\epsilon_2< E,\\
%		\epsilon_{min}<\epsilon_3,\epsilon_2}}p^3(E-\epsilon_2-\epsilon_3)p^2(\epsilon_2)p^1(\epsilon_3)
%\end{eqnarray}
%We note that during an infinitesmal time  the point $2$ only exchanging energy with points $1$ and $3$ does not change the total energy of the system which only changes if points $1$ and $3$ exchange energies with the thermal reservoirs. We hence have
%Hence,
%\begin{eqnarray}
%\frac{ dP(E)}{dt}&&=\sum_{ \epsilon' \in   [\epsilon_{min}, \epsilon_{max}]} \sum_{\substack{\epsilon_2,\epsilon_3:\\
%		\epsilon_2+\epsilon_3< E,\\
%		\epsilon_{min}<\epsilon_2,\epsilon_3}}[W^1_{E-\epsilon_2-\epsilon_3, \epsilon'}p^1(\epsilon' )p^2(\epsilon_2)p^3(\epsilon_3)  
%+W^3_{E-\epsilon_2-\epsilon_3, \epsilon'}p^3(\epsilon' )p^2(\epsilon_2)p^1(\epsilon_3) ]\nonumber\\
%&&=\sum_{\epsilon_{min}<\epsilon_2<E}   [ \sum_{ \substack{
%		\epsilon_{min}< \epsilon_3< E-\epsilon_2 \\\epsilon' \in   [\epsilon_{min}, \epsilon_{max}]
%}}W^1_{E-\epsilon_2-\epsilon_3, \epsilon'}p^1(\epsilon' ) p^3(\epsilon_3)  
%+ \   \sum_{ \substack{
%		\epsilon_{min}< \epsilon'< E-\epsilon_2 \nonumber\\\epsilon_3 \in   [\epsilon_{min}, \epsilon_{max}]
%}}W^3_{E-\epsilon_2-\epsilon', \epsilon_3}p^1(\epsilon' ) p^3(\epsilon_3)]p^2(\epsilon_2) \nonumber\\
%\end{eqnarray}
%
% 
%\begin{eqnarray}
%P(E'\rightarrow E)&&= \sum_{ \substack{\epsilon''+\epsilon' < E-\epsilon_{min}  , E'-\epsilon_{min} }}  [ W^1_{E-\epsilon' -\epsilon'', E'-\epsilon' -\epsilon'' } p^1(E'-\epsilon'-\epsilon'' )p^2(\epsilon')p^3(\epsilon'')\nonumber\\
%&&+  W^3_{E-\epsilon' -\epsilon'', E'-\epsilon' -\epsilon'' } p^3(E'-\epsilon'-\epsilon'' )p^2(\epsilon')p^1(\epsilon'') ]
%\end{eqnarray}
%because $W^{1,2}_{E-\epsilon' -\epsilon'', E'-\epsilon' -\epsilon'' } = e^{-\beta^{1,2}(E-E')}W^{1,2}W^{1,2}_{E'-\epsilon' -\epsilon'', E-\epsilon' -\epsilon'' }$, we get back Eq.\ref{fluctuation_theorem}.
%
%We can hence extend the arguments above to a system instead   made up of many points or   continuous in nature while still being connected to heat baths at two points, the energy exchanges with the surroundings would happen at these two points and we would then still get the fluctuation theorem derived above. It is not difficult to see that if a system is connected to $N$ thermal reservoirs at $N$ points, the following relationship should hold
%
%\begin{eqnarray}
%\frac{P(E_1\rightarrow E_2 \rightarrow ...E_n) }{P(E_n\rightarrow E_{n-1} \rightarrow ...E_1)} = \Pi_{i=1,n-1} [\sum_{j=1,N} \alpha^J_{E_i,E_{i+1}}  \frac{ P^J(E_i \rightarrow E_{i+1})}{ P^J(E_{i+1}\rightarrow E_i)} ]
%\end{eqnarray}
%Here  $P^{J }(E',E)$  correspond to the probability of the system making jumps from $E'$ to $E$ if the entire system    is at   temperature $T_J$ and $\alpha^J_{E_i,E_{i+1}}$is the probability that the system jumped from state $E_i$ to state $E_{i+1}$ because of heat exchanged with thermal reservoir labelled by $J$. 
%% 
%%If we consider fluctuation theorems for such a system we see that the probability of the entire system to evolve is simply product of probabilities of the who individual sites evolving in a specific way. So we get that
%%\begin{eqnarray}
%%\ln \frac{P(\Pi)}{P(\Pi')} = \frac{\delta Q_1}{T_1} + \frac{\delta Q_2}{T_2} + \delta s_1 + \delta s_2
%%\end{eqnarray}
%%However now the RHS is not the total entropy of the system plus surroundings. This is because in the steady state each term on the RHS is zero. However the system is not in a equilibrium and hence cannot be a 
%%
%% 

\section*{IV. Discussion} 
The Markovian master equation  in Eq.\ref{master_equation} has a unique steady state solution $P_{steady}(\epsilon_i)$  such that

 \begin{eqnarray}
  \sum_{\epsilon_j} W_{\epsilon_i,\epsilon_j}P_{steady}(\epsilon_j) - \sum_{\epsilon_j} W_{\epsilon_j,\epsilon_i}P_{steady}(\epsilon_i) = 0\nonumber\\
\end{eqnarray}
 
 The uniqueness of $P_{steady}(\epsilon_i)$   is guaranteed by the linearity of the above set of  equations. The entropy production rate is written as \cite{seifert2}
\begin{eqnarray}
&&\frac{k_B}{2}\sum_{\epsilon_i, \epsilon_j} [W_{\epsilon_i, \epsilon_j}P_{steady}(\epsilon_j) - W_{\epsilon_j, \epsilon_i} P_{steady}(\epsilon_i)]\nonumber\\
 &&\times \ln \frac{ W_{\epsilon_i, \epsilon_j}P_{steady}(\epsilon_j)}{  W_{\epsilon_j, \epsilon_i} P_{steady}(\epsilon_i)}.\nonumber\\  
\label{entropy_production_rate}
\end{eqnarray}

We know that in case the system is connected to a single thermal reservoir, the steady state corresponds to the equilibrium state that obeys detailed balance and hence the entropy production at equilibrium is zero.  For a system connected to multiple reservoirs, be it thermal or particle reservoirs etc,    if a Markovian evolution is assumed  the above entropy production is greater than zero at the steady state and this is used to derive thermodynamic uncertainity relations \cite{seifert1}, \cite{seifert2}, \cite{currents_nature}
What we have shown in our work is that atleast for systems connected to multiple thermal reservoirs we cannot use Eq.\ref{sum} to evaluate the rate of entropy production. We have also shown that the evolution of such systems may not be Markovian hence Eq.\ref{entropy_production_rate} is not the rate of entropy production in the steady state.   Because of   non-Markovian evolution one cannot extend known ideas such as fluctuation theorems to such systems as a whole, despite the fact that parts of the system can individually satisfy fluctuation theorems. Since $S_{system}$ is not relevant to the second law, the question naturally arises if any other entropy definition that is dependent on probability distribution of the system as a whole could be used to construct a second law.  We also note that the environment for a particular point also includes the other point and hence the $S_{environment}$ as discussed in the above calculation does not consider just the energy exchanged by the composite system of points $1$ and $2$ with the environment, which is the universe minus the points $1$ and $2$, but also the energy exchanged by points $1$ and $2$ with each other. It is however, the universe minus the points $1$ and $2$ which would be of relevance if we were to frame a second law by constructing a entropy definition that only utilized $P(E)$. How  to accomplish this is an open question   and would require further research. 

%\section*{V. Acknowledgements}
%We would like to thank Prof Ranjan Mukhopadhyay for discussions on statistical thermodynamics. 
% 
 
 
 \appendix
 \section{  }
 
 Eq.7 can be written as 
 	or
 \begin{eqnarray}
 &&\sum_{ \epsilon_j, \epsilon_j \neq \epsilon_i  } [ W^1_{\epsilon_j, \epsilon_i} [e^{-\beta^{1 } (\epsilon_i-\epsilon_j)}P_{steady}(\epsilon_j) - P_{steady}(\epsilon_i) ] \nonumber\\
 && + \sum_{ \epsilon_j, \epsilon_j \neq \epsilon_i  } W^2_{\epsilon_j, \epsilon_i} [e^{-\beta^{ 2} (\epsilon_i-\epsilon_j)}  P_{steady}(\epsilon_j) - P_{steady}(\epsilon_i)] =   0 \nonumber\\
 \end{eqnarray}	
 
 $M(\beta_1, \beta_2) =
 \begin{pmatrix}
 0 & a_{	\epsilon_1, \epsilon_2}e^{\beta_1(\epsilon_1-\epsilon_2)} & a_{	\epsilon_1, \epsilon_3}e^{\beta_1 (\epsilon_1-\epsilon_3)}\\
 a_{	\epsilon_1, \epsilon_2}	  & 0 &  a_{	\epsilon_2, \epsilon_3}e^{\beta_1(\epsilon_2-\epsilon_3)}\\
 	 	 a_{	\epsilon_3, \epsilon_1}	 &  a_{	\epsilon_3, \epsilon_2}	 & 0\\
 \end{pmatrix}$
 where 
 \begin{eqnarray}
 a_{\epsilon_i, \epsilon_j} = [e^{-\beta^{1 } (\epsilon_i-\epsilon_j)}P_{steady}(\epsilon_j) - P_{steady}(\epsilon_i) ]
 \end{eqnarray}
 We can see that $det(M) $ is not equal to zero. 
 
 \section{ }
 	To see this note that the  rate of change of the system entropy is 
 
 \begin{eqnarray}
 	&&\dot{S}_{system} =\frac{d}{dt}[-k_B \sum_{\epsilon_i} P(\epsilon_i)\ln P(\epsilon_i)] =  -k_B \sum_{\epsilon_i} \frac{dP(\epsilon_i)}{dt} \ln P(\epsilon_i) \nonumber\\
 	&&= -k_B \sum_{\epsilon_i, \epsilon_j}[ W_{\epsilon_i, \epsilon_j}P(\epsilon_j)-W_{\epsilon_j, \epsilon_i}P(\epsilon_i)]\ln P(\epsilon_i) \nonumber\\
 	&&=\frac{k_B}{2} \sum_{\epsilon_i, \epsilon_j} [W_{\epsilon_i, \epsilon_j}P(\epsilon_j) -W_{\epsilon_j, \epsilon_i}P(\epsilon_i) ] \ln \frac{P(\epsilon_j)}{ P(\epsilon_i)}  \nonumber\\
 	&&=\underbrace{\frac{k_B}{2} \sum_{k=1,n} \sum_{\epsilon_i, \epsilon_j} [W^k_{\epsilon_i, \epsilon_j}P(\epsilon_j) -W^k_{\epsilon_j, \epsilon_i}P(\epsilon_i) ] \ln \frac{W^k_{\epsilon_i, \epsilon_j}P(\epsilon_j)}{W^k_{\epsilon_j, \epsilon_i} P(\epsilon_i)}}_{\geq 0}  \nonumber\\
 	&&+\underbrace{\frac{k_B}{2} \sum_{k=1,n} \sum_{\epsilon_i, \epsilon_j} [W^k_{\epsilon_i, \epsilon_j}P(\epsilon_j) -W^k_{\epsilon_j, \epsilon_i}P(\epsilon_i) ] \ln \frac{W^k_{\epsilon_j, \epsilon_i}  }{W^k_{\epsilon_i, \epsilon_j} } }_{=\sum_{k=1,n}\frac{\dot{Q}^k}{T^k} = -\dot{S}_{environment}}  \nonumber\\
 	\label{2ndlaw}
 \end{eqnarray}
 Hence, we see that using Eq.\ref{sum}  we get the second law of thermodynamics $ \dot{S}_{system}+ 	\dot{S}_{environment}>0$. The above argument is taken from \cite{stochastic1}.
\section*{Acknowledgements}
We would like to thank Karsten Kruse for  comments on the manuscript, Ranjan Mukhopadhyay for discussions on statistical thermodynamics and for comments on the manuscript,    M. Bhaskaran and Abitosh Upadhyay for going over the proof in section II.

\section*{Data Availability}
There is no data associated with this research. 

\section*{Conflict of Interest}
There are no conflict of interests in this research as Vaibhav Wasnik is the sole author of this paper. 

\begin{thebibliography}{9}
	\bibitem{stochastic1}
	 C. Van den Broeck and M. Esposito, Ensemble and
	trajectory thermodynamics: A brief introduction, Physica
	(Amsterdam) 418A, 6 (2015)
	\href{https://ui.adsabs.harvard.edu/link_gateway/2015PhyA..418....6V/doi:10.1016/j.physa.2014.04.035}{DOI: $
		10.1016/j.physa.2014.04.035
		$ }
	\bibitem{stochastic2}
	 C. Jarzynski, Equilibrium free-energy differences from
	nonequilibrium measurements: A master-equation ap-
	proach, Phys. Rev. E 56, 5018 (1997).
	\href{https://doi.org/10.1103/PhysRevE.56.5018}{DOI :$https://doi.org/10.1103/PhysRevE.56.5018$}
		\bibitem{stochastic3}
	G. E. Crooks, Entropy production fluctuation theorem
	and the nonequilibrium work relation for free energy
	differences, Phys. Rev. E 60, 2721 (1999).
	\href{DOI:https://doi.org/10.1103/PhysRevE.60.2721}{DOI: $https://doi.org/10.1103/PhysRevE.60.2721$}
		\bibitem{stochastic4}
	C. Jarzynski, Equalities and inequalities: Irreversibility and
	the second law of thermodynamics at the nanoscale, Annu.
	Rev. Condens. Matter Phys. 2, 329 (2011).
	\href{ https://doi.org/10.1146/annurev-conmatphys-062910-140506 }{DOI: $ https://doi.org/10.1146/annurev-conmatphys-062910-140506 $}
	\bibitem{stochastic5}
	  G. E. Crooks, Nonequilibrium measurements of free energy
	differences for microscopically reversible Markovian sys-
	tems, J. Stat. Phys. 90, 1481 (1998).
  \href{https://doi.org/10.1023/A:1023208217925 }{DOI: $ https://doi.org/10.1023/A:1023208217925$}
 \bibitem{stochastic6}
 U. Seifert, Stochastic thermodynamics, fluctuation theorems
 and molecular machines, Rep. Prog. Phys. 75, 126001
 (2012).
 \href{doi:10.1088/0034-4885/75/12/126001}{doi:10.1088/0034-4885/75/12/126001}
	\bibitem{stochastic7}
	 C. Maes, On the origin and the use of fluctuation relations
	for the entropy, Semin. Poincar e 2, 29 (2003).
	\bibitem{stochastic8}
	  K. Sekimoto, Stochastic Energetics (Springer, Berlin,
	Germany, 2010).
 \href{DOI 10.1007/978-3-642-05411-2.}{DOI 10.1007/978-3-642-05411-2.}
 
 
	\bibitem{demon1}
	Strasberg, P., Schaller, G., Brandes, T.,  $\&$  Esposito, M. (2013). Thermodynamics of a physical model implementing a Maxwell demon. Physical review letters, 110(4), 040601.
	\href{DOI:https://doi.org/10.1103/PhysRevLett.110.040601}{DOI:https://doi.org/10.1103/PhysRevLett.110.040601}
	\bibitem{demon2}
	Esposito, M., $\&$ Schaller, G. (2012). Stochastic thermodynamics for “Maxwell demon” feedbacks. EPL (Europhysics Letters), 99(3), 30003.
	\href{10.1209/0295-5075/99/30003}{DOI: 10.1209/0295-5075/99/30003}
	\bibitem{sum_1}
	Barato, A. C., $\&$ Seifert, U. (2014). Stochastic thermodynamics with information reservoirs. Physical Review E, 90(4), 042150.
	\href{DOI:https://doi.org/10.1103/PhysRevE.90.042150}{DOI: $https://doi.org/10.1103/PhysRevE.90.042150$ }
\bibitem{multiple1}
Esposito, M. (2012). Stochastic thermodynamics under coarse graining. Physical Review E, 85(4), 041125.
\href{DOI:https://doi.org/10.1103/PhysRevE.85.041125}{DOI : $  https://doi.org/10.1103/PhysRevE.85.041125$}
\bibitem{multiple2}
Proesmans, K., $\&$ Fiore, C. E. (2019). General linear thermodynamics for periodically driven systems with multiple reservoirs. Physical Review E, 100(2), 022141.
\href{DOI:https://doi.org/10.1103/PhysRevE.100.022141}{DOI $:https://doi.org/10.1103/PhysRevE.100.022141$}
  \bibitem{seifert1}
  Barato, A. C., $\&$ Seifert, U. (2015). Thermodynamic uncertainty relation for biomolecular processes. Physical review letters, 114(15), 158101.
  \href{DOI:https://doi.org/10.1103/PhysRevLett.114.158101}{DOI: $https://doi.org/10.1103/PhysRevLett.114.158101$}
  \bibitem{seifert2}
  Seifert, U. (2018). Stochastic thermodynamics: From principles to the cost of precision. Physica A: Statistical Mechanics and its Applications, 504, 176-191.
  \href{https://doi.org/10.1016/j.physa.2017.10.024}{DOI: $https://doi.org/10.1016/j.physa.2017.10.024$ }
  \bibitem{currents_nature}
  Horowitz, J. M., $\&$ Gingrich, T. R. (2020). Thermodynamic uncertainty relations constrain non-equilibrium fluctuations. Nature Physics, 16(1), 15-20.
  \href{https://doi.org/10.1038/s41567-019-0702-6}{DOI : $https://doi.org/10.1038/s41567-019-0702-6$}
  \bibitem{kardar}
     Kardar, M. (2007). Statistical physics of particles. Cambridge University Press.
\end{thebibliography} 
\end{document}

\section*{Left vs Right}
 we should have
\begin{eqnarray}
W^{1}_{\epsilon_i, \epsilon_j}&&= W^{1,L}_{\epsilon_i, \epsilon_j}+  W^{1,R}_{\epsilon_i, \epsilon_j}\nonumber\\
W^{2}_{\epsilon_i, \epsilon_j}&&= W^{2,L}_{\epsilon_i, \epsilon_j}+  W^{2,R}_{\epsilon_i, \epsilon_j}\nonumber\\
\end{eqnarray}
Where $ W^{1,L}_{\epsilon_i, \epsilon_j}, W^{1,R}_{\epsilon_i, \epsilon_j}$ are the rates at which heat enters the point $1$ from the left and right taking the point from energy $\epsilon_j$ to $\epsilon_i$. 

We should have
\begin{eqnarray}
W^{1,2}_{\epsilon_i, \epsilon_j} p^{1,2}(\epsilon_j)&&=  W^{1,2}_{\epsilon_j, \epsilon_i}  p^{1,2}(\epsilon_i)
\end{eqnarray}
or
\begin{eqnarray}
[W^{1,L}_{\epsilon_i, \epsilon_j}+  W^{1,R}_{\epsilon_i, \epsilon_j}]p_1(\epsilon_j)&&=   [W^{1,L}_{\epsilon_j, \epsilon_i}+  W^{1,R}_{\epsilon_j, \epsilon_i} ] p_1(\epsilon_j)\nonumber\\
\end{eqnarray} 
\begin{eqnarray}
[W^{2,L}_{\epsilon_i, \epsilon_j}+  W^{2,R}_{\epsilon_i, \epsilon_j}]p_2(\epsilon_j)&&=   [W^{2,L}_{\epsilon_j, \epsilon_i}+  W^{2,R}_{\epsilon_j, \epsilon_i} ] p_2(\epsilon_j)\nonumber\\
\end{eqnarray} 
% Also we have
%\begin{eqnarray}
%W^{1,R}_{\epsilon_i, \epsilon_j} &&=  W^{2,L}_{\epsilon_j, \epsilon_i}
%\end{eqnarray} 
%Hence,
% \begin{eqnarray}
%W^{1,L}_{\epsilon_i, \epsilon_j}+  W^{2,L}_{\epsilon_j, \epsilon_i}&&= e^{-\beta_{1 }(\epsilon_j-\epsilon_i)} [W^{1,L}_{\epsilon_j, \epsilon_i}+  W^{2,L}_{\epsilon_i, \epsilon_j} ]\nonumber\\
%W^{2,L}_{\epsilon_j, \epsilon_i}+  W^{2,R}_{\epsilon_j, \epsilon_i}&&= e^{-\beta_{2 }(\epsilon_i-\epsilon_j)} [W^{2,L}_{\epsilon_i, \epsilon_j}+  W^{2,R}_{\epsilon_i, \epsilon_j}]\nonumber\\
%\end{eqnarray} 

The heat from the surroundings at equilibrium entering the side with temperature $T_1$ should exit the side at temperature $T_2$
\begin{eqnarray}
&&  \sum_{\epsilon,\epsilon' } [ W^{1,L}_{\epsilon, \epsilon'}p^1_{\epsilon'} - W^{1,L}_{\epsilon', \epsilon}p^1_{\epsilon} ] [\epsilon-\epsilon']\nonumber\\ 
&=&    - \sum_{\epsilon,\epsilon' } [ W^{2,R}_{\epsilon, \epsilon'}p^2_{\epsilon'} - W^{2,R}_{\epsilon', \epsilon}p^2_{\epsilon} ] [\epsilon-\epsilon'] 
\end{eqnarray}
or
\begin{eqnarray}
&&  \sum_{\epsilon_i,\epsilon_j } [ W^{1,L}_{\epsilon_i, \epsilon_j}p^1_{\epsilon_j} +W^{2,R}_{\epsilon_i, \epsilon_j}p^2_{\epsilon_j} ]  [\epsilon_i-\epsilon_j] = 0  \nonumber\\ 
\end{eqnarray}
or
\begin{eqnarray}
&&  \sum_{\epsilon_i,\epsilon_j } [ W^{1,L}_{\epsilon_i, \epsilon_j}p^1_{\epsilon_j} -W^{2,R}_{\epsilon_j, \epsilon_i}p^2_{\epsilon_i} ]  [\epsilon_i-\epsilon_j] = 0  \nonumber\\ 
\end{eqnarray}
%similarly
%\begin{eqnarray}
%&&  \sum_{\epsilon_i,\epsilon_j } [ W^{2,L}_{\epsilon_i, \epsilon_j}p^2_{\epsilon_j} -W^{1,R}_{\epsilon_j, \epsilon_i}p^1_{\epsilon_i} ]  [\epsilon_i-\epsilon_j] = 0  \nonumber\\ 
%\end{eqnarray}
\begin{eqnarray}
W^{1,L}_{\epsilon_i, \epsilon_j}\frac{ e^{-\beta_1 \epsilon_j} }{Z_1} = W^{2,R}_{\epsilon_j, \epsilon_i}\frac{e^{-  \beta_2 \epsilon_i }}{Z_2}
\end{eqnarray}

To understand this result note that because both points $1$ and $2$ are at thermal equilibrium, we should have that every possible net flow between   states at these points to vanish.  This argument is used to get detail balance equations below.

 \begin{eqnarray}
 [W^{1,L}_{\epsilon_i, \epsilon_j}+  W^{1,R}_{\epsilon_i, \epsilon_j}]p_1(\epsilon_j)&&=   [W^{1,L}_{\epsilon_j, \epsilon_i}+  W^{1,R}_{\epsilon_j, \epsilon_i} ] p_1(\epsilon_j)\nonumber\\
 \end{eqnarray} 
 \begin{eqnarray}
 [W^{2,L}_{\epsilon_i, \epsilon_j}+  W^{2,R}_{\epsilon_i, \epsilon_j}]p_2(\epsilon_j)&&=   [W^{2,L}_{\epsilon_j, \epsilon_i}+  W^{2,R}_{\epsilon_j, \epsilon_i} ] p_2(\epsilon_j)\nonumber\\
 \end{eqnarray} 


A similar logic also implies that the flow between two states because of heat flowing from the left at any point should be cancelled by flows because of heat flowing out of the right.  Extending this argument to both the points, we see that 
 
\begin{eqnarray}
W^{1,L}_{\epsilon_j, \epsilon_i}\frac{ e^{-\beta_1 \epsilon_i} }{Z_1} = W^{2,R}_{\epsilon_i, \epsilon_j}\frac{e^{-  \beta_2 \epsilon_j }}{Z_2}
\end{eqnarray}
Hence
\begin{eqnarray}
 \frac{W^{1,L}_{\epsilon_i, \epsilon_j} }{W^{1,L}_{\epsilon_j, \epsilon_i} } e^{ \beta_1 (\epsilon_i - \epsilon_j)} = \frac{ W^{2,R}_{\epsilon_j, \epsilon_i}}{ W^{2,R}_{\epsilon_i, \epsilon_j}} e^{   \beta_2 (\epsilon_j-\epsilon_i) }
\end{eqnarray}

or
\begin{eqnarray}
\frac{W^{1,L}_{\epsilon_i, \epsilon_j} }{W^{1,L}_{\epsilon_j, \epsilon_i} } \frac{ W^{2,R}_{\epsilon_i, \epsilon_j}}{ W^{2,R}_{\epsilon_j, \epsilon_i}}   =  e^{-  \beta_2 (\epsilon_i-\epsilon_j) }  e^{-  \beta_1 (\epsilon_i-\epsilon_j) }
\end{eqnarray}
The fluctuation theorem for this system hence becomes 

\end{document}
%and
%\begin{eqnarray}
%\frac{W^{2,L}_{\epsilon_i, \epsilon_j} }{W^{2,L}_{\epsilon_j, \epsilon_i} } \frac{ W^{1,R}_{\epsilon_i, \epsilon_j}}{ W^{1,R}_{\epsilon_j, \epsilon_i}}   =  e^{-  \beta_2 (\epsilon_i-\epsilon_j) }  e^{-  \beta_1 (\epsilon_i-\epsilon_j) }
%\end{eqnarray}
%or
%\begin{eqnarray}
%\frac{W^{1,L}_{\epsilon_i, \epsilon_j}+  W^{1,R}_{\epsilon_i, \epsilon_j}}{ W^{1,L}_{\epsilon_j, \epsilon_i}+  W^{1,R}_{\epsilon_j, \epsilon_i}   }&&= e^{-\beta_{1 }(\epsilon_j-\epsilon_i)}  \nonumber\\
%\frac{W^{2,L}_{\epsilon_i, \epsilon_j}+  W^{2,R}_{\epsilon_i, \epsilon_j}}{   W^{2,L}_{\epsilon_j, \epsilon_i}+  W^{2,R}_{\epsilon_j, \epsilon_i}  }&&= e^{-\beta_{2 }(\epsilon_j-\epsilon_i)} \nonumber\\
%\end{eqnarray}
%and
%\begin{eqnarray}
% \frac{W^{1,L}_{\epsilon_i, \epsilon_j}+  W^{1,R}_{\epsilon_i, \epsilon_j}}{ W^{1,L}_{\epsilon_j, \epsilon_i}+  W^{1,R}_{\epsilon_j, \epsilon_i}   }\frac{W^{2,L}_{\epsilon_i, \epsilon_j}+  W^{2,R}_{\epsilon_i, \epsilon_j}}{   W^{2,L}_{\epsilon_j, \epsilon_i}+  W^{2,R}_{\epsilon_j, \epsilon_i}}  &=&\frac{W^{1,L}_{\epsilon_i, \epsilon_j} }{W^{1,L}_{\epsilon_j, \epsilon_i} } \frac{ W^{2,R}_{\epsilon_i, \epsilon_j}}{ W^{2,R}_{\epsilon_j, \epsilon_i}}  \nonumber\\
% = \frac{W^{1,L}_{\epsilon_i, \epsilon_j} W^{2,L}_{\epsilon_i, \epsilon_j} +W^{1,R}_{\epsilon_i, \epsilon_j} W^{2,R}_{\epsilon_i, \epsilon_j} + W^{1,R}_{\epsilon_i, \epsilon_j} W^{2,L}_{\epsilon_i, \epsilon_j} }{  W^{1,L}_{\epsilon_j, \epsilon_i} W^{2,L}_{\epsilon_j, \epsilon_i} + W^{1,R}_{\epsilon_j, \epsilon_i} W^{2,R}_{\epsilon_j, \epsilon_i}+W^{1,R}_{\epsilon_j, \epsilon_i} W^{2,L}_{\epsilon_j, \epsilon_i}    }
%\end{eqnarray} 
%\end{document}
%
%
%
%
%
%
%
%
%
%
%
%
%
%
%
%
%
%
%
%
%
%
%
%or
%\begin{eqnarray}
% e^{ \beta_{2 }(\epsilon_j-\epsilon_i)} W^{2,L}_{\epsilon_i, \epsilon_j}+ e^{ \beta_{2 }(\epsilon_j-\epsilon_i)}  W^{2,R}_{\epsilon_i, \epsilon_j}&&=   [W^{2,L}_{\epsilon_j, \epsilon_i}+  W^{2,R}_{\epsilon_j, \epsilon_i}]\nonumber\\
%\end{eqnarray}
%subtracting the above from the first equation above
%\begin{eqnarray}
% W^{1,L}_{\epsilon_i, \epsilon_j} -  W^{2,R}_{\epsilon_j, \epsilon_i}&&=  e^{-\beta_{1 }(\epsilon_j-\epsilon_i)} [W^{1,L}_{\epsilon_j, \epsilon_i}+  W^{2,L}_{\epsilon_i, \epsilon_j} ] - e^{ \beta_{2 }(\epsilon_j-\epsilon_i)} W^{2,L}_{\epsilon_i, \epsilon_j}- e^{ \beta_{2 }(\epsilon_j-\epsilon_i)}  W^{2,R}_{\epsilon_i, \epsilon_j}
%\end{eqnarray}
% 
%We hence have
%\begin{eqnarray}
%e^{-\beta_{1 }(\epsilon_j-\epsilon_i)} W^{1,L}_{\epsilon_j, \epsilon_i}   - W^{1,L}_{\epsilon_i, \epsilon_j}&&=  W^{2,L}_{\epsilon_j, \epsilon_i} - e^{-\beta_{1 }(\epsilon_j-\epsilon_i)} W^{2,L}_{\epsilon_i, \epsilon_j}   
%\end{eqnarray}
%hence,
%\begin{eqnarray}
%e^{-\beta_{1 }(\epsilon_i-\epsilon_j)} W^{1,L}_{\epsilon_i, \epsilon_j}   - W^{1,L}_{\epsilon_j, \epsilon_i}&&=  W^{2,L}_{\epsilon_i, \epsilon_j} - e^{-\beta_{1 }(\epsilon_i-\epsilon_j)} W^{2,L}_{\epsilon_j, \epsilon_i}   
%\end{eqnarray}
%multiplying by $e^{ \beta_{1 }(\epsilon_i-\epsilon_j)}$ gives
%\begin{eqnarray}
%  W^{1,L}_{\epsilon_i, \epsilon_j}   - e^{ \beta_{1 }(\epsilon_i-\epsilon_j)}W^{1,L}_{\epsilon_j, \epsilon_i}&&=  e^{ \beta_{1 }(\epsilon_i-\epsilon_j)} W^{2,L}_{\epsilon_i, \epsilon_j} -  W^{2,L}_{\epsilon_j, \epsilon_i}   
%\end{eqnarray}
% plugging in to
%  \begin{eqnarray}
% W^{2,L}_{\epsilon_i, \epsilon_j}+  W^{2,R}_{\epsilon_i, \epsilon_j}&&= e^{-\beta_{2 }(\epsilon_j-\epsilon_i)} [W^{2,L}_{\epsilon_j, \epsilon_i}+  W^{2,R}_{\epsilon_j, \epsilon_i}]\nonumber\\
% \end{eqnarray} 
% gives
%  \begin{eqnarray}
% W^{2,L}_{\epsilon_i, \epsilon_j}+  W^{2,R}_{\epsilon_i, \epsilon_j}&&= e^{-\beta_{2 }(\epsilon_j-\epsilon_i)} [   -W^{1,L}_{\epsilon_i, \epsilon_j}   +  e^{ \beta_{1 }(\epsilon_i-\epsilon_j)}W^{1,L}_{\epsilon_j, \epsilon_i}+ e^{ \beta_{1 }(\epsilon_i-\epsilon_j)} W^{2,L}_{\epsilon_i, \epsilon_j} +  W^{2,R}_{\epsilon_j, \epsilon_i}]\nonumber\\
% \end{eqnarray} 
% 
%\section*{Entropy Increase}
% 
%   The rate at which Entropy of the system changes is given by 
%   
%   \begin{eqnarray}
%\dot{S} &=&  -k_B \sum_E \dot{P(E)} \ln P(E) \nonumber\\
%&=&  -k_B \sum_E [    \sum_{\epsilon  , \epsilon' }[ W^1_{E-\epsilon', \epsilon}  +  \sum_{\epsilon, \epsilon'} W^2_{E-\epsilon, \epsilon'}  ] p^1( \epsilon ) p^2( \epsilon') \nonumber\\
%   \end{eqnarray}
%   
%   
%   or 
%   
%    \begin{eqnarray}
%   \dot{S}+\sum_\nu\frac{dQ^\nu}{T^\nu} 
%   &=&  -k_B \sum_{\epsilon  , \epsilon',E } [      W^1_{E-\epsilon', \epsilon}  +    W^2_{E-\epsilon, \epsilon'}  ] p^1( \epsilon ) p^2( \epsilon')\ln \sum_{\epsilon''}    p^1(E-\epsilon'') p^2(\epsilon'')  \nonumber\\
%   %& + &  \sum_{\epsilon , \epsilon',E } [  W^1_{  \epsilon' , E-\epsilon}   +     W^2_{ \epsilon',  \epsilon} ] p^1(E- \epsilon ) p^2(  \epsilon)  \ln \sum_{\epsilon''}    p^1(E-\epsilon'') p^2(\epsilon'') \nonumber\\ 
%   &+& \frac{k_B}{2} \sum_{\epsilon,\epsilon',\nu} [ W^\nu_{\epsilon, \epsilon'}p^\nu_{\epsilon'} - W^\nu_{\epsilon', \epsilon}p^\nu_{\epsilon} ]\ln \frac{ W^\nu_{\epsilon, \epsilon'} }{ W^\nu_{\epsilon', \epsilon}}
%   \end{eqnarray}
%   
%   hence
%     
%   \begin{eqnarray}
%   \dot{S}+\sum_\nu\frac{dQ^\nu}{T^\nu} &=&  -k_B \sum_E \dot{P(E)} \ln P(E) \nonumber\\
%   &=&  -k_B \sum_{\epsilon  , \epsilon',E } [      W^1_{E-\epsilon', \epsilon}  +    W^2_{E-\epsilon, \epsilon'}  ] p^1( \epsilon ) p^2( \epsilon')\ln \sum_{\epsilon''}    p^1(E-\epsilon'') p^2(\epsilon'')  \nonumber\\
%   %& + &  \sum_{\epsilon , \epsilon',E } [  W^1_{  \epsilon' , E-\epsilon}   +     W^2_{ \epsilon',  \epsilon} ] p^1(E- \epsilon ) p^2(  \epsilon)  \ln \sum_{\epsilon''}    p^1(E-\epsilon'') p^2(\epsilon'') \nonumber\\ 
%   &+& \frac{k_B}{2} \sum_{\epsilon,\epsilon',\nu} [ W^\nu_{\epsilon, \epsilon'}p^\nu_{\epsilon'} - W^\nu_{\epsilon', \epsilon}p^\nu_{\epsilon } ]\ln \frac{ W^\nu_{\epsilon, \epsilon'}p^\nu_{\epsilon'} }{ W^\nu_{\epsilon', \epsilon}p^\nu_{\epsilon }} \nonumber\\
%     &-& \frac{k_B}{2} \sum_{\epsilon,\epsilon',\nu} [ W^\nu_{\epsilon, \epsilon'}p^\nu_{\epsilon'} - W^\nu_{\epsilon', \epsilon}p^\nu_{\epsilon } ]\ln \frac{  p^\nu_{\epsilon'} }{  p^\nu_{\epsilon }} \nonumber\\
%      &=&  -k_B \sum_{\epsilon  , \epsilon',E } [      W^1_{E-\epsilon', \epsilon}  +    W^2_{E-\epsilon, \epsilon'}  ] p^1( \epsilon ) p^2( \epsilon')\ln \sum_{\epsilon''}    p^1(E-\epsilon'') p^2(\epsilon'')  \nonumber\\
%     %& + &  \sum_{\epsilon , \epsilon',E } [  W^1_{  \epsilon' , E-\epsilon}   +     W^2_{ \epsilon',  \epsilon} ] p^1(E- \epsilon ) p^2(  \epsilon)  \ln \sum_{\epsilon''}    p^1(E-\epsilon'') p^2(\epsilon'') \nonumber\\ 
%     &+& \frac{k_B}{2} \sum_{\epsilon,\epsilon',\nu} [ W^\nu_{\epsilon, \epsilon'}p^\nu_{\epsilon'} - W^\nu_{\epsilon', \epsilon}p^\nu_{\epsilon } ]\ln \frac{ W^\nu_{\epsilon, \epsilon'}p^\nu_{\epsilon'} }{ W^\nu_{\epsilon', \epsilon}p^\nu_{\epsilon }} \nonumber\\
%     &+ &  k_B \sum_{\epsilon,\epsilon' } [ W^\nu_{\epsilon, \epsilon'}p^1_{\epsilon'}  ]\ln    p^1_{\epsilon }  + k_B \sum_{\epsilon,\epsilon' } [ W^2_{\epsilon, \epsilon'}p^2_{\epsilon'}  ]\ln    p^2_{\epsilon } \nonumber\\
%      &=&   -  k_B  \sum_{\epsilon  , \epsilon',E } [      W^1_{E-\epsilon , \epsilon'}     p^1( \epsilon' )]p^2( \epsilon )  \ln \sum_{\epsilon''}    p^1(E-\epsilon'') p^2(\epsilon'')  \nonumber\\
%    &&  -k_B \sum_{\epsilon  , \epsilon',E } [          W^2_{E-\epsilon', \epsilon}    p^1( \epsilon' )] p^2( \epsilon)\ln \sum_{\epsilon''}    p^1(E-\epsilon'') p^2(\epsilon'')  \nonumber\\
%     %& + &  \sum_{\epsilon , \epsilon',E } [  W^1_{  \epsilon' , E-\epsilon}   +     W^2_{ \epsilon',  \epsilon} ] p^1(E- \epsilon ) p^2(  \epsilon)  \ln \sum_{\epsilon''}    p^1(E-\epsilon'') p^2(\epsilon'') \nonumber\\ 
%     &+& \frac{k_B}{2} \sum_{\epsilon,\epsilon',\nu} [ W^\nu_{\epsilon, \epsilon'}p^\nu_{\epsilon'} - W^\nu_{\epsilon', \epsilon}p^\nu_{\epsilon } ]\ln \frac{ W^\nu_{\epsilon, \epsilon'}p^\nu_{\epsilon'} }{ W^\nu_{\epsilon', \epsilon}p^\nu_{\epsilon }} \nonumber\\
%     &+ &  k_B \sum_{\epsilon,\epsilon' } [ W^1_{\epsilon, \epsilon'}p^1 ( \epsilon')  ]\ln    p^1 (\epsilon )  + k_B \sum_{\epsilon,\epsilon' } [ W^2_{\epsilon, \epsilon'}p^2(\epsilon') ]\ln    p^2(\epsilon ) \nonumber\\
%   \end{eqnarray}
%    
%    
%    Now,
%    
%    \begin{eqnarray}
%  & &      k_B  \sum_{\epsilon  , \epsilon',E } [      W^1_{E-\epsilon , \epsilon'}     p^1( \epsilon' )]p^2( \epsilon )  \ln \frac{1}{\sum_{\epsilon''}    p^1(E-\epsilon'')  p^2(\epsilon'') } \nonumber\\
%  &>&   k_B  \sum_{\epsilon  , \epsilon',E } [      W^1_{E-\epsilon , \epsilon'}     p^1( \epsilon' )]p^2( \epsilon )  \ln \frac{1}{\Pi_{\epsilon''}    p^1(E-\epsilon'')  p^2(\epsilon'') } \nonumber\\
%   &>&   k_B  \sum_{\epsilon  , \epsilon',E } [      W^1_{E-\epsilon , \epsilon'}     p^1( \epsilon' )]p^2( \epsilon )  \ln \frac{1}{    p^1(E-\epsilon'')  } \quad \; for \; any \; \epsilon''\nonumber\\
%    \end{eqnarray}
%    where the $\epsilon''$ in the second line corresponds to the value of $\epsilon''$ for which $p^1(E-\epsilon'')$ is a minimum
% \end{document}