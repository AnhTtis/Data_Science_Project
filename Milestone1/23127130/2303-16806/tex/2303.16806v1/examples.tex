\section{Motivating Examples}
\label{sec:suboptimal-example}

In this section, we present several example games that motivate our study on the emergence of \phenom{} in finitely repeated games. 

\subsection{Example Game where \PheNom{} Occurs}
\label{sec:example-off-nash}

We first present a simple example game where \phenom{} occurs to give a flavor of how such phenomena arise.

\begin{example}
\label{example:off-nash}

\begin{table}[ht]
    \centering
    \begin{tabular}{c|c|c|}
    & $a_2$ & $b_2$ \\
    \midrule
    $a_1$ & (3,1) & (0,1) \\
       \midrule
   $b_1$ & (2,1) & (1,1) \\
   \midrule
    \end{tabular}
    \caption{Example stage game $G$ where \phenom{} occurs in an SPE of $G(2)$. }
    \label{tab:off_nash}
\end{table}

\Cref{tab:off_nash} presents an example stage game $G$ where \phenom{} occurs in an SPE of the repeated game $G(2)$. The game is represented in matrix form. Row player chooses from actions $a_1$ and $b_1$, column player chooses from actions $a_2$ and $b_2$. In each entry of the matrix, the first value is the payoff of the row player, and the second value is the payoff of the column player. 

The strategy profile $(b_1, a_2)$ is not an NE of the stage game $G$. However, the following is an SPE of the 2-round repeated game $G(2)$, in which the strategy profile in the first round is $(b_1, a_2)$:
\begin{itemize}
    \item In the second round, if the row player plays $a_1$ in the first round, play $(b_1, b_2)$; else, play $(a_1, a_2)$.
    \item In the first round, play $(b_1, a_2)$.
\end{itemize}
Notice that although the row player can obtain an addition payoff of 1 in the first round by switching to play $a_1$ in the first round, they will lose a payoff of 2 in the second round. This is why the above strategy profile is an SPE of the repeated game. Intuitively, the column player `threatens' the row player by stating (implicitly through the column player's strategy) that if the row player deviates in the first round, the column player will play according to the stage-game NE that gives a lower payoff to the row player in the second round.
\end{example}


\subsection{Example Game where SPE with \PheNom{} Strictly Dominates SPEs without \PheNom{}}

Here we present an example game where in the repeated game, some SPE in which \phenom{} occurs strictly dominates all SPEs where \phenom{} does not occur. 

\begin{example}

\begin{table}[ht]
    \centering
    \begin{tabular}{c|c|c|c|}
    & $a_2$ & $b_2$ & $c_2$ \\
    \midrule
    $a_1$ & (3,3) & (0,4) & (0,0) \\
      \midrule
     $b_1$ & (4,0) & (2,2) & (0,1) \\
       \midrule
   $c_1$ & (0,0) & (1,0) & (1,1) \\
   \midrule
    \end{tabular}
    \caption{Example stage game where in the repeated game, some SPE in which \phenom{} occurs strictly dominates all SPEs where \phenom{} does not occur.}
    \label{tab:off_nash_dom}
\end{table}

\Cref{tab:off_nash_dom} presents the example stage game in matrix form. This stage game $G$ has three Nash equilibria: $(b_1, b_2)$, $(c_1, c_2)$, and a mixed NE $(\sigma_1, \sigma_2)$ where $\sigma_1(b_1) = \sigma_1(c_1) = \sigma_2(b_2) = \sigma_2(c_2) = 0.5$. The payoffs of each of the above NEs are: $(2,2)$, $(1,1)$, and $(1,1)$ respectively. Therefore, for a $T$-round repeated game $G(T)$, in any SPE where \phenom{} does not occur, the total payoff of each player is at most $2T$. We argue that for any $T>2$, the following strategy profile is an SPE of $G(T)$:
\begin{itemize}
    \item In the first $T-2$ rounds, the row player plays $a_1$ and the column player plays $a_2$ (note that this strategy profile is not an NE of the stage game $G$). If any player deviates to other actions in any round, the two players immediately switch to play $(c_1, c_2)$ for the rest of the game.
    \item In the last 2 rounds, players play $(b_1, b_2)$.
\end{itemize}

The total payoff of each player under the above SPE is $3T-2$. For all $T>2$, $3T-2 > 2T$. Therefore, the above SPE in which \phenom{} occurs strictly dominates any SPE in which \phenom{} does not occur.
\end{example}


\subsection{Example Games Demonstrating Difference Between \PheNom{} and the Property in Folk Theorems}
\label{sec:example-diff}

Under the theme of analyzing equilibrium solutions in repeated games, a large body of work focuses on Folk Theorems \cite{Benoit1985,benoit1987,gossner1995,Smith1995,gonzalez2006finitely}, where the property of interest is: all {\em feasible} (i.e., the payoff profile lies in the convex hull of the set of all possible payoff profiles of the stage game) and {\em individually rational} (i.e., the payoff of each player is at least their minmax payoff in the stage game) payoff profiles can be attained in equilibria of the repeated game. Here we show that the property considered in Folk Theorems and the \phenom{} property we consider in this research are different, and the two properties do not have direct implications in either direction. We present 1) an example game where  \phenom{} can occur, but not all feasible and individually rational payoffs can be attained in the repeated game, and 2) an example game where all feasible and individually rational payoffs can be attained in the repeated game, but \phenom{} cannot occur.

\begin{example}
\Cref{tab:example-no-folk} presents an example stage game $G$ where \phenom{} can occur in the repeated game, but not all feasible and individually rational payoffs can be attained in the repeated game. This example is taken from \cite{Benoit1985}. This game contains 3 players. Player 1 selects rows ($a_1, b_1, c_1$), player 2 selects columns ($a_2, b_2$), and player 3 selects matrices ($a_3, b_3$). While \cite{Benoit1985} analyzes this example with only pure strategies allowed, we consider the general case where mixed strategies are allowed. There are three Nash equilibria: (i) $(a_1, a_2, a_3)$; (ii) $(a_1, b_2, b_3)$; (iii) $(a_1, \sigma_2, \sigma_3)$ where $\sigma_2(a_2) = \sigma_2(b_2) = 0.5$, $\sigma_3(a_3) = 0.25$, $\sigma_3(b_3) = 0.75$. These equilibria achieve payoffs of $(3,3,3)$, $(2,2,2)$, $(1.5,1.5,1.5)$ respectively. Following a similar idea in \Cref{example:off-nash}, it is easy to construct an SPE in a repeated game where \phenom{} occurs. For example, the following strategy profile is an SPE of $G(4)$:
\begin{itemize}
    \item In the first round, play $(a_1, a_2, b_3)$.
    \item In the last three rounds, if players play in the first round is $(a_1, a_2, b_3)$, play $(a_1, a_2, a_3)$; otherwise, play $(a_1, b_2, b_3)$.
\end{itemize}
In this SPE, the first round play does not form a stage-game NE. Therefore, \phenom{} occurs.

\begin{table}[ht]
    \centering
    \begin{tabular}{c|c|c|}
     & $a_2$ & $b_2$ \\
    \midrule
     $a_1$ & (3,3,3) & (0,0,0) \\
    \midrule
    $b_1$ & (0,0,0) & (0,0,0) \\
    \midrule
    $c_1$ & (0,1,1) & (0,0,0) \\
   \midrule
   \multicolumn{3}{c}{$a_3$}
    \end{tabular}
    \quad
    \begin{tabular}{c|c|c|}
     & $a_2$ & $b_2$ \\
    \midrule
     $a_1$ & (1,1,1) & (2,2,2) \\
    \midrule
    $b_1$ & (0,1,1) & (0,1,1) \\
    \midrule
    $c_1$ & (0,1,1) & (0,0,0) \\
   \midrule
   \multicolumn{3}{c}{$b_3$}
    \end{tabular}
    \caption{Example stage game where \phenom{} can occur, but not all feasible and individually rational payoffs can be attained in the repeated game. }
    \label{tab:example-no-folk}
\end{table}

For this stage game $G$, each player's minmax payoff is 0. We follow a similar argument as \cite{Benoit1985}. Denote $w_i(T)$ as the worst payoff that player $i$ can get in any SPE of $G(T)$, the $T$-round repeated game. We claim that for $i=2,3$, $w_i(T)/T\geq 0.5$, therefore not all feasible and individually rational payoffs can be approximated. We use induction. The claim is true for $T=1$. Suppose $w_i(T-1)\geq 0.5(T-1)$. Consider the strategy profile $\vmu$ in $G(T)$ that attains $w_2(T)$ and $w_3(T)$. Notice that player 2 and 3 always get the same payoff in this game, so $w_2(T)$ and $w_3(T)$ will be attained at the same time. Consider the behavior strategy profile in the first round $\vsigma = (\sigma_1, \sigma_2, \sigma_3)$ as specified in $\vmu$. If $\sigma_1(c_1)\cdot \sigma_2(b_2) \leq 0.5$, then player 3 playing $b_3$ in the first round gives them at least a total payoff of $0.5 + w_3(T-1)$. This implies $w_3(T)\geq 0.5 + w_3(T-1)$ and we are done by the induction hypothesis. If $\sigma_1(c_1)\cdot \sigma_2(b_2) > 0.5$, then player 2 playing $a_2$ in the first round gives them at least a total payoff of $0.5+w_2(T-1)$. This implies $w_2(T)\geq 0.5 + w_2(T-1)$ and we are done by the induction hypothesis.
\end{example}

\begin{example}
\Cref{tab:every_ne} presents an example stage game $G$ where all feasible and individually rational payoffs can be attained in the repeated game, but \phenom{} cannot occur. For this stage game $G$, the set of all feasible and individually rational payoff profiles is $\condSet{(u_1, u_2)}{0\leq u_1 \leq 1, 0\leq u_2 \leq 1}$. All these feasible and individually rational payoffs can be attained in the stage game $G$ itself, which is also a one-round repeated game $G(1)$. But every possible strategy profile in $G$ is an NE of $G$, so \phenom{} cannot occur.

\begin{table}[ht]
    \centering
    \begin{tabular}{c|c|c|}
    & $a_2$ & $b_2$ \\
    \midrule
    $a_1$ & (0,0)   & (1,0) \\
    \midrule
    $b_1$ & (0,1)  & (1,1) \\
   \midrule
    \end{tabular}
    \caption{Example stage game where every possible strategy profile is an NE.}
    \label{tab:every_ne}
\end{table}

\end{example}


\subsection{Example Game where \PheNom{} Only Occurs with Large $T$}
\label{sec:example-largeT}

One of the research questions we consider is in the computational aspect: given an arbitrary stage game $G$, how to (algorithmically) decide if there exists some $T$ and some SPE of $G(T)$ where \phenom{} occurs? A naive approach is to enumerate over $T$, solve for all SPEs for each $G(T)$, and check if off-(stage-game)-Nash behavior occurs. Here, we present a construction of games where \phenom{} only occurs with arbitrarily large $T$. This means that the naive approach above might need to check an arbitrarily large number of $T$'s before returning a result.

\begin{example}

\begin{table}[ht]
    \centering
    \begin{tabular}{c|c|c|}
    & $a_2$ & $b_2$ \\
    \midrule
    $a_1$ & (3,2)   & ($\alpha$,1) \\
      \midrule
     $b_1$ & (3,2)   & (2,2) \\
      \midrule
      $c_1$ & ($\alpha$,1)  & (2,2) \\
       \midrule
    \end{tabular}
    \caption{Example stage game where \phenom{} only occurs with large $T$.}
    \label{tab:large_T}
\end{table}

\Cref{tab:large_T} presents a construction of stage games where \phenom{} only occurs with arbitrarily large $T$. We claim that for any $\alpha < 2$, \phenom{} cannot occur with any $T < \frac{1}{2}(2-\alpha)$, and \phenom{} can occur with any $T > 3 - \alpha$. Therefore, as $\alpha$ becomes smaller, we have games where \phenom{} only occurs with arbitrarily large $T$. We present the proof as follows.

It is easy to see that the set of Nash equilibria of this stage game $G$ is
\begin{itemize}
    \item $(\sigma_1, a_2)$ where $\sigma_1(a_1) = \lambda, \sigma_1(b_1) = 1 - \lambda$ for all $0\leq \lambda \leq 1$,
    \item $(b_1, \sigma_2)$ where $\sigma_2(a_2) = \lambda, \sigma_2(b_2) = 1 - \lambda$ for all $0\leq \lambda \leq 1$,
    \item $(\sigma_1, b_2)$ where $\sigma_1(b_1) = \lambda, \sigma_1(c_1) = 1 - \lambda$ for all $0\leq \lambda \leq 1$.
\end{itemize}
It follows that $V_1 = [2,3]$, the continuous range from 2 to 3, and $V_2 = \{2\}$. 

First, we show that for any $T > 3 - \alpha$, \phenom{} can occur. Here is an SPE where \phenom{} occurs:
\begin{itemize}
    \item The first round strategy profile is $(\hat{\sigma}_1, b_2)$ where $\hat{\sigma}_1(a_1) = \frac{1}{4}$ and $\hat{\sigma}_1(c_1) = \frac{3}{4}$. $(\hat{\sigma}_1, b_2)\notin \Nash(G)$ since player 1 is not playing a best response (but player 2 is playing a best response).
    \item If player 1's first round play is $b_1$ or $c_1$, we let the players play a stage game Nash equilibrium that achieves $u_1=2$ (minimum payoff for player 1) in all the remaining $T-1$ rounds. 
    \item If player 1's first round play is $a_1$, we let the players play a sequence of stage game Nash equilibria that achieves a total payoff $U_1 = 2(T-1) + 2 - \alpha$ in the remaining $T-1$ rounds. This is possible since $T-1 > 2 - \alpha$ and $V_1$ contains the continuous interval between 2 and 3.
\end{itemize}

Now let $T^*$ be the smallest $T$ such that \phenom{} can occur in $G(T)$. Let $\vmu^*$ to be any SPE of $G(T^*)$ where \phenom{} occurs. Denote $\vsigma^* = (\sigma_1^*, \sigma_2^*)$ to be the first round strategy profile in $\vmu^*$. It follows that $\vsigma^*\notin \Nash(G)$, and all strategy profiles in all later rounds in $\vmu^*$ belongs to $\Nash(G)$. Since $|V_2| = 1$, player 2 must play a best response in $\vsigma^*$. Therefore, $\sigma_1^*$ must assign positive probabilities in both $a_1$ and $c_1$, since otherwise $\vsigma^*\in \Nash(G)$. For $\sigma_2^*$, either $\sigma_2^*(a_2)\geq 0.5$ or $\sigma_2^*(b_2)\geq 0.5$. If $\sigma_2^*(b_2)\geq 0.5$, we have $u_1(b_1, \sigma_2^*) - u_1(a_1, \sigma_2^*) \geq 0.5(2 - \alpha)$. Denote $U_1(\vmu^*_{|b_1})$ as the expected total payoff for player 1 in the last $T^*-1$ rounds given player 1 plays $b_1$ in the first round, and similarly for $U_1(\vmu^*_{|a_1})$. For $\vmu^*$ to be an SPE, we must have $U_1(\vmu^*_{|a_1}) - U_1(\vmu^*_{|b_1}) \geq u_1(b_1, \sigma_2^*) - u_1(a_1, \sigma_2^*) \geq 0.5(2-\alpha)$. But we also have $U_1(\vmu^*_{|a_1}) - U_1(\vmu^*_{|b_1}) \leq 3(T^*-1) - 2(T^* - 1) = T^* - 1$, so $T^* > 0.5(2-\alpha)$. The same argument can be applied to the case where $\sigma_2^*(a_2)\geq 0.5$. Therefore, \phenom{} cannot occur with any $T < \frac{1}{2}(2-\alpha)$.

    
\end{example}
