\section{The General Case}
\label{sec:mixed}

Now we consider the general case for 2-player games where mixed strategies are allowed. 
%We use $\Nash^{m,m}(G)$ to denote the set of Nash equilibria of the stage game $G$ when both players can use mixed strategies, and $V_i^{m,m} = \condSet{u_i(\vsigma)}{\vsigma\in\Nash^{m,m}(G)}$ to denote the set of payoff values attainable at stage game Nash equilibria for player $i$ when both players can use mixed strategies. 
We use $\Nash^{m,m}(G)$, $\spe^{m,m}(G,T)$, and $V_i^{m,m}$ to denote the corresponding concepts of $\Nash(G)$, $\spe(G,T)$, and $V_i$ when both players can use mixed strategies.
Since mixed Nash equilibrium always exists for $G$ \cite{nash1951non}, $|V_1^{m,m}|\geq 1$, $|V_2^{m,m}|\geq 1$. We use $\glsmm$ and $\glomm$ to denote the partition of the set of all stage games $\mathcal{G}$ when both players can use mixed strategies. The following theorem presents a complete mathematical characterization of $\glsmm$.


\begin{theorem} [2-player, general case]
\label{thm:general}
For general 2-player games (mixed strategies allowed),
a sufficient and necessary condition on the stage game $G$ for there exists some $T$ and some SPE of $G(T)$ where \phenom{} occurs is:
\begin{enumerate}
    \item $|V_1^{m,m}| > 1$, $|V_2^{m,m}| > 1$, and there exists some $\hat{\sigma}_1 \in \Delta A_1, \hat{\sigma}_2\in \Delta A_2$ where $(\hat{\sigma}_1, \hat{\sigma}_2) \notin \Nash^{m,m}(G)$, OR
    \item $|V_1^{m,m}| > 1$, $|V_2^{m,m}| = 1$, and there exists $\hat{\sigma}_1\in \Delta A_1, \hat{\sigma}_2 \in \Delta A_2, a_1'\in A_1$ where $u_1(\hat{\sigma}_1, \hat{\sigma}_2) < u_1(a_1', \hat{\sigma}_2)$ and $\hat{\sigma}_2$ is a best response to $\hat{\sigma}_1$, OR
    \item same as (2) but exchange player 1 and 2.
\end{enumerate}
\end{theorem}

We first establish some useful lemmas.

\begin{lemma}
\label{lemma:mixed-pure}
For any two-player game $G$, if there exists $\sigma_1\in\Delta A_1$ and $\sigma_2\in\Delta A_2$ where $(\sigma_1, \sigma_2)\notin \Nash^{m,m}(G)$, then there exists $a_1\in A_1$ and $a_2\in A_2$ where $(a_1, a_2) \notin \Nash^{m,m}(G)$.
\end{lemma}

\begin{proof}
$(\sigma_1, \sigma_2)\notin \Nash^{m,m}(G)$ implies that there exists some $a_1'\in A_1$ where $u_1(\sigma_1, \sigma_2) < u_1(a_1', \sigma_2)$, or there exists some $a_2'\in A_2$ where $u_2(\sigma_1, \sigma_2) < u_2(\sigma_1, a_2')$. We consider the case of there exists some $a_1'$ where $u_1(\sigma_1, \sigma_2) < u_1(a_1', \sigma_2)$, and the same argument applies to the other case. Since $u_1(\sigma_1, \sigma_2) \geq \min_{a_1\in S_{\sigma_1}} u_1 (a_1, \sigma_2)$, there exists some $a_1, a_1', \sigma_2$ where $u_1(a_1, \sigma_2) < u_1(a_1', \sigma_2)$. So $u_1(a_1', \sigma_2) - u_1(a_1, \sigma_2) = \sum_{a_2\in S_{\sigma_2}} \sigma_2(a_2)\cdot \Big(u_1(a_1', a_2) - u_1(a_1, a_2) \Big) > 0$, which means there exists some $a_2$ where $u_1(a_1', a_2) - u_1(a_1, a_2) > 0$. This $(a_1, a_2)\notin \Nash^{m,m}(G)$, which finishes the proof.
\end{proof}


\begin{lemma}
\label{lemma:matrix}
For any two-player game $G$, define 
\begin{align*}
    I = \condSet{(i,j)}{i\in A_1, j\in A_2, u_2(i, j) = \max_{j'\in A_2} u_2(i, j')}
\end{align*}
as the set of pure strategy profiles where player 2 plays a best response. If
\begin{enumerate}
    \item $|V_1^{m,m}| > 1$, $|V_2^{m,m}|=1$, and
    \item there does not exist $\hat{a}_1\in A_1, \hat{\sigma}_2 \in \Delta A_2, a_1'\in A_1$ where $u_1(\hat{a}_1, \hat{\sigma}_2) < u_1(a_1', \hat{\sigma}_2)$ and $\hat{\sigma}_2$ is a best response to $\hat{a}_1$, and
    \item there does not exist $a_1\in A_1$ and $a_2, a_2' \in A_2$ where $a_2\neq a_2'$ and both $a_2$ and $a_2'$ are best responses to $a_1$,
\end{enumerate}
then, 
\begin{itemize}
    \item [(a).] $I\subseteq \Nash^{m,m}(G)$,
    \item [(b).] for each $i\in A_1$, there is a unique $j\in A_2$ such that $(i,j)\in I$,
    \item [(c).] there exists $b\in \real$ such that for all $(i,j)\in I$, $u_2(i, j) = b$, and for all $(i',j')\notin I$, $u_2(i', j') < b$,
    \item [(d).] there exists $(i,j), (i',j')\in I$ such that $u_1(i, j) \neq u_1(i',j')$, and $i\neq i'$, $j\neq j'$.
\end{itemize}
\end{lemma}

\begin{proof}
(2) directly implies (a). From the definition of $I$, for each $i\in A_1$, there is at least one $j\in A_2$ such that $(i,j)\in I$. This combines with (3) implies (b).

Since $|V_2^{m,m}|=1$ and $I\subseteq \Nash^{m,m}(G)$, for all $(i,j)\in I$, $u_2(i,j)=b$ where $b$ is the only element in $V_2^{m,m}$. It then follows from the definition of $I$ that for all $(i',j')\notin I$, $u_2(i', j') < b$. So (c) follows.

For (d), assume in contradiction that all $u_1(i,j)$ for $(i,j)\in I$ are the same. Since $|V_1^{m,m}|>1$, there exists $\vsigma, \vsigma'\in \Nash^{m,m}(G)$ such that $u_1(\vsigma) \neq u_1(\vsigma')$. Denote $\mathcal{S}_{\vsigma}$ as the set of pure strategy profiles $(i,j)$ that occur with non-zero probability under the strategy profile $\vsigma$. Then at least one of $\mathcal{S}_{\vsigma}$ and $\mathcal{S}_{\vsigma'}$ needs to contain elements not in $I$, since otherwise $u_1(\vsigma) = u_1(\vsigma')$. WLOG, let $\mathcal{S}_{\vsigma}$ contain elements not in $I$. By (c), $u_2(\vsigma) < b$, which contradicts with $|V_2^{m,m}|=1$. So there exists $(i,j), (i',j')\in I$ such that $u_1(i, j) \neq u_1(i',j')$. For such $(i,j), (i',j')$, if $i=i'$, then (b) implies that $j=j'$, which contradicts with $u_1(i, j) \neq u_1(i',j')$. So $i\neq i'$. And since $I\subseteq \Nash^{m,m}(G)$ (due to (a)), $(i,j), (i',j')$ are both NEs with different payoffs for player 1, so $j\neq j'$. Therefore, (d) follows. 

\end{proof}

Now we are ready to prove \Cref{thm:general}.

\begin{proof}[Proof of \Cref{thm:general}]

Following the same argument as the proof for the pure strategy case (\Cref{thm:pure}), we can show the condition is necessary.

We prove the condition is sufficient by showing if the condition is satisfied, we can always construct some $T$ and some SPE where \phenom{} occurs. If the condition is satisfied, then at least one of (1),(2),(3) must be satisfied. We consider each case here.

If (1) is satisfied, there exists some $\hat{\sigma}_1 \in \Delta A_1, \hat{\sigma}_2\in \Delta A_2$ where $(\hat{\sigma}_1, \hat{\sigma}_2) \notin \Nash^{m,m}(G)$. By \Cref{lemma:mixed-pure}, there exists $\hat{a}_1\in A_1$ and $\hat{a}_2\in A_2$ where $(\hat{a}_1, \hat{a}_2) \notin \Nash^{m,m}(G)$. Then we can use the same construction that is used in the proof of the pure strategy case here (see the proof of sufficiency in \Cref{thm:pure}, the part that handles the case where (1) is satisfied).

The rest of the proof focus on the case when (2) is satisfied. The same argument applies for the case where (3) is satisfied. We first notice that all games that satisfy (2) can be categorized into the following 3 disjoint cases: 
\begin{itemize}
    \item[(a).] There exists $\hat{a}_1\in A_1, \hat{\sigma}_2 \in \Delta A_2, a_1'\in A_1$ where $u_1(\hat{a}_1, \hat{\sigma}_2) < u_1(a_1', \hat{\sigma}_2)$ and $\hat{\sigma}_2$ is a best response to $\hat{a}_1$.
    \item[(b).] (a) is false, and there exists $a_1\in A_1$ and $a_2, a_2' \in A_2$ where $a_2\neq a_2'$ and both $a_2$ and $a_2'$ are best responses to $a_1$.
    \item[(c).] Both (a) and (b) are false.
\end{itemize}
We consider each case here.
\\
\\
\textbf{Case (a).} We can use the same construction that is used in the proof of sufficiency for the pure strategy case (\Cref{thm:pure}), the part that handles the case where (2) is satisfied.
\\
\\
\textbf{Case (b).} (a) is false implies that for all $a_1\in A_1$, for all $\sigma_2\in \Delta A_2$ that is a best response to $a_1$, $a_1$ is also a best response to $\sigma_2$. \Cref{tab:no_best_response} is an example of such games. We know that there exists some $a^*\in A_1$, $b_1^*, b_2^* \in A_2$ where $b_1^*\neq b_2^*$ and both $b_1^*$ and $b_2^*$ are best responses to $a^*$. Therefore, $a^*$ is a best response to both $b_1^*$ and $b_2^*$. WLOG, let $u_1(a^*, b_1^*) \geq u_1(a^*, b_2^*)$. Denote $\sigma_{\lambda}\in \Delta A_2$ as the mixed strategy for player 2 which assigns $\sigma_{\lambda}(b_1^*) = \lambda$ and $\sigma_{\lambda}(b_2^*) = 1 - \lambda$. Then for all $0\leq \lambda \leq 1$, $(a^*, \sigma_{\lambda})\in \Nash^{m,m}(G)$.

Since (2) is satisfied, there exists $\hat{\sigma}_1\in \Delta A_1, \hat{\sigma}_2 \in \Delta A_2, a_1'\in A_1$ where $u_1(\hat{\sigma}_1, \hat{\sigma}_2) < u_1(a_1', \hat{\sigma}_2)$ and $\hat{\sigma}_2$ is a best response to $\hat{\sigma}_1$. We construct $T = 1 + T_1 + T_2$ and a strategy profile $\vmu^*$ for $G(T)$ with the following structure:
\begin{itemize}
    \item In the first round, play $(\hat{\sigma}_1, \hat{\sigma}_2)$.
    \item For the later rounds, if player 1's first round play is $i\in S_{\hat{\sigma}_1}$, players play their corresponding strategies according to SPE $\vmu^i$ of $G(T-1)$; otherwise, players play their corresponding strategies according to SPE $\vmu^{\bot}$ of $G(T-1)$.
\end{itemize}

Since $|V_1^{m,m}| > 1$, let $\vsmin, \vsmax\in \Nash^{m,m}(G)$ such that $u_1(\vsmin) = \min (V_1^{m,m})$ and $u_1(\vsmax) = \max (V_1^{m,m})$, so $u_1(\vsmax) > u_1(\vsmin)$. We construct the SPEs $\vmu^{\bot}, \{\vmu^i\}_{i\in S_{\hat{\sigma}_1}}$ as follows:

\begin{itemize}
    \item $\vmu^{\bot}$ is players playing $\vsmin$ repeatedly for $T_1 + T_2$ rounds.
    \item For all $\vmu^i$, the last $T_2$ rounds consist of players repeated playing $\vsmax$. $T_2$ is chosen to be large enough such that $U_1(\vmu^i) - U_1(\vmu^{\bot}) > \max_{a,a'\in A_1} u_1(a,\hat{\sigma}_2) - u_1(a',\hat{\sigma}_2)$ for every $i\in S_{\hat{\sigma}_1}$. This makes sure that in $\vmu^*$, player 1 deviating to any $i\notin S_{\hat{\sigma}_1}$ in the first round will reduce their total payoff in $G(T)$.
    \item The first $T_1$ rounds strategies for each $\vmu^i$ adopt the following structure:
    \begin{itemize}
        \item In the first round, play $(a^*, \sigma_{\lambda_i})$, where $\lambda_i$ is a parameter to be set for each $i$.
        \item In the latter $T_1 - 1$ rounds, if player 2 plays $b_1^*$ in the first round, players repeatedly play $\vsmax$; otherwise, players repeatedly play $\vsmin$. 
    \end{itemize}
    Pick $i^m \in S_{\hat{\sigma}_1}$ such that $u_1(i^m, \hat{\sigma}_2) = \max_{i \in S_{\hat{\sigma}_1}} u_1(i, \hat{\sigma}_2)$. We set $\lambda_{i^m} = 0$. For each $i\in S_{\hat{\sigma}_1} \setminus \{i^m\}$, we set $\lambda_i$ such that $U_1(\vmu^i) + u_1(i,\hat{\sigma}_2) = U_1(\vmu^{i^m}) + u_1(i^m,\hat{\sigma}_2)$. This makes sure that in $\vmu^*$, player 1 choosing any $i\in S_{\hat{\sigma}_1}$ in the first round will obtain the same total payoff in $G(T)$. We argue that with large enough $T_1$, such choice of $\lambda_i$'s is always possible. Consider the difference between two sides of the equation as a function of $\lambda_i$, $f(\lambda_i) = U_1(\vmu^i) - U_1(\vmu^{i^m}) + u_1(i,\hat{\sigma}_2) - u_1(i^m,\hat{\sigma}_2)$. By choosing $T_1 \geq \frac{\max_{i\in S_{\hat{\sigma}_1}} u_1(i^m, \hat{\sigma}_2) - u_1(i, \hat{\sigma}_2) }{\max(V_1^{m,m}) - \min(V_1^{m,m})} + 1$, we have $f(0)\leq 0$ and $f(1) \geq 0$. Since $f(\lambda_i)$ is a continuous function, there must exist some $\lambda_i \in [0,1]$ such that $f(\lambda_i) = 0$ as desired.
\end{itemize}

One can easily verify that $\vmu^{\bot}$ and $\{\vmu^i\}_{i\in S_{\hat{\sigma}_1}}$ are SPEs of $G(T-1)$. In addition, their construction ensures that $\vmu^*$ is an NE of the root game $G(T)$. Therefore, $\vmu^*$ is an SPE of $G(T)$ where the first round strategy profile does not form an NE of the stage game $G$.
\\
\\
\textbf{Case (c).} Since both (a) and (b) are false, and $|V_1^{m,m}| > 1$, $|V_2^{m,m}| = 1$, applying \Cref{lemma:matrix}, we know that there exists $(i_1,j_1), (i_2,j_2)\in I$ such that $i_1\neq i_2$, $j_1\neq j_2$, where $I$ is the set of pure strategy profiles where player 2 plays a best response, as defined in \Cref{lemma:matrix}. Take such $(i_1,j_1), (i_2,j_2)$, \Cref{lemma:matrix} further implies that for all $j\neq j_1$, $u_2(i_1, j) < b$, and for all $j\neq j_2$, $u_2(i_2, j) < b$, where $b$ is the only element in $V_2^{m,m}$. Denote $\hat{\sigma}_{\lambda}\in \Delta A_1$ as the mixed strategy for player 1 which assigns $\hat{\sigma}_{\lambda}(i_1) = \lambda$ and $\hat{\sigma}_{\lambda}(i_2) = 1 - \lambda$. Denote $J(\lambda) = \condSet{a_2}{a_2\in A_2, a_2 \textrm{ is a best response to }\hat{\sigma}_{\lambda}}$ as the set of best response pure strategies for player 2 against $\hat{\sigma}_{\lambda}$. It is helpful to consider a geometric interpretation of $J(\lambda)$. For each $j\in A_2$, $u_2(\hat{\sigma}_{\lambda}, j) = \lambda\cdot u_2(i_1, j) + (1 - \lambda)\cdot u_2(i_2, j)$ is a linear function in $\lambda$. We can plot the function $f_j(\lambda)=u_2(\hat{\sigma}_{\lambda}, j)$ for each $j\in A_2$, which gives $|A_2|$ straight lines within domain $[0,1]$. $J(\lambda)$ is then the set of lines that attains the maximum value at $\lambda$. We know that $J(0) = \{j_2\}$ and $J(1) = \{j_1\}$, so there must exist some $\lambda_1\in (0,1)$ where $|J(\lambda_1)| > 1$, which corresponds to some intersection point. Take such $\lambda_1$ and $\hat{j}_1, \hat{j}_2 \in J(\lambda_1)$ where $\hat{j}_1\neq \hat{j}_2$. Denote $\hat{\sigma}_{\rho}\in \Delta A_2$ as the mixed strategy for player 2 which assigns $\hat{\sigma}_{\rho}(\hat{j}_2) = \rho$ and $\hat{\sigma}_{\rho}(\hat{j}_1) = 1 - \rho$. Then for all $\rho\in[0,1]$, $\hat{\sigma}_{\rho}$ is a best response to $\hat{\sigma}_{\lambda_1}$, which implies that $\hat{\sigma}_{\lambda_1}$ is not a best response to $\hat{\sigma}_{\rho}$. This is because if $\hat{\sigma}_{\lambda_1}$ is a best response to $\hat{\sigma}_{\rho}$, then $(\hat{\sigma}_{\lambda_1}, \hat{\sigma}_{\rho})\in \Nash^{m,m}(G)$, but $u_2(\hat{\sigma}_{\lambda_1}, \hat{\sigma}_{\rho}) < b$, which contradicts with $|V_2^{m,m}| = 1$.

Now we show that we can always construct some $T$ and some SPE $\vmu^*$ of $G(T)$ where the first round strategy profile is $(\hat{\sigma}_{\lambda_1}, \hat{\sigma}_{\rho})$ for some $\rho$. Since the above argument shows that $(\hat{\sigma}_{\lambda_1}, \hat{\sigma}_{\rho})\notin \Nash^{m,m}(G)$, \phenom{} occurs in $\vmu^*$. We treat the case where $u_1(i_1, \hat{j}_1) = u_1(i_2, \hat{j}_1)$ and $u_1(i_1, \hat{j}_1) \neq u_1(i_2, \hat{j}_1)$ separately.

If $u_1(i_1, \hat{j}_1) = u_1(i_2, \hat{j}_1)$, we construct $\vmu^*$ as: 
\begin{itemize}
    \item In the first round, play $(\hat{\sigma}_{\lambda}, \hat{j}_1)$. ($\hat{j}_1$ is $\hat{\sigma}_{\rho}$ with $\rho = 0$)
    \item For the later rounds, if player 1's first round play is $i_1$ or $i_2$, players repeatedly play $\vsmax$; otherwise, players repeatedly play $\vsmin$.
\end{itemize}
Again, $\vsmin, \vsmax\in \Nash^{m,m}(G)$ such that $u_1(\vsmin) = \min (V_1^{m,m})$ and $u_1(\vsmax) = \max (V_1^{m,m})$. $T$ is chosen to be large enough such that player 1 deviating to any $i\notin \{i_1, i_2\}$ in the first round will reduce their total payoff in $G(T)$. It can be easily checked that $\vmu^*$ is an SPE of $G(T)$.

If $u_1(i_1, \hat{j}_1) \neq u_1(i_2, \hat{j}_1)$, WLOG, assume $u_1(i_1, \hat{j}_1) > u_1(i_2, \hat{j}_1)$. We construct $T = 1 + T_1 + T_2$ and $\vmu^*$ with the following structure:
\begin{itemize}
    \item In the first round, play $(\hat{\sigma}_{\lambda}, \hat{\sigma}_{\rho})$.
    \item For the later rounds, if the first round play is $(i_1, \hat{j}_2)$, players play their corresponding strategy according to SPE $\vmu^1$ of $G(T-1)$; if the first round play is $(i_2, \hat{j}_2)$, $(i_1, \hat{j}_1)$ or $(i_2, \hat{j}_1)$, players play their corresponding strategy according to SPE $\vmu^2$ of $G(T-1)$; otherwise, players play their corresponding strategy according to SPE $\vmu^{\bot}$ of $G(T-1)$.
    \begin{itemize}
        \item $\vmu^{\bot}$ is players play $\vsmin$ repeatedly for $T_1 + T_2$ rounds.
        \item $\vmu^2$ is players play $\vsmax$ repeatedly for $T_1 + T_2$ rounds.
        \item $\vmu^1$ is players play $\vsmin$ repeatedly for $T_1$ rounds, and then $\vsmax$ for $T_2$ rounds.
    \end{itemize}
\end{itemize}
$T_2$ is chosen to be large enough such that player 1 deviating to any $i\notin \{i_1, i_2\}$ in the first round will reduce their total payoff in $G(T)$. In order for $\vmu^*$ to be an NE of the root game, player 1 choosing $i_1$ and $i_2$ in the first round need to yield the same total payoff in $G(T)$. The total payoff of player 1 achieved by choosing $i_1$ in the first round under $\vmu^*$ is $\rho\cdot \Big(u_1(i_1, \hat{j}_2) + U_1(\vmu^1) \Big) + (1-\rho)\cdot \Big( u_1(i_1, \hat{j}_1) + U_1(\vmu^2) \Big)$, and the total payoff by choosing $i_2$ is $\rho\cdot \Big(u_1(i_2, \hat{j}_2) + U_1(\vmu^2) \Big) + (1-\rho)\cdot \Big( u_1(i_2, \hat{j}_1) + U_1(\vmu^2) \Big)$. Consider the difference between these two quantities as a function of $\rho$, $g(\rho) = \rho \cdot \Big(U_1(\vmu^1) - U_1(\vmu^2)\Big) + \rho \cdot \Big(u_1(i_1, \hat{j}_2) - u_1(i_2, \hat{j}_2)\Big) + (1-\rho) \cdot \Big( u_1(i_1, \hat{j}_1) - u_1(i_2, \hat{j}_1) \Big)$. We have $g(0) > 0$. By choosing a large enough $T_1$, $g(1) < 0$. Since $g(\rho)$ is a continuous function, there must exist some $\rho \in (0,1)$ such that $g(\rho) = 0$. With this value of $\rho$, $\vmu^*$ is an SPE of $G(T)$ as desired.

\end{proof}

Again, from the constructions of SPEs where \phenom{} occurs used in the above proof, we can obtain the following corollary regarding the value of $T$ above which \phenom{} can occur if the condition in \Cref{thm:general} is satisfied:

\begin{corollary}
For general 2-player games (mixed strategies allowed), given a stage game $G$:
\begin{enumerate}
    \item If $|V_1^{m,m}| > 1$, $|V_2^{m,m}| > 1$, and there exists some $\hat{\sigma}_1 \in \Delta A_1, \hat{\sigma}_2\in \Delta A_2$ where $(\hat{\sigma}_1, \hat{\sigma}_2) \notin \Nash^{m,m}(G)$, then by \Cref{lemma:mixed-pure}, there exists $\hat{a}_1\in A_1$ and $\hat{a}_2\in A_2$ where $(\hat{a}_1, \hat{a}_2) \notin \Nash^{m,m}(G)$, for all $T \geq 2\cdot \max\Big( \frac{\delta_1}{\max (V_1^{m,m})-\min (V_1^{m,m})}, \frac{\delta_2}{\max (V_2^{m,m})- \min (V_2^{m,m})} \Big) +1$ where $\delta_1 = \max_{a_1\in A_1} u_1(a_1, \hat{a}_2) - u_1(\hat{a}_1, \hat{a}_2)$ and $\delta_2 = \max_{a_2\in A_2} u_2(\hat{a}_1, a_2) - u_2(\hat{a}_1, \hat{a}_2)$, there exists some SPE of $G(T)$ where \phenom{} occurs.
    \item If $|V_1^{m,m}| > 1$, $|V_2^{m,m}| = 1$, and there exists $\hat{\sigma}_1\in \Delta A_1, \hat{\sigma}_2 \in \Delta A_2, a_1'\in A_1$ where $u_1(\hat{\sigma}_1, \hat{\sigma}_2) < u_1(a_1', \hat{\sigma}_2)$ and $\hat{\sigma}_2$ is a best response to $\hat{\sigma}_1$, then
    \begin{enumerate}
        \item If there exists $\hat{a}_1\in A_1, \hat{\sigma}_2 \in \Delta A_2, a_1'\in A_1$ where $u_1(\hat{a}_1, \hat{\sigma}_2) < u_1(a_1', \hat{\sigma}_2)$ and $\hat{\sigma}_2$ is a best response to $\hat{a}_1$, then for all $T\geq \frac{\max_{a_1\in A_1} u_1(a_1, \hat{\sigma}_2) - u_1(\hat{a}_1, \hat{\sigma}_2)}{\max (V_1^{m,m})-\min (V_1^{m,m})} + 1$, there exists some SPE of $G(T)$ where \phenom{} occurs.
        \item If there does not exist $\hat{a}_1\in A_1, \hat{\sigma}_2 \in \Delta A_2, a_1'\in A_1$ where $u_1(\hat{a}_1, \hat{\sigma}_2) < u_1(a_1', \hat{\sigma}_2)$ and $\hat{\sigma}_2$ is a best response to $\hat{a}_1$, and there exists $a_1\in A_1$ and $a_2, a_2' \in A_2$ where $a_2\neq a_2'$ and both $a_2$ and $a_2'$ are best responses to $a_1$, then for all $T \geq 3 + \frac{\max_{a, a'\in S_{\hat{\sigma}_1}} u_1(a, \hat{\sigma}_2) - u_1(a', \hat{\sigma}_2) }{\max(V_1^{m,m}) - \min(V_1^{m,m})} + \frac{\max_{a,a'\in A_1} u_1(a,\hat{\sigma}_2) - u_1(a',\hat{\sigma}_2)}{\max(V_1^{m,m}) - \min(V_1^{m,m})}$, there exists some SPE of $G(T)$ where \phenom{} occurs.
        \item If there does not exist $\hat{a}_1\in A_1, \hat{\sigma}_2 \in \Delta A_2, a_1'\in A_1$ where $u_1(\hat{a}_1, \hat{\sigma}_2) < u_1(a_1', \hat{\sigma}_2)$ and $\hat{\sigma}_2$ is a best response to $\hat{a}_1$, and there does not exist $a_1\in A_1$ and $a_2, a_2' \in A_2$ where $a_2\neq a_2'$ and both $a_2$ and $a_2'$ are best responses to $a_1$. By \Cref{lemma:matrix}, there exists $(i_1,j_1), (i_2,j_2)\in I$ such that $i_1\neq i_2$, $j_1\neq j_2$, where $I$ is the set of pure strategy profiles where player 2 plays a best response. Denote $\hat{\sigma}_{\lambda}\in \Delta A_1$ as the mixed strategy for player 1 which assigns $\hat{\sigma}_{\lambda}(i_1) = \lambda$ and $\hat{\sigma}_{\lambda}(i_2) = 1 - \lambda$. By the proof of \Cref{thm:general}, there exists $\lambda_1\in (0,1)$, $\hat{j}_1, \hat{j}_2 \in A_2$ where $\hat{j}_1, \hat{j}_2$ are both best responses to $\hat{\sigma}_{\lambda_1}$ and $\hat{j}_1\neq \hat{j}_2$. If $u_1(i_1, \hat{j}_1) = u_1(i_2, \hat{j}_1)$, then for all $T\geq \frac{\max_{a_1\in A_1} u_1(a_1, \hat{j}_1) - u_1(i_1, \hat{j}_1)}{\max (V_1^{m,m})-\min (V_1^{m,m})} + 1$, there exists some SPE of $G(T)$ where \phenom{} occurs. If $u_1(i_1, \hat{j}_1) \neq u_1(i_2, \hat{j}_1)$, then for all $T \geq 3 + \frac{\max_{a_1\in A_1, i\in \{i_1, i_2\}, j\in \{ \hat{j}_1, \hat{j}_2 \}} u_1(a_1, j) - u_1(i, j) }{\max (V_1^{m,m})-\min (V_1^{m,m})} + \frac{|u_1(i_1, \hat{j}_2) - u_1(i_2, \hat{j}_2)|}{\max (V_1^{m,m})-\min (V_1^{m,m})}$, there exists some SPE of $G(T)$ where \phenom{} occurs.
    \end{enumerate}
    \item same as (2) but exchange player 1 and 2.
\end{enumerate}
\end{corollary}
