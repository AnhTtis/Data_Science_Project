\section{Introduction}

We study the emergence of locally suboptimal behavior in finitely repeated games. Locally suboptimal behavior refers to players play suboptimally in some rounds of the repeated game (i.e., not maximizing their payoffs in those rounds) while maximizing their total payoffs in the whole repeated game. The emergence of locally suboptimal behavior reflects some fundamental psychological and social phenomena, such as delayed gratification, threats, and enforced cooperation. 

We focus on the emergence of locally suboptimal behavior in subgame-perfect equilibria (SPE) of finitely repeated games with complete information.
A widely known result that appears in many textbooks and lecture notes is that if the stage game $G$ has a unique Nash equilibrium payoff for every player, then in any SPE of any finitely repeated game $G(T)$ with any $T$ rounds, the strategy profile at each round forms an NE of the stage game $G$ \cite{fudenberg1991game,gibbons1992primer,osborne1994course}. This is proved using backward induction. 
It is also known that there are stage games $G$ where in some SPEs of the repeated game $G(T)$ for some $T$, the strategy profile at some round does not form a stage-game NE (\Cref{sec:example-off-nash} presents an example).
Such off-(stage-game)-Nash play occurs due to {\em `threats'} between players that are stated implicitly through players' strategies.

We define such off-(stage-game)-Nash plays in repeated games as {\em \phenom{}}.
As we have seen, for some stage games, \phenom{} can occur in some SPE of some repeated games; for other stage games, \phenom{} can never occur in any SPE of any repeated games.
Therefore, we can partition the set of all stage games $\mathcal{G}$ into two disjoint subsets $\gls$ and $\glo$. $\gls$ is the set of stage games $G$ where \phenom{} occurs in some SPE of $G(T)$ for some $T$; $\glo$ is the set of stage games $G$ where \phenom{} never occurs in any SPE of $G(T)$ for any $T$ (LO stands for locally optimal).
Our goal in this research is to completely characterize which stage games belong to $\gls$ and $\glo$. The central research question we aim to tackle is:

\begin{question}
\label{ques:cond}
What is a sufficient and necessary condition on the stage game $G$ that ensures that, for all $T$ and all subgame-perfect equilibria of the repeated game $G(T)$, the strategy profile at every round of $G(T)$ forms a Nash equilibrium of the stage game $G$?
\end{question}

The answer to \Cref{ques:cond} completely characterizes $\gls$ and $\glo$. As we have discussed, a sufficient condition for \Cref{ques:cond} is widely known (uniqueness of Nash equilibrium payoff for each player). However, this condition is not necessary; in fact, no previous work establishes a sufficient and necessary condition. A large body of work focuses on Folk Theorems \cite{Benoit1985,benoit1987,gossner1995,Smith1995,gonzalez2006finitely}, where the property of interest is: all feasible (i.e., the payoff profile lies in the convex hull of the set of all possible payoff profiles of the stage game) and individually rational (i.e., the payoff of each player is at least their minmax payoff in the stage game) payoff profiles can be attained in the equilibrium of the repeated game. As we show in \Cref{sec:example-diff}, the property considered in Folk Theorems and the \phenom{} property we consider in this work are different, and the two properties do not have direct implications in either direction. Therefore, the conditions established for Folk Theorems in the literature do not solve the problem we consider.

In addition to the complete mathematical characterization of the partitioning between $\gls$ and $\glo$, we also consider the computational aspect of the problem: 

\begin{question}
\label{ques:comp}
    Given an arbitrary stage game $G$, how to (algorithmically) decide if there exists some $T$ and some SPE of $G(T)$ where \phenom{} occurs? Is this problem decidable?
\end{question}

A naive approach is to enumerate over $T$, solve for all SPEs for each $G(T)$, and check if off-Nash behavior occurs. Such an approach is not only computationally inefficient, but also not guaranteed to terminate due to the unboundedness of $T$. In fact, we show that there are stage games where \phenom{} only occurs in repeated games with very large $T$, and we can construct games where this minimum $T$ for \phenom{} to occur can be arbitrarily large (\Cref{sec:example-largeT}). These facts motivate the study of \Cref{ques:comp}.

\subsection{Summary of Results}

\subsubsection{Sufficient and Necessary Conditions for 2-Player Games}

A main theoretical contribution of this part is that we prove sufficient and necessary conditions for \Cref{ques:cond} for 2-player games. We prove the conditions for three cases: 1) only pure strategies are allowed (\Cref{thm:pure}), 2) the general case where mixed strategies are allowed (\Cref{thm:general}), and 3) one player can only use pure strategies and the other player can use mixed strategies (\Cref{thm:pure-vs-mixed}). 
This is the first sufficient and necessary condition for off-(stage-game)-Nash plays to occur in SPEs of 2-player finitely repeated games.

From the perspective of partitioning the set of stage games $\mathcal{G}$, denote $\glspp$ as the set of stage games $G$ where \phenom{} occurs in some SPE of $G(T)$ for some $T$ when both players can only use pure strategies, $\glopp$ as the set of stage games $G$ where \phenom{} never occurs in any SPE of $G(T)$ for any $T$ when both players can only use pure strategies, $\glsmm$ and $\glomm$ as the corresponding partitioning when both players can use mixed strategies, and $\glsmp$ and $\glomp$ as the corresponding partitioning when player 1 can use mixed strategies and player 2 can only use pure strategies.
Essentially, we obtain complete mathematical characterizations of the partitioning of $\mathcal{G}$ for cases (1), (2), and (3) above: 1) $\glspp$ and $\glopp$, 2) $\glsmm$ and $\glomm$, and 3) $\glsmp$ and $\glomp$.

\subsubsection{Effect of Changing from Pure Strategies to Mixed Strategies on the Emergence of Local Suboptimality}

Based on the above results, we further study the effect of changing from pure strategies to mixed strategies on the emergence of \phenom{}. We aim to answer the following question: under what conditions on the stage game $G$ will allowing players to play mixed strategies change whether \phenom{} can ever occur in some repeated game $G(T)$? Essentially, we aim to study the relationships between $\glspp$, $\glsmp$, and $\glsmm$.

We prove that $\glspp \subseteq \glsmp \subseteq \glsmm$ (\Cref{thm:pp-to-mp-1,thm:mp-to-mm-1,thm:pp-to-mm-1}), i.e., if \phenom{} can occur before the change, then after changing any player (or both players) from pure-strategies-only to mixed-strategies-allowed, \phenom{} can still occur. This is because we prove that any SPE of the repeated game before the change is still an SPE after the change, and any strategy profile that is not a stage-game NE before the change is still not a stage-game NE after the change.

On the other hand, we show that $\glspp \neq \glsmp$ and $\glsmp \neq \glsmm$ (so $\glspp$ is a proper subset of $\glsmp$ and $\glsmp$ is a proper subset of $\glsmm$), i.e., there are games where \phenom{} can never occur before the change, but after changing one player (or both players) from pure-strategies-only to mixed-strategies-allowed, \phenom{} can occur. We present complete characterizations of the sets $\glsmp \setminus \glspp$, $\glsmm \setminus \glsmp$, and $\glsmm \setminus \glspp$, by proving sufficient and necessary conditions on the stage game $G$ such that \phenom{} can never occur before the change but can occur after the change (\Cref{thm:pp-to-mp-2,thm:mp-to-mm-2,thm:pp-to-mm-2}). Our characterizations are fine-grained based on $|V_1|$ and $|V_2|$, the number of payoff values attainable at stage-game NEs for each player. For example, we show that under certain preconditions on $|V_1|$ and $|V_2|$, $\glspp = \glsmp$; under other preconditions on $|V_1|$ and $|V_2|$, $\glspp \neq \glsmp$, and for each of such cases, we present an example stage game $G$ where $G\notin \glspp$ and $G\in \glsmp$. These examples demonstrate different mechanisms of how changing a player from pure-strategies-only to mixed-strategies-allowed can lead to the emergence of \phenom{}. We perform the same fine-grained analyses on $\glsmm \setminus \glsmp$ and $\glsmm \setminus \glspp$ as well.


\subsubsection{Computational Aspects}

We propose an algorithm for deciding \Cref{ques:comp} for 2-player games for the general case where mixed strategies are allowed and analyze the computational complexity of this algorithm. This shows that \Cref{ques:comp} is decidable for 2-player games where mixed strategies are allowed. This algorithm provides a method for computationally deciding if \phenom{} can ever happen for a given stage game. The algorithm is based on the sufficient and necessary condition established in \Cref{thm:general}. We design several efficient methods for checking different parts of the condition by utilizing properties we prove for general games. Naive methods for checking these parts of the condition take exponential time in the worst case, whereas our methods for checking these parts of the condition take polynomial time in the worst case.


\subsubsection{Generalization to $n$-Player Games}

We prove a separate sufficient condition and necessary condition for \Cref{ques:cond} for $n$-player games. These conditions are both tighter than what is previously known in the literature (again, only a sufficient condition is known previously, i.e., there is a unique Nash equilibrium payoff for each player \cite{fudenberg1991game,gibbons1992primer,osborne1994course}). 

The proof of a sufficient and necessary condition for the 2-player case relies on some properties that we prove to hold for 2-player games (\Cref{lemma:matrix} and the subsequent arguments in the proof of \Cref{thm:general} that uses \Cref{lemma:matrix} to show there exists a connected component of off-Nash strategy profiles). It is not clear whether similar properties hold for $n$-player games. Therefore, the questions of 1) what is a sufficient and necessary condition for $n$-player games, and 2) is \Cref{ques:comp} decidable for $n$-player games, remain open.
