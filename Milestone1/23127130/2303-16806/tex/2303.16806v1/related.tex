\section{Related Work}
\label{sec:suboptimal-related}

%\paragraph{Folk Theorem}
Under the theme of analyzing equilibrium solutions in repeated games, a large body of work focuses on Folk Theorems, where the property of interest is: all feasible and individually rational payoff profiles can be attained in equilibria of the repeated game. 
In the context of infinitely repeated games, the original Folk Theorem asserts that all feasible and individually rational (see \Cref{sec:example-diff} for the definitions) payoff profiles can be attained in Nash equilibria of infinitely repeated games with sufficiently little discounting. 
This result is widely known in the field but not formally published, which is why it is called Folk Theorem.
\cite{aumann1994long,rubinstein1979equilibrium} show that the same result holds when we consider subgame-perfect equilibria and assume no discounting. \cite{fudenberg1986folk} proves a sufficient condition for Folk Theorem for subgame-perfect equilibria in infinitely repeated games with discounting. \cite{friedman1971non,friedman1982oligopoly} consider a variation of Folk Theorem where they show that any feasible payoff profile that Pareto dominates a Nash equilibrium of the stage game can be attained in a subgame-perfect equilibrium of the infinitely repeated game with discounting.
In the context of finitely repeated games, \cite{Benoit1985} obtained sufficient conditions for Folk Theorem for subgame-perfect equilibria, and later \cite{Smith1995} establishes necessary and sufficient conditions for subgame-perfect equilibria. Both results rely on mixed strategies are observable, meaning that players can directly observe the mixed strategies (i.e., probability distributions) used by other players in previous rounds of the game, not just the realized actions in the previous rounds; \cite{gossner1995} establishes sufficient conditions for subgame-perfect equilibrium without this assumption. 
\cite{benoit1987} obtained sufficient conditions for Nash equilibria, and \cite{gonzalez2006finitely} establishes sufficient and necessary conditions for Nash equilibria. Folk Theorem has also been studied in a broader class of repeated game models. \cite{fudenberg1986folk} considers Folk Theorem for finitely repeated game with incomplete information. \cite{fudenberg1994folk} considers infinitely repeated game with imperfect monitoring. \cite{dutta1995folk} considers infinite horizon stochastic games with perfect monitoring, and later \cite{fudenberg2011folk} considers infinite horizon stochastic games with imperfect monitoring.

A major difference between the above line of work and this work is that Folk Theorems consider the set of payoffs attainable, whereas this work considers the occurrence of off-(stage-game)-Nash play. As we demonstrate in \Cref{sec:example-diff}, the property considered in Folk Theorems and the \phenom{} property considered in this work do not have direct implications in either direction. Therefore, unlike this research, none of the above research establishes a sufficient and necessary condition for off-(stage-game)-Nash play to occur in finitely repeated games.

%\paragraph{Equilibrium value set in repeated games}
Several works in the literature establish additional characterizations on the equilibrium value set in repeated games. When the preconditions of Folk Theorems do not hold, these results provide some characterizations on the equilibrium value set. 
\cite{demeze2020complete} provides a complete characterization of the set of pure strategy SPE payoff profiles in the limit as the time horizon increases for finitely repeated games with perfect monitoring. 
\cite{radner1985repeated,radner1986example} characterize limiting behavior of the equilibrium value set of infinitely repeated games with imperfect monitoring as the discount factor approaches 1.
\cite{abreu1990toward} further proves properties of the equilibrium value set in infinitely repeated games with discounting and imperfect monitoring. Again, this line of work considers the set of payoffs attainable, whereas our work considers the occurrence of off-(stage-game)-Nash play. Unlike our research, none of the above research establishes a sufficient and necessary condition for off-(stage-game)-Nash play to occur in finitely repeated games.

%\paragraph{Equilibrium analyses in repeated games}
%In a broader context, \cite{aumann1995repeated,mertens2015repeated} are two books that present comprehensive studies on equilibrium solutions in repeated games. To the best of our knowledge, the research presented in this thesis is the first to establish a complete characterization of when off-(stage-game)-Nash play can occur in two-player finitely repeated games.

%Besides equilibrium value set, there is research that analyzes other aspects of equilibrium solutions in repeated games. \cite{kreps1982rational} shows that cooperation occurs in sequential equilibria of finitely repeated prisoner's dilemma when there is certain incomplete information on one of the players. \cite{rubinstein1986finite} shows that cooperation cannot occur in infinitely repeated prisoner's dilemma when players use finite automata as their strategies. \cite{fudenberg1983subgame} shows that SPEs of infinitely repeated games arise in the limit from $\epsilon$-SPEs of finitely repeated games.
