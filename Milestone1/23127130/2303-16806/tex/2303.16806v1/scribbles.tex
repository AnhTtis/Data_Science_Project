
\section{Nash Equilibrium}

\begin{theorem}
For any dot covering game, there exists a pure Nash equilibrium.
\end{theorem}

\begin{proof}
Dot covering game is a potential game, with potential function:
\begin{equation*}
    \phi(f) = \sum_d \sum_{i=0}^{n_d} r_d(i)
\end{equation*}
Within the set of states that is achievable by strategies in hierarchy $\strategyLevel{k}$, there is one maximum in $\phi$. This point will be a PNE. Therefore PNE exists.
\end{proof}

\begin{theorem}
Finding a pure Nash equilibrium for a dot covering game is in $\mathcal{P}$, if we use unary representation for $r_g$ and $r_m$ (psuedo-polynomial).
\end{theorem}

\begin{proof}

We present a polynomial time algorithm for finding PNE.

\begin{itemize}
    \item Starting with an initial set of strategies.
    \item In each iteration, fixing all other player's strategy, and find the best strategy for player $i$, which achieves an increase of player $i$'s payoff $\delta_i$. Try it for each $i$ and choose the one with largest $\delta_i$ and adopt that update.
    \item Iterate until all $\delta_i\leq 0$. Then we find a PNE.
\end{itemize}

Initial $\phi_0\geq MPr_m(1)$, final maximum found $\phi_{max} \leq NPr_g(1)$. Since every iteration at least increase $\phi$ by 1, $N_{iter} \leq NPr_g(1) - MPr_m(1)$, so $N_{iter} = \mathcal{O}(NP)$.

DP Algorithm for finding best strategy of player $i$ fixing all others' strategy:
\begin{itemize}
    \item Express all points in one sequence, ordered by $x$, e.g. $g^1, g^0, m^1, m^3, \dots$. Each point associate a counter. For each of the other players, increment the counter for points it covers. Then we have the payoff if player $i$ covers point $t$, $r_t$. $\mathcal{O}(NP)$.
    \item For each location $t$, line $l$, record a cumulative sum of $r_t$ that belongs to line $l$ up to $t$, $s_{t,l}$. $\mathcal{O}(NL)$
    \item Do DP. Subproblem $(n, k)$ is the largest payoff $R_{n,k}$ for player $i$ if they use $k$ segments on the first $n$ points. $R_{n,k} = \max_{n'}(R_{n', k-1} + \max_l (s_{n, l} - s_{n', l}))$. So each subproblem requires $\mathcal{O}(NL)$, and there are $\mathcal{O}(NK)$ subproblems.
\end{itemize}
So complexity of the above DP is $\mathcal{O}(NP + N^2KL)$. Need to solve for all players, so each overall iteration takes $\mathcal{O}(NP^2 + N^2PKL)$. Overall algorithm takes $\mathcal{O}(N^2P^3 + N^3P^2KL)$ to terminate.

\end{proof}

\paragraph{Remark}
So PNE is a good model of behavior for dot covering game.


\section{Alternating ordering}

There are $4M+2$ points in the sequence, $t=0,\dots, 4M+1$. And the ordering is:
\begin{itemize}
    \item for $j=0,\dots,M-1$:
    \begin{itemize}
        \item $y_{4j}=1, \alpha_{4j}=\textrm{gold}$
        \item $y_{4j+1}=0, \alpha_{4j+1}=\textrm{gold}$
        \item $y_{4j+2}=1, \alpha_{4j+2}=\textrm{mine}$
        \item $y_{4j+3}=0, \alpha_{4j+3}=\textrm{mine}$
    \end{itemize}
    \item $y_{4M}=1, \alpha_{4M}=\textrm{gold}$; $y_{4M+1}=0, \alpha_{4M+1}=\textrm{gold}$
\end{itemize}

\subsection{Proof ideas}

\optima*
\begin{proofidea}
This lemma essentially states that the segments of an optimal strategy can only start and end at particular locations in the sequence. We prove this lemma by showing that if $f^*_i \in \mathcal{F}_k$ does not satisfy the above condition, then there exists $f'_i$ which uses $k'\leq k$ segments and achieves a higher payoff $u_i(f'_i, \textbf{f}_{-i}) > u_i(f^*_i, \textbf{f}_{-i})$, therefore contradicting the fact that $f^*_i$ is optimal.
\end{proofidea}

\form*
\begin{proofidea}
By applying \cref{lemma:optima} to the general form of $\mathcal{F}_k$ presented in the preliminary section, we can show that $f_i^*$ satisfies condition $S_1$ to $S_4$ in the respective cases. Then by carefully counting the number of gold and mines covered by each segment, we can prove that $f_i^*$ always covers the specified number of gold and mines.
\end{proofidea}

\exact*
\begin{proofidea}
We prove by induction. The base case $b=1$ is trivial.

For the induction step $b=k$, first we show that if one of the strategies in a PNE uses exactly $k$ segments, then the other strategy must also use exactly $k$ segments. WLOG, let $f_1^* \in \mathcal{F}_k$. Depending on whether $k$ is odd or even, by \cref{lemma:form}, $f_1^*$ must satisfy the corresponding conditions in $S_1$ to $S_4$. For each case, we can construct $\hat{f}_2$ according to $f_1^*$ such that $\hat{f}_2 \in \mathcal{F}_k$ and achieves best coverage. Denote the payoff of $\hat{f}_2$ as $\hat{u}_2$. For any $f'_2 \in \mathcal{F}_{k'}$ where $k' < k$, \cref{lemma:form} specifies the number of gold and mines it covers, so by applying \cref{lemma:cover}, we can show that $f'_2$ achieves a payoff $u'_2 < \hat{u}_2$. Therefore, $f^*_2 \in \mathcal{F}_k$.

We then show that there is no PNE where both strategies use less than $k$ segments. We prove by contradiction. Assume there is a PNE $(f_1^*, f_2^*)$ where both $f_1^*$ and $f_2^*$ use less than $k$ segments. By the induction hypothesis, $f_1^*, f_2^* \in \mathcal{F}_{k-1}$. Depending on whether $k$ is odd or even, by \cref{lemma:form}, $f_1^*$ must satisfy the corresponding conditions in $S_1$ to $S_4$. For each case, we can construct $\hat{f}_2$ according to $f_1^*$ such that $\hat{f}_2 \in \mathcal{F}_k$ and achieves best coverage. By \cref{lemma:form}, $f_2^*$ covers a specific number of gold and mines, so by applying \cref{lemma:cover} we can show that $u_2^* < \hat{u}_2$, which contradicts with the fact that $(f_1^*, f_2^*)$ is a PNE. This completes the induction step.

\end{proofidea}

\aogthm*
\begin{proofidea}
For any PNE $(f_1^*, f_2^*)$, by \cref{lemma:exact}, $f_1^*, f_2^* \in \mathcal{F}_b$. Depending on whether $b$ is odd or even, by \cref{lemma:form}, $f_1^*$ must satisfy the corresponding conditions in $S_1$ to $S_4$. For each case, we can construct $\hat{f}_2$ according to $f_1^*$ such that $\hat{f}_2 \in \mathcal{F}_k$ and achieves a payoff $\hat{u}_2$. By \cref{lemma:form}, $f_2^*$ covers a specified number of gold and mines, so by applying \cref{lemma:cover}, we can show that $u_2^* \leq \hat{u}_2$. Thus, $u_2^* = \hat{u}_2$, and similarly we can show that all $u_1^*$'s have the same value. Therefore, we can show that all PNEs have the same social welfare of the stated expressions.
\end{proofidea}


\section{Empirical results for general gold and mines game}

\subsection{For general ordering, not all Nash has same payoff. Show examples.}

\subsubsection{Example 1}

\begin{tikzpicture}
\foreach \i in {0,1} {
    \foreach \j in {0,1,2} {
        \filldraw[black] (\i*4+\j*1,1) circle (2pt) node{};
        \filldraw[black] (\i*4+\j*1+0.5,0) circle (2pt) node{};
    }
    \draw (\i*4+3,1) node[cross=2pt] {};
    \draw (\i*4+3.5,0) node[cross=2pt] {};
}
\foreach \j in {0,1,2} {
    \filldraw[black] (2*4+\j*1,1) circle (2pt) node{};
    \filldraw[black] (2*4+\j*1+0.5,0) circle (2pt) node{};
}
\draw[blue, thick, opacity=0.5] (0,1) -- (10.25, 1);
\draw[blue, thick, opacity=0.5] (10.25, 1) -- (10.25, 0);
\draw[blue, thick, opacity=0.5] (10.25, 0) -- (11, 0);
\draw[red, thick, opacity=0.5] (0.25,0) -- (11, 0);
\draw[red, thick, opacity=0.5] (0.25, 1) -- (0.25, 0);
\draw[red, thick, opacity=0.5] (0, 1) -- (0.25, 1);
\end{tikzpicture}

v.s.

\noindent
\begin{tikzpicture}
\foreach \i in {0,1} {
    \foreach \j in {0,1,2} {
        \filldraw[black] (\i*4+\j*1,1) circle (2pt) node{};
        \filldraw[black] (\i*4+\j*1+0.5,0) circle (2pt) node{};
    }
    \draw (\i*4+3,1) node[cross=2pt] {};
    \draw (\i*4+3.5,0) node[cross=2pt] {};
}
\foreach \j in {0,1,2} {
    \filldraw[black] (2*4+\j*1,1) circle (2pt) node{};
    \filldraw[black] (2*4+\j*1+0.5,0) circle (2pt) node{};
}
\draw[blue, thick, opacity=0.5] (0,1) -- (9.25, 1);
\draw[blue, thick, opacity=0.5] (9.25, 1) -- (9.25, 0);
\draw[blue, thick, opacity=0.5] (9.25, 0) -- (11, 0);
\draw[red, thick, opacity=0.5] (0,0) -- (8.75, 0);
\draw[red, thick, opacity=0.5] (8.75, 1) -- (8.75, 0);
\draw[red, thick, opacity=0.5] (8.75, 1) -- (11, 1);
\end{tikzpicture}

\subsubsection{Example 2}

\begin{tikzpicture}
\foreach \i in {0,1,2} {
\filldraw[black] (\i*2,1) circle (2pt) node{};
\filldraw[black] (\i*2+0.5,0) circle (2pt) node{};
\draw (\i*2+1,1) node[cross=2pt] {};
\draw (\i*2+1.5,0) node[cross=2pt] {};
}
\filldraw[black] (3*2,1) circle (2pt) node{};
\filldraw[black] (3*2+0.5,0) circle (2pt) node{};
\filldraw[black] (1.8,1) circle (2pt) node{};
\filldraw[black] (2.2,1) circle (2pt) node{};
\filldraw[black] (4.7,0) circle (2pt) node{};
\draw[blue, thick, opacity=0.5] (0,1) -- (4.25, 1);
\draw[blue, thick, opacity=0.5] (4.25,1) -- (4.25, 0);
\draw[blue, thick, opacity=0.5] (4.25,0) -- (7, 0);
\draw[red, thick, opacity=0.5] (0,0) -- (5.25, 0);
\draw[red, thick, opacity=0.5] (5.25,1) -- (5.25, 0);
\draw[red, thick, opacity=0.5] (5.25,1) -- (7, 1);
\end{tikzpicture}

v.s.

\noindent
\begin{tikzpicture}
\foreach \i in {0,1,2} {
\filldraw[black] (\i*2,1) circle (2pt) node{};
\filldraw[black] (\i*2+0.5,0) circle (2pt) node{};
\draw (\i*2+1,1) node[cross=2pt] {};
\draw (\i*2+1.5,0) node[cross=2pt] {};
}
\filldraw[black] (3*2,1) circle (2pt) node{};
\filldraw[black] (3*2+0.5,0) circle (2pt) node{};
\filldraw[black] (1.8,1) circle (2pt) node{};
\filldraw[black] (2.2,1) circle (2pt) node{};
\filldraw[black] (4.7,0) circle (2pt) node{};
\draw[blue, thick, opacity=0.5] (0,1) -- (2.3, 1);
\draw[blue, thick, opacity=0.5] (2.3,1) -- (2.3, 0);
\draw[blue, thick, opacity=0.5] (2.3,0) -- (7, 0);
\draw[red, thick, opacity=0.5] (0,0) -- (1.25, 0);
\draw[red, thick, opacity=0.5] (1.25,1) -- (1.25, 0);
\draw[red, thick, opacity=0.5] (1.25,1) -- (7, 1);
\end{tikzpicture}

\section{Price of anarchy of \aog{}}


\begin{definition}
\textbf{Global price of anarchy}

Denote the best social welfare under no restrictions on the strategy space (i.e. $b\rightarrow +\infty$) as $W_{\textrm{best}}^* = 2M+2$. Then
\begin{equation*}
    POA_{\textrm{global}}(b) = \frac{W_{\textrm{best}}^*}{W_{\textrm{Nash}}(b)}
\end{equation*}
\end{definition}


\paragraph{Trends of $POA_{\textrm{global}}$}
For price of anarchy to be well defined, the social welfare should be positive, so we only discuss $\mu > -1$ here.
The trend is inverse to the trend of $W_{\textrm{Nash}}$. See Figure xxx.


$W_{\textrm{best}}(1) = W_{\textrm{Nash}}(1)$, but $W_{\textrm{best}}$ always increases faster than $W_{\textrm{Nash}}$.


\paragraph{Remark}
We see that by restricting the strategy space, we can obtain better Nash equilibrium (in terms of social welfare). There is an optimal point on the strategy space hierarchy that gives the best social welfare (lowest global price of anarchy).
