\documentclass{amsart}
\usepackage[utf8]{inputenc}
\usepackage{amsmath,amssymb,amsthm, graphics, comment,bm}
\usepackage{mathtools}
\usepackage{amsthm}
\usepackage{diagmac2}
\usepackage{textcomp}
\usepackage{color}
\usepackage[T1]{fontenc}
\usepackage[alphabetic]{amsrefs}
\usepackage[all,cmtip,color,matrix,arrow]{xy}
\usepackage{yfonts}
\usepackage{tikz}
\usepackage{tikz-cd}
\usepackage{stmaryrd}
\usepackage{mathrsfs}
\usepackage{enumerate}
\usepackage{hyperref}
\hypersetup{
	colorlinks=true,
	linkcolor=blue,
	filecolor=magenta,      
	urlcolor=cyan,
}
\usepackage[mathscr]{euscript}
\usepackage[all,cmtip]{xy}
\usepackage{cleveref}

\setlength{\textwidth}{\paperwidth}
\addtolength{\textwidth}{-2in}
\calclayout

\newcommand{\dotr}{\mbox{$\boldsymbol{\cdot}$}}
\newcommand{\tilda}{\accentset{\sim}}
\newcommand{\Exal}{\operatorname{Exal}}
\newcommand{\Exalstack}{\underline{\operatorname{Exal}}}
\newcommand{\bgm}{B\mathbb{G}_m}
\newcommand{\Diff}{\operatorname{Diff}}
\newcommand{\noname}{[\mathbb{A}^1/\mathbb{G}_m]}
\newcommand{\bA}{\mathbb{A}}
\newcommand{\bC}{\mathbb{C}}
\newcommand{\bF}{\mathbb{F}}
\newcommand{\bG}{\mathbb{G}}
\newcommand{\bL}{\mathbb{L}}
\newcommand{\bN}{\mathbb{N}}
\newcommand{\bP}{\mathbb{P}}
\newcommand{\bQ}{\mathbb{Q}}
\newcommand{\bT}{\mathbb{T}}
\newcommand{\bZ}{\mathbb{Z}}
\newcommand{\kC}{\mathfrak{C}}
\newcommand{\kM}{\mathfrak{M}}
\newcommand{\km}{\mathfrak{m}}
\newcommand{\sA}{\mathscr{A}}
\newcommand{\sB}{\mathscr{B}}
\newcommand{\sC}{\mathscr{C}}
\newcommand{\sD}{\mathscr{D}}
\newcommand{\sE}{\mathscr{E}}
\newcommand{\sF}{\mathscr{F}}
\newcommand{\sG}{\mathscr{G}}
\newcommand{\sH}{\mathscr{H}}
\newcommand{\sI}{\mathscr{I}}
\newcommand{\sK}{\mathscr{K}}
\newcommand{\sL}{\mathscr{L}}
\newcommand{\sM}{\mathscr{M}}
\newcommand{\oH}{\operatorname{H}}
\newcommand{\sN}{\mathscr{N}}
\newcommand{\sO}{\mathscr{O}}
\newcommand{\sP}{\mathscr{P}}
\newcommand{\sS}{\mathscr{S}}
\newcommand{\sU}{\mathscr{U}}
\newcommand{\sX}{\mathscr{X}}
\newcommand{\sZ}{\mathscr{Z}}
\newcommand{\cA}{\mathcal{A}}
\newcommand{\cB}{\mathcal{B}}
\newcommand{\cC}{\mathcal{C}}
\newcommand{\cD}{\mathcal{D}}
\newcommand{\cE}{\mathcal{E}}
\newcommand{\cF}{\mathcal{F}}
\newcommand{\cG}{\mathcal{G}}
\newcommand{\cH}{\mathcal{H}}
\newcommand{\cI}{\mathcal{I}}
\newcommand{\cK}{\mathcal{K}}
\newcommand{\cL}{\mathcal{L}}
\newcommand{\cM}{\mathcal{M}}
\newcommand{\cN}{\mathcal{N}}
\newcommand{\cO}{\mathcal{O}}
\newcommand{\cP}{\mathcal{P}}
\newcommand{\cQ}{\mathcal{Q}}
\newcommand{\cR}{\mathcal{R}}
\newcommand{\cS}{\mathcal{S}}
\newcommand{\cT}{\mathcal{T}}
\newcommand{\cU}{\mathcal{U}}
\newcommand{\cV}{\mathcal{V}}
\newcommand{\cW}{\mathcal{W}}
\newcommand{\cX}{\mathcal{X}}
\newcommand{\cY}{\mathcal{Y}}
\newcommand{\cZ}{\mathcal{Z}}

\newcommand{\qB}{R}
\newcommand{\qR}{\cB}

\newcommand{\Ass}{\operatorname{Ass}}
\newcommand{\E}{\operatorname{Ext}}
\newcommand{\spec}{\operatorname{Spec}}
\newcommand{\Kg}{\mathcal{K}}
\newcommand{\sY}{\mathscr{Y}}
\newcommand{\supp}{\operatorname{Supp}}
\newcommand{\Ker}{\operatorname{Ker}}
\newcommand{\lcm}{\operatorname{lcm}}
\newcommand{\Tr}{\operatorname{Tr}}
\newcommand{\Sym}{\operatorname{Sym}}
\newcommand{\Hom}{\operatorname{Hom}}
\newcommand{\Gr}{\operatorname{Gr}}
\newcommand{\Div}{\operatorname{Div}}
\newcommand{\Pic}{\operatorname{Pic}}
\newcommand{\diag}{\operatorname{diag}}
\newcommand{\End}{\operatorname{End}}
\newcommand{\Gm}{\bG_{{\rm m}}}
\newcommand{\GL}{\operatorname{GL}}
\newcommand{\SL}{\operatorname{SL}}
\newcommand{\Hilb}{\operatorname{Hilb}}
\newcommand{\Aut}{\operatorname{Aut}}
\newcommand{\PGL}{\operatorname{PGL}}
\newcommand{\im}{\operatorname{im}}
\newcommand{\Id}{\operatorname{Id}}
\newcommand{\Ind}{\operatorname{Ind}}
\newcommand{\sgn}{\operatorname{sgn}}
\newcommand{\Imm}{\operatorname{Imm}}
\newcommand{\Stab}{\operatorname{Stab}}
\newcommand{\Isom}{\operatorname{Isom}}
\newcommand{\uIsom}{\underline{\Isom}}
\newcommand{\Ext}{\operatorname{E}}
\newcommand{\I}{\operatorname{I}}
\newcommand{\bmu}{\bm{\mu}}
\newtheorem{theorem}{Theorem}[section]
\newtheorem{Teo}[theorem]{Theorem}
\newtheorem*{Teo*}{Theorem}
\newtheorem{Lemma}[theorem]{Lemma}

\newtheorem{Cor}[theorem]{Corollary}
\newtheorem{example}[theorem]{Example}

\newtheorem{Prop}[theorem]{Proposition}
\newtheorem*{Ques*}{Question}
\newtheorem{Specialcase}[theorem]{Special case of cite}
\newtheorem{Claim}{Claim}
\newtheorem{Step}{Step}
\newtheorem{Step_1}{Blow-up}
\newtheorem{Step_2}{Step}
\newtheorem{Step_3}{Step}

\theoremstyle{definition}
\newtheorem{Oss}[theorem]{Remark}
\newtheorem*{Oss'}{Remark}
\newtheorem{EG}[theorem]{Example}
\newtheorem{Def}[theorem]{Definition}
\newtheorem*{Def*}{Definition}
\newtheorem{Notation}[theorem]{Notation}
\newtheorem{Remark}[theorem]{Remark}

\newcommand{\marg}[1]{\normalsize{{\color{red}\footnote{{\color{blue}#1}}}{\marginpar[{\color{red}\hfill\tiny\thefootnote$\rightarrow$}]{{\color{red}$\leftarrow$\tiny\thefootnote}}}}}
\newcommand{\Gio}[1]{\marg{(Giovanni) #1}}

\newcommand{\A}[1]{\marg{(Andrea) #1}}


\begin{document}
\title[Effective morphisms and quotient stacks]{Effective morphisms and quotient stacks}
	\author[A. Di Lorenzo]{Andrea Di Lorenzo}
	\address[A. Di Lorenzo]{Humboldt Universit\"{a}t zu Berlin, Germany}
	\email{andrea.dilorenzo@hu-berlin.de}
	\author[G. Inchiostro]{Giovanni Inchiostro}
	\address[G. Inchiostro]{University of Washington, Seattle, WA}
	\email{ginchios@uw.edu}
	\maketitle
	\begin{abstract}
		We give a valuative criterion for when a smooth algebraic stack with a separated good moduli space is the quotient of a separated Deligne-Mumford stack by a torus. For doing so, we introduce a new class of morphisms, the so-called \textit{effective} morphisms, which are a generalization of separated morphisms.
	\end{abstract}
\section{Introduction}
It has been clear since Grothendieck, that often there are advantages by considering objects as their functor of points. When one parametrizes objects with automorphisms,  it is natural to consider algebraic stacks.  As affine schemes are the building blocks for schemes, quotient stacks are the the building blocks for a wide class of stacks \cite{luna}.  Therefore, having some criteria for when an algebraic stack is a quotient stack is a foundational question which is worth investigating, both for purely theoretical reasons, and (as one expects) because it has some consequences on the moduli problem represented by the stack. For example, smooth quotient stacks have a well-defined notion of integral Chow ring \cites{EG}, and some times they measure the failure of $\operatorname{Br}= \operatorname{Br}'$ for schemes \cite{edidin2001brauer}.

While for schemes one has a cohomological criteria to determine if a scheme is affine, it is not at all obvious to come up with a criteria for when an algebraic stack is a quotient stack, nor that a criteria that completely characterizes quotient stacks should exist in the first place. 
Therefore in this paper we are interested in mildly relaxing the quotient assumption. In particular, we are interested in understanding when an algebraic stack $\cX$ is a quotient of a Deligne-Mumford stack $\cY$: one can understand Deligne-Mumford stacks as the the simplest stacks one encounters, with non-trivial stacky structure. 

In \Cref{section effective} we introduce a class of morphisms, the so called \textit{effective} morphisms, which have the following advantages:
\begin{enumerate}
\item Generalize separated morphisms,
\item Are defined by a valuative criterion, and
\item Are stable under composition, base change, and are not \'etale local on the target.
\end{enumerate}
The relevance of effective morphisms lies in the fact that they can be used to recognize quotient of DM stacks by torus actions. What follows is our main result.
\begin{Teo}[\Cref{teo quotient implies effective} and \Cref{teo effettivo implica ho il gen set}]\label{teo intro}Let $\cX$ be a smooth algebraic stack and $\pi:\cX\to X$ a good moduli space morphism.  Assume that $X$ is separated. Then $\pi$ is effective if and only $\cX$ is the quotient of a separated Deligne-Mumford stack by a split torus.
\end{Teo}
Effective morphisms are a generalization of separated morphisms, in the sense that all separated morphisms are effective, but effective morphisms are not \'etale local on the target. Observe that not being \'etale local on the target is essential for our application, as from \cite{luna} any class of morphisms which is \'etale local on the target would not measure the property of being a quotient stack.

The complete definition of effective is \Cref{def:effective},  but on a first approximation one can understand it as follows. Consider $R'$ a DVR and $\spec(R')\cup_\eta \spec(R')$ the non-separated $\spec(R')$, namely the scheme obtained by gluing two copies of $\spec(R')$ along their generic points.
Assume that there is an action of $\bmu_2$ on $\spec(R')\cup_\eta \spec(R')$ which  away from the origins is a Galois action,  and it swaps the two origins. We can take the quotient of this action, $P$. One can check that $P$ is non-separated, but it has a morphism $P\to \spec(R)$, where $R$ is the a DVR whose extension is $R'$, and such that the field extension $K(R)\subseteq K(R')$ has degree 2. An effective morphism of stacks $\cX\to \cY$ has to satisfy the following codimension one filling property:
 $$\begin{tikzcd}
        P\ar[r] \ar[d,] & \cX \ar[d] \\
        \spec(R)\ar[r] \ar[ur,dotted] &\cY.
    \end{tikzcd}$$
Finally, we remark that along the way we prove a structure theorem for classifying spaces of twisted tori (see \Cref{theorem from semidirect product to nonsplit torus}) which might be of independent interest.



The paper is organized as follows. In \Cref{section effective} we introduce effective morphisms, and we prove that if $\cX$ is an algebraic stack with a good moduli space admitting a separated, relatively Deligne-Mumford morphism $\cX\to \cB \Gm^n$, then the good moduli space map is effective (\Cref{teo quotient implies effective}). In \Cref{section torsors and twisted tori} we report some results about torsors and twisted tori; in particular we prove that the classifying space of a twisted torus is a certain fiber product (\Cref{theorem from semidirect product to nonsplit torus}).
In \Cref{section when BG not eff} we prove that $\cB G$ is effective if and only if $G$ is a central extension of a finite group by a split torus, whereas in \Cref{section from eff to quotient} we prove the other direction of \Cref{teo intro}
\subsection{Conventions}
All algebraic stacks will have affine (hence separated) diagonal. We work over a field $k$ which is of characteristic 0. Given two morphisms $X\to \spec(A)$ and $\spec(B)\to \spec(A)$, we denote by $X_B:=X\times_{\spec(A)}\spec(B)$. All groups will be quasi-projective, and all actions will be left actions. We will use the shorthand DM for Deligne-Mumford.
\subsection*{Acknowledgements} We are thankful to Jarod Alper, Dori Bejleri, Siddarth Mathur, Minseon Shin, David Rydh and Angelo Vistoli for helpful conversations. 

\section{Effective morphisms}\label{section effective}

In this section we introduce the main class of morphisms of algebraic stacks, the so-called \textit{effective morphisms}, which are a mild generalization of separated morphisms.

\subsection{The ``local bug-eyed cover”}In this section we recall a construction of a special push-out, related
to the so called ``bug-eyed cover” (\cite[Example 2.19]{kollar1997quotient} and \cite[Example 3.9.2]{Alp21}). 

Consider a DVR $R$ with fraction field $K(R)$, let $K(R')$ be a finite extension of $K(R)$ and let $R'$ be a DVR with fraction field $K(R')$ such that the inclusions $R\to K(R) \to K(R')$ lands in $R'$; namely, the ring $R'$ is the localization along the closed point of the normalization of $\spec(R)$ in $K(R')$. Assume also that $\spec(R')\to \spec(R) $ is \'etale. We have the following diagram:
$$\xymatrix{\spec(K(R'))\ar[r] \ar[d] & \spec(K(R))\ar[d] \\ \spec(R')\ar[r] & \spec(R). }$$
Recall that as $R\to R'$ is \'etale, the uniformizar of $R$ is also an uniformizer for $R'$. If we denote by $k_R$ and $k_{R'} $ the residue fields of $R $ and $R'$ respectively, we have an extension $k_R\subseteq k_{R'}$. 

It follows from \cite[Theorem B]{rydh2011etale} that if this extension is surjective, the diagram above is a pushout in the category of algebraic stacks. In other terms, for every algebraic stack $\cX$, the data of a map $\spec(R)\to \cX$ is equivalent to two maps $a:\spec(R')\to \cX$ and $b:\spec(K(R))\to \cX$, together with an isomorphism $\sigma:a|_{K(R')}\to b|_{K(R')}$. However, if $k_R\subsetneq k_{R'}$, the two maps $\spec(K(R'))\to \spec(K(R))$ and $\spec(K(R'))\to\spec(R')$ admit a pushout $P:=P_{R,R'}$ (from \cite{rydh2011etale}) which is \textit{not} isomorphic to $\spec(R)$.
\begin{Notation}
    We denote by $P_{R,R'}$ the pushout described above. When there is no ambiguity, we drop the subscripts $R$ and $R'$.
\end{Notation}
\begin{EG}
We describe the push-out $P$ in the special case where $[K(R):K(R')] = 2$ (and $k_R \neq k_{R'}$). In this case there is a $\bmu_2$-action on $\spec(R')$, whose associated quotient scheme is $\spec(R)$. We can pull back the cartesian diagram on the left along $\spec(R')\to\spec(R)$ to obtain the one on the right:
\[
\begin{tikzcd}
\spec(K(R')) \ar[r] \ar[d] & \spec(R') \ar[d] \\
\spec(K(R)) \ar[r] & P,
\end{tikzcd}
\quad
\begin{tikzcd}
\spec(K(R'))\times\bmu_2 \ar[r] \ar[d] & \spec(R')\times\bmu_2 \ar[d] \\
\spec(K(R')) \ar[r] & P'.
\end{tikzcd}
\]
Observe that $P'$ is still a pushout, because formation of the pushout commutes with base change \cite[Theorem C]{rydh2011etale}.
By construction, the closed points of $P'$ correspond to two copies of $\spec(k_{R'})$: we can then regard $P'$ as two copies of $\spec(R')$ glued at the generic points. It follows then that $P$ is the algebraic space obtained by quotiening $P'$ by the action of $\bmu_2$ that on the generic point agrees with the Galois action, and it switches the two closed points.
\end{EG}
\subsection{Effective morphisms}
We are ready to introduce a class of morphisms this paper revolves around.
\begin{Def}\label{def:effective}
    Let $\cX\to\cY$ be a morphism of algebraic stacks, let $R$ (resp. $R')$ be a DVR with fraction field $K(R)$ (resp. $K(R')$) and $\beta:\spec(R')\to\spec(R)$ be an \'etale extension. Let $P$ be the pushout in the category of algebraic stacks of $\spec(K(R'))\to\spec(K(R))$ and $\spec(K(R'))\to\spec(R')$. Then we say that $\cX\to\cY$ is \emph{effective} if, given any $R$ and $R'$ as above and a diagram of solid arrows as the one below, one can always find a dotted arrow such that the resulting diagram commutes:
    \[
    \begin{tikzcd}
        P \ar[r] \ar[d] & \cX \ar[d] \\
        \spec(R) \ar[r] \ar[ur, dotted] & \cY.
    \end{tikzcd}
    \]
    As usual, if $k$ denotes the ground field, we say that $\cX$ is effective if $\cX\to \spec(k)$ is effective.
\end{Def}
\begin{Remark}
More explicitly, one can think of effective morphisms as morphisms of algebraic stacks having the following property: for any morphism $f:\spec(R)\to \cY$ suppose that
\begin{enumerate}
\item we can lift $f$ generically, i.e. there exists a lifting $f_{K(R)}:\spec(K(R))\to\cX$,
\item we can lift $f$ globally up to \'etale covers, in the sense that there exists an \'etale extension $\beta:\spec(R')\to\spec(R)$ and a lifting $g:\spec(R')\to\cX$, and
\item there is an isomorphism $\varphi:(\beta|_{K(R')})^*f_{K(R)}\to g|_{K(R')}$
\end{enumerate}
then we can actually lift $f$ to a morphism $\spec(R)\to\cX$, in a way compatible with $f_{K(R)}$, $g_{K(R')}$ and $\varphi$. This situation arises typically when $\cX\to\cY$ is universally closed, thanks to the existence part of the valuative criterion of properness for algebraic stacks \cite{stacks-project}*{\href{https://stacks.math.columbia.edu/tag/0CLW}{Tag 0CLW},\href{https://stacks.math.columbia.edu/tag/0CLX} {Tag 0CLX}}.
\end{Remark}
\begin{EG} The classifying stack $\cB \GL_2$ is not effective. We will see in \Cref{section when BG not eff}, in particular \Cref{Teo BG effective implies G is a central ext of a torus} that $\cB G$ is effective if and only if $G$ is a central extension of a split torus by a finite group. This can also be proven directly, using the same argument of \Cref{prop sl2 not effective}.
%The morphism $\cB\GL_2\to\spec(k)$ is not effective. Indeed, in the situation of \Cref{def:effective}, let $\pi$ be the uniformizer of $R$ (hence of $R'$). Let $v$ be an element of $R'$ whose class in the residue field $k_{R'}$ is not contained in $k_{R}$. We can form the diagram
%\[
%\begin{tikzcd} \spec(K(R')) \ar[r] \ar[d] & \spec(R') \ar[d]\\ \spec(K(R)) \ar[r] & \cB\GL_2 \end{tikzcd} \]
%which corresponds to the following data:
%\begin{itemize}
    %\item the maps to $\cB\GL_2$ correspond both to the trivial rank two vector bundle;
    %\item the 2-commutativity of the diagram is given by the isomorphism of free modules $K(R')^{\oplus 2}\xrightarrow{\simeq}K(R')^{\oplus 2}$ defined via the matrix
    %\[A= \begin{pmatrix} 1 & v \\ 0 & \pi \end{pmatrix}.\]
%\end{itemize}
%This data defines a rank two vector bundle $V$ on $P$. To show that the induced map $P\to\cB\GL_2$ does factors through $\spec(R)$, it is enough to show that $V$ is not trivial. To trivialize this vector bundle is equivalent to finding matrices $B\in \GL_2(R')$ and $C\in\GL_2(K(R))$ such that $BAC=\Id$.
%Suppose that this is possible: then we would have $BA=C^{-1}$. Observe that the matrix on the right has entries in $K(R)$, whereas the one on the left has entries in $R'$. Therefore, both matrices have entries in $K(R)\cap R'=R$. In particular, for $B=(b_{ij})$, we deduce
%\[b_{i1} \in R, \quad  b_{i1}v+b_{i2}\pi \in R, \quad i=1,2.\]
%This implies that the equivalence class of $b_{i1}v$ is contained in $k_R$, and we already know that the equivalence class of $b_{i1}$ is in $k_R$. As $v$ does not belong to $k_R$ by hypothesis, the only possibility left is that $b_{i1}=\pi b'_{i1}$ for some $b_{i1}'$ in $R$. From this we deduce that the determinant of $B$ is equal to $\pi z$, where $z$ is an element in $R'$.
%On the other hand, as $B$ belongs to $\GL_2(R')$, its determinant must be invertible in $R'$, hence it cannot be of the form $\pi z$. We reached a contradiction, thus the rank two vector bundle $V$ on $P$ cannot be trivial.
\end{EG}
It turns out that effective morphisms are a generalization of effective morphisms.
\begin{Prop}\label{prop:sep to eff}
    Assume that $\cX\to\cY$ is separated. Then it is also effective.
\end{Prop}
Before proving this result, we need the following.
\begin{Lemma}\label{lm:rel diag affine}
    With our conventions, for every morphism $\cX \to \cY$ the relative diagonal $\delta_{\cX/\cY}:\cX\to \cX\times_{\cY}\cX$ is affine.
\end{Lemma}
\begin{proof}
We want to show that for any affine scheme $Z$ mapping to $\cX\times_\cY\cX$, the fiber product $$V:=Z\times_{\delta_{\cX/\cY}} (\cX\times_\cY \cX)$$ is affine. Consider the following diagram
\[
\begin{tikzcd}
    V \ar[r]\ar[d] & Z \ar[d, "f"] \ar[dr, "\Id"] & \\
    F' \ar[r]\ar[d] & F\ar[d] \ar[r] & Z \ar[d] \\
    \cX \ar[r] & \cX\times_{\cY}\cX \ar[r, "g"] & \cX\times\cX,
\end{tikzcd}
\]
where the squares are all cartesian. Observe that $F'$ is affine, because the diagonal of $\cX$ is affine and $Z$ is affine. Observe also that $F$ is affine: indeed, the diagonal of $\cY$ is affine, and we can write the morphism $g$ as
\[ \cX\times_{\cY}\cX \simeq \cX\times_{\cY}\cY\times_{\cY}\cX \xrightarrow{\Id_{\cX}\times_{\cY}(\delta_{\cY})\times_{\cY}\Id_{\cX}} \cX\times_{\cY}(\cY\times\cY)\times_{\cY}\cX\simeq \cX\times\cX. \]
This shows that $g$ is obtained by pulling back $\delta_{\cY}$, hence it is affine as well, which implies that $F$ is affine. This implies that $f$ is affine, hence also $V\to F'$ is affine, from which we conclude that $V$ is affine.
\end{proof}
\begin{proof}[Proof of \Cref{prop:sep to eff}]
Suppose to have a (2-commutative) diagram of solid arrows as the one in \Cref{def:effective}: this is equivalent to having the following diagram of solid arrows 
\begin{equation}
\begin{tikzcd}
    \spec(K(R'))\ar[r, "j"]\ar[d] & \spec(R')  \ar[r, "\xi"] & \cX \ar[d] \\
    \spec(K(R)) \ar[r] \ar[urr] & \spec(R) \ar[r] \ar[from=u, crossing over] \ar[ur, dotted] & \cY.
\end{tikzcd}
\end{equation}
Our goal is to produce a dotted arrow as in the diagram above: our strategy consists in making the arrow $\xi:\spec(R')\to\cX$ descend along the \'etale cover $\spec(R')\to\spec(R)$. For this, consider the projections
\[\begin{tikzcd} \pi_1,\pi_2:\spec(R')\times_{\spec(R)}\spec(R') \ar[r] & \spec(R') 
\end{tikzcd}\]
\[
\begin{tikzcd}
    \rho_1,\rho_2:\spec(K(R'))\times_{\spec(K(R)}\spec(K(R'))\ar[r] & \spec(K(R'))
\end{tikzcd}
\]
respectively on the first and the second factor. We want to construct an isomorphism $\varphi:\pi_1^*\xi \simeq \pi_2^*\xi$ in $\cX(\spec(R')\times_{\spec(R)}\spec(R'))$ that satisfies the cocycle condition: in other words, we want a section of
\[ \uIsom(\pi_1^*\xi,\pi_2^*\xi) \longrightarrow \spec(R')\times_{\spec(R)}\spec(R').  \]
Observe now that we have a cartesian diagram
\begin{equation}
    \begin{tikzcd}
        \uIsom(\rho_1^*j^*\xi,\rho_2^*j^*\xi) \ar[r] \ar[d, "q"] & \uIsom(\pi_1^*\xi,\pi_2^*\xi) \ar[r] \ar[d, "p"] & \cX \ar[d, "\delta_{\cX/\cY}"] \\
        \spec(K(R'))\times_{\spec(K(R)}\spec(K(R')) \ar[r] & \spec(R')\times_{\spec(R)}\spec(R') \ar[r] & \cX\times_{\cY}\cX
    \end{tikzcd}
\end{equation}
This is a cartesian diagram of schemes, because the diagonal $\delta_{\cX/\cY}$ is affine, hence representable (\Cref{lm:rel diag affine}). The morphism $p$ is proper, because by definition of separatedness $\delta_{\cX/\cY}$ is proper.

Observe that the morphism $j\circ\xi$ descends by construction along $\spec(K(R'))\to\spec(K(R))$, hence there is an isomorphism $\rho_1^*j^*\xi\simeq \rho_2^*j^*\xi$, i.e. a section of $q$. 

The scheme $\spec(K(R'))\times_{\spec(K(R)}\spec(K(R'))$ is the generic point of $\spec(R')\times_{\spec(R)}\spec(R')$, thus we have constructed a generic section from this scheme to $\uIsom(\pi_1^*\xi,\pi_2^*\xi)$. 

As $\spec(R')\times_{\spec(R)}\spec(R')$ is regular of dimension one and $p$ is proper, we can extend the generic section to a regular section of $p$. This gives the desired isomorphism $\varphi$; the cocycle condition is satisfied, because we already know that it is satisfied on the generic point.
\end{proof}
\begin{Cor}\label{cor hom from spec r sono determinati da hom from P}
    Let $R\to R'$ be an unramified extension of DVRs, and let $P$ be their bugged-eye cover. Then for every algebraic stack $\cX$ (with affine diagonal), the map $Hom(R,\cX)\to Hom(P,\cX)$ is fully faithful.
\end{Cor}
\begin{proof}Consider two morphisms $f,g:\spec(R)\to \cX$ which are isomorphic when we restrict them to $P$. They give rise to the following diagram
$$\begin{tikzcd}
        \uIsom(f|_P,g|_P) \ar[r] \ar[d,"q"] & \uIsom(f,g) \ar[r] \ar[d, "p"] & \cX \ar[d, "\delta_{\cX/\cY}"] \\
        P \ar[r] &\spec(R) \ar[r,"f\times g"] & \cX\times_{\cY}\cX.
    \end{tikzcd}$$
    Since $f|_P\cong g|_P$, the map $q$ has a section. Then $p$ has a section when we pull it back to $P$, and as $p$ is affine hence separated, it has a section by definition of effective. 
\end{proof}
\begin{Cor}\label{cor composition effective implies the first one effective}
Assume that the composition $X\to Y \to Z$ is effective. Then $X\to Y$ is effective.
\end{Cor}
\begin{proof}Consider the following diagram on the left:
$$\xymatrix{P\ar[d]_\pi \ar[r] ^a & X\ar[d]^p&\\ \spec(R)\ar[r]^-b &Y\ar[r] & Z} \text{ }\text{ }\text{ }\text{ }\text{ }\text{ }\text{ }\text{ }\xymatrix{P\ar[d]_\pi \ar[r] ^a & X\ar[d]^p\\ \spec(R)\ar[r]_-b \ar[ru]^-f&Y}$$
Since $X\to Z$ is effective, there is a map $f:\spec(R)\to X$ such that $a$ and $ f\circ \pi$ are isomorphic. But then $p\circ a$ and $p\circ f \circ \pi$ are isomorphic, and $p\circ a$ is isomorphic to $b\circ \pi$. So from \Cref{cor hom from spec r sono determinati da hom from P} also  $p\circ f $ and $b$ are isomorphic. In particular the diagram on the right is commutative, so $p$ is effective.\end{proof}
Non separated morphisms can be non-effective, even in the case of algebraic spaces.
\begin{EG}
    Let $R=\mathbb{C}(x^2)[y]_{(y)}$ and consider the \'etale extension of DVRs
    \[R\longrightarrow \mathbb{C}(x)[y]_{(y)}=:R'.\]
    Let $P$ be the pushout of $\spec(K(R'))\to\spec(K(R))$ and $\spec(K(R'))\to\spec(R')$. Then $P\to\spec(R)$ is a morphism of algebraic spaces that is not effective: indeed, if it were effective, we would have a section of the field extension $\mathbb{C}(x^2) \hookrightarrow \mathbb{C}(x)$, which is not the case.
\end{EG}
\begin{Lemma}\label{lm:composition}
    Composition of effective morphisms is effective, and being effective is stable under base change.
\end{Lemma}
\begin{proof}
    The first statement is straightforward. To prove the second one, for any effective morphism $f:\cX\to\cY$ and any morphism $g:\cZ\to\cY$, we have a 2-commutative diagram of solid arrows
    \[
    \begin{tikzcd}
        P \ar[r] \ar[d, "p"] & \cX \times_{\cY} \cZ \ar[r] & \cX \ar[d, "f"] \\
        \spec(R) \ar[r] \ar[urr, dotted] & \cZ \ar[from=u, crossing over] \ar[r, "g"] & \cY
    \end{tikzcd}
    \]
    Effectiveness of $f$ implies that there is a lifting given by the dotted arrow above. In this way, we have well defined morphisms $\xi:\spec(R)\to\cX$ and $\zeta:\spec(R)\to\cZ$; to obtain a morphism $\spec(R)\to\cX\times_{\cY}\cZ$, the only missing data is an isomorphism $\alpha:f(\xi)\simeq g(\zeta)$ or, in other terms, a section of $\uIsom_{\spec(R)}(f(\xi),g(\zeta))$. 

    The morphism $P\to \cX\times_{\cY}\cZ$ corresponds to the data $p^*\xi$, $p^*\zeta$ and an isomorphism $p^*\xi \simeq p^*\zeta$. In particular, we have a commutative diagram of solid arrows.
    \[
    \begin{tikzcd}
        P \ar[r] \ar[d] & \uIsom_{\spec(R)}(f(\xi),g(\zeta)) \ar[d, "q"] \\
        \spec(R) \ar[r, "\Id"] \ar[ur, dotted] & \spec(R)
    \end{tikzcd}
    \]
    The morphism $q$ is affine because by hypothesis $\cY$ has affine diagonal, hence it is separated and thus effective by \Cref{prop:sep to eff}. This means that we have a lifting given by the dotted arrow in the diagram above, which is the desired section of $\uIsom_{\spec(R)}(f(\xi),g(\zeta))$.  
\end{proof}
\begin{Prop}
    The morphism $\cB\Gm^n\to\spec(k)$ is effective.
\end{Prop}
\begin{proof}
    First we prove the case $n=1$. Let $R$ be a DVR with parameter $\pi$, let $f:\spec(R')\to \spec(R)$ be an \'etale extension of DVRs and suppose to have a 2-commutative diagram of solid arrows
    \begin{equation}\label{eq:BGm eff diag}
    \begin{tikzcd}
        P\ar[r] \ar[d, "p"] & \cB\Gm \ar[d] \\
        \spec(R) \ar[r] \ar[ur,dotted, "g"] & \spec(k)
    \end{tikzcd}
    \end{equation}
    where $P$ denotes as usual the pushout of $j:\spec(K(R'))\to\spec(R')$ and $q:\spec(K(R'))\to\spec(K(R))$. We want to construct a dotted arrow $g$ as in the diagram above.

    The data of a map $P\to\cB\Gm$ corresponds to a triple $(L',M,\varphi)$ where $L'$ (resp. $M$) is a line bundle on $\spec(R')$ (resp. on $\spec(K(R))$) and $\varphi:j^*L'\simeq q^*M$ is an isomorphism of line bundles. Observe that both $L'$ and $M$ are trivial, hence we can think of $\varphi$ as an element of $\Gm(K(R'))$, i.e. $\alpha=\pi^du$ where $d$ is an integer and $u$ is invertible in $R'$.

    The data of a map $g:\spec(R)\to\cB\Gm$ corresponds to a (trivial) line bundle $L$ on $\spec(R)$, and its pullback $p^*g:P\to\cB\Gm$ corresponds to the triple $(f^*L,L_{K(R)},\Id)$. The arrow $g$ makes the diagram (\ref{eq:BGm eff diag}) 2-commutative if and only if there exists an isomorphism of the triple $(f^*L,L_{K(R)},\Id)$ with $(L',M,\alpha)$, i.e. if there exist isomorphisms $a:f^*L\simeq L'$, $b:M\simeq L_{(K(R)}$ such that $q^*b\circ\varphi\circ j^*a = \Id$. 
    
    As all the line bundles involved are trivial, this is equivalent to finding $a\in \Gm(R')$ and $b\in\Gm(K(R))$ such that $a\cdot \pi^d u\cdot b=1$. We can pick $a= u^{-1}$ and $b=\pi^{-d}$ to obtain this equality, thus proving that the map $g:\spec(R)\to\cB\Gm$ given by the trivial line bundle makes (\ref{eq:BGm eff diag}) commutative.

    For $n\geq 1$, we argue by induction: indeed, there is a cartesian diagram
    \[
    \begin{tikzcd}
        \cB\Gm^n \ar[r] \ar[d] & \cB\Gm \ar[d] \\
        \cB\Gm^{n-1} \ar[r] & \spec(k).
    \end{tikzcd}
    \]
    Both the right vertical arrow and the bottom horizontal arrow are effective by inductive hypothesis, and the left vertical arrow is effective because being effective is stable under base change (\Cref{lm:composition}). As the composition of effective arrows is effective by the same lemma, we obtain the desired conclusion.
\end{proof}
\begin{Prop}\label{prop:BG eff}
Let $G$ be an algebraic group fitting in the short exact sequence
\[1\longrightarrow \Gm^n\overset{i}{\longrightarrow} G\longrightarrow F\longrightarrow 1\]
where the normal subgroup $i(\Gm^n)$ is central and $F$ is finite. Then $\cB G\to\spec(k)$ is effective.
\end{Prop}
\begin{proof}
    Due to the fact that $\Gm^n$ is central in $G$, we have \cite[Section 1.2.2]{cesnavic} long exact sequences
    \[
    \begin{tikzcd}
        \oH^1(\spec(R),G) \ar[r, "j_R"] \ar[d, "p^*"] & \oH^1(\spec(R),F) \ar[d, "p^*"] \ar[r, "\delta_R"] & \oH^2(\spec(R),\Gm^n) \ar[d, "p^*"]\\
        \oH^1(P, G) \ar[r, "j_P"] & \oH^1(P, F) \ar[r, "\delta_P"] & \oH^2(P, \Gm^n).
    \end{tikzcd}
    \]
    Let $x$ be an element in $\oH^1(P,G)$. Then $j_P(x)$ can be lifted to an element $y$ in $\oH^1(\spec(R), F)$ because $\cB F\to\spec(k)$ is separated, hence effective because of \Cref{prop:sep to eff}. Observe that $\delta_P(p^*y)=0=p^*(\delta_R(y))$, thus if we prove that the last vertical arrow is injective, we deduce that $\delta_R(y)=0$ and consequently $y=j_R(z)$, with $p^*z=x$, and we are done.

    Observe that the we have a commutative diagram
    \[
    \begin{tikzcd}
    \oH^2(\spec(R),\Gm^n) \ar[r] \ar[d] & \oH^2(\spec(K(R)),\Gm^n) \ar[d, "\simeq"] \\
    \oH^2(P, \Gm^n) \ar[r] & \oH^2(\spec(K(R)),\Gm^n)
    \end{tikzcd}
    \]
    and the right vertical arrow is an isomorphism. The homomorphism $\oH^2(\spec(R),\Gm)\to \oH^2(\spec(K(R),\Gm)$ is injective: indeed, as $\spec(R)$ is regular and affine, we have that $\oH^2(\spec(R),\Gm)$ is isomorphic to the Brauer group, and the Brauer group of $\spec(R)$ injects into the Brauer group of its generic point \cite[Proposition 3.1.3.3]{lieblich2008twisted}. From this it follows that the top horizontal arrow in the last diagram is injective, hence also the left vertical arrow is injective. This concludes the proof.
\end{proof}
We are ready to prove one of the two directions of our main result.
\begin{theorem}\label{teo quotient implies effective}
    Assume that $\cX$ admits a DM morphism $\phi:\cX\to \cB \Gm^n$, and a separated good moduli space $X$. Then the good moduli space map $\cX\to X$ is effective.
\end{theorem}
\begin{proof}
    Let $\cY\to \cX$ the $\Gm^n$-torsor associated to $\phi$. Then $\cY\to \cX$ is an affine morphism, so it is also S-complete from \cite[Proposition 3.42]{AHH}. So the composition
    $\cY\to \cX \to X$ is S-complete since being S-complete is stable under compositions. But then $\cY$ is separated, since a DM stack is S-complete if and only if it is separated. In particular, the morphism $\phi$ is separated. So also from \cite[\href{https://stacks.math.columbia.edu/tag/04YV}{Tag 04YV}]{stacks-project} the morphism $\cX\to X\times \cB \Gm^n$ is separated. In particular, it is effective from \Cref{prop:sep to eff}; since $\cB \Gm^n\to \spec(k)$ is effective also $X\times \cB \Gm^n\to X$ is effective from \Cref{lm:composition}. Since composition of effective morphisms is effective, also $\cX\to X$ is effective as desired.
\end{proof}
\begin{Def}\label{def effective group}
    We say that a group $G$ is effective if $\cB G\to\spec(k)$ is effective.
\end{Def}
\section{Torsors, twisted tori and gerbes}\label{section torsors and twisted tori}
In this section we recall some well-known facts about torsors, twisted tori and gerbes that will be needed in the rest of the paper. In particular, along the way we prove that the classifying space of a twisted torus is isomorphic to a certain fiber product. We will use these results in \Cref{section when BG not eff}.
\subsection{Torsors}
We will be mostly interested in (\'etale) torsors over the spectrum of a field $L$, and $G$ will be a finite type group scheme over $\spec(L)$. We begin with the following:

\begin{Def}
    If $f:X\to Y$ is a $G$-torsor, the automorphisms of $f$ are the automorphisms of $X$ which commute with $f$ and the $G$-action.
\end{Def}
\begin{EG}\label{example aut trivial G torsor}
    If $f:G\times Y\to Y$ is the trivial torsor, there is a bijection between $G$ and the automorphisms of $f$. This bijection sends $G\ni g\mapsto ((h,y)\mapsto (hg^{-1},y))$. This is clearly an isomorphism of $G\times Y$ which commutes with $f$, and it commutes with the action since $h_1(h_2g^{-1})=(h_1h_2)g^{-1}$: the multiplication in $G$ is associative, and our actions are left actions.
\end{EG}

Torsors can be described using descent. Let us first fix some notation.
\begin{Notation}
    In what follows we use use $G$ to denote a group scheme over $k$, we use $X$ for a $G$-torsor which gets trivialized by a Galois extension $k\subseteq k'$, and finally we use $\Gamma$ to denote the Galois group of the extension $k\subset k'$. 
\end{Notation}
In particular $\Gamma$ acts on $G_{k'}$, the action commutes with the group operation in $G_{k'}$, and there is a bijection between the set of $G$-torsors $X\to \spec(k)$ and $H^1(\Gamma, G_{k'})$ (see \cite[Chapter 1 Section 5]{Galois_cohom}); in the following remark we construct one of the two arrows. 

\begin{Remark}\label{remark from cocycle to t functions}
In the context of group cohomology, consider a cocycle 
    $$\Gamma\longrightarrow G_{k'},\quad \gamma\longmapsto g_\gamma.$$
The cocycle condition is that $g_{\gamma\delta}=g_{\gamma}(\gamma*g_{\delta})$ where $*$ is the action of $\Gamma$ on $G_{k'}$. By descent, this defines a $G$-torsor over $\spec(k)$, which we can understand as follows. 

The cocycle defines an action of $\Gamma$ on $G_{k'}$ via $(\gamma,x)\mapsto (\gamma*x)\cdot g^{-1}_\gamma$, where we denoted by $\cdot$ the group multiplication. The cocycle condition guarantees that this is indeed an action, namely $$x\mapsto (\gamma*x)\cdot g^{-1}_\gamma\mapsto \delta*((\gamma*x)\cdot g^{-1}_\gamma)\cdot g_{\delta}^{-1} = ((\delta\gamma)*x)\cdot (\delta*g^{-1}_\gamma)\cdot g_{\delta}^{-1} =((\delta\gamma)*x) \cdot g_{\delta\gamma}^{-1}.$$
By taking the quotient of $G_{k'}$ with respect to this action, we get the desired $G$-torsor.
For example, the trivial cocycle $\gamma \mapsto 1$ gives rise to the trivial $G$-torsor: the $G$-torsor given by taking the $\Gamma$-invariants in $G_{k'}$. 
\end{Remark}
\begin{Def}
    Given a cocycle $\Phi:\Gamma\to G_{k'}$, $\gamma\mapsto g_\gamma$, for every $\gamma\in \Gamma$ we have an isomorphism $G_{k'}\to G_{k'}$, $x\mapsto (\gamma\ast x)\cdot g_\gamma^{-1}$. We will call these isomorphisms the \textit{transition functions} of the cocycle $\Phi$.
\end{Def}
 The transition functions are the ones that define a ``twisted'' action of $\Gamma$ on $G_{k'}$; the quotient of $G_{k'}$ with respect to this action is the $G$-torsor associated to the cocycle.
\begin{EG}\label{example cocycle that gives the galois extension}
    Consider the trivial action of $\Gamma$ on itself, and consider the $\Gamma$-torsor associated to the cocycle $\Id:\Gamma\to \Gamma$, whose associated transition functions are $g \mapsto g\gamma^{-1}$. Then the resulting $\Gamma$-torsor over $\spec(k)$ is the quotient of $\Gamma$ by the action on itself defined by the transition functions: this quotient is exactly $\spec(k')\to \spec(k)$, with the $\Gamma$-action being the action of the Galois group.
\end{EG}
\begin{Lemma}\label{lemma automorphisms of a torsor}
    Assume that $f:X\to \spec(k)$ is a $G$-torsor, assume that $f$ is trivialized on the Galois extension $k\subseteq k'$ with Galois group $\Gamma$, and assume that $X$ is given by the cocycle $\Gamma\to G_{k'}$, $\gamma\mapsto g_\gamma$. Then there is a (non-canonical) bijection between $\Aut(X)$ and the set $$ \{a\in G_{k'}\text{ such that }\gamma * a = g_{\gamma}^{-1}\cdot a\cdot g_\gamma\}.$$
\end{Lemma}
\begin{proof}
    The automorphisms of $f$ are the automorphisms of $f_{k'}:X_{k'}\to \spec(k')$ which descend, i.e. which commute with the transition functions. As $X_{k'}$ is a trivial torsor, up to choosing an isomorphism $X_{k'}\to G_{k'}$, we can identify the automorphisms of $X_{k'}$ with $G_{k'}$, as in \Cref{example aut trivial G torsor}. Those which descend are the $\beta$'s which make the following diagram commutative for every $\gamma\in \Gamma$:
    $$\xymatrix{G_{k'}\ar[d]_{x\mapsto (\gamma \ast x)\cdot g_\gamma^{-1}} \ar[r]^\beta & G_{k'}\ar[d]^{x\mapsto (\gamma \ast x)\cdot g_\gamma^{-1}} \\ G_{k'}\ar[r]^\beta & G_{k'}.}$$
    As $\beta$ is given by left multiplication for $a\in G_{k'}$, it is now straightforward to see the desired bijection.
\end{proof}


\subsection{Twisted tori and gerbes}
\begin{Def}
   A twisted torus over $k$ is a group scheme $G$ which admits an isomorphism $G_{\overline{k}}\cong (\Gm^n)_{\overline{k}}$ for a certain $n$, where $\overline{k}$ is the algebraic closure of $k$.
\end{Def}
It is well known that if $G$ is a twisted torus, then there is a finite Galois extension $k'/k$ together with an isomorphism $G_{k'}\cong (\Gm^n)_{k'}$. We now recall \cite[Proposition 2.4]{Arithmetictoricvars}:
\begin{Prop}\label{prop from homs to twisted tori} Let $k\subseteq k'$ a Galois extension with Galois group $\Gamma$.
    Then there is a bijection between homomorphisms $\Gamma\to \GL_n(\mathbb{Z})$ and pairs $(T,f)$ where $T\to \spec(k)$ is a twisted torus and $f:T_{k'}\cong (\Gm^n)_{k'}$ is an isomorphism. So this induces a bijection between conjugacy classes of homomorphisms $\Gamma\to \GL_n(\mathbb{Z})$ and twisted tori $T\to \spec(k)$ such that $T_{k'}\cong (\Gm^n)_{k'}$.
\end{Prop}
 The argument essentially follows from the fact that twisted tori as above are parametrized by the group $H^1(\Gamma, \Aut((\Gm^n)_{k'}))$, and the action of $\Gamma$ on $\Aut((\Gm)_{k'})$ is trivial. So the cocycle condition $a_{\gamma \delta}=a_\gamma \gamma\ast a_\delta$ becomes $a_{\gamma\delta}=a_\gamma a_\delta$ (namely, $a$ is a homomorphism); and two cocycles $a$ and $a'$ are conjugated if and only $a$ is the composition of $a'$ with an inner automorphism. 
 
\begin{Notation}\label{notation twisted torus}Given an homomorphism $\phi:\Gamma\to \GL_n(\bZ)$, we denote by $T_\phi$ the twisted torus induced by $\phi$ as in \Cref{prop from homs to twisted tori}.
\end{Notation}
\begin{Lemma}\label{lemma points of nonsplit torus using descent}
    Let $T_\phi$ be a torus twisted by $\phi$. Then, for every scheme $S$ over $\spec(k)$, there is a bijection $$T_\phi(S)\longleftrightarrow \{x \in \Gm^n(S_{k'})\text{ such that }x =\phi(\gamma)(\gamma \ast x)\}.$$
\end{Lemma}
\begin{proof}
 This follows from descent. Indeed, a morphism $S\to T_\phi$ is equivalent, via descent, to a morphism $S_{k'}\to \Gm^n$ which satisfies the descent condition. This is a morphism $x\in \Gm^n(S_{k'})$ which makes the following diagram commutative
 $$\xymatrix{S_{k'}\ar[d]^{\Id} \ar[r]^x & \Gm^n\ar[d]^{g\mapsto \phi(\gamma)(\gamma\ast g)} \\S_{k'}\ar[r]^x &\Gm^n.} $$
\end{proof}




Now, given a homomorphism $\phi:\Gamma\to \GL_n(\bZ)$ one can perform another construction, the semidirect product $\Gm^n\rtimes_\phi \Gamma$. The following result clarifies the relation between the group $\Gm^n\rtimes_\phi \Gamma$ and $\cB T_{\phi}$.
\begin{Teo}\label{theorem from semidirect product to nonsplit torus}
    Let $\phi:\Gamma \to \GL_n(\bZ)$ be a homomorphism, let $G:=\Gm^n\rtimes_\phi \Gamma$, and let $\spec(k)\to \cB\Gamma$ be the morphism associated to the $\Gamma$-torsor $\spec(k')\to \spec(k)$. Consider the following fiber diagram, where the map $\cB G\to \cB\Gamma$ is associated to the surjection $G\to \Gamma$, $(g,\gamma)\mapsto \gamma$:
    $$\xymatrix{\cG = \spec(k)\times_{\cB\Gamma}\cB G\ar[r] \ar[d] & \cB G\ar[d] \\\spec(k)\ar[r] &\cB \Gamma.}$$
    Then $\cG = \cB T_\phi$.
\end{Teo}
\begin{proof}
    As $\cB  G\to\cB\Gamma$ is a gerbe, we already know that $\cG\to\spec(k)$ is a gerbe. It suffices to prove that, if we denote with $\pi:\cG\to \spec(k)$ the first projection, then $\pi$ has a section $\xi$, and that there is an isomorphism $\underline{\Aut}_k(\xi)\cong T_\phi$: indeed, from the first statement it would follow that $\cG\simeq \cB \underline{\Aut}_k(\xi)$, which combined with the second statement would give us the desired conclusion.

    \underline{The map $\pi$ has a section:} from the universal property of the fiber product, it suffices to construct a $G$-torsor $\cF$ over $\spec(k)$ whose associated $\Gamma$-torsor $\cF/\Gm^n$ is isomorphic to $\spec(k')$: the latter statement is equivalent to proving that (1) the pullback to $\spec(k')$ of $\cF/\Gm^n$ is trivial and (2) the associated cocycle is the one in \Cref{example cocycle that gives the galois extension}.

    First observe that there is an action of $\Gamma$ on $(\Gm^n)_{k'} = \spec(k'[t_1^{\pm},...,t_n^{\pm}])$, which we denote by $\ast_{\Gm^n}$, as there is an action of $\Gamma$ on $\spec(k')$ (the Galois action). We also use $\ast_{\Gamma}$ to indicate the trivial action of $\Gamma$ on itself, so that $\sigma\ast_\Gamma \gamma :=\gamma$. This gives an action of $\Gamma$ on $G= \Gm^n\rtimes_\phi \Gamma$, via $$\sigma *(g,\gamma):=(\sigma\ast_{\Gm^n}g,\gamma).$$
    As $\ast_{\Gm^n}$ acts only on the coefficients of $(\Gm^n)_{k'} = \spec(k'[t_1^{\pm},...,t_n^{\pm}])$, for every $\sigma$ and $\gamma\in \Gamma$, the operation $\sigma \ast_{\Gm^n}$ commutes 
    with $\phi(\gamma)$, so $\ast$ is an action of $\Gamma$ on $G$ that commutes with the multiplication on $G$. So now we have an action of $\Gamma$ on $G$, and we can check that $$\Gamma\longrightarrow G,\quad \gamma\longmapsto a_\gamma = (1,\gamma)$$
    satisfies the cocycle condition. In particular, it induces a $G$-torsor over $\spec(k)$, which we denote by $\cF\to \spec(k)$. This induces a map $\spec(k)\to \cB G$.

    The $\Gamma$-torsor associated to $\cF$ is obtained as follows. By construction, $\cF_{k'}$ is the trivial $G$-torsor, so if we fix an isomorphism between $\cF_{k'}\to G_{k'}$ we can identify the source with the target. The trivial torsor $G_{k'}$ has a (left) action of $G_{k'}$, so one can take its quotient by $\Gm^n\subseteq G$, to get $\Gm^n\backslash G_{k'}$. From \Cref{remark from cocycle to t functions}, the cocycle for $\cF$ (namely, the cocycle
    $a_\gamma$) descends for a cocycle for $\Gm^n\backslash G_{k'}$ (namely, the transition functions commute with taking the $\Gm^n$-quotient). This is clear from \Cref{remark from cocycle to t functions} as $G_{k'}$ is associative: the elements $a_\gamma^{-1}$ are multiplied on the right, whereas the action is a left action.
    
    So $a_\gamma$ descend and give transition functions on $\Gm^n\backslash G_{k'}$. If we fix the isomorphism $\Gamma \to \Gm^n\backslash G_{k'}$, $\gamma \mapsto [(1,\gamma)]$ so that we can identify the quotient $\Gm^n\backslash G_{k'}$ with $\Gamma$, one can see that the cocycle $a_\gamma$ is the same as the one in \Cref{example cocycle that gives the galois extension}. In particular, $\Gm^n\backslash\cF$ is isomorphic to the $\Gamma$-torsor $\spec(k')\to \spec(k)$. So from the universal property of the fiber product, $\pi$ has a section $\xi:\spec(k)\to\cG$.

    \underline{$\underline{\Aut}_k(\xi)\cong T_\phi$:} from \Cref{lemma automorphisms of a torsor} we have
    \[\Aut_k(\cF) = \{a = (\lambda,\mu)\in (G)_{k'}\text{ such that }\gamma * a = g_{\gamma}^{-1}\cdot a\cdot g_\gamma\},\]
    and from the universal property of the fiber product the automorphisms of $\xi$ are the automorphisms of $\cF$ of the form $(\lambda, 1)$. Since conjugating with $g_{\gamma^{-1}}$ acts as $\phi(\gamma^{-1}) = \phi^{-1}(\gamma)$, we deduce 
\begin{align*}
    \Aut_k(\cF)
    =&\{(\lambda,1)\in (G)_{k'}\text{ such that }\gamma * a = g_{\gamma}^{-1}\cdot a\cdot g_\gamma\} \\
    =&\{\lambda \in (\Gm)_{k'} \text{ such that }\gamma * \lambda = \phi^{-1}(\gamma)(\lambda)\} \\
    =&\{\lambda \in (\Gm)_{k'} \text{ such that }\phi(\gamma)(\gamma * \lambda) = \lambda\} \\
    =&T_\phi(\spec(k))
\end{align*}
where for the last bijection we used \Cref{lemma points of nonsplit torus using descent}.
\end{proof}

\section{When $\cB G$ is not effective}\label{section when BG not eff}

In this section we investigate when the classifying spaces of a group $G$ is not effective. We begin with the following criteria, which will be our main tool to prove that $\cB G$ is not effective.
\begin{theorem}\label{teo criteria for BG to be effective with double quotient}
    Let $G$ be a group,let $b:\spec(R')\to \spec(R)$ be an \'etale morphism of DVRs, let $\alpha:\spec(K(R'))\to \spec(R')$ and $a: \spec(K(R))\to \spec(R)$ be the inclusions of the generic points, and let $\beta:\spec(K(R))\to \spec(K(R'))$ be the map between the generic points. Let $\cG_1\to \spec(K(R))$ and $\cG_2\to \spec(R')$ be two $G$-torsors, which are isomorphic when pulled back to $\spec(K(R'))$. If $\cB G$ is effective, then the double quotient 
    $$\Aut(\cG_1)\backslash \operatorname{Isom}(\beta^*\cG_1,\alpha^*\cG_2)/\Aut(\cG_2)$$
    is a single element.
\end{theorem}
\begin{proof}
    Let $P:=P_{R,R'}$. A $G$-torsor over $P$ consists of two $G$-torsors over $\spec(K(R))$ and $\spec(R')$, and an isomorphism between their restrictions to $\spec(K(R'))$.
We will argue by contradiction: let $\psi$ and $\widetilde{\psi}$ be two elements in $$\operatorname{Isom}(\cG_1|_{ K(R')},\cG_2|_{K(R')})$$ which are not in the same equivalence class. Let $\cG_P$ be the torsor
associated to $\cG_1\to \spec(K(R))$, $\cG_2\to \spec(R')$ and $\psi$; and $\widetilde{\cG}_P$ the one associated to $\cG_1\to \spec(K(R))$, $\cG_2\to \spec(R')$ and $\widetilde{\psi}$.

First, observe that $\cG_P$ and $\widetilde{\cG}_P$ are not isomorphic. Indeed, an isomorphism between $\cG_P$ and $\widetilde{\cG}_P$ consists of two elements $\sigma\in \Aut(\cG_1)$ an $\tau \in \Aut(\cG_2)$ such that $\psi = \sigma\circ \widetilde{\psi}\circ \tau$. Since $\psi$ and $\widetilde{\psi}$ are not in the same equivalence class, such an isomorphism cannot exist. 

Then notice that $\cG_P$ and $\widetilde{\cG}_P$ restrict to the same $G$-torsor over $\spec(K(R))$. In particular the morphism $\oH^1(P,G)\to \oH^1(\spec(K(R)),G)$ is \textit{not} injective. So the map $\oH^1(\spec(R),G)\to \oH^1(P,G)$ cannot be surjective as the composition $\oH^1(\spec(R),G)\to \oH^1(P,G)\to \oH^1(\spec(K(R)),G)$ is injective, as the Grothendieck–Serre conjecture is known over DVRs \cites{nis82, nis84, guo20, vcesnavivcius2022problems} (in particular $\cG_P$ and $\widetilde{\cG}_P$ cannot be both pullbacks of $G$-torsors over $\spec(R)$). This shows that $\cB G\to\spec(k)$ cannot be effective.
\end{proof}

\begin{Prop}\label{prop sl2 not effective}
The morphism $\cB \SL_2\to \spec(k)$ is not effective.
\end{Prop}
\begin{proof}
Let $R\to R'$ be an unramified extension of DVRs, and pick an element $u\in R'$ such that its image in $k_{R'}$ does not belong to $k_R$. Since all the $\SL_2$-torsors on a local ring are trivial (as $\SL_2$ is special), from \Cref{teo criteria for BG to be effective with double quotient} to prove the proposition it suffices to prove that the double quotient
$$\SL_{2,R'}\backslash \SL_{2,K(R')}/\SL_{2,K(R)}$$
is not trivial, where $\SL_{2,R'}$ acts by multiplication on the left and $\SL_{2,K(R)}$ by multiplication on the right. In particular, if $\pi$ is an uniformizer for $R$, it suffices to check that the following matrix is not in the orbit of the identity:
$$M :=\begin{bmatrix}\pi & 0\\u & \pi^{-1}\end{bmatrix}.$$
If it was in the orbit of the identity, we could find a matrix $$B:= \begin{bmatrix}
    a & b \\ c & d
\end{bmatrix}\in\SL_{2,K(R)} \text{ such that }MB =\begin{bmatrix}\pi a & \pi b\\ ua + \pi^{-1}c& ub + \pi^{-1}d\end{bmatrix} \in\SL_{2,R'}.$$

Since $B$ is invertible, not both $a$ and $b$ can be 0. So assume $a\neq 0$, the case $b\neq 0$ is similar.
If we denote by $v$ the valuation on $K(R')$, we have $v(a)\ge -1$ since $\pi a\in R'$. Then since $ua + \pi^{-1}c\in R'$, we have $\pi(ua + \pi^{-1}c)\in R'$ so either $c=0$ or $v(c)\ge 0$, so either way $c\in R'$. Therefore since $ua + \pi^{-1}c\in R'$, we must have $v(a) = -1$ and $v(b) = -1$. So we can write $a = \pi^{-1}a_1$ with $a_1$ unit in $R$, and $\pi^{-1}(ua_1 + c)\in R$. Therefore 
$$v(ua_1 + c)> 0.$$
If we consider $ua_1 + c$ in $k_{R'}:=R'/(\pi)$, we have that the image of $u$ via $R'\to R'/(\pi)$ belongs to $k_R$ (it is equal to $-\frac{c}{a_1}$ modulo $\pi$), which is a contradiction.
\end{proof}
\begin{Prop}\label{prop nonsplit tori not eff} Let $L\subseteq L'$ be a Galois extension of $L$ with Galois group $\Gamma$, let $\phi:\Gamma\to \GL_n(\bZ)$ be a homomorphism, and let $T_\phi\to \spec(L)$ be its associated non-split torus as in \Cref{notation twisted torus}. Then $\cB T_\phi$ is not effective.
\end{Prop}
\begin{proof} 
First, consider the fields $L(t)$ and $L'(t)$. The extension $L(t)\subseteq L'(t)$ is still Galois with Galois group $\Gamma$, and $\Gamma$ acts trivially on $t$. In particular, it induces an \'etale extension of DVRs $\spec(L'[t]_{(t)})\to  \spec(L[t]_{(t)})$, we denote by $R$ (resp. $R'$) the ring $L[t]_{(t)}$ (resp. $L'[t]_{(t)}$). 

We plan to apply \Cref{teo criteria for BG to be effective with double quotient}, using \Cref{lemma points of nonsplit torus using descent} to show that not all the elements of the form $(1,...,1,t,1,...,1)$ are in the same equivalence class of the identity in 
$$(\Gm^n)_{R'}\backslash (\Gm^n)_{L'(t)}/\Aut(T_\phi(L(t)).$$

Set $\cG_1:=T_{\phi,R}\to \spec(R)$, $\cG_2:={\Gm^n}_{,R'}\to \spec(R')$ and $f:(\cG_1)|_{L'(t)}\overset{\simeq}{\to} (\Gm)_{L'(t)}^n$ an isomorphism. We will use $f$ to identify $$(\Gm^n)_{L'(t)}\simeq \Isom_{\Gm^n\text{-torsors}}((\Gm^n)_{L'(t)},(\Gm^n)_{L'(t)}) \overset{\simeq}{\longrightarrow}\operatorname{Isom}((\cG_1)|_{L'(t)},(\cG_2)|_{L'(t)}).$$
There is a valuation $v:(\Gm)_{L'(t)}\to \mathbb{Z}$, which sends $p(t)$ to its valuation at $0\in \bA^1_{L'}$. By definition, the valuation is a group homomorphism from the multiplicative group $(\Gm)_{L'(t)}$ to the additive group $\bZ$. In this way we get a homomorphism
$$v^n:(\Gm^n)_{L'(t)}\longrightarrow \mathbb{Z}^n.$$
Moreover, for every $n\times n$ matrix $M$ with integer coefficients, one has a homomorphism $F_M:(\Gm^n)_{L'(t)}\to (\Gm^n)_{L'(t)}$ defined as
\[(a_{i,j})\cdot (\lambda_1,...,\lambda_n)=((\lambda_1^{a_{1,1}}\lambda_2^{a_{1,2}}\cdots\lambda_n^{a_{1,n}}),\ldots,(\lambda_1^{a_{n,1}}\lambda_2^{a_{n,2}}\cdots \lambda_n^{a_{n,n}})).\]
Similarly, every matrix $M = (a_{i,j})$ gives a homomorphism $G_M:\bZ^n\to \bZ^n$ by matrix multiplication. It is straightforward to check that $v^n\circ F_M=G_M\circ v^n$. In particular, every element $\phi(\gamma)$ induces a homomorphism $G_{\phi(\gamma)}$, and $\phi(\gamma) = \Id$ if and only if $G_{\phi(\gamma)} = \Id$. 
Moreover, as $\Gamma$ acts trivially on $t$, for every $\gamma\in \Gamma$ one has $$v^n(\gamma \ast \lambda)= v^n(\lambda)$$ where we denoted by $\ast$ the action on $(\Gm^n)_{L'(t)}$ induced by the Galois action of $\Gamma$ on $L'$. In particular, using \Cref{lemma points of nonsplit torus using descent}, the valuation $v^n$ sends the $L$-points of $(T_\phi)_{L'(t)}$ to the vectors in $\bZ^n$ that commute with $G_{\phi(\gamma)}$ for every $\gamma \in \Gamma$. Similarly, it sends all the elements in $(\Gm^n)_{R'}$ to 0. Thus, since
\begin{enumerate}
    \item we can identify the automorphisms of the trivial $T_\phi$-torsor with $T_\phi$ as in \Cref{example aut trivial G torsor},
    \item we have the bijection of \Cref{lemma points of nonsplit torus using descent}, and
    \item $v^n(\gamma \ast g)= v^n(g)$,
\end{enumerate}
we have a surjective map
$$(\Gm^n)_{\spec(R')}\backslash (\Gm^n)_{\spec(L'(t))}/\Aut(T_\phi(\spec(L(t))))\to \bZ^n/\{w\in \bZ^n \text{ such that }G_{\phi(\gamma)}(w) = w \text{ for every }\gamma \in \Gamma\}.$$
As there is a $\gamma$ such that $\phi(\gamma)$ is not trivial, there is a $\gamma$ such that $G_{\phi(\gamma)}$ is not trivial. In particular, it cannot fix all the elements in the standard basis of $\bZ^n$: this implies that the right hand side is not a single element, hence the left hand side cannot be a single element.
\end{proof}
\begin{Cor}\label{cor semidirect product is not effective unless phi is trivial}
    Assume that $F$ is a finite group, and let $\phi:F\to \GL_n(\bZ)$ be a homomorphism. Then  $\cB \Gm^n\rtimes_\phi F$ is effective if and only if $\phi$ is trivial.
\end{Cor}
\begin{proof} If $\phi$ is trivial, then $\cB (\Gm^n\times F)$ is effective from \Cref{prop:BG eff}, so let's assume that $\phi$ is not trivial, let $G:=\Gm^n\rtimes_\phi F$ and consider $L\subset L'$ a field extension with Galois group $F$ (which exists, for example, if $L=\spec(k(t))$). From \Cref{theorem from semidirect product to nonsplit torus} and \Cref{prop nonsplit tori not eff}, the map $\cB G\times_{\cB F}\spec(L)\to \spec(L)$ is not effective. Then from \Cref{lm:composition} the stack $\cB G$ cannot be effective.
\end{proof}
We are ready to prove the main result of this section.
\begin{Teo}\label{Teo BG effective implies G is a central ext of a torus}
    Assume that $G$ is a group such that $\cB G$ is effective. Then the connected component of the identity of $G$ is a split torus contained in the center of $G$.
\end{Teo}
\begin{proof}First we show that the connected component of the identity of $G$, which we denote by $H$, is a torus. Observe that the map $\cB H\to \cB G$ is separated, hence effective from \Cref{prop:sep to eff}. Therefore if $\cB G$ was effective, also $\cB H$ would be effective.

Every reductive connected group which is not a torus admits a homomorphism $\SL_2\to H$ with finite kernel. Indeed, up to a finite base change, we can assume that $H$ is split reductive: by picking a root $\alpha$ of $H$, we can consider the associated split reductive subgroup $H_\alpha\subset H$ of rank one \cite{Mil}*{Theorem 21.11}; there is then a homomorphism $SL_2\to H_\alpha$ which is an isogeny onto the derived subgroup of $H_\alpha$ \cite{Mil}*{Proposition 20.32}, and the composition $\SL_2\to H_\alpha\to H$ has the claimed property. 

Hence we have a separated morphism $\cB\SL_2\to \cB H$. Therefore if $\cB H\to \spec(k)$ was effective, also $\cB\SL_2\to \spec(k)$ would be effective. \Cref{prop sl2 not effective} gives the desired contradiction.

So $H$ is a torus, and from \Cref{prop nonsplit tori not eff} it has to be split. From \cite{brion2015extensions}, if we denote by $F:=G/H$, we have a finite and surjective morphism $f:H\rtimes_\phi F\to G$. So $\cB(H\rtimes_\phi F)\to \cB G$ is effective, thus $\cB(H\rtimes_\phi F)$ is effective. It follows then that $\phi$ is trivial from \Cref{cor semidirect product is not effective unless phi is trivial}, and $H$ is contained in the center of $H\rtimes F$; as $f$ is surjective $f(H)$ is contained in the center of $G$ as desired.
\end{proof}
\begin{Cor}\label{cor:stab are extensions of tori}
    Assume that $\cX$ is an algebraic stack over $\spec(k)$ with a separated good moduli space $X$, and assume that $\cX\to X$ is effective. Then the stabilizers of the closed points are central extensions of split tori by finite groups.
\end{Cor}
\begin{proof}
This follows as the inclusion of the residual gerbe is a closed embedding (hence effective), composition of effective morphisms is effective, and \Cref{Teo BG effective implies G is a central ext of a torus}.
\end{proof}

\section{From effective to the quotient presentation}\label{section from eff to quotient}
In this section we prove the other direction of \Cref{teo intro}, namely that if a good moduli space morphism $\cX\to X$ is effective, then $X$ is a quotient of a DM stack by a torus (see \Cref{teo effettivo implica ho il gen set}). After introducing a general definition that will be useful later, we will proceed with the main result in two steps. First, in \Cref{ssection gms of dim 0} we will prove the desired result in the case $X=\spec(L)$ is the spectrum of a field. Then, in \Cref{ss extending gset} we prove the general result by spreading out line bundles and using \Cref{prop passo induttivo generating set}.  
\begin{Def}\label{def:generating line bundles}
    We say that an algebraic stack $\cX$ admits $n$ \textit{generating line bundles} if there are $n$ line bundles on $\cX$ such that the corresponding map $\cX\to \cB \Gm^n$ is representable in DM stacks.
\end{Def}
The following lemma motivates the name above.
\begin{Lemma}\label{lm:generating generate}
    Let $G$ be a group and assume that $\cB G$ has a generating set of line bundles. Then any such set form a generating set for the rational Picard group $\Pic(\cB G)_\bQ$; viceversa any set of line bundles whose classes generate $\Pic(\cB G)_{\bQ}$ also form a generating set on $\cB G$ in the sense of \Cref{def:generating line bundles}.
\end{Lemma}
\begin{proof}
    Let $\cL_1,\ldots,\cL_n$ be a generating set in the sense of \Cref{def:generating line bundles}. Identifying $\Pic(\cB G)_\bQ$ with the vector space of rational characters of $G$, let $\chi_1,\ldots,\chi_n$ be the associated characters, and let $\phi:G\to\Gm^n$ be the associated homomorphism of groups. Observe that $\phi$ induces precisely the DM morphism $\cB G\to\cB \Gm^n$, and that the torsor associated to $\cL_1\oplus\cdots\oplus \cL_n$ is $[(\bA^1\smallsetminus\{0\})^n/G]$, where the action of $G$ is the defined by $\phi$; the fact that this quotient stack is DM is equivalent to saying that the kernel of $\phi$ is finite: let us denote it $F$, and say that it has order $e$.

    Given any character $\chi$ of $G$, we have that $\chi^e(F)=0$, hence the character $\chi:G\to\Gm$ factors as $G\overset{\phi}{\to}\Gm^n\to\Gm$. This implies that $\phi^*$ induces a surjective morphism between the spaces of rational characters, and that $\chi_1,\ldots\chi_n$ generate $X(G)_\bQ.$ In terms of classifying stacks, this amounts to saying that $\cL_1,\ldots,\cL_n$ generate $\Pic(\cB G)_\bQ$.

    Now let $\cF_1,\ldots,\cF_m$ be generators for $\Pic(\cB G)_\bQ$, and let $\eta_1,\ldots\eta_n$ be the associated characters, so that there exist coefficients $a_{ij}\in \bQ$ with $\chi_i=\sum a_{ij}\eta_j$. Define $\psi:G\to\Gm^m$ as $g\mapsto (\eta_1(g),\ldots,\eta_m(g))$. There exists an integer $d$ such that the matrix $dA=(da_{ij})$ has integer coefficients and it induces a homomorphism $\ell_A:\Gm^m\to\Gm^n$ having the property that the composition $\ell_A\circ \psi$ is equal to $\phi^d$. This implies, as the kernel of $\phi$ is finite, that also the kernel of $\psi$ must be finite, hence the torsor associated to $\cF_1\oplus\cdots\oplus\cF_m$ is a DM stack, i.e. these line bundles form a generating set in the sense of \Cref{def:generating line bundles}. 
\end{proof}
\subsection{Base step: effective good moduli spaces of dimension zero}\label{ssection gms of dim 0}
In this subsection we prove one direction of \Cref{teo intro}, in the case where the good moduli space of $\cX$ is a point.
\begin{Lemma}\label{lemma la conj holds for gerbes}
    Assume that $f:\cX\to \spec(L)$ is a gerbe, and it is effective. Then there are $n$ generating line bundles on $\cX$.
\end{Lemma}
Part of the following argument was suggested by Siddarth Mathur and Minseon Shin to the second author.
\begin{proof}
    We begin by rigidifying $f$ (see \cite[Section 6.2.8]{Alp21}) Define $\cX'$ as the stack whose objects are the same objects of $\cX$ and
    \[ \Aut_{\cX'}(\xi) := \Aut_{\cX}(\xi)/\Aut_{\cX}(\xi)^0. \]
    Then we have a factorization $\cX\xrightarrow{h} \cX'\xrightarrow{g} \spec(L)$ where the diagonal relative to $h$ is represented by torsors under connected group schemes, and $\cX'$ is DM over $\spec(L)$. As $f$ is effective, by \Cref{cor composition effective implies the first one effective} the morphism $h$ is also effective.
    
    This implies by \Cref{cor:stab are extensions of tori} that the automorphism groups of points
    are split tori. In particular, they are abelian, so from \cite[Exercise 6.24.4]{Alp21} the morphism $h$ is a \textit{banded} gerbe.
%\textbf{Esercizio Jarod (lo metto qui che magari torna utile, poi lo si toglie)}
    %Let $f:\cX\to S$ be a gerbe with abelian stabilizers. By definition of gerbes, there exists a cover $S'\to S$ with $\cX\times_{S}S'\simeq \cB G'$ for some group $G'$ over $S'$. If we show that $G'$ descends along $S'\to S$, we are done. 
    %By construction, we have gluing data $\cB G'_{S''} \simeq \cB G'_{S''}$ over $S''=S'\times_S S'$: this induces an isomorphism of the universal torsor over $\cB G'_{S''}$, and consequently an isomorphism $\psi:G'\simeq G'$ of the trivial $G'_{S''}$-torsor which satisfies the cocycle condition up to automorphisms of $G'$-torsors , i.e. up to the inner action of $G'_{S''}$ on itself. As $G'$ is abelian, the inner action is trivial, thus $\psi$ satisfies the cocycle condition and therefore $G'$ descends to an abelian group over $S$.\textbf{Fine esercizio}
    Let $\cG\to \spec(L)$ be a band. Observe that we can choose $\cG$ to be a split torus: indeed, by definition of banded gerbe, there exists an \'etale cover $\spec(L')\to\spec(L)$ such that $\cX'_{L'}\simeq \cB \cG_{L'}$; as the base change of effective morphisms is effective (\Cref{lm:composition}), the map $\cB \cG_{L'}\to \spec(L')$ is effective, hence by \Cref{Teo BG effective implies G is a central ext of a torus} together with the fact that $\cG_{L'}$ is connected, we deduce that $\cG_L'\simeq {\Gm}_{,L'}^n$, so we can choose $\cG=\Gm^n$.
    
    In particular, the gerbe $\cX\to\cX'$ corresponds to an element in $\oH^2(\cX',\bG_m^n) = \bigoplus_{i=0}^n \oH^2(\cX',\bG_m)$, and the latter group is torsion because $\cX'$ is a smooth DM stack: indeed, the stack $\cX'$ being DM implies from \cite[Theorem 4.5.1]{Alp21} that there exists a finite flat cover $\phi:U\to \cX'$ of some degree $d$ where $U$ is a regular scheme, and the composition
    $$\oH^2(\cX',\bG_m)\xrightarrow{\phi^*} \oH^2(U,\bG_m)\xrightarrow{\phi_*}\oH^2(\cX',\bG_m)$$
    is the multiplication by $d$. As $\oH^2(U,\bG_m)$ is torsion, also $\oH^2(\cX',\bG_m)$ is torsion.
    
    In particular, the class of $\cX$ in $\oH^2(\cX',\Gm^n)$ is of $m$-torsion for some integer $m$, i.e. there exists $n$ line bundles on $\cX$ which are $m$-twisted, for $m$ divisible enough \cite{lieblich2008twisted}. This implies that the corresponding morphism $\cX\to \cX'\times \cB \Gm^n$ is relatively DM. As $\cX'\to \spec(L)$ is DM, the composition $\cX\to \cX'\times \cB \Gm^n\to \cB \Gm^n$ is relatively DM.
\end{proof}
\begin{Lemma}\label{lemma extend line bundles quotient}
    Let $\cX\to X$ be a good moduli space. Assume that $X=\spec(B)$ where $B$ is a semi-local ring and that the residual gerbe at the closed points of $\cX$ is $i:\bigsqcup\cB G\to \cX$, for a group $G$.  Then:
    \begin{enumerate}
        \item the restriction map $i^*\colon \operatorname{Pic}(\cX)\to \bigoplus\operatorname{Pic}(\cB G)$ is injective,
        \item If $B$ is local and $\cX\cong [\spec(A)/G]$, the restriction map $i^*\colon \operatorname{Pic}(\cX)\to \operatorname{Pic}(\cB G)$ is an isomorphism, and
        \item if $B = \spec(L)$ is the spectrum of a field and $\cL_1$, $\cL_2$ are two line bundles on $\cX$ whose restrictions to $\cB G$ are isomorphic, then $\Hom_\cX(\cL_1,\cL_2)\cong \Hom_{\cB G}(\cL_1|_{\cB G},\cL_2|_{\cB G})$.
    \end{enumerate}
\end{Lemma}
\begin{proof}
(1) follows from \cite[Theorem 10.3]{alp}. Indeed, if $i^*\cL$ is trivial, then the action of $G$ on the fiber of $\cL$ at the closed point is trivial, so from \textit{loc. cit.} $\cL$ is pulled back from a line bundle on $\spec(B)$. As the latter is semi-local, the only line bundle is the trivial line bundle.

For (2), surjectivity of $i^*$ follows since the map $\cB G\to [\spec(A)/G]$ has a section given by the torsor $\spec(A)\to [\spec(A)/G]$, so we have a diagram as follows whose composition is an isomorphism:
$$\cB G\to [\spec(A)/G]\to \cB G. $$
In particular $i^*$ is surjective as it has a right inverse.

For (3) observe that $$\Hom_\cX(\cL_1,\cL_2) \cong \operatorname{H}^0(\cX,\cL_1^{-1}\otimes\cL_2) = \oH^0(\spec(L),\pi_*(\cL_1^{-1}\otimes\cL_2)) $$
    where $\pi:\cX\to \spec(A^G)$ is the good moduli space map. As $\cL_1$ and $\cL_2$ are isomorphic once restricted to $\cB G$, the character of $(\cL_1^{-1}\otimes\cL_2)$ is 0, so from \cite[Theorem 10.3]{alp} the push-forward $\pi_*(\cL_1^{-1}\otimes\cL_2)$ is a line bundle on $\spec(L)$ (namely, the trivial line bundle);
    so $\oH^0(\spec(L),\pi_*(\cL_1^{-1}\otimes\cL_2)) \cong L$. We have a similar identification by replacing $\cX$ with $\cB G$ and $\cL_i$ with $\cL_i|_{\cB G}$, which gives the desired isomorphism.
\end{proof}
\begin{Lemma}\label{remark local pic determided by res gerbe}
    With the assumptions of \Cref{lemma extend line bundles quotient} and $B$ local, if $\cX$ admits a generating set of line bundles, then the morphism $\Pic(\cX)_{\bQ}\to\Pic(\cB G)_{\bQ}$ is an isomorphism. If $\cB G$ is effective, then the vice versa hold.
\end{Lemma}
\begin{proof}
In one direction, we only have to prove surjectivity. Given a DM morphism $\cX\to\cB\Gm^n$, the composition $\cB G\to \cX \to \cB \Gm^n$ is also DM. In particular, there is a $\Gm^n$-torsor $\cP\to \cB G$ which is a DM stack. As shown in the proof of \Cref{lm:generating generate}, this datum is equivalent to the datum of a homomorphism of groups $\varphi:G\to\Gm^n$ having finite kernel $F$ and such that $\varphi^*:X(\Gm^n)_{\bQ}\to X(G/F)_{\bQ}$ is surjective, hence $\Pic(\cB \Gm^n)_{\bQ}\to \Pic(\cB G)_{\bQ}$ is surjective. As the latter map factors through $\Pic(\cX)_{\bQ}$, we deduce that $\Pic(\cX)_{\bQ}\to\Pic(\cB G)_{\bQ}$ is surjective as well. 

In the other direction, suppose that $\Pic(\cX)_{\bQ}\to\Pic(\cB G)_{\bQ}$ is surjective: then if there are generating line bundles on $\cB G$, they extend (up to taking high enough tensor powers) to line bundles over $\cX$, and the associated extended torsors will still be DM by upper-semicontinuity of the dimension of the automorphism groups of points. But we know that, thanks to \Cref{lemma la conj holds for gerbes}, for $\cB G$ effective the stack $\cB G$ has generating line bundles.
\end{proof}

\begin{Prop}\label{prop la conj holds for 0 dim gms}
    Assume that $\pi:\cX \to \spec(L)$ is an effective good moduli space map, and $L$ is a field. Then $\cX$ admits $n$ generating line bundles.
\end{Prop}
\begin{proof}
First consider the residual gerbe $j:\cG\to \cX$. Recall that $j$ is a closed embedding. Indeed, it is locally closed from \cite[Proposition 3.5.16]{Alp21}, and by passing to a local quotient presentation \cite{luna}, one can check that it corresponds to the inclusion of the \textit{closed} orbit.
So the composition $\pi\circ j$ is effective.
In particular, from \Cref{lemma la conj holds for gerbes}, there are generating line bundles $\cL_1,\ldots,\cL_n$ on the residual gerbe. We want to prove that they extend to line bundles $\cL_1,...,\cL_n$ on $\cX$.

From \cite{luna}, there is an extension $\spec(L')\to \spec(L)$ and an isomorphism $[\spec(A)/G]\cong \cX\times_{\spec(L)}\spec(L')$. If we denote by $\cG' :=\cG\times_{\spec(L')}\spec(L)$ and by $\pi: \cG' \to \cG $ the first projection, then $\pi^*\cL_i$ extends to a line bundle on $[\spec(A)/G]$ from \Cref{lemma extend line bundles quotient} points (1) and (2). We denote these extensions by $\widetilde{\cL_1}, ...,\widetilde{\cL_n}$. Now we use \Cref{lemma extend line bundles quotient} part (3) to show that $\widetilde{\cL_i}$ descends. Indeed, by construction they descend when restricted to $\cG'$, and the descend data extends to $[\spec(A)/G]$ from \Cref{lemma extend line bundles quotient} part (3). The corresponding map $\cX\to \cB\Gm^n$ is representable in DM stacks because it is so once restricted to the closed point of $\cX$.
\end{proof}


\subsection{Inductive step: extending a generating set}\label{ss extending gset}In this subsection we finally complete the proof of \Cref{teo intro}.

\begin{Lemma}\label{lemma extending gen set}
    Consider a smooth algebraic stack $\cX:=[\spec(A)/G]$ with good moduli space $\cX\to X=\spec(A^G)$, where $G$ is effective and $A^G$ is local. Let $U$ be an open subscheme of $\spec(A^G)$. Assume that $\cU:= [\spec(A)/G]\times_{\spec(A^G)}U$ admits $n$ generating line bundles $\{\widetilde{\cL}_i\}_{i=1}^n$. Then there are $n+m$ generating line bundles $\{\cL_i\}$ on $\cX$, such that $(\cL_i)|_{\cU}\cong \widetilde{\cL}_i$ if $i\le n$, and $(\cL_i)|_{\cU}\cong \cO_{\cU}$ otherwise.
\end{Lemma}
\begin{proof}
    As $\cX$ is smooth, we have a surjection $\Pic(\cX)_{\bQ}\to \Pic(\cU)_{\bQ}$. In particular, we can extend the generating line bundles $\widetilde{\cL}_1,\ldots,\widetilde{\cL}_n$ on $\cU$ to line bundles $\cL_1,\ldots,\cL_n$ on $\cX$.

    By \Cref{lemma extend line bundles quotient} we have an isomorphism $\Pic(\cX)\overset{\cong}{\to}\Pic(\cB G)$: in particular, as the $\bQ$-vector space $\Pic(\cB G)_{\bQ}$ is finitely generated, we must have $\Pic(\cX)_{\bQ}\simeq \bQ^{n+m}$ for some value of $m$. Complete $\cL_1,\ldots,\cL_n$ to a basis $\cL_1,\ldots,\cL_{n+m}$ of $\Pic(\cX)_{\bQ}$: then by construction we have that $\cL_i|_{\cU}$ is trivial for $i>n$.
    
    Observe moreover that $\cB G$ has a generating set of line bundles because of \Cref{lemma la conj holds for gerbes}. Then, as the restriction of $\cL_1,\ldots,\cL_{n+m}$ generate the rational Picard group of $\cB G$, by \Cref{lm:generating generate} they also form a generating set for $\cB G$ in the sense of \Cref{def:generating line bundles}, i.e. the torsor associated to the restriction of $\cL_1\oplus\cdots\oplus\cL_{n+m}$ is DM. By upper-semicontinuity of the dimension of dimension of stabilizer groups for $G$-actions, the torsor associated to $\cL_1\oplus\cdots\oplus\cL_{n+m}$ must be DM too.
    \end{proof}
\begin{Prop}\label{prop passo induttivo generating set}
    Assume that $\cX\to \spec(R)$ good moduli space with $R$ local ring and $\cX$ smooth. Let $p$ be the closed point of $\spec(R)$, and if we denote by $U=\spec(R)\smallsetminus\{p\}$, we assume that $\cU:=U\times_{\spec(R)}\cX$ admits a generating set. Then also $\cX$ admits a generating set.
\end{Prop}
Before proceeding with the proof, we recall the following fact, which is a slight improvement of Zariski's main theorem.
\begin{Teo}\label{teo zmt enhanced}
    Let $X$ be a separated DM stack and $Y\to X$ an \'etale, quasi-finite surjective morphism from a Noetherian scheme $Y$. Then there is a finite morphism $F\to X$ such that the second projection $Y\times_X F\to F$, when restricted to the connected components of $Y\times_X F$, is an open embedding.
\end{Teo}
The proof is essentially the proof of \cite[Theorem 4.5.1]{Alp21} (see in particular the second diagram of page 178). Note that among the hypotheses in \textit{loc. cit} it is required the stack $X$ being of finite type, but the whole proof works as well with the hypotheses above.
\begin{proof}[Proof of \Cref{prop passo induttivo generating set}] First observe that the stabilizer of the geometric closed point of $\cX$ is an effective group. So there is a finite extension $L$ of the residue field of $R$, such that the residual gerbe of $\cX\times_{\spec(R)}\spec(L)$ is $\cB G$, where $G$ is an effective group. From the main result of \cite{luna}, there is a cartesian diagram as follows, with \'etale surjective horizontal arrows, where $A^G$ has a unique closed point:
$$\xymatrix{[\spec(A)/G]\ar[d]\ar[r] & \cX\ar[d] \\ \spec(A^G) \ar[r]^j & \spec(R).}$$
We can now apply \Cref{teo zmt enhanced} to $j$ to get the following diagram, where every square is cartesian
$$\xymatrix{\cF' \ar[rr] \ar[dd]\ar[dr]^{i_F} & & [\spec(A)/G]\ar[dr]^i\ar[dd] &\\
 & \cF\ar[dd]\ar[rr] & &\cX\ar[dd]_\pi \\
 \spec(R_F\otimes_R A^G)\ar[dr]^-{j_F}\ar[rr] & & \spec(A^G)\ar[dr]^-j & \\
 & \spec(R_F) \ar[rr] & & \spec(R).}$$
 We denote by $\pi_F$ the morhphism $\cF\to \spec(R_F)$; and up to replacing $R_F$ with the normalization of one of the connected components of $\spec(R_F)$ which dominate $\spec(R)$, we can assume that $R_F$ is normal. Moreover, up to further extending, we can assume that $\spec(R_F)\to\spec(R)$ generically is Galois with Galois group $\Gamma$. In particular, $\Gamma$ acts on $\spec(R_F)$, $\spec(R_F\otimes_R A^G)$, $\cF$ and $\cF'$ in a way such that all the arrows in the diagram above are equivariant. Finally, by assumption, $i_F$ and $j_F$ are open embeddings once restricted to each connected component of their domains. We first plan on descending a generating set on $\cF'$ to $\cF$, using \Cref{lemma extending gen set}.

 First, we restrict $\pi$ to $\cU$. Let $m$ be the number of line bundles in the generating set of $\cU$. We can pull back this set to $\cU_A:= \cU\times_{\cX}[\spec(A)/G]$, which is by construction a quotient stack by $G$, and extend it to a generating set for $[\spec(A)/G]$ using \Cref{lemma extending gen set}; now we have $n+m$ line bundles on $[\spec(A)/G]$ which are a generating set. We can pull back this generating set to $\cF'$, and now  so we have $n+m$ line bundles $\cL_1,...,\cL_{n+m}$ on $\cF'$. 

  We now study the action of $\Gamma$ on $\cL_i$. Consider the inclusion of the closed residual gerbe
  $\cB G_L\to [\spec(A)/G]$, where $L$ is a field, and its pull-back $\bigsqcup_{\ell = 1}^f \cB G'_{L'} = \cB G_L\times_{[\spec(A)/G]}\cF'$.
 There is an action of $\Gamma$ on $\bigsqcup_{\ell = 1}^f \cB G'_{L'}$, which acts transitively on its topological space. From \Cref{lemma extend line bundles quotient}, there are injective morphisms $\Pic(\cF')\to \bigoplus_{\ell = 1}^f  \Pic(\cB G'_{L'})$ so one can understand
 the action of $\Gamma$ on $\cL_i$ by understanding its action on $\bigsqcup_{\ell = 1}^f \cB G'_{L'}$. In particular, $\Gamma$ acts trivially on $\cL_i$, as they are pulled back from $[\spec(A)/G]$, namely $\bigotimes_{\gamma \in \Gamma}\gamma^*\cL_{i} = \cL_i^{\otimes|\Gamma|}$. Therefore $\{\cL_i\}$ is a $\Gamma$-invariant generating set:
 \begin{center}
  the characters induced by $\cL_i$ on each connected component of $\bigsqcup\cB G'_{L'}$  are the same.
 \end{center}

 The scheme $\spec(R_F)$ has an open subset $U_F=U\times_{\spec(R)}F$ and $d$ closed points $p_1,...,p_d$, and the action of $\Gamma$ is transitive on the closed points. For each of the points $p_i$, we can take an open subscheme $\spec(R_i)$ of a connected component of $\spec(R_F\otimes_R A^G)$ such that the composition $\spec(R_i)\to \spec(R_F)$ is the localization of $\spec(R_F)$ at $p_i$. So each $\cF_i:=\spec(R_i)\times_{\spec(R_F\otimes_R A^G)}\cF'$ has $n+m$ generating line bundles,
 and we plan on gluing those along the generic points of $\spec(R_i)$ for every $i$ to get $n+m$ generating line bundles on $\cF$. But the $n+m$ generating line bundles we choose glue along the generic fiber of $\cF_i\to \spec(R_i)$: the first $n$ of them came from line bundles on $\cX\times_{\spec(R)}\cU$, and the other $m$ restrict to the trivial line bundle generically. So $\cF$ admits a generating set, we denote it by $\cG_1,...,\cG_{n+m}$, where the first $m$ line bundles are the pull-back of a generating set of $\cX\times_{\spec(R)}\cU$.

 We now plan on descending a generating set from $\cF$ to $\cX$. Recall that we have an action of $\Gamma$ on $\spec(R_F)$, so we have the following diagram, where all squares are cartesian:
 $$\xymatrix{\cF \ar[d] \ar[r] &\cX' \ar[r] \ar[d] & \cX\ar[d]^\pi \\ \spec(R_F)\ar[r] & [\spec(R_F)/\Gamma]\ar[r] & \spec(R).}$$
 The data of a line bundle on $\cX'$ is the data of a $\Gamma$-equivariant line bundle on $\cF$. So for $i\le n$ we define $\cG_i':=\cG_i$ and for $1\ge i \ge n$ we define $\cG_{m+i}':=\bigotimes_{\gamma \in \Gamma}\gamma^*\cG_{m+i}$. Now the line bundles $\cG_i'$ admit an action by $\Gamma$. To check that $\{\cG_{m+i}'\}$ is still a generating set we pull them back to $\cF$. Since $\cL_i$ was a $\Gamma$-\textit{invariant} generating set, the pull-backs of $\cG_i'$ forms a generating set.
 
 So they descend to $n+m$ line bundles on $\cX'$, which we denote by $\cH_1,...,\cH_{n+m}$. Since $\cX$ is smooth, the scheme $\spec(R)$ is normal, so $[\spec(R_F)/\Gamma]\to \spec(R)$ is a coarse moduli space. So up to replacing $\cH_i$ with some powers, they descend to a generating set on $\cX$ as desired.\end{proof}

\begin{theorem}\label{teo effettivo implica ho il gen set}
Assume that $\cX$ is a smooth algebraic stack admitting a good moduli space $\pi:\cX\to X$. If $\pi$ is effective, then $\cX$ has a generating set, namely it admits a morphism $\cX\to \cB \Gm^n$ for $n$ big enough, which is representable in DM stacks.
\end{theorem}
\begin{proof}Consider the following set
$$\sU:=\{\cU \subseteq \cX:\cU\text{ is open and admists a generating set}\}.$$
It is ordered by inclusion, and every two elements $\cU_1,\cU_2\in \sU$ admit $\cU_3$ which contains both. Indeed, since $\cX$ is smooth, we can extend each line bundle in the generating sets of $\cU_1$ and $\cU_2$ to $\cU_1\cup \cU_2$. Consider now the direct sum of direct sum of the extensions of the line bundles coming from $\cU_1$ and $\cU_2$: the associated $\Gm^{n_1+n_2}$-torsor will be DM both when restricted to $U_1$ and when restricted to $U_2$, thus $\cU_1\cup \cU_2\in \sU$. 

 Observe also that $\sU$ is not empty, as from \Cref{prop la conj holds for 0 dim gms} we have a generating set over the generic fiber of $\pi$, which we can spread out.

Consider then $\cU$ a maximal element of $\sU$, and assume by contradiction that $\cU\neq\cX$. Let $p\in \cX\smallsetminus \cU$ be a generic point of an irreducible component of $ \cX\smallsetminus \cU$. From \Cref{prop passo induttivo generating set} the stack $\cX\times_{X}\spec(R)$ admits a generating set, which we can spread out to a generating set of $\cU'$, an open subset of $\cX$ \textit{containing} $p$. Then $\cU\cup \cU'$ admits a generating set, which contradicts the maximality of $\cU$. So $\cU=\cX$ as desired. 
\end{proof}

\begin{Remark}
    Observe that if $\cX$ admits a DM morphism $\cX\to \cB \Gm^n$, then its moduli space is always effective $\cX\to X$ if $X$ is separated; while for the vice versa we need some assumptions on the singularities of $\cX$, to extend the line bundles. While we do not expect the vice versa to be true
    for arbitrarly singular stacks, it would be interesting to give a finer characterization of the singularities on $\cX$ for which the thesis of \Cref{teo effettivo implica ho il gen set} still holds. 
\end{Remark}



\bibliographystyle{amsalpha}
\bibliography{bibliography}

\end{document}
