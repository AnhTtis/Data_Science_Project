\usepackage[utf8]{inputenc}
%\usepackage[T1]{fontenc}
%\usepackage[LY1]{fontenc}
%\usepackage[OT1]{fontenc}
%\usepackage{lmodern}
%\usepackage{libertine}
%\usepackage[libertine]{newtxmath}
%\usepackage{lmodern}


%%% Font setup
\usepackage[lining,semibold]{libertine} % a bit lighter than Times--no osf in math
%\usepackage{txfonts}
%\usepackage[T1]{fontenc} % best for Western European languages
%\usepackage{textcomp} % required to get special symbols
%\usepackage[varqu,varl]{inconsolata}% a typewriter font must be defined
%\usepackage{amsthm}% must be loaded before newtxmath
\usepackage[libertine, cmintegrals, bigdelims, vvarbb]{newtxmath}
%\usepackage[bigdelims, vvarbb]{newtxmath}
%\usepackage[scr=rsfso]{mathalfa}
%\usepackage{bm}% load after all math to give access to bold math
%After loading math package, switch to osf in text.
%\useosf % for osf in normal text

\usepackage{amsmath}
\usepackage{amsfonts}
\usepackage{mathrsfs}
\usepackage{gensymb}
\usepackage{bbm}
\usepackage{dsfont}

\usepackage{kbordermatrix}
\renewcommand{\kbldelim}{(}
\renewcommand{\kbrdelim}{)}

\usepackage{chemformula}
\usepackage[caption=false]{subfig}
\usepackage{chemfig}
\usepackage[version=3]{mhchem}

\usepackage{tikz}
\usetikzlibrary{matrix,positioning,decorations.pathreplacing}

\usepackage{soul}
\usepackage{xcolor}

\usepackage{scalerel} % To change size things in math environment \scaleto{...}{1pt}
\usepackage{comment}

\usepackage{xltabular}
\usepackage{tabularray}
\usepackage{booktabs}


%\usepackage{filecontents}
%\usepackage{natbib}
%\usepackage[resetlabels,labeled]{multibib}
%\newcites{supp}{Supplementary References}


%%%%%%%%%%%%%%%%%%%%%%%%%%%%%%%%%%%%%%%%%%%%%%%%%%%%%%%
%%%%%%%%%%%%%%%%%%%%%%%%%%%%%%%%%%%%%%%%%%%%%%%%%%%%%%%
%%%%%%%%%%%%%%%%%%%%%%%%%%%%%%%%%%%%%%%%%%%%%%%%%%%%%%%

\definecolor{webgreen}{rgb}{0,.5,0}
\definecolor{webbrown}{rgb}{.6,0,0}
\definecolor{grigio}{rgb}{.85,.85,.85} 
\definecolor{RoyalBlue}{rgb}{0.0, 0.14, 0.4}
\definecolor{skyblue1}{rgb}{0.45,0.62,0.81}
\definecolor{skyblue2}{rgb}{0.2,0.39,0.64}
\definecolor{skyblue3}{rgb}{0.13,0.29,0.53}
\definecolor{scarlet1}{rgb}{0.93,0.16,0.16}
\definecolor{scarlet2}{rgb}{0.8,0,0}
\definecolor{scarlet3}{rgb}{0.64,0,0}

\definecolor{g}{gray}{0.50}


\usepackage{hyperref}
\hypersetup{%
    %hyperfootnotes=false,pdfpagelabels,%
    %draft,	% = elimina tutti i link (utile per stampe in bianco e nero)
    colorlinks=true, linktocpage=true, pdfstartpage=1, pdfstartview=FitV,%
    % decommenta la riga seguente per avere link in nero (per esempio per la stampa in bianco e nero)
    %colorlinks=false, linktocpage=false, pdfborder={0 0 0}, pdfstartpage=1, pdfstartview=FitV,% 
    breaklinks=true, pdfpagemode=UseNone, pageanchor=true, pdfpagemode=UseOutlines,%
    plainpages=false, bookmarksnumbered, bookmarksopen=true, bookmarksopenlevel=1,%
    hypertexnames=true, pdfhighlight=/O,%nesting=true,%frenchlinks,%
    urlcolor=webbrown, linkcolor=RoyalBlue, citecolor=webgreen, %pagecolor=RoyalBlue,%
    %urlcolor=Black, linkcolor=Black, citecolor=Black, %pagecolor=Black,%
    pdftitle={},%
    pdfauthor={Francesco Avanzini},%
    pdfsubject={},%
    pdfkeywords={},%
    pdfcreator={pdfLaTeX},%
    pdfproducer={LaTeX REVTeX}%
}