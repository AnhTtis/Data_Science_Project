\documentclass[9pt,twocolumn,twoside]{pnas-new}
% Use the lineno option to display guide line numbers if required.

\templatetype{pnasresearcharticle} % Choose template 
% {pnasresearcharticle} = Template for a two-column research article
% {pnasmathematics} %= Template for a one-column mathematics article
% {pnasinvited} %= Template for a PNAS invited submission

% additional package for chemical equations with annotations above arrows
\usepackage{chemarr}
% additional package to locate floats
\usepackage{float}
\usepackage{soul} % -> \st{}

\usepackage{epstopdf}
%\epstopdfDeclareGraphicsRule{.tiff}{png}{.png}{convert #1 \OutputFile}
%\AppendGraphicsExtensions{.tiff}

\title{Nonequilibrium calcium dynamics optimizes the energetic efficiency of mitochondrial metabolism}

% Use letters for affiliations, numbers to show equal authorship (if applicable) and to indicate the corresponding author
\author[a,b,1]{Valérie Voorsluijs}
\author[a,1]{Francesco Avanzini} 
\author[a,c,1]{Gianmaria Falasco}
\author[a,1]{Massimiliano Esposito}
\author[a,b,d,1]{Alexander Skupin}

\affil[a]{Complex Systems and Statistical Mechanics, Department of Physics and Materials Science, University of Luxembourg, L-1511 Luxembourg, Luxembourg.}
\affil[b]{Luxembourg Centre for Systems Biomedicine, University of Luxembourg, L-4365 Esch-sur-Alzette, Luxembourg.}
\affil[c]{Department of Physics and Astronomy, University of Padova, Via Marzolo 8, I-35131 Padova, Italy.}
\affil[d]{Department of Neuroscience, University of California San Diego, CA 92093, USA.}
% Please give the surname of the lead author for the running footer
\leadauthor{Voorsluijs} 

% Please add a significance statement to explain the relevance of your work (<= 120 words)
\significancestatement{Mitochondria are crucial to cellular energy management and impairments in mitochondrial energy metabolism are linked to a variety of pathologies including neurodegeneration, diabetes and cancer. Homeostasis of mitochondrial metabolism is regulated by diverse active processes to maintain cells in a favorable energy state, where calcium signaling is a predominant mechanism. The cross-talk between calcium dynamics and mitochondrial metabolism is based on the energy-dependent intracellular calcium management and the activating action of calcium on mitochondrial energy production. Here we show that this essential regulatory interplay exhibits robustly a maximal efficiency at the onset of calcium oscillations which maximizes ATP production and therefore represents an essential regulation of cell maintenance.}

% Please include corresponding author, author contribution and author declaration information
\authorcontributions{Author contributions: A.S., M.E., V.V., F.A. and G.F. designed research; V.V., F.A., G.F., A.S. and M.E. performed research; V.V., F.A., G.F., A.S. and M.E. analyzed data; and V.V., A.S., F.A., G.F. and M.E. wrote the paper.}
\authordeclaration{The authors declare no conflicts of interest.}
%\equalauthors{\textsuperscript{1}A.O.(Author One) contributed equally to this work with A.T. (Author Two) (remove if not applicable).}
\correspondingauthor{\textsuperscript{1}To whom correspondence should be addressed. E-mail: valerie.voorsluijs@uni.lu, francesco.avanzini@uni.lu, gianmaria.falasco@unipd.it, massimiliano.esposito@uni.lu, alexander.skupin@uni.lu}

% At least three keywords are required at submission. Please provide three to five keywords, separated by the pipe symbol.
\keywords{calcium signaling $|$ metabolism $|$ nonequilibrium thermodynamics $|$ energetic cost} 

\begin{abstract}
%Please provide an abstract of no more than 250 words in a single paragraph. Abstracts should explain to the general reader the major contributions of the article. References in the abstract must be cited in full within the abstract itself and cited in the text.
Living organisms continuously harness energy to perform complex functions for their adaptation and survival while part of that energy is dissipated in the form of heat or chemical waste. 
%{\color{orange}However, these processes are not fully efficient as part of their energy input is dissipated in the form of heat or chemical waste.} 
Determining the energetic cost and the efficiency of specific cellular processes remains a largely open problem.
Here, we analyze the efficiency of mitochondrial adenosine triphosphate (ATP) production through the tricarboxylic acid (TCA) cycle and oxidative phosphorylation that generates most of the cellular chemical energy in eukaryotes.
The regulation of this pathway by calcium signaling represents a well-characterized example of a regulatory cross-talk that can affect the energetic output of a metabolic pathway, but its concrete energetic impact remains elusive.
On the one hand, calcium enhances ATP production by activating key enzymes of the TCA cycle, but on the other hand calcium homeostasis depends on ATP availability.
To evaluate how calcium signaling impacts the efficiency of mitochondrial metabolism, we propose a detailed kinetic model describing the calcium-mitochondria cross-talk and we analyze it using a nonequilibrium thermodynamic approach:
% {\color{blue}that is thermodynamically consistent}.
%From the identified set of effective reactions maintaining mitochondrial metabolism out of equilibrium and their associated work contributions, the  thermodynamic efficiency of the metabolic machinery was determined under different conditions. 
%Rephrasing by Massi (I put everything in present tense):
after identifying the effective reactions driving mitochondrial metabolism out of equilibrium, we quantify the thermodynamic efficiency of the metabolic machinery for different physiological conditions.
% After identifying the effective reactions driving mitochondrial metabolism out of equilibrium, we {\color{blue}use a nonequilibrium thermodynamics approach to analyze} the thermodynamic efficiency of the metabolic machinery for different physiological conditions.
We find that calcium oscillations increase the efficiency with a maximum close to substrate-limited conditions, suggesting a compensatory effect of calcium signaling on the energetics of mitochondrial metabolism.
\end{abstract}

\dates{This manuscript was compiled on \today}
\doi{\url{www.pnas.org/cgi/doi/10.1073/pnas.XXXXXXXXXX}}

\begin{document}

\maketitle
\thispagestyle{firststyle}
\ifthenelse{\boolean{shortarticle}}{\ifthenelse{\boolean{singlecolumn}}{\abscontentformatted}{\abscontent}}{}

% If your first paragraph (i.e. with the \dropcap) contains a list environment (quote, quotation, theorem, definition, enumerate, itemize...), the line after the list may have some extra indentation. If this is the case, add \parshape=0 to the end of the list environment.
% \dropcap{T}his PNAS journal template is provided to help you write your work in the correct journal format. Instructions for use are provided below. 

% Note: please start your introduction without including the word ``Introduction'' as a section heading (except for math articles in the Physical Sciences section); this heading is implied in the first paragraphs. 

% \section*{Guide to using this template on Overleaf}

% Please note that whilst this template provides a preview of the typeset manuscript for submission, to help in this preparation, it will not necessarily be the final publication layout. For more detailed information please see the \href{https://www.pnas.org/page/authors/format}{PNAS Information for Authors}.

% If you have a question while using this template on Overleaf, please use the help menu (``?'') on the top bar to search for \href{https://www.overleaf.com/help}{help and tutorials}. You can also \href{https://www.overleaf.com/contact}{contact the Overleaf support team} at any time with specific questions about your manuscript or feedback on the template.

% \subsection*{Author Affiliations}

% Include department, institution, and complete address, with the ZIP/postal code, for each author. Use lower case letters to match authors with institutions, as shown in the example. PNAS strongly encourages authors to supply an \href{https://orcid.org/}{ORCID identifier} for each author. Individual authors must link their ORCID account to their PNAS account at \href{http://www.pnascentral.org/}{www.pnascentral.org}. For proper authentication, authors must provide their ORCID at submission and are not permitted to add ORCIDs on proofs.

% \subsection*{Submitting Manuscripts}
% All authors must submit their articles at \href{http://www.pnascentral.org/cgi-bin/main.plex}{PNAScentral}. If you are using Overleaf to write your article, you can use the ``Submit to PNAS'' option in the top bar of the editor window. 

% \subsection*{Format}
% Many authors find it useful to organize their manuscripts with the following order of sections;  title, author line and affiliations, keywords, abstract, significance statement, introduction, results, discussion, materials and methods, acknowledgments, and references. Other orders and headings are permitted.

% \subsection*{Manuscript Length}
% A standard 6-page article is approximately 4,000 words, 50 references, and 4 medium-size graphical elements (i.e., figures and tables). The preferred length of articles remains at 6 pages, but PNAS will allow articles up to a maximum of 12 pages.

% \subsection*{References}
% References should be cited in numerical order as they appear in text; this will be done automatically via bibtex, e.g. \cite{belkin2002using} and \cite{berard1994embedding,coifman2005geometric}. All references cited in the main text should be included in the main manuscript file.

% \subsection*{Data Archival}
% PNAS must be able to archive the data essential to a published article. Where such archiving is not possible, deposition of data in public databases, such as GenBank, ArrayExpress, Protein Data Bank, Unidata, and others outlined in the \href{https://www.pnas.org/page/authors/journal-policies#xi}{Information for Authors}, is acceptable.

% \subsection*{Language-Editing Services}
% Prior to submission, authors who believe their manuscripts would benefit from professional editing are encouraged to use a language-editing service (see list at www.pnas.org/page/authors/language-editing). PNAS does not take responsibility for or endorse these services, and their use has no bearing on acceptance of a manuscript for publication. 

% \begin{figure}%[tbhp]
% \centering
% \includegraphics[width=.8\linewidth]{frog}
% \caption{Placeholder image of a frog with a long example legend to show justification setting.}
% \label{fig:frog}
% \end{figure}

% \begin{SCfigure*}[\sidecaptionrelwidth][t]
% \centering
% \includegraphics[width=11.4cm,height=11.4cm]{frog}
% \caption{This legend would be placed at the side of the figure, rather than below it.}\label{fig:side}
% \end{SCfigure*}

% \subsection*{Digital Figures}
% EPS, high-resolution PDF, and PowerPoint are preferred formats for figures that will be used in the main manuscript. Authors may submit PRC or U3D files for 3D images; these must be accompanied by 2D representations in TIFF, EPS, or high-resolution PDF format. Color images must be in RGB (red, green, blue) mode. Include the font files for any text. 

% Images must be provided at final size, preferably 1 column width (8.7cm). Figures wider than 1 column should be sized to 11.4cm or 17.8cm wide. Numbers, letters, and symbols should be no smaller than 6 points (2mm) and no larger than 12 points (6mm) after reduction and must be consistent. 

% Figures and tables should be labelled and referenced in the standard way using the \verb|\label{}| and \verb|\ref{}| commands.

% Figure \ref{fig:frog} shows an example of how to insert a column-wide figure. To insert a figure wider than one column, please use the \verb|\begin{figure*}...\end{figure*}| environment. Figures wider than one column should be sized to 11.4 cm or 17.8 cm wide. Use \verb|\begin{SCfigure*}...\end{SCfigure*}| for a wide figure with side legends.

% \subsection*{Tables}
% Tables should be included in the main manuscript file and should not be uploaded separately.

% \subsection*{Single column equations}
% Authors may use 1- or 2-column equations in their article, according to their preference.

% To allow an equation to span both columns, use the \verb|\begin{figure*}...\end{figure*}| environment mentioned above for figures.

% Note that the use of the \verb|widetext| environment for equations is not recommended, and should not be used. 

% \begin{figure*}[bt!]
% \begin{align*}
% (x+y)^3&=(x+y)(x+y)^2\\
%       &=(x+y)(x^2+2xy+y^2) \numberthis \label{eqn:example} \\
%       &=x^3+3x^2y+3xy^3+x^3. 
% \end{align*}
% \end{figure*}


% \begin{table}%[tbhp]
% \centering
% \caption{Comparison of the fitted potential energy surfaces and ab initio benchmark electronic energy calculations}
% \begin{tabular}{lrrr}
% Species & CBS & CV & G3 \\
% \midrule
% 1. Acetaldehyde & 0.0 & 0.0 & 0.0 \\
% 2. Vinyl alcohol & 9.1 & 9.6 & 13.5 \\
% 3. Hydroxyethylidene & 50.8 & 51.2 & 54.0\\
% \bottomrule
% \end{tabular}

% \addtabletext{nomenclature for the TSs refers to the numbered species in the table.}
% \end{table}

% \subsection*{Supporting Information Appendix (SI)}

% Authors should submit SI as a single separate SI Appendix PDF file, combining all text, figures, tables, movie legends, and SI references. SI will be published as provided by the authors; it will not be edited or composed. Additional details can be found in the \href{https://www.pnas.org/authors/submitting-your-manuscript#manuscript-formatting-guidelines}{PNAS Author Center}. The PNAS Overleaf SI template can be found \href{https://www.overleaf.com/latex/templates/pnas-template-for-supplementary-information/wqfsfqwyjtsd}{here}. Refer to the SI Appendix in the manuscript at an appropriate point in the text. Number supporting figures and tables starting with S1, S2, etc.

% Authors who place detailed materials and methods in an SI Appendix must provide sufficient detail in the main text methods to enable a reader to follow the logic of the procedures and results and also must reference the SI methods. If a paper is fundamentally a study of a new method or technique, then the methods must be described completely in the main text.

% \subsubsection*{SI Datasets} 

% Supply .xlsx, .csv, .txt, .rtf, or .pdf files. This file type will be published in raw format and will not be edited or composed.


% \subsubsection*{SI Movies}

% Supply Audio Video Interleave (avi), Quicktime (mov), Windows Media (wmv), animated GIF (gif), or MPEG files. Movie legends should be included in the SI Appendix file. All movies should be submitted at the desired reproduction size and length. Movies should be no more than 10MB in size.

% \subsubsection*{3D Figures}

% Supply a composable U3D or PRC file so that it may be edited and composed. Authors may submit a PDF file but please note it will be published in raw format and will not be edited or composed.

\dropcap{L}ife relies on permanent conversions between different forms of energy, a phenomenon referred to as energy transduction. A wide range of cellular processes are fueled by the chemical energy stored in adenosine triphosphate (ATP), but the compartmentalization of eukaryotic cells also enables the storage of potential energy across the membranes of organelles \cite{nicholls_bioenergetics_1992}. Energy transduction is mediated by enzymes and pumps driven in a nonequilibrium thermodynamic manner by the hydrolysis of ATP, chemical gradients or membrane potentials.

In optimal scenarios where transduction is fully efficient, the input energy is completely transformed into usable work. However, biological processes are typically accompanied by entropy production, \textit{i.e.}, dissipation of energy in the form of heat and/or chemical waste that is unusable for transduction~\cite{calisto_mechanisms_2021}. For example, the action of many transmembrane ionic pumps transporting ions against their concentration gradient is often based on catalysing the hydrolysis of ATP. The chemical energy released by hydrolysis is partly used to drive ionic transport while another part is dissipated. In the extreme case of pump uncoupling, also known as ``slippage'', all the energy of ATP hydrolysis is dissipated without any ion transport~\cite{berman_slippage_2001}.


Different nonequilibrium kinetic models have been developed to account for energy loss in pumps \cite{graber_bioenergetics_1997, rubi_energy_2007, hill_free_2012, wikstrom_thermodynamic_2020} but have only provided limited insights into energetic costs at the pathway level. New approaches based on metabolic network reconstruction and nonequilibrium thermodynamics are gradually emerging to rationalize the energetic costs of cellular processes~\cite{yang_physical_2021} including gene regulation \cite{estrada_information_2016}, repair mechanisms \cite{sartori_thermodynamics_2015, goloubinoff_chaperones_2018}, enzymatic catalysis \cite{flamholz_glycolytic_2013}, information processing \cite{parrondo_thermodynamics_2015} or signaling \cite{cao_free-energy_2015, rodenfels_heat_2019}.
%These processes are often energy-demanding, but their potentially activating or stabilising effect on ATP-producing pathways can lead to an improved net energetic balance of the coupled pathways. 
%I would like to insert somewhere something along these lines: 
A framework to study energy transduction in complex open chemical reaction networks (CRN) has recently been proposed and used to study the efficiency of pathways of the central energy metabolism in the absence of regulations \cite{wachtel_free-energy_2022}. Evaluating the efficiency of tightly coupled transduction processes, \textit{i.e.} processes whose input and output currents are equal, is straightforward as it does not depend on the net reaction flux. However, when regulations come into play, this tight coupling can be lost and kinetic models become indispensable to evaluate the flux of the different processes contributing to the efficiency.
%The efficiency of tightly coupled transduction processes is easy to evaluate as it does not depend on the net reaction flux, but in the absence of tight coupling, kinetic models become indispensable to evaluate the flux of the different processes contributing to the efficiency. 
%Then we somhow say that this is what we do here

Here, we resort to a such a kinetically-detailed nonequilibrium thermodynamic approach to show and quantify how active signaling can have a beneficial energetic impact on metabolism. In particular, we analyze the efficiency of the mitochondrial production of ATP \textit{via} the tricarboxylic acid (TCA) cycle and oxidative phosphorylation (OXPHOS), and take into account its regulation by calcium (Ca\textsuperscript{2+}). Since intracellular Ca\textsuperscript{2+} dynamics is strongly nonlinear (which can lead to oscillations in Ca\textsuperscript{2+} concentration) and depends itself on ATP availability, evaluating the net effect of signaling on the energetic efficiency of mitochondria is not straightforward. Our analysis quantifies the energetic efficiency of this essential cellular process beyond steady-state conditions, such as in an oscillatory regime. Overall, the proposed framework is laying the foundations for a more comprehensive characterization of energetic costs in biology.

%In particular, {\color{purple}we analyze the efficiency of the mitochondrial production of ATP \textit{via} the tricarboxylic acid (TCA) cycle and oxidative phosphorylation (OXPHOS), regulated by calcium (Ca\textsuperscript{2+})}.{\color{purple}In mitochondria, }Ca\textsuperscript{2+} activates two key enzymes of the TCA cycle (isocitrate dehydrogenase and $\alpha$-ketoglutarate dehydrogenase)~\cite{mccormack_characterization_1985, hajnoczky_decoding_1995, griffiths_mitochondrial_2009, denton_regulation_2009} and thereby increases the flux of high energy electrons feeding the electron transport chain. The successive redox reactions in the mitochondrial membrane contribute to the establishment of the proton motive force eventually driving the mitochondrial synthesis of ATP by F1F0-ATPase. {\color{purple}Depending on the availability of cytosolic ATP, Ca\textsuperscript{2+} can, however, be sequestrated into cell compartments other than mitochondria, such as the endoplasmic reticulum (ER) \textit{via} the sarcoendoplasmic reticulum Ca\textsuperscript{2+} ATPase (SERCA), or extruded to the extracellular space~\cite{berridge_calcium--life_1998}. These mechanisms ensure that Ca\textsuperscript{2+} does not accumulate in the cytosol, a persistent high cytosolic Ca\textsuperscript{2+} concentration being toxic for the cell. Given the nonlinear dynamics of intracellular Ca\textsuperscript{2+}, assessing its net impact on the energetic efficiency of mitochondria is nontrivial.}

%To address this challenge, we developed a curated model for the essential Ca\textsuperscript{2+}-metabolism system integrating different modules~\cite{magnus_minimal_1997, magnus_model_1998-c, magnus_model_1998-m, dudycha_detailed_2000, cortassa_integrated_2003, bertram_simplified_2006, wei_mitochondrial_2011, komin_multiscale_2015, berndt_physiology-based_2015, wacquier_interplay_2016}. {\color{purple}These kinetic models originally aimed at identifying the essential mechanisms underlying the Ca\textsuperscript{2+}-metabolism interplay and at rationalising experimental data about the response of Ca\textsuperscript{2+} signals to changes in mitochondrial activity (and \textit{vice versa}). After refining these models to combine them in a coherent way, and make them thermodynamically and experimentally consistent (see details in Section 1 of SI Appendix),} we analyzed the coupled pathways by a nonequilibrium thermodynamic description of CRN \cite{rao_nonequilibrium_2016, rao_conservation_2018, wachtel_thermodynamically_2018, avanzini_thermodynamics_2020, avanzini_nonequilibrium_2021, avanzini_thermodynamics_2022, wachtel_free-energy_2022}. With this strategy, our analysis quantifies the energetic efficiency of this essential cellular process for active regulation and non-steady-state conditions such as in an oscillatory regime. In this respect, the proposed framework is laying the foundations for a more comprehensive characterization of energetic costs in biology.

% {\color{red} Comment by Francesco: what about reformulating the introduction focusing more on metabolism and energetics? 
% We could discuss 1) flux-balance analysis which (mainly) focuses on the dynamics of large nonlinear networks and 2) kinetic proofreading where the trade-off between speed, accuracy, and dissipation is examined in small \textbf{linear (or pseudo-linear)} networks (see for instance works by Pigolotti and Igoshin).
% From this point of view what we do is to go beyond the linear set-up for a large/complex network using a systematic formulation of thermodynamics of CRNs.}

\section*{Biological background}
In mitochondria, Ca\textsuperscript{2+} activates two key enzymes of the TCA cycle (isocitrate dehydrogenase and $\alpha$-ketoglutarate dehydrogenase)~\cite{mccormack_characterization_1985, hajnoczky_decoding_1995, griffiths_mitochondrial_2009, denton_regulation_2009} and thereby increases the flux of high energy electrons, in the form of NADH, feeding the electron transport chain. The successive redox reactions in the mitochondrial membrane contribute to the establishment of the proton motive force 
%eventually 
driving the mitochondrial synthesis of ATP by F1F0-ATPase. Depending on the concentration of cytosolic ATP, Ca\textsuperscript{2+} can, however, be sequestrated into cell compartments other than mitochondria, such as the endoplasmic reticulum (ER) \textit{via} the sarcoendoplasmic reticulum Ca\textsuperscript{2+} ATPase (SERCA), or extruded to the extracellular space~\cite{berridge_calcium--life_1998}. These mechanisms ensure that Ca\textsuperscript{2+} does not accumulate in the cytosol, as a persistent high cytosolic Ca\textsuperscript{2+} concentration is toxic for the cell.
The central coupling enabling the Ca\textsuperscript{2+}-mitochondria cross-talk is thus given by the Ca\textsuperscript{2+} fluxes between the cytosol and the ER or mitochondria (Fig.~\ref{fig:fluxes}).
The Ca\textsuperscript{2+} release from the ER, by leakage or \textit{via} channels (IP\textsubscript{3}Rs) upon stimulation by inositol 1,4,5-trisphosphate (IP\textsubscript{3}), and Ca\textsuperscript{2+} exchanges with mitochondria are ATP-independent, as opposed to Ca\textsuperscript{2+} transport into the ER that relies on ATP-consuming SERCA pumps.
%Increased Ca\textsuperscript{2+} levels in mitochondria trigger TCA and OXPHOS activity and subsequent ATP production. The inositol 1,4,5-trisphosphate (IP\textsubscript{3}) and acetyl coenzyme A (AcCoA) are the key species controlling Ca\textsuperscript{2+} and mitochondrial dynamics, respectively.

\begin{figure}[t!]
    \centering
    \includegraphics[width=\linewidth]{Fig1_20230124}
    \caption{Representation of the model components, conceptualization of mitochondria as a chemical engine and corresponding abbreviations. Balanced chemical equations, detailed expressions of the reaction rates, thermodynamic forces and reference parameter values are given in Tables S1, S2, S3 and S4 in SI Appendix, respectively. (\textit{A}) The upper part depicts the Ca\textsuperscript{2+} (red) and ATP (blue) fluxes responsible for the cross-talk between Ca\textsuperscript{2+} dynamics and mitochondrial metabolism. The bottom part is a detailed description of the model components. The kinetic rates for TCA cycle fluxes and processes involving exchanges across the mitochondrial membrane are respectively originating from Dudycha \cite{dudycha_detailed_2000} and Magnus-Keizer models \cite{magnus_minimal_1997, magnus_model_1998-c, magnus_model_1998-m}, except for the transformation of MAL into OAA, which is described more realistically by a reversible flux \cite{berndt_physiology-based_2015}. Here, OXPHOS corresponds to the net redox reaction resulting from the electron transport chain (Ox) and the synthesis of ATP by the F1F0-ATPase (F1). A last module, consisting of Ca\textsuperscript{2+} exchanges across the ER membrane and cytosolic ATP hydrolysis are taken from the models from Komin \textit{et al.} \cite{komin_multiscale_2015} and from Wacquier \textit{et al.} \cite{wacquier_interplay_2016}. Controlled species (\textit{i.e.}, species whose concentration is assumed to be constant) are shown in gray, dynamical species in black and dashed arrows represent regulations. Processes are annotated in yellow and black boxes for mitochondrial and cytosolic/ER processes, respectively. (\textit{B}) Mitochondrial metabolism is conceptualized as an open chemical engine that transforms ADP\textsubscript{c} into ATP\textsubscript{c} through a set of 2 emergent cycles split in 3 effective reactions. Some of the controlled species involved in the internal reactions are buffered at a constant concentrations (green), while Na\textsuperscript{+} (turquoise) and Ca\textsuperscript{2+} (red) regulate reaction rates by activating specific enzymes or acting on the mitochondrial membrane potential.
    Abbreviations: AcCoA -- acetyl coenzyme A, $\alpha$KG -- alpha-ketoglutarate, ATP -- Adenosine triphosphate, ADP -- Adenosine diphosphate, CIT -- citrate, CoA -- coenzyme A, CoQ/CoQH\textsubscript{2} -- coenzyme Q10, FUM -- fumarate, IP\textsubscript{3} -- inositol 1,4,5-trisphosphate, ISOC -- isocitrate, MAL -- malate, NAD\textsuperscript{+}/NADH -- nicotinamide adenine dinucleotide, OAA -- oxaloacetate, Pi -- inorganic phosphate, SUC -- succinate, SCoA -- succinyl coenzyme A.}
    \label{fig:fluxes}
\end{figure}

\section*{Modeling and theoretical frameworks}
We developed a curated model for the essential Ca\textsuperscript{2+}-metabolism system integrating different modules~\cite{magnus_minimal_1997, magnus_model_1998-c, magnus_model_1998-m, dudycha_detailed_2000, cortassa_integrated_2003, bertram_simplified_2006, wei_mitochondrial_2011, komin_multiscale_2015, berndt_physiology-based_2015, wacquier_interplay_2016}. Its comprehensive parameterization on experimental data represent a major step towards the detailed analysis of the mitochondrial regulation by Ca\textsuperscript{2+}. The underlying kinetic models originally aimed at capturing the essential mechanisms of the the Ca\textsuperscript{2+}-metabolism interplay and at rationalizing experimental data about the response of Ca\textsuperscript{2+} signals to changes in mitochondrial activity (and \textit{vice versa}). After refining these models to combine them in a coherent way, 
% and {\color{blue}make them thermodynamically consistent}, 
we analyzed the coupled pathways by a nonequilibrium thermodynamic description of CRN \cite{rao_nonequilibrium_2016, rao_conservation_2018, wachtel_thermodynamically_2018, avanzini_thermodynamics_2020, avanzini_nonequilibrium_2021, avanzini_thermodynamics_2022, wachtel_free-energy_2022}.

%Mechanism of the Ca\textsuperscript{2+}-metabolism cross-talk (\textit{A}) Main fluxes underlying the interplay between Ca\textsuperscript{2+} signalling and mitochondrial metabolism (top) for which detailed fluxes, dynamical variables and controlled species are shown in Fig.~1. Mitochondrial metabolism is conceptualized as an open chemical engine (bottom) that transforms ADP\textsubscript{c} into ATP\textsubscript{c} through a set of effective reactions. 
% Part of the chemical species involved in internal reactions are tuned externally. The concentration of some reactants and products are buffered at a constant level (green). Na\textsuperscript{+} (turquoise) and Ca\textsuperscript{2+} (red) do not appear in the stoichiometry of internal reactions, but regulate reaction rates by activating specific enzymes or acting on the mitochondrial membrane potential.

%Some of the chemical species involved in internal reactions are buffered at a constant concentrations (green), while Na\textsuperscript{+} (turquoise) and Ca\textsuperscript{2+} (red) regulate reaction rates by activating specific enzymes or acting on the mitochondrial membrane potential. (\textit{B}-\textit{C}) $\mathrm{Ca}^{2+}_\mathrm{c}$ and ATP\textsubscript{c} concentrations over time for $\left[\mathrm{AcCoA}\right]=1\,\mathrm{\mu M}$ and (\textit{B}) $\left[\mathrm{IP_3}\right]=0.12\,\mathrm{\mu M}$ or (\textit{C}) $\left[\mathrm{IP_3}\right]=0.20\,\mathrm{\mu M}$. (\textit{D}) Effect of $\left[\mathrm{IP_3}\right]$ and $\left[\mathrm{AcCoA}\right]$ on the oscillation period. (\textit{E}-\textit{F}) Average concentration of $\mathrm{Ca}^{2+}_\mathrm{c}$ and ATP\textsubscript{c} as a function of (\textit{E}) $\left[\mathrm{IP_3}\right]$ for $\left[\mathrm{AcCoA}\right]=1\,\mathrm{\mu M}$ or as a function of (\textit{F}) $\left[\mathrm{AcCoA}\right]$ for $\left[\mathrm{IP_3}\right]=0.12\,\mathrm{\mu M}$. Empty and filled dots represent steady-state and oscillatory regimes, respectively, and the boundaries of the shaded areas correspond to the minimum and maximum concentrations. Parameter values are given in SI Appendix.

To compute their metabolic efficiency, we analyzed mitochondria as out-of-equilibrium chemical engines (Fig.~\ref{fig:fluxes}\textit{B}) satisfying the second law of thermodynamics~\cite{rao_conservation_2018, avanzini_nonequilibrium_2021}:
\begin{equation}
    T \sigma = - \mathrm{d}_t \mathcal{G} + \dot{w}_\mathrm{nc} + \dot{w}_\mathrm{driv}\,.\label{eq:2law}
\end{equation}
Mitochondrial metabolism constitutes an open CRN that continuously harnesses the free energy stored in buffered species (e.g., AcCoA, CoQ, O\textsubscript{2}, $\mathrm{H}^{+}_\mathrm{m}$) to synthesize ATP\textsubscript{c} from ADP\textsubscript{c} while being influenced by $\mathrm{Na^{+}}$ homeostasis and cytosolic processes such as Ca\textsuperscript{2+} signaling and ATP\textsubscript{c} consumption. From a thermodynamic perspective, the synthesis of ATP\textsubscript{c} and the regulations correspond to free energy exchanges between the mitochondrial engine and its surroundings. They appear in the second law (Eq.~\ref{eq:2law}) as the nonconservative work rate, $\dot{w}_\mathrm{nc}$, and the driving work rate, $\dot{w}_\mathrm{driv}$, respectively. The difference between their sum and the variation in time of the internal Gibbs free energy of mitochondria, $\mathcal{G}$, equals the free energy dissipated by the mitochondrial reactions, \textit{i.e.} the entropy production rate (EPR) $\sigma$ times the absolute temperature $T$.
% The former appears in the second law~(\ref{eq:2law}) as the nonconservative work rate $\dot{w}_\mathrm{nc}$,
% while latter as the driving work rate $\dot{w}_\mathrm{driv}$.
%They balance the variation of the internal (free) energy of mitochondria, quantified by the so-called (semigrand) Gibbs free energy $\mathcal{G}$, and the free energy dissipated by the mitochondrial reactions, quantified by the entropy production rate (EPR) $\sigma$. 

\begin{figure*}[h!]
%{\tiny
    \begin{align}
        \mathrm{ADP}_\mathrm{c} + \mathrm{Pi}_\mathrm{m} & \xrightleftharpoons[]{\mathrm{\mathbf{r1_{out}}}} \mathrm{ATP}_\mathrm{c} + \mathrm{H}_2 \mathrm{O}_\mathrm{m} \label{eq:chemOutput}\\
        \frac{3}{22}\, \mathrm{O}_2 + \frac{1}{11}\, \mathrm{AcCoA} + \frac{1}{11}\, \mathrm{CoQ} & \xrightleftharpoons[]{\mathrm{\mathbf{r1_{in}}}} \frac{1}{11}\, \mathrm{CoA} + \frac{2}{11}\, \mathrm{CO}_2 + \frac{1}{11}\, \mathrm{CoQH}_2 \label{eq:chemInputTCA}\\
      \mathrm{H}^{+}_\mathrm{m} +  \frac{1}{22}\, \mathrm{O}_2 + \frac{1}{33}\, \mathrm{AcCoA} + \frac{1}{33}\, \mathrm{CoQ} & \xrightleftharpoons[]{\mathbf{r2}} \mathrm{H}^{+}_\mathrm{c} + \frac{1}{33}\, \mathrm{CoA} + \frac{2}{33}\, \mathrm{CO}_2 + \frac{1}{33}\, \mathrm{CoQH}_2. \label{eq:emHc}
    \end{align}
%}
\end{figure*}

The expressions of the thermodynamic quantities in Eq.~\ref{eq:2law} are derived for mitochondrial metabolism using a topological analysis (developed in refs.~\cite{avanzini_thermodynamics_2020, avanzini_nonequilibrium_2021}) of the corresponding CRN, which allowed us to identify conservation laws and emergent cycles. The conservation laws define parts of molecules that remain intact in all mitochondrial reactions and are instrumental to determine the Gibbs free energy~$\mathcal{G}$. The emergent cycles define the 3 effective reactions in Eqs.~\ref{eq:chemOutput}-\ref{eq:emHc} and split the nonconservative work rate into the sum of 3 contributions: $\dot{w}_\mathrm{nc} = \dot{w}_\mathrm{r1_{out}} + \dot{w}_\mathrm{r1_{in}}+ \dot{w}_\mathrm{r2}$, where $\dot{w}_\mathrm{r1_{out}}$ quantifies the mitochondrial free energy output corresponding to the synthesis of ATP in the cytosol~(Eq.~\ref{eq:chemOutput}), while $\dot{w}_\mathrm{r1_{in}}$ and $\dot{w}_\mathrm{r2}$ quantify the mitochondrial free energy power source due the interconversion of the buffered species \textit{via} reactions in Eqs.~\ref{eq:chemInputTCA} and~\ref{eq:emHc}, respectively.
%%%%%%%%%%%%%%%%%%%%%%%%%%%%%%%%%%%
\begin{figure}[t!]
\centering
\includegraphics[width=\linewidth]{Fig2_20230124}
\caption{Kinetic behavior of the system. (\textit{A}-\textit{B}) $\mathrm{Ca}^{2+}_\mathrm{c}$ and ATP\textsubscript{c} concentrations over time for $\left[\mathrm{AcCoA}\right]=1\,\mathrm{\mu M}$ and (\textit{A}) $\left[\mathrm{IP_3}\right]=0.12\,\mathrm{\mu M}$ or (\textit{B}) $\left[\mathrm{IP_3}\right]=0.20\,\mathrm{\mu M}$. (\textit{C}) Effect of $\left[\mathrm{IP_3}\right]$ and $\left[\mathrm{AcCoA}\right]$ on the oscillation period. (\textit{D}-\textit{E}) Average concentration of $\mathrm{Ca}^{2+}_\mathrm{c}$ and ATP\textsubscript{c} as a function of (\textit{D}) $\left[\mathrm{IP_3}\right]$ for $\left[\mathrm{AcCoA}\right]=1\,\mathrm{\mu M}$ or as a function of (\textit{E}) $\left[\mathrm{AcCoA}\right]$ for $\left[\mathrm{IP_3}\right]=0.12\,\mathrm{\mu M}$. Empty and filled dots represent steady-state and oscillatory regimes, respectively, and the boundaries of the shaded areas correspond to the minimum and maximum concentrations. Parameter values are given in Table S4 of SI Appendix. Fig.~S1B illustrates the behavior of $\left[\mathrm{ATP}\right]_\mathrm{c}$ for an extended range of $\left[\mathrm{AcCoA}\right]$ and $\left[\mathrm{IP_3}\right]$.
}
\label{fig:kinetics}
\end{figure}

The average thermodynamic efficiency $\bar{\eta}$ of mitochondria
% , \textit{i.e.}, at steady-state or averaged over one period of Ca\textsuperscript{2+} oscillations, 
can then be calculated as
\begin{equation}
    \bar{\eta} = - \frac{\bar{w}_\mathrm{nc}^\mathrm{output}}{\bar{w}_\mathrm{nc}^\mathrm{input} + \bar{w}_\mathrm{driv}}
\end{equation}
where $\bar{w}_\mathrm{nc}^\mathrm{input}=\bar{w}_\mathrm{r1_{in}}+\bar{w}_\mathrm{r2}$ and $\bar{w}_\mathrm{nc}^\mathrm{output} = \bar{w}_\mathrm{r1_{out}}$,
and the overline denotes either steady-state quantities or averages over one period of Ca\textsuperscript{2+} oscillations (notice that $\overline{\mathrm{d}_t \mathcal{G}} = 0$).
% since the internal free energy $\mathcal{G}$ remains constant. 

The EPR nonconservative and driving work contributions vanish at equilibrium according to the second law of thermodynamics but take finite values in nonequilibrium regimes (Fig.~\ref{fig:thermo}\textit{B}). 
The nonequilibrium kinetics of the system was assessed for different stimulation conditions and mitochondrial substrate concentrations (\textit{i.e.} for different $\left[\mathrm{IP_3}\right]$ and $\left[\mathrm{AcCoA}\right]$ in the simulations), which allowed the calculation of the corresponding nonconservative and driving work contributions and, ultimately, of the efficiency of mitochondrial metabolism.
%Model development, methodology and calculations are detailed in the SI Appendix, which is organized as follows: (1) Description of the kinetic model, (2) Concepts of biothermodynamics and (3) Nonequilibrium thermodynamics analysis.


% - - - - - - - - - - - - - - - - - - - - - - - - - - - - - - - - - - - - - -

% - - - - - - - - - - - - - - - - - - - - - - - - - - - - - - - - - - - - - -

% - - - - - - - - - - - - - - - - - - - - - - - - - - - - - - - - - - - - - -

% - - Valerie's version - - 

% To compute the efficiency of mitochondrial metabolism under regulation of Ca\textsuperscript{2+} signalling, the mitochondrial machinery is analyzed as an open chemical reaction network (CRN) (Fig.~\ref{fig:kinetics}\textit{A}) whose internal behavior depends on the concentrations of \textit{exchanged species}, namely the input and output species (\textit{i.e.} ADP\textsubscript{c} and ATP\textsubscript{c}), buffered species and regulators ($\mathrm{Ca^{2+}_m}$, $\mathrm{Ca^{2+}_c}$, $\mathrm{Na^{+}_m}$, $\mathrm{Na^{+}_c}$). While buffered species, $\mathrm{Na^{+}_m}$ and $\mathrm{Na^{+}_c}$ are assumed to have fixed concentrations in time (controlled species), the concentrations of ADP\textsubscript{c}, ATP\textsubscript{c}, $\mathrm{Ca^{2+}_m}$ and $\mathrm{Ca^{2+}_c}$ evolve according to reaction and exchange fluxes in the whole cell. 

% The first step in the calculation of the energetic efficiency is the identification of the effective reactions for energy input and output of the mitochondrial machinery (Fig.~\ref{fig:kinetics}\textit{A}). A topological analysis of the CRN allowed us to identify global properties of the network, \textit{i.e.} emergent cycles and conservation laws. The former are sets of reactions that leave the internal mitochondrial dynamics unchanged and correspond to effective reactions between exchanged species. The latter are instrumental to identify the intrinsic thermodynamic potential of the CRN, $\mathcal{G}$. Together they allowed us to decompose the entropy production rate (EPR), which quantifies the amount of energy dissipated by a system, into the time variation of the intrinsic thermodynamic potential, $- \mathrm{d}_t\mathcal{G}$, and the different work contributions maintaining the system out of equilibrium, $\dot{w}_\mathrm{nc}$ and $\dot{w}_\mathrm{driv}$, corresponding respectively to nonconservative and driving work \cite{rao_conservation_2018, avanzini_nonequilibrium_2021}:
% \begin{equation}
%     T \sigma = - \mathrm{d}_t \mathcal{G} + \dot{w}_\mathrm{nc} + \dot{w}_\mathrm{driv}.
% \end{equation}
% More detailed definitions of these quantities can be found in the SI Appendix. 

% Based on this decomposition, the average thermodynamic efficiency of the system can be calculated as
% \begin{equation}
%     \bar{\eta} = - \frac{\bar{w}_\mathrm{nc}^\mathrm{output}}{\bar{w}_\mathrm{nc}^\mathrm{input} + \bar{w}_\mathrm{driv}}
% \end{equation}
% where $\bar{w}_\mathrm{nc}^\mathrm{input}=\bar{w}_\mathrm{r1_{in}}+\bar{w}_\mathrm{r2_{in}}$ and $\bar{w}_\mathrm{nc}^\mathrm{output} = \bar{w}_\mathrm{r1_{out}}$ are the nonconservative work contributions, at steady-state or averaged over one period, associated to the (split) emergent cycles $\mathbf{r1}$ and $\mathbf{r2}$ (Eqs. (\ref{eq:chemOutput})-(\ref{eq:emHc})). The  driving work $\bar{w}_\mathrm{driv}$ is identically zero at steady-state.

% % \begin{figure}[t]
% %     \centering
% %     \includegraphics[width=\linewidth]{Fig1}
% %     \caption{Representation of the model components. The upper part depicts the Ca\textsuperscript{2+} (red) and ATP (blue) fluxes responsible for the cross-talk between Ca\textsuperscript{2+} dynamics and mitochondrial metabolism. The bottom part is a detailed description of the model components. The kinetic rates for TCA cycle fluxes and processes involving exchanges across the mitochondrial membrane are respectively originating from Dudycha \cite{dudycha_detailed_2000} and Magnus-Keizer models \cite{magnus_minimal_1997, magnus_model_1998-c, magnus_model_1998-m}, except for the transformation of MAL into OAA, which is described more realistically by a reversible flux \cite{berndt_physiology-based_2015}. Here, OXPHOS is split into Ox and F1 reactions, corresponding respectively to the effective redox reaction of the electron transport chain and the synthesis of ATP by the F1F0-ATPase. A last module, consisting of Ca\textsuperscript{2+} exchanges across the ER membrane and cytosolic ATP hydrolysis are taken from the models from Komin \textit{et al.} \cite{komin_multiscale_2015} and from Wacquier \textit{et al.} \cite{wacquier_interplay_2016}. Controlled species (\textit{i.e.} species whose concentration is assumed to be constant) are shown in gray, dynamical species in black and dashed arrows represent regulations. Processes (Table S1 in SI Appendix) are annotated in yellow and black boxes for mitochondrial and cytosolic/ER processes, respectively. Abbreviations: AcCoA -- acetyl coenzyme A, $\alpha$KG -- alpha-ketoglutarate, ATP -- Adenosine triphosphate, ADP -- Adenosine diphosphate, CIT -- citrate, CoA -- coenzyme A, CoQ/CoQH\textsubscript{2} -- coenzyme Q10, FUM -- fumarate, IP\textsubscript{3} -- inositol 1,4,5-trisphosphate, ISOC -- isocitrate, MAL -- malate, NAD\textsuperscript{+}/NADH -- nicotinamide adenine dinucleotide, OAA -- oxaloacetate, Pi -- inorganic phosphate, SUC -- succinate, SCoA -- succinyl coenzyme A.}
% %     \label{fig:fluxes}
% % \end{figure}

% % \begin{figure*}[ht!]
% % %{\tiny
% %     \begin{align}
% %         \mathrm{ADP}_\mathrm{c} + \mathrm{Pi}_\mathrm{m} & \xrightleftharpoons[]{\mathrm{\mathbf{r1_{out}}}} \mathrm{ATP}_\mathrm{c} + \mathrm{H}_2 \mathrm{O}_\mathrm{m} \label{eq:chemOutput}\\
% %         \frac{3}{22}\, \mathrm{O}_2 + \frac{1}{11}\, \mathrm{AcCoA} + \frac{1}{11}\, \mathrm{CoQ} & \xrightleftharpoons[]{\mathrm{\mathbf{r1_{in}}}} \frac{1}{11}\, \mathrm{CoA} + \frac{2}{11}\, \mathrm{CO}_2 + \frac{1}{11}\, \mathrm{CoQH}_2 \label{eq:chemInputTCA}\\
% %       \mathrm{H}^{+}_\mathrm{m} +  \frac{1}{22}\, \mathrm{O}_2 + \frac{1}{33}\, \mathrm{AcCoA} + \frac{1}{33}\, \mathrm{CoQ} & \xrightleftharpoons[]{\mathbf{r2}} \mathrm{H}^{+}_\mathrm{c} + \frac{1}{33}\, \mathrm{CoA} + \frac{2}{33}\, \mathrm{CO}_2 + \frac{1}{33}\, \mathrm{CoQH}_2. \label{eq:emHc}
% %     \end{align}
% % %}
% % \end{figure*}

% % \begin{figure}%[t!]
% % \centering
% % \includegraphics[width=\linewidth]{Fig2}
% % \caption{Mechanism of the Ca\textsuperscript{2+}-metabolism cross-talk and kinetic behavior of the system. (\textit{A}) Main fluxes underlying the interplay between Ca\textsuperscript{2+} signalling and mitochondrial metabolism (top) for which detailed fluxes, dynamical variables and controlled species are shown in Fig.~1. Mitochondrial metabolism is conceptualized as an open chemical engine (bottom) that transforms ADP\textsubscript{c} into ATP\textsubscript{c} through a set of effective reactions. Part of the chemical species involved in internal reactions are tuned externally. The concentration of some reactants and products are buffered at a constant level (green). Na\textsuperscript{+} (turquoise) and Ca\textsuperscript{2+} (red) do not appear in the stoichiometry of internal reactions, but regulate reaction rates by activating specific enzymes or acting on the mitochondrial membrane potential. (\textit{B}-\textit{C}) $\mathrm{Ca}^{2+}_\mathrm{c}$ and ATP\textsubscript{c} concentrations over time for $\left[\mathrm{AcCoA}\right]=1\,\mathrm{\mu M}$ and (\textit{B}) $\left[\mathrm{IP_3}\right]=0.12\,\mathrm{\mu M}$ or (\textit{C}) $\left[\mathrm{IP_3}\right]=0.20\,\mathrm{\mu M}$. (\textit{D}) Effect of $\left[\mathrm{IP_3}\right]$ and $\left[\mathrm{AcCoA}\right]$ on the oscillation period. (\textit{E}-\textit{F}) Average concentration of $\mathrm{Ca}^{2+}_\mathrm{c}$ and ATP\textsubscript{c} as a function of (\textit{E}) $\left[\mathrm{IP_3}\right]$ for $\left[\mathrm{AcCoA}\right]=1\,\mathrm{\mu M}$ or (\textit{F}) as a function of $\left[\mathrm{AcCoA}\right]$ for $\left[\mathrm{IP_3}\right]=0.12\,\mathrm{\mu M}$. Empty and filled dots represent steady-state and oscillatory regimes, respectively, and the boundaries of the shaded areas correspond to the minimum and maximum concentrations. Parameter values are given in SI Appendix.
% % }
% % \label{fig:kinetics}
% % \end{figure}

% The EPR, nonconservative and driving work contributions vanish at equilibrium according to the second law of thermodynamics but are expected to take finite values in nonequilibrium regimes (Fig.~\ref{fig:thermo}\textit{B}). The nonequilibrium kinetics of the system was assessed for different cell stimulation conditions, which allowed the calculation of the corresponding nonconservative and driving work contributions and, ultimately, of the efficiency of mitochondrial metabolism (SI Appendix).


% - - - - - - - - - - - - - - - - - - - - - - - - - - - - - - - - - - - - - -

% - - - - - - - - - - - - - - - - - - - - - - - - - - - - - - - - - - - - - -

% - - - - - - - - - - - - - - - - - - - - - - - - - - - - - - - - - - - - - -

\section*{Results}
\subsection*{Ca\textsuperscript{2+}-metabolism cross-talk affects the oscillation period and the production of ATP}
To validate the kinetic model, we compared our simulation results to experimental and simulation data from the literature. Slow spiking is found around the bifurcation point corresponding to the transition from steady-state to oscillations, which marks the onset of the signaling machinery. A decrease in the oscillation period is observed as $\left[\mathrm{IP}_3\right]$ is increased (Fig.~\ref{fig:kinetics}\textit{A}-\textit{C}) or as $\left[\mathrm{AcCoA}\right]$ is decreased (Fig.~\ref{fig:kinetics}\textit{C}). These trends are in agreement with stimulation experiments performed in various cell types \cite{woods_repetitive_1986, falcke_reading_2004, dupont_calcium_2007, thurley_reliable_2014, moein_dissecting_2017} and with behaviors reported for limited availability of mitochondrial substrate \cite{jouaville_synchronization_1995, wacquier_interplay_2016, moein_dissecting_2017}.
%While the average $\left[\mathrm{Ca}^{2+}\right]_\mathrm{c}$ seems to be unaffected by the onset of oscillations, a cusp in the average of $\left[\mathrm{ATP}\right]_\mathrm{c}$ is observed at the critical point (Fig.~\ref{fig:kinetics}\textit{D-E}).
In the oscillatory regime, $\left[\mathrm{ATP}\right]_\mathrm{c}$ displays a maximum in dependence on $\left[\mathrm{IP}_3\right]$ and $\left[\mathrm{AcCoA}\right]$, a feature that is also predicted by the model of Wacquier \textit{et al.}~\cite{wacquier_interplay_2016}. In our simulations, a cusp in the average of $\left[\mathrm{ATP}\right]_\mathrm{c}$ is additionally observed at the critical point (Fig.~\ref{fig:kinetics}\textit{D-E}).

Most of these observations can be rationalized based on the dependence of SERCA pumps on ATP\textsubscript{c}, which enables the switch between ER and mitochondrial Ca\textsuperscript{2+} sequestration and is a key signature of the Ca\textsuperscript{2+}-metabolism cross-talk. As $\left[\mathrm{IP}_3\right]$ increases, more Ca\textsuperscript{2+} is released into the cytosol through IP\textsubscript{3}Rs. The steady-state $\left[\mathrm{ATP}\right]_\mathrm{c}$ thus decreases due to a more demanding maintenance of the basal $\left[\mathrm{Ca}^{2+}\right]_\mathrm{c}$ \textit{via} SERCA pumps. At the critical $\left[\mathrm{IP}_3\right]$ corresponding to the onset of oscillations, mitochondrial sequestration of Ca\textsuperscript{2+} becomes significant, which not only relieves SERCA pumps but also enables the activation of Ca\textsuperscript{2+}-sensitive dehydrogenases of the TCA cycle (Fig.~S2). These combined effects result in an increase of the average $\left[\mathrm{ATP}\right]_\mathrm{c}$.
% see bif diagrams of Jserca and Jmcu vs IP3
Increasing $\left[\mathrm{IP_3}\right]$ further leads to saturation in mitochondrial buffering of Ca\textsuperscript{2+} (Fig.~S2). More intense Ca\textsuperscript{2+} sequestration \textit{via} SERCA pumps is then required and the associated ATP\textsubscript{c} consumption is no longer counterbalanced by the Ca\textsuperscript{2+}-enhanced mitochondrial activity, which results in a slow decrease in average $\left[\mathrm{ATP}\right]_\mathrm{c}$.
%The dependence of SERCA pumps on $\left[\mathrm{ATP}\right]_\mathrm{c}$ also explains the switches between ER and mitochondrial sequestrations and is a key signature of the Ca\textsuperscript{2+}-metabolism cross-talk.
Meanwhile, increasing stimulation by IP\textsubscript{3} favors more frequent opening of the IP\textsubscript{3}Rs, which results in an increase of the oscillation period.
In most mathematical models for Ca\textsuperscript{2+} signaling and in agreement with experimental observations, the oscillation period saturates at high $\left[\mathrm{IP_3}\right]$ \cite{eisner_study_1986} and, beyond a critical $\left[\mathrm{IP_3}\right]$, oscillations disappear. The cell then exhibits a high-$\left[\mathrm{Ca^{2+}}\right]_\mathrm{c}$ steady-state \cite{falcke_reading_2004}, as reproduced by our simulations (Fig.~S2). Further stimulation by IP\textsubscript{3} does not affect the steady-state concentrations reached after termination of the oscillations (see Fig.~\ref{fig:thermo}\textit{C} bottom for $\left[\mathrm{ATP}\right]_\mathrm{c}$ and Fig.~S2 for $\left[\mathrm{Ca}^{2+}\right]_\mathrm{c}$ and $\left[\mathrm{Ca}^{2+}\right]_\mathrm{m}$), suggesting that IP\textsubscript{3}Rs have reached their maximal release rate and contribute to saturation effect.

The impact of AcCoA level on $\left[\mathrm{ATP}\right]_\mathrm{c}$ and oscillation period is more visible in starving conditions, \textit{i.e.} $\left[\mathrm{AcCoA}\right] \leq 1~\mathrm{\mu M}$, and for low stimulation by IP\textsubscript{3}, \textit{i.e.} $\left[\mathrm{IP}_3\right] \leq 0.24~\mathrm{\mu M}$. ATP production decreases as $\left[\mathrm{AcCoA}\right]$ is decreased except at the onset of oscillations, where a cusp in $\left[\mathrm{ATP}\right]_\mathrm{c}$ occurs (Figs. \ref{fig:kinetics}\textit{E} and S1\textit{B}). At this critical point, $\left[\mathrm{ATP}\right]_\mathrm{c}$ has become limiting for Ca\textsuperscript{2+} uptake by SERCAs and mitochondrial exchanges take over. This switch allows activation of TCA cycle enzymes by Ca\textsuperscript{2+} and locally rescues ATP production. Mitochondrial exchanges intensify as $\left[\mathrm{AcCoA}\right]$ is further decreased. The larger Ca\textsuperscript{2+} efflux from mitochondria exerts a positive feedback on IP\textsubscript{3}Rs which open more frequently, hence the decrease in oscillation period.

\subsection*{The efficiency of mitochondrial metabolism displays a maximum in the regime of Ca\textsuperscript{2+} spiking}
The nonlinear ATP production (Fig.~\ref{fig:kinetics}D) observed for different stimulation conditions in non-starving cells suggests variations in the output work of mitochondria and, possibly, in the thermodynamic efficiency of their metabolism.
As confirmed computationally, the output nonconservative work ($\bar{w}_\mathrm{r1out}$) displays a minimum (corresponding to maximal \textit{export} of energy from mitochondria) that coincides with the maximal $\left[\mathrm{ATP}\right]_\mathrm{c}$ in the kinetic simulations (Fig.~\ref{fig:thermo}\textit{A} top and \ref{fig:thermo}\textit{C} bottom).
On the other hand, both the non-conservative input work contributions ($\bar{w}_\mathrm{r1in}$ and $\bar{w}_\mathrm{r2}$) increase with $[\mathrm{IP}_3]$ (almost monotonically until the end of the oscillatory regime), while the driving work ($\bar{w}_\mathrm{driv}$) is always negligible compared to the total dissipation (Fig.~\ref{fig:thermo}\textit{A}).

%On the other hand, increasing $[\mathrm{IP}_3]$ increases {\color{orange}(almost monotonically until the end of the oscillatory regime)} both the non-conservative input work contributions ($\bar{w}_\mathrm{r1in}$ and $\bar{w}_\mathrm{r2}$), while the driving work ($\bar{w}_\mathrm{driv}$) is always negligible compared to the total dissipation (Fig.~\ref{fig:thermo}\textit{A}).

\begin{figure*}[t!]
\centering
\includegraphics[width=\linewidth]{Fig3_20230124}
\caption{Stimulating the Ca\textsuperscript{2+} signaling machinery impacts the dissipation and efficiency of mitochondrial metabolism \textit{via} the Ca\textsuperscript{2+}-metabolism cross-talk. (\textit{A})~Nonconservative work contributions, driving work and dissipation for different $\left[\mathrm{IP_3}\right]$. The driving work represents less than 0.01\% of the EPR. At high stimulation, oscillations disappear in the favor of a nonequilibrium steady-state regime. (\textit{B})~Expected work contributions in out-of-equilibrium conditions. (\textit{C})~Efficiency and ATP\textsubscript{c} concentration as a function of $\left[\mathrm{IP_3}\right]$. (\textit{D)}~Efficiency as a function of the total dissipation for a range of $\left[\mathrm{IP_3}\right]$ extended up to $5\, \mu\mathrm{M}$. (\textit{E}-\textit{F})~Plots corresponding to (\textit{C}-\textit{D})~for $V_{max}^\mathrm{SERCA} = 0.08\, \mathrm{\mu M\, s^{-1}}$. Empty and filled dots correspond to steady-state or period-averaged quantities, respectively. Unless specified otherwise, parameter values are the same as in Fig.~\ref{fig:kinetics}\textit{D}.}
\label{fig:thermo}
\end{figure*}

Importantly, the maximum in $\left[\mathrm{ATP}\right]_\mathrm{c}$ translates into a maximum in the efficiency of mitochondrial metabolism (Fig.~\ref{fig:thermo}\textit{C}). Such maxima are not systematically observed when $\left[\mathrm{AcCoA}\right]$ is varied at fixed $\left[\mathrm{IP_3}\right]$ (Fig. S1\textit{A-B} in SI Appendix). However, the increase in efficiency at the onset of the oscillatory regime is a robust feature that points to the stabilising effect of Ca\textsuperscript{2+} spikes on mitochondrial energetics.


\subsection*{Efficiency and total dissipation saturate at high stimulation of the Ca\textsuperscript{2+} signaling machinery}
Like in other biological processes such as the migration of molecular motors along microtubules, kinetic proofreading or the regulation of circadian clocks \cite{baiesi_life_2018}, the system's efficiency is maximal at intermediate levels of dissipation, corresponding to a limited range of $\left[\mathrm{IP_3}\right]$ (Fig.~\ref{fig:thermo}\textit{D}). This finding indicates that overstimulation of the signaling machinery is counterproductive since it only increases dissipation (Fig.~\ref{fig:thermo}\textit{A} bottom). By exploring the behavior of efficiency and total EPR at large $\left[\mathrm{IP_3}\right]$, we observed a saturation effect leading to limiting values for the efficiency ($\approx 0.236$) and total dissipation ($\approx 5680\, \mathrm{JL^{-1}s^{-1}}$).

Interestingly, the dependency of efficiency on the total dissipation is highly nonlinear (Fig. \ref{fig:thermo}\textit{D}). Around the onset of oscillations (and for a limited range of dissipation rates), a given dissipation rate can be associated to different efficiencies, in which case the highest efficiency is always reached for the highest Ca\textsuperscript{2+} spiking frequency, while the lowest efficiency corresponds to the steady-state regime. Reversely, different dissipative regimes can yield the same efficiency. In that case, steady-state regimes display the lowest EPR while the fast-spiking regimes are the more dissipative regimes. We hypothesize that in such instances, the selection of the dissipative regime could be guided by constraints imposed by the global energy budget of the cell.

\subsection*{Robustness of the efficiency-rescuing effect of Ca\textsuperscript{2+} oscillations}
Our proposed mechanism for the maximum in efficiency relies on the dependence of the SERCA flux ($J_\mathrm{SERCA}$) on the hydrolysis of ATP\textsubscript{c} and on the resulting modulation of Ca\textsuperscript{2+} sequestration mechanisms. If Ca\textsuperscript{2+} homeostasis was ATP-independent, Ca\textsuperscript{2+} would always exert a positive feedback on the TCA cycle flux and the efficiency of metabolism would increase monotonically with the Ca\textsuperscript{2+} release from IP\textsubscript{3}Rs. We validated this hypothesis by performing simulations with a modified SERCA flux that is uncoupled from ATP\textsubscript{c} hydrolysis. The degradation of ATP\textsubscript{c} then relies exclusively on other ATP-consuming processes mimicking cellular activity (Hyd reaction in Fig.~\ref{fig:fluxes}\textit{A}). As expected, the Ca\textsuperscript{2+}-enhanced ATP production by mitochondria is not restrained upon more intense stimulation by IP\textsubscript{3} (Fig.~\ref{fig:uncoupled}\textit{A}).
%This uncoupling also disables the feedback of ATP production by mitochondria on Ca\textsuperscript{2+} signaling, responsible for the decrease in spike period observed as the access to mitochondrial substrate becomes limited.
This uncoupling also disables the feedback of ATP production by mitochondria on Ca\textsuperscript{2+} oscillations: instead of decreasing, the spike period is barely changed as $\left[\mathrm{AcCoA}\right]$ decreases (Fig.~\ref{fig:uncoupled}\textit{B}).

\begin{figure}[t!]
\centering
\includegraphics[width=\linewidth]{Fig4_20230124}
\caption{Efficiency of mitochondrial metabolism regulated by Ca\textsuperscript{2+} signaling, Ca\textsuperscript{2+} sequestration fluxes and Ca\textsuperscript{2+}-dependent TCA cycle reaction fluxes, without and with the coupling of SERCA pumps to ATP\textsubscript{c} hydrolysis. (\textit{A}) Stimulating Ca\textsuperscript{2+} release by IP\textsubscript{3}Rs (that is, increasing $\left[\mathrm{IP_3}\right]$) monotonically increases the efficiency in the uncoupled case, which strongly contrasts with the nonmonotonic dependency on $\left[\mathrm{IP_3}\right]$ in the coupled case. While varying over different ranges, the period evolves according to the same trends in both cases. (\textit{B}) Both systems display similar responses in their efficiency upon variations in $\left[\mathrm{AcCoA}\right]$, but the oscillation period of the uncoupled system does not decrease -- and is even slightly increasing -- in stressing conditions corresponding to substrate depletion. As this observation is in contradiction with experimental evidence, the larger efficiencies reached in the uncoupled case are non physiological. Empty and filled dots correspond respectively to steady-state and oscillatory regimes -- note the use of linear colorbar schemes for the period. (\textit{C}) Phase portraits of Ca\textsuperscript{2+}-dependent TCA cycle currents (purple) and mitochondrial Ca\textsuperscript{2+} uptake (green), namely $J_\mathrm{IDH}$, $J_\mathrm{KGDH}$ and $J_\mathrm{UNI}$, \textit{vs.} ER Ca\textsuperscript{2+} uptake, namely $J_\mathrm{SERCA}$. Note that $J_\mathrm{IDH}$ and $J_\mathrm{KGDH}$ are indistinguishable. Symbols denote steady-state values. Triangles ($\left[\mathrm{IP_3}\right]=0.34\,\mathrm{\mu M}$) and dotted curves ($\left[\mathrm{IP_3}\right]=0.42\,\mathrm{\mu M}$) correspond to the uncoupled case, while circles ($\left[\mathrm{IP_3}\right]=0.10\,\mathrm{\mu M}$) and solid curves ($\left[\mathrm{IP_3}\right]=0.24\,\mathrm{\mu M}$) represent the coupled case. (\textit{A} and \textit{C}) $\left[\mathrm{AcCoA}\right]=1\,\mathrm{\mu M}$, (\textit{B}) $\left[\mathrm{IP_3}\right]=0.36\,\mathrm{\mu M}$ and the other parameter values are the same as in Figs.~\ref{fig:kinetics}--\ref{fig:thermo}.}
\label{fig:uncoupled}
\end{figure}

In the uncoupled case, $J_\mathrm{SERCA}$ is not limited by the depletion of ATP\textsubscript{c} and both removal mechanisms proceed synchronously, although Ca\textsuperscript{2+} uptake to the ER is predominant (Fig. \ref{fig:uncoupled}\textit{C}, green dotted curve). While the mitochondrial Ca\textsuperscript{2+} influx $J_\mathrm{UNI}$ slightly increases with $J_\mathrm{SERCA}$, the Ca\textsuperscript{2+}-dependent currents of the TCA cycle, $J_\mathrm{IDH}$ and $J_\mathrm{KGDH}$ (Fig. \ref{fig:uncoupled}\textit{C}, purple dotted curves), are barely affected. Upon regulation by ATP\textsubscript{c}, $J_\mathrm{SERCA}$ varies over a more restricted range, is on average smaller and proceeds with a slight phase shift with respect to $J_\mathrm{UNI}$, which allows for a larger Ca\textsuperscript{2+} influx in mitochondria and a more intense activation of the TCA cycle enzymes (Fig. \ref{fig:uncoupled}\textit{C} solid curves).
%compared to the steady-state obtained for subthreshold $\left[\mathrm{IP_3}\right]$.
This mechanism supports the ``efficiency-rescuing'' role of mitochondrial buffering that is visible near the onset of oscillations in the coupled case (Figs.~\ref{fig:thermo}\textit{C} and \ref{fig:uncoupled}\textit{A} bottom), while no such cusp-like transition is observed in the uncoupled case (Fig.~\ref{fig:uncoupled}\textit{A} top).

We also explored the robustness of our results against perturbations in the uptake rate of SERCA pumps of the original model. We mimicked the inhibition of SERCA pumps by decreasing the limiting rate $V_{max}^\mathrm{SERCA}$. Interestingly, the efficiency-dissipation relation displays the same features as in the non-inhibited case (Fig.~\ref{fig:thermo}\textit{F} \textit{vs} Fig.~\ref{fig:thermo}\textit{D}). In the extreme case where oscillations disappear for large stimulation by IP\textsubscript{3}, the efficiency drops and reaches a plateau (Fig.~\ref{fig:thermo}\textit{E} top).

Together, these results confirm that the cross-talk between Ca\textsuperscript{2+} signaling and mitochondrial energy metabolism is a major mechanism underlying the maximum in efficiency arising in the spiking regime, even when the amplitude of this coupling is reduced due to the inhibition of SERCA pumps.

\section*{Discussion}
Here, we examined the impact of Ca\textsuperscript{2+} signaling on the efficiency of mitochondrial metabolism by using tools from the nonequilibrium thermodynamics of CRNs on a detailed and experimentally validated kinetic model of the Ca\textsuperscript{2+}-metabolism cross-talk. Our results highlight that, despite a usually higher dissipation rate compared to steady-state regimes, Ca\textsuperscript{2+} oscillations can enhance the efficiency of mitochondrial metabolism. In particular, stimulation by IP\textsubscript{3} reduces the steady-state efficiency of metabolism but at the onset of oscillations
% , for a typical IP\textsubscript{3} concentration range where signalling is triggered to initiate a physiological response from the cell, 
the efficiency raises with a cusp-like transition and reaches a maximum of about 30\% before decreasing again at higher stimulation. This value corresponds to the efficiency of the TCA cycle estimated in the absence of regulation with a non-equilibrium thermodynamic approach \cite{wachtel_free-energy_2022}. Moreover, slow-spiking is less dissipative than fast-spiking. Thus, we hypothesize that, for a given cell state, there exists an optimal stimulation level leading to slow-spiking/low-dissipation oscillations which maximize the efficiency of metabolism during signaling. For higher stimulation, the Ca\textsuperscript{2+} signaling machinery then generates more dissipative regimes of gradually decreasing efficiency.

In the broader context of physical bioenergetics, energetic costs are usually assessed by evaluating the Gibbs free energy of reaction ($\Delta_r G$) dissipated or the equivalent number of ATP molecules produced/consumed along the processes of interest \cite{cao_free-energy_2015, rodenfels_heat_2019, yang_physical_2021}. However, such purely thermodynamic approaches do not account for reaction kinetics and thus cannot quantify the rates of free energy transduction and dissipation. Significant efforts have been made in the direction of adding thermodynamic constraints in flux balance analysis of metabolic networks \cite{beard_energy_2002, niebel_upper_2019}. A few attempts have also been made to account for more complex kinetic effects such as enzyme saturation, leading to insights into the trade-offs between energy production and enzyme costs in glycolysis \cite{noor_note_2013, flamholz_glycolytic_2013, stettner_cost_2013}. Nevertheless, all these approaches rely on optimized nonequilibrium steady-states, which may not correspond to physiological conditions and cannot capture the energetic impact of time-dependent behaviors, such as the energy-rescuing effect of Ca\textsuperscript{2+} oscillations quantified here. Our novel approach overcomes these limitations, based on the rigorous thermodynamic analysis of a curated dynamical model.

Finally, a key mechanism of the Ca\textsuperscript{2+}-mitochondria cross-talk and the metabolic efficiency management is the dynamical switch between SERCA and mitochondrial uptakes. Alterations in Ca\textsuperscript{2+} removal mechanisms due to mutations, generation of reactive oxygen species or remodeling of channel and pump expression are ubiquitous in pathological states such as mitochondrial \cite{visch_ca2-mobilizing_2006} and neurodegenerative diseases \cite{celsi_mitochondria_2009, filadi_mitochondrial_2020}, cancer \cite{monteith_calciumcancer_2017} or diabetes \cite{guerrero-hernandez_calcium_2014}, and therapeutic strategies targeting Ca\textsuperscript{2+} homeostasis and signaling have started to emerge \cite{giorgi_mitochondrial_2012, dejos_two-way_2020}. Overall, our methodology thus paves the way for a more systematic characterization of the dynamical energetic impact of metabolism regulation, which could improve the current understanding of pathway selection mechanisms in health and disease.

\matmethods{Model development, methodology and calculations are detailed in the SI Appendix, which is organized as follows: (1) Description of the kinetic model, (2) Concepts of biothermodynamics and (3) Nonequilibrium thermodynamics analysis. A systematic description of the nonequilibrium thermodynamics of CRNs can be found in refs. \cite{rao_conservation_2018, avanzini_nonequilibrium_2021}. This description can be applied to effective coarse-grained fluxes \cite{wachtel_thermodynamically_2018, avanzini_thermodynamics_2020} such as the ones we used to characterize the enzyme-catalysed processes of our system.
%Fig.~S1 illustrates the behavior of the mitochondrial efficiency for an extended range of AcCoA and IP\textsubscript{3} concentrations.
The chemical reactions incorporated in the model are listed in Table S1 and the corresponding fluxes and forces are given in Tables S2 and S3, respectively. Lastly, the reference simulation parameters and a short description of their physical meaning can be found in Table S4.}

% Outline of the SI Appendix

\showmatmethods{} % Display the Materials and Methods section

\acknow{
%Please include your acknowledgments here, set in a single paragraph. Please do not include any acknowledgments in the Supporting Information, or anywhere else in the manuscript.
VV is funded by the Complex Living Systems Initiative at the University of Luxembourg. FA and ME are funded by the Luxembourg National Research Fund, grant ChemComplex (C21/MS/16356329). GF is funded by the European Union -- NextGenerationEU -- and by the program STARS@UNIPD with project "ThermoComplex". FA, AS and ME acknowledge financial support of the Institute for Advanced Studies of the University of Luxembourg through an Audacity Grant (IDAE-2020).}

\showacknow{} % Display the acknowledgments section


% Bibliography
\bibliography{ThermoMitoCa}

\end{document}
