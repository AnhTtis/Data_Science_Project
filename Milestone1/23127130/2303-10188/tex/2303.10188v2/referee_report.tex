Authors (hereafter, A):
--------
We thank the anonymous referee for their careful reading of the manuscript.
We implemented the changes suggested by the referee, added a few references and corrected some typos in the text.
We highlighted all the changes in the revised manuscript with boldface.

In the following, we reply to the report. 

######################################################################

Reviewer's (hereafter, R)  Comments:

R: This paper presents a very interesting comparison  of the results of a set of N-body simulations with some observed properties of the Hyades cluster. The goal is to identify possible signatures of the presence of black holes (BHs) in the Hyades cluster and establish a connection between the possible presence of BHs and the cluster's structure.

The study is very detailed and well presented, and it is deserving of publication on MNRAS.

I have only a few minor comments and suggestions I would like the authors to address before I can recommend the paper for publication.


###########


R: Section 1
'...new objects have been discovered (e.g.....)'
Here the following references could be added
Hunt & Reffert 2023
https://arxiv.org/abs/2303.13424
Chi et al. 2023
https://arxiv.org/abs/2303.10380

A: We added the references as suggested by the referee.

###########

Fig.1

There is no dotted line in the top left panel of Fig.1 (the panel for 0 BHs).

A: In that panel the two lines are not distinguishable because they overlap. We now specify it in the caption.


###########

Section 2.2.1

R: Could the authors add information about whether the initial Plummer model has been truncated and what the initial tidal radius (rt) and rh/rt are for the different models explored?

A: The initial Plummer models were truncated at their tidal radius, we now specify it in the text. The typical initial tidal radius is 12 pc, while the ratio rh/rt spans from 0.03 to 0.18. We added this information in the text.  


###########

Sections 3.2, 3.3, Figures 3 and 4

R: In these sections, the authors discuss the results presented in Figures 3 and 4 and conclude that models with a few BHs are favored over models with 0 BHs because, for example, 'the distribution of rhm of the NBH=0 sample is mostly inconsistent with the observed value of the Hyades cluster'

The histograms shown in Figure 3 and 4 are, however, are a consequence of the range of initial parameters explored and differences in the survival of models with and without BHs.
Consideration of different range of initial values would lead to different distributions of rhm and of rhm,h/rhm,l and possibly to a different conclusion concerning the consistency of models with 0 BHs with observations.

Since all models (at least those ending with NBH<=3) produce a number of systems consistent with observations, it would be interesting to include figures showing the range of initial conditions corresponding to models evolving to properties  consistent with observations and with NBH=0, NBH=1, NBH=2, and NBH=3.

A: We thank the referee for the insightful comment. We have introduced a new figure (Fig. 9) and a new section (Sect. 4.1), where we explain that we can not use the final distribution as posteriors, since our sampling for the initial parameters was done on a rigid grid with fixed number of models at each grid point. Then, we discuss the dependence of our results on the initial parameters. 

In Fig. 9, we show the percentage distributions of models that match the observations, for each value of N_BH<3, and as a function of M_0 and r_hm,0. We define these models as those that lie within our mass cut (150<Mass/MSun<190) and that have a half-mass radius close (within 20\%) to the observed value. 

Most of the models with NBH < 3 lie well within the initial mass range, with lower percentages at the low-and at the high-mass end. No clear dependence on the initial radius is found. In contrast, star clusters with 3 BHs mainly result from initial masses and  radii at the upper boundary of the parameter distributions. This is mainly due to the larger number of massive progenitors, which enhance the cluster mass loss. At the same time, models with larger radii retain more BHs (fewer dynamical interactions) and they therefore need to be more massive.

Our analysis suggests that more massive and extended initial conditions may produce Hyades-like clusters, but they are expected to result in models with N_BH >= 3. Thus, a more extensive exploration of the initial parameter space is expected to strengthen the conclusion that a fraction of BHs needs to be retained within the cluster to match the observed properties of the Hyades. In contrast, there is no hint at the fact that, by extending the range of initial conditions, we will find different conclusions on the consistency of models with 0 BHs and observations.


###########

Figure 5

R: The caption mentions a 'red filled area'. There is no red area in my version of the paper.

A: We corrected the typo. Now we indicate it as 'green filled area'.

###########

Figure 5

R: This figure shows that only models with recent ejection of BHs lead to values of rh consistent with observations. Could the authors briefly comment on whether considering initial conditions with slightly larger initial values of rh could also lead to present-day values of rh consistent with observations, no BHs, and without requiring recent BH ejection?

A: As discussed in the new section (Sect. 4.1) and in the previous point of the report, our analysis suggests that, by extending the range of initial conditions, we do not expect t find different conclusions on the consistency of the models with 0 BHs with observations. We added a reference to the discussion in Sect. 4.1.

We also attach to this report a figure that shows three sub-distributions of rhm, for the models with tBH=0 Myr in Fig. 5. The three distributions, corresponding to different initial rhm, are very similar to each other, indicating that models with 0 BHs end up too
small, independently on their initial radius. If we consider clusters with larger initial radii and no retained BHs, we expect that they shrink, as also found for the case of Palomar 5 (Gieles et al. 2021). 



###########

Section 3.6

R: '...so-called potential escapers (Fukushige & Heggie 2000), which are
particularly numerous in low-mass clusters (Baumgardt 2001). '

I would suggest to slightly revise this sentence.
What the simulations show is that the fraction of potential escapers increases as the fraction of the initial number of stars remaining in a cluster decreases. A cluster might have a low mass but close to its initial mass and in that case the expected fraction of potential escapers would be a small.

I would recommend to clarify the connection between the estimate of 30 per cent potential escapers in the Hyades and the results of Claydon et al. 2017,2019.
Maybe the authors could specifically mention the typical values of Nc/N0 in their simulations and the results shown in Fig.1 of Claydon et al. 2017.

We also repeated the calculation of the dynamical mass, without including the potential escapers. In this case, we find values that are consistent with the actual total mass.

Also, what is the fraction of potential escapers in the initial conditions used for this study?

A: We revised the sentence as suggested by the referee. 

In our N-body models, the fraction of the initial stars retained within the cluster is $\sim 0.2$ for those models that lie within the Hyades mass range.
For these models, the fraction of potential escapers is generally $\sim 40\%$. This is consistent with Claydon et al. (2017), where models initialized with a Kroupa (2001) IMF (from 0.1 to 1 MSun) that evolve in a Navarro-Frenk-White (1995) potential (the same
as for the dark matter halo in the MWPotential2014) display similar fraction of potential escapers. 
In our models, the initial fraction of potential escapers is $\lesssim 5\%$. 
We added this information in the text.

Finally, we repeated the calculation of the dynamical mass without including potential escapers. In this case, we find mass values consistent with the actual total mass. We added this information in the text.

###########

Section 4.3.3

R: 'it is interesting to interrogate the sources'
Not clear what the authors mean by 'interrogate the sources'

A: We replaced the sentence with "As we are searching for significant-mass BHs, we focus on the sources..."