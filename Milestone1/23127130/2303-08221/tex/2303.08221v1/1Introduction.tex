
\section{Introduction}
\label{sec:introduction}

% Currently deployed electronic payment systems do not protect user privacy. Data such as the user's identity and the date and location of the payment are leaked. From such data, privacy-sensitive information like the user's whereabouts, her health status, or her religious or political affiliations can be inferred.

At the present moment, there is a pressing need for private electronic cash (e-cash), as shown by growing interest in blockchain-based systems and Centrally-Banked Digital Currencies (CBDCs), but no privacy-enhanced and decentralized system exists that scales to the requirements of this application scenario. To address this problem, we introduce the first decentralized offline e-cash scheme with provable security and full implementation, that can efficiently support electronic payments. In contrast to the current privacy-enhanced blockchain, systems do not require a constant online presence of the payees. The issuance of coins is non-interactive, i.e., the authorities do not need to synchronise, as we do not rely on MPC protocols.
Moreover, our scheme maps to distributed payment systems such as CBDCs, a technology that urgently requires further attention in terms of privacy.
The idea of distributing issuance among a quorum of authorities has been so far explored in the context 
of attribute-based credentials~\cite{DBLP:conf/ndss/SonninoABMD19, cryptoeprint:2022:011} and online anonymous e-cash schemes~\cite{DBLP:journals/iacr/BaudetSKD22}.
\newline\noindent \textit{\textbf{Contributions:}}
Our paper makes the following contributions:
\begin{itemize}[leftmargin=*, noitemsep,topsep=0pt]
    \item We introduce a construction $\mathrm{\Pi}_{\CEC}$ for threshold issuance offline e-cash ($\CEC$). To the best of our knowledge, this is the first offline e-cash scheme with threshold issuance. To this end, we define the system model and the security properties in the ideal-world/real-world paradigm~\cite{DBLP:conf/focs/Canetti01} by proposing an ideal functionality $\Functionality_{\CEC}$ for $\CEC$ (Sections \S\ref{sec:offlineecash}, \S\ref{sec:constructionCEC}).
    \item We propose two instantiations of  $\mathrm{\Pi}_{\CEC}$ based on compact~\cite{DBLP:conf/eurocrypt/CamenischHL05} and divisible~\cite{DBLP:conf/pkc/PointchevalST17} e-cash (Section \S\ref{sec:instantiation}). Our schemes are more efficient than~\cite{DBLP:conf/eurocrypt/CamenischHL05, DBLP:conf/pkc/PointchevalST17} thanks to the use of Pointcheval-Sanders signatures in the random oracle model and to decreasing the number of coin secrets. 
    We formally prove that our construction $\mathrm{\Pi}_{\CEC}$ realizes $\Functionality_{\CEC}$ when instantiated with the algorithms of our compact and divisible e-cash schemes (Sections \S\ref{sec:securityProofCompact},  \S\ref{sec:securityProofDivisible}). 
    \item We provide an open-source Rust implementation of both schemes and present an extensive evaluation of their performance and trade-offs (Section \S\ref{sec:efficiencycomparison}). To the best of our knowledge, this is first such practical comparison.
    % Regarding the spending phase, the computation cost of the divisible $\CEC$ scheme is $125$ ms, whereas the cost of our compact $\CEC$ scheme ranges from $51$ ms to $468$ ms depending on the size of the price range and the set of denominations considered.
    \item We conclude by outlining how our schemes can be integrated with a blockchain-based bulletin board (Section \S\ref{sec:integration}). This would allow our scheme to fulfil the requirements for a distributed privacy-enhanced CBDC~\cite{digitaleuro} and even provide scalability via offline transactions for existing blockchain systems like ZCash~\cite{zcash}. 
\end{itemize}

\section{Background and Motivation}
\label{background}
Anonymous e-cash was originally proposed by Chaum as a digital analog of regular cash, which also allows for private payments~\cite{DBLP:conf/crypto/Chaum82}. 
% Anonymous e-cash systems guarantee that no coalition of dishonest users, providers (i.e. merchants), and a bank can link the spending of a coin with its withdrawal or link multiple spendings performed by the same user. Moreover, no coalition of dishonest users and providers is able to deposit more coins than have been withdrawn. 
Unlike physical cash, e-coins are easy to duplicate, hence e-cash schemes must prevent double-spending and typically that is done by having a centralized bank~\cite{DBLP:conf/eurocrypt/CamenischHL05,DBLP:conf/pairing/BelenkiyCKL09,DBLP:conf/eurocrypt/CanardG07,DBLP:conf/acns/CanardPST15,DBLP:conf/pkc/CanardPST15,DBLP:conf/pkc/PointchevalST17,DBLP:conf/asiacrypt/BoursePS19,DBLP:conf/crypto/OkamotoO89,DBLP:conf/crypto/OkamotoO91,DBLP:conf/pkc/BaldimtsiCFK15,DBLP:conf/pkc/BauerFQ21}. Thus, there has been a revival of interest in adopting privacy to decentralized blockchain systems, in which coins are authenticated by proving in ZK that they belong to a public list of valid coins maintained on the blockchain, thus they do not require a central bank to prevent double-spending~\cite{DBLP:conf/sp/MiersG0R13}. 

Although blockchain-based privacy-enhanced systems such as ZCash and Monero have users~\cite{zcash, cryptonote,DBLP:journals/ledger/NoetherM16}, such 
decentralized e-cash systems require being online to check the status of payments, which simply does not scale to the speed of transactions needed by real-world payment systems or support the reality of transactions that need to be made without internet access, so the usage of blockchains for payments remains small in proportion to traditional payments. Also, as could happen to any other low-transaction fee blockchain that advertises high throughput, the ZCash blockchain has suffered an  attack of \emph{`spam'} transactions that have increased the blockchain size to such an extent that ZCash has suffered from what is effectively a \emph{denial-of-service} attack that has collapsed its throughput (i.e., simply downloading the blockchain becomes nearly impossible)~\cite{zcashattack1}. Similarly, `layer 2' solutions based on blockchain technologies such as zero-knowledge roll-ups, which increase transaction speed and (in some cases) privacy, are also vulnerable to these attacks~\cite{zamyatin2021sok}.  By virtue of not requiring a merchant to be online all the time but only needing eventual settlement over regular epochs, our system avoids these issues while maintaining the advantages of decentralization. Hence, it is to be expected that current privacy-enhanced blockchain systems that require an online setting will evolve into decentralized offline e-cash systems similar to the one presented in this paper. While proposals exist to enable offline payments in blockchain-based cryptocurrencies
such as payment channel networks~\cite{DBLP:conf/ccs/0001M17}, 
they typically do not offer strong privacy protection as only users who share a channel can transact with each other and so users who make a payment are not anonymous, similarly as in~\cite{10.1145/3052973.3052980}, and cashing out payments still requires online blockchain interactions. Even privacy-preserving offline payment channels effectively restrict payment transactions and the network topology of payment channels can lead participants to be effectively de-anoymized~\cite{ DBLP:conf/fc/KapposYPKDMM21, sharma2022anonymity}. 
%Therefore, offline anonymous e-cash schemes with threshold issuance offer
%an interesting alternative. 
% Recent market research on CBD released by the ECB further shows increasing interest in the offline approach to electronic payments~\cite{digitaleuro}. 

On the other side of the spectrum, centralized CBDCs have gained increasing interest in over a hundred countries and will soon be deployed in Europe and China~\cite{digitaleuro, xu2022developments}. CBDC systems typically sacrifice user transaction privacy from the settlement layer and can lead to the possibility of dystopian surveillance of user financial transactions~\cite{danezis2015centrally}. Although financial transaction data could be considered a matter of national security, MIT and the Federal Reserve of Massachusetts in the United States have begun exploring blockchain systems for its CBDC efforts without transaction unlinkability~\cite{lovejoy2022high}. In stark contrast, Switzerland's Central Bank  has put forward a centralized online privacy-preserving scheme called `eCash 2.0', but offline payments would require special hardware and are currently not supported. The European Central Bank (ECB) recently published a list of requirements for the Digital Euro, which include privacy of user transactions from the settlement layer and offline transactions~\cite{digitaleuro}. Our scheme would fulfil those criteria, and we present how it could be integrated with a blockchain.
% , which the ECB seems to think will require special hardware, i.e. “a card-based solution should be supported”~\cite{digitaleuro}. 
Furthermore, distributing the issuing power would be practical for emerging multi-nation economic proposals for joint issuance of currencies such as the South American joint reserve currency SUR in the `Banco del Sur' recently endorsed by Brazil and Argentina~\cite{marshall2009financing}. Even more importantly, practically preventing byzantine faults in centralized CBDCs requires it to be managed by a set of distributed parties, similar to Facebook's Diem project's consensus protocol based on HotStuff~\cite{yin2019hotstuff}. In this manner, our system unifies the objectives of blockchain research for decentralization while maintaining compatibility with the requirements for privacy-preserving CBDC by decentralizing e-cash.

% A user withdraws a wallet with a number of coins from the bank. 
% A typical anonymous e-cash scheme consists of three types of entities: the \emph{bank}, \emph{users} and \emph{providers}. The interaction between those parties takes place through a \emph{withdrawal}, \emph{spend} and \emph{deposit} phase.  In the withdrawal phase, the user obtains a wallet with a number of coins from the bank.
% At a later point, the user spends one or several coins from her wallet with a provider, who then deposits the received payments at the bank in exchange for funds. 

% , i.e. it must be possible to detect whether a coin is spent more than once.
% To enable double-spending detection,
\vspace{-5pt}
\paragraph{Online vs Offline Ecash}
To revisit the original e-cash proposal, the solution to double-spending is to associate each coin with a unique \emph{serial number}, which is used to detect double-spending by dishonest users and prevent dishonest providers from depositing a payment twice~\cite{DBLP:conf/crypto/Chaum82}.
% which is revealed after the coin is spent and used in the deposit phase, where the bank compares the coin's serial number with the serial numbers of previously spent coins, and accepts the coin only if there is no match.
% This detects double-spending by dishonest users, and also prevents dishonest providers from depositing a payment twice.
% The serial number is hidden from the bank during the withdrawal phase, but it is revealed after the coin is spent and can be used in the deposit phase, where the bank compares the coin's serial number with the serial numbers of previously spent coins, and accepts the coin only if there is no match. This detects double-spending by dishonest users, and also prevents dishonest providers from depositing a payment twice.
In \emph{online e-cash} schemes~\cite{DBLP:conf/crypto/Chaum82,DBLP:conf/ndss/SonninoABMD19}, providers are constantly connected to the bank
and can then check if a coin has been double-spent before accepting a payment. An alternative and more realistic solution space is given by \emph{offline e-cash} schemes~\cite{DBLP:conf/eurocrypt/CamenischHL05,DBLP:conf/pairing/BelenkiyCKL09,DBLP:conf/eurocrypt/CanardG07,DBLP:conf/acns/CanardPST15,DBLP:conf/pkc/CanardPST15,DBLP:conf/pkc/PointchevalST17,DBLP:conf/asiacrypt/BoursePS19}, which do not require a permanent online connection between a provider and the bank.
% , and instead allow for double spenders identification. 
The provider can accept payment and deposit it at a later settlement stage as there is a guarantee that users who double-spend  will be identified by the bank.

An important issue in the design of anonymous e-cash schemes is paying the exact amount as users cannot receive change,
since the providers are not anonymous towards the bank. If a user receives change, the user would in fact become a provider and lose anonymity. 
% Therefore, anonymous e-cash schemes need to allow users to pay the exact amount.
% Consider a user that possesses a coin with denomination $\$100$ and wants to make a payment of $\$75$.
In online e-cash, the user can contact the bank in order to exchange a coin for lower denominations in order to make the payment. In offline e-cash, contacting the bank is not allowed during the spending phase.
% (Otherwise, the scheme would be online.)
\emph{Transferable e-cash} schemes~\cite{DBLP:conf/crypto/OkamotoO89,DBLP:conf/eurocrypt/ChaumP92,DBLP:conf/pkc/BaldimtsiCFK15} are one solution to this problem,
allowing a user to further spent a previously received coin without interacting with the bank. Hence, providers can return the change to the users that paid.
% a provider that receives a payment that exceeds the price to be paid is able to give change to the user that payed.
% First, \emph{anonymous transferable e-cash}~\cite{DBLP:conf/crypto/OkamotoO89,DBLP:conf/eurocrypt/ChaumP92,DBLP:conf/pkc/BaldimtsiCFK15} allows a user that receives a coin to spend it again, i.e.\ the user does not have to deposit it. Thanks to that, a provider that receives a payment that exceeds the price to be paid is able to give change to the user that payed.
%Note that, in non-transferable e-cash schemes, this is not possible, because the user that receives change would have to deposit it, and then her anonymity would be lost.
Although transferable e-cash is appealing, state-of-the-art schemes~\cite{DBLP:conf/pkc/BauerFQ21} are much less efficient than non-transferable ones.

An alternative solution to preserve user unlinkability is to use coins of the smallest denomination. However, the large number of coins that may need to be spent easily yields an inefficient scheme. To solve this problem, researchers have focused on designing offline e-cash protocols whose complexity does not depend on the number of coins withdrawn or spent.
% This way, it is guaranteed that the user will be able to pay the exact amount.
% The problem is the large number of coins that may need to be spent, which could yield an inefficient scheme. In our example, paying the amount of $\$75$ would involve spending $7500$ coins. To solve this problem, researchers have focused on designing offline e-cash protocols whose complexity does not depend on the number of coins withdrawn or spent.
In \emph{anonymous compact e-cash} schemes~\cite{DBLP:conf/eurocrypt/CamenischHL05,DBLP:conf/pairing/BelenkiyCKL09}, the cost of storing a wallet of $N$ coins and the cost of withdrawing $N$ coins is independent of $N$. However, the cost of spending $n \leq N$ coins grows linearly with $n$.
\emph{Anonymous divisible e-cash} schemes~\cite{DBLP:conf/crypto/OkamotoO91,DBLP:conf/eurocrypt/CanardG07,DBLP:conf/acns/CanardPST15,DBLP:conf/pkc/CanardPST15,DBLP:conf/pkc/PointchevalST17,DBLP:conf/asiacrypt/BoursePS19} improve the efficiency of compact e-cash and allow the user to spend $n \leq N$ coins with cost independent of $n$. Therefore, in the last decade, research has focused solely on divisible e-cash. However, divisible schemes achieve constant spending cost at the expense of much more expensive deposit and identification phases, and to our knowledge, efficiency analysis of compact e-cash schemes with multiple denominations has not been conducted, as well as a fair comparison between practical implementations of divisible and compact e-cash has never been done. Therefore, it is unclear which scheme is better for the use-case of offline e-cash as required by privacy-enhanced CBDCs~\cite{digitaleuro} and blockchain-based scalability. 
Our analysis shows that compact e-cash with multiple denominations is preferable for small payments, which would naturally compose the majority of offline e-cash transactions in application scenarios such as a user-facing blockchain or CBDC where practical deployment concerns would necessitate distributed authorities.
% Although constructions in~\cite{DBLP:conf/eurocrypt/CanardG07,DBLP:conf/pkc/CanardPST15, DBLP:conf/acns/CanardPST15} allow to spend
% only a number of coins given by intervals of the form $[1+j2^k, \allowbreak (j+1)2^k] \allowbreak \in \allowbreak N$, the scheme proposed in~\cite{DBLP:conf/pkc/PointchevalST17} allows the user to spend any number of coins for any interval $[a,b] \allowbreak \in \allowbreak N$ was proposed.
% have the drawback that they do not allow to spend any number of coins $n \leq N$ with cost independent of $n$, but only a number of coins given by intervals of the form $[1+j2^k, \allowbreak (j+1)2^k] \allowbreak \in \allowbreak N$. In~\cite{DBLP:conf/pkc/PointchevalST17}, an anonymous divisible e-cash scheme that allows the user to spend any number of coins for any interval $[a,b] \allowbreak \in \allowbreak N$ was proposed.

% Although spending one coin is more efficient in compact e-cash than in divisible e-cash schemes, completing a payment requires multiple coins to be spent, and thus compact e-cash schemes would be less efficient. A solution proposed in the literature to improve the efficiency of compact e-cash schemes consists of running in parallel several instances of the compact e-cash scheme each associated with a different  denomination. This would reduce dramatically the number of coins that need to be spent in a payment. However, to our knowledge, an efficiency analysis of compact e-cash schemes with multiple denominations has not been conducted.


% Most anonymous e-cash schemes~\cite{DBLP:conf/eurocrypt/CamenischHL05,DBLP:conf/pairing/BelenkiyCKL09,DBLP:conf/eurocrypt/CanardG07,DBLP:conf/acns/CanardPST15,DBLP:conf/pkc/CanardPST15,DBLP:conf/pkc/PointchevalST17,DBLP:conf/asiacrypt/BoursePS19,DBLP:conf/crypto/OkamotoO89,DBLP:conf/crypto/OkamotoO91,DBLP:conf/pkc/BaldimtsiCFK15,DBLP:conf/pkc/BauerFQ21} are unsuitable for distributed environments and incompatible with existing blockchain systems because of the use of a central bank. 
% Another possible solution to remove the central bank is to distribute the responsibility among a quorum of authorities using threshold cryptography.
% This solution has been explored recently in the context 
% of attribute-based credentials~\cite{DBLP:conf/ndss/SonninoABMD19} and online anonymous e-cash schemes~\cite{DBLP:journals/iacr/BaudetSKD22} although without security proofs. 
% While the functionality executed by the bank could be replaced with an MPC protocol, that implies the need for the authorities to communicate with each other during coins issuance, rendering a much more inefficient scheme than when using threshold cryptography. 
In order to address the urgent scalability issues of blockchain systems and possibly dangerous centralization of CBDCs without privacy, a formal treatment of offline e-cash is needed, including a fair comparison of compact and divisible e-cash. 

% \alf{Somewhere we need to explain that the reason why we added e-cash transfers is to protect the privacy of providers in the deposit phase (and thanks to that, our schemes can offer privacy for the parties that receive payments, like Zcash or similar. At the moment, I'm unsure whether adding e-cash transfer to this paper is a good idea. One  problem could be that a reviewer could realize that, with e-cash transfers, giving change is possible, and thus a more efficient offline e-cash scheme can be designed. A second problem is that reviewers may think that, if e-cash transfers are important for our system, then they should be added to the main body, leading to a second major revision. If we remove them, the problem could be that a reviewer suggests that our schemes do not protect the privacy of payment recipients like Zcash, but they haven't realized that in the first submission, and we can always argue that Zcash and other schemes are not offline.}

%nother alternative for removing the need for a central bank is given by the concept of decentralized private e-cash~\cite{DBLP:conf/sp/MiersG0R13,DBLP:conf/ccs/DanezisFKP13,DBLP:journals/iacr/Ben-SassonCG0MTV14}, in which coins are authenticated by proving in ZK that they belong to a public list of valid coins maintained on the blockchain. Other methods can also be used to provide private decentralized payments, like those used in e.g. Monero~\cite{cryptonote,DBLP:journals/ledger/NoetherM16}, Dash~\cite{dash} or Mimblewimble~\cite{DBLP:conf/eurocrypt/FuchsbauerOS19}. However, all of those approaches have the drawback of being online, i.e.\ a payment needs to be checked to detect double-spending before it can be accepted by a provider. This requires merchants to be constantly connected with the blockchain. 
%As recently shown in the example of Zcash, the online approach might be exploited by attackers performing spam attacks~\cite{zcashattack1, zcashattack2, zcashattack3}.  
%Offline schemes are advantageous because they enable faster payments with offline providers and do not require publishing large payments on the blockchain.  

%There are proposals to enable offline payments in blockchain-based cryptocurrencies, however, none of them offers strong privacy protection. In~\cite{10.1145/3052973.3052980}, a protocol for offline payments in Bitcoin is described in which users who double spend can be revoked. We remark that, in that protocol, users who make a payment are not anonymous. The user's public key is revealed to the recipient of the payment, and later the public key can be used for revocation if double spending is found. In contrast, in an anonymous e-cash, a user's public key can only be obtained from payments if the user double spent. 
%Alternatively, payment channel networks~\cite{DBLP:conf/ccs/0001M17} allow making arbitrarily many off-chain payments between channel creators. However, only users who share a channel can transact with each other, and
%similarly as in the case  of~\cite{10.1145/3052973.3052980} users who make a payment are not anonymous~\cite{DBLP:conf/fc/KapposYPKDMM21}. Moreover, cashing out payments still requires blockchain interactions. 
%Therefore, offline anonymous e-cash schemes with threshold issuance offer
%an interesting alternative. 
% Recent market research on CBD released by the ECB further shows increasing interest in the offline approach to electronic payments~\cite{digitaleuro}. 



% \textit{\textbf{Outline}}
% In \S\ref{sec:offlineecash}, we define the system model, security properties and the ideal functionality $\Functionality_{\CEC}$   
% of an offline anonymous e-cash with threshold issuance ($\CEC$) scheme. In Section~\ref{sec:constructionCEC}, we describe the construction which realises the ideal functionality. 
% In Section~\ref{sec:instantiation}, we present two instantiations of $\Functionality_{\CEC}$ using compact and divisible e-cash schemes. 
% In \S\ref{sec:efficiencycomparison}, we analyze and compare those schemes in terms of efficiency. 
% In Section~\ref{sec:integration}, we discuss how our schemes can be further extended to provide transferability and integrated with a blockchain-based bulletin board. 
% We conclude in~\S\ref{sec:conclusion}.
% In~\S\ref{sec:securitydefinitionsbuildingblocks}, we define the security properties of the cryptographic primitives used in our $\CEC$ schemes. 
% In \S\ref{sec:securityProofCompact} and \S\ref{sec:securityProofDivisible}, we provide a formal security analysis to show that $\mathrm{\Pi}_{\CEC}$ realizes $\Functionality_{\CEC}$ when instantiated with the algorithms of our compact and divisible e-cash schemes. 
% Finally, in \S\ref{sec:compactECashRangeProof}, we describe an instantiation of our compact $\CEC$ scheme with a concrete set membership proof, and in \S\ref{sec:divisibleEcashSpend}, we describe how the NIZK arguments of our divisible $\CEC$ scheme are instantiated.




% Our compact $\CEC$ scheme is based on the compact e-cash scheme in~\cite{DBLP:conf/eurocrypt/CamenischHL05}, which we choose for its efficiency. Our divisible $\CEC$ scheme is based on the divisible e-cash scheme in~\cite{DBLP:conf/pkc/PointchevalST17}, which is appealing because it allows the user to spend any number of coins. In~\cite{DBLP:conf/asiacrypt/BoursePS19}, it is shown that there is a security problem in the standard model instantiations of previous divisible e-cash schemes, including~\cite{DBLP:conf/pkc/PointchevalST17}. However, since we aim at designing practical schemes, both our compact and our divisible $\CEC$ schemes use the random oracle model, with non-interactive zero-knowledge (ZK) arguments of knowledge computed via the Fiat-Shamir heuristic, and they avoid that security problem. In~\cite{DBLP:conf/asiacrypt/BoursePS19}, a framework for divisible e-cash schemes is proposed, which allows the design of secure schemes in both the random oracle and the standard model.

% In both~\cite{DBLP:conf/eurocrypt/CamenischHL05,DBLP:conf/pkc/PointchevalST17}, a wallet is a signature obtained by a user through a blind signing protocol run with the bank. In our schemes, the bank is replaced by the authorities $(\fcecAuthority_1, \allowbreak \ldots, \allowbreak \fcecAuthority_n)$. To enable threshold issuance, we instantiate the signature scheme used to compute wallets with Pointcheval-Sanders (PS) signatures~\cite{DBLP:conf/ctrsa/PointchevalS16}, and we leverage the threshold issuance protocol for PS signatures in~\cite{DBLP:conf/ndss/SonninoABMD19} with the ~~

% We modify the scheme in~\cite{DBLP:conf/eurocrypt/CamenischHL05} to improve its efficiency. In~\cite{DBLP:conf/eurocrypt/CamenischHL05}, a wallet is a signature on $(\ssk_{\fcecUser_j}, \allowbreak \cecsn, \allowbreak \cect)$, where $\ssk_{\fcecUser_j}$ is a user secret key and $\cecsn$ and $\cect$ are coin secrets such that, in the spending phase, $\cecsn$ is used to compute serial numbers and $\cect$ is used to compute double spending tags. In our scheme, a wallet is a signature on $(\ssk_{\fcecUser_j}, \allowbreak \cecsn)$, i.e. only one coin secret is needed. This allows us to reduce the wallet size and to improve the efficiency of the withdrawal and spend protocols in comparison to~\cite{DBLP:conf/eurocrypt/CamenischHL05}.

% In both~\cite{DBLP:conf/eurocrypt/CamenischHL05,DBLP:conf/pkc/PointchevalST17}, in the blind signature protocol, both the user and the bank contribute randomness to generate the coin secret $\cecsn$. In our schemes, the user chooses $\cecsn$, which improves the efficiency of the blind signature protocol. As also noted in~\cite{DBLP:conf/asiacrypt/BoursePS19}, this change does not compromise the security of our schemes.

% We compare our compact $\CEC$ scheme when using multiple denominations with our divisible $\CEC$ scheme. To this end, we analyze the issue of choosing the optimal set of denominations. Then, for several sets of chosen denominations and price ranges, we calculate the average number of coins that need to be spent.


% In the deposit phase, our compact $\CEC$ scheme clearly outperforms our divisible $\CEC$ scheme. The reason is twofold. First, in the compact $\CEC$ scheme, an authority only needs to compare serial numbers to detect double spending, whereas in the divisible $\CEC$ scheme, the authority, given a payment, needs to compute the serial numbers for that payment before comparing them with the serial numbers of other payments. Second, when double spending is detected, the compact $\CEC$ scheme can identify the user guilty of double spending with cost independent of the number of users, whereas in the divisible e-cash scheme the cost of identification grows linearly with the number of users.



% \ania{TODO: explain why our design is more interesting than Coconut or Zef, aka what are their limitations.}
% In this work, we propose two offline anonymous e-cash schemes with threshold issuance. First, we describe a compact e-cash scheme based on the scheme in~\cite{DBLP:conf/eurocrypt/CamenischHL05}. Second, we describe a divisible e-cash scheme based on the scheme in~\cite{DBLP:conf/pkc/PointchevalST17}. For both schemes, the protocol for threshold issuance is based on~\cite{DBLP:conf/ndss/SonninoABMD19} with the changes described in~\cite{cryptoeprint:2022:011}. To the best of our knowledge these are the first offline e-cash schemes with threshold issuance.
% To the best of our knowledge, no offline e-cash schemes with threshold issuance have yet been proposed.

% \vspace{2mm}
% \noindent \textbf{Our contributions:}
% In this paper, we make the following contributions.
% \begin{itemize}[noitemsep,topsep=0pt]
%     \item \ania{TODO: Needs updating} In~\cite{DBLP:conf/eurocrypt/CamenischHL05,DBLP:conf/pkc/PointchevalST17}, a wallet is a signature obtained by a user through a blind signing protocol run with the bank. Both our schemes remove the central bank in~\cite{DBLP:conf/eurocrypt/CamenischHL05,DBLP:conf/pkc/PointchevalST17} and consider instead $n$ authorities $(\fcecAuthority_1, \allowbreak \ldots, \allowbreak \fcecAuthority_n)$. In the setup phase, a verification key $\spk$ for wallets is generated, and each authority receives a secret key share for $\spk$. In the withdrawal phase, to obtain a wallet, a user needs to run a blind signing protocol with at least $t \allowbreak \leq \allowbreak n$ authorities. Each authority issues a signature by using its secret key share, and $t$ signatures from $t$ different authorities can be turned into a signature verifiable with $\spk$.

    % The spending and deposit phase of our compact e-cash scheme follows the scheme in~\cite{DBLP:conf/eurocrypt/CamenischHL05}. Our divisible e-cash scheme is a version in the random oracle model of the scheme in~\cite{DBLP:conf/pkc/PointchevalST17}, which is secure in the standard model. (\alf{Mention here the security problem in~\cite{DBLP:conf/pkc/PointchevalST17}}.) This change allows us to simplify the scheme in~\cite{DBLP:conf/pkc/PointchevalST17} and to improve its efficiency.

    % On an abstract level, the difference between the compact e-cash scheme and the divisible e-cash scheme is the following. In compact e-cash, serial numbers of each of the coins spent are generated by the user during the spending phase, and thus the cost of this phase grows with the number of coins spent. In divisible e-cash, the serial numbers are generated by the authority in the deposit phase, with information provided by the user during the spending phase and additional public parameters generated during the setup phase.

    % \item Next, we formally prove the security of our schemes in the ideal-world/real-world paradigm~\cite{DBLP:conf/focs/Canetti01}.
    % To this end, we define the ideal functionality for offline anonymous e-cash with threshold issuance. \ania{Is there something more you would like to highlight about the security analysis?}
    %which we briefly describe in~\S\ref{sec:securitymodel}.


%\end{itemize}


