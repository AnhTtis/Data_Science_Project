
Decentralized, offline, and privacy-preserving e-cash could fulfil the need for both scalable and byzantine fault-resistant payment systems. 
Existing offline anonymous e-cash schemes are unsuitable for distributed environments due to a central bank. 
We construct a distributed offline anonymous e-cash scheme, in which the role of the bank is performed by a quorum of authorities, and present its two instantiations. Our first scheme is compact, i.e. the cost of the issuance protocol and the size of a wallet are independent of the number of coins issued, but the cost of payment grows linearly with the number of coins spent. Our second scheme is divisible and thus the cost of payments is also independent of the number of coins spent, but the verification of deposits is more costly. We provide formal security proof of both schemes and compare the efficiency of their implementations. 
% Finally, we show how the schemes can support anonymous deposits as well as transferability in conjunction with a bulletin board like a blockchain, making offline e-cash a compelling next step in the evolution of payment systems. \alf{In this sentence, I would remove ``in conjunction with a bulletin board like a blockchain''. The bulletin board is needed in general for double spending detection, not only for transferability.}



