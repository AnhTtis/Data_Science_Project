

\section{Threshold Issuance Offline E-Cash}
\label{sec:offlineecash}

In this section, we introduce an offline e-cash scheme with threshold issuance ($\CEC$). First, we outline its system model and informally define the security properties. 
To define formally the security properties of $\CEC$, we construct the ideal functionality $\Functionality_{\CEC}$ and explain how it guarantees those properties.

% Our formal security definition for $\CEC$ is given in~\S\ref{sec:idealfunctionalityCEC}. In~\S\ref{sec:constructionCEC}, we describe a construction compatible with the security definition, which uses the algorithms defined here as building blocks.

% In \S\ref{sec:idealfunctionalityCEC}, we give a security definition for offline e-cash with threshold issuance ($\CEC$) in the form of an ideal functionality $\Functionality_{\CEC}$. In \S\ref{sec:constructionCEC}, we propose a construction $\mathrm{\Pi}_{\CEC}$ that realizes $\Functionality_{\CEC}$. $\mathrm{\Pi}_{\CEC}$ uses as building block algorithms for $\CEC$ that we define in this section. In \S\ref{sec:compactecash} and in \S\ref{sec:divisibleecash}, we depict how these algorithms are instantiated for our compact and our divisible e-cash schemes respectively.
\subsection{System model}

An $\CEC$ scheme involves $n$ authorities $(\fcecAuthority_1, \allowbreak \ldots, \allowbreak \fcecAuthority_n)$, any number of users $\fcecUser_j$ and any number of providers $\fcecProvider_k$.
The interaction between those parties takes place through a \emph{setup}, \emph{withdrawal}, \emph{spend} and \emph{deposit} phase. Users withdraw wallets containing one or more electronic coins from the authorities and spend them with providers, who then deposit them back with the authorities.  

In the \textbf{setup} phase, a trusted third party 
generates the public parameters $\cecparams$. Next, the public verification key  
$\spk$ is generated, alongside key pairs $(\ssk_{\fcecAuthority_i}, \allowbreak \spk_{\fcecAuthority_i})_{i\in[1,n]}$ for each of the authorities $\fcecAuthority_i$. The keys are generated in such a way that a user needs to receive a withdrawal from at least $t$ authorities in order to create a wallet.
The key generation can be executed by a trusted third party or run in a distributed way~\cite{DBLP:journals/iacr/KateHG12, cryptoeprint:2021:339}. 
Finally, each user $\fcecUser_j$ generates a key pair $(\ssk_{\fcecUser_j}, \allowbreak \spk_{\fcecUser_j})$. 

To obtain a wallet with $L$ coins, where $L$ is a public parameter of the scheme, a user $\fcecUser_j$ engages in the \textbf{withdrawal} protocol. To this end, $\fcecUser_j$ sends a request $\cecrequest$ to a set of $t$ different authorities. 
% The request contains a digital signature under $\ssk_{\fcecUser_j}$ on a message that describes the number of coins the user wishes to withdraw and includes additional protocol-specific values.
Each $\fcecAuthority_i$ verifies  $\cecrequest$ and using its secret key $\ssk_{\fcecAuthority_i}$ issues back a response $\cecresponse$. 
% The user obtains the signature from each validator through a blind signature protocol~\ref{}, thus 
% $\fcecAuthority_i$ does not learn any of the signed values, only the user's public key $\spk_{\fcecUser_j}$ \alf{This sentence describes security properties. I would remove it from here.} 
$\fcecUser_j$ verifies $\cecresponse$ and extracts from it a partial wallet $\cecwallet_i$. Once $\fcecUser_j$ has completed the protocol with at least $t$ authorities, 
and collected the threshold number of partial wallets,  
$\fcecUser_j$ aggregates them to form
a single consolidated wallet $\cecwallet$ with $L$ coins of the same monetary value. To \textbf{spend} $V$ coins  with a provider $\fcecProvider_k$ the user generates a payment $\cecpayment$ using her wallet $\cecwallet$ and payment information $\cecpaymentinfo$. $\cecpaymentinfo$ contains the provider's identity and other information about the payment and must be unique for each payment. $\fcecUser_j$ sends $\cecpayment$ to $\fcecProvider_k$, who verifies it. 
To \textbf{deposit} the payment $\cecpayment$, provider $\fcecProvider_k$ sends $\cecpayment$ to a bulletin board $\BB$. Each authority reads $\cecpayment$ from $\BB$, verifies it and checks it against all the payments previously written on $\BB$, in order to rule out double spending and double depositing. 
The double spending detection mechanism reveals the public key of the user $\fcecUser_j$ if the user double-spent any coin, while double depositing detection reveals that the payment is deposited twice if two payments contain the same payment information $\cecpaymentinfo$. Otherwise, the payment is deposited successfully.



\subsection{Security Properties}
\label{sec:securityoffec}
As defined in~\cite{DBLP:conf/asiacrypt/BoursePS19}, secure anonymous offline e-cash schemes should satisfy four properties. We describe them informally, taking into account that, in our schemes, the bank is replaced by a number of authorities.

\begin{description}[leftmargin=*, noitemsep,topsep=0pt]

    \item[Traceability:] guarantees that no more coins can be deposited than those that have been withdrawn. In particular, adversarial parties are not able to forge wallets. It also guarantees that an honest authority is able to identify a user who double-spends a coin. Double-depositing is also detected by the authority.

    \item[Unlinkability:] ensures that no coalition of dishonest authorities, users and providers is able to link the withdrawal of the wallet with the spending of its coins. It also guarantees that multiple spendings performed by the same user cannot be linked with each other.

    % \item[Balance] guarantees that no coalition of dishonest users and providers is able to deposit more coins than the number of coins withdrawn.

    % \item[Traceability] guarantees that if a user is proven guilty of double-spending, then their identity is revealed.

    \item[Exculpability:] requires that an honest user cannot be found guilty of double-spending.

    \item[Clearance:] ensures that only the provider that receives a payment is able to deposit it.

\end{description}