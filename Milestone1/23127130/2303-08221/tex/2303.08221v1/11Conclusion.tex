\section{Integration}\label{sec:integration}
In our schemes, deposited payments are verified by the authorities. Double-spending is detected by storing previous payments on a bulletin board and checking that serial
numbers in a deposited payment are not equal to any serial number
of previously submitted payments.
A blockchain can be used to implement the bulletin board. When a provider wishes to deposit a payment, the payment is broadcast to authorities, which could also be $n$ validators in a proof-of-stake blockchain. The consensus mechanism of the blockchain is then used to agree on whether the payment is valid and, in that case, recorded in the blockchain. This allows decentralizing also the deposit phase of our protocols. 
% In order to prevent the same problem of scaling as affected ZCash, in which the blockchain bloats with too many transactions~\cite{zcashattack1}, we can limit the information which needs to be submitted into the blockchain to just the serial number and double spending tag.\alf{Regarding this, I think it is more relevant what is said below, i.e. the ability to reset the bulletin board and avoid an ever growing block-chain.} 
% Moreover, the transferability mechanism (\Cref{sec:ecashtransfer}) further limits the attack applicability.\alf{How does it limit it?}  

Also, note that authorities in this setting may leave and enter the system dynamically. However, we must take into account that an authority that leaves the system still possesses a valid share of the secret key, and we assume that, after leaving the system, the authority becomes corrupt.  Similarly to decentralized e-cash schemes, in our schemes we assume an honest majority, i.e. if $n = 2a + 1$, there are at most $a$ corrupt authorities. For our schemes to be secure, if $a$ is the number of corrupt authorities currently in the system, and $b$ is the number of authorities that have left, we need that $a + b < t$, where $t$ is the threshold. Therefore, algorithm $\cecKeyGenA$ needs to be run periodically to create a new public verification key $\spk$ and new key pairs $(\ssk_{\fcecAuthority_i}, \allowbreak \spk_{\fcecAuthority_i})_{i\in[1,n]}$ so as to ensure that $a + b < t$. This implies that e-cash expires whenever a new key is created. A time interval in which users can convert old wallets to use the new secret key can be given~\cite{chaum2021issue}. Once this interval ends, e-cash expiration makes it possible to delete the payments stored for double-spending detection and reset the bulletin board, which avoids an ever-growing blockchain~\cite{zcashattack1}. The blockchain can thus be used as a settlement layer for offline e-cash transactions. 

This scheme would get the best of both worlds, that of online blockchain and offline e-cash, but further research needs to be done to specify this model formally and parameterize a real-world implementation. Threshold-issuance offline e-cash may end up solving the pressing problems of privacy and scalability of payments in both blockchain and even CBDCs as e-cash moves from theory into practice. 


\section{Conclusion}
\label{sec:conclusion} 

In this work, we proposed the first offline anonymous e-cash scheme with threshold issuance, motivated by the concrete scalability concerns of blockchains and concerns with centralization in CBDCs~\cite{digitaleuro}. We define the ideal functionality and propose two instantiations based on an improved compact and a divisible e-cash. We have shown that our schemes realize the ideal functionality and formally prove their security. We have also implemented both schemes and compared their efficiency, showing that compact e-cash is more efficient and feasible for smaller transactions, which would naturally compose the majority of offline e-cash transactions in application scenarios such as a user-facing blockchain or CBDC where practical deployment concerns would necessitate distributed authorities. 
% Although a lack of space prevents their discussion, we further outline how our schemes can be extended to provide transferability, merchant anonymity and value obfuscation, both informally described in Appendix~\ref{sec:ecashtransfer} and formalized in Appendix~\ref{sec:ecashtransfers}. 




% In that comparison, we have considered the use of several denominations for our compact e-cash scheme. Our results show that our compact e-cash scheme is more suitable than our divisible e-cash scheme in certain settings. 

% We have proposed a compact and a divisible anonymous e-cash scheme with threshold issuance. We have described an ideal functionality for anonymous e-cash with threshold issuance and we have shown that our schemes realize it. We have also implemented our schemes and we have compared them in terms of efficiency. In that comparison, we have considered the use of several denominations for our compact e-cash scheme. Our results show that our compact e-cash scheme is more suitable than our divisible e-cash scheme in certain settings. 

% As future work, we consider a performance analysis of our schemes in a blockchain environment. The blockchain would act as the bulletin board in which payments are deposited. This would allow us to measure the efficiency of double spending detection and identification by taking into account the cost of blockchain transactions.
