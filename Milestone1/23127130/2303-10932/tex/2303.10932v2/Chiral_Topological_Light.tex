\documentclass[%
 reprint,
 amsmath,amssymb,
 aps
,longbibliography]{revtex4-1}

\usepackage{braket}
\usepackage{graphicx}% Include figure files
\usepackage{dcolumn}% Align table columns on decimal point
\usepackage{bm}% bold math
\usepackage{xcolor}

\begin{document}


\title{Chiral topological light for detecting robust enantio-sensitive observables}

\author{N. Mayer$^{1}$, D. Ayuso$^{2}$, P. Decleva$^{3}$, M. Khokhlova$^{1,4}$, E. Pisanty$^{4}$, M. Ivanov$^{1,2,5,6}$,
O. Smirnova $^{1,6,7}$}


\address{
$^1$Max-Born-Institut, Max-Born str. 2A, 12489 Berlin, Germany \\
$^2$Department of Physics, Imperial College London, Prince Consort Rd, London, SW7 2NW, United Kingdom \\
$^3$CNR IOM and Dipartimento di Scienze Chimiche Farmaceutiche, Universit\'a degli Studi di Trieste, Via Licio Giorgieri 1, Trieste, 34127, Italy \\
$^4$Attosecond Quantum Physics Laboratory, Department of Physics, King's College London, Strand, London, WC2R 2LS, United Kingdom \\
$^5$Department of Physics, Humboldt University, Newtonstr. 15, D-12489 Berlin, Germany\\
$^6$ Technion - Israel Institute of Technology, 3200003, Haifa, Israel\\
$^7$ Technische Universit\"at Berlin, Str. des 17. Juni 135, Berlin, 10623, Germany}


\begin{abstract}

The topological response of matter to electromagnetic fields is a property in high demand in materials design and metrology due to its robustness against noise and decoherence, stimulating recent advances in ultrafast photonics. 
Embedding topological properties into the enantio-sensitive optical response of chiral molecules could therefore enhance  the efficiency and robustness of chiral optical discrimination.
Here we achieve such a topological embedding by introducing the concept of chiral topological light~-- a light beam which displays chirality locally, with an azimuthal distribution of its handedness described globally by a topological charge.
The topological charge is mapped onto the azimuthal intensity modulation of the non-linear optical response, where enantio-sensitivity is encoded into its spatial rotation.
The spatial rotation is robust against intensity fluctuations and imperfect local polarization states of the driving field.
Our theoretical results show that chiral topological light enables detection of percentage-level enantiomeric excesses in randomly oriented mixtures of chiral molecules, opening a way to new, extremely sensitive and robust chiro-optical spectroscopies with attosecond time resolution.
\end{abstract}
\maketitle


The topological properties of the electronic response to electromagnetic fields in solid state systems, as well as in photonic structures, are being actively harvested to obtain robust observables, such as e.g.\ edge currents  protected from material imperfections in topological insulators~\cite{RevModPhys.82.3045} or topologically protected light propagation pathways in their photonic analogs~\cite{Khanikaev:2013aa, rechtsman2013photonic}.
A similar robustness in the enantio-sensitive optical response of chiral molecular gases or liquids is much desired for analytical purposes, but is currently missing. While the first ideas connecting topological and chiral properties of electronic responses~\cite{Ordonez:2019aa, Ordonez:2022aa, Ordonez:2021arxiv} or microwave signals in molecular gases~\cite{Kai:topological,Kai:microwave} are starting to emerge, they do not map onto the optical response, which encodes the ultrafast, attosecond chiral electronic dynamics \cite{Cireasa:2015aa, Smirnova:2009aa, Smirnova:2009ab, Baker:2006aa}.

Topologically non-trivial optical signals can be achieved by using vortex beams, which carry orbital angular momentum~(OAM). They are characterized by an integer topological charge representing the number of helical revolutions of light's wavefront in space within one wavelength~\cite{Shen:2019aa}. Pertinent work established the chirality of vortex light in the linear regime~\cite{Forbes:2018,Forbes:2021}, exploited and manipulated ultrafast non-linear optical responses to vortex beams in atoms~\cite{Rego:2019aa, Rego:2019ab, Dorney:2019aa}, including the discovery of new synthetic topologies~\cite{Pisanty:2019aa, Pisanty:2019ab}, as well as in chiral molecules~\cite{Ashish:2023aa, Begin:2023aa}, where vortex light has also been successfully used for chiral detection in the hard X-ray region~\cite{Rouxel:2022aa, Rouxel:2022ab}. 
However its natural enantio-sensitivity in the optical domain is weak due to the orders-of-magnitude mismatch between the wavelength of the light and the size of the molecules.

This limitation can be overcome by encoding chirality in the Lissajous figure drawn by the polarization vector of an electromagnetic wave in time.
Such fields employ only electric-dipole transitions to drive non-linear enantio-sensitive signals, and have been devised~\cite{Kral:2001aa}, applied in the microwave region~\cite{Eibenberger:2017aa}, and extended to the optical domain~\cite{Ayuso:2019aa}.
The handedness of this light can be controlled with the phase delay between its frequency components, both locally --at every point in space-- and globally in the interaction region~\cite{Ayuso:2021aa, Khokhlova:2021aa}.

Here we introduce the concept of chiral topological light, which takes advantage of the global topological structure of vortex light and the high enantio-sensitivity of synthetic chiral light~\cite{Ayuso:2019aa}, embedding  robust topological properties into the highly enantio-sensitive ultrafast optical response.

\begin{figure*}[t]
\centering
\includegraphics[width=13cm, keepaspectratio=true]{Fig1.png}
\caption{The concept of chiral vortex light for bicircular counter-rotating $m_{\omega}=-m_{2\omega}=1$ beams carrying OAMs $\ell_{\omega}=-\ell_{2\omega}=1$. \textbf{a)}~Tight-focusing of bicircular counter-rotating Gaussian beams induces a longitudinal field, resulting in a synthetic chiral field whose polarization vector draws a chiral Lissajous curve over one laser cycle (inset). 
\textbf{b)}~Evolution of the chiral Lissajous curves with respect to the azimuthal angle $\theta$ at a given radial position $\rho=\sqrt{x^2+y^2}$ at $z=0$ for a chiral vortex with $\ell_\omega=-\ell_{2\omega}=1$. \textbf{c)}~Slices through the electric field distribution at $z=0$. The figures show the total intensity of the electric field~$|\mathbf{E}|^2$, the absolute value of the chiral correlation function $|h^{(5)}|$ and its phase distribution $\arg\mathopen{}\left[h^{(5)}\right]\mathclose{}$. The phase distribution of $h^{(5)}$ describes the spatial distribution of the handedness of light and is characterized by a topological charge $C=6$. The $x$ and $y$ coordinates are scaled to the waist $W_0$ of the beams at the focus.}
\label{Fig1}
\end{figure*}

Our key idea is to imprint the topological parameter of the vortex beam on the azimuthal phase of the chiral correlation function $h$~\cite{Ayuso:2019aa} characterising the handedness of the synthetic chiral light: $\arg[{h(\theta)}]=C\theta+\phi_L$, where~$\theta$ is azimuthal angle, $C$ is the topological charge, and $\phi_L$ is the enantio-sensitive phase of synthetic chiral light.

We show that the intensity of the nonlinear optical emission of a chiral molecular medium triggered by such light depends on both chiral and topological phases of $h$ as well as the enantio-sensitive phase $\phi_M$ introduced by the molecular medium: $I(\theta)\propto \cos(\phi_M-\phi_L+C\theta)$. Using numerical simulations, we demonstrate that the azimuthal intensity profile is patterned in a topologically robust and molecule-specific way, leading to a large enantio-sensitive offset $\Delta\theta=\pi/C$ between the intensity maxima (or minima) in opposite enantiomers. What's more, the topologically controlled angular offset is robust with respect to imperfections of light polarization and intensity fluctuations, persists for small amounts of enantiomeric excess and can be used to probe chirality in dilute mixtures.

To demonstrate these ideas, we now focus on a specific realization of chiral topological light. It involves two Laguerre-Gaussian beams with counter-rotating circular polarizations, propagating along the $z$-axis with frequencies $\omega$ and $2\omega$ and orbital angular momenta $\ell_\omega$ and $\ell_{2\omega}$ (see Methods). Near the focus,
the field develops a longitudinal component given by $E_z=-(\text{i}/k)\nabla_{\perp}\cdot\mathbf{E}_{\perp}$ in the first post-paraxial approximation~\cite{Bliokh:2015aa} (see Fig.~\ref{Fig1}a), taking the light polarization vector out of the $(x,y)$ plane -- a prerequisite for creating synthetic chiral light.

As a result, the Lissajous figure drawn by the polarization vector in one point in space over a laser cycle becomes chiral (see inset in Fig.~\ref{Fig1}a. Its handedness is controlled by the two-color phase $\phi_{2\omega,\omega}=2\phi_\omega-\phi_{2\omega}$, which depends on the azimuthal coordinate, forming a chiral vortex with the topological charge (see Methods): 
%
\begin{equation}\label{Eq1}C=2\ell_{\omega}-\ell_{2\omega}+2m_{\omega}-m_{2\omega}.\end{equation}
%
Here $m_{r\omega}$ indicates right ($m_{r\omega}=1$) or left ($m_{r\omega}=-1$) circular polarization. The Lissajous curve drawn by the polarization vector of the electric field over one laser cycle changes with the azimuthal angle, switching handedness $2|C|$ times as the azimuthal angle cycles over one revolution (Fig.~\ref{Fig1}b). Thus, the superposition of two tightly-focused OAM beams at commensurate frequencies gives rise to a chiral vortex, i.e. a vortex beam displaying chirality locally at each given point with an azimuthally varying handedness characterized by an integer topological charge $C$.

Fig. \ref{Fig1}c visualizes the chiral vortex by displaying the beam total intensity $|\mathbf{E}(x,y)|^2$, the absolute value $|h^{(5)}(x,y)|$ and the phase $\arg[h^{(5)}(x,y)]$ of the chiral correlation function for OAM $(\ell_\omega,\ell_{2\omega})=\left(1,-1\right)$ and SAM $(m_{\omega},m_{2\omega})=(1,-1)$. Both the chirality and the total intensity maximize along rings (see Fig.~\ref{Fig1}c), typical for vortex beams, while the topological charge $C=6$ characterizes the azimuthal phase distribution of the light's handedness quantified by the chiral correlation function.

The chiral topological charge $C$ is highly tunable thanks to 
its dependence on the OAM of the two beams, which can take any integer value from $-\infty$ to $\infty$,  enabling chiral vortices with arbitrarily high (and also arbitrarily low) chiral topological charge. 
By controlling the OAM of the beams, we can thus create chiral vortex beams with controlled properties. If $C=0$, then the chiral vortex has the same local handedness everywhere in space. Otherwise, the field's handedness displays a non-trivial spatial structure which is characterized by $C$.

\begin{figure}[t]
\centering
\includegraphics[width=8cm, keepaspectratio=true]{Fig2.png}
\caption{Enantio-sensitive high-harmonic spectroscopy using chiral topological light with topological charge $C=6$. \textbf{a,b)}~show the near-field spatial profile of H18 for L-fenchone (\textbf{a}) and R-fenchone (\textbf{b}). The $x$ and $y$ axes are given in units of the field waist at the focus $W_0$. \textbf{c,d)}~show the corresponding far-field spatial profiles for the two enantiomers. Here $k_x$ and $k_y$ are given in units of the reciprocal waist of the field at the focus, $1/W_0$. All profiles are normalized to their maximum value, which is the same for opposite enantiomers. The angles in the far-field picture indicate the position of the first peak in the outer ring of the profile, where we set the zero angle along the positive $k_x$ direction. For $C=6$ we have that $\phi_L=\phi_R+\pi/3$.}
\label{Fig2}
\end{figure}

We have modeled the highly nonlinear response of randomly oriented chiral molecules to this realization of chiral topological light depicted in Fig. 1\ref{Fig1} using a DFT-based S-matrix approach (see Methods).
Figure \ref{Fig2}a,b shows  the near-field  intensity of harmonic 18 generated in R- and L-fenchone, for fundamental frequency $\omega=0.044$~a.u.\ (1033 nm), peak intensity $I_0=5\cdot10^{14}$~W/cm$^2$ and a beam waist of $W_0=2.5$~$\mu\text{m}$ at the jet position $z=0$.

The azimutal distribution of the near-field intensity records both the topology of the driving laser field and the handedness of the medium.
It results from the interference between chiral and achiral multiphoton pathways.
The maxima occur at angles $\theta=\left[2\pi n+(\phi_L-\phi_M)\right]/C$, where the two pathways interfere constructively. The angular position of the peaks is therefore enantio-sensitive: swapping the molecular enantiomer leads to to a $\pi$ shift in the molecular phase $\phi_M\rightarrow\phi_M+\pi$, shifting the minima and maxima of the intensity pattern by $\pi/C$. The number of peaks is controlled by the topological charge $|C|=6$.
Importantly, the same topological structure is preserved in the far-field response, Fig.~\ref{Fig2}c,d.

\begin{figure*}[t]
\centering
\includegraphics[width=13cm, keepaspectratio=true]{Fig3.png}
\caption{Results as a function of the enantiomeric excess. \textbf{a,b,c)}~The far-field spatial profiles of H18 for an enantiomeric excess $ee=(C_R-C_L)/(C_R+C_L)$ of -4\%, 0\% and 4\% (respectively \textbf{a,b,c}), where positive enantiomeric excess corresponds to a larger concentration of R-fenchone in the sample. \textbf{d)}~Angle-resolved, radially-integrated far-field signal of the outer ring of the spatial profile ($|kW_0|>10$) as a function of the enantiomeric excess. The black line on the right shows the phase of the Fourier component of the spatial profile oscillating at frequency $\ell=6$ as a function of the enantiomeric excess. The overlapping red line shows the result accounting for intensity fluctuations. The $\pi$ jump at $ee=0\%$ indicates the enantio-sensitive rotation of the spatial profile. \textbf{e)}~Phase of the Fourier component (black solid line) and one obtained including intensity fluctuations (red dotted line) for enantiomeric excess between -5\% and 5\%.}
\label{Fig3}
\end{figure*}

Encoding the topological charge $C$ into the molecular response and extracting the enantio-sensitive offset angle, controlled by $C$, allows us to measure the enantiomeric excess $ee=(C_R-C_L)/(C_R+C_L)$ in macroscopic mixtures of left and right molecules with concentrations $C_L$ and $C_{R}$. 
Even for very small values of $ee$, we observe the appearance of the $C$-fold structure in the inner and outer rings, as well as the corresponding enantio-sensitive rotation of the spatial profile, see Figs. \ref{Fig3}a,c. For $ee=0\%$  (Fig.~\ref{Fig3}b), chiral channels are suppressed and a topologically different $2C$-fold structure is observed as a result of the interference between the two strongest open achiral channels (see Methods and Supplementary Information, SI).

The enantiosensitive rotation of the $C$-fold structure in the outer ring is apparent in the angle-resolved, radially-integrated signal (Fig.~\ref{Fig3}d). It manifests in the abrupt switching of the azimuthal angle which maximises the signal, as one changes the enantiomeric excess from positive to negative. The enantiosensitive rotation can be easily separated by performing  a Fourier analysis of the signal with respect to the azimuthal angle as a function of the enantiomeric excess. The solid black line in Fig.~\ref{Fig3}d shows the phase of the Fourier component $f_6$ oscillating at the $C=6$ frequency of the outer ring signal, as a function of the enantiomeric excess. A clear $\pi$ phase jump is observed at $ee=0\%$, indicating the switch in the handedness of the mixture. The sharpness of this jump (Fig.\ref{Fig3}e) characterizes the accuracy of resolving left and right molecules in mixtures with vanishingly small enantiomeric excess.

We now show how the enantiosensitive signal is robust with respect to imperfections in the laser beams.
First, we have included noise in our simulations (see Methods) via 2$\%$ intensity fluctuations of the driving fields. The red line in Figs.\ref{Fig3}d,e shows the phase of the Fourier component $f_6$ when noise is included. It is clear that the $\pi$ jump of the phase is robust against noise. Positive and negative enantiomeric excess can be distinguished with high fidelity due to the extremely sharp jump of the signal (Fig.\ref{Fig3}e) confined to very small values of enantiomeric excess (about 0.1$\%$). Indeed, the topological structure is imprinted via azimuthal interference of chiral and achiral responses and will survive as long as the two-color phase remains stable.
Given that in multicycle two-color fields such phase is routinely controlled with 
extremely high accuracy~\cite{Pedatzur:2015aa, Fleischer:2014aa}, we expect robust encoding of topological information in the molecular gas and robust read-out of the chiral topological signal.

The experimentally pertinent imperfection is also related to the SAM (imperfect circularity) and OAM content of the beams. Such imperfections affect the topological charge but not the concept of enantio-sensitive rotation of the non-linear response in the polarization plane.

We now focus on the analysis of chiral topological light created by elliptically polarized drivers with imperfect circularity (see SI for imperfections in the OAM content), which brings additional opportunities. 
To understand the effect of the 
imperfect circularity of the driving field on our observables, we express the elliptical field in terms of  two counter-rotating circularly polarized components: $\mathbf{E}(\omega)=[(1+\epsilon)\mathbf{E}_+(\omega)$ $+$ $(1-\epsilon)\exp(\text{i}\delta)\mathbf{E}_-(\omega)]/\sqrt{2(1+\epsilon^2)}$. Here $\delta$ is the phase delay between the components, which corresponds to the orientation of the resulting elliptical polarization and can be well controlled in the experiment \cite{Comby:2023aa}, and $|\epsilon|\le 1$ is the ellipticity, which is difficult to control with few-percent accuracy. Note that $\delta=0,\pi$ correspond to elliptical light ``squeezed'' along the $x$- and $y$-axis respectively (see Fig.\ref{Fig4}a).

The appearance  of the additional counter-rotating component in the elliptical beam leads to two interrelated consequences: (i)~the change of the topological structure of the harmonic emission due to the presence of new SAM components in the beams (see Eq. \ref{Eq1}), and (ii)~the appearance of two strong multiphoton pathways contributing to the achiral harmonic signal and effectively masking a weaker chiral signal driven by the longitudinal polarization. These issues are addressed by realizing an analogue of the lock-in method for the amplification of the chiral signal, based on its dependence on $\delta$.
%
\begin{figure*}[t]
\centering
\includegraphics[width=13cm, keepaspectratio=true]{Fig4.png}
\caption{Fourier analysis to recover the enantiosensitive rotation of the spatial profile in the case of elliptical fields. \textbf{a)}~Ellipse of an elliptical field and its orientation in the $x,y$ plane with respect to the phase delay $\delta$ between the counter-rotating components. \textbf{b)}~Spatially-integrated far-field signal for H18 and H19 as a function of the phase delay $\delta$. The red (blue) dotted line corresponds to the signal $S_L$ ($S_R$) from (R-)L-fenchone, while the black dotted line corresponds to the chiral dichroism signal $2(S_R-S_L)/(S_R+S_L)$. \textbf{c)}~Far-field spatial profiles for H18 and H19 obtained after Fourier transform with respect to the phase delay $\delta$ at the $\tilde{\delta}=1$ Fourier component for H18 (top plots) and H19 (bottom plots). The left (center) column shows the results for L-(R-)fenchone, while the right column shows the difference signal $S_R-S_L$. \textbf{d,e)}~Radially-integrated signal as a function of the azimuthal angle of the far-field spatial profiles for H18~(d) and H19~(e). The solid red (blue) lines corresponds to L-(R-)fenchone, while the black line shows the chiral dichroism signal $2(S_R-S_L)/(S_R+S_L)$.}
\label{Fig4}
\end{figure*}

Fig. \ref{Fig4}b shows the total far-field intensity for H18 and H19 as a function of the phase delay $\delta$ between the counter-rotating components of the $\omega$ field for both R- ($S_R$, blue dotted line) and L-fenchone ($S_L$, red dotted line), as well as the chiral dichroism~$2(S_R-S_L)/(S_R+S_L)$ (black dotted line), for globally chiral topological light with fundamental beam ellipticity of $\epsilon_\omega=0.9$.
The other parameters were kept as above. 
The strength of the far-field signal changes as one rotates the ellipse of the $\omega$ field, while the chiral dichroism in the signal intensity is maximized at around 20\%.

Fourier transforming the far-field intensity profile with respect to the phase delay $\delta$ separates the contributions of different pathways, because they experience different modulation with $\delta$. The two achiral pathways interfere in the third Fourier component ($\tilde\delta=3$) with respect to $\delta$, while the dominant contribution between the chiral and achiral pathways corresponds to the first Fourier component ($\tilde\delta=1$). Fig.~\ref{Fig4}c shows the far-field spatial profiles for $\tilde\delta=1$ for both enantiomers and both H18 and H19, as well as the their difference, while the polar plots in Fig.~\ref{Fig4}d show the radially-integrated signals and chiral dichroism. We see that the Fourier filtering recovers the enantiosensitive rotation, although the dominant topological charge is now $C=2$, (Eq.~\ref{Eq1} for $m_\omega=m_{2\omega}=-1$).

The additional benefit of using elliptical drivers is the ability to produce enantiosensitive rotation at $3N+1$ harmonic orders, which are naturally much stronger than $3N$ harmonics. Given the experimental ability to precisely control the orientation of the polarization ellipse of the driving infrared pulses, chiral topological light generated by such drivers stands out as a robust probe of molecular chirality, capable of inducing strongly enantio-sensitive total intensity signals as well as giant rotations of intense spectral profiles.

The concept of chiral topological light introduced here is not limited to vortex beams: other members of the larger family of structured light beams~\cite{Angelsky:2020aa, Forbes:2021aa, Dunlop:2017aa} can be used to create locally and globally chiral topological light. We envision using radially polarized beams, which are known to posses strong longitudinal components, central to the concept of local chirality, under tight focusing conditions~\cite{Dorn:2003aa}. Skyrmionic beams~\cite{Cuevas:2021aa, Du:2019aa} are the other exciting candidates, which could also be used in order to induce topological distributions with radially-dependent topological charges. We expect to find new robust and efficient enantiosensitive observables associated with such fields. From the perspective of structured light \cite{Angelsky:2020aa, Forbes:2021aa, Dunlop:2017aa, Bliokh:2023aa} the temporally chiral vortex introduced here represents a new kind of polarization singularity, which could be analyzed by extending the current framework from the monochromatic three dimensional fields \cite{Bliokh:2019aa,Alonso:2023aa} to the polychromatic 3D fields~\cite{Pisanty:2019aa, Sugic:2020aa,Kessler:2003aa}. Our method is not limited to high harmonics. Its extension to low-order parametric processes such as chiral sum-frequency generation~\cite{Vogwell:2023aa} or chiral nonlinear Stark shifts~\cite{Khokhlova:2021aa} has potential for applications in ultrafast control of artificial chirality~\cite{Mayer:2022aa}, for non-destructive enantio-sensitive imaging in the UV region, and for exploiting intrinsically low-order nonlinearities for enantio-sensitive detection in the X-ray domain~\cite{Rouxel:2022aa, Rouxel:2022ab}.

\subsection*{Online content}


\section*{Methods}

\subsection*{Spatial structure of vortex beams creating chiral topological light}

We use two Laguerre-Gaussian beams with  counter-rotating circular polarizations, propagating along the $z$-axis with frequencies $\omega$ and $2\omega$ and orbital angular momenta $\ell_\omega$ and $\ell_{2\omega}$. We set the radial indices to $p_\omega=p_{2\omega}=0$. The generalization to the case of non-zero radial index is straightforward. At the focal plane of the beams $z=0$, the Cartesian components of the fields in the transversal plane $(x,y)$ are
\begin{eqnarray}
\mathbf{E}^{\perp}_{\pm,r\omega}=
&&
\mathcal{E}_{r\omega}e^{-\frac{\rho^2}{W^2_0}}\left(\frac{\sqrt{2}\rho}{W_0}\right)^{|\ell_{r\omega}|}e^{\text{i}\ell_{r\omega}\theta}e^{\text{i}\phi_{r\omega}}
\nonumber \\
&&
\times 
\frac{(\mathbf{e}_x-\text{i}m_{r\omega}\mathbf{e}_y)}{\sqrt{2}}
,
\end{eqnarray}
% % %
where $\mathcal{E}_{r\omega}$
%=\sqrt{I_{r\omega}}$ 
is the field strength, $W_0$ is the beam waist, $\phi_{r\omega}$ is the carrier-envelope phase (CEP), $\rho=\sqrt{x^2+y^2}$ and $\theta=\arctan(y/x)$ are the radial and azimuthal coordinates, and $m_{r\omega}$ indicates right ($m_{r\omega}=1$) or left ($m_{r\omega}=-1$) circular polarization.
Near the focus this field develops a longitudinal component along the $z$-axis given by $E_z=-(\text{i}/k)\nabla_{\perp}\cdot\mathbf{E}_{\perp}$ in the first post-paraxial approximation~\cite{Bliokh:2015aa}:
%
\begin{eqnarray}\mathbf{E}^z_{\pm,r\omega}=&&-\frac{\text{i}\mathcal{E}_{r\omega}}{\sqrt{2}k_{r\omega}}e^{-\frac{\rho^2}{W^2_0}}\left(\frac{\sqrt{2}}{W_0}\right)^{|\ell_{r\omega}|}\rho^{|\ell_{r\omega}|-1}\\\nonumber
&&\times e^{\text{i}(l_{r\omega}+m_{r\omega})\theta}\left(|\ell_{r\omega}|-m_{r\omega}\ell_{r\omega}-\frac{2\rho^2}{W^2_0}\right)\mathbf{e}_z,
\end{eqnarray}
%
The total bichromatic electric field $\mathbf{E}(x,y)=\mathbf{E}_{\pm,\omega}+\mathbf{E}_{\pm,2\omega}$, combining the longitudinal and transverse field components for each color $\mathbf{E}_{\pm,r\omega}=\mathbf{E}^{\perp}_{\pm,r\omega}+\mathbf{E}^{z}_{\pm,r\omega}$ ($r=1,2$), is an example of a synthetic chiral light~\cite{Ayuso:2019aa}.

\subsection*{Chiral correlation function}

We report here the analytical expression for the chiral correlation function~\cite{Ayuso:2019aa} $h^{(5)}(-2\omega,-\omega,\omega,\omega,\omega)=\mathbf{E}^*(2\omega)\cdot\left[\mathbf{E}^*(\omega)\times\mathbf{E}(\omega)\right]\left(\mathbf{E}(\omega)\cdot\mathbf{E}(\omega)\right)$ for the general case of two OAM-carrying beams with frequencies $\omega$ and $2\omega$, SAM $m_{\omega}$ and $m_{2\omega}$ and OAM $\ell_\omega$ and $\ell_{2\omega}$. 
%

\begin{eqnarray}&&h^{(5)}(\rho,\theta)=-\frac{\mathcal{E}_{2\omega}\mathcal{E}^4_\omega}{\sqrt{2}2k^2_\omega}e^{-5\frac{\rho^2}{W^2_0}}\left(\frac{\sqrt{2}}{W_0}\right)^{4|\ell_\omega|+|\ell_{2\omega}|}\nonumber\\
&&\rho^{4|\ell_{\omega}|+|\ell_{2\omega}|-3}\left(|\ell_{\omega}|-m_\omega\ell_\omega-\frac{2\rho^2}{W^2_0}\right)^2\nonumber\\
&&\left\{\frac{|\ell_\omega|-m_\omega\ell_\omega-2\frac{\rho^2}{W^2_0}}{2k_\omega}\left[e^{im_\omega\theta}(m_\omega-m_{2\omega})-\nonumber\right.\right.\\
&&\left.e^{-im_\omega\theta}(m_\omega+m_{2\omega})\right]\nonumber\\
&&\left.+\frac{m_{2\omega}}{k_{2\omega}}(|\ell_{2\omega}|-m_{2\omega}\ell_{2\omega}-\frac{2\rho^2}{W^2_0})e^{-im_{2\omega}\theta}\right\}\nonumber\\
&&e^{i(2\phi_\omega-\phi_{2\omega})}e^{i(2\ell_\omega+2m_\omega-\ell_{2\omega})\theta}\end{eqnarray}
%
It is easy to verify that both in the counter-rotating $m_\omega=-m_{2\omega}$ and co-rotating case $m_\omega=m_{2\omega}$ the azimuthal dependence of the chiral correlation function is given by $C\theta$, where $C=2(\ell_\omega+m_\omega)-(\ell_{2\omega}+m_{2\omega})$.

\subsection*{DFT-based SFA simulations in fenchone}

The method was adapted from Refs. \cite{Ayuso:2019aa,Ayuso:2018aa,Ayuso:2018ab,Smirnova:2014aa} to describe HHG in a chiral molecule subjected to a strong-field. The macroscopic dipole moment in an ensemble of randomly oriented molecules arises form the coherent summation of the contributions from all possible molecular orientations
%
$$\mathbf{D}(N\omega)=\int d\Omega \int d\beta\,\mathbf{D}_{\Omega\beta}(N\omega)$$
%
where $\omega$ is the fundamental frequency, $N$ is the harmonic number and $\mathbf{D}_{\Omega\beta}$ is the harmonic dipole associated with a molecular orientation characterized by the three Euler angles, here denoted in terms of the solid angle $\Omega$ and the angle $\beta$. In the Strong-Field Approximation, the harmonic dipole for a given orientation is given by \cite{Ayuso:2019aa,Smirnova:2014aa}
%
\begin{eqnarray}\mathbf{D}_{\Omega\beta}(N\omega)=&&e^{\text{i}N\omega t'_r}a_{\mathrm{rec}}\,\mathbf{d}(\mathrm{U}_{\Omega\beta}\mathrm{Re}[\mathbf{k}(t'_r)])a_{\mathrm{prop}}\,\nonumber\\
&&e^{-\text{i}S(\mathbf{p}_s,t_i,t_r)}\,a_{\mathrm{ion}}\,\Psi_{D}(\mathrm{U}_{\Omega\beta}\mathrm{Re}[\mathbf{k}(t'_i)])\end{eqnarray}
%
where $\mathbf{d}(\mathbf{k})$ is the recombination matrix element in the laboratory frame and $\mathbf{k}(t)=\mathbf{p}+\mathbf{A}(t)$. Here $\mathrm{U}_{\Omega\beta}$ is the rotation matrix that transforms the laboratory frame $(\mathbf{e}_1,\mathbf{e}_2,\mathbf{e}_3)$ to the molecular $(\mathbf{i}_1,\mathbf{i}_2,\mathbf{i}_3)$ frame, with elements $\mathrm{U}_{ij}=\langle\mathbf{e}_i|\mathbf{i}_j\rangle$ for a given orientation.Here $\Psi_D(\mathbf{k})=\langle\mathbf{k}|\Psi_D\rangle$ is the overlap between the Volkov state with kinetic momentum $\mathbf{k}$ and the Dyson orbital, where the latter is the overlap between the neutral N-electron wavefunction and the ionic N-1-electron wavefunction $|\Psi_D\rangle=\langle\Psi^{N-1}|\Psi^{N}\rangle$. The integral over the solid angle $d\Omega=d\alpha d\beta\sin(\beta)$ is performed using the Lebedev quadrature method \cite{Lebedev:1975aa}, while the integral over the $\beta$ angle is done by trapezoid method. In order to find the rotation matrix, we first assume that the $x$-axis of the molecular frame points toward a given Lebedev point, and then rotate by an angle $\beta$ around the $x$-axis. For all simulations we used a 17th-order Lebedev quadrature (for a total of 110 points) and 40 $\beta$ angles evenly distributed on the $[0,2\pi]$ interval.

In the expression for the harmonic dipole, $\mathbf{p}$, $t_i=t'_i+\text{i}t''_i$, $t_r=t'_r+\text{i}t''_r$ are the complex momenta and times of ionization and recombination resulting from the application of the saddle-point method \cite{Smirnova:2014aa}. $S(\mathbf{p},t_i,t_r)=\frac12 \int_{t_i}^{t_r}dt'\,\left[\mathbf{p}+\mathbf{A}(t')\right]^2+I_p(t_r-t_i)$ is the action from the (complex) times of ionization and recombination. The terms associated with the saddle-point method on $(t_i,t_r,\mathbf{p})$ are given by
%
$$a(\mathbf{p},t_i,t_r)=a_\mathrm{ion}a_\mathrm{prop}a_\mathrm{rec}$$
$$a_\mathrm{ion}=\sqrt{\frac{2\pi}{\partial^2_{t_i}S}}$$
$$a_\mathrm{rec}=\sqrt{\frac{2\pi}{\partial^2_{t_r}S}}$$
$$a_\mathrm{prop}=\left(\frac{2\pi}{\text{i}(t_r-t_i)}\right)^{3/2}$$
%
where the second derivatives of the action are given explicitly by
%
$$\partial^2_{t_i}S=-\mathbf{E}(t_i)\cdot\mathbf{k}(t_i)$$
$$\partial^2_{t_i}S=\mathbf{E}(t_r)\cdot\mathbf{k}(t_r)$$
%
where $\mathbf{E}(t)$ is the electric field
and all expressions for the prefactor are calculated at the complex times.

The transition matrix elements of the right and left-handed molecules are related by 
%
\begin{equation}
\mathbf{D}_R(\mathbf{k})=-\mathbf{D}_L(-\mathbf{k})
\end{equation}
%
while for the overlap between the Dyson orbital and the Volkov wavefunction we have that
%
\begin{equation}\Psi^R_D(\mathbf{k})=\Psi^L_D(-\mathbf{k})\end{equation}
%
The matrix elements and the Dyson orbitals for fenchone are calculated using DFT methods described in \cite{Ayuso:2021aa, TOFFOLI200225}.

\subsection*{Multiphoton picture}

The multiphoton picture of enantiosensitive HHG driven by chiral topological light can be understood by analyzing the contributing chiral and achiral multiphoton pathways. To do so, we classify the multiphoton pathways by indicating with a subscript the SAM of the photon, so that e.g. $(N)\omega_+$ indicate the absorption of $N$ $\omega$ photons with SAM $m=1$ and $(-1)\omega_z$ indicates the emission of one $\omega$ photon with SAM $m=0$.

In the specific case of bicircular counter-rotating fields, if the field has no longitudinal component along its direction of propagation (i.e.\ if we consider an achiral field in the dipole approximation), conservation of SAM results in a harmonic spectrum with doublets at $3N+1$ and $3N+2$ harmonic frequencies, where the $3N+1$ harmonics ($3N+2$) co-rotate with the $\omega$ ($2\omega$) field \cite{Fleischer:2014aa,Hickstein:2015aa}. $3N$ harmonic orders are forbidden in achiral media, since their generation requires absorption of an equal number of photons from both drivers. In chiral media, the $3N$ harmonic orders can instead be generated due to the broken parity of the medium, but are polarized along the direction of propagation of the fields (the $z$-axis in our case), and thus are not detectable in the far-field. We label this pathway as
%
\begin{equation}C_z=\left[(N)\omega_+,(N)2\omega_-\right].\end{equation}
%
Focusing on the specific case of $3N$ harmonic orders, if the field is chiral (i.e.\ if it posses a longitudinal component along the propagation direction) in the case of achiral media the following multiphoton pathways can now lead to symmetry-allowed HHG:
%
\begin{eqnarray}\mathrm{AC}_+&=&\left[(N-2)\cdot\omega_+, (2)\omega_z, (N-1)\cdot2\omega_-\right]\\
\mathrm{AC}_-&=&\left[(N-1)\cdot\omega_+,(-1)\omega_z, (N)\cdot2\omega_{-}, (1)2\omega_z\right]\end{eqnarray}
%
corresponding respectively to the emission of a photon with SAM $m=1$ and $m=-1$. We label these pathways as achiral pathways (i.e. $\mathrm{AC}_{m}$, with $m$ the SAM of the harmonic photon), since they occur already in achiral media driven by a chiral field as they require the absorption and emission of an odd number of photons. If the medium is chiral, two new pathways including absorption of an equal number of $\omega$ and $2\omega$ photons open, i.e.
%
\begin{eqnarray}\mathrm{C}_+&=&\left[(N)\cdot\omega_+, (N-1)\cdot2\omega_-, (1)2\omega_z\right]\\
\mathrm{C}_-&=&\left[(N-1)\cdot\omega_+,(1)\omega_z, (N)\cdot2\omega_{-}\right]\end{eqnarray}
%
corresponding again respectively to the emission of a photon with SAM $m=1$ and $m=-1$. We label these pathways as chiral pathways ($\mathrm{C}_m$) since they can occur only in chiral media. Finding the corresponding OAM of all pathways indicated above is straightforward, once we remember that the longitudinal components of the fields carry OAMs of $\ell_{\omega_z}=\ell_{\omega_+}+m_{\omega}$ and $\ell_{2\omega_z}=\ell_{2\omega_-}+m_{2\omega}$. Obviously, other chiral and achiral pathways including the absorption of a larger number of z-polarized photons from either drivers are also in principle accessible: yet, since the longitudinal component is relatively weak, we restrict ourselves here to the photon pathways that include the absorption or emission of the fewest number of z-polarized photons. Fig. 1a of the SI) shows schematically the multiphoton pathways $\mathrm{C}_z$, $\mathrm{AC}_{m}$ and $\mathrm{C}_m$ for the case of a 3N harmonic order.

The results from the SFA simulations confirm the considerations above; in Fig.~1b  of the SI) we show the near-field OAM distributions for H18 in (R)fenchone driven by a field with $\ell_\omega=-\ell_{2\omega}=1$ and $m_{\omega}=-m_{2\omega}=1$. For comparison, we also report the OAM content for an artificial atom with ionization potential equal to fenchone driven by the same chiral field and the OAM content in fenchone for an achiral field with same OAM of the driving beams, obtained by manually setting the longitudinal component of the field to zero.\\
When the field is achiral, H18 in an atom is absent, while in the case of fenchone we observe a $\ell=0$ component polarized along the z-axis: this corresponds to the pathway $\mathrm{C}_z$ denoted above. When the field is chiral, circularly polarized components with $\ell=\pm5$ are observed for both the atom and the molecule: these are the achiral pathways $\mathrm{AC}_{+}$ and $\mathrm{AC}_{-}$ denoted above. Finally, the chiral pathways $\mathrm{C}_{+}$ and $\mathrm{C}_{-}$ correspond to the OAMs $\ell=\pm1$ and are only seen in a chiral molecule, since they require the absorption of an even number of photons.
Note that in the far-field only the SAM $m=\pm1$ components are going to be observed, since $m=0$ polarization (corresponding to the $\mathrm{C}_z$ pathway in black in Fig.~1 of the SI) will propagate in a direction orthogonal with respect to the propagation axis of the beams.

The different OAM content of an atom and chiral molecule driven by a chiral bicircular field is directly reflect in the far-field profile of H18, shown in Fig. 1c of the SI). In an atom (left figure of Fig. 1 of the SI), where for a given SAM there is only one contributing OAM, the far-field profile of H18 is a ring where the intensity is mostly constant, while in fenchone we observe an azimuthal interference pattern with periodicity determined by the topological charge $C$, corresponding in modulus to the net difference between the OAMs of chiral and achiral pathways. The enantio-sensitive rotation of the spatial profile can be understood from the perspective of the multiphoton pathways by accounting a shift by $\pi$ of the phase of the chiral pathways $\mathrm{C}_\pm$ when changing the molecular enantiomer.
The enantio-sensitive rotation of the spatial profile of the high-harmonics in the far-field allows one also to use HHG driven by chiral vortices as a highly-sensitive method to infer the enantiomeric excess in a mixture of right and left molecular enantiomers.

Next order pathways can be identified using the same approach. In the case of achiral channels the next order pathway includes the absorption of two more longitudinal photons (see Fig. 2 of the SI) and is respectively two order of magnitude smaller. The next order chiral pathway is 4 order of magnitude smaller, corresponding to the absorption of 4 more longitudinal photons, and so on.

As mentioned in the main text in the case of an elliptically polarized $\omega$ field two new achiral pathways dominate the response, whose photon diagrams we report in Fig. 3 of the SI. For a 3N harmonic order both new achiral pathways contribute to the final SAM of $m=-1$ and are in particular
%
\begin{eqnarray}\mathrm{AC}^{\epsilon}_{1}&=&\left[(N-2)\cdot\omega_+,(2)\omega_-,(N-1)\cdot2\omega_-\right]\\
\mathrm{AC}^\epsilon_2&=&\left[(N-1)\cdot\omega_+,(-1)\omega_-,(N+1)\cdot2\omega_-\right]\end{eqnarray}
%
where $\omega_-$ refers now to the counter-rotating component of the elliptically polarized field at $\omega$ frequency. Since each elliptically polarized photon carries a phase delay dependece of $\exp(\text{i}\delta)$, the interference between these two achiral pathways oscillates with respect to the phase delay as $3\delta$. This explains why choosing the $\tilde{\delta}=1$ component of the harmonic profile after Fourier analysis allows one to recover the enantiosensitive rotation of the spatial profile.

\subsection*{Noise (intensity fluctuations) simulations}

In order to include the effect of noise on HHG driven by chiral vortex light, we take the following approach.
For a given electric field strength $E_0$ (which we assume to be the same for both fields) the Laguerre-Gaussian beam is given in the near-field by $\mathbf{E}(\mathbf{r})=E_0\mathbf{LG}_{l,p}(\mathbf{r})$, where $\mathbf{LG}_{l,p}=LG_{l,p}(\mathbf{r})\mathbf{e}_L(\mathbf{r})$. Here $LG_{l,p}$ is a Laguerre-Gaussian mode and $\mathbf{e}_L$ is the polarization vector of the field. The corresponding laser intensity is $I_0=|E_0|^2$. We then pick a value for the laser intensity from a normal distribution of noise centered at $I_0$ with width $\gamma$. We call this electric field intensity $I_1$. Then, for each point $\mathbf{r}$ in the focus, we introduce intensity fluctuations such that at a given position the electric field strength is given by
%
\begin{equation}\mathcal{I}(\mathbf{r})=I_1\,LG_{l,p}(\mathbf{r})(1+\delta_I(\mathbf{r}))\end{equation}
%
where $\delta_I(\mathbf{r})=C\lambda(\mathbf{r})$. $\lambda(\mathbf{r})$ is chosen from a Gaussian distribution centered at zero with width 1 and $C=0.1$ is a constant. There is therefore 68.2\% probability that the fluctuation will be below 0.1\% of the signal at the given point. We produce 16 electric fields using this approach, choosing a central intensity of $I_0=5\cdot10^{14}$ W/cm$^2$ with width $\gamma=3.51\cdot10^{13}$ W/cm$^2$, and calculate the resulting far-field picture for left and right fenchone. The average intensity fluctuations are on the order of 2\%, on par with standard experimental parameters \cite{Astrella}. We then scan the enantiomeric excesses $ee$ between $-100\%$ and $100\%$ in 1001 steps. For each step, we pick a random index $i$ between 1 and 16, selecting one of the far-field profiles for right- and left-fenchone $\mathbf{d}^{R/L}_i$. The resulting far-field image at a given enantiomeric excess $ee=(C_R-C_L)/(C_R+C_L)$ for normalized concentrations $C_R+C_L=1$ is given by $\mathbf{d}^{ee}=C_R\mathbf{d}^{R}_i+C_L\mathbf{d}^{L}_i$ and the phase of the $\ell=6$ Fourier component of the outer ring $kW_0>10$ is then calculated. We then repeat the procedure 16 times and for each enantiomeric excess calculate the mean phase as $\bar{\phi}=\sum_{i=1}^{16}\phi_i/16$. The result is the red solid line shown in Fig. 3.


\subsection*{Data availability}

The data that supports the plots within this paper and other findings of this study are available from the corresponding authors upon reasonable request.

\subsection*{Acknowledgements}

The authors would like to acknowledge helpful discussions with A.\ Ord\'{o}\~{n}ez and O. Kornilov.
O.S.\ and M.I.\ acknowledge the hospitality of the Technion -- Israel Institute of
Technology, especially during the week of October 8, 2023. This
project has received funding from the EU Horizon 2020 programme (grant 
agreement No 899794) and European Union (ERC, ULISSES, 101054696).
D.A., M.K.\ and E.P.\ acknowledge Royal Society funding under 
URF\textbackslash{}R1\textbackslash{}201333, URF\textbackslash{}R1\textbackslash{}231460, and URF\textbackslash{}R1\textbackslash{}211390.

% --- BIBLIOGRAPHY --- %
%
\bibliographystyle{unsrt}
\bibliography{biblio}


\end{document}