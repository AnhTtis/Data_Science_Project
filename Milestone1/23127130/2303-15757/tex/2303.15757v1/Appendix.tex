\section{Effective Hamiltonian}
\label{sec:Effective_Hamiltonian}

%\begin{table*}
%\caption{Effective Hamiltonian: We present for different resonant momenta the contributions of the asymptotic expansion of  $\hat{H}_\text{eff}$  in orders  of $\alpha_n$ in the low-gain regime and in orders of $\varepsilon$ in the high-gain regime, respectively. For the first and the second resonance, $p=q/2$ and $p=q$, we give the effective Hamiltonian up to third order. Moreover, we have calculated the fourth-order contribution in the case of $p=q$. For the third resonance $p=3q/2$ we have restricted ourselves to the low-gain regime.}
%\begin{tabular}{lcc}
%   \toprule
%  resonance & low gain: $ \hat{H}_\text{eff}\cong$ &  high gain: %$\hat{H}_\text{eff}\cong$ \\
%   \midrule
%   $\displaystyle p=\frac{q}{2}$ 
%   &
%   \begin{math}\displaystyle
%   \alpha_n\left(\hat{\sigma}_{1,0}+\hat{\sigma}_{0,1}\right)
%   \end{math}
%   & 
%   \begin{math} \displaystyle  
%     \varepsilon\left(\hat{a}_{\L} \hat{\Upsilon}_{1,0} 
%     + \hat{a}_{\L}^\dagger \hat{\Upsilon}_{0,1}  \right)
%    \end{math}
%   \\ 
%    & 
%    \begin{math}\displaystyle
%     + \alpha_n^2\left[
%      -\frac{1}{2}\left(\hat{\sigma}_{0,0}+\hat{\sigma}_{1,1}\right)
%      +\sum\limits_{\mu\neq0,1} %\frac{\hat{\sigma}_{\mu,\mu}}{2\mu(\mu-1)}\right]
%    \end{math}
%    &
%     \begin{math} \displaystyle  
%     +\frac{\varepsilon^2}{2}\left[\left(\hat{n}+1\right)
%     \sum\limits_{\mu\neq 0}\frac{1}{\mu}
%     \left(\hat{\Upsilon}_{\mu+1,\mu+1}-\hat{\Upsilon}_{\mu,\mu} %\right)-
%     \sum\limits_{\mu\neq %0}\frac{1}{\mu}\hat{\Upsilon}_{\mu+1,\mu}\hat{\Upsilon}_{\mu,\mu+1}\righ%t]
%    \end{math}    
%    \\
%    &
%    \begin{math}\displaystyle
%     +\alpha_n^3\left[-\frac{1}{4}
%      \left(\hat{\sigma}_{0,1}+\hat{\sigma}_{1,0}-\hat{\sigma}_{-1,2}-\h%at{\sigma}_{2,-1}   
%      \right)\right]
%    \end{math}
%    & 
%     \begin{math} \displaystyle  
%     \begin{aligned}
%    + \frac{\varepsilon^3}{4} \!\left\{
%     \hat{a}_{\L}\!\!\!\!\!\sum\limits_{\mu \neq %-1,0}\!\!\frac{\hat{\Upsilon}_{2\mu +2,2\mu +1}
%     \hat{\Upsilon}_{\mu,\mu+2}}{\mu(\mu+1)(2\mu+1)}
%     \!+\!\frac{3\hat{a}_{\L}}{2}\left(\hat{\Upsilon}_{0,-1}\hat{\Upsilo%n}_{-1,1}-\!\hat{\Upsilon}_{0,2}\hat{\Upsilon}_{2,1}\right)
%     \right.
%     \\
%     \left.
%     -\left[\sum\limits_{\mu\neq 0}\frac{1}{2\mu^2}
%     \!\left(\hat{\Upsilon}_{\mu+1,\mu+1}- \!\hat{\Upsilon}_{\mu,\mu} %\right) \!+\!\hat{n} \!+\!\frac{1}{2}\right]
%    \hat{a}_{\L}\hat{\Upsilon}_{0,1}
%     %\\
%     %\left.
%     +\hat{a}_{\L}^3\hat{\Upsilon}_{-1,2}+\text{h.c.}\right\}
%     \end{aligned}    
%    \end{math}
%\\
%    \midrule
%$
%p=q$
%&
%
%%& 

% \\ 
%      &
    
%      &

%   \\   
%     &
%     \begin{math}
%      \alpha_n^2\left(
%      \hat{\sigma}_{0,2}+\hat{\sigma}_{2,0}
%      +\sum\limits_{\mu} %\frac{2\,\hat{\sigma}_{\mu,\mu}}{(2\mu-3)(2\mu-1)}\right)
%    \end{math}  
%     &
%    \begin{math}
%    \begin{aligned}
%       \varepsilon^2 %\left[\hat{a}_{\L}{}^2\hat{\Upsilon}_{0,2}+\hat{a}_{\L}^     
%       \dagger{}
%      ^2\hat{\Upsilon}_{2,0}+\hat{n}\!\sum\limits_\mu\frac{\hat{\Upsilon%}_{\mu+1,\mu+1}}{2\mu-1}
%   \!- \left(\hat{n}+1\right)\sum\limits_\mu\frac{\hat{\Upsilon}_{\mu,\m%u}}{2\mu-1}\right.\\
%      +\!\sum\limits_\mu\!\frac{1}{4\mu -2}\!
%    \left(\hat{\Upsilon}_{\mu+1,\mu+1}\!-\!\hat{\Upsilon}_{\mu+1,\mu}\ha%t{\Upsilon}_{\mu,\mu   +1}\right)
%    \\ \left.
%    +\sum\limits_\mu\frac{1}{4\mu-2} %\left(\hat{\Upsilon}_{\mu,\mu}-\hat{\Upsilon}_{\mu,\mu+1}\hat{\Upsilon}_%{\mu+1,\mu}\right)\right]
%      \end{aligned}
%    \end{math}  
%   \\ 
% &
% \begin{math}\displaystyle
%  \begin{aligned}
%+ \, \alpha_n^4\left[-\frac{16}{9}(\hat{\sigma}_{2,0}+\hat{\sigma}_{0,2}%)
%+\frac{1}{36}(\hat{\sigma}_{3,-1}+\hat{\sigma}_{-1,3})\right.\\ \left. 
%+\!\!\sum\limits_{\mu\neq %0}\frac{\hat{\sigma}_{\mu+1,\mu+1}+\hat{\sigma}_{\mu,\mu}}{64\mu %$(\mu^2-1/4)^2}
%-\!\!\sum\limits_{\mu}\frac{\hat{\sigma}_{\mu+1,\mu+1}-\hat{\sigma}_{\mu%,\mu}}{8(\mu-1/2)^3 \left[(\mu-1/2)^2-1\right]^2}\right]
%\end{aligned}
% \end{math} 
% &
% \\
%    \midrule   
% $\displaystyle p=\frac{3q}{2}$    
% &
%$\alpha_n\left(\hat{\sigma}_{1,2}+\hat{\sigma}_{2,1}\right)$
% &   
% \\  
% &
% \begin{math}\displaystyle
%   +\alpha_n^2\left[
%  -\frac{1}{2}\left(\hat{\sigma}_{1,1}+\hat{\sigma}_{2,2}\right)
%  +\sum\limits_{\mu\neq 1,2} %\frac{\hat{\sigma}_{\mu,\mu}}{2(\mu-1)(\mu-2)}\right]
% \end{math}
% &
% \\
% & 
% \begin{math}\displaystyle
% +\alpha_n^3\left[\frac{1}{4}
%\left(\hat{\sigma}_{0,3}+\hat{\sigma}_{3,0}-\hat{\sigma}_{1,2}-\hat{\sig%ma}_{2,1}\right)\right]
% \end{math}
 %&
 %\\   
 %   \bottomrule
 % \end{tabular}
 % \label{tab:eff_Ham} 
%\end{table*} 
    
\begin{table*}
\caption{Effective Hamiltonian: We present for different resonant momenta the contributions of the asymptotic expansion of  $\hat{H}_\text{eff}$  in orders  of $\alpha_n$ in the low-gain regime and in orders of $\varepsilon$ in the high-gain regime, respectively. For the first and the second resonance, $p=q/2$ and $p=q$, we give the effective Hamiltonian up to third order. Moreover, we have calculated the fourth-order contribution in the case of $p=q$. For the third resonance $p=3q/2$ we have restricted ourselves to the low-gain regime.}
\begin{tabular}{lcc}
   \toprule
   & low gain: $ \hat{H}_\text{eff}\cong$ &  high gain: $\hat{H}_\text{eff}\cong$ \\
   \midrule
   \begin{tabular}{l}
      $\displaystyle p=\frac{q}{2}$ 
   \end{tabular}
   &
    \begin{tabular}{c}
      \begin{math}\displaystyle
        \alpha_n\left[\hat{\sigma}_{1,0}+\hat{\sigma}_{0,1}\vphantom{\hat{a}_{\text{L}}^\dagger \hat{\Upsilon}_{0,1}}\right]
        \end{math}
        \\
        \begin{math}\displaystyle
             + \alpha_n^2\left[
              -\frac{1}{2}\left(\hat{\sigma}_{0,0}+\hat{\sigma}_{1,1}\right)
              +\sum\limits_{\mu\neq0,1} \frac{\hat{\sigma}_{\mu,\mu}}{2\mu(\mu-1)}\right]
        \end{math}
        \\
        \begin{math}\displaystyle
            -\frac{\alpha_n^3}{4}
            \left[\hat{\sigma}_{0,1}+\hat{\sigma}_{1,0}-\hat{\sigma}_{-1,2}-\hat{\sigma}_{2,-1}   
            \right] \vphantom{\Bigg[}
        \end{math}
        \\
        \begin{math}\displaystyle
           \vphantom{\Bigg[}
        \end{math}
   \end{tabular}
    &
    \begin{tabular}{c}
       \begin{math} \displaystyle  
             \varepsilon\left[\hat{a}_{\text{L}} \hat{\Upsilon}_{1,0} 
             + \hat{a}_{\text{L}}^\dagger \hat{\Upsilon}_{0,1}  \right]
        \end{math}
        \\
        \begin{math} \displaystyle  
             +\frac{\varepsilon^2}{2}\left[\left(\hat{n}+1\right)
             \sum\limits_{\mu\neq 0}\frac{1}{\mu}
             \left(\hat{\Upsilon}_{\mu+1,\mu+1}-\hat{\Upsilon}_{\mu,\mu} \right)-
             \sum\limits_{\mu\neq 0}\frac{1}{\mu}\hat{\Upsilon}_{\mu+1,\mu}\hat{\Upsilon}_{\mu,\mu+1}\right]
        \end{math}
        \\
         \begin{math} \displaystyle  
            + \frac{\varepsilon^3}{4} \!\Bigg[
             \hat{a}_{\text{L}}\!\!\!\!\!\sum\limits_{\mu \neq -1,0}\!\!\frac{\hat{\Upsilon}_{2\mu +2,2\mu +1}
             \hat{\Upsilon}_{\mu,\mu+2}}{\mu(\mu+1)(2\mu+1)}
             \!+\!\frac{3\hat{a}_{\text{L}}}{2}\left(\hat{\Upsilon}_{0,-1}\hat{\Upsilon}_{-1,1}-\!\hat{\Upsilon}_{0,2}\hat{\Upsilon}_{2,1}\right)
        \end{math}
        \\
        \begin{math} \displaystyle 
           \phantom{+ \frac{\varepsilon^3}{4}} -\left(\sum\limits_{\mu\neq 0}\frac{\hat{\Upsilon}_{\mu+1,\mu+1}- \!\hat{\Upsilon}_{\mu,\mu}}{2\mu^2}
             \! +\!\hat{n} \!+\!\frac{1}{2}\right)
            \hat{a}_{\text{L}}\hat{\Upsilon}_{0,1}
             +\hat{a}_{\text{L}}^3\hat{\Upsilon}_{-1,2}+\text{h.c.}\Bigg]
        \end{math}        
    \end{tabular}
    \\
    \midrule
   \begin{tabular}{l}
      $\displaystyle p=q$ 
   \end{tabular}
    &
    \begin{tabular}{c}
        \begin{math}\displaystyle
          \alpha_n^2\left[
          \hat{\sigma}_{0,2}+\hat{\sigma}_{2,0}
          +\sum\limits_{\mu} \frac{2\,\hat{\sigma}_{\mu,\mu}}{(2\mu-3)(2\mu-1)}\right]
        \end{math}
        \\
         \begin{math}\displaystyle
            + \, \alpha_n^4\Bigg[-\frac{16}{9}(\hat{\sigma}_{2,0}+\hat{\sigma}_{0,2})
            +\frac{1}{36}(\hat{\sigma}_{3,-1}+\hat{\sigma}_{-1,3})
        \end{math}
        \\
        \begin{math}\displaystyle
            %+\!\!\sum\limits_{\mu\neq 0}\frac{\hat{\sigma}_{\mu+1,\mu+1}+\hat{\sigma}_{\mu,\mu}}{64\mu (\mu^2-1/4)^2}
            -\sum\limits_{\mu}\frac{\hat{\sigma}_{\mu+1,\mu+1}-\hat{\sigma}_{\mu,\mu}}{8(\mu-1/2)^3 \left\lbrace(\mu-1/2)^2-1\right\rbrace^2}
            %\Bigg]
        \end{math}
        \\
        \begin{math}\displaystyle
            +\!\!\sum\limits_{\mu\neq 0}\frac{\hat{\sigma}_{\mu+1,\mu+1}+\hat{\sigma}_{\mu,\mu}}{64\mu (\mu^2-1/4)^2}
            %-\sum\limits_{\mu}\frac{\hat{\sigma}_{\mu+1,\mu+1}-\hat{\sigma}_{\mu,\mu}}{8(\mu-1/2)^3 \left[(\mu-1/2)^2-1\right]^2}
            \Bigg]
        \end{math}
    \end{tabular}
     &
    \begin{tabular}{c}
        \begin{math}\displaystyle
            \varepsilon^2 \Bigg[\hat{a}_{\text{L}}{}^2\hat{\Upsilon}_{0,2}+ \text{ h.c.} +\hat{n}\!\sum\limits_\mu\frac{\hat{\Upsilon}_{\mu+1,\mu+1}}{2\mu-1} - \left(\hat{n}+1\right)\sum\limits_\mu\frac{\hat{\Upsilon}_{\mu,\mu}}{2\mu-1}
        \end{math}
        \\
        \begin{math}\displaystyle
        \phantom{ \varepsilon^2 \Bigg[}
           +\!\sum\limits_\mu\!\frac{\hat{\Upsilon}_{\mu+1,\mu+1}+\hat{\Upsilon}_{\mu,\mu}\!-\!\hat{\Upsilon}_{\mu+1,\mu}\hat{\Upsilon}_{\mu,\mu   +1}-\hat{\Upsilon}_{\mu,\mu+1}\hat{\Upsilon}_{\mu+1,\mu}}{4\mu -2}\!\Bigg]
        \end{math} 
        \\
        \begin{math}\displaystyle
           \phantom{
            -\sum\limits_{\mu}\frac{\hat{\sigma}_{\mu+1,\mu+1}}{\left\lbrace(\mu)^2\right\rbrace^2}
            }
        \end{math}
        \\
        \begin{math}\displaystyle
            \phantom{
            \sum\limits_{\mu\neq 0}\frac{\hat{\sigma}_{\mu+1,\mu+1}}{ (\mu^2)^2}
            \Bigg]
            }
        \end{math}
    \end{tabular}
    \\
    \midrule   
   \begin{tabular}{l}
      $\displaystyle p=\frac{3q}{2}$
    \end{tabular}
    &
    \begin{tabular}{c}
        $ \displaystyle \alpha_n\left[\hat{\sigma}_{1,2}+\hat{\sigma}_{2,1}\right]$
        \\
        \begin{math}\displaystyle
            +\alpha_n^2\left[
            -\frac{1}{2}\left(\hat{\sigma}_{1,1}+\hat{\sigma}_{2,2}\right)
            +\sum\limits_{\mu\neq 1,2} \frac{\hat{\sigma}_{\mu,\mu}}{2(\mu-1)(\mu-2)}\right]
        \end{math}
        \\
         \begin{math}\displaystyle
            +\frac{\alpha_n^3}{4}\left[
            \hat{\sigma}_{0,3}+\hat{\sigma}_{3,0}-\hat{\sigma}_{1,2}-\hat{\sigma}_{2,1}\right]
        \end{math}
    \end{tabular}
    &   
  \\   
    \bottomrule
  \end{tabular}
  \label{tab:eff_Ham} 
\end{table*} 
%%%%PETER: mag meine Tabelle lieber, weil man da einen besseren Vergleich der verschiedenen Ordnungen hat. Ist aber leider wegen der vertikalen Ausrichtung an der mittleren Zeile nicht optimal.    


 
We start with the dimensionless Hamiltonian in the high-gain regime~\cite{PRA2019}
 \begin{equation}
    \label{eq:H_many}
        \hat{H}\!\equiv \! \varepsilon\!\!\sum\limits_\mu \!\left(\e{\I 2\tau\left[\frac{p}{q}-\left(\mu +\frac{1}{2}\right)\right]}\!\hat{a}_{\text{L}}\hat{\Upsilon}_{\mu,\mu+1}\!+ \text{ h.c.}\right)
      \end{equation}
in the interaction picture with the dimensionless time variable $\tau \equiv \omega_\text{r}t$. To obtain the single-electron and semi-classical Hamiltonian for a low-gain FEL, we simply have to replace the collective operators $\hat{\Upsilon}_{\mu,\nu}$ by their single-particle counterparts $\hat{\sigma}_{\mu,\nu}$ and approximate $\hat{a}_\text{L}\approx \hat{a}_\text{L}^\dagger\approx \sqrt{n}\approx \text{const}$. We note that the commutation relation
\begin{equation}
\label{eq:Upsilon_comm}
\left[\hat{\Upsilon}_{\mu,\nu},\hat{\Upsilon}_{\rho,\sigma}\right]
=\updelta_{\nu,\rho}\hat{\Upsilon}_{\mu,\sigma}
-\updelta_{\sigma,\mu}\hat{\Upsilon}_{\rho,\nu}
\end{equation}
for the jump operators is the same for the collective model as in the single-electron limit. However, the properties of products of these operators differ~\cite{PRA2019}.  

The asymptotic method of averaging~\cite{bogoliubov,higher,peter} is suitable for a Hamiltonian $\hat{H}$ which can be represented as a Fourier series in terms of the phase $ \tau$ and its integer multiples. We separate slow and rapid dynamics in the state vector %   
%\begin{equation}
%\label{eq:app_trafo}
$\ket{\Psi(\tau)}\equiv \exp[{-\hat{F}(\tau)]}\ket{\Phi(\tau)}$, %
%\end{equation}
where $\hat{F}$ describes the rapidly varying part, while $\ket{\Phi}$ gives the slowly-varying part. With the help of this ansatz we derive the effective Hamiltonian~\cite{higher}
\begin{equation}\label{eq:H_eff_allg}
\hat{H}_\text{eff}=\sum\limits_{j=0}^\infty\frac{1}{(j+1)!}\left[\hat{F},\I\frac{\D\hat{F}}{\D\tau}\right]_j+\sum\limits_{j=0}^\infty\frac{1}{j!}\left[\hat{F},\hat{H}\right]_j
\end{equation}
of the Schr{\"o}dinger equation for $\ket{\Phi}$, where the subscript $j$ indicates a $j$ times nested commutator.


We proceed by asymptotically expanding $\hat{H}_\text{eff}$ and $\hat{F}$ in powers of $\alpha_n$ -- or in powers of $\varepsilon\equiv g/\omega_\text{r}$ in the high-gain regime. In each order of this expansion we have to ensure that the effective Hamiltonian is independent of time, that is $\hat{H}_\text{eff}\neq \hat{H}_\text{eff}(\tau)$. Hereby, we avoid secular contributions which otherwise lead to unphysically growing terms~\cite{nayfeh}. The dynamics dictated by $\hat{H}_\text{eff}$ can then to be solved non-perturbatively. In contrast, we can rely on perturbation theory for the rapidly-varying dynamics since here the secular terms are excluded by construction.       
 
   

Depending on the specific initial momentum $p=\nu q/2$ with integer $\nu$ we  obtain from Eq.~\eqref{eq:H_many} the explicit  expressions for the Fourier components  of $\hat{H}$.
By inserting these components into $\hat{H}_{\text{eff}}$ from Eq.~\eqref{eq:H_eff_allg} and calculating the occurring commutators we finally obtain the effective Hamiltonian for low and high gain and for different resonances. We have listed the   
explicit expressions in Tab.~\ref{tab:eff_Ham}.


\section{Population of Momentum Levels}
\label{sec:Population_of_Momentum_Levels}

In this appendix we discuss the population probabilities of the momentum levels for an electron in a low-gain FEL resulting from the asymptotic method of averaging. For the first resonance $p=q/2$ we refer to Ref.~\cite{NJP2015}, where the population probabilities for the momentum levels are listed up to third order in $\alpha_n$ for the frequency and up to  second order  for the amplitude. In the following we consider the second and the third resonance. 
  
\subsection{Second resonance}

The initial state of an electron for the second resonance is given by the momentum eigenstate   $\ket{\Psi(0)}=\ket{p}$ with $p=q$. However, due to the transformation from $\ket{\Psi}$ to 
$\ket{\Phi}$ we calculate the transformed initial state 
$\ket{\Phi(0)}=\exp[{\hat{F}(0)]}\ket{\Psi(0)}$ perturbatively up to second order of $\alpha_n$.

%with  
%\begin{equation}
%\label{eq:app_init}
%\hat{S}_\pm (\tau) \equiv \mathds{1}\pm \alpha_n \hat{F}^{(1)}(\tau)
%+\frac{\alpha_n^2}{2}\left(\hat{F}^{(1)}(\tau)\right)^2
%\pm\alpha_n^2 \hat{F}^{(2)}(\tau)\,.
%\end{equation}
%Here we have to explicitly calculate $\hat{F}$ up to second order in $\alpha_n$ with the help of the prescriptions in App.~\ref{sec:Effective_Hamiltonian}. %and the expression for $\hat{\mathcal{H}}_k$.

We expand the state $\ket{\Phi}$  in the discretized momentum basis with probability amplitudes $ \braket{p-\mu q|\Phi(\tau)}$. The Schr{\"o}dinger equation corresponding to the effective Hamiltonian from Tab.~\ref{tab:eff_Ham}  then translates to a system of linear  differential equations which we easily solve with respect to the initial conditions for $\ket{\Phi}$.

Then, we transform the result for  $\ket{\Phi}$ back to the original state 
$\ket{\Psi}$ via the relation $\ket{\Psi(\tau)}=\exp[{-\hat{F}(\tau)]}\ket{\Phi(\tau)}$, and again restrict ourselver to terms up to second order of $\alpha_n$. Finally, we calculate the probabilities %  
%\begin{equation}
$P_{p-\mu q}(\tau)\equiv \left|\braket{p-\mu q|\Psi(\tau)}\right|^2$ %
%\end{equation}
for the population of the momentum levels up to the order $\alpha_n^2$ in amplitude and $\alpha_n^4$ in frequency.

By this procedure, we find the explicit expressions%\begin{widetext}
\allowdisplaybreaks
\begin{align}
%\begin{split}
P_{2q}(\tau)&=\! \frac{\alpha_n^2}{9}\!\left(\cos^2{\xi_1 \tau}\!+\!\cos^2{\xi_2 \tau} \!-\!2\cos{\xi_1\tau} 
\cos{\xi_2\tau}
\cos{\xi_3 \tau}\right)\nonumber \\ \nonumber
P_q(\tau)&= \cos^2{\xi_1 \tau} +2\alpha_n^2\cos{\xi_1\tau} \\ \nonumber
& \ \ \ \ \ \times\left(-\frac{10}{9}\cos{\xi_1\tau}
+\cos{\xi_4\tau}  +\frac{1}{9}\cos{\xi_2\tau}
\cos{\xi_3\tau}\right)\\
%\label{eq:app_P0}
P_{0}(\tau)&=2\alpha_n^2\left\{1-\cos{\left[\left(\xi_1+\xi_4\right)\tau\right]}\right\}\nonumber \\ \nonumber
P_{-q}(\tau)&= \sin^2{\xi_1\tau}
+2\alpha_n^2\sin{\xi_1\tau}\\ \nonumber
& \ \ \ \ \ \times \left(-\frac{10}{9}\sin{\xi_1\tau}
-\sin{\xi_4\tau}+\frac{1}{9}\sin{\xi_2\tau}
\cos{\xi_3\tau}\right)
 \ \ \ \ \ \ 
\\
P_{-2q}(\tau)&= \frac{\alpha_n^2}{9}\!\left(\sin^2{\xi_1\tau}+\sin^2{\xi_2\tau}-2\sin{\xi_1\tau} 
\sin{\xi_2\tau}
\cos{\xi_3\tau}\right) \nonumber
%\end{split}
\end{align}
%\end{widetext}
with 
\allowdisplaybreaks
\begin{align}
%\begin{split}
  \xi_1 & \equiv  \alpha_n^2\left(1-\frac{16\alpha_n^2}{9}\right)
  \nonumber\\
  \xi_2 &\equiv \frac{\alpha_n^4}{36}\sqrt{1+\left(\frac{124}{125}\right)^2} \nonumber\\ 
  \xi_3 &\equiv 3- \frac{8\alpha_n^2}{15}\left(1-\frac{16\alpha_n^2}{5}\right) \nonumber\\
  \xi_4 &\equiv 1+\frac{8\alpha_n^2}{3}\left(1-7\left(\frac{8\alpha_n}{15}\right)^2\right)\,. \nonumber
  %\end{split}
\end{align}
We note that the sum over these probabilities equals unity.

 
\subsection{Third resonance}

For the third resonance, $p=3q/2$, we neglect the amplitude corrections and assume that  $\ket{\Psi}\approx\ket{\Phi}$. With the help of the effective Hamiltonian in Tab.~\ref{tab:eff_Ham}
we obtain the probabilities 
\begin{align}
\label{eq:app_P3q}
P_{3q/2}(\tau) = \cos^2{\left(\frac{\alpha_n^3\tau}{4}\right)} \ \   \text{and} \ \   
P_{-3q/2} (\tau)= \sin^2{\left(\frac{\alpha_n^3\tau}{4}\right)}
\end{align}
for the population of the momentum levels $3q/2$ and $-3q/2$, respectively.


\section{Calculations in High-Gain Regime} 
\label{sec:Calculations_in_High-Gain_Regime}

We calculate the time evolution of the mean photon number for a high-gain FEL in the quantum regime at the second resonance. For that we employ (i) an analytical approximation and (ii) a numerical simulation. 
 
\subsection{Analytical approximation}
\label{sec:Analytic approximation}

The momentum jump operators appearing in the two-photon Dicke Hamiltonian $\hat{H}_{2\text{ph}}$ from Eq.~\eqref{eq:H2ph} can be treated analogously to ladder operators of angular momenta.  
For simplicity, we employ the Schwinger representation of angular momentum~\cite{schwinger} by introducing the bosonic annihilation and creation operators, $\hat{b}_s$ and $\hat{b}_s^\dagger$, respectively for two modes $s=0,2$. We then identify the relations $\hat{\Upsilon}_{0,2}\equiv \hat{b}_0^\dagger \hat{b}_2$ and
$\hat{\Upsilon}_{2,0}\equiv \hat{b}_2^\dagger \hat{b}_0$. Hence, we obtain the Hamiltonian
\begin{equation}
\hat{H}_{2\text{ph}}=\varepsilon^2 \left(\hat{a}_{\text{L}}{}^2\hat{b}_0^\dagger\hat{b}_2+\hat{a}_{\text{L}}^\dagger{}^2\hat{b}_2^\dagger\hat{b}_0\right)
\end{equation}
from which we derive via the Heisenberg equations of motion the two constants of motion 
%\begin{equation}
 % \begin{aligned}
   $\hat{A}  \equiv \hat{N}_0 + \hat{N}_2=\text{const}$, and % \ \ \ \ \text{and} \ \ \ \
   $\hat{B}  \equiv 2\hat{N}_0 + \hat{n}=\text{const}$ %
 % \end{aligned}
%\end{equation}
with $\hat{N}_k\equiv \hat{b}_k^\dagger \hat{b}_k $ and $\hat{n}\equiv\hat{a}_{\text{L}}^\dagger\hat{a}_{\text{L}} $~\cite{kumar}.

In the following we approximate the bosonic operators as classical but dynamical changing variables. The Hamiltonian equation of motion for a dynamical quantity $f$ 
then reads
\begin{equation}
\frac{\D f}{\D \tau}=\left\{f,H_{2\text{ph}}\right\}\equiv -\I \!\! \! \sum\limits_{s=0,2,\text{L}}\left(\frac{\partial f}{\partial b_s}\frac{\partial H_{2\text{ph}}}{\partial b_s^*}-\frac{\partial f}{\partial b_s^*}\frac{\partial H_{2\text{ph}}}{\partial b_s}\right),
\end{equation}
where we have defined the Poisson brackets for the complex amplitudes $b_0$, $b_2$ und $b_{\text{L}}\equiv a_{\text{L}}$ of three harmonic oscillators. This semi-classical approximation neglects contributions that are responsible for spontaneous emission and thus we deduce that our approximation works for a seeded FEL, but breaks down for self-amplified spontaneous emission (SASE).

For the time evolution of the photon number $n\equiv |a_{\text{L}}|^2$ we obtain the second-order differential equation
\begin{equation}
\label{eq:ddot}
\ddot{n}=4\varepsilon^4\left[4n N_0 N_2+n_0^2 (N_0-N_2)\right]
\end{equation}  
with $N_s\equiv |b_s|^2 $. We assume that the two constants of motion, $\hat{A}$ and $\hat{B}$, are described by their initial expectations values, that is $A=N $ and $B=2N+n_0$, respectively. With the help of these relations we eliminate $N_0$ and $N_2$ in Eq.~\eqref{eq:ddot} and obtain a closed equation for $n$. After integrating twice with respect to time $\tau$ we observe
\begin{equation}
2\alpha_N^2\tau=\int\limits_{n_0/N}^{n/N}\frac{\D\xi}{\xi \sqrt{\left(\xi-\frac{n_0}{N}\right)
\left(2+\frac{n_0}{N}-\xi\right)}}
\end{equation}
which can be solved analytically. Finally, we arrive at the expression in Eq.~\eqref{eq:qhigh_nkl_zweite} for the evolution of the photon number $n=n(L)$, where we have introduced the interaction length $L$ via the relation $\alpha_N\tau=L/(2L_g)$~\cite{PRA2019}. 

\subsection{Numerical simulation}


To find a numerical solution for the dynamics dictated by the effective Hamiltonian for $p=q$  we first consider the contribution corresponding to the two-photon Dicke Hamiltonian $\hat{H}_{2\text{ph}}$. Similarly to Ref.~\cite{PRR2021} we notice the analogy of the jump operators to angular momentum, that is $ \hat{J}_+ = \hat{\Upsilon}_{0,2}$, $\hat{J}_- = \hat{\Upsilon}_{2,0}$, and
$\hat{J}_z=(\hat{\Upsilon}_{0,0}-\hat{\Upsilon}_{2,2})/2$. By applying the ladder operators $\hat{J}_\pm$ on the state $\ket{r,m}$ we obtain the relation~\cite{ct}
\begin{equation}
\label{eq:app_Jpm}
\hat{J}_\pm \ket{r,m}=\sqrt{(r\pm m+1)(r\mp m)}\ket{r,m\pm 1}\,,
\end{equation}
where $r$ and $m$ correspond to the quantum numbers of total angular momentum and its $z$-component, respectively.

In this description, the initial state of the electrons is given by % 
%\begin{equation}
$\ket{N/2,N/2}=\ket{p,p,...,p}$. %
%\end{equation}
In this case, only superpositions of the following states 
\begin{equation}
\label{eq:app_mu}
\ket{\mu}\equiv \ket{n_0+2\mu}\ket{N/2,N/2-\mu}
\end{equation}
can be populated by $\hat{H}_{2\text{ph}}$, if we assume that the laser field starts from a Fock state with $n_0$ photons~\cite{walls70}. The quantum number $\mu$ runs from $0$ to $N$, due to 
$-r \leq m \leq r$ with $r=N/2$. 

  

We note that the second contribution $\hat{\Delta}\equiv \hat{H}_\text{eff}-\hat{H}_{2\text{ph}}$ to the effective Hamiltonian (compare to Tab.~\ref{tab:eff_Ham}) includes operators outside this angular momentum algebra. To proceed, we write the electron part of the state in Eq.~\eqref{eq:app_mu} in the  form
\begin{equation}
\ket{N/2,N/2-\mu}=\frac{1}{\sqrt{\mu !}}\sqrt{\frac{(N-\mu)!}{N!}}\,\,\hat{J}_-^\mu\,  \ket{N/2,N/2}
\end{equation}
which follows from Eq.~\eqref{eq:app_Jpm}. With the help of this relation and the commutation relation for the jump operators in Eq.~\eqref{eq:Upsilon_comm} we calculate the action of $\hat{\Delta}$ on the state $\ket{\mu}$ and find that it is an eigenstate of  $\hat{\Delta}$. Hence, we still can rely on the formalism for $\hat{H}_{2\text{ph}}$ for the full effective Hamiltonian since $\hat{\Delta}$ reproduces only states in the form of Eq.~\eqref{eq:app_mu}.


After expanding the quantum state $\ket{\Psi}$ of the total system in terms of the basis states
$\ket{\mu}$, and applying the Schrödinger equation with the effective Hamiltonian for an initial momentum $p$, we finally obtain the equation of motion  
\begin{equation}
\label{eq:cdot}
\I\frac{\D c_\mu(L)}{\D(L/L_g)}=a_p(\mu)c_{\mu-1}(L)+a(\mu+1)c_{\mu+1}(L)+d_p(\mu)c_\mu(L)
\end{equation}
for the expansion coefficients $c_\mu\equiv \braket{\mu|\Psi} $.
For $p=q$, the off-diagonal terms
\begin{subequations}
\begin{equation}
a_q(\mu)\equiv \frac{\alpha_N}{2}\sqrt{(n_0+2\mu-1)(n_0+2\mu)}\sqrt{\frac{\mu}{N}}\sqrt{1-\frac{\mu-1}{N}}
\end{equation}
emerge from the two-photon Dicke Hamiltonian $\hat{H}_{2\text{ph}}$ and
\begin{align}
d_q(\mu)=\alpha_N\left[\frac{2}{3}\mu\left(1-\frac{1}{N}\right)+\frac{1}{3}n_0+\frac{1}{2}\right]
\end{align}
\end{subequations}
represents the additional diagonal contributions arising from $\hat{\Delta}$. Similarly to App.~\ref{sec:Analytic approximation}, we transformed from $\tau$ to $L$. The probability amplitudes $c_\mu$ contain all information of the quantum state of the system and after computing them numerically by diagonalizing a $(N+1)\times (N+1)$ tri-diagonal matrix we are able to evaluate any expectation value.  


Analogously, we find for the resonance $p=q/2$ a dynamical equation of the same form as Eq.~\eqref{eq:cdot} using the corresponding effective Hamiltonian from Tab.~\ref{tab:eff_Ham} up to third order. In this case, the ladder operators of angular momentum  are given by $\hat{\Upsilon}_{1,0}$ and $\hat{\Upsilon}_{0,1}$. We obtain the expressions
\begin{subequations}
\begin{equation}
\label{eq:aq2}
a_{q/2}(\mu)\equiv
\frac{1}{2}\!\left[1\!-\!\frac{\alpha_N^2}{8}\!\left(1+2\frac{n_0+1}{N}\right)\!\right]\sqrt{\mu(n_0+\mu)}\sqrt{1-\frac{\mu-1}{N}}
\end{equation}
and
\begin{equation}
\label{eq:dq2}
d_{q/2}(\mu)\equiv -\frac{\alpha_N}{4}\left[n_0+\mu\left(1+\frac{1}{N}\right)\right]\,
\end{equation}
\end{subequations}
for the off-diagonal and diagonal terms in the differential equation.




