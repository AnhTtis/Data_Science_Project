%% LyX 2.3.6.2 created this file.  For more info, see http://www.lyx.org/.
%% Do not edit unless you really know what you are doing.
\documentclass[a4paper]{amsart}
\usepackage[T1]{fontenc}
\usepackage{amstext}
\usepackage{amsthm}
\usepackage{amssymb}
\usepackage{stackrel}
\usepackage[unicode=true,pdfusetitle,
 bookmarks=true,bookmarksnumbered=false,bookmarksopen=false,
 breaklinks=false,pdfborder={0 0 0},pdfborderstyle={},backref=false,colorlinks=false]
 {hyperref}

\makeatletter

%%%%%%%%%%%%%%%%%%%%%%%%%%%%%% LyX specific LaTeX commands.
\special{papersize=\the\paperwidth,\the\paperheight}


%%%%%%%%%%%%%%%%%%%%%%%%%%%%%% Textclass specific LaTeX commands.
\numberwithin{equation}{section}
\numberwithin{figure}{section}
\theoremstyle{plain}
\newtheorem{thm}{\protect\theoremname}[section]
\theoremstyle{plain}
\newtheorem{prop}[thm]{\protect\propositionname}
\theoremstyle{remark}
\newtheorem{rem}[thm]{\protect\remarkname}
\ifx\proof\undefined
\newenvironment{proof}[1][\protect\proofname]{\par
	\normalfont\topsep6\p@\@plus6\p@\relax
	\trivlist
	\itemindent\parindent
	\item[\hskip\labelsep\scshape #1]\ignorespaces
}{%
	\endtrivlist\@endpefalse
}
\providecommand{\proofname}{Proof}
\fi
\theoremstyle{plain}
\newtheorem{cor}[thm]{\protect\corollaryname}
\theoremstyle{plain}
\newtheorem{lem}[thm]{\protect\lemmaname}
\theoremstyle{definition}
\newtheorem{defn}[thm]{\protect\definitionname}

%%%%%%%%%%%%%%%%%%%%%%%%%%%%%% User specified LaTeX commands.
\usepackage{tikz-cd}
\usepackage{mathtools}
\usepackage{adjustbox}
\usepackage{enumitem}
\tikzcdset{scale cd/.style={every label/.append style={scale=#1},
    cells={nodes={scale=#1}}}}
\usepackage{eucal}

\makeatother


\providecommand{\corollaryname}{Corollary}
\providecommand{\definitionname}{Definition}
\providecommand{\lemmaname}{Lemma}
\providecommand{\propositionname}{Proposition}
\providecommand{\remarkname}{Remark}
\providecommand{\theoremname}{Theorem}

\begin{document}
\global\long\def\sf#1{\mathsf{#1}}%

\global\long\def\scr#1{\mathscr{{#1}}}%

\global\long\def\cal#1{\mathcal{#1}}%

\global\long\def\bb#1{\mathbb{#1}}%

\global\long\def\bf#1{\mathbf{#1}}%

\global\long\def\frak#1{\mathfrak{#1}}%

\global\long\def\fr#1{\mathfrak{#1}}%

\global\long\def\u#1{\underline{#1}}%

\global\long\def\tild#1{\widetilde{#1}}%

\global\long\def\mrm#1{\mathrm{#1}}%

\global\long\def\pr#1{\left(#1\right)}%

\global\long\def\abs#1{\left|#1\right|}%

\global\long\def\inp#1{\left\langle #1\right\rangle }%

\global\long\def\br#1{\left\{  #1\right\}  }%

\global\long\def\norm#1{\left\Vert #1\right\Vert }%

\global\long\def\hat#1{\widehat{#1}}%

\global\long\def\opn#1{\operatorname{#1}}%

\global\long\def\bigmid{\,\middle|\,}%

\global\long\def\Top{\sf{Top}}%

\global\long\def\Set{\sf{Set}}%

\global\long\def\SS{\sf{sSet}}%

\global\long\def\Kan{\sf{Kan}}%

\global\long\def\Cat{\mathcal{C}\sf{at}}%

\global\long\def\imfld{\cal M\mathsf{fld}}%

\global\long\def\ids{\cal D\sf{isk}}%

\global\long\def\ich{\cal C\sf h}%

\global\long\def\SW{\mathcal{SW}}%

\global\long\def\SHC{\mathcal{SHC}}%

\global\long\def\B{\sf B}%

\global\long\def\Spaces{\sf{Spaces}}%

\global\long\def\Mod{\sf{Mod}}%

\global\long\def\Nec{\sf{Nec}}%

\global\long\def\Fin{\sf{Fin}}%

\global\long\def\Ch{\sf{Ch}}%

\global\long\def\Ab{\sf{Ab}}%

\global\long\def\SA{\sf{sAb}}%

\global\long\def\P{\mathsf{POp}}%

\global\long\def\Op{\mathcal{O}\mathsf{p}}%

\global\long\def\Opg{\mathcal{O}\mathsf{p}_{\infty}^{\mathrm{gn}}}%

\global\long\def\Tup{\mathsf{Tup}}%

\global\long\def\Del{\mathbf{\Delta}}%

\global\long\def\id{\operatorname{id}}%

\global\long\def\Aut{\operatorname{Aut}}%

\global\long\def\End{\operatorname{End}}%

\global\long\def\Hom{\operatorname{Hom}}%

\global\long\def\Ext{\operatorname{Ext}}%

\global\long\def\sk{\operatorname{sk}}%

\global\long\def\ihom{\underline{\operatorname{Hom}}}%

\global\long\def\N{\mathrm{N}}%

\global\long\def\-{\text{-}}%

\global\long\def\op{\mathrm{op}}%

\global\long\def\To{\Rightarrow}%

\global\long\def\rr{\rightrightarrows}%

\global\long\def\rl{\rightleftarrows}%

\global\long\def\mono{\rightarrowtail}%

\global\long\def\epi{\twoheadrightarrow}%

\global\long\def\comma{\downarrow}%

\global\long\def\ot{\leftarrow}%

\global\long\def\corr{\leftrightsquigarrow}%

\global\long\def\lim{\operatorname{lim}}%

\global\long\def\colim{\operatorname{colim}}%

\global\long\def\holim{\operatorname{holim}}%

\global\long\def\hocolim{\operatorname{hocolim}}%

\global\long\def\Ran{\operatorname{Ran}}%

\global\long\def\Lan{\operatorname{Lan}}%

\global\long\def\Sk{\operatorname{Sk}}%

\global\long\def\Sd{\operatorname{Sd}}%

\global\long\def\Ex{\operatorname{Ex}}%

\global\long\def\Cosk{\operatorname{Cosk}}%

\global\long\def\Sing{\operatorname{Sing}}%

\global\long\def\Sp{\operatorname{Sp}}%

\global\long\def\Spc{\operatorname{Spc}}%

\global\long\def\Ho{\operatorname{Ho}}%

\global\long\def\Fun{\operatorname{Fun}}%

\global\long\def\map{\operatorname{map}}%

\global\long\def\diag{\operatorname{diag}}%

\global\long\def\Gap{\operatorname{Gap}}%

\global\long\def\cc{\operatorname{cc}}%

\global\long\def\Ob{\operatorname{Ob}}%

\global\long\def\Map{\operatorname{Map}}%

\global\long\def\Rfib{\operatorname{RFib}}%

\global\long\def\Lfib{\operatorname{LFib}}%

\global\long\def\Tw{\operatorname{Tw}}%

\global\long\def\Equiv{\operatorname{Equiv}}%

\global\long\def\Arr{\operatorname{Arr}}%

\global\long\def\Cyl{\operatorname{Cyl}}%

\global\long\def\Path{\operatorname{Path}}%

\global\long\def\Alg{\operatorname{Alg}}%

\global\long\def\ho{\operatorname{ho}}%

\global\long\def\Comm{\operatorname{Comm}}%

\global\long\def\Triv{\operatorname{Triv}}%

\global\long\def\triv{\operatorname{triv}}%

\global\long\def\Env{\operatorname{Env}}%

\global\long\def\Act{\operatorname{Act}}%

\global\long\def\act{\operatorname{act}}%

\global\long\def\loc{\operatorname{loc}}%

\global\long\def\Assem{\operatorname{Assem}}%

\global\long\def\Nat{\operatorname{Nat}}%

\global\long\def\lax{\mathrm{lax}}%

\global\long\def\weq{\mathrm{weq}}%

\global\long\def\fib{\mathrm{fib}}%

\global\long\def\cof{\mathrm{cof}}%

\global\long\def\inj{\mathrm{inj}}%

\global\long\def\univ{\mathrm{univ}}%

\global\long\def\Ker{\opn{Ker}}%

\global\long\def\Coker{\opn{Coker}}%

\global\long\def\Im{\opn{Im}}%

\global\long\def\Coim{\opn{Im}}%

\global\long\def\coker{\opn{coker}}%

\global\long\def\im{\opn{\mathrm{im}}}%

\global\long\def\coim{\opn{coim}}%

\global\long\def\gn{\mathrm{gn}}%

\global\long\def\Mon{\mathrm{Mon}}%

\global\long\def\Un{\mathrm{Un}}%

\global\long\def\St{\mathrm{St}}%

\global\long\def\CA{\operatorname{CAlg}}%

\global\long\def\rd{\mathrm{rd}}%

\global\long\def\xmono#1#2{\stackrel[#2]{#1}{\rightarrowtail}}%

\global\long\def\xepi#1#2{\stackrel[#2]{#1}{\twoheadrightarrow}}%

\global\long\def\adj{\stackrel[\longleftarrow]{\longrightarrow}{\bot}}%

\global\long\def\btimes{\boxtimes}%

\global\long\def\ps#1#2{\prescript{}{#1}{#2}}%

\global\long\def\ups#1#2{\prescript{#1}{}{#2}}%

\global\long\def\hofib{\mathrm{hofib}}%

\global\long\def\cofib{\mathrm{cofib}}%

\global\long\def\Vee{\bigvee}%

\global\long\def\w{\wedge}%

\global\long\def\t{\otimes}%

\global\long\def\bp{\boxplus}%

\global\long\def\rcone{\triangleright}%

\global\long\def\lcone{\triangleleft}%

\global\long\def\S{\mathsection}%

\global\long\def\p{\prime}%
 

\global\long\def\pp{\prime\prime}%

\global\long\def\W{\overline{W}}%

\global\long\def\o#1{\overline{#1}}%

\title[Monoidal Envelopes of Families and Operadic
Kan Extensions]{Monoidal Envelopes of Families of $\infty$-Operads and $\infty$-Operadic
Kan Extensions}
\author{Kensuke Arakawa}
\email{arakawa.kensuke.22c@st.kyoto-u.ac.jp}
\address{Department of Mathematics, Kyoto University, Kyoto, 606-8502, Japan}
\subjclass[2020]{18N70, 55P48}
\begin{abstract}
We provide details of the proof of Lurie's theorem on operadic Kan
extensions \cite[Theorem 3.1.2.3]{HA}. Along the way, we generalize
the construction of monoidal envelopes of $\infty$-operads to families
of $\infty$-operads and use it to construct the fiberwise direct
sum functor, both of which we characterize by certain universal properties.
Aside from their uses in the proof of Lurie's theorem, these results
and constructions have their independent interest.
\end{abstract}

\maketitle
\tableofcontents{}

\section*{Introduction}

In his book \cite{HA}, Lurie introduces (among other things) the
notion of $\infty$-operads, an $\infty$-categorical analog of (colored)
operads. Given an $\infty$-operad $\cal O^{\t}$ and a symmetric
monoidal $\infty$-category $\cal C^{\t}$, we can form the $\infty$-category
$\Alg_{\cal O}\pr{\cal C}$ of $\cal O$-algebras. If $f:\cal O^{\t}\to\cal O^{\p\t}$
is a morphism of $\infty$-operads, there is the induced forgetful
map $f^{*}:\Alg_{\cal O'}\pr{\cal C}\to\Alg_{\cal O}\pr{\cal C}$.
In analogy with restrictions and extensions of scalars, it is natural
to ask whether the map $f^{*}$ has a left adjoint. Lurie gives an
affirmative answer to this question in \cite[Corollary 3.1.3.5]{HA},
provided that $\cal C$ has enough colimits. Because of its fundamental
importance, the result has been applied repeatedly in his work. 

Lurie's proof of \cite[Corollary 3.1.3.5]{HA} relies on another major
theorem on operadic Kan extensions \cite[Theorem 3.1.2.3]{HA} (or
Theorem \ref{thm:3.1.2.3}), which we call the \textbf{fundamental
theorem of operadic Kan extensions}, or FTOK for short. The proof
of FTOK, as Lurie himself acknowledges, is very long. Perhaps because
of the length of the proof, he omits some crucial details of the proof.
This note aims to provide all the details by making necessary constructions
and proving results on them. (In particular, we do not aim to provide
a more concise proof of FTOK than the one provided in \cite{HA}.)

We can roughly divide the details we provide into two parts: The first
one is the datum of ``coherent homotopy.'' In the proof of \cite[Theorem 3.1.2.3]{HA}
(to be more precise, on p. 337), Lurie claims that certain diagrams
are ``equivalent'' without writing the actual equivalence. Such
a practice is fairly common in the literature and is understandable
to some extent. However, in our case, we must be more attentive because
a considerable amount of effort is required to write down the actual
equivalence. We thus give a complete treatment of the equivalence
in this note. (See around Step 1 of the proof of Theorem \ref{thm:3.1.2.3}.)
The second missing detail is the verification that all the combinatorics
fits together. (This corresponds to Step 3 of Theorem \ref{thm:3.1.2.3}.)
Such detail, again, could be left to the reader if the verification
is trivial. But in our case, the verification is so complicated that
a sizable portion of readers will benefit from having it written down.
We thus record every single detail of the verification. 

Here is an outline of this note. In Section \ref{sec:A-Result-on},
we will prove an equivalent formulation of families of $\infty$-operads.
In Section \ref{sec:Monoidal-Envelopes-of}, we generalize Lurie's
monoidal envelopes to families of $\infty$-operads. Using monoidal
envelopes, we can define the fiberwise direct sum functor, whose properties
we discuss in Section \ref{sec:Direct_sum}. The constructions and
results in Sections \ref{sec:A-Result-on}, \ref{sec:Monoidal-Envelopes-of},
and \ref{sec:Direct_sum} will be used in writing down the equivalence
discussed in the previous paragraph, but they also have some independent
interest. To the author's knowledge, these constructions and results
have not appeared elsewhere. Finally, in Section \ref{sec:FTOK},
we will give a complete proof of FTOK.

\section*{Notation and Terminology}

Our notation and terminology mostly follow those of \cite{HA}. Here
are some deviations.
\begin{itemize}
\item We will say that a morphism of simplicial sets is \textbf{final} if
it is cofinal in the sense of \cite{HTT}, and \textbf{initial} if
its opposite is final. 
\item We will write $\Delta^{-1}$ for the empty simplicial set. 
\item If $\cal C$ is a category and $\alpha$ is an ordinal (or more generally
a well-ordered set), then an \textbf{$\alpha$-sequence} in $\cal C$
is a functor $F:\alpha\to\cal C$ such that the map $\colim_{\beta<\lambda}F\beta\to F\beta$
is an isomorphism for each limit ordinal less than $\alpha$. 
\item If $X$ is a simplicial set, then we denote the cone point of the
simplicial set $X^{\rcone}$ by $\infty$.
\end{itemize}

\section{\label{sec:A-Result-on}A Result on Families of $\infty$-Operads}

Let $\cal C$ be an $\infty$-category. Recall that a $\cal C$\textbf{-family
of $\infty$-operads} \cite[Definition 2.3.2.1]{HA} is a categorical
fibration $p:\cal M^{\t}\to\cal C\times N\pr{\Fin_{\ast}}$ satisfying
the following conditions:
\begin{itemize}
\item [(a)]For each object $M\in\cal M^{\t}$ with image $\pr{C,\inp m}\in\cal C\times N\pr{\Fin_{\ast}}$
and for each inert map $\alpha:\inp m\to\inp n$ in $N\pr{\Fin_{\ast}}$,
the morphism $\pr{\id_{C},\alpha}$ admits a $p$-cocartesian lift.
\item [(b)]Let $M\in\cal M^{\t}$ be an object with image $\pr{C,\inp m}\in\cal C\times N\pr{\Fin_{\ast}}$,
where $m\geq1$. Choose for each $1\leq i\leq m$ a $p$-cocartesian
lift $f_{i}:M\to M_{i}$ over $\pr{\id_{C},\rho^{i}}:\pr{C,\inp n}\to\pr{C,\inp 1}$.
Then the morphisms $f_{i}$ form a $p$-limit cone. Every object in
$\cal M_{\inp 0}^{\t}$ is $p$-terminal.
\item [(c)]Let $n\geq1$ and $C\in\cal C$. Given objects $M_{1},\dots,M_{n}\in\cal M_{C}$,
there is an object $M\in\cal M^{\t}$ lying over $\pr{C,\inp n}$
which admits $p$-cocartesian morphisms $M\to M_{i}$ over $\pr{\id_{C},\rho^{i}}:\pr{C,\inp n}\to\pr{C,\inp 1}$.
\end{itemize}
The goal of this section is to prove the following equivalent formulation
of families of $\infty$-operads:
\begin{prop}
\label{prop:sec1_main}Let $\cal C$ be an $\infty$-category and
let $p:\cal M^{\t}\to\cal C\times N\pr{\Fin_{\ast}}$ be a categorical
fibration satisfying conditions (a) and (b) above. Then the following
conditions are equivalent:
\begin{itemize}
\item [(c-i)]The map $p$ satisfies condition (c).
\item [(c-ii)]For each $1\leq i\leq n$, let $\rho_{!}^{i}:\cal M_{\inp n}^{\t}\to\cal M_{\inp 1}^{\t}$
be the functor over $\cal C$ induced by the morphism $\rho^{i}$.
Then for each $n\geq1$, the functor
\[
\pr{\rho_{!}^{i}}_{1\leq i\leq n}:\cal M_{\inp n}^{\t}\to\cal M\times_{\cal C}\cdots\times_{\cal C}\cal M
\]
is an equivalence of $\infty$-categories. 
\end{itemize}
\end{prop}
Here the functor $\rho_{!}^{i}:\cal M_{\inp n}^{\t}\to\cal M$ is
obtained in the following way: Since every inert morphism in $\cal C\times N\pr{\Fin_{\ast}}$
admits a $p$-cocartesian lift, it is possible to choose a natural
transformation $\cal M_{\inp n}^{\t}\times\Delta^{1}\to\cal M^{\t}$
fitting into the commutative diagram % https://q.uiver.app/?q=WzAsNCxbMSwwLCIoXFxtYXRoY2Fse019Xlxcb3RpbWVzICxcXG1hdGhjYWx7RX0pIl0sWzEsMSwiXFxtYXRoY2Fse0N9XlxcZmxhdFxcdGltZXMgTihcXG1hdGhzZntGaW59X1xcYXN0KV5cXG5hdHVyYWwsIl0sWzAsMSwiKFxcbWF0aGNhbHtNfV5cXG90aW1lcyBfe1xcbGFuZ2xlIG5cXHJhbmdsZX0pXlxcZmxhdFxcdGltZXMgKFxcRGVsdGFeMSleXFxzaGFycCJdLFswLDAsIihcXG1hdGhjYWx7TX1eXFxvdGltZXMgX3tcXGxhbmdsZSBuXFxyYW5nbGV9KV5cXGZsYXRcXHRpbWVzIFxcezBcXH0iXSxbMiwxLCJwXzJcXHRpbWVzIFxccmhvXmkiLDJdLFswLDFdLFszLDBdLFszLDJdLFsyLDAsIiIsMSx7InN0eWxlIjp7ImJvZHkiOnsibmFtZSI6ImRhc2hlZCJ9fX1dXQ==
\[\begin{tikzcd}
	{(\mathcal{M}^\otimes _{\langle n\rangle})^\flat\times \{0\}} & {(\mathcal{M}^\otimes ,\mathcal{E})} \\
	{(\mathcal{M}^\otimes _{\langle n\rangle})^\flat\times (\Delta^1)^\sharp} & {\mathcal{C}^\flat\times N(\mathsf{Fin}_\ast)^\natural,}
	\arrow["{p_2\times \rho^i}"', from=2-1, to=2-2]
	\arrow[from=1-2, to=2-2]
	\arrow[from=1-1, to=1-2]
	\arrow[from=1-1, to=2-1]
	\arrow[dashed, from=2-1, to=1-2]
\end{tikzcd}\]where $\cal E$ denotes the set of $p$-cocartesian edges over inert
morphisms in $\cal C\times N\pr{\Fin_{\ast}}$ whose image in $\cal C$
is degenerate, and $N\pr{\Fin_{\ast}}^{\natural}$ is the marked simplicial
set whose underlying simplicial set is $N\pr{\Fin_{\ast}}$ and with
inert morphisms marked. We understand that $\rho_{!}^{i}$ is the
restriction of the filler, so that it is a functor over $\cal C$
and its homotopy class over $\cal C$ is well-defined.
\begin{rem}
A reader conversant with the language of generalized $\infty$-operads
will recognize Proposition \ref{prop:sec1_main} as \cite[Proposition 2.3.2.11]{HA}.
However, in general, the diagram appearing in (2) of \cite[Definition 2.3.2.1]{HA}
is not strictly commutative but is commutative up to a specified natural
equivalence. So Proposition \ref{prop:sec1_main} is slightly different
from the cited proposition and should be regarded as its ``strictified
version.''
\end{rem}
The hardest part of the proof is the rectification of cocartesian
fibrations. We will overcome this problem by appealing to the simplicial
Grothendieck construction, which we explain in subsection \ref{subsec:Simplicial-Grothendieck-Construction}.
Proposition \ref{prop:sec1_main} will then be proved in subsection
\ref{subsec:Main-Result}.

\subsection{\label{subsec:Simplicial-Grothendieck-Construction}Simplicial Grothendieck
Construction}

In this subsection, we describe a convenient method to construct a
cocartesian fibration out of a functor $F:\cal C\to\sf{Cat}_{\Delta}$
of ordinary categories.

Let $\cal C$ be an ordinary category and $F:\cal C\to\sf{Cat}_{\Delta}$
a projectively fibrant functor. The (\textbf{simplicial}) \textbf{Gronthendieck
construction} $\int F$ is the simplicial category which is defined
as follows:
\begin{enumerate}
\item The set of objects of $F$ is given by $\coprod_{C\in\cal C}\opn{ob}F\pr C$.
\item Let $C,C'\in\cal C$ and $X\in F\pr C$, $X'\in F\pr{C'}$. Then the
hom-simplicial sets is 
\[
\pr{\int F}\pr{X,X'}=\coprod_{f\in\cal C\pr{C,C'}}F\pr{C'}\pr{Ff\pr X,X'}.
\]
\end{enumerate}
Composition is given by the composites 
\begin{align*}
F\pr{C''}\pr{Fg\pr{X'},X''}\times F\pr{C'}\pr{Ff\pr X,X'} & \to F\pr{C''}\pr{Fg\pr{X'},X''}\times F\pr{C''}\pr{FgFf\pr X,FgX'}\\
 & \xrightarrow{\circ}F\pr{C''}\pr{F\pr{gf}\pr X,X''}.
\end{align*}
This construction defines a functor
\[
\int:\Fun\pr{\cal C,\sf{Cat}_{\Delta}}\to\pr{\sf{Cat}_{\Delta}}_{/\cal C}.
\]

The Grothendieck construction is closely related to Lurie's relative
nerve functor \cite[$\S$ 3.2.5]{HTT}:
\begin{prop}
\label{prop:relative_nerve_of_Gr_const}Let $\cal C$ be a category
and $F:\cal C\to\sf{Cat}_{\Delta}$ a functor. There is an isomorphism
of simplicial sets 
\[
N\pr{\int F}\cong N_{N\circ F}\pr{\cal C}
\]
between the homotopy coherent nerve of $\int F$ and the relative
nerve of the composite $N\circ F:\cal C\to\sf{Cat}_{\Delta}\xrightarrow{N}\SS$,
which commutes with the projection to $N\pr{\cal C}$. The isomorphism
is natural in $F$.
\end{prop}
\begin{proof}
An $n$-simplex of the relative nerve $N_{N\circ F}\pr{\cal C}$,
by definition, consists of an $n$-simplex $\sigma=C_{0}\xrightarrow{f_{1}}\cdots\xrightarrow{f_{n}}C_{n}$
in $N\pr{\cal C}$ and a simplicial functor $\varphi_{I}:\fr C[\Delta^{I}]\to F\pr{C_{\max I}}$
for each subset $I\subset[n]$, such that for any inclusions $I\subset J\subset[n]$
of subsets, the diagram % https://q.uiver.app/?q=WzAsNCxbMCwwLCJcXG1hdGhmcmFre0N9W1xcRGVsdGFeSV0iXSxbMSwwLCJGKENfe1xcbWF4IEl9KSJdLFsxLDEsIkYoQ197XFxtYXggSn0pIl0sWzAsMSwiXFxtYXRoZnJha3tDfVtcXERlbHRhXkpdIl0sWzAsMV0sWzEsMl0sWzAsM10sWzMsMl1d
\[\begin{tikzcd}
	{\mathfrak{C}[\Delta^I]} & {F(C_{\max I})} \\
	{\mathfrak{C}[\Delta^J]} & {F(C_{\max J})}
	\arrow[from=1-1, to=1-2]
	\arrow[from=1-2, to=2-2]
	\arrow[from=1-1, to=2-1]
	\arrow[from=2-1, to=2-2]
\end{tikzcd}\]commutes. Given such an $n$-simplex $\pr{\sigma,\pr{x_{I}}_{I}}$,
we define a simplicial functor $\varphi:\fr C[\Delta^{n}]\to\int F$
as follows:
\begin{itemize}
\item For each integer $0\leq i\leq n$, we set $\varphi\pr i=\varphi_{[i]}\pr i$.
\item For each pair of integers $0\leq i\leq j\leq n$, the map
\[
\varphi:\fr C[\Delta^{n}]\pr{i,j}\to\pr{\int F}\pr{\varphi_{[i]}\pr i,\varphi_{[i]}\pr j}
\]
is the composite 
\begin{align*}
\fr C[\Delta^{n}]\pr{i,j} & \xrightarrow{\varphi_{[j]}}F\pr{C_{j}}\pr{\varphi_{[j]}\pr i,\varphi_{[j]}\pr j}\\
 & =F\pr{C_{j}}\pr{F\pr{f_{ij}}\pr{\varphi_{[i]}\pr i},\varphi_{[j]}\pr j}\\
 & \hookrightarrow\pr{\int F}\pr{\varphi_{[i]}\pr i,\varphi_{[i]}\pr j}.
\end{align*}
Here $f_{ij}:C_{i}\to C_{j}$ is the composite $f_{j}\cdots f_{i+1}$. 
\end{itemize}
The assignment $\pr{\sigma,\pr{x_{I}}_{I}}\mapsto\varphi$ determines
a morphism of simplicial sets
\[
\Phi:N_{N\circ F}\pr{\cal C}\to N\pr{\int F}
\]
over $N\pr{\cal C}$. We claim that it is an isomorphism of simplicial
sets by constructing an inverse. 

Let $\varphi:\fr C[\Delta^{n}]\to\int F$ be a simplicial functor.
Its projection to $\cal C$ determines an $n$-simplex $\sigma=C_{0}\xrightarrow{f_{1}}\cdots\xrightarrow{f_{n}}C_{n}$
in $N\pr{\cal C}$. For each subinterval $I\subset[n]$, define a
simplicial functor $\varphi_{I}:\fr C[\Delta^{I}]\to F\pr{C_{\max I}}$
as follows:
\begin{itemize}
\item For each integer $i\in I$, we set $\varphi_{I}\pr i=Ff_{i,\max I}\pr{\varphi\pr i}$.
\item For each pair of integers $i,j\in I$ with $i\le j$, the map
\[
\varphi_{I}:\fr C[\Delta^{I}]\pr{i,j}\to F\pr{C_{\max I}}\pr{\varphi_{I}\pr i,\varphi_{I}\pr j}
\]
is the composite
\begin{align*}
\fr C[\Delta^{I}]\pr{i,j} & =\fr C[\Delta^{n}]\pr{i,j}\\
 & \xrightarrow{\varphi}F\pr{C_{j}}\pr{F\pr{f_{ij}}\pr{\varphi\pr i},\varphi\pr j}\\
 & \xrightarrow{Ff_{j,\max I}}F\pr{C_{\max I}}\pr{F\pr{f_{i,\max I}}\pr{\varphi\pr i},F\pr{f_{j,\max I}}\pr{\varphi\pr j}}\\
 & =F\pr{C_{\max I}}\pr{\varphi_{I}\pr i,\varphi_{I}\pr j}.
\end{align*}
\end{itemize}
Next, for an arbitrary subset $J\subset[n]$, we define a simplicial
functor $\varphi_{J}:\fr C[\Delta^{J}]\to F\pr{C_{\max J}}$ to be
the restriction of $\varphi_{[\max J]}$. Then the pair $\pr{\sigma,\pr{\varphi_{I}}_{I}}$
is an $n$-simplex of $N_{N\circ F}\pr{\cal C}$, and the assignment
$\varphi\mapsto\pr{\sigma,\pr{\varphi_{I}}_{I}}$ determines an inverse
of $\Phi$.
\end{proof}
\begin{cor}
\label{cor:sgr}Let $\cal C$ be a category and $F:\cal C\to\sf{Cat}_{\Delta}$
a projectively fibrant functor. Then the functor 
\[
p:N\pr{\int F}\to N\pr{\cal C}
\]
is a cocartesian fibration. A morphism $\pr{f,g}:\pr{C,X}\to\pr{D,Y}$
in $N\pr{\int F}$ is $p$-cocartesian if and only if the morphism
$g:\pr{Ff}\pr X\to Y$ of $N\pr{F\pr D}$ is an equivalence.
\end{cor}
\begin{proof}
Define $\widetilde{F}:\cal C\to\SS^{+}$ by $\widetilde{F}C=N\pr{FC}^{\natural}$,
the homotopy coherent nerve of $FC$ with equivalences marked. Then
$\widetilde{F}$ is a projectively fibrant functor, so by \cite[Proposition 3.2.5.18]{HTT}
its relative nerve $N_{\widetilde{F}}^{+}\pr{\cal C}\to N\pr{\cal C}^{\sharp}$
is a fibrant object of $\SS^{+}/N\pr{\cal C}$ equipped with the cocartesian
model structure. The claim now follows from Proposition \ref{prop:relative_nerve_of_Gr_const}.
\end{proof}
\begin{rem}
The advantage of working with the simplicial Grothendieck construction
is that we can be very explicit about the functors induced by cocartesian
fibrations. Recall that if $p:\cal A\to\cal B$ is a cocartesian fibration
of $\infty$-categories and $f:X\to Y$ is a morphism in $\cal B$,
the induced functor $f_{!}:\cal A_{X}=\cal A\times_{\cal B}\{X\}\to\cal A_{Y}$
is defined by choosing a cocartesian natural transformation $\cal A_{X}\times\Delta^{1}\to\cal B$
fitting into the commutative diagram % https://q.uiver.app/?q=WzAsNSxbMCwwLCJcXG1hdGhjYWx7QX1fWFxcdGltZXMgXFx7MFxcfSJdLFswLDEsIlxcbWF0aGNhbHtBfV9YXFx0aW1lcyBcXERlbHRhXjEiXSxbMiwwLCJcXG1hdGhjYWx7QX0iXSxbMiwxLCJcXG1hdGhjYWx7Qn0iXSxbMSwxLCJcXERlbHRhXjEiXSxbMCwxXSxbMCwyXSxbMiwzXSxbMSwyLCIiLDEseyJzdHlsZSI6eyJib2R5Ijp7Im5hbWUiOiJkYXNoZWQifX19XSxbMSw0XSxbNCwzLCJmIiwyXV0=
\[\begin{tikzcd}
	{\mathcal{A}_X\times \{0\}} && {\mathcal{A}} \\
	{\mathcal{A}_X\times \Delta^1} & {\Delta^1} & {\mathcal{B}}
	\arrow[from=1-1, to=2-1]
	\arrow[from=1-1, to=1-3]
	\arrow[from=1-3, to=2-3]
	\arrow[dashed, from=2-1, to=1-3]
	\arrow[from=2-1, to=2-2]
	\arrow["f"', from=2-2, to=2-3]
\end{tikzcd}\]and then restricting it to $\cal A_{X}\times\{1\}$. 

In the case $p$ is the nerve of a simplicial functor of the form
$\int F\to\cal C$, given a morphism $f:C\to C'$ in $\cal C$, the
induced functor $f_{!}:N\pr{FC}\to N\pr{FC'}$ is nothing but $Nf$.
Indeed, the collection
\[
\{\id_{FfX}:FfX\to FfX\}_{X\in FC}
\]
defines a simplicial natural transformation from the inclusion $FC\hookrightarrow\int F$
to the composite $FC\xrightarrow{Ff}FC'\hookrightarrow\int F$, and
the nerve of this simplicial natural transformation gives us a natural
transformation $N\pr{FC}\times\Delta^{1}\to N\pr{\int F}$ over the
morphism $f$ which is pointwise cocartesian by Corollary \ref{cor:sgr}.
\end{rem}

\subsection{\label{subsec:Main-Result}Proof of Proposition \ref{prop:sec1_main}}

In this subsection, we prove Proposition \ref{prop:sec1_main} using
the results in the previous subsection.

Let $\pr{\sf{Cat}_{\Delta}}_{\fib}$ denote the full subcategory of
$\sf{Cat}_{\Delta}$ spanned by the fibrant simplicial categories.
We let $N^{\natural}:\pr{\sf{Cat}_{\Delta}}_{\fib}\to\SS^{+}$ denote
the functor obtained from the homotopy coherent nerve functor by marking
equivalences. 
\begin{lem}
\label{lem:replacement}Let $\cal C$ be a small category and let
$F,G:\cal C\to\SS_{\mathrm{cocart}}^{+}$ be projectively fibrant
functors. For any natural transformation $f:F\to G$, we can find
projectively fibrant functors $F',G':\cal C\to\sf{Cat}_{\Delta}$,
a projective fibration $f':F'\to G'$, and a commutative diagram % https://q.uiver.app/?q=WzAsNCxbMCwwLCJGIl0sWzEsMCwiTl5cXG5hdHVyYWxcXGNpcmMgRiciXSxbMSwxLCJOXlxcbmF0dXJhbFxcY2lyYyBHJyJdLFswLDEsIkciXSxbMCwxLCJcXHNpbWVxIl0sWzEsMiwiTl5cXG5hdHVyYWxcXGNpcmMgZiciXSxbMCwzLCJmIiwyXSxbMywyLCJcXHNpbWVxIiwyXV0=
\[\begin{tikzcd}
	F & {N^\natural\circ F'} \\
	G & {N^\natural\circ G'}
	\arrow["\simeq", from=1-1, to=1-2]
	\arrow["{N^\natural\circ f'}", from=1-2, to=2-2]
	\arrow["f"', from=1-1, to=2-1]
	\arrow["\simeq"', from=2-1, to=2-2]
\end{tikzcd}\]in $\Fun\pr{\cal C,\SS^{+}}$ whose horizontal arrows are weak equivalences.
\end{lem}
\begin{proof}
Let $F_{\flat},G_{\flat}:\cal C\to\SS$ denote the functors obtained
by forgetting the markings. Since $F,G$ are projectively fibrant,
for each object $C\in\cal C$, the simplicial set $F_{\flat}\pr C$
is an $\infty$-category and the marked edges of $F\pr C$ are precisely
the equivalences of $F_{\flat}\pr C$. Now find a commutative diagram
% https://q.uiver.app/?q=WzAsNCxbMCwwLCJcXG1hdGhmcmFre0N9XFxjaXJjIEZfXFxmbGF0Il0sWzAsMSwiXFxtYXRoZnJha3tDfVxcY2lyYyBHX1xcZmxhdCJdLFsxLDAsIkYnIl0sWzEsMSwiRyciXSxbMCwxLCJcXG1hdGhmcmFre0N9XFxjaXJjIGYiLDJdLFswLDIsIlxcc2ltZXEiXSxbMiwzLCJmJyJdLFsxLDMsIlxcc2ltZXEiLDJdXQ==
\[\begin{tikzcd}
	{\mathfrak{C}\circ F_\flat} & {F'} \\
	{\mathfrak{C}\circ G_\flat} & {G'}
	\arrow["{\mathfrak{C}\circ f}"', from=1-1, to=2-1]
	\arrow["\simeq", from=1-1, to=1-2]
	\arrow["{f'}", from=1-2, to=2-2]
	\arrow["\simeq"', from=2-1, to=2-2]
\end{tikzcd}\]in $\Fun\pr{\cal C,\sf{Cat}_{\Delta}}$, where $F',G'$ are projectively
fibrant, $\alpha'$ is a projective fibration, and the horizontal
arrows are weak equivalences. Since the adjunction
\[
\fr C:\SS_{\mathrm{Joyal}}\adj\sf{Cat}_{\Delta}:N
\]
is a Quillen equivalence, the horizontal arrows of the induced square
% https://q.uiver.app/?q=WzAsNCxbMCwwLCJGX1xcZmxhdCJdLFswLDEsIkdfXFxmbGF0Il0sWzEsMCwiTlxcY2lyYyBGJyJdLFsxLDEsIk5cXGNpcmMgRyciXSxbMCwxLCJmIiwyXSxbMCwyLCJcXHNpbWVxIl0sWzIsMywiTlxcY2lyYyBmJyJdLFsxLDMsIlxcc2ltZXEiLDJdXQ==
\[\begin{tikzcd}
	{F_\flat} & {N\circ F'} \\
	{G_\flat} & {N\circ G'}
	\arrow["f"', from=1-1, to=2-1]
	\arrow["\simeq", from=1-1, to=1-2]
	\arrow["{N\circ f'}", from=1-2, to=2-2]
	\arrow["\simeq"', from=2-1, to=2-2]
\end{tikzcd}\]in $\Fun\pr{\cal C,\SS}$ are weak equivalences (i.e., pointwise categorical
equivalences). By marking the equivalences, we obtain the desired
square.
\end{proof}
\begin{defn}
Let $n\geq1$. We let $\cal B\pr n$ denote the nerve of the category
$\{\infty\}\star\inp n^{\circ}$, where $\inp n^{\circ}=\{1,\dots,n\}$
is regarded as a discrete category. For each $1\leq i\leq n$, we
will denote the unique morphism $\infty\to i$ by $\rho^{i}$.
\end{defn}
\begin{prop}
\label{prop:pre-equivalence}Let $n\geq1$, and consider a commutative
diagram% https://q.uiver.app/?q=WzAsMyxbMCwwLCJcXG1hdGhjYWx7Q30iXSxbMiwwLCJcXG1hdGhjYWx7RH0iXSxbMSwxLCJcXG1hdGhjYWx7Qn0obikiXSxbMCwxLCJwIl0sWzEsMiwiciJdLFswLDIsInEiLDJdXQ==
\[\begin{tikzcd}
	{\mathcal{C}} && {\mathcal{D}} \\
	& {\mathcal{B}(n)}
	\arrow["p", from=1-1, to=1-3]
	\arrow["r", from=1-3, to=2-2]
	\arrow["q"', from=1-1, to=2-2]
\end{tikzcd}\]of $\infty$-categories. Assume the following:
\begin{enumerate}
\item The functors $q$ and $r$ are cocartesian fibrations, and the functor
$p$ is a categorical fibration.
\item The functor $p$ preserves cocartesian morphisms over $\cal B\pr n$.
\item The functor $\pr{\rho_{!}^{i}}_{i}:\cal D_{\infty}\to\prod_{1\leq i\leq n}\cal D_{i}$
induces an injection between the set of equivalence classes of objects
of $\cal D_{\inp n}$ and that of $\prod_{1\leq i\leq n}\cal D_{i}$.
\item For each object $C\in\cal C_{\infty}$, the $q$-cocartesian morphisms
$C\to C_{i}$ over $\rho^{i}$ form a $p$-limit cone. 
\item Given an object $\pr{C_{i}}_{i}\in\prod_{1\leq i\leq n}\cal C_{i}$,
there is an object $C\in\cal C$ which admits a $q$-cocartesian morphism
$C\to C_{i}$ over the morphism $\rho^{i}$ for every $i$.
\end{enumerate}
Then the functor
\[
\cal C_{\infty}\to\prod_{1\leq i\leq n}\cal C_{i}\times_{\prod_{1\leq i\leq n}\cal D_{i}}\cal D_{\infty}
\]
is an equivalence of $\infty$-categories.
\end{prop}
In the proof, we shall make use of the adjunction 
\[
r_{!}:\SS^{+}/\cal B\pr n\adj\Fun\pr{\{\infty\}\star\inp n^{\circ},\SS^{+}}:r^{*}
\]
between the (marked) relative nerve functor $r^{*}$ and its left
adjoint $r_{!}$, given by $r_{!}\pr X\pr x=X\times_{\cal B\pr n^{\sharp}}\cal B\pr n_{/x}^{\sharp}$.
\begin{proof}
The statement of the proposition is a bit vague, because the functors
$\cal C_{\infty}\to\cal C_{i}$ and $\cal D_{\infty}\to\cal D_{i}$
are defined only up to natural equivalence. A more precise formulation
is the following: Since $p$ is a categorical fibration, the map $\cal C^{\natural}\to\cal D^{\natural}$
is a fibration in the cocartesian model structures over $\cal B\pr n$,
where $\cal C^{\natural}$ is the marked simplicial set whose marked
edges are the $q$-cocartesian edges, and $\cal D^{\natural}$ is
defined similarly. (This is the opposite convention of \cite[$\S 3.2$]{HTT},
but it shouldn't case any confusion.) Therefore, for each object $i\in\inp n^{\circ}$,
there is a natural transformation $\cal C_{\infty}\times\Delta^{1}\to\cal C$
which fits into the commutative diagram % https://q.uiver.app/?q=WzAsNSxbMCwwLCJcXG1hdGhjYWx7Q31eXFxuYXR1cmFsX1xcaW5mdHlcXHRpbWVzIChcXHswXFx9KV5cXHNoYXJwIl0sWzAsMSwiXFxtYXRoY2Fse0N9XlxcbmF0dXJhbF9cXGluZnR5XFx0aW1lcyAoXFxEZWx0YV4xKV5cXHNoYXJwIl0sWzEsMSwiXFxtYXRoY2Fse0R9XlxcbmF0dXJhbF9cXGluZnR5XFx0aW1lcyAoXFxEZWx0YV4xKV5cXHNoYXJwIl0sWzIsMSwiXFxtYXRoY2Fse0R9XlxcbmF0dXJhbCJdLFsyLDAsIlxcbWF0aGNhbHtDfV5cXG5hdHVyYWwiXSxbMCwxXSxbMSwyXSxbMiwzXSxbNCwzXSxbMCw0XSxbMSw0LCIiLDEseyJzdHlsZSI6eyJib2R5Ijp7Im5hbWUiOiJkYXNoZWQifX19XV0=
\[\begin{tikzcd}
	{\mathcal{C}^\natural_\infty\times (\{0\})^\sharp} && {\mathcal{C}^\natural} \\
	{\mathcal{C}^\natural_\infty\times (\Delta^1)^\sharp} & {\mathcal{D}^\natural_\infty\times (\Delta^1)^\sharp} & {\mathcal{D}^\natural,}
	\arrow[from=1-1, to=2-1]
	\arrow[from=2-1, to=2-2]
	\arrow[from=2-2, to=2-3]
	\arrow[from=1-3, to=2-3]
	\arrow[from=1-1, to=1-3]
	\arrow[dashed, from=2-1, to=1-3]
\end{tikzcd}\]where the functor $\cal D^{\natural}\times\pr{\Delta^{1}}^{\sharp}\to\cal D^{\natural}$
is a $r$-cocartesian natural transformation over $\infty\to i$ which
starts from the inclusion $\cal D_{\infty}\hookrightarrow\cal D$.
The restriction of the filler determines a commutative diagram% https://q.uiver.app/?q=WzAsNCxbMCwwLCJcXG1hdGhjYWx7Q31fXFxpbmZ0eSJdLFsxLDAsIlxcbWF0aGNhbHtDfV9pIl0sWzEsMSwiXFxtYXRoY2Fse0R9X2kuIl0sWzAsMSwiXFxtYXRoY2Fse0R9X1xcaW5mdHkiXSxbMCwxLCJcXHJob15pXyEiXSxbMSwyLCJwIl0sWzAsMywicCIsMl0sWzMsMiwiXFxyaG9eaV8hIiwyXV0=
\[\begin{tikzcd}
	{\mathcal{C}_\infty} & {\mathcal{C}_i} \\
	{\mathcal{D}_\infty} & {\mathcal{D}_i.}
	\arrow["{\rho^i_!}", from=1-1, to=1-2]
	\arrow["p", from=1-2, to=2-2]
	\arrow["p"', from=1-1, to=2-1]
	\arrow["{\rho^i_!}"', from=2-1, to=2-2]
\end{tikzcd}\]The claim is that, for any such choice of functors $\cal D_{\infty}\to\cal D_{i}$
and $\cal C_{\infty}\to\cal C_{i}$, the resulting functor $\phi:\cal C_{\infty}\to\prod_{1\leq i\leq n}\cal C_{i}\times_{\prod_{1\leq i\leq n}\cal D_{i}}\cal D_{\infty}$
is an equivalence of $\infty$-categories.

We begin by proving the essential surjectivity. Let $\pr{\pr{C_{i}}_{i},D}\in\prod_{1\leq i\leq n}\cal C_{i}\times_{\prod_{1\leq i\leq n}\cal D_{i}}\cal D_{\infty}$
be an arbitrary object. By our hypothesis (5), we can find an object
$C\in\cal C_{\infty}$ which admits a $q$-cocartesian morphism $C\to C_{i}$
over $\rho^{i}$ for each $1\leq i\leq n$. By the uniqueness of cocartesian
morphisms, there is an equivalence to $\pr{C_{i}}_{i}\simeq\pr{\rho_{!}^{i}\pr C}_{i}$
in $\prod_{1\leq i\leq n}\cal C_{i}$. Its image in $\prod_{1\leq i\leq n}\cal D_{i}$
determines an equivalence $\pr{\rho_{!}^{i}\pr D}_{i}\simeq\pr{\rho_{!}^{i}\pr{p\pr C}}_{i}$.
By virtue of assumption (3), we obtain an equivalence $g:D\xrightarrow{\simeq}p(C)$
in $\cal D_{\infty}$. Using the fact that $p$ is a categorical fibration,
we can lift the morphism $g$ to an equivalence $\widetilde{g}:\widetilde{C}\xrightarrow{\simeq}C$
in $\cal C$. Therefore, replacing $f_{i}$ by $f_{i}\widetilde{g}$
if necessary, we may assume that $p\pr C=D$. For each $1\leq i\leq n$,
the morphism $f_{i}$ determines a functor $\cal C_{\infty}^{\natural}\times\{0\}^{\sharp}\cup\{C\}^{\sharp}\times\pr{\Delta^{1}}^{\sharp}\to\cal C^{\natural}$.
Since the inclusion
\[
\cal C_{\infty}^{\natural}\times\{0\}^{\sharp}\cup\{C\}^{\sharp}\times\pr{\Delta^{1}}^{\sharp}\hookrightarrow\cal C_{\infty}^{\natural}\times\pr{\Delta^{1}}^{\sharp}
\]
is a trivial cofibration in the cocartesian model structure over $\cal B\pr n$
\cite[Proposition 3.1.2.2]{HTT}, we can find a filler of the diagram
% https://q.uiver.app/?q=WzAsNSxbMCwwLCJcXG1hdGhjYWx7Q31eXFxuYXR1cmFsX1xcaW5mdHlcXHRpbWVzIChcXHswXFx9KV5cXHNoYXJwXFxjdXBcXHtDXFx9Xlxcc2hhcnAgXFx0aW1lcyAoXFxEZWx0YV4xKV5cXHNoYXJwIl0sWzAsMSwiXFxtYXRoY2Fse0N9XlxcbmF0dXJhbF9cXGluZnR5XFx0aW1lcyAoXFxEZWx0YV4xKV5cXHNoYXJwIl0sWzEsMSwiXFxtYXRoY2Fse0R9XlxcbmF0dXJhbF9cXGluZnR5XFx0aW1lcyAoXFxEZWx0YV4xKV5cXHNoYXJwIl0sWzIsMSwiXFxtYXRoY2Fse0R9XlxcbmF0dXJhbCJdLFsyLDAsIlxcbWF0aGNhbHtDfV5cXG5hdHVyYWwiXSxbMCwxXSxbMSwyXSxbMiwzXSxbNCwzXSxbMCw0XSxbMSw0LCIiLDEseyJzdHlsZSI6eyJib2R5Ijp7Im5hbWUiOiJkYXNoZWQifX19XV0=
\[\begin{tikzcd}
	{\mathcal{C}^\natural_\infty\times (\{0\})^\sharp\cup\{C\}^\sharp \times (\Delta^1)^\sharp} && {\mathcal{C}^\natural} \\
	{\mathcal{C}^\natural_\infty\times (\Delta^1)^\sharp} & {\mathcal{D}^\natural_\infty\times (\Delta^1)^\sharp} & {\mathcal{D}^\natural.}
	\arrow[from=1-1, to=2-1]
	\arrow[from=2-1, to=2-2]
	\arrow[from=2-2, to=2-3]
	\arrow[from=1-3, to=2-3]
	\arrow[from=1-1, to=1-3]
	\arrow[dashed, from=2-1, to=1-3]
\end{tikzcd}\]The filler determines a functor $\cal C_{\infty}\to\cal C_{i}$ which
is naturally equivalent to $\rho_{!}^{i}$ over $\cal D_{i}$. So
the functor $\phi':\cal C_{\infty}\to\prod_{1\leq i\leq n}\cal C_{i}\times_{\prod_{1\leq i\leq n}\cal D_{i}}\cal D_{\infty}$
obtained from these fillers is naturally equivalent to $\phi$. By
construction, we have $\phi'\pr C=\pr{\pr{C_{i}}_{i},D}$. This proves
that $\phi$ is essentially surjective.

Next, to prove that $\phi$ is fully faithful, we consider the commutative
diagram % https://q.uiver.app/?q=WzAsMTAsWzEsMSwiXFxtYXRoY2Fse0N9XlxcbmF0dXJhbF9cXGluZnR5XFx0aW1lcyAoXFxEZWx0YV4xKV5cXHNoYXJwIl0sWzEsMywiXFxtYXRoY2Fse0R9XlxcbmF0dXJhbF9cXGluZnR5XFx0aW1lcyAoXFxEZWx0YV4xKV5cXHNoYXJwIl0sWzMsMywiXFxwcm9kX3sxXFxsZXEgaVxcbGVxIG59XFxtYXRoY2Fse0R9XlxcbmF0dXJhbFxcdGltZXMgX3soXFxtYXRoY2Fse0J9KG4pKV5cXHNoYXJwfSh7XFx7XFxpbmZ0eVxcfVxcc3Rhclxce2lcXH19KV5cXHNoYXJwLiJdLFszLDEsIlxccHJvZF97MVxcbGVxIGlcXGxlcSBufVxcbWF0aGNhbHtDfV5cXG5hdHVyYWxcXHRpbWVzIF97KFxcbWF0aGNhbHtCfShuKSleXFxzaGFycH0oe1xce1xcaW5mdHlcXH1cXHN0YXJcXHtpXFx9fSleXFxzaGFycCJdLFs0LDAsIlxccHJvZF97MVxcbGVxIGlcXGxlcSBufVxcbWF0aGNhbHtDfV9pXlxcbmF0dXJhbCJdLFs0LDIsIlxccHJvZF97MVxcbGVxIGlcXGxlcSBufVxcbWF0aGNhbHtEfV9pXlxcbmF0dXJhbCJdLFsyLDAsIlxcbWF0aGNhbHtDfV9cXGluZnR5XlxcbmF0dXJhbFxcdGltZXMgXFx7MVxcfV5cXHNoYXJwIl0sWzIsMiwiXFxtYXRoY2Fse0R9X1xcaW5mdHleXFxuYXR1cmFsXFx0aW1lcyBcXHsxXFx9Xlxcc2hhcnAiXSxbMCwyLCJcXG1hdGhjYWx7Q31fXFxpbmZ0eV5cXG5hdHVyYWxcXHRpbWVzIFxcezBcXH1eXFxzaGFycCJdLFswLDQsIlxcbWF0aGNhbHtEfV9cXGluZnR5XlxcbmF0dXJhbFxcdGltZXMgXFx7MFxcfV5cXHNoYXJwIl0sWzAsMV0sWzEsMl0sWzMsMl0sWzAsM10sWzQsMywiXFxzaW1lcSIsMix7InN0eWxlIjp7InRhaWwiOnsibmFtZSI6Imhvb2siLCJzaWRlIjoiYm90dG9tIn19fV0sWzQsNV0sWzUsMiwiXFxzaW1lcSIsMCx7InN0eWxlIjp7InRhaWwiOnsibmFtZSI6Imhvb2siLCJzaWRlIjoiYm90dG9tIn19fV0sWzYsNF0sWzYsN10sWzcsNV0sWzcsMSwiXFxzaW1lcSJdLFs2LDAsIlxcc2ltZXEiXSxbOCwwLCJcXHNpbWVxIl0sWzksMSwiXFxzaW1lcSJdLFs4LDldLFs4LDMsIiIsMSx7InN0eWxlIjp7InRhaWwiOnsibmFtZSI6Imhvb2siLCJzaWRlIjoidG9wIn19fV0sWzksMiwiIiwxLHsic3R5bGUiOnsidGFpbCI6eyJuYW1lIjoiaG9vayIsInNpZGUiOiJ0b3AifX19XV0=
\[\begin{tikzcd}[scale cd=.65]
	&& {\mathcal{C}_\infty^\natural\times \{1\}^\sharp} && {\prod_{1\leq i\leq n}\mathcal{C}_i^\natural} \\
	& {\mathcal{C}^\natural_\infty\times (\Delta^1)^\sharp} && {\prod_{1\leq i\leq n}\mathcal{C}^\natural\times _{(\mathcal{B}(n))^\sharp}({\{\infty\}\star\{i\}})^\sharp} \\
	{\mathcal{C}_\infty^\natural\times \{0\}^\sharp} && {\mathcal{D}_\infty^\natural\times \{1\}^\sharp} && {\prod_{1\leq i\leq n}\mathcal{D}_i^\natural} \\
	& {\mathcal{D}^\natural_\infty\times (\Delta^1)^\sharp} && {\prod_{1\leq i\leq n}\mathcal{D}^\natural\times _{(\mathcal{B}(n))^\sharp}({\{\infty\}\star\{i\}})^\sharp.} \\
	{\mathcal{D}_\infty^\natural\times \{0\}^\sharp}
	\arrow[from=2-2, to=4-2]
	\arrow[from=4-2, to=4-4]
	\arrow[from=2-4, to=4-4]
	\arrow[from=2-2, to=2-4]
	\arrow["\simeq"', hook', from=1-5, to=2-4]
	\arrow[from=1-5, to=3-5]
	\arrow["\simeq", hook', from=3-5, to=4-4]
	\arrow[from=1-3, to=1-5]
	\arrow[from=1-3, to=3-3]
	\arrow[from=3-3, to=3-5]
	\arrow["\simeq", from=3-3, to=4-2]
	\arrow["\simeq", from=1-3, to=2-2]
	\arrow["\simeq", from=3-1, to=2-2]
	\arrow["\simeq", from=5-1, to=4-2]
	\arrow[from=3-1, to=5-1]
	\arrow[hook, from=3-1, to=2-4]
	\arrow[hook, from=5-1, to=4-4]
\end{tikzcd}\]We wish to show that the back face of this diagram is homotopy cartesian
in the cocartesian model structure in $\SS^{+}$. By \cite[Proposition 2.10]{StUn},
the maps labeled with $\simeq$ are weak equivalences in $\SS^{+}$.
It will therefore suffice to show that the front face is homotopy
cartesian. Note that the front face can be identified with the diagram
% https://q.uiver.app/?q=WzAsNCxbMCwwLCJyXyEoXFxtYXRoY2Fse0N9XlxcbmF0dXJhbCkoXFxpbmZ0eSkiXSxbMCwxLCJyXyEoXFxtYXRoY2Fse0R9XlxcbmF0dXJhbCkoXFxpbmZ0eSkiXSxbMSwwLCJcXHByb2RfezFcXGxlcSBpXFxsZXEgbn1yXyEoXFxtYXRoY2Fse0N9XlxcbmF0dXJhbCkoaSkiXSxbMSwxLCJcXHByb2RfezFcXGxlcSBpXFxsZXEgbn1yXyEoXFxtYXRoY2Fse0R9XlxcbmF0dXJhbCkoaSkiXSxbMCwxXSxbMCwyXSxbMSwzXSxbMiwzXV0=
\[\begin{tikzcd}
	{r_!(\mathcal{C}^\natural)(\infty)} & {\prod_{1\leq i\leq n}r_!(\mathcal{C}^\natural)(i)} \\
	{r_!(\mathcal{D}^\natural)(\infty)} & {\prod_{1\leq i\leq n}r_!(\mathcal{D}^\natural)(i),}
	\arrow[from=1-1, to=2-1]
	\arrow[from=1-1, to=1-2]
	\arrow[from=2-1, to=2-2]
	\arrow[from=1-2, to=2-2]
\end{tikzcd}\]where $r_{!}:\SS^{+}/\cal B\pr n\to\Fun\pr{\{\infty\}\star\inp n^{\circ},\SS^{+}}$
is the left adjoint of the relative nerve functor.

Find a commutative diagram % https://q.uiver.app/?q=WzAsNCxbMCwwLCJyXyEoXFxtYXRoY2Fse0N9XlxcbmF0dXJhbCkiXSxbMSwwLCJGIl0sWzEsMSwiRyJdLFswLDEsInJfIShcXG1hdGhjYWx7RH1eXFxuYXR1cmFsKSJdLFswLDEsIlxcc2ltZXEiXSxbMSwyLCJmIl0sWzMsMiwiXFxzaW1lcSIsMl0sWzAsMywicl8hKHApIiwyXV0=
\[\begin{tikzcd}
	{r_!(\mathcal{C}^\natural)} & F \\
	{r_!(\mathcal{D}^\natural)} & G
	\arrow["\simeq", from=1-1, to=1-2]
	\arrow["f", from=1-2, to=2-2]
	\arrow["\simeq"', from=2-1, to=2-2]
	\arrow["{r_!(p)}"', from=1-1, to=2-1]
\end{tikzcd}\]in $\Fun\pr{\{\infty\}\star\inp n^{\circ},\SS^{+}}$, where $F$ and
$G$ are projectively fibrant functors, $f$ is a projective fibration,
and the horizontal arrows are weak equivalences. According to Lemma
\ref{lem:replacement}, we may assume that the natural transformation
\[
f:\{\infty\}\star\inp n^{\circ}\times[1]\to\SS^{+}
\]
can be factored as 
\[
\{\infty\}\star\inp n^{\circ}\times[1]\xrightarrow{f'}\pr{\sf{Cat}_{\Delta}}_{\fib}\xrightarrow{N^{\natural}}\SS^{+}.
\]
We set $f'_{0}=F'$ and $g'_{0}=G'$, and write $\cal C_{\Delta}=\int F'$
and $\cal D_{\Delta}=\int G'$. We are then reduced to showing that
the square % https://q.uiver.app/?q=WzAsNCxbMCwwLCJOKFxcbWF0aGNhbHtDfV97XFxEZWx0YSxcXGluZnR5fSleXFxuYXR1cmFsIl0sWzEsMCwiXFxwcm9kX3sxXFxsZXEgaVxcbGVxIG59TihcXG1hdGhjYWx7Q31fe1xcRGVsdGEsaX0pXlxcbmF0dXJhbCJdLFsxLDEsIlxccHJvZF97MVxcbGVxIGlcXGxlcSBufU4oXFxtYXRoY2Fse0R9X3tcXERlbHRhLGl9KV5cXG5hdHVyYWwiXSxbMCwxLCJOKFxcbWF0aGNhbHtEfV97XFxEZWx0YSxcXGluZnR5fSleXFxuYXR1cmFsIl0sWzAsMV0sWzEsMl0sWzAsM10sWzMsMl1d
\[\begin{tikzcd}
	{N(\mathcal{C}_{\Delta,\infty})^\natural} & {\prod_{1\leq i\leq n}N(\mathcal{C}_{\Delta,i})^\natural} \\
	{N(\mathcal{D}_{\Delta,\infty})^\natural} & {\prod_{1\leq i\leq n}N(\mathcal{D}_{\Delta,i})^\natural}
	\arrow[from=1-1, to=1-2]
	\arrow[from=1-2, to=2-2]
	\arrow[from=1-1, to=2-1]
	\arrow[from=2-1, to=2-2]
\end{tikzcd}\]of marked simplicial sets is homotopy cartesian. Since the vertical
arrows are fibrations in $\SS_{\mathrm{cocart}}^{+}$ and all the
objects are fibrant, it suffices to show that the map 
\[
N\pr{\cal C_{\Delta,\infty}}^{\natural}\to N\pr{\cal D_{\Delta,\infty}}^{\natural}\times_{\prod_{1\leq i\leq n}N\pr{\cal D_{\Delta,i}}^{\natural}}\prod_{1\leq i\leq n}N\pr{\cal C_{\Delta,i}}^{\natural}
\]
is a weak equivalence in $\SS_{\mathrm{cocart}}^{+}$. This is equivalent
to the condition that the map
\[
\theta':N\pr{\cal C_{\Delta,\infty}}\to N\pr{\cal D_{\Delta,\infty}}\times_{\prod_{1\leq i\leq n}N\pr{\cal D_{\Delta,i}}}\prod_{1\leq i\leq n}N\pr{\cal C_{\Delta,i}}
\]
is an equivalence of $\infty$-categories. 

According to Proposition \ref{prop:relative_nerve_of_Gr_const}, the
map $r^{*}\pr f$ can be identified with the map
\[
N\pr{\int f'}:N\pr{\cal C_{\Delta}}^{\natural}\to N\pr{\cal D_{\Delta}}^{\natural}.
\]
Moreover, since the adjunction $r_{!}\dashv r^{*}$ is a Quillen equivalence,
the maps $\cal C^{\natural}\to r^{*}\pr F$ and $\cal D^{\natural}\to r^{*}\pr G$
are weak equivalences in the cocartesian model structure over $\cal B\pr n$.
So the functor $N\pr{\cal C_{\Delta}}\to N\pr{\cal D_{\Delta}}$ has
all the properties (1)-(5). So the proof of the essential surjectivity
of $\theta$ applies as well to $\theta'$, showing that $\theta'$
is essentially surjective. It remains to verify that $\theta'$ is
fully faithful. For this, it suffices to show that for each pair of
objects $X,Y\in\cal C_{\Delta,\infty}$, the diagram% https://q.uiver.app/?q=WzAsNCxbMCwwLCJcXG1hdGhjYWx7Q31fe1xcRGVsdGEsXFxpbmZ0eX0oWCxZKSJdLFswLDEsIlxcbWF0aGNhbHtEfV97XFxEZWx0YSxcXGluZnR5fShmWCxmWSkiXSxbMSwwLCJcXHByb2RfezFcXGxlcSBpXFxsZXEgbn1cXG1hdGhjYWx7Q31fe1xcRGVsdGEsaX0oKEYnXFxyaG9eaSlYLChGJ1xccmhvXmkpWSkiXSxbMSwxLCJcXHByb2RfezFcXGxlcSBpXFxsZXEgbn1cXG1hdGhjYWx7RH1fe1xcRGVsdGEsaX0oKEcnXFxyaG9eaSlmWCwoRydcXHJob15pKWZZKSJdLFswLDFdLFswLDJdLFsyLDNdLFsxLDNdXQ==
\[\begin{tikzcd}
	{\mathcal{C}_{\Delta,\infty}(X,Y)} & {\prod_{1\leq i\leq n}\mathcal{C}_{\Delta,i}((F'\rho^i)X,(F'\rho^i)Y)} \\
	{\mathcal{D}_{\Delta,\infty}(fX,fY)} & {\prod_{1\leq i\leq n}\mathcal{D}_{\Delta,i}((G'\rho^i)fX,(G'\rho^i)fY)}
	\arrow[from=1-1, to=2-1]
	\arrow[from=1-1, to=1-2]
	\arrow[from=1-2, to=2-2]
	\arrow[from=2-1, to=2-2]
\end{tikzcd}\]of Kan complexes is homotopy cartesian. Consider the commutative diagram
% https://q.uiver.app/?q=WzAsOCxbMCwxLCJcXG1hdGhjYWx7Q31fe1xcRGVsdGEsXFxpbmZ0eX0oWCxZKSJdLFswLDMsIlxcbWF0aGNhbHtEfV97XFxEZWx0YSxcXGluZnR5fShmWCxmWSkiXSxbMiwxLCJcXHByb2RfezFcXGxlcSBpXFxsZXEgbn1cXG1hdGhjYWx7Q31fe1xcRGVsdGEsaX0oKEYnXFxyaG9eaSlYLChGJ1xccmhvXmkpWSkiXSxbMiwzLCJcXHByb2RfezFcXGxlcSBpXFxsZXEgbn1cXG1hdGhjYWx7RH1fe1xcRGVsdGEsaX0oKEcnXFxyaG9eaSkgZlgsKEcnXFxyaG9eaSlmWSkiXSxbMSwwLCJcXG1hdGhjYWx7Q31fXFxEZWx0YShYLFkpIl0sWzMsMCwiXFxwcm9kIF97MVxcbGVxIGlcXGxlcSBufVxcbWF0aGNhbHtDfV9cXERlbHRhKFgsKEYnXFxyaG9eaSlZKSJdLFszLDIsIlxccHJvZF97MVxcbGVxIGlcXGxlcSBufVxcbWF0aGNhbHtEfV97XFxEZWx0YX0oZlgsKEcnXFxyaG9eaSlmWSkuIl0sWzEsMiwiXFxtYXRoY2Fse0R9X3tcXERlbHRhfShmWCxmWSkiXSxbMCwxXSxbMCwyXSxbMiwzXSxbMSwzXSxbMCw0XSxbMiw1XSxbNCw1XSxbMyw2XSxbNCw3XSxbNyw2XSxbMSw3XSxbNSw2XV0=
\[\begin{tikzcd}[column sep=small, scale cd=.75]
	& {\mathcal{C}_\Delta(X,Y)} && {\prod _{1\leq i\leq n}\mathcal{C}_\Delta(X,(F'\rho^i)Y)} \\
	{\mathcal{C}_{\Delta,\infty}(X,Y)} && {\prod_{1\leq i\leq n}\mathcal{C}_{\Delta,i}((F'\rho^i)X,(F'\rho^i)Y)} \\
	& {\mathcal{D}_{\Delta}(fX,fY)} && {\prod_{1\leq i\leq n}\mathcal{D}_{\Delta}(fX,(G'\rho^i)fY).} \\
	{\mathcal{D}_{\Delta,\infty}(fX,fY)} && {\prod_{1\leq i\leq n}\mathcal{D}_{\Delta,i}((G'\rho^i) fX,(G'\rho^i)fY)}
	\arrow[from=2-1, to=4-1]
	\arrow[from=2-1, to=2-3]
	\arrow[from=2-3, to=4-3]
	\arrow[from=4-1, to=4-3]
	\arrow[from=2-1, to=1-2]
	\arrow[from=2-3, to=1-4]
	\arrow[from=1-2, to=1-4]
	\arrow[from=4-3, to=3-4]
	\arrow[from=1-2, to=3-2]
	\arrow[from=3-2, to=3-4]
	\arrow[from=4-1, to=3-2]
	\arrow[from=1-4, to=3-4]
\end{tikzcd}\]The slanted arrows are isomorphisms of simplicial sets, so it suffices
to show that the back face is homotopy cartesian. This follows from
our assumption (4).
\end{proof}
We can now prove Proposition \ref{prop:sec1_main}.
\begin{proof}
[Proof of  Proposition \ref{prop:sec1_main}]Clearly (c-ii) implies
(c-i). Conversely, suppose that condition (c-i) is satisfied. For
each $n\geq1$, there is a unique functor $\cal B\pr n\to N\pr{\Fin_{\ast}}$
which is the identity map on morphisms. Applying Proposition \ref{prop:pre-equivalence}
to the functor $\cal M^{\t}\times_{N\pr{\Fin_{\ast}}}\cal B\pr n\to\cal C\times\cal B\pr n$,
we see that condition (c-ii) holds true.
\end{proof}

\section{\label{sec:Monoidal-Envelopes-of}Monoidal Envelopes of Families
of $\infty$-Operads}

In \cite[Section 2.2.4]{HA}, Lurie introduces the universal procedure
to make an arbitrary $\infty$-operad into a symmetric monoidal $\infty$-category.
The resulting symmetric monoidal $\infty$-category is called the
\textbf{monoidal envelope}. In this section, we will show that given
a family of $\infty$-operads, we can take the monoidal envelope of
each fiber to obtain a family of symmetric monoidal $\infty$-categories.
We will also prove that monoidal envelopes of families of $\infty$-operads
enjoys the expected universal property. 

We remark that the proofs of the results in this section are only
slight modifications of Lurie's original proofs of various results
concerning monoidal envelopes, which can be found in \cite[Section 2.2.4]{HA}.
Nevertheless, for the sake of completeness, we will record the proofs
whenever Lurie's proof does not apply verbatim.

\subsection{Definition and Construction}

In this subsection, we define monoidal envelopes of families of $\infty$-operads
and show that they are again families of $\infty$-operads.

We begin with a generalization of Lurie's construction \cite[Construction 2.2.4.1]{HA}:
\begin{defn}
Let $\cal M^{\t}$ be a generalized $\infty$-operad \cite[Definition 2.3.2.1]{HA}.
We define $\Act\pr{\cal M^{\t}}\subset\Fun\pr{\Delta^{1},\cal M^{\t}}$
to be the full subcategory spanned by the active morphisms. If $p:\cal M^{\t}\to\cal N^{\t}$
is a fibration of generalized $\infty$-operads, the \textbf{$\cal N$-monoidal
envelope} of $\cal M^{\t}$ is defined to be the fiber product
\[
\Env_{\cal N}\pr{\cal M}^{\t}=\cal M^{\t}\times_{\Fun\pr{\{0\},\cal N^{\t}}}\Act\pr{\cal N^{\t}}.
\]
In the case $\cal N^{\t}=N\pr{\Fin_{\ast}}$, we will write $\Env_{\cal N}\pr{\cal M}^{\t}=\Env\pr{\cal M}^{\t}$.
We will regard $\Env_{\cal N}\pr{\cal M}^{\t}$ as an $\infty$-category
over $\cal N^{\t}$ via the composite
\[
\Env_{\cal N}\pr{\cal M}^{\t}\to\Act\pr{\cal N^{\t}}\xrightarrow{\opn{ev}_{1}}\cal N^{\t}.
\]
\end{defn}
Thus, if $p:\cal M^{\t}\to N\pr{\Fin_{\ast}}$ is a generalized $\infty$-operad,
then $\Env\pr{\cal M}^{\t}$ is the $\infty$-category of pairs $\pr{M,\alpha:p\pr M\to\inp k}$,
where $M\in\cal M^{\t}$ and $\alpha$ is an active morphism of $N\pr{\Fin_{\ast}}$.
In particular, the $\infty$-category $\Env\pr{\cal M}=\Env\pr{\cal M}_{\inp 1}^{\t}$
may be identified with $\cal M_{\act}^{\t}$. The intuition here is
that an object $M\in\cal M_{\inp n}^{\t}$ is regarded as a ``formal
tensor product'' of the objects $\rho_{!}^{i}\pr M\in\cal M$.

Note that the (fully faithful) diagonal embedding $\cal N^{\t}\to\Act\pr{\cal N^{\t}}$
induces a fully faithful embedding $\cal M^{\t}\hookrightarrow\Env_{\cal N}\pr{\cal M}^{\t}$,
which partly justifies the terminology ``envelope.'' In fact, the
monoidal envelope enjoys a certain universal property, as we will
see in Subsection \ref{subsec:Universal-Property-of_monoidal_env}.

The goal of this note is to prove the following result, which is a
generalization of \cite[Proposition 2.2.4.4]{HA}.
\begin{prop}
\label{prop:2.2.4.4}Let $\cal C$ be an $\infty$-category and let
$p:\cal M^{\t}\to\cal C\times N\pr{\Fin_{\ast}}$ be a $\cal C$-family
of $\infty$-operads. Let 
\[
q:\Env\pr{\cal M}^{\t}\to\cal C\times N\pr{\Fin_{\ast}}
\]
denote the functor induced by the functors $\Env\pr{\cal M}^{\t}\to N\pr{\Fin_{\ast}}$
and $\Env\pr{\cal M}^{\t}\to\cal M^{\t}\to\cal C$. Then:
\begin{itemize}
\item [(a)]The functor $q$ makes $\Env\pr{\cal M}^{\t}$ into a $\cal C$-family
of $\infty$-operads. 
\item [(b)]A morphism of $\Env\pr{\cal M}^{\t}$ whose image in $\cal M^{\t}$
is inert is $q$-cocartesian. The converse holds if it lies over an
inert morphism in $\cal C\times N\pr{\Fin_{\ast}}$.
\end{itemize}
\end{prop}
\begin{rem}
In the situation of Proposition \ref{prop:2.2.4.4}, the fiber $\Env\pr{\cal M}_{C}^{\t}=\Env\pr{\cal M}^{\t}\times_{\cal C}\{C\}$
over an object $C\in\cal C$ is the monoidal envelope of the $\infty$-operad
$\cal M_{C}^{\t}$, which is a symmetric monoidal $\infty$-category
by \cite[Proposition 2.2.4.4]{HA}. 
\end{rem}
We will return to the proof after a sequence of lemmas.
\begin{lem}
\cite[Lemma 2.2.4.11]{HA}\label{lem:2.2.4.11}Let $p:\cal E\to\cal D$
be a cocartesian fibration of $\infty$-categories. Suppose there
is a full subcategory $\cal C\subset\cal E$ which has the following
properties:

\begin{enumerate}[label=(\roman*)]

\item For each object $D\in\cal D$, the inclusion $\cal C_{D}\subset\cal E_{D}$
admits a left adjoint $L_{D}:\cal E_{D}\to\cal C_{D}$.

\item For each morphism $f:D\to D'$ in $\cal D$, the associated
functor $f_{!}:\cal E_{D}\to\cal E_{D'}$ carries $L_{D}$-equivalences
(i.e., its image under $L_{D}$ is an equivalence) to $L_{D'}$-equivalences.

\end{enumerate}

Let $q=p\vert_{\cal C}:\cal C\to\cal D$ denote the restriction of
$p$. The following holds:
\begin{enumerate}
\item The functor $q=p\vert_{\cal C}:\cal C\to\cal D$ is a cocartesian
fibration.
\item Let $g:D\to D'$ be a morphism in $\cal D$ and $f:C\to C'$ a morphism
in $\cal C$ lifting $g$. Then $f$ is $q$-cocartesian if and only
if the map $g_{!}C\to C'$ is an $L_{D'}$-equivalence, where $g_{!}:\cal E_{D}\to\cal E_{D'}$
is the functor induced by $g$.
\end{enumerate}
\end{lem}
%
\begin{lem}
\cite[Remark 2.2.4.12]{HA}\label{lem:2.2.4.12}Let $p:\cal E\to\cal D$
be a cocartesian fibration of $\infty$-categories and $\cal C\subset\cal E$
a full subcategory. Consider the following conditions for $\cal C$:

\begin{enumerate}[label=(\roman*)]

\item For each object $D\in\cal D$, the inclusion $\cal C_{D}\subset\cal E_{D}$
admits a left adjoint $L_{D}:\cal E_{D}\to\cal C_{D}$.

\item For each morphism $f:D\to D'$ in $\cal D$, the associated
functor $f_{!}:\cal E_{D}\to\cal E_{D'}$ carries $L_{D}$-equivalences
(i.e., its image under $L_{D}$ is an equivalence) to $L_{D'}$-equivalences.

\end{enumerate}

\begin{enumerate}[label=(\roman*')]

\item The inclusion $\cal C\subset\cal E$ admits a left adjoint
$L:\cal E\to\cal C$.

\item The functor $p$ carries each $L$-equivalecne to to an equivalence.

\end{enumerate}

The conditions (i) and (ii) are equivalent to the conditions (i')
and (ii'). Moreover, if these conditions are satisfied, then for each
object $D\in\cal D$, a morphism in $\cal C_{D}$ is an $L$-equivalence
if and only if it is an $L_{D}$-equivalence.
\end{lem}
%
\begin{lem}
\cite[Lemma 2.2.4.13]{HA}\label{lem:2.2.4.13}Let $p:\cal C\to\cal D$
be an inner fibration of $\infty$-categories and $\cal D'\subset\cal D$
a full subcategory. Set $\cal C'=\cal C\times_{\cal D'}\cal D$. Assume
the following:

\begin{enumerate}[label=(\roman*)]

\item The inclusion $\cal D'\subset\cal D$ admits a left adjoint.

\item For each object $C\in\cal C$, the morphism $pC\to D'$ which
exhibits $D$ as a $\cal D'$-localization of $pC$ lifts to a $p$-cocartesian
morphism $C\to C'$.

\end{enumerate}

Then $\cal C'$ is a reflective subcategory of $\cal C$, and a morphism
$f:C\to C'$ in $\cal C$ exhibits $C'$ as a $\cal C'$-localization
of $C$ if and only if $f$ is $p$-cocartesian and the morphism $p\pr f$
exhibits $p\pr{C'}$ as a $\cal D'$-localization of $p\pr C$.
\end{lem}
%
\begin{lem}
\cite[Lemma 2.2.4.14]{HA}\label{lem:2.2.4.14}Let $p:\cal M^{\t}\to\cal N^{\t}$
be a fibration of generalized $\infty$-operads. Set $\cal D=\cal M^{\t}\times_{\Fun\pr{\{0\},\cal N^{\t}}}\Fun\pr{\Delta^{1},\cal N^{\t}}$.
The inclusion
\[
\Env_{\cal N}\pr{\cal M}^{\t}\subset\cal D
\]
admits a left adjoint. Moreover, a morphism $\alpha:D\to D'$ in $\cal D$
exhibits $D'$ as an $\Env_{\cal N}\pr{\cal M}^{\t}$-localization
of $D$ if and only if $D'\in\Env_{\cal N}\pr{\cal M}^{\t}$, the
image of $\alpha$ in $\cal M^{\t}$ is inert, and the image of $\alpha$
in $\cal N^{\t}$ is an equivalence.
\end{lem}
\begin{proof}
We recall from \cite[Lemma 5.2.8.19]{HTT} that the inclusion $\Act\pr{\cal N^{\t}}\subset\Fun\pr{\Delta^{1},\cal N^{\t}}$
admits a left adjoint and that a morphism $\alpha:g\to g'$ in $\Fun\pr{\Delta^{1},\cal N^{\t}}$
corresponding to a square % https://q.uiver.app/?q=WzAsNCxbMCwwLCJcXGJ1bGxldCJdLFsxLDAsIlxcYnVsbGV0Il0sWzEsMSwiXFxidWxsZXQiXSxbMCwxLCJcXGJ1bGxldCJdLFswLDEsImYiXSxbMSwyLCJnJyJdLFszLDIsImYnIiwyXSxbMCwzLCJnIiwyXV0=
\[\begin{tikzcd}
	\bullet & \bullet \\
	\bullet & \bullet
	\arrow["f", from=1-1, to=1-2]
	\arrow["{g'}", from=1-2, to=2-2]
	\arrow["{f'}"', from=2-1, to=2-2]
	\arrow["g"', from=1-1, to=2-1]
\end{tikzcd}\]exhibits $g'$ as an $\Act\pr{\cal N^{\t}}$-localization if and only
if $g'$ is active, $f$ is inert, and $f'$ is an equivalence. 

Now let $\pr{M,g:p\pr M\to N}$ be an object of $\Env_{\cal N}\pr{\cal M}^{\t}$,
and find a morphism $\alpha:g\to g'$ as in the previous paragraph.
According to Lemma \ref{lem:2.2.4.13}, it will suffice to show that
$\alpha$ admits a lift with domain $\pr{M,g}$ which is cocartesian
with respect to the projection $\cal D\to\Fun\pr{\Delta^{1},\cal N^{\t}}$.
Since $f$ is inert and $p$ is a fibration of generalized $\infty$-operads,
we can lift the morphism $f$ to an inert morphism $\beta:M\to M'$
in $\cal M^{\t}$. Then $\pr{\beta,\alpha}$ determines a morphism
$\pr{M,g}\to\pr{M',g'}$ which lifts $\alpha$. Since its image in
$\cal M^{\t}$ is $p$-cocartesian, the morphism $\pr{\alpha,\beta}$
is cocartesian with respect to $\cal D\to\Fun\pr{\Delta^{1},\cal N^{\t}}$.
The claim follows.
\end{proof}
\begin{lem}
\cite[Corollary 2.4.7.12]{HTT}\label{lem:corr_cocart}Let $f:\cal C\to\cal D$
be a functor of $\infty$-categories. The evaluation at $1\in\Delta^{1}$
induces a cocartesian fibration $p:\cal E=\cal C\times_{\Fun\pr{\{0\},\cal D}}\Fun\pr{\Delta^{1},\cal D}\to\cal D$,
and a morphism in $\cal E$ is $p$-cocartesian if and only if its
image in $\cal C$ is an equivalence.
\end{lem}
%
\begin{lem}
\cite[Lemma 2.2.4.15]{HA}\label{lem:2.2.4.15}Let $p:\cal M^{\t}\to\cal N^{\t}$
be a fibration of generalized $\infty$-operads. Set $\cal D=\cal M^{\t}\times_{\Fun\pr{\{0\},\cal N^{\t}}}\Fun\pr{\Delta^{1},\cal N^{\t}}$.
The following holds:
\begin{enumerate}
\item Evaluation at $1\in\Delta^{1}$ induces a cocartesian fibration $q':\cal D\to\cal N^{\t}$. 
\item A morphism in $\cal D$ is $q'$-cocartesian if and only if its image
in $\cal M^{\t}$ is an equivalence.
\item The map $q'$ restricts to a cocartesian fibration $q:\Env_{\cal N}\pr{\cal M}^{\t}\to\cal N^{\t}$.
\item A morphism $f$ in $\Env_{\cal N}\pr{\cal M}^{\t}$ is $q$-cocartesian
if and only if its image in $\cal M^{\t}$ is inert.
\end{enumerate}
\end{lem}
\begin{proof}
Assertions (1) and (2) follow from Lemma \ref{lem:corr_cocart}. Let
now $L:\cal D\to\Env_{\cal N}\pr{\cal M}^{\t}$ denote the left adjoint
of the inclusion, which exists by Lemma \ref{lem:2.2.4.14}. According
to this lemma, the functor $q'$ maps $L$-equivalences to equivalences.
Thus, by Lemmas \ref{lem:2.2.4.11} and \ref{lem:2.2.4.12}, the restriction
$q=q'\vert\Env_{\cal N}\pr{\cal M}^{\t}$ is a cocartesian fibration,
and a morphism $\pr{f,g}:\pr{M,\alpha}\to\pr{M',\alpha'}$ in $\Env_{\cal N}\pr{\cal M}^{\t}$
is $q$-cocartesian if and only if the map $\theta:g_{!}\pr{M,\alpha}\to\pr{M',\alpha'}$
is an $L$-equivalence. Here $g_{!}:\cal D_{X}\to\cal D_{X'}$ is
the functor induced by the morphism $g$. Now up to equivalence, the
map $\theta$ may be identified with the map $\pr{f,\id}:\pr{M,g\circ\alpha}\to\pr{M',\alpha'}$.
So Lemma \ref{lem:2.2.4.14} tells us that the latter condition holds
if and only if $f:M\to M'$ is inert. This proves (4).
\end{proof}
\begin{lem}
\cite[Lemma 2.2.4.16]{HA}\label{lem:2.2.4.16}Let $n\geq0$, and
let $\beta:\inp m\to\inp n$ and $\beta_{i}:\inp{m_{i}}\to\inp 1$
be active morphisms for $1\leq i\leq n$. Suppose we are given a commutative
diagram % https://q.uiver.app/?q=WzAsNCxbMCwwLCJcXGxhbmdsZSBtXFxyYW5nbGUiXSxbMSwwLCJcXGxhbmdsZSBtX2lcXHJhbmdsZSJdLFsxLDEsIlxcbGFuZ2xlIDEgXFxyYW5nbGUiXSxbMCwxLCJcXGxhbmdsZSBuXFxyYW5nbGUiXSxbMCwxLCJcXGdhbW1hX2kiXSxbMSwyLCJcXGJldGFfaSJdLFszLDIsIlxccmhvXmkiLDJdLFswLDMsIlxcYmV0YSIsMl1d
\[\begin{tikzcd}
	{\langle m\rangle} & {\langle m_i\rangle} \\
	{\langle n\rangle} & {\langle 1 \rangle}
	\arrow["{\gamma_i}", from=1-1, to=1-2]
	\arrow["{\beta_i}", from=1-2, to=2-2]
	\arrow["{\rho^i}"', from=2-1, to=2-2]
	\arrow["\beta"', from=1-1, to=2-1]
\end{tikzcd}\]for $1\leq i\leq n$ such that $\gamma_{i}$ is inert. Then the morphisms
$\{\beta\to\beta_{i}\}_{1\leq i\leq n}$ form a limit cone relative
to the map $p=\opn{ev}_{1}:\Act\pr{N\pr{\Fin_{\ast}}}\to N\pr{\Fin_{\ast}}$.
\end{lem}
%
\begin{proof}
[Proof of Proposition \ref{prop:2.2.4.4}]We begin with the first
half of part (b). Let $r:\cal C\times N\pr{\Fin_{\ast}}\to N\pr{\Fin_{\ast}}$
denote the projection. If a morphism $f$ in $\Env\pr{\cal M}^{\t}$
has the property that its image in $\cal M^{\t}$ is inert, then we
know from Lemma \ref{lem:2.2.4.15} that $f$ is $rq$-cocartesian,
and the image of $f$ in $\cal C$ is an equivalence. Thus $q\pr f$
ir $r$-cocartesian. Hence $f$ is $q$-cocartesian.

Next we proceed to (a). We must prove the following.
\begin{enumerate}
\item The functor $q$ is a categorical fibration.
\item Let $\pr{M,\alpha:\inp{n_{M}}\to\inp k}$ be an object of $\Env\pr{\cal M}^{\t}$
with image $C\in\cal C$, and let $\beta:\inp k\to\inp l$ be an inert
map. There is a $q$-cocartesian morphism $\pr{M,\alpha}\to\pr{M',\alpha'}$
lying over $\pr{\id_{C},\beta}$ whose image in $\cal M^{\t}$ is
inert.
\item Let $E\in\Env\pr{\cal M}^{\t}$ and set $\pr{C,\inp m}=q\pr E$. Then
the $q$-cocartesian morphisms $\gamma=\{E\to E_{i}\}_{1\leq i\leq k}$
lying over the morphisms $\{\pr{\id_{C},\rho^{i}}:\pr{C,\inp m}\to\pr{C,\inp 1}\}_{1\leq i\leq k}$
form a $q$-limit cone.
\item Let $k\geq1$, $C\in\cal C$, and $E_{1},\dots,E_{k}\in\Env\pr{\cal M}_{\pr{X,\inp 1}}^{\t}$.
There is an object $E\in\Env\pr{\cal M}_{\pr{C,\inp k}}^{\t}$ and
$q$-cocartesian morphisms $E\to E_{i}$ over $\rho^{i}$.
\end{enumerate}
Note that (2) and the first half of part (b) implies the latter half
of (b).

Assertion (1) is clear, because $q$ is the composite
\[
\Env\pr{\cal M}^{\t}\to\cal C\times\Act\pr{N\pr{\Fin_{\ast}}}\xrightarrow{\id\times\opn{ev}_{1}}\cal C\times N\pr{\Fin_{\ast}},
\]
and the first map is a pullback of $p$ and the second map is a categorical
fibration. For assertion (2), find a commutative diagram % https://q.uiver.app/?q=WzAsNCxbMCwwLCJcXGxhbmdsZSBuX01cXHJhbmdsZSJdLFsxLDAsIlxcbGFuZ2xlIGtcXHJhbmdsZSJdLFsxLDEsIlxcbGFuZ2xlIGxcXHJhbmdsZSJdLFswLDEsIlxcbGFuZ2xlICBuX00nXFxyYW5nbGUiXSxbMCwxLCJcXGFscGhhIl0sWzEsMiwiXFxiZXRhIl0sWzAsMywiXFxnYW1tYSIsMl0sWzMsMiwiXFxhbHBoYSciLDJdXQ==
\[\begin{tikzcd}
	{\langle n_M\rangle} & {\langle k\rangle} \\
	{\langle  n_M'\rangle} & {\langle l\rangle}
	\arrow["\alpha", from=1-1, to=1-2]
	\arrow["\beta", from=1-2, to=2-2]
	\arrow["\gamma"', from=1-1, to=2-1]
	\arrow["{\alpha'}"', from=2-1, to=2-2]
\end{tikzcd}\]in $\Fin_{\ast}$ with $\gamma$ inert and $\alpha'$ active. Since
$p$ is an $\cal C$-family of $\infty$-operads, there is an inert
morphism $f:M\to M'$ in $\cal M^{\t}$ lying over $\pr{\id_{C},\gamma}$.
The maps $f,\gamma,\beta$ determines a morphism $\pr{M,\alpha}\to\pr{M',\alpha'}$
in $\Env\pr{\cal M}^{\t}$ lying over $\pr{\id_{C},\beta}$, which
is $q$-cocartesian by the ``if'' part of part (b). 

For assertion (3), set $\cal D=\cal M^{\t}\times_{\Fun\pr{\{0\},\cal N^{\t}}}\Fun\pr{\Delta^{1},\cal N^{\otimes}}$
and consider the commutative diagram % https://q.uiver.app/?q=WzAsOCxbMCwwLCJcXG9wZXJhdG9ybmFtZXtFbnZ9X3tcXG1hdGhjYWx7Q31cXHRpbWVzIE4oXFxtYXRoc2Z7RmlufV9cXGFzdCl9KFxcbWF0aGNhbHtNfSleXFxvdGltZXMgIl0sWzAsMSwiXFxtYXRoY2Fse0N9XFx0aW1lcyBcXG1hdGhybXtBY3R9KE4oXFxtYXRoc2Z7RmlufV9cXGFzdCkpIl0sWzEsMCwiXFxtYXRoY2Fse0R9Il0sWzEsMSwiXFxtYXRoY2Fse0N9XFx0aW1lcyBcXG9wZXJhdG9ybmFtZXtGdW59KFxcRGVsdGFeMSxOKFxcbWF0aHNme0Zpbn1fXFxhc3QpKSJdLFsyLDAsIlxcbWF0aGNhbHtNfV5cXG90aW1lcyAiXSxbMiwxLCJcXG1hdGhjYWx7Q31cXHRpbWVzIE4oXFxtYXRoc2Z7RmlufV9cXGFzdCkiXSxbMSwyLCJcXG1hdGhjYWx7Q31cXHRpbWVzIE4oXFxtYXRoc2Z7RmlufV9cXGFzdCkiXSxbMCwyLCJcXG1hdGhjYWx7Q31cXHRpbWVzIE4oXFxtYXRoc2Z7RmlufV9cXGFzdCkiXSxbMCwxLCJxXzAiLDJdLFswLDIsIlxcYWxwaGEiXSxbMiwzXSxbMiw0LCJcXGJldGEiXSxbNCw1LCJwIl0sWzMsNSwiXFxvcGVyYXRvcm5hbWV7aWR9X3tcXG1hdGhjYWx7Q319XFx0aW1lcyBcXG9wZXJhdG9ybmFtZXtldn1fMCIsMl0sWzEsM10sWzMsNiwiXFxvcGVyYXRvcm5hbWV7aWR9X3tcXG1hdGhjYWx7Q319XFx0aW1lcyBcXG9wZXJhdG9ybmFtZXtldn1fMSJdLFsxLDcsIlxcb3BlcmF0b3JuYW1le2lkfV97XFxtYXRoY2Fse0N9fVxcdGltZXMgXFxvcGVyYXRvcm5hbWV7ZXZ9XzEiLDJdLFs3LDYsImVxdWFsIiwyXV0=
\[\begin{tikzcd}
	{\operatorname{Env}_{\mathcal{C}\times N(\mathsf{Fin}_\ast)}(\mathcal{M})^\otimes } & {\mathcal{D}} & {\mathcal{M}^\otimes } \\
	{\mathcal{C}\times \mathrm{Act}(N(\mathsf{Fin}_\ast))} & {\mathcal{C}\times \operatorname{Fun}(\Delta^1,N(\mathsf{Fin}_\ast))} & {\mathcal{C}\times N(\mathsf{Fin}_\ast)} \\
	{\mathcal{C}\times N(\mathsf{Fin}_\ast)} & {\mathcal{C}\times N(\mathsf{Fin}_\ast)}
	\arrow["{q_0}"', from=1-1, to=2-1]
	\arrow["\alpha", from=1-1, to=1-2]
	\arrow[from=1-2, to=2-2]
	\arrow["\beta", from=1-2, to=1-3]
	\arrow["p", from=1-3, to=2-3]
	\arrow["{\operatorname{id}_{\mathcal{C}}\times \operatorname{ev}_0}"', from=2-2, to=2-3]
	\arrow[from=2-1, to=2-2]
	\arrow["{\operatorname{id}_{\mathcal{C}}\times \operatorname{ev}_1}", from=2-2, to=3-2]
	\arrow["{\operatorname{id}_{\mathcal{C}}\times \operatorname{ev}_1}"', from=2-1, to=3-1]
	\arrow[equal, from=3-1, to=3-2]
\end{tikzcd}\]whose squares in the top rows are cartesian. We must show that $\gamma$
is an $\pr{\id_{\cal C}\times\opn{ev}_{1}}\circ q_{0}$-limit cone.
Since relative limits are stable under pullbacks \cite[Proposition 4.3.1.5]{HTT},
it suffices to show that $\beta\alpha\gamma$ is a $p$-limit cone
and that $q_{0}\gamma$ is an $\id_{\cal C}\times\opn{ev}_{1}$-limit
cone. First we consider $\beta\alpha\gamma$. Write $E=\pr{M,\alpha:\inp{n_{M}}\to\inp n}$
and $E_{i}=\pr{M_{i},\alpha_{i}:\inp{n_{M_{i}}}\to\inp 1}$. The morphism
$\gamma_{i}:E\to E_{i}$ can be depicted by a diagram% https://q.uiver.app/?q=WzAsNixbMSwwLCJcXGxhbmdsZSBuX01cXHJhbmdsZSJdLFsxLDEsIlxcbGFuZ2xlIG5fe01faX1cXHJhbmdsZSJdLFsyLDAsIlxcbGFuZ2xlIG1cXHJhbmdsZSJdLFsyLDEsIlxcbGFuZ2xlIDFcXHJhbmdsZSJdLFswLDAsIk0iXSxbMCwxLCJNX2kiXSxbMiwzLCJcXHJob15pIl0sWzEsMywiXFxhbHBoYV9pIiwyXSxbMCwyLCJcXGFscGhhIl0sWzAsMSwiXFxiZXRhX2kiLDJdLFs0LDUsImZfaSIsMl1d
\[\begin{tikzcd}
	M & {\langle n_M\rangle} & {\langle m\rangle} \\
	{M_i} & {\langle n_{M_i}\rangle} & {\langle 1\rangle}
	\arrow["{\rho^i}", from=1-3, to=2-3]
	\arrow["{\alpha_i}"', from=2-2, to=2-3]
	\arrow["\alpha", from=1-2, to=1-3]
	\arrow["{\beta_i}"', from=1-2, to=2-2]
	\arrow["{f_i}"', from=1-1, to=2-1]
\end{tikzcd}\]Since $\gamma_{i}$ is $q$-cocartesian, the morphism $f_{i}$ is
inert. Hence $\beta_{i}$ is inert. Since the horizontal arrows are
active, the commutativity of the diagram implies that the map $\beta_{i}:\inp{n_{M}}\to\inp{n_{M_{i}}}$
is the inert map which induces a bijection $\alpha^{-1}\pr i\cong\inp{n_{M_{i}}}^{\circ}$.
Combining this with the fact that $p$ is a $\cal C$-family of $\infty$-operads,
we deduce that the morphisms $\{f_{i}\}_{i}=\{\beta\alpha\gamma_{i}\}_{i}$
form a $p$-limit cone. For $q_{0}\gamma$, we use Lemma \ref{lem:2.2.4.16}.

To prove (4), write $E_{i}=\pr{M_{i},\alpha_{i}:\inp{n_{M_{i}}}\to\inp 1}$.
Let $n_{M}=n_{M_{1}}+\cdots+n_{M_{k}}$, and fix a bijection $\inp{n_{M}}\cong\Vee_{i=1}^{k}\inp{n_{M_{i}}}$.
Since $\cal M_{C}^{\t}$ is an $\infty$-operad, we can find an object
$M\in\cal C_{\pr{C,\inp{n_{M}}}}^{\t}$ and which admits a $p$-cocartesian
morphism $f_{i}:M\to M_{i}$ lying over the inert map $\beta_{i}:\inp{n_{M}}\to\inp{n_{M_{i}}}$
for each $1\leq i\leq k$. Let $\alpha:\inp{n_{M}}\to\inp k$ denote
the function which maps $\inp{n_{M_{i}}}^{\circ}$ to $i\in\inp k^{\circ}$,
and set $E=\pr{M,\alpha:\inp{n_{M}}\to\inp k}$. The triple $\pr{f_{i},\beta_{i},\rho^{i}}$
determines a morphism $E\to E_{i}$ which is $q$-cocartesian by part
(b). This proves (4).
\end{proof}

\subsection{\label{subsec:Universal-Property-of_monoidal_env}Universal Property
of Monoidal Envelopes}

In this subsection, we prove that the monoidal envelope of families
of $\infty$-operads has the expected universal property. The contents
of this subsection will not be used elsewhere in the note.

We begin with a definition.
\begin{defn}
Let $\cal C$ be an $\infty$-category and let $p:\cal M^{\t}\to\cal C\times N\pr{\Fin_{\ast}}$
a $\cal C$-family of $\infty$-operads. Let $q:\cal D^{\t}\to N\pr{\Fin_{\ast}}$
be a symmetric monoidal $\infty$-category. We let $\Fun^{\t}\pr{\Env\pr{\cal M},\cal D}\subset\Fun_{N\pr{\Fin_{\ast}}}\pr{\Env\pr{\cal M}^{\t},\cal D^{\t}}$
denote the full subcategory spanned by the functors which restrict
to a symmetric monoidal functor $\Env\pr{\cal M_{C}}^{\t}\to\cal D^{\t}$
for each $C\in\cal C$. 
\end{defn}
Here is the universal property of monoidal envelopes. The result is
a generalization of \cite[Proposition 2.2.4.9]{HA}.
\begin{prop}
Let $\cal C$ be an $\infty$-category and $p:\cal M^{\t}\to\cal C\times N\pr{\Fin_{\ast}}$
a $\cal C$-family of $\infty$-operads. Let $q:\cal D^{\t}\to N\pr{\Fin_{\ast}}$
be a symmetric monoidal $\infty$-category. The embedding $i:\cal M^{\t}\hookrightarrow\Env\pr{\cal M}^{\t}$
induces an equivalence of $\infty$-categories
\[
\Fun^{\t}\pr{\Env\pr{\cal M},\cal D}\xrightarrow{\simeq}\Alg_{\cal M}\pr{\cal D}.
\]
\end{prop}
\begin{proof}
Let $\cal E^{\t}$ denote the essential image of $i$. In other words,
$\cal E^{\t}$ is the full subcategory of $\Env\pr{\cal M}^{\t}$
spanned by those objects $\pr{M,\alpha:p\pr M\to\inp n}$ such that
$\alpha$ is an equivalence. The restriction map $\Alg_{\cal E}\pr{\cal D}\to\Alg_{\cal M}\pr{\cal D}$
is an equivalence of $\infty$-categories, so it will suffice to show
that the map
\[
\Fun^{\t}\pr{\Env\pr{\cal M},\cal D}\to\Alg_{\cal E}\pr{\cal D}
\]
is a trivial fibration. For this, it suffices to prove the following:

\begin{enumerate}[label=(\alph*)]

\item Every functor $\cal E^{\t}\to\cal D^{\t}$ over $N\pr{\Fin_{\ast}}$
admits a $q$-left Kan extension $\Env\pr{\cal M}^{\t}\to\cal D^{\t}$.

\item A functor $F\in\Fun_{N\pr{\Fin_{\ast}}}\pr{\Env\pr{\cal M},\cal D}$
restricts to a symmetric monoidal functor $\Env\pr{\cal M_{C}}^{\t}\to\cal D^{\t}$
for each $C\in\cal C$ if and only if it is a $q$-left Kan extension
of $F\vert_{\cal E^{\t}}$ and $F\vert_{\cal E^{\t}}$ is a morphism
of generalized $\infty$-operads.

\end{enumerate}

We start from (a). For each object $\pr{M,\alpha:p_{2}\pr M\to\inp n}$
in $\Env\pr{\cal M}^{\t}$, the $\infty$-category $\cal E_{/\pr{M,\alpha}}^{\t}$
has a terminal object, given by the map $\pr{\id_{M},\alpha}:\pr{M,\id_{p_{2}\pr M}}\to\pr{M,\alpha}$
(Lemma \ref{lem:2.2.4.13}). Therefore, if $F:\cal E^{\t}\to\cal D^{\t}$
is an arbitrary functor over $N\pr{\Fin_{\ast}}$, then the diagram
$\cal E_{/\pr{M,\alpha}}^{\t}\to\cal E^{\t}\xrightarrow{F}\cal D^{\t}$
has a $q$-colimit cone if and only if there is a $q$-cocartesian
morphism $F\pr{M,\id_{p_{2}\pr M}}\to D$ in $\cal D$ lifting $\alpha$.
The existence of such a $q$-cocartesian morphism follows from the
fact that $q$ is a cocartesian fibration.

For (b), let $p':\Env\pr{\cal M}^{\t}\to\cal C\times N\pr{\Fin_{\ast}}$
denote the projection. By the discussion in the previous paragraph,
it will suffice to show that $F$ preserves $p'$-cocartesian morphisms
lying over degenerate edges of $\cal X$ if and only if for each object
$\pr{M,\alpha:p_{2}\pr M\to\inp n}\in\Env\pr{\cal M}^{\t}$, the map
$F\pr{M,\id_{p_{2}\pr M}}\to F\pr{M,\alpha}$ is a $q$-cocartesian
morphism. Necessity is clear, since the morphism $\pr{M,\id_{p_{2}\pr M}}\to\pr{M,\alpha}$
is $p'$-cocartesian by Proposition \ref{prop:2.2.4.4}. For sufficiency,
suppose that every morphism of the form $\pr{M,\id_{p_{2}\pr M}}\to\pr{M,\alpha}$
is mapped to a $q$-cocartesian morphism. Let $\pr{f,g}:\pr{M,\alpha:p_{2}\pr M\to\inp n}\to\pr{M',\alpha':p_{2}\pr{M'}\to\inp{n'}}$
be a $p'$-cocartesian morphism in $\Env\pr{\cal M^{\t}}$ lying over
a degenerate edge of $\cal C$. We wish to show that $F\pr{f,g}$
is $q$-cocartesian. By hypothesis, the map $F\pr{M,\id_{p_{2}\pr M}}\to F\pr{M,\alpha}$
is $q$-cocartesian. So it suffice to show that the composite $F\pr{M,\id_{p_{2}\pr M}}\to F\pr{M,\alpha}\to F\pr{M',\alpha'}$
is $q$-cocartesian. We may therefore reduce to the case $\alpha=\id_{p_{2}\pr M}$.
Factor the map $g:p_{2}\pr M\to\inp{n'}$ as 
\[
p_{2}\pr M\xrightarrow{g'}\inp{n''}\xrightarrow{g''}\inp{n'},
\]
where $g'$ is inert and $g''$ is active. Lift $\pr{\id_{p_{1}\pr M},g'}$
to a $p$-cocartesian morphism $f':M\to M''$ in $\cal M^{\t}$. The
morphism $\pr{f',g'}:\pr{M,\id_{p_{2}\pr M}}\to\pr{M'',\id_{\inp{n''}}}$
is $p'$-cocartesian, so we can find a triangle in $\Env\pr{\cal M_{p_{1}\pr M}}^{\t}$
which we depict as % https://q.uiver.app/?q=WzAsNixbMCwzLCJwXzIoTSkiXSxbMSwyLCJcXGxhbmdsZSBuJydcXHJhbmdsZSJdLFsyLDMsIlxcbGFuZ2xlIG4nXFxyYW5nbGUiXSxbMiwxLCJwXzIoTScpIl0sWzEsMCwicF8yKE0nJykiXSxbMCwxLCJwXzIoTSkiXSxbMCwxLCJnJyJdLFsxLDIsImcnJyJdLFswLDIsInAoZykiLDJdLFszLDIsIlxcYWxwaGEnIl0sWzQsMywicChmJycpIl0sWzUsNCwicChmJykiXSxbNCwxLCJlcXVhbCIsMix7ImxhYmVsX3Bvc2l0aW9uIjo3MH1dLFs1LDAsImVxdWFsIiwyXSxbNSwzLCJwKGYpIl1d
\[\begin{tikzcd}
	& {p_2(M'')} \\
	{p_2(M)} && {p_2(M')} \\
	& {\langle n''\rangle} \\
	{p_2(M)} && {\langle n'\rangle}
	\arrow["{g'}", from=4-1, to=3-2]
	\arrow["{g''}", from=3-2, to=4-3]
	\arrow["{p(g)}"', from=4-1, to=4-3]
	\arrow["{\alpha'}", from=2-3, to=4-3]
	\arrow["{p(f'')}", from=1-2, to=2-3]
	\arrow["{p(f')}", from=2-1, to=1-2]
	\arrow[equal, from=1-2, to=3-2]
	\arrow[equal, from=2-1, to=4-1]
	\arrow["{p(f)}", from=2-1, to=2-3]
\end{tikzcd}\]The morphism $f''$ is both active and inert, so it is an equivalence.
Thus, by hypothesis, the map $F\pr{f'',g''}$ is a $q$-cocartesian.
The morphism $\pr{f',g'}$ is also $q$-cocartesian, since it is an
inert morphism of $\cal E^{\t}$. Hence $F\pr{f,g}$ is $q$-cocartesian,
as required.
\end{proof}

\section{\label{sec:Direct_sum}The Fiberwise Direct Sum Functor}

Let $\cal O^{\t}$ be an $\infty$-operad. Given objects $X_{1},\dots,X_{n}\in\cal O$,
we can find an object $X\in\cal O_{\inp n}^{\t}$ which admits a $p$-cocartesian
morphism $X\to X_{i}$ over $\rho^{i}:\inp n\to\inp 1$ for each $i$.
Such an object is denoted by $X_{1}\oplus\cdots\oplus X_{n}$ \cite[Remark 2.1.1.15]{HA}.
By inspection, the object $X_{1}\oplus\cdots\oplus X_{n}$ is the
equivalent to the image of the object $\pr{X_{1},\dots,X_{n}}\in\pr{\cal O_{\act}^{\t}}^{n}$
under the tensor product of $\Env\pr{\cal O}=\cal O_{\act}^{\t}$.
So the operation $\oplus$ can be made functorial using monoidal envelopes.
In fact, we can do it fiberwise:
\begin{defn}
Let $\cal C$ be an $\infty$-category, let $\cal M^{\t}\to\cal C\times N\pr{\Fin_{\ast}}$
be a $\cal C$-family of $\infty$-operads, and let $n\geq1$ be a
positive integer. We define a functor
\[
\bigoplus_{i=1}^{n}:\pr{\cal M_{\act}^{\t}}^{n}\times_{\cal C^{n}}\cal C=\cal M_{\act}^{\t}\times_{\cal C}\cdots\times_{\cal C}\cal M_{\act}^{\t}\to\cal M_{\act}^{\t}
\]
over $\cal C$, well-defined up to natural equivalence over $\cal C$,
as follows: According to Proposition \ref{prop:sec1_main}, there
is an equivalence of $\infty$-categories
\[
\Env\pr{\cal M}_{\inp n}^{\t}\xrightarrow{\simeq}\pr{\cal M_{\act}^{\t}}^{n}\times_{\cal C^{n}}\cal C
\]
over $\cal C$, obtained from the functors $\rho_{!}^{i}:\Env\pr{\cal M}_{\inp n}^{\t}\to\Env\pr{\cal M}_{\inp 1}^{\t}=\cal M_{\act}^{\t}$.
The functors $\Env\pr{\cal M}_{\inp n}^{\t}\to\cal C$ and $\pr{\cal M_{\act}^{\t}}^{n}\times_{\cal C^{n}}\cal C\to\cal C$
are categorical fibrations, so the above equivalence admits an inverse
equivalence
\[
\pr{\cal M_{\act}^{\t}}^{n}\times_{\cal C^{n}}\cal C\xrightarrow{\simeq}\Env\pr{\cal M}_{\inp n}^{\t}
\]
over $\cal C$. The functor $\bigoplus_{i=1}^{n}$ is the composite
\[
\pr{\cal M_{\act}^{\t}}^{n}\times_{\cal C^{n}}\cal C\xrightarrow{\simeq}\Env\pr{\cal M}_{\inp n}^{\t}\xrightarrow{\text{forget}}\cal M_{\act}^{\t}.
\]
\end{defn}
In this section, we will prove two important properties of the direct
sum functor: Its universal property and the interaction with slices.

\subsection{\label{subsec:Universal-Property-of}Universal Property of the Direct
Sum Functor }

In this subsection, we will characterize the direct sum functor by
a certain universal property (Corollary \ref{cor:oplus_p-limit}).
\begin{prop}
\label{prop:inverting_weak_equiv_of_pairs}Let $\cal M$ be a model
category. Suppose we are given a commutative diagram % https://q.uiver.app/?q=WzAsNCxbMCwwLCJFIl0sWzEsMCwiRSciXSxbMSwxLCJCJyJdLFswLDEsIkIiXSxbMCwxLCJmIl0sWzEsMiwicCciLDAseyJzdHlsZSI6eyJoZWFkIjp7Im5hbWUiOiJlcGkifX19XSxbMCwzLCJwIiwyLHsic3R5bGUiOnsiaGVhZCI6eyJuYW1lIjoiZXBpIn19fV0sWzMsMiwicSIsMix7InN0eWxlIjp7ImhlYWQiOnsibmFtZSI6ImVwaSJ9fX1dLFszLDIsIlxcc2ltZXEiXSxbMCwxLCJcXHNpbWVxIiwyXV0=
\[\begin{tikzcd}
	E & {E'} \\
	B & {B'}
	\arrow["f", from=1-1, to=1-2]
	\arrow["{p'}", two heads, from=1-2, to=2-2]
	\arrow["p"', two heads, from=1-1, to=2-1]
	\arrow["q"', two heads, from=2-1, to=2-2]
	\arrow["\simeq", from=2-1, to=2-2]
	\arrow["\simeq"', from=1-1, to=1-2]
\end{tikzcd}\]in $\cal M$. Assume the following:
\begin{enumerate}
\item The maps $p,p',q$ are fibrations.
\item The maps $f$ and $q$ are weak equivalences.
\item The object $E'$ is cofibrant. 
\item The map $q$ has a section $s:B'\to B$.
\end{enumerate}
Then there is a map $g:E'\to E$ rendering the diagram % https://q.uiver.app/?q=WzAsNCxbMCwwLCJFJyJdLFswLDEsIkInIl0sWzEsMCwiRSJdLFsxLDEsIkIiXSxbMCwxLCJwJyJdLFswLDIsImciXSxbMSwzLCJzIiwyXSxbMiwzLCJwIl1d
\[\begin{tikzcd}
	{E'} & E \\
	{B'} & B
	\arrow["{p'}", from=1-1, to=2-1]
	\arrow["g", from=1-1, to=1-2]
	\arrow["s"', from=2-1, to=2-2]
	\arrow["p", from=1-2, to=2-2]
\end{tikzcd}\]commutative, such that the composite $fg:E'\to E'$ is homotopic to
the identity in $\cal M_{/B'}$.
\end{prop}
\begin{proof}
Consider the diagram % https://q.uiver.app/?q=WzAsNixbMiwwLCJFIl0sWzIsMSwiQiJdLFsxLDEsIkInIl0sWzEsMCwiRSciXSxbMCwwLCJFIl0sWzAsMSwiQiJdLFswLDEsInAiLDJdLFsyLDEsInMiLDJdLFszLDIsInAnIiwyXSxbNSwyLCJxIiwyXSxbNCw1LCJwIiwyXSxbNCwzLCJmIl0sWzMsMCwiIiwyLHsic3R5bGUiOnsiYm9keSI6eyJuYW1lIjoiZGFzaGVkIn19fV1d
\[\begin{tikzcd}
	E & {E'} & E \\
	B & {B'} & B
	\arrow["p"', from=1-3, to=2-3]
	\arrow["s"', from=2-2, to=2-3]
	\arrow["{p'}"', from=1-2, to=2-2]
	\arrow["q"', from=2-1, to=2-2]
	\arrow["p"', from=1-1, to=2-1]
	\arrow["f", from=1-1, to=1-2]
	\arrow[dashed, from=1-2, to=1-3]
\end{tikzcd}\]in $\cal M_{/B'}$. Since $p$ is a fibration between fibrant objects
and $E'$ is cofibrant, \cite[Proposition A.2.3.1]{HTT} shows that
there is a map $g:E'\to E$ rendering the diagram commutative, such
that $[g][f]=[\id_{E}]$ in $\ho\pr{\cal M_{/B'}}$. Since $[f]$
is an isomorphism in $\ho\pr{\cal M_{/B'}}$, the uniqueness of inverses
implies that $[f][g]=[\id_{E'}]$ in $\ho\pr{\cal M_{/B'}}$. Since
$E'$ is a fibrant-cofibrant object of $\cal M_{/B'}$, we deduce
that $fg$ is homotopic to the identity in $\cal M_{/B'}$.
\end{proof}
\begin{defn}
Let $n\geq1$. Given integers $m_{1},\dots,m_{n}\geq0$, we shall
identify the pointed set $\Vee_{i=1}^{n}\inp{m_{i}}$ with the set
$\inp{m_{1}+\cdots+m_{n}}$ via the map
\[
\Vee_{i=1}^{n}\inp{m_{i}}\to\inp{m_{1}+\cdots+m_{n}}
\]
which maps $k\in\Vee_{i=1}^{n}\inp{m_{i}}$ to $\sum_{j<i}m_{j}+k$.
The maps $\Vee_{i=1}^{n}\inp{m_{i}}\to\inp{m_{i}}$ define an inert
natural transformation
\[
h_{i}:N\pr{\Fin_{\ast}}_{\act}^{n}\times\Delta^{1}\to N\pr{\Fin_{\ast}}.
\]
\end{defn}
\begin{prop}
\label{prop:oplus_p-limit}Let $n\ge1$, let $\cal C$ be an $\infty$-category,
and let $p:\cal M^{\t}\to\cal C\times N\pr{\Fin_{\ast}}$ be a $\cal C$-family
of $\infty$-operads. We can construct the functor $\bigoplus_{1\leq i\leq n}:\pr{\cal M_{\act}^{\t}}^{n}\times_{\cal C^{n}}\cal C\to\cal M^{\t}$
so that for each $1\leq i\leq n$, there is an inert natural transformation
$\bigoplus_{1\leq i\leq n}\to\opn{pr}_{i}$ rendering the diagram
% https://q.uiver.app/?q=WzAsNCxbMCwwLCIoKFxcbWF0aGNhbHtNfV5cXG90aW1lc197XFxtYXRocm17YWN0fX0pXm5cXHRpbWVzIF97XFxtYXRoY2Fse0N9Xm59XFxtYXRoY2Fse0N9KVxcdGltZXMgXFxEZWx0YSBeMSJdLFsxLDAsIlxcbWF0aGNhbHtNfV5cXG90aW1lcyAiXSxbMSwxLCJcXG1hdGhjYWx7Q31cXHRpbWVzIE4oXFxtYXRoc2Z7RmlufV9cXGFzdCkiXSxbMCwxLCJcXG1hdGhjYWx7Q31cXHRpbWVzIE4oXFxtYXRoc2Z7RmlufV9cXGFzdCApXm5fe1xcbWF0aHJte2FjdH19XFx0aW1lcyBcXERlbHRhXjEiXSxbMCwxLCJcXHdpZGV0aWxkZXtofV9pIl0sWzEsMiwicCJdLFszLDIsIlxcb3BlcmF0b3JuYW1le2lkfV97XFxtYXRoY2Fse0N9fVxcdGltZXMgaF9pIiwyXSxbMCwzXV0=
\[\begin{tikzcd}
	{((\mathcal{M}^\otimes_{\mathrm{act}})^n\times _{\mathcal{C}^n}\mathcal{C})\times \Delta ^1} & {\mathcal{M}^\otimes } \\
	{\mathcal{C}\times N(\mathsf{Fin}_\ast )^n_{\mathrm{act}}\times \Delta^1} & {\mathcal{C}\times N(\mathsf{Fin}_\ast)}
	\arrow["{\widetilde{h}_i}", from=1-1, to=1-2]
	\arrow["p", from=1-2, to=2-2]
	\arrow["{\operatorname{id}_{\mathcal{C}}\times h_i}"', from=2-1, to=2-2]
	\arrow[from=1-1, to=2-1]
\end{tikzcd}\]commutative.
\end{prop}
\begin{proof}
We begin with the construction of the functor $\bigoplus_{1\leq i\leq n}$.
For each $1\leq i\leq n$, there is an inert natural transformation
$g_{i}:\Env\pr{N\pr{\Fin_{\ast}}}_{\inp n}^{\t}\times\Delta^{1}\to\Env\pr{N\pr{\Fin_{\ast}}}^{\t}$
from the inclusion to the functor 
\[
\pr{\alpha:\inp k\to\inp n}\to\pr{\alpha^{-1}\pr i_{\ast}\to\inp 1},
\]
where $\alpha^{-1}\pr i_{\ast}$ denotes the unique object $\inp m\in\Fin_{\ast}$
which admits an order-preserving bijection $\alpha^{-1}\pr i\cong\inp m^{\circ}$.
This natural transformation covers $\rho^{i}:\inp n\to\inp 1$. Thus
we may construct the functor $\rho_{!}^{i}:\Env\pr{N\pr{\Fin_{\ast}}}_{\inp n}^{\t}\to N\pr{\Fin_{\ast}}_{\act}$
by $\rho^{i}\pr{\alpha:\inp k\to\inp n}=\alpha^{-1}\pr i_{\ast}$.
Since the functor $p$ is a categorical fibration, it induces a fibration
$\Env\pr{\cal M}^{\t}\to\cal C\times\Env\pr{N\pr{\Fin_{\ast}}}^{\t}$
of generalized $\infty$-operads. Therefore, we can find an inert
natural transformation $\Env\pr{\cal M}_{\inp n}^{\t}\times\Delta^{1}\to\Env\pr{\cal M}^{\t}$
rendering the diagram % https://q.uiver.app/?q=WzAsNSxbMCwwLCJcXG9wZXJhdG9ybmFtZXtFbnZ9KFxcbWF0aGNhbHtNfSleXFxvdGltZXMgX3tcXGxhbmdsZSBuXFxyYW5nbGV9XFx0aW1lcyBcXHswXFx9Il0sWzIsMCwiXFxvcGVyYXRvcm5hbWV7RW52fShcXG1hdGhjYWx7TX0pXlxcb3RpbWVzICJdLFsyLDEsIlxcbWF0aGNhbHtDfVxcdGltZXMgXFxvcGVyYXRvcm5hbWV7RW52fShOKFxcbWF0aHNme0Zpbn1fXFxhc3QpKV5cXG90aW1lcyAiXSxbMCwxLCJcXG9wZXJhdG9ybmFtZXtFbnZ9KFxcbWF0aGNhbHtNfSleXFxvdGltZXMgX3tcXGxhbmdsZSBuXFxyYW5nbGV9XFx0aW1lcyBcXERlbHRhXjEiXSxbMSwxLCJcXG1hdGhjYWx7Q31cXHRpbWVzIFxcb3BlcmF0b3JuYW1le0Vudn0oTihcXG1hdGhzZntGaW59X1xcYXN0KSleXFxvdGltZXMgX3tcXGxhbmdsZSBuXFxyYW5nbGV9XFx0aW1lcyBcXERlbHRhXjEiXSxbMSwyXSxbMCwzXSxbMywxLCIiLDEseyJzdHlsZSI6eyJib2R5Ijp7Im5hbWUiOiJkYXNoZWQifX19XSxbMyw0XSxbNCwyLCJcXG9wZXJhdG9ybmFtZXtpZH1fe1xcbWF0aGNhbHtDfX1cXHRpbWVzIGdfaSIsMl0sWzAsMV1d
\[\begin{tikzcd}
	{\operatorname{Env}(\mathcal{M})^\otimes _{\langle n\rangle}\times \{0\}} && {\operatorname{Env}(\mathcal{M})^\otimes } \\
	{\operatorname{Env}(\mathcal{M})^\otimes _{\langle n\rangle}\times \Delta^1} & {\mathcal{C}\times \operatorname{Env}(N(\mathsf{Fin}_\ast))^\otimes _{\langle n\rangle}\times \Delta^1} & {\mathcal{C}\times \operatorname{Env}(N(\mathsf{Fin}_\ast))^\otimes }
	\arrow[from=1-3, to=2-3]
	\arrow[from=1-1, to=2-1]
	\arrow[dashed, from=2-1, to=1-3]
	\arrow[from=2-1, to=2-2]
	\arrow["{\operatorname{id}_{\mathcal{C}}\times g_i}"', from=2-2, to=2-3]
	\arrow[from=1-1, to=1-3]
\end{tikzcd}\]commutative. We use this inert natural transformation to define the
functor $\rho_{!}^{i}:\Env\pr{\cal M}_{\inp n}^{\t}\to\cal M_{\act}^{\t}$.
This will ensure that the diagram

% https://q.uiver.app/?q=WzAsNCxbMCwxLCJcXG1hdGhjYWx7Q31cXHRpbWVzIChOKFxcbWF0aHNme0Zpbn1fXFxhc3QpX3tcXG1hdGhybXthY3R9fSlebiJdLFsyLDEsIlxcbWF0aGNhbHtDfVxcdGltZXMgXFxvcGVyYXRvcm5hbWV7RW52fShOKFxcbWF0aHNme0Zpbn1fXFxhc3QpKV5cXG90aW1lcyBfe1xcbGFuZ2xlIG4gXFxyYW5nbGV9Il0sWzIsMCwiXFxvcGVyYXRvcm5hbWV7RW52fShcXG1hdGhjYWx7TX0pXlxcb3RpbWVzIF97XFxsYW5nbGUgbiBcXHJhbmdsZX0iXSxbMCwwLCIoXFxtYXRoY2Fse019Xlxcb3RpbWVzIF97XFxtYXRocm17YWN0fX0pXm5cXHRpbWVzIF97XFxtYXRoY2Fse0N9Xm59XFxtYXRoY2Fse0N9Il0sWzMsMF0sWzEsMCwiXFxvcGVyYXRvcm5hbWV7aWR9X3tcXG1hdGhjYWx7Q319XFx0aW1lcyhcXHJob15pXyEpX3tpPTF9Xm4iXSxbMiwxXSxbMiwzLCIoXFxyaG9eaV8hKV97aT0xfV5uIiwyXV0=
\[\begin{tikzcd}
	{(\mathcal{M}^\otimes _{\mathrm{act}})^n\times _{\mathcal{C}^n}\mathcal{C}} && {\operatorname{Env}(\mathcal{M})^\otimes _{\langle n \rangle}} \\
	{\mathcal{C}\times (N(\mathsf{Fin}_\ast)_{\mathrm{act}})^n} && {\mathcal{C}\times \operatorname{Env}(N(\mathsf{Fin}_\ast))^\otimes _{\langle n \rangle}}
	\arrow[from=1-1, to=2-1]
	\arrow["{\operatorname{id}_{\mathcal{C}}\times(\rho^i_!)_{i=1}^n}", from=2-3, to=2-1]
	\arrow[from=1-3, to=2-3]
	\arrow["{(\rho^i_!)_{i=1}^n}"', from=1-3, to=1-1]
\end{tikzcd}\]is commutative. Now the functor $\pr{\rho_{!}^{i}}_{i=1}^{n}:\Env\pr{N\pr{\Fin_{\ast}}}_{\inp n}^{\t}\to\pr{\pr{N\pr{\Fin_{\ast}}}_{\act}}^{n}$
is a trivial fibration, and it has a section $\phi:\pr{N\pr{\Fin_{\ast}}_{\act}}^{n}\to\Env\pr{N\pr{\Fin_{\ast}}}_{\inp n}^{\t}$
given by 
\[
\pr{\inp{k_{i}}}_{i=1}^{n}\mapsto\pr{\Vee_{i=1}^{n}\inp{k_{i}}\to\Vee_{i=1}^{n}\inp 1=\inp n}.
\]
Applying Proposition \ref{prop:inverting_weak_equiv_of_pairs} to
the commutative diagram (\ref{d1}) and the section $\id_{\cal C}\times\phi$,
we can find a functor $\Phi:\pr{\cal M_{\act}^{\t}}^{n}\to\Env\pr{\cal M}_{\inp n}^{\t}$
with the following properties:
\begin{itemize}
\item The diagram % https://q.uiver.app/?q=WzAsNCxbMCwxLCJcXG1hdGhjYWx7Q31cXHRpbWVzIChOKFxcbWF0aHNme0Zpbn1fXFxhc3QpX3tcXG1hdGhybXthY3R9fSlebiJdLFsyLDEsIlxcbWF0aGNhbHtDfVxcdGltZXMgXFxvcGVyYXRvcm5hbWV7RW52fShOKFxcbWF0aHNme0Zpbn1fXFxhc3QpKV5cXG90aW1lcyBfe1xcbGFuZ2xlIG4gXFxyYW5nbGV9Il0sWzIsMCwiXFxvcGVyYXRvcm5hbWV7RW52fShcXG1hdGhjYWx7TX0pXlxcb3RpbWVzIF97XFxsYW5nbGUgbiBcXHJhbmdsZX0iXSxbMCwwLCIoXFxtYXRoY2Fse019Xlxcb3RpbWVzIF97XFxtYXRocm17YWN0fX0pXm5cXHRpbWVzIF97XFxtYXRoY2Fse0N9Xm59XFxtYXRoY2Fse0N9Il0sWzMsMF0sWzAsMSwiXFxvcGVyYXRvcm5hbWV7aWR9X3tcXG1hdGhjYWx7Q319XFx0aW1lc1xccGhpIiwyXSxbMiwxXSxbMywyLCJcXFBoaSJdXQ==
\[\begin{tikzcd}
	{(\mathcal{M}^\otimes _{\mathrm{act}})^n\times _{\mathcal{C}^n}\mathcal{C}} && {\operatorname{Env}(\mathcal{M})^\otimes _{\langle n \rangle}} \\
	{\mathcal{C}\times (N(\mathsf{Fin}_\ast)_{\mathrm{act}})^n} && {\mathcal{C}\times \operatorname{Env}(N(\mathsf{Fin}_\ast))^\otimes _{\langle n \rangle}}
	\arrow[from=1-1, to=2-1]
	\arrow["{\operatorname{id}_{\mathcal{C}}\times\phi}"', from=2-1, to=2-3]
	\arrow[from=1-3, to=2-3]
	\arrow["\Phi", from=1-1, to=1-3]
\end{tikzcd}\]is commutative. 
\item The composite $\pr{\rho_{!}^{i}}_{1\leq i\leq n}\circ\Phi$ is naturally
equivalent over $\cal C\times\pr{N\pr{\Fin_{\ast}}_{\act}}^{n}$ to
the identity functor. 
\end{itemize}
We will construct the functor $\bigoplus_{1\leq i\leq n}$ as the
composite
\[
\pr{\cal M_{\act}^{\t}}^{n}\times_{\cal C^{n}}\cal C\xrightarrow{\Phi}\Env\pr{\cal M}_{\inp n}^{\t}\xrightarrow{\text{forget}}\cal M^{\t}.
\]

Next, we show that, with our construction of the functor $\bigoplus_{1\leq i\leq n}$,
there is an inert natural transformation $\widetilde{h}_{i}:\pr{\pr{\cal M_{\act}^{\t}}^{n}\times_{\cal C^{n}}\cal C}\times\Delta^{1}\to\cal M^{\t}$
rendering the diagram in the statement commutative. Since $\opn{pr}_{i}$
is naturally equivalent over $\cal C\times N\pr{\Fin_{\ast}}$ to
the composite $\rho_{!}^{i}\Phi$, it suffices (by \cite[Proposition A.2.3.1]{HTT})
to find an inert natural transformation rendering the diagram % https://q.uiver.app/?q=WzAsNSxbMCwwLCIoKFxcbWF0aGNhbHtNfV5cXG90aW1lc197XFxtYXRocm17YWN0fX0pXm5cXHRpbWVzIF97XFxtYXRoY2Fse0N9Xm59XFxtYXRoY2Fse0N9KVxcdGltZXMgXFxwYXJ0aWFsXFxEZWx0YV4xIl0sWzAsMSwiKChcXG1hdGhjYWx7TX1eXFxvdGltZXNfe1xcbWF0aHJte2FjdH19KV5uXFx0aW1lcyBfe1xcbWF0aGNhbHtDfV5ufVxcbWF0aGNhbHtDfSlcXHRpbWVzIFxcRGVsdGFeMSJdLFsxLDEsIlxcbWF0aGNhbHtDfVxcdGltZXMgKE4oXFxtYXRoc2Z7RmlufV9cXGFzdClfe1xcbWF0aHJte2FjdH19KV5uXFx0aW1lcyBcXERlbHRhXjEiXSxbMiwxLCJcXG1hdGhjYWx7Q31cXHRpbWVzIE4oXFxtYXRoc2Z7RmlufV9cXGFzdCkiXSxbMiwwLCJcXG1hdGhjYWx7TX1eXFxvdGltZXMgIl0sWzAsMV0sWzEsMl0sWzIsMywiXFxvcGVyYXRvcm5hbWV7aWR9X3tcXG1hdGhjYWx7Q319XFx0aW1lcyBoX2kiLDJdLFs0LDMsInAiXSxbMCw0LCIoXFxiaWdvcGx1c197aT0xfV5uLFxccmhvXmlfIVxcUGhpICkiXSxbMSw0LCIiLDIseyJzdHlsZSI6eyJib2R5Ijp7Im5hbWUiOiJkYXNoZWQifX19XV0=
\[\begin{tikzcd}
	{((\mathcal{M}^\otimes_{\mathrm{act}})^n\times _{\mathcal{C}^n}\mathcal{C})\times \partial\Delta^1} && {\mathcal{M}^\otimes } \\
	{((\mathcal{M}^\otimes_{\mathrm{act}})^n\times _{\mathcal{C}^n}\mathcal{C})\times \Delta^1} & {\mathcal{C}\times (N(\mathsf{Fin}_\ast)_{\mathrm{act}})^n\times \Delta^1} & {\mathcal{C}\times N(\mathsf{Fin}_\ast)}
	\arrow[from=1-1, to=2-1]
	\arrow[from=2-1, to=2-2]
	\arrow["{\operatorname{id}_{\mathcal{C}}\times h_i}"', from=2-2, to=2-3]
	\arrow["p", from=1-3, to=2-3]
	\arrow["{(\bigoplus_{i=1}^n,\rho^i_!\Phi )}", from=1-1, to=1-3]
	\arrow[dashed, from=2-1, to=1-3]
\end{tikzcd}\]commutative. The outer rectangle is equal to that of the diagram % https://q.uiver.app/?q=WzAsNyxbMCwwLCIoKFxcbWF0aGNhbHtNfV5cXG90aW1lc197XFxtYXRocm17YWN0fX0pXm5cXHRpbWVzIF97XFxtYXRoY2Fse0N9Xm59XFxtYXRoY2Fse0N9KVxcdGltZXMgXFxwYXJ0aWFsXFxEZWx0YV4xIl0sWzAsMSwiKChcXG1hdGhjYWx7TX1eXFxvdGltZXNfe1xcbWF0aHJte2FjdH19KV5uXFx0aW1lc197XFxtYXRoY2Fse0N9Xm59XFxtYXRoY2Fse0N9KVxcdGltZXMgIFxcRGVsdGFeMSJdLFsyLDEsIlxcbWF0aGNhbHtDfVxcdGltZXMgXFxvcGVyYXRvcm5hbWV7RW52fShOKFxcbWF0aHNme0Zpbn1fXFxhc3QpKV5cXG90aW1lcyBfe1xcbGFuZ2xlIG5cXHJhbmdsZX1cXHRpbWVzIFxcRGVsdGFeMSJdLFszLDEsIlxcbWF0aGNhbHtDfVxcdGltZXMgTihcXG1hdGhzZntGaW59X1xcYXN0KSJdLFszLDAsIlxcbWF0aGNhbHtNfV5cXG90aW1lcyAiXSxbMSwxLCJcXG1hdGhjYWx7Q31cXHRpbWVzIFxcb3BlcmF0b3JuYW1le0Vudn0oXFxtYXRoY2Fse019KV5cXG90aW1lcyBfe1xcbGFuZ2xlIG5cXHJhbmdsZX1cXHRpbWVzIFxcRGVsdGFeMSJdLFsxLDAsIlxcbWF0aGNhbHtDfVxcdGltZXMgXFxvcGVyYXRvcm5hbWV7RW52fShcXG1hdGhjYWx7TX0pXlxcb3RpbWVzIF97XFxsYW5nbGUgbiBcXHJhbmdsZX1cXHRpbWVzIFxccGFydGlhbFxcRGVsdGFeMSJdLFswLDFdLFsyLDMsIlxcb3BlcmF0b3JuYW1le2lkfV97XFxtYXRoY2Fse0N9fVxcdGltZXMgZ19pIiwyXSxbNSwyXSxbMSw1LCJcXFBoaVxcdGltZXMgXFxvcGVyYXRvcm5hbWV7aWR9IiwyXSxbMCw2LCJcXFBoaVxcdGltZXMgXFxvcGVyYXRvcm5hbWV7aWR9Il0sWzYsNCwiKFxcdGV4dHtmb3JnZXR9LCBcXHJob15pXyEpIl0sWzYsNV0sWzQsM11d
\[\begin{tikzcd}[column sep=small, scale cd=.8]
	{((\mathcal{M}^\otimes_{\mathrm{act}})^n\times _{\mathcal{C}^n}\mathcal{C})\times \partial\Delta^1} & {\mathcal{C}\times \operatorname{Env}(\mathcal{M})^\otimes _{\langle n \rangle}\times \partial\Delta^1} && {\mathcal{M}^\otimes } \\
	{((\mathcal{M}^\otimes_{\mathrm{act}})^n\times_{\mathcal{C}^n}\mathcal{C})\times  \Delta^1} & {\mathcal{C}\times \operatorname{Env}(\mathcal{M})^\otimes _{\langle n\rangle}\times \Delta^1} & {\mathcal{C}\times \operatorname{Env}(N(\mathsf{Fin}_\ast))^\otimes _{\langle n\rangle}\times \Delta^1} & {\mathcal{C}\times N(\mathsf{Fin}_\ast)}
	\arrow[from=1-1, to=2-1]
	\arrow["{\operatorname{id}_{\mathcal{C}}\times g_i}"', from=2-3, to=2-4]
	\arrow[from=2-2, to=2-3]
	\arrow["{\Phi\times \operatorname{id}}"', from=2-1, to=2-2]
	\arrow["{\Phi\times \operatorname{id}}", from=1-1, to=1-2]
	\arrow["{(\text{forget}, \rho^i_!)}", from=1-2, to=1-4]
	\arrow[from=1-2, to=2-2]
	\arrow[from=1-4, to=2-4]
\end{tikzcd}\]so the existence of the desired filler follows from the definition
of $\rho_{!}^{i}$.
\end{proof}
%
\begin{cor}
[Universal Property of the Direct Sum Functor]\label{cor:oplus_p-limit}Let
$\cal C$ be an $\infty$-category and let $p:\cal M^{\t}\to\cal C\times N\pr{\Fin_{\ast}}$
be a $\cal C$-family of $\infty$-operads. Let $q:K\to\cal C$ be
an object of $\SS/\cal C$, let $n\geq1$, and let $f,g_{1},\dots,g_{n}:K\to\cal M_{\act}^{\t}$
be diagrams. Assume the following:
\begin{enumerate}
\item For each $1\leq i\leq n$, there is an inert natural transformation
$\alpha_{i}:f\to g_{i}$ over $\cal C$.
\item For each vertex $v$ in $K$, the maps $\alpha_{i}$ give rise to
a $p$-limit cone $\{f\pr v\to g_{i}\pr v\}_{1\leq i\le n}$. 
\end{enumerate}
Then $f$ is naturally equivalent over $\cal C$ to the composite
\[
G:K\xrightarrow{\pr{g_{i}}_{i=1}^{n}}\pr{\cal M_{\act}^{\t}}^{n}\times_{\cal C^{n}}\cal C\xrightarrow{\bigoplus_{1\leq i\leq n}}\cal M_{\act}^{\t}.
\]
\end{cor}
\begin{proof}
For each vertex $v$ in $K$, write $p_{2}\pr{g_{i}\pr v}=\inp{m_{i}\pr v}$
and $p_{2}\pr{f\pr v}=\inp{m\pr v}$. Let $\eta:K\times\Delta^{1}\to N\pr{\Fin_{\ast}}$
denote the natural equivalence which satisfies the following conditions: 
\begin{itemize}
\item The restriction $\eta\vert K\times\{0\}$ is given by $K\xrightarrow{\pr{p_{2}g_{i}}_{i}}N\pr{\Fin_{\ast}}_{\act}^{n}\xrightarrow{\Vee_{i=1}^{n}}N\pr{\Fin_{\ast}}$.
\item The restriction $\eta\vert K\times\{1\}$ is given by $p_{2}f$.
\item For each vertex $v\in K$ and $1\leq i\leq n$, the composite
\[
\inp{m\pr v}\xrightarrow{\eta}\inp{m\pr v}\xrightarrow{p_{2}\alpha_{i}}\inp{m_{i}\pr v}
\]
is defined as follows: Given $k\in\inp{m\pr v}^{\circ}$, the integer
$\sum_{j<i}m_{j}\pr v+k\in\inp{m\pr v}$ is mapped to $k$. The remaining
elements are mapped to the base point of $\inp{m_{i}\pr v}$.
\end{itemize}
Using the fact that $p$ is a categorical equivalence, we can find
a functor $f':K\to\cal M^{\t}$ and a natural equivalence $f'\xrightarrow{\simeq}f$
which lifts the natural equivalence $\pr{q\circ\opn{pr}_{1},\eta}:K\times\Delta^{1}\to\cal C\times N\pr{\Fin_{\ast}}$.
Replacing $f$ by $f'$ if necessary, we may assume that the diagram
% https://q.uiver.app/?q=WzAsNCxbMCwwLCJLXFx0aW1lcyBcXERlbHRhXjEiXSxbMSwwLCJcXG1hdGhjYWx7TX1eXFxvdGltZXMgIl0sWzEsMSwiXFxtYXRoY2Fse0N9XFx0aW1lcyBOKFxcbWF0aHNme0Zpbn1fXFxhc3QpIl0sWzAsMSwiKFxcbWF0aGNhbHtDfVxcdGltZXMgKE4oXFxtYXRoc2Z7RmlufV9cXGFzdClfe1xcbWF0aHJte2FjdH19KV5uKVxcdGltZXMgXFxEZWx0YV4xIl0sWzAsMSwiXFxhbHBoYV9pIl0sWzEsMiwicCJdLFszLDIsIlxcb3BlcmF0b3JuYW1le2lkfV97XFxtYXRoY2Fse0N9fVxcdGltZXMgaF9pIiwyXSxbMCwzLCIocSwocF8yZ19qKV9qKVxcdGltZXMgXFxvcGVyYXRvcm5hbWV7aWR9IiwyXV0=
\[\begin{tikzcd}
	{K\times \Delta^1} & {\mathcal{M}^\otimes } \\
	{(\mathcal{C}\times (N(\mathsf{Fin}_\ast)_{\mathrm{act}})^n)\times \Delta^1} & {\mathcal{C}\times N(\mathsf{Fin}_\ast)}
	\arrow["{\alpha_i}", from=1-1, to=1-2]
	\arrow["p", from=1-2, to=2-2]
	\arrow["{\operatorname{id}_{\mathcal{C}}\times h_i}"', from=2-1, to=2-2]
	\arrow["{(q,(p_2g_j)_j)\times \operatorname{id}}"', from=1-1, to=2-1]
\end{tikzcd}\]commutes for each $1\leq i\leq n$. Since relative limits in functor
categories can be formed objectwise \cite[\href{https://kerodon.net/tag/02XK}{Tag 02XK}]{kerodon},
the morphisms $\alpha_{i}:f\to g_{i}$ in $\Fun\pr{K,\cal M^{\t}}$
form a $\Fun\pr{K,p}$-limit cone. Moreover, since relative limits
are stable under pullbacks \cite[Proposition 4.3.1.5]{HTT}, we deduce
that the morphisms $\{\alpha_{i}\}_{1\leq i\leq n}$ of $\Fun_{\cal C}\pr{K,\cal M^{\t}}$
form a $\Fun_{\cal C}\pr{K,p}$-limit cone. 

Now using Proposition \ref{prop:oplus_p-limit}, we can construct
the functor $\bigoplus_{1\leq i\leq n}:\pr{\cal M_{\act}^{\t}}^{n}\times_{\cal C^{n}}\cal C\to\cal M^{\t}$
so that there is an inert natural transformation $\widetilde{h}_{i}:\bigoplus_{1\leq i\leq n}\to\opn{pr}_{i}$
which lifts the composite $\pr{\cal M_{\act}^{\t}}^{n}\times_{\cal C^{n}}\cal C\to\cal C\times N\pr{\Fin_{\ast}}_{\act}^{n}\times\Delta^{1}\xrightarrow{\id_{\cal C}\times h_{i}}\cal C\times N\pr{\Fin_{\ast}}$.
Again, the induced natural transformations $\{G\to g_{i}\}_{1\leq i\leq n}$
form a $\Fun_{\cal C}\pr{K,p}$-limit cone covering the same cone
as $\{\alpha_{i}\}$. Thus we obtain the desired equivalence $G\xrightarrow{\simeq}f$
in $\Fun_{\cal C}\pr{K,\cal M^{\t}}$.
\end{proof}
The above corollary admits the following further corollary, which
we shall use in the next section.
\begin{cor}
\label{cor:3.1.1.8}Let $q:\cal C^{\t}\to\cal O^{\t}$ be a fibration
of $\infty$-operads. Let $K_{1},\dots,K_{n}$ be simplicial sets,
and let $K=\prod_{1\leq i\leq n}K_{i}$. Let $\overline{f}:K^{\rcone}\to\cal C^{\t}$
and $\{\overline{f_{i}}:K^{\rcone}\to\cal C^{\t}\}_{1\leq i\leq n}$
be active diagrams, and suppose there are inert natural transformations
$\{\overline{f}\to\overline{f}_{i}\circ\pr{\opn{pr}_{i}}^{\rcone}\}_{1\le i\leq n}$
such that for each vertex $v$ in $K^{\rcone}$, the induced map $\{\overline{f}\pr v\to\overline{f}_{i}\circ\pr{\opn{pr}_{i}}^{\rcone}\pr v\}_{1\le i\leq n}$
form a $q$-limit cone. If each $\overline{f_{i}}$ is an operadic
$q$-colimit diagram, so is $\overline{f}$.
\end{cor}
\begin{proof}
According to Corollary \ref{cor:oplus_p-limit}, the diagram $\overline{f}$
is naturally equivalent to the composite
\[
K^{\rcone}\to\prod_{1\leq i\leq n}\pr{K_{i}^{\rcone}}\xrightarrow{\prod_{1\leq i\leq n}\overline{f_{i}}}\pr{\cal C_{\act}^{\t}}^{n}\xrightarrow{\bigoplus_{1\leq i\leq n}}\cal C^{\t}.
\]
The claim is thus a consequence of \cite[Proposition 3.1.1.8]{HA}.
\end{proof}

\subsection{Direct Sum and Slice}

Let $\cal O^{\t}$ be an $\infty$-operad and let $Y\in\cal O^{\t}$
be an object. If $Y$ lies over an object $\inp n\in N\pr{\Fin_{\ast}}$
with $n\geq2$, then we can find objects $Y_{i}\in\cal O$ and an
equivalence $Y\simeq\bigoplus_{i=1}^{n}Y_{i}$. Given objects $X_{1},\dots,X_{n}\in\cal O^{\t}$
and active maps $f_{i}:X_{i}\to Y_{i}$, their direct sum $\bigoplus_{i=1}^{n}f_{i}:\bigoplus_{i=1}^{n}X_{i}\to\bigoplus_{i=1}^{n}Y_{i}\simeq Y$
is an active morphism with codomain $Y$. Conversely, given an active
morphism $f:X\to Y$ in $\cal O^{\t}$, we can write $f\simeq\bigoplus_{i=1}^{n}f_{i}$,
where $f_{i}$ is the active map obtained by factoring the composite
$X\to Y\to Y_{i}$ into an inert map followed by an active map. The
following proposition, which is the only result of this subsection,
asserts that this ``direct sum decomposition'' of morphisms is an
equivalence on the level of $\infty$-categories: 
\begin{prop}
\label{prop:directsum_slice_equiv}Let $\cal C$ be an $\infty$-category,
let $C\in\cal C$ be an object, and let $\cal M^{\t}\to\cal C\times N\pr{\Fin_{\ast}}$
be a $\cal C$-family of $\infty$-operads. For any integer $n\geq1$
and any objects $M_{1},\dots,M_{n}\in\cal M\times_{\cal C}\{C\}$,
the direct sum functor induces an equivalence of $\infty$-categories
\[
\pr{\pr{\cal M_{\act}^{\t}}^{n}\times_{\cal C^{n}}\cal C}_{/\pr{M_{1},\dots,M_{n}}}\xrightarrow{\simeq}\pr{\cal M_{\act}^{\t}}_{/\bigoplus_{i=1}^{n}M_{i}}.
\]
\end{prop}
\begin{proof}
Recall that the direct sum functor $\bigoplus_{i=1}^{n}$ is obtained
as the composite 
\[
\pr{\cal M_{\act}^{\t}}^{n}\times_{\cal C^{n}}\cal C\xrightarrow[\Phi]{\simeq}\Env\pr{\cal M}_{\inp n}^{\t}\xrightarrow{\text{forget}}\cal M_{\act}^{\t},
\]
where the map $\Phi$ is the inverse equivalence over $\cal C$ of
the functor $\pr{\rho_{!}^{i}}_{1\leq i\le n}:\Env\pr{\cal M}_{\inp n}^{\t}\xrightarrow{\simeq}\pr{\cal M_{\act}^{\t}}^{n}\times_{\cal C^{n}}\cal C$
. The functor $\Phi$ maps the object $\pr{M_{1},\dots,M_{n}}$ to
the object $\pr{\bigoplus_{i=1}^{n}M_{i},\alpha:\inp n\to\inp n}$,
where $\alpha$ is a bijection. It will therefore suffice to prove
that the functor 
\[
\theta:\pr{\Env\pr{\cal M}_{\inp n}^{\t}}_{/\pr{\bigoplus_{i=1}^{n}M_{i},\alpha}}\to\pr{\cal M_{\act}^{\t}}_{/\bigoplus_{i=1}^{n}M_{i}}
\]
is an equivalence of $\infty$-categories. Set $M=\bigoplus_{i=1}^{n}M_{i}$.
By definition, we have
\[
\pr{\Env\pr{\cal M}_{\inp n}^{\t}}_{/\pr{M,\alpha}}=\pr{\Act\pr{N\pr{\Fin_{\ast}}}_{\inp n}^{\t}}_{/\alpha}\times_{N\pr{\Fin_{\ast}}_{/\inp n}}\cal M_{/M}^{\t}.
\]
Since the evaluation at $0\in\Delta^{1}$ induces an isomorphism of
simplicial sets
\[
\pr{\Act\pr{N\pr{\Fin_{\ast}}}_{\inp n}^{\t}}_{/\alpha}\xrightarrow{\cong}\pr{N\pr{\Fin_{\ast}}_{\act}}_{/\inp n},
\]
we deduce that the functor $\theta$ is an \textit{isomorphism} of
simplicial sets.
\end{proof}

\section{\label{sec:FTOK}The Fundamental Theorem of Operadic Kan Extensions}

In this final section, we will provide complete details of the proof
of \cite[Theorem 3.1.2.3]{HA}, using tools and results in the preceding
sections.

\subsection{Recollection}

In this subsection, we recall the definition of operadic Kan extensions
and the statement of the fundamental theorem of operadic Kan extensions. 
\begin{defn}
\cite[Definition 3.1.2.2]{HA}Let $\cal M^{\t}\to N\pr{\Fin_{\ast}}\times\Delta^{1}$
be a $\Delta^{1}$-family of $\infty$-operads and let $q:\cal C^{\t}\to\cal O^{\t}$
be a fibration of $\infty$-operads. Set $\cal A^{\t}=\cal M^{\t}\times_{\Delta^{1}}\{0\}$,
$\cal B^{\t}=\cal M^{\t}\times_{\Delta^{1}}\{1\}$. A map $F:\cal M^{\t}\to\cal C^{\t}$
is called an \textbf{operadic $q$-left Kan extension of $F\vert\cal A^{\t}$
}if the following condition is satisfied for every object $B\in\cal B$:
\begin{itemize}
\item [($\ast$)]The composite map
\[
\pr{\pr{\cal M_{\act}^{\t}}_{/B}\times_{\Delta^{1}}\{0\}}^{\rcone}\to\pr{\cal M_{/B}^{\t}}^{\rcone}\to\cal M^{\t}\xrightarrow{\overline{F}}\cal C^{\t}
\]
is an operadic $q$-colimit diagram.
\end{itemize}
\end{defn}
Lurie imposes condition ($\ast$) to hold for every object $B\in\cal B^{\t}$
in \cite[Definition 3.1.2.2]{HA}. This is not a problem, because
of the following proposition, which seems to be implicitly used in
\cite{HA}.
\begin{prop}
\label{prop:operadic_Kan_equiv}Let $p:\cal M^{\t}\to N\pr{\Fin_{\ast}}\times\Delta^{1}$
be a correspondence from an $\infty$-operad $\cal A^{\t}$ to another
$\infty$-operad $\cal B^{\t}$. Let $q:\cal C^{\t}\to\cal O^{\t}$
be a fibration of $\infty$-operads and let $F:\cal M^{\t}\to\cal C^{\t}$
be a map of generalized $\infty$-operads. Suppose that $F$ is an
operadic $q$-left Kan extension of \textbf{$F\vert\cal A^{\t}$}.
Then for every object $B\in\cal B^{\t},$the map 
\[
\theta:\pr{\pr{\cal M_{\act}^{\t}}_{/B}\times_{\Delta^{1}}\{0\}}^{\rcone}\to\pr{\cal M_{/B}^{\t}}^{\rcone}\to\cal M^{\t}\xrightarrow{F}\cal C^{\t}
\]
is an operadic $q$-colimit diagram.
\end{prop}
\begin{proof}
Let $\inp n$ be the image of $B$ in $N\pr{\Fin_{\ast}}$. We consider
two cases, depending on whether or not $n$ is equal to $0$.

Suppose first that $n=0$. Then we have $\pr{\cal M_{\act}^{\t}}_{/B}=\pr{\cal M_{\inp 0}^{\t}}_{/B}$,
since there is no active map $\inp k\to\inp 0$ with $k\geq1$. Since
every object of $\cal M_{\inp 0}^{\t}$ is $p$-terminal, the functor
$\cal M_{\inp 0}^{\t}\to\Delta^{1}$ is a trivial fibration. It follows
that the functor $\pr{\cal M_{\act}^{\t}}_{/B}\times_{\Delta^{1}}\{0\}\to\Delta_{/1}^{1}\times_{\Delta^{1}}\{0\}\cong\Delta^{0}$
is also a trivial fibration. It will therefore suffice to show that
$F$ carries a morphism of the form $A\to B$ in $\cal M^{\t}$ with
$A\in\cal M_{\inp 0}^{\t}\times_{\Delta^{1}}\{0\}$ to an equivalence
in $\cal C^{\t}$. This is clear, since $\cal C_{\inp 0}^{\t}$ is
a contractible Kan complex.

If $n\geq1$, choose for each $1\leq i\leq n$ an inert map $B\to B_{i}$
in $\cal B^{\t}$ over $\rho^{i}:\inp n\to\inp 1$, and choose also
a direct sum functor $\bigoplus_{i=1}^{n}$ for $\cal M^{\t}$ and
$\cal C^{\t}$. Replacing $\bigoplus_{i=1}^{n}$ by a functor naturally
equivalent one, we may assume that $B=\bigoplus_{i=1}^{n}$. According
to Proposition \ref{prop:directsum_slice_equiv}, the direct sum functor
induces an equivalence of $\infty$-categories
\[
\phi:\pr{\pr{\cal M_{\act}^{\t}}^{n}\times_{\pr{\Delta^{1}}^{n}}\Delta^{1}}_{/\pr{B_{1},\dots,B_{n}}}\xrightarrow{\simeq}\pr{\cal M_{\act}^{\t}}_{/B}.
\]
There is also an isomorphism of simplicial sets
\[
\psi:\prod_{i=1}^{n}\pr{\cal M_{\act}^{\t}}_{/B_{i}}\times_{\Delta^{1}}\{0\}\cong\pr{\pr{\cal M_{\act}^{\t}}^{n}\times_{\pr{\Delta^{1}}^{n}}\Delta^{1}}_{/\pr{B_{1},\dots,B_{n}}}\times_{\Delta^{1}}\{0\}.
\]
It will therefore suffice to show that the composite $\theta\circ\phi^{\rcone}\circ\psi^{\rcone}$
is an operadic $q$-colimit diagram. For this, we consider the diagram
% https://q.uiver.app/?q=WzAsOSxbMCwwLCIoXFxwcm9kX3tpPTF9Xm4oXFxtYXRoY2Fse019Xlxcb3RpbWVzIF97XFxtYXRocm17YWN0fX0pX3svQl9pfVxcdGltZXMgX3tcXERlbHRhXjF9XFx7MFxcfSleXFx0cmlhbmdsZXJpZ2h0Il0sWzEsMCwiKCgoXFxtYXRoY2Fse019Xlxcb3RpbWVzIF97XFxtYXRocm17YWN0fX0pXm5cXHRpbWVzX3soXFxEZWx0YV4xKV5ufVxcRGVsdGFeMSlfey8oQl8xLFxcZG90cyAsQl9uKX1cXHRpbWVzX3tcXERlbHRhXjF9XFx7MFxcfSleXFx0cmlhbmdsZXJpZ2h0Il0sWzAsMSwiXFxwcm9kX3tpPTF9Xm4oKFxcbWF0aGNhbHtNfV5cXG90aW1lcyBfe1xcbWF0aHJte2FjdH19KV97L0JfaX1cXHRpbWVzIF97XFxEZWx0YV4xfVxcezBcXH0pXlxcdHJpYW5nbGVyaWdodCJdLFsxLDEsIihcXG1hdGhjYWx7TX1eXFxvdGltZXMgX3tcXG1hdGhybXthY3R9fSleblxcdGltZXNfeyhcXERlbHRhXjEpXm59XFxEZWx0YV4xIl0sWzIsMCwiKChcXG1hdGhjYWx7TX1eXFxvdGltZXMgX3tcXG1hdGhybXthY3R9fSlfey9CfVxcdGltZXMgX3tcXERlbHRhXjF9XFx7MFxcfSleXFx0cmlhbmdsZXJpZ2h0Il0sWzIsMSwiXFxtYXRoY2Fse019Xlxcb3RpbWVzIF97XFxtYXRocm17YWN0fX0iXSxbMCwyLCIoXFxtYXRoY2Fse019Xlxcb3RpbWVzIF97XFxtYXRocm17YWN0fX0pXm4iXSxbMSwyLCIoXFxtYXRoY2Fse0N9Xlxcb3RpbWVzIF97XFxtYXRocm17YWN0fX0pXm4iXSxbMiwyLCJcXG1hdGhjYWx7Q31eXFxvdGltZXMgX3tcXG1hdGhybXthY3R9fSJdLFswLDJdLFsxLDNdLFswLDEsIlxcY29uZyJdLFsxLDQsIlxcc2ltZXEiXSxbMCwxLCJcXHBzaV5cXHRyaWFuZ2xlcmlnaHQiLDJdLFsxLDQsIlxccGhpXlxcdHJpYW5nbGVyaWdodCIsMl0sWzQsNV0sWzMsNSwiXFxiaWdvcGx1c197aT0xfV5uIiwyXSxbMyw2LCIiLDIseyJzdHlsZSI6eyJ0YWlsIjp7Im5hbWUiOiJob29rIiwic2lkZSI6ImJvdHRvbSJ9fX1dLFsyLDZdLFs2LDcsIlxccHJvZF97aT0xfV5uRiIsMl0sWzMsNywiRyJdLFs1LDgsIkYiXSxbNyw4LCJcXGJpZ29wbHVzX3tpPTF9Xm4iLDJdXQ==
\[\begin{tikzcd}[column sep=small, scale cd=.8]
	{(\prod_{i=1}^n(\mathcal{M}^\otimes _{\mathrm{act}})_{/B_i}\times _{\Delta^1}\{0\})^\triangleright} & {(((\mathcal{M}^\otimes _{\mathrm{act}})^n\times_{(\Delta^1)^n}\Delta^1)_{/(B_1,\dots ,B_n)}\times_{\Delta^1}\{0\})^\triangleright} & {((\mathcal{M}^\otimes _{\mathrm{act}})_{/B}\times _{\Delta^1}\{0\})^\triangleright} \\
	{\prod_{i=1}^n((\mathcal{M}^\otimes _{\mathrm{act}})_{/B_i}\times _{\Delta^1}\{0\})^\triangleright} & {(\mathcal{M}^\otimes _{\mathrm{act}})^n\times_{(\Delta^1)^n}\Delta^1} & {\mathcal{M}^\otimes _{\mathrm{act}}} \\
	{(\mathcal{M}^\otimes _{\mathrm{act}})^n} & {(\mathcal{C}^\otimes _{\mathrm{act}})^n} & {\mathcal{C}^\otimes _{\mathrm{act}}.}
	\arrow[from=1-1, to=2-1]
	\arrow[from=1-2, to=2-2]
	\arrow["\cong", from=1-1, to=1-2]
	\arrow["\simeq", from=1-2, to=1-3]
	\arrow["{\psi^\triangleright}"', from=1-1, to=1-2]
	\arrow["{\phi^\triangleright}"', from=1-2, to=1-3]
	\arrow[from=1-3, to=2-3]
	\arrow["{\bigoplus_{i=1}^n}"', from=2-2, to=2-3]
	\arrow[hook', from=2-2, to=3-1]
	\arrow[from=2-1, to=3-1]
	\arrow["{\prod_{i=1}^nF}"', from=3-1, to=3-2]
	\arrow["G", from=2-2, to=3-2]
	\arrow["F", from=2-3, to=3-3]
	\arrow["{\bigoplus_{i=1}^n}"', from=3-2, to=3-3]
\end{tikzcd}\]Here the map $G$ is the restriction of the map $\prod_{i=1}^{n}F$.
The left column and the top right square are commutative. The bottom
right square commutes up to natural equivalence by Proposition \ref{prop:oplus_p-limit}
and Corollary \ref{cor:oplus_p-limit}. Hence the diagram $\theta\circ\phi^{\rcone}\circ\psi^{\rcone}$
is naturally equivalent to the composite
\[
\theta':\pr{\prod_{i=1}^{n}\pr{\cal M_{\act}^{\t}}_{/B_{i}}\times_{\Delta^{1}}\{0\}}^{\rcone}\to\prod_{i=1}^{n}\pr{\pr{\cal M_{\act}^{\t}}_{/B_{i}}\times_{\Delta^{1}}\{0\}}^{\rcone}\to\pr{\cal C_{\act}^{\t}}^{n}\xrightarrow{\bigoplus_{i=1}^{n}}\cal C_{\act}^{\t}.
\]
The diagram $\theta'$ is an operadic $q$-colimit diagram by \cite[Proposition 3.1.1.8]{HA}.
\end{proof}
We can now state the fundamental theorem of opeardic Kan extensions.
\begin{thm}
\cite[Theorem 3.1.2.3]{HA}\label{thm:3.1.2.3}Let $p:\cal M^{\t}\to N\pr{\Fin_{\ast}}\times\Delta^{n}$
be a $\Delta^{n}$-family of generalized $\infty$-operads, and let
$q:\cal C^{\t}\to\cal O^{\t}$ be a fibration of $\infty$-operads.
Consider a commutative diagram% https://q.uiver.app/?q=WzAsNCxbMCwwLCJcXG1hdGhjYWx7TX1eXFxvdGltZXMgXFx0aW1lcyBfe1xcRGVsdGFebn1cXExhbWJkYSBebl8wIl0sWzAsMSwiXFxtYXRoY2Fse019Xlxcb3RpbWVzICJdLFsxLDAsIlxcbWF0aGNhbHtDfV5cXG90aW1lcyAiXSxbMSwxLCJcXG1hdGhjYWx7T31eXFxvdGltZXMgLiJdLFswLDEsIiIsMCx7InN0eWxlIjp7InRhaWwiOnsibmFtZSI6Imhvb2siLCJzaWRlIjoidG9wIn19fV0sWzAsMiwiZl8wIl0sWzIsMywicSJdLFsxLDMsImciLDJdLFsxLDIsIiIsMCx7InN0eWxlIjp7ImJvZHkiOnsibmFtZSI6ImRhc2hlZCJ9fX1dXQ==
\[\begin{tikzcd}
	{\mathcal{M}^\otimes \times _{\Delta^n}\Lambda ^n_0} & {\mathcal{C}^\otimes } \\
	{\mathcal{M}^\otimes } & {\mathcal{O}^\otimes}
	\arrow[hook, from=1-1, to=2-1]
	\arrow["{f_0}", from=1-1, to=1-2]
	\arrow["q", from=1-2, to=2-2]
	\arrow["g"', from=2-1, to=2-2]
	\arrow[dashed, from=2-1, to=1-2]
\end{tikzcd}\]of simplicial sets. Suppose that for each vertex $i$ in $\Lambda_{0}^{n}$
and each vertex $j$ in $\Delta^{n}$, the induced maps $\cal M^{\t}\times_{\Delta^{n}}\{v\}\to\cal C^{\t}$
and $\cal M^{\t}\times_{\Delta^{n}}\{j\}\to\cal O^{\t}$ are morphisms
of $\infty$-operads.
\begin{itemize}
\item [(A)]If $n=1$, the following conditions are equivalent:
\begin{itemize}
\item [(a)]There is a dashed filler which is an operadic $q$-left Kan
extension of $f_{0}$.
\item [(b)]For each object $B\in\cal M^{\t}\times_{\Delta^{1}}\{1\}$,
the diagram
\[
\{0\}\times_{\Delta^{1}}\pr{\pr{\cal M_{\act}^{\t}}_{/B}}\to\{0\}\times_{\Delta^{1}}\cal M^{\t}\xrightarrow{f_{0}}\cal C^{\t}
\]
admits an operadic $q$-colimit cone which lifts the map
\[
\pr{\{0\}\times_{\Delta^{1}}\pr{\pr{\cal M_{\act}^{\t}}_{/B}}}^{\rcone}\to\pr{\cal M_{/B}^{\t}}^{\rcone}\to\cal M^{\t}\xrightarrow{g}\cal O^{\t}.
\]
\end{itemize}
\item [(B)]If $n>1$ and the restriction $f_{0}\vert\cal M^{\t}\times_{\Delta^{n}}\Delta^{\{0,1\}}$
is an operadic $q$-left Kan extension of $f_{0}\vert\cal M^{\t}\times_{\Delta^{n}}\{0\}$,
then there is a dashed arrow rendering the diagram commutative.
\end{itemize}
\end{thm}
The rest of this note is devoted to elaborating Lurie's proof of the
above theorem.

\subsection{Preliminary Results}

In this subsection, we collect some results which will be used in
the proof of Theorem \ref{prop:operadic_Kan_equiv}. We recommend
that the reader skip to the next subsection and refer to this subsection
as needed.
\begin{prop}
\label{prop:relative_terminal_obj_in_slice}Let $p:\cal C\to\cal D$
be an inner fibration of $\infty$-categories, $K$ a simplicial set,
and $\overline{f}:K^{\rcone}\to\cal C$ a diagram. Set $f=\overline{f}\vert K$
and let $q$ denote the functor $\cal C_{f/}\to\cal D_{pf/}$. If
$\overline{f}$ maps the cone point to a $p$-terminal object, then
$\overline{f}$ is $q$-terminal.
\end{prop}
\begin{proof}
We must show that the map 
\[
\pr{\cal C_{f/}}_{/\overline{f}}\to\cal C_{f/}\times_{\cal D_{pf/}}\pr{\cal D_{pf/}}_{/q\overline{f}}
\]
is a trivial fibration. Let $C=\overline{f}\pr{\infty}$. Given a
monomorphism $A\to B$ of simplicial sets, a lifting problem on the
left hand side corresponds under adjunction to a lifting problem on
the right hand side:% https://q.uiver.app/?q=WzAsOCxbMCwwLCJBIl0sWzAsMSwiQiJdLFsxLDAsIihcXG1hdGhjYWx7Q31fe2YvfSlfey9cXG92ZXJsaW5le2Z9fSJdLFsxLDEsIlxcbWF0aGNhbHtDfV97Zi99XFx0aW1lcyBfe1xcbWF0aGNhbHtEfV97cGYvfX0oXFxtYXRoY2Fse0R9X3twZi99KV97L3FcXG92ZXJsaW5le2Z9fSJdLFsyLDAsIktcXHN0YXIgQSJdLFsyLDEsIktcXHN0YXIgQiJdLFszLDAsIlxcbWF0aGNhbHtDfV97L0N9Il0sWzMsMSwiXFxtYXRoY2Fse0N9XFx0aW1lcyBfe1xcbWF0aGNhbHtEfX1cXG1hdGhjYWx7RH1fey9wQ30iXSxbMCwyXSxbMiwzXSxbMCwxXSxbMSwzXSxbMSwyLCIiLDEseyJzdHlsZSI6eyJib2R5Ijp7Im5hbWUiOiJkYXNoZWQifX19XSxbNCw1XSxbNCw2XSxbNiw3XSxbNSw3XSxbNSw2LCIiLDEseyJzdHlsZSI6eyJib2R5Ijp7Im5hbWUiOiJkYXNoZWQifX19XV0=
\[\begin{tikzcd}
	A & {(\mathcal{C}_{f/})_{/\overline{f}}} & {K\star A} & {\mathcal{C}_{/C}} \\
	B & {\mathcal{C}_{f/}\times _{\mathcal{D}_{pf/}}(\mathcal{D}_{pf/})_{/q\overline{f}}} & {K\star B} & {\mathcal{C}\times _{\mathcal{D}}\mathcal{D}_{/pC}.}
	\arrow[from=1-1, to=1-2]
	\arrow[from=1-2, to=2-2]
	\arrow[from=1-1, to=2-1]
	\arrow[from=2-1, to=2-2]
	\arrow[dashed, from=2-1, to=1-2]
	\arrow[from=1-3, to=2-3]
	\arrow[from=1-3, to=1-4]
	\arrow[from=1-4, to=2-4]
	\arrow[from=2-3, to=2-4]
	\arrow[dashed, from=2-3, to=1-4]
\end{tikzcd}\]The right hand lifting problem is solvable because $C$ is $p$-terminal.
\end{proof}
\begin{cor}
\label{cor:Step2}Let $q:\cal C^{\t}\to\cal O^{\t}$ be a fibration
of $\infty$-operads, let $K$ be a simplicial set, and let $n\geq0$.
Consider a lifting problem % https://q.uiver.app/?q=WzAsNCxbMCwwLCJLXFxzdGFyIFxccGFydGlhbFxcRGVsdGEgXm4iXSxbMCwxLCJLXFxzdGFyIFxcRGVsdGEgXm4iXSxbMSwwLCJcXG1hdGhjYWx7Q31eXFxvdGltZXMgIl0sWzEsMSwiXFxtYXRoY2Fse099Xlxcb3RpbWVzICJdLFswLDEsIiIsMCx7InN0eWxlIjp7InRhaWwiOnsibmFtZSI6Imhvb2siLCJzaWRlIjoidG9wIn19fV0sWzIsMywicSJdLFswLDIsImYiXSxbMSwzLCJnIiwyXSxbMSwyLCIiLDEseyJzdHlsZSI6eyJib2R5Ijp7Im5hbWUiOiJkYXNoZWQifX19XV0=
\[\begin{tikzcd}
	{K\star \partial\Delta ^n} & {\mathcal{C}^\otimes } \\
	{K\star \Delta ^n} & {\mathcal{O}^\otimes.}
	\arrow[hook, from=1-1, to=2-1]
	\arrow["q", from=1-2, to=2-2]
	\arrow["f", from=1-1, to=1-2]
	\arrow["g"', from=2-1, to=2-2]
	\arrow[dashed, from=2-1, to=1-2]
\end{tikzcd}\]If the map $g$ maps the terminal vertex of $\Delta^{n}$ to an object
in $\cal O_{\inp 0}^{\t}$, then the lifting problem admits a solution.
\end{cor}
\begin{proof}
If $n=0$, find an object $C\in\cal C_{\inp 0}^{\t}$ which lies over
$g\pr 0$. Such an object exists because the functor $\cal C_{\inp 0}^{\t}\to\cal O_{\inp 0}^{\t}$
is a trivial fibration. The object $C$ is $q$-terminal, so the functor
\[
\cal C_{/C}^{\t}\to\cal O_{/q\pr C}^{\t}\times_{\cal O^{\t}}\cal C^{\t}
\]
is a trivial fibration. This implies the existence of the filler.

If $n>1$, then set $h=f\vert K$. We must solve a lifting problem
% https://q.uiver.app/?q=WzAsNCxbMCwwLCJcXHBhcnRpYWxcXERlbHRhIF5uIl0sWzAsMSwiXFxEZWx0YSBebiJdLFsxLDAsIlxcbWF0aGNhbHtDfV5cXG90aW1lcyBfe2gvfSJdLFsxLDEsIlxcbWF0aGNhbHtPfV5cXG90aW1lc197cWgvfSJdLFswLDEsIiIsMCx7InN0eWxlIjp7InRhaWwiOnsibmFtZSI6Imhvb2siLCJzaWRlIjoidG9wIn19fV0sWzIsMywicSciXSxbMCwyXSxbMSwzXSxbMSwyLCIiLDEseyJzdHlsZSI6eyJib2R5Ijp7Im5hbWUiOiJkYXNoZWQifX19XV0=
\[\begin{tikzcd}
	{\partial\Delta ^n} & {\mathcal{C}^\otimes _{h/}} \\
	{\Delta ^n} & {\mathcal{O}^\otimes_{qh/}.}
	\arrow[hook, from=1-1, to=2-1]
	\arrow["{q'}", from=1-2, to=2-2]
	\arrow[from=1-1, to=1-2]
	\arrow[from=2-1, to=2-2]
	\arrow[dashed, from=2-1, to=1-2]
\end{tikzcd}\]For this, it will suffice to show that the image of the vertex $n\in\partial\Delta^{n}$
under the top horizontal arrow is $q'$-terminal. This follows from
Proposition \ref{prop:relative_terminal_obj_in_slice}.
\end{proof}
\begin{lem}
\cite[Lemma 3.1.2.5]{HA}\label{lem:3.1.2.5}Let $\cal C$ be an $\infty$-category
and $\cal C^{0}\subset\cal C$ a full subcategory. Let $\sigma:\Delta^{n}\to\cal C$
be a nondegenerate simplex such that $\sigma\pr i\not\in\cal C^{0}$
for each $0\leq i\leq n$. Consider the following simplicial sets:
\begin{enumerate}
\item The simplicial subset $K\subset\cal C$ consisting of those simplices
$\tau:\Delta^{k}\star\Delta^{l}\to\cal C$, where $k,l\geq-1$, $\tau\vert\Delta^{k}$
factors through $\cal C^{0}$ and $\tau\vert\Delta^{l}$ factors through
$\sigma$.
\item The simplicial subset $K_{0}\subset\cal C$ consisting of those simplices
$\tau:\Delta^{k}\star\Delta^{l}\to\cal C$, where $k,l\geq-1$, $\tau\vert\Delta^{k}$
factors through $\cal C^{0}$ and $\tau\vert\Delta^{l}$ factors through
$\sigma\vert\partial\Delta^{n}$.
\item The simplicial subset $L\subset\cal C$ generated by the simplices
$\widetilde{\sigma}:\Delta^{k}\star\Delta^{n}\to\cal C$, where $k\geq-1$,
$\widetilde{\sigma}\vert\Delta^{k}$ factors through $\cal C^{0}$,
and $\widetilde{\sigma}\vert\Delta^{n}=\sigma$. 
\end{enumerate}
Then:
\begin{itemize}
\item [(a)]The map $\cal C_{/\sigma}^{0}\star\Delta^{n}\amalg_{\cal C_{/\sigma}^{0}\star\partial\Delta^{n}}K_{0}\to K_{0}\cup L$
is an isomorphism of simplicial sets.
\item [(b)]The inclusion
\[
K_{0}\cup L\subset K
\]
is a trivial cofibration in the Joyal model structure.
\end{itemize}
\end{lem}
\begin{proof}
We begin with (a). It is clear that the simplicial set $K_{0}\cup L$
is the union of $K_{0}$ and the image of the map $\cal C_{/\sigma}^{0}\star\Delta^{n}\to\cal C$.
Therefore, it will suffice to show that if a simplex $x$ of $\cal C_{/\sigma}^{0}\star\Delta^{n}$
is mapped to a simplex in $K_{0}$, then $x$ belongs to $\cal C_{/\sigma}^{0}\star\partial\Delta^{n}$.
The image of $x$ can be represented by a simplex of the form
\[
\Delta^{k}\star\Delta^{l}\xrightarrow{\id\star u}\Delta^{k}\star\Delta^{n}\xrightarrow{\widetilde{\sigma}}\cal C,
\]
where $k,l\geq-1$, $u:[l]\to[n]$ is a poset map, and $\widetilde{\sigma}$
is an extension of $\sigma$ which maps the vertices of $\Delta^{k}$
into $\cal C^{0}$. We wish to show that if $\sigma u:\Delta^{l}\to\cal C$
factors through $\sigma\vert\partial\Delta^{n}$, then $u$ factors
through $\partial\Delta^{n}$. This is clear, since $\sigma$ is nondegenerate.

Next we prove (b). For each $m\geq n$, let $X_{m}$ denote the simplicial
subset of $K$ consisting of the simplices in $K_{0}$ and the simplices
of the form
\[
\tau:\Delta^{k}\star\Delta^{m'}\to\cal C,
\]
where $k\geq-1$, $n\leq m'<m$, $\tau\vert\Delta^{k}$ factors through
$\cal C^{0}$, and $\tau\vert\Delta^{m'}$ is a degeneration of $\sigma$.
Then we have a filtration 
\[
K_{0}\cup L=X_{n}\subset X_{n+1}\subset\cdots
\]
and $K=\bigcup_{i\geq n}X_{i}$. To prove (b), it will suffice to
show that each inclusion $X_{i}\subset X_{i+1}$ is a trivial cofibration.

Let $\varepsilon:[m]\to[n]$ be a surjective poset map. We let $\cal J\pr{\varepsilon}\subset\Del_{[m]//[n]}$
denote the full subcategory spanned by the factorizations $\pr{\varepsilon',\varepsilon''}=[m]\xrightarrow{\varepsilon'}[m']\xrightarrow{\varepsilon''}[n]$
of $\varepsilon$ into two surjective poset maps, and let $\cal J_{0}\pr{\varepsilon}$
denote its full subcategory spanned by the object other than $\pr{\id,\varepsilon}$.
Let $A\pr{\varepsilon}\subset\cal C_{/\sigma\varepsilon}^{0}$ denote
the union of the image of the maps $\cal C_{/\sigma\varepsilon''}^{0}\to\cal C_{/\sigma\varepsilon''\varepsilon'}^{0}=\cal C_{/\sigma\varepsilon}^{0}$,
where $\pr{\varepsilon',\varepsilon''}$ ranges over the objects of
$\cal J_{0}\pr{\varepsilon}$. We consider the commutative diagram
% https://q.uiver.app/?q=WzAsNCxbMCwwLCJcXGNvcHJvZF97XFx2YXJlcHNpbG9uOlttXVxcdHdvaGVhZHJpZ2h0YXJyb3dbbl19KEEoXFx2YXJlcHNpbG9uKVxcc3RhciBcXERlbHRhIF5tKVxcY3VwKFxcbWF0aGNhbHtDfV4wX3svXFxzaWdtYVxcdmFyZXBzaWxvbn1cXHN0YXJcXHBhcnRpYWxcXERlbHRhXnttfSkiXSxbMCwxLCJYX20iXSxbMSwwLCJcXGNvcHJvZF97XFx2YXJlcHNpbG9uOlttXVxcdHdvaGVhZHJpZ2h0YXJyb3dbbl19XFxtYXRoY2Fse0N9XjBfey9cXHNpZ21hXFx2YXJlcHNpbG9ufVxcc3RhclxcRGVsdGFee219Il0sWzEsMSwiWF97bSsxfSJdLFswLDFdLFswLDJdLFsyLDNdLFsxLDNdXQ==
\[\begin{tikzcd}
	{\coprod_{\varepsilon:[m]\twoheadrightarrow[n]}(A(\varepsilon)\star \Delta ^m)\cup(\mathcal{C}^0_{/\sigma\varepsilon}\star\partial\Delta^{m})} & {\coprod_{\varepsilon:[m]\twoheadrightarrow[n]}\mathcal{C}^0_{/\sigma\varepsilon}\star\Delta^{m}} \\
	{X_m} & {X_{m+1}}
	\arrow[from=1-1, to=2-1]
	\arrow[from=1-1, to=1-2]
	\arrow[from=1-2, to=2-2]
	\arrow[from=2-1, to=2-2]
\end{tikzcd}\]where the coproducts are indexed by the surjective poset maps $[m]\epi[n]$.
To complete the proof, it suffices to prove the following:
\begin{itemize}
\item [(i)]The square is a pushout.
\item [(ii)]For each surjection $\varepsilon:[m]\to[n]$, the inclusion
\[
A\pr{\varepsilon}\star\Delta^{m}\cup\cal C_{/\sigma\varepsilon}^{0}\star\partial\Delta^{m}\to\cal C_{/\sigma\varepsilon}^{0}\star\Delta^{m}
\]
is a trivial cofibration.
\end{itemize}
We start with (i). Let $x$ be a simplex in $\cal C_{/\sigma\varepsilon}^{0}\star\Delta^{m}$
corresponding to a pair $\pr{\widetilde{\sigma\varepsilon}:\Delta^{k}\star\Delta^{m}\to\cal C,\,u:\Delta^{l}\to\Delta^{m}}$,
$k,l\geq-1$, $k+l+1\geq0$. Its image in $\cal C$ is the composition
\[
\alpha:\Delta^{k}\star\Delta^{l}\xrightarrow{\id\star u}\Delta^{k}\star\Delta^{m}\xrightarrow{\widetilde{\sigma\varepsilon}}\cal C.
\]
We wish to show that if $\alpha$ belongs to $X_{m}$, then $\pr{\widetilde{\sigma\varepsilon},u}$
belongs to $A\pr{\varepsilon}\star\Delta^{m}\cup\pr{\cal C_{/\sigma\varepsilon}^{0}\star\partial\Delta^{m}}$.
The claim is obvious if $u$ is not surjective on vertices. So assume
that $u$ is surjective on vertices. Since $\alpha\vert\Delta^{l}$
is a degeneration of $\sigma$, it cannot factor through $\sigma\vert\partial\Delta^{n}$.
Thus $\alpha$ does not belong to $K_{0}$. Hence (since the vertices
of $\sigma$ do not belong to $\cal C^{0}$) we can find a factorization
of $\alpha$ of the form 
\[
\alpha:\Delta^{k}\star\Delta^{l}\xrightarrow{\id\star u'}\Delta^{k}\star\Delta^{m'}\xrightarrow{\widetilde{\sigma\varepsilon'}}\cal C,
\]
where $m'<m$, $\varepsilon':[m']\to[n]$ is a surjection, and $\widetilde{\sigma\varepsilon'}$
is an extension of $\sigma\varepsilon'$ which maps $\Delta^{k}$
into $\cal C^{0}$. Replacing $\Delta^{m'}$ by $\Delta^{u'[l]}$
if necessary, we may assume that $u'$ is surjective on vertices.
Factor the map $\widetilde{\sigma\varepsilon'}$ as 
\[
\Delta^{k}\star\Delta^{m'}\xrightarrow{s\star s'}\Delta^{p}\star\Delta^{m''}\xrightarrow{\tau}\cal C,
\]
where $p,m''\geq-1$, $s,s'$ are surjective on vertices and $\tau$
is nondegenerate. (Such a factorization exists because the vertices
of $\sigma$ do not belong to $\cal C^{0}$.) Then $\sigma\varepsilon'$
is a degeneration of $\tau\vert\Delta^{q}$, so Eilenberg-Zilber's
lemma implies that $\tau\vert\Delta^{m''}$ is a degeneration of $\sigma$.
Therefore, we can write $\tau\vert\Delta^{m''}=\sigma\varepsilon''$,
where $\varepsilon'':[m'']\to[n]$ is a surjective poset map. By Eilenberg-Zilber's
lemma, $\widetilde{\sigma\varepsilon'}$ is a degeneartion of $\tau$.
Since $m''<m$, this implies that the simplex $\pr{\widetilde{\sigma\varepsilon},u}$
belongs to $A\pr{\varepsilon}\star\Delta^{m}$, as required.

We next prove (ii). Since join functors preserves trivial cofibrations
in each variable, it will suffice to show that the inclusion $A\pr{\varepsilon}\subset\cal C_{/\sigma\varepsilon}^{0}$
is a trivial cofibration. Arguing as in the previous paragraph, we
see that $A\pr{\varepsilon}$ is the colimit of the diagram $F:\cal J_{0}\pr{\varepsilon}^{\op}\to\SS$
defined by $\pr{\varepsilon',\varepsilon''}\mapsto\cal C_{/\sigma\varepsilon'}^{0}$.
This diagram admits a natural transformation to the constant diagram
at $\cal C_{/\sigma\varepsilon}^{0}$ whose components are trivial
fibrations. It will therefore suffice to prove that the diagrams $F$
and the constant diagram at $\cal C_{/\sigma\varepsilon}^{0}$ are
projectively cofibrant. There is a functor $\cal J_{0}\pr{\varepsilon}^{\op}\to\omega$
which maps a factorization $[m]\to[m']\to[n]$ of $\varepsilon$ to
the integer $m'$. This functor endows the category $\cal J_{0}\pr{\varepsilon}^{\op}$
with the structure of a direct category. The claim is thus a consequence
of \cite[Theorem 5.1.3]{Hovey2007}.
\end{proof}
\begin{defn}
Let $X$ be a simplicial set over $\Delta^{1}$. Given a simplex $\Delta^{k}\to X$,
we define the \textbf{head} of $X$ to be the simplex $\Delta^{u^{-1}\pr 1}\to\Delta^{k}\to X$
and the \textbf{tail} to be the simplex $\Delta^{u^{-1}\pr 0}\to\Delta^{k}\to X$,
where $u$ denotes the composite $\Delta^{k}\to X\to\Delta^{1}$.
\end{defn}
\begin{prop}
\label{prop:heads_and_tails}Let % https://q.uiver.app/?q=WzAsMyxbMCwwLCJYIl0sWzIsMCwiWSJdLFsxLDEsIlxcRGVsdGFeMSJdLFswLDEsInAiXSxbMSwyXSxbMCwyXV0=
\[\begin{tikzcd}
	X && Y \\
	& {\Delta^1}
	\arrow["p", from=1-1, to=1-3]
	\arrow[from=1-3, to=2-2]
	\arrow[from=1-1, to=2-2]
\end{tikzcd}\]be a commutative diagram of simplicial sets. Let $n\geq0$, and let
$S\subset S'\subset Y_{1}=Y\times_{\Delta^{1}}\{1\}$ be simplicial
subsets satisfying the following conditions:
\begin{enumerate}
\item The simplicial set $S$ contains the $\pr{n-1}$-skeleton of $Y_{1}$.
\item The simplicial set $S'$ is generated by $S$ and a set $\Sigma$
of nondegenerate $n$-simplices of $Y_{1}$ which do not belong to
$S'$.
\end{enumerate}
Let $X\pr S$ denote the simplicial subset of $X$ spanned by the
simplices whose head lies over $S$, and define $X\pr{S'}$ similarly.
Let $\{\sigma_{a}\}_{a\in A}$ be the enumeration of all nondegenerate
simplices of $X_{1}$ whose image in $Y$ is a degeneration of a simplex
in $\Sigma$. Choose an ordering of $A$ so that that dimension of
$\sigma_{a}$ is a non-decreasing function of $a\in A$. For each
$a\in A$, define simplicial sets $X\pr{S'}_{<a}$, $X\pr{S'}_{\leq a}$,
$K_{a}$, and $K_{0,a}$ as follows:
\begin{itemize}
\item $X\pr{S'}_{<a}\subset X$ is the simplicial subset generated by $X\pr S$
and those simplices of $X$ whose head factors through $\sigma_{b}$
for some $b<a$.
\item $X\pr{S'}_{\leq a}\subset X$ is the simplicial subset generated by
$X\pr S$ and those simplices of $X$ whose head factors through $\sigma_{b}$
for some $b\leq a$.
\item $K_{a}\subset X$ is the simplicial subset consisting of the simplices
whose head factors through $\sigma_{a}$.
\item $K_{0,a}\subset X$ is the simplicial subset consisting of the simplices
whose head factors through $\partial\sigma_{a}=\sigma_{a}\vert\partial\Delta^{\dim\sigma_{a}}$.
\end{itemize}
Then the following holds:
\begin{enumerate}
\item $X\pr{S'}=X\pr S\cup\bigcup_{a\in A}X\pr{S'}_{\leq a}$.
\item For each $a\in A$, the square % https://q.uiver.app/?q=WzAsNCxbMCwwLCJLX3swLGF9Il0sWzAsMSwiS19hIl0sWzEsMCwiWChTJylfezxhfSJdLFsxLDEsIlgoUycpX3tcXGxlcSBhfSJdLFswLDFdLFswLDJdLFsyLDNdLFsxLDNdXQ==
\begin{equation}\label{d:h_and_t}
\begin{tikzcd}
	{K_{0,a}} & {X(S')_{<a}} \\
	{K_a} & {X(S')_{\leq a}}
	\arrow[from=1-1, to=2-1]
	\arrow[from=1-1, to=1-2]
	\arrow[from=1-2, to=2-2]
	\arrow[from=2-1, to=2-2]
\end{tikzcd}
\end{equation}of simplicial sets is cocartesian.
\end{enumerate}
\end{prop}
\begin{proof}
We start with (1). The containment $X\pr S\cup\bigcup_{a\in A}X\pr{S'}_{\leq a}\subset X\pr{S'}$
holds trivially. For the reverse inclusion, let $x$ be an arbitrary
simplex of $X\pr{S'}$. We must show that $x$ belongs to $X\pr S\cup\bigcup_{a\in A}X\pr{S'}_{\leq a}$.
Let $\tau\pr x:\Delta^{m}\to X_{1}$ denote the head of $x$. If $p\tau\pr x$
belongs to $S$, then $x$ belongs to $X\pr S$ and we are done. If
$\tau\pr x$ does not belong to $S$, then $p\tau\pr x$ factors through
a simplex $\sigma$ in $\Sigma$. If $p\tau\pr x$ factors through
the boundary of $\sigma$, then $p\tau\pr x$ belongs to the $\pr{n-1}$-skeleton
of $Y_{1}$ and hence $\tau\pr x$ belongs to $S$, a contradiction.
Therefore, $p\tau\pr x$ is a degeneration of $\sigma$. Now $\tau\pr x$
is a degeneration of some nondegenerate simplex $\tau'\pr x$ of $X_{1}$.
Then $p\tau\pr x$ is a degeneration of $p\tau'\pr x$, so Eilenberg-Zilber's
lemma implies that $p\tau'\pr x$ is a degeneration of $\sigma$.
Hence $\tau'\pr x=\sigma_{a}$ for some $a\in A$, and hence $x\in X\pr{S'}_{\leq a}$.

We next prove (2). We will write $K_{0}=K_{0,a}$ and $K=K_{a}$.
First we remark that $X\pr{S'}_{<a}$ contains $K_{0}$. Indeed, suppose
we are given a simplex $x$ of $X$ whose head $\tau\pr x:\Delta^{m}\to X_{1}$
facotors through $\partial\sigma_{a}$. By construction, there is
a commutative diagram % https://q.uiver.app/?q=WzAsNixbMSwwLCJcXERlbHRhXntcXGRpbVxcc2lnbWFfYX0iXSxbMiwwLCJcXERlbHRhXm4iXSxbMiwxLCJZXzEsIl0sWzEsMSwiWF8xIl0sWzAsMCwiXFxwYXJ0aWFsXFxEZWx0YV57XFxkaW1cXHNpZ21hX2F9Il0sWzAsMSwiXFxEZWx0YV5tIl0sWzAsMSwicyJdLFsxLDIsIlxcc2lnbWEiXSxbMywyLCJwIiwyXSxbMCwzLCJcXHNpZ21hX2EiLDJdLFs0LDAsIiIsMCx7InN0eWxlIjp7InRhaWwiOnsibmFtZSI6Imhvb2siLCJzaWRlIjoidG9wIn19fV0sWzUsMywiXFx0YXUoeCkiLDJdLFs1LDRdXQ==
\[\begin{tikzcd}
	{\partial\Delta^{\dim\sigma_a}} & {\Delta^{\dim\sigma_a}} & {\Delta^n} \\
	{\Delta^m} & {X_1} & {Y_1,}
	\arrow["s", from=1-2, to=1-3]
	\arrow["\sigma", from=1-3, to=2-3]
	\arrow["p"', from=2-2, to=2-3]
	\arrow["{\sigma_a}"', from=1-2, to=2-2]
	\arrow[hook, from=1-1, to=1-2]
	\arrow["{\tau(x)}"', from=2-1, to=2-2]
	\arrow[from=2-1, to=1-1]
\end{tikzcd}\]where $s$ is surjective on vertices. If the map $\Delta^{m}\to\Delta^{n}$
is not surjective on vertices, then $p\tau\pr x$ factors through
the $\pr{n-1}$-skeleton of $Y_{1}$ and hence $x$ belongs to $X\pr S$.
If the map $\Delta^{m}\to\Delta^{n}$ is surjective on vertices, then
write $\tau\pr x$ as a degeneration of a nondegenerate simplex $\tau'\pr x$
of $X_{1}$. Eilenberg-Zilber's lemma implies that the simplex $p\tau'\pr x$
must be a degeneration of $\sigma$. Thus $p\tau'\pr x=\sigma_{b}$
for some $b\in A$. Since $\tau\pr x$ factors through the boundary
of $\sigma_{a}$, the dimension of $\sigma_{b}$ is strictly smaller
than that of $\sigma_{a}$. Thus $b<a$. Hence $x$ belongs to $X\pr{S'}_{<a}$.

By what we have just shown in the previous paragraph, the diagram
(\ref{d:h_and_t}) is well-defined. We now show that it is cocartesian.
By definition, $X\pr{S'}_{\leq a}$ is the union of $K$ and $X\pr{S'}_{<a}$.
Therefore, it suffices to show that $K_{0}$ is the intersection of
$K$ and $X\pr{S'}_{<a}$. So let $x$ be a simplex of $K$. We must
show that, if $x$ does not belong to $K_{0}$, then $x$ does not
belong to $X\pr{S'}_{<a}$ either. Let $\tau\pr x:\Delta^{m}\to X_{1}$
be the head of $x$. Since $x$ does not belong to $K_{0}$, $\tau\pr x$
is a degeneration of $\sigma_{a}$. Thus $p\pr{\tau\pr x}$ is a degeneration
of some simplex $\sigma\in\Sigma$. In particular, $p\pr{\tau\pr x}$
does not belong to $S$, so $\tau\pr x$ does not belong to $X\pr S$.
Therefore, should $\tau\pr x$ belong to $X\pr{S'}_{<a}$, then $\tau\pr x$
must factor through some $\sigma_{b}$ for some $b<a$. If $\tau\pr x$
factors through $\partial\sigma_{b}$ for some $b<a$, then we would
have $\dim\sigma_{a}<\dim\sigma_{b}$, a contradiction. So $\tau\pr x$
is a degeneartion of $\sigma_{b}$; but then $a=b$, a contradiction.
Thus $x$ does not belong to $X\pr{S'}_{<a}$, as required.
\end{proof}

\subsection{Proof of Theorem \ref{thm:3.1.2.3}}

In this final subsection of this note, we will give a proof of Theorem
\ref{thm:3.1.2.3} by elaborating the proof given by Lurie in \cite{HA}.

The proof proceeds by a simplex-by-simplex argument. For this, we
will classify simplices of $N\pr{\Fin_{\ast}}\times\Delta^{\{1,\dots,n\}}$
into five (somewhat artificial) groups.
\begin{defn}
Let $n\geq1$ and let $\alpha$ be a morphism in $N\pr{\Fin_{\ast}}\times\Delta^{\{1,\dots,n\}}$
with image $\alpha_{0}:\inp m\to\inp n$ in $N\pr{\Fin_{\ast}}$.
We say that $\alpha$ is:
\begin{enumerate}
\item \textbf{active} if $\alpha_{0}$ is active;
\item \textbf{strongly inert} if $\alpha_{0}$ is inert, the induced injection
$\inp n^{\circ}\to\inp m^{\circ}$ is order-perserving, and the image
of $\alpha$ in $\Delta^{\{1,\dots,n\}}$ is degenerate; and
\item \textbf{neutral} if it is neigher active nor strongly inert.
\end{enumerate}
Note that active morphisms and strongly inert morphisms are closed
under composition. Also, every morphism in $N\pr{\Fin_{\ast}}\times\Delta^{\{1,\dots,n\}}$
can be factored uniquely as a composition of a strongly inert map
followed by an active map.

Let $\sigma$ be an $m$-simplex of $N\pr{\Fin_{\ast}}\times\Delta^{\{1,\dots,n\}}$
depicted as
\[
\pr{\inp{k_{0}},e_{0}}\xrightarrow{\alpha_{\sigma}\pr 1}\cdots\xrightarrow{\alpha_{\sigma}\pr m}\pr{\inp{k_{m}},e_{m}}.
\]
We will say that $\sigma$ is \textbf{closed} if $k_{m}=1$, and \textbf{open}
otherwise. We say that $\sigma$ is \textbf{complete}\footnote{Lurie uses te term ``new'' instead of ``complete.''}
if $\{e_{0},\dots,e_{m}\}=\{1,\dots,n\}$ and \textbf{incomplete}
otherwise. Every nondegenerate simplex of $N\pr{\Fin_{\ast}}\times\Delta^{\{1,\dots,n\}}$
is a face of a nondegenerate complete simplex. 

We partition the set of nondegenerate complete simplices of $N\pr{\Fin_{\ast}}\times\Delta^{\{1,\dots n\}}$
into five groups $G_{\pr 1},G_{\pr 2},G'_{\pr 2},G_{\pr 3},G_{\pr 3}'$
as follows: $m$-simplex of $N\pr{\Fin_{\ast}}\times\Delta^{\{1,\dots,n\}}$.
Write $\alpha_{\sigma}\pr i=\sigma\vert\Delta^{\{i,i+1\}}$. Let $0\le k\leq m$
be the minimal integer such that $\alpha_{\sigma}\pr i$ is strongly
inert for every $i>k$, and let $0\leq j\leq k$ be the minimal integer
such that $\alpha_{\sigma}\pr i$ is active for every $j<i\leq k$.
\begin{itemize}
\item If $j=0$, $k=m$, and $\sigma$ is closed, then $\sigma$ belongs
to $G_{\pr 1}$.
\item If $j=0$, $k<m$, and $\sigma$ is closed, then $\sigma$ belongs
to $G_{\pr 2}$.
\item If $j=0$ and $\sigma$ is open, then $\sigma$ belongs to $G'_{\pr 2}$.
\item If $j\geq1$ and $\alpha_{\sigma}\pr j$ is strongly inert, then $\sigma$
belongs to $G_{\pr 3}$.
\item If $j\geq1$ and $\alpha_{\sigma}\pr j$ is neutral, then $\sigma$
belongs to $G_{\pr 3}'$.
\end{itemize}
Given an $m$-simplex $\sigma\in G_{\pr 2}$, we define its \textbf{associate}
$\sigma'\in G_{\pr 2}'$ by $\sigma'=\sigma\vert\Delta^{\{0,\dots,m-1\}}$.
Given an $m$-simplex $\sigma\in G_{\pr 3}$, then we define its \textbf{associate}
$\sigma'$ by $\sigma'=\sigma\partial_{j}\in G_{\pr 3}'$, where $j$
is the integer defined as above.
\end{defn}
%
\begin{proof}
[Proof of Theorems \ref{thm:3.1.2.3}]We will regard $\cal M^{\t}$
and $N\pr{\Fin_{\ast}}\times\Delta^{n}$ as a simplicial set over
$\Delta^{1}$ by means of the map $\Delta^{n}\to\Delta^{1}$ which
maps the vertex $0\in\Delta^{n}$ to the vertex $0\in\Delta^{1}$
and the remaining vertices to the vertex $1\in\Delta^{1}$. Given
a simplicial subset $S\subset N\pr{\Fin_{\ast}}\times\Delta^{\{1,\dots,n\}}$,
we let $\cal M_{S}^{\t}\subset\cal M^{\t}$ denote the simplicial
subset consisting of the simplices whose tail lies over $S$. We will
also write $\overline{S}=\pr{N\pr{\Fin_{\ast}}\times\Delta^{\{1,\dots,n\}}}_{S}$.

The implication (a)$\implies$(b) for part (A) is obvious. Assume
therefore that condition (b) is satisfied if $n=1$. For each $m\geq0$,
let $F\pr m$ denote the simplicial subset of $N\pr{\Fin_{\ast}}\times\Delta^{\{1,\dots,n\}}$
generated by the nondegenerate simplices $\sigma$ satisfying one
of the following conditions:
\begin{itemize}
\item $\sigma$ is incomplete.
\item $\sigma$ has dimension less than $m$.
\item $\sigma$ has dimension $m$ and belongs to $G_{\pr 2}$ or $G_{\pr 3}$.
\end{itemize}
Observe that $\cal M_{F\pr 0}^{\t}=\cal M^{\t}\times_{\Delta^{n}}\Lambda_{0}^{n}$.
We will complete the proof by inductively constructing a map $f_{m}:\cal M_{F\pr m}^{\t}\to\cal C^{\t}$
which makes the diagram % https://q.uiver.app/?q=WzAsNCxbMCwwLCJcXG1hdGhjYWx7TX1eXFxvdGltZXMgX3tGKG0tMSl9Il0sWzAsMSwiXFxtYXRoY2Fse019Xlxcb3RpbWVzIF97RihtKX0iXSxbMSwxLCJcXG1hdGhjYWx7T31eXFxvdGltZXMgIl0sWzEsMCwiXFxtYXRoY2Fse0N9Xlxcb3RpbWVzIl0sWzAsMSwiIiwwLHsic3R5bGUiOnsidGFpbCI6eyJuYW1lIjoiaG9vayIsInNpZGUiOiJ0b3AifX19XSxbMSwyLCJnXFx2ZXJ0XFxtYXRoY2Fse019Xlxcb3RpbWVzIF97RihtKX0iLDJdLFszLDIsInEiXSxbMCwzLCJmX3ttLTF9Il0sWzEsMywiZl97bX0iLDFdXQ==
\[\begin{tikzcd}
	{\mathcal{M}^\otimes _{F(m-1)}} & {\mathcal{C}^\otimes} \\
	{\mathcal{M}^\otimes _{F(m)}} & {\mathcal{O}^\otimes }
	\arrow[hook, from=1-1, to=2-1]
	\arrow["{g\vert\mathcal{M}^\otimes _{F(m)}}"', from=2-1, to=2-2]
	\arrow["q", from=1-2, to=2-2]
	\arrow["{f_{m-1}}", from=1-1, to=1-2]
	\arrow["{f_{m}}"{description}, from=2-1, to=1-2]
\end{tikzcd}\]commutative, and such that $f_{1}$ has the following special properties
if $n=1$:
\begin{itemize}
\item [(i)]For each object $B\in\cal M^{\t}\times_{\Delta^{1}}\{1\}$,
the map 
\[
\pr{\pr{\cal M_{\act}^{\t}}_{/B}\times_{\Delta^{1}}\{0\}}^{\rcone}\to\cal M_{\pr 1}^{\t}\xrightarrow{f_{1}}\cal C^{\t}
\]
is an operadic $q$-colimit diagram.
\item [(ii)]For every inert morphism $e:M'\to M$ in $\cal M^{\t}\times_{\Delta^{1}}\{1\}$
such that $M\in\cal M$, the functor $f_{1}$ carries $e$ to an inert
morphism in $\cal C^{\t}$.
\end{itemize}
Fix $m>0$, and suppose that $f_{m-1}$ has been constructed. Observe
that $F\pr m$ is obtained from $F\pr{m-1}$ by adjoining the following
simplices:
\begin{itemize}
\item The $\pr{m-1}$-simplices in $G_{\pr 1}$.
\item The $\pr{m-1}$-simplices in $G'_{\pr 2}$ without associates.
\item The $m$-simplices in $G_{\pr 2}$ and $G_{\pr 3}$.
\end{itemize}
We define simplicial subsets $F'\pr m\subset F''\pr m\subset F\pr m$
as follows: $F'\pr m$ is generated by $F\pr{m-1}$ and the $\pr{m-1}$-simplices
in $G_{\pr 1}$; $F'\pr{m-1}$ is generated by $F'\pr m$ and the
$\pr{m-1}$-simplices of $G'_{\pr 2}$ without associates. Our strategy
is to extend $f_{m-1}$ to $\cal M_{F'\pr m}^{\t}$, then to $\cal M_{F''\pr m}^{\t}$,
and then to $\cal M_{F\pr m}^{\t}$. 

\begin{enumerate}[label=(\textbf{Step \arabic*}),wide =0.5\parindent, listparindent=1.5em]

\item We will extend $f_{m-1}$ to a map $f'_{m}:\cal M_{F'\pr m}^{\t}\to\cal C^{\t}$
over $\cal O^{\t}$. 

Let $\{\sigma_{a}\}_{a\in A}$ be the collection of nondegenerate
simplices of $\cal M^{\t}\times_{\Delta^{n}}\Delta^{\{1,\dots,n\}}$
whose image in $N\pr{\Fin_{\ast}}\times\Delta^{\{1,\dots,n\}}$ is
a degeneration of some $\pr{m-1}$-simplex of $G_{\pr 1}$. Choose
a well-ordering on $A$ so that $\dim\sigma_{a}$ is non-decreasing
in $a\in A$. For each $a\in A$, let $\cal M_{<a}^{\t}$ denote the
simplicial subset spanned by $\cal M_{F\pr{m-1}}^{\t}$ and the simplices
of $\cal M^{\t}$ whose tail factors through $\sigma_{b}$ for some
$b<a$. We define $\cal M_{\leq a}^{\t}$ similarly. According to
Proposition \ref{prop:heads_and_tails}, we have $\cal M_{F'\pr m}^{\t}=\bigcup_{a\in A}\cal M_{\leq a}^{\t}$,
so it suffices to extend $f_{m-1}$ to an $A$-sequence $f^{\leq a}:\cal M_{\leq a}^{\t}\to\cal C^{\t}$
over $\cal O^{\t}$. 

The construction is inductive. Let $a\in A$, and suppose that $f^{\leq b}$
has been constructed for $b<a$. These maps determine a map $f^{<a}:\cal M_{<a}^{\t}\to\cal C^{\t}$
extending $f_{m-1}$. Let $K_{0}\subset K\subset\cal M^{\t}$ denote
the simplicial subset consisting of the simplices of $\cal M^{\t}$
whose head factors through $\sigma_{a}\vert\partial\Delta^{\dim\sigma_{a}}$
and $\sigma_{a}$, respectively. According to Lemma \ref{lem:3.1.2.5},
the left hand square of the commutative diagram% https://q.uiver.app/?q=WzAsNixbMCwwLCIoXFxtYXRoY2Fse019Xlxcb3RpbWVzIF97L1xcc2lnbWFfYX1cXHRpbWVzX3tcXERlbHRhXm59IFxcezBcXH0pXFxzdGFyIFxccGFydGlhbFxcRGVsdGEgXntcXGRpbVxcc2lnbWEgX2F9Il0sWzAsMSwiKFxcbWF0aGNhbHtNfV5cXG90aW1lcyBfey9cXHNpZ21hX2F9XFx0aW1lc197XFxEZWx0YV5ufSBcXHswXFx9KVxcc3RhciBcXERlbHRhIF57XFxkaW1cXHNpZ21hIF9hfSJdLFsxLDAsIktfMCJdLFsxLDEsIksiXSxbMiwwLCJcXG1hdGhjYWx7TX1eXFxvdGltZXMgX3s8YX0iXSxbMiwxLCJcXG1hdGhjYWx7TX1eXFxvdGltZXMgX3tcXGxlcSBhfSJdLFswLDJdLFsxLDNdLFsyLDNdLFsyLDRdLFs0LDVdLFszLDVdLFswLDFdXQ==
\[\begin{tikzcd}
	{(\mathcal{M}^\otimes _{/\sigma_a}\times_{\Delta^n} \{0\})\star \partial\Delta ^{\dim\sigma _a}} & {K_0} & {\mathcal{M}^\otimes _{<a}} \\
	{(\mathcal{M}^\otimes _{/\sigma_a}\times_{\Delta^n} \{0\})\star \Delta ^{\dim\sigma _a}} & K & {\mathcal{M}^\otimes _{\leq a}}
	\arrow[from=1-1, to=1-2]
	\arrow[from=2-1, to=2-2]
	\arrow[from=1-2, to=2-2]
	\arrow[from=1-2, to=1-3]
	\arrow[from=1-3, to=2-3]
	\arrow[from=2-2, to=2-3]
	\arrow[from=1-1, to=2-1]
\end{tikzcd}\]is homotopy cocartesian. The right hand square is cocartesian by Proposition
\ref{prop:heads_and_tails}. It follows that the map
\[
\pr{\cal M_{/\sigma_{a}}^{\t}\times_{\Delta^{n}}\{0\}}\star\Delta^{\dim\sigma_{a}}\amalg_{\pr{\cal M_{/\sigma_{a}}^{\t}\times_{\Delta^{n}}\{0\}}\star\partial\Delta^{\dim\sigma_{a}}}\cal M_{<a}^{\t}\to\cal M_{\leq a}^{\t}
\]
is a trivial cofibration in the Joyal model structure. Thus we only
need to extend the composite
\[
g_{0}:\pr{\cal M_{/\sigma_{a}}^{\t}\times_{\Delta^{n}}\{0\}}\star\partial\Delta^{\dim\sigma_{a}}\to\cal M_{<a}^{\t}\xrightarrow{f^{<a}}\cal C^{\t}
\]
to a map $\pr{\cal M_{/\sigma_{a}}^{\t}\times_{\Delta^{n}}\{0\}}\star\Delta^{\dim\sigma_{a}}\to\cal C^{\t}$
over $\cal O^{\t}$.

Assume first that $\sigma_{a}$ is zero-dimensional, so that, in particular,
$m=1$. If $n>1$, then $F\pr 0=F'\pr 1$ and there is nothing to
do. If $n=1$, then let $B\in\cal M^{\t}$ be the image of $\sigma_{a}$.
Since $\sigma_{a}$ is closed, the object $B$ lies in $\cal M\times_{\Delta^{1}}\{1\}$.
Using the inert-active factorization system in $\cal M^{\t}$, we
see that the inclusion $\pr{\pr{\cal M_{\act}^{\t}}_{/B}\times_{\Delta^{1}}\{0\}}\subset\cal M_{/B}^{\t}\times_{\Delta^{1}}\{0\}$
is a right adjoint, hence final. So our assumption (b) ensures that
we can find the desired extension $g$. Note that condition (i) is
satisfied with this particular construction.

Assume next that $\sigma_{a}$ has positive dimension. Since $\Delta^{\dim\sigma_{a}}$
has an initial vertex, we see as in the previous paragraph that the
inclusion $\pr{\cal M_{\act}^{\t}}_{/\sigma_{a}}\times_{\Delta^{n}}\{0\}\subset\cal M_{/\sigma_{a}}^{\t}\times_{\Delta^{1}}\{0\}$
is a right adjoint, and hence final. Thus, by virtue of \cite[Proposition 3.1.1.7]{HA},
it suffices to show that the map
\[
\pr{\pr{\cal M_{\act}^{\t}}_{/\sigma_{a}}\times_{\Delta^{n}}\{0\}}\star\{0\}\to\cal M_{<a}^{\t}\xrightarrow{f^{<a}}\cal C^{\t}
\]
is an operadic $q$-colimit diagram. Let $B=\sigma_{a}\pr 0$. The
map $\pr{\cal M_{\act}^{\t}}_{/\sigma_{a}}\times_{\Delta^{n}}\{0\}\to\pr{\cal M_{\act}^{\t}}_{/B}\times_{\Delta^{n}}\{0\}$
is a trivial fibration, so it suffices to show that the map
\[
\phi:\pr{\pr{\cal M_{\act}^{\t}}_{/B}\times_{\Delta^{n}}\{0\}}\star\{0\}\to\cal M_{<a}^{\t}\xrightarrow{f^{<a}}\cal C^{\t}
\]
is an operadic $q$-colimit cone. Let $\inp p\in N\pr{\Fin_{\ast}}$
be the image of the object $B$. If $p=0$, then $\pr{\cal M_{\act}^{\t}}_{/B}\times_{\Delta^{n}}\{0\}$
is a contractible Kan complex, and for each object $\alpha:A\to B$
in $\pr{\cal M_{\act}^{\t}}_{/B}\times_{\Delta^{n}}\{0\}$, the map
$\phi\vert\{\alpha\}\star\{0\}$ is an equivalence in $\cal C^{\t}$
(since $\cal C_{\inp 0}^{\t}$ is a contractible Kan complex). Thus
$\phi$ is an operadic $q$-colimit cone by \cite[Example 3.1.1.6]{HA}.
If $p=1$, the claim follows from (i). If $p>1$, choose for each
$1\leq i\leq p$ an inert map $B\to B_{i}$ in $\cal M_{\act}^{\t}\times_{\Delta^{n}}\{1\}$
over $\rho^{i}:\inp p\to\inp 1$. Note that since $p>1$, we have
$m\geq2$, so that $f^{<a}$ is defined on $\cal M_{F\pr 1}^{\t}$. 

Using Proposition \ref{prop:oplus_p-limit}, choose a direct sum functor
$\bigoplus_{i=1}^{n}:\pr{\cal M_{\act}^{\t}}^{p}\times_{\pr{\Delta^{1}}^{p}}\Delta^{1}\to\cal M^{\t}$
so that there is an inert natural transformation $\widetilde{h}_{i}:\bigoplus_{i=1}^{n}\to\opn{pr}_{i}$
making the diagram % https://q.uiver.app/?q=WzAsNCxbMCwwLCIoKFxcbWF0aGNhbHtNfV5cXG90aW1lc197XFxtYXRocm17YWN0fX0pXnBcXHRpbWVzIF97KFxcRGVsdGFebilecH1cXERlbHRhXm4pXFx0aW1lcyBcXERlbHRhIF4xIl0sWzEsMCwiXFxtYXRoY2Fse019Xlxcb3RpbWVzICJdLFsxLDEsIlxcRGVsdGFeblxcdGltZXMgTihcXG1hdGhzZntGaW59X1xcYXN0KSJdLFswLDEsIihcXERlbHRhXm5cXHRpbWVzIE4oXFxtYXRoc2Z7RmlufV9cXGFzdCApX3tcXG1hdGhybXthY3R9fSlecClcXHRpbWVzIFxcRGVsdGFeMSJdLFswLDEsIlxcd2lkZXRpbGRle2h9X2kiXSxbMSwyLCJwIl0sWzMsMiwiXFxvcGVyYXRvcm5hbWV7aWR9X3tcXERlbHRhXm59XFx0aW1lcyBoX2kiLDJdLFswLDNdXQ==
\[\begin{tikzcd}
	{((\mathcal{M}^\otimes_{\mathrm{act}})^p\times _{(\Delta^n)^p}\Delta^n)\times \Delta ^1} & {\mathcal{M}^\otimes } \\
	{(\Delta^n\times N(\mathsf{Fin}_\ast )_{\mathrm{act}})^p)\times \Delta^1} & {\Delta^n\times N(\mathsf{Fin}_\ast)}
	\arrow["{\widetilde{h}_i}", from=1-1, to=1-2]
	\arrow["p", from=1-2, to=2-2]
	\arrow["{\operatorname{id}_{\Delta^n}\times h_i}"', from=2-1, to=2-2]
	\arrow[from=1-1, to=2-1]
\end{tikzcd}\]commutative. There is an equivalence $\alpha:B\xrightarrow{\simeq}\bigoplus_{i=1}^{p}B_{i}$
lying over the identity of $\inp p$. There is a natural equivalence
\[
H:\pr{\pr{\cal M_{\act}^{\t}}_{/\alpha}\times_{\Delta^{n}}\{0\}}^{\rcone}\times\Delta^{1}\to\cal M^{\t}
\]
which makes the diagram % https://q.uiver.app/?q=WzAsNixbMCwwLCIoKFxcbWF0aGNhbHtNfV5cXG90aW1lcyBfe1xcbWF0aHJte2FjdH19KV97L1xcYWxwaGF9XFx0aW1lcyBfe1xcRGVsdGFebn1cXHswXFx9KV5cXHRyaWFuZ2xlcmlnaHRcXHRpbWVzIFxcezBcXH0iXSxbMSwwLCIoKFxcbWF0aGNhbHtNfV5cXG90aW1lcyBfe1xcbWF0aHJte2FjdH19KV97L0J9XFx0aW1lcyBfe1xcRGVsdGFebn1cXHswXFx9KV5cXHRyaWFuZ2xlcmlnaHQiXSxbMCwxLCIoKFxcbWF0aGNhbHtNfV5cXG90aW1lcyBfe1xcbWF0aHJte2FjdH19KV97L1xcYWxwaGF9XFx0aW1lcyBfe1xcRGVsdGFebn1cXHswXFx9KV5cXHRyaWFuZ2xlcmlnaHRcXHRpbWVzIFxcRGVsdGFeMSJdLFsxLDEsIlxcbWF0aGNhbHtNfV5cXG90aW1lcyAiXSxbMCwyLCIoKFxcbWF0aGNhbHtNfV5cXG90aW1lcyBfe1xcbWF0aHJte2FjdH19KV97L1xcYWxwaGF9XFx0aW1lcyBfe1xcRGVsdGFebn1cXHswXFx9KV5cXHRyaWFuZ2xlcmlnaHRcXHRpbWVzIFxcezFcXH0iXSxbMSwyLCIoKFxcbWF0aGNhbHtNfV5cXG90aW1lcyBfe1xcbWF0aHJte2FjdH19KV97L1xcYmlnb3BsdXNfe2k9MX1ebkJfaX1cXHRpbWVzIF97XFxEZWx0YV5ufVxcezBcXH0pXlxcdHJpYW5nbGVyaWdodFxcdGltZXMgXFx7MVxcfSJdLFswLDFdLFswLDJdLFsxLDNdLFsyLDNdLFs0LDJdLFs1LDNdLFs0LDVdXQ==
\[\begin{tikzcd}
	{((\mathcal{M}^\otimes _{\mathrm{act}})_{/\alpha}\times _{\Delta^n}\{0\})^\triangleright\times \{0\}} & {((\mathcal{M}^\otimes _{\mathrm{act}})_{/B}\times _{\Delta^n}\{0\})^\triangleright} \\
	{((\mathcal{M}^\otimes _{\mathrm{act}})_{/\alpha}\times _{\Delta^n}\{0\})^\triangleright\times \Delta^1} & {\mathcal{M}^\otimes } \\
	{((\mathcal{M}^\otimes _{\mathrm{act}})_{/\alpha}\times _{\Delta^n}\{0\})^\triangleright\times \{1\}} & {((\mathcal{M}^\otimes _{\mathrm{act}})_{/\bigoplus_{i=1}^nB_i}\times _{\Delta^n}\{0\})^\triangleright\times \{1\}}
	\arrow[from=1-1, to=1-2]
	\arrow[from=1-1, to=2-1]
	\arrow[from=1-2, to=2-2]
	\arrow[from=2-1, to=2-2]
	\arrow[from=3-1, to=2-1]
	\arrow[from=3-2, to=2-2]
	\arrow[from=3-1, to=3-2]
\end{tikzcd}\]commutative, such that the restriction of $H$ to $\pr{\pr{\cal M_{\act}^{\t}}_{/\alpha}\times_{\Delta^{n}}\{0\}}\times\Delta^{1}$
is the constant natural transformation at the projection $\pr{\pr{\cal M_{\act}^{\t}}_{/\alpha}\times_{\Delta^{n}}\{0\}}\to\cal M^{\t}$
and the restriction $H\vert\{\infty\}\times\Delta^{1}$ is given by
$\alpha$. This natural equivalence takes values in $\cal M_{F\pr 1}^{\t}$,
because $\alpha$ lies in $\cal M_{F\pr 1}^{\t}$. So it suffices
to show that the composite 
\[
\phi:\pr{\pr{\cal M_{\act}^{\t}}_{/\bigoplus_{i=1}^{p}B_{i}}\times_{\Delta^{n}}\{0\}}^{\rcone}\to\cal M_{F\pr 1}^{\t}\xrightarrow{f_{1}}\cal C^{\t}
\]
is an operadic $q$-colimit diagram. Now consider the commutative
diagram % https://q.uiver.app/?q=WzAsNSxbMCwwLCIoXFxwcm9kX3sxXFxsZXEgaVxcbGVxIHB9KFxcbWF0aGNhbHtNfV5cXG90aW1lcyBfe1xcbWF0aHJte2FjdH19KV97L0JfaX1cXHRpbWVzIF97XFxEZWx0YV5ufVxcezBcXH0pXlxcdHJpYW5nbGVyaWdodCJdLFsxLDAsIigoKFxcbWF0aGNhbHtNfV5cXG90aW1lcyBfe1xcbWF0aHJte2FjdH19KV5wXFx0aW1lcyBfeyhcXERlbHRhXm4pXnB9XFxEZWx0YV5uKV97LyhCXzEsXFxkb3RzICxCX3ApfVxcdGltZXMgX3tcXERlbHRhXm59XFx7MFxcfSleXFx0cmlhbmdsZXJpZ2h0Il0sWzEsMSwiKChcXG1hdGhjYWx7TX1eXFxvdGltZXMgX3tcXG1hdGhybXthY3R9fSlfey9cXGJpZ29wbHVzX3tpPTF9XnBCX2l9XFx0aW1lcyBfe1xcRGVsdGFebn1cXHswXFx9KV5cXHRyaWFuZ2xlcmlnaHQiXSxbMSwyLCJcXG1hdGhjYWx7TX1eXFxvdGltZXMgLiJdLFswLDIsIihcXG1hdGhjYWx7TX1eXFxvdGltZXMgX3tcXG1hdGhybXthY3R9fSlecFxcdGltZXMgX3soXFxEZWx0YV5uKV5wfVxcRGVsdGFebiJdLFswLDEsIlxcY29uZyJdLFsxLDIsIlxccHNpIl0sWzIsM10sWzQsMywiXFxiaWdvcGx1c197aT0xfV5wIiwyXSxbMCw0XSxbMSwyLCJcXHNpbWVxIiwyXV0=
\[\begin{tikzcd}
	{(\prod_{1\leq i\leq p}(\mathcal{M}^\otimes _{\mathrm{act}})_{/B_i}\times _{\Delta^n}\{0\})^\triangleright} & {(((\mathcal{M}^\otimes _{\mathrm{act}})^p\times _{(\Delta^n)^p}\Delta^n)_{/(B_1,\dots ,B_p)}\times _{\Delta^n}\{0\})^\triangleright} \\
	& {((\mathcal{M}^\otimes _{\mathrm{act}})_{/\bigoplus_{i=1}^pB_i}\times _{\Delta^n}\{0\})^\triangleright} \\
	{(\mathcal{M}^\otimes _{\mathrm{act}})^p\times _{(\Delta^n)^p}\Delta^n} & {\mathcal{M}^\otimes .}
	\arrow["\cong", from=1-1, to=1-2]
	\arrow["\psi", from=1-2, to=2-2]
	\arrow[from=2-2, to=3-2]
	\arrow["{\bigoplus_{i=1}^p}"', from=3-1, to=3-2]
	\arrow[from=1-1, to=3-1]
	\arrow["\simeq"', from=1-2, to=2-2]
\end{tikzcd}\]Here the map $\psi$ is the equivalence of $\infty$-categories induced
by the direct sum functor (Proposition \ref{prop:directsum_slice_equiv}).
In light of the commutativity of this diagram, the composite
\[
\eta:\pr{\prod_{1\leq i\leq p}\pr{\cal M_{\act}^{\t}}_{/B_{i}}\times_{\Delta^{n}}\{0\}}^{\rcone}\to\pr{\cal M_{\act}^{\t}}^{p}\times_{\pr{\Delta^{n}}^{p}}\Delta^{n}\xrightarrow{\bigoplus_{i=1}^{p}}\cal M^{\t}
\]
takes values in $\cal M_{F\pr 1}^{\t}$, and it suffices to show that
the composite $f_{1}\eta$ is an operadic $q$-colimit diagram. Now
the inert natural transformation $\widetilde{h}_{i}$ induces an inert
natural transfromation from $\eta$ to the composite 
\[
\eta_{i}:\pr{\prod_{1\leq i\leq p}\pr{\cal M_{\act}^{\t}}_{/B_{i}}\times_{\Delta^{n}}\{0\}}^{\rcone}\to\pr{\pr{\cal M_{\act}^{\t}}_{/B_{i}}\times_{\Delta^{n}}\{0\}}^{\rcone}\to\cal M^{\t}.
\]
Since $F\pr 1$ contains the $1$-simplices in $G_{\pr 2}$, this
natural transformation takes values in $\cal M_{F\pr 1}^{\t}$. Since
$f_{1}$ satisfies (ii), we deduce that the composite $f_{1}\eta$
admits an inert natural transformation $H_{i}$ to the composite $f_{1}\eta_{i}$,
such that for each vertex $v$ in $\pr{\prod_{1\leq i\leq p}\pr{\cal M_{\act}^{\t}}_{/B_{i}}\times_{\Delta^{n}}\{0\}}^{\rcone}$,
the components $\{H_{i}\pr v\}_{1\leq i\leq p}$ form a $q$-limit
cone. It follows from Corollary \ref{cor:3.1.1.8} and (i) that $f_{1}\eta$
is an operadic $q$-colimit diagram, as desired.

\item We will extend the map $f'_{m}$ in Step 1 to a map $f''_{m}:\cal M_{F''\pr m}^{\t}\to\cal C^{\t}$
over $\cal O^{\t}$. 

We argue as in Step 1. Let $\{\sigma_{a}\}_{a\in A}$ be the set of
all nondegenerate simplices of $\cal M^{\t}\times_{\Delta^{n}}\Delta^{\{1,\dots,n\}}$
whose image in $N\pr{\Fin_{\ast}}\times\Delta^{\{1,\dots,n\}}$ is
a degeneration of an $\pr{m-1}$-simplex in $G'_{\pr 2}$ without
associates. Choose a well-ordering of the set $A$ so that $\dim\sigma_{a}$
is a non-decreasing function of $a$. For each $a\in A$, let $\cal M_{<a}^{\t}$
denote the simplicial subset spanned by $\cal M_{F'\pr m}^{\t}$ and
the simplices of $\cal M^{\t}$ whose tail factors through $\sigma_{b}$
for some $b<a$. We define $\cal M_{\leq a}^{\t}$ similarly. (The
notations $\cal M_{<a}^{\t}$ and $\cal M_{\leq a}^{\t}$ are in conflict
with the ones introduced in Step 1, but there should not be any confusion.)
By Proposition \ref{prop:heads_and_tails}, we have $\cal M_{F''\pr m}^{\t}=\bigcup_{a\in A}\cal M_{\leq a}^{\t}$,
so it suffices to extend $f'_{m}$ to an $A$-sequence $f^{\leq a}:\cal M_{\leq a}^{\t}\to\cal C^{\t}$
over $\cal O^{\t}$. The construction is inductive. Suppose $f^{\leq b}$
has been constructed for $b<a$, and let $f^{<a}:\cal M_{<a}^{\t}\to\cal C^{\t}$
be their amalgamation. Just as in Step 1, we are reduced to solving
a lifting problem of the form % https://q.uiver.app/?q=WzAsNixbMCwwLCIoXFxtYXRoY2Fse019Xlxcb3RpbWVzIF97L1xcc2lnbWFfYX1cXHRpbWVzX3tcXERlbHRhXm59IFxcezBcXH0pXFxzdGFyIFxccGFydGlhbFxcRGVsdGEgXntcXGRpbVxcc2lnbWEgX2F9Il0sWzAsMSwiKFxcbWF0aGNhbHtNfV5cXG90aW1lcyBfey9cXHNpZ21hX2F9XFx0aW1lc197XFxEZWx0YV5ufSBcXHswXFx9KVxcc3RhciBcXERlbHRhIF57XFxkaW1cXHNpZ21hIF9hfSJdLFsxLDAsIlxcbWF0aGNhbHtNfV5cXG90aW1lcyBfezxhfSJdLFsyLDAsIlxcbWF0aGNhbHtDfV5cXG90aW1lcyJdLFsyLDEsIlxcbWF0aGNhbHtPfV5cXG90aW1lcyJdLFsxLDEsIlxcbWF0aGNhbHtNfV5cXG90aW1lcyAiXSxbMCwxXSxbMCwyXSxbMiwzLCJmXns8YX0iXSxbMSw1XSxbNSw0XSxbMyw0LCJxIl0sWzEsMywiIiwxLHsic3R5bGUiOnsiYm9keSI6eyJuYW1lIjoiZGFzaGVkIn19fV1d
\[\begin{tikzcd}
	{(\mathcal{M}^\otimes _{/\sigma_a}\times_{\Delta^n} \{0\})\star \partial\Delta ^{\dim\sigma _a}} & {\mathcal{M}^\otimes _{<a}} & {\mathcal{C}^\otimes} \\
	{(\mathcal{M}^\otimes _{/\sigma_a}\times_{\Delta^n} \{0\})\star \Delta ^{\dim\sigma _a}} & {\mathcal{M}^\otimes } & {\mathcal{O}^\otimes.}
	\arrow[from=1-1, to=2-1]
	\arrow[from=1-1, to=1-2]
	\arrow["{f^{<a}}", from=1-2, to=1-3]
	\arrow[from=2-1, to=2-2]
	\arrow[from=2-2, to=2-3]
	\arrow["q", from=1-3, to=2-3]
	\arrow[dashed, from=2-1, to=1-3]
\end{tikzcd}\]The existence of such a lift follows from Corollary \ref{cor:Step2}.

\item We complete the proof by extending the map $f''_{m}$ in Step
2 to a map $f_{m}:\cal M_{F\pr m}^{\t}\to\cal C^{\t}$ over $\cal O^{\t}$.
Let $\{\sigma'_{a}\}_{a\in A}$ be the collection of all $\pr{m-1}$-simplices
in $G'_{\pr 2}$ and $G'_{\pr 3}$ which have at least one associate.
Choose a well-ordering on $A$, and for each $a\in A$, let $F_{\leq a}$
denote the simplicial subset of $F\pr m$ generated by $F''\pr m$
and the associates of the simplices $\sigma'_{b}$ for $b\leq a$.
Define $F_{<a}$ similarly. We will choose the ordering on $A$ so
that the following condition is satisfied:
\begin{itemize}
\item [($\blacklozenge$)]Let $a\in A$ and let $\sigma$ be an associate
of $\sigma'_{a}$. Let $0\leq l\leq m$ be the (unique) integer such
that $d_{l}\sigma=\sigma'_{a}$. Then for each $i\in[m]\setminus\{l\}$,
the simplex $d_{i}\sigma$ belongs to $F_{<a}$.
\end{itemize}
The existence of such an ordering is nontrivial, but we will not get
into it presently. The construction of such an ordering can be found
at the very end of this step. Note that ($\blacklozenge$) implies
that:
\begin{itemize}
\item [($\blacklozenge\blacklozenge$)]For every $a\in A$, the simplex
$\sigma'_{a}$ does not belong to $F_{<a}$.
\end{itemize}
Indeed, suppose $\sigma'_{a}$ belongs to $F_{<a}$. Choose a minimal
element $b\in A$ such that $\sigma'_{a}$ belongs to $F_{<b}$. Then
$\sigma'_{a}$ factors through one of the following simplices:
\begin{enumerate}
\item Incomplete simplices.
\item Nondegenerate simplices of dimensions less than $m-1$.
\item $\pr{m-1}$-simplices in $G_{\pr 1}\cup G_{\pr 2}\cup G_{\pr 3}$.
\item $\pr{m-1}$-simplices in $G'_{\pr 2}$ without associates.
\item Associates of $\sigma'_{c}$ for some $c<b$.
\end{enumerate}
Since $\sigma'_{a}$ is complete and nondegenerate and has dimension
$m-1$, the cases (1), (2), (3), and (4) are immediately ruled out.
We show that the case (5) is impossible by reasoning by contradiction.
Suppose that there are an index $c<b$ and an associate $\sigma$
of $\sigma'_{c}$ through which $\sigma'_{a}$ factors. We have $\sigma'_{a}=d_{i}\sigma$
for some $0\leq i\leq m$ for dimensional reasons. Using ($\blacklozenge$),
we deduce that $\sigma'_{a}$ belongs to $F_{<c}$, contrary to the
minimality of $b$.

We now get back to the construction of $f_{m}$. We will construct
$f_{m}$ as an amalgamation of an $A$-sequence $\{f_{\leq a}:\cal M_{F_{\leq a}}^{\t}\to\cal C^{\t}\}_{a\in A}$
over $\cal O^{\t}$ which extends $f''_{m}$. The construction is
inductive. Suppose that $f_{\leq b}$ has been constructed for $b\leq a$,
so that they together determine a map $f_{<a}:\cal M_{F_{<a}}^{\t}\to\cal C^{\t}$.
We must extend $f_{<a}$ to $\cal M_{F_{\leq a}}^{\t}$. We consider
two cases, depending on whether $\sigma'_{a}$ belongs to $G'_{\pr 2}$
or to $G'_{\pr 3}$.

\begin{enumerate}[label=(Case \arabic*),wide =0.5\parindent=, listparindent=1.5em]

\item Suppose that $\sigma'_{a}$ belongs to $G'_{\pr 2}$. We shall
deploy an argument which is a variant of Proposition \ref{prop:heads_and_tails}.
Let $\{\tau_{\lambda}\}_{\lambda\in\Lambda}$ be the collection of
all nondegenerate simplices of $\cal M^{\t}$ the image of whose head
in $N\pr{\Fin_{\ast}}\times\Delta^{\{1,\dots,n\}}$ is a degneration
of $\sigma'_{a}$. Choose a well-ordering of $\Lambda$ so that $\dim\tau_{\lambda}$
is a non-decreasing function of $\lambda$. For each $\lambda\in\Lambda$,
we let $\cal N_{\leq\lambda}\subset\cal M^{\t}$ denote the simplicial
subset generated by $\cal M_{F_{<a}}^{\t}$ and the simplices $\tau:\Delta^{p}\to\cal M^{\t}$
for which there is an integer $0\leq p'<p$ such that $\tau\vert\Delta^{\{0,\dots,p'\}}$
factors through some $\tau_{\mu}$ for some $\mu\leq\lambda$, $\tau\vert\Delta^{\{p',p'+1\}}$
is inert, and $\tau\vert\Delta^{\{p'+1,\dots,p\}}$ factors through
$\cal M$. We define $\cal N_{<\lambda}$ similarly. 

We shall prove the following assertion later:
\begin{itemize}
\item [($*$)]$\cal M_{F_{\leq a}}^{\t}$ is the union of $\cal M_{F_{<a}}^{\t}$
and $\{\cal N_{\leq\lambda}\}_{\lambda\in\Lambda}$. 
\end{itemize}
Accepting ($\ast$) for now, we complete the proof as follows. It
will suffice construct a $\Lambda$-sequence $\{f^{\leq\lambda}:\cal N_{\leq\lambda}\to\cal C^{\t}\}_{\lambda\in\Lambda}$
of maps over $\cal O^{\t}$ which extends $f_{<a}$. The construction
is inductive. Suppose $f^{\leq\mu}$ has been constructed for $\mu<\lambda$,
and let $f^{<\lambda}:\cal N_{<\lambda}\to\cal C^{\t}$ denote the
map obtained by amalgamating the maps $\{f^{\leq\mu}\}_{\mu<\lambda}$.
Let $\pr{\inp k,n}$ be the final vertex of $\tau_{\lambda}$. Note
that $k\geq2$. There are $k$ inert maps  $\inp k\to\inp 1$, and
these maps and $p\tau_{\lambda}$ combine to determine a diagram $\Delta^{\dim\tau_{\lambda}}\star\inp k^{\circ}\to N\pr{\Fin_{\ast}}\times\Delta^{n}$.
Let $\cal X=\pr{\Delta^{\dim\tau_{\lambda}}\star\inp k^{\circ}}\times_{N\pr{\Fin_{\ast}}\times\Delta^{n}}\cal M^{\t}$
and let $\overline{\tau}_{\lambda}:\Delta^{\dim\tau_{\lambda}}\to\cal X$
denote the induced diagram. For each $1\leq i\leq k$, let $\cal X_{i}$
denote the fiber of $\cal X$ over $i\in\inp k^{\circ}$ (which is
isomorphic to $\cal M\times_{\Delta^{n}}\{n\}$), and set $\cal X^{0}=\bigcup_{1\leq i\leq k}\cal X_{i}$
and $\cal X_{\overline{\tau}_{\lambda}/}^{0}=\cal X^{0}\times_{\cal X}\cal X_{\overline{\tau}_{\lambda}/}$.
We now consider the following diagram: % https://q.uiver.app/?q=WzAsNixbMCwwLCJcXHBhcnRpYWxcXERlbHRhXntcXGRpbVxcb3ZlcmxpbmVcXHRhdV9cXGxhbWJkYX1cXHN0YXJcXG1hdGhjYWx7WH1eMF97XFxvdmVybGluZVxcdGF1X1xcbGFtYmRhL30iXSxbMCwxLCJcXERlbHRhXntcXGRpbVxcb3ZlcmxpbmVcXHRhdV9cXGxhbWJkYX1cXHN0YXJcXG1hdGhjYWx7WH1eMF97XFxvdmVybGluZVxcdGF1X1xcbGFtYmRhL30iXSxbMSwwLCJLXzAiXSxbMSwxLCJLIl0sWzIsMCwiXFxtYXRoY2Fse059X3s8XFxsYW1iZGF9Il0sWzIsMSwiXFxtYXRoY2Fse059X3tcXGxlcSBcXGxhbWJkYX0iXSxbMCwxXSxbMCwyXSxbMSwzXSxbMiwzXSxbMiw0XSxbMyw1XSxbNCw1XV0=
\[\begin{tikzcd}
	{\partial\Delta^{\dim\overline\tau_\lambda}\star\mathcal{X}^0_{\overline\tau_\lambda/}} & {K_0} & {\mathcal{N}_{<\lambda}} \\
	{\Delta^{\dim\overline\tau_\lambda}\star\mathcal{X}^0_{\overline\tau_\lambda/}} & K & {\mathcal{N}_{\leq \lambda}.}
	\arrow[from=1-1, to=2-1]
	\arrow[from=1-1, to=1-2]
	\arrow[from=2-1, to=2-2]
	\arrow[from=1-2, to=2-2]
	\arrow[from=1-2, to=1-3]
	\arrow[from=2-2, to=2-3]
	\arrow[from=1-3, to=2-3]
\end{tikzcd}\]Here $K\subset\cal X$ denotes the simplicial subset spanned by the
simplices whose tail factors through $\overline{\tau}_{\lambda}$,
and $K_{0}$ is its simplicial subset obtained by replacing $\overline{\tau}_{\lambda}$
by $\partial\overline{\tau}_{\lambda}$, where we regard $\cal X$
as a simplicial set over $\Delta^{1}$ by the map $\Delta^{\dim\tau_{\lambda}}\star\inp k^{\circ}\to\{0\}\star\{1\}=\Delta^{1}$.
The left hand square is homotopy cocartesian by Lemma \ref{lem:3.1.2.5}.
We shall prove later that:
\begin{itemize}
\item [($**$)]The horizontal arrows of the the right hand square are well-defined,
and the right hand square is cocartesian.
\end{itemize}
We are now reduced to solving the lifting problem % https://q.uiver.app/?q=WzAsNixbMCwwLCJcXHBhcnRpYWxcXERlbHRhXntcXGRpbVxcb3ZlcmxpbmVcXHRhdV9cXGxhbWJkYX1cXHN0YXJcXG1hdGhjYWx7WH1eMF97XFxvdmVybGluZVxcdGF1X1xcbGFtYmRhL30iXSxbMCwxLCJcXERlbHRhXntcXGRpbVxcb3ZlcmxpbmVcXHRhdV9cXGxhbWJkYX1cXHN0YXJcXG1hdGhjYWx7WH1eMF97XFxvdmVybGluZVxcdGF1X1xcbGFtYmRhL30iXSxbMSwwLCJcXG1hdGhjYWx7Tn1fezxcXGxhbWJkYX0iXSxbMSwxLCJcXG1hdGhjYWx7Tn1fe1xcbGVxIFxcbGFtYmRhfSJdLFsyLDAsIlxcbWF0aGNhbHtDfV5cXG90aW1lcyAiXSxbMiwxLCJcXG1hdGhjYWx7T31eXFxvdGltZXMgLiJdLFswLDFdLFsyLDRdLFszLDVdLFs0LDVdLFsxLDQsIiIsMSx7InN0eWxlIjp7ImJvZHkiOnsibmFtZSI6ImRhc2hlZCJ9fX1dLFswLDJdLFsxLDNdXQ==
\[\begin{tikzcd}
	{\partial\Delta^{\dim\overline\tau_\lambda}\star\mathcal{X}^0_{\overline\tau_\lambda/}} & {\mathcal{N}_{<\lambda}} & {\mathcal{C}^\otimes } \\
	{\Delta^{\dim\overline\tau_\lambda}\star\mathcal{X}^0_{\overline\tau_\lambda/}} & {\mathcal{N}_{\leq \lambda}} & {\mathcal{O}^\otimes .}
	\arrow[from=1-1, to=2-1]
	\arrow[from=1-2, to=1-3]
	\arrow[from=2-2, to=2-3]
	\arrow[from=1-3, to=2-3]
	\arrow[dashed, from=2-1, to=1-3]
	\arrow[from=1-1, to=1-2]
	\arrow[from=2-1, to=2-2]
\end{tikzcd}\]Now the $\infty$-category $\cal X_{\overline{\tau}_{\lambda}/}^{0}$
is the disjoint union of the $\infty$-categories $\pr{\cal X_{i}}_{\overline{\tau}_{\lambda}/}=\cal X_{i}\times_{\cal X}\cal X_{\overline{\tau}_{\lambda}/}$.
Each $\infty$-category $\pr{\cal X_{i}}_{\overline{\tau}_{\lambda}/}$
has an initial object, given by a cone $\phi_{i}:\pr{\Delta^{\dim\overline{\tau}_{\lambda}}}^{\rcone}\to\cal X$
which maps the last edge to an inert morphism over $\pr{\rho^{i},\id}:\pr{\inp k,n}\to\pr{\inp 1,n}$.
Set $S=\coprod_{i}\{\phi_{i}\}$. The inclusion $S\subset\cal X_{\overline{\tau}_{\lambda}/}^{0}$
is initial, so we are reduced to solving the lifting problem % https://q.uiver.app/?q=WzAsOCxbMSwwLCJcXHBhcnRpYWxcXERlbHRhXntcXGRpbVxcb3ZlcmxpbmVcXHRhdV9cXGxhbWJkYX1cXHN0YXJcXG1hdGhjYWx7WH1eMF97XFxvdmVybGluZVxcdGF1X1xcbGFtYmRhL30iXSxbMSwxLCJcXERlbHRhXntcXGRpbVxcb3ZlcmxpbmVcXHRhdV9cXGxhbWJkYX1cXHN0YXJcXG1hdGhjYWx7WH1eMF97XFxvdmVybGluZVxcdGF1X1xcbGFtYmRhL30iXSxbMiwwLCJcXG1hdGhjYWx7Tn1fezxcXGxhbWJkYX0iXSxbMiwxLCJcXG1hdGhjYWx7Tn1fe1xcbGVxIFxcbGFtYmRhfSJdLFszLDAsIlxcbWF0aGNhbHtDfV5cXG90aW1lcyAiXSxbMywxLCJcXG1hdGhjYWx7T31eXFxvdGltZXMgLiJdLFswLDAsIlxccGFydGlhbFxcRGVsdGFee1xcZGltXFxvdmVybGluZVxcdGF1X1xcbGFtYmRhfVxcc3RhciBTIl0sWzAsMSwiXFxEZWx0YV57XFxkaW1cXG92ZXJsaW5lXFx0YXVfXFxsYW1iZGF9XFxzdGFyIFMiXSxbMiw0XSxbMyw1XSxbNCw1XSxbMCwyXSxbMSwzXSxbNiwwXSxbNywxXSxbNyw0LCIiLDEseyJzdHlsZSI6eyJib2R5Ijp7Im5hbWUiOiJkYXNoZWQifX19XSxbNiw3XV0=
\[\begin{tikzcd}
	{\partial\Delta^{\dim\overline\tau_\lambda}\star S} & {\partial\Delta^{\dim\overline\tau_\lambda}\star\mathcal{X}^0_{\overline\tau_\lambda/}} & {\mathcal{N}_{<\lambda}} & {\mathcal{C}^\otimes } \\
	{\Delta^{\dim\overline\tau_\lambda}\star S} & {\Delta^{\dim\overline\tau_\lambda}\star\mathcal{X}^0_{\overline\tau_\lambda/}} & {\mathcal{N}_{\leq \lambda}} & {\mathcal{O}^\otimes .}
	\arrow[from=1-3, to=1-4]
	\arrow[from=2-3, to=2-4]
	\arrow[from=1-4, to=2-4]
	\arrow[from=1-2, to=1-3]
	\arrow[from=2-2, to=2-3]
	\arrow[from=1-1, to=1-2]
	\arrow[from=2-1, to=2-2]
	\arrow[dashed, from=2-1, to=1-4]
	\arrow[from=1-1, to=2-1]
\end{tikzcd}\] If the dimension of $\overline{\tau}_{\lambda}$ is positive, then
the claim is immediate since the restriction of the top horizontal
arrow to $\{\dim\overline{\tau}_{\lambda}\}\star S$ is a $q$-limit
cone. If $\overline{\tau}_{\lambda}$ is zero-dimensional (in which
case $m=n=1$), let $C_{i}$ denote the image of $\phi_{i}\in S$
under the top horizontal map. The bottom horizontal arrow classifies
a diagram $X\to q\pr{C_{i}}$ of inert maps $\{\alpha_{i}:X\to q\pr{C_{i}}\}_{1\leq i\leq k}$
lying over $\{\rho^{i}:\inp k\to\inp 1\}_{1\leq i\leq k}$, and we
wish to lift this to a diagram $S^{\lcone}\to\cal C^{\t}$ in $\cal C^{\t}$
which maps each $\phi_{i}\in S$ to the object $C_{i}$. Since $q$
is a fibration of $\infty$-operads, we can in fact find such a lift
consisting of inert morphisms. Note that with such a choice of lift,
condition (ii) is satisfied.

We now move on to the verification of ($\ast$) and ($\ast\ast$).
We begin with ($\ast$). It is clear that $\cal N_{\leq\lambda}$
and $\cal M_{F_{<a}}^{\t}$ are contained in $\cal M_{F_{\leq a}}^{\t}$.
For the reverse implication, let $x$ be an arbitrary simplex of $\cal M_{F_{\leq a}}^{\t}$.
We must show that $x$ belongs to either $\cal M_{F_{<a}}^{\t}$ or
one of the $\cal N_{\leq\lambda}$'s. If $x$ belongs to $\cal M_{F_{<a}}^{\t}$,
we are done. So assume not. Let $r$ denote the dimension of $x$.
Since $x$ does not belong to $\cal M_{F_{<a}}^{\t}$, its head is
nonempty. Find an integer $0\leq r'\leq r$ such that $x\vert\Delta^{\{r',\dots,r\}}$
is the head of $x$. Since $x$ does not belong to $\cal M_{F_{<a}}^{\t}$,
the simplex $px\vert\Delta^{\{r',\dots,r\}}$ factors through an associate
$\sigma$ of $\sigma'_{a}$. Let $\Delta^{r}\xrightarrow{u}\Delta^{m}\xrightarrow{\sigma}N\pr{\Fin_{\ast}}\times\Delta^{\{1,\dots,n\}}$
be such a factorization. If the image of $u$ does not contain some
integer $i\in\{0,\dots,m-1\}$, then $px\vert\Delta^{\{r',\dots,r\}}$
factors through $d_{i}\sigma$ and hnece through $F_{<a}$ by ($\blacklozenge$),
a contradiction. So the image of $u$ contains every integer in $\{0,\dots,m-1\}$.
Let $0\leq r''<r$ be the largest integer such that $u\pr{r''}<m$.
There are now several cases to consider:
\begin{itemize}
\item Suppose that $u$ is surjective on vertices and that the map $x\pr{r''}\to x\pr{r''+1}$
is inert. We write $x\vert\Delta^{\{0,\dots,r''\}}=s^{*}y$, where
$s:[r']\to[k]$ is a surjection and $y$ is nondegenerate. We claim
that $y$ is one of the $\tau_{\lambda}$'s, so that $x$ belongs
to $\cal N_{\leq\lambda}$. Let $k'=s\pr{r'}$. Then $y\vert\Delta^{\{k',\dots,k\}}$
is the head of $y$. We write $py\vert\Delta^{\{k',\dots,k\}}=s^{\p\ast}z$,
where $z$ is nondegenerate and $s'$ is a surjection. Then we obtain
the following commutative diagram: % https://q.uiver.app/?q=WzAsOCxbMSwwLCJcXERlbHRhXntcXHtyJyxcXGRvdHMscicnXFx9fSJdLFsyLDAsIlxcRGVsdGFee20tMX0iXSxbMSwxLCJcXG1hdGhjYWx7TX1eXFxvdGltZXMgIl0sWzIsMSwiTihcXG1hdGhzZntGaW59X1xcYXN0KVxcdGltZXMgXFxEZWx0YV5uLiJdLFswLDAsIlxcRGVsdGFee1xcezAsXFxkb3RzLHInJ1xcfX0iXSxbMSwyLCJcXERlbHRhXmwiXSxbMCwyLCJcXERlbHRhXntcXHtrJyxcXGRvdHNtLGtcXH19Il0sWzAsMSwiXFxEZWx0YV57a30iXSxbMCwxLCJ1XFx2ZXJ0XFxEZWx0YV57XFx7cicsXFxkb3RzLHInJ1xcfX0iLDAseyJzdHlsZSI6eyJoZWFkIjp7Im5hbWUiOiJlcGkifX19XSxbMSwzLCJcXHNpZ21hX2EnIl0sWzIsMywicCIsMl0sWzQsNywicyIsMix7InN0eWxlIjp7ImhlYWQiOnsibmFtZSI6ImVwaSJ9fX1dLFs3LDIsInkiLDJdLFswLDQsIiIsMCx7InN0eWxlIjp7InRhaWwiOnsibmFtZSI6Imhvb2siLCJzaWRlIjoiYm90dG9tIn19fV0sWzYsNywiIiwyLHsic3R5bGUiOnsidGFpbCI6eyJuYW1lIjoiaG9vayIsInNpZGUiOiJib3R0b20ifX19XSxbNiw1LCJzJyIsMix7InN0eWxlIjp7ImhlYWQiOnsibmFtZSI6ImVwaSJ9fX1dLFs2LDJdLFs1LDMsInoiLDJdLFs0LDIsInhcXHZlcnRcXERlbHRhXntcXHswLFxcZG90cyxyJydcXH19IiwxXV0=
\[\begin{tikzcd}
	{\Delta^{\{0,\dots,r''\}}} & {\Delta^{\{r',\dots,r''\}}} & {\Delta^{m-1}} \\
	{\Delta^{k}} & {\mathcal{M}^\otimes } & {N(\mathsf{Fin}_\ast)\times \Delta^n.} \\
	{\Delta^{\{k',\dotsm,k\}}} & {\Delta^l}
	\arrow["{u\vert\Delta^{\{r',\dots,r''\}}}", two heads, from=1-2, to=1-3]
	\arrow["{\sigma_a'}", from=1-3, to=2-3]
	\arrow["p"', from=2-2, to=2-3]
	\arrow["s"', two heads, from=1-1, to=2-1]
	\arrow["y"', from=2-1, to=2-2]
	\arrow[hook', from=1-2, to=1-1]
	\arrow[hook', from=3-1, to=2-1]
	\arrow["{s'}"', two heads, from=3-1, to=3-2]
	\arrow[from=3-1, to=2-2]
	\arrow["z"', from=3-2, to=2-3]
	\arrow["{x\vert\Delta^{\{0,\dots,r''\}}}"{description}, from=1-1, to=2-2]
\end{tikzcd}\]Applying Eilenberg-Zilber's lemma to $px\vert\Delta^{\{r',\dots,r''\}}$,
we deduce that $z=\sigma'_{a}$. Thus $y$ is one of the simplices
in $\{\tau_{\lambda}\}_{\lambda\in\Lambda}$, as required.
\item Suppose that $u$ is surjective on vertices and that the map $\theta:u\pr{r''}\to u\pr{r''+1}$
is not inert. By factoring the map $\theta$ into an inert map followed
by an active map, we can find an $\pr{r+1}$-simplex $y$ of $\cal M_{F_{\leq a}}^{\t}$
such that $d_{r''+1}y=x$ and $y\vert\Delta^{\{r'',r''+1,r''+2\}}$
is the chosen factorization of $\theta$. By the previous point, the
simplex $y$ belongs to $\cal N_{\leq\lambda}$ for some $\lambda$,
and hence so must $x$.
\item Suppose $u$ is not surjective on vertices. Find an inert map $\theta:u\pr r\to X$
over the last edge of $\sigma$, and let $y$ be an $\pr{r+1}$-simplex
$y$ of $\cal M_{F_{\leq a}}^{\t}$ such that $d_{r+1}y=x$ and $y\vert\Delta^{\{r,r+1\}}$
is equal to $\theta$. By the first point, the simplex $y$ belongs
to $\cal N_{\leq\lambda}$ for some $\lambda$, and hence so must
$x$.
\end{itemize}
This completes the proof of ($\ast$).

Next we prove ($\ast\ast$). First we show that the maps $K\to\cal N_{\leq\lambda}$
and $K_{0}\to\cal N_{<\lambda}$ are well-defined. A typical simplex
in the image of the map $K\to\cal M^{\t}$ has the form $x:\Delta^{r}\to\cal M^{\t}$,
where there is an integer $-1\leq r'\leq r$ such that $x\vert\Delta^{\{0,\dots,r'\}}$
factors through $\tau_{\lambda}$ and, if $r'<r$, then $p\pr{x\vert\Delta^{\{r',r'+1\}}}$
is inert and and $p\pr{x\vert\Delta^{\{r'+1,\dots,r\}}}$ is the constant
map at $\pr{\inp 1,n}$. (When $r'=-1$, the symbol $\Delta^{\{0,\dots,r'\}}$
denotes the empty simplicial set.) By choosing an inert map $x\pr{r'}\to X$
with $X\in\cal M$ if $r'<r$, or by factoring the map $x\pr{r'}\to x\pr{r'+1}$
into an inert map followed by an active map, we can find a simplex
$y$ belonging to $\cal N_{\leq\lambda}$ and satisfying $d_{r'+1}y=x$.
Hence $x$ belongs to $\cal N_{\leq\lambda}$, showing that the map
$K\to\cal N_{\leq\lambda}$ is well-defined. Next, for the map $K_{0}\to\cal N_{<\lambda}$,
assume further $x\vert\Delta^{\{0,\dots,r'\}}$ factors through the
boundary of $\tau_{\lambda}$. We must show that $x$ belongs to $\cal N_{<\lambda}$.
We have two cases to consider:
\begin{itemize}
\item Suppose that the head of the simplex $p\pr{x\vert\Delta^{\{0,\dots,r'\}}}$
factors through $\partial\sigma'_{a}$. Then the head of $p\pr x$
factors through $d_{i}\sigma$ for some $0\leq i<m$ for some associate
$\sigma$ of $\sigma'_{a}$, so $x$ belongs to $\cal M_{F_{<a}}^{\t}$
by ($\blacklozenge$). 
\item Suppose that the head of the simplex $p\pr{x\vert\Delta^{\{0,\dots,r'\}}}$
is a degeneration of $\sigma'_{a}$. Write $x\vert\Delta^{\{0,\dots,r'\}}=s^{*}y$,
where $y$ is nondegenerate and $s$ is a surjection. Eilenberg-Zilber's
lemma implies that the simplex $p\pr y$ is again a degeneration of
$\sigma'_{a}$, so $y=\tau_{\mu}$ for some $\mu\in\Lambda$. Since
$x$ factors through the boundary of $\tau_{\lambda}$, the dimension
of $y$ must be smaller than that of $\tau_{\lambda}$. Thus $\mu<\lambda$.
The above argument (that the map $K\to\cal N_{\leq\lambda}$ is well-defined)
shows that $y$ belongs to $\cal N_{\leq\mu}\subset\cal N_{<\lambda}$,
so $x$ belongs to $\cal N_{<\lambda}$.
\end{itemize}
This completes the verification of the first half of ($\ast\ast$).
We next proceed to the latter half: We show that the square % https://q.uiver.app/?q=WzAsNCxbMCwwLCJLXzAiXSxbMCwxLCJLIl0sWzEsMCwiXFxtYXRoY2Fse059X3s8XFxsYW1iZGF9Il0sWzEsMSwiXFxtYXRoY2Fse059X3tcXGxlcSBcXGxhbWJkYX0iXSxbMCwxXSxbMCwyXSxbMSwzXSxbMiwzXV0=
\[\begin{tikzcd}
	{K_0} & {\mathcal{N}_{<\lambda}} \\
	K & {\mathcal{N}_{\leq \lambda}}
	\arrow[from=1-1, to=2-1]
	\arrow[from=1-1, to=1-2]
	\arrow[from=2-1, to=2-2]
	\arrow[from=1-2, to=2-2]
\end{tikzcd}\]is cocartesian. Clearly $\cal N_{\leq\lambda}$ is the union of the
images of $K$ and $\cal N_{<\lambda}$. So it suffices to show that
if a simplex $z$ of $K$ is mapped into $\cal N_{<\lambda}$, then
$z$ belongs to $K_{0}$. Taking the contrapositive, we will show
that if $z$ does not belong to $K_{0}$, then its image in $\cal N_{\leq\lambda}$
does not belong to $\cal N_{<\lambda}$. Let $x:\Delta^{r}\to\cal N_{\leq\lambda}$
be the image of $x$. By the definition of the map $K\to\cal N_{\leq\lambda}$
and by the hypothesis that $z$ does not belong to $K_{0}$, there
is an integer $0\leq r'\leq r$ such that $x\vert\Delta^{\{0,\dots,r'\}}$
is a degeneration of $\tau_{\lambda}$ and, if $r'<r$, then $p\pr{x\vert\Delta^{\{r',r'+1\}}}$
is inert and and $p\pr{x\vert\Delta^{\{r'+1,\dots,r\}}}$ is the constant
map at $\pr{\inp 1,n}$. The head of $p\pr x$ is thus a degeneration
of either $\sigma'_{a}$ or its associate, and so it does not belong
to $F_{<a}$ by ($\blacklozenge\blacklozenge$). Therefore, $x$ does
not belong to $\cal M_{F_{<a}}^{\t}$. So should $x$ belong to $\cal N_{<\lambda}$,
then $x\vert\Delta^{\{0,\dots,r'\}}$ must factor through $\tau_{\mu}$
for some $\mu<\lambda$. This is impossible because $x\vert\Delta^{\{0,\dots,r'\}}$
is a degeneration of $\tau_{\lambda}$. Hence $x$ does not belong
to $\cal N_{<\lambda}$.

\item Suppose that $\sigma'_{a}$ belongs to $G'_{\pr 3}$. We will
show that the inclusion $\cal M_{<a}^{\t}\hookrightarrow\cal M_{\leq a}^{\t}$
is a weak categorical equivalence. The desired extension of $f_{<a}$
can then be found because the map $q$ is a categorical fibration.

Let $\sigma$ be the unique associate of $\sigma'_{a}$, which we
depict as 
\[
\pr{\inp{k_{0}},e_{0}}\xrightarrow{\alpha_{\sigma}\pr 1}\cdots\xrightarrow{\alpha_{\sigma}\pr m}\pr{\inp{k_{m}},e_{m}}.
\]
Choose integers $0<j<k\leq m$ such that $\alpha_{\sigma}\pr i$ is
strongly inert for $i=j$, active for $j<i\leq k$, and strongly inert
for $i>k$. Set $Y=\pr{N\pr{\Fin_{\ast}}\times\Delta^{n}}_{/\sigma}\times_{\Delta^{n}}\{0\}$.
Using ($\blacklozenge$), we may consider the following commutative
diagram: % https://q.uiver.app/?q=WzAsNCxbMCwwLCJZXFxzdGFyIFxcTGFtYmRhIF5tX2oiXSxbMCwxLCJZXFxzdGFyIFxcRGVsdGEgXm0iXSxbMSwwLCJcXG92ZXJsaW5le0ZfezxhfX0iXSxbMSwxLCJcXG92ZXJsaW5le0Zfe1xcbGVxIGF9fSJdLFswLDFdLFswLDJdLFsxLDNdLFsyLDNdXQ==
\[\begin{tikzcd}
	{Y\star \Lambda ^m_j} & {\overline{F_{<a}}} \\
	{Y\star \Delta ^m} & {\overline{F_{\leq a}}.}
	\arrow[from=1-1, to=2-1]
	\arrow[from=1-1, to=1-2]
	\arrow[from=2-1, to=2-2]
	\arrow[from=1-2, to=2-2]
\end{tikzcd}\]Using ($\blacklozenge\blacklozenge$), we deduce that this square
is cocartesian. Therefore, it suffices to show that the inclusion
\[
\pr{Y\star\Lambda_{j}^{m}}\times_{N\pr{\Fin_{\ast}}\times\Delta^{n}}\cal M^{\t}\to\pr{Y\star\Delta^{m}}\times_{N\pr{\Fin_{\ast}}\times\Delta^{n}}\cal M^{\t}
\]
is a weak categorical equivalence. We will prove more generally that
for any morphism $Y'\to Y$ of simplicial sets, the map 
\[
\eta_{Y'}:\pr{Y'\star\Lambda_{j}^{m}}\times_{N\pr{\Fin_{\ast}}\times\Delta^{n}}\cal M^{\t}\to\pr{Y'\star\Delta^{m}}\times_{N\pr{\Fin_{\ast}}\times\Delta^{n}}\cal M^{\t}
\]
is a weak categorical equivalence. The assignment $Y'\mapsto\eta_{Y'}$
defines a functor from $\SS/Y'$ to the arrow category $\SS^{[1]}$
which commutes with filtered colimits. Since weak categorical equivalences
are stable under filtered colimits, we may assume that $Y'$ is a
finite simplicial set. If $Y'$ is empty, the claim follows from \cite[Lemma 2.4.4.6]{HA}.
For the inductive step, we can find a pushout diagram % https://q.uiver.app/?q=WzAsNCxbMCwwLCJcXHBhcnRpYWxcXERlbHRhXnAiXSxbMSwwLCJYIl0sWzEsMSwiWSciXSxbMCwxLCJcXERlbHRhXnAiXSxbMCwxXSxbMSwyXSxbMCwzXSxbMywyXV0=
\[\begin{tikzcd}
	{\partial\Delta^p} & X \\
	{\Delta^p} & {Y'}
	\arrow[from=1-1, to=1-2]
	\arrow[from=1-2, to=2-2]
	\arrow[from=1-1, to=2-1]
	\arrow[from=2-1, to=2-2]
\end{tikzcd}\]in $\SS/Y$, such that the claim holds for $X$. The map $\eta_{Y'}$
factors as
\begin{align*}
\pr{Y'\star\Lambda_{j}^{m}}\times_{N\pr{\Fin_{\ast}}\times\Delta^{n}}\cal M^{\t} & \xrightarrow{\phi}\pr{Y'\star\Lambda_{j}^{m}\cup X\star\Delta^{m}}\times_{N\pr{\Fin_{\ast}}\times\Delta^{n}}\cal M^{\t}\\
 & \xrightarrow{\psi}\pr{Y'\star\Delta^{m}}\times_{N\pr{\Fin_{\ast}}\times\Delta^{n}}\cal M^{\t}.
\end{align*}
The map $\phi$ is a pushout of $\eta_{X}$, and hence is a trivial
cofibration. The map $\psi$ is a pushout of the inclusion
\[
\psi':\pr{\Delta^{p}\star\Lambda_{j}^{m}\cup\partial\Delta^{p}\star\Delta^{m}}\times_{N\pr{\Fin_{\ast}}\times\Delta^{n}}\cal M^{\t}\to\pr{\Delta^{p}\star\Delta^{m}}\times_{N\pr{\Fin_{\ast}}\times\Delta^{n}}\cal M^{\t}.
\]
Using the isomorphism $\Delta^{p}\star\Lambda_{j}^{m}\cup\partial\Delta^{p}\star\Delta^{m}\cong\Lambda_{p+1+j}^{m+p+1}$
and Lemma \cite[Lemma 2.4.4.6]{HA}, we deduce that $\psi'$ is a
weak categorical equivalence. Hence $\psi$ is a weak categorical
equivalence, completing the treatment of Case 2.

\end{enumerate}

It remains to construct a well-ordering on $A$ which satisfies ($\blacklozenge$).
For each $a\in A$, define integers $u_{\mathrm{neut}}\pr a,u_{\act}\pr a,u_{\mathrm{oc}}\pr a,u_{\mathrm{as}}\pr a$
as follows: Let $\sigma$ be an associate of $\sigma'_{a}$, and set
$\alpha_{\sigma}\pr i=\sigma\vert\Delta^{\{i-1,i\}}$ for $1\leq i\leq m$.
Then:
\begin{itemize}
\item $u_{\mathrm{neut}}\pr a$ is the number of integers $1\leq i\leq m$
such that $\alpha_{\sigma}\pr i$ is neutral.
\item $u_{\act}\pr a$ is the number of integers $1\leq i\leq m$ such that
$\alpha_{\sigma}\pr i$ is active.
\item $u_{\mathrm{oc}}\pr a$ is set equal to $0$ if $\sigma$ is closed,
and is set equal to $1$ if $\sigma$ is open.
\item $u_{\mathrm{as}}\pr a$ is the number of pairs of integers $1\leq i<j\leq m$
such that $\alpha_{\sigma}\pr i$ is active and $\alpha_{\sigma}\pr j$
is strictly inert.
\end{itemize}
We choose a well-ordering on $A$ so that the function 
\[
A\ni a\mapsto\pr{u_{\mathrm{neut}}\pr a,u_{\act}\pr a,u_{\mathrm{oc}}\pr a,u_{\mathrm{as}}\pr a}\in\bb Z^{4}
\]
is non-decreasing, where $\bb Z^{4}$ is equipped with the lexicographic
ordering. (Thus $\pr{n_{1},n_{2},n_{3},n_{4}}<\pr{n'_{1},n'_{2},n'_{3},n'_{4}}$
if and only if $n_{i}<n'_{i}$, where $i$ is the minimal integer
for which $n_{i}\neq n'_{i}$.) We will show that this ordering does
the job.

Let $a,\sigma,i$ be as in ($\blacklozenge$). We wish to show that
$d_{i}\sigma$ belongs to $F_{<a}$. If $d_{i}\sigma$ is incomplete
or degenerate, then it belongs to $F\pr{m-1}$ and we are done. So
assume that $d_{i}\sigma$ is complete and nondegenerate. Then $d_{i}\sigma$
belongs to (exactly) one of the sets $G_{\pr 1},G_{\pr 2},G'_{\pr 2},G_{\pr 3},G'_{\pr 3}$.
If $d_{i}\sigma$ belongs to $G_{\pr 1}\cup G_{\pr 2}\cup G_{\pr 3}$,
or if it belongs to $G'_{\pr 2}$ and has no associates, then it belongs
to $F''\pr m$ and we are done. So we will assume that $d_{i}\sigma$
belongs to $G'_{\pr 2}\cup G'_{\pr 3}$ and has an associate, so that
$d_{i}\sigma=\sigma'_{b}$ for some $b\in A$. We wish to show that
$b<a$.

We will make use of the following notations. Let $\alpha_{\sigma}\pr i=\sigma\vert\Delta^{\{i-1,i\}}$.
Let $0\leq k\leq m$ be the minimal integer such that $\alpha_{\sigma}\pr i$
is strongly inert for every $i>k$, and let $0\leq j\leq k$ be the
minimal integer such that $\alpha_{\sigma}\pr i$ is active for every
$j<i\leq k$. 

Suppose first that $\sigma'_{a}\in G'_{\pr 2}$. Our assumption on
$d_{i}\sigma$ implies that $i=k$ and that the composite $\alpha_{\sigma}\pr{k+1}\circ\alpha_{\sigma}\pr k$
is neutral. It follows that the associate $\tau$ of $d_{i}\sigma$
satisfies $\alpha_{\tau}\pr s=\alpha_{\sigma}\pr s$ for $s\neq k,k+1$,
$\alpha_{\tau}\pr i$ is strictly inert, and $\alpha_{\tau}\pr{k+1}$
is active. Thus $u_{\mathrm{neut}}\pr a=u_{\mathrm{neut}}\pr b$,
$u_{\act}\pr a=u_{\act}\pr b$, $u_{\mathrm{oc}}\pr a=u_{\mathrm{oc}}\pr b$,
and $u_{\mathrm{as}}\pr a>u_{\mathrm{as}}\pr b$. Hence $a>b$, as
required.

Suppose next that $\sigma'_{a}\in G'_{\pr 3}$ and that $d_{i}\sigma\in G'_{\pr 2}$. 
\begin{itemize}
\item If $u_{\mathrm{neut}}\pr a>0$, we are done, since $u_{\mathrm{neut}}\pr b=0$.
\item If $u_{\mathrm{neut}}\pr a=0$, then each $\alpha_{\sigma}\pr i$
is either active or strongly inert. It follows that $u_{\act}\pr a$
is not less than the number of active morphisms in $d_{i}\sigma$,
which is equal to $u_{\act}\pr b$. So $u_{\act}\pr a\geq u_{\act}\pr b$.
If $u_{\act}\pr a>u_{\act}\pr b$, we are done. 
\item If $u_{\mathrm{neut}}\pr a=0$ and $u_{\act}\pr a=u_{\act}\pr b$,
then $\sigma$ is open. Indeed, if $\sigma$ were closed, then $i=m$
since $d_{i}\sigma$ is open. But then $d_{i}\sigma$ must contain
a subsequence consisting of a strictly inert morphism followed by
an active morphism. This is impossible because $d_{i}\sigma$ belongs
to $G'_{\pr 2}$. Hence $u_{\mathrm{oc}}\pr a>u_{\mathrm{oc}}\pr b$
and we are done.
\end{itemize}
Finally, suppose that $\sigma'_{a}\in G'_{\pr 3}$ and that $d_{i}\sigma\in G'_{\pr 3}$.
Note that our assumption on $d_{i}\sigma$ forces $i=j-1$ or $i=k$.
\begin{itemize}
\item The number of neutral morphisms in $d_{i}\sigma$ is at most $u_{\mathrm{neut}}\pr a+1$,
so 
\begin{equation}
u_{\mathrm{neut}}\pr a\geq\#\{\text{neutral morphisms in }d_{i}\sigma\}-1=u_{\mathrm{neut}}\pr b.\label{eq:1}
\end{equation}
If $u_{\mathrm{neut}}\pr a>u_{\mathrm{neut}}\pr b$, we are done. 
\item Suppose $u_{\mathrm{neut}}\pr a=u_{\mathrm{neut}}\pr b$, so that
the equality holds in (\ref{eq:1}). Then $0<i<m$, the composite
$\alpha_{\sigma}\pr{i+1}\circ\alpha_{\sigma}\pr i$ is neutral, and
$\alpha_{\sigma}\pr i$ is active. ($\alpha_{\sigma}\pr i$ is necessarily
strictly inert since $i=j-1$ or $i=k$.) It follows that the associate
$\tau$ of $d_{i}\sigma$ satisfies $\alpha_{\tau}\pr s=\alpha_{\sigma}\pr s$
for $s\neq i,i+1$, $\alpha_{\tau}\pr i$ is strictly inert, and $\alpha_{\tau}\pr{i+1}$
is active. Thus $u_{\mathrm{neut}}\pr a=u_{\mathrm{neut}}\pr b$,
$u_{\act}\pr a=u_{\act}\pr b$, $u_{\mathrm{oc}}\pr a=u_{\mathrm{oc}}\pr b$,
and $u_{\mathrm{as}}\pr a>u_{\mathrm{as}}\pr b$. Hence $a>b$, as
desired.
\end{itemize}
This completes the proof of ($\blacklozenge$) and hence the proof
of the theorem.

\end{enumerate}
\end{proof}
%% LyX 2.3.6.2 created this file.  For more info, see http://www.lyx.org/.
%% Do not edit unless you really know what you are doing.
\documentclass[a4paper]{amsart}
\usepackage[T1]{fontenc}
\usepackage{amstext}
\usepackage{amsthm}
\usepackage{amssymb}
\usepackage{stackrel}
\usepackage[unicode=true,pdfusetitle,
 bookmarks=true,bookmarksnumbered=false,bookmarksopen=false,
 breaklinks=false,pdfborder={0 0 0},pdfborderstyle={},backref=false,colorlinks=false]
 {hyperref}

\makeatletter

%%%%%%%%%%%%%%%%%%%%%%%%%%%%%% LyX specific LaTeX commands.
\special{papersize=\the\paperwidth,\the\paperheight}


%%%%%%%%%%%%%%%%%%%%%%%%%%%%%% Textclass specific LaTeX commands.
\numberwithin{equation}{section}
\numberwithin{figure}{section}
\theoremstyle{plain}
\newtheorem{thm}{\protect\theoremname}[section]
\theoremstyle{plain}
\newtheorem{prop}[thm]{\protect\propositionname}
\theoremstyle{remark}
\newtheorem{rem}[thm]{\protect\remarkname}
\ifx\proof\undefined
\newenvironment{proof}[1][\protect\proofname]{\par
	\normalfont\topsep6\p@\@plus6\p@\relax
	\trivlist
	\itemindent\parindent
	\item[\hskip\labelsep\scshape #1]\ignorespaces
}{%
	\endtrivlist\@endpefalse
}
\providecommand{\proofname}{Proof}
\fi
\theoremstyle{plain}
\newtheorem{cor}[thm]{\protect\corollaryname}
\theoremstyle{plain}
\newtheorem{lem}[thm]{\protect\lemmaname}
\theoremstyle{definition}
\newtheorem{defn}[thm]{\protect\definitionname}

%%%%%%%%%%%%%%%%%%%%%%%%%%%%%% User specified LaTeX commands.
\usepackage{tikz-cd}
\usepackage{mathtools}
\usepackage{adjustbox}
\usepackage{enumitem}
\tikzcdset{scale cd/.style={every label/.append style={scale=#1},
    cells={nodes={scale=#1}}}}
\usepackage{eucal}

\makeatother


\providecommand{\corollaryname}{Corollary}
\providecommand{\definitionname}{Definition}
\providecommand{\lemmaname}{Lemma}
\providecommand{\propositionname}{Proposition}
\providecommand{\remarkname}{Remark}
\providecommand{\theoremname}{Theorem}

\begin{document}
\global\long\def\sf#1{\mathsf{#1}}%

\global\long\def\scr#1{\mathscr{{#1}}}%

\global\long\def\cal#1{\mathcal{#1}}%

\global\long\def\bb#1{\mathbb{#1}}%

\global\long\def\bf#1{\mathbf{#1}}%

\global\long\def\frak#1{\mathfrak{#1}}%

\global\long\def\fr#1{\mathfrak{#1}}%

\global\long\def\u#1{\underline{#1}}%

\global\long\def\tild#1{\widetilde{#1}}%

\global\long\def\mrm#1{\mathrm{#1}}%

\global\long\def\pr#1{\left(#1\right)}%

\global\long\def\abs#1{\left|#1\right|}%

\global\long\def\inp#1{\left\langle #1\right\rangle }%

\global\long\def\br#1{\left\{  #1\right\}  }%

\global\long\def\norm#1{\left\Vert #1\right\Vert }%

\global\long\def\hat#1{\widehat{#1}}%

\global\long\def\opn#1{\operatorname{#1}}%

\global\long\def\bigmid{\,\middle|\,}%

\global\long\def\Top{\sf{Top}}%

\global\long\def\Set{\sf{Set}}%

\global\long\def\SS{\sf{sSet}}%

\global\long\def\Kan{\sf{Kan}}%

\global\long\def\Cat{\mathcal{C}\sf{at}}%

\global\long\def\imfld{\cal M\mathsf{fld}}%

\global\long\def\ids{\cal D\sf{isk}}%

\global\long\def\ich{\cal C\sf h}%

\global\long\def\SW{\mathcal{SW}}%

\global\long\def\SHC{\mathcal{SHC}}%

\global\long\def\B{\sf B}%

\global\long\def\Spaces{\sf{Spaces}}%

\global\long\def\Mod{\sf{Mod}}%

\global\long\def\Nec{\sf{Nec}}%

\global\long\def\Fin{\sf{Fin}}%

\global\long\def\Ch{\sf{Ch}}%

\global\long\def\Ab{\sf{Ab}}%

\global\long\def\SA{\sf{sAb}}%

\global\long\def\P{\mathsf{POp}}%

\global\long\def\Op{\mathcal{O}\mathsf{p}}%

\global\long\def\Opg{\mathcal{O}\mathsf{p}_{\infty}^{\mathrm{gn}}}%

\global\long\def\Tup{\mathsf{Tup}}%

\global\long\def\Del{\mathbf{\Delta}}%

\global\long\def\id{\operatorname{id}}%

\global\long\def\Aut{\operatorname{Aut}}%

\global\long\def\End{\operatorname{End}}%

\global\long\def\Hom{\operatorname{Hom}}%

\global\long\def\Ext{\operatorname{Ext}}%

\global\long\def\sk{\operatorname{sk}}%

\global\long\def\ihom{\underline{\operatorname{Hom}}}%

\global\long\def\N{\mathrm{N}}%

\global\long\def\-{\text{-}}%

\global\long\def\op{\mathrm{op}}%

\global\long\def\To{\Rightarrow}%

\global\long\def\rr{\rightrightarrows}%

\global\long\def\rl{\rightleftarrows}%

\global\long\def\mono{\rightarrowtail}%

\global\long\def\epi{\twoheadrightarrow}%

\global\long\def\comma{\downarrow}%

\global\long\def\ot{\leftarrow}%

\global\long\def\corr{\leftrightsquigarrow}%

\global\long\def\lim{\operatorname{lim}}%

\global\long\def\colim{\operatorname{colim}}%

\global\long\def\holim{\operatorname{holim}}%

\global\long\def\hocolim{\operatorname{hocolim}}%

\global\long\def\Ran{\operatorname{Ran}}%

\global\long\def\Lan{\operatorname{Lan}}%

\global\long\def\Sk{\operatorname{Sk}}%

\global\long\def\Sd{\operatorname{Sd}}%

\global\long\def\Ex{\operatorname{Ex}}%

\global\long\def\Cosk{\operatorname{Cosk}}%

\global\long\def\Sing{\operatorname{Sing}}%

\global\long\def\Sp{\operatorname{Sp}}%

\global\long\def\Spc{\operatorname{Spc}}%

\global\long\def\Ho{\operatorname{Ho}}%

\global\long\def\Fun{\operatorname{Fun}}%

\global\long\def\map{\operatorname{map}}%

\global\long\def\diag{\operatorname{diag}}%

\global\long\def\Gap{\operatorname{Gap}}%

\global\long\def\cc{\operatorname{cc}}%

\global\long\def\Ob{\operatorname{Ob}}%

\global\long\def\Map{\operatorname{Map}}%

\global\long\def\Rfib{\operatorname{RFib}}%

\global\long\def\Lfib{\operatorname{LFib}}%

\global\long\def\Tw{\operatorname{Tw}}%

\global\long\def\Equiv{\operatorname{Equiv}}%

\global\long\def\Arr{\operatorname{Arr}}%

\global\long\def\Cyl{\operatorname{Cyl}}%

\global\long\def\Path{\operatorname{Path}}%

\global\long\def\Alg{\operatorname{Alg}}%

\global\long\def\ho{\operatorname{ho}}%

\global\long\def\Comm{\operatorname{Comm}}%

\global\long\def\Triv{\operatorname{Triv}}%

\global\long\def\triv{\operatorname{triv}}%

\global\long\def\Env{\operatorname{Env}}%

\global\long\def\Act{\operatorname{Act}}%

\global\long\def\act{\operatorname{act}}%

\global\long\def\loc{\operatorname{loc}}%

\global\long\def\Assem{\operatorname{Assem}}%

\global\long\def\Nat{\operatorname{Nat}}%

\global\long\def\lax{\mathrm{lax}}%

\global\long\def\weq{\mathrm{weq}}%

\global\long\def\fib{\mathrm{fib}}%

\global\long\def\cof{\mathrm{cof}}%

\global\long\def\inj{\mathrm{inj}}%

\global\long\def\univ{\mathrm{univ}}%

\global\long\def\Ker{\opn{Ker}}%

\global\long\def\Coker{\opn{Coker}}%

\global\long\def\Im{\opn{Im}}%

\global\long\def\Coim{\opn{Im}}%

\global\long\def\coker{\opn{coker}}%

\global\long\def\im{\opn{\mathrm{im}}}%

\global\long\def\coim{\opn{coim}}%

\global\long\def\gn{\mathrm{gn}}%

\global\long\def\Mon{\mathrm{Mon}}%

\global\long\def\Un{\mathrm{Un}}%

\global\long\def\St{\mathrm{St}}%

\global\long\def\CA{\operatorname{CAlg}}%

\global\long\def\rd{\mathrm{rd}}%

\global\long\def\xmono#1#2{\stackrel[#2]{#1}{\rightarrowtail}}%

\global\long\def\xepi#1#2{\stackrel[#2]{#1}{\twoheadrightarrow}}%

\global\long\def\adj{\stackrel[\longleftarrow]{\longrightarrow}{\bot}}%

\global\long\def\btimes{\boxtimes}%

\global\long\def\ps#1#2{\prescript{}{#1}{#2}}%

\global\long\def\ups#1#2{\prescript{#1}{}{#2}}%

\global\long\def\hofib{\mathrm{hofib}}%

\global\long\def\cofib{\mathrm{cofib}}%

\global\long\def\Vee{\bigvee}%

\global\long\def\w{\wedge}%

\global\long\def\t{\otimes}%

\global\long\def\bp{\boxplus}%

\global\long\def\rcone{\triangleright}%

\global\long\def\lcone{\triangleleft}%

\global\long\def\S{\mathsection}%

\global\long\def\p{\prime}%
 

\global\long\def\pp{\prime\prime}%

\global\long\def\W{\overline{W}}%

\global\long\def\o#1{\overline{#1}}%

\title[Monoidal Envelopes of Families and Operadic
Kan Extensions]{Monoidal Envelopes of Families of $\infty$-Operads and $\infty$-Operadic
Kan Extensions}
\author{Kensuke Arakawa}
\email{arakawa.kensuke.22c@st.kyoto-u.ac.jp}
\address{Department of Mathematics, Kyoto University, Kyoto, 606-8502, Japan}
\subjclass[2020]{18N70, 55P48}
\begin{abstract}
We provide details of the proof of Lurie's theorem on operadic Kan
extensions \cite[Theorem 3.1.2.3]{HA}. Along the way, we generalize
the construction of monoidal envelopes of $\infty$-operads to families
of $\infty$-operads and use it to construct the fiberwise direct
sum functor, both of which we characterize by certain universal properties.
Aside from their uses in the proof of Lurie's theorem, these results
and constructions have their independent interest.
\end{abstract}

\maketitle
\tableofcontents{}

\section*{Introduction}

In his book \cite{HA}, Lurie introduces (among other things) the
notion of $\infty$-operads, an $\infty$-categorical analog of (colored)
operads. Given an $\infty$-operad $\cal O^{\t}$ and a symmetric
monoidal $\infty$-category $\cal C^{\t}$, we can form the $\infty$-category
$\Alg_{\cal O}\pr{\cal C}$ of $\cal O$-algebras. If $f:\cal O^{\t}\to\cal O^{\p\t}$
is a morphism of $\infty$-operads, there is the induced forgetful
map $f^{*}:\Alg_{\cal O'}\pr{\cal C}\to\Alg_{\cal O}\pr{\cal C}$.
In analogy with restrictions and extensions of scalars, it is natural
to ask whether the map $f^{*}$ has a left adjoint. Lurie gives an
affirmative answer to this question in \cite[Corollary 3.1.3.5]{HA},
provided that $\cal C$ has enough colimits. Because of its fundamental
importance, the result has been applied repeatedly in his work. 

Lurie's proof of \cite[Corollary 3.1.3.5]{HA} relies on another major
theorem on operadic Kan extensions \cite[Theorem 3.1.2.3]{HA} (or
Theorem \ref{thm:3.1.2.3}), which we call the \textbf{fundamental
theorem of operadic Kan extensions}, or FTOK for short. The proof
of FTOK, as Lurie himself acknowledges, is very long. Perhaps because
of the length of the proof, he omits some crucial details of the proof.
This note aims to provide all the details by making necessary constructions
and proving results on them. (In particular, we do not aim to provide
a more concise proof of FTOK than the one provided in \cite{HA}.)

We can roughly divide the details we provide into two parts: The first
one is the datum of ``coherent homotopy.'' In the proof of \cite[Theorem 3.1.2.3]{HA}
(to be more precise, on p. 337), Lurie claims that certain diagrams
are ``equivalent'' without writing the actual equivalence. Such
a practice is fairly common in the literature and is understandable
to some extent. However, in our case, we must be more attentive because
a considerable amount of effort is required to write down the actual
equivalence. We thus give a complete treatment of the equivalence
in this note. (See around Step 1 of the proof of Theorem \ref{thm:3.1.2.3}.)
The second missing detail is the verification that all the combinatorics
fits together. (This corresponds to Step 3 of Theorem \ref{thm:3.1.2.3}.)
Such detail, again, could be left to the reader if the verification
is trivial. But in our case, the verification is so complicated that
a sizable portion of readers will benefit from having it written down.
We thus record every single detail of the verification. 

Here is an outline of this note. In Section \ref{sec:A-Result-on},
we will prove an equivalent formulation of families of $\infty$-operads.
In Section \ref{sec:Monoidal-Envelopes-of}, we generalize Lurie's
monoidal envelopes to families of $\infty$-operads. Using monoidal
envelopes, we can define the fiberwise direct sum functor, whose properties
we discuss in Section \ref{sec:Direct_sum}. The constructions and
results in Sections \ref{sec:A-Result-on}, \ref{sec:Monoidal-Envelopes-of},
and \ref{sec:Direct_sum} will be used in writing down the equivalence
discussed in the previous paragraph, but they also have some independent
interest. To the author's knowledge, these constructions and results
have not appeared elsewhere. Finally, in Section \ref{sec:FTOK},
we will give a complete proof of FTOK.

\section*{Notation and Terminology}

Our notation and terminology mostly follow those of \cite{HA}. Here
are some deviations.
\begin{itemize}
\item We will say that a morphism of simplicial sets is \textbf{final} if
it is cofinal in the sense of \cite{HTT}, and \textbf{initial} if
its opposite is final. 
\item We will write $\Delta^{-1}$ for the empty simplicial set. 
\item If $\cal C$ is a category and $\alpha$ is an ordinal (or more generally
a well-ordered set), then an \textbf{$\alpha$-sequence} in $\cal C$
is a functor $F:\alpha\to\cal C$ such that the map $\colim_{\beta<\lambda}F\beta\to F\beta$
is an isomorphism for each limit ordinal less than $\alpha$. 
\item If $X$ is a simplicial set, then we denote the cone point of the
simplicial set $X^{\rcone}$ by $\infty$.
\end{itemize}

\section{\label{sec:A-Result-on}A Result on Families of $\infty$-Operads}

Let $\cal C$ be an $\infty$-category. Recall that a $\cal C$\textbf{-family
of $\infty$-operads} \cite[Definition 2.3.2.1]{HA} is a categorical
fibration $p:\cal M^{\t}\to\cal C\times N\pr{\Fin_{\ast}}$ satisfying
the following conditions:
\begin{itemize}
\item [(a)]For each object $M\in\cal M^{\t}$ with image $\pr{C,\inp m}\in\cal C\times N\pr{\Fin_{\ast}}$
and for each inert map $\alpha:\inp m\to\inp n$ in $N\pr{\Fin_{\ast}}$,
the morphism $\pr{\id_{C},\alpha}$ admits a $p$-cocartesian lift.
\item [(b)]Let $M\in\cal M^{\t}$ be an object with image $\pr{C,\inp m}\in\cal C\times N\pr{\Fin_{\ast}}$,
where $m\geq1$. Choose for each $1\leq i\leq m$ a $p$-cocartesian
lift $f_{i}:M\to M_{i}$ over $\pr{\id_{C},\rho^{i}}:\pr{C,\inp n}\to\pr{C,\inp 1}$.
Then the morphisms $f_{i}$ form a $p$-limit cone. Every object in
$\cal M_{\inp 0}^{\t}$ is $p$-terminal.
\item [(c)]Let $n\geq1$ and $C\in\cal C$. Given objects $M_{1},\dots,M_{n}\in\cal M_{C}$,
there is an object $M\in\cal M^{\t}$ lying over $\pr{C,\inp n}$
which admits $p$-cocartesian morphisms $M\to M_{i}$ over $\pr{\id_{C},\rho^{i}}:\pr{C,\inp n}\to\pr{C,\inp 1}$.
\end{itemize}
The goal of this section is to prove the following equivalent formulation
of families of $\infty$-operads:
\begin{prop}
\label{prop:sec1_main}Let $\cal C$ be an $\infty$-category and
let $p:\cal M^{\t}\to\cal C\times N\pr{\Fin_{\ast}}$ be a categorical
fibration satisfying conditions (a) and (b) above. Then the following
conditions are equivalent:
\begin{itemize}
\item [(c-i)]The map $p$ satisfies condition (c).
\item [(c-ii)]For each $1\leq i\leq n$, let $\rho_{!}^{i}:\cal M_{\inp n}^{\t}\to\cal M_{\inp 1}^{\t}$
be the functor over $\cal C$ induced by the morphism $\rho^{i}$.
Then for each $n\geq1$, the functor
\[
\pr{\rho_{!}^{i}}_{1\leq i\leq n}:\cal M_{\inp n}^{\t}\to\cal M\times_{\cal C}\cdots\times_{\cal C}\cal M
\]
is an equivalence of $\infty$-categories. 
\end{itemize}
\end{prop}
Here the functor $\rho_{!}^{i}:\cal M_{\inp n}^{\t}\to\cal M$ is
obtained in the following way: Since every inert morphism in $\cal C\times N\pr{\Fin_{\ast}}$
admits a $p$-cocartesian lift, it is possible to choose a natural
transformation $\cal M_{\inp n}^{\t}\times\Delta^{1}\to\cal M^{\t}$
fitting into the commutative diagram % https://q.uiver.app/?q=WzAsNCxbMSwwLCIoXFxtYXRoY2Fse019Xlxcb3RpbWVzICxcXG1hdGhjYWx7RX0pIl0sWzEsMSwiXFxtYXRoY2Fse0N9XlxcZmxhdFxcdGltZXMgTihcXG1hdGhzZntGaW59X1xcYXN0KV5cXG5hdHVyYWwsIl0sWzAsMSwiKFxcbWF0aGNhbHtNfV5cXG90aW1lcyBfe1xcbGFuZ2xlIG5cXHJhbmdsZX0pXlxcZmxhdFxcdGltZXMgKFxcRGVsdGFeMSleXFxzaGFycCJdLFswLDAsIihcXG1hdGhjYWx7TX1eXFxvdGltZXMgX3tcXGxhbmdsZSBuXFxyYW5nbGV9KV5cXGZsYXRcXHRpbWVzIFxcezBcXH0iXSxbMiwxLCJwXzJcXHRpbWVzIFxccmhvXmkiLDJdLFswLDFdLFszLDBdLFszLDJdLFsyLDAsIiIsMSx7InN0eWxlIjp7ImJvZHkiOnsibmFtZSI6ImRhc2hlZCJ9fX1dXQ==
\[\begin{tikzcd}
	{(\mathcal{M}^\otimes _{\langle n\rangle})^\flat\times \{0\}} & {(\mathcal{M}^\otimes ,\mathcal{E})} \\
	{(\mathcal{M}^\otimes _{\langle n\rangle})^\flat\times (\Delta^1)^\sharp} & {\mathcal{C}^\flat\times N(\mathsf{Fin}_\ast)^\natural,}
	\arrow["{p_2\times \rho^i}"', from=2-1, to=2-2]
	\arrow[from=1-2, to=2-2]
	\arrow[from=1-1, to=1-2]
	\arrow[from=1-1, to=2-1]
	\arrow[dashed, from=2-1, to=1-2]
\end{tikzcd}\]where $\cal E$ denotes the set of $p$-cocartesian edges over inert
morphisms in $\cal C\times N\pr{\Fin_{\ast}}$ whose image in $\cal C$
is degenerate, and $N\pr{\Fin_{\ast}}^{\natural}$ is the marked simplicial
set whose underlying simplicial set is $N\pr{\Fin_{\ast}}$ and with
inert morphisms marked. We understand that $\rho_{!}^{i}$ is the
restriction of the filler, so that it is a functor over $\cal C$
and its homotopy class over $\cal C$ is well-defined.
\begin{rem}
A reader conversant with the language of generalized $\infty$-operads
will recognize Proposition \ref{prop:sec1_main} as \cite[Proposition 2.3.2.11]{HA}.
However, in general, the diagram appearing in (2) of \cite[Definition 2.3.2.1]{HA}
is not strictly commutative but is commutative up to a specified natural
equivalence. So Proposition \ref{prop:sec1_main} is slightly different
from the cited proposition and should be regarded as its ``strictified
version.''
\end{rem}
The hardest part of the proof is the rectification of cocartesian
fibrations. We will overcome this problem by appealing to the simplicial
Grothendieck construction, which we explain in subsection \ref{subsec:Simplicial-Grothendieck-Construction}.
Proposition \ref{prop:sec1_main} will then be proved in subsection
\ref{subsec:Main-Result}.

\subsection{\label{subsec:Simplicial-Grothendieck-Construction}Simplicial Grothendieck
Construction}

In this subsection, we describe a convenient method to construct a
cocartesian fibration out of a functor $F:\cal C\to\sf{Cat}_{\Delta}$
of ordinary categories.

Let $\cal C$ be an ordinary category and $F:\cal C\to\sf{Cat}_{\Delta}$
a projectively fibrant functor. The (\textbf{simplicial}) \textbf{Gronthendieck
construction} $\int F$ is the simplicial category which is defined
as follows:
\begin{enumerate}
\item The set of objects of $F$ is given by $\coprod_{C\in\cal C}\opn{ob}F\pr C$.
\item Let $C,C'\in\cal C$ and $X\in F\pr C$, $X'\in F\pr{C'}$. Then the
hom-simplicial sets is 
\[
\pr{\int F}\pr{X,X'}=\coprod_{f\in\cal C\pr{C,C'}}F\pr{C'}\pr{Ff\pr X,X'}.
\]
\end{enumerate}
Composition is given by the composites 
\begin{align*}
F\pr{C''}\pr{Fg\pr{X'},X''}\times F\pr{C'}\pr{Ff\pr X,X'} & \to F\pr{C''}\pr{Fg\pr{X'},X''}\times F\pr{C''}\pr{FgFf\pr X,FgX'}\\
 & \xrightarrow{\circ}F\pr{C''}\pr{F\pr{gf}\pr X,X''}.
\end{align*}
This construction defines a functor
\[
\int:\Fun\pr{\cal C,\sf{Cat}_{\Delta}}\to\pr{\sf{Cat}_{\Delta}}_{/\cal C}.
\]

The Grothendieck construction is closely related to Lurie's relative
nerve functor \cite[$\S$ 3.2.5]{HTT}:
\begin{prop}
\label{prop:relative_nerve_of_Gr_const}Let $\cal C$ be a category
and $F:\cal C\to\sf{Cat}_{\Delta}$ a functor. There is an isomorphism
of simplicial sets 
\[
N\pr{\int F}\cong N_{N\circ F}\pr{\cal C}
\]
between the homotopy coherent nerve of $\int F$ and the relative
nerve of the composite $N\circ F:\cal C\to\sf{Cat}_{\Delta}\xrightarrow{N}\SS$,
which commutes with the projection to $N\pr{\cal C}$. The isomorphism
is natural in $F$.
\end{prop}
\begin{proof}
An $n$-simplex of the relative nerve $N_{N\circ F}\pr{\cal C}$,
by definition, consists of an $n$-simplex $\sigma=C_{0}\xrightarrow{f_{1}}\cdots\xrightarrow{f_{n}}C_{n}$
in $N\pr{\cal C}$ and a simplicial functor $\varphi_{I}:\fr C[\Delta^{I}]\to F\pr{C_{\max I}}$
for each subset $I\subset[n]$, such that for any inclusions $I\subset J\subset[n]$
of subsets, the diagram % https://q.uiver.app/?q=WzAsNCxbMCwwLCJcXG1hdGhmcmFre0N9W1xcRGVsdGFeSV0iXSxbMSwwLCJGKENfe1xcbWF4IEl9KSJdLFsxLDEsIkYoQ197XFxtYXggSn0pIl0sWzAsMSwiXFxtYXRoZnJha3tDfVtcXERlbHRhXkpdIl0sWzAsMV0sWzEsMl0sWzAsM10sWzMsMl1d
\[\begin{tikzcd}
	{\mathfrak{C}[\Delta^I]} & {F(C_{\max I})} \\
	{\mathfrak{C}[\Delta^J]} & {F(C_{\max J})}
	\arrow[from=1-1, to=1-2]
	\arrow[from=1-2, to=2-2]
	\arrow[from=1-1, to=2-1]
	\arrow[from=2-1, to=2-2]
\end{tikzcd}\]commutes. Given such an $n$-simplex $\pr{\sigma,\pr{x_{I}}_{I}}$,
we define a simplicial functor $\varphi:\fr C[\Delta^{n}]\to\int F$
as follows:
\begin{itemize}
\item For each integer $0\leq i\leq n$, we set $\varphi\pr i=\varphi_{[i]}\pr i$.
\item For each pair of integers $0\leq i\leq j\leq n$, the map
\[
\varphi:\fr C[\Delta^{n}]\pr{i,j}\to\pr{\int F}\pr{\varphi_{[i]}\pr i,\varphi_{[i]}\pr j}
\]
is the composite 
\begin{align*}
\fr C[\Delta^{n}]\pr{i,j} & \xrightarrow{\varphi_{[j]}}F\pr{C_{j}}\pr{\varphi_{[j]}\pr i,\varphi_{[j]}\pr j}\\
 & =F\pr{C_{j}}\pr{F\pr{f_{ij}}\pr{\varphi_{[i]}\pr i},\varphi_{[j]}\pr j}\\
 & \hookrightarrow\pr{\int F}\pr{\varphi_{[i]}\pr i,\varphi_{[i]}\pr j}.
\end{align*}
Here $f_{ij}:C_{i}\to C_{j}$ is the composite $f_{j}\cdots f_{i+1}$. 
\end{itemize}
The assignment $\pr{\sigma,\pr{x_{I}}_{I}}\mapsto\varphi$ determines
a morphism of simplicial sets
\[
\Phi:N_{N\circ F}\pr{\cal C}\to N\pr{\int F}
\]
over $N\pr{\cal C}$. We claim that it is an isomorphism of simplicial
sets by constructing an inverse. 

Let $\varphi:\fr C[\Delta^{n}]\to\int F$ be a simplicial functor.
Its projection to $\cal C$ determines an $n$-simplex $\sigma=C_{0}\xrightarrow{f_{1}}\cdots\xrightarrow{f_{n}}C_{n}$
in $N\pr{\cal C}$. For each subinterval $I\subset[n]$, define a
simplicial functor $\varphi_{I}:\fr C[\Delta^{I}]\to F\pr{C_{\max I}}$
as follows:
\begin{itemize}
\item For each integer $i\in I$, we set $\varphi_{I}\pr i=Ff_{i,\max I}\pr{\varphi\pr i}$.
\item For each pair of integers $i,j\in I$ with $i\le j$, the map
\[
\varphi_{I}:\fr C[\Delta^{I}]\pr{i,j}\to F\pr{C_{\max I}}\pr{\varphi_{I}\pr i,\varphi_{I}\pr j}
\]
is the composite
\begin{align*}
\fr C[\Delta^{I}]\pr{i,j} & =\fr C[\Delta^{n}]\pr{i,j}\\
 & \xrightarrow{\varphi}F\pr{C_{j}}\pr{F\pr{f_{ij}}\pr{\varphi\pr i},\varphi\pr j}\\
 & \xrightarrow{Ff_{j,\max I}}F\pr{C_{\max I}}\pr{F\pr{f_{i,\max I}}\pr{\varphi\pr i},F\pr{f_{j,\max I}}\pr{\varphi\pr j}}\\
 & =F\pr{C_{\max I}}\pr{\varphi_{I}\pr i,\varphi_{I}\pr j}.
\end{align*}
\end{itemize}
Next, for an arbitrary subset $J\subset[n]$, we define a simplicial
functor $\varphi_{J}:\fr C[\Delta^{J}]\to F\pr{C_{\max J}}$ to be
the restriction of $\varphi_{[\max J]}$. Then the pair $\pr{\sigma,\pr{\varphi_{I}}_{I}}$
is an $n$-simplex of $N_{N\circ F}\pr{\cal C}$, and the assignment
$\varphi\mapsto\pr{\sigma,\pr{\varphi_{I}}_{I}}$ determines an inverse
of $\Phi$.
\end{proof}
\begin{cor}
\label{cor:sgr}Let $\cal C$ be a category and $F:\cal C\to\sf{Cat}_{\Delta}$
a projectively fibrant functor. Then the functor 
\[
p:N\pr{\int F}\to N\pr{\cal C}
\]
is a cocartesian fibration. A morphism $\pr{f,g}:\pr{C,X}\to\pr{D,Y}$
in $N\pr{\int F}$ is $p$-cocartesian if and only if the morphism
$g:\pr{Ff}\pr X\to Y$ of $N\pr{F\pr D}$ is an equivalence.
\end{cor}
\begin{proof}
Define $\widetilde{F}:\cal C\to\SS^{+}$ by $\widetilde{F}C=N\pr{FC}^{\natural}$,
the homotopy coherent nerve of $FC$ with equivalences marked. Then
$\widetilde{F}$ is a projectively fibrant functor, so by \cite[Proposition 3.2.5.18]{HTT}
its relative nerve $N_{\widetilde{F}}^{+}\pr{\cal C}\to N\pr{\cal C}^{\sharp}$
is a fibrant object of $\SS^{+}/N\pr{\cal C}$ equipped with the cocartesian
model structure. The claim now follows from Proposition \ref{prop:relative_nerve_of_Gr_const}.
\end{proof}
\begin{rem}
The advantage of working with the simplicial Grothendieck construction
is that we can be very explicit about the functors induced by cocartesian
fibrations. Recall that if $p:\cal A\to\cal B$ is a cocartesian fibration
of $\infty$-categories and $f:X\to Y$ is a morphism in $\cal B$,
the induced functor $f_{!}:\cal A_{X}=\cal A\times_{\cal B}\{X\}\to\cal A_{Y}$
is defined by choosing a cocartesian natural transformation $\cal A_{X}\times\Delta^{1}\to\cal B$
fitting into the commutative diagram % https://q.uiver.app/?q=WzAsNSxbMCwwLCJcXG1hdGhjYWx7QX1fWFxcdGltZXMgXFx7MFxcfSJdLFswLDEsIlxcbWF0aGNhbHtBfV9YXFx0aW1lcyBcXERlbHRhXjEiXSxbMiwwLCJcXG1hdGhjYWx7QX0iXSxbMiwxLCJcXG1hdGhjYWx7Qn0iXSxbMSwxLCJcXERlbHRhXjEiXSxbMCwxXSxbMCwyXSxbMiwzXSxbMSwyLCIiLDEseyJzdHlsZSI6eyJib2R5Ijp7Im5hbWUiOiJkYXNoZWQifX19XSxbMSw0XSxbNCwzLCJmIiwyXV0=
\[\begin{tikzcd}
	{\mathcal{A}_X\times \{0\}} && {\mathcal{A}} \\
	{\mathcal{A}_X\times \Delta^1} & {\Delta^1} & {\mathcal{B}}
	\arrow[from=1-1, to=2-1]
	\arrow[from=1-1, to=1-3]
	\arrow[from=1-3, to=2-3]
	\arrow[dashed, from=2-1, to=1-3]
	\arrow[from=2-1, to=2-2]
	\arrow["f"', from=2-2, to=2-3]
\end{tikzcd}\]and then restricting it to $\cal A_{X}\times\{1\}$. 

In the case $p$ is the nerve of a simplicial functor of the form
$\int F\to\cal C$, given a morphism $f:C\to C'$ in $\cal C$, the
induced functor $f_{!}:N\pr{FC}\to N\pr{FC'}$ is nothing but $Nf$.
Indeed, the collection
\[
\{\id_{FfX}:FfX\to FfX\}_{X\in FC}
\]
defines a simplicial natural transformation from the inclusion $FC\hookrightarrow\int F$
to the composite $FC\xrightarrow{Ff}FC'\hookrightarrow\int F$, and
the nerve of this simplicial natural transformation gives us a natural
transformation $N\pr{FC}\times\Delta^{1}\to N\pr{\int F}$ over the
morphism $f$ which is pointwise cocartesian by Corollary \ref{cor:sgr}.
\end{rem}

\subsection{\label{subsec:Main-Result}Proof of Proposition \ref{prop:sec1_main}}

In this subsection, we prove Proposition \ref{prop:sec1_main} using
the results in the previous subsection.

Let $\pr{\sf{Cat}_{\Delta}}_{\fib}$ denote the full subcategory of
$\sf{Cat}_{\Delta}$ spanned by the fibrant simplicial categories.
We let $N^{\natural}:\pr{\sf{Cat}_{\Delta}}_{\fib}\to\SS^{+}$ denote
the functor obtained from the homotopy coherent nerve functor by marking
equivalences. 
\begin{lem}
\label{lem:replacement}Let $\cal C$ be a small category and let
$F,G:\cal C\to\SS_{\mathrm{cocart}}^{+}$ be projectively fibrant
functors. For any natural transformation $f:F\to G$, we can find
projectively fibrant functors $F',G':\cal C\to\sf{Cat}_{\Delta}$,
a projective fibration $f':F'\to G'$, and a commutative diagram % https://q.uiver.app/?q=WzAsNCxbMCwwLCJGIl0sWzEsMCwiTl5cXG5hdHVyYWxcXGNpcmMgRiciXSxbMSwxLCJOXlxcbmF0dXJhbFxcY2lyYyBHJyJdLFswLDEsIkciXSxbMCwxLCJcXHNpbWVxIl0sWzEsMiwiTl5cXG5hdHVyYWxcXGNpcmMgZiciXSxbMCwzLCJmIiwyXSxbMywyLCJcXHNpbWVxIiwyXV0=
\[\begin{tikzcd}
	F & {N^\natural\circ F'} \\
	G & {N^\natural\circ G'}
	\arrow["\simeq", from=1-1, to=1-2]
	\arrow["{N^\natural\circ f'}", from=1-2, to=2-2]
	\arrow["f"', from=1-1, to=2-1]
	\arrow["\simeq"', from=2-1, to=2-2]
\end{tikzcd}\]in $\Fun\pr{\cal C,\SS^{+}}$ whose horizontal arrows are weak equivalences.
\end{lem}
\begin{proof}
Let $F_{\flat},G_{\flat}:\cal C\to\SS$ denote the functors obtained
by forgetting the markings. Since $F,G$ are projectively fibrant,
for each object $C\in\cal C$, the simplicial set $F_{\flat}\pr C$
is an $\infty$-category and the marked edges of $F\pr C$ are precisely
the equivalences of $F_{\flat}\pr C$. Now find a commutative diagram
% https://q.uiver.app/?q=WzAsNCxbMCwwLCJcXG1hdGhmcmFre0N9XFxjaXJjIEZfXFxmbGF0Il0sWzAsMSwiXFxtYXRoZnJha3tDfVxcY2lyYyBHX1xcZmxhdCJdLFsxLDAsIkYnIl0sWzEsMSwiRyciXSxbMCwxLCJcXG1hdGhmcmFre0N9XFxjaXJjIGYiLDJdLFswLDIsIlxcc2ltZXEiXSxbMiwzLCJmJyJdLFsxLDMsIlxcc2ltZXEiLDJdXQ==
\[\begin{tikzcd}
	{\mathfrak{C}\circ F_\flat} & {F'} \\
	{\mathfrak{C}\circ G_\flat} & {G'}
	\arrow["{\mathfrak{C}\circ f}"', from=1-1, to=2-1]
	\arrow["\simeq", from=1-1, to=1-2]
	\arrow["{f'}", from=1-2, to=2-2]
	\arrow["\simeq"', from=2-1, to=2-2]
\end{tikzcd}\]in $\Fun\pr{\cal C,\sf{Cat}_{\Delta}}$, where $F',G'$ are projectively
fibrant, $\alpha'$ is a projective fibration, and the horizontal
arrows are weak equivalences. Since the adjunction
\[
\fr C:\SS_{\mathrm{Joyal}}\adj\sf{Cat}_{\Delta}:N
\]
is a Quillen equivalence, the horizontal arrows of the induced square
% https://q.uiver.app/?q=WzAsNCxbMCwwLCJGX1xcZmxhdCJdLFswLDEsIkdfXFxmbGF0Il0sWzEsMCwiTlxcY2lyYyBGJyJdLFsxLDEsIk5cXGNpcmMgRyciXSxbMCwxLCJmIiwyXSxbMCwyLCJcXHNpbWVxIl0sWzIsMywiTlxcY2lyYyBmJyJdLFsxLDMsIlxcc2ltZXEiLDJdXQ==
\[\begin{tikzcd}
	{F_\flat} & {N\circ F'} \\
	{G_\flat} & {N\circ G'}
	\arrow["f"', from=1-1, to=2-1]
	\arrow["\simeq", from=1-1, to=1-2]
	\arrow["{N\circ f'}", from=1-2, to=2-2]
	\arrow["\simeq"', from=2-1, to=2-2]
\end{tikzcd}\]in $\Fun\pr{\cal C,\SS}$ are weak equivalences (i.e., pointwise categorical
equivalences). By marking the equivalences, we obtain the desired
square.
\end{proof}
\begin{defn}
Let $n\geq1$. We let $\cal B\pr n$ denote the nerve of the category
$\{\infty\}\star\inp n^{\circ}$, where $\inp n^{\circ}=\{1,\dots,n\}$
is regarded as a discrete category. For each $1\leq i\leq n$, we
will denote the unique morphism $\infty\to i$ by $\rho^{i}$.
\end{defn}
\begin{prop}
\label{prop:pre-equivalence}Let $n\geq1$, and consider a commutative
diagram% https://q.uiver.app/?q=WzAsMyxbMCwwLCJcXG1hdGhjYWx7Q30iXSxbMiwwLCJcXG1hdGhjYWx7RH0iXSxbMSwxLCJcXG1hdGhjYWx7Qn0obikiXSxbMCwxLCJwIl0sWzEsMiwiciJdLFswLDIsInEiLDJdXQ==
\[\begin{tikzcd}
	{\mathcal{C}} && {\mathcal{D}} \\
	& {\mathcal{B}(n)}
	\arrow["p", from=1-1, to=1-3]
	\arrow["r", from=1-3, to=2-2]
	\arrow["q"', from=1-1, to=2-2]
\end{tikzcd}\]of $\infty$-categories. Assume the following:
\begin{enumerate}
\item The functors $q$ and $r$ are cocartesian fibrations, and the functor
$p$ is a categorical fibration.
\item The functor $p$ preserves cocartesian morphisms over $\cal B\pr n$.
\item The functor $\pr{\rho_{!}^{i}}_{i}:\cal D_{\infty}\to\prod_{1\leq i\leq n}\cal D_{i}$
induces an injection between the set of equivalence classes of objects
of $\cal D_{\inp n}$ and that of $\prod_{1\leq i\leq n}\cal D_{i}$.
\item For each object $C\in\cal C_{\infty}$, the $q$-cocartesian morphisms
$C\to C_{i}$ over $\rho^{i}$ form a $p$-limit cone. 
\item Given an object $\pr{C_{i}}_{i}\in\prod_{1\leq i\leq n}\cal C_{i}$,
there is an object $C\in\cal C$ which admits a $q$-cocartesian morphism
$C\to C_{i}$ over the morphism $\rho^{i}$ for every $i$.
\end{enumerate}
Then the functor
\[
\cal C_{\infty}\to\prod_{1\leq i\leq n}\cal C_{i}\times_{\prod_{1\leq i\leq n}\cal D_{i}}\cal D_{\infty}
\]
is an equivalence of $\infty$-categories.
\end{prop}
In the proof, we shall make use of the adjunction 
\[
r_{!}:\SS^{+}/\cal B\pr n\adj\Fun\pr{\{\infty\}\star\inp n^{\circ},\SS^{+}}:r^{*}
\]
between the (marked) relative nerve functor $r^{*}$ and its left
adjoint $r_{!}$, given by $r_{!}\pr X\pr x=X\times_{\cal B\pr n^{\sharp}}\cal B\pr n_{/x}^{\sharp}$.
\begin{proof}
The statement of the proposition is a bit vague, because the functors
$\cal C_{\infty}\to\cal C_{i}$ and $\cal D_{\infty}\to\cal D_{i}$
are defined only up to natural equivalence. A more precise formulation
is the following: Since $p$ is a categorical fibration, the map $\cal C^{\natural}\to\cal D^{\natural}$
is a fibration in the cocartesian model structures over $\cal B\pr n$,
where $\cal C^{\natural}$ is the marked simplicial set whose marked
edges are the $q$-cocartesian edges, and $\cal D^{\natural}$ is
defined similarly. (This is the opposite convention of \cite[$\S 3.2$]{HTT},
but it shouldn't case any confusion.) Therefore, for each object $i\in\inp n^{\circ}$,
there is a natural transformation $\cal C_{\infty}\times\Delta^{1}\to\cal C$
which fits into the commutative diagram % https://q.uiver.app/?q=WzAsNSxbMCwwLCJcXG1hdGhjYWx7Q31eXFxuYXR1cmFsX1xcaW5mdHlcXHRpbWVzIChcXHswXFx9KV5cXHNoYXJwIl0sWzAsMSwiXFxtYXRoY2Fse0N9XlxcbmF0dXJhbF9cXGluZnR5XFx0aW1lcyAoXFxEZWx0YV4xKV5cXHNoYXJwIl0sWzEsMSwiXFxtYXRoY2Fse0R9XlxcbmF0dXJhbF9cXGluZnR5XFx0aW1lcyAoXFxEZWx0YV4xKV5cXHNoYXJwIl0sWzIsMSwiXFxtYXRoY2Fse0R9XlxcbmF0dXJhbCJdLFsyLDAsIlxcbWF0aGNhbHtDfV5cXG5hdHVyYWwiXSxbMCwxXSxbMSwyXSxbMiwzXSxbNCwzXSxbMCw0XSxbMSw0LCIiLDEseyJzdHlsZSI6eyJib2R5Ijp7Im5hbWUiOiJkYXNoZWQifX19XV0=
\[\begin{tikzcd}
	{\mathcal{C}^\natural_\infty\times (\{0\})^\sharp} && {\mathcal{C}^\natural} \\
	{\mathcal{C}^\natural_\infty\times (\Delta^1)^\sharp} & {\mathcal{D}^\natural_\infty\times (\Delta^1)^\sharp} & {\mathcal{D}^\natural,}
	\arrow[from=1-1, to=2-1]
	\arrow[from=2-1, to=2-2]
	\arrow[from=2-2, to=2-3]
	\arrow[from=1-3, to=2-3]
	\arrow[from=1-1, to=1-3]
	\arrow[dashed, from=2-1, to=1-3]
\end{tikzcd}\]where the functor $\cal D^{\natural}\times\pr{\Delta^{1}}^{\sharp}\to\cal D^{\natural}$
is a $r$-cocartesian natural transformation over $\infty\to i$ which
starts from the inclusion $\cal D_{\infty}\hookrightarrow\cal D$.
The restriction of the filler determines a commutative diagram% https://q.uiver.app/?q=WzAsNCxbMCwwLCJcXG1hdGhjYWx7Q31fXFxpbmZ0eSJdLFsxLDAsIlxcbWF0aGNhbHtDfV9pIl0sWzEsMSwiXFxtYXRoY2Fse0R9X2kuIl0sWzAsMSwiXFxtYXRoY2Fse0R9X1xcaW5mdHkiXSxbMCwxLCJcXHJob15pXyEiXSxbMSwyLCJwIl0sWzAsMywicCIsMl0sWzMsMiwiXFxyaG9eaV8hIiwyXV0=
\[\begin{tikzcd}
	{\mathcal{C}_\infty} & {\mathcal{C}_i} \\
	{\mathcal{D}_\infty} & {\mathcal{D}_i.}
	\arrow["{\rho^i_!}", from=1-1, to=1-2]
	\arrow["p", from=1-2, to=2-2]
	\arrow["p"', from=1-1, to=2-1]
	\arrow["{\rho^i_!}"', from=2-1, to=2-2]
\end{tikzcd}\]The claim is that, for any such choice of functors $\cal D_{\infty}\to\cal D_{i}$
and $\cal C_{\infty}\to\cal C_{i}$, the resulting functor $\phi:\cal C_{\infty}\to\prod_{1\leq i\leq n}\cal C_{i}\times_{\prod_{1\leq i\leq n}\cal D_{i}}\cal D_{\infty}$
is an equivalence of $\infty$-categories.

We begin by proving the essential surjectivity. Let $\pr{\pr{C_{i}}_{i},D}\in\prod_{1\leq i\leq n}\cal C_{i}\times_{\prod_{1\leq i\leq n}\cal D_{i}}\cal D_{\infty}$
be an arbitrary object. By our hypothesis (5), we can find an object
$C\in\cal C_{\infty}$ which admits a $q$-cocartesian morphism $C\to C_{i}$
over $\rho^{i}$ for each $1\leq i\leq n$. By the uniqueness of cocartesian
morphisms, there is an equivalence to $\pr{C_{i}}_{i}\simeq\pr{\rho_{!}^{i}\pr C}_{i}$
in $\prod_{1\leq i\leq n}\cal C_{i}$. Its image in $\prod_{1\leq i\leq n}\cal D_{i}$
determines an equivalence $\pr{\rho_{!}^{i}\pr D}_{i}\simeq\pr{\rho_{!}^{i}\pr{p\pr C}}_{i}$.
By virtue of assumption (3), we obtain an equivalence $g:D\xrightarrow{\simeq}p(C)$
in $\cal D_{\infty}$. Using the fact that $p$ is a categorical fibration,
we can lift the morphism $g$ to an equivalence $\widetilde{g}:\widetilde{C}\xrightarrow{\simeq}C$
in $\cal C$. Therefore, replacing $f_{i}$ by $f_{i}\widetilde{g}$
if necessary, we may assume that $p\pr C=D$. For each $1\leq i\leq n$,
the morphism $f_{i}$ determines a functor $\cal C_{\infty}^{\natural}\times\{0\}^{\sharp}\cup\{C\}^{\sharp}\times\pr{\Delta^{1}}^{\sharp}\to\cal C^{\natural}$.
Since the inclusion
\[
\cal C_{\infty}^{\natural}\times\{0\}^{\sharp}\cup\{C\}^{\sharp}\times\pr{\Delta^{1}}^{\sharp}\hookrightarrow\cal C_{\infty}^{\natural}\times\pr{\Delta^{1}}^{\sharp}
\]
is a trivial cofibration in the cocartesian model structure over $\cal B\pr n$
\cite[Proposition 3.1.2.2]{HTT}, we can find a filler of the diagram
% https://q.uiver.app/?q=WzAsNSxbMCwwLCJcXG1hdGhjYWx7Q31eXFxuYXR1cmFsX1xcaW5mdHlcXHRpbWVzIChcXHswXFx9KV5cXHNoYXJwXFxjdXBcXHtDXFx9Xlxcc2hhcnAgXFx0aW1lcyAoXFxEZWx0YV4xKV5cXHNoYXJwIl0sWzAsMSwiXFxtYXRoY2Fse0N9XlxcbmF0dXJhbF9cXGluZnR5XFx0aW1lcyAoXFxEZWx0YV4xKV5cXHNoYXJwIl0sWzEsMSwiXFxtYXRoY2Fse0R9XlxcbmF0dXJhbF9cXGluZnR5XFx0aW1lcyAoXFxEZWx0YV4xKV5cXHNoYXJwIl0sWzIsMSwiXFxtYXRoY2Fse0R9XlxcbmF0dXJhbCJdLFsyLDAsIlxcbWF0aGNhbHtDfV5cXG5hdHVyYWwiXSxbMCwxXSxbMSwyXSxbMiwzXSxbNCwzXSxbMCw0XSxbMSw0LCIiLDEseyJzdHlsZSI6eyJib2R5Ijp7Im5hbWUiOiJkYXNoZWQifX19XV0=
\[\begin{tikzcd}
	{\mathcal{C}^\natural_\infty\times (\{0\})^\sharp\cup\{C\}^\sharp \times (\Delta^1)^\sharp} && {\mathcal{C}^\natural} \\
	{\mathcal{C}^\natural_\infty\times (\Delta^1)^\sharp} & {\mathcal{D}^\natural_\infty\times (\Delta^1)^\sharp} & {\mathcal{D}^\natural.}
	\arrow[from=1-1, to=2-1]
	\arrow[from=2-1, to=2-2]
	\arrow[from=2-2, to=2-3]
	\arrow[from=1-3, to=2-3]
	\arrow[from=1-1, to=1-3]
	\arrow[dashed, from=2-1, to=1-3]
\end{tikzcd}\]The filler determines a functor $\cal C_{\infty}\to\cal C_{i}$ which
is naturally equivalent to $\rho_{!}^{i}$ over $\cal D_{i}$. So
the functor $\phi':\cal C_{\infty}\to\prod_{1\leq i\leq n}\cal C_{i}\times_{\prod_{1\leq i\leq n}\cal D_{i}}\cal D_{\infty}$
obtained from these fillers is naturally equivalent to $\phi$. By
construction, we have $\phi'\pr C=\pr{\pr{C_{i}}_{i},D}$. This proves
that $\phi$ is essentially surjective.

Next, to prove that $\phi$ is fully faithful, we consider the commutative
diagram % https://q.uiver.app/?q=WzAsMTAsWzEsMSwiXFxtYXRoY2Fse0N9XlxcbmF0dXJhbF9cXGluZnR5XFx0aW1lcyAoXFxEZWx0YV4xKV5cXHNoYXJwIl0sWzEsMywiXFxtYXRoY2Fse0R9XlxcbmF0dXJhbF9cXGluZnR5XFx0aW1lcyAoXFxEZWx0YV4xKV5cXHNoYXJwIl0sWzMsMywiXFxwcm9kX3sxXFxsZXEgaVxcbGVxIG59XFxtYXRoY2Fse0R9XlxcbmF0dXJhbFxcdGltZXMgX3soXFxtYXRoY2Fse0J9KG4pKV5cXHNoYXJwfSh7XFx7XFxpbmZ0eVxcfVxcc3Rhclxce2lcXH19KV5cXHNoYXJwLiJdLFszLDEsIlxccHJvZF97MVxcbGVxIGlcXGxlcSBufVxcbWF0aGNhbHtDfV5cXG5hdHVyYWxcXHRpbWVzIF97KFxcbWF0aGNhbHtCfShuKSleXFxzaGFycH0oe1xce1xcaW5mdHlcXH1cXHN0YXJcXHtpXFx9fSleXFxzaGFycCJdLFs0LDAsIlxccHJvZF97MVxcbGVxIGlcXGxlcSBufVxcbWF0aGNhbHtDfV9pXlxcbmF0dXJhbCJdLFs0LDIsIlxccHJvZF97MVxcbGVxIGlcXGxlcSBufVxcbWF0aGNhbHtEfV9pXlxcbmF0dXJhbCJdLFsyLDAsIlxcbWF0aGNhbHtDfV9cXGluZnR5XlxcbmF0dXJhbFxcdGltZXMgXFx7MVxcfV5cXHNoYXJwIl0sWzIsMiwiXFxtYXRoY2Fse0R9X1xcaW5mdHleXFxuYXR1cmFsXFx0aW1lcyBcXHsxXFx9Xlxcc2hhcnAiXSxbMCwyLCJcXG1hdGhjYWx7Q31fXFxpbmZ0eV5cXG5hdHVyYWxcXHRpbWVzIFxcezBcXH1eXFxzaGFycCJdLFswLDQsIlxcbWF0aGNhbHtEfV9cXGluZnR5XlxcbmF0dXJhbFxcdGltZXMgXFx7MFxcfV5cXHNoYXJwIl0sWzAsMV0sWzEsMl0sWzMsMl0sWzAsM10sWzQsMywiXFxzaW1lcSIsMix7InN0eWxlIjp7InRhaWwiOnsibmFtZSI6Imhvb2siLCJzaWRlIjoiYm90dG9tIn19fV0sWzQsNV0sWzUsMiwiXFxzaW1lcSIsMCx7InN0eWxlIjp7InRhaWwiOnsibmFtZSI6Imhvb2siLCJzaWRlIjoiYm90dG9tIn19fV0sWzYsNF0sWzYsN10sWzcsNV0sWzcsMSwiXFxzaW1lcSJdLFs2LDAsIlxcc2ltZXEiXSxbOCwwLCJcXHNpbWVxIl0sWzksMSwiXFxzaW1lcSJdLFs4LDldLFs4LDMsIiIsMSx7InN0eWxlIjp7InRhaWwiOnsibmFtZSI6Imhvb2siLCJzaWRlIjoidG9wIn19fV0sWzksMiwiIiwxLHsic3R5bGUiOnsidGFpbCI6eyJuYW1lIjoiaG9vayIsInNpZGUiOiJ0b3AifX19XV0=
\[\begin{tikzcd}[scale cd=.65]
	&& {\mathcal{C}_\infty^\natural\times \{1\}^\sharp} && {\prod_{1\leq i\leq n}\mathcal{C}_i^\natural} \\
	& {\mathcal{C}^\natural_\infty\times (\Delta^1)^\sharp} && {\prod_{1\leq i\leq n}\mathcal{C}^\natural\times _{(\mathcal{B}(n))^\sharp}({\{\infty\}\star\{i\}})^\sharp} \\
	{\mathcal{C}_\infty^\natural\times \{0\}^\sharp} && {\mathcal{D}_\infty^\natural\times \{1\}^\sharp} && {\prod_{1\leq i\leq n}\mathcal{D}_i^\natural} \\
	& {\mathcal{D}^\natural_\infty\times (\Delta^1)^\sharp} && {\prod_{1\leq i\leq n}\mathcal{D}^\natural\times _{(\mathcal{B}(n))^\sharp}({\{\infty\}\star\{i\}})^\sharp.} \\
	{\mathcal{D}_\infty^\natural\times \{0\}^\sharp}
	\arrow[from=2-2, to=4-2]
	\arrow[from=4-2, to=4-4]
	\arrow[from=2-4, to=4-4]
	\arrow[from=2-2, to=2-4]
	\arrow["\simeq"', hook', from=1-5, to=2-4]
	\arrow[from=1-5, to=3-5]
	\arrow["\simeq", hook', from=3-5, to=4-4]
	\arrow[from=1-3, to=1-5]
	\arrow[from=1-3, to=3-3]
	\arrow[from=3-3, to=3-5]
	\arrow["\simeq", from=3-3, to=4-2]
	\arrow["\simeq", from=1-3, to=2-2]
	\arrow["\simeq", from=3-1, to=2-2]
	\arrow["\simeq", from=5-1, to=4-2]
	\arrow[from=3-1, to=5-1]
	\arrow[hook, from=3-1, to=2-4]
	\arrow[hook, from=5-1, to=4-4]
\end{tikzcd}\]We wish to show that the back face of this diagram is homotopy cartesian
in the cocartesian model structure in $\SS^{+}$. By \cite[Proposition 2.10]{StUn},
the maps labeled with $\simeq$ are weak equivalences in $\SS^{+}$.
It will therefore suffice to show that the front face is homotopy
cartesian. Note that the front face can be identified with the diagram
% https://q.uiver.app/?q=WzAsNCxbMCwwLCJyXyEoXFxtYXRoY2Fse0N9XlxcbmF0dXJhbCkoXFxpbmZ0eSkiXSxbMCwxLCJyXyEoXFxtYXRoY2Fse0R9XlxcbmF0dXJhbCkoXFxpbmZ0eSkiXSxbMSwwLCJcXHByb2RfezFcXGxlcSBpXFxsZXEgbn1yXyEoXFxtYXRoY2Fse0N9XlxcbmF0dXJhbCkoaSkiXSxbMSwxLCJcXHByb2RfezFcXGxlcSBpXFxsZXEgbn1yXyEoXFxtYXRoY2Fse0R9XlxcbmF0dXJhbCkoaSkiXSxbMCwxXSxbMCwyXSxbMSwzXSxbMiwzXV0=
\[\begin{tikzcd}
	{r_!(\mathcal{C}^\natural)(\infty)} & {\prod_{1\leq i\leq n}r_!(\mathcal{C}^\natural)(i)} \\
	{r_!(\mathcal{D}^\natural)(\infty)} & {\prod_{1\leq i\leq n}r_!(\mathcal{D}^\natural)(i),}
	\arrow[from=1-1, to=2-1]
	\arrow[from=1-1, to=1-2]
	\arrow[from=2-1, to=2-2]
	\arrow[from=1-2, to=2-2]
\end{tikzcd}\]where $r_{!}:\SS^{+}/\cal B\pr n\to\Fun\pr{\{\infty\}\star\inp n^{\circ},\SS^{+}}$
is the left adjoint of the relative nerve functor.

Find a commutative diagram % https://q.uiver.app/?q=WzAsNCxbMCwwLCJyXyEoXFxtYXRoY2Fse0N9XlxcbmF0dXJhbCkiXSxbMSwwLCJGIl0sWzEsMSwiRyJdLFswLDEsInJfIShcXG1hdGhjYWx7RH1eXFxuYXR1cmFsKSJdLFswLDEsIlxcc2ltZXEiXSxbMSwyLCJmIl0sWzMsMiwiXFxzaW1lcSIsMl0sWzAsMywicl8hKHApIiwyXV0=
\[\begin{tikzcd}
	{r_!(\mathcal{C}^\natural)} & F \\
	{r_!(\mathcal{D}^\natural)} & G
	\arrow["\simeq", from=1-1, to=1-2]
	\arrow["f", from=1-2, to=2-2]
	\arrow["\simeq"', from=2-1, to=2-2]
	\arrow["{r_!(p)}"', from=1-1, to=2-1]
\end{tikzcd}\]in $\Fun\pr{\{\infty\}\star\inp n^{\circ},\SS^{+}}$, where $F$ and
$G$ are projectively fibrant functors, $f$ is a projective fibration,
and the horizontal arrows are weak equivalences. According to Lemma
\ref{lem:replacement}, we may assume that the natural transformation
\[
f:\{\infty\}\star\inp n^{\circ}\times[1]\to\SS^{+}
\]
can be factored as 
\[
\{\infty\}\star\inp n^{\circ}\times[1]\xrightarrow{f'}\pr{\sf{Cat}_{\Delta}}_{\fib}\xrightarrow{N^{\natural}}\SS^{+}.
\]
We set $f'_{0}=F'$ and $g'_{0}=G'$, and write $\cal C_{\Delta}=\int F'$
and $\cal D_{\Delta}=\int G'$. We are then reduced to showing that
the square % https://q.uiver.app/?q=WzAsNCxbMCwwLCJOKFxcbWF0aGNhbHtDfV97XFxEZWx0YSxcXGluZnR5fSleXFxuYXR1cmFsIl0sWzEsMCwiXFxwcm9kX3sxXFxsZXEgaVxcbGVxIG59TihcXG1hdGhjYWx7Q31fe1xcRGVsdGEsaX0pXlxcbmF0dXJhbCJdLFsxLDEsIlxccHJvZF97MVxcbGVxIGlcXGxlcSBufU4oXFxtYXRoY2Fse0R9X3tcXERlbHRhLGl9KV5cXG5hdHVyYWwiXSxbMCwxLCJOKFxcbWF0aGNhbHtEfV97XFxEZWx0YSxcXGluZnR5fSleXFxuYXR1cmFsIl0sWzAsMV0sWzEsMl0sWzAsM10sWzMsMl1d
\[\begin{tikzcd}
	{N(\mathcal{C}_{\Delta,\infty})^\natural} & {\prod_{1\leq i\leq n}N(\mathcal{C}_{\Delta,i})^\natural} \\
	{N(\mathcal{D}_{\Delta,\infty})^\natural} & {\prod_{1\leq i\leq n}N(\mathcal{D}_{\Delta,i})^\natural}
	\arrow[from=1-1, to=1-2]
	\arrow[from=1-2, to=2-2]
	\arrow[from=1-1, to=2-1]
	\arrow[from=2-1, to=2-2]
\end{tikzcd}\]of marked simplicial sets is homotopy cartesian. Since the vertical
arrows are fibrations in $\SS_{\mathrm{cocart}}^{+}$ and all the
objects are fibrant, it suffices to show that the map 
\[
N\pr{\cal C_{\Delta,\infty}}^{\natural}\to N\pr{\cal D_{\Delta,\infty}}^{\natural}\times_{\prod_{1\leq i\leq n}N\pr{\cal D_{\Delta,i}}^{\natural}}\prod_{1\leq i\leq n}N\pr{\cal C_{\Delta,i}}^{\natural}
\]
is a weak equivalence in $\SS_{\mathrm{cocart}}^{+}$. This is equivalent
to the condition that the map
\[
\theta':N\pr{\cal C_{\Delta,\infty}}\to N\pr{\cal D_{\Delta,\infty}}\times_{\prod_{1\leq i\leq n}N\pr{\cal D_{\Delta,i}}}\prod_{1\leq i\leq n}N\pr{\cal C_{\Delta,i}}
\]
is an equivalence of $\infty$-categories. 

According to Proposition \ref{prop:relative_nerve_of_Gr_const}, the
map $r^{*}\pr f$ can be identified with the map
\[
N\pr{\int f'}:N\pr{\cal C_{\Delta}}^{\natural}\to N\pr{\cal D_{\Delta}}^{\natural}.
\]
Moreover, since the adjunction $r_{!}\dashv r^{*}$ is a Quillen equivalence,
the maps $\cal C^{\natural}\to r^{*}\pr F$ and $\cal D^{\natural}\to r^{*}\pr G$
are weak equivalences in the cocartesian model structure over $\cal B\pr n$.
So the functor $N\pr{\cal C_{\Delta}}\to N\pr{\cal D_{\Delta}}$ has
all the properties (1)-(5). So the proof of the essential surjectivity
of $\theta$ applies as well to $\theta'$, showing that $\theta'$
is essentially surjective. It remains to verify that $\theta'$ is
fully faithful. For this, it suffices to show that for each pair of
objects $X,Y\in\cal C_{\Delta,\infty}$, the diagram% https://q.uiver.app/?q=WzAsNCxbMCwwLCJcXG1hdGhjYWx7Q31fe1xcRGVsdGEsXFxpbmZ0eX0oWCxZKSJdLFswLDEsIlxcbWF0aGNhbHtEfV97XFxEZWx0YSxcXGluZnR5fShmWCxmWSkiXSxbMSwwLCJcXHByb2RfezFcXGxlcSBpXFxsZXEgbn1cXG1hdGhjYWx7Q31fe1xcRGVsdGEsaX0oKEYnXFxyaG9eaSlYLChGJ1xccmhvXmkpWSkiXSxbMSwxLCJcXHByb2RfezFcXGxlcSBpXFxsZXEgbn1cXG1hdGhjYWx7RH1fe1xcRGVsdGEsaX0oKEcnXFxyaG9eaSlmWCwoRydcXHJob15pKWZZKSJdLFswLDFdLFswLDJdLFsyLDNdLFsxLDNdXQ==
\[\begin{tikzcd}
	{\mathcal{C}_{\Delta,\infty}(X,Y)} & {\prod_{1\leq i\leq n}\mathcal{C}_{\Delta,i}((F'\rho^i)X,(F'\rho^i)Y)} \\
	{\mathcal{D}_{\Delta,\infty}(fX,fY)} & {\prod_{1\leq i\leq n}\mathcal{D}_{\Delta,i}((G'\rho^i)fX,(G'\rho^i)fY)}
	\arrow[from=1-1, to=2-1]
	\arrow[from=1-1, to=1-2]
	\arrow[from=1-2, to=2-2]
	\arrow[from=2-1, to=2-2]
\end{tikzcd}\]of Kan complexes is homotopy cartesian. Consider the commutative diagram
% https://q.uiver.app/?q=WzAsOCxbMCwxLCJcXG1hdGhjYWx7Q31fe1xcRGVsdGEsXFxpbmZ0eX0oWCxZKSJdLFswLDMsIlxcbWF0aGNhbHtEfV97XFxEZWx0YSxcXGluZnR5fShmWCxmWSkiXSxbMiwxLCJcXHByb2RfezFcXGxlcSBpXFxsZXEgbn1cXG1hdGhjYWx7Q31fe1xcRGVsdGEsaX0oKEYnXFxyaG9eaSlYLChGJ1xccmhvXmkpWSkiXSxbMiwzLCJcXHByb2RfezFcXGxlcSBpXFxsZXEgbn1cXG1hdGhjYWx7RH1fe1xcRGVsdGEsaX0oKEcnXFxyaG9eaSkgZlgsKEcnXFxyaG9eaSlmWSkiXSxbMSwwLCJcXG1hdGhjYWx7Q31fXFxEZWx0YShYLFkpIl0sWzMsMCwiXFxwcm9kIF97MVxcbGVxIGlcXGxlcSBufVxcbWF0aGNhbHtDfV9cXERlbHRhKFgsKEYnXFxyaG9eaSlZKSJdLFszLDIsIlxccHJvZF97MVxcbGVxIGlcXGxlcSBufVxcbWF0aGNhbHtEfV97XFxEZWx0YX0oZlgsKEcnXFxyaG9eaSlmWSkuIl0sWzEsMiwiXFxtYXRoY2Fse0R9X3tcXERlbHRhfShmWCxmWSkiXSxbMCwxXSxbMCwyXSxbMiwzXSxbMSwzXSxbMCw0XSxbMiw1XSxbNCw1XSxbMyw2XSxbNCw3XSxbNyw2XSxbMSw3XSxbNSw2XV0=
\[\begin{tikzcd}[column sep=small, scale cd=.75]
	& {\mathcal{C}_\Delta(X,Y)} && {\prod _{1\leq i\leq n}\mathcal{C}_\Delta(X,(F'\rho^i)Y)} \\
	{\mathcal{C}_{\Delta,\infty}(X,Y)} && {\prod_{1\leq i\leq n}\mathcal{C}_{\Delta,i}((F'\rho^i)X,(F'\rho^i)Y)} \\
	& {\mathcal{D}_{\Delta}(fX,fY)} && {\prod_{1\leq i\leq n}\mathcal{D}_{\Delta}(fX,(G'\rho^i)fY).} \\
	{\mathcal{D}_{\Delta,\infty}(fX,fY)} && {\prod_{1\leq i\leq n}\mathcal{D}_{\Delta,i}((G'\rho^i) fX,(G'\rho^i)fY)}
	\arrow[from=2-1, to=4-1]
	\arrow[from=2-1, to=2-3]
	\arrow[from=2-3, to=4-3]
	\arrow[from=4-1, to=4-3]
	\arrow[from=2-1, to=1-2]
	\arrow[from=2-3, to=1-4]
	\arrow[from=1-2, to=1-4]
	\arrow[from=4-3, to=3-4]
	\arrow[from=1-2, to=3-2]
	\arrow[from=3-2, to=3-4]
	\arrow[from=4-1, to=3-2]
	\arrow[from=1-4, to=3-4]
\end{tikzcd}\]The slanted arrows are isomorphisms of simplicial sets, so it suffices
to show that the back face is homotopy cartesian. This follows from
our assumption (4).
\end{proof}
We can now prove Proposition \ref{prop:sec1_main}.
\begin{proof}
[Proof of  Proposition \ref{prop:sec1_main}]Clearly (c-ii) implies
(c-i). Conversely, suppose that condition (c-i) is satisfied. For
each $n\geq1$, there is a unique functor $\cal B\pr n\to N\pr{\Fin_{\ast}}$
which is the identity map on morphisms. Applying Proposition \ref{prop:pre-equivalence}
to the functor $\cal M^{\t}\times_{N\pr{\Fin_{\ast}}}\cal B\pr n\to\cal C\times\cal B\pr n$,
we see that condition (c-ii) holds true.
\end{proof}

\section{\label{sec:Monoidal-Envelopes-of}Monoidal Envelopes of Families
of $\infty$-Operads}

In \cite[Section 2.2.4]{HA}, Lurie introduces the universal procedure
to make an arbitrary $\infty$-operad into a symmetric monoidal $\infty$-category.
The resulting symmetric monoidal $\infty$-category is called the
\textbf{monoidal envelope}. In this section, we will show that given
a family of $\infty$-operads, we can take the monoidal envelope of
each fiber to obtain a family of symmetric monoidal $\infty$-categories.
We will also prove that monoidal envelopes of families of $\infty$-operads
enjoys the expected universal property. 

We remark that the proofs of the results in this section are only
slight modifications of Lurie's original proofs of various results
concerning monoidal envelopes, which can be found in \cite[Section 2.2.4]{HA}.
Nevertheless, for the sake of completeness, we will record the proofs
whenever Lurie's proof does not apply verbatim.

\subsection{Definition and Construction}

In this subsection, we define monoidal envelopes of families of $\infty$-operads
and show that they are again families of $\infty$-operads.

We begin with a generalization of Lurie's construction \cite[Construction 2.2.4.1]{HA}:
\begin{defn}
Let $\cal M^{\t}$ be a generalized $\infty$-operad \cite[Definition 2.3.2.1]{HA}.
We define $\Act\pr{\cal M^{\t}}\subset\Fun\pr{\Delta^{1},\cal M^{\t}}$
to be the full subcategory spanned by the active morphisms. If $p:\cal M^{\t}\to\cal N^{\t}$
is a fibration of generalized $\infty$-operads, the \textbf{$\cal N$-monoidal
envelope} of $\cal M^{\t}$ is defined to be the fiber product
\[
\Env_{\cal N}\pr{\cal M}^{\t}=\cal M^{\t}\times_{\Fun\pr{\{0\},\cal N^{\t}}}\Act\pr{\cal N^{\t}}.
\]
In the case $\cal N^{\t}=N\pr{\Fin_{\ast}}$, we will write $\Env_{\cal N}\pr{\cal M}^{\t}=\Env\pr{\cal M}^{\t}$.
We will regard $\Env_{\cal N}\pr{\cal M}^{\t}$ as an $\infty$-category
over $\cal N^{\t}$ via the composite
\[
\Env_{\cal N}\pr{\cal M}^{\t}\to\Act\pr{\cal N^{\t}}\xrightarrow{\opn{ev}_{1}}\cal N^{\t}.
\]
\end{defn}
Thus, if $p:\cal M^{\t}\to N\pr{\Fin_{\ast}}$ is a generalized $\infty$-operad,
then $\Env\pr{\cal M}^{\t}$ is the $\infty$-category of pairs $\pr{M,\alpha:p\pr M\to\inp k}$,
where $M\in\cal M^{\t}$ and $\alpha$ is an active morphism of $N\pr{\Fin_{\ast}}$.
In particular, the $\infty$-category $\Env\pr{\cal M}=\Env\pr{\cal M}_{\inp 1}^{\t}$
may be identified with $\cal M_{\act}^{\t}$. The intuition here is
that an object $M\in\cal M_{\inp n}^{\t}$ is regarded as a ``formal
tensor product'' of the objects $\rho_{!}^{i}\pr M\in\cal M$.

Note that the (fully faithful) diagonal embedding $\cal N^{\t}\to\Act\pr{\cal N^{\t}}$
induces a fully faithful embedding $\cal M^{\t}\hookrightarrow\Env_{\cal N}\pr{\cal M}^{\t}$,
which partly justifies the terminology ``envelope.'' In fact, the
monoidal envelope enjoys a certain universal property, as we will
see in Subsection \ref{subsec:Universal-Property-of_monoidal_env}.

The goal of this note is to prove the following result, which is a
generalization of \cite[Proposition 2.2.4.4]{HA}.
\begin{prop}
\label{prop:2.2.4.4}Let $\cal C$ be an $\infty$-category and let
$p:\cal M^{\t}\to\cal C\times N\pr{\Fin_{\ast}}$ be a $\cal C$-family
of $\infty$-operads. Let 
\[
q:\Env\pr{\cal M}^{\t}\to\cal C\times N\pr{\Fin_{\ast}}
\]
denote the functor induced by the functors $\Env\pr{\cal M}^{\t}\to N\pr{\Fin_{\ast}}$
and $\Env\pr{\cal M}^{\t}\to\cal M^{\t}\to\cal C$. Then:
\begin{itemize}
\item [(a)]The functor $q$ makes $\Env\pr{\cal M}^{\t}$ into a $\cal C$-family
of $\infty$-operads. 
\item [(b)]A morphism of $\Env\pr{\cal M}^{\t}$ whose image in $\cal M^{\t}$
is inert is $q$-cocartesian. The converse holds if it lies over an
inert morphism in $\cal C\times N\pr{\Fin_{\ast}}$.
\end{itemize}
\end{prop}
\begin{rem}
In the situation of Proposition \ref{prop:2.2.4.4}, the fiber $\Env\pr{\cal M}_{C}^{\t}=\Env\pr{\cal M}^{\t}\times_{\cal C}\{C\}$
over an object $C\in\cal C$ is the monoidal envelope of the $\infty$-operad
$\cal M_{C}^{\t}$, which is a symmetric monoidal $\infty$-category
by \cite[Proposition 2.2.4.4]{HA}. 
\end{rem}
We will return to the proof after a sequence of lemmas.
\begin{lem}
\cite[Lemma 2.2.4.11]{HA}\label{lem:2.2.4.11}Let $p:\cal E\to\cal D$
be a cocartesian fibration of $\infty$-categories. Suppose there
is a full subcategory $\cal C\subset\cal E$ which has the following
properties:

\begin{enumerate}[label=(\roman*)]

\item For each object $D\in\cal D$, the inclusion $\cal C_{D}\subset\cal E_{D}$
admits a left adjoint $L_{D}:\cal E_{D}\to\cal C_{D}$.

\item For each morphism $f:D\to D'$ in $\cal D$, the associated
functor $f_{!}:\cal E_{D}\to\cal E_{D'}$ carries $L_{D}$-equivalences
(i.e., its image under $L_{D}$ is an equivalence) to $L_{D'}$-equivalences.

\end{enumerate}

Let $q=p\vert_{\cal C}:\cal C\to\cal D$ denote the restriction of
$p$. The following holds:
\begin{enumerate}
\item The functor $q=p\vert_{\cal C}:\cal C\to\cal D$ is a cocartesian
fibration.
\item Let $g:D\to D'$ be a morphism in $\cal D$ and $f:C\to C'$ a morphism
in $\cal C$ lifting $g$. Then $f$ is $q$-cocartesian if and only
if the map $g_{!}C\to C'$ is an $L_{D'}$-equivalence, where $g_{!}:\cal E_{D}\to\cal E_{D'}$
is the functor induced by $g$.
\end{enumerate}
\end{lem}
%
\begin{lem}
\cite[Remark 2.2.4.12]{HA}\label{lem:2.2.4.12}Let $p:\cal E\to\cal D$
be a cocartesian fibration of $\infty$-categories and $\cal C\subset\cal E$
a full subcategory. Consider the following conditions for $\cal C$:

\begin{enumerate}[label=(\roman*)]

\item For each object $D\in\cal D$, the inclusion $\cal C_{D}\subset\cal E_{D}$
admits a left adjoint $L_{D}:\cal E_{D}\to\cal C_{D}$.

\item For each morphism $f:D\to D'$ in $\cal D$, the associated
functor $f_{!}:\cal E_{D}\to\cal E_{D'}$ carries $L_{D}$-equivalences
(i.e., its image under $L_{D}$ is an equivalence) to $L_{D'}$-equivalences.

\end{enumerate}

\begin{enumerate}[label=(\roman*')]

\item The inclusion $\cal C\subset\cal E$ admits a left adjoint
$L:\cal E\to\cal C$.

\item The functor $p$ carries each $L$-equivalecne to to an equivalence.

\end{enumerate}

The conditions (i) and (ii) are equivalent to the conditions (i')
and (ii'). Moreover, if these conditions are satisfied, then for each
object $D\in\cal D$, a morphism in $\cal C_{D}$ is an $L$-equivalence
if and only if it is an $L_{D}$-equivalence.
\end{lem}
%
\begin{lem}
\cite[Lemma 2.2.4.13]{HA}\label{lem:2.2.4.13}Let $p:\cal C\to\cal D$
be an inner fibration of $\infty$-categories and $\cal D'\subset\cal D$
a full subcategory. Set $\cal C'=\cal C\times_{\cal D'}\cal D$. Assume
the following:

\begin{enumerate}[label=(\roman*)]

\item The inclusion $\cal D'\subset\cal D$ admits a left adjoint.

\item For each object $C\in\cal C$, the morphism $pC\to D'$ which
exhibits $D$ as a $\cal D'$-localization of $pC$ lifts to a $p$-cocartesian
morphism $C\to C'$.

\end{enumerate}

Then $\cal C'$ is a reflective subcategory of $\cal C$, and a morphism
$f:C\to C'$ in $\cal C$ exhibits $C'$ as a $\cal C'$-localization
of $C$ if and only if $f$ is $p$-cocartesian and the morphism $p\pr f$
exhibits $p\pr{C'}$ as a $\cal D'$-localization of $p\pr C$.
\end{lem}
%
\begin{lem}
\cite[Lemma 2.2.4.14]{HA}\label{lem:2.2.4.14}Let $p:\cal M^{\t}\to\cal N^{\t}$
be a fibration of generalized $\infty$-operads. Set $\cal D=\cal M^{\t}\times_{\Fun\pr{\{0\},\cal N^{\t}}}\Fun\pr{\Delta^{1},\cal N^{\t}}$.
The inclusion
\[
\Env_{\cal N}\pr{\cal M}^{\t}\subset\cal D
\]
admits a left adjoint. Moreover, a morphism $\alpha:D\to D'$ in $\cal D$
exhibits $D'$ as an $\Env_{\cal N}\pr{\cal M}^{\t}$-localization
of $D$ if and only if $D'\in\Env_{\cal N}\pr{\cal M}^{\t}$, the
image of $\alpha$ in $\cal M^{\t}$ is inert, and the image of $\alpha$
in $\cal N^{\t}$ is an equivalence.
\end{lem}
\begin{proof}
We recall from \cite[Lemma 5.2.8.19]{HTT} that the inclusion $\Act\pr{\cal N^{\t}}\subset\Fun\pr{\Delta^{1},\cal N^{\t}}$
admits a left adjoint and that a morphism $\alpha:g\to g'$ in $\Fun\pr{\Delta^{1},\cal N^{\t}}$
corresponding to a square % https://q.uiver.app/?q=WzAsNCxbMCwwLCJcXGJ1bGxldCJdLFsxLDAsIlxcYnVsbGV0Il0sWzEsMSwiXFxidWxsZXQiXSxbMCwxLCJcXGJ1bGxldCJdLFswLDEsImYiXSxbMSwyLCJnJyJdLFszLDIsImYnIiwyXSxbMCwzLCJnIiwyXV0=
\[\begin{tikzcd}
	\bullet & \bullet \\
	\bullet & \bullet
	\arrow["f", from=1-1, to=1-2]
	\arrow["{g'}", from=1-2, to=2-2]
	\arrow["{f'}"', from=2-1, to=2-2]
	\arrow["g"', from=1-1, to=2-1]
\end{tikzcd}\]exhibits $g'$ as an $\Act\pr{\cal N^{\t}}$-localization if and only
if $g'$ is active, $f$ is inert, and $f'$ is an equivalence. 

Now let $\pr{M,g:p\pr M\to N}$ be an object of $\Env_{\cal N}\pr{\cal M}^{\t}$,
and find a morphism $\alpha:g\to g'$ as in the previous paragraph.
According to Lemma \ref{lem:2.2.4.13}, it will suffice to show that
$\alpha$ admits a lift with domain $\pr{M,g}$ which is cocartesian
with respect to the projection $\cal D\to\Fun\pr{\Delta^{1},\cal N^{\t}}$.
Since $f$ is inert and $p$ is a fibration of generalized $\infty$-operads,
we can lift the morphism $f$ to an inert morphism $\beta:M\to M'$
in $\cal M^{\t}$. Then $\pr{\beta,\alpha}$ determines a morphism
$\pr{M,g}\to\pr{M',g'}$ which lifts $\alpha$. Since its image in
$\cal M^{\t}$ is $p$-cocartesian, the morphism $\pr{\alpha,\beta}$
is cocartesian with respect to $\cal D\to\Fun\pr{\Delta^{1},\cal N^{\t}}$.
The claim follows.
\end{proof}
\begin{lem}
\cite[Corollary 2.4.7.12]{HTT}\label{lem:corr_cocart}Let $f:\cal C\to\cal D$
be a functor of $\infty$-categories. The evaluation at $1\in\Delta^{1}$
induces a cocartesian fibration $p:\cal E=\cal C\times_{\Fun\pr{\{0\},\cal D}}\Fun\pr{\Delta^{1},\cal D}\to\cal D$,
and a morphism in $\cal E$ is $p$-cocartesian if and only if its
image in $\cal C$ is an equivalence.
\end{lem}
%
\begin{lem}
\cite[Lemma 2.2.4.15]{HA}\label{lem:2.2.4.15}Let $p:\cal M^{\t}\to\cal N^{\t}$
be a fibration of generalized $\infty$-operads. Set $\cal D=\cal M^{\t}\times_{\Fun\pr{\{0\},\cal N^{\t}}}\Fun\pr{\Delta^{1},\cal N^{\t}}$.
The following holds:
\begin{enumerate}
\item Evaluation at $1\in\Delta^{1}$ induces a cocartesian fibration $q':\cal D\to\cal N^{\t}$. 
\item A morphism in $\cal D$ is $q'$-cocartesian if and only if its image
in $\cal M^{\t}$ is an equivalence.
\item The map $q'$ restricts to a cocartesian fibration $q:\Env_{\cal N}\pr{\cal M}^{\t}\to\cal N^{\t}$.
\item A morphism $f$ in $\Env_{\cal N}\pr{\cal M}^{\t}$ is $q$-cocartesian
if and only if its image in $\cal M^{\t}$ is inert.
\end{enumerate}
\end{lem}
\begin{proof}
Assertions (1) and (2) follow from Lemma \ref{lem:corr_cocart}. Let
now $L:\cal D\to\Env_{\cal N}\pr{\cal M}^{\t}$ denote the left adjoint
of the inclusion, which exists by Lemma \ref{lem:2.2.4.14}. According
to this lemma, the functor $q'$ maps $L$-equivalences to equivalences.
Thus, by Lemmas \ref{lem:2.2.4.11} and \ref{lem:2.2.4.12}, the restriction
$q=q'\vert\Env_{\cal N}\pr{\cal M}^{\t}$ is a cocartesian fibration,
and a morphism $\pr{f,g}:\pr{M,\alpha}\to\pr{M',\alpha'}$ in $\Env_{\cal N}\pr{\cal M}^{\t}$
is $q$-cocartesian if and only if the map $\theta:g_{!}\pr{M,\alpha}\to\pr{M',\alpha'}$
is an $L$-equivalence. Here $g_{!}:\cal D_{X}\to\cal D_{X'}$ is
the functor induced by the morphism $g$. Now up to equivalence, the
map $\theta$ may be identified with the map $\pr{f,\id}:\pr{M,g\circ\alpha}\to\pr{M',\alpha'}$.
So Lemma \ref{lem:2.2.4.14} tells us that the latter condition holds
if and only if $f:M\to M'$ is inert. This proves (4).
\end{proof}
\begin{lem}
\cite[Lemma 2.2.4.16]{HA}\label{lem:2.2.4.16}Let $n\geq0$, and
let $\beta:\inp m\to\inp n$ and $\beta_{i}:\inp{m_{i}}\to\inp 1$
be active morphisms for $1\leq i\leq n$. Suppose we are given a commutative
diagram % https://q.uiver.app/?q=WzAsNCxbMCwwLCJcXGxhbmdsZSBtXFxyYW5nbGUiXSxbMSwwLCJcXGxhbmdsZSBtX2lcXHJhbmdsZSJdLFsxLDEsIlxcbGFuZ2xlIDEgXFxyYW5nbGUiXSxbMCwxLCJcXGxhbmdsZSBuXFxyYW5nbGUiXSxbMCwxLCJcXGdhbW1hX2kiXSxbMSwyLCJcXGJldGFfaSJdLFszLDIsIlxccmhvXmkiLDJdLFswLDMsIlxcYmV0YSIsMl1d
\[\begin{tikzcd}
	{\langle m\rangle} & {\langle m_i\rangle} \\
	{\langle n\rangle} & {\langle 1 \rangle}
	\arrow["{\gamma_i}", from=1-1, to=1-2]
	\arrow["{\beta_i}", from=1-2, to=2-2]
	\arrow["{\rho^i}"', from=2-1, to=2-2]
	\arrow["\beta"', from=1-1, to=2-1]
\end{tikzcd}\]for $1\leq i\leq n$ such that $\gamma_{i}$ is inert. Then the morphisms
$\{\beta\to\beta_{i}\}_{1\leq i\leq n}$ form a limit cone relative
to the map $p=\opn{ev}_{1}:\Act\pr{N\pr{\Fin_{\ast}}}\to N\pr{\Fin_{\ast}}$.
\end{lem}
%
\begin{proof}
[Proof of Proposition \ref{prop:2.2.4.4}]We begin with the first
half of part (b). Let $r:\cal C\times N\pr{\Fin_{\ast}}\to N\pr{\Fin_{\ast}}$
denote the projection. If a morphism $f$ in $\Env\pr{\cal M}^{\t}$
has the property that its image in $\cal M^{\t}$ is inert, then we
know from Lemma \ref{lem:2.2.4.15} that $f$ is $rq$-cocartesian,
and the image of $f$ in $\cal C$ is an equivalence. Thus $q\pr f$
ir $r$-cocartesian. Hence $f$ is $q$-cocartesian.

Next we proceed to (a). We must prove the following.
\begin{enumerate}
\item The functor $q$ is a categorical fibration.
\item Let $\pr{M,\alpha:\inp{n_{M}}\to\inp k}$ be an object of $\Env\pr{\cal M}^{\t}$
with image $C\in\cal C$, and let $\beta:\inp k\to\inp l$ be an inert
map. There is a $q$-cocartesian morphism $\pr{M,\alpha}\to\pr{M',\alpha'}$
lying over $\pr{\id_{C},\beta}$ whose image in $\cal M^{\t}$ is
inert.
\item Let $E\in\Env\pr{\cal M}^{\t}$ and set $\pr{C,\inp m}=q\pr E$. Then
the $q$-cocartesian morphisms $\gamma=\{E\to E_{i}\}_{1\leq i\leq k}$
lying over the morphisms $\{\pr{\id_{C},\rho^{i}}:\pr{C,\inp m}\to\pr{C,\inp 1}\}_{1\leq i\leq k}$
form a $q$-limit cone.
\item Let $k\geq1$, $C\in\cal C$, and $E_{1},\dots,E_{k}\in\Env\pr{\cal M}_{\pr{X,\inp 1}}^{\t}$.
There is an object $E\in\Env\pr{\cal M}_{\pr{C,\inp k}}^{\t}$ and
$q$-cocartesian morphisms $E\to E_{i}$ over $\rho^{i}$.
\end{enumerate}
Note that (2) and the first half of part (b) implies the latter half
of (b).

Assertion (1) is clear, because $q$ is the composite
\[
\Env\pr{\cal M}^{\t}\to\cal C\times\Act\pr{N\pr{\Fin_{\ast}}}\xrightarrow{\id\times\opn{ev}_{1}}\cal C\times N\pr{\Fin_{\ast}},
\]
and the first map is a pullback of $p$ and the second map is a categorical
fibration. For assertion (2), find a commutative diagram % https://q.uiver.app/?q=WzAsNCxbMCwwLCJcXGxhbmdsZSBuX01cXHJhbmdsZSJdLFsxLDAsIlxcbGFuZ2xlIGtcXHJhbmdsZSJdLFsxLDEsIlxcbGFuZ2xlIGxcXHJhbmdsZSJdLFswLDEsIlxcbGFuZ2xlICBuX00nXFxyYW5nbGUiXSxbMCwxLCJcXGFscGhhIl0sWzEsMiwiXFxiZXRhIl0sWzAsMywiXFxnYW1tYSIsMl0sWzMsMiwiXFxhbHBoYSciLDJdXQ==
\[\begin{tikzcd}
	{\langle n_M\rangle} & {\langle k\rangle} \\
	{\langle  n_M'\rangle} & {\langle l\rangle}
	\arrow["\alpha", from=1-1, to=1-2]
	\arrow["\beta", from=1-2, to=2-2]
	\arrow["\gamma"', from=1-1, to=2-1]
	\arrow["{\alpha'}"', from=2-1, to=2-2]
\end{tikzcd}\]in $\Fin_{\ast}$ with $\gamma$ inert and $\alpha'$ active. Since
$p$ is an $\cal C$-family of $\infty$-operads, there is an inert
morphism $f:M\to M'$ in $\cal M^{\t}$ lying over $\pr{\id_{C},\gamma}$.
The maps $f,\gamma,\beta$ determines a morphism $\pr{M,\alpha}\to\pr{M',\alpha'}$
in $\Env\pr{\cal M}^{\t}$ lying over $\pr{\id_{C},\beta}$, which
is $q$-cocartesian by the ``if'' part of part (b). 

For assertion (3), set $\cal D=\cal M^{\t}\times_{\Fun\pr{\{0\},\cal N^{\t}}}\Fun\pr{\Delta^{1},\cal N^{\otimes}}$
and consider the commutative diagram % https://q.uiver.app/?q=WzAsOCxbMCwwLCJcXG9wZXJhdG9ybmFtZXtFbnZ9X3tcXG1hdGhjYWx7Q31cXHRpbWVzIE4oXFxtYXRoc2Z7RmlufV9cXGFzdCl9KFxcbWF0aGNhbHtNfSleXFxvdGltZXMgIl0sWzAsMSwiXFxtYXRoY2Fse0N9XFx0aW1lcyBcXG1hdGhybXtBY3R9KE4oXFxtYXRoc2Z7RmlufV9cXGFzdCkpIl0sWzEsMCwiXFxtYXRoY2Fse0R9Il0sWzEsMSwiXFxtYXRoY2Fse0N9XFx0aW1lcyBcXG9wZXJhdG9ybmFtZXtGdW59KFxcRGVsdGFeMSxOKFxcbWF0aHNme0Zpbn1fXFxhc3QpKSJdLFsyLDAsIlxcbWF0aGNhbHtNfV5cXG90aW1lcyAiXSxbMiwxLCJcXG1hdGhjYWx7Q31cXHRpbWVzIE4oXFxtYXRoc2Z7RmlufV9cXGFzdCkiXSxbMSwyLCJcXG1hdGhjYWx7Q31cXHRpbWVzIE4oXFxtYXRoc2Z7RmlufV9cXGFzdCkiXSxbMCwyLCJcXG1hdGhjYWx7Q31cXHRpbWVzIE4oXFxtYXRoc2Z7RmlufV9cXGFzdCkiXSxbMCwxLCJxXzAiLDJdLFswLDIsIlxcYWxwaGEiXSxbMiwzXSxbMiw0LCJcXGJldGEiXSxbNCw1LCJwIl0sWzMsNSwiXFxvcGVyYXRvcm5hbWV7aWR9X3tcXG1hdGhjYWx7Q319XFx0aW1lcyBcXG9wZXJhdG9ybmFtZXtldn1fMCIsMl0sWzEsM10sWzMsNiwiXFxvcGVyYXRvcm5hbWV7aWR9X3tcXG1hdGhjYWx7Q319XFx0aW1lcyBcXG9wZXJhdG9ybmFtZXtldn1fMSJdLFsxLDcsIlxcb3BlcmF0b3JuYW1le2lkfV97XFxtYXRoY2Fse0N9fVxcdGltZXMgXFxvcGVyYXRvcm5hbWV7ZXZ9XzEiLDJdLFs3LDYsImVxdWFsIiwyXV0=
\[\begin{tikzcd}
	{\operatorname{Env}_{\mathcal{C}\times N(\mathsf{Fin}_\ast)}(\mathcal{M})^\otimes } & {\mathcal{D}} & {\mathcal{M}^\otimes } \\
	{\mathcal{C}\times \mathrm{Act}(N(\mathsf{Fin}_\ast))} & {\mathcal{C}\times \operatorname{Fun}(\Delta^1,N(\mathsf{Fin}_\ast))} & {\mathcal{C}\times N(\mathsf{Fin}_\ast)} \\
	{\mathcal{C}\times N(\mathsf{Fin}_\ast)} & {\mathcal{C}\times N(\mathsf{Fin}_\ast)}
	\arrow["{q_0}"', from=1-1, to=2-1]
	\arrow["\alpha", from=1-1, to=1-2]
	\arrow[from=1-2, to=2-2]
	\arrow["\beta", from=1-2, to=1-3]
	\arrow["p", from=1-3, to=2-3]
	\arrow["{\operatorname{id}_{\mathcal{C}}\times \operatorname{ev}_0}"', from=2-2, to=2-3]
	\arrow[from=2-1, to=2-2]
	\arrow["{\operatorname{id}_{\mathcal{C}}\times \operatorname{ev}_1}", from=2-2, to=3-2]
	\arrow["{\operatorname{id}_{\mathcal{C}}\times \operatorname{ev}_1}"', from=2-1, to=3-1]
	\arrow[equal, from=3-1, to=3-2]
\end{tikzcd}\]whose squares in the top rows are cartesian. We must show that $\gamma$
is an $\pr{\id_{\cal C}\times\opn{ev}_{1}}\circ q_{0}$-limit cone.
Since relative limits are stable under pullbacks \cite[Proposition 4.3.1.5]{HTT},
it suffices to show that $\beta\alpha\gamma$ is a $p$-limit cone
and that $q_{0}\gamma$ is an $\id_{\cal C}\times\opn{ev}_{1}$-limit
cone. First we consider $\beta\alpha\gamma$. Write $E=\pr{M,\alpha:\inp{n_{M}}\to\inp n}$
and $E_{i}=\pr{M_{i},\alpha_{i}:\inp{n_{M_{i}}}\to\inp 1}$. The morphism
$\gamma_{i}:E\to E_{i}$ can be depicted by a diagram% https://q.uiver.app/?q=WzAsNixbMSwwLCJcXGxhbmdsZSBuX01cXHJhbmdsZSJdLFsxLDEsIlxcbGFuZ2xlIG5fe01faX1cXHJhbmdsZSJdLFsyLDAsIlxcbGFuZ2xlIG1cXHJhbmdsZSJdLFsyLDEsIlxcbGFuZ2xlIDFcXHJhbmdsZSJdLFswLDAsIk0iXSxbMCwxLCJNX2kiXSxbMiwzLCJcXHJob15pIl0sWzEsMywiXFxhbHBoYV9pIiwyXSxbMCwyLCJcXGFscGhhIl0sWzAsMSwiXFxiZXRhX2kiLDJdLFs0LDUsImZfaSIsMl1d
\[\begin{tikzcd}
	M & {\langle n_M\rangle} & {\langle m\rangle} \\
	{M_i} & {\langle n_{M_i}\rangle} & {\langle 1\rangle}
	\arrow["{\rho^i}", from=1-3, to=2-3]
	\arrow["{\alpha_i}"', from=2-2, to=2-3]
	\arrow["\alpha", from=1-2, to=1-3]
	\arrow["{\beta_i}"', from=1-2, to=2-2]
	\arrow["{f_i}"', from=1-1, to=2-1]
\end{tikzcd}\]Since $\gamma_{i}$ is $q$-cocartesian, the morphism $f_{i}$ is
inert. Hence $\beta_{i}$ is inert. Since the horizontal arrows are
active, the commutativity of the diagram implies that the map $\beta_{i}:\inp{n_{M}}\to\inp{n_{M_{i}}}$
is the inert map which induces a bijection $\alpha^{-1}\pr i\cong\inp{n_{M_{i}}}^{\circ}$.
Combining this with the fact that $p$ is a $\cal C$-family of $\infty$-operads,
we deduce that the morphisms $\{f_{i}\}_{i}=\{\beta\alpha\gamma_{i}\}_{i}$
form a $p$-limit cone. For $q_{0}\gamma$, we use Lemma \ref{lem:2.2.4.16}.

To prove (4), write $E_{i}=\pr{M_{i},\alpha_{i}:\inp{n_{M_{i}}}\to\inp 1}$.
Let $n_{M}=n_{M_{1}}+\cdots+n_{M_{k}}$, and fix a bijection $\inp{n_{M}}\cong\Vee_{i=1}^{k}\inp{n_{M_{i}}}$.
Since $\cal M_{C}^{\t}$ is an $\infty$-operad, we can find an object
$M\in\cal C_{\pr{C,\inp{n_{M}}}}^{\t}$ and which admits a $p$-cocartesian
morphism $f_{i}:M\to M_{i}$ lying over the inert map $\beta_{i}:\inp{n_{M}}\to\inp{n_{M_{i}}}$
for each $1\leq i\leq k$. Let $\alpha:\inp{n_{M}}\to\inp k$ denote
the function which maps $\inp{n_{M_{i}}}^{\circ}$ to $i\in\inp k^{\circ}$,
and set $E=\pr{M,\alpha:\inp{n_{M}}\to\inp k}$. The triple $\pr{f_{i},\beta_{i},\rho^{i}}$
determines a morphism $E\to E_{i}$ which is $q$-cocartesian by part
(b). This proves (4).
\end{proof}

\subsection{\label{subsec:Universal-Property-of_monoidal_env}Universal Property
of Monoidal Envelopes}

In this subsection, we prove that the monoidal envelope of families
of $\infty$-operads has the expected universal property. The contents
of this subsection will not be used elsewhere in the note.

We begin with a definition.
\begin{defn}
Let $\cal C$ be an $\infty$-category and let $p:\cal M^{\t}\to\cal C\times N\pr{\Fin_{\ast}}$
a $\cal C$-family of $\infty$-operads. Let $q:\cal D^{\t}\to N\pr{\Fin_{\ast}}$
be a symmetric monoidal $\infty$-category. We let $\Fun^{\t}\pr{\Env\pr{\cal M},\cal D}\subset\Fun_{N\pr{\Fin_{\ast}}}\pr{\Env\pr{\cal M}^{\t},\cal D^{\t}}$
denote the full subcategory spanned by the functors which restrict
to a symmetric monoidal functor $\Env\pr{\cal M_{C}}^{\t}\to\cal D^{\t}$
for each $C\in\cal C$. 
\end{defn}
Here is the universal property of monoidal envelopes. The result is
a generalization of \cite[Proposition 2.2.4.9]{HA}.
\begin{prop}
Let $\cal C$ be an $\infty$-category and $p:\cal M^{\t}\to\cal C\times N\pr{\Fin_{\ast}}$
a $\cal C$-family of $\infty$-operads. Let $q:\cal D^{\t}\to N\pr{\Fin_{\ast}}$
be a symmetric monoidal $\infty$-category. The embedding $i:\cal M^{\t}\hookrightarrow\Env\pr{\cal M}^{\t}$
induces an equivalence of $\infty$-categories
\[
\Fun^{\t}\pr{\Env\pr{\cal M},\cal D}\xrightarrow{\simeq}\Alg_{\cal M}\pr{\cal D}.
\]
\end{prop}
\begin{proof}
Let $\cal E^{\t}$ denote the essential image of $i$. In other words,
$\cal E^{\t}$ is the full subcategory of $\Env\pr{\cal M}^{\t}$
spanned by those objects $\pr{M,\alpha:p\pr M\to\inp n}$ such that
$\alpha$ is an equivalence. The restriction map $\Alg_{\cal E}\pr{\cal D}\to\Alg_{\cal M}\pr{\cal D}$
is an equivalence of $\infty$-categories, so it will suffice to show
that the map
\[
\Fun^{\t}\pr{\Env\pr{\cal M},\cal D}\to\Alg_{\cal E}\pr{\cal D}
\]
is a trivial fibration. For this, it suffices to prove the following:

\begin{enumerate}[label=(\alph*)]

\item Every functor $\cal E^{\t}\to\cal D^{\t}$ over $N\pr{\Fin_{\ast}}$
admits a $q$-left Kan extension $\Env\pr{\cal M}^{\t}\to\cal D^{\t}$.

\item A functor $F\in\Fun_{N\pr{\Fin_{\ast}}}\pr{\Env\pr{\cal M},\cal D}$
restricts to a symmetric monoidal functor $\Env\pr{\cal M_{C}}^{\t}\to\cal D^{\t}$
for each $C\in\cal C$ if and only if it is a $q$-left Kan extension
of $F\vert_{\cal E^{\t}}$ and $F\vert_{\cal E^{\t}}$ is a morphism
of generalized $\infty$-operads.

\end{enumerate}

We start from (a). For each object $\pr{M,\alpha:p_{2}\pr M\to\inp n}$
in $\Env\pr{\cal M}^{\t}$, the $\infty$-category $\cal E_{/\pr{M,\alpha}}^{\t}$
has a terminal object, given by the map $\pr{\id_{M},\alpha}:\pr{M,\id_{p_{2}\pr M}}\to\pr{M,\alpha}$
(Lemma \ref{lem:2.2.4.13}). Therefore, if $F:\cal E^{\t}\to\cal D^{\t}$
is an arbitrary functor over $N\pr{\Fin_{\ast}}$, then the diagram
$\cal E_{/\pr{M,\alpha}}^{\t}\to\cal E^{\t}\xrightarrow{F}\cal D^{\t}$
has a $q$-colimit cone if and only if there is a $q$-cocartesian
morphism $F\pr{M,\id_{p_{2}\pr M}}\to D$ in $\cal D$ lifting $\alpha$.
The existence of such a $q$-cocartesian morphism follows from the
fact that $q$ is a cocartesian fibration.

For (b), let $p':\Env\pr{\cal M}^{\t}\to\cal C\times N\pr{\Fin_{\ast}}$
denote the projection. By the discussion in the previous paragraph,
it will suffice to show that $F$ preserves $p'$-cocartesian morphisms
lying over degenerate edges of $\cal X$ if and only if for each object
$\pr{M,\alpha:p_{2}\pr M\to\inp n}\in\Env\pr{\cal M}^{\t}$, the map
$F\pr{M,\id_{p_{2}\pr M}}\to F\pr{M,\alpha}$ is a $q$-cocartesian
morphism. Necessity is clear, since the morphism $\pr{M,\id_{p_{2}\pr M}}\to\pr{M,\alpha}$
is $p'$-cocartesian by Proposition \ref{prop:2.2.4.4}. For sufficiency,
suppose that every morphism of the form $\pr{M,\id_{p_{2}\pr M}}\to\pr{M,\alpha}$
is mapped to a $q$-cocartesian morphism. Let $\pr{f,g}:\pr{M,\alpha:p_{2}\pr M\to\inp n}\to\pr{M',\alpha':p_{2}\pr{M'}\to\inp{n'}}$
be a $p'$-cocartesian morphism in $\Env\pr{\cal M^{\t}}$ lying over
a degenerate edge of $\cal C$. We wish to show that $F\pr{f,g}$
is $q$-cocartesian. By hypothesis, the map $F\pr{M,\id_{p_{2}\pr M}}\to F\pr{M,\alpha}$
is $q$-cocartesian. So it suffice to show that the composite $F\pr{M,\id_{p_{2}\pr M}}\to F\pr{M,\alpha}\to F\pr{M',\alpha'}$
is $q$-cocartesian. We may therefore reduce to the case $\alpha=\id_{p_{2}\pr M}$.
Factor the map $g:p_{2}\pr M\to\inp{n'}$ as 
\[
p_{2}\pr M\xrightarrow{g'}\inp{n''}\xrightarrow{g''}\inp{n'},
\]
where $g'$ is inert and $g''$ is active. Lift $\pr{\id_{p_{1}\pr M},g'}$
to a $p$-cocartesian morphism $f':M\to M''$ in $\cal M^{\t}$. The
morphism $\pr{f',g'}:\pr{M,\id_{p_{2}\pr M}}\to\pr{M'',\id_{\inp{n''}}}$
is $p'$-cocartesian, so we can find a triangle in $\Env\pr{\cal M_{p_{1}\pr M}}^{\t}$
which we depict as % https://q.uiver.app/?q=WzAsNixbMCwzLCJwXzIoTSkiXSxbMSwyLCJcXGxhbmdsZSBuJydcXHJhbmdsZSJdLFsyLDMsIlxcbGFuZ2xlIG4nXFxyYW5nbGUiXSxbMiwxLCJwXzIoTScpIl0sWzEsMCwicF8yKE0nJykiXSxbMCwxLCJwXzIoTSkiXSxbMCwxLCJnJyJdLFsxLDIsImcnJyJdLFswLDIsInAoZykiLDJdLFszLDIsIlxcYWxwaGEnIl0sWzQsMywicChmJycpIl0sWzUsNCwicChmJykiXSxbNCwxLCJlcXVhbCIsMix7ImxhYmVsX3Bvc2l0aW9uIjo3MH1dLFs1LDAsImVxdWFsIiwyXSxbNSwzLCJwKGYpIl1d
\[\begin{tikzcd}
	& {p_2(M'')} \\
	{p_2(M)} && {p_2(M')} \\
	& {\langle n''\rangle} \\
	{p_2(M)} && {\langle n'\rangle}
	\arrow["{g'}", from=4-1, to=3-2]
	\arrow["{g''}", from=3-2, to=4-3]
	\arrow["{p(g)}"', from=4-1, to=4-3]
	\arrow["{\alpha'}", from=2-3, to=4-3]
	\arrow["{p(f'')}", from=1-2, to=2-3]
	\arrow["{p(f')}", from=2-1, to=1-2]
	\arrow[equal, from=1-2, to=3-2]
	\arrow[equal, from=2-1, to=4-1]
	\arrow["{p(f)}", from=2-1, to=2-3]
\end{tikzcd}\]The morphism $f''$ is both active and inert, so it is an equivalence.
Thus, by hypothesis, the map $F\pr{f'',g''}$ is a $q$-cocartesian.
The morphism $\pr{f',g'}$ is also $q$-cocartesian, since it is an
inert morphism of $\cal E^{\t}$. Hence $F\pr{f,g}$ is $q$-cocartesian,
as required.
\end{proof}

\section{\label{sec:Direct_sum}The Fiberwise Direct Sum Functor}

Let $\cal O^{\t}$ be an $\infty$-operad. Given objects $X_{1},\dots,X_{n}\in\cal O$,
we can find an object $X\in\cal O_{\inp n}^{\t}$ which admits a $p$-cocartesian
morphism $X\to X_{i}$ over $\rho^{i}:\inp n\to\inp 1$ for each $i$.
Such an object is denoted by $X_{1}\oplus\cdots\oplus X_{n}$ \cite[Remark 2.1.1.15]{HA}.
By inspection, the object $X_{1}\oplus\cdots\oplus X_{n}$ is the
equivalent to the image of the object $\pr{X_{1},\dots,X_{n}}\in\pr{\cal O_{\act}^{\t}}^{n}$
under the tensor product of $\Env\pr{\cal O}=\cal O_{\act}^{\t}$.
So the operation $\oplus$ can be made functorial using monoidal envelopes.
In fact, we can do it fiberwise:
\begin{defn}
Let $\cal C$ be an $\infty$-category, let $\cal M^{\t}\to\cal C\times N\pr{\Fin_{\ast}}$
be a $\cal C$-family of $\infty$-operads, and let $n\geq1$ be a
positive integer. We define a functor
\[
\bigoplus_{i=1}^{n}:\pr{\cal M_{\act}^{\t}}^{n}\times_{\cal C^{n}}\cal C=\cal M_{\act}^{\t}\times_{\cal C}\cdots\times_{\cal C}\cal M_{\act}^{\t}\to\cal M_{\act}^{\t}
\]
over $\cal C$, well-defined up to natural equivalence over $\cal C$,
as follows: According to Proposition \ref{prop:sec1_main}, there
is an equivalence of $\infty$-categories
\[
\Env\pr{\cal M}_{\inp n}^{\t}\xrightarrow{\simeq}\pr{\cal M_{\act}^{\t}}^{n}\times_{\cal C^{n}}\cal C
\]
over $\cal C$, obtained from the functors $\rho_{!}^{i}:\Env\pr{\cal M}_{\inp n}^{\t}\to\Env\pr{\cal M}_{\inp 1}^{\t}=\cal M_{\act}^{\t}$.
The functors $\Env\pr{\cal M}_{\inp n}^{\t}\to\cal C$ and $\pr{\cal M_{\act}^{\t}}^{n}\times_{\cal C^{n}}\cal C\to\cal C$
are categorical fibrations, so the above equivalence admits an inverse
equivalence
\[
\pr{\cal M_{\act}^{\t}}^{n}\times_{\cal C^{n}}\cal C\xrightarrow{\simeq}\Env\pr{\cal M}_{\inp n}^{\t}
\]
over $\cal C$. The functor $\bigoplus_{i=1}^{n}$ is the composite
\[
\pr{\cal M_{\act}^{\t}}^{n}\times_{\cal C^{n}}\cal C\xrightarrow{\simeq}\Env\pr{\cal M}_{\inp n}^{\t}\xrightarrow{\text{forget}}\cal M_{\act}^{\t}.
\]
\end{defn}
In this section, we will prove two important properties of the direct
sum functor: Its universal property and the interaction with slices.

\subsection{\label{subsec:Universal-Property-of}Universal Property of the Direct
Sum Functor }

In this subsection, we will characterize the direct sum functor by
a certain universal property (Corollary \ref{cor:oplus_p-limit}).
\begin{prop}
\label{prop:inverting_weak_equiv_of_pairs}Let $\cal M$ be a model
category. Suppose we are given a commutative diagram % https://q.uiver.app/?q=WzAsNCxbMCwwLCJFIl0sWzEsMCwiRSciXSxbMSwxLCJCJyJdLFswLDEsIkIiXSxbMCwxLCJmIl0sWzEsMiwicCciLDAseyJzdHlsZSI6eyJoZWFkIjp7Im5hbWUiOiJlcGkifX19XSxbMCwzLCJwIiwyLHsic3R5bGUiOnsiaGVhZCI6eyJuYW1lIjoiZXBpIn19fV0sWzMsMiwicSIsMix7InN0eWxlIjp7ImhlYWQiOnsibmFtZSI6ImVwaSJ9fX1dLFszLDIsIlxcc2ltZXEiXSxbMCwxLCJcXHNpbWVxIiwyXV0=
\[\begin{tikzcd}
	E & {E'} \\
	B & {B'}
	\arrow["f", from=1-1, to=1-2]
	\arrow["{p'}", two heads, from=1-2, to=2-2]
	\arrow["p"', two heads, from=1-1, to=2-1]
	\arrow["q"', two heads, from=2-1, to=2-2]
	\arrow["\simeq", from=2-1, to=2-2]
	\arrow["\simeq"', from=1-1, to=1-2]
\end{tikzcd}\]in $\cal M$. Assume the following:
\begin{enumerate}
\item The maps $p,p',q$ are fibrations.
\item The maps $f$ and $q$ are weak equivalences.
\item The object $E'$ is cofibrant. 
\item The map $q$ has a section $s:B'\to B$.
\end{enumerate}
Then there is a map $g:E'\to E$ rendering the diagram % https://q.uiver.app/?q=WzAsNCxbMCwwLCJFJyJdLFswLDEsIkInIl0sWzEsMCwiRSJdLFsxLDEsIkIiXSxbMCwxLCJwJyJdLFswLDIsImciXSxbMSwzLCJzIiwyXSxbMiwzLCJwIl1d
\[\begin{tikzcd}
	{E'} & E \\
	{B'} & B
	\arrow["{p'}", from=1-1, to=2-1]
	\arrow["g", from=1-1, to=1-2]
	\arrow["s"', from=2-1, to=2-2]
	\arrow["p", from=1-2, to=2-2]
\end{tikzcd}\]commutative, such that the composite $fg:E'\to E'$ is homotopic to
the identity in $\cal M_{/B'}$.
\end{prop}
\begin{proof}
Consider the diagram % https://q.uiver.app/?q=WzAsNixbMiwwLCJFIl0sWzIsMSwiQiJdLFsxLDEsIkInIl0sWzEsMCwiRSciXSxbMCwwLCJFIl0sWzAsMSwiQiJdLFswLDEsInAiLDJdLFsyLDEsInMiLDJdLFszLDIsInAnIiwyXSxbNSwyLCJxIiwyXSxbNCw1LCJwIiwyXSxbNCwzLCJmIl0sWzMsMCwiIiwyLHsic3R5bGUiOnsiYm9keSI6eyJuYW1lIjoiZGFzaGVkIn19fV1d
\[\begin{tikzcd}
	E & {E'} & E \\
	B & {B'} & B
	\arrow["p"', from=1-3, to=2-3]
	\arrow["s"', from=2-2, to=2-3]
	\arrow["{p'}"', from=1-2, to=2-2]
	\arrow["q"', from=2-1, to=2-2]
	\arrow["p"', from=1-1, to=2-1]
	\arrow["f", from=1-1, to=1-2]
	\arrow[dashed, from=1-2, to=1-3]
\end{tikzcd}\]in $\cal M_{/B'}$. Since $p$ is a fibration between fibrant objects
and $E'$ is cofibrant, \cite[Proposition A.2.3.1]{HTT} shows that
there is a map $g:E'\to E$ rendering the diagram commutative, such
that $[g][f]=[\id_{E}]$ in $\ho\pr{\cal M_{/B'}}$. Since $[f]$
is an isomorphism in $\ho\pr{\cal M_{/B'}}$, the uniqueness of inverses
implies that $[f][g]=[\id_{E'}]$ in $\ho\pr{\cal M_{/B'}}$. Since
$E'$ is a fibrant-cofibrant object of $\cal M_{/B'}$, we deduce
that $fg$ is homotopic to the identity in $\cal M_{/B'}$.
\end{proof}
\begin{defn}
Let $n\geq1$. Given integers $m_{1},\dots,m_{n}\geq0$, we shall
identify the pointed set $\Vee_{i=1}^{n}\inp{m_{i}}$ with the set
$\inp{m_{1}+\cdots+m_{n}}$ via the map
\[
\Vee_{i=1}^{n}\inp{m_{i}}\to\inp{m_{1}+\cdots+m_{n}}
\]
which maps $k\in\Vee_{i=1}^{n}\inp{m_{i}}$ to $\sum_{j<i}m_{j}+k$.
The maps $\Vee_{i=1}^{n}\inp{m_{i}}\to\inp{m_{i}}$ define an inert
natural transformation
\[
h_{i}:N\pr{\Fin_{\ast}}_{\act}^{n}\times\Delta^{1}\to N\pr{\Fin_{\ast}}.
\]
\end{defn}
\begin{prop}
\label{prop:oplus_p-limit}Let $n\ge1$, let $\cal C$ be an $\infty$-category,
and let $p:\cal M^{\t}\to\cal C\times N\pr{\Fin_{\ast}}$ be a $\cal C$-family
of $\infty$-operads. We can construct the functor $\bigoplus_{1\leq i\leq n}:\pr{\cal M_{\act}^{\t}}^{n}\times_{\cal C^{n}}\cal C\to\cal M^{\t}$
so that for each $1\leq i\leq n$, there is an inert natural transformation
$\bigoplus_{1\leq i\leq n}\to\opn{pr}_{i}$ rendering the diagram
% https://q.uiver.app/?q=WzAsNCxbMCwwLCIoKFxcbWF0aGNhbHtNfV5cXG90aW1lc197XFxtYXRocm17YWN0fX0pXm5cXHRpbWVzIF97XFxtYXRoY2Fse0N9Xm59XFxtYXRoY2Fse0N9KVxcdGltZXMgXFxEZWx0YSBeMSJdLFsxLDAsIlxcbWF0aGNhbHtNfV5cXG90aW1lcyAiXSxbMSwxLCJcXG1hdGhjYWx7Q31cXHRpbWVzIE4oXFxtYXRoc2Z7RmlufV9cXGFzdCkiXSxbMCwxLCJcXG1hdGhjYWx7Q31cXHRpbWVzIE4oXFxtYXRoc2Z7RmlufV9cXGFzdCApXm5fe1xcbWF0aHJte2FjdH19XFx0aW1lcyBcXERlbHRhXjEiXSxbMCwxLCJcXHdpZGV0aWxkZXtofV9pIl0sWzEsMiwicCJdLFszLDIsIlxcb3BlcmF0b3JuYW1le2lkfV97XFxtYXRoY2Fse0N9fVxcdGltZXMgaF9pIiwyXSxbMCwzXV0=
\[\begin{tikzcd}
	{((\mathcal{M}^\otimes_{\mathrm{act}})^n\times _{\mathcal{C}^n}\mathcal{C})\times \Delta ^1} & {\mathcal{M}^\otimes } \\
	{\mathcal{C}\times N(\mathsf{Fin}_\ast )^n_{\mathrm{act}}\times \Delta^1} & {\mathcal{C}\times N(\mathsf{Fin}_\ast)}
	\arrow["{\widetilde{h}_i}", from=1-1, to=1-2]
	\arrow["p", from=1-2, to=2-2]
	\arrow["{\operatorname{id}_{\mathcal{C}}\times h_i}"', from=2-1, to=2-2]
	\arrow[from=1-1, to=2-1]
\end{tikzcd}\]commutative.
\end{prop}
\begin{proof}
We begin with the construction of the functor $\bigoplus_{1\leq i\leq n}$.
For each $1\leq i\leq n$, there is an inert natural transformation
$g_{i}:\Env\pr{N\pr{\Fin_{\ast}}}_{\inp n}^{\t}\times\Delta^{1}\to\Env\pr{N\pr{\Fin_{\ast}}}^{\t}$
from the inclusion to the functor 
\[
\pr{\alpha:\inp k\to\inp n}\to\pr{\alpha^{-1}\pr i_{\ast}\to\inp 1},
\]
where $\alpha^{-1}\pr i_{\ast}$ denotes the unique object $\inp m\in\Fin_{\ast}$
which admits an order-preserving bijection $\alpha^{-1}\pr i\cong\inp m^{\circ}$.
This natural transformation covers $\rho^{i}:\inp n\to\inp 1$. Thus
we may construct the functor $\rho_{!}^{i}:\Env\pr{N\pr{\Fin_{\ast}}}_{\inp n}^{\t}\to N\pr{\Fin_{\ast}}_{\act}$
by $\rho^{i}\pr{\alpha:\inp k\to\inp n}=\alpha^{-1}\pr i_{\ast}$.
Since the functor $p$ is a categorical fibration, it induces a fibration
$\Env\pr{\cal M}^{\t}\to\cal C\times\Env\pr{N\pr{\Fin_{\ast}}}^{\t}$
of generalized $\infty$-operads. Therefore, we can find an inert
natural transformation $\Env\pr{\cal M}_{\inp n}^{\t}\times\Delta^{1}\to\Env\pr{\cal M}^{\t}$
rendering the diagram % https://q.uiver.app/?q=WzAsNSxbMCwwLCJcXG9wZXJhdG9ybmFtZXtFbnZ9KFxcbWF0aGNhbHtNfSleXFxvdGltZXMgX3tcXGxhbmdsZSBuXFxyYW5nbGV9XFx0aW1lcyBcXHswXFx9Il0sWzIsMCwiXFxvcGVyYXRvcm5hbWV7RW52fShcXG1hdGhjYWx7TX0pXlxcb3RpbWVzICJdLFsyLDEsIlxcbWF0aGNhbHtDfVxcdGltZXMgXFxvcGVyYXRvcm5hbWV7RW52fShOKFxcbWF0aHNme0Zpbn1fXFxhc3QpKV5cXG90aW1lcyAiXSxbMCwxLCJcXG9wZXJhdG9ybmFtZXtFbnZ9KFxcbWF0aGNhbHtNfSleXFxvdGltZXMgX3tcXGxhbmdsZSBuXFxyYW5nbGV9XFx0aW1lcyBcXERlbHRhXjEiXSxbMSwxLCJcXG1hdGhjYWx7Q31cXHRpbWVzIFxcb3BlcmF0b3JuYW1le0Vudn0oTihcXG1hdGhzZntGaW59X1xcYXN0KSleXFxvdGltZXMgX3tcXGxhbmdsZSBuXFxyYW5nbGV9XFx0aW1lcyBcXERlbHRhXjEiXSxbMSwyXSxbMCwzXSxbMywxLCIiLDEseyJzdHlsZSI6eyJib2R5Ijp7Im5hbWUiOiJkYXNoZWQifX19XSxbMyw0XSxbNCwyLCJcXG9wZXJhdG9ybmFtZXtpZH1fe1xcbWF0aGNhbHtDfX1cXHRpbWVzIGdfaSIsMl0sWzAsMV1d
\[\begin{tikzcd}
	{\operatorname{Env}(\mathcal{M})^\otimes _{\langle n\rangle}\times \{0\}} && {\operatorname{Env}(\mathcal{M})^\otimes } \\
	{\operatorname{Env}(\mathcal{M})^\otimes _{\langle n\rangle}\times \Delta^1} & {\mathcal{C}\times \operatorname{Env}(N(\mathsf{Fin}_\ast))^\otimes _{\langle n\rangle}\times \Delta^1} & {\mathcal{C}\times \operatorname{Env}(N(\mathsf{Fin}_\ast))^\otimes }
	\arrow[from=1-3, to=2-3]
	\arrow[from=1-1, to=2-1]
	\arrow[dashed, from=2-1, to=1-3]
	\arrow[from=2-1, to=2-2]
	\arrow["{\operatorname{id}_{\mathcal{C}}\times g_i}"', from=2-2, to=2-3]
	\arrow[from=1-1, to=1-3]
\end{tikzcd}\]commutative. We use this inert natural transformation to define the
functor $\rho_{!}^{i}:\Env\pr{\cal M}_{\inp n}^{\t}\to\cal M_{\act}^{\t}$.
This will ensure that the diagram

% https://q.uiver.app/?q=WzAsNCxbMCwxLCJcXG1hdGhjYWx7Q31cXHRpbWVzIChOKFxcbWF0aHNme0Zpbn1fXFxhc3QpX3tcXG1hdGhybXthY3R9fSlebiJdLFsyLDEsIlxcbWF0aGNhbHtDfVxcdGltZXMgXFxvcGVyYXRvcm5hbWV7RW52fShOKFxcbWF0aHNme0Zpbn1fXFxhc3QpKV5cXG90aW1lcyBfe1xcbGFuZ2xlIG4gXFxyYW5nbGV9Il0sWzIsMCwiXFxvcGVyYXRvcm5hbWV7RW52fShcXG1hdGhjYWx7TX0pXlxcb3RpbWVzIF97XFxsYW5nbGUgbiBcXHJhbmdsZX0iXSxbMCwwLCIoXFxtYXRoY2Fse019Xlxcb3RpbWVzIF97XFxtYXRocm17YWN0fX0pXm5cXHRpbWVzIF97XFxtYXRoY2Fse0N9Xm59XFxtYXRoY2Fse0N9Il0sWzMsMF0sWzEsMCwiXFxvcGVyYXRvcm5hbWV7aWR9X3tcXG1hdGhjYWx7Q319XFx0aW1lcyhcXHJob15pXyEpX3tpPTF9Xm4iXSxbMiwxXSxbMiwzLCIoXFxyaG9eaV8hKV97aT0xfV5uIiwyXV0=
\[\begin{tikzcd}
	{(\mathcal{M}^\otimes _{\mathrm{act}})^n\times _{\mathcal{C}^n}\mathcal{C}} && {\operatorname{Env}(\mathcal{M})^\otimes _{\langle n \rangle}} \\
	{\mathcal{C}\times (N(\mathsf{Fin}_\ast)_{\mathrm{act}})^n} && {\mathcal{C}\times \operatorname{Env}(N(\mathsf{Fin}_\ast))^\otimes _{\langle n \rangle}}
	\arrow[from=1-1, to=2-1]
	\arrow["{\operatorname{id}_{\mathcal{C}}\times(\rho^i_!)_{i=1}^n}", from=2-3, to=2-1]
	\arrow[from=1-3, to=2-3]
	\arrow["{(\rho^i_!)_{i=1}^n}"', from=1-3, to=1-1]
\end{tikzcd}\]is commutative. Now the functor $\pr{\rho_{!}^{i}}_{i=1}^{n}:\Env\pr{N\pr{\Fin_{\ast}}}_{\inp n}^{\t}\to\pr{\pr{N\pr{\Fin_{\ast}}}_{\act}}^{n}$
is a trivial fibration, and it has a section $\phi:\pr{N\pr{\Fin_{\ast}}_{\act}}^{n}\to\Env\pr{N\pr{\Fin_{\ast}}}_{\inp n}^{\t}$
given by 
\[
\pr{\inp{k_{i}}}_{i=1}^{n}\mapsto\pr{\Vee_{i=1}^{n}\inp{k_{i}}\to\Vee_{i=1}^{n}\inp 1=\inp n}.
\]
Applying Proposition \ref{prop:inverting_weak_equiv_of_pairs} to
the commutative diagram (\ref{d1}) and the section $\id_{\cal C}\times\phi$,
we can find a functor $\Phi:\pr{\cal M_{\act}^{\t}}^{n}\to\Env\pr{\cal M}_{\inp n}^{\t}$
with the following properties:
\begin{itemize}
\item The diagram % https://q.uiver.app/?q=WzAsNCxbMCwxLCJcXG1hdGhjYWx7Q31cXHRpbWVzIChOKFxcbWF0aHNme0Zpbn1fXFxhc3QpX3tcXG1hdGhybXthY3R9fSlebiJdLFsyLDEsIlxcbWF0aGNhbHtDfVxcdGltZXMgXFxvcGVyYXRvcm5hbWV7RW52fShOKFxcbWF0aHNme0Zpbn1fXFxhc3QpKV5cXG90aW1lcyBfe1xcbGFuZ2xlIG4gXFxyYW5nbGV9Il0sWzIsMCwiXFxvcGVyYXRvcm5hbWV7RW52fShcXG1hdGhjYWx7TX0pXlxcb3RpbWVzIF97XFxsYW5nbGUgbiBcXHJhbmdsZX0iXSxbMCwwLCIoXFxtYXRoY2Fse019Xlxcb3RpbWVzIF97XFxtYXRocm17YWN0fX0pXm5cXHRpbWVzIF97XFxtYXRoY2Fse0N9Xm59XFxtYXRoY2Fse0N9Il0sWzMsMF0sWzAsMSwiXFxvcGVyYXRvcm5hbWV7aWR9X3tcXG1hdGhjYWx7Q319XFx0aW1lc1xccGhpIiwyXSxbMiwxXSxbMywyLCJcXFBoaSJdXQ==
\[\begin{tikzcd}
	{(\mathcal{M}^\otimes _{\mathrm{act}})^n\times _{\mathcal{C}^n}\mathcal{C}} && {\operatorname{Env}(\mathcal{M})^\otimes _{\langle n \rangle}} \\
	{\mathcal{C}\times (N(\mathsf{Fin}_\ast)_{\mathrm{act}})^n} && {\mathcal{C}\times \operatorname{Env}(N(\mathsf{Fin}_\ast))^\otimes _{\langle n \rangle}}
	\arrow[from=1-1, to=2-1]
	\arrow["{\operatorname{id}_{\mathcal{C}}\times\phi}"', from=2-1, to=2-3]
	\arrow[from=1-3, to=2-3]
	\arrow["\Phi", from=1-1, to=1-3]
\end{tikzcd}\]is commutative. 
\item The composite $\pr{\rho_{!}^{i}}_{1\leq i\leq n}\circ\Phi$ is naturally
equivalent over $\cal C\times\pr{N\pr{\Fin_{\ast}}_{\act}}^{n}$ to
the identity functor. 
\end{itemize}
We will construct the functor $\bigoplus_{1\leq i\leq n}$ as the
composite
\[
\pr{\cal M_{\act}^{\t}}^{n}\times_{\cal C^{n}}\cal C\xrightarrow{\Phi}\Env\pr{\cal M}_{\inp n}^{\t}\xrightarrow{\text{forget}}\cal M^{\t}.
\]

Next, we show that, with our construction of the functor $\bigoplus_{1\leq i\leq n}$,
there is an inert natural transformation $\widetilde{h}_{i}:\pr{\pr{\cal M_{\act}^{\t}}^{n}\times_{\cal C^{n}}\cal C}\times\Delta^{1}\to\cal M^{\t}$
rendering the diagram in the statement commutative. Since $\opn{pr}_{i}$
is naturally equivalent over $\cal C\times N\pr{\Fin_{\ast}}$ to
the composite $\rho_{!}^{i}\Phi$, it suffices (by \cite[Proposition A.2.3.1]{HTT})
to find an inert natural transformation rendering the diagram % https://q.uiver.app/?q=WzAsNSxbMCwwLCIoKFxcbWF0aGNhbHtNfV5cXG90aW1lc197XFxtYXRocm17YWN0fX0pXm5cXHRpbWVzIF97XFxtYXRoY2Fse0N9Xm59XFxtYXRoY2Fse0N9KVxcdGltZXMgXFxwYXJ0aWFsXFxEZWx0YV4xIl0sWzAsMSwiKChcXG1hdGhjYWx7TX1eXFxvdGltZXNfe1xcbWF0aHJte2FjdH19KV5uXFx0aW1lcyBfe1xcbWF0aGNhbHtDfV5ufVxcbWF0aGNhbHtDfSlcXHRpbWVzIFxcRGVsdGFeMSJdLFsxLDEsIlxcbWF0aGNhbHtDfVxcdGltZXMgKE4oXFxtYXRoc2Z7RmlufV9cXGFzdClfe1xcbWF0aHJte2FjdH19KV5uXFx0aW1lcyBcXERlbHRhXjEiXSxbMiwxLCJcXG1hdGhjYWx7Q31cXHRpbWVzIE4oXFxtYXRoc2Z7RmlufV9cXGFzdCkiXSxbMiwwLCJcXG1hdGhjYWx7TX1eXFxvdGltZXMgIl0sWzAsMV0sWzEsMl0sWzIsMywiXFxvcGVyYXRvcm5hbWV7aWR9X3tcXG1hdGhjYWx7Q319XFx0aW1lcyBoX2kiLDJdLFs0LDMsInAiXSxbMCw0LCIoXFxiaWdvcGx1c197aT0xfV5uLFxccmhvXmlfIVxcUGhpICkiXSxbMSw0LCIiLDIseyJzdHlsZSI6eyJib2R5Ijp7Im5hbWUiOiJkYXNoZWQifX19XV0=
\[\begin{tikzcd}
	{((\mathcal{M}^\otimes_{\mathrm{act}})^n\times _{\mathcal{C}^n}\mathcal{C})\times \partial\Delta^1} && {\mathcal{M}^\otimes } \\
	{((\mathcal{M}^\otimes_{\mathrm{act}})^n\times _{\mathcal{C}^n}\mathcal{C})\times \Delta^1} & {\mathcal{C}\times (N(\mathsf{Fin}_\ast)_{\mathrm{act}})^n\times \Delta^1} & {\mathcal{C}\times N(\mathsf{Fin}_\ast)}
	\arrow[from=1-1, to=2-1]
	\arrow[from=2-1, to=2-2]
	\arrow["{\operatorname{id}_{\mathcal{C}}\times h_i}"', from=2-2, to=2-3]
	\arrow["p", from=1-3, to=2-3]
	\arrow["{(\bigoplus_{i=1}^n,\rho^i_!\Phi )}", from=1-1, to=1-3]
	\arrow[dashed, from=2-1, to=1-3]
\end{tikzcd}\]commutative. The outer rectangle is equal to that of the diagram % https://q.uiver.app/?q=WzAsNyxbMCwwLCIoKFxcbWF0aGNhbHtNfV5cXG90aW1lc197XFxtYXRocm17YWN0fX0pXm5cXHRpbWVzIF97XFxtYXRoY2Fse0N9Xm59XFxtYXRoY2Fse0N9KVxcdGltZXMgXFxwYXJ0aWFsXFxEZWx0YV4xIl0sWzAsMSwiKChcXG1hdGhjYWx7TX1eXFxvdGltZXNfe1xcbWF0aHJte2FjdH19KV5uXFx0aW1lc197XFxtYXRoY2Fse0N9Xm59XFxtYXRoY2Fse0N9KVxcdGltZXMgIFxcRGVsdGFeMSJdLFsyLDEsIlxcbWF0aGNhbHtDfVxcdGltZXMgXFxvcGVyYXRvcm5hbWV7RW52fShOKFxcbWF0aHNme0Zpbn1fXFxhc3QpKV5cXG90aW1lcyBfe1xcbGFuZ2xlIG5cXHJhbmdsZX1cXHRpbWVzIFxcRGVsdGFeMSJdLFszLDEsIlxcbWF0aGNhbHtDfVxcdGltZXMgTihcXG1hdGhzZntGaW59X1xcYXN0KSJdLFszLDAsIlxcbWF0aGNhbHtNfV5cXG90aW1lcyAiXSxbMSwxLCJcXG1hdGhjYWx7Q31cXHRpbWVzIFxcb3BlcmF0b3JuYW1le0Vudn0oXFxtYXRoY2Fse019KV5cXG90aW1lcyBfe1xcbGFuZ2xlIG5cXHJhbmdsZX1cXHRpbWVzIFxcRGVsdGFeMSJdLFsxLDAsIlxcbWF0aGNhbHtDfVxcdGltZXMgXFxvcGVyYXRvcm5hbWV7RW52fShcXG1hdGhjYWx7TX0pXlxcb3RpbWVzIF97XFxsYW5nbGUgbiBcXHJhbmdsZX1cXHRpbWVzIFxccGFydGlhbFxcRGVsdGFeMSJdLFswLDFdLFsyLDMsIlxcb3BlcmF0b3JuYW1le2lkfV97XFxtYXRoY2Fse0N9fVxcdGltZXMgZ19pIiwyXSxbNSwyXSxbMSw1LCJcXFBoaVxcdGltZXMgXFxvcGVyYXRvcm5hbWV7aWR9IiwyXSxbMCw2LCJcXFBoaVxcdGltZXMgXFxvcGVyYXRvcm5hbWV7aWR9Il0sWzYsNCwiKFxcdGV4dHtmb3JnZXR9LCBcXHJob15pXyEpIl0sWzYsNV0sWzQsM11d
\[\begin{tikzcd}[column sep=small, scale cd=.8]
	{((\mathcal{M}^\otimes_{\mathrm{act}})^n\times _{\mathcal{C}^n}\mathcal{C})\times \partial\Delta^1} & {\mathcal{C}\times \operatorname{Env}(\mathcal{M})^\otimes _{\langle n \rangle}\times \partial\Delta^1} && {\mathcal{M}^\otimes } \\
	{((\mathcal{M}^\otimes_{\mathrm{act}})^n\times_{\mathcal{C}^n}\mathcal{C})\times  \Delta^1} & {\mathcal{C}\times \operatorname{Env}(\mathcal{M})^\otimes _{\langle n\rangle}\times \Delta^1} & {\mathcal{C}\times \operatorname{Env}(N(\mathsf{Fin}_\ast))^\otimes _{\langle n\rangle}\times \Delta^1} & {\mathcal{C}\times N(\mathsf{Fin}_\ast)}
	\arrow[from=1-1, to=2-1]
	\arrow["{\operatorname{id}_{\mathcal{C}}\times g_i}"', from=2-3, to=2-4]
	\arrow[from=2-2, to=2-3]
	\arrow["{\Phi\times \operatorname{id}}"', from=2-1, to=2-2]
	\arrow["{\Phi\times \operatorname{id}}", from=1-1, to=1-2]
	\arrow["{(\text{forget}, \rho^i_!)}", from=1-2, to=1-4]
	\arrow[from=1-2, to=2-2]
	\arrow[from=1-4, to=2-4]
\end{tikzcd}\]so the existence of the desired filler follows from the definition
of $\rho_{!}^{i}$.
\end{proof}
%
\begin{cor}
[Universal Property of the Direct Sum Functor]\label{cor:oplus_p-limit}Let
$\cal C$ be an $\infty$-category and let $p:\cal M^{\t}\to\cal C\times N\pr{\Fin_{\ast}}$
be a $\cal C$-family of $\infty$-operads. Let $q:K\to\cal C$ be
an object of $\SS/\cal C$, let $n\geq1$, and let $f,g_{1},\dots,g_{n}:K\to\cal M_{\act}^{\t}$
be diagrams. Assume the following:
\begin{enumerate}
\item For each $1\leq i\leq n$, there is an inert natural transformation
$\alpha_{i}:f\to g_{i}$ over $\cal C$.
\item For each vertex $v$ in $K$, the maps $\alpha_{i}$ give rise to
a $p$-limit cone $\{f\pr v\to g_{i}\pr v\}_{1\leq i\le n}$. 
\end{enumerate}
Then $f$ is naturally equivalent over $\cal C$ to the composite
\[
G:K\xrightarrow{\pr{g_{i}}_{i=1}^{n}}\pr{\cal M_{\act}^{\t}}^{n}\times_{\cal C^{n}}\cal C\xrightarrow{\bigoplus_{1\leq i\leq n}}\cal M_{\act}^{\t}.
\]
\end{cor}
\begin{proof}
For each vertex $v$ in $K$, write $p_{2}\pr{g_{i}\pr v}=\inp{m_{i}\pr v}$
and $p_{2}\pr{f\pr v}=\inp{m\pr v}$. Let $\eta:K\times\Delta^{1}\to N\pr{\Fin_{\ast}}$
denote the natural equivalence which satisfies the following conditions: 
\begin{itemize}
\item The restriction $\eta\vert K\times\{0\}$ is given by $K\xrightarrow{\pr{p_{2}g_{i}}_{i}}N\pr{\Fin_{\ast}}_{\act}^{n}\xrightarrow{\Vee_{i=1}^{n}}N\pr{\Fin_{\ast}}$.
\item The restriction $\eta\vert K\times\{1\}$ is given by $p_{2}f$.
\item For each vertex $v\in K$ and $1\leq i\leq n$, the composite
\[
\inp{m\pr v}\xrightarrow{\eta}\inp{m\pr v}\xrightarrow{p_{2}\alpha_{i}}\inp{m_{i}\pr v}
\]
is defined as follows: Given $k\in\inp{m\pr v}^{\circ}$, the integer
$\sum_{j<i}m_{j}\pr v+k\in\inp{m\pr v}$ is mapped to $k$. The remaining
elements are mapped to the base point of $\inp{m_{i}\pr v}$.
\end{itemize}
Using the fact that $p$ is a categorical equivalence, we can find
a functor $f':K\to\cal M^{\t}$ and a natural equivalence $f'\xrightarrow{\simeq}f$
which lifts the natural equivalence $\pr{q\circ\opn{pr}_{1},\eta}:K\times\Delta^{1}\to\cal C\times N\pr{\Fin_{\ast}}$.
Replacing $f$ by $f'$ if necessary, we may assume that the diagram
% https://q.uiver.app/?q=WzAsNCxbMCwwLCJLXFx0aW1lcyBcXERlbHRhXjEiXSxbMSwwLCJcXG1hdGhjYWx7TX1eXFxvdGltZXMgIl0sWzEsMSwiXFxtYXRoY2Fse0N9XFx0aW1lcyBOKFxcbWF0aHNme0Zpbn1fXFxhc3QpIl0sWzAsMSwiKFxcbWF0aGNhbHtDfVxcdGltZXMgKE4oXFxtYXRoc2Z7RmlufV9cXGFzdClfe1xcbWF0aHJte2FjdH19KV5uKVxcdGltZXMgXFxEZWx0YV4xIl0sWzAsMSwiXFxhbHBoYV9pIl0sWzEsMiwicCJdLFszLDIsIlxcb3BlcmF0b3JuYW1le2lkfV97XFxtYXRoY2Fse0N9fVxcdGltZXMgaF9pIiwyXSxbMCwzLCIocSwocF8yZ19qKV9qKVxcdGltZXMgXFxvcGVyYXRvcm5hbWV7aWR9IiwyXV0=
\[\begin{tikzcd}
	{K\times \Delta^1} & {\mathcal{M}^\otimes } \\
	{(\mathcal{C}\times (N(\mathsf{Fin}_\ast)_{\mathrm{act}})^n)\times \Delta^1} & {\mathcal{C}\times N(\mathsf{Fin}_\ast)}
	\arrow["{\alpha_i}", from=1-1, to=1-2]
	\arrow["p", from=1-2, to=2-2]
	\arrow["{\operatorname{id}_{\mathcal{C}}\times h_i}"', from=2-1, to=2-2]
	\arrow["{(q,(p_2g_j)_j)\times \operatorname{id}}"', from=1-1, to=2-1]
\end{tikzcd}\]commutes for each $1\leq i\leq n$. Since relative limits in functor
categories can be formed objectwise \cite[\href{https://kerodon.net/tag/02XK}{Tag 02XK}]{kerodon},
the morphisms $\alpha_{i}:f\to g_{i}$ in $\Fun\pr{K,\cal M^{\t}}$
form a $\Fun\pr{K,p}$-limit cone. Moreover, since relative limits
are stable under pullbacks \cite[Proposition 4.3.1.5]{HTT}, we deduce
that the morphisms $\{\alpha_{i}\}_{1\leq i\leq n}$ of $\Fun_{\cal C}\pr{K,\cal M^{\t}}$
form a $\Fun_{\cal C}\pr{K,p}$-limit cone. 

Now using Proposition \ref{prop:oplus_p-limit}, we can construct
the functor $\bigoplus_{1\leq i\leq n}:\pr{\cal M_{\act}^{\t}}^{n}\times_{\cal C^{n}}\cal C\to\cal M^{\t}$
so that there is an inert natural transformation $\widetilde{h}_{i}:\bigoplus_{1\leq i\leq n}\to\opn{pr}_{i}$
which lifts the composite $\pr{\cal M_{\act}^{\t}}^{n}\times_{\cal C^{n}}\cal C\to\cal C\times N\pr{\Fin_{\ast}}_{\act}^{n}\times\Delta^{1}\xrightarrow{\id_{\cal C}\times h_{i}}\cal C\times N\pr{\Fin_{\ast}}$.
Again, the induced natural transformations $\{G\to g_{i}\}_{1\leq i\leq n}$
form a $\Fun_{\cal C}\pr{K,p}$-limit cone covering the same cone
as $\{\alpha_{i}\}$. Thus we obtain the desired equivalence $G\xrightarrow{\simeq}f$
in $\Fun_{\cal C}\pr{K,\cal M^{\t}}$.
\end{proof}
The above corollary admits the following further corollary, which
we shall use in the next section.
\begin{cor}
\label{cor:3.1.1.8}Let $q:\cal C^{\t}\to\cal O^{\t}$ be a fibration
of $\infty$-operads. Let $K_{1},\dots,K_{n}$ be simplicial sets,
and let $K=\prod_{1\leq i\leq n}K_{i}$. Let $\overline{f}:K^{\rcone}\to\cal C^{\t}$
and $\{\overline{f_{i}}:K^{\rcone}\to\cal C^{\t}\}_{1\leq i\leq n}$
be active diagrams, and suppose there are inert natural transformations
$\{\overline{f}\to\overline{f}_{i}\circ\pr{\opn{pr}_{i}}^{\rcone}\}_{1\le i\leq n}$
such that for each vertex $v$ in $K^{\rcone}$, the induced map $\{\overline{f}\pr v\to\overline{f}_{i}\circ\pr{\opn{pr}_{i}}^{\rcone}\pr v\}_{1\le i\leq n}$
form a $q$-limit cone. If each $\overline{f_{i}}$ is an operadic
$q$-colimit diagram, so is $\overline{f}$.
\end{cor}
\begin{proof}
According to Corollary \ref{cor:oplus_p-limit}, the diagram $\overline{f}$
is naturally equivalent to the composite
\[
K^{\rcone}\to\prod_{1\leq i\leq n}\pr{K_{i}^{\rcone}}\xrightarrow{\prod_{1\leq i\leq n}\overline{f_{i}}}\pr{\cal C_{\act}^{\t}}^{n}\xrightarrow{\bigoplus_{1\leq i\leq n}}\cal C^{\t}.
\]
The claim is thus a consequence of \cite[Proposition 3.1.1.8]{HA}.
\end{proof}

\subsection{Direct Sum and Slice}

Let $\cal O^{\t}$ be an $\infty$-operad and let $Y\in\cal O^{\t}$
be an object. If $Y$ lies over an object $\inp n\in N\pr{\Fin_{\ast}}$
with $n\geq2$, then we can find objects $Y_{i}\in\cal O$ and an
equivalence $Y\simeq\bigoplus_{i=1}^{n}Y_{i}$. Given objects $X_{1},\dots,X_{n}\in\cal O^{\t}$
and active maps $f_{i}:X_{i}\to Y_{i}$, their direct sum $\bigoplus_{i=1}^{n}f_{i}:\bigoplus_{i=1}^{n}X_{i}\to\bigoplus_{i=1}^{n}Y_{i}\simeq Y$
is an active morphism with codomain $Y$. Conversely, given an active
morphism $f:X\to Y$ in $\cal O^{\t}$, we can write $f\simeq\bigoplus_{i=1}^{n}f_{i}$,
where $f_{i}$ is the active map obtained by factoring the composite
$X\to Y\to Y_{i}$ into an inert map followed by an active map. The
following proposition, which is the only result of this subsection,
asserts that this ``direct sum decomposition'' of morphisms is an
equivalence on the level of $\infty$-categories: 
\begin{prop}
\label{prop:directsum_slice_equiv}Let $\cal C$ be an $\infty$-category,
let $C\in\cal C$ be an object, and let $\cal M^{\t}\to\cal C\times N\pr{\Fin_{\ast}}$
be a $\cal C$-family of $\infty$-operads. For any integer $n\geq1$
and any objects $M_{1},\dots,M_{n}\in\cal M\times_{\cal C}\{C\}$,
the direct sum functor induces an equivalence of $\infty$-categories
\[
\pr{\pr{\cal M_{\act}^{\t}}^{n}\times_{\cal C^{n}}\cal C}_{/\pr{M_{1},\dots,M_{n}}}\xrightarrow{\simeq}\pr{\cal M_{\act}^{\t}}_{/\bigoplus_{i=1}^{n}M_{i}}.
\]
\end{prop}
\begin{proof}
Recall that the direct sum functor $\bigoplus_{i=1}^{n}$ is obtained
as the composite 
\[
\pr{\cal M_{\act}^{\t}}^{n}\times_{\cal C^{n}}\cal C\xrightarrow[\Phi]{\simeq}\Env\pr{\cal M}_{\inp n}^{\t}\xrightarrow{\text{forget}}\cal M_{\act}^{\t},
\]
where the map $\Phi$ is the inverse equivalence over $\cal C$ of
the functor $\pr{\rho_{!}^{i}}_{1\leq i\le n}:\Env\pr{\cal M}_{\inp n}^{\t}\xrightarrow{\simeq}\pr{\cal M_{\act}^{\t}}^{n}\times_{\cal C^{n}}\cal C$
. The functor $\Phi$ maps the object $\pr{M_{1},\dots,M_{n}}$ to
the object $\pr{\bigoplus_{i=1}^{n}M_{i},\alpha:\inp n\to\inp n}$,
where $\alpha$ is a bijection. It will therefore suffice to prove
that the functor 
\[
\theta:\pr{\Env\pr{\cal M}_{\inp n}^{\t}}_{/\pr{\bigoplus_{i=1}^{n}M_{i},\alpha}}\to\pr{\cal M_{\act}^{\t}}_{/\bigoplus_{i=1}^{n}M_{i}}
\]
is an equivalence of $\infty$-categories. Set $M=\bigoplus_{i=1}^{n}M_{i}$.
By definition, we have
\[
\pr{\Env\pr{\cal M}_{\inp n}^{\t}}_{/\pr{M,\alpha}}=\pr{\Act\pr{N\pr{\Fin_{\ast}}}_{\inp n}^{\t}}_{/\alpha}\times_{N\pr{\Fin_{\ast}}_{/\inp n}}\cal M_{/M}^{\t}.
\]
Since the evaluation at $0\in\Delta^{1}$ induces an isomorphism of
simplicial sets
\[
\pr{\Act\pr{N\pr{\Fin_{\ast}}}_{\inp n}^{\t}}_{/\alpha}\xrightarrow{\cong}\pr{N\pr{\Fin_{\ast}}_{\act}}_{/\inp n},
\]
we deduce that the functor $\theta$ is an \textit{isomorphism} of
simplicial sets.
\end{proof}

\section{\label{sec:FTOK}The Fundamental Theorem of Operadic Kan Extensions}

In this final section, we will provide complete details of the proof
of \cite[Theorem 3.1.2.3]{HA}, using tools and results in the preceding
sections.

\subsection{Recollection}

In this subsection, we recall the definition of operadic Kan extensions
and the statement of the fundamental theorem of operadic Kan extensions. 
\begin{defn}
\cite[Definition 3.1.2.2]{HA}Let $\cal M^{\t}\to N\pr{\Fin_{\ast}}\times\Delta^{1}$
be a $\Delta^{1}$-family of $\infty$-operads and let $q:\cal C^{\t}\to\cal O^{\t}$
be a fibration of $\infty$-operads. Set $\cal A^{\t}=\cal M^{\t}\times_{\Delta^{1}}\{0\}$,
$\cal B^{\t}=\cal M^{\t}\times_{\Delta^{1}}\{1\}$. A map $F:\cal M^{\t}\to\cal C^{\t}$
is called an \textbf{operadic $q$-left Kan extension of $F\vert\cal A^{\t}$
}if the following condition is satisfied for every object $B\in\cal B$:
\begin{itemize}
\item [($\ast$)]The composite map
\[
\pr{\pr{\cal M_{\act}^{\t}}_{/B}\times_{\Delta^{1}}\{0\}}^{\rcone}\to\pr{\cal M_{/B}^{\t}}^{\rcone}\to\cal M^{\t}\xrightarrow{\overline{F}}\cal C^{\t}
\]
is an operadic $q$-colimit diagram.
\end{itemize}
\end{defn}
Lurie imposes condition ($\ast$) to hold for every object $B\in\cal B^{\t}$
in \cite[Definition 3.1.2.2]{HA}. This is not a problem, because
of the following proposition, which seems to be implicitly used in
\cite{HA}.
\begin{prop}
\label{prop:operadic_Kan_equiv}Let $p:\cal M^{\t}\to N\pr{\Fin_{\ast}}\times\Delta^{1}$
be a correspondence from an $\infty$-operad $\cal A^{\t}$ to another
$\infty$-operad $\cal B^{\t}$. Let $q:\cal C^{\t}\to\cal O^{\t}$
be a fibration of $\infty$-operads and let $F:\cal M^{\t}\to\cal C^{\t}$
be a map of generalized $\infty$-operads. Suppose that $F$ is an
operadic $q$-left Kan extension of \textbf{$F\vert\cal A^{\t}$}.
Then for every object $B\in\cal B^{\t},$the map 
\[
\theta:\pr{\pr{\cal M_{\act}^{\t}}_{/B}\times_{\Delta^{1}}\{0\}}^{\rcone}\to\pr{\cal M_{/B}^{\t}}^{\rcone}\to\cal M^{\t}\xrightarrow{F}\cal C^{\t}
\]
is an operadic $q$-colimit diagram.
\end{prop}
\begin{proof}
Let $\inp n$ be the image of $B$ in $N\pr{\Fin_{\ast}}$. We consider
two cases, depending on whether or not $n$ is equal to $0$.

Suppose first that $n=0$. Then we have $\pr{\cal M_{\act}^{\t}}_{/B}=\pr{\cal M_{\inp 0}^{\t}}_{/B}$,
since there is no active map $\inp k\to\inp 0$ with $k\geq1$. Since
every object of $\cal M_{\inp 0}^{\t}$ is $p$-terminal, the functor
$\cal M_{\inp 0}^{\t}\to\Delta^{1}$ is a trivial fibration. It follows
that the functor $\pr{\cal M_{\act}^{\t}}_{/B}\times_{\Delta^{1}}\{0\}\to\Delta_{/1}^{1}\times_{\Delta^{1}}\{0\}\cong\Delta^{0}$
is also a trivial fibration. It will therefore suffice to show that
$F$ carries a morphism of the form $A\to B$ in $\cal M^{\t}$ with
$A\in\cal M_{\inp 0}^{\t}\times_{\Delta^{1}}\{0\}$ to an equivalence
in $\cal C^{\t}$. This is clear, since $\cal C_{\inp 0}^{\t}$ is
a contractible Kan complex.

If $n\geq1$, choose for each $1\leq i\leq n$ an inert map $B\to B_{i}$
in $\cal B^{\t}$ over $\rho^{i}:\inp n\to\inp 1$, and choose also
a direct sum functor $\bigoplus_{i=1}^{n}$ for $\cal M^{\t}$ and
$\cal C^{\t}$. Replacing $\bigoplus_{i=1}^{n}$ by a functor naturally
equivalent one, we may assume that $B=\bigoplus_{i=1}^{n}$. According
to Proposition \ref{prop:directsum_slice_equiv}, the direct sum functor
induces an equivalence of $\infty$-categories
\[
\phi:\pr{\pr{\cal M_{\act}^{\t}}^{n}\times_{\pr{\Delta^{1}}^{n}}\Delta^{1}}_{/\pr{B_{1},\dots,B_{n}}}\xrightarrow{\simeq}\pr{\cal M_{\act}^{\t}}_{/B}.
\]
There is also an isomorphism of simplicial sets
\[
\psi:\prod_{i=1}^{n}\pr{\cal M_{\act}^{\t}}_{/B_{i}}\times_{\Delta^{1}}\{0\}\cong\pr{\pr{\cal M_{\act}^{\t}}^{n}\times_{\pr{\Delta^{1}}^{n}}\Delta^{1}}_{/\pr{B_{1},\dots,B_{n}}}\times_{\Delta^{1}}\{0\}.
\]
It will therefore suffice to show that the composite $\theta\circ\phi^{\rcone}\circ\psi^{\rcone}$
is an operadic $q$-colimit diagram. For this, we consider the diagram
% https://q.uiver.app/?q=WzAsOSxbMCwwLCIoXFxwcm9kX3tpPTF9Xm4oXFxtYXRoY2Fse019Xlxcb3RpbWVzIF97XFxtYXRocm17YWN0fX0pX3svQl9pfVxcdGltZXMgX3tcXERlbHRhXjF9XFx7MFxcfSleXFx0cmlhbmdsZXJpZ2h0Il0sWzEsMCwiKCgoXFxtYXRoY2Fse019Xlxcb3RpbWVzIF97XFxtYXRocm17YWN0fX0pXm5cXHRpbWVzX3soXFxEZWx0YV4xKV5ufVxcRGVsdGFeMSlfey8oQl8xLFxcZG90cyAsQl9uKX1cXHRpbWVzX3tcXERlbHRhXjF9XFx7MFxcfSleXFx0cmlhbmdsZXJpZ2h0Il0sWzAsMSwiXFxwcm9kX3tpPTF9Xm4oKFxcbWF0aGNhbHtNfV5cXG90aW1lcyBfe1xcbWF0aHJte2FjdH19KV97L0JfaX1cXHRpbWVzIF97XFxEZWx0YV4xfVxcezBcXH0pXlxcdHJpYW5nbGVyaWdodCJdLFsxLDEsIihcXG1hdGhjYWx7TX1eXFxvdGltZXMgX3tcXG1hdGhybXthY3R9fSleblxcdGltZXNfeyhcXERlbHRhXjEpXm59XFxEZWx0YV4xIl0sWzIsMCwiKChcXG1hdGhjYWx7TX1eXFxvdGltZXMgX3tcXG1hdGhybXthY3R9fSlfey9CfVxcdGltZXMgX3tcXERlbHRhXjF9XFx7MFxcfSleXFx0cmlhbmdsZXJpZ2h0Il0sWzIsMSwiXFxtYXRoY2Fse019Xlxcb3RpbWVzIF97XFxtYXRocm17YWN0fX0iXSxbMCwyLCIoXFxtYXRoY2Fse019Xlxcb3RpbWVzIF97XFxtYXRocm17YWN0fX0pXm4iXSxbMSwyLCIoXFxtYXRoY2Fse0N9Xlxcb3RpbWVzIF97XFxtYXRocm17YWN0fX0pXm4iXSxbMiwyLCJcXG1hdGhjYWx7Q31eXFxvdGltZXMgX3tcXG1hdGhybXthY3R9fSJdLFswLDJdLFsxLDNdLFswLDEsIlxcY29uZyJdLFsxLDQsIlxcc2ltZXEiXSxbMCwxLCJcXHBzaV5cXHRyaWFuZ2xlcmlnaHQiLDJdLFsxLDQsIlxccGhpXlxcdHJpYW5nbGVyaWdodCIsMl0sWzQsNV0sWzMsNSwiXFxiaWdvcGx1c197aT0xfV5uIiwyXSxbMyw2LCIiLDIseyJzdHlsZSI6eyJ0YWlsIjp7Im5hbWUiOiJob29rIiwic2lkZSI6ImJvdHRvbSJ9fX1dLFsyLDZdLFs2LDcsIlxccHJvZF97aT0xfV5uRiIsMl0sWzMsNywiRyJdLFs1LDgsIkYiXSxbNyw4LCJcXGJpZ29wbHVzX3tpPTF9Xm4iLDJdXQ==
\[\begin{tikzcd}[column sep=small, scale cd=.8]
	{(\prod_{i=1}^n(\mathcal{M}^\otimes _{\mathrm{act}})_{/B_i}\times _{\Delta^1}\{0\})^\triangleright} & {(((\mathcal{M}^\otimes _{\mathrm{act}})^n\times_{(\Delta^1)^n}\Delta^1)_{/(B_1,\dots ,B_n)}\times_{\Delta^1}\{0\})^\triangleright} & {((\mathcal{M}^\otimes _{\mathrm{act}})_{/B}\times _{\Delta^1}\{0\})^\triangleright} \\
	{\prod_{i=1}^n((\mathcal{M}^\otimes _{\mathrm{act}})_{/B_i}\times _{\Delta^1}\{0\})^\triangleright} & {(\mathcal{M}^\otimes _{\mathrm{act}})^n\times_{(\Delta^1)^n}\Delta^1} & {\mathcal{M}^\otimes _{\mathrm{act}}} \\
	{(\mathcal{M}^\otimes _{\mathrm{act}})^n} & {(\mathcal{C}^\otimes _{\mathrm{act}})^n} & {\mathcal{C}^\otimes _{\mathrm{act}}.}
	\arrow[from=1-1, to=2-1]
	\arrow[from=1-2, to=2-2]
	\arrow["\cong", from=1-1, to=1-2]
	\arrow["\simeq", from=1-2, to=1-3]
	\arrow["{\psi^\triangleright}"', from=1-1, to=1-2]
	\arrow["{\phi^\triangleright}"', from=1-2, to=1-3]
	\arrow[from=1-3, to=2-3]
	\arrow["{\bigoplus_{i=1}^n}"', from=2-2, to=2-3]
	\arrow[hook', from=2-2, to=3-1]
	\arrow[from=2-1, to=3-1]
	\arrow["{\prod_{i=1}^nF}"', from=3-1, to=3-2]
	\arrow["G", from=2-2, to=3-2]
	\arrow["F", from=2-3, to=3-3]
	\arrow["{\bigoplus_{i=1}^n}"', from=3-2, to=3-3]
\end{tikzcd}\]Here the map $G$ is the restriction of the map $\prod_{i=1}^{n}F$.
The left column and the top right square are commutative. The bottom
right square commutes up to natural equivalence by Proposition \ref{prop:oplus_p-limit}
and Corollary \ref{cor:oplus_p-limit}. Hence the diagram $\theta\circ\phi^{\rcone}\circ\psi^{\rcone}$
is naturally equivalent to the composite
\[
\theta':\pr{\prod_{i=1}^{n}\pr{\cal M_{\act}^{\t}}_{/B_{i}}\times_{\Delta^{1}}\{0\}}^{\rcone}\to\prod_{i=1}^{n}\pr{\pr{\cal M_{\act}^{\t}}_{/B_{i}}\times_{\Delta^{1}}\{0\}}^{\rcone}\to\pr{\cal C_{\act}^{\t}}^{n}\xrightarrow{\bigoplus_{i=1}^{n}}\cal C_{\act}^{\t}.
\]
The diagram $\theta'$ is an operadic $q$-colimit diagram by \cite[Proposition 3.1.1.8]{HA}.
\end{proof}
We can now state the fundamental theorem of opeardic Kan extensions.
\begin{thm}
\cite[Theorem 3.1.2.3]{HA}\label{thm:3.1.2.3}Let $p:\cal M^{\t}\to N\pr{\Fin_{\ast}}\times\Delta^{n}$
be a $\Delta^{n}$-family of generalized $\infty$-operads, and let
$q:\cal C^{\t}\to\cal O^{\t}$ be a fibration of $\infty$-operads.
Consider a commutative diagram% https://q.uiver.app/?q=WzAsNCxbMCwwLCJcXG1hdGhjYWx7TX1eXFxvdGltZXMgXFx0aW1lcyBfe1xcRGVsdGFebn1cXExhbWJkYSBebl8wIl0sWzAsMSwiXFxtYXRoY2Fse019Xlxcb3RpbWVzICJdLFsxLDAsIlxcbWF0aGNhbHtDfV5cXG90aW1lcyAiXSxbMSwxLCJcXG1hdGhjYWx7T31eXFxvdGltZXMgLiJdLFswLDEsIiIsMCx7InN0eWxlIjp7InRhaWwiOnsibmFtZSI6Imhvb2siLCJzaWRlIjoidG9wIn19fV0sWzAsMiwiZl8wIl0sWzIsMywicSJdLFsxLDMsImciLDJdLFsxLDIsIiIsMCx7InN0eWxlIjp7ImJvZHkiOnsibmFtZSI6ImRhc2hlZCJ9fX1dXQ==
\[\begin{tikzcd}
	{\mathcal{M}^\otimes \times _{\Delta^n}\Lambda ^n_0} & {\mathcal{C}^\otimes } \\
	{\mathcal{M}^\otimes } & {\mathcal{O}^\otimes}
	\arrow[hook, from=1-1, to=2-1]
	\arrow["{f_0}", from=1-1, to=1-2]
	\arrow["q", from=1-2, to=2-2]
	\arrow["g"', from=2-1, to=2-2]
	\arrow[dashed, from=2-1, to=1-2]
\end{tikzcd}\]of simplicial sets. Suppose that for each vertex $i$ in $\Lambda_{0}^{n}$
and each vertex $j$ in $\Delta^{n}$, the induced maps $\cal M^{\t}\times_{\Delta^{n}}\{v\}\to\cal C^{\t}$
and $\cal M^{\t}\times_{\Delta^{n}}\{j\}\to\cal O^{\t}$ are morphisms
of $\infty$-operads.
\begin{itemize}
\item [(A)]If $n=1$, the following conditions are equivalent:
\begin{itemize}
\item [(a)]There is a dashed filler which is an operadic $q$-left Kan
extension of $f_{0}$.
\item [(b)]For each object $B\in\cal M^{\t}\times_{\Delta^{1}}\{1\}$,
the diagram
\[
\{0\}\times_{\Delta^{1}}\pr{\pr{\cal M_{\act}^{\t}}_{/B}}\to\{0\}\times_{\Delta^{1}}\cal M^{\t}\xrightarrow{f_{0}}\cal C^{\t}
\]
admits an operadic $q$-colimit cone which lifts the map
\[
\pr{\{0\}\times_{\Delta^{1}}\pr{\pr{\cal M_{\act}^{\t}}_{/B}}}^{\rcone}\to\pr{\cal M_{/B}^{\t}}^{\rcone}\to\cal M^{\t}\xrightarrow{g}\cal O^{\t}.
\]
\end{itemize}
\item [(B)]If $n>1$ and the restriction $f_{0}\vert\cal M^{\t}\times_{\Delta^{n}}\Delta^{\{0,1\}}$
is an operadic $q$-left Kan extension of $f_{0}\vert\cal M^{\t}\times_{\Delta^{n}}\{0\}$,
then there is a dashed arrow rendering the diagram commutative.
\end{itemize}
\end{thm}
The rest of this note is devoted to elaborating Lurie's proof of the
above theorem.

\subsection{Preliminary Results}

In this subsection, we collect some results which will be used in
the proof of Theorem \ref{prop:operadic_Kan_equiv}. We recommend
that the reader skip to the next subsection and refer to this subsection
as needed.
\begin{prop}
\label{prop:relative_terminal_obj_in_slice}Let $p:\cal C\to\cal D$
be an inner fibration of $\infty$-categories, $K$ a simplicial set,
and $\overline{f}:K^{\rcone}\to\cal C$ a diagram. Set $f=\overline{f}\vert K$
and let $q$ denote the functor $\cal C_{f/}\to\cal D_{pf/}$. If
$\overline{f}$ maps the cone point to a $p$-terminal object, then
$\overline{f}$ is $q$-terminal.
\end{prop}
\begin{proof}
We must show that the map 
\[
\pr{\cal C_{f/}}_{/\overline{f}}\to\cal C_{f/}\times_{\cal D_{pf/}}\pr{\cal D_{pf/}}_{/q\overline{f}}
\]
is a trivial fibration. Let $C=\overline{f}\pr{\infty}$. Given a
monomorphism $A\to B$ of simplicial sets, a lifting problem on the
left hand side corresponds under adjunction to a lifting problem on
the right hand side:% https://q.uiver.app/?q=WzAsOCxbMCwwLCJBIl0sWzAsMSwiQiJdLFsxLDAsIihcXG1hdGhjYWx7Q31fe2YvfSlfey9cXG92ZXJsaW5le2Z9fSJdLFsxLDEsIlxcbWF0aGNhbHtDfV97Zi99XFx0aW1lcyBfe1xcbWF0aGNhbHtEfV97cGYvfX0oXFxtYXRoY2Fse0R9X3twZi99KV97L3FcXG92ZXJsaW5le2Z9fSJdLFsyLDAsIktcXHN0YXIgQSJdLFsyLDEsIktcXHN0YXIgQiJdLFszLDAsIlxcbWF0aGNhbHtDfV97L0N9Il0sWzMsMSwiXFxtYXRoY2Fse0N9XFx0aW1lcyBfe1xcbWF0aGNhbHtEfX1cXG1hdGhjYWx7RH1fey9wQ30iXSxbMCwyXSxbMiwzXSxbMCwxXSxbMSwzXSxbMSwyLCIiLDEseyJzdHlsZSI6eyJib2R5Ijp7Im5hbWUiOiJkYXNoZWQifX19XSxbNCw1XSxbNCw2XSxbNiw3XSxbNSw3XSxbNSw2LCIiLDEseyJzdHlsZSI6eyJib2R5Ijp7Im5hbWUiOiJkYXNoZWQifX19XV0=
\[\begin{tikzcd}
	A & {(\mathcal{C}_{f/})_{/\overline{f}}} & {K\star A} & {\mathcal{C}_{/C}} \\
	B & {\mathcal{C}_{f/}\times _{\mathcal{D}_{pf/}}(\mathcal{D}_{pf/})_{/q\overline{f}}} & {K\star B} & {\mathcal{C}\times _{\mathcal{D}}\mathcal{D}_{/pC}.}
	\arrow[from=1-1, to=1-2]
	\arrow[from=1-2, to=2-2]
	\arrow[from=1-1, to=2-1]
	\arrow[from=2-1, to=2-2]
	\arrow[dashed, from=2-1, to=1-2]
	\arrow[from=1-3, to=2-3]
	\arrow[from=1-3, to=1-4]
	\arrow[from=1-4, to=2-4]
	\arrow[from=2-3, to=2-4]
	\arrow[dashed, from=2-3, to=1-4]
\end{tikzcd}\]The right hand lifting problem is solvable because $C$ is $p$-terminal.
\end{proof}
\begin{cor}
\label{cor:Step2}Let $q:\cal C^{\t}\to\cal O^{\t}$ be a fibration
of $\infty$-operads, let $K$ be a simplicial set, and let $n\geq0$.
Consider a lifting problem % https://q.uiver.app/?q=WzAsNCxbMCwwLCJLXFxzdGFyIFxccGFydGlhbFxcRGVsdGEgXm4iXSxbMCwxLCJLXFxzdGFyIFxcRGVsdGEgXm4iXSxbMSwwLCJcXG1hdGhjYWx7Q31eXFxvdGltZXMgIl0sWzEsMSwiXFxtYXRoY2Fse099Xlxcb3RpbWVzICJdLFswLDEsIiIsMCx7InN0eWxlIjp7InRhaWwiOnsibmFtZSI6Imhvb2siLCJzaWRlIjoidG9wIn19fV0sWzIsMywicSJdLFswLDIsImYiXSxbMSwzLCJnIiwyXSxbMSwyLCIiLDEseyJzdHlsZSI6eyJib2R5Ijp7Im5hbWUiOiJkYXNoZWQifX19XV0=
\[\begin{tikzcd}
	{K\star \partial\Delta ^n} & {\mathcal{C}^\otimes } \\
	{K\star \Delta ^n} & {\mathcal{O}^\otimes.}
	\arrow[hook, from=1-1, to=2-1]
	\arrow["q", from=1-2, to=2-2]
	\arrow["f", from=1-1, to=1-2]
	\arrow["g"', from=2-1, to=2-2]
	\arrow[dashed, from=2-1, to=1-2]
\end{tikzcd}\]If the map $g$ maps the terminal vertex of $\Delta^{n}$ to an object
in $\cal O_{\inp 0}^{\t}$, then the lifting problem admits a solution.
\end{cor}
\begin{proof}
If $n=0$, find an object $C\in\cal C_{\inp 0}^{\t}$ which lies over
$g\pr 0$. Such an object exists because the functor $\cal C_{\inp 0}^{\t}\to\cal O_{\inp 0}^{\t}$
is a trivial fibration. The object $C$ is $q$-terminal, so the functor
\[
\cal C_{/C}^{\t}\to\cal O_{/q\pr C}^{\t}\times_{\cal O^{\t}}\cal C^{\t}
\]
is a trivial fibration. This implies the existence of the filler.

If $n>1$, then set $h=f\vert K$. We must solve a lifting problem
% https://q.uiver.app/?q=WzAsNCxbMCwwLCJcXHBhcnRpYWxcXERlbHRhIF5uIl0sWzAsMSwiXFxEZWx0YSBebiJdLFsxLDAsIlxcbWF0aGNhbHtDfV5cXG90aW1lcyBfe2gvfSJdLFsxLDEsIlxcbWF0aGNhbHtPfV5cXG90aW1lc197cWgvfSJdLFswLDEsIiIsMCx7InN0eWxlIjp7InRhaWwiOnsibmFtZSI6Imhvb2siLCJzaWRlIjoidG9wIn19fV0sWzIsMywicSciXSxbMCwyXSxbMSwzXSxbMSwyLCIiLDEseyJzdHlsZSI6eyJib2R5Ijp7Im5hbWUiOiJkYXNoZWQifX19XV0=
\[\begin{tikzcd}
	{\partial\Delta ^n} & {\mathcal{C}^\otimes _{h/}} \\
	{\Delta ^n} & {\mathcal{O}^\otimes_{qh/}.}
	\arrow[hook, from=1-1, to=2-1]
	\arrow["{q'}", from=1-2, to=2-2]
	\arrow[from=1-1, to=1-2]
	\arrow[from=2-1, to=2-2]
	\arrow[dashed, from=2-1, to=1-2]
\end{tikzcd}\]For this, it will suffice to show that the image of the vertex $n\in\partial\Delta^{n}$
under the top horizontal arrow is $q'$-terminal. This follows from
Proposition \ref{prop:relative_terminal_obj_in_slice}.
\end{proof}
\begin{lem}
\cite[Lemma 3.1.2.5]{HA}\label{lem:3.1.2.5}Let $\cal C$ be an $\infty$-category
and $\cal C^{0}\subset\cal C$ a full subcategory. Let $\sigma:\Delta^{n}\to\cal C$
be a nondegenerate simplex such that $\sigma\pr i\not\in\cal C^{0}$
for each $0\leq i\leq n$. Consider the following simplicial sets:
\begin{enumerate}
\item The simplicial subset $K\subset\cal C$ consisting of those simplices
$\tau:\Delta^{k}\star\Delta^{l}\to\cal C$, where $k,l\geq-1$, $\tau\vert\Delta^{k}$
factors through $\cal C^{0}$ and $\tau\vert\Delta^{l}$ factors through
$\sigma$.
\item The simplicial subset $K_{0}\subset\cal C$ consisting of those simplices
$\tau:\Delta^{k}\star\Delta^{l}\to\cal C$, where $k,l\geq-1$, $\tau\vert\Delta^{k}$
factors through $\cal C^{0}$ and $\tau\vert\Delta^{l}$ factors through
$\sigma\vert\partial\Delta^{n}$.
\item The simplicial subset $L\subset\cal C$ generated by the simplices
$\widetilde{\sigma}:\Delta^{k}\star\Delta^{n}\to\cal C$, where $k\geq-1$,
$\widetilde{\sigma}\vert\Delta^{k}$ factors through $\cal C^{0}$,
and $\widetilde{\sigma}\vert\Delta^{n}=\sigma$. 
\end{enumerate}
Then:
\begin{itemize}
\item [(a)]The map $\cal C_{/\sigma}^{0}\star\Delta^{n}\amalg_{\cal C_{/\sigma}^{0}\star\partial\Delta^{n}}K_{0}\to K_{0}\cup L$
is an isomorphism of simplicial sets.
\item [(b)]The inclusion
\[
K_{0}\cup L\subset K
\]
is a trivial cofibration in the Joyal model structure.
\end{itemize}
\end{lem}
\begin{proof}
We begin with (a). It is clear that the simplicial set $K_{0}\cup L$
is the union of $K_{0}$ and the image of the map $\cal C_{/\sigma}^{0}\star\Delta^{n}\to\cal C$.
Therefore, it will suffice to show that if a simplex $x$ of $\cal C_{/\sigma}^{0}\star\Delta^{n}$
is mapped to a simplex in $K_{0}$, then $x$ belongs to $\cal C_{/\sigma}^{0}\star\partial\Delta^{n}$.
The image of $x$ can be represented by a simplex of the form
\[
\Delta^{k}\star\Delta^{l}\xrightarrow{\id\star u}\Delta^{k}\star\Delta^{n}\xrightarrow{\widetilde{\sigma}}\cal C,
\]
where $k,l\geq-1$, $u:[l]\to[n]$ is a poset map, and $\widetilde{\sigma}$
is an extension of $\sigma$ which maps the vertices of $\Delta^{k}$
into $\cal C^{0}$. We wish to show that if $\sigma u:\Delta^{l}\to\cal C$
factors through $\sigma\vert\partial\Delta^{n}$, then $u$ factors
through $\partial\Delta^{n}$. This is clear, since $\sigma$ is nondegenerate.

Next we prove (b). For each $m\geq n$, let $X_{m}$ denote the simplicial
subset of $K$ consisting of the simplices in $K_{0}$ and the simplices
of the form
\[
\tau:\Delta^{k}\star\Delta^{m'}\to\cal C,
\]
where $k\geq-1$, $n\leq m'<m$, $\tau\vert\Delta^{k}$ factors through
$\cal C^{0}$, and $\tau\vert\Delta^{m'}$ is a degeneration of $\sigma$.
Then we have a filtration 
\[
K_{0}\cup L=X_{n}\subset X_{n+1}\subset\cdots
\]
and $K=\bigcup_{i\geq n}X_{i}$. To prove (b), it will suffice to
show that each inclusion $X_{i}\subset X_{i+1}$ is a trivial cofibration.

Let $\varepsilon:[m]\to[n]$ be a surjective poset map. We let $\cal J\pr{\varepsilon}\subset\Del_{[m]//[n]}$
denote the full subcategory spanned by the factorizations $\pr{\varepsilon',\varepsilon''}=[m]\xrightarrow{\varepsilon'}[m']\xrightarrow{\varepsilon''}[n]$
of $\varepsilon$ into two surjective poset maps, and let $\cal J_{0}\pr{\varepsilon}$
denote its full subcategory spanned by the object other than $\pr{\id,\varepsilon}$.
Let $A\pr{\varepsilon}\subset\cal C_{/\sigma\varepsilon}^{0}$ denote
the union of the image of the maps $\cal C_{/\sigma\varepsilon''}^{0}\to\cal C_{/\sigma\varepsilon''\varepsilon'}^{0}=\cal C_{/\sigma\varepsilon}^{0}$,
where $\pr{\varepsilon',\varepsilon''}$ ranges over the objects of
$\cal J_{0}\pr{\varepsilon}$. We consider the commutative diagram
% https://q.uiver.app/?q=WzAsNCxbMCwwLCJcXGNvcHJvZF97XFx2YXJlcHNpbG9uOlttXVxcdHdvaGVhZHJpZ2h0YXJyb3dbbl19KEEoXFx2YXJlcHNpbG9uKVxcc3RhciBcXERlbHRhIF5tKVxcY3VwKFxcbWF0aGNhbHtDfV4wX3svXFxzaWdtYVxcdmFyZXBzaWxvbn1cXHN0YXJcXHBhcnRpYWxcXERlbHRhXnttfSkiXSxbMCwxLCJYX20iXSxbMSwwLCJcXGNvcHJvZF97XFx2YXJlcHNpbG9uOlttXVxcdHdvaGVhZHJpZ2h0YXJyb3dbbl19XFxtYXRoY2Fse0N9XjBfey9cXHNpZ21hXFx2YXJlcHNpbG9ufVxcc3RhclxcRGVsdGFee219Il0sWzEsMSwiWF97bSsxfSJdLFswLDFdLFswLDJdLFsyLDNdLFsxLDNdXQ==
\[\begin{tikzcd}
	{\coprod_{\varepsilon:[m]\twoheadrightarrow[n]}(A(\varepsilon)\star \Delta ^m)\cup(\mathcal{C}^0_{/\sigma\varepsilon}\star\partial\Delta^{m})} & {\coprod_{\varepsilon:[m]\twoheadrightarrow[n]}\mathcal{C}^0_{/\sigma\varepsilon}\star\Delta^{m}} \\
	{X_m} & {X_{m+1}}
	\arrow[from=1-1, to=2-1]
	\arrow[from=1-1, to=1-2]
	\arrow[from=1-2, to=2-2]
	\arrow[from=2-1, to=2-2]
\end{tikzcd}\]where the coproducts are indexed by the surjective poset maps $[m]\epi[n]$.
To complete the proof, it suffices to prove the following:
\begin{itemize}
\item [(i)]The square is a pushout.
\item [(ii)]For each surjection $\varepsilon:[m]\to[n]$, the inclusion
\[
A\pr{\varepsilon}\star\Delta^{m}\cup\cal C_{/\sigma\varepsilon}^{0}\star\partial\Delta^{m}\to\cal C_{/\sigma\varepsilon}^{0}\star\Delta^{m}
\]
is a trivial cofibration.
\end{itemize}
We start with (i). Let $x$ be a simplex in $\cal C_{/\sigma\varepsilon}^{0}\star\Delta^{m}$
corresponding to a pair $\pr{\widetilde{\sigma\varepsilon}:\Delta^{k}\star\Delta^{m}\to\cal C,\,u:\Delta^{l}\to\Delta^{m}}$,
$k,l\geq-1$, $k+l+1\geq0$. Its image in $\cal C$ is the composition
\[
\alpha:\Delta^{k}\star\Delta^{l}\xrightarrow{\id\star u}\Delta^{k}\star\Delta^{m}\xrightarrow{\widetilde{\sigma\varepsilon}}\cal C.
\]
We wish to show that if $\alpha$ belongs to $X_{m}$, then $\pr{\widetilde{\sigma\varepsilon},u}$
belongs to $A\pr{\varepsilon}\star\Delta^{m}\cup\pr{\cal C_{/\sigma\varepsilon}^{0}\star\partial\Delta^{m}}$.
The claim is obvious if $u$ is not surjective on vertices. So assume
that $u$ is surjective on vertices. Since $\alpha\vert\Delta^{l}$
is a degeneration of $\sigma$, it cannot factor through $\sigma\vert\partial\Delta^{n}$.
Thus $\alpha$ does not belong to $K_{0}$. Hence (since the vertices
of $\sigma$ do not belong to $\cal C^{0}$) we can find a factorization
of $\alpha$ of the form 
\[
\alpha:\Delta^{k}\star\Delta^{l}\xrightarrow{\id\star u'}\Delta^{k}\star\Delta^{m'}\xrightarrow{\widetilde{\sigma\varepsilon'}}\cal C,
\]
where $m'<m$, $\varepsilon':[m']\to[n]$ is a surjection, and $\widetilde{\sigma\varepsilon'}$
is an extension of $\sigma\varepsilon'$ which maps $\Delta^{k}$
into $\cal C^{0}$. Replacing $\Delta^{m'}$ by $\Delta^{u'[l]}$
if necessary, we may assume that $u'$ is surjective on vertices.
Factor the map $\widetilde{\sigma\varepsilon'}$ as 
\[
\Delta^{k}\star\Delta^{m'}\xrightarrow{s\star s'}\Delta^{p}\star\Delta^{m''}\xrightarrow{\tau}\cal C,
\]
where $p,m''\geq-1$, $s,s'$ are surjective on vertices and $\tau$
is nondegenerate. (Such a factorization exists because the vertices
of $\sigma$ do not belong to $\cal C^{0}$.) Then $\sigma\varepsilon'$
is a degeneration of $\tau\vert\Delta^{q}$, so Eilenberg-Zilber's
lemma implies that $\tau\vert\Delta^{m''}$ is a degeneration of $\sigma$.
Therefore, we can write $\tau\vert\Delta^{m''}=\sigma\varepsilon''$,
where $\varepsilon'':[m'']\to[n]$ is a surjective poset map. By Eilenberg-Zilber's
lemma, $\widetilde{\sigma\varepsilon'}$ is a degeneartion of $\tau$.
Since $m''<m$, this implies that the simplex $\pr{\widetilde{\sigma\varepsilon},u}$
belongs to $A\pr{\varepsilon}\star\Delta^{m}$, as required.

We next prove (ii). Since join functors preserves trivial cofibrations
in each variable, it will suffice to show that the inclusion $A\pr{\varepsilon}\subset\cal C_{/\sigma\varepsilon}^{0}$
is a trivial cofibration. Arguing as in the previous paragraph, we
see that $A\pr{\varepsilon}$ is the colimit of the diagram $F:\cal J_{0}\pr{\varepsilon}^{\op}\to\SS$
defined by $\pr{\varepsilon',\varepsilon''}\mapsto\cal C_{/\sigma\varepsilon'}^{0}$.
This diagram admits a natural transformation to the constant diagram
at $\cal C_{/\sigma\varepsilon}^{0}$ whose components are trivial
fibrations. It will therefore suffice to prove that the diagrams $F$
and the constant diagram at $\cal C_{/\sigma\varepsilon}^{0}$ are
projectively cofibrant. There is a functor $\cal J_{0}\pr{\varepsilon}^{\op}\to\omega$
which maps a factorization $[m]\to[m']\to[n]$ of $\varepsilon$ to
the integer $m'$. This functor endows the category $\cal J_{0}\pr{\varepsilon}^{\op}$
with the structure of a direct category. The claim is thus a consequence
of \cite[Theorem 5.1.3]{Hovey2007}.
\end{proof}
\begin{defn}
Let $X$ be a simplicial set over $\Delta^{1}$. Given a simplex $\Delta^{k}\to X$,
we define the \textbf{head} of $X$ to be the simplex $\Delta^{u^{-1}\pr 1}\to\Delta^{k}\to X$
and the \textbf{tail} to be the simplex $\Delta^{u^{-1}\pr 0}\to\Delta^{k}\to X$,
where $u$ denotes the composite $\Delta^{k}\to X\to\Delta^{1}$.
\end{defn}
\begin{prop}
\label{prop:heads_and_tails}Let % https://q.uiver.app/?q=WzAsMyxbMCwwLCJYIl0sWzIsMCwiWSJdLFsxLDEsIlxcRGVsdGFeMSJdLFswLDEsInAiXSxbMSwyXSxbMCwyXV0=
\[\begin{tikzcd}
	X && Y \\
	& {\Delta^1}
	\arrow["p", from=1-1, to=1-3]
	\arrow[from=1-3, to=2-2]
	\arrow[from=1-1, to=2-2]
\end{tikzcd}\]be a commutative diagram of simplicial sets. Let $n\geq0$, and let
$S\subset S'\subset Y_{1}=Y\times_{\Delta^{1}}\{1\}$ be simplicial
subsets satisfying the following conditions:
\begin{enumerate}
\item The simplicial set $S$ contains the $\pr{n-1}$-skeleton of $Y_{1}$.
\item The simplicial set $S'$ is generated by $S$ and a set $\Sigma$
of nondegenerate $n$-simplices of $Y_{1}$ which do not belong to
$S'$.
\end{enumerate}
Let $X\pr S$ denote the simplicial subset of $X$ spanned by the
simplices whose head lies over $S$, and define $X\pr{S'}$ similarly.
Let $\{\sigma_{a}\}_{a\in A}$ be the enumeration of all nondegenerate
simplices of $X_{1}$ whose image in $Y$ is a degeneration of a simplex
in $\Sigma$. Choose an ordering of $A$ so that that dimension of
$\sigma_{a}$ is a non-decreasing function of $a\in A$. For each
$a\in A$, define simplicial sets $X\pr{S'}_{<a}$, $X\pr{S'}_{\leq a}$,
$K_{a}$, and $K_{0,a}$ as follows:
\begin{itemize}
\item $X\pr{S'}_{<a}\subset X$ is the simplicial subset generated by $X\pr S$
and those simplices of $X$ whose head factors through $\sigma_{b}$
for some $b<a$.
\item $X\pr{S'}_{\leq a}\subset X$ is the simplicial subset generated by
$X\pr S$ and those simplices of $X$ whose head factors through $\sigma_{b}$
for some $b\leq a$.
\item $K_{a}\subset X$ is the simplicial subset consisting of the simplices
whose head factors through $\sigma_{a}$.
\item $K_{0,a}\subset X$ is the simplicial subset consisting of the simplices
whose head factors through $\partial\sigma_{a}=\sigma_{a}\vert\partial\Delta^{\dim\sigma_{a}}$.
\end{itemize}
Then the following holds:
\begin{enumerate}
\item $X\pr{S'}=X\pr S\cup\bigcup_{a\in A}X\pr{S'}_{\leq a}$.
\item For each $a\in A$, the square % https://q.uiver.app/?q=WzAsNCxbMCwwLCJLX3swLGF9Il0sWzAsMSwiS19hIl0sWzEsMCwiWChTJylfezxhfSJdLFsxLDEsIlgoUycpX3tcXGxlcSBhfSJdLFswLDFdLFswLDJdLFsyLDNdLFsxLDNdXQ==
\begin{equation}\label{d:h_and_t}
\begin{tikzcd}
	{K_{0,a}} & {X(S')_{<a}} \\
	{K_a} & {X(S')_{\leq a}}
	\arrow[from=1-1, to=2-1]
	\arrow[from=1-1, to=1-2]
	\arrow[from=1-2, to=2-2]
	\arrow[from=2-1, to=2-2]
\end{tikzcd}
\end{equation}of simplicial sets is cocartesian.
\end{enumerate}
\end{prop}
\begin{proof}
We start with (1). The containment $X\pr S\cup\bigcup_{a\in A}X\pr{S'}_{\leq a}\subset X\pr{S'}$
holds trivially. For the reverse inclusion, let $x$ be an arbitrary
simplex of $X\pr{S'}$. We must show that $x$ belongs to $X\pr S\cup\bigcup_{a\in A}X\pr{S'}_{\leq a}$.
Let $\tau\pr x:\Delta^{m}\to X_{1}$ denote the head of $x$. If $p\tau\pr x$
belongs to $S$, then $x$ belongs to $X\pr S$ and we are done. If
$\tau\pr x$ does not belong to $S$, then $p\tau\pr x$ factors through
a simplex $\sigma$ in $\Sigma$. If $p\tau\pr x$ factors through
the boundary of $\sigma$, then $p\tau\pr x$ belongs to the $\pr{n-1}$-skeleton
of $Y_{1}$ and hence $\tau\pr x$ belongs to $S$, a contradiction.
Therefore, $p\tau\pr x$ is a degeneration of $\sigma$. Now $\tau\pr x$
is a degeneration of some nondegenerate simplex $\tau'\pr x$ of $X_{1}$.
Then $p\tau\pr x$ is a degeneration of $p\tau'\pr x$, so Eilenberg-Zilber's
lemma implies that $p\tau'\pr x$ is a degeneration of $\sigma$.
Hence $\tau'\pr x=\sigma_{a}$ for some $a\in A$, and hence $x\in X\pr{S'}_{\leq a}$.

We next prove (2). We will write $K_{0}=K_{0,a}$ and $K=K_{a}$.
First we remark that $X\pr{S'}_{<a}$ contains $K_{0}$. Indeed, suppose
we are given a simplex $x$ of $X$ whose head $\tau\pr x:\Delta^{m}\to X_{1}$
facotors through $\partial\sigma_{a}$. By construction, there is
a commutative diagram % https://q.uiver.app/?q=WzAsNixbMSwwLCJcXERlbHRhXntcXGRpbVxcc2lnbWFfYX0iXSxbMiwwLCJcXERlbHRhXm4iXSxbMiwxLCJZXzEsIl0sWzEsMSwiWF8xIl0sWzAsMCwiXFxwYXJ0aWFsXFxEZWx0YV57XFxkaW1cXHNpZ21hX2F9Il0sWzAsMSwiXFxEZWx0YV5tIl0sWzAsMSwicyJdLFsxLDIsIlxcc2lnbWEiXSxbMywyLCJwIiwyXSxbMCwzLCJcXHNpZ21hX2EiLDJdLFs0LDAsIiIsMCx7InN0eWxlIjp7InRhaWwiOnsibmFtZSI6Imhvb2siLCJzaWRlIjoidG9wIn19fV0sWzUsMywiXFx0YXUoeCkiLDJdLFs1LDRdXQ==
\[\begin{tikzcd}
	{\partial\Delta^{\dim\sigma_a}} & {\Delta^{\dim\sigma_a}} & {\Delta^n} \\
	{\Delta^m} & {X_1} & {Y_1,}
	\arrow["s", from=1-2, to=1-3]
	\arrow["\sigma", from=1-3, to=2-3]
	\arrow["p"', from=2-2, to=2-3]
	\arrow["{\sigma_a}"', from=1-2, to=2-2]
	\arrow[hook, from=1-1, to=1-2]
	\arrow["{\tau(x)}"', from=2-1, to=2-2]
	\arrow[from=2-1, to=1-1]
\end{tikzcd}\]where $s$ is surjective on vertices. If the map $\Delta^{m}\to\Delta^{n}$
is not surjective on vertices, then $p\tau\pr x$ factors through
the $\pr{n-1}$-skeleton of $Y_{1}$ and hence $x$ belongs to $X\pr S$.
If the map $\Delta^{m}\to\Delta^{n}$ is surjective on vertices, then
write $\tau\pr x$ as a degeneration of a nondegenerate simplex $\tau'\pr x$
of $X_{1}$. Eilenberg-Zilber's lemma implies that the simplex $p\tau'\pr x$
must be a degeneration of $\sigma$. Thus $p\tau'\pr x=\sigma_{b}$
for some $b\in A$. Since $\tau\pr x$ factors through the boundary
of $\sigma_{a}$, the dimension of $\sigma_{b}$ is strictly smaller
than that of $\sigma_{a}$. Thus $b<a$. Hence $x$ belongs to $X\pr{S'}_{<a}$.

By what we have just shown in the previous paragraph, the diagram
(\ref{d:h_and_t}) is well-defined. We now show that it is cocartesian.
By definition, $X\pr{S'}_{\leq a}$ is the union of $K$ and $X\pr{S'}_{<a}$.
Therefore, it suffices to show that $K_{0}$ is the intersection of
$K$ and $X\pr{S'}_{<a}$. So let $x$ be a simplex of $K$. We must
show that, if $x$ does not belong to $K_{0}$, then $x$ does not
belong to $X\pr{S'}_{<a}$ either. Let $\tau\pr x:\Delta^{m}\to X_{1}$
be the head of $x$. Since $x$ does not belong to $K_{0}$, $\tau\pr x$
is a degeneration of $\sigma_{a}$. Thus $p\pr{\tau\pr x}$ is a degeneration
of some simplex $\sigma\in\Sigma$. In particular, $p\pr{\tau\pr x}$
does not belong to $S$, so $\tau\pr x$ does not belong to $X\pr S$.
Therefore, should $\tau\pr x$ belong to $X\pr{S'}_{<a}$, then $\tau\pr x$
must factor through some $\sigma_{b}$ for some $b<a$. If $\tau\pr x$
factors through $\partial\sigma_{b}$ for some $b<a$, then we would
have $\dim\sigma_{a}<\dim\sigma_{b}$, a contradiction. So $\tau\pr x$
is a degeneartion of $\sigma_{b}$; but then $a=b$, a contradiction.
Thus $x$ does not belong to $X\pr{S'}_{<a}$, as required.
\end{proof}

\subsection{Proof of Theorem \ref{thm:3.1.2.3}}

In this final subsection of this note, we will give a proof of Theorem
\ref{thm:3.1.2.3} by elaborating the proof given by Lurie in \cite{HA}.

The proof proceeds by a simplex-by-simplex argument. For this, we
will classify simplices of $N\pr{\Fin_{\ast}}\times\Delta^{\{1,\dots,n\}}$
into five (somewhat artificial) groups.
\begin{defn}
Let $n\geq1$ and let $\alpha$ be a morphism in $N\pr{\Fin_{\ast}}\times\Delta^{\{1,\dots,n\}}$
with image $\alpha_{0}:\inp m\to\inp n$ in $N\pr{\Fin_{\ast}}$.
We say that $\alpha$ is:
\begin{enumerate}
\item \textbf{active} if $\alpha_{0}$ is active;
\item \textbf{strongly inert} if $\alpha_{0}$ is inert, the induced injection
$\inp n^{\circ}\to\inp m^{\circ}$ is order-perserving, and the image
of $\alpha$ in $\Delta^{\{1,\dots,n\}}$ is degenerate; and
\item \textbf{neutral} if it is neigher active nor strongly inert.
\end{enumerate}
Note that active morphisms and strongly inert morphisms are closed
under composition. Also, every morphism in $N\pr{\Fin_{\ast}}\times\Delta^{\{1,\dots,n\}}$
can be factored uniquely as a composition of a strongly inert map
followed by an active map.

Let $\sigma$ be an $m$-simplex of $N\pr{\Fin_{\ast}}\times\Delta^{\{1,\dots,n\}}$
depicted as
\[
\pr{\inp{k_{0}},e_{0}}\xrightarrow{\alpha_{\sigma}\pr 1}\cdots\xrightarrow{\alpha_{\sigma}\pr m}\pr{\inp{k_{m}},e_{m}}.
\]
We will say that $\sigma$ is \textbf{closed} if $k_{m}=1$, and \textbf{open}
otherwise. We say that $\sigma$ is \textbf{complete}\footnote{Lurie uses te term ``new'' instead of ``complete.''}
if $\{e_{0},\dots,e_{m}\}=\{1,\dots,n\}$ and \textbf{incomplete}
otherwise. Every nondegenerate simplex of $N\pr{\Fin_{\ast}}\times\Delta^{\{1,\dots,n\}}$
is a face of a nondegenerate complete simplex. 

We partition the set of nondegenerate complete simplices of $N\pr{\Fin_{\ast}}\times\Delta^{\{1,\dots n\}}$
into five groups $G_{\pr 1},G_{\pr 2},G'_{\pr 2},G_{\pr 3},G_{\pr 3}'$
as follows: $m$-simplex of $N\pr{\Fin_{\ast}}\times\Delta^{\{1,\dots,n\}}$.
Write $\alpha_{\sigma}\pr i=\sigma\vert\Delta^{\{i,i+1\}}$. Let $0\le k\leq m$
be the minimal integer such that $\alpha_{\sigma}\pr i$ is strongly
inert for every $i>k$, and let $0\leq j\leq k$ be the minimal integer
such that $\alpha_{\sigma}\pr i$ is active for every $j<i\leq k$.
\begin{itemize}
\item If $j=0$, $k=m$, and $\sigma$ is closed, then $\sigma$ belongs
to $G_{\pr 1}$.
\item If $j=0$, $k<m$, and $\sigma$ is closed, then $\sigma$ belongs
to $G_{\pr 2}$.
\item If $j=0$ and $\sigma$ is open, then $\sigma$ belongs to $G'_{\pr 2}$.
\item If $j\geq1$ and $\alpha_{\sigma}\pr j$ is strongly inert, then $\sigma$
belongs to $G_{\pr 3}$.
\item If $j\geq1$ and $\alpha_{\sigma}\pr j$ is neutral, then $\sigma$
belongs to $G_{\pr 3}'$.
\end{itemize}
Given an $m$-simplex $\sigma\in G_{\pr 2}$, we define its \textbf{associate}
$\sigma'\in G_{\pr 2}'$ by $\sigma'=\sigma\vert\Delta^{\{0,\dots,m-1\}}$.
Given an $m$-simplex $\sigma\in G_{\pr 3}$, then we define its \textbf{associate}
$\sigma'$ by $\sigma'=\sigma\partial_{j}\in G_{\pr 3}'$, where $j$
is the integer defined as above.
\end{defn}
%
\begin{proof}
[Proof of Theorems \ref{thm:3.1.2.3}]We will regard $\cal M^{\t}$
and $N\pr{\Fin_{\ast}}\times\Delta^{n}$ as a simplicial set over
$\Delta^{1}$ by means of the map $\Delta^{n}\to\Delta^{1}$ which
maps the vertex $0\in\Delta^{n}$ to the vertex $0\in\Delta^{1}$
and the remaining vertices to the vertex $1\in\Delta^{1}$. Given
a simplicial subset $S\subset N\pr{\Fin_{\ast}}\times\Delta^{\{1,\dots,n\}}$,
we let $\cal M_{S}^{\t}\subset\cal M^{\t}$ denote the simplicial
subset consisting of the simplices whose tail lies over $S$. We will
also write $\overline{S}=\pr{N\pr{\Fin_{\ast}}\times\Delta^{\{1,\dots,n\}}}_{S}$.

The implication (a)$\implies$(b) for part (A) is obvious. Assume
therefore that condition (b) is satisfied if $n=1$. For each $m\geq0$,
let $F\pr m$ denote the simplicial subset of $N\pr{\Fin_{\ast}}\times\Delta^{\{1,\dots,n\}}$
generated by the nondegenerate simplices $\sigma$ satisfying one
of the following conditions:
\begin{itemize}
\item $\sigma$ is incomplete.
\item $\sigma$ has dimension less than $m$.
\item $\sigma$ has dimension $m$ and belongs to $G_{\pr 2}$ or $G_{\pr 3}$.
\end{itemize}
Observe that $\cal M_{F\pr 0}^{\t}=\cal M^{\t}\times_{\Delta^{n}}\Lambda_{0}^{n}$.
We will complete the proof by inductively constructing a map $f_{m}:\cal M_{F\pr m}^{\t}\to\cal C^{\t}$
which makes the diagram % https://q.uiver.app/?q=WzAsNCxbMCwwLCJcXG1hdGhjYWx7TX1eXFxvdGltZXMgX3tGKG0tMSl9Il0sWzAsMSwiXFxtYXRoY2Fse019Xlxcb3RpbWVzIF97RihtKX0iXSxbMSwxLCJcXG1hdGhjYWx7T31eXFxvdGltZXMgIl0sWzEsMCwiXFxtYXRoY2Fse0N9Xlxcb3RpbWVzIl0sWzAsMSwiIiwwLHsic3R5bGUiOnsidGFpbCI6eyJuYW1lIjoiaG9vayIsInNpZGUiOiJ0b3AifX19XSxbMSwyLCJnXFx2ZXJ0XFxtYXRoY2Fse019Xlxcb3RpbWVzIF97RihtKX0iLDJdLFszLDIsInEiXSxbMCwzLCJmX3ttLTF9Il0sWzEsMywiZl97bX0iLDFdXQ==
\[\begin{tikzcd}
	{\mathcal{M}^\otimes _{F(m-1)}} & {\mathcal{C}^\otimes} \\
	{\mathcal{M}^\otimes _{F(m)}} & {\mathcal{O}^\otimes }
	\arrow[hook, from=1-1, to=2-1]
	\arrow["{g\vert\mathcal{M}^\otimes _{F(m)}}"', from=2-1, to=2-2]
	\arrow["q", from=1-2, to=2-2]
	\arrow["{f_{m-1}}", from=1-1, to=1-2]
	\arrow["{f_{m}}"{description}, from=2-1, to=1-2]
\end{tikzcd}\]commutative, and such that $f_{1}$ has the following special properties
if $n=1$:
\begin{itemize}
\item [(i)]For each object $B\in\cal M^{\t}\times_{\Delta^{1}}\{1\}$,
the map 
\[
\pr{\pr{\cal M_{\act}^{\t}}_{/B}\times_{\Delta^{1}}\{0\}}^{\rcone}\to\cal M_{\pr 1}^{\t}\xrightarrow{f_{1}}\cal C^{\t}
\]
is an operadic $q$-colimit diagram.
\item [(ii)]For every inert morphism $e:M'\to M$ in $\cal M^{\t}\times_{\Delta^{1}}\{1\}$
such that $M\in\cal M$, the functor $f_{1}$ carries $e$ to an inert
morphism in $\cal C^{\t}$.
\end{itemize}
Fix $m>0$, and suppose that $f_{m-1}$ has been constructed. Observe
that $F\pr m$ is obtained from $F\pr{m-1}$ by adjoining the following
simplices:
\begin{itemize}
\item The $\pr{m-1}$-simplices in $G_{\pr 1}$.
\item The $\pr{m-1}$-simplices in $G'_{\pr 2}$ without associates.
\item The $m$-simplices in $G_{\pr 2}$ and $G_{\pr 3}$.
\end{itemize}
We define simplicial subsets $F'\pr m\subset F''\pr m\subset F\pr m$
as follows: $F'\pr m$ is generated by $F\pr{m-1}$ and the $\pr{m-1}$-simplices
in $G_{\pr 1}$; $F'\pr{m-1}$ is generated by $F'\pr m$ and the
$\pr{m-1}$-simplices of $G'_{\pr 2}$ without associates. Our strategy
is to extend $f_{m-1}$ to $\cal M_{F'\pr m}^{\t}$, then to $\cal M_{F''\pr m}^{\t}$,
and then to $\cal M_{F\pr m}^{\t}$. 

\begin{enumerate}[label=(\textbf{Step \arabic*}),wide =0.5\parindent, listparindent=1.5em]

\item We will extend $f_{m-1}$ to a map $f'_{m}:\cal M_{F'\pr m}^{\t}\to\cal C^{\t}$
over $\cal O^{\t}$. 

Let $\{\sigma_{a}\}_{a\in A}$ be the collection of nondegenerate
simplices of $\cal M^{\t}\times_{\Delta^{n}}\Delta^{\{1,\dots,n\}}$
whose image in $N\pr{\Fin_{\ast}}\times\Delta^{\{1,\dots,n\}}$ is
a degeneration of some $\pr{m-1}$-simplex of $G_{\pr 1}$. Choose
a well-ordering on $A$ so that $\dim\sigma_{a}$ is non-decreasing
in $a\in A$. For each $a\in A$, let $\cal M_{<a}^{\t}$ denote the
simplicial subset spanned by $\cal M_{F\pr{m-1}}^{\t}$ and the simplices
of $\cal M^{\t}$ whose tail factors through $\sigma_{b}$ for some
$b<a$. We define $\cal M_{\leq a}^{\t}$ similarly. According to
Proposition \ref{prop:heads_and_tails}, we have $\cal M_{F'\pr m}^{\t}=\bigcup_{a\in A}\cal M_{\leq a}^{\t}$,
so it suffices to extend $f_{m-1}$ to an $A$-sequence $f^{\leq a}:\cal M_{\leq a}^{\t}\to\cal C^{\t}$
over $\cal O^{\t}$. 

The construction is inductive. Let $a\in A$, and suppose that $f^{\leq b}$
has been constructed for $b<a$. These maps determine a map $f^{<a}:\cal M_{<a}^{\t}\to\cal C^{\t}$
extending $f_{m-1}$. Let $K_{0}\subset K\subset\cal M^{\t}$ denote
the simplicial subset consisting of the simplices of $\cal M^{\t}$
whose head factors through $\sigma_{a}\vert\partial\Delta^{\dim\sigma_{a}}$
and $\sigma_{a}$, respectively. According to Lemma \ref{lem:3.1.2.5},
the left hand square of the commutative diagram% https://q.uiver.app/?q=WzAsNixbMCwwLCIoXFxtYXRoY2Fse019Xlxcb3RpbWVzIF97L1xcc2lnbWFfYX1cXHRpbWVzX3tcXERlbHRhXm59IFxcezBcXH0pXFxzdGFyIFxccGFydGlhbFxcRGVsdGEgXntcXGRpbVxcc2lnbWEgX2F9Il0sWzAsMSwiKFxcbWF0aGNhbHtNfV5cXG90aW1lcyBfey9cXHNpZ21hX2F9XFx0aW1lc197XFxEZWx0YV5ufSBcXHswXFx9KVxcc3RhciBcXERlbHRhIF57XFxkaW1cXHNpZ21hIF9hfSJdLFsxLDAsIktfMCJdLFsxLDEsIksiXSxbMiwwLCJcXG1hdGhjYWx7TX1eXFxvdGltZXMgX3s8YX0iXSxbMiwxLCJcXG1hdGhjYWx7TX1eXFxvdGltZXMgX3tcXGxlcSBhfSJdLFswLDJdLFsxLDNdLFsyLDNdLFsyLDRdLFs0LDVdLFszLDVdLFswLDFdXQ==
\[\begin{tikzcd}
	{(\mathcal{M}^\otimes _{/\sigma_a}\times_{\Delta^n} \{0\})\star \partial\Delta ^{\dim\sigma _a}} & {K_0} & {\mathcal{M}^\otimes _{<a}} \\
	{(\mathcal{M}^\otimes _{/\sigma_a}\times_{\Delta^n} \{0\})\star \Delta ^{\dim\sigma _a}} & K & {\mathcal{M}^\otimes _{\leq a}}
	\arrow[from=1-1, to=1-2]
	\arrow[from=2-1, to=2-2]
	\arrow[from=1-2, to=2-2]
	\arrow[from=1-2, to=1-3]
	\arrow[from=1-3, to=2-3]
	\arrow[from=2-2, to=2-3]
	\arrow[from=1-1, to=2-1]
\end{tikzcd}\]is homotopy cocartesian. The right hand square is cocartesian by Proposition
\ref{prop:heads_and_tails}. It follows that the map
\[
\pr{\cal M_{/\sigma_{a}}^{\t}\times_{\Delta^{n}}\{0\}}\star\Delta^{\dim\sigma_{a}}\amalg_{\pr{\cal M_{/\sigma_{a}}^{\t}\times_{\Delta^{n}}\{0\}}\star\partial\Delta^{\dim\sigma_{a}}}\cal M_{<a}^{\t}\to\cal M_{\leq a}^{\t}
\]
is a trivial cofibration in the Joyal model structure. Thus we only
need to extend the composite
\[
g_{0}:\pr{\cal M_{/\sigma_{a}}^{\t}\times_{\Delta^{n}}\{0\}}\star\partial\Delta^{\dim\sigma_{a}}\to\cal M_{<a}^{\t}\xrightarrow{f^{<a}}\cal C^{\t}
\]
to a map $\pr{\cal M_{/\sigma_{a}}^{\t}\times_{\Delta^{n}}\{0\}}\star\Delta^{\dim\sigma_{a}}\to\cal C^{\t}$
over $\cal O^{\t}$.

Assume first that $\sigma_{a}$ is zero-dimensional, so that, in particular,
$m=1$. If $n>1$, then $F\pr 0=F'\pr 1$ and there is nothing to
do. If $n=1$, then let $B\in\cal M^{\t}$ be the image of $\sigma_{a}$.
Since $\sigma_{a}$ is closed, the object $B$ lies in $\cal M\times_{\Delta^{1}}\{1\}$.
Using the inert-active factorization system in $\cal M^{\t}$, we
see that the inclusion $\pr{\pr{\cal M_{\act}^{\t}}_{/B}\times_{\Delta^{1}}\{0\}}\subset\cal M_{/B}^{\t}\times_{\Delta^{1}}\{0\}$
is a right adjoint, hence final. So our assumption (b) ensures that
we can find the desired extension $g$. Note that condition (i) is
satisfied with this particular construction.

Assume next that $\sigma_{a}$ has positive dimension. Since $\Delta^{\dim\sigma_{a}}$
has an initial vertex, we see as in the previous paragraph that the
inclusion $\pr{\cal M_{\act}^{\t}}_{/\sigma_{a}}\times_{\Delta^{n}}\{0\}\subset\cal M_{/\sigma_{a}}^{\t}\times_{\Delta^{1}}\{0\}$
is a right adjoint, and hence final. Thus, by virtue of \cite[Proposition 3.1.1.7]{HA},
it suffices to show that the map
\[
\pr{\pr{\cal M_{\act}^{\t}}_{/\sigma_{a}}\times_{\Delta^{n}}\{0\}}\star\{0\}\to\cal M_{<a}^{\t}\xrightarrow{f^{<a}}\cal C^{\t}
\]
is an operadic $q$-colimit diagram. Let $B=\sigma_{a}\pr 0$. The
map $\pr{\cal M_{\act}^{\t}}_{/\sigma_{a}}\times_{\Delta^{n}}\{0\}\to\pr{\cal M_{\act}^{\t}}_{/B}\times_{\Delta^{n}}\{0\}$
is a trivial fibration, so it suffices to show that the map
\[
\phi:\pr{\pr{\cal M_{\act}^{\t}}_{/B}\times_{\Delta^{n}}\{0\}}\star\{0\}\to\cal M_{<a}^{\t}\xrightarrow{f^{<a}}\cal C^{\t}
\]
is an operadic $q$-colimit cone. Let $\inp p\in N\pr{\Fin_{\ast}}$
be the image of the object $B$. If $p=0$, then $\pr{\cal M_{\act}^{\t}}_{/B}\times_{\Delta^{n}}\{0\}$
is a contractible Kan complex, and for each object $\alpha:A\to B$
in $\pr{\cal M_{\act}^{\t}}_{/B}\times_{\Delta^{n}}\{0\}$, the map
$\phi\vert\{\alpha\}\star\{0\}$ is an equivalence in $\cal C^{\t}$
(since $\cal C_{\inp 0}^{\t}$ is a contractible Kan complex). Thus
$\phi$ is an operadic $q$-colimit cone by \cite[Example 3.1.1.6]{HA}.
If $p=1$, the claim follows from (i). If $p>1$, choose for each
$1\leq i\leq p$ an inert map $B\to B_{i}$ in $\cal M_{\act}^{\t}\times_{\Delta^{n}}\{1\}$
over $\rho^{i}:\inp p\to\inp 1$. Note that since $p>1$, we have
$m\geq2$, so that $f^{<a}$ is defined on $\cal M_{F\pr 1}^{\t}$. 

Using Proposition \ref{prop:oplus_p-limit}, choose a direct sum functor
$\bigoplus_{i=1}^{n}:\pr{\cal M_{\act}^{\t}}^{p}\times_{\pr{\Delta^{1}}^{p}}\Delta^{1}\to\cal M^{\t}$
so that there is an inert natural transformation $\widetilde{h}_{i}:\bigoplus_{i=1}^{n}\to\opn{pr}_{i}$
making the diagram % https://q.uiver.app/?q=WzAsNCxbMCwwLCIoKFxcbWF0aGNhbHtNfV5cXG90aW1lc197XFxtYXRocm17YWN0fX0pXnBcXHRpbWVzIF97KFxcRGVsdGFebilecH1cXERlbHRhXm4pXFx0aW1lcyBcXERlbHRhIF4xIl0sWzEsMCwiXFxtYXRoY2Fse019Xlxcb3RpbWVzICJdLFsxLDEsIlxcRGVsdGFeblxcdGltZXMgTihcXG1hdGhzZntGaW59X1xcYXN0KSJdLFswLDEsIihcXERlbHRhXm5cXHRpbWVzIE4oXFxtYXRoc2Z7RmlufV9cXGFzdCApX3tcXG1hdGhybXthY3R9fSlecClcXHRpbWVzIFxcRGVsdGFeMSJdLFswLDEsIlxcd2lkZXRpbGRle2h9X2kiXSxbMSwyLCJwIl0sWzMsMiwiXFxvcGVyYXRvcm5hbWV7aWR9X3tcXERlbHRhXm59XFx0aW1lcyBoX2kiLDJdLFswLDNdXQ==
\[\begin{tikzcd}
	{((\mathcal{M}^\otimes_{\mathrm{act}})^p\times _{(\Delta^n)^p}\Delta^n)\times \Delta ^1} & {\mathcal{M}^\otimes } \\
	{(\Delta^n\times N(\mathsf{Fin}_\ast )_{\mathrm{act}})^p)\times \Delta^1} & {\Delta^n\times N(\mathsf{Fin}_\ast)}
	\arrow["{\widetilde{h}_i}", from=1-1, to=1-2]
	\arrow["p", from=1-2, to=2-2]
	\arrow["{\operatorname{id}_{\Delta^n}\times h_i}"', from=2-1, to=2-2]
	\arrow[from=1-1, to=2-1]
\end{tikzcd}\]commutative. There is an equivalence $\alpha:B\xrightarrow{\simeq}\bigoplus_{i=1}^{p}B_{i}$
lying over the identity of $\inp p$. There is a natural equivalence
\[
H:\pr{\pr{\cal M_{\act}^{\t}}_{/\alpha}\times_{\Delta^{n}}\{0\}}^{\rcone}\times\Delta^{1}\to\cal M^{\t}
\]
which makes the diagram % https://q.uiver.app/?q=WzAsNixbMCwwLCIoKFxcbWF0aGNhbHtNfV5cXG90aW1lcyBfe1xcbWF0aHJte2FjdH19KV97L1xcYWxwaGF9XFx0aW1lcyBfe1xcRGVsdGFebn1cXHswXFx9KV5cXHRyaWFuZ2xlcmlnaHRcXHRpbWVzIFxcezBcXH0iXSxbMSwwLCIoKFxcbWF0aGNhbHtNfV5cXG90aW1lcyBfe1xcbWF0aHJte2FjdH19KV97L0J9XFx0aW1lcyBfe1xcRGVsdGFebn1cXHswXFx9KV5cXHRyaWFuZ2xlcmlnaHQiXSxbMCwxLCIoKFxcbWF0aGNhbHtNfV5cXG90aW1lcyBfe1xcbWF0aHJte2FjdH19KV97L1xcYWxwaGF9XFx0aW1lcyBfe1xcRGVsdGFebn1cXHswXFx9KV5cXHRyaWFuZ2xlcmlnaHRcXHRpbWVzIFxcRGVsdGFeMSJdLFsxLDEsIlxcbWF0aGNhbHtNfV5cXG90aW1lcyAiXSxbMCwyLCIoKFxcbWF0aGNhbHtNfV5cXG90aW1lcyBfe1xcbWF0aHJte2FjdH19KV97L1xcYWxwaGF9XFx0aW1lcyBfe1xcRGVsdGFebn1cXHswXFx9KV5cXHRyaWFuZ2xlcmlnaHRcXHRpbWVzIFxcezFcXH0iXSxbMSwyLCIoKFxcbWF0aGNhbHtNfV5cXG90aW1lcyBfe1xcbWF0aHJte2FjdH19KV97L1xcYmlnb3BsdXNfe2k9MX1ebkJfaX1cXHRpbWVzIF97XFxEZWx0YV5ufVxcezBcXH0pXlxcdHJpYW5nbGVyaWdodFxcdGltZXMgXFx7MVxcfSJdLFswLDFdLFswLDJdLFsxLDNdLFsyLDNdLFs0LDJdLFs1LDNdLFs0LDVdXQ==
\[\begin{tikzcd}
	{((\mathcal{M}^\otimes _{\mathrm{act}})_{/\alpha}\times _{\Delta^n}\{0\})^\triangleright\times \{0\}} & {((\mathcal{M}^\otimes _{\mathrm{act}})_{/B}\times _{\Delta^n}\{0\})^\triangleright} \\
	{((\mathcal{M}^\otimes _{\mathrm{act}})_{/\alpha}\times _{\Delta^n}\{0\})^\triangleright\times \Delta^1} & {\mathcal{M}^\otimes } \\
	{((\mathcal{M}^\otimes _{\mathrm{act}})_{/\alpha}\times _{\Delta^n}\{0\})^\triangleright\times \{1\}} & {((\mathcal{M}^\otimes _{\mathrm{act}})_{/\bigoplus_{i=1}^nB_i}\times _{\Delta^n}\{0\})^\triangleright\times \{1\}}
	\arrow[from=1-1, to=1-2]
	\arrow[from=1-1, to=2-1]
	\arrow[from=1-2, to=2-2]
	\arrow[from=2-1, to=2-2]
	\arrow[from=3-1, to=2-1]
	\arrow[from=3-2, to=2-2]
	\arrow[from=3-1, to=3-2]
\end{tikzcd}\]commutative, such that the restriction of $H$ to $\pr{\pr{\cal M_{\act}^{\t}}_{/\alpha}\times_{\Delta^{n}}\{0\}}\times\Delta^{1}$
is the constant natural transformation at the projection $\pr{\pr{\cal M_{\act}^{\t}}_{/\alpha}\times_{\Delta^{n}}\{0\}}\to\cal M^{\t}$
and the restriction $H\vert\{\infty\}\times\Delta^{1}$ is given by
$\alpha$. This natural equivalence takes values in $\cal M_{F\pr 1}^{\t}$,
because $\alpha$ lies in $\cal M_{F\pr 1}^{\t}$. So it suffices
to show that the composite 
\[
\phi:\pr{\pr{\cal M_{\act}^{\t}}_{/\bigoplus_{i=1}^{p}B_{i}}\times_{\Delta^{n}}\{0\}}^{\rcone}\to\cal M_{F\pr 1}^{\t}\xrightarrow{f_{1}}\cal C^{\t}
\]
is an operadic $q$-colimit diagram. Now consider the commutative
diagram % https://q.uiver.app/?q=WzAsNSxbMCwwLCIoXFxwcm9kX3sxXFxsZXEgaVxcbGVxIHB9KFxcbWF0aGNhbHtNfV5cXG90aW1lcyBfe1xcbWF0aHJte2FjdH19KV97L0JfaX1cXHRpbWVzIF97XFxEZWx0YV5ufVxcezBcXH0pXlxcdHJpYW5nbGVyaWdodCJdLFsxLDAsIigoKFxcbWF0aGNhbHtNfV5cXG90aW1lcyBfe1xcbWF0aHJte2FjdH19KV5wXFx0aW1lcyBfeyhcXERlbHRhXm4pXnB9XFxEZWx0YV5uKV97LyhCXzEsXFxkb3RzICxCX3ApfVxcdGltZXMgX3tcXERlbHRhXm59XFx7MFxcfSleXFx0cmlhbmdsZXJpZ2h0Il0sWzEsMSwiKChcXG1hdGhjYWx7TX1eXFxvdGltZXMgX3tcXG1hdGhybXthY3R9fSlfey9cXGJpZ29wbHVzX3tpPTF9XnBCX2l9XFx0aW1lcyBfe1xcRGVsdGFebn1cXHswXFx9KV5cXHRyaWFuZ2xlcmlnaHQiXSxbMSwyLCJcXG1hdGhjYWx7TX1eXFxvdGltZXMgLiJdLFswLDIsIihcXG1hdGhjYWx7TX1eXFxvdGltZXMgX3tcXG1hdGhybXthY3R9fSlecFxcdGltZXMgX3soXFxEZWx0YV5uKV5wfVxcRGVsdGFebiJdLFswLDEsIlxcY29uZyJdLFsxLDIsIlxccHNpIl0sWzIsM10sWzQsMywiXFxiaWdvcGx1c197aT0xfV5wIiwyXSxbMCw0XSxbMSwyLCJcXHNpbWVxIiwyXV0=
\[\begin{tikzcd}
	{(\prod_{1\leq i\leq p}(\mathcal{M}^\otimes _{\mathrm{act}})_{/B_i}\times _{\Delta^n}\{0\})^\triangleright} & {(((\mathcal{M}^\otimes _{\mathrm{act}})^p\times _{(\Delta^n)^p}\Delta^n)_{/(B_1,\dots ,B_p)}\times _{\Delta^n}\{0\})^\triangleright} \\
	& {((\mathcal{M}^\otimes _{\mathrm{act}})_{/\bigoplus_{i=1}^pB_i}\times _{\Delta^n}\{0\})^\triangleright} \\
	{(\mathcal{M}^\otimes _{\mathrm{act}})^p\times _{(\Delta^n)^p}\Delta^n} & {\mathcal{M}^\otimes .}
	\arrow["\cong", from=1-1, to=1-2]
	\arrow["\psi", from=1-2, to=2-2]
	\arrow[from=2-2, to=3-2]
	\arrow["{\bigoplus_{i=1}^p}"', from=3-1, to=3-2]
	\arrow[from=1-1, to=3-1]
	\arrow["\simeq"', from=1-2, to=2-2]
\end{tikzcd}\]Here the map $\psi$ is the equivalence of $\infty$-categories induced
by the direct sum functor (Proposition \ref{prop:directsum_slice_equiv}).
In light of the commutativity of this diagram, the composite
\[
\eta:\pr{\prod_{1\leq i\leq p}\pr{\cal M_{\act}^{\t}}_{/B_{i}}\times_{\Delta^{n}}\{0\}}^{\rcone}\to\pr{\cal M_{\act}^{\t}}^{p}\times_{\pr{\Delta^{n}}^{p}}\Delta^{n}\xrightarrow{\bigoplus_{i=1}^{p}}\cal M^{\t}
\]
takes values in $\cal M_{F\pr 1}^{\t}$, and it suffices to show that
the composite $f_{1}\eta$ is an operadic $q$-colimit diagram. Now
the inert natural transformation $\widetilde{h}_{i}$ induces an inert
natural transfromation from $\eta$ to the composite 
\[
\eta_{i}:\pr{\prod_{1\leq i\leq p}\pr{\cal M_{\act}^{\t}}_{/B_{i}}\times_{\Delta^{n}}\{0\}}^{\rcone}\to\pr{\pr{\cal M_{\act}^{\t}}_{/B_{i}}\times_{\Delta^{n}}\{0\}}^{\rcone}\to\cal M^{\t}.
\]
Since $F\pr 1$ contains the $1$-simplices in $G_{\pr 2}$, this
natural transformation takes values in $\cal M_{F\pr 1}^{\t}$. Since
$f_{1}$ satisfies (ii), we deduce that the composite $f_{1}\eta$
admits an inert natural transformation $H_{i}$ to the composite $f_{1}\eta_{i}$,
such that for each vertex $v$ in $\pr{\prod_{1\leq i\leq p}\pr{\cal M_{\act}^{\t}}_{/B_{i}}\times_{\Delta^{n}}\{0\}}^{\rcone}$,
the components $\{H_{i}\pr v\}_{1\leq i\leq p}$ form a $q$-limit
cone. It follows from Corollary \ref{cor:3.1.1.8} and (i) that $f_{1}\eta$
is an operadic $q$-colimit diagram, as desired.

\item We will extend the map $f'_{m}$ in Step 1 to a map $f''_{m}:\cal M_{F''\pr m}^{\t}\to\cal C^{\t}$
over $\cal O^{\t}$. 

We argue as in Step 1. Let $\{\sigma_{a}\}_{a\in A}$ be the set of
all nondegenerate simplices of $\cal M^{\t}\times_{\Delta^{n}}\Delta^{\{1,\dots,n\}}$
whose image in $N\pr{\Fin_{\ast}}\times\Delta^{\{1,\dots,n\}}$ is
a degeneration of an $\pr{m-1}$-simplex in $G'_{\pr 2}$ without
associates. Choose a well-ordering of the set $A$ so that $\dim\sigma_{a}$
is a non-decreasing function of $a$. For each $a\in A$, let $\cal M_{<a}^{\t}$
denote the simplicial subset spanned by $\cal M_{F'\pr m}^{\t}$ and
the simplices of $\cal M^{\t}$ whose tail factors through $\sigma_{b}$
for some $b<a$. We define $\cal M_{\leq a}^{\t}$ similarly. (The
notations $\cal M_{<a}^{\t}$ and $\cal M_{\leq a}^{\t}$ are in conflict
with the ones introduced in Step 1, but there should not be any confusion.)
By Proposition \ref{prop:heads_and_tails}, we have $\cal M_{F''\pr m}^{\t}=\bigcup_{a\in A}\cal M_{\leq a}^{\t}$,
so it suffices to extend $f'_{m}$ to an $A$-sequence $f^{\leq a}:\cal M_{\leq a}^{\t}\to\cal C^{\t}$
over $\cal O^{\t}$. The construction is inductive. Suppose $f^{\leq b}$
has been constructed for $b<a$, and let $f^{<a}:\cal M_{<a}^{\t}\to\cal C^{\t}$
be their amalgamation. Just as in Step 1, we are reduced to solving
a lifting problem of the form % https://q.uiver.app/?q=WzAsNixbMCwwLCIoXFxtYXRoY2Fse019Xlxcb3RpbWVzIF97L1xcc2lnbWFfYX1cXHRpbWVzX3tcXERlbHRhXm59IFxcezBcXH0pXFxzdGFyIFxccGFydGlhbFxcRGVsdGEgXntcXGRpbVxcc2lnbWEgX2F9Il0sWzAsMSwiKFxcbWF0aGNhbHtNfV5cXG90aW1lcyBfey9cXHNpZ21hX2F9XFx0aW1lc197XFxEZWx0YV5ufSBcXHswXFx9KVxcc3RhciBcXERlbHRhIF57XFxkaW1cXHNpZ21hIF9hfSJdLFsxLDAsIlxcbWF0aGNhbHtNfV5cXG90aW1lcyBfezxhfSJdLFsyLDAsIlxcbWF0aGNhbHtDfV5cXG90aW1lcyJdLFsyLDEsIlxcbWF0aGNhbHtPfV5cXG90aW1lcyJdLFsxLDEsIlxcbWF0aGNhbHtNfV5cXG90aW1lcyAiXSxbMCwxXSxbMCwyXSxbMiwzLCJmXns8YX0iXSxbMSw1XSxbNSw0XSxbMyw0LCJxIl0sWzEsMywiIiwxLHsic3R5bGUiOnsiYm9keSI6eyJuYW1lIjoiZGFzaGVkIn19fV1d
\[\begin{tikzcd}
	{(\mathcal{M}^\otimes _{/\sigma_a}\times_{\Delta^n} \{0\})\star \partial\Delta ^{\dim\sigma _a}} & {\mathcal{M}^\otimes _{<a}} & {\mathcal{C}^\otimes} \\
	{(\mathcal{M}^\otimes _{/\sigma_a}\times_{\Delta^n} \{0\})\star \Delta ^{\dim\sigma _a}} & {\mathcal{M}^\otimes } & {\mathcal{O}^\otimes.}
	\arrow[from=1-1, to=2-1]
	\arrow[from=1-1, to=1-2]
	\arrow["{f^{<a}}", from=1-2, to=1-3]
	\arrow[from=2-1, to=2-2]
	\arrow[from=2-2, to=2-3]
	\arrow["q", from=1-3, to=2-3]
	\arrow[dashed, from=2-1, to=1-3]
\end{tikzcd}\]The existence of such a lift follows from Corollary \ref{cor:Step2}.

\item We complete the proof by extending the map $f''_{m}$ in Step
2 to a map $f_{m}:\cal M_{F\pr m}^{\t}\to\cal C^{\t}$ over $\cal O^{\t}$.
Let $\{\sigma'_{a}\}_{a\in A}$ be the collection of all $\pr{m-1}$-simplices
in $G'_{\pr 2}$ and $G'_{\pr 3}$ which have at least one associate.
Choose a well-ordering on $A$, and for each $a\in A$, let $F_{\leq a}$
denote the simplicial subset of $F\pr m$ generated by $F''\pr m$
and the associates of the simplices $\sigma'_{b}$ for $b\leq a$.
Define $F_{<a}$ similarly. We will choose the ordering on $A$ so
that the following condition is satisfied:
\begin{itemize}
\item [($\blacklozenge$)]Let $a\in A$ and let $\sigma$ be an associate
of $\sigma'_{a}$. Let $0\leq l\leq m$ be the (unique) integer such
that $d_{l}\sigma=\sigma'_{a}$. Then for each $i\in[m]\setminus\{l\}$,
the simplex $d_{i}\sigma$ belongs to $F_{<a}$.
\end{itemize}
The existence of such an ordering is nontrivial, but we will not get
into it presently. The construction of such an ordering can be found
at the very end of this step. Note that ($\blacklozenge$) implies
that:
\begin{itemize}
\item [($\blacklozenge\blacklozenge$)]For every $a\in A$, the simplex
$\sigma'_{a}$ does not belong to $F_{<a}$.
\end{itemize}
Indeed, suppose $\sigma'_{a}$ belongs to $F_{<a}$. Choose a minimal
element $b\in A$ such that $\sigma'_{a}$ belongs to $F_{<b}$. Then
$\sigma'_{a}$ factors through one of the following simplices:
\begin{enumerate}
\item Incomplete simplices.
\item Nondegenerate simplices of dimensions less than $m-1$.
\item $\pr{m-1}$-simplices in $G_{\pr 1}\cup G_{\pr 2}\cup G_{\pr 3}$.
\item $\pr{m-1}$-simplices in $G'_{\pr 2}$ without associates.
\item Associates of $\sigma'_{c}$ for some $c<b$.
\end{enumerate}
Since $\sigma'_{a}$ is complete and nondegenerate and has dimension
$m-1$, the cases (1), (2), (3), and (4) are immediately ruled out.
We show that the case (5) is impossible by reasoning by contradiction.
Suppose that there are an index $c<b$ and an associate $\sigma$
of $\sigma'_{c}$ through which $\sigma'_{a}$ factors. We have $\sigma'_{a}=d_{i}\sigma$
for some $0\leq i\leq m$ for dimensional reasons. Using ($\blacklozenge$),
we deduce that $\sigma'_{a}$ belongs to $F_{<c}$, contrary to the
minimality of $b$.

We now get back to the construction of $f_{m}$. We will construct
$f_{m}$ as an amalgamation of an $A$-sequence $\{f_{\leq a}:\cal M_{F_{\leq a}}^{\t}\to\cal C^{\t}\}_{a\in A}$
over $\cal O^{\t}$ which extends $f''_{m}$. The construction is
inductive. Suppose that $f_{\leq b}$ has been constructed for $b\leq a$,
so that they together determine a map $f_{<a}:\cal M_{F_{<a}}^{\t}\to\cal C^{\t}$.
We must extend $f_{<a}$ to $\cal M_{F_{\leq a}}^{\t}$. We consider
two cases, depending on whether $\sigma'_{a}$ belongs to $G'_{\pr 2}$
or to $G'_{\pr 3}$.

\begin{enumerate}[label=(Case \arabic*),wide =0.5\parindent=, listparindent=1.5em]

\item Suppose that $\sigma'_{a}$ belongs to $G'_{\pr 2}$. We shall
deploy an argument which is a variant of Proposition \ref{prop:heads_and_tails}.
Let $\{\tau_{\lambda}\}_{\lambda\in\Lambda}$ be the collection of
all nondegenerate simplices of $\cal M^{\t}$ the image of whose head
in $N\pr{\Fin_{\ast}}\times\Delta^{\{1,\dots,n\}}$ is a degneration
of $\sigma'_{a}$. Choose a well-ordering of $\Lambda$ so that $\dim\tau_{\lambda}$
is a non-decreasing function of $\lambda$. For each $\lambda\in\Lambda$,
we let $\cal N_{\leq\lambda}\subset\cal M^{\t}$ denote the simplicial
subset generated by $\cal M_{F_{<a}}^{\t}$ and the simplices $\tau:\Delta^{p}\to\cal M^{\t}$
for which there is an integer $0\leq p'<p$ such that $\tau\vert\Delta^{\{0,\dots,p'\}}$
factors through some $\tau_{\mu}$ for some $\mu\leq\lambda$, $\tau\vert\Delta^{\{p',p'+1\}}$
is inert, and $\tau\vert\Delta^{\{p'+1,\dots,p\}}$ factors through
$\cal M$. We define $\cal N_{<\lambda}$ similarly. 

We shall prove the following assertion later:
\begin{itemize}
\item [($*$)]$\cal M_{F_{\leq a}}^{\t}$ is the union of $\cal M_{F_{<a}}^{\t}$
and $\{\cal N_{\leq\lambda}\}_{\lambda\in\Lambda}$. 
\end{itemize}
Accepting ($\ast$) for now, we complete the proof as follows. It
will suffice construct a $\Lambda$-sequence $\{f^{\leq\lambda}:\cal N_{\leq\lambda}\to\cal C^{\t}\}_{\lambda\in\Lambda}$
of maps over $\cal O^{\t}$ which extends $f_{<a}$. The construction
is inductive. Suppose $f^{\leq\mu}$ has been constructed for $\mu<\lambda$,
and let $f^{<\lambda}:\cal N_{<\lambda}\to\cal C^{\t}$ denote the
map obtained by amalgamating the maps $\{f^{\leq\mu}\}_{\mu<\lambda}$.
Let $\pr{\inp k,n}$ be the final vertex of $\tau_{\lambda}$. Note
that $k\geq2$. There are $k$ inert maps  $\inp k\to\inp 1$, and
these maps and $p\tau_{\lambda}$ combine to determine a diagram $\Delta^{\dim\tau_{\lambda}}\star\inp k^{\circ}\to N\pr{\Fin_{\ast}}\times\Delta^{n}$.
Let $\cal X=\pr{\Delta^{\dim\tau_{\lambda}}\star\inp k^{\circ}}\times_{N\pr{\Fin_{\ast}}\times\Delta^{n}}\cal M^{\t}$
and let $\overline{\tau}_{\lambda}:\Delta^{\dim\tau_{\lambda}}\to\cal X$
denote the induced diagram. For each $1\leq i\leq k$, let $\cal X_{i}$
denote the fiber of $\cal X$ over $i\in\inp k^{\circ}$ (which is
isomorphic to $\cal M\times_{\Delta^{n}}\{n\}$), and set $\cal X^{0}=\bigcup_{1\leq i\leq k}\cal X_{i}$
and $\cal X_{\overline{\tau}_{\lambda}/}^{0}=\cal X^{0}\times_{\cal X}\cal X_{\overline{\tau}_{\lambda}/}$.
We now consider the following diagram: % https://q.uiver.app/?q=WzAsNixbMCwwLCJcXHBhcnRpYWxcXERlbHRhXntcXGRpbVxcb3ZlcmxpbmVcXHRhdV9cXGxhbWJkYX1cXHN0YXJcXG1hdGhjYWx7WH1eMF97XFxvdmVybGluZVxcdGF1X1xcbGFtYmRhL30iXSxbMCwxLCJcXERlbHRhXntcXGRpbVxcb3ZlcmxpbmVcXHRhdV9cXGxhbWJkYX1cXHN0YXJcXG1hdGhjYWx7WH1eMF97XFxvdmVybGluZVxcdGF1X1xcbGFtYmRhL30iXSxbMSwwLCJLXzAiXSxbMSwxLCJLIl0sWzIsMCwiXFxtYXRoY2Fse059X3s8XFxsYW1iZGF9Il0sWzIsMSwiXFxtYXRoY2Fse059X3tcXGxlcSBcXGxhbWJkYX0iXSxbMCwxXSxbMCwyXSxbMSwzXSxbMiwzXSxbMiw0XSxbMyw1XSxbNCw1XV0=
\[\begin{tikzcd}
	{\partial\Delta^{\dim\overline\tau_\lambda}\star\mathcal{X}^0_{\overline\tau_\lambda/}} & {K_0} & {\mathcal{N}_{<\lambda}} \\
	{\Delta^{\dim\overline\tau_\lambda}\star\mathcal{X}^0_{\overline\tau_\lambda/}} & K & {\mathcal{N}_{\leq \lambda}.}
	\arrow[from=1-1, to=2-1]
	\arrow[from=1-1, to=1-2]
	\arrow[from=2-1, to=2-2]
	\arrow[from=1-2, to=2-2]
	\arrow[from=1-2, to=1-3]
	\arrow[from=2-2, to=2-3]
	\arrow[from=1-3, to=2-3]
\end{tikzcd}\]Here $K\subset\cal X$ denotes the simplicial subset spanned by the
simplices whose tail factors through $\overline{\tau}_{\lambda}$,
and $K_{0}$ is its simplicial subset obtained by replacing $\overline{\tau}_{\lambda}$
by $\partial\overline{\tau}_{\lambda}$, where we regard $\cal X$
as a simplicial set over $\Delta^{1}$ by the map $\Delta^{\dim\tau_{\lambda}}\star\inp k^{\circ}\to\{0\}\star\{1\}=\Delta^{1}$.
The left hand square is homotopy cocartesian by Lemma \ref{lem:3.1.2.5}.
We shall prove later that:
\begin{itemize}
\item [($**$)]The horizontal arrows of the the right hand square are well-defined,
and the right hand square is cocartesian.
\end{itemize}
We are now reduced to solving the lifting problem % https://q.uiver.app/?q=WzAsNixbMCwwLCJcXHBhcnRpYWxcXERlbHRhXntcXGRpbVxcb3ZlcmxpbmVcXHRhdV9cXGxhbWJkYX1cXHN0YXJcXG1hdGhjYWx7WH1eMF97XFxvdmVybGluZVxcdGF1X1xcbGFtYmRhL30iXSxbMCwxLCJcXERlbHRhXntcXGRpbVxcb3ZlcmxpbmVcXHRhdV9cXGxhbWJkYX1cXHN0YXJcXG1hdGhjYWx7WH1eMF97XFxvdmVybGluZVxcdGF1X1xcbGFtYmRhL30iXSxbMSwwLCJcXG1hdGhjYWx7Tn1fezxcXGxhbWJkYX0iXSxbMSwxLCJcXG1hdGhjYWx7Tn1fe1xcbGVxIFxcbGFtYmRhfSJdLFsyLDAsIlxcbWF0aGNhbHtDfV5cXG90aW1lcyAiXSxbMiwxLCJcXG1hdGhjYWx7T31eXFxvdGltZXMgLiJdLFswLDFdLFsyLDRdLFszLDVdLFs0LDVdLFsxLDQsIiIsMSx7InN0eWxlIjp7ImJvZHkiOnsibmFtZSI6ImRhc2hlZCJ9fX1dLFswLDJdLFsxLDNdXQ==
\[\begin{tikzcd}
	{\partial\Delta^{\dim\overline\tau_\lambda}\star\mathcal{X}^0_{\overline\tau_\lambda/}} & {\mathcal{N}_{<\lambda}} & {\mathcal{C}^\otimes } \\
	{\Delta^{\dim\overline\tau_\lambda}\star\mathcal{X}^0_{\overline\tau_\lambda/}} & {\mathcal{N}_{\leq \lambda}} & {\mathcal{O}^\otimes .}
	\arrow[from=1-1, to=2-1]
	\arrow[from=1-2, to=1-3]
	\arrow[from=2-2, to=2-3]
	\arrow[from=1-3, to=2-3]
	\arrow[dashed, from=2-1, to=1-3]
	\arrow[from=1-1, to=1-2]
	\arrow[from=2-1, to=2-2]
\end{tikzcd}\]Now the $\infty$-category $\cal X_{\overline{\tau}_{\lambda}/}^{0}$
is the disjoint union of the $\infty$-categories $\pr{\cal X_{i}}_{\overline{\tau}_{\lambda}/}=\cal X_{i}\times_{\cal X}\cal X_{\overline{\tau}_{\lambda}/}$.
Each $\infty$-category $\pr{\cal X_{i}}_{\overline{\tau}_{\lambda}/}$
has an initial object, given by a cone $\phi_{i}:\pr{\Delta^{\dim\overline{\tau}_{\lambda}}}^{\rcone}\to\cal X$
which maps the last edge to an inert morphism over $\pr{\rho^{i},\id}:\pr{\inp k,n}\to\pr{\inp 1,n}$.
Set $S=\coprod_{i}\{\phi_{i}\}$. The inclusion $S\subset\cal X_{\overline{\tau}_{\lambda}/}^{0}$
is initial, so we are reduced to solving the lifting problem % https://q.uiver.app/?q=WzAsOCxbMSwwLCJcXHBhcnRpYWxcXERlbHRhXntcXGRpbVxcb3ZlcmxpbmVcXHRhdV9cXGxhbWJkYX1cXHN0YXJcXG1hdGhjYWx7WH1eMF97XFxvdmVybGluZVxcdGF1X1xcbGFtYmRhL30iXSxbMSwxLCJcXERlbHRhXntcXGRpbVxcb3ZlcmxpbmVcXHRhdV9cXGxhbWJkYX1cXHN0YXJcXG1hdGhjYWx7WH1eMF97XFxvdmVybGluZVxcdGF1X1xcbGFtYmRhL30iXSxbMiwwLCJcXG1hdGhjYWx7Tn1fezxcXGxhbWJkYX0iXSxbMiwxLCJcXG1hdGhjYWx7Tn1fe1xcbGVxIFxcbGFtYmRhfSJdLFszLDAsIlxcbWF0aGNhbHtDfV5cXG90aW1lcyAiXSxbMywxLCJcXG1hdGhjYWx7T31eXFxvdGltZXMgLiJdLFswLDAsIlxccGFydGlhbFxcRGVsdGFee1xcZGltXFxvdmVybGluZVxcdGF1X1xcbGFtYmRhfVxcc3RhciBTIl0sWzAsMSwiXFxEZWx0YV57XFxkaW1cXG92ZXJsaW5lXFx0YXVfXFxsYW1iZGF9XFxzdGFyIFMiXSxbMiw0XSxbMyw1XSxbNCw1XSxbMCwyXSxbMSwzXSxbNiwwXSxbNywxXSxbNyw0LCIiLDEseyJzdHlsZSI6eyJib2R5Ijp7Im5hbWUiOiJkYXNoZWQifX19XSxbNiw3XV0=
\[\begin{tikzcd}
	{\partial\Delta^{\dim\overline\tau_\lambda}\star S} & {\partial\Delta^{\dim\overline\tau_\lambda}\star\mathcal{X}^0_{\overline\tau_\lambda/}} & {\mathcal{N}_{<\lambda}} & {\mathcal{C}^\otimes } \\
	{\Delta^{\dim\overline\tau_\lambda}\star S} & {\Delta^{\dim\overline\tau_\lambda}\star\mathcal{X}^0_{\overline\tau_\lambda/}} & {\mathcal{N}_{\leq \lambda}} & {\mathcal{O}^\otimes .}
	\arrow[from=1-3, to=1-4]
	\arrow[from=2-3, to=2-4]
	\arrow[from=1-4, to=2-4]
	\arrow[from=1-2, to=1-3]
	\arrow[from=2-2, to=2-3]
	\arrow[from=1-1, to=1-2]
	\arrow[from=2-1, to=2-2]
	\arrow[dashed, from=2-1, to=1-4]
	\arrow[from=1-1, to=2-1]
\end{tikzcd}\] If the dimension of $\overline{\tau}_{\lambda}$ is positive, then
the claim is immediate since the restriction of the top horizontal
arrow to $\{\dim\overline{\tau}_{\lambda}\}\star S$ is a $q$-limit
cone. If $\overline{\tau}_{\lambda}$ is zero-dimensional (in which
case $m=n=1$), let $C_{i}$ denote the image of $\phi_{i}\in S$
under the top horizontal map. The bottom horizontal arrow classifies
a diagram $X\to q\pr{C_{i}}$ of inert maps $\{\alpha_{i}:X\to q\pr{C_{i}}\}_{1\leq i\leq k}$
lying over $\{\rho^{i}:\inp k\to\inp 1\}_{1\leq i\leq k}$, and we
wish to lift this to a diagram $S^{\lcone}\to\cal C^{\t}$ in $\cal C^{\t}$
which maps each $\phi_{i}\in S$ to the object $C_{i}$. Since $q$
is a fibration of $\infty$-operads, we can in fact find such a lift
consisting of inert morphisms. Note that with such a choice of lift,
condition (ii) is satisfied.

We now move on to the verification of ($\ast$) and ($\ast\ast$).
We begin with ($\ast$). It is clear that $\cal N_{\leq\lambda}$
and $\cal M_{F_{<a}}^{\t}$ are contained in $\cal M_{F_{\leq a}}^{\t}$.
For the reverse implication, let $x$ be an arbitrary simplex of $\cal M_{F_{\leq a}}^{\t}$.
We must show that $x$ belongs to either $\cal M_{F_{<a}}^{\t}$ or
one of the $\cal N_{\leq\lambda}$'s. If $x$ belongs to $\cal M_{F_{<a}}^{\t}$,
we are done. So assume not. Let $r$ denote the dimension of $x$.
Since $x$ does not belong to $\cal M_{F_{<a}}^{\t}$, its head is
nonempty. Find an integer $0\leq r'\leq r$ such that $x\vert\Delta^{\{r',\dots,r\}}$
is the head of $x$. Since $x$ does not belong to $\cal M_{F_{<a}}^{\t}$,
the simplex $px\vert\Delta^{\{r',\dots,r\}}$ factors through an associate
$\sigma$ of $\sigma'_{a}$. Let $\Delta^{r}\xrightarrow{u}\Delta^{m}\xrightarrow{\sigma}N\pr{\Fin_{\ast}}\times\Delta^{\{1,\dots,n\}}$
be such a factorization. If the image of $u$ does not contain some
integer $i\in\{0,\dots,m-1\}$, then $px\vert\Delta^{\{r',\dots,r\}}$
factors through $d_{i}\sigma$ and hnece through $F_{<a}$ by ($\blacklozenge$),
a contradiction. So the image of $u$ contains every integer in $\{0,\dots,m-1\}$.
Let $0\leq r''<r$ be the largest integer such that $u\pr{r''}<m$.
There are now several cases to consider:
\begin{itemize}
\item Suppose that $u$ is surjective on vertices and that the map $x\pr{r''}\to x\pr{r''+1}$
is inert. We write $x\vert\Delta^{\{0,\dots,r''\}}=s^{*}y$, where
$s:[r']\to[k]$ is a surjection and $y$ is nondegenerate. We claim
that $y$ is one of the $\tau_{\lambda}$'s, so that $x$ belongs
to $\cal N_{\leq\lambda}$. Let $k'=s\pr{r'}$. Then $y\vert\Delta^{\{k',\dots,k\}}$
is the head of $y$. We write $py\vert\Delta^{\{k',\dots,k\}}=s^{\p\ast}z$,
where $z$ is nondegenerate and $s'$ is a surjection. Then we obtain
the following commutative diagram: % https://q.uiver.app/?q=WzAsOCxbMSwwLCJcXERlbHRhXntcXHtyJyxcXGRvdHMscicnXFx9fSJdLFsyLDAsIlxcRGVsdGFee20tMX0iXSxbMSwxLCJcXG1hdGhjYWx7TX1eXFxvdGltZXMgIl0sWzIsMSwiTihcXG1hdGhzZntGaW59X1xcYXN0KVxcdGltZXMgXFxEZWx0YV5uLiJdLFswLDAsIlxcRGVsdGFee1xcezAsXFxkb3RzLHInJ1xcfX0iXSxbMSwyLCJcXERlbHRhXmwiXSxbMCwyLCJcXERlbHRhXntcXHtrJyxcXGRvdHNtLGtcXH19Il0sWzAsMSwiXFxEZWx0YV57a30iXSxbMCwxLCJ1XFx2ZXJ0XFxEZWx0YV57XFx7cicsXFxkb3RzLHInJ1xcfX0iLDAseyJzdHlsZSI6eyJoZWFkIjp7Im5hbWUiOiJlcGkifX19XSxbMSwzLCJcXHNpZ21hX2EnIl0sWzIsMywicCIsMl0sWzQsNywicyIsMix7InN0eWxlIjp7ImhlYWQiOnsibmFtZSI6ImVwaSJ9fX1dLFs3LDIsInkiLDJdLFswLDQsIiIsMCx7InN0eWxlIjp7InRhaWwiOnsibmFtZSI6Imhvb2siLCJzaWRlIjoiYm90dG9tIn19fV0sWzYsNywiIiwyLHsic3R5bGUiOnsidGFpbCI6eyJuYW1lIjoiaG9vayIsInNpZGUiOiJib3R0b20ifX19XSxbNiw1LCJzJyIsMix7InN0eWxlIjp7ImhlYWQiOnsibmFtZSI6ImVwaSJ9fX1dLFs2LDJdLFs1LDMsInoiLDJdLFs0LDIsInhcXHZlcnRcXERlbHRhXntcXHswLFxcZG90cyxyJydcXH19IiwxXV0=
\[\begin{tikzcd}
	{\Delta^{\{0,\dots,r''\}}} & {\Delta^{\{r',\dots,r''\}}} & {\Delta^{m-1}} \\
	{\Delta^{k}} & {\mathcal{M}^\otimes } & {N(\mathsf{Fin}_\ast)\times \Delta^n.} \\
	{\Delta^{\{k',\dotsm,k\}}} & {\Delta^l}
	\arrow["{u\vert\Delta^{\{r',\dots,r''\}}}", two heads, from=1-2, to=1-3]
	\arrow["{\sigma_a'}", from=1-3, to=2-3]
	\arrow["p"', from=2-2, to=2-3]
	\arrow["s"', two heads, from=1-1, to=2-1]
	\arrow["y"', from=2-1, to=2-2]
	\arrow[hook', from=1-2, to=1-1]
	\arrow[hook', from=3-1, to=2-1]
	\arrow["{s'}"', two heads, from=3-1, to=3-2]
	\arrow[from=3-1, to=2-2]
	\arrow["z"', from=3-2, to=2-3]
	\arrow["{x\vert\Delta^{\{0,\dots,r''\}}}"{description}, from=1-1, to=2-2]
\end{tikzcd}\]Applying Eilenberg-Zilber's lemma to $px\vert\Delta^{\{r',\dots,r''\}}$,
we deduce that $z=\sigma'_{a}$. Thus $y$ is one of the simplices
in $\{\tau_{\lambda}\}_{\lambda\in\Lambda}$, as required.
\item Suppose that $u$ is surjective on vertices and that the map $\theta:u\pr{r''}\to u\pr{r''+1}$
is not inert. By factoring the map $\theta$ into an inert map followed
by an active map, we can find an $\pr{r+1}$-simplex $y$ of $\cal M_{F_{\leq a}}^{\t}$
such that $d_{r''+1}y=x$ and $y\vert\Delta^{\{r'',r''+1,r''+2\}}$
is the chosen factorization of $\theta$. By the previous point, the
simplex $y$ belongs to $\cal N_{\leq\lambda}$ for some $\lambda$,
and hence so must $x$.
\item Suppose $u$ is not surjective on vertices. Find an inert map $\theta:u\pr r\to X$
over the last edge of $\sigma$, and let $y$ be an $\pr{r+1}$-simplex
$y$ of $\cal M_{F_{\leq a}}^{\t}$ such that $d_{r+1}y=x$ and $y\vert\Delta^{\{r,r+1\}}$
is equal to $\theta$. By the first point, the simplex $y$ belongs
to $\cal N_{\leq\lambda}$ for some $\lambda$, and hence so must
$x$.
\end{itemize}
This completes the proof of ($\ast$).

Next we prove ($\ast\ast$). First we show that the maps $K\to\cal N_{\leq\lambda}$
and $K_{0}\to\cal N_{<\lambda}$ are well-defined. A typical simplex
in the image of the map $K\to\cal M^{\t}$ has the form $x:\Delta^{r}\to\cal M^{\t}$,
where there is an integer $-1\leq r'\leq r$ such that $x\vert\Delta^{\{0,\dots,r'\}}$
factors through $\tau_{\lambda}$ and, if $r'<r$, then $p\pr{x\vert\Delta^{\{r',r'+1\}}}$
is inert and and $p\pr{x\vert\Delta^{\{r'+1,\dots,r\}}}$ is the constant
map at $\pr{\inp 1,n}$. (When $r'=-1$, the symbol $\Delta^{\{0,\dots,r'\}}$
denotes the empty simplicial set.) By choosing an inert map $x\pr{r'}\to X$
with $X\in\cal M$ if $r'<r$, or by factoring the map $x\pr{r'}\to x\pr{r'+1}$
into an inert map followed by an active map, we can find a simplex
$y$ belonging to $\cal N_{\leq\lambda}$ and satisfying $d_{r'+1}y=x$.
Hence $x$ belongs to $\cal N_{\leq\lambda}$, showing that the map
$K\to\cal N_{\leq\lambda}$ is well-defined. Next, for the map $K_{0}\to\cal N_{<\lambda}$,
assume further $x\vert\Delta^{\{0,\dots,r'\}}$ factors through the
boundary of $\tau_{\lambda}$. We must show that $x$ belongs to $\cal N_{<\lambda}$.
We have two cases to consider:
\begin{itemize}
\item Suppose that the head of the simplex $p\pr{x\vert\Delta^{\{0,\dots,r'\}}}$
factors through $\partial\sigma'_{a}$. Then the head of $p\pr x$
factors through $d_{i}\sigma$ for some $0\leq i<m$ for some associate
$\sigma$ of $\sigma'_{a}$, so $x$ belongs to $\cal M_{F_{<a}}^{\t}$
by ($\blacklozenge$). 
\item Suppose that the head of the simplex $p\pr{x\vert\Delta^{\{0,\dots,r'\}}}$
is a degeneration of $\sigma'_{a}$. Write $x\vert\Delta^{\{0,\dots,r'\}}=s^{*}y$,
where $y$ is nondegenerate and $s$ is a surjection. Eilenberg-Zilber's
lemma implies that the simplex $p\pr y$ is again a degeneration of
$\sigma'_{a}$, so $y=\tau_{\mu}$ for some $\mu\in\Lambda$. Since
$x$ factors through the boundary of $\tau_{\lambda}$, the dimension
of $y$ must be smaller than that of $\tau_{\lambda}$. Thus $\mu<\lambda$.
The above argument (that the map $K\to\cal N_{\leq\lambda}$ is well-defined)
shows that $y$ belongs to $\cal N_{\leq\mu}\subset\cal N_{<\lambda}$,
so $x$ belongs to $\cal N_{<\lambda}$.
\end{itemize}
This completes the verification of the first half of ($\ast\ast$).
We next proceed to the latter half: We show that the square % https://q.uiver.app/?q=WzAsNCxbMCwwLCJLXzAiXSxbMCwxLCJLIl0sWzEsMCwiXFxtYXRoY2Fse059X3s8XFxsYW1iZGF9Il0sWzEsMSwiXFxtYXRoY2Fse059X3tcXGxlcSBcXGxhbWJkYX0iXSxbMCwxXSxbMCwyXSxbMSwzXSxbMiwzXV0=
\[\begin{tikzcd}
	{K_0} & {\mathcal{N}_{<\lambda}} \\
	K & {\mathcal{N}_{\leq \lambda}}
	\arrow[from=1-1, to=2-1]
	\arrow[from=1-1, to=1-2]
	\arrow[from=2-1, to=2-2]
	\arrow[from=1-2, to=2-2]
\end{tikzcd}\]is cocartesian. Clearly $\cal N_{\leq\lambda}$ is the union of the
images of $K$ and $\cal N_{<\lambda}$. So it suffices to show that
if a simplex $z$ of $K$ is mapped into $\cal N_{<\lambda}$, then
$z$ belongs to $K_{0}$. Taking the contrapositive, we will show
that if $z$ does not belong to $K_{0}$, then its image in $\cal N_{\leq\lambda}$
does not belong to $\cal N_{<\lambda}$. Let $x:\Delta^{r}\to\cal N_{\leq\lambda}$
be the image of $x$. By the definition of the map $K\to\cal N_{\leq\lambda}$
and by the hypothesis that $z$ does not belong to $K_{0}$, there
is an integer $0\leq r'\leq r$ such that $x\vert\Delta^{\{0,\dots,r'\}}$
is a degeneration of $\tau_{\lambda}$ and, if $r'<r$, then $p\pr{x\vert\Delta^{\{r',r'+1\}}}$
is inert and and $p\pr{x\vert\Delta^{\{r'+1,\dots,r\}}}$ is the constant
map at $\pr{\inp 1,n}$. The head of $p\pr x$ is thus a degeneration
of either $\sigma'_{a}$ or its associate, and so it does not belong
to $F_{<a}$ by ($\blacklozenge\blacklozenge$). Therefore, $x$ does
not belong to $\cal M_{F_{<a}}^{\t}$. So should $x$ belong to $\cal N_{<\lambda}$,
then $x\vert\Delta^{\{0,\dots,r'\}}$ must factor through $\tau_{\mu}$
for some $\mu<\lambda$. This is impossible because $x\vert\Delta^{\{0,\dots,r'\}}$
is a degeneration of $\tau_{\lambda}$. Hence $x$ does not belong
to $\cal N_{<\lambda}$.

\item Suppose that $\sigma'_{a}$ belongs to $G'_{\pr 3}$. We will
show that the inclusion $\cal M_{<a}^{\t}\hookrightarrow\cal M_{\leq a}^{\t}$
is a weak categorical equivalence. The desired extension of $f_{<a}$
can then be found because the map $q$ is a categorical fibration.

Let $\sigma$ be the unique associate of $\sigma'_{a}$, which we
depict as 
\[
\pr{\inp{k_{0}},e_{0}}\xrightarrow{\alpha_{\sigma}\pr 1}\cdots\xrightarrow{\alpha_{\sigma}\pr m}\pr{\inp{k_{m}},e_{m}}.
\]
Choose integers $0<j<k\leq m$ such that $\alpha_{\sigma}\pr i$ is
strongly inert for $i=j$, active for $j<i\leq k$, and strongly inert
for $i>k$. Set $Y=\pr{N\pr{\Fin_{\ast}}\times\Delta^{n}}_{/\sigma}\times_{\Delta^{n}}\{0\}$.
Using ($\blacklozenge$), we may consider the following commutative
diagram: % https://q.uiver.app/?q=WzAsNCxbMCwwLCJZXFxzdGFyIFxcTGFtYmRhIF5tX2oiXSxbMCwxLCJZXFxzdGFyIFxcRGVsdGEgXm0iXSxbMSwwLCJcXG92ZXJsaW5le0ZfezxhfX0iXSxbMSwxLCJcXG92ZXJsaW5le0Zfe1xcbGVxIGF9fSJdLFswLDFdLFswLDJdLFsxLDNdLFsyLDNdXQ==
\[\begin{tikzcd}
	{Y\star \Lambda ^m_j} & {\overline{F_{<a}}} \\
	{Y\star \Delta ^m} & {\overline{F_{\leq a}}.}
	\arrow[from=1-1, to=2-1]
	\arrow[from=1-1, to=1-2]
	\arrow[from=2-1, to=2-2]
	\arrow[from=1-2, to=2-2]
\end{tikzcd}\]Using ($\blacklozenge\blacklozenge$), we deduce that this square
is cocartesian. Therefore, it suffices to show that the inclusion
\[
\pr{Y\star\Lambda_{j}^{m}}\times_{N\pr{\Fin_{\ast}}\times\Delta^{n}}\cal M^{\t}\to\pr{Y\star\Delta^{m}}\times_{N\pr{\Fin_{\ast}}\times\Delta^{n}}\cal M^{\t}
\]
is a weak categorical equivalence. We will prove more generally that
for any morphism $Y'\to Y$ of simplicial sets, the map 
\[
\eta_{Y'}:\pr{Y'\star\Lambda_{j}^{m}}\times_{N\pr{\Fin_{\ast}}\times\Delta^{n}}\cal M^{\t}\to\pr{Y'\star\Delta^{m}}\times_{N\pr{\Fin_{\ast}}\times\Delta^{n}}\cal M^{\t}
\]
is a weak categorical equivalence. The assignment $Y'\mapsto\eta_{Y'}$
defines a functor from $\SS/Y'$ to the arrow category $\SS^{[1]}$
which commutes with filtered colimits. Since weak categorical equivalences
are stable under filtered colimits, we may assume that $Y'$ is a
finite simplicial set. If $Y'$ is empty, the claim follows from \cite[Lemma 2.4.4.6]{HA}.
For the inductive step, we can find a pushout diagram % https://q.uiver.app/?q=WzAsNCxbMCwwLCJcXHBhcnRpYWxcXERlbHRhXnAiXSxbMSwwLCJYIl0sWzEsMSwiWSciXSxbMCwxLCJcXERlbHRhXnAiXSxbMCwxXSxbMSwyXSxbMCwzXSxbMywyXV0=
\[\begin{tikzcd}
	{\partial\Delta^p} & X \\
	{\Delta^p} & {Y'}
	\arrow[from=1-1, to=1-2]
	\arrow[from=1-2, to=2-2]
	\arrow[from=1-1, to=2-1]
	\arrow[from=2-1, to=2-2]
\end{tikzcd}\]in $\SS/Y$, such that the claim holds for $X$. The map $\eta_{Y'}$
factors as
\begin{align*}
\pr{Y'\star\Lambda_{j}^{m}}\times_{N\pr{\Fin_{\ast}}\times\Delta^{n}}\cal M^{\t} & \xrightarrow{\phi}\pr{Y'\star\Lambda_{j}^{m}\cup X\star\Delta^{m}}\times_{N\pr{\Fin_{\ast}}\times\Delta^{n}}\cal M^{\t}\\
 & \xrightarrow{\psi}\pr{Y'\star\Delta^{m}}\times_{N\pr{\Fin_{\ast}}\times\Delta^{n}}\cal M^{\t}.
\end{align*}
The map $\phi$ is a pushout of $\eta_{X}$, and hence is a trivial
cofibration. The map $\psi$ is a pushout of the inclusion
\[
\psi':\pr{\Delta^{p}\star\Lambda_{j}^{m}\cup\partial\Delta^{p}\star\Delta^{m}}\times_{N\pr{\Fin_{\ast}}\times\Delta^{n}}\cal M^{\t}\to\pr{\Delta^{p}\star\Delta^{m}}\times_{N\pr{\Fin_{\ast}}\times\Delta^{n}}\cal M^{\t}.
\]
Using the isomorphism $\Delta^{p}\star\Lambda_{j}^{m}\cup\partial\Delta^{p}\star\Delta^{m}\cong\Lambda_{p+1+j}^{m+p+1}$
and Lemma \cite[Lemma 2.4.4.6]{HA}, we deduce that $\psi'$ is a
weak categorical equivalence. Hence $\psi$ is a weak categorical
equivalence, completing the treatment of Case 2.

\end{enumerate}

It remains to construct a well-ordering on $A$ which satisfies ($\blacklozenge$).
For each $a\in A$, define integers $u_{\mathrm{neut}}\pr a,u_{\act}\pr a,u_{\mathrm{oc}}\pr a,u_{\mathrm{as}}\pr a$
as follows: Let $\sigma$ be an associate of $\sigma'_{a}$, and set
$\alpha_{\sigma}\pr i=\sigma\vert\Delta^{\{i-1,i\}}$ for $1\leq i\leq m$.
Then:
\begin{itemize}
\item $u_{\mathrm{neut}}\pr a$ is the number of integers $1\leq i\leq m$
such that $\alpha_{\sigma}\pr i$ is neutral.
\item $u_{\act}\pr a$ is the number of integers $1\leq i\leq m$ such that
$\alpha_{\sigma}\pr i$ is active.
\item $u_{\mathrm{oc}}\pr a$ is set equal to $0$ if $\sigma$ is closed,
and is set equal to $1$ if $\sigma$ is open.
\item $u_{\mathrm{as}}\pr a$ is the number of pairs of integers $1\leq i<j\leq m$
such that $\alpha_{\sigma}\pr i$ is active and $\alpha_{\sigma}\pr j$
is strictly inert.
\end{itemize}
We choose a well-ordering on $A$ so that the function 
\[
A\ni a\mapsto\pr{u_{\mathrm{neut}}\pr a,u_{\act}\pr a,u_{\mathrm{oc}}\pr a,u_{\mathrm{as}}\pr a}\in\bb Z^{4}
\]
is non-decreasing, where $\bb Z^{4}$ is equipped with the lexicographic
ordering. (Thus $\pr{n_{1},n_{2},n_{3},n_{4}}<\pr{n'_{1},n'_{2},n'_{3},n'_{4}}$
if and only if $n_{i}<n'_{i}$, where $i$ is the minimal integer
for which $n_{i}\neq n'_{i}$.) We will show that this ordering does
the job.

Let $a,\sigma,i$ be as in ($\blacklozenge$). We wish to show that
$d_{i}\sigma$ belongs to $F_{<a}$. If $d_{i}\sigma$ is incomplete
or degenerate, then it belongs to $F\pr{m-1}$ and we are done. So
assume that $d_{i}\sigma$ is complete and nondegenerate. Then $d_{i}\sigma$
belongs to (exactly) one of the sets $G_{\pr 1},G_{\pr 2},G'_{\pr 2},G_{\pr 3},G'_{\pr 3}$.
If $d_{i}\sigma$ belongs to $G_{\pr 1}\cup G_{\pr 2}\cup G_{\pr 3}$,
or if it belongs to $G'_{\pr 2}$ and has no associates, then it belongs
to $F''\pr m$ and we are done. So we will assume that $d_{i}\sigma$
belongs to $G'_{\pr 2}\cup G'_{\pr 3}$ and has an associate, so that
$d_{i}\sigma=\sigma'_{b}$ for some $b\in A$. We wish to show that
$b<a$.

We will make use of the following notations. Let $\alpha_{\sigma}\pr i=\sigma\vert\Delta^{\{i-1,i\}}$.
Let $0\leq k\leq m$ be the minimal integer such that $\alpha_{\sigma}\pr i$
is strongly inert for every $i>k$, and let $0\leq j\leq k$ be the
minimal integer such that $\alpha_{\sigma}\pr i$ is active for every
$j<i\leq k$. 

Suppose first that $\sigma'_{a}\in G'_{\pr 2}$. Our assumption on
$d_{i}\sigma$ implies that $i=k$ and that the composite $\alpha_{\sigma}\pr{k+1}\circ\alpha_{\sigma}\pr k$
is neutral. It follows that the associate $\tau$ of $d_{i}\sigma$
satisfies $\alpha_{\tau}\pr s=\alpha_{\sigma}\pr s$ for $s\neq k,k+1$,
$\alpha_{\tau}\pr i$ is strictly inert, and $\alpha_{\tau}\pr{k+1}$
is active. Thus $u_{\mathrm{neut}}\pr a=u_{\mathrm{neut}}\pr b$,
$u_{\act}\pr a=u_{\act}\pr b$, $u_{\mathrm{oc}}\pr a=u_{\mathrm{oc}}\pr b$,
and $u_{\mathrm{as}}\pr a>u_{\mathrm{as}}\pr b$. Hence $a>b$, as
required.

Suppose next that $\sigma'_{a}\in G'_{\pr 3}$ and that $d_{i}\sigma\in G'_{\pr 2}$. 
\begin{itemize}
\item If $u_{\mathrm{neut}}\pr a>0$, we are done, since $u_{\mathrm{neut}}\pr b=0$.
\item If $u_{\mathrm{neut}}\pr a=0$, then each $\alpha_{\sigma}\pr i$
is either active or strongly inert. It follows that $u_{\act}\pr a$
is not less than the number of active morphisms in $d_{i}\sigma$,
which is equal to $u_{\act}\pr b$. So $u_{\act}\pr a\geq u_{\act}\pr b$.
If $u_{\act}\pr a>u_{\act}\pr b$, we are done. 
\item If $u_{\mathrm{neut}}\pr a=0$ and $u_{\act}\pr a=u_{\act}\pr b$,
then $\sigma$ is open. Indeed, if $\sigma$ were closed, then $i=m$
since $d_{i}\sigma$ is open. But then $d_{i}\sigma$ must contain
a subsequence consisting of a strictly inert morphism followed by
an active morphism. This is impossible because $d_{i}\sigma$ belongs
to $G'_{\pr 2}$. Hence $u_{\mathrm{oc}}\pr a>u_{\mathrm{oc}}\pr b$
and we are done.
\end{itemize}
Finally, suppose that $\sigma'_{a}\in G'_{\pr 3}$ and that $d_{i}\sigma\in G'_{\pr 3}$.
Note that our assumption on $d_{i}\sigma$ forces $i=j-1$ or $i=k$.
\begin{itemize}
\item The number of neutral morphisms in $d_{i}\sigma$ is at most $u_{\mathrm{neut}}\pr a+1$,
so 
\begin{equation}
u_{\mathrm{neut}}\pr a\geq\#\{\text{neutral morphisms in }d_{i}\sigma\}-1=u_{\mathrm{neut}}\pr b.\label{eq:1}
\end{equation}
If $u_{\mathrm{neut}}\pr a>u_{\mathrm{neut}}\pr b$, we are done. 
\item Suppose $u_{\mathrm{neut}}\pr a=u_{\mathrm{neut}}\pr b$, so that
the equality holds in (\ref{eq:1}). Then $0<i<m$, the composite
$\alpha_{\sigma}\pr{i+1}\circ\alpha_{\sigma}\pr i$ is neutral, and
$\alpha_{\sigma}\pr i$ is active. ($\alpha_{\sigma}\pr i$ is necessarily
strictly inert since $i=j-1$ or $i=k$.) It follows that the associate
$\tau$ of $d_{i}\sigma$ satisfies $\alpha_{\tau}\pr s=\alpha_{\sigma}\pr s$
for $s\neq i,i+1$, $\alpha_{\tau}\pr i$ is strictly inert, and $\alpha_{\tau}\pr{i+1}$
is active. Thus $u_{\mathrm{neut}}\pr a=u_{\mathrm{neut}}\pr b$,
$u_{\act}\pr a=u_{\act}\pr b$, $u_{\mathrm{oc}}\pr a=u_{\mathrm{oc}}\pr b$,
and $u_{\mathrm{as}}\pr a>u_{\mathrm{as}}\pr b$. Hence $a>b$, as
desired.
\end{itemize}
This completes the proof of ($\blacklozenge$) and hence the proof
of the theorem.

\end{enumerate}
\end{proof}
%% LyX 2.3.6.2 created this file.  For more info, see http://www.lyx.org/.
%% Do not edit unless you really know what you are doing.
\documentclass[a4paper]{amsart}
\usepackage[T1]{fontenc}
\usepackage{amstext}
\usepackage{amsthm}
\usepackage{amssymb}
\usepackage{stackrel}
\usepackage[unicode=true,pdfusetitle,
 bookmarks=true,bookmarksnumbered=false,bookmarksopen=false,
 breaklinks=false,pdfborder={0 0 0},pdfborderstyle={},backref=false,colorlinks=false]
 {hyperref}

\makeatletter

%%%%%%%%%%%%%%%%%%%%%%%%%%%%%% LyX specific LaTeX commands.
\special{papersize=\the\paperwidth,\the\paperheight}


%%%%%%%%%%%%%%%%%%%%%%%%%%%%%% Textclass specific LaTeX commands.
\numberwithin{equation}{section}
\numberwithin{figure}{section}
\theoremstyle{plain}
\newtheorem{thm}{\protect\theoremname}[section]
\theoremstyle{plain}
\newtheorem{prop}[thm]{\protect\propositionname}
\theoremstyle{remark}
\newtheorem{rem}[thm]{\protect\remarkname}
\ifx\proof\undefined
\newenvironment{proof}[1][\protect\proofname]{\par
	\normalfont\topsep6\p@\@plus6\p@\relax
	\trivlist
	\itemindent\parindent
	\item[\hskip\labelsep\scshape #1]\ignorespaces
}{%
	\endtrivlist\@endpefalse
}
\providecommand{\proofname}{Proof}
\fi
\theoremstyle{plain}
\newtheorem{cor}[thm]{\protect\corollaryname}
\theoremstyle{plain}
\newtheorem{lem}[thm]{\protect\lemmaname}
\theoremstyle{definition}
\newtheorem{defn}[thm]{\protect\definitionname}

%%%%%%%%%%%%%%%%%%%%%%%%%%%%%% User specified LaTeX commands.
\usepackage{tikz-cd}
\usepackage{mathtools}
\usepackage{adjustbox}
\usepackage{enumitem}
\tikzcdset{scale cd/.style={every label/.append style={scale=#1},
    cells={nodes={scale=#1}}}}
\usepackage{eucal}

\makeatother


\providecommand{\corollaryname}{Corollary}
\providecommand{\definitionname}{Definition}
\providecommand{\lemmaname}{Lemma}
\providecommand{\propositionname}{Proposition}
\providecommand{\remarkname}{Remark}
\providecommand{\theoremname}{Theorem}

\begin{document}
\global\long\def\sf#1{\mathsf{#1}}%

\global\long\def\scr#1{\mathscr{{#1}}}%

\global\long\def\cal#1{\mathcal{#1}}%

\global\long\def\bb#1{\mathbb{#1}}%

\global\long\def\bf#1{\mathbf{#1}}%

\global\long\def\frak#1{\mathfrak{#1}}%

\global\long\def\fr#1{\mathfrak{#1}}%

\global\long\def\u#1{\underline{#1}}%

\global\long\def\tild#1{\widetilde{#1}}%

\global\long\def\mrm#1{\mathrm{#1}}%

\global\long\def\pr#1{\left(#1\right)}%

\global\long\def\abs#1{\left|#1\right|}%

\global\long\def\inp#1{\left\langle #1\right\rangle }%

\global\long\def\br#1{\left\{  #1\right\}  }%

\global\long\def\norm#1{\left\Vert #1\right\Vert }%

\global\long\def\hat#1{\widehat{#1}}%

\global\long\def\opn#1{\operatorname{#1}}%

\global\long\def\bigmid{\,\middle|\,}%

\global\long\def\Top{\sf{Top}}%

\global\long\def\Set{\sf{Set}}%

\global\long\def\SS{\sf{sSet}}%

\global\long\def\Kan{\sf{Kan}}%

\global\long\def\Cat{\mathcal{C}\sf{at}}%

\global\long\def\imfld{\cal M\mathsf{fld}}%

\global\long\def\ids{\cal D\sf{isk}}%

\global\long\def\ich{\cal C\sf h}%

\global\long\def\SW{\mathcal{SW}}%

\global\long\def\SHC{\mathcal{SHC}}%

\global\long\def\B{\sf B}%

\global\long\def\Spaces{\sf{Spaces}}%

\global\long\def\Mod{\sf{Mod}}%

\global\long\def\Nec{\sf{Nec}}%

\global\long\def\Fin{\sf{Fin}}%

\global\long\def\Ch{\sf{Ch}}%

\global\long\def\Ab{\sf{Ab}}%

\global\long\def\SA{\sf{sAb}}%

\global\long\def\P{\mathsf{POp}}%

\global\long\def\Op{\mathcal{O}\mathsf{p}}%

\global\long\def\Opg{\mathcal{O}\mathsf{p}_{\infty}^{\mathrm{gn}}}%

\global\long\def\Tup{\mathsf{Tup}}%

\global\long\def\Del{\mathbf{\Delta}}%

\global\long\def\id{\operatorname{id}}%

\global\long\def\Aut{\operatorname{Aut}}%

\global\long\def\End{\operatorname{End}}%

\global\long\def\Hom{\operatorname{Hom}}%

\global\long\def\Ext{\operatorname{Ext}}%

\global\long\def\sk{\operatorname{sk}}%

\global\long\def\ihom{\underline{\operatorname{Hom}}}%

\global\long\def\N{\mathrm{N}}%

\global\long\def\-{\text{-}}%

\global\long\def\op{\mathrm{op}}%

\global\long\def\To{\Rightarrow}%

\global\long\def\rr{\rightrightarrows}%

\global\long\def\rl{\rightleftarrows}%

\global\long\def\mono{\rightarrowtail}%

\global\long\def\epi{\twoheadrightarrow}%

\global\long\def\comma{\downarrow}%

\global\long\def\ot{\leftarrow}%

\global\long\def\corr{\leftrightsquigarrow}%

\global\long\def\lim{\operatorname{lim}}%

\global\long\def\colim{\operatorname{colim}}%

\global\long\def\holim{\operatorname{holim}}%

\global\long\def\hocolim{\operatorname{hocolim}}%

\global\long\def\Ran{\operatorname{Ran}}%

\global\long\def\Lan{\operatorname{Lan}}%

\global\long\def\Sk{\operatorname{Sk}}%

\global\long\def\Sd{\operatorname{Sd}}%

\global\long\def\Ex{\operatorname{Ex}}%

\global\long\def\Cosk{\operatorname{Cosk}}%

\global\long\def\Sing{\operatorname{Sing}}%

\global\long\def\Sp{\operatorname{Sp}}%

\global\long\def\Spc{\operatorname{Spc}}%

\global\long\def\Ho{\operatorname{Ho}}%

\global\long\def\Fun{\operatorname{Fun}}%

\global\long\def\map{\operatorname{map}}%

\global\long\def\diag{\operatorname{diag}}%

\global\long\def\Gap{\operatorname{Gap}}%

\global\long\def\cc{\operatorname{cc}}%

\global\long\def\Ob{\operatorname{Ob}}%

\global\long\def\Map{\operatorname{Map}}%

\global\long\def\Rfib{\operatorname{RFib}}%

\global\long\def\Lfib{\operatorname{LFib}}%

\global\long\def\Tw{\operatorname{Tw}}%

\global\long\def\Equiv{\operatorname{Equiv}}%

\global\long\def\Arr{\operatorname{Arr}}%

\global\long\def\Cyl{\operatorname{Cyl}}%

\global\long\def\Path{\operatorname{Path}}%

\global\long\def\Alg{\operatorname{Alg}}%

\global\long\def\ho{\operatorname{ho}}%

\global\long\def\Comm{\operatorname{Comm}}%

\global\long\def\Triv{\operatorname{Triv}}%

\global\long\def\triv{\operatorname{triv}}%

\global\long\def\Env{\operatorname{Env}}%

\global\long\def\Act{\operatorname{Act}}%

\global\long\def\act{\operatorname{act}}%

\global\long\def\loc{\operatorname{loc}}%

\global\long\def\Assem{\operatorname{Assem}}%

\global\long\def\Nat{\operatorname{Nat}}%

\global\long\def\lax{\mathrm{lax}}%

\global\long\def\weq{\mathrm{weq}}%

\global\long\def\fib{\mathrm{fib}}%

\global\long\def\cof{\mathrm{cof}}%

\global\long\def\inj{\mathrm{inj}}%

\global\long\def\univ{\mathrm{univ}}%

\global\long\def\Ker{\opn{Ker}}%

\global\long\def\Coker{\opn{Coker}}%

\global\long\def\Im{\opn{Im}}%

\global\long\def\Coim{\opn{Im}}%

\global\long\def\coker{\opn{coker}}%

\global\long\def\im{\opn{\mathrm{im}}}%

\global\long\def\coim{\opn{coim}}%

\global\long\def\gn{\mathrm{gn}}%

\global\long\def\Mon{\mathrm{Mon}}%

\global\long\def\Un{\mathrm{Un}}%

\global\long\def\St{\mathrm{St}}%

\global\long\def\CA{\operatorname{CAlg}}%

\global\long\def\rd{\mathrm{rd}}%

\global\long\def\xmono#1#2{\stackrel[#2]{#1}{\rightarrowtail}}%

\global\long\def\xepi#1#2{\stackrel[#2]{#1}{\twoheadrightarrow}}%

\global\long\def\adj{\stackrel[\longleftarrow]{\longrightarrow}{\bot}}%

\global\long\def\btimes{\boxtimes}%

\global\long\def\ps#1#2{\prescript{}{#1}{#2}}%

\global\long\def\ups#1#2{\prescript{#1}{}{#2}}%

\global\long\def\hofib{\mathrm{hofib}}%

\global\long\def\cofib{\mathrm{cofib}}%

\global\long\def\Vee{\bigvee}%

\global\long\def\w{\wedge}%

\global\long\def\t{\otimes}%

\global\long\def\bp{\boxplus}%

\global\long\def\rcone{\triangleright}%

\global\long\def\lcone{\triangleleft}%

\global\long\def\S{\mathsection}%

\global\long\def\p{\prime}%
 

\global\long\def\pp{\prime\prime}%

\global\long\def\W{\overline{W}}%

\global\long\def\o#1{\overline{#1}}%

\title[Monoidal Envelopes of Families and Operadic
Kan Extensions]{Monoidal Envelopes of Families of $\infty$-Operads and $\infty$-Operadic
Kan Extensions}
\author{Kensuke Arakawa}
\email{arakawa.kensuke.22c@st.kyoto-u.ac.jp}
\address{Department of Mathematics, Kyoto University, Kyoto, 606-8502, Japan}
\subjclass[2020]{18N70, 55P48}
\begin{abstract}
We provide details of the proof of Lurie's theorem on operadic Kan
extensions \cite[Theorem 3.1.2.3]{HA}. Along the way, we generalize
the construction of monoidal envelopes of $\infty$-operads to families
of $\infty$-operads and use it to construct the fiberwise direct
sum functor, both of which we characterize by certain universal properties.
Aside from their uses in the proof of Lurie's theorem, these results
and constructions have their independent interest.
\end{abstract}

\maketitle
\tableofcontents{}

\section*{Introduction}

In his book \cite{HA}, Lurie introduces (among other things) the
notion of $\infty$-operads, an $\infty$-categorical analog of (colored)
operads. Given an $\infty$-operad $\cal O^{\t}$ and a symmetric
monoidal $\infty$-category $\cal C^{\t}$, we can form the $\infty$-category
$\Alg_{\cal O}\pr{\cal C}$ of $\cal O$-algebras. If $f:\cal O^{\t}\to\cal O^{\p\t}$
is a morphism of $\infty$-operads, there is the induced forgetful
map $f^{*}:\Alg_{\cal O'}\pr{\cal C}\to\Alg_{\cal O}\pr{\cal C}$.
In analogy with restrictions and extensions of scalars, it is natural
to ask whether the map $f^{*}$ has a left adjoint. Lurie gives an
affirmative answer to this question in \cite[Corollary 3.1.3.5]{HA},
provided that $\cal C$ has enough colimits. Because of its fundamental
importance, the result has been applied repeatedly in his work. 

Lurie's proof of \cite[Corollary 3.1.3.5]{HA} relies on another major
theorem on operadic Kan extensions \cite[Theorem 3.1.2.3]{HA} (or
Theorem \ref{thm:3.1.2.3}), which we call the \textbf{fundamental
theorem of operadic Kan extensions}, or FTOK for short. The proof
of FTOK, as Lurie himself acknowledges, is very long. Perhaps because
of the length of the proof, he omits some crucial details of the proof.
This note aims to provide all the details by making necessary constructions
and proving results on them. (In particular, we do not aim to provide
a more concise proof of FTOK than the one provided in \cite{HA}.)

We can roughly divide the details we provide into two parts: The first
one is the datum of ``coherent homotopy.'' In the proof of \cite[Theorem 3.1.2.3]{HA}
(to be more precise, on p. 337), Lurie claims that certain diagrams
are ``equivalent'' without writing the actual equivalence. Such
a practice is fairly common in the literature and is understandable
to some extent. However, in our case, we must be more attentive because
a considerable amount of effort is required to write down the actual
equivalence. We thus give a complete treatment of the equivalence
in this note. (See around Step 1 of the proof of Theorem \ref{thm:3.1.2.3}.)
The second missing detail is the verification that all the combinatorics
fits together. (This corresponds to Step 3 of Theorem \ref{thm:3.1.2.3}.)
Such detail, again, could be left to the reader if the verification
is trivial. But in our case, the verification is so complicated that
a sizable portion of readers will benefit from having it written down.
We thus record every single detail of the verification. 

Here is an outline of this note. In Section \ref{sec:A-Result-on},
we will prove an equivalent formulation of families of $\infty$-operads.
In Section \ref{sec:Monoidal-Envelopes-of}, we generalize Lurie's
monoidal envelopes to families of $\infty$-operads. Using monoidal
envelopes, we can define the fiberwise direct sum functor, whose properties
we discuss in Section \ref{sec:Direct_sum}. The constructions and
results in Sections \ref{sec:A-Result-on}, \ref{sec:Monoidal-Envelopes-of},
and \ref{sec:Direct_sum} will be used in writing down the equivalence
discussed in the previous paragraph, but they also have some independent
interest. To the author's knowledge, these constructions and results
have not appeared elsewhere. Finally, in Section \ref{sec:FTOK},
we will give a complete proof of FTOK.

\section*{Notation and Terminology}

Our notation and terminology mostly follow those of \cite{HA}. Here
are some deviations.
\begin{itemize}
\item We will say that a morphism of simplicial sets is \textbf{final} if
it is cofinal in the sense of \cite{HTT}, and \textbf{initial} if
its opposite is final. 
\item We will write $\Delta^{-1}$ for the empty simplicial set. 
\item If $\cal C$ is a category and $\alpha$ is an ordinal (or more generally
a well-ordered set), then an \textbf{$\alpha$-sequence} in $\cal C$
is a functor $F:\alpha\to\cal C$ such that the map $\colim_{\beta<\lambda}F\beta\to F\beta$
is an isomorphism for each limit ordinal less than $\alpha$. 
\item If $X$ is a simplicial set, then we denote the cone point of the
simplicial set $X^{\rcone}$ by $\infty$.
\end{itemize}

\section{\label{sec:A-Result-on}A Result on Families of $\infty$-Operads}

Let $\cal C$ be an $\infty$-category. Recall that a $\cal C$\textbf{-family
of $\infty$-operads} \cite[Definition 2.3.2.1]{HA} is a categorical
fibration $p:\cal M^{\t}\to\cal C\times N\pr{\Fin_{\ast}}$ satisfying
the following conditions:
\begin{itemize}
\item [(a)]For each object $M\in\cal M^{\t}$ with image $\pr{C,\inp m}\in\cal C\times N\pr{\Fin_{\ast}}$
and for each inert map $\alpha:\inp m\to\inp n$ in $N\pr{\Fin_{\ast}}$,
the morphism $\pr{\id_{C},\alpha}$ admits a $p$-cocartesian lift.
\item [(b)]Let $M\in\cal M^{\t}$ be an object with image $\pr{C,\inp m}\in\cal C\times N\pr{\Fin_{\ast}}$,
where $m\geq1$. Choose for each $1\leq i\leq m$ a $p$-cocartesian
lift $f_{i}:M\to M_{i}$ over $\pr{\id_{C},\rho^{i}}:\pr{C,\inp n}\to\pr{C,\inp 1}$.
Then the morphisms $f_{i}$ form a $p$-limit cone. Every object in
$\cal M_{\inp 0}^{\t}$ is $p$-terminal.
\item [(c)]Let $n\geq1$ and $C\in\cal C$. Given objects $M_{1},\dots,M_{n}\in\cal M_{C}$,
there is an object $M\in\cal M^{\t}$ lying over $\pr{C,\inp n}$
which admits $p$-cocartesian morphisms $M\to M_{i}$ over $\pr{\id_{C},\rho^{i}}:\pr{C,\inp n}\to\pr{C,\inp 1}$.
\end{itemize}
The goal of this section is to prove the following equivalent formulation
of families of $\infty$-operads:
\begin{prop}
\label{prop:sec1_main}Let $\cal C$ be an $\infty$-category and
let $p:\cal M^{\t}\to\cal C\times N\pr{\Fin_{\ast}}$ be a categorical
fibration satisfying conditions (a) and (b) above. Then the following
conditions are equivalent:
\begin{itemize}
\item [(c-i)]The map $p$ satisfies condition (c).
\item [(c-ii)]For each $1\leq i\leq n$, let $\rho_{!}^{i}:\cal M_{\inp n}^{\t}\to\cal M_{\inp 1}^{\t}$
be the functor over $\cal C$ induced by the morphism $\rho^{i}$.
Then for each $n\geq1$, the functor
\[
\pr{\rho_{!}^{i}}_{1\leq i\leq n}:\cal M_{\inp n}^{\t}\to\cal M\times_{\cal C}\cdots\times_{\cal C}\cal M
\]
is an equivalence of $\infty$-categories. 
\end{itemize}
\end{prop}
Here the functor $\rho_{!}^{i}:\cal M_{\inp n}^{\t}\to\cal M$ is
obtained in the following way: Since every inert morphism in $\cal C\times N\pr{\Fin_{\ast}}$
admits a $p$-cocartesian lift, it is possible to choose a natural
transformation $\cal M_{\inp n}^{\t}\times\Delta^{1}\to\cal M^{\t}$
fitting into the commutative diagram % https://q.uiver.app/?q=WzAsNCxbMSwwLCIoXFxtYXRoY2Fse019Xlxcb3RpbWVzICxcXG1hdGhjYWx7RX0pIl0sWzEsMSwiXFxtYXRoY2Fse0N9XlxcZmxhdFxcdGltZXMgTihcXG1hdGhzZntGaW59X1xcYXN0KV5cXG5hdHVyYWwsIl0sWzAsMSwiKFxcbWF0aGNhbHtNfV5cXG90aW1lcyBfe1xcbGFuZ2xlIG5cXHJhbmdsZX0pXlxcZmxhdFxcdGltZXMgKFxcRGVsdGFeMSleXFxzaGFycCJdLFswLDAsIihcXG1hdGhjYWx7TX1eXFxvdGltZXMgX3tcXGxhbmdsZSBuXFxyYW5nbGV9KV5cXGZsYXRcXHRpbWVzIFxcezBcXH0iXSxbMiwxLCJwXzJcXHRpbWVzIFxccmhvXmkiLDJdLFswLDFdLFszLDBdLFszLDJdLFsyLDAsIiIsMSx7InN0eWxlIjp7ImJvZHkiOnsibmFtZSI6ImRhc2hlZCJ9fX1dXQ==
\[\begin{tikzcd}
	{(\mathcal{M}^\otimes _{\langle n\rangle})^\flat\times \{0\}} & {(\mathcal{M}^\otimes ,\mathcal{E})} \\
	{(\mathcal{M}^\otimes _{\langle n\rangle})^\flat\times (\Delta^1)^\sharp} & {\mathcal{C}^\flat\times N(\mathsf{Fin}_\ast)^\natural,}
	\arrow["{p_2\times \rho^i}"', from=2-1, to=2-2]
	\arrow[from=1-2, to=2-2]
	\arrow[from=1-1, to=1-2]
	\arrow[from=1-1, to=2-1]
	\arrow[dashed, from=2-1, to=1-2]
\end{tikzcd}\]where $\cal E$ denotes the set of $p$-cocartesian edges over inert
morphisms in $\cal C\times N\pr{\Fin_{\ast}}$ whose image in $\cal C$
is degenerate, and $N\pr{\Fin_{\ast}}^{\natural}$ is the marked simplicial
set whose underlying simplicial set is $N\pr{\Fin_{\ast}}$ and with
inert morphisms marked. We understand that $\rho_{!}^{i}$ is the
restriction of the filler, so that it is a functor over $\cal C$
and its homotopy class over $\cal C$ is well-defined.
\begin{rem}
A reader conversant with the language of generalized $\infty$-operads
will recognize Proposition \ref{prop:sec1_main} as \cite[Proposition 2.3.2.11]{HA}.
However, in general, the diagram appearing in (2) of \cite[Definition 2.3.2.1]{HA}
is not strictly commutative but is commutative up to a specified natural
equivalence. So Proposition \ref{prop:sec1_main} is slightly different
from the cited proposition and should be regarded as its ``strictified
version.''
\end{rem}
The hardest part of the proof is the rectification of cocartesian
fibrations. We will overcome this problem by appealing to the simplicial
Grothendieck construction, which we explain in subsection \ref{subsec:Simplicial-Grothendieck-Construction}.
Proposition \ref{prop:sec1_main} will then be proved in subsection
\ref{subsec:Main-Result}.

\subsection{\label{subsec:Simplicial-Grothendieck-Construction}Simplicial Grothendieck
Construction}

In this subsection, we describe a convenient method to construct a
cocartesian fibration out of a functor $F:\cal C\to\sf{Cat}_{\Delta}$
of ordinary categories.

Let $\cal C$ be an ordinary category and $F:\cal C\to\sf{Cat}_{\Delta}$
a projectively fibrant functor. The (\textbf{simplicial}) \textbf{Gronthendieck
construction} $\int F$ is the simplicial category which is defined
as follows:
\begin{enumerate}
\item The set of objects of $F$ is given by $\coprod_{C\in\cal C}\opn{ob}F\pr C$.
\item Let $C,C'\in\cal C$ and $X\in F\pr C$, $X'\in F\pr{C'}$. Then the
hom-simplicial sets is 
\[
\pr{\int F}\pr{X,X'}=\coprod_{f\in\cal C\pr{C,C'}}F\pr{C'}\pr{Ff\pr X,X'}.
\]
\end{enumerate}
Composition is given by the composites 
\begin{align*}
F\pr{C''}\pr{Fg\pr{X'},X''}\times F\pr{C'}\pr{Ff\pr X,X'} & \to F\pr{C''}\pr{Fg\pr{X'},X''}\times F\pr{C''}\pr{FgFf\pr X,FgX'}\\
 & \xrightarrow{\circ}F\pr{C''}\pr{F\pr{gf}\pr X,X''}.
\end{align*}
This construction defines a functor
\[
\int:\Fun\pr{\cal C,\sf{Cat}_{\Delta}}\to\pr{\sf{Cat}_{\Delta}}_{/\cal C}.
\]

The Grothendieck construction is closely related to Lurie's relative
nerve functor \cite[$\S$ 3.2.5]{HTT}:
\begin{prop}
\label{prop:relative_nerve_of_Gr_const}Let $\cal C$ be a category
and $F:\cal C\to\sf{Cat}_{\Delta}$ a functor. There is an isomorphism
of simplicial sets 
\[
N\pr{\int F}\cong N_{N\circ F}\pr{\cal C}
\]
between the homotopy coherent nerve of $\int F$ and the relative
nerve of the composite $N\circ F:\cal C\to\sf{Cat}_{\Delta}\xrightarrow{N}\SS$,
which commutes with the projection to $N\pr{\cal C}$. The isomorphism
is natural in $F$.
\end{prop}
\begin{proof}
An $n$-simplex of the relative nerve $N_{N\circ F}\pr{\cal C}$,
by definition, consists of an $n$-simplex $\sigma=C_{0}\xrightarrow{f_{1}}\cdots\xrightarrow{f_{n}}C_{n}$
in $N\pr{\cal C}$ and a simplicial functor $\varphi_{I}:\fr C[\Delta^{I}]\to F\pr{C_{\max I}}$
for each subset $I\subset[n]$, such that for any inclusions $I\subset J\subset[n]$
of subsets, the diagram % https://q.uiver.app/?q=WzAsNCxbMCwwLCJcXG1hdGhmcmFre0N9W1xcRGVsdGFeSV0iXSxbMSwwLCJGKENfe1xcbWF4IEl9KSJdLFsxLDEsIkYoQ197XFxtYXggSn0pIl0sWzAsMSwiXFxtYXRoZnJha3tDfVtcXERlbHRhXkpdIl0sWzAsMV0sWzEsMl0sWzAsM10sWzMsMl1d
\[\begin{tikzcd}
	{\mathfrak{C}[\Delta^I]} & {F(C_{\max I})} \\
	{\mathfrak{C}[\Delta^J]} & {F(C_{\max J})}
	\arrow[from=1-1, to=1-2]
	\arrow[from=1-2, to=2-2]
	\arrow[from=1-1, to=2-1]
	\arrow[from=2-1, to=2-2]
\end{tikzcd}\]commutes. Given such an $n$-simplex $\pr{\sigma,\pr{x_{I}}_{I}}$,
we define a simplicial functor $\varphi:\fr C[\Delta^{n}]\to\int F$
as follows:
\begin{itemize}
\item For each integer $0\leq i\leq n$, we set $\varphi\pr i=\varphi_{[i]}\pr i$.
\item For each pair of integers $0\leq i\leq j\leq n$, the map
\[
\varphi:\fr C[\Delta^{n}]\pr{i,j}\to\pr{\int F}\pr{\varphi_{[i]}\pr i,\varphi_{[i]}\pr j}
\]
is the composite 
\begin{align*}
\fr C[\Delta^{n}]\pr{i,j} & \xrightarrow{\varphi_{[j]}}F\pr{C_{j}}\pr{\varphi_{[j]}\pr i,\varphi_{[j]}\pr j}\\
 & =F\pr{C_{j}}\pr{F\pr{f_{ij}}\pr{\varphi_{[i]}\pr i},\varphi_{[j]}\pr j}\\
 & \hookrightarrow\pr{\int F}\pr{\varphi_{[i]}\pr i,\varphi_{[i]}\pr j}.
\end{align*}
Here $f_{ij}:C_{i}\to C_{j}$ is the composite $f_{j}\cdots f_{i+1}$. 
\end{itemize}
The assignment $\pr{\sigma,\pr{x_{I}}_{I}}\mapsto\varphi$ determines
a morphism of simplicial sets
\[
\Phi:N_{N\circ F}\pr{\cal C}\to N\pr{\int F}
\]
over $N\pr{\cal C}$. We claim that it is an isomorphism of simplicial
sets by constructing an inverse. 

Let $\varphi:\fr C[\Delta^{n}]\to\int F$ be a simplicial functor.
Its projection to $\cal C$ determines an $n$-simplex $\sigma=C_{0}\xrightarrow{f_{1}}\cdots\xrightarrow{f_{n}}C_{n}$
in $N\pr{\cal C}$. For each subinterval $I\subset[n]$, define a
simplicial functor $\varphi_{I}:\fr C[\Delta^{I}]\to F\pr{C_{\max I}}$
as follows:
\begin{itemize}
\item For each integer $i\in I$, we set $\varphi_{I}\pr i=Ff_{i,\max I}\pr{\varphi\pr i}$.
\item For each pair of integers $i,j\in I$ with $i\le j$, the map
\[
\varphi_{I}:\fr C[\Delta^{I}]\pr{i,j}\to F\pr{C_{\max I}}\pr{\varphi_{I}\pr i,\varphi_{I}\pr j}
\]
is the composite
\begin{align*}
\fr C[\Delta^{I}]\pr{i,j} & =\fr C[\Delta^{n}]\pr{i,j}\\
 & \xrightarrow{\varphi}F\pr{C_{j}}\pr{F\pr{f_{ij}}\pr{\varphi\pr i},\varphi\pr j}\\
 & \xrightarrow{Ff_{j,\max I}}F\pr{C_{\max I}}\pr{F\pr{f_{i,\max I}}\pr{\varphi\pr i},F\pr{f_{j,\max I}}\pr{\varphi\pr j}}\\
 & =F\pr{C_{\max I}}\pr{\varphi_{I}\pr i,\varphi_{I}\pr j}.
\end{align*}
\end{itemize}
Next, for an arbitrary subset $J\subset[n]$, we define a simplicial
functor $\varphi_{J}:\fr C[\Delta^{J}]\to F\pr{C_{\max J}}$ to be
the restriction of $\varphi_{[\max J]}$. Then the pair $\pr{\sigma,\pr{\varphi_{I}}_{I}}$
is an $n$-simplex of $N_{N\circ F}\pr{\cal C}$, and the assignment
$\varphi\mapsto\pr{\sigma,\pr{\varphi_{I}}_{I}}$ determines an inverse
of $\Phi$.
\end{proof}
\begin{cor}
\label{cor:sgr}Let $\cal C$ be a category and $F:\cal C\to\sf{Cat}_{\Delta}$
a projectively fibrant functor. Then the functor 
\[
p:N\pr{\int F}\to N\pr{\cal C}
\]
is a cocartesian fibration. A morphism $\pr{f,g}:\pr{C,X}\to\pr{D,Y}$
in $N\pr{\int F}$ is $p$-cocartesian if and only if the morphism
$g:\pr{Ff}\pr X\to Y$ of $N\pr{F\pr D}$ is an equivalence.
\end{cor}
\begin{proof}
Define $\widetilde{F}:\cal C\to\SS^{+}$ by $\widetilde{F}C=N\pr{FC}^{\natural}$,
the homotopy coherent nerve of $FC$ with equivalences marked. Then
$\widetilde{F}$ is a projectively fibrant functor, so by \cite[Proposition 3.2.5.18]{HTT}
its relative nerve $N_{\widetilde{F}}^{+}\pr{\cal C}\to N\pr{\cal C}^{\sharp}$
is a fibrant object of $\SS^{+}/N\pr{\cal C}$ equipped with the cocartesian
model structure. The claim now follows from Proposition \ref{prop:relative_nerve_of_Gr_const}.
\end{proof}
\begin{rem}
The advantage of working with the simplicial Grothendieck construction
is that we can be very explicit about the functors induced by cocartesian
fibrations. Recall that if $p:\cal A\to\cal B$ is a cocartesian fibration
of $\infty$-categories and $f:X\to Y$ is a morphism in $\cal B$,
the induced functor $f_{!}:\cal A_{X}=\cal A\times_{\cal B}\{X\}\to\cal A_{Y}$
is defined by choosing a cocartesian natural transformation $\cal A_{X}\times\Delta^{1}\to\cal B$
fitting into the commutative diagram % https://q.uiver.app/?q=WzAsNSxbMCwwLCJcXG1hdGhjYWx7QX1fWFxcdGltZXMgXFx7MFxcfSJdLFswLDEsIlxcbWF0aGNhbHtBfV9YXFx0aW1lcyBcXERlbHRhXjEiXSxbMiwwLCJcXG1hdGhjYWx7QX0iXSxbMiwxLCJcXG1hdGhjYWx7Qn0iXSxbMSwxLCJcXERlbHRhXjEiXSxbMCwxXSxbMCwyXSxbMiwzXSxbMSwyLCIiLDEseyJzdHlsZSI6eyJib2R5Ijp7Im5hbWUiOiJkYXNoZWQifX19XSxbMSw0XSxbNCwzLCJmIiwyXV0=
\[\begin{tikzcd}
	{\mathcal{A}_X\times \{0\}} && {\mathcal{A}} \\
	{\mathcal{A}_X\times \Delta^1} & {\Delta^1} & {\mathcal{B}}
	\arrow[from=1-1, to=2-1]
	\arrow[from=1-1, to=1-3]
	\arrow[from=1-3, to=2-3]
	\arrow[dashed, from=2-1, to=1-3]
	\arrow[from=2-1, to=2-2]
	\arrow["f"', from=2-2, to=2-3]
\end{tikzcd}\]and then restricting it to $\cal A_{X}\times\{1\}$. 

In the case $p$ is the nerve of a simplicial functor of the form
$\int F\to\cal C$, given a morphism $f:C\to C'$ in $\cal C$, the
induced functor $f_{!}:N\pr{FC}\to N\pr{FC'}$ is nothing but $Nf$.
Indeed, the collection
\[
\{\id_{FfX}:FfX\to FfX\}_{X\in FC}
\]
defines a simplicial natural transformation from the inclusion $FC\hookrightarrow\int F$
to the composite $FC\xrightarrow{Ff}FC'\hookrightarrow\int F$, and
the nerve of this simplicial natural transformation gives us a natural
transformation $N\pr{FC}\times\Delta^{1}\to N\pr{\int F}$ over the
morphism $f$ which is pointwise cocartesian by Corollary \ref{cor:sgr}.
\end{rem}

\subsection{\label{subsec:Main-Result}Proof of Proposition \ref{prop:sec1_main}}

In this subsection, we prove Proposition \ref{prop:sec1_main} using
the results in the previous subsection.

Let $\pr{\sf{Cat}_{\Delta}}_{\fib}$ denote the full subcategory of
$\sf{Cat}_{\Delta}$ spanned by the fibrant simplicial categories.
We let $N^{\natural}:\pr{\sf{Cat}_{\Delta}}_{\fib}\to\SS^{+}$ denote
the functor obtained from the homotopy coherent nerve functor by marking
equivalences. 
\begin{lem}
\label{lem:replacement}Let $\cal C$ be a small category and let
$F,G:\cal C\to\SS_{\mathrm{cocart}}^{+}$ be projectively fibrant
functors. For any natural transformation $f:F\to G$, we can find
projectively fibrant functors $F',G':\cal C\to\sf{Cat}_{\Delta}$,
a projective fibration $f':F'\to G'$, and a commutative diagram % https://q.uiver.app/?q=WzAsNCxbMCwwLCJGIl0sWzEsMCwiTl5cXG5hdHVyYWxcXGNpcmMgRiciXSxbMSwxLCJOXlxcbmF0dXJhbFxcY2lyYyBHJyJdLFswLDEsIkciXSxbMCwxLCJcXHNpbWVxIl0sWzEsMiwiTl5cXG5hdHVyYWxcXGNpcmMgZiciXSxbMCwzLCJmIiwyXSxbMywyLCJcXHNpbWVxIiwyXV0=
\[\begin{tikzcd}
	F & {N^\natural\circ F'} \\
	G & {N^\natural\circ G'}
	\arrow["\simeq", from=1-1, to=1-2]
	\arrow["{N^\natural\circ f'}", from=1-2, to=2-2]
	\arrow["f"', from=1-1, to=2-1]
	\arrow["\simeq"', from=2-1, to=2-2]
\end{tikzcd}\]in $\Fun\pr{\cal C,\SS^{+}}$ whose horizontal arrows are weak equivalences.
\end{lem}
\begin{proof}
Let $F_{\flat},G_{\flat}:\cal C\to\SS$ denote the functors obtained
by forgetting the markings. Since $F,G$ are projectively fibrant,
for each object $C\in\cal C$, the simplicial set $F_{\flat}\pr C$
is an $\infty$-category and the marked edges of $F\pr C$ are precisely
the equivalences of $F_{\flat}\pr C$. Now find a commutative diagram
% https://q.uiver.app/?q=WzAsNCxbMCwwLCJcXG1hdGhmcmFre0N9XFxjaXJjIEZfXFxmbGF0Il0sWzAsMSwiXFxtYXRoZnJha3tDfVxcY2lyYyBHX1xcZmxhdCJdLFsxLDAsIkYnIl0sWzEsMSwiRyciXSxbMCwxLCJcXG1hdGhmcmFre0N9XFxjaXJjIGYiLDJdLFswLDIsIlxcc2ltZXEiXSxbMiwzLCJmJyJdLFsxLDMsIlxcc2ltZXEiLDJdXQ==
\[\begin{tikzcd}
	{\mathfrak{C}\circ F_\flat} & {F'} \\
	{\mathfrak{C}\circ G_\flat} & {G'}
	\arrow["{\mathfrak{C}\circ f}"', from=1-1, to=2-1]
	\arrow["\simeq", from=1-1, to=1-2]
	\arrow["{f'}", from=1-2, to=2-2]
	\arrow["\simeq"', from=2-1, to=2-2]
\end{tikzcd}\]in $\Fun\pr{\cal C,\sf{Cat}_{\Delta}}$, where $F',G'$ are projectively
fibrant, $\alpha'$ is a projective fibration, and the horizontal
arrows are weak equivalences. Since the adjunction
\[
\fr C:\SS_{\mathrm{Joyal}}\adj\sf{Cat}_{\Delta}:N
\]
is a Quillen equivalence, the horizontal arrows of the induced square
% https://q.uiver.app/?q=WzAsNCxbMCwwLCJGX1xcZmxhdCJdLFswLDEsIkdfXFxmbGF0Il0sWzEsMCwiTlxcY2lyYyBGJyJdLFsxLDEsIk5cXGNpcmMgRyciXSxbMCwxLCJmIiwyXSxbMCwyLCJcXHNpbWVxIl0sWzIsMywiTlxcY2lyYyBmJyJdLFsxLDMsIlxcc2ltZXEiLDJdXQ==
\[\begin{tikzcd}
	{F_\flat} & {N\circ F'} \\
	{G_\flat} & {N\circ G'}
	\arrow["f"', from=1-1, to=2-1]
	\arrow["\simeq", from=1-1, to=1-2]
	\arrow["{N\circ f'}", from=1-2, to=2-2]
	\arrow["\simeq"', from=2-1, to=2-2]
\end{tikzcd}\]in $\Fun\pr{\cal C,\SS}$ are weak equivalences (i.e., pointwise categorical
equivalences). By marking the equivalences, we obtain the desired
square.
\end{proof}
\begin{defn}
Let $n\geq1$. We let $\cal B\pr n$ denote the nerve of the category
$\{\infty\}\star\inp n^{\circ}$, where $\inp n^{\circ}=\{1,\dots,n\}$
is regarded as a discrete category. For each $1\leq i\leq n$, we
will denote the unique morphism $\infty\to i$ by $\rho^{i}$.
\end{defn}
\begin{prop}
\label{prop:pre-equivalence}Let $n\geq1$, and consider a commutative
diagram% https://q.uiver.app/?q=WzAsMyxbMCwwLCJcXG1hdGhjYWx7Q30iXSxbMiwwLCJcXG1hdGhjYWx7RH0iXSxbMSwxLCJcXG1hdGhjYWx7Qn0obikiXSxbMCwxLCJwIl0sWzEsMiwiciJdLFswLDIsInEiLDJdXQ==
\[\begin{tikzcd}
	{\mathcal{C}} && {\mathcal{D}} \\
	& {\mathcal{B}(n)}
	\arrow["p", from=1-1, to=1-3]
	\arrow["r", from=1-3, to=2-2]
	\arrow["q"', from=1-1, to=2-2]
\end{tikzcd}\]of $\infty$-categories. Assume the following:
\begin{enumerate}
\item The functors $q$ and $r$ are cocartesian fibrations, and the functor
$p$ is a categorical fibration.
\item The functor $p$ preserves cocartesian morphisms over $\cal B\pr n$.
\item The functor $\pr{\rho_{!}^{i}}_{i}:\cal D_{\infty}\to\prod_{1\leq i\leq n}\cal D_{i}$
induces an injection between the set of equivalence classes of objects
of $\cal D_{\inp n}$ and that of $\prod_{1\leq i\leq n}\cal D_{i}$.
\item For each object $C\in\cal C_{\infty}$, the $q$-cocartesian morphisms
$C\to C_{i}$ over $\rho^{i}$ form a $p$-limit cone. 
\item Given an object $\pr{C_{i}}_{i}\in\prod_{1\leq i\leq n}\cal C_{i}$,
there is an object $C\in\cal C$ which admits a $q$-cocartesian morphism
$C\to C_{i}$ over the morphism $\rho^{i}$ for every $i$.
\end{enumerate}
Then the functor
\[
\cal C_{\infty}\to\prod_{1\leq i\leq n}\cal C_{i}\times_{\prod_{1\leq i\leq n}\cal D_{i}}\cal D_{\infty}
\]
is an equivalence of $\infty$-categories.
\end{prop}
In the proof, we shall make use of the adjunction 
\[
r_{!}:\SS^{+}/\cal B\pr n\adj\Fun\pr{\{\infty\}\star\inp n^{\circ},\SS^{+}}:r^{*}
\]
between the (marked) relative nerve functor $r^{*}$ and its left
adjoint $r_{!}$, given by $r_{!}\pr X\pr x=X\times_{\cal B\pr n^{\sharp}}\cal B\pr n_{/x}^{\sharp}$.
\begin{proof}
The statement of the proposition is a bit vague, because the functors
$\cal C_{\infty}\to\cal C_{i}$ and $\cal D_{\infty}\to\cal D_{i}$
are defined only up to natural equivalence. A more precise formulation
is the following: Since $p$ is a categorical fibration, the map $\cal C^{\natural}\to\cal D^{\natural}$
is a fibration in the cocartesian model structures over $\cal B\pr n$,
where $\cal C^{\natural}$ is the marked simplicial set whose marked
edges are the $q$-cocartesian edges, and $\cal D^{\natural}$ is
defined similarly. (This is the opposite convention of \cite[$\S 3.2$]{HTT},
but it shouldn't case any confusion.) Therefore, for each object $i\in\inp n^{\circ}$,
there is a natural transformation $\cal C_{\infty}\times\Delta^{1}\to\cal C$
which fits into the commutative diagram % https://q.uiver.app/?q=WzAsNSxbMCwwLCJcXG1hdGhjYWx7Q31eXFxuYXR1cmFsX1xcaW5mdHlcXHRpbWVzIChcXHswXFx9KV5cXHNoYXJwIl0sWzAsMSwiXFxtYXRoY2Fse0N9XlxcbmF0dXJhbF9cXGluZnR5XFx0aW1lcyAoXFxEZWx0YV4xKV5cXHNoYXJwIl0sWzEsMSwiXFxtYXRoY2Fse0R9XlxcbmF0dXJhbF9cXGluZnR5XFx0aW1lcyAoXFxEZWx0YV4xKV5cXHNoYXJwIl0sWzIsMSwiXFxtYXRoY2Fse0R9XlxcbmF0dXJhbCJdLFsyLDAsIlxcbWF0aGNhbHtDfV5cXG5hdHVyYWwiXSxbMCwxXSxbMSwyXSxbMiwzXSxbNCwzXSxbMCw0XSxbMSw0LCIiLDEseyJzdHlsZSI6eyJib2R5Ijp7Im5hbWUiOiJkYXNoZWQifX19XV0=
\[\begin{tikzcd}
	{\mathcal{C}^\natural_\infty\times (\{0\})^\sharp} && {\mathcal{C}^\natural} \\
	{\mathcal{C}^\natural_\infty\times (\Delta^1)^\sharp} & {\mathcal{D}^\natural_\infty\times (\Delta^1)^\sharp} & {\mathcal{D}^\natural,}
	\arrow[from=1-1, to=2-1]
	\arrow[from=2-1, to=2-2]
	\arrow[from=2-2, to=2-3]
	\arrow[from=1-3, to=2-3]
	\arrow[from=1-1, to=1-3]
	\arrow[dashed, from=2-1, to=1-3]
\end{tikzcd}\]where the functor $\cal D^{\natural}\times\pr{\Delta^{1}}^{\sharp}\to\cal D^{\natural}$
is a $r$-cocartesian natural transformation over $\infty\to i$ which
starts from the inclusion $\cal D_{\infty}\hookrightarrow\cal D$.
The restriction of the filler determines a commutative diagram% https://q.uiver.app/?q=WzAsNCxbMCwwLCJcXG1hdGhjYWx7Q31fXFxpbmZ0eSJdLFsxLDAsIlxcbWF0aGNhbHtDfV9pIl0sWzEsMSwiXFxtYXRoY2Fse0R9X2kuIl0sWzAsMSwiXFxtYXRoY2Fse0R9X1xcaW5mdHkiXSxbMCwxLCJcXHJob15pXyEiXSxbMSwyLCJwIl0sWzAsMywicCIsMl0sWzMsMiwiXFxyaG9eaV8hIiwyXV0=
\[\begin{tikzcd}
	{\mathcal{C}_\infty} & {\mathcal{C}_i} \\
	{\mathcal{D}_\infty} & {\mathcal{D}_i.}
	\arrow["{\rho^i_!}", from=1-1, to=1-2]
	\arrow["p", from=1-2, to=2-2]
	\arrow["p"', from=1-1, to=2-1]
	\arrow["{\rho^i_!}"', from=2-1, to=2-2]
\end{tikzcd}\]The claim is that, for any such choice of functors $\cal D_{\infty}\to\cal D_{i}$
and $\cal C_{\infty}\to\cal C_{i}$, the resulting functor $\phi:\cal C_{\infty}\to\prod_{1\leq i\leq n}\cal C_{i}\times_{\prod_{1\leq i\leq n}\cal D_{i}}\cal D_{\infty}$
is an equivalence of $\infty$-categories.

We begin by proving the essential surjectivity. Let $\pr{\pr{C_{i}}_{i},D}\in\prod_{1\leq i\leq n}\cal C_{i}\times_{\prod_{1\leq i\leq n}\cal D_{i}}\cal D_{\infty}$
be an arbitrary object. By our hypothesis (5), we can find an object
$C\in\cal C_{\infty}$ which admits a $q$-cocartesian morphism $C\to C_{i}$
over $\rho^{i}$ for each $1\leq i\leq n$. By the uniqueness of cocartesian
morphisms, there is an equivalence to $\pr{C_{i}}_{i}\simeq\pr{\rho_{!}^{i}\pr C}_{i}$
in $\prod_{1\leq i\leq n}\cal C_{i}$. Its image in $\prod_{1\leq i\leq n}\cal D_{i}$
determines an equivalence $\pr{\rho_{!}^{i}\pr D}_{i}\simeq\pr{\rho_{!}^{i}\pr{p\pr C}}_{i}$.
By virtue of assumption (3), we obtain an equivalence $g:D\xrightarrow{\simeq}p(C)$
in $\cal D_{\infty}$. Using the fact that $p$ is a categorical fibration,
we can lift the morphism $g$ to an equivalence $\widetilde{g}:\widetilde{C}\xrightarrow{\simeq}C$
in $\cal C$. Therefore, replacing $f_{i}$ by $f_{i}\widetilde{g}$
if necessary, we may assume that $p\pr C=D$. For each $1\leq i\leq n$,
the morphism $f_{i}$ determines a functor $\cal C_{\infty}^{\natural}\times\{0\}^{\sharp}\cup\{C\}^{\sharp}\times\pr{\Delta^{1}}^{\sharp}\to\cal C^{\natural}$.
Since the inclusion
\[
\cal C_{\infty}^{\natural}\times\{0\}^{\sharp}\cup\{C\}^{\sharp}\times\pr{\Delta^{1}}^{\sharp}\hookrightarrow\cal C_{\infty}^{\natural}\times\pr{\Delta^{1}}^{\sharp}
\]
is a trivial cofibration in the cocartesian model structure over $\cal B\pr n$
\cite[Proposition 3.1.2.2]{HTT}, we can find a filler of the diagram
% https://q.uiver.app/?q=WzAsNSxbMCwwLCJcXG1hdGhjYWx7Q31eXFxuYXR1cmFsX1xcaW5mdHlcXHRpbWVzIChcXHswXFx9KV5cXHNoYXJwXFxjdXBcXHtDXFx9Xlxcc2hhcnAgXFx0aW1lcyAoXFxEZWx0YV4xKV5cXHNoYXJwIl0sWzAsMSwiXFxtYXRoY2Fse0N9XlxcbmF0dXJhbF9cXGluZnR5XFx0aW1lcyAoXFxEZWx0YV4xKV5cXHNoYXJwIl0sWzEsMSwiXFxtYXRoY2Fse0R9XlxcbmF0dXJhbF9cXGluZnR5XFx0aW1lcyAoXFxEZWx0YV4xKV5cXHNoYXJwIl0sWzIsMSwiXFxtYXRoY2Fse0R9XlxcbmF0dXJhbCJdLFsyLDAsIlxcbWF0aGNhbHtDfV5cXG5hdHVyYWwiXSxbMCwxXSxbMSwyXSxbMiwzXSxbNCwzXSxbMCw0XSxbMSw0LCIiLDEseyJzdHlsZSI6eyJib2R5Ijp7Im5hbWUiOiJkYXNoZWQifX19XV0=
\[\begin{tikzcd}
	{\mathcal{C}^\natural_\infty\times (\{0\})^\sharp\cup\{C\}^\sharp \times (\Delta^1)^\sharp} && {\mathcal{C}^\natural} \\
	{\mathcal{C}^\natural_\infty\times (\Delta^1)^\sharp} & {\mathcal{D}^\natural_\infty\times (\Delta^1)^\sharp} & {\mathcal{D}^\natural.}
	\arrow[from=1-1, to=2-1]
	\arrow[from=2-1, to=2-2]
	\arrow[from=2-2, to=2-3]
	\arrow[from=1-3, to=2-3]
	\arrow[from=1-1, to=1-3]
	\arrow[dashed, from=2-1, to=1-3]
\end{tikzcd}\]The filler determines a functor $\cal C_{\infty}\to\cal C_{i}$ which
is naturally equivalent to $\rho_{!}^{i}$ over $\cal D_{i}$. So
the functor $\phi':\cal C_{\infty}\to\prod_{1\leq i\leq n}\cal C_{i}\times_{\prod_{1\leq i\leq n}\cal D_{i}}\cal D_{\infty}$
obtained from these fillers is naturally equivalent to $\phi$. By
construction, we have $\phi'\pr C=\pr{\pr{C_{i}}_{i},D}$. This proves
that $\phi$ is essentially surjective.

Next, to prove that $\phi$ is fully faithful, we consider the commutative
diagram % https://q.uiver.app/?q=WzAsMTAsWzEsMSwiXFxtYXRoY2Fse0N9XlxcbmF0dXJhbF9cXGluZnR5XFx0aW1lcyAoXFxEZWx0YV4xKV5cXHNoYXJwIl0sWzEsMywiXFxtYXRoY2Fse0R9XlxcbmF0dXJhbF9cXGluZnR5XFx0aW1lcyAoXFxEZWx0YV4xKV5cXHNoYXJwIl0sWzMsMywiXFxwcm9kX3sxXFxsZXEgaVxcbGVxIG59XFxtYXRoY2Fse0R9XlxcbmF0dXJhbFxcdGltZXMgX3soXFxtYXRoY2Fse0J9KG4pKV5cXHNoYXJwfSh7XFx7XFxpbmZ0eVxcfVxcc3Rhclxce2lcXH19KV5cXHNoYXJwLiJdLFszLDEsIlxccHJvZF97MVxcbGVxIGlcXGxlcSBufVxcbWF0aGNhbHtDfV5cXG5hdHVyYWxcXHRpbWVzIF97KFxcbWF0aGNhbHtCfShuKSleXFxzaGFycH0oe1xce1xcaW5mdHlcXH1cXHN0YXJcXHtpXFx9fSleXFxzaGFycCJdLFs0LDAsIlxccHJvZF97MVxcbGVxIGlcXGxlcSBufVxcbWF0aGNhbHtDfV9pXlxcbmF0dXJhbCJdLFs0LDIsIlxccHJvZF97MVxcbGVxIGlcXGxlcSBufVxcbWF0aGNhbHtEfV9pXlxcbmF0dXJhbCJdLFsyLDAsIlxcbWF0aGNhbHtDfV9cXGluZnR5XlxcbmF0dXJhbFxcdGltZXMgXFx7MVxcfV5cXHNoYXJwIl0sWzIsMiwiXFxtYXRoY2Fse0R9X1xcaW5mdHleXFxuYXR1cmFsXFx0aW1lcyBcXHsxXFx9Xlxcc2hhcnAiXSxbMCwyLCJcXG1hdGhjYWx7Q31fXFxpbmZ0eV5cXG5hdHVyYWxcXHRpbWVzIFxcezBcXH1eXFxzaGFycCJdLFswLDQsIlxcbWF0aGNhbHtEfV9cXGluZnR5XlxcbmF0dXJhbFxcdGltZXMgXFx7MFxcfV5cXHNoYXJwIl0sWzAsMV0sWzEsMl0sWzMsMl0sWzAsM10sWzQsMywiXFxzaW1lcSIsMix7InN0eWxlIjp7InRhaWwiOnsibmFtZSI6Imhvb2siLCJzaWRlIjoiYm90dG9tIn19fV0sWzQsNV0sWzUsMiwiXFxzaW1lcSIsMCx7InN0eWxlIjp7InRhaWwiOnsibmFtZSI6Imhvb2siLCJzaWRlIjoiYm90dG9tIn19fV0sWzYsNF0sWzYsN10sWzcsNV0sWzcsMSwiXFxzaW1lcSJdLFs2LDAsIlxcc2ltZXEiXSxbOCwwLCJcXHNpbWVxIl0sWzksMSwiXFxzaW1lcSJdLFs4LDldLFs4LDMsIiIsMSx7InN0eWxlIjp7InRhaWwiOnsibmFtZSI6Imhvb2siLCJzaWRlIjoidG9wIn19fV0sWzksMiwiIiwxLHsic3R5bGUiOnsidGFpbCI6eyJuYW1lIjoiaG9vayIsInNpZGUiOiJ0b3AifX19XV0=
\[\begin{tikzcd}[scale cd=.65]
	&& {\mathcal{C}_\infty^\natural\times \{1\}^\sharp} && {\prod_{1\leq i\leq n}\mathcal{C}_i^\natural} \\
	& {\mathcal{C}^\natural_\infty\times (\Delta^1)^\sharp} && {\prod_{1\leq i\leq n}\mathcal{C}^\natural\times _{(\mathcal{B}(n))^\sharp}({\{\infty\}\star\{i\}})^\sharp} \\
	{\mathcal{C}_\infty^\natural\times \{0\}^\sharp} && {\mathcal{D}_\infty^\natural\times \{1\}^\sharp} && {\prod_{1\leq i\leq n}\mathcal{D}_i^\natural} \\
	& {\mathcal{D}^\natural_\infty\times (\Delta^1)^\sharp} && {\prod_{1\leq i\leq n}\mathcal{D}^\natural\times _{(\mathcal{B}(n))^\sharp}({\{\infty\}\star\{i\}})^\sharp.} \\
	{\mathcal{D}_\infty^\natural\times \{0\}^\sharp}
	\arrow[from=2-2, to=4-2]
	\arrow[from=4-2, to=4-4]
	\arrow[from=2-4, to=4-4]
	\arrow[from=2-2, to=2-4]
	\arrow["\simeq"', hook', from=1-5, to=2-4]
	\arrow[from=1-5, to=3-5]
	\arrow["\simeq", hook', from=3-5, to=4-4]
	\arrow[from=1-3, to=1-5]
	\arrow[from=1-3, to=3-3]
	\arrow[from=3-3, to=3-5]
	\arrow["\simeq", from=3-3, to=4-2]
	\arrow["\simeq", from=1-3, to=2-2]
	\arrow["\simeq", from=3-1, to=2-2]
	\arrow["\simeq", from=5-1, to=4-2]
	\arrow[from=3-1, to=5-1]
	\arrow[hook, from=3-1, to=2-4]
	\arrow[hook, from=5-1, to=4-4]
\end{tikzcd}\]We wish to show that the back face of this diagram is homotopy cartesian
in the cocartesian model structure in $\SS^{+}$. By \cite[Proposition 2.10]{StUn},
the maps labeled with $\simeq$ are weak equivalences in $\SS^{+}$.
It will therefore suffice to show that the front face is homotopy
cartesian. Note that the front face can be identified with the diagram
% https://q.uiver.app/?q=WzAsNCxbMCwwLCJyXyEoXFxtYXRoY2Fse0N9XlxcbmF0dXJhbCkoXFxpbmZ0eSkiXSxbMCwxLCJyXyEoXFxtYXRoY2Fse0R9XlxcbmF0dXJhbCkoXFxpbmZ0eSkiXSxbMSwwLCJcXHByb2RfezFcXGxlcSBpXFxsZXEgbn1yXyEoXFxtYXRoY2Fse0N9XlxcbmF0dXJhbCkoaSkiXSxbMSwxLCJcXHByb2RfezFcXGxlcSBpXFxsZXEgbn1yXyEoXFxtYXRoY2Fse0R9XlxcbmF0dXJhbCkoaSkiXSxbMCwxXSxbMCwyXSxbMSwzXSxbMiwzXV0=
\[\begin{tikzcd}
	{r_!(\mathcal{C}^\natural)(\infty)} & {\prod_{1\leq i\leq n}r_!(\mathcal{C}^\natural)(i)} \\
	{r_!(\mathcal{D}^\natural)(\infty)} & {\prod_{1\leq i\leq n}r_!(\mathcal{D}^\natural)(i),}
	\arrow[from=1-1, to=2-1]
	\arrow[from=1-1, to=1-2]
	\arrow[from=2-1, to=2-2]
	\arrow[from=1-2, to=2-2]
\end{tikzcd}\]where $r_{!}:\SS^{+}/\cal B\pr n\to\Fun\pr{\{\infty\}\star\inp n^{\circ},\SS^{+}}$
is the left adjoint of the relative nerve functor.

Find a commutative diagram % https://q.uiver.app/?q=WzAsNCxbMCwwLCJyXyEoXFxtYXRoY2Fse0N9XlxcbmF0dXJhbCkiXSxbMSwwLCJGIl0sWzEsMSwiRyJdLFswLDEsInJfIShcXG1hdGhjYWx7RH1eXFxuYXR1cmFsKSJdLFswLDEsIlxcc2ltZXEiXSxbMSwyLCJmIl0sWzMsMiwiXFxzaW1lcSIsMl0sWzAsMywicl8hKHApIiwyXV0=
\[\begin{tikzcd}
	{r_!(\mathcal{C}^\natural)} & F \\
	{r_!(\mathcal{D}^\natural)} & G
	\arrow["\simeq", from=1-1, to=1-2]
	\arrow["f", from=1-2, to=2-2]
	\arrow["\simeq"', from=2-1, to=2-2]
	\arrow["{r_!(p)}"', from=1-1, to=2-1]
\end{tikzcd}\]in $\Fun\pr{\{\infty\}\star\inp n^{\circ},\SS^{+}}$, where $F$ and
$G$ are projectively fibrant functors, $f$ is a projective fibration,
and the horizontal arrows are weak equivalences. According to Lemma
\ref{lem:replacement}, we may assume that the natural transformation
\[
f:\{\infty\}\star\inp n^{\circ}\times[1]\to\SS^{+}
\]
can be factored as 
\[
\{\infty\}\star\inp n^{\circ}\times[1]\xrightarrow{f'}\pr{\sf{Cat}_{\Delta}}_{\fib}\xrightarrow{N^{\natural}}\SS^{+}.
\]
We set $f'_{0}=F'$ and $g'_{0}=G'$, and write $\cal C_{\Delta}=\int F'$
and $\cal D_{\Delta}=\int G'$. We are then reduced to showing that
the square % https://q.uiver.app/?q=WzAsNCxbMCwwLCJOKFxcbWF0aGNhbHtDfV97XFxEZWx0YSxcXGluZnR5fSleXFxuYXR1cmFsIl0sWzEsMCwiXFxwcm9kX3sxXFxsZXEgaVxcbGVxIG59TihcXG1hdGhjYWx7Q31fe1xcRGVsdGEsaX0pXlxcbmF0dXJhbCJdLFsxLDEsIlxccHJvZF97MVxcbGVxIGlcXGxlcSBufU4oXFxtYXRoY2Fse0R9X3tcXERlbHRhLGl9KV5cXG5hdHVyYWwiXSxbMCwxLCJOKFxcbWF0aGNhbHtEfV97XFxEZWx0YSxcXGluZnR5fSleXFxuYXR1cmFsIl0sWzAsMV0sWzEsMl0sWzAsM10sWzMsMl1d
\[\begin{tikzcd}
	{N(\mathcal{C}_{\Delta,\infty})^\natural} & {\prod_{1\leq i\leq n}N(\mathcal{C}_{\Delta,i})^\natural} \\
	{N(\mathcal{D}_{\Delta,\infty})^\natural} & {\prod_{1\leq i\leq n}N(\mathcal{D}_{\Delta,i})^\natural}
	\arrow[from=1-1, to=1-2]
	\arrow[from=1-2, to=2-2]
	\arrow[from=1-1, to=2-1]
	\arrow[from=2-1, to=2-2]
\end{tikzcd}\]of marked simplicial sets is homotopy cartesian. Since the vertical
arrows are fibrations in $\SS_{\mathrm{cocart}}^{+}$ and all the
objects are fibrant, it suffices to show that the map 
\[
N\pr{\cal C_{\Delta,\infty}}^{\natural}\to N\pr{\cal D_{\Delta,\infty}}^{\natural}\times_{\prod_{1\leq i\leq n}N\pr{\cal D_{\Delta,i}}^{\natural}}\prod_{1\leq i\leq n}N\pr{\cal C_{\Delta,i}}^{\natural}
\]
is a weak equivalence in $\SS_{\mathrm{cocart}}^{+}$. This is equivalent
to the condition that the map
\[
\theta':N\pr{\cal C_{\Delta,\infty}}\to N\pr{\cal D_{\Delta,\infty}}\times_{\prod_{1\leq i\leq n}N\pr{\cal D_{\Delta,i}}}\prod_{1\leq i\leq n}N\pr{\cal C_{\Delta,i}}
\]
is an equivalence of $\infty$-categories. 

According to Proposition \ref{prop:relative_nerve_of_Gr_const}, the
map $r^{*}\pr f$ can be identified with the map
\[
N\pr{\int f'}:N\pr{\cal C_{\Delta}}^{\natural}\to N\pr{\cal D_{\Delta}}^{\natural}.
\]
Moreover, since the adjunction $r_{!}\dashv r^{*}$ is a Quillen equivalence,
the maps $\cal C^{\natural}\to r^{*}\pr F$ and $\cal D^{\natural}\to r^{*}\pr G$
are weak equivalences in the cocartesian model structure over $\cal B\pr n$.
So the functor $N\pr{\cal C_{\Delta}}\to N\pr{\cal D_{\Delta}}$ has
all the properties (1)-(5). So the proof of the essential surjectivity
of $\theta$ applies as well to $\theta'$, showing that $\theta'$
is essentially surjective. It remains to verify that $\theta'$ is
fully faithful. For this, it suffices to show that for each pair of
objects $X,Y\in\cal C_{\Delta,\infty}$, the diagram% https://q.uiver.app/?q=WzAsNCxbMCwwLCJcXG1hdGhjYWx7Q31fe1xcRGVsdGEsXFxpbmZ0eX0oWCxZKSJdLFswLDEsIlxcbWF0aGNhbHtEfV97XFxEZWx0YSxcXGluZnR5fShmWCxmWSkiXSxbMSwwLCJcXHByb2RfezFcXGxlcSBpXFxsZXEgbn1cXG1hdGhjYWx7Q31fe1xcRGVsdGEsaX0oKEYnXFxyaG9eaSlYLChGJ1xccmhvXmkpWSkiXSxbMSwxLCJcXHByb2RfezFcXGxlcSBpXFxsZXEgbn1cXG1hdGhjYWx7RH1fe1xcRGVsdGEsaX0oKEcnXFxyaG9eaSlmWCwoRydcXHJob15pKWZZKSJdLFswLDFdLFswLDJdLFsyLDNdLFsxLDNdXQ==
\[\begin{tikzcd}
	{\mathcal{C}_{\Delta,\infty}(X,Y)} & {\prod_{1\leq i\leq n}\mathcal{C}_{\Delta,i}((F'\rho^i)X,(F'\rho^i)Y)} \\
	{\mathcal{D}_{\Delta,\infty}(fX,fY)} & {\prod_{1\leq i\leq n}\mathcal{D}_{\Delta,i}((G'\rho^i)fX,(G'\rho^i)fY)}
	\arrow[from=1-1, to=2-1]
	\arrow[from=1-1, to=1-2]
	\arrow[from=1-2, to=2-2]
	\arrow[from=2-1, to=2-2]
\end{tikzcd}\]of Kan complexes is homotopy cartesian. Consider the commutative diagram
% https://q.uiver.app/?q=WzAsOCxbMCwxLCJcXG1hdGhjYWx7Q31fe1xcRGVsdGEsXFxpbmZ0eX0oWCxZKSJdLFswLDMsIlxcbWF0aGNhbHtEfV97XFxEZWx0YSxcXGluZnR5fShmWCxmWSkiXSxbMiwxLCJcXHByb2RfezFcXGxlcSBpXFxsZXEgbn1cXG1hdGhjYWx7Q31fe1xcRGVsdGEsaX0oKEYnXFxyaG9eaSlYLChGJ1xccmhvXmkpWSkiXSxbMiwzLCJcXHByb2RfezFcXGxlcSBpXFxsZXEgbn1cXG1hdGhjYWx7RH1fe1xcRGVsdGEsaX0oKEcnXFxyaG9eaSkgZlgsKEcnXFxyaG9eaSlmWSkiXSxbMSwwLCJcXG1hdGhjYWx7Q31fXFxEZWx0YShYLFkpIl0sWzMsMCwiXFxwcm9kIF97MVxcbGVxIGlcXGxlcSBufVxcbWF0aGNhbHtDfV9cXERlbHRhKFgsKEYnXFxyaG9eaSlZKSJdLFszLDIsIlxccHJvZF97MVxcbGVxIGlcXGxlcSBufVxcbWF0aGNhbHtEfV97XFxEZWx0YX0oZlgsKEcnXFxyaG9eaSlmWSkuIl0sWzEsMiwiXFxtYXRoY2Fse0R9X3tcXERlbHRhfShmWCxmWSkiXSxbMCwxXSxbMCwyXSxbMiwzXSxbMSwzXSxbMCw0XSxbMiw1XSxbNCw1XSxbMyw2XSxbNCw3XSxbNyw2XSxbMSw3XSxbNSw2XV0=
\[\begin{tikzcd}[column sep=small, scale cd=.75]
	& {\mathcal{C}_\Delta(X,Y)} && {\prod _{1\leq i\leq n}\mathcal{C}_\Delta(X,(F'\rho^i)Y)} \\
	{\mathcal{C}_{\Delta,\infty}(X,Y)} && {\prod_{1\leq i\leq n}\mathcal{C}_{\Delta,i}((F'\rho^i)X,(F'\rho^i)Y)} \\
	& {\mathcal{D}_{\Delta}(fX,fY)} && {\prod_{1\leq i\leq n}\mathcal{D}_{\Delta}(fX,(G'\rho^i)fY).} \\
	{\mathcal{D}_{\Delta,\infty}(fX,fY)} && {\prod_{1\leq i\leq n}\mathcal{D}_{\Delta,i}((G'\rho^i) fX,(G'\rho^i)fY)}
	\arrow[from=2-1, to=4-1]
	\arrow[from=2-1, to=2-3]
	\arrow[from=2-3, to=4-3]
	\arrow[from=4-1, to=4-3]
	\arrow[from=2-1, to=1-2]
	\arrow[from=2-3, to=1-4]
	\arrow[from=1-2, to=1-4]
	\arrow[from=4-3, to=3-4]
	\arrow[from=1-2, to=3-2]
	\arrow[from=3-2, to=3-4]
	\arrow[from=4-1, to=3-2]
	\arrow[from=1-4, to=3-4]
\end{tikzcd}\]The slanted arrows are isomorphisms of simplicial sets, so it suffices
to show that the back face is homotopy cartesian. This follows from
our assumption (4).
\end{proof}
We can now prove Proposition \ref{prop:sec1_main}.
\begin{proof}
[Proof of  Proposition \ref{prop:sec1_main}]Clearly (c-ii) implies
(c-i). Conversely, suppose that condition (c-i) is satisfied. For
each $n\geq1$, there is a unique functor $\cal B\pr n\to N\pr{\Fin_{\ast}}$
which is the identity map on morphisms. Applying Proposition \ref{prop:pre-equivalence}
to the functor $\cal M^{\t}\times_{N\pr{\Fin_{\ast}}}\cal B\pr n\to\cal C\times\cal B\pr n$,
we see that condition (c-ii) holds true.
\end{proof}

\section{\label{sec:Monoidal-Envelopes-of}Monoidal Envelopes of Families
of $\infty$-Operads}

In \cite[Section 2.2.4]{HA}, Lurie introduces the universal procedure
to make an arbitrary $\infty$-operad into a symmetric monoidal $\infty$-category.
The resulting symmetric monoidal $\infty$-category is called the
\textbf{monoidal envelope}. In this section, we will show that given
a family of $\infty$-operads, we can take the monoidal envelope of
each fiber to obtain a family of symmetric monoidal $\infty$-categories.
We will also prove that monoidal envelopes of families of $\infty$-operads
enjoys the expected universal property. 

We remark that the proofs of the results in this section are only
slight modifications of Lurie's original proofs of various results
concerning monoidal envelopes, which can be found in \cite[Section 2.2.4]{HA}.
Nevertheless, for the sake of completeness, we will record the proofs
whenever Lurie's proof does not apply verbatim.

\subsection{Definition and Construction}

In this subsection, we define monoidal envelopes of families of $\infty$-operads
and show that they are again families of $\infty$-operads.

We begin with a generalization of Lurie's construction \cite[Construction 2.2.4.1]{HA}:
\begin{defn}
Let $\cal M^{\t}$ be a generalized $\infty$-operad \cite[Definition 2.3.2.1]{HA}.
We define $\Act\pr{\cal M^{\t}}\subset\Fun\pr{\Delta^{1},\cal M^{\t}}$
to be the full subcategory spanned by the active morphisms. If $p:\cal M^{\t}\to\cal N^{\t}$
is a fibration of generalized $\infty$-operads, the \textbf{$\cal N$-monoidal
envelope} of $\cal M^{\t}$ is defined to be the fiber product
\[
\Env_{\cal N}\pr{\cal M}^{\t}=\cal M^{\t}\times_{\Fun\pr{\{0\},\cal N^{\t}}}\Act\pr{\cal N^{\t}}.
\]
In the case $\cal N^{\t}=N\pr{\Fin_{\ast}}$, we will write $\Env_{\cal N}\pr{\cal M}^{\t}=\Env\pr{\cal M}^{\t}$.
We will regard $\Env_{\cal N}\pr{\cal M}^{\t}$ as an $\infty$-category
over $\cal N^{\t}$ via the composite
\[
\Env_{\cal N}\pr{\cal M}^{\t}\to\Act\pr{\cal N^{\t}}\xrightarrow{\opn{ev}_{1}}\cal N^{\t}.
\]
\end{defn}
Thus, if $p:\cal M^{\t}\to N\pr{\Fin_{\ast}}$ is a generalized $\infty$-operad,
then $\Env\pr{\cal M}^{\t}$ is the $\infty$-category of pairs $\pr{M,\alpha:p\pr M\to\inp k}$,
where $M\in\cal M^{\t}$ and $\alpha$ is an active morphism of $N\pr{\Fin_{\ast}}$.
In particular, the $\infty$-category $\Env\pr{\cal M}=\Env\pr{\cal M}_{\inp 1}^{\t}$
may be identified with $\cal M_{\act}^{\t}$. The intuition here is
that an object $M\in\cal M_{\inp n}^{\t}$ is regarded as a ``formal
tensor product'' of the objects $\rho_{!}^{i}\pr M\in\cal M$.

Note that the (fully faithful) diagonal embedding $\cal N^{\t}\to\Act\pr{\cal N^{\t}}$
induces a fully faithful embedding $\cal M^{\t}\hookrightarrow\Env_{\cal N}\pr{\cal M}^{\t}$,
which partly justifies the terminology ``envelope.'' In fact, the
monoidal envelope enjoys a certain universal property, as we will
see in Subsection \ref{subsec:Universal-Property-of_monoidal_env}.

The goal of this note is to prove the following result, which is a
generalization of \cite[Proposition 2.2.4.4]{HA}.
\begin{prop}
\label{prop:2.2.4.4}Let $\cal C$ be an $\infty$-category and let
$p:\cal M^{\t}\to\cal C\times N\pr{\Fin_{\ast}}$ be a $\cal C$-family
of $\infty$-operads. Let 
\[
q:\Env\pr{\cal M}^{\t}\to\cal C\times N\pr{\Fin_{\ast}}
\]
denote the functor induced by the functors $\Env\pr{\cal M}^{\t}\to N\pr{\Fin_{\ast}}$
and $\Env\pr{\cal M}^{\t}\to\cal M^{\t}\to\cal C$. Then:
\begin{itemize}
\item [(a)]The functor $q$ makes $\Env\pr{\cal M}^{\t}$ into a $\cal C$-family
of $\infty$-operads. 
\item [(b)]A morphism of $\Env\pr{\cal M}^{\t}$ whose image in $\cal M^{\t}$
is inert is $q$-cocartesian. The converse holds if it lies over an
inert morphism in $\cal C\times N\pr{\Fin_{\ast}}$.
\end{itemize}
\end{prop}
\begin{rem}
In the situation of Proposition \ref{prop:2.2.4.4}, the fiber $\Env\pr{\cal M}_{C}^{\t}=\Env\pr{\cal M}^{\t}\times_{\cal C}\{C\}$
over an object $C\in\cal C$ is the monoidal envelope of the $\infty$-operad
$\cal M_{C}^{\t}$, which is a symmetric monoidal $\infty$-category
by \cite[Proposition 2.2.4.4]{HA}. 
\end{rem}
We will return to the proof after a sequence of lemmas.
\begin{lem}
\cite[Lemma 2.2.4.11]{HA}\label{lem:2.2.4.11}Let $p:\cal E\to\cal D$
be a cocartesian fibration of $\infty$-categories. Suppose there
is a full subcategory $\cal C\subset\cal E$ which has the following
properties:

\begin{enumerate}[label=(\roman*)]

\item For each object $D\in\cal D$, the inclusion $\cal C_{D}\subset\cal E_{D}$
admits a left adjoint $L_{D}:\cal E_{D}\to\cal C_{D}$.

\item For each morphism $f:D\to D'$ in $\cal D$, the associated
functor $f_{!}:\cal E_{D}\to\cal E_{D'}$ carries $L_{D}$-equivalences
(i.e., its image under $L_{D}$ is an equivalence) to $L_{D'}$-equivalences.

\end{enumerate}

Let $q=p\vert_{\cal C}:\cal C\to\cal D$ denote the restriction of
$p$. The following holds:
\begin{enumerate}
\item The functor $q=p\vert_{\cal C}:\cal C\to\cal D$ is a cocartesian
fibration.
\item Let $g:D\to D'$ be a morphism in $\cal D$ and $f:C\to C'$ a morphism
in $\cal C$ lifting $g$. Then $f$ is $q$-cocartesian if and only
if the map $g_{!}C\to C'$ is an $L_{D'}$-equivalence, where $g_{!}:\cal E_{D}\to\cal E_{D'}$
is the functor induced by $g$.
\end{enumerate}
\end{lem}
%
\begin{lem}
\cite[Remark 2.2.4.12]{HA}\label{lem:2.2.4.12}Let $p:\cal E\to\cal D$
be a cocartesian fibration of $\infty$-categories and $\cal C\subset\cal E$
a full subcategory. Consider the following conditions for $\cal C$:

\begin{enumerate}[label=(\roman*)]

\item For each object $D\in\cal D$, the inclusion $\cal C_{D}\subset\cal E_{D}$
admits a left adjoint $L_{D}:\cal E_{D}\to\cal C_{D}$.

\item For each morphism $f:D\to D'$ in $\cal D$, the associated
functor $f_{!}:\cal E_{D}\to\cal E_{D'}$ carries $L_{D}$-equivalences
(i.e., its image under $L_{D}$ is an equivalence) to $L_{D'}$-equivalences.

\end{enumerate}

\begin{enumerate}[label=(\roman*')]

\item The inclusion $\cal C\subset\cal E$ admits a left adjoint
$L:\cal E\to\cal C$.

\item The functor $p$ carries each $L$-equivalecne to to an equivalence.

\end{enumerate}

The conditions (i) and (ii) are equivalent to the conditions (i')
and (ii'). Moreover, if these conditions are satisfied, then for each
object $D\in\cal D$, a morphism in $\cal C_{D}$ is an $L$-equivalence
if and only if it is an $L_{D}$-equivalence.
\end{lem}
%
\begin{lem}
\cite[Lemma 2.2.4.13]{HA}\label{lem:2.2.4.13}Let $p:\cal C\to\cal D$
be an inner fibration of $\infty$-categories and $\cal D'\subset\cal D$
a full subcategory. Set $\cal C'=\cal C\times_{\cal D'}\cal D$. Assume
the following:

\begin{enumerate}[label=(\roman*)]

\item The inclusion $\cal D'\subset\cal D$ admits a left adjoint.

\item For each object $C\in\cal C$, the morphism $pC\to D'$ which
exhibits $D$ as a $\cal D'$-localization of $pC$ lifts to a $p$-cocartesian
morphism $C\to C'$.

\end{enumerate}

Then $\cal C'$ is a reflective subcategory of $\cal C$, and a morphism
$f:C\to C'$ in $\cal C$ exhibits $C'$ as a $\cal C'$-localization
of $C$ if and only if $f$ is $p$-cocartesian and the morphism $p\pr f$
exhibits $p\pr{C'}$ as a $\cal D'$-localization of $p\pr C$.
\end{lem}
%
\begin{lem}
\cite[Lemma 2.2.4.14]{HA}\label{lem:2.2.4.14}Let $p:\cal M^{\t}\to\cal N^{\t}$
be a fibration of generalized $\infty$-operads. Set $\cal D=\cal M^{\t}\times_{\Fun\pr{\{0\},\cal N^{\t}}}\Fun\pr{\Delta^{1},\cal N^{\t}}$.
The inclusion
\[
\Env_{\cal N}\pr{\cal M}^{\t}\subset\cal D
\]
admits a left adjoint. Moreover, a morphism $\alpha:D\to D'$ in $\cal D$
exhibits $D'$ as an $\Env_{\cal N}\pr{\cal M}^{\t}$-localization
of $D$ if and only if $D'\in\Env_{\cal N}\pr{\cal M}^{\t}$, the
image of $\alpha$ in $\cal M^{\t}$ is inert, and the image of $\alpha$
in $\cal N^{\t}$ is an equivalence.
\end{lem}
\begin{proof}
We recall from \cite[Lemma 5.2.8.19]{HTT} that the inclusion $\Act\pr{\cal N^{\t}}\subset\Fun\pr{\Delta^{1},\cal N^{\t}}$
admits a left adjoint and that a morphism $\alpha:g\to g'$ in $\Fun\pr{\Delta^{1},\cal N^{\t}}$
corresponding to a square % https://q.uiver.app/?q=WzAsNCxbMCwwLCJcXGJ1bGxldCJdLFsxLDAsIlxcYnVsbGV0Il0sWzEsMSwiXFxidWxsZXQiXSxbMCwxLCJcXGJ1bGxldCJdLFswLDEsImYiXSxbMSwyLCJnJyJdLFszLDIsImYnIiwyXSxbMCwzLCJnIiwyXV0=
\[\begin{tikzcd}
	\bullet & \bullet \\
	\bullet & \bullet
	\arrow["f", from=1-1, to=1-2]
	\arrow["{g'}", from=1-2, to=2-2]
	\arrow["{f'}"', from=2-1, to=2-2]
	\arrow["g"', from=1-1, to=2-1]
\end{tikzcd}\]exhibits $g'$ as an $\Act\pr{\cal N^{\t}}$-localization if and only
if $g'$ is active, $f$ is inert, and $f'$ is an equivalence. 

Now let $\pr{M,g:p\pr M\to N}$ be an object of $\Env_{\cal N}\pr{\cal M}^{\t}$,
and find a morphism $\alpha:g\to g'$ as in the previous paragraph.
According to Lemma \ref{lem:2.2.4.13}, it will suffice to show that
$\alpha$ admits a lift with domain $\pr{M,g}$ which is cocartesian
with respect to the projection $\cal D\to\Fun\pr{\Delta^{1},\cal N^{\t}}$.
Since $f$ is inert and $p$ is a fibration of generalized $\infty$-operads,
we can lift the morphism $f$ to an inert morphism $\beta:M\to M'$
in $\cal M^{\t}$. Then $\pr{\beta,\alpha}$ determines a morphism
$\pr{M,g}\to\pr{M',g'}$ which lifts $\alpha$. Since its image in
$\cal M^{\t}$ is $p$-cocartesian, the morphism $\pr{\alpha,\beta}$
is cocartesian with respect to $\cal D\to\Fun\pr{\Delta^{1},\cal N^{\t}}$.
The claim follows.
\end{proof}
\begin{lem}
\cite[Corollary 2.4.7.12]{HTT}\label{lem:corr_cocart}Let $f:\cal C\to\cal D$
be a functor of $\infty$-categories. The evaluation at $1\in\Delta^{1}$
induces a cocartesian fibration $p:\cal E=\cal C\times_{\Fun\pr{\{0\},\cal D}}\Fun\pr{\Delta^{1},\cal D}\to\cal D$,
and a morphism in $\cal E$ is $p$-cocartesian if and only if its
image in $\cal C$ is an equivalence.
\end{lem}
%
\begin{lem}
\cite[Lemma 2.2.4.15]{HA}\label{lem:2.2.4.15}Let $p:\cal M^{\t}\to\cal N^{\t}$
be a fibration of generalized $\infty$-operads. Set $\cal D=\cal M^{\t}\times_{\Fun\pr{\{0\},\cal N^{\t}}}\Fun\pr{\Delta^{1},\cal N^{\t}}$.
The following holds:
\begin{enumerate}
\item Evaluation at $1\in\Delta^{1}$ induces a cocartesian fibration $q':\cal D\to\cal N^{\t}$. 
\item A morphism in $\cal D$ is $q'$-cocartesian if and only if its image
in $\cal M^{\t}$ is an equivalence.
\item The map $q'$ restricts to a cocartesian fibration $q:\Env_{\cal N}\pr{\cal M}^{\t}\to\cal N^{\t}$.
\item A morphism $f$ in $\Env_{\cal N}\pr{\cal M}^{\t}$ is $q$-cocartesian
if and only if its image in $\cal M^{\t}$ is inert.
\end{enumerate}
\end{lem}
\begin{proof}
Assertions (1) and (2) follow from Lemma \ref{lem:corr_cocart}. Let
now $L:\cal D\to\Env_{\cal N}\pr{\cal M}^{\t}$ denote the left adjoint
of the inclusion, which exists by Lemma \ref{lem:2.2.4.14}. According
to this lemma, the functor $q'$ maps $L$-equivalences to equivalences.
Thus, by Lemmas \ref{lem:2.2.4.11} and \ref{lem:2.2.4.12}, the restriction
$q=q'\vert\Env_{\cal N}\pr{\cal M}^{\t}$ is a cocartesian fibration,
and a morphism $\pr{f,g}:\pr{M,\alpha}\to\pr{M',\alpha'}$ in $\Env_{\cal N}\pr{\cal M}^{\t}$
is $q$-cocartesian if and only if the map $\theta:g_{!}\pr{M,\alpha}\to\pr{M',\alpha'}$
is an $L$-equivalence. Here $g_{!}:\cal D_{X}\to\cal D_{X'}$ is
the functor induced by the morphism $g$. Now up to equivalence, the
map $\theta$ may be identified with the map $\pr{f,\id}:\pr{M,g\circ\alpha}\to\pr{M',\alpha'}$.
So Lemma \ref{lem:2.2.4.14} tells us that the latter condition holds
if and only if $f:M\to M'$ is inert. This proves (4).
\end{proof}
\begin{lem}
\cite[Lemma 2.2.4.16]{HA}\label{lem:2.2.4.16}Let $n\geq0$, and
let $\beta:\inp m\to\inp n$ and $\beta_{i}:\inp{m_{i}}\to\inp 1$
be active morphisms for $1\leq i\leq n$. Suppose we are given a commutative
diagram % https://q.uiver.app/?q=WzAsNCxbMCwwLCJcXGxhbmdsZSBtXFxyYW5nbGUiXSxbMSwwLCJcXGxhbmdsZSBtX2lcXHJhbmdsZSJdLFsxLDEsIlxcbGFuZ2xlIDEgXFxyYW5nbGUiXSxbMCwxLCJcXGxhbmdsZSBuXFxyYW5nbGUiXSxbMCwxLCJcXGdhbW1hX2kiXSxbMSwyLCJcXGJldGFfaSJdLFszLDIsIlxccmhvXmkiLDJdLFswLDMsIlxcYmV0YSIsMl1d
\[\begin{tikzcd}
	{\langle m\rangle} & {\langle m_i\rangle} \\
	{\langle n\rangle} & {\langle 1 \rangle}
	\arrow["{\gamma_i}", from=1-1, to=1-2]
	\arrow["{\beta_i}", from=1-2, to=2-2]
	\arrow["{\rho^i}"', from=2-1, to=2-2]
	\arrow["\beta"', from=1-1, to=2-1]
\end{tikzcd}\]for $1\leq i\leq n$ such that $\gamma_{i}$ is inert. Then the morphisms
$\{\beta\to\beta_{i}\}_{1\leq i\leq n}$ form a limit cone relative
to the map $p=\opn{ev}_{1}:\Act\pr{N\pr{\Fin_{\ast}}}\to N\pr{\Fin_{\ast}}$.
\end{lem}
%
\begin{proof}
[Proof of Proposition \ref{prop:2.2.4.4}]We begin with the first
half of part (b). Let $r:\cal C\times N\pr{\Fin_{\ast}}\to N\pr{\Fin_{\ast}}$
denote the projection. If a morphism $f$ in $\Env\pr{\cal M}^{\t}$
has the property that its image in $\cal M^{\t}$ is inert, then we
know from Lemma \ref{lem:2.2.4.15} that $f$ is $rq$-cocartesian,
and the image of $f$ in $\cal C$ is an equivalence. Thus $q\pr f$
ir $r$-cocartesian. Hence $f$ is $q$-cocartesian.

Next we proceed to (a). We must prove the following.
\begin{enumerate}
\item The functor $q$ is a categorical fibration.
\item Let $\pr{M,\alpha:\inp{n_{M}}\to\inp k}$ be an object of $\Env\pr{\cal M}^{\t}$
with image $C\in\cal C$, and let $\beta:\inp k\to\inp l$ be an inert
map. There is a $q$-cocartesian morphism $\pr{M,\alpha}\to\pr{M',\alpha'}$
lying over $\pr{\id_{C},\beta}$ whose image in $\cal M^{\t}$ is
inert.
\item Let $E\in\Env\pr{\cal M}^{\t}$ and set $\pr{C,\inp m}=q\pr E$. Then
the $q$-cocartesian morphisms $\gamma=\{E\to E_{i}\}_{1\leq i\leq k}$
lying over the morphisms $\{\pr{\id_{C},\rho^{i}}:\pr{C,\inp m}\to\pr{C,\inp 1}\}_{1\leq i\leq k}$
form a $q$-limit cone.
\item Let $k\geq1$, $C\in\cal C$, and $E_{1},\dots,E_{k}\in\Env\pr{\cal M}_{\pr{X,\inp 1}}^{\t}$.
There is an object $E\in\Env\pr{\cal M}_{\pr{C,\inp k}}^{\t}$ and
$q$-cocartesian morphisms $E\to E_{i}$ over $\rho^{i}$.
\end{enumerate}
Note that (2) and the first half of part (b) implies the latter half
of (b).

Assertion (1) is clear, because $q$ is the composite
\[
\Env\pr{\cal M}^{\t}\to\cal C\times\Act\pr{N\pr{\Fin_{\ast}}}\xrightarrow{\id\times\opn{ev}_{1}}\cal C\times N\pr{\Fin_{\ast}},
\]
and the first map is a pullback of $p$ and the second map is a categorical
fibration. For assertion (2), find a commutative diagram % https://q.uiver.app/?q=WzAsNCxbMCwwLCJcXGxhbmdsZSBuX01cXHJhbmdsZSJdLFsxLDAsIlxcbGFuZ2xlIGtcXHJhbmdsZSJdLFsxLDEsIlxcbGFuZ2xlIGxcXHJhbmdsZSJdLFswLDEsIlxcbGFuZ2xlICBuX00nXFxyYW5nbGUiXSxbMCwxLCJcXGFscGhhIl0sWzEsMiwiXFxiZXRhIl0sWzAsMywiXFxnYW1tYSIsMl0sWzMsMiwiXFxhbHBoYSciLDJdXQ==
\[\begin{tikzcd}
	{\langle n_M\rangle} & {\langle k\rangle} \\
	{\langle  n_M'\rangle} & {\langle l\rangle}
	\arrow["\alpha", from=1-1, to=1-2]
	\arrow["\beta", from=1-2, to=2-2]
	\arrow["\gamma"', from=1-1, to=2-1]
	\arrow["{\alpha'}"', from=2-1, to=2-2]
\end{tikzcd}\]in $\Fin_{\ast}$ with $\gamma$ inert and $\alpha'$ active. Since
$p$ is an $\cal C$-family of $\infty$-operads, there is an inert
morphism $f:M\to M'$ in $\cal M^{\t}$ lying over $\pr{\id_{C},\gamma}$.
The maps $f,\gamma,\beta$ determines a morphism $\pr{M,\alpha}\to\pr{M',\alpha'}$
in $\Env\pr{\cal M}^{\t}$ lying over $\pr{\id_{C},\beta}$, which
is $q$-cocartesian by the ``if'' part of part (b). 

For assertion (3), set $\cal D=\cal M^{\t}\times_{\Fun\pr{\{0\},\cal N^{\t}}}\Fun\pr{\Delta^{1},\cal N^{\otimes}}$
and consider the commutative diagram % https://q.uiver.app/?q=WzAsOCxbMCwwLCJcXG9wZXJhdG9ybmFtZXtFbnZ9X3tcXG1hdGhjYWx7Q31cXHRpbWVzIE4oXFxtYXRoc2Z7RmlufV9cXGFzdCl9KFxcbWF0aGNhbHtNfSleXFxvdGltZXMgIl0sWzAsMSwiXFxtYXRoY2Fse0N9XFx0aW1lcyBcXG1hdGhybXtBY3R9KE4oXFxtYXRoc2Z7RmlufV9cXGFzdCkpIl0sWzEsMCwiXFxtYXRoY2Fse0R9Il0sWzEsMSwiXFxtYXRoY2Fse0N9XFx0aW1lcyBcXG9wZXJhdG9ybmFtZXtGdW59KFxcRGVsdGFeMSxOKFxcbWF0aHNme0Zpbn1fXFxhc3QpKSJdLFsyLDAsIlxcbWF0aGNhbHtNfV5cXG90aW1lcyAiXSxbMiwxLCJcXG1hdGhjYWx7Q31cXHRpbWVzIE4oXFxtYXRoc2Z7RmlufV9cXGFzdCkiXSxbMSwyLCJcXG1hdGhjYWx7Q31cXHRpbWVzIE4oXFxtYXRoc2Z7RmlufV9cXGFzdCkiXSxbMCwyLCJcXG1hdGhjYWx7Q31cXHRpbWVzIE4oXFxtYXRoc2Z7RmlufV9cXGFzdCkiXSxbMCwxLCJxXzAiLDJdLFswLDIsIlxcYWxwaGEiXSxbMiwzXSxbMiw0LCJcXGJldGEiXSxbNCw1LCJwIl0sWzMsNSwiXFxvcGVyYXRvcm5hbWV7aWR9X3tcXG1hdGhjYWx7Q319XFx0aW1lcyBcXG9wZXJhdG9ybmFtZXtldn1fMCIsMl0sWzEsM10sWzMsNiwiXFxvcGVyYXRvcm5hbWV7aWR9X3tcXG1hdGhjYWx7Q319XFx0aW1lcyBcXG9wZXJhdG9ybmFtZXtldn1fMSJdLFsxLDcsIlxcb3BlcmF0b3JuYW1le2lkfV97XFxtYXRoY2Fse0N9fVxcdGltZXMgXFxvcGVyYXRvcm5hbWV7ZXZ9XzEiLDJdLFs3LDYsImVxdWFsIiwyXV0=
\[\begin{tikzcd}
	{\operatorname{Env}_{\mathcal{C}\times N(\mathsf{Fin}_\ast)}(\mathcal{M})^\otimes } & {\mathcal{D}} & {\mathcal{M}^\otimes } \\
	{\mathcal{C}\times \mathrm{Act}(N(\mathsf{Fin}_\ast))} & {\mathcal{C}\times \operatorname{Fun}(\Delta^1,N(\mathsf{Fin}_\ast))} & {\mathcal{C}\times N(\mathsf{Fin}_\ast)} \\
	{\mathcal{C}\times N(\mathsf{Fin}_\ast)} & {\mathcal{C}\times N(\mathsf{Fin}_\ast)}
	\arrow["{q_0}"', from=1-1, to=2-1]
	\arrow["\alpha", from=1-1, to=1-2]
	\arrow[from=1-2, to=2-2]
	\arrow["\beta", from=1-2, to=1-3]
	\arrow["p", from=1-3, to=2-3]
	\arrow["{\operatorname{id}_{\mathcal{C}}\times \operatorname{ev}_0}"', from=2-2, to=2-3]
	\arrow[from=2-1, to=2-2]
	\arrow["{\operatorname{id}_{\mathcal{C}}\times \operatorname{ev}_1}", from=2-2, to=3-2]
	\arrow["{\operatorname{id}_{\mathcal{C}}\times \operatorname{ev}_1}"', from=2-1, to=3-1]
	\arrow[equal, from=3-1, to=3-2]
\end{tikzcd}\]whose squares in the top rows are cartesian. We must show that $\gamma$
is an $\pr{\id_{\cal C}\times\opn{ev}_{1}}\circ q_{0}$-limit cone.
Since relative limits are stable under pullbacks \cite[Proposition 4.3.1.5]{HTT},
it suffices to show that $\beta\alpha\gamma$ is a $p$-limit cone
and that $q_{0}\gamma$ is an $\id_{\cal C}\times\opn{ev}_{1}$-limit
cone. First we consider $\beta\alpha\gamma$. Write $E=\pr{M,\alpha:\inp{n_{M}}\to\inp n}$
and $E_{i}=\pr{M_{i},\alpha_{i}:\inp{n_{M_{i}}}\to\inp 1}$. The morphism
$\gamma_{i}:E\to E_{i}$ can be depicted by a diagram% https://q.uiver.app/?q=WzAsNixbMSwwLCJcXGxhbmdsZSBuX01cXHJhbmdsZSJdLFsxLDEsIlxcbGFuZ2xlIG5fe01faX1cXHJhbmdsZSJdLFsyLDAsIlxcbGFuZ2xlIG1cXHJhbmdsZSJdLFsyLDEsIlxcbGFuZ2xlIDFcXHJhbmdsZSJdLFswLDAsIk0iXSxbMCwxLCJNX2kiXSxbMiwzLCJcXHJob15pIl0sWzEsMywiXFxhbHBoYV9pIiwyXSxbMCwyLCJcXGFscGhhIl0sWzAsMSwiXFxiZXRhX2kiLDJdLFs0LDUsImZfaSIsMl1d
\[\begin{tikzcd}
	M & {\langle n_M\rangle} & {\langle m\rangle} \\
	{M_i} & {\langle n_{M_i}\rangle} & {\langle 1\rangle}
	\arrow["{\rho^i}", from=1-3, to=2-3]
	\arrow["{\alpha_i}"', from=2-2, to=2-3]
	\arrow["\alpha", from=1-2, to=1-3]
	\arrow["{\beta_i}"', from=1-2, to=2-2]
	\arrow["{f_i}"', from=1-1, to=2-1]
\end{tikzcd}\]Since $\gamma_{i}$ is $q$-cocartesian, the morphism $f_{i}$ is
inert. Hence $\beta_{i}$ is inert. Since the horizontal arrows are
active, the commutativity of the diagram implies that the map $\beta_{i}:\inp{n_{M}}\to\inp{n_{M_{i}}}$
is the inert map which induces a bijection $\alpha^{-1}\pr i\cong\inp{n_{M_{i}}}^{\circ}$.
Combining this with the fact that $p$ is a $\cal C$-family of $\infty$-operads,
we deduce that the morphisms $\{f_{i}\}_{i}=\{\beta\alpha\gamma_{i}\}_{i}$
form a $p$-limit cone. For $q_{0}\gamma$, we use Lemma \ref{lem:2.2.4.16}.

To prove (4), write $E_{i}=\pr{M_{i},\alpha_{i}:\inp{n_{M_{i}}}\to\inp 1}$.
Let $n_{M}=n_{M_{1}}+\cdots+n_{M_{k}}$, and fix a bijection $\inp{n_{M}}\cong\Vee_{i=1}^{k}\inp{n_{M_{i}}}$.
Since $\cal M_{C}^{\t}$ is an $\infty$-operad, we can find an object
$M\in\cal C_{\pr{C,\inp{n_{M}}}}^{\t}$ and which admits a $p$-cocartesian
morphism $f_{i}:M\to M_{i}$ lying over the inert map $\beta_{i}:\inp{n_{M}}\to\inp{n_{M_{i}}}$
for each $1\leq i\leq k$. Let $\alpha:\inp{n_{M}}\to\inp k$ denote
the function which maps $\inp{n_{M_{i}}}^{\circ}$ to $i\in\inp k^{\circ}$,
and set $E=\pr{M,\alpha:\inp{n_{M}}\to\inp k}$. The triple $\pr{f_{i},\beta_{i},\rho^{i}}$
determines a morphism $E\to E_{i}$ which is $q$-cocartesian by part
(b). This proves (4).
\end{proof}

\subsection{\label{subsec:Universal-Property-of_monoidal_env}Universal Property
of Monoidal Envelopes}

In this subsection, we prove that the monoidal envelope of families
of $\infty$-operads has the expected universal property. The contents
of this subsection will not be used elsewhere in the note.

We begin with a definition.
\begin{defn}
Let $\cal C$ be an $\infty$-category and let $p:\cal M^{\t}\to\cal C\times N\pr{\Fin_{\ast}}$
a $\cal C$-family of $\infty$-operads. Let $q:\cal D^{\t}\to N\pr{\Fin_{\ast}}$
be a symmetric monoidal $\infty$-category. We let $\Fun^{\t}\pr{\Env\pr{\cal M},\cal D}\subset\Fun_{N\pr{\Fin_{\ast}}}\pr{\Env\pr{\cal M}^{\t},\cal D^{\t}}$
denote the full subcategory spanned by the functors which restrict
to a symmetric monoidal functor $\Env\pr{\cal M_{C}}^{\t}\to\cal D^{\t}$
for each $C\in\cal C$. 
\end{defn}
Here is the universal property of monoidal envelopes. The result is
a generalization of \cite[Proposition 2.2.4.9]{HA}.
\begin{prop}
Let $\cal C$ be an $\infty$-category and $p:\cal M^{\t}\to\cal C\times N\pr{\Fin_{\ast}}$
a $\cal C$-family of $\infty$-operads. Let $q:\cal D^{\t}\to N\pr{\Fin_{\ast}}$
be a symmetric monoidal $\infty$-category. The embedding $i:\cal M^{\t}\hookrightarrow\Env\pr{\cal M}^{\t}$
induces an equivalence of $\infty$-categories
\[
\Fun^{\t}\pr{\Env\pr{\cal M},\cal D}\xrightarrow{\simeq}\Alg_{\cal M}\pr{\cal D}.
\]
\end{prop}
\begin{proof}
Let $\cal E^{\t}$ denote the essential image of $i$. In other words,
$\cal E^{\t}$ is the full subcategory of $\Env\pr{\cal M}^{\t}$
spanned by those objects $\pr{M,\alpha:p\pr M\to\inp n}$ such that
$\alpha$ is an equivalence. The restriction map $\Alg_{\cal E}\pr{\cal D}\to\Alg_{\cal M}\pr{\cal D}$
is an equivalence of $\infty$-categories, so it will suffice to show
that the map
\[
\Fun^{\t}\pr{\Env\pr{\cal M},\cal D}\to\Alg_{\cal E}\pr{\cal D}
\]
is a trivial fibration. For this, it suffices to prove the following:

\begin{enumerate}[label=(\alph*)]

\item Every functor $\cal E^{\t}\to\cal D^{\t}$ over $N\pr{\Fin_{\ast}}$
admits a $q$-left Kan extension $\Env\pr{\cal M}^{\t}\to\cal D^{\t}$.

\item A functor $F\in\Fun_{N\pr{\Fin_{\ast}}}\pr{\Env\pr{\cal M},\cal D}$
restricts to a symmetric monoidal functor $\Env\pr{\cal M_{C}}^{\t}\to\cal D^{\t}$
for each $C\in\cal C$ if and only if it is a $q$-left Kan extension
of $F\vert_{\cal E^{\t}}$ and $F\vert_{\cal E^{\t}}$ is a morphism
of generalized $\infty$-operads.

\end{enumerate}

We start from (a). For each object $\pr{M,\alpha:p_{2}\pr M\to\inp n}$
in $\Env\pr{\cal M}^{\t}$, the $\infty$-category $\cal E_{/\pr{M,\alpha}}^{\t}$
has a terminal object, given by the map $\pr{\id_{M},\alpha}:\pr{M,\id_{p_{2}\pr M}}\to\pr{M,\alpha}$
(Lemma \ref{lem:2.2.4.13}). Therefore, if $F:\cal E^{\t}\to\cal D^{\t}$
is an arbitrary functor over $N\pr{\Fin_{\ast}}$, then the diagram
$\cal E_{/\pr{M,\alpha}}^{\t}\to\cal E^{\t}\xrightarrow{F}\cal D^{\t}$
has a $q$-colimit cone if and only if there is a $q$-cocartesian
morphism $F\pr{M,\id_{p_{2}\pr M}}\to D$ in $\cal D$ lifting $\alpha$.
The existence of such a $q$-cocartesian morphism follows from the
fact that $q$ is a cocartesian fibration.

For (b), let $p':\Env\pr{\cal M}^{\t}\to\cal C\times N\pr{\Fin_{\ast}}$
denote the projection. By the discussion in the previous paragraph,
it will suffice to show that $F$ preserves $p'$-cocartesian morphisms
lying over degenerate edges of $\cal X$ if and only if for each object
$\pr{M,\alpha:p_{2}\pr M\to\inp n}\in\Env\pr{\cal M}^{\t}$, the map
$F\pr{M,\id_{p_{2}\pr M}}\to F\pr{M,\alpha}$ is a $q$-cocartesian
morphism. Necessity is clear, since the morphism $\pr{M,\id_{p_{2}\pr M}}\to\pr{M,\alpha}$
is $p'$-cocartesian by Proposition \ref{prop:2.2.4.4}. For sufficiency,
suppose that every morphism of the form $\pr{M,\id_{p_{2}\pr M}}\to\pr{M,\alpha}$
is mapped to a $q$-cocartesian morphism. Let $\pr{f,g}:\pr{M,\alpha:p_{2}\pr M\to\inp n}\to\pr{M',\alpha':p_{2}\pr{M'}\to\inp{n'}}$
be a $p'$-cocartesian morphism in $\Env\pr{\cal M^{\t}}$ lying over
a degenerate edge of $\cal C$. We wish to show that $F\pr{f,g}$
is $q$-cocartesian. By hypothesis, the map $F\pr{M,\id_{p_{2}\pr M}}\to F\pr{M,\alpha}$
is $q$-cocartesian. So it suffice to show that the composite $F\pr{M,\id_{p_{2}\pr M}}\to F\pr{M,\alpha}\to F\pr{M',\alpha'}$
is $q$-cocartesian. We may therefore reduce to the case $\alpha=\id_{p_{2}\pr M}$.
Factor the map $g:p_{2}\pr M\to\inp{n'}$ as 
\[
p_{2}\pr M\xrightarrow{g'}\inp{n''}\xrightarrow{g''}\inp{n'},
\]
where $g'$ is inert and $g''$ is active. Lift $\pr{\id_{p_{1}\pr M},g'}$
to a $p$-cocartesian morphism $f':M\to M''$ in $\cal M^{\t}$. The
morphism $\pr{f',g'}:\pr{M,\id_{p_{2}\pr M}}\to\pr{M'',\id_{\inp{n''}}}$
is $p'$-cocartesian, so we can find a triangle in $\Env\pr{\cal M_{p_{1}\pr M}}^{\t}$
which we depict as % https://q.uiver.app/?q=WzAsNixbMCwzLCJwXzIoTSkiXSxbMSwyLCJcXGxhbmdsZSBuJydcXHJhbmdsZSJdLFsyLDMsIlxcbGFuZ2xlIG4nXFxyYW5nbGUiXSxbMiwxLCJwXzIoTScpIl0sWzEsMCwicF8yKE0nJykiXSxbMCwxLCJwXzIoTSkiXSxbMCwxLCJnJyJdLFsxLDIsImcnJyJdLFswLDIsInAoZykiLDJdLFszLDIsIlxcYWxwaGEnIl0sWzQsMywicChmJycpIl0sWzUsNCwicChmJykiXSxbNCwxLCJlcXVhbCIsMix7ImxhYmVsX3Bvc2l0aW9uIjo3MH1dLFs1LDAsImVxdWFsIiwyXSxbNSwzLCJwKGYpIl1d
\[\begin{tikzcd}
	& {p_2(M'')} \\
	{p_2(M)} && {p_2(M')} \\
	& {\langle n''\rangle} \\
	{p_2(M)} && {\langle n'\rangle}
	\arrow["{g'}", from=4-1, to=3-2]
	\arrow["{g''}", from=3-2, to=4-3]
	\arrow["{p(g)}"', from=4-1, to=4-3]
	\arrow["{\alpha'}", from=2-3, to=4-3]
	\arrow["{p(f'')}", from=1-2, to=2-3]
	\arrow["{p(f')}", from=2-1, to=1-2]
	\arrow[equal, from=1-2, to=3-2]
	\arrow[equal, from=2-1, to=4-1]
	\arrow["{p(f)}", from=2-1, to=2-3]
\end{tikzcd}\]The morphism $f''$ is both active and inert, so it is an equivalence.
Thus, by hypothesis, the map $F\pr{f'',g''}$ is a $q$-cocartesian.
The morphism $\pr{f',g'}$ is also $q$-cocartesian, since it is an
inert morphism of $\cal E^{\t}$. Hence $F\pr{f,g}$ is $q$-cocartesian,
as required.
\end{proof}

\section{\label{sec:Direct_sum}The Fiberwise Direct Sum Functor}

Let $\cal O^{\t}$ be an $\infty$-operad. Given objects $X_{1},\dots,X_{n}\in\cal O$,
we can find an object $X\in\cal O_{\inp n}^{\t}$ which admits a $p$-cocartesian
morphism $X\to X_{i}$ over $\rho^{i}:\inp n\to\inp 1$ for each $i$.
Such an object is denoted by $X_{1}\oplus\cdots\oplus X_{n}$ \cite[Remark 2.1.1.15]{HA}.
By inspection, the object $X_{1}\oplus\cdots\oplus X_{n}$ is the
equivalent to the image of the object $\pr{X_{1},\dots,X_{n}}\in\pr{\cal O_{\act}^{\t}}^{n}$
under the tensor product of $\Env\pr{\cal O}=\cal O_{\act}^{\t}$.
So the operation $\oplus$ can be made functorial using monoidal envelopes.
In fact, we can do it fiberwise:
\begin{defn}
Let $\cal C$ be an $\infty$-category, let $\cal M^{\t}\to\cal C\times N\pr{\Fin_{\ast}}$
be a $\cal C$-family of $\infty$-operads, and let $n\geq1$ be a
positive integer. We define a functor
\[
\bigoplus_{i=1}^{n}:\pr{\cal M_{\act}^{\t}}^{n}\times_{\cal C^{n}}\cal C=\cal M_{\act}^{\t}\times_{\cal C}\cdots\times_{\cal C}\cal M_{\act}^{\t}\to\cal M_{\act}^{\t}
\]
over $\cal C$, well-defined up to natural equivalence over $\cal C$,
as follows: According to Proposition \ref{prop:sec1_main}, there
is an equivalence of $\infty$-categories
\[
\Env\pr{\cal M}_{\inp n}^{\t}\xrightarrow{\simeq}\pr{\cal M_{\act}^{\t}}^{n}\times_{\cal C^{n}}\cal C
\]
over $\cal C$, obtained from the functors $\rho_{!}^{i}:\Env\pr{\cal M}_{\inp n}^{\t}\to\Env\pr{\cal M}_{\inp 1}^{\t}=\cal M_{\act}^{\t}$.
The functors $\Env\pr{\cal M}_{\inp n}^{\t}\to\cal C$ and $\pr{\cal M_{\act}^{\t}}^{n}\times_{\cal C^{n}}\cal C\to\cal C$
are categorical fibrations, so the above equivalence admits an inverse
equivalence
\[
\pr{\cal M_{\act}^{\t}}^{n}\times_{\cal C^{n}}\cal C\xrightarrow{\simeq}\Env\pr{\cal M}_{\inp n}^{\t}
\]
over $\cal C$. The functor $\bigoplus_{i=1}^{n}$ is the composite
\[
\pr{\cal M_{\act}^{\t}}^{n}\times_{\cal C^{n}}\cal C\xrightarrow{\simeq}\Env\pr{\cal M}_{\inp n}^{\t}\xrightarrow{\text{forget}}\cal M_{\act}^{\t}.
\]
\end{defn}
In this section, we will prove two important properties of the direct
sum functor: Its universal property and the interaction with slices.

\subsection{\label{subsec:Universal-Property-of}Universal Property of the Direct
Sum Functor }

In this subsection, we will characterize the direct sum functor by
a certain universal property (Corollary \ref{cor:oplus_p-limit}).
\begin{prop}
\label{prop:inverting_weak_equiv_of_pairs}Let $\cal M$ be a model
category. Suppose we are given a commutative diagram % https://q.uiver.app/?q=WzAsNCxbMCwwLCJFIl0sWzEsMCwiRSciXSxbMSwxLCJCJyJdLFswLDEsIkIiXSxbMCwxLCJmIl0sWzEsMiwicCciLDAseyJzdHlsZSI6eyJoZWFkIjp7Im5hbWUiOiJlcGkifX19XSxbMCwzLCJwIiwyLHsic3R5bGUiOnsiaGVhZCI6eyJuYW1lIjoiZXBpIn19fV0sWzMsMiwicSIsMix7InN0eWxlIjp7ImhlYWQiOnsibmFtZSI6ImVwaSJ9fX1dLFszLDIsIlxcc2ltZXEiXSxbMCwxLCJcXHNpbWVxIiwyXV0=
\[\begin{tikzcd}
	E & {E'} \\
	B & {B'}
	\arrow["f", from=1-1, to=1-2]
	\arrow["{p'}", two heads, from=1-2, to=2-2]
	\arrow["p"', two heads, from=1-1, to=2-1]
	\arrow["q"', two heads, from=2-1, to=2-2]
	\arrow["\simeq", from=2-1, to=2-2]
	\arrow["\simeq"', from=1-1, to=1-2]
\end{tikzcd}\]in $\cal M$. Assume the following:
\begin{enumerate}
\item The maps $p,p',q$ are fibrations.
\item The maps $f$ and $q$ are weak equivalences.
\item The object $E'$ is cofibrant. 
\item The map $q$ has a section $s:B'\to B$.
\end{enumerate}
Then there is a map $g:E'\to E$ rendering the diagram % https://q.uiver.app/?q=WzAsNCxbMCwwLCJFJyJdLFswLDEsIkInIl0sWzEsMCwiRSJdLFsxLDEsIkIiXSxbMCwxLCJwJyJdLFswLDIsImciXSxbMSwzLCJzIiwyXSxbMiwzLCJwIl1d
\[\begin{tikzcd}
	{E'} & E \\
	{B'} & B
	\arrow["{p'}", from=1-1, to=2-1]
	\arrow["g", from=1-1, to=1-2]
	\arrow["s"', from=2-1, to=2-2]
	\arrow["p", from=1-2, to=2-2]
\end{tikzcd}\]commutative, such that the composite $fg:E'\to E'$ is homotopic to
the identity in $\cal M_{/B'}$.
\end{prop}
\begin{proof}
Consider the diagram % https://q.uiver.app/?q=WzAsNixbMiwwLCJFIl0sWzIsMSwiQiJdLFsxLDEsIkInIl0sWzEsMCwiRSciXSxbMCwwLCJFIl0sWzAsMSwiQiJdLFswLDEsInAiLDJdLFsyLDEsInMiLDJdLFszLDIsInAnIiwyXSxbNSwyLCJxIiwyXSxbNCw1LCJwIiwyXSxbNCwzLCJmIl0sWzMsMCwiIiwyLHsic3R5bGUiOnsiYm9keSI6eyJuYW1lIjoiZGFzaGVkIn19fV1d
\[\begin{tikzcd}
	E & {E'} & E \\
	B & {B'} & B
	\arrow["p"', from=1-3, to=2-3]
	\arrow["s"', from=2-2, to=2-3]
	\arrow["{p'}"', from=1-2, to=2-2]
	\arrow["q"', from=2-1, to=2-2]
	\arrow["p"', from=1-1, to=2-1]
	\arrow["f", from=1-1, to=1-2]
	\arrow[dashed, from=1-2, to=1-3]
\end{tikzcd}\]in $\cal M_{/B'}$. Since $p$ is a fibration between fibrant objects
and $E'$ is cofibrant, \cite[Proposition A.2.3.1]{HTT} shows that
there is a map $g:E'\to E$ rendering the diagram commutative, such
that $[g][f]=[\id_{E}]$ in $\ho\pr{\cal M_{/B'}}$. Since $[f]$
is an isomorphism in $\ho\pr{\cal M_{/B'}}$, the uniqueness of inverses
implies that $[f][g]=[\id_{E'}]$ in $\ho\pr{\cal M_{/B'}}$. Since
$E'$ is a fibrant-cofibrant object of $\cal M_{/B'}$, we deduce
that $fg$ is homotopic to the identity in $\cal M_{/B'}$.
\end{proof}
\begin{defn}
Let $n\geq1$. Given integers $m_{1},\dots,m_{n}\geq0$, we shall
identify the pointed set $\Vee_{i=1}^{n}\inp{m_{i}}$ with the set
$\inp{m_{1}+\cdots+m_{n}}$ via the map
\[
\Vee_{i=1}^{n}\inp{m_{i}}\to\inp{m_{1}+\cdots+m_{n}}
\]
which maps $k\in\Vee_{i=1}^{n}\inp{m_{i}}$ to $\sum_{j<i}m_{j}+k$.
The maps $\Vee_{i=1}^{n}\inp{m_{i}}\to\inp{m_{i}}$ define an inert
natural transformation
\[
h_{i}:N\pr{\Fin_{\ast}}_{\act}^{n}\times\Delta^{1}\to N\pr{\Fin_{\ast}}.
\]
\end{defn}
\begin{prop}
\label{prop:oplus_p-limit}Let $n\ge1$, let $\cal C$ be an $\infty$-category,
and let $p:\cal M^{\t}\to\cal C\times N\pr{\Fin_{\ast}}$ be a $\cal C$-family
of $\infty$-operads. We can construct the functor $\bigoplus_{1\leq i\leq n}:\pr{\cal M_{\act}^{\t}}^{n}\times_{\cal C^{n}}\cal C\to\cal M^{\t}$
so that for each $1\leq i\leq n$, there is an inert natural transformation
$\bigoplus_{1\leq i\leq n}\to\opn{pr}_{i}$ rendering the diagram
% https://q.uiver.app/?q=WzAsNCxbMCwwLCIoKFxcbWF0aGNhbHtNfV5cXG90aW1lc197XFxtYXRocm17YWN0fX0pXm5cXHRpbWVzIF97XFxtYXRoY2Fse0N9Xm59XFxtYXRoY2Fse0N9KVxcdGltZXMgXFxEZWx0YSBeMSJdLFsxLDAsIlxcbWF0aGNhbHtNfV5cXG90aW1lcyAiXSxbMSwxLCJcXG1hdGhjYWx7Q31cXHRpbWVzIE4oXFxtYXRoc2Z7RmlufV9cXGFzdCkiXSxbMCwxLCJcXG1hdGhjYWx7Q31cXHRpbWVzIE4oXFxtYXRoc2Z7RmlufV9cXGFzdCApXm5fe1xcbWF0aHJte2FjdH19XFx0aW1lcyBcXERlbHRhXjEiXSxbMCwxLCJcXHdpZGV0aWxkZXtofV9pIl0sWzEsMiwicCJdLFszLDIsIlxcb3BlcmF0b3JuYW1le2lkfV97XFxtYXRoY2Fse0N9fVxcdGltZXMgaF9pIiwyXSxbMCwzXV0=
\[\begin{tikzcd}
	{((\mathcal{M}^\otimes_{\mathrm{act}})^n\times _{\mathcal{C}^n}\mathcal{C})\times \Delta ^1} & {\mathcal{M}^\otimes } \\
	{\mathcal{C}\times N(\mathsf{Fin}_\ast )^n_{\mathrm{act}}\times \Delta^1} & {\mathcal{C}\times N(\mathsf{Fin}_\ast)}
	\arrow["{\widetilde{h}_i}", from=1-1, to=1-2]
	\arrow["p", from=1-2, to=2-2]
	\arrow["{\operatorname{id}_{\mathcal{C}}\times h_i}"', from=2-1, to=2-2]
	\arrow[from=1-1, to=2-1]
\end{tikzcd}\]commutative.
\end{prop}
\begin{proof}
We begin with the construction of the functor $\bigoplus_{1\leq i\leq n}$.
For each $1\leq i\leq n$, there is an inert natural transformation
$g_{i}:\Env\pr{N\pr{\Fin_{\ast}}}_{\inp n}^{\t}\times\Delta^{1}\to\Env\pr{N\pr{\Fin_{\ast}}}^{\t}$
from the inclusion to the functor 
\[
\pr{\alpha:\inp k\to\inp n}\to\pr{\alpha^{-1}\pr i_{\ast}\to\inp 1},
\]
where $\alpha^{-1}\pr i_{\ast}$ denotes the unique object $\inp m\in\Fin_{\ast}$
which admits an order-preserving bijection $\alpha^{-1}\pr i\cong\inp m^{\circ}$.
This natural transformation covers $\rho^{i}:\inp n\to\inp 1$. Thus
we may construct the functor $\rho_{!}^{i}:\Env\pr{N\pr{\Fin_{\ast}}}_{\inp n}^{\t}\to N\pr{\Fin_{\ast}}_{\act}$
by $\rho^{i}\pr{\alpha:\inp k\to\inp n}=\alpha^{-1}\pr i_{\ast}$.
Since the functor $p$ is a categorical fibration, it induces a fibration
$\Env\pr{\cal M}^{\t}\to\cal C\times\Env\pr{N\pr{\Fin_{\ast}}}^{\t}$
of generalized $\infty$-operads. Therefore, we can find an inert
natural transformation $\Env\pr{\cal M}_{\inp n}^{\t}\times\Delta^{1}\to\Env\pr{\cal M}^{\t}$
rendering the diagram % https://q.uiver.app/?q=WzAsNSxbMCwwLCJcXG9wZXJhdG9ybmFtZXtFbnZ9KFxcbWF0aGNhbHtNfSleXFxvdGltZXMgX3tcXGxhbmdsZSBuXFxyYW5nbGV9XFx0aW1lcyBcXHswXFx9Il0sWzIsMCwiXFxvcGVyYXRvcm5hbWV7RW52fShcXG1hdGhjYWx7TX0pXlxcb3RpbWVzICJdLFsyLDEsIlxcbWF0aGNhbHtDfVxcdGltZXMgXFxvcGVyYXRvcm5hbWV7RW52fShOKFxcbWF0aHNme0Zpbn1fXFxhc3QpKV5cXG90aW1lcyAiXSxbMCwxLCJcXG9wZXJhdG9ybmFtZXtFbnZ9KFxcbWF0aGNhbHtNfSleXFxvdGltZXMgX3tcXGxhbmdsZSBuXFxyYW5nbGV9XFx0aW1lcyBcXERlbHRhXjEiXSxbMSwxLCJcXG1hdGhjYWx7Q31cXHRpbWVzIFxcb3BlcmF0b3JuYW1le0Vudn0oTihcXG1hdGhzZntGaW59X1xcYXN0KSleXFxvdGltZXMgX3tcXGxhbmdsZSBuXFxyYW5nbGV9XFx0aW1lcyBcXERlbHRhXjEiXSxbMSwyXSxbMCwzXSxbMywxLCIiLDEseyJzdHlsZSI6eyJib2R5Ijp7Im5hbWUiOiJkYXNoZWQifX19XSxbMyw0XSxbNCwyLCJcXG9wZXJhdG9ybmFtZXtpZH1fe1xcbWF0aGNhbHtDfX1cXHRpbWVzIGdfaSIsMl0sWzAsMV1d
\[\begin{tikzcd}
	{\operatorname{Env}(\mathcal{M})^\otimes _{\langle n\rangle}\times \{0\}} && {\operatorname{Env}(\mathcal{M})^\otimes } \\
	{\operatorname{Env}(\mathcal{M})^\otimes _{\langle n\rangle}\times \Delta^1} & {\mathcal{C}\times \operatorname{Env}(N(\mathsf{Fin}_\ast))^\otimes _{\langle n\rangle}\times \Delta^1} & {\mathcal{C}\times \operatorname{Env}(N(\mathsf{Fin}_\ast))^\otimes }
	\arrow[from=1-3, to=2-3]
	\arrow[from=1-1, to=2-1]
	\arrow[dashed, from=2-1, to=1-3]
	\arrow[from=2-1, to=2-2]
	\arrow["{\operatorname{id}_{\mathcal{C}}\times g_i}"', from=2-2, to=2-3]
	\arrow[from=1-1, to=1-3]
\end{tikzcd}\]commutative. We use this inert natural transformation to define the
functor $\rho_{!}^{i}:\Env\pr{\cal M}_{\inp n}^{\t}\to\cal M_{\act}^{\t}$.
This will ensure that the diagram

% https://q.uiver.app/?q=WzAsNCxbMCwxLCJcXG1hdGhjYWx7Q31cXHRpbWVzIChOKFxcbWF0aHNme0Zpbn1fXFxhc3QpX3tcXG1hdGhybXthY3R9fSlebiJdLFsyLDEsIlxcbWF0aGNhbHtDfVxcdGltZXMgXFxvcGVyYXRvcm5hbWV7RW52fShOKFxcbWF0aHNme0Zpbn1fXFxhc3QpKV5cXG90aW1lcyBfe1xcbGFuZ2xlIG4gXFxyYW5nbGV9Il0sWzIsMCwiXFxvcGVyYXRvcm5hbWV7RW52fShcXG1hdGhjYWx7TX0pXlxcb3RpbWVzIF97XFxsYW5nbGUgbiBcXHJhbmdsZX0iXSxbMCwwLCIoXFxtYXRoY2Fse019Xlxcb3RpbWVzIF97XFxtYXRocm17YWN0fX0pXm5cXHRpbWVzIF97XFxtYXRoY2Fse0N9Xm59XFxtYXRoY2Fse0N9Il0sWzMsMF0sWzEsMCwiXFxvcGVyYXRvcm5hbWV7aWR9X3tcXG1hdGhjYWx7Q319XFx0aW1lcyhcXHJob15pXyEpX3tpPTF9Xm4iXSxbMiwxXSxbMiwzLCIoXFxyaG9eaV8hKV97aT0xfV5uIiwyXV0=
\[\begin{tikzcd}
	{(\mathcal{M}^\otimes _{\mathrm{act}})^n\times _{\mathcal{C}^n}\mathcal{C}} && {\operatorname{Env}(\mathcal{M})^\otimes _{\langle n \rangle}} \\
	{\mathcal{C}\times (N(\mathsf{Fin}_\ast)_{\mathrm{act}})^n} && {\mathcal{C}\times \operatorname{Env}(N(\mathsf{Fin}_\ast))^\otimes _{\langle n \rangle}}
	\arrow[from=1-1, to=2-1]
	\arrow["{\operatorname{id}_{\mathcal{C}}\times(\rho^i_!)_{i=1}^n}", from=2-3, to=2-1]
	\arrow[from=1-3, to=2-3]
	\arrow["{(\rho^i_!)_{i=1}^n}"', from=1-3, to=1-1]
\end{tikzcd}\]is commutative. Now the functor $\pr{\rho_{!}^{i}}_{i=1}^{n}:\Env\pr{N\pr{\Fin_{\ast}}}_{\inp n}^{\t}\to\pr{\pr{N\pr{\Fin_{\ast}}}_{\act}}^{n}$
is a trivial fibration, and it has a section $\phi:\pr{N\pr{\Fin_{\ast}}_{\act}}^{n}\to\Env\pr{N\pr{\Fin_{\ast}}}_{\inp n}^{\t}$
given by 
\[
\pr{\inp{k_{i}}}_{i=1}^{n}\mapsto\pr{\Vee_{i=1}^{n}\inp{k_{i}}\to\Vee_{i=1}^{n}\inp 1=\inp n}.
\]
Applying Proposition \ref{prop:inverting_weak_equiv_of_pairs} to
the commutative diagram (\ref{d1}) and the section $\id_{\cal C}\times\phi$,
we can find a functor $\Phi:\pr{\cal M_{\act}^{\t}}^{n}\to\Env\pr{\cal M}_{\inp n}^{\t}$
with the following properties:
\begin{itemize}
\item The diagram % https://q.uiver.app/?q=WzAsNCxbMCwxLCJcXG1hdGhjYWx7Q31cXHRpbWVzIChOKFxcbWF0aHNme0Zpbn1fXFxhc3QpX3tcXG1hdGhybXthY3R9fSlebiJdLFsyLDEsIlxcbWF0aGNhbHtDfVxcdGltZXMgXFxvcGVyYXRvcm5hbWV7RW52fShOKFxcbWF0aHNme0Zpbn1fXFxhc3QpKV5cXG90aW1lcyBfe1xcbGFuZ2xlIG4gXFxyYW5nbGV9Il0sWzIsMCwiXFxvcGVyYXRvcm5hbWV7RW52fShcXG1hdGhjYWx7TX0pXlxcb3RpbWVzIF97XFxsYW5nbGUgbiBcXHJhbmdsZX0iXSxbMCwwLCIoXFxtYXRoY2Fse019Xlxcb3RpbWVzIF97XFxtYXRocm17YWN0fX0pXm5cXHRpbWVzIF97XFxtYXRoY2Fse0N9Xm59XFxtYXRoY2Fse0N9Il0sWzMsMF0sWzAsMSwiXFxvcGVyYXRvcm5hbWV7aWR9X3tcXG1hdGhjYWx7Q319XFx0aW1lc1xccGhpIiwyXSxbMiwxXSxbMywyLCJcXFBoaSJdXQ==
\[\begin{tikzcd}
	{(\mathcal{M}^\otimes _{\mathrm{act}})^n\times _{\mathcal{C}^n}\mathcal{C}} && {\operatorname{Env}(\mathcal{M})^\otimes _{\langle n \rangle}} \\
	{\mathcal{C}\times (N(\mathsf{Fin}_\ast)_{\mathrm{act}})^n} && {\mathcal{C}\times \operatorname{Env}(N(\mathsf{Fin}_\ast))^\otimes _{\langle n \rangle}}
	\arrow[from=1-1, to=2-1]
	\arrow["{\operatorname{id}_{\mathcal{C}}\times\phi}"', from=2-1, to=2-3]
	\arrow[from=1-3, to=2-3]
	\arrow["\Phi", from=1-1, to=1-3]
\end{tikzcd}\]is commutative. 
\item The composite $\pr{\rho_{!}^{i}}_{1\leq i\leq n}\circ\Phi$ is naturally
equivalent over $\cal C\times\pr{N\pr{\Fin_{\ast}}_{\act}}^{n}$ to
the identity functor. 
\end{itemize}
We will construct the functor $\bigoplus_{1\leq i\leq n}$ as the
composite
\[
\pr{\cal M_{\act}^{\t}}^{n}\times_{\cal C^{n}}\cal C\xrightarrow{\Phi}\Env\pr{\cal M}_{\inp n}^{\t}\xrightarrow{\text{forget}}\cal M^{\t}.
\]

Next, we show that, with our construction of the functor $\bigoplus_{1\leq i\leq n}$,
there is an inert natural transformation $\widetilde{h}_{i}:\pr{\pr{\cal M_{\act}^{\t}}^{n}\times_{\cal C^{n}}\cal C}\times\Delta^{1}\to\cal M^{\t}$
rendering the diagram in the statement commutative. Since $\opn{pr}_{i}$
is naturally equivalent over $\cal C\times N\pr{\Fin_{\ast}}$ to
the composite $\rho_{!}^{i}\Phi$, it suffices (by \cite[Proposition A.2.3.1]{HTT})
to find an inert natural transformation rendering the diagram % https://q.uiver.app/?q=WzAsNSxbMCwwLCIoKFxcbWF0aGNhbHtNfV5cXG90aW1lc197XFxtYXRocm17YWN0fX0pXm5cXHRpbWVzIF97XFxtYXRoY2Fse0N9Xm59XFxtYXRoY2Fse0N9KVxcdGltZXMgXFxwYXJ0aWFsXFxEZWx0YV4xIl0sWzAsMSwiKChcXG1hdGhjYWx7TX1eXFxvdGltZXNfe1xcbWF0aHJte2FjdH19KV5uXFx0aW1lcyBfe1xcbWF0aGNhbHtDfV5ufVxcbWF0aGNhbHtDfSlcXHRpbWVzIFxcRGVsdGFeMSJdLFsxLDEsIlxcbWF0aGNhbHtDfVxcdGltZXMgKE4oXFxtYXRoc2Z7RmlufV9cXGFzdClfe1xcbWF0aHJte2FjdH19KV5uXFx0aW1lcyBcXERlbHRhXjEiXSxbMiwxLCJcXG1hdGhjYWx7Q31cXHRpbWVzIE4oXFxtYXRoc2Z7RmlufV9cXGFzdCkiXSxbMiwwLCJcXG1hdGhjYWx7TX1eXFxvdGltZXMgIl0sWzAsMV0sWzEsMl0sWzIsMywiXFxvcGVyYXRvcm5hbWV7aWR9X3tcXG1hdGhjYWx7Q319XFx0aW1lcyBoX2kiLDJdLFs0LDMsInAiXSxbMCw0LCIoXFxiaWdvcGx1c197aT0xfV5uLFxccmhvXmlfIVxcUGhpICkiXSxbMSw0LCIiLDIseyJzdHlsZSI6eyJib2R5Ijp7Im5hbWUiOiJkYXNoZWQifX19XV0=
\[\begin{tikzcd}
	{((\mathcal{M}^\otimes_{\mathrm{act}})^n\times _{\mathcal{C}^n}\mathcal{C})\times \partial\Delta^1} && {\mathcal{M}^\otimes } \\
	{((\mathcal{M}^\otimes_{\mathrm{act}})^n\times _{\mathcal{C}^n}\mathcal{C})\times \Delta^1} & {\mathcal{C}\times (N(\mathsf{Fin}_\ast)_{\mathrm{act}})^n\times \Delta^1} & {\mathcal{C}\times N(\mathsf{Fin}_\ast)}
	\arrow[from=1-1, to=2-1]
	\arrow[from=2-1, to=2-2]
	\arrow["{\operatorname{id}_{\mathcal{C}}\times h_i}"', from=2-2, to=2-3]
	\arrow["p", from=1-3, to=2-3]
	\arrow["{(\bigoplus_{i=1}^n,\rho^i_!\Phi )}", from=1-1, to=1-3]
	\arrow[dashed, from=2-1, to=1-3]
\end{tikzcd}\]commutative. The outer rectangle is equal to that of the diagram % https://q.uiver.app/?q=WzAsNyxbMCwwLCIoKFxcbWF0aGNhbHtNfV5cXG90aW1lc197XFxtYXRocm17YWN0fX0pXm5cXHRpbWVzIF97XFxtYXRoY2Fse0N9Xm59XFxtYXRoY2Fse0N9KVxcdGltZXMgXFxwYXJ0aWFsXFxEZWx0YV4xIl0sWzAsMSwiKChcXG1hdGhjYWx7TX1eXFxvdGltZXNfe1xcbWF0aHJte2FjdH19KV5uXFx0aW1lc197XFxtYXRoY2Fse0N9Xm59XFxtYXRoY2Fse0N9KVxcdGltZXMgIFxcRGVsdGFeMSJdLFsyLDEsIlxcbWF0aGNhbHtDfVxcdGltZXMgXFxvcGVyYXRvcm5hbWV7RW52fShOKFxcbWF0aHNme0Zpbn1fXFxhc3QpKV5cXG90aW1lcyBfe1xcbGFuZ2xlIG5cXHJhbmdsZX1cXHRpbWVzIFxcRGVsdGFeMSJdLFszLDEsIlxcbWF0aGNhbHtDfVxcdGltZXMgTihcXG1hdGhzZntGaW59X1xcYXN0KSJdLFszLDAsIlxcbWF0aGNhbHtNfV5cXG90aW1lcyAiXSxbMSwxLCJcXG1hdGhjYWx7Q31cXHRpbWVzIFxcb3BlcmF0b3JuYW1le0Vudn0oXFxtYXRoY2Fse019KV5cXG90aW1lcyBfe1xcbGFuZ2xlIG5cXHJhbmdsZX1cXHRpbWVzIFxcRGVsdGFeMSJdLFsxLDAsIlxcbWF0aGNhbHtDfVxcdGltZXMgXFxvcGVyYXRvcm5hbWV7RW52fShcXG1hdGhjYWx7TX0pXlxcb3RpbWVzIF97XFxsYW5nbGUgbiBcXHJhbmdsZX1cXHRpbWVzIFxccGFydGlhbFxcRGVsdGFeMSJdLFswLDFdLFsyLDMsIlxcb3BlcmF0b3JuYW1le2lkfV97XFxtYXRoY2Fse0N9fVxcdGltZXMgZ19pIiwyXSxbNSwyXSxbMSw1LCJcXFBoaVxcdGltZXMgXFxvcGVyYXRvcm5hbWV7aWR9IiwyXSxbMCw2LCJcXFBoaVxcdGltZXMgXFxvcGVyYXRvcm5hbWV7aWR9Il0sWzYsNCwiKFxcdGV4dHtmb3JnZXR9LCBcXHJob15pXyEpIl0sWzYsNV0sWzQsM11d
\[\begin{tikzcd}[column sep=small, scale cd=.8]
	{((\mathcal{M}^\otimes_{\mathrm{act}})^n\times _{\mathcal{C}^n}\mathcal{C})\times \partial\Delta^1} & {\mathcal{C}\times \operatorname{Env}(\mathcal{M})^\otimes _{\langle n \rangle}\times \partial\Delta^1} && {\mathcal{M}^\otimes } \\
	{((\mathcal{M}^\otimes_{\mathrm{act}})^n\times_{\mathcal{C}^n}\mathcal{C})\times  \Delta^1} & {\mathcal{C}\times \operatorname{Env}(\mathcal{M})^\otimes _{\langle n\rangle}\times \Delta^1} & {\mathcal{C}\times \operatorname{Env}(N(\mathsf{Fin}_\ast))^\otimes _{\langle n\rangle}\times \Delta^1} & {\mathcal{C}\times N(\mathsf{Fin}_\ast)}
	\arrow[from=1-1, to=2-1]
	\arrow["{\operatorname{id}_{\mathcal{C}}\times g_i}"', from=2-3, to=2-4]
	\arrow[from=2-2, to=2-3]
	\arrow["{\Phi\times \operatorname{id}}"', from=2-1, to=2-2]
	\arrow["{\Phi\times \operatorname{id}}", from=1-1, to=1-2]
	\arrow["{(\text{forget}, \rho^i_!)}", from=1-2, to=1-4]
	\arrow[from=1-2, to=2-2]
	\arrow[from=1-4, to=2-4]
\end{tikzcd}\]so the existence of the desired filler follows from the definition
of $\rho_{!}^{i}$.
\end{proof}
%
\begin{cor}
[Universal Property of the Direct Sum Functor]\label{cor:oplus_p-limit}Let
$\cal C$ be an $\infty$-category and let $p:\cal M^{\t}\to\cal C\times N\pr{\Fin_{\ast}}$
be a $\cal C$-family of $\infty$-operads. Let $q:K\to\cal C$ be
an object of $\SS/\cal C$, let $n\geq1$, and let $f,g_{1},\dots,g_{n}:K\to\cal M_{\act}^{\t}$
be diagrams. Assume the following:
\begin{enumerate}
\item For each $1\leq i\leq n$, there is an inert natural transformation
$\alpha_{i}:f\to g_{i}$ over $\cal C$.
\item For each vertex $v$ in $K$, the maps $\alpha_{i}$ give rise to
a $p$-limit cone $\{f\pr v\to g_{i}\pr v\}_{1\leq i\le n}$. 
\end{enumerate}
Then $f$ is naturally equivalent over $\cal C$ to the composite
\[
G:K\xrightarrow{\pr{g_{i}}_{i=1}^{n}}\pr{\cal M_{\act}^{\t}}^{n}\times_{\cal C^{n}}\cal C\xrightarrow{\bigoplus_{1\leq i\leq n}}\cal M_{\act}^{\t}.
\]
\end{cor}
\begin{proof}
For each vertex $v$ in $K$, write $p_{2}\pr{g_{i}\pr v}=\inp{m_{i}\pr v}$
and $p_{2}\pr{f\pr v}=\inp{m\pr v}$. Let $\eta:K\times\Delta^{1}\to N\pr{\Fin_{\ast}}$
denote the natural equivalence which satisfies the following conditions: 
\begin{itemize}
\item The restriction $\eta\vert K\times\{0\}$ is given by $K\xrightarrow{\pr{p_{2}g_{i}}_{i}}N\pr{\Fin_{\ast}}_{\act}^{n}\xrightarrow{\Vee_{i=1}^{n}}N\pr{\Fin_{\ast}}$.
\item The restriction $\eta\vert K\times\{1\}$ is given by $p_{2}f$.
\item For each vertex $v\in K$ and $1\leq i\leq n$, the composite
\[
\inp{m\pr v}\xrightarrow{\eta}\inp{m\pr v}\xrightarrow{p_{2}\alpha_{i}}\inp{m_{i}\pr v}
\]
is defined as follows: Given $k\in\inp{m\pr v}^{\circ}$, the integer
$\sum_{j<i}m_{j}\pr v+k\in\inp{m\pr v}$ is mapped to $k$. The remaining
elements are mapped to the base point of $\inp{m_{i}\pr v}$.
\end{itemize}
Using the fact that $p$ is a categorical equivalence, we can find
a functor $f':K\to\cal M^{\t}$ and a natural equivalence $f'\xrightarrow{\simeq}f$
which lifts the natural equivalence $\pr{q\circ\opn{pr}_{1},\eta}:K\times\Delta^{1}\to\cal C\times N\pr{\Fin_{\ast}}$.
Replacing $f$ by $f'$ if necessary, we may assume that the diagram
% https://q.uiver.app/?q=WzAsNCxbMCwwLCJLXFx0aW1lcyBcXERlbHRhXjEiXSxbMSwwLCJcXG1hdGhjYWx7TX1eXFxvdGltZXMgIl0sWzEsMSwiXFxtYXRoY2Fse0N9XFx0aW1lcyBOKFxcbWF0aHNme0Zpbn1fXFxhc3QpIl0sWzAsMSwiKFxcbWF0aGNhbHtDfVxcdGltZXMgKE4oXFxtYXRoc2Z7RmlufV9cXGFzdClfe1xcbWF0aHJte2FjdH19KV5uKVxcdGltZXMgXFxEZWx0YV4xIl0sWzAsMSwiXFxhbHBoYV9pIl0sWzEsMiwicCJdLFszLDIsIlxcb3BlcmF0b3JuYW1le2lkfV97XFxtYXRoY2Fse0N9fVxcdGltZXMgaF9pIiwyXSxbMCwzLCIocSwocF8yZ19qKV9qKVxcdGltZXMgXFxvcGVyYXRvcm5hbWV7aWR9IiwyXV0=
\[\begin{tikzcd}
	{K\times \Delta^1} & {\mathcal{M}^\otimes } \\
	{(\mathcal{C}\times (N(\mathsf{Fin}_\ast)_{\mathrm{act}})^n)\times \Delta^1} & {\mathcal{C}\times N(\mathsf{Fin}_\ast)}
	\arrow["{\alpha_i}", from=1-1, to=1-2]
	\arrow["p", from=1-2, to=2-2]
	\arrow["{\operatorname{id}_{\mathcal{C}}\times h_i}"', from=2-1, to=2-2]
	\arrow["{(q,(p_2g_j)_j)\times \operatorname{id}}"', from=1-1, to=2-1]
\end{tikzcd}\]commutes for each $1\leq i\leq n$. Since relative limits in functor
categories can be formed objectwise \cite[\href{https://kerodon.net/tag/02XK}{Tag 02XK}]{kerodon},
the morphisms $\alpha_{i}:f\to g_{i}$ in $\Fun\pr{K,\cal M^{\t}}$
form a $\Fun\pr{K,p}$-limit cone. Moreover, since relative limits
are stable under pullbacks \cite[Proposition 4.3.1.5]{HTT}, we deduce
that the morphisms $\{\alpha_{i}\}_{1\leq i\leq n}$ of $\Fun_{\cal C}\pr{K,\cal M^{\t}}$
form a $\Fun_{\cal C}\pr{K,p}$-limit cone. 

Now using Proposition \ref{prop:oplus_p-limit}, we can construct
the functor $\bigoplus_{1\leq i\leq n}:\pr{\cal M_{\act}^{\t}}^{n}\times_{\cal C^{n}}\cal C\to\cal M^{\t}$
so that there is an inert natural transformation $\widetilde{h}_{i}:\bigoplus_{1\leq i\leq n}\to\opn{pr}_{i}$
which lifts the composite $\pr{\cal M_{\act}^{\t}}^{n}\times_{\cal C^{n}}\cal C\to\cal C\times N\pr{\Fin_{\ast}}_{\act}^{n}\times\Delta^{1}\xrightarrow{\id_{\cal C}\times h_{i}}\cal C\times N\pr{\Fin_{\ast}}$.
Again, the induced natural transformations $\{G\to g_{i}\}_{1\leq i\leq n}$
form a $\Fun_{\cal C}\pr{K,p}$-limit cone covering the same cone
as $\{\alpha_{i}\}$. Thus we obtain the desired equivalence $G\xrightarrow{\simeq}f$
in $\Fun_{\cal C}\pr{K,\cal M^{\t}}$.
\end{proof}
The above corollary admits the following further corollary, which
we shall use in the next section.
\begin{cor}
\label{cor:3.1.1.8}Let $q:\cal C^{\t}\to\cal O^{\t}$ be a fibration
of $\infty$-operads. Let $K_{1},\dots,K_{n}$ be simplicial sets,
and let $K=\prod_{1\leq i\leq n}K_{i}$. Let $\overline{f}:K^{\rcone}\to\cal C^{\t}$
and $\{\overline{f_{i}}:K^{\rcone}\to\cal C^{\t}\}_{1\leq i\leq n}$
be active diagrams, and suppose there are inert natural transformations
$\{\overline{f}\to\overline{f}_{i}\circ\pr{\opn{pr}_{i}}^{\rcone}\}_{1\le i\leq n}$
such that for each vertex $v$ in $K^{\rcone}$, the induced map $\{\overline{f}\pr v\to\overline{f}_{i}\circ\pr{\opn{pr}_{i}}^{\rcone}\pr v\}_{1\le i\leq n}$
form a $q$-limit cone. If each $\overline{f_{i}}$ is an operadic
$q$-colimit diagram, so is $\overline{f}$.
\end{cor}
\begin{proof}
According to Corollary \ref{cor:oplus_p-limit}, the diagram $\overline{f}$
is naturally equivalent to the composite
\[
K^{\rcone}\to\prod_{1\leq i\leq n}\pr{K_{i}^{\rcone}}\xrightarrow{\prod_{1\leq i\leq n}\overline{f_{i}}}\pr{\cal C_{\act}^{\t}}^{n}\xrightarrow{\bigoplus_{1\leq i\leq n}}\cal C^{\t}.
\]
The claim is thus a consequence of \cite[Proposition 3.1.1.8]{HA}.
\end{proof}

\subsection{Direct Sum and Slice}

Let $\cal O^{\t}$ be an $\infty$-operad and let $Y\in\cal O^{\t}$
be an object. If $Y$ lies over an object $\inp n\in N\pr{\Fin_{\ast}}$
with $n\geq2$, then we can find objects $Y_{i}\in\cal O$ and an
equivalence $Y\simeq\bigoplus_{i=1}^{n}Y_{i}$. Given objects $X_{1},\dots,X_{n}\in\cal O^{\t}$
and active maps $f_{i}:X_{i}\to Y_{i}$, their direct sum $\bigoplus_{i=1}^{n}f_{i}:\bigoplus_{i=1}^{n}X_{i}\to\bigoplus_{i=1}^{n}Y_{i}\simeq Y$
is an active morphism with codomain $Y$. Conversely, given an active
morphism $f:X\to Y$ in $\cal O^{\t}$, we can write $f\simeq\bigoplus_{i=1}^{n}f_{i}$,
where $f_{i}$ is the active map obtained by factoring the composite
$X\to Y\to Y_{i}$ into an inert map followed by an active map. The
following proposition, which is the only result of this subsection,
asserts that this ``direct sum decomposition'' of morphisms is an
equivalence on the level of $\infty$-categories: 
\begin{prop}
\label{prop:directsum_slice_equiv}Let $\cal C$ be an $\infty$-category,
let $C\in\cal C$ be an object, and let $\cal M^{\t}\to\cal C\times N\pr{\Fin_{\ast}}$
be a $\cal C$-family of $\infty$-operads. For any integer $n\geq1$
and any objects $M_{1},\dots,M_{n}\in\cal M\times_{\cal C}\{C\}$,
the direct sum functor induces an equivalence of $\infty$-categories
\[
\pr{\pr{\cal M_{\act}^{\t}}^{n}\times_{\cal C^{n}}\cal C}_{/\pr{M_{1},\dots,M_{n}}}\xrightarrow{\simeq}\pr{\cal M_{\act}^{\t}}_{/\bigoplus_{i=1}^{n}M_{i}}.
\]
\end{prop}
\begin{proof}
Recall that the direct sum functor $\bigoplus_{i=1}^{n}$ is obtained
as the composite 
\[
\pr{\cal M_{\act}^{\t}}^{n}\times_{\cal C^{n}}\cal C\xrightarrow[\Phi]{\simeq}\Env\pr{\cal M}_{\inp n}^{\t}\xrightarrow{\text{forget}}\cal M_{\act}^{\t},
\]
where the map $\Phi$ is the inverse equivalence over $\cal C$ of
the functor $\pr{\rho_{!}^{i}}_{1\leq i\le n}:\Env\pr{\cal M}_{\inp n}^{\t}\xrightarrow{\simeq}\pr{\cal M_{\act}^{\t}}^{n}\times_{\cal C^{n}}\cal C$
. The functor $\Phi$ maps the object $\pr{M_{1},\dots,M_{n}}$ to
the object $\pr{\bigoplus_{i=1}^{n}M_{i},\alpha:\inp n\to\inp n}$,
where $\alpha$ is a bijection. It will therefore suffice to prove
that the functor 
\[
\theta:\pr{\Env\pr{\cal M}_{\inp n}^{\t}}_{/\pr{\bigoplus_{i=1}^{n}M_{i},\alpha}}\to\pr{\cal M_{\act}^{\t}}_{/\bigoplus_{i=1}^{n}M_{i}}
\]
is an equivalence of $\infty$-categories. Set $M=\bigoplus_{i=1}^{n}M_{i}$.
By definition, we have
\[
\pr{\Env\pr{\cal M}_{\inp n}^{\t}}_{/\pr{M,\alpha}}=\pr{\Act\pr{N\pr{\Fin_{\ast}}}_{\inp n}^{\t}}_{/\alpha}\times_{N\pr{\Fin_{\ast}}_{/\inp n}}\cal M_{/M}^{\t}.
\]
Since the evaluation at $0\in\Delta^{1}$ induces an isomorphism of
simplicial sets
\[
\pr{\Act\pr{N\pr{\Fin_{\ast}}}_{\inp n}^{\t}}_{/\alpha}\xrightarrow{\cong}\pr{N\pr{\Fin_{\ast}}_{\act}}_{/\inp n},
\]
we deduce that the functor $\theta$ is an \textit{isomorphism} of
simplicial sets.
\end{proof}

\section{\label{sec:FTOK}The Fundamental Theorem of Operadic Kan Extensions}

In this final section, we will provide complete details of the proof
of \cite[Theorem 3.1.2.3]{HA}, using tools and results in the preceding
sections.

\subsection{Recollection}

In this subsection, we recall the definition of operadic Kan extensions
and the statement of the fundamental theorem of operadic Kan extensions. 
\begin{defn}
\cite[Definition 3.1.2.2]{HA}Let $\cal M^{\t}\to N\pr{\Fin_{\ast}}\times\Delta^{1}$
be a $\Delta^{1}$-family of $\infty$-operads and let $q:\cal C^{\t}\to\cal O^{\t}$
be a fibration of $\infty$-operads. Set $\cal A^{\t}=\cal M^{\t}\times_{\Delta^{1}}\{0\}$,
$\cal B^{\t}=\cal M^{\t}\times_{\Delta^{1}}\{1\}$. A map $F:\cal M^{\t}\to\cal C^{\t}$
is called an \textbf{operadic $q$-left Kan extension of $F\vert\cal A^{\t}$
}if the following condition is satisfied for every object $B\in\cal B$:
\begin{itemize}
\item [($\ast$)]The composite map
\[
\pr{\pr{\cal M_{\act}^{\t}}_{/B}\times_{\Delta^{1}}\{0\}}^{\rcone}\to\pr{\cal M_{/B}^{\t}}^{\rcone}\to\cal M^{\t}\xrightarrow{\overline{F}}\cal C^{\t}
\]
is an operadic $q$-colimit diagram.
\end{itemize}
\end{defn}
Lurie imposes condition ($\ast$) to hold for every object $B\in\cal B^{\t}$
in \cite[Definition 3.1.2.2]{HA}. This is not a problem, because
of the following proposition, which seems to be implicitly used in
\cite{HA}.
\begin{prop}
\label{prop:operadic_Kan_equiv}Let $p:\cal M^{\t}\to N\pr{\Fin_{\ast}}\times\Delta^{1}$
be a correspondence from an $\infty$-operad $\cal A^{\t}$ to another
$\infty$-operad $\cal B^{\t}$. Let $q:\cal C^{\t}\to\cal O^{\t}$
be a fibration of $\infty$-operads and let $F:\cal M^{\t}\to\cal C^{\t}$
be a map of generalized $\infty$-operads. Suppose that $F$ is an
operadic $q$-left Kan extension of \textbf{$F\vert\cal A^{\t}$}.
Then for every object $B\in\cal B^{\t},$the map 
\[
\theta:\pr{\pr{\cal M_{\act}^{\t}}_{/B}\times_{\Delta^{1}}\{0\}}^{\rcone}\to\pr{\cal M_{/B}^{\t}}^{\rcone}\to\cal M^{\t}\xrightarrow{F}\cal C^{\t}
\]
is an operadic $q$-colimit diagram.
\end{prop}
\begin{proof}
Let $\inp n$ be the image of $B$ in $N\pr{\Fin_{\ast}}$. We consider
two cases, depending on whether or not $n$ is equal to $0$.

Suppose first that $n=0$. Then we have $\pr{\cal M_{\act}^{\t}}_{/B}=\pr{\cal M_{\inp 0}^{\t}}_{/B}$,
since there is no active map $\inp k\to\inp 0$ with $k\geq1$. Since
every object of $\cal M_{\inp 0}^{\t}$ is $p$-terminal, the functor
$\cal M_{\inp 0}^{\t}\to\Delta^{1}$ is a trivial fibration. It follows
that the functor $\pr{\cal M_{\act}^{\t}}_{/B}\times_{\Delta^{1}}\{0\}\to\Delta_{/1}^{1}\times_{\Delta^{1}}\{0\}\cong\Delta^{0}$
is also a trivial fibration. It will therefore suffice to show that
$F$ carries a morphism of the form $A\to B$ in $\cal M^{\t}$ with
$A\in\cal M_{\inp 0}^{\t}\times_{\Delta^{1}}\{0\}$ to an equivalence
in $\cal C^{\t}$. This is clear, since $\cal C_{\inp 0}^{\t}$ is
a contractible Kan complex.

If $n\geq1$, choose for each $1\leq i\leq n$ an inert map $B\to B_{i}$
in $\cal B^{\t}$ over $\rho^{i}:\inp n\to\inp 1$, and choose also
a direct sum functor $\bigoplus_{i=1}^{n}$ for $\cal M^{\t}$ and
$\cal C^{\t}$. Replacing $\bigoplus_{i=1}^{n}$ by a functor naturally
equivalent one, we may assume that $B=\bigoplus_{i=1}^{n}$. According
to Proposition \ref{prop:directsum_slice_equiv}, the direct sum functor
induces an equivalence of $\infty$-categories
\[
\phi:\pr{\pr{\cal M_{\act}^{\t}}^{n}\times_{\pr{\Delta^{1}}^{n}}\Delta^{1}}_{/\pr{B_{1},\dots,B_{n}}}\xrightarrow{\simeq}\pr{\cal M_{\act}^{\t}}_{/B}.
\]
There is also an isomorphism of simplicial sets
\[
\psi:\prod_{i=1}^{n}\pr{\cal M_{\act}^{\t}}_{/B_{i}}\times_{\Delta^{1}}\{0\}\cong\pr{\pr{\cal M_{\act}^{\t}}^{n}\times_{\pr{\Delta^{1}}^{n}}\Delta^{1}}_{/\pr{B_{1},\dots,B_{n}}}\times_{\Delta^{1}}\{0\}.
\]
It will therefore suffice to show that the composite $\theta\circ\phi^{\rcone}\circ\psi^{\rcone}$
is an operadic $q$-colimit diagram. For this, we consider the diagram
% https://q.uiver.app/?q=WzAsOSxbMCwwLCIoXFxwcm9kX3tpPTF9Xm4oXFxtYXRoY2Fse019Xlxcb3RpbWVzIF97XFxtYXRocm17YWN0fX0pX3svQl9pfVxcdGltZXMgX3tcXERlbHRhXjF9XFx7MFxcfSleXFx0cmlhbmdsZXJpZ2h0Il0sWzEsMCwiKCgoXFxtYXRoY2Fse019Xlxcb3RpbWVzIF97XFxtYXRocm17YWN0fX0pXm5cXHRpbWVzX3soXFxEZWx0YV4xKV5ufVxcRGVsdGFeMSlfey8oQl8xLFxcZG90cyAsQl9uKX1cXHRpbWVzX3tcXERlbHRhXjF9XFx7MFxcfSleXFx0cmlhbmdsZXJpZ2h0Il0sWzAsMSwiXFxwcm9kX3tpPTF9Xm4oKFxcbWF0aGNhbHtNfV5cXG90aW1lcyBfe1xcbWF0aHJte2FjdH19KV97L0JfaX1cXHRpbWVzIF97XFxEZWx0YV4xfVxcezBcXH0pXlxcdHJpYW5nbGVyaWdodCJdLFsxLDEsIihcXG1hdGhjYWx7TX1eXFxvdGltZXMgX3tcXG1hdGhybXthY3R9fSleblxcdGltZXNfeyhcXERlbHRhXjEpXm59XFxEZWx0YV4xIl0sWzIsMCwiKChcXG1hdGhjYWx7TX1eXFxvdGltZXMgX3tcXG1hdGhybXthY3R9fSlfey9CfVxcdGltZXMgX3tcXERlbHRhXjF9XFx7MFxcfSleXFx0cmlhbmdsZXJpZ2h0Il0sWzIsMSwiXFxtYXRoY2Fse019Xlxcb3RpbWVzIF97XFxtYXRocm17YWN0fX0iXSxbMCwyLCIoXFxtYXRoY2Fse019Xlxcb3RpbWVzIF97XFxtYXRocm17YWN0fX0pXm4iXSxbMSwyLCIoXFxtYXRoY2Fse0N9Xlxcb3RpbWVzIF97XFxtYXRocm17YWN0fX0pXm4iXSxbMiwyLCJcXG1hdGhjYWx7Q31eXFxvdGltZXMgX3tcXG1hdGhybXthY3R9fSJdLFswLDJdLFsxLDNdLFswLDEsIlxcY29uZyJdLFsxLDQsIlxcc2ltZXEiXSxbMCwxLCJcXHBzaV5cXHRyaWFuZ2xlcmlnaHQiLDJdLFsxLDQsIlxccGhpXlxcdHJpYW5nbGVyaWdodCIsMl0sWzQsNV0sWzMsNSwiXFxiaWdvcGx1c197aT0xfV5uIiwyXSxbMyw2LCIiLDIseyJzdHlsZSI6eyJ0YWlsIjp7Im5hbWUiOiJob29rIiwic2lkZSI6ImJvdHRvbSJ9fX1dLFsyLDZdLFs2LDcsIlxccHJvZF97aT0xfV5uRiIsMl0sWzMsNywiRyJdLFs1LDgsIkYiXSxbNyw4LCJcXGJpZ29wbHVzX3tpPTF9Xm4iLDJdXQ==
\[\begin{tikzcd}[column sep=small, scale cd=.8]
	{(\prod_{i=1}^n(\mathcal{M}^\otimes _{\mathrm{act}})_{/B_i}\times _{\Delta^1}\{0\})^\triangleright} & {(((\mathcal{M}^\otimes _{\mathrm{act}})^n\times_{(\Delta^1)^n}\Delta^1)_{/(B_1,\dots ,B_n)}\times_{\Delta^1}\{0\})^\triangleright} & {((\mathcal{M}^\otimes _{\mathrm{act}})_{/B}\times _{\Delta^1}\{0\})^\triangleright} \\
	{\prod_{i=1}^n((\mathcal{M}^\otimes _{\mathrm{act}})_{/B_i}\times _{\Delta^1}\{0\})^\triangleright} & {(\mathcal{M}^\otimes _{\mathrm{act}})^n\times_{(\Delta^1)^n}\Delta^1} & {\mathcal{M}^\otimes _{\mathrm{act}}} \\
	{(\mathcal{M}^\otimes _{\mathrm{act}})^n} & {(\mathcal{C}^\otimes _{\mathrm{act}})^n} & {\mathcal{C}^\otimes _{\mathrm{act}}.}
	\arrow[from=1-1, to=2-1]
	\arrow[from=1-2, to=2-2]
	\arrow["\cong", from=1-1, to=1-2]
	\arrow["\simeq", from=1-2, to=1-3]
	\arrow["{\psi^\triangleright}"', from=1-1, to=1-2]
	\arrow["{\phi^\triangleright}"', from=1-2, to=1-3]
	\arrow[from=1-3, to=2-3]
	\arrow["{\bigoplus_{i=1}^n}"', from=2-2, to=2-3]
	\arrow[hook', from=2-2, to=3-1]
	\arrow[from=2-1, to=3-1]
	\arrow["{\prod_{i=1}^nF}"', from=3-1, to=3-2]
	\arrow["G", from=2-2, to=3-2]
	\arrow["F", from=2-3, to=3-3]
	\arrow["{\bigoplus_{i=1}^n}"', from=3-2, to=3-3]
\end{tikzcd}\]Here the map $G$ is the restriction of the map $\prod_{i=1}^{n}F$.
The left column and the top right square are commutative. The bottom
right square commutes up to natural equivalence by Proposition \ref{prop:oplus_p-limit}
and Corollary \ref{cor:oplus_p-limit}. Hence the diagram $\theta\circ\phi^{\rcone}\circ\psi^{\rcone}$
is naturally equivalent to the composite
\[
\theta':\pr{\prod_{i=1}^{n}\pr{\cal M_{\act}^{\t}}_{/B_{i}}\times_{\Delta^{1}}\{0\}}^{\rcone}\to\prod_{i=1}^{n}\pr{\pr{\cal M_{\act}^{\t}}_{/B_{i}}\times_{\Delta^{1}}\{0\}}^{\rcone}\to\pr{\cal C_{\act}^{\t}}^{n}\xrightarrow{\bigoplus_{i=1}^{n}}\cal C_{\act}^{\t}.
\]
The diagram $\theta'$ is an operadic $q$-colimit diagram by \cite[Proposition 3.1.1.8]{HA}.
\end{proof}
We can now state the fundamental theorem of opeardic Kan extensions.
\begin{thm}
\cite[Theorem 3.1.2.3]{HA}\label{thm:3.1.2.3}Let $p:\cal M^{\t}\to N\pr{\Fin_{\ast}}\times\Delta^{n}$
be a $\Delta^{n}$-family of generalized $\infty$-operads, and let
$q:\cal C^{\t}\to\cal O^{\t}$ be a fibration of $\infty$-operads.
Consider a commutative diagram% https://q.uiver.app/?q=WzAsNCxbMCwwLCJcXG1hdGhjYWx7TX1eXFxvdGltZXMgXFx0aW1lcyBfe1xcRGVsdGFebn1cXExhbWJkYSBebl8wIl0sWzAsMSwiXFxtYXRoY2Fse019Xlxcb3RpbWVzICJdLFsxLDAsIlxcbWF0aGNhbHtDfV5cXG90aW1lcyAiXSxbMSwxLCJcXG1hdGhjYWx7T31eXFxvdGltZXMgLiJdLFswLDEsIiIsMCx7InN0eWxlIjp7InRhaWwiOnsibmFtZSI6Imhvb2siLCJzaWRlIjoidG9wIn19fV0sWzAsMiwiZl8wIl0sWzIsMywicSJdLFsxLDMsImciLDJdLFsxLDIsIiIsMCx7InN0eWxlIjp7ImJvZHkiOnsibmFtZSI6ImRhc2hlZCJ9fX1dXQ==
\[\begin{tikzcd}
	{\mathcal{M}^\otimes \times _{\Delta^n}\Lambda ^n_0} & {\mathcal{C}^\otimes } \\
	{\mathcal{M}^\otimes } & {\mathcal{O}^\otimes}
	\arrow[hook, from=1-1, to=2-1]
	\arrow["{f_0}", from=1-1, to=1-2]
	\arrow["q", from=1-2, to=2-2]
	\arrow["g"', from=2-1, to=2-2]
	\arrow[dashed, from=2-1, to=1-2]
\end{tikzcd}\]of simplicial sets. Suppose that for each vertex $i$ in $\Lambda_{0}^{n}$
and each vertex $j$ in $\Delta^{n}$, the induced maps $\cal M^{\t}\times_{\Delta^{n}}\{v\}\to\cal C^{\t}$
and $\cal M^{\t}\times_{\Delta^{n}}\{j\}\to\cal O^{\t}$ are morphisms
of $\infty$-operads.
\begin{itemize}
\item [(A)]If $n=1$, the following conditions are equivalent:
\begin{itemize}
\item [(a)]There is a dashed filler which is an operadic $q$-left Kan
extension of $f_{0}$.
\item [(b)]For each object $B\in\cal M^{\t}\times_{\Delta^{1}}\{1\}$,
the diagram
\[
\{0\}\times_{\Delta^{1}}\pr{\pr{\cal M_{\act}^{\t}}_{/B}}\to\{0\}\times_{\Delta^{1}}\cal M^{\t}\xrightarrow{f_{0}}\cal C^{\t}
\]
admits an operadic $q$-colimit cone which lifts the map
\[
\pr{\{0\}\times_{\Delta^{1}}\pr{\pr{\cal M_{\act}^{\t}}_{/B}}}^{\rcone}\to\pr{\cal M_{/B}^{\t}}^{\rcone}\to\cal M^{\t}\xrightarrow{g}\cal O^{\t}.
\]
\end{itemize}
\item [(B)]If $n>1$ and the restriction $f_{0}\vert\cal M^{\t}\times_{\Delta^{n}}\Delta^{\{0,1\}}$
is an operadic $q$-left Kan extension of $f_{0}\vert\cal M^{\t}\times_{\Delta^{n}}\{0\}$,
then there is a dashed arrow rendering the diagram commutative.
\end{itemize}
\end{thm}
The rest of this note is devoted to elaborating Lurie's proof of the
above theorem.

\subsection{Preliminary Results}

In this subsection, we collect some results which will be used in
the proof of Theorem \ref{prop:operadic_Kan_equiv}. We recommend
that the reader skip to the next subsection and refer to this subsection
as needed.
\begin{prop}
\label{prop:relative_terminal_obj_in_slice}Let $p:\cal C\to\cal D$
be an inner fibration of $\infty$-categories, $K$ a simplicial set,
and $\overline{f}:K^{\rcone}\to\cal C$ a diagram. Set $f=\overline{f}\vert K$
and let $q$ denote the functor $\cal C_{f/}\to\cal D_{pf/}$. If
$\overline{f}$ maps the cone point to a $p$-terminal object, then
$\overline{f}$ is $q$-terminal.
\end{prop}
\begin{proof}
We must show that the map 
\[
\pr{\cal C_{f/}}_{/\overline{f}}\to\cal C_{f/}\times_{\cal D_{pf/}}\pr{\cal D_{pf/}}_{/q\overline{f}}
\]
is a trivial fibration. Let $C=\overline{f}\pr{\infty}$. Given a
monomorphism $A\to B$ of simplicial sets, a lifting problem on the
left hand side corresponds under adjunction to a lifting problem on
the right hand side:% https://q.uiver.app/?q=WzAsOCxbMCwwLCJBIl0sWzAsMSwiQiJdLFsxLDAsIihcXG1hdGhjYWx7Q31fe2YvfSlfey9cXG92ZXJsaW5le2Z9fSJdLFsxLDEsIlxcbWF0aGNhbHtDfV97Zi99XFx0aW1lcyBfe1xcbWF0aGNhbHtEfV97cGYvfX0oXFxtYXRoY2Fse0R9X3twZi99KV97L3FcXG92ZXJsaW5le2Z9fSJdLFsyLDAsIktcXHN0YXIgQSJdLFsyLDEsIktcXHN0YXIgQiJdLFszLDAsIlxcbWF0aGNhbHtDfV97L0N9Il0sWzMsMSwiXFxtYXRoY2Fse0N9XFx0aW1lcyBfe1xcbWF0aGNhbHtEfX1cXG1hdGhjYWx7RH1fey9wQ30iXSxbMCwyXSxbMiwzXSxbMCwxXSxbMSwzXSxbMSwyLCIiLDEseyJzdHlsZSI6eyJib2R5Ijp7Im5hbWUiOiJkYXNoZWQifX19XSxbNCw1XSxbNCw2XSxbNiw3XSxbNSw3XSxbNSw2LCIiLDEseyJzdHlsZSI6eyJib2R5Ijp7Im5hbWUiOiJkYXNoZWQifX19XV0=
\[\begin{tikzcd}
	A & {(\mathcal{C}_{f/})_{/\overline{f}}} & {K\star A} & {\mathcal{C}_{/C}} \\
	B & {\mathcal{C}_{f/}\times _{\mathcal{D}_{pf/}}(\mathcal{D}_{pf/})_{/q\overline{f}}} & {K\star B} & {\mathcal{C}\times _{\mathcal{D}}\mathcal{D}_{/pC}.}
	\arrow[from=1-1, to=1-2]
	\arrow[from=1-2, to=2-2]
	\arrow[from=1-1, to=2-1]
	\arrow[from=2-1, to=2-2]
	\arrow[dashed, from=2-1, to=1-2]
	\arrow[from=1-3, to=2-3]
	\arrow[from=1-3, to=1-4]
	\arrow[from=1-4, to=2-4]
	\arrow[from=2-3, to=2-4]
	\arrow[dashed, from=2-3, to=1-4]
\end{tikzcd}\]The right hand lifting problem is solvable because $C$ is $p$-terminal.
\end{proof}
\begin{cor}
\label{cor:Step2}Let $q:\cal C^{\t}\to\cal O^{\t}$ be a fibration
of $\infty$-operads, let $K$ be a simplicial set, and let $n\geq0$.
Consider a lifting problem % https://q.uiver.app/?q=WzAsNCxbMCwwLCJLXFxzdGFyIFxccGFydGlhbFxcRGVsdGEgXm4iXSxbMCwxLCJLXFxzdGFyIFxcRGVsdGEgXm4iXSxbMSwwLCJcXG1hdGhjYWx7Q31eXFxvdGltZXMgIl0sWzEsMSwiXFxtYXRoY2Fse099Xlxcb3RpbWVzICJdLFswLDEsIiIsMCx7InN0eWxlIjp7InRhaWwiOnsibmFtZSI6Imhvb2siLCJzaWRlIjoidG9wIn19fV0sWzIsMywicSJdLFswLDIsImYiXSxbMSwzLCJnIiwyXSxbMSwyLCIiLDEseyJzdHlsZSI6eyJib2R5Ijp7Im5hbWUiOiJkYXNoZWQifX19XV0=
\[\begin{tikzcd}
	{K\star \partial\Delta ^n} & {\mathcal{C}^\otimes } \\
	{K\star \Delta ^n} & {\mathcal{O}^\otimes.}
	\arrow[hook, from=1-1, to=2-1]
	\arrow["q", from=1-2, to=2-2]
	\arrow["f", from=1-1, to=1-2]
	\arrow["g"', from=2-1, to=2-2]
	\arrow[dashed, from=2-1, to=1-2]
\end{tikzcd}\]If the map $g$ maps the terminal vertex of $\Delta^{n}$ to an object
in $\cal O_{\inp 0}^{\t}$, then the lifting problem admits a solution.
\end{cor}
\begin{proof}
If $n=0$, find an object $C\in\cal C_{\inp 0}^{\t}$ which lies over
$g\pr 0$. Such an object exists because the functor $\cal C_{\inp 0}^{\t}\to\cal O_{\inp 0}^{\t}$
is a trivial fibration. The object $C$ is $q$-terminal, so the functor
\[
\cal C_{/C}^{\t}\to\cal O_{/q\pr C}^{\t}\times_{\cal O^{\t}}\cal C^{\t}
\]
is a trivial fibration. This implies the existence of the filler.

If $n>1$, then set $h=f\vert K$. We must solve a lifting problem
% https://q.uiver.app/?q=WzAsNCxbMCwwLCJcXHBhcnRpYWxcXERlbHRhIF5uIl0sWzAsMSwiXFxEZWx0YSBebiJdLFsxLDAsIlxcbWF0aGNhbHtDfV5cXG90aW1lcyBfe2gvfSJdLFsxLDEsIlxcbWF0aGNhbHtPfV5cXG90aW1lc197cWgvfSJdLFswLDEsIiIsMCx7InN0eWxlIjp7InRhaWwiOnsibmFtZSI6Imhvb2siLCJzaWRlIjoidG9wIn19fV0sWzIsMywicSciXSxbMCwyXSxbMSwzXSxbMSwyLCIiLDEseyJzdHlsZSI6eyJib2R5Ijp7Im5hbWUiOiJkYXNoZWQifX19XV0=
\[\begin{tikzcd}
	{\partial\Delta ^n} & {\mathcal{C}^\otimes _{h/}} \\
	{\Delta ^n} & {\mathcal{O}^\otimes_{qh/}.}
	\arrow[hook, from=1-1, to=2-1]
	\arrow["{q'}", from=1-2, to=2-2]
	\arrow[from=1-1, to=1-2]
	\arrow[from=2-1, to=2-2]
	\arrow[dashed, from=2-1, to=1-2]
\end{tikzcd}\]For this, it will suffice to show that the image of the vertex $n\in\partial\Delta^{n}$
under the top horizontal arrow is $q'$-terminal. This follows from
Proposition \ref{prop:relative_terminal_obj_in_slice}.
\end{proof}
\begin{lem}
\cite[Lemma 3.1.2.5]{HA}\label{lem:3.1.2.5}Let $\cal C$ be an $\infty$-category
and $\cal C^{0}\subset\cal C$ a full subcategory. Let $\sigma:\Delta^{n}\to\cal C$
be a nondegenerate simplex such that $\sigma\pr i\not\in\cal C^{0}$
for each $0\leq i\leq n$. Consider the following simplicial sets:
\begin{enumerate}
\item The simplicial subset $K\subset\cal C$ consisting of those simplices
$\tau:\Delta^{k}\star\Delta^{l}\to\cal C$, where $k,l\geq-1$, $\tau\vert\Delta^{k}$
factors through $\cal C^{0}$ and $\tau\vert\Delta^{l}$ factors through
$\sigma$.
\item The simplicial subset $K_{0}\subset\cal C$ consisting of those simplices
$\tau:\Delta^{k}\star\Delta^{l}\to\cal C$, where $k,l\geq-1$, $\tau\vert\Delta^{k}$
factors through $\cal C^{0}$ and $\tau\vert\Delta^{l}$ factors through
$\sigma\vert\partial\Delta^{n}$.
\item The simplicial subset $L\subset\cal C$ generated by the simplices
$\widetilde{\sigma}:\Delta^{k}\star\Delta^{n}\to\cal C$, where $k\geq-1$,
$\widetilde{\sigma}\vert\Delta^{k}$ factors through $\cal C^{0}$,
and $\widetilde{\sigma}\vert\Delta^{n}=\sigma$. 
\end{enumerate}
Then:
\begin{itemize}
\item [(a)]The map $\cal C_{/\sigma}^{0}\star\Delta^{n}\amalg_{\cal C_{/\sigma}^{0}\star\partial\Delta^{n}}K_{0}\to K_{0}\cup L$
is an isomorphism of simplicial sets.
\item [(b)]The inclusion
\[
K_{0}\cup L\subset K
\]
is a trivial cofibration in the Joyal model structure.
\end{itemize}
\end{lem}
\begin{proof}
We begin with (a). It is clear that the simplicial set $K_{0}\cup L$
is the union of $K_{0}$ and the image of the map $\cal C_{/\sigma}^{0}\star\Delta^{n}\to\cal C$.
Therefore, it will suffice to show that if a simplex $x$ of $\cal C_{/\sigma}^{0}\star\Delta^{n}$
is mapped to a simplex in $K_{0}$, then $x$ belongs to $\cal C_{/\sigma}^{0}\star\partial\Delta^{n}$.
The image of $x$ can be represented by a simplex of the form
\[
\Delta^{k}\star\Delta^{l}\xrightarrow{\id\star u}\Delta^{k}\star\Delta^{n}\xrightarrow{\widetilde{\sigma}}\cal C,
\]
where $k,l\geq-1$, $u:[l]\to[n]$ is a poset map, and $\widetilde{\sigma}$
is an extension of $\sigma$ which maps the vertices of $\Delta^{k}$
into $\cal C^{0}$. We wish to show that if $\sigma u:\Delta^{l}\to\cal C$
factors through $\sigma\vert\partial\Delta^{n}$, then $u$ factors
through $\partial\Delta^{n}$. This is clear, since $\sigma$ is nondegenerate.

Next we prove (b). For each $m\geq n$, let $X_{m}$ denote the simplicial
subset of $K$ consisting of the simplices in $K_{0}$ and the simplices
of the form
\[
\tau:\Delta^{k}\star\Delta^{m'}\to\cal C,
\]
where $k\geq-1$, $n\leq m'<m$, $\tau\vert\Delta^{k}$ factors through
$\cal C^{0}$, and $\tau\vert\Delta^{m'}$ is a degeneration of $\sigma$.
Then we have a filtration 
\[
K_{0}\cup L=X_{n}\subset X_{n+1}\subset\cdots
\]
and $K=\bigcup_{i\geq n}X_{i}$. To prove (b), it will suffice to
show that each inclusion $X_{i}\subset X_{i+1}$ is a trivial cofibration.

Let $\varepsilon:[m]\to[n]$ be a surjective poset map. We let $\cal J\pr{\varepsilon}\subset\Del_{[m]//[n]}$
denote the full subcategory spanned by the factorizations $\pr{\varepsilon',\varepsilon''}=[m]\xrightarrow{\varepsilon'}[m']\xrightarrow{\varepsilon''}[n]$
of $\varepsilon$ into two surjective poset maps, and let $\cal J_{0}\pr{\varepsilon}$
denote its full subcategory spanned by the object other than $\pr{\id,\varepsilon}$.
Let $A\pr{\varepsilon}\subset\cal C_{/\sigma\varepsilon}^{0}$ denote
the union of the image of the maps $\cal C_{/\sigma\varepsilon''}^{0}\to\cal C_{/\sigma\varepsilon''\varepsilon'}^{0}=\cal C_{/\sigma\varepsilon}^{0}$,
where $\pr{\varepsilon',\varepsilon''}$ ranges over the objects of
$\cal J_{0}\pr{\varepsilon}$. We consider the commutative diagram
% https://q.uiver.app/?q=WzAsNCxbMCwwLCJcXGNvcHJvZF97XFx2YXJlcHNpbG9uOlttXVxcdHdvaGVhZHJpZ2h0YXJyb3dbbl19KEEoXFx2YXJlcHNpbG9uKVxcc3RhciBcXERlbHRhIF5tKVxcY3VwKFxcbWF0aGNhbHtDfV4wX3svXFxzaWdtYVxcdmFyZXBzaWxvbn1cXHN0YXJcXHBhcnRpYWxcXERlbHRhXnttfSkiXSxbMCwxLCJYX20iXSxbMSwwLCJcXGNvcHJvZF97XFx2YXJlcHNpbG9uOlttXVxcdHdvaGVhZHJpZ2h0YXJyb3dbbl19XFxtYXRoY2Fse0N9XjBfey9cXHNpZ21hXFx2YXJlcHNpbG9ufVxcc3RhclxcRGVsdGFee219Il0sWzEsMSwiWF97bSsxfSJdLFswLDFdLFswLDJdLFsyLDNdLFsxLDNdXQ==
\[\begin{tikzcd}
	{\coprod_{\varepsilon:[m]\twoheadrightarrow[n]}(A(\varepsilon)\star \Delta ^m)\cup(\mathcal{C}^0_{/\sigma\varepsilon}\star\partial\Delta^{m})} & {\coprod_{\varepsilon:[m]\twoheadrightarrow[n]}\mathcal{C}^0_{/\sigma\varepsilon}\star\Delta^{m}} \\
	{X_m} & {X_{m+1}}
	\arrow[from=1-1, to=2-1]
	\arrow[from=1-1, to=1-2]
	\arrow[from=1-2, to=2-2]
	\arrow[from=2-1, to=2-2]
\end{tikzcd}\]where the coproducts are indexed by the surjective poset maps $[m]\epi[n]$.
To complete the proof, it suffices to prove the following:
\begin{itemize}
\item [(i)]The square is a pushout.
\item [(ii)]For each surjection $\varepsilon:[m]\to[n]$, the inclusion
\[
A\pr{\varepsilon}\star\Delta^{m}\cup\cal C_{/\sigma\varepsilon}^{0}\star\partial\Delta^{m}\to\cal C_{/\sigma\varepsilon}^{0}\star\Delta^{m}
\]
is a trivial cofibration.
\end{itemize}
We start with (i). Let $x$ be a simplex in $\cal C_{/\sigma\varepsilon}^{0}\star\Delta^{m}$
corresponding to a pair $\pr{\widetilde{\sigma\varepsilon}:\Delta^{k}\star\Delta^{m}\to\cal C,\,u:\Delta^{l}\to\Delta^{m}}$,
$k,l\geq-1$, $k+l+1\geq0$. Its image in $\cal C$ is the composition
\[
\alpha:\Delta^{k}\star\Delta^{l}\xrightarrow{\id\star u}\Delta^{k}\star\Delta^{m}\xrightarrow{\widetilde{\sigma\varepsilon}}\cal C.
\]
We wish to show that if $\alpha$ belongs to $X_{m}$, then $\pr{\widetilde{\sigma\varepsilon},u}$
belongs to $A\pr{\varepsilon}\star\Delta^{m}\cup\pr{\cal C_{/\sigma\varepsilon}^{0}\star\partial\Delta^{m}}$.
The claim is obvious if $u$ is not surjective on vertices. So assume
that $u$ is surjective on vertices. Since $\alpha\vert\Delta^{l}$
is a degeneration of $\sigma$, it cannot factor through $\sigma\vert\partial\Delta^{n}$.
Thus $\alpha$ does not belong to $K_{0}$. Hence (since the vertices
of $\sigma$ do not belong to $\cal C^{0}$) we can find a factorization
of $\alpha$ of the form 
\[
\alpha:\Delta^{k}\star\Delta^{l}\xrightarrow{\id\star u'}\Delta^{k}\star\Delta^{m'}\xrightarrow{\widetilde{\sigma\varepsilon'}}\cal C,
\]
where $m'<m$, $\varepsilon':[m']\to[n]$ is a surjection, and $\widetilde{\sigma\varepsilon'}$
is an extension of $\sigma\varepsilon'$ which maps $\Delta^{k}$
into $\cal C^{0}$. Replacing $\Delta^{m'}$ by $\Delta^{u'[l]}$
if necessary, we may assume that $u'$ is surjective on vertices.
Factor the map $\widetilde{\sigma\varepsilon'}$ as 
\[
\Delta^{k}\star\Delta^{m'}\xrightarrow{s\star s'}\Delta^{p}\star\Delta^{m''}\xrightarrow{\tau}\cal C,
\]
where $p,m''\geq-1$, $s,s'$ are surjective on vertices and $\tau$
is nondegenerate. (Such a factorization exists because the vertices
of $\sigma$ do not belong to $\cal C^{0}$.) Then $\sigma\varepsilon'$
is a degeneration of $\tau\vert\Delta^{q}$, so Eilenberg-Zilber's
lemma implies that $\tau\vert\Delta^{m''}$ is a degeneration of $\sigma$.
Therefore, we can write $\tau\vert\Delta^{m''}=\sigma\varepsilon''$,
where $\varepsilon'':[m'']\to[n]$ is a surjective poset map. By Eilenberg-Zilber's
lemma, $\widetilde{\sigma\varepsilon'}$ is a degeneartion of $\tau$.
Since $m''<m$, this implies that the simplex $\pr{\widetilde{\sigma\varepsilon},u}$
belongs to $A\pr{\varepsilon}\star\Delta^{m}$, as required.

We next prove (ii). Since join functors preserves trivial cofibrations
in each variable, it will suffice to show that the inclusion $A\pr{\varepsilon}\subset\cal C_{/\sigma\varepsilon}^{0}$
is a trivial cofibration. Arguing as in the previous paragraph, we
see that $A\pr{\varepsilon}$ is the colimit of the diagram $F:\cal J_{0}\pr{\varepsilon}^{\op}\to\SS$
defined by $\pr{\varepsilon',\varepsilon''}\mapsto\cal C_{/\sigma\varepsilon'}^{0}$.
This diagram admits a natural transformation to the constant diagram
at $\cal C_{/\sigma\varepsilon}^{0}$ whose components are trivial
fibrations. It will therefore suffice to prove that the diagrams $F$
and the constant diagram at $\cal C_{/\sigma\varepsilon}^{0}$ are
projectively cofibrant. There is a functor $\cal J_{0}\pr{\varepsilon}^{\op}\to\omega$
which maps a factorization $[m]\to[m']\to[n]$ of $\varepsilon$ to
the integer $m'$. This functor endows the category $\cal J_{0}\pr{\varepsilon}^{\op}$
with the structure of a direct category. The claim is thus a consequence
of \cite[Theorem 5.1.3]{Hovey2007}.
\end{proof}
\begin{defn}
Let $X$ be a simplicial set over $\Delta^{1}$. Given a simplex $\Delta^{k}\to X$,
we define the \textbf{head} of $X$ to be the simplex $\Delta^{u^{-1}\pr 1}\to\Delta^{k}\to X$
and the \textbf{tail} to be the simplex $\Delta^{u^{-1}\pr 0}\to\Delta^{k}\to X$,
where $u$ denotes the composite $\Delta^{k}\to X\to\Delta^{1}$.
\end{defn}
\begin{prop}
\label{prop:heads_and_tails}Let % https://q.uiver.app/?q=WzAsMyxbMCwwLCJYIl0sWzIsMCwiWSJdLFsxLDEsIlxcRGVsdGFeMSJdLFswLDEsInAiXSxbMSwyXSxbMCwyXV0=
\[\begin{tikzcd}
	X && Y \\
	& {\Delta^1}
	\arrow["p", from=1-1, to=1-3]
	\arrow[from=1-3, to=2-2]
	\arrow[from=1-1, to=2-2]
\end{tikzcd}\]be a commutative diagram of simplicial sets. Let $n\geq0$, and let
$S\subset S'\subset Y_{1}=Y\times_{\Delta^{1}}\{1\}$ be simplicial
subsets satisfying the following conditions:
\begin{enumerate}
\item The simplicial set $S$ contains the $\pr{n-1}$-skeleton of $Y_{1}$.
\item The simplicial set $S'$ is generated by $S$ and a set $\Sigma$
of nondegenerate $n$-simplices of $Y_{1}$ which do not belong to
$S'$.
\end{enumerate}
Let $X\pr S$ denote the simplicial subset of $X$ spanned by the
simplices whose head lies over $S$, and define $X\pr{S'}$ similarly.
Let $\{\sigma_{a}\}_{a\in A}$ be the enumeration of all nondegenerate
simplices of $X_{1}$ whose image in $Y$ is a degeneration of a simplex
in $\Sigma$. Choose an ordering of $A$ so that that dimension of
$\sigma_{a}$ is a non-decreasing function of $a\in A$. For each
$a\in A$, define simplicial sets $X\pr{S'}_{<a}$, $X\pr{S'}_{\leq a}$,
$K_{a}$, and $K_{0,a}$ as follows:
\begin{itemize}
\item $X\pr{S'}_{<a}\subset X$ is the simplicial subset generated by $X\pr S$
and those simplices of $X$ whose head factors through $\sigma_{b}$
for some $b<a$.
\item $X\pr{S'}_{\leq a}\subset X$ is the simplicial subset generated by
$X\pr S$ and those simplices of $X$ whose head factors through $\sigma_{b}$
for some $b\leq a$.
\item $K_{a}\subset X$ is the simplicial subset consisting of the simplices
whose head factors through $\sigma_{a}$.
\item $K_{0,a}\subset X$ is the simplicial subset consisting of the simplices
whose head factors through $\partial\sigma_{a}=\sigma_{a}\vert\partial\Delta^{\dim\sigma_{a}}$.
\end{itemize}
Then the following holds:
\begin{enumerate}
\item $X\pr{S'}=X\pr S\cup\bigcup_{a\in A}X\pr{S'}_{\leq a}$.
\item For each $a\in A$, the square % https://q.uiver.app/?q=WzAsNCxbMCwwLCJLX3swLGF9Il0sWzAsMSwiS19hIl0sWzEsMCwiWChTJylfezxhfSJdLFsxLDEsIlgoUycpX3tcXGxlcSBhfSJdLFswLDFdLFswLDJdLFsyLDNdLFsxLDNdXQ==
\begin{equation}\label{d:h_and_t}
\begin{tikzcd}
	{K_{0,a}} & {X(S')_{<a}} \\
	{K_a} & {X(S')_{\leq a}}
	\arrow[from=1-1, to=2-1]
	\arrow[from=1-1, to=1-2]
	\arrow[from=1-2, to=2-2]
	\arrow[from=2-1, to=2-2]
\end{tikzcd}
\end{equation}of simplicial sets is cocartesian.
\end{enumerate}
\end{prop}
\begin{proof}
We start with (1). The containment $X\pr S\cup\bigcup_{a\in A}X\pr{S'}_{\leq a}\subset X\pr{S'}$
holds trivially. For the reverse inclusion, let $x$ be an arbitrary
simplex of $X\pr{S'}$. We must show that $x$ belongs to $X\pr S\cup\bigcup_{a\in A}X\pr{S'}_{\leq a}$.
Let $\tau\pr x:\Delta^{m}\to X_{1}$ denote the head of $x$. If $p\tau\pr x$
belongs to $S$, then $x$ belongs to $X\pr S$ and we are done. If
$\tau\pr x$ does not belong to $S$, then $p\tau\pr x$ factors through
a simplex $\sigma$ in $\Sigma$. If $p\tau\pr x$ factors through
the boundary of $\sigma$, then $p\tau\pr x$ belongs to the $\pr{n-1}$-skeleton
of $Y_{1}$ and hence $\tau\pr x$ belongs to $S$, a contradiction.
Therefore, $p\tau\pr x$ is a degeneration of $\sigma$. Now $\tau\pr x$
is a degeneration of some nondegenerate simplex $\tau'\pr x$ of $X_{1}$.
Then $p\tau\pr x$ is a degeneration of $p\tau'\pr x$, so Eilenberg-Zilber's
lemma implies that $p\tau'\pr x$ is a degeneration of $\sigma$.
Hence $\tau'\pr x=\sigma_{a}$ for some $a\in A$, and hence $x\in X\pr{S'}_{\leq a}$.

We next prove (2). We will write $K_{0}=K_{0,a}$ and $K=K_{a}$.
First we remark that $X\pr{S'}_{<a}$ contains $K_{0}$. Indeed, suppose
we are given a simplex $x$ of $X$ whose head $\tau\pr x:\Delta^{m}\to X_{1}$
facotors through $\partial\sigma_{a}$. By construction, there is
a commutative diagram % https://q.uiver.app/?q=WzAsNixbMSwwLCJcXERlbHRhXntcXGRpbVxcc2lnbWFfYX0iXSxbMiwwLCJcXERlbHRhXm4iXSxbMiwxLCJZXzEsIl0sWzEsMSwiWF8xIl0sWzAsMCwiXFxwYXJ0aWFsXFxEZWx0YV57XFxkaW1cXHNpZ21hX2F9Il0sWzAsMSwiXFxEZWx0YV5tIl0sWzAsMSwicyJdLFsxLDIsIlxcc2lnbWEiXSxbMywyLCJwIiwyXSxbMCwzLCJcXHNpZ21hX2EiLDJdLFs0LDAsIiIsMCx7InN0eWxlIjp7InRhaWwiOnsibmFtZSI6Imhvb2siLCJzaWRlIjoidG9wIn19fV0sWzUsMywiXFx0YXUoeCkiLDJdLFs1LDRdXQ==
\[\begin{tikzcd}
	{\partial\Delta^{\dim\sigma_a}} & {\Delta^{\dim\sigma_a}} & {\Delta^n} \\
	{\Delta^m} & {X_1} & {Y_1,}
	\arrow["s", from=1-2, to=1-3]
	\arrow["\sigma", from=1-3, to=2-3]
	\arrow["p"', from=2-2, to=2-3]
	\arrow["{\sigma_a}"', from=1-2, to=2-2]
	\arrow[hook, from=1-1, to=1-2]
	\arrow["{\tau(x)}"', from=2-1, to=2-2]
	\arrow[from=2-1, to=1-1]
\end{tikzcd}\]where $s$ is surjective on vertices. If the map $\Delta^{m}\to\Delta^{n}$
is not surjective on vertices, then $p\tau\pr x$ factors through
the $\pr{n-1}$-skeleton of $Y_{1}$ and hence $x$ belongs to $X\pr S$.
If the map $\Delta^{m}\to\Delta^{n}$ is surjective on vertices, then
write $\tau\pr x$ as a degeneration of a nondegenerate simplex $\tau'\pr x$
of $X_{1}$. Eilenberg-Zilber's lemma implies that the simplex $p\tau'\pr x$
must be a degeneration of $\sigma$. Thus $p\tau'\pr x=\sigma_{b}$
for some $b\in A$. Since $\tau\pr x$ factors through the boundary
of $\sigma_{a}$, the dimension of $\sigma_{b}$ is strictly smaller
than that of $\sigma_{a}$. Thus $b<a$. Hence $x$ belongs to $X\pr{S'}_{<a}$.

By what we have just shown in the previous paragraph, the diagram
(\ref{d:h_and_t}) is well-defined. We now show that it is cocartesian.
By definition, $X\pr{S'}_{\leq a}$ is the union of $K$ and $X\pr{S'}_{<a}$.
Therefore, it suffices to show that $K_{0}$ is the intersection of
$K$ and $X\pr{S'}_{<a}$. So let $x$ be a simplex of $K$. We must
show that, if $x$ does not belong to $K_{0}$, then $x$ does not
belong to $X\pr{S'}_{<a}$ either. Let $\tau\pr x:\Delta^{m}\to X_{1}$
be the head of $x$. Since $x$ does not belong to $K_{0}$, $\tau\pr x$
is a degeneration of $\sigma_{a}$. Thus $p\pr{\tau\pr x}$ is a degeneration
of some simplex $\sigma\in\Sigma$. In particular, $p\pr{\tau\pr x}$
does not belong to $S$, so $\tau\pr x$ does not belong to $X\pr S$.
Therefore, should $\tau\pr x$ belong to $X\pr{S'}_{<a}$, then $\tau\pr x$
must factor through some $\sigma_{b}$ for some $b<a$. If $\tau\pr x$
factors through $\partial\sigma_{b}$ for some $b<a$, then we would
have $\dim\sigma_{a}<\dim\sigma_{b}$, a contradiction. So $\tau\pr x$
is a degeneartion of $\sigma_{b}$; but then $a=b$, a contradiction.
Thus $x$ does not belong to $X\pr{S'}_{<a}$, as required.
\end{proof}

\subsection{Proof of Theorem \ref{thm:3.1.2.3}}

In this final subsection of this note, we will give a proof of Theorem
\ref{thm:3.1.2.3} by elaborating the proof given by Lurie in \cite{HA}.

The proof proceeds by a simplex-by-simplex argument. For this, we
will classify simplices of $N\pr{\Fin_{\ast}}\times\Delta^{\{1,\dots,n\}}$
into five (somewhat artificial) groups.
\begin{defn}
Let $n\geq1$ and let $\alpha$ be a morphism in $N\pr{\Fin_{\ast}}\times\Delta^{\{1,\dots,n\}}$
with image $\alpha_{0}:\inp m\to\inp n$ in $N\pr{\Fin_{\ast}}$.
We say that $\alpha$ is:
\begin{enumerate}
\item \textbf{active} if $\alpha_{0}$ is active;
\item \textbf{strongly inert} if $\alpha_{0}$ is inert, the induced injection
$\inp n^{\circ}\to\inp m^{\circ}$ is order-perserving, and the image
of $\alpha$ in $\Delta^{\{1,\dots,n\}}$ is degenerate; and
\item \textbf{neutral} if it is neigher active nor strongly inert.
\end{enumerate}
Note that active morphisms and strongly inert morphisms are closed
under composition. Also, every morphism in $N\pr{\Fin_{\ast}}\times\Delta^{\{1,\dots,n\}}$
can be factored uniquely as a composition of a strongly inert map
followed by an active map.

Let $\sigma$ be an $m$-simplex of $N\pr{\Fin_{\ast}}\times\Delta^{\{1,\dots,n\}}$
depicted as
\[
\pr{\inp{k_{0}},e_{0}}\xrightarrow{\alpha_{\sigma}\pr 1}\cdots\xrightarrow{\alpha_{\sigma}\pr m}\pr{\inp{k_{m}},e_{m}}.
\]
We will say that $\sigma$ is \textbf{closed} if $k_{m}=1$, and \textbf{open}
otherwise. We say that $\sigma$ is \textbf{complete}\footnote{Lurie uses te term ``new'' instead of ``complete.''}
if $\{e_{0},\dots,e_{m}\}=\{1,\dots,n\}$ and \textbf{incomplete}
otherwise. Every nondegenerate simplex of $N\pr{\Fin_{\ast}}\times\Delta^{\{1,\dots,n\}}$
is a face of a nondegenerate complete simplex. 

We partition the set of nondegenerate complete simplices of $N\pr{\Fin_{\ast}}\times\Delta^{\{1,\dots n\}}$
into five groups $G_{\pr 1},G_{\pr 2},G'_{\pr 2},G_{\pr 3},G_{\pr 3}'$
as follows: $m$-simplex of $N\pr{\Fin_{\ast}}\times\Delta^{\{1,\dots,n\}}$.
Write $\alpha_{\sigma}\pr i=\sigma\vert\Delta^{\{i,i+1\}}$. Let $0\le k\leq m$
be the minimal integer such that $\alpha_{\sigma}\pr i$ is strongly
inert for every $i>k$, and let $0\leq j\leq k$ be the minimal integer
such that $\alpha_{\sigma}\pr i$ is active for every $j<i\leq k$.
\begin{itemize}
\item If $j=0$, $k=m$, and $\sigma$ is closed, then $\sigma$ belongs
to $G_{\pr 1}$.
\item If $j=0$, $k<m$, and $\sigma$ is closed, then $\sigma$ belongs
to $G_{\pr 2}$.
\item If $j=0$ and $\sigma$ is open, then $\sigma$ belongs to $G'_{\pr 2}$.
\item If $j\geq1$ and $\alpha_{\sigma}\pr j$ is strongly inert, then $\sigma$
belongs to $G_{\pr 3}$.
\item If $j\geq1$ and $\alpha_{\sigma}\pr j$ is neutral, then $\sigma$
belongs to $G_{\pr 3}'$.
\end{itemize}
Given an $m$-simplex $\sigma\in G_{\pr 2}$, we define its \textbf{associate}
$\sigma'\in G_{\pr 2}'$ by $\sigma'=\sigma\vert\Delta^{\{0,\dots,m-1\}}$.
Given an $m$-simplex $\sigma\in G_{\pr 3}$, then we define its \textbf{associate}
$\sigma'$ by $\sigma'=\sigma\partial_{j}\in G_{\pr 3}'$, where $j$
is the integer defined as above.
\end{defn}
%
\begin{proof}
[Proof of Theorems \ref{thm:3.1.2.3}]We will regard $\cal M^{\t}$
and $N\pr{\Fin_{\ast}}\times\Delta^{n}$ as a simplicial set over
$\Delta^{1}$ by means of the map $\Delta^{n}\to\Delta^{1}$ which
maps the vertex $0\in\Delta^{n}$ to the vertex $0\in\Delta^{1}$
and the remaining vertices to the vertex $1\in\Delta^{1}$. Given
a simplicial subset $S\subset N\pr{\Fin_{\ast}}\times\Delta^{\{1,\dots,n\}}$,
we let $\cal M_{S}^{\t}\subset\cal M^{\t}$ denote the simplicial
subset consisting of the simplices whose tail lies over $S$. We will
also write $\overline{S}=\pr{N\pr{\Fin_{\ast}}\times\Delta^{\{1,\dots,n\}}}_{S}$.

The implication (a)$\implies$(b) for part (A) is obvious. Assume
therefore that condition (b) is satisfied if $n=1$. For each $m\geq0$,
let $F\pr m$ denote the simplicial subset of $N\pr{\Fin_{\ast}}\times\Delta^{\{1,\dots,n\}}$
generated by the nondegenerate simplices $\sigma$ satisfying one
of the following conditions:
\begin{itemize}
\item $\sigma$ is incomplete.
\item $\sigma$ has dimension less than $m$.
\item $\sigma$ has dimension $m$ and belongs to $G_{\pr 2}$ or $G_{\pr 3}$.
\end{itemize}
Observe that $\cal M_{F\pr 0}^{\t}=\cal M^{\t}\times_{\Delta^{n}}\Lambda_{0}^{n}$.
We will complete the proof by inductively constructing a map $f_{m}:\cal M_{F\pr m}^{\t}\to\cal C^{\t}$
which makes the diagram % https://q.uiver.app/?q=WzAsNCxbMCwwLCJcXG1hdGhjYWx7TX1eXFxvdGltZXMgX3tGKG0tMSl9Il0sWzAsMSwiXFxtYXRoY2Fse019Xlxcb3RpbWVzIF97RihtKX0iXSxbMSwxLCJcXG1hdGhjYWx7T31eXFxvdGltZXMgIl0sWzEsMCwiXFxtYXRoY2Fse0N9Xlxcb3RpbWVzIl0sWzAsMSwiIiwwLHsic3R5bGUiOnsidGFpbCI6eyJuYW1lIjoiaG9vayIsInNpZGUiOiJ0b3AifX19XSxbMSwyLCJnXFx2ZXJ0XFxtYXRoY2Fse019Xlxcb3RpbWVzIF97RihtKX0iLDJdLFszLDIsInEiXSxbMCwzLCJmX3ttLTF9Il0sWzEsMywiZl97bX0iLDFdXQ==
\[\begin{tikzcd}
	{\mathcal{M}^\otimes _{F(m-1)}} & {\mathcal{C}^\otimes} \\
	{\mathcal{M}^\otimes _{F(m)}} & {\mathcal{O}^\otimes }
	\arrow[hook, from=1-1, to=2-1]
	\arrow["{g\vert\mathcal{M}^\otimes _{F(m)}}"', from=2-1, to=2-2]
	\arrow["q", from=1-2, to=2-2]
	\arrow["{f_{m-1}}", from=1-1, to=1-2]
	\arrow["{f_{m}}"{description}, from=2-1, to=1-2]
\end{tikzcd}\]commutative, and such that $f_{1}$ has the following special properties
if $n=1$:
\begin{itemize}
\item [(i)]For each object $B\in\cal M^{\t}\times_{\Delta^{1}}\{1\}$,
the map 
\[
\pr{\pr{\cal M_{\act}^{\t}}_{/B}\times_{\Delta^{1}}\{0\}}^{\rcone}\to\cal M_{\pr 1}^{\t}\xrightarrow{f_{1}}\cal C^{\t}
\]
is an operadic $q$-colimit diagram.
\item [(ii)]For every inert morphism $e:M'\to M$ in $\cal M^{\t}\times_{\Delta^{1}}\{1\}$
such that $M\in\cal M$, the functor $f_{1}$ carries $e$ to an inert
morphism in $\cal C^{\t}$.
\end{itemize}
Fix $m>0$, and suppose that $f_{m-1}$ has been constructed. Observe
that $F\pr m$ is obtained from $F\pr{m-1}$ by adjoining the following
simplices:
\begin{itemize}
\item The $\pr{m-1}$-simplices in $G_{\pr 1}$.
\item The $\pr{m-1}$-simplices in $G'_{\pr 2}$ without associates.
\item The $m$-simplices in $G_{\pr 2}$ and $G_{\pr 3}$.
\end{itemize}
We define simplicial subsets $F'\pr m\subset F''\pr m\subset F\pr m$
as follows: $F'\pr m$ is generated by $F\pr{m-1}$ and the $\pr{m-1}$-simplices
in $G_{\pr 1}$; $F'\pr{m-1}$ is generated by $F'\pr m$ and the
$\pr{m-1}$-simplices of $G'_{\pr 2}$ without associates. Our strategy
is to extend $f_{m-1}$ to $\cal M_{F'\pr m}^{\t}$, then to $\cal M_{F''\pr m}^{\t}$,
and then to $\cal M_{F\pr m}^{\t}$. 

\begin{enumerate}[label=(\textbf{Step \arabic*}),wide =0.5\parindent, listparindent=1.5em]

\item We will extend $f_{m-1}$ to a map $f'_{m}:\cal M_{F'\pr m}^{\t}\to\cal C^{\t}$
over $\cal O^{\t}$. 

Let $\{\sigma_{a}\}_{a\in A}$ be the collection of nondegenerate
simplices of $\cal M^{\t}\times_{\Delta^{n}}\Delta^{\{1,\dots,n\}}$
whose image in $N\pr{\Fin_{\ast}}\times\Delta^{\{1,\dots,n\}}$ is
a degeneration of some $\pr{m-1}$-simplex of $G_{\pr 1}$. Choose
a well-ordering on $A$ so that $\dim\sigma_{a}$ is non-decreasing
in $a\in A$. For each $a\in A$, let $\cal M_{<a}^{\t}$ denote the
simplicial subset spanned by $\cal M_{F\pr{m-1}}^{\t}$ and the simplices
of $\cal M^{\t}$ whose tail factors through $\sigma_{b}$ for some
$b<a$. We define $\cal M_{\leq a}^{\t}$ similarly. According to
Proposition \ref{prop:heads_and_tails}, we have $\cal M_{F'\pr m}^{\t}=\bigcup_{a\in A}\cal M_{\leq a}^{\t}$,
so it suffices to extend $f_{m-1}$ to an $A$-sequence $f^{\leq a}:\cal M_{\leq a}^{\t}\to\cal C^{\t}$
over $\cal O^{\t}$. 

The construction is inductive. Let $a\in A$, and suppose that $f^{\leq b}$
has been constructed for $b<a$. These maps determine a map $f^{<a}:\cal M_{<a}^{\t}\to\cal C^{\t}$
extending $f_{m-1}$. Let $K_{0}\subset K\subset\cal M^{\t}$ denote
the simplicial subset consisting of the simplices of $\cal M^{\t}$
whose head factors through $\sigma_{a}\vert\partial\Delta^{\dim\sigma_{a}}$
and $\sigma_{a}$, respectively. According to Lemma \ref{lem:3.1.2.5},
the left hand square of the commutative diagram% https://q.uiver.app/?q=WzAsNixbMCwwLCIoXFxtYXRoY2Fse019Xlxcb3RpbWVzIF97L1xcc2lnbWFfYX1cXHRpbWVzX3tcXERlbHRhXm59IFxcezBcXH0pXFxzdGFyIFxccGFydGlhbFxcRGVsdGEgXntcXGRpbVxcc2lnbWEgX2F9Il0sWzAsMSwiKFxcbWF0aGNhbHtNfV5cXG90aW1lcyBfey9cXHNpZ21hX2F9XFx0aW1lc197XFxEZWx0YV5ufSBcXHswXFx9KVxcc3RhciBcXERlbHRhIF57XFxkaW1cXHNpZ21hIF9hfSJdLFsxLDAsIktfMCJdLFsxLDEsIksiXSxbMiwwLCJcXG1hdGhjYWx7TX1eXFxvdGltZXMgX3s8YX0iXSxbMiwxLCJcXG1hdGhjYWx7TX1eXFxvdGltZXMgX3tcXGxlcSBhfSJdLFswLDJdLFsxLDNdLFsyLDNdLFsyLDRdLFs0LDVdLFszLDVdLFswLDFdXQ==
\[\begin{tikzcd}
	{(\mathcal{M}^\otimes _{/\sigma_a}\times_{\Delta^n} \{0\})\star \partial\Delta ^{\dim\sigma _a}} & {K_0} & {\mathcal{M}^\otimes _{<a}} \\
	{(\mathcal{M}^\otimes _{/\sigma_a}\times_{\Delta^n} \{0\})\star \Delta ^{\dim\sigma _a}} & K & {\mathcal{M}^\otimes _{\leq a}}
	\arrow[from=1-1, to=1-2]
	\arrow[from=2-1, to=2-2]
	\arrow[from=1-2, to=2-2]
	\arrow[from=1-2, to=1-3]
	\arrow[from=1-3, to=2-3]
	\arrow[from=2-2, to=2-3]
	\arrow[from=1-1, to=2-1]
\end{tikzcd}\]is homotopy cocartesian. The right hand square is cocartesian by Proposition
\ref{prop:heads_and_tails}. It follows that the map
\[
\pr{\cal M_{/\sigma_{a}}^{\t}\times_{\Delta^{n}}\{0\}}\star\Delta^{\dim\sigma_{a}}\amalg_{\pr{\cal M_{/\sigma_{a}}^{\t}\times_{\Delta^{n}}\{0\}}\star\partial\Delta^{\dim\sigma_{a}}}\cal M_{<a}^{\t}\to\cal M_{\leq a}^{\t}
\]
is a trivial cofibration in the Joyal model structure. Thus we only
need to extend the composite
\[
g_{0}:\pr{\cal M_{/\sigma_{a}}^{\t}\times_{\Delta^{n}}\{0\}}\star\partial\Delta^{\dim\sigma_{a}}\to\cal M_{<a}^{\t}\xrightarrow{f^{<a}}\cal C^{\t}
\]
to a map $\pr{\cal M_{/\sigma_{a}}^{\t}\times_{\Delta^{n}}\{0\}}\star\Delta^{\dim\sigma_{a}}\to\cal C^{\t}$
over $\cal O^{\t}$.

Assume first that $\sigma_{a}$ is zero-dimensional, so that, in particular,
$m=1$. If $n>1$, then $F\pr 0=F'\pr 1$ and there is nothing to
do. If $n=1$, then let $B\in\cal M^{\t}$ be the image of $\sigma_{a}$.
Since $\sigma_{a}$ is closed, the object $B$ lies in $\cal M\times_{\Delta^{1}}\{1\}$.
Using the inert-active factorization system in $\cal M^{\t}$, we
see that the inclusion $\pr{\pr{\cal M_{\act}^{\t}}_{/B}\times_{\Delta^{1}}\{0\}}\subset\cal M_{/B}^{\t}\times_{\Delta^{1}}\{0\}$
is a right adjoint, hence final. So our assumption (b) ensures that
we can find the desired extension $g$. Note that condition (i) is
satisfied with this particular construction.

Assume next that $\sigma_{a}$ has positive dimension. Since $\Delta^{\dim\sigma_{a}}$
has an initial vertex, we see as in the previous paragraph that the
inclusion $\pr{\cal M_{\act}^{\t}}_{/\sigma_{a}}\times_{\Delta^{n}}\{0\}\subset\cal M_{/\sigma_{a}}^{\t}\times_{\Delta^{1}}\{0\}$
is a right adjoint, and hence final. Thus, by virtue of \cite[Proposition 3.1.1.7]{HA},
it suffices to show that the map
\[
\pr{\pr{\cal M_{\act}^{\t}}_{/\sigma_{a}}\times_{\Delta^{n}}\{0\}}\star\{0\}\to\cal M_{<a}^{\t}\xrightarrow{f^{<a}}\cal C^{\t}
\]
is an operadic $q$-colimit diagram. Let $B=\sigma_{a}\pr 0$. The
map $\pr{\cal M_{\act}^{\t}}_{/\sigma_{a}}\times_{\Delta^{n}}\{0\}\to\pr{\cal M_{\act}^{\t}}_{/B}\times_{\Delta^{n}}\{0\}$
is a trivial fibration, so it suffices to show that the map
\[
\phi:\pr{\pr{\cal M_{\act}^{\t}}_{/B}\times_{\Delta^{n}}\{0\}}\star\{0\}\to\cal M_{<a}^{\t}\xrightarrow{f^{<a}}\cal C^{\t}
\]
is an operadic $q$-colimit cone. Let $\inp p\in N\pr{\Fin_{\ast}}$
be the image of the object $B$. If $p=0$, then $\pr{\cal M_{\act}^{\t}}_{/B}\times_{\Delta^{n}}\{0\}$
is a contractible Kan complex, and for each object $\alpha:A\to B$
in $\pr{\cal M_{\act}^{\t}}_{/B}\times_{\Delta^{n}}\{0\}$, the map
$\phi\vert\{\alpha\}\star\{0\}$ is an equivalence in $\cal C^{\t}$
(since $\cal C_{\inp 0}^{\t}$ is a contractible Kan complex). Thus
$\phi$ is an operadic $q$-colimit cone by \cite[Example 3.1.1.6]{HA}.
If $p=1$, the claim follows from (i). If $p>1$, choose for each
$1\leq i\leq p$ an inert map $B\to B_{i}$ in $\cal M_{\act}^{\t}\times_{\Delta^{n}}\{1\}$
over $\rho^{i}:\inp p\to\inp 1$. Note that since $p>1$, we have
$m\geq2$, so that $f^{<a}$ is defined on $\cal M_{F\pr 1}^{\t}$. 

Using Proposition \ref{prop:oplus_p-limit}, choose a direct sum functor
$\bigoplus_{i=1}^{n}:\pr{\cal M_{\act}^{\t}}^{p}\times_{\pr{\Delta^{1}}^{p}}\Delta^{1}\to\cal M^{\t}$
so that there is an inert natural transformation $\widetilde{h}_{i}:\bigoplus_{i=1}^{n}\to\opn{pr}_{i}$
making the diagram % https://q.uiver.app/?q=WzAsNCxbMCwwLCIoKFxcbWF0aGNhbHtNfV5cXG90aW1lc197XFxtYXRocm17YWN0fX0pXnBcXHRpbWVzIF97KFxcRGVsdGFebilecH1cXERlbHRhXm4pXFx0aW1lcyBcXERlbHRhIF4xIl0sWzEsMCwiXFxtYXRoY2Fse019Xlxcb3RpbWVzICJdLFsxLDEsIlxcRGVsdGFeblxcdGltZXMgTihcXG1hdGhzZntGaW59X1xcYXN0KSJdLFswLDEsIihcXERlbHRhXm5cXHRpbWVzIE4oXFxtYXRoc2Z7RmlufV9cXGFzdCApX3tcXG1hdGhybXthY3R9fSlecClcXHRpbWVzIFxcRGVsdGFeMSJdLFswLDEsIlxcd2lkZXRpbGRle2h9X2kiXSxbMSwyLCJwIl0sWzMsMiwiXFxvcGVyYXRvcm5hbWV7aWR9X3tcXERlbHRhXm59XFx0aW1lcyBoX2kiLDJdLFswLDNdXQ==
\[\begin{tikzcd}
	{((\mathcal{M}^\otimes_{\mathrm{act}})^p\times _{(\Delta^n)^p}\Delta^n)\times \Delta ^1} & {\mathcal{M}^\otimes } \\
	{(\Delta^n\times N(\mathsf{Fin}_\ast )_{\mathrm{act}})^p)\times \Delta^1} & {\Delta^n\times N(\mathsf{Fin}_\ast)}
	\arrow["{\widetilde{h}_i}", from=1-1, to=1-2]
	\arrow["p", from=1-2, to=2-2]
	\arrow["{\operatorname{id}_{\Delta^n}\times h_i}"', from=2-1, to=2-2]
	\arrow[from=1-1, to=2-1]
\end{tikzcd}\]commutative. There is an equivalence $\alpha:B\xrightarrow{\simeq}\bigoplus_{i=1}^{p}B_{i}$
lying over the identity of $\inp p$. There is a natural equivalence
\[
H:\pr{\pr{\cal M_{\act}^{\t}}_{/\alpha}\times_{\Delta^{n}}\{0\}}^{\rcone}\times\Delta^{1}\to\cal M^{\t}
\]
which makes the diagram % https://q.uiver.app/?q=WzAsNixbMCwwLCIoKFxcbWF0aGNhbHtNfV5cXG90aW1lcyBfe1xcbWF0aHJte2FjdH19KV97L1xcYWxwaGF9XFx0aW1lcyBfe1xcRGVsdGFebn1cXHswXFx9KV5cXHRyaWFuZ2xlcmlnaHRcXHRpbWVzIFxcezBcXH0iXSxbMSwwLCIoKFxcbWF0aGNhbHtNfV5cXG90aW1lcyBfe1xcbWF0aHJte2FjdH19KV97L0J9XFx0aW1lcyBfe1xcRGVsdGFebn1cXHswXFx9KV5cXHRyaWFuZ2xlcmlnaHQiXSxbMCwxLCIoKFxcbWF0aGNhbHtNfV5cXG90aW1lcyBfe1xcbWF0aHJte2FjdH19KV97L1xcYWxwaGF9XFx0aW1lcyBfe1xcRGVsdGFebn1cXHswXFx9KV5cXHRyaWFuZ2xlcmlnaHRcXHRpbWVzIFxcRGVsdGFeMSJdLFsxLDEsIlxcbWF0aGNhbHtNfV5cXG90aW1lcyAiXSxbMCwyLCIoKFxcbWF0aGNhbHtNfV5cXG90aW1lcyBfe1xcbWF0aHJte2FjdH19KV97L1xcYWxwaGF9XFx0aW1lcyBfe1xcRGVsdGFebn1cXHswXFx9KV5cXHRyaWFuZ2xlcmlnaHRcXHRpbWVzIFxcezFcXH0iXSxbMSwyLCIoKFxcbWF0aGNhbHtNfV5cXG90aW1lcyBfe1xcbWF0aHJte2FjdH19KV97L1xcYmlnb3BsdXNfe2k9MX1ebkJfaX1cXHRpbWVzIF97XFxEZWx0YV5ufVxcezBcXH0pXlxcdHJpYW5nbGVyaWdodFxcdGltZXMgXFx7MVxcfSJdLFswLDFdLFswLDJdLFsxLDNdLFsyLDNdLFs0LDJdLFs1LDNdLFs0LDVdXQ==
\[\begin{tikzcd}
	{((\mathcal{M}^\otimes _{\mathrm{act}})_{/\alpha}\times _{\Delta^n}\{0\})^\triangleright\times \{0\}} & {((\mathcal{M}^\otimes _{\mathrm{act}})_{/B}\times _{\Delta^n}\{0\})^\triangleright} \\
	{((\mathcal{M}^\otimes _{\mathrm{act}})_{/\alpha}\times _{\Delta^n}\{0\})^\triangleright\times \Delta^1} & {\mathcal{M}^\otimes } \\
	{((\mathcal{M}^\otimes _{\mathrm{act}})_{/\alpha}\times _{\Delta^n}\{0\})^\triangleright\times \{1\}} & {((\mathcal{M}^\otimes _{\mathrm{act}})_{/\bigoplus_{i=1}^nB_i}\times _{\Delta^n}\{0\})^\triangleright\times \{1\}}
	\arrow[from=1-1, to=1-2]
	\arrow[from=1-1, to=2-1]
	\arrow[from=1-2, to=2-2]
	\arrow[from=2-1, to=2-2]
	\arrow[from=3-1, to=2-1]
	\arrow[from=3-2, to=2-2]
	\arrow[from=3-1, to=3-2]
\end{tikzcd}\]commutative, such that the restriction of $H$ to $\pr{\pr{\cal M_{\act}^{\t}}_{/\alpha}\times_{\Delta^{n}}\{0\}}\times\Delta^{1}$
is the constant natural transformation at the projection $\pr{\pr{\cal M_{\act}^{\t}}_{/\alpha}\times_{\Delta^{n}}\{0\}}\to\cal M^{\t}$
and the restriction $H\vert\{\infty\}\times\Delta^{1}$ is given by
$\alpha$. This natural equivalence takes values in $\cal M_{F\pr 1}^{\t}$,
because $\alpha$ lies in $\cal M_{F\pr 1}^{\t}$. So it suffices
to show that the composite 
\[
\phi:\pr{\pr{\cal M_{\act}^{\t}}_{/\bigoplus_{i=1}^{p}B_{i}}\times_{\Delta^{n}}\{0\}}^{\rcone}\to\cal M_{F\pr 1}^{\t}\xrightarrow{f_{1}}\cal C^{\t}
\]
is an operadic $q$-colimit diagram. Now consider the commutative
diagram % https://q.uiver.app/?q=WzAsNSxbMCwwLCIoXFxwcm9kX3sxXFxsZXEgaVxcbGVxIHB9KFxcbWF0aGNhbHtNfV5cXG90aW1lcyBfe1xcbWF0aHJte2FjdH19KV97L0JfaX1cXHRpbWVzIF97XFxEZWx0YV5ufVxcezBcXH0pXlxcdHJpYW5nbGVyaWdodCJdLFsxLDAsIigoKFxcbWF0aGNhbHtNfV5cXG90aW1lcyBfe1xcbWF0aHJte2FjdH19KV5wXFx0aW1lcyBfeyhcXERlbHRhXm4pXnB9XFxEZWx0YV5uKV97LyhCXzEsXFxkb3RzICxCX3ApfVxcdGltZXMgX3tcXERlbHRhXm59XFx7MFxcfSleXFx0cmlhbmdsZXJpZ2h0Il0sWzEsMSwiKChcXG1hdGhjYWx7TX1eXFxvdGltZXMgX3tcXG1hdGhybXthY3R9fSlfey9cXGJpZ29wbHVzX3tpPTF9XnBCX2l9XFx0aW1lcyBfe1xcRGVsdGFebn1cXHswXFx9KV5cXHRyaWFuZ2xlcmlnaHQiXSxbMSwyLCJcXG1hdGhjYWx7TX1eXFxvdGltZXMgLiJdLFswLDIsIihcXG1hdGhjYWx7TX1eXFxvdGltZXMgX3tcXG1hdGhybXthY3R9fSlecFxcdGltZXMgX3soXFxEZWx0YV5uKV5wfVxcRGVsdGFebiJdLFswLDEsIlxcY29uZyJdLFsxLDIsIlxccHNpIl0sWzIsM10sWzQsMywiXFxiaWdvcGx1c197aT0xfV5wIiwyXSxbMCw0XSxbMSwyLCJcXHNpbWVxIiwyXV0=
\[\begin{tikzcd}
	{(\prod_{1\leq i\leq p}(\mathcal{M}^\otimes _{\mathrm{act}})_{/B_i}\times _{\Delta^n}\{0\})^\triangleright} & {(((\mathcal{M}^\otimes _{\mathrm{act}})^p\times _{(\Delta^n)^p}\Delta^n)_{/(B_1,\dots ,B_p)}\times _{\Delta^n}\{0\})^\triangleright} \\
	& {((\mathcal{M}^\otimes _{\mathrm{act}})_{/\bigoplus_{i=1}^pB_i}\times _{\Delta^n}\{0\})^\triangleright} \\
	{(\mathcal{M}^\otimes _{\mathrm{act}})^p\times _{(\Delta^n)^p}\Delta^n} & {\mathcal{M}^\otimes .}
	\arrow["\cong", from=1-1, to=1-2]
	\arrow["\psi", from=1-2, to=2-2]
	\arrow[from=2-2, to=3-2]
	\arrow["{\bigoplus_{i=1}^p}"', from=3-1, to=3-2]
	\arrow[from=1-1, to=3-1]
	\arrow["\simeq"', from=1-2, to=2-2]
\end{tikzcd}\]Here the map $\psi$ is the equivalence of $\infty$-categories induced
by the direct sum functor (Proposition \ref{prop:directsum_slice_equiv}).
In light of the commutativity of this diagram, the composite
\[
\eta:\pr{\prod_{1\leq i\leq p}\pr{\cal M_{\act}^{\t}}_{/B_{i}}\times_{\Delta^{n}}\{0\}}^{\rcone}\to\pr{\cal M_{\act}^{\t}}^{p}\times_{\pr{\Delta^{n}}^{p}}\Delta^{n}\xrightarrow{\bigoplus_{i=1}^{p}}\cal M^{\t}
\]
takes values in $\cal M_{F\pr 1}^{\t}$, and it suffices to show that
the composite $f_{1}\eta$ is an operadic $q$-colimit diagram. Now
the inert natural transformation $\widetilde{h}_{i}$ induces an inert
natural transfromation from $\eta$ to the composite 
\[
\eta_{i}:\pr{\prod_{1\leq i\leq p}\pr{\cal M_{\act}^{\t}}_{/B_{i}}\times_{\Delta^{n}}\{0\}}^{\rcone}\to\pr{\pr{\cal M_{\act}^{\t}}_{/B_{i}}\times_{\Delta^{n}}\{0\}}^{\rcone}\to\cal M^{\t}.
\]
Since $F\pr 1$ contains the $1$-simplices in $G_{\pr 2}$, this
natural transformation takes values in $\cal M_{F\pr 1}^{\t}$. Since
$f_{1}$ satisfies (ii), we deduce that the composite $f_{1}\eta$
admits an inert natural transformation $H_{i}$ to the composite $f_{1}\eta_{i}$,
such that for each vertex $v$ in $\pr{\prod_{1\leq i\leq p}\pr{\cal M_{\act}^{\t}}_{/B_{i}}\times_{\Delta^{n}}\{0\}}^{\rcone}$,
the components $\{H_{i}\pr v\}_{1\leq i\leq p}$ form a $q$-limit
cone. It follows from Corollary \ref{cor:3.1.1.8} and (i) that $f_{1}\eta$
is an operadic $q$-colimit diagram, as desired.

\item We will extend the map $f'_{m}$ in Step 1 to a map $f''_{m}:\cal M_{F''\pr m}^{\t}\to\cal C^{\t}$
over $\cal O^{\t}$. 

We argue as in Step 1. Let $\{\sigma_{a}\}_{a\in A}$ be the set of
all nondegenerate simplices of $\cal M^{\t}\times_{\Delta^{n}}\Delta^{\{1,\dots,n\}}$
whose image in $N\pr{\Fin_{\ast}}\times\Delta^{\{1,\dots,n\}}$ is
a degeneration of an $\pr{m-1}$-simplex in $G'_{\pr 2}$ without
associates. Choose a well-ordering of the set $A$ so that $\dim\sigma_{a}$
is a non-decreasing function of $a$. For each $a\in A$, let $\cal M_{<a}^{\t}$
denote the simplicial subset spanned by $\cal M_{F'\pr m}^{\t}$ and
the simplices of $\cal M^{\t}$ whose tail factors through $\sigma_{b}$
for some $b<a$. We define $\cal M_{\leq a}^{\t}$ similarly. (The
notations $\cal M_{<a}^{\t}$ and $\cal M_{\leq a}^{\t}$ are in conflict
with the ones introduced in Step 1, but there should not be any confusion.)
By Proposition \ref{prop:heads_and_tails}, we have $\cal M_{F''\pr m}^{\t}=\bigcup_{a\in A}\cal M_{\leq a}^{\t}$,
so it suffices to extend $f'_{m}$ to an $A$-sequence $f^{\leq a}:\cal M_{\leq a}^{\t}\to\cal C^{\t}$
over $\cal O^{\t}$. The construction is inductive. Suppose $f^{\leq b}$
has been constructed for $b<a$, and let $f^{<a}:\cal M_{<a}^{\t}\to\cal C^{\t}$
be their amalgamation. Just as in Step 1, we are reduced to solving
a lifting problem of the form % https://q.uiver.app/?q=WzAsNixbMCwwLCIoXFxtYXRoY2Fse019Xlxcb3RpbWVzIF97L1xcc2lnbWFfYX1cXHRpbWVzX3tcXERlbHRhXm59IFxcezBcXH0pXFxzdGFyIFxccGFydGlhbFxcRGVsdGEgXntcXGRpbVxcc2lnbWEgX2F9Il0sWzAsMSwiKFxcbWF0aGNhbHtNfV5cXG90aW1lcyBfey9cXHNpZ21hX2F9XFx0aW1lc197XFxEZWx0YV5ufSBcXHswXFx9KVxcc3RhciBcXERlbHRhIF57XFxkaW1cXHNpZ21hIF9hfSJdLFsxLDAsIlxcbWF0aGNhbHtNfV5cXG90aW1lcyBfezxhfSJdLFsyLDAsIlxcbWF0aGNhbHtDfV5cXG90aW1lcyJdLFsyLDEsIlxcbWF0aGNhbHtPfV5cXG90aW1lcyJdLFsxLDEsIlxcbWF0aGNhbHtNfV5cXG90aW1lcyAiXSxbMCwxXSxbMCwyXSxbMiwzLCJmXns8YX0iXSxbMSw1XSxbNSw0XSxbMyw0LCJxIl0sWzEsMywiIiwxLHsic3R5bGUiOnsiYm9keSI6eyJuYW1lIjoiZGFzaGVkIn19fV1d
\[\begin{tikzcd}
	{(\mathcal{M}^\otimes _{/\sigma_a}\times_{\Delta^n} \{0\})\star \partial\Delta ^{\dim\sigma _a}} & {\mathcal{M}^\otimes _{<a}} & {\mathcal{C}^\otimes} \\
	{(\mathcal{M}^\otimes _{/\sigma_a}\times_{\Delta^n} \{0\})\star \Delta ^{\dim\sigma _a}} & {\mathcal{M}^\otimes } & {\mathcal{O}^\otimes.}
	\arrow[from=1-1, to=2-1]
	\arrow[from=1-1, to=1-2]
	\arrow["{f^{<a}}", from=1-2, to=1-3]
	\arrow[from=2-1, to=2-2]
	\arrow[from=2-2, to=2-3]
	\arrow["q", from=1-3, to=2-3]
	\arrow[dashed, from=2-1, to=1-3]
\end{tikzcd}\]The existence of such a lift follows from Corollary \ref{cor:Step2}.

\item We complete the proof by extending the map $f''_{m}$ in Step
2 to a map $f_{m}:\cal M_{F\pr m}^{\t}\to\cal C^{\t}$ over $\cal O^{\t}$.
Let $\{\sigma'_{a}\}_{a\in A}$ be the collection of all $\pr{m-1}$-simplices
in $G'_{\pr 2}$ and $G'_{\pr 3}$ which have at least one associate.
Choose a well-ordering on $A$, and for each $a\in A$, let $F_{\leq a}$
denote the simplicial subset of $F\pr m$ generated by $F''\pr m$
and the associates of the simplices $\sigma'_{b}$ for $b\leq a$.
Define $F_{<a}$ similarly. We will choose the ordering on $A$ so
that the following condition is satisfied:
\begin{itemize}
\item [($\blacklozenge$)]Let $a\in A$ and let $\sigma$ be an associate
of $\sigma'_{a}$. Let $0\leq l\leq m$ be the (unique) integer such
that $d_{l}\sigma=\sigma'_{a}$. Then for each $i\in[m]\setminus\{l\}$,
the simplex $d_{i}\sigma$ belongs to $F_{<a}$.
\end{itemize}
The existence of such an ordering is nontrivial, but we will not get
into it presently. The construction of such an ordering can be found
at the very end of this step. Note that ($\blacklozenge$) implies
that:
\begin{itemize}
\item [($\blacklozenge\blacklozenge$)]For every $a\in A$, the simplex
$\sigma'_{a}$ does not belong to $F_{<a}$.
\end{itemize}
Indeed, suppose $\sigma'_{a}$ belongs to $F_{<a}$. Choose a minimal
element $b\in A$ such that $\sigma'_{a}$ belongs to $F_{<b}$. Then
$\sigma'_{a}$ factors through one of the following simplices:
\begin{enumerate}
\item Incomplete simplices.
\item Nondegenerate simplices of dimensions less than $m-1$.
\item $\pr{m-1}$-simplices in $G_{\pr 1}\cup G_{\pr 2}\cup G_{\pr 3}$.
\item $\pr{m-1}$-simplices in $G'_{\pr 2}$ without associates.
\item Associates of $\sigma'_{c}$ for some $c<b$.
\end{enumerate}
Since $\sigma'_{a}$ is complete and nondegenerate and has dimension
$m-1$, the cases (1), (2), (3), and (4) are immediately ruled out.
We show that the case (5) is impossible by reasoning by contradiction.
Suppose that there are an index $c<b$ and an associate $\sigma$
of $\sigma'_{c}$ through which $\sigma'_{a}$ factors. We have $\sigma'_{a}=d_{i}\sigma$
for some $0\leq i\leq m$ for dimensional reasons. Using ($\blacklozenge$),
we deduce that $\sigma'_{a}$ belongs to $F_{<c}$, contrary to the
minimality of $b$.

We now get back to the construction of $f_{m}$. We will construct
$f_{m}$ as an amalgamation of an $A$-sequence $\{f_{\leq a}:\cal M_{F_{\leq a}}^{\t}\to\cal C^{\t}\}_{a\in A}$
over $\cal O^{\t}$ which extends $f''_{m}$. The construction is
inductive. Suppose that $f_{\leq b}$ has been constructed for $b\leq a$,
so that they together determine a map $f_{<a}:\cal M_{F_{<a}}^{\t}\to\cal C^{\t}$.
We must extend $f_{<a}$ to $\cal M_{F_{\leq a}}^{\t}$. We consider
two cases, depending on whether $\sigma'_{a}$ belongs to $G'_{\pr 2}$
or to $G'_{\pr 3}$.

\begin{enumerate}[label=(Case \arabic*),wide =0.5\parindent=, listparindent=1.5em]

\item Suppose that $\sigma'_{a}$ belongs to $G'_{\pr 2}$. We shall
deploy an argument which is a variant of Proposition \ref{prop:heads_and_tails}.
Let $\{\tau_{\lambda}\}_{\lambda\in\Lambda}$ be the collection of
all nondegenerate simplices of $\cal M^{\t}$ the image of whose head
in $N\pr{\Fin_{\ast}}\times\Delta^{\{1,\dots,n\}}$ is a degneration
of $\sigma'_{a}$. Choose a well-ordering of $\Lambda$ so that $\dim\tau_{\lambda}$
is a non-decreasing function of $\lambda$. For each $\lambda\in\Lambda$,
we let $\cal N_{\leq\lambda}\subset\cal M^{\t}$ denote the simplicial
subset generated by $\cal M_{F_{<a}}^{\t}$ and the simplices $\tau:\Delta^{p}\to\cal M^{\t}$
for which there is an integer $0\leq p'<p$ such that $\tau\vert\Delta^{\{0,\dots,p'\}}$
factors through some $\tau_{\mu}$ for some $\mu\leq\lambda$, $\tau\vert\Delta^{\{p',p'+1\}}$
is inert, and $\tau\vert\Delta^{\{p'+1,\dots,p\}}$ factors through
$\cal M$. We define $\cal N_{<\lambda}$ similarly. 

We shall prove the following assertion later:
\begin{itemize}
\item [($*$)]$\cal M_{F_{\leq a}}^{\t}$ is the union of $\cal M_{F_{<a}}^{\t}$
and $\{\cal N_{\leq\lambda}\}_{\lambda\in\Lambda}$. 
\end{itemize}
Accepting ($\ast$) for now, we complete the proof as follows. It
will suffice construct a $\Lambda$-sequence $\{f^{\leq\lambda}:\cal N_{\leq\lambda}\to\cal C^{\t}\}_{\lambda\in\Lambda}$
of maps over $\cal O^{\t}$ which extends $f_{<a}$. The construction
is inductive. Suppose $f^{\leq\mu}$ has been constructed for $\mu<\lambda$,
and let $f^{<\lambda}:\cal N_{<\lambda}\to\cal C^{\t}$ denote the
map obtained by amalgamating the maps $\{f^{\leq\mu}\}_{\mu<\lambda}$.
Let $\pr{\inp k,n}$ be the final vertex of $\tau_{\lambda}$. Note
that $k\geq2$. There are $k$ inert maps  $\inp k\to\inp 1$, and
these maps and $p\tau_{\lambda}$ combine to determine a diagram $\Delta^{\dim\tau_{\lambda}}\star\inp k^{\circ}\to N\pr{\Fin_{\ast}}\times\Delta^{n}$.
Let $\cal X=\pr{\Delta^{\dim\tau_{\lambda}}\star\inp k^{\circ}}\times_{N\pr{\Fin_{\ast}}\times\Delta^{n}}\cal M^{\t}$
and let $\overline{\tau}_{\lambda}:\Delta^{\dim\tau_{\lambda}}\to\cal X$
denote the induced diagram. For each $1\leq i\leq k$, let $\cal X_{i}$
denote the fiber of $\cal X$ over $i\in\inp k^{\circ}$ (which is
isomorphic to $\cal M\times_{\Delta^{n}}\{n\}$), and set $\cal X^{0}=\bigcup_{1\leq i\leq k}\cal X_{i}$
and $\cal X_{\overline{\tau}_{\lambda}/}^{0}=\cal X^{0}\times_{\cal X}\cal X_{\overline{\tau}_{\lambda}/}$.
We now consider the following diagram: % https://q.uiver.app/?q=WzAsNixbMCwwLCJcXHBhcnRpYWxcXERlbHRhXntcXGRpbVxcb3ZlcmxpbmVcXHRhdV9cXGxhbWJkYX1cXHN0YXJcXG1hdGhjYWx7WH1eMF97XFxvdmVybGluZVxcdGF1X1xcbGFtYmRhL30iXSxbMCwxLCJcXERlbHRhXntcXGRpbVxcb3ZlcmxpbmVcXHRhdV9cXGxhbWJkYX1cXHN0YXJcXG1hdGhjYWx7WH1eMF97XFxvdmVybGluZVxcdGF1X1xcbGFtYmRhL30iXSxbMSwwLCJLXzAiXSxbMSwxLCJLIl0sWzIsMCwiXFxtYXRoY2Fse059X3s8XFxsYW1iZGF9Il0sWzIsMSwiXFxtYXRoY2Fse059X3tcXGxlcSBcXGxhbWJkYX0iXSxbMCwxXSxbMCwyXSxbMSwzXSxbMiwzXSxbMiw0XSxbMyw1XSxbNCw1XV0=
\[\begin{tikzcd}
	{\partial\Delta^{\dim\overline\tau_\lambda}\star\mathcal{X}^0_{\overline\tau_\lambda/}} & {K_0} & {\mathcal{N}_{<\lambda}} \\
	{\Delta^{\dim\overline\tau_\lambda}\star\mathcal{X}^0_{\overline\tau_\lambda/}} & K & {\mathcal{N}_{\leq \lambda}.}
	\arrow[from=1-1, to=2-1]
	\arrow[from=1-1, to=1-2]
	\arrow[from=2-1, to=2-2]
	\arrow[from=1-2, to=2-2]
	\arrow[from=1-2, to=1-3]
	\arrow[from=2-2, to=2-3]
	\arrow[from=1-3, to=2-3]
\end{tikzcd}\]Here $K\subset\cal X$ denotes the simplicial subset spanned by the
simplices whose tail factors through $\overline{\tau}_{\lambda}$,
and $K_{0}$ is its simplicial subset obtained by replacing $\overline{\tau}_{\lambda}$
by $\partial\overline{\tau}_{\lambda}$, where we regard $\cal X$
as a simplicial set over $\Delta^{1}$ by the map $\Delta^{\dim\tau_{\lambda}}\star\inp k^{\circ}\to\{0\}\star\{1\}=\Delta^{1}$.
The left hand square is homotopy cocartesian by Lemma \ref{lem:3.1.2.5}.
We shall prove later that:
\begin{itemize}
\item [($**$)]The horizontal arrows of the the right hand square are well-defined,
and the right hand square is cocartesian.
\end{itemize}
We are now reduced to solving the lifting problem % https://q.uiver.app/?q=WzAsNixbMCwwLCJcXHBhcnRpYWxcXERlbHRhXntcXGRpbVxcb3ZlcmxpbmVcXHRhdV9cXGxhbWJkYX1cXHN0YXJcXG1hdGhjYWx7WH1eMF97XFxvdmVybGluZVxcdGF1X1xcbGFtYmRhL30iXSxbMCwxLCJcXERlbHRhXntcXGRpbVxcb3ZlcmxpbmVcXHRhdV9cXGxhbWJkYX1cXHN0YXJcXG1hdGhjYWx7WH1eMF97XFxvdmVybGluZVxcdGF1X1xcbGFtYmRhL30iXSxbMSwwLCJcXG1hdGhjYWx7Tn1fezxcXGxhbWJkYX0iXSxbMSwxLCJcXG1hdGhjYWx7Tn1fe1xcbGVxIFxcbGFtYmRhfSJdLFsyLDAsIlxcbWF0aGNhbHtDfV5cXG90aW1lcyAiXSxbMiwxLCJcXG1hdGhjYWx7T31eXFxvdGltZXMgLiJdLFswLDFdLFsyLDRdLFszLDVdLFs0LDVdLFsxLDQsIiIsMSx7InN0eWxlIjp7ImJvZHkiOnsibmFtZSI6ImRhc2hlZCJ9fX1dLFswLDJdLFsxLDNdXQ==
\[\begin{tikzcd}
	{\partial\Delta^{\dim\overline\tau_\lambda}\star\mathcal{X}^0_{\overline\tau_\lambda/}} & {\mathcal{N}_{<\lambda}} & {\mathcal{C}^\otimes } \\
	{\Delta^{\dim\overline\tau_\lambda}\star\mathcal{X}^0_{\overline\tau_\lambda/}} & {\mathcal{N}_{\leq \lambda}} & {\mathcal{O}^\otimes .}
	\arrow[from=1-1, to=2-1]
	\arrow[from=1-2, to=1-3]
	\arrow[from=2-2, to=2-3]
	\arrow[from=1-3, to=2-3]
	\arrow[dashed, from=2-1, to=1-3]
	\arrow[from=1-1, to=1-2]
	\arrow[from=2-1, to=2-2]
\end{tikzcd}\]Now the $\infty$-category $\cal X_{\overline{\tau}_{\lambda}/}^{0}$
is the disjoint union of the $\infty$-categories $\pr{\cal X_{i}}_{\overline{\tau}_{\lambda}/}=\cal X_{i}\times_{\cal X}\cal X_{\overline{\tau}_{\lambda}/}$.
Each $\infty$-category $\pr{\cal X_{i}}_{\overline{\tau}_{\lambda}/}$
has an initial object, given by a cone $\phi_{i}:\pr{\Delta^{\dim\overline{\tau}_{\lambda}}}^{\rcone}\to\cal X$
which maps the last edge to an inert morphism over $\pr{\rho^{i},\id}:\pr{\inp k,n}\to\pr{\inp 1,n}$.
Set $S=\coprod_{i}\{\phi_{i}\}$. The inclusion $S\subset\cal X_{\overline{\tau}_{\lambda}/}^{0}$
is initial, so we are reduced to solving the lifting problem % https://q.uiver.app/?q=WzAsOCxbMSwwLCJcXHBhcnRpYWxcXERlbHRhXntcXGRpbVxcb3ZlcmxpbmVcXHRhdV9cXGxhbWJkYX1cXHN0YXJcXG1hdGhjYWx7WH1eMF97XFxvdmVybGluZVxcdGF1X1xcbGFtYmRhL30iXSxbMSwxLCJcXERlbHRhXntcXGRpbVxcb3ZlcmxpbmVcXHRhdV9cXGxhbWJkYX1cXHN0YXJcXG1hdGhjYWx7WH1eMF97XFxvdmVybGluZVxcdGF1X1xcbGFtYmRhL30iXSxbMiwwLCJcXG1hdGhjYWx7Tn1fezxcXGxhbWJkYX0iXSxbMiwxLCJcXG1hdGhjYWx7Tn1fe1xcbGVxIFxcbGFtYmRhfSJdLFszLDAsIlxcbWF0aGNhbHtDfV5cXG90aW1lcyAiXSxbMywxLCJcXG1hdGhjYWx7T31eXFxvdGltZXMgLiJdLFswLDAsIlxccGFydGlhbFxcRGVsdGFee1xcZGltXFxvdmVybGluZVxcdGF1X1xcbGFtYmRhfVxcc3RhciBTIl0sWzAsMSwiXFxEZWx0YV57XFxkaW1cXG92ZXJsaW5lXFx0YXVfXFxsYW1iZGF9XFxzdGFyIFMiXSxbMiw0XSxbMyw1XSxbNCw1XSxbMCwyXSxbMSwzXSxbNiwwXSxbNywxXSxbNyw0LCIiLDEseyJzdHlsZSI6eyJib2R5Ijp7Im5hbWUiOiJkYXNoZWQifX19XSxbNiw3XV0=
\[\begin{tikzcd}
	{\partial\Delta^{\dim\overline\tau_\lambda}\star S} & {\partial\Delta^{\dim\overline\tau_\lambda}\star\mathcal{X}^0_{\overline\tau_\lambda/}} & {\mathcal{N}_{<\lambda}} & {\mathcal{C}^\otimes } \\
	{\Delta^{\dim\overline\tau_\lambda}\star S} & {\Delta^{\dim\overline\tau_\lambda}\star\mathcal{X}^0_{\overline\tau_\lambda/}} & {\mathcal{N}_{\leq \lambda}} & {\mathcal{O}^\otimes .}
	\arrow[from=1-3, to=1-4]
	\arrow[from=2-3, to=2-4]
	\arrow[from=1-4, to=2-4]
	\arrow[from=1-2, to=1-3]
	\arrow[from=2-2, to=2-3]
	\arrow[from=1-1, to=1-2]
	\arrow[from=2-1, to=2-2]
	\arrow[dashed, from=2-1, to=1-4]
	\arrow[from=1-1, to=2-1]
\end{tikzcd}\] If the dimension of $\overline{\tau}_{\lambda}$ is positive, then
the claim is immediate since the restriction of the top horizontal
arrow to $\{\dim\overline{\tau}_{\lambda}\}\star S$ is a $q$-limit
cone. If $\overline{\tau}_{\lambda}$ is zero-dimensional (in which
case $m=n=1$), let $C_{i}$ denote the image of $\phi_{i}\in S$
under the top horizontal map. The bottom horizontal arrow classifies
a diagram $X\to q\pr{C_{i}}$ of inert maps $\{\alpha_{i}:X\to q\pr{C_{i}}\}_{1\leq i\leq k}$
lying over $\{\rho^{i}:\inp k\to\inp 1\}_{1\leq i\leq k}$, and we
wish to lift this to a diagram $S^{\lcone}\to\cal C^{\t}$ in $\cal C^{\t}$
which maps each $\phi_{i}\in S$ to the object $C_{i}$. Since $q$
is a fibration of $\infty$-operads, we can in fact find such a lift
consisting of inert morphisms. Note that with such a choice of lift,
condition (ii) is satisfied.

We now move on to the verification of ($\ast$) and ($\ast\ast$).
We begin with ($\ast$). It is clear that $\cal N_{\leq\lambda}$
and $\cal M_{F_{<a}}^{\t}$ are contained in $\cal M_{F_{\leq a}}^{\t}$.
For the reverse implication, let $x$ be an arbitrary simplex of $\cal M_{F_{\leq a}}^{\t}$.
We must show that $x$ belongs to either $\cal M_{F_{<a}}^{\t}$ or
one of the $\cal N_{\leq\lambda}$'s. If $x$ belongs to $\cal M_{F_{<a}}^{\t}$,
we are done. So assume not. Let $r$ denote the dimension of $x$.
Since $x$ does not belong to $\cal M_{F_{<a}}^{\t}$, its head is
nonempty. Find an integer $0\leq r'\leq r$ such that $x\vert\Delta^{\{r',\dots,r\}}$
is the head of $x$. Since $x$ does not belong to $\cal M_{F_{<a}}^{\t}$,
the simplex $px\vert\Delta^{\{r',\dots,r\}}$ factors through an associate
$\sigma$ of $\sigma'_{a}$. Let $\Delta^{r}\xrightarrow{u}\Delta^{m}\xrightarrow{\sigma}N\pr{\Fin_{\ast}}\times\Delta^{\{1,\dots,n\}}$
be such a factorization. If the image of $u$ does not contain some
integer $i\in\{0,\dots,m-1\}$, then $px\vert\Delta^{\{r',\dots,r\}}$
factors through $d_{i}\sigma$ and hnece through $F_{<a}$ by ($\blacklozenge$),
a contradiction. So the image of $u$ contains every integer in $\{0,\dots,m-1\}$.
Let $0\leq r''<r$ be the largest integer such that $u\pr{r''}<m$.
There are now several cases to consider:
\begin{itemize}
\item Suppose that $u$ is surjective on vertices and that the map $x\pr{r''}\to x\pr{r''+1}$
is inert. We write $x\vert\Delta^{\{0,\dots,r''\}}=s^{*}y$, where
$s:[r']\to[k]$ is a surjection and $y$ is nondegenerate. We claim
that $y$ is one of the $\tau_{\lambda}$'s, so that $x$ belongs
to $\cal N_{\leq\lambda}$. Let $k'=s\pr{r'}$. Then $y\vert\Delta^{\{k',\dots,k\}}$
is the head of $y$. We write $py\vert\Delta^{\{k',\dots,k\}}=s^{\p\ast}z$,
where $z$ is nondegenerate and $s'$ is a surjection. Then we obtain
the following commutative diagram: % https://q.uiver.app/?q=WzAsOCxbMSwwLCJcXERlbHRhXntcXHtyJyxcXGRvdHMscicnXFx9fSJdLFsyLDAsIlxcRGVsdGFee20tMX0iXSxbMSwxLCJcXG1hdGhjYWx7TX1eXFxvdGltZXMgIl0sWzIsMSwiTihcXG1hdGhzZntGaW59X1xcYXN0KVxcdGltZXMgXFxEZWx0YV5uLiJdLFswLDAsIlxcRGVsdGFee1xcezAsXFxkb3RzLHInJ1xcfX0iXSxbMSwyLCJcXERlbHRhXmwiXSxbMCwyLCJcXERlbHRhXntcXHtrJyxcXGRvdHNtLGtcXH19Il0sWzAsMSwiXFxEZWx0YV57a30iXSxbMCwxLCJ1XFx2ZXJ0XFxEZWx0YV57XFx7cicsXFxkb3RzLHInJ1xcfX0iLDAseyJzdHlsZSI6eyJoZWFkIjp7Im5hbWUiOiJlcGkifX19XSxbMSwzLCJcXHNpZ21hX2EnIl0sWzIsMywicCIsMl0sWzQsNywicyIsMix7InN0eWxlIjp7ImhlYWQiOnsibmFtZSI6ImVwaSJ9fX1dLFs3LDIsInkiLDJdLFswLDQsIiIsMCx7InN0eWxlIjp7InRhaWwiOnsibmFtZSI6Imhvb2siLCJzaWRlIjoiYm90dG9tIn19fV0sWzYsNywiIiwyLHsic3R5bGUiOnsidGFpbCI6eyJuYW1lIjoiaG9vayIsInNpZGUiOiJib3R0b20ifX19XSxbNiw1LCJzJyIsMix7InN0eWxlIjp7ImhlYWQiOnsibmFtZSI6ImVwaSJ9fX1dLFs2LDJdLFs1LDMsInoiLDJdLFs0LDIsInhcXHZlcnRcXERlbHRhXntcXHswLFxcZG90cyxyJydcXH19IiwxXV0=
\[\begin{tikzcd}
	{\Delta^{\{0,\dots,r''\}}} & {\Delta^{\{r',\dots,r''\}}} & {\Delta^{m-1}} \\
	{\Delta^{k}} & {\mathcal{M}^\otimes } & {N(\mathsf{Fin}_\ast)\times \Delta^n.} \\
	{\Delta^{\{k',\dotsm,k\}}} & {\Delta^l}
	\arrow["{u\vert\Delta^{\{r',\dots,r''\}}}", two heads, from=1-2, to=1-3]
	\arrow["{\sigma_a'}", from=1-3, to=2-3]
	\arrow["p"', from=2-2, to=2-3]
	\arrow["s"', two heads, from=1-1, to=2-1]
	\arrow["y"', from=2-1, to=2-2]
	\arrow[hook', from=1-2, to=1-1]
	\arrow[hook', from=3-1, to=2-1]
	\arrow["{s'}"', two heads, from=3-1, to=3-2]
	\arrow[from=3-1, to=2-2]
	\arrow["z"', from=3-2, to=2-3]
	\arrow["{x\vert\Delta^{\{0,\dots,r''\}}}"{description}, from=1-1, to=2-2]
\end{tikzcd}\]Applying Eilenberg-Zilber's lemma to $px\vert\Delta^{\{r',\dots,r''\}}$,
we deduce that $z=\sigma'_{a}$. Thus $y$ is one of the simplices
in $\{\tau_{\lambda}\}_{\lambda\in\Lambda}$, as required.
\item Suppose that $u$ is surjective on vertices and that the map $\theta:u\pr{r''}\to u\pr{r''+1}$
is not inert. By factoring the map $\theta$ into an inert map followed
by an active map, we can find an $\pr{r+1}$-simplex $y$ of $\cal M_{F_{\leq a}}^{\t}$
such that $d_{r''+1}y=x$ and $y\vert\Delta^{\{r'',r''+1,r''+2\}}$
is the chosen factorization of $\theta$. By the previous point, the
simplex $y$ belongs to $\cal N_{\leq\lambda}$ for some $\lambda$,
and hence so must $x$.
\item Suppose $u$ is not surjective on vertices. Find an inert map $\theta:u\pr r\to X$
over the last edge of $\sigma$, and let $y$ be an $\pr{r+1}$-simplex
$y$ of $\cal M_{F_{\leq a}}^{\t}$ such that $d_{r+1}y=x$ and $y\vert\Delta^{\{r,r+1\}}$
is equal to $\theta$. By the first point, the simplex $y$ belongs
to $\cal N_{\leq\lambda}$ for some $\lambda$, and hence so must
$x$.
\end{itemize}
This completes the proof of ($\ast$).

Next we prove ($\ast\ast$). First we show that the maps $K\to\cal N_{\leq\lambda}$
and $K_{0}\to\cal N_{<\lambda}$ are well-defined. A typical simplex
in the image of the map $K\to\cal M^{\t}$ has the form $x:\Delta^{r}\to\cal M^{\t}$,
where there is an integer $-1\leq r'\leq r$ such that $x\vert\Delta^{\{0,\dots,r'\}}$
factors through $\tau_{\lambda}$ and, if $r'<r$, then $p\pr{x\vert\Delta^{\{r',r'+1\}}}$
is inert and and $p\pr{x\vert\Delta^{\{r'+1,\dots,r\}}}$ is the constant
map at $\pr{\inp 1,n}$. (When $r'=-1$, the symbol $\Delta^{\{0,\dots,r'\}}$
denotes the empty simplicial set.) By choosing an inert map $x\pr{r'}\to X$
with $X\in\cal M$ if $r'<r$, or by factoring the map $x\pr{r'}\to x\pr{r'+1}$
into an inert map followed by an active map, we can find a simplex
$y$ belonging to $\cal N_{\leq\lambda}$ and satisfying $d_{r'+1}y=x$.
Hence $x$ belongs to $\cal N_{\leq\lambda}$, showing that the map
$K\to\cal N_{\leq\lambda}$ is well-defined. Next, for the map $K_{0}\to\cal N_{<\lambda}$,
assume further $x\vert\Delta^{\{0,\dots,r'\}}$ factors through the
boundary of $\tau_{\lambda}$. We must show that $x$ belongs to $\cal N_{<\lambda}$.
We have two cases to consider:
\begin{itemize}
\item Suppose that the head of the simplex $p\pr{x\vert\Delta^{\{0,\dots,r'\}}}$
factors through $\partial\sigma'_{a}$. Then the head of $p\pr x$
factors through $d_{i}\sigma$ for some $0\leq i<m$ for some associate
$\sigma$ of $\sigma'_{a}$, so $x$ belongs to $\cal M_{F_{<a}}^{\t}$
by ($\blacklozenge$). 
\item Suppose that the head of the simplex $p\pr{x\vert\Delta^{\{0,\dots,r'\}}}$
is a degeneration of $\sigma'_{a}$. Write $x\vert\Delta^{\{0,\dots,r'\}}=s^{*}y$,
where $y$ is nondegenerate and $s$ is a surjection. Eilenberg-Zilber's
lemma implies that the simplex $p\pr y$ is again a degeneration of
$\sigma'_{a}$, so $y=\tau_{\mu}$ for some $\mu\in\Lambda$. Since
$x$ factors through the boundary of $\tau_{\lambda}$, the dimension
of $y$ must be smaller than that of $\tau_{\lambda}$. Thus $\mu<\lambda$.
The above argument (that the map $K\to\cal N_{\leq\lambda}$ is well-defined)
shows that $y$ belongs to $\cal N_{\leq\mu}\subset\cal N_{<\lambda}$,
so $x$ belongs to $\cal N_{<\lambda}$.
\end{itemize}
This completes the verification of the first half of ($\ast\ast$).
We next proceed to the latter half: We show that the square % https://q.uiver.app/?q=WzAsNCxbMCwwLCJLXzAiXSxbMCwxLCJLIl0sWzEsMCwiXFxtYXRoY2Fse059X3s8XFxsYW1iZGF9Il0sWzEsMSwiXFxtYXRoY2Fse059X3tcXGxlcSBcXGxhbWJkYX0iXSxbMCwxXSxbMCwyXSxbMSwzXSxbMiwzXV0=
\[\begin{tikzcd}
	{K_0} & {\mathcal{N}_{<\lambda}} \\
	K & {\mathcal{N}_{\leq \lambda}}
	\arrow[from=1-1, to=2-1]
	\arrow[from=1-1, to=1-2]
	\arrow[from=2-1, to=2-2]
	\arrow[from=1-2, to=2-2]
\end{tikzcd}\]is cocartesian. Clearly $\cal N_{\leq\lambda}$ is the union of the
images of $K$ and $\cal N_{<\lambda}$. So it suffices to show that
if a simplex $z$ of $K$ is mapped into $\cal N_{<\lambda}$, then
$z$ belongs to $K_{0}$. Taking the contrapositive, we will show
that if $z$ does not belong to $K_{0}$, then its image in $\cal N_{\leq\lambda}$
does not belong to $\cal N_{<\lambda}$. Let $x:\Delta^{r}\to\cal N_{\leq\lambda}$
be the image of $x$. By the definition of the map $K\to\cal N_{\leq\lambda}$
and by the hypothesis that $z$ does not belong to $K_{0}$, there
is an integer $0\leq r'\leq r$ such that $x\vert\Delta^{\{0,\dots,r'\}}$
is a degeneration of $\tau_{\lambda}$ and, if $r'<r$, then $p\pr{x\vert\Delta^{\{r',r'+1\}}}$
is inert and and $p\pr{x\vert\Delta^{\{r'+1,\dots,r\}}}$ is the constant
map at $\pr{\inp 1,n}$. The head of $p\pr x$ is thus a degeneration
of either $\sigma'_{a}$ or its associate, and so it does not belong
to $F_{<a}$ by ($\blacklozenge\blacklozenge$). Therefore, $x$ does
not belong to $\cal M_{F_{<a}}^{\t}$. So should $x$ belong to $\cal N_{<\lambda}$,
then $x\vert\Delta^{\{0,\dots,r'\}}$ must factor through $\tau_{\mu}$
for some $\mu<\lambda$. This is impossible because $x\vert\Delta^{\{0,\dots,r'\}}$
is a degeneration of $\tau_{\lambda}$. Hence $x$ does not belong
to $\cal N_{<\lambda}$.

\item Suppose that $\sigma'_{a}$ belongs to $G'_{\pr 3}$. We will
show that the inclusion $\cal M_{<a}^{\t}\hookrightarrow\cal M_{\leq a}^{\t}$
is a weak categorical equivalence. The desired extension of $f_{<a}$
can then be found because the map $q$ is a categorical fibration.

Let $\sigma$ be the unique associate of $\sigma'_{a}$, which we
depict as 
\[
\pr{\inp{k_{0}},e_{0}}\xrightarrow{\alpha_{\sigma}\pr 1}\cdots\xrightarrow{\alpha_{\sigma}\pr m}\pr{\inp{k_{m}},e_{m}}.
\]
Choose integers $0<j<k\leq m$ such that $\alpha_{\sigma}\pr i$ is
strongly inert for $i=j$, active for $j<i\leq k$, and strongly inert
for $i>k$. Set $Y=\pr{N\pr{\Fin_{\ast}}\times\Delta^{n}}_{/\sigma}\times_{\Delta^{n}}\{0\}$.
Using ($\blacklozenge$), we may consider the following commutative
diagram: % https://q.uiver.app/?q=WzAsNCxbMCwwLCJZXFxzdGFyIFxcTGFtYmRhIF5tX2oiXSxbMCwxLCJZXFxzdGFyIFxcRGVsdGEgXm0iXSxbMSwwLCJcXG92ZXJsaW5le0ZfezxhfX0iXSxbMSwxLCJcXG92ZXJsaW5le0Zfe1xcbGVxIGF9fSJdLFswLDFdLFswLDJdLFsxLDNdLFsyLDNdXQ==
\[\begin{tikzcd}
	{Y\star \Lambda ^m_j} & {\overline{F_{<a}}} \\
	{Y\star \Delta ^m} & {\overline{F_{\leq a}}.}
	\arrow[from=1-1, to=2-1]
	\arrow[from=1-1, to=1-2]
	\arrow[from=2-1, to=2-2]
	\arrow[from=1-2, to=2-2]
\end{tikzcd}\]Using ($\blacklozenge\blacklozenge$), we deduce that this square
is cocartesian. Therefore, it suffices to show that the inclusion
\[
\pr{Y\star\Lambda_{j}^{m}}\times_{N\pr{\Fin_{\ast}}\times\Delta^{n}}\cal M^{\t}\to\pr{Y\star\Delta^{m}}\times_{N\pr{\Fin_{\ast}}\times\Delta^{n}}\cal M^{\t}
\]
is a weak categorical equivalence. We will prove more generally that
for any morphism $Y'\to Y$ of simplicial sets, the map 
\[
\eta_{Y'}:\pr{Y'\star\Lambda_{j}^{m}}\times_{N\pr{\Fin_{\ast}}\times\Delta^{n}}\cal M^{\t}\to\pr{Y'\star\Delta^{m}}\times_{N\pr{\Fin_{\ast}}\times\Delta^{n}}\cal M^{\t}
\]
is a weak categorical equivalence. The assignment $Y'\mapsto\eta_{Y'}$
defines a functor from $\SS/Y'$ to the arrow category $\SS^{[1]}$
which commutes with filtered colimits. Since weak categorical equivalences
are stable under filtered colimits, we may assume that $Y'$ is a
finite simplicial set. If $Y'$ is empty, the claim follows from \cite[Lemma 2.4.4.6]{HA}.
For the inductive step, we can find a pushout diagram % https://q.uiver.app/?q=WzAsNCxbMCwwLCJcXHBhcnRpYWxcXERlbHRhXnAiXSxbMSwwLCJYIl0sWzEsMSwiWSciXSxbMCwxLCJcXERlbHRhXnAiXSxbMCwxXSxbMSwyXSxbMCwzXSxbMywyXV0=
\[\begin{tikzcd}
	{\partial\Delta^p} & X \\
	{\Delta^p} & {Y'}
	\arrow[from=1-1, to=1-2]
	\arrow[from=1-2, to=2-2]
	\arrow[from=1-1, to=2-1]
	\arrow[from=2-1, to=2-2]
\end{tikzcd}\]in $\SS/Y$, such that the claim holds for $X$. The map $\eta_{Y'}$
factors as
\begin{align*}
\pr{Y'\star\Lambda_{j}^{m}}\times_{N\pr{\Fin_{\ast}}\times\Delta^{n}}\cal M^{\t} & \xrightarrow{\phi}\pr{Y'\star\Lambda_{j}^{m}\cup X\star\Delta^{m}}\times_{N\pr{\Fin_{\ast}}\times\Delta^{n}}\cal M^{\t}\\
 & \xrightarrow{\psi}\pr{Y'\star\Delta^{m}}\times_{N\pr{\Fin_{\ast}}\times\Delta^{n}}\cal M^{\t}.
\end{align*}
The map $\phi$ is a pushout of $\eta_{X}$, and hence is a trivial
cofibration. The map $\psi$ is a pushout of the inclusion
\[
\psi':\pr{\Delta^{p}\star\Lambda_{j}^{m}\cup\partial\Delta^{p}\star\Delta^{m}}\times_{N\pr{\Fin_{\ast}}\times\Delta^{n}}\cal M^{\t}\to\pr{\Delta^{p}\star\Delta^{m}}\times_{N\pr{\Fin_{\ast}}\times\Delta^{n}}\cal M^{\t}.
\]
Using the isomorphism $\Delta^{p}\star\Lambda_{j}^{m}\cup\partial\Delta^{p}\star\Delta^{m}\cong\Lambda_{p+1+j}^{m+p+1}$
and Lemma \cite[Lemma 2.4.4.6]{HA}, we deduce that $\psi'$ is a
weak categorical equivalence. Hence $\psi$ is a weak categorical
equivalence, completing the treatment of Case 2.

\end{enumerate}

It remains to construct a well-ordering on $A$ which satisfies ($\blacklozenge$).
For each $a\in A$, define integers $u_{\mathrm{neut}}\pr a,u_{\act}\pr a,u_{\mathrm{oc}}\pr a,u_{\mathrm{as}}\pr a$
as follows: Let $\sigma$ be an associate of $\sigma'_{a}$, and set
$\alpha_{\sigma}\pr i=\sigma\vert\Delta^{\{i-1,i\}}$ for $1\leq i\leq m$.
Then:
\begin{itemize}
\item $u_{\mathrm{neut}}\pr a$ is the number of integers $1\leq i\leq m$
such that $\alpha_{\sigma}\pr i$ is neutral.
\item $u_{\act}\pr a$ is the number of integers $1\leq i\leq m$ such that
$\alpha_{\sigma}\pr i$ is active.
\item $u_{\mathrm{oc}}\pr a$ is set equal to $0$ if $\sigma$ is closed,
and is set equal to $1$ if $\sigma$ is open.
\item $u_{\mathrm{as}}\pr a$ is the number of pairs of integers $1\leq i<j\leq m$
such that $\alpha_{\sigma}\pr i$ is active and $\alpha_{\sigma}\pr j$
is strictly inert.
\end{itemize}
We choose a well-ordering on $A$ so that the function 
\[
A\ni a\mapsto\pr{u_{\mathrm{neut}}\pr a,u_{\act}\pr a,u_{\mathrm{oc}}\pr a,u_{\mathrm{as}}\pr a}\in\bb Z^{4}
\]
is non-decreasing, where $\bb Z^{4}$ is equipped with the lexicographic
ordering. (Thus $\pr{n_{1},n_{2},n_{3},n_{4}}<\pr{n'_{1},n'_{2},n'_{3},n'_{4}}$
if and only if $n_{i}<n'_{i}$, where $i$ is the minimal integer
for which $n_{i}\neq n'_{i}$.) We will show that this ordering does
the job.

Let $a,\sigma,i$ be as in ($\blacklozenge$). We wish to show that
$d_{i}\sigma$ belongs to $F_{<a}$. If $d_{i}\sigma$ is incomplete
or degenerate, then it belongs to $F\pr{m-1}$ and we are done. So
assume that $d_{i}\sigma$ is complete and nondegenerate. Then $d_{i}\sigma$
belongs to (exactly) one of the sets $G_{\pr 1},G_{\pr 2},G'_{\pr 2},G_{\pr 3},G'_{\pr 3}$.
If $d_{i}\sigma$ belongs to $G_{\pr 1}\cup G_{\pr 2}\cup G_{\pr 3}$,
or if it belongs to $G'_{\pr 2}$ and has no associates, then it belongs
to $F''\pr m$ and we are done. So we will assume that $d_{i}\sigma$
belongs to $G'_{\pr 2}\cup G'_{\pr 3}$ and has an associate, so that
$d_{i}\sigma=\sigma'_{b}$ for some $b\in A$. We wish to show that
$b<a$.

We will make use of the following notations. Let $\alpha_{\sigma}\pr i=\sigma\vert\Delta^{\{i-1,i\}}$.
Let $0\leq k\leq m$ be the minimal integer such that $\alpha_{\sigma}\pr i$
is strongly inert for every $i>k$, and let $0\leq j\leq k$ be the
minimal integer such that $\alpha_{\sigma}\pr i$ is active for every
$j<i\leq k$. 

Suppose first that $\sigma'_{a}\in G'_{\pr 2}$. Our assumption on
$d_{i}\sigma$ implies that $i=k$ and that the composite $\alpha_{\sigma}\pr{k+1}\circ\alpha_{\sigma}\pr k$
is neutral. It follows that the associate $\tau$ of $d_{i}\sigma$
satisfies $\alpha_{\tau}\pr s=\alpha_{\sigma}\pr s$ for $s\neq k,k+1$,
$\alpha_{\tau}\pr i$ is strictly inert, and $\alpha_{\tau}\pr{k+1}$
is active. Thus $u_{\mathrm{neut}}\pr a=u_{\mathrm{neut}}\pr b$,
$u_{\act}\pr a=u_{\act}\pr b$, $u_{\mathrm{oc}}\pr a=u_{\mathrm{oc}}\pr b$,
and $u_{\mathrm{as}}\pr a>u_{\mathrm{as}}\pr b$. Hence $a>b$, as
required.

Suppose next that $\sigma'_{a}\in G'_{\pr 3}$ and that $d_{i}\sigma\in G'_{\pr 2}$. 
\begin{itemize}
\item If $u_{\mathrm{neut}}\pr a>0$, we are done, since $u_{\mathrm{neut}}\pr b=0$.
\item If $u_{\mathrm{neut}}\pr a=0$, then each $\alpha_{\sigma}\pr i$
is either active or strongly inert. It follows that $u_{\act}\pr a$
is not less than the number of active morphisms in $d_{i}\sigma$,
which is equal to $u_{\act}\pr b$. So $u_{\act}\pr a\geq u_{\act}\pr b$.
If $u_{\act}\pr a>u_{\act}\pr b$, we are done. 
\item If $u_{\mathrm{neut}}\pr a=0$ and $u_{\act}\pr a=u_{\act}\pr b$,
then $\sigma$ is open. Indeed, if $\sigma$ were closed, then $i=m$
since $d_{i}\sigma$ is open. But then $d_{i}\sigma$ must contain
a subsequence consisting of a strictly inert morphism followed by
an active morphism. This is impossible because $d_{i}\sigma$ belongs
to $G'_{\pr 2}$. Hence $u_{\mathrm{oc}}\pr a>u_{\mathrm{oc}}\pr b$
and we are done.
\end{itemize}
Finally, suppose that $\sigma'_{a}\in G'_{\pr 3}$ and that $d_{i}\sigma\in G'_{\pr 3}$.
Note that our assumption on $d_{i}\sigma$ forces $i=j-1$ or $i=k$.
\begin{itemize}
\item The number of neutral morphisms in $d_{i}\sigma$ is at most $u_{\mathrm{neut}}\pr a+1$,
so 
\begin{equation}
u_{\mathrm{neut}}\pr a\geq\#\{\text{neutral morphisms in }d_{i}\sigma\}-1=u_{\mathrm{neut}}\pr b.\label{eq:1}
\end{equation}
If $u_{\mathrm{neut}}\pr a>u_{\mathrm{neut}}\pr b$, we are done. 
\item Suppose $u_{\mathrm{neut}}\pr a=u_{\mathrm{neut}}\pr b$, so that
the equality holds in (\ref{eq:1}). Then $0<i<m$, the composite
$\alpha_{\sigma}\pr{i+1}\circ\alpha_{\sigma}\pr i$ is neutral, and
$\alpha_{\sigma}\pr i$ is active. ($\alpha_{\sigma}\pr i$ is necessarily
strictly inert since $i=j-1$ or $i=k$.) It follows that the associate
$\tau$ of $d_{i}\sigma$ satisfies $\alpha_{\tau}\pr s=\alpha_{\sigma}\pr s$
for $s\neq i,i+1$, $\alpha_{\tau}\pr i$ is strictly inert, and $\alpha_{\tau}\pr{i+1}$
is active. Thus $u_{\mathrm{neut}}\pr a=u_{\mathrm{neut}}\pr b$,
$u_{\act}\pr a=u_{\act}\pr b$, $u_{\mathrm{oc}}\pr a=u_{\mathrm{oc}}\pr b$,
and $u_{\mathrm{as}}\pr a>u_{\mathrm{as}}\pr b$. Hence $a>b$, as
desired.
\end{itemize}
This completes the proof of ($\blacklozenge$) and hence the proof
of the theorem.

\end{enumerate}
\end{proof}
%% LyX 2.3.6.2 created this file.  For more info, see http://www.lyx.org/.
%% Do not edit unless you really know what you are doing.
\documentclass[a4paper]{amsart}
\usepackage[T1]{fontenc}
\usepackage{amstext}
\usepackage{amsthm}
\usepackage{amssymb}
\usepackage{stackrel}
\usepackage[unicode=true,pdfusetitle,
 bookmarks=true,bookmarksnumbered=false,bookmarksopen=false,
 breaklinks=false,pdfborder={0 0 0},pdfborderstyle={},backref=false,colorlinks=false]
 {hyperref}

\makeatletter

%%%%%%%%%%%%%%%%%%%%%%%%%%%%%% LyX specific LaTeX commands.
\special{papersize=\the\paperwidth,\the\paperheight}


%%%%%%%%%%%%%%%%%%%%%%%%%%%%%% Textclass specific LaTeX commands.
\numberwithin{equation}{section}
\numberwithin{figure}{section}
\theoremstyle{plain}
\newtheorem{thm}{\protect\theoremname}[section]
\theoremstyle{plain}
\newtheorem{prop}[thm]{\protect\propositionname}
\theoremstyle{remark}
\newtheorem{rem}[thm]{\protect\remarkname}
\ifx\proof\undefined
\newenvironment{proof}[1][\protect\proofname]{\par
	\normalfont\topsep6\p@\@plus6\p@\relax
	\trivlist
	\itemindent\parindent
	\item[\hskip\labelsep\scshape #1]\ignorespaces
}{%
	\endtrivlist\@endpefalse
}
\providecommand{\proofname}{Proof}
\fi
\theoremstyle{plain}
\newtheorem{cor}[thm]{\protect\corollaryname}
\theoremstyle{plain}
\newtheorem{lem}[thm]{\protect\lemmaname}
\theoremstyle{definition}
\newtheorem{defn}[thm]{\protect\definitionname}

%%%%%%%%%%%%%%%%%%%%%%%%%%%%%% User specified LaTeX commands.
\usepackage{tikz-cd}
\usepackage{mathtools}
\usepackage{adjustbox}
\usepackage{enumitem}
\tikzcdset{scale cd/.style={every label/.append style={scale=#1},
    cells={nodes={scale=#1}}}}
\usepackage{eucal}

\makeatother


\providecommand{\corollaryname}{Corollary}
\providecommand{\definitionname}{Definition}
\providecommand{\lemmaname}{Lemma}
\providecommand{\propositionname}{Proposition}
\providecommand{\remarkname}{Remark}
\providecommand{\theoremname}{Theorem}

\begin{document}
\global\long\def\sf#1{\mathsf{#1}}%

\global\long\def\scr#1{\mathscr{{#1}}}%

\global\long\def\cal#1{\mathcal{#1}}%

\global\long\def\bb#1{\mathbb{#1}}%

\global\long\def\bf#1{\mathbf{#1}}%

\global\long\def\frak#1{\mathfrak{#1}}%

\global\long\def\fr#1{\mathfrak{#1}}%

\global\long\def\u#1{\underline{#1}}%

\global\long\def\tild#1{\widetilde{#1}}%

\global\long\def\mrm#1{\mathrm{#1}}%

\global\long\def\pr#1{\left(#1\right)}%

\global\long\def\abs#1{\left|#1\right|}%

\global\long\def\inp#1{\left\langle #1\right\rangle }%

\global\long\def\br#1{\left\{  #1\right\}  }%

\global\long\def\norm#1{\left\Vert #1\right\Vert }%

\global\long\def\hat#1{\widehat{#1}}%

\global\long\def\opn#1{\operatorname{#1}}%

\global\long\def\bigmid{\,\middle|\,}%

\global\long\def\Top{\sf{Top}}%

\global\long\def\Set{\sf{Set}}%

\global\long\def\SS{\sf{sSet}}%

\global\long\def\Kan{\sf{Kan}}%

\global\long\def\Cat{\mathcal{C}\sf{at}}%

\global\long\def\imfld{\cal M\mathsf{fld}}%

\global\long\def\ids{\cal D\sf{isk}}%

\global\long\def\ich{\cal C\sf h}%

\global\long\def\SW{\mathcal{SW}}%

\global\long\def\SHC{\mathcal{SHC}}%

\global\long\def\B{\sf B}%

\global\long\def\Spaces{\sf{Spaces}}%

\global\long\def\Mod{\sf{Mod}}%

\global\long\def\Nec{\sf{Nec}}%

\global\long\def\Fin{\sf{Fin}}%

\global\long\def\Ch{\sf{Ch}}%

\global\long\def\Ab{\sf{Ab}}%

\global\long\def\SA{\sf{sAb}}%

\global\long\def\P{\mathsf{POp}}%

\global\long\def\Op{\mathcal{O}\mathsf{p}}%

\global\long\def\Opg{\mathcal{O}\mathsf{p}_{\infty}^{\mathrm{gn}}}%

\global\long\def\Tup{\mathsf{Tup}}%

\global\long\def\Del{\mathbf{\Delta}}%

\global\long\def\id{\operatorname{id}}%

\global\long\def\Aut{\operatorname{Aut}}%

\global\long\def\End{\operatorname{End}}%

\global\long\def\Hom{\operatorname{Hom}}%

\global\long\def\Ext{\operatorname{Ext}}%

\global\long\def\sk{\operatorname{sk}}%

\global\long\def\ihom{\underline{\operatorname{Hom}}}%

\global\long\def\N{\mathrm{N}}%

\global\long\def\-{\text{-}}%

\global\long\def\op{\mathrm{op}}%

\global\long\def\To{\Rightarrow}%

\global\long\def\rr{\rightrightarrows}%

\global\long\def\rl{\rightleftarrows}%

\global\long\def\mono{\rightarrowtail}%

\global\long\def\epi{\twoheadrightarrow}%

\global\long\def\comma{\downarrow}%

\global\long\def\ot{\leftarrow}%

\global\long\def\corr{\leftrightsquigarrow}%

\global\long\def\lim{\operatorname{lim}}%

\global\long\def\colim{\operatorname{colim}}%

\global\long\def\holim{\operatorname{holim}}%

\global\long\def\hocolim{\operatorname{hocolim}}%

\global\long\def\Ran{\operatorname{Ran}}%

\global\long\def\Lan{\operatorname{Lan}}%

\global\long\def\Sk{\operatorname{Sk}}%

\global\long\def\Sd{\operatorname{Sd}}%

\global\long\def\Ex{\operatorname{Ex}}%

\global\long\def\Cosk{\operatorname{Cosk}}%

\global\long\def\Sing{\operatorname{Sing}}%

\global\long\def\Sp{\operatorname{Sp}}%

\global\long\def\Spc{\operatorname{Spc}}%

\global\long\def\Ho{\operatorname{Ho}}%

\global\long\def\Fun{\operatorname{Fun}}%

\global\long\def\map{\operatorname{map}}%

\global\long\def\diag{\operatorname{diag}}%

\global\long\def\Gap{\operatorname{Gap}}%

\global\long\def\cc{\operatorname{cc}}%

\global\long\def\Ob{\operatorname{Ob}}%

\global\long\def\Map{\operatorname{Map}}%

\global\long\def\Rfib{\operatorname{RFib}}%

\global\long\def\Lfib{\operatorname{LFib}}%

\global\long\def\Tw{\operatorname{Tw}}%

\global\long\def\Equiv{\operatorname{Equiv}}%

\global\long\def\Arr{\operatorname{Arr}}%

\global\long\def\Cyl{\operatorname{Cyl}}%

\global\long\def\Path{\operatorname{Path}}%

\global\long\def\Alg{\operatorname{Alg}}%

\global\long\def\ho{\operatorname{ho}}%

\global\long\def\Comm{\operatorname{Comm}}%

\global\long\def\Triv{\operatorname{Triv}}%

\global\long\def\triv{\operatorname{triv}}%

\global\long\def\Env{\operatorname{Env}}%

\global\long\def\Act{\operatorname{Act}}%

\global\long\def\act{\operatorname{act}}%

\global\long\def\loc{\operatorname{loc}}%

\global\long\def\Assem{\operatorname{Assem}}%

\global\long\def\Nat{\operatorname{Nat}}%

\global\long\def\lax{\mathrm{lax}}%

\global\long\def\weq{\mathrm{weq}}%

\global\long\def\fib{\mathrm{fib}}%

\global\long\def\cof{\mathrm{cof}}%

\global\long\def\inj{\mathrm{inj}}%

\global\long\def\univ{\mathrm{univ}}%

\global\long\def\Ker{\opn{Ker}}%

\global\long\def\Coker{\opn{Coker}}%

\global\long\def\Im{\opn{Im}}%

\global\long\def\Coim{\opn{Im}}%

\global\long\def\coker{\opn{coker}}%

\global\long\def\im{\opn{\mathrm{im}}}%

\global\long\def\coim{\opn{coim}}%

\global\long\def\gn{\mathrm{gn}}%

\global\long\def\Mon{\mathrm{Mon}}%

\global\long\def\Un{\mathrm{Un}}%

\global\long\def\St{\mathrm{St}}%

\global\long\def\CA{\operatorname{CAlg}}%

\global\long\def\rd{\mathrm{rd}}%

\global\long\def\xmono#1#2{\stackrel[#2]{#1}{\rightarrowtail}}%

\global\long\def\xepi#1#2{\stackrel[#2]{#1}{\twoheadrightarrow}}%

\global\long\def\adj{\stackrel[\longleftarrow]{\longrightarrow}{\bot}}%

\global\long\def\btimes{\boxtimes}%

\global\long\def\ps#1#2{\prescript{}{#1}{#2}}%

\global\long\def\ups#1#2{\prescript{#1}{}{#2}}%

\global\long\def\hofib{\mathrm{hofib}}%

\global\long\def\cofib{\mathrm{cofib}}%

\global\long\def\Vee{\bigvee}%

\global\long\def\w{\wedge}%

\global\long\def\t{\otimes}%

\global\long\def\bp{\boxplus}%

\global\long\def\rcone{\triangleright}%

\global\long\def\lcone{\triangleleft}%

\global\long\def\S{\mathsection}%

\global\long\def\p{\prime}%
 

\global\long\def\pp{\prime\prime}%

\global\long\def\W{\overline{W}}%

\global\long\def\o#1{\overline{#1}}%

\title[Monoidal Envelopes of Families and Operadic
Kan Extensions]{Monoidal Envelopes of Families of $\infty$-Operads and $\infty$-Operadic
Kan Extensions}
\author{Kensuke Arakawa}
\email{arakawa.kensuke.22c@st.kyoto-u.ac.jp}
\address{Department of Mathematics, Kyoto University, Kyoto, 606-8502, Japan}
\subjclass[2020]{18N70, 55P48}
\begin{abstract}
We provide details of the proof of Lurie's theorem on operadic Kan
extensions \cite[Theorem 3.1.2.3]{HA}. Along the way, we generalize
the construction of monoidal envelopes of $\infty$-operads to families
of $\infty$-operads and use it to construct the fiberwise direct
sum functor, both of which we characterize by certain universal properties.
Aside from their uses in the proof of Lurie's theorem, these results
and constructions have their independent interest.
\end{abstract}

\maketitle
\tableofcontents{}

\section*{Introduction}

In his book \cite{HA}, Lurie introduces (among other things) the
notion of $\infty$-operads, an $\infty$-categorical analog of (colored)
operads. Given an $\infty$-operad $\cal O^{\t}$ and a symmetric
monoidal $\infty$-category $\cal C^{\t}$, we can form the $\infty$-category
$\Alg_{\cal O}\pr{\cal C}$ of $\cal O$-algebras. If $f:\cal O^{\t}\to\cal O^{\p\t}$
is a morphism of $\infty$-operads, there is the induced forgetful
map $f^{*}:\Alg_{\cal O'}\pr{\cal C}\to\Alg_{\cal O}\pr{\cal C}$.
In analogy with restrictions and extensions of scalars, it is natural
to ask whether the map $f^{*}$ has a left adjoint. Lurie gives an
affirmative answer to this question in \cite[Corollary 3.1.3.5]{HA},
provided that $\cal C$ has enough colimits. Because of its fundamental
importance, the result has been applied repeatedly in his work. 

Lurie's proof of \cite[Corollary 3.1.3.5]{HA} relies on another major
theorem on operadic Kan extensions \cite[Theorem 3.1.2.3]{HA} (or
Theorem \ref{thm:3.1.2.3}), which we call the \textbf{fundamental
theorem of operadic Kan extensions}, or FTOK for short. The proof
of FTOK, as Lurie himself acknowledges, is very long. Perhaps because
of the length of the proof, he omits some crucial details of the proof.
This note aims to provide all the details by making necessary constructions
and proving results on them. (In particular, we do not aim to provide
a more concise proof of FTOK than the one provided in \cite{HA}.)

We can roughly divide the details we provide into two parts: The first
one is the datum of ``coherent homotopy.'' In the proof of \cite[Theorem 3.1.2.3]{HA}
(to be more precise, on p. 337), Lurie claims that certain diagrams
are ``equivalent'' without writing the actual equivalence. Such
a practice is fairly common in the literature and is understandable
to some extent. However, in our case, we must be more attentive because
a considerable amount of effort is required to write down the actual
equivalence. We thus give a complete treatment of the equivalence
in this note. (See around Step 1 of the proof of Theorem \ref{thm:3.1.2.3}.)
The second missing detail is the verification that all the combinatorics
fits together. (This corresponds to Step 3 of Theorem \ref{thm:3.1.2.3}.)
Such detail, again, could be left to the reader if the verification
is trivial. But in our case, the verification is so complicated that
a sizable portion of readers will benefit from having it written down.
We thus record every single detail of the verification. 

Here is an outline of this note. In Section \ref{sec:A-Result-on},
we will prove an equivalent formulation of families of $\infty$-operads.
In Section \ref{sec:Monoidal-Envelopes-of}, we generalize Lurie's
monoidal envelopes to families of $\infty$-operads. Using monoidal
envelopes, we can define the fiberwise direct sum functor, whose properties
we discuss in Section \ref{sec:Direct_sum}. The constructions and
results in Sections \ref{sec:A-Result-on}, \ref{sec:Monoidal-Envelopes-of},
and \ref{sec:Direct_sum} will be used in writing down the equivalence
discussed in the previous paragraph, but they also have some independent
interest. To the author's knowledge, these constructions and results
have not appeared elsewhere. Finally, in Section \ref{sec:FTOK},
we will give a complete proof of FTOK.

\section*{Notation and Terminology}

Our notation and terminology mostly follow those of \cite{HA}. Here
are some deviations.
\begin{itemize}
\item We will say that a morphism of simplicial sets is \textbf{final} if
it is cofinal in the sense of \cite{HTT}, and \textbf{initial} if
its opposite is final. 
\item We will write $\Delta^{-1}$ for the empty simplicial set. 
\item If $\cal C$ is a category and $\alpha$ is an ordinal (or more generally
a well-ordered set), then an \textbf{$\alpha$-sequence} in $\cal C$
is a functor $F:\alpha\to\cal C$ such that the map $\colim_{\beta<\lambda}F\beta\to F\beta$
is an isomorphism for each limit ordinal less than $\alpha$. 
\item If $X$ is a simplicial set, then we denote the cone point of the
simplicial set $X^{\rcone}$ by $\infty$.
\end{itemize}

\section{\label{sec:A-Result-on}A Result on Families of $\infty$-Operads}

Let $\cal C$ be an $\infty$-category. Recall that a $\cal C$\textbf{-family
of $\infty$-operads} \cite[Definition 2.3.2.1]{HA} is a categorical
fibration $p:\cal M^{\t}\to\cal C\times N\pr{\Fin_{\ast}}$ satisfying
the following conditions:
\begin{itemize}
\item [(a)]For each object $M\in\cal M^{\t}$ with image $\pr{C,\inp m}\in\cal C\times N\pr{\Fin_{\ast}}$
and for each inert map $\alpha:\inp m\to\inp n$ in $N\pr{\Fin_{\ast}}$,
the morphism $\pr{\id_{C},\alpha}$ admits a $p$-cocartesian lift.
\item [(b)]Let $M\in\cal M^{\t}$ be an object with image $\pr{C,\inp m}\in\cal C\times N\pr{\Fin_{\ast}}$,
where $m\geq1$. Choose for each $1\leq i\leq m$ a $p$-cocartesian
lift $f_{i}:M\to M_{i}$ over $\pr{\id_{C},\rho^{i}}:\pr{C,\inp n}\to\pr{C,\inp 1}$.
Then the morphisms $f_{i}$ form a $p$-limit cone. Every object in
$\cal M_{\inp 0}^{\t}$ is $p$-terminal.
\item [(c)]Let $n\geq1$ and $C\in\cal C$. Given objects $M_{1},\dots,M_{n}\in\cal M_{C}$,
there is an object $M\in\cal M^{\t}$ lying over $\pr{C,\inp n}$
which admits $p$-cocartesian morphisms $M\to M_{i}$ over $\pr{\id_{C},\rho^{i}}:\pr{C,\inp n}\to\pr{C,\inp 1}$.
\end{itemize}
The goal of this section is to prove the following equivalent formulation
of families of $\infty$-operads:
\begin{prop}
\label{prop:sec1_main}Let $\cal C$ be an $\infty$-category and
let $p:\cal M^{\t}\to\cal C\times N\pr{\Fin_{\ast}}$ be a categorical
fibration satisfying conditions (a) and (b) above. Then the following
conditions are equivalent:
\begin{itemize}
\item [(c-i)]The map $p$ satisfies condition (c).
\item [(c-ii)]For each $1\leq i\leq n$, let $\rho_{!}^{i}:\cal M_{\inp n}^{\t}\to\cal M_{\inp 1}^{\t}$
be the functor over $\cal C$ induced by the morphism $\rho^{i}$.
Then for each $n\geq1$, the functor
\[
\pr{\rho_{!}^{i}}_{1\leq i\leq n}:\cal M_{\inp n}^{\t}\to\cal M\times_{\cal C}\cdots\times_{\cal C}\cal M
\]
is an equivalence of $\infty$-categories. 
\end{itemize}
\end{prop}
Here the functor $\rho_{!}^{i}:\cal M_{\inp n}^{\t}\to\cal M$ is
obtained in the following way: Since every inert morphism in $\cal C\times N\pr{\Fin_{\ast}}$
admits a $p$-cocartesian lift, it is possible to choose a natural
transformation $\cal M_{\inp n}^{\t}\times\Delta^{1}\to\cal M^{\t}$
fitting into the commutative diagram % https://q.uiver.app/?q=WzAsNCxbMSwwLCIoXFxtYXRoY2Fse019Xlxcb3RpbWVzICxcXG1hdGhjYWx7RX0pIl0sWzEsMSwiXFxtYXRoY2Fse0N9XlxcZmxhdFxcdGltZXMgTihcXG1hdGhzZntGaW59X1xcYXN0KV5cXG5hdHVyYWwsIl0sWzAsMSwiKFxcbWF0aGNhbHtNfV5cXG90aW1lcyBfe1xcbGFuZ2xlIG5cXHJhbmdsZX0pXlxcZmxhdFxcdGltZXMgKFxcRGVsdGFeMSleXFxzaGFycCJdLFswLDAsIihcXG1hdGhjYWx7TX1eXFxvdGltZXMgX3tcXGxhbmdsZSBuXFxyYW5nbGV9KV5cXGZsYXRcXHRpbWVzIFxcezBcXH0iXSxbMiwxLCJwXzJcXHRpbWVzIFxccmhvXmkiLDJdLFswLDFdLFszLDBdLFszLDJdLFsyLDAsIiIsMSx7InN0eWxlIjp7ImJvZHkiOnsibmFtZSI6ImRhc2hlZCJ9fX1dXQ==
\[\begin{tikzcd}
	{(\mathcal{M}^\otimes _{\langle n\rangle})^\flat\times \{0\}} & {(\mathcal{M}^\otimes ,\mathcal{E})} \\
	{(\mathcal{M}^\otimes _{\langle n\rangle})^\flat\times (\Delta^1)^\sharp} & {\mathcal{C}^\flat\times N(\mathsf{Fin}_\ast)^\natural,}
	\arrow["{p_2\times \rho^i}"', from=2-1, to=2-2]
	\arrow[from=1-2, to=2-2]
	\arrow[from=1-1, to=1-2]
	\arrow[from=1-1, to=2-1]
	\arrow[dashed, from=2-1, to=1-2]
\end{tikzcd}\]where $\cal E$ denotes the set of $p$-cocartesian edges over inert
morphisms in $\cal C\times N\pr{\Fin_{\ast}}$ whose image in $\cal C$
is degenerate, and $N\pr{\Fin_{\ast}}^{\natural}$ is the marked simplicial
set whose underlying simplicial set is $N\pr{\Fin_{\ast}}$ and with
inert morphisms marked. We understand that $\rho_{!}^{i}$ is the
restriction of the filler, so that it is a functor over $\cal C$
and its homotopy class over $\cal C$ is well-defined.
\begin{rem}
A reader conversant with the language of generalized $\infty$-operads
will recognize Proposition \ref{prop:sec1_main} as \cite[Proposition 2.3.2.11]{HA}.
However, in general, the diagram appearing in (2) of \cite[Definition 2.3.2.1]{HA}
is not strictly commutative but is commutative up to a specified natural
equivalence. So Proposition \ref{prop:sec1_main} is slightly different
from the cited proposition and should be regarded as its ``strictified
version.''
\end{rem}
The hardest part of the proof is the rectification of cocartesian
fibrations. We will overcome this problem by appealing to the simplicial
Grothendieck construction, which we explain in subsection \ref{subsec:Simplicial-Grothendieck-Construction}.
Proposition \ref{prop:sec1_main} will then be proved in subsection
\ref{subsec:Main-Result}.

\subsection{\label{subsec:Simplicial-Grothendieck-Construction}Simplicial Grothendieck
Construction}

In this subsection, we describe a convenient method to construct a
cocartesian fibration out of a functor $F:\cal C\to\sf{Cat}_{\Delta}$
of ordinary categories.

Let $\cal C$ be an ordinary category and $F:\cal C\to\sf{Cat}_{\Delta}$
a projectively fibrant functor. The (\textbf{simplicial}) \textbf{Gronthendieck
construction} $\int F$ is the simplicial category which is defined
as follows:
\begin{enumerate}
\item The set of objects of $F$ is given by $\coprod_{C\in\cal C}\opn{ob}F\pr C$.
\item Let $C,C'\in\cal C$ and $X\in F\pr C$, $X'\in F\pr{C'}$. Then the
hom-simplicial sets is 
\[
\pr{\int F}\pr{X,X'}=\coprod_{f\in\cal C\pr{C,C'}}F\pr{C'}\pr{Ff\pr X,X'}.
\]
\end{enumerate}
Composition is given by the composites 
\begin{align*}
F\pr{C''}\pr{Fg\pr{X'},X''}\times F\pr{C'}\pr{Ff\pr X,X'} & \to F\pr{C''}\pr{Fg\pr{X'},X''}\times F\pr{C''}\pr{FgFf\pr X,FgX'}\\
 & \xrightarrow{\circ}F\pr{C''}\pr{F\pr{gf}\pr X,X''}.
\end{align*}
This construction defines a functor
\[
\int:\Fun\pr{\cal C,\sf{Cat}_{\Delta}}\to\pr{\sf{Cat}_{\Delta}}_{/\cal C}.
\]

The Grothendieck construction is closely related to Lurie's relative
nerve functor \cite[$\S$ 3.2.5]{HTT}:
\begin{prop}
\label{prop:relative_nerve_of_Gr_const}Let $\cal C$ be a category
and $F:\cal C\to\sf{Cat}_{\Delta}$ a functor. There is an isomorphism
of simplicial sets 
\[
N\pr{\int F}\cong N_{N\circ F}\pr{\cal C}
\]
between the homotopy coherent nerve of $\int F$ and the relative
nerve of the composite $N\circ F:\cal C\to\sf{Cat}_{\Delta}\xrightarrow{N}\SS$,
which commutes with the projection to $N\pr{\cal C}$. The isomorphism
is natural in $F$.
\end{prop}
\begin{proof}
An $n$-simplex of the relative nerve $N_{N\circ F}\pr{\cal C}$,
by definition, consists of an $n$-simplex $\sigma=C_{0}\xrightarrow{f_{1}}\cdots\xrightarrow{f_{n}}C_{n}$
in $N\pr{\cal C}$ and a simplicial functor $\varphi_{I}:\fr C[\Delta^{I}]\to F\pr{C_{\max I}}$
for each subset $I\subset[n]$, such that for any inclusions $I\subset J\subset[n]$
of subsets, the diagram % https://q.uiver.app/?q=WzAsNCxbMCwwLCJcXG1hdGhmcmFre0N9W1xcRGVsdGFeSV0iXSxbMSwwLCJGKENfe1xcbWF4IEl9KSJdLFsxLDEsIkYoQ197XFxtYXggSn0pIl0sWzAsMSwiXFxtYXRoZnJha3tDfVtcXERlbHRhXkpdIl0sWzAsMV0sWzEsMl0sWzAsM10sWzMsMl1d
\[\begin{tikzcd}
	{\mathfrak{C}[\Delta^I]} & {F(C_{\max I})} \\
	{\mathfrak{C}[\Delta^J]} & {F(C_{\max J})}
	\arrow[from=1-1, to=1-2]
	\arrow[from=1-2, to=2-2]
	\arrow[from=1-1, to=2-1]
	\arrow[from=2-1, to=2-2]
\end{tikzcd}\]commutes. Given such an $n$-simplex $\pr{\sigma,\pr{x_{I}}_{I}}$,
we define a simplicial functor $\varphi:\fr C[\Delta^{n}]\to\int F$
as follows:
\begin{itemize}
\item For each integer $0\leq i\leq n$, we set $\varphi\pr i=\varphi_{[i]}\pr i$.
\item For each pair of integers $0\leq i\leq j\leq n$, the map
\[
\varphi:\fr C[\Delta^{n}]\pr{i,j}\to\pr{\int F}\pr{\varphi_{[i]}\pr i,\varphi_{[i]}\pr j}
\]
is the composite 
\begin{align*}
\fr C[\Delta^{n}]\pr{i,j} & \xrightarrow{\varphi_{[j]}}F\pr{C_{j}}\pr{\varphi_{[j]}\pr i,\varphi_{[j]}\pr j}\\
 & =F\pr{C_{j}}\pr{F\pr{f_{ij}}\pr{\varphi_{[i]}\pr i},\varphi_{[j]}\pr j}\\
 & \hookrightarrow\pr{\int F}\pr{\varphi_{[i]}\pr i,\varphi_{[i]}\pr j}.
\end{align*}
Here $f_{ij}:C_{i}\to C_{j}$ is the composite $f_{j}\cdots f_{i+1}$. 
\end{itemize}
The assignment $\pr{\sigma,\pr{x_{I}}_{I}}\mapsto\varphi$ determines
a morphism of simplicial sets
\[
\Phi:N_{N\circ F}\pr{\cal C}\to N\pr{\int F}
\]
over $N\pr{\cal C}$. We claim that it is an isomorphism of simplicial
sets by constructing an inverse. 

Let $\varphi:\fr C[\Delta^{n}]\to\int F$ be a simplicial functor.
Its projection to $\cal C$ determines an $n$-simplex $\sigma=C_{0}\xrightarrow{f_{1}}\cdots\xrightarrow{f_{n}}C_{n}$
in $N\pr{\cal C}$. For each subinterval $I\subset[n]$, define a
simplicial functor $\varphi_{I}:\fr C[\Delta^{I}]\to F\pr{C_{\max I}}$
as follows:
\begin{itemize}
\item For each integer $i\in I$, we set $\varphi_{I}\pr i=Ff_{i,\max I}\pr{\varphi\pr i}$.
\item For each pair of integers $i,j\in I$ with $i\le j$, the map
\[
\varphi_{I}:\fr C[\Delta^{I}]\pr{i,j}\to F\pr{C_{\max I}}\pr{\varphi_{I}\pr i,\varphi_{I}\pr j}
\]
is the composite
\begin{align*}
\fr C[\Delta^{I}]\pr{i,j} & =\fr C[\Delta^{n}]\pr{i,j}\\
 & \xrightarrow{\varphi}F\pr{C_{j}}\pr{F\pr{f_{ij}}\pr{\varphi\pr i},\varphi\pr j}\\
 & \xrightarrow{Ff_{j,\max I}}F\pr{C_{\max I}}\pr{F\pr{f_{i,\max I}}\pr{\varphi\pr i},F\pr{f_{j,\max I}}\pr{\varphi\pr j}}\\
 & =F\pr{C_{\max I}}\pr{\varphi_{I}\pr i,\varphi_{I}\pr j}.
\end{align*}
\end{itemize}
Next, for an arbitrary subset $J\subset[n]$, we define a simplicial
functor $\varphi_{J}:\fr C[\Delta^{J}]\to F\pr{C_{\max J}}$ to be
the restriction of $\varphi_{[\max J]}$. Then the pair $\pr{\sigma,\pr{\varphi_{I}}_{I}}$
is an $n$-simplex of $N_{N\circ F}\pr{\cal C}$, and the assignment
$\varphi\mapsto\pr{\sigma,\pr{\varphi_{I}}_{I}}$ determines an inverse
of $\Phi$.
\end{proof}
\begin{cor}
\label{cor:sgr}Let $\cal C$ be a category and $F:\cal C\to\sf{Cat}_{\Delta}$
a projectively fibrant functor. Then the functor 
\[
p:N\pr{\int F}\to N\pr{\cal C}
\]
is a cocartesian fibration. A morphism $\pr{f,g}:\pr{C,X}\to\pr{D,Y}$
in $N\pr{\int F}$ is $p$-cocartesian if and only if the morphism
$g:\pr{Ff}\pr X\to Y$ of $N\pr{F\pr D}$ is an equivalence.
\end{cor}
\begin{proof}
Define $\widetilde{F}:\cal C\to\SS^{+}$ by $\widetilde{F}C=N\pr{FC}^{\natural}$,
the homotopy coherent nerve of $FC$ with equivalences marked. Then
$\widetilde{F}$ is a projectively fibrant functor, so by \cite[Proposition 3.2.5.18]{HTT}
its relative nerve $N_{\widetilde{F}}^{+}\pr{\cal C}\to N\pr{\cal C}^{\sharp}$
is a fibrant object of $\SS^{+}/N\pr{\cal C}$ equipped with the cocartesian
model structure. The claim now follows from Proposition \ref{prop:relative_nerve_of_Gr_const}.
\end{proof}
\begin{rem}
The advantage of working with the simplicial Grothendieck construction
is that we can be very explicit about the functors induced by cocartesian
fibrations. Recall that if $p:\cal A\to\cal B$ is a cocartesian fibration
of $\infty$-categories and $f:X\to Y$ is a morphism in $\cal B$,
the induced functor $f_{!}:\cal A_{X}=\cal A\times_{\cal B}\{X\}\to\cal A_{Y}$
is defined by choosing a cocartesian natural transformation $\cal A_{X}\times\Delta^{1}\to\cal B$
fitting into the commutative diagram % https://q.uiver.app/?q=WzAsNSxbMCwwLCJcXG1hdGhjYWx7QX1fWFxcdGltZXMgXFx7MFxcfSJdLFswLDEsIlxcbWF0aGNhbHtBfV9YXFx0aW1lcyBcXERlbHRhXjEiXSxbMiwwLCJcXG1hdGhjYWx7QX0iXSxbMiwxLCJcXG1hdGhjYWx7Qn0iXSxbMSwxLCJcXERlbHRhXjEiXSxbMCwxXSxbMCwyXSxbMiwzXSxbMSwyLCIiLDEseyJzdHlsZSI6eyJib2R5Ijp7Im5hbWUiOiJkYXNoZWQifX19XSxbMSw0XSxbNCwzLCJmIiwyXV0=
\[\begin{tikzcd}
	{\mathcal{A}_X\times \{0\}} && {\mathcal{A}} \\
	{\mathcal{A}_X\times \Delta^1} & {\Delta^1} & {\mathcal{B}}
	\arrow[from=1-1, to=2-1]
	\arrow[from=1-1, to=1-3]
	\arrow[from=1-3, to=2-3]
	\arrow[dashed, from=2-1, to=1-3]
	\arrow[from=2-1, to=2-2]
	\arrow["f"', from=2-2, to=2-3]
\end{tikzcd}\]and then restricting it to $\cal A_{X}\times\{1\}$. 

In the case $p$ is the nerve of a simplicial functor of the form
$\int F\to\cal C$, given a morphism $f:C\to C'$ in $\cal C$, the
induced functor $f_{!}:N\pr{FC}\to N\pr{FC'}$ is nothing but $Nf$.
Indeed, the collection
\[
\{\id_{FfX}:FfX\to FfX\}_{X\in FC}
\]
defines a simplicial natural transformation from the inclusion $FC\hookrightarrow\int F$
to the composite $FC\xrightarrow{Ff}FC'\hookrightarrow\int F$, and
the nerve of this simplicial natural transformation gives us a natural
transformation $N\pr{FC}\times\Delta^{1}\to N\pr{\int F}$ over the
morphism $f$ which is pointwise cocartesian by Corollary \ref{cor:sgr}.
\end{rem}

\subsection{\label{subsec:Main-Result}Proof of Proposition \ref{prop:sec1_main}}

In this subsection, we prove Proposition \ref{prop:sec1_main} using
the results in the previous subsection.

Let $\pr{\sf{Cat}_{\Delta}}_{\fib}$ denote the full subcategory of
$\sf{Cat}_{\Delta}$ spanned by the fibrant simplicial categories.
We let $N^{\natural}:\pr{\sf{Cat}_{\Delta}}_{\fib}\to\SS^{+}$ denote
the functor obtained from the homotopy coherent nerve functor by marking
equivalences. 
\begin{lem}
\label{lem:replacement}Let $\cal C$ be a small category and let
$F,G:\cal C\to\SS_{\mathrm{cocart}}^{+}$ be projectively fibrant
functors. For any natural transformation $f:F\to G$, we can find
projectively fibrant functors $F',G':\cal C\to\sf{Cat}_{\Delta}$,
a projective fibration $f':F'\to G'$, and a commutative diagram % https://q.uiver.app/?q=WzAsNCxbMCwwLCJGIl0sWzEsMCwiTl5cXG5hdHVyYWxcXGNpcmMgRiciXSxbMSwxLCJOXlxcbmF0dXJhbFxcY2lyYyBHJyJdLFswLDEsIkciXSxbMCwxLCJcXHNpbWVxIl0sWzEsMiwiTl5cXG5hdHVyYWxcXGNpcmMgZiciXSxbMCwzLCJmIiwyXSxbMywyLCJcXHNpbWVxIiwyXV0=
\[\begin{tikzcd}
	F & {N^\natural\circ F'} \\
	G & {N^\natural\circ G'}
	\arrow["\simeq", from=1-1, to=1-2]
	\arrow["{N^\natural\circ f'}", from=1-2, to=2-2]
	\arrow["f"', from=1-1, to=2-1]
	\arrow["\simeq"', from=2-1, to=2-2]
\end{tikzcd}\]in $\Fun\pr{\cal C,\SS^{+}}$ whose horizontal arrows are weak equivalences.
\end{lem}
\begin{proof}
Let $F_{\flat},G_{\flat}:\cal C\to\SS$ denote the functors obtained
by forgetting the markings. Since $F,G$ are projectively fibrant,
for each object $C\in\cal C$, the simplicial set $F_{\flat}\pr C$
is an $\infty$-category and the marked edges of $F\pr C$ are precisely
the equivalences of $F_{\flat}\pr C$. Now find a commutative diagram
% https://q.uiver.app/?q=WzAsNCxbMCwwLCJcXG1hdGhmcmFre0N9XFxjaXJjIEZfXFxmbGF0Il0sWzAsMSwiXFxtYXRoZnJha3tDfVxcY2lyYyBHX1xcZmxhdCJdLFsxLDAsIkYnIl0sWzEsMSwiRyciXSxbMCwxLCJcXG1hdGhmcmFre0N9XFxjaXJjIGYiLDJdLFswLDIsIlxcc2ltZXEiXSxbMiwzLCJmJyJdLFsxLDMsIlxcc2ltZXEiLDJdXQ==
\[\begin{tikzcd}
	{\mathfrak{C}\circ F_\flat} & {F'} \\
	{\mathfrak{C}\circ G_\flat} & {G'}
	\arrow["{\mathfrak{C}\circ f}"', from=1-1, to=2-1]
	\arrow["\simeq", from=1-1, to=1-2]
	\arrow["{f'}", from=1-2, to=2-2]
	\arrow["\simeq"', from=2-1, to=2-2]
\end{tikzcd}\]in $\Fun\pr{\cal C,\sf{Cat}_{\Delta}}$, where $F',G'$ are projectively
fibrant, $\alpha'$ is a projective fibration, and the horizontal
arrows are weak equivalences. Since the adjunction
\[
\fr C:\SS_{\mathrm{Joyal}}\adj\sf{Cat}_{\Delta}:N
\]
is a Quillen equivalence, the horizontal arrows of the induced square
% https://q.uiver.app/?q=WzAsNCxbMCwwLCJGX1xcZmxhdCJdLFswLDEsIkdfXFxmbGF0Il0sWzEsMCwiTlxcY2lyYyBGJyJdLFsxLDEsIk5cXGNpcmMgRyciXSxbMCwxLCJmIiwyXSxbMCwyLCJcXHNpbWVxIl0sWzIsMywiTlxcY2lyYyBmJyJdLFsxLDMsIlxcc2ltZXEiLDJdXQ==
\[\begin{tikzcd}
	{F_\flat} & {N\circ F'} \\
	{G_\flat} & {N\circ G'}
	\arrow["f"', from=1-1, to=2-1]
	\arrow["\simeq", from=1-1, to=1-2]
	\arrow["{N\circ f'}", from=1-2, to=2-2]
	\arrow["\simeq"', from=2-1, to=2-2]
\end{tikzcd}\]in $\Fun\pr{\cal C,\SS}$ are weak equivalences (i.e., pointwise categorical
equivalences). By marking the equivalences, we obtain the desired
square.
\end{proof}
\begin{defn}
Let $n\geq1$. We let $\cal B\pr n$ denote the nerve of the category
$\{\infty\}\star\inp n^{\circ}$, where $\inp n^{\circ}=\{1,\dots,n\}$
is regarded as a discrete category. For each $1\leq i\leq n$, we
will denote the unique morphism $\infty\to i$ by $\rho^{i}$.
\end{defn}
\begin{prop}
\label{prop:pre-equivalence}Let $n\geq1$, and consider a commutative
diagram% https://q.uiver.app/?q=WzAsMyxbMCwwLCJcXG1hdGhjYWx7Q30iXSxbMiwwLCJcXG1hdGhjYWx7RH0iXSxbMSwxLCJcXG1hdGhjYWx7Qn0obikiXSxbMCwxLCJwIl0sWzEsMiwiciJdLFswLDIsInEiLDJdXQ==
\[\begin{tikzcd}
	{\mathcal{C}} && {\mathcal{D}} \\
	& {\mathcal{B}(n)}
	\arrow["p", from=1-1, to=1-3]
	\arrow["r", from=1-3, to=2-2]
	\arrow["q"', from=1-1, to=2-2]
\end{tikzcd}\]of $\infty$-categories. Assume the following:
\begin{enumerate}
\item The functors $q$ and $r$ are cocartesian fibrations, and the functor
$p$ is a categorical fibration.
\item The functor $p$ preserves cocartesian morphisms over $\cal B\pr n$.
\item The functor $\pr{\rho_{!}^{i}}_{i}:\cal D_{\infty}\to\prod_{1\leq i\leq n}\cal D_{i}$
induces an injection between the set of equivalence classes of objects
of $\cal D_{\inp n}$ and that of $\prod_{1\leq i\leq n}\cal D_{i}$.
\item For each object $C\in\cal C_{\infty}$, the $q$-cocartesian morphisms
$C\to C_{i}$ over $\rho^{i}$ form a $p$-limit cone. 
\item Given an object $\pr{C_{i}}_{i}\in\prod_{1\leq i\leq n}\cal C_{i}$,
there is an object $C\in\cal C$ which admits a $q$-cocartesian morphism
$C\to C_{i}$ over the morphism $\rho^{i}$ for every $i$.
\end{enumerate}
Then the functor
\[
\cal C_{\infty}\to\prod_{1\leq i\leq n}\cal C_{i}\times_{\prod_{1\leq i\leq n}\cal D_{i}}\cal D_{\infty}
\]
is an equivalence of $\infty$-categories.
\end{prop}
In the proof, we shall make use of the adjunction 
\[
r_{!}:\SS^{+}/\cal B\pr n\adj\Fun\pr{\{\infty\}\star\inp n^{\circ},\SS^{+}}:r^{*}
\]
between the (marked) relative nerve functor $r^{*}$ and its left
adjoint $r_{!}$, given by $r_{!}\pr X\pr x=X\times_{\cal B\pr n^{\sharp}}\cal B\pr n_{/x}^{\sharp}$.
\begin{proof}
The statement of the proposition is a bit vague, because the functors
$\cal C_{\infty}\to\cal C_{i}$ and $\cal D_{\infty}\to\cal D_{i}$
are defined only up to natural equivalence. A more precise formulation
is the following: Since $p$ is a categorical fibration, the map $\cal C^{\natural}\to\cal D^{\natural}$
is a fibration in the cocartesian model structures over $\cal B\pr n$,
where $\cal C^{\natural}$ is the marked simplicial set whose marked
edges are the $q$-cocartesian edges, and $\cal D^{\natural}$ is
defined similarly. (This is the opposite convention of \cite[$\S 3.2$]{HTT},
but it shouldn't case any confusion.) Therefore, for each object $i\in\inp n^{\circ}$,
there is a natural transformation $\cal C_{\infty}\times\Delta^{1}\to\cal C$
which fits into the commutative diagram % https://q.uiver.app/?q=WzAsNSxbMCwwLCJcXG1hdGhjYWx7Q31eXFxuYXR1cmFsX1xcaW5mdHlcXHRpbWVzIChcXHswXFx9KV5cXHNoYXJwIl0sWzAsMSwiXFxtYXRoY2Fse0N9XlxcbmF0dXJhbF9cXGluZnR5XFx0aW1lcyAoXFxEZWx0YV4xKV5cXHNoYXJwIl0sWzEsMSwiXFxtYXRoY2Fse0R9XlxcbmF0dXJhbF9cXGluZnR5XFx0aW1lcyAoXFxEZWx0YV4xKV5cXHNoYXJwIl0sWzIsMSwiXFxtYXRoY2Fse0R9XlxcbmF0dXJhbCJdLFsyLDAsIlxcbWF0aGNhbHtDfV5cXG5hdHVyYWwiXSxbMCwxXSxbMSwyXSxbMiwzXSxbNCwzXSxbMCw0XSxbMSw0LCIiLDEseyJzdHlsZSI6eyJib2R5Ijp7Im5hbWUiOiJkYXNoZWQifX19XV0=
\[\begin{tikzcd}
	{\mathcal{C}^\natural_\infty\times (\{0\})^\sharp} && {\mathcal{C}^\natural} \\
	{\mathcal{C}^\natural_\infty\times (\Delta^1)^\sharp} & {\mathcal{D}^\natural_\infty\times (\Delta^1)^\sharp} & {\mathcal{D}^\natural,}
	\arrow[from=1-1, to=2-1]
	\arrow[from=2-1, to=2-2]
	\arrow[from=2-2, to=2-3]
	\arrow[from=1-3, to=2-3]
	\arrow[from=1-1, to=1-3]
	\arrow[dashed, from=2-1, to=1-3]
\end{tikzcd}\]where the functor $\cal D^{\natural}\times\pr{\Delta^{1}}^{\sharp}\to\cal D^{\natural}$
is a $r$-cocartesian natural transformation over $\infty\to i$ which
starts from the inclusion $\cal D_{\infty}\hookrightarrow\cal D$.
The restriction of the filler determines a commutative diagram% https://q.uiver.app/?q=WzAsNCxbMCwwLCJcXG1hdGhjYWx7Q31fXFxpbmZ0eSJdLFsxLDAsIlxcbWF0aGNhbHtDfV9pIl0sWzEsMSwiXFxtYXRoY2Fse0R9X2kuIl0sWzAsMSwiXFxtYXRoY2Fse0R9X1xcaW5mdHkiXSxbMCwxLCJcXHJob15pXyEiXSxbMSwyLCJwIl0sWzAsMywicCIsMl0sWzMsMiwiXFxyaG9eaV8hIiwyXV0=
\[\begin{tikzcd}
	{\mathcal{C}_\infty} & {\mathcal{C}_i} \\
	{\mathcal{D}_\infty} & {\mathcal{D}_i.}
	\arrow["{\rho^i_!}", from=1-1, to=1-2]
	\arrow["p", from=1-2, to=2-2]
	\arrow["p"', from=1-1, to=2-1]
	\arrow["{\rho^i_!}"', from=2-1, to=2-2]
\end{tikzcd}\]The claim is that, for any such choice of functors $\cal D_{\infty}\to\cal D_{i}$
and $\cal C_{\infty}\to\cal C_{i}$, the resulting functor $\phi:\cal C_{\infty}\to\prod_{1\leq i\leq n}\cal C_{i}\times_{\prod_{1\leq i\leq n}\cal D_{i}}\cal D_{\infty}$
is an equivalence of $\infty$-categories.

We begin by proving the essential surjectivity. Let $\pr{\pr{C_{i}}_{i},D}\in\prod_{1\leq i\leq n}\cal C_{i}\times_{\prod_{1\leq i\leq n}\cal D_{i}}\cal D_{\infty}$
be an arbitrary object. By our hypothesis (5), we can find an object
$C\in\cal C_{\infty}$ which admits a $q$-cocartesian morphism $C\to C_{i}$
over $\rho^{i}$ for each $1\leq i\leq n$. By the uniqueness of cocartesian
morphisms, there is an equivalence to $\pr{C_{i}}_{i}\simeq\pr{\rho_{!}^{i}\pr C}_{i}$
in $\prod_{1\leq i\leq n}\cal C_{i}$. Its image in $\prod_{1\leq i\leq n}\cal D_{i}$
determines an equivalence $\pr{\rho_{!}^{i}\pr D}_{i}\simeq\pr{\rho_{!}^{i}\pr{p\pr C}}_{i}$.
By virtue of assumption (3), we obtain an equivalence $g:D\xrightarrow{\simeq}p(C)$
in $\cal D_{\infty}$. Using the fact that $p$ is a categorical fibration,
we can lift the morphism $g$ to an equivalence $\widetilde{g}:\widetilde{C}\xrightarrow{\simeq}C$
in $\cal C$. Therefore, replacing $f_{i}$ by $f_{i}\widetilde{g}$
if necessary, we may assume that $p\pr C=D$. For each $1\leq i\leq n$,
the morphism $f_{i}$ determines a functor $\cal C_{\infty}^{\natural}\times\{0\}^{\sharp}\cup\{C\}^{\sharp}\times\pr{\Delta^{1}}^{\sharp}\to\cal C^{\natural}$.
Since the inclusion
\[
\cal C_{\infty}^{\natural}\times\{0\}^{\sharp}\cup\{C\}^{\sharp}\times\pr{\Delta^{1}}^{\sharp}\hookrightarrow\cal C_{\infty}^{\natural}\times\pr{\Delta^{1}}^{\sharp}
\]
is a trivial cofibration in the cocartesian model structure over $\cal B\pr n$
\cite[Proposition 3.1.2.2]{HTT}, we can find a filler of the diagram
% https://q.uiver.app/?q=WzAsNSxbMCwwLCJcXG1hdGhjYWx7Q31eXFxuYXR1cmFsX1xcaW5mdHlcXHRpbWVzIChcXHswXFx9KV5cXHNoYXJwXFxjdXBcXHtDXFx9Xlxcc2hhcnAgXFx0aW1lcyAoXFxEZWx0YV4xKV5cXHNoYXJwIl0sWzAsMSwiXFxtYXRoY2Fse0N9XlxcbmF0dXJhbF9cXGluZnR5XFx0aW1lcyAoXFxEZWx0YV4xKV5cXHNoYXJwIl0sWzEsMSwiXFxtYXRoY2Fse0R9XlxcbmF0dXJhbF9cXGluZnR5XFx0aW1lcyAoXFxEZWx0YV4xKV5cXHNoYXJwIl0sWzIsMSwiXFxtYXRoY2Fse0R9XlxcbmF0dXJhbCJdLFsyLDAsIlxcbWF0aGNhbHtDfV5cXG5hdHVyYWwiXSxbMCwxXSxbMSwyXSxbMiwzXSxbNCwzXSxbMCw0XSxbMSw0LCIiLDEseyJzdHlsZSI6eyJib2R5Ijp7Im5hbWUiOiJkYXNoZWQifX19XV0=
\[\begin{tikzcd}
	{\mathcal{C}^\natural_\infty\times (\{0\})^\sharp\cup\{C\}^\sharp \times (\Delta^1)^\sharp} && {\mathcal{C}^\natural} \\
	{\mathcal{C}^\natural_\infty\times (\Delta^1)^\sharp} & {\mathcal{D}^\natural_\infty\times (\Delta^1)^\sharp} & {\mathcal{D}^\natural.}
	\arrow[from=1-1, to=2-1]
	\arrow[from=2-1, to=2-2]
	\arrow[from=2-2, to=2-3]
	\arrow[from=1-3, to=2-3]
	\arrow[from=1-1, to=1-3]
	\arrow[dashed, from=2-1, to=1-3]
\end{tikzcd}\]The filler determines a functor $\cal C_{\infty}\to\cal C_{i}$ which
is naturally equivalent to $\rho_{!}^{i}$ over $\cal D_{i}$. So
the functor $\phi':\cal C_{\infty}\to\prod_{1\leq i\leq n}\cal C_{i}\times_{\prod_{1\leq i\leq n}\cal D_{i}}\cal D_{\infty}$
obtained from these fillers is naturally equivalent to $\phi$. By
construction, we have $\phi'\pr C=\pr{\pr{C_{i}}_{i},D}$. This proves
that $\phi$ is essentially surjective.

Next, to prove that $\phi$ is fully faithful, we consider the commutative
diagram % https://q.uiver.app/?q=WzAsMTAsWzEsMSwiXFxtYXRoY2Fse0N9XlxcbmF0dXJhbF9cXGluZnR5XFx0aW1lcyAoXFxEZWx0YV4xKV5cXHNoYXJwIl0sWzEsMywiXFxtYXRoY2Fse0R9XlxcbmF0dXJhbF9cXGluZnR5XFx0aW1lcyAoXFxEZWx0YV4xKV5cXHNoYXJwIl0sWzMsMywiXFxwcm9kX3sxXFxsZXEgaVxcbGVxIG59XFxtYXRoY2Fse0R9XlxcbmF0dXJhbFxcdGltZXMgX3soXFxtYXRoY2Fse0J9KG4pKV5cXHNoYXJwfSh7XFx7XFxpbmZ0eVxcfVxcc3Rhclxce2lcXH19KV5cXHNoYXJwLiJdLFszLDEsIlxccHJvZF97MVxcbGVxIGlcXGxlcSBufVxcbWF0aGNhbHtDfV5cXG5hdHVyYWxcXHRpbWVzIF97KFxcbWF0aGNhbHtCfShuKSleXFxzaGFycH0oe1xce1xcaW5mdHlcXH1cXHN0YXJcXHtpXFx9fSleXFxzaGFycCJdLFs0LDAsIlxccHJvZF97MVxcbGVxIGlcXGxlcSBufVxcbWF0aGNhbHtDfV9pXlxcbmF0dXJhbCJdLFs0LDIsIlxccHJvZF97MVxcbGVxIGlcXGxlcSBufVxcbWF0aGNhbHtEfV9pXlxcbmF0dXJhbCJdLFsyLDAsIlxcbWF0aGNhbHtDfV9cXGluZnR5XlxcbmF0dXJhbFxcdGltZXMgXFx7MVxcfV5cXHNoYXJwIl0sWzIsMiwiXFxtYXRoY2Fse0R9X1xcaW5mdHleXFxuYXR1cmFsXFx0aW1lcyBcXHsxXFx9Xlxcc2hhcnAiXSxbMCwyLCJcXG1hdGhjYWx7Q31fXFxpbmZ0eV5cXG5hdHVyYWxcXHRpbWVzIFxcezBcXH1eXFxzaGFycCJdLFswLDQsIlxcbWF0aGNhbHtEfV9cXGluZnR5XlxcbmF0dXJhbFxcdGltZXMgXFx7MFxcfV5cXHNoYXJwIl0sWzAsMV0sWzEsMl0sWzMsMl0sWzAsM10sWzQsMywiXFxzaW1lcSIsMix7InN0eWxlIjp7InRhaWwiOnsibmFtZSI6Imhvb2siLCJzaWRlIjoiYm90dG9tIn19fV0sWzQsNV0sWzUsMiwiXFxzaW1lcSIsMCx7InN0eWxlIjp7InRhaWwiOnsibmFtZSI6Imhvb2siLCJzaWRlIjoiYm90dG9tIn19fV0sWzYsNF0sWzYsN10sWzcsNV0sWzcsMSwiXFxzaW1lcSJdLFs2LDAsIlxcc2ltZXEiXSxbOCwwLCJcXHNpbWVxIl0sWzksMSwiXFxzaW1lcSJdLFs4LDldLFs4LDMsIiIsMSx7InN0eWxlIjp7InRhaWwiOnsibmFtZSI6Imhvb2siLCJzaWRlIjoidG9wIn19fV0sWzksMiwiIiwxLHsic3R5bGUiOnsidGFpbCI6eyJuYW1lIjoiaG9vayIsInNpZGUiOiJ0b3AifX19XV0=
\[\begin{tikzcd}[scale cd=.65]
	&& {\mathcal{C}_\infty^\natural\times \{1\}^\sharp} && {\prod_{1\leq i\leq n}\mathcal{C}_i^\natural} \\
	& {\mathcal{C}^\natural_\infty\times (\Delta^1)^\sharp} && {\prod_{1\leq i\leq n}\mathcal{C}^\natural\times _{(\mathcal{B}(n))^\sharp}({\{\infty\}\star\{i\}})^\sharp} \\
	{\mathcal{C}_\infty^\natural\times \{0\}^\sharp} && {\mathcal{D}_\infty^\natural\times \{1\}^\sharp} && {\prod_{1\leq i\leq n}\mathcal{D}_i^\natural} \\
	& {\mathcal{D}^\natural_\infty\times (\Delta^1)^\sharp} && {\prod_{1\leq i\leq n}\mathcal{D}^\natural\times _{(\mathcal{B}(n))^\sharp}({\{\infty\}\star\{i\}})^\sharp.} \\
	{\mathcal{D}_\infty^\natural\times \{0\}^\sharp}
	\arrow[from=2-2, to=4-2]
	\arrow[from=4-2, to=4-4]
	\arrow[from=2-4, to=4-4]
	\arrow[from=2-2, to=2-4]
	\arrow["\simeq"', hook', from=1-5, to=2-4]
	\arrow[from=1-5, to=3-5]
	\arrow["\simeq", hook', from=3-5, to=4-4]
	\arrow[from=1-3, to=1-5]
	\arrow[from=1-3, to=3-3]
	\arrow[from=3-3, to=3-5]
	\arrow["\simeq", from=3-3, to=4-2]
	\arrow["\simeq", from=1-3, to=2-2]
	\arrow["\simeq", from=3-1, to=2-2]
	\arrow["\simeq", from=5-1, to=4-2]
	\arrow[from=3-1, to=5-1]
	\arrow[hook, from=3-1, to=2-4]
	\arrow[hook, from=5-1, to=4-4]
\end{tikzcd}\]We wish to show that the back face of this diagram is homotopy cartesian
in the cocartesian model structure in $\SS^{+}$. By \cite[Proposition 2.10]{StUn},
the maps labeled with $\simeq$ are weak equivalences in $\SS^{+}$.
It will therefore suffice to show that the front face is homotopy
cartesian. Note that the front face can be identified with the diagram
% https://q.uiver.app/?q=WzAsNCxbMCwwLCJyXyEoXFxtYXRoY2Fse0N9XlxcbmF0dXJhbCkoXFxpbmZ0eSkiXSxbMCwxLCJyXyEoXFxtYXRoY2Fse0R9XlxcbmF0dXJhbCkoXFxpbmZ0eSkiXSxbMSwwLCJcXHByb2RfezFcXGxlcSBpXFxsZXEgbn1yXyEoXFxtYXRoY2Fse0N9XlxcbmF0dXJhbCkoaSkiXSxbMSwxLCJcXHByb2RfezFcXGxlcSBpXFxsZXEgbn1yXyEoXFxtYXRoY2Fse0R9XlxcbmF0dXJhbCkoaSkiXSxbMCwxXSxbMCwyXSxbMSwzXSxbMiwzXV0=
\[\begin{tikzcd}
	{r_!(\mathcal{C}^\natural)(\infty)} & {\prod_{1\leq i\leq n}r_!(\mathcal{C}^\natural)(i)} \\
	{r_!(\mathcal{D}^\natural)(\infty)} & {\prod_{1\leq i\leq n}r_!(\mathcal{D}^\natural)(i),}
	\arrow[from=1-1, to=2-1]
	\arrow[from=1-1, to=1-2]
	\arrow[from=2-1, to=2-2]
	\arrow[from=1-2, to=2-2]
\end{tikzcd}\]where $r_{!}:\SS^{+}/\cal B\pr n\to\Fun\pr{\{\infty\}\star\inp n^{\circ},\SS^{+}}$
is the left adjoint of the relative nerve functor.

Find a commutative diagram % https://q.uiver.app/?q=WzAsNCxbMCwwLCJyXyEoXFxtYXRoY2Fse0N9XlxcbmF0dXJhbCkiXSxbMSwwLCJGIl0sWzEsMSwiRyJdLFswLDEsInJfIShcXG1hdGhjYWx7RH1eXFxuYXR1cmFsKSJdLFswLDEsIlxcc2ltZXEiXSxbMSwyLCJmIl0sWzMsMiwiXFxzaW1lcSIsMl0sWzAsMywicl8hKHApIiwyXV0=
\[\begin{tikzcd}
	{r_!(\mathcal{C}^\natural)} & F \\
	{r_!(\mathcal{D}^\natural)} & G
	\arrow["\simeq", from=1-1, to=1-2]
	\arrow["f", from=1-2, to=2-2]
	\arrow["\simeq"', from=2-1, to=2-2]
	\arrow["{r_!(p)}"', from=1-1, to=2-1]
\end{tikzcd}\]in $\Fun\pr{\{\infty\}\star\inp n^{\circ},\SS^{+}}$, where $F$ and
$G$ are projectively fibrant functors, $f$ is a projective fibration,
and the horizontal arrows are weak equivalences. According to Lemma
\ref{lem:replacement}, we may assume that the natural transformation
\[
f:\{\infty\}\star\inp n^{\circ}\times[1]\to\SS^{+}
\]
can be factored as 
\[
\{\infty\}\star\inp n^{\circ}\times[1]\xrightarrow{f'}\pr{\sf{Cat}_{\Delta}}_{\fib}\xrightarrow{N^{\natural}}\SS^{+}.
\]
We set $f'_{0}=F'$ and $g'_{0}=G'$, and write $\cal C_{\Delta}=\int F'$
and $\cal D_{\Delta}=\int G'$. We are then reduced to showing that
the square % https://q.uiver.app/?q=WzAsNCxbMCwwLCJOKFxcbWF0aGNhbHtDfV97XFxEZWx0YSxcXGluZnR5fSleXFxuYXR1cmFsIl0sWzEsMCwiXFxwcm9kX3sxXFxsZXEgaVxcbGVxIG59TihcXG1hdGhjYWx7Q31fe1xcRGVsdGEsaX0pXlxcbmF0dXJhbCJdLFsxLDEsIlxccHJvZF97MVxcbGVxIGlcXGxlcSBufU4oXFxtYXRoY2Fse0R9X3tcXERlbHRhLGl9KV5cXG5hdHVyYWwiXSxbMCwxLCJOKFxcbWF0aGNhbHtEfV97XFxEZWx0YSxcXGluZnR5fSleXFxuYXR1cmFsIl0sWzAsMV0sWzEsMl0sWzAsM10sWzMsMl1d
\[\begin{tikzcd}
	{N(\mathcal{C}_{\Delta,\infty})^\natural} & {\prod_{1\leq i\leq n}N(\mathcal{C}_{\Delta,i})^\natural} \\
	{N(\mathcal{D}_{\Delta,\infty})^\natural} & {\prod_{1\leq i\leq n}N(\mathcal{D}_{\Delta,i})^\natural}
	\arrow[from=1-1, to=1-2]
	\arrow[from=1-2, to=2-2]
	\arrow[from=1-1, to=2-1]
	\arrow[from=2-1, to=2-2]
\end{tikzcd}\]of marked simplicial sets is homotopy cartesian. Since the vertical
arrows are fibrations in $\SS_{\mathrm{cocart}}^{+}$ and all the
objects are fibrant, it suffices to show that the map 
\[
N\pr{\cal C_{\Delta,\infty}}^{\natural}\to N\pr{\cal D_{\Delta,\infty}}^{\natural}\times_{\prod_{1\leq i\leq n}N\pr{\cal D_{\Delta,i}}^{\natural}}\prod_{1\leq i\leq n}N\pr{\cal C_{\Delta,i}}^{\natural}
\]
is a weak equivalence in $\SS_{\mathrm{cocart}}^{+}$. This is equivalent
to the condition that the map
\[
\theta':N\pr{\cal C_{\Delta,\infty}}\to N\pr{\cal D_{\Delta,\infty}}\times_{\prod_{1\leq i\leq n}N\pr{\cal D_{\Delta,i}}}\prod_{1\leq i\leq n}N\pr{\cal C_{\Delta,i}}
\]
is an equivalence of $\infty$-categories. 

According to Proposition \ref{prop:relative_nerve_of_Gr_const}, the
map $r^{*}\pr f$ can be identified with the map
\[
N\pr{\int f'}:N\pr{\cal C_{\Delta}}^{\natural}\to N\pr{\cal D_{\Delta}}^{\natural}.
\]
Moreover, since the adjunction $r_{!}\dashv r^{*}$ is a Quillen equivalence,
the maps $\cal C^{\natural}\to r^{*}\pr F$ and $\cal D^{\natural}\to r^{*}\pr G$
are weak equivalences in the cocartesian model structure over $\cal B\pr n$.
So the functor $N\pr{\cal C_{\Delta}}\to N\pr{\cal D_{\Delta}}$ has
all the properties (1)-(5). So the proof of the essential surjectivity
of $\theta$ applies as well to $\theta'$, showing that $\theta'$
is essentially surjective. It remains to verify that $\theta'$ is
fully faithful. For this, it suffices to show that for each pair of
objects $X,Y\in\cal C_{\Delta,\infty}$, the diagram% https://q.uiver.app/?q=WzAsNCxbMCwwLCJcXG1hdGhjYWx7Q31fe1xcRGVsdGEsXFxpbmZ0eX0oWCxZKSJdLFswLDEsIlxcbWF0aGNhbHtEfV97XFxEZWx0YSxcXGluZnR5fShmWCxmWSkiXSxbMSwwLCJcXHByb2RfezFcXGxlcSBpXFxsZXEgbn1cXG1hdGhjYWx7Q31fe1xcRGVsdGEsaX0oKEYnXFxyaG9eaSlYLChGJ1xccmhvXmkpWSkiXSxbMSwxLCJcXHByb2RfezFcXGxlcSBpXFxsZXEgbn1cXG1hdGhjYWx7RH1fe1xcRGVsdGEsaX0oKEcnXFxyaG9eaSlmWCwoRydcXHJob15pKWZZKSJdLFswLDFdLFswLDJdLFsyLDNdLFsxLDNdXQ==
\[\begin{tikzcd}
	{\mathcal{C}_{\Delta,\infty}(X,Y)} & {\prod_{1\leq i\leq n}\mathcal{C}_{\Delta,i}((F'\rho^i)X,(F'\rho^i)Y)} \\
	{\mathcal{D}_{\Delta,\infty}(fX,fY)} & {\prod_{1\leq i\leq n}\mathcal{D}_{\Delta,i}((G'\rho^i)fX,(G'\rho^i)fY)}
	\arrow[from=1-1, to=2-1]
	\arrow[from=1-1, to=1-2]
	\arrow[from=1-2, to=2-2]
	\arrow[from=2-1, to=2-2]
\end{tikzcd}\]of Kan complexes is homotopy cartesian. Consider the commutative diagram
% https://q.uiver.app/?q=WzAsOCxbMCwxLCJcXG1hdGhjYWx7Q31fe1xcRGVsdGEsXFxpbmZ0eX0oWCxZKSJdLFswLDMsIlxcbWF0aGNhbHtEfV97XFxEZWx0YSxcXGluZnR5fShmWCxmWSkiXSxbMiwxLCJcXHByb2RfezFcXGxlcSBpXFxsZXEgbn1cXG1hdGhjYWx7Q31fe1xcRGVsdGEsaX0oKEYnXFxyaG9eaSlYLChGJ1xccmhvXmkpWSkiXSxbMiwzLCJcXHByb2RfezFcXGxlcSBpXFxsZXEgbn1cXG1hdGhjYWx7RH1fe1xcRGVsdGEsaX0oKEcnXFxyaG9eaSkgZlgsKEcnXFxyaG9eaSlmWSkiXSxbMSwwLCJcXG1hdGhjYWx7Q31fXFxEZWx0YShYLFkpIl0sWzMsMCwiXFxwcm9kIF97MVxcbGVxIGlcXGxlcSBufVxcbWF0aGNhbHtDfV9cXERlbHRhKFgsKEYnXFxyaG9eaSlZKSJdLFszLDIsIlxccHJvZF97MVxcbGVxIGlcXGxlcSBufVxcbWF0aGNhbHtEfV97XFxEZWx0YX0oZlgsKEcnXFxyaG9eaSlmWSkuIl0sWzEsMiwiXFxtYXRoY2Fse0R9X3tcXERlbHRhfShmWCxmWSkiXSxbMCwxXSxbMCwyXSxbMiwzXSxbMSwzXSxbMCw0XSxbMiw1XSxbNCw1XSxbMyw2XSxbNCw3XSxbNyw2XSxbMSw3XSxbNSw2XV0=
\[\begin{tikzcd}[column sep=small, scale cd=.75]
	& {\mathcal{C}_\Delta(X,Y)} && {\prod _{1\leq i\leq n}\mathcal{C}_\Delta(X,(F'\rho^i)Y)} \\
	{\mathcal{C}_{\Delta,\infty}(X,Y)} && {\prod_{1\leq i\leq n}\mathcal{C}_{\Delta,i}((F'\rho^i)X,(F'\rho^i)Y)} \\
	& {\mathcal{D}_{\Delta}(fX,fY)} && {\prod_{1\leq i\leq n}\mathcal{D}_{\Delta}(fX,(G'\rho^i)fY).} \\
	{\mathcal{D}_{\Delta,\infty}(fX,fY)} && {\prod_{1\leq i\leq n}\mathcal{D}_{\Delta,i}((G'\rho^i) fX,(G'\rho^i)fY)}
	\arrow[from=2-1, to=4-1]
	\arrow[from=2-1, to=2-3]
	\arrow[from=2-3, to=4-3]
	\arrow[from=4-1, to=4-3]
	\arrow[from=2-1, to=1-2]
	\arrow[from=2-3, to=1-4]
	\arrow[from=1-2, to=1-4]
	\arrow[from=4-3, to=3-4]
	\arrow[from=1-2, to=3-2]
	\arrow[from=3-2, to=3-4]
	\arrow[from=4-1, to=3-2]
	\arrow[from=1-4, to=3-4]
\end{tikzcd}\]The slanted arrows are isomorphisms of simplicial sets, so it suffices
to show that the back face is homotopy cartesian. This follows from
our assumption (4).
\end{proof}
We can now prove Proposition \ref{prop:sec1_main}.
\begin{proof}
[Proof of  Proposition \ref{prop:sec1_main}]Clearly (c-ii) implies
(c-i). Conversely, suppose that condition (c-i) is satisfied. For
each $n\geq1$, there is a unique functor $\cal B\pr n\to N\pr{\Fin_{\ast}}$
which is the identity map on morphisms. Applying Proposition \ref{prop:pre-equivalence}
to the functor $\cal M^{\t}\times_{N\pr{\Fin_{\ast}}}\cal B\pr n\to\cal C\times\cal B\pr n$,
we see that condition (c-ii) holds true.
\end{proof}

\section{\label{sec:Monoidal-Envelopes-of}Monoidal Envelopes of Families
of $\infty$-Operads}

In \cite[Section 2.2.4]{HA}, Lurie introduces the universal procedure
to make an arbitrary $\infty$-operad into a symmetric monoidal $\infty$-category.
The resulting symmetric monoidal $\infty$-category is called the
\textbf{monoidal envelope}. In this section, we will show that given
a family of $\infty$-operads, we can take the monoidal envelope of
each fiber to obtain a family of symmetric monoidal $\infty$-categories.
We will also prove that monoidal envelopes of families of $\infty$-operads
enjoys the expected universal property. 

We remark that the proofs of the results in this section are only
slight modifications of Lurie's original proofs of various results
concerning monoidal envelopes, which can be found in \cite[Section 2.2.4]{HA}.
Nevertheless, for the sake of completeness, we will record the proofs
whenever Lurie's proof does not apply verbatim.

\subsection{Definition and Construction}

In this subsection, we define monoidal envelopes of families of $\infty$-operads
and show that they are again families of $\infty$-operads.

We begin with a generalization of Lurie's construction \cite[Construction 2.2.4.1]{HA}:
\begin{defn}
Let $\cal M^{\t}$ be a generalized $\infty$-operad \cite[Definition 2.3.2.1]{HA}.
We define $\Act\pr{\cal M^{\t}}\subset\Fun\pr{\Delta^{1},\cal M^{\t}}$
to be the full subcategory spanned by the active morphisms. If $p:\cal M^{\t}\to\cal N^{\t}$
is a fibration of generalized $\infty$-operads, the \textbf{$\cal N$-monoidal
envelope} of $\cal M^{\t}$ is defined to be the fiber product
\[
\Env_{\cal N}\pr{\cal M}^{\t}=\cal M^{\t}\times_{\Fun\pr{\{0\},\cal N^{\t}}}\Act\pr{\cal N^{\t}}.
\]
In the case $\cal N^{\t}=N\pr{\Fin_{\ast}}$, we will write $\Env_{\cal N}\pr{\cal M}^{\t}=\Env\pr{\cal M}^{\t}$.
We will regard $\Env_{\cal N}\pr{\cal M}^{\t}$ as an $\infty$-category
over $\cal N^{\t}$ via the composite
\[
\Env_{\cal N}\pr{\cal M}^{\t}\to\Act\pr{\cal N^{\t}}\xrightarrow{\opn{ev}_{1}}\cal N^{\t}.
\]
\end{defn}
Thus, if $p:\cal M^{\t}\to N\pr{\Fin_{\ast}}$ is a generalized $\infty$-operad,
then $\Env\pr{\cal M}^{\t}$ is the $\infty$-category of pairs $\pr{M,\alpha:p\pr M\to\inp k}$,
where $M\in\cal M^{\t}$ and $\alpha$ is an active morphism of $N\pr{\Fin_{\ast}}$.
In particular, the $\infty$-category $\Env\pr{\cal M}=\Env\pr{\cal M}_{\inp 1}^{\t}$
may be identified with $\cal M_{\act}^{\t}$. The intuition here is
that an object $M\in\cal M_{\inp n}^{\t}$ is regarded as a ``formal
tensor product'' of the objects $\rho_{!}^{i}\pr M\in\cal M$.

Note that the (fully faithful) diagonal embedding $\cal N^{\t}\to\Act\pr{\cal N^{\t}}$
induces a fully faithful embedding $\cal M^{\t}\hookrightarrow\Env_{\cal N}\pr{\cal M}^{\t}$,
which partly justifies the terminology ``envelope.'' In fact, the
monoidal envelope enjoys a certain universal property, as we will
see in Subsection \ref{subsec:Universal-Property-of_monoidal_env}.

The goal of this note is to prove the following result, which is a
generalization of \cite[Proposition 2.2.4.4]{HA}.
\begin{prop}
\label{prop:2.2.4.4}Let $\cal C$ be an $\infty$-category and let
$p:\cal M^{\t}\to\cal C\times N\pr{\Fin_{\ast}}$ be a $\cal C$-family
of $\infty$-operads. Let 
\[
q:\Env\pr{\cal M}^{\t}\to\cal C\times N\pr{\Fin_{\ast}}
\]
denote the functor induced by the functors $\Env\pr{\cal M}^{\t}\to N\pr{\Fin_{\ast}}$
and $\Env\pr{\cal M}^{\t}\to\cal M^{\t}\to\cal C$. Then:
\begin{itemize}
\item [(a)]The functor $q$ makes $\Env\pr{\cal M}^{\t}$ into a $\cal C$-family
of $\infty$-operads. 
\item [(b)]A morphism of $\Env\pr{\cal M}^{\t}$ whose image in $\cal M^{\t}$
is inert is $q$-cocartesian. The converse holds if it lies over an
inert morphism in $\cal C\times N\pr{\Fin_{\ast}}$.
\end{itemize}
\end{prop}
\begin{rem}
In the situation of Proposition \ref{prop:2.2.4.4}, the fiber $\Env\pr{\cal M}_{C}^{\t}=\Env\pr{\cal M}^{\t}\times_{\cal C}\{C\}$
over an object $C\in\cal C$ is the monoidal envelope of the $\infty$-operad
$\cal M_{C}^{\t}$, which is a symmetric monoidal $\infty$-category
by \cite[Proposition 2.2.4.4]{HA}. 
\end{rem}
We will return to the proof after a sequence of lemmas.
\begin{lem}
\cite[Lemma 2.2.4.11]{HA}\label{lem:2.2.4.11}Let $p:\cal E\to\cal D$
be a cocartesian fibration of $\infty$-categories. Suppose there
is a full subcategory $\cal C\subset\cal E$ which has the following
properties:

\begin{enumerate}[label=(\roman*)]

\item For each object $D\in\cal D$, the inclusion $\cal C_{D}\subset\cal E_{D}$
admits a left adjoint $L_{D}:\cal E_{D}\to\cal C_{D}$.

\item For each morphism $f:D\to D'$ in $\cal D$, the associated
functor $f_{!}:\cal E_{D}\to\cal E_{D'}$ carries $L_{D}$-equivalences
(i.e., its image under $L_{D}$ is an equivalence) to $L_{D'}$-equivalences.

\end{enumerate}

Let $q=p\vert_{\cal C}:\cal C\to\cal D$ denote the restriction of
$p$. The following holds:
\begin{enumerate}
\item The functor $q=p\vert_{\cal C}:\cal C\to\cal D$ is a cocartesian
fibration.
\item Let $g:D\to D'$ be a morphism in $\cal D$ and $f:C\to C'$ a morphism
in $\cal C$ lifting $g$. Then $f$ is $q$-cocartesian if and only
if the map $g_{!}C\to C'$ is an $L_{D'}$-equivalence, where $g_{!}:\cal E_{D}\to\cal E_{D'}$
is the functor induced by $g$.
\end{enumerate}
\end{lem}
%
\begin{lem}
\cite[Remark 2.2.4.12]{HA}\label{lem:2.2.4.12}Let $p:\cal E\to\cal D$
be a cocartesian fibration of $\infty$-categories and $\cal C\subset\cal E$
a full subcategory. Consider the following conditions for $\cal C$:

\begin{enumerate}[label=(\roman*)]

\item For each object $D\in\cal D$, the inclusion $\cal C_{D}\subset\cal E_{D}$
admits a left adjoint $L_{D}:\cal E_{D}\to\cal C_{D}$.

\item For each morphism $f:D\to D'$ in $\cal D$, the associated
functor $f_{!}:\cal E_{D}\to\cal E_{D'}$ carries $L_{D}$-equivalences
(i.e., its image under $L_{D}$ is an equivalence) to $L_{D'}$-equivalences.

\end{enumerate}

\begin{enumerate}[label=(\roman*')]

\item The inclusion $\cal C\subset\cal E$ admits a left adjoint
$L:\cal E\to\cal C$.

\item The functor $p$ carries each $L$-equivalecne to to an equivalence.

\end{enumerate}

The conditions (i) and (ii) are equivalent to the conditions (i')
and (ii'). Moreover, if these conditions are satisfied, then for each
object $D\in\cal D$, a morphism in $\cal C_{D}$ is an $L$-equivalence
if and only if it is an $L_{D}$-equivalence.
\end{lem}
%
\begin{lem}
\cite[Lemma 2.2.4.13]{HA}\label{lem:2.2.4.13}Let $p:\cal C\to\cal D$
be an inner fibration of $\infty$-categories and $\cal D'\subset\cal D$
a full subcategory. Set $\cal C'=\cal C\times_{\cal D'}\cal D$. Assume
the following:

\begin{enumerate}[label=(\roman*)]

\item The inclusion $\cal D'\subset\cal D$ admits a left adjoint.

\item For each object $C\in\cal C$, the morphism $pC\to D'$ which
exhibits $D$ as a $\cal D'$-localization of $pC$ lifts to a $p$-cocartesian
morphism $C\to C'$.

\end{enumerate}

Then $\cal C'$ is a reflective subcategory of $\cal C$, and a morphism
$f:C\to C'$ in $\cal C$ exhibits $C'$ as a $\cal C'$-localization
of $C$ if and only if $f$ is $p$-cocartesian and the morphism $p\pr f$
exhibits $p\pr{C'}$ as a $\cal D'$-localization of $p\pr C$.
\end{lem}
%
\begin{lem}
\cite[Lemma 2.2.4.14]{HA}\label{lem:2.2.4.14}Let $p:\cal M^{\t}\to\cal N^{\t}$
be a fibration of generalized $\infty$-operads. Set $\cal D=\cal M^{\t}\times_{\Fun\pr{\{0\},\cal N^{\t}}}\Fun\pr{\Delta^{1},\cal N^{\t}}$.
The inclusion
\[
\Env_{\cal N}\pr{\cal M}^{\t}\subset\cal D
\]
admits a left adjoint. Moreover, a morphism $\alpha:D\to D'$ in $\cal D$
exhibits $D'$ as an $\Env_{\cal N}\pr{\cal M}^{\t}$-localization
of $D$ if and only if $D'\in\Env_{\cal N}\pr{\cal M}^{\t}$, the
image of $\alpha$ in $\cal M^{\t}$ is inert, and the image of $\alpha$
in $\cal N^{\t}$ is an equivalence.
\end{lem}
\begin{proof}
We recall from \cite[Lemma 5.2.8.19]{HTT} that the inclusion $\Act\pr{\cal N^{\t}}\subset\Fun\pr{\Delta^{1},\cal N^{\t}}$
admits a left adjoint and that a morphism $\alpha:g\to g'$ in $\Fun\pr{\Delta^{1},\cal N^{\t}}$
corresponding to a square % https://q.uiver.app/?q=WzAsNCxbMCwwLCJcXGJ1bGxldCJdLFsxLDAsIlxcYnVsbGV0Il0sWzEsMSwiXFxidWxsZXQiXSxbMCwxLCJcXGJ1bGxldCJdLFswLDEsImYiXSxbMSwyLCJnJyJdLFszLDIsImYnIiwyXSxbMCwzLCJnIiwyXV0=
\[\begin{tikzcd}
	\bullet & \bullet \\
	\bullet & \bullet
	\arrow["f", from=1-1, to=1-2]
	\arrow["{g'}", from=1-2, to=2-2]
	\arrow["{f'}"', from=2-1, to=2-2]
	\arrow["g"', from=1-1, to=2-1]
\end{tikzcd}\]exhibits $g'$ as an $\Act\pr{\cal N^{\t}}$-localization if and only
if $g'$ is active, $f$ is inert, and $f'$ is an equivalence. 

Now let $\pr{M,g:p\pr M\to N}$ be an object of $\Env_{\cal N}\pr{\cal M}^{\t}$,
and find a morphism $\alpha:g\to g'$ as in the previous paragraph.
According to Lemma \ref{lem:2.2.4.13}, it will suffice to show that
$\alpha$ admits a lift with domain $\pr{M,g}$ which is cocartesian
with respect to the projection $\cal D\to\Fun\pr{\Delta^{1},\cal N^{\t}}$.
Since $f$ is inert and $p$ is a fibration of generalized $\infty$-operads,
we can lift the morphism $f$ to an inert morphism $\beta:M\to M'$
in $\cal M^{\t}$. Then $\pr{\beta,\alpha}$ determines a morphism
$\pr{M,g}\to\pr{M',g'}$ which lifts $\alpha$. Since its image in
$\cal M^{\t}$ is $p$-cocartesian, the morphism $\pr{\alpha,\beta}$
is cocartesian with respect to $\cal D\to\Fun\pr{\Delta^{1},\cal N^{\t}}$.
The claim follows.
\end{proof}
\begin{lem}
\cite[Corollary 2.4.7.12]{HTT}\label{lem:corr_cocart}Let $f:\cal C\to\cal D$
be a functor of $\infty$-categories. The evaluation at $1\in\Delta^{1}$
induces a cocartesian fibration $p:\cal E=\cal C\times_{\Fun\pr{\{0\},\cal D}}\Fun\pr{\Delta^{1},\cal D}\to\cal D$,
and a morphism in $\cal E$ is $p$-cocartesian if and only if its
image in $\cal C$ is an equivalence.
\end{lem}
%
\begin{lem}
\cite[Lemma 2.2.4.15]{HA}\label{lem:2.2.4.15}Let $p:\cal M^{\t}\to\cal N^{\t}$
be a fibration of generalized $\infty$-operads. Set $\cal D=\cal M^{\t}\times_{\Fun\pr{\{0\},\cal N^{\t}}}\Fun\pr{\Delta^{1},\cal N^{\t}}$.
The following holds:
\begin{enumerate}
\item Evaluation at $1\in\Delta^{1}$ induces a cocartesian fibration $q':\cal D\to\cal N^{\t}$. 
\item A morphism in $\cal D$ is $q'$-cocartesian if and only if its image
in $\cal M^{\t}$ is an equivalence.
\item The map $q'$ restricts to a cocartesian fibration $q:\Env_{\cal N}\pr{\cal M}^{\t}\to\cal N^{\t}$.
\item A morphism $f$ in $\Env_{\cal N}\pr{\cal M}^{\t}$ is $q$-cocartesian
if and only if its image in $\cal M^{\t}$ is inert.
\end{enumerate}
\end{lem}
\begin{proof}
Assertions (1) and (2) follow from Lemma \ref{lem:corr_cocart}. Let
now $L:\cal D\to\Env_{\cal N}\pr{\cal M}^{\t}$ denote the left adjoint
of the inclusion, which exists by Lemma \ref{lem:2.2.4.14}. According
to this lemma, the functor $q'$ maps $L$-equivalences to equivalences.
Thus, by Lemmas \ref{lem:2.2.4.11} and \ref{lem:2.2.4.12}, the restriction
$q=q'\vert\Env_{\cal N}\pr{\cal M}^{\t}$ is a cocartesian fibration,
and a morphism $\pr{f,g}:\pr{M,\alpha}\to\pr{M',\alpha'}$ in $\Env_{\cal N}\pr{\cal M}^{\t}$
is $q$-cocartesian if and only if the map $\theta:g_{!}\pr{M,\alpha}\to\pr{M',\alpha'}$
is an $L$-equivalence. Here $g_{!}:\cal D_{X}\to\cal D_{X'}$ is
the functor induced by the morphism $g$. Now up to equivalence, the
map $\theta$ may be identified with the map $\pr{f,\id}:\pr{M,g\circ\alpha}\to\pr{M',\alpha'}$.
So Lemma \ref{lem:2.2.4.14} tells us that the latter condition holds
if and only if $f:M\to M'$ is inert. This proves (4).
\end{proof}
\begin{lem}
\cite[Lemma 2.2.4.16]{HA}\label{lem:2.2.4.16}Let $n\geq0$, and
let $\beta:\inp m\to\inp n$ and $\beta_{i}:\inp{m_{i}}\to\inp 1$
be active morphisms for $1\leq i\leq n$. Suppose we are given a commutative
diagram % https://q.uiver.app/?q=WzAsNCxbMCwwLCJcXGxhbmdsZSBtXFxyYW5nbGUiXSxbMSwwLCJcXGxhbmdsZSBtX2lcXHJhbmdsZSJdLFsxLDEsIlxcbGFuZ2xlIDEgXFxyYW5nbGUiXSxbMCwxLCJcXGxhbmdsZSBuXFxyYW5nbGUiXSxbMCwxLCJcXGdhbW1hX2kiXSxbMSwyLCJcXGJldGFfaSJdLFszLDIsIlxccmhvXmkiLDJdLFswLDMsIlxcYmV0YSIsMl1d
\[\begin{tikzcd}
	{\langle m\rangle} & {\langle m_i\rangle} \\
	{\langle n\rangle} & {\langle 1 \rangle}
	\arrow["{\gamma_i}", from=1-1, to=1-2]
	\arrow["{\beta_i}", from=1-2, to=2-2]
	\arrow["{\rho^i}"', from=2-1, to=2-2]
	\arrow["\beta"', from=1-1, to=2-1]
\end{tikzcd}\]for $1\leq i\leq n$ such that $\gamma_{i}$ is inert. Then the morphisms
$\{\beta\to\beta_{i}\}_{1\leq i\leq n}$ form a limit cone relative
to the map $p=\opn{ev}_{1}:\Act\pr{N\pr{\Fin_{\ast}}}\to N\pr{\Fin_{\ast}}$.
\end{lem}
%
\begin{proof}
[Proof of Proposition \ref{prop:2.2.4.4}]We begin with the first
half of part (b). Let $r:\cal C\times N\pr{\Fin_{\ast}}\to N\pr{\Fin_{\ast}}$
denote the projection. If a morphism $f$ in $\Env\pr{\cal M}^{\t}$
has the property that its image in $\cal M^{\t}$ is inert, then we
know from Lemma \ref{lem:2.2.4.15} that $f$ is $rq$-cocartesian,
and the image of $f$ in $\cal C$ is an equivalence. Thus $q\pr f$
ir $r$-cocartesian. Hence $f$ is $q$-cocartesian.

Next we proceed to (a). We must prove the following.
\begin{enumerate}
\item The functor $q$ is a categorical fibration.
\item Let $\pr{M,\alpha:\inp{n_{M}}\to\inp k}$ be an object of $\Env\pr{\cal M}^{\t}$
with image $C\in\cal C$, and let $\beta:\inp k\to\inp l$ be an inert
map. There is a $q$-cocartesian morphism $\pr{M,\alpha}\to\pr{M',\alpha'}$
lying over $\pr{\id_{C},\beta}$ whose image in $\cal M^{\t}$ is
inert.
\item Let $E\in\Env\pr{\cal M}^{\t}$ and set $\pr{C,\inp m}=q\pr E$. Then
the $q$-cocartesian morphisms $\gamma=\{E\to E_{i}\}_{1\leq i\leq k}$
lying over the morphisms $\{\pr{\id_{C},\rho^{i}}:\pr{C,\inp m}\to\pr{C,\inp 1}\}_{1\leq i\leq k}$
form a $q$-limit cone.
\item Let $k\geq1$, $C\in\cal C$, and $E_{1},\dots,E_{k}\in\Env\pr{\cal M}_{\pr{X,\inp 1}}^{\t}$.
There is an object $E\in\Env\pr{\cal M}_{\pr{C,\inp k}}^{\t}$ and
$q$-cocartesian morphisms $E\to E_{i}$ over $\rho^{i}$.
\end{enumerate}
Note that (2) and the first half of part (b) implies the latter half
of (b).

Assertion (1) is clear, because $q$ is the composite
\[
\Env\pr{\cal M}^{\t}\to\cal C\times\Act\pr{N\pr{\Fin_{\ast}}}\xrightarrow{\id\times\opn{ev}_{1}}\cal C\times N\pr{\Fin_{\ast}},
\]
and the first map is a pullback of $p$ and the second map is a categorical
fibration. For assertion (2), find a commutative diagram % https://q.uiver.app/?q=WzAsNCxbMCwwLCJcXGxhbmdsZSBuX01cXHJhbmdsZSJdLFsxLDAsIlxcbGFuZ2xlIGtcXHJhbmdsZSJdLFsxLDEsIlxcbGFuZ2xlIGxcXHJhbmdsZSJdLFswLDEsIlxcbGFuZ2xlICBuX00nXFxyYW5nbGUiXSxbMCwxLCJcXGFscGhhIl0sWzEsMiwiXFxiZXRhIl0sWzAsMywiXFxnYW1tYSIsMl0sWzMsMiwiXFxhbHBoYSciLDJdXQ==
\[\begin{tikzcd}
	{\langle n_M\rangle} & {\langle k\rangle} \\
	{\langle  n_M'\rangle} & {\langle l\rangle}
	\arrow["\alpha", from=1-1, to=1-2]
	\arrow["\beta", from=1-2, to=2-2]
	\arrow["\gamma"', from=1-1, to=2-1]
	\arrow["{\alpha'}"', from=2-1, to=2-2]
\end{tikzcd}\]in $\Fin_{\ast}$ with $\gamma$ inert and $\alpha'$ active. Since
$p$ is an $\cal C$-family of $\infty$-operads, there is an inert
morphism $f:M\to M'$ in $\cal M^{\t}$ lying over $\pr{\id_{C},\gamma}$.
The maps $f,\gamma,\beta$ determines a morphism $\pr{M,\alpha}\to\pr{M',\alpha'}$
in $\Env\pr{\cal M}^{\t}$ lying over $\pr{\id_{C},\beta}$, which
is $q$-cocartesian by the ``if'' part of part (b). 

For assertion (3), set $\cal D=\cal M^{\t}\times_{\Fun\pr{\{0\},\cal N^{\t}}}\Fun\pr{\Delta^{1},\cal N^{\otimes}}$
and consider the commutative diagram % https://q.uiver.app/?q=WzAsOCxbMCwwLCJcXG9wZXJhdG9ybmFtZXtFbnZ9X3tcXG1hdGhjYWx7Q31cXHRpbWVzIE4oXFxtYXRoc2Z7RmlufV9cXGFzdCl9KFxcbWF0aGNhbHtNfSleXFxvdGltZXMgIl0sWzAsMSwiXFxtYXRoY2Fse0N9XFx0aW1lcyBcXG1hdGhybXtBY3R9KE4oXFxtYXRoc2Z7RmlufV9cXGFzdCkpIl0sWzEsMCwiXFxtYXRoY2Fse0R9Il0sWzEsMSwiXFxtYXRoY2Fse0N9XFx0aW1lcyBcXG9wZXJhdG9ybmFtZXtGdW59KFxcRGVsdGFeMSxOKFxcbWF0aHNme0Zpbn1fXFxhc3QpKSJdLFsyLDAsIlxcbWF0aGNhbHtNfV5cXG90aW1lcyAiXSxbMiwxLCJcXG1hdGhjYWx7Q31cXHRpbWVzIE4oXFxtYXRoc2Z7RmlufV9cXGFzdCkiXSxbMSwyLCJcXG1hdGhjYWx7Q31cXHRpbWVzIE4oXFxtYXRoc2Z7RmlufV9cXGFzdCkiXSxbMCwyLCJcXG1hdGhjYWx7Q31cXHRpbWVzIE4oXFxtYXRoc2Z7RmlufV9cXGFzdCkiXSxbMCwxLCJxXzAiLDJdLFswLDIsIlxcYWxwaGEiXSxbMiwzXSxbMiw0LCJcXGJldGEiXSxbNCw1LCJwIl0sWzMsNSwiXFxvcGVyYXRvcm5hbWV7aWR9X3tcXG1hdGhjYWx7Q319XFx0aW1lcyBcXG9wZXJhdG9ybmFtZXtldn1fMCIsMl0sWzEsM10sWzMsNiwiXFxvcGVyYXRvcm5hbWV7aWR9X3tcXG1hdGhjYWx7Q319XFx0aW1lcyBcXG9wZXJhdG9ybmFtZXtldn1fMSJdLFsxLDcsIlxcb3BlcmF0b3JuYW1le2lkfV97XFxtYXRoY2Fse0N9fVxcdGltZXMgXFxvcGVyYXRvcm5hbWV7ZXZ9XzEiLDJdLFs3LDYsImVxdWFsIiwyXV0=
\[\begin{tikzcd}
	{\operatorname{Env}_{\mathcal{C}\times N(\mathsf{Fin}_\ast)}(\mathcal{M})^\otimes } & {\mathcal{D}} & {\mathcal{M}^\otimes } \\
	{\mathcal{C}\times \mathrm{Act}(N(\mathsf{Fin}_\ast))} & {\mathcal{C}\times \operatorname{Fun}(\Delta^1,N(\mathsf{Fin}_\ast))} & {\mathcal{C}\times N(\mathsf{Fin}_\ast)} \\
	{\mathcal{C}\times N(\mathsf{Fin}_\ast)} & {\mathcal{C}\times N(\mathsf{Fin}_\ast)}
	\arrow["{q_0}"', from=1-1, to=2-1]
	\arrow["\alpha", from=1-1, to=1-2]
	\arrow[from=1-2, to=2-2]
	\arrow["\beta", from=1-2, to=1-3]
	\arrow["p", from=1-3, to=2-3]
	\arrow["{\operatorname{id}_{\mathcal{C}}\times \operatorname{ev}_0}"', from=2-2, to=2-3]
	\arrow[from=2-1, to=2-2]
	\arrow["{\operatorname{id}_{\mathcal{C}}\times \operatorname{ev}_1}", from=2-2, to=3-2]
	\arrow["{\operatorname{id}_{\mathcal{C}}\times \operatorname{ev}_1}"', from=2-1, to=3-1]
	\arrow[equal, from=3-1, to=3-2]
\end{tikzcd}\]whose squares in the top rows are cartesian. We must show that $\gamma$
is an $\pr{\id_{\cal C}\times\opn{ev}_{1}}\circ q_{0}$-limit cone.
Since relative limits are stable under pullbacks \cite[Proposition 4.3.1.5]{HTT},
it suffices to show that $\beta\alpha\gamma$ is a $p$-limit cone
and that $q_{0}\gamma$ is an $\id_{\cal C}\times\opn{ev}_{1}$-limit
cone. First we consider $\beta\alpha\gamma$. Write $E=\pr{M,\alpha:\inp{n_{M}}\to\inp n}$
and $E_{i}=\pr{M_{i},\alpha_{i}:\inp{n_{M_{i}}}\to\inp 1}$. The morphism
$\gamma_{i}:E\to E_{i}$ can be depicted by a diagram% https://q.uiver.app/?q=WzAsNixbMSwwLCJcXGxhbmdsZSBuX01cXHJhbmdsZSJdLFsxLDEsIlxcbGFuZ2xlIG5fe01faX1cXHJhbmdsZSJdLFsyLDAsIlxcbGFuZ2xlIG1cXHJhbmdsZSJdLFsyLDEsIlxcbGFuZ2xlIDFcXHJhbmdsZSJdLFswLDAsIk0iXSxbMCwxLCJNX2kiXSxbMiwzLCJcXHJob15pIl0sWzEsMywiXFxhbHBoYV9pIiwyXSxbMCwyLCJcXGFscGhhIl0sWzAsMSwiXFxiZXRhX2kiLDJdLFs0LDUsImZfaSIsMl1d
\[\begin{tikzcd}
	M & {\langle n_M\rangle} & {\langle m\rangle} \\
	{M_i} & {\langle n_{M_i}\rangle} & {\langle 1\rangle}
	\arrow["{\rho^i}", from=1-3, to=2-3]
	\arrow["{\alpha_i}"', from=2-2, to=2-3]
	\arrow["\alpha", from=1-2, to=1-3]
	\arrow["{\beta_i}"', from=1-2, to=2-2]
	\arrow["{f_i}"', from=1-1, to=2-1]
\end{tikzcd}\]Since $\gamma_{i}$ is $q$-cocartesian, the morphism $f_{i}$ is
inert. Hence $\beta_{i}$ is inert. Since the horizontal arrows are
active, the commutativity of the diagram implies that the map $\beta_{i}:\inp{n_{M}}\to\inp{n_{M_{i}}}$
is the inert map which induces a bijection $\alpha^{-1}\pr i\cong\inp{n_{M_{i}}}^{\circ}$.
Combining this with the fact that $p$ is a $\cal C$-family of $\infty$-operads,
we deduce that the morphisms $\{f_{i}\}_{i}=\{\beta\alpha\gamma_{i}\}_{i}$
form a $p$-limit cone. For $q_{0}\gamma$, we use Lemma \ref{lem:2.2.4.16}.

To prove (4), write $E_{i}=\pr{M_{i},\alpha_{i}:\inp{n_{M_{i}}}\to\inp 1}$.
Let $n_{M}=n_{M_{1}}+\cdots+n_{M_{k}}$, and fix a bijection $\inp{n_{M}}\cong\Vee_{i=1}^{k}\inp{n_{M_{i}}}$.
Since $\cal M_{C}^{\t}$ is an $\infty$-operad, we can find an object
$M\in\cal C_{\pr{C,\inp{n_{M}}}}^{\t}$ and which admits a $p$-cocartesian
morphism $f_{i}:M\to M_{i}$ lying over the inert map $\beta_{i}:\inp{n_{M}}\to\inp{n_{M_{i}}}$
for each $1\leq i\leq k$. Let $\alpha:\inp{n_{M}}\to\inp k$ denote
the function which maps $\inp{n_{M_{i}}}^{\circ}$ to $i\in\inp k^{\circ}$,
and set $E=\pr{M,\alpha:\inp{n_{M}}\to\inp k}$. The triple $\pr{f_{i},\beta_{i},\rho^{i}}$
determines a morphism $E\to E_{i}$ which is $q$-cocartesian by part
(b). This proves (4).
\end{proof}

\subsection{\label{subsec:Universal-Property-of_monoidal_env}Universal Property
of Monoidal Envelopes}

In this subsection, we prove that the monoidal envelope of families
of $\infty$-operads has the expected universal property. The contents
of this subsection will not be used elsewhere in the note.

We begin with a definition.
\begin{defn}
Let $\cal C$ be an $\infty$-category and let $p:\cal M^{\t}\to\cal C\times N\pr{\Fin_{\ast}}$
a $\cal C$-family of $\infty$-operads. Let $q:\cal D^{\t}\to N\pr{\Fin_{\ast}}$
be a symmetric monoidal $\infty$-category. We let $\Fun^{\t}\pr{\Env\pr{\cal M},\cal D}\subset\Fun_{N\pr{\Fin_{\ast}}}\pr{\Env\pr{\cal M}^{\t},\cal D^{\t}}$
denote the full subcategory spanned by the functors which restrict
to a symmetric monoidal functor $\Env\pr{\cal M_{C}}^{\t}\to\cal D^{\t}$
for each $C\in\cal C$. 
\end{defn}
Here is the universal property of monoidal envelopes. The result is
a generalization of \cite[Proposition 2.2.4.9]{HA}.
\begin{prop}
Let $\cal C$ be an $\infty$-category and $p:\cal M^{\t}\to\cal C\times N\pr{\Fin_{\ast}}$
a $\cal C$-family of $\infty$-operads. Let $q:\cal D^{\t}\to N\pr{\Fin_{\ast}}$
be a symmetric monoidal $\infty$-category. The embedding $i:\cal M^{\t}\hookrightarrow\Env\pr{\cal M}^{\t}$
induces an equivalence of $\infty$-categories
\[
\Fun^{\t}\pr{\Env\pr{\cal M},\cal D}\xrightarrow{\simeq}\Alg_{\cal M}\pr{\cal D}.
\]
\end{prop}
\begin{proof}
Let $\cal E^{\t}$ denote the essential image of $i$. In other words,
$\cal E^{\t}$ is the full subcategory of $\Env\pr{\cal M}^{\t}$
spanned by those objects $\pr{M,\alpha:p\pr M\to\inp n}$ such that
$\alpha$ is an equivalence. The restriction map $\Alg_{\cal E}\pr{\cal D}\to\Alg_{\cal M}\pr{\cal D}$
is an equivalence of $\infty$-categories, so it will suffice to show
that the map
\[
\Fun^{\t}\pr{\Env\pr{\cal M},\cal D}\to\Alg_{\cal E}\pr{\cal D}
\]
is a trivial fibration. For this, it suffices to prove the following:

\begin{enumerate}[label=(\alph*)]

\item Every functor $\cal E^{\t}\to\cal D^{\t}$ over $N\pr{\Fin_{\ast}}$
admits a $q$-left Kan extension $\Env\pr{\cal M}^{\t}\to\cal D^{\t}$.

\item A functor $F\in\Fun_{N\pr{\Fin_{\ast}}}\pr{\Env\pr{\cal M},\cal D}$
restricts to a symmetric monoidal functor $\Env\pr{\cal M_{C}}^{\t}\to\cal D^{\t}$
for each $C\in\cal C$ if and only if it is a $q$-left Kan extension
of $F\vert_{\cal E^{\t}}$ and $F\vert_{\cal E^{\t}}$ is a morphism
of generalized $\infty$-operads.

\end{enumerate}

We start from (a). For each object $\pr{M,\alpha:p_{2}\pr M\to\inp n}$
in $\Env\pr{\cal M}^{\t}$, the $\infty$-category $\cal E_{/\pr{M,\alpha}}^{\t}$
has a terminal object, given by the map $\pr{\id_{M},\alpha}:\pr{M,\id_{p_{2}\pr M}}\to\pr{M,\alpha}$
(Lemma \ref{lem:2.2.4.13}). Therefore, if $F:\cal E^{\t}\to\cal D^{\t}$
is an arbitrary functor over $N\pr{\Fin_{\ast}}$, then the diagram
$\cal E_{/\pr{M,\alpha}}^{\t}\to\cal E^{\t}\xrightarrow{F}\cal D^{\t}$
has a $q$-colimit cone if and only if there is a $q$-cocartesian
morphism $F\pr{M,\id_{p_{2}\pr M}}\to D$ in $\cal D$ lifting $\alpha$.
The existence of such a $q$-cocartesian morphism follows from the
fact that $q$ is a cocartesian fibration.

For (b), let $p':\Env\pr{\cal M}^{\t}\to\cal C\times N\pr{\Fin_{\ast}}$
denote the projection. By the discussion in the previous paragraph,
it will suffice to show that $F$ preserves $p'$-cocartesian morphisms
lying over degenerate edges of $\cal X$ if and only if for each object
$\pr{M,\alpha:p_{2}\pr M\to\inp n}\in\Env\pr{\cal M}^{\t}$, the map
$F\pr{M,\id_{p_{2}\pr M}}\to F\pr{M,\alpha}$ is a $q$-cocartesian
morphism. Necessity is clear, since the morphism $\pr{M,\id_{p_{2}\pr M}}\to\pr{M,\alpha}$
is $p'$-cocartesian by Proposition \ref{prop:2.2.4.4}. For sufficiency,
suppose that every morphism of the form $\pr{M,\id_{p_{2}\pr M}}\to\pr{M,\alpha}$
is mapped to a $q$-cocartesian morphism. Let $\pr{f,g}:\pr{M,\alpha:p_{2}\pr M\to\inp n}\to\pr{M',\alpha':p_{2}\pr{M'}\to\inp{n'}}$
be a $p'$-cocartesian morphism in $\Env\pr{\cal M^{\t}}$ lying over
a degenerate edge of $\cal C$. We wish to show that $F\pr{f,g}$
is $q$-cocartesian. By hypothesis, the map $F\pr{M,\id_{p_{2}\pr M}}\to F\pr{M,\alpha}$
is $q$-cocartesian. So it suffice to show that the composite $F\pr{M,\id_{p_{2}\pr M}}\to F\pr{M,\alpha}\to F\pr{M',\alpha'}$
is $q$-cocartesian. We may therefore reduce to the case $\alpha=\id_{p_{2}\pr M}$.
Factor the map $g:p_{2}\pr M\to\inp{n'}$ as 
\[
p_{2}\pr M\xrightarrow{g'}\inp{n''}\xrightarrow{g''}\inp{n'},
\]
where $g'$ is inert and $g''$ is active. Lift $\pr{\id_{p_{1}\pr M},g'}$
to a $p$-cocartesian morphism $f':M\to M''$ in $\cal M^{\t}$. The
morphism $\pr{f',g'}:\pr{M,\id_{p_{2}\pr M}}\to\pr{M'',\id_{\inp{n''}}}$
is $p'$-cocartesian, so we can find a triangle in $\Env\pr{\cal M_{p_{1}\pr M}}^{\t}$
which we depict as % https://q.uiver.app/?q=WzAsNixbMCwzLCJwXzIoTSkiXSxbMSwyLCJcXGxhbmdsZSBuJydcXHJhbmdsZSJdLFsyLDMsIlxcbGFuZ2xlIG4nXFxyYW5nbGUiXSxbMiwxLCJwXzIoTScpIl0sWzEsMCwicF8yKE0nJykiXSxbMCwxLCJwXzIoTSkiXSxbMCwxLCJnJyJdLFsxLDIsImcnJyJdLFswLDIsInAoZykiLDJdLFszLDIsIlxcYWxwaGEnIl0sWzQsMywicChmJycpIl0sWzUsNCwicChmJykiXSxbNCwxLCJlcXVhbCIsMix7ImxhYmVsX3Bvc2l0aW9uIjo3MH1dLFs1LDAsImVxdWFsIiwyXSxbNSwzLCJwKGYpIl1d
\[\begin{tikzcd}
	& {p_2(M'')} \\
	{p_2(M)} && {p_2(M')} \\
	& {\langle n''\rangle} \\
	{p_2(M)} && {\langle n'\rangle}
	\arrow["{g'}", from=4-1, to=3-2]
	\arrow["{g''}", from=3-2, to=4-3]
	\arrow["{p(g)}"', from=4-1, to=4-3]
	\arrow["{\alpha'}", from=2-3, to=4-3]
	\arrow["{p(f'')}", from=1-2, to=2-3]
	\arrow["{p(f')}", from=2-1, to=1-2]
	\arrow[equal, from=1-2, to=3-2]
	\arrow[equal, from=2-1, to=4-1]
	\arrow["{p(f)}", from=2-1, to=2-3]
\end{tikzcd}\]The morphism $f''$ is both active and inert, so it is an equivalence.
Thus, by hypothesis, the map $F\pr{f'',g''}$ is a $q$-cocartesian.
The morphism $\pr{f',g'}$ is also $q$-cocartesian, since it is an
inert morphism of $\cal E^{\t}$. Hence $F\pr{f,g}$ is $q$-cocartesian,
as required.
\end{proof}

\section{\label{sec:Direct_sum}The Fiberwise Direct Sum Functor}

Let $\cal O^{\t}$ be an $\infty$-operad. Given objects $X_{1},\dots,X_{n}\in\cal O$,
we can find an object $X\in\cal O_{\inp n}^{\t}$ which admits a $p$-cocartesian
morphism $X\to X_{i}$ over $\rho^{i}:\inp n\to\inp 1$ for each $i$.
Such an object is denoted by $X_{1}\oplus\cdots\oplus X_{n}$ \cite[Remark 2.1.1.15]{HA}.
By inspection, the object $X_{1}\oplus\cdots\oplus X_{n}$ is the
equivalent to the image of the object $\pr{X_{1},\dots,X_{n}}\in\pr{\cal O_{\act}^{\t}}^{n}$
under the tensor product of $\Env\pr{\cal O}=\cal O_{\act}^{\t}$.
So the operation $\oplus$ can be made functorial using monoidal envelopes.
In fact, we can do it fiberwise:
\begin{defn}
Let $\cal C$ be an $\infty$-category, let $\cal M^{\t}\to\cal C\times N\pr{\Fin_{\ast}}$
be a $\cal C$-family of $\infty$-operads, and let $n\geq1$ be a
positive integer. We define a functor
\[
\bigoplus_{i=1}^{n}:\pr{\cal M_{\act}^{\t}}^{n}\times_{\cal C^{n}}\cal C=\cal M_{\act}^{\t}\times_{\cal C}\cdots\times_{\cal C}\cal M_{\act}^{\t}\to\cal M_{\act}^{\t}
\]
over $\cal C$, well-defined up to natural equivalence over $\cal C$,
as follows: According to Proposition \ref{prop:sec1_main}, there
is an equivalence of $\infty$-categories
\[
\Env\pr{\cal M}_{\inp n}^{\t}\xrightarrow{\simeq}\pr{\cal M_{\act}^{\t}}^{n}\times_{\cal C^{n}}\cal C
\]
over $\cal C$, obtained from the functors $\rho_{!}^{i}:\Env\pr{\cal M}_{\inp n}^{\t}\to\Env\pr{\cal M}_{\inp 1}^{\t}=\cal M_{\act}^{\t}$.
The functors $\Env\pr{\cal M}_{\inp n}^{\t}\to\cal C$ and $\pr{\cal M_{\act}^{\t}}^{n}\times_{\cal C^{n}}\cal C\to\cal C$
are categorical fibrations, so the above equivalence admits an inverse
equivalence
\[
\pr{\cal M_{\act}^{\t}}^{n}\times_{\cal C^{n}}\cal C\xrightarrow{\simeq}\Env\pr{\cal M}_{\inp n}^{\t}
\]
over $\cal C$. The functor $\bigoplus_{i=1}^{n}$ is the composite
\[
\pr{\cal M_{\act}^{\t}}^{n}\times_{\cal C^{n}}\cal C\xrightarrow{\simeq}\Env\pr{\cal M}_{\inp n}^{\t}\xrightarrow{\text{forget}}\cal M_{\act}^{\t}.
\]
\end{defn}
In this section, we will prove two important properties of the direct
sum functor: Its universal property and the interaction with slices.

\subsection{\label{subsec:Universal-Property-of}Universal Property of the Direct
Sum Functor }

In this subsection, we will characterize the direct sum functor by
a certain universal property (Corollary \ref{cor:oplus_p-limit}).
\begin{prop}
\label{prop:inverting_weak_equiv_of_pairs}Let $\cal M$ be a model
category. Suppose we are given a commutative diagram % https://q.uiver.app/?q=WzAsNCxbMCwwLCJFIl0sWzEsMCwiRSciXSxbMSwxLCJCJyJdLFswLDEsIkIiXSxbMCwxLCJmIl0sWzEsMiwicCciLDAseyJzdHlsZSI6eyJoZWFkIjp7Im5hbWUiOiJlcGkifX19XSxbMCwzLCJwIiwyLHsic3R5bGUiOnsiaGVhZCI6eyJuYW1lIjoiZXBpIn19fV0sWzMsMiwicSIsMix7InN0eWxlIjp7ImhlYWQiOnsibmFtZSI6ImVwaSJ9fX1dLFszLDIsIlxcc2ltZXEiXSxbMCwxLCJcXHNpbWVxIiwyXV0=
\[\begin{tikzcd}
	E & {E'} \\
	B & {B'}
	\arrow["f", from=1-1, to=1-2]
	\arrow["{p'}", two heads, from=1-2, to=2-2]
	\arrow["p"', two heads, from=1-1, to=2-1]
	\arrow["q"', two heads, from=2-1, to=2-2]
	\arrow["\simeq", from=2-1, to=2-2]
	\arrow["\simeq"', from=1-1, to=1-2]
\end{tikzcd}\]in $\cal M$. Assume the following:
\begin{enumerate}
\item The maps $p,p',q$ are fibrations.
\item The maps $f$ and $q$ are weak equivalences.
\item The object $E'$ is cofibrant. 
\item The map $q$ has a section $s:B'\to B$.
\end{enumerate}
Then there is a map $g:E'\to E$ rendering the diagram % https://q.uiver.app/?q=WzAsNCxbMCwwLCJFJyJdLFswLDEsIkInIl0sWzEsMCwiRSJdLFsxLDEsIkIiXSxbMCwxLCJwJyJdLFswLDIsImciXSxbMSwzLCJzIiwyXSxbMiwzLCJwIl1d
\[\begin{tikzcd}
	{E'} & E \\
	{B'} & B
	\arrow["{p'}", from=1-1, to=2-1]
	\arrow["g", from=1-1, to=1-2]
	\arrow["s"', from=2-1, to=2-2]
	\arrow["p", from=1-2, to=2-2]
\end{tikzcd}\]commutative, such that the composite $fg:E'\to E'$ is homotopic to
the identity in $\cal M_{/B'}$.
\end{prop}
\begin{proof}
Consider the diagram % https://q.uiver.app/?q=WzAsNixbMiwwLCJFIl0sWzIsMSwiQiJdLFsxLDEsIkInIl0sWzEsMCwiRSciXSxbMCwwLCJFIl0sWzAsMSwiQiJdLFswLDEsInAiLDJdLFsyLDEsInMiLDJdLFszLDIsInAnIiwyXSxbNSwyLCJxIiwyXSxbNCw1LCJwIiwyXSxbNCwzLCJmIl0sWzMsMCwiIiwyLHsic3R5bGUiOnsiYm9keSI6eyJuYW1lIjoiZGFzaGVkIn19fV1d
\[\begin{tikzcd}
	E & {E'} & E \\
	B & {B'} & B
	\arrow["p"', from=1-3, to=2-3]
	\arrow["s"', from=2-2, to=2-3]
	\arrow["{p'}"', from=1-2, to=2-2]
	\arrow["q"', from=2-1, to=2-2]
	\arrow["p"', from=1-1, to=2-1]
	\arrow["f", from=1-1, to=1-2]
	\arrow[dashed, from=1-2, to=1-3]
\end{tikzcd}\]in $\cal M_{/B'}$. Since $p$ is a fibration between fibrant objects
and $E'$ is cofibrant, \cite[Proposition A.2.3.1]{HTT} shows that
there is a map $g:E'\to E$ rendering the diagram commutative, such
that $[g][f]=[\id_{E}]$ in $\ho\pr{\cal M_{/B'}}$. Since $[f]$
is an isomorphism in $\ho\pr{\cal M_{/B'}}$, the uniqueness of inverses
implies that $[f][g]=[\id_{E'}]$ in $\ho\pr{\cal M_{/B'}}$. Since
$E'$ is a fibrant-cofibrant object of $\cal M_{/B'}$, we deduce
that $fg$ is homotopic to the identity in $\cal M_{/B'}$.
\end{proof}
\begin{defn}
Let $n\geq1$. Given integers $m_{1},\dots,m_{n}\geq0$, we shall
identify the pointed set $\Vee_{i=1}^{n}\inp{m_{i}}$ with the set
$\inp{m_{1}+\cdots+m_{n}}$ via the map
\[
\Vee_{i=1}^{n}\inp{m_{i}}\to\inp{m_{1}+\cdots+m_{n}}
\]
which maps $k\in\Vee_{i=1}^{n}\inp{m_{i}}$ to $\sum_{j<i}m_{j}+k$.
The maps $\Vee_{i=1}^{n}\inp{m_{i}}\to\inp{m_{i}}$ define an inert
natural transformation
\[
h_{i}:N\pr{\Fin_{\ast}}_{\act}^{n}\times\Delta^{1}\to N\pr{\Fin_{\ast}}.
\]
\end{defn}
\begin{prop}
\label{prop:oplus_p-limit}Let $n\ge1$, let $\cal C$ be an $\infty$-category,
and let $p:\cal M^{\t}\to\cal C\times N\pr{\Fin_{\ast}}$ be a $\cal C$-family
of $\infty$-operads. We can construct the functor $\bigoplus_{1\leq i\leq n}:\pr{\cal M_{\act}^{\t}}^{n}\times_{\cal C^{n}}\cal C\to\cal M^{\t}$
so that for each $1\leq i\leq n$, there is an inert natural transformation
$\bigoplus_{1\leq i\leq n}\to\opn{pr}_{i}$ rendering the diagram
% https://q.uiver.app/?q=WzAsNCxbMCwwLCIoKFxcbWF0aGNhbHtNfV5cXG90aW1lc197XFxtYXRocm17YWN0fX0pXm5cXHRpbWVzIF97XFxtYXRoY2Fse0N9Xm59XFxtYXRoY2Fse0N9KVxcdGltZXMgXFxEZWx0YSBeMSJdLFsxLDAsIlxcbWF0aGNhbHtNfV5cXG90aW1lcyAiXSxbMSwxLCJcXG1hdGhjYWx7Q31cXHRpbWVzIE4oXFxtYXRoc2Z7RmlufV9cXGFzdCkiXSxbMCwxLCJcXG1hdGhjYWx7Q31cXHRpbWVzIE4oXFxtYXRoc2Z7RmlufV9cXGFzdCApXm5fe1xcbWF0aHJte2FjdH19XFx0aW1lcyBcXERlbHRhXjEiXSxbMCwxLCJcXHdpZGV0aWxkZXtofV9pIl0sWzEsMiwicCJdLFszLDIsIlxcb3BlcmF0b3JuYW1le2lkfV97XFxtYXRoY2Fse0N9fVxcdGltZXMgaF9pIiwyXSxbMCwzXV0=
\[\begin{tikzcd}
	{((\mathcal{M}^\otimes_{\mathrm{act}})^n\times _{\mathcal{C}^n}\mathcal{C})\times \Delta ^1} & {\mathcal{M}^\otimes } \\
	{\mathcal{C}\times N(\mathsf{Fin}_\ast )^n_{\mathrm{act}}\times \Delta^1} & {\mathcal{C}\times N(\mathsf{Fin}_\ast)}
	\arrow["{\widetilde{h}_i}", from=1-1, to=1-2]
	\arrow["p", from=1-2, to=2-2]
	\arrow["{\operatorname{id}_{\mathcal{C}}\times h_i}"', from=2-1, to=2-2]
	\arrow[from=1-1, to=2-1]
\end{tikzcd}\]commutative.
\end{prop}
\begin{proof}
We begin with the construction of the functor $\bigoplus_{1\leq i\leq n}$.
For each $1\leq i\leq n$, there is an inert natural transformation
$g_{i}:\Env\pr{N\pr{\Fin_{\ast}}}_{\inp n}^{\t}\times\Delta^{1}\to\Env\pr{N\pr{\Fin_{\ast}}}^{\t}$
from the inclusion to the functor 
\[
\pr{\alpha:\inp k\to\inp n}\to\pr{\alpha^{-1}\pr i_{\ast}\to\inp 1},
\]
where $\alpha^{-1}\pr i_{\ast}$ denotes the unique object $\inp m\in\Fin_{\ast}$
which admits an order-preserving bijection $\alpha^{-1}\pr i\cong\inp m^{\circ}$.
This natural transformation covers $\rho^{i}:\inp n\to\inp 1$. Thus
we may construct the functor $\rho_{!}^{i}:\Env\pr{N\pr{\Fin_{\ast}}}_{\inp n}^{\t}\to N\pr{\Fin_{\ast}}_{\act}$
by $\rho^{i}\pr{\alpha:\inp k\to\inp n}=\alpha^{-1}\pr i_{\ast}$.
Since the functor $p$ is a categorical fibration, it induces a fibration
$\Env\pr{\cal M}^{\t}\to\cal C\times\Env\pr{N\pr{\Fin_{\ast}}}^{\t}$
of generalized $\infty$-operads. Therefore, we can find an inert
natural transformation $\Env\pr{\cal M}_{\inp n}^{\t}\times\Delta^{1}\to\Env\pr{\cal M}^{\t}$
rendering the diagram % https://q.uiver.app/?q=WzAsNSxbMCwwLCJcXG9wZXJhdG9ybmFtZXtFbnZ9KFxcbWF0aGNhbHtNfSleXFxvdGltZXMgX3tcXGxhbmdsZSBuXFxyYW5nbGV9XFx0aW1lcyBcXHswXFx9Il0sWzIsMCwiXFxvcGVyYXRvcm5hbWV7RW52fShcXG1hdGhjYWx7TX0pXlxcb3RpbWVzICJdLFsyLDEsIlxcbWF0aGNhbHtDfVxcdGltZXMgXFxvcGVyYXRvcm5hbWV7RW52fShOKFxcbWF0aHNme0Zpbn1fXFxhc3QpKV5cXG90aW1lcyAiXSxbMCwxLCJcXG9wZXJhdG9ybmFtZXtFbnZ9KFxcbWF0aGNhbHtNfSleXFxvdGltZXMgX3tcXGxhbmdsZSBuXFxyYW5nbGV9XFx0aW1lcyBcXERlbHRhXjEiXSxbMSwxLCJcXG1hdGhjYWx7Q31cXHRpbWVzIFxcb3BlcmF0b3JuYW1le0Vudn0oTihcXG1hdGhzZntGaW59X1xcYXN0KSleXFxvdGltZXMgX3tcXGxhbmdsZSBuXFxyYW5nbGV9XFx0aW1lcyBcXERlbHRhXjEiXSxbMSwyXSxbMCwzXSxbMywxLCIiLDEseyJzdHlsZSI6eyJib2R5Ijp7Im5hbWUiOiJkYXNoZWQifX19XSxbMyw0XSxbNCwyLCJcXG9wZXJhdG9ybmFtZXtpZH1fe1xcbWF0aGNhbHtDfX1cXHRpbWVzIGdfaSIsMl0sWzAsMV1d
\[\begin{tikzcd}
	{\operatorname{Env}(\mathcal{M})^\otimes _{\langle n\rangle}\times \{0\}} && {\operatorname{Env}(\mathcal{M})^\otimes } \\
	{\operatorname{Env}(\mathcal{M})^\otimes _{\langle n\rangle}\times \Delta^1} & {\mathcal{C}\times \operatorname{Env}(N(\mathsf{Fin}_\ast))^\otimes _{\langle n\rangle}\times \Delta^1} & {\mathcal{C}\times \operatorname{Env}(N(\mathsf{Fin}_\ast))^\otimes }
	\arrow[from=1-3, to=2-3]
	\arrow[from=1-1, to=2-1]
	\arrow[dashed, from=2-1, to=1-3]
	\arrow[from=2-1, to=2-2]
	\arrow["{\operatorname{id}_{\mathcal{C}}\times g_i}"', from=2-2, to=2-3]
	\arrow[from=1-1, to=1-3]
\end{tikzcd}\]commutative. We use this inert natural transformation to define the
functor $\rho_{!}^{i}:\Env\pr{\cal M}_{\inp n}^{\t}\to\cal M_{\act}^{\t}$.
This will ensure that the diagram

% https://q.uiver.app/?q=WzAsNCxbMCwxLCJcXG1hdGhjYWx7Q31cXHRpbWVzIChOKFxcbWF0aHNme0Zpbn1fXFxhc3QpX3tcXG1hdGhybXthY3R9fSlebiJdLFsyLDEsIlxcbWF0aGNhbHtDfVxcdGltZXMgXFxvcGVyYXRvcm5hbWV7RW52fShOKFxcbWF0aHNme0Zpbn1fXFxhc3QpKV5cXG90aW1lcyBfe1xcbGFuZ2xlIG4gXFxyYW5nbGV9Il0sWzIsMCwiXFxvcGVyYXRvcm5hbWV7RW52fShcXG1hdGhjYWx7TX0pXlxcb3RpbWVzIF97XFxsYW5nbGUgbiBcXHJhbmdsZX0iXSxbMCwwLCIoXFxtYXRoY2Fse019Xlxcb3RpbWVzIF97XFxtYXRocm17YWN0fX0pXm5cXHRpbWVzIF97XFxtYXRoY2Fse0N9Xm59XFxtYXRoY2Fse0N9Il0sWzMsMF0sWzEsMCwiXFxvcGVyYXRvcm5hbWV7aWR9X3tcXG1hdGhjYWx7Q319XFx0aW1lcyhcXHJob15pXyEpX3tpPTF9Xm4iXSxbMiwxXSxbMiwzLCIoXFxyaG9eaV8hKV97aT0xfV5uIiwyXV0=
\[\begin{tikzcd}
	{(\mathcal{M}^\otimes _{\mathrm{act}})^n\times _{\mathcal{C}^n}\mathcal{C}} && {\operatorname{Env}(\mathcal{M})^\otimes _{\langle n \rangle}} \\
	{\mathcal{C}\times (N(\mathsf{Fin}_\ast)_{\mathrm{act}})^n} && {\mathcal{C}\times \operatorname{Env}(N(\mathsf{Fin}_\ast))^\otimes _{\langle n \rangle}}
	\arrow[from=1-1, to=2-1]
	\arrow["{\operatorname{id}_{\mathcal{C}}\times(\rho^i_!)_{i=1}^n}", from=2-3, to=2-1]
	\arrow[from=1-3, to=2-3]
	\arrow["{(\rho^i_!)_{i=1}^n}"', from=1-3, to=1-1]
\end{tikzcd}\]is commutative. Now the functor $\pr{\rho_{!}^{i}}_{i=1}^{n}:\Env\pr{N\pr{\Fin_{\ast}}}_{\inp n}^{\t}\to\pr{\pr{N\pr{\Fin_{\ast}}}_{\act}}^{n}$
is a trivial fibration, and it has a section $\phi:\pr{N\pr{\Fin_{\ast}}_{\act}}^{n}\to\Env\pr{N\pr{\Fin_{\ast}}}_{\inp n}^{\t}$
given by 
\[
\pr{\inp{k_{i}}}_{i=1}^{n}\mapsto\pr{\Vee_{i=1}^{n}\inp{k_{i}}\to\Vee_{i=1}^{n}\inp 1=\inp n}.
\]
Applying Proposition \ref{prop:inverting_weak_equiv_of_pairs} to
the commutative diagram (\ref{d1}) and the section $\id_{\cal C}\times\phi$,
we can find a functor $\Phi:\pr{\cal M_{\act}^{\t}}^{n}\to\Env\pr{\cal M}_{\inp n}^{\t}$
with the following properties:
\begin{itemize}
\item The diagram % https://q.uiver.app/?q=WzAsNCxbMCwxLCJcXG1hdGhjYWx7Q31cXHRpbWVzIChOKFxcbWF0aHNme0Zpbn1fXFxhc3QpX3tcXG1hdGhybXthY3R9fSlebiJdLFsyLDEsIlxcbWF0aGNhbHtDfVxcdGltZXMgXFxvcGVyYXRvcm5hbWV7RW52fShOKFxcbWF0aHNme0Zpbn1fXFxhc3QpKV5cXG90aW1lcyBfe1xcbGFuZ2xlIG4gXFxyYW5nbGV9Il0sWzIsMCwiXFxvcGVyYXRvcm5hbWV7RW52fShcXG1hdGhjYWx7TX0pXlxcb3RpbWVzIF97XFxsYW5nbGUgbiBcXHJhbmdsZX0iXSxbMCwwLCIoXFxtYXRoY2Fse019Xlxcb3RpbWVzIF97XFxtYXRocm17YWN0fX0pXm5cXHRpbWVzIF97XFxtYXRoY2Fse0N9Xm59XFxtYXRoY2Fse0N9Il0sWzMsMF0sWzAsMSwiXFxvcGVyYXRvcm5hbWV7aWR9X3tcXG1hdGhjYWx7Q319XFx0aW1lc1xccGhpIiwyXSxbMiwxXSxbMywyLCJcXFBoaSJdXQ==
\[\begin{tikzcd}
	{(\mathcal{M}^\otimes _{\mathrm{act}})^n\times _{\mathcal{C}^n}\mathcal{C}} && {\operatorname{Env}(\mathcal{M})^\otimes _{\langle n \rangle}} \\
	{\mathcal{C}\times (N(\mathsf{Fin}_\ast)_{\mathrm{act}})^n} && {\mathcal{C}\times \operatorname{Env}(N(\mathsf{Fin}_\ast))^\otimes _{\langle n \rangle}}
	\arrow[from=1-1, to=2-1]
	\arrow["{\operatorname{id}_{\mathcal{C}}\times\phi}"', from=2-1, to=2-3]
	\arrow[from=1-3, to=2-3]
	\arrow["\Phi", from=1-1, to=1-3]
\end{tikzcd}\]is commutative. 
\item The composite $\pr{\rho_{!}^{i}}_{1\leq i\leq n}\circ\Phi$ is naturally
equivalent over $\cal C\times\pr{N\pr{\Fin_{\ast}}_{\act}}^{n}$ to
the identity functor. 
\end{itemize}
We will construct the functor $\bigoplus_{1\leq i\leq n}$ as the
composite
\[
\pr{\cal M_{\act}^{\t}}^{n}\times_{\cal C^{n}}\cal C\xrightarrow{\Phi}\Env\pr{\cal M}_{\inp n}^{\t}\xrightarrow{\text{forget}}\cal M^{\t}.
\]

Next, we show that, with our construction of the functor $\bigoplus_{1\leq i\leq n}$,
there is an inert natural transformation $\widetilde{h}_{i}:\pr{\pr{\cal M_{\act}^{\t}}^{n}\times_{\cal C^{n}}\cal C}\times\Delta^{1}\to\cal M^{\t}$
rendering the diagram in the statement commutative. Since $\opn{pr}_{i}$
is naturally equivalent over $\cal C\times N\pr{\Fin_{\ast}}$ to
the composite $\rho_{!}^{i}\Phi$, it suffices (by \cite[Proposition A.2.3.1]{HTT})
to find an inert natural transformation rendering the diagram % https://q.uiver.app/?q=WzAsNSxbMCwwLCIoKFxcbWF0aGNhbHtNfV5cXG90aW1lc197XFxtYXRocm17YWN0fX0pXm5cXHRpbWVzIF97XFxtYXRoY2Fse0N9Xm59XFxtYXRoY2Fse0N9KVxcdGltZXMgXFxwYXJ0aWFsXFxEZWx0YV4xIl0sWzAsMSwiKChcXG1hdGhjYWx7TX1eXFxvdGltZXNfe1xcbWF0aHJte2FjdH19KV5uXFx0aW1lcyBfe1xcbWF0aGNhbHtDfV5ufVxcbWF0aGNhbHtDfSlcXHRpbWVzIFxcRGVsdGFeMSJdLFsxLDEsIlxcbWF0aGNhbHtDfVxcdGltZXMgKE4oXFxtYXRoc2Z7RmlufV9cXGFzdClfe1xcbWF0aHJte2FjdH19KV5uXFx0aW1lcyBcXERlbHRhXjEiXSxbMiwxLCJcXG1hdGhjYWx7Q31cXHRpbWVzIE4oXFxtYXRoc2Z7RmlufV9cXGFzdCkiXSxbMiwwLCJcXG1hdGhjYWx7TX1eXFxvdGltZXMgIl0sWzAsMV0sWzEsMl0sWzIsMywiXFxvcGVyYXRvcm5hbWV7aWR9X3tcXG1hdGhjYWx7Q319XFx0aW1lcyBoX2kiLDJdLFs0LDMsInAiXSxbMCw0LCIoXFxiaWdvcGx1c197aT0xfV5uLFxccmhvXmlfIVxcUGhpICkiXSxbMSw0LCIiLDIseyJzdHlsZSI6eyJib2R5Ijp7Im5hbWUiOiJkYXNoZWQifX19XV0=
\[\begin{tikzcd}
	{((\mathcal{M}^\otimes_{\mathrm{act}})^n\times _{\mathcal{C}^n}\mathcal{C})\times \partial\Delta^1} && {\mathcal{M}^\otimes } \\
	{((\mathcal{M}^\otimes_{\mathrm{act}})^n\times _{\mathcal{C}^n}\mathcal{C})\times \Delta^1} & {\mathcal{C}\times (N(\mathsf{Fin}_\ast)_{\mathrm{act}})^n\times \Delta^1} & {\mathcal{C}\times N(\mathsf{Fin}_\ast)}
	\arrow[from=1-1, to=2-1]
	\arrow[from=2-1, to=2-2]
	\arrow["{\operatorname{id}_{\mathcal{C}}\times h_i}"', from=2-2, to=2-3]
	\arrow["p", from=1-3, to=2-3]
	\arrow["{(\bigoplus_{i=1}^n,\rho^i_!\Phi )}", from=1-1, to=1-3]
	\arrow[dashed, from=2-1, to=1-3]
\end{tikzcd}\]commutative. The outer rectangle is equal to that of the diagram % https://q.uiver.app/?q=WzAsNyxbMCwwLCIoKFxcbWF0aGNhbHtNfV5cXG90aW1lc197XFxtYXRocm17YWN0fX0pXm5cXHRpbWVzIF97XFxtYXRoY2Fse0N9Xm59XFxtYXRoY2Fse0N9KVxcdGltZXMgXFxwYXJ0aWFsXFxEZWx0YV4xIl0sWzAsMSwiKChcXG1hdGhjYWx7TX1eXFxvdGltZXNfe1xcbWF0aHJte2FjdH19KV5uXFx0aW1lc197XFxtYXRoY2Fse0N9Xm59XFxtYXRoY2Fse0N9KVxcdGltZXMgIFxcRGVsdGFeMSJdLFsyLDEsIlxcbWF0aGNhbHtDfVxcdGltZXMgXFxvcGVyYXRvcm5hbWV7RW52fShOKFxcbWF0aHNme0Zpbn1fXFxhc3QpKV5cXG90aW1lcyBfe1xcbGFuZ2xlIG5cXHJhbmdsZX1cXHRpbWVzIFxcRGVsdGFeMSJdLFszLDEsIlxcbWF0aGNhbHtDfVxcdGltZXMgTihcXG1hdGhzZntGaW59X1xcYXN0KSJdLFszLDAsIlxcbWF0aGNhbHtNfV5cXG90aW1lcyAiXSxbMSwxLCJcXG1hdGhjYWx7Q31cXHRpbWVzIFxcb3BlcmF0b3JuYW1le0Vudn0oXFxtYXRoY2Fse019KV5cXG90aW1lcyBfe1xcbGFuZ2xlIG5cXHJhbmdsZX1cXHRpbWVzIFxcRGVsdGFeMSJdLFsxLDAsIlxcbWF0aGNhbHtDfVxcdGltZXMgXFxvcGVyYXRvcm5hbWV7RW52fShcXG1hdGhjYWx7TX0pXlxcb3RpbWVzIF97XFxsYW5nbGUgbiBcXHJhbmdsZX1cXHRpbWVzIFxccGFydGlhbFxcRGVsdGFeMSJdLFswLDFdLFsyLDMsIlxcb3BlcmF0b3JuYW1le2lkfV97XFxtYXRoY2Fse0N9fVxcdGltZXMgZ19pIiwyXSxbNSwyXSxbMSw1LCJcXFBoaVxcdGltZXMgXFxvcGVyYXRvcm5hbWV7aWR9IiwyXSxbMCw2LCJcXFBoaVxcdGltZXMgXFxvcGVyYXRvcm5hbWV7aWR9Il0sWzYsNCwiKFxcdGV4dHtmb3JnZXR9LCBcXHJob15pXyEpIl0sWzYsNV0sWzQsM11d
\[\begin{tikzcd}[column sep=small, scale cd=.8]
	{((\mathcal{M}^\otimes_{\mathrm{act}})^n\times _{\mathcal{C}^n}\mathcal{C})\times \partial\Delta^1} & {\mathcal{C}\times \operatorname{Env}(\mathcal{M})^\otimes _{\langle n \rangle}\times \partial\Delta^1} && {\mathcal{M}^\otimes } \\
	{((\mathcal{M}^\otimes_{\mathrm{act}})^n\times_{\mathcal{C}^n}\mathcal{C})\times  \Delta^1} & {\mathcal{C}\times \operatorname{Env}(\mathcal{M})^\otimes _{\langle n\rangle}\times \Delta^1} & {\mathcal{C}\times \operatorname{Env}(N(\mathsf{Fin}_\ast))^\otimes _{\langle n\rangle}\times \Delta^1} & {\mathcal{C}\times N(\mathsf{Fin}_\ast)}
	\arrow[from=1-1, to=2-1]
	\arrow["{\operatorname{id}_{\mathcal{C}}\times g_i}"', from=2-3, to=2-4]
	\arrow[from=2-2, to=2-3]
	\arrow["{\Phi\times \operatorname{id}}"', from=2-1, to=2-2]
	\arrow["{\Phi\times \operatorname{id}}", from=1-1, to=1-2]
	\arrow["{(\text{forget}, \rho^i_!)}", from=1-2, to=1-4]
	\arrow[from=1-2, to=2-2]
	\arrow[from=1-4, to=2-4]
\end{tikzcd}\]so the existence of the desired filler follows from the definition
of $\rho_{!}^{i}$.
\end{proof}
%
\begin{cor}
[Universal Property of the Direct Sum Functor]\label{cor:oplus_p-limit}Let
$\cal C$ be an $\infty$-category and let $p:\cal M^{\t}\to\cal C\times N\pr{\Fin_{\ast}}$
be a $\cal C$-family of $\infty$-operads. Let $q:K\to\cal C$ be
an object of $\SS/\cal C$, let $n\geq1$, and let $f,g_{1},\dots,g_{n}:K\to\cal M_{\act}^{\t}$
be diagrams. Assume the following:
\begin{enumerate}
\item For each $1\leq i\leq n$, there is an inert natural transformation
$\alpha_{i}:f\to g_{i}$ over $\cal C$.
\item For each vertex $v$ in $K$, the maps $\alpha_{i}$ give rise to
a $p$-limit cone $\{f\pr v\to g_{i}\pr v\}_{1\leq i\le n}$. 
\end{enumerate}
Then $f$ is naturally equivalent over $\cal C$ to the composite
\[
G:K\xrightarrow{\pr{g_{i}}_{i=1}^{n}}\pr{\cal M_{\act}^{\t}}^{n}\times_{\cal C^{n}}\cal C\xrightarrow{\bigoplus_{1\leq i\leq n}}\cal M_{\act}^{\t}.
\]
\end{cor}
\begin{proof}
For each vertex $v$ in $K$, write $p_{2}\pr{g_{i}\pr v}=\inp{m_{i}\pr v}$
and $p_{2}\pr{f\pr v}=\inp{m\pr v}$. Let $\eta:K\times\Delta^{1}\to N\pr{\Fin_{\ast}}$
denote the natural equivalence which satisfies the following conditions: 
\begin{itemize}
\item The restriction $\eta\vert K\times\{0\}$ is given by $K\xrightarrow{\pr{p_{2}g_{i}}_{i}}N\pr{\Fin_{\ast}}_{\act}^{n}\xrightarrow{\Vee_{i=1}^{n}}N\pr{\Fin_{\ast}}$.
\item The restriction $\eta\vert K\times\{1\}$ is given by $p_{2}f$.
\item For each vertex $v\in K$ and $1\leq i\leq n$, the composite
\[
\inp{m\pr v}\xrightarrow{\eta}\inp{m\pr v}\xrightarrow{p_{2}\alpha_{i}}\inp{m_{i}\pr v}
\]
is defined as follows: Given $k\in\inp{m\pr v}^{\circ}$, the integer
$\sum_{j<i}m_{j}\pr v+k\in\inp{m\pr v}$ is mapped to $k$. The remaining
elements are mapped to the base point of $\inp{m_{i}\pr v}$.
\end{itemize}
Using the fact that $p$ is a categorical equivalence, we can find
a functor $f':K\to\cal M^{\t}$ and a natural equivalence $f'\xrightarrow{\simeq}f$
which lifts the natural equivalence $\pr{q\circ\opn{pr}_{1},\eta}:K\times\Delta^{1}\to\cal C\times N\pr{\Fin_{\ast}}$.
Replacing $f$ by $f'$ if necessary, we may assume that the diagram
% https://q.uiver.app/?q=WzAsNCxbMCwwLCJLXFx0aW1lcyBcXERlbHRhXjEiXSxbMSwwLCJcXG1hdGhjYWx7TX1eXFxvdGltZXMgIl0sWzEsMSwiXFxtYXRoY2Fse0N9XFx0aW1lcyBOKFxcbWF0aHNme0Zpbn1fXFxhc3QpIl0sWzAsMSwiKFxcbWF0aGNhbHtDfVxcdGltZXMgKE4oXFxtYXRoc2Z7RmlufV9cXGFzdClfe1xcbWF0aHJte2FjdH19KV5uKVxcdGltZXMgXFxEZWx0YV4xIl0sWzAsMSwiXFxhbHBoYV9pIl0sWzEsMiwicCJdLFszLDIsIlxcb3BlcmF0b3JuYW1le2lkfV97XFxtYXRoY2Fse0N9fVxcdGltZXMgaF9pIiwyXSxbMCwzLCIocSwocF8yZ19qKV9qKVxcdGltZXMgXFxvcGVyYXRvcm5hbWV7aWR9IiwyXV0=
\[\begin{tikzcd}
	{K\times \Delta^1} & {\mathcal{M}^\otimes } \\
	{(\mathcal{C}\times (N(\mathsf{Fin}_\ast)_{\mathrm{act}})^n)\times \Delta^1} & {\mathcal{C}\times N(\mathsf{Fin}_\ast)}
	\arrow["{\alpha_i}", from=1-1, to=1-2]
	\arrow["p", from=1-2, to=2-2]
	\arrow["{\operatorname{id}_{\mathcal{C}}\times h_i}"', from=2-1, to=2-2]
	\arrow["{(q,(p_2g_j)_j)\times \operatorname{id}}"', from=1-1, to=2-1]
\end{tikzcd}\]commutes for each $1\leq i\leq n$. Since relative limits in functor
categories can be formed objectwise \cite[\href{https://kerodon.net/tag/02XK}{Tag 02XK}]{kerodon},
the morphisms $\alpha_{i}:f\to g_{i}$ in $\Fun\pr{K,\cal M^{\t}}$
form a $\Fun\pr{K,p}$-limit cone. Moreover, since relative limits
are stable under pullbacks \cite[Proposition 4.3.1.5]{HTT}, we deduce
that the morphisms $\{\alpha_{i}\}_{1\leq i\leq n}$ of $\Fun_{\cal C}\pr{K,\cal M^{\t}}$
form a $\Fun_{\cal C}\pr{K,p}$-limit cone. 

Now using Proposition \ref{prop:oplus_p-limit}, we can construct
the functor $\bigoplus_{1\leq i\leq n}:\pr{\cal M_{\act}^{\t}}^{n}\times_{\cal C^{n}}\cal C\to\cal M^{\t}$
so that there is an inert natural transformation $\widetilde{h}_{i}:\bigoplus_{1\leq i\leq n}\to\opn{pr}_{i}$
which lifts the composite $\pr{\cal M_{\act}^{\t}}^{n}\times_{\cal C^{n}}\cal C\to\cal C\times N\pr{\Fin_{\ast}}_{\act}^{n}\times\Delta^{1}\xrightarrow{\id_{\cal C}\times h_{i}}\cal C\times N\pr{\Fin_{\ast}}$.
Again, the induced natural transformations $\{G\to g_{i}\}_{1\leq i\leq n}$
form a $\Fun_{\cal C}\pr{K,p}$-limit cone covering the same cone
as $\{\alpha_{i}\}$. Thus we obtain the desired equivalence $G\xrightarrow{\simeq}f$
in $\Fun_{\cal C}\pr{K,\cal M^{\t}}$.
\end{proof}
The above corollary admits the following further corollary, which
we shall use in the next section.
\begin{cor}
\label{cor:3.1.1.8}Let $q:\cal C^{\t}\to\cal O^{\t}$ be a fibration
of $\infty$-operads. Let $K_{1},\dots,K_{n}$ be simplicial sets,
and let $K=\prod_{1\leq i\leq n}K_{i}$. Let $\overline{f}:K^{\rcone}\to\cal C^{\t}$
and $\{\overline{f_{i}}:K^{\rcone}\to\cal C^{\t}\}_{1\leq i\leq n}$
be active diagrams, and suppose there are inert natural transformations
$\{\overline{f}\to\overline{f}_{i}\circ\pr{\opn{pr}_{i}}^{\rcone}\}_{1\le i\leq n}$
such that for each vertex $v$ in $K^{\rcone}$, the induced map $\{\overline{f}\pr v\to\overline{f}_{i}\circ\pr{\opn{pr}_{i}}^{\rcone}\pr v\}_{1\le i\leq n}$
form a $q$-limit cone. If each $\overline{f_{i}}$ is an operadic
$q$-colimit diagram, so is $\overline{f}$.
\end{cor}
\begin{proof}
According to Corollary \ref{cor:oplus_p-limit}, the diagram $\overline{f}$
is naturally equivalent to the composite
\[
K^{\rcone}\to\prod_{1\leq i\leq n}\pr{K_{i}^{\rcone}}\xrightarrow{\prod_{1\leq i\leq n}\overline{f_{i}}}\pr{\cal C_{\act}^{\t}}^{n}\xrightarrow{\bigoplus_{1\leq i\leq n}}\cal C^{\t}.
\]
The claim is thus a consequence of \cite[Proposition 3.1.1.8]{HA}.
\end{proof}

\subsection{Direct Sum and Slice}

Let $\cal O^{\t}$ be an $\infty$-operad and let $Y\in\cal O^{\t}$
be an object. If $Y$ lies over an object $\inp n\in N\pr{\Fin_{\ast}}$
with $n\geq2$, then we can find objects $Y_{i}\in\cal O$ and an
equivalence $Y\simeq\bigoplus_{i=1}^{n}Y_{i}$. Given objects $X_{1},\dots,X_{n}\in\cal O^{\t}$
and active maps $f_{i}:X_{i}\to Y_{i}$, their direct sum $\bigoplus_{i=1}^{n}f_{i}:\bigoplus_{i=1}^{n}X_{i}\to\bigoplus_{i=1}^{n}Y_{i}\simeq Y$
is an active morphism with codomain $Y$. Conversely, given an active
morphism $f:X\to Y$ in $\cal O^{\t}$, we can write $f\simeq\bigoplus_{i=1}^{n}f_{i}$,
where $f_{i}$ is the active map obtained by factoring the composite
$X\to Y\to Y_{i}$ into an inert map followed by an active map. The
following proposition, which is the only result of this subsection,
asserts that this ``direct sum decomposition'' of morphisms is an
equivalence on the level of $\infty$-categories: 
\begin{prop}
\label{prop:directsum_slice_equiv}Let $\cal C$ be an $\infty$-category,
let $C\in\cal C$ be an object, and let $\cal M^{\t}\to\cal C\times N\pr{\Fin_{\ast}}$
be a $\cal C$-family of $\infty$-operads. For any integer $n\geq1$
and any objects $M_{1},\dots,M_{n}\in\cal M\times_{\cal C}\{C\}$,
the direct sum functor induces an equivalence of $\infty$-categories
\[
\pr{\pr{\cal M_{\act}^{\t}}^{n}\times_{\cal C^{n}}\cal C}_{/\pr{M_{1},\dots,M_{n}}}\xrightarrow{\simeq}\pr{\cal M_{\act}^{\t}}_{/\bigoplus_{i=1}^{n}M_{i}}.
\]
\end{prop}
\begin{proof}
Recall that the direct sum functor $\bigoplus_{i=1}^{n}$ is obtained
as the composite 
\[
\pr{\cal M_{\act}^{\t}}^{n}\times_{\cal C^{n}}\cal C\xrightarrow[\Phi]{\simeq}\Env\pr{\cal M}_{\inp n}^{\t}\xrightarrow{\text{forget}}\cal M_{\act}^{\t},
\]
where the map $\Phi$ is the inverse equivalence over $\cal C$ of
the functor $\pr{\rho_{!}^{i}}_{1\leq i\le n}:\Env\pr{\cal M}_{\inp n}^{\t}\xrightarrow{\simeq}\pr{\cal M_{\act}^{\t}}^{n}\times_{\cal C^{n}}\cal C$
. The functor $\Phi$ maps the object $\pr{M_{1},\dots,M_{n}}$ to
the object $\pr{\bigoplus_{i=1}^{n}M_{i},\alpha:\inp n\to\inp n}$,
where $\alpha$ is a bijection. It will therefore suffice to prove
that the functor 
\[
\theta:\pr{\Env\pr{\cal M}_{\inp n}^{\t}}_{/\pr{\bigoplus_{i=1}^{n}M_{i},\alpha}}\to\pr{\cal M_{\act}^{\t}}_{/\bigoplus_{i=1}^{n}M_{i}}
\]
is an equivalence of $\infty$-categories. Set $M=\bigoplus_{i=1}^{n}M_{i}$.
By definition, we have
\[
\pr{\Env\pr{\cal M}_{\inp n}^{\t}}_{/\pr{M,\alpha}}=\pr{\Act\pr{N\pr{\Fin_{\ast}}}_{\inp n}^{\t}}_{/\alpha}\times_{N\pr{\Fin_{\ast}}_{/\inp n}}\cal M_{/M}^{\t}.
\]
Since the evaluation at $0\in\Delta^{1}$ induces an isomorphism of
simplicial sets
\[
\pr{\Act\pr{N\pr{\Fin_{\ast}}}_{\inp n}^{\t}}_{/\alpha}\xrightarrow{\cong}\pr{N\pr{\Fin_{\ast}}_{\act}}_{/\inp n},
\]
we deduce that the functor $\theta$ is an \textit{isomorphism} of
simplicial sets.
\end{proof}

\section{\label{sec:FTOK}The Fundamental Theorem of Operadic Kan Extensions}

In this final section, we will provide complete details of the proof
of \cite[Theorem 3.1.2.3]{HA}, using tools and results in the preceding
sections.

\subsection{Recollection}

In this subsection, we recall the definition of operadic Kan extensions
and the statement of the fundamental theorem of operadic Kan extensions. 
\begin{defn}
\cite[Definition 3.1.2.2]{HA}Let $\cal M^{\t}\to N\pr{\Fin_{\ast}}\times\Delta^{1}$
be a $\Delta^{1}$-family of $\infty$-operads and let $q:\cal C^{\t}\to\cal O^{\t}$
be a fibration of $\infty$-operads. Set $\cal A^{\t}=\cal M^{\t}\times_{\Delta^{1}}\{0\}$,
$\cal B^{\t}=\cal M^{\t}\times_{\Delta^{1}}\{1\}$. A map $F:\cal M^{\t}\to\cal C^{\t}$
is called an \textbf{operadic $q$-left Kan extension of $F\vert\cal A^{\t}$
}if the following condition is satisfied for every object $B\in\cal B$:
\begin{itemize}
\item [($\ast$)]The composite map
\[
\pr{\pr{\cal M_{\act}^{\t}}_{/B}\times_{\Delta^{1}}\{0\}}^{\rcone}\to\pr{\cal M_{/B}^{\t}}^{\rcone}\to\cal M^{\t}\xrightarrow{\overline{F}}\cal C^{\t}
\]
is an operadic $q$-colimit diagram.
\end{itemize}
\end{defn}
Lurie imposes condition ($\ast$) to hold for every object $B\in\cal B^{\t}$
in \cite[Definition 3.1.2.2]{HA}. This is not a problem, because
of the following proposition, which seems to be implicitly used in
\cite{HA}.
\begin{prop}
\label{prop:operadic_Kan_equiv}Let $p:\cal M^{\t}\to N\pr{\Fin_{\ast}}\times\Delta^{1}$
be a correspondence from an $\infty$-operad $\cal A^{\t}$ to another
$\infty$-operad $\cal B^{\t}$. Let $q:\cal C^{\t}\to\cal O^{\t}$
be a fibration of $\infty$-operads and let $F:\cal M^{\t}\to\cal C^{\t}$
be a map of generalized $\infty$-operads. Suppose that $F$ is an
operadic $q$-left Kan extension of \textbf{$F\vert\cal A^{\t}$}.
Then for every object $B\in\cal B^{\t},$the map 
\[
\theta:\pr{\pr{\cal M_{\act}^{\t}}_{/B}\times_{\Delta^{1}}\{0\}}^{\rcone}\to\pr{\cal M_{/B}^{\t}}^{\rcone}\to\cal M^{\t}\xrightarrow{F}\cal C^{\t}
\]
is an operadic $q$-colimit diagram.
\end{prop}
\begin{proof}
Let $\inp n$ be the image of $B$ in $N\pr{\Fin_{\ast}}$. We consider
two cases, depending on whether or not $n$ is equal to $0$.

Suppose first that $n=0$. Then we have $\pr{\cal M_{\act}^{\t}}_{/B}=\pr{\cal M_{\inp 0}^{\t}}_{/B}$,
since there is no active map $\inp k\to\inp 0$ with $k\geq1$. Since
every object of $\cal M_{\inp 0}^{\t}$ is $p$-terminal, the functor
$\cal M_{\inp 0}^{\t}\to\Delta^{1}$ is a trivial fibration. It follows
that the functor $\pr{\cal M_{\act}^{\t}}_{/B}\times_{\Delta^{1}}\{0\}\to\Delta_{/1}^{1}\times_{\Delta^{1}}\{0\}\cong\Delta^{0}$
is also a trivial fibration. It will therefore suffice to show that
$F$ carries a morphism of the form $A\to B$ in $\cal M^{\t}$ with
$A\in\cal M_{\inp 0}^{\t}\times_{\Delta^{1}}\{0\}$ to an equivalence
in $\cal C^{\t}$. This is clear, since $\cal C_{\inp 0}^{\t}$ is
a contractible Kan complex.

If $n\geq1$, choose for each $1\leq i\leq n$ an inert map $B\to B_{i}$
in $\cal B^{\t}$ over $\rho^{i}:\inp n\to\inp 1$, and choose also
a direct sum functor $\bigoplus_{i=1}^{n}$ for $\cal M^{\t}$ and
$\cal C^{\t}$. Replacing $\bigoplus_{i=1}^{n}$ by a functor naturally
equivalent one, we may assume that $B=\bigoplus_{i=1}^{n}$. According
to Proposition \ref{prop:directsum_slice_equiv}, the direct sum functor
induces an equivalence of $\infty$-categories
\[
\phi:\pr{\pr{\cal M_{\act}^{\t}}^{n}\times_{\pr{\Delta^{1}}^{n}}\Delta^{1}}_{/\pr{B_{1},\dots,B_{n}}}\xrightarrow{\simeq}\pr{\cal M_{\act}^{\t}}_{/B}.
\]
There is also an isomorphism of simplicial sets
\[
\psi:\prod_{i=1}^{n}\pr{\cal M_{\act}^{\t}}_{/B_{i}}\times_{\Delta^{1}}\{0\}\cong\pr{\pr{\cal M_{\act}^{\t}}^{n}\times_{\pr{\Delta^{1}}^{n}}\Delta^{1}}_{/\pr{B_{1},\dots,B_{n}}}\times_{\Delta^{1}}\{0\}.
\]
It will therefore suffice to show that the composite $\theta\circ\phi^{\rcone}\circ\psi^{\rcone}$
is an operadic $q$-colimit diagram. For this, we consider the diagram
% https://q.uiver.app/?q=WzAsOSxbMCwwLCIoXFxwcm9kX3tpPTF9Xm4oXFxtYXRoY2Fse019Xlxcb3RpbWVzIF97XFxtYXRocm17YWN0fX0pX3svQl9pfVxcdGltZXMgX3tcXERlbHRhXjF9XFx7MFxcfSleXFx0cmlhbmdsZXJpZ2h0Il0sWzEsMCwiKCgoXFxtYXRoY2Fse019Xlxcb3RpbWVzIF97XFxtYXRocm17YWN0fX0pXm5cXHRpbWVzX3soXFxEZWx0YV4xKV5ufVxcRGVsdGFeMSlfey8oQl8xLFxcZG90cyAsQl9uKX1cXHRpbWVzX3tcXERlbHRhXjF9XFx7MFxcfSleXFx0cmlhbmdsZXJpZ2h0Il0sWzAsMSwiXFxwcm9kX3tpPTF9Xm4oKFxcbWF0aGNhbHtNfV5cXG90aW1lcyBfe1xcbWF0aHJte2FjdH19KV97L0JfaX1cXHRpbWVzIF97XFxEZWx0YV4xfVxcezBcXH0pXlxcdHJpYW5nbGVyaWdodCJdLFsxLDEsIihcXG1hdGhjYWx7TX1eXFxvdGltZXMgX3tcXG1hdGhybXthY3R9fSleblxcdGltZXNfeyhcXERlbHRhXjEpXm59XFxEZWx0YV4xIl0sWzIsMCwiKChcXG1hdGhjYWx7TX1eXFxvdGltZXMgX3tcXG1hdGhybXthY3R9fSlfey9CfVxcdGltZXMgX3tcXERlbHRhXjF9XFx7MFxcfSleXFx0cmlhbmdsZXJpZ2h0Il0sWzIsMSwiXFxtYXRoY2Fse019Xlxcb3RpbWVzIF97XFxtYXRocm17YWN0fX0iXSxbMCwyLCIoXFxtYXRoY2Fse019Xlxcb3RpbWVzIF97XFxtYXRocm17YWN0fX0pXm4iXSxbMSwyLCIoXFxtYXRoY2Fse0N9Xlxcb3RpbWVzIF97XFxtYXRocm17YWN0fX0pXm4iXSxbMiwyLCJcXG1hdGhjYWx7Q31eXFxvdGltZXMgX3tcXG1hdGhybXthY3R9fSJdLFswLDJdLFsxLDNdLFswLDEsIlxcY29uZyJdLFsxLDQsIlxcc2ltZXEiXSxbMCwxLCJcXHBzaV5cXHRyaWFuZ2xlcmlnaHQiLDJdLFsxLDQsIlxccGhpXlxcdHJpYW5nbGVyaWdodCIsMl0sWzQsNV0sWzMsNSwiXFxiaWdvcGx1c197aT0xfV5uIiwyXSxbMyw2LCIiLDIseyJzdHlsZSI6eyJ0YWlsIjp7Im5hbWUiOiJob29rIiwic2lkZSI6ImJvdHRvbSJ9fX1dLFsyLDZdLFs2LDcsIlxccHJvZF97aT0xfV5uRiIsMl0sWzMsNywiRyJdLFs1LDgsIkYiXSxbNyw4LCJcXGJpZ29wbHVzX3tpPTF9Xm4iLDJdXQ==
\[\begin{tikzcd}[column sep=small, scale cd=.8]
	{(\prod_{i=1}^n(\mathcal{M}^\otimes _{\mathrm{act}})_{/B_i}\times _{\Delta^1}\{0\})^\triangleright} & {(((\mathcal{M}^\otimes _{\mathrm{act}})^n\times_{(\Delta^1)^n}\Delta^1)_{/(B_1,\dots ,B_n)}\times_{\Delta^1}\{0\})^\triangleright} & {((\mathcal{M}^\otimes _{\mathrm{act}})_{/B}\times _{\Delta^1}\{0\})^\triangleright} \\
	{\prod_{i=1}^n((\mathcal{M}^\otimes _{\mathrm{act}})_{/B_i}\times _{\Delta^1}\{0\})^\triangleright} & {(\mathcal{M}^\otimes _{\mathrm{act}})^n\times_{(\Delta^1)^n}\Delta^1} & {\mathcal{M}^\otimes _{\mathrm{act}}} \\
	{(\mathcal{M}^\otimes _{\mathrm{act}})^n} & {(\mathcal{C}^\otimes _{\mathrm{act}})^n} & {\mathcal{C}^\otimes _{\mathrm{act}}.}
	\arrow[from=1-1, to=2-1]
	\arrow[from=1-2, to=2-2]
	\arrow["\cong", from=1-1, to=1-2]
	\arrow["\simeq", from=1-2, to=1-3]
	\arrow["{\psi^\triangleright}"', from=1-1, to=1-2]
	\arrow["{\phi^\triangleright}"', from=1-2, to=1-3]
	\arrow[from=1-3, to=2-3]
	\arrow["{\bigoplus_{i=1}^n}"', from=2-2, to=2-3]
	\arrow[hook', from=2-2, to=3-1]
	\arrow[from=2-1, to=3-1]
	\arrow["{\prod_{i=1}^nF}"', from=3-1, to=3-2]
	\arrow["G", from=2-2, to=3-2]
	\arrow["F", from=2-3, to=3-3]
	\arrow["{\bigoplus_{i=1}^n}"', from=3-2, to=3-3]
\end{tikzcd}\]Here the map $G$ is the restriction of the map $\prod_{i=1}^{n}F$.
The left column and the top right square are commutative. The bottom
right square commutes up to natural equivalence by Proposition \ref{prop:oplus_p-limit}
and Corollary \ref{cor:oplus_p-limit}. Hence the diagram $\theta\circ\phi^{\rcone}\circ\psi^{\rcone}$
is naturally equivalent to the composite
\[
\theta':\pr{\prod_{i=1}^{n}\pr{\cal M_{\act}^{\t}}_{/B_{i}}\times_{\Delta^{1}}\{0\}}^{\rcone}\to\prod_{i=1}^{n}\pr{\pr{\cal M_{\act}^{\t}}_{/B_{i}}\times_{\Delta^{1}}\{0\}}^{\rcone}\to\pr{\cal C_{\act}^{\t}}^{n}\xrightarrow{\bigoplus_{i=1}^{n}}\cal C_{\act}^{\t}.
\]
The diagram $\theta'$ is an operadic $q$-colimit diagram by \cite[Proposition 3.1.1.8]{HA}.
\end{proof}
We can now state the fundamental theorem of opeardic Kan extensions.
\begin{thm}
\cite[Theorem 3.1.2.3]{HA}\label{thm:3.1.2.3}Let $p:\cal M^{\t}\to N\pr{\Fin_{\ast}}\times\Delta^{n}$
be a $\Delta^{n}$-family of generalized $\infty$-operads, and let
$q:\cal C^{\t}\to\cal O^{\t}$ be a fibration of $\infty$-operads.
Consider a commutative diagram% https://q.uiver.app/?q=WzAsNCxbMCwwLCJcXG1hdGhjYWx7TX1eXFxvdGltZXMgXFx0aW1lcyBfe1xcRGVsdGFebn1cXExhbWJkYSBebl8wIl0sWzAsMSwiXFxtYXRoY2Fse019Xlxcb3RpbWVzICJdLFsxLDAsIlxcbWF0aGNhbHtDfV5cXG90aW1lcyAiXSxbMSwxLCJcXG1hdGhjYWx7T31eXFxvdGltZXMgLiJdLFswLDEsIiIsMCx7InN0eWxlIjp7InRhaWwiOnsibmFtZSI6Imhvb2siLCJzaWRlIjoidG9wIn19fV0sWzAsMiwiZl8wIl0sWzIsMywicSJdLFsxLDMsImciLDJdLFsxLDIsIiIsMCx7InN0eWxlIjp7ImJvZHkiOnsibmFtZSI6ImRhc2hlZCJ9fX1dXQ==
\[\begin{tikzcd}
	{\mathcal{M}^\otimes \times _{\Delta^n}\Lambda ^n_0} & {\mathcal{C}^\otimes } \\
	{\mathcal{M}^\otimes } & {\mathcal{O}^\otimes}
	\arrow[hook, from=1-1, to=2-1]
	\arrow["{f_0}", from=1-1, to=1-2]
	\arrow["q", from=1-2, to=2-2]
	\arrow["g"', from=2-1, to=2-2]
	\arrow[dashed, from=2-1, to=1-2]
\end{tikzcd}\]of simplicial sets. Suppose that for each vertex $i$ in $\Lambda_{0}^{n}$
and each vertex $j$ in $\Delta^{n}$, the induced maps $\cal M^{\t}\times_{\Delta^{n}}\{v\}\to\cal C^{\t}$
and $\cal M^{\t}\times_{\Delta^{n}}\{j\}\to\cal O^{\t}$ are morphisms
of $\infty$-operads.
\begin{itemize}
\item [(A)]If $n=1$, the following conditions are equivalent:
\begin{itemize}
\item [(a)]There is a dashed filler which is an operadic $q$-left Kan
extension of $f_{0}$.
\item [(b)]For each object $B\in\cal M^{\t}\times_{\Delta^{1}}\{1\}$,
the diagram
\[
\{0\}\times_{\Delta^{1}}\pr{\pr{\cal M_{\act}^{\t}}_{/B}}\to\{0\}\times_{\Delta^{1}}\cal M^{\t}\xrightarrow{f_{0}}\cal C^{\t}
\]
admits an operadic $q$-colimit cone which lifts the map
\[
\pr{\{0\}\times_{\Delta^{1}}\pr{\pr{\cal M_{\act}^{\t}}_{/B}}}^{\rcone}\to\pr{\cal M_{/B}^{\t}}^{\rcone}\to\cal M^{\t}\xrightarrow{g}\cal O^{\t}.
\]
\end{itemize}
\item [(B)]If $n>1$ and the restriction $f_{0}\vert\cal M^{\t}\times_{\Delta^{n}}\Delta^{\{0,1\}}$
is an operadic $q$-left Kan extension of $f_{0}\vert\cal M^{\t}\times_{\Delta^{n}}\{0\}$,
then there is a dashed arrow rendering the diagram commutative.
\end{itemize}
\end{thm}
The rest of this note is devoted to elaborating Lurie's proof of the
above theorem.

\subsection{Preliminary Results}

In this subsection, we collect some results which will be used in
the proof of Theorem \ref{prop:operadic_Kan_equiv}. We recommend
that the reader skip to the next subsection and refer to this subsection
as needed.
\begin{prop}
\label{prop:relative_terminal_obj_in_slice}Let $p:\cal C\to\cal D$
be an inner fibration of $\infty$-categories, $K$ a simplicial set,
and $\overline{f}:K^{\rcone}\to\cal C$ a diagram. Set $f=\overline{f}\vert K$
and let $q$ denote the functor $\cal C_{f/}\to\cal D_{pf/}$. If
$\overline{f}$ maps the cone point to a $p$-terminal object, then
$\overline{f}$ is $q$-terminal.
\end{prop}
\begin{proof}
We must show that the map 
\[
\pr{\cal C_{f/}}_{/\overline{f}}\to\cal C_{f/}\times_{\cal D_{pf/}}\pr{\cal D_{pf/}}_{/q\overline{f}}
\]
is a trivial fibration. Let $C=\overline{f}\pr{\infty}$. Given a
monomorphism $A\to B$ of simplicial sets, a lifting problem on the
left hand side corresponds under adjunction to a lifting problem on
the right hand side:% https://q.uiver.app/?q=WzAsOCxbMCwwLCJBIl0sWzAsMSwiQiJdLFsxLDAsIihcXG1hdGhjYWx7Q31fe2YvfSlfey9cXG92ZXJsaW5le2Z9fSJdLFsxLDEsIlxcbWF0aGNhbHtDfV97Zi99XFx0aW1lcyBfe1xcbWF0aGNhbHtEfV97cGYvfX0oXFxtYXRoY2Fse0R9X3twZi99KV97L3FcXG92ZXJsaW5le2Z9fSJdLFsyLDAsIktcXHN0YXIgQSJdLFsyLDEsIktcXHN0YXIgQiJdLFszLDAsIlxcbWF0aGNhbHtDfV97L0N9Il0sWzMsMSwiXFxtYXRoY2Fse0N9XFx0aW1lcyBfe1xcbWF0aGNhbHtEfX1cXG1hdGhjYWx7RH1fey9wQ30iXSxbMCwyXSxbMiwzXSxbMCwxXSxbMSwzXSxbMSwyLCIiLDEseyJzdHlsZSI6eyJib2R5Ijp7Im5hbWUiOiJkYXNoZWQifX19XSxbNCw1XSxbNCw2XSxbNiw3XSxbNSw3XSxbNSw2LCIiLDEseyJzdHlsZSI6eyJib2R5Ijp7Im5hbWUiOiJkYXNoZWQifX19XV0=
\[\begin{tikzcd}
	A & {(\mathcal{C}_{f/})_{/\overline{f}}} & {K\star A} & {\mathcal{C}_{/C}} \\
	B & {\mathcal{C}_{f/}\times _{\mathcal{D}_{pf/}}(\mathcal{D}_{pf/})_{/q\overline{f}}} & {K\star B} & {\mathcal{C}\times _{\mathcal{D}}\mathcal{D}_{/pC}.}
	\arrow[from=1-1, to=1-2]
	\arrow[from=1-2, to=2-2]
	\arrow[from=1-1, to=2-1]
	\arrow[from=2-1, to=2-2]
	\arrow[dashed, from=2-1, to=1-2]
	\arrow[from=1-3, to=2-3]
	\arrow[from=1-3, to=1-4]
	\arrow[from=1-4, to=2-4]
	\arrow[from=2-3, to=2-4]
	\arrow[dashed, from=2-3, to=1-4]
\end{tikzcd}\]The right hand lifting problem is solvable because $C$ is $p$-terminal.
\end{proof}
\begin{cor}
\label{cor:Step2}Let $q:\cal C^{\t}\to\cal O^{\t}$ be a fibration
of $\infty$-operads, let $K$ be a simplicial set, and let $n\geq0$.
Consider a lifting problem % https://q.uiver.app/?q=WzAsNCxbMCwwLCJLXFxzdGFyIFxccGFydGlhbFxcRGVsdGEgXm4iXSxbMCwxLCJLXFxzdGFyIFxcRGVsdGEgXm4iXSxbMSwwLCJcXG1hdGhjYWx7Q31eXFxvdGltZXMgIl0sWzEsMSwiXFxtYXRoY2Fse099Xlxcb3RpbWVzICJdLFswLDEsIiIsMCx7InN0eWxlIjp7InRhaWwiOnsibmFtZSI6Imhvb2siLCJzaWRlIjoidG9wIn19fV0sWzIsMywicSJdLFswLDIsImYiXSxbMSwzLCJnIiwyXSxbMSwyLCIiLDEseyJzdHlsZSI6eyJib2R5Ijp7Im5hbWUiOiJkYXNoZWQifX19XV0=
\[\begin{tikzcd}
	{K\star \partial\Delta ^n} & {\mathcal{C}^\otimes } \\
	{K\star \Delta ^n} & {\mathcal{O}^\otimes.}
	\arrow[hook, from=1-1, to=2-1]
	\arrow["q", from=1-2, to=2-2]
	\arrow["f", from=1-1, to=1-2]
	\arrow["g"', from=2-1, to=2-2]
	\arrow[dashed, from=2-1, to=1-2]
\end{tikzcd}\]If the map $g$ maps the terminal vertex of $\Delta^{n}$ to an object
in $\cal O_{\inp 0}^{\t}$, then the lifting problem admits a solution.
\end{cor}
\begin{proof}
If $n=0$, find an object $C\in\cal C_{\inp 0}^{\t}$ which lies over
$g\pr 0$. Such an object exists because the functor $\cal C_{\inp 0}^{\t}\to\cal O_{\inp 0}^{\t}$
is a trivial fibration. The object $C$ is $q$-terminal, so the functor
\[
\cal C_{/C}^{\t}\to\cal O_{/q\pr C}^{\t}\times_{\cal O^{\t}}\cal C^{\t}
\]
is a trivial fibration. This implies the existence of the filler.

If $n>1$, then set $h=f\vert K$. We must solve a lifting problem
% https://q.uiver.app/?q=WzAsNCxbMCwwLCJcXHBhcnRpYWxcXERlbHRhIF5uIl0sWzAsMSwiXFxEZWx0YSBebiJdLFsxLDAsIlxcbWF0aGNhbHtDfV5cXG90aW1lcyBfe2gvfSJdLFsxLDEsIlxcbWF0aGNhbHtPfV5cXG90aW1lc197cWgvfSJdLFswLDEsIiIsMCx7InN0eWxlIjp7InRhaWwiOnsibmFtZSI6Imhvb2siLCJzaWRlIjoidG9wIn19fV0sWzIsMywicSciXSxbMCwyXSxbMSwzXSxbMSwyLCIiLDEseyJzdHlsZSI6eyJib2R5Ijp7Im5hbWUiOiJkYXNoZWQifX19XV0=
\[\begin{tikzcd}
	{\partial\Delta ^n} & {\mathcal{C}^\otimes _{h/}} \\
	{\Delta ^n} & {\mathcal{O}^\otimes_{qh/}.}
	\arrow[hook, from=1-1, to=2-1]
	\arrow["{q'}", from=1-2, to=2-2]
	\arrow[from=1-1, to=1-2]
	\arrow[from=2-1, to=2-2]
	\arrow[dashed, from=2-1, to=1-2]
\end{tikzcd}\]For this, it will suffice to show that the image of the vertex $n\in\partial\Delta^{n}$
under the top horizontal arrow is $q'$-terminal. This follows from
Proposition \ref{prop:relative_terminal_obj_in_slice}.
\end{proof}
\begin{lem}
\cite[Lemma 3.1.2.5]{HA}\label{lem:3.1.2.5}Let $\cal C$ be an $\infty$-category
and $\cal C^{0}\subset\cal C$ a full subcategory. Let $\sigma:\Delta^{n}\to\cal C$
be a nondegenerate simplex such that $\sigma\pr i\not\in\cal C^{0}$
for each $0\leq i\leq n$. Consider the following simplicial sets:
\begin{enumerate}
\item The simplicial subset $K\subset\cal C$ consisting of those simplices
$\tau:\Delta^{k}\star\Delta^{l}\to\cal C$, where $k,l\geq-1$, $\tau\vert\Delta^{k}$
factors through $\cal C^{0}$ and $\tau\vert\Delta^{l}$ factors through
$\sigma$.
\item The simplicial subset $K_{0}\subset\cal C$ consisting of those simplices
$\tau:\Delta^{k}\star\Delta^{l}\to\cal C$, where $k,l\geq-1$, $\tau\vert\Delta^{k}$
factors through $\cal C^{0}$ and $\tau\vert\Delta^{l}$ factors through
$\sigma\vert\partial\Delta^{n}$.
\item The simplicial subset $L\subset\cal C$ generated by the simplices
$\widetilde{\sigma}:\Delta^{k}\star\Delta^{n}\to\cal C$, where $k\geq-1$,
$\widetilde{\sigma}\vert\Delta^{k}$ factors through $\cal C^{0}$,
and $\widetilde{\sigma}\vert\Delta^{n}=\sigma$. 
\end{enumerate}
Then:
\begin{itemize}
\item [(a)]The map $\cal C_{/\sigma}^{0}\star\Delta^{n}\amalg_{\cal C_{/\sigma}^{0}\star\partial\Delta^{n}}K_{0}\to K_{0}\cup L$
is an isomorphism of simplicial sets.
\item [(b)]The inclusion
\[
K_{0}\cup L\subset K
\]
is a trivial cofibration in the Joyal model structure.
\end{itemize}
\end{lem}
\begin{proof}
We begin with (a). It is clear that the simplicial set $K_{0}\cup L$
is the union of $K_{0}$ and the image of the map $\cal C_{/\sigma}^{0}\star\Delta^{n}\to\cal C$.
Therefore, it will suffice to show that if a simplex $x$ of $\cal C_{/\sigma}^{0}\star\Delta^{n}$
is mapped to a simplex in $K_{0}$, then $x$ belongs to $\cal C_{/\sigma}^{0}\star\partial\Delta^{n}$.
The image of $x$ can be represented by a simplex of the form
\[
\Delta^{k}\star\Delta^{l}\xrightarrow{\id\star u}\Delta^{k}\star\Delta^{n}\xrightarrow{\widetilde{\sigma}}\cal C,
\]
where $k,l\geq-1$, $u:[l]\to[n]$ is a poset map, and $\widetilde{\sigma}$
is an extension of $\sigma$ which maps the vertices of $\Delta^{k}$
into $\cal C^{0}$. We wish to show that if $\sigma u:\Delta^{l}\to\cal C$
factors through $\sigma\vert\partial\Delta^{n}$, then $u$ factors
through $\partial\Delta^{n}$. This is clear, since $\sigma$ is nondegenerate.

Next we prove (b). For each $m\geq n$, let $X_{m}$ denote the simplicial
subset of $K$ consisting of the simplices in $K_{0}$ and the simplices
of the form
\[
\tau:\Delta^{k}\star\Delta^{m'}\to\cal C,
\]
where $k\geq-1$, $n\leq m'<m$, $\tau\vert\Delta^{k}$ factors through
$\cal C^{0}$, and $\tau\vert\Delta^{m'}$ is a degeneration of $\sigma$.
Then we have a filtration 
\[
K_{0}\cup L=X_{n}\subset X_{n+1}\subset\cdots
\]
and $K=\bigcup_{i\geq n}X_{i}$. To prove (b), it will suffice to
show that each inclusion $X_{i}\subset X_{i+1}$ is a trivial cofibration.

Let $\varepsilon:[m]\to[n]$ be a surjective poset map. We let $\cal J\pr{\varepsilon}\subset\Del_{[m]//[n]}$
denote the full subcategory spanned by the factorizations $\pr{\varepsilon',\varepsilon''}=[m]\xrightarrow{\varepsilon'}[m']\xrightarrow{\varepsilon''}[n]$
of $\varepsilon$ into two surjective poset maps, and let $\cal J_{0}\pr{\varepsilon}$
denote its full subcategory spanned by the object other than $\pr{\id,\varepsilon}$.
Let $A\pr{\varepsilon}\subset\cal C_{/\sigma\varepsilon}^{0}$ denote
the union of the image of the maps $\cal C_{/\sigma\varepsilon''}^{0}\to\cal C_{/\sigma\varepsilon''\varepsilon'}^{0}=\cal C_{/\sigma\varepsilon}^{0}$,
where $\pr{\varepsilon',\varepsilon''}$ ranges over the objects of
$\cal J_{0}\pr{\varepsilon}$. We consider the commutative diagram
% https://q.uiver.app/?q=WzAsNCxbMCwwLCJcXGNvcHJvZF97XFx2YXJlcHNpbG9uOlttXVxcdHdvaGVhZHJpZ2h0YXJyb3dbbl19KEEoXFx2YXJlcHNpbG9uKVxcc3RhciBcXERlbHRhIF5tKVxcY3VwKFxcbWF0aGNhbHtDfV4wX3svXFxzaWdtYVxcdmFyZXBzaWxvbn1cXHN0YXJcXHBhcnRpYWxcXERlbHRhXnttfSkiXSxbMCwxLCJYX20iXSxbMSwwLCJcXGNvcHJvZF97XFx2YXJlcHNpbG9uOlttXVxcdHdvaGVhZHJpZ2h0YXJyb3dbbl19XFxtYXRoY2Fse0N9XjBfey9cXHNpZ21hXFx2YXJlcHNpbG9ufVxcc3RhclxcRGVsdGFee219Il0sWzEsMSwiWF97bSsxfSJdLFswLDFdLFswLDJdLFsyLDNdLFsxLDNdXQ==
\[\begin{tikzcd}
	{\coprod_{\varepsilon:[m]\twoheadrightarrow[n]}(A(\varepsilon)\star \Delta ^m)\cup(\mathcal{C}^0_{/\sigma\varepsilon}\star\partial\Delta^{m})} & {\coprod_{\varepsilon:[m]\twoheadrightarrow[n]}\mathcal{C}^0_{/\sigma\varepsilon}\star\Delta^{m}} \\
	{X_m} & {X_{m+1}}
	\arrow[from=1-1, to=2-1]
	\arrow[from=1-1, to=1-2]
	\arrow[from=1-2, to=2-2]
	\arrow[from=2-1, to=2-2]
\end{tikzcd}\]where the coproducts are indexed by the surjective poset maps $[m]\epi[n]$.
To complete the proof, it suffices to prove the following:
\begin{itemize}
\item [(i)]The square is a pushout.
\item [(ii)]For each surjection $\varepsilon:[m]\to[n]$, the inclusion
\[
A\pr{\varepsilon}\star\Delta^{m}\cup\cal C_{/\sigma\varepsilon}^{0}\star\partial\Delta^{m}\to\cal C_{/\sigma\varepsilon}^{0}\star\Delta^{m}
\]
is a trivial cofibration.
\end{itemize}
We start with (i). Let $x$ be a simplex in $\cal C_{/\sigma\varepsilon}^{0}\star\Delta^{m}$
corresponding to a pair $\pr{\widetilde{\sigma\varepsilon}:\Delta^{k}\star\Delta^{m}\to\cal C,\,u:\Delta^{l}\to\Delta^{m}}$,
$k,l\geq-1$, $k+l+1\geq0$. Its image in $\cal C$ is the composition
\[
\alpha:\Delta^{k}\star\Delta^{l}\xrightarrow{\id\star u}\Delta^{k}\star\Delta^{m}\xrightarrow{\widetilde{\sigma\varepsilon}}\cal C.
\]
We wish to show that if $\alpha$ belongs to $X_{m}$, then $\pr{\widetilde{\sigma\varepsilon},u}$
belongs to $A\pr{\varepsilon}\star\Delta^{m}\cup\pr{\cal C_{/\sigma\varepsilon}^{0}\star\partial\Delta^{m}}$.
The claim is obvious if $u$ is not surjective on vertices. So assume
that $u$ is surjective on vertices. Since $\alpha\vert\Delta^{l}$
is a degeneration of $\sigma$, it cannot factor through $\sigma\vert\partial\Delta^{n}$.
Thus $\alpha$ does not belong to $K_{0}$. Hence (since the vertices
of $\sigma$ do not belong to $\cal C^{0}$) we can find a factorization
of $\alpha$ of the form 
\[
\alpha:\Delta^{k}\star\Delta^{l}\xrightarrow{\id\star u'}\Delta^{k}\star\Delta^{m'}\xrightarrow{\widetilde{\sigma\varepsilon'}}\cal C,
\]
where $m'<m$, $\varepsilon':[m']\to[n]$ is a surjection, and $\widetilde{\sigma\varepsilon'}$
is an extension of $\sigma\varepsilon'$ which maps $\Delta^{k}$
into $\cal C^{0}$. Replacing $\Delta^{m'}$ by $\Delta^{u'[l]}$
if necessary, we may assume that $u'$ is surjective on vertices.
Factor the map $\widetilde{\sigma\varepsilon'}$ as 
\[
\Delta^{k}\star\Delta^{m'}\xrightarrow{s\star s'}\Delta^{p}\star\Delta^{m''}\xrightarrow{\tau}\cal C,
\]
where $p,m''\geq-1$, $s,s'$ are surjective on vertices and $\tau$
is nondegenerate. (Such a factorization exists because the vertices
of $\sigma$ do not belong to $\cal C^{0}$.) Then $\sigma\varepsilon'$
is a degeneration of $\tau\vert\Delta^{q}$, so Eilenberg-Zilber's
lemma implies that $\tau\vert\Delta^{m''}$ is a degeneration of $\sigma$.
Therefore, we can write $\tau\vert\Delta^{m''}=\sigma\varepsilon''$,
where $\varepsilon'':[m'']\to[n]$ is a surjective poset map. By Eilenberg-Zilber's
lemma, $\widetilde{\sigma\varepsilon'}$ is a degeneartion of $\tau$.
Since $m''<m$, this implies that the simplex $\pr{\widetilde{\sigma\varepsilon},u}$
belongs to $A\pr{\varepsilon}\star\Delta^{m}$, as required.

We next prove (ii). Since join functors preserves trivial cofibrations
in each variable, it will suffice to show that the inclusion $A\pr{\varepsilon}\subset\cal C_{/\sigma\varepsilon}^{0}$
is a trivial cofibration. Arguing as in the previous paragraph, we
see that $A\pr{\varepsilon}$ is the colimit of the diagram $F:\cal J_{0}\pr{\varepsilon}^{\op}\to\SS$
defined by $\pr{\varepsilon',\varepsilon''}\mapsto\cal C_{/\sigma\varepsilon'}^{0}$.
This diagram admits a natural transformation to the constant diagram
at $\cal C_{/\sigma\varepsilon}^{0}$ whose components are trivial
fibrations. It will therefore suffice to prove that the diagrams $F$
and the constant diagram at $\cal C_{/\sigma\varepsilon}^{0}$ are
projectively cofibrant. There is a functor $\cal J_{0}\pr{\varepsilon}^{\op}\to\omega$
which maps a factorization $[m]\to[m']\to[n]$ of $\varepsilon$ to
the integer $m'$. This functor endows the category $\cal J_{0}\pr{\varepsilon}^{\op}$
with the structure of a direct category. The claim is thus a consequence
of \cite[Theorem 5.1.3]{Hovey2007}.
\end{proof}
\begin{defn}
Let $X$ be a simplicial set over $\Delta^{1}$. Given a simplex $\Delta^{k}\to X$,
we define the \textbf{head} of $X$ to be the simplex $\Delta^{u^{-1}\pr 1}\to\Delta^{k}\to X$
and the \textbf{tail} to be the simplex $\Delta^{u^{-1}\pr 0}\to\Delta^{k}\to X$,
where $u$ denotes the composite $\Delta^{k}\to X\to\Delta^{1}$.
\end{defn}
\begin{prop}
\label{prop:heads_and_tails}Let % https://q.uiver.app/?q=WzAsMyxbMCwwLCJYIl0sWzIsMCwiWSJdLFsxLDEsIlxcRGVsdGFeMSJdLFswLDEsInAiXSxbMSwyXSxbMCwyXV0=
\[\begin{tikzcd}
	X && Y \\
	& {\Delta^1}
	\arrow["p", from=1-1, to=1-3]
	\arrow[from=1-3, to=2-2]
	\arrow[from=1-1, to=2-2]
\end{tikzcd}\]be a commutative diagram of simplicial sets. Let $n\geq0$, and let
$S\subset S'\subset Y_{1}=Y\times_{\Delta^{1}}\{1\}$ be simplicial
subsets satisfying the following conditions:
\begin{enumerate}
\item The simplicial set $S$ contains the $\pr{n-1}$-skeleton of $Y_{1}$.
\item The simplicial set $S'$ is generated by $S$ and a set $\Sigma$
of nondegenerate $n$-simplices of $Y_{1}$ which do not belong to
$S'$.
\end{enumerate}
Let $X\pr S$ denote the simplicial subset of $X$ spanned by the
simplices whose head lies over $S$, and define $X\pr{S'}$ similarly.
Let $\{\sigma_{a}\}_{a\in A}$ be the enumeration of all nondegenerate
simplices of $X_{1}$ whose image in $Y$ is a degeneration of a simplex
in $\Sigma$. Choose an ordering of $A$ so that that dimension of
$\sigma_{a}$ is a non-decreasing function of $a\in A$. For each
$a\in A$, define simplicial sets $X\pr{S'}_{<a}$, $X\pr{S'}_{\leq a}$,
$K_{a}$, and $K_{0,a}$ as follows:
\begin{itemize}
\item $X\pr{S'}_{<a}\subset X$ is the simplicial subset generated by $X\pr S$
and those simplices of $X$ whose head factors through $\sigma_{b}$
for some $b<a$.
\item $X\pr{S'}_{\leq a}\subset X$ is the simplicial subset generated by
$X\pr S$ and those simplices of $X$ whose head factors through $\sigma_{b}$
for some $b\leq a$.
\item $K_{a}\subset X$ is the simplicial subset consisting of the simplices
whose head factors through $\sigma_{a}$.
\item $K_{0,a}\subset X$ is the simplicial subset consisting of the simplices
whose head factors through $\partial\sigma_{a}=\sigma_{a}\vert\partial\Delta^{\dim\sigma_{a}}$.
\end{itemize}
Then the following holds:
\begin{enumerate}
\item $X\pr{S'}=X\pr S\cup\bigcup_{a\in A}X\pr{S'}_{\leq a}$.
\item For each $a\in A$, the square % https://q.uiver.app/?q=WzAsNCxbMCwwLCJLX3swLGF9Il0sWzAsMSwiS19hIl0sWzEsMCwiWChTJylfezxhfSJdLFsxLDEsIlgoUycpX3tcXGxlcSBhfSJdLFswLDFdLFswLDJdLFsyLDNdLFsxLDNdXQ==
\begin{equation}\label{d:h_and_t}
\begin{tikzcd}
	{K_{0,a}} & {X(S')_{<a}} \\
	{K_a} & {X(S')_{\leq a}}
	\arrow[from=1-1, to=2-1]
	\arrow[from=1-1, to=1-2]
	\arrow[from=1-2, to=2-2]
	\arrow[from=2-1, to=2-2]
\end{tikzcd}
\end{equation}of simplicial sets is cocartesian.
\end{enumerate}
\end{prop}
\begin{proof}
We start with (1). The containment $X\pr S\cup\bigcup_{a\in A}X\pr{S'}_{\leq a}\subset X\pr{S'}$
holds trivially. For the reverse inclusion, let $x$ be an arbitrary
simplex of $X\pr{S'}$. We must show that $x$ belongs to $X\pr S\cup\bigcup_{a\in A}X\pr{S'}_{\leq a}$.
Let $\tau\pr x:\Delta^{m}\to X_{1}$ denote the head of $x$. If $p\tau\pr x$
belongs to $S$, then $x$ belongs to $X\pr S$ and we are done. If
$\tau\pr x$ does not belong to $S$, then $p\tau\pr x$ factors through
a simplex $\sigma$ in $\Sigma$. If $p\tau\pr x$ factors through
the boundary of $\sigma$, then $p\tau\pr x$ belongs to the $\pr{n-1}$-skeleton
of $Y_{1}$ and hence $\tau\pr x$ belongs to $S$, a contradiction.
Therefore, $p\tau\pr x$ is a degeneration of $\sigma$. Now $\tau\pr x$
is a degeneration of some nondegenerate simplex $\tau'\pr x$ of $X_{1}$.
Then $p\tau\pr x$ is a degeneration of $p\tau'\pr x$, so Eilenberg-Zilber's
lemma implies that $p\tau'\pr x$ is a degeneration of $\sigma$.
Hence $\tau'\pr x=\sigma_{a}$ for some $a\in A$, and hence $x\in X\pr{S'}_{\leq a}$.

We next prove (2). We will write $K_{0}=K_{0,a}$ and $K=K_{a}$.
First we remark that $X\pr{S'}_{<a}$ contains $K_{0}$. Indeed, suppose
we are given a simplex $x$ of $X$ whose head $\tau\pr x:\Delta^{m}\to X_{1}$
facotors through $\partial\sigma_{a}$. By construction, there is
a commutative diagram % https://q.uiver.app/?q=WzAsNixbMSwwLCJcXERlbHRhXntcXGRpbVxcc2lnbWFfYX0iXSxbMiwwLCJcXERlbHRhXm4iXSxbMiwxLCJZXzEsIl0sWzEsMSwiWF8xIl0sWzAsMCwiXFxwYXJ0aWFsXFxEZWx0YV57XFxkaW1cXHNpZ21hX2F9Il0sWzAsMSwiXFxEZWx0YV5tIl0sWzAsMSwicyJdLFsxLDIsIlxcc2lnbWEiXSxbMywyLCJwIiwyXSxbMCwzLCJcXHNpZ21hX2EiLDJdLFs0LDAsIiIsMCx7InN0eWxlIjp7InRhaWwiOnsibmFtZSI6Imhvb2siLCJzaWRlIjoidG9wIn19fV0sWzUsMywiXFx0YXUoeCkiLDJdLFs1LDRdXQ==
\[\begin{tikzcd}
	{\partial\Delta^{\dim\sigma_a}} & {\Delta^{\dim\sigma_a}} & {\Delta^n} \\
	{\Delta^m} & {X_1} & {Y_1,}
	\arrow["s", from=1-2, to=1-3]
	\arrow["\sigma", from=1-3, to=2-3]
	\arrow["p"', from=2-2, to=2-3]
	\arrow["{\sigma_a}"', from=1-2, to=2-2]
	\arrow[hook, from=1-1, to=1-2]
	\arrow["{\tau(x)}"', from=2-1, to=2-2]
	\arrow[from=2-1, to=1-1]
\end{tikzcd}\]where $s$ is surjective on vertices. If the map $\Delta^{m}\to\Delta^{n}$
is not surjective on vertices, then $p\tau\pr x$ factors through
the $\pr{n-1}$-skeleton of $Y_{1}$ and hence $x$ belongs to $X\pr S$.
If the map $\Delta^{m}\to\Delta^{n}$ is surjective on vertices, then
write $\tau\pr x$ as a degeneration of a nondegenerate simplex $\tau'\pr x$
of $X_{1}$. Eilenberg-Zilber's lemma implies that the simplex $p\tau'\pr x$
must be a degeneration of $\sigma$. Thus $p\tau'\pr x=\sigma_{b}$
for some $b\in A$. Since $\tau\pr x$ factors through the boundary
of $\sigma_{a}$, the dimension of $\sigma_{b}$ is strictly smaller
than that of $\sigma_{a}$. Thus $b<a$. Hence $x$ belongs to $X\pr{S'}_{<a}$.

By what we have just shown in the previous paragraph, the diagram
(\ref{d:h_and_t}) is well-defined. We now show that it is cocartesian.
By definition, $X\pr{S'}_{\leq a}$ is the union of $K$ and $X\pr{S'}_{<a}$.
Therefore, it suffices to show that $K_{0}$ is the intersection of
$K$ and $X\pr{S'}_{<a}$. So let $x$ be a simplex of $K$. We must
show that, if $x$ does not belong to $K_{0}$, then $x$ does not
belong to $X\pr{S'}_{<a}$ either. Let $\tau\pr x:\Delta^{m}\to X_{1}$
be the head of $x$. Since $x$ does not belong to $K_{0}$, $\tau\pr x$
is a degeneration of $\sigma_{a}$. Thus $p\pr{\tau\pr x}$ is a degeneration
of some simplex $\sigma\in\Sigma$. In particular, $p\pr{\tau\pr x}$
does not belong to $S$, so $\tau\pr x$ does not belong to $X\pr S$.
Therefore, should $\tau\pr x$ belong to $X\pr{S'}_{<a}$, then $\tau\pr x$
must factor through some $\sigma_{b}$ for some $b<a$. If $\tau\pr x$
factors through $\partial\sigma_{b}$ for some $b<a$, then we would
have $\dim\sigma_{a}<\dim\sigma_{b}$, a contradiction. So $\tau\pr x$
is a degeneartion of $\sigma_{b}$; but then $a=b$, a contradiction.
Thus $x$ does not belong to $X\pr{S'}_{<a}$, as required.
\end{proof}

\subsection{Proof of Theorem \ref{thm:3.1.2.3}}

In this final subsection of this note, we will give a proof of Theorem
\ref{thm:3.1.2.3} by elaborating the proof given by Lurie in \cite{HA}.

The proof proceeds by a simplex-by-simplex argument. For this, we
will classify simplices of $N\pr{\Fin_{\ast}}\times\Delta^{\{1,\dots,n\}}$
into five (somewhat artificial) groups.
\begin{defn}
Let $n\geq1$ and let $\alpha$ be a morphism in $N\pr{\Fin_{\ast}}\times\Delta^{\{1,\dots,n\}}$
with image $\alpha_{0}:\inp m\to\inp n$ in $N\pr{\Fin_{\ast}}$.
We say that $\alpha$ is:
\begin{enumerate}
\item \textbf{active} if $\alpha_{0}$ is active;
\item \textbf{strongly inert} if $\alpha_{0}$ is inert, the induced injection
$\inp n^{\circ}\to\inp m^{\circ}$ is order-perserving, and the image
of $\alpha$ in $\Delta^{\{1,\dots,n\}}$ is degenerate; and
\item \textbf{neutral} if it is neigher active nor strongly inert.
\end{enumerate}
Note that active morphisms and strongly inert morphisms are closed
under composition. Also, every morphism in $N\pr{\Fin_{\ast}}\times\Delta^{\{1,\dots,n\}}$
can be factored uniquely as a composition of a strongly inert map
followed by an active map.

Let $\sigma$ be an $m$-simplex of $N\pr{\Fin_{\ast}}\times\Delta^{\{1,\dots,n\}}$
depicted as
\[
\pr{\inp{k_{0}},e_{0}}\xrightarrow{\alpha_{\sigma}\pr 1}\cdots\xrightarrow{\alpha_{\sigma}\pr m}\pr{\inp{k_{m}},e_{m}}.
\]
We will say that $\sigma$ is \textbf{closed} if $k_{m}=1$, and \textbf{open}
otherwise. We say that $\sigma$ is \textbf{complete}\footnote{Lurie uses te term ``new'' instead of ``complete.''}
if $\{e_{0},\dots,e_{m}\}=\{1,\dots,n\}$ and \textbf{incomplete}
otherwise. Every nondegenerate simplex of $N\pr{\Fin_{\ast}}\times\Delta^{\{1,\dots,n\}}$
is a face of a nondegenerate complete simplex. 

We partition the set of nondegenerate complete simplices of $N\pr{\Fin_{\ast}}\times\Delta^{\{1,\dots n\}}$
into five groups $G_{\pr 1},G_{\pr 2},G'_{\pr 2},G_{\pr 3},G_{\pr 3}'$
as follows: $m$-simplex of $N\pr{\Fin_{\ast}}\times\Delta^{\{1,\dots,n\}}$.
Write $\alpha_{\sigma}\pr i=\sigma\vert\Delta^{\{i,i+1\}}$. Let $0\le k\leq m$
be the minimal integer such that $\alpha_{\sigma}\pr i$ is strongly
inert for every $i>k$, and let $0\leq j\leq k$ be the minimal integer
such that $\alpha_{\sigma}\pr i$ is active for every $j<i\leq k$.
\begin{itemize}
\item If $j=0$, $k=m$, and $\sigma$ is closed, then $\sigma$ belongs
to $G_{\pr 1}$.
\item If $j=0$, $k<m$, and $\sigma$ is closed, then $\sigma$ belongs
to $G_{\pr 2}$.
\item If $j=0$ and $\sigma$ is open, then $\sigma$ belongs to $G'_{\pr 2}$.
\item If $j\geq1$ and $\alpha_{\sigma}\pr j$ is strongly inert, then $\sigma$
belongs to $G_{\pr 3}$.
\item If $j\geq1$ and $\alpha_{\sigma}\pr j$ is neutral, then $\sigma$
belongs to $G_{\pr 3}'$.
\end{itemize}
Given an $m$-simplex $\sigma\in G_{\pr 2}$, we define its \textbf{associate}
$\sigma'\in G_{\pr 2}'$ by $\sigma'=\sigma\vert\Delta^{\{0,\dots,m-1\}}$.
Given an $m$-simplex $\sigma\in G_{\pr 3}$, then we define its \textbf{associate}
$\sigma'$ by $\sigma'=\sigma\partial_{j}\in G_{\pr 3}'$, where $j$
is the integer defined as above.
\end{defn}
%
\begin{proof}
[Proof of Theorems \ref{thm:3.1.2.3}]We will regard $\cal M^{\t}$
and $N\pr{\Fin_{\ast}}\times\Delta^{n}$ as a simplicial set over
$\Delta^{1}$ by means of the map $\Delta^{n}\to\Delta^{1}$ which
maps the vertex $0\in\Delta^{n}$ to the vertex $0\in\Delta^{1}$
and the remaining vertices to the vertex $1\in\Delta^{1}$. Given
a simplicial subset $S\subset N\pr{\Fin_{\ast}}\times\Delta^{\{1,\dots,n\}}$,
we let $\cal M_{S}^{\t}\subset\cal M^{\t}$ denote the simplicial
subset consisting of the simplices whose tail lies over $S$. We will
also write $\overline{S}=\pr{N\pr{\Fin_{\ast}}\times\Delta^{\{1,\dots,n\}}}_{S}$.

The implication (a)$\implies$(b) for part (A) is obvious. Assume
therefore that condition (b) is satisfied if $n=1$. For each $m\geq0$,
let $F\pr m$ denote the simplicial subset of $N\pr{\Fin_{\ast}}\times\Delta^{\{1,\dots,n\}}$
generated by the nondegenerate simplices $\sigma$ satisfying one
of the following conditions:
\begin{itemize}
\item $\sigma$ is incomplete.
\item $\sigma$ has dimension less than $m$.
\item $\sigma$ has dimension $m$ and belongs to $G_{\pr 2}$ or $G_{\pr 3}$.
\end{itemize}
Observe that $\cal M_{F\pr 0}^{\t}=\cal M^{\t}\times_{\Delta^{n}}\Lambda_{0}^{n}$.
We will complete the proof by inductively constructing a map $f_{m}:\cal M_{F\pr m}^{\t}\to\cal C^{\t}$
which makes the diagram % https://q.uiver.app/?q=WzAsNCxbMCwwLCJcXG1hdGhjYWx7TX1eXFxvdGltZXMgX3tGKG0tMSl9Il0sWzAsMSwiXFxtYXRoY2Fse019Xlxcb3RpbWVzIF97RihtKX0iXSxbMSwxLCJcXG1hdGhjYWx7T31eXFxvdGltZXMgIl0sWzEsMCwiXFxtYXRoY2Fse0N9Xlxcb3RpbWVzIl0sWzAsMSwiIiwwLHsic3R5bGUiOnsidGFpbCI6eyJuYW1lIjoiaG9vayIsInNpZGUiOiJ0b3AifX19XSxbMSwyLCJnXFx2ZXJ0XFxtYXRoY2Fse019Xlxcb3RpbWVzIF97RihtKX0iLDJdLFszLDIsInEiXSxbMCwzLCJmX3ttLTF9Il0sWzEsMywiZl97bX0iLDFdXQ==
\[\begin{tikzcd}
	{\mathcal{M}^\otimes _{F(m-1)}} & {\mathcal{C}^\otimes} \\
	{\mathcal{M}^\otimes _{F(m)}} & {\mathcal{O}^\otimes }
	\arrow[hook, from=1-1, to=2-1]
	\arrow["{g\vert\mathcal{M}^\otimes _{F(m)}}"', from=2-1, to=2-2]
	\arrow["q", from=1-2, to=2-2]
	\arrow["{f_{m-1}}", from=1-1, to=1-2]
	\arrow["{f_{m}}"{description}, from=2-1, to=1-2]
\end{tikzcd}\]commutative, and such that $f_{1}$ has the following special properties
if $n=1$:
\begin{itemize}
\item [(i)]For each object $B\in\cal M^{\t}\times_{\Delta^{1}}\{1\}$,
the map 
\[
\pr{\pr{\cal M_{\act}^{\t}}_{/B}\times_{\Delta^{1}}\{0\}}^{\rcone}\to\cal M_{\pr 1}^{\t}\xrightarrow{f_{1}}\cal C^{\t}
\]
is an operadic $q$-colimit diagram.
\item [(ii)]For every inert morphism $e:M'\to M$ in $\cal M^{\t}\times_{\Delta^{1}}\{1\}$
such that $M\in\cal M$, the functor $f_{1}$ carries $e$ to an inert
morphism in $\cal C^{\t}$.
\end{itemize}
Fix $m>0$, and suppose that $f_{m-1}$ has been constructed. Observe
that $F\pr m$ is obtained from $F\pr{m-1}$ by adjoining the following
simplices:
\begin{itemize}
\item The $\pr{m-1}$-simplices in $G_{\pr 1}$.
\item The $\pr{m-1}$-simplices in $G'_{\pr 2}$ without associates.
\item The $m$-simplices in $G_{\pr 2}$ and $G_{\pr 3}$.
\end{itemize}
We define simplicial subsets $F'\pr m\subset F''\pr m\subset F\pr m$
as follows: $F'\pr m$ is generated by $F\pr{m-1}$ and the $\pr{m-1}$-simplices
in $G_{\pr 1}$; $F'\pr{m-1}$ is generated by $F'\pr m$ and the
$\pr{m-1}$-simplices of $G'_{\pr 2}$ without associates. Our strategy
is to extend $f_{m-1}$ to $\cal M_{F'\pr m}^{\t}$, then to $\cal M_{F''\pr m}^{\t}$,
and then to $\cal M_{F\pr m}^{\t}$. 

\begin{enumerate}[label=(\textbf{Step \arabic*}),wide =0.5\parindent, listparindent=1.5em]

\item We will extend $f_{m-1}$ to a map $f'_{m}:\cal M_{F'\pr m}^{\t}\to\cal C^{\t}$
over $\cal O^{\t}$. 

Let $\{\sigma_{a}\}_{a\in A}$ be the collection of nondegenerate
simplices of $\cal M^{\t}\times_{\Delta^{n}}\Delta^{\{1,\dots,n\}}$
whose image in $N\pr{\Fin_{\ast}}\times\Delta^{\{1,\dots,n\}}$ is
a degeneration of some $\pr{m-1}$-simplex of $G_{\pr 1}$. Choose
a well-ordering on $A$ so that $\dim\sigma_{a}$ is non-decreasing
in $a\in A$. For each $a\in A$, let $\cal M_{<a}^{\t}$ denote the
simplicial subset spanned by $\cal M_{F\pr{m-1}}^{\t}$ and the simplices
of $\cal M^{\t}$ whose tail factors through $\sigma_{b}$ for some
$b<a$. We define $\cal M_{\leq a}^{\t}$ similarly. According to
Proposition \ref{prop:heads_and_tails}, we have $\cal M_{F'\pr m}^{\t}=\bigcup_{a\in A}\cal M_{\leq a}^{\t}$,
so it suffices to extend $f_{m-1}$ to an $A$-sequence $f^{\leq a}:\cal M_{\leq a}^{\t}\to\cal C^{\t}$
over $\cal O^{\t}$. 

The construction is inductive. Let $a\in A$, and suppose that $f^{\leq b}$
has been constructed for $b<a$. These maps determine a map $f^{<a}:\cal M_{<a}^{\t}\to\cal C^{\t}$
extending $f_{m-1}$. Let $K_{0}\subset K\subset\cal M^{\t}$ denote
the simplicial subset consisting of the simplices of $\cal M^{\t}$
whose head factors through $\sigma_{a}\vert\partial\Delta^{\dim\sigma_{a}}$
and $\sigma_{a}$, respectively. According to Lemma \ref{lem:3.1.2.5},
the left hand square of the commutative diagram% https://q.uiver.app/?q=WzAsNixbMCwwLCIoXFxtYXRoY2Fse019Xlxcb3RpbWVzIF97L1xcc2lnbWFfYX1cXHRpbWVzX3tcXERlbHRhXm59IFxcezBcXH0pXFxzdGFyIFxccGFydGlhbFxcRGVsdGEgXntcXGRpbVxcc2lnbWEgX2F9Il0sWzAsMSwiKFxcbWF0aGNhbHtNfV5cXG90aW1lcyBfey9cXHNpZ21hX2F9XFx0aW1lc197XFxEZWx0YV5ufSBcXHswXFx9KVxcc3RhciBcXERlbHRhIF57XFxkaW1cXHNpZ21hIF9hfSJdLFsxLDAsIktfMCJdLFsxLDEsIksiXSxbMiwwLCJcXG1hdGhjYWx7TX1eXFxvdGltZXMgX3s8YX0iXSxbMiwxLCJcXG1hdGhjYWx7TX1eXFxvdGltZXMgX3tcXGxlcSBhfSJdLFswLDJdLFsxLDNdLFsyLDNdLFsyLDRdLFs0LDVdLFszLDVdLFswLDFdXQ==
\[\begin{tikzcd}
	{(\mathcal{M}^\otimes _{/\sigma_a}\times_{\Delta^n} \{0\})\star \partial\Delta ^{\dim\sigma _a}} & {K_0} & {\mathcal{M}^\otimes _{<a}} \\
	{(\mathcal{M}^\otimes _{/\sigma_a}\times_{\Delta^n} \{0\})\star \Delta ^{\dim\sigma _a}} & K & {\mathcal{M}^\otimes _{\leq a}}
	\arrow[from=1-1, to=1-2]
	\arrow[from=2-1, to=2-2]
	\arrow[from=1-2, to=2-2]
	\arrow[from=1-2, to=1-3]
	\arrow[from=1-3, to=2-3]
	\arrow[from=2-2, to=2-3]
	\arrow[from=1-1, to=2-1]
\end{tikzcd}\]is homotopy cocartesian. The right hand square is cocartesian by Proposition
\ref{prop:heads_and_tails}. It follows that the map
\[
\pr{\cal M_{/\sigma_{a}}^{\t}\times_{\Delta^{n}}\{0\}}\star\Delta^{\dim\sigma_{a}}\amalg_{\pr{\cal M_{/\sigma_{a}}^{\t}\times_{\Delta^{n}}\{0\}}\star\partial\Delta^{\dim\sigma_{a}}}\cal M_{<a}^{\t}\to\cal M_{\leq a}^{\t}
\]
is a trivial cofibration in the Joyal model structure. Thus we only
need to extend the composite
\[
g_{0}:\pr{\cal M_{/\sigma_{a}}^{\t}\times_{\Delta^{n}}\{0\}}\star\partial\Delta^{\dim\sigma_{a}}\to\cal M_{<a}^{\t}\xrightarrow{f^{<a}}\cal C^{\t}
\]
to a map $\pr{\cal M_{/\sigma_{a}}^{\t}\times_{\Delta^{n}}\{0\}}\star\Delta^{\dim\sigma_{a}}\to\cal C^{\t}$
over $\cal O^{\t}$.

Assume first that $\sigma_{a}$ is zero-dimensional, so that, in particular,
$m=1$. If $n>1$, then $F\pr 0=F'\pr 1$ and there is nothing to
do. If $n=1$, then let $B\in\cal M^{\t}$ be the image of $\sigma_{a}$.
Since $\sigma_{a}$ is closed, the object $B$ lies in $\cal M\times_{\Delta^{1}}\{1\}$.
Using the inert-active factorization system in $\cal M^{\t}$, we
see that the inclusion $\pr{\pr{\cal M_{\act}^{\t}}_{/B}\times_{\Delta^{1}}\{0\}}\subset\cal M_{/B}^{\t}\times_{\Delta^{1}}\{0\}$
is a right adjoint, hence final. So our assumption (b) ensures that
we can find the desired extension $g$. Note that condition (i) is
satisfied with this particular construction.

Assume next that $\sigma_{a}$ has positive dimension. Since $\Delta^{\dim\sigma_{a}}$
has an initial vertex, we see as in the previous paragraph that the
inclusion $\pr{\cal M_{\act}^{\t}}_{/\sigma_{a}}\times_{\Delta^{n}}\{0\}\subset\cal M_{/\sigma_{a}}^{\t}\times_{\Delta^{1}}\{0\}$
is a right adjoint, and hence final. Thus, by virtue of \cite[Proposition 3.1.1.7]{HA},
it suffices to show that the map
\[
\pr{\pr{\cal M_{\act}^{\t}}_{/\sigma_{a}}\times_{\Delta^{n}}\{0\}}\star\{0\}\to\cal M_{<a}^{\t}\xrightarrow{f^{<a}}\cal C^{\t}
\]
is an operadic $q$-colimit diagram. Let $B=\sigma_{a}\pr 0$. The
map $\pr{\cal M_{\act}^{\t}}_{/\sigma_{a}}\times_{\Delta^{n}}\{0\}\to\pr{\cal M_{\act}^{\t}}_{/B}\times_{\Delta^{n}}\{0\}$
is a trivial fibration, so it suffices to show that the map
\[
\phi:\pr{\pr{\cal M_{\act}^{\t}}_{/B}\times_{\Delta^{n}}\{0\}}\star\{0\}\to\cal M_{<a}^{\t}\xrightarrow{f^{<a}}\cal C^{\t}
\]
is an operadic $q$-colimit cone. Let $\inp p\in N\pr{\Fin_{\ast}}$
be the image of the object $B$. If $p=0$, then $\pr{\cal M_{\act}^{\t}}_{/B}\times_{\Delta^{n}}\{0\}$
is a contractible Kan complex, and for each object $\alpha:A\to B$
in $\pr{\cal M_{\act}^{\t}}_{/B}\times_{\Delta^{n}}\{0\}$, the map
$\phi\vert\{\alpha\}\star\{0\}$ is an equivalence in $\cal C^{\t}$
(since $\cal C_{\inp 0}^{\t}$ is a contractible Kan complex). Thus
$\phi$ is an operadic $q$-colimit cone by \cite[Example 3.1.1.6]{HA}.
If $p=1$, the claim follows from (i). If $p>1$, choose for each
$1\leq i\leq p$ an inert map $B\to B_{i}$ in $\cal M_{\act}^{\t}\times_{\Delta^{n}}\{1\}$
over $\rho^{i}:\inp p\to\inp 1$. Note that since $p>1$, we have
$m\geq2$, so that $f^{<a}$ is defined on $\cal M_{F\pr 1}^{\t}$. 

Using Proposition \ref{prop:oplus_p-limit}, choose a direct sum functor
$\bigoplus_{i=1}^{n}:\pr{\cal M_{\act}^{\t}}^{p}\times_{\pr{\Delta^{1}}^{p}}\Delta^{1}\to\cal M^{\t}$
so that there is an inert natural transformation $\widetilde{h}_{i}:\bigoplus_{i=1}^{n}\to\opn{pr}_{i}$
making the diagram % https://q.uiver.app/?q=WzAsNCxbMCwwLCIoKFxcbWF0aGNhbHtNfV5cXG90aW1lc197XFxtYXRocm17YWN0fX0pXnBcXHRpbWVzIF97KFxcRGVsdGFebilecH1cXERlbHRhXm4pXFx0aW1lcyBcXERlbHRhIF4xIl0sWzEsMCwiXFxtYXRoY2Fse019Xlxcb3RpbWVzICJdLFsxLDEsIlxcRGVsdGFeblxcdGltZXMgTihcXG1hdGhzZntGaW59X1xcYXN0KSJdLFswLDEsIihcXERlbHRhXm5cXHRpbWVzIE4oXFxtYXRoc2Z7RmlufV9cXGFzdCApX3tcXG1hdGhybXthY3R9fSlecClcXHRpbWVzIFxcRGVsdGFeMSJdLFswLDEsIlxcd2lkZXRpbGRle2h9X2kiXSxbMSwyLCJwIl0sWzMsMiwiXFxvcGVyYXRvcm5hbWV7aWR9X3tcXERlbHRhXm59XFx0aW1lcyBoX2kiLDJdLFswLDNdXQ==
\[\begin{tikzcd}
	{((\mathcal{M}^\otimes_{\mathrm{act}})^p\times _{(\Delta^n)^p}\Delta^n)\times \Delta ^1} & {\mathcal{M}^\otimes } \\
	{(\Delta^n\times N(\mathsf{Fin}_\ast )_{\mathrm{act}})^p)\times \Delta^1} & {\Delta^n\times N(\mathsf{Fin}_\ast)}
	\arrow["{\widetilde{h}_i}", from=1-1, to=1-2]
	\arrow["p", from=1-2, to=2-2]
	\arrow["{\operatorname{id}_{\Delta^n}\times h_i}"', from=2-1, to=2-2]
	\arrow[from=1-1, to=2-1]
\end{tikzcd}\]commutative. There is an equivalence $\alpha:B\xrightarrow{\simeq}\bigoplus_{i=1}^{p}B_{i}$
lying over the identity of $\inp p$. There is a natural equivalence
\[
H:\pr{\pr{\cal M_{\act}^{\t}}_{/\alpha}\times_{\Delta^{n}}\{0\}}^{\rcone}\times\Delta^{1}\to\cal M^{\t}
\]
which makes the diagram % https://q.uiver.app/?q=WzAsNixbMCwwLCIoKFxcbWF0aGNhbHtNfV5cXG90aW1lcyBfe1xcbWF0aHJte2FjdH19KV97L1xcYWxwaGF9XFx0aW1lcyBfe1xcRGVsdGFebn1cXHswXFx9KV5cXHRyaWFuZ2xlcmlnaHRcXHRpbWVzIFxcezBcXH0iXSxbMSwwLCIoKFxcbWF0aGNhbHtNfV5cXG90aW1lcyBfe1xcbWF0aHJte2FjdH19KV97L0J9XFx0aW1lcyBfe1xcRGVsdGFebn1cXHswXFx9KV5cXHRyaWFuZ2xlcmlnaHQiXSxbMCwxLCIoKFxcbWF0aGNhbHtNfV5cXG90aW1lcyBfe1xcbWF0aHJte2FjdH19KV97L1xcYWxwaGF9XFx0aW1lcyBfe1xcRGVsdGFebn1cXHswXFx9KV5cXHRyaWFuZ2xlcmlnaHRcXHRpbWVzIFxcRGVsdGFeMSJdLFsxLDEsIlxcbWF0aGNhbHtNfV5cXG90aW1lcyAiXSxbMCwyLCIoKFxcbWF0aGNhbHtNfV5cXG90aW1lcyBfe1xcbWF0aHJte2FjdH19KV97L1xcYWxwaGF9XFx0aW1lcyBfe1xcRGVsdGFebn1cXHswXFx9KV5cXHRyaWFuZ2xlcmlnaHRcXHRpbWVzIFxcezFcXH0iXSxbMSwyLCIoKFxcbWF0aGNhbHtNfV5cXG90aW1lcyBfe1xcbWF0aHJte2FjdH19KV97L1xcYmlnb3BsdXNfe2k9MX1ebkJfaX1cXHRpbWVzIF97XFxEZWx0YV5ufVxcezBcXH0pXlxcdHJpYW5nbGVyaWdodFxcdGltZXMgXFx7MVxcfSJdLFswLDFdLFswLDJdLFsxLDNdLFsyLDNdLFs0LDJdLFs1LDNdLFs0LDVdXQ==
\[\begin{tikzcd}
	{((\mathcal{M}^\otimes _{\mathrm{act}})_{/\alpha}\times _{\Delta^n}\{0\})^\triangleright\times \{0\}} & {((\mathcal{M}^\otimes _{\mathrm{act}})_{/B}\times _{\Delta^n}\{0\})^\triangleright} \\
	{((\mathcal{M}^\otimes _{\mathrm{act}})_{/\alpha}\times _{\Delta^n}\{0\})^\triangleright\times \Delta^1} & {\mathcal{M}^\otimes } \\
	{((\mathcal{M}^\otimes _{\mathrm{act}})_{/\alpha}\times _{\Delta^n}\{0\})^\triangleright\times \{1\}} & {((\mathcal{M}^\otimes _{\mathrm{act}})_{/\bigoplus_{i=1}^nB_i}\times _{\Delta^n}\{0\})^\triangleright\times \{1\}}
	\arrow[from=1-1, to=1-2]
	\arrow[from=1-1, to=2-1]
	\arrow[from=1-2, to=2-2]
	\arrow[from=2-1, to=2-2]
	\arrow[from=3-1, to=2-1]
	\arrow[from=3-2, to=2-2]
	\arrow[from=3-1, to=3-2]
\end{tikzcd}\]commutative, such that the restriction of $H$ to $\pr{\pr{\cal M_{\act}^{\t}}_{/\alpha}\times_{\Delta^{n}}\{0\}}\times\Delta^{1}$
is the constant natural transformation at the projection $\pr{\pr{\cal M_{\act}^{\t}}_{/\alpha}\times_{\Delta^{n}}\{0\}}\to\cal M^{\t}$
and the restriction $H\vert\{\infty\}\times\Delta^{1}$ is given by
$\alpha$. This natural equivalence takes values in $\cal M_{F\pr 1}^{\t}$,
because $\alpha$ lies in $\cal M_{F\pr 1}^{\t}$. So it suffices
to show that the composite 
\[
\phi:\pr{\pr{\cal M_{\act}^{\t}}_{/\bigoplus_{i=1}^{p}B_{i}}\times_{\Delta^{n}}\{0\}}^{\rcone}\to\cal M_{F\pr 1}^{\t}\xrightarrow{f_{1}}\cal C^{\t}
\]
is an operadic $q$-colimit diagram. Now consider the commutative
diagram % https://q.uiver.app/?q=WzAsNSxbMCwwLCIoXFxwcm9kX3sxXFxsZXEgaVxcbGVxIHB9KFxcbWF0aGNhbHtNfV5cXG90aW1lcyBfe1xcbWF0aHJte2FjdH19KV97L0JfaX1cXHRpbWVzIF97XFxEZWx0YV5ufVxcezBcXH0pXlxcdHJpYW5nbGVyaWdodCJdLFsxLDAsIigoKFxcbWF0aGNhbHtNfV5cXG90aW1lcyBfe1xcbWF0aHJte2FjdH19KV5wXFx0aW1lcyBfeyhcXERlbHRhXm4pXnB9XFxEZWx0YV5uKV97LyhCXzEsXFxkb3RzICxCX3ApfVxcdGltZXMgX3tcXERlbHRhXm59XFx7MFxcfSleXFx0cmlhbmdsZXJpZ2h0Il0sWzEsMSwiKChcXG1hdGhjYWx7TX1eXFxvdGltZXMgX3tcXG1hdGhybXthY3R9fSlfey9cXGJpZ29wbHVzX3tpPTF9XnBCX2l9XFx0aW1lcyBfe1xcRGVsdGFebn1cXHswXFx9KV5cXHRyaWFuZ2xlcmlnaHQiXSxbMSwyLCJcXG1hdGhjYWx7TX1eXFxvdGltZXMgLiJdLFswLDIsIihcXG1hdGhjYWx7TX1eXFxvdGltZXMgX3tcXG1hdGhybXthY3R9fSlecFxcdGltZXMgX3soXFxEZWx0YV5uKV5wfVxcRGVsdGFebiJdLFswLDEsIlxcY29uZyJdLFsxLDIsIlxccHNpIl0sWzIsM10sWzQsMywiXFxiaWdvcGx1c197aT0xfV5wIiwyXSxbMCw0XSxbMSwyLCJcXHNpbWVxIiwyXV0=
\[\begin{tikzcd}
	{(\prod_{1\leq i\leq p}(\mathcal{M}^\otimes _{\mathrm{act}})_{/B_i}\times _{\Delta^n}\{0\})^\triangleright} & {(((\mathcal{M}^\otimes _{\mathrm{act}})^p\times _{(\Delta^n)^p}\Delta^n)_{/(B_1,\dots ,B_p)}\times _{\Delta^n}\{0\})^\triangleright} \\
	& {((\mathcal{M}^\otimes _{\mathrm{act}})_{/\bigoplus_{i=1}^pB_i}\times _{\Delta^n}\{0\})^\triangleright} \\
	{(\mathcal{M}^\otimes _{\mathrm{act}})^p\times _{(\Delta^n)^p}\Delta^n} & {\mathcal{M}^\otimes .}
	\arrow["\cong", from=1-1, to=1-2]
	\arrow["\psi", from=1-2, to=2-2]
	\arrow[from=2-2, to=3-2]
	\arrow["{\bigoplus_{i=1}^p}"', from=3-1, to=3-2]
	\arrow[from=1-1, to=3-1]
	\arrow["\simeq"', from=1-2, to=2-2]
\end{tikzcd}\]Here the map $\psi$ is the equivalence of $\infty$-categories induced
by the direct sum functor (Proposition \ref{prop:directsum_slice_equiv}).
In light of the commutativity of this diagram, the composite
\[
\eta:\pr{\prod_{1\leq i\leq p}\pr{\cal M_{\act}^{\t}}_{/B_{i}}\times_{\Delta^{n}}\{0\}}^{\rcone}\to\pr{\cal M_{\act}^{\t}}^{p}\times_{\pr{\Delta^{n}}^{p}}\Delta^{n}\xrightarrow{\bigoplus_{i=1}^{p}}\cal M^{\t}
\]
takes values in $\cal M_{F\pr 1}^{\t}$, and it suffices to show that
the composite $f_{1}\eta$ is an operadic $q$-colimit diagram. Now
the inert natural transformation $\widetilde{h}_{i}$ induces an inert
natural transfromation from $\eta$ to the composite 
\[
\eta_{i}:\pr{\prod_{1\leq i\leq p}\pr{\cal M_{\act}^{\t}}_{/B_{i}}\times_{\Delta^{n}}\{0\}}^{\rcone}\to\pr{\pr{\cal M_{\act}^{\t}}_{/B_{i}}\times_{\Delta^{n}}\{0\}}^{\rcone}\to\cal M^{\t}.
\]
Since $F\pr 1$ contains the $1$-simplices in $G_{\pr 2}$, this
natural transformation takes values in $\cal M_{F\pr 1}^{\t}$. Since
$f_{1}$ satisfies (ii), we deduce that the composite $f_{1}\eta$
admits an inert natural transformation $H_{i}$ to the composite $f_{1}\eta_{i}$,
such that for each vertex $v$ in $\pr{\prod_{1\leq i\leq p}\pr{\cal M_{\act}^{\t}}_{/B_{i}}\times_{\Delta^{n}}\{0\}}^{\rcone}$,
the components $\{H_{i}\pr v\}_{1\leq i\leq p}$ form a $q$-limit
cone. It follows from Corollary \ref{cor:3.1.1.8} and (i) that $f_{1}\eta$
is an operadic $q$-colimit diagram, as desired.

\item We will extend the map $f'_{m}$ in Step 1 to a map $f''_{m}:\cal M_{F''\pr m}^{\t}\to\cal C^{\t}$
over $\cal O^{\t}$. 

We argue as in Step 1. Let $\{\sigma_{a}\}_{a\in A}$ be the set of
all nondegenerate simplices of $\cal M^{\t}\times_{\Delta^{n}}\Delta^{\{1,\dots,n\}}$
whose image in $N\pr{\Fin_{\ast}}\times\Delta^{\{1,\dots,n\}}$ is
a degeneration of an $\pr{m-1}$-simplex in $G'_{\pr 2}$ without
associates. Choose a well-ordering of the set $A$ so that $\dim\sigma_{a}$
is a non-decreasing function of $a$. For each $a\in A$, let $\cal M_{<a}^{\t}$
denote the simplicial subset spanned by $\cal M_{F'\pr m}^{\t}$ and
the simplices of $\cal M^{\t}$ whose tail factors through $\sigma_{b}$
for some $b<a$. We define $\cal M_{\leq a}^{\t}$ similarly. (The
notations $\cal M_{<a}^{\t}$ and $\cal M_{\leq a}^{\t}$ are in conflict
with the ones introduced in Step 1, but there should not be any confusion.)
By Proposition \ref{prop:heads_and_tails}, we have $\cal M_{F''\pr m}^{\t}=\bigcup_{a\in A}\cal M_{\leq a}^{\t}$,
so it suffices to extend $f'_{m}$ to an $A$-sequence $f^{\leq a}:\cal M_{\leq a}^{\t}\to\cal C^{\t}$
over $\cal O^{\t}$. The construction is inductive. Suppose $f^{\leq b}$
has been constructed for $b<a$, and let $f^{<a}:\cal M_{<a}^{\t}\to\cal C^{\t}$
be their amalgamation. Just as in Step 1, we are reduced to solving
a lifting problem of the form % https://q.uiver.app/?q=WzAsNixbMCwwLCIoXFxtYXRoY2Fse019Xlxcb3RpbWVzIF97L1xcc2lnbWFfYX1cXHRpbWVzX3tcXERlbHRhXm59IFxcezBcXH0pXFxzdGFyIFxccGFydGlhbFxcRGVsdGEgXntcXGRpbVxcc2lnbWEgX2F9Il0sWzAsMSwiKFxcbWF0aGNhbHtNfV5cXG90aW1lcyBfey9cXHNpZ21hX2F9XFx0aW1lc197XFxEZWx0YV5ufSBcXHswXFx9KVxcc3RhciBcXERlbHRhIF57XFxkaW1cXHNpZ21hIF9hfSJdLFsxLDAsIlxcbWF0aGNhbHtNfV5cXG90aW1lcyBfezxhfSJdLFsyLDAsIlxcbWF0aGNhbHtDfV5cXG90aW1lcyJdLFsyLDEsIlxcbWF0aGNhbHtPfV5cXG90aW1lcyJdLFsxLDEsIlxcbWF0aGNhbHtNfV5cXG90aW1lcyAiXSxbMCwxXSxbMCwyXSxbMiwzLCJmXns8YX0iXSxbMSw1XSxbNSw0XSxbMyw0LCJxIl0sWzEsMywiIiwxLHsic3R5bGUiOnsiYm9keSI6eyJuYW1lIjoiZGFzaGVkIn19fV1d
\[\begin{tikzcd}
	{(\mathcal{M}^\otimes _{/\sigma_a}\times_{\Delta^n} \{0\})\star \partial\Delta ^{\dim\sigma _a}} & {\mathcal{M}^\otimes _{<a}} & {\mathcal{C}^\otimes} \\
	{(\mathcal{M}^\otimes _{/\sigma_a}\times_{\Delta^n} \{0\})\star \Delta ^{\dim\sigma _a}} & {\mathcal{M}^\otimes } & {\mathcal{O}^\otimes.}
	\arrow[from=1-1, to=2-1]
	\arrow[from=1-1, to=1-2]
	\arrow["{f^{<a}}", from=1-2, to=1-3]
	\arrow[from=2-1, to=2-2]
	\arrow[from=2-2, to=2-3]
	\arrow["q", from=1-3, to=2-3]
	\arrow[dashed, from=2-1, to=1-3]
\end{tikzcd}\]The existence of such a lift follows from Corollary \ref{cor:Step2}.

\item We complete the proof by extending the map $f''_{m}$ in Step
2 to a map $f_{m}:\cal M_{F\pr m}^{\t}\to\cal C^{\t}$ over $\cal O^{\t}$.
Let $\{\sigma'_{a}\}_{a\in A}$ be the collection of all $\pr{m-1}$-simplices
in $G'_{\pr 2}$ and $G'_{\pr 3}$ which have at least one associate.
Choose a well-ordering on $A$, and for each $a\in A$, let $F_{\leq a}$
denote the simplicial subset of $F\pr m$ generated by $F''\pr m$
and the associates of the simplices $\sigma'_{b}$ for $b\leq a$.
Define $F_{<a}$ similarly. We will choose the ordering on $A$ so
that the following condition is satisfied:
\begin{itemize}
\item [($\blacklozenge$)]Let $a\in A$ and let $\sigma$ be an associate
of $\sigma'_{a}$. Let $0\leq l\leq m$ be the (unique) integer such
that $d_{l}\sigma=\sigma'_{a}$. Then for each $i\in[m]\setminus\{l\}$,
the simplex $d_{i}\sigma$ belongs to $F_{<a}$.
\end{itemize}
The existence of such an ordering is nontrivial, but we will not get
into it presently. The construction of such an ordering can be found
at the very end of this step. Note that ($\blacklozenge$) implies
that:
\begin{itemize}
\item [($\blacklozenge\blacklozenge$)]For every $a\in A$, the simplex
$\sigma'_{a}$ does not belong to $F_{<a}$.
\end{itemize}
Indeed, suppose $\sigma'_{a}$ belongs to $F_{<a}$. Choose a minimal
element $b\in A$ such that $\sigma'_{a}$ belongs to $F_{<b}$. Then
$\sigma'_{a}$ factors through one of the following simplices:
\begin{enumerate}
\item Incomplete simplices.
\item Nondegenerate simplices of dimensions less than $m-1$.
\item $\pr{m-1}$-simplices in $G_{\pr 1}\cup G_{\pr 2}\cup G_{\pr 3}$.
\item $\pr{m-1}$-simplices in $G'_{\pr 2}$ without associates.
\item Associates of $\sigma'_{c}$ for some $c<b$.
\end{enumerate}
Since $\sigma'_{a}$ is complete and nondegenerate and has dimension
$m-1$, the cases (1), (2), (3), and (4) are immediately ruled out.
We show that the case (5) is impossible by reasoning by contradiction.
Suppose that there are an index $c<b$ and an associate $\sigma$
of $\sigma'_{c}$ through which $\sigma'_{a}$ factors. We have $\sigma'_{a}=d_{i}\sigma$
for some $0\leq i\leq m$ for dimensional reasons. Using ($\blacklozenge$),
we deduce that $\sigma'_{a}$ belongs to $F_{<c}$, contrary to the
minimality of $b$.

We now get back to the construction of $f_{m}$. We will construct
$f_{m}$ as an amalgamation of an $A$-sequence $\{f_{\leq a}:\cal M_{F_{\leq a}}^{\t}\to\cal C^{\t}\}_{a\in A}$
over $\cal O^{\t}$ which extends $f''_{m}$. The construction is
inductive. Suppose that $f_{\leq b}$ has been constructed for $b\leq a$,
so that they together determine a map $f_{<a}:\cal M_{F_{<a}}^{\t}\to\cal C^{\t}$.
We must extend $f_{<a}$ to $\cal M_{F_{\leq a}}^{\t}$. We consider
two cases, depending on whether $\sigma'_{a}$ belongs to $G'_{\pr 2}$
or to $G'_{\pr 3}$.

\begin{enumerate}[label=(Case \arabic*),wide =0.5\parindent=, listparindent=1.5em]

\item Suppose that $\sigma'_{a}$ belongs to $G'_{\pr 2}$. We shall
deploy an argument which is a variant of Proposition \ref{prop:heads_and_tails}.
Let $\{\tau_{\lambda}\}_{\lambda\in\Lambda}$ be the collection of
all nondegenerate simplices of $\cal M^{\t}$ the image of whose head
in $N\pr{\Fin_{\ast}}\times\Delta^{\{1,\dots,n\}}$ is a degneration
of $\sigma'_{a}$. Choose a well-ordering of $\Lambda$ so that $\dim\tau_{\lambda}$
is a non-decreasing function of $\lambda$. For each $\lambda\in\Lambda$,
we let $\cal N_{\leq\lambda}\subset\cal M^{\t}$ denote the simplicial
subset generated by $\cal M_{F_{<a}}^{\t}$ and the simplices $\tau:\Delta^{p}\to\cal M^{\t}$
for which there is an integer $0\leq p'<p$ such that $\tau\vert\Delta^{\{0,\dots,p'\}}$
factors through some $\tau_{\mu}$ for some $\mu\leq\lambda$, $\tau\vert\Delta^{\{p',p'+1\}}$
is inert, and $\tau\vert\Delta^{\{p'+1,\dots,p\}}$ factors through
$\cal M$. We define $\cal N_{<\lambda}$ similarly. 

We shall prove the following assertion later:
\begin{itemize}
\item [($*$)]$\cal M_{F_{\leq a}}^{\t}$ is the union of $\cal M_{F_{<a}}^{\t}$
and $\{\cal N_{\leq\lambda}\}_{\lambda\in\Lambda}$. 
\end{itemize}
Accepting ($\ast$) for now, we complete the proof as follows. It
will suffice construct a $\Lambda$-sequence $\{f^{\leq\lambda}:\cal N_{\leq\lambda}\to\cal C^{\t}\}_{\lambda\in\Lambda}$
of maps over $\cal O^{\t}$ which extends $f_{<a}$. The construction
is inductive. Suppose $f^{\leq\mu}$ has been constructed for $\mu<\lambda$,
and let $f^{<\lambda}:\cal N_{<\lambda}\to\cal C^{\t}$ denote the
map obtained by amalgamating the maps $\{f^{\leq\mu}\}_{\mu<\lambda}$.
Let $\pr{\inp k,n}$ be the final vertex of $\tau_{\lambda}$. Note
that $k\geq2$. There are $k$ inert maps  $\inp k\to\inp 1$, and
these maps and $p\tau_{\lambda}$ combine to determine a diagram $\Delta^{\dim\tau_{\lambda}}\star\inp k^{\circ}\to N\pr{\Fin_{\ast}}\times\Delta^{n}$.
Let $\cal X=\pr{\Delta^{\dim\tau_{\lambda}}\star\inp k^{\circ}}\times_{N\pr{\Fin_{\ast}}\times\Delta^{n}}\cal M^{\t}$
and let $\overline{\tau}_{\lambda}:\Delta^{\dim\tau_{\lambda}}\to\cal X$
denote the induced diagram. For each $1\leq i\leq k$, let $\cal X_{i}$
denote the fiber of $\cal X$ over $i\in\inp k^{\circ}$ (which is
isomorphic to $\cal M\times_{\Delta^{n}}\{n\}$), and set $\cal X^{0}=\bigcup_{1\leq i\leq k}\cal X_{i}$
and $\cal X_{\overline{\tau}_{\lambda}/}^{0}=\cal X^{0}\times_{\cal X}\cal X_{\overline{\tau}_{\lambda}/}$.
We now consider the following diagram: % https://q.uiver.app/?q=WzAsNixbMCwwLCJcXHBhcnRpYWxcXERlbHRhXntcXGRpbVxcb3ZlcmxpbmVcXHRhdV9cXGxhbWJkYX1cXHN0YXJcXG1hdGhjYWx7WH1eMF97XFxvdmVybGluZVxcdGF1X1xcbGFtYmRhL30iXSxbMCwxLCJcXERlbHRhXntcXGRpbVxcb3ZlcmxpbmVcXHRhdV9cXGxhbWJkYX1cXHN0YXJcXG1hdGhjYWx7WH1eMF97XFxvdmVybGluZVxcdGF1X1xcbGFtYmRhL30iXSxbMSwwLCJLXzAiXSxbMSwxLCJLIl0sWzIsMCwiXFxtYXRoY2Fse059X3s8XFxsYW1iZGF9Il0sWzIsMSwiXFxtYXRoY2Fse059X3tcXGxlcSBcXGxhbWJkYX0iXSxbMCwxXSxbMCwyXSxbMSwzXSxbMiwzXSxbMiw0XSxbMyw1XSxbNCw1XV0=
\[\begin{tikzcd}
	{\partial\Delta^{\dim\overline\tau_\lambda}\star\mathcal{X}^0_{\overline\tau_\lambda/}} & {K_0} & {\mathcal{N}_{<\lambda}} \\
	{\Delta^{\dim\overline\tau_\lambda}\star\mathcal{X}^0_{\overline\tau_\lambda/}} & K & {\mathcal{N}_{\leq \lambda}.}
	\arrow[from=1-1, to=2-1]
	\arrow[from=1-1, to=1-2]
	\arrow[from=2-1, to=2-2]
	\arrow[from=1-2, to=2-2]
	\arrow[from=1-2, to=1-3]
	\arrow[from=2-2, to=2-3]
	\arrow[from=1-3, to=2-3]
\end{tikzcd}\]Here $K\subset\cal X$ denotes the simplicial subset spanned by the
simplices whose tail factors through $\overline{\tau}_{\lambda}$,
and $K_{0}$ is its simplicial subset obtained by replacing $\overline{\tau}_{\lambda}$
by $\partial\overline{\tau}_{\lambda}$, where we regard $\cal X$
as a simplicial set over $\Delta^{1}$ by the map $\Delta^{\dim\tau_{\lambda}}\star\inp k^{\circ}\to\{0\}\star\{1\}=\Delta^{1}$.
The left hand square is homotopy cocartesian by Lemma \ref{lem:3.1.2.5}.
We shall prove later that:
\begin{itemize}
\item [($**$)]The horizontal arrows of the the right hand square are well-defined,
and the right hand square is cocartesian.
\end{itemize}
We are now reduced to solving the lifting problem % https://q.uiver.app/?q=WzAsNixbMCwwLCJcXHBhcnRpYWxcXERlbHRhXntcXGRpbVxcb3ZlcmxpbmVcXHRhdV9cXGxhbWJkYX1cXHN0YXJcXG1hdGhjYWx7WH1eMF97XFxvdmVybGluZVxcdGF1X1xcbGFtYmRhL30iXSxbMCwxLCJcXERlbHRhXntcXGRpbVxcb3ZlcmxpbmVcXHRhdV9cXGxhbWJkYX1cXHN0YXJcXG1hdGhjYWx7WH1eMF97XFxvdmVybGluZVxcdGF1X1xcbGFtYmRhL30iXSxbMSwwLCJcXG1hdGhjYWx7Tn1fezxcXGxhbWJkYX0iXSxbMSwxLCJcXG1hdGhjYWx7Tn1fe1xcbGVxIFxcbGFtYmRhfSJdLFsyLDAsIlxcbWF0aGNhbHtDfV5cXG90aW1lcyAiXSxbMiwxLCJcXG1hdGhjYWx7T31eXFxvdGltZXMgLiJdLFswLDFdLFsyLDRdLFszLDVdLFs0LDVdLFsxLDQsIiIsMSx7InN0eWxlIjp7ImJvZHkiOnsibmFtZSI6ImRhc2hlZCJ9fX1dLFswLDJdLFsxLDNdXQ==
\[\begin{tikzcd}
	{\partial\Delta^{\dim\overline\tau_\lambda}\star\mathcal{X}^0_{\overline\tau_\lambda/}} & {\mathcal{N}_{<\lambda}} & {\mathcal{C}^\otimes } \\
	{\Delta^{\dim\overline\tau_\lambda}\star\mathcal{X}^0_{\overline\tau_\lambda/}} & {\mathcal{N}_{\leq \lambda}} & {\mathcal{O}^\otimes .}
	\arrow[from=1-1, to=2-1]
	\arrow[from=1-2, to=1-3]
	\arrow[from=2-2, to=2-3]
	\arrow[from=1-3, to=2-3]
	\arrow[dashed, from=2-1, to=1-3]
	\arrow[from=1-1, to=1-2]
	\arrow[from=2-1, to=2-2]
\end{tikzcd}\]Now the $\infty$-category $\cal X_{\overline{\tau}_{\lambda}/}^{0}$
is the disjoint union of the $\infty$-categories $\pr{\cal X_{i}}_{\overline{\tau}_{\lambda}/}=\cal X_{i}\times_{\cal X}\cal X_{\overline{\tau}_{\lambda}/}$.
Each $\infty$-category $\pr{\cal X_{i}}_{\overline{\tau}_{\lambda}/}$
has an initial object, given by a cone $\phi_{i}:\pr{\Delta^{\dim\overline{\tau}_{\lambda}}}^{\rcone}\to\cal X$
which maps the last edge to an inert morphism over $\pr{\rho^{i},\id}:\pr{\inp k,n}\to\pr{\inp 1,n}$.
Set $S=\coprod_{i}\{\phi_{i}\}$. The inclusion $S\subset\cal X_{\overline{\tau}_{\lambda}/}^{0}$
is initial, so we are reduced to solving the lifting problem % https://q.uiver.app/?q=WzAsOCxbMSwwLCJcXHBhcnRpYWxcXERlbHRhXntcXGRpbVxcb3ZlcmxpbmVcXHRhdV9cXGxhbWJkYX1cXHN0YXJcXG1hdGhjYWx7WH1eMF97XFxvdmVybGluZVxcdGF1X1xcbGFtYmRhL30iXSxbMSwxLCJcXERlbHRhXntcXGRpbVxcb3ZlcmxpbmVcXHRhdV9cXGxhbWJkYX1cXHN0YXJcXG1hdGhjYWx7WH1eMF97XFxvdmVybGluZVxcdGF1X1xcbGFtYmRhL30iXSxbMiwwLCJcXG1hdGhjYWx7Tn1fezxcXGxhbWJkYX0iXSxbMiwxLCJcXG1hdGhjYWx7Tn1fe1xcbGVxIFxcbGFtYmRhfSJdLFszLDAsIlxcbWF0aGNhbHtDfV5cXG90aW1lcyAiXSxbMywxLCJcXG1hdGhjYWx7T31eXFxvdGltZXMgLiJdLFswLDAsIlxccGFydGlhbFxcRGVsdGFee1xcZGltXFxvdmVybGluZVxcdGF1X1xcbGFtYmRhfVxcc3RhciBTIl0sWzAsMSwiXFxEZWx0YV57XFxkaW1cXG92ZXJsaW5lXFx0YXVfXFxsYW1iZGF9XFxzdGFyIFMiXSxbMiw0XSxbMyw1XSxbNCw1XSxbMCwyXSxbMSwzXSxbNiwwXSxbNywxXSxbNyw0LCIiLDEseyJzdHlsZSI6eyJib2R5Ijp7Im5hbWUiOiJkYXNoZWQifX19XSxbNiw3XV0=
\[\begin{tikzcd}
	{\partial\Delta^{\dim\overline\tau_\lambda}\star S} & {\partial\Delta^{\dim\overline\tau_\lambda}\star\mathcal{X}^0_{\overline\tau_\lambda/}} & {\mathcal{N}_{<\lambda}} & {\mathcal{C}^\otimes } \\
	{\Delta^{\dim\overline\tau_\lambda}\star S} & {\Delta^{\dim\overline\tau_\lambda}\star\mathcal{X}^0_{\overline\tau_\lambda/}} & {\mathcal{N}_{\leq \lambda}} & {\mathcal{O}^\otimes .}
	\arrow[from=1-3, to=1-4]
	\arrow[from=2-3, to=2-4]
	\arrow[from=1-4, to=2-4]
	\arrow[from=1-2, to=1-3]
	\arrow[from=2-2, to=2-3]
	\arrow[from=1-1, to=1-2]
	\arrow[from=2-1, to=2-2]
	\arrow[dashed, from=2-1, to=1-4]
	\arrow[from=1-1, to=2-1]
\end{tikzcd}\] If the dimension of $\overline{\tau}_{\lambda}$ is positive, then
the claim is immediate since the restriction of the top horizontal
arrow to $\{\dim\overline{\tau}_{\lambda}\}\star S$ is a $q$-limit
cone. If $\overline{\tau}_{\lambda}$ is zero-dimensional (in which
case $m=n=1$), let $C_{i}$ denote the image of $\phi_{i}\in S$
under the top horizontal map. The bottom horizontal arrow classifies
a diagram $X\to q\pr{C_{i}}$ of inert maps $\{\alpha_{i}:X\to q\pr{C_{i}}\}_{1\leq i\leq k}$
lying over $\{\rho^{i}:\inp k\to\inp 1\}_{1\leq i\leq k}$, and we
wish to lift this to a diagram $S^{\lcone}\to\cal C^{\t}$ in $\cal C^{\t}$
which maps each $\phi_{i}\in S$ to the object $C_{i}$. Since $q$
is a fibration of $\infty$-operads, we can in fact find such a lift
consisting of inert morphisms. Note that with such a choice of lift,
condition (ii) is satisfied.

We now move on to the verification of ($\ast$) and ($\ast\ast$).
We begin with ($\ast$). It is clear that $\cal N_{\leq\lambda}$
and $\cal M_{F_{<a}}^{\t}$ are contained in $\cal M_{F_{\leq a}}^{\t}$.
For the reverse implication, let $x$ be an arbitrary simplex of $\cal M_{F_{\leq a}}^{\t}$.
We must show that $x$ belongs to either $\cal M_{F_{<a}}^{\t}$ or
one of the $\cal N_{\leq\lambda}$'s. If $x$ belongs to $\cal M_{F_{<a}}^{\t}$,
we are done. So assume not. Let $r$ denote the dimension of $x$.
Since $x$ does not belong to $\cal M_{F_{<a}}^{\t}$, its head is
nonempty. Find an integer $0\leq r'\leq r$ such that $x\vert\Delta^{\{r',\dots,r\}}$
is the head of $x$. Since $x$ does not belong to $\cal M_{F_{<a}}^{\t}$,
the simplex $px\vert\Delta^{\{r',\dots,r\}}$ factors through an associate
$\sigma$ of $\sigma'_{a}$. Let $\Delta^{r}\xrightarrow{u}\Delta^{m}\xrightarrow{\sigma}N\pr{\Fin_{\ast}}\times\Delta^{\{1,\dots,n\}}$
be such a factorization. If the image of $u$ does not contain some
integer $i\in\{0,\dots,m-1\}$, then $px\vert\Delta^{\{r',\dots,r\}}$
factors through $d_{i}\sigma$ and hnece through $F_{<a}$ by ($\blacklozenge$),
a contradiction. So the image of $u$ contains every integer in $\{0,\dots,m-1\}$.
Let $0\leq r''<r$ be the largest integer such that $u\pr{r''}<m$.
There are now several cases to consider:
\begin{itemize}
\item Suppose that $u$ is surjective on vertices and that the map $x\pr{r''}\to x\pr{r''+1}$
is inert. We write $x\vert\Delta^{\{0,\dots,r''\}}=s^{*}y$, where
$s:[r']\to[k]$ is a surjection and $y$ is nondegenerate. We claim
that $y$ is one of the $\tau_{\lambda}$'s, so that $x$ belongs
to $\cal N_{\leq\lambda}$. Let $k'=s\pr{r'}$. Then $y\vert\Delta^{\{k',\dots,k\}}$
is the head of $y$. We write $py\vert\Delta^{\{k',\dots,k\}}=s^{\p\ast}z$,
where $z$ is nondegenerate and $s'$ is a surjection. Then we obtain
the following commutative diagram: % https://q.uiver.app/?q=WzAsOCxbMSwwLCJcXERlbHRhXntcXHtyJyxcXGRvdHMscicnXFx9fSJdLFsyLDAsIlxcRGVsdGFee20tMX0iXSxbMSwxLCJcXG1hdGhjYWx7TX1eXFxvdGltZXMgIl0sWzIsMSwiTihcXG1hdGhzZntGaW59X1xcYXN0KVxcdGltZXMgXFxEZWx0YV5uLiJdLFswLDAsIlxcRGVsdGFee1xcezAsXFxkb3RzLHInJ1xcfX0iXSxbMSwyLCJcXERlbHRhXmwiXSxbMCwyLCJcXERlbHRhXntcXHtrJyxcXGRvdHNtLGtcXH19Il0sWzAsMSwiXFxEZWx0YV57a30iXSxbMCwxLCJ1XFx2ZXJ0XFxEZWx0YV57XFx7cicsXFxkb3RzLHInJ1xcfX0iLDAseyJzdHlsZSI6eyJoZWFkIjp7Im5hbWUiOiJlcGkifX19XSxbMSwzLCJcXHNpZ21hX2EnIl0sWzIsMywicCIsMl0sWzQsNywicyIsMix7InN0eWxlIjp7ImhlYWQiOnsibmFtZSI6ImVwaSJ9fX1dLFs3LDIsInkiLDJdLFswLDQsIiIsMCx7InN0eWxlIjp7InRhaWwiOnsibmFtZSI6Imhvb2siLCJzaWRlIjoiYm90dG9tIn19fV0sWzYsNywiIiwyLHsic3R5bGUiOnsidGFpbCI6eyJuYW1lIjoiaG9vayIsInNpZGUiOiJib3R0b20ifX19XSxbNiw1LCJzJyIsMix7InN0eWxlIjp7ImhlYWQiOnsibmFtZSI6ImVwaSJ9fX1dLFs2LDJdLFs1LDMsInoiLDJdLFs0LDIsInhcXHZlcnRcXERlbHRhXntcXHswLFxcZG90cyxyJydcXH19IiwxXV0=
\[\begin{tikzcd}
	{\Delta^{\{0,\dots,r''\}}} & {\Delta^{\{r',\dots,r''\}}} & {\Delta^{m-1}} \\
	{\Delta^{k}} & {\mathcal{M}^\otimes } & {N(\mathsf{Fin}_\ast)\times \Delta^n.} \\
	{\Delta^{\{k',\dotsm,k\}}} & {\Delta^l}
	\arrow["{u\vert\Delta^{\{r',\dots,r''\}}}", two heads, from=1-2, to=1-3]
	\arrow["{\sigma_a'}", from=1-3, to=2-3]
	\arrow["p"', from=2-2, to=2-3]
	\arrow["s"', two heads, from=1-1, to=2-1]
	\arrow["y"', from=2-1, to=2-2]
	\arrow[hook', from=1-2, to=1-1]
	\arrow[hook', from=3-1, to=2-1]
	\arrow["{s'}"', two heads, from=3-1, to=3-2]
	\arrow[from=3-1, to=2-2]
	\arrow["z"', from=3-2, to=2-3]
	\arrow["{x\vert\Delta^{\{0,\dots,r''\}}}"{description}, from=1-1, to=2-2]
\end{tikzcd}\]Applying Eilenberg-Zilber's lemma to $px\vert\Delta^{\{r',\dots,r''\}}$,
we deduce that $z=\sigma'_{a}$. Thus $y$ is one of the simplices
in $\{\tau_{\lambda}\}_{\lambda\in\Lambda}$, as required.
\item Suppose that $u$ is surjective on vertices and that the map $\theta:u\pr{r''}\to u\pr{r''+1}$
is not inert. By factoring the map $\theta$ into an inert map followed
by an active map, we can find an $\pr{r+1}$-simplex $y$ of $\cal M_{F_{\leq a}}^{\t}$
such that $d_{r''+1}y=x$ and $y\vert\Delta^{\{r'',r''+1,r''+2\}}$
is the chosen factorization of $\theta$. By the previous point, the
simplex $y$ belongs to $\cal N_{\leq\lambda}$ for some $\lambda$,
and hence so must $x$.
\item Suppose $u$ is not surjective on vertices. Find an inert map $\theta:u\pr r\to X$
over the last edge of $\sigma$, and let $y$ be an $\pr{r+1}$-simplex
$y$ of $\cal M_{F_{\leq a}}^{\t}$ such that $d_{r+1}y=x$ and $y\vert\Delta^{\{r,r+1\}}$
is equal to $\theta$. By the first point, the simplex $y$ belongs
to $\cal N_{\leq\lambda}$ for some $\lambda$, and hence so must
$x$.
\end{itemize}
This completes the proof of ($\ast$).

Next we prove ($\ast\ast$). First we show that the maps $K\to\cal N_{\leq\lambda}$
and $K_{0}\to\cal N_{<\lambda}$ are well-defined. A typical simplex
in the image of the map $K\to\cal M^{\t}$ has the form $x:\Delta^{r}\to\cal M^{\t}$,
where there is an integer $-1\leq r'\leq r$ such that $x\vert\Delta^{\{0,\dots,r'\}}$
factors through $\tau_{\lambda}$ and, if $r'<r$, then $p\pr{x\vert\Delta^{\{r',r'+1\}}}$
is inert and and $p\pr{x\vert\Delta^{\{r'+1,\dots,r\}}}$ is the constant
map at $\pr{\inp 1,n}$. (When $r'=-1$, the symbol $\Delta^{\{0,\dots,r'\}}$
denotes the empty simplicial set.) By choosing an inert map $x\pr{r'}\to X$
with $X\in\cal M$ if $r'<r$, or by factoring the map $x\pr{r'}\to x\pr{r'+1}$
into an inert map followed by an active map, we can find a simplex
$y$ belonging to $\cal N_{\leq\lambda}$ and satisfying $d_{r'+1}y=x$.
Hence $x$ belongs to $\cal N_{\leq\lambda}$, showing that the map
$K\to\cal N_{\leq\lambda}$ is well-defined. Next, for the map $K_{0}\to\cal N_{<\lambda}$,
assume further $x\vert\Delta^{\{0,\dots,r'\}}$ factors through the
boundary of $\tau_{\lambda}$. We must show that $x$ belongs to $\cal N_{<\lambda}$.
We have two cases to consider:
\begin{itemize}
\item Suppose that the head of the simplex $p\pr{x\vert\Delta^{\{0,\dots,r'\}}}$
factors through $\partial\sigma'_{a}$. Then the head of $p\pr x$
factors through $d_{i}\sigma$ for some $0\leq i<m$ for some associate
$\sigma$ of $\sigma'_{a}$, so $x$ belongs to $\cal M_{F_{<a}}^{\t}$
by ($\blacklozenge$). 
\item Suppose that the head of the simplex $p\pr{x\vert\Delta^{\{0,\dots,r'\}}}$
is a degeneration of $\sigma'_{a}$. Write $x\vert\Delta^{\{0,\dots,r'\}}=s^{*}y$,
where $y$ is nondegenerate and $s$ is a surjection. Eilenberg-Zilber's
lemma implies that the simplex $p\pr y$ is again a degeneration of
$\sigma'_{a}$, so $y=\tau_{\mu}$ for some $\mu\in\Lambda$. Since
$x$ factors through the boundary of $\tau_{\lambda}$, the dimension
of $y$ must be smaller than that of $\tau_{\lambda}$. Thus $\mu<\lambda$.
The above argument (that the map $K\to\cal N_{\leq\lambda}$ is well-defined)
shows that $y$ belongs to $\cal N_{\leq\mu}\subset\cal N_{<\lambda}$,
so $x$ belongs to $\cal N_{<\lambda}$.
\end{itemize}
This completes the verification of the first half of ($\ast\ast$).
We next proceed to the latter half: We show that the square % https://q.uiver.app/?q=WzAsNCxbMCwwLCJLXzAiXSxbMCwxLCJLIl0sWzEsMCwiXFxtYXRoY2Fse059X3s8XFxsYW1iZGF9Il0sWzEsMSwiXFxtYXRoY2Fse059X3tcXGxlcSBcXGxhbWJkYX0iXSxbMCwxXSxbMCwyXSxbMSwzXSxbMiwzXV0=
\[\begin{tikzcd}
	{K_0} & {\mathcal{N}_{<\lambda}} \\
	K & {\mathcal{N}_{\leq \lambda}}
	\arrow[from=1-1, to=2-1]
	\arrow[from=1-1, to=1-2]
	\arrow[from=2-1, to=2-2]
	\arrow[from=1-2, to=2-2]
\end{tikzcd}\]is cocartesian. Clearly $\cal N_{\leq\lambda}$ is the union of the
images of $K$ and $\cal N_{<\lambda}$. So it suffices to show that
if a simplex $z$ of $K$ is mapped into $\cal N_{<\lambda}$, then
$z$ belongs to $K_{0}$. Taking the contrapositive, we will show
that if $z$ does not belong to $K_{0}$, then its image in $\cal N_{\leq\lambda}$
does not belong to $\cal N_{<\lambda}$. Let $x:\Delta^{r}\to\cal N_{\leq\lambda}$
be the image of $x$. By the definition of the map $K\to\cal N_{\leq\lambda}$
and by the hypothesis that $z$ does not belong to $K_{0}$, there
is an integer $0\leq r'\leq r$ such that $x\vert\Delta^{\{0,\dots,r'\}}$
is a degeneration of $\tau_{\lambda}$ and, if $r'<r$, then $p\pr{x\vert\Delta^{\{r',r'+1\}}}$
is inert and and $p\pr{x\vert\Delta^{\{r'+1,\dots,r\}}}$ is the constant
map at $\pr{\inp 1,n}$. The head of $p\pr x$ is thus a degeneration
of either $\sigma'_{a}$ or its associate, and so it does not belong
to $F_{<a}$ by ($\blacklozenge\blacklozenge$). Therefore, $x$ does
not belong to $\cal M_{F_{<a}}^{\t}$. So should $x$ belong to $\cal N_{<\lambda}$,
then $x\vert\Delta^{\{0,\dots,r'\}}$ must factor through $\tau_{\mu}$
for some $\mu<\lambda$. This is impossible because $x\vert\Delta^{\{0,\dots,r'\}}$
is a degeneration of $\tau_{\lambda}$. Hence $x$ does not belong
to $\cal N_{<\lambda}$.

\item Suppose that $\sigma'_{a}$ belongs to $G'_{\pr 3}$. We will
show that the inclusion $\cal M_{<a}^{\t}\hookrightarrow\cal M_{\leq a}^{\t}$
is a weak categorical equivalence. The desired extension of $f_{<a}$
can then be found because the map $q$ is a categorical fibration.

Let $\sigma$ be the unique associate of $\sigma'_{a}$, which we
depict as 
\[
\pr{\inp{k_{0}},e_{0}}\xrightarrow{\alpha_{\sigma}\pr 1}\cdots\xrightarrow{\alpha_{\sigma}\pr m}\pr{\inp{k_{m}},e_{m}}.
\]
Choose integers $0<j<k\leq m$ such that $\alpha_{\sigma}\pr i$ is
strongly inert for $i=j$, active for $j<i\leq k$, and strongly inert
for $i>k$. Set $Y=\pr{N\pr{\Fin_{\ast}}\times\Delta^{n}}_{/\sigma}\times_{\Delta^{n}}\{0\}$.
Using ($\blacklozenge$), we may consider the following commutative
diagram: % https://q.uiver.app/?q=WzAsNCxbMCwwLCJZXFxzdGFyIFxcTGFtYmRhIF5tX2oiXSxbMCwxLCJZXFxzdGFyIFxcRGVsdGEgXm0iXSxbMSwwLCJcXG92ZXJsaW5le0ZfezxhfX0iXSxbMSwxLCJcXG92ZXJsaW5le0Zfe1xcbGVxIGF9fSJdLFswLDFdLFswLDJdLFsxLDNdLFsyLDNdXQ==
\[\begin{tikzcd}
	{Y\star \Lambda ^m_j} & {\overline{F_{<a}}} \\
	{Y\star \Delta ^m} & {\overline{F_{\leq a}}.}
	\arrow[from=1-1, to=2-1]
	\arrow[from=1-1, to=1-2]
	\arrow[from=2-1, to=2-2]
	\arrow[from=1-2, to=2-2]
\end{tikzcd}\]Using ($\blacklozenge\blacklozenge$), we deduce that this square
is cocartesian. Therefore, it suffices to show that the inclusion
\[
\pr{Y\star\Lambda_{j}^{m}}\times_{N\pr{\Fin_{\ast}}\times\Delta^{n}}\cal M^{\t}\to\pr{Y\star\Delta^{m}}\times_{N\pr{\Fin_{\ast}}\times\Delta^{n}}\cal M^{\t}
\]
is a weak categorical equivalence. We will prove more generally that
for any morphism $Y'\to Y$ of simplicial sets, the map 
\[
\eta_{Y'}:\pr{Y'\star\Lambda_{j}^{m}}\times_{N\pr{\Fin_{\ast}}\times\Delta^{n}}\cal M^{\t}\to\pr{Y'\star\Delta^{m}}\times_{N\pr{\Fin_{\ast}}\times\Delta^{n}}\cal M^{\t}
\]
is a weak categorical equivalence. The assignment $Y'\mapsto\eta_{Y'}$
defines a functor from $\SS/Y'$ to the arrow category $\SS^{[1]}$
which commutes with filtered colimits. Since weak categorical equivalences
are stable under filtered colimits, we may assume that $Y'$ is a
finite simplicial set. If $Y'$ is empty, the claim follows from \cite[Lemma 2.4.4.6]{HA}.
For the inductive step, we can find a pushout diagram % https://q.uiver.app/?q=WzAsNCxbMCwwLCJcXHBhcnRpYWxcXERlbHRhXnAiXSxbMSwwLCJYIl0sWzEsMSwiWSciXSxbMCwxLCJcXERlbHRhXnAiXSxbMCwxXSxbMSwyXSxbMCwzXSxbMywyXV0=
\[\begin{tikzcd}
	{\partial\Delta^p} & X \\
	{\Delta^p} & {Y'}
	\arrow[from=1-1, to=1-2]
	\arrow[from=1-2, to=2-2]
	\arrow[from=1-1, to=2-1]
	\arrow[from=2-1, to=2-2]
\end{tikzcd}\]in $\SS/Y$, such that the claim holds for $X$. The map $\eta_{Y'}$
factors as
\begin{align*}
\pr{Y'\star\Lambda_{j}^{m}}\times_{N\pr{\Fin_{\ast}}\times\Delta^{n}}\cal M^{\t} & \xrightarrow{\phi}\pr{Y'\star\Lambda_{j}^{m}\cup X\star\Delta^{m}}\times_{N\pr{\Fin_{\ast}}\times\Delta^{n}}\cal M^{\t}\\
 & \xrightarrow{\psi}\pr{Y'\star\Delta^{m}}\times_{N\pr{\Fin_{\ast}}\times\Delta^{n}}\cal M^{\t}.
\end{align*}
The map $\phi$ is a pushout of $\eta_{X}$, and hence is a trivial
cofibration. The map $\psi$ is a pushout of the inclusion
\[
\psi':\pr{\Delta^{p}\star\Lambda_{j}^{m}\cup\partial\Delta^{p}\star\Delta^{m}}\times_{N\pr{\Fin_{\ast}}\times\Delta^{n}}\cal M^{\t}\to\pr{\Delta^{p}\star\Delta^{m}}\times_{N\pr{\Fin_{\ast}}\times\Delta^{n}}\cal M^{\t}.
\]
Using the isomorphism $\Delta^{p}\star\Lambda_{j}^{m}\cup\partial\Delta^{p}\star\Delta^{m}\cong\Lambda_{p+1+j}^{m+p+1}$
and Lemma \cite[Lemma 2.4.4.6]{HA}, we deduce that $\psi'$ is a
weak categorical equivalence. Hence $\psi$ is a weak categorical
equivalence, completing the treatment of Case 2.

\end{enumerate}

It remains to construct a well-ordering on $A$ which satisfies ($\blacklozenge$).
For each $a\in A$, define integers $u_{\mathrm{neut}}\pr a,u_{\act}\pr a,u_{\mathrm{oc}}\pr a,u_{\mathrm{as}}\pr a$
as follows: Let $\sigma$ be an associate of $\sigma'_{a}$, and set
$\alpha_{\sigma}\pr i=\sigma\vert\Delta^{\{i-1,i\}}$ for $1\leq i\leq m$.
Then:
\begin{itemize}
\item $u_{\mathrm{neut}}\pr a$ is the number of integers $1\leq i\leq m$
such that $\alpha_{\sigma}\pr i$ is neutral.
\item $u_{\act}\pr a$ is the number of integers $1\leq i\leq m$ such that
$\alpha_{\sigma}\pr i$ is active.
\item $u_{\mathrm{oc}}\pr a$ is set equal to $0$ if $\sigma$ is closed,
and is set equal to $1$ if $\sigma$ is open.
\item $u_{\mathrm{as}}\pr a$ is the number of pairs of integers $1\leq i<j\leq m$
such that $\alpha_{\sigma}\pr i$ is active and $\alpha_{\sigma}\pr j$
is strictly inert.
\end{itemize}
We choose a well-ordering on $A$ so that the function 
\[
A\ni a\mapsto\pr{u_{\mathrm{neut}}\pr a,u_{\act}\pr a,u_{\mathrm{oc}}\pr a,u_{\mathrm{as}}\pr a}\in\bb Z^{4}
\]
is non-decreasing, where $\bb Z^{4}$ is equipped with the lexicographic
ordering. (Thus $\pr{n_{1},n_{2},n_{3},n_{4}}<\pr{n'_{1},n'_{2},n'_{3},n'_{4}}$
if and only if $n_{i}<n'_{i}$, where $i$ is the minimal integer
for which $n_{i}\neq n'_{i}$.) We will show that this ordering does
the job.

Let $a,\sigma,i$ be as in ($\blacklozenge$). We wish to show that
$d_{i}\sigma$ belongs to $F_{<a}$. If $d_{i}\sigma$ is incomplete
or degenerate, then it belongs to $F\pr{m-1}$ and we are done. So
assume that $d_{i}\sigma$ is complete and nondegenerate. Then $d_{i}\sigma$
belongs to (exactly) one of the sets $G_{\pr 1},G_{\pr 2},G'_{\pr 2},G_{\pr 3},G'_{\pr 3}$.
If $d_{i}\sigma$ belongs to $G_{\pr 1}\cup G_{\pr 2}\cup G_{\pr 3}$,
or if it belongs to $G'_{\pr 2}$ and has no associates, then it belongs
to $F''\pr m$ and we are done. So we will assume that $d_{i}\sigma$
belongs to $G'_{\pr 2}\cup G'_{\pr 3}$ and has an associate, so that
$d_{i}\sigma=\sigma'_{b}$ for some $b\in A$. We wish to show that
$b<a$.

We will make use of the following notations. Let $\alpha_{\sigma}\pr i=\sigma\vert\Delta^{\{i-1,i\}}$.
Let $0\leq k\leq m$ be the minimal integer such that $\alpha_{\sigma}\pr i$
is strongly inert for every $i>k$, and let $0\leq j\leq k$ be the
minimal integer such that $\alpha_{\sigma}\pr i$ is active for every
$j<i\leq k$. 

Suppose first that $\sigma'_{a}\in G'_{\pr 2}$. Our assumption on
$d_{i}\sigma$ implies that $i=k$ and that the composite $\alpha_{\sigma}\pr{k+1}\circ\alpha_{\sigma}\pr k$
is neutral. It follows that the associate $\tau$ of $d_{i}\sigma$
satisfies $\alpha_{\tau}\pr s=\alpha_{\sigma}\pr s$ for $s\neq k,k+1$,
$\alpha_{\tau}\pr i$ is strictly inert, and $\alpha_{\tau}\pr{k+1}$
is active. Thus $u_{\mathrm{neut}}\pr a=u_{\mathrm{neut}}\pr b$,
$u_{\act}\pr a=u_{\act}\pr b$, $u_{\mathrm{oc}}\pr a=u_{\mathrm{oc}}\pr b$,
and $u_{\mathrm{as}}\pr a>u_{\mathrm{as}}\pr b$. Hence $a>b$, as
required.

Suppose next that $\sigma'_{a}\in G'_{\pr 3}$ and that $d_{i}\sigma\in G'_{\pr 2}$. 
\begin{itemize}
\item If $u_{\mathrm{neut}}\pr a>0$, we are done, since $u_{\mathrm{neut}}\pr b=0$.
\item If $u_{\mathrm{neut}}\pr a=0$, then each $\alpha_{\sigma}\pr i$
is either active or strongly inert. It follows that $u_{\act}\pr a$
is not less than the number of active morphisms in $d_{i}\sigma$,
which is equal to $u_{\act}\pr b$. So $u_{\act}\pr a\geq u_{\act}\pr b$.
If $u_{\act}\pr a>u_{\act}\pr b$, we are done. 
\item If $u_{\mathrm{neut}}\pr a=0$ and $u_{\act}\pr a=u_{\act}\pr b$,
then $\sigma$ is open. Indeed, if $\sigma$ were closed, then $i=m$
since $d_{i}\sigma$ is open. But then $d_{i}\sigma$ must contain
a subsequence consisting of a strictly inert morphism followed by
an active morphism. This is impossible because $d_{i}\sigma$ belongs
to $G'_{\pr 2}$. Hence $u_{\mathrm{oc}}\pr a>u_{\mathrm{oc}}\pr b$
and we are done.
\end{itemize}
Finally, suppose that $\sigma'_{a}\in G'_{\pr 3}$ and that $d_{i}\sigma\in G'_{\pr 3}$.
Note that our assumption on $d_{i}\sigma$ forces $i=j-1$ or $i=k$.
\begin{itemize}
\item The number of neutral morphisms in $d_{i}\sigma$ is at most $u_{\mathrm{neut}}\pr a+1$,
so 
\begin{equation}
u_{\mathrm{neut}}\pr a\geq\#\{\text{neutral morphisms in }d_{i}\sigma\}-1=u_{\mathrm{neut}}\pr b.\label{eq:1}
\end{equation}
If $u_{\mathrm{neut}}\pr a>u_{\mathrm{neut}}\pr b$, we are done. 
\item Suppose $u_{\mathrm{neut}}\pr a=u_{\mathrm{neut}}\pr b$, so that
the equality holds in (\ref{eq:1}). Then $0<i<m$, the composite
$\alpha_{\sigma}\pr{i+1}\circ\alpha_{\sigma}\pr i$ is neutral, and
$\alpha_{\sigma}\pr i$ is active. ($\alpha_{\sigma}\pr i$ is necessarily
strictly inert since $i=j-1$ or $i=k$.) It follows that the associate
$\tau$ of $d_{i}\sigma$ satisfies $\alpha_{\tau}\pr s=\alpha_{\sigma}\pr s$
for $s\neq i,i+1$, $\alpha_{\tau}\pr i$ is strictly inert, and $\alpha_{\tau}\pr{i+1}$
is active. Thus $u_{\mathrm{neut}}\pr a=u_{\mathrm{neut}}\pr b$,
$u_{\act}\pr a=u_{\act}\pr b$, $u_{\mathrm{oc}}\pr a=u_{\mathrm{oc}}\pr b$,
and $u_{\mathrm{as}}\pr a>u_{\mathrm{as}}\pr b$. Hence $a>b$, as
desired.
\end{itemize}
This completes the proof of ($\blacklozenge$) and hence the proof
of the theorem.

\end{enumerate}
\end{proof}
\input{HA_3.1.2.3_v2.bbl}

\end{document}


\end{document}


\end{document}


\end{document}
