\section{Introduction}
\label{sec:intro}

\subsection{Background and Motivation}
It had become common sense that bot accounts flooded almost every popular social network platform. Many research works have realized these issues and made solutions like Twitter \cite{cresci2017paradigm}, Facebook \cite{santia2019detecting}, and Reddit \cite{hurtado2019bot}. Along with the breaking out social bot population, user-experience of normal users deteriorate quickly due to malicious bot direct or indirect disturb, e.g., fake news \cite{heidari2021bert}, spam \cite{rodrigues2022real}, rumor \cite{huang2022social}, and misinformation\cite{himelein2021bots}. Furthermore, it will finally push platforms to improve a large amount of maintenance and risk control costs or otherwise suffer from user loss. Therefore, designing highly accurate and easily deployed bot detection systems is substantial for research and industry.


\subsection{Bot Detection Approaches}
\label{subsec:related}
Recent social bot detection works could be categorized into three strategies: account-based, content-based, and graph-based. 

\textbf{Account-based} approach mainly focuses on mining risk signatures as features from account meta-data and statistical indicators for machine learning algorithms to do classification. Kai-Cheng {\em et al.} \cite{yang2020scalable} extracted two types of features, including raw data like follower count and derived features like follower growth rate and the length of the screen name. Saleh {\em et al.} \cite{ahmad2021spam} summarized a bunch of features according to account information like age, length of the name, and count of followers and then used a support vector machine to separate bots from genuine users. Hrushikesh {\em et al.} \cite{shukla2021enhanced} ensemble the prediction results of three machine learning models to improve the classifying accuracy after a comprehensive account features collection, e.g., location, profile image, and daily average tweet count, and so on. Maryam {\em et al.} \cite{heidari2022online} utilized GloVe \cite{pennington2014glove} to create account embedding based on age, gender, education, and personality from meta-data, which firstly proposes the user embedding concept. 


\textbf{Content-based} method systems shift the spotlight to public texts posted by users, like tweets,  and apply natural language processing techniques to mine risk words or embedding semantics. The primary assumption of those works is that content posted by bots tends to uncover their fraud intentions,  and various intentions are clusters in high-dimensional language embedding space. Anisha {\em et al.} \cite{rodrigues2022real} applies the TF-IDF and Bag of Words technique to generate text features for the downstream machine learning model to train and predict. BGSRD \cite{guo2021social} combined Bert \cite{devlin2018bert} and GCN(Graph Convolutional Networks) \cite{kipf2016semi} algorithm to jointly learn representation from multiple historical tweets of each account for bot classification tasks. DeepSBD \cite{fazil2021deepsbd} learned text representation based on historical tweets and mixed content embedding with account features via joint representing. Maryam {\em et al.} \cite{heidari2021bert} also uses BERT to generate text embedding from tweets, showing a significant performance in detecting fake news about the COVID-19 topic. 

\textbf{Graph-based} approach gets popular in the bot detection domain after the rapid evolution of the graph representing techniques. After 2016, newly proposed algorithms GraphSAGE \cite{hamilton2017inductive} and GCN \cite{kipf2016semi} perform significantly among various network relationships in the real world, like social networks, online shopping, and citation map. The graph-based approach reuses useful account-based and content-based features by transforming them into node embedding. Moreover, it explores the optimal network topology to share and transfer node information among neighbors, outperforming traditional methods. Some recent works show a start-of-art accuracy by designing appropriate graph structures and node features. Seyed {\em et al.} \cite{ali2019detect} firstly apply graph convolutional neural networks to learn one node's representation based on account features of itself and its neighbors. Shangbin {\em et al.} \cite{feng2021botrgcn} applied GCN algorithm on the user following relationship graph, then represented raw node features including user profile, categorical and numerical data of account activity. Shangbin {\em et al.} \cite{feng2022heterogeneity} constructs two kinds of heterogeneity structures, including relation and influence, leveraging the topology to identify the difference between genuine users and social bots. 


\subsection{Existing Problems}
\label{subsec:prob}
Most \emph{account-based works} focus on evaluating the effects of different machine learning models without a deep digging into the behavior patterns of bots. 
\textbf{The first problem} is that most works tend to input coarse-grain statistical indicators as features into models without controlling variables. For example, the count of followers is an essential feature in many bot detection models. However, the count of followers of one account will naturally increase and accumulate day-to-day after registering. Engineers usually calculate the indicators on the day of building models. However, the registering dates of accounts are different. As a result, these calibrated features become noisy data to prevent the classifier from making an accurate prediction. In short,  adequate log data is not utilized sufficiently in this way. 
\textbf{The second problem} is that most works consider bots isolated rather than gangs of attackers. Consequently, little research is focusing on discovering behavior similarity among bot accounts.
In \emph{graph-based methods}, accounts will exchange node information with neighbors. However, most focus on finding a better network topology rather than exploring behavioral node features.

\subsection{Our Contributions} 
To summarize, this paper makes the following contributions:
\begin{itemize}
\item We first propose a novel type of feature to profile a social account's behavioral pattern, supported by a solid measurement work to observe distribution divergence between genuine users and bot accounts. Based on behavioral sequences, we adopt proper algorithms to catch their significant patterns for prediction. 

\item We design a bot detection system \BotShape integrating an automatic behavioral feature generation process and implementing a complete pipeline log processing, feature engineering, and prediction. Input parameters are simplified into a minimal range to improve the automation degree of the whole process. 

\item \BotShape performs high accuracy, with an average accuracy of 98.52\% and an average f1-score of 96.65\% in evaluating experiments across four classifiers and three ground truths. It presents a steady performance improvement compared to \emph{account-based features} and also shows the crucial role of extracting\emph{behavioral patterns}. \BotShape is easy to combine with other bot detection systems by providing compelling behavioral features. For example, \emph{graph-based} approach could use behavioral features as the node features and 
spread pattern information into the whole network.

\end{itemize}



  
