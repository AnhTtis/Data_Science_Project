
\section{Dataset}
\label{sec:dataset}

\begin{table*}[h]
\centering
\caption{Key attributes of account and tweeting data in \Cresci data set}
\label{tab:attributes}
\vspace{-6pt}
\renewcommand{\arraystretch}{1.1}
\small
\begin{tabular}{|l|p{4.5in}|}
\hline
{\bf Attribute} & {\bf Descriptions}\\
\hline
User ID & It refers to the unique identifier for each tweet account. \\
\hline
Created At & It referred to the time stamp at the time of the account registration.  \\
\hline
Account Information & There are several essential attributes, such as the nickname of an account, the personal introduction, whether the avatar is the default, and the common location.
\\
\hline
Interaction (user-level) & They refer to the number of accounts that interact with an account, potentially reflecting the social vitality of an account, such as the count of followers, friends, and favorites. \\
\hline
Tweet ID & It refers to the unique identifier for each tweet post. It has a unique id of its author. \\
\hline
Tweet Created At & It is the time stamp of the posting time of a work like a tweet.\\
\hline
Post Content & It refers to the raw text or voice of a post.\\
\hline
Interaction (tweet-level) & They are statistics of re-tweeting and replying to a post, reflecting its popularity. \\
\hline
\end{tabular}
\end{table*}
Cresci {\em et al.} \cite{cresci2017paradigm} published a data set of Twitter (called \Cresci) of four types of accounts, including genuine users, social bots, traditional users, and fake followers. It collected user information and event logs (mainly tweeting records with timestamps, identifiers, and text) in the real world. The total number of social bots is 4912, consisting of three small data sets collected at different periods (from the easiest tweet publish time to the last one): (i) from March 17, 2009, to May 26, 2014 (ii) from September 9, 2008, to March 22, 2014 (iii) from September 14, 2008, to April 11, 2014. The data group of genuine users begins on January 22, 2007, and ends on April 20, 2015, with 3474 accounts in total (only 1083 of them have tweeting logs). There are two extra spam account data sets with complete tweeting logs: (i) traditional spam bots: from July 4, 2007, to March 8, 2010, with 1000 accounts that have tweeting logs (ii) fake followers: from December 7, 2007, to April 30, 2013, with 3351 accounts. The collected tweets come across a very long period, sufficient to observe accounts' early and long-term behavior. \Cresci supports abundant real-world user and tweet data for analysis and modeling. Based on domain knowledge, we highlight important ones and summarize them in Table~\ref{tab:attributes}.
