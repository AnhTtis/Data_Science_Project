
\section{Design}

We present \BotShape, a social bots detection framework. It takes account logs as input, does an automatic behavioral feature-generating process, and then detects social bots based on machine learning classifiers. The feature engineering process has two steps: (i) extract behavioral sequences from account registration logs and event logs in various time intervals. (ii) compress raw behavior sequences time series to seasonality attributes and distinctive shapelets features \cite{ye2009time}. Behavioral features are actual multiple numerical vectors compatible with many machine learning classifiers. 

\subsection{Main Idea}


\subsubsection{\textbf{How to detect bots at an earlier time?}}Figure~\ref{fig:arch} shows the \BotShape architecture, where each module is reasonable and has a corresponding supporting point in section~\ref{sec:measure}. \BotShape takes full advantage of the raw account event logs to discover the most compelling features for a classifier to distinguish bot accounts and user accounts. Different from \textbf{account-based} method, taking indicators accumulated for a more extended period as input, \BotShape could be deployed anytime after the registration due to its fine-grain observation view.


\subsubsection{\textbf{How to compress dimensionality and keep useful features?}} 
Extracting the best feature set is not trivial because the original features constructed by setting various time-related parameters are high-dimensional, leading to the \emph{Curse of Dimensionality} problem \cite{donoho2000high}. In other words, it will not give rise to a high detection accuracy if only piling up a mass of features. Therefore, it is necessary to mine prominent features from many raw behavioral series for the downriver model to make prediction. 

%解决方法: compress + two step(automatiion)
\subsubsection{\textbf{How to design an automatic system?}} We assess the automation degree of a  system based on the complexity of the manual parameter-setting process. Namely, it is highly automotive if a system could directly produce compelling features after setting a few parameters without too many attempts. We design \BotShape according to this principle. \BotShape applies a two-step separate feature engineering process with flexible parameter management, decoupling features election from raw feature production. The first step is generating a batch of statistical behavior sequences via different parameter settings such as time interval, period, and statistic functions. The second step focuses on mining the critical fluctuation points of those sequences based on the Shapelets method, which vastly reduces the dimensionality of feature vectors. Users could tune parameters respectively and flexibly for the two parts.


\begin{figure}[htb]
\centering
\includegraphics[width=\linewidth]{figs/framework.pdf}
\vspace{-6pt}
\caption{\BotShape architecture.}
\label{fig:arch}
\end{figure}
\label{sec:design}


\subsection{Behavior Sequence Features}

\label{subsec:design_bs}

\begin{algorithm}[t]
	\label{alg:bhv}
    \caption{Generate Behavioral Sequences (Time Series)}
    \begin{algorithmic}[1]
    \Require {\\ \textbf{Registration Logs:} $Log_{reg} = (id_{user}, t_{reg})$ \\
     \textbf{Event Logs:} $Log_{bhv} = \{(id, t_1), (id, t_2),..., (id, t_n)\}$ \\
     \textbf{Parameters:} $dur$, $gran$}
    \Ensure \emph{Behavioral Time-series} $ts$
    	
    \Function {$gen\_bhv\_sequence(Log_{bhv}, t_{reg}, dur, gran)$}{}
    	\State {$win = floor(\frac{dur}{gran})$}
    	\State {$ts =$ \textbf{new} $int[win]$}
    	\For {$k = 1, ....win $}
    		\State {$T_{st} = gran * (k-1)$}
    		\State {$T_{ed} = gran * k$}
    		\State {$cnt = 0$}
    		\For {$i=1, 2, ..., n$}
    			\If {$t_i - t_{reg} > T_{st}$ \textbf{and} $t_i - t_{reg} <= T_{ed}$} {$cnt = cnt + 1$}
    			\EndIf
    		\EndFor
    		\State {$ts[k] = cnt$}
    	\EndFor
    	\State{\textbf{return} $ts$}
    \EndFunction
    \end{algorithmic}
\end{algorithm}

We now elucidate the first step of the feature engineering process about generating behavior sequences using event logs $Log_{bhv}$ and registration information $Log_{reg}$. We format behavior sequences as multiple time series data. The program extracts behavioral time series from raw logs based on two time-related parameters. Parameter $dur$ is the duration from the registration to the computation time. Parameter $gran$ refers to the time granularity, such as day, week, and month, which is the window size for calculating statistical features like the number of tweets.

 Pseudo-code~\ref{alg:bhv} introduces how to generate behavioral sequences in a time series format. It takes two kinds of data as input, including the registration time stamp of an account and its event logs. Event logs could be any action on the social platform, like FOLLOW, SHARE, POST, and EDIT PROFILE. The computing is individual for each account, so this process can be deployed paralleled and distributed. In the generation process, it firstly computes the count of windows $win$, equaling $dur$ dividing $gran$, then it scans each time window to calculate the behavioral statistic like the count of tweets. In each round, the program sets the start time $T_{st}$ and end time of $T_{ed}$ each window for precisely distributing each $Log_{bhv}$ item to its time window. \BotShape outputs the sequences of behavioral statistics in chronological order, forming a time series. Engineers could generate a bunch of time series by setting various function parameters.
 
\subsection{Behavior Pattern Features}
\label{subsec:design_ts}
\subsubsection{\textbf{Seasonality Decomposition}}
In a time series data, \emph{seasonality} is the periodic fluctuation correlated strongly to the time attribute. Some time series data could perform a seasonal variation because the related entity changes regularly over time. For example, the number of online users in a network correlates with the working and sleeping time (busy or idle state) of people \cite{wu2018cellpad}, so the time series indicator could perform a repeat and similar pattern. Plus, social platforms for online activity are highly affected by user usage habits and tend to present hourly, daily, and weekly seasonality. Also, in sub section~\ref{subsec:busy_idle}, we compare the behavioral seasonality distributions of bots and genuine users, hourly and daily, respectively, showing an evident divergence.

Assuming that classifiers could separate bots from genuine users after exploring the seasonality of behavioral sequences, \BotShape owns a continued process of seasonality extraction. In detail, After generating a time series, \BotShape continues to compute the seasonality, which is still a sequence where each item is the mean value of corresponding elements. For example, for extracting the weekly seasonality extraction from a day-gran time series, \BotShape firstly queries the day of the week of each time-stamp and then allocates them to seven groups (Sunday, Monday, ..., Saturday), finally averages each group and organizes them into a new sequence in order.


\subsubsection{\textbf{Shapelets Representation}}
Shapelets \cite{ye2009time} are subsequences of a time series that are prominently distinct characteristics of its class. The Shapelets algorithm performs outstandingly in time series classification problems compared to raw data. The algorithm targets finding the best splitting strategy for maximizing the information gain (difference between entropy before and after the splitting). 

In sub section~\ref{subsec:busy_idle}, we observe an apparent similar time series shape in six clusters of bot behavioral sequences. If \BotShape could represent the pattern correctly, it would vastly improve the detection accuracy for bots. Therefore, \BotShape also takes continued Shapelet mining after constructing behavioral sequences. It applies a python package called tslearn \cite{tslearn} to learn shapelets. It learns from the raw time series and their labels from the train set to discover the best splitting points for the most accurate classification and then extracts the shapelets of the time series in the test set via the same splitting manner. In the default setting, \BotShape extracts shapelets from weekly and monthly behavioral sequences as new features, with a default subsequence length setting of 30\% of sequence length for weekly and 50\% for monthly.

\subsection{Bot Detection}
Behavioral data constructed by \BotShape, including behavioral sequence, seasonality, and shapelets, are potential features for machine learning classifiers to fit and predict. We emphasize that \BotShape focuses on behavioral feature engineering, providing extra general bot features, rather than \textbf{account-based} attributes. To evaluate its general utility, \BotShape integrates multiple prediction algorithms instead of fixing the classifier. We evaluate prediction accuracy in sub-section~\ref {subsec:eval_features} shows behavioral features all perform a very high detection accuracy when using different classifiers to predict. Eventually, secondary time series features (seasonality plus shapelets) perform better.
