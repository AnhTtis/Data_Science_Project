% \documentclass{article}
% \usepackage[utf8]{inputenc}
% \usepackage{amssymb}
% \usepackage{srcltx}
% \usepackage{amsmath}
% \usepackage{CJK}
% \usepackage{graphicx}
% \usepackage{epstopdf}
% \usepackage{dcolumn}
% \usepackage{bm}
% \usepackage{multicol}
% \usepackage{float}
% \usepackage{algorithm}
% \usepackage{algorithmic}
% \usepackage{MnSymbol}
% \newtheorem{theorem}{Theorem}
% \newtheorem{lemma}{Lemma}
% \newtheorem{corollary}{Corollary}
% \newtheorem{proof}{Proof}[section]
% \newtheorem{proposition}{Proposition}
% \newcommand{\holderspace}[3]{\mathcal{W}^{#1}_{#2}(#3)}
% \begin{document}
\subsection{Preliminary}
\paragraph{Additional definitions} The theory of empirical processes \citep{van1996weak} will be involved heavily in the proof. Therefore we briefly introduce some common notations: For a function class $\mathcal{F}$, define $\cn{\epsilon}{\mathcal{F}}{\|\cdot\|}$ to be the covering number of $\mathcal{F}$ with respect to norm $\|\cdot\|$ under radius $\epsilon$, and define $\bn{\epsilon}{\mathcal{F}}{\|\cdot\|}$ to be the bracketing number of $\mathcal{F}$ with respect to norm $\|\cdot\|$ under radius $\epsilon$. We use $\vcdim{\mathcal{F}}$ to denote the VC-dimension of $\mathcal{F}$. Moreover, we use the notation $a \lesssim b$ to denote $a \le C b$ for some positive constant $C$.\par 
Before proving theorem \ref{thm: rate_pf} and \ref{thm: rate_fn},
% we first illustrate the relationship between our learning objective and the defined metric.
we introduce some additional notations that will be useful throughout the proof process. \par

In the PF scheme, define
\begin{align*}
    l(T,\delta,Z; h, m,\theta) =& \delta \log g_{\theta}\left(e^{m(Z)}\int_{0}^{T}e^{h(s)}ds\right)+\delta h(T)+\delta m(Z)\\
    &-G_{\theta}\left(e^{m(Z)}\int_{0}^{T}e^{h(s)}ds\right),
\end{align*}
where we denote $g_\theta = G^\prime(\theta)$. Under the definition of the sieve space stated in condition \ref{cond: sieve_PF}, we restate the parameter estimates as
\begin{align*}
    \left(\widehat{h}_{n},\widehat{m}_{n},\widehat{\theta}_{n}\right) = \mathop{\mathrm{argmax}}\limits_{
    \widehat{h} \in \mathcal{H}_n,\widehat{m} \in \mathcal{M}_n,\theta \in \Theta
    }\frac{1}{n}\sum_{i\in [n]}l(T_{i},\delta_{i},Z_{i};\widehat{h},\widehat{m},\theta).
\end{align*}
% The estimated function $\widehat{h_{n}}(t),\widehat{m}_{n}(z)$ and the estimated frailty parameter $\widehat{\theta}_{n}$ are obtained through
% \begin{equation*}
% \left(\widehat{h}_{n}(t),\widehat{m}_{n}(z),\widehat{\theta}_{n}\right) = \mathop{\mathrm{argmax}}\limits_{\widehat{h}(t;\mathbf{W^{h}},\mathbf{b^{h}}),\widehat{m}(z;\mathbf{W^{m}},\mathbf{b^{m}}),\theta}\frac{1}{n}\sum_{i\in [n]}l(T_{i},\delta_{i},Z_{i};\widehat{h}(t;\mathbf{W^{h}},\mathbf{b^{h}}),\widehat{m}(z;\mathbf{W^{m}},\mathbf{b^{m}}),\theta).
% \end{equation*}

Similarly, in the FN scheme, we define 
\begin{align*}
    l(T,\delta,Z; \nu,\theta) = \delta \log g_{\theta}\left(\int_{0}^{T} e^{\nu(s,Z)}ds\right)+\delta \nu(T,Z) - G_{\theta}\left(\int_{0}^{T} e^{\nu(s,Z)}ds\right)
\end{align*}
Under the definition of the sieve space stated in condition \ref{cond: sieve_FN}, we restate the parameter estimates as
\begin{align*}
    \left(\widehat{\nu}_{n}(t,z), \widehat{\theta}_{n}\right) = \mathop{\mathrm{argmax}}\limits_{
    \widehat{\nu} \in \mathcal{V}_n,\theta \in \Theta
    }\frac{1}{n}\sum_{i \in [n]}l(T_{i},\delta_{i},Z_{i};\widehat{\nu},\theta).
\end{align*}
% The estimated function $\widehat{\nu}_{n}(t,z)$ and the estimated frailty parameter $\widehat{\theta}_{n}$ are obtained through
% \begin{equation*}
% \left(\widehat{\nu}_{n}(t,z), \widehat{\theta}_{n}\right) = \mathop{\mathrm{argmax}}\limits_{\widehat{\nu}(t,z;\mathbf{W},\mathbf{b}),\theta}\frac{1}{n}\sum_{i \in [n]}l(T_{i},\delta_{i},Z_{i};\widehat{\nu}(t,z;\mathbf{W},\mathbf{b}),\theta).
% \end{equation*}

We denote the conditional density function and survival function of the event time $\tilde{T}$ given $Z$ by $f_{\tilde{T}\mid Z}(t)$ and $S_{\tilde{T}\mid Z}(t)$, respectively. Similarly, we denote the conditional density function and survival function of the censoring time $C$ given $Z$ by $f_{C\mid Z}(t)$ and $S_{C\mid Z}(t)$. Under the assumption that $\tilde{T} \ind C \mid Z$, the joint conditional density of the observed time $T$ and the censoring indicator $\delta$ given $Z$ can be expressed as the following:
\begin{eqnarray*}
    p(T,\delta \mid Z) &=& f_{\tilde{T}\mid Z}(T)^{\delta}S_{\tilde{T}\mid Z}(T)^{1-\delta}f_{C\mid Z}(T)^{1-\delta}S_{C\mid Z}(T)^{\delta}\\
    &=& \lambda_{\tilde{T}\mid Z}(T)^{\delta}S_{\tilde{T}\mid Z}(T)f_{C\mid Z}(T)^{1-\delta}S_{C\mid Z}(T)^{\delta},
\end{eqnarray*}
where $\lambda_{\tilde{T}\mid Z}(T)$ is the conditional hazard function of the survival time $\tilde{T}$ given $Z$.


Under the model assumption of PF scheme, $p(T,\delta \mid Z)$ can be expressed by
\begin{eqnarray*}
p(T,\delta \mid Z;h,m,\theta) = \exp\left(l(T,\delta,Z;h,m,\theta)\right)f_{C\mid Z}(T)^{1-\delta}S_{C\mid Z}(T)^{\delta}.
\end{eqnarray*}

For $\phi_{0}=(h_{0},m_{0},\theta_{0})$ and an estimator  $\widehat{\phi}=(\widehat{h},\widehat{m},\widehat{\theta})$, the defined distance $d_{\textsf{PF}}\left(\widehat{\phi},\phi_{0}\right)$ can be explicitly expresses by
\begin{eqnarray*}
    d_{\textsf{FN}}\left(\widehat{\psi},\psi_{0}\right) = \sqrt{\mathbb{E}_Z\left[\int\left|\sqrt{p(T,\delta\mid Z;\widehat{h},\widehat{m},\widehat{\theta})}-\sqrt{p(T,\delta\mid Z;h_{0},m_{0},\theta_{0})}\right|^{2}\mu(dT\times d\delta)\right]}.
\end{eqnarray*}
Here the dominating measure $\mu$ is defined such that for any (measurable) function $r(T,\delta)$
\begin{align*}
    \int r(T,\delta)\mu(dT\times d\delta)=\int_{0}^{\tau}r(T,\delta = 1)dT+\int_{0}^{\tau}r(T,\delta=0)dT
\end{align*}
% \wrf{Preciser statements about $\mu$ is desired, it is the product measure of counting measure and Lebesgue measure? My measure theory knowledge is fainting now}

Under the model assumption of FN scheme, $p(T,\delta \mid Z)$ can be expressed by
\begin{eqnarray*}
p(T,\delta \mid Z;\nu,\theta) = \exp\left(l(T,\delta,Z;\nu, \theta)\right)f_{C\mid Z}(T)^{1-\delta}S_{C\mid Z}(T)^{\delta}.
\end{eqnarray*}

For $\psi_{0}=(\nu_{0},\theta_{0})$ and an estimator $\widehat{\psi}=(\widehat{\nu},\widehat{\theta})$, the defined distance $d_{\textsf{FN}}\left(\widehat{\psi},\psi_{0}\right)$ can be explicitly expresses by
\begin{eqnarray*}
    d_{\textsf{FN}}\left(\widehat{\psi},\psi_{0}\right) = \sqrt{\mathbb{E}_Z\left[\int\left|\sqrt{p(T,\delta\mid Z;\widehat{\nu},\widehat{\theta})}-\sqrt{p(T,\delta\mid Z;\nu_{0},\theta_{0})}\right|^{2}\mu(dT\times d\delta)\right]}.
\end{eqnarray*}
% Here for any function $r(T,\delta)$, $\int r(T,\delta)\mu(dT\times d\delta)=\int_{0}^{\tau}r(T,\delta = 1)dT+\int_{0}^{\tau}r(T,\delta=0)dT$.

\subsection{Technical lemmas}
The following lemmas are needed for the proof of Theorem \ref{thm: rate_pf} and \ref{thm: rate_fn}. Hereafter for notational convenience, we will use $\widehat{h}, \widehat{m}$ for arbitrary elements in the corresponding sieve space listed in condition \ref{cond: sieve_PF}, $\widehat{\nu}$ for an arbitrary element in the sieve space listed in condition \ref{cond: sieve_FN}, and $\widehat{\theta}$ for an arbitrary element in $\Theta$.

\begin{lemma}\label{lem: pf_l_bound}
Under condition \ref{cond: param_PF}, \ref{cond: sieve_PF}, \ref{cond: G}, for $(T,\delta,Z)\in [0,\tau]\times\{0,1\}\times[-1,1]^{d}$, the following terms are bounded:
\begin{enumerate}[leftmargin=*]
    \item $l(T,\delta,Z;h_{0},m_{0},\theta_{0})$ with true parameter $(h_{0},m_{0},\theta_{0})$
    \item $l(T,\delta,Z;\widehat{h},\widehat{m},\widehat{\theta})$ with parameter estimates $(\widehat{h},\widehat{m},\widehat{\theta})$ in any sieve space listed in condition \ref{cond: sieve_PF}.
\end{enumerate}
% $l(T,\delta,Z;h_{0},m_{0},\theta_{0})$ and \ $l(T,\delta,Z;\widehat{h},\widehat{m},\widehat{\theta})$ are bounded for $(T,\delta,Z)\in [0,\tau]\times\{0,1\}\times[-1,1]^{d}$. Here $h_{0},m_{0},\theta_{0}$ are the true functions and parameters satisfying condition 1 and $\widehat{h},\widehat{m},\widehat{\theta}$ are arbitrary approximating functions and parameters satisfying condition 3.
\end{lemma}

\begin{lemma}\label{lem: fn_l_bound}
Under condition \ref{cond: param_FN}, \ref{cond: sieve_FN}, \ref{cond: G}, for $(T,\delta,Z)\in [0,\tau]\times\{0,1\}\times[-1,1]^{d}$, the following terms are bounded:
\begin{enumerate}[leftmargin=*]
    \item $l(T,\delta,Z;\nu_{0},\theta_{0})$ with true parameter $(\nu_{0},\theta_{0})$
    \item $l(T,\delta,Z;\widehat{\nu},\widehat{\theta})$ with parameter estimates $(\widehat{\nu},\widehat{\theta})$ in any sieve space listed in condition \ref{cond: sieve_FN}.
\end{enumerate}
% $l(T,\delta,Z;\nu_{0},\theta_{0})$ and \ $l(T,\delta,Z;\widehat{\nu},\widehat{\theta})$ are bounded for $(T,\delta,Z)\in [0,\tau]\times\{0,1\}\times[-1,1]^{d}$. Here $\nu_{0},\theta_{0}$ are the true functions and parameters satisfying condition 2 and $\widehat{\nu},\widehat{\theta}$ are arbitrary approximating functions and parameters satisfying condition 4.
\end{lemma}

\begin{lemma}\label{lem: pf_norm_decomp}
Under condition \ref{cond: param_PF}, \ref{cond: sieve_PF}, \ref{cond: G}, let
% for an arbitrary triple of arbitrary approximating functions and parameters
$(\widehat{h},\widehat{m},\widehat{\theta})$, $(\widehat{h}_{1},\widehat{m}_{1},\widehat{\theta}_{1})$, and $(\widehat{h}_{2},\widehat{m}_{2},\widehat{\theta}_{2})$ be arbitrary three parameter triples inside the sieve space defined in condition \ref{cond: sieve_PF}, 
the following two inequalities hold.
\begin{align*}
    &\|l(T,\delta,Z;h_{0},m_{0},\theta_{0})-l(T,\delta,Z;\widehat{h},\widehat{m},\widehat{\theta})\|_{\infty}\lesssim |\theta_{0}-\widehat{\theta}|+\|h_{0}-\widehat{h}\|_{\infty}+\|m_{0}-\widehat{m}\|_{\infty} \\
    &\|l(T,\delta,Z;\widehat{h}_{1},\widehat{m}_{1},\widehat{\theta}_{1})-l(T,\delta,Z;\widehat{h}_{2},\widehat{m}_{2},\widehat{\theta}_{2})\|_{\infty}\lesssim
    |\widehat{\theta}_{1}-\widehat{\theta}_{2}|+\|\widehat{h}_{1}-\widehat{h}_{2}\|_{\infty}+\|\widehat{m}_{1}-\widehat{m}_{2}\|_{\infty}.
\end{align*}
% \begin{equation*}
% \|l(T,\delta,Z;h_{0},m_{0},\theta_{0})-l(T,\delta,Z;\widehat{h},\widehat{m},\widehat{\theta})\|_{\infty}\lesssim |\theta_{0}-\widehat{\theta}|+\|h_{0}-\widehat{h}\|_{\infty}+\|m_{0}-\widehat{m}\|_{\infty},
% \end{equation*}
% and
% \begin{equation*}
% \|l(T,\delta,Z;\widehat{h}_{1},\widehat{m}_{1},\widehat{\theta}_{1})-l(T,\delta,Z;\widehat{h}_{2},\widehat{m}_{2},\widehat{\theta}_{2})\|_{\infty}\lesssim
% |\widehat{\theta}_{1}-\widehat{\theta}_{2}|+\|\widehat{h}_{1}-\widehat{h}_{2}\|_{\infty}+\|\widehat{m}_{1}-\widehat{m}_{2}\|_{\infty}.
% \end{equation*}
\end{lemma}

\begin{lemma}\label{lem: fn_norm_decomp}
    Under condition \ref{cond: param_FN}, \ref{cond: sieve_FN}, \ref{cond: G}, let $(\widehat{\nu},\widehat{\theta})$, $(\widehat{\nu}_{1},\widehat{\theta}_{1})$, and $(\widehat{\nu}_{2},\widehat{\theta}_{2})$ be arbitrary three parameter tuples inside the sieve space defined in condition \ref{cond: sieve_FN},, the following inequalities hold.
    \begin{align*}
    &\|l(T,\delta,Z;\nu_{0},\theta_{0})-l(T,\delta,Z;\widehat{\nu},\widehat{\theta})\|_{\infty}\lesssim |\theta_{0}-\widehat{\theta}|+\|\nu_{0}-\widehat{\nu}\|_{\infty}\\
    &\|l(T,\delta,Z;\widehat{\nu}_{1},\widehat{\theta}_{1})-l(T,\delta,Z;\widehat{\nu}_{2},\widehat{\theta}_{2})\|_{\infty}\lesssim
    |\widehat{\theta}_{1}-\widehat{\theta}_{2}|+\|\widehat{\nu}_{1}-\widehat{\nu}_{2}\|_{\infty}.
    \end{align*}
\end{lemma}

\begin{lemma}[Approximating error of PF scheme]\label{lem: pf_approx}
    In the PF scheme, for any $n$, there exists an element in the corresponding sieve space $\pi_{n}\phi_{0}=(\pi_{n}h_{0},\pi_{n}m_{0},\pi_{n}\theta_{0})$, satisfying $d_{\textsf{PF}}\left(\pi_{n}\phi_{0},\phi_{0}\right) = O\left(n^{-\frac{\beta}{\beta+d}}\right)$.
\end{lemma}

\begin{lemma}[Approximating error of FN scheme]\label{lem: fn_approx}
    In the FN scheme, for any $n$, there exists an element in the corresponding sieve space  $\pi_{n}\psi = (\pi_{n}\nu_{0},\pi_{n}\theta_{0})$ satisfying 
    $d_{\textsf{FN}}\left(\pi_{n}\psi_{0},\psi_{0}\right)= O\left(n^{-\frac{\beta}{\beta+d+1}}\right)$.
\end{lemma}

\begin{lemma}\label{lem: covering_number}
    Suppose $\mathcal{F}$ is a class of functions satisfying that $N(\varepsilon,\mathcal{F},\|\cdot\|)<\infty$ for $\forall \varepsilon>0$. We define $\widetilde{N}(\varepsilon,\mathcal{F},\|\cdot\|)$ to be the minimal number of $\varepsilon$-balls $B(f,\varepsilon)=\{g: \|g-f\|<\varepsilon\}$ needed to cover $\mathcal{F}$ with radius $\varepsilon$ and further constrain that $f\in \mathcal{F}$. Then we have 
    \begin{align*}
        N(\varepsilon,\mathcal{F},\|\cdot\|)\leq\widetilde{N}(\varepsilon,\mathcal{F},\|\cdot\|)\leq N(\frac{\varepsilon}{2},\mathcal{F},\|\cdot\|).
    \end{align*}
\end{lemma}

\begin{lemma}\label{lem: bracketing_number}
    Suppose $\mathcal{F}$ is a class of functions satisfying that $N_{[]}(\varepsilon,\mathcal{F},\|\cdot\|_{\infty})<\infty$ for $\forall \varepsilon>0$. We define $\widetilde{N}_{[]}(\varepsilon,\mathcal{F},\|\cdot\|_{\infty})$ to be the minimal number of brackets $[l,u]$ needed to cover $\mathcal{F}$ with $\|l-u\|_{\infty}< \varepsilon$ and further constrain that $f\in \mathcal{F}$, $l=f-\frac{\varepsilon}{2}$ and $u=f+\frac{\varepsilon}{2}$. Then we have
    \begin{align*}
        N_{[]}(\varepsilon,\mathcal{F},\|\cdot\|_{\infty})\leq\widetilde{N}_{[]}(\varepsilon,\mathcal{F},\|\cdot\|_{\infty})\leq N_{[]}(\frac{\varepsilon}{2},\mathcal{F},\|\cdot\|_{\infty})
    \end{align*}
    Furthermore, we have $\widetilde{N}_{[]}(\varepsilon,\mathcal{F},\|\cdot\|_{\infty})=\widetilde{N}(\frac{\varepsilon}{2},\mathcal{F},\|\cdot\|_{\infty})$.
\end{lemma}

\begin{lemma}[Model capacity of PF scheme]\label{lem: pf_capacity}
    Let $\mathcal{F}_{n}=\{l(T,\delta,Z;\widehat{h},\widehat{m},\widehat{\theta}): \widehat{h}\in \mathcal{H}_{n},\widehat{m} \in \mathcal{M}_{n},\widehat{\theta} \in \Theta\}$. Under condition \ref{cond: G}, with $s_{h}=\frac{2\beta}{2\beta+1}$ and $s_{m}=\frac{2\beta}{2\beta+d}$, there exist constants $c_{h}$ and $c_{m}>0$ such that
    \begin{eqnarray*}
    N_{[]}(\varepsilon,\mathcal{F}_{n},\|\cdot\|_{\infty})\lesssim \frac{1}{\varepsilon} N(c_{h}\varepsilon^{1/s_{h}},\mathcal{H}_{n},\|\cdot\|_{2})\times N(c_{m}\varepsilon^{1/s_{m}},\mathcal{M}_{n},\|\cdot\|_{2}).
    \end{eqnarray*}
\end{lemma}

\begin{lemma}[Model capacity of FN scheme]\label{lem: fn_capacity}
    Let $\mathcal{G}_{n}=\{l(T,\delta,Z;\widehat{\nu},\widehat{\theta}):\widehat{\nu}\in\mathcal{V}_n,\widehat{\theta}\in\Theta\}$. Under condition \ref{cond: G}, with $s_{\nu}=\frac{2\beta}{2\beta+d+1}$, there exists a constant $c_{\nu}>0$ such that
    \begin{eqnarray*}
    N_{[]}(\varepsilon,\mathcal{G}_{n},\|\cdot\|_{\infty})\lesssim \frac{1}{\varepsilon} N(c_{\nu}\varepsilon^{1/s_{\nu}},\mathcal{V}_n,\|\cdot\|_{2}).
    \end{eqnarray*}
\end{lemma}

\subsection{Proofs of theorem \ref{thm: rate_pf} and \ref{thm: rate_fn}}
\begin{proof}[Proof of theorem \ref{thm: rate_pf}]
    The proof is divided into four steps.
    
    \paragraph{Step $1$} We denote $\phi_{0}=(h_{0},m_{0},\theta_{0})$ and $\widehat{\phi}=(\widehat{h},\widehat{m},\widehat{\theta})$, where $\widehat{h}\in\mathcal{H}_{n}$,$\widehat{m}\in\mathcal{M}_{n}$ and $\widehat{\theta}\in\Theta$. For arbitrary small $\varepsilon>0$, we have that
    \begin{eqnarray*}
    &&\inf_{d_{\textsf{PF}}\left(\widehat{\phi},\phi_{0}\right)\geq \varepsilon} \mathbb{E}\left[ l(T,\delta,Z;h_{0},m_{0},\theta_{0})-l(T,\delta,Z;\widehat{h},\widehat{m},\widehat{\theta})\right]\\
    &&=\inf_{d_{\textsf{PF}}\left(\widehat{\phi},\phi_{0}\right)\geq \varepsilon} \mathbb{E}_Z\left[\mathbb{E}_{T,\delta\mid Z}\left[\log p(T,\delta \mid Z;h_{0},m_{0},\theta_{0})-\log p(T,\delta \mid Z;\widehat{h},\widehat{m},\widehat{\theta})\right]\right]\\
    &&=\inf_{d_{\textsf{PF}}\left(\widehat{\phi},\phi_{0}\right)\geq \varepsilon}\mathbb{E}_Z\left[ 
    \fdivergence{KL}{\mathbb{P}_{\widehat{\phi},Z}}{\mathbb{P}_{\phi_{0},Z}}
    \right]
    \end{eqnarray*}
    Using the fact that $\fdivergence{KL}{\mathbb{P}_{\widehat{\phi},Z}}{\mathbb{P}_{\phi_{0},Z}}\geq 2 H^{2}(\mathbb{P}_{\widehat{\phi},Z}\parallel\mathbb{P}_{\phi_{0},Z})$. Thus, we further obtain that
    \begin{eqnarray*}
    &&\inf_{d_{\textsf{PF}}\left(\widehat{\phi},\phi_{0}\right)\geq \varepsilon} \mathbb{E}\left[ l(T,\delta,Z;h_{0},m_{0},\theta_{0})-l(T,\delta,Z;\widehat{h},\widehat{m},\widehat{\theta})\right]\\
    &&\geq \inf_{d_{\textsf{PF}}\left(\widehat{\phi},\phi_{0}\right)\geq \varepsilon}2\mathbb{E}_Z \left[H^{2}(\mathbb{P}_{\widehat{\phi},Z}\parallel\mathbb{P}_{\phi_{0},Z})\right]\\
    &&=2\inf_{d_{\textsf{PF}}\left(\widehat{\phi},\phi_{0}\right)\geq \varepsilon}d^{2}_{\textsf{PF}}\left(\widehat{\phi},\phi_{0}\right)\\
    &&\geq 2\varepsilon^{2}.
    \end{eqnarray*}
    
    \paragraph{Step $2$} Consider the following derivation.
    \begin{eqnarray*}
    &&\sup_{d_{\textsf{PF}}\left(\widehat{\phi},\phi_{0}\right)\leq \varepsilon}\var\left[l(T,\delta,Z;h_{0},m_{0},\theta_{0})-l(T,\delta,Z;\widehat{h},\widehat{m},\widehat{\theta})\right]\\
    &&\leq \sup_{d_{\textsf{PF}}\left(\widehat{\phi},\phi_{0}\right)\leq \varepsilon} \mathbb{E}\left[\left(l(T,\delta,Z;h_{0},m_{0},\theta_{0})-l(T,\delta,Z;\widehat{h},\widehat{m},\widehat{\theta})\right)^{2}\right]\\
    &&=\sup_{d_{\textsf{PF}}\left(\widehat{\phi},\phi_{0}\right)\leq \varepsilon}\mathbb{E}_Z\mathbb{E}_{T,\delta\mid Z}\left[\log p(T,\delta,Z;h_{0},m_{0},\theta_{0})-\log p(T,\delta,Z;\widehat{h},\widehat{m},\widehat{\theta})\right]^{2}\\
    % &&=4\sup_{d_{\textsf{PF}}\left(\widehat{\phi},\phi_{0}\right)\leq \varepsilon}\mathbb{E}_Z\int \left[p(T,\delta,Z;h_{0},m_{0},\theta_{0})\left(\log \sqrt{p(T,\delta,Z;h_{0},m_{0},\theta_{0})}-\log \sqrt{p(T,\delta,Z;\widehat{h},\widehat{m},\widehat{\theta})}\right)^{2}\right]\mu(dT\times d\delta)\\
    &&=4\sup_{d_{\textsf{PF}}\left(\widehat{\phi},\phi_{0}\right)\leq \varepsilon}\mathbb{E}_Z\left[\int \left(p(T,\delta,Z;h_{0},m_{0},\theta_{0})\left(\log \sqrt{\dfrac{p(T,\delta,Z;h_{0},m_{0},\theta_{0})}{p(T,\delta,Z;\widehat{h},\widehat{m},\widehat{\theta})}}\right)^{2}\right)\mu(dT\times d\delta)\right]
    \end{eqnarray*}
    By Taylor's expansion on $\log x$, there exists $\xi(T,\delta,Z)$ between $p^{\frac{1}{2}}(T,\delta,Z;h_{0},m_{0},\theta_{0})$ and $p^{\frac{1}{2}}(T,\delta,Z;\widehat{h},\widehat{m},\widehat{\theta})$ pointwisely such that
    \begin{eqnarray*}
    &&p(T,\delta,Z;h_{0},m_{0},\theta_{0})\left(\log \sqrt{\dfrac{p(T,\delta,Z;h_{0},m_{0},\theta_{0})}{p(T,\delta,Z;\widehat{h},\widehat{m},\widehat{\theta})}}\right)^{2}\\
    &&=p(T,\delta,Z;h_{0},m_{0},\theta_{0})\left(\log \sqrt{p(T,\delta,Z;h_{0},m_{0},\theta_{0})}-\log \sqrt{p(T,\delta,Z;\widehat{h},\widehat{m},\widehat{\theta})}\right)^{2}\\
    &&= \frac{p(T,\delta,Z;h_{0},m_{0},\theta_{0})}{\xi(T,\delta,Z)^{2}}\left(\sqrt{p(T,\delta,Z;h_{0},m_{0},\theta_{0})}-\sqrt{p(T,\delta,Z;\widehat{h},\widehat{m},\widehat{\theta})}\right)^{2}
    \end{eqnarray*}
    Since 
    \begin{align*}
        \dfrac{p(T,\delta,Z;h_{0},m_{0},\theta_{0})}{p(T,\delta,Z;\widehat{h},\widehat{m},\widehat{\theta})}=e^{l(T,\delta,Z;h_{0},m_{0},\theta_{0})-l(T,\delta,Z;\widehat{h},\widehat{m},\widehat{\theta})}
    \end{align*}
    by lemma \ref{lem: pf_l_bound}, $l(T,\delta,Z;h_{0},m_{0},\theta_{0})$ and $l(T,\delta,Z;\widehat{h},\widehat{m},\widehat{\theta})$ are bounded among$[0,\tau]\times\{0,1\}\times[-1,1]^{d}$ uniformly on all $\widehat{\phi}=(\widehat{h},\widehat{m},\widehat{\theta})$. Thus, there exist constants $C_{1}$ and $C_{2}$ such that $0<C_{1}\leq p(T,\delta,Z;h_{0},m_{0},\theta_{0})/p(T,\delta,Z;\widehat{h},\widehat{m},\widehat{\theta})\leq C_{2}$. This leads to the fact that $p(T,\delta,Z;h_{0},m_{0},\theta_{0})\frac{1}{\xi(T,\delta,Z)^{2}}$ is bounded. We further obtained that
    \begin{eqnarray*}
    &&p(T,\delta,Z;h_{0},m_{0},\theta_{0})\left(\log \sqrt{p(T,\delta,Z;h_{0},m_{0},\theta_{0})}-\log\sqrt{p(T,\delta,Z;\widehat{h},\widehat{m},\widehat{\theta})}\right)^{2}\\
    &&\lesssim \left|\sqrt{p(T,\delta,Z;h_{0},m_{0},\theta_{0})}-\sqrt{p(T,\delta,Z;\widehat{h},\widehat{m},\widehat{\theta})}\right|^{2}.
    \end{eqnarray*}
    Thus, we have that
    \begin{eqnarray*}
    &&\sup_{d_{\textsf{PF}}\left[\widehat{\phi},\phi_{0}\right]\leq \varepsilon}\var(l(T,\delta,Z;h_{0},m_{0},\theta_{0})-l(T,\delta,Z;\widehat{h},\widehat{m},\widehat{\theta}))\\
    &&\lesssim\sup_{d_{\textsf{PF}}\left(\widehat{\phi},\phi_{0}\right)\leq \varepsilon}\mathbb{E}_Z\left[\int\left|\sqrt{p(T,\delta,Z;h_{0},m_{0},\theta_{0})}-\sqrt{p(T,\delta,Z;\widehat{h},\widehat{m},\widehat{\theta})}\right|^{2}\mu(dT\times d\delta)\right]\\
    &&=\sup_{d_{\textsf{PF}}\left(\widehat{\phi},\phi_{0}\right)\leq \varepsilon}d^{2}_{\textsf{PF}}\left(\widehat{\phi},\phi_{0}\right)\\
    &&\leq \varepsilon^{2}.
    \end{eqnarray*}
    
    \paragraph{Step $3$} We define that $\widetilde{\mathcal{F}}_{n}=\{l(T,\delta,Z;\widehat{h},\widehat{m},\widehat{\theta})-l(T,\delta,Z;\pi_{n}h_{0},\pi_{n}m_{0},\pi_{n}\theta_{0}):\widehat{h}\in\mathcal{H}_{n},\widehat{m}\in\mathcal{M}_{n},\widehat{\theta}\in\Theta\}$.
    Here $(\pi_{n}h_{0},\pi_{n}m_{0},\pi_{n}\theta_{0})$ have been defined in lemma \ref{lem: pf_approx}. Obviously, we have that $\log N_{[]}(\varepsilon,\widetilde{\mathcal{F}}_{n},\|\cdot\|_{\infty}) = \log N_{[]}(\varepsilon,\mathcal{F}_{n},\|\cdot\|_{\infty})$, where $\mathcal{F}$ is defined in lemma \ref{lem: pf_capacity}. By lemma \ref{lem: pf_capacity}, we further have that
    \begin{eqnarray*}
    \log N_{[]}(\varepsilon,\mathcal{F}_{n},\|\cdot\|_{\infty})\lesssim \log\frac{1}{\varepsilon}+ \log N(c_{h}\varepsilon^{1/s_{h}},\mathcal{H}_{n},\|\cdot\|_{2})+ \log N(c_{m}\varepsilon^{1/s_{m}},\mathcal{M}_{n},\|\cdot\|_{2}).
    \end{eqnarray*}
    According to \citet[Theorem 7]{bartlett2019nearly}, under condition \ref{cond: sieve_PF}, we have that the VC-dimension of $\mathcal{H}_{n}$ and $\mathcal{M}_{n}$ satisfy that $\vcdim{\mathcal{H}_{n}} \lesssim n^{\frac{1}{\beta+d}}\log^{3}n$ and $\vcdim{\mathcal{M}_{n}}\lesssim n^{\frac{d}{\beta+d}}\log^{3}n$. Thus, we obtain that
    \begin{eqnarray*}
    \log N(c_{h}\varepsilon^{1/s_{h}},\mathcal{H}_{n},\|\cdot\|_{2})\lesssim \frac{\vcdim{\mathcal{H}_{n}}}{s_{h}}\log\frac{1}{\varepsilon}\lesssim n^{\frac{1}{\beta+d}}\log^{3}n\log\frac{1}{\varepsilon},
    \end{eqnarray*}
    and
    \begin{eqnarray*}
    \log N(c_{m}\varepsilon^{1/s_{m}},\mathcal{M}_{n},\|\cdot\|_{2})\lesssim \frac{\vcdim{\mathcal{M}_{n}}}{s_{\nu}}\log\frac{1}{\varepsilon}\lesssim n^{\frac{d}{\beta+d}}\log^{3}n\log\frac{1}{\varepsilon}.
    \end{eqnarray*}
    Thus, we obtain that $\log N_{[]}(\varepsilon,\widetilde{\mathcal{F}}_{n},\|\cdot\|_{\infty})\lesssim n^{\frac{d}{\beta+d}}\log^{3}n\log\frac{1}{\varepsilon}$.
    
    \paragraph{Step $4$} By the Cauchy-Schwartz inequality, we have that
    \begin{eqnarray*}
    \sqrt{\mathbb{E}\left[l(T,\delta,Z;\pi_{n}h_{0},\pi_{n}m_{0},\pi_{n}\theta_{0})-l(T,\delta,Z;h_{0},m_{0},\theta_{0})\right]}\\
    \leq \left[\mathbb{E}(l(T,\delta,Z;\pi_{n}h_{0},\pi_{n}m_{0},\pi_{n}\theta_{0})-l(T,\delta,Z;h_{0},m_{0},\theta_{0}))^{2}\right]^{\frac{1}{4}}.
    \end{eqnarray*}
    Similar to the second part and by lemma \ref{lem: pf_approx}, we further have that
    \begin{eqnarray*}
    \sqrt{\mathbb{E}\left[l(T,\delta,Z;\pi_{n}h_{0},\pi_{n}m_{0},\pi_{n}\theta_{0})-l(T,\delta,Z;h_{0},m_{0},\theta_{0})\right]}\lesssim \sqrt{d_{\textsf{PF}}\left(\pi_{n}\phi_{0},\phi_{0}\right)}\lesssim n^{-\frac{\beta}{2\beta+2d}}.
    \end{eqnarray*}
    Now let 
    \begin{align*}
        \tau = \frac{\beta}{2\beta+2d}-2\frac{\log\log n}{\log n}
    \end{align*}
    By Step 1,2,3 and \citet[Theorem 1]{shen1994convergence}, we have
    \begin{align*}
        d_{\textsf{PF}}\left(\widehat{\phi}_{n},\phi_{0}\right)&= \max\left(n^{-\tau},d_{\textsf{PF}}\left(\pi_{n}\phi_{0},\phi_{0}\right), \right. \\
        &\left. \sqrt{\mathbb{E}\left[l(T,\delta,Z;\pi_{n}h_{0},\pi_{n}m_{0},\pi_{n}\theta_{0})-l(T,\delta,Z;h_{0},m_{0},\theta_{0})\right]}\right)
    \end{align*}
    By lemma \ref{lem: pf_approx}, $d_{\textsf{PF}}\left(\pi_{n}\phi_{0},\phi_{0}\right)=O(n^{-\frac{\beta}{\beta+d}})$.\par 
    By Step 4, $\sqrt{\mathbb{E}\left[l(T,\delta,Z;\pi_{n}h_{0},\pi_{n}m_{0},\pi_{n}\theta_{0})-l(T,\delta,Z;h_{0},m_{0},\theta_{0})\right]}=O\left(n^{-\frac{\beta}{2\beta+2d}}\right)$.
    Thus, we have $d_{\textsf{PF}}\left(\widehat{\phi}_{n},\phi_{0}\right) = O(n^{-\frac{\beta}{2\beta+2d}}\log^{2}n) = \widetilde{O}(n^{-\frac{\beta}{2\beta+2d}})$.
\end{proof}

\begin{proof}[Proof of theorem \ref{thm: rate_fn}]
    The proof is divided into four steps.
    
    \paragraph{Step $1$} We denote $\psi_{0}=(\nu_{0},\theta_{0})$ and $\widehat{\psi}=(\widehat{\nu},\widehat{\theta})$, where $\widehat{\nu}\in\mathcal{V}_{n}$ and $\widehat{\theta}\in\Theta$. For arbitrary $0<\varepsilon\leq 1$, we have that
    \begin{eqnarray*}
    &&\inf_{d_{\textsf{FN}}\left(\widehat{\psi},\psi_{0}\right)\geq \varepsilon} \mathbb{E}\left[ l(T,\delta,Z;\nu_{0},\theta_{0})-l(T,\delta,Z;\widehat{\nu},\widehat{\theta})\right]\\
    &&=\inf_{d_{\textsf{FN}}\left(\widehat{\psi},\psi_{0}\right)\geq \varepsilon} \mathbb{E}_Z\left[\mathbb{E}_{T,\delta\mid Z}\left[\log p(T,\delta \mid Z;\nu_{0},\theta_{0})-\log p(T,\delta \mid Z;\widehat{\nu},\widehat{\theta})\right]\right]\\
    &&=\inf_{d_{\textsf{FN}}\left(\widehat{\psi},\psi_{0}\right)\geq \varepsilon}\mathbb{E}_Z \left[\fdivergence{KL}{\mathbb{P}_{\widehat{\psi},Z}}{\parallel\mathbb{P}_{\psi_{0},Z}}\right]
    \end{eqnarray*}
    Using the fact that $KL(\mathbb{P}_{\widehat{\psi},Z}\parallel\mathbb{P}_{\psi_{0},Z})\geq 2 H^{2}(\mathbb{P}_{\widehat{\psi},Z}\parallel\mathbb{P}_{\psi_{0},Z})$. Thus, we further obtain that
    \begin{eqnarray*}
    &&\inf_{d_{\textsf{FN}}\left(\widehat{\psi},\psi_{0}\right)\geq \varepsilon} \mathbb{E}\left[ l(T,\delta,Z;\nu_{0},\theta_{0})-l(T,\delta,Z;\widehat{\nu},\widehat{\theta})\right]\\
    &&\geq \inf_{d_{\textsf{FN}}\left(\widehat{\psi},\psi_{0}\right)\geq \varepsilon}2\mathbb{E}_Z\left[H^{2}(\mathbb{P}_{\widehat{\psi},Z}\parallel\mathbb{P}_{\psi_{0},Z})\right]\\
    &&=2\inf_{d_{\textsf{FN}}\left(\widehat{\psi},\psi_{0}\right)\geq \varepsilon}d^{2}_{\textsf{FN}}\left(\widehat{\psi},\psi_{0}\right)\\
    &&\geq 2\varepsilon^{2}.
    \end{eqnarray*}
    
    \paragraph{Step $2$} We consider the following derivation.
    \begin{eqnarray*}
    &&\sup_{d_{\textsf{FN}}\left(\widehat{\psi},\psi_{0}\right)\leq \varepsilon}\var\left[l(T,\delta,Z;\nu_{0},\theta_{0})-l(T,\delta,Z;\widehat{\nu},\widehat{\theta})\right]\\
    &&\leq \sup_{d_{\textsf{FN}}\left(\widehat{\psi},\psi_{0}\right)\leq \varepsilon} \mathbb{E}\left[\left(l(T,\delta,Z;\nu_{0},\theta_{0})-l(T,\delta,Z;\widehat{\nu},\widehat{\theta})\right)^{2}\right]\\
    &&=\sup_{d_{\textsf{FN}}\left(\widehat{\psi},\psi_{0}\right)\leq \varepsilon}\mathbb{E}_Z\left[\mathbb{E}_{T,\delta\mid Z}\left[\left(\log p(T,\delta,Z;\nu_{0},\theta_{0})-\log p(T,\delta,Z;\widehat{\nu},\widehat{\theta})\right)^{2}\right]\right]\\
    &&=4\sup_{d_{\textsf{FN}}\left(\widehat{\psi},\psi_{0}\right)\leq \varepsilon}\mathbb{E}_Z\left[\int \left(p(T,\delta,Z;\nu_{0},\theta_{0})(\log \sqrt{\dfrac{p(T,\delta,Z;\nu_{0},\theta_{0})}{p(T,\delta,Z;\widehat{\nu},\widehat{\theta})}})^{2}\right)\mu(dT\times d\delta)\right]\\
    \end{eqnarray*}
    By Taylor's expansion on $\log x$, there exists $\eta(T,\delta,Z)$ between $\sqrt{p(T,\delta,Z;\nu_{0},\theta_{0})}$ and $\sqrt{p(T,\delta,Z;\widehat{\nu},\widehat{\theta})}$ pointwisely such that
    \begin{eqnarray*}
    &&p(T,\delta,Z;\nu_{0},\theta_{0})(\log \sqrt{\dfrac{p(T,\delta,Z;\nu_{0},\theta_{0})}{p(T,\delta,Z;\widehat{\nu},\widehat{\theta})}})^{2}\\
    &&=p(T,\delta,Z;\nu_{0},\theta_{0})\left(\log \sqrt{p(T,\delta,Z;\nu_{0},\theta_{0})}-\log \sqrt{p(T,\delta,Z;\widehat{\nu},\widehat{\theta})}\right)^{2}\\
    &&= \dfrac{p(T,\delta,Z;\nu_{0},\theta_{0})}{\eta(T,\delta,Z)^{2}}\left(\sqrt{p(T,\delta,Z;\nu_{0},\theta_{0}})-\sqrt{p(T,\delta,Z;\widehat{\nu},\widehat{\theta})}\right)^{2}
    \end{eqnarray*}
    Since $p(T,\delta,Z;\nu_{0},\theta_{0})/p(T,\delta,Z;\widehat{\nu},\widehat{\theta})=e^{l(T,\delta,Z;\nu_{0},\theta_{0})-l(T,\delta,Z;\widehat{\nu},\widehat{\theta})}$, by lemma \ref{lem: fn_l_bound}, $l(T,\delta,Z;\nu_{0},\theta_{0})$ and $l(T,\delta,Z;\widehat{\nu},\widehat{\theta})$ are bounded on $[0,\tau]\times\{0,1\}\times[-1,1]^{d}$ uniformly for all $\widehat{\psi}=(\widehat{\nu},\widehat{\theta})$. Thus there exist constants $C_{3}$ and $C_{4}$ such that $0<C_{3}\leq p(T,\delta,Z;\nu_{0},\theta_{0})/p(T,\delta,Z;\widehat{\nu},\widehat{\theta})\leq C_{4}$. This leads to the fact that $p(T,\delta,Z;\nu_{0},\theta_{0})\frac{1}{\eta(T,\delta,Z)^{2}}$ is bounded. We further have that
    \begin{eqnarray*}
    &&p(T,\delta,Z;\nu_{0},\theta_{0})\left(\log \sqrt{p(T,\delta,Z;\nu_{0},\theta_{0})}-\log \sqrt{p(T,\delta,Z;\widehat{\nu},\widehat{\theta})}\right)^{2}\\
    &&\lesssim \left|\sqrt{p(T,\delta,Z;\nu_{0},\theta_{0})}-\sqrt{p(T,\delta,Z;\widehat{\nu},\widehat{\theta})}\right|^{2}.
    \end{eqnarray*}
    Thus, we have that
    \begin{eqnarray*}
    &&\sup_{d_{\textsf{FN}}\left(\widehat{\psi},\psi_{0}\right)\leq \varepsilon}\var\left[l(T,\delta,Z;\nu_{0},\theta_{0})-l(T,\delta,Z;\widehat{\nu},\widehat{\theta})\right]\\
    &&\lesssim\sup_{d_{\textsf{FN}}\left(\widehat{\psi},\psi_{0}\right)\leq \varepsilon}\mathbb{E}_Z\left[\int\left|\sqrt{p(T,\delta,Z;\nu_{0},\theta_{0})}-\sqrt{p(T,\delta,Z;\widehat{\nu},\widehat{\theta})}\right|^{2}\mu(dT\times d\delta)\right]\\
    &&=\sup_{d_{\textsf{FN}}\left(\widehat{\psi},\psi_{0}\right)\leq \varepsilon}d^{2}_{\textsf{FN}}\left(\widehat{\psi},\psi_{0}\right)\\
    &&\leq \varepsilon^{2}.
    \end{eqnarray*}
    
    \paragraph{Step $3$} We define that $\widetilde{\mathcal{G}}_{n}=\{l(T,\delta,Z;\widehat{\nu},\widehat{\theta})-l(T,\delta,Z;\pi_{n}\nu_{0},\pi_{n}\theta_{0}):\widehat{\nu}\in\mathcal{V}_{n},\theta\in\Theta\}$.
    Here $(\pi_{n}\nu_{0},\pi_{n}\theta_{0})$ have been defined in lemma \ref{lem: fn_approx}. Obviously, we have that $\log N_{[]}(\varepsilon,\widetilde{\mathcal{G}}_{n},\|\cdot\|_{\infty}) = \log N_{[]}(\varepsilon,\mathcal{G}_{n},\|\cdot\|_{\infty})$, where $\mathcal{G}$ is defined in lemma \ref{lem: fn_capacity}. By lemma \ref{lem: fn_capacity}, we further obtain that
    \begin{eqnarray*}
    \log N_{[]}(\varepsilon,\mathcal{G}_{n},\|\cdot\|_{\infty})\lesssim \log\frac{1}{\varepsilon}+ \log N(c_{\nu}\varepsilon^{1/s_{\nu}},\mathcal{V}_{n},\|\cdot\|_{2}).
    \end{eqnarray*}
    According to \citet[Theorem 7]{bartlett2019nearly}, under condition \ref{cond: sieve_FN}, we have that the VC-dimension of $\mathcal{V}_{n}$ satisfies that $\vcdim{\mathcal{V}_{n}}\lesssim n^{\frac{d+1}{\beta+d+1}}\log^{3}n$. Thus, we obtain that
    \begin{eqnarray*}
    \log N(c_{h}\varepsilon^{1/s_{\nu}},\mathcal{V}_{n},\|\cdot\|_{2})\lesssim \frac{\vcdim{\mathcal{V}_{n}}}{s_{\nu}}\log\frac{1}{\varepsilon}\lesssim n^{\frac{d+1}{\beta+d+1}}\log^{3}n\log\frac{1}{\varepsilon}.
    \end{eqnarray*}
    Furthermore, we get that $\log N_{[]}(\varepsilon,\widetilde{\mathcal{G}}_{n},\|\cdot\|_{\infty})\lesssim n^{\frac{d+1}{\beta+d+1}}\log^{3}n\log\frac{1}{\varepsilon}$.
    
    \paragraph{Step $4$} By the Cauchy-Schwartz inequality, we have that
    \begin{eqnarray*}
    \sqrt{\mathbb{E}[l(T,\delta,Z;\pi_{n}\nu_{0},\pi_{n}\theta_{0})-l(T,\delta,Z;\nu_{0},\theta_{0})]}\leq \left[\mathbb{E}\left(l(T,\delta,Z;\pi_{n}\nu_{0},\pi_{n}\theta_{0})-l(T,\delta,Z;\nu_{0},\theta_{0})\right)^{2}\right]^{\frac{1}{4}}.
    \end{eqnarray*}
    Similar to the second part and by lemma \ref{lem: fn_approx}, we further obtain that
    \begin{eqnarray*}
    &&\sqrt{\mathbb{E}[l(T,\delta,Z;\pi_{n}\nu_{0},\pi_{n}\theta_{0})-l(T,\delta,Z;\nu_{0},\theta_{0})]}\lesssim \sqrt{d_{\textsf{FN}}\left(\pi_{n}\psi_{0},\psi_{0}\right)}\lesssim n^{-\frac{\beta}{2\beta+2d+2}}
    \end{eqnarray*}
    Now let
    \begin{align*}
        \tau = \frac{\beta}{2\beta+2d+2}-2\frac{\log\log n}{\log n}.
    \end{align*}
    By step 1,2,3 and Step 1,2,3 and \citet[Theorem 1]{shen1994convergence},
    \begin{align*}
        d_{\textsf{FN}}\left(\widehat{\psi}_{n},\psi_{0}\right) &= \max\left(n^{-\tau},d_{\textsf{FN}}\left(\pi_{n}\psi_{0},\psi_{0}\right),\right.\\
        &\left.\sqrt{\mathbb{E}[l(T,\delta,Z;\pi_{n}\nu_{0},\pi_{n}\theta_{0})-l(T,\delta,Z;\nu_{0},\theta_{0})]}\right)
    \end{align*}
    By lemma\ref{lem: fn_approx}, $d_{\textsf{FN}}\left(\pi_{n}\psi_{0},\psi_{0}\right)=O(n^{-\frac{\beta}{\beta+d+1}})$ \par
    By Step 4, $\sqrt{\mathbb{E}[l(T,\delta,Z;\pi_{n}\nu_{0},\pi_{n}\theta_{0})-l(T,\delta,Z;\nu_{0},\theta_{0})]} = O(n^{-\frac{\beta}{2\beta+2d+2}})$.
    Thus, we have  $d_{\textsf{FN}}\left(\widehat{\psi}_{n},\psi_{0}\right) = O(n^{-\frac{\beta}{2\beta+2d+2}}\log^{2}n) = \widetilde{O}(n^{-\frac{\beta}{2\beta+2d+2}})$.
\end{proof}

\subsection{Proofs of technical lemmas}

\begin{proof}[Proof of lemma \ref{lem: pf_l_bound}]
    Since $h_{0}(T)\in \holderspace{\beta}{M}{[0, \tau]}$ and $m_{0}(Z)\in \holderspace{\beta}{M}{[-1, 1]^d}$, we have that $h_{0}(T)\leq M$, $m_{0}(Z)\leq M$ and $e^{m_{0}(Z)}\int_{0}^{T}h_{0}(s)ds\leq \tau e^{2M}$. Let $\mathcal{B}=[0,\tau e^{2M}]$, we have that
    \begin{eqnarray*}
    &&|l(T,\delta,Z;h_{0},m_{0},\theta_{0})|\\
    &&\leq \left|\log g_{\theta_{0}}\left(e^{m_{0}(Z)}\int_{0}^{T}e^{h_{0}(s)}ds\right)\right|+|h_{0}(T)|+ |m_{0}(Z)|+\left|G_{\theta_{0}}\left(e^{m_{0}(Z)}\int_{0}^{T}e^{h_{0}(s)}ds\right)\right|\\
    &&\leq 2M+\sup_{x\in \mathcal{B}}\left|\log g_{\theta_{0}}(x)\right|+\sup_{x\in \mathcal{B}}\left|G_{\theta_{0}}(x)\right|
    \end{eqnarray*}
    By condition 5, we have that $l(T,\delta,Z;h_{0},m_{0},\theta_{0})$ is bounded for $(T,\delta,Z)\in [0,\tau]\times\{0,1\}\times[-1,1]^{d}$. The proof of the boundness of $l(T,\delta,Z;\widehat{h},\widehat{m},\widehat{\theta})$ is similar.
\end{proof}


\begin{proof}[Proof of lemma \ref{lem: fn_l_bound}]
    Since $\nu_{0}(T,Z)\in \holderspace{\beta}{M}{[0, \tau] \times [-1, 1]^d}$, we have $\nu_{0}(T,Z)\leq M$ and $\int_{0}^{T} e^{\nu(s,Z)}ds\leq \tau e^{M}$. Let $\mathcal{B}=[0,\tau e^{M}]$, we have that
    \begin{eqnarray*}
    &&|l(T,\delta,Z;\nu_{0},\theta_{0})|\\
    &&\leq \left|\log G^{\prime}_{\theta_{0}}\left(\int_{0}^{T} e^{\nu_{0}(s,Z)}ds\right)\right|+ |\nu_{0}(T,Z)|+\left|G_{\theta_{0}}\left(\int_{0}^{T} e^{\nu_{0}(s,Z)}ds\right)\right|\\
    &&\leq M+\sup_{x\in \mathcal{B}}\left|\log G^{\prime}_{\theta_{0}}(x)\right|+\sup_{x\in \mathcal{B}}\left|G_{\theta_{0}}(x)\right|.
    \end{eqnarray*}
    By condition \ref{cond: G}, we have that $l(T,\delta,Z;\nu_{0},\theta_{0})$ is bounded among $(T,\delta,Z)\in [0,\tau]\times\{0,1\}\times[-1,1]^{d}$ The proof of the boundness of $l(T,\delta,Z;\widehat{\nu},\widehat{\theta})$ is similar.
\end{proof}


\begin{proof}[Proof of lemma \ref{lem: pf_norm_decomp}]
    By definition, we have that
    \begin{eqnarray*}
    &&|l(T,\delta,Z;h_{0},m_{0},\theta_{0})-l(T,\delta,Z;\widehat{h},\widehat{m},\widehat{\theta})|\\
    &&\leq
    \left|\log g_{\theta_{0}}\left(e^{m_{0}(Z)}\int_{0}^{T}e^{h_{0}(s)}ds\right)-\log g_{\widehat{\theta}}\left(e^{\widehat{m}(Z)}\int_{0}^{T}e^{\widehat{h}(s)}ds\right)\right|+\left|h_{0}(T)-\widehat{h}(T)\right|\\
    &&+|m_{0}(Z)-\widehat{m}(Z)|+\left|G_{\theta_{0}}\left(e^{m_{0}(Z)}\int_{0}^{T}e^{h_{0}(s)}ds\right)-G_{\widehat{\theta}}\left(e^{\widehat{m}(Z)}\int_{0}^{T}e^{\widehat{h}(s)}ds\right)\right|.
    \end{eqnarray*}
    Let $\mathcal{B}=[0,\tau\max(e^{2M},e^{M_{h}+M_{m}})]$. By Taylor's expansion, we can further show that
    \begin{eqnarray*}
    &&|l(T,\delta,Z;h_{0},m_{0},\theta_{0})-l(T,\delta,Z;\widehat{h},\widehat{m},\widehat{\theta})|\\
    &&\leq \sup_{\tilde{\theta}\in\Theta,\tilde{x}\in \mathcal{B}}\left|\frac{\partial \log g_{\tilde{\theta}}(\tilde{x})}{\partial_{\tilde{\theta}}}\right|\cdot\left|\theta_{0}-\widehat{\theta}\right|\\
    &&+\sup_{\tilde{\theta}\in\Theta,\tilde{x}\in \mathcal{B}}\left|\frac{\partial \log g_{\tilde{\theta}}(\tilde{x})}{\partial_{\tilde{x}}}\right|\cdot\left|e^{m_{0}(Z)}\int_{0}^{T}e^{h_{0}(s)}ds-e^{\widehat{m}(Z)}\int_{0}^{T}e^{\widehat{h}(s)}ds\right|\\
    &&+|h_{0}(T)-\widehat{h}(T)|+|m_{0}(Z)-\widehat{m}(Z)|+\sup_{\tilde{\theta}\in\Theta,\tilde{x}\in \mathcal{B}}\left|\frac{\partial G_{\tilde{\theta}}(\tilde{x})}{\partial_{\tilde{\theta}}}\right|\cdot\left|\theta_{0}-\widehat{\theta}\right|\\
    &&+\sup_{\tilde{\theta}\in\Theta,\tilde{x}\in \mathcal{B}}\left|\frac{\partial G_{\tilde{\theta}}(\tilde{x})}{\partial_{\tilde{x}}}\right|\cdot\left|e^{m_{0}(Z)}\int_{0}^{T}e^{h_{0}(s)}ds-e^{\widehat{m}(Z)}\int_{0}^{T}e^{\widehat{h}(s)}ds\right|.
    \end{eqnarray*}
    Again, by Taylor's expansion, we have that
    \begin{eqnarray*}
    &&\left|e^{m_{0}(Z)}\int_{0}^{T}e^{h_{0}(s)}ds-e^{\widehat{m}(Z)}\int_{0}^{T}e^{\widehat{h}(s)}ds\right|\\
    &&\leq \left|e^{m_{0}(Z)}\int_{0}^{T}(e^{h_{0}(s)}-e^{\widehat{h}(s)})ds\right|+\left|(e^{m_{0}(Z)}-e^{\widehat{m}(Z)})\int_{0}^{T}e^{\widehat{h}(s)}ds\right|\\
    &&\leq e^{M}\cdot\tau e^{\max(M,M_{h})}\left\|h_{0}-\widehat{h}\right\|_{\infty}+\tau e^{M_{h}}\cdot e^{\max(M,M_{m})}\|m_{0}-\widehat{m}\|_{\infty}.
    \end{eqnarray*}
    Finally, we obtain that
    \begin{eqnarray*}
    &&|l(T,\delta,Z;h_{0},m_{0},\theta_{0})-l(T,\delta,Z;\widehat{h},\widehat{m},\widehat{\theta})|\\
    &&\leq \sup_{\tilde{\theta}\in\Theta,\tilde{x}\in \mathcal{B}}\left|\frac{\partial \log g_{\tilde{\theta}}(\tilde{x})}{\partial_{\tilde{x}}}\right|\cdot \left[ e^{M}\cdot\tau e^{\max(M,M_{h})}\|h_{0}-\widehat{h}\|_{\infty}+\tau e^{M_{h}}\cdot e^{\max(M,M_{m})}\|m_{0}-\widehat{m}\|_{\infty}\right]\\
    &&+\sup_{\tilde{\theta}\in\Theta,\tilde{x}\in \mathcal{B}}\left|\frac{\partial \log g_{\tilde{\theta}}(\tilde{x})}{\partial_{\tilde{\theta}}}\right|\cdot\left|\theta_{0}-\widehat{\theta}\right|+\left|h_{0}(T)-\widehat{h}(T)\right|+|m_{0}(Z)-\widehat{m}(Z)|\\
    &&+\sup_{\tilde{\theta}\in\Theta,\tilde{x}\in \mathcal{B}}\left|\frac{\partial G_{\tilde{\theta}}(\tilde{x})}{\partial_{\tilde{\theta}}}\right|\cdot\left|\theta_{0}-\widehat{\theta}\right|\\
    &&+\sup_{\tilde{\theta}\in\Theta,\tilde{x}\in \mathcal{B}}\left|\frac{\partial G_{\tilde{\theta}}(\tilde{x})}{\partial_{\tilde{x}}}\right|\cdot \left[e^{M}\cdot\tau e^{\max(M,M_{h})}\|h_{0}-\widehat{h}\|_{\infty}+\tau e^{M_{h}}\cdot e^{\max(M,M_{m})}\|m_{0}-\widehat{m}\|_{\infty}\right].
    \end{eqnarray*}
    Taking supremum on both sides, we conclude that
    \begin{equation*}
    \|l(T,\delta,Z;h_{0},m_{0},\theta_{0})-l(T,\delta,Z;\widehat{h},\widehat{m},\widehat{\theta})\|_{\infty}\lesssim |\theta_{0}-\widehat{\theta}|+\|h_{0}-\widehat{h}\|_{\infty}+\|m_{0}-\widehat{m}\|_{\infty}.
    \end{equation*}
    The proof of the second inequality is similar.
\end{proof}


\begin{proof}[Proof of lemma \ref{lem: fn_norm_decomp}]
    By definition, we have that
    \begin{eqnarray*}
    &&|l(T,\delta,Z;\nu_{0},\theta_{0})-l(T,\delta,Z;\widehat{\nu},\widehat{\theta})|\\
    &&\leq
    \left|\log g_{\theta_{0}}\left(\int_{0}^{T}e^{\nu_{0}(s,Z)}ds\right)-\log g_{\widehat{\theta}}\left(\int_{0}^{T}e^{\widehat{\nu}(s,Z)}ds\right)\right|+\left|\nu_{0}(T,Z)-\widehat{\nu}(T,Z)\right|\\
    &&+\left|G_{\theta_{0}}\left(\int_{0}^{T}e^{\nu_{0}(s,Z)}ds\right)-G_{\widehat{\theta}}\left(\int_{0}^{T}e^{\widehat{\nu}(s,Z)}ds\right)\right|.
    \end{eqnarray*}
    Let $\mathcal{B}=[0,\tau\max(e^{M},e^{M_{\nu}})]$. By Taylor's expansion, we can further show that
    \begin{eqnarray*}
    &&|l(T,\delta,Z;\nu_{0},\theta_{0})-l(T,\delta,Z;\widehat{\nu},\widehat{\theta})|\\
    &&\leq \sup_{\tilde{\theta}\in\Theta,\tilde{x}\in \mathcal{B}}\left|\frac{\partial \log g_{\tilde{\theta}}(\tilde{x})}{\partial_{\tilde{\theta}}}\right|\cdot\left|\theta_{0}-\widehat{\theta}\right|+\sup_{\tilde{\theta}\in\Theta,\tilde{x}\in \mathcal{B}}\left|\frac{\partial \log g_{\tilde{\theta}}(\tilde{x})}{\partial_{\tilde{x}}}\right|\cdot\left|\int_{0}^{T}e^{\nu_{0}(s,Z)}ds-\int_{0}^{T}e^{\widehat{\nu}(s,Z)}ds\right|\\
    &&+|\nu_{0}(T,Z)-\widehat{\nu}(T,Z)|+\sup_{\tilde{\theta}\in\Theta,\tilde{x}\in \mathcal{B}}\left|\frac{\partial G_{\tilde{\theta}}(\tilde{x})}{\partial_{\tilde{\theta}}}\right|\cdot|\theta_{0}-\widehat{\theta}|\\
    &&+\sup_{\tilde{\theta}\in\Theta,\tilde{x}\in \mathcal{B}}\left|\frac{\partial G_{\tilde{\theta}}(\tilde{x})}{\partial_{\tilde{x}}}\right|\cdot\left|\int_{0}^{T}e^{\nu_{0}(s,Z)}ds-\int_{0}^{T}e^{\widehat{\nu}(s,Z)}ds\right|.
    \end{eqnarray*}
    Again, by Taylor's expansion, 
    \begin{eqnarray*}
    \left|\int_{0}^{T}e^{\nu_{0}(s,Z)}ds-\int_{0}^{T}e^{\widehat{\nu}(s,Z)}ds\right|\leq\tau e^{\max(M,M_{\nu})}\|\nu_{0}-\widehat{\nu}\|_{\infty},
    \end{eqnarray*}
    Finally, we obtain that
    \begin{eqnarray*}
    &&\left|l(T,\delta,Z;\nu_{0},\theta_{0})-l(T,\delta,Z;\widehat{\nu},\widehat{\theta})\right|\\
    &&\leq \sup_{\tilde{\theta}\in\Theta,\tilde{x}\in \mathcal{B}}\left|\frac{\partial \log g_{\tilde{\theta}}(\tilde{x})}{\partial_{\tilde{\theta}}}\right|\cdot\left|\theta_{0}-\widehat{\theta}\right| + \sup_{\tilde{\theta}\in\Theta,\tilde{x}\in \mathcal{B}}\left|\frac{\partial \log g_{\tilde{\theta}}(\tilde{x})}{\partial_{\tilde{x}}}\right|\cdot\tau e^{\max(M,M_{\nu})}\left\|\nu_{0}-\widehat{\nu}\right\|_{\infty}\\
    &&+|\nu_{0}(T,Z)-\widehat{\nu}(T,Z)|+\sup_{\tilde{\theta}\in\Theta,\tilde{x}\in \mathcal{B}}\left|\frac{\partial G_{\tilde{\theta}}(\tilde{x})}{\partial_{\tilde{\theta}}}\right|\cdot\left|\theta_{0}-\widehat{\theta}\right|\\
    &&+\sup_{\tilde{\theta}\in\Theta,\tilde{x}\in \mathcal{B}}\left|\frac{\partial G_{\tilde{\theta}}(\tilde{x})}{\partial_{\tilde{x}}}\right|\cdot\tau e^{\max(M,M_{\nu})}\left\|\nu_{0}-\widehat{\nu}\right\|_{\infty}.
    \end{eqnarray*}
    Taking supremum on both sides, we conclude that
    \begin{equation*}
    \|l(T,\delta,Z;\nu_{0},\theta_{0})-l(T,\delta,Z;\widehat{\nu},\widehat{\theta})\|_{\infty}\lesssim |\theta_{0}-\widehat{\theta}|+\|\nu_{0}-\widehat{\nu}\|_{\infty},
    \end{equation*}
    The proof of the second inequality is similar.
\end{proof}

\begin{proof}[Proof of lemma \ref{lem: pf_approx}]
    According to \citet[Theorem 1]{yarotsky2017error}, there exist approximating functions $\widehat{h}^{*}$ and $\widehat{m}^{*}$ such that $\|\widehat{h}^{*}-h_{0}\|_{\infty} = O\left(n^{-\frac{\beta}{\beta+d}}\right)$ and $\|\widehat{m}^{*}-m_{0}\|_{\infty} = O\left(n^{-\frac{\beta}{\beta+d}}\right)$. Let $\pi_{n}h_{0}=\widehat{h}^{*}$, $\pi_{n}m_{0}=\widehat{m}^{*}$, and $\pi_{n}\theta=\theta_{0}$. We have that
    \begin{eqnarray*}
    &&d_{\textsf{PF}}\left(\pi_{n}\phi_{0},\phi_{0}\right)\\
    &&=\sqrt{\mathbb{E}_Z\left[\int |\sqrt{p(T,\delta \mid Z;\pi_{n}h_{0},\pi_{n}m_{0},\pi_{n}\theta_{0})}-\sqrt{p(T,\delta \mid Z;h_{0},m_{0},\theta_{0})}|^{2}\mu(dT\times d\delta)\right]}\\
    &&=\sqrt{\mathbb{E}_Z\left[\int [e^{\frac{1}{2}l(T,\delta,Z;\pi_{n}h_{0},\pi_{n}m_{0},\pi_{n}\theta_{0})}-e^{\frac{1}{2}l(T,\delta,Z;h_{0},m_{0},\theta_{0})}]^{2}f_{C\mid Z}(T)^{1-\delta}S_{C\mid Z}(T)^{\delta}\mu(dT\times d\delta)\right]}\\
    &&\leq \left\|e^{\frac{1}{2}l(T,\delta,Z;\pi_{n}h_{0},\pi_{n}m_{0},\pi_{n}\theta_{0})}-e^{\frac{1}{2}l(T,\delta,Z;h_{0},m_{0},\theta_{0})}\right\|_{\infty} \\
    &&\qquad \qquad \times \sqrt{\mathbb{E}_Z\left[\int f_{C\mid Z}(T)^{1-\delta}S_{C\mid Z}(T)^{\delta}\mu(dT\times d\delta)\right]}.
    \end{eqnarray*}
    By lemma \ref{lem: pf_l_bound} and \ref{lem: pf_norm_decomp}, we have that
    \begin{eqnarray*}
    &&\|e^{\frac{1}{2}l(T,\delta,Z;\pi_{n}h_{0},\pi_{n}m_{0},\pi_{n}\theta_{0})}-e^{\frac{1}{2}l(T,\delta,Z;h_{0},m_{0},\theta_{0})}\|_{\infty}\\
    &&\lesssim \|\pi_{n}\theta_{0}-\theta_{0}\|+\|\pi_{n}h_{0}-h_{0}\|_{\infty}+\|\pi_{n}m_{0}-m_{0}\|_{\infty}\\
    &&=O\left(n^{-\frac{\beta}{\beta+d}}\right).
    \end{eqnarray*}
    Since $f_{C\mid Z}(T)^{1-\delta}\leq f_{C\mid Z}(T)+1$ and $S_{C\mid Z}(T)^{\delta}\leq 1$, we also have that
    \begin{eqnarray*}
    \sqrt{\mathbb{E}_Z\left[\int f_{C\mid Z}(T)^{1-\delta}S_{C\mid Z}(T)^{\delta}\mu(dT\times d\delta)\right]} &\leq&  \sqrt{\mathbb{E}_Z\left[\int (1+f_{C\mid Z}(T))\mu(dT\times d\delta)\right]}\\
    &\leq& \sqrt{2+2\tau}.
    \end{eqnarray*}
    Thus, we obtain that $d_{\textsf{PF}}\left(\pi_{n}\phi_{0},\phi_{0}\right) = O\left(n^{-\frac{\beta}{\beta+d}}\right)$.
\end{proof}

\begin{proof}[Proof of lemma \ref{lem: fn_approx}]
    According to \citet[Theorem 1]{yarotsky2017error}, there exists an approximating function $\widehat{\nu}^{*}$ such that $\|\widehat{\nu}^{*}-\nu_{0}\|_{\infty}=O\left(n^{-\frac{\beta}{\beta+d+1}}\right)$. Let $\pi_{n}\nu_{0} = \widehat{\nu}^{*}$ and $\pi_{n}\theta_{0}=\theta_{0}$. We have that
    \begin{eqnarray*}
    &&d_{\textsf{FN}}\left(\pi_{n}\psi_{0},\psi_{0}\right)\\
    &&=\sqrt{\mathbb{E}_Z\left[\int \left|\sqrt{p(T,\delta \mid Z;\pi_{n}\nu_{0},\pi_{n}\theta_{0})}-\sqrt{p(T,\delta \mid Z;\nu_{0},\theta_{0})}\right|^{2}\mu(dT\times d\delta)\right]}\\
    &&=\sqrt{\mathbb{E}_Z\left[\int \left[e^{\frac{1}{2}l(T,\delta,Z;\pi_{n}\nu_{0},\pi_{n}\theta_{0})}-e^{\frac{1}{2}l(T,\delta,Z;\nu_{0},\theta_{0})}\right]^{2}f_{C\mid Z}(T)^{1-\delta}S_{C\mid Z}(T)^{\delta}\mu(dT\times d\delta)\right]}\\
    &&\leq \left\|\frac{1}{2}e^{l(T,\delta,Z;\pi_{n}\nu_{0},\pi_{n}\theta_{0})}-\frac{1}{2}e^{l(T,\delta,Z;\nu_{0},\theta_{0})}\right\|_{\infty}\sqrt{\mathbb{E}_Z\left[\int f_{C\mid Z}(T)^{1-\delta}S_{C\mid Z}(T)^{\delta}\mu(dT\times d\delta)\right]}.
    \end{eqnarray*}
    By lemma \ref{lem: fn_l_bound} and \ref{lem: fn_norm_decomp}, we have that
    \begin{align*}
        \left\|e^{\frac{1}{2}l(T,\delta,Z;\pi_{n}\nu_{0},\pi_{n}\theta_{0})}-e^{\frac{1}{2}l(T,\delta,Z;\nu_{0},\theta_{0})}\right\|_{\infty} &\lesssim \|\pi_{n}\theta_{0}-\theta_{0}\|+\|\pi_{n}\nu_{0}-\nu_{0}\|_{\infty}\\
        &=O\left(n^{-\frac{\beta}{\beta+d+1}}\right).
    \end{align*}
    Since $f_{C\mid Z}(T)^{1-\delta}\leq f_{C\mid Z}(T)+1$ and $S_{C\mid Z}(T)^{\delta}\leq 1$, we also have that
    \begin{eqnarray*}
    \sqrt{\mathbb{E}_Z\left[\int f_{C\mid Z}(T)^{1-\delta}S_{C\mid Z}(T)^{\delta}\mu(dT\times d\delta)\right]} &\leq&  \sqrt{\mathbb{E}_Z\left[\int (1+f_{C\mid Z}(T))\mu(dT\times d\delta)\right]}\\
    &\leq& \sqrt{2+2\tau}.
    \end{eqnarray*}
    Thus, we obtain that $d_{\textsf{FN}}\left(\pi_{n}\psi_{0},\psi_{0}\right)= O\left(n^{-\frac{\beta}{\beta+d+1}}\right)$.
\end{proof}

\begin{proof}[Proof of lemma \ref{lem: covering_number}]
    The left inequality is trivial according to the definition of covering number.  We need to show that the correctness of the right inequality.

    Suppose that we have $\{B(g_{i},\frac{\varepsilon}{2})\},i=1\ldots,N$, where $N=N(\frac{\varepsilon}{2},\mathcal{F},\|\cdot\|)$, are the minimal number of $\frac{\varepsilon}{2}$-ball that covers $\mathcal{F}$. Then there exists at least one $f_{i}\in\mathcal{F}$ such that $f_{i}\in B(g_{i},\varepsilon)$. Consider the following $\varepsilon-balls$ $\{B(f_{i},\varepsilon)\},i=1\ldots,N$. For arbitrary $f\in \mathcal{F}\cap B(g_{i},\frac{\varepsilon}{2}),$ we have that $\|f-f_{i}\|\leq\|f-g_{i}\|+\|f_{i}-g_{i}\|\leq \varepsilon$. Thus $\{B(f_{i},\varepsilon)\},i=1\ldots,N$ forms a $\varepsilon$-covering of $\mathcal{F}$. By definition, we have that $\widetilde{N}(\varepsilon,\mathcal{F},\|\cdot\|)\leq N(\frac{\varepsilon}{2},\mathcal{F},\|\cdot\|)$.
\end{proof}

\begin{proof}[Proof of lemma \ref{lem: bracketing_number}]
    The proof of the first two inequalities follows exactly the same steps of lemma \ref{lem: covering_number}. Here we just need to mention the rest of the statement that $\widetilde{N}_{[]}(\varepsilon,\mathcal{F},\|\cdot\|_{\infty})=\widetilde{N}(\frac{\varepsilon}{2},\mathcal{F},\|\cdot\|_{\infty})$.
    We first choose a set of $\frac{\varepsilon}{2}$-covering balls $\{B(f_{i},\frac{\varepsilon}{2})\},i=1,\ldots,N_{1}$, where $N_{1}=\widetilde{N}(\frac{\varepsilon}{2},\mathcal{F},\|\cdot\|_{\infty})$. Now we construct a set of brackets $\{[l_{i},u_{i}]\},i=1\ldots,N_{1}$, where $l_{i}=f_{i}-\frac{\varepsilon}{2}$ and $u_{i}=f_{i}+\frac{\varepsilon}{2}$. Noting that the bracket $\{[l_{i},u_{i}]\}$ is exactly the same as $B(f_{i},\frac{\varepsilon}{2})$, The set $\{[l_{i},u_{i}]\},i=1,\ldots,N_{1}$ covers $\mathcal{F}$, which leads to $\widetilde{N}_{[]}(\varepsilon,\mathcal{F},\|\cdot\|_{\infty})\leq\widetilde{N}(\frac{\varepsilon}{2},\mathcal{F},\|\cdot\|_{\infty})$. Likewise, we have that $\widetilde{N}_{[]}(\varepsilon,\mathcal{F},\|\cdot\|_{\infty})\geq\widetilde{N}(\frac{\varepsilon}{2},\mathcal{F},\|\cdot\|_{\infty})$. Consequently,  we have that
    $\widetilde{N}_{[]}(\varepsilon,\mathcal{F},\|\cdot\|_{\infty})=\widetilde{N}(\frac{\varepsilon}{2},\mathcal{F},\|\cdot\|_{\infty})$.
\end{proof}


\begin{proof}[Proof of lemma \ref{lem: pf_capacity}]
    By lemma \ref{lem: bracketing_number}, first we have that $N_{[]}(\varepsilon,\mathcal{F}_{n},\|\cdot\|_{\infty})\leq \widetilde{N}_{[]}(\varepsilon,\mathcal{F}_{n},\|\cdot\|_{\infty})$. By lemma \ref{lem: pf_norm_decomp}, there exists a constant $c_{1}>0$ such that for arbitrary $\widehat{h}_{1},\widehat{h}_{2}\in\mathcal{H}_{n}$,$\widehat{m}_{1},\widehat{m}_{2}\in\mathcal{M}_{n}$ and $\widehat{\theta}_{1},\widehat{\theta}_{2}\in\Theta$, we have that
    \begin{eqnarray*}
    \|l(T,\delta,Z;\widehat{h}_{1},\widehat{m}_{1},\theta_{1})-l(T,\delta,Z;\widehat{h}_{2},\widehat{m}_{2},\theta_{2})\|_{\infty}\leq c_{1}[
    |\widehat{\theta}_{1}-\widehat{\theta}_{2}|+\|\widehat{h}_{1}-\widehat{h}_{2}\|_{\infty}+\|\widehat{m}_{1}-\widehat{m}_{2}\|_{\infty}],
    \end{eqnarray*}
    which indicates that as long as $|\widehat{\theta}_{1}-\widehat{\theta}_{2}|\leq\frac{\varepsilon}{3c_{1}}$, $\|\widehat{h}_{1}-\widehat{h}_{2}\|_{\infty}\leq\frac{\varepsilon}{3c_{1}}$ and $\|\widehat{m}_{1}-\widehat{m}_{2}\|_{\infty}\leq\frac{\varepsilon}{3c_{1}}$, we have that $\|l(T,\delta,Z;\widehat{h}_{1},\widehat{m}_{1},\theta_{1})-l(T,\delta,Z;\widehat{h}_{2},\widehat{m}_{2},\theta_{2})\|_{\infty}\leq \varepsilon$. Consequently, we have that
    \begin{eqnarray*}
    \widetilde{N}_{[]}(\varepsilon,\mathcal{F}_{n},\|\cdot\|_{\infty})\leq \widetilde{N}_{[]}(\frac{\varepsilon}{3c_{1}},\Theta,\|\cdot\|_{\infty})\times
    \widetilde{N}_{[]}(\frac{\varepsilon}{3c_{1}},\mathcal{H}_{n},\|\cdot\|_{\infty})\times\widetilde{N}_{[]}(\frac{\varepsilon}{3c_{1}},\mathcal{M}_{n},\|\cdot\|_{\infty}).
    \end{eqnarray*}
    
    Since $\Theta$ is a compact set on $\mathbb{R}$, by lemma \ref{lem: bracketing_number} and traditional volume argument, we have that $\widetilde{N}_{[]}(\frac{\varepsilon}{3c_{1}},\Theta,\|\cdot\|_{\infty})\leq N_{[]}(\frac{\varepsilon}{6c_{1}},\Theta,\|\cdot\|_{\infty})\lesssim\frac{1}{\varepsilon}$.
    
    For $\widetilde{N}_{[]}(\frac{\varepsilon}{3c_{1}},\mathcal{H}_{n},\|\cdot\|_{\infty})$, by lemma \ref{lem: bracketing_number}, we have that $\widetilde{N}_{[]}(\frac{\varepsilon}{3c_{1}},\mathcal{H}_{n},\|\cdot\|_{\infty})=\widetilde{N}(\frac{\varepsilon}{3c_{1}},\mathcal{H}_{n},\|\cdot\|_{\infty})$. By \citet[Lemma 2]{chen1998sieve}, there exists a constant $c_{2}>0$ such that $\|\widehat{h}_{1}-\widehat{h}_{2}\|_{\infty}\leq c_{2}\|\widehat{h}_{1}-\widehat{h}_{2}\|_{2}^{s_{h}}$, which leads to $\widetilde{N}(\frac{\varepsilon}{3c_{1}},\mathcal{H}_{n},\|\cdot\|_{\infty})\leq \widetilde{N}(\frac{\varepsilon^{1/s_{h}}}{(3c_{1}c_{2})^{1/s_{h}}},\mathcal{H}_{n},\|\cdot\|_{2})$. By lemma \ref{lem: covering_number} we further have that $\widetilde{N}(\frac{\varepsilon^{1/s_{h}}}{(3c_{1}c_{2})^{1/s_{h}}},\mathcal{H}_{n},\|\cdot\|_{2}) \leq N(\frac{\varepsilon^{1/s_{h}}}{2(3c_{1}c_{2})^{1/s_{h}}},\mathcal{H}_{n},\|\cdot\|_{2})$. Let $c_{h}=\frac{1}{2(3c_{1}c_{2})^{1/s_{h}}}$. We have that $\widetilde{N}_{[]}(\frac{\varepsilon}{3c_{1}},\mathcal{H}_{n},\|\cdot\|_{\infty})\leq N(c_{h}\varepsilon^{1/s_{h}},\mathcal{H}_{n},\|\cdot\|_{2})$.
    
    Similarly, there exists a constant $c_{m}>0$ such that $\widetilde{N}_{[]}(\frac{\varepsilon}{3c_{1}},\mathcal{M}_{n},\|\cdot\|_{\infty})\leq N(c_{m}\varepsilon^{1/s_{m}},\mathcal{M}_{n},\|\cdot\|_{2})$.
    
    Thus, finally we can obtain that
    \begin{eqnarray*}
    N_{[]}(\varepsilon,\mathcal{F}_{n},\|\cdot\|_{\infty})\lesssim \frac{1}{\varepsilon} N(c_{h}\varepsilon^{1/s_{h}},\mathcal{H}_{n},\|\cdot\|_{2})\times N(c_{m}\varepsilon^{1/s_{m}},\mathcal{M}_{n},\|\cdot\|_{2}).
    \end{eqnarray*}
\end{proof}

\begin{proof}[Proof of lemma \ref{lem: fn_capacity}]
    By lemma \ref{lem: bracketing_number}, first we have $N_{[]}(\varepsilon,\mathcal{G}_{n},\|\cdot\|_{\infty})\leq \widetilde{N}_{[]}(\varepsilon,\mathcal{G}_{n},\|\cdot\|_{\infty})$. By lemma \ref{lem: fn_norm_decomp}, there exists a constant $c_{3}>0$ such that for arbitrary $\widehat{\nu}_{1},\widehat{\nu}_{2}\in\mathcal{V}_{n}$ and $\widehat{\theta}_{1},\widehat{\theta}_{2}\in\Theta$, we have that
    \begin{eqnarray*}
    \|l(T,\delta,Z;\widehat{\nu}_{1},\widehat{\theta}_{1})-l(T,\delta,Z;\widehat{\nu}_{2},\widehat{\theta}_{2})\|_{\infty}\leq c_{3}[
    |\widehat{\theta}_{1}-\widehat{\theta}_{2}|+\|\widehat{\nu}_{1}-\widehat{\nu}_{2}\|_{\infty}],
    \end{eqnarray*}
    which indicates that as long as $|\widehat{\theta}_{1}-\widehat{\theta}_{2}|\leq\frac{\varepsilon}{2c_{3}}$and $\|\widehat{\nu}_{1}-\widehat{\nu}_{2}\|_{\infty}\leq\frac{\varepsilon}{2c_{3}}$, we have that $\|l(T,\delta,Z;\widehat{\nu}_{1},\widehat{\theta}_{1})-l(T,\delta,Z;\widehat{\nu}_{2},\widehat{\theta}_{2})\|_{\infty}\leq \varepsilon$. Thus, we have:
    \begin{eqnarray*}
    \widetilde{N}_{[]}(\varepsilon,\mathcal{G}_{n},\|\cdot\|_{\infty})\leq \widetilde{N}_{[]}(\frac{\varepsilon}{2c_{3}},\Theta,\|\cdot\|_{\infty})\times
    \widetilde{N}_{[]}(\frac{\varepsilon}{2c_{3}},\mathcal{V}_{n},\|\cdot\|_{\infty}).
    \end{eqnarray*}
    
    Since $\Theta$ is a compact set on $\mathbb{R}$, by lemma \ref{lem: bracketing_number} and traditional volume argument, we have that $\widetilde{N}_{[]}(\frac{\varepsilon}{2c_{3}},\Theta,\|\cdot\|_{\infty})\leq N_{[]}(\frac{\varepsilon}{4c_{3}},\Theta,\|\cdot\|_{\infty})\lesssim\frac{1}{\varepsilon}$.
    
    For $\widetilde{N}_{[]}(\frac{\varepsilon}{2c_{3}},\mathcal{V}_{n},\|\cdot\|_{\infty})$, by lemma \ref{lem: bracketing_number}, we have that $\widetilde{N}_{[]}(\frac{\varepsilon}{2c_{3}},\mathcal{V}_{n},\|\cdot\|_{\infty})=\widetilde{N}(\frac{\varepsilon}{2c_{3}},\mathcal{V}_{n},\|\cdot\|_{\infty})$. By \citet[Lemma 2]{chen1998sieve}, there exists a constant $c_{4}>0$ such that $\|\widehat{\nu}_{1}-\widehat{\nu}_{2}\|_{\infty}\leq c_{4}\|\widehat{\nu}_{1}-\widehat{\nu}_{2}\|_{2}^{s_{h}}$, which leads to $\widetilde{N}(\frac{\varepsilon}{2c_{3}},\mathcal{V}_{n},\|\cdot\|_{\infty})\leq \widetilde{N}(\frac{\varepsilon^{1/s_{\nu}}}{(2c_{3}c_{4})^{1/s_{\nu}}},\mathcal{V}_{n},\|\cdot\|_{2})$. By lemma \ref{lem: covering_number} we further have $\widetilde{N}(\frac{\varepsilon^{1/s_{\nu}}}{(2c_{3}c_{4})^{1/s_{\nu}}},\mathcal{V}_{n},\|\cdot\|_{2}) \leq N(\frac{\varepsilon^{1/s_{\nu}}}{2(2c_{3}c_{4})^{1/s_{\nu}}},\mathcal{V}_{n},\|\cdot\|_{2})$. Let $c_{\nu}=\frac{1}{2(2c_{3}c_{4})^{1/s_{\nu}}}$, we have that $\widetilde{N}_{[]}(\frac{\varepsilon}{2c_{3}},\mathcal{V}_{n},\|\cdot\|_{\infty})\leq N(c_{\nu}\varepsilon^{1/s_{\nu}},\mathcal{V}_{n},\|\cdot\|_{2})$.
    
    Thus, finally we can obtain that
    \begin{eqnarray*}
    N_{[]}(\varepsilon,\mathcal{G}_{n},\|\cdot\|_{\infty})\lesssim \frac{1}{\varepsilon} N(c_{\nu}\varepsilon^{1/s_{\nu}},\mathcal{V}_{n},\|\cdot\|_{2}).
    \end{eqnarray*}
\end{proof}

% \end{document}


