\section{Introduction}
Recently, metamaterial-based reconfigurable holographic surfaces~(\acrshort{rhs}s) have been proposed as novel cost-efficient antenna arrays, which possess high potential in improving the performance of wireless networks~\cite{Deng2021Reconfigurable}.
RHSs are characterized by their large number of densely arranged metamaterial antenna elements (\emph{\acrshort{me}s}), which have smaller spatial intervals than half of their working wavelength~\cite{Boyarsky2021Electronically}.
Such dense placement enables \acrshort{rhs}s to have strong manipulation capability for \acrfull{em} waves and can synthesize various beam forms through \acrfull{and} beamforming~\cite{Zhang2022Holographic}.
By leveraging this capability, an \acrfull{bs} equipped with an \acrshort{rhs} can focus transmitted signals towards users for communication enhancement.

To achieve this enhancement, the \acrshort{rhs}-based \acrshort{bs}s need to know the \acrfull{csi} or the positions of the users.
However, the channel state estimation problem of the \acrshort{rhs}s is highly complicated due to the large number of \acrshort{me}s lacking signal processing capability~\cite{Wei2021Channel}.
Alternatively, using the \acrshort{rhs}, a \acrshort{bs} can effectively positioning a target user and focus its beam towards it. 
This indicates that the \acrshort{rhs}-based \acrshort{bs}s are intrinsically fit for \acrfull{isac} in the sense that the communication can benefit from the positioning results obtained by the \acrshort{rhs}.

In this regard, the performance of an \acrshort{rhs}-based \acrshort{isac} system is fundamentally influenced by its positioning precision.
In literature, several works have considered the positioning aspect of \acrshort{rhs}-based \acrshort{isac} systems.
In~\cite{Zhang2022Holographic}, the authors proposed a beamforming algorithm to simultaneously generate beams for communication and sensing purposes with high directional gains.
In~\cite{ZhangX2022Holographic}, the authors used \acrshort{rhs} for target detection and obtained high accuracy with low cost and power consumption.

Additionally, given the essential similarity between reconfigurable intelligent surfaces~(RISs) and \acrshort{rhs}s, the positioning techniques for \acrshort{ris}-based systems have the potential to be applied in \acrshort{rhs}-based systems\footnote{The main difference between \acrshort{ris}s and \acrshort{rhs}s is that \acrshort{ris}s perform beamforming by reflecting signals, while \acrshort{rhs}s radiate the signals themselves.}.
In~\cite{zhang2020towards}, the authors utilized an \acrshort{ris} to generate distinguishable signals strength values for different locations to facilitate positioning. 
In~\cite{Elzanaty2021Reconfigurable}, the authors proposed a \acrshort{snr}-based \acrshort{ris} configuration profile that increases the positioning precision.

However, in existing works, the fundamental limitation is that the signal bandwidths for positioning are not broad enough.
This reduces the positioning precision, as the spatial resolution of positioning typically increases with the bandwidth\cite{Sturm2009ANovel}.
Though the authors in~\cite{Ma2021Indoor} considered using \acrfull{uwb} technique in \acrshort{ris}-based positioning systems, it is challenging to implement \acrshort{ris} or \acrshort{rhs} that can support beamforming of \acrshort{uwb} signals in practice, due to the high frequency selectivity of the reconfigurable \acrshort{me}s~\cite{Deng2021Reconfigurable, Boyarsky2021Electronically}.




In this paper, we propose the \acrfull{mb} \acrshort{rhs}-based \acrshort{isac} system, where multiple uplink bands are utilized in a \acrfull{tdd} manner to enhance the positioning precision and eventually leads to better communication capacity.
The \acrshort{mb} technique is a promising alternative of \acrshort{uwb}~\cite{Jafari2015TDOA}.
As the feasibility of \acrshort{mb} \acrshort{me} has been addressed in literature~\cite{Jagadeesan2015}, the \acrshort{mb} \acrshort{rhs} becomes a promising solution to get around the bandwidth limitation and achieve high positioning precision.
To fully exploit the strength of the proposed system, we propose a \emph{\gls{Protocol}} for the system.
Based on the protocol, we establish the channel model of the system in a rich scattering environment.
Then, we derive the positioning precision of the system in terms of the \acrfull{crlb} given the \acrshort{and} beamforming of the \acrshort{rhs}.

Nevertheless, as multiple bands are adopted, the large number of controlling variables in the \acrshort{and} beamforming is further multiplied, which cannot be handled by conventional small-scale optimization algorithms efficiently. 
To handle this challenge, we derive a closed-form formula of the \acrshort{crlb}'s gradient and then propose an efficient \acrfull{bcd} algorithm where the digital and analog beamforming variables are optimized alternatingly.
Simulation results verify that with the proposed algorithm, the \acrshort{mb} \acrshort{rhs}-based system achieves lower \acrshort{crlb} in positioning, which leads to lower positioning error and much smaller communication capacity loss compared with benchmarks.

The rest of the paper is organized as follows.
In Section~\ref{sec: sys mod}, we propose the system model, the protocol, and the channel model of the \gls{SysName}.
In Section~\ref{sec: problem formulation}, the \acrshort{crlb} of positioning is derived, and the problem for beamforming optimization is formulated.
In Section~\ref{sec: alg design}, the gradients of \acrshort{crlb} are derived, and the algorithm to solve the formulated problem is designed.
Simulation results are provided in Section~\ref{sec: simulation result}, and a conclusion is drawn in Section~\ref{sec: conclu}.

\emph{Notations}: 
$\overline{(\cdot)}$, $(\cdot)^{\top}$, and $(\cdot)^{\hil}$ denote the conjugate, transpose, and Hermitian transpose. 
$\odot$ and $\otimes$ denote the Hadamard and Kronecker products. 
$\trace(\cdot)$ and $\diag(\cdot)$ return the trace and the main diagonal vector of a matrix.
$\{\bm X_i\}_i$ is the set of $\bm X_i$ for all $i$ in its value range.
$[\bm X]_{i}$ and $[\bm X]_{i,j}$ indicate the $i$-th row vector and the $(i,j)$-th element of $\bm X$.
Besides, $[\bm X]_{i:j}$ is the sub-matrix of $\bm X$ composed of its $i$-th to $j$-th row vectors.
Moreover, $\bm I_{N}$ denotes the $N$-dim unit matrix.
$\bm 1_{N}$ and $\bm J_{N}$ denote all-ones $N$-dim vector and $N\times N$ matrix, respectively.
$\Re(\cdot)$ is the real part of the argument.

\endinput