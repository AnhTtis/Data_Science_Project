%%%%%%%%%%%%%%%
\section{Problem Formulation}
%%%%%%%%%%%%%%%
\label{sec: problem formulation}

In this section, we formulate the \acrshort{and} beamforming optimization problem of the \gls{SysName}.

As in~\cite{Zohair2021Near}, we adopt the average \acrshort{crlb} in the \acrshort{roi} to evaluate the system's positioning performance, which indicates the lower-bound of the \acrfull{mse} of positioning.
Given the \acrshort{and} beamforming of the \acrshort{rhs}, the \acrshort{crlb} for the system to position a user at $\gls{userPos}$ can be derived by Proposition~\ref{prop: crlb expression}.

\begin{proposition}
\label{prop: crlb expression}
	The \acrshort{crlb} for \gls{SysName} to position a user at $\gls{userPos}\in \mathbb R^{1\times 3}$ can be calculated by
	\beq
\label{equ: general crlb expression}
\mathrm{CRLB}(\gls{userPos}) = \sum_{u=1}^3\big[
\bm I^{-1}_{\mathrm{FIM}} (\gls{userPos})
\big]_{uu}.
\eeq
Here, $\bm I_{\mathrm{FIM}}(\gls{userPos})\in\mathbb R^{3\times 3}$ is the \acrfull{fim} of $\gls{recvSigMat}$ w.r.t. $\gls{userPos}$, whose element can be calculated as:
\begin{align}
\label{equ: fim matrix}
& [\bm I_{\mathrm{FIM}} (\gls{userPos})]_{u,v} \!=\!  2\Re\Big(\sum_{i=1}^{\gls{numBand}}\! (\frac{\partial \hat{\bm y}_i}{\partial p_u^{\user}})^{\hil}\! \bm \varLambda_i^{-1} \!(\frac{\partial \hat{\bm y}_i}{\partial p_v^{\user}})\Big),\\
&   {\partial\hat{\bm y}_i \over \partial p_{u}^{\user}}= \diag\Big( (  \dot{\bm H}^{\los}_{i,u} \otimes \bm 1_{\gls{numFrame}} ) \gls{holoMat_i}^{\tran}\Big),~\dot{\bm H}^{\los}_{i,u}= {\partial\bm H_i^{\los}\over \partial p_u^{\user}},\\
&  \bm\varLambda_i = ( \bm K_{\mathrm{f},i} \!\otimes\! \bm J_{\gls{numFrame}})\!\odot\! ( \bm J_{\gls{numSBand}}\!\otimes\!\bm K_{\mathrm{t},i} ) \!\odot\! (\bm T_i \bm V_i \bm T_i^{\hil}) \!+\! \gls{noisePower}\bm I,\\
\label{equ: decorrelation matrices}
& [\bm K_{\mathrm{f},i}]_{j_1, j_2} =  \rho_{\mathrm{f},i}(j_1, j_2), ~[\bm K_{\mathrm{t},i}]_{q_1, q_2} =  \rho_{\mathrm{t},i}(q_1, q_2).
\end{align}
\end{proposition}
\begin{IEEEproof}
The \acrshort{fim} in~(\ref{equ: fim matrix}) can be obtained by substituting the channel model in~(\ref{equ: recvSigVec_i}) into the \acrshort{fim} Eq.~(6.55) in~\cite{Schreier2010Statistical}, and the \acrshort{crlb} can be derived based on Eq.~(27) in~\cite{Elzanaty2021Reconfigurable}.
\end{IEEEproof}


Then, the \acrshort{and} beamforming optimization problem of the system for \acrshort{crlb} minimization can be formulated as:
\begin{subequations}
\label{optprob: P1}
	\begin{align}
\text{(P1)}:~\min_{\gls{sigSet}, \gls{codeSet}}~&  \mathbb E_{\gls{userPos} \in \gls{userDistribu}}\big(\mathrm{CRLB}(\gls{userPos})\big),\\
\text{s.t.}~\qquad & \text{(\ref{equ: matrix forms elements})$\sim$(\ref{equ: decorrelation matrices}),}\\
& \sum_{j=1}^{\gls{numSBand}}(\gls{sigVec_ijq})^{\hil}\gls{sigVec_ijq} = \gls{userMaxPower},\\
& \bm 0\preceq \gls{codeMat_i} \preceq \bm 1.
\end{align}
\end{subequations}

The challenges in solving~(P1) lie in the following aspects.
\emph{Firstly}, due to the non-convex relationship between \gls{sigSet}, \gls{codeSet} and the \acrshort{crlb}, (P1) is non-convex and NP-hard.
\emph{Secondly}, the objective function (\ref{optprob: P1}a) is of high computational cost since it is calculated over the \acrshort{roi}.
\emph{Thirdly}, as multiple bands are used, the number of variables can be very large, i.e., $\gls{numBand}\gls{numFrame}(\gls{numSBand}\gls{numFeed}+\gls{numElem})$, which makes it hard for conventional algorithms to handle efficiently.
Therefore, it is necessary to design novel algorithm to solve (P1) efficiently.


\endinput