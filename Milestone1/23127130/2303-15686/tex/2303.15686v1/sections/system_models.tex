%%%%%%%%%%%%%%%
\section{System Model}
%%%%%%%%%%%%%%%
\label{sec: sys mod}

In this section, we introduce the \gls{SysName}.
We describe its components, design a working protocol for it, and establish the channel model of the system.


\begin{figure}[!t]
\centerline{ \includegraphics[width=0.8\linewidth]{./formal_figures/1_system_model.pdf} }
\caption{\gls{SysName}}
\label{fig: 1 sys mod}
\end{figure}

%==============
\subsection{System Components}
%==============
As shown in Fig.~\ref{fig: 1 sys mod}, the system contains a \acrshort{bs} equipped with a \acrshort{rhs} serving multiple users in a \acrfull{roi}.
Each user has an omni-directional antenna for \acrshort{tx} and \acrshort{rx}.
The \acrshort{bs} uses the \acrshort{rhs} for \acrfull{rx} and \acrfull{tx} \acrfull{ofdm} signals for positioning and transmitting data, while it also has an omni-directional antenna to broadcast beacons for link construction.
The \acrshort{bs} is connected to the \acrshort{rhs} via $\gls{numFeed}$ feeds, and each feed can send or receive signals from all the $\gls{numElem}$ \acrshort{me}s through on-board signal propagation.
The $\gls{numElem}$ \acrshort{me}s are arranged as a square with the interval between adjacent \acrshort{me}s being \gls{elemInterval}.
Each \acrshort{me} can be electronically configured into multiple \emph{states}, each with its own radiation coefficient.
The radiation coefficients of all the \acrshort{me}s are referred to as the \emph{configuration} of the \acrshort{rhs}, which is denoted by vector $\bm c\mathbb \in [0,1]^{1\times\gls{numElem}}$~\cite{Zhang2022Holographic}.

The \acrshort{rhs} and \acrshort{bs} are able to transmit and receive signals in $\gls{numBand}$ bands in a \acrshort{tdd} manner by adjusting their configurations.
Each band $i$~($i \in \{1,...,\gls{numBand}\}$) is centered at frequency $\gls{cenFreq_i}$ and composed of $\gls{numSBand}$ orthogonal sub-bands, each with center frequency $\gls{subFreq_ij}$~($j\in\{1,...,\gls{numSBand}\}$) and bandwidth $\gls{subBandwidth}$.
In each band, the \acrshort{bs} is able to apply \acrshort{and} beamforming. % for positioning and communication enhancement.
To be specific, the \acrshort{bs} performs the analog beamforming by controlling the configuration $\bm c$ of the \acrshort{rhs}, and performs the digital beamforming through a weighted combination of the \acrshort{tx}/\acrshort{rx} symbols for different feeds, with combining vector denoted by $\bm s\in \mathbb C^{1\times\gls{numFeed}}$.
The \acrshort{and} beamforming capability enables the \acrshort{bs} to steer \acrshort{tx}/\acrshort{rx} beams, enhancing positioning and communication.

A user positioning process is performed by the \acrshort{bs} processing the \acrshort{rx} signals from the users.
The users are within the \acrshort{roi}, which is modeled as a 3D spatial region with dimensions $\gls{roi_sideLen_x}\times\gls{roi_sideLen_y}\times \gls{roi_sideLen_z}$ m$^3$ and is rich in scatterers.
The probability distribution of user position within the \acrshort{roi} is $\gls{userDistribu}$, and the maximum speed of each user is $\gls{userMaxSpeed}$.
We assume that $\gls{userMaxSpeed}$ is not large and thus the users' positions can be considered fixed during each positioning process.


%==============
\subsection{Protocol Design}
%==============
\begin{figure}[!t]
\centerline{ \includegraphics[width=1\linewidth]{./formal_figures/2_protocol.pdf} }
\vspace{-0.5em}
\caption{Position-then-transmit protocol}
\label{fig: 2 protocol}\
\vspace{-0.5em}
\end{figure}

We propose a \gls{Protocol} to coordinate the system.
As shown in Fig.~\ref{fig: 2 protocol}, the protocol contains three phases, i.e., the \emph{beaconing phase}, the \emph{positioning phase}, and the \emph{transmission phase}.
In the beaconing phase, the \acrshort{bs} broadcasts beacons for link construction. 
The users who want to communicate with the \acrshort{bs} reply acknowledgement frames to the \acrshort{bs}.
The \acrshort{bs} selects one of the users who have constructed an available link, responds to it, and proceeds to the next phase.

In the positioning phase, the selected user transmits signals to the \acrshort{bs} through \gls{numBand} bands sequentially, and the \acrshort{bs} receives the signals for $\gls{numFrame}$ frames in each band.
Specifically, in the frame~$q$~($q\in\{1,...,\gls{numFrame}\}$) of band $i$, the configuration of the \acrshort{rhs} is $\gls{codeVec_iq}$, and the combining vector is $\gls{sigVec_ijq}$ for sub-band $j$ of band $i$.
Besides, the combining vector of the \acrshort{bs} is bound by the maximum power \gls{userMaxPower}.
Denoting the received signals of the \acrshort{bs} in sub-band $j$ of band $i$ over the $\gls{numFrame}$ frames as vector $\bm y_{i,j}\in\mathbb C^{1\times \gls{numFrame}}$, the received signals in this phase can be arranged as matrix $\gls{recvSigMat}\in\mathbb C^{\gls{numBand}\times \gls{numSBand}\gls{numFrame}}$, whose block-wise elements are
$[\gls{recvSigMat}]_{i,j} = \bm y_{i,j}$.
At the end of this phase, the \acrshort{bs} uses a \emph{positioning function} to estimate the position of the user, and the result  is denoted by \gls{userPos_est}.
Moreover, despite incurring additional overhead in this phase, it should be noted that the user's transmission can also be used for sensing purposes:
A user can function as a monostatic radio-frequency sensor for the surrounding environment and motion~\cite{Chen2022ISACoT}. 
In this regard, this phase can be integrated into a higher-level ISAC network, and the overhead can be justified.

In the transmission phase, the \acrshort{bs} steers its beam to maximize the channel capacity towards \gls{userPos_est}. The \acrshort{bs} and the user start downlink/uplink transmission over one of the \gls{numBand} bands.

%==============
\subsection{Channel Model}
%==============
We establish the channel model by analyzing the equivalent base-band expression of the \acrshort{bs}'s received signals when a user at \gls{userPos} is transmitting.
Without loss of generality, we model the received signal in sub-band~$j$ of band~$i$ at frame~$q$.
Based on~\cite{goldsmith2005wireless}, for \acrshort{me}~$m$~($m\in \gls{valueSet_m}$), its received signal can be expressed as
\begin{align}
\label{equ: tau i j m}
& \tau_{i,j,m}^{(q)} = (\gls{losgain_ijm}(\gls{userPos}) + \gls{mpgain_ijmq}) \cdot x,
\end{align}
where $x$ indicates the \acrshort{tx} symbol of the user. 
Here, $\gls{losgain_ijm}$ is the gain of the \acrfull{los} path from the user at \gls{userPos} to the \acrshort{me} $m$, which can be expressed as 
\begin{align}
& \gls{losgain_ijm}(\gls{userPos}) = \frac{\gls{speedLight} \cdot g_i^{\elem} \cdot g_i^{\user} \cdot \exp(
-\iu \frac{2\pi f_{i,j}}{\gls{speedLight}} \cdot \|\gls{userPos} - \gls{posElem_m}\|_2
) }{4\pi f_{i,j} \cdot \|\gls{userPos} - \gls{posElem_m}\|_2}, \nonumber
\end{align}
where $\gls{speedLight}$ is the speed of light, 
$\iu$ is the imaginary unit, 
$g_i^{\user}$ and $g_i^{\elem}$ denote the gains of user's antenna and the \acrshort{me}, respectively, 
and $\gls{posElem_m}$ is the position of \acrshort{me}~$m$.

Besides, in~(\ref{equ: tau i j m}), $\gls{mpgain_ijmq}$ denotes the overall multipath gain.
Based on the multi-path gain model for \acrshort{ofdm} signals in rich-scattering environments~\cite{goldsmith2005wireless}, we can model $\gls{mpgain_ijmq}$ as a complex random variable following \acrfull{wss} Gaussian distribution.
Specifically, denoting $\gls{mpgainVec_ijq} = (h^{\mulpath,(q)}_{i,j,1},...,h^{\mulpath,(q)}_{i,j,\gls{numElem}})^\tran$ and with the help of~\cite{Barriac2006Space}, $\gls{mpgainVec_ijq}\sim \mathcal{CN}(0, \gls{covmat_i})$, where covariance matrix $\gls{covmat_i}$ can be obtained by the expectation of the outer product of \acrshort{rhs}'s array response $\gls{angularResponse_i}(\bm \theta)\in \mathbb C^{\gls{numElem}}$ over the angular domain, i.e., 
\beq
\label{equ: V i}
\bm V_i \!=\! \mathbb E \!\left( 
\gls{angularResponse_i}(\bm \theta) \gls{angularResponse_i}(\bm \theta)^{\hil} \right) 
\!=\!\oint \!\gls{angularResponse_i}(\bm \theta) \gls{angularResponse_i}(\bm \theta)^{\hil}\! P_{\mathrm{pap}}\!(\bm \theta) \!\dd \!\bm \theta, 
\eeq
where $[\gls{angularResponse_i}(\bm \theta)]_{m} = \exp(\iu \frac{2\pi \gls{cenFreq_i}}{\gls{speedLight}} (\gls{posElem_m} - \gls{posElem_1}) \cdot \hat{\bm n}(\bm \theta)) \cdot g^{\elem}_i$ with $\hat{\bm n}(\bm \theta)$ being the unit normal vector for $\bm \theta$, 
and $P_{\mathrm{pap}}(\bm \theta)$ is the \emph{power-angle profile} of the \acrshort{rhs}.
Based on~\cite{Barriac2006Space}, we model the covariance between the multi-path gains in the same band as 
\begin{align}
&\mathbb E\left[\bm h_{i, j_1}^{\mulpath, (q_1)}\big(\bm h_{i, j_2}^{\mulpath, (q_2)}\big)^{\hil}\right] = \gls{covCoefFuncFreq_i}(j_1,j_2)\cdot \gls{covCoefFuncTime_i}(q_1, q_2)\cdot \bm V_i,\nonumber\\
\label{equ: covCoefFuncFreq}
&\quad \gls{covCoefFuncFreq_i}(j_1,j_2) = (1+\iu 2\pi \gls{rmsValue_i} (f_{i,j_1} - f_{i, j_2}))^{-1},\\
\label{equ: covCoefFuncTime}
&\quad \gls{covCoefFuncTime_i}(q_1, q_2) = \mathcal J_0(2\pi \gls{dopplerFreq_i} (q_1 - q_2)\gls{durFrame}).
\end{align}

In~(\ref{equ: covCoefFuncFreq}), $\gls{rmsValue_i}$ indicates the \emph{root mean square~(rms) power delay spread} of band~$i$.
In~(\ref{equ: covCoefFuncTime}), $\mathcal J_0$ is the \emph{zeroth-order Bessel function of the first kind}, $\gls{dopplerFreq_i}$ indicates the maximum Doppler frequency, which can be calculated by $\gls{dopplerFreq_i} = \gls{userMaxSpeed} \gls{cenFreq_i} / \gls{speedLight}$, and \gls{durFrame} denotes the time duration of a frame.
We assume that different bands have larger spectral intervals than the coherence bandwidth of the channel, and thus the multi-path gains of different bands have zero covariance.

Then, based on~(\ref{equ: tau i j m}), the received signal at feed $k$~($k\in\{1,...,\gls{numFeed}\}$) is
\begin{align}
\label{equ: y i j k q}
& y_{i,j,k}^{(q)} = \sum_{m=1}^{\gls{numElem}} \tau_{i,j,m}^{(q)} \cdot \kappa(f_{i,j}, \bm p^{\mathrm{E}}_{m}, \bm p^{\mathrm{F}}_{k}),
\end{align}
where $\kappa(f_{i,j}, \bm p^{\mathrm{E}}_{m}, \bm p^{\mathrm{F}}_{k})$ is the onboard propagation gain.
Based on~\cite{Zhang2022Holographic}, it can be calculated as
\begin{align}
\label{equ: }
& \kappa(f, \bm p_{\mathrm{s}}, \bm p_{\mathrm{d}}) = \exp( -\iu \cdot\frac{2\pi n_{\mathrm{r}}f}{\gls{speedLight}} \cdot \|\bm p_{\mathrm{d}} - \bm p_{\mathrm{s}}\|_2 ),
\end{align}
where $n_{\mathrm{r}}$ is the relative refractive index of the \acrshort{rhs} board.

Combine the received signals at the $\gls{numFeed}$ feeds with weight $\gls{sigVec_ijq}$, and the received signal at the \acrshort{bs} can be modeled as
\begin{align}
\label{equ: y i j q}
& y_{i,j}^{(q)} =  \sum_{k=1}^{\gls{numFeed}} s_{i,j,k}^{(q)}y_{i,j,k}^{(q)} + e,
\end{align}
where $e \sim \mathcal{CN}(0, \gls{noisePower})$ is the thermal noise.
Given power spectral density of noise being \gls{noisePowerDensity}, noise power $\gls{noisePower} = \gls{noisePowerDensity}\gls{subBandwidth}$.

Finally, based on~(\ref{equ: tau i j m}),~(\ref{equ: y i j k q}), and~(\ref{equ: y i j q}), the received signals in the positioning phase can be formulated into matrix form, i.e., 
\begin{align}
\label{equ: recvSigVec_i}
& [\gls{recvSigMat}]_{i} =  \diag\big(\big(\bm  H_i^{\los}(\gls{userPos}) \otimes \bm 1_{\gls{numFrame}} + \bm H_i^{\mulpath}\big) \gls{holoMat_i}^\tran \big) x  + \bm e,
\end{align}
where the elements of the above matrices can be expressed as
\begin{align}
& [\bm H_i^{\los}(\gls{userPos})]_{j, m} \!=\! \gls{losgain_ijm}(\gls{userPos}),[\bm H_i^{\mulpath}]_{(q-1)\gls{numSBand}\!+\!j, m} \!=\! \gls{mpgain_ijmq}, \nonumber\\
\label{equ: matrix forms elements}
&[\bm T_i]_{j} = \bm C_i \odot \big( \bm S_{i,j} \bm B_{i,j} \big),~[\bm S_{i,j}]_{q} = \bm s_{i,j}^{(q)}, \\
&[\bm C_{i}]_{q} = \bm c_i^{(q)}, [\bm B_{i,j}]_{k,m} = \kappa(f_{i,j}, \bm p^{\mathrm{E}}_{m}, \bm p^{\mathrm{F}}_{k}). \nonumber
\end{align}
Based on channel reciprocity, when the \acrshort{bs} transmits, the received signals at the user can also be calculated by~(\ref{equ: recvSigVec_i}).

Furthermore, we model the channel capacity in the transmission phase of band $i$ given unit symbol.
Assuming that the \acrshort{bs} adopts combining vectors $\hat{\bm s}_{i,1}, ... \hat{\bm s}_{i, \gls{numSBand}}$ and \acrshort{rhs} configuration $\hat{\bm c}_i$, the channel capacity of the band $i$ can be calculated by 
\beq
\label{equ: capacity}
R_i = \sum_{j=1}^{\gls{numSBand}}\!W\!\log_2 (1 + \frac{\|\hat{\bm t}_{i,j}[\bm H_i^{\los}(\gls{userPos})]_{j}\|_2^2 + \hat{\bm t}_{i,j}\bm V_i \hat{\bm t}_{i,j}^{\hil}}{\gls{noisePower}}),
\eeq
where the received signal power of both the LoS and the multi-path channels are considered, and $\hat{\bm t}_{i,j} = \hat{\bm c}_i \odot \hat{\bm s}_{i,j}\bm B_{i,j}$.


\endinput