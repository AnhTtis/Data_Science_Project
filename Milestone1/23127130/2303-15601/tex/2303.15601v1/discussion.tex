\section{Discussion and open problems}\label{sec:Disc}
It is natural to ask whether the assumptions of our main Theorem \ref{thm:CLT} are needed. The very first obstacle is given by the use of Theorem \ref{thm:poissonapp}. One could keep track of the dependence of $m$ (resp., $\epsilon$) in their error bounds, and thus extend our result by allowing some moderate growth (resp., decay).  \\ \\
It is worth remarking that Lemma \eqref{lem:TotalScore} holds for all decks, while the bound in Lemma \eqref{lem:berryEssen} continues to hold as long as the conditional variance goes to infinity. In particular, under this condition we are guaranteed to have convergence to a mixture of normal random variables. Notice that the fact that $m$ remains bounded plays no role in this part of the proof, while we needed our condition on $\epsilon$. While this may not be sharp, some care has to be taken if one type of card has a much higher multiplicity than all the others: for instance, in the case of finite $n$, this may be an obstacle to the convergence to a mixture of normal (see \cite{DG81} for the case where $n=2$). 
\\ \\
As for the convergence to a normal random variable (i.e., a trivial mixture) our method relies on $m$ being finite, and it is an interesting problem to determine if this is a true limitation. This is intimately related to the understanding of the concentration properties of $\widetilde W_j$, i.e., the ``number of ties at top", when $j$ grows. At least for some regimes of $j$, a closely related result is available in \cite{ottolini2020oscillations}, where asymptotic for $\widetilde W_j$ are shown if the hypergeometric process is replaced by the multinomial process (i.e., if cards are reinserted in the deck).
\\ \\
It is worth pointing out that, while there is hope for a CLT to hold even when $m$ grows much faster than $n$, the asymptotic of the expected value (first part of Theorem \ref{thm:CLT}) eventually breaks down, as shown in \cite{ottolini2022guessing}.
\section{Acknowledgements}
We thank Persi Diaconis for suggesting the problem, and Jimmy He for the idea behind Lemma \ref{lem:TotalScore}.