\documentclass[pra,reprint,twocolumn,amsmath,amssymb,superscriptaddress]{revtex4-1}
\usepackage{color}

\usepackage{graphicx}
\usepackage{epsfig}
\usepackage{bm}
\usepackage{soul}
\usepackage[normalem]{ulem}

\newcommand{\bra}[1]{\ensuremath{\left\langle{#1}\right\vert}}
\newcommand{\braket}[1]{\ensuremath{\left\langle{#1}\right\rangle}}
\newcommand{\ket}[1]{\ensuremath{\left|{#1}\right\rangle}}
\newcommand{\ad}{\ensuremath{a^\dagger}}
\newcommand{\fref}[1]{Fig.~\ref{#1}}
\def\bea{\begin{eqnarray}}
\def\eea{\end{eqnarray}}
\newcommand{\eref}[1]{Eq.~(\ref{#1})}

\newcommand{\jl}[1]{{\color{red}{#1}}}
\newcommand{\mbp}[1]{{\color{magenta}{#1}}}
\newcommand{\nl}[1]{{\color{blue}{#1}}}
\newcommand{\dt}[1]{{\color{green}{#1}}}
\newcommand{\sfh}[1]{{\color{cyan}{#1}}} 
\raggedbottom

\newcommand{\DT}[2]{\textcolor{green}{\textbf{Dario:} \sout{#1} {#2} } }  

\begin{document}


\begin{widetext}

\newpage
%\vspace{20cm}

\begin{center}
{\large\bf Supplemental Material: Systematic coarse-graining of environments for \\ the non-perturbative simulation of open quantum systems}
\end{center}

\section{Phonon spectral density of the FMO complex}

The experimentally estimated phonon spectral density of the FMO complex consists of a quasi-continuous spectrum of low-frequency protein modes and 62 high-frequency intra-pigment vibrational modes~\cite{SI_RatsepJL2007}. The low-frequency spectrum is modelled by the Adolphs-Renger (AR) spectral density
\begin{equation}
	J_{\rm AR}(\omega)=\frac{S}{s_1+s_2}\sum_{i=1}^{2}\frac{s_i}{7! 2 \omega_i^4}\omega^5 e^{-(\omega/\omega_i)^{1/2}},
	\label{eq:B777_SD}
\end{equation}
where $S=0.29$, $s_1=0.8$, $s_2=0.5$, $\omega_1=0.069\,{\rm meV}$ and $\omega_2=0.24\,{\rm meV}$. The intra-pigment vibrational modes are modelled by a sum of 62 Lorentzian functions
\begin{equation}
    J_{h}(\omega) = \sum_{k=1}^{62} \frac{4 \omega_k s_k \gamma_k (\omega_{k}^2+\gamma_{k}^2)\omega}{\pi((\omega+\omega_k)^{2}+\gamma_{k}^2)((\omega-\omega_k)^{2}+\gamma_{k}^2)},
\end{equation}
where the $k$-th Lorentzian peak is centered at vibrational frequency $\omega_k$ with a height being proportional to the Huang-Rhys factor $s_k$. The experimentally estimated values of the vibrational frequencies and Huang-Rhys factors~\cite{SI_RatsepJL2007} are summarized in Table~\ref{Table_intrapigment}. The width of the Lorentzian functions, determining the vibrational damping rate of the intra-pigment modes, is taken to be $\gamma_k = 1\,{\rm ps}^{-1} \approx 5\,{\rm cm}^{-1}$. The total phonon spectral density of the FMO complex is given by $J(\omega)=J_{\rm AR}(\omega)+J_{h}(\omega)$.

\begin{table}[b]
\caption{Vibrational frequencies $\omega_k$ and Huang-Rhys factors $s_k$ of the 62 intra-pigment modes of the FMO complex~\cite{SI_RatsepJL2007}.}
\label{Table_intrapigment}
\begin{ruledtabular}
\begin{tabular}{llllllllllllll}
$k$ & 1 & 2 & 3 & 4 & 5 & 6 & 7 & 8 & 9 & 10\\
\hline
$\omega_k\,[{\rm cm}^{-1}]$ & 46 & 68 & 117 & 167 & 180 & 191 & 202 & 243 & 263 & 284 \\
$s_k$ & 0.011 & 0.011 & 0.009 & 0.009 & 0.010 & 0.011 & 0.011 & 0.012 & 0.003 & 0.008 \\
\hline\hline
$k$ & 11 & 12 & 13 & 14 & 15 & 16 & 17 & 18 & 19 & 20 \\
\hline
$\omega_k\,[{\rm cm}^{-1}]$ & 291 & 327 & 366 & 385 & 404 & 423 & 440 & 481 & 541 & 568 \\
$s_k$ & 0.008 & 0.003 & 0.006 & 0.002 & 0.002 & 0.002 & 0.001 & 0.002 & 0.004 & 0.007\\
\hline\hline
$k$ & 21 & 22 & 23 & 24 & 25 & 26 & 27 & 28 & 29 & 30 \\
\hline
$\omega_k\,[{\rm cm}^{-1}]$ & 582 & 597 & 630 & 638 & 665 & 684 & 713 & 726 & 731 & 750 \\
$s_k$ & 0.004 & 0.004 & 0.003 & 0.006 & 0.004 & 0.003 & 0.007 & 0.010 & 0.005 & 0.004 \\
\hline\hline
$k$ & 31 & 32 & 33 & 34 & 35 & 36 & 37 & 38 & 39 & 40\\
\hline
$\omega_k\,[{\rm cm}^{-1}]$ & 761 & 770 & 795 & 821 & 856 & 891 & 900 & 924 & 929 & 946 \\
$s_k$ & 0.009 & 0.018 & 0.007 & 0.006 & 0.007 & 0.003 & 0.004 & 0.001 & 0.001 & 0.002\\
\hline\hline
$k$ & 41 & 42 & 43 & 44 & 45 & 46 & 47 & 48 & 49 & 50 \\
\hline
$\omega_k\,[{\rm cm}^{-1}]$ & 966 & 984 & 1004 & 1037 & 1058 & 1094 & 1104 & 1123 & 1130 & 1162 \\
$s_k$ & 0.002 & 0.003 & 0.001 & 0.002 & 0.002 & 0.001 & 0.001 & 0.003 & 0.003 & 0.009 \\
\hline\hline
$k$ & 51 & 52 & 53 & 54 & 55 & 56 & 57 & 58 & 59 & 60 \\
\hline
$\omega_k\,[{\rm cm}^{-1}]$ & 1175 & 1181 & 1201 & 1220 & 1283 & 1292 & 1348 & 1367 & 1386 & 1431 \\
$s_k$ & 0.007 & 0.010 & 0.003 & 0.005 & 0.002 & 0.004 & 0.007 & 0.002 & 0.004 & 0.002 \\
\hline\hline
$k$ & 61 & 62 \\
\hline
$\omega_k\,[{\rm cm}^{-1}]$ & 1503 & 1545 \\
$s_k$ & 0.003 & 0.003 \\
\end{tabular}
\end{ruledtabular}
\end{table}

In the main manuscript, an effective phonon spectral density constructed based on the bath correlation function of the full FMO spectral density $J_{\rm AR}(\omega)+J_{h}(\omega)$ is introduced, which can reproduce numerically exact absorption spectra of dimers computed based on the full FMO phonon spectral density. In HEOM simulations, the effective phonon spectral density was modelled by the sum of the AR spectral density and five Lorentzian functions
\begin{equation}
    J_{\rm effective}(\omega) = J_{\rm AR}(\omega)+\sum_{k=1}^{5} \frac{4 \Omega_k S_k \Gamma_k (\Omega_{k}^2+\Gamma_{k}^2)\omega}{\pi((\omega+\Omega_k)^{2}+\Gamma_{k}^2)((\omega-\Omega_k)^{2}+\Gamma_{k}^2)},
\end{equation}
where the parameters $\{\Omega_k,S_k,\Gamma_k\}$ of the Lorentzian peaks are summarized in Table~\ref{Table_L5}. In DAMPF simulations~\cite{SI_SomozaPRL2019}, the 62 intra-pigment modes were effectively described by five pseudo-modes~\cite{SI_tama18,SI_mascherpa20} with parameters in Table~\ref{Table_L5} and the AR spectral density was modelled by an additional pseudo-mode with $\Omega_6 = 160\,{\rm cm}^{-1}$, $S_6 = 0.164$ and $\Gamma_6=133\,{\rm cm}^{-1}$. The temperature of the Markovian baths coupled to the pseudo-modes was taken to be $T=77\,{\rm K}$. We note that a single pseudo-mode is sufficient to describe the influence of the AR spectral density on absorption spectra, as the simulated results do not show any appreciable changes even if the AR spectral density is modelled by 3, 5, or 15 pseudo-modes for a more accurate description (not shown here).

\begin{table}[t]
\caption{The parameters $\{\Omega_k,S_k,\Gamma_k\}$ of the five Lorentzians considered in the effective phonon spectral density $J_{\rm effective}(\omega)$.}
\label{Table_L5}
\begin{ruledtabular}
\begin{tabular}{lllllllll}
$k$ & 1 & 2 & 3 & 4 & 5 \\
\hline
$\Omega_k\,[{\rm cm}^{-1}]$ & 247 & 763 & 1175 & 1356 & 1521 \\
$S_k$  & 0.056 & 0.133 & 0.049 & 0.019 & 0.006 \\
$\Gamma_k\,[{\rm cm}^{-1}]$  & 53 & 76 & 29 & 29 & 15 \\
\end{tabular}
\end{ruledtabular}
\end{table}

In the main manuscript, a conventional coarse-grained spectral density is introduced, consisting of the AR spectral density and a single broad Lorentzian peak
\begin{equation}
    J_{\rm conventional}(\omega) = J_{\rm AR}(\omega) + \frac{4 \tilde{\Omega} \tilde{S} \tilde{\Gamma} (\tilde{\Omega}^2+\tilde{\Gamma}^2)\omega}{\pi((\omega+\tilde{\Omega})^{2}+\tilde{\Gamma}^2)((\omega-\tilde{\Omega})^{2}+\tilde{\Gamma}^2)},
\end{equation}
where $\tilde{\Omega}=1000\,{\rm cm}^{-1}$, $\tilde{S}=0.2093$, and $\tilde{\Gamma}=(20\,{\rm fs})^{-1}\approx 265\,{\rm cm}^{-1}$.

\begin{table}[b]
\caption{${\rm Q}_y$-transition dipole moments of the FMO complex of {\it C. tepidum}.}
\label{Table_FMO}
\begin{ruledtabular}
\begin{tabular}{lllllllll}
Site & 1 & 2 & 3 & 4 & 5 & 6 & 7 \\
\hline
x-component & 0.741 & 0.857 & 0.197 & 0.799 & 0.736 & 0.135 & 0.495 \\
y-component & 0.560 & -0.503 & -0.957 & 0.533 & -0.655 & 0.879 & 0.708 \\
z-component & 0.369 & 0.107 & 0.210 & 0.276 & -0.164 & -0.456 & 0.503
\end{tabular}
\end{ruledtabular}
\end{table}

\section{Theory of absorption spectra}

Within the Franck-Condon approximation, the linear absorption spectrum of isotropic samples is described by~\cite{SI_Mukamel1995}
\begin{equation}
    A(\omega)\propto  {\rm Re}{\int_{0}^{\infty}} dt e^{i\omega t} d(t)={\rm Re}{\int_{0}^{\infty}} dt e^{i\omega t}{\rm tr}[{\bm \mu}\cdot (e^{-iHt}{\bm \mu}\rho(0)e^{iHt})],
    \label{eq:absorption}
\end{equation}
with $d(t)$ denoting dipole-dipole correlation function and $\rho(0)=\ket{g}\bra{g}\otimes\rho_{v}(T)$ an initial state where $\ket{g}$ is the global electronic ground state and $\rho_{v}(T)$ the thermal state of vibrational modes at temperature $T$~\cite{SI_Mukamel1995}. The transition dipole moment operator ${\bm \mu}$ is modelled by
\begin{equation}
    {\bm \mu}=\sum_{i=1}^{N}{\bm \mu}_{i}(\ket{\epsilon_i}\bra{g}+\ket{g}\bra{\epsilon_i}),
\end{equation}
with ${\bm \mu}_{i}$ representing the transition dipole moment of site $i$. In the FMO simulations, the transition dipole moments of sites 1-7 summarized in Table~\ref{Table_FMO} were considered. In dimer simulations, the transition dipole moments of two pigments were assumed to be mutually orthogonal (${\bm \mu}_1 \cdot {\bm \mu}_2=0$), so that both low- and high-energy excitons are bright states. The lifetime of optical coherences, ${\rm tr}_{v}[e^{-iHt}{\bm \mu}\rho(0)e^{iHt}]$, and corresponding absorption line widths are determined by the homogeneous and inhomogeneous broadenings induced by the vibronic coupling $H_{e-v}$ and static disorder, respectively.

\section{Simulation-time dependence of coarse-graining}

In the main manuscript, we considered the bath correlation function (BCF) of the full FMO complex up to $300\,{\rm fs}$ to construct an effective spectral density. This is the time scale of absorption measurements, namely the lifetime of optical coherences between electronic ground and excited states. It is found that 62 intra-pigment modes present in the actual FMO spectral density can be effectively described by five Lorentzian peaks using our course-graining scheme, which can be estimated based on the number of peaks present in the frequency spectrum of the BCF up to $300\,{\rm fs}$, as shown in Fig.~\ref{Fig1}(a).

As the time scale of interest increases, a larger number of peaks appears in the frequency spectrum of the BCF. In two-dimensional (2D) electronic spectroscopy, a molecular sample is perturbed by three femtosecond laser pulses with controlled time delays~\cite{SI_JonasARPC2003,SI_Brixner2004,SI_LimPRL2019}. The time delay $t_1$ between first and second pulses enables one to monitor the phase evolution of the optical coherences created by the first pulse, decaying within $\sim 300\,{\rm fs}$ as is the case of linear absorption. The molecular dynamics, such as energy/charge transfer and vibrational motion, is monitored as a function of the time delay $t_2$ between second and third pulses. In typical 2D experiments on PPCs, $t_2 \lesssim 1\,{\rm ps}$ has been considered to study photosynthetic energy/charge transfer and quantum coherence effects occurring on a sub-ps time scale~\cite{PanitchayangkoonPNAS2010,ColliniN2010,RomeroNP2014}. The optical coherences created by the third laser pulse induce oscillating transition dipole moments generating nonlinear 2D signals. The optical coherences decay within $\sim 300\,{\rm fs}$, described by an additional time variable $t_3$. In simulations, one needs to compute the time evolution of a molecular system as a function of time $t=t_1+t_2+t_3 \lesssim 2\,{\rm ps}$, leading to 2D electronic spectra when the time-domain simulated data are Fourier-transformed with respect to $t_1$ and $t_3$. Therefore, one needs to simulate the dynamics of a molecular system on a picosecond time scale, namely $\sim 2\,{\rm ps}$, to compute 2D electronic spectra.

In Fig.~\ref{Fig1}(b) and (c), respectively, we consider the typical time scale of the energy transfer in PPCs ($1\,{\rm ps}$) and the time scale associated with the simulations of 2D electronic spectroscopy ($2\,{\rm ps}$). It is notable that the number of peaks present in the frequency spectrum is increased from 5, via 16, to 28, as the time scale of interest increases from $300\,{\rm fs}$, via $1\,{\rm ps}$, to $2\,{\rm ps}$, but they are still smaller than the total number of intra-pigment modes in the full FMO spectral density ($M=62$). Therefore, our coarse-graining scheme becomes less efficient as the time scale of interest increases, but it still requires less computational resources than the full FMO spectral density.

\begin{figure}
\includegraphics[width=0.8\textwidth]{Fig9.pdf}
\caption{(a)-(c) Fourier transform (FT) of the bath correlation function $C(t)$ of the full FMO spectral density at $77\,{\rm K}$ weighted by a Gaussian function $e^{-t^{2}/2\tilde{\sigma}^2}$ with $3\tilde{\sigma}\in\{300\,{\rm fs}, 1\,{\rm ps}, 2\,{\rm ps}\}$. The number of peaks is increased from 5, via 16, to 28 as the time scale of interest is increased from $300\,{\rm fs}$ (absorption measurements), via $1\,{\rm ps}$ (energy transfer dynamics), to $2\,{\rm ps}$ (nonlinear spectroscopy measurements).}
\label{Fig1}
\end{figure}

\begin{figure}
\includegraphics[width=0.8\textwidth]{SI_Fig3.pdf}
\caption{Fourier transform (FT) of the bath correlation function $C(t)$ at $77\,{\rm K}$ of (a) the special pair in bacterial reaction centre, (b) WSCP from cauliflowers, and (c) PCB158c pigment in the PC645 complex from marine algae. The FT of the BCF without a Gaussian filter is shown in blue, while the FT of the BCF weighted by a Gaussian function $e^{-t^{2}/2\tilde{\sigma}^2}$ with $3\tilde{\sigma} = 300\,{\rm fs}$ ($3\tilde{\sigma} = 1\,{\rm ps}$) is shown in black (red).}
\label{Fig2}
\end{figure}

So far, we have shown that the complexity of an effective environment, capturing the full environmental effects induced by the experimentally determined phonon spectral density of the FMO complex, can be significantly reduced by considering a finite time scale. Here, we demonstrate that these observations are also valid for other photosynthetic pigment-protein complexes. In Fig.~\ref{Fig2}(a)-(c), respectively, we consider experimentally determined phonon spectral density of the special pair (SP) in bacterial reaction centre~\cite{SI_ZazubovichJPCB2001}, that of water-soluble chlorophyll-binding protein (WSCP) from cauliflower~\cite{SI_PieperJPCB2011}, and theoretically computed phonon spectral density of the PCB158c pigment present in the PC645 complex from marine algae~\cite{SI_BlauPNAS2018}. These complexes have all highly structured phonon spectral densities, as shown in blue, where the multi-peak structure originates from the discrete vibrational spectrum of the intra-pigment modes~\cite{SI_StuartWiley2005}. For all the three cases, the effective environments constructed based on a finite time scale $300\,{\rm fs}$ ($1\,{\rm ps}$) consist of $\sim 8$ ($\sim 15$) peaks, as shown in black (red), similar to the case of the FMO complex. These results demonstrate that the computational costs of non-perturbative simulations of PPCs can be reduced by our coarse-graining scheme, which is not limited to the FMO complex considered in the main manuscript. We note that our coarse-graining scheme does not depend on the electronic energy-level structures of PPCs, or more generally on the parameters of open-quantum systems. Rather, it solely depends on the BCF and finite time scale of interest.

\section{Time scale analysis of absorption spectra of the FMO complex}

In the main manuscript, we constructed an effective environment based on the FT of the BCF weighted by a Gaussian function $e^{-t^2/2 \tilde{\sigma}^2}$ where the standard deviation $\tilde{\sigma}$ is taken to be $3\tilde{\sigma} = 300\,{\rm fs}$, so that the Gaussian-weighted BCF decays within $300\,{\rm fs}$. This is based on the assumption that ensemble-averaged optical coherences decay within $300\,{\rm fs}$, which can be easily estimated based on the simulations in the absence of environments where the dephasing of optical coherences is induced solely by static disorder. In case the time scale $\tau$ of the optical coherence dynamics determining absorption lineshapes is shorter than $300\,{\rm fs}$, it is sufficient to construct an effective environment with a reduced standard deviation $3\tilde{\sigma} = \tau < 300\,{\rm fs}$ for numerically exact simulations of absorption spectra.

\begin{figure}
\includegraphics[width=0.8\textwidth]{SI_Fig4.pdf}
\caption{(a) Fourier transform of the Gaussian-filtered bath correlation function of the full FMO spectral density at $77\,{\rm K}$ where the standard deviation of the Gaussian function is taken to be $3\tilde{\sigma} = 300\,{\rm fs}$ (blue), $100\,{\rm fs}$ (red), $50\,{\rm fs}$ (orange) and $5\,{\rm fs}$ (green). (b) Absorption spectra of the FMO complex and (c) corresponding dipole-dipole correlation functions $d(t)$ obtained using the four effective environments shown in (a).}
\label{Fig3}
\end{figure}

To investigate how small standard deviations can reproduce the numerically exact absorption spectra of the FMO complex at $77\,{\rm K}$, we constructed effective environments for $3\tilde{\sigma}\in\{5,50,100,300\}\,{\rm fs}$. The results are shown in Fig.~\ref{Fig3}(a). As the standard deviation is decreased from $3\tilde{\sigma}=300\,{\rm fs}$ to $3\tilde{\sigma}=5\,{\rm fs}$, the structure of the effective environments becomes simpler and broader. As shown in Fig.~\ref{Fig3}(b), the numerically exact absorption spectra can be well reproduced when $3\tilde{\sigma}\gtrsim 100 \,{\rm fs}$. The numerical results obtained using $3\tilde{\sigma}\ll 100 \,{\rm fs}$ deviate from numerically exact ones, particularly in the high-energy domain above $12.5\times 10^3 {\rm cm}^{-1}$. These results can be rationalized by analysing optical coherence dynamics. In Fig.~\ref{Fig3}(c), the dipole-dipole correlation function $d(t)$ in the presence of static disorder is shown as a function of time $t$. Note that $d(t)$ mainly decays within $\sim 100\,{\rm fs}$, with small fluctuations persisting up to $\sim 300\,{\rm fs}$. Since absorption spectrum is proportional to the Fourier transform of $d(t)$, the large-amplitude dynamics of $d(t)$ at early times $t\lesssim 100\,{\rm fs}$ may play a crucial role in determining the absorption spectra. This is in line with the numerical results shown in Fig.~\ref{Fig3}(b) where numerically exact absorption spectra are well described by effective environments constructed using $3\tilde{\sigma}\gtrsim 100 \,{\rm fs}$. These results demonstrate that the performance of our coarse-graining scheme can be further enhanced by optimizing the time scale $\tau=3\tilde{\sigma}$ of open-system dynamics governing experimentally observable quantities, although this necessitates repetitions of non-perturbative simulations.


\begin{figure}
\includegraphics[width=0.8\textwidth]{SI_Fig5.pdf}
\caption{(a) Photon spectral density of the electromagnetic environment of a silver dimer nanoantenna embedded in a dielectric microsphere~\cite{SI_MedinaPRL2021}. (b) Corresponding bath correlation function (BCF) at zero temperature. (c) Fourier transform of the full BCF (blue) and those of Gaussian-filtered BCFs with $3\tilde{\sigma} = 400\,{\rm fs}$ (red) or $3\tilde{\sigma} = 100\,{\rm fs}$ (black). The offset values of the FTs are shifted for better visibility.}
\label{Fig4}
\end{figure}

\newpage

\section{Coarse-graining of emitter-photon interaction}

In this work, we demonstrated that the structure of phonon spectral densities of photosynthetic pigment-protein complexes, or molecular systems in general, can be systematically coarse-grained while preserving full environmental effects on a timescale of interest. Our coarse-graining scheme can be applied to any harmonic environments, which are initially in a thermal state and linearly coupled to open systems. A relevant example is provided by polaritonic systems where an emitter is coupled to photonic environments. As shown in Ref.~\cite{SI_MedinaPRL2021}, the emitter-photon coupling spectrum of polaritonic systems is characterized by a high degree of structure, reproduced in Fig.~\ref{Fig4}(a). In Ref.~\cite{SI_MedinaPRL2021}, the emitter dynamics up to $400\,{\rm fs}$ was simulated using a non-perturbative method where the continuous photonic environment is described by a finite number of discrete modes under Lindblad noise. Due to the highly structured emitter-photon coupling spectrum, characterised by sharp Fano-like profiles, the interaction between discrete modes was introduced to take into account environmental effects with high accuracy, albeit introducing increased computational costs~\cite{SI_MedinaPRL2021}. As shown in Fig.~\ref{Fig4}(b), the bath correlation function of the photonic environment mainly decays within $100\,{\rm fs}$, due to the broad frequency spectrum of the photon spectral density shown in Fig.~\ref{Fig4}(a), with small amplitudes persisting up to $400\,{\rm fs}$. This implies that the emitter dynamics is mainly governed by the bath correlation function at early times (below $100\,{\rm fs}$). In Fig.~\ref{Fig4}(b), the effective spectral densities obtained by our coarse-graining scheme with $3\tilde{\sigma}\in\{100,400\}\,{\rm fs}$ are shown. Note that the narrow Fano-like features are significantly suppressed by our scheme, which may enable one to perform numerically exact simulations without resorting to consider the interaction between discrete modes, thus reducing computational costs.

\vspace{1cm}

\section{Computational costs of DAMPF: the FMO complex and linear chains}

\begin{figure}
\includegraphics[width=0.8\textwidth]{Fig8.pdf}
\caption{(a) Optical coherence dynamics of the full FMO model ($N=7$ sites and $M=62$ intra-pigment modes per site) weighted by a Gaussian function $e^{-t^2/2\tilde{\sigma}^2}$ with $3\tilde{\sigma}=300\,{\rm fs}$, computed by DAMPF with bond dimensions $\chi=4$ and $\chi=30$. (b) The difference between DAMPF results obtained with $\chi=4$ and $\chi=30$.}
\label{Fig5}
\end{figure}

In this section we discuss the computational costs of DAMPF for absorption simulations of the FMO complex and linear chains.

\begin{figure}
\includegraphics[width=0.5\textwidth]{SI_Fig6.pdf}
\caption{(a) Computational time and (b) memory required for DAMPF simulations of the absorption spectra of the linear chains consisting of 7, 14, 21 sites (see the main text for more details).}
\label{Fig6}
\end{figure}

\begin{table}[b]
\caption{Site energies of a seven-site linear chain considered in Fig.~\ref{Fig6}.}
\label{Table_chain}
\begin{ruledtabular}
\begin{tabular}{llllllll}
Site & 1 & 2 & 3 & 4 & 5 & 6 & 7 \\
\hline
$\epsilon_i\,[{\rm cm}^{-1}]$ & 1141 & 1478 & 1031 & 1233 & 1163 & 1420 & 1282
\end{tabular}
\end{ruledtabular}
\end{table}

In Fig.~\ref{Fig5}, we show the optical coherence dynamics of the full FMO model, consisting of $N=7$ pigments and $M=62$ intra-pigment modes per site, weighted by a Gaussian function
\begin{equation*}
    A(t) = {\rm tr}[{\bm \mu}\cdot (e^{-iHt}{\bm \mu}\rho(0)e^{iHt})] e^{-t^2/2\tilde{\sigma}^2}
\end{equation*}
with $3\tilde{\sigma}=300\,{\rm fs}$. In simulations, the mean site energies of the FMO complex~\cite{SI_RengerBJ2006} were considered. Here the Gaussian function makes $A(t)$ decay within $300\,{\rm fs}$, so that the Fourier transform of $A(t)$ leads to well-defined absorption line shapes without ringing artifacts. As shown in Fig.~\ref{Fig5}, $A(t)$ computed up to $300\,{\rm fs}$ by DAMPF with bond dimensions $\chi=4$ and $\chi=30$ are quantitatively well-matched, implying that low bond dimensions $\sim 4$ are sufficient to obtain numerically converged absorption spectra of the full FMO model. For the effective environment considered in the main manuscript, we found that the bond dimension required for numerically exact DAMPF simulations of the absorption spectra of the FMO complex is $\chi=2$.

In Fig.~\ref{Fig6}, we considered linear molecular chains consisting of 7, 14, 21 sites. For simplicity, we assumed that the electronic coupling $V=50\,{\rm cm}^{-1}$ between nearby sites is uniform and each site is coupled to a local phonon environment at $77\,{\rm K}$ described by the experimentally determined phonon spectral density of the FMO complex. For the seven-site chain, we considered randomly generated site energies with detunings being smaller in magnitude than $500\,{\rm cm}^{-1}$, as summarised in Table~\ref{Table_chain}. For the longer chains, we repeatedly used the site energies in Table~\ref{Table_chain}, treating them as if multiple seven-site chains were connected. In Fig.~\ref{Fig6}(a) and (b), respectively, the computational time and memory required for DAMPF simulations of the absorption spectra of the linear chains are shown. Here the computational costs for dealing with the full FMO spectral density and effective environments, respectively, are shown in blue and red. For the linear chains consisting of 7, 14, 23 sites, respectively, the computational costs are reduced from 51, 434, 1283 min to 4, 17, 46 min (from 0.7, 7, 29 GB to 7, 105, 355 MB) when the effective environment is considered instead of the full FMO spectral density (all simulations were executed using 15 cores in an Intel Xeon 6252 Gold CPU). The bond dimensions required for numerically exact DAMPF simulations of the linear chains consisting of 7, 14, 21 sites are 3, 4, 4 (9, 10, 11), respectively, when the effective (full) environment is considered, which is larger than the FMO case. These results demonstrate that our coarse-graining scheme combined with DAMPF enables one to perform numerically exact simulations of large vibronic systems consisting of several tens of sites where each site is coupled to a highly structured phonon environment at finite temperatures.



\begin{thebibliography}{99}

\bibitem{SI_RatsepJL2007} M. R{\"a}tsep and A. Freiberg. Electron Phonon and Vibronic Couplings in the FMO Bacteriochlorophyll a Antenna Complex Studied by Difference Fluorescence Line Narrowing. J. Lumin. {\bf 127}, 251 (2007).

\bibitem{SI_SomozaPRL2019} A. D. Somoza, O. Marty, J. Lim, S. F. Huelga, and M. B. Plenio. Dissipation-Assisted Matrix Product Factorization. Phys. Rev. Lett. {\bf 123}, 100502 (2019).

\bibitem{SI_tama18} D. Tamascelli, A. Smirne, S. F. Huelga, and M. B. Plenio. Nonperturbative treatment of non-Markovian dynamics of open quantum systems. Phys. Rev. Lett. {\bf 120}, 030402 (2018).

\bibitem{SI_mascherpa20} F. Mascherpa, A. Smirne, A. D. Somoza, P. Fernández-Acebal, S. Donadi, D. Tamascelli, S. F. Huelga, and M. B. Plenio. Optimized Auxiliary Oscillators for the Simulation of General Open Quantum Systems. Phys. Rev. A {\bf 101}, 052108 (2020).

\bibitem{SI_Mukamel1995} S. Mukamel. Principles of Nonlinear Optical Spectroscopy (Oxford University Press, 1995).

\bibitem{SI_JonasARPC2003} D. M. Jonas. Two-Dimensional Femtosecond Spectroscopy. Annu. Rev. Phys. Chem. {\bf 54}, 425 (2003).

\bibitem{SI_Brixner2004} T. Brixner, T. Man{\v c}al, I. V. Stiopkin, and G. R. Fleming. Phase-Stabilized Two-Dimensional Electronic Spectroscopy. J. Chem. Phys. {\bf 121}, 4221 (2004).

\bibitem{SI_LimPRL2019} J. Lim, C. M. Bösen, A. D. Somoza, C. P. Koch, M. B. Plenio, and S. F. Huelga. Multi-Color Quantum Control for Suppressing Ground State Coherences in Two-Dimensional Electronic Spectroscopy. Phys. Rev. Lett. {\bf 123}, 233201 (2019).

\bibitem{PanitchayangkoonPNAS2010} G. Panitchayangkoon, D. Hayes, K. A. Fransted, J. R. Caram, E. Harel, J. Wen, R. E. Blankenship, and G. S. Engel. Long-Lived Quantum Coherence in Photosynthetic Complexes at Physiological Temperature. Proc. Natl Acad. Sci. USA {\bf 107}, 12766 (2010).

\bibitem{ColliniN2010} E. Collini, C. Y. Wong, K. E. Wilk, P. M. G. Curmi, P. Brumer, and G. D. Scholes. Coherently Wired Light-Harvesting in Photosynthetic Marine Algae at Ambient Temperature. Nature {\bf 463}, 644 (2010).

\bibitem{RomeroNP2014} E. Romero, R. Augulis, V. I. Novoderezhkin, M. Ferretti, J. Thieme, D. Zigmantas, and R. van Grondelle. Quantum Coherence in Photosynthesis for Efficient Solar-Energy Conversion. Nat. Phys. {\bf 10}, 676 (2014).


\bibitem{SI_ZazubovichJPCB2001} V. Zazubovich, I. Tibe, and G. J. Small. Bacteriochlorophyll a Franck-Condon Factors for the S0-S1(Qy) Transition. J. Phys. Chem. B, {\bf 105}, 12410 (2001).

\bibitem{SI_PieperJPCB2011} J. Pieper, M. R{\"a}tsep, I. Trostmann, H. Paulsen, G. Renger, and A. Freiberg. Excitonic Energy Level Structure and Pigment-Protein Interactions in the Recombinant Water-Soluble Chlorophyll Protein. I. Difference Fluorescence Line-Narrowing. J. Phys. Chem. B  {\bf 115}, 4042 (2011).

\bibitem{SI_BlauPNAS2018} S. M. Blau, D. I. G. Bennett, C. Kreisbeck, G. D. Scholes, and A. Aspuru-Guzik. Local Protein Solvation Drives Direct Down-Conversion in Phycobiliprotein PC645 via Incoherent Vibronic Transport. Proc. Natl Acad. Sci. USA {\bf 115}, E3342 (2018).

\bibitem{SI_StuartWiley2005} B. Stuart. Infrared Spectroscopy: Fundamentals and Applications (Wiley, 2005).

\bibitem{SI_MedinaPRL2021} I. Medina, F. J. García-Vidal, A. I. Fernández-Domínguez, and J. Feist. Few-Mode Field Quantization of Arbitrary Electromagnetic Spectral Densities. Phys. Rev. Lett. {\bf 126}, 093601 (2021).

\bibitem{SI_RengerBJ2006} J. Adolphs and T. Renger. How Proteins Trigger Excitation Energy Transfer in the FMO Complex of Green Sulfur Bacteria. Biophys. J. {\bf 91}, 2778 (2006).

\end{thebibliography}

\end{widetext}

\end{document}
