\documentclass[pra,reprint,twocolumn,amsmath,amssymb,superscriptaddress]{revtex4-1}
\usepackage{color}

\usepackage{graphicx}
\usepackage{epsfig}
\usepackage{bm}
\usepackage[normalem]{ulem}

\newcommand{\bra}[1]{\ensuremath{\left\langle{#1}\right\vert}}
\newcommand{\braket}[1]{\ensuremath{\left\langle{#1}\right\rangle}}
\newcommand{\ket}[1]{\ensuremath{\left|{#1}\right\rangle}}
\newcommand{\ad}{\ensuremath{a^\dagger}}
\newcommand{\fref}[1]{Fig.~\ref{#1}}
\def\bea{\begin{eqnarray}}
\def\eea{\end{eqnarray}}
\newcommand{\eref}[1]{Eq.~(\ref{#1})}

\newcommand{\jl}[1]{{\color{red}{#1}}}
\newcommand{\mbp}[1]{{\color{magenta}{#1}}}
\newcommand{\nl}[1]{{\color{blue}{#1}}}
\newcommand{\sfh}[1]{{\color{cyan}{#1}}}
\raggedbottom

\newcommand{\DT}[2]{\textcolor{green}{\textbf{Dario:} \sout{#1} {#2} } }  

\begin{document}

\title{Systematic coarse-graining of environments for \\ the non-perturbative simulation of open quantum systems}

\author{Nicola Lorenzoni}
\affiliation{Institut f\"ur Theoretische Physik und IQST, Albert-Einstein-Allee 11, Universit\"at Ulm, D-89081 Ulm, Germany}
\author{Namgee Cho}
\affiliation{Institut f\"ur Theoretische Physik und IQST, Albert-Einstein-Allee 11, Universit\"at Ulm, D-89081 Ulm, Germany}
\author{James Lim}
\affiliation{Institut f\"ur Theoretische Physik und IQST, Albert-Einstein-Allee 11, Universit\"at Ulm, D-89081 Ulm, Germany}
\author{Dario Tamascelli}
\affiliation{Institut f\"ur Theoretische Physik und IQST, Albert-Einstein-Allee 11, Universit\"at Ulm, D-89081 Ulm, Germany}
\affiliation{Dipartimento di Fisica ``Aldo Pontremoli'', Universit{\`a} degli Studi di Milano, Via Celoria 16, 20133 Milano-Italy}
\author{Susana F. Huelga}
\affiliation{Institut f\"ur Theoretische Physik und IQST, Albert-Einstein-Allee 11, Universit\"at Ulm, D-89081 Ulm, Germany}
\author{Martin B. Plenio}\email{martin.plenio@uni-ulm.de}
\affiliation{Institut f\"ur Theoretische Physik und IQST, Albert-Einstein-Allee 11, Universit\"at Ulm, D-89081 Ulm, Germany}

\begin{abstract}
The numerical description of the electron-vibrational dynamics of bio-molecular systems presents significant challenges due to the numerous required vibrational modes, necessitating the use of approximations. In this study, we demonstrate that conventional criteria for coarse-grained environments, such as conserving total electronic-vibrational coupling strength and testing accuracy based on monomer absorption spectra, do not accurately describe optical responses in multi-chromophoric systems. To address this issue, we develop a method that leverages the finite lifetime of optical coherences to construct effective phonon spectral densities that precisely reproduce absorption spectra. Our approach reduces computational costs associated with non-perturbative methods and enables simulation of optical responses under full environmental fluctuations. To demonstrate the efficacy of our method, we apply it to simulate the absorption spectrum of the entire Fenna-Matthews-Olson complex, a model multi-chromophoric system, using an experimentally determined environmental spectral density.
\end{abstract}
\maketitle

{\it Introduction.} The interaction between electronic and vibrational degrees of freedom governs various dynamical processes in molecular complexes, such as energy and charge transfer~\cite{May2011,JangPRL2004,RengerBJ2006,WomickJPCB2011,RomeroNature2017,JangRMP2018,MattioniPRX2021,SomozaarXiv2022} and chirality-induced spin selectivity~\cite{ZhangPRB2020,DuPRB2020,FranssonPRB2020,VittmannJPCL2023}. The electronic-vibrational (vibronic) interaction is typically the result of a quasi-continuous low-frequency phonon spectrum and several tens of underdamped high-frequency vibrational modes, where vibrational frequencies are comparable to or larger than room temperature. Here electronic states interact not only with near-resonant vibrational modes~\cite{PriorCH+10,ChinNP2013,TiwariPNAS2013,PlenioAH13, ChenuSR2013,RomeroNP2014,FullerNC2014,ButkusJCP2014,LimNC2015}, but also with multiple off-resonant modes when the vibronic coupling strength is beyond the weak coupling regime. This is the case even for biological photosynthetic pigment-protein complexes (PPCs)~\cite{Caycedo2022} where the vibronic coupling is generally weaker than that achievable in engineered molecular systems, such as synthetic dyes and aggregates~\cite{SpanoACR2010,FarouilJPCA2018,BarclayJPCL2022}. Non-perturbative simulations of electronic dynamics and their optical responses are a challenging task for highly structured vibrational environments beyond the weak coupling regime. As a result, while perturbative simulation methods typical in the study of open quantum systems, such as a line shape theory~\cite{NovoderezhkinJPCB2004,NovoderezhkinBJ2005,NovoderezhkinBJ2006,AbramaviciusJCP2010,NovoderezhkinBJ2010,LewisJPCA2013,DinhJCP2015,GelzinisJCP2015,RancovaJPCB2016,GelzinisJCP2017,RancovaJPCB2018,KhmelnitskiyJPCB2019}, have been widely employed, non-perturbative simulations have been made possible only by severely coarse-graining the structure of experimentally determined vibronic coupling spectra~\cite{IshizakiPNAS2009,StrumpferJCP2009,KreisbeckJPCL2012,KreisbeckJCTC2014,BlauPNAS2018}. The coarse-grained phonon spectral densities have been mainly constructed based on two criteria~\cite{BlauPNAS2018}, the conservation of the total vibronic coupling strength, quantified by reorganization energy, and the agreement of monomer absorption spectra computed based on the actual and coarse-grained vibronic coupling spectra, which can be readily computed in a non-perturbative manner. However, the validity of the conventional coarse-graining scheme has never been rigorously tested for multi-chromophoric systems like dimers due to the high computational costs of non-perturbative methods.

In this work, we employ HEOM~\cite{Caycedo2022,TanimuraJPSJ1989} as a non-perturbative simulation tool to demonstrate that the conventional coarse-graining scheme can fail to describe correctly absorption spectra of dimeric electronic systems coupled to the experimentally estimated phonon spectral density of the FMO complex. By leveraging the observation that accurate evaluation of absorption spectra only requires knowledge of short-time electronic system dynamics, we develop a reliable method for constructing coarse-grained environments that reproduce exact short-time dynamics and absorption spectra with great accuracy. Our method enables capturing the influence of environmental fluctuations on electronic system dynamics with significantly reduced computational costs. Importantly, our approach extends beyond absorption simulations to model the dynamics and spectral responses of open quantum systems more broadly based on existing numerical methods.

\vspace{0.2cm}

{\it Model.} Linear optical responses of excitonic systems are determined by the dynamics of optical coherences between electronic ground and excited states. In the simplest case of a dimeric system consisting of only two pigments, the coherent electronic interaction is modelled by a Hamiltonian of the form,
\begin{equation}
    H_e=\sum_{j=1}^{2}\epsilon_j\ket{\epsilon_j}\bra{\epsilon_j}+V(\ket{\epsilon_1}\bra{\epsilon_2}+\ket{\epsilon_2}\bra{\epsilon_1}),
    \label{eq:H_e}
\end{equation}
where $\ket{\epsilon_j}$ denotes a local electronic excitation of pigment $j$ with transition energy $\epsilon_j$ between electronic ground and first excited states, and $V$ represents the inter-pigment electronic coupling. Here we assume that higher energy excited states do not contribute to the low-energy part of linear optical spectra, which is a reliable description of PPCs consisting of bacteriochlorophylls such as the FMO complex~\cite{RengerBJ2006}. 
For PPCs, the site energies $\epsilon_j$ of pigments can vary due to non-identical local environments, leading to static disorder that can be accounted for by sampling the site energies $\epsilon_j$ from independent Gaussian distributions with mean values $\langle\epsilon_j\rangle$ and standard deviation $\sigma$. In absorption simulations present in this work, we take into account the static disorder with $\sigma = 80\,{\rm cm}^{-1}$, a typical value of PPCs~\cite{RengerBJ2006,Caycedo2022}, and compute ensemble averaged optical responses over multiple samples, unless stated otherwise. 
The eigenstates of the electronic Hamiltonian are expressed as delocalized excitonic states $\ket{E_+}= \cos(\theta/2)\ket{\epsilon_1}+\sin(\theta/2)\ket{\epsilon_2}$ and $\ket{E_-} = -\sin(\theta/2)\ket{\epsilon_1}+\cos(\theta/2)\ket{\epsilon_2}$,
with mixing angle $\theta=\tan^{-1}(2V/(\epsilon_1-\epsilon_2))$. Here $\ket{E_+}$ and $\ket{E_-}$ represent high- and low-energy exciton states, respectively, with energies $E_\pm = (\epsilon_1+\epsilon_2\pm\Delta)/2$, where $\Delta=E_+-E_-=\sqrt{(\epsilon_1-\epsilon_2)^{2}+4V^2}$ denotes an excitonic splitting.

For PPCs, the vibronic interaction is modeled by a Hamiltonian term that induces local dephasing of local excitations, namely
\begin{align}
    H_{e-v}=\sum_{j=1}^{2}\ket{\epsilon_j}\bra{\epsilon_j}\sum_{k}\omega_k\sqrt{s_k}(b_{j,k}+b_{j,k}^{\dagger}),
    \label{eq:H_ev}
\end{align}
where $b_{j,k}$ and $b_{j,k}^{\dagger}$ denote the annihilation and creation operators, respectively, of a vibrational mode with frequency $\omega_k$, which is locally coupled to pigment $j$ with vibronic coupling strength quantified by the Huang-Rhys factor $s_k$. The interaction Hamiltonian $H_{e-v}$ is fully characterized by a phonon spectral density $J(\omega)=\sum_{k}\omega_{k}^{2}s_k\delta(\omega-\omega_k)$, and the total vibronic coupling strength is often quantified by the reorganization energy $\int_{0}^{\infty}d\omega J(\omega)/\omega$. In this work, we consider the experimentally estimated phonon spectral density of the FMO complex~\cite{RatsepJL2007}, shown in blue in Fig.~\ref{Fig1}(a), consisting of a quasi-continuous protein spectrum $J_{\rm AR}(\omega)$ and 62 intra-pigment vibrational modes described by narrow Lorentzian peaks (see the SI for more details). Given the vibrational Hamiltonian $H_v=\sum_{j=1}^{2}\sum_{k}\omega_k b_{j,k}^{\dagger}b_{j,k}$, the total Hamiltonian is modeled by $H=H_e+H_v+H_{e-v}$.

\begin{figure}
\includegraphics[width=0.48\textwidth]{New_Fig1.pdf}
\caption{(a) Experimentally estimated phonon spectral density of the FMO complex from green sulfur bacteria~\cite{RatsepJL2007} and coarse-grained spectral density constructed based on conventional criteria (see the main text), shown in blue and red, respectively. (b) Monomer absorption spectra at $77\,{\rm K}$ computed based on the full and coarse-grained spectral densities in (a). (c,d) Dimer absorption spectra at $77\,{\rm K}$ as a function of electronic coupling $V$, computed by HEOM with (c) the full FMO phonon spectral density or with (d) the coarse-grained model.}
\label{Fig1}
\end{figure}

{\it Conventional coarse-graining.} The computational costs of several non-perturbative simulation tools, such as HEOM~\cite{KramerJCC2018} and DAMPF~\cite{SomozaPRL2019,SomozaarXiv2022}, strongly depend on the number of peaks present in an input spectral density. In Fig.~\ref{Fig1}(a), a typical coarse-grained spectral density considered in previous studies~\cite{KreisbeckJPCL2012,KreisbeckJCTC2014,BlauPNAS2018} is shown in red, where the 62 intra-pigment modes are replaced by a single broad Lorentzian peak. Here the parameters of the broad Lorentzian were determined in such a way that the total reorganization energy of the 62 intra-pigment modes is conserved, and the monomer absorption spectra at $77\,{\rm K}$ ($\epsilon_1=\epsilon_2$ and $V=0$ in Eq.~(\ref{eq:H_e})) computed based on the full and coarse-grained spectral densities are well matched, as shown in Fig.~\ref{Fig1}(b). In Fig.~\ref{Fig1}(c) and (d), dimer absorption spectra computed by HEOM with the full and coarse-grained spectral densities, respectively, are shown as a function of the electronic coupling strength $V$, for the case that the mean site energies are identical ($\langle\epsilon_1\rangle= \langle\epsilon_2\rangle$) and the transition dipole moments between pigments are mutually orthogonal (see the SI for more details). On the one hand, these results demonstrate that the conventional coarse-graining approach can fail to describe the optical responses of multi-chromophoric systems coupled to actual vibrational environments. On the other hand, the failure of the conventional approach implies that the influence of vibrational environments on electronic system dynamics is not fully characterized by the total vibronic coupling strength and its impact on electronic dynamics of monomer.

{\it BCF-based coarse-graining.} The influence of bosonic environments on reduced system dynamics is determined by the bath correlation function (BCF), which takes the form $C(t)=\int_{0}^{\infty}d\omega J(\omega)(\coth(\omega/2k_B T)\cos(\omega t)-i\sin(\omega t))$ where $J(\omega)$ is the environmental spectral density. For open systems that are linearly coupled to bosonic environments initially in thermal states at temperature $T$, such as the vibronic coupling model of PPCs, the BCF fully characterizes the environmental influence.

Of particular importance is the fact that the reduced system density matrix at a finite time $t_0$ is determined solely by the bath correlation function $C(t)$ for $0\le t\le t_0$. This means that the BCF beyond $t_0$ has no effect on the reduced system dynamics \cite{HartmannJCP2019}. Therefore, when experimentally observable quantities are determined by the reduced system dynamics over a finite time scale $0\le t\le t_0$, a coarse-grained spectral density can effectively describe a highly structured spectral density, provided that their BCFs are well matched for $0\le t\le t_0$.

As detailed in the SI, absorption spectra are obtained by the Fourier transformation (FT) of the optical coherence dynamics between electronic ground and excited states. The absorption line widths are determined by a finite lifetime of optical coherences, induced by the vibronic coupling $H_{e-v}$ and static disorder. In case that one aims to compute absorption spectra for a given set of electronic and vibrational parameters, an upper bound on the lifetime of optical coherences can be estimated based on the degree of static disorder. As summarized in the SI, the static disorder in site energies $\epsilon_j$ of a dimer, modeled by independent Gaussian distributions with a standard deviation of $\sigma$, can be alternatively described by the static disorder in site energy difference $(\epsilon_1-\epsilon_2)$ and the static disorder in average site energy $(\epsilon_1+\epsilon_2)/2$, where the latter leads to a Gaussian dephasing  $e^{-\sigma^{2} t^{2}/4}$ of optical coherences. As a consequence, the upper bound on the lifetime of optical coherences can be estimated from the Gaussian dephasing, which is approximately $300\,{\rm fs}$ for $\sigma=80\,{\rm cm}^{-1}$ considered in our simulations. This implies that the optical coherence dynamics and corresponding absorption spectra are determined by $C(t)$ for $0\le t\le 300\,{\rm fs}$, while $C(t)$ beyond $t>300\,{\rm fs}$ cannot affect absorption line shapes. Therefore numerically exact dimer absorption spectra can be obtained by using a coarse-grained phonon spectral density that quantitatively describes the bath correlation function $C(t)$ of the full phonon spectral density up to $300\,{\rm fs}$.

\begin{figure}
\includegraphics[width=0.48\textwidth]{New_Fig2.pdf}
\caption{(a) Fourier transformation (FT) of the bath correlation function of the full FMO spectral density at $77\,{\rm K}$ weighted by a Gaussian filter, $C(t)e^{-t^2/2(100\,{\rm fs})^{2}}$, over a finite time window $0\le t\le 300\,{\rm fs}$, shown in blue. A sum of the FT spectrum of the Gaussian-filtered bath correlation function of the low-frequency phonon spectrum $J_{\rm AR}(\omega)$ of the FMO complex and those of five effective Lorentzian spectral densities, shown in red (see the SI for more details). (b) The full FMO phonon spectral density and the coarse-grained phonon spectral density considered in (a), shown in blue and red, respectively. (c) Dimer absorption spectra at $77\,{\rm K}$ as a function of electronic coupling V, computed by HEOM with the coarse-grained spectral density, which are well matched to the full model results shown in Fig.~\ref{Fig1}(c). (d) The real part of the bath correlation function of the full FMO complex, shown in blue, and that of a truncated T-TEDOPA chain consisting of 51 oscillators, shown in orange (see the corresponding discrete spectral density of the truncated T-TEDOPA chain, shown in black dots in (b)).}
\label{Fig2}
\end{figure}

In Fig.~\ref{Fig2}(a), the FT of $C(t)$ of the full FMO spectral density over $0\le t\le 300\,{\rm fs}$ is shown in blue. Since a sharp cut-off of $C(t)$ at $t=300\,{\rm fs}$ leads to ringing artifacts that make it difficult to identify the frequency components present in $C(t)$ within $300\,{\rm fs}$, we apply a Gaussian filter before the FT, so that $C(t)e^{-t^2/2 \tilde{\sigma}^{2}}$ with a standard deviation $\tilde{\sigma} = 100\,{\rm fs}$ has less than $1\,\%$ of its maximum amplitude at $3\tilde{\sigma} = 300\,{\rm fs}$. Here we consider a Gaussian filter instead of exponential to better capture the initial dynamics of $C(t)$. It is found that the resulting FT spectrum can be well reproduced by a sum of the Gaussian-filtered bath correlation function of the low-frequency protein spectrum of the FMO complex, $J_{\rm AR}(\omega)$, and those of five effective Lorentzian spectral densities, as shown in red in Fig.~\ref{Fig2}(a) (see the SI for more details). Here the parameters of the five Lorentzian functions, shown in red in Fig.~\ref{Fig2}(b), were determined in such a way that the total reorganization energy and Huang-Rhys factor of the 62 intra-pigment modes of the FMO complex are conserved. The absorption spectra computed by HEOM with the coarse-grained phonon spectral density, shown in Fig.~\ref{Fig2}(c), are quantitatively well matched to Fig.~\ref{Fig1}(c) where the full FMO spectral density is considered. These results demonstrate that the bath correlation function $C(t)$ within the lifetime of optical coherences should be considered when constructing a coarse-grained vibrational environment of multi-chromophoric systems. As summarized in the SI, the numerically exact dimer absorption spectra shown in Fig.~\ref{Fig1}(c) can be explained based on avoided crossings induced by multimode vibronic mixing~\cite{Caycedo2022}, while the non-perturbative simulated results cannot be reproduced by a perturbative line shape theory.

{\it TEDOPA-based coarse-graining.} The coarse-grained environmental structure governing system dynamics up to 300\,${\rm fs}$ can also be estimated by using the parameters of T-TEDOPA~\cite{TamascelliPRL2019}, an alternative numerically exact method for simulating open quantum system dynamics. Here a vibrational environment is mapped into a semi-infinite one-dimensional chain of quantum harmonic oscillators, where the electronic states couple only to the first site of the chain and every oscillator interacts only with its nearest neighbor~\cite{PriorCH+10,ChinRH+10}.
In a crucial step towards the simulation of spectral densities at finite temperature, T-TEDOPA implements a transformed spectral density determined such that all the environmental oscillators in the chain are initialized in their vacuum states and the system dynamics is provably identical to that of the original finite temperature problem~\cite{TamascelliPRL2019}. As a consequence, the first oscillator directly coupled to the electronic states is populated at early times and then the population is transferred through the semi-infinite chain. This allows a truncation of the chain in such a way that the bath correlation function $C(t)$ is well described within a finite time scale until the population reaches the truncated site, thus ensuring arbitrarily small error in the system observables such as spectra~\cite{MascherpaSH+17}. The finite number of oscillators of the truncated chain results in a discrete phonon spectral density~\cite{TamascelliMC2022}, which does not show any ringing artifacts, contrary to the previously described FT-based approach. Fig.~\ref{Fig2}(d) shows the bath correlation function of the truncated T-TEDOPA chain consisting of 51 oscillators, which reproduces the bath correlation function $C(t)$ of the full FMO spectral density up to $300\,{\rm fs}$ (as the number of oscillators increases, the bath correlation function can be reproduced for a longer time scale). The corresponding discrete phonon spectral density, shown in black dots in Fig.~\ref{Fig2}(b), is qualitatively similar to the effective phonon spectral density constructed based on the FT-based approach. This implies that the effective spectral density of a truncated T-TEDOPA chain can be a useful guideline when estimating the structure of a coarse-grained environment that accurately describes reduced system dynamics on a finite time scale.

{\it Coarse-graining and computational costs.} So far we have demonstrated that dimer absorption spectra for the full FMO spectral density with its 62 intra-pigment modes (see Fig.~\ref{Fig2}(b)) can be computed in a numerically accurate manner by considering five effective Lorentzian peaks. Such a reduction of the number of peaks of an environmental spectral density entering non-perturbative simulation tools can significantly reduce their computational costs. In conventional HEOM simulations~\cite{KramerJCC2018}, the information about the correlations between system and environments is encoded in a hierarchical structure of auxiliary operators whose dimensions are identical to that of a reduced system density matrix. For a typical PPC model consisting of $N$ pigments coupled to local vibrational environments comprised of $M$ Lorentzian peaks, the total number of auxiliary operators up to the $L$-th hierarchical layer is approximately given by $(2NM+L)!/(2NM)!/L!$~\cite{KramerJCC2018}, where a larger $L$ is required to obtain numerically converged reduced system dynamics as the Lorentzian peaks become narrower~\cite{Caycedo2022}. In this estimate, the Matsubara terms are not considered, which may increase the HEOM simulation costs further~\cite{LimPRL2019}. For a dimer ($N=2$), absorption simulation costs are reduced by almost $6$ orders of magnitude, from $\sim 0.27\,{\rm TB}$ to $\sim 1.7\,{\rm MB}$ as the number of Lorentzian peaks is decreased from $M=62$ to $M=5$, when a typical hierarchical depth $L=5$ is considered. Note that the width of the Lorentzian peaks of the full FMO spectral density is of the order of $(1\,{\rm ps})^{-1}\approx 5\,{\rm cm}^{-1}$, which is an order of magnitude narrower than the five effective Lorentzian peaks considered in coarse-graining (see Fig.~\ref{Fig2}(b) and the SI for more details), thus the full model simulations with $M=62$ require a higher $L\approx 9$~\cite{Caycedo2022}, increasing the HEOM simulation costs further. For multi-chromophoric systems consisting of $N=10, 20, 30$ pigments, when our coarse-grained spectral density with $M=5$ is considered with $L=5$, the HEOM simulation costs of absorption simulations are approximately $\sim 0.015, 0.92, 10\,{\rm TB}$, respectively. This will enable HEOM simulations of functionally relevant PPC units, such as the FMO complex ($N=7$)~\cite{MilderPR2010}, the PC645 complex ($N=8$) from marine algae~\cite{BlauPNAS2018,ColliniNature2010}, and LH2 ($N=27$) from purple bacteria~\cite{ScholesJPCB1999,LawMMB2004}.

\begin{figure}
\includegraphics[width=0.48\textwidth]{New_Fig4.pdf}
\caption{(a) Absorption spectra of a seven-site FMO model at $77\,{\rm K}$, computed by DAMPF, where each site is coupled to 62 intra-pigment modes (blue) or five effective Lorentzian peaks (red) as shown in (b).}
\label{Fig3}
\end{figure}

Our coarse-graining approach to the spectral density can also greatly reduce the computational costs of DAMPF~\cite{SomozaPRL2019}. We demonstrate this potential in Fig.~\ref{Fig3}(a), which shows absorption spectra of the full FMO complex consisting of seven sites ($N=7$) computed by DAMPF where each site is coupled to a local environment consisting of $62$ intra-pigment vibrational modes or five effective Lorentzian peaks, as shown in Fig.~\ref{Fig3}(b). Here we consider the mean site energies of the FMO complex~\cite{RengerBJ2006} together with Gaussian line shapes to approximately take into account inhomogeneous broadening, in such a way that optical coherence dynamics weighted by a Gaussian function $e^{-t^2/2\tilde{\sigma}^2}$ decays within $3\tilde{\sigma}=300\,{\rm fs}$. We found that the computational time is reduced from $40$ min to $3$ min (using 15 cores in an Intel Xeon 6252 Gold CPU) as the number of Lorentzians is decreased from $M=62$ to $M=5$, while the memory costs are surprisingly low for both cases ($200\,{\rm MB}$ to $4\,{\rm MB}$, respectively). The low computational costs are mainly due to the low bond dimensions $\sim 4$ required to obtain numerically converged absorption spectra (see the SI for more details). Since an accurate description of the inhomogeneous broadening induced by static disorder requires the repetitions of DAMPF simulations with randomly generated electronic Hamiltonian, typically requiring $\sim 10^{3}$ samples, the coarse-graining scheme can significantly reduce the overall computational costs. This will be helpful in molecular parameter estimation where the parameters of electronic system Hamiltonian are optimized until experimental spectra are quantitatively reproduced~\cite{Caycedo2022}.



\vspace{0.2cm}

{\it Conclusions.} We have developed a systematic method to construct coarse-grained vibrational environments which can efficiently describe electronic dynamics subject to the complete vibrational environment, by considering the bath correlation function only within the characteristic time scale of the physical quantity of interest. This has allowed to model the absorption spectrum of the whole FMO complex subject to the experimentally determined phonon fluctuation spectrum. As the time scale of interest increases, for example, from a sub-picosecond time scale of linear optical responses to a picosecond time scale of energy/charge transfer dynamics and nonlinear optical responses, the primary environmental structure responsible for measured quantities can become more complicated as the multimode nature of the vibrational environments is further revealed, nonetheless requiring less computational resources than the full spectral density (see the SI for explicit details). The systematic coarse-graining proposed in this work will be useful in general non-perturbative simulations of multi-partite systems and in particular in studies concerning the feasibility and resilience of proposed molecular information processors~\cite{WasielewskiNRC2020}. More generally, our findings provide an insight on the primary environmental structure responsible for open-system dynamics and its fingerprint on experimentally observable quantities.

{\it Acknowledgements}. We thank Thomas Renger for useful discussions. This work was support by the ERC Synergy grant HyperQ (Grant No. 856432), the BMBF project PhoQuant (grant no. 13N16110) and the state of Baden-Württemberg through bwHPC and the German Research Foundation (DFG) through grant no INST 40/575-1 FUGG (JUSTUS 2 cluster).

\begin{thebibliography}{99}

\bibitem{May2011} V. May and O. K{\"u}hn. Charge and Energy Transfer Dynamics in Molecular Systems (Wiley-VCH, 2011).

\bibitem{JangPRL2004} S. Jang, M. D. Newton, and R. J. Silbey. Multichromophoric F{\"o}rster Resonance Energy Transfer. Phys. Rev. Lett. {\bf 92}, 218301 (2004).

\bibitem{RengerBJ2006} J. Adolphs and T. Renger. How Proteins Trigger Excitation Energy Transfer in the FMO Complex of Green Sulfur Bacteria. Biophys. J. {\bf 91}, 2778 (2006).

\bibitem{WomickJPCB2011} J. M. Womick and A. M. Moran. Vibronic Enhancement of Exciton Sizes and Energy Transport in Photosynthetic Complexes. J. Phys. Chem. B {\bf 115}, 1347 (2011).

\bibitem{RomeroNature2017} E. Romero, V. I. Novoderezhkin, and R. van Grondelle. Quantum Design of Photosynthesis of Bio-Inspired Solar-Energy Conversion. Nature {\bf 543}, 355 (2017).

\bibitem{JangRMP2018} S. J. Jang and B. Mennucci. Delocalized Excitons in Natural Light-Harvesting Complexes. Rev. Mod. Phys. {\bf 90}, 035003 (2018).

\bibitem{MattioniPRX2021} A. Mattioni, F. Caycedo-Soler, S. F. Huelga, and M. B. Plenio. Design Principles for Long-Range Energy Transfer at Room Temperature. Phys. Rev. X {\bf 11}, 041003 (2021).

\bibitem{SomozaarXiv2022} A. D. Somoza, N. Lorenzoni, J. Lim, S. F. Huelga, and M. B. Plenio. Driving Force and Nonequilibrium Vibronic Dynamics in Charge Separation of Strongly Bound Electron-Hole Pairs. arXiv:2205.06623 (2022).

\bibitem{ZhangPRB2020} L. Zhang, Y. Hao, W. Qin, S. Xie, F. Qu. Chiral-Induced Spin Selectivity: A Polaron Transport Model. Phys. Rev. B {\bf 102}, 214303 (2020).

\bibitem{DuPRB2020} G.-F. Du, H.-H. Fu, R. Wu. Vibration-Enhanced Spin-Selective Transport of Electrons in the DNA Double Helix. Phys. Rev. B {\bf 102}, 035431 (2020).

\bibitem{FranssonPRB2020} J. Fransson. Vibrational Origin of Exchange Splitting and Chiral-Induced Spin Selectivity. Phys. Rev. B {\bf 102}, 235416 (2020).

\bibitem{VittmannJPCL2023} C. Vittmann, J. Lim, D. Tamascelli, S. F. Huelga, M. B. Plenio. Spin-Dependent Momentum Conservation of Electron-Phonon Scattering in Chirality-Induced Spin Selectivity. J. Phys. Chem. Lett. {\bf 14}, 340 (2023).

\bibitem{PriorCH+10} J. Prior, A.W. Chin, S.F. Huelga, and M.B. Plenio. Efficient simulation of strong system-environment interactions. Phys. Rev. Lett. {\bf 105}, 050404 (2010).

\bibitem{ChinNP2013} A. W. Chin, J. Prior, R. Rosenbach, F. Caycedo-Soler, S. F. Huelga, and M. B. Plenio. The Role of Non-Equilibrium Vibrational Structures in
Electronic Coherence and Recoherence in Pigment-Protein Complexes. Nat. Phys. {\bf 9}, 113 (2013).

\bibitem{TiwariPNAS2013} V. Tiwari, W. K. Peters, and D. M. Jonas. Electronic Resonance with Anticorrelated Pigment Vibrations Drives Photosynthetic Energy Transfer Outside the Adiabatic Framework. Proc. Natl. Acad. Sci. U.S.A. {\bf 110}, 1203 (2013). 

\bibitem{PlenioAH13} M.B. Plenio, J. Almeida and S.F. Huelga. Origin of long-lived oscillations in 2D-spectra of a Quantum Vibronic Model: Electronic versus Vibrational coherence. J. Chem. Phys. {\bf 139}, 235102 (2013).

\bibitem{ChenuSR2013} A. Chenu, N. Christensson, H.F. Kauffmann, and T. Man{\v c}al. Enhancement of Vibronic and Ground-State Vibrational Coherences in 2D Spectra of Photosynthetic Complexes. Sci. Rep. {\bf 3}, 2029 (2013).

\bibitem{RomeroNP2014} E. Romero, R. Augulis, V. I. Novoderezhkin, M. Ferretti, J. Thieme, D. Zigmantas, and R. van Grondelle. Quantum Coherence in Photosynthesis for Efficient Solar-Energy Conversion. Nat. Phys. {\bf 10}, 676 (2014).

\bibitem{FullerNC2014} F. D. Fuller, J. Pan, A. Gelzinis, V. Butkus, S. S. Senlik, D. E. Wilcox, C. F. Yocum, L. Valkunas, D. Abramavicius, and J. P. Ogilvie. Vibronic Coherence in Oxygenic Photosynthesis. Nat. Chem. {\bf 6}, 706 (2014).

\bibitem{ButkusJCP2014} V. Butkus, L. Valkunas, and D. Abramavicius. Vibronic Phenomena and Exciton–Vibrational Interference in Two-Dimensional Spectra of Molecular Aggregates. J. Chem. Phys. {\bf 140}, 034306 (2014).

\bibitem{LimNC2015} J. Lim, D. Pale{\v c}ek, F. Caycedo-Soler, C. N. Lincoln, J. Prior, H. von Berlepsch, S. F. Huelga, M. B. Plenio, D. Zigmantas, and J. Hauer. Vibronic Origin of Long-Lived Coherence in an Artificial Molecular Light Harvester. Nat. Commun. {\bf 6}, 7755 (2015).

\bibitem{Caycedo2022} F. Caycedo-Soler, A. Mattioni, J. Lim, T. Renger, S. F. Huelga, and M. B. Plenio. Exact Simulation of Pigment-Protein Complexes Unveils Vibronic Renormalization of Electronic Parameters in Ultrafast Spectroscopy. Nat. Commun. {\bf 13}, 2912 (2022).

\bibitem{SpanoACR2010} F. C. Spano. The Spectral Signatures of Frenkel Polarons in H- and J-Aggregates. Acc. Chem. Res. {\bf 43}, 429 (2010).

\bibitem{FarouilJPCA2018} L. Farouil, F. Alary, E. Bedel-Pereira, and J.-L. Heully. Revisiting the Vibrational and Optical Properties of P3HT: A Combined Experimental and Theoretical Study. J. Phys. Chem. A {\bf 122}, 6532 (2018).

\bibitem{BarclayJPCL2022} M. S. Barclay, J. S. Huff, R. D. Pensack, P. H. Davis, W. B. Knowlton, B. Yurke, J. C. Dean, P. C. Arpin, and D. B. Turner. Characterizing Mode Anharmonicity and Huang-Rhys Factors Using Models of Femtosecond Coherence Spectra. J. Phys. Chem. Lett. {\bf 13}, 5413 (2022).

\bibitem{NovoderezhkinJPCB2004} V. I. Novoderezhkin, M. A. Palacios, H. van Amerongen, and R. van Grondelle. Energy-Transfer Dynamics in the LHCII Complex of Higher Plants: Modified Redfield Approach. J. Phys. Chem. B {\bf 108}, 10363 (2004).

\bibitem{NovoderezhkinBJ2005} V. I. Novoderezhkin, E. G. Andrizhiyevskaya, J. P. Dekker, and R. van Grondelle. Pathways and Timescales of Primary Charge Separation in the Photosystem II Reaction Center as Revealed by a Simultaneous Fit of Time-Resolved Fluorescence and Transient Absorption. Biophys. J. {\bf 89}, 1464 (2005).

\bibitem{NovoderezhkinBJ2006} V. I. Novoderezhkin, D. Rutkauskas, and R. van Grondelle. Dynamics of the Emission Spectrum of a Single LH2 Complex: Interplay of Slow and Fast Nuclear Motions. Biophys. J. {\bf 90}, 2890 (2006).

\bibitem{AbramaviciusJCP2010} D. Abramavicius and S. Mukamel. Energy-Transfer and Charge-Separation Pathways in the Reaction Center of Photosystem II Revealed by Coherent Two-Dimensional Optical Spectroscopy. J. Chem. Phys. {\bf 133}, 184501 (2010).

\bibitem{NovoderezhkinBJ2010} V. I. Novoderezhkin, A. B. Doust, C. Curutchet, G. D. Scholes, and R. van Grondelle. Excitation Dynamics in Phycoerythrin 545: Modeling of Steady-State Spectra and Transient Absorption with Modified Redfield Theory. Biophys. J. {\bf 99}, 344 (2010).

\bibitem{LewisJPCA2013} K. L. M. Lewis, F. D. Fuller, J. A. Myers, C. F. Yocum, S. Mukamel, D. Abramavicius, and J. P. Ogilvie. Simulations of the Two-Dimensional Electronic Spectroscopy of the Photosystem II Reaction Center. J. Phys. Chem. A {\bf 117}, 34 (2013).

\bibitem{DinhJCP2015} T.-C. Dinh and T. Renger. Towards an Exact Theory of Linear Absorbance and Circular Dichroism of Pigment-Protein Complexes: Importance of Non-Secular Contributions. J. Chem. Phys. {\bf 142}, 034104 (2015).

\bibitem{GelzinisJCP2015} A. Gelzinis, D. Abramavicius, and L. Valkunas. Absorption Lineshapes of Molecular Aggregates Revisited. J. Chem. Phys. {\bf 142}, 154107 (2015).

\bibitem{RancovaJPCB2016} O. Rancova, R. Jankowiak, A. Kell, M. Jassas, and D. Abramavicius. Band Structure of the Rhodobacter Sphaeroides Photosynthetic Reaction Center from Low-Temperature Absorption and Hole-Burned Spectra. J. Phys. Chem. B {\bf 120}, 5601 (2016).

\bibitem{GelzinisJCP2017} A. Gelzinis, D. Abramavicius, J. P. Ogilvie, and L. Valkunas. Spectroscopic Properties of Photosystem II Reaction Center Revisited. J. Chem. Phys. {\bf 147}, 115102 (2017).

\bibitem{RancovaJPCB2018} O. Rancova, R. Jankowiak, and D. Abramavicius. Role of Bath Fluctuations in the Double-Excitation Manifold in Shaping the 2DES of Bacterial Reaction Centers at Low Temperature. J. Phys. Chem. B {\bf 122}, 1348 (2018).

\bibitem{KhmelnitskiyJPCB2019} A. Khmelnitskiy, T. Reinot, and R. Jankowiak. Mixed Upper Exciton State of the Special Pair in Bacterial Reaction Centers. J. Phys. Chem. B {\bf 123}, 852 (2019).

\bibitem{StrumpferJCP2009} J. Str{\"u}mpfer and K. Schulten. Light Harvesting Complex II B850 Excitation Dynamics. J. Chem. Phys. {\bf 131}, 225101 (2009).

\bibitem{IshizakiPNAS2009} A. Ishizaki and G. R. Fleming. Theoretical Examination of Quantum Coherence in a Photosynthetic System at Physiological Temperature. Proc. Natl. Acad. Sci. {\bf 106}, 17255 (2009).

\bibitem{KreisbeckJPCL2012} C. Kreisbeck and T. Kramer. Long-Lived Electronic Coherence in Dissipative Exciton Dynamics of Light-Harvesting Complexes. J. Phys. Chem. Lett. {\bf 3}, 2828 (2012).

\bibitem{KreisbeckJCTC2014} C. Kreisbeck, T. Kramer and A. Aspuru-Guzik. Scalable High-Performance Algorithm for the Simulation of Exciton Dynamics. Application to the Light-Harvesting Complex II in the Presence of Resonant Vibrational Modes. J. Chem. Theory Comput. {\bf 10}, 4045 (2014).

\bibitem{BlauPNAS2018} S. M. Blau, D. I. G. Bennett, C. Kreisbeck, G. D. Scholes, and A. Aspuru-Guzik. Local Protein Solvation Drives Direct Down-Conversion in Phycobiliprotein PC645 via Incoherent Vibronic Transport. Proc. Natl Acad. Sci. USA {\bf 115}, E3342 (2018).

\bibitem{TanimuraJPSJ1989} Y. Tanimura and R. Kubo. Time Evolution of a Quantum System in Contact with a Nearly Gaussian–Markoffian Noise Bath. J. Phys. Soc. Jpn. {\bf 58}, 101 (1989).

\bibitem{RatsepJL2007} M. R{\"a}tsep and A. Freiberg. Electron Phonon and Vibronic Couplings in the FMO Bacteriochlorophyll a Antenna Complex Studied by Difference Fluorescence Line Narrowing. J. Lumin. {\bf 127}, 251 (2007).

\bibitem{KramerJCC2018} T. Kramer, M. Noack, A. Reinefeld, M. Rodr{\' i}guez, and Y. Zelinskyy. Efficient Calculation of Open Quantum System Dynamics and Time-Resolved Spectroscopy with Distributed Memory HEOM (DM-HEOM). J. Comput. Chem. {\bf 39}, 1779 (2018).

\bibitem{SomozaPRL2019} A. D. Somoza, O. Marty, J. Lim, S. F. Huelga, and M. B. Plenio. Dissipation-Assisted Matrix Product Factorization. Phys. Rev. Lett. {\bf 123}, 100502 (2019).

\bibitem{HartmannJCP2019} R. Hartmann, M. Werther, F. Grossmann, and W. T. Strunz. Exact Open Quantum System Dynamics: Optimal Frequency vs Time Representation of Bath Correlations. J. Chem. Phys. {\bf 150}, 234105 (2019).

\bibitem{TamascelliPRL2019} D. Tamascelli, A. Smirne, J. Lim, S. F. Huelga, and M. B. Plenio. Efficient Simulation of Finite-Temperature Open Quantum Systems. Phys. Rev. Lett. {\bf 123}, 090402 (2019).

\bibitem{ChinRH+10} A. W. Chin, A. Rivas, S. F. Huelga and M. B. Plenio. Exact mapping between system-reservoir quantum models and semi-infinite discrete chains using orthogonal polynomials. J. Math. Phys. {\bf 51}, 092109 (2010).

\bibitem{MascherpaSH+17} F. Mascherpa, A. Smirne, S.F. Huelga and M.B. Plenio. Open systems with error bounds: spin boson model with spectral density variations. Phys. Rev. Lett. {\bf 118}, 100401 (2017).

\bibitem{TamascelliMC2022} A. N{\"u}{\ss}eler, D. Tamascelli, A. Smirne, J. Lim, S.~F. Huelga, and M.~B. Plenio. Fingerprint and Universal Markovian Closure of Structured Bosonic Environments. Phys. Rev. Lett. {\bf 129}, 140604 (2022).

\bibitem{LimPRL2019} J. Lim, C. M. B{\"o}sen, A. D. Somoza, C. P. Koch, M. B. Plenio, and S. F. Huelga. Multicolor Quantum Control for Suppressing Ground State Coherences in Two-Dimensional Electronic Spectroscopy. Phys. Rev. Lett. {\bf 123}, 233201 (2019).

\bibitem{MilderPR2010} M. T. W. Milder, B. Br{\"u}ggemann, R. van Grondelle, and J. L. Herek. Revisiting the Optical Properties of the FMO Protein. Photosynth Res. {\bf 104}, 257 (2010).

\bibitem{ColliniNature2010} E. Collini, C. Y. Wong, K. E. Wilk, P. M. G. Curmi, P. Brumer, G. D. and Scholes. Coherently Wired Light-Harvesting in Photosynthetic Marine Algae at Ambient Temperature. Nature {\bf 463}, 644 (2010).

\bibitem{ScholesJPCB1999} G. D. Scholes, I. R. Gould, R. J. Cogdell, and G. R. Fleming. Ab Initio Molecular Orbital Calculations of Electronic Couplings in the LH2 Bacterial Light-Harvesting Complex of Rps. Acidophila. J. Phys. Chem. B {\bf 103}, 2543 (1999).

\bibitem{LawMMB2004} C. J. Law, A. W. Roszak, J. Southall, A. T. Gardiner, N. W. Isaacs, and R. J. Cogdell. The Structure and Function of Bacterial Light-Harvesting Complexes (Review). Mol. Membr. Biol. {\bf 21}, 183 (2004).

\bibitem{WasielewskiNRC2020} M. R. Wasielewski, M. D. E. Forbes, N. L. Frank, K. Kowalski, G. D. Scholes, J. Yuen-Zhou, M. A. Baldo, D. E. Freedman, R. H. Goldsmith, T. Goodson III, M. L. Kirk, J. K. McCusker, J. P. Ogilvie, D. A. Shultz, S. Stoll and K. B. Whaley. Exploiting Chemistry and Molecular Systems for Quantum Information Science. Nat. Rev. Chem. {\bf 4}, 490 (2020).

\end{thebibliography}



%%% SI

\begin{widetext}

\newpage
%\vspace{20cm}

\begin{center}
{\large\bf Supplementary Information}
\end{center}

\section{Phonon spectral density of the FMO complex}

The experimentally estimated phonon spectral density of the FMO complex consists of a quasi-continuous spectrum of low-frequency protein modes and 62 high-frequency intra-pigment vibrational modes~\cite{SI_RatsepJL2007}. The low-frequency spectrum is modelled by the Adolphs-Renger (AR) spectral density
\begin{equation}
	J_{\rm AR}(\omega)=\frac{S}{s_1+s_2}\sum_{i=1}^{2}\frac{s_i}{7! 2 \omega_i^4}\omega^5 e^{-(\omega/\omega_i)^{1/2}},
	\label{eq:B777_SD}
\end{equation}
where $S=0.29$, $s_1=0.8$, $s_2=0.5$, $\omega_1=0.069\,{\rm meV}$ and $\omega_2=0.24\,{\rm meV}$. The intra-pigment vibrational modes are modelled by a sum of 62 Lorentzian functions
\begin{equation}
    J_{h}(\omega) = \sum_{k=1}^{62} \frac{4 \omega_k s_k \gamma_k (\omega_{k}^2+\gamma_{k}^2)\omega}{\pi((\omega+\omega_k)^{2}+\gamma_{k}^2)((\omega-\omega_k)^{2}+\gamma_{k}^2)},
\end{equation}
where the $k$-th Lorentzian peak is centered at vibrational frequency $\omega_k$ with a height being proportional to the Huang-Rhys factor $s_k$. The experimentally estimated values of the vibrational frequencies and Huang-Rhys factors~\cite{SI_RatsepJL2007} are summarized in Table~\ref{Table_intrapigment}. The width of the Lorentzian functions, determining the vibrational damping rate of the intra-pigment modes, is taken to be $\gamma_k = 1\,{\rm ps}^{-1} \approx 5\,{\rm cm}^{-1}$. The total phonon spectral density of the FMO complex is given by $J(\omega)=J_{\rm AR}(\omega)+J_{h}(\omega)$.

\begin{table}[b]
\caption{Vibrational frequencies $\omega_k$ and Huang-Rhys factors $s_k$ of the 62 intra-pigment modes of the FMO complex~\cite{SI_RatsepJL2007}.}
\label{Table_intrapigment}
\begin{ruledtabular}
\begin{tabular}{llllllllllllll}
$k$ & 1 & 2 & 3 & 4 & 5 & 6 & 7 & 8 & 9 & 10\\
\hline
$\omega_k\,[{\rm cm}^{-1}]$ & 46 & 68 & 117 & 167 & 180 & 191 & 202 & 243 & 263 & 284 \\
$s_k$ & 0.011 & 0.011 & 0.009 & 0.009 & 0.010 & 0.011 & 0.011 & 0.012 & 0.003 & 0.008 \\
\hline\hline
$k$ & 11 & 12 & 13 & 14 & 15 & 16 & 17 & 18 & 19 & 20 \\
\hline
$\omega_k\,[{\rm cm}^{-1}]$ & 291 & 327 & 366 & 385 & 404 & 423 & 440 & 481 & 541 & 568 \\
$s_k$ & 0.008 & 0.003 & 0.006 & 0.002 & 0.002 & 0.002 & 0.001 & 0.002 & 0.004 & 0.007\\
\hline\hline
$k$ & 21 & 22 & 23 & 24 & 25 & 26 & 27 & 28 & 29 & 30 \\
\hline
$\omega_k\,[{\rm cm}^{-1}]$ & 582 & 597 & 630 & 638 & 665 & 684 & 713 & 726 & 731 & 750 \\
$s_k$ & 0.004 & 0.004 & 0.003 & 0.006 & 0.004 & 0.003 & 0.007 & 0.010 & 0.005 & 0.004 \\
\hline\hline
$k$ & 31 & 32 & 33 & 34 & 35 & 36 & 37 & 38 & 39 & 40\\
\hline
$\omega_k\,[{\rm cm}^{-1}]$ & 761 & 770 & 795 & 821 & 856 & 891 & 900 & 924 & 929 & 946 \\
$s_k$ & 0.009 & 0.018 & 0.007 & 0.006 & 0.007 & 0.003 & 0.004 & 0.001 & 0.001 & 0.002\\
\hline\hline
$k$ & 41 & 42 & 43 & 44 & 45 & 46 & 47 & 48 & 49 & 50 \\
\hline
$\omega_k\,[{\rm cm}^{-1}]$ & 966 & 984 & 1004 & 1037 & 1058 & 1094 & 1104 & 1123 & 1130 & 1162 \\
$s_k$ & 0.002 & 0.003 & 0.001 & 0.002 & 0.002 & 0.001 & 0.001 & 0.003 & 0.003 & 0.009 \\
\hline\hline
$k$ & 51 & 52 & 53 & 54 & 55 & 56 & 57 & 58 & 59 & 60 \\
\hline
$\omega_k\,[{\rm cm}^{-1}]$ & 1175 & 1181 & 1201 & 1220 & 1283 & 1292 & 1348 & 1367 & 1386 & 1431 \\
$s_k$ & 0.007 & 0.010 & 0.003 & 0.005 & 0.002 & 0.004 & 0.007 & 0.002 & 0.004 & 0.002 \\
\hline\hline
$k$ & 61 & 62 \\
\hline
$\omega_k\,[{\rm cm}^{-1}]$ & 1503 & 1545 \\
$s_k$ & 0.003 & 0.003 \\
\end{tabular}
\end{ruledtabular}
\end{table}

In the main manuscript, a coarse-grained phonon spectral density constructed based on the bath correlation function of the full FMO spectral density $J_{\rm AR}(\omega)+J_{h}(\omega)$ is introduced, which can reproduce numerically exact absorption spectra of dimers computed based on the full FMO phonon spectral density. The coarse-grained phonon spectral density is modelled by a sum of the AR spectral density and five Lorentzians
\begin{equation}
    J_{5L}(\omega) = J_{\rm AR}(\omega) + \sum_{k=1}^{5} \frac{4 \Omega_k S_k \Gamma_k (\Omega_{k}^2+\Gamma_{k}^2)\omega}{\pi((\omega+\Omega_k)^{2}+\Gamma_{k}^2)((\omega-\Omega_k)^{2}+\Gamma_{k}^2)},
\end{equation}
where the parameters $\{\Omega_k,S_k,\Gamma_k\}$ of the Lorentzian functions are summarized in Table~\ref{Table_L5}.

\begin{table}[t]
\caption{The parameters $\{\Omega_k,S_k,\Gamma_k\}$ of the five Lorentzians of a coarse-grained phonon spectral density $J_{5L}(\omega)$ considered in the main manuscript.}
\label{Table_L5}
\begin{ruledtabular}
\begin{tabular}{lllllllll}
$k$ & 1 & 2 & 3 & 4 & 5 \\
\hline
$\Omega_k\,[{\rm cm}^{-1}]$ & 258 & 776 & 1156 & 1348 & 1524 \\
$S_k$ & 0.140 & 0.105 & 0.047 & 0.021 & 0.006 \\
$\Gamma_k\,[{\rm cm}^{-1}]$ & 177 & 53 & 21 & 27 & 18 \\
\end{tabular}
\end{ruledtabular}
\end{table}

In the main manuscript, a coarse-grained spectral density constructed based on conventional criteria is introduced, consisting of the AR spectral density and a single broad Lorentzian peak
\begin{equation}
    J_{1L}(\omega) = J_{\rm AR}(\omega) + \frac{4 \tilde{\Omega} \tilde{S} \tilde{\Gamma} (\tilde{\Omega}^2+\tilde{\Gamma}^2)\omega}{\pi((\omega+\tilde{\Omega})^{2}+\tilde{\Gamma}^2)((\omega-\tilde{\Omega})^{2}+\tilde{\Gamma}^2)},
\end{equation}
where $\tilde{\Omega}=1000\,{\rm cm}^{-1}$, $\tilde{S}=0.2093$, and $\tilde{\Gamma}=(20\,{\rm fs})^{-1}\approx 265\,{\rm cm}^{-1}$.



\section{Theory of absorption spectra}

Within the Franck-Condon approximation, the linear absorption spectrum of isotropic samples is described by
\begin{equation}
    A(\omega)\propto {\rm Re}{\int_{0}^{\infty}} dt e^{i\omega t}{\rm tr}[{\bm \mu}\cdot (e^{-iHt}{\bm \mu}\rho(0)e^{iHt})],
    \label{eq:absorption}
\end{equation}
with $\rho(0)=\ket{g}\bra{g}\otimes\rho_{v}(T)$ representing an initial state where $\ket{g}$ is the global electronic ground state and $\rho_{v}(T)$ the thermal state of vibrational modes at temperature $T$~\cite{SI_Mukamel1995}. For a dimer, the transition dipole moment operator ${\bm \mu}$ is given by
\begin{equation}
    {\bm \mu}=\sum_{j=1}^{2}{\bm \mu}_{j}(\ket{\epsilon_j}\bra{g}+\ket{g}\bra{\epsilon_j}).
\end{equation}
For simplicity, we assume that the transition dipole moments of two pigments are mutually orthogonal (${\bm \mu}_1 \cdot {\bm \mu}_2=0$), as is the case of the first two lowest-energy pigments (sites 3 and 4) of the FMO complex~\cite{SI_MilderPR2010}, so that both low- and high-energy excitons are bright ($\bra{E_{\pm}}{\bm \mu}\ket{g}\neq 0$). The lifetime of optical coherences, ${\rm tr}_{v}[e^{-iHt}{\bm \mu}\rho(0)e^{iHt}]$, and corresponding absorption line widths are determined by the homogeneous and inhomogeneous broadenings induced by the vibronic coupling $H_{e-v}$ and static disorder, respectively.



\section{Static disorder}

When the site energies $\epsilon_j$ of a dimer are randomly generated from independent Gaussian distributions with mean values $\langle\epsilon_j\rangle$ and standard deviation $\sigma$, the ensemble-averaged dipole-dipole correlation function is given by
\begin{equation}
    \langle D(t)\rangle_{\epsilon_1,\epsilon_2} = \frac{1}{2\pi\sigma^2}\int_{-\infty}^{\infty}\int_{-\infty}^{\infty}d\epsilon_1 d\epsilon_2 \exp\left(-\frac{1}{2\sigma^2}(\epsilon_1-\langle\epsilon_1\rangle)^{2}-\frac{1}{2\sigma^2}(\epsilon_2-\langle\epsilon_2\rangle)^{2}\right){\rm tr}[{\bm \mu}\cdot (e^{-iHt}{\bm \mu}\rho(0)e^{iHt})],
\end{equation}
where the total Hamiltonian $H$ includes the electronic Hamiltonian $H_e$ depending on the site energies $\epsilon_j$. When the Gaussian distributions are expressed as a function of site energy difference $\xi_1=\epsilon_1-\epsilon_2$ and average site energy $\xi_2=(\epsilon_1+\epsilon_2)/2$, the ensemble-averaged dipole-dipole correlation function is given by
\begin{equation}
    \langle D(t)\rangle_{\epsilon_1,\epsilon_2} = \frac{1}{2\pi\sigma^2}\int_{-\infty}^{\infty}\int_{-\infty}^{\infty}d\xi_1 d\xi_2 \exp\left(-\frac{1}{4\sigma^2}(\xi_1-\langle\xi_1\rangle)^{2}-\frac{1}{\sigma^2}(\xi_2-\langle\xi_2\rangle)^{2}\right){\rm tr}[{\bm \mu}\cdot (e^{-iHt}{\bm \mu}\rho(0)e^{iHt})].
\end{equation}
Since the average site energy $\xi_2$ describes a global shift in site energies, the ensemble average over $\xi_2$ does not require the repetition of numerical simulations as its influence on the dipole-dipole correlation function can be computed analytically as
\begin{equation}
    \langle D(t)\rangle_{\epsilon_1,\epsilon_2} = \frac{1}{\sqrt{\pi}\sigma}\int_{-\infty}^{\infty}d\xi_2 \exp\left(-\frac{1}{\sigma^2}(\xi_2-\langle\xi_2\rangle)^{2}\right) \langle D(t)\rangle_{\xi_1,\langle\xi_2\rangle} = e^{-\sigma^2 t^2/4} \langle D(t)\rangle_{\xi_1,\langle\xi_2\rangle},
\end{equation}
where $\langle D(t)\rangle_{\xi_1,\langle\xi_2\rangle}$ represents the ensemble average over $\xi_1$ for a fixed value of $\xi_2=\langle \xi_2 \rangle$. Note that $e^{-\sigma^2 t^2/4}$ is the Gaussian dephasing discussed in the main manuscript.



\section{Line shape theory}

For a monomer, the dynamics of the optical coherence $\ket{e}\bra{g}$ between electronic ground $\ket{g}$ and excited state $\ket{e}$ can be readily computed in a non-perturbative manner, resulting in monomer absorption spectra~\cite{SI_DinhJCP2015}
\begin{equation}
	A(\omega)\, {\propto}\, {\rm Re} \int_{0}^{\infty} dt |{\bm\mu}|^{2}e^{i(\omega-\epsilon+\lambda)t+G(t)-G(0)},
\end{equation}
where $|{\bm\mu}|^{2}$ denotes the transition dipole strength of the monomer with the transition energy $\epsilon$ between electronic states, $\lambda = \int_{0}^{\infty} d\omega J(\omega)/\omega$ is the reorganization energy and $G(t)$ describes the influence of electronic-vibrational interaction on optical coherence dynamics
\begin{equation}
	G(t)=\int_{0}^{\infty}d\omega \frac{J(\omega)}{\omega^2}((n(\omega)+1)e^{-i\omega t}+n(\omega) e^{i\omega t}).
\end{equation}

In case of a dimer, the dynamics of the optical coherences $\ket{E_\pm}\bra{g}$ between electronic ground $\ket{g}$ and exciton states $\ket{E_\pm}$ requires numerical simulations. In a line shape theory, the diagonal vibronic couplings in the exciton basis (${\propto}\ket{E_\pm}\bra{E_\pm}$) are treated non-perturbatively, while the off-diagonal vibronic couplings in the exciton basis (${\propto}\ket{E_\pm}\bra{E_\mp}$) are considered perturbatively, leading to dimer absorption spectra in the form
\begin{equation}
	A(\omega)\, {\propto}\, {\rm Re} {\int_{0}^{\infty}} dt {\sum_{k = \pm}} |{\bm\mu}_{E_j}|^{2} e^{i(\omega-E_{k}+\lambda_{k})t+G_{k}(t)-G_{k}(0)-\frac{1}{2}\Gamma_{k\rightarrow k'}t}.
\end{equation}
Here $|{\bm\mu}_{E_k}|^{2}=|\bra{E_k}{\bm\mu}\ket{g}|^{2}$ denotes the transition dipole strength of an exciton state $\ket{E_k}$, while $\lambda_k=\sum_{j=1}^{2}|\langle \epsilon_j | E_k\rangle |^{4}\lambda$ and $G_{k}(t)=\sum_{j=1}^{2}|\langle \epsilon_j | E_k\rangle |^{4} G(t)$ with $\lambda$ and $G(t)$ defined for the monomer case. $\lambda_k$ and $G_k(t)$ originate from the diagonal vibronic couplings and their magnitudes are reduced as the excitons are delocalized over two pigments.

The perturbative treatment of the off-diagonal vibronic couplings within the Born-Markov-secular approximations~\cite{SI_Petruccione2000,SI_DinhJCP2015} leads to an incoherent relaxation rate $\Gamma_{k\rightarrow k'}$ from $\ket{E_{k=\pm}}$ to the other exciton state $\ket{E_{k'=\mp}}$, described by
\begin{equation}
	\Gamma_{k\rightarrow k'} = 2\pi \sum_{j=1}^{2} |\langle E_k | \epsilon_j\rangle\langle \epsilon_j | E_{k'}\rangle |^{2} (J(\Delta_{kk'}) (n(\Delta_{kk'})+1)+J(-\Delta_{kk'})n(-\Delta_{kk'})),
\end{equation}
with $J(\omega)=0$ for $\omega<0$ and $n(\omega)=(\exp(\hbar\omega/k_B T)-1)^{-1}$, where $\Delta_{kk'}=E_k-E_k'$ represents the energy-gap between exciton states $\ket{E_k}$ and $\ket{E_{k'}}$, and $T$ the temperature of vibrational environments. In addition to the exciton relaxation, the perturbative treatment of the off-diagonal vibronic couplings induces exciton energy renormalization described by the Lamb shift where the exciton energies $E_\pm$ are shifted to
\begin{equation}
	E_{\pm}\rightarrow E_{\pm}+\sum_{j=1}^{2}|\langle E_+ | \epsilon_j \rangle \langle \epsilon_j | E_-\rangle |^{2} {\rm P} \int_{-\infty}^{\infty} d\omega \frac{J(\omega)(n(\omega)+1)+J(-\omega)n(-\omega)}{\pm \Delta -\omega},
\end{equation}
where ${\rm P}$ denotes the Cauchy principal value and $\Delta=E_+-E_->0$ a bare excitonic splitting.



\section{Avoided crossing in dimer absorption spectra}

\begin{figure*}
\includegraphics[width=\textwidth]{Fig2v7.pdf}
\caption{(a) Approximate linear absorption spectra of dimers shown as a function of inter-site electronic coupling $V$, simulated by the line shape theory where off-diagonal vibronic couplings are ignored. The main peak $\ket{E_\pm,0}$ and vibrational sideband $\ket{E_\pm,1_k}$ of exciton states are highlighted by dashed lines, leading to multiple crossings between $\ket{E_+,0}$ and $\ket{E_-,1_k}$. (b) Numerically exact absorption spectra computed by HEOM where the off-diagonal vibronic couplings are treated non-perturbatively. (c,d) Approximate absorption spectra obtained by the line shape theory where the off-diagonal vibronic couplings are treated perturbatively, leading to incoherent relaxation rates between excitons and Lamb shift. In (c), only the incoherent exciton relaxation is considered, while in (d), both the exciton relaxation and Lamb shift are considered. In all the simulations, the full FMO spectral density $J_{\rm AR}(\omega)+J_{h}(\omega)$ is considered.}
\label{FigS1}
\end{figure*}

To investigate the role of the off-diagonal vibronic couplings in linear optical responses, we compare numerically exact absorption spectra at $77\,{\rm K}$ computed by HEOM with approximate results obtained by the line shape theory. Here we consider a homodimer with identical mean site energies $\langle\epsilon_1\rangle= \langle\epsilon_2\rangle$ and the full FMO spectral density $J_{\rm AR}(\omega)+J_{h}(\omega)$.

In Fig.~\ref{FigS1}(a), approximate absorption spectra obtained by neglecting the off-diagonal vibronic couplings are shown as a function of the electronic coupling $V$.
Here the diagonal vibronic couplings lead to a main absorption peak and relatively weak vibrational sideband for each exciton state $\ket{E_\pm}$. The vibrational sideband of a monomer ($V=0$) consists of two small peaks (see the main manuscript). This leads to a total of six peaks in dimer absorption spectra, where three of them originate from the low-energy exciton state $\ket{E_-}$, red-shifted as $V$ increases, and three from the high-energy exciton state $\ket{E_+}$, blue-shifted as a function of $V$. 
The dependence of the main absorption peaks on $V$ is well described by the analytical form of exciton energies $E_\pm$ (see the main manuscript). It is notable that the main absorption peak of the high-energy exciton $\ket{E_+}$ crosses the vibrational sideband of the low-energy exciton $\ket{E_-}$ when the electronic coupling strength is around $300 \lesssim V \lesssim 600\,{\rm cm}^{-1}$. Such an electronic coupling strength has been found from several PPCs, such as light-harvesting complex (LH2)~\cite{SI_ScholesJPCB1999} and bacterial reaction center~\cite{SI_KhmelnitskiyJPCB2019} from purple bacteria, and PC645 complex~\cite{SI_ColliniNature2010} from marine algae.

\begin{figure}
\includegraphics[width=0.75\textwidth]{New_Fig5.pdf}
\caption{(a) Dimer absorption spectra at $77\,{\rm K}$, computed by HEOM, are shown as a function of electronic coupling $V$. For weak ($V\lesssim 300\,{\rm cm}^{-1}$) and strong ($V\gtrsim 600\,{\rm cm}^{-1}$) electronic coupling, absorption spectra are dominated by two main peaks with an energy-gap $\Delta'$ being smaller and larger than bare excitonic splitting $\Delta\approx 2V$, respectively. (b) Absorption spectra for a few selected electronic couplings $V\in\{100,300,450,600\}\,{\rm cm}^{-1}$ are shown where absorption line shapes exhibit more than two main peaks for $V=450\,{\rm cm}^{-1}$ due to multiple avoided crossings. (c) The full FMO phonon spectral density considered in HEOM simulations is shown together with three regimes where absorption line shapes are modified by the avoided crossing (or equivalently multimode vibronic mixing) in different manners. When the bare excitonic splitting $\Delta$ is off-resonant from the vibrational frequency spectrum of intra-pigment modes, the energy-gap $\Delta'$ between two main absorption peaks is red- or blue-shifted from $\Delta$. Otherwise more than two absorption peaks with comparable amplitudes may appear due to multiple avoided crossings.}
\label{FigS2}
\end{figure}

\begin{figure}
\includegraphics[width=0.8\textwidth]{Fig4v2.pdf}
\caption{(a)-(c) Linear absorption spectra of heterodimers shown as a function of inter-site electronic coupling $V$, simulated by HEOM in a numerically exact manner for mean site energy differences (a) $\langle\epsilon_2 - \epsilon_1\rangle = 300\,{\rm cm}^{-1}$, (b) $600\,{\rm cm}^{-1}$ and (c) $900\,{\rm cm}^{-1}$. The model parameters are as in the homodimer case ($\langle\epsilon_2 - \epsilon_1\rangle = 0$) considered in the main manuscript, including static disorder.}
\label{FigS3}
\end{figure}

In Fig.~\ref{FigS1}(b), numerically exact absorption spectra are shown where both diagonal and off-diagonal vibronic couplings are treated non-perturbatively by using HEOM. It is notable that absorption line shapes are significantly modified over a broad range of $V$ around the crossing points marked in Fig.~\ref{FigS1}(a). 
These results can be rationalized based on the fact that the off-diagonal vibronic couplings induce transitions between exciton states while creating or annihilating a phonon. Since the vibrational sideband of the low-energy exciton state mainly originates from vibrationally hot states $\ket{E_-,1_k}$ where one of the intra-pigment modes is singly excited, while the main absorption peak of the high-energy exciton comes from a vibrationally cold state $\ket{E_+,0}$ where all the intra-pigment modes are in their vacuum states, the off-diagonal vibronic couplings can induce a vibronic mixing between them, resulting in avoided crossings. It is notable that the avoided crossing causes a red- and blue-shift of the main peak of the high-energy exciton state $\ket{E_+}$ for weak ($V \lesssim 300\,{\rm cm}^{-1}$) and strong ($V \gtrsim 600\,{\rm cm}^{-1}$) electronic couplings, respectively, as highlighted in Fig.~\ref{FigS2}(a). This is in line with the results reported in Ref.~\cite{SI_Caycedo2022} where the discussions have been mainly limited to the cases where the bare excitonic splitting $\Delta$ is smaller or larger than the vibrational frequency spectrum of intra-pigment modes and absorption spectra are dominated by two main absorption peaks with a shifted energy-gap $\Delta'$. Note that when the bare excitonic splitting $\Delta$ is in the middle of the vibrational frequency spectrum of intra-pigment modes, absorption line shapes exhibit more than two peaks with comparable amplitudes, as shown in Fig.~\ref{FigS2}(b), which cannot be characterized by a shift $\Delta\rightarrow\Delta'$ of two main absorption peaks. Our results therefore provide a comprehensive picture of vibronic modification of absorption line shapes for the entire range of excitonic splitting $\Delta$ across the vibrational frequency spectrum of PPCs, as schematically shown in Fig.~\ref{FigS2}(c). The avoided crossing effect cannot be described by the conventional line shape theory~\cite{SI_Caycedo2022}, as shown in Fig.~\ref{FigS1}(c) and (d), where the incoherent exciton relaxation is considered without and with the Lamb shift, respectively. The avoided crossing effect dominates absorption line shapes even in the presence of detunings in mean site energies, $\langle \epsilon_1-\epsilon_2\rangle \in \{300,600,900\}\,{\rm cm}^{-1}$, as shown in Fig.~\ref{FigS3}.

We note that the line shape theory has been widely employed to simulate the optical spectra of molecular systems and its accuracy has been demonstrated for a few broad model spectral densities by perturbatively taking into account non-secular and non-Markov corrections~\cite{SI_DinhJCP2015} or by using a numerically exact HEOM method~\cite{SI_GelzinisJCP2015}. On the contrary, our results show that when underdamped intra-pigment modes, present in real systems, are considered, the line shape theory cannot reproduce numerically exact absorption spectra of dimers unless the bare excitonic splitting $\Delta\approx 2V$ is sufficiently off-resonant from the vibrational frequency spectrum of molecular systems, namely $|V|\ll 100\,{\rm cm}^{-1}$ or $|V|\gg 800\,{\rm cm}^{-1}$ where a vibronic mixing induced by off-diagonal vibronic couplings becomes negligibly small, as shown in Fig.~\ref{FigS1}.



\section{Multimode vibronic mixing}

\begin{figure*}
\includegraphics[width=\textwidth]{Fig6v7.pdf}
\caption{(a) Linear absorption spectra of homodimers with $\langle\epsilon_1\rangle = \langle\epsilon_2\rangle = 12\times 10^3 {\rm cm}^{-1}$, $V=450\,{\rm cm}^{-1}$ computed exactly by using HEOM (blue) or approximately by using a reduced vibronic model (red). Neglecting static disorder, the fidelity $P_0^{E_+}$ ($P_0^{E_-}$) between bright vibronic eigenstates of the reduced vibronic model and vibrationally cold exciton state $|E_+,0\rangle$ ($|E_-,0\rangle$) is shown in black (red) dots. The energies of the vibronic eigenstates are red-shifted by the half of the total reorganization energy, induced by center-of-mass modes, as only relative-motion modes are considered when computing the vibronic eigenstates~\cite{SI_Caycedo2022}. (b)-(f) For a few selected vibronic eigenstates marked in (a), the fidelity $P_{0}^{E_\pm}$ between vibronic eigenstates and vibrationally cold exciton states $\ket{E_\pm,0}$ and the average population $P_{q}^{E_\pm}$ of the $q$-th relative intra-pigment mode conditional to exciton states are shown. $P_{0}^{E_\pm}$ and $P_{q}^{E_\pm}$ are shown as a function of vibrational frequency, namely $P_{0}^{E_\pm}$ centered at zero frequency and $P_{q}^{E_\pm}$ at the vibrational frequency $\omega_q$ of the $q$-th intra-pigment vibrational mode summarized in Table~\ref{Table_intrapigment}, to visualize which intra-pigment modes are involved in a vibronic mixing. For better visibility, the phonon spectral density of the FMO complex is shown with the vibrational frequencies $\omega_q$ of 62 intra-pigment modes marked by vertical lines. In (c)-(f), the maximum of $P_{q}^{E_-}$ is 0.280, 0.361, 0.396, 0.681, respectively. With $\omega_Q$ denoting the vibrational frequency of the maximally populated mode ($P_{Q}^{E_-}>P_{q}^{E_-}$ for all $q\neq Q$), the sum of $P_{q}^{E_-}$ over $\omega_Q-50\,{\rm cm}^{-1}\le \omega_q \le \omega_Q+50\,{\rm cm}^{-1}$, marked in red rectangles, is 0.751, 0.794, 0.620, 0.871, respectively. The sum of the remaining $P_{q}^{E_-}$ with $\omega_q < \omega_Q-50\,{\rm cm}^{-1}$ or $\omega_q>\omega_Q+50\,{\rm cm}^{-1}$ is 0.161, 0.133, 0.273, 0.047, respectively.}
\label{FigS4}
\end{figure*}

\begin{figure*}
\includegraphics[width=\textwidth]{Fig7.pdf}
\caption{(a) Linear absorption spectra of homodimers with $\langle\epsilon_1\rangle = \langle\epsilon_2\rangle = 12\times 10^3 {\rm cm}^{-1}$, $V=100\,{\rm cm}^{-1}$ computed exactly by using HEOM (blue) or approximately by using a reduced vibronic model (red). Neglecting static disorder, the fidelity $P_0^{E_+}$ ($P_0^{E_-}$) between bright vibronic eigenstates of the reduced vibronic model and vibrationally cold exciton state $|E_+,0\rangle$ ($|E_-,0\rangle$) is shown in black (red) dots. The energies of the vibronic eigenstates are red-shifted by the half of the total reorganization energy, induced by center-of-mass modes, as only relative-motion modes are considered when computing the vibronic eigenstates~\cite{SI_Caycedo2022}. (b)-(c) For two vibronic eigenstates marked in (a), the fidelity $P_{0}^{E_\pm}$ between vibronic eigenstates and vibrationally cold exciton states $\ket{E_\pm,0}$ and the average population $P_{q}^{E_\pm}$ of the $q$-th relative intra-pigment mode conditional to exciton states are shown. In (c), the maximum of $P_{q}^{E_-}$ is 0.0541. With $\omega_Q$ denoting the vibrational frequency of the maximally populated mode ($P_{Q}^{E_-}>P_{q}^{E_-}$ for all $q\neq Q$), the sum of $P_{q}^{E_-}$ over $\omega_Q-50\,{\rm cm}^{-1}\le \omega_q \le \omega_Q+50\,{\rm cm}^{-1}$, marked in a red rectangle, is 0.305. The sum of the remaining $P_{q}^{E_-}$ with $\omega_q < \omega_Q-50\,{\rm cm}^{-1}$ or $\omega_q>\omega_Q+50\,{\rm cm}^{-1}$ is 0.138.}
\label{FigS5}
\end{figure*}

Here we demonstrate that a multimode vibronic mixing is responsible for the avoided crossings observed in dimer absorption spectra by using a reduced vibronic model constructed within double vibrational excitation subspace in the polaron basis.

In this work, we consider a dimer where each site is coupled to an independent vibrational environment. As discussed in Ref.~\cite{SI_Caycedo2022}, two local phonon baths can be mapped to two shared phonon baths consisting of center-of-mass and relative-motion modes, respectively. The center-of-mass modes cause the dephasing of optical coherences, which can be readily described in a non-perturbative manner. The relative-motion modes are responsible for a vibronic mixing between excitons mediated by the off-diagonal vibronic couplings in the exciton basis. For simplicity, we explicitly consider the relative motion of 62 intra-pigment modes of the FMO complex, for which double vibrational excitation subspace is spanned by $\ket{0}$, where all the relative intra-pigment modes are in their vacuum states, $\ket{1_k}$ ($\ket{2_k}$), where only the $k$-th mode is singly (doubly) excited while all the other modes are in their vacuum states, and $\ket{1_k,1_{k'}}$, where two different modes $k\neq k'$ are singly excited at the same time. Here the off-diagonal vibronic couplings in the exciton basis can induce a vibronic mixing between vibrationally cold and singly excited states, such as $\ket{E_\pm,0}$ and $\ket{E_\mp,1_k}$, and a mixing between singly and doubly excited vibrational states, such as $\ket{E_\mp,1_k}$, $\ket{E_\pm,2_k}$ and $\ket{E_\pm,1_k,1_{k'}}$. The vibronic eigenstates $\ket{\psi_l}$ within the double vibrational excitation subspace can be computed numerically, enabling one to investigate which vibrational modes are involved in the vibronic mixing.

As an example, we consider an electronic coupling $V=450\,{\rm cm}^{-1}$ of a homodimer in Fig.~\ref{FigS4}, for which numerically exact absorption spectrum exhibits multiple avoided crossings. Fig.~\ref{FigS4}(a) demonstrates that the numerically exact absorption spectrum computed by HEOM, shown in blue, is quantitatively well matched to approximate results obtained by the reduced vibronic model, shown in red, where the blue line is hardly visible due to an excellent overlap with the red line. In addition, for $\epsilon_1=\epsilon_2$ (no static disorder), the fidelity between vibronic eigenstates $\ket{\psi_l}$ and vibrationally cold exciton state $\ket{E_+,0}$ ($\ket{E_-,0}$) is shown in black (red) dots, defined by
\begin{equation}
    P_{0}^{E_\pm}=|\langle E_\pm,0 | \psi_l\rangle |^{2}.
\end{equation}
Fig.~\ref{FigS4}(a) demonstrates that the lowest-energy absorption peak is induced by a single vibronic eigenstate well described by $\ket{E_-,0}$, while the other high-lying absorption peaks originate from multiple vibronic eigenstates with comparable overlaps with $\ket{E_+,0}$, namely $P_{0}^{E_+}\lesssim 0.1$.

To analyze the vibronic mixing structure of each vibronic eigenstate $\ket{\psi_l}$, we consider the average population of the $q$-th relative intra-pigment mode conditional to the exciton states
\begin{equation}
    P_{q=1,2,\cdots,62}^{E_\pm}=\langle \psi_l | (|E_\pm\rangle\langle E_\pm|\otimes N_{q=1,2,\cdots,62}) | \psi_l\rangle,
\end{equation}
with $N_q=b^\dagger_q b_q$ denoting the number operator of the $q$-th mode. For the four absorption peaks marked in Fig.~\ref{FigS4}(a), we found that the vibronic eigenstates contributing to the same peak have a similar structure of vibronic mixing. Fig.~\ref{FigS4}(b) shows that the lowest-energy absorption peak is dominated by a single vibronic eigenstate that can be well described as the vibrationally cold low-energy exciton state $\ket{E_-,0}$ with $P_0^{E_-}\approx 1$. Fig.~\ref{FigS4}(c)-(f) show the vibronic mixing structure of the brightest vibronic eigenstate contributing to each absorption peak, marked by black arrows in Fig.~\ref{FigS4}(a), where the vibrationally cold high-energy exciton state $\ket{E_+,0}$ is mixed with multiple vibrationally hot low-energy exciton states, such as $\ket{E_-,1_k}$, described by a broad distribution of $P_q^{E_-}$. It is notable that the vibrational frequencies of the intra-pigment modes involved in the vibronic mixing are not necessarily near-resonant with the bare excitonic splitting $\Delta\approx 2V$ of a dimer, hinting that vibronic coupling strengths are beyond the weak coupling regime.

We note that when the bare excitonic splitting $\Delta$ is smaller or larger than the vibrational frequency spectrum of intra-pigment modes, dimer absorption spectra are dominated by two main peaks originating from vibronic eigenstates $\ket{\psi_+}$ and $\ket{\psi_-}$ with a relatively strong excitonic character, namely $P_{0}^{E_+}\gtrsim 0.1$ and $P_{0}^{E_-}\approx 1$, respectively~\cite{SI_Caycedo2022}. As an example, Fig.~\ref{FigS5} show the vibronic eigenstate analysis of dimer absorption spectra for $V=100\,{\rm cm}^{-1}$, where the lowest-energy peak is induced by a vibronic eigenstate in the form $\ket{E_-,0}$, while the high-energy main peak is dominated by a vibronic eigenstate with $P_{0}^{E_+}=0.528$.

We remark that the avoided crossings in dimer absorption spectra can also be observed even if two pigments are coupled to different phonon spectral densities. This is due to the fact that the avoided crossing features are determined by the effective spectral density of the relative motion modes, which can be constructed even if the local phonon baths have different vibronic coupling spectra. In addition, the energy-levels of vibronic eigenstates, underlying the avoided crossing structures, are determined by the vibronic coupling spectra, but not by the angle between the transition dipole moments of pigments, which is taken to be $90^\circ$, as an example, in our simulations. The variation of the angle only redistributes the brightness of the vibronic eigenstates, affecting the degree of visibility of avoided crossing features (not shown here).



\section{Circular dichroism}

\begin{figure}
\includegraphics[width=0.8\textwidth]{Fig5v4.pdf}
\caption{(a)-(c) Circular dichroism spectra of homodimers ($\langle\epsilon_2 - \epsilon_1\rangle = 0$) shown as a function of inter-site electronic coupling $V$, simulated by (a) HEOM or by (b,c) a line shape theory where the off-diagonal vibronic couplings in the exciton basis are treated perturbatively (b) without and (c) with the Lamb shift. The model parameters are as in the absorption case considered in the main manuscript.}
\label{FigS6}
\end{figure}

So far we have shown that the absorption line shapes of dimers are dominated by multiple avoided crossings resulting from multimode vibronic mixing. To demonstrate that such avoided crossing features also appear in other linear optical spectra, we now consider circular dichroism (CD) defined as the difference in absorbance of left and right circularly polarized light.

For isotropic samples, the CD spectrum is described by
\begin{equation}
	{\rm CD}(\omega)\, {\propto}\, {\rm Re}\int_{0}^{\infty} dt e^{i\omega t}{\rm Tr}[{\bm\mu}\cdot (U(t) {\bm m} \rho(0) U^{\dagger}(t))],
\end{equation}
where ${\bm m}$ represents the magnetic transition dipole moment
\begin{equation}
	{\bm m}= \sum_{j=1}^{2}({\bm R}_{j}\times{\bm\mu}_{j})(\ket{\epsilon_j}\bra{g}+\ket{g}\bra{\epsilon_j}),
\end{equation}
with ${\bm \mu}_{j}$ and ${\bm R}_{j}$ denoting the electric transition dipole moment and center position, respectively, of pigment $j$. It can be shown that the CD spectrum is invariant under the translation ${\bm R}_{j} \rightarrow {\bm R}_{j}+{\bm R}_{0}$ for arbitrary ${\bm R}_{0}$. Fig.~\ref{FigS6} shows the CD spectra of homodimers computed by HEOM and line shape theory. As is the case of the absorption spectra, the numerically exact CD spectra cannot be reproduced by the perturbative line shape theory. 



\section{Simulation-time dependence of coarse-graining}

In the main manuscript, the bath correlation function (BCF) of the full FMO complex up to $300\,{\rm fs}$ is considered when constructing a coarse-grained spectral density, as it 
is associated with the time scale of absorption measurements, namely the typical lifetime of optical coherences between electronic ground and excited states. It is found that 62 intra-pigment modes present in the actual FMO spectral density can be approximately described by five Lorentzian peaks based on our course-graining scheme, which can be estimated from the number of peaks present in the frequency spectrum of the BCF up to $300\,{\rm fs}$, as shown in Fig.~\ref{Fig9}(a).

As the time scale of interest increases, the frequency spectrum of the BCF can have a larger number of peaks. As examples, in Fig.~\ref{Fig9}(b) and (c), we consider the typical time scale of energy transfer dynamics in PPCs ($1\,{\rm ps}$), and the time scale associated with nonlinear spectroscopy measurements ($2\,{\rm ps}$), such as pump probe and two-dimensional electronic spectroscopy. It is notable that the number of peaks present in the frequency spectrum is increased from 5, via 16, to 28, as the time scale of interest increases from $300\,{\rm fs}$, via $1\,{\rm ps}$, to $2\,{\rm ps}$, but they are still smaller than the total number of intra-pigment modes in the full FMO spectral density ($M=62$). Therefore, although our coarse-graining scheme becomes less efficient as the time scale of interest increases, it still requires less computational resources than the full FMO spectral density when computational costs strongly depend on the number of peaks present in the input spectral density entering non-perturbative simulation methods, such as HEOM and DAMPF (see the main manuscript).

\begin{figure}
\includegraphics[width=0.85\textwidth]{Fig9.pdf}
\caption{(a)-(c) Fourier transformation of the bath correlation function $C(t)$ of the full FMO spectral density at $77\,{\rm K}$ weighted by a Gaussian function $e^{-t^{2}/2\tilde{\sigma}^2}$ with $3\tilde{\sigma}=300\,{\rm fs}, 1\,{\rm ps}, 2\,{\rm ps}$. The number of peaks in the FT spectra is increased from 5, via 16, to 28 as the time scale of interest is increased from $300\,{\rm fs}$ (absorption measurements), via $1\,{\rm ps}$ (energy transfer dynamics), to $2\,{\rm ps}$ (nonlinear spectroscopy measurements). The total number of intra-pigment modes considered in the FMO spectral density is 62.}
\label{Fig9}
\end{figure}



\section{Bond dimension required for DAMPF simulations of absorption spectra}

\begin{figure}
\includegraphics[width=0.8\textwidth]{Fig8.pdf}
\caption{(a) Optical coherence dynamics of the full FMO model ($N=7$ sites and $M=62$ intra-pigment modes per site) weighted by a Gaussian function $e^{-t^2/2\tilde{\sigma}^2}$ with $3\tilde{\sigma}=300\,{\rm fs}$, computed by DAMPF with bond dimensions $\chi=4$ and $\chi=30$. (b) The difference between DAMPF results obtained with $\chi=4$ and $\chi=30$.}
\label{Fig8}
\end{figure}

As discussed in the main manuscript, we found that DAMPF simulations of absorption spectra of a seven-site FMO model require low bond dimensions $\chi$~\cite{SI_SomozaPRL2019}. In Fig.~\ref{Fig8}, we show the optical coherence dynamics of the full FMO model, consisting of $N=7$ pigments and $M=62$ intra-pigment modes per site, weighted by a Gaussian function as in
\begin{equation*}
    A(t) = {\rm tr}[{\bm \mu}\cdot (e^{-iHt}{\bm \mu}\rho(0)e^{iHt})] e^{-t^2/2\tilde{\sigma}^2}
\end{equation*}
with $3\tilde{\sigma}=300\,{\rm fs}$, where $H$ is the electronic Hamiltonian characterized by mean site energies. Here the Gaussian function makes $A(t)$ decays within $300\,{\rm fs}$, so that the Fourier transformation of $A(t)$ leads to well-defined absorption line shapes without ringing artifacts. As shown in Fig.~\ref{Fig8}, $A(t)$ computed up to $300\,{\rm fs}$ by DAMPF with bond dimensions $\chi=4$ and $\chi=30$ are quantitatively well-matched, implying that low bond dimensions $\sim 4$ are sufficient to obtained numerically converged absorption spectra of the full FMO model.



\begin{thebibliography}{99}

\bibitem{SI_RatsepJL2007} M. R{\"a}tsep and A. Freiberg. Electron Phonon and Vibronic Couplings in the FMO Bacteriochlorophyll a Antenna Complex Studied by Difference Fluorescence Line Narrowing. J. Lumin. {\bf 127}, 251 (2007).

\bibitem{SI_Mukamel1995} S. Mukamel. Principles of Nonlinear Optical Spectroscopy (Oxford University Press, 1995).

\bibitem{SI_MilderPR2010} M. T. W. Milder, B. Br{\"u}ggemann, R. van Grondelle, and J. L. Herek. Revisiting the Optical Properties of the FMO Protein. Photosynth Res. {\bf 104}, 257 (2010).

\bibitem{SI_DinhJCP2015} T.-C. Dinh and T. Renger. Towards an Exact Theory of Linear Absorbance and Circular Dichroism of Pigment-Protein Complexes: Importance of Non-Secular Contributions. J. Chem. Phys. {\bf 142}, 034104 (2015).

\bibitem{SI_Petruccione2000} H. P. Breuer and F. Petruccione. The Theory of Open Quantum Systems (Oxford University Press, 2002).

\bibitem{SI_ScholesJPCB1999} G. D. Scholes, I. R. Gould, R. J. Cogdell, and G. R. Fleming. Ab Initio Molecular Orbital Calculations of Electronic Couplings in the LH2 Bacterial Light-Harvesting Complex of Rps. Acidophila. J. Phys. Chem. B {\bf 103}, 2543 (1999).

\bibitem{SI_KhmelnitskiyJPCB2019} A. Khmelnitskiy, T. Reinot, and R. Jankowiak. Mixed Upper Exciton State of the Special Pair in Bacterial Reaction Centers. J. Phys. Chem. B {\bf 123}, 852 (2019).

\bibitem{SI_ColliniNature2010} E. Collini, C. Y. Wong, K. E. Wilk, P. M. G. Curmi, P. Brumer, G. D. and Scholes. Coherently Wired Light-Harvesting in Photosynthetic Marine Algae at Ambient Temperature. Nature {\bf 463}, 644 (2010).

\bibitem{SI_Caycedo2022} F. Caycedo-Soler, A. Mattioni, J. Lim, T. Renger, S. F. Huelga, and M. B. Plenio. Exact Simulation of Pigment-Protein Complexes Unveils Vibronic Renormalization of Electronic Parameters in Ultrafast Spectroscopy. Nat. Commun. {\bf 13}, 2912 (2022).

\bibitem{SI_GelzinisJCP2015} A. Gelzinis, D. Abramavicius, and L. Valkunas. Absorption Lineshapes of Molecular Aggregates Revisited. J. Chem. Phys. {\bf 142}, 154107 (2015).

\bibitem{SI_SomozaPRL2019} A. D. Somoza, O. Marty, J. Lim, S. F. Huelga, and M. B. Plenio. Dissipation-Assisted Matrix Product Factorization. Phys. Rev. Lett. {\bf 123}, 100502 (2019).

\end{thebibliography}

\end{widetext}

\end{document}





