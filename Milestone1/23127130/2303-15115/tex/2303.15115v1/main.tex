%\documentclass[conference]{IEEEtran}
\documentclass[letterpaper, 10 pt, conference]{ieeeconf}
\IEEEoverridecommandlockouts
% \overrideIEEEmargins                                                             
                                                          
\synctex=1 
%\usepackage{cite}
\usepackage{amsmath,amssymb,amsfonts}
%\usepackage{graphicx}
\usepackage[pdftex]{graphicx}
\usepackage{textcomp}
\usepackage{xcolor}
\usepackage{hyperref}
\usepackage{amsmath}
\usepackage[noend]{algpseudocode}
\usepackage{comment}
\usepackage[normalem]{ulem}
\usepackage[colorinlistoftodos]{todonotes}
\usepackage[framemethod=tikz]{mdframed}
\newtheorem{dfn}{Definition}
\newtheorem{asm}{Assumption}
\newtheorem{con}{Conjecture}
\newtheorem{problem}{Problem}
\newtheorem{remark}{Remark}

\def\BibTeX{{\rm B\kern-.05em{\sc i\kern-.025em b}\kern-.08em
    T\kern-.1667em\lower.7ex\hbox{E}\kern-.125emX}}
    

\usepackage{graphicx}
\usepackage{psfrag}
\usepackage{pstool}
\usepackage{multirow}
\usepackage{soul}
\usepackage{gensymb}
\algnewcommand\algorithmicforeach{\textbf{for each}}
\algdef{S}[FOR]{ForEach}[1]{\algorithmicforeach\ #1\ \algorithmicdo}



\newcommand{\argmin}{\mathop{\mathrm{arg\,min}}}




% Multiple Reference to footnote 
\newcommand{\footlabel}[2]{%
    \addtocounter{footnote}{1}%
    \footnotetext[\thefootnote]{%
        \addtocounter{footnote}{-1}%
        \refstepcounter{footnote}\label{#1}%
        #2%
    }%
    $^{\ref{#1}}$%
}
\newcommand{\usefootref}[1]{%
    $^{\ref{#1}}$%
}

\DeclareUnicodeCharacter{0301}{\'{e}}

\usepackage{xspace}

\makeatletter

\DeclareRobustCommand\onedot{\futurelet\@let@token\@onedot}
\def\@onedot{\ifx\@let@token.\else.\null\fi\xspace}

\def\eg{\emph{e.g}\onedot} \def\Eg{\emph{E.g}\onedot}
\def\ie{\emph{i.e.}\onedot} \def\Ie{\emph{I.e}\onedot}
\def\cf{\emph{c.f}\onedot} \def\Cf{\emph{C.f}\onedot}
\def\etc{\emph{etc}\onedot} \def\vs{\emph{vs}\onedot}
\def\wrt{w.r.t\onedot} \def\dof{d.o.f\onedot}
\def\etal{\emph{et al}\onedot}


% \usepackage{etoolbox}
% \makeatletter
% \patchcmd{\@makecaption}
%   {\scshape}
%   {}
%   {}
%   {}
% \makeatother

\usepackage{algorithm,algpseudocode,setspace}

\DeclareMathOperator*{\argmax}{arg\,max}
\begin{comment}

\newcounter{algsubstate}
\makeatletter
\renewcommand{\thealgsubstate}{\arabic{ALG@line}.\arabic{algsubstate}}
\makeatother
\newenvironment{algsubstates}
  {\setcounter{algsubstate}{0}%
   \renewcommand{\State}{%
     \refstepcounter{algsubstate}%
     \Statex {\footnotesize\arabic{algsubstate}:}\space}
    \renewcommand{\Ifsub}{%
     \refstepcounter{algsubstate}%
     \Statex {\footnotesize\arabic{algsubstate}:}\space}
    \renewcommand{\EndIfsub}{%
     \refstepcounter{algsubstate}%
     \Statex {\footnotesize\arabic{algsubstate}:}\space}
     }
  {}
\end{comment}
  

\usepackage{etoolbox}
\makeatletter
\patchcmd{\@makecaption}
  {\scshape}
  {}
  {}
  {}
\makeatletter
\patchcmd{\@makecaption}
  {\\}
  {.\ }
  {}
  {}
\makeatother
\def\tablename{Table}

%.............................................................
%...File per la generazione di simboli matematici in grassetto
%.............................................................

\font\bfmath=cmmib10
\textfont9=\bfmath
\def\bmit{\fam=9}

\mathchardef\Gamma="7100
\mathchardef\Delta="7101
\mathchardef\Theta="7102
\mathchardef\Lambda="7103
\mathchardef\Xi="7104
\mathchardef\Pi="7105
\mathchardef\Sigma="7106
\mathchardef\Upsilon="7107
\mathchardef\Phi="7108
\mathchardef\Psi="7109
\mathchardef\Omega="710A

\mathchardef\alpha="710B
\mathchardef\beta="710C
\mathchardef\gamma="710D
\mathchardef\delta="710E
\mathchardef\epsilon="710F
\mathchardef\zeta="7110
\mathchardef\eta="7111
\mathchardef\theta="7112
\mathchardef\iota="7113
\mathchardef\kappa="7114
\mathchardef\lambda="7115
\mathchardef\mu="7116
\mathchardef\nu="7117
\mathchardef\xi="7118
\mathchardef\pi="7119
\mathchardef\rho="711A
\mathchardef\sigma="711B
\mathchardef\tau="711C
\mathchardef\upsilon="711D
\mathchardef\phi="711E
\mathchardef\chi="711F
\mathchardef\psi="7120
\mathchardef\omega="7121
\mathchardef\epsilon="7122

\mathchardef\varepsilon="7122
\mathchardef\vartheta="7123
\mathchardef\varpi="7124
\mathchardef\varrho="7125
\mathchardef\varsigma="7126
\mathchardef\varphi="7127
\mathchardef\imath="717B
\mathchardef\jmath="717C

\def\zero{{\bf 0}}
\def\Zero{{\mbox{\boldmath $O$}}}

\def\bfa{{\mbox{\boldmath $a$}}}
\def\bfb{{\mbox{\boldmath $b$}}}
\def\bfc{{\mbox{\boldmath $c$}}}
\def\bfd{{\mbox{\boldmath $d$}}}
\def\bfe{{\mbox{\boldmath $e$}}}
\def\bff{{\mbox{\boldmath $f$}}}
\def\bfg{{\mbox{\boldmath $g$}}}
\def\bfh{{\mbox{\boldmath $h$}}}
\def\bfi{{\mbox{\boldmath $i$}}}
\def\bfj{{\mbox{\boldmath $j$}}}
\def\bfk{{\mbox{\boldmath $k$}}}
\def\bfl{{\mbox{\boldmath $l$}}}
\def\bfm{{\mbox{\boldmath $m$}}}
\def\bfn{{\mbox{\boldmath $n$}}}
\def\bfo{{\mbox{\boldmath $o$}}}
\def\bfp{{\mbox{\boldmath $p$}}}
\def\bfq{{\mbox{\boldmath $q$}}}
\def\bfr{{\mbox{\boldmath $r$}}}
\def\bfs{{\mbox{\boldmath $s$}}}
\def\bft{{\mbox{\boldmath $t$}}}
\def\bfu{{\mbox{\boldmath $u$}}}
\def\bfv{{\mbox{\boldmath $v$}}}
\def\bfx{{\mbox{\boldmath $x$}}}
\def\bfy{{\mbox{\boldmath $y$}}}
\def\bfw{{\mbox{\boldmath $w$}}}
\def\bfz{{\mbox{\boldmath $z$}}}

\def\bfA{{\mbox{\boldmath $A$}}}
\def\bfB{{\mbox{\boldmath $B$}}}
\def\bfC{{\mbox{\boldmath $C$}}}
\def\bfD{{\mbox{\boldmath $D$}}}
\def\bfE{{\mbox{\boldmath $E$}}}
\def\bfF{{\mbox{\boldmath $F$}}}
\def\bfG{{\mbox{\boldmath $G$}}}
\def\bfH{{\mbox{\boldmath $H$}}}
\def\bfI{{\mbox{\boldmath $I$}}}
\def\bfJ{{\mbox{\boldmath $J$}}}
\def\bfK{{\mbox{\boldmath $K$}}}
\def\bfL{{\mbox{\boldmath $L$}}}
\def\bfM{{\mbox{\boldmath $M$}}}
\def\bfN{{\mbox{\boldmath $N$}}}
\def\bfO{{\mbox{\boldmath $O$}}}
\def\bfP{{\mbox{\boldmath $P$}}}
\def\bfQ{{\mbox{\boldmath $Q$}}}
\def\bfR{{\mbox{\boldmath $R$}}}
\def\bfS{{\mbox{\boldmath $S$}}}
\def\bfT{{\mbox{\boldmath $T$}}}
\def\bfU{{\mbox{\boldmath $U$}}}
\def\bfX{{\mbox{\boldmath $X$}}}
\def\bfY{{\mbox{\boldmath $Y$}}}
\def\bfW{{\mbox{\boldmath $W$}}}
\def\bfV{{\mbox{\boldmath $V$}}}
\def\bfZ{{\mbox{\boldmath $Z$}}}

\def\bfalpha{{\mbox{\boldmath $\alpha$}}}
\def\bfbeta{{\mbox{\boldmath $\beta$}}}
\def\bfgamma{{\mbox{\boldmath $\gamma$}}}
\def\bfdelta{{\mbox{\boldmath $\delta$}}}
\def\bfepsilon{{\mbox{\boldmath $\epsilon$}}}
\def\bfzeta{{\mbox{\boldmath $\zeta$}}}
\def\bfeta{{\mbox{\boldmath $\eta$}}}
\def\bftheta{{\mbox{\boldmath $\theta$}}}
\def\bfiota{{\mbox{\boldmath $\iota$}}}
\def\bfkappa{{\mbox{\boldmath $\kappa$}}}
\def\bflambda{{\mbox{\boldmath $\lambda$}}}
\def\bfmu{{\mbox{\boldmath $\mu$}}}
\def\bfnu{{\mbox{\boldmath $\nu$}}}
\def\bfxi{{\mbox{\boldmath $\xi$}}}
\def\bfpi{{\mbox{\boldmath $\pi$}}}
\def\bfrho{{\mbox{\boldmath $\rho$}}}
\def\bfsigma{{\mbox{\boldmath $\sigma$}}}
\def\bftau{{\mbox{\boldmath $\tau$}}}
\def\bfupsilon{{\mbox{\boldmath $\upsilon$}}}
\def\bfphi{{\mbox{\boldmath $\phi$}}}
\def\bfchi{{\mbox{\boldmath $\chi$}}}
\def\bfpsi{{\mbox{\boldmath $\psi$}}}
\def\bfomega{{\mbox{\boldmath $\omega$}}}

\def\bfvarepsilon{{\mbox{\boldmath $\varepsilon$}}}
\def\bfvartheta{{\mbox{\boldmath $\vartheta$}}}
\def\bfvarpi{{\mbox{\boldmath $\varpi$}}}
\def\bfvarrho{{\mbox{\boldmath $\varrho$}}}
\def\bfvarsigma{{\mbox{\boldmath $\varsigma$}}}
\def\bfvarphi{{\mbox{\boldmath $\varphi$}}}
\def\bfimath{{\mbox{\boldmath $\imath$}}}
\def\vfjmath{{\mbox{\boldmath $\jmath$}}}

\def\bfGamma{{\mbox{\boldmath $\Gamma$}}}
\def\bfDelta{{\mbox{\boldmath $\Delta$}}}
\def\bfTheta{{\mbox{\boldmath $\Theta$}}}
\def\bfLambda{{\mbox{\boldmath $\Lambda$}}}
\def\bfXi{{\mbox{\boldmath $\Xi$}}}
\def\bfPi{{\mbox{\boldmath $\Pi$}}}
\def\bfSigma{{\mbox{\boldmath $\Sigma$}}}
\def\bfUpsilon{{\mbox{\boldmath $\Upsilon$}}}
\def\bfPhi{{\mbox{\boldmath $\Phi$}}}
\def\bfPsi{{\mbox{\boldmath $\Psi$}}}
\def\bfOmega{{\mbox{\boldmath $\Omega$}}}

%%%%%%%%%%%%%%%%%%%%%%%%%%%%%%%%%%%%%%%%%%%%%%%%%%%%%%%%%%%%
%                 TEX MACRO FILE, F. BONNANS
%                       APRIL 27, 2003
%%%%%%%%%%%%%%%%%%%%%%%%%%%%%%%%%%%%%%%%%%%%%%%%%%%%%%%%%%%%

\def\dd{{\rm d}}
\newcommand{\ddt}{\frac{\rm d}{{\rm d} t} }
\newcommand{\dotex}[1]{{\frac{\displaystyle d #1}{\displaystyle dt}}}
\newcommand{\ideal}[1]{\langle #1 \rangle}

%%%%%%%%%%%%%%%%%%%%%%%%%%%%%%%%%%%%%%% EMPHAZISE, ETC 
\newcommand{\new}[1]{{\em #1}}
\newcommand{\newd}[2]{{\em #1}\index{#2}}
\newcommand{\prob}[1]{{\em #1}\index{#1}}
\newcommand{\mrm}[1]{\text{\rm #1}}
%%%%%%%%%%%%%%%%%%%%%%%%%%%%%%%%%%%%% GRAPHES 
% \newcommand{\arc}[2]{(#1,#2)}
\def\arcset{{\mathcal A}}
\def\nodeset{{\mathcal N}}
\def\incid{{M}}
\newcommand{\edge}[2]{\{#1,#2\}}
%%%%%%%%%%%%%%%%%%%%%%%%%%%%%%%%%%%%% DIVERS S. Gaubert

\newcommand{\ov}[1]{\overline{#1}}
\newcommand{\arcm}[2]{#1#2}

\newcommand{\transp}{^\top}
\newcommand{\wglobal}[1]{\typeout{warning: #1 global definition}#1}
\newcommand{\promis}{\typeout{warning: verifier que c est fait}}
\newcommand{\ext}{\operatorname{extr}}
\newcommand{\co}{\operatorname{co}}
\newcommand{\cob}{\operatorname{\overline{co}}}
\newcommand{\set}[2]{\{#1\mid\,#2\}}

\newenvironment{sgdelicat}{\small}{}
\newenvironment{sgremarque}{\begin{remark}}{\hfill $\bullet$\end{remark}}

\def\weight(#1,#2){c_{#1,#2}}
%%%%%%%%%%%%%%%%%%%%%%%%%%%%%%%%%%%%% DIVERS RO
\def\Kh{\hat{K}} 
\def\Eb{\bar{E}}
\def\Adj{{\rm Adj}}
\def\Adjc{{\rm Adjc}}
\def\deg{{\rm deg}}
\def\red{{\rm red}}
\def\ra{{\rm ra}}
\def\ovdeg{{\overline{\rm deg}}}
\def\degu{\underline{\rm deg}}
\def\Mh {{\hat M}}
%%%%%%%%%%%%%%%%%%%%%%%%%%% Prog Dyn
\def\qbo {{\bf q}}
\def\ubo {{\bf u}}
\def\ybo {{\bf y}}
\def\uv {{\underline v}}

%%%%%%%%%%%%%%%%%%%%%% FRENCH %%%%%%%%%%%%%%%%%%%%%%%%%%%%%%
%%%%%%%%%%%%%%   FRANCAIS / ANGLAIS
%%% VERSION FRANCAISE
\newcommand{\fa}[2]{#2}
% TEXTE
\newcommand{\Alors}{Alors }
\newcommand{\alors}{alors }
\newcommand{\alorsv}{alors, }
\newcommand{\avec}{avec }
\newcommand{\dans}{dans }
\newcommand{\debdem}{{\noindent \bf D\'emonstration.}\quad}
\newcommand{\et}{et }
\newcommand{\ett}{et }
\newcommand{\est}{est }
\newcommand{\ilexiste}{il existe }
\newcommand{\mesurable}{mesurable}
\newcommand{\ou}{ou }
\newcommand{\pour}{pour }
\newcommand{\pourtout}{pour tout }
\newcommand{\pp}{p.p. }
\newcommand{\quand}{quand }
\newcommand{\Si}{Si }
\newcommand{\si}{si }
\newcommand{\sinon}{sinon }
\newcommand{\Soient}{Soient }
\newcommand{\Soit}{Soit }
\newcommand{\soit}{soit }
\newcommand{\sur}{sur }
\newcommand{\telque}{tel que }
\newcommand{\telleque}{telle que }
\newcommand{\thm}{th\'eor\`eme }
\newcommand{\un}{un }
% MATHS
\newcommand\meas{\mathop{\rm mes}}
%\newcommand\rank{\mathop{\rm rang}}
%
%%% VERSION ANGLAISE
 \if {
  \renewcommand{\fa}[2]{#1}
% TEXTE
  \renewcommand{\Alors}{Then }
  \renewcommand{\alors}{then }
  \renewcommand{\alorsv}{then, }
  \renewcommand{\avec}{with }
  \renewcommand{\dans}{in }
  \renewcommand{\debdem}{\noindent {\bf Proof.}\quad}
  \renewcommand{\et}{and }
  \renewcommand{\ett}{and }
  \renewcommand{\est}{is }
  \renewcommand{\ilexiste}{there exists }
  \renewcommand{\mesurable}{measurable}
  \renewcommand{\ou}{or }
  \renewcommand{\pour}{for }
  \renewcommand{\pourtout}{for all }
  \renewcommand{\pp}{a.e. }
  \renewcommand{\quand}{when }
  \renewcommand{\Si}{If }
  \renewcommand{\si}{if }
  \renewcommand{\sinon}{otherwise }
  \renewcommand{\Soient}{Let }
  \renewcommand{\Soit}{Let }
  \renewcommand{\soit}{let }
  \renewcommand{\sur}{on }
  \renewcommand{\telque}{such that }
  \renewcommand{\telleque}{such that }
  \renewcommand{\thm}{theorem }
  \renewcommand{\un}{a }
% MATHS
  \renewcommand\meas{\mathop{\rm meas}}  
  \renewcommand\rank{\mathop{\rm rank}}
 } \fi
%%%%%%%%%%%%%%%%%%%%%%%%%%%%%%%%%%%%%%%%%%%%%%%%%%%%%%%%%%

\if { 
\renewcommand{\chapterl}[1]{\chapter{\LARGE \bf {#1}}\vspace{-40mm}}
\renewcommand{\sectionl}[1]{\section{\LARGE \bf {#1}}}
\renewcommand{\subsectionl}[1]{\subsection{\LARGE \bf {#1}}}
\renewcommand{\subsubsectionl}[1]{\subsubsection{\Large \bf {#1}}}
\renewcommand{\paragraphl}[1]{\paragraph{\LARGE \bf {#1}}}
\renewcommand{\subparagraphl}[1]{\subparagraph{\LARGE \bf {#1}}}

\renewcommand{\captionl}[1]{\caption{\LARGE \bf {#1}}}

\renewcommand{\Largel}{\Large}}{{\caption}}
} \fi

%%%%%%%%%%%%%%%%%%%%%%%%%%%%%%%%%%%%%%%%%%%%%%%%%%%%%%%%%%%%
\newcommand\findem{\hfill{$\blacksquare$}\medskip}
%%%%%%%%%%%%%%%%%%%%%%%%%%%%%%%%%%%%%%%%%%%%%%%%%%%%%%%%%%%%


%%%%%%%%%%%%%%%%%%%%%%%%%% PROOFS %%%%%%%%%%%%%%%%%%%%%%%%%%%%
% \newcommand{\debdem}{\begin{proof}} \def\findem{\end{proof}}

%%% \newenvironment{proof}{{\noindent \bf D\'emonstration.}\quad}{\hfill{$\blacksquare$}\medskip}
\newenvironment{proofdu}[1]{{\noindent \bf D\'emonstration #1.}\quad}{\hfill{$\blacksquare$}\medskip}

%%% \newcommand{\qed}{\findem}
%%%%%%%%%%%%%%%%%%%%%%%%%%%%%%%%%%%%%%%%%%%%%%%%%%%%%%%%%%%%

%%%%%%%%%%%%%%%%%%%%%%%%%%%%%%%% HAT %%%%%%%%%%%%%%%%%%%%%%%%
\def\ah{\hat{a}}
\def\bh{\hat{b}}
\def\ch{\hat{c}}
\def\dh{\hat{d}}
\def\eh{\hat{e}}
\def\fh{\hat{f}}
\def\gh{\hat{g}}
\def\hh{\hat{h}}
\def\ih{\hat{i}}
\def\jh{\hat{j}}
\def\kh{\hat{k}}
\def\lh{\hat{l}}
\def\mh{\hat{m}}
\def\nh{\hat{n}}
\def\oh{\hat{o}}
\def\ph{\hat{p}}
\def\qh{\hat{q}}
\def\rh{\hat{r}}
\def\sh{\hat{s}}
\def\th{\hat{t}}
\def\uh{\hat{u}}
\def\vh{\hat{v}}
\def\wh{\hat{w}}
\def\xh{\hat{x}}
\def\yh{\hat{y}}
\def\zh{\hat{z}}
%%%%%%%%%%%%%%%%%%%%%%%%%%%%%%%% BAR %%%%%%%%%%%%%%%%%%%%%%%%
\def\ab{\bar{a}}
\def\bb{\bar{b}}
\def\cb{\bar{c}}
\def\db{\bar{d}}
\def\eb{\bar{e}}
\def\fb{\bar{f}}
\def\gb{\bar{g}}
\def\hb{\bar{h}}
\def\ib{\bar{i}}
\def\jb{\bar{j}}
\def\kb{\bar{k}}
\def\lb{\bar{l}}
\def\mb{\bar{m}}
\def\nb{\bar{n}}
\def\ob{\bar{o}}
\def\pb{\bar{p}}
\def\qb{\bar{q}}
\def\rb{\bar{r}}
\def\sb{\bar{s}}
\def\tb{\bar{t}}
\def\ub{\bar{u}}
\def\vb{\bar{v}}
\def\wb{\bar{w}}
\def\xb{\bar{x}}
\def\yb{\bar{y}}
\def\zb{\bar{z}} 

%%%%%%%%%%%%%%%%%%%%%%%%%%%%%%%% UPPERCASE BAR %%%%%%%%%%%%%%
\def\Ab{\bar{A}}
\def\Bb{\bar{B}}
\def\Cb{\bar{C}}
\def\Db{\bar{D}}
\def\Eb{\bar{E}}
\def\Fb{\bar{F}}
\def\Gb{\bar{G}}
\def\Hb{\bar{H}}
\def\Ib{\bar{I}}
\def\Jb{\bar{J}}
\def\Kb{\bar{K}}
\def\Lb{\bar{L}}
\def\Mb{\bar{M}}
\def\Nb{\bar{N}}
\def\Ob{\bar{O}}
\def\Pb{\bar{P}}
\def\Qb{\bar{Q}}
\def\Rb{\bar{R}}
\def\Sb{\bar{S}}
\def\Tb{\bar{T}}
\def\Ub{\bar{U}}
\def\Vb{\bar{V}}
\def\Wb{\bar{W}}
\def\Xb{\bar{X}}
\def\Yb{\bar{Y}}
\def\Zb{\bar{Z}}

\def\Cbar{\bar{C}}
\def\Tbar{\bar{T}}
\def\Zbar{\bar{Z}}
%%%%%%%%%%%%%%%%%%%%%%%%%%%%%%%% TILDE %%%%%%%%%%%%%%%%%%%%%%%%

\def\at{\tilde{a}}
\def\bt{\tilde{b}}
\def\ct{\tilde{c}}
\def\dt{\tilde{d}}
\def\Dt{\Delta t}
\def\Dx{\Delta x}
\def\et{\tilde{e}}
\def\ft{\tilde{f}}
\def\gt{\tilde{g}}
\def\hit{\tilde{h}}
\def\iit{\tilde{i}}
\def\jt{\tilde{j}}
\def\kt{\tilde{k}}
\def\lt{\tilde{l}}
\def\mt{\tilde{m}}
\def\nt{\tilde{n}}
\def\ot{\tilde{o}}
\def\pt{\tilde{p}}
\def\qt{\tilde{q}}
\def\rt{\tilde{r}}
\def\st{\tilde{s}}
\def\tt{\tilde{t}}
\def\ut{\tilde{u}}
\def\vt{\tilde{v}}
\def\wt{\tilde{w}}
\def\xt{\tilde{x}}
\def\yt{\tilde{y}}
\def\zt{\tilde{z}}
%%%%%%%%%%%%%%%%%%%%%%%%%%%%%%%% UPPERCASE TILDE %%%%%%%%%%%%%%%%
\def\tA{\tilde{A}} 
\def\tL{\tilde{L}} 
\def\tS{\tilde{S}} 
\def\tV{\tilde{V}} 
\def\tW{\tilde{W}} 
\def\tX{\tilde{X}} 
\def\tY{\tilde{Y}} 
\def\tZ{\tilde{Z}} 
%%%%%%%%%%%%%%%%%%%%%%%%%%%%%%%% BOLDFACE %%%%%%%%%%%%%%%%%%%%%%%
\def\ubf { {\bf u}}  
\def\vbf { {\bf v}}  
\def\wbf { {\bf w}}  
\def\xbf { {\bf x}}  
\def\ybf { {\bf y}}  
%%%%%%%%%%%%%%%%%%%%%%%%%%%%%%%% CAL %%%%%%%%%%%%%%%%%%%%%%%%%%%%
\def\cala{{\cal A}}
\def\calb{{\cal B}}
\def\calc{{\cal C}}
\def\cald{{\cal D}}
\def\cale{{\cal E}}
\def\calf{{\cal F}}
\def\calg{{\cal G}}
\def\calh{{\cal H}}
\def\cali{{\cal I}}
\def\calj{{\cal J}}
\def\calk{{\cal K}}
\def\call{{\cal L}}
\def\calm{{\cal M}}
\def\caln{{\cal N}}
\def\calo{{\cal O}}
\def\calp{{\cal P}}
\def\calq{{\cal Q}}
\def\calr{{\cal R}}
\def\cals{{\cal S}}
\def\calt{{\cal T}}
\def\calu{{\cal U}}
\def\calv{{\cal V}}
\def\calw{{\cal W}}
\def\calx{{\cal X}}
\def\caly{{\cal Y}}
\def\calz{{\cal Z}}
%%%%%%%%%%%%%%%%%%%%%%%%%%%% OTHER CAL %%%%%%%%%%%%%%%%%%%%%%%%%%
\def\cL{{\cal L}}
\def\cP{{\cal P}}

\def\cPt{\widetilde{\cal P}}
%%%%%%%%%%%%%%%%%%%%%%%%%%%%%%%% MATHCAL %%%%%%%%%%%%%%%%%%%%%%%%
\def\A{\mathcal{A}}
\def\B{\mathcal{B}}
\def\C{\mathcal{C}}
\def\E{\mathcal{E}}
\def\F{\mathcal{F}}
\def\G{\mathcal{G}}
\def\H{\mathcal{H}}
\def\I{\mathcal{I}}
\def\J{\mathcal{J}}
\def\K{\mathcal{K}}
\def\L{\mathcal{L}}
\def\M{\mathcal{M}}
\def\N{\mathcal{N}}
\def\P{\mathcal{P}}
\def\Q{\mathcal{Q}}
\def\R{\mathcal{R}}
\def\SS{\mathcal{S}}
\def\T{\mathcal{T}}
\def\U{\mathcal{U}}
\def\V{\mathcal{V}}
\def\W{\mathcal{W}}
\def\X{\mathcal{X}}
\def\Y{\mathcal{Y}}
\def\Z{\mathcal{Z}}
%%%%%%%%%%%%%%%%%%%%%%%%%%%% AUTRES MATHCAL %%%%%%%%%%%%%%%%%%
\newcommand{\sB}{\mathcal{B}}
\newcommand{\sC}{\mathcal{C}}
\newcommand{\sE}{\mathcal{E}}
\newcommand{\sL}{\mathcal{L}}
\newcommand{\sS}{\mathcal{S}}
\newcommand{\sT}{\mathcal{T}}
\newcommand{\sV}{\mathcal{V}}
%%%%%%%%%%%%%%%%%%%%%%% SPECIAL   %%%%%%%%%%%%%%%%%%%%%%%%
% NOTATION HAMILTONIEN
\def\HH{\overline{\bf H}}
%%%%%%%%%%%%%%%%%%%%%%% UNDERLINE %%%%%%%%%%%%%%%%%%%%%%%%
\def\xu{\underline{x}}
%
\def\uV{\underline{V}}

%%%%%%%%%%%%%%%%%%%%%%% GREEKS %%%%%%%%%%%%%%%%%%%%%%%%
\def\eps{\varepsilon}
\def\Om{{\Omega}}
\def\om{{\omega}}
\def\ml{\lambda}
\def\mlb{\bar{\lambda}}
\def\mub{\bar{\mu}}
\def\nub{\bar{\nu}}
\def\taub{{\bar\tau}}
%%%%%%%%%%%%%%%%%%%%%%%% MATH OPERATORS %%%%%%%%%%%%%%%%
\newcommand{\ceil}[1]{\lceil #1 \rceil}
\newcommand{\floor}[1]{\lfloor #1 \rfloor}

\def\ad{\mathop{\rm ad}}
\def\affhull{\mathop{\rm affhull}}
\def\argmin{\mathop{\rm argmin}}
\def\argmax{\mathop{\rm argmax}}
\def\cl{\mathop{\rm cl}}
\def\cone{\mathop{\rm cone}}
\def\conv{\mathop{\rm conv}}
\def\convbar{\mathop{\rm \overline{conv}}}
\def\deg{\mathop{\rm deg}}
\def\det{\mathop{\rm det}}
\def\diag{{\mathop{\rm diag}}}
\def\dist{\mathop{\rm dist}}
\def\ddiv{\mathop{\rm div}}
\def\dom{\mathop{{\rm dom}}}
\def\epi{\mathop{\text{\'epi}}}
\def\intt{\mathop{\rm int}}
\def\inv{\mathop{\rm inv}}
\def\lin{\mathop{\rm lin}}
\def\isom{\mathop{\rm Isom}}
\def\range{\mathop{\rm Im}}
\def\sign{\mathop{\rm sign}}
\def\supp{\mathop{\rm supp}}
\def\trace{\mathop{\rm trace}}
\def\val{\mathop{\rm val}}
\def\var{\mathop{\rm var}}

\def\Inf{\mathop{\rm Inf}}
\def\Ker{\mathop{\rm Ker}}
\def\Min{\mathop{\rm Min}}
\def\Max{\mathop{\rm Max}}

\def\dw{\mathop{\delta w}}

%%%%%%%%%%%%%%%% UNDERLINED AND OVERLINED LIMINF AND LIMSUP %%%
\def\liminfu{\mathop{\underline{\lim}}}
\def\limsupo{\mathop{\overline{\lim}}}
%%%%%%%%%%%%%%%%%%%%%%%%%%%%%%%% NUMBERS %%%%%%%%%%%%%%%%%%
\def\half{\mbox{$\frac{1}{2}$}}
\def\sixinv{\mbox{$\frac{1}{6}$}}
\def\1B{{\bf  1}}
\def\sbdeux#1#2{\mbox{\scriptsize$#1$}\atop\mbox{\scriptsize$#2$}}
\def\sbtrois#1#2#3{\mbox{\scriptsize$#1$}\atop\mbox{\scriptsize$#2$}\atop\mbox{\scriptsize$#3$}}
%%%%%%%%%%%%%%%%%%%%%%%%%%%%%%%%%  NUMBER SPACES  %%%%
\newcommand{\NN}{\mathbb{N}}
\newcommand{\ZZ}{\mathbb{Z}}
\newcommand{\WW}{\mathbb{W}}
\newcommand{\QQ}{\mathbb{Q}}
\newcommand{\OO}{\mathcal{O}}
\newcommand{\RR}{\mathbb{R}}
\newcommand{\TR}{ \mbox{Tr}}
\newcommand{\gA}{\mathsf{A}}
\newcommand{\gG}{\mathsf{G}}
\newcommand{\gN}{\mathsf{N}}
\newcommand{\gT}{\mathsf{T}}
\newcommand{\gV}{\mathsf{V}}
\newcommand{\gE}{\mathsf{E}}
\newcommand{\sG}{\mathcal{G}}
\newcommand{\sN}{\mathcal{N}}
\newcommand{\sA}{\mathcal{A}}

% \def\cR{I\hspace{-1mm}R}

\def\cC{\mathbb{C}}
\def\cN{\mathbb{N}}
\def\II{\mathbb{I}}
\def\EE{\mathbb{E}}
\def\PP{\mathbb{P}}
\def\cQ{\mathbb{Q}}
\def\cR{\mathbb{R}}
\def\cZ{\mathbb{Z}}

\newcommand{\cE}{I\!\!E}
% \newcommand{\cR}{I\!\! R} \newcommand{\cN}{I\!\!N}
\newcommand{\cRbar}{\bar {I\!\! R}}
\newcommand{\rbar}{\overline{\mathbb{R}}}
\newcommand{\rbartosn}{\rbar^{_{\scriptstyle\sN}}}
%%%%%%%%%%%%%%%%%%%%%%%%%%%%%%%%  ENVIRONMENTS %%%%%%%%%%%%%%%%%%
\newcommand\be{\begin{equation}}
\newcommand\ee{\end{equation}}
\newcommand\ba{\begin{array}}
\newcommand\ea{\end{array}}
\newcommand{\bean}{\begin{eqnarray*}}
\newcommand{\eean}{\end{eqnarray*}}

\newenvironment{paoplist}{\vspace{-1ex}\begin{list}{-}
{\itemsep 0mm \leftmargin 2mm \labelwidth 0mm}}
{\end{list} \vspace{-1ex}}

\newenvironment{myenumerate}{
\renewcommand{\theenumi}{\roman{enumi}}
\renewcommand{\labelenumi}{(\theenumi)}
\begin{enumerate}}{\end{enumerate}}
%%%%%%%%%%%%%%%%%%%%%%%%%%%%%%%%  SPACING %%%%%%%%%%%%%%%%%%
\newcommand{\noi}{\noindent}
\newcommand{\bs}{\bigskip}
\newcommand{\ms}{\medskip}
\newcommand{\msn}{{\bigskip \noindent}}
%%%%%%%%%%%%%%%%%%%%%%%%%%%%%%%%  MISC %%%%%%%%%%%%%%%%%%
\newcommand{\refeq}[1]{(\ref{#1})}
%\newcommand{\eqref}[1]{(\ref{#1})}

\def\ATT{\marginpar{$\leftarrow$}}

\def\rar{\rightarrow}
\def\fmap{\rightarrow}

\def\ds{\displaystyle}
\def\disp{\displaystyle}

\def\La{\langle}
\def\la{\langle}
\def\ra{\rangle}
%%%%%%%%%%%%%%%%%%%%%%%%%% FIGURES %%%%%%%%%%%%%%%%%%%%%%%%%%%%%%%%%%

\newcommand{\mypsfig}[3]
           {\begin{figure}[hbtp]
            \centerline{\input #1}
            \caption{\rm{#2}} \label{#3}
            \end{figure}}








%%% EOF



\usepackage[style=ieee,doi=false,mincitenames=1,maxcitenames=20,minbibnames=1,maxbibnames=20,isbn=false,url=true]{biblatex}
\addbibresource{bibliography.bib}


\makeatother
\usepackage[belowskip=-7pt,aboveskip=10pt,font={small}]{caption}
%\usepackage[font={small}]{caption}
\usepackage[font={small}]{subcaption}
\urlstyle{same}
\makeatletter


%% end biblio font
\begin{document}


\title{ 
Ensemble Latent Space Roadmap for Improved Robustness \\in Visual Action Planning

}

\author{Martina Lippi*$^{1}$, Michael C. Welle*$^{2}$, Andrea Gasparri$^{1}$, Danica Kragic$^{2}$% <-this % stops a space
\thanks{*These authors contributed equally (listed in alphabetical order).}% <-this % stops a space
\thanks{ ${}^1$Roma Tre University, Rome, Italy  {\it\small \{martina.lippi,andrea.gasparri\}@uniroma3.it} }%
\thanks{ ${}^2$KTH Royal Institute of Technology Stockholm, Sweden, {\it\small \{mwelle,dani\}@kth.se}}%
%\thanks{ ${}^3$University of Cassino and Southern Lazio, Cassino, Italy {\tt\small al.marino@unicas.it}}%
% \thanks{This work was supported by the Swedish Research Council, Knut and Alice Wallenberg Foundationm, by the European Research Council (ERC-884807), and by the European Commission (Project CANOPIES-101016906).}
}




\maketitle

\begin{abstract}


 Planning in learned latent spaces helps to decrease the dimensionality of raw observations.
In this work, we propose to leverage the ensemble paradigm to 
enhance the robustness of latent planning systems. We rely on our Latent Space Roadmap (LSR) framework, which builds a graph in a learned structured latent space to perform planning.  
Given multiple LSR  framework instances, that differ either on their latent spaces or on the parameters for constructing the graph, we use the action information as well as the embedded nodes of the produced plans to define similarity measures. These are then utilized to select the most promising plans. We validate the performance of our Ensemble LSR (ENS-LSR) on simulated box stacking and grape harvesting tasks as well as on a real-world robotic T-shirt folding experiment. 
\end{abstract}


\section{Introduction}
\label{sec:intro}

Generating plans from raw observations can be more flexible than relying on handcrafted state extractors or modeling the system explicitly.
This can be particularly advantageous in complex tasks, such as manipulating deformable objects or operating in dynamic environments, where accurately defining the underlying system state can prove to be challenging.
Moreover, the generation of plans from raw observations offers the potential to create both visual and action plans, which we refer to as visual action plans. The visual plan captures the sequence of images that the system transitions through while executing the action plan.  This  allows human operators to easily comprehend the  plan and gain a better understanding of the system behavior.



Nonetheless, 
the high-dimensionality of raw observations 
challenges the effectiveness of classical planning approaches~\cite{bellman1961curse}. In this regard, representation learning  may help to retrieve compact representations from high-dimensional observations. Given these representations, classical planning approaches can be then employed. Based on this principle, we proposed in our previous works \cite{lippi2022enabling, lippi2022augment} a Latent Space Roadmap (LSR) framework.  
This involves training a deep neural network to learn a low-dimensional latent space of the high-dimensional observation space. 
A roadmap is then constructed within this space to capture the connectivity between different states, and is used to perform planning.


\begin{figure}[t!]
    \centering
    \includegraphics[width=\linewidth]{figures/ens_fig1.png}
    \vspace{-15pt}
    \caption{Depiction of the Ensemble Latent Space Roadmap framework for a folding task. Given start and goal observations, multiple models produce different visual action plans, and the ones with the highest degree of similarity to one another are selected. 
}
    \label{fig:en_overview}
\end{figure}

 While utilizing learned models for representation is generally advantageous, they might lead to unreliable plans. For instance, this can occur when working with non-representative data samples during the training process.  
In this work, we aim to enhance the robustness of latent space planning by  incorporating 
the  ensemble paradigm \cite{rokach2019ensemble}: 
several models are collected and their plans are combined based on similarity measures. 
 To this aim, we 
 take into account both the sequence of actions and the composition of transited nodes in the latent space.
% 
 In line with majority voting ensemble approaches \cite{ruta2005classifier}, we select the plans which are the most similar to the others.
We name the resulting framework Ensemble LSR (ENS-LSR) and show an overview in Fig.~\ref{fig:en_overview}.
% 
We demonstrate its effectiveness  compared to \cite{lippi2022enabling, lippi2022augment} in simulated box stacking and grape harvesting tasks and on a real-world folding task. 
% 
In detail, our contributions are:
\begin{itemize}
    \item The design of a novel ensemble algorithm for latent space planning along with the definition of appropriate similarity measures. 
    \item An extensive performance comparison on multiple tasks, including a new simulated agricultural task and a real-world manipulation task with deformable objects.
    \item Extensive ablation studies analyzing different components of the algorithm. 
    \item The release of all datasets, code, and execution videos for real-world experiments on the project website\footnote{\label{fn:website} \url{https://visual-action-planning.github.io/ens-lsr/}}.
\end{itemize}
Note that, although we focus on LSR framework, the proposed algorithm can be  adapted to any latent space planning method. 




\section{Related Work} \label{sec:rw}

\subsection{Visual planning}
Visual planning  allows robots to generate plans and make decisions based on the visual information they perceive from their environment. 
Methods directly working in the image space have been proposed in the literature. For instance, the method in \cite{finn2017deep} is based on Long-Short Term Memory blocks to generate a video prediction model, which is  then integrated into a Model Predictive Control (MPC) framework to produce  visual plans; the approach in \cite{wang2019learning} produces visual foresight plans for rope manipulation using GAN models and then resorts to a learned rope inverse dynamics. 
However, as mentioned in the Introduction, latent spaces can be employed to reduce the high-dimensionality of the image space. For instance, an RRT-based algorithm in the latent space with collision checking  is adopted in \cite{Ichter2019}; interaction features conditioned on object images are learned and integrated within Logic-Geometric Programming for planning in \cite{Toussaint_ral2022};  model-free Reinforcement Learning (RL) in the off-policy case combined with auto-encoders is investigated in~\cite{yarats2021improving}. 
Despite the dimensionality reduction of the space, 
the approaches above typically require a large amount of data to operate. In contrast, the LSR framework is able to accomplish visual planning in a data-efficient manner by leveraging a contrastive loss to structure the latent space.   

\subsection{Ensemble methods}
While the ensemble principle has primarily been utilized within the machine learning community \cite{dong2020survey}, it has also been implemented in robotic planning to combine multiple planning algorithms or models and improve the overall performance of the system.
Multiple planners in parallel,  which work under diverse assumptions, are executed according to the method in  \cite{Scherer_ICRA2015}. An ensemble selection is then made based on learned priors on planning performance. The combination of multiple heuristics in greedy best-first search planning algorithm is investigated in \cite{Helmert_2021}. 
An online ensemble learning method based on different predictors is proposed instead in \cite{Zambelli_TCDS2017} to achieve an accurate robot forward model that can be beneficial for imitation behavior.   
The study in \cite{Adil_Access2020} adopts 
an ensemble framework with neural networks to mitigate the error in stereo vision systems and then performs planning for manipulation. 
 Finally, several applications of the ensemble paradigm to RL approaches can be found in the literature, such as in 
\cite{Wiering_TSMC2008,buckman2018sample,lee2021sunrise}. In detail, the ensemble method in \cite{Wiering_TSMC2008} combines the value functions of five different RL algorithms, the approach in \cite{buckman2018sample}  dynamically interpolates between rollouts with different horizon lengths, while the study in \cite{lee2021sunrise} proposes ensemble-based weighted Bellman backups. 
However, to the best of our knowledge, none of the above approaches is able to realize visual action planning. 









\section{The Semi-Oblivious Chase Procedure}\label{sec:semi}
%

The semi-oblivious chase (or simply chase) takes as input a database $D$ and a set $\dep$ of TGDs, and constructs an instance that contains $D$ and satisfies $\dep$.
%
A central notion in this context is that of trigger.
%are those of trigger, active trigger, and trigger application.

\begin{definition}%[\textbf{Trigger Application}]
	Given a set $\dep$ of TGDs and an instance $I$, a {\em trigger} for $\dep$  on $I$ is a pair $(\sigma,h)$, where $\sigma \in \dep$ and $h$ is a homomorphism from $\body{\sigma}$ to $I$.
	%
	The {\em result} of $(\sigma,h)$, denoted $\result{\sigma}{h}$, is the set $\mu(\head{\sigma})$, where $\mu : \var{\head{\sigma}} \ra \ins{C} \cup \ins{N}$ is defined as follows:
	%
	%$\mu(x) = h(x)$ if $x \in \fr{\sigma}$, and $\mu(x) = \bot_{\sigma,h_{|\fr{\sigma}}}^{x}$ otherwise,
	\[
	\mu(x)\
	=\ \left\{
	\begin{array}{ll}
	h(x) & \quad \text{if } x \in \fr{\sigma}\\
	&\\
	\bot_{\sigma,h_{|\fr{\sigma}}}^{x} & \quad \text{otherwise}
	\end{array} \right.
	\]
	where $\bot_{\sigma,h_{|\fr{\sigma}}}^{x} \in \ins{N}$.  Let $T(\dep,I)$ be the set of triggers for $\dep$ on $I$.	\hfill\markfull
\end{definition}




Observe that in the definition of $\result{\sigma}{h}$, each existentially quantified variable $x$ of $\head{\sigma}$ is mapped by $\mu$ to a null value of $\ins{N}$ whose name is uniquely determined by the trigger $(\sigma,h)$ and the variable $x$ itself. This means that, given a trigger $(\sigma,h)$, we can unambiguously construct the set of atoms $\result{\sigma}{h}$.
%
The central idea of the chase is, starting from a database $D$, to exhaustively apply triggers for the given set $\dep$ of TGDs on the instance constructed so far.
%
More precisely, given a database $D$ and a set $\dep$ of TGDs, let
\[
\mathsf{chase}^{0}(D,\dep)\ =\ D,
\]
and for each $i>0$, let
\[
\mathsf{chase}^{i}(D,\dep)\ =\ \mathsf{chase}^{i-1}(D,\dep)\ \cup\ \bigcup_{(\sigma,h) \in S} \result{\sigma}{h},
\]
where $S = T(\dep,\mathsf{chase}^{i-1}(D,\dep))$. 
%
We finally define {\em the result of the chase of $D$ w.r.t.~$\dep$} as the (possibly infinite) instance
\[
\chase{D}{\dep}\ =\ \bigcup_{i \geq 0} \mathsf{chase}^{i}(D,\dep).
\]


\ignore{
The semi-oblivious chase procedure (or simply chase) takes as input a database $D$ and a set $\dep$ of TGDs, and constructs an instance that contains $D$ and satisfies $\dep$.
%
Central notions in this context are those of trigger, active trigger, and trigger application.

\begin{definition}%[\textbf{Trigger Application}]
	Given a set $\dep$ of TGDs and an instance $I$, a {\em trigger} for $\dep$  on $I$ is a pair $(\sigma,h)$, where $\sigma \in \dep$ and $h$ is a homomorphism from $\body{\sigma}$ to $I$.
	%
	The {\em result} of $(\sigma,h)$, denoted $\result{\sigma}{h}$, is the set $\mu(\head{\sigma})$, where $\mu : \var{\head{\sigma}} \ra \ins{C} \cup \ins{N}$ is defined as follows:
	%
	%$\mu(x) = h(x)$ if $x \in \fr{\sigma}$, and $\mu(x) = \bot_{\sigma,h_{|\fr{\sigma}}}^{x}$ otherwise,
	\[
	\mu(x)\
	=\ \left\{
	\begin{array}{ll}
	h(x) & \quad \text{if } x \in \fr{\sigma}\\
	&\\
	\bot_{\sigma,h_{|\fr{\sigma}}}^{x} & \quad \text{otherwise}
	\end{array} \right.
	\]
	where $\bot_{\sigma,h_{|\fr{\sigma}}}^{x}$ is a null value from $\ins{N}$.
	%
	The trigger $(\sigma,h)$ is {\em active} if $\result{\sigma}{h} \not\subseteq I$.
	%
	The {\em application} of $(\sigma,h)$ to $I$ returns the instance $J = I \cup \result{\sigma}{h}$ and is denoted as $I \app{\sigma}{h} J$.
	\hfill\markfull
\end{definition}


Observe that in the definition of $\result{\sigma}{h}$ above, each existentially quantified variable $x$ of $\head{\sigma}$ is mapped by $\mu$ to a null value of $\ins{N}$ whose name is uniquely determined by the trigger $(\sigma,h)$ and the variable $x$ itself. This means that, given a trigger $(\sigma,h)$, we can unambiguously extract the set of atoms 
$\result{\sigma}{h}$.



%\medskip

%\noindent
%\textbf{Semi-Oblivious Chase.}
The central idea of the chase is, starting from a database $D$, to exhaustively apply active triggers for the given set $\dep$ of TGDs on the instance constructed so far. This is formalized via the notion of (semi-oblivious) chase derivation, which can be finite or infinite.


\begin{definition}
	Consider a database $D$ and a set $\dep$ of TGDs.
	%We consider the two cases where a derivation is finite or infinite:
	\begin{itemize}
		\item A finite sequence $(I_i)_{0 \leq i \leq n}$ of instances, with $D = I_0$ and $n \geq 0$, is a {\em chase derivation} of $D$ w.r.t.~$\dep$ if, for each $i \in \{0,\ldots,n-1\}$, there is an active trigger $(\sigma,h)$ for $\dep$ on $I_i$ with $I_i \app{\sigma}{h} I_{i+1}$, and there is no active trigger for $\dep$ on $I_n$. The {\em result} of such a chase derivation is the instance $I_n$.
		
		
		\item An infinite sequence $(I_i)_{i \geq 0}$ of instances, with $D = I_0$, is a {\em chase derivation} of $D$ w.r.t.~$\dep$ if, for each $i \geq 0$, there is an active trigger $(\sigma,h)$ for $\dep$ on $I_i$ such that $I_i \app{\sigma}{h} I_{i+1}$. Moreover, $(I_i)_{i \geq 0}$ is {\em fair} if, for each $i \geq 0$, and for every active trigger $(\sigma,h)$ for $\dep$ on $I_i$, there exists $j > i$ such that $(\sigma,h)$ is not an active trigger for $\dep$ on $I_j$. 
		%The latter is known as the {\em fairness condition}, and guarantees that all the active triggers will be deactivated. %
		The {\em result} of such a chase derivation is the instance $\bigcup_{i \geq 0} \, I_i$.
	\end{itemize}
	%
	%The {\em result} of a chase derivation is defined as the union of all the instances occurring in it. 
	A chase derivation is {\em valid} if it is finite or infinite and fair.  \hfill\markfull
\end{definition}


Let us stress that infinite but unfair chase derivations are not considered as valid ones since they do not serve the main purpose of the chase, that is, to build an instance that satisfies the given set of TGDs. Indeed, given the set $\dep$ consisting of the TGDs
\[
\sigma\ =\ R(x,y) \ra \exists z \, R(y,z) \qquad \sigma'\ =\ R(x,y) \ra P(x,y),
\]
the result of the unfair chase derivation of $D = \{R(a,b)\}$ w.r.t.~$\dep$ that involves only triggers of the form $(\sigma,\cdot)$, i.e., only the TGD $\sigma$ is used, does not satisfy $\sigma'$, and thus, it does not satisfy $\dep$.
%
Interestingly, for every database $D$ and set $\dep$ of TGDs, any two valid chase derivations of $D$ w.r.t.~$\dep$ have always the same result, which implies that all valid chase derivations are either finite or infinite~\cite{GrOn18}. Therefore, in the rest of the paper, we can safely refer to {\em the} result of the chase of $D$ w.r.t. $\dep$, which we will denote by $\chase{D}{\dep}$. 
}


%\subsection{Non-Uniform Chase Termination}\label{sec:problem}
%

\medskip

\noindent
\textbf{Chase Termination.}
The result of the chase may be infinite even for very simple settings: it is easy to see that for $D = \{R(a,b)\}$ and $\dep = \{R(x,y) \ra \exists z \, R(y,z)\}$, $\chase{D}{\dep}$ is infinite.
%; in particular, $\chase{D}{\dep} = \{R(a,b),R(b,\bot_1),R(\bot_1,\bot_2),R(\bot_2,\bot_3),\ldots\}$, where $\bot_1,\bot_2,\ldots$ are null values.
%
This leads to the following problem, parameterized by a class $\class{C}$ of TGDs such as $\class{SL}$ (the class of simple-linear TGDs) and $\class{L}$ (the class of linear TGDs):


\medskip

\begin{center}
	\fbox{
		\begin{tabular}{ll}
			%{\small PROBLEM} : & %$\mathsf{ChaseTermination}(\class{C})$
			%\\
			{\small INPUT} : & A database $D$ and a set $\dep$ of TGDs from $\class{C}$.
			\\
			{\small QUESTION} : &  Is the instance $\chase{D}{\dep}$ finite?
	\end{tabular}}
\end{center}

\medskip

\noindent This problem has been recently studied in~\cite{CaGP22} for the classes of simple-linear and linear TGDs. Interestingly, for both classes, the finiteness of the result of the chase has been syntactically characterized by exploiting the notion of non-uniform weak-acyclicity. 
%
We proceed to recall this acyclicity notion, and then present the characterizations established in~\cite{CaGP22}, which in turn lead to simple algorithms for checking the finiteness of the result of the chase.
%
Note that, for the sake of clarity, in the rest of the paper we assume TGDs with a non-empty frontier, i.e., we assume that there is at least one variable in a TGD $\sigma$ that occurs both in $\body{\sigma}$ and $\head{\sigma}$. This assumption can be made without loss of generality since, given a database $D$ and a set $\dep$ of TGDs, we can easily construct a set $\dep'$ of TGDs with a non-empty frontier by slightly modifying $\dep$ such that $\chase{D}{\dep}$ is finite iff $\chase{D}{\dep'}$ is finite.


\medskip

\noindent
\textbf{Non-Uniform Weak-Acyclicity.} Weak-acyclicity was introduced in~\cite{FKMP05} as the main formalism for data exchange purposes, which guarantees the finiteness of the result of the chase for {\em every} input database. Non-uniform weak-acyclicity is the database-dependent variant of weak-acyclicity introduced in~\cite{CaGP22}. We proceed to give the formal definitions.
%
We first need to recall the notion of the {\em dependency graph} of a set $\dep$ of TGDs, 
%which symbolically encodes how terms may propagate during the chase.
%The {\em dependency graph} of set $\dep$ of TGDs 
defined as a directed multigraph $\depg{\dep}=(N,E)$, where $N = \pos{\sch{\dep}}$ and $E$ contains {\em only} the following edges.
%
For each TGD $\sigma \in \dep$ with $\head{\sigma} = \{\alpha_1,\ldots,\alpha_k\}$, for each $x \in \frontier{\sigma}$, and for each position $\pi \in \posvar{\body{\sigma}}{x}$:
\begin{itemize}
	\item For each $i \in [k]$ and for each $\pi' \in \posvar{\alpha_i}{x}$, there exists a \emph{normal} edge $(\pi,\pi') \in E$.
	%
	\item For each existentially quantified variable $z$ in $\sigma$, $i \in [k]$, and $\pi' \in \posvar{\alpha_i}{z}$, there is a \emph{special} edge $(\pi,\pi') \in E$.
\end{itemize}
%
We further need to define when a predicate is reachable from another predicate. 
%
Given predicates $R,P \in \sch{\dep}$, {\em $P$ is reachable from $R$ (w.r.t.~$\dep$)} if $R = P$, or there exists a path in $\depg{\dep}$ from a position of the form $(R,i)$ to a position of the form $(P,j)$.
%
%we write $R \ra_\dep P$  if $R = P$, or there exists a TGD $\sigma \in \dep$ such that $R$ occurs in $\body{\sigma}$ and $P$ occurs in $\head{\sigma}$. We say that {\em $P$ is reachable from $R$ (w.r.t.~$\dep$)}, denoted $R \reach{\dep} P$, if (i) $R \ra_\dep P$, or (ii) there exists $T \in \sch{\dep}$ such that $R \reach{\dep} T$ and $T \ra_\dep P$.
%in $\depg{\dep}$, denoted $R \reach{\dep} P$, if there exists a path in $\depg{\dep}$ from a position $(R,i)$ to a position $(P,j)$, for some $i \in [\arity{R}]$ and $j \in [\arity{P}]$.
Given a database $D$, we say that a (not necessarily simple and possibly cyclic) path $C$ in $\depg{\dep}$ is \emph{$D$-supported} if there exists an atom $R(\bar t) \in D$ and a node of the form $(P,i)$ in $C$ such that $P$ is reachable from $R$.
%
We are now ready to recall (non-uniform) weak-acyclicity.



\begin{definition}\label{def:dwa}
	Consider a database $D$ and a set $\dep$ of TGDs. We say that $\dep$ is {\em weakly-acyclic w.r.t.~$D$}, or {\em $D$-weakly-acyclic}, if there is no $D$-supported cycle in $\depg{\dep}$ with a special edge. 
	%
	We say that $\dep$ is {\em weakly-acyclic} if there is no cycle in $\depg{\dep}$ with a special edge. \hfill\markfull
\end{definition}


\smallskip

\noindent
\textbf{Characterizing the Finiteness of the Chase.}
It is not very difficult to show that whenever a set $\dep$ of TGDs (not necessarily linear) is $D$-weakly-acyclic, then the instance $\chase{D}{\dep}$ is finite. In other words, the $D$-weak-acyclicity of $\dep$ is a sufficient condition for the finiteness of $\chase{D}{\dep}$. What is more interesting is that, assuming that $\dep$ is a set of simple-linear TGDs, the $D$-weak-acyclicity of $\dep$ is also a necessary condition for the finiteness of $\chase{D}{\dep}$. This leads to the following characterization established in~\cite{CaGP22}:

\begin{theorem}\label{the:characterization-simple-linear}
	Consider a database $D$ and a set $\dep \in \class{SL}$ of TGDs. It holds that $\chase{D}{\dep}$ is finite iff $\dep$ is $D$-weakly-acyclic.
\end{theorem}

For linear TGDs, it turned out that non-uniform weak-acyclicity is not powerful enough for characterizing the finiteness of the chase instance. Here is an example given in~\cite{CaGP22} that illustrates this fact:
%This is illustrated by the following example.


\begin{example}
	Consider the database $D = \{R(a,b)\}$ and the singleton set $\dep$ consisting of the (non-simple) linear TGD
	\[
	R(x,x)\ \ra\ \exists z \, R(z,x). 
	\]
	It is easy to see that there is no trigger for $\dep$ on $D$. This means that $\chase{D}{\dep} = D$ is finite, whereas $\dep$ is {\em not} $D$-weakly-acyclic. \hfill\markfull
\end{example}


To obtain a characterization analogous to Theorem~\ref{the:characterization-simple-linear}, the authors of~\cite{CaGP22} used the technique of {\em simplification} to convert linear TGDs into simple-linear TGDs, while preserving the finiteness of the chase instance. We proceed to recall this technique.
%
Let $\bar t = (t_1,\ldots,t_n)$ be a tuple of (not necessarily distinct) terms. We write $\unique{\bar t}$ for the tuple obtained from $\bar t$ by keeping only the first occurrence of each term in $\bar t$.
%
For example, if $\bar t = (x,y,x,z,y)$, then $\unique{\bar t} = (x,y,z)$.
%
For each $i \in [n]$, the \emph{identifier of $t_i$ in $\bar t$}, denoted $\id{\bar t}{t_i}$, is the integer that identifies the position of $\unique{\bar t}$ at which $t_i$ appears. 
%
We write $\id{}{\bar t}$ for the tuple $(\id{\bar t}{t_1},\ldots,\id{\bar t}{t_n})$.
%
For example, if $\bar t = (x,y,x,z,y)$, then $\id{}{\bar t} = (1,2,1,3,2)$.
%
For an atom $\alpha = R(\bar t)$, the {\em simplification of $\alpha$}, denoted $\simple{\alpha}$, is the atom $R_{\id{}{\bar t}}(\unique{\bar t})$, whereas the {\em shape of $\alpha$}, denoted $\shape{\alpha}$, is the predicate $R_{\id{}{\bar t}}$. We can naturally refer to the simplification and the shape of a set of atoms.
%
For a tuple of variables $\bar x = (x_1,\ldots,x_n)$, a \emph{specialization of $\bar x$} is a function $f$ from $\bar x$ to $\bar x$ such that $f(x_1) = x_1$, and $f(x_i) \in \{f(x_1),\ldots,f(x_{i-1}),x_i\}$, for each $i \in \{2,\ldots,n\}$.
We write $f(\bar x)$ for $(f(x_1),\ldots,f(x_n))$. We are now ready to recall how a set of linear TGDs is converted into a set of simple-linear TGDs.

\begin{definition}\label{def:simplification}
	Consider a linear TGD $\sigma$ of the form
	\[
	R(\bar x) \ra \exists \bar z\, \psi(\bar y,\bar z), 
	\]
	where $\bar y \subseteq \bar x$, and a specialization $f$ of $\bar x$. The {\em simplification of $\sigma$ induced by $f$} is the simple-linear TGD
	\[
	\simple{R(f(\bar x))} \rightarrow \exists \bar z\, \simple{\psi(f(\bar y),\bar z)}.
	\]
	We write $\simple{\sigma}$ for the set of all simplifications of $\sigma$ induced by some specialization of $\bar x$.
	%
	For a set $\dep \in \class{L}$ of TGDs, the {\em simplification of $\dep$} is defined as the set
	\[
	\simple{\dep}\ =\ \bigcup_{\sigma \in \dep} \simple{\sigma}
	\]
	consisting only of simple-linear TGDs. \hfill\markfull
\end{definition}

We can now recall the characterization for the finiteness of the chase instance for linear TGDs, established in~\cite{CaGP22}, which is similar to the one for simple-linear TGDs, with the key difference that first we need to simplify both the database and the set of linear TGDs:

\begin{theorem}\label{the:characterization-linear}
	Consider a database $D$ and a set $\dep \in \class{L}$ of TGDs. Then, $\chase{D}{\dep}$ is finite iff $\simple{\dep}$ is $\simple{D}$-weakly-acyclic.
\end{theorem}

It is clear that Theorems~\ref{the:characterization-simple-linear} and~\ref{the:characterization-linear} provide simple algorithms for checking whether the chase instance is finite. In particular, given a database $D$ and a set $\dep$ of simple-linear TGDs, we simply need to check whether $\dep$ is $D$-weakly-acyclic, in which case the algorithm returns \true; otherwise, it returns \false. The same holds when $\dep$ is a set of linear TGDs, with the difference that the algorithm first needs to simplify $D$ and $\dep$, and then perform the acyclicity check.
%
Our goal is to experimentally evaluate the above algorithms with the aim of understanding which input parameters affect their performance, clarifying whether they can be applied in a practical context, and revealing their performance limitations. Of course, a naive implementation of the above algorithms, especially for linear TGDs where the expensive simplification must be applied, will lead to poor performance, and thus, will not be very useful towards our goal. Hence, we need to somehow convert the above theoretical algorithms into practical algorithms that are amenable to efficient implementations. This is the subject of the next section.
\section{Method}
\label{sec:method}

% \ml{``Inconsistent'' to ``large variation''}

% In this section, we propose our methods based on the observations in Section \ref{sec:motivation}.
In this section, we propose two techniques to further enhance the strong baseline to capture the variation of activation distributions better.
We first introduce spatial re-scaling to adapt the network to pixel-to-pixel variation.
We then propose channel-wise shifting and re-scaling to better capture the channel-to-channel variation.
Meanwhile, as both of the two methods are image-dependent, the image-to-image variation can be captured naturally.
By combining the two methods with our strong baseline, we build our enhanced BNN for SR, named EBSR.

% Because the activation distributions among pixels, channels and images have large variations \red{**are highly inconsistent} in SR networks, we introduce spatial re-scaling to adapt to pixel-wise variations and channel shift and re-scaling to adapt to channel-wise variations. And both of them are image-dependent to adapt to image-wise variations, which means during inference our network re-scales and shifts the distributions of activations flexibly for different input images. Based on these methods, we build an enhanced binary neural network for image super-resolution (EBSR).

% According to [3], the difference of activation magnitudes indicates different scaling factors are needed for each pixel.

\subsection{Spatial Re-scaling}
% It is better to use different scaling factors for different pixels to reduce the quantization error and retain more detailed information for image super-resolution. 

% \ml{In the main method, we do not need to introduce the previous works but can focus on introducing our own method. Channel rescaling in Real-to-binary Net is not relevant in this context.}

% Re-scaling the output of binary convolutions was proposed at the birth of BNN in XNOR-Net \cite{rastegari2016xnor} to reduce quantization error and improve accuracy for image classification tasks.
% It is computed as below:
% \begin{equation}
% \mathcal{A} * \mathcal{W} \approx(\operatorname{sign}(\mathcal{A}) \circledast \operatorname{sign}(\mathcal{W})) \odot \mathcal{K} \alpha
% \label{eq:xnor-net rescale}
% \end{equation}
% where $\circledast$ denotes the binary convolution and $\odot$ denotes the element-wise multiplication.
% $\mathcal{A}$, $\mathcal{W}$, $\alpha$, and $\mathcal{K}$ denote the activation, weight, weight scaling factor, and activation scaling factor, respectively.
%  Later in XNOR-Net++ \cite{bulat2019xnor}, Bulat et al. fuse the activation and weight scaling factors into a single one that is learned end-to-end based on gradients and this improves the classification accuracy on ImageNet dataset.

% % It is computed as Eq.~\ref{eq:xnor-net rescale}, where $\circledast$ denotes 
% %  the binary convolution and $\odot$ denotes the element-wise multiplication. The binary convolution of $\mathcal{A}$ and $\mathcal{W}$ is rescaled by the weight scaling factor $\alpha$ and the activation scaling factor $\mathcal{K}$, both of which are calculated analytically.


% \zc{Similarly, you should explain the meaning of A, W and the operators $\circledast$ in the formula}
% Then in Real-to-binary Net \cite{martinez2020training}, Martinez et al. used a data-driven channel re-scaling module that takes the pre-convolution activations as input to predict the activation scaling factor. Unlike that in XNOR-Net++ \cite{bulat2019xnor}, these scaling factors are not fixed during inference but rather inferred from data. By doing this, they further improved the classification accuracy on ImageNet over XNOR-Net++. 
As is shown in Figure \ref{fig:pixel}, activation distributions have large pixel-to-pixel variation in SR networks
and the difference of activation magnitudes indicates different scaling factors are preferred for different pixels.
Inspired by \cite{martinez2020training}, we propose spatial re-scaling to better adapt the network to the spatial variation
of activation distributions in SR networks.
% fit the various pixel-wise distributions in SR networks.
We take the real-valued activations $A$ before convolution as input and predict pixel-wise scaling factors $S(A)$, which re-scale the binary convolution output. Spatial re-scaling process can be formulated as follows:
\begin{equation}
A * W \approx(\operatorname{sign}(A) \circledast \operatorname{sign}(W)) \odot \alpha \odot S(A)
\label{eq:spatial rescale}
\end{equation}
where $\circledast$ denotes 
the binary convolution and $\odot$ denotes the element-wise multiplication. $A$, $W$, $\alpha$, and $S\left(A\right)$ denote real-valued activations, weights, the scaling factor of weights, and the spatial-wise scaling factor of activations respectively. $S\left(A\right) \in \mathbb{R}^{1\times H\times W}$ can be calculated with a convolution and a sigmoid function.
% as $\sigma\left( CONV\left(A\right)\right)$. 
As shown in Figure \ref{fig:method}(a), real-valued activations first go through a convolution layer,
which has an input channel of $C$ and an output channel of 1, 
and then pass through a sigmoid function to produce the scaling factors $S(A)$ along the spatial dimension.
During inference, the scaling factor will change dynamically according to different input feature maps.
By re-scaling binary convolution output using $S(A)$, we can reduce the quantization error and the original pixel-wise information in FP activation
will be preserved much better.
Spatial re-scaling leads to a large PSNR improvement of 0.24 dB (from 30.30 dB to 31.54 dB) on Set5 and 0.22 dB (from 25.09 dB to 25.31 dB)
on Urban100 compared with our strong baseline. 

\subsection{Channel-wise Shifting and Re-scaling}

\begin{table}[!tb]
\centering
\caption{Comparison between whether to fuse channel-wise shifting and re-scaling or not based on our baseline with spatial re-scaling. }
\label{tab:fusing}

\scalebox{0.65}{
\begin{tabular}{c|cc|cc|cc}
\hline
\multirow{2}{*}{Method}     & \multirow{2}{*}{OPs} & \multirow{2}{*}{Params} & \multicolumn{2}{c|}{Set5} & \multicolumn{2}{c}{Urban100} \\ \cline{4-7} 
                            &                      &                         & PSNR        & SSIM        & PSNR          & SSIM         \\ \hline
Baseline + spatial re-scale & 2.16G                & 0.05M                   & 31.54       & 0.883       & 25.31         & 0.759        \\
+ channel-wise shift and re-scale             & 2.34G                & 0.09M                   & 31.61       & 0.885       & 25.35         & 0.761        \\
+ w/ fusing                   & 2.27G                & 0.08M                   & \textbf{31.64}       & \textbf{0.885}       & \textbf{25.36}         & \textbf{0.761}        \\ \hline
\end{tabular}
}
\end{table}

In SR networks, activation distributions exhibit larger channel-to-channel variation (Figure \ref{fig:chl}).
Both the mean and magnitude of the activation distributions vary significantly across channels.
% Thus we use channel-wise shifting and re-scaling to adapt to various channel-wise distributions. 
\cite{martinez2020training} has proposed the data-driven channel re-scaling, 
but our method differs from them in further introducing data-driven thresholds to handle the channel-wise variation of both mean and magnitude.
Since the blocks to generate the scaling factors and thresholds are very similar, we further propose to fuse them into one module.
% and fusing channel-wise shifting and re-scaling into one module.
We evaluate the effect of fusing the two blocks in Table \ref{tab:fusing}.
With channel-wise shifting and re-scaling fused, our models have fewer operations and parameters overhead and slightly higher performance.

For the specific process, we take the real-valued activations as input and predict different thresholds and scaling factors for each channel. They are also image dependent, e.g., $\beta_{i}$ in Eq.\ref{eq:act_binarize} is no longer fixed during inference but generated according to different input feature maps. Channel-wise shifting and re-scaling can be formulated as follows:
\begin{equation}
A * W \approx(\operatorname{sign}(A-C_s(A)) \circledast \operatorname{sign}(W)) \odot \alpha \odot C_r(A)
\label{eq:channel-wise_shift_and_rescale}
\end{equation}
where $\circledast$ denotes 
the binary convolution and $\odot$ denotes the element-wise multiplication. $C_s(A), C_r(A) \in \mathbb{R}^{C\times1\times1}$ denote the channel-wise threshold and scaling factor, respectively. 
We show the block diagram in Figure \ref{fig:method}(b).
The real-valued input feature map is first squeezed to a ${C\times1\times1}$ vector by a global average pooling (GAP) layer.
The subsequent fully connected layers and ReLU learn the channel-wise information and output a ${2C\times1\times1}$ vector.
Then the ${2C\times1\times1}$ vector is split into two ${C\times1\times1}$ vectors.
We use the first $C$ channels as the channel-wise bias and pass the last $C$ channels through a sigmoid layer 
as the channel-wise scaling factor, which are used to shift the real-valued activations and re-scale the binary convolution output, respectively. 


% \ml{We can mention previously, channel-wise re-scale has been proposed. We propose to fuse them. Add the comparison between fuse v.s. no fuse.}

\begin{figure}[!tbp]%
  \centering
    \includegraphics[width=0.4\textwidth]{fig/methods.png}
  
% \subfloat[channel-wise shifting\&re-scale]{
%     \label{subfig:channel-wise shifting and re-scale}
%     \includegraphics[width=0.2\textwidth]{fig/chl shift and rescale.png}
%   }

  \caption{Block diagram for spatial re-scaling, and channel-wise shifting and re-scaling.} 
  % Input A is the real-valued activation tensor and C, H, and W denote its dimension. GAP stands for global average pooling. The reduction ratio r is set to 16 for a better trade-off between the performance and the number of operations and parameters.}
  \label{fig:method}
\end{figure}


\subsection{Network Structure}

Combining the spatial re-scaling and the channel-wise shifting and re-scaling methods, we construct the enhanced convolution layer (E-Conv).
Then we build our EBSR model based on E-Conv.
In Figure \ref{fig:E-conv}, we compare the binary convolution layer used in the baseline network and our proposed E-Conv.
We use spatial and channel-wise scaling factors to re-scale the binary convolution output,
and use channel-wise shifting to learn appropriate thresholds for each channel before binarization.
The scaling factors and threshold used in E-Conv are learnable and depend on the real-valued input activations.
In this way, our proposed EBSR can adapt to pixel-to-pixel, channel-to-channel, and image-to-image variations
to reduce the large binarization error and preserve more details.
% In this way, our proposed E-Conv reduces the large quantization error caused by binarization and keeps the original information of input feature maps to a large extent.


\begin{figure}[!tb]%
  \centering

    \includegraphics[width=0.5\textwidth]{fig/E-conv.png}

  \caption{Comparison of (a) the binary convolution layer with a skip connection used in our baseline network and (b) the proposed E-Conv.}
  \label{fig:E-conv}
\end{figure}


Figure \ref{fig:network} shows the basic block based on the E-Conv and our EBSR composed of the basic blocks. Following existing works, the convolution layers in the head and tail modules are not binarized. We choose the lightweight EDSR which has 16 basic blocks and 64 channels, and EDSR which has 32 basic blocks and 256 channels as our backbones, which correspond to EBSR-light and EBSR, respectively.

\begin{figure}[!tb]%
  \centering
  {
    \includegraphics[width=0.35\textwidth]{fig/network.png}
  }
  
  \caption{The structure of our proposed EBSR.  Convolution layers in purple are real-valued vanilla 3x3 convolutions.}
  \label{fig:network}
\end{figure}
We present in section~\ref{ssec:faces} an application of PnP-HVAE on face images, using a pretrained state-of-the-art hierarchical VAE. 
Next, we study the application of our framework to natural images. To that end, we introduce  in section~\ref{ssec:patchVDVAE}  a patch hierachical VAE architecture, that is able to model natural images of different resolutions. In section~\ref{ssec:app_nat}, we provide deblurring, super-resolution and inpainting experiments to demonstrate the relevance of the proposed method.

Additional results are presented in Appendix~\ref{app:add}. All experiments can be reproduced using the code available at \url{https://github.com/jprost76/PnP-HVAE}.



\subsection{Face Image restoration (FFHQ)}\label{ssec:faces}
We first demonstrate the effectiveness of PnP-HVAE on highly structured data, by performing face image restoration.
Latent variable generative models can accurately model structured images such as face images \cite{karras2019style,vahdat2020nvae,child2021very,kingma2018glow}, and then be used to produce high quality restoration of such data. 
In our experiments, we use the VDVAE model of~\cite{child2021very}, pre-trained on the FFHQ dataset~\cite{karras2019style}, as our hierarchical VAE prior.
VDVAE has $L=66$ latent variable groups in its hierarchy and generates images at resolution $256\times256$.

We compare PnP-HVAE with the intermediate layer optimization algorithm (ILO)~\cite{daras2021intermediate} that is based on a different class of generative models than HVAE. ILO is a GAN inversion method which optimizes the image latent code along with the intermediate layer representation of a StyleGAN to generate an image consistent with a degraded observation.
We use the official implementation of ILO, along with a StyleGAN2 model~\cite{karras2020analyzing, stylegan2pytorch}, that was trained for 550k iterations on images of resolution $256\times256$ from FFHQ.  
As VDVAE and StyleGAN models are not trained on the same train-test split of FFHQ, we chose to evaluate the methods on a subset of 100 images from the CelebA dataset~\cite{liu2018large}. 
For super-resolution, the degradation model corresponds to the application of a gaussian low-pass filter followed by a $\times 4$ sub-sampling, and the addition of a gaussian white noise with $\sigma=3$.
For the deblurring, we considered motion blur and  gaussian kernels, both with a noise level $\sigma=8$. %

We provide quantitative comparisons in table~\ref{table:comp_ILO}, along with a visual comparison of the results in figure~\ref{fig:face_restoration}.
PnP-HVAE has the best  PSNR and SSIM results for all the considered restoration tasks, while ILO provides better results  for the perceptual distance.
By jointly optimizing the image and its latent variable, PnP-HVAE provides  results that are both realistic and consistent with the degraded observation.
On the other hand,  ILO  only optimizes on an extended latent space. This method generates  sharp and realistic images with better LPIPS scores,   
but the results lack  of consistency with respect to the observation, which explains the overall lower PSNR performance. 






\subsection{PatchVDVAE: a HVAE for natural images}\label{ssec:patchVDVAE}
Available generative models in the literature operate on images of  fixed resolutions and
are either restrained to datasets of limited diversity, or even to registered face images~\cite{kingma2018glow,child2021very, vahdat2020nvae, karras2019style}, or requiring additional class information~\cite{brock2018large, dhariwal2021diffusion, song2020score, luhman2022optimizing}.
Fitting an unconditional model on natural images appears to be a more difficult task, as their resolution can change, and their content is highly diverse.
The complexity of the problem can be reduced by learning a prior model on patches of reduced dimension. 
For image restoration problems, the patch model can be reused on images of higher dimensions~\cite{zoran2011learning,prost2021learning,altekruger2022patchnr}. When the model is a full CNN, the prior on the set of the  patches can  be computed efficiently by applying the network on the full image~\cite{prost2021learning}.

We thus introduce  patchVDVAE, a fully convolutional hierarchical VAE.
Contrary to existing HVAE models whose resolution is constrained by the constant tensor at the input of the top-down block, patchVDVAE can generate images of different resolutions by controlling the dimension of the input latent. 
This amounts to defining a prior on patches whose dimension corresponds to the receptive field of the VAE. A similar model is used for image denoising in~\cite{prakash2021interpretable}.

 
For PatchVDVAE architecture, we use the same bottom-up and top-down blocks as VDVAE~\cite{child2021very}, and replace the constant trainable input in the first top-down block by a latent variable, to make the model fully convolutional (details on the  architecture are given in Appendix~\ref{app:details}). 
The training dataset is composed of $128\times 128$ patches extracted from a combination of DIV2K~\cite{agustsson2017ntire} and Flickr2K~\cite{Lim_2017_CVPR_workshops} datasets.
We perform data augmentation by extracting  patches at $3$ resolutions: HR-images and $\times 2$ and $\times 4$ downscaled images. 
The model is trained for $7.10^5$ iterations with a batch size of $64$. Following the recommendation of~\cite{hazami2022efficient}, we use Adamax optimizer with an exponential moving average and gradient smoothing of the variance.
We set the decoder model to be a gaussian with diagonal covariance, as in~\cite{luhman2022optimizing}.
PatchVDVAE is fully convolutional and can generate images of dimension that are multiples of $64$ as illustrated by
figure~\ref{fig:vdvae}.

\newlength{\patchwidth}
\setlength{\patchwidth}{0.135\columnwidth}
\begin{figure}[!ht]
    \centering
    \begin{subfigure}[t]{.34\columnwidth}\hspace{0.1cm}
        \setlength{\tabcolsep}{0.02pt}
\renewcommand{\arraystretch}{0}
        \begin{tabular}{*{2}{p{1.03\patchwidth}}}
            \includegraphics[width=\patchwidth]{figures_arxiv/patchVDVAE/samples/generated/64x64/setup-5-image-0018.png} &
            \includegraphics[width=\patchwidth]{figures_arxiv/patchVDVAE/samples/generated/64x64/setup-5-image-0016.png} \\
            \includegraphics[width=\patchwidth]{figures_arxiv/patchVDVAE/samples/generated/64x64/setup-5-image-0008.png} &
            \includegraphics[width=\patchwidth]{figures_arxiv/patchVDVAE/samples/generated/64x64/setup-5-image-0019.png}   
        \end{tabular}
    \end{subfigure}\hspace{-0.15cm}
    \begin{subfigure}[t]{.64\columnwidth}
\begin{tabular}{cc}\vspace{-0.1cm}
\includegraphics[width=2\patchwidth]{figures_arxiv/patchVDVAE/samples/generated/256x256/setup-2-image-0009.png}&
        \includegraphics[width=2\patchwidth]{figures_arxiv/patchVDVAE/samples/generated/256x256/setup-2-image-0002.png}\end{tabular}

    \end{subfigure}
    \caption{\label{fig:vdvae} Left: $64\times64$ patches samples from our patchVDVAE model trained on patches from natural images.
    Right: PatchVDVAE is fully convolutional and it can generate images of higher resolution (here: $128\times128$).\vspace{-0.2cm}}
\end{figure}

\subsection{Natural images restoration}\label{ssec:app_nat}
We  evaluate PnP-HVAE on natural image restoration.
For each task, we report the average value of the PSNR, the SSIM, and the LPIPS metrics on $20$ images from the test set of the BSD dataset~\cite{MartinFTM01}.\\


\noindent
{\bf Image deblurring.}
In the experiments, we consider $2$ gaussian kernels and $2$ motion blur kernels from~\cite{levin2009understanding}, with $3$ different noise levels 
$\sigma \in \{2.55, 7.65, 12.75\}$.
As a baseline we consider  EPLL~\cite{zoran2011learning}, which learns a prior on image patches with a gaussian mixture model.
We also compare PnP-HVAE  with PnP-MMO and GS-PnP, $2$ competing convergent Plug-and-Play methods based on CNN denoisers.
PnP-MMO~\cite{pesquet2021learning} restricts the denoiser to be contraction in order to guarantee the convergence of the PnP forward-backard algorithm. GS-PnP~\cite{hurault2022gradient} considers a gradient step denoiser and reaches state-of-the-art performances of non converging methods~\cite{zhang2021plug}.
We set the temperature $\tau$  in our method as $0.95$, $0.8$ and $0.6$ for noise levels $2.55$, $7.65$ and $12.75$ respectively, and we let it run for a maximum of $50$ iterations. 
For the three compared methods we use the official implementations and pre-trained models provided by the respective authors. 
Details on the choice of hyperparameters for the concurrent methods are provided in the Appendix~\ref{app:details}
Figure~\ref{fig:deblurring_bsd} illustrates that our method provides correct deblurring results. 

According to table~\ref{tab:deb}, the performance of PnP-HVAE is between those of EPLL and GS-PnP and it outperforms PnP-MMO for large noise levels.\\

\begin{table}
\begin{center}\footnotesize
    \begin{tabular}{>{\centering}m{.3cm}*{5}{c}}
    $\sigma$ &Method & PSNR$\uparrow$ & SSIM$\uparrow$ & LPIPS$\downarrow$  \\ 
    \hline
    \multirow{4}{*}{\vcell{$2.55$}}
    & PnP-HVAE & $27.75$ & $0.79$ & $0.31$\\
    & GS-PNP \cite{hurault2022gradient} & $\mathbf{29.59}$ & $\mathbf{0.84}$ & $\mathbf{0.22}$\\
    & EPLL \cite{zoran2011learning} & $26.49$ & $0.71$ & $0.36$\\ 
    & PnP-MMO \cite{pesquet2021learning} & $\underbar{29.50}$ & $\underbar{0.83}$ & $\underbar{0.20}$ \\ \hline
    \multirow{4}{*}{\vcell{$7.65$}}
    & PnP-HVAE & $\underbar{26.36}$ & $\underbar{0.72}$ & $\underbar{0.40}$\\
    & GS-PNP \cite{hurault2022gradient} & $\mathbf{27.33}$ & $\mathbf{0.77}$ & $\mathbf{0.31}$\\
    & EPLL \cite{zoran2011learning} & $24.04$ & $0.66$ & $0.45$ \\ 
    & PnP-MMO \cite{pesquet2021learning} & $25.34$ & $0.69$ & $0.34$\\
    \hline
    \multirow{4}{*}{\vcell{$12.75$}}
    & PnP-HVAE & $\underbar{25.12}$ & $\mathbf{0.73}$ & $\underbar{0.47}$\\
    & GS-PNP \cite{hurault2022gradient} & $\mathbf{26.32}$ & $\mathbf{0.73}$ & $\mathbf{0.37}$\\
    & EPLL \cite{zoran2011learning} & $23.28$ & $0.61$ & $0.51$ \\ 
    & PnP-MMO \cite{pesquet2021learning} & $22.42$ & $0.53$& $0.54$ \\
    \hline
    &\vspace*{-.3cm}\\
            \multicolumn{2}{c}{Blur and motion kernels}& \multicolumn{3}{c}{
        \includegraphics*[scale=1]{figures_arxiv/kernels/4.png}\;\includegraphics*[scale=1]{figures_arxiv/kernels/7.png}\;\includegraphics*[scale=1]{figures_arxiv/kernels/9.png}\;\includegraphics*[scale=1]{figures_arxiv/kernels/11.png}} 
    \end{tabular}
        \caption{\label{tab:deb}Comparison  of PnP-HVAE  and other restoration methods on deblurring. Results are averaged on $4$ kernels.\vspace{-0.2cm}}% on image deblurring.}
    \end{center}
\end{table}

\begin{figure}
    
    \begin{subfigure}[h]{\linewidth}
        \centering
        \includegraphics*[width=\columnwidth]{figures_arxiv/deb_s255_k7.pdf}\vspace{-0.1cm}
        \caption{Gaussian blur, $\sigma=2.55$}
    \end{subfigure}
    \begin{subfigure}[h]{\linewidth}
        \centering
        \includegraphics*[width=\columnwidth]{figures_arxiv/deb_s765_k11.pdf}\vspace{-0.1cm}
        \caption{Motion blur, $\sigma=7.65$}
    \end{subfigure}\vspace*{-0.1cm}
    \caption{\label{fig:deblurring_bsd} Natural image deblurring\vspace{-0.1cm}}
\end{figure}

\noindent {\bf Effect of the temperature.}
PnP-HVAE gives control on the temperature of the prior over the latent space.
In figure~\ref{fig:temp_effect}, we illustrate that reducing the temperature increases the strength of the regularization prior. In this example the tuning $\tau=0.7$ produces the best performance.\\
\begin{figure}[!ht]
   
    \includegraphics[width=\columnwidth]{figures_arxiv/demo_temp.pdf}\vspace{-0.15cm}
    \caption{ \label{fig:temp_effect} Effect of the temperature in PnP-VAE on a deblurring problem, with $\sigma=7.65$.\vspace{-0.15cm}}
\end{figure}


\noindent
{\bf Image inpainting.}
Next we consider the task of noisy image inpainting. 
We compose a test-set of 10 images from the validation set of BSD~\cite{MartinFTM01} and we create masks
  by occluding diverse objects of small size in the images. 
A gaussian white noise with $\sigma=3$ is added to the images.
As a comparaison, we still consider GS-PnP and EPLL.
For PnP-HVAE, the temperature is set to $\tau=0.6$, and the algorithm is run for a maximum of $200$ iterations, unless the residual $||\x_{k+1}-\x_k||$ is on a plateau.
We provide on Table~\ref{tab:inpainting_bsd} the distortion metrics with the ground truth, as well as a visual
\begin{table}



\begin{center}
    \begin{tabular}{cccc}
        & PSNR$\uparrow$ & SSIM$\uparrow$ &LPIPS$\downarrow$ \\\hline
        PnP-HVAE  & $\mathbf{29.54}$ & $\mathbf{0.93}$ & $\mathbf{0.06}$\\
        GS-PNP & $28.52$ & $\mathbf{0.93}$ & $0.09$\\
        EPLL & $\underline{29.16}$ & $\mathbf{0.93}$ & $\mathbf{0.06}$\\
    \end{tabular}
    \caption{\label{tab:inpainting_bsd}Quantitative evaluation for inpainting on BSD.}
    \end{center}
\end{table}
comparison on figure~\ref{fig:inpainting_bsd}. 
With its hierarchical structure,  PnP-HVAE outperforms the compared methods. \vspace{0.05cm}



\begin{figure}[!h]
    \includegraphics[width=\columnwidth]{figures_arxiv/demo_inp_bsd2.pdf}\vspace{-0.1cm}
    \caption{\label{fig:inpainting_bsd}Natural image inpainting\vspace{-0.3cm}}
\end{figure}











\section{Conclusions}
We consider the phase-extraction problem, and we showed that, given a unitary $U = e^{i\pi H}$ and its inverse $U^{\dag}$, we could implement a block-encoding of $\phi(H)$ for some smooth function $\phi(x)$. The word `smooth' here means existence and continuity of the derivatives: the higher the number of continuous derivatives that a function has, the faster its Fourier sum (and thus the Laurent polynomial on the eigenphases) uniformly converges to that function. We are confident this can have many more applications beyond what is shown in this work. It is also worth remarking that Jackson showed that the convergence rate of a Fourier series is almost-optimal, in the sense that no trigonometric (or, equivalently, complex exponential) series can approximate the desired function faster, up to that $\log d$ factor~\cite[p.\ 21]{jacksonTheoryApproximation1930a}. Also remember that `smoothing' a function, i.e., replacing its derivative with a continuous function, does not give faster convergence for free in general, as its derivative will become steep in the points where we smooth out discontinuities, and this translates to a high Lipschitz constant: a~clear example is given by Eq.~\ref{eq:lipschitz-constant-recurrence-solution}, but in that case, fortunately, nothing depends on the size of the input $N$, and thus does not influence the asymptotic query complexity of Algorithm~\ref{alg:prop-sampling-qsp}, although the constant factor can become large even for $p = 20$. From a theoretical point of view, this work shows that, for any $\eta > 0$, there is an algorithm with query complexity 
$$\Tilde{\bigO}\left(\frac{1}{\bar{c}^{\frac{1}{2} + \eta}} \frac{1}{\epsilon^\eta} \right)$$
solving the proportional-sampling problem. This statement seems to suggest there exists an algorithm which directly solves the problem with $\eta = 0$, and an open question would be to find such algorithm.


It is also interesting to remark that Theorems~\ref{thm:haah-construction},~\ref{thm:haah-completion} indeed allow the construction for any $\phi$, even complex-valued, provided that its absolute value is reciprocal.

One could think that, in Section~\ref{sec:prop-sampling}, instead of using the linear function in the phase-extraction subroutine, we could approximate the square root and then apply the transformation directly on $e^{i \pi c(x)}$. However, in the case of proportional sampling this would be inconvenient, as the derivative of the square root function has a discontinuity with an infinite jump around 0, and we could not choose a constant $\delta$ if we had values of the oracle that are too close to $0$.



% \bibliographystyle{ieeetr}
% \bibliography{bibliography}

\AtNextBibliography{\scriptsize}

\printbibliography






\end{document}
