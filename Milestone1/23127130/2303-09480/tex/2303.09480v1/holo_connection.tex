\section{Holomorphic Levi-Civita Connection}
\label{app:holo_connection}

In this part, we introduce the notion of a (linear/affine) holomorphic connection on a complex manifold $(X,J)$ (cf. the end of Section 4.2 in \cite{huybrechts2005}). In particular, we define the holomorphic Levi-Civita connection $\nabla^h$ induced by a holomorphic metric $h = h_R + ih_I$ on $(X,J)$. We show that the standard Levi-Civita connections $\nabla^{h_R}$ and $\nabla^{h_I}$ of the real and imaginary part $h_R$ and $h_I$ agree with each other, that the holomorphic Levi-Civita connection $\nabla^h$ is just the complexification of $\nabla^{h_R} = \nabla^{h_I}$, and that holomorphic normal coordinates of $h$ are also normal coordinates of $h_R$.\\
We start by defining a holomorphic connection $\nabla$ on a complex manifold $(X,J)$ (cf. \cite{huybrechts2005}):

\begin{Def}[Holomorphic connection]\label{def:holo_connection}
 Let $X$ be a complex manifold with complex structure $J$ and let $\Gamma (T^{(1,0)}U)$ be the complex vector space of holomorphic vector fields on any open subset $U\subset X$. We call $\nabla$ a \textbf{holomorphic connection} on $(X,J)$ iff $\nabla$ is a collection
 \begin{gather*}
  \{\nab{U}:\hvec{U}\times\hvec{U}\to\hvec{U}\mid U\subset X\text{\normalfont\ open}\}
 \end{gather*}
 of $\Cx$-bilinear maps satisfying for all open subsets $U^\prime\subset U\subset X$, all holomorphic functions $f:U\to\Cx$, and all holomorphic vector fields $X,Y\in\hvec{U}$: 
 \begin{enumerate}[label = (\arabic*)]
  \item Tensorial in first component:
  \begin{gather*}
   \nab{U}_{fX} Y = f\nab{U}_X Y.
  \end{gather*}
  \item Leibniz rule in second component:
  \begin{gather*}
   \nab{U}_X fY = X(f) Y + f\nab{U}_X Y.
  \end{gather*}
  \item Presheaf property:
  \begin{gather*}
   (\nab{U}_X Y)\vert_{U^\prime} = \nab{U^\prime}_{X\vert_{U^\prime}} Y\vert_{U^\prime}.
  \end{gather*}
 \end{enumerate}
\end{Def}

\begin{Rem}[Presheaf property]\label{rem:presheaf}
 The standard definition of a (smooth) connection $\nabla$ on a (smooth) manifold $M$\footnote{To be precise, a linear/affine connection $\nabla$ on the vector bundle $TM\to M$.} is formulated globally and usually does not include the presheaf property, since the existence of (smooth) partitions of unity together with Property (1) and (2) implies the presheaf property in those cases. However, there are no holomorphic partitions of unity for complex manifolds, which is why we need to enforce Property (3) manually. Otherwise, the zero map would be a holomorphic connection on the complex torus $\Cx^n/\Gamma$ ($\Gamma$: lattice), since the (global) Leibniz rule reduces to linearity in this case.
\end{Rem}

As in the real case, a holomorphic connection can be computed locally:

\begin{Prop}[$\nabla$ is local]
 Let $(X,J)$ be a complex manifold with holomorphic connection $\nabla$. For all points $p\in X$, all open neighborhoods $U^\prime\subset U\subset X$ of $p$, and all holomorphic vector fields $X,Y\in \hvec{U}$ on $U$, the expression $\nab{U}_X Y (p)$ is completely determined by $X(p)$ and $Y\vert_{U^\prime}$.
\end{Prop}

\begin{proof}
 Take the notations from above, then:
 \begin{gather*}
  \nab{U}_X Y (p) = (\nab{U}_X Y)\vert_{U^\prime} (p) \stackrel{(3)}{=} \nab{U^\prime}_{X\vert_{U^\prime}} Y\vert_{U^\prime} (p)
 \end{gather*}
 Thus, $\nab{U}_X Y(p)$ only depends on $X\vert_{U^\prime}$ and $Y\vert_{U^\prime}$. By going into holomorphic charts of $X$ near $p$ and using Property (1), we see that $\nab{U}_X Y (p)$ only depends on $X(p)$ and $Y\vert_{U^\prime}$.
\end{proof}

Next, we want to define the holomorphic Levi-Civita connection $\nabla^h$ of a holomorphic metric $h$ on $(X,J)$. As in the real case, $\nabla^h$ should be the unique holomorphic connection on $X$ which is torsion-free and compatible with $h$. Thus, we first need to define the torsion $T$ of a holomorphic connection $\nabla$:

\begin{Def}[Torsion $T$]
 Let $(X,J)$ be a complex manifold with holomorphic connection $\nabla$. The \textbf{torsion}\footnote{Sometimes also called torsion tensor or torsion (tensor) field} $T$ of $\nabla$ is the collection of maps
 \begin{gather*}
  \{\tor{U}:\hvec{U}\times\hvec{U}\to\hvec{U}\mid U\subset X\text{\normalfont\ open}\}
 \end{gather*}
 defined by
 \begin{gather*}
  \tor{U}(X,Y)\coloneqq \nab{U}_X Y - \nab{U}_Y X - [X,Y]\quad\forall X,Y\in\hvec{U}\ \forall U\subset X\text{\normalfont\ open}.
 \end{gather*}
 $\nabla$ is said to be \textbf{torsion-free} iff $\tor{U}\equiv 0$ for all open subsets $U\subset X$.
\end{Def}

\begin{Rem*}[$T$ is a tensor]
 $T$ is bitensorial and satisfies the presheaf property.
\end{Rem*}

We also need to say what it means for a holomorphic connection $\nabla$ to be compatible with a holomorphic metric $h$:

\begin{Def}[Metric compatibility]
 Let $(X,J)$ be a complex manifold with holomorphic connection $\nabla$. Furthermore, let $h$ be a holomorphic metric on $X$. The \textbf{metric compatibility tensor} $\nabla h$ of $\nabla$ and $h$ is the collection of maps
 \begin{gather*}
  \{\nab{U}h:\hvec{U}^3\to\mathcal{O}_U\coloneqq\{f:U\to\Cx\mid f\text{\normalfont\ holomorphic}\}\mid U\subset X\text{\normalfont\ open}\}
 \end{gather*}
 defined by
 \begin{gather*}
  \nab{U}h(X,Y,Z)\coloneqq X(h\vert_U (Y,Z)) - h\vert_U (\nab{U}_X Y, Z) - h\vert_U (Y, \nab{U}_X Z)
 \end{gather*}
 for all holomorphic vector fields $X,Y,Z\in\hvec{U}$ and all open subsets $U\subset X$. We say that $\nabla$ is \textbf{compatible} with $h$ iff $\nab{U}h\equiv 0$ for every open subset $U\subset X$.
\end{Def}

\begin{Rem*}[$\nabla h$ is a tensor.]
 $\nabla h$ is tritensorial and satisfies the presheaf property.
\end{Rem*}

We are now ready to define the holomorphic Levi-Civita connection $\nabla^h$:

\begin{Lem}[Levi-Civita connection $\nabla^h$]\label{lem:holo_levi-civita}
 Let $(X,J)$ be a complex manifold together with a holomorphic metric $h$. Then, there exists exactly one holomorphic connection $\nabla^h$ on $X$, the holomorphic \textbf{Levi-Civita connection}, which is torsion-free and compatible with $h$.
\end{Lem}

\begin{proof}
 Take the notations from above. The proof works very similarly to the real case by exploiting the Koszul formula.\\\\
 \noindent \textbf{Uniqueness:} Let $\nabla^{h,1}$ and $\nabla^{h,2}$ be two holomorphic connections which are torsion-free and compatible with $h$. As in the real case, one can show that $\nabla^{h,1}$ and $\nabla^{h,2}$ satisfy the Koszul formula ($U\subset X$ open; $X,Y,Z\in\hvec{U}$):
 \begin{align*}
  2h\vert_U (\nab{U}^{h,1}_X Y, Z) &= X(h\vert_U (Y,Z)) + Y(h\vert_U (Z,X)) - Z (h\vert_U (X,Y))\\
  &+ h\vert_U ([X,Y], Z) - h\vert_U ([Y,Z], X) - h\vert_U([X,Z],Y)\\
  &= 2h\vert_U (\nab{U}^{h,2} Y, Z).
 \end{align*}
 Now let $U\subset X$ be any open subset and $X,Y\in\hvec{U}$ be two holomorphic vector fields on $U$. Pick any point $p\in U$ and a holomorphic chart $(z_1,\ldots, z_n):U^\prime\to V^\prime\subset\Cx^n$ of $X$ near $p$ such that $U^\prime\subset U$. Due to the presheaf property of $\nabla^{h,i}$, we have ($i\in\{1,2\}$):
 \begin{gather*}
  \nab{U}^{h,i}_X Y (p) = \nab{U^\prime}^{h,i}_{X\vert_{U^\prime}} Y\vert_{U^\prime} (p).
 \end{gather*}
 We now set $Z\vert_{U^\prime} = \partial_{z_j}$ ($j\in\{1,\ldots, n\}$) in the formula above (evaluated at $p$) and obtain using the previous equation:
 \begin{gather*}
  h\vert_{U^\prime} (\nab{U^\prime}^{h,1}_{X\vert_{U^\prime}} Y\vert_{U^\prime}, \partial_{z_j}) (p) = h\vert_{U^\prime} (\nab{U^\prime}^{h,2}_{X\vert_{U^\prime}} Y\vert_{U^\prime}, \partial_{z_j}) (p).
 \end{gather*}
 Since the last equation holds for every $j\in\{1,\ldots, n\}$, we can combine the previous two equations to find:
 \begin{gather*}
  \nab{U}^{h,1}_X Y (p) = \nab{U^\prime}^{h,1}_{X\vert_{U^\prime}} Y\vert_{U^\prime} (p) = \nab{U^\prime}^{h,2}_{X\vert_{U^\prime}} Y\vert_{U^\prime} (p) = \nab{U}^{h,2}_X Y (p).
 \end{gather*}
 As this argument can be repeated for every $p\in U$, we find:
 \begin{gather*}
  \nab{U}^{h,1}_X Y = \nab{U}^{h,2}_X Y.
 \end{gather*}
 Again, we can repeat the argument for all $X,Y\in\hvec{U}$ and any open subset $U\subset X$ implying $\nabla^{h,1} = \nabla^{h,2}$.\\\\
 \noindent \textbf{Existence:} The Koszul formula gives us an expression for
 \begin{gather*}
  h\vert_U (\nab{U}^{h}_X Y, Z).
 \end{gather*}
 A priori, it is not cleared whether this expression completely determines $\nab{U}^h_X Y$, as $U$ might be ``too large'' to admit enough linearly independent, holomorphic vector fields $Z$. However, we can always shrink $U$ by the presheaf property. Especially, if we shrink $U$ to be a chart domain, then $U$ admits enough holomorphic vector fields. Thus, we can define $\nab{U}^h_X Y (p)$ for any point $p\in U$ via the Koszul formula in a small chart near $p$. Since the Koszul formula is independent of the choice of charts, the resulting holomorphic connection $\nabla^h$ is well-defined. One easily checks that $\nabla^h$ defined this way is torsion-free and compatible with $h$.
\end{proof}

Next, we want to consider normal coordinates of a holomorphic connection. To do so, we need to define geodesics first:

\begin{Def}[Complex derivative $\nabla/dz$ along $\gamma$ and geodesics]
 Let $(X,J)$ be a complex manifold with holomorphic connection $\nabla$. Further, let $D\subset\Cx$ be an open and connected subset, $\gamma:D\to X$ be a holomorphic curve, and $Y$ be a holomorphic vector field along $\gamma$, i.e., a holomorphic section of the pullback bundle $\gamma^\ast T^{(1,0)}X$. Analogously to the real case, we can define the \textbf{complex derivative} $\nabla/dz$ along $\gamma$. In holomorphic charts $\phi = (z_1,\ldots, z_n):U^\prime\to V^\prime$ of $X$, $\nabla/dz$ is defined by:
 \begin{gather*}
  \left(\frac{\nabla}{dz} Y\right)_j (z)\coloneqq Y_j^\prime (z) + \sum^n_{k,l = 1}\gamma_k^\prime (z)\cdot Y_l (z)\cdot \Gamma^j_{kl} (\gamma (z)),
 \end{gather*}
 where ``$\prime$'' denotes the usual complex derivative and $\nab{U^\prime}_{\partial_{z_k}} \partial_{z_l}\eqqcolon \sum_{j} \Gamma^j_{kl}\cdot \partial_{z_j}$ are the holomorphic \textbf{Christoffel symbols} of $\nabla$. We call $\gamma$ a \textbf{geodesic} of $\nabla$ iff $\nabla/dz\ \gamma^\prime \equiv 0$.
\end{Def}

The following existence and uniqueness result regarding geodesics of $\nabla$ is reminiscent of the real case:

\begin{Prop}[Existence and uniqueness of geodesics]
 Let $(X,J)$ be a complex manifold with holomorphic connection $\nabla$. Further, let $p\in X$ and $v\in T^{(1,0)}_p X$. Then, for every $z_0\in\Cx$, there exists an open and connected neighborhood $D\subset\Cx$ of $z_0$ and a geodesic $\gamma:D\to X$ of $\nabla$ such that $\gamma (z_0) = p$ and $\gamma^\prime (z_0) = v$. Moreover, if $\gamma_1, \gamma_2:D\to X$ are two geodesics of $\nabla$ with $\gamma_1 (z_0) = \gamma_2 (z_0) = p$ and $\gamma^\prime_1 (z_0) = \gamma^\prime_2 (z_0) = v$ for any open and connected neighborhood $D\subset\Cx$ of $z_0$, then $\gamma_1 \equiv \gamma_2$.
\end{Prop}

\begin{proof}
 We observe that the geodesic equation is locally a second order complex differential equation. By introducing new variables $v_i\coloneqq \gamma^\prime_i$, thus, doubling the number of equations, we can rewrite the second order differential equation into a first order one. Since the geodesic equation has no explicit time-dependence ($z$-dependence), we can interpret the first order differential equation obtained this way as the integral curve equation of a holomorphic vector field. The existence and uniqueness results for geodesics now follow from the existence and uniqueness results for integral curves of holomorphic vector fields (cf. Proposition \autoref{prop:holo_traj}).\\
 There is also an alternative way to prove uniqueness: Write the geodesic equation in holomorphic charts near $p$ and expand the coordinates of $\gamma_1$, $\gamma_2$ in a power series around $z_0$. This way, the geodesic equation becomes a recursive formula for the coefficients of the power series. Furthermore, we realize that this recursive formula completely determines all coefficients if we fix the first two terms in each power series. Hence, fixing $\gamma_1 (z_0) = \gamma_2 (z_0) = p$ and $\gamma^\prime_1 (z_0) = \gamma^\prime_2 (z_0) = v$ uniquely determines the coefficients of the power series. The rest now follows from the identity theorem.
\end{proof}

\begin{Rem}[Geodesics depend holomorphically on initial values]
 Proposition \autoref{prop:holo_traj} also shows that a geodesic $\gamma$ of a holomorphic connection $\nabla$ with $\gamma (z_0) = p$ and $\gamma^\prime (z_0) = v$ depends holomorphically on $z_0$, $v$, and $p$.
\end{Rem}

We can now use the geodesics of $\nabla$ to define the exponential map of $\nabla$:

\begin{Def}[Exponential map]
 Let $(X,J)$ be a complex manifold with holomorphic connection $\nabla$  and $p\in X$ be any point. The \textbf{exponential map} of $\nabla$, $\exp_p:V\to X$ with a suitable open neighborhood $V\subset T^{(1,0)}_p X$ of $0$, is defined by:
 \begin{gather*}
  \exp_p (v) \coloneqq \gamma^v_p (1),
 \end{gather*}
 where $\gamma^v_p$ is a geodesic of $\nabla$ satisfying $\gamma^v_p (0) = p$ and $\gamma^{v\prime}_p (0) = v$. Here, $V\subset T^{(1,0)}_p X$ is chosen small enough such that $\gamma^v_p (1)$ is uniquely defined and unaffected by monodromy effects (cf. \autoref{subsec:holo_traj}). For instance, choose $V\subset T^{(1,0)}_pX$ such that the domain of the geodesic $\gamma^v_p:D\to X$ can be chosen to be $D = \{z\in\Cx\mid |z|<2\}$ for every $v\in V$.
\end{Def}

\begin{Rem}[Exponential map is holomorphic]
 By the previous remark, the exponential map $\exp_p$ of a holomorphic connection $\nabla$ is holomorphic.
\end{Rem}

We can now shrink the domain of the exponential map to obtain a biholomorphism:

\begin{Prop}
 Let $(X,J)$ be a complex manifold with holomorphic connection $\nabla$  and let $p\in X$ be any point. Then, there exists an open neighborhood $V\subset T^{(1,0)}_p X$ of $0$ such that $\exp_p: V\to \exp_p (V)$ is a biholomorphism.
\end{Prop}

\begin{proof}
 Apply the holomorphic inverse function theorem to $d\exp_p\vert_0 = \text{id}_{T^{1,0}_p X}$.
\end{proof}

Lastly, we use this biholomorphism to define normal coordinates:

\begin{Def}[Normal coordinates]
 Let $(X,J)$ be a complex manifold with holomorphic connection $\nabla$  and let $p\in X$ be any point. Further, choose a $\Cx$-linear isomorphism $l:\Cx^n\to T^{(1,0)}_p X$. We call the coordinates $(z_1,\ldots, z_n)$ of the holomorphic chart $l^{-1}\circ \exp_p^{-1}$ \textbf{holomorphic normal coordinates} of $(X,\nabla)$ near $p$. If $\nabla = \nabla^h$ is the Levi-Civita connection of a holomorphic metric $h$ on $X$, we additionally require that the vectors $v_1\coloneqq l (\hat e_1),\ldots, v_n\coloneqq l (\hat e_n)$ ($\hat e_1,\ldots, \hat e_n$: standard basis of $\Cx^n$) satisfy:
 \begin{gather*}
  h (v_i, v_j) = \delta_{ij}.
 \end{gather*}
\end{Def}

\begin{Rem*}
 Note that in the complex case all non-degenerate, symmetric bilinear forms are isomorphic, while in the real case two non-degenerate, symmetric bilinear forms are isomorphic if and only if they have the same signature.
\end{Rem*}

Holomorphic normal coordinates satisfy properties similar to their real counterparts:

\newpage

\begin{Prop}[Properties of holomorphic normal coordinates]
 Let $(X,J)$ be a complex manifold with holomorphic metric $h$ and corresponding Levi-Civita connection $\nabla^h$. Further, let $p\in X$ be any point and $\phi = (z_1,\ldots, z_n):U\to V$ be holomorphic normal coordinates of $(X,\nabla^h)$ near $p$. Then, we have in coordinates $(z_1,\ldots, z_n)$:
 \begin{enumerate}[label = (\arabic*)]
  \item $h_{ij} (p) = \delta_{ij}$
  \item $\partial_{z_k}h_{ij} (p) = 0$
  \item $\Gamma^k_{ij} (p) = 0$
 \end{enumerate}
\end{Prop}

\begin{proof}
 We only show (3), as (1) is obvious and (2) follows from (3) by exploiting the vanishing of the metric compatibility tensor. Define for $x = (x_1,\ldots, x_n)\in\Cx^n$ the curve $\gamma (z)\coloneqq \phi^{-1}(z\cdot x)$. By definition of the holomorphic normal coordinates, $\gamma$ is a geodesic of $\nabla^h$ with $\gamma^\prime (0) = d\phi^{-1}\vert_0 (x)$. In coordinates $(z_1,\ldots, z_n)$, the geodesic equation reads:
 \begin{gather*}
  \gamma^{\prime\prime}_{j}(z) + \sum^n_{k,l = 1} \Gamma^j_{kl} (\gamma (z))\gamma^\prime_k (z)\gamma^\prime_l (z) = 0
 \end{gather*}
 For $z = 0$, this gives:
 \begin{gather*}
  \sum^n_{k,l = 1} \Gamma^j_{kl} (p)x_kx_l = 0
 \end{gather*}
 The last equation is true for any $x\in\Cx^n$, thus, it enforces $\Gamma^j_{kl} (p) + \Gamma^j_{lk}(p) = 0$. Since $\nabla^h$ is symmetric, $\Gamma^j_{kl} (p)$ is symmetric in $k$ and $l$. This implies $2\Gamma^j_{kl} (p) = 0$ concluding the proof.
\end{proof}

Our next goal is to relate the holomorphic Levi-Civita connection $\nabla^{h}$ to the standard Levi-Civita connections $\nabla^{h_R}$ and $\nabla^{h_I}$. In particular, we aim to relate the holomorphic normal coordinates of $(X,\nabla^h)$ to the standard normal coordinates of $(X,\nabla^{h_R})$ and $(X,\nabla^{h_I})$ for holomorphic metrics $h = h_R + ih_I$. To achieve that, we first prove the following proposition:

\begin{Prop}\label{prop:h_R_in_coordinates}
 Let $(X,J)$ be a complex manifold with holomorphic metric $h = h_R + ih_I$ and corresponding Levi-Civita connections $\nabla^h$, $\nabla^{h_R}$, and $\nabla^{h_I}$. Further, let $p\in X$ be any point and $\phi = (z_1 = x_1 + iy_1,\ldots, z_n = x_n + iy_n):U\to V$ be holomorphic normal coordinates of $(X,\nabla^h)$ near $p$. Then:
 \begin{enumerate}[label = (\arabic*)]
  \item $h_R\vert_p (\partial_{x_i}\vert_p, \partial_{x_j}\vert_p) = -h_R\vert_p (\partial_{y_i}\vert_p, \partial_{y_j}\vert_p) = \delta_{ij},\quad h_I\vert_p (\partial_{x_i}\vert_p, \partial_{y_j}\vert_p) = \delta_{ij}$\\
  $h_R\vert_p (\partial_{x_i}\vert_p, \partial_{y_j}\vert_p) = h_I\vert_p (\partial_{x_i}\vert_p, \partial_{x_j}\vert_p)\ \ = h_I\vert_p (\partial_{y_i}\vert_p, \partial_{y_j}\vert_p)\qquad\ \ = 0$
  \item All derivatives of $h_R$ and $h_I$ vanish at $p$ in coordinates $(x_1,\ldots, x_n, y_1,\ldots, y_n)$.
  \item All Christoffel symbols of $\nabla^{h_R}$ and $\nabla^{h_I}$ vanish at $p$ in coordinates\linebreak $(x_1,\ldots, x_n, y_1,\ldots, y_n)$.
 \end{enumerate}
\end{Prop}

\begin{proof}
 (1) follows from (1) of the previous proposition:
 \begin{gather*}
  h_R\vert_p + i h_I\vert_p = h\vert_p = \sum^n_{j = 1} dz^2_j\vert_p = \sum^n_{j = 1}\left(dx^2_j\vert_p - dy^2_j\vert_p\right) + i\sum^n_{j = 1}\left(dx_j\vert_p\otimes dy_j\vert_p + dy_j\vert_p\otimes dx_j\vert_p\right).
 \end{gather*}
 (2) follows from (2) of the previous proposition and the fact that the components $h_{ij}$ are holomorphic, i.e., $\partial_{\bar{z}_k} h_{ij} (p) = 0$.\\
 (3) follows from (2) of the proposition at hand and the Koszul formula for Christoffel symbols of standard Levi-Civita connections.
\end{proof}

The last proposition implies that the Levi-Civita connections $\nabla^{h_R}$ and $\nabla^{h_I}$ describe the same connection.

\begin{Cor}[$\nabla^{h_R} = \nabla^{h_I}$]
 Let $(X,J)$ be a complex manifold with holomorphic metric $h = h_R + ih_I$. Then, the (standard) Levi-Civita connections $\nabla^{h_R}$ and $\nabla^{h_I}$ coincide, i.e., $\nabla^{h_R} = \nabla^{h_I}$.
\end{Cor}

\begin{proof}
 This is an immediate consequence of the previous proposition: For every point $p\in X$, there are coordinates in which the Christoffel symbols of $\nabla^{h_R}$ and $\nabla^{h_I}$ agree at $p$, hence, $\nabla^{h_R} = \nabla^{h_I}$ by definition of the Christoffel symbols.
\end{proof}

We will see later on that the holomorphic Levi-Civita connection $\nabla^h$ can be regarded as the complexification of the connection $\nabla^{h_R} = \nabla^{h_I}$. Before we can make this statement precise, we first need to investigate how the complex structure $J$ interacts with the connections $\nabla^{h}$ and $\nabla^{h_R} = \nabla^{h_I}$. For this, we introduce the following definition:

\begin{Def}[Complex connection]
 Let $X$ be a smooth manifold with almost complex structure $J$. We call a connection $\nabla:\Gamma (TX)\times \Gamma (TX)\to \Gamma (TX)$ on $X$ \textbf{complex} with respect to $J$ iff $\nabla J \equiv 0$.
\end{Def}

\begin{Rem*}[Definition of $\nabla J$]
 Recall that for all $X,Y\in\Gamma (TX)$, one has:
 \begin{gather*}
  (\nabla_X J) (Y) = \nabla_X (J(Y)) - J(\nabla_X Y).
 \end{gather*}
\end{Rem*}

We find that $\nabla^{h_R} = \nabla^{h_I}$ is complex:

\begin{Prop}[$\nabla^{h_R} = \nabla^{h_I}$ is complex]
 Let $(X,J)$ be a complex manifold with holomorphic metric $h = h_R + ih_I$. Then, the connection $\nabla^{h_R} = \nabla^{h_I}$ is complex with respect to $J$.
\end{Prop}

\begin{proof}
 We need to show:
 \begin{gather*}
  (\nabla^{h_R}_X J) (Y) = \nabla^{h_R}_X (J(Y)) - J(\nabla^{h_R}_X Y) = 0\quad\forall X,Y\in\Gamma (TX).
 \end{gather*}
 The object $(\nabla^{h_R}_X J) (Y)$ is tensorial in $X$ and $Y$, hence, it suffices to compute $\nabla^{h_R} J$ at any point $p\in X$ for the tangent vector fields associated to the holomorphic normal coordinates $(z_1 = x_1 + iy_1,\ldots, z_n = x_n + iy_n)$ near $p$. By Proposition \autoref{prop:h_R_in_coordinates}, we find:
 \begin{align*}
  \nabla^{h_R}_{\partial_{x_i}}\left( J (\partial_{x_j})\right) (p) &= \nabla^{h_R}_{\partial_{x_i}} \partial_{y_j} (p) = 0,\quad \nabla^{h_R}_{\partial_{x_i}}\left( J (\partial_{y_j})\right) = \ldots\\
  J(\nabla^{h_R}_{\partial_{x_i}} \partial_{x_j}) (p) &= 0,\quad J(\nabla^{h_R}_{\partial_{x_i}} \partial_{y_j}) (p) =\ldots
 \end{align*}
 Hence, $\nabla^{h_R} J = 0$.
\end{proof}

Before we move on with the investigation of the relation between $\nabla^h$ and $\nabla^{h_R} = \nabla^{h_I}$, let us take a moment to understand what it means for a connection to be complex. To do that, we first realize that every connection $\nabla:\Gamma (TX)\times \Gamma (TX)\to \Gamma (TX)$ on a smooth manifold $X$ can be complexified by $\Cx$-linearity, i.e., can be turned into a connection $\nabla:\Gamma (T_\Cx X)\times \Gamma (T_\Cx X)\to \Gamma (T_\Cx X)$ acting on smooth complex vector fields. Now we can rephrase the property ``complex'' as follows:

\begin{Prop}
 Let $X$ be a smooth manifold with almost complex structure $J$. A connection $\nabla:\Gamma (TX)\times \Gamma (TX)\to \Gamma (TX)$ is complex if and only if its complexification $\nabla:\Gamma (T_\Cx X)\times \Gamma (T_\Cx X)\to \Gamma (T_\Cx X)$ satisfies:
 \begin{gather*}
  \nabla_X Y\in \Gamma_{C^\infty} (T^{(1,0)}X)\quad \forall X\in \Gamma (T_\Cx X)\ \forall Y\in \Gamma_{C^\infty} (T^{(1,0)}X),
 \end{gather*}
 where $\Gamma_{C^\infty} (T^{(1,0)}X)$ is the space of smooth sections of $T^{(1,0)}X$.
\end{Prop}

\begin{proof}
 ``$\Rightarrow$'': Let $X\in \Gamma (T_\Cx X)$ and $Y\in \Gamma_{C^\infty} (T^{(1,0)}X)$, then $Y = 1/2 (\hat Y - iJ (\hat Y))$ for a unique real vector field $\hat Y$ on $X$. Thus, we obtain:
 \begin{gather*}
  \nabla_X Y = \frac{1}{2}\left(\nabla_X\hat Y - i\nabla_X (J(\hat Y))\right) \stackrel{\nabla J = 0}{=} \frac{1}{2}\left(\nabla_X\hat Y - iJ (\nabla_X \hat Y)\right)\in\Gamma_{C^\infty} (T^{(1,0)}X).
 \end{gather*}
 ``$\Leftarrow$'': Let $X\in\Gamma (TX)$ and $\hat Y\in \Gamma (TX)$, then $Y\coloneqq 1/2 (\hat Y - iJ(\hat Y))\in\Gamma_{C^\infty} (T^{(1,0)}X)$. Thus, we find $\nabla_X Y \in\Gamma_{C^\infty} (T^{(1,0)}X)$. Since
 \begin{gather*}
  \text{Re} \left(\nabla_X Y\right) = \frac{1}{2} \nabla_X \hat Y,
 \end{gather*}
 we must have:
 \begin{gather*}
  \text{Im} \left(\nabla_X Y\right) = -\frac{1}{2} J(\nabla_X \hat Y).
 \end{gather*}
 In total, this gives:
 \begin{gather*}
  \frac{1}{2}\nabla_X (\hat Y - i J(\hat Y)) = \frac{1}{2}\left(\nabla_X \hat Y - i J(\nabla_X \hat Y)\right)
 \end{gather*}
 Hence:
 \begin{gather*}
  \nabla_X J(\hat Y) = J(\nabla_X\hat Y)\ \Rightarrow\ \nabla J = 0
 \end{gather*}
\end{proof}

We now see that, at first glance, the definition of a complex connection and a holomorphic connection look very similar. A complex connection $\nabla:\Gamma (TX)\times \Gamma (TX)\to \Gamma (TX)$ can be regarded as a map $\nabla:\Gamma_{C^\infty} (T^{(1,0)}X)\times \Gamma_{C^\infty} (T^{(1,0)}X)\to \Gamma_{C^\infty} (T^{(1,0)}X)$ (via complexification) satisfying Property (1), (2), and (3)\footnote{We can convince ourselves that a complex connection satisfies Property (3) by exploiting smooth partitions of unity (cf. Remark \autoref{rem:presheaf}).} from Definition \autoref{def:holo_connection}, where all vector fields $X$ and $Y$ are taken to be smooth complex vector fields and all functions $f$ are taken to be smooth $\Cx$-valued functions. The main difference between complex and holomorphic connections is that holomorphic connections are only defined for integrable $J$ and send holomorphic vector fields to holomorphic vector fields, while complex connections are defined on general almost complex manifolds $(X,J)$. In particular, if $J$ is non-integrable, there is no notion of holomorphic vector fields. Even if $J$ is integrable, complex connections are not required to send holomorphic vector fields to holomorphic vector fields. In this sense, the notion of a holomorphic connection is more restrictive than the notion of a complex connection.\\
We now have all tools at hand to show that $\nabla^h$ is the complexification of $\nabla^{h_R} = \nabla^{h_I}$:

\begin{Lem}[$\nabla^h = \nabla^{h_R} = \nabla^{h_I}$]
 Let $(X,J)$ be a complex manifold with holomorphic metric $h = h_R + ih_I$. Then, the holomorphic Levi-Civita connection $\nabla^h$ is given by the complexification\footnote{We also denote the complexification of $\nabla^{h_R} = \nabla^{h_I}$ by $\nabla^{h_R} = \nabla^{h_I}$.} of the standard Levi-Civita connections $\nabla^{h_R} = \nabla^{h_I}$, i.e.:
 \begin{gather*}
  \nab{U}^h_X Y = \nabla^{h_R\vert_U}_X Y = \nabla^{h_I\vert_U}_X Y\quad\forall X,Y\in\Gamma (T^{(1,0)}X)\ \forall U\subset X\text{ open}.
 \end{gather*}
\end{Lem}

\begin{proof}
 For smooth real vector fields $X,Y,Z\in\Gamma (TX)$, we can express the terms
 \begin{gather*}
  2h_R \left(\nabla^{h_R}_X Y, Z\right)\quad\text{and}\quad 2h_I \left(\nabla^{h_I}_X Y, Z\right)
 \end{gather*}
 with the help of the Koszul formula as in the proof of Lemma \autoref{lem:holo_levi-civita}. We realize that the Koszul formula also holds for complex smooth vector fields $X,Y,Z\in\Gamma (T_\Cx X)$ if we complexify the expressions above and the Koszul formula by $\Cx$-linearity in $X,Y,Z$. In particular, for holomorphic vector fields $X,Y,Z\in\Gamma (T^{(1,0)}X)$, this gives:
 \begin{align*}
  2h (\nabla^{h_R}_X Y, Z) &= 2h_R (\nabla^{h_R}_X Y, Z) + i 2h_I (\nabla^{h_R}_X Y, Z) = 2h_R (\nabla^{h_R}_X Y, Z) + i 2h_I (\nabla^{h_I}_X Y, Z)\\
  &= (\text{Koszul formula for $h_R$}) + i (\text{Koszul formula for $h_I$})\\
  &= \text{Koszul formula for $h$} = 2h (\nabla^h_X Y, Z).
 \end{align*}
 Thus, we have:
 \begin{gather*}
  h(\nabla^{h}_X Y - \nabla^{h_R}_X Y, Z) = 0.
 \end{gather*}
 A priori, the last equation does not enforce $\nabla^h = \nabla^{h_R}$, since the holomorphic metric $h$ is only non-degenerate on $T^{(1,0)}X$, but vanishes on $T^{(0,1)}X$. However, since $\nabla^h$ is holomorphic and $\nabla^{h_R}$ is complex, the vector $\nabla^{h}_X Y - \nabla^{h_R}_X Y$ is of type $(1,0)$ concluding the proof.
\end{proof}

\begin{Rem}[Holomorphic connections as complexifications of real connections]
 Locally, every holomorphic connection $\nabla$ on $X$ can be interpreted as the complexification $\hat\nabla:\Gamma (T_\Cx U)\times \Gamma (T_\Cx U)\to \Gamma (T_\Cx U)$ of a real connection $\hat\nabla:\Gamma (T U)\times \Gamma (T U)\to \Gamma (T U)$ on a chart domain $U\subset X$. For instance, we can define $\hat\nabla$ by\footnote{These equations determine by $\Cx$-linearity the expressions $\hat\nabla_{\pa_{x_i}}\pa_{x_j}$, $\hat\nabla_{\pa_{x_i}}\pa_{y_j}$,\ldots, which in turn determine the Christoffel symbols of $\hat\nabla$. Since $\hat\nabla$ satisfies $\hat\nabla_{\pa_{z_i}}\pa_{z_j} = \overline{\hat\nabla_{\pa_{\bar{z}_i}}\pa_{\bar{z}_j}}$ and $\hat\nabla_{\pa_{\bar{z}_i}}\pa_{z_j} = \overline{\hat\nabla_{\pa_{z_i}}\pa_{\bar{z}_j}}$, the Christoffel symbols of $\hat\nabla$ w.r.t. the chart $(x_1,\ldots, y_n)$ are real implying that $\hat\nabla$ is indeed a real connection on $U$.}:
 \begin{gather*}
  \hat\nabla_{\pa_{z_i}}\pa_{z_j} = \nab{U}_{\pa_{z_i}}\pa_{z_j},\quad \hat\nabla_{\pa_{\bar{z}_i}}\pa_{\bar{z}_j} = \overline{\nab{U}_{\pa_{z_i}}\pa_{z_j}},\quad \hat\nabla_{\pa_{\bar{z}_i}}\pa_{z_j} = \hat\nabla_{\pa_{z_i}}\pa_{\bar{z}_j} = 0.
 \end{gather*}
 However, such a connection $\hat\nabla$ is not unique. In fact, replacing the last equation by
 \begin{gather*}
  \hat\nabla_{\pa_{\bar{z}_i}} \pa_{z_j} = \overline{\hat\nabla_{\pa_{z_i}} \pa_{\bar{z}_j}} = \sum^n_{k=1} f^k_{ij}\cdot \pa_{z_k} + g^k_{ij}\cdot \pa_{\bar{z}_k}
 \end{gather*}
 with $f^k_{ij}, g^k_{ij}\in C^\infty (U,\Cx)$ also yields a suitable connection $\hat\nabla$.
\end{Rem}

Lastly, we want to examine the relation between the holomorphic normal coordinates of $\nabla^h$ and the normal coordinates of $\nabla^{h_R}$. The following formula will be helpful in the upcoming discussion.

\begin{Prop}
 Let $(X,J)$ be a complex manifold with holomorphic metric $h = h_R + i h_I$ and Levi-Civita connection $\nabla^h\equiv \nabla^{h_R}\equiv \nabla^{h_I}$. Then:
 \begin{gather*}
  \nabla^{h_R}_{J(X)} Y = i\nabla^{h_R}_X Y\quad\forall X\in\Gamma (TX)\ \forall Y\in\Gamma (T^{(1,0)}X).
 \end{gather*}
\end{Prop}

\begin{proof}
 Take the notations from above and pick any point $p\in X$. We show:
 \begin{gather*}
  \nabla^{h_R}_{J(X)} Y (p) = i\nabla^{h_R}_X Y (p).
 \end{gather*}
 Let $(z_1 = x_1 + iy_1,\ldots, z_n = x_n + iy_n)$ be holomorphic normal coordinates of $(X,\nabla^h)$ near $p$ and write
 \begin{gather*}
  Y = \sum^n_{j = 1} c_j\pa_{z_j}
 \end{gather*}
 for some locally defined holomorphic functions $c_j$. Then:
 \begin{align*}
  \nabla^{h_R}_{J(X)} Y (p) &= \sum^n_{j = 1} c_j (p)\cdot \nabla^{h_R}_{J(X)}\pa_{z_j} (p) + dc_j (J(X)) (p)\cdot\pa_{z_j}\vert_p\\
  &= \sum^n_{j = 1} c_j (p)\cdot \nabla^{h_R}_{J(X)}\pa_{z_j} (p) + i\cdot dc_j (X) (p)\cdot\pa_{z_j}\vert_p\\
  &= \sum^n_{j = 1} i\cdot dc_j (X) (p)\cdot\pa_{z_j}\vert_p\\
  &= i\sum^n_{j = 1} c_j (p)\cdot \nabla^{h_R}_{X}\pa_{z_j} (p) + dc_j (X) (p)\cdot\pa_{z_j}\vert_p = i\nabla^{h_R}_X Y (p),
 \end{align*}
 where we used that $\nabla^{h_R}$ is $\Cx$-linear in both components, tensorial in the first component, and $\nabla^{h_R}_{\pa_{x_i}} \pa_{x_j} (p) = \nabla^{h_R}_{\pa_{x_i}} \pa_{y_j} (p) = \ldots = 0$ (cf. Proposition \autoref{prop:h_R_in_coordinates}).
\end{proof}

The formula we have just derived now allows us to link the (holomorphic) geodesics of $\nabla^h$ with the geodesics of $\nabla^{h_R} = \nabla^{h_I}$.

\begin{Prop}
 Let $(X,J)$ be a complex manifold with holomorphic metric $h = h_R + ih_I$  and Levi-Civita connection $\nabla^h\equiv\nabla^{h_R}\equiv\nabla^{h_I}$. Further, let $U\coloneqq [t_0, t_1] + i[s_0, s_1]$ be a domain in $\Cx$ and $\gamma:U\to X$ be a holomorphic curve. Define the curves $\gamma_s:[t_0,t_1]\to X$ and $\gamma_t: [s_0, s_1]\to X$ by $\gamma_s (t) \coloneqq \gamma (t+is) \eqqcolon \gamma_t (s)$. Then, $\gamma$ is a (holomorphic) geodesic of $\nabla^h$ iff $\gamma_s$ is a geodesic of $\nabla^{h_R} = \nabla^{h_I}$ for every $s\in[s_0, s_1]$ iff $\gamma_t$ is a geodesic of $\nabla^{h_R} = \nabla^{h_I}$ for every $t\in[t_0, t_1]$.
\end{Prop}

\begin{proof}
 This statement is mostly a consequence of the previous formula and the fact that $\nabla^{h_R}$ is complex: For $z = t+is$, we can write:
 \begin{gather*}
  \gamma^\prime (z) = \frac{1}{2}\left(\pa_t\gamma (z) - iJ(\pa_t\gamma (z))\right) = \frac{1}{2}\left(-J(\pa_s\gamma (z)) - i\pa_s\gamma (z)\right).
 \end{gather*}
 Hence:
 \begin{align*}
  \frac{\nabla^h}{dz}\gamma^\prime &= \nabla^h_{\gamma^\prime} \gamma^\prime = \nabla^{h_R}_{\gamma^\prime} \gamma^\prime = \nabla^{h_R}_{1/2 (\pa_t\gamma - iJ(\pa_t\gamma))} \gamma^\prime = \nabla^{h_R}_{\pa_t\gamma}\gamma^\prime = \frac{1}{2}\left(\nabla^{h_R}_{\pa_t\gamma} \pa_t\gamma - iJ\left(\nabla^{h_R}_{\pa_t\gamma} \pa_t\gamma\right)\right)\\
  &= \frac{1}{2}\left(\frac{\nabla^{h_R}}{dt}\frac{d\gamma_s}{dt} - iJ\left(\frac{\nabla^{h_R}}{dt}\frac{d\gamma_s}{dt}\right)\right).
 \end{align*}
 A similar expression can be found for $\gamma_t$ concluding the proof.
\end{proof}

We are now able to prove the following lemma:

\begin{Lem}
 Let $(X,J)$ be a complex manifold with holomorphic metric $h = h_R + ih_I$. Let $p\in X$ be any point. Then, holomorphic normal coordinates\linebreak $(z_1 = x_1 + iy_1,\ldots, z_n = x_n + iy_n)$ of $(X,\nabla^h)$ near $p$ give rise to normal coordinates $(x_1,\ldots, y_n)$ of $(X, \nabla^{h_R})$ near $p$.
\end{Lem}

\begin{proof}
 Combine all previous results from this section. Note that the same result is only true for $h_I$ after applying a linear transformation, since
 \begin{gather*}
  h_R\vert_p = \sum^n_{j = 1} dx^2_j\vert_p - dy^2_j\vert_p
 \end{gather*}
 is in standard form at $p$ in the coordinates $(x_1,\ldots, y_n)$, while the same is not true for
 \begin{gather*}
  h_I\vert_p = \sum^n_{j=1} dx_j\vert_p\otimes dy_j\vert_p + dy_j\vert_p\otimes dx_j\vert_p.
 \end{gather*}
\end{proof}
