\section{Almost Complex Structures on $\mathbf{TM}$ and $\mathbf{T^\ast M}$}
\label{app:almost_complex_structures}

In this part, we explain how a connection $\nabla$ on a manifold $M$ induces an almost complex structure $J_\nabla$ on its tangent bundle $TM$ (cf. \cite{Cieliebak1994}). If $\nabla = \nabla^g$ is the Levi-Civita connection of a semi-Riemannian metric $g$ on $M$, then $\nabla^g$ also defines an almost complex structure $J^\ast_{\nabla^g}$ on the cotangent bundle $T^\ast M$ via the bundle isomorphism $G:TM\to T^\ast M, v\mapsto \iota_v g$. In this case, $J^\ast_{\nabla^g}$ is compatible with the canonical symplectic form $\omega_\can$ on $T^\ast M$ in the sense that $\omega_\can (\cdot, J^\ast_{\nabla^g}\cdot)$ is a semi-Riemannian metric on $T^\ast M$ of signature $(2s, 2t)$, where $(s,t)$ is the signature of $g$. Furthermore, we will see that $J_{\nabla^g}$ or, equivalently, $J^\ast_{\nabla^g}$ is integrable if and only if $\nabla^g$ or, equivalently, $g$ is flat.\\
Let $M$ a smooth manifold of dimension $n$, $\pi_E:E\to M$ be a (smooth) vector bundle, and $\nabla$ be a linear connection\footnote{Sometimes, the term ``affine connection'' is used instead of ``linear connection''.} on the vector bundle $E\to M$. The (fiberwise) kernel of the differential $d\pi_E: TE\to TM$ yields the vertical subbundle $VE$ of $TE$, while $\nabla$ defines the horizontal subbundle $HE$ of $TE$. In fact, the notion of a horizontal subbundle $HE$ is equivalent to the notion of a linear connection $\nabla$ on $E\to M$. To see this, let $K:TE = VE\oplus HE\to E$ be the vertical projection\footnote{
 Technically speaking, the map $\hat K:TE = VE\oplus HE\to VE$ is the vertical projection. To obtain $K$ from $\hat K$, we have already exploited the fact that $E\to M$ is a vector bundle allowing us to identify the fibers of $VE$ with the fibers of $E$ via the linear isomorphism
 \begin{gather*}
  E_p\to V_wE, v\mapsto \left.\frac{d}{dt}\right\vert_{t = 0} (w + vt)
 \end{gather*}
 for $p\in M$ and $w\in E_p = \pi_E^{-1}(p)$.
} (cf. \cite{Eliasson1967} for the construction of $K$). The data $HE$ and $K$ are equivalent, since, given the horizontal subbundle $HE$, we can always define the vertical projection $K$ and, given the map $K$, we can always define the horizontal bundle $HE$ to be the (fiberwise) kernel of $K$. Likewise, the data $\nabla$ and $K$ are equivalent. Their relation is encoded in the following formula:
\begin{gather*}
 \nabla_X Y = K\circ dY (X)\quad\forall X\in TM,
\end{gather*}
where the section $Y\in\Gamma (E)$ is viewed as a smooth map $Y:M\to E$.\\
Now observe that the map $f^\nabla \coloneqq (d\pi_E, K):TE\to TM\oplus E$ is a bundle map over\linebreak $\pi_E:E\to M$, i.e., the diagram
\begin{center}
 \begin{tikzcd}
  TE \arrow[r, "f^\nabla"] \arrow[d]
  & TM\oplus E \arrow[d] \\
  E \arrow[r, "\pi_E"]
  & M
 \end{tikzcd}
\end{center}
commutes, where the vertical arrows are the base point projections of the vector bundles $TE$ and $TM\oplus E$. Fiberwise, the bundle map $f^\nabla$ is a linear isomorphism. Thus, $TE$ is isomorphic to the pullback bundle $\pi_E^\ast (TM\oplus E)$.\\
Now take $E\to M$ to be the tangent bundle $TM\to M$ of $M$ with base point projection $\pi\equiv \pi_{TM}:TM\to M$. Then, $f^\nabla$ allows us to define the almost complex structure $J_\nabla$ on the manifold $TM$ by ``pulling back'' the almost complex structure\linebreak $J_{TM\oplus TM}:TM\oplus TM \to TM\oplus TM, (w_1, w_2)\mapsto (w_2, -w_1)$ of the vector bundle\linebreak $TM\oplus TM\to M$ to the vector bundle $T(TM)\to TM$. Explicitly speaking, the almost complex structure $J_\nabla$ is completely determined by the following equations:
\begin{gather*}
 d\pi\circ J_\nabla = K;\quad K\circ J_\nabla = -d\pi.
\end{gather*}
Next, we wish to express $J_\nabla$ in local coordinates. For this, we need to parametrize the vertical and horizontal subspaces first. Choose a point $p\in M$ and normal coordinates $\psi = (x_1,\ldots, x_n):U\to V\subset\R^n$ of $(M,\nabla)$ near $p$, i.e., a chart $\psi$ in which all lines through the origin $\psi (p) = 0$ are geodesics. We denote by $T\psi = (\hat x_1,\ldots, \hat x_n, v_1,\ldots, v_n)$ the coordinates of $TM$ near any point $w\in TM$ with $\pi (w) = p$, which are defined by:
\begin{gather*}
 (T\psi)^{-1} (\hat x_1,\ldots, \hat x_n, v_1,\ldots, v_n)\coloneqq \sum^n_{k = 1}v_k\partial_{x_k}\vert_{\psi^{-1}(\hat x_1,\ldots, \hat x_n)}.
\end{gather*}
We then find:
\begin{gather*}
 d\pi (\partial_{v_k}) = \left.\frac{d}{dt}\right\vert_{t = 0} \pi\left(\sum^n_{l\neq k}v_l\partial_{x_l} + (v_k + t)\partial_{x_k}\right) = \left.\frac{d}{dt}\right\vert_{t = 0} \pi\left(\sum^n_{l = 1}v_l\partial_{x_l}\right) = 0.
\end{gather*}
Thus, the vector fields $\partial_{v_1},\ldots, \partial_{v_n}$ span the vertical subspaces.\\
To parametrize the horizontal subspaces, we consider the local vector field $X_c\coloneqq \sum_k c_k\partial_{x_k}$ with constants $c_k\in\R$. We find:
\begin{gather*}
 \nabla_{\partial_{x_k}} X_c (p) = K\circ dX_c\vert_p (\partial_{x_k}\vert_p) = K\left(\left.\frac{d}{dt}\right\vert_{t = 0}\left(X_c\circ \psi^{-1}(\psi (p) + t\hat e_k)\right)\right) = K(\partial_{\hat x_k}\vert_{X_c (p)})
\end{gather*}
Hence, the vectors $\partial_{\hat x_1}\vert_w,\ldots, \partial_{\hat x_n}\vert_w$ span the horizontal subspaces $H(TM) = \ker (K)$ for any $w = X_c (p)\in T_pM$ if and only if the equation
\begin{gather*}
 \nabla_{\partial_{x_i}}\partial_{x_j} (p) = 0
\end{gather*}
holds for all $i,j\in\{1,\ldots, n\}$. The last equation is satisfied for all normal coordinates $(x_1,\ldots, x_n)$ near any point $p\in M$ if and only if $\nabla$ is symmetric, i.e., satisfies:
\begin{gather*}
 \nabla_X Y - \nabla_Y X = [X,Y]\quad \forall X,Y\in\Gamma (TM).
\end{gather*}
Thus, we shall henceforth assume that the connection $\nabla$ is symmetric.\\
In total, we have found that the vertical subspaces at $w\in T_pM$ are spanned by the vectors $\partial_{v_1}\vert_w,\ldots, \partial_{v_n}\vert_w$, while the horizontal subspaces at $w\in T_pM$ are spanned by the vectors $\partial_{\hat x_1}\vert_w,\ldots, \partial_{\hat x_n}\vert_w$ for normal coordinates $\psi = (x_1,\ldots, x_n)$ of $(M,\nabla)$ near $p\in M$ with $T\psi = (\hat x_1,\ldots, \hat x_n, v_1,\ldots, v_n)$. If we express an arbitrary vector $u\in T_w (TM)$ and its image $J_\nabla(u)\in T_w (TM)$ as
\begin{gather*}
 u = \sum^n_{k = 1}a_k\partial_{v_k}\vert_w + b_k\partial_{\hat x_k}\vert_w\quad\text{and}\quad J_\nabla(u) = \sum^n_{k = 1}c_k\partial_{v_k}\vert_w + d_k\partial_{\hat x_k}\vert_w,
\end{gather*}
we can compute the coefficients $c_k$ and $d_k$ in terms of $a_k$ and $b_k$:
\begin{alignat*}{2}
 -\sum^n_{k = 1}b_k\partial_{x_k}\vert_p &= -d\pi (u) = K\circ J_\nabla (u) = \sum^n_{k = 1}c_k\partial_{x_k}\vert_p\quad &\Rightarrow c_k = -b_k\\
 \sum^n_{k = 1} a_k\partial_{x_k}\vert_p &= K(u) = d\pi\circ J_\nabla (u) = \sum^n_{k =1}d_k\partial_{x_k}\vert_p\quad &\Rightarrow d_k = a_k,
\end{alignat*}
where we used $V(TM) = \ker (d\pi)$, $H(TM) = \ker (K)$, and $d\pi (\partial_{\hat x_k}\vert_w) = \partial_{x_k}\vert_p = K (\partial_{v_k}\vert_w)$. This gives us:
\begin{gather*}
 J_\nabla (\partial_{v_k}\vert_w) = \partial_{\hat x_k}\vert_w;\quad J_\nabla (\partial_{\hat x_k}\vert_w) = -\partial_{v_k}\vert_w.
\end{gather*}
We now see that the almost complex structure $J_\nabla$ assumes the standard form in normal coordinates near $p\in M$ for points $w\in T_pM$. This does not mean, however, that $J_\nabla$ is integrable, since the last equation is not necessarily true for all points $w$ within the chart domain $TU$. This is the case if $\nabla_{\partial_{x_i}}\partial_{x_j} \equiv 0$, i.e., if $\nabla$ is flat\footnote{If, additionally, $\nabla = \nabla^g$ is the Levi-Civita connection of some semi-Riemannian metric $g$, this becomes an ``if and only if''-statement, i.e., $J_{\nabla^g}$ is integrable if and only if $\nabla^g$ is flat if and only if $g$ is flat (this statement was proven for Riemannian metrics by Michel Grueneberg \cite{Grueneberg2001}, but the same proof also works for semi-Riemannian metrics).}.\\
Next, we want to transfer the almost complex structure $J_\nabla$ from $TM$ to $T^\ast M$. In general, we can pick any bundle isomorphism $TM\to T^\ast M$, which is then also a diffeomorphism between the manifolds $TM$ and $T^\ast M$, and translate $J_\nabla$ using this diffeomorphism. However, there is no canonical choice of bundle isomorphism for generic manifolds $M$ with connection $\nabla$. The situation is different if $M$ is equipped with a semi-Riemannian metric $g$ and $\nabla$ is the Levi-Civita connection $\nabla^g$. In this case, we can choose $G:TM\to T^\ast M$, $v\mapsto \iota_v g$ as our bundle isomorphism and define the almost complex structure $J^\ast_{\nabla^g}$ on $T^\ast M$ via
\begin{gather*}
 J^\ast_{\nabla^g}\coloneqq dG\circ J_{\nabla^g}\circ dG^{-1}.
\end{gather*}
Again, we aim to express $J^\ast_{\nabla^g}$ in normal coordinates $\psi = (x_1,\ldots, x_n)$ of $(M, g)$ near $p\in M$. Note that the chosen normal coordinates satisfy
\begin{gather*}
 g(p) = \sum^s_{k = 1} dx_k\vert^2_p - \sum^{n = s+t}_{k = s+1} dx_k\vert^2_p\quad\text{and}\quad \partial_{x_k} g_{lm} (p) = 0,
\end{gather*}
where $(s,t)$ is the signature of $g$. As before, we introduce the notation\linebreak $T\psi = (\hat x_1,\ldots, \hat x_n, v_1,\ldots, v_n)$ for the induced coordinates on $TM$. Similarly, we employ the notation $T^\ast \psi = (q_1,\ldots, q_n, p_1,\ldots, p_n)$ for the induced coordinates on $T^\ast M$:
\begin{gather*}
 (T^\ast\psi)^{-1} (q_1,\ldots, q_n, p_1,\ldots, p_n) \coloneqq \sum^n_{k = 1} p_k dx_k\vert_{\psi^{-1}(q_1,\ldots, q_n)}.
\end{gather*}
In these coordinates, $G$ is given by:
\begin{align*}
 T^\ast\psi\circ G\circ (T\psi)^{-1} (0,\ldots, 0, v_1,\ldots, v_n) = (0,\ldots, 0, v_1,\ldots, v_s, -v_{s+1},\ldots, -v_n).
\end{align*}
Together with $\partial_{x_k} g_{lm} (p) = 0$, this implies:
\begin{gather*}
 dG\vert_w (\partial_{\hat x_k}\vert_w) = \partial_{q_k}\vert_{G(w)},\quad dG\vert_w (\partial_{v_k}\vert_w) = \begin{cases}\partial_{p_k}\vert_{G(w)}\text{ for }1\leq k\leq s\\ -\partial_{p_k}\vert_{G(w)}\text{ for } k>s\end{cases}\quad \forall w\in T_pM.
\end{gather*}
This allows us to compute $J^\ast_{\nabla^g}$ in coordinates for points $\alpha\in T^\ast_p M$:
\begin{alignat*}{3}
 &J^\ast_{\nabla^g} (\partial_{q_k}\vert_\alpha) = -\partial_{p_k}\vert_\alpha;\quad &&J^\ast_{\nabla^g} (\partial_{p_k}\vert_\alpha) = \phantom{-} \partial_{q_k}\vert_\alpha\quad &&(1\leq k\leq s),\\
 &J^\ast_{\nabla^g} (\partial_{q_k}\vert_\alpha) = \phantom{-} \partial_{p_k}\vert_\alpha;\quad &&J^\ast_{\nabla^g} (\partial_{p_k}\vert_\alpha) = -\partial_{q_k}\vert_\alpha\quad &&(k>s).
\end{alignat*}
Again, this does not imply that $J^\ast_{\nabla^g}$ is integrable, as the equations above are only true for points $\alpha\in T^\ast_p M$ and not necessarily the entire chart domain $T^\ast U$. However, $J^\ast_{\nabla^g}$ is integrable if and only if $\nabla^g$ is flat, i.e., if and only if $g$ is flat (cf. \cite{Grueneberg2001}).\\
The reason why we are interested in the almost complex structure $J^\ast_{\nabla^g}$ is the curious fact that $J^\ast_{\nabla^g}$ is naturally compatible with the canonical symplectic form $\omega_\can$ on $T^\ast M$, as one easily checks: In the coordinates $(q_1,\ldots, q_n, p_1,\ldots, p_n)$ of $T^\ast M$ from above, $\omega_\can$ is given by:
\begin{gather*}
 \omega_\can\vert_{T^\ast U} = \sum^n_{k = 1} dp_k\wedge dq_k.
\end{gather*}
Thus, $\omega_\can (\cdot, J^\ast_{\nabla^g}\cdot)$ is a semi-Riemannian metric on $T^\ast M$ of signature $(2s, 2t)$:
\begin{gather*}
 \omega_\can\vert_\alpha (\cdot, J^\ast_{\nabla^g}\cdot) = \sum^s_{k = 1} dq_k\vert^2_\alpha + dp_k\vert^2_\alpha - \sum^{n}_{k = s+1} dq_k\vert^2_\alpha + dp_k\vert^2_\alpha\quad\forall \alpha\in T_pM.
\end{gather*}
As we can find normal coordinates near any $p\in M$, we have proven the following theorem:
\begin{Thm}[Almost complex structures on $TM$ and $T^\ast M$]
 Let $M$ be a smooth,\linebreak $n$-dimensional manifold together with a connection $\nabla$ on it. Then, there exists a unique almost complex structure $J_\nabla$ on the tangent bundle $TM$ such that
 \begin{gather*}
  d\pi\circ J_\nabla = K;\quad K\circ J_\nabla = -d\pi,
 \end{gather*}
 where $\pi:TM\to M$ is the base point projection of $TM$ and $K:T(TM)\to TM$ is the vertical projection corresponding to $\nabla$. If $\nabla$ is symmetric, then $J_\nabla$ can be expressed as
 \begin{gather*}
  J_\nabla (\partial_{v_k}\vert_w) = \partial_{\hat x_k}\vert_w;\quad J_\nabla (\partial_{\hat x_k}\vert_w) = -\partial_{v_k}\vert_w,
 \end{gather*}
 where $w\in TM$ is a point, $\psi = (x_1,\ldots, x_n)$ are normal coordinates of $(M,\nabla)$ near $p = \pi (w)$, and $T\psi = (\hat x_1,\ldots, \hat x_n, v_1,\ldots, v_n)$ are the induced coordinates on $TM$. If, additionally, $\nabla$ is flat, then $J_{\nabla}$ is integrable.\\
 If $\nabla = \nabla^g$ is the Levi-Civita connection of a semi-Riemannian metric $g$ on $M$ of signature $(s,t)$, then there exists a (canonical) almost complex structure $J^\ast_{\nabla^g}$ on $T^\ast M$ such that $\omega_\can (\cdot, J^\ast_{\nabla^g}\cdot)$ is a semi-Riemannian metric on $T^\ast M$ of signature $(2s, 2t)$, where $\omega_\can$ is the canonical symplectic form on $T^\ast M$. In coordinates, $J^\ast_{\nabla^g}$ is given by:
 \begin{alignat*}{3}
  &J^\ast_{\nabla^g} (\partial_{q_k}\vert_\alpha) = -\partial_{p_k}\vert_\alpha;\quad &&J^\ast_{\nabla^g} (\partial_{p_k}\vert_\alpha) = \phantom{-} \partial_{q_k}\vert_\alpha\quad &&(1\leq k\leq s),\\
  &J^\ast_{\nabla^g} (\partial_{q_k}\vert_\alpha) = \phantom{-} \partial_{p_k}\vert_\alpha;\quad &&J^\ast_{\nabla^g} (\partial_{p_k}\vert_\alpha) = -\partial_{q_k}\vert_\alpha\quad &&(k>s),
 \end{alignat*}
 where $\alpha\in T^\ast_p M$ is a point, $\psi = (x_1,\ldots, x_n)$ are normal coordinates of $(M,g)$ near $p\in M$, and $T^\ast \psi = (q_1,\ldots, q_n, p_1,\ldots, p_n)$ are the induced coordinates of $T^\ast M$. Furthermore, $J^\ast_{\nabla^g}$ is integrable if and only if $g$ is flat.
\end{Thm}
