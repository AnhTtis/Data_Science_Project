\section{Holomorphic Hamiltonian Systems}
\label{sec:HHS}

In this section, we recreate the results known from RHSs (laid out in \autoref{sec:intro}% or in \autoref{sec:RHS}
) for holomorphic Hamiltonian systems. We first cover the basic notion of a holomorphic Hamiltonian system (\autoref{subsec:def_HHS}), including the definition of a holomorphic symplectic manifold and Darboux's theorem for holomorphic symplectic manifolds, afterwards discuss the properties of holomorphic trajectories (\autoref{subsec:holo_traj}), then formulate an action principle for holomorphic Hamiltonian systems (\autoref{subsec:holo_action_fun_and_prin}), and, lastly, apply HHSs to investigate the relation between Lefschetz and almost toric fibrations (\autoref{subsec:Lefschetz}).

\subsection{HHS: Basic Definitions and Notions}
\label{subsec:def_HHS}

To begin with, we define a holomorphic symplectic manifold:
\begin{Def}[Holomorphic symplectic manifold]\label{def:holo_sym_man}
 A pair $(X,\Omega)$ is called \textbf{holomorphic symplectic manifold}\footnote{Warning: In some branches of algebraic geometry, the term ``holomorphic symplectic manifold'' is also used, but defined with additional constraints on $X$ and $\Omega$!} (HSM) iff $X$ is a complex manifold and $\Omega$ is a holomorphic $2$-form on $X$ which is closed and non-degenerate on the $(1,0)$-tangent bundle $T^{(1,0)}X$ of $X$. In this setup, $\Omega$ is called the holomorphic symplectic $2$-form of $(X,\Omega)$.
\end{Def}
Let us spend some time understanding the definition of a HSM. Recall that a complex manifold $X$ is defined via an atlas $\{(\phi_\alpha, U_\alpha)\}_{\alpha\in I}$ of charts with values in $\mathbb{C}^m$ such that their transition functions are holomorphic. If $\phi = (z_1,\ldots, z_m):U\to V\subset\mathbb{C}^m$ is such a holomorphic chart, then a holomorphic $2$-form $\Omega$ on $X$ can locally be written as:
\begin{gather*}
 \Omega\vert_U = \sum^{m}_{i,j = 1} \Omega_{ij} dz_i\wedge dz_j,
\end{gather*}
where $\Omega_{ij}:U\to\mathbb{C}$ are holomorphic functions on $U$. Similarly to the real case, one can define an exterior derivative $d$ on complex-valued forms such that closedness of $\Omega$ simply equates to $d\Omega = 0$. To understand the non-degeneracy in the definition of a HSM, recall that every complex manifold $X$ implicitly defines an integrable (almost) complex structure $J$ on $X$ and that, further, the complexified tangent and cotangent bundle of $X$, viewed as a real manifold, each decompose into a direct sum of two subbundles:
\begin{gather*}
 T_\mathbb{C}X = T^{(1,0)}X\oplus T^{(0,1)}X;\quad T^\ast_\mathbb{C}X = T^{\ast, (1,0)}X\oplus T^{\ast, (0,1)}X,
\end{gather*}
where the $(1,0)$- and $(0,1)$-bundles are fiberwise eigenspaces of $J$ (or its dual $J^\ast$) with eigenvalue $i$ and $-i$, respectively. By construction, the local forms $dz_i$ are local sections of $T^{\ast, (1,0)}X$ and, hence, map elements of $T^{(0,1)}X$ to zero. This implies that holomorphic $2$-forms can never be non-degenerate on the entire complexified tangent bundle, as they always vanish on the $(0,1)$-bundle. For a holomorphic $2$-form $\Omega$, we can at most achieve non-degeneracy on $T^{(1,0)}X$, i.e.:
\begin{gather*}
 \forall x\in X\ \forall V\in T^{(1,0)}_xX\backslash\{0\}\ \exists W\in T^{(1,0)}_xX:\quad \Omega_x(V, W) \neq 0.
\end{gather*}
In particular, as in the real case, non-degeneracy of $\Omega$ implies that the complex dimension $m$ of $X$ is even. Also note that, by construction, $\Omega$ and $J$ satisfy the following relations:
\begin{gather}\label{eq:J-anticompatible}
 \Omega (J\cdot, \cdot) = \Omega (\cdot, J\cdot) = i\cdot\Omega;\quad \Omega (J\cdot, J\cdot) = -\Omega.
\end{gather}
As in ``real'' symplectic geometry, there are two standard examples of HSMs: the first one is $X = \mathbb{C}^{2n}$ together with the standard form $\Omega = \sum^n_{j = 1} dP_j\wedge dQ_j$, where $(Q_1,\ldots, Q_n, P_1,\ldots, P_n)\in\mathbb{C}^{2n}$. The other one is the holomorphic cotangent bundle $X = T^{\ast, (1,0)}Y$ of a complex manifold $Y$ with canonical $2$-form $\Omega_\text{can} = d\Lambda_\text{can}$, where $\Lambda_\text{can}$ is the holomorphic Liouville $1$-form.\\
We know by Darboux's theorem that symplectic manifolds exhibit no local invariants, since they all are locally isomorphic to the standard symplectic manifold\linebreak $(\mathbb{R}^{2n}, \sum^n_{i = 1} dp_i\wedge dq_i)$. The same statement is true for HSMs:
\begin{Thm}[Darboux's theorem for HSMs]\label{thm:holo_Darboux}
 Let $(X,\Omega)$ be a HSM of complex dimension $\text{\normalfont dim}_\mathbb{C}(X) = 2n$ $(n\in\mathbb{N})$. Then, for every point $x\in X$, there is a holomorphic chart $\psi = (Q_1,\ldots, Q_n, P_1,\ldots, P_n):U\to V\subset\mathbb{C}^{2n}$ of $X$ near $x$ such that
 \begin{gather*}
  \Omega\vert_{U} = \sum\limits^n_{j = 1} dP_j\wedge dQ_j.
 \end{gather*}
\end{Thm}
The correctness of Darboux's theorem for HSMs is widely accepted in the mathematical community, however, no proof has yet been formally written down, at least to the author's extent of knowledge. For completeness' sake, a proof of Darboux's theorem for HSMs is provided in \autoref{app:darboux}. We will make use of Darboux's theorem for HSMs in \autoref{subsec:deforming_HHS}.\\
Now, let us turn our attention to holomorphic Hamiltonian systems:
\begin{Def}[Holomorphic Hamiltonian system]\label{def:holo_ham_sys}
 We call the triple $(X,\Omega, \mathcal{H})$ a \textbf{holomorphic Hamiltonian system} (HHS) iff $(X,\Omega)$ is a HSM and $\mH:X\to\mathbb{C}$ is a holomorphic function on $X$. In this setup, we call $\mH$ the \textbf{Hamilton function} or, simply, the Hamiltonian of the HHS $(X,\Omega, \mH)$.
\end{Def}
Examples of HHSs include the standard HSM $(\mathbb{C}^{2n}, \sum^n_{j = 1} dP_j\wedge dQ_j)$ together with any holomorphic function $\mH:\mathbb{C}^{2n}\to\mathbb{C}$ on it and holomorphic cotangent bundles $(T^{\ast, (1,0)}Y, \Omega_\text{can})$ together with natural Hamiltonians. We call a Hamiltonian on a holomorphic cotangent bundle natural iff it can be written as a sum of kinetic and potential energy, $\mH = \mathcal{T} + \mathcal{V}$. Hereby, we say a holomorphic function $\mathcal{V}:T^{\ast, (1,0)}Y\to\mathbb{C}$ denotes potential energy iff it factors through the holomorphic projection $\pi:T^{\ast, (1,0)}Y\to Y$, i.e., can be written as $\mathcal{V} = \mathcal{V}_0\circ\pi$ for some holomorphic function $\mathcal{V}_0:Y\to \mathbb{C}$. Furthermore, a holomorphic function $\mathcal{T}:T^{\ast, (1,0)}Y\to\mathbb{C}$ is called kinetic energy iff $2\mathcal{T}(x)\equiv g^\ast (x,x)$ $\forall x\in T^{\ast, (1,0)}Y$, where $g^\ast$ is the dualization of some holomorphic metric $g$ on $Y$. A holomorphic metric $g$ on $Y$ is a holomorphic symmetric $\mathbb{C}$-bilinear form which is non-degenerate on $T^{(1,0)}Y$, i.e., for any holomorphic chart $\phi = (z_1,\ldots, z_n):U\to V\subset\mathbb{C}^n$ of $Y$, $g$ can be written as
\begin{gather*}
 g\vert_U = \sum^n_{i,j = 1} g_{ij} dz_i\otimes dz_j,
\end{gather*}
where $g_{ij}:U\to\mathbb{C}$ are holomorphic functions satisfying $g_{ij} = g_{ji}$ and $\text{det}(g_{ij})\neq 0$.
We will study these examples in more detail in the upcoming subsections.


\subsection{Holomorphic Trajectories}
\label{subsec:holo_traj}

Our next goal is to investigate the dynamics of a HHS. As for RHSs, they are determined by a Hamiltonian vector field:
\begin{Def}[Holomorphic Hamiltonian vector field]\label{def:holo_ham_field}
 Let $(X,\Omega, \mH)$ be a HHS. We call the holomorphic vector field $X_\mH$ on $X$ defined by $\iota_{X_\mH}\Omega = -d\mH$ the (holomorphic) \textbf{Hamiltonian vector field} of the HHS $(X,\Omega, \mH)$.
\end{Def}
\begin{Rem}[$X_\mH$ is well-defined]\label{rem:ham_vec_field_well-def}
 Note that a holomorphic vector field $V$ on a complex manifold $X$ can -- in a holomorphic chart $\phi \equiv (z_1,\ldots, z_{m}):U\to V\subset\mathbb{C}^m$ -- be written as
 \begin{gather*}
  V\vert_U = \sum^m_{j = 1} V_j\partial_{z_j},
 \end{gather*}
 where $V_j:U\to\mathbb{C}$ are holomorphic functions on U. Thus, a holomorphic vector field on $X$ only attains values in the bundle $T^{(1,0)}X$. Together with the non-degeneracy of $\Omega$ on $T^{(1,0)}X$, this implies that the Hamiltonian vector field $X_\mH$ is well-defined.
\end{Rem}
Before we define what a trajectory of a HHS is, it is wise to study $X_\mH$ or, better yet, holomorphic vector fields in general. The following proposition is a standard result from complex geometry:
\begin{Prop}[Holomorphic vector fields $\Leftrightarrow$ $J$-preserving vector fields]\label{prop:holo_vec_field_equiv_J_pre_vec_field}
 Let $X$ be a complex manifold with complex structure $J\in\Gamma (\text{\normalfont End}(TX))$. Then, the tangent bundles $TX$\footnote{For $TX$, $X$ is viewed as a real manifold.} and $T^{(1,0)}X$ are isomorphic as smooth complex vector bundles via
 \begin{gather*}
  f:TX\to T^{(1,0)}X,\quad v\mapsto \frac{1}{2}(v - i\cdot J(v)),
 \end{gather*}
 where the fiberwise complex vector space structure of $TX$ is given by the complex multiplication $(a+ib)\odot v\coloneqq av + bJ(v)$ for every $a,b\in\mathbb{R}$ and $v\in TX$.\\
 Now consider the space $\Gamma_J (TX)$ of smooth real $J$-preserving\footnote{$J$-preserving vector fields are called infinitesimal automorphisms in \cite{kobayashi1969}.} vector fields on $X$:
 \begin{gather*}
  \Gamma_J (TX)\coloneqq \{V_R\in\Gamma (TX)\mid L_{V_R}J = 0\},
 \end{gather*}
 where $L_{V_R}J$ is the Lie derivative of $J$ with respect to $V_R$ and $\Gamma (TX)$ is the space of smooth real vector fields on $X$. Then, $\Gamma_J (TX)$ together with the standard commutator $[\cdot, \cdot]$ of vector fields and the complex structure $J$ forms a complex Lie algebra. In fact, $(\Gamma_J (TX), [\cdot, \cdot])$ is isomorphic as complex Lie algebras to the space $(\Gamma (T^{(1,0)}X), [\cdot,\cdot])$ of holomorphic vector fields on $X$ via
 \begin{gather*}
  F:\Gamma_J (TX)\to \Gamma (T^{(1,0)}X),\quad V_R\mapsto \frac{1}{2}(V_R - i\cdot J(V_R)).
 \end{gather*}
\end{Prop}
The proof of Proposition \autoref{prop:holo_vec_field_equiv_J_pre_vec_field} can be found in Chapter IX of \cite{kobayashi1969} (cf. Proposition 2.10 and 2.11). One important consequence of Proposition \autoref{prop:holo_vec_field_equiv_J_pre_vec_field} which we heavily use later on is the fact that the real and imaginary part of a holomorphic vector field commute:
\begin{Cor}[$V_R$ and $J(V_R)$ commute]\label{cor:commute}
 Let $X$ be a complex manifold with complex structure $J$ and $V_R\in\Gamma_J (TX)$ be a $J$-preserving vector field. Then:
 \begin{gather*}
  \left[V_R, J(V_R)\right] = 0.
 \end{gather*}
 In particular, the real and imaginary part of holomorphic vector fields on $X$ commute.
\end{Cor}
\begin{proof}
 By Proposition \autoref{prop:holo_vec_field_equiv_J_pre_vec_field}, we know that $(\Gamma_J (TX), [\cdot, \cdot])$ is a complex Lie algebra, hence:
 \begin{gather*}
  \left[V_R, J(V_R)\right] = J\left([V_R, V_R]\right) = 0.
 \end{gather*}
\end{proof}
By Proposition \autoref{prop:holo_vec_field_equiv_J_pre_vec_field}, we can associate with the Hamiltonian vector field $X_\mH$ of a HHS $(X,\Omega,\mH)$ a $J$-preserving vector field $X^R_\mH$ which is uniquely determined by
\begin{gather*}
 X_\mH = \frac{1}{2}(X^R_\mH - i\cdot J(X^R_\mH)).
\end{gather*}
Equipped with this knowledge, there are now two ways to define holomorphic trajectories of a HHS: the first one is to simply say that holomorphic trajectories are holomorphic integral curves of the holomorphic Hamiltonian vector field $X_\mH$. The second one is to define the holomorphic trajectories as analytic continuations of the integral curves of $X^R_\mH$. Both definitions are indeed equivalent and we make use of both of them. For our purposes, we use the first one as the actual definition and the second one to construct and investigate holomorphic trajectories afterwards:
\begin{Def}[Holomorphic trajectories]\label{def:holo_traj}
 Let $(X,\Omega,\mH)$ be a HHS and\linebreak $X_\mH = 1/2(X^R_\mH - i\cdot J(X^R_\mH))$ be its Hamiltonian vector field. We call a holomorphic map $\gamma:U\to X$ a \textbf{holomorphic trajectory} of the HHS $(X,\Omega, \mH)$ iff $\gamma$ satisfies the holomorphic integral curve equation:
 \begin{gather*}
  \gamma^\prime (z) = X_\mH (\gamma (z))\quad\forall z\in U,
 \end{gather*}
 where $U\subset\mathbb{C}$ is an open and connected subset and $\gamma^\prime$ is the complex derivative of $\gamma$. We call a holomorphic trajectory $\gamma:U\to X$ \textbf{maximal} iff for every holomorphic trajectory $\hat\gamma:\hat U\to X$ with $U\subset \hat U$ and $\hat\gamma\vert_U = \gamma$ one has $\hat U = U$ and $\hat \gamma = \gamma$. Sometimes, we call the integral curves of the real vector field $X^R_\mH$ the \textbf{real trajectories} of the HHS $(X,\Omega,\mH)$.
\end{Def}
Next, let us consider the existence and uniqueness of holomorphic trajectories:%In particular, we want to show that for any given initial value $x_0\in X$ and $z_0\in\mathbb{C}$ a holomorphic trajectory $\gamma^{z_0, x_0}:U\to X$ with $\gamma^{z_0, x_0} (z_0) = x_0$ exists and that two holomorphic trajectories $\gamma^{z_0, x_0}_1:U_1\to X$ and $\gamma^{z_0, x_0}_2:U_2\to X$ for the same initial value are equal iff their domains $U_1$ and $U_2$ are equal:

\begin{Prop}[Existence and uniqueness of holomorphic trajectories]\label{prop:holo_traj}
 Let $(X,\Omega,\mH)$ be a HHS. Then, for any $z_0\in\mathbb{C}$ and $x_0\in X$, there exists an open and connected subset $U\subset\mathbb{C}$ with $z_0\in U$ and a holomorphic trajectory\ $\gamma^{z_0, x_0}:U\to X$ of $(X,\Omega,\mH)$ with $\gamma^{z_0, x_0} (z_0) = x_0$. Two holomorphic trajectories $\gamma^{z_0, x_0}_1:U_1\to X$ and $\gamma^{z_0, x_0}_2:U_2\to X$ with $\gamma^{z_0, x_0}_1 (z_0) = x_0 = \gamma^{z_0, x_0}_2 (z_0)$ locally coincide, in particular, they are equal iff their domains $U_1$ and $U_2$ are equal. Furthermore, the holomorphic trajectory $\gamma^{z_0, x_0}$ depends holomorphically on $z_0$ and $x_0$.
\end{Prop}

\begin{proof}
 Let $(X,\Omega,\mH)$ be a HHS with Hamiltonian vector field $X_\mH = 1/2(X^R_\mH - iJ(X^R_\mH))$ and let $z_0\in\mathbb{C}$ and $x_0\in X$ be any points. To construct a holomorphic trajectory $\gamma^{z_0, x_0}$, we first realize that $t\mapsto\gamma^{z_0, x_0}(t + is)$ for fixed $s\in\mathbb{R}$ is a real trajectory. We can see this by taking the real part of the holomorphic integral curve equation. Thus, finding holomorphic trajectories amounts to finding analytic continuations of real trajectories. To accomplish this task, we observe that similarly $s\mapsto\gamma^{z_0, x_0}(t + is)$ for fixed $t\in\mathbb{R}$ is an integral curve of $J(X^R_\mH)$. %by considering the imaginary part of the holomorphic integral curve equation.
 Naively, one hopes that $\gamma^{z_0, x_0}(t + is)$ is given by:
 \begin{gather*}
   \gamma^{z_0, x_0}(t + is) = \varphi^{J(X^R_\mH)}_{s-s_0}\circ\varphi^{X^R_\mH}_{t-t_0} (x_0),
 \end{gather*}
 where $z_0 = t_0 + is_0$ and $\varphi^{J(X^R_\mH)}_{s-s_0}$ and $\varphi^{X^R_\mH}_{t-t_0}$ are the flows of the vector fields $J(X^R_\mH)$ and $X^R_\mH$ with times $s-s_0$ and $t-t_0$, respectively. In general, however, this expression is problematic: even though it is an integral curve of $J(X^R_\mH)$ for fixed $t$, it might not be an integral curve of $X^R_\mH$ for fixed $s$ anymore due to the composition with $\varphi^{J(X^R_\mH)}_{s-s_0}$. In order to avoid this problem, we need the composition of the flows to commute, at least for small times $t-t_0$ and $s-s_0$. This occurs if the vector fields $X^R_\mH$ and $J(X^R_\mH)$ themselves commute. In our situation, this is indeed the case, as we can use Corollary \autoref{cor:commute} and the fact that $X_\mH$ is a holomorphic vector field. Thus, we can define:
 \begin{gather*}
   \gamma^{z_0, x_0}(t + is)\coloneqq \varphi^{J(X^R_\mH)}_{s-s_0}\circ\varphi^{X^R_\mH}_{t-t_0} (x_0)\equiv \varphi^{X^R_\mH}_{t-t_0}\circ\varphi^{J(X^R_\mH)}_{s-s_0} (x_0)\equiv\varphi^{(t-t_0)X^R_\mH + (s-s_0)J(X^R_\mH)}_1 (x_0).
 \end{gather*}
 The expressions above are well-defined for $|t-t_0|,|s-s_0|<\varepsilon$ with $\varepsilon$ small enough and all identical due to the commutativity of $X^R_\mH$ and $J(X^R_\mH)$.\\
 Let us check that the given expressions for $\gamma^{z_0, x_0}$ indeed define a holomorphic trajectory. By construction, the map $\gamma^{z_0, x_0}$ is holomorphic, as it satisfies the Cauchy-Riemann equations. Hence, we only need to compute the complex derivative $\gamma^{z_0, x_0\ \prime}$.\linebreak If $\phi = (z_1,\ldots, z_{2n}):V\to W\subset\mathbb{C}^{2n}$ is a holomorphic chart of $X$ near $x_0$,\linebreak then we can define the complex derivative $\gamma^{z_0, x_0\ \prime}(z)$ for suitable $z$\linebreak using $(\gamma^{z_0, x_0}_1(z),\ldots, \gamma^{z_0, x_0}_{2n}(z))\coloneqq \phi\circ\gamma^{z_0, x_0} (z)$:
 \begin{gather*}
 \gamma^{z_0, x_0\ \prime} (z)\coloneqq\sum^{2n}_{j = 1} \gamma^{z_0, x_0\ \prime}_j (z)\cdot\left.\pa_{z_j}\right\vert_{\gamma^{z_0, x_0} (z)},
 \end{gather*}
 where $\gamma^{z_0, x_0\ \prime}_j (z)$ is the usual complex derivative of a holomorphic map from $\mathbb{C}$ to $\mathbb{C}$. A straightforward calculation reveals that the complex derivative $\gamma^{z_0, x_0\ \prime}(z)$ equates to:
 \begin{gather*}
  \gamma^{z_0, x_0\ \prime}(z) = \frac{1}{2}\left(\frac{\pa \gamma^{z_0, x_0}}{\pa t}(z) - i\cdot\frac{\pa \gamma^{z_0, x_0}}{\pa s}(z)\right).
 \end{gather*}
 By definition of $\gamma^{z_0, x_0}$, we have:
 \begin{gather*}
  \frac{\pa \gamma^{z_0, x_0}}{\pa t}(z) = X^R_\mH (\gamma^{z_0, x_0}(z));\quad \frac{\pa \gamma^{z_0, x_0}}{\pa s}(z) = J\left(X^R_\mH (\gamma^{z_0, x_0}(z))\right).
 \end{gather*}
 Putting everything together gives:
 \begin{gather*}
  \gamma^{z_0, x_0\ \prime}(z) = \frac{1}{2}\left(X^R_\mH (\gamma^{z_0, x_0}(z)) - i\cdot J\left(X^R_\mH (\gamma^{z_0, x_0}(z))\right)\right) = X_\mH (\gamma^{z_0, x_0}(z)).
 \end{gather*}
 Thus, $\gamma^{z_0, x_0}$ is indeed a holomorphic trajectory. Clearly, $\gamma^{z_0, x_0}$ satisfies $\gamma^{z_0, x_0}(z_0) = x_0$ proving the existence in Proposition \autoref{prop:holo_traj}.\\
 To show local uniqueness given an initial value, we recall that $\gamma^{z_0, x_0}$ is just an integral curve of $X^R_\mH$ along the $t$-axis satisfying $\gamma^{z_0, x_0}(z_0) = x_0$. Hence, every other holomorphic trajectory $\hat\gamma^{z_0, x_0}$ with $\hat\gamma^{z_0, x_0}(z_0) = x_0$ agrees with $\gamma^{z_0, x_0}$ for $s = s_0$ and $t$ near $t_0$. This allows us to apply the identity theorem for holomorphic functions to the coordinates of $\gamma^{z_0, x_0}$ and $\hat\gamma^{z_0, x_0}$ in a holomorphic chart near $x_0$ giving us the local uniqueness. In order to show that $\gamma^{z_0, x_0}$ and $\hat\gamma^{z_0, x_0}$ coincide completely iff their domains are equal, we cover the images of $\gamma^{z_0, x_0}$ and $\hat\gamma^{z_0, x_0}$ with holomorphic charts and repeatedly apply the identity theorem.\\
 Lastly, we need to show that $\gamma^{z_0, x_0}$ depends analytically on $z_0$ and $x_0$. For $z_0$, this is trivial, since $\gamma^{z_1, x_0}(z)$ and $\gamma^{z_2, x_0}(z)$ for $z_1\neq z_2$ only differ by a translation in $z$. For $x_0$, this is true if and only if the flows $\varphi^{X^R_\mH}_{t-t_0}$ and $\varphi^{J(X^R_\mH)}_{s-s_0}$ of $X^R_\mH$ and $J(X^R_\mH)$ are holomorphic maps from and to $X$. As explained in Chapter IX of \cite{kobayashi1969}, the $J$-preserving vector fields on $X$ are exactly those real vector fields on $X$ whose flow is holomorphic. Remembering that, by Proposition \autoref{prop:holo_vec_field_equiv_J_pre_vec_field}, the vector fields $X^R_\mH$ and $J(X^R_\mH)$ are $J$-preserving concludes the proof.
\end{proof}

\begin{Rem}\label{rem:holo_traj_alpha}
 In the last proof, we have used that a holomorphic trajectory $\gamma (t+is)$ of a HHS $(X,\Omega,\mH)$ is an integral curve of $X^R_\mH$ for fixed $s$ and an integral curve of $J(X^R_\mH)$ for fixed $t$. We can generalize this observation. If we express $t+is$ in polar coordinates, $t+is = re^{i\alpha}$, then $\gamma (re^{i\alpha})$ is an integral curve of $\cos(\alpha) X^R_\mH + \sin (\alpha)J(X^R_\mH)$ for fixed $\alpha$.
\end{Rem}

The properties we have found so far seem to indicate that holomorphic trajectories of a HHS exhibit the same behavior as trajectories of a RHS. However, this is not entirely true. In sharp contrast to the real case, the maximal holomorphic trajectories, given an initial value, do \underline{not} need to be unique, as the following counterexample demonstrates.

\begin{Ex}[Central problem in one complex dimension]\label{ex:holo_cen_prob}\normalfont
 {\textcolor{white}{Easter Egg}}\linebreak Let $X\coloneqq T^{\ast, (1,0)}\mathbb{C}^\times\cong\mathbb{C}^\times\times\mathbb{C}$ be the holomorphic cotangent bundle of $\mathbb{C}^\times\coloneqq\mathbb{C}\backslash\{0\}$ together with the standard form $\Omega = \Omega_\text{can} = dP\wedge dQ$, $(Q,P)\in X$, and the natural Hamiltonian\footnote{The given Hamiltonian is even regular, i.e., $d\mH\neq 0$ for all points of $X$.} $\mH (Q,P)\coloneqq \frac{P^2}{2} - \frac{1}{8Q^2}$. Physically speaking, the HHS $(X,\Omega, \mH)$ is the complexification of the RHS describing a single particle in one-dimensional position space subject to the almost Kepler-like central potential $V(q) = -\frac{1}{8q^2}$. The Hamiltonian vector field $X_\mH$ of the HHS $(X,\Omega,\mH)$ is given by
 \begin{gather*}
  X_\mH (Q,P) = P\cdot\pa_Q - \frac{1}{4Q^3}\cdot\pa_P.
 \end{gather*}
 Hence, the holomorphic trajectories $\gamma (z) = (Q(z), P(z))$ of $(X,\Omega, \mH)$ satisfy
 \begin{gather*}
  Q^\prime (z) = P (z);\quad P^\prime (z) = -\frac{1}{4Q^3 (z)}
 \end{gather*}
 or, combining both equations:
 \begin{gather*}
  Q^{\prime\prime} (z) = -\frac{1}{4Q^3 (z)}.
 \end{gather*}
 We want to determine the holomorphic trajectories $\gamma$ satisfying $\gamma (z_0) = x_0 = (Q_0,P_0)$ for $z_0, P_0\in\mathbb{C}$ and $Q_0\in\mathbb{C}^\times$. After translation in $z$, we can assume $z_0 = 0$. A straightforward computation reveals that, locally, the desired solutions are given by
 \begin{gather*}
  Q(z) = \sqrt{Q^2_0 + 2Q_0 P_0\cdot z + 2E_0\cdot z^2};\quad P(z) = Q^\prime (z),
 \end{gather*}
 where $E_0\coloneqq \mH (Q_0, P_0)$ and $\sqrt{\cdot}$ is chosen such that $\sqrt{Q^2_0} = Q_0$. Two square roots mapping $Q^2_0$ to $Q_0$ coincide on a small neighborhood of $Q^2_0$, however, they do not need to have the same domain. Let us make this precise by choosing values for $Q_0$ and $P_0$. Pick $Q_0 = 1$ and $P_0 = \frac{1}{2}$. Then, $E_0 = 0$ and $Q(z) = \sqrt{z + 1}$. Here, all square roots are admissible that coincide with the standard square root for real positive numbers. For instance, one can choose
 \begin{gather*}
  \sqrt{\cdot}^1:\{z\in\mathbb{C}\mid \text{Im}(z) \neq 0\text{ or }\text{Re}(z) > 0\}\to\mathbb{C},\quad z = re^{i\alpha}\mapsto \sqrt{r}e^{\frac{i\alpha}{2}}\ \left(\alpha \in (-\pi, \pi)\right)
 \end{gather*}
 or
 \begin{gather*}
  \sqrt{\cdot}^2:\{z\in\mathbb{C}\mid \text{Re}(z) \neq 0\text{ or }\text{Im}(z) > 0\}\to\mathbb{C},\quad z = re^{i\alpha}\mapsto \sqrt{r}e^{\frac{i\alpha}{2}}\ \left(\alpha \in \left(-\frac{\pi}{2}, \frac{3\pi}{2}\right)\right).
 \end{gather*}
 Using these square roots, we can define the holomorphic trajectories $\gamma_1: U_1\to X$ and $\gamma_2: U_2\to X$ given by
 \begin{align*}
  Q_1:U_1\coloneqq\{z\in\mathbb{C}\mid \text{Im}(z+1) \neq 0\text{ or }\text{Re}(z+1) > 0\}\to\mathbb{C},\quad Q_1(z)&\coloneqq \sqrt{z + 1}^1,\\
  Q_2:U_2\coloneqq\{z\in\mathbb{C}\mid \text{Re}(z+1) \neq 0\text{ or }\text{Im}(z+1) > 0\}\to\mathbb{C},\quad Q_2(z)&\coloneqq \sqrt{z + 1}^2.
 \end{align*}
 Clearly, $\gamma_1$ and $\gamma_2$ are maximal holomorphic trajectories satisfying $\gamma_1 (0) = (1,\frac{1}{2}) = \gamma_2 (0)$. However, their domains $U_1$ and $U_2$ differ showing that maximal trajectories are not unique, even given an initial value. In particular, the trajectories $\gamma_1$ and $\gamma_2$ yield different values for $z\in\{z\in\mathbb{C}\mid\text{Re}(z+1)<0\text{ and }\text{Im}(z+1)<0\}\subset U_1\cap U_2$, namely $\gamma_1 (z) = -\gamma_2 (z)$.
\end{Ex}

Even though the square root, which spoils the uniqueness of maximal trajectories in the previous example, is not well-defined on all of $\mathbb{C}\backslash\{0\}$, it is well-defined on the $2:1$ covering $z\mapsto z^2$ of $\mathbb{C}\backslash\{0\}$. In the same vein, the maximal trajectories of a HHS become unique after ``passing them down to a covering''. To understand this idea, recall that, for a RHS $(M,\omega, H)$, the energy hypersurface $H^{-1}(E)$ is foliated by maximal trajectories of $(M,\omega, H)$ if $E$ is a regular value of $H$. Similarly, there exists a foliation for regular energy hypersurfaces of a HHS $(X,\Omega, \mH)$. However, the leaves of the foliation are ``more than just'' the maximal trajectories this time:

\newpage

\begin{Prop}[Holomorphic foliation of a regular hypersurface]\label{prop:holo_foli}
 Let $(X,\Omega,\mH)$ be a HHS with complex structure $J$, Hamiltonian vector field $X_\mH = 1/2 (X^R_\mH - i J(X^R_\mH))$, and regular value $E$ of $\mH$. Then, the energy hypersurface $\mH^{-1}(E)$ admits a holomorphic\footnote{If the reader is unfamiliar with the notion of a holomorphic foliation, confer the proof of Proposition \autoref{prop:pseudo-holo_foli} for its definition.} foliation. The leaf $L_{x_0}$ of this foliation through a point $x_0\in\mH^{-1}(E)$ is given by
 \begin{align*}
  L_{x_0}\coloneqq \{y\in X\mid &y = \varphi^{X^R_\mH}_{t_1}\circ\varphi^{J(X^R_\mH)}_{s_1}\circ\varphi^{X^R_\mH}_{t_2}\circ\varphi^{J(X^R_\mH)}_{s_2}\circ\ldots\circ\varphi^{X^R_\mH}_{t_n}\circ\varphi^{J(X^R_\mH)}_{s_n} (x_0);\\
  &t_1,\ldots,t_n, s_1,\ldots, s_n\in\mathbb{R};\ n\in\mathbb{N}\},
 \end{align*}
 where $\varphi^{X^R_\mH}_{t_j}$ and $\varphi^{J(X^R_\mH)}_{s_j}$ are the flows of $X^R_\mH$ and $J(X^R_\mH)$ for time $t_j$ and $s_j$, respectively. Every holomorphic trajectory of $(X,\Omega,\mH)$ with energy $E$ is completely contained in one such leaf.
\end{Prop}

\begin{proof}
 Take the assumptions and notations from above. $E$ is a regular value of $\mH$, hence, $\mH^{-1}(E)$ is a complex submanifold of $X$. The (real) tangent space of $\mH^{-1} (E)$ consists of vectors $W$ in the (real) tangent space of $X$ satisfying $d\mH (W) = 0$. Using the holomorphicity of $\mH$, $d\mH\circ J = i\cdot d\mH$, we obtain
 \begin{align*}
  d\mH (X^R_\mH) &= d\mH (X_\mH) = -\Omega (X_\mH, X_\mH) = 0,\\
  d\mH (J(X^R_\mH)) &= i\cdot d\mH (X^R_\mH) = 0
 \end{align*}
 showing that $X^R_\mH$ and $J(X^R_\mH)$ live in the tangent space of $\mH^{-1}(E)$ at each point of $\mH^{-1}(E)$. This allows us to restrict $X^R_\mH$ and $J(X^R_\mH)$ to real vector fields on $\mH^{-1}(E)$.\\
 As $\mH$ is regular on $\mH^{-1} (E)$, neither $X^R_\mH$ nor $J(X^R_\mH)$ vanish on $\mH^{-1} (E)$. Furthermore, as real vector fields, they are $\mathbb{R}$-linearly independent at every point of $\mH^{-1}(E)$, since there is no real number which squares to $-1$. By Corollary \autoref{cor:commute}, $X^R_\mH$ and $J(X^R_\mH)$ also commute. This allows us to apply Frobenius' theorem to the distribution spanned by the real vector real fields $X^R_\mH$ and $J(X^R_\mH)$ giving us a foliation of $\mH^{-1}(E)$ whose leaves take the form described in Proposition \autoref{prop:holo_foli}.
 %As the vector space spanned by $X^R_\mH$ and $J(X^R_\mH)$ at each point of $\mH^{-1}(E)$ is closed under $J$, this foliation is $J$-holomorphic.
 As $X^R_\mH$ and $J(X^R_\mH)$, the vector fields generating the foliation, are real and imaginary part of a holomorphic vector field, the foliation itself is holomorphic by the holomorphic version of Frobenius' theorem (cf. Theorem 2.26 in \cite{voisin2002}). Comparing the construction of holomorphic trajectories in the proof of Proposition \autoref{prop:holo_traj} with the form of the leaves in Proposition \autoref{prop:holo_foli} reveals that a holomorphic trajectory is completely contained in one leaf concluding the proof.
\end{proof}

\begin{Rem}[Flows of $X^R_\mH$ and $J(X^R_\mH)$ do not commute globally]\label{rem:flows_do_not_commute}
 One might be tempted to set $n$ in the definition of the leaves in Proposition \autoref{prop:holo_foli} to $1$, since $X^R_\mH$ and $J(X^R_\mH)$\linebreak as well as their flows commute. However, this is only locally the case.\linebreak To illustrate this, consider Example \autoref{ex:holo_cen_prob} again. Choose the initial value\linebreak $x_0 = (Q_0, P_0) = (1, 1/2)$ and set $n=2$,  $t_1 = 0$, $s_1 = -2$, $t_2 = -2$, and $s_2 = 1$:
 \begin{gather*}
  \varphi^{X^R_\mH}_{0}\circ\varphi^{J(X^R_\mH)}_{-2}\circ\varphi^{X^R_\mH}_{-2}\circ\varphi^{J(X^R_\mH)}_{1} \left(1, \frac{1}{2}\right) = \left(\sqrt{2}\cdot e^{i\frac{5\pi}{8}}, \frac{1}{2\sqrt{2}}\cdot e^{-i\frac{5\pi}{8}}\right).
 \end{gather*}
 If we exchange the order of $s_1$ and $s_2$, the result is different:
 \begin{gather*}
  \varphi^{X^R_\mH}_{0}\circ\varphi^{J(X^R_\mH)}_{1}\circ\varphi^{X^R_\mH}_{-2}\circ\varphi^{J(X^R_\mH)}_{-2} \left(1, \frac{1}{2}\right) = -\left(\sqrt{2}\cdot e^{i\frac{5\pi}{8}}, \frac{1}{2\sqrt{2}}\cdot e^{-i\frac{5\pi}{8}}\right).
 \end{gather*}
\end{Rem}

In light of Proposition \autoref{prop:holo_foli}, one might say that the leaves of a HHS $(X,\Omega, \mH)$ should be considered to be the holomorphic counterpart to the maximal trajectories of a RHS $(M,\omega, H)$. Let us investigate this statement further. To do that, we first need to generalize the notion of holomorphic trajectories in such a way that we can view any Riemann surface as the domain of a trajectory, not only subsets of $\mathbb{C}$:

\begin{Def}[Geometric trajectory]\label{def:geo_traj}
 Let $(X,\Omega,\mH)$ be a HHS with regular value $E$ of $\mH$ and foliation $L = \{L_{x_0}\}_{x_0\in I}$ ($I$: some index set) of $\mH^{-1}(E)$ as in Proposition \autoref{prop:holo_foli}. Further, let $\Sigma$ be a Riemann surface, i.e., a connected, complex one-dimensional manifold. We call a holomorphic map $\gamma:\Sigma\to X$ a \textbf{geometric trajectory} of energy $E$ iff $\gamma$ is an immersion and the image of $\gamma$ is completely contained in one leaf $L_{x_0}$ of the foliation $L$.
\end{Def}

The definition of a geometric trajectory is reasonable, as every geometric trajectory is locally a holomorphic trajectory:

\begin{Prop}[Geometric trajectories are locally holomorphic trajectories]\label{prop:geo_traj}
 Let\linebreak $(X,\Omega,\mH)$ be a HHS with regular value $E$ of $\mH$, $\Sigma$ be a Riemann surface, and $\gamma:\Sigma\to X$ be a geometric trajectory of energy $E$. Then, for every $s_0\in\Sigma$, there exists an open neighborhood $V\subset\Sigma$ of $s_0$ and a holomorphic chart $\varphi:V\to U\subset\mathbb{C}$ of $\Sigma$ such that $\gamma\circ\varphi^{-1}:U\to X$ is a holomorphic trajectory of the HHS $(X,\Omega,\mH)$.
\end{Prop}

\begin{proof}
 Invoke the assumptions and notations from above. As $\gamma$ is a holomorphic immersion whose image is completely contained in one leaf and the leaves of $L$ are generated by the Hamiltonian vector field $X_\mH$, there exists for every $s\in\Sigma$ an uniquely determined vector $Y_\mH (s)\in T^{(1,0)}_s\Sigma$ such that:
 \begin{gather*}
  d\gamma\vert_s (Y_\mH (s)) = X_\mH (\gamma (s)).
 \end{gather*}
 These vectors form a holomorphic vector field $Y_\mH$ on $\Sigma$. Now pick $s_0\in\Sigma$ and $z_0\in\mathbb{C}$. By Proposition \autoref{prop:holo_traj} and Remark \autoref{rem:holo_vec_fields}, there exists an open and connected subset $U\subset\mathbb{C}$ such that $\varphi^{-1}:U\to\Sigma$ is a holomorphic integral curve of $Y_\mH$ satisfying $\varphi^{-1}(z_0) = s_0$. After shrinking $U$ if necessary, $\varphi^{-1}$ becomes a biholomorphism onto its image $\varphi^{-1}(U)\eqqcolon V$. Thus, $\varphi:V\to U$ is a holomorphic chart of $\Sigma$ near $s_0$. Furthermore, the curve $\gamma\circ\varphi^{-1}:U\to X$ is an integral curve of $X_\mH$:
 \begin{gather*}
  \left(\gamma\circ\varphi^{-1}\right)^\prime (z) = d\gamma\vert_{\varphi^{-1}(z)}\left(\varphi^{-1\,\prime} (z)\right) = d\gamma\vert_{\varphi^{-1}(z)}\left(Y_\mH(\varphi^{-1} (z))\right) = X_\mH (\gamma\circ\varphi^{-1}(z))\quad\forall z\in U,
 \end{gather*}
 hence, a holomorphic trajectory concluding the proof.
\end{proof}

Let us now assume that $(X,\Omega,\mH)$ is a HHS with regular value $E$ of $\mH$ and foliation $L = \{L_{x_0}\}_{x_0\in I}$ of $\mH^{-1}(E)$. Pick a leaf $L_{x_0}$. If $L_{x_0}$ is a complex submanifold of $X$, the inclusion $L_{x_0}\hookrightarrow X$ is clearly a geometric trajectory. If $L_{x_0}$ is not a complex submanifold of $X$, we can always equip $L_{x_0}$ with the structure of a complex manifold by choosing a suitable atlas such that the inclusion $L_{x_0}\hookrightarrow X$ becomes a geometric trajectory. The atlas in question consists of maps $\gamma^{-1}$ where $\gamma:U\to X$ is an injective holomorphic trajectory whose image is contained in $L_{x_0}$. In contrast to maximal trajectories, the geometric trajectories $L_{x_0}\hookrightarrow X$ are unique given an initial value $x_0\in \mH^{-1}(E)\subset X$ such that $x_0\in L_{x_0}$. In fact, the uniqueness can be expressed as a universal property: pick $x_0\in\mH^{-1}(E)$. Then, every geometric trajectory $\gamma:\Sigma\to X$ with initial value $x_0\in\gamma (\Sigma)$ factors uniquely through the geometric trajectory $L_{x_0}\hookrightarrow X$ and the geometric trajectory $L_{x_0}\hookrightarrow X$ is unique up to biholomorphisms with that property.\\
Often, the geometric trajectories $L_{x_0}\hookrightarrow X$ can be understood as coverings of or to be covered by maximal trajectories. Example \autoref{ex:holo_cen_prob} exemplifies this behavior. To see this, we first need to determine the leaves in Example \autoref{ex:holo_cen_prob}. We achieve this by applying the following proposition:

\begin{Prop}[Energy hypersurfaces of low-dimensional systems]\label{prop:low_dim}
 Let $(X,\Omega,\mH)$ be a HHS with $\text{\normalfont dim}_\mathbb{C}(X) = 2$ and regular value $E$ of $\mH$. Then, the leaves of $\mH^{-1} (E)$ are its connected components and, in particular, complex submanifolds of $X$.
\end{Prop}

\begin{proof}
 Take the notations and assumptions from above. $X$ is a complex two-dimensional manifold, hence, $\mH^{-1}(E)$ is a complex one-dimensional one. Likewise, the leaves of $\mH^{-1}(E)$ are one-dimensional complex manifolds, immersed in $\mH^{-1}(E)$. Thus, the leaves are open in $\mH^{-1}(E)$. By definition of a foliation, $\mH^{-1}(E)$ decomposes into a disjoint union of leaves. Therefore, the leaves are also closed in $\mH^{-1}(E)$ concluding the proof.
\end{proof}

Return to Example \autoref{ex:holo_cen_prob}. By Proposition \autoref{prop:low_dim}, the leaves in this example are just the connected components of the energy hypersurfaces $\mH^{-1}(E)$. For $E\neq 0$, the energy hypersurface is connected and only consists of one leaf, namely itself. For $E = 0$, we have:
\begin{gather*}
 \mH(Q,P) = \frac{P^2}{2} - \frac{1}{8Q^2} = \frac{1}{2}\left(P-\frac{1}{2Q}\right)\left(P+\frac{1}{2Q}\right) \stackrel{!}{=} 0.
\end{gather*}
We see that there are two connected components and, consequentially, two leaves this time: one with $Q\cdot P = \frac{1}{2}$ and another one with $Q\cdot P = -\frac{1}{2}$. We have already determined the maximal trajectories of the leaf with $Q\cdot P = \frac{1}{2}$: they are given by $Q (z) = \sqrt{z+1}$ and $P = Q^\prime (z)$ with appropriate square roots. Precomposing the maximal trajectories with the $2:1$ covering $z\mapsto z^2 -1$ gives us a well-defined map from $\mathbb{C}\backslash\{0\}$ to the leaf with $Q\cdot P = \frac{1}{2}$. In fact, this map is a biholomorphism. The leaf can be understood in this sense as the double cover of the maximal trajectories. For an example where a leaf is covered by a maximal trajectory, confer Example \autoref{ex:complex_torus}.\\
Before we conclude the section on holomorphic trajectories, we quickly add two comments. The first one concerns the notion of holomorphic and geometric trajectories: throughout the rest of this paper, we will not distinguish between holomorphic and geometric trajectories anymore and use both names interchangeably. The second one concerns holomorphic vector fields on general complex manifolds: 

\begin{Rem}\label{rem:holo_vec_fields}
 The previous results are not only true for holomorphic Hamiltonian vector fields and holomorphic trajectories of HHSs, but also for holomorphic vector fields and holomorphic integral curves on any complex manifold.
\end{Rem}


\subsection{Action Functionals and Principles for HHSs}
\label{subsec:holo_action_fun_and_prin}

As in the real case, we wish to link the holomorphic trajectories of a HHS $(X,\Omega,\mH)$ to critical points of some action functional. To achieve this, let us first study the holomorphic symplectic $2$-form $\Omega$ on $X$. By definition, $\Omega$ is non-degenerate on $T^{(1,0)}X$, but vanishes on $T^{(0,1)}X$. Every complex form $\Omega$ together with complex vectors $V$ and $W$ satisfies $\overline{\Omega (V,W)} = \overline{\Omega} (\overline{V},\overline{W})$, hence, the complex conjugate $\overline{\Omega}$ of the holomorphic symplectic form $\Omega$ is non-degenerate on $T^{(0,1)}X$, but vanishes on $T^{(1,0)}X$\footnote{Here, we have also used that the complex conjugation maps $T^{(1,0)}X$ to $T^{(0,1)}X$ and $T^{(0,1)}X$ to $T^{(1,0)}X$.}. In total, this implies that the real and imaginary part of $\Omega$,
\begin{gather*}
 \Omega_R = \frac{1}{2}(\Omega + \overline{\Omega})\quad\text{and}\quad\Omega_I = \frac{-i}{2}(\Omega - \overline{\Omega}),
\end{gather*}
are non-degenerate on the entire complexified tangent bundle $T_\mathbb{C}X = T^{(1,0)}X\oplus T^{(0,1)}X$ of $X$. Clearly, $\Omega_R$ and $\Omega_I$ are real $2$-forms on $X$ and, hence, must be already non-degenerate on the real tangent bundle $TX$. As $\Omega$ is holomorphic and closed, $\Omega_R$ and $\Omega_I$ are smooth and closed. Putting everything together, we find that the real and imaginary part $\Omega_R$ and $\Omega_I$ of $\Omega$, respectively, are symplectic $2$-forms on $X$ viewed as a real manifold.\\
Let us return to the HHS $(X,\Omega, \mH)$. Clearly, the real and imaginary part of the Hamiltonian $\mH = \mH_R + i\mH_I$ are smooth real functions on $X$. Thus, any HHS $(X,\Omega,\mH)$ gives rise to four \underline{RHSs}: $(X,\Omega_R,\mH_R)$, $(X,\Omega_I,\mH_R)$, $(X,\Omega_R,\mH_I)$, and $(X,\Omega_I,\mH_I)$\footnote{As we will see later on, it might be more appropriate to say that $(X,\Omega,\mH)$ gives rise to only two RHSs, since $(X,\Omega_R,\mH_R)$ and $(X,\Omega_I,\mH_I)$ as well as $(X,\Omega_I,\mH_R)$ and $(X,\Omega_R,-\mH_I)$ are subject to the same dynamics.}. Our next task is to determine the Hamiltonian vector fields of the four RHSs.\\
We start with the RHS $(X,\Omega_R,\mH_R)$. We write the holomorphic Hamiltonian vector field $X_\mH$ as $X_\mH = 1/2 (X^R_\mH - iJ(X^R_\mH))$ and compute $\iota_{X^R_\mH}\Omega_R$:
\begin{align*}
 \iota_{X^R_\mH}\Omega_R &= \frac{1}{2}\left(\iota_{X^R_\mH}\Omega + \iota_{X^R_\mH}\overline{\Omega}\right) = \frac{1}{2}\left(\iota_{X^R_\mH}\Omega + \iota_{\overline{X^R_\mH}}\overline{\Omega}\right) = \frac{1}{2}\left(\iota_{X^R_\mH}\Omega + \overline{\iota_{X^R_\mH}\Omega}\right)\\
 &= \frac{1}{2}\left(\iota_{X_\mH}\Omega + \overline{\iota_{X_\mH}\Omega}\right) = -\frac{1}{2}\left(d\mH + \overline{d\mH}\right) = -\frac{1}{2}d\left(\mH + \overline{\mH}\right) = - d\mH_R,
\end{align*}
where we used that $X^R_\mH$ is a real vector field on $X$, i.e., $\overline{X^R_\mH} = X^R_\mH$, and that\linebreak $\iota_{X_\mH}\Omega = \iota_{X^R_\mH}\Omega$ due to Equation \eqref{eq:J-anticompatible}. We deduce from the expression above that $X^R_\mH$ is the real Hamiltonian vector field of the RHS $(X,\Omega_R,\mH_R)$. Similarly, one can show the following proposition:

\begin{Prop}[HHS as four RHSs]\label{prop:four_RHS}
 Let $(X,\Omega = \Omega_R + i\Omega_I,\mH = \mH_R + i\mH_I)$ be a HHS with Hamiltonian vector field $X_\mH = 1/2 (X^R_\mH - iJ(X^R_\mH))$. Then:
 \begin{enumerate}
  \item $(X,\Omega_R, \mH_R)$ is a RHS with Hamiltonian vector field $X^R_\mH$.
  \item $(X,\Omega_R, \mH_I)$ is a RHS with Hamiltonian vector field $-J(X^R_\mH)$.
  \item $(X,\Omega_I, \mH_R)$ is a RHS with Hamiltonian vector field $J(X^R_\mH)$.
  \item $(X,\Omega_I, \mH_I)$ is a RHS with Hamiltonian vector field $X^R_\mH$.
 \end{enumerate}
\end{Prop}

\begin{Rem}[Cauchy-Riemann-like relations]\label{rem:cauchy-riemann}
 At first glance, one might be confused why the Hamiltonian vector fields of $(X,\Omega_R,\mH_R)$ and $(X,\Omega_I,\mH_I)$ coincide, while the Hamiltonian vector fields of $(X,\Omega_R,\mH_I)$ and $(X,\Omega_I,\mH_R)$ differ by a sign. However, this observation is just a consequence of the analyticity of the HHS $(X,\Omega,\mH)$ and one might think of it as Cauchy-Riemann-like relations:
 \begin{gather*}
  X^{\Omega_R}_{\mH_R} = X^{\Omega_I}_{\mH_I};\quad J\left(X^{\Omega_R}_{\mH_R}\right) = X^{\Omega_I}_{\mH_R} = - X^{\Omega_R}_{\mH_I},
 \end{gather*}
 where $X^{\Omega_a}_{\mH_b}$ is the Hamiltonian vector field of the RHS $(X,\Omega_a,\mH_b)$.
\end{Rem}

The upshot of Proposition \autoref{prop:four_RHS} is that the real trajectories of the HHS $(X,\Omega,\mH)$ are just the trajectories of the RHSs $(X,\Omega_R,\mH_R)$ and $(X,\Omega_I,\mH_I)$. In particular, the real trajectories of $(X,\Omega,\mH)$ are critical points of an action functional, at least if\linebreak $(X,\Omega = d\Lambda, \mH)$ is exact\footnote{A HHS $(X,\Omega,\mH)$ is exact iff $\Omega$ has a holomorphic primitive $\Lambda$.}. Of course, the same is true for the trajectories of $(X, \Omega_R, \mH_I)$ and $(X, \Omega_I, \mH_R)$:

\begin{Prop}[Action principle for real trajectories]\label{prop:action_fct_of_real_traj}
 Let $(X,\Omega = d\Lambda, \mH)$ be an exact HHS with Hamiltonian vector field $X_\mH = 1/2 (X^R_\mH - iJ(X^R_\mH))$ and decompositions\linebreak $\Omega = \Omega_R + i\Omega_I$, $\Lambda = \Lambda_R + i\Lambda_I$, and $\mH = \mH_R + i\mH_I$. Let $I_0\subset\mathbb{R}$ be an interval. We set $\mathcal{P}_{I_0}\coloneqq C^\infty (I_0,X)$ and define the action functionals $\mathcal{A}^{\Lambda_R}_{\mH_R}:\mathcal{P}_{I_0}\to\mathbb{R}$, $\mathcal{A}^{\Lambda_R}_{-\mH_I}:\mathcal{P}_{I_0}\to\mathbb{R}$\footnote{Note the different signs in the definition of $\mathcal{A}^{\Lambda_R}_{-\mH_I}$ due to the Cauchy-Riemann-like relations.}, $\mathcal{A}^{\Lambda_I}_{\mH_R}:\mathcal{P}_{I_0}\to\mathbb{R}$, $\mathcal{A}^{\Lambda_I}_{\mH_I}:\mathcal{P}_{I_0}\to\mathbb{R}$, $\mathcal{A}^{\Lambda}_{\mH}:\mathcal{P}_{I_0}\to\mathbb{C}$, and $\mathcal{A}^{\Lambda}_{i\mH}:\mathcal{P}_{I_0}\to\mathbb{C}$ by
 \begin{align*}
  \mathcal{A}^{\Lambda_R}_{\mH_R}[\gamma]&\coloneqq \int\limits_{I_0}\gamma^\ast\Lambda_R - \int\limits_{I_0}\mH_R\circ\gamma (t)dt,\quad \mathcal{A}^{\Lambda_R}_{-\mH_I}[\gamma]\coloneqq \int\limits_{I_0}\gamma^\ast\Lambda_R + \int\limits_{I_0}\mH_I\circ\gamma (t)dt,\\
  \mathcal{A}^{\Lambda_I}_{\mH_R}[\gamma]&\coloneqq \int\limits_{I_0}\gamma^\ast\Lambda_I - \int\limits_{I_0}\mH_R\circ\gamma (t)dt,\quad \mathcal{A}^{\Lambda_I}_{\mH_I}[\gamma]\coloneqq \int\limits_{I_0}\gamma^\ast\Lambda_I - \int\limits_{I_0}\mH_I\circ\gamma (t)dt,\\
  \mathcal{A}^{\Lambda}_{\mH}[\gamma]&\coloneqq \mathcal{A}^{\Lambda_R}_{\mH_R}[\gamma] + i\mathcal{A}^{\Lambda_I}_{\mH_I}[\gamma] = \int\limits_{I_0}\gamma^\ast\Lambda - \int\limits_{I_0}\mH\circ\gamma (t)dt,\\
  \mathcal{A}^{\Lambda}_{i\mH}[\gamma]&\coloneqq \mathcal{A}^{\Lambda_R}_{-\mH_I}[\gamma] + i\mathcal{A}^{\Lambda_I}_{\mH_R}[\gamma] = \int\limits_{I_0}\gamma^\ast\Lambda - i\int\limits_{I_0}\mH\circ\gamma (t)dt,
 \end{align*}
 where $\gamma\in\mathcal{P}_{I_0}$. Now, let $\gamma\in\mathcal{P}_{I_0}$ be any smooth path in $X$. Then, $\gamma$ is a real trajectory of $(X,\Omega,\mH)$ if and only if $\gamma$ is a ``critical point'' of the action functionals $\mathcal{A}^{\Lambda_R}_{\mH_R}$, $\mathcal{A}^{\Lambda_I}_{\mH_I}$, and $\mathcal{A}^{\Lambda}_{\mH}$. Similarly, $\gamma$ is a (real) integral curve of $J(X^R_\mH)$ if and only if $\gamma$ is a ``critical point'' of the action functionals $\mathcal{A}^{\Lambda_R}_{-\mH_I}$, $\mathcal{A}^{\Lambda_I}_{\mH_R}$, and $\mathcal{A}^{\Lambda}_{i\mH}$.
\end{Prop}

\begin{proof}
 Proposition \autoref{prop:action_fct_of_real_traj} is a consequence of Proposition \autoref{prop:four_RHS} and the action principle for RHSs.
\end{proof}

\begin{Rem}[Meaning of ``critical point'']\label{rem:critical_point}
 ``Critical point'' does not denote an actual critical point. ``Critical point'' in Proposition \autoref{prop:action_fct_of_real_traj} means that the first derivative of the action functionals vanishes at $\gamma$ \underline{only} for all variations of $\gamma$ \textbf{which keep the endpoints of $\mathbf{\gamma}$ fixed!} Often, one wishes to view trajectories as actual critical points of some action functional, not just with fixed endpoints. There are two main ways to achieve this:
 \begin{enumerate}
  \item One can only consider paths which start and end at points where the primitive of the symplectic form vanishes, usually Lagrangian submanifolds of the symplectic manifold.
  \item One can only consider periodic paths such that the boundary terms in the first derivative of the action cancel each other.
 \end{enumerate}
\end{Rem}

\begin{Rem}[Action functional for ``tilted'' trajectories]\label{rem:tilted_traj}
 The observation that the integral curves of $X^R_\mH$ and $J(X^R_\mH)$ are linked to the ``critical points'' of the action functionals $\mathcal{A}^{\Lambda}_{\mH}$ and $\mathcal{A}^{\Lambda}_{i\mH}$, respectively, can be generalized to ``tilted'' trajectories. For any $\alpha\in\mathbb{R}$, $\gamma$ is an integral curve of $\cos (\alpha)\cdot X^R_\mH + \sin (\alpha)\cdot J(X^R_\mH)$ if and only if $\gamma$ is a ``critical point'' of the action functional $\mathcal{A}^{\Lambda}_{e^{i\alpha}\mH}:\mathcal{P}_{I_0}\to\mathbb{C}$ defined by
 \begin{gather*}
  \mathcal{A}^{\Lambda}_{e^{i\alpha}\mH}[\gamma]\coloneqq \int\limits_{I_0}\gamma^\ast\Lambda - e^{i\alpha}\int\limits_{I_0}\mH\circ\gamma (t)dt.
 \end{gather*}
 Of course, the same action principle holds true if we only consider the real or imaginary part of $\mathcal{A}^{\Lambda}_{e^{i\alpha}\mH}$. This fact will be of great importance later on and in \autoref{app:action_functionals}.
\end{Rem}

As the holomorphic trajectories of the HHS $(X,\Omega,\mH)$ are analytic continuations of its real trajectories, one might be content with finding action functionals for the real trajectories. However, it is also possible to construct an action functional for the holomorphic trajectories of $(X,\Omega,\mH)$ by averaging the action functionals of the four underlying RHSs:
% The idea behind the construction is simple: Recall from the proof of Proposition \autoref{prop:holo_traj} that $\gamma$ is a holomorphic trajectory of a HHS $(X,\Omega, \mH)$ iff $\gamma (t + is)$ is an integral curve of $X^R_\mH$ for fixed $s$ and an integral curve of $J(X^R_\mH)$ for fixed $t$. By Proposition \autoref{prop:action_fct_of_real_traj}, the map $\gamma_s: t\mapsto \gamma (t+is)$, for any fixed $s$, is an integral curve of $X^R_\mH$ iff $\gamma_s$ is a ``critical point'' of $\mathcal{A}^{\Lambda_R}_{\mH_R}$. Likewise, the map $\gamma_t: s\mapsto \gamma (t+is)$, for any fixed $t$, is an integral curve of $J(X^R_\mH)$ iff $\gamma_t$ is a ``critical point'' of $\mathcal{A}^{\Lambda_R}_{-\mH_I}$. Similar statements still hold if we average the action functionals $\mathcal{A}^{\Lambda_R}_{\mH_R}[\gamma_s]$ and $\mathcal{A}^{\Lambda_R}_{-\mH_I}[\gamma_t]$ over $s$ and $t$, respectively. For instance, $\gamma_s: t\mapsto \gamma (t+is)$ is an integral curve of $X^R_\mH$ for \underline{all} $s$ iff $\gamma$ is a ``critical point'' of the functional
% \begin{gather*}
%  \gamma\mapsto \int \mathcal{A}^{\Lambda_R}_{\mH_R}[\gamma_s] ds.
% \end{gather*}
% Thus, averaging the action functionals $\mathcal{A}^{\Lambda_R}_{\mH_R}[\gamma_s]$ and $\mathcal{A}^{\Lambda_R}_{-\mH_I}[\gamma_t]$ over $s$ and $t$, respectively, and taking a suitable complex linear combination of them afterwards gives us the desired action functional for holomorphic trajectories:

\begin{Lem}[Action principle for holomorphic trajectories]\label{lem:holo_action_prin}
 Let $(X,\Omega = d\Lambda,\mH)$ be an exact HHS with $\Lambda = \Lambda_R + i\Lambda_I$. Furthermore, let $R\coloneqq [t_1,t_2] + i[s_1,s_2]\subset\mathbb{C}$ be a rectangle in the complex plane with real numbers $t_1< t_2$ and $s_1< s_2$. Denote the space of smooth maps from $R$ to $X$ by $\mathcal{P}_R$ and define the action functional $\mathcal{A}^R_\mH:\mathcal{P}_R\to\mathbb{C}$ by
 \begin{gather*}
  \mathcal{A}^R_\mH[\gamma]\coloneqq \int\limits^{t_2}_{t_1}\int\limits^{s_2}_{s_1}\left[\Lambda_R\vert_{\gamma (t+is)}\left(2\frac{\pa\gamma}{\pa z}(t+is)\right) - \mH\circ\gamma (t+is)\right] ds\, dt\ \text{with}\ \frac{\pa \gamma}{\pa z}\coloneqq \frac{1}{2}\left(\frac{\pa \gamma}{\pa t} - i\frac{\pa \gamma}{\pa s}\right)
 \end{gather*}
 for every $\gamma\in\mathcal{P}_R$. Now, let $\gamma\in\mathcal{P}_R$ be a smooth map from $R$ to $X$. Then, $\gamma$ is a holomorphic trajectory of $(X,\Omega,\mH)$ iff $\gamma$ is a ``critical point''\footnote{``Critical point'' means that only those variations are allowed which keep $\gamma$ fixed on the boundary $\partial R$.} of $\mathcal{A}^R_\mH$.
\end{Lem}

\begin{proof}
 Take the notations from above and decompose $\mH = \mH_R + i\mH_I$. Furthermore, let $\gamma\in\mathcal{P}_R$ be a smooth map and let $\gamma_s:[t_1,t_2]\to X$ and $\gamma_t:[s_1,s_2]\to X$ be defined by $\gamma_s (t) = \gamma (t+is) = \gamma_t (s)$ for any $s\in [s_1,s_2]$ and $t\in [t_1,t_2]$. Recall the action functionals from Proposition \autoref{prop:action_fct_of_real_traj}. A short calculation reveals that $\mathcal{A}^R_\mH [\gamma]$ can be expressed as
 \begin{gather*}
  \mathcal{A}^R_\mH [\gamma] = \int\limits^{s_2}_{s_1} \mathcal{A}^{\Lambda_R}_{\mH_R}[\gamma_s] ds - i\int\limits^{t_2}_{t_1} \mathcal{A}^{\Lambda_R}_{-\mH_I}[\gamma_t] dt.
 \end{gather*}
 $\gamma$ is a ``critical point'' of $\mathcal{A}^R_\mH$ iff $\gamma$ is a ``critical point'' of its real and imaginary part.\\
 Now consider the real part of $\mathcal{A}^R_\mH$. For any $s\in [s_1,s_2]$, $\gamma_s$ is a ``critical point'' of $\mathcal{A}^{\Lambda_R}_{\mH_R}$ iff $\gamma_s$ is an integral curve of $X^R_\mH$, where $X_\mH = 1/2(X^R_\mH - iJ(X^R_\mH))$ is the Hamiltonian vector field of $(X,\Omega,\mH)$. Explicitly writing down the first derivative of the functional $\gamma\mapsto\int \mathcal{A}^{\Lambda_R}_{\mH_R}[\gamma_s] ds$ shows that this property is preserved under averaging: $\gamma$ is a ``critical point'' of $\gamma\mapsto\int \mathcal{A}^{\Lambda_R}_{\mH_R}[\gamma_s] ds$ iff $\gamma_s$ is an integral curve of $X^R_\mH$ for every $s\in [s_1,s_2]$.\\
 Similarly, we find for the imaginary part of $\mathcal{A}^R_\mH$ that $\gamma$ is a ``critical point'' of\linebreak $\gamma\mapsto\int \mathcal{A}^{\Lambda_R}_{-\mH_I}[\gamma_t] dt$ iff $\gamma_t$ is an integral curve of $J(X^R_\mH)$ for every $t\in [t_1,t_2]$. Combining our results so far, we find that $\gamma$ is a ``critical point'' of $\mathcal{A}^R_\mH$ iff $\gamma_s$ is an integral curve of $X^R_\mH$ for every $s\in [s_1,s_2]$ and $\gamma_t$ is an integral curve of $J(X^R_\mH)$ for every $t\in [t_1,t_2]$. Recalling from \autoref{subsec:holo_traj} that holomorphic trajectories of $(X,\Omega,\mH)$ are exactly those smooth maps $\gamma$ that are integral curves of $X^R_\mH$ in $t$-direction and $J(X^R_\mH)$ in $s$-direction concludes the proof.
\end{proof}

To get a better understanding of Lemma \autoref{lem:holo_action_prin}, several remarks are in order:

\begin{Rem}\label{rem:several_remarks}
 {\textcolor{white}{Easter Egg}}
 \begin{enumerate}
  \item Note that the action functional $\mathcal{A}^R_\mH$ in Lemma \autoref{lem:holo_action_prin} only uses the real part $\Lambda_R$ and \underline{not} $\Lambda_I$. Of course, a similar action functional including $\Lambda_I$ exists, but we will not use it for reasons that become apparent in \autoref{sec:PHHS}.
  \item As before, one might wish to express holomorphic trajectories as actual critical points of some functional. Again, there are two main ways to achieve this: one may only consider smooth maps $\gamma$ from the rectangle $R$ to $X$ which\ldots
  \begin{enumerate}
   \item \dots map the boundary $\partial R$ to points in $X$ where the $1$-form $\Lambda_R$ vanishes, usually Lagrangian submanifolds of $X$.
   \item \dots are doubly-periodic, i.e., periodic in both $t$- and $s$-direction.
  \end{enumerate}
  Theoretically, one can even imagine a mix of both methods: one only considers maps $\gamma$ which are periodic in one direction and map the boundary orthogonal to the remaining direction to Lagrangian submanifolds of $X$.
  \item In more geometrical terms, the action $\mathcal{A}^R_\mH$ can be expressed as
  \begin{gather*}
   \mathcal{A}^R_\mH[\gamma] =  \iint\limits_{R}\left[\Lambda_R\vert_{\gamma (t+is)}\left(2\frac{\pa \gamma}{\pa z}(t+is)\right) - \mH\circ\gamma (t+is)\right] dt\wedge ds,
  \end{gather*}
  where $dt\wedge ds$ is the standard area form on $\mathbb{C}\cong\mathbb{R}^2$. If $\gamma$ is a ``critical point'' of $\mathcal{A}^R_\mH$ or simply a holomorphic curve, we find:
  \begin{gather*}
   \frac{\pa\gamma}{\pa z} (t+is) = \gamma^\prime (t+is) \in T^{(1,0)}X.
  \end{gather*}
  Using $\Lambda_R (V) = i\Lambda_I (V)$ for $V\in T^{(1,0)}X$, we obtain that the action at such $\gamma$ is given by
  \begin{gather*}
   \mathcal{A}^R_\mH[\gamma] =  \iint\limits_{R}\left[\Lambda\left(\gamma^\prime (t+is)\right) - \mH\circ\gamma (t+is)\right] dt\wedge ds,
  \end{gather*}
  where the expression in rectangular brackets is holomorphic in $z = t + is$.
  \item Upon closer inspection of Lemma \autoref{lem:holo_action_prin}, one might wonder whether Lemma \autoref{lem:holo_action_prin} is still true if one restricts the domain $\mathcal{P}_R$ of $\mathcal{A}^R_\mH$ to the holomorphic curves from $R$ to $X$ instead of varying over all smooth curves from $R$ to $X$. Clearly, this is not the case, as the values $\gamma$ attains at $\partial R$ completely determine one holomorphic curve, so variation over this space is not viable. A different perspective is offered by the action functional $\mathcal{A}^R_\mH$ itself. By writing $dt\wedge ds = i/2 \cdot dz\wedge d\bar{z}$ and recalling Point 3, $\mathcal{A}^R_\mH [\gamma]$ can be written as the integral of a form admitting a primitive for holomorphic $\gamma$. By Stokes' theorem, $\mathcal{A}^R_\mH[\gamma]$ then only depends on the values of $\gamma$ on the boundary $\partial R$. Since these values are kept fixed during variation, the action functional never changes in the variational process and gives us no information. An additional explanation for this behavior is presented in \autoref{app:action_functionals}.
 \end{enumerate}
\end{Rem}

We can define action functionals like $\mathcal{A}^R_\mH$ not only for rectangles, but for all kinds of domains in $\mathbb{C}$. A large selection of them is explored in \autoref{app:action_functionals}. Here, let us quickly introduce one generalization of $\mathcal{A}^R_\mH$, namely the action functional $\mathcal{A}^{P_\alpha}_\mH$ for parallelograms $P_\alpha$. For the sake of simplicity, we assume that the first vector spanning the parallelogram $P_\alpha$ is parallel to the real axis such that we can write $P_\alpha = [t_1,t_2] + e^{i\alpha}\cdot[r_1,r_2]$ for some angle $\alpha\in\mathbb{R}\backslash\{n\cdot\pi\mid n\in\mathbb{Z}\}$ and some real numbers $t_1 < t_2$ and $r_1 < r_2$. Using the standard area form $dt\wedge ds$, the generalization of $\mathcal{A}^R_\mH$ to $\mathcal{A}^{P_\alpha}_\mH$ is straightforward:
\begin{gather*}
 \mathcal{A}^{P_\alpha}_\mH[\gamma]\coloneqq \iint\limits_{P_\alpha}\left[\Lambda_R\vert_{\gamma (t+is)}\left(2\frac{\pa\gamma}{\pa z}(t+is)\right) - \mH\circ\gamma (t+is)\right] dt\wedge ds\quad\forall\gamma\in\mathcal{P}_{P_\alpha},
\end{gather*}
where we used the coordinates $z = t + is$ and defined $\partial\gamma/\partial z$ as in Lemma \autoref{lem:holo_action_prin}. To show that $\gamma$ is a ``critical point'' of $\mathcal{A}^{P_\alpha}_\mH$ iff $\gamma$ is a holomorphic trajectory, we express $\mathcal{A}^{P_\alpha}_\mH$ in the ``tilted'' coordinates $z = t + r\cdot e^{i\alpha}$ ($\alpha$ fixed):
\begin{align*}
 \mathcal{A}^{P_\alpha}_\mH[\gamma]&= \int\limits^{t_2}_{t_1}\int\limits^{r_2}_{r_1}\left[\Lambda_R\vert_{\gamma (t+re^{i\alpha})}\left(2\frac{\pa\gamma}{\pa z}(t+re^{i\alpha})\right) - \mH\circ\gamma (t+re^{i\alpha})\right]\, \sin(\alpha)\,dr\, dt\\
 &= \int\limits^{t_2}_{t_1}\int\limits^{r_2}_{r_1}\left[\Lambda_R\vert_{\gamma (t+re^{i\alpha})}\left(ie^{-i\alpha}\cdot \frac{d\gamma_r}{d t}(t) - i\cdot \frac{d\gamma_t}{dr}(r)\right)\right.\\
 &\qquad\quad \left.- \left(ie^{-i\alpha}\mH_R - i\text{Re}(e^{i\alpha}\mH)\right)\circ\gamma (t+re^{i\alpha})\right]\, dr\, dt\\
 &= ie^{-i\alpha}\int\limits^{r_2}_{r_1}\mathcal{A}^{\Lambda_R}_{\mH_R} [\gamma_r] dr - i\int\limits^{t_2}_{t_1}\text{Re} (\mathcal{A}^\Lambda_{e^{i\alpha}\mH} [\gamma_t]) dt,
\end{align*}
where, for any $t\in [t_1, t_2]$ and $r\in [r_1,r_2]$, the curves $\gamma_r:[t_1,t_2]\to X$ and $\gamma_t:[r_1,r_2]\to X$ are defined by $\gamma_r (t) = \gamma (t + re^{i\alpha}) = \gamma_t (r)$, $\text{Re}(\cdot)$ denotes the real part, and $\mathcal{A}^{\Lambda_R}_{\mH_R}$ and $\mathcal{A}^{\Lambda}_{e^{i\alpha}\mH}$ are the action functionals from Proposition \autoref{prop:action_fct_of_real_traj} and Remark \autoref{rem:tilted_traj}, respectively. For $\alpha\neq n\cdot \pi$, $n\in\mathbb{Z}$, the complex numbers $ie^{-i\alpha}$ and $-i$ form a $\mathbb{R}$-linear basis of $\mathbb{C}$. Thus, $\gamma$ is a ``critical point'' of $\mathcal{A}^{P_\alpha}_\mH$ iff $\gamma$ is a ``critical point'' of the functionals $\gamma\mapsto\int\mathcal{A}^{\Lambda_R}_{\mH_R} [\gamma_r] dr$ and $\gamma\mapsto\int\text{Re} (\mathcal{A}^\Lambda_{e^{i\alpha}\mH} [\gamma_t]) dt$. The rest now follows as in proof of Lemma \autoref{lem:holo_action_prin} by exploiting Proposition \autoref{prop:action_fct_of_real_traj}, Remark \autoref{rem:tilted_traj}, and the fact that holomorphic trajectories are exactly those smooth curves which are integral curves of $X^R_\mH$ in $t$-direction and integral curves of $\cos (\alpha) X^R_\mH + \sin (\alpha) J(X^R_\mH)$ in $r$-direction ($z = t + re^{i\alpha}$) for $\alpha\in\mathbb{R}\backslash\{n\cdot\pi\mid n\in\mathbb{Z}\}$. Summing up our results, we have just shown:

\begin{Prop}[Action principle for parallelograms]\label{prop:holo_action_prin_para}
 Let $(X,\Omega = d\Lambda,\mH)$ be an exact HHS with $\Lambda = \Lambda_R + i\Lambda_I$. For $\alpha\in\mathbb{R}\backslash\{n\cdot\pi\mid n\in\mathbb{Z}\}$, let $P_\alpha\coloneqq [t_1,t_2] + e^{i\alpha}[r_1,r_2]\subset\mathbb{C}$ be a parallelogram in the complex plane with real numbers $t_1< t_2$ and $r_1< r_2$. Denote the space of smooth maps from $P_\alpha$ to $X$ by $\mathcal{P}_{P_\alpha}$ and define the action functional $\mathcal{A}^{P_\alpha}_\mH:\mathcal{P}_{P_\alpha}\to\mathbb{C}$ by
\begin{gather*}
 \mathcal{A}^{P_\alpha}_\mH[\gamma]\coloneqq \iint\limits_{P_\alpha}\left[\Lambda_R\vert_{\gamma (t+is)}\left(2\frac{\pa\gamma}{\pa z}(t+is)\right) - \mH\circ\gamma (t+is)\right] dt\wedge ds\ \text{with}\ \frac{\pa\gamma}{\pa z}\coloneqq \frac{1}{2}\left(\frac{\pa\gamma}{\pa t} - i\frac{\pa\gamma}{\pa s}\right)
\end{gather*}
 for every $\gamma\in\mathcal{P}_{P_\alpha}$. Now, let $\gamma\in\mathcal{P}_{P_\alpha}$ be a smooth map from $P_\alpha$ to $X$. Then, $\gamma$ is a holomorphic trajectory of $(X,\Omega,\mH)$ iff $\gamma$ is a ``critical point''\footnote{Again, ``critical point'' means that only those variations are allowed which keep $\gamma$ fixed on the boundary $\partial P_\alpha$.} of $\mathcal{A}^{P_\alpha}_\mH$.
\end{Prop}

\begin{Rem}\label{rem:action_para}
 Proposition \autoref{prop:holo_action_prin_para} is a direct generalization of Lemma \autoref{lem:holo_action_prin}, since one obtains Lemma \autoref{lem:holo_action_prin} from Proposition \autoref{prop:holo_action_prin_para} by setting $\alpha = \pi/2$.
\end{Rem}

Before we conclude \autoref{subsec:holo_action_fun_and_prin}, let us inspect Point 2b) of Remark \autoref{rem:several_remarks} more closely. If a holomorphic curve $\gamma:P_\alpha\to X$ whose domain is a parallelogram $P_\alpha\subset\mathbb{C}$ is doubly-periodic, i.e., periodic in $t$- and $r$-direction for $z = t + re^{i\alpha}$, then we can also view $\gamma$ as a holomorphic map from a complex torus to $X$. In this sense, we can interpret holomorphic trajectories whose domains are complex tori as the holomorphic analogue to periodic orbits of RHSs. In contrast to periodic orbits of RHSs, however, the domains of two holomorphic periodic orbits do not need to be isomorphic. Indeed, the complex structure of such a torus is determined by the shape of the parallelogram $P_\alpha$. Therefore, the action functional $\mathcal{A}^{P_\alpha}_\mH$ is only sensitive to certain holomorphic periodic orbits, namely those whose domains share the complex structure induced by $P_\alpha$.\\
Non-constant holomorphic periodic orbits are rather rare and do not exist in most HHSs $(X,\Omega,\mH)$. For instance, take $X$ to be the standard example $\mathbb{C}^{2n}$. Due to the compactness of a complex torus $\mathbb{C}/\Gamma$, the maximum principle applies and any holomorphic map\linebreak $\gamma:\mathbb{C}/\Gamma\to X$ has to be constant. The same result applies if $X$ is Brody hyperbolic\footnote{A complex manifold $X$ is Brody hyperbolic iff every holomorphic map $\gamma:\mathbb{C}\to X$ defined on all of $\mathbb{C}$ is constant.}. Furthermore, if $X$ is compact, then all holomorphic trajectories are constant, since all Hamiltonians are constant by the maximum principle. Still, there are examples of HHSs $(X,\Omega,\mH)$ where a plethora of holomorphic periodic orbits exists.

\begin{Ex}[Natural Hamiltonians on complex tori $\mathbb{C}^n/\Gamma$]\label{ex:complex_torus}\normalfont
 Let $n\in\mathbb{N}$ be a natural number and $\Gamma\subset\mathbb{C}^n$ be a lattice, i.e.
 \begin{gather*}
  \Gamma\coloneqq \left\{\sum\limits^{2n}_{j=1} k_j\cdot e_j\middle| k_j\in\mathbb{Z}\right\},
 \end{gather*}
 where the vectors $e_1,\ldots, e_{2n}\in\mathbb{C}^n$ form an $\mathbb{R}$-linear basis of $\mathbb{C}^n$. Then, $\mathbb{C}^n/\Gamma$ is a complex torus of complex dimension $n$. Now consider the holomorphic cotangent bundle\linebreak $X\coloneqq T^{\ast, (1,0)}(\mathbb{C}^n/\Gamma)\cong \mathbb{C}^n/\Gamma\times\mathbb{C}^n$ with coordinates $([Q_1,\ldots, Q_j], P_1,\ldots, P_j)\in \mathbb{C}^n/\Gamma\times\mathbb{C}^n$ and canonical $2$-form $\Omega = \sum^n_{j=1} dP_j\wedge dQ_j$. We want to determine all natural Hamiltonians $\mH = \mathcal{T} + \mathcal{V}$ on the HSM $(X,\Omega)$. The potential energy $\mathcal{V}$ factors through a holomorphic function on $\mathbb{C}^n/\Gamma$. As $\mathbb{C}^n/\Gamma$ is compact, all holomorphic functions on it are constant due to the maximum principle. Since changing the Hamiltonian by a constant does not change the dynamics of the system, we can set the potential energy to zero without loss of generality. To compute the kinetic energy $\mathcal{T}$, we need to classify all holomorphic metrics $g$ on $\mathbb{C}^n/\Gamma$. The projection $\mathbb{C}^n\to\mathbb{C}^n/\Gamma$ gives rise to $n$ linearly independent holomorphic $1$-forms $dQ_j$, $j = 1,\ldots, n$, on the torus $\mathbb{C}^n/\Gamma$ which we have already used to express the canonical form $\Omega$. Using these $1$-forms, we can write $g$ as
 \begin{gather*}
  g = \sum^{2n}_{i,j = 1} g_{ij}dQ_i\otimes dQ_j,
 \end{gather*}
 where $g_{ij}$ are holomorphic functions on the torus. As before, these functions have to be constant implying that the space of holomorphic metrics $g$ on $\mathbb{C}^n/\Gamma$ is isomorphic to the space of symmetric and non-degenerate $\mathbb{C}$-bilinear forms on the complex vector space $\mathbb{C}^n$. By a standard result from linear algebra, every symmetric and non-degenerate $\mathbb{C}$-bilinear form on $\mathbb{C}^n$ can be brought into the standard form $g_{ij} = \delta_{ij}$\footnote{Here, $\delta_{ij}$ is the Kronecker delta!} by a $\mathbb{C}$-linear transformation. Hence, after transforming the lattice $\Gamma$ if necessary, we can assume that the metric $g$ is given by $g = \sum^{2n}_{j = 1} dQ^2_j$. In total, it suffices to investigate the dynamics of the HHS $(\mathbb{C}^n/\Gamma\times\mathbb{C}^n, \sum^n_{j=1} dP_j\wedge dQ_j, \mH)$ with Hamiltonian
 \begin{gather*}
  \mH (Q_1,\ldots, Q_n, P_1,\ldots, P_n) = \sum^{2n}_{j = 1} \frac{P^2_j}{2}
 \end{gather*}
 for all lattices $\Gamma\subset\mathbb{C}^n$ in order to study all natural Hamiltonians on a complex torus.\\
 Clearly, the Hamilton equations related to this problem are given by
 \begin{gather*}
  Q^\prime_j (z) = P_j (z);\quad P^\prime_j (z) = 0
 \end{gather*}
 and, given the initial value $\gamma (0) = ([Q^0_1,\ldots, Q^0_n], P^0_1,\ldots, P^0_n)$, obviously solved by the holomorphic trajectory $\gamma:\mathbb{C}\to X$,
 \begin{gather*}
  \gamma (z)\coloneqq ([Q^0_1 + z\cdot P^0_1,\ldots, Q^0_n + z\cdot P^0_n], P^0_1,\ldots, P^0_n).
 \end{gather*}
 Let us now define $P^0\coloneqq (P^0_1,\ldots, P^0_n)\in\mathbb{C}^n$ and consider different values for $P^0$:
 \begin{enumerate}
  \item If $P^0 = 0$, then $\gamma$ is simply a constant curve.
  \item If $P^0 \neq 0$ and $z\cdot P^0\notin\Gamma$ for every $z\in\mathbb{C}\backslash\{0\}$, then $\gamma$ is a regular holomorphic trajectory with no periodicity.
  \item If $P^0\neq 0$ and $z_1\cdot P^0\in\Gamma$ for at least one $z_1\neq 0$, then $\gamma$ is a regular holomorphic trajectory which is periodic in at least one direction. In this case, we can view $\gamma$ as a holomorphic map from a complex cylinder to $X$.
  \item If $P^0\neq 0$ and $z_1\cdot P^0, z_2\cdot P^0\in\Gamma$ for two $\mathbb{R}$-linearly independent complex numbers $z_1, z_2\in\mathbb{C}$, then $\gamma$ is a regular, doubly-periodic holomorphic trajectory. In this case, $\gamma$ is holomorphic periodic orbit.
 \end{enumerate}
 We observe that the topology and the complex structure of the domain of $\gamma$ changes depending on the momentum $P^0$.
\end{Ex}

\begin{Rem}[General Hamiltonians on a complex torus]\label{rem:ham_on_com_tor}
 As it turns out, Example \autoref{ex:complex_torus} covers all possible Hamiltonians on the HSM $(T^{\ast, (1,0)}(\mathbb{C}^n/\Gamma), \sum^{2n}_{j = 1} dP_j\wedge dQ_j)$. Let $\mH$ be any holomorphic function on $T^{\ast, (1,0)}(\mathbb{C}^n/\Gamma)$. Since $T^{\ast, (1,0)}(\mathbb{C}^n/\Gamma)$ is isomorphic to $\mathbb{C}^n/\Gamma\times\mathbb{C}^n$, $\mH$ cannot depend on the $Q_j$-coordinates due to maximum principle. This allows us to repeat the discussion from Example \autoref{ex:complex_torus} by replacing $P^0_j$ with $\partial\mH/\partial P_j (P^0)$ in the solution to the Hamilton equations.
\end{Rem}

\subsection{Application of HHSs: Lefschetz and Almost Toric Fibrations}
\label{subsec:Lefschetz}

One interesting aspect of HHSs is their interplay with two important structures in (symplectic) geometry: Lefschetz fibrations and almost toric fibrations. In this subsection, we briefly explore the connection between these structures. We begin the investigation by giving a short introduction to Lefschetz and almost toric fibrations.

\subsubsection*{Lefschetz Fibrations}

Let us first recall the definition of a Lefschetz fibration:

\begin{Def}[Lefschetz fibration]
 Let $X$ be a smooth $2m$-manifold and $C$ be a smooth $2$-manifold (both possibly with boundary). We call a smooth surjective map $\pi:X\to C$ a \textbf{Lefschetz fibration} iff the following three conditions are satisfied:
 \begin{enumerate}
  \item $\partial X = \pi^{-1} (\partial C)$
  \item All points on the boundary $\partial X$ are regular points of $\pi$.
  \item For every critical point $p\in\text{Crit}(\pi)\subset\text{Int}(X)$, there exists a smooth chart\linebreak $\psi_X:U_X\to V_X\subset\R^{2m}\cong \Cx^m$ near $p$ and a smooth chart $\psi_C:U_C\to V_C\subset\R^{2}\cong \Cx$ near $\pi (p)$ such that
   \begin{gather*}
    \psi_C\circ\pi\circ\psi_X^{-1} (z_1,\ldots,z_m) = \sum^m_{j=1}z^2_j.
   \end{gather*}
 \end{enumerate}
\end{Def}

Roughly speaking, a Lefschetz fibration generalizes the notion of a fiber bundle over a surface where we now also allow singular fibers. This aspect is captured by the following proposition:

\begin{Prop}[Lefschetz fibrations as fiber bundles]\label{prop:Lef_fiber_bundles}
 Let $\pi:X\to C$ be a Lefschetz fibration and $C^\ast$ be the set of regular values of $\pi$. Further, assume that $X$ is connected. If $\pi:\pi^{-1}(C^\ast)\to C^\ast$ is proper, then $\pi^{-1}(C^\ast)\stackrel{\pi}{\to} C^\ast$ is a fiber bundle. In particular, if $X$ (and then also $C$) is compact, $\pi^{-1}(C^\ast)\stackrel{\pi}{\to} C^\ast$ is a fiber bundle.
\end{Prop}

\begin{proof}
 This follows directly from the fact\footnote{Due to Ehresmann, cf. \cite{Ehresmann1952}.} that every smooth, surjective, and proper submersion between connected manifolds is a fiber bundle.
\end{proof}

Most authors include additional conditions in the definition of a Lefschetz fibration. Often, $X$ is assumed to be a compact, connected, and oriented fourfold. On one hand, this has historic reasons: While Lefschetz introduced these fibrations to study the topology of complex surfaces, Donaldson and Gompf brought Lefschetz fibrations to the attention of the symplectic community by showing that every compact symplectic fourfold admits the structure of a Lefschetz fibration (after blowing up if necessary) and vice versa\footnote{Under mild conditions.}. For a comprehensive overview of the history of fourfolds and Lefschetz fibrations, confer the introduction in \cite{Fuller2003}.\\
On the other hand, the case $m=2$ has a rich structure and is well understood: If $X$ is a closed fourfold, then the regular fibers of $\pi:X\to C$ are closed surfaces. Under mild conditions\footnote{Both $X$ and $C$ are oriented and connected and, additionally, $C$ is simply connected, cf. \cite{Naylor2016}.}, the regular fibers of $\pi:X\to C$ are even oriented and connected, hence, are surfaces of genus $g$. In this case, the singular fibers are pinched surfaces of genus $g$ (cf. \cite{Naylor2016}). In our investigation, the case $m=2$ also plays an important role.


\subsubsection*{Almost Toric Fibrations}

Almost toric fibrations generalize the notion of toric fibrations which themselves can be understood as the momentum map of an effective Hamiltonian torus action. To capture this idea, let us recall the famous convexity theorem of Atiyah, Guillemin, and Sternberg (confer, for instance, \cite{Audin2004} and \cite{Symington2002}):

\begin{Thm}[Convexity theorem]
 Let $(X^{2m},\omega)$ be a connected and closed symplectic manifold with an effective Hamiltonian $T^m$-action on it. Then, the image of the associated momentum map $\pi: X\to\R^m$ is a convex polytope.
\end{Thm}

The polytope has a natural stratification. The highest dimensional stratum, i.e., its interior is the set of regular values of $\pi$. The regular level sets are Lagrangian tori and one can express $\omega$ and $\pi$ in a neighborhood of any regular point as\footnote{This is just the equally famous Arnold-Liouville theorem.}
\begin{gather*}
 \omega = \sum^m_{j=1} dx_j\wedge dy_j;\quad \pi_j = x_j.
\end{gather*}
For $k<m$, the points on the $k$-dimensional stratum are critical values. Similarly as before, one can show that in a neighborhood of such a critical point we can write (cf. \cite{Symington2002} and \cite{Leung2003}):
\begin{gather}
 \omega = \sum^m_{j=1} dx_j\wedge dy_j;\quad \pi_j = x_j\text{ for }1\leq j\leq k;\quad \pi_j = x^2_j + y^2_j\text{ for }k<j\leq m.\label{eq:toric}
\end{gather}
We see that a toric fibration is a symplectic manifold viewed as a fiber bundle whose regular fibers are Lagrangian tori and whose singular fibers take the local form as described by \autoref{eq:toric}.\\
Almost toric fibrations, introduced by M. Symington in \cite{Symington2002}, exemplify a similar structure, but the local description of their singular fibers is broadened. The following definitions are taken from \cite{Symington2002} and \cite{Leung2003}:

\begin{Def}[Lagrangian and almost toric fibrations]\label{def:alm_tor}
 Let $C^m$ be a smooth and $(X^{2m},\omega)$ be a symplectic manifold (both possibly with boundary). Further, let $\pi:X\to C$ be a smooth and surjective map. We call $\pi$ a \textbf{Lagrangian fibration} of $(X,\omega)$ iff there exists an open and dense subset $C^\ast\subset C$ such that $\pi^{-1}(C^\ast)\stackrel{\pi}{\to} C^\ast$ is a fiber bundle with Lagrangian fibers. We call $\pi$ an \textbf{almost toric fibration} iff $\pi$ is a Lagrangian fibration and every critical point of $\pi$ has a neighborhood in which $\omega$ and $\pi$, after choosing charts, take the following form ($0\leq k<m$):
 \begin{enumerate}
  \item $\omega = \sum^m_{j=1} dx_j\wedge dy_j$,
  \item $\pi_j = x_j$ for $1\leq j\leq k$,
  \item $\pi_j = x^2_j + y^2_j$ or $(\pi_j, \pi_{j+1}) = (x_jy_j + x_{j+1}y_{j+1}, x_jy_{j+1} - x_{j+1}y_j)$ for $k<j\leq m$.
 \end{enumerate}
\end{Def}

\begin{Rem}[Fibers of almost toric fibrations]\label{rem:toric_fibers}
 If the regular fibers of an almost toric fibration are compact and connected, then they are diffeomorphic to a torus by the Arnold-Liouville theorem, explaining the name ``almost toric fibration''.
\end{Rem}

Let us now investigate the relation between Lefschetz and almost toric fibrations using HHSs. The idea is that we define a compatible holomorphic symplectic structure on a Lefschetz fibration. Such a fibration can be interpreted as a HHS. At the same time, a holomorphic symplectic Lefschetz fibration in real dimension four is, under mild conditions, an almost toric fibration, as we will show later on.\\
First, we formulate the notion of a Lefschetz fibration in a holomorphic setup:

\begin{Def}[Holomorphic Lefschetz fibration]
 Let $X$ be a complex $m$-dimensional and $C$ be a complex one-dimensional manifold (both possibly with boundary\footnote{A complex manifold with boundary is defined similarly to a real manifold with boundary, even though there are subtle differences.}). We say that a surjective holomorphic map $\pi:X\to C$ is a \textbf{holomorphic Lefschetz fibration} iff the following three conditions are satisfied:
 \begin{enumerate}
  \item $\partial X = \pi^{-1} (\partial C)$
  \item All points on the boundary $\partial X$ are regular points of $\pi$.
  \item Every critical point $p\in\text{Crit}(\pi)\subset\text{Int}(X)$ is non-degenerate.
 \end{enumerate}
\end{Def}

Here, a critical point $p\in\text{Crit}(\pi)$ is called non-degenerate iff its (complex) Hessian\footnote{Since $p$ is a critical point, the non-degeneracy of the Hessian is well-defined, i.e., independent of the choice of charts.} at $p$ is non-degenerate in some holomorphic charts of $X$ and $C$. As the name implies, a holomorphic Lefschetz fibration is also a (smooth) Lefschetz fibration:

\begin{Prop}[Holomorphic Lefschetz fibrations are Lefschetz fibrations]
 Let\linebreak $\pi:X\to C$ be a holomorphic Lefschetz fibration. Then $\pi:X\to C$ is a (smooth) Lefschetz fibration in the usual sense. In particular, the charts $\psi_X$ and $\psi_C$ can be chosen to be holomorphic.
\end{Prop}

\begin{proof}
 This is a direct consequence of the holomorphic Morse lemma.
\end{proof}

\begin{Lem}[Holomorphic Morse lemma]
 Let $X$ be a complex manifold, $f:X\to\Cx$ be a holomorphic function, and $p\in X$ be a non-degenerate critical point of $f$. Then, there exists a holomorphic chart $\psi_X:U_X\to V_X\subset\Cx^m$ of $X$ near $p$ with $\psi_X (p) = 0$ such that:
 \begin{gather*}
  f\circ\psi^{-1}_X (z_1,\ldots, z_m) = f(p) + \sum^m_{j=1} z^2_j\quad\forall (z_1,\ldots, z_m)\in V_X.
 \end{gather*}
\end{Lem}

\begin{Rem}[Signature of Hessian]
 In contrast to the real Morse lemma, we do not need to care about the signature of the (complex) Hessian in the holomorphic Morse lemma, since all non-degenerate symmetric $\Cx$-bilinear forms on $\Cx^m$ are isomorphic.
\end{Rem}

\begin{proof}
 The holomorphic Morse lemma can be shown in the same way as the usual Morse lemma (confer, for instance, the proof in \cite{Audin2014} on page 12 ff.), where, of course, we use the appropriate theorems from complex analysis instead of their smooth counterparts\footnote{For instance, the holomorphic implicit function theorem instead of the implicit function theorem.}.
\end{proof}

Now, we put a compatible symplectic structure on a holomorphic Lefschetz fibration:

\begin{Def}[Holomorphic symplectic Lefschetz fibration]
 We call a holomorphic Lefschetz fibration $\pi:X\to C$ \textbf{symplectic} iff $X$ admits the structure of a HSM $(X,\Omega)$ such that every critical point $p\in\text{Crit}(\pi)$ has holomorphic Morse Darboux charts near it, i.e., there are holomorphic charts $\psi_X = (z_1,\ldots, z_{2n}):U_X\to V_X\subset\Cx^{2n}$ of $X$ near $p$ and $\psi_C:U_C\to V_C\subset\Cx$ of $C$ near $\pi (p)$ satisfying:
 \begin{enumerate}
  \item $\Omega\vert_{U_X} = \sum^n_{j=1} dz_{j+n}\wedge dz_j$,
  \item $\psi_C\circ\pi\vert_{U_X} = \sum^{2n}_{j=1} z^2_j$.
 \end{enumerate}
\end{Def}

If $\pi:X\to C$ is a holomorphic symplectic Lefschetz fibration with underlying HSM $(X,\Omega)$, we can locally interpret $(X,\Omega,\pi)$ as a HHS after choosing a holomorphic chart $\psi_C$ of $C$. Of course, the Hamiltonian vector field of this HHS is not well-defined, since the Hamilton function depends on the choice of $\psi_C$. However, two Hamiltonian vector fields only differ by a holomorphic function:
\begin{gather*}
 X_{\hat \psi_C\circ\pi}\vert_p = (\hat\psi_C\circ\psi^{-1}_C)^\prime (\psi_C\circ\pi(p))\cdot X_{\psi_C\circ\pi}\vert_p\quad\text{for }p\in X.
\end{gather*}
Hence, the two Hamiltonian vector fields have the same trajectories, just parameterized differently. This implies that the holomorphic foliation by the regular level sets of the HHS is still well-defined, even though the Hamiltonian vector field is not. In particular, if $X$ is complex two-dimensional or, equivalently, real four-dimensional and the level sets of $\pi:X\to C$ are connected, then the leaves of this foliation are just the regular level sets themselves allowing us to interpret a holomorphic symplectic Lefschetz fibration as the holomorphic foliation of a HHS.\\
The next step is to link holomorphic symplectic Lefschetz fibrations to almost toric fibrations. We show that every holomorphic symplectic Lefschetz fibration in real dimension four is also an almost toric fibration if $\pi$ is proper.

\begin{Prop}[Holomorphic symplectic Lefschetz fibrations in real dimension four]\label{prop:holo_lef_toric}
 Let $\pi:X\to C$ be a holomorphic symplectic Lefschetz fibration with underlying HSM $(X,\Omega = \Omega_R + i\Omega_I)$. Assume that $X$ is connected and has real dimension four. If $\pi:X\to C$ is proper, then $\pi:X\to C$ is an almost toric fibration of $(X,\Omega_R)$. In particular, if $X$ is compact, then $\pi:X\to C$ is an almost toric fibration of $(X,\Omega_R)$.
\end{Prop}

\begin{proof}
 The set $C^\ast$ of regular values of $\pi$ is open and dense in $C$. By the same argument as in the proof of Proposition \autoref{prop:Lef_fiber_bundles}, $\pi^{-1}(C^\ast)\stackrel{\pi}{\to}C^\ast$ is a fiber bundle. We now consider the fibers of $\pi^{-1}(C^\ast)\stackrel{\pi}{\to}C^\ast$. The HHS $(X,\Omega,\pi)$ is complex two-dimensional, hence, the regular level sets of $\pi$ are complex Lagrangian submanifolds of $(X,\Omega)$. Thus, they are also Lagrangian submanifolds of $(X,\Omega_R)$, as $\Omega = \Omega_R + i\Omega_I$ is a holomorphic $2$-form. This implies that $\pi:X\to C$ is a Lagrangian fibration of $(X,\Omega_R)$.\\
 \pagebreak
 To conclude the proof, we need to consider a critical point $p$ of $\pi$. By definition, there are holomorphic charts $\psi_X = (z_1, z_2):U_X\to V_X\subset\Cx^2$ of $X$ near $p$ and $\psi_C:U_C\to V_C\subset\Cx$ of $C$ near $\pi (p)$ satisfying:
 \begin{align*}
  \Omega\vert_{U_X} &= dz_2\wedge dz_1,\\
  \psi_C\circ\pi\vert_{U_X} &= z^2_1 + z^2_2.
 \end{align*}
 In the new coordinates $z_1\eqqcolon (\hat z_2 - i\hat z_1)/\sqrt{2}$, $z_2\eqqcolon (\hat z_1 - i\hat z_2)/\sqrt{2}$, and $\hat \psi_C \coloneqq -\psi_C/2i$, we find:
 \begin{alignat*}{4}
  \Omega\vert_{\hat U_X} &= \, d\hat z_1\wedge d\hat z_2 &&= &&(d\hat x_1\wedge d\hat x_2 - d\hat y_1\wedge d\hat y_2) &&+ i(d\hat x_1\wedge d\hat y_2 + d\hat y_1\wedge d\hat x_2) ,\\
  \hat \psi_C\circ\pi\vert_{\hat U_X} &= \hat z_1\hat z_2 &&= &&(\hat x_1 \hat x_2 - \hat y_1\hat y_2) &&+ i(\hat x_1\hat y_2 + \hat y_1 \hat x_2),
 \end{alignat*}
 where we have used the decomposition $\hat z_j = \hat x_j + i\hat y_j$. In particular, we have\linebreak $\Omega_R\vert_{\hat U_X} = d\hat x_1\wedge d\hat x_2 - d\hat y_1\wedge d\hat y_2$ in these coordinates. Setting $x_1 \coloneqq \hat x_1$, $x_2\coloneqq -\hat y_1$, $y_1\coloneqq \hat x_2$, $y_2\coloneqq \hat y_2$, $\pi_1 \equiv \text{Re}(\hat \psi_C\circ\pi\vert_{\hat U_X})$, and $\pi_2 \equiv \text{Im}(\hat \psi_C\circ\pi\vert_{\hat U_X})$ reproduces the local structure near a critical point as in the definition of an almost toric fibration.
\end{proof}

\begin{Rem}[Proposition \autoref{prop:holo_lef_toric} for $(X,\Omega_I)$]
 With the same assumptions as in Proposition \autoref{prop:holo_lef_toric}, we find that $\pi:X\to C$ is also an almost toric fibration of $(X,\Omega_I)$ if $\pi:X\to C$ is proper. In that regard, a holomorphic symplectic Lefschetz fibration gives rise to two different almost toric fibrations.
\end{Rem}

\begin{Rem}[Critical points are non-toric]
 Observe that the critical points of a holomorphic symplectic Lefschetz fibration are non-toric, i.e., their local description matches the non-toric case in Definition \autoref{def:alm_tor}.
\end{Rem}

One might wonder how many manifolds $X$ Proposition \autoref{prop:holo_lef_toric} is applicable to. In the case that $X$ is closed, the answer is already known: Leung and Symington classified in \cite{Leung2003} all closed almost toric four-folds up to diffeomorphisms. They have shown that the only examples which are locally Lefschetz, i.e., whose critical points are non-toric are the K3 surface with base $C = S^2$ and its $\mathbb{Z}_2$-quotient, i.e., the Enriques surface with base $C = \R P^2$. Since $\R P^2$ is not orientable, it cannot admit a complex structure and, thus, the Enriques surface cannot be a holomorphic symplectic Lefschetz fibration. Hence, the only possible example of a closed holomorphic symplectic Lefschetz fibration in four dimensions is the K3 surface\footnote{By Remark \autoref{rem:toric_fibers}, the regular fibers of the K3 surface are tori.}.\\
To conclude this section, we generalize the last question and ask ourselves whether there is an obstruction for a holomorphic Lefschetz fibration to be symplectic. Of course, the space $X$ of a holomorphic Lefschetz fibration $\pi:X\to C$ needs to admit the structure of a HSM in order for $\pi:X\to C$ to be symplectic. This itself is a non-trivial condition. But even if $X$ is a HSM, it is not clear whether the Lefschetz fibration admits Morse Darboux charts near critical points. This question is especially interesting, since we already know that every HSM admits Darboux charts near any point and every holomorphic Lefschetz fibration admits Morse charts near critical points.\\
In the complex case, this is a non-trivial problem and not completely understood\footnote{At least not by the author.}. To gain some intuition, we consider the real case in dimension two instead.

\newpage

\textbf{Setting:} Let $M$ be a smooth manifold of real dimension two with symplectic form $\omega$ on it, $L$ be a smooth manifold of real dimension one, $f\in C^\infty (M,L)$, and $p\in M$ be a non-degenerate\footnote{Confer Definition \autoref{def:morse_index}.} critical point of $f$.\\

\textbf{Question:} Do smooth charts $\psi_M = (x,y):U_M\to V_M\subset\R^2$ of $M$ near $p$ and\linebreak $\psi_L:U_L\to V_L\subset\R$ of $L$ near $f(p)$ exist such that
\begin{enumerate}
 \item $\omega\vert_{U_M} = dx\wedge dy$,
 \item $\psi_L\circ f\vert_{U_M} = x^2 + y^2$?
\end{enumerate}

Of course, the answer to the question in full generality is negative due to the Morse index of $p$ which is absent in the complex case:

\begin{Def}[Morse index $\mu_f (p)$]\label{def:morse_index}
 Let $M$ be a smooth $m$-manifold, $L$ be a smooth $1$-manifold, $f\in C^\infty (M,L)$, and $p\in M$ be a critical point of $f$. We say that $p$ is non-degenerate iff the Hessian of $\psi_L\circ f\circ \psi^{-1}_M$ at $\psi_M (p)$ is non-degenerate for some charts $\psi_M$ of $M$ near $p$ and $\psi_L$ of $L$ near $f(p)$. The \textbf{Morse index} $\mu_f (p)$ is the number $\min\{k, m-k\}$ where $k$ is the usual Morse index of $\psi_L\circ f\circ \psi^{-1}_M$ at $\psi_M (p)$ for charts $\psi_M$ and $\psi_L$, i.e, the number of negative eigenvalues of its Hessian.
\end{Def}

\begin{Rem*}
 Even though the non-degeneracy of the Hessian is independent of the choice of charts, the number of negative eigenvalues of the Hessian is not. A orientation reversing transformation of the chart $\psi_L$, for instance $\psi_L\mapsto -\psi_L$, changes the number from $k$ to $m-k$, explaining the definition of the Morse index.
\end{Rem*}

The Morse index of $(x,y)\mapsto x^2 + y^2$ is $0$. Since the Morse index is invariant under change of charts, a necessary condition for the existence of Morse Darboux charts near $p$ is $\mu_f (p) = 0$. The question remains whether this condition is sufficient. Aside from regularity issues, this seems to be the case. To make this precise, consider the following two lemmata which are proven in \autoref{app:morse_darboux_lem}:

\begin{Lem}[Morse Darboux lemma I]\label{lem:morse_darboux_lem_I}
 Let $(M^2,\omega)$ be a symplectic $2$-manifold, $L^1$ be a smooth $1$-manifold, $f\in C^\infty (M,L)$, and $p\in M$ be a non-degenerate critical point of $f$ with Morse index $\mu_f (p) = 0$. Further, let $T>0$ be a positive real number. Then, there exists a $C^1$-chart $\psi_L:U_L\to V_L\subset\R$ of $L$ near $f(p)$ which is smooth on $U_L\backslash\{f(p)\}$ such that all non-constant trajectories near $p$ of the RHS $(U_M, \omega\vert_{U_M}, H)$ with $U_M\coloneqq f^{-1}(U_L)$ and $H\coloneqq \psi_L\circ f\vert_{U_M}$ are $T$-periodic.
\end{Lem}

\newpage
% 
% \begin{Lem}[Morse Darboux lemma II]\label{lem:morse_darboux_lem_II}
%  Let $(M^2,\omega)$ be a symplectic manifold and let\linebreak $H\in C^\infty (M,\R)$ be a smooth function on $M$ with non-degenerate critical point $p\in M$. Further, let $T>0$ be a positive real number. Then, the following statements are equivalent:
%  \begin{enumerate}
%   \item There exists a topological chart $\psi_M = (x,y):U_M\to V_M\subset\R^2$ of $M$ near $p$ which is smooth on $U_M\backslash\{p\}$ such that ($\psi_M (p) = 0$):
%   \begin{enumerate}[label = (\alph*)]
%    \item $H\vert_{U_M} = H(p) \pm \frac{\pi}{T}(x^2 + y^2)$,
%    \item $\omega\vert_{U_M} = dx\wedge dy$.
%   \end{enumerate}
%   \item There exists an open neighborhood $U_M\subset M$ of $p$ such that:
%   \begin{enumerate}[label = (\alph*)]
%    \item For every metric $d$ on $M$ compatible with its topology and every $\varepsilon >0$, there is a non-constant trajectory $\gamma$ of the RHS $(U_M, \omega\vert_{U_M}, H\vert_{U_M})$ such that $\sup_{t\in\R} d(\gamma (t), p) <\varepsilon$.
%    \item All non-constant trajectories of the RHS $(U_M, \omega\vert_{U_M}, H\vert_{U_M})$ are $T$-periodic.
%   \end{enumerate}
%   \item There exists a number $E_0 > 0$ such that:
%   \begin{enumerate}[label = (\alph*)]
%    \item $\mu_H (p) \neq 1$,
%    \item $\int_{U(E)} \omega = T\cdot E$ for every $E\in [0, E_0]$, where $U(E)$ is the connected component containing $p$ of the set $\{q\in M\mid |H(q)-H(p)|\leq E\}$.
%   \end{enumerate}
%  \end{enumerate}
% \end{Lem}

\begin{Lem}[Morse Darboux lemma II]\label{lem:morse_darboux_lem_II}
 Let $(M^2,\omega)$ be a symplectic $2$-manifold and let\linebreak $H\in C^\infty (M,\R)$ be a smooth function on $M$ with non-degenerate critical point $p\in M$ of Morse index $\mu_H (p)\neq 1$. Further, let $T>0$ be a positive real number. Then, the following statements are equivalent:
 \begin{enumerate}
  \item There exists a topological chart $\psi_M = (x,y):U_M\to V_M\subset\R^2$ of $M$ near $p$ which is smooth on $U_M\backslash\{p\}$ such that ($\psi_M (p) = 0$):
  \begin{enumerate}[label = (\alph*)]
   \item $H\vert_{U_M} = H(p) \pm \frac{\pi}{T}(x^2 + y^2)$,
   \item $\omega\vert_{U_M} = dx\wedge dy$.
  \end{enumerate}
  \item There exists an open neighborhood $U_M\subset M$ of $p$ such that all non-constant trajectories of the RHS $(U_M, \omega\vert_{U_M}, H\vert_{U_M})$ are $T$-periodic.
  \item There exists a number $E_0 > 0$ such that $\int_{U(E)} \omega = T\cdot E$ for every $E\in [0, E_0]$,\linebreak where $U(E)$ is the connected component containing $p$ of the set\linebreak $\{q\in M\mid |H(q)-H(p)|\leq E\}$.
 \end{enumerate}
\end{Lem}

% \begin{Rem*}
%  Condition (a) in Statement 2 and 3 of Lemma \autoref{lem:morse_darboux_lem_II} can be omitted if we include $\mu_H (p)\neq 1$ in the assumptions. We worded Lemma \autoref{lem:morse_darboux_lem_II} this way so that the equivalence of Statement 1 and 2 has a (weaker) holomorphic analogue (cf. Lemma \autoref{lem:holo_morse_darboux_lem_II}).
% \end{Rem*}

With Lemma \autoref{lem:morse_darboux_lem_I} and \autoref{lem:morse_darboux_lem_II} in mind, the rough idea to show the existence of Morse Darboux charts is clear: First, we use Lemma \autoref{lem:morse_darboux_lem_I} to find a chart $\psi_L$ in which all trajectories near $p$ have period $T = \pi$ and the usual Morse index of $\psi_L\circ f$ is $0$. Afterwards, we wish to apply Lemma \autoref{lem:morse_darboux_lem_II} to find a chart $\psi_M$ in which $\omega$ and $H = \psi_L\circ f$ assume their respective standard form. However, the regularity issues come into play here: We cannot apply Lemma \autoref{lem:morse_darboux_lem_II}\footnote{Note, however, that we \underline{can} apply Lemma \autoref{lem:morse_darboux_lem_II} after Lemma \autoref{lem:morse_darboux_lem_I} if all objects in Lemma \autoref{lem:morse_darboux_lem_I} are real analytic, as the chart $\psi_L$ and the Hamilton function $\psi_L\circ f$ are real analytic in this case. Confer Remark \autoref{rem:no_reg_issue} in \autoref{app:morse_darboux_lem} for details.}, since the Hamilton function $\psi_L\circ f$ is, in general, only $C^1$. This is troublesome for several reasons, not the least of which is that the notion of a Morse index does not even make sense for $C^1$-functions.\\
At this point, it is clear that a deeper analysis of this problem is in order, especially if one aims to also discuss the complex case. We stop this excursion here and move on to the main part of this paper.
