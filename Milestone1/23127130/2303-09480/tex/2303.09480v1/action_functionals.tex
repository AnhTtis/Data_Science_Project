\section{Various Action Functionals for HHSs and PHHSs}
\label{app:action_functionals}

In \autoref{subsec:holo_action_fun_and_prin} and \autoref{subsec:def_PHHS}, we have defined and studied action functionals for HHSs and PHHSs. The ``critical points'' of these action functionals gave us (pseudo-)holomorphic trajectories of the system under consideration whose domains are parallelograms in the complex plane. However, these action functionals are not the only functionals whose ``critical points'' can be linked to (pseudo-)holomorphic trajectories. There is, in fact, an abundance of action functionals that differ in the domain of their trajectories and the way the ``one-dimensional'' action functionals of their underlying RHSs are integrated. In this part of the appendix, we present and examine a large selection of such action functionals. First, we only formulate and explore action functionals for HHSs. Afterwards, we explain how these action functionals need to be modified in order to give action functionals for PHHSs. Hereby, we realize that the presented action functionals for PHHSs are all real-valued. From this point of view, a Floer-like theory for PHHSs revolving around these real-valued functionals might be possible.\\
We begin by defining an action functional for holomorphic trajectories whose domains are disks $D^{z_0}_R\subset\mathbb{C}$ of radius $R>0$ centered at $z_0\in\mathbb{C}$. To do that, we first need to partition the disk $D^{z_0}_R$ into lines. We choose the partition consisting of lines starting at the center $z_0$ and ending at any boundary point $z\in\partial D^{z_0}_R$. For every such radial line, we consider the action functional $\mathcal{A}^{\Lambda}_{e^{i\alpha}\mH}$ from Remark \autoref{rem:tilted_traj}. We now obtain an action functional for HHSs by integrating the action $\mathcal{A}^{\Lambda}_{e^{i\alpha}\mH}$ over all radial lines, i.e., $\alpha\in [0,2\pi]$:

\begin{Prop}[Action functional $\actdiski$]\label{prop:actdiski}
 Let $(X,\Omega = d\Lambda, \mH)$ be an exact\linebreak HHS, $D^{z_0}_R\coloneqq\{z\in\mathbb{C}\mid |z-z_0|\leq R\}$ be a disk of radius $R>0$ centered at $z_0\in\mathbb{C}$,\linebreak $\mathcal{P}_{D^{z_0}_R}\coloneqq C^\infty (D^{z_0}_R, X)$ be the set of smooth maps from $D^{z_0}_R$ to $X$, and $\actdiski:\mathcal{P}_{D^{z_0}_R}\to\mathbb{C}$ be the action functional defined by
 \begin{gather*}
  \actdiski [\gamma]\coloneqq \frac{1}{2\pi}\int\limits^{2\pi}_0\int\limits^R_0\left[\Lambda\vert_{\gamma_\alpha (r)}\left(\frac{d\gamma_\alpha}{dr}(r)\right) - e^{i\alpha}\cdot \mH\circ\gamma_\alpha (r)\right]dr\, d\alpha\quad\forall \gamma \in\mathcal{P}_{D^{z_0}_R},
 \end{gather*}
 where $\gamma_\alpha:[0,R]\to X$ is defined by $\gamma_\alpha (r)\coloneqq \gamma (z_0 + re^{i\alpha})$. Now let $\gamma\in\mathcal{P}_{D^{z_0}_R}$. Then, $\gamma$ is a holomorphic trajectory of the HHS $(X,\Omega,\mH)$ iff $\gamma$ is a ``critical point'' of $\actdiski$. Here, ``critical points'' means that we only allow for those variations of $\gamma$ which keep $\gamma$ fixed at the boundary $\partial D^{z_0}_R$ and the \underline{center} $z_0$.
\end{Prop}

\begin{proof}
 Take the notations from above. Using Remark \autoref{rem:tilted_traj} and writing $\actdiski$ as
 \begin{gather*}
  \actdiski [\gamma] = \frac{1}{2\pi}\int\limits^{2\pi}_0\mathcal{A}^{\Lambda}_{e^{i\alpha}\mH} [\gamma_\alpha] d\alpha,
 \end{gather*}
 we can show as in the proof of Lemma \autoref{lem:holo_action_prin} that $\gamma$ is a ``critical point'' of $\actdiski$ iff\linebreak $\gamma_\alpha:[0,R]\to X$ is a (real) integral curve of $\cos (\alpha)\cdot X^R_\mH + \sin (\alpha)\cdot J(X^R_\mH)$ for every $\alpha\in [0,2\pi]$, where $X_\mH = 1/2 (X^R_\mH - iJ(X^R_\mH))$ is the Hamiltonian vector field of $(X,\Omega,\mH)$. Thus, a ``critical point'' $\gamma$ is uniquely determined, given an initial value $x_0\coloneqq\gamma (z_0)$, by:
 \begin{gather*}
   \gamma(z_0 + re^{i\alpha}) = \varphi^{\cos (\alpha)\cdot X^R_\mH + \sin (\alpha)\cdot J(X^R_\mH)}_r (x_0) = \varphi^{r\cos (\alpha)\cdot X^R_\mH + r\sin (\alpha)\cdot J(X^R_\mH)}_1 (x_0),
 \end{gather*}
 where $\varphi^V_t$ is the time-$t$-flow of a real vector field $V$ on $X$. Comparing the last equation with the formula for the holomorphic trajectory $\gamma^{z_0,x_0}$ satisfying $\gamma (z_0) = x_0$ given in the proof of Proposition \autoref{prop:holo_traj} shows that $\gamma$ is a holomorphic trajectory of the HHS $(X,\Omega,\mH)$ iff $\gamma$ is a ``critical point'' of $\actdiski$.\\
 Lastly, we have to explain why a ``critical point'' $\gamma$ of $\actdiski$ needs to fix the variations of $\gamma$ at the boundary $\partial D^{z_0}_R$ and the center $z_0$. Recall the variation of $\mathcal{A}^{\Lambda}_{e^{i\alpha}\mH}$ at $\gamma_\alpha$. In general, the variation of this functional also includes terms associated with the boundary of the image of $\gamma_\alpha$. This boundary consists of two points, namely the center $z_0$ and one boundary point $z\in\partial D^{z_0}_R$. To get rid of these boundary terms in the variation of $\actdiski$, we have to keep $\gamma$ fixed at $z_0$ and $\partial D^{z_0}_R$.
\end{proof}

In \autoref{subsec:holo_action_fun_and_prin}, we have given two reasons why we need to vary over all smooth curves $\gamma$ and cannot simple restrict the variational problem to the space of holomorphic curves $\gamma$. The new-found action functional offers an additional perspective on that matter. It maps every holomorphic curve $\gamma$ to zero, hence, only varying it over the space of holomorphic curves is meaningless:

\begin{Prop}\label{prop:actdiski_value}
 Take the assumptions and notations from Proposition \autoref{prop:actdiski}. Further, let $\gamma:D^{z_0}_R\to X$ be any holomorphic map from $D^{z_0}_R$ to $X$. Then:
 \begin{gather*}
  \actdiski [\gamma] = 0.
 \end{gather*}
\end{Prop}

\begin{proof}
 Take the assumptions and notations from Proposition \autoref{prop:actdiski} and let $\gamma:D^{z_0}_R\to X$ be holomorphic. Using the relation $\Lambda (J\cdot) = i\cdot \Lambda$ for holomorphic $1$-forms, we find:
 \begin{gather*}
  \Lambda\vert_{\gamma_\alpha(r)}\left(\frac{d\gamma_\alpha}{dr}(r)\right) = e^{i\alpha}\cdot\Lambda\vert_{\gamma_\alpha (r)}\left( \gamma^{\prime}(z_0 + re^{i\alpha})\right),
 \end{gather*}
 where $\gamma^{\prime}$ is the complex derivative of $\gamma$. With this, we obtain:
 \begin{align*}
  \actdiski [\gamma] &= \frac{1}{2\pi}\int\limits^{2\pi}_0\int\limits^R_0\left[\Lambda\vert_{\gamma_\alpha (r)}\left(\frac{d\gamma_\alpha}{dr}(r)\right) - e^{i\alpha}\cdot \mH\circ\gamma_\alpha (r)\right]dr\, d\alpha\\
  &= \frac{1}{2\pi}\int\limits^{2\pi}_0\int\limits^R_0\left[\Lambda\vert_{\gamma (z_0 + re^{i\alpha})}\left(\gamma^\prime (z_0 + re^{i\alpha})\right) - \mH\circ\gamma (z_0 + re^{i\alpha})\right]dr\, e^{i\alpha}\, d\alpha\\
  &= \int\limits^R_0 \frac{1}{2\pi i}\oint\limits_{|z| = 1}\left[\Lambda\vert_{\gamma (z_0 + rz)}\left(\gamma^\prime (z_0 + rz)\right) - \mH\circ\gamma (z_0 + rz)\right]dz\, dr\\
  &= \int\limits^R_0 \text{\normalfont Res}_{z = 0}\left[\Lambda\vert_{\gamma (z_0 + rz)}\left(\gamma^\prime (z_0 + rz)\right) - \mH\circ\gamma (z_0 + rz)\right] dr,
 \end{align*}
 where $\text{\normalfont Res}_{z = 0} [f(z)]$ denotes the residue of a meromorphic function $f(z)$ at $z = 0$. In the last line of the computation, we have used Cauchy's theorem. The function for which we need to determine the residue is clearly holomorphic at $z = 0$. Thus, the residue is $0$ and we see that the action vanishes for holomorphic curves concluding the proof.
\end{proof}

Even though the ``critical values'' of $\actdiski$ are nice and easy to understand, the action functional itself does not appear to be particularly useful. Often, we want to modify action functionals such that trajectories become actual critical points. The standard ways to achieve this are to either put the boundary of the trajectory on Lagrangian submanifolds or to impose periodicity. Both ways do not appear to be meaningful here. For the presented action functional, periodicity means periodicity of the radial lines. Thus, a ``periodic'' curve $\gamma:D^{z_0}_R\to X$ needs to attain the same value on its boundary as on its center. However, the only holomorphic maps $\gamma:D^{z_0}_R\to X$ exhibiting such a behavior are constant curves by the identity theorem.\\
The other method, mapping the ``boundary'' to Lagrangian submanifolds, takes an unnatural and downright ugly form here, namely mapping $z_0$ and $\partial D^{z_0}_R$ to Lagrangian submanifolds. The action functional $\actdiskii$ improves on $\actdiski$ in that regard. To avoid boundary terms associated with $z_0$, which are at the center\footnote{Cum grano salis.} of our problem, we now partition the disk $D^{z_0}_R$ into lines starting at $z_0 - z$ and ending at $z_0 + z$ ($|z| = R$). In order to account for the doubled length of the radial lines, we only integrate over the angles $\alpha\in [0,\pi]$ this time:

\begin{Prop}[Action functional $\actdiskii$]\label{prop:actdiskii}
 Let $(X,\Omega = d\Lambda, \mH)$ be an exact\linebreak HHS, $D^{z_0}_R\coloneqq\{z\in\mathbb{C}\mid |z-z_0|\leq R\}$ be a disk of radius $R>0$ centered at $z_0\in\mathbb{C}$,\linebreak $\mathcal{P}_{D^{z_0}_R}\coloneqq C^\infty (D^{z_0}_R, X)$ be the set of smooth maps from $D^{z_0}_R$ to $X$, and $\actdiskii:\mathcal{P}_{D^{z_0}_R}\to\mathbb{C}$ be the action functional defined by
 \begin{gather*}
  \actdiskii [\gamma]\coloneqq \frac{i}{4R}\int\limits^{\pi}_0\int\limits^R_{-R}\left[\Lambda\vert_{\gamma_\alpha (r)}\left(\frac{d\gamma_\alpha}{dr}(r)\right) - e^{i\alpha}\cdot \mH\circ\gamma_\alpha (r)\right]dr\, d\alpha\quad\forall \gamma \in\mathcal{P}_{D^{z_0}_R},
 \end{gather*}
 where $\gamma_\alpha:[-R,R]\to X$ is defined by $\gamma_\alpha (r)\coloneqq \gamma (z_0 + re^{i\alpha})$. Now let $\gamma\in\mathcal{P}_{D^{z_0}_R}$. Then, $\gamma$ is a holomorphic trajectory of the HHS $(X,\Omega,\mH)$ iff $\gamma$ is a ``critical point'' of $\actdiskii$. Here, ``critical points'' means that we only allow for those variations of $\gamma$ which keep $\gamma$ fixed at the boundary $\partial D^{z_0}_R$.
\end{Prop}

\begin{proof}
 The proof works as the proof of Proposition \autoref{prop:actdiski} by writing $\actdiskii$ as
 \begin{gather*}
  \actdiskii [\gamma] = \frac{i}{4R}\int\limits^{\pi}_0\mathcal{A}^{\Lambda}_{e^{i\alpha}\mH} [\gamma_\alpha] d\alpha.
 \end{gather*}
 Here, the variations of $\gamma$ only need to keep $\gamma$ fixed at the boundary $\partial D^{z_0}_R$, since the radial lines start and end at $\partial D^{z_0}_R$.
\end{proof}

\begin{Rem}[No Proposition \autoref{prop:actdiski_value} for $\actdiskii$]\label{rem:actdiskii_value}
 Proposition \autoref{prop:actdiski_value} does not apply to $\actdiskii$. In fact, the normalization in Proposition \autoref{prop:actdiskii} is chosen such that the action of constant curves is given by the Hamilton function:
 \begin{gather*}
  \actdiskii [\gamma_{x_0}] = \mH (x_0),
 \end{gather*}
 where $\gamma_{x_0} (z)\coloneqq x_0\in X$ for every $z\in D^{z_0}_{R}$. Thus, any singular point $x_0$ of $\mH$ with $\mH (x_0)\neq 0$ provides a counterexample to Proposition \autoref{prop:actdiski_value} for $\actdiskii$.
\end{Rem}

If we modify $\actdiskii$ such that the holomorphic trajectories become actual critical points, we see that this action is a bit more reasonable. In the Lagrangian case, we now restrict the space of smooth curves $\gamma:D^{z_0}_R\to X$ to the space of those curves which only map the boundary $\partial D^{z_0}_R$ to Lagrangian submanifolds, as one would expect. However, the modification via periodicity still only gives trivial results. One can see this as follows: now, periodicity means periodicity of radial lines starting and ending at $\partial D^{z_0}_R$. In this sense, we say $\gamma:D^{z_0}_R\to X$ is ``periodic'' if it assigns the same value to opposite points on the boundary $\partial D^{z_0}_R$. For the sake of simplicity, let us now assume $z_0 = 0$. For such a ``periodic'' $\gamma$, define $\gamma_-$ by $\gamma_- (z)\coloneqq \gamma (-z)$. If $\gamma$ is holomorphic, then $\gamma_-$ is also holomorphic and, by assumption, attains the same values on $\partial D^{0}_R$ as $\gamma$. Hence, by the identity theorem, $\gamma$ and $\gamma_-$ denote the same map. However, if $\gamma$ is even a holomorphic trajectory, then $\gamma$ is an integral curve of the Hamiltonian vector field $X_\mH$ and we have:
\begin{gather*}
 X_\mH (\gamma (z)) = \frac{\pa}{\pa z}(\gamma (z)) = \frac{\pa}{\pa z} (\gamma (-z)) = - X_\mH (\gamma (-z)) = -X_\mH (\gamma (z)).
\end{gather*}
Thus, the Hamiltonian vector field vanishes in this case and $\gamma$ is a constant curve.\\
We cannot only formulate $\actdiski$ and $\actdiskii$ for disks $D^{z_0}_R$, but for any bounded star-shaped domain\footnote{Here, a domain $D\subset\mathbb{C}$ is a path-connected subset of $\mathbb{C}$ with non-empty interior $D^\circ$ dense in $D$.} $D\subset\mathbb{C}$ with smooth boundary\footnote{The boundary $b$ is parameterized by the polar angle $\alpha$ in the decomposition $z = z_0 + re^{i\alpha}\in D$.} $b:\mathbb{R}/2\pi\mathbb{Z}\to \partial D$:

\begin{Prop}[Action functionals $\acti$ and $\actii$ for bounded star-shaped domains] \label{prop:action_starshaped}
 Let $(X,\Omega = d\Lambda, \mH)$ be an exact HHS, let $D\subset\mathbb{C}$ be a bounded domain in $\mathbb{C}$ which is star-shaped with respect to $z_0$ and has smooth boundary $b:\mathbb{R}/2\pi\mathbb{Z}\to\partial D$, and let $\mathcal{P}_{D}\coloneqq C^\infty (D, X)$ be the set of smooth maps from $D$ to $X$. Then, we can define the action functionals $\acti:\mathcal{P}_{D}\to\mathbb{C}$ and $\actii:\mathcal{P}_D\to \mathbb{C}$ by
 \begin{align*}
  \acti [\gamma]&\coloneqq \frac{1}{2\pi}\int\limits^{2\pi}_0\int\limits^{R(\alpha)}_0\left[\Lambda\vert_{\gamma_\alpha (r)}\left(\frac{d\gamma_\alpha}{dr}(r)\right) - e^{i\alpha}\cdot \mH\circ\gamma_\alpha (r)\right]dr\, d\alpha,\\
  \actii [\gamma]&\coloneqq \frac{i}{4\hat{R}}\int\limits^{\pi}_0\int\limits^{R(\alpha)}_{-R(\alpha - \pi)}\left[\Lambda\vert_{\gamma_\alpha (r)}\left(\frac{d\gamma_\alpha}{dr}(r)\right) - e^{i\alpha}\cdot \mH\circ\gamma_\alpha (r)\right]dr\, d\alpha,
 \end{align*}
 where $\gamma\in\mathcal{P}_D$, $R:\mathbb{R}/2\pi\mathbb{Z}\to\mathbb{R}$ is defined by $R(\alpha)\coloneqq |b(\alpha)-z_0|$, $\gamma_\alpha:[-R(\alpha - \pi), R(\alpha)]\to X$ is given by $\gamma_\alpha (r)\coloneqq \gamma(z_0 + re^{i\alpha})$, and $\hat{R}$ is defined by
 \begin{gather*}
  \hat{R}\coloneqq \frac{i}{4}\left[\int\limits^{2\pi}_{\pi} R(\alpha)e^{i\alpha} d\alpha - \int\limits^{\pi}_{0} R(\alpha)e^{i\alpha} d\alpha\right].
 \end{gather*}
 Now let $\gamma\in\mathcal{P}_{D}$. Then, $\gamma$ is a holomorphic trajectory of the HHS $(X,\Omega,\mH)$ iff $\gamma$ is a ``critical point''\footnote{In the sense of Proposition \autoref{prop:actdiski}.} of $\acti$ iff $\gamma$ is a ``critical point''\footnote{In the sense of Proposition \autoref{prop:actdiskii}.} of $\actii$.
\end{Prop}

\begin{proof}
 Confer the proofs of Proposition \autoref{prop:actdiski} and \autoref{prop:actdiskii}.
\end{proof}

\begin{Rem}[Normalization of $\acti$ and $\actii$]\label{rem:normalization}
 The normalization of $\acti$ and $\actii$ are chosen such that they coincide with our previous definitions for $D = D^{z_0}_R$ being a disk. In particular, $\actii$ agrees with the Hamilton function $\mH$ for constant curves $\gamma$.
\end{Rem}

One might wonder how the action functionals $\acti$ and $\actii$ are related to the action functional $\mathcal{A}^{P_\alpha}_\mH$ for parallelograms $P_\alpha$ from \autoref{subsec:holo_action_fun_and_prin} and \autoref{subsec:def_PHHS}, especially because a parallelogram $P_\alpha$ is also a bounded star-shaped domain in $\mathbb{C}$. When we modify the functionals $\acti$ and $\actii$ to describe general PHHSs, we will see that $\acti$ and $\actii$ differ a lot from $\mathcal{A}^{P_\alpha}_\mH$. To compare $\mathcal{A}^{P_\alpha}_\mH$ directly with $\acti$ and $\actii$, let us express $\acti$ and $\actii$ in the same coordinates as $\mathcal{A}^{P_\alpha}_\mH$, namely Cartesian coordinates $z = t + is$:

\begin{Prop}[$\acti$ and $\actii$ in Cartesian coordinates]\label{prop:cartesian_coordinates}
 Employ the assumptions and notations from Proposition \autoref{prop:action_starshaped}. For $\gamma\in\mathcal{P}_D$, we define the following derivatives in Cartesian coordinates $z = t + is \in D$:
 \begin{gather*}
  \frac{\partial\gamma}{\partial z} (z)\coloneqq \frac{1}{2}\left(\frac{\partial\gamma}{\partial t} (z) - i\frac{\partial\gamma}{\partial s} (z)\right);\quad \frac{\partial\gamma}{\partial \bar{z}} (z)\coloneqq \frac{1}{2}\left(\frac{\partial\gamma}{\partial t} (z) + i\frac{\partial\gamma}{\partial s} (z)\right).
 \end{gather*}
 Furthermore, define the complex functions $f,g:D\to\mathbb{C}$ by:
 \begin{gather*}
  f(z)\coloneqq \Lambda\vert_{\gamma (z)}\left(\frac{\partial\gamma}{\partial z}(z)\right) - \mH\circ\gamma (z);\quad g(z)\coloneqq \Lambda\vert_{\gamma (z)}\left(\frac{\partial\gamma}{\partial \bar{z}}(z)\right).
 \end{gather*}
 Then, the action functionals $\acti$ and $\actii$ in Cartesian coordinates are given by:
 \begin{align*}
  \acti [\gamma] &= \frac{1}{2\pi}\iint\limits_{D}\left[\frac{f(z)}{\bar{z} - \bar{z_0}} + \frac{g(z)}{z - z_0}\right]dt\wedge ds,\\
  \actii [\gamma] &= \frac{i}{4\hat{R}}\iint\limits_{D^+}\left[\frac{f(z)}{\bar{z} - \bar{z_0}} + \frac{g(z)}{z - z_0}\right]dt\wedge ds - \frac{i}{4\hat{R}}\iint\limits_{D^-}\left[\frac{f(z)}{\bar{z} - \bar{z_0}} + \frac{g(z)}{z - z_0}\right]dt\wedge ds,
 \end{align*}
 where $z = t+is\in D$, $\bar{\cdot}$ denotes the complex conjugation, $D^+\coloneqq \{z\in D\mid \text{\normalfont Im}(z-z_0)\geq 0\}$, and\linebreak $D^-\coloneqq \{z\in D\mid \text{\normalfont Im}(z-z_0)\leq 0\}$.
\end{Prop}

\begin{proof}
 Take the assumptions and notations from above. We only show Proposition \autoref{prop:cartesian_coordinates} for $D = D^0_R\equiv D_R$ being a disk of radius $R>0$ centered at the origin. The general case can be shown similarly. Using the derivatives defined above, we can write:
 \begin{gather*}
  \frac{\partial\gamma}{\partial t} (z) = \frac{\partial\gamma}{\partial z} (z) + \frac{\partial\gamma}{\partial \bar{z}} (z);\quad \frac{\partial\gamma}{\partial s} (z) = i\left(\frac{\partial\gamma}{\partial z} (z) - \frac{\partial\gamma}{\partial \bar{z}} (z)\right).
 \end{gather*}
 Now consider the map $\gamma_\alpha:[-R,R]\to X$ defined in polar coordinates $z = re^{i\alpha}$ by\linebreak $\gamma_\alpha (r) = \gamma (re^{i\alpha})$ for every $\alpha\in [0,2\pi]$. $\Lambda$ applied to the derivative of $\gamma_\alpha$ gives:
 \begin{alignat*}{2}
  \Lambda\vert_{\gamma_\alpha (r)}\left(\frac{d\gamma_\alpha}{dr} (r)\right) &= \cos (\alpha)\cdot \Lambda\vert_{\gamma (re^{i\alpha})}\left(\frac{\partial\gamma}{\partial t} (re^{i\alpha})\right) &&+ \sin (\alpha)\cdot \Lambda\vert_{\gamma (re^{i\alpha})}\left(\frac{\partial\gamma}{\partial s} (re^{i\alpha})\right)\\
  &= e^{i\alpha}\cdot \Lambda\vert_{\gamma (re^{i\alpha})}\left(\frac{\partial\gamma}{\partial z} (re^{i\alpha})\right) &&+ e^{-i\alpha}\cdot \Lambda\vert_{\gamma (re^{i\alpha})}\left(\frac{\partial\gamma}{\partial \bar{z}} (re^{i\alpha})\right).
 \end{alignat*}
 Recalling the definition of $f$ and $g$, this allows us to write:
 \begin{align*}
  \Lambda\vert_{\gamma_\alpha (r)}\left(\frac{d\gamma_\alpha}{dr}(r)\right) - e^{i\alpha}\cdot \mH\circ\gamma_\alpha (r) &= e^{i\alpha}\cdot f(re^{i\alpha}) + e^{-i\alpha}\cdot g(re^{i\alpha})\\
  &= r\cdot\left[\frac{f(re^{i\alpha})}{re^{-i\alpha}} + \frac{g(re^{i\alpha})}{re^{i\alpha}}\right] = r\cdot\left[\frac{f(z)}{\bar{z}} + \frac{g(z)}{z}\right].
 \end{align*}
 The expression for $\acti$ is now obtained by inserting the last equation into the defining formula for $\acti$ and using $r\cdot dr\wedge d\alpha = dt\wedge ds$ for $re^{i\alpha} = z = t + is$. Observing that the integrands of $\acti$ and $\actii$ agree on $D^+_R$ and differ by a sign on $D^-_R$ concludes the proof.
\end{proof}

%\footnote{We have renamed the angle $\alpha$ of the parallelogram $P_\alpha$ to $\beta$ here to avoid confusion with the polar coordinates $z = re^{i\alpha}$.}

The form of $\acti$ and $\actii$ in Cartesian coordinates is rather remarkable. In fact, we can express the action functional $\mathcal{A}^{P_\alpha}_\mH$ from \autoref{subsec:holo_action_fun_and_prin} in the same form as $\acti$ for $D = P_\alpha$, just with different complex functions $f$ and $g$, namely by setting:
\begin{gather*}
 f(z) = 2\pi\overline{(z - z_0)}\cdot\left[\Lambda_R\vert_{\gamma (z)}\left(2\frac{\partial\gamma}{\partial z}(z)\right) - \mH\circ\gamma (z)\right];\quad g(z) = 0.
\end{gather*}
The similarities between $\acti$ and $\mathcal{A}^{P_\alpha}_\mH$ become even more apparent if we evaluate $\mathcal{A}^{P_\alpha}_\mH$ at holomorphic curves $\gamma:P_\alpha\to X$. As in Remark \autoref{rem:several_remarks}, Point 3, we find in this case:
\begin{gather*}
 \mathcal{A}^{P_\alpha}_\mH[\gamma] =  \iint\limits_{P_\alpha}\left[\Lambda\left(\frac{\partial\gamma}{\partial z}(z)\right) - \mH\circ\gamma (z)\right] dt\wedge ds.
\end{gather*}
Because the function $g$ as defined in Proposition \autoref{prop:cartesian_coordinates} vanishes for holomorphic $\gamma$, the only difference between $\acti$ and $\mathcal{A}^{P_\alpha}_\mH$ is now the factor $2\pi\overline{(z - z_0)}$ in the integrand. However, this seemingly small difference is rather impactful, as we will shortly see.\\
Lastly, we want to modify the actions $\acti$ and $\actii$ in such a way that they also apply to PHHSs. Recall that for an exact PHHS $(X,J;\Omega_R = d\Lambda_R,\mH_R)$ the induced $2$-form $\Omega_I$ is, in general, not closed. Hence, only the parts of $\acti$ and $\actii$ that do not include $\Lambda_I$ are well-defined for PHHSs. Precisely speaking, these are the real part of $\acti$ and the real part of $-4i\hat R\cdot \actii$. Still, these \underline{real-valued} functionals satisfy an action principle with respect to the pseudo-holomorphic trajectories of a PHHS. In fact, this can be shown in the same way as Proposition \autoref{prop:actdiski} by simply observing that Remark \autoref{rem:tilted_traj}, which is crucial for the proof of Proposition \autoref{prop:actdiski}, is also valid for the real part of the functional $\mathcal{A}^\Lambda_{e^{i\alpha}\mH}$.\\
The generalization of $\acti$ and $\actii$ to PHHSs has now revealed the most striking difference between $\acti$ and $\mathcal{A}^{P_\alpha}_\mH$: while the action principle related to $\acti$ still applies if we only consider the real part of $\acti$, both the real \underline{and} imaginary part of $\mathcal{A}^{P_\alpha}_\mH$ are crucial for the validity of the action principle related to $\mathcal{A}^{P_\alpha}_\mH$. In particular, there might exist a Floer-like theory related to the real part of $\acti$ or $-4i\hat R\cdot \actii$. For $\mathcal{A}^{P_\alpha}_\mH$, we have no intuition on how such a theory should look like, since we do not know how to interpret a complex function as a Morse function in the sense of Morse homology. Since we can turn $\acti$ and $\actii$ into real-valued action functionals, the same objections do not apply to them. Nevertheless, the question remains whether the real part of $\acti$ or $-4i\hat R\cdot \actii$ are indeed suitable Morse functions and whether the resulting Floer theories, if they exist, give any non-trivial result. At least for the (conjectured) Hamiltonian\footnote{In Hamiltonian Floer theory, one only considers periodic orbits.} Floer theory related to $\acti$ and $\actii$, it is most likely that it only gives trivial results, since all trajectories, which are ``periodic'' in a sense suitable for $\acti$ and $\actii$ as explained above, are automatically constant.
