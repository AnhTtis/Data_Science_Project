\section{Real Hamiltonian Systems}
\label{sec:RHS}

This section serves three purposes: Firstly, it is a reminder of the basic concepts of Hamiltonian mechanics, secondly, it allows us to fix signs and notations for the rest of the paper, and, lastly, it doubles as a blueprint for \autoref{sec:HHS}. The reader familiar with symplectic geometry and classical mechanics may skip this section entirely.


\subsection{RHS: Basic Definitions and Notions}
\label{subsec:def_RHS}

To start, we recall the definition of a symplectic manifold $(M,\omega)$ and a real Hamiltonian system $(M,\omega, H)$:
\begin{Def}[Symplectic manifold and real Hamiltonian system]\label{def:real_sym_man_and_ham_sys}
 A \textbf{symplectic manifold} $(M,\omega)$ is a smooth (real) manifold $M$ together with a non-degenerate and closed $2$-form $\omega$ on it. The triple $(M,\omega, H)$ consisting of a symplectic manifold $(M,\omega)$ and smooth function $H\in C^\infty (M,\mathbb{R})$ on $M$ is called \textbf{real Hamiltonian system} (RHS). In this setup, $H$ is usually called the \textbf{Hamilton function} or, simply, the Hamiltonian. A symplectic manifold $(M,\omega)$ or RHS $(M,\omega, H)$ is called \textbf{exact} iff $\omega$ is exact, i.e., possesses a primitive $\lambda$: $\omega = d\lambda$.
\end{Def}
RHSs are of particular interest in physics and classical mechanics, as they can be used to describe the dynamics of non-relativistic, non-quantized point-like particles. Hereby, the manifold $M$ of a RHS $(M,\omega, H)$ is identified with the phase space of a classical system, i.e., the space of ``states'' a classical system may assume. Usually, we can think of the phase space $M$ as the space of general positions $q_i$ of some particles together with their conjugate momenta $p_i$. This analogy can be made apparent by choosing $M = T^\ast Q$ to be the cotangent bundle of some configuration (position) space $Q$ together with the canonical symplectic $2$-form $\omega_\text{can} = d\lambda_\text{can}$, where $\lambda_\text{can}$ is the Liouville $1$-form, or by going into Darboux charts of the symplectic manifold $(M,\omega)$, i.e., by going into charts $(U, \psi = (q_1,\ldots, q_m,p_1,\ldots, p_m))$ of $M$ ($\text{dim}_\mathbb{R}(M) = 2m$) for which $\omega$ takes the following form:
\begin{gather*}
 \omega\vert_U = \sum^m_{i=1} dp_i\wedge dq_i.
\end{gather*}
By Darboux's theorem, every symplectic manifold $(M,\omega)$ admits such Darboux charts near any point of $M$.\\
The Hamilton function $H$ of a RHS $(M,\omega, H)$ describes the energy a point/state in the phase space $M$ assumes and is, therefore, often called the energy function in classical mechanics. Together with the symplectic form $\omega$, the Hamilton function $H$ completely determines the dynamics of the RHS $(M,\omega, H)$. To see this, let us consider the trajectories of the RHS $(M,\omega, H)$.


\subsection{Trajectories}
\label{subsec:traj}

First, we recall the definition of the (real) Hamiltonian vector field $X_H$ in order to specify the trajectories of a RHS $(M,\omega, H)$:
\begin{Def}[Real Hamiltonian vector field and trajectories]\label{def:real_ham_field_and_traj}
 Let $(M,\omega, H)$ be a RHS. We call the vector field $X_H$ on $M$ defined by $\iota_{X_H}\omega = -dH$ the (real) \textbf{Hamiltonian vector field} of the RHS $(M,\omega, H)$. Now let $I\subset\mathbb{R}$\footnote{$I$ plays the role of time in physics.} be a connected and open subset, i.e., an interval. We call a smooth curve $\gamma\in C^\infty (I,M)$ in $M$ a \textbf{trajectory} of the RHS $(M,\omega, H)$ iff $\gamma$ satisfies the integral curve equation
 \begin{gather*}
  \dot\gamma (t) = X_H (\gamma (t))\ \forall t\in I.
 \end{gather*}
 A trajectory $\gamma:I\to M$ of $(M,\omega, H)$ is called \textbf{maximal} iff for every trajectory $\hat\gamma:\hat I\to M$ of $(M,\omega, H)$ satisfying $I\subset \hat I$ and $\hat\gamma\vert_I = \gamma$ one has $\hat I = I$ and $\hat\gamma = \gamma$.
\end{Def}
\begin{Rem*}[$X_H$ well-defined]
 As the symplectic form $\omega$ of a RHS $(M,\omega, H)$ is non-degenerate, the Hamiltonian vector field $X_H$ is well-defined.
\end{Rem*}
To relate the given definition of trajectory to the notion known from classical mechanics, let us analyze the integral curve equation in Darboux charts: Let $(U,\psi = (q_1,\ldots, q_m, p_1,\ldots, p_m))$ be a Darboux chart of $(M,\omega)$, then the Hamiltonian vector field $X_H$ is given on $U$ by:
\begin{gather*}
 X_H\vert_U = \sum^m_{i=1} \frac{\partial H}{\partial p_i}\cdot \frac{\partial}{\partial q_i} - \frac{\partial H}{\partial q_i}\cdot \frac{\partial}{\partial p_i}.
\end{gather*}
For simplicity, let us assume that the image of the trajectory $\gamma:I\to M$ is\linebreak completely contained in $U$. Then, we can express $\gamma$ in the chart $(U,\psi)$ by\linebreak $\psi\circ \gamma(t)\coloneqq (q_1(t),\ldots, q_m(t), p_1(t),\ldots, p_m(t))$. With these notations, the integral curve equation becomes:
\begin{gather*}
 \dot q_i(t) = \frac{\partial H}{\partial p_i};\quad \dot p_i(t) = -\frac{\partial H}{\partial q_i}\quad\forall t\in I\ \forall i\in\{1,\ldots, m\},
\end{gather*}
where, of course, $\frac{\partial H}{\partial q_i}$ and $\frac{\partial H}{\partial p_i}$ are just abbreviations for $\frac{\partial H}{\partial q_i}(\gamma (t))$ and $\frac{\partial H}{\partial p_i}(\gamma (t))$, respectively. These equations are known to physicists as the Hamilton equations which describe the motion of classical particles. This shows that the symplectic description of Hamiltonian systems is equivalent to the Hamiltonian formulation of classical mechanics.\\
Before we move on to the action principle, let us quickly comment on the existence and uniqueness of trajectories. As the integral curve equation is just a first-order differential equation, the following remark is an immediate consequence of standard results from analysis:
\begin{Rem}[Existence and uniqueness of trajectories]\label{rem:traj}
 Let $(M,\omega, H)$ be a RHS and $x\in M$ be any point. Then, for any $t_0\in\mathbb{R}$ there exists an interval $I\subset\mathbb{R}$ with $t_0\in I$\linebreak and a trajectory $\gamma:I\to M$ of $(M,\omega, H)$ satisfying $\gamma (t_0) = x$. In particular, if\linebreak $\gamma_1,\gamma_2:I\to M$ are two trajectories of $(M,\omega, H)$ with the same domain satisfying\linebreak $\gamma_1 (t_0) = x = \gamma_2 (t_0)$, then $\gamma_1 = \gamma_2$. Furthermore, for any initial value $x\in M$ and $t_0\in\mathbb{R}$ there exists a maximal trajectory $\gamma$ of $(M,\omega, H)$ satisfying $\gamma (t_0) = x$ which is also unique by the previous statements.
\end{Rem}


\subsection{Action Functional and Principle}
\label{subsec:action_fun_and_prin}

In physics, the Hamilton equations are usually derived from a fundamental principle, namely the action principle. To understand this, let us define the action functional $\mathcal{A}_H$ of a RHS $(M,\omega, H)$:
\begin{Def}[Action functional]\label{def:real_action}
 Let $(M,\omega = d\lambda, H)$ be an exact RHS, $I\subset\mathbb{R}$ be some interval, and $\mathcal{P}_I\coloneqq C^\infty (I,M)$ be the space of smooth paths in $M$ with domain $I$. We define the \textbf{action functional} $\mathcal{A}_H:\mathcal{P}_I\to\mathbb{R}$ as follows:
 \begin{gather*}
  \mathcal{A}_H[\gamma]\equiv \mathcal{A}^\lambda_H[\gamma]\coloneqq \int\limits_I \gamma^\ast\lambda - \int\limits_I H\circ\gamma (t)\, dt\quad\forall \gamma\in\mathcal{P}_I.
 \end{gather*}
 $\mathcal{A}_H$ is often simply referred to as the action of the RHS $(M,\omega, H)$ in physics. 
\end{Def}
The action principle now states that the trajectories of the RHS $(M,\omega, H)$ are paths of stationary action. Mathematically speaking, this means that the curve $\gamma\in\mathcal{P}_I$ is a trajectory of the RHS $(M,\omega, H)$ iff $\gamma$ is -- in a suitable sense -- a critical point of the action $\mathcal{A}_H$. To be precise, the first variation of $\mathcal{A}_H$ at $\gamma$ has to vanish, whereby the variation of the action $\mathcal{A}_H$ is understood with fixed endpoints, i.e., only variations of $\gamma$ are admissible which keep the endpoints fixed. A short calculation reveals that the critical points (with fixed endpoints) of the action functional $\mathcal{A}_H$ are indeed the integral curves of the Hamiltonian vector field $X_H$, coinciding with our prior definitions:
\begin{Prop}[Action principle]\label{prop:real_action_principle}
 Let $(M,\omega = d\lambda, H)$ be an exact RHS, $I\subset\mathbb{R}$ be some interval, $\gamma\in\mathcal{P}_I$ be a path in $M$, and $\mathcal{A}_H:\mathcal{P}_I\to\mathbb{R}$ be the action of $(M,\omega, H)$. Then, $\gamma$ is a trajectory of $(M,\omega, H)$ iff $\gamma$ is a critical point of $\mathcal{A}_H$ (with fixed endpoints).
\end{Prop}
\begin{proof}
 We only sketch the proof here, as a detailed computation can be found in most textbooks on classical mechanics. Take the notations from above and denote $(a,b)\coloneqq I$. For $c>0$, we call a smooth map $\Gamma:I\times(-c, c)\to M$ a variation of $\gamma$, if $\Gamma$ satisfies the following two properties:
 \begin{enumerate}
  \item $\Gamma (t, 0) = \gamma (t)\quad\forall t\in I$,
  \item $\Gamma (a, \epsilon) = \gamma (a)$ and $\Gamma (b,\epsilon) = \gamma (b)\quad \forall\epsilon\in (-c,c)$.
 \end{enumerate}
 Usually, we denote a variation of $\gamma$ by $\gamma_\epsilon (t) = \Gamma (t,\epsilon)$. We want to compute the first of variation of $\mathcal{A}_H$ at the point $\gamma\in\mathcal{P}_I$. To do so, we need to calculate the derivative
 \begin{gather*}
  \left.\frac{d}{d\epsilon}\right\vert_{\epsilon = 0}\mathcal{A}_H[\gamma_\epsilon]
 \end{gather*}
 for every variation $\gamma_\epsilon$ of $\gamma$. One can show that this derivative is given by
 \begin{align*}
  \left.\frac{d}{d\epsilon}\right\vert_{\epsilon = 0}\mathcal{A}_H[\gamma_\epsilon] =& \int\limits^b_a \omega\vert_{\gamma (t)}\left(\left.\frac{\partial \gamma_\epsilon}{\partial \epsilon} (t)\right\vert_{\epsilon = 0}, \dot\gamma (t) - X_H (\gamma (t))\right)\ dt\\
  &+ \lambda\vert_{\gamma (b)} \left(\left.\frac{\partial \gamma_\epsilon}{\partial \epsilon} (b)\right\vert_{\epsilon = 0}\right) - \lambda\vert_{\gamma (a)} \left(\left.\frac{\partial \gamma_\epsilon}{\partial \epsilon} (a)\right\vert_{\epsilon = 0}\right),
 \end{align*}
 where $\gamma_\epsilon$ is any variation of $\gamma$. Since $\gamma_\epsilon$ is kept fixed at the endpoints, the boundary terms involving $\lambda$ vanish. Now, we say that $\gamma$ is a critical point of the action functional $\mathcal{A}_H$ iff the first variation of $\mathcal{A}_H$ vanishes at $\gamma$, i.e., the last equation is zero for every variation $\gamma_\epsilon$ of $\gamma$. This is only possible iff $\gamma$ satisfies the equation
 \begin{gather*}
  \dot\gamma (t) = X_H (\gamma (t))\quad\forall t\in I,
 \end{gather*}
 i.e., $\gamma$ is a trajectory.
\end{proof}
\begin{Rem*}[Trajectories as actual critical points of the action]
 Often, one wants to understand trajectories as actual critical points of some action functional, not just with fixed endpoints. There are two main ways to achieve this:
 \begin{enumerate}
  \item One restricts themselves to paths which only start and end at points where the primitive $\lambda$ vanishes, usually Lagrangian submanifolds of $(M,\omega)$.
  \item One restricts themselves to periodic paths such that the boundary terms in the first variation of $\mathcal{A}_H$ cancel each other.
 \end{enumerate}
\end{Rem*}
