\section{Proof of Darboux's Theorem for HSMs}
\label{app:darboux}

In this part of the appendix, we want to show that there is a holomorphic counterpart to Darboux's theorem.

\begin{Thm*}[Darboux's theorem for HSMs]
 Let $(X,\Omega)$ be a holomorphic symplectic manifold (HSM) of complex dimension $\text{\normalfont dim}_\mathbb{C}(X) = 2n$ $(n\in\mathbb{N})$. Then, for every point $x\in X$, there is a holomorphic chart $\psi = (Q_1,\ldots, Q_n, P_1,\ldots, P_n):U\to V\subset\mathbb{C}^{2n}$ of $X$ near $x$ such that
 \begin{gather*}
  \Omega\vert_{U} = \sum\limits^n_{j = 1} dP_j\wedge dQ_j.
 \end{gather*}
\end{Thm*}

\begin{proof}
 The proof of Darboux's theorem for HSMs works almost completely analogously to the proof of Darboux's theorem in the real setup given by Weinstein. In particular, we will apply Moser's trick to HSMs. A detailed transcription of Moser's trick to complex manifolds can be found in \cite{soldatenkov2021} by Soldatenkov and Verbitsky.\\
 Let $(X,\Omega)$ be a HSM of complex dimension $\text{\normalfont dim}_\mathbb{C}(X) = 2n$ ($n\in\mathbb{N}$), $x\in X$ be any point of $X$, and $(\hat U, \hat\psi = (\hat z_1,\ldots, \hat z_{2n}))$ be a holomorphic chart of $X$ near $x$. First, we observe that every complex symplectic form $\omega$ on a complex vector space $V$ of dimension $\text{\normalfont dim}_\mathbb{C}(V) = 2n$ can be brought into standard form, i.e., can be written as
 \begin{gather*}
  \omega = \sum\limits^n_{j=1} \theta_{j+n}\wedge\theta_{j}
 \end{gather*}
 for a basis $(\theta_1,\ldots,\theta_{2n})$ dual to some complex basis $(e_1,\ldots, e_{2n})$ of $V$. Thus, we can assume that $\Omega$ at $x$ in the chart $(\hat U,\hat\psi)$ takes the form
 \begin{gather*}
  \Omega\vert_x = \sum\limits^n_{j = 1} d\hat z_{j+n}\vert_x\wedge d\hat z_{j}\vert_x
 \end{gather*}
 by applying a $\mathbb{C}$-linear transformation to $(\hat U,\hat\psi)$ if necessary. Next, we define the following $2$-form on $\hat U$:
 \begin{gather*}
  \Omega_1\coloneqq\sum\limits^n_{j=1} d\hat z_{j+n}\wedge d\hat z_{j} \in\Omega^{(2,0)}(\hat U).
 \end{gather*}
 Clearly, both $\Omega_0\coloneqq \Omega\vert_{\hat U}$ and $\Omega_1$ are holomorphic symplectic $2$-forms on $\hat U$. Now, we define the $2$-form $\Omega_t$ on $\hat U$ as the interpolation of $\Omega_0$ and $\Omega_1$:
 \begin{gather*}
  \Omega_t\coloneqq \Omega_0 + t(\Omega_1 - \Omega_0)\quad\forall t\in\mathbb{R}.
 \end{gather*}
 For every $t\in\mathbb{R}$, the form $\Omega_t$ is holomorphic and closed, as $\Omega_0$ and $\Omega_1$ are holomorphic and closed. Further, we find that $\Omega_t\vert_x = \Omega_0\vert_x$ for every $t\in\mathbb{R}$, as $\Omega_0$ and $\Omega_1$ coincide at $x$ by construction. This means that $\Omega_t\vert_x$ is non-degenerate for every $t$. As non-degeneracy is an open property, we can construct an open subset $U^\prime\subset\hat U$ containing $x$ such that $\Omega_t\vert_{U^\prime}$ is a non-degenerate $2$-form for every $t\in [0,1]$, where we have also used the fact that $\Omega_t$ depends continuously on $t$ and $[0,1]$ is a compact interval. This turns ($U^\prime$, $\Omega_t\vert_{U^\prime}$) into a HSM for every $t\in [0,1]$. Moreover, we can assume that $U^\prime$ is contractible by shrinking $U^\prime$ if necessary. For the sake of simplicity and ease of notation, we assume from now on that $U^\prime = \hat U$. As $\hat U$ is contractible, its cohomology is trivial. This allows us to apply Theorem 2.5 (Moser's isotopy, version I) from \cite{soldatenkov2021}. Hence, we can find a $1$-form $\alpha$\footnote{In the paper by Soldatenkov and Verbitsky, the $1$-form $\alpha \equiv \alpha_t$ has, in general, a non-trivial $t$-dependence. In the present case, however, we can omit the $t$-dependence, as $d/dt\,\Omega_t$ does not depend on $t$ anymore.} on $\hat U$ of type $(1,0)$ such that
 \begin{gather*}
  \Omega_1 - \Omega_0 = \frac{d}{dt}\Omega_t = d\alpha.
 \end{gather*}
 Since the $2$-form $\Omega_1 - \Omega_0$ is of type $(2,0)$, the $1$-form $\alpha$ satisfies $\bar{\partial}\alpha = 0$. Hence, the form $\alpha$ is even holomorphic. We can assume without loss of generality that $\alpha$ satisfies $\alpha\vert_x = 0$ by replacing $\alpha$ with
 \begin{gather*}
  \alpha^\prime = \alpha - \alpha^x
 \end{gather*}
 if necessary, where $\alpha^x$ is a holomorphic $1$-form with $\alpha^x\vert_x = \alpha\vert_x$ and $d\alpha^x = 0$. For every $t\in [0,1]$, define the vector field $V_t$ on $\hat U$ with values in $T^{(1,0)}\hat U$ by
 \begin{gather*}
  \iota_{V_t}\Omega_t = -\alpha.
 \end{gather*}
 Note that $V_t$ is well-defined, as $\alpha$ is a $1$-form of type $(1,0)$ and $\Omega_t$ is non-degenerate on $T^{(1,0)}\hat U$, thus, $\Omega_t$ gives rise to an isomorphism from $T^{(1,0)}\hat U$ to $T^{\ast, (1,0)}\hat U$. Since $\alpha$ and $\Omega_t$ are holomorphic, $V_t$ is a holomorphic vector field on $\hat U$ for every $t\in [0,1]$. Using the closedness of $\Omega_t$ and Cartan's magic formula, we calculate the Lie derivative $L_{V_t}\Omega_t$:
 \begin{gather*}
  L_{V_t}\Omega_t = d\iota_{V_t}\Omega_t + \iota_{V_t}d\Omega_t = d\iota_{V_t}\Omega_t = -d\alpha = -\frac{d}{dt}\Omega_t\quad\forall t\in [0,1].
 \end{gather*}
 Recall\footnote{Confer Proposition \autoref{prop:holo_vec_field_equiv_J_pre_vec_field} and \cite{kobayashi1969} for details.} that every holomorphic vector field $V$ can be uniquely written as\linebreak $1/2(V^R - i\cdot J(V^R))$, where $J$ is the complex structure of the underlying complex manifold and $V^R$ is a real $J$-preserving vector field. Now let $V^R_t$ be the real $J$-preserving vector field corresponding to $V_t$ for every $t\in [0,1]$. Next, we want to show the following equation:
 \begin{gather*}
  L_{V^R_t}\Omega_t = L_{V_t}\Omega_t = -\frac{d}{dt}\Omega_t.
 \end{gather*}
 We do this by proving a more general auxiliary lemma:
 
 \begin{Aux*}
  Let $X$ be a complex manifold with complex structure $J\in\Gamma (\text{\normalfont End}(TX))$. Further, let $V^R\in\Gamma_J (TX)$ be a $J$-preserving vector field on $X$ with corresponding holomorphic vector field $V\coloneqq 1/2(V^R -iJ(V^R))$ and $T$ be a holomorphic tensor field on $X$, then the Lie derivatives of $T$ with respect to $V^R$ and $V$ coincide, i.e.:
  \begin{gather*}
   L_{V^R} T = L_{V}T.
  \end{gather*}
 \end{Aux*}

 \begin{proof}
  Let $X$ and $J$ be as above and let $V^R\in\Gamma_J (TX)$ be a $J$-preserving vector field on $X$ with corresponding holomorphic vector field $V\coloneqq 1/2(V^R -iJ(V^R))$. Further, let $T$ be a holomorphic $(k,l)$-tensor field on $X$. The Lie derivative is complex linear, thus, we have by definition of $V$:
  \begin{gather*}
   L_V T = \frac{1}{2}\left(L_{V^R}T - i\cdot L_{J(V^R)}T\right).
  \end{gather*}
  Hence, it suffices to show:
  \begin{gather}\label{eq:lie_i}
   L_{J(V^R)}T = i\cdot L_{V^R}T.
  \end{gather}
  In a holomorphic chart $\phi = (z_1,\ldots, z_n):U\to V\subset\mathbb{C}^n$ of $X$, $T$ can be expressed as
  \begin{gather*}
   T\vert_U = \sum^n_{i_1\ldots i_k, j_1\ldots j_l = 1} T^{i_1\ldots i_k}_{j_1\ldots j_l}\cdot dz_{j_1}\otimes\ldots\otimes dz_{j_l}\otimes \partial_{z_{i_1}}\otimes\ldots\otimes \partial_{z_{i_k}},
  \end{gather*}
  where $T^{i_1\ldots i_k}_{j_1\ldots j_l}:U\to\mathbb{C}$ are holomorphic functions on $U$. Since the Lie derivative can be computed locally and satisfies the Leibniz rule, it suffices to show Equation \eqref{eq:lie_i} for $T$ being $T^{i_1\ldots i_k}_{j_1\ldots j_l}$, $dz_{i}$, and $\partial_{z_j}$. As the Lie derivative also commutes with the exterior differential $d$ and $\partial_{z_j}$ is a (local) holomorphic vector field, it is sufficient to prove Equation \eqref{eq:lie_i} for $T$ being a holomorphic function $h$ and holomorphic vector field $W$. For $T = h$, we find:
  \begin{gather*}
   L_{J(V^R)} h = dh\left( J (V^R)\right) = i\cdot dh (V^R) = i\cdot L_{V^R}h,
  \end{gather*}
  where we used that $h$ is holomorphic, i.e. $dh\circ J = i\cdot dh$. For $T = W$, we can use Proposition \autoref{prop:holo_vec_field_equiv_J_pre_vec_field} to obtain:
  \begin{align*}
   L_{J(V^R)}W &= \left[J(V^R), W\right] = \frac{1}{2}\left([J(V^R), W^R] - i[J(V^R), J(W^R)]\right)\\
   &= \frac{1}{2}\left([V^R, J(W^R)] - i[V^R, J^2(W^R)]\right) = \frac{i}{2}\left([V^R, W^R] - i[V^R, J(W^R)]\right)\\
   &= i\left[V^R, W\right] = i\cdot L_{V^R}W,
  \end{align*}
  concluding the proof.
 \end{proof}
 
 Now return to the proof of Darboux's theorem for HSMs. Let $\varphi_t$ be the (possibly local) flow of the real time-dependent vector field $V^R_t$. The pull-back $\varphi^\ast_t\Omega_t$ is a solution of the initial value problem:
 \begin{gather*}
  \frac{d}{dt}(\varphi^\ast_t \Omega_t) = \varphi^\ast_t (L_{V^R_t}\Omega_t + \frac{d}{dt}\Omega_t) = 0,\quad \varphi^\ast_0\Omega_0 = \Omega_0,
 \end{gather*}
 where we suppressed the fact that $\varphi_t$ might not be defined on all of $\hat U$ for every $t\in [0,1]$ in our notation. Clearly, $\Omega_0$ is also a solution to the same initial value problem. As the solution to the given initial value problem is unique, we obtain:
 \begin{gather*}
  \varphi^\ast_t\Omega_t = \Omega_0.
 \end{gather*}
 We have to show that the last equation holds true for every $t\in [0,1]$ in some open neighborhood $U\subset\hat U$ of $x$. For this, we recall that $\alpha\vert_x = 0$ by construction. Thus, we have $V^R_t (x) = V_t (x) = 0$ for every $t\in [0,1]$. This implies that the flow $\varphi_t$ is stationary at $x$, i.e., $\varphi_t (x) = x$ for every $t\in [0,1]$. We can deduce from this that there exists an open neighborhood $U\subset\hat U$ of $x$ such that the flow $\varphi_t: U\to \varphi_t(U)\subset\hat U$ is a well-defined diffeomorphism for every $t\in [0,1]$. In particular, the time-$1$-map $\varphi_1:U\to\varphi_1 (U)$ satisfies:
 \begin{gather*}
  \varphi^\ast_1\Omega_1 = \Omega_0.
 \end{gather*}
 Hence, $(U,\psi\coloneqq\hat \psi\circ\varphi_1)$ is a smooth chart of $X$ near $x$ which satisfies:
 \begin{gather*}
  \psi^{-1\,\ast}\Omega\vert_U = {\hat\psi}^{-1\,\ast}\left(\varphi^{-1\,\ast}_1 \Omega_0\right) = {\hat\psi}^{-1\,\ast}\Omega_1 = \sum^n_{j=1} \theta_{j+n}\wedge \theta_{j},
 \end{gather*}
 where $\sum \theta_{j+n}\wedge\theta_{j}$ is the standard symplectic form on $\mathbb{C}^{2n}$. Hence, the holomorphic symplectic form $\Omega$ takes the following form on $(U,\psi \equiv (z_1,\ldots, z_{2n}))$:
 \begin{gather*}
  \Omega\vert_{U} = \sum\limits^n_{j = 1} dz_{j+n}\wedge dz_{j}.
 \end{gather*}
 We see that $(U,\psi)$ is a good candidate for the desired Darboux chart. To conclude the proof, we need to show that $(U,\psi)$ is also a holomorphic chart of $X$. For this, it suffices to prove that the map $\varphi_1: U\to \varphi_1 (U)$ is locally biholomorphic. The idea behind this proof is simple:
%  We deduce from Lemma (holo. int. curves\todo{Insert proper name here!}) that the flow $\varphi^{V^R}_t$ of a \underline{time-independent} real $J$-preserving vector field $V^R$ is locally biholomorphic.
 In Chapter IX of \cite{kobayashi1969}, it is shown that the \underline{time-independent} $J$-preserving\footnote{$J$-preserving vector fields are called infinitesimal automorphisms in \cite{kobayashi1969}.} vector fields $V^R$ on a complex manifold $X$ are exactly those real vector fields whose flow $\varphi^{V^R}_t$ is locally biholomorphic. However, we cannot directly apply this statement to $V^R_t$, as, in general, $V^R_t$ carries a non-trivial time-dependence. To account for this, we relate $V_t$ to a time-independent holomorphic vector field $V$ on $\hat U \times O \ni (x,t)$, where $O\subset\mathbb{C}$.\\
 First, we generalize the definition of $\Omega_t$ and allow for complex times $t\equiv\tau\in\mathbb{C}$. By the same arguments as before and by shrinking $\hat U$ if necessary, we find an open neighborhood $O\subset\mathbb{C}$ of $[0,1]$ such that $\Omega_\tau$ is non-degenerate for every $\tau\in O$. This allows us to generalize the definition of $V_\tau$ to all $\tau\in O$. Now observe that we cannot only view $\Omega_\tau$ as a time-dependent $(2,0)$-form on $\hat U$, but also as a time-independent $(2,0)$-form on $\hat U\times O$. As the time-dependence of $\Omega_\tau$ is clearly holomorphic, $\Omega_\tau$ as a form on $\hat U\times O$ is also holomorphic. Thus, $V_\tau$ understood as a vector field on $\hat U\times O$ is also holomorphic. Now consider the time-independent vector field $V$ on $\hat U\times O$:
 \begin{gather*}
  V(y,\tau^\prime)\coloneqq V_{\tau^\prime}(y) + \partial_\tau\vert_{(y,\tau^\prime)}\quad\forall (y,\tau^\prime)\in\hat U\times O.
 \end{gather*}
 As $V_\tau$ and $\partial_\tau$ are holomorphic, $V$ is also holomorphic. Now let $V^R$ be the real $J$-preserving vector field corresponding to $V$ and let $\Gamma:[0,1]\to\hat U\times O$, $\Gamma (r)\equiv (\gamma (r), \rho (r))$ be a smooth curve in $\hat U\times O$ with $\rho (0) = 0$. Then, we have the following auxiliary lemma:
 
 \begin{Aux*}
  $\Gamma$ is an integral curve of $V^R$ if and only if $\gamma:[0,1]\to\hat U$ is an integral curve of $V^R_t$ and $\rho (r) = r$ for every $r\in [0,1]$.
 \end{Aux*}
 
 \begin{proof}[Proof of auxiliary lemma]
  This follows from a quick computation: let $\tau = t + is$ be the decomposition of $\tau$ into real and imaginary part, then the vector field $V^R$ is given by:
  \begin{gather*}
   V^R(y,\tau^\prime) = V^R_{\tau^\prime}(y) + \partial_t\vert_{(y,\tau^\prime)}\quad\forall (y,\tau^\prime)\in\hat U\times O.
  \end{gather*}
  Further, let $\rho = \rho_R + i\rho_I$ be the decomposition of $\rho$ into real and imaginary part, then the integral curve equation of $V^R$ for $\Gamma$ can be written as:
  \begin{gather*}
   \dot\gamma (r) = V^R_{\rho (r)} (\gamma (r));\quad \dot \rho_R (r)\cdot\partial_t\vert_{\Gamma (r)} + \dot \rho_I (r)\cdot\partial_s\vert_{\Gamma (r)} = \partial_t\vert_{\Gamma (r)}\quad r\in [0,1].
  \end{gather*}
  As $\rho$ needs to satisfy the initial condition $\rho (0) = 0$, $\rho$ is given by $\text{id}_{[0,1]}$ if $\Gamma$ is an integral curve of $V^R$. In this case, $\gamma$ has to satisfy the following differential equation:
  \begin{gather*}
   \dot\gamma (r) = V^R_{r} (\gamma (r))\quad r\in [0,1].
  \end{gather*}
  Thus, if $\Gamma$ is an integral curve of $V^R$, then $\gamma$ is an integral curve of the time-dependent vector field $V^R_t$ on $\hat U$. The converse direction follows similarly.
 \end{proof}
 
 From the auxiliary lemma, it follows that the flow $\varphi^{V^R}_t$ of $V^R$ and the flow $\varphi_t$ of the time-dependent vector field $V^R_t$ are related in the following way:
 \begin{gather*}
  \varphi^{V^R}_t (y,0) = \left(\varphi_t (y), t\right)
 \end{gather*}
 for every suitable $y\in\hat U$. As discussed earlier, $\varphi^{V^R}_t$ is the flow of a holomorphic vector field and, hence, locally biholomorphic. This implies that $\varphi_t$ is also locally biholomorphic concluding the proof.
\end{proof}
