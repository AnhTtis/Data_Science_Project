In this paper, we first explore holomorphic Hamiltonian systems. In particular, we define action functionals for those systems and show that holomorphic trajectories obey an action principle, i.e., that they can be understood -- in some sense -- as critical points of these action functionals. As an application, we use holomorphic Hamiltonian systems to establish a relation between Lefschetz fibrations and almost toric fibrations. During the investigation of action functionals for holomorphic Hamiltonian systems, we observe that the complex structure $J$ corresponding to a holomorphic Hamiltonian system poses strong restrictions on the existence of certain trajectories. For instance, no non-trivial holomorphic trajectories with complex tori as domains can exist in $\mathbb{C}^{2n}$ due to the maximum principle. To lift this restriction, we generalize the notion of holomorphic Hamiltonian systems to systems with non-integrable almost complex structures $J$ leading us to the definition of pseudo-holomorphic Hamiltonian systems. We show that these systems exhibit properties very similar to their holomorphic counterparts, notably, that they are also subject to an action principle. Furthermore, we prove that the integrability of $J$ is equivalent to the closedness of the ``pseudo-holomorphic symplectic'' form. Lastly, we show that, aside from dimension four, the set of proper pseudo-holomorphic Hamiltonian systems is open and dense in the set of pseudo-holomorphic Hamiltonian systems by considering deformations of holomorphic Hamiltonian systems. This implies that proper pseudo-holomorphic Hamiltonian systems are generic.
