\section{Pseudo-Holomorphic Hamiltonian Systems}
\label{sec:PHHS}

In \autoref{subsec:holo_action_fun_and_prin}, we have seen that the existence of non-constant holomorphic periodic orbits is forbidden for a large class of HHSs $(X,\Omega, \mH)$. One possible obstruction to the existence of such orbits is the integrability of the almost complex structure $J$ of $X$. In this section, we drop the integrability of $J$ leading us to the notion of a pseudo-holomorphic Hamiltonian system (PHHS). We demonstrate in \autoref{subsec:def_PHHS} that PHHSs exhibit, by design, (almost) the same properties as found for HHSs in \autoref{sec:HHS}, in particular with respect to the existence and uniqueness of pseudo-holomorphic trajectories and with respect to action functionals and principles. In \autoref{subsec:rel_HHS_PHHS}, we explore the relation between HHSs and PHHSs and show that we recover a HHS from a PHHS if we restore the integrability of its almost complex structure $J$.


\subsection{PHHS: Basic Definitions, Notions, and Properties}
\label{subsec:def_PHHS}

In \autoref{subsec:holo_action_fun_and_prin}, we have found that most HHSs $(X,\Omega,\mH)$ do not possess non-constant holomorphic periodic orbits. Often, their existence was forbidden by the maximum principle. For instance, consider $X = \mathbb{R}^{4}\cong \mathbb{C}^2$ with standard complex structure $J = i$. As holomorphic periodic orbits are holomorphic maps $\gamma:\mathbb{C}/\Gamma\to X$ and a complex torus $\mathbb{C}/\Gamma$ is compact, the maximum principle applies and every holomorphic periodic orbit in $\mathbb{C}^2$ is constant.\\
In his beautiful paper \cite{moser1995} from 1995, Moser showed that the same argument does \underline{not} apply if we equip $\mathbb{R}^4$ with a different almost complex structure $J$. Let $\gamma$ be any smooth embedding of the $2$-torus into $\mathbb{R}^4$, e.g. the inclusion $S^1\times S^1 \subset \mathbb{R}^2\times\mathbb{R}^2\equiv \mathbb{R}^4$. Then, the image of $\gamma$ is a $2$-dimensional submanifold of $\mathbb{R}^4$ and its tangent bundle can be continued to a smooth $2$-dimensional distribution $D$ on $\mathbb{R}^4$. This can be seen as follows: the tangent bundle of $S^1\times S^1\subset\mathbb{R}^4$ is spanned by the two vector fields $V_1$ and $V_2$ on $S^1\times S^1$:
\begin{align*}
 V_1:&S^1\times S^1\to\mathbb{R}^4,\ (x_1,x_2,x_3,x_4)\mapsto (-x_2,x_1,0,0);\\ V_2:&S^1\times S^1\to\mathbb{R}^4,\ (x_1,x_2,x_3,x_4)\mapsto (0,0, -x_4, x_3).
\end{align*}
We show that there are two linearly independent vector fields $\hat V_1$ and $\hat V_2$ on $\mathbb{R}^4$ continuing $V_1$ and $V_2$, i.e., $\hat V_i\vert_{S^1\times S^1} = V_i$. To construct $\hat V_1$ and $\hat V_2$, we first define the functions $r_1\coloneqq \sqrt{x^2_1 + x^2_2}$, $r_2\coloneqq \sqrt{x^2_3 + x^2_4}$, and $R\coloneqq (1-r^2_1)^2 + (1-r^2_2)^2$. Next, define the vector fields $\hat V_1$, $W_1$, and $W_2$ on $\mathbb{R}^4$ as follows:
\begin{align*}
 \hat V_1:& \mathbb{R}^4\to\mathbb{R}^4,\ (x_1,x_2,x_3,x_4)\mapsto (-x_2,x_1,R,0);\\
 W_1:& \mathbb{R}^4\to\mathbb{R}^4,\ (x_1,x_2,x_3,x_4)\mapsto (-x_2x_4R,\, x_1x_4R,\, -r^2_1x_4,\, r^2_1x_3);\\
 W_2:& \mathbb{R}^4\to\mathbb{R}^4,\ (x_1,x_2,x_3,x_4)\mapsto (x_1,x_2,0,R).
\end{align*}
One easily checks that $\hat V_1$ is a continuation of $V_1$, vanishes nowhere, and is orthogonal to $W_1$ and $W_2$ with respect to the standard metric on $\mathbb{R}^4$. Furthermore, we notice that $W_1$ is a continuation of $V_2$. However, $W_1$ vanishes for $x_1 = x_2 = 0$ or $x_3 = x_4 = 0$. To rectify this, we take an appropriate linear combination of $W_1$ and $W_2$. For that, we first observe that $W_1$ does not vanish on $S^1\times S^1$. Hence, we can pick an open neighborhood $U\subset\mathbb{R}^4$ of $S^1\times S^1$ such that $W_1$ does not vanish on $U$. Next, we choose a partition of unity $\{f_1, f_2\}$ on $\mathbb{R}^4$ subordinate to the open covering $\{U, \mathbb{R}^4\backslash (S^1\times S^1)\}$ of $\mathbb{R}^4$, i.e, two smooth functions $f_1,f_2\in C^\infty (\mathbb{R}^4,\mathbb{R}_{\geq 0})$ satisfying:
\begin{enumerate}
 \item $f_1(x) + f_2(x) = 1\quad\forall x\in\mathbb{R}^4$,
 \item $\text{supp}(f_1)\subset U$ and $\text{supp}(f_2)\subset\mathbb{R}^4\backslash (S^1\times S^1)$.
\end{enumerate}
Now define the vector field $\hat V_2$ by $\hat V_2\coloneqq f_1\cdot W_1 + f_2\cdot W_2$. By construction, the vector field $\hat V_2$ is a continuation of $V_2$. Moreover, one can show that $\hat V_2$ vanishes nowhere by considering $\hat V_2$ separately on $S^1\times S^1$, $U\backslash (S^1\times S^1)$, and $\mathbb{R}^4\backslash U$. As $\hat V_1$ is orthogonal to $W_1$ and $W_2$, $\hat V_1$ is also orthogonal to the vector field $\hat V_2$. Two orthogonal vector fields which vanish nowhere are linearly independent, hence, the vector fields $\hat V_1$ and $\hat V_2$ are the desired continuations of $V_1$ and $V_2$. The distribution $D$ is now just the span of $\hat V_1$ and $\hat V_2$ at any point $(x_1,x_2,x_3,x_4)\in\mathbb{R}^4$.\\
Return to Moser's construction. Choose a Riemannian metric $g$ on $\mathbb{R}^4$ and consider\linebreak the orthogonal complement $D^\perp$ of $D$ with respect to $g$. $D^\perp$ is also a smooth\linebreak $2$-dimensional distribution on $\mathbb{R}^4$. Moreover, $D$ and $D^\perp$ span the tangent bundle of $\mathbb{R}^4$,\linebreak $T\mathbb{R}^4 = D\oplus D^\perp$. Now we can construct the almost complex structure $J$ on $\mathbb{R}^4$ as follows: choose orientations for $D$ and $D^\perp$ and define $J$ to be the $90^{\circ}$-rotation in $D$ and $D^\perp$ with respect to $g$ and the given orientations. After choosing a suitable complex structure $j$ on the $2$-torus, $\gamma$ becomes a pseudo-holomorphic\footnote{A smooth map $f:(X_1,J_1)\to (X_2, J_2)$ between smooth manifolds $X_1$ and $X_2$ with almost complex structures $J_1$ and $J_2$ is called \textbf{pseudo-holomorphic} iff $df\circ J_1 = J_2\circ df$. We call a pseudo-holomorphic map $f$ \textbf{holomorphic} if further $J_1$ and $J_2$ are integrable.} embedding, i.e., $d\gamma\circ j = J\circ d\gamma$. The almost complex structure $J$ constructed this way will, in general, \underline{not} be integrable.\\
Moser's example indicates that Hamiltonian systems with non-integrable almost complex structures $J$ might be richer than HHSs when it comes to (pseudo-)holomorphic periodic orbits. However, the generalization of HHSs is not straightforward, as the complex structure $J$ only enters most definitions regarding HHSs implicitly. To that end, let us recapitulate which objects and relations are essential to the definitions and discussions in \autoref{sec:HHS}. A HHS consists of six objects: a smooth manifold $X$ together with an integrable almost complex structure $J$ on it, two real $2$-forms $\Omega_R$ and $\Omega_I$ on $X$ which assemble to a holomorphic symplectic form $\Omega = \Omega_R + i\Omega_I$, and two smooth real functions $\mH_R$ and $\mH_I$ on $X$ forming a holomorphic function $\mH = \mH_R + i\mH_I$ on $X$. Closely tracing every step of \autoref{sec:HHS}, we see that these six objects need to satisfy the following relations:
\begin{enumerate}
 \item $\Omega_R$ must be closed\footnote{For the action functionals of holomorphic trajectories in \autoref{subsec:holo_action_fun_and_prin}, we need a primitive of $\Omega_R$, but not of $\Omega_I$!}.
 \item $J$, $\Omega_R$, and $\Omega_I$ need to satisfy the relations induced by Equation \eqref{eq:J-anticompatible}.
 \item $J$ and the Hamiltonian vector fields of the underlying RHSs have to fulfill Cauchy-Riemann-like relations formulated in Remark \autoref{rem:cauchy-riemann}.
 \item All Hamiltonian vector fields must commute reproducing Corollary \autoref{cor:commute}.
\end{enumerate}
Now, one way to define a pseudo-holomorphic Hamiltonian system is to simply impose these relations for any almost complex structure $J$:

\begin{Def}[Pseudo-holomorphic Hamiltonian system]\label{def:PHHS_1}
 We call a collection\linebreak $(X,J;\Omega_R,\Omega_I;\mH_R,\mH_I)$ a \textbf{pseudo-holomorphic Hamiltonian system} (PHHS) iff $X$ is a smooth manifold, $J$ is a (not necessarily integrable) almost complex structure on $X$, $\Omega_R$ and $\Omega_I$ are two smooth, non-degenerate, alternating, real $2$-forms on $X$, and $\mH_R$ and $\mH_I$ are two smooth real functions on $X$ satisfying:
 \begin{enumerate}
  \item $\Omega_R$ is closed, $d\Omega_R = 0$.
  \item $\Omega_R(J\cdot,\cdot) = \Omega_R(\cdot,J\cdot) = -\Omega_I;\quad \Omega_I(J\cdot,\cdot) = \Omega_I(\cdot,J\cdot) = \Omega_R;$\\
  $\Omega_R (J\cdot,J\cdot) = -\Omega_R;\qquad\qquad\ \,\ \Omega_I (J\cdot,J\cdot) = -\Omega_I$.
  \item $X^{\Omega_R}_{\mH_R} = X^{\Omega_I}_{\mH_I}$ and $J\left(X^{\Omega_R}_{\mH_R}\right) = X^{\Omega_I}_{\mH_R} = - X^{\Omega_R}_{\mH_I}$, where $X^{\Omega_a}_{\mH_b}$ is defined by $\iota_{X^{\Omega_a}_{\mH_b}}\Omega_a = -d\mH_b$.
  \item $[X^{\Omega_a}_{\mH_b}, X^{\Omega_c}_{\mH_d}] = 0$ for all $a,b,c,d\in\{R,I\}$.
 \end{enumerate}
\end{Def}

\begin{Rem}[Property 3 in Definition \autoref{def:PHHS_1}]\label{rem:H_pseudo-holo}
 Note that we can replace Property 3 in Definition \autoref{def:PHHS_1} with the condition that $\mH\coloneqq \mH_R + i\cdot \mH_I$ is pseudo-holomorphic. Indeed, Property 2 and 3 imply that $\mH:X\to\mathbb{C}$ is pseudo-holomorphic:
 \begin{align*}
  d\mH\circ J &= -\Omega_R (X^{\Omega_R}_{\mH_R}, J\cdot) - i\cdot \Omega_R (X^{\Omega_R}_{\mH_I}, J\cdot)\\
  &= \Omega_I (X^{\Omega_I}_{\mH_I}, \cdot) + i\cdot\Omega_I (-X^{\Omega_I}_{\mH_R},\cdot) = -d\mH_I + i\cdot d\mH_R = i\cdot d\mH.
 \end{align*}
 Conversely, Property 2 and $d\mH\circ J = i\cdot d\mH$ imply the Cauchy-Riemann-like equations in Property 3.
\end{Rem}

\begin{Rem}[The form $\Omega$, Part I]\label{rem:omega_I}
 As for complex manifolds, we can decompose the complexified tangent\footnote{Of course, similar remarks apply to the complexified cotangent bundle $T^{\ast}_\mathbb{C}X$ of $X$.} bundle $T_\mathbb{C}X$ of a manifold $X$ with almost complex structure $J$ into a direct sum of bundles $T^{(1,0)}X$ and $T^{(0,1)}X$ which are fiberwise given by the eigenspaces of $J$ with eigenvalue $i$ and $-i$, respectively. The difference is that now $T^{(1,0)}X$ is not a holomorphic , but merely a smooth complex vector bundle over $X$. Still, if we define the complex $2$-form $\Omega$ to be $\Omega_R + i\Omega_I$ for PHHSs, we find that $\Omega$ is of type $(2,0)$, i.e., vanishes on $T^{(0,1)}X$.
\end{Rem}

One might be confused why we only require $\Omega_R$ to be closed. The reason is that if we were to include the closedness of $\Omega_I$ into the definition of a PHHS, the almost complex structure $J$ would automatically be integrable rendering our construction pointless. The proof of this statement is given in \autoref{subsec:rel_HHS_PHHS}, when we explore the relation between HHSs and PHHSs.\\
Definition \autoref{def:PHHS_1} is convoluted, redundant, and rather unwieldy. For a better approach to PHHSs, let us first define pseudo-holomorphic symplectic manifolds:

\begin{Def}[Pseudo-holomorphic symplectic manifolds]\label{def:PHSM}
 We call a triple $(X,J;\Omega_R)$ \textbf{pseudo-holomorphic symplectic manifold} (PHSM) iff $(X,\Omega_R)$ is a symplectic manifold and $J$ is an almost complex structure on $X$ which is also $\Omega_R$-anticompatible, i.e. $\Omega_R (J\cdot, J\cdot) = -\Omega_R$. A PHSM $(X,J;\Omega_R)$ is called \textbf{proper} iff $J$ is not integrable.
\end{Def}

\begin{Rem}[The form $\Omega$, Part II]\label{rem:omega_II}
 Every PHSM $(X,J;\Omega_R)$ possesses forms $\Omega_I$ and $\Omega$ defined by
 \begin{gather*}
  \Omega_I\coloneqq -\Omega_R (J\cdot,\cdot);\quad \Omega\coloneqq \Omega_R + i\Omega_I.
 \end{gather*}
 It is easy to see that $\Omega_I$ is a smooth, non-degenerate, alternating $2$-form on $X$ which is also anticompatible with $J$. Furthermore, $\Omega$ is also anticompatible with $J$, satisfies $\Omega (J\cdot,\cdot) = \Omega (\cdot, J\cdot) = i\Omega$, i.e., $\Omega$ is of type $(2,0)$, and is non-degenerate on $T^{(1,0)}X$. However, neither $\Omega_I$ nor $\Omega$ are necessarily closed.
\end{Rem}

Now we can give an alternative definition of a PHHS:

\begin{Def}[Pseudo-holomorphic Hamiltonian system]\label{def:PHHS_2}
 We call a collection\linebreak $(X,J;\Omega_R, \mH_R)$ a \textbf{pseudo-holomorphic Hamiltonian system} (PHHS) iff $(X,J;\Omega_R)$ is a PHSM, $\mH_R:X\to\mathbb{R}$ is a smooth function on $X$, and the $1$-form $\Omega_R (J(X^{\Omega_R}_{\mH_R}),\cdot)$ is exact, where $\Omega_R (X^{\Omega_R}_{\mH_R},\cdot)\coloneqq -d\mH_R$. We call a PHHS $(X,J;\Omega_R,\mH_R)$ \textbf{proper} iff $J$ is not integrable.
\end{Def}

For both definitions to agree, it is obvious that the condition ``$\Omega_R (J(X^{\Omega_R}_{\mH_R}),\cdot)$ is exact'' is necessary, since by Definition \autoref{def:PHHS_1}: $\Omega_R (J(X^{\Omega_R}_{\mH_R}),\cdot) = d\mH_I$. It is also sufficient as guaranteed by the following proposition:

\begin{Prop}[PHHSs well-defined]\label{prop:def_PHHS_agree}
 Definition \autoref{def:PHHS_1} and Definition \autoref{def:PHHS_2} coincide.
\end{Prop}

\begin{proof}
 Clearly, every PHHS as in Definition \autoref{def:PHHS_1} also fulfills Definition \autoref{def:PHHS_2}. Now let\linebreak $(X,J;\Omega_R,\mH_R)$ be a PHHS as in Definition \autoref{def:PHHS_2}. Then, we define $\Omega_I\coloneqq -\Omega_R (J\cdot,\cdot)$ and take $\mH_I$ to be a primitive of the $1$-form $\Omega_R (J(X^{\Omega_R}_{\mH_R}),\cdot)$. We need to check that these data satisfy the properties 1-4 in Definition \autoref{def:PHHS_1}. Property 1 is trivially true by definition. Verifying Property 2 is a short and easy computation. To check Property 3, we recall Remark \autoref{rem:H_pseudo-holo}. It suffices to verify that the map $\mH = \mH_R + i\mH_I:X\to\mathbb{C}$ is pseudo-holomorphic which follows immediately:
 \begin{align*}
  d\mH_R\circ J &= -\Omega_R (X^{\Omega_R}_{\mH_R},J\cdot) = -\Omega_R (J(X^{\Omega_R}_{\mH_R}),\cdot) = -d\mH_I,\\
  d\mH_I\circ J &= \Omega_R (J(X^{\Omega_R}_{\mH_R}),J\cdot) = -\Omega_R (X^{\Omega_R}_{\mH_R},\cdot) = d\mH_R,\\
  \Rightarrow d\mH\circ J &= i\cdot d\mH.
 \end{align*}
 Lastly, we need to check Property 4. For this, remember that any symplectic manifold $(X,\Omega_R)$ admits a Poisson bracket $\{\cdot,\cdot\}:C^{\infty}(X,\mathbb{R})\times C^{\infty}(X,\mathbb{R})\to C^{\infty}(X,\mathbb{R})$ given by
 \begin{gather*}
  \{F,G\}\coloneqq \Omega_R (X_F, X_G),
 \end{gather*}
 where $X_F$ and $X_G$ are the Hamiltonian vector fields of the functions $F$ and $G$. Furthermore, recall that the map $X_\cdot:C^{\infty}(X,\mathbb{R})\to\Gamma (TX)$ is a Lie algebra homomorphism:
 \begin{gather*}
  X_{\{F,G\}} = [X_F, X_G]\quad\forall F,G\in C^{\infty}(X,\mathbb{R}).
 \end{gather*}
 Hence, it suffices to prove that $\{\mH_R,\mH_I\}$ vanishes in order to show that $X^{\Omega_R}_{\mH_R}$ and $X^{\Omega_R}_{\mH_I}$ commute. Let us calculate $\{\mH_R,\mH_I\}$ using Property 2 and 3:
 \begin{gather*}
  \{\mH_R,\mH_I\} = \Omega_R (X^{\Omega_R}_{\mH_R}, X^{\Omega_R}_{\mH_I}) = -\Omega_R (X^{\Omega_R}_{\mH_R}, J(X^{\Omega_R}_{\mH_R})) = \Omega_I (X^{\Omega_R}_{\mH_R},X^{\Omega_R}_{\mH_R}) = 0.
 \end{gather*}
 Commutativity of the remaining Hamiltonian vector fields follows from commutativity of $X^{\Omega_R}_{\mH_R}$ and $X^{\Omega_R}_{\mH_I}$ as well as Property 3 concluding the proof.
\end{proof}

Now that we have found a compact definition of PHHSs, we should briefly mention some examples of PHHSs. Of course, every HHS is, by design, a PHHS with integrable $J$. Even though the set of proper PHHSs is much larger than the set of HHSs, finding them is a bit more involved and, thus, relegated to \autoref{sec:deformation}. Partially, this is due to the fact that there are no ``standard'' examples of proper PHHSs like cotangent bundles\footnote{At least no canonical ones! In \autoref{sec:deformation}, we will equip the holomorphic cotangent bundle of a complex manifold with a non-canonical PHHS-structure.} as there are for RHSs and HHSs. On a deeper level, this is caused by the absence of a Darboux-like theorem. Clearly, there cannot be a counterpart to Darboux's theorem for PHSMs, since $J$ is usually not integrable and, hence, there are no coordinates in which $J$ assumes the standard form, let alone coordinates in which both $J$ and $\Omega = \Omega_R + i\Omega_I$ assume some standard form. Still, one can bring $J$ and $\Omega$ into standard form using local (usually non-integrable) frames:

\begin{Lem}[PHSMs in local frames]\label{lem:PHSM_in_sta_form}
 Let $(X,J;\Omega_R)$ be a PHSM with $\Omega\coloneqq \Omega_R - i\Omega_R (J\cdot,\cdot)$ and let $x_0\in X$ be any point. Then, there exists an open neighborhood $U\subset X$ of $x_0$ and a local frame $\theta^Q_1,\ldots, \theta^Q_n,\theta^P_1,\ldots,\theta^P_n$ of the smooth complex vector bundle $T^{\ast, (1,0)}X$ on $U$ such that $\Omega$ on $U$ can be expressed as
 \begin{gather*}
  \Omega\vert_U = \sum^{n}_{j} \theta^P_j\wedge \theta^Q_j.
 \end{gather*}
 In particular, the real dimension of $X$ is a multiple of $4$. The local frame can be chosen to be integrable (after shrinking $U$ if necessary) if and only if $J$ is integrable near $x_0$.
\end{Lem}

\begin{Rem}[$J$ in standard form]\label{rem:J_sta}
 $J$ is also in standard form in the dual frame of $\theta^Q_1,\ldots, \theta^Q_n,\theta^P_1,\ldots,\theta^P_n$. One can see this as follows: the real and imaginary part of the local frame $\theta^Q_1,\ldots, \theta^Q_n,\theta^P_1,\ldots,\theta^P_n$ ($\theta = \theta^x + i \theta^y$) give rise to a local frame of the real cotangent bundle $T^{\ast}X$. Its dual frame ${\hat e}^{Q, x}_{1},\ldots, {\hat e}^{P,y}_{n}$ is a local frame of the tangent bundle $TX$. By setting ${\hat e}\coloneqq 1/2({\hat e}^x - i{\hat e}^y)$, one obtains a local frame of $T^{(1,0)}X$. On $T^{(1,0)}X$, $J$ simply acts by $i$, thus, ${\hat e}^x$ and ${\hat e}^y$ satisfy $J({\hat e}^x) = {\hat e}^y$ and $J({\hat e}^y) = -{\hat e}^x$. This is the standard form of $J$.
\end{Rem}

\begin{proof}
 Lemma \autoref{lem:PHSM_in_sta_form} follows from the application of the symplectic Gram-Schmidt process, which can be found in any textbook on symplectic geometry, to a local frame of $T^{(1,0)}X$. Confer Proposition 2.8 in \cite{bogomolov2020} for the complex analogue of the symplectic Gram-Schmidt process. For completeness' sake, we repeat the explicit construction here. Let $(X,J;\Omega_R)$ be a PHSM with $\Omega\coloneqq \Omega_R - i\Omega_R (J\cdot,\cdot)$ and take the real dimension of $X$ to be\linebreak $\text{dim}_\mathbb{R} (X) = 2m$, $m\in\mathbb{N}$. The dimension of $X$ is even, as $X$ admits an almost complex structure $J$. Then, the complex rank of the complexified bundle $T_\mathbb{C}X$ is also given by $2m$. Now recall the decomposition $T_\mathbb{C}X = T^{(1,0)}X\oplus T^{(0,1)}X$. Since the complex vector bundles $T^{(1,0)}X$ and $T^{(0,1)}X$ are isomorphic via the complex conjugation $v + iw\mapsto v- iw$, their fibers have the same complex dimension, namely $m$. Now let $x_0\in X$ be any point and pick a local frame $v_1,\ldots, v_m$ of $T^{(1,0)}X$ on an open neighborhood $U\subset X$ of $x_0$. $\Omega$ is non-degenerate on $T^{(1,0)}X$ by Remark \autoref{rem:omega_II}, hence, there exists a vector ${\hat e}^Q_1\vert_{x_0}\in T^{(1,0)}_{x_0}X$ such that $\Omega\vert_{x_0} (v_1\vert_{x_0}, {\hat e}^Q_1\vert_{x_0})\neq 0$. $v_1,\ldots, v_m$ is a local frame of $T^{(1,0)}X$ near $x_0$, thus, we can write
 \begin{gather*}
  {\hat e}^Q_1\vert_{x_0} = \sum^m_{j = 1} c_j\cdot v_j\vert_{x_0}
 \end{gather*}
 for some constants $c_j\in\mathbb{C}$. Now define the local section ${\hat e}^Q_1 \coloneqq \sum_{j} c_j\cdot v_j$ of $T^{(1,0)}X$. After shrinking $U$ while preserving $x_0\in U$ if necessary, one obtains $\Omega\vert_x (v_1\vert_x, {\hat e}^Q_1\vert_x)\neq 0$ for every $x\in U$. Setting ${\hat e}^P_1\coloneqq v_1$ and changing the normalization of ${\hat e}^Q_1$ if necessary allows us to write $\Omega\vert_U ({\hat e}^P_1, {\hat e}^Q_1) = 1$.\\
 If $m = 2$, we simply define $\theta^Q_1, \theta^P_1$ to be the dual frame of ${\hat e}^Q_1, {\hat e}^P_1$. If $m> 2$, then we can pick one local section of the frame $v_1,\ldots, v_m$, say $v_2$, such that ${\hat e}^Q_1\vert_x$, ${\hat e}^P_1\vert_x$, and $v_2\vert_x$ are $\mathbb{C}$-linearly independent for every $x\in U$ after shrinking $U\ni x_0$ if necessary. We set:
 \begin{gather*}
  \hat v_2\coloneqq v_2 - \Omega\vert_U (v_2, {\hat e}^Q_1)\cdot {\hat e}^P_1 + \Omega\vert_U (v_2, {\hat e}^P_1)\cdot {\hat e}^Q_1.
 \end{gather*}
 Then, ${\hat e}^Q_1$, ${\hat e}^P_1$, and $\hat v_2$ are still $\mathbb{C}$-linearly independent on $U$ and $\hat v_2$ is $\Omega$-orthogonal to ${\hat e}^Q_1$ and ${\hat e}^P_1$, i.e., $\Omega\vert_U (\hat v_2, {\hat e}^Q_1) = \Omega\vert_U (\hat v_2, {\hat e}^P_1) = 0$. Again by the non-degeneracy of $\Omega$, we can find a vector $e^Q_2\vert_{x_0}\in T^{(1,0)}_{x_0}X$ such that $\Omega\vert_{x_0} (\hat v_2\vert_{x_0}, e^Q_2\vert_{x_0})\neq 0$. As before, we can write
 \begin{gather*}
  {e}^Q_2\vert_{x_0} = \sum^m_{j = 1} d_j\cdot v_j\vert_{x_0}
 \end{gather*}
 for some constants $d_j\in\mathbb{C}$ and define the local section ${e}^Q_2 \coloneqq \sum_{j} d_j\cdot v_j$ of $T^{(1,0)}$. After shrinking $U$ and changing the normalization of $e^Q_2$ if necessary, we obtain $\Omega\vert_U (\hat v_2, e^Q_2) = 1$. Now we set:
 \begin{gather*}
  {\hat e}^P_2\coloneqq \hat v_2;\quad {\hat e}^Q_2\coloneqq e^Q_2 - \Omega\vert_U (e^Q_2, {\hat e}^Q_1)\cdot {\hat e}^P_1 + \Omega\vert_U (e^Q_2, {\hat e}^P_1)\cdot {\hat e}^Q_1.
 \end{gather*}
 Proceeding inductively gives us a local frame ${\hat e}^Q_1,\ldots {\hat e}^Q_n$, ${\hat e}^P_1,\ldots, {\hat e}^P_n$ of $T^{(1,0)}X$ on some neighborhood $U$ of $x_0$ ($n\coloneqq m/2$) satisfying:
 \begin{gather*}
  \Omega\vert_U ({\hat e}^Q_i, {\hat e}^Q_j) = \Omega\vert_U ({\hat e}^P_i, {\hat e}^P_j) = 0;\quad \Omega\vert_U ({\hat e}^P_i, {\hat e}^Q_j) = \delta_{ij}.
 \end{gather*}
 Thus, the frame $\theta^Q_1,\ldots, \theta^Q_n$, $\theta^P_1,\ldots, \theta^P_n$ dual to the frame ${\hat e}^Q_1,\ldots {\hat e}^Q_n$, ${\hat e}^P_1,\ldots, {\hat e}^P_n$ is the desired local frame of $T^{\ast, (1,0)}X$ near $x_0$ in which $\Omega$ takes the form:
 \begin{gather*}
  \Omega\vert_U = \sum^{n}_{j} \theta^P_j\wedge \theta^Q_j.
 \end{gather*}
 In particular, the real dimension of $X$ is $4n$. If the frame $\theta^Q_1,\ldots$ is integrable, then the frame ${\hat e}^Q_1,\ldots$ is also integrable and there exists a (holomorphic) chart\linebreak $\phi = (Q_1,\ldots, Q_n, P_1,\ldots, P_n):U\to V\subset\mathbb{C}^{2n}$ near $x_0$ such that:
 \begin{gather*}
  {\hat e}^Q_j\equiv \pa_{Q_j};\quad {\hat e}^P_j\equiv \pa_{P_j};\quad \theta^Q_j\equiv dQ_j;\quad \theta^P_j\equiv dP_j.
 \end{gather*}
 By Remark \autoref{rem:J_sta}, $J$ also assumes its standard form in this chart, thus, the Nijenhuis tensor of $J$ vanishes on an open neighborhood of $x_0$ and $J$ is integrable near $x_0$. The converse direction follows from Theorem \autoref{thm:rel_HSM_PHSM} (cf. \autoref{subsec:rel_HHS_PHHS}) and Darboux's theorem for HSMs (cf. Theorem \autoref{thm:holo_Darboux} and \autoref{app:darboux}).
\end{proof}

Next, let us investigate the dynamics of a PHHS $(X,J;\Omega_R,\mH_R)$. To do so, we need to introduce pseudo-holomorphic Hamiltonian vector fields and trajectories. Hereby, we imitate the definitions from \autoref{sec:HHS}. For the sake of simplicity, we always associate from now on with a PHHS $(X,J;\Omega_R,\mH_R)$ the forms $\Omega_I\coloneqq -\Omega_R (J\cdot,\cdot)$ and $\Omega\coloneqq \Omega_R + i\Omega_I$ as well as the functions $\mH_I$ and $\mH\coloneqq \mH_R + i\mH_I$, where $\mH_I$ is a primitive of $\Omega_R (J(X^{\Omega_R}_{\mH_R}),\cdot)$.

\begin{Def}[Pseudo-holomorphic Hamiltonian vector fields and trajectories]\label{def:pseudo-holo_ham_field_and_traj}
 Let\linebreak $(X,J;\Omega_R, \mH_R)$ be a PHHS. We call the smooth section $X_\mH$ of $T^{(1,0)}X$ defined by\linebreak $\iota_{X_\mH}\Omega = -d\mH$ the (pseudo-holomorphic) \textbf{Hamiltonian vector field} of the\linebreak PHHS $(X,J;\Omega_R, \mH_R)$. Furthermore, we call a pseudo-holomorphic map $\gamma:U\to X$ a \textbf{pseudo-holomorphic trajectory} of the PHHS $(X,J;\Omega_R, \mH_R)$ iff $\gamma$ satisfies the pseudo-holomorphic integral curve equation:
 \begin{gather*}
  \frac{\pa\gamma}{\pa z} (z)\coloneqq \frac{1}{2}\left(\frac{\pa \gamma}{\pa t} (z) - i\frac{\pa\gamma}{\pa s} (z)\right) = X_\mH (\gamma (z))\quad\forall z = t + is\in U,
 \end{gather*}
 where $U\subset\mathbb{C}$ is an open and connected subset with standard complex structure $j = i$. We call a pseudo-holomorphic trajectory $\gamma:U\to X$ \textbf{maximal} iff for every pseudo-holomorphic trajectory $\hat\gamma:\hat U\to X$ with $U\subset \hat U$ and $\hat\gamma\vert_U = \gamma$ one has $\hat U = U$ and $\hat \gamma = \gamma$.
\end{Def}

Alternatively, one can define pseudo-holomorphic Hamiltonian vector fields and trajectories in terms of the vector field $X^{\Omega_R}_{\mH_R}$:

\begin{Prop}[Alternative Definition of $X_\mH$ and $\gamma$]\label{prop:pseudo-holo_ham_field_and_traj}
 Let $(X,J;\Omega_R,\mH_R)$ be a PHHS with vector field $X^{\Omega_R}_{\mH_R}$ defined by $\Omega_R (X^{\Omega_R}_{\mH_R},\cdot) = -d\mH_R$. Then:
 \begin{gather*}
  X_\mH = \frac{1}{2} \left( X^{\Omega_R}_{\mH_R} - i\cdot J(X^{\Omega_R}_{\mH_R})\right).
 \end{gather*}
 Now, let $U\subset\mathbb{C}$ be an open and connected subset and $\gamma:U\to X$ be a map. Then, $\gamma$ is a pseudo-holomorphic trajectory iff $\gamma_s$ defined by $\gamma_s (t)\coloneqq \gamma (t + is)$ is an integral curve of $X^{\Omega_R}_{\mH_R}$ for every suitable $s\in\mathbb{R}$ and $\gamma:U\to X$ is pseudo-holomorphic.
\end{Prop}

\begin{proof}
 A straightforward calculation verifies that $X_\mH$ as in Definition \autoref{def:pseudo-holo_ham_field_and_traj} satisfies the equation above. To prove the statement about pseudo-holomorphic trajectories, consider the real part of the pseudo-holomorphic integral curve equation and observe that due to the pseudo-holomorphicity of $\gamma$:
 \begin{gather*}
  \frac{\pa\gamma}{\pa s} = J(\frac{\pa\gamma}{\pa t}).
 \end{gather*}
\end{proof}

We designed PHHSs in such a way that all properties we found for HHSs in \autoref{sec:PHHS} (almost) completely transfer to PHHSs. For instance, we find the following PHHS-counterpart to Proposition \autoref{prop:holo_traj}:

\begin{Prop}[Existence and uniqueness of pseudo-holomorphic trajectories]\label{prop:pseudo-holo_traj}
 Let $(X,J;\Omega_R,\mH_R)$ be a PHHS. Then, for any $z_0\in\mathbb{C}$ and $x_0\in X$, there exists an open and connected subset $U\subset\mathbb{C}$ and a pseudo-holomorphic trajectory $\gamma^{z_0, x_0}:U\to X$ of $(X,J;\Omega_R,\mH_R)$ with $\gamma^{z_0, x_0} (z_0) = x_0$. Two pseudo-holomorphic trajectories\linebreak $\gamma^{z_0, x_0}_1:U_1\to X$ and $\gamma^{z_0, x_0}_2:U_2\to X$ with $\gamma^{z_0, x_0}_1 (z_0) = x_0 = \gamma^{z_0, x_0}_2 (z_0)$ locally coincide, in particular, they are equal iff their domains $U_1$ and $U_2$ are equal. Furthermore, the pseudo-holomorphic trajectory $\gamma^{z_0, x_0}$ depends pseudo-holomorphically on $z_0$, but, in general, \underline{only} smoothly on $x_0$.
\end{Prop}

\begin{proof}
 The proof of Proposition \autoref{prop:pseudo-holo_traj} works very similarly to the proof of Proposition \autoref{prop:holo_traj}. By Definition \autoref{def:PHHS_1} and Proposition \autoref{prop:pseudo-holo_ham_field_and_traj}, the real and imaginary part of $X_\mH$ commute, hence, we can proceed as in the proof of Proposition \autoref{prop:holo_traj} to show that pseudo-holomorphic trajectories $\gamma^{z_0,x_0}$ given an initial value $\gamma^{z_0, x_0} (z_0) = x_0$ exist.\\
 To prove uniqueness, we observe that the formula
 \begin{gather*}
  \gamma^{z_0, x_0}(z)\coloneqq \varphi^{J(X^{\Omega_R}_{\mH_R})}_{s-s_0}\circ\varphi^{X^{\Omega_R}_{\mH_R}}_{t-t_0} (x_0)\equiv \varphi^{X^{\Omega_R}_{\mH_R}}_{t-t_0}\circ\varphi^{J(X^{\Omega_R}_{\mH_R})}_{s-s_0} (x_0)\equiv\varphi^{(t-t_0)X^{\Omega_R}_{\mH_R} + (s-s_0)J(X^{\Omega_R}_{\mH_R})}_1 (x_0),
 \end{gather*}
 where $z = t+is$, uniquely determines $\gamma^{z_0, x_0}$ on a small rectangle in $\mathbb{C}$ near $z_0 = t_0 + is_0$. The rest now follows by covering a path between $z_0$ and any point $z_1$ in $U_1\equiv U_2$ with a finite number of such rectangles.\\
 Lastly, let us consider the dependence of $\gamma^{z_0, x_0}$ on $z_0\in\mathbb{C}$ and $x_0\in X$. Again, $\gamma^{z_1, x_0} (z)$ and $\gamma^{z_2, x_0} (z)$ only differ by a translation in $z$. As pseudo-holomorphic trajectories are pseudo-holomorphic maps, the $z_0$-dependence is also pseudo-holomorphic. For the $x_0$-dependence, we need to consider the flow of $X^{\Omega_R}_{\mH_R}$ and $J(X^{\Omega_R}_{\mH_R})$. Both $X^{\Omega_R}_{\mH_R}$ and $J(X^{\Omega_R}_{\mH_R})$ are smooth vector fields, thus, their flows are smooth as well concluding the proof.
\end{proof}

\begin{Rem}[$x_0$-dependence]\label{rem:x_0-dependance}
 Note that holomorphic trajectories of HHSs depend holomorphically on $x_0$, while pseudo-holomorphic trajectories of PHHSs do \underline{not} generally depend pseudo-holomorphically on $x_0$. This distinction can be traced back to the Hamiltonian vector field $X_\mH$. For HHSs, $X_\mH$ is a holomorphic vector field, in particular its real and imaginary part are $J$-preserving vector fields (cf. Proposition \autoref{prop:holo_vec_field_equiv_J_pre_vec_field}) implying that the differential of their flows commute with $J$. For PHHSs, this does not need to be the case anymore: neither $X^{\Omega_R}_{\mH_R}$ nor $J(X^{\Omega_R}_{\mH_R})$ are required to be $J$-preserving! In fact, we study an example of a proper PHHS in \autoref{sec:deformation} where $X^{\Omega_R}_{\mH_R}$ is $J$-preserving, but $J(X^{\Omega_R}_{\mH_R})$ is not.
\end{Rem}

As for HHSs, the maximal trajectories of a PHHS, given an initial value, do not need to be unique, however, we can still pseudo-holomorphically foliate energy hypersurfaces $\mH^{-1}(E)$ of a PHHS:

\begin{Prop}[Pseudo-holomorphic foliation of a regular hypersurface]\label{prop:pseudo-holo_foli}
 Let\linebreak $(X,J;\Omega_R,\mH_R)$ be a PHHS with Hamiltonian vector field $X_\mH = 1/2 (X^{\Omega_R}_{\mH_R} - i J(X^{\Omega_R}_{\mH_R}))$ and regular\footnote{As before, a PHHS $(X,J;\Omega_R,\mH_R)$ is regular at the energy $E\in\mathbb{C}$ iff $d\mH$ or, equivalently, $d\mH_R$ does not vanish on $\mH^{-1}(E)$.} value $E$ of $\mH$. Then, the energy hypersurface $\mH^{-1}(E)$ admits a pseudo-holomorphic foliation. The leaf $L_{x_0}$ of this foliation through a point $x_0\in\mH^{-1}(E)$ is given by
 \begin{align*}
  L_{x_0}\coloneqq \{y\in X\mid &y = \varphi^{X^{\Omega_R}_{\mH_R}}_{t_1}\circ\varphi^{J(X^{\Omega_R}_{\mH_R})}_{s_1}\circ\varphi^{X^{\Omega_R}_{\mH_R}}_{t_2}\circ\varphi^{J(X^{\Omega_R}_{\mH_R})}_{s_2}\circ\ldots\circ\varphi^{X^{\Omega_R}_{\mH_R}}_{t_n}\circ\varphi^{J(X^{\Omega_R}_{\mH_R})}_{s_n} (x_0);\\
  &t_1,\ldots,t_n, s_1,\ldots, s_n\in\mathbb{R};\ n\in\mathbb{N}\},
 \end{align*}
 where $\varphi^{X^{\Omega_R}_{\mH_R}}_{t_j}$ and $\varphi^{J(X^{\Omega_R}_{\mH_R})}_{s_j}$ are the flows of $X^{\Omega_R}_{\mH_R}$ and $J(X^{\Omega_R}_{\mH_R})$ for time $t_j$ and $s_j$, respectively. Every pseudo-holomorphic trajectory of $(X,J;\Omega_R,\mH_R)$ with energy $E$ is completely contained in one such leaf.
\end{Prop}

\begin{proof}
 First, we need to clarify the notion of a pseudo-holomorphic foliation. In order to do that, recall the definition of a holomorphic foliation. A ($d$-dimensional) holomorphic foliation $\{L_{x_0}\}_{x_0\in I}$ ($I$: index set) of a complex manifold $X$ is a decomposition of\linebreak $X = \bigcup_{x_0\in I} L_{x_0}$ into a disjoint union of leaves $L_{x_0}$, path-connected subsets of $X$, such that for every point $x\in X$ there exists a holomorphic chart $\phi = (z_1,\ldots, z_n):U\to V\subset\mathbb{C}^n$ of $X$ near $x$ fulfilling: for every leaf $L_{x_0}$ with $U\cap L_{x_0}\neq\emptyset$, the connected components of $U\cap L_{x_0}$ are given by $z_{d+1} = c_{d+1}$,\ldots, $z_n = c_n$ for some constants $c_{d+1},\ldots, c_n\in\mathbb{C}$. Clearly, we cannot directly transfer this definition to the non-integrable case, since generic almost complex manifolds do not admit holomorphic charts. Therefore, we call $\{L_{x_0}\}_{x_0\in I}$ a pseudo-holomorphic foliation of an almost complex manifold $(X,J)$ iff $\{L_{x_0}\}_{x_0\in I}$ is a (smooth) foliation of $X$ and the tangent spaces of the leaves $L_{x_0}$ are closed under the action of $J$. We can now prove the last proposition in the same way as Proposition \autoref{prop:holo_foli} by applying the Frobenius theorem to the vector fields $X^{\Omega_R}_{\mH_R}$ and $J(X^{\Omega_R}_{\mH_R})$. 
\end{proof}

Similarly to HHSs, we can also define the notion of geometric trajectories for\linebreak PHHSs. We simply copy Definition \autoref{def:geo_traj} and replace the term ``holomorphic'' with\linebreak ``pseudo-holomorphic''. All results we found in \autoref{subsec:holo_traj} for geometric trajectories of HHSs still hold in the pseudo-holomorphic case. In particular, Proposition \autoref{prop:geo_traj} is still true for PHHSs. The proof is essentially the same as in the holomorphic case. However, the vector field $Y_\mH$ on the Riemann surface $\Sigma$ is a priori only a smooth section of $T^{(1,0)}\Sigma$. $Y_\mH$ becomes a holomorphic vector field on $\Sigma$ by noting that the real and imaginary part of the Hamiltonian vector field $X_\mH$ commute by construction. Thus, the real and imaginary part of $Y_\mH$ also commute, as the push-forward of $\gamma$ is a Lie algebra homomorphism, i.e., $\gamma_\ast [V,W] = [\gamma_\ast V, \gamma_\ast W]$ for vector fields\footnote{Precisely speaking, this is not correct, since $\gamma:\Sigma\to X$ is only an immersion and not a diffeomorphism, hence, the push-forward of $\gamma$ is not well-defined. Nevertheless, the argument still holds if we consider $\gamma_\ast V$ and $\gamma_\ast W$ to be sections of the pull-back bundle $\gamma^\ast TX$ and adjust the definition of the Lie bracket accordingly.} $V$ and $W$ on $\Sigma$. Now note that, for a Riemann surface $\Sigma$, the real and imaginary part of a smooth section $V$ of $T^{(1,0)}\Sigma$ commute if and only if $V$ is a holomorphic vector field on $\Sigma$. This is easily verified in holomorphic charts of $\Sigma$. This shows the holomorphicity of $Y_\mH$. The proofs for the remaining results regarding geometric trajectories work as in the holomorphic case after adjusting the language where need be.\\
Before we conclude this subsection, we want to formulate action functionals and principles for pseudo-holomorphic trajectories. By construction of PHHSs, this can be done in the same way as in \autoref{subsec:holo_action_fun_and_prin}. First, we note that a PHHS $(X,J;\Omega_R,\mH_R)$ decomposes into multiple RHSs. In contrast to HHSs, we only obtain two RHSs this time, namely $(X,\Omega_R,\mH_R)$ and $(X,\Omega_R,\mH_I)$, since $\Omega_I$ is, in general, not closed. However, this suffices to find action functionals for pseudo-holomorphic trajectories, as only two of the four underlying RHSs of a HHS are subject to different dynamics. If the PHHS $(X,J;\Omega_R,\mH_R)$ is exact, i.e., $\Omega_R = d\Lambda_R$, the two RHSs $(X,\Omega_R,\mH_R)$ and $(X,\Omega_R,\mH_I)$ are also exact and possess themselves action functionals. As in the case of HHSs, we can now average these action functionals over the remaining time variable and take suitable linear combinations afterwards to find the following action functional for pseudo-holomorphic trajectories:

\begin{Lem}[Action principle for pseudo-holomorphic trajectories]\label{lem:pseudo-holo_action_prin_para}
 {\textcolor{white}{Easter Egg}}\linebreak Let $(X,J;\Omega_R = d\Lambda_R,\mH_R)$ be an exact PHHS. For $\alpha\in\mathbb{R}\backslash\{n\cdot\pi\mid n\in\mathbb{Z}\}$, let\linebreak $P_\alpha\coloneqq [t_1,t_2] + e^{i\alpha}[r_1,r_2]\subset\mathbb{C}$ be a parallelogram in the complex plane with real numbers $t_1< t_2$ and $r_1< r_2$. Denote the space of smooth maps from $P_\alpha$ to $X$ by $\mathcal{P}_{P_\alpha}$ and define the action functional $\mathcal{A}^{P_\alpha}_\mH:\mathcal{P}_{P_\alpha}\to\mathbb{C}$ by
\begin{gather*}
 \mathcal{A}^{P_\alpha}_\mH[\gamma]\coloneqq \iint\limits_{P_\alpha}\left[\Lambda_R\vert_{\gamma (t+is)}\left(2\frac{\pa\gamma}{\pa z}(t+is)\right) - \mH\circ\gamma (t+is)\right] dt\wedge ds\ \text{with}\ \frac{\pa\gamma}{\pa z}\coloneqq \frac{1}{2}\left(\frac{\pa\gamma}{\pa t} - i\frac{d\gamma}{\pa s}\right)
\end{gather*}
 for every $\gamma\in\mathcal{P}_{P_\alpha}$. Now, let $\gamma\in\mathcal{P}_{P_\alpha}$ be a smooth map from $P_\alpha$ to $X$. Then, $\gamma$ is a pseudo-holomorphic trajectory of $(X,J;\Omega_R,\mH_R)$ iff $\gamma$ is a ``critical point''\footnote{``Critical point'' means that only those variations are allowed which keep $\gamma$ fixed on the boundary $\partial P_\alpha$.} of $\mathcal{A}^{P_\alpha}_\mH$.
\end{Lem}

\begin{proof}
 For the proof of Lemma \autoref{lem:pseudo-holo_action_prin_para}, repeat the steps from \autoref{subsec:holo_action_fun_and_prin} for PHHSs, in particular the proof of Proposition \autoref{prop:holo_action_prin_para}.
\end{proof}

As before, if we wish to view pseudo-holomorphic trajectories as actual critical points of some functional, we can achieve this by either mapping the boundary $\partial P_\alpha$ to a Lagrangian submanifold of $(X,\Omega_R)$ or by imposing periodicity on the curves $\gamma$.


\newpage
\subsection{Relation between HHSs and PHHSs}
\label{subsec:rel_HHS_PHHS}

At this point, it is not clear how PHHSs relate to HHSs and why PHHSs are a ``reasonable'' generalization of HHSs with regard to the integrability of $J$. In particular, we do not know yet why the notion of PHHSs introduced in \autoref{subsec:def_PHHS} should coincide with the notion of HHSs when we restore the integrability of $J$. A priori, there is no reason why the $2$-form $\Omega\coloneqq \Omega_R - i\Omega_R (J\cdot,\cdot)$ associated with a PHHS $(X,J;\Omega_R,\mH_R)$ should be holomorphic or even closed for integrable $J$. That this is indeed the case is guaranteed by the following theorem:

\begin{Thm}[Relation between HSMs and PHSMs]\label{thm:rel_HSM_PHSM}
 Let $(X,J;\Omega_R)$ be a PHSM with $2$-forms $\Omega_I\coloneqq -\Omega_R (J\cdot,\cdot)$ and $\Omega\coloneqq \Omega_R + i\Omega_I$. Further, let $x_0\in X$ be any point. Then, $J$ is integrable near $x_0$ if and only if $d\Omega_I$ vanishes near $x_0$. Moreover, the following statements are equivalent:
 \begin{enumerate}
  \item $(X,\Omega)$ is a HSM with complex structure $J$.
  \item $\Omega_I$ is closed, $d\Omega_I = 0$.
  \item $J$ is integrable.
 \end{enumerate}
\end{Thm}

\begin{proof}
 We only show the equivalence of Statement 1, 2, and 3. The first part of Theorem \autoref{thm:rel_HSM_PHSM} is then just a local version of the equivalence. Direction ``1$\Rightarrow$2'' is trivially true by definition of a HSM. Implication ``2$\Rightarrow$3'' is due to Verbitsky (confer Theorem 3.5 in \cite{verbitsky2013} and Proposition 2.12 in \cite{bogomolov2020}). For completeness' sake, we include the proof here. Let $(X,J;\Omega_R)$ be a PHSM with $2$-forms $\Omega_I\coloneqq -\Omega_R (J\cdot,\cdot)$ and $\Omega\coloneqq \Omega_R + i\Omega_I$. Further, assume $d\Omega_I = 0$. We want to show that $J$ is integrable. By the Newlander-Nirenberg theorem, $J$ is integrable if and only if $J$ has no torsion, i.e., its Nijenhuis tensor vanishes. Now we apply Theorem 2.8 in Chapter IX of \cite{kobayashi1969}. Thus, $J$ is integrable if and only if the space of smooth sections of $T^{(0,1)}X$ is closed under the commutator $[\cdot,\cdot]$. From \autoref{subsec:def_PHHS}, we know that $\Omega$ is non-degenerate on $T^{(1,0)}X$, but vanishes on $T^{(0,1)}X$. Hence, a complex vector field $V$ on $X$ is a smooth section of $T^{(0,1)}$ if and only if $\iota_V\Omega = 0$. Therefore, $J$ is integrable if and only if for every pair of two complex vector fields $V$ and $W$ on $X$ satisfying $\iota_V\Omega = \iota_W\Omega = 0$ one has $\iota_{[V,W]}\Omega = 0$. Now let $V$ and $W$ be two complex vector fields on $X$ with $\iota_V\Omega = \iota_W\Omega = 0$. Recall that the interior product $\iota$ applied to forms fulfills the relation (cf. Proposition 3.10 in Chapter I of \cite{kobayashi1963}):
 \begin{gather*}
  \iota_{[V,W]} = [L_V, \iota_W],
 \end{gather*}
 where $L_V$ is the Lie derivative of $V$. We can calculate $L_V\Omega$ by using Cartan's magic formula, $\iota_V\Omega = 0$, and $d\Omega = 0$:
 \begin{gather*}
  L_V\Omega = d\iota_V\Omega + \iota_Vd\Omega = 0.
 \end{gather*}
 In total, we obtain using $\iota_W\Omega = 0$:
 \begin{gather*}
  \iota_{[V,W]}\Omega = [L_V,\iota_W]\Omega = L_V(\iota_W\Omega) - \iota_W(L_V\Omega) = 0
 \end{gather*}
 proving the integrability of $J$.
 \newpage
 The remaining direction ``3$\Rightarrow$1'' can be proven as follows: let $(X,J;\Omega_R)$ be a PHSM with $2$-forms $\Omega_I\coloneqq -\Omega_R (J\cdot,\cdot)$ and $\Omega\coloneqq \Omega_R + i\Omega_I$. Further, assume that $J$ is integrable. Then, $X$ is a complex manifold with complex structure $J$. We need to show that $\Omega$ is a closed, holomorphic $2$-form which is non-degenerate on $T^{(1,0)}X$. By Remark \autoref{rem:omega_II}, $\Omega$ is non-degenerate on $T^{(1,0)}X$ and of type $(2,0)$. Hence, $\Omega$ can be written in a holomorphic chart $\phi = (z_1,\ldots, z_{2m}):U\to V\subset\mathbb{C}^{2m}$ of $X$ as
 \begin{gather*}
  \Omega\vert_U = \sum\limits^{2m}_{i,j = 1} \Omega_{ij} dz_i\wedge dz_j,
 \end{gather*}
 where the coefficients $\Omega_{ij} = -\Omega_{ji}:U\to\mathbb{C}\cong\mathbb{R}^2$ are smooth functions on $U$. From $d\Omega_R = 0$, we deduce:
 \begin{align*}
  0 = 2d\Omega_R\vert_U = (\partial + \bar{\partial})(\Omega + \overline{\Omega})\vert_U &= \sum\limits^{2m}_{i,j,k = 1}\left(\frac{\partial\Omega_{ij}}{\partial z_k} dz_k\wedge dz_i\wedge dz_j + \frac{\partial\Omega_{ij}}{\partial \bar{z}_k} d\bar{z}_k\wedge dz_i\wedge dz_j\right.\\
  &\qquad\qquad + \left.\frac{\partial\overline{\Omega}_{ij}}{\partial z_k} dz_k\wedge d\bar{z}_i\wedge d\bar{z}_j + \frac{\partial\overline{\Omega}_{ij}}{\partial \bar{z}_k} d\bar{z}_k\wedge d\bar{z}_i\wedge d\bar{z}_j\right).
 \end{align*}
 This equation implies:
 \begin{gather*}
  \frac{\partial\Omega_{ij}}{\partial \bar{z}_k} = 0\quad\forall i,j,k\in\{1,\ldots, 2m\}.
 \end{gather*}
 Thus, the coefficients $\Omega_{ij}$ are holomorphic functions on $U$. As the last argument can be repeated for any holomorphic chart of $X$, the form $\Omega$ itself is holomorphic. Therefore, its exterior derivative $d\Omega$ is also a holomorphic form. In particular, $d\Omega$ satisfies:
 \begin{gather*}
  d\Omega (J\cdot,\cdot,\cdot) = i\cdot d\Omega.
 \end{gather*}
 As the exterior derivative is $\mathbb{C}$-linear, the decomposition of $d\Omega$ into real and imaginary part amounts to $d\Omega = d\Omega_R + id\Omega_I$. Combining this decomposition with the previous equation gives us
 \begin{gather*}
  d\Omega_I = -d\Omega_R (J\cdot,\cdot,\cdot) = 0,
 \end{gather*}
 where we have used the closedness of $\Omega_R$ again. This shows that $\Omega$ has the desired properties concluding the proof.
\end{proof}

\begin{Rem}[Closedness of $\Omega_R$]\label{rem:closedness_of_Omega_R}
 Note that the closedness of $\Omega_R$ is crucial for Theorem \autoref{thm:rel_HSM_PHSM}: if $(X,\Omega = \Omega_R + i\Omega_I)$ is a HSM and $f:X\to\mathbb{R}_+$ is a positive and smooth function on $X$, then $f\cdot\Omega_R$ is still a non-degenerate $2$-form on $X$ which is anticompatible with the integrable complex structure $J$, however, neither $f\cdot \Omega_R$ nor $f\cdot \Omega_I$ are necessarily closed. In fact, $f\cdot \Omega$ is, in general, not even holomorphic.
\end{Rem}

Of course, we can also formulate Theorem \autoref{thm:rel_HSM_PHSM} for Hamiltonian systems:

\begin{Cor}[Relation between HHSs and PHHSs]\label{cor:rel_HHS_PHHS}
 Let $(X,J;\Omega_R, \mH_R)$ be a PHHS with $2$-forms $\Omega_I\coloneqq -\Omega_R (J\cdot,\cdot)$ and $\Omega\coloneqq \Omega_R + i\Omega_I$ as well as a function $\mH\coloneqq \mH_R + i\mH_I$, where $\mH_I$ is any primitive of the $1$-form $\Omega_R (J(X^{\Omega_R}_{\mH_R}),\cdot)$. Then, the following statements are equivalent:
 \begin{enumerate}
  \item $(X,\Omega, \mH)$ is a HHS with complex structure $J$.
  \item $\Omega_I$ is closed, $d\Omega_I = 0$.
  \item $J$ is integrable.
 \end{enumerate}
\end{Cor}

\begin{proof}
 Corollary \autoref{cor:rel_HHS_PHHS} is a direct consequence of Theorem \autoref{thm:rel_HSM_PHSM} and Remark \autoref{rem:H_pseudo-holo}.
\end{proof}

We can interpret Theorem \autoref{thm:rel_HSM_PHSM} and Corollary \autoref{cor:rel_HHS_PHHS} as follows: the integrability of the almost complex structure $J$ of a PHSM $(X,J;\Omega_R)$ or a PHHS $(X,J;\Omega_R,\mH_R)$ is completely measured by the closedness of the imaginary part $\Omega_I\coloneqq -\Omega_R (J\cdot,\cdot)$ and vice versa. Moreover, these quantities are the only local invariants of a PHSM or a PHHS: we know by Darboux's theorem for HSMs (cf. Theorem \autoref{thm:holo_Darboux}) that any HSM can locally be brought into standard form. Similarly, we will see in \autoref{subsec:deforming_HHS} that (regular) HHSs can locally also be brought into standard form. The existence of coordinates in which some geometrical object assumes a standard form implies that said geometrical object exhibits no local invariant. In this sense, Theorem \autoref{thm:rel_HSM_PHSM} and Corollary \autoref{cor:rel_HHS_PHHS} state that the Nijenhuis tensor $N_J$ of $J$ or, equivalently, the exterior derivative $d\Omega_I$ are the only local invariants of PHSMs and (regular) PHHSs. For general PHHSs, the Nijenhuis tensor and the behavior of the Hamiltonian near singular points are the only local invariants.
