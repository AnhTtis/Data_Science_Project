\section{Construction of Proper PHHSs and Deformation of HHSs}
\label{sec:deformation}

This section serves two purposes. The first is to present examples of proper PHHSs. In fact, we provide a general method for constructing PHHSs out of HHSs (cf. \autoref{subsec:constructing_PHHS}), which, for instance, allows us to equip the holomorphic cotangent bundle of a complex manifold with a (non-canonical) PHHS-structure. Secondly, we study the ``size'' of the set of proper PHHSs within the set of all PHHSs. The main result of this investigation is that proper PHHSs are generic (cf. \autoref{subsec:deforming_HHS}). To prove this, we deform HHSs by proper PHHSs.
% We obtain such a deformation by locally expressing the given HHS in standard form and applying the construction from \autoref{subsec:constructing_PHHS} afterwards to deform this standard form by proper PHHSs.


\subsection{Constructing proper PHHSs out of HHSs}
\label{subsec:constructing_PHHS}

In this subsection, we wish to construct examples of proper PHHSs. The basic idea of the construction is to start with a HHS $(X,\Omega,\mH)$ and turn it into a PHHS by transforming its complex structure $J$ into an almost complex structure $J_g \coloneqq I_g\circ J\circ I_g$ using a $\Omega_R$-\underline{compatible}\footnote{Here, $I_g$ is $\Omega_R$-compatible iff $\Omega_R (I_g\cdot, I_g\cdot) = \Omega_R$. In the literature on symplectic geometry, this term is used differently. Usually, if $(X,\Omega_R)$ is a symplectic manifold and $I_g$ is an almost complex structure on $X$, one says that $I_g$ is $\Omega_R$-compatible if $-g\coloneqq \Omega_R (\cdot,I_g\cdot)$ is a Riemannian metric. This is stronger than our condition, since our condition only implies that $-g$ is a semi-Riemannian metric, but not necessarily positive definite. Also note that we use $g\coloneqq \Omega_R (I_g\cdot,\cdot)$ instead of $-g$ in the following computations, since the sign is not important for us.} almost complex structure $I_g$. Let us make this idea precise:\\
Pick a HHS $(X,\Omega = \Omega_R + i\Omega_I,\mH = \mH_R + i\mH_I)$ with complex structure $J$. Next, choose an almost complex structure $I_g$ on $X$ satisfying $\Omega_R (I_g\cdot, I_g\cdot) = \Omega_R$ and consider the smooth $(1,1)$-tensor field $J_g\coloneqq I_g\circ J\circ I_g$ on $X$. Clearly, $J_g$ is an almost complex structure on $X$:
\begin{gather*}
 J^2_g = \left(I_g\circ J\circ I_g\right)\circ \left(I_g\circ J\circ I_g\right) = - I_g\circ J^2\circ I_g = I^2_g = -1.
\end{gather*}
Moreover, $J_g$ is still $\Omega_R$-anticompatible:
\begin{gather*}
 \Omega_R (J_g\cdot, J_g\cdot) = \Omega_R (J\circ I_g\cdot, J\circ I_g\cdot) = -\Omega_R (I_g\cdot, I_g\cdot) = -\Omega_R.
\end{gather*}
Thus, $(X,J_g;\Omega_R)$ is a PHSM. If $I_g$ was chosen such that the $1$-form $\Omega_R (J_g (X^{\Omega_R}_{\mH_R}),\cdot)$ is exact, then $(X,J_g;\Omega_R,\mH_R)$ is even a PHHS. The PHHS constructed this way is, in general, proper. To check that, it suffices by Corollary \autoref{cor:rel_HHS_PHHS} to compute the exterior derivative of\linebreak $\Omega^g_I\coloneqq -\Omega_R (J_g\cdot,\cdot)$.\\
In practice, one can find $\Omega_R$-compatible almost complex structures $I_g$ by picking suitable semi-Riemannian metrics $g\coloneqq \Omega_R (I_g\cdot,\cdot)$. Given any HHS $(X,\Omega,\mH)$, however, an arbitrary $\Omega_R$-compatible almost complex structure $I_g$ is usually not compatible with $\mH_R$ in the sense explained above. Most often, it is simpler to \underline{first} pick an almost complex structure $I_g$ and \underline{afterwards} pick a suitable real function $\mH_R$. Since $X$ is contractible in most examples we wish to study, e.g. local considerations and $X = \mathbb{C}^{2m}$, one can often find suitable $\mH_R$ by solving the differential equation $d[\Omega_R (J_g (X^{\Omega_R}_{\mH_R}),\cdot)] = 0$.\\
To illustrate the construction, let us consider the simplest non-trivial example:

\begin{Ex}[PHHS on $X = \mathbb{C}^2$]\label{ex:constructing_PHHS}\normalfont
 Let $(X,\Omega,\mH)$ be the HHS with $X = \mathbb{C}^2$, $\Omega = dz_2\wedge dz_1$, where $(z_1,z_2)\in\mathbb{C}^2$, and $\mH (z_1,z_2) = i\cdot z_1$. With the decomposition $z_j = x_j + i\cdot y_j$, we obtain:
 \begin{gather*}
  \Omega_R = dx_2\wedge dx_1 - dy_2\wedge dy_1;\quad \mH_R = -y_1.
 \end{gather*}
 The complex structure $J$ of the HHS $(X,\Omega,\mH)$ is simply given by multiplication with $i$:
 \begin{gather*}
  J (\partial_{x_j})\coloneqq \partial_{y_j};\quad J (\partial_{y_j})\coloneqq -\partial_{x_j}.
 \end{gather*}
 Now we pick the following semi-Riemannian metric $g$ on $\mathbb{C}^2$:
 \begin{gather*}
  g(\pa_{x_1},\pa_{x_1}) = g(\pa_{x_2},\pa_{x_2})^{-1} = f;\quad g(\pa_{y_1},\pa_{y_1}) = g(\pa_{y_2},\pa_{y_2})^{-1} = h;\\
  g(\pa_{x_1},\pa_{x_2}) = g(\pa_{y_1},\pa_{y_2}) = g(\pa_{x_i},\pa_{y_j}) = 0,
 \end{gather*}
 where $f,h:\mathbb{C}^2\to\mathbb{R}$ are smooth, nowhere-vanishing functions. The corresponding almost complex structure $I_g$ is given by:
 \begin{gather*}
  I_g (\pa_{x_1}) = f\pa_{x_2},\ I_g (\pa_{x_2}) = -f^{-1}\pa_{x_1};\quad I_g (\pa_{y_1}) = -h\pa_{y_2},\ I_g (\pa_{y_2}) = h^{-1}\pa_{y_1}.
 \end{gather*}
 One easily checks that $I_g$ is indeed an almost complex structure and $\Omega_R$-compatible. We now employ the notation $r\coloneqq f/h$ and set $J_g\coloneqq I_g\circ J\circ I_g$. Then, $J_g$ is given by
 \begin{gather*}
  J_g (\pa_{x_1}) = r\pa_{y_1},\ J_g (\pa_{x_2}) = r^{-1}\pa_{y_2};\quad J_g (\pa_{y_1}) = -r^{-1}\pa_{x_1},\ J_g (\pa_{y_2}) = -r\pa_{x_2}.
 \end{gather*}
 By construction, $J_g$ is an almost complex structure and $\Omega_R$-anticompatible. Thus,\linebreak $(X,J_g;\Omega_R)$ is a PHSM. We find for the induced $2$-form $\Omega^g_I\coloneqq -\Omega_R (J_g\cdot,\cdot)$:
 \begin{gather*}
  \Omega^g_I = r^{-1}dx_2\wedge dy_1 + rdy_2\wedge dx_1.
 \end{gather*}
 Hence, the exterior derivative of $\Omega^g_I$ can be expressed as
 \begin{gather*}
  d\Omega^g_I %= -r^{-2}dr\wedge dx_2\wedge dy_1 + dr\wedge dy_2\wedge dx_1
  = dr\wedge\left( dy_2\wedge dx_1 - r^{-2}dx_2\wedge dy_1\right).
 \end{gather*}
 In general, the exterior derivative $d\Omega^g_I$ does not vanish. For instance, we can set\linebreak $f (z_1, z_2) = 1$ and $h(z_1, z_2) = e^{x_1}$ resulting in $r (z_1, z_2) = e^{-x_1}$. This choice yields
 \begin{gather*}
  d\Omega^g_I = e^{x_1} dx_1\wedge dx_2\wedge dy_1
 \end{gather*}
 which does not vanish at any point of $X$. Thus, $(X,J_g;\Omega_R)$ is a proper PHSM for the presented choice of $f$ and $h$.\\
 %Next, we want to define a smooth real function $H$ on $M$ such that $(M,\omega, H, J_g)$ becomes a proper $J_g$-anticompatible Hamiltonian system. This means that the function $H$ needs to be chosen such that $\iota_{J_g (X_H)}\omega$ is exact where $X_H$ is the Hamiltonian vector field of $(M,\omega, H)$. If the first cohomology of $M$ vanishes, which is the case for $M = \mathbb{C}^2$, as $\mathbb{C}^2$ is contractible, it suffices to check that $\iota_{J_g (X_H)}\omega$ is closed. There is no generic way to ensure this, so we will only work in the example $M = \mathbb{C}^2$ from now on.\\
 Next, we check whether the $1$-form $\Omega_R (J_g (X^{\Omega_R}_{\mH_R}),\cdot)$ is exact. Since $X = \mathbb{C}^2$ is contractible, it suffices to check whether $\Omega_R (J_g (X^{\Omega_R}_{\mH_R}),\cdot)$ is closed. For the sake of generality, we first perform the computation for any smooth real function $H:\mathbb{C}^2\to\mathbb{R}$ and afterwards insert\linebreak $H = \mH_R = -y_1$. We start by determining the Hamiltonian vector field $X_H\equiv X^{\Omega_R}_H$. It can be written as
 \begin{gather*}
  X_H = (\pa_{x_2}H)\cdot\pa_{x_1} - (\pa_{x_1}H)\cdot\pa_{x_2} - (\pa_{y_2}H)\cdot\pa_{y_1} + (\pa_{y_1}H)\cdot\pa_{y_2}.
 \end{gather*}
 Applying $J_g$ to $X_H$ yields:
 \begin{gather*}
  J_g (X_H) = (r^{-1}\pa_{y_2}H)\cdot \pa_{x_1} - (r\pa_{y_1}H)\cdot \pa_{x_2} + (r\pa_{x_2}H)\cdot \pa_{y_1} - (r^{-1}\pa_{x_1}H)\cdot\pa_{y_2}.
 \end{gather*}
 Contracting $J_g (X_H)$ with $\Omega_R$ gives:
 \begin{align*}
  \Omega_R (J_g (X_H),\cdot) &= -(r\pa_{y_1}H)\cdot dx_1 - (r^{-1}\pa_{y_2}H)\cdot dx_2 + (r^{-1}\pa_{x_1}H)\cdot dy_1 + (r\pa_{x_2}H)\cdot dy_2\\
  &\equiv w_{x_1} dx_1 + w_{x_2} dx_2 + w_{y_1} dy_1 + w_{y_2} dy_2.
 \end{align*}
 For $\Omega_R (J_g (X_H),\cdot)$ to be closed, the coefficients $w$ need to satisfy the following conditions:
 \begin{gather*}
  \pa_{x_i} w_{x_j} = \pa_{x_j} w_{x_i};\quad \pa_{y_i} w_{y_j} = \pa_{y_j} w_{y_i};\quad \pa_{x_i} w_{y_j} = \pa_{y_j} w_{x_i}.
 \end{gather*}
 Let us now set $H = \mH_R = -y_1$. Then, all coefficients except for $w_{x_1} = r$ vanish. In particular, $\Omega_R (J_g (X_H),\cdot)$ is exact iff $r$ only depends on $x_1$. In this case, the primitive $\mH^g_I$ of $\Omega_R (J_g (X_H),\cdot)$ is given by $R$, where $R$ only depends on $x_1$ and satisfies $\pa_{x_1}R = r$. For instance, we can again set $f (z_1, z_2) = 1$ and $h(z_1, z_2) = e^{x_1}$ leading to\linebreak $r (z_1, z_2) = e^{-x_1}$. Then, $(X,J_g;\Omega_R,\mH_R)$ is a proper PHHS for this choice, where\linebreak $\mH^g_I = -e^{-x_1}$ is one possibility for the imaginary part of the Hamiltonian.
\end{Ex}

Of course, the construction presented above is not the only way to obtain a PHSM out of a HSM. For instance, the simple structure of the example $(X = \mathbb{C}^2, \Omega = dz_2\wedge dz_1)$ allows us to transform the standard complex structure $J$ into a new $\Omega_R$-anticompatible almost complex structure $J_\varphi$ by rotation of the axes:
\begin{alignat*}{2}
 J_\varphi (\pa_{x_1})&\coloneqq \cos (\varphi)\pa_{y_1} - \sin (\varphi)\pa_{y_2},\quad &&J_\varphi (\pa_{x_2})\coloneqq \sin (\varphi)\pa_{y_1} + \cos (\varphi)\pa_{y_2};\\
 J_\varphi (\pa_{y_1})&\coloneqq -\cos (\varphi)\pa_{x_1} - \sin (\varphi)\pa_{x_2},\quad &&J_\varphi (\pa_{y_2})\coloneqq \sin (\varphi)\pa_{x_1} - \cos (\varphi)\pa_{x_2},
\end{alignat*}
where $\varphi:\mathbb{C}^2\to\mathbb{R}$ is any smooth function. The resulting PHSM $(X, J_\varphi; \Omega_R)$ is also, in general, proper.\\
The reason why we focus our attention on the construction involving $I_g$ is that this construction possesses an interesting connection to hyperk{\"a}hler manifolds. Consider Example \autoref{ex:constructing_PHHS} again. If we choose $g$ to be the standard Euclidean metric $\delta$ on $\mathbb{C}^2\cong\mathbb{R}^4$, i.e., $f \equiv h\equiv 1$, then $I_\delta$ anti-commutes with the standard complex structure $J$. Thus, $J_\delta$ coincides with $J$ and our construction does not change the base HSM $(\mathbb{C}^2, dz_2\wedge dz_1)$. In fact, $(\mathbb{C}^2, dz_2\wedge dz_1)$ comes from the hyperk{\"a}hler structure on $\mathbb{C}^2$ defined by $\delta$, $I_\delta$, $J$, and $K\coloneqq I_\delta\circ J$. Here, we call a collection $(X,g,I,J,K)$ a \textbf{hyperk{\"a}hler manifold} iff $X$ is a smooth manifold with Riemannian metric $g$ and integrable almost complex structures $I$, $J$, and $K$ on it such that $I$, $J$, and $K$ fulfill the relation $I^2 = J^2 = K^2 = IJK = -1$ and $\omega^a\coloneqq -g(a\cdot,\cdot)$ is a symplectic $2$-form on $X$ for every $a\in\{I,J,K\}$. Every hyperk{\"a}hler manifold $(X,g,I,J,K)$ admits three holomorphic symplectic forms, namely:
\begin{gather*}
 \Omega^I\coloneqq \omega^J + i\omega^K;\quad \Omega^J\coloneqq\omega^K + i\omega^I;\quad \Omega^K\coloneqq\omega^I + i\omega^J,
\end{gather*}
where $\Omega^a$ is holomorphic with respect to $a\in\{I,J,K\}$. For the hyperk{\"a}hler manifold $(X = \mathbb{C}^2,\delta, I_\delta, J, K)$, the holomorphic symplectic form $\Omega = dz_2\wedge dz_1$ coincides with $-i\Omega^J$.\\
Now, note that the space of pairs $(f,h)$ of smooth, nowhere-vanishing functions on $\mathbb{C}^2$ has four connected components which are isomorphic via $(f,h)\mapsto (-f,h)$ and\linebreak $(f,h)\mapsto (f,-h)$. Furthermore, each connected component is contractible. Thus, we can reach any pair $(f,h)$ in this space by a path starting at $(1,1)$ and a discrete transformation. Next, consider such a path in this space, i.e., a smooth $1$-parameter family of nowhere-vanishing functions $f^\varepsilon$ and $h^\varepsilon$ on $\mathbb{C}^2$ such that $f^0\equiv h^0 \equiv 1$. This family induces a $1$-parameter family $J^\varepsilon$ of almost complex structures on $\mathbb{C}^2$ satisfying $J^0 = J$.
%which we can interpret as a deformation of the HHS $(\mathbb{C}^2,dz_2\wedge dz_1)$ by PHHSs\footnote{Confer \autoref{subsec:deforming_HHS} for the precise definition of deformations.}.
By construction, $J^\varepsilon$ itself is defined via a $1$-parameter family of almost complex structures $I^\varepsilon$ satisfying $I^0 = I_\delta$. With our previous knowledge, we can interpret $I^\varepsilon$ as some sort of deformation of the hyperk{\"a}hler manifold $(X = \mathbb{C}^2,\delta, I_\delta, J, K)$. Hence, we can say that every almost complex structure $J_g$ as in Example \autoref{ex:constructing_PHHS} giving rise to a PHSM comes from such a ``deformation'' of a hyperk{\"a}hler manifold up to a discrete transformation $J_g\mapsto -J_g$. In general, this also seems to be the best application of our construction: pick a HSM $(X,\Omega = -i\Omega^J)$ coming from a hyperk{\"a}hler manifold $(X,g,I,J,K)$ and then deform $I$ as described to find PHSMs.\\
Next, we show that the construction $J_g = I_g\circ J\circ I_g$ of PHSMs is applicable to a rather large class of complex manifolds $X$, namely the class of holomorphic cotangent bundles $X = T^{\ast, (1,0)}Y$ of complex manifolds $Y$. For this, let $Y$ be a complex manifold and $X\coloneqq T^{\ast, (1,0)}Y$ be its holomorphic cotangent bundle. The complex structure of $Y$ induces a canonical complex structure $J$ on $X$. Furthermore, $X$ as a holomorphic cotangent bundle possesses a canonical HSM structure with $2$-form $\Omega_\text{can} = \Omega_R + i\Omega_I$. We observe that the real part $\Omega_R$ of $\Omega_\text{can}$ can be identified with the canonical symplectic $2$-form $\omega_\text{can}$ on the real cotangent bundle $T^\ast Y$:

\begin{Prop}[$F^\ast\omega_\text{can} = \Omega_R$]\label{prop:F}
 Let $Y$ be a complex manifold. Then, the map\linebreak $F:T^{\ast, (1,0)}Y\to T^\ast Y$
 \begin{gather*}
  F(\alpha)\coloneqq \text{\normalfont Re} (\alpha)\quad\forall \alpha\in T^{\ast, (1,0)}Y,
 \end{gather*}
 where $\text{\normalfont Re} (\alpha)$ denotes the real part of $\alpha$, is a bundle isomorphism between smooth real vector bundles. In particular, $F$ is a diffeomorphism between the smooth manifolds $T^{\ast, (1,0)}Y$ and $T^\ast Y$ satisfying $F^\ast \omega_\text{can} = \Omega_R$.
\end{Prop}

\begin{proof}
 Take the notations from above. Let $(Q_1 = Q_{x,1} + iQ_{y,1},\ldots, Q_n = Q_{x,n} + iQ_{y,n}) = \psi$ be a holomorphic chart of $Y$. Then, $T^{\ast, (1,0)}\psi$ and $T^{\ast}\psi$ are (smooth) charts of $T^{\ast, (1,0)}Y$ and $T^\ast Y$, respectively:
 \begin{align*}
  \left(T^{\ast, (1,0)}\psi\right)^{-1} (\tilde{Q}_{x,1},\ldots, P_{y,n}) &\coloneqq \sum^n_{j=1} P_j dQ_j\vert_{\psi^{-1}(\tilde{Q}_{1},\ldots, \tilde{Q}_{n})}\\
  = \sum^n_{j = 1} &P_{x,j}dQ_{x,j}\vert_{\ldots} - P_{y,j}dQ_{y,j}\vert_{\ldots} + i\sum^n_{j=1}P_{x,j}dQ_{y,j}\vert_{\ldots} + P_{y,j}dQ_{x,j}\vert_{\ldots},\\
  \left(T^{\ast}\psi\right)^{-1} (\tilde{q}_{x,1},\ldots, p_{y,n}) &\coloneqq \sum^n_{j = 1} p_{x,j}dQ_{x,j}\vert_{\psi^{-1}(\tilde{q}_{1},\ldots, \tilde{q}_{n})} + p_{y,j}dQ_{y,j}\vert_{\psi^{-1}(\tilde{q}_{1},\ldots, \tilde{q}_{n})}.
 \end{align*}
 We denote the components of $T^{\ast, (1,0)}\psi$ by $T^{\ast, (1,0)}\psi = (Q_{x,1},\ldots, P_{y,n})$, while we denote the components of $T^{\ast}\psi$ by $T^{\ast}\psi = (q_{x,1},\ldots, p_{y,n})$. In these coordinates, $\Omega_\text{can}$ and $\omega_\text{can}$ can be expressed as:
 \begin{align*}
  \Omega_\text{can} &= \sum^n_{j = 1} dP_j\wedge dQ_j = \sum^n_{j=1} dP_{x,j}\wedge dQ_{x,j} - dP_{y,j}\wedge dQ_{y,j} + i\sum^n_{j=1}dP_{x,j}\wedge dQ_{y,j} + dP_{y,j}\wedge dQ_{x,j},\\
  \omega_\text{can} &= \sum^n_{j=1} dp_{x,j}\wedge dq_{x,j} + dp_{y,j}\wedge dq_{y,j}.
 \end{align*}
 Expressing $F$ in these coordinates gives:
 \begin{gather*}
  T^\ast\psi\circ F\circ \left(T^{\ast, (1,0)}\psi\right)^{-1} (\tilde{Q}_{x,j}, \tilde{Q}_{y,j}, P_{x,j}, P_{y,j}) = (\tilde{Q}_{x,j}, \tilde{Q}_{y,j}, P_{x,j}, -P_{y,j}).
 \end{gather*}
 In total, we obtain:
 \begin{gather*}
  F^\ast\omega_\text{can} = \sum^n_{j=1} dP_{x,j}\wedge dQ_{x,j} - dP_{y,j}\wedge dQ_{y,j} = \text{Re} (\Omega_\text{can}) = \Omega_R.
 \end{gather*}
\end{proof}

Now choose a semi-Riemannian metric $g_Y$ on $Y$. In \autoref{app:almost_complex_structures}, we show that any semi-Riemannian metric $g_Y$ on $Y$ induces an almost complex structure $J^\ast_{\nabla^{g_Y}}$ on $T^\ast Y$ which is $\omega_\text{can}$-compatible. Using $F$, we can transfer this almost complex structure from $T^\ast Y$ to $X$. We denote the result by $I_g\coloneqq dF^{-1}\circ J^\ast_{\nabla^{g_Y}}\circ dF$. By Proposition \autoref{prop:F}, $I_g$ on $X$ is also $\Omega_R$-compatible. Thus, $J_g\coloneqq I_g\circ J\circ I_g$ is $\Omega_R$-anticompatible and $(X,J_g;\Omega_R)$ is a PHSM. Note that $J_g$ depends on a choice of metric $g_Y$.\\
In general, this PHSM is proper, since we have not imposed any relation between $g_Y$ and the complex structure on $Y$. However, there is a special case in which $J_g$ is integrable, namely if $g_Y = h_R$ is the real part of a holomorphic metric $h = h_R + ih_I$ on $Y$:

\begin{Lem}[$g_Y = h_R\ \Rightarrow\ J$ and $I_g$ commute]
 Let $Y$ be a complex manifold with holomorphic metric $h = h_R + ih_I$. Then, $I_g$ obtained from the construction above for $g_Y = h_R$ commutes with the complex structure $J$ on $X\coloneqq T^{\ast, (1,0)}Y$. In particular, $J_g \coloneqq I_g\circ J\circ I_g = -J$.
\end{Lem}

\begin{proof}
 The idea of the proof is to choose coordinates in which $J$ and $I_g$ take a simple form. Let $p\in Y$ be any point. We start by choosing holomorphic coordinates\linebreak $(Q_1 = Q_{x,1} + iQ_{y,1},\ldots, Q_n = Q_{x,n} + iQ_{y,n}) = \psi$ of $Y$ near $p$ which, at the same time, are normal coordinates of $(Y,h_R)$ near $p$. In \autoref{app:holo_connection}, we show that such coordinates exist by considering normal coordinates of the holomorphic Levi-Civita connection $\nabla^h$ of the holomorphic metric $h$. The holomorphic normal coordinates $\psi$ then give rise to holomorphic coordinates $T^{\ast, (1,0)}\psi = (Q_1,\ldots, P_n = P_{x,n} + iP_{y,n})$ of $X$. As $J$ is the complex structure of $X$, $J$ takes the following form in $T^{\ast, (1,0)}\psi$ ($\alpha\in T^{\ast, (1,0)}_p Y$):
 \begin{alignat*}{2}
  J\left(\partial_{Q_{x,j}}\vert_\alpha\right) &= \partial_{Q_{y,j}}\vert_\alpha,\quad J\left(\partial_{Q_{y,j}}\vert_\alpha\right) &&= -\partial_{Q_{x,j}}\vert_\alpha,\\
  J\left(\partial_{P_{x,j}}\vert_\alpha\right) &= \partial_{P_{y,j}}\vert_\alpha,\quad J\left(\partial_{P_{y,j}}\vert_\alpha\right) &&= -\partial_{P_{x,j}}\vert_\alpha.
 \end{alignat*}
 In \autoref{app:almost_complex_structures}, we show that $I_g$ takes the following form\footnote{To be precise, we have determined the form of $J^\ast_{\nabla^{h_R}}$ in coordinates $T^\ast \psi$ in \autoref{app:almost_complex_structures}. We obtain the form of $I_g\coloneqq dF^{-1}\circ J^\ast_{\nabla^{h_R}}\circ dF$ by applying $F$. Mind the change of signs for $P_{y,j}$ due to $F$.} in $T^{\ast, (1,0)}\psi$:
 \begin{alignat*}{2}
  I_g\left(\partial_{Q_{x,j}}\vert_\alpha\right) &= -\partial_{P_{x,j}}\vert_\alpha,\quad I_g\left(\partial_{Q_{y,j}}\vert_\alpha\right) &&= -\partial_{P_{y,j}}\vert_\alpha,\\
  I_g\left(\partial_{P_{x,j}}\vert_\alpha\right) &= \partial_{Q_{x,j}}\vert_\alpha,\quad I_g\left(\partial_{P_{y,j}}\vert_\alpha\right) &&= \partial_{Q_{y,j}}\vert_\alpha.
 \end{alignat*}
 Now, one can check by direct computation that $J$ and $I_g$ commute at the point $\alpha$. Since the same argument can be repeated for every $\alpha\in T^{\ast, (1,0)}_pY$ and $p\in Y$, the result is true for all points of $X$ concluding the proof.
\end{proof}

Before we conclude this subsection and turn our attention to the deformation of HHSs, we should briefly illustrate one feature of proper PHHSs which is absent in the case of HHSs: the fact that the Hamiltonian vector fields $X^{\Omega_R}_{\mH_R}$ and $J(X^{\Omega_R}_{\mH_R})$ of a PHHS $(X,J;\Omega_R,\mH_R)$ do not need to be $J$-preserving (cf. Remark \autoref{rem:x_0-dependance}). Example \autoref{ex:constructing_PHHS} beautifully demonstrates that feature. To see this, let us first develop a criterium that tells us in which cases the vector fields $X^{\Omega_R}_{\mH_R}$ and $J(X^{\Omega_R}_{\mH_R})$ are $J$-preserving:

\begin{Prop}[When $X^{\Omega_R}_{\mH_R}$ and $J(X^{\Omega_R}_{\mH_R})$ are $J$-preserving]\label{prop:J-preserving_criteria}
 Let $(X,J;\Omega_R,\mH_R)$ be a PHHS with Hamiltonian vector field $X_\mH = 1/2 (X^{\Omega_R}_{\mH_R} - iJ(X^{\Omega_R}_{\mH_R}))$ and $2$-form\linebreak $\Omega_I = -\Omega_R (J\cdot,\cdot)$. Then, $X^{\Omega_R}_{\mH_R}$ is $J$-preserving if and only if $d\Omega_I (X^{\Omega_R}_{\mH_R},\cdot,\cdot) \equiv 0$. Similarly, $J(X^{\Omega_R}_{\mH_R})$ is $J$-preserving if and only if $d\Omega_I (J(X^{\Omega_R}_{\mH_R}),\cdot,\cdot) \equiv 0$.
\end{Prop}

\begin{proof}
 Let $(X,J;\Omega_R,\mH_R)$ be a PHHS with $2$-form $\Omega_I = -\Omega_R (J\cdot,\cdot)$ and Hamiltonian vector field $X_\mH = 1/2 (X^{\Omega_R}_{\mH_R} - iJ(X^{\Omega_R}_{\mH_R}))$. Take $V\in\{X^{\Omega_R}_{\mH_R}, J(X^{\Omega_R}_{\mH_R})\}$. We want to determine in which cases $V$ is $J$-preserving. Thereto, we use Proposition 2.10 in Chapter IX of \cite{kobayashi1969}:\linebreak A (real) vector field $V$ on a smooth manifold $X$ with almost complex structure $J$ is $J$-preserving if and only if the following equation is fulfilled for every vector field $W$:
 \begin{gather*}
  [V, J(W)] = J([V,W]).
 \end{gather*}
 Since $\Omega_R$ is non-degenerate, we find that $V$ is $J$-preserving if and only if
 \begin{gather*}
  \iota_{[V, J(W)]}\Omega_R - \iota_{J([V,W])}\Omega_R = 0
 \end{gather*}
 for every vector field $W$ on $X$. Let us now compute the left-hand side of this equation. First, we obtain by definition of $\Omega_I$:
 \begin{gather*}
  \iota_{J([V,W])}\Omega_R = -\iota_{[V,W]}\Omega_I.
 \end{gather*}
 Now remember the relation (cf. Proposition 3.10 in Chapter I of \cite{kobayashi1963}):
 \begin{gather*}
  \iota_{[V,W]} = [L_V,\iota_W].
 \end{gather*}
 Furthermore, recall that by Definition \autoref{def:PHHS_1} the $1$-forms $\Omega_R (V,\cdot)$ and\linebreak $\Omega_I (V,\cdot) = -\Omega_R (J(V),\cdot)$ are exact. Together with $d\Omega_R = 0$ and Cartan's magic formula, this implies:
 \begin{gather*}
  L_V\Omega_R = 0;\quad L_V\Omega_I = \iota_V d\Omega_I,
 \end{gather*}
 where $L_V$ denotes the Lie derivative of $V$. These relations allow us to compute:
 \begin{align*}
  \iota_{[V,J(W)]}\Omega_R &= [L_V,\iota_{J(W)}]\Omega_R = L_V\left(\iota_{J(W)}\Omega_R\right) - \iota_{J(W)}\left(L_V\Omega_R\right)\\
  &= -L_V\left(\iota_W\Omega_I\right) = -[L_V,\iota_W]\Omega_I - \iota_W\left(L_V\Omega_I\right)\\
  &= -\iota_{[V,W]}\Omega_I - \iota_W\iota_V d\Omega_I = \iota_{J([V,W])}\Omega_R - \iota_W\iota_V d\Omega_I.
 \end{align*}
 In total, we find that $V$ is $J$-preserving if and only if
 \begin{gather*}
  \iota_{[V, J(W)]}\Omega_R - \iota_{J([V,W])}\Omega_R = - \iota_W\iota_V d\Omega_I = 0
 \end{gather*}
 for every vector field $W$ on $X$ concluding the proof.
\end{proof}

Now let us consider Example \autoref{ex:constructing_PHHS} again with $f = 1$, $h = e^{x_1}$, and $H = \mH_R = -y_1$. Then, the Hamiltonian vector fields and the exterior derivative of $\Omega^g_I$ are given by:
\begin{gather*}
 X^{\Omega_R}_{\mH_R} = -\pa_{y_2};\quad J_g(X^{\Omega_R}_{\mH_R}) = e^{-x_1}\pa_{x_2};\quad d\Omega^g_I = e^{x_1} dx_1\wedge dx_2\wedge dy_1.
\end{gather*}
Using Proposition \autoref{prop:J-preserving_criteria}, we immediately see that $X^{\Omega_R}_{\mH_R}$ is $J_g$-preserving, while $J_g(X^{\Omega_R}_{\mH_R})$ is not. By choosing $H$ to be a linear combination of $\mH_R = -y_1$ and $\mH^g_I = -e^{-x_1}$, Example \autoref{ex:constructing_PHHS} even shows that neither $X^{\Omega_R}_{\mH_R}$ nor $J(X^{\Omega_R}_{\mH_R})$ of a PHHS $(X,J;\Omega_R,\mH_R)$ need to be $J$-preserving.


\subsection{Deforming HHSs by proper PHHSs}
\label{subsec:deforming_HHS}

In this subsection, we examine the question: ``How `large' is the set of proper PHHSs within the set of all PHHSs?'' The answer and the main result of this subsection is that proper PHHSs are generic if $\text{dim}_\mathbb{R}(X)>4$. To prove this, we first reduce the genericity of proper PHHSs to the claim that every HHS can be deformed by proper PHHSs. Afterwards, we show this claim in two steps. The first step is to locally bring every HHS into standard form. Secondly, we deform the HHS standard form by proper PHHSs within a small neighborhood.\\
We start by recalling the definition of a generic property. For a topological space $B$ and a subset $A\subset B$, we call the property that an element is contained in $A$ \textbf{generic} iff $B\backslash A$ is a meagre subset of $B$ in the sense of Baire. In particular, the property $a\in A$ is generic if $A$ is an open and dense subset of $B$. Now let $X$ be a smooth manifold. We introduce the following notations for the set of almost complex structures, of PHSMs, and of PHHSs on $X$:
\begin{align*}
 \mathcal{J}_\text{a.c.} (X)&\coloneqq \{J\in\Gamma (\text{End}(TX))\mid J^2 = -1\},\\
 \text{PHSM} (X)&\coloneqq \{(J,\Omega_R)\in\mathcal{J}_\text{a.c.}(X)\times\Omega^2 (X)\mid (X,J;\Omega_R)\text{ is a PHSM}\},\\
 \text{PHHS} (X)&\coloneqq \{(J,\Omega_R,\mH_R)\in\mathcal{J}_\text{a.c.}(X)\times\Omega^2 (X)\times C^\infty (X,\mathbb{R})\mid (X,J;\Omega_R,\mH_R)\text{ is a PHHS}\}.
\end{align*}
Similarly, we denote the set of complex structures, of HSMs, and of HHSs on $X$ by:
\begin{align*}
 \mathcal{J}_\text{c} (X)&\coloneqq \{J\in\mathcal{J}_\text{a.c.}(X)\mid J\text{ is integrable}\},\\
 \text{HSM} (X)&\coloneqq \{(J,\Omega_R)\in \text{PHSM} (X)\mid J\text{ is integrable}\},\\
 \text{HHS} (X)&\coloneqq \{(J,\Omega_R,\mH_R)\in \text{PHHS} (X)\mid J\text{ is integrable}\}.
\end{align*}
Lastly, we write for the set of proper, i.e., non-integrable almost complex structures, of proper PHSMs, and of proper PHHSs on $X$:
\begin{align*}
 \mathcal{J}_\text{p} (X)&\coloneqq \mathcal{J}_\text{a.c.} (X)\backslash \mathcal{J}_\text{c} (X),\\
 \text{PHSM}_\text{p} (X)&\coloneqq \text{PHSM} (X)\backslash \text{HSM} (X),\\
 \text{PHHS}_\text{p} (X)&\coloneqq \text{PHHS} (X)\backslash \text{HHS} (X).
\end{align*}
We equip every set with the topology induced by the compact-open topology. We wish to show that $\mathcal{J}_\text{p} (X)\subset \mathcal{J}_\text{a.c.} (X)$, $\text{PHSM}_\text{p} (X)\subset \text{PHSM} (X)$, and $\text{PHHS}_\text{p} (X)\subset \text{PHHS} (X)$ are open and dense subsets if $\text{dim}_\mathbb{R} (X)>4$. Clearly, the sets in question are open subsets, since all former mentioned subsets contain exactly those elements from their respective supersets for which the Nijenhuis tensor $N_J$ of $J$ does not vanish. The argument is completed by noting that $N_J\neq 0$ is an open condition.\\
It is left to show that the subsets are dense in their respective supersets. This can be done by showing that the ``integrable'' sets $\mathcal{J}_\text{c} (X)$, $\text{HSM} (X)$, and $\text{HHS}(X)$ are contained in the boundary of the ``proper'' sets $\mathcal{J}_\text{p}$, $\text{PHSM}_\text{p} (X)$, and $\text{PHHS}_\text{p} (X)$, respectively. The last statement is true if the ``integrable'' elements can be deformed by ``proper'' elements:

\begin{Def}[(Proper) deformation]\label{def:deformation}
 Let $X$ be a smooth manifold with almost complex structure $J$ on it. $(X,J^\varepsilon)$ is called a \textbf{deformation} of $(X,J)$ iff $J^\varepsilon$ describes a smooth\footnote{Here, smooth path means that the map $J^\varepsilon:X\times\mathbb{R}\to \text{\normalfont End}(TX)$ is smooth. Similar remarks apply to $\Omega^\varepsilon_R$ and $\mH^\varepsilon_R$.} path in $\mathcal{J}_\text{\normalfont a.c.} (X)$ with start point $J^0 = J$. Now let $\Omega_R$ be a $2$-form on $X$ such that $(X,J;\Omega_R)$ is a PHSM. Then, $(X,J^\varepsilon; \Omega^\varepsilon_R)$ is a deformation of $(X,J;\Omega_R)$ iff $(J^\varepsilon, \Omega^\varepsilon_R)$ describes a smooth path in $\text{\normalfont PHSM}(X)$ with start point $(J^0, \Omega^0_R) = (J,\Omega_R)$. If, additionally, $\mH_R$ is a function on $X$ such that $(X,J;\Omega_R,\mH_R)$ is a PHHS, we say $(X,J^\varepsilon; \Omega^\varepsilon,\mH^\varepsilon_R)$ is a deformation of $(X,J;\Omega_R,\mH_R)$ iff $(J^\varepsilon, \Omega^\varepsilon_R, \mH^\varepsilon_R)$ describes a smooth path in $\text{\normalfont PHHS}(X)$ with start point $(J^0, \Omega^0_R,\mH^0_R) = (J,\Omega_R,\mH_R)$. We call a deformation \textbf{proper} iff the corresponding $J^\varepsilon$ is not integrable for $\varepsilon\neq 0$.
\end{Def}

Before we continue with the proof of genericity, we shall quickly address one aspect concerning our definitions: in the definition of a deformation, we have neglected the imaginary parts $\Omega_I$ and $\mH_I$, even though they are crucial for the definition of HSMs and HHSs. One might wonder whether this is justified, i.e., whether every deformation of a PHSM or a PHHS automatically gives us a suitable $1$-parameter family $\Omega^\varepsilon_I$ and $\mH^\varepsilon_I$ of imaginary parts. Regarding the form $\Omega_I$, this is certainly true by simply setting $\Omega^\varepsilon_I\coloneqq -\Omega^\varepsilon_R (J^\varepsilon\cdot,\cdot)$. However, $\mH_I$ defined as a primitive of $\Omega_R (J(X^{\Omega_R}_{\mH_R}),\cdot)$ might be more problematic. It is not obvious why a smooth $1$-parameter family of exact $1$-forms $\alpha^\varepsilon$ should be the differential of a smooth $1$-parameter family of functions $f^\varepsilon$. Nevertheless, this is still true in the case of $X$ being contractible:

\begin{Prop}[Primitive of $\alpha^\varepsilon$]\label{prop:primitive}
 Let $X$ be a smooth contractible manifold, $\alpha^\varepsilon$ be a smooth $1$-parameter family of exact $1$-forms on $X$, and $f\in C^\infty (X,\mathbb{R})$ such that $\alpha^0 = df$. Then, there exists a smooth $1$-parameter family of functions $f^\varepsilon$ on $X$ such that $\alpha^\varepsilon = df^\varepsilon$ and $f^0 = f$.
\end{Prop}

\begin{proof}
 Let $X$, $\alpha^\varepsilon$, and $f$ be as above. $X$ is contractible, so there is a smooth map $F:[0,1]\times X\to X$ such that $F(0,x) = x_0$ and $F(1,x) = x$ for every $x\in X$ and some $x_0\in X$. For fixed $\varepsilon$, $F^\ast\alpha^\varepsilon$ is a closed $1$-form on $[0,1]\times X$ and can be expressed as (using $F_t:X\to X$, $F_t (x)\coloneqq F (t,x)$ for every $t\in [0,1]$):
 \begin{gather*}
  F^\ast\alpha^\varepsilon = F^\ast_t\alpha^\varepsilon + \beta^\varepsilon_t \cdot dt,
 \end{gather*}
 where $\beta^\varepsilon_t$ is a function on $X$ smoothly depending on $\varepsilon$ and $t$. For fixed $t$ and $\varepsilon$, $F^\ast_t\alpha^\varepsilon$ can be understood as a closed $1$-form on $X$. The closedness of $F^\ast\alpha^\varepsilon$ and $F^\ast_t\alpha^\varepsilon$ together with the expression for $F^\ast\alpha^\varepsilon$ implies:
 \begin{gather*}
  \frac{d}{dt}F^\ast_t\alpha^\varepsilon = d\beta^\varepsilon_t.
 \end{gather*}
 In total, we obtain:
 \begin{align*}
  \alpha^\varepsilon = \text{id}^\ast_X\alpha^\varepsilon - \text{const}^\ast_{x_0}\alpha^\varepsilon = F^\ast_1\alpha^\varepsilon - F^\ast_0\alpha^\varepsilon = \int\limits^1_0 \frac{d}{dt}(F^\ast_t\alpha^\varepsilon)\, dt = d\int\limits^1_0 \beta^\varepsilon_t dt\eqqcolon df^\varepsilon.
 \end{align*}
 Shifting $f^\varepsilon$ by a constant to match $f$ at $\varepsilon = 0$ concludes the proof.
\end{proof}

To show that the property ``proper'' is generic, we will only use local deformations. In particular, we can always shrink the neighborhood in which we deform an integrable structure such that it becomes contractible. Thus, we can rightfully neglect deformations of the imaginary parts $\Omega_I$ and $\mH_I$ here.\\
Let us return to the proof of genericity. As explained, it suffices to find a proper deformation of every ``integrable'' element to show genericity. We construct the desired proper deformations in two steps:
\begin{enumerate}
 \item We bring the ``integrable'' elements locally into standard form.
 \item We deform the standard form ``properly'' within a small neighborhood.
\end{enumerate}
Regarding the first step, it is clear what the standard forms of complex manifolds and HSMs are and how to achieve them. For complex manifolds, the standard form is just $X = \mathbb{C}^m$ with complex structure $J = i$ and every complex manifold can be brought into standard form by holomorphic charts. For HSMs, the standard form is $X = \mathbb{C}^{2n}$ together with $\Omega = \sum_j dP_j\wedge dQ_j$ and every HSM can be brought into standard form by holomorphic Darboux charts (cf. Theorem \autoref{thm:holo_Darboux} and \autoref{app:darboux}).\\
For a HHS $(X,\Omega,\mH)$, we observe that $\mH$ is either locally constant or regular at some point, i.e., there exists $x_0\in X$ such that $d\mH\vert_{x_0}\neq 0$. If the given HHS is locally constant near a point, then holomorphic Darboux coordinates describe the standard form of the HHS in question near that point. If the given HHS is regular near a point, then the following lemma brings it into standard form:

\begin{Lem}[Regular HHSs in standard form]\label{lem:HHS_in_standard_form}
 Let $(X,\Omega,\mH)$ be a HHS and\linebreak $x_0\in X$ be a point such that $d\mH\vert_{x_0}\neq 0$. Then, there exists a holomorphic chart\linebreak $\phi = (z_1,\ldots, z_{2n}):U\to V\subset\mathbb{C}^{2n}$ of $X$ near $x_0\in U$ such that $\Omega\vert_U = \sum^n_{j = 1} dz_{j+n}\wedge dz_j$ and $\mH\vert_U = z_{2n}$.
\end{Lem}

\begin{proof}
 Let $(X,\Omega,\mH)$ be a HHS and $x_0\in X$ be a point such that $d\mH\vert_{x_0}\neq 0$. Without loss of generality, we can assume $\mH (x_0) = 0$. The proof consists of three steps:
 \begin{enumerate}
  \item Construct a holomorphic function $G$ defined locally near $x_0$ satisfying\linebreak $\{\mH, G\}\coloneqq \Omega (X_\mH, X_G) = 1$.
  \item Find a holomorphic chart $\phi_3 = (z^\#_1,\ldots, z^\#_{2n}):U_3\to V_3$ of $X$ near $x_0$ such that $G\vert_{U_3} = z^\#_n$, $\mH\vert_{U_3} = z^\#_{2n}$, $X_\mH\vert_{U_3} = \pa_{z^\#_n}$, and $X_G\vert_{U_3} = -\pa_{z^\#_{2n}}$.
  \item We have $\Omega\vert_{U_3} = dz^\#_{2n}\wedge dz^\#_{n} + \Sigma$ where $\Sigma$ is a closed $2$-form only depending on $z^\#_1,\ldots, z^\#_{n-1}, z^\#_{n+1},\ldots, z^\#_{2n-1}$. Bring $\Sigma$ into standard form using Darboux's theorem for HSMs (cf. Theorem \autoref{thm:holo_Darboux} and \autoref{app:darboux}).
 \end{enumerate}
 \textbf{Step 1:} We pick a small open neighborhood $U_1$ of $x_0$ such that $U_1$ is the domain of a holomorphic chart $\phi_1 = (\hat z_1,\ldots, \hat z_{2n}):U_1\to V_1\subset\mathbb{C}^{2n}$ with $\phi_1 (x_0) = 0$ and $d\mH$ does not vanish on $U_1$. $\pa_{\hat z_1},\ldots, \pa_{\hat z_{2n}}$ form a holomorphic frame of the tangent bundle $T^{(1,0)}U_1$. Since the holomorphic Hamiltonian vector field $X_\mH$ of the HHS $(X,\Omega,\mH)$ does not vanish on $U_1$, we can replace one coordinate vector field, say $\pa_{\hat z_n}$, in the collection $\pa_{\hat z_1},\ldots, \pa_{\hat z_{2n}}$ with $X_\mH$ and still obtain a frame of $T^{(1,0)}U_1$ after shrinking $U_1$ if necessary.\\
 Now let $\varphi^{X_\mH}_z:U_1\to X$ be the complex flow of the holomorphic Hamiltonian vector field $X_\mH$ for suitable $z\in\mathbb{C}$. $\varphi^{X_\mH}_z$ is constructed as follows: for $x\in X$, we denote by $\gamma_x$ the holomorphic curve from $\mathbb{C}$ to $X$ solving the following initial value problem:
 \begin{gather*}
  \gamma^\prime_x (z) = X_\mH (\gamma_x (z));\quad \gamma_x (0) = x.
 \end{gather*}
 By Proposition \autoref{prop:holo_traj}, the curves $\gamma_x$ exist and are locally unique. With this, we can set the flow of $X_\mH$ to be $\varphi^{X_\mH}_z (x)\coloneqq \gamma_x (z)$.\\
 Let us now consider the map $\phi^{-1}_2$ from $\mathbb{C}^{2n}$ to $X$ defined by
 \begin{gather*}
  \phi^{-1}_2(\tilde z_1,\ldots, \tilde z_{2n})\coloneqq \varphi^{X_\mH}_{\tilde z_n}\circ \phi^{-1}_1 (\tilde z_1,\ldots, \tilde z_{n-1}, 0, \tilde z_{n+1},\ldots, \tilde z_{2n}).
 \end{gather*}
 The differential of $\phi^{-1}_2$ at $0\in\mathbb{C}^{2n}$ maps the standard complex basis of $\mathbb{C}^{2n}$ to the vector fields $\pa_{\hat z_1},\ldots, \pa_{\hat z_{n-1}}, X_{\mH}, \pa_{\hat z_{n+1}}, \ldots, \pa_{\hat z_{2n}}$ at $x_0\in X$. These vector fields form a complex basis at $x_0$, hence, $\phi^{-1}_2$ is a local biholomorphism near $0$ by the holomorphic\linebreak version of the inverse function theorem. This gives us the holomorphic chart\linebreak $\phi_2 = (\tilde z_1,\ldots, \tilde z_{2n}):U_2\to V_2$, where $U_2$ is a small open neighborhood of $x_0$.\\
 Next, we set $G:U_2\to\mathbb{C}$ to be the coordinate $\tilde z_{n}$. We find:
 \begin{gather*}
  dG\vert_x (X_\mH\vert_x) = \left.\frac{d}{dz}\right\vert_{z = 0} (G\circ \varphi^{X_\mH}_z (x)) = 1\quad\forall x\in U_2.
 \end{gather*}
 This implies for the Poisson bracket of $\mH$ and $G$:
 \begin{gather*}
  \{\mH, G\} \coloneqq \Omega (X_\mH, X_G) = dG (X_\mH) = 1,
 \end{gather*}
 where $X_G$ is the holomorphic Hamiltonian vector field of the HHS $(U_2,\Omega\vert_{U_2}, G)$.\\
 \textbf{Step 2:} Taking the Hamiltonian vector field of a holomorphic function is a Lie algebra homomorphism, hence, we get:
 \begin{gather*}
  [X_\mH, X_G] = X_{\{\mH, G\}} = 0.
 \end{gather*}
 As in the real case, the commutativity of the vector fields implies the\linebreak commutativity of their flows. This allows us to find a holomorphic chart\linebreak $\phi_3 = (z^\#_1,\ldots, z^\#_{2n}):U_3\to V_3$ of $X$ near $x_0$ such that $G\vert_{U_3} = z^\#_n$, $\mH\vert_{U_3} = z^\#_{2n}$, $X_\mH\vert_{U_3} = \pa_{z^\#_n}$, and $X_G\vert_{U_3} = -\pa_{z^\#_{2n}}$. The construction of $\phi_3$ makes use of the regular value theorem for complex manifolds. Consider the holomorphic map $f\coloneqq(H\vert_{U_2},G):U_2\to\mathbb{C}^2$. The map $f$ is a submersion due to $\Omega (X_\mH, X_G) = 1$. By the regular value theorem, the level sets of $f$ are complex submanifolds of $U_2$. The tangent space of a level set is given by the kernel of $df$. Now pick the level set $W\coloneqq f^{-1}(\mH (x_0), 0) = f^{-1}(0,0)$ containing $x_0$ and a holomorphic chart $\psi$ of $W$ near $x_0$ with $\psi (x_0) = 0$. Next, we define the map $\phi^{-1}_3$ via:
 \begin{gather*}
  \phi^{-1}_3 (z^\#_1,\ldots, z^\#_{2n})\coloneqq \varphi^{X_\mH}_{z^\#_n}\circ \varphi^{X_G}_{-z^\#_{2n}}\circ \psi^{-1} (z^\#_1,\ldots, z^\#_{n-1}, z^\#_{n+1}, \ldots, z^\#_{2n-1}).
 \end{gather*}
 The differential of $\phi^{-1}_3$ at $0\in\mathbb{C}^{2n}$ maps the standard complex basis of $\mathbb{C}^{2n}$ to the vector fields $X_\mH$, $-X_G$, and the coordinate vector fields of $\psi$ at $x_0$. The coordinate vector fields $v_j$ of $\psi$ satisfy $d\mH (v_j) = dG (v_j) = 0$, hence, are orthogonal to $X_\mH$ and $X_G$ with respect to the symplectic form $\Omega$. Thus, by $\Omega (X_\mH, X_G) = 1$ and standard arguments from linear symplectic algebra, the coordinate vector fields of $\psi$ form together with $X_\mH$ and $-X_G$ a complex basis of $T^{(1,0)}_{x_0}X$. We can again apply the inverse function theorem to show that $\phi^{-1}_3$ is locally biholomorphic near $0$. From $\phi^{-1}_3$, we obtain the holomorphic chart $\phi_3 = (z^\#_1,\ldots, z^\#_{2n}):U_3\to V_3$ of $X$ near $x_0$. Using the commutativity of $\varphi^{X_\mH}_{z^\#_n}$ and $\varphi^{X_G}_{-z^\#_{2n}}$,\linebreak we easily compute $\pa_{z^\#_n} = X_\mH\vert_{U_3}$ and $\pa_{z^\#_{2n}} = -X_G\vert_{U_3}$. By construction of $\phi_3$ via level sets of $\mH$ and $G$, by exploiting the fact that Hamiltonian flows preserve $\Omega$, and by using $d\mH (X_\mH) = dG (X_G) = 0$, we deduce that the functions $\mH$ and $G$ only change in $X_G$- and $X_\mH$-direction, respectively. Therefore, $\mH$ only depends on $z^\#_{2n}$ and $G$ only depends on $z^\#_{n}$. Integrating $dG(X_\mH) = d\mH (-X_G) = \Omega (X_\mH, X_G) = 1$ gives us $G = z^\#_{n}$ and $\mH = z^\#_{2n}$.\\
 \textbf{Step 3:} Let us inspect $\Omega$ in the chart $\phi_3$ more closely. The coordinates of $\phi_3$ satisfy $\iota_{\pa_{z^\#_{n}}}\Omega\vert_{U_3} = -dz^\#_{2n}$ and $\iota_{\pa_{z^\#_{2n}}}\Omega\vert_{U_3} = dz^\#_{n}$. Therefore, $\Omega\vert_{U_3}$ takes the following form:
 \begin{gather*}
  \Omega\vert_{U_3} = dz^\#_{2n}\wedge dz^\#_n + \sum\limits_{i,j\neq n,2n} \Omega_{ij} dz^\#_{i}\wedge dz^\#_{j}\eqqcolon dz^\#_{2n}\wedge dz^\#_n + \Sigma.
 \end{gather*}
 $\Omega\vert_{U_3}$ and $dz^\#_{2n}\wedge dz^\#_n$ are closed $2$-forms on $U_3$, thus, $\Sigma$ is also a closed $2$-form on $U_3$. $d\Sigma = 0$ implies that the partial derivatives of $\Omega_{ij}$ with respect to $z^\#_{n}$ and $z^\#_{2n}$ have to vanish for every $i,j\neq n,2n$. Thus, the restriction of $\Sigma$ to a hyperplane $z^\#_n = c_1$ and $z^\#_{2n} = c_2$ does not depend on the values $c_1$ and $c_2$. Furthermore, the restriction of $\Sigma$ is a holomorphic symplectic $2$-form on such a hyperplane. This is a direct consequence of $\Omega\vert_{U_3}$ being a holomorphic symplectic $2$-form on $U_3$. Therefore, we can apply Theorem \autoref{thm:holo_Darboux} to $\Sigma$ giving us a holomorphic chart $\Psi$ of the hyperplane $z^\#_n = 0$ and $z^\#_{2n} = 0$ in which $\Sigma$ assumes the standard form. Replacing the coordinates $z^\#_1,\ldots, z^\#_{n-1}, z^\#_{n+1}, \ldots, z^\#_{2n-1}$ of $\phi_3$ with the coordinates of $\Psi$ yields the holomorphic chart $\phi = (z_1,\ldots, z_{2n}):U\to V$ of $X$ near $x_0$ (choose $\Psi (x_0) = 0$):
 \begin{gather*}
  \phi^{-1} (z_1,\ldots, z_{2n})\coloneqq \varphi^{X_\mH}_{z_n}\circ \varphi^{X_G}_{-z_{2n}}\circ \Psi^{-1} (z_1,\ldots, z_{n-1}, z_{n+1}, \ldots, z_{2n-1}).
 \end{gather*}
 In this chart, $\Omega$ and $\mH$ take the form $\Omega\vert_U = \sum^n_{j = 1} dz_{j+n}\wedge dz_j$ and $\mH\vert_U = z_{2n}$ concluding the proof.
\end{proof}

\begin{Cor}[Every regular HHS is locally integrable as a Hamiltonian system]\label{cor:integrable}
 Let $(X,\Omega,\mH)$ be a HHS and $x_0\in X$ be a point such that $d\mH\vert_{x_0}\neq 0$. Then, there exists an open neighborhood $U$ of $x_0$ and holomorphic functions $F_i,G_i:U\to\mathbb{C}$ for $i\in\{1,\ldots, n\}$, $2n\coloneqq \text{\normalfont dim}_\mathbb{C}(X)$, such that:
 \begin{gather*}
  F_n = \mH\vert_U,\quad \{F_i, G_j\} = \delta_{ij},\quad \{F_i, F_j\} = \{G_i,G_j\} = 0\quad\forall i,j\in\{1,\ldots, n\}.
 \end{gather*}
 In particular, every regular HHS is locally integrable as a Hamiltonian system.
\end{Cor}

\begin{proof}
 Take the chart $\phi$ from Lemma \autoref{lem:HHS_in_standard_form} and set $F_j\coloneqq z_{j + n}$ as well as $G_j\coloneqq z_j$.
\end{proof}

\begin{Rem}[Every regular RHS is locally integrable]\label{rem:integrable}
 Lemma \autoref{lem:HHS_in_standard_form} and Corollary \autoref{cor:integrable} are also true for regular RHSs with almost exactly the same proofs. In particular, every regular RHS is locally integrable.
\end{Rem}

With the ``integrable'' elements in standard form, let us now turn our attention to the second step. We need to construct local proper deformations of the standard form. Here, we only show how to deform HHSs explicitly. The deformations of HSMs and complex manifolds can be obtained in a similar way.

\begin{Prop}[Deformation of standard HHSs]\label{prop:deformation_of_standard_HHS}
 Let $(X,\Omega,\mH)$ be a HHS with complex structure $J$ and decompositions $\Omega = \Omega_R + i\Omega_I$ and $\mH = \mH_R + i\mH_I$, where $X = \mathbb{C}^{2n}\ (n>1)$, $J = i$, $\Omega = \sum^n_{j = 1} z_{j+n}\wedge z_j$, and $\mH \equiv c$ for some constant $c\in\mathbb{C}$ or $\mH = z_{2n}$. Furthermore, let $U\subset \mathbb{C}^{2n}$ be any non-empty open subset. Then, there exists a proper deformation $(X,J^\varepsilon;\Omega^\varepsilon_R,\mH^\varepsilon_R)$ of the HHS $(X,\Omega,\mH)$ such that $J^\varepsilon\vert_{X\backslash U} = J\vert_{X\backslash U}$, $\Omega^\varepsilon_R = \Omega_R$, and $\mH^\varepsilon_R = \mH_R$ for every $\varepsilon\in\mathbb{R}$.
\end{Prop}

\begin{proof}
 The idea for the deformation is based on Example \autoref{ex:constructing_PHHS}. Let $(X,\Omega,\mH)$ be a HHS as above with non-empty open subset $U\subset X$. Now pick a non-constant smooth function $f:X\cong\mathbb{R}^{4n}\to\mathbb{R}$ satisfying:
 \begin{gather*}
  f(x)\geq 0\ \forall x\in X,\quad f(x) = 0\ \forall x\in X\backslash U.
 \end{gather*}
 We define the $1$-parameter family of smooth functions $r^\varepsilon$ on $X$ as follows:
 \begin{gather*}
  r^\varepsilon (x)\coloneqq 1 + \varepsilon^2 f(x)\quad \forall x\in X\ \forall \varepsilon\in\mathbb{R}.
 \end{gather*}
 Using $r^\varepsilon$, we define the $1$-parameter family of almost complex structures $J^\varepsilon$ on $X$:
 \begin{align*}
  J^\varepsilon (\pa_{x_1}) \coloneqq r^\varepsilon\pa_{y_1},\ J^\varepsilon (\pa_{x_{n+1}}) \coloneqq \frac{1}{r^\varepsilon}\pa_{y_{n+1}}&;\quad J^\varepsilon (\pa_{y_1}) \coloneqq \frac{-1}{r^\varepsilon}\pa_{x_1},\ J^\varepsilon (\pa_{y_{n+1}}) \coloneqq -r^\varepsilon\pa_{x_{n+1}};\\
  J^\varepsilon (\pa_{x_j}) \coloneqq \pa_{y_j}&;\quad J^\varepsilon (\pa_{y_j}) \coloneqq -\pa_{x_j}\quad \forall j\in\{2,\ldots, n, n+2,\ldots, 2n\},
 \end{align*}
 where $\pa_{x_j}$ and $\pa_{y_j}$ are the vector fields coming from the coordinates\linebreak $(z_1 = x_1 + iy_1,\ldots, z_{2n} = x_{2n} + iy_{2n})\in\mathbb{C}^{2n}$. Clearly, $J^\varepsilon$ coincides with $J = i$ for $\varepsilon = 0$. Moreover, one easily checks that $J^\varepsilon$ satisfies $J^\varepsilon\vert_{X\backslash U} = J\vert_{X\backslash U}$ and is $\Omega_R$-anticompatible for every $\varepsilon\in\mathbb{R}$.\\
 Next, we set $\Omega^\varepsilon_R\equiv \Omega_R$ and $\Omega^\varepsilon_I\coloneqq -\Omega^\varepsilon_R (J^\varepsilon\cdot,\cdot)$ for every $\varepsilon\in\mathbb{R}$. As in Example \autoref{ex:constructing_PHHS}, the exterior derivative of $\Omega^\varepsilon_I$ can be expressed as
 \begin{gather*}
  d\Omega^\varepsilon_I = \varepsilon^2 \cdot df\wedge\left( dy_{n+1}\wedge dx_1 - \left(\frac{1}{r^\varepsilon}\right)^2dx_{n+1}\wedge dy_1\right).
 \end{gather*}
 We see that for every $\varepsilon\neq 0$ there exists a point $x_0\in U$ satisfying $d\Omega^\varepsilon_I\vert_{x_0}\neq 0$. Thus, $(X,J^\varepsilon;\Omega^\varepsilon_R)$ becomes a proper PHSM for every $\varepsilon\neq 0$ due to Theorem \autoref{thm:rel_HSM_PHSM}.\\
 The only thing left to check is that $\mH^\varepsilon_R\equiv \mH_R$ is compatible with the PHSM $(X,J^\varepsilon;\Omega^\varepsilon_R)$, i.e., $d[\Omega^\varepsilon_R (J^\varepsilon (X^{\Omega^\varepsilon_R}_{\mH^\varepsilon_R}),\cdot)] = 0$ for every $\varepsilon$. In the case of $\mH\equiv c$, this is trivially true, because then the Hamiltonian vector field $X^{\Omega^\varepsilon_R}_{\mH^\varepsilon_R}$ vanishes. For $\mH = z_{2n}$, we first realize that the equation above is equivalent to the condition that $\mH^\varepsilon_R$ is the real part of some pseudo-holomorphic function $\mH^\varepsilon$. We already know that $\mH^\varepsilon\equiv \mH$ is pseudo-holomorphic with respect to $J = i$. Since $\mH^\varepsilon$ only depends on the last component $z_{2n}$ and both $J^\varepsilon$ and $J = i$ act in the same way on the last components of $X$ for every $\varepsilon$ (as $n>1$), $\mH^\varepsilon\equiv\mH$ is also pseudo-holomorphic with respect to $J^\varepsilon$ for every $\varepsilon$. This turns $(X,J^\varepsilon;\Omega^\varepsilon_R,\mH^\varepsilon_R)$ into the desired deformation of $(X,\Omega,\mH)$ concluding the proof.
\end{proof}

\begin{Rem}[The case $\mH\equiv c$]\label{rem:constant_Ham}
 Upon closer inspection, we note that the proof also works for $n = 1$ if $\mH\equiv c$. Indeed, we only need the condition $n>1$ to ensure that $\mH^\varepsilon_R$ is the real part of some pseudo-holomorphic function $\mH^\varepsilon$. However, the constant function is pseudo-holomorphic with respect to any almost complex structure.
\end{Rem}

In order to find local proper deformations of the standard HSM, we simply use the deformation from the proof above and forget the Hamiltonian $\mH^\varepsilon_R$. Since we are not constrained by the compatibility with the Hamiltonian $\mH^\varepsilon_R$ in this case, this deformation even works for $n=1$. Similarly, we obtain a local proper deformation of the complex manifold $\mathbb{C}^m$ if we take the almost complex structure $J^\varepsilon$ from the proof. Here, we need to enforce $m>1$ to make sure that we even have four real dimensions to rescale.\\
Combining Lemma \autoref{lem:HHS_in_standard_form} with Proposition \autoref{prop:deformation_of_standard_HHS} proves the following theorem:
\newpage

\begin{Thm}[Proper PHHSs are generic]\label{thm:generic}
 Let $X$ be a smooth manifold, then the following statements apply depending on the real dimension of $X$:
 \begin{enumerate}
  \item If $\text{\normalfont dim}_\mathbb{R}(X) = 2$: Every almost complex structure on $X$ is integrable and automatically a complex structure\footnote{It is easy to check that the Nijenhuis tensor always vanishes in two dimensions.}.
  \item If $\text{\normalfont dim}_\mathbb{R}(X) > 2$: Every complex manifold $(X,J)$ and HSM $(X,\Omega)$ admits a proper deformation. In particular, the non-integrable almost complex structures and proper PHSMs on $X$ are generic within the set of all almost complex structures and PHSMs on $X$, respectively.
  \item If $\text{\normalfont dim}_\mathbb{R}(X) > 4$: Every HHS $(X,\Omega, \mH)$ admits a proper deformation. In particular, the proper PHHSs on $X$ are generic within the set of all PHHSs on $X$.
 \end{enumerate}
\end{Thm}

Before we conclude this section, let us quickly comment on PHHSs $(X,J;\Omega_R,\mH_R)$ in dimension $\text{dim}_\mathbb{R}(X) = 4$. These systems are the only geometrical object studied in this subsection to which Theorem \autoref{thm:generic} does \underline{not} apply. The reason is that Proposition \autoref{prop:deformation_of_standard_HHS} fails for $\text{dim}_\mathbb{R}(X) = 4$, since $\mH_R = x_2$ cannot be the real part of a complex function which is pseudo-holomorphic with respect to deformations $J^\varepsilon$ as chosen in the proof of Proposition \autoref{prop:deformation_of_standard_HHS}. Nevertheless, Statement 3 of Theorem \autoref{thm:generic} might still be true for $\text{dim}_\mathbb{R}(X) = 4$. One possible way to prove this could be to modify Proposition \autoref{prop:deformation_of_standard_HHS}. Instead of choosing $\mH^\varepsilon_R$ to be independent of $\varepsilon$, we could allow for general deformations $\mH^\varepsilon_R$ of $\mH_R$. Finding such deformations $\mH^\varepsilon_R$ involves solving a second order PDE with boundary conditions. However, the existence of non-trivial solutions to the given problem might be forbidden by the Nijenhuis tensor. We elaborate on this thought: let $X$ be a smooth manifold with almost complex structure $J$, $N_J$ be the Nijenhuis tensor of $J$, and $f:X\to\mathbb{C}$ be a pseudo-holomorphic map, i.e., $df\circ J = i\cdot df$. A straightforward calculation reveals that $df\left(N_J (V,W)\right) = 0$ for all vector fields $V$ and $W$ on $X$. Thus, the image of $N_J\vert_x$ is contained within the kernel of $df\vert_x$ for any pseudo-holomorphic function $f$ and any point $x\in X$. This implies that there are at most $1/2(\text{dim}_\mathbb{R}(X)-r_x)$ pseudo-holomorphic functions on $X$ whose differentials at a given point $x\in X$ are $\mathbb{C}$-linearly independent, where $r_x$ is the rank\footnote{The Nijenhuis tensor satisfies the relation $N_J (J(V),W) = -JN_J (V,W)$, hence, its rank is even.} of $N_J\vert_x$. In four dimensions, the rank of $N_J$ alone does not exclude the existence of non-trivial pseudo-holomorphic functions: let $V_1$, $V_2$, $J(V_1)$, and $J(V_2)$ be a local frame of $X$. One easily sees that, because of the symmetries of $N_J$, i.e., $N_J (V,W) = -N_J(W,V)$ and $N_J(J(V), W) = -JN_J (V,W)$, $N_J (V_1, V_2)$ and $N_J (V_1, J(V_2)) = -JN_J (V_1,V_2)$ are the only two components of $N_J$ in the local frame $V_1$, $V_2$, $J(V_1)$, and $J(V_2)$ which are not redundant. Thus, the rank of the Nijenhuis tensor is at most $2$ in four dimensions. For instance, the deformation $J^\varepsilon$ from above for $n=1$,
 \begin{gather*}
  J^\varepsilon (\pa_{x_1}) \coloneqq r^\varepsilon\pa_{y_1},\ J^\varepsilon (\pa_{x_{2}}) \coloneqq \frac{1}{r^\varepsilon}\pa_{y_{2}};\quad J^\varepsilon (\pa_{y_1}) \coloneqq \frac{-1}{r^\varepsilon}\pa_{x_1},\ J^\varepsilon (\pa_{y_{2}}) \coloneqq -r^\varepsilon\pa_{x_{2}},
 \end{gather*}
 yields for $r^\varepsilon\in C^\infty(\mathbb{R}^4,\mathbb{R}_{+})$:
 \begin{gather*}
  N_{J^\varepsilon} (\pa_{x_1},\pa_{x_2}) = \frac{1}{r^\varepsilon}J^\varepsilon\left(N_{J^\varepsilon} (\pa_{x_1},\pa_{y_2})\right) = \sum\limits_{a\in\{x,y\}}\sum^2_{i,j = 1;\ i\neq j}\left(\pa_{a_i}\ln (r^\varepsilon)\right)\cdot \pa_{a_j}.
 \end{gather*}
 Hence, the (real) rank of $N_{J^\varepsilon}\vert_x$ is $2$ for $dr^\varepsilon\vert_x \neq 0$ and $0$ for $dr^\varepsilon\vert_x = 0$.\\
 Nevertheless, the rank of $N_J$ is a rather weak bound for the number of independent pseudo-holomorphic functions. The exact number is given by $1/2(\text{dim}_\mathbb{R}(X) - k)$, where $k$ is the rank of the IJ-bundle\footnote{Confer \cite{muskarov1986} for the definition of the IJ-bundle and a detailed investigation of the relation between the Nijenhuis tensor and pseudo-holomorphic functions.} on $X$ containing the image of $N_J$. Even though there are non-integrable almost complex structures $J$ in four dimensions whose IJ-bundle does not have full rank, e.g. $J^\varepsilon$ with $r^\varepsilon = e^{-x_1}$ as in Example \autoref{ex:constructing_PHHS}, it is not clear at all why this should also apply to the almost complex structures $J^\varepsilon$ as above for general functions $r^\varepsilon$.
