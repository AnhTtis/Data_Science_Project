\section{Proof of Morse Darboux Lemmata}
\label{app:morse_darboux_lem}

Our goal in this part is to prove Lemma \autoref{lem:morse_darboux_lem_I} and \autoref{lem:morse_darboux_lem_II}. We start with Lemma \autoref{lem:morse_darboux_lem_I}.

\begin{Lem*}[Morse Darboux lemma I]
 Let $(M^2,\omega)$ be a symplectic $2$-manifold, $L^1$ be a smooth $1$-manifold, $f\in C^\infty (M,L)$, and $p\in M$ be a non-degenerate critical point of $f$ with Morse index $\mu_f (p) = 0$. Further, let $T>0$ be a positive real number. Then, there exists a $C^1$-chart $\psi_L:U_L\to V_L\subset\R$ of $L$ near $f(p)$ which is smooth on $U_L\backslash\{f(p)\}$ such that all non-constant trajectories near $p$ of the RHS $(U_M, \omega\vert_{U_M}, H)$ with $U_M\coloneqq f^{-1}(U_L)$ and $H\coloneqq \psi_L\circ f\vert_{U_M}$ are $T$-periodic.
\end{Lem*}

\begin{proof}
 The proof consists of three steps:
 \begin{enumerate}
  \item First, we convince ourselves that the non-constant trajectories $\gamma$ near $p$ are indeed periodic.
  \item Afterwards, we compute the period $\hat T(r)$ of a trajectory $\gamma$ near $p$ with $f\circ \gamma = r^2$ to show that $\hat T(r)$ is defined for $r\in (-\varepsilon,\varepsilon)$ ($\varepsilon>0$), depends smoothly on $r$, and is bounded from below by a positive constant.
  \item Lastly, we use these properties of $\hat T(r)$ to define a $C^1$-diffeomorphism\linebreak $\psi_L:U_L\to V_L\subset\R$ such that the trajectories of the rescaled RHS $(U_M,\omega\vert_{U_M}, H)$ with $U_M$ and $H$ as above have fixed period $T>0$.
 \end{enumerate}
 \textbf{Step 1}\\\\
 Without loss of generality, we can assume, after choosing appropriate charts, that $L = \R$, $f(p) = 0\in\R$, and that the (usual) Morse index $\mu_f (p)$ of $f:M\to\R$ is $0$. Now, we apply the Morse lemma to find a chart $\hat\psi_M = (\hat x, \hat y):\hat U_M\to \hat V_M$ of $M$ near $p$ with $\hat \psi_M (p) = 0$ such that $f\vert_{\hat U_M} = {\hat x}^2 + {\hat y}^2$. In this chart, we have $\omega\vert_{\hat U_M} = \hat v\, d\hat x\wedge d\hat y$, where $\hat v\in C^\infty (\hat U_M,\R)$. Since $\omega$ is non-degenerate, we can assume $\hat v >0$ (after permuting $\hat x$ and $\hat y$ if necessary). Now consider the RHS $(M,\omega, f)$ and its trajectories $\gamma$ near $p$. $f$ is constant along $\gamma$, so for $r>0$ small enough the trajectories $\gamma$ near $p$ with $f\circ \gamma = r^2$ move along the circle
 \begin{gather*}
  f^{-1}(r^2) = \hat \psi^{-1}_M (\{(\hat x,\hat y)\in\R^2\mid \hat x^2 + \hat y^2 = r^2\}) \cong S^1
 \end{gather*}
 with velocity $\dot\gamma\neq 0$. Hence, the trajectories near $p$ are periodic.\\\\
 \textbf{Step 2}\\\\
 Denote the period of $\gamma$ with $f\circ \gamma = r^2$ and $r>0$ by $\hat T(r)$. We calculate $\hat T(r)$ by going into polar coordinates:
 \begin{gather*}
  (\hat x,\hat y) = (r\cos (\varphi), r\sin (\varphi)).
 \end{gather*}
 Define $v \coloneqq \hat v\circ \hat\psi^{-1}_M$, then the Hamiltonian vector field $X_f$ is given by:
 \begin{gather*}
  d\hat\psi_M (X_f) = \frac{2}{v(\hat x, \hat y)}
  \begin{pmatrix}
   -\hat y\\ \hat x
  \end{pmatrix} = \frac{2}{v(r\cos (\varphi), r\sin (\varphi))}
  \begin{pmatrix}
   -r\sin (\varphi)\\ r\cos (\varphi)
  \end{pmatrix}.
 \end{gather*}
 Now parameterize an integral curve $\gamma:\R\to M$ of $X_f$ by $r_\gamma, \varphi_\gamma:\R\to\R$ in the following way:
 \begin{gather*}
  \hat \psi_M\circ \gamma (t) =
  \begin{pmatrix}
   r_\gamma (t)\cos (\varphi_\gamma (t))\\
   r_\gamma (t)\sin (\varphi_\gamma (t))
  \end{pmatrix}.
 \end{gather*}
 The integral curve equation $\dot\gamma = X_f (\gamma)$ now yields:
 \begin{gather*}
  \dot r_\gamma = 0;\quad \dot \varphi_\gamma = \frac{2}{v(r_\gamma\cos (\varphi_\gamma), r_\gamma\sin (\varphi_\gamma))}.
 \end{gather*}
 Thus, $r_\gamma$ is constant and, since $f\circ\gamma = r^2$, given by $r$. This allows us to define $\Phi:\R\to \R$ by
 \begin{align*}
  \Phi (t)&\coloneqq \varphi_\gamma (t) - \varphi_\gamma (0) = \int\limits^t_0 \dot\varphi_\gamma (t^\prime) dt^\prime\\
  &= \int\limits^t_0 \frac{2}{v(r\cos (\varphi_\gamma (t^\prime)), r\sin (\varphi_\gamma(t^\prime)))} dt^\prime.
 \end{align*}
 $\Phi$ is an orientation preserving diffeomorphism, since $\dot\Phi = \frac{2}{v}>0$. Hence, $\Phi^{-1}$ exists and is given by:
 \begin{gather*}
  \Phi^{-1} (\alpha) = \frac{1}{2}\int\limits^{\varphi_0+\alpha}_{\varphi_0} v(r\cos (\varphi), r\sin (\varphi))\, d\varphi
 \end{gather*}
 with $\varphi_0\coloneqq \varphi_\gamma (0)$, as
 \begin{align*}
  \frac{d\Phi^{-1}}{d\alpha} (\alpha) &\stackrel{\phantom{\varphi_\gamma = \varphi_0 + \Phi}}{=} \frac{1}{\dot\Phi (\Phi^{-1}(\alpha))} = \frac{1}{2}v (r\cos (\varphi_\gamma (\Phi^{-1}(\alpha))), r\sin (\varphi_\gamma (\Phi^{-1}(\alpha))))\\
  &\stackrel{\varphi_\gamma = \varphi_0 + \Phi}{=} \frac{1}{2} v(r\cos (\varphi_0 + \alpha), r\sin (\varphi_0 + \alpha)).
 \end{align*}
 We can now use $\Phi$ and $\Phi^{-1}$ to compute $\hat T(r)$:
 \begin{align*}
  &\varphi_\gamma (t + \hat T(r)) = \varphi_\gamma (t) + 2\pi\quad\forall t\in\R\quad \Rightarrow \Phi (\hat T(r)) = 2\pi\\
  \Rightarrow &\hat T(r) = \Phi^{-1} (2\pi) = \frac{1}{2}\int\limits^{\varphi_0+2\pi}_{\varphi_0} v(r\cos (\varphi), r\sin (\varphi))\, d\varphi = \frac{1}{2}\int\limits^{2\pi}_{0} v(r\cos (\varphi), r\sin (\varphi))\, d\varphi
 \end{align*}
 As we can see, $\hat T(r)$ depends smoothly on $r>0$. In fact, this formula allows us to define $\hat T(r)$ smoothly for $r\leq 0$ as well. It turns out that $\hat T(r)$ is even:
 \begin{align*}
  \hat T(-r) &= \frac{1}{2}\int\limits^{2\pi}_{0} v(-r\cos (\varphi), -r\sin (\varphi))\, d\varphi = \frac{1}{2}\int\limits^{2\pi}_{0} v(r\cos (\varphi+\pi), r\sin (\varphi+\pi))\, d\varphi\\
  &= \frac{1}{2}\int\limits^{2\pi}_{0} v(r\cos (\varphi), r\sin (\varphi))\, d\varphi = \hat T(r).
 \end{align*}
 After shrinking $\hat U_M$ if necessary, we can assume that $\hat U_M$ has a compact neighborhood. Thus, $\hat v\in C^\infty (\hat U_M,\R)$ is bounded from below by a positive constant $v_{\min}>0$. Therefore, $v\coloneqq \hat v\circ\hat\psi^{-1}_M$ is also bounded from below by $v_{\min}$. This implies:
 \begin{gather*}
  \hat T(r) = \frac{1}{2}\int\limits^{2\pi}_{0} v(r\cos (\varphi), r\sin (\varphi)) \geq \pi v_{\min} > 0\quad\forall r.
 \end{gather*}
 \textbf{Step 3}\\\\
 Lastly, we use these properties of $\hat T(r)$ to define a $C^1$-diffeomorphism $\psi_L:U_L\to V_L\subset\R$ on a neighborhood $U_L\subset L$ of $0\in\R = L$ which rescales the periods of the trajectories $\gamma$ to a fixed period $T > 0$. Consider a trajectory $\gamma$ near $p$ with $f\circ \gamma = r^2$ again. We want to define $\psi_L (s)$ with $s = r^2$ in such a way that the rescaled trajectory $\Gamma (t)\coloneqq \gamma (\hat T(r)\cdot t/T)$ is a trajectory of the rescaled RHS $(U_M,\omega\vert_{U_M},H)$ ($U_M$ and $H$ as above). Hence, we want $\Gamma$ to be an integral curve of $X_H$:
 \begin{align*}
  \dot\Gamma (t) &= \frac{\hat T(r)}{T}\dot\gamma \left(\frac{\hat T(r)}{T}t\right) = \frac{\hat T(r)}{T} X_f (\Gamma (t))\\
  &\stackrel{!}{=} X_H (\Gamma (t)) = \frac{d\psi_L}{ds} (f\circ\Gamma (t)) X_f (\Gamma (t)) = \frac{d\psi_L}{ds} (r^2) X_f (\Gamma (t))
 \end{align*}
 Thus, we obtain the following condition for $\psi_L$:
 \begin{gather}
  \frac{d\psi_L}{ds} (r^2) = \frac{\hat T(r)}{T}.\label{eq:psi_L}
 \end{gather}
 To solve Equation \eqref{eq:psi_L}, we define the function $g:\R\to\R$ by $g(s)\coloneqq \sqrt{|s|}$. $g$ is continuous on $\R$ and smooth on $\R\backslash\{0\}$. Thus, $\hat T\circ g$ is also continuous on a neighborhood $U_L$ of $0$ and smooth on $U_L\backslash\{0\}$. Now we define $\psi_L$ as
 \begin{gather*}
  \psi_L (s)\coloneqq \frac{1}{T}\int\limits^s_0 \hat T\circ g (s^\prime) ds^\prime = \frac{1}{T} \int\limits^s_0 \hat T (\sqrt{|s^\prime|}) ds^\prime.
 \end{gather*}
 Since $\hat T\circ g$ is continuous, $\psi_L$ exists, is a $C^1$-function on $U_L$, and smooth on $U_L\backslash\{0\}$. Furthermore:
 \begin{gather*}
  \frac{d\psi_L}{ds} (s) = \frac{\hat T (\sqrt{|s|})}{T}\geq \frac{\pi v_{\min}}{T} >0.
 \end{gather*}
 Therefore, $\psi_L$ is also a $C^1$-diffeomorphism. The last equation together with the fact that $\hat T (r)$ is even shows Equation \eqref{eq:psi_L} concluding the proof:
 \begin{gather*}
  \frac{d\psi_L}{ds} (r^2) = \frac{\hat T (|r|)}{T} = \frac{\hat T(r)}{T}.
 \end{gather*}
\end{proof}

\begin{Rem}[No regularity issues in a real analytic setup]\label{rem:no_reg_issue}
 If all objects in Lemma \autoref{lem:morse_darboux_lem_I} are real analytic instead of smooth, then the regularity issues do not occur, i.e., the chart $\psi_L$ can be chosen to be real analytic. We can see this as follows: By similar arguments as before, the map $\hat T:\R\to\R$ assigning the period $\hat T(r)$ to each radius $r$ is given by
 \begin{gather*}
  \hat T(r) = \frac{1}{2}\int\limits^{2\pi}_{0} v(r\cos (\varphi), r\sin (\varphi))\, d\varphi
 \end{gather*}
 and, hence, real analytic, as $v$ is real analytic. Thus, $\hat T$ can be written as a power series near $r = 0$:
 \begin{gather*}
  \hat T (r) = \sum^\infty_{k=0} a_k r^k.
 \end{gather*}
 Now recall that $\hat T$ is even, therefore, only even powers occur in the power series of $\hat T$:
 \begin{gather*}
  \hat T (r) = \sum^\infty_{k=0} a_{2k} r^{2k}.
 \end{gather*}
 This allows us to define the real analytic function $\hat t$ by
 \begin{gather*}
  \hat t (s) \coloneqq \sum^\infty_{k=0} a_{2k} s^k.
 \end{gather*}
 Obviously, $\hat T$ and $\hat t$ satisfy the relation: $\hat t (r^2) = \hat T (r)$. Now we define the real-analytic chart $\psi_L$ by
 \begin{gather*}
  \psi_L (s)\coloneqq \frac{1}{T}\int\limits^s_0 \hat t (s^\prime) ds^\prime.
 \end{gather*}
 As in the proof of Lemma \autoref{lem:morse_darboux_lem_I}, all non-constant trajectories of the Hamiltonian $\psi_L\circ f$ are $T$-periodic, since $\psi_L$ fulfills the equation
 \begin{gather*}
  \frac{d\psi_L}{ds} (r^2) = \frac{\hat t (r^2)}{T} = \frac{\hat T (r)}{T}.
 \end{gather*}
\end{Rem}

\noindent Next, we prove Lemma \autoref{lem:morse_darboux_lem_II}.

\newpage

\begin{Lem*}[Morse Darboux lemma II]
 Let $(M^2,\omega)$ be a symplectic $2$-manifold and let\linebreak $H\in C^\infty (M,\R)$ be a smooth function on $M$ with non-degenerate critical point $p\in M$ of Morse index $\mu_H (p)\neq 1$. Further, let $T>0$ be a positive real number. Then, the following statements are equivalent:
 \begin{enumerate}
  \item There exists a topological chart $\psi_M = (x,y):U_M\to V_M\subset\R^2$ of $M$ near $p$ which is smooth on $U_M\backslash\{p\}$ such that ($\psi_M (p) = 0$):
  \begin{enumerate}[label = (\alph*)]
   \item $H\vert_{U_M} = H(p) \pm \frac{\pi}{T}(x^2 + y^2)$,
   \item $\omega\vert_{U_M} = dx\wedge dy$.
  \end{enumerate}
  \item There exists an open neighborhood $U_M\subset M$ of $p$ such that all non-constant trajectories of the RHS $(U_M, \omega\vert_{U_M}, H\vert_{U_M})$ are $T$-periodic.
  \item There exists a number $E_0 > 0$ such that $\int_{U(E)} \omega = T\cdot E$ for every $E\in [0, E_0]$,\linebreak where $U(E)$ is the connected component containing $p$ of the set\linebreak $\{q\in M\mid |H(q)-H(p)|\leq E\}$.
 \end{enumerate}
\end{Lem*}

\begin{proof}
 The idea of the proof is simple: The implications ``$1.\Rightarrow 2.$'' and ``$2.\Rightarrow 3.$'' follow from straightforward computations. To show the remaining implication ``$3.\Rightarrow 1.$'', we first choose a Morse chart $\hat \psi_M = (\hat x, \hat y):\hat U_M\to \hat V_M$ such that $H\vert_{\hat U_M} = H(p) + \varepsilon\frac{\pi}{T} (\hat x^2 + \hat y^2)$, where $\varepsilon\in\{-1,+1\}$. In general, $\hat \psi_M$ is not a Darboux chart for $\omega$. Still, the trajectories of the RHS $(\hat U_M, \omega\vert_{\hat U_M}, H\vert_{\hat U_M})$ are circles, in particular periodic orbits. Solely the angular velocity of these circles might not be constant. To rectify this, we go into polar coordinates $(r,\varphi)$ and apply an appropriately chosen diffeomorphism to $\varphi$. This operation results in a new chart $\psi_M$. Since we have not changed the radius $r$, $\psi_M$ is still a Morse chart. However, the change of the angle coordinate turns $\psi_M$ into a Darboux chart.\\\\
 $\boxed{1.\Rightarrow 2.}$\\\\
 In the chart $\psi_M$, we find for the Hamiltonian vector field $X_H$:
 \begin{gather*}
  d\psi_M(X_H) = \pm\frac{2\pi}{T}\begin{pmatrix}-y\\ x\end{pmatrix}.
 \end{gather*}
 Hence, the integral curves $\gamma$ of $X_H$ are given by:
 \begin{gather*}
  \gamma (t) = \begin{pmatrix}r_0 \cos (\varphi_0 \pm \frac{2\pi}{T}t)\\ r_0 \sin (\varphi_0 \pm \frac{2\pi}{T}t)\end{pmatrix}.
 \end{gather*}
 Thus, the trajectories $\gamma$ near $p$ are $T$-periodic.\\\\
 $\boxed{2.\Rightarrow 3.}$\\\\
 $p$ is a non-degenerate critical point of $H$ with Morse index $\mu_H (p) \neq 1$, hence, we can find a Morse chart $\hat \psi_M = (\hat x,\hat y):\hat U_M\to \hat V_M$ of $M$ near $p$ such that ($\hat\psi_M (p) = 0$)
 \begin{gather*}
  H\vert_{\hat U_M} = H(p) + \varepsilon\frac{\pi}{T}(\hat x^2 + \hat y^2)
 \end{gather*}
 for $\varepsilon\in\{-1,+1\}$. In this chart, we have $\omega\vert_{\hat U_M} = \hat v\cdot d\hat x\wedge d\hat y$ for $\hat v\in C^\infty (\hat U_M,\R)$. After permuting $\hat x$ and $\hat y$ if necessary, we can assume that $\hat v >0$. Let $\gamma$ be an integral curve of $X_H$. In polar coordinates $(\hat x, \hat y) = (r\cos (\varphi), r\sin (\varphi))$, we can parameterize $\gamma$ via $r_\gamma, \varphi_\gamma:\R\to\R$ as follows:
 \begin{gather*}
  \gamma (t) = \begin{pmatrix}
                r_\gamma (t) \cos (\varphi_\gamma (t))\\
                r_\gamma (t) \sin (\varphi_\gamma (t))
               \end{pmatrix}.
 \end{gather*}
 With $v\coloneqq \hat v\circ \hat\psi^{-1}_M$, the integral curve equation becomes:
 \begin{gather*}
  \dot r_\gamma = 0;\quad \dot \varphi_\gamma = \frac{2\pi\varepsilon}{Tv(r_\gamma\cos (\varphi_\gamma), r_\gamma\sin (\varphi_\gamma))}.
 \end{gather*}
 Thus, $r_\gamma\equiv r$ is constant. This allows us to define $\Phi:\R\to \R$ by
 \begin{align*}
  \Phi (t)&\coloneqq \varphi_\gamma (t) - \varphi_\gamma (0) = \int\limits^t_0 \dot\varphi_\gamma (t^\prime) dt^\prime\\
  &= \int\limits^t_0 \frac{2\pi\varepsilon}{T v(r\cos (\varphi_\gamma (t^\prime)), r\sin (\varphi_\gamma(t^\prime)))} dt^\prime.
 \end{align*}
 As in the proof of Lemma \autoref{lem:morse_darboux_lem_I}, $\Phi^{-1}$ exists and is given by:
 \begin{gather*}
  \Phi^{-1} (\alpha) = \frac{T}{2\pi\varepsilon}\int\limits^{\varphi_0+\alpha}_{\varphi_0} v(r\cos (\varphi), r\sin (\varphi))\, d\varphi
 \end{gather*}
 with $\varphi_0\coloneqq \varphi_\gamma (0)$. We now use the fact that, by assumption, $\gamma$ is $T$-periodic, so $\Phi^{-1}$ satisfies:
 \begin{gather*}
  \Phi^{-1} (2\pi) = \varepsilon T\quad\Rightarrow \int\limits^{2\pi}_0 v(r\cos (\varphi), r\sin (\varphi)) d\varphi = 2\pi.
 \end{gather*}
 Observe that the last equation holds for all $r>0$ small enough. 
 %By continuity and the fact that polar coordinates are ``invariant'' under the transformation $r\mapsto -r$ and $\varphi\mapsto \varphi + \pi$, it even holds for all $r\in (-\varepsilon_0,\varepsilon_0)$ with $\varepsilon_0>0$ small enough.
 It allows us to compute the symplectic area $\int_{U(E)}\omega$:
 \begin{align*}
  \int\limits_{U(E)} \omega &= \int\limits^{\sqrt{\frac{TE}{\pi}}}_0 \int\limits^{2\pi}_0 v(r\cos(\varphi), r\sin(\varphi))\, rdr\, d\varphi\\
  &= \int\limits^{\sqrt{\frac{TE}{\pi}}}_0 \left(\int\limits^{2\pi}_0 v(r\cos(\varphi), r\sin(\varphi))\, d\varphi\right) rdr\\
  &= \int\limits^{\sqrt{\frac{TE}{\pi}}}_0 2\pi r\, dr = T\cdot E.
 \end{align*}
 $\boxed{3.\Rightarrow 1.}$\\\\
 As in ``$2.\Rightarrow 3.$'', we can find a Morse chart $\hat \psi_M = (\hat x,\hat y):\hat U_M\to \hat V_M$ of $M$ near $p$ such that ($\hat\psi_M (p) = 0$)
 \begin{gather*}
  H\vert_{\hat U_M} = H(p) + \varepsilon\frac{\pi}{T}(\hat x^2 + \hat y^2)
 \end{gather*}
 for $\varepsilon\in\{-1,+1\}$ and $\omega\vert_{\hat U_M} = \hat v\cdot d\hat x\wedge d\hat y$ for $\hat v\in C^\infty (\hat U_M,\R)$ with $\hat v >0$. By taking the derivative of $\int_{U(E)}\omega = T\cdot E$ with respect to $E$, we deduce that
 \begin{gather*}
  \int\limits^{2\pi}_0 v(r\cos (\varphi), r\sin (\varphi))\, d\varphi = 2\pi
 \end{gather*}
 for $r>0$ small enough and $v\coloneqq \hat v\circ \hat\psi^{-1}_M$. The last equation implies that the map\linebreak $P:(0,\varepsilon_0)\times S^1\to (0,\varepsilon_0)\times S^1$ given by
 \begin{gather*}
  P(r, [\varphi])\coloneqq \left(r, \left[\int\limits^{\varphi}_0 v(r\cos (\varphi^\prime), r\sin (\varphi^\prime))\, d\varphi^\prime\right]\right)
 \end{gather*}
 is well-defined for $\varepsilon_0 >0$ small enough. $P$ is a smooth diffeomorphism, since the Jacobian of $P$ is $v>0$. Denote the map associated with polar coordinates by $S:\R_+ \times S^1\to \R^2\backslash\{0\}$, i.e., $S(r, [\varphi])\coloneqq (r\cos (\varphi), r\sin (\varphi))$, and consider the map $S\circ P\circ S^{-1}:\dot D_{\varepsilon_0}\to \dot D_{\varepsilon_0}$, where $\dot D_{\varepsilon_0}\coloneqq \{x\in\R^2\backslash\{0\}\mid ||x||\leq \varepsilon_0\}$. $S\circ P\circ S^{-1}$ is a smooth diffeomorphism, since both $S$ and $P$ are smooth diffeomorphisms. Furthermore, $S\circ P\circ S^{-1}$ maps circles of radius $r$ to circles of radius $r$, hence, we can extend $S\circ P\circ S^{-1}$ to a homeomorphism on $D_{\varepsilon_0}\coloneqq\{x\in\R^2\mid ||x||\leq \varepsilon_0\}$ by setting $S\circ P\circ S^{-1} (0)\coloneqq 0$.\\
 Now consider the map $(x,y)\equiv \psi_M\coloneqq S\circ P\circ S^{-1}\circ\hat \psi_M:U_M\to V_M$. $\psi_M$ is a topological chart of $M$ near $p$ and smooth on $U_M\backslash\{p\}$. Recall that $\hat \psi_M$ is a Morse chart for $H$, thus, the level sets of $H$ are circles in the chart $\hat \psi_M$. Since the charts $\psi_M$ and $\hat \psi_M$ only differ by postcomposition with $S\circ P\circ S^{-1}$ which preserves circles, $\psi_M$ is also a Morse chart for $H$:
 \begin{gather*}
  H\vert_{U_M} = H(p) + \varepsilon\frac{\pi}{T}(x^2 + y^2).
 \end{gather*}
 Furthermore, the fact that the Jacobian of $P$ is $v$ implies that
 \begin{gather*}
  \omega\vert_{U_M} = dx\wedge dy
 \end{gather*}
 concluding the proof.
\end{proof}
