\section{Introduction}
\label{sec:intro}

Hamiltonian systems (HSs) as the mathematical model for classical mechanics have been central to the advance of modern physics and mathematics alike. In physics, HSs provide a theoretical foundation for several approaches to quantization. In mathematics, the interest in HSs has led to the study of symplectic geometry and topology. The methods developed in this study, e.g. Floer theory, have proven to be of great success for various branches of mathematics and physics alike, for instance celestial mechanics and string theory.\\
Simply put, a HS consists of three data: a smooth manifold $M$, a symplectic $2$-form $\omega$ on $M$, and smooth function $H\in C^\infty (M,\mathbb{R})$. In physical terms, the symplectic manifold $(M,\omega)$ can be understood as the phase space of the system, while the function $H$, often called Hamilton function or, simply, Hamiltonian, assigns to every point in phase space its energy. These data allow us to define the Hamiltonian vector field $X_H$ on $M$ via the equation $\iota_{X_H}\omega = -dH$. The dynamics of the HS $(M,\omega, H)$ is governed by the vector field $X_H$. Precisely speaking, the physical trajectories of point-like particles described by the HS $(M,\omega, H)$ are exactly the integral curves of $X_H$. The connection between the integral curve equation of $X_H$ and the Hamilton equations known from classical mechanics is given by Darboux's theorem: every symplectic form $\omega$ can locally be written as
\begin{gather*}
 \omega = \sum^n_{i = 1} dp_i\wedge dq_i.
\end{gather*}
In such Darboux charts, the integrable curve equation of $X_H$ reduces to the Hamilton equations:
\begin{gather*}
 \dot q_i(t) = \frac{\partial H}{\partial p_i};\quad \dot p_i(t) = -\frac{\partial H}{\partial q_i}\quad\forall t\in I\ \forall i\in\{1,\ldots, m\}.
\end{gather*}
Since the integral curve equation is just a first-order differential equation, there always exists an open interval $I$ and a trajectory $\gamma:I\to M$ with $\gamma (t_0) = x$ for any initial value $x\in M$ and $t_0\in\mathbb{R}$. Furthermore, two trajectories $\gamma_1:I_1\to M$ and $\gamma_2:I_2\to M$ are identical iff they have the same domain ($I_1 = I_2\equiv I$) and attain the same value at some point $t_0\in I$. In particular, maximal trajectories for a given initial value are unique.\\
Physical trajectories also obey the action principle, i.e., they can be obtained as ``critical points'' of the action functional $\mathcal{A}_H:C^\infty(I,M)\to\mathbb{R}$ assigned to an exact\linebreak HS $(M,\omega = d\lambda, H)$:
\begin{gather*}
 \mathcal{A}_H[\gamma]\equiv \mathcal{A}^\lambda_H[\gamma]\coloneqq \int\limits_I \gamma^\ast\lambda - \int\limits_I H\circ\gamma (t)\, dt.
\end{gather*}
Here, ``critical point'' means that the first variation of $\mathcal{A}_H$ has to vanish at the trajectory $\gamma\in C^\infty(I,M)$ where we only allow for variations of $\gamma$ which keep the endpoints of $\gamma$ fixed. Sometimes, for instance in Floer theory, one wishes to view certain trajectories as actual critical points of some action functional. There are several ways to achieve this, e.g. by putting the endpoints of a trajectory on a Lagrangian or by only considering periodic trajectories.\\
Since HSs are given in terms of \underline{real-valued} manifolds $M$, forms $\omega$, and functions $H$, it is only natural to ask whether a similar construction with similar properties exists for \underline{complex-valued} manifolds $X$, forms $\Omega$, and functions $\mathcal{H}$. The answer to this question directly leads to the notion of \textbf{holomorphic Hamiltonian systems} (HHSs). Similarly to real Hamiltonian systems\footnote{To distinguish real- and complex-valued Hamiltonian systems, we call real-valued Hamiltonian systems real Hamiltonian systems from now on.} (RHSs), HHSs are also described by three data (cf. \autoref{subsec:def_HHS}): a complex manifold $X$ (implicitly defining an integrable complex structure $J$), a holomorphic symplectic $2$-form $\Omega$ on $X$, and a holomorphic function $\mH:X\to\mathbb{C}$.
%As for RHSs, one can also associate a (holomorphic) Hamiltonian vector field $X_\mH$, (holomorphic) trajectories, and action functionals with a HHS.
HHSs have been studied since the early 2000s, e.g. by Gerdjikov and Kyuldjiev et al. \cite{gerd2001}, \cite{gerd2002}, \cite{gerd2004} or by Arathoon and Fontaine \cite{arathoon2020}. In the given references, HHSs are usually viewed as complexifications of RHSs and mostly used as a tool to study RHSs which arise as real forms\footnote{The terms ``complexifications and real forms of Hamiltonian systems'' can be defined properly, but we do not give an explicit definition here. For us, it suffices to know that complexifications and real forms of Hamiltonian systems are defined similarly to complexifications and real forms of manifolds: a real manifold $M$ is the real form of a complex manifold $X$ and $X$ is a complexification of $M$ iff $M$ is the fixed point set of some anti-holomorphic involution on $X$.} of HHSs. In \cite{arathoon2020}, for instance, an integrable and compact RHS is constructed out of the HHS obtained from the complexification of the spherical pendulum.\\
In the present paper, we take a different approach. We study HHSs on their own and try to recreate the results known from RHSs for HHSs. To start with, we discuss the existence and uniqueness of holomorphic trajectories. Similarly to RHSs, holomorphic trajectories are defined as the holomorphic integral curves of the holomorphic Hamiltonian vector field $X_\mH$. We show in \autoref{subsec:holo_traj} that, locally, holomorphic trajectories always exist and are unique, given an initial value. Maximal holomorphic trajectories, however, are not unique anymore, even given an initial value, due to the effects of monodromy\footnote{Recently, the monodromy of the complexified Kepler problem has been studied by Sun and You (cf. \cite{shanzhong2020}).}. This behavior is in sharp contrast to RHSs. Nevertheless, the holomorphic trajectories still give rise to a foliation by energy hypersurfaces $\mH^{-1}(E)$ for regular values $E$ of $\mH$, as shown in \autoref{subsec:holo_traj}.\\
After this discussion, we prove in \autoref{subsec:holo_action_fun_and_prin} that the holomorphic trajectories satisfy an action principle\footnote{To the extent of the author's knowledge, an action principle for HHSs has not been formulated before in the literature.}, i.e., that they can be understood -- in some sense -- as critical points of certain action functionals. These action functionals are obtained by first decomposing a HHS $(X,\Omega,\mH)$ into four RHSs, one for each combination of real and imaginary part of $\Omega$ and $\mH$. To each RHS, we can assign the usual action functional of a RHS. Afterwards, we average each of these action functionals over the imaginary (or real) time axis and take an appropriate linear combination to obtain the action functional for the HHS $(X,\Omega,\mH)$. In fact, this method gives rise to a plethora of action functionals for the HHS $(X,\Omega,\mH)$ which simply differ by how one averages and takes the linear combination. We conclude \autoref{sec:HHS} with an application of HHSs. Precisely speaking, we establish a relation between Lefschetz fibrations and almost toric fibrations in \autoref{subsec:Lefschetz} using HHSs.\\
During the investigation of action functionals for HHSs, we observe that $J$, the complex structure of $X$, poses rather strong restrictions on the existence of certain holomorphic trajectories. In \autoref{subsec:holo_action_fun_and_prin}, we consider holomorphic trajectories whose domains are complex tori and interpret them as the complexification of periodic orbits. However, by the maximum principle, such holomorphic trajectories are always constant if the complex manifold in question is $X=\mathbb{C}^{2n}$ equipped with the standard complex structure $J = i$. The same argument does not hold anymore if we allow $J$ to be any almost complex structure. In his beautiful paper \cite{moser1995} from 1995, Moser shows that it is possible to pseudo-holomorphically embed complex tori in $\mathbb{R}^4$, where $\mathbb{R}^4$ is equipped with a suitable, not necessarily integrable almost complex structure $J$.\\
To avoid constraints imposed by the integrability of $J$, we introduce special Hamiltonian systems in \autoref{sec:PHHS} which are described by the same data as HHSs, but whose almost complex structure $J$ does not need to be integrable anymore. These Hamiltonian systems are called \textbf{pseudo-holomorphic Hamiltonian systems} (PHHSs) and exhibit, by design, the same properties as HHSs. In particular, pseudo-holomorphic trajectories of PHHSs induce foliations by regular energy hypersurfaces $\mH^{-1}(E)$ and obey an action principle (cf. \autoref{subsec:def_PHHS}).\\
At first glance, PHHSs may appear to be contrived and artificial, especially since, by definition, the imaginary part of $\Omega$ does not need to be closed anymore. However, the non-closedness of the imaginary part of $\Omega$ is an unavoidable consequence of the non-integrability of $J$, as we show in \autoref{subsec:rel_HHS_PHHS}. In fact, we prove that we recover a HHS from a PHHS if and only if $J$ is integrable or, equivalently, the imaginary part of $\Omega$ is closed. To further strengthen our claim that PHHSs are indeed a natural generalization of HHSs, we show that the space of proper\footnote{A proper PHHS is a PHHS which is not simultaneously a HHS.} PHHSs is open and dense in the space of PHHSs on a fixed manifold $X$ with $\text{dim}_\mathbb{R}(X)>4$ implying that proper PHHSs are generic. To prove that proper PHHSs are generic, we first give a method to construct proper PHHSs out of HHSs (cf. \autoref{subsec:constructing_PHHS}). The method itself is very interesting, since it is related to hyperkähler structures and allows us to equip the cotangent bundle of a complex manifold with the structure of a PHHS. Lastly, we use this construction to deform HHSs by proper PHHSs (cf. \autoref{subsec:deforming_HHS}).
