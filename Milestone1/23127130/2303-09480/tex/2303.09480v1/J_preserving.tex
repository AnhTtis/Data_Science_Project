\section{Proof of Statement 2 in Proposition 1} %\autoref{prop:holo_vec_field_equiv_J_pre_vec_field}}
\label{app:J_preserving}

The goal of this part is to explore the relation between holomorphic and $J$-preserving vector fields on a complex manifold $X$. In particular, we want to prove the Statement 2 in the following proposition:

\begin{Prop*}[Holomorphic vector fields $\Leftrightarrow$ $J$-preserving vector fields]
 Let $X$ be a complex manifold with complex structure $J\in\Gamma (\text{\normalfont End}(TX))$. Then:
 \begin{enumerate}
  \item The tangent bundles $TX$ and $T^{(1,0)}X$ are isomorphic as smooth complex vector bundles via
  \begin{gather*}
   f:TX\to T^{(1,0)}X,\quad v\mapsto \frac{1}{2}(v - i\cdot J(v)),
  \end{gather*}
  where the fiberwise complex vector space structure of $TX$ is given by the complex multiplication $(a+ib)\odot v\coloneqq av + bJ(v)$ for every $a,b\in\mathbb{R}$ and $v\in TX$.\\
  Now consider the space $\Gamma_J (TX)$ of smooth real $J$-preserving vector fields on $X$:
  \begin{gather*}
   \Gamma_J (TX)\coloneqq \{V_R\in\Gamma (TX)\mid L_{V_R}J = 0\},
  \end{gather*}
  where $L_{V_R}J$ is the Lie derivative of $J$ with respect to $V_R$. Then, $\Gamma_J (TX)$ together with the standard commutator $[\cdot, \cdot]$ of vector fields and the complex structure $J$ forms a complex Lie algebra. In fact, $(\Gamma_J (TX), [\cdot, \cdot])$ is isomorphic as complex Lie algebras to the space $(\Gamma (T^{(1,0)}X), [\cdot,\cdot])$ of holomorphic vector fields on $X$ via
  \begin{gather*}
   F:\Gamma_J (TX)\to \Gamma (T^{(1,0)}X),\quad V_R\mapsto \frac{1}{2}(V_R - i\cdot J(V_R)).
  \end{gather*}
  \item If $V_R\in\Gamma_J (TX)$ is a $J$-preserving vector field on $X$ with corresponding holomorphic vector field $V\coloneqq F(V_R)$ and $T$ is a holomorphic tensor field on $X$, then the Lie derivatives of $T$ with respect to $V_R$ and $V$ coincide, i.e.:
  \begin{gather*}
   L_{V_R} T = L_{V}T.
  \end{gather*}
 \end{enumerate}
\end{Prop*}

\begin{proof}
 Let $X$ and $J$ be as above and let $V_R\in\Gamma_J (TX)$ be a $J$-preserving vector field on $X$ with corresponding holomorphic vector field $V\coloneqq 1/2(V_R -iJ(V_R))$. Further, let $T$ be a holomorphic $(k,l)$-tensor field on $X$. The Lie derivative is complex linear, thus, we have by definition of $V$:
 \begin{gather*}
  L_V T = \frac{1}{2}\left(L_{V_R}T - i\cdot L_{J(V_R)}T\right).
 \end{gather*}
 Hence, it suffices to show:
 \begin{gather}\label{eq:lie_i}
  L_{J(V_R)}T = i\cdot L_{V_R}T.
 \end{gather}
 In a holomorphic chart $\phi = (z_1,\ldots, z_n):U\to V\subset\mathbb{C}^n$ of $X$, $T$ can be expressed as
 \begin{gather*}
  T\vert_U = \sum^n_{i_1\ldots i_k, j_1\ldots j_l = 1} T^{i_1\ldots i_k}_{j_1\ldots j_l}\cdot dz_{j_1}\otimes\ldots\otimes dz_{j_l}\otimes \partial_{z_{i_1}}\otimes\ldots\otimes \partial_{z_{i_k}},
 \end{gather*}
 where $T^{i_1\ldots i_k}_{j_1\ldots j_l}:U\to\mathbb{C}$ are holomorphic functions on $U$. Since the Lie derivative can be computed locally and satisfies the Leibinz rule, it suffices to show \autoref{eq:lie_i} for $T$ being $T^{i_1\ldots i_k}_{j_1\ldots j_l}$, $dz_{i}$, and $\partial_{z_j}$. As the Lie derivative also commutes with the exterior differential $d$ and $\partial_{z_j}$ is a (local) holomorphic vector field, it is sufficient to prove \autoref{eq:lie_i} for $T$ being a holomorphic function $h$ and holomorphic vector field $W$. For $T = h$, we find:
 \begin{gather*}
  L_{J(V_R)} h = dh\left( J (V_R)\right) = i\cdot dh (V_R) = i\cdot L_{V_R}h,
 \end{gather*}
 where we used that $h$ is holomorphic, i.e. $dh\circ J = i\cdot dh$. For $T = W$, we can use Statement 1 of Proposition \autoref{prop:holo_vec_field_equiv_J_pre_vec_field} to obtain:
 \begin{align*}
  L_{J(V_R)}W &= \left[J(V_R), W\right] = \frac{1}{2}\left([J(V_R), W_R] - i[J(V_R), J(W_R)]\right)\\
  &= \frac{1}{2}\left([V_R, J(W_R)] - i[V_R, J^2(W_R)]\right) = \frac{i}{2}\left([V_R, W_R] - i[V_R, J(W_R)]\right)\\
  &= i\left[V_R, W\right] = i\cdot L_{V_R}W,
 \end{align*}
 concluding the proof.
\end{proof}
