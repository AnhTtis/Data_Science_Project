	
	\documentclass[10pt,a4paper,showkeys, reprint]{revtex4-1}
\usepackage[utf8]{inputenc}
\usepackage{amsmath}
\usepackage{amsfonts}
\usepackage{amssymb}
\usepackage{graphicx}
\DeclareMathOperator{\sech}{sech}
\DeclareMathOperator{\csch}{csch}
\DeclareMathOperator{\arcsinh}{arcsinh}
\usepackage{mathrsfs}
\usepackage{latexsym}
%\usepackage{natbib}
\bibliographystyle{apsrev4-2}
%\setcitestyle{numbers,open={[},close={]}}
\usepackage{bm}
\usepackage{subfigure}
\usepackage{color}
\usepackage{hyperref}
\usepackage{subfigure}





\usepackage[left=2cm,right=2cm,top=2cm,bottom=2cm]{geometry}



\begin{document}

\author{Tanima Duary\footnote{Email: td14ip021@iiserkol.ac.in},  Narayan Banerjee\footnote{Email: narayan@iiserkol.ac.in} and Ananda Dasgupta\footnote{Email: adg@iiserkol.ac.in}}
\affiliation{Department of Physical Sciences, Indian Institute of Science Education and Research Kolkata,
Mohanpur 741 246 ,WB, India}
\title{ Phase transition of a cosmological model that mimics the $\Lambda$CDM model}
\date{\today}



\begin{abstract}
Using the Hayward-Kodama temperature for the apparent horizon, it is found that matter content in the Universe is not thermodynamically stable, and the entry to the late accelerated expansion is actually a 
second order phase transition. The cosmological model used for the purpose is one that imitates the $\Lambda$CDM model, the favoured model for the present Universe.
\end{abstract}

\keywords{thermodynamics, stability, phase transition}
\maketitle
\vspace{0.3cm}


\section{Introduction:}
The idea of a horizon and the thermodynmical properties of cosmological models were brought into being inspired by black hole thermodynamics.
This served the purpose as a diagnostics of the validity of cosmological models, mainly 
by checking the validity of the ``Generalized Second Law of Thermodynamics (GSL)\cite{bekenstein1974generalized} in the model. The GSL states that the total entropy is non-decreasing. For a comprehensive review, we refer to the monograph by 
Faraoni\cite{faraoni2015horizons}. This helps one to determine the favoured model amongst two or more. For example, we refer to the recent work which indicates that freezing models are better in this respect than the thawing models\cite{duary2019thawing}. 
The thermodynamic stability of model can be ascertained from the properties of the second order derivatives of the internal entropy of the system. For some examples, see references \cite{Ferreira:2015iaa} and \cite{mukherjee2021nonparametric}. The motivation of 
the present work, to begin with, is to look at the thermodynamic stability of a cosmological model that mimics a $\Lambda$CDM model for the late time evolution. The method is to look for the concavity of the entropy function for the matter 
content in the Universe. This is done through the properties of the hessian matrix, which involves the second order derivatives, of the entropy. An example of such treatment in cosmology can be found in the work by Bhandari, Haldar and Chakraborty\cite{bhandari2017interacting}. 
With a metric ansatz where the scale factor is a hyperbolic function ($a\simeq {\sinh}^{\frac{2}{3}}$), that imitates a $\Lambda$CDM behaviour, it is found that the system is not thermodynamically stable. \\

The calculation yields a surprising result. It is found that although the entropy $S$ is continuous, $C_V$, the thermal capacity at constant volume has a discontinuity at some value of the redshift $z$, where the evolution transits from 
the decelerated to the accelerated state of expansion. Clearly this transition is a phase transition, and as the discontinuity is in $C_V$. It is also found that the deceleration parameter $q$ plays the role of the order parameter, and the 
order of the discontinuity is simply unity. \\

Unlike the case of a stationary black hole, the horizon in cosmology is evolving, and is defined as an apparent horizon. This motivates one to replace the Hawking temperature by Hayward-Kodama temperature\cite{hayward1998unified,hayward2009local} 
as the temperature of the horizon. In fact, this is a crucial difference. It has been checked that if Hawking temperature is used, this phase transition goes missing! This may be the reason behind this important connection between the 
signature flip in $q$ and a second order phase transition. It deserves mention that some recent research on evolving black holes embedded in a de-Sitter spacetime, a second order phase transition is 
indicated\cite{Zhang:2022aqg, zhao2018, ali2022thermodynamics, Zhang:2020odg, Ma:2018hni}. \\

The paper is organized as follows. Section 2 introduces the general thermodynamic stability conditions. In section 3, the cosmological model  used in the work is discussed.  Section 4 deals with the stability analysis and the phase transition 
in the model. The 5th and final section includes some concluding remarks.
 

\section{General stability condition:}
We consider a spatially flat, homogeneous and isotropic universe given by the metric,
\begin{equation}
ds^2=-dt^2+a^2(t)[dr^2+r^2 d\Omega^2],
\end{equation}
where $a=a(t)$ is the scale factor. Einstein field equations are, 
\begin{align} \label{fe}
3H^2 &=  \rho , \\
2\dot{H}&= - (\rho+p),
\end{align}
where $\rho$ and $p$ are the total energy density and pressure of the matter content, $H=\frac{\dot a}{a}$ is the Hubble parameter and an overhead dot indicates a derivative with respect to the cosmic time $t$. 
Units are chosen where $c=1$ and $8\pi G=1$. Radius of apparent horizon, $\tilde {r}_h$, defined as $g^{\mu\nu} \tilde{r}_{h_{,\mu}} \tilde{r}_{h_{,\nu}}=0$ , is ${\tilde {r}_h}=\frac{1}{H}$ for a spatially flat ($k=0$) 
universe\cite{faraoni2015horizons}.

We consider that the fluid inside the horizon is in thermodynamical equilibrium with the horizon. As no radiation component in the matter content will be considered, this is fairly a legitimate assumption\cite{mimosodiego2016}. 
The horizon is evolving, as opposed to that in the case of a stationary black hole. So the equilibrium temperature is assumed to be given by Hayward-Kodama temperature\cite{hayward1998unified,hayward2009local},
\begin{equation}\label{temp}
T= \frac{2H^2+\dot{H}}{4\pi H}.
\end{equation}

The entropy of the horizon is 
\begin{equation}
\label{hor-entr}
S_h=2\pi A,
\end{equation}
where $A=4\pi \tilde{r}_h^2$ is the area of apparent horizon\cite{faraoni2015horizons}.

Differentiating $S_h$ with respect to time $t$, one obtains,
\begin{equation}
\dot{S_h}= -16\pi^2\frac{\dot{H}}{H^3}.
\end{equation}

For the fluid inside the horizon, Gibbs' law looks like,  
\begin{equation}\label{Gibbs}
TdS_{in}=dU+pdV,
\end{equation}
where $S_{in}$, $U$ and $V$ denote the entropy, the internal energy and the volume of the fluid inside the horizon respectively. $V$ is bounded by the apparent horizon, 
\begin{align}\label{vol}
V &=\frac{4}{3}\pi \tilde{r}_h^3 \nonumber \\ 
&=\frac{4}{3}\pi \frac{1}{H^3}
\end{align}

Rate of change of entropy of fluid inside the horizon is,
\begin{align}\label{ent}
\dot{S}_{in}&= \frac{1}{T_h}\left[(\rho+p)\dot{V}+\dot{\rho}V\right]\nonumber\\
&= \frac{1}{T_h}(\rho+p)(\dot{V}-3HV).
\end{align}
Now inserting $T_h$ from equation \eqref{temp} and $V$ from equation \eqref{vol} in \eqref{ent}, one obatins the expression of $\dot{S}_{in}$ as,
\begin{align}
\dot{S}_{in}  = 16\pi^2\frac{\dot{H}}{H^3}\left(1+\frac{\dot{H}}{2H^2+\dot{H}} \right).
\end{align}


Therefore rate of change of the total entropy is,
\begin{align}
\dot{S}&=\dot{S}_h+\dot{S}_{in} \nonumber\\
&= 16\pi^2\frac{\dot{H}^2}{H^3}\left(\frac{1}{2H^2+\dot{H}}\right).
\end{align}
 


\section{A model that mimics $\Lambda$CDM}


We assume a simple ansatz for the scale factor as 
\begin{equation}
\label{ansatz-a}
 \frac{a}{a_0}\sim\frac{\sinh^{2/3}(t/t_0)}{\sinh^{2/3}(1)},
\end{equation}

which gives an accelerated expansion for a late time whereas as a decelerated expansion in the early matter dominated era. Here we have consiedered $a=a_0$ at $t=t_0$ and taken $t_0=1$. In fact this ansatz mimics the $\Lambda$CDM behaviour for the late time, the favoured model for the present 
Universe\cite{PADMANABHAN2003235}. The equation \eqref{ansatz-a} can be used to write $t/t_0 = \arcsinh\left((\frac{1}{1+z})^{3/2}\sinh(1)\right)$, where $z$ is the redshift, defined as $1+z = \frac{a_0}{a}$, where $a_0$ is the present value of 
the scale factor.
One can write Hubble parameter in terms of $z$ as, 
\begin{align}
H &=\frac{2}{3}\coth (t/t_0)\nonumber\\
 &=\frac{2\csch(1)}{3}\frac{\sqrt{1+\frac{\sinh^2(1)}{(1+z)^3}}}{\left(\frac{1}{1+z}\right)^{3/2}}.
\end{align}
 Deceleration parameter, defined as $q=-\left[1+\frac{\dot{H}}{H^2}\right]$, looks like,
  \begin{equation}\label{q}
  q(z)= -1+\frac{3}{2\left(1+\frac{\sinh^2(1)}{(1+z)^3}\right)},
\end{equation}    
in terms of $z$.

\begin{figure}
\boxed{\includegraphics[scale=0.6]{q.pdf}}
\caption{Plot of $q$ against $z$}
\end{figure}
Figure (1) shows the behaviour of $q$ against $z$. At $z=-1+2^{1/3}\sinh^{2/3}(1)\simeq 0.403$, expansion of the Universe transits from the deceleration to the acceleration phase. \\

With this model, the thermodynamic behaviour will be investigated.


\section{Thermodynamic Stability and Phase transition:}
  For thermodynamic stability, entropy of the fluid inside the horizon has to be maximized. In terms of the Hessian matrix of entropy, this can be realized as follows\cite{callen1998thermodynamics, kubo1968thermodynamics, carter2000classical,Muller1985}, all 
  the $k^{\text{th}}$ order principle minors of the matrix are $\leq$ 0 if $k$ is odd and $\geq$ 0 if $k$ is even. Hessian matrix $W$ of $S_{in} $ is
\begin{equation}
W=
\begin{bmatrix}
{S_{in}}_{UU} & {S_{in}}_{UV}\\
{S_{in}}_{VU} & {S_{in}}_{VV}
\end{bmatrix},
\end{equation}
where a suffix denotes partial derivative with respect to the particular variable.
Therefore the thermodynamic stability requires that the conditions
\begin{align} 
(i)\hspace{3mm}{S_{in}}_{UU}\leq 0, \label{condi1}\\
(ii)\hspace{3mm}{S_{in}}_{UU}{S_{in}}_{VV}-{S_{in}}_{UV}^2 \geq 0, \label{condi2}
\end{align}

are satisfied together. \\

Now , 

\begin{equation}\label{fc}
{S_{in}}_{UU} = -\frac{1}{T^2C_V},
\end{equation}
and 
\begin{equation}\label{sc}
 {S_{in}}_{UU}{S_{in}}_{VV}-{S_{in}}_{UV}^2=\frac{1}{C_VT^3V\beta_T} = \alpha.
\end{equation}

The second expression is denoted as $\alpha$ for the sake of brevity.  Here $T$ is the temperature, $C_V$ is heat capacity at constant volume 
and $\beta_T$ is the isothermal compressibility. \\

Heat capacity at constant volume ($C_V$) and that at constant pressure ($C_P$)  of the fluid are defined respectively as,
\begin{equation}\label{cv}
C_V= T\left(\frac{\partial S_{in}}{\partial T}\right)_V,
\end{equation}
and
\begin{equation}\label{cp}
C_P= T\left(\frac{\partial S_{in}}{\partial T}\right)_P.
\end{equation}
Isothermal compressibility is defined as, 
\begin{equation}\label{bet}
\beta_T =-\frac{1}{V}\left(\frac{\partial V }{\partial P}\right)_T.
\end{equation}
Using equation \eqref{Gibbs}, one can calculate the heat capacities and isothermal compressibility for the matter inside the event horizon.
  
 \begin{align}\label{cv-expr}
 C_V &= V\left(\frac{\partial \rho}{\partial T}\right)_V \nonumber \\
 &= 32\pi^2\frac{\dot{H}}{2H^2\dot{H}+H\ddot{H}-\dot{H}^2}\nonumber\\
&= \frac{144\pi^2}{-2+(1+z)^3\csch^2(1)},
 \end{align}
 
 \begin{align}
 C_P &=  V\left(\frac{\partial \rho}{\partial T}\right)_P+(\rho+P)\left(\frac{\partial V}{\partial T}\right)_P \nonumber \\
 &= 32\pi^2\frac{H^2\dot{H}+\dot{H^2}}{H^2(2H^2\dot{H}+H\ddot{H}-\dot{H}^2)} \nonumber\\
 &= -72\pi^2 \frac{\sinh^2(1)}{(1+z)^3+\sinh^2(1)},
 \end{align}
 
\begin{align}\label{beta-expr}
 \beta_T &= \frac{3\dot{H}(2H^2+\dot{H})}{2(H^2+\dot{H})(2H^2\dot{H}+H\ddot{H}-\dot{H}^2)} \nonumber\\
 &=-\frac{27}{4}\frac{4+(1+z)^3\csch^2(1)}{\left(-2+(1+z)^3\csch^2(1)\right)}.
 \end{align}
  
 
 \begin{figure}
 \centering
\boxed{\includegraphics[scale=0.6]{CV.pdf}}
\caption{Plot of $C_V$ against $z$}\label{cvfig}
\end{figure}.


\begin{figure}
\centering
\boxed{\includegraphics[scale=0.6]{CP.pdf}}
\centering\caption{Plot of $C_P$ against $z$}\label{cpfig}
\end{figure}


Figures(\ref{cvfig}), (\ref{cpfig}) describes the behaviour of $C_V$ and $C_P$  respectively against the redshift $Z$ for low $z$ ($0\leq z \leq 1$). Using the expressions for $C_V$ and $\beta_T$ from equations  
\eqref{cv-expr} and \eqref{beta-expr} in equations \eqref{fc} and \eqref{sc}, one can plot $S_{inUU}$ and $\alpha$. These are shown in figures \eqref{cond1} and \eqref{cond2} respectively. It is clearly seen that the two 
conditions \eqref{condi1} and \eqref{condi2} are never satisfied together for the low redshift range ($0\leq z \leq 1$). Thus the model is not thermodynamically stable in the said redshift range. \\

The crucial observation that one can make from the figure \eqref{cvfig} is that $C_V$ has a discontinuity, in fact a divergence, at $z=-1+2^{1/3}\sinh^{2/3}(1)\simeq 0.403$, the value of $z$ where the Universe flips from the 
decelerated to the accelerated phase of expansion! So, the transition from the decelareted to the accelerated state of expansion is actually a thermodynamic phase transition. It has been checked that the entropy $S$ does not have any such 
discontinuity in the mentioned redshift range, the discontinuity is rather in $C_V$. Thus the phase transition is definitely a second order phase transition. \\

It deserves mention that $C_V$ is negative for the present Universe, $z>-1+2^{1/3}\sinh^{2/3}(1)\simeq 0.403$. However, a negative heat capacity is not at all a surprise in gravitational systems (for a review, see \cite{padmanabhan1990statistical}). In fact,
Luongo and Quevedo\cite{luongo2014cosmographic} arrived a strong result that for a currently accelerating Universe, $C_V$ is required to be negative. \\

Using equation \eqref{q} in \eqref{cv-expr}, one can obtain $C_V$ in terms of $q$ as
\begin{equation}\label{cvq}
 C_V=24\pi^2\frac{1-2q}{q}.
\end{equation}

So it is clearly seen that the discontinuity in $C_V$ results from $q$ appearing in the denominator with an exponent $+1$. Thus $q$ serves as the order parameter and the discontinuity is of order unity. 



 
 \begin{figure}
\centering
 \boxed{\includegraphics[scale=0.6]{con1.pdf}}
\caption{Plot of ${S_{in}}_{UU}$  against $z$}\label{cond1}
 \end{figure}
 
 \begin{figure}
 \centering
 \boxed{\includegraphics[scale=0.6]{con2.pdf}}
 \caption{Plot of $\alpha$ against $z$}\label{cond2}
 \end{figure}

\section{Conclusion:}
The thermodynamical stability analysis is done for a model which mimics the $\Lambda$CDM model for the present Universe. As the horizon is evolving, the Hayward-Kodama temperature is considered as the horizon temperature. It is found that the 
cosmic matter inside the horizon is not thermodynamically stable. \\

The far reaching result obtained is that the matter content undergoes a phase transition as the Universe flips from the decelerated to the accelerated state of expansion. The phase transition is manifestly a second order one, as the 
discontinuity is in $C_V$. The deceleration parameter $q$ plays the role of the order parameter. One disclaimer is that it has not been the attempt to fit in the observational value of $z$ at $q=0$. The investigation was really that 
of the qualitative thermodynamic nature of the signature flip in $q$. The reason for earlier investigations being unable to find the second order phase transition at the commencement of the accelerated phase of expansion of the Universe is perhaps because of the use of the Hawking temperature of the horizon and thus ignoring the fact that the apparent horizon is evolving.



\begin{acknowledgements}
TD wants to thank Council of Scientific \& Industrial Research, India (CSIR) for financial support. 
\end{acknowledgements}

\bibliography{biblio}


\end{document}