\documentclass[runningheads]{llncs}

\usepackage{graphicx}
\usepackage{tikz}
\usepackage{float}
\usepackage{siunitx}
\usepackage[export]{adjustbox}


\begin{document}
%
\title{Supplementary Material}
\author{}
\institute{}
\maketitle
%
%
\section{Additional Results on the SkullBreak Dataset}
\begin{figure}[h]
\centering
\resizebox{\textwidth}{!}{
	\begin{tikzpicture}
	    \node[] at (0, 0)       {\includegraphics[height=0.16\textwidth]{SkullBreak_Supp/supp_fronto_27_r.png}};
    	\node[] at (0, 1.25)    {\scriptsize Fronto-Orbital};
    	\node[] at (2, 0)       {\includegraphics[height=0.16\textwidth]{SkullBreak_Supp/supp_bil_45_r.png}};
    	\node[] at (2, 1.25)    {\scriptsize Bilateral};
    	\node[] at (4, 0)       {\includegraphics[height=0.16\textwidth]{SkullBreak_Supp/supp_parie_49_r.png}};
    	\node[] at (4, 1.23)    {\scriptsize Parieto-Temporal};
    	\node[] at (6, 0)       {\includegraphics[height=0.16\textwidth]{SkullBreak_Supp/supp_rand1_57_r.png}};
    	\node[] at (6, 1.25)    {\scriptsize Random 1};
    	\node[] at (8, 0)       {\includegraphics[height=0.16\textwidth]{SkullBreak_Supp/supp_rand2_30_r.png}};
    	\node[] at (8, 1.25)    {\scriptsize Random 2};
    	\node[] at (0, -2)      {\includegraphics[height=0.16\textwidth]{SkullBreak_Supp/supp_fronto_9_r.png}};
    	\node[] at (2, -2)      {\includegraphics[height=0.16\textwidth]{SkullBreak_Supp/supp_bil_9_r.png}};
    	\node[] at (4, -2)      {\includegraphics[height=0.16\textwidth]{SkullBreak_Supp/supp_parie_9_r.png}};
    	\node[] at (6, -2)      {\includegraphics[height=0.16\textwidth]{SkullBreak_Supp/supp_rand1_9_r.png}};
    	\node[] at (8, -2)      {\includegraphics[height=0.16\textwidth]{SkullBreak_Supp/supp_rand2_9_r.png}};
	\end{tikzpicture}}
\caption{Additional results of our method for the five different defect classes of the SkullBreak dataset.}
\label{SkullBreak}
\end{figure}
%
%
\section{Additional Results on the SkullFix Dataset}
\begin{figure}[h]
\centering
\resizebox{\textwidth}{!}{
	\begin{tikzpicture}
	    \node[] at (0, 0)   {\includegraphics[height=0.16\textwidth]{SkullFix_Supp/supp_sf3_r.png}};
    	\node[] at (2, 0)   {\includegraphics[height=0.16\textwidth]{SkullFix_Supp/supp_sf10_r.png}};
    	\node[] at (4, 0)   {\includegraphics[height=0.16\textwidth]{SkullFix_Supp/supp_sf11_r.png}};
    	\node[] at (6, 0)   {\includegraphics[height=0.16\textwidth]{SkullFix_Supp/supp_sf21_r.png}};
    	\node[] at (8, 0)   {\includegraphics[height=0.16\textwidth]{SkullFix_Supp/supp_sf30_r.png}};
    	\node[] at (0, -2)  {\includegraphics[height=0.16\textwidth]{SkullFix_Supp/supp_sf52_r.png}};
    	\node[] at (2, -2)  {\includegraphics[height=0.16\textwidth]{SkullFix_Supp/supp_sf53_r.png}};
    	\node[] at (4, -2)  {\includegraphics[height=0.16\textwidth]{SkullFix_Supp/supp_sf67_r.png}};
    	\node[] at (6, -2)  {\includegraphics[height=0.16\textwidth]{SkullFix_Supp/supp_sf73_r.png}};
    	\node[] at (8, -2)  {\includegraphics[height=0.16\textwidth]{SkullFix_Supp/supp_sf84_r.png}};
	\end{tikzpicture}}
\caption{Additional results of our method for the SkullFix dataset.}
\label{SkullFix}
\end{figure}
%
%
\section{Additional Results Ensembling Method}
\begin{figure}[h]
\centering
\resizebox{\textwidth}{!}{
	\begin{tikzpicture}
	    \node[] at (0, 0)  {\adjincludegraphics[width=0.16\textwidth, trim={.15\width, .1\width, .15\width, .4\width}, clip]{Ensembling_Supp/0.png}};
    	\node[] at (0, 1)  {\scriptsize Implant 1};
    	\node[] at (2, 0)  {\adjincludegraphics[width=0.16\textwidth, trim={.15\width, .1\width, .15\width, .4\width}, clip]{Ensembling_Supp/1.png}};
    	\node[] at (2, 1)  {\scriptsize Implant 2};
    	\node[] at (3.5, 0){$...$};
    	\node[] at (5, 0)  {\adjincludegraphics[width=0.16\textwidth, trim={.15\width, .1\width, .15\width, .4\width}, clip]{Ensembling_Supp/4.png}};
    	\node[] at (5, 1)  {\scriptsize Implant 5};
    	\node[] at (7, 0)  {\adjincludegraphics[width=0.16\textwidth, trim={.15\width, .1\width, .15\width, .4\width}, clip]{Ensembling_Supp/mean.png}};
    	\node[] at (7, 1)  {\scriptsize Mean};
    	\node[] at (9, 0)  {\adjincludegraphics[width=0.16\textwidth, trim={.15\width, .1\width, .15\width, .4\width}, clip]{Ensembling_Supp/var.png}};
    	\node[] at (9, 1)  {\scriptsize Variance Map};
	\end{tikzpicture}}
\caption{Different implants, mean implant and variance map for a single skull defect.}
\label{Ensembling}
\end{figure}
%
%
\section{Network Architecture of the Point Cloud Diffusion Model}
\begin{table}[h]
\caption{Architecture of the proposed point cloud diffusion model. The input point cloud is subsequently passed through SA1-4, FP1-4 and a MLP. Time embedding is concatenated to the point features in front of every SA \& FP block.}
\begin{center}
\resizebox{.45\textwidth}{!}{
    \begin{tabular}{|l|c|c|c|c|}
    \hline
        \multicolumn{5}{|c|}{Input size: $30720\times 3$}\\\hline
        \multicolumn{5}{|c|}{Input time embedding size: $64$}\\\hline\hline
        \multicolumn{5}{|c|}{\textbf{Time embedding}}\\\hline
        \multicolumn{5}{|c|}{Sinusoidal embedding dimension: 64}\\
        \multicolumn{5}{|c|}{MLP$(64,64)$}\\
        \multicolumn{5}{|c|}{LeakyReLU$(0.1)$}\\
        \multicolumn{5}{|c|}{MLP$(64,64)$}\\\hline\hline
        \multicolumn{5}{|c|}{\textbf{Set Abstraction (SA) Layers}}\\\hline
                                & SA1     & SA2     & SA3   & SA4\\\hline
        Number of PVConv blocks & $2$     & $3$     & $3$   & $0$ \\
        Input channels          & $3$     & $32$    & $64$  & - \\
        Output channels         & $32$    & $64$    & $128$ & - \\
        Voxelization resolution & $32$    & $16$    & $8$   & - \\
        Use attention           & False   & True    & False & - \\\hline
        Number of centers       & $10240$ & $2560$  & $640$ & $160$ \\
        Radius                  & $0.1$   & $0.2$   & $0.4$ & $0.8$ \\
        Number of neighbors     & $128$   & $128$   & $128$ & $128$ \\\hline\hline
        \multicolumn{5}{|c|}{\textbf{Feature Propagation (FP) Layers}}\\\hline
                                & FP1     & FP2     & FP3   & FP4 \\\hline
        Number of PVConv blocks & $2$     & $3$     & $3$   & $0$ \\
        Input channels          & $128$   & $256$   & $256$ & $128$ \\
        Output channels & $256$ & $256$   & $128$   & $64$ \\
        Voxelization resolution & $8$     & $8$     & $16$  & $32$ \\
        Use attention           & False   & True    & False & False \\
        \hline\hline
        \multicolumn{5}{|c|}{MLP$(64,3)$}\\\hline
        \multicolumn{5}{|c|}{Output size: $30720\times 3$}\\\hline
    \end{tabular}}
    \label{tab:}
    \end{center}
\end{table}
\end{document}