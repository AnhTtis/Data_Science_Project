\documentclass[10pt,conference,hidelinks]{IEEEtran}

\usepackage{url}
\usepackage{subfig}
\usepackage{graphicx}
\usepackage{amsmath,amssymb}
\usepackage{hyperref}
%\usepackage{subcaption}
\usepackage{listings}
\usepackage{tabularx}
\usepackage{xcolor}
\usepackage{comment}
\usepackage{listings,newtxtt}
\lstset{basicstyle=\ttfamily, keywordstyle=\bfseries}

\usepackage{tikz}
\usepackage{xcolor}
\newcommand*\circled[1]{\tikz[baseline=(char.base)]{
            \node[shape=circle,fill,inner sep=1pt] (char) {\textcolor{white}{#1}};}}


\usepackage{pbox}
\usepackage{comment}
\usepackage{csvsimple}
\usepackage{fancyhdr}
\usepackage{mathtools}
\usepackage{caption}
\usepackage{amsmath}
\usepackage{makecell}
\usepackage{tikz}
\def\checkmark{\tikz\fill[scale=0.4](0,.35) -- (.25,0) -- (1,.7) -- (.25,.15) -- cycle;}
%\usepackage[top=1in, bottom=1in, left=0.75in, right=0.75in]{geometry}
\usepackage{siunitx}
\sisetup{
    table-figures-decimal = 3,
    table-number-alignment=center-decimal-marker
}

\newcommand\Ayaz[1]{\textcolor{black}{#1}}
\newcommand\note[1]{\textcolor{blue}{#1}}

\begin{document}

% Use these macros to cite our tools, etc.
% This ensures consistency.
\newcommand{\gem}{gem{5}}

\title{A Cycle-level Unified DRAM Cache Controller Model for 3DXPoint Memory Systems in \gem{}}

% I don't like the title, we should find a better one
%\text{NOTE: ISPASS 2022 Submission, Currently Under Review  -- \\
%This Is a Confidential Draft -- Do NOT Distribute.}
\author{
%\normalsize{NOTE: ISPASS 2022 Submission, Currently Under Review  -- This Is a Confidential Draft -- Do NOT Distribute.}\\
\\\IEEEauthorblockN{Maryam Babaie, Ayaz Akram, Jason Lowe-Power}
\IEEEauthorblockA{Department of Computer Science,
University of California, Davis\\
Email: \{mbabaie, yazakram, jlowepower\}@ucdavis.edu}}

\maketitle

\begin{abstract}

    To accommodate the growing memory footprints of today's applications, CPU vendors have employed large DRAM caches,
    backed by large non-volatile memories like Intel Optane (e.g., Intel's Cascade Lake).
    The existing computer architecture simulators do not provide support to
    model and evaluate systems which use DRAM devices as a cache to the non-volatile main memory.
    In this work, we present a cycle-level DRAM cache model which is integrated with \gem{}.
    % a general DRAM cache protocol for the aforementioned systems.
    % We extend gem5 (widely used full-system simulator) with a unified DRAM cache and main memory controller
    % (\textit{UDCC}) to implement the DRAM cache protocol. \textit{UDCC} controls two different memory devices at the same time,
    % in which one of them caches data for the other one and data is kept consistent between the two.
    This model leverages the flexibility of \gem{}'s memory devices models and full system support to enable exploration of many different DRAM cache designs.
    We demonstrate the usefulness of this new tool by exploring the design space of a DRAM cache controller through several case studies including the impact of
    scheduling policies, required buffering, combining different memory technologies (e.g., HBM, DDR3/4/5, 3DXPoint, High latency)
    as the cache and main memory, and the effect of wear-leveling when DRAM cache is backed by NVM main memory.
    We also perform experiments with real workloads in full-system simulations to validate the proposed model
    and show the sensitivity of these workloads to the DRAM cache sizes.
    % We address the complexities involved in timing and micro-architecture details of \textit{UDCC} that are vital to support the data
    % consistency across DRAM cache and main memory and reason about the performance degradation caused by these complexities.

\end{abstract}

\begin{IEEEkeywords}
computer architecture, simulation, dram caches
\end{IEEEkeywords}


%\section{Introduction}
\label{sec:introduction}
% \begin{itemize}
%     % Diffusion of FL
%     \item {\st{Diffusion of FL}}
%     % Security threats to FL
%     \item {\st{Security threats to FL with particular focus on model poisoning}}
%     % Limitations of existing countermeasures
%     \item {\st{Current countermeasures (e.g., KRUM) and their limitations}}
%     % Proposed method and its advantages
%     \item {\st{Intuitive description of the proposed method and its difference (i.e., advantages) w.r.t. state of the art}}
%     % Main contributions
%     \item {\st{Summary of the main contributions of this work}}
%     % Paper's structure and organization
%     \item {\st{Paper's structure and organization}}
% \end{itemize}

% Diffusion of FL
Recently, {\em federated learning} (FL) has emerged as the leading paradigm for training distributed, large-scale, and privacy-preserving machine learning (ML) systems~\cite{mcmahan2017googleai,mcmahan2017aistats}. 
The core idea of FL is to allow multiple edge clients to collaboratively train a shared, global model without disclosing their local private training data.
%Specifically, an FL system consists of a central server and many edge clients; 
A typical FL round involves the following steps: {\em(i)} the server randomly picks some clients and sends them the current, global model; {\em(ii)} each selected client locally trains its model with its own private data; then, it sends the resulting local model to the server;\footnote{Whenever we refer to global/local model, we mean global/local model {\em parameters}.} {\em(iii)} the server updates the global model by computing an \emph{aggregation function}, usually the average (FedAvg), on the local models received from clients.
% \begin{enumerate}
%     \item[{\em(i)}] the server sends the current, global model to the clients and appoints some of them for training;
%     \item[{\em(ii)}] each selected client locally trains its copy of the global model with its own private data; then, it sends the resulting local model back to the server;\footnote{Whenever we refer to global/local model, we mean global/local model {\em parameters}.}
%     \item[{\em(iii)}] the server updates the global model by computing an \emph{aggregation function} on the local models received from clients (by default, the average, also referred to as FedAvg~\cite{mcmahan2017aistats}).
% \end{enumerate}
This process goes on until the global model converges. %(e.g., after a certain number of rounds or other similar stopping criteria).
%\\
% The advantages of FL over the traditional, centralized learning paradigm are undoubtedly clear in terms of flexibility/scalability (clients can join/disconnect from the FL network dynamically), network communications (only model weights\footnote{We will use \textit{parameters} and \textit{weights} interchangeably.} are exchanged between clients and server), and privacy (each client's private training data is kept local at the client's end and not uploaded to the server).
\\
% Security threats to FL
%However, the growing adoption of FL also raises security concerns~\cite{costa2022covert}, particularly about its confidentiality, integrity, and availability.
Although its advantages over standard ML, FL also raises security concerns~\cite{costa2022covert}. %, particularly about its confidentiality, integrity, and availability~\cite{costa2022covert}.
% OLD, LONG VERSION
% Indeed, some work deals with privacy leakage that may expose the local data of some clients~\cite{melis2019sp}. 
% A large body of work, instead, investigates attacks that usually aim to detriment the predictive accuracy of the learned global model. For instance, \emph{data poisoning} attacks achieve this goal by letting an adversary pollute the training set of some corrupt FL clients with maliciously crafted examples~\cite{jagielski2018sp}.
% Similarly, in \emph{model poisoning} the attacker attempts to tweak the global model weights~\cite{bhagoji2019pmlr} by directly perturbing the local model's weights of some infected FL clients before these are sent to the central server for aggregation, usually via so-called Byzantine attacks. 
% It turns out that Byzantine model poisoning attacks severely impact standard FedAvg; therefore, more robust aggregation functions must be designed to make FL systems secure.
Here, we focus on \emph{untargeted model poisoning} attacks~\cite{bhagoji2019pmlr}, where an adversary attempts to tweak the global model weights %\footnote{We will use the terms \textit{parameters} and \textit{weights} interchangeably.} 
by directly perturbing the local model's parameters of some infected clients before these are sent to the central server for aggregation.
In doing so, the adversary aims to jeopardize the global model \textit{indiscriminately} at inference time.
Such model poisoning attacks severely impact standard FedAvg; therefore, more robust aggregation functions must be designed to secure FL systems.
\\
% In this paper, we focus on designing a novel robust aggregation scheme at the server's end to contrast the effect of Byzantine model poisoning attacks.
%
% Current countermeasures and their limitations
%Several countermeasures have been proposed in the literature to combat model poisoning attacks on FL systems.
% Some methods use simple statistics more robust than plain average to smooth the impact of malicious updates (e.g., Trimmed Mean and FedMedian~\cite{yin2018icml}). 
% Other defenses implement outlier detection techniques to discard malicious updates from the aggregation performed at the server's end. Those are either based on heuristics (e.g., Krum/Multi-Krum~\cite{blanchard2017nips} and Bulyan~\cite{mhamdi2018pmlr}) or data-driven approaches (e.g., K-means clustering~\cite{shen2016acm} or DnC via spectral analysis~\cite{shejwalkar2021ndss}). 
% Finally, some strategies rely on a centralized ``source of trust'' to spot potential malicious updates (e.g., FLTrust~\cite{cao2020fltrust}).
% Several countermeasures have been proposed in the literature to combat model poisoning attacks on FL systems, i.e., to discard possible malicious local updates from the aggregation performed at the server's end. 
% These techniques range from simple statistics more robust than plain average (e.g., Trimmed Mean and FedMedian~\cite{yin2018icml}) to outlier detection heuristics (e.g., Krum/Multi-Krum~\cite{blanchard2017nips} and Bulyan~\cite{mhamdi2018pmlr}) or data-driven approaches (e.g., spectral analysis via K-means clustering~\cite{shen2016acm} or spectral analysis), or methods based on ``source of trust'' (e.g., FLTrust~\cite{cao2020fltrust}).
% OLD, LONG VERSION
%Several countermeasures have been proposed in the literature to combat Byzantine model poisoning attacks on FL systems.
% Descriptive statistics
% For example, Trimmed Mean and FedMedian aggregate local model updates using more robust statistics than standard average~\cite{yin2018icml}.
%
% % Heuristics for outlier detection
% Many existing Byzantine-resilient strategies implement some outlier detection heuristics to discard the model updates sent by potentially malicious clients from the input of the aggregation function.
% One of the most popular heuristics is Krum~\cite{blanchard2017nips}.
% This strategy tries to mitigate the impact of Byzantine attacks by selecting as a global model the local model with the smallest sum of Euclidean distances to {\em all} the other local models.
% Although powerful, Krum requires the server to know (or, at least, estimate) the number of malicious FL clients upfront, which is generally impossible in a realistic attack scenario. %
% Moreover, Krum may become ineffective for complex, high-dimensional model parameter spaces due to the curse of dimensionality.
% Bulyan~\cite{mhamdi2018pmlr} tries to overcome this issue by combining Krum with a variant of Trimmed Mean.
% % Data-driven outlier detection
% Other strategies use data-driven outlier detection techniques -- e.g., via K-means clustering~\cite{shen2016acm} -- to spot potential malicious local model updates. 
% %For instance, Shen et al. propose to cluster local model updates with K-means and thus identify outliers.
%
% % Other techniques
% As far as the server is concerned, any local model received can be from a potential malicious client. 
% FLTrust~\cite{cao2020fltrust} assumes the server acts as a client, i.e., trains a local model on an additional {\em trustworthy} dataset at the server's end and compares it against all the local models from other clients. 
% This way, the server can rely on some ``source of trust'' when discarding potentially malicious clients.
%\\
% Limitations of existing Byzantine-resilient strategies
Unfortunately, existing defense mechanisms either rely on simple heuristics (e.g., Trimmed Mean and FedMedian by~\cite{yin2018icml}) or need strong and unrealistic assumptions to work effectively (e.g., foreknowledge or estimation of the number of malicious clients in the FL system, as for Krum/Multi-Krum~\cite{blanchard2017nips} and Bulyan~\cite{mhamdi2018pmlr}, which, however, cannot exceed a fixed threshold).
Furthermore, outlier detection methods using K-means clustering~\cite{shen2016acm} or spectral analysis like DnC~\cite{shejwalkar2021ndss} do not directly consider the temporal evolution of local model updates received.
Finally, strategies like FLTrust~\cite{cao2020fltrust} require the server to collect its own dataset and act as a proper client, thereby altering the standard FL protocol.
\\
% OLD, LONG VERSION
% Overall, existing Byzantine-resilient strategies are either simple heuristics (e.g., FedMedian) or, if they are more complex, they rely on strong and unrealistic assumptions to work effectively (e.g., knowing the number of malicious clients in the FL system in advance, as for Krum and alike).
% Furthermore, data-driven outlier detection methods do not consider the temporary evolution of local model updates received (e.g., K-means clustering). 
% Finally, strategies like FLTrust requires the server to collect its own dataset and act as a proper client, thereby altering the standard FL protocol.
%
% Description of the proposed method
This work introduces a novel pre-aggregation \textit{filter} robust to untargeted model poisoning attacks. Notably, this filter $(i)$ operates without requiring prior knowledge or constraints on the number of malicious clients and $(ii)$ inherently integrates temporal dependencies. 
The FL server can employ this filter as a preprocessing step before applying \textit{any} aggregation function, be it standard like FedAvg or robust like Krum or Bulyan.
Specifically, we formulate the problem of identifying corrupted updates as a multidimensional (i.e., matrix-valued) time series anomaly detection task. 
The key idea is that legitimate local updates, resulting from well-calibrated iterative procedures like stochastic gradient descent (SGD) with an appropriate learning rate, show \textit{higher predictability} compared to malicious updates. This hypothesis stems from the fact that the sequence of gradients (thus, model parameters) observed during legitimate training exhibit regular patterns, as validated in Section~\ref{subsec:intuition}. %until convergence. 
%This regularity may be more pronounced for smooth convex loss functions, but it can still be captured within an appropriate time window, even for more complex and convoluted loss surfaces. 
%We provide evidence of this claim in Appendix~B, where we show that the average mutual information (i.e., ``predictability''), calculated over pairs of legitimate model updates sent at different FL rounds, is significantly higher than the corresponding computation for a malicious client.
\\
Inspired by the matrix autoregressive (MAR) framework for multidimensional time series forecasting~\cite{chen2021je}, we propose the FLANDERS ({\em \textbf{F}ederated \textbf{L}earning meets \textbf{AN}omaly \textbf{DE}tection for a \textbf{R}obust and \textbf{S}ecure}) filter.
The main advantages of FLANDERS over existing strategies like FLDetector~\cite{zhao2020multivariate} are its resilience to large-scale attacks, where $50\%$ or more FL participants are hostile, and the capability of working under realistic non-iid scenarios.
We attribute such a capability to two key factors: $(i)$ FLANDERS works without knowing a priori the ratio of corrupted clients, and $(ii)$ it embodies temporal dependencies between intra- and inter-client updates, quickly recognizing local model drifts caused by evil players. Below, we summarize our main contributions:

\begin{itemize}
\item[{\em(i)}]
We provide empirical evidence that the sequence of models sent by legitimate clients is more predictable than those of malicious participants performing untargeted model poisoning attacks.
\\
\item[{\em(ii)}] 
We introduce FLANDERS, the first pre-aggregation filter for FL robust to untargeted model poisoning based on multidimensional time series anomaly detection.
\\
\item[{\em(iii)}] 
We integrate FLANDERS into Flower,\footnote{\scriptsize{\url{https://flower.dev/}}} a popular FL simulation framework for reproducibility.
\\
\item[{\em(iv)}] 
We show that FLANDERS improves the robustness of the existing aggregation methods under multiple settings: different datasets, client's data distribution (non-iid), models, and attack scenarios.
\\
\item[{\em(v)}] 
We publicly release all the implementation code of FLANDERS along with our experiments.\footnote{\scriptsize{\url{https://anonymous.4open.science/r/flanders_exp-7EEB}}}
\end{itemize}

% Paper's structure and organization
The remainder of the paper is structured as follows. %some related work and the current state-of-the-art solutions to security issues that FL entails. 
Section~\ref{sec:background} covers background and preliminaries. 
In Section~\ref{sec:related}, we discuss related work.
Section~\ref{sec:problem} and Section~\ref{sec:method} describe the problem formulation and the method proposed. % to tackle it. 
Section~\ref{sec:experiments} gathers experimental results. %, and Section~\ref{sec:limitations} discusses some limitations of this work.
Finally, we conclude in Section~\ref{sec:conclusion}.
 %discusses the limitations of this work and draws future research directions.
%reports conclusions and draws perspectives for future research directions.

%%%%%%% OLD %%%%%%%
%to overcome the resilience of Byzantine failures in distributed Stochastic Gradient Descent computations. 
% The strength of Krum is its time complexity, which is linear in the gradient dimension. 
% However, the robustness of the approach is guaranteed for gradient-based learning applications only when the majority of the clients are not compromised. 
% Besides, the aggregation mechanism of Krum, as well as that of similar methods, is robust from a coarse-grained perspective and does not provide solutions to errors and perturbations that may occur at inference time.
%A related approach to~\cite{blanchard2017nips} is the work of Su et al.~\cite{su2016dc}. Here, the authors propose an iterated approximate agreement to tackle a multi-layer scenario attacked by Byzantine agents. 
%However, the method works efficiently on the sole discrete context and it is inapplicable to continuous state environments.
%\gabri{Maybe, we should just talk about the main limitations of existing countermeasures without digging into their details (or, we can just mention Krum as this is the most popular one). I will move the description of all these methods to the Related Work section.}
\section{Introduction}

The last decade has seen significant academic research on DRAM caches, and today
these ideas are becoming a reality with CPU vendors implementing DRAM cache-based computer systems, e.g., Intel's Cascade Lake and Sapphire Rapids.
Hardware-managed DRAM caches are seen as one way to enable heterogeneous memory systems (e.g., systems with DRAM and non-volatile memory) to be more easily programmable.
DRAM caches are transparent to the programmer and easier to use than manual data movement.

However, recent work has shown that these transparent hardware-based data movement designs are much less efficient than manual data movement~\cite{hildebrand2021case}.
While the work by Hildebrand et al.~\cite{hildebrand2021case} and other recent work investigating Intel's Cascade Lake systems provides some insight into real implementations on DRAM caches~\cite{izraelevitz2019basic,wang2020characterizing}, there is a gap in the community's access to cycle-level simulation models for DRAM caches.
This paper describes a new \gem{}-based model of a unified DRAM cache controller inspired by the Cascade Lake hardware to fill this gap.

Previous work has explored many aspects of DRAM cache design in simulation such as the replacement policy, caching granularity~\cite{qureshi2012fundamental,jevdjic2013stacked}, dram cache tag placement~\cite{huang2014atcache,loh2012supporting,loh2011efficiently}, associativity~\cite{qureshi2012fundamental,kotra2018chameleon,young2018accord}, and other metadata to improve performance~\cite{loh2011efficiently,jevdjic2013stacked,young2018accord}.
These mostly high-level memory system design investigations can appropriately be evaluated with trace-based or non-cycle-level simulation.
However, as shown in recent work, the micro-architecture of the unified DRAM and non-volatile main memory (NVRAM) controller can lead to unexpected performance pathologies not captured in these prior works (e.g., Hildebrand et al. showed that a dirty miss to the DRAM cache requires up to \emph{five accesses} to memory~\cite{hildebrand2021case}).

Thus, to better understand these realistic DRAM cache systems, it is imperative to build a detailed DRAM cache simulation model which can be used to perform a design space exploration around the DRAM cache idea.
The previous research works on DRAM cache design improvements do not provide any (open-source) DRAM cache modeling platform for a detailed micro-architectural and timing analysis.
To the best of our knowledge, most research works do not consider systems where the hardware-managed DRAM cache and NVRAM are sharing the same physical interface and are controlled by a unified memory controller (as is the case in real platforms like Intel Cascade Lake).

In this work, we describe our unified DRAM cache and main memory controller (\textit{UDCC}) cycle-level DRAM cache model for \gem{}~\cite{lowepower2020gem5}.
The protocol takes inspiration from the actual hardware providing DRAM cache, such as Intel's Cascade Lake, in which an NVRAM accompanies a DRAM cache as the off-chip main memory sharing the same bus.
To model such hardware, we leverage the cycle-level DRAM~\cite{hansson2014simulating} and NVRAM~\cite{gem5-workshop-presentation} models in \gem{}.
Our model implements the timing and micro-architectural details enforced by the memory interfaces including the DRAM timing constraints, scheduling policies, buffer sizes, and internal queues.
We propose a DRAM cache model that is direct-mapped, insert-on-miss, and write-back to model Intel's Cascade Lake design.
% FUTURE WORK: though in our model these policy decisions are parameterized \note{IS THIS TRUE???}.

Using this model, we present validation data and investigate five case studies.

\emph{What is the impact of memory scheduling policies in a unified DRAM cache and memory controller?}
We find that using FR-FCFS is highly impactful when the cache hit ratio is high, but less so when the hit ratio is low and the NVRAM's bandwidth limits performance.

\emph{What is the impact of DRAM technology on performance and memory controller architecture?}
We find that higher performing memory technologies require more buffering to achieve peak performance. 
Moreover, we find that the composition of the memory access patterns and their hit/miss ratio on DRAM cache, 
can also affect the amount of buffering to achieve the peak bandwidth.

\emph{What is the impact of backing ``main memory'' performance?}
We find that while slower backing memory hurts performance, the performance of the backing memory does not have a significant affect on the micro-architecture of the cache controller.

\emph{What is the impact of the UDCC model for full-system applications?}
We find that our model shows similar performance characteristics on real applications as previously shown on real hardware providing further evidence for the importance of cycle-level simulation.

\emph{What is the impact of NVRAM wear leveling on memory system performance with a DRAM cache?}
We find that while wear leveling has a very small direct impact, the impact when using NVRAM as backing memory with a DRAM cache can be much higher. Although only 1 in 14,000 requests experience a wear-leveling event, the performance impact is up to an 8\% slowdown.

Our model is open-source and publicly available for the use of research community~\cite{dcacheGem5Code} and will be integrated into \gem{} mainstream. Using this new model which implements the micro-architectural details of realistic DRAM caches on a simulator can help find any potential improvement for the next generation of memory systems.

%The rest of the paper is organized as follows...

\section{Background on Network Calculus}
\label{sec: background}


\begin{figure*}[tbh]
\centering
\begin{subfigure}[b]{0.3\textwidth}
    \centering
    \includegraphics[width=\linewidth]{images/in-out.png}
    \caption{Arrival and departure data and their relation with delay $d(t)$ and backlog $b(t)$. For a FIFO system, the delay is the horizontal distance between $R(t)$ and $R^*(t)$ but some other multiplexing techniques may shift the data to a later priority, causing a longer delay.}
    \label{fig: data in-out}
\end{subfigure}
\hfill
\begin{subfigure}[b]{0.35\textwidth}
    \centering
    \includegraphics[width=\linewidth]{images/arrival-service.png}
    \caption{Characteristics of an arrival curve and a service curve. From any point of observation, the arriving data never exceeds its arrival curve; the departure data is also never less than the service curve with respect to the data arrival.}
    \label{fig: arrival-service curves}
\end{subfigure}
\hfill
\begin{subfigure}[b]{0.33\textwidth}
    \centering
    \includegraphics[width=\linewidth]{images/bound.png}
    \caption{Delay and backlog bounds of a system. Backlog is the maximum vertical distance between $\alpha(t)$ and $\beta(t)$; FIFO delay is their maximum horizontal distance; but for arbitrary multiplexing, the delay guarantee is when the system clears its buffer, thus it's the intersection of $\alpha(t)$ and $\beta(t)$.}
    \label{fig: system bounds}
\end{subfigure}
\caption{Network calculus framework. We let $R(t)$ and $R^*(t)$ be the arrival and departure data flow of a system; $\alpha(t)$ be the piecewise linear concave arrival curve and $\beta(t)$ be the piecewise linear convex service curve of a system.}
% \hossein{Better to show piece-wise linear concave arrival curve and piece-wise linear convex service curve instead of token-bucket and rate-latency.}}
\end{figure*}

We recall some of the network calculus essentials for a better understanding of the framework used in Saihu. In the following context, we use the following notation: $\mbb{R}^+$ is the set of non-negative real numbers; $[x]_+$ denotes $\max(0, x)$

The data flow is by convention modeled as a left-continuous wide-sense increasing function $R(t): \mbb{R}^+ \mapsto \mbb{R}^+$ with respect to time $t$~\cite{ncbook2001leboudec}. 

A system $\mcal{S}$ receives arrival data described as a cumulative function $R(t)$ and delivers departure data as another cumulative function $R^*(t)$. Figure~\ref{fig: data in-out} illustrates such a system $\mcal{S}$. The benefit of representing a system like this is that we can observe system backlog and delay with such a model. 

\begin{definition}[Backlog and Delay~\cite{ncbook2001leboudec}]
    The backlog of a system at time~$t$ is
    \begin{equation}
        b(t) = R(t) - R^*(t)
    \end{equation}
    
    The virtual delay of a FIFO system at time $t$ is
    \begin{equation}
        d_{FIFO}(t) = \inf \lbp \tau \geq 0 : R(t) \leq R^*(t+\tau) \rbp
    \end{equation}
\end{definition}



The backlog of a system can be viewed as the vertical distance between $R$ and $R^*$. The FIFO (\textit{First-in First-out}) delay is the horizontal distance between $R$ and $R^*$. One may obtain other delay values if the multiplexing technique is not FIFO.

% \begin{figure}
%     \centering
%     \includegraphics[width=0.9\linewidth]{images/in-out.png}
%     \caption{In/out data flow; delay and backlog}
%     \label{fig: data in-out}
% \end{figure}

Since we are interested in the system guarantee instead of a single instance of data flow, we would like to have general bounds to the arrival and departure data flows. Therefore, we define \textit{arrival curve} and \textit{service curve} as the bounds of arrival and departure data flows.

\begin{definition}[Arrival Curve~\cite{ncbook2001leboudec}]
    Given a wide-sense increasing function $\alpha: \mbb{R}^+ \mapsto \mbb{R}^+$, we say that a flow $R(t)$ is $\alpha$-constrained if and only if for all $s \leq t$:
    \begin{equation}
        R(t) - R(s) \leq \alpha(t-s)
    \end{equation}
    We say $R(t)$ has $\alpha$ as an arrival curve.
\end{definition}

\begin{definition}[Service Curve~\cite{ncbook2001leboudec}]
    Given a wide-sense increasing function $\beta: \mbb{R}^+ \mapsto \mbb{R}^+$ and $\beta(0) = 0$. A system $\mcal{S}$ having $R(t)$ and $R^*(t)$ as its arrival and departure flows. We say $\mcal{S}$ offers a service curve $\beta$ if and only if
    \begin{equation}
        R^*(t) \geq (R \otimes \beta)(t) =: \inf_{s \leq t} \lbp R(s) + \beta(t-s) \rbp
    \end{equation}
    where $\otimes$ denotes the min-plus convolution
\end{definition}

Figure~\ref{fig: arrival-service curves} illustrates the arrival and service curves. Any segment of arrival flow $R(t)$ is constrained by arrival curve $\alpha$ and the output curve $R^*(t)$ is always no less than the curve $R\otimes\beta$. As a result, an arrival curve upper bounds the incoming traffic, and a service curve lower bounds the outgoing traffic.

% \begin{figure}
%     \centering
%     \includegraphics[width=\linewidth]{images/arrival-service.png}
%     \caption{Arrival/Service curve}
%     \label{fig: arrival-service curves}
% \end{figure}

We consider 2 special types of curves throughout this paper, \textit{token-bucket} (or sometimes called \textit{leaky-bucket}) curve and \textit{rate-Latency} curve.

\begin{definition}[Token-bucket and Rate-latency~\cite{ncbook2001leboudec}]
    A token-bucket curve $\gamma_{r,b}$ with arrival rate $r$ and burst $b$ is defined as
    \begin{equation}
        \gamma_{r,b}(t) = b + rt
    \end{equation}

    A rate-latency curve $\beta_{R,T}$ with service rate $R$ and latency $T$ is defined as
    \begin{equation}
        \beta_{R,T}(t) = R \lb t - T \rb_+
    \end{equation}
\end{definition}

A token-bucket curve is determined by a burst $b$ and an arrival rate~$r$. Burst represents the maximum possible data volume that can arrive simultaneously, and arrival rate represents the maximum long-term data rate~\cite{bouillard2022tradeoff}.
A rate-latency curve is determined by a latency~$T$ and a service rate~$R$. Latency represents the time a server needs before starting to process the incoming data, and service rate represents the minimum rate to process data after the initial latency.

With the help of arrival and service curves, we can derive delay and backlog bounds for a system $\mcal{S}$ illustrated in Figure~\ref{fig: system bounds}. Suppose a system $\mcal{S}$ has arrival curve $\alpha$ and service curve~$\beta$, its worst-case backlog $b^*$ is the maximum vertical distance between~$\alpha$ and~$\beta$. Similarly, depending on the multiplexing technique applied to the system, its worst-case delay bound $d^*$ is the maximum horizontal distance between $\alpha$ and $\beta$ if $\mcal{S}$ is a FIFO system. If we don't have any information about its multiplexing technique, referred to as arbitrary multiplexing, the best we can say is that when $\alpha$ and $\beta$ intersect each other, where all data has been delivered out of the system. Consequently, the worst-case delay bound for arbitrary multiplexing is the time required for $\mcal{S}$ to clear its buffer.

% \begin{figure}
%     \centering
%     \includegraphics[width=\linewidth]{images/bound.png}
%     \caption{System delay/backlog bounds}
%     \label{fig: system bounds}
% \end{figure}

While a service curve captures the slowest possible output speed of a system, a link's transmission capacity limits the speed as well. Hence, we model this phenomenon using a \textit{greedy shaper} with a sub-additive function $\sigma: \mbb{R}^+ \mapsto \mbb{R}^+$ concatenated with a server. We consider a concatenation as shown in Figure \ref{fig: system}. By convention we assume $\sigma(0) = 0$ and $\beta(t) \leq \sigma(t), \forall t \in \mbb{R}^+$, meaning that the buffer is cleared at the beginning and the service never exceed its physical limitation. With the above definition, such greedy shaper conserves the service provided by the system due to theorem \ref{thm: shaping}.

\begin{figure}[thb]
    \centering
    \includegraphics[width=0.7\linewidth]{images/system.png}
    \caption{Shaping of departure data. A flow that has an arrival curve $\alpha$ feeds into a server with an arrival data flow $R(t)$. The server having service curve $\beta$ takes $R(t)$ and gives a departure data flow $R^*(t)$ to a shaper with shaping function $\sigma$. The shaper takes $R^*(t)$ and shape the data flow as another departure $D(t)$.}
    \label{fig: system}
\end{figure}


\begin{theorem}[Shaping conserves service \cite{ncbook2001leboudec}]
\label{thm: shaping}
Following the system shown in Figure \ref{fig: system}, we have
\begin{equation}
     D = R^* \otimes \sigma \geq \lp R \otimes \beta \rp \otimes \sigma = R \otimes \lp \beta \otimes \sigma \rp = R \otimes \beta
\end{equation}
\end{theorem}

In the following context, we model the shaping function $\sigma$ as a token-bucket curve $\gamma_{C,L}$ with transmission capacity $C$ and the packet size $L$ to capture the link capacity and packetization~\cite{bouillard2022tradeoff}.

\section{Design}
\label{s:design}
In this section, we will first present the core of our system. Then we present some analysis of the system along with some extensions to address a few practical concerns. We will present details of our cloud implementation separately in the next section.

\subsection{Delivery Based Ordering}
Our solution is composed of three parts. 
\subsubsection{Delivery Clock\\}
\noindent\textbf{What we do.}
Each RB maintains a delivery clock. This delivery clock essentially tracks time relative to when market data was delivered to the participant. We use $DC(i,a)$ to represent delivery clock of participant $i$ at time when trade $(i,a)$ is submitted. Delivery clock is a lexicographical tuple.
\begin{align}
    DC(i,a) = \langle ld(i,a), S(i,a)-D(i, ld(i,a))\rangle.
\end{align}
where $ld(i,a)$ is the latest data point that was delivered to MP$_i$ by time S(i,a), i.e., $D(i,ld(i,a)) \leq S(i,a) < D(i,ld(i,a)+1)$). 
Interval, $S(i,a)-D(i, ld(i,a))$, corresponds to the time that has elapsed since the last delivery and can be measured locally at the RB without requiring any clock synchronization (challenge 1). 

\noindent
\textit{Monotonicity:} Delivery clocks advance monotonically with submission time. As a result, DBO trivially satisfies the causality condition (Equation~\ref{eq:causality}). Further, it incentivizes the participants to submit trades as early as possible. Therefore, \emph{a participant cannot gain any advantage by delaying trades.} %\pg{should this point have a heading of its own}
Finally, we also leverage the monotonic property to overcome challenge 3 (\S\ref{ss:enforcing_ordering}). Figure~\ref{fig:delivery_clock} shows how delivery clock advances with time.

%\pg{I tried to reduce the notation here. I defined delivery clock slightly differently.}

\begin{figure}[t]
\centering
    \includegraphics[width=0.8\columnwidth]{figures/delivery_clock.pdf}
    \caption{\small{\bf Delivery Clock.}}% \pg{Redraw}}% \pg{Eashan see Ranveer's comment}}% \pg{Eashan can you redraw this figure in powerpoint or something.}}}
    \label{fig:delivery_clock}
    \vspace{-2.5mm}
\end{figure}

All incoming trades are marked with the delivery clock at the trade submission time. The ordering buffer uses this delivery clock time to order trades. Formally, the ordering in DBO is given by,  

\vspace{-2mm}
\begin{align}
    O(i,a) = DC(i, a). 
    \label{eq:ordering_with_dc}
\end{align}


\begin{figure}[t]
\centering
    \includegraphics[trim={0 0 0 2mm},clip,width=0.8\columnwidth]{figures/dbo_correct.pdf}
    \vspace{-4mm}
    \caption{\small{{\bf DBO can help correct for late delivery of data.} Delivery of market data to MP$_i$ is lagging behind MP$_j$. There are two trades $(i,a)$ and $(j,b)$ generated in response to the same market data $x$. $(j,b)$ was submitted before $(i,a)$ but
    %, i.e., $S_j(l) < A_i(k)$. 
    response time of $(i,a)$ is less than $(j,b)$.
    %, i.e., $rt_i(k) < rt_j(l)$. 
    In this example, $DC(i,a) (= \langle x, RT(i,a)\rangle) < DC(j,b) (= \langle x, RT(j,b)\rangle)$ and trade $(i,a)$ is correctly ordered ahead of $(j,b)$.}} %Ordering based on the submission time leads to incorrect ordering.}
    %\pg{Correct figure}}
    \label{fig:dbo_correction}
    \vspace{-3mm}
\end{figure}


\noindent\textbf{Why it works.}
When the trigger point of trade $(i,a)$ is indeed the last data point (i.e., $x = TP(i,a) = ld(i, a)$), then, DBO respects condition C2 for LRTF. Figure~\ref{fig:dbo_correction} shows an illustrative example of this.
This is because, the delivery clock directly tracks the response time of $i,a$ in this case and $O(i,a) = DC(i, a) = \langle x, RT(i,a)\rangle$. For a competing trade $(j,b)$ with higher response time, the delivery clock at time of submission will either read $O(j,b) = DC(j, b) = \langle x, RT(j,b)\rangle$ (if S(j,b)<D(j,x+1)) or $DC(j, b) = \langle y, S(j,b)-D(j,y)\rangle$ with $y>x$. In both cases, $O(i,a) < O(j,b)$.


At a high level, in our ordering we are correcting for latency differences in data delivery by using the delivery time of the last data point. When the last data point is not the trigger point for trade $(i,a)$, DBO satisfies the LRTF condition C2, if the following condition holds, 
\begin{align}
    D(i,ld(i,a))-D(i,x) = D(j,ld(i,a))-D(j,x),
    \label{eq:cond_delivery_lrtf}
\end{align}
where $x = TP(i,a)$.  
While it is impossible to ensure that inter-delivery times remain the same for all participants for all points, by pacing data at the RB it is indeed possible to ensure that the above condition is always met.% \radhika{you meant C2 or the above condition?}. \pg{the above condition only}
The main reason why we can meet the above condition is that condition C2 limits that the trigger point $x$ cannot be any arbitrary data point in the past, and that the trigger point must have been delivered recently  $S(i,a)-D(i,x) < \delta$.
%and we only need to ensure same inter-delivery times for. 
In the next subsection, we will show how we can achieve this and solve challenge 2. %\pg{Is this easy to follow?}



%\pg{FIX: say delivery clocks helps correct has static differences in latency. Why are delivery clocks so good on their own, give more intuition and experimentation. Potential things to include, see 6.1. Maybe make a section of.delivery clock on its own. correct the equation here in terms of response time as well.}
%\pg{Should we include results on necessary conditions on delivery times for achieving LRTF. Maybe its a bit of an overkill.}

\noindent
\textit{Remark:} In our cloud experiments, we find that DBO achieves fairness with very high probability. This is because network latency (from CES to any given participant) exhibits temporal correlation in latency especially over  short periods of time. When temporal correlation is high, inter-delivery time at any participant is close to the inter-generation time at the CES. In such cases, condition given by Equation~\ref{eq:cond_delivery_lrtf} is satisfied with high probability.

\noindent
\textbf{Difference with traditional logical clocks:} Logical clocks are commonly used in distributed systems. The most famous ones are lamport clocks~\cite{lamportSeminalPaper} and vector clocks. These clocks can be used for achieving total causal ordering of events. While these clocks can track causality of events, they cannot be used to achieve response time fairness. In particular, these clocks don't say anything about how two competing trades generated using the same market data should be ordered as these two trades have no direct causality relation. Unlike delivery clocks, such logical clocks also have no notion of measuring time between occurrences of two events. Time difference between events is critical to achieve fairnesss for exchanges. 

\noindent\textit{Note:} Several major financial exchanges already rely on heartbeats~\cite{nyse-client} for liveness when traffic is low.


\begin{figure}[t]
\centering
    \includegraphics[width=0.8\columnwidth]{figures/batching_pacing.pdf}
    \vspace{-2mm}
    \caption{\small{\bf Batching and Pacing. Inter-delivery time for consecutive batches is equal to or more than $\delta$.}}% \pg{Redraw}}% \pg{Eashan see Ranveer's comment}}% \pg{Eashan can you redraw this figure in powerpoint or something.}}}
    \label{fig:batching_pacing}
    \vspace{-4.5mm}
\end{figure}

\subsubsection{Batching and Pacing\\}
\noindent
\textbf{What we do.}
In DBO, the CES breaks data into batches. Each new batch contains all data points in the duration $(1+\kappa) \cdot \delta$ after the previous batch. Here $\kappa > 0$. Each release buffer delivers all data points in a batch at the same time. %Two points $x,y$ belonging to the same batch are delivered simultaneously to each participant, i.e., $D(j,y)=D(j,x), \forall j$.
The release buffer delivers batches as quickly as possible while ensuring that the time between delivery of two consecutive batches is atleast $\delta$. Figure~\ref{fig:batching_pacing} shows an illustration of batching. Both batching and pacing increase the delivery time of data points. In the next subsection we will analyze the impact of the two on latency. Note that in the event of queue build up at the RB, since the batch generation rate ($\frac{1}{(1+\kappa) \cdot \delta}$) is slower than the batch dequeue rate($\frac{1}{\delta}$), the queue at the RB eventually gets drained(\S\ref{ss:understanding_latency}).


\noindent
\textbf{Why it works.} With batching and pacing, DBO achieves LRTF. In particular, 
consider a trade $(i,a)$ with response time less than $\delta$. Because of pacing, consecutive batches are separated atleast by $\delta$. This means that the trigger point ($x=TP(i,a)$) must be within the last received batch. The point $ld(i,a)$ is also the last point in this batch and $D(i,ld(i,a)) = D(i,x)$. \emph{With batching and pacing, the delivery clock again directly tracks the response time of $(i,a)$} and $O(i,a) = DC(i,a) = <ld(i,a), RT(i,a)>$.
With batching, for participant $j$, $x$ and $ld(i,a)$ also belong to the same batch $D(j,ld(i,a)) = D(j,x)$.
For a competing trade $(j,b)$ with higher response time, the delivery clock at the time of submission will either read $O(j,b) = DC(j,b)) = \langle ld(i,a)), RT(j,b)\rangle$ (if $(j,b)$ was submitted before the next batch, i.e., $S(j,b) < D(j,ld(i,a)+1)$) or $DC(j, b) = \langle y, S(j,b)-D(j,y)\rangle$ with $y>ld(i,a)$. In both cases, $O(i,a) < O(j,b)$.

\if 0
\begin{figure}[t]
\centering
    \includegraphics[width=0.8\columnwidth, angle = -90]{images/pq_hb.jpg}
    \vspace{-2.5mm}
    \caption{\small{\bf Enforcing the ordering.} \pg{Redraw}}% \pg{Eashan see Ranveer's comment}}% \pg{Eashan can you redraw this figure in powerpoint or something.}}}
    \label{fig:pq_hb}
    \vspace{-2.5mm}
\end{figure}
\fi

\subsubsection{Enforcing the ordering\\}
\label{ss:enforcing_ordering}
OB contains a priority queue where all incoming trades are sorted based on the delivery clock timestamp (Equation~\ref{eq:ordering_with_dc}). A trade $(i,a)$ at the head of the priority queue should be forwarded to the CES only when the OB has received all trades $(j,b)$ with lower ordering $DC(j,b) < DC(i,a)$. 

\noindent
\textit{OB's Heartbeat Handler:} In DBO, each RB sends a heartbeat periodically every $\tau$ seconds to the CES. The heartbeat $(i,h)$, from participant $i$ contains the delivery clock timestamp at the time the heartbeat was generated ($DC(i,h)$). Since data in delivered in order and because delivery clock advances monotonically with time, heartbeat $(i,h)$ tells the OB that it has received all trades from participant $i$ with delivery clock less than $DC(i,h)$. The ordering buffer forwards trade $(i,a)$ if it has received heartbeats from all the participants with delivery clock timestamp higher than $DC(i,a)$. 


\subsection{Understanding DBO}

\subsubsection{Latency, parameter setting and straggler mitigation\\}
\label{ss:understanding_latency}

We will first derive the optimal latency for any ordering system that achieves response time fairness. We will then discuss how DBO compares to  optimal latency. We will also present guidelines for setting parameters and how to mitigate stragglers that can impact latency.

We define latency for trade $(i,a)$, $L(i,a)$, as the sum of latency in delivering data (from generation time) and latency in trade forwarding to the CES (from trade submission time). Formally,
\begin{align}
    L(i,a) = (D(i, x) - G(x)) + (F(i,a) - S(i,a)),\nonumber\\
    L(i,a) = F(i,a) - G(x) - RT(i,a),
    \label{eq:latency_def}
\end{align}
where $x=TP(i,a)$.

\noindent
\textbf{Optimal Latency:} Formally trade $(i,a)$ can only be forwarded to the CES's ME only when the CES has received all potential competing trades $(j,b)$ with lower response times ($RT(j,b) < RT(i,a)$). Let $R(i, x, RT)$ represent the time when the CES receives trade $(i,a)$ whose whose trigger point is x and response time is RT. Formally, 
\begin{align}
    F(i,a) = \max_{j}(R(j, x=TP(i,a), RT=RT(i,a))). 
\end{align}
A subtle point to note here is that even if participant $j$ does not produce any trades, we still need to wait for that participant till $R(j, x=TP(i,a), RT(i,a))$. Before this time, fundamentally the CES cannot be sure that it will not receive a trade from participant $j$ with a lower response time. 

We use $RTT(i, x, RT)$ to represent the sum of raw network latency for point x from CES to MP $_i$ and latency of trade from MP$_i$ to the CES (whose trigger point is x and response time RT).  In the best case scenario for latency (no buffering at any point in the path) we get
\begin{align}
    R(i, x, RT) = G(x) + RTT(i, x, RT) + RT.
\end{align}


Using the above two equations, we can write the following theorem.
\begin{theorem}
For any ordering system that achieves response time fairness, the minimum latency for trade $(i,a)$ is given by,
\begin{align}
    L(i,a) = \max_{j}(RTT(j, x=TP(i,a), RT=RT(i,a))).
\end{align}
\vspace{-2mm}
\label{thm:latency}
\end{theorem}

Put it simply, the above theorem states for achieving response time fairness, the minimum latency is bounded by the maximum round trip time across all participants. This means that fundamentally bad latency for a participant affects the latency of all trades. To achieve low latency consistently, we would like to ensure that latency of all the participants is well behaved majority of the times. How to better achieve this goal is left as a subject for future work.

%This theorem implies that even in cloud settings exchanges should ask for  network latency  

%With a very large number of participants thus pose a 
%\pg{fundamental issue with scalability}

\noindent
\textbf{How does DBO compare with the optimal?} DBO achieves close to optimal latency.  Compared to the optimal, batching and pacing introduce additional delay in delivery of market data points.  Since heartbeats are  generated only periodically they can  introduce an additional delay of $\tau$ at the ordering buffer. We now discuss the delay due to each of these components and how do the parameters $\kappa$, $\delta$ and $\tau$ affect latency. %\pg{Include a table here for parameters?}

\noindent
\textbf{Impact of batching:} Batching can introduce an additional delay of $(1+\kappa)\cdot \delta$ in the worst case. 

\noindent
\textit{Setting $\delta$:} $\delta$ thus presents a trade-off between latency and fairness (how large of a horizon can we pick). The right trade-off really depends on the needs of the exchange. Ideally, the exchange should pick the minimum value of $\delta$ that accommodates the response time of the fastest participants in a race. Our conversations reveal that fastest participants typically respond within a few microseconds and majority of the speed races last 5-10 $\mu s$. For our cloud experiments we  use $\delta = 20 \mu s$.

\begin{figure}[t]
    \centering
    \includegraphics[trim={0 0 0 0mm},clip,width=0.8\linewidth]{images/latency_b+p.pdf}
    \vspace{-5mm}
    \caption{\small{\textbf{Latency in data delivery:} x-axis shows the generation time of the market data. y-axis shows the latency from generation time to data delivery. $\kappa$  governs the average slope of the orange line immediately after latency spike (slope = $\frac{\kappa}{1+\kappa}$}).} %\pg{Include orange line and the base latency. Change labels to DBO and direct-delivery. Slope is $\kappa/(1+kappa)$}}
    %\pg{Eashan: Include the drain rate, make the colored lines thicker and use different linestyles for the three schemes..}}% \pg{Maybe label the drain rate in the figure for S1 and S2.}}
    \label{fig:latency_b+p}
    \vspace{-5mm}
\end{figure}

\noindent
\textbf{Impact of pacing.} Pacing restricts the batch dequeue rate at the RB. When network latency to a participant is not varying, the batch arrival/enqueue rate at the RB ($\frac{1}{(1+\kappa) \cdot \delta}$) is higher than the batch dequeue rate limit ($\frac{1}{\delta}$) and there is no queue build up. However, when network latency to a participant is decreasing (e.g., after a latency spike), batch arrival rate at the RB can exceed the dequeue rate limit leading to a queue build up. The overall queue - dequeue rate can be given by $\text{batch size} \cdot \text{batch rate limit} = 1+\kappa$. Figure~\ref{fig:latency_b+p} shows the impact of batching and pacing on latency in delivery of data in the event of a queue build up. The figure also shows the latency when data is delivered directly (raw network latency). The smaller sawtooths in the batching + pacing are because of batching. The deviation in direct delivery and batching + pacing is because of the rate limit imposed by pacing.

\noindent
\textit{Setting $\kappa$:} Increasing $\kappa$ increases batching delay but also increases the queue drain rate in the event of queue build up due to tail latency spikes. Increasing $\kappa$ thus presents a trade-off between reducing tail latency and increasing average latency. In our experiments we use $\kappa = 0.25$.
 
\noindent\textbf{Impact of heartbeats:} Heartbeats present a trade-off. Too frequent heartbeats can overwhelm the network, the ordering buffer or the release buffer. 
Infrequent heartbeats can increase the time OB has to wait of the participants. In particular, hearbeats can introduce an additional wait time of $\tau$. Note that the number of heartbeats, the OB needs to process increases linearly with the number of participants. In the next section we show how the heartbeat handler can be sharded for scalability.

\noindent\textit{Setting $\tau$:} Ideally we want to pick as low of a value as possible for the heartbeats without overwhelming the system. This number is very much dependent on the capabilities of the network and the processing power of the RB and the OB. In our cloud implementation we use $\tau = 20 \mu s$.

\noindent\textit{A note on latency:} When the network latency to participants is not varying with time, there is no queue build up at the release buffers. In such cases, DBO adds maximum of $((1+\kappa)\cdot \delta) + \tau$ additional latency over the optimal.

\noindent\textbf{Straggler Mitigation and RB/MP failure} In the event a  participant or release buffer crashes, DBO can stall processing trades. Further, the overall system latency also gets impacted when a certain participant is experiencing unusually high network latency (see Theorem~\ref{thm:latency}). Here we have the option to wait for the delayed participant and take a latency hit but not let the fairness be impacted. Ideally, we want to let the system continue with low latency with only the affected participant incurring unfairness. In DBO, we use a simple strategy to mitigate this. Using the heartbeats and the generation time of data points, the OB tracks the round trip latency to each participant. If this latency goes beyond a certain threshold for a participant, then the OB does not wait for heartbeats from such straggler participant before forwarding trades. When the round trip latency goes down, OB again starts waiting for heartbeats from the straggler. In the event of crashes, OB might not hear any heartbeats. If the OB does not hear a heartbeat from a particular participant for the above threshold, then it concludes that round trip latency exceeds the threshold and the OB deems the participant a straggler. 
 
\noindent\textit{OB failure:} In the event, the OB crashes all trades in the priority queue will be lost. System will incur unfairness in such cases. 

%The above strategy is also helpful in controlling overall system latency when a certain participant is experiencing unusually high network latency.


\subsubsection{Is Batching and Pacing necessary?\\}
\textbf{Batching and pacing contribute delays; are they necessary?} The answer is yes. Similar to Lemma~\ref{lemma:inter_delivery_imp}, we can derive the necessary conditions for achieving LRTF. 
\begin{corollary}
When trigger points are unknown, the \textit{necessary} conditions on the delivery processes for achieving response time fairness with any ordering system is given by,
\vspace{-1mm}
\begin{align*}
    \text{If }  D(i,y) - D(i,x) &< \delta, \text{ then},\nonumber\\
    D(i, y) - D(i,x) &= D(i,y) - D(i,x), & \forall i,j.
\end{align*}
\label{cor:inter_delivery_lrtf}
\vspace{-6mm}
\end{corollary}

\begin{proof}
Please see Appendix~\ref{app:cor_inter_delivery_lrtf}.
\end{proof}
\vspace{-1mm}
In contrast to Lemma~\ref{lemma:inter_delivery_imp}, the above condition states that the inter-delivery time of two points should be same across all participants only if they are separated by less than $\delta$ for some participant. Batching and pacing indeed satisfies this, for two points x and y in a batch, the inter-delivery times across all participants is indeed zero and hence equal. For point $x$ and $y$ belonging to different batches, since the inter-delivery time is greater than $\delta$ across all participants, there is no additional contraint on inter-delivery times being equal.
 
\subsubsection{Impact of RB to MP latency\\}
In scenarios where RB and the participant cannot be colocated, DBO can incur unfairness. If this latency is unbounded, then, it might be impossible to achieve fairness. If latency is bounded, however, then DBO provides the following fairness guarantees.

\begin{theorem}
    If round trip network latency from release buffer $i$ to it's corresponding participant is bounded between $B_l(i)$ and $B_h(i)$, then, DBO achieves the following guarantee for ordering trades.
    \begin{align*}
    C3: &\text{ if } TP(i,a)= TP(j,b) = x\\ 
    &\land RT(i,a) < RT(j,b) - (B_h(i)-B_l(j)), \\
    & \land RT(i,a) < \delta - B_h(i),\\
    &\text{ then, }O(i,a) < O(j,b).
\end{align*}
    \label{thm:rb_to_mp_latency}
    \vspace{-5mm}
\end{theorem}

\vspace{-1mm}
\begin{proof}
See Appendix~\ref{app:rb_to_mp_latency}.
\end{proof}
\vspace{-1mm}

Compared to LRTF, the above condition reduces the bound on response time for the faster trade $(i,a)$ to $\delta - B_h(i)$.
Additionally, the above condition states that trades are ordered fairly only when the response time of the faster trade is lower than the response time of the competing trade by atleast the variability in latency ($B_h(i)-B_l(j)$). This theorem essentially states that when RB and MP cannot be colocated, for better fairness we should ensure that latency between them is both consistent (across participants) and the upper bound is small.



\subsubsection{Impact of Losses\\}

Although infrequent, packet losses can occur in cloud environments. Such losses can impact fairness in DBO. However, only the fairness for trades that are lost and trades  whose trigger point is lost is impacted (see Appendix~\ref{app:impact_losses}).



\if 0
\subsubsection{Excessive queing at RB and OB\\}
\pg{This can be cut?}

Even though DBO employs straggler mitigation to limit the latency at the OB, it can build up a large queue if it receives a very large number of trades (little's law). The RB can also overflow in scenarios where the network latency is decreasing (Figure~\ref{fig:latency_b+p}) for a large period of time. 

\noindent
\textbf{RB:} In the event a release buffer's queue fills up (exceeds a certain threshold), to avoid overflow the release buffer forgoes pacing and starts releasing data as fast as possible to reduce the queue. In such cases, the delivery clock advances faster than as dictated by pacing. As a result, trades from such a participant might unfairly get ordered behind. The fairness for trades from other participants remains unaffected. When the queue goes down the RB resumes normal operation.

\noindent
\textbf{OB overflow:} In the event the order buffer's queue fills up, the OB starts releasing trades as fast as possible without waiting for heartbeats from participants. Once the queue goes down, OB resumes normal operation. In such cases, fairness of all trades are impacted. 
\fi

\subsubsection{Thwarting front-running attacks\\}

%Monotonicity of delivery clocks ensures that participants are incentivized to submit trades as early as possible and delaying trades does not offer any competitive advantage.% and participants are incentivized to be honest.
There is a front-running attack possible in our system. In particular, if a participant receives a market data point $x$ through some other way before RB delivers the data point $x$ to the participant then the participant has a competitive advantage. This scenario (though unlikely) is still possible. 

A simple to avoid this is to limit that a participant cannot talk to anyone beyond the CES. 
%\pg{External participants}
However, we would like the participant machine to use other  ``helper'' machines in the cloud, e.g.,  to aid computation. We also want to allow the participants to be able to talk to machines outside the cloud, e.g., to get a news stream. %stream.%\footnote{Participants use external news streams update trading strategies and make trading decisions.} 

In Appendix~\ref{app:front_running}, we show how we can prevent such front running attacks. In our solution, the participant and its helpers cannot communicate with any other participants or their helpers using the cloud network. 
To prevent scenarios where a participant uses a proxy machine outside the cloud to send market data to other  participants (faster than the network), we precisely add additional latency for data being sent outside the cloud.
While our solution introduces latency for data going out, the latency of speed trades remains unaffected.

\if 0

While monotonicity of delivery clocks ensure that participants are incentivized to submit trades as early as possible an delaying trades does offer any competitive advantage, there is still a potential front-running attack possible in our system. In particular, if a participant receives a market data point $x$ through some other way before RB delivers the data point $x$ to the participant then it has a competitive advantage. This scenario though unlikely is still possible.
A simple to avoid this is to limit that participant cannot talk to anyone beyond the CES. 

However, we would like the participant machine to use other  ``helper'' machines in the cloud to aid computation. We also want to allow the participants to be able to talk to machines outside the cloud. Participants do use external news streams and feeds from other exchanges to update trading strategies and make trading decisions. We will discuss fairness with respect to such streams shortly.  

Allowing such communication naively can lead to attacks.
By restricting communication, it is possible to ensure that no participant gets early access to market data %(at the cost of introducing latency in messages from the front-end to helpers outside the cloud)
and thwart such front-running attacks. 

%
%\pg{Which of two alternatives is better?}
%
To this end, we impose two simple constraints on communication. \begin{enumerate*}[label=(\arabic*)]\item A participant machine and its helper machines can communicate with each other freely but they cannot communicate with any other machines in the cloud. This restriction can be imposed easily by cloud providers today using security groups. This restriction ensures that a participant machine cannot get market data from other participant machines in the cloud directly. Next, we will ensure that a participant machine cannot get an earlier market data feed from outside the cloud. 
We will do so by restricting that a participant can only send data point x out of the cloud, when x has been delivered to all participants in the cloud. This way, market data points can only be available outside the cloud once they have been delivered to all the participants.
\item The helper machines cannot send data outside the cloud. Any data (excluding the trade orders) from a participant being sent outside the cloud is tagged by the delivery clock at the RB and buffered at a gateway. The data sent by the participant could potentially be a market data point with id less than or equal to the last point id (first tuple) of the delivery clock time stamp. The gateway thus buffers this data until it is sure that the all data points with id less than the last data point id in the delivery clock time stamp have been delivered. For this purpose, RB's periodically communicate their delivery clock to the gateway. 
%
%A simple way to achieve this is for each RB to send other RBs periodic beacons communicating the status of its delivery clock. This way each RB can maintain a lower bound on the delivery clocks at other RBs. 
\end{enumerate*}
\pg{include this? a bit hand-wavy and not clean. There is one challenge to be solved though. If data delivery to a particular participant is straggling then the gateway buffer can get bloated. It is not necessary for the gateway to wait for such straggler if we disable the incoming data to the straggler. The gateway can identify such stragglers and then disable any data coming from outside the cloud.}

Note that the above solution adds additionaly latency for data being sent outside the cloud. However, the latency of speed trades remains unaffected.
%There are other ways to thwart front-running that impose weaker restrictions on communication or are easier to implement. We chose to present this one for its simplicity.


\fi



\subsubsection{Limtations of DBO: Fairness beyond LRTF\\}
\label{ss:beyond_fairness}

With DBO, it is not guaranteed that trades that do not directly follow the LRTF model (Theorem~\ref{thm:1} and Equation~\ref{eq:cm})are ordered fairly. However, DBO still ensures that fairness for the most latency-sensitive speed trades. While ensuring guaranteed fairness for trades that do not follow the might be impossible, we will discuss potential some solutions.


%This will impose some system challenges. Another challenge is that different participants might be requesting different external streams. 
%


\noindent\textbf{Trades with response time > $\delta$:} DBO does not provide any guarantees for trades with response time greater than $\delta$. %If the inter-delivery times for batches across participants are same then DBO provides response time fairness for such trades. Again achieving the same inter-delivery times for all the batches is impossible. 
In case we have access to synchronized clocks, we can try and ensure (to the extent possible) that batches are indeed delivered at the same time across participants. 
When batches are delivered simultaneously, delivery clocks also get synchronized and DBO simply orders trades in the order of submission time. DBO thus ensures better fairness for such trades (when data is delivered simultaneous) while always guaranteeing LRTF. %\pg{Is this clear?}


%Regardless of whether using clocksync or not for deliverying the data, the performance of DBO for such trades is comparable to 


\noindent\textbf{Generalized compute model for trades:} A trade's submission time might be governed by delivery times of multiple data points. Again in such cases if we have access to synchronized clocks, we can try and ensure simultaneous delivery to the extent possible and achieve better fairness for such trades.


\noindent\textbf{External data streams:} In theory, external data streams like news events or market data from a competing exchange can trigger speed races. While DBO does not delay delivery of such streams to the participants (Appendix~\ref{app:front_running}), as described it does not guarantee fairness with respect to such streams. Existing exchanges do not provide any simultaneous delivery guarantees with respect to such external streams. Such streams typically traverse the internet, and the variability is network latency is substantially higher (order of milliseconds) than the market data stream (order of microseconds). Potentially, the exchange can serialize such external streams with the market data stream and ensure LRTF with respect to such a super stream. Such a serialization might not be trivial. Participants are requesting different data streams. We need to think carefully about what constitutes a fair serialization.
%\pg{Talk about how  further system challenges.}


%\subsubsection{\pg{Miscellaneous, do if time:}}
%\pg {Radhika advidce here would be helpful}

%\pg{1. Impact of clock drift rate, 3. Is batching and pacing necessary 4. Discussion, sharding for scalability, a separate RB for each asset class}













\if 0

\subsubsection{Delivery Clock\\}
Each RB maintains a delivery clock. This delivery clock essentially tracks time relative to when market data was delivered to the participant. We use $DC(i,t)$ to represent deliver clock of participant $i$ at time $t$. Delivery clock is a lexicographical tuple.
\begin{align}
    DC(i,t) = \langle ld(i,t), t-D(i, ld(i,t))\rangle.
\end{align}
where $ld(i,t)$ is the latest data point that was delivered to MP$_i$ at time t.% (i.e., $D_i(x_l(t)) \leq t < D_i(x_l(t)+1)$). 
Interval, $t-D(i, ld(i,t))$, corresponds to the time that has elapsed since the last delivery and can be measured locally at the RB without requiring any clock synchronization (challenge 1). Delivery clock advance monotonically with time. This property will help us overcome challenge 3 and also guard us against certain attack. (\pg{forward pointers}). Figure~\ref{fig:delivery_clock} shows how delivery clock advances with time.

\begin{figure}[t]
\centering
    \includegraphics[width=0.8\columnwidth]{images/delivery_clock.jpg}
    \vspace{-2.5mm}
    \caption{\small{\bf Delivery Clock.} \pg{Redraw}}% \pg{Eashan see Ranveer's comment}}% \pg{Eashan can you redraw this figure in powerpoint or something.}}}
    \label{fig:delivery_clock}
    \vspace{-2.5mm}
\end{figure}

All incoming trades are market with the delivery clock at the trade submission time. The ordering buffer uses this delivery clock time to order trades. Formally, the ordering in DBO is given by,  

\begin{align}
    O(i,a) = DC(i, S(i,a)). 
    \label{eq:ordering_with_dc}
\end{align}


\begin{figure}[t]
\centering
    \includegraphics[trim={0 0 0 2mm},clip,width=0.9\columnwidth]{hotnets-images/time series visualization (3).pdf}
    \vspace{-3mm}
    \caption{\small{{\bf DBO can help correct for late delivery of data.} Delivery of market data to MP$_i$ is lagging behind MP$_j$. There are two trades $(i,a)$ and $(j,b)$ generated in response to the same market data $x$. $(j,l)$ was submitted before $(i,k)$ but
    %, i.e., $S_j(l) < A_i(k)$. 
    response time of $(i,k)$ is less than $(j,l)$.
    %, i.e., $rt_i(k) < rt_j(l)$. 
    With DBO, $O(i,a) (= \langle x, RT(i,a)\rangle) < O(j,b) (= \langle x, RT(j,b)\rangle)$ and trade $(i,a)$ is correctly ordered ahead of $(j,b)$.} %Ordering based on the submission time leads to incorrect ordering.}
    \pg{Correct figure}}
    \label{fig:dbo_correction}
    \vspace{-4mm}
\end{figure}


When the trigger point of trade $(i,a)$ is indeed the last data point (i.e., $x = TP(i,a) = ld(i, S(i,a))$), then, DBO respects condition C2 for LRTF. Figure~\ref{fig:dbo_correction} shows an illustrative example of this.
This is because $O(i,a) = DC(i, S(i,a)) = \langle x, RT(i,a)\rangle$. For, a competing trade $(j,b)$ with higher response time, the delivery clock at time of submission will either read $O(j,b) = DC(j, S(j,b)) = \langle x, RT(j,b)\rangle$ (if D(j,x+1)>S(j,b)) or $DC(j, S(j,b) = \langle y, S(j,b)-D(j,y)\rangle$ with $y>x$. In both cases, $O(i,a) < O(j,b)$.


\noindent
\t
At a high level, in our ordering we are correcting for latency differences in data delivery by using the delivery time of the last data point. When the last data point is not the trigger point for trade $(i,a)$, DBO satisfies the LRTF condition C2, if the following condition holds, 
\begin{align}
    D(i,ld(i,t))-D(i,x) = D(j,ld(i,t))-D(j,x),
    \label{eq:cond_delivery_lrtf}
\end{align}
where $x = TP(i,a)$.  
While it is impossible to ensure that inter-delivery times remain the same for all participants for all points, by pacing data at the RB it is indeed possible to ensure that the above condition is always met. 
The main reason why we can do so is thaat condition C2 limits that the trigger point $x$ cannot be any arbitrary data point in the past ($S(i,a)-D(i,x) < \delta$).
%and we only need to ensure same inter-delivery times for. 
In the next subsection, we will show how we can achieve this and solve challenge 2. \pg{Is this easy to follow?}

\pg{Should we include results on necessary conditions on delivery times for achieving LRTF}

\noindent
\textit{Remark:} In our cloud experiments, we find that DBO achieves fairness with very high probability. This is because network latency (from CES to any given participant) exhibits temporal correlation in latency especially over  short periods of time. When temporal correlation is high, inter-delivery time at any participant is close to the inter-generation time at the CES. In such cases, condition given by Equation~\ref{eq:cond_delivery_lrtf} is satisfied with high probability.

\begin{figure}[t]
\centering
    \includegraphics[width=0.8\columnwidth]{images/batching_pacing.jpg}
    \vspace{-2.5mm}
    \caption{\small{\bf Batching and Pacing.} \pg{Redraw}}% \pg{Eashan see Ranveer's comment}}% \pg{Eashan can you redraw this figure in powerpoint or something.}}}
    \label{fig:batching_pacing}
    \vspace{-2.5mm}
\end{figure}

\subsubsection{Batching and Pacing\\}
In DBO, the CES breaks data into batches. Each new batch contains all data points in the duration $(1+\kappa) \cdot \delta$ after the previous batch. Here $\kappa > 0$. Each release buffer delivers all data points in a batch at the same time. %Two points $x,y$ belonging to the same batch are delivered simultaneously to each participant, i.e., $D(j,y)=D(j,x), \forall j$.
The release buffer delivers batches as quickly as possible while ensuring that the time between delivery of two consecutive batches is atleast $\delta$. Figure~\ref{fig:batching_pacing} shows an illustration of batching. Both batching and pacing increase the delivery time of data points. In the next subsection we will analyze the impact of the two on latency. Note that since $\kappa > 0$ batch generation rate is slower than batch drain rate and build up queue because of pacing will eventually get drained. 



With batching and pacing, DBO achieves LRTF. In particular, 
consider a trade $(i,a)$ with response time less than $\delta$. Because of pacing, batches are separated by $\delta$. This means that the trigger point ($x=TP(i,a)$) must be within the last received batch. The point $ld(i,S(i,a))$ is also the last point in this batch and $D(i,ld(i,S(i,a)) = D(i,x)$. $O(i,a) = DC(i,S(i,a)) = <ld(i,S(i,a)), RT(i,a)>$.
With batching, for participant $j$, $x$ and $ld(i,S(i,a))$ also belong to the same batch $D(j,ld(i,S(i,a)) = D(j,x)$.
For, a competing trade $(j,b)$ with higher response time, the delivery clock at the time of submission will either read $O(j,b) = DC(j, S(j,b)) = \langle ld(i,S(i,a)), RT(j,b)\rangle$ (if $(j,b)$ was submitted before the next batch, i.e., $D(j,ld(i,S(i,a))+1) > S(j,b)$,) or $DC(j, S(j,b) = \langle y, S(j,b)-D(j,y)\rangle$ with $y>ld(i,S(i,a))$. In both cases, $O(i,a) < O(j,b)$.

\fi

\if 0
\subsection{Compute Model of the HFT Trader and Definition of Fairness}

\begin{enumerate}
    \item $MD_R(i, x):$ Receive time of market data at the gateway/RBi
    \item $TO_G(i, a):$ Generation time of trade order a by trader i
    \item $TP(i,a):$ Trigger/stimuli for trade (i,a)
    \item $RT(i,a):$ Response time of for trade (i,a) 
\end{enumerate}


\textbf{Compute Model:}
Time of generation of trade= time participant received the market point that triggered the trade + response time (or time it took to generate the trade)
\begin{equation}
    TO_G(i,a) = MD_R(i,TP(i,a)) + RT(i,a)
\end{equation}


\textbf{Perceived Fairness with respect to participant i}
If all other participants received the market data at the same time as i, then how should the trades be ordered
\begin{align*}
    \text{Trade (i,a) should be ordered ahead if}\\
    TO_G(i,a) &< MD_R(i,y) + RT(j,b)\\
    TO_G(i,a) - MD_R(i,y) &< TO_G(j,b) - MD_R(j,y)
\end{align*}
This definition states for two orders trades we need to measure time relative to event y

alternatively what if i goes into j's time domain
\begin{align*}
    &\text{Trade (i,a) should be ordered ahead iff O(i,a)<O(j,b)}\\
    MD_R(j,x) + RT(i,a) &< TO_G(j,b)\\
    TO_G(i,a)-MD_R(i,x) &< TO_G(j,b) - MD_R(j,x)
\end{align*}

Correction, relative ordering




\textbf{Achieving fairness}
There are two challenges,
\begin{outline}
    \1 How do you decide how to order these trades when TP y is unknown. \pg{Three options 1) Delivery Clocks 2) Equal RTT 3) Directly to limited fairness} \pg{Time domain: two options a) I's domain b) zero latency time doman. Fairness for trades using different data points.}
        \2 Don't know which x, recency \pg{equivalence between equal inter-delivery and correcting one way latency}
        \2 Clocks are not synced
        \2 Monotonic ordering with time
    \1 How do you enforce the ordering process. In particular, trades may take an arbitrary amount of time to reach the OB.
\end{outline}

What is the lowest RTT possible with this system?\\
Say you knew the trigger points x,y what then, \\
Say you didn't know the trigger points\\
Enforcing the ordering: key insight Enforcing an ordering at a single point is easier than controlling things at multiple RBs\\
What about trades with response time greater than delta\\


Question: Fairness wrt to external data stream

\textbf{Practical Considerations}

\begin{enumerate}
    \item Collusion attacks: Ensure that any market data point is delivered only after all participants have received it
    \item external participants: Have all participants submit trade via a dummy MP machine (we dont support fairness for such particpants)
    \item External data streams:
    \item Stragglers: 
\end{enumerate}


\textit{Correction by latency pitch}
\begin{align*}
    TO_G(i,a) - MD_R(i,y) &< TO_G(j,b) - MD_R(j,y)\\
    TO_G(i,a) - (G(y) - MD_R(i,y))) &< TO_G(j,b) +(G(y)- MD_R(j,y))
\end{align*}

\pg{Alternatively fairness in the same or equal or zero latency time domain?}
\begin{align*}
    &\text{Trade (i,a) should be ordered ahead iff O(i,a)<O(j,b)}\\
    G(x) + RT(i,a) &< G(y) + RT(j,b))\\
    TO_G(i,a) + (G(x)-MD_R(i,x)) &< TO_G(j,b) + (G(y) - MD_R(j,y))
\end{align*}


\textbf{Final Pitch Attempt}
\begin{enumerate}
    \item Introduce generalized compute model
    \item Talk about zero latency model for fairness. Three problems clocksync, which x to use, how to enforce ordering. \pg{Introduce C1 from strong fairness here?}
    \item clocksync: We are interested in competing trades that are generated using the same data point \pg{is clocksync really necessary to force this}
    \item which x to use: the last x since trades are fast. What about latency for trades with response time greater than delta
    \item how to enforce ordering: monotonic ordering process \pg{unclear if monotonic is time property is even needed (if )} 
    \item part of above? No fooling: C1 property of strong fairness
    \item \pg{Limitations: Our solution doesn't work with this model for trades generated using different data points. What about approx fairness? This is kind of nice because it talks about latency/}
\end{enumerate}
\fi
\subsection{Evaluation procedure}
\label{sec_validation}
\newcommand{\iterDayCmp}{j}
\newcommand{\iterDayExp}{i}
\newcommand{\iterDay}{i}
The objective of the evaluation procedure is to answer whether the new price- and forecast-aware controller saves money when compared to the existing benchmark controller described in Section \ref{sec_test_house}. The key performance indicator is daily cost given the weather conditions. It is inherently difficult to benchmark and validate the performance of a controller operating in a complex environment with many uncontrollable external factors such as weather and occupant activities. Further, the long time-constants play a significant role by demanding long test periods. Ideally, the benchmark  and \MPC-controller should be run in parallel on exact copies of the same building placed at the same location, with occupants doing the same activities. Although, some buildings support such circumstances, this can obviously not be asked of the occupants. Instead, a benchmark \dataSet\ from the same house is used for the evaluation. The benchmark \dataSet\ is based on data collected from the former heating period (2021-2022) where the original benchmark controller was operating. The data is sorted into full days creating a collection of comparison days $\daySetCmp$, seen in \eqref{eq_days}, from which appropriate subsets can be selected. The daily generated data on set form is:
% \begin{align}
%     \label{eq_days}
%     \daily^i = \{\Eg^\iterDay, \Ta^\iterDay, \priceElecBuySeti{\iterDay}, \Epv^\iterDay \in \realv{\Nday}\}, \hspace{0.1cm} \daily^i \in \mathcal{D}
% \end{align}
\newcommand{\iterDayOne}{n}
\begin{align}
    \label{eq_days}
    &\mathcal{D}_\compare = \left\{day^\iterDayOne = \left(\Eg^\iterDayOne, \Ta^\iterDayOne, \priceElecBuySeti{\iterDayOne},\Epv^\iterDayOne  \right)\right\}\\
    &\iterDayOne = 1,\dots,\Ncmp, \hspace{4mm} \Eg^\iterDayOne, \Ta^\iterDayOne, \priceElecBuySeti{\iterDayOne},\Epv^\iterDayOne \in \realv{\Nday}
    %\daily^i = \{\Eg^\iterDay, \Ta^\iterDay, \priceElecBuySeti{\iterDay}, \Epv^\iterDay \in \realv{\Nday}\}, \hspace{0.1cm} \daily^i \in \mathcal{D}
\end{align}
with $\Eg^\iterDay$ and $\Epv^\iterDay$ being the electricity consumption from grid and production from \pv\ in \si{\kWh} during day $i$, respectively, $\Ta^\iterDay$ the ambient temperature, $\priceElecBuySeti{\iterDay}$ the hourly electricity price for day $i$, and $\Nday = 24$. Note that benchmark days where the system has been manipulated or a significant amount of data is missing are dropped to minimise pollution of the results. A similar data collection, $\daySetExp$, is generated from the experiment period. %To evaluate daily controller performance a subset of days, $\daySetCmp^\iterDayExp \subset \daySetCmp$, with similar average ambient temperature and sun irradiation are drawn from the comparison collection $\daySetCmp$ for each experiment day $\iterDayExp$:
The \MPC-controller is evaluated daily by comparing the operation cost of day $\iterDayExp$ to a subset of benchmark days, $\daySetCmp^\iterDayExp \subset \daySetCmp$, drawn from the full benchmark \dataSet. The subset, $\daySetCmp^\iterDayExp$, is drawn according to the following rule:
% \begin{align}
%     \label{eq_subset_cmp_days}
%     \mathcal{D}_{\compare}^\iterDayExp &= \{day^\iterDayCmp \hspace{1mm}\vert\hspace{1mm}  \nonumber\\ &-\Delta\avgTai{\dn} \leq \avgTaCmp^{\iterDayCmp} - \avgTaExp^\iterDayExp \leq \Delta\avgTai{\up}, \nonumber\\
%     &-\Delta\Epvi{\dn}  \leq \Sigma \EpvCmp^{\iterDayCmp}-\Sigma\EpvExp^\iterDayExp \leq \Delta\Epvi{\up} \nonumber \\
%     &day^\iterDayCmp \in \daySetCmp \}
% \end{align}
\begin{align}
    \label{eq_subset_cmp_days}
    \mathcal{D}_{\compare}^\iterDayExp &= \{day \hspace{1mm}\vert\hspace{1mm}  \nonumber\\ &-\Delta\avgTai{\dn} \leq \avgTaCmp - \avgTaExp^\iterDayExp \leq \Delta\avgTai{\up}, \nonumber\\
    &-\Delta\Epvi{\dn}  \leq \Sigma \EpvCmp-\Sigma\EpvExp^\iterDayExp \leq \Delta\Epvi{\up}, \nonumber \\
    & \avgTaCmp,\Sigma \EpvCmp \in day \in \daySetCmp \}
\end{align}
with $\avgTa$, $\Sigma \Epv$ being average ambient temperature and accumulated electricity production from \pv, respectively. The constants $\Delta\avgTai{\dn}$ and $\Delta\avgTai{\up}$ are the down- and up-search range for ambient temperature, respectively. Similar, $\Delta\Epvi{\dn}$, $\Delta\Epvi{\up}$ makes out the search-range for accumulated electricity produced by the \pv. Here the \pv\ is used as an indicator for sun radiation. This is not a perfect indicator, since the sun altitude and intensity vary with the seasons, thereby creating a bias. However, it is found to be a good indicator for dealing with cloud conditions on-site, since it directly measures the level of shadow on the building. With ambient temperature and sun irradiation accounted for, factors such as occupant behavior and previous day heating patterns are left out. This undeniably causes noise, making the electricity consumption of the \hp\ distribute randomly for any given day. To decrease the influence of the noise, the controller is run over a long period to obtain more consistent results.

We calculate a virtual cost for benchmark day $\iterDayCmp$, with respect to experiment day $\iterDayExp$,
\begin{align}
    \label{eq_cost_comp}
    &\costElecCmp^\iterDayCmp = \sum_{k = 0}^{\Nday} \priceElecBuySeti{\iterDay}(k)\Eg^\iterDayCmp(k)\nonumber \\  &\priceElecBuySeti{\iterDay} \in \dailyExp^\iterDayExp, \hspace{0.4cm} \Eg^\iterDayCmp \in \dailyCmp^\iterDayCmp \in \daySetCmp^\iterDayExp
\end{align}
% \begin{align}
%     \label{eq_cost_comp}
%     \costElecCmp^\iterDayCmp = &\sum_{k = 0}^{\Nday-1} \priceElecBuySeti{\iterDay}(k)\Eg(k) \nonumber \\ &  \priceElecBuySeti{\iterDay} \in \dailyExp^\iterDayExp,\hspace{0.1cm} \Eg \in \dailyCmp \in \daySetCmp^\iterDayExp
% \end{align}
It simply means that electricity consumption from similar benchmark days are imposed onto the price of the experiment day to calculate the virtual cost. This provides a plausible alternate outcome for the case where the benchmark controller had been running instead. This is done since the benchmark controller is price ignorant and thereby acts independently of the price. This manoeuvre would not be possible if the comparison was between two price-aware controllers. In that case price curves would have to be accounted for as well. The cost of the experiment day $i$, $\costElecExp^\iterDayExp$, is of course calculated using the actual electricity consumption for the day.
\section{Methodology}
\label{sec:simMethod}

For all case studies except real workloads under full system simulation, we used \gem{}'s traffic generators instead of execution-based workloads.
Using traffic generators allows us to explore the behavior of the DRAM cache design more directly and more clearly understand the fundamental trade-offs in its design.
We use \gem{}'s traffic generator to generate two patterns of addresses: linear or random.
If not specified, the tests are run with a linear traffic generator.

% In our studies, where a full system simulation was not needed, 
% we considered UDCC receives 
% the requests from a traffic generator included in \gem{}, 
% known as PyTrafficGen. gem5's PyTrafficGen is 
% implemented as a simobject and can be used as a data requestor component. 
% It creates a syntehtic traffic which can be fed to a memory subsystem such as UDCC. 
% PyTrafficGen's output can be either probabilistic or trace-based. In the 
% trace-based mode, it uses a memory trace to produce the requests. In 
% the probabilistic mode it takes a list of parameteres which describe the 
% characteristics of its output pattern. %Table \ref{tab:traffGenParam} describes these parameteres.
% These parameters are listed below:
% \begin{itemize}
%   \item Duration: The duration of producing requests (in ps).
%   \item Start Address:  The lower bound of produced addresses.
%   \item End Address: The upper bound of produced addresses.
%   \item Minimum Period: The shortest timing gap between two consecutive requests.
%   \item Maximum Period: The longest timing gap between two consecutive requests.
%   \item Request Size: The size of read/written data by each request (in bytes).
%   \item Read Percentage: The percentage of reads among all the requests (the rest of requests are writes).
% \end{itemize}

To implement the DRAM cache we used the DRAM models (``interfaces'') provided by gem5.
We also used the NVM interface provided in gem5 with its default configuration unless otherwise specified (timing parameters are shown in the `Base' column in Table~\ref{tab:nvmtimes}).
%\note{What is the default NVM config? Maybe point forward to the table later?}
We extended \gem to model DDR5 (based on DDR5 data-sheets) and HBM (based on a performance comparison) for our studies.

In the case studies presented below, we are not concerned with the detailed configuration of the \textit{UDCC} (e.g., the size of DRAM cache). 
Instead, we are interested in the study of the behavior of the \textit{UDCC} through specific scenarios 
which enables us to evaluate the best, worst, or in between performances of the system. For this purpose, we have used 
patterns which are either Read-Only (RO), Write-Only (WO), or a combination of reads and writes (70\% reads 30\% writes). 
% For the latter we used a pattern of 70\% read and 30\% writes. 
% These patterns were enforced using the 
% Read Percentage parameter of the traffic generator. 
The other traffic pattern characteristic we varied is the hit (or miss) ratio of the generated pattern. 
This factor was enforced by two different parameters (1) the size of the DRAM cache and (2) the range of addresses requested by the traffic 
generator. In all of the case studies in which the traffic generator was involved, we used a DRAM cache of 16 MB size backed-up 
by NVRAM as the main memory. Unless otherwise specified, we used 
DDR4 to implement the DRAM cache in \textit{UDCC} with a total
buffer size of 256 entries. In order to get 0\%, 
25\%, 50\%, 75\%, and 100\% hit ratios, we set the range of addresses to be 6GB, 64MB, 32MB, 20MB, and 6MB. 
% In this way we were able 
%to test the UDCC for a combination of RO, WO, or 70\% read and 30\% writes with the aforementioned hit ratios. 
For instance, to study 
the behavior of a write intensive application with a memory foot print larger than the DRAM cache capacity on our proposed model, 
we set the traffic generator to give WO accesses within the range of 6GB for a DRAM cache of 16MB capacity. 
In this way, we were able to test the \textit{UDCC} in a reasonable 
simulation time. 
We used a cache block size of 64B in all tests to match current systems.
All the tests were simulated for 1 second. 
%% should I say any margine for the hit ratio err?



% \begin{center}
%   \begin{tabular}{ | m{2.5cm} | m{5cm} | }
%     %\caption gem5's PyTrafficGen parameteres in probabilistic mode} \\
%     \hline
%      Duration & The duration of producing requests (in ps) \\ 
%     \hline
%      Start Address &  The lower bound of produced addresses \\ 
%     \hline
%      End Address & The upper bound of produced addresses \\ 
%      \hline
%      Minimum Period & The shortest timing gap between two consecutive requests \\ 
%      \hline
%      Maximum Period & The longest timing gap between two consecutive requests \\ 
%      \hline
%      Request Size & The size of read/written data by each request (in bytes) \\ 
%      \hline
%      Read Percentage & The percentage of reads among all the requests (the rest of requests
%      are write requests) \\ 
%     \hline
%   \end{tabular}
%   \label{tab:traffGenParam}
%   \end{center}

% \section{Case Studies}
% \label{sec:case_studies}


\section{Case Study 1: Impact of Scheduling Policy on DRAM Caches Performance}
\label{sec:sched}

\subsection{Background and Methodology}

The policy to choose which memory request to service has a significant impact on the performance of the DRAM memory system.
% Different scheduling policies for memory subsystems have been proposed to provide fairness and high quality of service.
In this work we consider two scheduling policies: (i) first-come, first-serve (\textit{FCFS}), and (ii) first-ready, first-come, first-serve (\textit{FRFCFS}).
\textit{FCFS} is the naive policy which processes the memory requests in the order they are received by the memory controller.
%\textit{FCFS} can be beneficial where the demand's addresses are uniform and are well-distributed across the banks, taking advantage of bank-level parallelism.
While it is simple and easy to implement, \textit{FCFS} adds many row-switching delay penalties, leading to lower bus utilization.
Rixner et al.~\cite{rixner2000memory} proposed \textit{FRFCFS} trying to take advantage of maximizing row-buffer hit rate. \textit{FRFCFS} reorders
the received requests so the ones that hit on the currently-opened row would be serviced earlier than any other requests which map to the other rows that are currently closed.
%  switching penalty as much as possible. Amongst all the requests that hit on the currently-opened row, whichever was received first,
% would be serviced first. In case of no request's address mapped to the currently-opened row, it will switch to another row and pay the switching delay penalty.
% Generally, \textit{FCFS} is easy to implement as it requires a simpler logical circuit than \textit{FRFCFS}, and it's a default scheduling policy leveraged in some CPU vendors' products.
In heterogeneous memory systems, Wang et al.~\cite{wang2020characterizing} reported that Intel's Cascade Lake's NVRAM interface deploys \textit{FCFS}.
The question that arises here is, in a memory system with two cooperative devices, one as DRAM cache and the other as main memory, with several internal requests buffers to arbitrate from, how important is the choice of scheduling policy?
% Can \textit{FRFCFS} be a better choice in such memory systems than \textit{FCFS}?

We extended \textit{UDCC} with \textit{FCFS} and \textit{FRFCFS} scheduling policies to answer this question.
\textit{UDCC} employs the same scheduling policy for both DRAM cache and main memory.
We tested \textit{UDCC} with each
scheduling policy to measure the improvement of bandwidth observed by LLC with \textit{FRFCFS} over \textit{FCFS}.
We have run tests with different hit ratios and different read and write request combinations to test
the sensitivity of bus utilization and the system performance to the scheduling policy.

% We configured \textit{UDCC} with a total
% buffer size of 256 entries, a 16MB DDR4 DRAM cache, and NVRAM as the main memory.
%\note{Why is it OK to only use 16 MB???}
The results are based on request patterns of 0\% hit ratio and 100\% hit ratio for RO and WO.
We also ran patterns containing 70\% read and 30\% write requests, with 100\%, 75\%, 50\%, 25\% and 0\% hit ratios.

\subsection{Results and Discussions}

\begin{figure}
    \centering
    \includegraphics[scale=0.6]{figures/woro-bw.pdf}
    \vspace{-1ex}
    \caption{Bandwidth seen by LLC (GB/s). \textit{UDCC} has been tested with a total buffer size of 256 entries, 
    DDR4 DRAM cache, and NVRAM main memory for \textit{FCFS} and
     \textit{FRFCFS} scheduling policies. On the X-axis read-only (RO) and write-only (WO) patterns with 0\% and 100\% hit ratios are
     shown for the two scheduling policies.}
    \label{fig:woroBW}
  \end{figure}

  \begin{figure}
    \centering
    \includegraphics[scale=0.6]{figures/r70-bw.pdf}
    \vspace{-1ex}
    \caption{Bandwidth seen by LLC (GB/s). \textit{UDCC} has been tested with a total buffer size of 256 entries, 
    DDR4 DRAM cache, and NVRAM main memory for \textit{FCFS} and \textit{FRFCFS} scheduling policies. \textit{UDCC} has been fed by 70\% read requests and 30\% write requests.
    As shown on the X-axis, different hit ratios (0\%, 25\%, 50\%, 75\%, 100\%) have been applied to the request patterns.}
    \label{fig:r70BW}
  \end{figure}

%\begin{figure}
%  \subfloat[\scriptsize{DRAM}]{
%    \includegraphics[width=.45\linewidth, scale=0.5]{figures/woro-busUtil-dram.pdf}
%    \label{fig:dram1}
%  }
%   \subfloat[\scriptsize{NVM}]{
%    \includegraphics[width=.45\linewidth, scale=0.5]{figures/woro-busUtil-nvm.pdf}
%    \label{fig:nvm1}
%  }
%   \caption{Total bus utilization percentage. \textit{UDCC} has been tested with an outstanding
%   request buffer size of 256 entries, 16MB DDR4 DRAM cache, and NVRAM main memory for \textit{FCFS}
%   and \textit{FRFCFS} scheduling policies. On the X-axis Read-Only and Write-Only patterns
%    with different hit ratios (0\%, 100\%) are shown for the two scheduling policies.}
%   \label{fig:woroBusUtil}
%  \end{figure}

\begin{figure}
  \subfloat[\scriptsize{DRAM}]{
    \includegraphics[width=.45\linewidth, scale=0.5]{figures/r70-busUtil-dram.pdf}
    \label{fig:dram}
  }
   \subfloat[\scriptsize{NVRAM}]{
    \includegraphics[width=.45\linewidth, scale=0.5]{figures/r70-busUtil-nvm.pdf}
    \label{fig:nvm}
  }
    \caption{Bus utilization percentage of DRAM and NVRAM. \textit{UDCC} has been tested with a total buffer size of 256 entries, 
    DDR4 DRAM cache, and NVRAM main memory for \textit{FCFS} and \textit{FRFCFS} scheduling policies. \textit{UDCC} has been fed by 70\% read requests and 30\% write requests.
     As shown on the X-axis, different hit ratios (0\%, 25\%, 50\%, 75\%, 100\%) have been applied to the request patterns.}
    \label{fig:r70BusUtil}
  \end{figure}

% Why is this indented???

Figure \ref{fig:woroBW} compares the observed bandwidth at the LLC, while running requests patterns which are RO and WO with 100\% and 0\% hit ratio DRAM cache.
For RO 100\% hit ratio and WO 100\% hit ratio, \textit{FRFCFS} achieved higher bandwidth than \textit{FCFS} by
2.56x and 2.85x, respectively. For RO 0\% hit ratio and WO 0\% hit ratio the observed bandwidth with \textit{FRFCFS} compared to \textit{FCFS} are
1.65x and 1x.

From this data, we conclude that the scheduling policy is more important for workloads which are more likely to saturate the DRAM bandwidth and are not limited by NVRAM bandwidth.
For instance, in the WO with 0\% hit ratio case, the performance is completely limited by the NVRAM device bandwidth and the scheduling policy does not affect the performance at all.
  % Overall, \textit{FRFCFS} can give up to 2.85x bandwidth improvement over \textit{FCFS} for DRAM
  % cache (i.e., 2.85x for WO with 100\% hit ratio).

  To look at a slightly more realistic situation, Figure~\ref{fig:r70BW} shows the observed bandwidth by LLC while running a pattern consisting of 70\% read, and
  30\% write accesses and the hit ratio increases on the X-axis with 25\% steps. As the hit ratio increases, the observed bandwidth increases
  for both \textit{FRFCFS} and \textit{FCFS} because more hits results in more DRAM accesses which is higher bandwidth.
  %The reason is, in
  %order to handle the misses, NVRAM reads (to fetch any missing lines from NVRAM) or writes
  %(to write any evicted dirty lines of DRAM cache to NVRAM)
  % accesses are required. So, a single request by LLC may turn into multiple memory accesses and bus transactions within the memory controller,
  % exacerbating the overall observed performance of the system. Thus, the fewer misses on DRAM cache (higher hit ratio) happens, the higher the
  % performance becomes.
  Figure~\ref{fig:r70BW} also shows that as the hit ratio increases, the improvement of bandwidth by \textit{FRFCFS} over \textit{FCFS}, increases.
  % Overall, \textit{FRFCFS} gives a higher bandwidth than \textit{FCFS} by 1.08x, 1.15x, 1.31x, 2.17x, and
  % 2.69x for the 0\%, 25\%, 50\%, 75\%, 100\% hit ratios, respectively.
  % Ultimately, \textit{FRFCFS} has improved bandwidth up to 169\%
  % over \textit{FCFS} for DRAM cache for an access pattern consisting of 70\% read, and 30\% write (i.e., 2.69x for a hit ratio of 100\%).

% \note{Below: Have 1 short paragraph about how FRFCFS improves bus utilization significantly for DRAM, then one short paragraph about how it improves bus utilization less for NVRAM. Thus, when miss rates are high (more NVM than DRAM accesses) the policy matters less.}

To better understand the performance improvement of the \textit{FRFCFC} policy 
Figure~\ref{fig:r70BusUtil} shows 
the bus utilization percentage for 70\% read 
and 30\% write case, for both DRAM and NVRAM devices.
%In Figure~\ref{fig:dram}, DRAM bus utilization of \textit{FRFCFS} is higher. 
% than \textit{FCFS} by 2.56x, 2.85x, and 1.65x for RO 100\% hit ratio, WO 100\% hit ratio,
  % and RO 0\% hit ratio, respectively. 
% For the cases of WO with 0\% hit ratio, DRAM bus utilization of \textit{FRFCFS} compared to \textit{FCFS} is
%   only slightly higher by 0.13\%. 
% In this case, there's an increase in the number of dirty lines evicted from DRAM cache needed to
%   be written back to the NVRAM.
% \note{I don't understand this last part...}
% \note{Also, we could drop Figure~\ref{fig:woroBusUtil}}

% In Figure \ref{fig:nvm1}, NVRAM bus utilization of RO 100\% hit ratio and WO 100\% hit ratio,
% are zero as there are no misses to be handled through NVRAM.
% For the cases of RO with 0\% hit ratio, NVRAM
  % bus utilization of \textit{FRFCFS} compared to \textit{FCFS} is
  % higher by 1.65x. For the cases of WO with 0\% hit ratio, NVRAM bus utilization of \textit{FRFCFS} compared to \textit{FCFS} is
  % slightly higher by 0.13\%. In this case, there's an increase in the number of dirty lines evicted from DRAM cache needed to
  % be written back to the NVRAM. %%%% AYAZ
  In Figure \ref{fig:dram}, 
  DRAM bus utilization of \textit{FRFCFS} is
  higher than FCFS in all hit ratios. 
  % than \textit{FCFS} by 1.08x, 1.15x, 1.31x, 2.17, and 2.69 for
  % 0\%, 25\%, 50\%, 75\%, 100\% hit ratios, respectively.
  Besides, as the hit ratio increases, the improvement of
  DRAM bus utilization by \textit{FRFCFS} compared to \textit{FCFS} also increases.
  In Figure \ref{fig:nvm}, the NVRAM bus utilization decreases for both \textit{FRFCFS} and \textit{FCFS}, as the hit ratio increases 
  since there are less misses to be handled by NVRAM.
  % than \textit{FCFS} by 1.08x, 1.15x, 1.31x, and 2.16 for
  % 0\%, 25\%, 50\%, 75\% hit ratios, respectively. 
  % Since there's no misses needed to be handled through NVRAM
  % for the case of 100\% hit ratio, the NVRAM bus utilization is zero for this case.
  Overall, DRAM interface bus utilization has benefited more from \textit{FRFCFS} (compared to \textit{FCFS}), than NVRAM. Moreover,
  where the hit ratio is higher, the improvement of bus utilization by \textit{FRFCFS} over \textit{FCFS} is higher.

  Based on the proposed DRAM cache model, there are several internal buffers at the controller per interface for reads and writes. The fair
  arbitration of requests in each of these buffers can highly impact the utilization of the resources, affecting the performance of the DRAM
  cache. 
  % In this section we demonstrated the impact of scheduling policy from two different points. First, we discussed the
  %effect of scheduling policy on bandwidth seen by LLC, as an UDCC-external effect. 
  We showed
  \textit{FRFCFS} can achieve up-to 2.85x bandwidth improvement over \textit{FCFS} (in WO with 100\% hit ratio) and that the improved DRAM bus utilization is the main factor contributing to this improvement.
  % Second, we discussed
  % the effect of scheduling policy on bus utilizations of DRAM and NVRAM devices, as an UDCC-external effect.
  % We showed bus utilizations improvement of \textit{FRFCFS} over \textit{FCFS} can go up-to 185\% (in WO with 100\% hit ratio) for DRAM and 116\% (for the case of
  % 70\% read and 30\% writes with 75\% hit ratio).
  % \textit{UDCC} with \textit{FRFCFS} can achieve up to 2.69x higher bandwidth compared to \textit{FCFS}.
  % In none of the test cases, \textit{UDCC} with \textit{FRFCFS} was outperformed by \textit{FCFS}.
  Overall, even though implementation of \textit{FRFCFS} would be more
  costly than \textit{FCFS} due to more complex and associative circuits, the performance gap between these two policies is significant.
  % on a system containing a DRAM cache with
  % multiple buffers to arbitrate requests from is not negligible. % In the end,
  % our simulation results suggest \textit{FRFCFS} can be more beneficial than \textit{FCFS} for DRAM caches.

%%%%%%%%%%%%%%%%%%%%%%%%%%%%%%%%%%%%%%%%%%%%%%%%%%%%%%%%%%%%%%%%%%%%%%%%%%%%%%%%

\section{Case Study 2: Performance Analysis of Different DRAM Technologies as DRAM Cache}

\begin{figure}
  \subfloat[\scriptsize{Maximum bandwidth achieved}]{
    \includegraphics[width=.48\linewidth, scale=0.5]{figures/bws.pdf}
    \label{fig:bws}
  }
  \subfloat[\scriptsize{Buffer size for maximum bandwidth}]{
    \includegraphics[width=.48\linewidth, scale=0.5]{figures/orb.pdf}
    \label{fig:orb}
  }
    \caption{Buffer size needed to achieve the maximum bandwidth seen by LLC for 
    \textbf{DDR3}, \textbf{DDR4}, and \textbf{DDR5} as DRAM cache. The traffic patterns 
    shown in the figure include read-only (RO) and write-only (WO), 
    each having 100\% and 0\% hit ratio.
    }
    \label{fig:orb_bw}
  \end{figure}


\subsection{Background and Methodology}

Different commercial products have adapted different DRAM technologies for their DRAM caches (e.g., HBM in Sapphire Rapids and DDR4 in Cascade Lake).
%  For instance, Intel's Knights Landing
% and their newest product Sapphire Rapids have used HBM, and Cascade Lake has used DDR4, as their DRAM cache.
These technologies have different characteristics (e.g., peak bandwidth) that gives different applicability to each.
In this
study we want to answer these questions: how many total buffers are required for each DRAM technology to fully utilize it as DRAM cache?
And, what is the peak performance of each device when the hit ratio and the percentage of read-write in access patterns change?

To address these questions, we configured \textit{UDCC} to use DDR3, DDR4, DDR5 and HBM models implemented for \gem, as DRAM cache.
The theoretical peak bandwidth of DDR3, DDR4, DDR5 and HBM are 12.8 GB/s, 19.2 GB/s, 33.6 GB/s, and 256 GB/s, respectively.
% We use a traffic generator sending synthetic traffic patterns to the \textit{UDCC} to model requests sent by the LLC.
% Like the previous case study, these patterns were 
The results are based on request patterns of 
RO 100\% hit ratio, RO 100\% miss ratio, WO 100\% hit ratio, and WO 100\% miss ratio.
We ran these patterns across all DRAM technologies for buffer sizes from 2 to 1024 by powers of 2.
For each
case we looked for the buffer size where the observed bandwidth by LLC reached to its maximum state and
would not get improved by increasing the size of the buffers.
To separate out the total number of buffers from the specific micro-architectural decisions (e.g., how to size each buffer), we only constrained the ``outstanding request buffer'' and allowed all of the internal buffers to be any size.

\subsection{Results and Discussion}

Figure \ref{fig:bws} shows the maximum achieved bandwidth for the DRAM cache, using DDR3, DDR4 and DDR5,
for the case of RO 100\% hit ratio, RO 100\% miss ratio, WO 100\% hit ratio, and WO 100\% miss ratio.
Moreover, Figure \ref{fig:orb} shows the amount of total buffers required in order to reach to the maximum bandwidth shown
in Figure \ref{fig:bws} for each case.

In RO 100\% hit ratio access pattern, each request requires only a single access to fetch the data along with tag and metadata from DRAM.
Thus, it is expected to achieve to a bandwidth very close to the theoretical peak bandwidth, and Figure~\ref{fig:bws} shows this is the case.
% The peak bandwidths in this case are 12.74 GB/s, 19.12 GB/s, and 33.27 GB/s for DDR3, DDR4, and DDR5, respectively.
Moreover,
the Figure \ref{fig:orb} shows that DDR3, DDR4, and DDR5 have reached to this bandwidth at buffer size of 128, 256, 256, respectively. Thus, increasing the buffer size would not help for RO 100\% hit traffic pattern.
%increasing the buffer size to any size larger than these sizes has not been effective to improve %the achieved bandwidth.
%We assume these buffer
%sizes for each device as their baseline size, since they achieve a bandwidth close to the peak %theoretical bandwidth of their technology.
%\note{Why not the buffer size that has peak bandwidth for writes?}

In WO 100\% hit ratio access pattern, each request requires two accesses to fetch tag and metadata from DRAM and then writing the data to the DRAM.
The peak bandwidth in this case for each device is lower than the theoretical peak by about a factor of two (5.58 GB/s, 8.2 GB/s, and 16.47 GB/s for DDR3, DDR4, and DDR5, respectively).
Since these requests require two DRAM accesses, the latency of each request is higher and more buffering is required to reach the peak bandwidth.
Specifically, Figure~\ref{fig:orb} shows that DDR3, DDR4, and DDR5 have reached to this bandwidth at buffer size of 256, 512, 1024~\footnote{DDR5 shows bandwidth improvement of 8\% going from 512 to 1024 entries}, respectively.
Comparing this case with RO 100\% hit ratio shows this case has gotten some bandwidth improvement from increasing the buffer size.
% However,
% any buffer size larger than these sizes did not get a higher bandwidth.

In RO 100\% miss ratio access pattern, each request requires three accesses: one for fetching tag and metadata from DRAM, then fetching
the line from NVRAM, and finally writing the data to the DRAM.
The peak bandwidth in this case for each device is 3.47 GB/s, 4.45 GB/s, and 6.21 GB/s for DDR3, DDR4, and DDR5, respectively.
Figure \ref{fig:orb} shows that DDR3, DDR4, and DDR5 have reached to this bandwidth at buffer size of 256, 256, 1024, respectively.
It suggests DDR5 has gotten some bandwidth improvement from increasing the buffer size. 
However, the bandwidth improvement from 512 to 1024 buffer entries is 3.8\% (and 2\% for 256 to 512 buffer size change).
%\note{Is this improvement too small to call this bandwidth the peak bandwidth?}
%However,
%any buffer size larger than these sizes did not improved achieved bandwidth for this case.

Finally, for WO 100\% miss ratio access pattern, each request requires four accesses: one for fetching tag and metadata from DRAM, then fetching
the line from NVRAM, writing the data to the DRAM, and a write to NVRAM for dirty line write back. 
The peak bandwidth in this case is 1.91 GB/s, 1.90 GB/s, and 1.90 GB/s for DDR3, DDR4, and DDR5, respectively.
The Figure \ref{fig:orb} shows that DDR3, DDR4, and DDR5 have reached to this bandwidth at buffer size of 16, 16, 32, respectively.
Comparing this case with the previous cases shows that, in this case the bandwidth gets saturated in smaller
buffer sizes than the other cases.

%\note{It would be interesting to see what the speedup of the ``best'' buffer for write only and read misses is over the ``baseline'' of 128 buffers.}

% The reason that in each case the bandwidth is not improved after a certain buffer size, is that the bandwidth becomes saturated and
% the buffer size is not a limiting factor anymore in that situation. In fact, the limiting factor becomes the latency and the performance
% of the device which is already at its peak by that specific buffer size. In this situation, increasing the buffer size
% will lead to higher latency according to the Little's Law. Another important factor to consider for the amount of total buffers required, is the
% access amplification of the requests. When the access amplification is higher, the amount of total jobs (e.g., multiple accesses to multiple memory interfaces)
% needed to be internally done by the controller is more. Thus, the bandwidth becomes saturated with fewer outstanding requests within the memory controller.
% Moreover, if the access amplification is not too high, the technologies with higher theoretical bandwidth, can benefit from larger buffers.

The main takeaway from this case study is that the buffer size required to achieve peak bandwidth largely depends on
the composition of memory traffic. Secondly, the DRAM cache controller might need a large number of buffers to reach the peak
bandwidth, primarily if the device can provide a large bandwidth (e.g., over 1000 in the case of DDR5).
More memory controller buffers are needed because of the increased latency of accesses since each write request and read miss request requires multiple accesses to the memory devices.

% Overall, this study shows that each DRAM technology benefit from a specific buffer size to get fully utilized as the DRAM cache,
% since they have different characteristic. This is an important point which is needed to be considered at the design time of the DRAM caches.


\subsection{HBM}

We separate out HBM from the discussion of other DDR memory technologies as its memory controller design may be different from the \textit{UDCC} design described in this paper and used in the Intel's Cascade Lake.
For instance, in Intel's Knight's Landing, the DRAM cache was implemented with a four-state coherence protocol and the HBM interfaces were physically separated from the DRAM interfaces through the on-chip network~\cite{sodani2015knights}.

% \Ayaz{The latest DRAM cache based commercial processors (e.g. Sapphire Rappids) rely on HBM as a DRAM cache.
% Since HBM devices are characterized by their high bandwidths, it raises a question that what buffer sizes would be sufficient in \textit{UDCC} to take full advantage of HBM as a cache.
% Secondly, what is the utilization of HBM in the cases when packets do not always hit in the DRAM cache?
% To answer these questions 
We modeled an HBM interface in \gem{} such that it can provide the theoretical peak bandwidth of 256 GB/s in a single memory channel (equivalent to the sum of all pseudo channels in an HBM2 device).
Figure~\ref{fig:hbm} shows the bandwidth that is achieved for different total buffer sizes, for two cases: 1) when all accesses hit in the DRAM cache, and 2) when all accesses miss in the DRAM cache.
% The results are based on a synthetic linear traffic pattern generated using gem5's traffic generator.
In case of all hits, we see close to the maximum possible bandwidth with an 2048 buffers.
In case of all misses, the achievable bandwidth is limited by the main memory used in the experiment (a NVRAM device) and does not get better after buffer size of 1024 entries.

This data implies that the buffer sizes required for HBM are not that much higher than for high-performance DDR DRAMs (e.g., DDR5).
The main reason that HBM does not require significantly more buffering is because even for a small miss rate (13\%), we are limited by the NVRAM's performance and saturate bandwidth at much less than the theoretical peak as shown in Figure~\ref{fig:hbm}.
However, if the HBM cache was backed by higher performance memory (e.g., DDR4 or DDR5 DRAM) the total number of buffers to achieve the maximum performance would likely be much higher.


\begin{figure}
  \centering
  \includegraphics[scale=0.6]{figures/hbmbw.pdf}
  \vspace{-1ex}
  \caption{Buffer size impact on read bandwidth for an HBM based DRAM cache (theoretical peak bandwidth : 256 GB/s).}
  \label{fig:hbm}
\end{figure}


\section{Case Study 3: Performance Analysis of DRAM Caches While Backed-up by Different Main Memory Models}

\subsection{Background and Methodology}


\begin{table}
	\centering
       \caption{NVRAM Interface Timing Parameters}
      \begin{tabular}{|p{2.2cm}|p{1.32cm}|p{1.50cm}|p{1.40cm}|}
      \Xhline{2\arrayrulewidth}
      \textbf{Parameter} & \textbf{Slow} & \textbf{Base} & \textbf{Fast} \\ \Xhline{2\arrayrulewidth}
      \textbf{tREAD} & 300ns & 150ns & 75ns \\ \hline
      \textbf{tWRITE} & 1000ns & 500ns & 250ns \\ \hline
      \textbf{tSEND} & 28.32ns & 14.16ns & 7.08ns \\ \hline
      \textbf{tBURST} & 6.664ns & 3.332ns & 1.666ns \\ \Xhline{2\arrayrulewidth}
    \end{tabular}
    \label{tab:nvmtimes}
\end{table}

% Heterogenous Memory (HM) emerged as a promising direction to increase the overall capacity and bandwidth
% by combining different memory technologies within the same system. Usually, HM consists of multiple memory components
% known as fast (or near) and slow (or far) memories. Many combination of devices can be considered in such system, e.g.,
% Intel's Cascade Lake provides a smaller-capacity fast-memory DDR4 compared to its large-capacity slow PCM memory connected
% to the same channel. In some of the HM systems, fast (near) memory can serve as the DRAM cache for the slow (far)
% memory. For instance, in Intel's Cascade Lake the DDR4 can serve PCM as DRAM cache, and in Intel's
% Sapphire Rapid a HBM memory is served as DRAM cache for NVRAM. Such combination of technologies in the DRAM caching idea,
% rises this question that how much the performance (such as latency) of the main memory can affect the performance of DRAM cache
% observed by the LLC? Since every miss on a fill-on-miss DRAM cache needs to be handled by communicating to the main memory,
% it is expected that the latency of the main memory can affect the overall bandwidth and latency of the DRAM cache.

In this section, we focus on the question how much does the performance (such as latency) of the main memory affect the performance of the memory system as observed by the LLC in DRAM cache-based systems? 
We extended the baseline
NVRAM interface model of gem5 to implement a faster NVRAM and slower NVRAM compared to the baseline performance.
The baseline, fast, and slow models of gem5 NVRAM provide 19.2 GB/s, 38.4 GB/s, and 9.6 GB/s bandwidth, respectively.
Table \ref{tab:nvmtimes} shows the timing constraints of all three cases.
% We considered a 16 MB DRAM cache using gem5 DDR4 model.
% \note{is this a repeat from the main methodology section?}
% Connected to \textit{UDCC}, a gem5's traffic generator produced the requests sent by LLC. 
The results are based on
an access pattern with a high miss ratio and large number of dirty line evictions from DRAM cache,
to highly challenge the performance of the main memory for both read and write accesses to it.
For this purpose, we used a WO with 100\% miss ratio access pattern which generates dirty lines along the way,
requiring write-backs to the main memory. This access pattern highly engages the NVRAM during miss 
handling to fetch the missed lines
and write-back of dirty lines, enabling us to evaluate the performance of the system 
in all three cases of slow, fast, and baseline NVRAMs. Moreover, we have tested them for a pattern 
consisting of RO 100\% miss ratio, since this case also requires interaction with NVRAM (for fetching 
the missed line). In both patterns, the results are based on a total buffer size of 512 entries.

\subsection{Results and Discussion}

\begin{figure}
  \centering
  \includegraphics[scale=0.5]{figures/nvmcs3.pdf}
  \vspace{-1ex}
  \caption{Observed bandwidth by the LLC for read-only (RO) 100\% miss ratio and write-only (WO) 100\% miss ratio.}
  \label{fig:bw-wo100M-cs4}
\end{figure}

% \begin{figure}
%   \centering
%   \includegraphics[scale=0.6]{figures/avgNvmQuLat-wo100miss-fastSlowNVM-caseStudy4.pdf}
%   \vspace{-1ex}
%   \caption{Average NVRAM write buffer queue latency for the write-only 100\% miss ratio access pattern.}
%   \label{fig:avgNvmQL-cs4}
% \end{figure}

First we investigate the effect of the different NVRAMs from the UDCC-external point of view.
Figure \ref{fig:bw-wo100M-cs4} compares the bandwidth seen by the LLC for three
different NVRAMs, slow, baseline, and fast, for RO 100\% miss ratio and WO 100\% miss ratio 
requests patterns. Note that NVRAM has a 
dedicated write buffer in \textit{UDCC} whose size is enforced by the NVRAM interface (128 entries in gem5 NVRAM models). 
Our resuts showed, the average queuing latency for this buffer is 76.19~$\mu$s,
38.49~$\mu$s, and 19.52~$\mu$s for slow, baseline, and fast NVRAMs, respectively.
In other words, the write queuing latency at the NVRAM interface, gets shorter
once the speed of NVRAM increases, as expected. 
In the WO 100\% miss ratio, the highest achieved bandwidth is 0.96 GB/s, 1.91 GB/s, and 3.8 GB/s for slow, baseline and fast NVRAMs, 
respectively. This interprets to 49.7\% bandwidth degradation once the
slow NVRAM was used as the main memory, and 98.9\% bandwidth improvement when the fast NVRAM
was used as the main memory, compared to the baseline NVRAM. Note that in this request pattern 
there are two NVRAM accesses (one read and one write).

In the RO 100\% miss ratio, the highest achieved bandwidth is 3.39 GB/s, 4.55 GB/s, and 5.1 GB/s for slow, baseline and fast NVRAMs, 
respectively. This interprets to 25.49\% bandwidth degradation once the
slow NVRAM was used as the main memory, and 12.08\% bandwidth improvement when the fast NVRAM
was used as the main memory, compared to the baseline NVRAM. Note that in this request pattern 
there is only one NVRAM (read) access.

These results suggest that if an access pattern requires more interaction with NVRAM (e.g., 
where the DRAM cache miss ratio is higher, or when there are more dirty line evictions), 
the improvement of bandwidth observed by LLC with faster NVRAM is higher, for a system consisting of DRAM cache. 
Moreover, these results are based on a total buffer size of 512 entries as our further investigations showed that 
none of the devices gains more bandwidth by providing larger buffer size. Thus, even though slowing down 
the NVRAM device hurts performance, it does not affect the microarchitectural details of the controller.


% \note{why WO 100\% miss? If you're interested in the buffer size, then you 
% should look at 100\% miss reads or 100\% hits writes as that's what you found required the most buffers before.}

% Note that slow and baseline NVRAM have reached to the saturated bandwidth at small buffer size (16 entries), and have not
% gained any bandwidth improvement by increasing the size of the buffers after 16 entries. The fast NVRAM has gained bandwidth
% improvement by increasing the buffer size up-to 256 entries. In all three cases, increasing the size of buffers after the bandwidth is 
% reached to its saturated state, 
% is not effective on improving performance. The reason is that 
% the system is now limited by the NVRAM and DRAM devices performance (rather than the size of buffer before bandwidth saturation).
% In other words, the devices are at their full utilization and increasing the amount of total buffers is not improving bandwidth, yet it can lead 
% to longer latency.

% Now, we look for the effect of different NVRAM speed from the UDCC-internal point of view.
% Figure \ref{fig:avgNvmQL-cs4} compares the average NVRAM queuing latency per requests for slow,
% baseline, and fast NVRAMs. The amount of total buffers is swept over, on the X-axis. At total buffer size of 512, the average latency is 76.19~$\mu$s,
% 38.49~$\mu$s, and 19.52~$\mu$s for slow, baseline, and fast NVRAMs, respectively.
% This means the queuing latency at the NVRAM interface, gets shorter
% once the speed of NVRAM increases, as expected. However, for all three NVRAMs, as the buffer size increases, the average queuing latency
% also increases. 
% We know from Figure \ref{fig:bw-wo100M-cs4}
% the bandwidth is saturated at the buffer size of 16 entries for slow and baseline NVRAMs, and at the buffer size of 256 entries
% for fast NVRAM. At the same time, from Figure \ref{fig:avgNvmQL-cs4}
% we can observe the average queueing time of NVRAM keeps getting longer as the buffer size increases. 
% Note that NVRAM has a
% dedicated write buffer in UDCC whose size is enforced by the NVRAM interface (128 entries in gem5 NVRAM models).
% As explained before, once the bandwidth is saturated, NVRAM and DRAM devices are at their full utilization and adding more buffer size
% is not beneficial to improve the performance. Instead, it adds pressure on the internal buffers such as NVRAM write buffer in this case,
% causing longer latency for each requests in those buffers. As a result, choosing the proper size of buffers is important to avoid any indirect
% performance degradation.

%%%%%%%%%%%%%%%%%%%%%%%%%%%%%%%%%%%%%%%%%%%%%%%%%%%%%%%%%%%%%%%%%%%%%%%%%%%%%%%%
\section{Case Study 4: Evaluating DRAM Caches for Real Applications}
\label{sec:realBench}

In this case study, we performed an evaluation of DRAM caches for real-world applications, i.e., NPB~\cite{bailey1991parallel} and GAPBS~\cite{beamer2015gap}.
The DRAM caches are expected to perform similar to a DRAM main memory if the working set of the workload fits in the cache.
Therefore, a case of particular interest for us is to evaluate the performance of the DRAM cache-based system when the workload does not fit in the DRAM cache.
To accomplish the previously mentioned goal, we model a scaled-down system with 64MB DRAM cache and run NPB and GAPBS workloads for one second of simulation time.

We run all workloads in three different configurations.
The first two configurations (\textit{NVRAM, DRAM}) model a system without a DRAM cache and the main memory as NVRAM or DRAM.
The third configuration (\textit{DCache\_64MB}) uses a 64MB DRAM cache and NVRAM as main memory.
Figure~\ref{fig:npb} shows a million instructions per second (MIPS) values for NPB in three configurations.
In most cases, \textit{DCache\_64MB} performs the worst, with the most prominent performance degradation for \textit{lu.C} and \textit{bt.C}.
The only exception is \textit{is.C}, where \textit{DCache\_64MB} performs better than \textit{NVRAM}.
The performance of \textit{DCache\_64MB} correlates with the DRAM cache misses per thousand instructions (MPKI) values shown in Figure~\ref{fig:npbmiss}.
For example, \textit{is.C} shows the smallest and \textit{lu.C} shows the largest MPKI values.

Figure~\ref{fig:gapbs} shows a million instructions per second (MIPS) values for GAPBS in the previously mentioned three configurations.
Figure~\ref{fig:gapbmiss} shows the associated MPKI values.
The graph workloads on a DRAM cache perform mostly similarly to NVRAM alone in our simulation runs.
The only exception is \textit{bfs} which shows a significantly lower MIPS value than \textit{NVRAM}.
The DRAM cache MPKI value for \textit{bfs} alone does not explain the reason for this behavior.
\textit{bfs} has the highest fraction of writes than reads in the requests that hit the DRAM cache (29\% more writes than reads in case of \textit{bfs} in contrast to 36\% less writes than reads for the rest of the workloads)

Since a write hit also leads to access amplification (tag read before a write hit), the impact of this amplification is then seen in the performance degradation with a DRAM cache.
We conclude from this observation that the extra DRAM accesses (for tag reads) can also impact the workload's performance on a DRAM cache.

\begin{figure}
  \subfloat[\scriptsize{Performance}]{
    \includegraphics[width=.55\linewidth, scale=0.6]{figures/npbNew.pdf}
    \label{fig:npb}
  }
   \subfloat[\scriptsize{Cache miss rate}]{
    \includegraphics[width=.45\linewidth, scale=0.5]{figures/npb_miss.pdf}
    \label{fig:npbmiss}
  }
    \caption{NAS Parallel Benchmarks on DRAM Cache}
    \label{fig:npbRes}
  \end{figure}

\begin{comment}
\begin{figure}
  \centering
  \includegraphics[scale=0.7]{figures/gapbsNew.pdf}
  \vspace{-1ex}
  \caption{GAPBS.}
  \label{fig:gapbs}
\end{figure}

\begin{figure}
  \centering
  \includegraphics[scale=0.7]{figures/gapbs_miss.pdf}
  \vspace{-1ex}
  \caption{DRAM cache misses per thousand instructions}
  \label{fig:gapbs}
\end{figure}
\end{comment}

\begin{figure}
  \subfloat[\scriptsize{Performance}]{
    \includegraphics[width=.55\linewidth, scale=0.6]{figures/gapbsNew.pdf}
    \label{fig:gapbs}
  }
   \subfloat[\scriptsize{Cache miss rate}]{
    \includegraphics[width=.45\linewidth, scale=0.5]{figures/gapbs_miss.pdf}
    \label{fig:gapbmiss}
  }
    \caption{GAPBS on DRAM Cache}
    \label{fig:gapbsRes}
  \end{figure}


%%%%%%%%%%%%%%%%%%%%%%%%%%%%%%%%%%%%%%%%%%%%%%%%%%%%%%%%%%%%%%%%%%%%%%%%%%%%%%%%
\section{Case Study 5: NVRAM Write Wear Leveling Effect}

\subsection{Background and Methodology}
Non-volatile storage and memory devices, such as Phase Change Memories (PCM) used as NVRAM
and flash memory, are known to have a limited write endurance.
Write endurance is defined as the number of writes to a block in the device, before it becomes
unreliable. One of the techniques used for such devices to prolong their lifetime, is
wear leveling. Usually NVRAM wear-leveling technique tries to evenly distribute wear-out by moving data from one
highly-written location to a less worn-out location. Wang et al.~\cite{wang2020characterizing} measured the frequency and latency overhead
of data migration due to wear leveling in NVRAMs through probing and profiling. They reported a long tail latency of 60~$\mu$s
for almost every 14000 writes to each 256B region of NVRAM. Wear leveling can affect the performance of DRAM caches. 
This effect is more noticeable while running write-intensive
workloads which their memory footprint is larger than the DRAM cache capacity. In such situation, many dirty lines eviction will happen
that are needed to be written back to NVRAM; thus, an overall increase on NVRAM writes can be expected. As a result, frequent data migration for
wear-leveling (with a long tail latency) will be done by NVRAM interface and a performance degradation
can be expected in this case.

In this section we investigate the effect of this long latency on DRAM cache performance. We extended NVRAM interface model of
gem5 so for every 14000 write accesses the interface adds 60~$\mu$s extra latency, delaying the service time of the next request.
% We configured \textit{UDCC} with 16MB DDR4 DRAM cache and used an ORB with 256 entries.
% \note{repeated???}
The access pattern was set to be all writes and 100\% miss and 8x bigger than the DRAM cache size. This will
increase the number of write-backs of dirty lines
from DRAM cache to the NVRAM, to pressure the NVRAM write buffer and see the effect of wear leveling delay on the overall system performance.

\subsection{Results and Discussion}

\begin{table}
	\centering
       \caption{Maximum write bandwidth (and average NVM write latency) achieved with and without wear-leveling}
      \begin{tabular}{|p{2.3cm}|p{2.45cm}|p{2.75cm}|}
      \Xhline{2\arrayrulewidth}
      \textbf{Parameter} & \textbf{With wear-leveling} & \textbf{Without wear-leveling} \\ \Xhline{2\arrayrulewidth}
      \textbf{Bandwidth} & 1.77 GB/s & 1.92 GB/s \\ \hline
      %\textbf{Buffer size} & 256 & 256 \\ \hline
      \textbf{NVM write latency} & 42.84~$\mu$s & 39.71~$\mu$s \\ \Xhline{2\arrayrulewidth}
    \end{tabular}
    \label{tab:nvmwear}
\end{table}

\begin{comment}
\begin{figure}
  \centering
  \includegraphics[scale=0.6]{figures/bw-wearLeveling-cs6.pdf}
  \vspace{-1ex}
  \caption{[They will be bar graphs] Write bandwidth observed by LLC (GB/s) with and without NVRAM wear leveling delay.}
  \label{fig:bw-wearLeveling}
\end{figure}

\begin{figure}
  \centering
  \includegraphics[scale=0.6]{figures/avgNvmWrQT-wearleveling-cs6.pdf}
  \vspace{-1ex}
  \caption{[They will be bar graphs] Average NVRAM write queuing time with and without NVRAM wear leveling delay.}
  \label{fig:avgNvmWrQT-wearleveling}
\end{figure}
\end{comment}

Table \ref{tab:nvmwear} compares the overall bandwidth seen by LLC in two cases; with wear leveling and
without wear leveling. 
%The ORB size is 256 entries which gives the highest bandwidth possible for both cases.
Without wear leveling the peak bandwidth
is 1.92~GB/s while it drops to~1.77 GB/s when wear leveling is activated. These results show a 7.8\% performance degradation which directly comes
from wear leveling overhead.
Table \ref{tab:nvmwear} also shows the average queuing latency measured for NVRAM write buffer. This latency is 42.84~$\mu$s for
NVRAM-with-wear-leveling, and 39.71~$\mu$s for NVRAM-without-wear-leveling.
Thus, the 60~$\mu$s latency considered for data migration during wear leveling, has caused 7.8\% latency overhead on NVRAM write buffer as well.
This 7.8\% overhead is larger than expected given the rarity of wear leveling events showing that these rare events have an outsized impact when the system is configured with a DRAM cache.

% Note that due to the access amplification in this case, the bandwidth observed by LLC without wear leveling is much less than the theoretical
% bandwidth of DRAM cache device (19.2 GB/s for DDR4). Thus, 7.8\% additional performance degradation caused by wear-leveling is
% another point which should be considered in the design of DRAM caches.

%\note{could probably drop the table to save space.}

\section{Related work}
\noindent \textbf{Video foundation models.}
With sufficient computational power and an abundant source of data, there have been attempts to build a single large-scale foundation model that can be adapted to diverse downstream tasks.
Along with the success of foundations models in the natural language processing domain~\cite{brown2020language,chen2021evaluating,devlin2019bert} and in computer vision~\cite{bertasius2021space,jia2021scaling,radford2021learning}, video data has become another data type of interest, as it has grown in scale due to numerous internet video-sharing platforms.
Accordingly, several methods to train a video foundation model have been proposed.
Due to the innate multi-modality of video data, \textit{i.e.}, a combination of visual $\cdot$ vocal $\cdot$ textual context, most works have centered around the variations of the cross-modal attention mechanism \cite{akbari2021vatt,bertasius2021space,gabeur2020multi,luo2020univl,neimark2021video,tan2021look,wei2020multi,yang2021taco}.
In addition, as most video data lack proper labels or descriptions, contrastive learning methods were studied to learn meaningful feature representations or enhance video-text alignment in a self-supervised manner \cite{akbari2021vatt,kuang2021video,luo2020univl,yang2021taco}.

More specifically, MERLOT \cite{zellers2021merlot} proposed a multi-modal representation learning method for visual commonsense reasoning, which also performed well in twelve video reasoning tasks.
VATT \cite{akbari2021vatt} introduced a multi-modal learning method via contrastive learning. 
The pre-trained model performed well in a variety of vision tasks from image classification to video action recognition and zero-shot video retrieval.
Another representative work, UniVL \cite{luo2020univl} proposed a straightforward pre-training method with auxiliary loss functions. 
After fine-tuning on a specific task, the pre-trained model performed outstandingly in a wide range of tasks of text-to-video retrieval, action segmentation, action step localization, video sentiment analysis, and video captioning.
Other foundation models for multiple video tasks include \cite{li2020hero,sun2019learning,sun2019videobert,zhu2020actbert,fu2021violet,wang2022all}. 

\noindent \textbf{Auxiliary learning.}
In order to enhance the performance of one or a multitude of primary tasks, auxiliary learning methods can be incorporated.
\cite{ruder2017overview} introduced Multi-task learning (MTL) to the deep neural networks by training a single model with multiple task losses to assist learning on the main task.
Such a method is generally adapted to pre-train the foundation models in the self-supervised manner~\cite{li2020hero,sun2019learning,sun2019videobert,zhu2020actbert,fu2021violet,wang2022all}.
However, these various pretext task losses used in the pre-training phase are ignored in the fine-tuning phase, and only the primary task loss is minimized.

Recently, meta-learning methods have been introduced for auxiliary learning.
\cite{liu2019self,navon2020auxiliary,shu2019meta} proposed a meta-learning method in which the model learns auxiliary tasks to generalize well to unseen data. 
In these settings, a separate subset of data is held out as the primary task, while the others are used as auxiliary tasks that aid the primary task's performance.
Similar methods were adopted for computer vision tasks such as semantic segmentation \cite{xu2021leveraging}.
Other domain applications include navigation tasks with reinforcement learning \cite{ye2021auxiliary}, or self-supervised learning methods on graph data \cite{hwang2020self}.

\section{Conclusion}

In this work, we described our detailed cycle-level DRAM cache
simulation model implemented and validated on \gem{}, which can enable performing a design space exploration in
DRAM cache research.
% As the applications memory footprint is scaling out and demanding larger and faster memories,
% and heterogeneous memory systems are becoming more available
% in the market to address this demand, new hardware techniques are required
% to enable efficient data management in the memory systems.
This new tool can be used to explore new unified DRAM cache and memory controller designs as part of an agile and full-system simulation platforms.
% are required to allow exploration of new techniques to address these needs.
The tool we presented in this work
can enable many interesting research work in the domain of heterogeneous memory systems.
For instance, using \textit{UDCC} we can address questions such as what is
the efficient data movement or data placement in systems composed of fast and slow memories.
Since, our tool provides full system simulation platform, also it can address the hardware and software co-design
ideas to explore design space of heterogeneous memories.

Moreover, \gem{} is highly modular and allows composing a simulated system based on a variety of components.
\textit{UDCC} can enable experimenting with different and new memory device models which might have features to be a better fit to be used as a cache to a backing memory.

%\section*{ACKNOWLEDGMENT}

\bibliographystyle{IEEEtran}
\bibliography{IEEEabrv,references}

\end{document}
