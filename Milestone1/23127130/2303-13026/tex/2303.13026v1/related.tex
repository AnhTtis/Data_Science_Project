\section{Related work}

Most of the prior research work in DRAM cache organization do not provide detailed methodologies required to model a DRAM cache during the simulation. In terms of modeling of a 
DRAM cache, Gulur et al.~\cite{gulur2015comprehensive} presented an analytical performance model of DRAM cache for in-SRAM and in-DRAM tag storage organizations. 
Their model considers parameters such as DRAM Cache's and off-chip memory's timing values, cache block size, tag cache/predictor hit rate and workload characteristic, 
to estimate average miss penalty and bandwidth seen by the last level on chip SRAM cache (LLSC). Their work is based on a prior model called ANATOMY~\cite{gulur2014anatomy} which is a trace-based analytical model estimating 
key workload characteristics like arrival rate, row-buffer hit rate, and request spread, that are used as inputs to the network-like queuing model to statistically estimate memory performance. 
Even though this work accounts for the timing constraints of DRAM for the cache management, it is agnostic of the microarchitectural and timing constraints of main memory technologies cooperating with DRAM cache, 
and still leaves a gap for a full system DRAM cache simulation for detailed architectural analysis.

% To the best of our knowledge, there is only one work discussing an analytical performance model of DRAM cache for in-SRAM and in-DRAM tag storage
% organizations~\cite{gulur2015comprehensive}. Gulur et al.~\cite{gulur2015comprehensive} proposed an analytical DRAM cache model to estimate the average miss penalty and the bandwidth seen by the last level (on-chip) cache. Though this work considers the timing constraints of DRAM for cache management, it is agnostic to the microarchitecture and timing constraints of main memory technologies cooperating with DRAM cache and still leaves a gap for a full-system DRAM cache simulation for detailed microarchitectural analysis.

%\note{Add notes on VANS/LENS}

Wang et al.~\cite{wang2020characterizing} presented VANS, a cycle-level NVRAM simulator which, models the microarchitectural details of Optane DIMMs.
However, their simulation model does not support using an NVRAM device as a backing memory of a DRAM cache.
VANS can be integrated with other simulators like \gem{}.
In our work, we rely on \gem{}'s default NVM model, but we plan to use the VANS detailed NVRAM model in the future.