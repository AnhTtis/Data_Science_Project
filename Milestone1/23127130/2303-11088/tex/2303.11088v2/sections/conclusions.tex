
\section{Conclusions}\label{sec:conclusions}

In this study, we benchmark the scalability of distributed stream processing frameworks particularly suited to be used within microservices.
We experimentally evaluate the frameworks  Apache Flink, Apache Kafka Streams, Hazelcast Jet, and Apache Beam with the Apache Flink runner and the Apache Samza runner in a private cloud environment and in the Google cloud.
We find that all frameworks show linear scalability for most use cases, however with partially significantly different resource demands.
There is no clear superior framework. Instead, depending on the use case Flink, Hazelcast Jet, or Kafka Streams show the lowest increase in resource demand.
Using Apache Beam as abstraction layer still comes with a significant negative impact on performance, leading to a significantly steeper increase in resource demand, independent of the use case.
We observe our results regardless of scaling load on a microservice, scaling the computational work performed inside the microservice, the selected cloud environment, and whether the microservice is scaled over multiple nodes or on a single node.
The latter means that vertical scaling distributed stream processing frameworks can---to some extent---also complement horizontal scaling.
In summary we found that while scalable microservices can be designed with all evaluated frameworks, the choice of a framework and its deployment has a considerable impact on the cost of operating it.
This emphasizes also a key benefit of designing stream processing applications using microservice-based architectures: the ability for each service to select the most suitable stream framework based on its specific use case, requirements for scalability, and other relevant considerations.
Our findings leave room for future research on exploring the multi-dimensional space of cloud deployment options and finding an optimal one.
Moreover, our work paves the way for deeper analysis of the underlying reasons for our observed results by benchmarking further task samples, deployments, and configuration options.





