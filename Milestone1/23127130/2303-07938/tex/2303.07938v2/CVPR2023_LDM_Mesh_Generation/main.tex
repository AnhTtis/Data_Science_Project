% CVPR 2023 Paper Template
% based on the CVPR template provided by Ming-Ming Cheng (https://github.com/MCG-NKU/CVPR_Template)
% modified and extended by Stefan Roth (stefan.roth@NOSPAMtu-darmstadt.de)

\documentclass[10pt,twocolumn,letterpaper]{article}

%%%%%%%%% PAPER TYPE  - PLEASE UPDATE FOR FINAL VERSION
% \usepackage[review]{cvpr}      % To produce the REVIEW version
% \usepackage{cvpr}              % To produce the CAMERA-READY version
\usepackage[pagenumbers]{cvpr} % To force page numbers, e.g. for an arXiv version

% Include other packages here, before hyperref.
\usepackage{graphicx}
\usepackage{mmstyles}
\usepackage{amsmath}
\usepackage{amssymb}
\usepackage{booktabs}
\usepackage{multirow}
% \usepackage{subfigure}
\input{math_command}

% It is strongly recommended to use hyperref, especially for the review version.
% hyperref with option pagebackref eases the reviewers' job.
% Please disable hyperref *only* if you encounter grave issues, e.g. with the
% file validation for the camera-ready version.
%
% If you comment hyperref and then uncomment it, you should delete
% ReviewTempalte.aux before re-running LaTeX.
% (Or just hit 'q' on the first LaTeX run, let it finish, and you
%  should be clear).
\usepackage[pagebackref,breaklinks,colorlinks]{hyperref}


% Support for easy cross-referencing
\usepackage[capitalize]{cleveref}
\crefname{section}{Sec.}{Secs.}
\Crefname{section}{Section}{Sections}
\Crefname{table}{Table}{Tables}
\crefname{table}{Tab.}{Tabs.}


%%%%%%%%% PAPER ID  - PLEASE UPDATE
\def\cvprPaperID{6089} % *** Enter the CVPR Paper ID here
\def\confName{CVPR}
\def\confYear{2023}

\newcommand{\lyu}[1]{\textcolor{red}{Lyu: #1}}
\newcommand{\DDM}{Point cloud representation}
\newcommand{\LDM}{Sparse latent point representaiton}

\begin{document}

%%%%%%%%% TITLE - PLEASE UPDATE
\title{Controllable Mesh Generation Through Sparse Latent Point Diffusion Models}

% \author{First Author\\
% Institution1\\
% Institution1 address\\
% {\tt\small firstauthor@i1.org}
% % For a paper whose authors are all at the same institution,
% % omit the following lines up until the closing ``}''.
% % Additional authors and addresses can be added with ``\and'',
% % just like the second author.
% % To save space, use either the email address or home page, not both
% \and
% Second Author\\
% Institution2\\
% First line of institution2 address\\
% {\tt\small secondauthor@i2.org}
% }
\newcommand{\AuthorSpace}{\hspace{1.2em}}
\author{%
Zhaoyang Lyu$^{1}$\thanks{Equal Contribution.} \AuthorSpace{} Jinyi Wang$^{1,2*}$\AuthorSpace{} Yuwei An$^{1,4}$\AuthorSpace{} Ya Zhang$^{1,2}$\AuthorSpace{} Dahua Lin$^{1,3}$\AuthorSpace{} Bo Dai$^{1}$ \\
$^1$Shanghai AI Laboratory \hspace{1em}
$^2$Shanghai Jiao Tong University\\
$^3$The Chinese University of Hong Kong \hspace{1em}
$^4$Tsinghua University \\
% \texttt{lvzhaoyang@pjlab.org.cn, jinyi.wang@sjtu.edu.cn, anyuwei@pjlab.org.cn} \\
\texttt{lyuzhaoyang@link.cuhk.edu.hk, jinyi.wang@sjtu.edu.cn} \\
\texttt{anyuwei@pjlab.org.cn, ya\_zhang@sjtu.edu.cn}\\
\texttt{dhlin@ie.cuhk.edu.hk, daibo@pjlab.org.cn}
}
% \author{%
% Zhaoyang Lyu$^{1}$\thanks{Equal Contribution.} \AuthorSpace{} Jinyi Wang$^{1,2*}$\AuthorSpace{} Yuwei An$^{4}$\AuthorSpace{} Ya Zhang$^{1,2}$\AuthorSpace{} Dahua Lin$^{1,3}$\AuthorSpace{} Bo Dai$^{1}$ \\
% $^1$Shanghai AI Laboratory \hspace{1em}
% $^2$Shanghai Jiao Tong University\\
% $^3$The Chinese University of Hong Kong \hspace{1em}
% $^4$Tsinghua University \\
% \texttt{lyuzhaoyang@link.cuhk.edu.hk, jinyi.wang@sjtu.edu.cn,anyuwei@qinghua.com} \\
% \texttt{ya_zhang@sjtu.edu.cn, dhlin@ie.cuhk.edu.hk, daibo@pjlab.org.cn}
% }
\maketitle
%%%%%%%%% ABSTRACT
\begin{figure*}[h!]
\vspace{-3em}
\centering
\includegraphics[width=1\textwidth]{Figures/method_figures/controllable_generation.png}
\caption{We can use the sparse latent points to control the shape of the generated meshes. Red points are stationary, and blue points are moving. Black arrows indicate the moving direction of the blue points. Some points are invisible because they are within the mesh. 
Note that the latent points are not always strictly lying on the surface of the generated meshes. This is because our point cloud decoder assumes that some noises exist in the positions of the sparse latent points. It will generate a mesh that best fits the latent points, but avoid generating defective meshes just to strictly fit the latent points.}
\label{fig:controllable_generation}
\vspace{-1.5em}
\end{figure*}
\begin{abstract}
\begin{abstract}
As models continue to grow in size, the development of memory optimization methods (MOMs) has emerged as a solution to address the memory bottleneck encountered when training large models. To comprehensively examine the practical value of various MOMs, we have conducted a thorough analysis of existing literature from a systems perspective. 
% Furthermore, we have evaluated the most widely adopted MOMs employed in mainstream frameworks for both vision and language models.
Our analysis has revealed a notable challenge within the research community: the absence of standardized metrics for effectively evaluating the efficacy of MOMs. The scarcity of informative evaluation metrics hinders the ability of researchers and practitioners to compare and benchmark different approaches reliably. Consequently, drawing definitive conclusions and making informed decisions regarding the selection and application of MOMs becomes a challenging endeavor.
To address the challenge, this paper summarizes the scenarios in which MOMs prove advantageous for model training. We propose the use of distinct evaluation metrics under different scenarios. By employing these metrics, we evaluate the prevailing MOMs and find that their benefits are not universal. We present insights derived from experiments and discuss the circumstances in which they can be advantageous.

\end{abstract}
\end{abstract}

%%%%%%%%% BODY TEXT
\vspace{-1em}
\section{Introduction}
\label{sec:intro}
\section{Introduction}
\IEEEPARstart{T}{he} method Neural Radiance Fields (NeRF)~\cite{mildenhall2020nerf} is proposed for photorealistic novel view synthesis. Given many views of the scene, it creates implicit multi-view geometry and learns for view synthesis. However, it has poor generalizations to new scenes and requires retraining or fine-tuning on each scene. 
 
 Recent work~\cite{Yu_2021_CVPR,Trevithick_2021_ICCV} has explored the ways of using a single image to train NeRF. They introduce a convolutional feature encoder to learn the image representation which gives it some limited generalization abilities to unseen scenes.  But, without fine-tuning, these methods produce many floats and artifacts in rendering novel views. 
 
  Multi-Plane Images (MPI) representation that learns multiple RGB images from a single image is also used in \cite{Wu_2021_ICCV,Tucker_2020_CVPR,wu2022remote} for  novel view synthesis. However, MPI heavily relies on the qualities of the planar images and needs plenty of image planes to avoid blurs. There is no strong 3D geometry constraint and it fails in many complex scenes.
  
  MINE~\cite{Li_2021_ICCV2} introduces the volume rendering of NeRF into the MPI. It runs faster and produces better depth rendering quality compared with single-view NeRFs~\cite{Yu_2021_CVPR,Trevithick_2021_ICCV}. However, the rendering quality heavily relies on the number of image planes. It needs high-resolution 4D volumes to store the 4-channel  (RGB and volume density) image planes that cost a large amount of GPU memory in both training and 
 prediction.  
 

 
 \begin{figure}[t]
\setlength{\abovecaptionskip}{7pt}
\setlength{\belowcaptionskip}{0pt}
	\centering
% 	\subfigure[MINE (PSNR:14.9)]{  % for AAAI
	\subfloat[MINE (PSNR:14.9)]{
%			\centering
			\includegraphics[width=0.23\textwidth]{figure/intro/DJI_20200223_163206_598_0_MINE.png}
%			\label{subfig:pixelnerf}
	}\subfloat[MINE (depth)]{
%			\centering
			\includegraphics[width=0.23\textwidth]{figure/intro/MINE_disp.png}
%			\label{subfig:mpi}
	}
	\\[-3mm]
	\subfloat[Ours (PSNR:17.0)]{
%			\centering
			\includegraphics[width=0.23\textwidth]{figure/intro/DJI_20200223_163206_598_0_ours.png} 
	}\subfloat[Ours (depth)]{
%			\centering
			\includegraphics[width=0.23\textwidth]{figure/intro/ours_disp.png}
	}
	\caption{Comparison with state-of-the-art methods. (a-b) RGB and depth rendering results of  \cite{Li_2021_ICCV2}. It produces many blurs and floats in the occluded regions and at the object/depth edges. 
	(c-d) Our method employs a joint rendering mechanism that preserves more image details and predicts sharp depth edges.}
	\label{fig:performance_illustration}
\end{figure}
 
 In this paper, we propose a joint rendering mechanism that takes the MPI strategy for coarse sampling proposals and the MLP\&volume-based rendering~\cite{mildenhall2020nerf} for fine sampling and rendering. Then, both the coarse point samples and the fine samples are combined according to their geometry distribution to realize a more accurate joint rendering. More importantly, we introduce a depth teacher net that serves as the guidance for the joint rendering. The monocular depth teacher predicts dense pseudo depth maps that assist the consistent 3D geometry learning between the MPI, the fine volume, and the joint rendering. It also boosts the multi-view geometry consistency between the source view and the target novel views that 
helps handle the occlusions, reduce the blurs and floats, and render accurate depths. 
 
In the experiments,  we verify the effectiveness of our method on three challenging real-scene datasets (RealEstate10K~\cite{zhou2018stereo}, NYU~\cite{silberman2012indoor} and  NeRF-LLFF~\cite{mildenhall2020nerf}) for novel view synthesis or depth estimation. Given a single image as input, our method is shown able to produce higher qualities in both the RGB image rendering and depth map prediction. It far outperforms state-of-the-art methods~\cite{Li_2021_ICCV2,Yu_2021_CVPR} with improvements of 5$\sim$20\% in PSNR and SSIM for the RGB rendering and reduces 20$\sim$50\% of the errors for the depth prediction.

\vspace{-1em}
\section{Background}
\label{sec:back}
\section{Background and Motivation}
This section first introduces mini-batch GNN training, 
% in Section~\ref{sec:background}.
and then 
% discusses the data loading-induced performance bottleneck and 
elaborates on the limitations of existing optimizations. 
% , such as caching and hybrid parallelism, 
% in Section~\ref{sec:problem}.

\begin{figure}[ht]
    \centering
    \includegraphics[width=\linewidth]{figures/mini-batch.pdf}
    \caption{Example of a mini-batch.}
    \label{fig:minibatch}
\end{figure}

\subsection{Mini-Batch GNN Training and Data Parallelism} 
A GNN model is defined as a sequence of \emph{GNN layers}.\footnote{In the following, the term ``layer'' will refer to GNN layers, not to neural network layers unless otherwise stated.}
During each mini-batch training iteration, there are three phases: \textit{sampling}, \textit{loading}, and \textit{training}.
The sampling phase randomly selects a mini-batch starting from the target vertices.
A mini-batch with two target vertices is shown in Figure~\ref{fig:comparison}(a). 
In the loading phase, the input features of the vertices in the bottom layer of the mini-batch are loaded into the GPUs.
During forward propagation, each GNN layer $l > 0$ aggregates and transforms the features of the vertices in the layer $l-1$ of the sample and produces the features of the vertices in the layer $l$ (see Figure~\ref{fig:minibatch}).
The last GNN layer computes the features of the target vertices, which are then used to calculate the loss. 
During backward propagation, the layers are executed in reverse order to compute gradients. 
Finally, all GPUs aggregate and apply the computed gradients.


\begin{figure}[ht]
    \centering
    \begin{subfigure}[t]{0.45\linewidth}
        \centering
        \includegraphics[width=\textwidth]{results/breakdown/orkut_epoch_breakdown.pdf}
        \caption{The fraction of sampling, loading, and training time per epoch of DGL, P3* and Quiver on Orkut with the GAT model.}
        \label{fig:orkut_epoch_breakdown}
    \end{subfigure}
    \hfill
    \begin{subfigure}[t]{0.45\linewidth}
        \centering
        \includegraphics[width=\textwidth]{results/breakdown/quiver_epoch_breakdown.pdf}
        \caption{The percentage of sampling, loading, and training time per epoch of Quiver on Orkut and Papers100M with the GraphSage model. }
        \label{fig:quiver_epoch_breakdown}
    \end{subfigure}
    \caption{Epoch time breakdown of existing systems on a four-GPU host with NVLink.}
    \label{fig:epoch_breakdown}
\end{figure}

% \begin{figure}[h!]
%     \centering
%     \advance\leftskip-1cm
% % \advance\rightskip-3cm
% \includegraphics[keepaspectratio=true,width=.5\textwidth]{results/epoch_breakdown/sage_epoch_breakdown_arxiv.png}
%     % \includegraphics[width=0.85\columnwidth]{results/epoch_breakdown/sage_epoch_breakdown.png}
%      \caption{The fraction of sampling, loading, and training time per epoch of existing systems when training the GraphSage model on a four GPU host with NVLink. \ms{New figures. This figure is out of bounds. Larger fonts. Edit text as needed}
%      %DGL and P3 cache no feature data in all experiments. Quiver caches 44\% for Papers100M, and 12\% percent for Friendster.  All systems use the GAT model with fanouts [15, 15, 15] and batch size 1024. 
%      }
%     \label{fig:epoch_breakdown}
% \end{figure}

Data parallelism is the most commonly used training strategy for mini-batch GNN training. 
In data parallel training, the target vertices are partitioned among GPUs, where each partition corresponds to a separate \textit{micro-batch} (see Figure~\ref{fig:comparison}(a)). 
Each GPU independently loads the input features of all the vertices in the bottom layer of its micro-batch and trains on it.
This approach has two limitations: a high cost of data loading and a high degree of computational and data loading redundancy.

% Instead of creating multiple independent and overlapping micro-batches as done by data parallelism, \tname generates a single mini-batch for all the target vertices and splits it without overlaps, avoiding redundant computation and loads.

\mypar{Data loading bottleneck}
Input feature loading is a major overhead in data-parallel GNN training, which contributes to a large fraction of the total training time and prevents a good utilization of the GPUs.
% When training a GNN using a mini-batch with N target vertices, the GPUs need to load the feature vectors of the vertices in the k-hop neighborhood of these target vertices into their memory. 
% 
Figure \ref{fig:orkut_epoch_breakdown} shows the time breakdown of sampling, feature loading, and forward/backward pass per epoch with three GNN systems: DGL, Quiver, and P3*.
We will initially focus on DGL, which is a standard data-parallel baseline, and discuss optimizations shortly.
% We observe that data loading can take up to 69\% of the epoch time for DGL, 48\% for P3*, and 79\% for Quiver. 
We observe that data loading can take more than 60\% of the epoch time for DGL. 
This data loading-induced performance bottleneck is also reported in other GNN training literature~\citep{quiver, pagraph, wholegraph,ugache}. 

\mypar{Redundant loading and computation}
Table~\ref{tab:redundancy} further reports the degree of computational and data loading redundancy in data-parallel training.
With 4 GPUs, data parallelism creates 4 separate micro-batches (``Micro''), causing up to $1.2\times$ compute and $2.5\times$ feature loading compared to having only a single mini-batch (``Mini'').

% \begin{table}[t]
% \centering 
% \tabcolsep=0.08cm
% \begin{tabular}{|l||c|c|c|}
% \hline 
% \textbf{Dataset} & \textbf{4x Micro} & \textbf{1x Mini} & \textbf{\% redundancy} \\ \hline \hline
% % ogbn-products & 195M& 155M & 25.5\% \\
% orkut & 926M & 751M & 23\% \\
% % papers100M & 327M & 131M & 148.9\% \\
% papers100M & 452M & 389M & 16\% \\
% % friendster & 452M & 389M & 16\% \\
% % amazon & 473M & 263M & 79.2\% \\
% \hline
% \end{tabular}
% \caption{Redundant computation. The total number of edges computed over one epoch when each mini-batch is sampled as 4 micro-batches of size 1024 (\texttt{4x Micro}) vs. 1 mini-batch of size 4096 (\texttt{1x Mini}). \ms{Numbers for the other graphs?}}
% \label{tab:overlap-edges} 
% \end{table}

\begin{table}[ht]
\centering
% \resizebox{\columnwidth}{!}{%
\small 
\tabcolsep=0.02cm
\begin{tabular}{|c|ccc|ccc|}
\hline
\multirow{2}{*}{\textbf{Graph}} & \multicolumn{3}{c|}{\textbf{\# Edges Computed}}                                                 & \multicolumn{3}{c|}{\textbf{\# Feature Vectors Loaded}}                                                \\ \cline{2-7} 
                                & \multicolumn{1}{c|}{\textbf{ Micro}} & \multicolumn{1}{c|}{\textbf{ Mini}} & \textbf{Ratio} & \multicolumn{1}{c|}{\textbf{ Micro}} & \multicolumn{1}{c|}{\textbf{ Mini}} & \textbf{Ratio} \\ \hline
\textbf{Orkut}                  & \multicolumn{1}{c|}{926M}              & \multicolumn{1}{c|}{751M}             & 1.2x           & \multicolumn{1}{c|}{422M}              & \multicolumn{1}{c|}{169M}             & 2.5x          \\ \hline
\textbf{Papers100M}             & \multicolumn{1}{c|}{452M}              & \multicolumn{1}{c|}{389M}             & 1.2x          & \multicolumn{1}{c|}{231M}              & \multicolumn{1}{c|}{154M}             & 1.5x           \\ \hline
\textbf{Friendster}             & \multicolumn{1}{c|}{13.4B}             & \multicolumn{1}{c|}{13.1B}            & 1.0x          & \multicolumn{1}{c|}{11.4B}             & \multicolumn{1}{c|}{9.4B}             & 1.2x           \\ \hline
\end{tabular}%
% }
\caption{Redundant computation and data loading. The total number of edges computed and feature data loaded over one epoch when each mini-batch is sampled as 4 micro-batches of size 1024 (Micro) vs. 1 mini-batch of size 4096 (Mini).}
\vspace{-0.2cm}
\label{tab:redundancy}
\end{table}




% For example, in Figure~\ref{fig:comparison}(a), although there are only two target vertices in that mini-batch, a total of eight input features need to be loaded into the GPU memory for forward and backward propagation on a 2-layer GNN. 

% The data loading phase can take up a significant portion of GNN training time and even become the performance bottleneck, depending on the system, the graph characteristics, and the GNN model.


\subsection{Limitations of Existing Optimizations} \label{sec:problem}

\begin{comment}
Table~\ref{tab:loading-bottleneck} shows the time spent on the three phases, sampling, data loading, and training, when training two GraphSAGE and GAT models on two graph datasets, Papers100M and Orkut, using DGL~\citep{dgl} on a four-GPU server with NVLink.
DGL is one of the well-established GNN training frameworks that support mini-batch data parallel training. 
Details on the experiment settings are in Section~\ref{sec:settings}.
% DGL doesn't support caching only a part of a graph in GPU memory and both graphs we consider are too big to fit. 
From the table, we observe that data loading can take up to XX\% of the epoch time. \ms{update number}
This data loading-induced performance bottleneck is also reported in other GNN training literature~\citep{quiver, pagraph}. 

\begin{table*}[t]
\small 
\begin{tabular}{|c||c|c||c|c|c|c||c|c|c|c|}
\hline 
\multirow{2}{*}{ Graph }  & \multirow{2}{*}{ System}  & \multirow{2}{*}{ Cache \%} &\multicolumn{4}{|c||}{SAGE} & \multicolumn{4}{|c|}{GAT} \\
\cline{4-11}
 & & & S  & L  & FB & Total & S  & L  & FB  & Total  \\  
\hline \hline  
 & DGL & no cache & 6.57 & 14.46 & 11.89 & 32.92 & 6.36 & 14.42 & 31.36 & 52.14\\
Papers100M & Quiver & SAGE: 51\% - GAT: 44\%  &11.17 & 17.23 & 8.99 & 37.38 & 11.19 & 22.09 & 25.06 & 58.33 \\
& $P^{3*}$ & OOM & OOM & OOM & OOM & OOM & OOM & OOM & OOM & OOM \\
\hline
& DGL & no cache &0.96 & 96.59 & 13.29 & 110.84 & 0.98 & 96.32 & 17.55 & 114.85 \\
Orkut & Quiver & 100\% & 4.08 & 6.85 & 12.81 & 23.74 & 3.83 & 6.91 & 17.44 & 28.18\\
& $P^{3*}$ & 100\% & 0.97 & 2 & 15.72 & 18.68 & 1.01 & 1.83 & 25.89 & 28.73 \\
\hline
\end{tabular}
\caption{Data loading bottleneck on a system with NVLink. 
\ms{Explain S/L/T once finalized. make it look less similar to the evaluation results (since they are the same numbers). }
}
\label{tab:loading-bottleneck}
\end{table*}
% 
\end{comment}

% \begin{figure}
%     \centering
%     \includegraphics[width=0.49\columnwidth]{results/epoch_breakdown/Papers100M_epoch_breakdown.png}
%     \includegraphics[width=0.49\columnwidth]{results/epoch_breakdown/Friendster_epoch_breakdown.png}

%      \caption{The fraction of sampling, loading, and training time per epoch of existing systems when training GAT on a 4 GPU host with NVLink. 
%      %DGL and P3 cache no feature data in all experiments. Quiver caches 44\% for Papers100M, and 12\% percent for Friendster.  All systems use the GAT model with fanouts [15, 15, 15] and batch size 1024. 
%      }
%     \label{fig:epoch_breakdown}
% \end{figure}


% \subsection{Existing Optimizations} \label{sec:limitations}
% \mypar{Existing optimizations}
Many approaches have been proposed to address the data loading-induced performance bottleneck of mini-batch training. 
These approaches fall broadly into three categories: caching~\citep{pagraph, gnnlab, quiver, wholegraph}, hybrid parallelism~\citep{gandhi2021p3}, and algorithmic optimizations~\citep{dong2021global,twolevel, ramezani2020gcn,liu2023bgl}. 
In this work, 
% we do not consider algorithmic optimizations that impose approximation choices onto the users and impact model accuracy. 
we focus on optimizations that can be used in general-purpose GNN training systems that scale to multiple GPUs without imposing modeling choices that can impact the model accuracy or semantics, such as using specific sampling algorithms or relaxing synchrony.

We now discuss caching and hybrid parallelism approaches 
% for single-host mini-batch training 
and show that they still suffer from high data loading costs.
Besides caching, none of these solutions addresses the fundamental problem of redundant loading and computation that is inherent in data parallelism.

\mypar{Limitations of data-parallel caching}
To reduce data loading time, several systems maintain a static cache in the main memory of the GPUs. 
This cache is populated offline with frequently-accessed input features~\citep{pagraph, gnnlab}.
The latest systems use a distributed shared memory to enable GPUs to fetch features from other GPUs' memory using fast GPU-to-GPU interconnects like NVLink \citep{quiver, dsp, ugache}.
As shown in Figure~\ref{fig:orkut_epoch_breakdown}, the distributed shared-memory caching mechanism in Quiver~\citep{quiver} can reduce loading time for the Orkut graph, whose features are too large to fit in a single GPU memory but can be fully cached across multiple GPUs. 

Distributed caching, however, does not fully address the performance issue of data loading, which can still take a significant fraction of the epoch time as shown in Figure~\ref{fig:quiver_epoch_breakdown}.
For the GraphSage model, data loading over NVLink can be relatively expensive for Quiver with a graph that can be fully cached like Orkut.
For larger graphs such as Papers100M, only a part of input features can be cached on GPUs. 
Data loading can still stress the PCIe bus between the host and devices, resulting in unsatisfactory training performance. 
With Papers100M, only up to 60\% of the input features can be cached and the data loading time remains high.
%We fixed this bug of Quiver and termed the improved system DistCache. DGL can still outperform DistCache}
%Even with Orkut, which can be fully cached in the distributed GPU memory, Quiver still needs to spend up to XX\% of the time loading data.\ms{update numbers}

\begin{comment}
Table~\ref{tab:loading-bottleneck} reports the maximum percentage of features that can be cached on a four-GPU server with NVLink interconnect. 
With Friendster, only up to XX\% of the input features can be cached and the data loading time remains high (up to 14\% of the epoch time).  
Even with Orkut, which can be fully cached in the distributed GPU memory, Quiver still needs to spend up to XX\% of the time loading data.\ms{update numbers}
\end{comment}


\begin{comment}    
These experiments consider a system equipped with NVLink.
Without NVLink, data loads from the host memory or other GPUs always need to occur over PCIe.
This makes distributed GPU caching not effective, as we will show in our experiments.
\ms{Review this later. PCIe numbers are still inconclusive.}
\end{comment}

% % \begin{table}[t]
% \centering 
% \tabcolsep=0.08cm
% \begin{tabular}{|l||c|c|c|}
% \hline 
% \textbf{Dataset} & \textbf{4x Micro} & \textbf{1x Mini} & \textbf{\% redundancy} \\ \hline \hline
% % ogbn-products & 195M& 155M & 25.5\% \\
% orkut & 926M & 751M & 23\% \\
% % papers100M & 327M & 131M & 148.9\% \\
% papers100M & 452M & 389M & 16\% \\
% % friendster & 452M & 389M & 16\% \\
% % amazon & 473M & 263M & 79.2\% \\
% \hline
% \end{tabular}
% \caption{Redundant computation. The total number of edges computed over one epoch when each mini-batch is sampled as 4 micro-batches of size 1024 (\texttt{4x Micro}) vs. 1 mini-batch of size 4096 (\texttt{1x Mini}). \ms{Numbers for the other graphs?}}
% \label{tab:overlap-edges} 
% \end{table}

\begin{table}[ht]
\centering
% \resizebox{\columnwidth}{!}{%
\small 
\tabcolsep=0.02cm
\begin{tabular}{|c|ccc|ccc|}
\hline
\multirow{2}{*}{\textbf{Graph}} & \multicolumn{3}{c|}{\textbf{\# Edges Computed}}                                                 & \multicolumn{3}{c|}{\textbf{\# Feature Vectors Loaded}}                                                \\ \cline{2-7} 
                                & \multicolumn{1}{c|}{\textbf{ Micro}} & \multicolumn{1}{c|}{\textbf{ Mini}} & \textbf{Ratio} & \multicolumn{1}{c|}{\textbf{ Micro}} & \multicolumn{1}{c|}{\textbf{ Mini}} & \textbf{Ratio} \\ \hline
\textbf{Orkut}                  & \multicolumn{1}{c|}{926M}              & \multicolumn{1}{c|}{751M}             & 1.2x           & \multicolumn{1}{c|}{422M}              & \multicolumn{1}{c|}{169M}             & 2.5x          \\ \hline
\textbf{Papers100M}             & \multicolumn{1}{c|}{452M}              & \multicolumn{1}{c|}{389M}             & 1.2x          & \multicolumn{1}{c|}{231M}              & \multicolumn{1}{c|}{154M}             & 1.5x           \\ \hline
\textbf{Friendster}             & \multicolumn{1}{c|}{13.4B}             & \multicolumn{1}{c|}{13.1B}            & 1.0x          & \multicolumn{1}{c|}{11.4B}             & \multicolumn{1}{c|}{9.4B}             & 1.2x           \\ \hline
\end{tabular}%
% }
\caption{Redundant computation and data loading. The total number of edges computed and feature data loaded over one epoch when each mini-batch is sampled as 4 micro-batches of size 1024 (Micro) vs. 1 mini-batch of size 4096 (Mini).}
\vspace{-0.2cm}
\label{tab:redundancy}
\end{table}

\begin{figure*}[ht]
    \centering
    \includegraphics[width=\textwidth]{figures/overview.drawio.pdf}
    \vspace{-5mm}
    \caption{Overview of the \name training pipeline.}
    \label{fig:overview}
\end{figure*}

\mypar{Limitations of existing hybrid parallelism} 
The alternative to data-parallelism for mini-batch GNN training is called \emph{push-pull parallelism} proposed by the P3 system~\citep{gandhi2021p3}.
P3 targets distributed multi-host systems and aims to avoid transferring input features among hosts.
Each host keeps a slice of the feature vector of each vertex in host memory.
Like in data-parallel training, each GPU is associated with a micro-batch.
However, in push-pull parallelism, each GPU computes the input layer of \emph{all} micro-batches on its feature slice.
GPUs then exchange partial activations and continue the iteration in a data-parallel fashion.

$P^3$ is a multi-host system that does not use GPUs for caching, so it was not previously evaluated in a single-host multi-GPU setting.
To fill this gap, we have implemented its push-pull parallelism approach in a single-host multi-GPU setting.
We call this implementation P3*.
Figure~\ref{fig:comparison}(b) illustrates how P3* applies hybrid parallelism. 
Like the original $P^3$, this implementation partitions the cached features among GPUs at the cost of a cross-GPU push-pull shuffle.
For the Orkut graph, which can be fully cached, P3* does not exchange features among GPUs during the loading time, as Quiver does.
Instead, it pushes the bottom layer of all micro-batches to all GPUs, which induces a much lower loading cost.
The additional overhead of push-pull shuffling outweighs the gains in terms of loading time during the forward and backward pass, which results in higher overall training costs (see Figure~\ref{fig:orkut_epoch_breakdown}).
For other graphs that cannot be fully cached, P3* loads all the features in the mini-batch.
The training time still is higher than the other two systems due to the cost of shuffling.

\begin{comment}
\mypar{Implications for Training Cost}
One of the most important reasons for optimizing the training time of ML models is to reduce the dollar cost of training.
This depends not only on the training time but also on the type of server that is used for training.
GNN models are very small and they can be easily replicated at each GPU, unlike for example large language models that need to be partitioned among multiple GPUs.
For example, the models we consider range around XXX \ms{report size}.
The computation in GNN training is also relatively lightweight.
Therefore, the benefit of using high-end servers with NVLink mainly comes from mitigating the data loading bottleneck.
However, are these speedups sufficient to justify the additional cost of using high-end GPU servers for GNN training?

To answer this question, we consider the cost of the two AWS instance types we used for our evaluation and evaluate the cost per epoch of each system.
We consider a p3.8xlarge instance with NVLink, which currently costs \$12.24 per hour and a g4dn.12xlarge without NVLink, which costs \$3.912 per hour.
The cost per epoch is a good metric to measure the overall training cost because all the systems we consider run the models unmodified using synchronous training and have similar convergence rates per epoch, as we validated experimentally.

Figure XXX shows the cost per epoch achieved by different systems. 
For the Papers100M graph, which cannot be entirely cached in the distributed GPU cache, running DGL on an instance \emph{without} NVLink results in a much lower cost per epoch than using a caching-based system on NVLink.
Using a cheaper instance is preferable unless minimizing the absolute training time is the main goal.

The Orkut graph can be entirely cached in the distributed GPU cache.
In this case, using NVLink has a similar cost per epoch as PCIe, so using a high-end server is preferable since it can speed up training at no extra cost.
\end{comment}

% \mypar{Summary and motivation}
% Data-parallel training suffers from a fundamental problem, which is \emph{redundant data loading}. 
% In each training iteration of data-parallel training, two micro-batches (prepared for two GPUs separately) can share the same input vertices due to the interconnected nature of graphs. 
% The input features of these overlapped vertices need to be loaded twice, one for each GPU. 
% Caching speeds up some of these data loads but it does not fundamentally solve the redundant loading problem.
% The push-pull parallelism approach proposed by P3 eliminates redundant data loads but it introduces an expensive shuffle during training that adds a significant cost to the end-to-end training time.
% This cost often outweighs its benefits compared to data parallelism with caching.


\section{Sparse Latent Point Diffusion Models}
\label{sec:Method}
\begin{figure*}[t]
    \centering
    \includegraphics[width=\linewidth]{fig/ModelStructure.pdf}
    \caption{Overall architecture of \name. The left part represents different modality-specific encoders to extract latent features and the multimodal fusion module to integrate multimodal representations. The right part represents the contextual relational model decoders to get the similarity score and the decision fusion module to make the final prediction on all modalities.}
    \label{fig:model}
\end{figure*}

\section{Methodology}

Formally, a knowledge graph is defined as $\mathcal{G} = \langle \mathcal{E}, \mathcal{R}, \mathcal{T} \rangle$, where $\mathcal{E}$ and $\mathcal{R}$ indicate sets of entities and relations, respectively. 
$\mathcal{T} = \{(h, r, t) | h, t \in \mathcal{E}, r \in \mathcal{R}\}$ represents relational triples of the KG.
In multimodal KGs, each entity in KGs is represented by multiple features from different modalities.
Here, we define the set of modalities $\mathcal{K} = \{s, v, t, m\}$ where $s, v, t, m$ denote structural, visual, textual and multimodal modality, respectively.
Due to the complexity of real-world knowledge, it is almost impossible to take all the triples into account.
Therefore, given a well-formulated KG, the \emph{Link Prediction} task aims at predicting missing links between entities.
Specifically, link prediction models expect to learn a score function of relational triples to estimate the likelihood of a triple, which is always formulated as $\psi : \mathcal{E} \times \mathcal{R} \times \mathcal{E} \to \mathbb{R}$.


\subsection{Overall Architecture}

In order to fully exploit the complicated interaction between different modalities, we propose a two-stage fusion model instead of simply considering the multimodal information separately in a unified vector space.
As shown in Figure~\ref{fig:model}, \name consists of four key components:
\begin{itemize}[leftmargin=*]
	\item[1] The Modality-Specific Encoders are used for extracting structural, visual and textual features as the input of multimodal fusion stage.
	\item[2] The Multimodal Fusion Module, which is the first fusion stage, effectively models bilinear interactions between different modalities based on \textit{Tucker} decomposition and contrastive learning.
	\item[3] The Contextual Relational Model calculates the similarity of contextual entity representations to formulate triple scores as modality-specific predictions for decision fusion stage.
	\item[4]  The Decision Fusion Module, which is the second fusion stage, takes all the similarity scores from structural, visual, textual and multimodal models into account to make the final prediction.
\end{itemize}

\subsection{Modality-Specific Encoders}
In this subsection, we first introduce the pre-trained encoders used for different modalities.
These encoders are not fine-tuned during training and we treat them as fixed feature extractors to obtain the modality-specific entity representations.
Note that \name is a general framework and it is straightforward to replace them with other up-to-date encoders or add ones for new modalities into \name.

\subsubsection{Structural Encoder}

From the most basic view, the structural information of KG, we employ a Graph Attention Network (GAT)\footnote{https://github.com/Diego999/pyGAT}~\cite{DBLP:conf/iclr/VelickovicCCRLB18} with TransE loss.

Specifically, our GAT encoder takes L1 distance of neighbor aggregated representations as energy function of triples, which is $E(h, r, t) = ||\mathbf{h}+\mathbf{r}-\mathbf{t}||$.
In the training process, we minimize the following Hinge loss~\eqref{eq-gat-loss}:
\begin{equation}\label{eq-gat-loss}
    \begin{split}
        \mathcal{L}_{GAT} = & \sum_{(h,r,t) \in \mathcal{T}}\sum_{(h', r, t') \in \mathcal{T'}} \mathrm{max} \{0,  \\
        &\gamma + E(h,r,t) - E(h',r,t')\}
    \end{split}
\end{equation}
where $\gamma$ is margin hyper-parameter and $\mathcal{T'}$ denotes set of negative triples derived from $\mathcal{T}$. 
$\mathcal{T'}$ is created by randomly replacing head or tail entities of triples in $\mathcal{T}$, which is~\eqref{eq-gat-neg}:
\begin{equation}\label{eq-gat-neg}
    \mathcal{T'} = \{(h',r,t)|h' \in \mathcal{E} \backslash h\} \cup \{(h,r,t')|t' \in \mathcal{E} \backslash t\}
\end{equation}

\subsubsection{Visual Encoder} 
Visual features are greatly expressive while providing different views of knowledge from traditional KGs. 
To effectively extract visual features, we utilize VGG16\footnote{https://github.com/machrisaa/tensorflow-vgg} pre-trained on \textit{ImageNet}\footnote{https://image-net.org/} to get image embeddings of corresponding entities following~\cite{DBLP:conf/esws/LiuLGNOR19}.
Specifically, we take outputs of the last hidden layer before softmax operation as visual features, which are 4096-dimensional vectors.

\subsubsection{Textual Encoder} 
Entity descriptions contain much richer but more complex knowledge than pure KGs.
To fully extract the complex knowledge, we employ BERT~\cite{DBLP:conf/naacl/DevlinCLT19} as the textual encoder, which is very expressive to get description embeddings of corresponding entities.
The textual features are 768-dimensional vectors, i.e., pooled outputs of pre-trained BERT-Base model\footnote{https://github.com/huggingface/transformers}.

\subsection{Multimodal Fusion}
The multimodal fusion stage aims to effectively get multimodal representations, which fully capture the complex interactions between different modalities.
Many existing multimodal fusion methods have achieved promising results in many tasks like VQA (Visual Question Answering).
However, most of them aim at finding the commonality to get more precise representations by modality projecting~\cite{DBLP:conf/nips/FromeCSBDRM13,DBLP:conf/aaai/CollellZM17} or cross-modal attention~\cite{DBLP:conf/aaai/PerezSVDC18}.
These types of methods will suffer from the loss of unique information in different modalities and can not achieve sufficient interaction between modalities.
To this end, we propose to employ the bilinear models, which have a strong ability to realize full parameters interaction as the cornerstone to perform the fusion of multimodal information.
Specifically, we extend the \textit{Tucker} decomposition, which decomposes the tensor into a core tensor transformed by a matrix along with each mode to 4-mode factors as expressed in Equation~\eqref{eq-tucker}:
\begin{equation}\label{eq-tucker}
    \mathcal{P} = (((\mathcal{P}_c \times \mathbf{M}_s) \times \mathbf{M}_v) \times \mathbf{M}_t) \times \mathbf{M}_d
\end{equation}
where $\mathbf{M}_s \in \mathbb{R}^{d_s \times t_s}$, $\mathbf{M}_v \in \mathbb{R}^{d_v \times t_v}$, $\mathbf{M}_t \in \mathbb{R}^{d_t \times t_t}$,  $\mathbf{M}_d \in \mathbb{R}^{\mathcal{D} \times t_d}$ denotes transformation matrix and $\mathcal{P}_c \in \mathbb{R}^{t_s \times t_v \times t_t \times t_d}$ denotes a smaller core tensor.

In such a situation, entity embeddings are first projected into a low-dimensional space and then fused with the core tensor $\mathcal{P}_c$.
Following~\cite{DBLP:conf/iccv/Ben-younesCCT17}, we further reduce the computation complexity by decomposing the core tensor $\mathcal{P}_c$ to merge representations of all modalities into a unified space with element-wise product.
The detailed calculation process is expressed as Equation~\eqref{eq-fusion}:
\begin{equation}\label{eq-fusion}
    \mathbf{e}_m = \tilde{\mathbf{e}}_s^\mathsf{T} \mathbf{M}_d^s * \tilde{\mathbf{e}}_v^\mathsf{T} \mathbf{M}_d^v * \tilde{\mathbf{e}}_t^\mathsf{T} \mathbf{M}_d^t
\end{equation}
where $\tilde{\mathbf{e}}_k = \mathrm{ReLU}(\mathbf{e}_k\mathbf{M}_k) \in \mathbb{R}^{t_k}$ denotes latent representations and $\mathbf{e}_k \in \mathbb{R}^{d_k}$ is the original embedding representations and $\mathbf{M}_d^k \in \mathbb{R}^{t_k \times t_d}$ is decomposed transformation matrix for each modality $k \in \{s, v, t\}$.

However, the multimodal bilinear fusion has no bound limitation while the gradient produced by the final prediction result can only implicitly guide parameter learning.
To alleviate this problem, we add constraints to limit the correlation between different modality representations of the same entity to be stronger.
Therefore, we further leverage contrastive learning~\cite{DBLP:conf/icml/ChenK0H20,DBLP:conf/nips/LiSGJXH21,DBLP:conf/cvpr/Yuan0K0WMKF21} between different entities and modalities as an additional learning objective for regularization.
In the settings of contrastive learning, we take the pairs of representations of the same entity of different modalities as positive samples and the pairs of representations of different entities as negative samples.
As shown in Figure~\ref{fig:cl}, we aim at limiting the distance of negative samples to be larger than positive samples to enhance multimodal fusion, which is:
\begin{equation}
    d(f(x), f(x^+)) << d(f(x), f(x^-))
\end{equation}
where $d(\cdot, \cdot)$ denotes the distance measure and $f(\cdot)$ denotes the embedding function. The superscript $+, -$ represent the positive and negative samples, respectively.

\begin{figure}
    \centering
    \includegraphics[width=\linewidth]{fig/ContrastiveLearning.pdf}
    \caption{Example of multimodal contrastive learning. The distance between the representations of the same entity in different modalities is minimized, while the distance between the representations of different entities is maximized.}
    \label{fig:cl}
\end{figure}

Specifically, we randomly sample $N$ entities from the entity set as a minibatch and define contrastive learning loss upon it.
The positive pairs are naturally obtained with the same entities while the negative pairs are constructed by negative sharing~\cite{DBLP:conf/kdd/ChenSSH17} of all other entities.
We take the latent representations $\tilde{\mathbf{e}}_k = \mathrm{ReLU}(\mathbf{e}_k\mathbf{M}_k) \in \mathbb{R}^{t_k}$ and leverage cosine similarity $d(u, v) = - \mathbf{u}^\mathsf{T}\mathbf{v}/||\mathbf{u}||\mathbf{v}||$ as distance measure.
Then we have the following contrastive loss function for each entity $i$:
\begin{equation}\label{eq-cl}
    \mathcal{L}_{CLi} = \frac{1}{3N} \sum_{p,q \in \mathcal{M}} \sum_{j=1}^N  d(e_i^{p}, e_i^{q}) - d(e_i^{p}, e_j^{q}) + 2
\end{equation}
where $\mathcal{M} = \{(s, v), (s, t), (v, t)\}$ is set of modality pairs.

\subsection{Contextual Relational Model}
After obtaining representations of each modality and multimodal, we then design a contextual relational model, which takes relations in triples as contextual information for scoring, to get the predictions.
Note that this relational model can be easily replaced by any scoring function like TransE.

Due to the variety and complexity of relations in KGs, we argue that improving the degree of parameter interaction~\cite{DBLP:conf/aaai/VashishthSNAT20} is crucial for better modeling the relational triples.
The degree of parameter interaction means the calculation ratio of each parameter to all other parameters. 
For example, dot product could achieve $1/d$ degree while cross product could achieve $(d-1)/d$ degree.
Based on this assumption, we propose to use bilinear outer product between entity and relation embeddings to incorporate contextual information into entity representations.
Instead of taking relations as input as in previous studies, our contextual relational model utilizes relations to provide context in the transformation matrix of entity embeddings.
Then, entity embeddings are projected using the contextual transformation matrix to get \emph{contextual embeddings}, which are used for calculating similarity with all candidate entities.
The learning objective is to minimize the binary cross-entropy loss.
For each modality $k \in \mathcal{K}$, the computation details are shown as Equation~\eqref{eq-crm} to Equation~\eqref{eq-loss}:
\begin{gather}
    \hat{\mathbf{e}}_k = \mathbf{e}_k^\mathsf{T}\mathbf{W}_k^r  + \mathbf{b} = \mathbf{e}_k^\mathsf{T}\mathbf{W}_k\mathbf{r} + \mathbf{b}_k \label{eq-crm} \\
    \mathbf{y}_k = \sigma(\mathrm{cosine}(\mathbf{e}_k, \hat{\mathbf{e}}_k)) = \sigma (\frac{\mathbf{e}_k \cdot \hat{\mathbf{e}}_k}   
    {|\mathbf{e}_k| |\hat{\mathbf{e}}_k|}) \label{eq-sim} \\
    \mathcal{L}_k = -\frac{1}{N} \sum_{i=1}^N (t_i \cdot \mathrm{log}(y_{i,k})+(1-t_i) \cdot \mathrm{log}(1-y_{i,k})) \label{eq-loss}
\end{gather}
where $\mathbf{e}_k$ and $\hat{\mathbf{e}}_k$ are original and contextual entity embeddings respectively;
$\mathbf{W}_k^r = \mathbf{W}_k \mathbf{r}$ denotes contextual transformation matrix which is obtained by matrix multiplication of weight matrix $\mathbf{W}_k$ and relation vectors $\mathbf{r}$ while $\mathbf{b}_k$ is a bias vector;
$\sigma$ is sigmoid function and $\mathbf{y}_k = [y_{1,k},y_{2,k},...,y_{N,k}]$ is final prediction of modality $k$.

\subsection{Decision Fusion}
Existing multimodal approaches mainly focus on projecting different modality representations into a unified space and predicting with commonality between modalities, which will fail to preserve the modality-specific knowledge.
We alleviate this problem in the decision fusion stage by joint learning and combining predictions of different modalities to further leverage the complementarity.

Under the multimodal settings, we assign different contextual relational models for each modality and utilize their own results for training in different views.
Recall the contrastive learning loss in Equation~\eqref{eq-cl}, the overall training objective is to minimize the joint loss shown in Equation~\eqref{eq-mmloss}:
\begin{equation}\label{eq-mmloss}
    \mathcal{L}_{Joint} = \gamma_s \mathcal{L}_s + \gamma_v \mathcal{L}_v + \gamma_t \mathcal{L}_t + \gamma_m \mathcal{L}_{m} + \mathcal{L}_{CL}
\end{equation}
where $\mathcal{L}_k$ denotes binary cross entropy loss for modality $k$ as Equation~\eqref{eq-loss} and $\gamma_k$ is a learned weight parameter.

\begin{algorithm}[t]
\caption{Optimization Algorithm.}\label{alg:optim}
\begin{algorithmic}[1]
\STATE \textbf{Input:} Multimodal Knowledge Graph $\mathcal{G}$
\STATE \textbf{Output:} Trained Model $\mathcal{M}$
\STATE Pre-train structural encoder GAT on $\mathcal{G}$ with the loss in Equation(1)
\STATE Obtain pre-trained visual encoder VGG16 and textual encoder BERT-base
\STATE Initialize the entity embeddings $\mathbf{E}_s, \mathbf{E}_v, \mathbf{E}_t$ in $\mathcal{M}$ with the outputs of pre-trained encoders
\WHILE{not converge}
    \STATE Sample a batch of entities from $\mathcal{G}$
    \FOR{Entity $e$ in batch}
    \STATE Obtain the structural, visual, textual embeddings $\mathbf{e}_s, \mathbf{e}_v, \mathbf{e}_t$ of entity $e$
    \STATE Compute the multimodal fused embeddings $\mathbf{e}_m$ of entity $e$ with Equation (4)
    \STATE Compute the contrastive learning loss $\mathcal{L}_{CL}$ with Equation (6)
    \STATE Compute the loss $\mathcal{L}_s, \mathcal{L}_v, \mathcal{L}_t, \mathcal{L}_m$ with modality-specific scorers via Equation (7) - Equation (9)
    \STATE Compute the joint loss $\mathcal{L}_{Joint}$ with the above losses $\mathcal{L}_s, \mathcal{L}_v, \mathcal{L}_t, \mathcal{L}_m, \mathcal{L}_{CL}$ via Equation (10)
    \STATE Update model parameters of $\mathcal{M}$ by minimizing $\mathcal{L}_{Joint}$
    \ENDFOR
\ENDWHILE
\RETURN $\mathcal{M}$
\end{algorithmic}
\end{algorithm}

To better illustrate the whole training process of \name, we describe it via the pseudo-code of the optimization algorithm.
As shown in Algorithm~\ref{alg:optim}, we first obtain the pre-trained encoders of structural, visual and textual and utilize them for entity embeddings (line 3-5).
Since the pre-trained models are much larger and more complex than \name, they are not fine-tuned and their outputs are directly used as inputs of \name.
The multimodal embeddings are obtained by multimodal fusion while contrastive learning is applied to further enhance the fusion stage (line 9-11).
During training, each modality delivers its own prediction and loss via the modality-specific scorers (line 12), and then the joint prediction and loss are computed based on all modalities including multimodal ones (line 14).

For inference, we propose to jointly consider the predictions of each modality as well as multimodal ones.
Specifically, the overall predictions are shown in Equation~\eqref{eq-df}:
\begin{equation}\label{eq-df}
    \mathbf{y}_{Joint} = \frac{\gamma_s \mathbf{y}_s + \gamma_v \mathbf{y}_v + \gamma_t \mathbf{y}_t + \gamma_m \mathbf{y}_m} {\gamma_s + \gamma_v + \gamma_t + \gamma_m}
\end{equation}
where $\gamma_k$ denotes weight for modality $k$ as same as Equation~\eqref{eq-mmloss} while the values in $\mathbf{y}$ are in [0, 1].




\section{Related Work}
\label{sec:related}
% \begin{table*}[tbh!]
\vspace{-2em}
\centering
\caption{MMD comparison of meshes generated by our method and baselines. CD, EMD, and normal consistency (N.C.) losses are multiplied by 1000, 100, and 10, respectively. We find that the N.C. loss computed MMDs best match human evaluations based on examples in Figure~\ref{fig:mesh_comparison}.} 
\label{tbl:mesh_mmd}
\scalebox{0.8}
{
\begin{tabular}{c|ccc|ccc|ccc|ccc|ccc}
\hline
 & \multicolumn{3}{c|}{Airplane} & \multicolumn{3}{c|}{Cabinet} & \multicolumn{3}{c|}{Car} & \multicolumn{3}{c|}{Chair} & \multicolumn{3}{c}{Lamp}\\ \cline{2-16}
  & CD & EMD & N.C. & CD & EMD & N.C. & CD & EMD & N.C. & CD & EMD & N.C. & CD & EMD & N.C. \\ 
 \hline
TreeGan&5.01&4.04&3.79&10.51&7.42&4.04&5.17&3.59&3.94&16.1&9.02&5.06&26.68&12.28&6.31\\ShapeGF&4.33&4.03&3.26&9.75&6.49&2.98&4.46&3.10&3.4&14.43&8.53&4.53&19.99&11.04&5.43\\PVD&5.11&4.29&3.42&10.91&7.00&3.09&4.80&3.20&3.45&15.89&8.69&4.40&24.85&12.99&5.11\\DPM&4.46&4.00&4.80&9.67&7.39&4.69&4.99&3.57&4.95&14.09&9.51&6.20&19.63&12.02&6.84\\SPGAN&5.09&4.03&3.39&10.52&7.18&3.04&5.02&3.49&3.54&18.86&9.67&4.68&23.64&12.63&4.87\\
% DDPM&4.54&3.66&3.05&9.76&6.71&2.86&4.46&3.15&3.32&14.37&8.41&4.17&21.7&11.33&4.38\\
% DDPM no normal&4.55&3.64&3.07&9.75&6.43&2.86&4.46&3.12&3.34&14.39&8.37&4.19&21.74&11.27&4.43\\
% Latent DDPM&4.48&3.81&3.08&9.53&6.74&2.75&4.57&3.28&3.21&14.86&8.48&4.17&23.81&11.75&4.33\\
% Latent DDPM no normal&4.46&3.77&3.07&9.52&6.50&2.76&4.55&3.20&3.23&14.76&8.49&4.19&23.61&11.87&4.38\\
\hline
Ours&4.46&3.77&3.07&9.52&6.50&2.76&4.55&3.20&3.23&14.76&8.49&4.19&23.61&11.87&4.38\\

 \hline
\end{tabular}
}
% \vspace{-1em}
\end{table*}
\begin{figure*}[t]
    \vspace{-2em}
    \centering
    \includegraphics[width=1\textwidth]{Figures/method_figures/mesh_comparison.png}
    \vspace{-2em}
    \caption{Mesh generated by our methods and baselines. 
    % The meshes are reconstructed from the generated point clouds using SAP~\cite{peng2021shape}. 
    We can see that meshes generated by our method are more visually appealing. More examples of other baselines and our method are provided in Appendix B.5 and B.8.}
    \label{fig:mesh_comparison}
    \vspace{-1.5em}
\end{figure*}

\begin{figure}[t]
    \centering
    \includegraphics[width=0.5\textwidth]{Figures/method_figures/car_pc.png}
    \caption{Point clouds generated by our method and baselines. More examples are provided in Appendix B.7 and B.8.}
    \label{fig:car_pc}
    \vspace{-1em}
\end{figure}

% \vspace{-1em}
\paragraph{Mesh Generation.}
Most existing mesh generation methods rely on deforming a template mesh or another mesh~\cite{wang2018pixel2mesh, wen2019pixel2mesh++, gupta2020neural, liu2021deepmetahandles, yifan2020neural, jakab2021keypointdeformer, jiang2020shapeflow}, but meshes generated in this way are usually limited by the topology of the template or the initial mesh. And large deformations could cause defects. In contrast, our method is able to generate meshes from scratch with diverse topologies. 
Another line of works uses implicit representations of 3D shapes~\cite{DBLP:conf/cvpr/ChenZ19, park2019deepsdf, sitzmann2020implicit, genova2020local, zheng2022sdf, kleineberg2020adversarial,Chen_2021_ICCV, mittal2022autosdf}, but it usually requires dense neural network evaluations to extract meshes from the learned model.

\vspace{-1.5em}
\paragraph{Point cloud generation.}
Many learning-based methods are proposed to model the distribution of point clouds.
Some works use generative adversarial networks (GANs) to generate point clouds~\cite{DBLP:conf/iccv/TreeGan,li2021sp,achlioptas2018learning,li2018point}.
\cite{achlioptas2018learning} also trains a latent GAN in the latent space of a point cloud autoencoder, but the autoencoder they use can only encode a point cloud to a global feature.
Other works~\cite{yang2019pointflow,DBLP:conf/nips/SoftFlow,DBLP:conf/eccv/DPF-Net} use normalizing
flows to model the distribution of point clouds.
ShapeGF~\cite{DBLP:conf/eccv/CaiYAHBSH20} learns gradient fields to move randomly sampled points to the surface of the objects.
DDPMs have also been applied to point cloud generation~\cite{luo2021diffusion, zhou20213d}.
The generated point clouds of these methods can be transformed to meshes through surface reconstruction techniques~\cite{hanocka2020point2mesh,wei2021deep,chen2022neural,jiang2020shapeflow, williams2019deep, chibane2020implicit, gao2020learning, shen2021deep}.
In this work, we choose SAP~\cite{peng2021shape} for surface reconstruction for its efficiency and reconstruction quality.
\vspace{-1.5em}
\paragraph{Diffusion models.}
DDPMs are a kind of likelihood-based generative model that generate samples by gradually denoising a Gaussian noise~\cite{ho2020denoising,sohl2015deep}.
They have shown promising results for 3D point cloud generation~\cite{luo2021diffusion, zhou20213d}.
Our work is based on the recently proposed
latent diffusion models~\cite{rombach2022high, vahdat2021score}.
Latent diffusion models train diffusion models in the latent space of an autoencoder that encodes data samples to a more compact representation, and thus makes the training and sampling process of DDPMs faster.

\vspace{-1.5em}
\paragraph{Concurrent works.}
The concurrent work, LION~\cite{zeng2022lion}, also proposes to use a latent diffusion model to learn the distribution of point clouds and then use SAP~\cite{peng2021shape} to reconstruct meshes from point clouds, but the latent point cloud representation they use is a noisy point cloud with the same number of points as the original clean point cloud.
In contrast, we encode the original clean point cloud to a sparse set of latent points with features, which is a more compact representation and thus leads to faster training and sampling for DDPMs.
This representation also enables us to perform controllable generation using the sparse latent points.
Concurrently, NVMG~\cite{zheng2022neural} proposes to use voxels as the latent representations of meshes, but computational cost increases rapidly as the resolution of the 3D grid increases.


\section{Experiment}
\label{sec:experiment}

We present our main experiment results in this section. 
We use ShapeNet~\cite{chang2015shapenet} to train our mesh generative model and compare it with other baselines.
We use the pre-processed ShapeNet dataset provided by~\cite{peng2021shape}.
% See more details of the dataset in Appendix B.1.
% and some interesting applications based on our \LDM method. 
The detailed setups and complete experiment results are provided in Appendix B. 

% \subsection{Dataset}
% \label{sec:dataset}
% We use ShapeNet~\cite{chang2015shapenet} to train our mesh generative model and compare with other baselines.
% We use the pre-processed ShapeNet dataset provided by~\cite{peng2021shape}.
% It contains $13$ categories of objects: airplane, bench, cabinet, car,
% chair, display, lamp, loudspeaker, rifle, sofa, table, telephone, and watercraft.
% It splits the dataset into training set and validation set.
% For each shape, it provide $10000$ points sampled from the mesh surface, and the signed distance field (SDF) of the shape discretized on a $128^3$ grid.
% Most shapes in it are normalized in a unit bounding box, namely, the center of the bounding box is placed at the origin, and the maximum length of the three sides of the bounding box is scaled to $1$ such that each shape ranges from $-0.5$ to $0.5$.
% We scale the shapes in the dataset by a factor of $2$ to make them ranges from $-1$ to $1$. 
% before training our generative model and other baselines.

% To fully validate the effectiveness of our two methods, we use the most wildly used dataset, ShapeNet, to benchmark. (To be finished): dataset splits and preprocssing. We use per shape normalization when preprocess the datasets. 

\subsection{Evaluation Metrics}
To evaluate the quality of generated meshes, we uniformly sample point clouds ($2048$ points) with normals from the generated meshes and reference meshes from the validation set. 
Then we use the commonly used point cloud evaluation metrics 1-NN~\cite{yang2019pointflow}, Minimum Matching Distance (MMD) and Coverage (COV) as our main evaluation tools.
All of the metrics require a distance metric to compute the distance between two point clouds.
We use the commonly used Chamfer distance (CD) and earth mover distance (EMD).
We also use the normal consistency loss between two point clouds with normals.
We find that it can better reflect the surface curvature differences between the two underlying meshes.
Details of the normal consistency loss are described in Appendix B.3.
% We also use the normal consistency loss between two point clouds $\mX$ and $\mY$:
% \begin{align*}
%     L_{\text{normal}} = \sum_{\vx \in \mX} [1 - |\cos(\vn_x, \vn_{y^*})|] + \sum_{\vy \in \mY} [1 - |\cos(\vn_{x^*}, \vn_y^)|],
% \end{align*}
% where $\vn_x, \vn_y$ denotes the normal of the points $\vx, \vy$, and $\vy^* = \argmin_{\vy \in \mY} ||\vx-\vy||$, $\vx^* = \argmin_{\vx \in \mX} ||\vx-\vy||$.
% $L_{\text{normal}}$ can be roughly interpreted as the CD loss between the normals of two point clouds. 
% We find it can better reflect the surface curvature differences between the two underlying meshes from which the point clouds are sampled.

% \section{Surface Reconstruction Experiments}
% As mentioned in Section~\ref{sec:pointcloud_representation}, we need to train an upsampling network to upsample the point cloud ($2048$ points) generated by the generative model, then use the Differentiable Poisson Surface Reconstruction (DPSR) method to reconstruct a mesh from the upsampled dense point cloud.
% We use the improved PointNet++ proposed in~\cite{lyu2021conditional} as our upsampling network, and train it following the same procedures in~\cite{peng2021shape} across the all of the $13$ categories in the pre-processed ShapeNet dataset~\cite{peng2021shape}.
% The ground truth SDF values discritized on the $128^3$ grid are used as supervision for the upsampling network.
% The MSE loss between the reconstructed SDF values and the ground truth SDF values discritized on the $128^3$ grid is $4.56 \times 10^{-4}$ tested on the validation set.

% \subsection{Autoencoder Exps}
% We train the autoencoders described in 

% \begin{table*}[tbh!]
% \vspace{-0.5em}
\centering
\caption{Point Cloud: 1-NN(Percentage)} 
\label{tbl:exp1}
\scalebox{0.7}
{
\begin{tabular}{c|cccccccccc}
\hline
 & \multicolumn{2}{c}{Airplane} & \multicolumn{2}{c}{Cabinet} & \multicolumn{2}{c}{Car} & \multicolumn{2}{c}{Chair} & \multicolumn{2}{c}{Lamp}\\ \cline{2-3} \cline{4-5} \cline{6-7} \cline{8-9} \cline{10-11}
  & CD & EMD & CD & EMD & CD & EMD & CD & EMD & CD & EMD \\ 
 \hline
TreeGan&79.95&99.26&71.66&99.68&91.12&97.4&73.86&94.83&69.7&93.72  \\  ShapeGF&64.23&74.01&59.87&62.74&63.08&64.49&54.65&69.35&56.06&56.49  \\  PVD&85.52&81.06&78.66&86.62&82.31&77.17&80.5&78.36&78.35&81.82  \\  DPM&77.48&65.47&68.47&65.29&82.18&69.43&63.15&61.3&66.67&64.29  \\  SPGAN&79.7&80.69&70.38&86.31&81.51&83.58&79.69&83.01&69.26&69.48  \\   \hline 
% DDPM&60.64&51.36&53.5&53.82&63.75&51.87&57.46&54.43&55.41&54.33  \\  
% Latent DDPM kps=0.999&64.85&71.91&54.78&63.69&59.75&65.42&55.17&52.66&53.68&55.19 \\   
% Latent DDPM kps=0.9999 &62.5&70.79&57.96&64.97&58.21&63.28&55.83&53.62&58.23&56.93  \\ 
SLPD (Ours) &62.5&70.79&54.78&63.69&58.21&63.28&55.17&52.66&53.68&55.19  \\

 \hline
\end{tabular}
}
% \vspace{-1em}
\end{table*}
% \begin{table*}[tbh!]
% \vspace{-0.5em}
\centering
\caption{Point Cloud: MMD (CD multi 1000, EMD multi 100)} 
\label{tbl:exp1}
\scalebox{0.7}
{
\begin{tabular}{c|cccccccccc}
\hline
 & \multicolumn{2}{c}{Airplane} & \multicolumn{2}{c}{Cabinet} & \multicolumn{2}{c}{Car} & \multicolumn{2}{c}{Chair} & \multicolumn{2}{c}{Lamp}\\ \cline{2-3} \cline{4-5} \cline{6-7} \cline{8-9} \cline{10-11}
  & CD & EMD & CD & EMD & CD & EMD & CD & EMD & CD & EMD \\ 
 \hline
TreeGan&4.24&9.54&9.74&23.1&4.86&4.74&15.28&12.83&21.06&19.75\\
ShapeGF&3.96&4.03&9.35&6.84&4.28&3.2&13.83&9.39&17.92&10.83\\PVD&4.9&3.93&10.7&7.56&4.69&3.2&15.82&9.01&24.77&12.68\\
DPM&3.85&3.84&9.23&7.22&4.67&3.22&12.44&8.8&17.13&12.07\\
SPGAN&4.94&4.15&10.32&8.49&4.82&3.64&18.6&10.03&21.97&13.24\\ \hline
% DDPM&4.5&3.48&9.45&6.39&4.42&2.97&14.21&8.39&20.88&11.07\\
% Latent DDPM kps=0.999&4.28&3.97&8.72&6.47&4.08&3.04&14.0&8.26&21.99&11.79\\
% Latent DDPM kps=0.9999&4.17&3.84&9.03&6.67&4.05&3.11&14.17&8.27&22.46&11.88\\
SLPD (Ours)&4.17&3.84&8.72&6.47&4.05&3.11&14.0&8.26&21.99&11.79\\

 \hline
\end{tabular}
}
% \vspace{-1em}
\end{table*}
% \begin{table*}[tbh!]
% \vspace{-0.5em}
\centering
\caption{Point Cloud: COV (multi 100)} 
\label{tbl:exp1}
\scalebox{0.7}
{
\begin{tabular}{c|cccccccccc}
\hline
 & \multicolumn{2}{c}{Airplane} & \multicolumn{2}{c}{Cabinet} & \multicolumn{2}{c}{Car} & \multicolumn{2}{c}{Chair} & \multicolumn{2}{c}{Lamp}\\ \cline{2-3} \cline{4-5} \cline{6-7} \cline{8-9} \cline{10-11}
  & CD & EMD & CD & EMD & CD & EMD & CD & EMD & CD & EMD \\ 
 \hline
TreeGan&47.28&15.84&43.95&12.74&37.52&23.23&48.01&22.9&48.05&25.97  \\  ShapeGF&52.23&40.35&49.04&52.87&45.39&43.93&51.85&41.36&52.38&55.84  \\  PVD&35.15&40.59&36.94&35.67&30.31&37.38&34.27&35.89&36.36&36.36  \\  DPM&41.83&49.01&52.23&48.41&28.57&42.06&46.82&48.89&49.78&51.52  \\  SPGAN&39.36&33.66&42.68&36.94&32.18&37.25&28.95&25.55&46.75&48.05  \\   \hline 
% DDPM&44.55&51.73&49.68&52.87&43.52&49.4&48.15&52.29&52.81&53.68  \\  
% Latent DDPM kps=0.999&45.3&41.58&52.87&47.13&41.79&41.39&49.93&49.93&53.68&52.38  \\   
% Latent DDPM kps=0.9999 &48.76&42.08&47.77&44.59&40.19&40.05&49.63&50.07&47.19&51.95  \\  
SLPD (Ours) &48.76&42.08&52.87&47.13&40.19&40.05&49.93&49.93&53.68&52.38 \\

\hline
\end{tabular}
}
% \vspace{-1em}
\end{table*}
% \begin{table*}[tbh!]
% \vspace{-0.5em}
\centering
\caption{Mesh: 1-NN(Percentage)} 
\label{tbl:exp1}
\scalebox{0.7}
{
\begin{tabular}{c|ccccccccccccccc}
\hline
 & \multicolumn{3}{c}{Airplane} & \multicolumn{3}{c}{Cabinet} & \multicolumn{3}{c}{Car} & \multicolumn{3}{c}{Chair} & \multicolumn{3}{c}{Lamp}\\ \cline{2-16}
  & CD & EMD & NORMAL & CD & EMD & NORMAL & CD & EMD & NORMAL & CD & EMD & NORMAL & CD & EMD & NORMAL \\ 
 \hline
TreeGan&81.31&71.78&84.65&69.75&74.2&55.73&91.32&75.23&87.18&70.24&62.04&56.94&74.68&59.74&50.43  \\  ShapeGF&73.39&71.29&76.36&61.46&53.82&56.37&67.29&59.01&54.47&56.57&54.14&54.65&57.79&51.3&54.33  \\  PVD&88.86&83.91&86.14&79.62&69.43&68.47&85.05&67.02&58.48&81.39&73.71&71.79&77.27&80.09&84.42  \\  DPM&75.5&68.81&50.37&62.42&63.69&49.68&86.38&79.44&50.00&66.17&69.5&49.93&66.23&61.04&52.6  \\  SPGAN&82.05&70.42&83.29&70.38&61.15&60.19&85.18&75.23&59.61&79.03&75.85&77.99&67.97&64.07&70.13  \\   \hline 
% DDPM&65.72&62.0&72.65&57.64&59.55&52.55&67.56&63.75&53.6&57.61&53.03&52.73&58.44&56.93&54.11  \\  
% DDPM no normal&65.84&58.04&71.29&56.69&52.23&52.55&69.09&61.48&53.94&58.64&53.03&52.66&57.79&55.41&55.41  \\  
% Latent DDPM&71.29&65.35&72.77&56.69&62.42&55.73&70.23&66.49&53.0&57.46&51.62&53.18&57.79&56.49&53.68  \\   
% Latent DDPM no normal &70.17&65.84&72.40&55.73&59.24&55.41&69.09&64.62&53.40&56.72&51.18&53.77&58.44&58.87&52.81  \\  
SLPD (Ours) &70.17&65.84&72.40&55.73&59.24&55.41&69.09&64.62&53.40&56.72&51.18&53.77&58.44&58.87&52.81  \\  



 \hline
\end{tabular}
}
% \vspace{-1em}
\end{table*}
% \begin{table*}[tbh!]
\vspace{-2em}
\centering
\caption{MMD comparison of meshes generated by our method and baselines. CD, EMD, and normal consistency (N.C.) losses are multiplied by 1000, 100, and 10, respectively. We find that the N.C. loss computed MMDs best match human evaluations based on examples in Figure~\ref{fig:mesh_comparison}.} 
\label{tbl:mesh_mmd}
\scalebox{0.8}
{
\begin{tabular}{c|ccc|ccc|ccc|ccc|ccc}
\hline
 & \multicolumn{3}{c|}{Airplane} & \multicolumn{3}{c|}{Cabinet} & \multicolumn{3}{c|}{Car} & \multicolumn{3}{c|}{Chair} & \multicolumn{3}{c}{Lamp}\\ \cline{2-16}
  & CD & EMD & N.C. & CD & EMD & N.C. & CD & EMD & N.C. & CD & EMD & N.C. & CD & EMD & N.C. \\ 
 \hline
TreeGan&5.01&4.04&3.79&10.51&7.42&4.04&5.17&3.59&3.94&16.1&9.02&5.06&26.68&12.28&6.31\\ShapeGF&4.33&4.03&3.26&9.75&6.49&2.98&4.46&3.10&3.4&14.43&8.53&4.53&19.99&11.04&5.43\\PVD&5.11&4.29&3.42&10.91&7.00&3.09&4.80&3.20&3.45&15.89&8.69&4.40&24.85&12.99&5.11\\DPM&4.46&4.00&4.80&9.67&7.39&4.69&4.99&3.57&4.95&14.09&9.51&6.20&19.63&12.02&6.84\\SPGAN&5.09&4.03&3.39&10.52&7.18&3.04&5.02&3.49&3.54&18.86&9.67&4.68&23.64&12.63&4.87\\
% DDPM&4.54&3.66&3.05&9.76&6.71&2.86&4.46&3.15&3.32&14.37&8.41&4.17&21.7&11.33&4.38\\
% DDPM no normal&4.55&3.64&3.07&9.75&6.43&2.86&4.46&3.12&3.34&14.39&8.37&4.19&21.74&11.27&4.43\\
% Latent DDPM&4.48&3.81&3.08&9.53&6.74&2.75&4.57&3.28&3.21&14.86&8.48&4.17&23.81&11.75&4.33\\
% Latent DDPM no normal&4.46&3.77&3.07&9.52&6.50&2.76&4.55&3.20&3.23&14.76&8.49&4.19&23.61&11.87&4.38\\
\hline
Ours&4.46&3.77&3.07&9.52&6.50&2.76&4.55&3.20&3.23&14.76&8.49&4.19&23.61&11.87&4.38\\

 \hline
\end{tabular}
}
% \vspace{-1em}
\end{table*}    
% \begin{table*}[tbh!]
% \vspace{-0.5em}
\centering
\caption{Mesh: COV (CD EMD multi 100, NORMAL multi 100)} 
\label{tbl:exp1}
\scalebox{0.7}
{
\begin{tabular}{c|ccccccccccccccc}
\hline
 & \multicolumn{3}{c}{Airplane} & \multicolumn{3}{c}{Cabinet} & \multicolumn{3}{c}{Car} & \multicolumn{3}{c}{Chair} & \multicolumn{3}{c}{Lamp}\\ \cline{2-16}
  & CD & EMD & NORMAL & CD & EMD & NORMAL & CD & EMD & NORMAL & CD & EMD & NORMAL & CD & EMD & NORMAL \\  
 \hline
TreeGan&46.29&43.81&31.68&43.31&50.96&29.94&33.24&34.18&8.28&49.63&48.89&26.29&47.19&46.75&29.87  \\  ShapeGF&49.5&41.58&41.34&45.86&49.04&38.85&44.73&47.8&14.82&52.88&50.96&34.12&50.65&55.84&36.80  \\  PVD&34.41&34.9&32.67&36.94&45.22&29.94&30.31&40.59&10.55&35.75&44.02&27.47&38.53&43.72&28.14  \\  DPM&43.32&48.76&32.18&49.68&50.32&33.76&33.24&35.11&7.21&44.02&47.27&26.44&48.48&52.38&29.44  \\  SPGAN&41.09&46.29&34.65&39.49&42.68&32.48&32.58&36.72&12.68&30.13&31.91&21.86&46.75&49.78&34.63  \\   \hline 
% DDPM&44.31&48.02&42.82&47.13&49.68&40.76&39.52&41.26&14.29&48.15&50.22&37.08&49.78&50.65&37.23  \\  
% DDPM no normal&42.57&51.24&45.05&45.22&51.59&39.49&39.52&41.79&13.22&47.12&52.14&35.45&50.22&51.95&38.53  \\  
% Latent DDPM&50.0&44.55&39.60&50.96&45.86&36.94&38.72&39.39&12.95&48.45&52.58&34.86&52.81&50.22&36.80  \\   
% Latent DDPM no normal &48.51&46.29&39.85&52.87&47.77&37.58&38.99&39.92&12.55&49.19&49.63&34.71&52.81&52.38&36.36  \\  
SLPD (Ours) &48.51&46.29&39.85&52.87&47.77&37.58&38.99&39.92&12.55&49.19&49.63&34.71&52.81&52.38&36.36  \\  

 \hline
\end{tabular}
}
% \vspace{-1em}
\end{table*}

\subsection{Point Cloud and Mesh Generation}
We train our sparse latent point diffusion model on $5$ categories of the ShapeNet dataset: Airplane, cabinet, car, chair, and lamp.
And compare with baselines TreeGan~\cite{DBLP:conf/iccv/TreeGan}, 
SPGAN~\cite{li2021sp},
ShapeGF~\cite{cai2020learning}, PVD~\cite{zhou20213d}, 
DPM~\cite{luo2021diffusion}.
All the baselines are trained by ourselves using their public codebase.
We compare both the point clouds that they generate and meshes reconstructed from the point clouds using SAP.
Meshes generated by our method and baselines are shown in Figure~\ref{fig:mesh_comparison}.
More examples and generated point clouds are shown in Appendix B.6, B.7, and B.8.
We can see that our method generates meshes of the highest visual quality, with smooth surfaces and sharp details.
Since all the meshes are reconstructed from the generated point clouds using the same method, SAP.
This means the quality of the generated point clouds greatly affects the quality of the reconstructed meshes.
We provide an example of generated point clouds in Figure~\ref{fig:car_pc}. More point cloud examples are provided in Appendix B.7. 
Indeed, we can see that point clouds generated by our method spread more uniformly on the surface of the objects, and bear less noise compared with other methods. 
We attribute this to the design of our novel point cloud autoencoder.
% Compared with PVD~\cite{zhou20213d}, which directly trains DDPMs on dense point clouds, our sparse latent point diffusion model demonstrates great advantages in visual quality.
% In addition, the average generation time for a single point cloud of PVD~\cite{zhou20213d} is 13.3s tested a NVIDIA GeForce RTX 2080Ti GPU, while our sparse latent point diffusion model only need 0.18 s to generate a point cloud tested on a NVIDIA A100 GPU. Even if we count the 
% \footnote{We have not been able to run PVD on an A100 GPU. We replace the Point-Voxel CNN in it with an improved PointNet++~\cite{lyu2021conditional} and test it on an A100 GPU. The original PVD need 13.3s to generate a point cloud tested on an RTX 2080Ti GPU.}.
Quantitatively, we compute 1-NN, MMD, and COV on both generated point clouds and reconstructed meshes. Results are shown in Appendix B.4 and B.6.
% Point clouds evaluation results are in Appendix B.5.
% We find that MMDs computed by normal consistency loss best match human evaluations based on samples shown in Figure~\ref{fig:mesh_comparison}, and is shown in Table~\ref{tbl:mesh_mmd}. 1-NN and COV metrics of generated meshes are shown in Appendix B.4.
In terms of efficiency, the average generation time for a single point cloud of our method is about 0.20s (See Appendix B.10 for more details of the generation time.) tested on a single NVIDIA A100 GPU, while the DDPM-based method that directly trains generative models on dense point clouds, PVD~\cite{zhou20213d}, need 2.93s to generate a point cloud.
% \vspace{-1.5em}
% \paragraph{Ablation Study.}
We also conduct an ablation study on the number of sparse latent points and the method to sample them. Results are shown in Appendix B.10.



\begin{figure}[t]
% \vspace{-1em}
    \centering
    \includegraphics[width=0.4\textwidth]{Figures/method_figures/shape_generation_diversity.png}
    \vspace{-1em}
    \caption{Our method is able to generate diverse meshes for the same set of sparse latent points due to the stochasticity in the feature generation process. Here are two pairs of generated lamps for the same set of latent points.}
    \label{fig:generation_diversity}
\vspace{-1.5em}
\end{figure}

\begin{figure}[t]
% \vspace{-2em}
    \centering
    \includegraphics[width=0.95\linewidth]{Figures/method_figures/local_controllable_generation.png}
    \vspace{-1em}
    \caption{
    % Left side are FPS sampled sparse latent points (top) and human placed sparse latent points (bottom). We can 
    Use manually placed sparse latent points to control the rear legs of the generated chairs. 
    The blue points are moving and the red points are fixed.
    The top row generates new features for all sparse latent points, and the bottom row generates new features only for moved points and fixes the features of the rest points.}
    \label{fig:local_controllable_generation}
% \vspace{-1em}
\end{figure}

\begin{figure}[t]
\vspace{-1em}
    \centering
    \includegraphics[width=0.7\linewidth]{Figures/method_figures/shape_combination.png}
    \vspace{-0.5em}
    \caption{Perform shape combination. 
    The first row are the sparse latent points of the original two lamps and the combined sparse latent points.
    % The first row are the sparse latent points (with features) of the original two lamps (left side and right side) and the two combined sparse latent points (the middle two).
    The second row are the original two lamps (two sides) and the two lamps (middle two) obtained by combining the top part and bottom part of the original lamps.}
    \label{fig:shape_combination}
\vspace{-1.5em}
\end{figure}

% \begin{figure}[t]
% \vspace{-1em}
%     \centering
%     \includegraphics[width=0.9\linewidth]{Figures/method_figures/decompose_feature_and_position.png}
%     \vspace{-0.5em}
%     \caption{Decompose positions and features of the sparse latent points during interpolation. See Appendix B.8 for the complete interpolation process bewteen the two shapes.}
%     \label{fig:decompose_feature_and_position}
% \vspace{-1.5em}
% \end{figure}


% \subsection{Controllable Generation.}
\vspace{-2em}
\paragraph{Controllable Generation.}
As mentioned in Section~\ref{sec:train_latent_ddpm}, we can use the sparse latent points to control the generated mesh.
Specifically, we can change the positions of the sparse latent points, then use the second DDPM to generate features at the latent points, and finally decode them to a point cloud and reconstruct the mesh.
Several examples are shown in Figure~\ref{fig:controllable_generation}. 
It shows that we can use the sparse latent points to control the overall scale of the generated mesh as well as change the position, scale, or shape of a part of the mesh.
It is worth noting that we achieve this without any part annotations of the dataset.
Our method is also able to generate diverse meshes even for the same set of sparse latent points due to the stochasticity in the feature generation process.
Figure~\ref{fig:generation_diversity} gives two pairs of examples.

The sparse latent points in Figure~\ref{fig:controllable_generation} are obtained by FPS. 
At inference, we can also manually place the sparse latent points at regions of interest other than FPS sampled points and control the corresponding part.
This is because we augment the FPS sampled sparse latent points with Gaussian noises during training and it makes our model robust to the positions of the sparse latent points.
Figure~\ref{fig:local_controllable_generation} gives an example where we manually select the sparse latent points and control the rear legs of a chair.
In addition, if we want to keep the rest part of a shape fixed while changing the part we want to edit, we can use the second DDPM to sample features only for moved sparse latent points and fix the features of rest points.
See Figure~\ref{fig:local_controllable_generation} for an example.
This is achieved by an algorithm similar to DDPM-based image inpainting and is described in Appendix A.5.

\begin{figure}[t]
\vspace{-2em}
    \centering
    \includegraphics[width=0.43\textwidth]{Figures/method_figures/interpolation_used.png}
    \vspace{-0.5em}
    \caption{Our method is able to perform both global and local interpolations. The first row is an example of global interpolation. The second row interpolates between the bottom of the two lamps.}
    \label{fig:interpolation}
\vspace{-1em}
\end{figure}
% \subsection{Shape Interpolation.}
\vspace{-1em}
\paragraph{Shape Interpolation.}
% It is straightforward to perform interpolation using our sparse latent point representation of 3D shapes.
To interpolate two shapes,
we can interpolate both the positions and features between the corresponding latent points of two shapes.
See Appendix B.8 for how to establish correspondence between two sets of sparse latent points of two shapes.
The top row of Figure~\ref{fig:interpolation} is an example of global interpolation.
Our method is also able to perform local interpolation.
We can interpolate only a part of the latent points, and keep the positions and features of the rest part of the latent points fixed.
The bottom row of Figure~\ref{fig:interpolation} is an example of local interpolation.
% We can also decompose positions and features of the sparse latent points during interpolation.
% Figure~\ref{fig:decompose_feature_and_position} gives an example where we only interpolate positions or features of the sparse latent points between two shapes.
% We can see that positions of the sparse latent points mainly control the overall size and structure of the shape, while features mainly control the local geometry of the shape.
\vspace{-1.8em}
\paragraph{Shape combination.}
We can also perform shape combinations using our sparse latent point-based representation of 3D shapes.
We can simply combine the sparse latent points and their features from two or more source shapes to form new shapes.
See Figure~\ref{fig:shape_combination} for an example.
% \subsection{Single-Class 3D Shape Generation}

% \subsection{Many-class Unconditional 3D Shape Generation}

% \subsection{Sampling Time}

\section{Conclusion}
\label{sec:conclusion}
\section{Conclusion}

% In this work, we propose PGKD to distill the knowledge from high-accuracy GNNs to low-latency MLPs.
% The distillation process is edge-free and the learned MLP students are structure-aware.
% Firstly, we analyze the impact of graph structure~(graph edges) on GNNs.
% Specifically, we categorize the graph edges into Intra-class edges and Inter-class edges and study their impact, respectively.
% Based on the analysis, we design two corresponding losses via class prototypes to transfer the graph structural knowledge from GNNs to MLPs.
% Experiments on popular benchmarks demonstrate the effectiveness of our proposed PGKD.
% Further analysis indicate that PGKD is robust to noisy node features and performs well in different training settings.

% For future work, we would consider to apply PGKD to other graph tasks other than node classification.
% Moreover, generating the prototypes basing on the node representations rather than the class labels would be another interesting topic.

A novel PGKD scheme has been proposed to distill the knowledge from high-accuracy GNNs to low-latency MLPs, wherein the distillation process is edge-free and the learned MLP students are structure-aware. 
Specifically, we analyze the impact of graph structure~(graph edges) on GNNs and categorize them into intra-class and inter-class edges. 
Two corresponding losses via class prototypes are designed to transfer the graph structural knowledge from GNNs to MLPs.
Experiments on popular benchmarks demonstrate the effectiveness of PGKD.
Additionally, we show PGKD is robust to noisy node features, and performs well under different training settings.

For our future work, PGKD will be generalized to other graph tasks beyond node classification. 
Another interesting direction will be to generate prototypes utilizing node representations rather than class labels.


\clearpage
\section*{Limitations}
In PGKD, we adopt the class prototypes to capture graph structural information for MLPs in an edge-free setting.
Subsequently, PGKD requires slightly more computing cost compared to the baseline GLNN.
Meanwhile, the gap between the MLP learned by PGKD and its teacher GNN under the inductive setting is larger than that under the transductive setting, especially on Cora and Penn94 datasets.
More effort to improve the performance under the inductive setting is required underway.


% \section{Table}
% \label{sec:table}
% \begin{table*}[tbh!]
% \vspace{-0.5em}
\centering
\caption{Point Cloud: MMD (CD multi 1000, EMD multi 100)} 
\label{tbl:exp1}
\scalebox{0.7}
{
\begin{tabular}{c|cccccccccc}
\hline
 & \multicolumn{2}{c}{Airplane} & \multicolumn{2}{c}{Cabinet} & \multicolumn{2}{c}{Car} & \multicolumn{2}{c}{Chair} & \multicolumn{2}{c}{Lamp}\\ \cline{2-3} \cline{4-5} \cline{6-7} \cline{8-9} \cline{10-11}
  & CD & EMD & CD & EMD & CD & EMD & CD & EMD & CD & EMD \\ 
 \hline
TreeGan&4.24&9.54&9.74&23.1&4.86&4.74&15.28&12.83&21.06&19.75\\ShapeGF&3.96&4.03&9.35&6.84&4.28&3.2&13.83&9.39&17.92&10.83\\PVD&4.9&3.93&10.7&7.56&4.69&3.2&15.82&9.01&24.77&12.68\\DPM&3.85&3.84&9.23&7.22&4.67&3.22&12.44&8.8&17.13&12.07\\SPGAN&4.94&4.15&10.32&8.49&4.82&3.64&18.6&10.03&21.97&13.24\\DDPM&4.5&3.48&9.45&6.39&4.42&2.97&14.21&8.39&20.88&11.07\\Latent DDPM kps=0.999&4.28&3.97&8.72&6.47&4.08&3.04&14.0&8.26&21.99&11.79\\Latent DDPM kps=0.9999&4.17&3.84&9.03&6.67&4.05&3.11&14.17&8.27&22.46&11.88\\

 \hline
\end{tabular}
}
% \vspace{-1em}
\end{table*}

\begin{table*}[tbh!]
% \vspace{-0.5em}
\centering
\caption{Point Cloud: COV (multi 100)} 
\label{tbl:exp1}
\scalebox{0.7}
{
\begin{tabular}{c|cccccccccc}
\hline
 & \multicolumn{2}{c}{Airplane} & \multicolumn{2}{c}{Cabinet} & \multicolumn{2}{c}{Car} & \multicolumn{2}{c}{Chair} & \multicolumn{2}{c}{Lamp}\\ \cline{2-3} \cline{4-5} \cline{6-7} \cline{8-9} \cline{10-11}
  & CD & EMD & CD & EMD & CD & EMD & CD & EMD & CD & EMD \\ 
 \hline
TreeGan&47.28&15.84&43.95&12.74&37.52&23.23&48.01&22.9&48.05&25.97  \\  ShapeGF&52.23&40.35&49.04&52.87&45.39&43.93&51.85&41.36&52.38&55.84  \\  PVD&35.15&40.59&36.94&35.67&30.31&37.38&34.27&35.89&36.36&36.36  \\  DPM&41.83&49.01&52.23&48.41&28.57&42.06&46.82&48.89&49.78&51.52  \\  SPGAN&39.36&33.66&42.68&36.94&32.18&37.25&28.95&25.55&46.75&48.05  \\   \hline DDPM&44.55&51.73&49.68&52.87&43.52&49.4&48.15&52.29&52.81&53.68  \\  Latent DDPM kps=0.999&45.3&41.58&52.87&47.13&41.79&41.39&49.93&49.93&53.68&52.38  \\   Latent DDPM kps=0.9999 &48.76&42.08&47.77&44.59&40.19&40.05&49.63&50.07&47.19&51.95  \\  


 \hline
\end{tabular}
}
% \vspace{-1em}
\end{table*}

\begin{table*}[tbh!]
% \vspace{-0.5em}
\centering
\caption{Point Cloud: 1-NN(Percentage)} 
\label{tbl:exp1}
\scalebox{0.7}
{
\begin{tabular}{c|cccccccccc}
\hline
 & \multicolumn{2}{c}{Airplane} & \multicolumn{2}{c}{Cabinet} & \multicolumn{2}{c}{Car} & \multicolumn{2}{c}{Chair} & \multicolumn{2}{c}{Lamp}\\ \cline{2-3} \cline{4-5} \cline{6-7} \cline{8-9} \cline{10-11}
  & CD & EMD & CD & EMD & CD & EMD & CD & EMD & CD & EMD \\ 
 \hline
TreeGan&79.95&99.26&71.66&99.68&91.12&97.4&73.86&94.83&69.7&93.72  \\  ShapeGF&64.23&74.01&59.87&62.74&63.08&64.49&54.65&69.35&56.06&56.49  \\  PVD&85.52&81.06&78.66&86.62&82.31&77.17&80.5&78.36&78.35&81.82  \\  DPM&77.48&65.47&68.47&65.29&82.18&69.43&63.15&61.3&66.67&64.29  \\  SPGAN&79.7&80.69&70.38&86.31&81.51&83.58&79.69&83.01&69.26&69.48  \\   \hline DDPM&60.64&51.36&53.5&53.82&63.75&51.87&57.46&54.43&55.41&54.33  \\  Latent DDPM kps=0.999&64.85&71.91&54.78&63.69&59.75&65.42&55.17&52.66&53.68&55.19  \\   Latent DDPM kps=0.9999 &62.5&70.79&57.96&64.97&58.21&63.28&55.83&53.62&58.23&56.93  \\  

 \hline
\end{tabular}
}
% \vspace{-1em}
\end{table*}
% \begin{table*}[tbh!]
% \vspace{-0.5em}
\centering
\caption{Mesh: MMD (CD multi 1000,EMD mutil 100, NORMAL multi 10)} 
\label{tbl:exp1}
\scalebox{0.7}
{
\begin{tabular}{c|ccccccccccccccc}
\hline
 & \multicolumn{3}{c}{Airplane} & \multicolumn{3}{c}{Cabinet} & \multicolumn{3}{c}{Car} & \multicolumn{3}{c}{Chair} & \multicolumn{3}{c}{Lamp}\\ \cline{2-16}
  & CD & EMD & NORMAL & CD & EMD & NORMAL & CD & EMD & NORMAL & CD & EMD & NORMAL & CD & EMD & NORMAL \\ 
 \hline
TreeGan&5.01&4.04&3.79&10.51&7.42&4.04&5.17&3.59&3.94&16.1&9.02&5.06&26.68&12.28&6.31\\ShapeGF&4.33&4.03&3.26&9.75&6.49&2.98&4.46&3.10&3.4&14.43&8.53&4.53&19.99&11.04&5.43\\PVD&5.11&4.29&3.42&10.91&7.00&3.09&4.80&3.20&3.45&15.89&8.69&4.40&24.85&12.99&5.11\\DPM&4.46&4.00&4.80&9.67&7.39&4.69&4.99&3.57&4.95&14.09&9.51&6.20&19.63&12.02&6.84\\SPGAN&5.09&4.03&3.39&10.52&7.18&3.04&5.02&3.49&3.54&18.86&9.67&4.68&23.64&12.63&4.87\\DDPM&4.54&3.66&3.05&9.76&6.71&2.86&4.46&3.15&3.32&14.37&8.41&4.17&21.7&11.33&4.38\\DDPM no normal&4.55&3.64&3.07&9.75&6.43&2.86&4.46&3.12&3.34&14.39&8.37&4.19&21.74&11.27&4.43\\Latent DDPM&4.48&3.81&3.08&9.53&6.74&2.75&4.57&3.28&3.21&14.86&8.48&4.17&23.81&11.75&4.33\\Latent DDPM no normal&4.46&3.77&3.07&9.52&6.50&2.76&4.55&3.20&3.23&14.76&8.49&4.19&23.61&11.87&4.38\\
 \hline
\end{tabular}
}
% \vspace{-1em}
\end{table*}



\begin{table*}[tbh!]
% \vspace{-0.5em}
\centering
\caption{Mesh: COV (CD EMD multi 100, NORMAL multi 100)} 
\label{tbl:exp1}
\scalebox{0.7}
{
\begin{tabular}{c|ccccccccccccccc}
\hline
 & \multicolumn{3}{c}{Airplane} & \multicolumn{3}{c}{Cabinet} & \multicolumn{3}{c}{Car} & \multicolumn{3}{c}{Chair} & \multicolumn{3}{c}{Lamp}\\ \cline{2-16}
  & CD & EMD & NORMAL & CD & EMD & NORMAL & CD & EMD & NORMAL & CD & EMD & NORMAL & CD & EMD & NORMAL \\  
 \hline
TreeGan&46.29&43.81&31.68&43.31&50.96&29.94&33.24&34.18&8.28&49.63&48.89&26.29&47.19&46.75&29.87  \\  ShapeGF&49.5&41.58&41.34&45.86&49.04&38.85&44.73&47.8&14.82&52.88&50.96&34.12&50.65&55.84&36.80  \\  PVD&34.41&34.9&32.67&36.94&45.22&29.94&30.31&40.59&10.55&35.75&44.02&27.47&38.53&43.72&28.14  \\  DPM&43.32&48.76&32.18&49.68&50.32&33.76&33.24&35.11&7.21&44.02&47.27&26.44&48.48&52.38&29.44  \\  SPGAN&41.09&46.29&34.65&39.49&42.68&32.48&32.58&36.72&12.68&30.13&31.91&21.86&46.75&49.78&34.63  \\   \hline DDPM&44.31&48.02&42.82&47.13&49.68&40.76&39.52&41.26&14.29&48.15&50.22&37.08&49.78&50.65&37.23  \\  DDPM no normal&42.57&51.24&45.05&45.22&51.59&39.49&39.52&41.79&13.22&47.12&52.14&35.45&50.22&51.95&38.53  \\  Latent DDPM&50.0&44.55&39.60&50.96&45.86&36.94&38.72&39.39&12.95&48.45&52.58&34.86&52.81&50.22&36.80  \\   Latent DDPM no normal &48.51&46.29&39.85&52.87&47.77&37.58&38.99&39.92&12.55&49.19&49.63&34.71&52.81&52.38&36.36  \\  

 \hline
\end{tabular}
}
% \vspace{-1em}
\end{table*}


\begin{table*}[tbh!]
% \vspace{-0.5em}
\centering
\caption{Mesh: 1-NN(Percentage)} 
\label{tbl:exp1}
\scalebox{0.7}
{
\begin{tabular}{c|ccccccccccccccc}
\hline
 & \multicolumn{3}{c}{Airplane} & \multicolumn{3}{c}{Cabinet} & \multicolumn{3}{c}{Car} & \multicolumn{3}{c}{Chair} & \multicolumn{3}{c}{Lamp}\\ \cline{2-16}
  & CD & EMD & NORMAL & CD & EMD & NORMAL & CD & EMD & NORMAL & CD & EMD & NORMAL & CD & EMD & NORMAL \\ 
 \hline
TreeGan&81.31&71.78&84.65&69.75&74.2&55.73&91.32&75.23&87.18&70.24&62.04&56.94&74.68&59.74&50.43  \\  ShapeGF&73.39&71.29&76.36&61.46&53.82&56.37&67.29&59.01&54.47&56.57&54.14&54.65&57.79&51.3&54.33  \\  PVD&88.86&83.91&86.14&79.62&69.43&68.47&85.05&67.02&58.48&81.39&73.71&71.79&77.27&80.09&84.42  \\  DPM&75.5&68.81&50.37&62.42&63.69&49.68&86.38&79.44&50.00&66.17&69.5&49.93&66.23&61.04&52.6  \\  SPGAN&82.05&70.42&83.29&70.38&61.15&60.19&85.18&75.23&59.61&79.03&75.85&77.99&67.97&64.07&70.13  \\   \hline DDPM&65.72&62.0&72.65&57.64&59.55&52.55&67.56&63.75&53.6&57.61&53.03&52.73&58.44&56.93&54.11  \\  DDPM no normal&65.84&58.04&71.29&56.69&52.23&52.55&69.09&61.48&53.94&58.64&53.03&52.66&57.79&55.41&55.41  \\  Latent DDPM&71.29&65.35&72.77&56.69&62.42&55.73&70.23&66.49&53.0&57.46&51.62&53.18&57.79&56.49&53.68  \\   Latent DDPM no normal &70.17&65.84&72.40&55.73&59.24&55.41&69.09&64.62&53.40&56.72&51.18&53.77&58.44&58.87&52.81  \\  


 \hline
\end{tabular}
}
% \vspace{-1em}
\end{table*}


%%%%%%%%% REFERENCES
{\small
\bibliographystyle{ieee_fullname}
\bibliography{egbib}
}

\end{document}
