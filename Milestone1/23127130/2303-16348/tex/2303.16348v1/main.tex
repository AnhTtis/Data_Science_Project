
\documentclass[11pt,twoside]{article}

\setlength{\textwidth}{160mm} \setlength{\textheight}{210mm}
\setlength{\parindent}{8mm} \frenchspacing
\setlength{\oddsidemargin}{0pt} \setlength{\evensidemargin}{0pt}
\thispagestyle{empty}
\usepackage{mathrsfs,amsfonts,amsmath,amssymb}
%\usepackage[english,russian]{babel}
\usepackage{latexsym}
\usepackage{comment} 



%\usepackage[
%backend=biber,
%style=alphabetic,
%sorting=ynt
%]{biblatex}



\pagestyle{myheadings}
%\input{polchar}
%\markboth{\centerline{\sc{\small  }}}
\newtheorem{satz}{Theorem}
\newtheorem{proposition}[satz]{Proposition}
\newtheorem{theorem}[satz]{Theorem}
\newtheorem{lemma}[satz]{Lemma}
\newtheorem{definition}[satz]{Definition}
\newtheorem{corollary}[satz]{Corollary}
\newtheorem{remark}[satz]{Remark}
\newtheorem{example}[satz]{Example}
\newcommand{\lf}{\left\lfloor}
\newcommand{\rf}{\right\rfloor}
\newcommand{\lc}{\left\lceil}
\newcommand{\rc}{\right\rceil}

%\newcommand{\qed}{{} \hfill \mbox{$\Box$}}
%\renewcommand{\baselinestretch}{1.5}
\renewcommand{\thefootnote}{}
%\renewcommand{\labelenumi{\roman{enumi}}}
\def\proof{\par \noindent \hbox{\rm P\,r\,o\,o\,f.}~}
\def\no{\noindent}
\def\sbeq{\subseteq}
\def\N{\mathbb {N}}
\def\Z{\mathbb {Z}}
\def\F{\mathbb {F}}
\def\E{\mathsf{E}}
\def\e{\varepsilon}
\def\l{\lambda}
\def\s{\sigma}
\def\I{{\cal I}}
\def\a{\alpha}
\def\S{{ S_{\alpha}}}
\def\C{\mathbb{C}}
\def\P{{\cal P}}
\def\h{\widehat}
\def\p{\varphi}
\def\d{\delta}
\def\o{\omega}
\def\({\big (}
\def\){\big )}
\def\g{\gamma}
\def\G{\Gamma}
\def\s{\setminus}
\def\b{\beta}
\def\ls{\leqslant}
\def\gs{\geqslant}
\def\dim{{\rm dim}}
\def\codim{{\rm codim}}
\def\le{\leqslant}
\def\ge{\geqslant}
\def\_phi{\varphi}
\def\eps{\varepsilon}
\def\m{\times}
\def\Gr{{\mathbf G}}
\def\FF{\widehat}
\def\k{\kappa}
\def\ov{\overline}
\def\Spec{{\rm Spec\,}}
\renewcommand{\thetheorem}{\arabic{theorem}.}


\def\t{\tilde}
\def\Span{{\rm Span\,}}
\def\f{{\mathbb F}}
\def\Cf{{\mathcal C}}
\def\mP{{\mathcal P}}
\def\mD{{\mathcal D}}
\def\la{\lambda}
\def\D{\Delta}
\def\Var{\mathsf{Var}}
\def\supp{\mathsf{supp}}


\def\T{\mathsf{T}}
\def\Q{\mathsf{Q}}
\def\C{\mathbb{C}}
\def\R{\mathbb{R}}
\def\M{\mathsf{M}}
\def\cI{\mathsf{I}}
\def\tr{\mathrm{tr}}


\def\SL{{\rm SL}}
\def\Conj{{\rm Conj}}
\def\Stab{{\rm Stab}}
%\def\tr{{\rm tr}}
\def\Aff{{\rm Aff}}
\def\cov{{\rm cov}}

\def\L{\mathcal{L}}




\newcommand{\commIl}[2][]{\todo[#1,color=yellow]{Ilya: #2}}


\newcommand{\bp}{\bigskip}






\author{I.D. Shkredov}
%*%\title{Two--dimensional corners and the higher energies}
\title{Some new results on  the higher energies I}
\date{}
\begin{document}
	\maketitle  





%\begin{document}


\begin{center}
	Annotation.
\end{center}



{\it \small
    We obtain a generalization of the recent Kelley--Meka result on sets avoiding arithmetic progressions of length three. 
%*%    We generalize the recent Kelley--Meka result on sets avoiding arithmetic progressions of length three for two--dimensional corners.  
    In our proof we develop the theory of the higher energies. 
    Also, we discuss the case of longer arithmetic progressions. 
}
\\


\section{Introduction}

The famous  Erd\H{o}s--Tur\'an conjecture \cite{ET} asks is it true that for any integer $k\ge 3$  any set of positive integers $A=\{n_1 < n_2 <\dots < n_m < \dots \}$
satisfying 
\begin{equation}\label{conj:ET}
    \sum_{j=1}^\infty \frac{1}{n_j} = \infty 
\end{equation} 
contains an arithmetic progression of length $k$ (we say $A$ has APk), that is the sequence of the form $x,x+y,\dots,x+(k-1)y \in A$? 
This question has a rich history  see, e.g., \cite{Gowers_4}, \cite{Gowers_m} or \cite{sh_Sz_survey}  and is considered a central one in the area of classical additive combinatorics due to its connection with many adjecent fields as combinatorical ergodic theory and graphs/hypergraphs theory, we just mention some papers 
\cite{Szemeredi_4}, \cite{Szemeredi_m}, \cite{furstenberg2014recurrence}, \cite{furstenberg1982ergodic}, \cite{shelah1988primitive}, \cite{gowers2007hypergraph}, \cite{van1927beweis}, \cite{Tao_removal}, \cite{RS_regularity}, \cite{NRS_counting} etc. 
If one defines 
\[
    r_k (N) = \frac{1}{N} \max\{ |A| ~:~ A\subseteq \{1,\dots, N\}\,, \quad A \mbox{ has no APk} \} \,,
\]
then the condition \eqref{conj:ET} means, roughly, that 
\begin{equation}\label{conj:ET_rk}
    r_k (N) \ll \frac{1}{\log N \cdot (\log \log N)^{1+\eps}} \,, \quad \quad N \to \infty 
\end{equation} 
for an arbitrary $\eps>0$. 


The case of arithmetic progressions of length   three was considered to be special thanks to the Fourier approach of Roth \cite{Roth1953}, the required information and references can be found in \cite{bloom2023kelley}, \cite{kelley2023strong}, as well as in \cite{sh_Sz_survey}. 
Bloom and Sisask in \cite{bloom2020breaking} proved that $r_3 (N) \ll (\log N)^{-1-c_1}$ for a certain $c_1>0$ and hence established  conjecture \eqref{conj:ET_rk} in the case of $k=3$. 
Recently, Kelley and Meka \cite{kelley2023strong} achieved a remarkable progress in this question and proved that 
\[
    r_k (N) \ll \exp (- O((\log N)^{c_2})) \,,
\]
where $c_1>0$ is an absolute constant. 
One of the ideas of paper  \cite{kelley2023strong} was to use the higher energy $\E^k_2$ and the uniformity relatively to $\E^k_2$ (all definitions can be found in Sections \ref{sec:def}, \ref{sec:E^k_l}) with a growing parameter $k$ to control  the number of arithmetic progressions in an arbitrary set. 
Namely, bound \eqref{conj:ET_rk} is an immediate consequence of   the following result (for simplicity we consider the group $\F_p^n$). 


\begin{theorem}
    Let $\Gr=\F_p^n$, $A\subseteq \Gr$ be a set, $|A|=\d N$, and $\eps>0$ be a parameter.
    Then there is a subspace $V\subseteq \Gr$ and $x\in \Gr$ such that $A\cap (V+x)$ is $\eps$--uniform relatively to $\E^k_2$, $\mu_{V+x} (A) \ge \d$, and 
\begin{equation}\label{f:uniform_KM_intr}
    \codim V \ll \eps^{-14} k^4 \L^3 (\d) \L^{2} (\eps \d) \,.
    %7->14
\end{equation}
\label{t:uniform_KM_intr}
\end{theorem}


%We generalize Theorem \ref{t:uniform_KM} for the higher energies $\E^k_l$.  


The aim of  this paper is to generalize Kelley--Meka  results to a more wide additive--combinatorial family of energies $\E^k_l$ see, e.g., \cite{sh_Ek}. In our regime the parameter $l$ below is $l=O(1)$. 



\begin{theorem}
    Let $\Gr=\F_p^n$, $A\subseteq \Gr$ be a set, $|A|=\d N$, and $\eps \in (0,1]$ be a parameter.
    Then there is a subspace $V\subseteq \Gr$ and $x\in \Gr$ such that $A\cap (V+x)$ is $\eps$--uniform relatively to $\E^k_l$, $\mu_{V+x}(A) \ge \d$ and 
\begin{equation}\label{f:uniform_KM_k_intr}
    \codim V \ll 
%    \eps^{-3l^2} k^4  l^{9l} \L^{4l} (\eps) \L^{5l} (\d) 
\eps^{-28l^l} (8l)^{28l^l} k^4 \L^{4l} (\eps) \L^{5l} (\d)
    \,.
\end{equation}
\label{t:uniform_KM_k_intr}
\end{theorem}


%*%
%The exact formulation of our results can be found in Section \ref{sec:E^k_l}, see Theorem \ref{t:uniform_KM_k} and Proposition \ref{p:L2_increment_Ekl}. 
Theorem \ref{t:uniform_KM_k_intr} is interesting in its own right and can be used to 
%finding 
solve more general equations and systems than $x+y=2z$, which corresponds to the case of AP3. 
Our forthcoming paper will be devoted to finding some applications of this result. 


%*%
\begin{comment}

As an application we consider a two--dimensional generalization of the problem about arithmetic progression of length three, namely, the {\it corners}, see  
%which is called 
\cite{furstenberg2014recurrence}, \cite{Gowers_m}, \cite{sh_Izv}. 
Recall, that a triple $\{ (x,y), (x+d,y), (x,y+d) \} \in \Z^2 \times \Z^2 \times \Z^2$ with  $d>0$ is called a {\it corner}. 
Let 
\[
L (N) = \frac{1}{N^2} \max\{ |A| ~:~ A\subseteq \{1,\dots, N\}^2 \,, \quad A \mbox{ has no corners} \} \,.
\]
The  problem of estimating the function $L(N)$ was considered in \cite{furstenberg2014recurrence}, 
\cite{solymosi_corners},
\cite{Green_models}, 
\cite{Lacey_corners}, \cite{sh_Izv}, \cite{sh_Sz_survey}, \cite{sh_Izv_ab} and in other papers. 
It is easy to see that
\[
    r_3 (N) \ll L(N)
\]
and that is why we say that the problem 
%about corners 
on obtaining upper bounds for $L(N)$
generalizes the question about AP3. In  \cite{sh_Sz_LMS} the author proved that
\begin{equation}\label{f:L(N)_old}
    L(N) \ll \frac{1}{(\log \log N)^C} \,,
\end{equation} 
where $C>0$ is an absolute constant. The case of general abelian group was considered in \cite{sh_Izv_ab}. In our new result we reduce one logarithm in bound \eqref{f:L(N)_old} in the case of the group $\F_2^n$. 







\begin{theorem}
    There is an absolute constant $c>0$ such that any set $A\subseteq \F^n_2 \times \F^n_2$  of size 
\[
    |A| \gg \frac{4^n}{n^c} 
\]
    %Then $A$ 
    contains a non--trivial corner, that is a configuration of the form $\{ (x,y), (x+d,y), (x,y+d) \}$ with non--zero $d$. 
\label{t:corners_new}
\end{theorem} 



Our Theorem \ref{t:corners_new} significantly improves previous results  \cite{Green_models}, 
\cite{Lacey_corners}, \cite{sh_Sz_survey} and \cite{sh_Izv_ab} (for the group $\Gr = \F_2^n$) and of course in our proof we use the arguments from  \cite{sh_Izv}, \cite{sh_Sz_LMS} and \cite{sh_Izv_ab}. 
Usually the case of the groups $\F_p^n$ for prime $p$ is considered to be a model one, see the excellent survey \cite{Green_models} about this theme and an appropriate generalization for all abelian groups $\Gr$ is known to be a technical task. We will prove the correspondent result in our forthcoming paper. 

\end{comment} 


Usually the case of the groups $\F_p^n$ for prime $p$ is considered to be a model one, see the excellent survey \cite{Green_models} about this theme and an appropriate generalization for all abelian groups $\Gr$ is known to be a technical task. We will prove the correspondent result in our forthcoming paper. 


The approach develops the strategy of \cite{Gowers_4}, \cite{Gowers_m}, the method of the higher energies (see, e.g., \cite{SS_higher}, \cite{sh_Ek}) and of course \cite{kelley2023strong}. 
Also, we extensively use a brilliant exposition \cite{bloom2023kelley}, where the Kelley--Meka results were discussed.
%*%
\begin{comment} 
The main message of our proof is the following: 
%that 
although the case  of arithmetic progressions of length three is a rather special one and corresponds to the energies $\E^k_2$  but nevertheless the argument  of \cite{kelley2023strong} can be adopted for higher energies $\E^k_l$ and in our particular  problem about corners it is required to have deal with the energy $\E^{k}_4$. 
%corners as well.  
%longer progressions.      
\end{comment} 


In the appendix we discuss the original Erd\H{o}s--Tur\'an conjecture, that is the case of longer arithmetic progressions and show that there is series of difficulties on the  conceptual and on the technical levels which make the question about generalizations of  the methods from \cite{kelley2023strong} rather hard. 
The author thinks this part is also interesting in its own right because it allows us to understand the limits of the  Kelley--Meka approach.  
 

\section{Definitions and preliminaries}  
\label{sec:def} 


Let  $\Gr$ be a finite abelian group 
and
denote by $N$ the cardinality of $\Gr$.
%$N:=|\Gr|$ and
We use the same capital letter to denote  set $A\subseteq \Gr$ and   its characteristic function $A: \Gr \to \{0,1 \}$. 
Let us define $\mu_A (x) = A(x)/|A|$, that is $\sum_{x\in \Gr} \mu_A (x) =1$. 
Given two sets $A,B\subset \Gr$, define  
%the \textit{product set} (
the {\it sumset} 
%in the abelian case) 
of $A$ and $B$ as 
$$A+B:=\{a+b ~:~ a\in{A},\,b\in{B}\}\,.$$
In a similar way we define the {\it difference sets} and {\it higher sumsets}, e.g., $2A-A$ is $A+A-A$.



Let $f$ be a function from $\Gr$ to $\mathbb{C}.$  We denote the Fourier transform of $f$ by~$\FF{f},$
\begin{equation}\label{F:Fourier}
  \FF{f}(\xi) =  \sum_{x \in \Gr} f(x) \overline{\chi (x)} \,,
\end{equation}
where $\chi \in \FF{\Gr}$ is a character of $\Gr$. 
We rely on the following basic identities
\begin{equation}\label{F_Par}
    \sum_{x\in \Gr} |f(x)|^2
        =
            \frac{1}{N} \sum_{\chi \in \FF{\Gr}} \big|\widehat{f} (\chi)\big|^2 \,,
\end{equation}
%\begin{equation}\label{svertka}
%    \sum_{y\in \Gr} \Big|\sum_{x\in \Gr} f(x) g(y-x) \Big|^2
%        = \frac{1}{N} \sum_{\xi \in \FF{\Gr}} \big|\widehat{f} (\xi)\big|^2 \big|\widehat{g} (\xi)\big|^2 \,.
%\end{equation}
and
\begin{equation}\label{f:inverse}
    f(x) = \frac{1}{N} \sum_{\chi \in \FF{\Gr}} \FF{f}(\chi)  \chi(x) \,.
\end{equation}
If
$$
    (f*g) (x) := \sum_{y\in \Gr} f(y) g(x-y) \quad \mbox{ and } \quad (f\circ g) (x) := \sum_{y\in \Gr} f(y) g(y+x) \,,
$$
 then
\begin{equation}\label{f:F_svertka}
    \FF{f*g} = \FF{f} \FF{g} 
%\quad \mbox{ and } \quad \FF{f \circ g} = \FF{f}^c \FF{g} = \ov{\FF{\ov{f}}} \FF{g} \,,
    %(\F{fg}) (x) = \frac{1}{N} (\F{f} * \F{g}) (x) \,.
\end{equation}
and similar for $f\circ g$. 
%where for a function $f:\Gr \to \mathbb{C}$ we put $f^c (x):= f(-x)$.
 Clearly,  $(f*g) (x) = (g*f) (x)$ and $(f\circ g)(x) = (g \circ f) (-x)$, $x\in \Gr$.
 The $k$--fold convolution, $k\in \N$  we denote by 
 $f^{(k)}$,
 so $f^{(2)} = f*f$ and $f^{(3)} = f*f*f$ for example. 




We need some formalism concerning higher convolutions see, e.g.,  
\cite{sh_Ek}. 
Let  $l$ be a positive integer. 
%Let us  define 
Consider 
two operators $\mD_l$, $\mP_l : \Gr \to \Gr^l$ such that for a variable $x$ one has $\mD_l (x) = (x,\dots,x) \in \Gr^l$ and $\mP_l (x) = (x_1,\dots,x_l) \in \Gr^l$. 
Notice that $\mP_1 (x) = \mD_1 (x) = x$. 
In the same way these operators act on functions, e.g., 
%$f:\Gr \to \C$.
%For example, 
$\mP_l (f) (x_1,\dots,x_l) = f(x_1) \dots f(x_l)$ for $f:\Gr \to \C$ and for $F:\Gr^l \to \C$ one has $\mD_l (F)(x_1,\dots,x_l) =F(x_1,\dots,x_1)$ if $x_1=\dots=x_l$ and zero otherwise. 
%namely, $\mP_l (f) (x_1,\dots,x_l) = f(x_1) \dots f(x_l)$ and $\mD_l (f) (x_1,\dots,x_l) = f^l(x_1)$ if $x_1=\dots=x_l$ and zero otherwise. 
Now given  a function $f:\Gr \to \C$ and a positive integer $l$ define the generalized convolution 
\begin{equation}\label{def:Cf}
    \Cf_l (f) (x_1,\dots,x_l) = \sum_{z\in \Gr} f(z+x_1) \dots f(z+x_l) = (\mD_l (\Gr) \circ \mP_l (f)) (x_1,\dots,x_l) 
\end{equation} 
\begin{equation}\label{def:Cf+}
    := \sum_{z\in \Gr} f_{x_1,\dots,x_l} (z) \,.
\end{equation} 
In a similar way we can consider $\Cf_l (f_1,\dots,f_l) (x_1,\dots,x_l)$ for any functions $f_1,\dots,f_l : \Gr \to \C$. 
One has 
\begin{equation}\label{f:E^k_l_symmetries} 
    \Cf_l (f) (x_1,\dots,x_l) = \Cf_l (f) (x_1+w,\dots,x_l+w) = \Cf_l (f) ((x_1,\dots,x_l) + \mD_l (w))
\end{equation} 
for any $w\in \Gr$. 
Let us 
%underline 
emphasise 
that definitions \eqref{def:Cf}, \eqref{def:Cf+} differ slightly from the usual one, see, e.g., \cite{sh_Ek} by a linear change of the variables. Namely, it is a little bit more traditional to put 
\begin{equation}\label{f:E^k_l:C,C'-}
    f'_{x_1,\dots,x_{l}} (z) = f_{0,x_1,\dots,x_l} (z)= f(z) f(z+x_1) \dots f(z+x_{l}) \,,
\end{equation} 
and
\begin{equation}\label{f:E^k_l:C,C'}
    \Cf'_{l+1} (f) (x_1,\dots,x_l) = \sum_{z\in \Gr} f(z) f(z+x_1) \dots f(z+x_l) =  \Cf_l (f) (0, x_1,\dots,x_l)\,.
\end{equation} 
Definitions \eqref{f:E^k_l:C,C'-}, \eqref{f:E^k_l:C,C'} have  an advantage that they allow to consider infinite groups $\Gr$ as well. 
To this end we use the dual notation  $\| f\|^{kl}_{\E^k_l} = \bar{\E}^{k}_l (f) = N^{-1} \E^k_l (f)$. 
%Further
Now having $k,l\ge 2$ and a function $f:\Gr \to \C$ one can consider 
\begin{equation}\label{def:E_kl}
    \E^k_{l} (f) = \sum_{x_1,\dots,x_l} \Cf^k_l (f) (x_1,\dots,x_l) = \E^l_{k} (f) 
\end{equation}
and it was showed in 
%Appendix 
\cite[Proposition 30]{sh_Ek} that for a real function $f$ and even $k,l$ the formula $(\E^k_{l} (f))^{1/kl}$ defines a norm of our function $f$. 
If one put $l=1$ in \eqref{def:E_kl}, then we formally obtain $\E_1^k (f) =N(\sum_z f(x) )^k$ and this is not a norm for any $k$. Nevertheless, it is convenient to consider the quantities $\E_1^k (f)$ sometimes. 
 Notice that $\E^k_{l} (f) \ge 0$, as well as the triangle inequality for  $\E^k_{l} (f)$ takes place, provided  at least one of $k,l$ is even but, nevertheless,  it cannot be a norm in this case, see \cite[Sections 4,7]{sh_Ek}.  

%\bp 

Let $\eps \in (0,1]$ be a real number. We write $\mathcal{L} (\eps)$ for $\log(2/\eps)$.
%, if $\eps\le 1$ and $\mathcal{L} (\eps) = 1$ otherwise. 
Let us 
%assume 
make a convention 
that if a product is taken over an empty set, then it equals one. 
The signs $\ll$ and $\gg$ are the usual Vinogradov symbols.
When the constants in the signs  depend on a parameter $M$, we write $\ll_M$ and $\gg_M$. 
%If $a\ll_M b$ and $b\ll_M a$, then we write $a\sim_M b$. 
All logarithms are to base $2$.
By $\F_p$ denote $\F_p = \Z/p\Z$ for a prime $p$. %Let $\F^*_p = \F_p \setminus \{0\}$. 
%If we have a set $A$, then we will write $a \lesssim b$ or $b \gtrsim a$ if $a = O(b \cdot \log^c |A|)$, $c>0$.
Let us denote by $[n]$ the set $\{1,2,\dots, n\}$.


%\section{}






\section{Some results on $\E^k_l$--norms}
\label{sec:E^k_l}


In this section we obtain some generalization of Kelley--Meka results which were obtained for $\E_2^k$--norm to  $\E_l^k$--norm. Also, we  discuss some special properties of such norms. Our results naturally break down into two cases: uniform and non--uniform. 


\subsection{Uniform sets in the sense of $\E^k_l$--norm}
\label{subsec:uniformity}


Let us give the  main definition of this subsection. 

\begin{definition}
    Let $\Gr$ be an abelian group, $A\subseteq \Gr$ be a set,  $|A|=\d N$, and $\eps>0$ be a parameter.
    %and $\| \cdot \|$ be a norm. 
    Then we say that $A$ is $\eps$--uniform relatively to (the energy) $\E^k_l$ if 
\begin{equation}\label{def:uniformity_kl} 
    \| f_A\|^{kl}_{\E^k_l} \le \eps^{kl} \d^{kl} N^{k+l} \,.
\end{equation}
%    Similarly, we that $A$ is $\eps$--uniform relatively to (the energy) $\mathcal{E}^k_{s,t}$ if  
%\begin{equation}\label{def:uniformity_kst} 
 %   \| f_A\|^{kst}_{\mathcal{E}^k_{s,t}} \le \eps^{kst} \d^{kst} N^{k+s+t} \,.
%\end{equation}
\end{definition} 


Usually the number $\eps$ belongs to $(0,1]$ but sometimes $\eps>1$ and hence one can consider the quantity $\eps$ as the  definition of the energy $ \| f_A\|^{kl}_{\E^k_l}$, that is 
$\| f_A\|^{kl}_{\E^k_l} := \eps^{kl} \d^{kl} N^{k+l}$. 
%\eqref{def:uniformity_kl}
Further by the H\"older inequality, we have 
\begin{equation}\label{f:Holder_Ekl}
    (\E^{k-1}_{l} (f))^{k} \le (\E^{k}_{l} (f))^{k-1} N^l 
\end{equation} 
and hence if $A$ is $\eps$--uniform relatively to $\E^k_l$, then $A$ is $\eps$--uniform relatively to $\E^{k'}_{l'}$ for $k'\le k$, $l'\le l$. 
%In contrary, 
On the other hand, 
it is easy to see that a smaller norm does not control the higher one. 

%\bp 

\begin{example}
%{\bf Example.} 
Let $\Gr = \F_2^n$, $H< \F_2^n$, $\Lambda \subseteq \F_2^n/H$ be a random set such that $|\Lambda| = \d N/|H|$. 
Also, suppose that $\d^2 \gg |H|/N$ and thus with high probability $\Lambda - \Lambda \approx \F_2^n/H$. 
Let $A$ be the direct sum of $H$ and $\Lambda$, then $|A|=\d N$. 
It is easy to see that for a random $x\in A-A \approx \Gr$ one has $|A_x| \sim \d^2 N$ but for $x,y\in H$ one has $A_x= A$ and $A_{x,y} = A$.
Thus for any $k\ge 2$ the following holds 
\[\E_2^k (A) \sim  (\d^2 N)^k N + (\d N)^k |H| \sim (\d^2 N)^k N \,, \]
provided $|H| \ll \d^k N$ but taking an arbitrary $l$,  we see that
\[\E^l_3 (A) \sim (\d^3 N)^l N^2 + (\d N)^l |H|^2 \gg (\d N)^l |H|^2 \,, 
\]
provided $|H| \gg \d^l N$. 
It follows that one can take $l=k+1$ and construct a set $A$ such that $A$ is  $\E_2^k$--uniform but not $\E_3^{k+1}$--uniform. 
\end{example} 


%\bp 



%\bp 

Now let us obtain the characteristic property of the energy $\E^k_l$.


\begin{lemma}
    Let $l,k \ge $ be positive integers number, $lk$ be an even number and $A_j \subseteq \Gr$, $j\in [l]$ be sets. 
    %$\eps$--uniform sets relatively to the energy $\E^k_l$.
    Then for any function $g:\Gr \to \R$ one has 
\begin{equation}\label{f:uniformity} 
    \sum_{x} \prod_{j=1}^l (f_{A_j} \circ g) (x)
        \le 
            \| g\|^{l(1-1/k)}_1 \|\Cf_l (g)\|^{1/k}_\infty  \cdot \prod_{j=1}^l \| f_{A_j} \|_{\ov{\E}^k_l} \,.
\end{equation}
\label{l:uniformity} 
\end{lemma} 
\begin{proof} 
    By the H\"older inequality and \cite[Lemma 29]{sh_Ek} and the duality one has 
\[
    \sum_{x} \prod_{j=1}^l (f_{A_j} \circ g) (x)  = N^{-1} \sum_{|z|=l} \Cf_l (f_{A_1}, \dots, f_{A_l}) (z) \Cf_l (g) (z) 
\]
\[
    \le 
    N^{-1+1/k} \prod_{j=1}^l \| f_{A_j} \|_{\ov{\E}^k_l} \cdot 
    \left( \sum_{|z|=l} |\Cf^{k/(k-1)}_l (g) (z)| \right)^{1-1/k}
%\]
%\[
    \le 
    %\| f\|^l_{\ov{\E}^k_l}  
    \prod_{j=1}^l \| f_{A_j} \|_{\ov{\E}^k_l} \cdot 
    \|\Cf_l (g) \|^{1/k}_\infty \| g \|^{l(1-1/k)}_1
    %\,.
\]
%as required. 
This completes the proof. 
$\hfill\Box$
\end{proof}

\begin{corollary}
    Let $A,B\subseteq \Gr$ be sets, $|A|=\d N$, $|B| = \beta N$, and $l$ be a positive integer.  
    %For any $C>0$ take $k =\lceil C l \log (1/\beta) \rceil$ 
    Take $k =2 \lceil e l \log (1/\beta) \rceil$
    and suppose that $A$ is 
    $\eps$--uniform relatively to the energy $\E^k_l$.
    Then
\begin{equation}\label{f:A_circ_B}
    \sum_{x} (A\circ B)^l (x) \le  \delta^l |B|^l N  \cdot \min\{ 5/4 (1+\eps)^l, (1+1.25 \eps)^l\}  \,.
\end{equation}
\label{c:A_circ_B}
\end{corollary} 
\begin{proof} 
    Using the formula $A(x) = f_A(x) +\d$, combining with Lemma \ref{l:uniformity},  we see that the left--hand side of \eqref{f:A_circ_B} is 
\[
    \sum_{j=0}^l \binom{l}{j} (\d |B|)^{l-j} \sum_x (f\circ B)^j (x) 
    \le 
    \sum_{j=0}^l \binom{l}{j} (\d |B|)^{l-j} \| f\|^j_{\ov{\E}^k_j} |B|^{j(1-1/k)+1/k} 
\]
\[
    \le \d^l |B|^{l} N \sum_{j=0}^l \binom{l}{j} \eps^j \beta^{-(j-1)/k}
    \le \frac{5}{4} \delta^l |B|^l N (1+\eps)^l 
    %\,.
\]
as required and similarly the second bound. 
This completes the proof. 
$\hfill\Box$
\end{proof}


\bp

%\begin{remark}
   Let us remark that, of course,  the energy $\E_2^k$ solely allows us to control sums from \eqref{f:A_circ_B} but our task is to obtain the correct power of $\d$ and $|B|$ in the right--hand side of this estimate. 
    %estimate \eqref{f:A_circ_B}
%%    More precisely, we can the following characterization.  
%\end{remark}

%\begin{proposition}
%\end{proposition}

\bp 

%*%
%We need one more result about uniform set, which we will use in Section \ref{sec:corners}. 
We need one more result about uniform set, which is useful for applications. 



\begin{lemma}
    Let $A_1,\dots, A_l \subseteq \Gr$ be sets, $|A_j| = \d_j N$. 
    Suppose that all $A_j$ are $\eps$--uniform relatively to $\E^k_l$ and 
    %$\eps \le (2l)^{-l}$. 
    $2 l \eps^k \le 1$. 
    Then 
\begin{equation}\label{f:dispersion}
    \sum_{|x|=l} \left( \Cf_l (A_1,\dots, A_l) (x) - N\prod_{j=1}^l \d_j \right)^k
    \le 
    \eps^{k} 
    %(2l)^{kl} 
    l 2^{kl+1}
    N^{l+k} \left( \prod_{j=1}^l \d_j  \right)^k \,.
\end{equation} 
\label{l:dispersion}
\end{lemma}
\begin{proof} 
    Put $\Pi = \prod_{j=1}^l \d_j$. 
Then the left--hand side of \eqref{f:dispersion} is 
\[
    \sigma:= \sum_{|x|=l} \left( \sum_{\emptyset \neq S \subseteq [l]} \Cf_{l} (f_{1}, \dots, f_l) (x) \right)^k
    = \sum_{|x|=l} \left( \sum_{\emptyset \neq S \subseteq [l]} F_S (x) \right)^k \,,
\]
    where for $j\in S$ we put $f_j=f_{A_j}$ and if $j\notin S$, then $f_j = \d_j$. 
    Using $\eps$--uniformity of all sets $A_j$, combining with \cite[Lemma 29]{sh_Ek}, we get 
\[
    \| F_S\|^{kl}_{\E^k_l} \le \eps^{|S|k} \Pi^k N^{l+k} \,.
\]
%    Thus by the norm property of $\E^k_l$ one has 
    Thus by the H\"older inequality one has
\[
    \sigma  \le 2^{kl} \sum_{\emptyset \neq S \subseteq [l]} \| F_S\|^{kl}_{\E^k_l}
    \le 
    2^{kl} \Pi^k N^{k+l} ((1+\eps^k)^l - 1)
    \le l2^{kl+1} \eps^k \Pi^k N^{k+l} 
\]
\begin{comment}
\[
    \sigma \le  \left( \sum_{\emptyset \neq S \subseteq [l]} \| F_S\|_{\E^k_l} \right)^{kl}
    \le \Pi^k N^{k+l}   \left( \sum_{\emptyset \neq S \subseteq [l]} \eps^{|S|/l} \right)^{kl}
    \le \Pi^k N^{k+l} \left( (1+\eps^{1/l})^l -1 \right)^{kl}
\]
\[
    \le \Pi^k N^{k+l} (2l)^{kl} \eps^k 
    %\,.
\]
\end{comment} 
as required. 
%This completes the proof. 
$\hfill\Box$
\end{proof}


\bp 


Let us consider one more  example which shows that one can delete/add a tiny subset from a non--uniform set to obtain a uniform one. This phenomenon has no place if we consider the classical uniformity in terms of the Fourier transform or in terms of Gowers norms \cite{Gowers_m}, say. The reason is  normalization \eqref{def:uniformity_kl}, of course. 

\bp 


%{\bf Example.} 
\begin{example} 
\label{exm:removing}
Let $\Gr = \F_2^n$, $H< \F_2^n$, $|H|=\beta N$, $\Lambda \subseteq \Gr$ be a random set, $|\Lambda| = \d N$, $\beta \le \d$ and put $A=\tilde{H} \bigsqcup \Lambda$, where $\tilde{H} = H \setminus \Lambda$. Then with high probability  $|\tilde{H}| \sim \beta(1-\d) N = \eps |A|$, the set $\Lambda$ is uniform in any possible sense but $A$ is non $\eta$--uniform set with rather large $\eta$. 
Indeed, by Kelley--Meka method \cite{kelley2023strong} or just see Lemma \ref{l:uniformity}, we know that 
\begin{equation}\label{tmp:23.03_1}
    \sigma:= \sum_{x} (A\circ A)(x) H(x) = (\d+(1-\d)\beta)^2 \beta N^2 + \theta \eta^2 \d^2 \beta N^2 \,,
\end{equation}
    where $|\theta| \le 4$, say,  is a certain number and  
    $A$ is supposed to be $\E_2^k$--uniform with $k \sim \L(\beta)$.
    On the other hand, the direct calculation shows 
\begin{equation}\label{tmp:23.03_2}
    \sigma =  |\tilde{H}|^2 + 2\d\beta^2 (1-\d) N^2 + \d^2 |H|N 
\end{equation}
    plus a negligible  error term.
    Comparing \eqref{tmp:23.03_1} and \eqref{tmp:23.03_2}, we obtain 
\[
    \eta^2 \d^3 \eps \gg 2\d \beta \eps + \beta \eps^2 -  \eps^2 \d^2 \gg \d \eps^3 
\]
    and thus $\eta \gg \eps/\d$ which is much larger than $\eps$ for small $\d$. 


    Similarly, 
    %let us  
    one can 
    show that deleting 
    %from an arbitrary  set $A$ 
    a subspace $H$ from a random set $\Lambda$, $|H| = \eps |A|$, $H$ lives on the first coordinates, say, we obtain a non $\eta$--uniform set with $\eta \gg 1$ thanks to the equality $\sum_{x} H(x) (A\circ A) (x) = 0$. 
\end{example} 

%\bp 


\subsection{Non--uniformity and almost periodicity}
\label{subsec:NU+AP}


The aim of this subsection is to obtain Sanders' almost periodicity result for {\it higher  convolutions}, see Lemma \ref{l:Sanders_V} below. 

%\cite{sanders2011roth}, \cite{sanders2012bogolyubov}, \cite{sanders2013structure}  and  \cite{CS}, \cite{SS4}, reflecting the almost periodicity properties of \cite[Theorem 3.2]{SS4} \cite{CS}  (also, see \cite{sanders2011roth}, \cite{sanders2012bogolyubov}, \cite{sanders2013structure} and, especially, \cite[Theorem 3.2]{SS4})

%Now 
At the beginning 
we want to transfer 
%the large 
a lower bound for the 
energy $\E^k_l (f_A)$ to 
%large energy 
the largeness of the energy 
$\E^k_l (A)$.  
We follow a more simple method from \cite{bloom2023kelley} which differs from the approach of \cite{kelley2023strong} by some logarithms. The dependence on $l$ in the first multiple in  \eqref{f:e_to_d} is, probably, can be improved  significantly (also, see Remark \ref{r:ET_fail} from the appendix)
but in our regime $l=O(1)$ and 
%the dependence  
thus it is not so critical for us. 



\begin{lemma}
    Let $A\subseteq \Gr$ be a set, $|A|= \d N$ and 
%    $\eps \in (0,1)$ be a parameter.
    $\eps>0$ be a parameter, $\eps_* := \min \{\eps, 1\}$.
    Suppose that for an odd $k\ge 5$ one has 
\begin{equation}\label{cond:e_to_d}
    \E_l^k (f_A) = \eps^{lk} \d^{lk} N^{l+k} \,,
\end{equation} 
    and that for $k_* = O(kl \eps^{-l}_* \L(\eps_*))$ the set $A$ is $\frac{\eps \eps^{l-1}_*}{8l}$--uniform relatively to $\E^{k_*}_{l-1}$.
    Then there is $k_1 \le k_*$ such that 
\begin{equation}\label{f:e_to_d}
    \E_l^{k_1} (A) \ge \left(1+\frac{\eps \eps^{l-1}_*}{8l}  \right)^{lk_1} \d^{lk_1} N^{l+k_1} \,.
\end{equation} 
\label{l:e_to_d}
\end{lemma}
\begin{proof} 
    Write $f(x)= f_A (x)$. 
   We have 
\[
    \eps^{lk} \d^{lk} N^{l+k} \le \sum_{|x|=l} \Cf^{k-1}_l (f) (x) |\Cf_l (f) (x)| + \sum_{|x|=l} \Cf^{k}_l (f) (x) 
    = 2 \sum_{|x|=l} \Cf^{k-1}_l (f) (x) \cdot \max\{ 0, \Cf_l (f) (x)\} \,,
\]
    and thus considering the set $P = \{ x ~:~ \Cf_l (f) (x) \ge 0\}$, we get 
\begin{equation}\label{tmp:03.03_1}
    \sum_{x\in P} \Cf^{k}_l (f) (x) \ge 2^{-1} \eps^{lk} \d^{lk} N^{l+k} \,.
\end{equation} 
    Now let consider the subset of the set $P$
\[
    P_\eps := \{ x ~:~ \Cf^{}_l (f) (x) \ge \frac{3}{4} \eps^l \d^l N\} \,.
\]
    Then we have 
\begin{equation}\label{tmp:03.03_2}
    \sum_{x\notin P_\eps} \Cf^{k}_l (f) (x) \le 
    \left( \frac{3}{4} \eps^l \d^l N \right)^k N^l 
    \le 
    2^{-2} \eps^{lk} \d^{lk} N^{l+k} \,.  
\end{equation} 
    Combining \eqref{tmp:03.03_1}, \eqref{tmp:03.03_2} and using the H\"older inequality, we obtain 
\begin{equation}\label{tmp:03.03_3}
    |P_\eps| \E^{2k}_{l} (f) \ge 2^{-4} \eps^{2lk} \d^{2lk} N^{2l+2k} \,. 
\end{equation} 
    By the norm property of $\E^{2k}_{l} (f)$ one has 
\[
    \E^{2k}_{l} (f) \le \left( \| A\|_{\E^{2k}_l} + \| \d\|_{\E^{2k}_l} \right)^{2kl} 
    \le 
    (2+\eps/2)^{2kl} \d^{2kl} N^{l+2k} 
%    <
%     (5\d/2)^{2kl} N^{l+2k} 
\]
    otherwise there is nothing to prove with $k_1=2k$ and much larger $\eps$. 
    Thus we derive from \eqref{tmp:03.03_3} that 
    $|P_\eps| \ge (2\eps_*/5)^{2kl} N^l$. 
    %and hence
    Now 
\[
    \Cf_l (A) (x) = \Cf_l (f+\d) (x) =\d^l N + \Cf_l (f) + \sum_{ S\subseteq [l] ~:~ 1\le |S|<l} \d^{l-|S|} \Cf_{|S|} (f) (x_S) = \d^l N + \Cf_l (f) + \mathcal{E} (x) \,,
\]
    where for a set $S\subseteq [l]$
    %we have denoted 
    the vector $x_S$ has coordinates $x_j$, $j\in S$. 
    By the triangle inequality for $L_{k_1}$--norm, we have 
\begin{equation}\label{tmp:06.03_1}
    (\E^{k_1}_{l} (A))^{1/k_1} = \| \Cf_l (A) \|_{k_1} = \| \Cf_l (f+\d) \|_{k_1}
    \ge 
    \| \d^l N + \Cf_l (f) \|_{k_1} - \| \mathcal{E}\|_{k_1} \,.
\end{equation} 
    Using our bound for the cardinality of the set $P_\eps$, we get
\[
    \| \d^l N + \Cf_l (f) \|^{k_1}_{k_1} \ge \sum_{x\in P_\eps} (\d^l N + \Cf_l (f))^{k_1} (x) \ge (2\eps_*/5)^{2kl} N^l \cdot (1+3\eps^l/4)^{k_1} \d^{lk_1} N^{k_1} 
\]
\begin{equation}\label{tmp:06.03_2}
    \ge 
    (1+\eps^l/2)^{k_1} \d^{lk_1} N^{k_1+l}  \,,
\end{equation} 
    provided $k_1 \ge 20 kl \eps^{-l}_* \L(\eps_*)$. 
    On the other hand, by our assumption the set $A$ is $\zeta:=\frac{\eps^l}{8l}$--uniform relatively to 
    $\E^{k_*}_j$ for all $j<l$ and $k_* = k_1$.
    It follows that 
\begin{equation}\label{tmp:06.03_3}
    \| \mathcal{E}\|_{k_1} \le \sum_{ S\subseteq [l] ~:~ 1\le |S|<l} \d^{l-|S|}
    N^{\frac{l-|S|}{k_1}} \| f\|^{|S|}_{\E^{k_1}_{|S|}} 
    \le \d^l 
    N^{\frac{l+k_1}{k_1}} ((1+\zeta)^{l}-1) \,.
\end{equation} 
    Combining \eqref{tmp:06.03_1}, \eqref{tmp:06.03_2} and \eqref{tmp:06.03_3}, we obtain 
\[
    \E^{k_1}_{l} (A) \ge \d^{lk_1} N^{k_1+l} \left( 2 +\eps^l/2  -  (1+\zeta)^{l} ) \right)^{k_1}
    \ge  \d^{lk_1} N^{k_1+l} \left(1+\frac{\eps \eps^{l-1}_*}{8l} \right)^{lk_1} 
\]
as required. 
%This completes the proof. 
$\hfill\Box$
\end{proof}


\bp 

\begin{comment} 
The following result shows that removing small subsets of a set $A\subseteq \Gr$ we do not decrease the energy $\E_l^k (A)$ too much. 

\begin{lemma}
    Let $U\subseteq A \subseteq \Gr$, $A':=A\setminus U$  be sets, $|A|= \d N$, $|U|=\o |A|$, and 
\begin{equation}\label{cond:A,U}
    \E_l^k (f_A) = \eps^{lk} \d^{lk} N^{l+k} \,.
%    \E^k_l (A) \ge (1+\eps)^{lk} \d^{lk} N^{l+k} \,.
\end{equation}
    Suppose that  for $k_* = O(kl \eps^{-l}_* \L(\eps_*))$ the set $A$ is $\frac{\eps^l}{8l}$--uniform relatively to $\E^{k_*}_{l-1}$.
    Then 
    Then 
\begin{equation}\label{f:A,U}
    \E^k_l (A') \ge (1+\eps)^{lk} \d^{lk} N^{l+k} \,.
\end{equation}
\label{l:A,U}
\end{lemma}
\begin{proof} 
    Applying Lemma \ref{l:e_to_d} we find $k_1 \le k_*$ such that 
\begin{equation*}\label{}
    \E_l^{k_1} (A) \ge \left(1+\frac{\eps \eps^{l-1}_*}{8l}  \right)^{lk_1} \d^{lk_1} N^{l+k_1} \,.
\end{equation*} 
    Now $\| A'\|_{\E^k_l} \ge \| A\|_{\E^k_l} - \| U \|_{\E^k_l}$ and 
    $\| U \|^{kl}_{\E^k_l} \le \sum_x $
as required. 
%This completes the proof. 
$\hfill\Box$
\end{proof}

\bigskip 
\end{comment} 

Now we use duality \eqref{def:E_kl} to  obtain an appropriate version of multi--dimensional version of the Balog--Szemer\'edi--Gowers theorem as was done in \cite{Schoen_BSzG} (also, see \cite[Theorem 17]{sh_str_survey}).
Thanks to duality \eqref{f:E^k_st_duality} one can show that 
%the same 
a similar 
result takes place for more general energies $\mathcal{E}^k_{s,t}$, see the appendix. 
Of course, in this case one needs to replace 
%$N^{-1} \Cf_{|x|} (\cdot) (x)$ to $N^{-2} \Cf_{|x||z|} (\cdot) (x\oplus z)$ or  $N^{-2} \Cf_{|y||z|} (\cdot) (y\oplus z)$ 
$\Cf_{|x|} (\cdot) (x)$ to $\Cf_{|x||z|} (\cdot) (x\oplus z)$ or  $\Cf_{|y||z|} (\cdot) (y\oplus z)$ 
and use symmetries \eqref{f:E^k_st_symmetries}  instead of the symmetry \eqref{f:E^k_l_symmetries} below.


\begin{lemma}
    Let $A \subseteq \Gr$ be a set, $|A|= \d N$ and $\eps>0$, $\eta \in (0,1/2)$ be parameters. 
    Suppose that for some integers $k,l\ge 2$ with $kl \ge 4\eps^{-1}_* \L(\eta)$ one has  
\begin{equation}\label{cond:BSzG_k}
    \E^k_l (A) \ge (1+\eps)^{lk} \d^{lk} N^{l+k} \,.
\end{equation} 
    Define the set 
\begin{equation}\label{def:BSzG_k}
    S = \{ |x|=l ~:~ \Cf_l (A) (x) \ge (1+\eps/4)^l \d^l N \} \,.
\end{equation} 
    Then there is a set $B$ such that 
\begin{equation}\label{f:BSzG_k}
    N^{-1} \sum_{|x|=l} S(x) \Cf_l (B) (x) \ge (1-2\eta) |B|^l \,,
\end{equation} 
    and 
$|B| > 2^{-1/(l-1)} (1+\eps)^k  \d^k N$.
\label{l:BSzG_k}
\end{lemma}
\begin{proof} 
    For {\it any} set $S \subseteq \Gr^l$ with the property $S(x + \mD_l(t)) = S(x)$, $t\in \Gr$, $x\in \Gr^l$,  we have 
\[
    \sum_{|x|=l} S(x) \Cf^k_l (A) (x)
    =
    \sum_{|x|=l} S(x) \sum_{|z|=k} A^k (z+\mD_k (x_1)) \dots A^k (z+\mD_k (x_l))
\]
\begin{equation}\label{tmp:28.02_0}
    =
    \sum_{|z|=k}\, \sum_{|x|=l} S(x) A_{z} (x_1) \dots A_z (x_l)
    =
    N^{-1} \sum_{|z|=k}\, \sum_{|x|=l} S(x) \Cf_l (A_z) (x) \,,
\end{equation} 
where we have made the change of the variables $x_j \to x_j+t$ in the last formula. 
Clearly, we have from identity \eqref{f:E^k_l_symmetries} and definition \eqref{def:BSzG_k} that 
    $S(x + \mD_l(t)) = S(x)$ for and $t\in \Gr$ and $x\in \Gr^l$ and thus the argument above can be applied for the set $S$ as well. 
%    On the other hand, 
    Thus using the definition of the set $S$, as well as the conditions \eqref{cond:BSzG_k} and  $kl \ge 4\eps^{-1}_* \log (4/\eta)$, we get 
\begin{equation}\label{tmp:28.02_2*}
    \sum_{x\notin S}  \Cf^k_l (A) (x) \le (1+\eps/4)^{lk} \d^{lk} N^{k+l}
    \le 
    2^{-2} \eta (1+\eps)^{lk} \d^{lk} N^{k+l} 
    \le 
    2^{-2} \eta \E^k_l (A) \,.
\end{equation} 
    Now define  the set 
\[
    \Omega = \left\{ |z|=k ~:~ |A_z| \ge 2^{-1/(l-1)} (1+\eps)^k  \d^k N \right\} \,.
\]
    By the definition of the set $\Omega$, we derive 
\begin{equation}\label{tmp:28.02_2**}
\sum_{z\notin \Omega} |A_z|^l \le 
    (\max_{z\notin \Omega} |A_z|)^{l-1} |A|^k N \le 
    2^{-1} (1+\eps)^{lk} \d^{lk} N^{l+k} \le 
    2^{-1} \E^k_l (A)  \,. 
\end{equation} 
In view of 
%inequalities 
bounds 
\eqref{tmp:28.02_2*}, \eqref{tmp:28.02_2**} one has 
\begin{equation}\label{f:prob}
    N^{-1} \sum_{z\in \Omega} \left( \sum_{x\in S}  \Cf_l (A_z) (x) - \eta^{-1} \sum_{x\notin S}  \Cf_l (A_z) (x) \right) \ge 2^{-1} \sum_{|z|=k} |A_z|^l - 2^{-1} \E^k_l (A) = 0 \,.
\end{equation} 
    Hence there is $z\in \Omega$ such that inequality \eqref{f:BSzG_k} holds for $B=A_z$ and $|B| > 2^{-1/(l-1)} (1+\eps)^k  \d^k N$ as required. 
\begin{comment}     

\bigskip 

    Now let us consider the case of the energy  $\mathcal{E}^k_{s,t}$.  
    Let $l=st$. 
    %One has 
    We have an analogue of \eqref{tmp:28.02_0} 
\[
    \sum_{|x|=s}\, \sum_{|y|=t} S(x\oplus y) \Cf^k_l (A) (x\oplus y) = 
    N^{-1} \sum_{|z|=k}\, \sum_{|x|=s}\, \sum_{|y|=t} S(x\oplus y) \Cf_l (A_z) (x\oplus y)
\] 
    and thus we need to define 
\[
    \Omega = 
    %\left\{ z ~:~  \sum_{|x|=s}\, \sum_{|y|=t} S(x\oplus y) \Cf_l (A_z) (x\oplus y) \le (1-2\eta) \E_s^t (A_z) N \right\} \,.
    \left\{ z ~:~ |A_z| >  2^{-\frac{1}{s+t-2}}  (1+\eps)^{\frac{stk}{s+t-2}} \d^{\frac{k(st-1)}{s+t-2}} N \right\} \,.
\]
    Then as in \eqref{tmp:28.02_2**} one has 
    \[
\sum_{z\notin \Omega} \E_{s}^t (A_z) 
\le N \sum_{z\notin \Omega} |A_z|^{s+t-1} \le (\max_{z\notin \Omega} |A_z|)^{s+t-2} |A|^k N^2 \le 
2^{-1} (1+\eps)^{lk} \d^{lk} N^{s+t+k} \le 2^{-1} \mathcal{E}^k_{s,t} (A) \,.
 \]
    Also, by the definition of the set $S$, namely, 
\[
   S = \{ |x|=s, |y|=t  ~:~ \Cf_l (A) (x\oplus y) \ge (1+\eps/4)^l \d^l N \} 
   %\,.
\]    
    we see that the sum over $x\notin S$ is negligible 
    (consult 
    %calculations 
    computations 
    in  \eqref{tmp:28.02_2*}). 
    By \eqref{f:E^k_st_duality} we know that $\sum_{|z|=k}  \E_s^t (A_z) = \mathcal{E}^k_{s,t} (A)$ and hence we can repeat the calculations as in \eqref{f:prob}. 
    %the contribution over $z\in \Omega$ is negligible similar to 
    %as in  \eqref{tmp:28.02_1}. 
    Namely, thanks to identity \eqref{f:E^k_st_expectation} one has 
\[
    N^{-1} \sum_{z\in \Omega} \left( \sum_{(x\oplus y) \in S}   \Cf_l (A_z) (x\oplus y) - \eta^{-1} \sum_{(x\oplus y) \in S}  \Cf_l (A_z) (x \oplus y) \right) \ge 2^{-1} \sum_{|z|=k} \E_s^t (A_z) - 2^{-1} \mathcal{E}^k_l (A) = 0 
    %\,.
\]
    and the existence of the required set  $B=A_z$ follows immediately. 
\end{comment} 
%as required. 
%This completes the proof. 
$\hfill\Box$
\end{proof}


\bigskip 

Now we need an analogue of the almost periodicity result \cite{CS}  (also, see \cite{sanders2011roth}, \cite{sanders2012bogolyubov}, \cite{sanders2013structure} and, especially, 
\cite[Theorem 3.2]{SS4}) for the higher convolutions. 
This theme is rather well--known and thus we give just a scheme of the proof emphasizing the necessary distinctions we must make.  
%%(also, some details can be found in Lemma \ref{l:CS_new} from the appendix where even more difficult situation is considered).
%%Again, we prove our results for $\Cf_{|x|} (x)$ (actually, for $\Cf'_{|x|+1} (x)$) 
%for convenience 
%%but the same is true for $\Cf_{|x||z|} (x\oplus z)$ or for $\Cf_{|y||z|} (y\oplus z)$. 
For the convolution  $\Cf_{|x||z|} (x\oplus z)$ a similar result takes place, see Lemma  \ref{l:CS_new} from the appendix. 


\begin{lemma}
    Let $\Gr=\F_p^n$, $l$ be an integer and $\epsilon  \in (0,1]$ be a real parameter.
    Also, let $B\subseteq \Gr$ be a set, $|B|=\beta N$, and $f:\Gr^l \to [-1,1]$ be a function. 
    Then there is a subspace $V\le \Gr$ with 
\begin{equation}\label{f:Sanders_V_codim}
        \mathrm{codim} V \ll \epsilon^{-2} l \L^2 (\beta) \L^2 (\epsilon \beta^l)
\end{equation} 
    and such that 
\begin{equation}\label{f:Sanders_V}
    \left| \sum_{|x|=l} f(x) (B^l \circ \mD_l (B * \mu_V)) (x) - \sum_{|x|=l} f(x) (B^l \circ \mD_l (B)) (x)  \right| 
    \le \epsilon |B|^{l+1} \,.
\end{equation} 
\begin{comment} 
    A similar result is true for 
    %the energy $\mathcal{E}^k_{s,t}$, 
    $\Cf_{st} (B) (x\oplus y)$, 
    namely, for any $s,t\ge 3$, $l=st-1$, we get 
\[
%    \left| 
    |\sum_{|x|=s-1}\, \sum_{|y|=t-1} f(\bar{x}\oplus \bar{y}) (B^l \circ \mD_l (B * \mu_V)) (\bar{x}\oplus \bar{y}) -
\]
\begin{equation}\label{f:Sanders_V+}
    \sum_{|x|=s-1}\, \sum_{|y|=t-1} f(\bar{x}\oplus \bar{y}) (B^l \circ \mD_l (B)) (\bar{x}\oplus \bar{y})
    |
 %   \right| 
    \le \eps \bar{\E}^t_{s} (B) \,.
\end{equation} 
\end{comment} 
\label{l:Sanders_V}
\end{lemma} 
\begin{proof}
    We begin with a rather general argument which takes place in any abelian group $\Gr$. 
    Let $k\ge 2$ be an integer parameter and $q\ge 2$ be a real parameter.  
    Applying the Croot--Sisask lemma \cite{CS}, \cite{sanders2012bogolyubov} (clearly, one has  $|\mD_l (B) + \mD_l (\Gr)| \le \beta^{-1} |\mD_l (B)|$, so the set $\mD_l (B)$ has the small doubling), 
%    one has 
    we find a set $T\subseteq \Gr$, $|T|\ge |B| \exp(-O(\epsilon^{-2} qk^2 \log (1/\beta)))$    and such that for any $t\in kT$ the following holds 
\[
    \sum_{|x|=l} \left|  (f \circ \mD_l (B)) (x+\mD_l (t)) - 
    \sum_{|x|=l} (f \circ \mD_l (B)) (x) \right|^q 
\]
\begin{equation}\label{tmp:28.02_5}
        \le 
            \left(\frac{\epsilon}{4} \right)^q \| f\|_q^q |B|^q 
                \le      
                \left(\frac{\epsilon}{4} \right)^q |B|^q N^l \,.
\end{equation}
    Fixing $t\in kT$ and using the H\"older inequality, combining with estimate  \eqref{tmp:28.02_5}, we get 
\[
    \left| \sum_{|x|=l} f(x) (B^l \circ \mD_l (B)) (x+\mD_l (t)) 
    - \sum_{|x|=l} f(x) (B^l \circ \mD_l (B)) (x) \right| 
        \le 
        \frac{\epsilon}{4} |B| N^{l/q} |B|^{l(1-1/q)}
\]
\begin{equation}\label{tmp:28.02_5.5}
    =
        \frac{\epsilon}{4} \beta^{-l/q} |B|^{l+1}
        \le 
        \frac{\epsilon}{2} |B|^{l+1} \,,
\end{equation}
    where we have taken $q=C l \log (1/\beta)$ for a sufficiently large constant $C>0$. 
    It follows that 
\begin{equation}\label{tmp:28.02_6}
    \left| \sum_{|x|=l} f(x) (B^l \circ \mD_l (B * \mu^{(k)}_T)) (x) - \sum_{|x|=l} f(x) (B^l \circ \mD_l (B)) (x) \right| \le 2^{-1} \epsilon |B|^{l+1} \,. 
\end{equation}
    Let us analyze the sum $\sigma:= |\sum_{|x|=l} f(x) (B^l \circ \mD_l (B * \mu^{(k)}_T)) (x)|$ from \eqref{tmp:28.02_6}. 
    Clearly, $\FF{\mu}_T (r_1,\dots, r_l) = |T|^{-1} \FF{T}(r_1+\dots+r_l)$ and thus 
\begin{equation}\label{tmp:28.02_7}
    \sigma \le \frac{|B|}{|T|^k N^l} \sum_{z} |\FF{T} (z)|^k
    \sum_{r_1+\dots+r_l=z} |\FF{f} (r_1,\dots,r_l)| |\FF{B}(r_1)| \dots |\FF{B}(r_l)| \,.
\end{equation}
    As usual let us estimate the last sum over $z\in \Spec_c (T)$, where  $c\in (0,1]$ is a parameter  and  over $z\notin \Spec_c (T)$, 
    where we have defined 
\[
    \Spec_c (T) = \{ z\in \Gr ~:~ |\FF{T} (z)| \ge c |T|\} \,. 
\]
By the definition of the set $\Spec_c (T)$, the H\"older inequality and the Parseval identity, we have 
\[
    \sigma_1 :=
\frac{|B|}{|T|^k N^l} \sum_{z\notin \Spec_c (T)} |\FF{T} (z)|^k
    \sum_{r_1+\dots+r_l=z} |\FF{f} (r_1,\dots,r_l)| |\FF{B}(r_1)| \dots |\FF{B}(r_l)| 
\]
\[ 
    \le 
    \frac{c^k |B|}{N^l} \sum_{r_2,\dots,r_l} 
    |\FF{B}(r_2)| \dots |\FF{B}(r_l)| 
    \sum_{z} |\FF{f} (z-r_2+\dots+r_l,\dots,r_l)| |\FF{B} (z-r_2+\dots+r_l)|
\]
\[
    \le 
    \frac{c^k |B|^{3/2}}{N^{l-1}} \sum_{r_2,\dots,r_l} 
    |\FF{B}(r_2)| \dots |\FF{B}(r_l)| \left( \sum_a |\FF{f}_a (r_2,\dots,r_l)|^2 \right)^{1/2} \,,
\]
    where $f_a(x_2,\dots,x_l) = f(a,x_2,\dots,x_l)$. 
    Using the H\"older inequality and the Parseval formula one more time, we derive 
\[
    \sigma_1 \le \frac{c^k |B|^{(l+2)/2}}{N^{(l-1)/2}}
    \left( \sum_{r_2,\dots,r_l} \sum_a |\FF{f}_a (r_2,\dots,r_l)|^2  \right)^{1/2}
    \le 
    c^k |B|^{(l+2)/2} N^{l/2} \le 2^{-1}\epsilon |B|^{l+1} \,.
\]
    Here we have taken $c = 1/2$ and $k = \lceil 2\L (\epsilon \beta^l) \rceil$, say. 
    For the sum over $z\in \Spec_{c} (T)$ we use the Chang lemma \cite{chang2002polynomial} about the dimension of the spectrum and find a subspace $V$ such that \eqref{f:Sanders_V} takes place and 
\[
    \mathrm{codim} V \ll \log (N/|T|) \ll \epsilon^{-2} l \L^2 (\beta) k^2
    \ll
     \epsilon^{-2} l \L^2 (\beta) \L^2 (\epsilon \beta^l) \,,
\]
%as required. 
see details in \cite{CS}, \cite{sanders2012bogolyubov}, \cite{sanders2013structure} or in \cite[Section 5]{sh_str_survey}. 
\begin{comment}

\bigskip 


Now let us obtain \eqref{f:Sanders_V+}. 
%Denote by $\sigma$ the left--hand side of this bound. 
Let $\sigma_*$ be the left--hand side of \eqref{f:Sanders_V+}. 
In this case we use an appropriate variant of the Croot--Sisask result (see Lemma \ref{l:CS_new} from the appendix) and obtain as in \eqref{tmp:28.02_5}, \eqref{tmp:28.02_5.5} 
\[
    \sigma_* \le \frac{\eps}{4} \left( |B|^{q-1} \sum_{|y|=t-1} \Cf'_t (|f|^q)^{s-1} (y) \cdot \Cf'_t (B,|f|^q, \dots, |f|^q) (y) \right)^{1/q} \cdot \left( \sum_{|x|=s-1}\, \sum_{|y|=t-1} B^l (\bar{x}\oplus \bar{y}) \right)^{1-1/q} 
\]
\begin{equation}\label{tmp:04.03_1}
    \le \frac{\eps}{4}  |B| N^{\frac{s+t-2}{q}} \bar{\E}^{s-1}_t (B)^{1-1/q} \,.
\end{equation} 
Now by the H\"older inequality, we get 
\begin{equation}\label{f:E_st_Holder}
    \E^{s-1}_t (B) \le (\E^{s}_t (B))^{\frac{s-2}{s-1}} |B|^{\frac{t}{s-1}} 
        \quad \quad \mbox{and} \quad \quad 
    \E^{s}_t (B) \ge \beta^{st} N^{s+t} \,.
\end{equation} 
    Substituting estimate \eqref{f:E_st_Holder} to \eqref{tmp:04.03_1}, we derive
\[
    \sigma_* \le \frac{\eps}{4} \bar{\E}^{s}_t (B) \cdot (\bar{\E}^{s}_t (B))^{-\left(\frac{1}{q}+ \frac{1}{s-1} \left(1-\frac{1}{q}\right)\right)} |B|^{1+\frac{t}{s-1}\left(1-\frac{1}{q}\right)} N^{\frac{s+t-2}{q}}
    \le 
    \frac{\eps}{4} \beta^{-\frac{ts}{q}} \bar{\E}^{s}_t (B)
    \le 
     \frac{\eps}{2} \bar{\E}^{s}_t (B)
\]
    thanks to our choice of $q$. 
    Arguing as above, we see that it remains to estimate the Fourier transform sum  
\[\sum_{|x|=s-1}\, \sum_{|y|=t-1} f(\bar{x}\oplus \bar{y}) (B^l \circ \mD_l (B*\mu^{(k)}_T)) (\bar{x}\oplus \bar{y})
\]
    outside the spectrum $\Spec_\epsilon (T)$. 
\end{comment} 
This completes the proof. 
$\hfill\Box$
\end{proof}



\section{Some generalizations of Kelley--Meka results}

Using the density increment Kelley--Meka \cite{kelley2023strong} 
(or just repeat the calculations of the previous section, combining forthcoming  Proposition \ref{p:increment_Ekl} in the case $l=2$) 
obtained the following result. 

\begin{theorem}
    Let $\Gr=\F_p^n$, $A\subseteq \Gr$ be a set, $|A|=\d N$, and $\eps>0$ be a parameter.
    Then there is a subspace $V\subseteq \Gr$ and $x\in \Gr$ such that $A\cap (V+x)$ is $\eps$--uniform relatively to $\E^k_2$, $\mu_{V+x} (A) \ge \d$, and 
\begin{equation}\label{f:uniform_KM}
    \codim V \ll \eps^{-14} k^4 \L^3 (\d) \L^{2} (\eps \d) \cdot \L^{4} (\eps) \,.
    %7->14
\end{equation}
\label{t:uniform_KM}
\end{theorem}

\begin{remark}
    Actually, we formulate Theorem \ref{t:uniform_KM} in the form of Bloom--Sisask \cite{bloom2023kelley}. Kelley--Meka \cite{kelley2023strong} obtained this result without $\L^{4} (\eps)$ in \eqref{f:uniform_KM}. 
\end{remark} 


We generalize Theorem \ref{t:uniform_KM} for the higher energies $\E^k_l$.  



\begin{theorem}
    Let $\Gr=\F_p^n$, $A\subseteq \Gr$ be a set, $|A|=\d N$, and $\eps \in (0,1]$ be a parameter.
    Then there is a subspace $V\subseteq \Gr$ and $x\in \Gr$ such that $A\cap (V+x)$ is $\eps$--uniform relatively to $\E^k_l$, $\mu_{V+x}(A) \ge \d$ and 
\begin{equation}\label{f:uniform_KM_k}
    \codim V \ll 
%    \eps^{-3l^2} k^4  l^{9l} \L^{4l} (\eps) \L^{5l} (\d) 
\eps^{-28l^l} (8l)^{28l^l} k^4 \L^{4l} (\eps) \L^{5l} (\d)
    \,.
\end{equation}
\label{t:uniform_KM_k}
\end{theorem}




Now we are ready to obtain our driving result about the density increment. 
As always we will apply Proposition \ref{p:increment_Ekl} in an iterative way  and we see that estimate \eqref{cond:increment_2_Ekl}  allows us to do it in at most $O(\eps^{-1} \L(\d))$ times. 

\begin{proposition}
    Let $\Gr = \F_p^n$, $A\subseteq \Gr$, $|A|=\d N$, $\eps>0$ be a real number and $k,l \ge 2$ be positive integers, $kl \gg \eps^{-1} \L(\eps)$. 
    Suppose that 
\begin{equation}\label{cond:increment_Ekl}
    \E^k_{l} (A) \ge (1+\eps)^{kl} \d^{kl} N^{k+l} \,,
\end{equation} 
    and that $A$ is $\eps/5$--uniform relatively to $\E^{k_*}_{l-1}$ for  $k_* = O(kl \L (\delta))$.
    Then there is a subspace $V \subseteq \Gr$ such that 
\begin{equation}\label{cond:increment_1_Ekl}
    \mathrm{codim} V \ll \eps^{-2} l^3 k^4 \L^2 (\d) \L^2 (\eps \d)
    \,,
\end{equation} 
    and for a certain $x\in \Gr$ one has 
\begin{equation}\label{cond:increment_2_Ekl} 
    |A\cap (V+x)| \ge (1+\eps/8) \d |V| \,.
\end{equation} 
\label{p:increment_Ekl}
\end{proposition}
\begin{proof} 
%    By our assumption  $kl \gg \eps^{-1} \L(\eps)$. 
    %Hence 
    Applying Lemma \ref{l:BSzG_k} for the energy $\E^k_{l} (A)$ with the parameters $\epsilon = \eps$, $\eta = \eps/30$, we construct the set 
\[
    S = \{ |x|=l ~:~ \Cf_l (A) (x) \ge (1+\eps/4)^l \d^l N \}
\]
and such that for a certain set $B \subseteq \Gr$,  $|B| > 2^{-1/(l-1)} (1+\eps)^k \d^k N := \beta N$ the following holds 
\begin{equation}\label{tmp:01.03_1_Ekl} 
    N^{-1} \sum_{|x|=l} S (x) \Cf_l (B) (x) \ge (1-2\eta) |B|^l \,.
\end{equation} 
    We have $kl \ge 4\eps^{-1} \L (\eta)$ and thus Lemma \ref{l:BSzG_k} can be applied indeed. 
    Using formulae \eqref{f:E^k_l_symmetries} and making the required change of the variables (in \eqref{def:Cf} we put, consequently, $z\to z-x_1$) one can see that \eqref{tmp:01.03_1_Ekl} is equivalent to 
\begin{equation}\label{tmp:01.03_2_Ekl} 
    %N^{-2} 
    \sum_{|x|=l-1} \bar{S} (x) (B^{l-1} \circ \mD_{l-1} (B)) (x)  \ge (1-2\eta) |B|^l \,,
\end{equation} 
    where $\bar{S} \subseteq \Gr^{l-1}$ is a certain set which is constructed via the set $S$, see formulae \eqref{f:E^k_l_symmetries}, \eqref{f:E^k_l:C,C'}. %\eqref{f:Cf_to_Cf'+}. 
    Now we apply Lemma \ref{l:Sanders_V} with $f=\bar{S}$, $B=B$, $\epsilon   = \eta$, and $l=l-1$.
    By this result and inequality \eqref{tmp:01.03_2_Ekl} we find a subspace $V\subseteq \Gr$ such that 
    the co--dimension of $V$ is controlled 
    %thanks to  
    by estimate 
    \eqref{f:Sanders_V_codim} and 
\begin{equation}\label{tmp:01.03_3_Ekl} 
    \sum_{|x|=l-1} \bar{S} (x) (B^{l-1} \circ \mD_{l-1} (B * \mu_V)) (x)  \ge (1-3\eta) |B|^l \,.
\end{equation} 
    By the definition of the set $S$ (and hence $\bar{S}$), we have $\Cf_l (A) (x) \ge (1+\eps/4)^l \d^l N$
    for any $x \in \bar{S}$ and hence inequality \eqref{tmp:01.03_3_Ekl}  give us 
\[
    (1-3\eta) (1+\eps/4)^l \d^l |B|^l N \le \sum_{|x|=l-1} 
    (A^{l-1} \circ \mD_{l-1} (A)) (x)
    (B^{l-1} \circ \mD_{l-1} (B * \mu_V)) (x)
\]
\[
    =
%    \sum_{|x|=s-1}\, \sum_{|z|=k-1} \mD_{l-1} (\mu_V \circ A) (\bar{x}\oplus \bar{z}) (B^{l-1} \circ A^{l-1} \circ \mD_{l-1} (B)) (\bar{x}\oplus \bar{z}) \,.
    \sum_{\a} (A*B*\mu_V) (\a)  
    \sum_{|x|=l-1} A^{l-1} (x) B^{l-1} (x + \mD_{l-1} (\a)) 
    = \sum_{\a} (A*B*\mu_V) (\a) (A\circ B)^{l-1} (\a) 
\]
\begin{equation}\label{tmp:01.03_4_Ekl} 
\le 
    |B| \| A * \mu_V \|_\infty  \sum_{\a}  (A\circ B)^{l-1} (\a) 
\,.
\end{equation} 
    Now by our assumption $A$ is $\eps/5$--uniform relatively to $\E^{k_*}_{l-1}$ and a certain  $k_* = O(kl \L (\delta))$.
    Using Corollary \ref{c:A_circ_B}, we derive
%    It follows that 
\[
    \| A * \mu_V \|_\infty \ge \d (1-3\eta) (1+\eps/4) \ge \d (1+\eps/8) \,.
\]
%    provided $l\ge 5/\eps$. 
%as required. 
    Finally, thanks to \eqref{f:Sanders_V_codim}, we get
\begin{equation}\label{tmp:codim}
    \codim V \ll \eps^{-2} l \L^2 (\beta) \L^2 (\eps \beta^l) 
    \ll 
     \eps^{-2} l^3 k^4 \L^2 (\d) \L^2 (\eps \d) \,.
\end{equation} 
This completes the proof. 
$\hfill\Box$
\end{proof}


\bp 

Now we  
%ready to prove 
can prove our new Theorem \ref{t:uniform_KM_k}. 


As always 
the proof follows the density increment scheme and our aim is to construct a shift of a subspace $V_\eps (l,k)$ where the set $A$ is $\eps$--uniform relatively to $\E_l^k$. 
Also, to obtain Theorem \ref{t:uniform_KM_k} we use induction on parameter $l\ge 2$ and the first step of the induction for $l=2$ and an arbitrary $k$ one can use either  
Kelley--Meka  Theorem \ref{t:uniform_KM}
%Alternatively, we prefer follow 
or the arguments of 
%of the previous section, 
Section \ref{sec:E^k_l} (one can check that we do not need any uniformity conditions in this case),  
combining with  Proposition \ref{p:increment_Ekl}. 
%They give us the following bound for the co--dimension of the subspace where our set $A$ is uniform
%\begin{equation}\label{f:basis}
%    \codim V \ll 
%\end{equation} 
Now let $l\ge 3$ and suppose that the set $A$ is not $\eps$--uniform relatively to $\E^k_{l}$ for a certain $k$ because otherwise there is nothing to prove. 
Of course (see, e.g., inequality \eqref{f:Holder_Ekl}), one can take $k$ to be a sufficiently large number and we choose $k$ such that $k \gg l\eps^{-l} \L (\eps)$.  
Put $\eps_l=\frac{\eps^l}{8l}$.
We can freely assume that our set $A$ is $\eps_l/5$--uniform relatively to $\E^{k_*}_{l-1}$ with 
%$k_* =O(k l \eps^{-l} \L(\eps))$ 
$k_* = k_* (l) =O(k l^2 \eps^{-l} \L(\eps) \L(\d))$ in a shift of a subspace $V_{\eps_l/5} (l-1,k_*(l))$ 
thanks to Theorem \ref{t:uniform_KM} in the case $l=3$ or by the induction assumption for larger $l$. 
%It coasts us just co--dimension \eqref{f:uniform_KM}, namely,
%\begin{equation}\label{f:induction_l-1}
%    \codim V_{l-1} \ll  \eps^{-3}_l l^3  \L^3 (\d) \L^2 (\eps_l \d) \cdot (k_* l \eps^{-l} \L(\eps))^4 \,.
%\end{equation}
Now we apply Lemma \ref{l:e_to_d} and find $k_1 = O(kl \eps^{-l} \L(\eps))$ such that  
\[
    \E_l^{k_1} (A) \ge \left(1+ \eps_l \right)^{lk_1} \d^{lk_1} N^{l+k_1} \,.
\]
After that we use Proposition \ref{p:increment_Ekl} with 
$\eps = \eps_l$.
One can check that  $kl \gg \eps_l^{-1} \L(\eps_l)$ and that $A$ is sufficiently uniform set (to apply our proposition) relatively to $\E^{k_*}_{l-1}$ thanks to our choice of $k_*$.  
%Clearly, 
Estimate \eqref{cond:increment_2_Ekl}  implies that 
the procedure must stop after $O(\eps^{-1}_l \L(\d))$ number of steps and thus the final co--dimension is  
\[
   \codim V_\eps (l,k) \ll 
   \eps^{-1}_l \L(\d) \codim V_{\eps_l/5} (l-1,k_* (l)) \ll 
   l \eps^{-l} \L(\d) \codim V_{\eps_l/5} (l-1, k l^2 \eps^{-l} \L(\eps) \L(\d)) \,.
\]   
    Put $L=l!$. 
    Solving the functional inequality above and using \eqref{f:uniform_KM}, we get 
\begin{equation}\label{tmp:09.03_1}
    \codim V_\eps (l,k) \ll L \eps^{-l(l-1)/2} \L^{l-2} (\d) \codim V_{(\eps/8l)^{2L}} (2, k(l!)^2  \eps^{-l(l-1)/2} (\L(\eps) \L(\d))^{l-2})
\end{equation} 
\[
    \ll \eps^{-28l^l} (8l)^{28l^l} k^4 \L^{4l} (\eps) \L^{5l} (\d)
\]
as required. 
Actually, one can see that the number of steps of our algorithm is at most $O(\eps^{-1}_l \L(\d))$ (due to every time we increase the density $\d$ to $\d (1+\eps_l/8)$ and hence we do not need the first multiple in \eqref{tmp:09.03_1}). Nevertheless, it gives us a bound of the same sort.   
This concludes the proof of the theorem. 
%Theorem \ref{t:uniform_KM_k}. 
$\hfill\Box$


\bp


Theorem \ref{t:uniform_KM_k} allows us to find a shift of subspace where our set $A$ is uniform relatively to $\E^k_l$. 
A modification of this argument allows to prove more
and this is important for applications. 
Actually, for any relatively dense $A\subseteq \F_p^n$, $|A| = \d N$ there is a partition of $\F_p^n$ onto shifts of subspaces $V_j$ such that 
\begin{equation}\label{f:A_L2}
    A = \left(\bigsqcup_j A\cap (V_j+ x_j) \right) \bigsqcup \Omega 
    %=  \bisqcup_j A_j
\end{equation} 
such that $|\Omega|= \omega N$, each set $(A-x_j)$ is $\eps$--uniform in $V_j$ relatively to $\E^k_l$ and $\codim V_j = O_{\eps,\omega, \d} (1)$. 
Such results are not new see, e.g.,  \cite{GT_primes} and \cite{Green_models}, \cite{sh_Sz_LMS},  \cite{sh_Sz_survey}.
The idea is to replace $L_\infty$--increment to $L_q$--increment with a controllable parameter $q$.  
%*%Let us 
In this section we obtain the driving proposition, which we will use in the future paper. 
%which we will use in the next section. 


\begin{proposition}
    Let $\Gr = \F_p^n$, $A\subseteq \Gr$, $|A|=\d N$, $\eps \in (0,1/2)$ be a real number and $k,l \ge 2$ be positive integers, and $k \gg l\eps^{-l} \L(\eps)$.
    %, $l\le \eps^{-1}$. 
    Suppose that 
\begin{equation}\label{cond:L2_ increment_Ekl}
    \E^k_{l} (f_A) = \eps^{kl} \d^{kl} N^{k+l} \,,
\end{equation} 
    and that $A$ is $2^{-11l} l^{-l} \eps^{l^2}$--uniform with respect to $\E^{k_*}_{l-1}$ for  $k_* = O(kl^2 \eps^{-l} \L(\eps) \L (\delta))$.
    Then there is a subspace $V \subseteq \Gr$ such that 
\begin{equation}\label{cond:L2_increment_1_Ekl}
    \mathrm{codim} V \ll \eps^{-2l} l^5 k^4 \L^2 (\d) \L^2 (\eps \d)
    \,,
\end{equation} 
%    and  
%    For $l=2$ one has 
\begin{equation}\label{cond:L2_increment_2_Ekl+} 
    N^{-1} \sum_x (f_A * \mu_V)^l (x) \ge  
%    2^{-7l} \eps^l l^{-l} \d^l 
    2 \left(\frac{\eps \d}{2^7} \right)^l
%    2^{-2} \eps     \d^{2} \,.
    \,.
\end{equation} 
    In particular, 
\begin{equation}\label{cond:L2_increment_new} 
    N^{-1} \sum_x (A * \mu_V)^l (x) \ge  
%    2^{-7l} \eps^l l^{-l} \d^l 
    \d^l \left(1+\frac{\eps^l}{2^{7l}} \right)
%    2^{-2} \eps     \d^{2} \,.
    \,.
\end{equation}
    Further suppose that $A$ is $2^{-11} \eps^l/l^2$--uniform with respect to $\E^{k_*}_{l-1}$ for  $k_* = O(kl^2 \eps^{-l} \L(\eps) \L (\delta))$. 
    Then for a certain even $q$, $q \ll l^3 \L(\eps \delta)$ and $l>2$ the following holds 
\begin{equation}\label{cond:L2_increment_2_Ekl} 
    N^{-1} \sum_x (f_A * \mu_V)^q (x) \ge  
    %2^{-2q} \eps 
    (4\d)^{q} \,.
\end{equation} 
    In particular,
\begin{equation}\label{cond:L2_increment_2_Ekl_A} 
    N^{-1} \sum_x (A * \mu_V)^q (x) \ge  (2\d)^{q} \,.
\end{equation} 
%    Finally, let $A_1 \subseteq A$, $|A_1|=\d_1 N$ be any non $\eps_1$--uniform set relatively $\E_l^k$. 
%    Then 
%\begin{equation}\label{cond:L2_increment_2_Ekl_A_*} 
%    N^{-1} \sum_x (A_* * \mu_V)^q (x) \ge  (2\d)^{q} \,.
%\end{equation
\label{p:L2_increment_Ekl}
\end{proposition}
\begin{proof} 
    Using Lemma \ref{l:e_to_d} we find
%    $k_1 \le k_* \ll kl \eps^{-l} \L(\eps)$
    $k_1 \ll kl \eps^{-l} \L(\eps)$ such that 
\[
    \E^{k_1}_{l} (A) \ge 
    \left(1+\frac{\eps \eps^{l-1}_*}{8l} \right)^{k_1 l} \d^{k_1 l} N^{k_1+l}
    =
    (1+\eps_l)^{k_1 l} \d^{k_1 l} N^{k_1+l} \,.
\]
    Let $\zeta = 2^{-8l} \eps^l_l$, $\zeta_1 = 2^{-8} \eps_l/l$ and  for any positive integer $n$ we put 
    $S (n) = N^{-1} \sum_\a (f_A * \mu_V)^n (\a)$. 
    %and $q=Cl \L(\eps \d)$ for a sufficiently large constant $C>1$, $q$ is an even number.  
%    Also, put $S(q) = N^{-1} \sum_\a (f_A * \mu_V)^q (\a)$. 
    %and choose $\eta= \eps/100$.
We use the same argument as in Proposition \ref{p:increment_Ekl} (the parameter $\eta$ equals  $\eps_l/30$)
before inequality \eqref{tmp:01.03_4_Ekl}.
One has 
\begin{equation}\label{tmp:03.09_2}
    (1-3\eta) (1+\eps_l/4)^l \d^l |B|^l N = \d |B| \sum_\a (A\circ B)^{l-1} (\a) + \sum_\a (f_A * \mu_V) (\a) ((A\circ B)^{l-1} \circ B) (\a) = \sigma_1 +\sigma_2 \,.
\end{equation} 
    Here we have the required upper bound for the dimension of $V$ similar to \eqref{tmp:codim} 
\begin{equation*}
    \codim V \ll \eps^{-2}_l l \L^2 (\beta) \L^2 (\eps_l \beta^l) 
    \ll 
     \eps^{-2l} l^5 k^4 \L^2 (\d) \L^2 (\eps \d) \,.
\end{equation*}
    We need to consider the cases $l=2$ and $l>2$ separately. 
    In the case $l=2$ we follow the method from \cite{kelley2023strong}.
    %and show that one can choose $q=2$ in \eqref{cond:L2_increment_2_Ekl}. 
    Clearly, if $l=2$, then $\sigma_1=\d |B|^2 |A|$ and hence
\begin{equation}\label{tmp:11.03_1}
    \frac{7\eps_2}{18} \d^2 |B|^2 N \le \sum_\a ((f_A \circ  f_A) * \mu_V) (\a) (B \circ B) (\a) \,.
\end{equation} 
    Using  the Fourier transform twice and the fact that $\FF{\mu}_V (r) = V^{\perp} (r)$, we get
\[
     \frac{7\eps_2}{18} \d^2 |B|^2 N 
     \le 
     N^{-1} \sum_{r} \FF{\mu}_V (r) |\FF{f}_A (r)|^2 
     |\FF{B} (r)|^2
     \le 
     N^{-1} |B|^2 \sum_{r} \FF{\mu}_V (r) |\FF{f}_A (r)|^2 =
     |B|^2 \sum_{\a} (f_A\circ f_A) (\a) \mu_V (\a) \,.
\]
    It remains to notice that $\mu_V \circ \mu_V = \mu_V$. Let us give another proof which does not use the Fourier approach. Using the identity $\mu_V \circ \mu_V = \mu_V$ we estimate the right--hand side of 
    \eqref{tmp:11.03_1} as 
\[
    \| ((f_A \circ  f_A) * (\mu_V \circ \mu_V)) (\a) \|_\infty \cdot 
    \sum_\a  (B \circ B) (\a) 
    = |B|^2  \sum_x (f_A * \mu_V)^2(x)  
\]
as required. 


    Now consider the general case. 
    To calculate the sum $\sigma_1$ we apply Corollary \ref{c:A_circ_B} and our assumption on $l$ to derive 
\begin{equation}\label{tmp:03.09_3}
    (\d |B|)^{-1} \sigma_1 \le \d^{l-1} |B|^{l-1} N (1+1.25 \zeta)^{l-1} 
    \le \d^{l-1} |B|^{l-1} N ( 1 + 2\zeta l) 
\,.
\end{equation} 
    In view of our choice of $\eta$ and $\zeta \le \zeta_1 = 2^{-8} \eps_l/l$ it gives us
\begin{equation}\label{tmp:03.09_3'}
%    2^{-3} \eps \d^l |B|^l N 
%        \le 
    2^{-3} \eps_l l \d^l |B|^l N
    \le 
    ((1-3\eta)(1+\eps_l l/4)- 1-2\zeta l)
    \d^l |B|^l N
    \le 
    \sigma_2
   \,.
\end{equation}  
    Applying the H\"older inequality several times for even $l$, we derive
\[
    \sigma_2 = \sum_\a (f_A * \mu_V) (\a) ((A\circ B)^{l-1} \circ B) (\a)
\]
\begin{equation}\label{tmp:S(l)_calculation-}
    \le 
     \left(\sum_\a (f_A * \mu_V)^l (\a) \right)^{1/l}
     \left( \sum_\a ((A\circ B)^{l-1} \circ B)^{l/(l-1)} (\a) \right)^{1-1/l}
\end{equation} 
\begin{equation}\label{tmp:S(l)_calculation}
     \le
     |B| S^{1/l} (l) N^{1/l} 
     \left( \sum_\a (A\circ B)^{l} (\a) \right)^{1-1/l}  N^{1-1/l} \,.
\end{equation} 
    Now using Corollary \ref{c:A_circ_B} and recalling estimate \eqref{tmp:03.09_3'}, we see that 
\begin{equation}\label{tmp:19.03_even}
    2^{-4l+2} l^l \eps^l_l \d^l \le S(l) = N^{-1} \sum_\a (f_A * \mu_V)^l (\a)
    %\,.
\end{equation} 
    as required. 
    %For odd $l$ the argument is similar, we do not replace the function $A$ to $f_A$ in $A* \mu_V$  as we have done in \eqref{tmp:03.09_2}, after that we use the  H\"older inequality and finally we replace $A$ to $f_A$ in $\sum_\a (A * \mu_V)^l (\a)$, using the binomial theorem, see Remark {\bf ??????}
    For odd $l$ the argument is the similar, one has 
\[
    \sum_\a |(f_A * \mu_V)^{l} (\a)| 
    \le 
    \sum_\a (f_A * \mu_V)^{l-1} (\a) ( (f_A * \mu_V) (\a) + 2\d)
    =
    N S(l) + 2\d \sum_\a (f_A * \mu_V)^{l-1} (\a)
\]
\[
    \le 
    N S(l) + 2 \zeta^{l-1} \d^{l} N
    \le  N S(l) + 2^{-4l} l^l \eps^l_l \d^l N
\]
    and hence we have obtain a lower bound for $S(l)$ similar to \eqref{tmp:19.03_even}. 
    Thus we have 
    %obtained 
    proved 
    \eqref{cond:L2_increment_2_Ekl+}. 
    To derive  \eqref{cond:L2_increment_new}, we write 
\[
    N^{-1} \sum_{\a} (A* \mu_V)^l (\a) = S(l) + \d^l +
    \sum_{0<j<l} \binom{l}{j} \d^{l-j} \sum_{\a} (f_A* \mu_V)^j (\a) 
\]
    and apply Lemma \ref{l:e_to_d}
\[
    N^{-1} \sum_{\a} (A* \mu_V)^l (\a) \ge S(l) + \d^l - 2l \d^l \zeta \ge \d^l (1+2^{-4l} l^l \eps^l_l) 
    %\,.
\]
    as required. 

\bigskip 

Now let us obtain \eqref{cond:L2_increment_2_Ekl} and  \eqref{cond:L2_increment_2_Ekl_A}.  
    Returning to \eqref{tmp:03.09_2} and \eqref{tmp:03.09_3'}, we have 
%        One has 
\[
    \sigma_2 =\sum_{j=1}^{l-1} \binom{l-1}{j} (\d |B|)^{l-1-j} \sum_\a ((f_A \circ B)^j \circ B) (\a) (f_A * \mu_V)  (\a)
\]
\begin{equation}\label{tmp:11.03_2}
    = \sum_{j=1}^{l-1} \binom{l-1}{j} (\d |B|)^{l-1-j} \sigma(j) \,.
\end{equation}  
    For even $j$ we can apply formula \eqref{tmp:S(l)_calculation-} with $l=q$  
    %\eqref{tmp:03.09_3} 
    %$\mu_V \circ \mu_V$ 
    again, then the estimate $\| ((f_A \circ B)^j \circ B) (\a) \|_\infty \le |B|^{j+1}$, 
    as well as 
    %Corollary \ref{c:A_circ_B} 
    Lemma \ref{l:uniformity} 
    with the  parameter $j<l$ and derive 
    %, we have 
\[
    \sigma^q (j) \le |B|^{q+j} \left( \sum_\a (f_A \circ B)^{j} (\a) \right)^{q-1}
    \cdot 
    %\left( 
    \sum_\a (f_A * \mu_V)^q  (\a) %\right)^{1/q} 
\]
\begin{equation}\label{tmp:12.03_1}
    \le \zeta^{j(q-1)}_1 \d^{j(q-1)} |B|^{(j+1)q} N^{q}  \cdot S(q)
%    \sum_\a (f_A * \mu_V)^q  (\a) 
%    \,.
\end{equation}  
For odd $j$, we write 
\[
    \sigma (j) = \sum_\a ((f_A \circ B)^{j-1} \cdot (A\circ B) \circ B) (\a) (f_A * \mu_V)  (\a) 
    - \d |B| \sigma(j-1) 
    = \sigma_* (j) - \d |B| \sigma(j-1) 
\]
    and thus we need to estimate $\sigma_* (j)$   because in  $\sigma(j-1)$  the argument $j-1$ is even.
    %%As for  
    To bound the sum 
    $\sigma_* (j)$  
    %to estimate
    we use the same argument as in \eqref{tmp:12.03_1} (and the identity  $(A\circ B) (\a) = (f_A\circ B) (\a) + \d |B|$, of course) to show that 
\[
    \sigma^q_* (j) \le |B|^{q+j} \left( \sum_\a (f_A \circ B)^{j-1} (\a) ( A\circ B) (\a) \right)^{q-1}
    \cdot 
    %\left( 
    \sum_\a (f_A * \mu_V)^q  (\a) %\right)^{1/q} 
\]
\begin{equation}\label{tmp:12.03_1'}
    \le 2^{q-1} \zeta^{j(q-1)}_1 \d^{j(q-1)} |B|^{(j+1)q} N^{q}  \cdot S(q) \,.
%    \sum_\a (f_A * \mu_V)^q  (\a) 
%    \,.
\end{equation}  
    Thus in view of \eqref{tmp:03.09_3'}, \eqref{tmp:12.03_1}, \eqref{tmp:12.03_1'} as well as our choice of $q$, we get 
\[
    2^{-5} \eps_l l \d \le l((1+\zeta_1)^{l-1} - 1) S(q)^{1/q} 
    \le 2\zeta_1 l^2 S(q)^{1/q} 
    %N^{-1} \sum_\a (f_A * \mu_V)^q (\a)
\]
    and hence 
\begin{equation}\label{tmp:4delta} 
   %2^{-2q} \eps \d^{q} N  \le 
   %2^{-6q} (\eps/\zeta) 
    (4\d)^{q} \le  N^{-1} \sum_\a (f_A * \mu_V)^q (\a)
\end{equation} 
as required. 
To get \eqref{cond:L2_increment_2_Ekl_A} just use the identity  $(f_A * \mu_V) (\a) = (A * \mu_V) (\a) - \d$ and apply the binomial formula to \eqref{tmp:4delta}.
%
%
%    It remains to obtain \eqref{cond:L2_increment_2_Ekl_A_*}. 
This completes the proof. 
$\hfill\Box$
\end{proof}


\begin{remark}
    At the fist view it looks strange that estimates \eqref{cond:L2_increment_2_Ekl}, \eqref{cond:L2_increment_2_Ekl_A} do not depend on $\eps$ but we just pay for this taking larger codimension   in \eqref{cond:L2_increment_1_Ekl}. 
\label{r:codim_absorbe}
\end{remark}



\begin{comment} 

%\section{Applications}
\section{An application}
\label{sec:corners}



    Let $\Gr$ be an abelian group. Any triple of the form 
\[
    \{ (x,y), (x+d,y), (x,y+d) \} \in (\Gr \times \Gr)^3 
\]
    is called a {\it corner}. If $d\neq 0$, then we say that our corner is a {\it  non--trivial} one. 
    %Let $\Gr=\F_2^n$ and 
    Further let 
    $A\subseteq S_1 \times S_2 \subseteq \Gr \times \Gr$, $|A|=\d |S_1||S_2|$, $|S_1|=\sigma_1 N$, $|S_2|=\sigma_2 N$. 
    We say that $A$ is {\it $\eps$--uniform with respect to the rectangular norm} if 
\[
    \|f_A\|^4_{\Box} := \sum_{x,y} \left| \sum_{y} f_A (x,y) f_A (x',y) \right|^2 \le \eps^4 \d^4 |S_1|^2 |S_2|^2 \,.
\]
    Let us obtain a counting result on the number of corners in uniform sets, see \cite{Green_models}, \cite{sh_Sz_survey}. 

\begin{theorem}
    Let $\Gr=\F_2^n$ and $A\subseteq S_1 \times S_2 \subseteq \Gr \times \Gr$, $|A|=\d |S_1||S_2|$, $|S_1|=\sigma_1 N$, $|S_2|=\sigma_2 N$. 
    Suppose that $A$ is $\eta$--uniform with respect to the rectangular norm, $\eta \le \d^{3/2}/4$, 
    and $S_1,S_2$ are $\eps$--uniform relatively to $\E^k_4$ with $\eps = 2^{-12} \eta^4 \d^2$ and $k=O(\L (\d \sigma_1 \sigma_2))$. 
    Then $A$ contains at least $2^{-1} \d^3 \sigma^2_1 \sigma^2_2 N^3$ corners. 
    In particular, if $N>2\d^{-2} \sigma^{-1}_1 \sigma^{-1}_2$, then $A$ has a non--trivial corner. 
\label{t:counting_corners}
\end{theorem} 
\begin{proof} 
Let $f_1, f_2, f_3$ be three arbitrary functions on $\Gr \times \Gr :=\mathscr{P}$. Consider the functional 
\[ T\left(f_1, f_2, f_3\right)=\sum_{x, y, z} f_1(x, y) f_2(y+z, y) f_3(x, x+z) \,.
\]
It is clear that $T$ is linear in each of the arguments. Moreover, the value $T(A, A, A)$ is equal to the number of triples $\{(x, y),(x+d, y),(x, y+d)\}$ in $A$. 
We have 
$T(A, A, A)=\delta T(\mathscr{P}, A, A)+T(f_A, A, A)$ and $\|f_A\|^4_{\Box} \leqslant \eta^4 \d^4 \sigma_1^2 \sigma_2^2 N^4$.


Let $g(z)=\sum_x A(x, x+z)$. Then $T(\mathscr{P}, A, A)=\sum_z g(z)^2$ (here we use that our group is $\F_2^n$). We have $\sum_z g(z)=\delta \sigma_1 \sigma_2 N^2$ and thus by the H\"older inequality,
\begin{equation}\label{tmp:03.13_1}
T(A, A, A) \geqslant \delta^3 \sigma_1^2 \sigma_2^2 N^3+T(f, A, A) \,.
\end{equation} 
Let us estimate the second term on the right-hand side of \eqref{tmp:03.13_1}. Using once again the H\"older inequality, we see that
$$
\begin{aligned}\label{tmp:03.13_2}
T(f, A, A)= & \sum_{y, z} A(y+z, y) \sum_x S_1(y+z) f(x, y) A(x, x+z)\\
& \le \left( \sum_{y,z}  A(y+z, y) \right)^{1/2}\\
& \times \left( \sum_{x,x',y,z} S_1(y+z) A(x,x+z) A(x',x'+z) f_A(x,y) f_A (x',y) \right)^{1/2} \,.
\end{aligned} 
$$
We have $\sum_{y, z}  A(y+z, y) = \d \sigma_1 \sigma_2 N^2$. Further,
$$
\begin{aligned}
\sigma & :=\sum_{x, x^{\prime}, y, z} S_1(y+z) A(x, x+z) A\left(x^{\prime}, x^{\prime}+z\right) f(x, y) f\left(x^{\prime}, y\right) \\
& =\sum_{x, x^{\prime}, z} A(x, x+z) A\left(x^{\prime}, x^{\prime}+z\right) \sum_y S_1(y+z) S_2(x+z) S_2\left(x^{\prime}+z\right) f(x, y) f\left(x^{\prime}, y\right)
\end{aligned}
$$
Let 
\[
    \omega\left(x, x^{\prime}, y, y^{\prime}\right)=\sum_z S_1(y+z) S_1\left(y^{\prime}+z\right) S_2(x+z) S_2\left(x^{\prime}+z\right) = \Cf_4 (S_2,S_2,S_1,S_1) (x,x',y,y') \,. 
\]
A third use of the H\"older inequality gives us
$$
\begin{aligned}
\sigma \leqslant & \left( \sum_{x, x^{\prime}, z} S_1(x) S_1\left(x^{\prime}\right) S_2(x+z) S_2\left(x^{\prime}+z\right)\right)^{1 / 2} \\
& \times\left(\sum_{x, x^{\prime}, y, y^{\prime}} \omega\left(x, x^{\prime}, y, y^{\prime}\right) f(x, y) f\left(x^{\prime}, y\right) f\left(x, y^{\prime}\right) f\left(x^{\prime}, y^{\prime}\right)\right)^{1 / 2} .
\end{aligned}
$$
Applying Corollary \ref{c:A_circ_B} with $l=2$, $A=S_1$, $B=S_2$, we obtain 
$$
\sum_{x, x^{\prime}, z} S_1(x) S_1\left(x^{\prime}\right) S_2(x+z) 
S_2\left(x^{\prime}+z\right) 
= \sum_z (S_1 \circ S_2)^2 (z) \leqslant 2 \sigma_1^2 \sigma_2^2 N^3 
%\,.
$$
    due to $k \gg \L (\sigma_2)$ and $\eps \le 1/4$. 
Further using inequalities 
\[
    T(f,A,A)^4 \le 2 \d^2 \sigma^4_1 \sigma^4_2 N^7 \sum_{x,x',y,y'} \omega (x,x',y,y') f(x,y) f(x',y) f(x,y') f(x',y') 
\]
\[
    =
    2 \d^2 \sigma^6_1 \sigma^6_2 N^8 
   \| f\|^4_{\Box}
    +
    2 \d^2 \sigma^4_1 \sigma^4_2 N^7 \sum_{x,x',y,y'} (\omega (x,x',y,y') - \sigma^2_1 \sigma^2_2 N) f(x,y) f(x',y) f(x,y') f(x',y') 
\]
\begin{equation}\label{tmp:03.13}
    \le 
    2 \eta^4 \delta^6 \sigma^8_1 \sigma^8_2 N^{12}
     + 2 \d^2 \sigma^4_1 \sigma^4_2 N^7 S_* 
    \,.
\end{equation} 
    Here we have used the assumption that the set $A$ is $\eta$--uniform with respect to the rectangular norm. 
    To estimate the sum $S_*$ we apply Lemma \ref{l:dispersion} with $l=4$ and the H\"older inequality and obtain 
\[
    S_* \le 2^{12} \eps  (\sigma^{2k}_1 \sigma^{2k}_2 N^{4+k})^{1/k} (4\d^{2} \sigma^2_1 \sigma^2_2 N^4 )^{1-1/k}
    \le 2^{15} \eps \d^2  \sigma^4_1 \sigma^4_2 N^5 
\]
    due to $k=C \L (\d \sigma_1 \sigma_2)$, where $C>1$ is a sufficiently large absolute constant. 
    Returning to \eqref{tmp:03.13} and recalling that $\eps = 2^{-12} \eta^4 \d^2$, we get 
\[
    T(f,A,A)^4 \le 2 \eta^4 \delta^6 \sigma^8_1 \sigma^8_2 N^{12} + 2^{15} \eps \d^4 \sigma^8_1 \sigma^8_2 N^{12}
    \le 16 \eta^4 \delta^6 \sigma^8_1 \sigma^8_2 N^{12} \,.
\]
    The last bound and \eqref{tmp:03.13_1} give us 
$$
T(A, A, A) \geqslant
\sigma^2_1 \sigma^2_2 N^3 (\d^3 - 2\eta \d^{3/2}) \ge 2^{-1} \d^3 \sigma^2_1 \sigma^2_2 N^3 
$$
as required. 
%This completes the proof. 
$\hfill\Box$
\end{proof}


\bp 


The following purely graph--theoretical result was obtained in \cite{Green_models}, \cite[Proposition 8]{sh_Sz_survey}. 

\begin{lemma}
    Let $A\subseteq S_1 \times S_2$, $\mu_{S_1\times S_2} (A) = \d$ and $A$ is not $\eta$--uniform with respect to the rectangular norm. 
    Then there are $S'_j \subseteq S_j$, $|S'_j| \ge 2^{-8}(\eta \d)^4 |S_j|$, $j\in [2]$  such that $\mu_{S'_1\times S'_2} (A) \ge \d + 2^{-14} (\eta \d)^8$. 
\label{l:E_inc}
\end{lemma}


\begin{corollary}
     Let $\Gr=\F_2^n$ and $A\subseteq S_1 \times S_2 \subseteq \Gr \times \Gr$, $|A|=\d |S_1||S_2|$, $|S_1|=\sigma_1 N$, $|S_2|=\sigma_2 N$. 
     Suppose that $S_1,S_2$ are $\eps$--uniform relatively to $\E_4^k$ with $\eps = 2^{-20} \d^{8}$, $k=O(\L (\d \sigma_1 \sigma_2))$ and $A$ has no non--trivial corners.
    Assume that 
\begin{equation}\label{f:1+2}
    N>2\d^{-2} \sigma^{-1}_1 \sigma^{-1}_2 \,. 
\end{equation}
    Then there are $S'_j \subseteq S_j$, $|S'_j| \ge 2^{-16} \d^{10} |S_j|$, $j\in [2]$  such that
\begin{equation}\label{f:1+2_inc}
    \mu_{S'_1\times S'_2} (A) \ge \d + 2^{-30} \d^{20} \,.
\end{equation} 
\label{c:1+2}
\end{corollary}


Thus Theorem \ref{t:counting_corners} combining with Lemma \ref{l:E_inc} say that for any set $A$ without corners  we have a certain density increment in the Cartesian product $S'_1\times S'_2$. Of course, that one can try to iterate this argument but to do this we need to regularize our sets $S'_1, S'_2$.
%possesses  
To this end 
%Now 
one can use Proposition \ref{p:L2_increment_Ekl} to obtain decomposition \eqref{f:A_L2} of each $S'_1$, $S'_2$ and hence to reach a density increment of $A$  in some $\E^k_l$--uniform subsets of $S'_1$, $S'_2$, see  \cite{Green_models} or \cite[Proposition 10]{sh_Sz_survey}.
%We formulate it 
We combine this type of arguments  
in the following result. 


\begin{proposition}
     Let $\Gr = \F_2^n$, $A\subseteq S_1 \times S_2 \subseteq \Gr \times \Gr$, 
     $|S_j| = \sigma_j N$, $j\in [2]$, $\d, \tau \in (0,1]$, $\mu_{S_1\times S_2} (A) = \d + \tau \d$ and $k,l\ge 2$ be some integers $k \gg l \eps^{-l} \L(\eps)$.
     Suppose that 
\begin{equation}\label{cond:L2_corners}
    n \gg 2^{200l^{l+1}} l^{(10l)^{l+1}} \eps^{-20l^{l+1}} 
    \left(\tau^{-2} \L^{} (\tau) \L^{} (\eps)  \L(\d) \L^{}(\sigma) \right)^{4l+4} k^4
\end{equation}
    Then there is $W\subseteq \Gr$ with 
\begin{equation}\label{f:L2_corners_1}
    \codim (W) \ll 2^{200l^{l+1}} l^{(10l)^{l+1}} \eps^{-20l^{l+1}} 
    \left(\tau^{-2} \L^{} (\tau) \L^{} (\eps)  \L(\d) \L^{}(\sigma) \right)^{4l+4} k^4
    \,,
\end{equation}
    and $t_1, t_2 \in \Gr$ such that for $\tilde{S}_j = (S_j-t_j) \cap W$, $j\in [2]$ the following holds\\
$\bullet~$ the sets $\tilde{S}_1$, $\tilde{S}_2$ are $\eps$--uniform in $W$ relatively to $\E^k_l$; \\
$\bullet~$ $\mu_{\tilde{S}_1 \times \tilde{S}_2} (A-(t_1,t_2)) \ge \d + \tau \d/4$; \\ 
$\bullet~$ 
%$\mu_W (\tilde{S}_j) \ge 2^{-2} \tau \sigma_j$, $j\in [2]$.
$\mu_{W\times W} (\tilde{S}_1 \times \tilde{S}_2) \ge \sigma_1 \sigma_2 \cdot \exp( - O( \tau^{-2} \L(\tau) \L(\d \tau \sigma_1 \sigma_2)))$.
\label{p:L2_corners}
\end{proposition} 
\begin{proof} 
    Put $\sigma = \sigma_1 \sigma_2$, $S=S_1\times S_2$ and let $M>1$ be a sufficiently large absolute constant. 
    Also, let 
    %$l=4$ and   
    $q=M l^2 \L(\eps \sigma)$. 
    If either $S_1$ or $S_2$ is not $\eps$--uniform in $\Gr$ relatively to $\E^k_l$ but $2^{-11} l^{-2} \eps^l$--uniform relatively to $\E^{k_*}_{l-1}$, $k_l = Mkl^2 \eps^{-l} \L(\eps) \L(\sigma)$, %$j\in [2]$, 
    then we use Proposition \ref{p:L2_increment_Ekl} to split $\Gr\times \Gr$ onto cells $C=(V+t_1)\times (V+t_2)$
    %$= V_1 \times V_2$ 
    and inspect each such a cell to iterate our procedure. 
    Considering the function $f_j (\eps) = 2^{-11} j^{-2} \eps^j$, we write $\eps_{l-1} = f_l(\eps_l)$, where $\eps_l=\eps$. 
    We say the a cell $C$ is {\it thin} if 
    %$\mu_{C} (S) < 2^{-2} \tau \sigma$. 
    $\mu_{C} (S) < 2^{-2} \tau \d \mu_{\Gr \times \Gr} (S)$. 
    Here $\Gr \times \Gr$ is our initial cell and below the set we define thin cells according to the ambient cells which were obtained at the previous steps of our procedure. 
    Further a non--thin cell $C$ is {\it uniform} if both sets $S_1,S_2$ are  $\eps_i$--uniform with respect to $\E^{k_{i}}_{i}$, $\eps_{i} = f_{i+1} (\eps_{i+1})$,  $k_{i} = M l^2 k_{i+1} \eps^{-i}_{i+1} \L(\eps_{i+1}) \L(\sigma)$, 
    $i = l, l-1, \dots, 2$
    %, $j\in [2]$ 
    in the sides $V + t_1,V + t_2$ of $C$ and {\it non--uniform} otherwise. 
%    Here $\eps_* = 2^{-10} \eps$, say. 
    Let $\mathcal{T}_s$, $\mathcal{U}_s$, $\mathcal{N}_s$ is the collections of thin, uniform and non--uniform cells at $s$th step of our algorithm, correspondingly.  
    Put $\mathcal{F}_s = \mathcal{T}_s \bigsqcup \mathcal{U}_s \bigsqcup \mathcal{N}_s$. 
    Of course, one can assume that for all $C \in \mathcal{U}_s$ the followings holds 
\begin{equation}\label{f:U_density} 
    \mu_{C\cap S} (A) \le \d + \tau \d /4 = \mu_S (A) + \tau \mu_S (A) /4 
    %\,.
\end{equation} 
    due to otherwise there is nothing to prove. 
    Further let us make rather  general calculation and describe an auxiliary sub--procedure. 
%    Consider any cell $C$ with $\mu_C (A) \ge (\d+\tau) \sigma$ (e.g., $C=\F_2^n \times \F_2^n$) and then
    By assumption $\mu_{S} (A) = \d +\tau \d$ and hence  in view of \eqref{f:U_density} for any $s$, we have 
\[
    (\d + \tau \d) \sigma N^2 
    =
    \mu_S (A) 
    \sigma N^2 = |A\cap \mathcal{T}_s| + |A\cap \mathcal{U}_s| + |A \cap \mathcal{N}_s| 
\]
\begin{equation}\label{f:N_U-}
    \le 
    2^{-2} \tau \d \sigma N^2 + (\d + \tau \d/4) |S\cap \mathcal{U}_s| + |A \cap \mathcal{N}_s| \,,
\end{equation} 
    and hence
\begin{equation}\label{f:N_U}
    2^{-2} \tau \d \sum_{C\in \mathcal{U}_s} |S\cap C| + 
    (\d +\tau \d) \sum_{C\in \mathcal{N}_s} |S\cap C|
    \le
    \sum_{C\in \mathcal{N}_s} |A\cap C|
\end{equation} 
    In particular, it means that the density of $A$ in $\mathcal{N}_s$ does not decrease and thus there are cells in $\mathcal{N}_s$ with density at least $\d + \tau \d$. 
    Split the set $\mathcal{N}_s$ as $\mathcal{N}^{+}_s$, $\mathcal{N}^{-}_s$, collecting all sets such that the density $A$ is at least   $\d + \tau \d/2$ and strictly less than  $\d + \tau \d/2$, correspondingly. 
    As we have seen the set $\mathcal{N}^{+}_s$ is always nonempty and the density of $A$ in  $\mathcal{N}^{+}_s$  does not decrease due to 
\begin{equation}\label{f:N+_U}
    2^{-2} \tau \d \sum_{C\in \mathcal{U}_s \bigsqcup \mathcal{N}^{-}_s} |S\cap C| + 
    (\d +\tau \d) \sum_{C\in \mathcal{N}^{+}_s} |S\cap C|
    \le
    \sum_{C\in \mathcal{N}^{+}_s} |A\cap C| 
    %\,.
\end{equation} 
    similar to inequality \eqref{f:N_U}. 
    Also, if one has  $|S\cap (\mathcal{U}_s \bigsqcup \mathcal{N}^{-}_s)| \ge \omega |S \cap \mathcal{N}^{+}_s|$ with some $\omega >0$, then we find a cell $C\in \mathcal{N}^{+}_s$ where $A$ has a larger density 
\begin{equation}\label{f:A_inc_in_N}
    \mu_{C \cap S}(A) \ge 
    \d + 2^{-2} \o \tau \d =
    %\d+\tau + 2^{-2} \o \tau = 
    \mu_{S} (A) + 2^{-2} \o \tau \mu_{S} (A) \,.
\end{equation}     
    After that we can use the same argument for the cell $C$ and iterate this sub--procedure. 
%    Of course, in \eqref{f:N_U-}---\eqref{f:A_inc_in_N} we just use the assumption \eqref{f:U_density}.
%    Also, notice that our cell $C$ is not thin  and a fortiori density of $A$ in $C$ is at least $2^{-2} \tau \d >0$.  
    It means that after at most  $K_1:= 8 (\omega \tau)^{-1}$ steps %either 
    we obtain a cell $C$ such that
    %the uniform part 
    the part $\mathcal{U}_s \bigsqcup \mathcal{N}^{-}_s$
    of this cell is in $\omega$ times smaller than the non--uniform part $\mathcal{N}^{+}_s$.
    %or for a certain non--uniform cell $C\in \mathcal{N}_s$. 
    Let us underline one more time that we use the fact that the definition of thin cells depends on the ambient cells. 
    Notice that after applying our procedure we obtain a cell with density of our set $A$ at least $\d +\d\tau$ and the density of the set $S$ in this cell is at least  $(2^{-2} \tau \d)^{K_1} \sigma$. 


    \bigskip 

    Now take $C=W_1 \times W_2\in \mathcal{N}^{+}_s$, suppose, for concreteness, that $S_1 \subset W_1$ is not a uniform set  and choose the minimal $i$ such that $S_1$ is $\eps_{i-1}$--uniform with respect to $\E^{k_{i-1}}_{i-1}$ but $S_1$ is not $\eps_{i}$--uniform with respect to $\E^{k_{i}}_{i}$. 
    Then we can use our splitting Proposition \ref{p:L2_increment_Ekl}  and 
    from  inequality \eqref{cond:L2_increment_2_Ekl_A},  
    %with $\eps = 2^{-4}\eps^2 (S_1)$, 
    where $\| f_{S_1} \|^{kl}_{\E^k_l (W_1)} := \eps^{kl} (S_1) \mu^{kl}_{W_1} (S_1) |W_1|^{k+l}$, 
    %\eqref{cond:L2_increment_2_Ekl_A}, 
    we have for a certain subspace $V$ that 
\begin{equation}\label{f:ind_inc_l>2}
    |V| \sum_{t_1\in W_1/V} (S_1 * \mu_V)^q (t_1) \ge  (2 \mu_{W_1} (S_1))^q |W_1| \,,
\end{equation} 
    and similar for $S_2$. 
    Split shifts $(t_1,t_2)\in  W_1/V \times W_2/V$ above onto three classes $\mathcal{N}_{s+1} (C)$,  $\mathcal{U}_{s+1} (C)$ and $\mathcal{T}_{s+1} (C)$ of non--uniform/uniform intersections of such shifts with the sets $S_1$, $S_2$, correspondingly and thin subcells. 
    Also, let 
\[
    S^{\mathcal{N}^{\pm}_{s+1}}  = \bigsqcup_{(t_1,t_2) \in \mathcal{N}^{\pm}_{s+1} (C)} (S \cap ((V+t_1) \times (V+t_2))) \,,
    \quad \quad 
    S^{\mathcal{U}_{s+1}} = \bigsqcup_{(t_1,t_2) \in \mathcal{U}_{s+1} (C)} (S \cap ((V+t_1) \times (V+t_2)))  \,.
\]
    By the arguments as in \eqref{f:N_U}, \eqref{f:A_inc_in_N} we can freely assume that $|S^{\mathcal{U}_{s+1}} \bigsqcup S^{\mathcal{N}^{-}_{s+1}}| < \omega |S^{\mathcal{N}^{+}_{s+1}}|$ and let us choose $\omega = \tau/8$. Clearly, the set $S^{\mathcal{N}^{+}_{s+1}}$ has density at least 
\begin{equation}\label{f:S^N}
    \mu_{C} (S^{\mathcal{N}^{+}_{s+1}}) \ge (1-\tau/4 - \omega) \mu_{C} (S) \ge \frac{5}{8}  \mu_{C} (S) \,.
\end{equation} 
    %, say, and the density of $A$ in $S^{\mathcal{N}_{s+1}}$ is greater than $\d + \tau - \tau/4 - \omega \ge \d + \tau/2$. 
    %It is easy to see that one can split the set $S^{\mathcal{N}_{s+1}}$ onto three Cartesian products according uniform/non--uniform sides of $S^{\mathcal{N}_{s+1}}$, denote these sets as  $X,Y,Z$, say.  
    %We do not consider thin sets among $X,Y,Z$, as well as sets where the density of $A$ is less than $\d+\tau/4$. 
    %By previous calculations the family of remaining sets is non--empty and if these sets are  $\eps$--uniform relatively $\E^k_l$, then we are done. 







    Now take any family of cells $\mathcal{Q}_s$ and at an arbitrary step $s$ of our algorithm consider 
\begin{equation}\label{def:ind}
    \mathrm{ind}_s (S; \mathcal{Q}_s) = N^{-2} \sum_{C \in \mathcal{Q}_s} |C| \mu^q_{C} (S)  \le 1 \,.
\end{equation} 
    It is easy to see from the H\"older inequality that the index of any set does not decrease if we 
    %consider 
    take 
    finer partitions.  
    In particular, for any $s$ one has 
\begin{equation}\label{tmp:lower_ind}
    \mathrm{ind}_{s} (S; \mathcal{F}_s) \ge \sigma^q 
%    \,,
    \quad \quad \mbox{and} \quad \quad 
    \mathrm{ind}_{s} (S; \mathcal{N}^{+}_s) \ge (2^{-2} \tau \d \sigma)^q \,,
%    \mathrm{ind}_{s} (A; \mathcal{F}_s) \ge (\d+\tau)^q \sigma^q \,.
\end{equation}  
    and any splitting process (e.g., our sub--procedure/main procedure) does not decrease the index.  
    The second bound in \eqref{tmp:lower_ind} follows from \eqref{f:N+_U}. 
%    Now we stop our algorithm if 
%\begin{equation}\label{f:non-uniform_stop}
%   \mathrm{ind}_s (S_1\times S_2; \mathcal{N}_s) \le 
%\end{equation}  
%    Take a cell $C \in \mathcal{N}_s$
    In view of inequalities  \eqref{f:ind_inc_l>2}, \eqref{f:S^N} one has 
\[
    \mathrm{ind}_{s+1} (S; \mathcal{N}^{+}_{s+1}) \ge (2 \cdot 5/8)^q \mathrm{ind}_{s} (S; \mathcal{N}^{+}_s)
    =
    (5/4)^q
    \mathrm{ind}_{s} (S; \mathcal{N}^{+}_s)
\]
%    and thus the negotiation of \eqref{f:non-uniform_stop} gives us 
%\[
%   \mathrm{ind}_{s+1} (S_1\times S_2; \mathcal{F}_s) \ge 2^{q/2} \mathrm{ind}_{s} (S_1\times S_2; \mathcal{F}_s) \,.
%\]
    In other words, in view of the second estimate of \eqref{tmp:lower_ind} our algorithm must stop after $K_2:=O(\L(\d \tau \sigma))$ number of steps.  
    After that we can apply our sub--procedure and repeat the process. We see that it stops in $K = K_1 K_2$ steps and hence we obtain a partition of $\Gr \times \Gr$ onto uniform cells (plus some thin cells) such that the minimal density $\tilde{\sigma}$ of $S$ in such uniform cells is at least 
\begin{equation}\label{tmp:sigma_final}
    \tilde{\sigma} \ge  (2^{-2} \tau)^K \sigma 
    \ge 
    \sigma \cdot \exp( - O( \tau^{-2} \L(\tau) \L(\d \tau \sigma))) \,.
\end{equation} 
    In view of \eqref{f:N_U} (also, see \eqref{f:U_density}) it means that we find $\tilde{S}_1$, $\tilde{S}_2$ with all required properties. 
    It remains to calculate the co--dimension of the resulting subspaces of our partition. 
    By definition of the function $f_j$ one can see that 
\begin{equation}\label{tmp:eps2}
    \eps_2 = (f_2 \circ \dots \circ f_{l-1} \circ f_l) (\eps) \ge 2^{-12 l^l} l^{-(4l)^l} \eps^{l^l} \,,
\end{equation} 
    and using this bound, we obtain 
\begin{equation}\label{tmp:k2}
    k_2 \ge (M l^2 \L(\tilde{\sigma}))^l (12l^l (4l)^l l^l \L(\eps) \log l)^l
    (2^{-12 l^l} l^{-(4l)^l} \eps^{l^l})^{-4} k
    \ge M^l 2^{50 l^l} l^{(5l)^l} \L^{l} (\eps) \L^l(\tilde{\sigma}) \eps^{-4l^l} k
    \,.
\end{equation} 
    It remains to combine estimate \eqref{cond:L2_increment_1_Ekl} of Proposition \ref{p:L2_increment_Ekl} with inequalities 
    %\eqref{tmp:sigma_final}, \eqref{tmp:eps2}, \eqref{tmp:k2} 
    \eqref{tmp:sigma_final}---\eqref{tmp:k2} 
    and obtain 
\[
    \mathrm{codim} V \ll \eps^{-2l}_2 l^6 k_2^4 \L^2 (\tilde{\sigma}) \L^2 (\eps_2 \tilde{\sigma})
    \ll 
    2^{200l^{l+1}} l^{(10l)^{l+1}} \eps^{-20l^{l+1}} 
    \left(\tau^{-2} \L^{} (\tau) \L^{} (\eps)  \L(\d) \L^{}(\sigma) \right)^{4l+4} k^4
%    \,,
\]  
as required. 
%This completes the proof. 
$\hfill\Box$
\end{proof}


\bp 


One can remark that the scheme of the proof of Proposition \ref{p:L2_corners} differs from the argument of \cite{Green_models}, \cite{sh_Sz_LMS}. Indeed, in \cite{Green_models}, \cite{sh_Sz_LMS} we use an analogue of the index \eqref{def:ind} and increase it at each step of the correspondent algorithm by a certain quantity. 
Now we have a rather strong inequality \eqref{cond:L2_increment_2_Ekl_A} of  Proposition \ref{p:L2_increment_Ekl} and our task to save this exponential gain in the proof of  Proposition \ref{p:L2_corners}. That is why the last result requires an additional sub--procedure which was absent in \cite{Green_models}, \cite{sh_Sz_LMS}. 

\bp 

Now we can prove Theorem \ref{t:corners_new}, the proof follows the ordinary scheme, see \cite{Green_models}, \cite{sh_Sz_LMS}. 
Let $A\subseteq \F_2^n \times \F_2^n$, $|A| = \d N^2$ and $A$ has no non--trivial corners. 
Put $S^{(0)}_1 = S^{(0)}_2 = \Gr$ and for an integer $i\ge 0$ write $|S^{(i)}_1| = \sigma^{(i)}_1 N$, $|S^{(i)}_2| = \sigma^{(i)}_2 N$, and $\sigma^{(i)} = \sigma^{(i)}_1 \sigma^{(i)}_2$. 
We subsequently use Corollary \ref{c:1+2} and Proposition \ref{p:L2_corners} with $l=4$ to increase the density of $A$ in some Cartesian products $S^{(i)}_1 \times S^{(i)}_2$ of $\E^k_4$--uniform sets $S^{(i)}_1$, $S^{(i)}_2$ for sufficiently large $k$.
Thanks to the increment \eqref{f:1+2_inc}, as well as the second part of  Proposition \ref{p:L2_corners}, we see that the algorithm must stop at $K:=O(\d^{-23})$ steps. 
Now the third part of  Proposition \ref{p:L2_corners} (the parameter $\tau$ equals $2^{-30} \d^{23}$ and $\eps = 2^{20} \d^{10}$) gives us 
\begin{equation}\label{f:sigma_K}
    \sigma_K \ge \exp(-O( (\tau^{-2} \L^2 (\tau) \L(\d))^K \L(\sigma))) \gg 
    \d^{-\d^{-C_1}}
\end{equation}
    as well as for $k_K \ll \L(\d \sigma_K)$ one has 
\begin{equation}\label{f:codim_K}
    \codim (W) \ll \eps^{-C_2} k^4_K (\tau^{-2} \L(\tau) \L(\eps) \L(\d) \L(\sigma_K))^{20} 
    \ll 
    \d^{-C_3}
    \,,
\end{equation} 
    where $C_1,C_2, C_3 >0$ are some  absolute constants.
    Also, at the last step of our algorithm we need to check that 
\begin{equation}\label{f:N_K}
    N > 2 \d^{-2} \sigma_K \gg \d^{-2-\d^{-C_1}} \,.
\end{equation} 
    Bounds \eqref{f:codim_K}, \eqref{f:N_K} give the required estimate for $\d$. 
%as required. 
This completes the proof. 
$\hfill\Box$


\bp 

One can show that in Theorem \ref{t:corners_new} one can take $c=1/2000$, say. 
%%1/1656+1
Of course, one can decrease this constant, using the methods of \cite{Lacey_corners}, \cite{sh_Izv_ab} or just apply Proposition \ref{p:L2_increment_Ekl} in the proof of Proposition \ref{p:L2_corners} in our particular case $l=4$ to decrease the constants $C_1,C_2,C_3$ in \eqref{f:sigma_K}, \eqref{f:codim_K}. We left this task for the interested reader. 


\end{comment}

\section{Appendix}
\label{sec:appendix}





We need a generalization of norms \eqref{def:E_kl}. Having vectors $x=(x_1,\dots,x_s) \in \Gr^s$ and $y=(y_1,\dots,y_t) \in \Gr^t$ (we write that $|x|=s$ and $|y|=t$) define its ``Minkowski'' sum as $x\oplus y \in \Gr^{st}$, where the components of $x\oplus y$ are $x_i+y_j$, $i\in [s]$, $j\in [t]$ (and similarly for higher sums). 
%Write 
Put
\begin{equation}\label{def:mE_kl}
    \mathcal{E}^k_{s,t} (f) = \sum_{|x|=s}\, \sum_{|y|=t} \Cf^k_{st} (f) (x\oplus y) =
    \sum_{|x|=s}\, \sum_{|y|=t}\, \sum_{|z|=k} f(x\oplus y \oplus z) 
    \,. 
\end{equation} 
In these terms 
\begin{equation}\label{def:mE_kl_pre}
    \E^k_l (f) =  \sum_{|x|=l}\, \sum_{|y|=k}\, f(x\oplus y) \,. 
\end{equation}
For even $k,s,t$ and a real function  $f$ one has $\mathcal{E}^k_{s,t} (f) \ge \E^k_{s} (f), \E^k_{t} (f) \ge 0$ and the triangle inequality for $\mathcal{E}^k_{s,t}$ can be obtained exactly as in \cite[Appendix]{sh_Ek} (but of course one needs an additional application of the H\"older inequality due to we have the longer sum in \eqref{def:mE_kl} than in \eqref{def:mE_kl_pre}). 
Thus 
%$\| f\|_{\mathcal{E}^k_{s,t}} := (\mathcal{E}^k_{s,t} (f))^{1/kst}$ 
$(\mathcal{E}^k_{s,t} (f))^{1/kst}$ 
defines a norm of $f:\Gr \to \R$ in the case of even $k,s,t$. 
Notice that similar to $\E^k_l (f)$ the quantity $\mathcal{E}^k_{s,t} (f) \ge 0$ and the triangle inequality for  $\mathcal{E}^k_{s,t} (f)$ takes place, provided at least one of $k,s,t$ is even but, nevertheless,  it is not  always a norm in this case, see \cite[Section 4]{sh_Ek}.  
By some symmetricity  reasons (see, e.g., formulae \eqref{f:E^k_st_symmetries} below) we make a normalization and put 
$$
\| f\|_{\mathcal{E}^k_{s,t}} := (|\Gr|^{-2} \mathcal{E}^k_{s,t} (f))^{1/kst} := (\bar{\mathcal{E}}^k_{s,t} (f))^{1/kst} 
$$
for $f:\Gr \to \R$. 
Clearly, one has 
\begin{equation}\label{f:mE_kl_1}
    \mathcal{E}^k_{s,t} (f)
    = 
    \sum_{|x|=s}\, \sum_{|z|=k}  \Cf^{t}_s (f_z) (x) 
    =
    \sum_{|x|=s}\, \sum_{|z|=k}  \Cf^{t}_{sk} (f) (x\oplus z) \,,
\end{equation}
    and
\begin{equation}\label{f:mE_kl_2}    
    \mathcal{E}^k_{s,t} (f)
    =
    \sum_{|y|=t}\, \sum_{|z|=k}  \Cf^{s}_t (f_z) (y) 
    =
    \sum_{|y|=t}\, \sum_{|z|=k}  \Cf^{s}_{tk} (f) (y \oplus z) 
    \,.
\end{equation} 
Thus we have the duality relation similar to \eqref{def:E_kl}
\begin{equation}\label{f:E^k_st_duality}
    \mathcal{E}^k_{t,s} (f) = \mathcal{E}^k_{s,t} (f) = \mathcal{E}^t_{s,k} (f) = \mathcal{E}^s_{t,k} (f)\,. 
\end{equation} 
%Finally, 
Also, let us remark that the expectations over $x\oplus y$ of the generalized convolution of any real function $f : \Gr \to \mathbb{R}$ is connected with the higher energies  
\begin{equation}\label{f:E^k_st_expectation}
    \sum_{|x|=s}\, \sum_{|y|=t} \Cf_{st} (f) (x\oplus y) = N \E^t_{s} (f) \,.
\end{equation}
In particular, the expectation above is always non--negative if $s$ or $t$ is an even number and we see immediately that the duality \eqref{def:E_kl} takes place. Formula \eqref{f:E^k_st_expectation} can be proved directly or it follows from  \eqref{f:mE_kl_1}, \eqref{f:mE_kl_2} and the fact that $\mathcal{E}^k_{t,1} (f) = N \E^k_{t} (f)$.
Finally, notice that in contrast to $\Cf_l (x)$ the function $\Cf_{st}(x\oplus y)$ enjoys even two symmetries, namely,
\begin{equation}\label{f:E^k_st_symmetries}
    \Cf_{st} (f) (x\oplus y) = \Cf_{st} (f) ((x+\mD_s (w_1))\oplus y) = 
    \Cf_{st} (f) (x\oplus (y + \mD_t(w_2))) 
\end{equation} 
for any $w_1,w_2\in \Gr$. 
Ii gives, in particular, 
\begin{equation}\label{f:Cf_to_Cf'+}
    \Cf_{st} (f) (x\oplus y) = \Cf_{st} (f) ((x - \mD_s (x_1))\oplus (y-\mD_t (y_1))) =  N^2 \Cf'_{st} (f) (w) \,,
\end{equation} 
where $|w|=st-1$ and, 
%more precisely, 
more concretely, 
$w_{ij} =(x_i-x_1)+(y_j-y_1)$, $i\in [s]$, $j\in [t]$ and $(i,j) \neq (1,1)$.  



\bp 

Now we are ready to obtain our counting lemma. 
%We 
Let us 
write $L(x,y) = \a x+\beta y + \gamma$ for a non--trivial linear form.
%Having 
Given  
a real number $q>1$ put $q^* = \frac{q}{q-1}$. 

\begin{theorem}
    Let $N$ be a prime number and 
$k=4$, $l_1,l_2 \ge 2$ be positive integers.
    %and $\mathcal{L}_i = \prod_{j\equiv i\pmod 2} l_j$, $i=0,1$. 
    %$\mathcal{L}_1 = \sum_{j\equiv 1\pmod 2} l_j$. 
    Also, let $f_1,\dots, f_k : \Z/N\Z \to \R$ be functions  and $L_1,\dots,L_k$ be non--proportional linear forms such that $L_2,\dots,L_k$ depend on both variables. Then  
\begin{equation}\label{f:counting}
    \left|\sum_{x,y} f_1( L_1(x,y)) \dots f_k( L_k(x,y)) \right| 
    \le 
    %\prod_{j=1}^{k-2} \| f_j\|_{\FF{l}_j l^*_j} \cdot \| f_{k-1} 
    \| f_1\|_{l^*_1} \| f_2\|_{l^*_2}
    \| f_3 \|_{\mathcal{E}^2_{l_1,l_2}} 
    \| f_{4} \|_{\mathcal{E}^2_{l_1,l_2}} \,.
\end{equation}
%    where $\FF{l}_j = \prod_{i<j,\, i\equiv j \pmod 2} l_i$. 
\label{t:counting}
\end{theorem}
\begin{proof} 
    Let $\sigma$ be the left--hand side of \eqref{f:counting}. 
    Without loss of  generality one can assume that $L_j(x,y)=\a_j x + \beta_j y$, $j\in [k]$. 
    Consider the nonzero form $L_1$ and suppose for concreteness that  $\a_1\neq 0$. 
    Changing the variables $\a_1 x + \beta_1 y \to x$, we obtain
\begin{equation}\label{tmp:sigma1}
    \sigma = \sum_{x,y} f_1(x) f_2( \tilde{L}_2(x,y)) \dots f_k( \tilde{L}_k(x,y)) \,,
\end{equation} 
    where here and below we write $\tilde{L}_j(x,y) =  L_{j} (x,y) = \a_j x + \beta_j y$ and the coefficients  $\a_j$, $\beta_j$  may change from line to line.
    Anyway one can check that all new forms $\tilde{L}_2,\dots, \tilde{L}_k$ in \eqref{tmp:sigma1} are nonzero and non--proportional.
    Moreover, by assumption the initial forms $L_2,\dots,L_k$ depend on both variables and we see that the new forms in \eqref{tmp:sigma1} depend on both variables as well. 
    Now we use the H\"older inequality and get
\[
    (\sigma/\|f_1\|_{l^*_1})^{l_1}
    \le
    \sum_{x} \left( \sum_y f_2( L_2(x,y)) \dots f_k( L_k(x,y)) \right)^{l_1}
\]
\[
    =
    \sum_{x,y} \mP_{l_1} (f_2) ( L_2(\mD_{l_1} (x),\mP_{l_1} (y))) \dots \mP_{l_1} (f_k) ( L_k(\mD_{l_1} (x),\mP_{l_1} (y))) \,.
\]
    Notice that we have decreased the number of our linear forms (but increased the number of variables). 
    Now let us make the  changing of the variables  similar to above, namely, $\a_1 \mD_{l_1} (x) + \beta_1 \mP_{l_1} (y) \to \mP_{l_1} (y)$ and again one can easily check that we preserve all conditions on our linear forms $L_3, \dots, L_k$. 
    Thus one has 
\[
    (\sigma/\|f_1\|_{l^*_1})^{l_1} \le 
    \sum_y \mP_{l_1} (f_2) (\mP_{l_1} (y)) \sum_x \mP_{l_1} (f_3) ( L_3(\mD_{l_1} (x),\mP_{l_1} (y))) \dots \mP_{l_1} (f_k) ( L_k(\mD_{l_1} (x),\mP_{l_1} (y)))
\]
    and using the H\"older inequality 
    %again,  
    one more time, 
    as well as the obvious identity 
\begin{equation}\label{tmp:26.02_-1}
    \left(\sum_y \mP^{l^*_2}_{l_1} (f_2) (\mP_{l_1} (y))  \right)^{l_2-1}  = \| f_2 \|^{l_1 l_2}_{l^*_2} \,, 
\end{equation} 
    we derive
\[
    (\sigma/\|f_1\|_{l^*_1} \|f_2\|_{l^*_2})^{l_1 l_2 } 
\]
\begin{equation}\label{tmp:sigma_k_2}
    \le 
    \sum_{x,y} \mP_{l_1 l_2} (f_3) ( L_3(\mP_{l_2} \mD_{l_1} (x),\mD_{l_2} \mP_{l_1} (y))) \mP_{l_1 l_2} (f_4) ( L_4(\mP_{l_2} \mD_{l_1} (x),\mD_{l_2} \mP_{l_1} (y))) \,.
\end{equation} 
    Now 
    let us analyse the right--hand side of formula \eqref{tmp:sigma_k_2}. 
%    the last formula. 
    First of all, it is easy to see that there are $l_2$ different variables $x_i$ and  $l_1$ different variables $y_j$ in 
    %both expressions \eqref{tmp:sigma_k_1}, 
    \eqref{tmp:sigma_k_2}. 
    Secondly, take the form $L_{k-1}$ (for $L_k$ the argument is the same) and notice that it depends on $\a_{k-1} x_i + \beta_{k-1} y_j$, $i\in [l_2]$, $j\in [l_1]$ and that every such expression appears exactly once. 
    Now introducing two more variables $z,w$ such that $x_i\to x_i+z$, $y_j \to y_j + w$ and then replacing $z,w$ to other variables $Z,W$, where $Z=\a_{k-1} z + \beta_{k-1} w$, $W = \a_{k} z + \beta_{k} w$ (this change of the variables is allowable because the forms $L_{k-1}$, $L_k$ are not proportional), we arrive to the quantities  $\Cf_{l_1l_2} (f_{k-1})$, 
    $\Cf_{l_1l_2} (f_{k})$
    \eqref{tmp:sigma_k_2}.
    Writing  $x= (x_1,\dots, x_{l_2})$, $y= (y_1,\dots, y_{l_1})$, we have finally 
\[
    (\sigma/\|f_1\|_{l^*_1} \|f_2\|_{l^*_2})^{l_1 l_2 } 
    \le 
    N^{-2}
    \sum_{\vec{x},\vec{y}} 
    \Cf_{l_1 l_2} (f_{k-1}) (\a_{k-1} \cdot x \oplus   \beta_{k-1} \cdot y)
    \Cf_{l_1 l_2} (f_{k}) (\a_{k} \cdot x \oplus  \beta_{k} \cdot y) \,.
\]
    Using the H\"older inequality the last time, as well as the fact that $\a_{k-1},\a_k,\beta_{k-1}, \beta_k \neq 0$, we obtain 
\[
 (\sigma/\|f_1\|_{l^*_1} \|f_2\|_{l^*_2})^{l_1 l_2 } 
\]
\[
    \le 
    \left( N^{-2}\sum_{|x| =l_2,\, |y|=l_1}  
    \Cf^2_{l_1 l_2} (f_{k-1}) (x \oplus y)
    \right)^{1/2} 
    \left( N^{-2} \sum_{|x| =l_2,\, |y|=l_1}  \Cf^2_{l_1 l_2} (f_{k}) (x \oplus y)
    \right)^{1/2} 
\]
\[    
    = 
    \| f_{k-1} \|^{l_1 l_2}_{\mathcal{E}^2_{l_1,l_2}} \cdot 
    \| f_{k} \|^{l_1 l_2}_{\mathcal{E}^2_{l_1,l_2}}
\]
as required. 
%This completes the proof. 
$\hfill\Box$
\end{proof}


\begin{remark}
    One can check that for any $l_1,\dots, l_{k-2}$ one has 
\[
    \frac{1}{l^*_1} + \frac{1}{l^*_2} +\frac{1}{l_1} + \frac{1}{l_2}= 2
\]
    and hence the right--hand side of bound \eqref{f:counting} has the correct order in $N$. 
    Similarly, taking $f_j (x) = f_A (x)$, $j\in [4]$ be the balanced function of a set $A$ and $l_1 \sim l_2 \sim \L (\d)$ we see that the dependence on $\d$ is also correct. 
\end{remark}



\begin{remark}
    As we have said in the previous remark the optimal dependence on the parameters $l_1,l_2$ in Theorem \ref{t:counting} is $l_1 \sim l_2 \sim \L (\d)$. 
    Suppose that the dependence on $\eps$ in Lemma \ref{l:e_to_d} and in all statements below is almost optimal, say, $c\eps$ for a constant $c\in (0,1)$. 
    Thanks to the induction scheme of the proof, it gives us the multiple $c^{\L(\d)} = \d^{-C}$ for a certain $C>0$ in codimention of the subspace $V$, where our set $A$ is uniform.
    But $\d^{-C}$ is more or less that usual Gowers' method gives to us and hence we have no special gain. 
    Thus there are considerable difficulties to extend the Kelley--Meka method for more complicated objects than arithmetic progressions of length three even on  the technical level. 
\label{r:ET_fail}
\end{remark}


We conclude the appendix showing that the convolutions $\Cf_{st} (f) (x\oplus y)$, $|x|=s$, $|y|=t$ 
%satisfy the same 
enjoy the almost periodicity properties similar to the ordinary convolutions $\Cf_s (f) (x)$. 
For $x=(x_1,\dots,x_r)$ let us write  for convenience $\bar{x} = (0,x_1,\dots,x_r)$.


\begin{lemma}
    Let $\eps \in (0,1]$ be a real number, $s,t,q\ge 2$ be positive integers, $l:=st-1$, $B\subseteq \Gr$, $|B| = \beta N$ and $F: \Gr^l \to \R$.
    Then there is a set $T\subseteq \Gr$, $|T|\ge |B| \exp(-O(\eps^{-2} q \log (1/\beta)))$    and such that for any $t\in T$ one has 
\[
    \sum_{|x|=s-1}\, \sum_{|y|=t-1} \left| (F\circ \mD_{l} (B+t) )(\bar{x}\oplus \bar{y}) - (F\circ \mD_{l} (B)) (\bar{x}\oplus \bar{y}) \right|^q 
\]
\begin{equation}\label{f:CS_new}
    \le 
    \eps^q |B|^{q-1} \sum_{|y|=t-1} \Cf'_t (|F|^q)^{s-1} (y) \cdot \Cf'_t (B,|F|^q, \dots, |F|^q) (y) \,.
\end{equation} 
\label{l:CS_new}
\end{lemma} 
\begin{proof} 
    We choose $k = O(\eps^{-2}q)$  random points $b_1,\dots, b_k \in B$ uniformly and independently and let $Z_j ((\bar{x}\oplus \bar{y}) = F((\bar{x}\oplus \bar{y}) + \mD_l (b_j)) - (F\circ \mD_{l} (\mu_B)) (\bar{x}\oplus \bar{y})$. 
    Clearly, the random variables $Z_j$ are independent, have zero expectation and their variances do not exceed $(|F|^2\circ \mD_{l} (\mu_B)) (\bar{x}\oplus \bar{y})$.
    By the Khintchine inequality for sums of independent random variables,
$$
    \| \sum_{j=1}^k Z_j (\bar{x}\oplus \bar{y}) \|_{L_p (\mu^k_B)} \ll (|F|^2\circ \mD_{l} (\mu_B)) (\bar{x}\oplus \bar{y})^{1/2} \,.
$$
Raising the last inequality to the power $q$, dividing by $k^q$, summing over  $\bar{x}\oplus \bar{y}$, and using the H\"older inequality, which gives $(|F|^2\circ \mD_{l} (\mu_B)) (\bar{x}\oplus \bar{y})^{q/2} \le (|F|^q\circ \mD_{l} (\mu_B)) (\bar{x}\oplus \bar{y})$, we get that
$$
    \sum_{|x|=s-1}\, \sum_{|y|=t-1} \int  \left| 
    \frac{1}{k} \sum_{j=1}^k F((\bar{x}\oplus \bar{y}) + \mD_l (b_j)) - (F\circ \mD_{l} (\mu_B)) (\bar{x}\oplus \bar{y})
     \right|^q  d\mu_B^k (x_1,\dots,x_k)
$$
$$
        \ll
        (qk^{-1})^{q/2} \sum_{|x|=s-1}\, \sum_{|y|=t-1} (|F|^q\circ \mD_{l} (\mu_B)) (\bar{x}\oplus \bar{y})
$$
$$
        =
        (qk^{-1})^{q/2} |B|^{-1} \sum_{|y|=t-1} \Cf'_t (|F|^q)^{s-1} (y) \cdot \Cf'_t (B,|F|^q, \dots, |F|^q) (y) \,.
$$
 After that we repeat the argument from 
 \cite{CS}, \cite{sanders2012bogolyubov}, \cite{sanders2013structure} and 
\cite[Theorem  15]{sh_str_survey}. 
    This completes the proof. 
$\hfill\Box$
\end{proof}



\bibliographystyle{abbrv}

\bibliography{bibliography}{}


%\addbibresource{bibliography.bib}
%\printbibliography



\end{document}







\bigskip 


\noindent{I.D.~Shkredov\\
Steklov Mathematical Institute,\\
ul. Gubkina, 8, Moscow, Russia, 119991}
\\
%and
%\\
%IITP RAS,  \\
%Bolshoy Ka ny per. 19, Moscow, Russia, 127994\\
%and 
%\\
%MIPT, \\ 
%Institutskii per. 9, Dolgoprudnii, Russia, 141701\\
{\tt ilya.shkredov@gmail.com}



%%%%%%%%%%%%




%%%%%%%%%%%


\bigskip

Now we are ready to obtain our driving result about the density increment. 

\begin{proposition}
    Let $\Gr = \F_p^n$, $A\subseteq \Gr$, $|A|=\d N$ and $\eps \in (0, 1/2)$ be a real number.
    Suppose that 
\begin{equation}\label{cond:increment}
    \mathcal{E}^k_{s,t} (A) \ge (1+\eps)^{kst} \d^{kst} N^{s+t+k} \,.
\end{equation} 
    Then there is a subspace $V \subseteq \Gr$ such that 
\begin{equation}\label{cond:increment_1}
    \mathrm{codim} V \ll 
    \,,
\end{equation} 
    and for a certain $x\in \Gr$ one has 
\begin{equation}\label{cond:increment_2} 
    |A\cap (V+x)| \ge (\d + ) |V| \,.
\end{equation} 
\label{p:increment}
\end{proposition}
\begin{proof} 
    It is more convenient for us to have deal with the energy $\mathcal{E}^s_{t,k} (A)$ than with the initial energy 
    $\mathcal{E}^k_{s,t} (A)$ but of course it is the same thanks to \eqref{f:E^k_st_duality}. 
    Let $l=ks$.
    Applying Lemma \ref{l:BSzG_k} for the energy $\mathcal{E}^t_{s,k} (A)$ with the parameters $\eps = \eps$, $\eta = 1/6$, we construct the set 
\[
    S = \{ (x,z) ~:~ |x|=s,\, |z| = k,\, \Cf_l (A) (x\oplus z) \ge (1+\eps/4)^l \d^l N \}
\]
and such that for a certain set $B \subseteq \Gr$,  $|B| >  2^{-1/(l-1)} (1+\eps)^t \d^t N := \beta N$ the following holds 
\begin{equation}\label{tmp:01.03_1} 
    N^{-1} \sum_{|x|=s}\, \sum_{|z|=k} S (x\oplus z) \Cf_l (B) (x\oplus z) \ge (1-2\eta) \E^k_{s} (B) \,.
\end{equation} 
    We have $kl \ge 2\eps^{-1} \log (4/\eta)$ and thus Lemma \ref{l:BSzG_k} can be applied indeed 
    {\bf ???}.
    For $x=(x_1,\dots,x_r)$ write $\bar{x} = (0,x_1,\dots,x_r)$.
    Also, define for $x=(x_1,\dots,x_{r_1})$ and $y=(y_1,\dots,y_{r_2})$ the vector $\bar{x} \oplus \bar{y}$ as the vector with $r_1r_2-1$ coordinates which equal $x_i+y_j$, $x_i$, $y_j$, where $i\in [r_1]$ and $j\in [r_2]$. 
    Using formulae \eqref{f:E^k_st_symmetries} and making the required change of the variables (in \eqref{def:Cf} we put, consequently, $z\to z-x_1$ and then $z\to z-y_1$) one can see that \eqref{tmp:01.03_1} is equivalent to 
\begin{equation}\label{tmp:01.03_2} 
    %N^{-2} 
    \sum_{|x|=s-1}\, \sum_{|z|=k-1} \bar{S} (\bar{x}\oplus \bar{z}) (B^{l-1} \circ \mD_{l-1} (B)) (\bar{x}\oplus \bar{z})  \ge (1-2\eta) \bar{\E}^k_{s} (B) \,,
\end{equation} 
    where $\bar{S} \subseteq \Gr^{l-1}$ is a certain set which is constructed via the set $S$, see formula \eqref{f:Cf_to_Cf'+}. 
    Now we apply Lemma \ref{l:Sanders_V} with $f=\bar{S}$, $B=B$, $\eps = \eta$, and $l=l-1$.
    By this result and inequality \eqref{tmp:01.03_2} we find a subspace $V\subseteq \Gr$ such that 
    the co--dimension of $V$ is controlled 
    %thanks to  
    by estimate 
    \eqref{f:Sanders_V_codim} and 
\begin{equation}\label{tmp:01.03_3} 
    \sum_{|x|=s-1}\, \sum_{|z|=k-1} \bar{S} (\bar{x}\oplus \bar{z}) (B^{l-1} \circ \mD_{l-1} (B * \mu_V)) (\bar{x}\oplus \bar{z})  \ge (1-3\eta) \bar{\E}^k_{s} (B) \,.
\end{equation} 
    By the definition of the set $S$ (and hence $\bar{S}$), we have $\Cf_l (A) (\bar{x}\oplus \bar{z}) \ge (1+\eps/4)^l \d^l N$
    for any $\bar{x}\oplus \bar{z} \in \bar{S}$ and hence inequality \eqref{tmp:01.03_3}  gives us 
\[
    (1-3\eta) (1+\eps/4)^l \d^l \bar{\E}^k_{s} (B) N \le \sum_{|x|=s-1}\, \sum_{|z|=k-1} 
    (A^{l-1} \circ \mD_{l-1} (A)) (\bar{x}\oplus \bar{z})
    (B^{l-1} \circ \mD_{l-1} (B * \mu_V)) (\bar{x}\oplus \bar{z})
\]
\begin{equation}\label{tmp:01.03_4} 
    =
%    \sum_{|x|=s-1}\, \sum_{|z|=k-1} \mD_{l-1} (\mu_V \circ A) (\bar{x}\oplus \bar{z}) (B^{l-1} \circ A^{l-1} \circ \mD_{l-1} (B)) (\bar{x}\oplus \bar{z}) \,.
    \sum_{\a} (A*B*\mu_V) (\a)  
    \sum_{|x|=s-1}\, \sum_{|z|=k-1} A^{l-1} (\bar{x}\oplus \bar{z}) B^{l-1} ((\bar{x}\oplus \bar{z}) + \mD_{l-1} (\a)) \,.
\end{equation} 
    It follows that 
\[
    \| A * \mu_V \|_\infty \ge \d (1-3\eta) (1+\eps/4)^l \ge \d (1+\eps/8)^l \,,
\]
    provided $l\ge 5/\eps$. 
%as required. 
This completes the proof. 
$\hfill\Box$
\end{proof}


%%%%%%%%%%%%%%%


\section{Proof of the main result} 

\begin{theorem}
    Let $\Gr = \F_p^n$, where $p$ is a prime number, $p>$. 
    Let $k\ge 3$ be a positive integer and $A$ be a set having no arithmetic progressions of length $k$.
    Then 
\label{t:main}
\end{theorem}
\begin{proof}
    Let $\d = |A|/N$ and $f(x) = f_A(x)$ be  the balanced function of $A$. 
    Given some functions $f_1,\dots, f_k : \Gr \to \C$ denote by $S(f_1,\dots,f_k)$ the functional 
\[
    S(f_1,\dots,f_k) 
    %= S_k (f_1,\dots,f_k) 
    = \sum_{x,y} f_1(x) f_2(x+y) \dots f_k (x+(k-1)y) \,.
\]
    Clearly, $S (A,\dots, A)$ equals the number of all (trivial and non--trivial) progressions in $A$. 
    Let us calculate $S (A,\dots, A)$ in terms of the balanced function of our set $A$. 
    Using the formula $A(x) = f (x) + \delta$ several times, as well as the trivial identity $\sum_x f(x) = 0$ one has 
\begin{equation}\label{tmp:S(A)}
    S (A,\dots, A) = \d^k N^2 + \sum_{2\le i<j\le k} \d^{k-i-1} S(A,\dots,A,f,1,\dots,1,f,1,\dots,1) \,,
\end{equation} 
    where the first function $f$ in \eqref{tmp:S(A)} appears at the place $i$ and the second one at the place $j$. 
    By assumption the set $A$ has no non--trivial arithmetic progressions and hence $S(A,\dots,A) = |A| = \d N \le 2^{-1} \d^{k}N^2$, provided $N\ge 2\d^{1-k}$. 
    Using this fact, formula \eqref{tmp:S(A)} and the triangle inequality, we see that there are $i<j$ with 
\begin{equation}\label{tmp:26.02_1}
    S(A,\dots,A,f,1,\dots,1,f,1,\dots,1) \ge \frac{\d^{i+1} N^2}{k^2} := \eps_k \d^{i+1} N^2 \,.
\end{equation} 
    Let us emphasize one more time that there are some functions $A$ in the last formula and hence to estimate quantity \eqref{tmp:26.02_1} one can use our counting Theorem \ref{t:counting}. 
    For concreteness let $i=k-1$ and $j=k$. 
    Take $l_1=l_2= l = M\log (1/\d)$, where $M>1$ is a sufficiently large constant and $l\ge 2$ is an even number  and put $l_3=\dots=l_{k-2} =2$.
    Let us write for brevity $s=\mathcal{L}_0$ and $t=\mathcal{L}_1$, where $\mathcal{L}_i = \prod_{j\equiv i\pmod 2} l_j$, $i=0,1$. 
    Using Theorem \ref{t:counting} with these parameters, we obtain 
\begin{equation}\label{tmp:26.02_2}
    \| f \|^{2st}_{\mathcal{E}^2_{s,t}} \ge \eps^{2st}_* \d^{2st} N^{s+t} \,,
\end{equation} 
    where $\eps = \eps_k \cdot $.
    This completes the proof. 
$\hfill\Box$
\end{proof}


%%%%%%%

\begin{comment}
    By the H\"older inequality, we get
\[
    \frac{\eps^2}{7} \d^4 |B|^4 N^2 \le \sum_\a (f_A * \mu_V)^2 (\a) \cdot \sum_\a (f_A \circ f_A)(\a) (B\circ B * B \circ B) (\a) \,.
\]
    Using Lemma \ref{l:uniformity} and the fact that $\| f\|^{2k}_{\E^k_2} = \eps^{2k}_* \d^{2k} N^{k+2}$ with $\eps = \eps^{2}_*/16$, we see that 
\[
    \frac{\eps^2}{7} \d^4 |B|^4 N^2 \le 4\eps^{1/2} \d^2 N^{1+1/k} |B|^2 |B|^{-1/k}
\] 
\end{comment} 


%%%%%%%%%%%%%%%%


    As for the sum $\sigma_2$ we use the H\"older inequality and obtain 
\begin{equation}\label{tmp:03.09_4}
\sigma^2_2 \le \sum_\a (f_A * \mu_V)^2 (\a) \cdot \sum_\a 
(A\circ B)^{l-1} (\a) 
((A\circ B)^{l-1} \circ B \circ B) (\a) = \sigma_3 \sigma_4
\,.
\end{equation} 
To estimate $\sigma_4$ we use the same argument as in Lemma \ref{l:uniformity} and derive
\[
    \sigma^q_4 \le  \sum_\a (A\circ B)^{(l-1)q} (\a) \cdot |B|^{l+1}   \left( \sum_\a (A\circ B)^{l-1} (\a) |B|^2 \right)^{q-1}
\]
\[
    \le 
    |B|^{2q+l-1}  
    \left( \sum_\a (A\circ B)^{(l-1)q} (\a) \right)^{2-1/q} N^{(1-1/q)(q-1)}
\]
and hence by Lemma \ref{l:uniformity}, we see that
\[
    \sigma^q_4 \le (2\d)^{(l-1)(2q-1)} |B|^{2q+l-1 + (l-1)(2q-1)} N^{q} 
    = (2\d)^{-(l-1)}  (2\d)^{2(l-1)q} |B|^{2lq} N^q \,.
\]
Combining the last bound with \eqref{tmp:03.09_3'}, \eqref{tmp:03.09_4}, we derive
\[
    2^{-6} \eps^2 \d^{2l} |B|^{2l} N^2 
    \le 
    \sum_\a (f_A * \mu_V)^2 (\a) \cdot 
    (2\d)^{2l-2} |B|^{2l}  N (2\d)^{-(l-1)/q} 
\]

%%%%%%%%%%%%%%%%%%%%%%%%%%%%%%%


\begin{comment} 
Lower bounds of Behrend \cite{Behrend} and Rankin \cite{Rankin} (the history of the question and some small improvements can be found in \cite{o2008sets}, say) give us 
\begin{equation}\label{f:Rankin}
    r_k (N) \gg N \cdot \exp (-C^n \log^{1/n} N) \,,
\end{equation} 
where $C>1$ is an absolute constant and $n= \lceil \log k\rceil$. 

So, we have improved the bounds for $r_K(N)$ from \cite{Szemeredi_4}, \cite{Szemeredi_m}, \cite{Gowers_4}, \cite{Gowers_m}. 
Of course, our result implies that the  primes contain arithmetic progressions of any length, see \cite{GT_primes}. 
\end{comment} 


%%\cite{GT_AP4} 