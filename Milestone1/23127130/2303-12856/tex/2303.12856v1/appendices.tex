
\section{Omitted proofs}
\label{sec:otherproofs}
\begin{proof}[Proof of \cref{ABequiv}]
Let $\PBnorm{\psi}=\inf\{\Bnorm{f}|\AP f=\psi\}$. Then,
\[
\begin{aligned}
\ABnorm{\psi}
&=\inf\{\:\:\Bnorm{f}\:\:|\:\AS f=\psi\}
\\&=\inf\{\:\:\Bnorm{f}\:\:|\:\AP(\sqrt{n!}f)=\psi\}
\\&=\inf\{\:\:\Bnorm{\tfrac1{\sqrt{n!}}g}\:\:|\:\AP(g)=\psi\}
=\tfrac1{\sqrt{n!}}\PBnorm{\psi}.
\end{aligned}
\]
So it suffices to show that $\PBnorm{\psi}=\Bnorm{\psi}$ for any anti-symmetric function $\psi$.

    $\PBnorm{\psi}\le\Bnorm{\psi}$ holds because the infimum $\PBnorm{\psi}=\inf\{\Bnorm{f}|\AP f=\psi\}$ over $f$ includes $f=\psi$.
    
    $\PBnorm{\psi}\ge\Bnorm{\psi}$. To show this, fix $\psi$ and let $f_\rho$ be such that $\AP{f_\rho}=\psi$ and $\varphi(\rho)\to\PBnorm{\psi}$. We need a representation of $\AP f_\rho$ as $f_{\rho'}$ for some measure $\rho'$ to bound its raw Barron norm. But indeed, $\psi=\AP f_\rho=f_{\rho'}$ where
    \[\rho'=\frac1{n!}\sum_{\pi\in S_n}\rho_\pi\quad\rho_\pi(a,b,\w)=\rho((-1)^{\pi}a,b,\pi(\w)).\]
    $\varphi(\rho_\pi)=\varphi(\rho)$ for each $\pi$, so the same holds for $\rho'$. Now $\Bnorm{\psi}\le\varphi(\rho')=\varphi(\rho)\to\PBnorm{\psi}$ which proves the inequality.

\end{proof}
    


\subsection{The Fourier inversion formula for ReLU}
\label{invformula}
%\NA{define lp and hp again}
We verify \cref{Fdef} for the ReLU activation with $\widehat\activation$ given by \cref{Frelu}. \cref{Fdef} claims that $\LP\activation\gamma=p_\gamma+O(\gamma g)$ where $p$ is a low-degree polynomial and $g$ is bounded by a polynomial. Here we have defined $\LP\activation\gamma=\activation-\HP\activation\gamma$.
%\begin{enumerate}
   % \item $\activation=\opn{ReLU}$: 
    We first evaluate the high-pass part
\[\HP\activation{\gamma}(y):=
\frac1{\sqrt{2\pi}}\int_{|\theta|>\gamma}\hat\sigma(\theta)e^{i\theta y}d\theta
=|y|/2-\frac{\cos(\gamma y)}{\pi \gamma }-\frac{y\opn{Si}(\gamma y)}\pi,\]
where $\opn{Si}(y)=\int_0^y\frac{\sin s}sds$. Since $\opn{ReLU}(y)=|y|/2+y/2$,
\begin{align}
\LP\activation{\gamma }(y):=\activation(y)-\HP\activation\gamma(y)
&=y/2+\frac{\cos(\gamma y)}{\pi \gamma }+\frac{y\opn{Si}(\gamma y)}\pi.
\end{align}
Write $\LP\activation\gamma=p_\gamma+\varepsilon$ where
\[p_\gamma(y)=y/2+\frac1{\pi\gamma}.\]
Then the remainder satisfies
\begin{equation}\label{boundlp}
|\varepsilon|\le|\frac{\cos(\gamma y)-1}{\pi \gamma }|+|\frac{y\opn{Si}(\gamma y)}\pi|\le\frac{(\gamma y)^2}{2\pi \gamma }+\frac{y\cdot(\gamma y)}{\pi}=\gamma g(y),\qquad g(y):=\frac{3 }{2\pi}y^2.\end{equation}




\subsection{Infra-red estimate}
\label{ests}

\begin{proof}[Proof of \cref{AAbound}]
By \cref{Frelu} we have $|\widehat\relu(\theta)|=\frac1{\sqrt{2\pi}\theta^2}$ for $|\theta|>0$, so \cref{limeps} implies
\begin{equation}\label{intwp}
\Xnorm{A_{\w,b;\gamma}-A_{\w,b}}\vphantom{\frac11}\le
\frac1{2\pi}\int_{|\theta|<\gamma}\frac{\Xnorm{\aa_{\theta\w}}}{\theta^2}d\theta,
\end{equation}
Now apply \cref{Dttlowbound} to get the bound
\begin{equation}\label{sep}
\Xnorm{\aa_{\theta\w}}^2\le\Big(\frac{|\theta|}{2\gamma}\Big)^2\Big(\frac{|\theta|}{2\gamma}\Big)^{\Omega(n^{1+1/d})}\quad \text{for }|\theta|\le\gamma.
\end{equation}
Here we have taken out two powers of $\theta/(2\gamma)$ in order to cancel the $1/\theta^2$ in \cref{intwp}. Rearranging \cref{sep} yields
\begin{equation}\label{loose}
\frac{\Xnorm{\aa_{\theta\w}}}{\theta^2}\le\gamma^2\Big(\frac{|\theta|}{2\gamma}\Big)^{\Omega(n^{1+1/d})}\le\gamma^2 2^{-\Omega(n^{1+1/d})}.
\end{equation}
Now substitute \cref{loose} into \cref{intwp} and use $\gamma=O(1)$ to obtain the result.
\end{proof}


%\item $\activation=\tanh$:
%We can write the low-pass part as an absolutely convergent integral as
%\[\LP\activation{\epsilon }(y)=\frac1{\sqrt{2\pi}}\int_{-\epsilon }^\epsilon \frac{-i\sqrt{\pi/2}}{\sinh(\pi\theta/2)}(e^{i\theta y}-1)d\theta.\]
%Let $p,C_\epsilon\equiv0$ and bound
%\begin{align}
%    |\LP\activation{\epsilon }(y)|
%   % \le&\frac1{\sqrt{2\pi}}\int_{-\epsilon }^\epsilon \Big|\frac{-i\sqrt{\pi/2}}{\sinh(\pi\theta/2)}(e^{i\theta y}-1)\Big| d\theta.
%   % \\
%    \le&\frac1{\sqrt{2\pi}}\int_{-\epsilon }^\epsilon \frac{\sqrt{\pi/2}}{|\pi\theta/2|}|\theta y| d\theta
%    =
%    \epsilon g(y),\qquad g(y):=\frac2\pi |y|.
%\end{align}
%\end{enumerate}

%\subsection{Proof of \cref{convexcombi}}
%\begin{proof}
%Write the complex measure $\mu$ as $\mu=\zeta|\mu|$ where $|\mu|$ is a non-negative measure and $\zeta$ is measurable function with values on the complex unit circle. Define the probability distribution $p=|\mu|/\|\mu\|$. Then $\psi=\int f d\mu(f)=\|\mu\|\EE_{f\sim p}[\zeta(f) f]$.
%
%Sample $f_1,\ldots,f_m\sim p$ according to the and let $\psi_m=\frac1m\sum_i\zeta(f_i)f_i$.
%\[
%\EE[\Xnorm{\psi_m-\psi}^2]=
%\EE[\Xnorm{\psi_m-\EE[\psi_m]}^2]=
%\]
%%The calculation is similar to \cite{e_barron_2022}: \NA{Since this part is standard we'll probably move it to the appendix}\LL{ sure} 
%    \[
%    \begin{aligned}
%    &\EE_{\tilde a_k,\tilde\w_k}\EE_{X}[|\tilde\psi_\epsilon(X)-\psi_\epsilon(X)|^2]
%    \\=&
%    \EE_{X}\EE_{\tilde a_k,\tilde\w_k}[|\tilde\psi_\epsilon(X)-\EE_{\tilde a_k,\tilde\w_k}[\tilde\psi_\epsilon(X)]|^2]
%    \\=&
%    \EE_{X}\operatorname{Var}[\tilde\psi_\epsilon(X)],
%    \end{aligned}
%    \]
%    where $\Var Y=\Var[\Re Y]+\Var[\Im Y]$. So we have
%    \[
%    \begin{aligned}
%    \EE_{\tilde a_k,\tilde\w_k}\Xnorm{\tilde\psi_\epsilon-\psi_\epsilon}^2
%    &=
%    \frac1m\EE_{X}\operatorname{Var}[\tilde a\alpha_{\tilde\w}(X)]
%    \\&\le
%    \frac1m\EE_{X}\EE_{\tilde a,\tilde\w}[|\tilde a\alpha_{\tilde\w}(X)|^2]
%    \\&=
%    \frac1m\EE_{\tilde a,\tilde\w}[|\tilde a|^2\Xnorm{\alpha_{\tilde\w}}^2].
%    \end{aligned}
%    \]
%\end{proof}



%\section{Bound on the overlap kernel for the Gaussian envelope}
%\label{sec:det_upper_bound}
%
%
%\begin{restatable}{lemma}{lemmalowranksum}
%    \label{lem:rank_one_terms}
%Let $\v=(v_1,\ldots,v_n)^T\in\RR^{n\times d}$ and $\w=(w_1,\ldots,w_n)^T\in\RR^{n\times d}$. Then $(e^{v_i\cdot w_j})_{ij}=\sum_{k=0}^\infty Q_k$ where
%\begin{equation}\opn{rank}Q_k\le\binom{k+d-1}{d-1},\qquad\|Q_k\|\le \frac{n(\|v\|_\infty\|w\|_\infty d)^k}{k!}.\label{Qk}\end{equation}
%\end{restatable}
%%\lemmalowranksum*
%\begin{proof}
%Let $(c_1,\ldots,c_d)$ and $(\tilde c_1,\dots,\tilde c_d)$ be the columns of $v$ and $w$ and let $\entwise$ denote elementwise operations. 
%\begin{equation}
%    \label{columndecomp}
%(e^{v_i\cdot w_j})_{ij}=e^{\entwise\sum_{i=1}^dc_i\tilde c_i^T}=\entp_{i=1}^d e^{\entwise c_i\tilde c_i^T}.
%\end{equation}
%We first consider each factor $e^{\entwise c_i\tilde c_i^T}$ separately. Elementwise multiplication of rank-one matrices given as outer products corresponds to elementwise multiplication of the vectors, $ab^T\entp \tilde a\tilde b^T=(a\entp\tilde a)(b\entp\tilde b)^T$. Therefore, applying the Taylor expansion entrywise,
%\begin{equation}
%    \label{singlecolumn}
%e^{\entwise c\tilde c^T}
%=
%\sum_{k=0}^\infty\frac{(c\tilde c^T)^{\entwise k}}{k!}
%=
%\sum_{k=0}^\infty\frac{(c^{\entwise k})(\tilde c^{\entwise k})^T}{k!},
%\end{equation}
%where $c=c_i$, $\tilde c=\tilde c_i$ are column vectors. Apply \eqref{singlecolumn} to each factor of \eqref{columndecomp},
%\begin{align}
%\entp_{i=1}^de^{\entwise c_i\tilde c_i^T}
%&=
%\sum_{k_1,\ldots,k_d=0}^\infty\frac{(\entp_{i=1}^d c_i^{\entwise k_i})(\entp_{i=1}^d \tilde c_i^{\entwise k_i})^T}{\prod_{i=1}^dk_i!}.
%\\&=
%\sum_{k=0}^\infty
%\sum_{k_1+\ldots+k_d=k}^\infty\frac1{k!}\binom{k}{k_1,\ldots,k_d}(\entp_{i=1}^d c_i^{\entwise k_i})(\entp_{i=1}^d \tilde c_i^{\entwise k_i})^T\label{doublesum}
%%\\&=\sum_{k=0}^\infty Q_k,
%\end{align}
%Let $Q_k$ be the innermost sum of \eqref{doublesum}. We estimate the maximum over the entries,  
%\[\|Q_k\|\submax\le\frac{\|\v\|_\infty^{k}\|\w\|_\infty^k}{k!}\sum_{k_1+\ldots+k_d=k}^\infty\binom{k}{k_1,\ldots,k_d}=\frac{\|\v\|_\infty^k\|\w\|_\infty^kd^k}{k!}\]
%and apply the inequality $\|Q_k\|\le n\|Q_k\|\submax$.  
%\end{proof}
%
%\begin{lemma}\label{lem:eigbound}
%Let $\lambda_0\ge\lambda_1\ge\ldots$ be the absolute values of the eigenvalues of $(e^{v_i\cdot w_j})_{ij}$ and let $\mu=\|\v\|_\infty\|\w\|_\infty d$. Then $\lambda_0\le ne^\mu$, and for $\mu\le1/2$,
%\begin{equation}
%    \label{binomkd}
%    \lambda_L\le\frac{2n}{p!}\mu^p,\qquad L=\binom{p+d-1}{d},
%\end{equation}
%where the case $p=0$ of \eqref{binomkd} holds with the interpretation $L=\binom{d-1}{d}=0$, $\lambda_0\le ne^{1/2}\le 2n$.
%\end{lemma}
%\begin{proof}
%From the identity
%\[
%\binom{p+d-1}{d}=1+d+\binom{d+1}{d-1}+\cdots+\binom{p+d-2}{d-1},
%\]
%there are 
%\[1+d+\binom{d+1}{d-1}+\cdots+\binom{p+d-2}{d-1}\ge\rank Q_0+\cdots+\rank Q_{p-1}\]
%eigenvalues in front of $\lambda_L$ where $L=\binom{p+d-1}{d}$, and we have used \cref{lem:rank_one_terms}. By the min-max principle,
%\[\lambda_L\le\left\|\sum_{k=p}^\infty Q_k\right\|\le n\sum_{k=p}^\infty\frac{\mu ^k}{k!}=\frac{n}{p!}\sum_{k=p}^\infty\mu^k=\frac{n}{p!}\frac{\mu^p}{1-\mu}\le \frac{2n}{p!}\mu^p.\]
%\end{proof}
%
%\propdetbound*
%\begin{proof}
%Let $\v=\w$. By \cref{lem:eigbound} and the assumptions on $p$ we have $\lambda_{\lfloor n/2\rfloor}\le\frac{2n}{p!}\mu^p\le\frac{\mu^p}{2n}$ and $\lambda_0\le 2n$ where $\mu=d\|\w\|_\infty^2$, so it follows that 
%$|\det((e^{w_i\cdot w_j})_{ij})|\le\lambda_0^{n/2}\lambda_{n/2}^{n/2}\le(\mu^p)^{n/2}=(\|\w\|_\infty\sqrt d)^{pn}=(\frac12\frac{\|\w\|_\infty}{\gamma})^{pn}$. This holds when $d\|\w\|_\infty^2=\mu\le1/4$, i.e., when $\|\w\|_\infty^2\le\gamma^2$.
%%=(\nu/2)^{pn}$. Set $\v=\w$.
%\end{proof}
%
%
%
%
%
%
%
%
%%%%%%%%%%%%%%%%%%%%%%%%%%%%%%%%%%%%%%%%%%%%%%%%%%%%%%%%%%%%%%%%%%%%%%%%%%%%%%%%%%%%%%%%%%%%%%%%%%%%%







%\begin{restatable}{lemma}{lemmalpbound}\label{lem:lpbound}
%Let $\lp=\LP\activation{t}$ be the low-pass at threshold $t=(2\sqrt d\|w\|_\infty)^{-1}$. 
%If $w$ is typical then\showfornips{\footnote{This originally said $t=\Theta(\sqrt{n/\log n})$ when it should say $t=\Omega(\sqrt{n/\log n})$ (and the lower bound is what we need). Moreover, $t=\tilde\Theta(\sqrt n)$ is true (\cref{sec:errors}).}} $t=\Omega(\sqrt{n/\log n})$ and $\XnormN{\AS\lp_w}=O(2^{-\Omega(n^{1+1/d})})$.
%\end{restatable}
%%\lemmalpbound*
%\begin{proof}
%The lower bound on $t$ follows directly from its definition and the definition of $w$ being typical.
%
%Bound $\DttN{w}$ as in \eqref{Dttlowbound} and write $|\isoF\activation(\theta)|=O(|\theta|^{-r}+1)$. The triangle inequality yields $\XnormN{\LP\activation{t}}=O(2^{-pn/2}\int_{-t}^t(|\theta|/t)^{pn/2}(|\theta|^{-r}+1)d\theta)=O(2^{-pn/2}t(t^{-r}+1))$ where $p=\Omega(n^{1/d})$. Here we have cancelled the pole $|\theta|^{-r}$ by writing $(|\theta|/t)^{pn/2}|\theta|^{-r}\le(|\theta|/t)^r|\theta|^{-r}=t^{-r}$ so that the integrand is bounded by $t^{-r}+1$.
%\end{proof}
%

