%
\documentclass[preprint]{elsarticle}

%%%%%%%%%%%%%%%%%%%%%%%%%%%%%%%%%%%%%%%%%%%%%%%%%%
% set to \iftrue before submission
%%%%%%%%%%%%%%%%%%%%%%%%%%%%%%%%%%%%%%%%%%%%%%%%%%
%\newcommand\ifhidecomments\iffalse
\newcommand\ifhidecomments\iftrue

\newcommand\paperversion{jcp}
% \usepackage[latin1]{inputenc}
\usepackage[british]{babel}
\usepackage[all]{xy}
\usepackage{amscd}
\usepackage{amssymb}
\usepackage{amsthm}
\usepackage{enumitem}
\usepackage{mathrsfs,bbm}
\usepackage{xcolor,graphicx}
\usepackage{graphics}
\usepackage{soul}
\usepackage{comment}
\usepackage[all]{xy}
\usepackage{amscd}
\usepackage{amssymb,amsmath,latexsym}
\usepackage{amsthm}
\usepackage{enumitem}
\usepackage{mathrsfs,bbm}
\usepackage{dsfont}
\usepackage{tikz-cd}
\usepackage[T1]{fontenc}
\usepackage[utf8]{inputenc}  
 %
%%%%%%%%%%%%%%%%%%%%%%%%%%%%%%%%%%
%pagestyle
%%%%%%%%%%%%%%%%%%%%%%%%%%%%%%%%%%
%\pagestyle{plain}
\textwidth=430pt
\headsep=.7cm
\evensidemargin=15pt
\oddsidemargin=15pt
\leftmargin=0cm
\rightmargin=0cm
%%
%%%%%%%%%%%%%%%%%%%%%%%
\newcommand*\fixitem {\item[]%
  \refstepcounter{enumi}\hskip-\leftmargin\labelenumi\hskip\labelsep}
\newtheorem*{mainthm}{Main Theorem}
\newtheorem*{mainthm1}{Theorem}
\newtheorem*{maincor}{Corollary}
\usepackage[colorlinks=true]{hyperref}
\DeclareMathOperator{\Forall}{\forall}
\DeclareMathOperator{\Exists}{\exists}
\DeclareMathOperator{\ord}{ord}
\newcommand{\phiD}{\varphi_D}
\newcommand{\phiDI}{\varphi_{\mathbf{D}_I}}
\newcommand{\phiDIj}{\varphi_{\mathbf{D}_I (j)}}
\newcommand{\phiH}{\varphi_H}
\newcommand{\phiTimes}{\phiD \otimes \phiH}
\newcommand{\phiTimesDI}{\varphi_{\mathbf{D}_I} \otimes \phiH}
\newcommand{\R}{\mathscr{A}}
\newcommand{\X}{\mathscr{X}}
\newcommand{\Xf}{\mathscr{X}_{(k_0 ,i)}[r_0]}
\newcommand{\Xfr}{\mathscr{X}_{(k_0,i)}[r]}
\newcommand{\hotimes}{\widehat{\otimes}}
\newcommand{\C}{\mathbb{C}_p}
\newcommand{\V}{\mathscr{V}}
\newcommand{\B}{\mathscr{B}}
\newcommand{\dualD}{\mathfrak{D}}
\newcommand{\Dg}{\mathbf{D}}
\newcommand{\DD}{\mathcal{D}^0}
\newcommand{\DDg}{\mathcal{D}}
\newcommand{\DV}{\mathcal{D}}
\newcommand{\W}{\mathscr{W}_N}
\newcommand{\Ao}{\mathbf{A}^\circ}
\newcommand{\AoK}{\mathbf{A}^\circ_{\K}}
\newcommand{\AK}{\mathbf{A}_{/\K}}
\newcommand{\OOO}{\mathscr{A}^\circ}
\newcommand{\K}{\mathcal{K}} 
\newcommand{\OK}{\mathcal{O}_{\K}}
\newcommand{\varprojlog}[1]{\underleftarrow{\log\!^{#1}}}
\newcommand{\T}{\mathscr{T}}
\newcommand{\TT}{\mathbf{T}}
\newcommand{\VV}{\mathbf{V}}
\newcommand{\HH}{\mathcal{H}}
\newcommand{\hh}{\mathcal{H}^+}
\newcommand{\HG}[2]{\mathcal{H}_{#1}(#2)}
\newcommand{\hhl}{\mathcal{H}^{+,[l]}}
\newcommand{\hhj}{\mathcal{H}^{+,[j]}}
\newcommand{\hhjj}{\mathcal{H}^{+,[l,l']}}
\newcommand{\GS}{G_{\mathbb{Q},S}}
\newcommand{\Rf}{R_{(k_0 ,i)}[r_0]}
\newcommand{\Rfr}{R_{(k_0 ,i)}[r]}
\newcommand{\parT}{\langle T\rangle}
\newcommand{\Zf}{Z_{(k_0 ,i)}[r_0]}
\newcommand{\Zfr}{\mathscr{Z}_{(k_0 ,i)}[r]}
\newcommand{\ZFf}{\mathscr{Z}_{(k_0 ,i)}[r_0]}
\newcommand{\ZFfr}{\mathscr{Z}_{(k_0 ,i)}[r]}
\newcommand{\ZF}{\mathscr{Z}}
\begin{document}


\title{
Anti-symmetric Barron functions and their approximation with sums of determinants
}

\author[1,2]{Nilin Abrahamsen\corref{cor1}}
\ead{nilin@berkeley.edu}

\author[1,3]{Lin Lin}
\ead{linlin@math.berkeley.edu}

\cortext[cor1]{Corresponding author}

\address[1]{
Department of Mathematics,
University of California, Berkeley, CA 94720 USA
}
\address[2]{
The Simons Institute for the Theory of Computing,
Berkeley, CA 94720 USA
}
\address[3]{
Applied Mathematics and Computational Research Division, Lawrence Berkeley National Laboratory, Berkeley, CA 94720, USA
}





\begin{abstract}


Over the past few years, there has been a significant amount of research focused on studying the ReLU activation function, with the aim of achieving neural network convergence through over-parametrization. However, recent developments in the field of Large Language Models (LLMs) have sparked interest in the use of exponential activation functions, specifically in the attention mechanism.

Mathematically, we define the neural function $F: \R^{d \times m} \times  \mathbb{R}^d \rightarrow \mathbb{R}$ using an exponential activation function. Given a set of data points with labels $\{(x_1, y_1), (x_2, y_2), \dots, (x_n, y_n)\} \subset \mathbb{R}^d \times \mathbb{R}$ where $n$ denotes the number of the data. Here $F(W(t),x)$ can be expressed as $F(W(t),x) := \sum_{r=1}^m a_r \exp(\langle w_r, x \rangle)$, where $m$ represents the number of neurons, and $w_r(t)$ are weights at time $t$. It's standard in literature that $a_r$ are the fixed weights and it's never changed during the training. We initialize the weights $W(0) \in \mathbb{R}^{d \times m}$ with random Gaussian distributions, such that $w_r(0) \sim \mathcal{N}(0, I_d)$ and initialize $a_r$ from random sign distribution for each $r \in [m]$.

Using the gradient descent algorithm, we can find a weight $W(T)$ such that $\| F(W(T), X) - y \|_2 \leq \epsilon$ holds with probability $1-\delta$, where $\epsilon \in (0,0.1)$ and $m = \Omega(n^{2+o(1)}\log(n/\delta))$. To optimize the over-parametrization bound $m$, we employ several tight analysis techniques from previous studies [Song and Yang arXiv 2019, Munteanu, Omlor, Song and Woodruff ICML 2022]. 

 

\end{abstract}


\maketitle

\section{Introduction}
%
To simulate a physical system it is essential to construct a model which respects the \emph{symmetries} of the real-world problem. A prominent example is when a function is defined on \emph{sets} of points \cite{zaheer_deep_2017,santoro_simple_2017,yarotsky_universal_2022}, in which case the function can be viewed as a \emph{permutation-invariant} function of an input vector. A symmetry need not mean that a function is invariant to transformations of its input; more generally it can map transformations of the input to transformations of the output through a group homomorphism. \emph{Equivariance} is one such example where a transformation of the input gives rise to the same transformation on the output. Another such symmetry is \emph{anti-symmetry} where a permutation of the input vector multiplies the output by the \emph{sign} of the permutation.

Accurate modeling of fermionic systems is one of the most challenging and interesting problems in science. For example, the solution of the Schr\"odinger equation underlies all chemical properties of a given atomic system. 
Due to the Pauli exclusion principle, the fermionic wavefunction is anti-symmetric with respect to particle exchange. 
When the number of fermions grows, effective parametrization of such  wavefunctions can become increasingly difficult for many systems of interest. 
Anti-symmetric functions also arise in other contexts in machine learning such as determinantal point processes \cite{kulesza_determinantal_2012} where they are used to ensure diverse samples. 




In the past decade there has been an explosive growth of techniques using neural networks (NN) as universal function approximators. The practical applicability of NNs is brought about by new software tools, hardware optimizations, as well as improved algorithms. NN approximators also significantly broaden the parameterization class for anti-symmetric functions in quantum physics~\cite{LuoClark2019,HanZhangE2019,HermannSchaetzleNoe2020,PfauSpencerMatthewsEtAl2020,StokesMorenoPnevmatikakisEtAl2020,LinGoldshlagerLin_vmcnet}. The combination of NN with variational Monte Carlo (VMC) methods provides a new path towards a low-scaling, systematically improvable method to approach the exact solution.


Despite recent progresses it is unclear how to construct a universal NN representation of anti-symmetric functions that does not obviously suffer from the curse of the dimensionality~\cite{LuoClark2019,PfauSpencerMatthewsEtAl2020,Hutter2020,HanLiLinEtAl2019}.
%It is  worth noting that 
In the absence of symmetry constraints, very simple NN structures such as an NN with one hidden layer (sometimes also referred to as a ``two-layer'' NN) is already a universal function approximator~\cite{Cybenko1989,HornikStinchcombeWhite1989,Barron1993}.
Therefore in principle, explicitly antisymmetrizing a NN with one hidden layer can parameterize  universal anti-symmetric functions. 
One obvious drawback of this strategy is that the computational cost of the antisymmetrization step still increases factorially with respect to the system size. 
Nonetheless, such an explicitly anti-symmetrized NN structure has been recently studied in VMC calculations, which can yield effectively the exact ground state energy for small atoms and molecules~\cite{LinGoldshlagerLin_vmcnet}.

Conversely, determinant-based NN constructions of anti-symmetric functions can be evaluated efficiently. But their expressive power is unclear except in the setting of a combinatorially large number of determinants spanning the entire anti-symmetric subspace. It is therefore prudent to know if the efficient determinant-based constructions are able to capture the a priori intractable class of explicitly anti-symmetrized neural networks. 

\subsection{Contribution}
The \emph{Barron space}, defined in \cite{e_barron_2022} based on the seminal work of Barron \cite{Barron1993}, characterizes functions that can be approximated by an infinite neural network with one hidden layer. We consider the subspace of antisymmetric functions in the Barron space and prove (\cref{mainthm}):
\begin{enumerate}

\item A function in the anti-symmetric Barron space can be efficiently approximated using a sum of determinants.

\item The theoretical error bound factorially improves the error estimate in the standard Barron space.

\end{enumerate}
The Fourier transform is central to our analysis. 
This is because anti-symmetrizing a complex-valued plane wave gives rise to a determinant (called a Slater determinant). Each plane wave can the viewed a single hidden neuron with an exponential activation function 


\subsection{Background and related works}


Consider a system of $n$ \emph{indistinguishable particles} in a $d$-dimensional space $\Omega\subset\RR^d$ ($d=1,2,3$), and let $N=nd$. The $n$-particle wave function is defined on inputs $\x=(x_1,\ldots,x_n)\in\Omega^n\subset\RR^N$ where each $x_i$ is in $\Omega\subset\RR^d$.
Indistinguishability means that  $\psi(x_1,\ldots,x_n)$ satisfies permutation symmetry of the norm $|\psi(x_1,\ldots,x_n)|$ under interchange of the $n$ inputs $x_i\in\Omega$. \emph{Fermions} are indistinguishable particles which satisfy the \emph{Pauli exclusion principle} and correspond to an \emph{anti-symmetric} wave function $\psi$. Anti-symmetry means that for a permutation $\pi\in S_n$, whose sign we denote by $(-1)^\pi$,
\[\pi(\psi)=(-1)^\pi\psi,\]where we have defined $\pi(\psi):\RR^{nd}\mapsto\CC$ by $\pi(\psi)(\x):=\psi(x_{\pi^{-1}(1)},\ldots,x_{\pi^{-1}(n)})$ for $x\in\RR^{nd}$. Let $\ASF_n$ denote the set of anti-symmetric functions $(\RR^d)^n\to\CC$.

In the machine learning literature there is a rich body of works related to permutation-invariant data, i.e., when the input data is a set \cite{zaheer_deep_2017,santoro_simple_2017,yarotsky_universal_2022,zweig_functional_nodate}. 
A widely used class of Ansatzes for anti-symmetric functions takes the form of a \textit{sum of Slater determinants}.  A Slater determinant, denoted $\phi_1\wedge\cdots\wedge\phi_n$, is constructed through $n$ \emph{orbitals}, i.e., functions $\phi_1,\ldots,\phi_n:\RR^d\to\CC$. The Slater determinant is the function defined by $(\phi_1\wedge\cdots\wedge \phi_n)(x_1,\ldots,x_n)=\det[E(\x)]$, where $(E(\x))_{ij}=\frac1{\sqrt{n!}}\phi_j(x_i),i,j=1,\ldots,n$. 

The representation of anti-symmetric functions is extensively studied in physics, but the literature on anti-symmetrized neural networks is sparse. 
Slater determinants can span a dense subset of the anti-symmetric space but the representation is very inefficient.  
Indeed, even in the case of a finite single-particle state space $|\Omega|=O(n)$ we would require $\binom{|\Omega|}{n}$ Slater determinants to span the anti-symmetric space. 
\cite{zweig_towards_2022} finds certain anti-symmetric functions that cannot be efficiently approximated using a simple sum of Slater determinants, but can be effectively expressed using a more complex Ansatz called the Slater-Jastrow form.



The FermiNet \cite{PfauSpencerMatthewsEtAl2020} and PauliNet \cite{HermannSchaetzleNoe2020} Ansatz have significantly expanded the representation power of the sum of determinants, by composing the orbitals with an equivariant mapping, and by parameterizing both using neural networks. The resulting structure can be expressed as a sum of generalized determinants
\begin{equation}
\psi(\x)=\sum_{k=1}^m \det(E_k(\x)),
\label{eq:composite}
\end{equation}
where $E_k:\RR^{nd}\to\RR^{n\times n}$ is equivariant, meaning that $E_k(\pi\x)=\pi(E_k(\x))$ and the permutation $\pi$ only exchanges rows of $E_k$. 

While the representation power of the Ansatz of the form \eqref{eq:composite} remains unclear, \cite{PfauSpencerMatthewsEtAl2020} provides an argument that with a sufficiently general mapping $E$, it is sufficient to choose $m=1$ to represent \textit{any} anti-symmetric function $\psi$. We recall their argument below.

%\noindent\textbf{\cite[Appendix B]{PfauSpencerMatthewsEtAl2020}: }
\noindent{Proof of universality from \cite{PfauSpencerMatthewsEtAl2020}:}
\textit{Introduce an ordering $\le$ on vectors in $\RR^d$, for example a dictionary ordering on the $d$ coordinates. Given $\x\in\RR^{nd}$ let $\pi$ be the permutation such that $\pi^{-1}\x$ is sorted. That is, $\x=(\tilde x_{\pi(1)},\ldots,\tilde x_{\pi(n)})$ for some sorted $\tilde x_1\le\ldots\le\tilde x_n$. Then for any anti-symmetric function $\psi$, it is sufficient to choose $E(\x)=\pi\tilde\y$ where $\tilde\y=\diag((-1)^\pi\psi(\x),1,\ldots,1)$.}

The argument above has the drawback that the mapping $E$ is highly discontinuous. In this work we therefore aim to approximate a more restricted class of anti-symmetric functions with an Ansatz which is continuous with respect to $\x$ and obtain a quantizative error bound relative to a known complexity measure. 







\subsection{Setup}
Based on the seminal work of Barron \cite{Barron1993}, 
the \emph{Barron space} is defined in \cite{e_barron_2022} as those functions which can be approximated by a continuous generalization of neural networks with one hidden layer. 

\begin{definition}[Barron space and norm \cite{e_barron_2022}]
\label{def:Barronspace}
The \emph{Barron space} $\Bsp$ is the set of functions of the form
    \begin{equation}\label{Bspdef}f_\rhoB(\x)=\int a\relu(\w\cdot \x+b)d\rhoB(a,b,\w),\end{equation}
    where $\relu(x)=\max\{0,x\}$ is the ReLU activation function and $\rhoB$ is a finite measure. 
 \cite{e_barron_2022} defines \emph{Barron norm} of $f$ as
    \begin{align}\label{eq:Barronfn}
    \Bnormti{f}&=\inf\{\tilde\varphi(\rho)\:|\:f_\rho=f\},\text{ where}\\
    \tilde\varphi(\rho)&=\int|a|(\|\w\|_1+|b|)d\rhoB(a,b,\w).
    \end{align}
    We further define the {translation-invariant} Barron norm $\Bnorm{f}\le\Bnormti{f}$ by
    \begin{align}
    \Bnorm{f}&=\inf\{\varphi(\rho)\:|\:f_\rho=f\},\text{ where}\\
    \varphi(\rho)&=\int|a|\|\w\|_1d\rho(a,b,\w).\label{varphidef}
    \end{align}
\end{definition}
We call $\rho$ a \emph{Barron measure} for $f$ if $f_\rho=f$.
We state our upper bound in terms of the translation-invariant Barron norm $\Bnorm{f}$ which implies the same bound relative to the larger norm $\Bnormti{f}$. 


\section{Main result}
\label{sec:results}
We state our approximation result in terms of an upper bound on the error in the norm on $L^2(\Omega^n,\xdist^{\otimes n})$ where $\xdist=\cN(0,I)$ is a standard Gaussian envelope function. The interpretation is that the actual wave function is a normalized function $\Psi\in L^2(\Omega^n)$ which we represent as $\Psi(\x)=\sqrt\Xdist(\x)\psi(\x)$ since $\Psi$ is localized. Writing $\tilde\Psi(\x)=\sqrt\Xdist(\x)\tilde\psi(x)$ we then have that $\Xnorm{\tilde\psi-\psi}=\|\tilde\Psi-\Psi\|$ is the standard $L^2$-distance between the wave functions $\tilde\Psi$ and $\Psi$.


%Let $\psi$ be an anti-symmetric function and $\tilde\psi$ an anti-symmetric approximation to $\psi$. 
\begin{theorem}
\label{mainthm}
    Let $\psi$ be an antisymmetric function which belongs to the Barron space and has translation-invariant Barron norm $\Bnorm\psi$ (\cref{def:Barronspace}). 
Then for each $m\in\NN$ there exists a linear combination $\psi_m=\frac1m\sum_{k=1}^m \mu_k\aa_{\w_k}$ of $m$ Slater determinants of the form $\aa_{\w}=e_{w_1}\wedge\cdots\wedge e_{w_n}$ with $e_w(x)=e^{\ii w\cdot x}$ such that 
    \begin{equation}
    \Xnorm{\psi_m-\psi}\le\frac{\Bnorm{\psi}}{\sqrt{n!}}\Big(\frac{C}{\sqrt m}+2^{-\Omega(n)}\Big).
    \end{equation}
     Here, $C=2\sqrt{d}/\pi$. In particular $\psi_m$ is of the form \cref{eq:composite} with $m$ determinants. 
\end{theorem}

%\subsubsection{Examples}
It was previously known that a Barron function $f$ can be approximated by \emph{finite} neural networks with one hidden layer of $m$ neurons up to error $\Bnorm{f}/\sqrt m$ \cite{Barron1993,e_barron_2022}. Our determinant-based approximation in \cref{mainthm} improves factorially on this estimate by a factor $1/\sqrt{n!}$ in the anti-symmetric setting.
This illustrates that the approximation is highly inefficient if the anti-symmetry condition is not explicitly built into the Ansatz.


\cref{mainthm} motivates the following definition:



\begin{definition}[Anti-symmetric Barron space and norm]\label{def:Anorm}
    The anti-symmetric Barron space is the subspace of the Barron space consisting of anti-symmetric functions, that is,
    \begin{equation}
    \ABsp=\Bsp\cap\ASF.
    \end{equation}
    For $\psi\in\ABsp$ we define its \emph{anti-symmetric Barron norm} as
    \begin{equation}\label{eq:Anorm}
    \ABnorm{\psi}=\frac{\Bnorm{\psi}}{\sqrt{n!}}.
    \end{equation}
\end{definition}


We can then restate our result as follows:
\begin{corollary}
\label{maincor1}
For any $\psi\in\ABsp$ and each $m\in\NN$ there exists a linear combination $\psi_m=\frac1m\sum_{k=1}^m \mu_k\aa_{\w_k}$ of $m$ Slater determinants $\aa_{\w}$ such that 
    \begin{equation}
    \Xnorm{\psi_m-\psi}\le\ABnorm{\psi}\Big(\frac{C}{\sqrt m}+2^{-\Omega(n)}\Big),
    \end{equation}
    where $\ABnorm{\psi}$ is given by \cref{eq:Anorm}.
\end{corollary}


\citet[Section IX]{Barron1993} provided a number of examples of Barron functions. 
Some care must be taken when restricting these to the anti-symmetric case. For example, any radial function $f(\x)=g(|\x|)$, or any function $f(\x)$ that is symmetric with respect to two of its coordinates vanishes after anti-symmetrization. This issue can be overcome by applying a translation by some vector $(x_1,\ldots,x_n)\in\RR^{nd}$ with distinct components $x_i\neq x_j$ before anti-symmetrizing. %Less obviously, anti-symmetrization of low-order polynomials also vanishes (\cref{lem:polynomial}). 
%However, we may compose a Barron function $f(\x)$ with an anisotropic linear transformation $\x\mapsto B\x$, and the resulting function $\AS f(B\x)$ may not vanish. 
More general anti-symmetric Barron functions can be constructed from anisotropic ridge functions $f(\x)=g(\a\cdot \x)$, anisotropic radial functions $f(\x)=g(|A\x|)$, and anisotropic integral representations $f(\x)=\int K(\a\cdot \x+b) d\rhoB(\a,b)$, to name a few.


A Slater determinant $\psi=\phi_1\wedge\cdots\wedge\phi_n$ can be written as $\psi=\sqrt{n!}\AP(\phi_1\otimes\cdots\otimes\phi_n)$ where $\AP$ is the projection onto the subspace of anti-symmetric functions. It therefore follows that \[\ABnorm{\phi_1\wedge\cdots\wedge\phi_n}=\Bnorm{\phi_1\otimes\cdots\otimes\phi_n}.\]


\section{Proof sketch}
We now outline the proof of \cref{mainthm}.
We can state the property of being antisymmetric as $\psi=\AP\psi$ where $\AP$ is the projection onto $\ASF$. Given a basis expansion $\psi=\int f_w d\mu(w)$ of $\psi\in\ASF$ we will take the projection inside the integral and write $\psi=\int\AP(f_w) d\mu(w)$. We therefore need a basis expansion such that:
\begin{enumerate}
\item\label{it:easyP} We can analyze anti-symmetric projection of $f_w$ and estimate the magnitude of $\AP(f_w)$. 
\item\label{it:B_to_exp} The expansion $\rho$ of a Barron function $\psi_\rho$ (\cref{def:Barronspace}) gives rise to a basis expansion $\mu(w)$ into functions $f_w$.
\end{enumerate}
We will show that the Fourier transform provides such an expansion.

The Fourier basis functions are complex plane waves $\x\mapsto e^{i\w\cdot\x}$.
To analyze their anti-symmetrization (\cref{it:easyP}), observe that they factor into a product:
\begin{equation}
\ee_{\w}(\x):=\e(\w\cdot\x)=e^{\i\w\cdot\x}=\prod_{j=1}^ne^{\i w_j\cdot x_j}.
\end{equation}
where $\i\in\CC$ is the complex unit. %As remarked in \cref{productslater} its anti-symmetrization can then be computed by a determinant formula:
The projection of $\ee_\w$ onto $\ASF$ can then be computed as a Slater determinant. Specifically, 
\begin{align}
\label{eq:Pew}
\AP\ee_{\w}=\AP(\otimes_{j=1}^n e_{w_j})=\frac1{\sqrt{n!}}\aa_\w,
\end{align}
where $\aa_\w=e_{w_1}\wedge\cdots\wedge e_{w_n}$ is as in \cref{mainthm}.
To obtain \cref{it:B_to_exp} we use the Barron expansion \cref{eq:Barronfn} of a function $\psi\in\Bsp$ to obtain its Fourier decomposition. Concretely, \cref{eq:Barronfn} given a decomposition of $\psi$ into ridge functions 
\begin{equation}\label{eq:ridgefn}
\sigmaa_{\w,b}(\x)=\sigma(\w\cdot\x+b).
\end{equation}
We then apply the one-dimensional Fourier decomposition of ReLU to decompose the ridge function into Fourier basis functions on $\RR^{nd}$.


The remainder of this sketch is a formal derivation which will require additional work in the following sections to be made rigorous. Assume that the activation function satisfies the Fourier inversion formula:
\begin{equation}
\label{formal0}
\sigma(y)=\frac1{\sqrt{2\pi}}\int \hat\sigma(\theta)e^{i\theta y}d\theta.
\end{equation}
This is not immediately well-defined in the case of ReLU because $\hat\sigma$ is not absolutely integrable. Substituting $\w\cdot\x+b$ into \cref{formal0} yields a decomposition of the ridge function $\sigmaa_{\w,b}$ on $\RR^{nd}$. By \cref{eq:Pew}, projecting this ridge function onto the anti-symmetric subspace yields
\begin{equation}
\label{ASridge}
\AP\sigmaa_{\w,b}=\frac1{\sqrt{2\pi n!}}\int\hat\sigma(\theta)e^{ib\theta}\aa_{\theta\w}d\theta.
\end{equation}
The anti-symmetric Barron function $\psi$ is of the form $\psi=f_\rho$ for some measure $\rho$. We antisymmetrize the integral representation of $f_\rho$ and expand the anti-symmetrized ridge function as in \cref{ASridge} to obtain
\begin{align}
f_\rho=\AP f_\rho
&=\int a\,\AP(\sigmaa_{\w,b})\,d\rho(a,b,\w)\\
&=\frac1{\sqrt{2\pi n!}}\iint a\hat\sigma(\theta)e^{ib\theta}\aa_{\theta\w}d\rho(a,b,\w),
\label{bigint}
\end{align}
which is an expansion as an integral over the basis functions $\aa_{\theta\w}$ against a complex measure $\mu$. By a standard sampling argument this can be approximated up to error $\|\mu\|/\sqrt m$ with a finite sum of $m$ terms, where $\|\mu\|$ is the total variation of the measure. 



\begin{figure}
    \centering
    \includegraphics[width=.9\textwidth]{figures/hp.pdf}
    \caption{High-passed version $\HP\relu\gamma$ of ReLU (\cref{uvrelu}) for thresholds $\gamma=1\text{ (opaque)}$ and $\gamma=\frac14,\frac12,2$ (faint blue, with $\gamma=1/4$ being the large and slowly oscillating curve and $\gamma=2$ being the fast-oscillating curve). %\LL{ which line corresponds to which case? Also I am not sure  the meaning of the boldfont for 1}  
    We use anti-symmetrized ridge functions $\AP(\tilde\sigmaa_{\w,b})$ defined with the activation function $\tilde\sigma=\HP\relu\gamma$ to approximate anti-symmetrized ridge functions $\AP(\sigmaa_{\w,b})$ defined with the ReLU activation. %The same is not true for $\tilde\sigmaa_{\w,b}$ vs. $\sigmaa_{\w,b}$ without the anti-symmetrization. %\LL{ I am not sure what this sentence means} 
    }\label{fig:hp}
\end{figure}

The formal identities \cref{formal0}--\cref{bigint} do not directly apply for ReLU due to a divergence at $\theta\to0$. To overcome this subtlety we decompose the ReLU activation function into a \emph{high-passed} or \emph{ultraviolet} part and a \emph{low-passed} or \emph{infrared} remainder. 
Specifically, the ultraviolet part of the ReLU activation function is
\begin{equation}\label{uvrelu}
\HP\activation{\gamma}(y)=
|y|/2-\frac{\cos(\gamma y)}{\pi \gamma }-\frac{y\opn{Si}(\gamma y)}\pi,\end{equation}
where $\opn{Si}(y)=\int_0^y\frac{\sin s}sds$. 

We prove asymptotic bounds on $\Xnorm{\aa_\w}$ for small $\w$ which show that the contributions from the infrared remainder are exponentially small after anti-symmetrization. We can therefore truncate away the infra-red part to avoid the divergence at small $\theta$ at the cost of an exponentially small error term. This truncation is equivalent with replacing the ReLU activation by its high-passed part (\cref{fig:hp}). We emphasize that the magnitude of the discarded infrared remainder is not small as a one-dimensional function. Rather, its smoothness means as a multidimensional ridge function it is near-orthogonal to the antisymmetric subspace.




\section{A renormalized anti-symmetrization operator}
It is natural to renormalize the anti-symmetric projection to
\begin{equation}\label{ASdef}\AS f=\sqrt{n!}\:\AP f=\frac1{\sqrt{n!}}\sum_{\pi\in S_n}(-1)^\pi\pi(f).\end{equation}
In particular, if
$f=\phi_1\otimes\cdots\otimes \phi_n$ is a tensor product of single-particle orbitals, then $\psi(x)=(\AS \phi)(x)$ is the Slater determinant $\phi_1\wedge\cdots\wedge \phi_n$. The normalization in Eq. \eqref{ASdef} is such that if $\phi_i$ are orthonormal functions on $L^2(\Omega,\xdist)$ then $\psi=\AS f$ is normalized in $L^2(\Omega^{n},\xdist^{\otimes n})$. This follows from Pythagoras' theorem because orthogonality of $\psi_i$ implies that the $n!$ terms $\pi(f)$ in \cref{ASdef} are orthonormal.
\label{productslater}



With the renormalized antisymmetrization operator we have another equivalent definition of the antisymmetric Barron norm.
\begin{lemma}\label{ABequiv}
The \emph{anti-symmetric Barron space} is equal to $\ABsp=\AS\Bsp:=\{\AS f|f\in\Bsp\}$,
and the \emph{anti-symmetric Barron norm} of an anti-symmetric $\psi$ is
\begin{equation}\label{ABdef}
\begin{aligned}
\ABnorm{\psi}&=\inf\{\Bnorm{f}\:|\:\AS f=\psi\}.
%\\&=\inf\{\:\EE_{\rhoB}|a|(\|w\|_1+|b|)\:|\:\rhoB\text{ such that }\AS f_\rhoB=\psi\:\}.
\end{aligned}
\end{equation}
\end{lemma}
We include the straightforward derivation of \cref{ABequiv} in \ref{sec:otherproofs}.






\section{Generalized Fourier inversion formula}
\label{sec:proofs}


\label{divergentF}
Our proof of \cref{mainthm} uses the Fourier decomposition of $f_\rhoB$ which we characterize using the Fourier transform of the ReLU activation function. %We then give an explicit formula for the anti-symmetrization of the Fourier basis functions. 
Since ReLU is not integrable its Fourier transform is not defined as a convergent integral but rather in the sense of \emph{tempered distributions} \cite{reed_i_1981}. It this sense, ReLU has the Fourier transform
\begin{equation}\widehat\relu(\theta)=\frac{-1}{\sqrt{2\pi}\cdot\theta^2}+\sqrt{\tfrac\pi2}i\delta'(\theta).\end{equation}



We will not need the precise definition of $\hat\sigma$ but only that it satisfies the following, which we term the \emph{ultraviolet Fourier inversion formula}: For $t>0$,
\begin{equation}
\sigma(y)=\frac1{\sqrt{2\pi}}\int_{|\theta|>t}\hat\sigma(\theta)e^{i\theta y}d\theta
+p_t(y)+O(t g(y)),
\label{Fdef}
\end{equation}
where $p_t$ is a polynomial whose degree is bounded (uniformly in $t$), and $g$ is a non-negative function bounded by a polynomial. 
In \ref{invformula} we show that ReLU satisfies \cref{Fdef} with
\begin{equation}\label{Frelu}\widehat\relu(\theta)=\frac{-1}{\sqrt{2\pi}\cdot\theta^2},\qquad|\theta|>0.\end{equation}
and with remainders $p_\gamma(y)=y/2+\frac1{\pi\gamma}$, $g(y)=y^2$.



To state the generalized Fourier inversion formula more compactly, define the high-frequency part of an activation function $\sigma$ as follows: %\LL{ why change  $\HP{\sigma}{\gamma}$ to  $\HP{\sigma}{t}$?} 
\begin{definition}\label{def:hpdef}
For $\sigma:\RR\to\CC$ define its \emph{high-pass} or $\HP{\sigma}{\gamma}$ at threshold $\gamma>0$ by
\begin{equation}
\label{fourieractivation}
\HP{\sigma}{\gamma}(y)=\frac1{\sqrt{2\pi}}\int_{|\theta|>\gamma}\hat\sigma(\theta)e^{i\theta y}d\theta.
\end{equation}
We define the low-pass $\LP\sigma{\gamma}$ as the remainder $\sigma-\HP\sigma{\gamma}$. 
\end{definition}
Then the ultraviolet Fourier inversion formula \cref{Fdef} holds when the remainder $\sigma-\HP\sigma{\gamma}$ is of the form
\begin{equation}%\LP{\sigma}{\gamma}
\label{uvi}
\sigma-\HP\sigma{\gamma}=p_\gamma+O(\gamma g).\end{equation}
The high-pass of ReLU is \cref{uvrelu} as shown in (\ref{invformula}).

We will prove that we can replace $\sigma$ by its high-pass and that the error incurred becomes exponentially small after anti-symmetrization. 
%
As a first step towards this error bound, observe that for an anti-symmetric Barron function, the contribution from the term $p_\gamma$ in the Fourier inversion formula vanishes.
\begin{lemma}\label{lem:polynomial}
If $f:\RR^{nd}\to\CC$ is a polynomial of degree $\deg f\le n-2$, then $\AS f\equiv 0$. In particular $\AS\sigmaa_{\w,b}\equiv 0$ if $\sigma$ is an activation function which is a polynomial of degree $\deg\sigma\le n-2$.
\end{lemma}
\begin{proof}
By linearity it suffices to prove the claim when $f$ is a monomial $f(\x)=\prod_{i=1}^n x_i^{r_i}$ where $r_i\in\NN_0$. Since $\deg f=\sum_i r_i\le n-2$ there exists a pair $i\neq j$ such that $r_i,r_j=0$. Let $\pi_{ij}$ be the permutation which swaps $i$ and $j$. Then $f(\pi_{ij}(\x))=f(\x)$ because $f$ does not depend on $x_i,x_j$. But we also have $f(\pi_{ij}(\x))=-f(\x)$ by anti-symmetry, so $f(\x)=0$.
\end{proof}



We substitute $y=\w\cdot\x+b$ into the ultraviolet Fourier inversion formula \eqref{Fdef} to obtain a decomposition of ridge functions on $\RR^{nd}$
\begin{equation}\label{FdefRnd}
\begin{aligned}
\sigmaa_{\w,b}(\x)&=\frac1{\sqrt{2\pi}}\int_{|\theta|>\gamma}\hat\sigma(\theta)e^{i\theta b}e^{i\theta\w\cdot\x}d\theta
\\&+p_\gamma(\w\cdot\x)+O(\gamma g(\w\cdot\x)),
\end{aligned}
\end{equation}
We anti-symmetrize this ridge function and apply \cref{lem:polynomial} which yields that for $n\ge\deg p_\gamma+2$,
\begin{align}\label{FdefA}
\AS\sigmaa_{\w,b}&=\lim_{\epsilon\to0}\frac1{\sqrt{2\pi}}\int_{|\theta|>\epsilon}e^{ib\theta}\hat\sigma(\theta)\aa_{\theta\w}d\theta,
\end{align}
where convergence is in the $\Ltnd$-norm ($g(\w\cdot\x)$ is bounded in this norm because of the fast-decaying Gaussian envelope $\Xdist$). 

\section{Properties of the anti-symmetrized Fourier basis functions}
%Recall the definition of the anti-symmetrized Fourier basis functions from \cref{expandaa}:


\cref{fig:Dtt} illustrates that $\Xnorm{\aa_\w}$ is bounded by $1$ and vanishes for small $\w$ (\cref{fig:Dtt}). \cref{onebound} and \cref{prop:detbound} below capture this fact rigorously.
\begin{figure}[h]
    \centering
    \includegraphics[width=.9\textwidth]{figures/D.pdf}
    \caption{
    The function $\theta\mapsto\Xnorm{\aa_{\theta\w}}^2$ (for a randomly chosen $\w$ scaled to have $\|\w\|_\infty=1$). $n=100$, $d=3$.}\label{fig:Dtt}
\end{figure}
To analyze the behavior of $\Xnorm{\aa_w}$ we use that the overlap between two Slater determinants is the determinant of the \emph{overlap matrix} \cite{lowdin_quantum_1955}, meaning that 
\begin{equation}\label{AeAe}
    \bracket{\aa_\v}{\aa_\w}=\bracket{\wedge_i e_{v_i}}{\wedge_i e_{w_i}}=\det B^{(\v,\w)},
\end{equation}
where $B^{(\v,\w)}\in\RR^{n\times n}$ is given by
\begin{equation}
B^{(\v,\w)}_{ij}=\bracket{e_{v_i}}{e_{w_j}}.
\end{equation}
By \cref{AeAe} the problem of bounding the norms and overlaps of functions $\aa_\w$ corresponds to bounding the magnitude of a determinant. We begin with a simple uniform bound \cref{onebound} before proving the more technical $\w$-dependent bound (\cref{prop:detbound} below) which will lead to the exponentially small error term in \cref{mainthm}.

\begin{lemma}\label{onebound}
$\Xnorm{\aa_\w}\le1$ for all $\w\in\RR^{nd}$.
\end{lemma}
\begin{proof}
$B_{ij}^{(\w,\w)}$ is the Gram matrix of the $n$ vectors $e_{w_i}\in\Ltd$ so it is a positive semidefinite matrix. It then satisfies $\det(B^{(\w,\w)})\le\prod_i B_{ii}^{(\w,\w)}$ by Hadamard's theorem. So
\begin{align}\label{normsqB}
\Xnorm{\aa_\w}^2&=\det(B^{(\w,\w)})
\\&\le\prod_i B_{ii}^{(\w,\w)}=\prod_{i=1}\bracket{e_{w_i}}{e_{w_i}}^n=1.
\end{align}
\end{proof}


When the envelope is the standard Gaussian $\Xdistr=\cN(0,I_{nd})$ and $\v=\w$, the overlap matrix specializes to (letting $Y\sim\cN(0,1)$)
\begin{align}\label{GaussianD0}
B^{(\w,\w)}_{ij}
&=\EE[\overline{e^{iw_{i1}Y}}e^{iw_{j1}Y}]\cdots\EE[\overline{e^{iw_{id}Y}}e^{iw_{jd}Y}]
\\&=\EE[e^{i(w_{j1}-w_{i1})Y}]\cdots\EE[e^{i(w_{j1}-w_{i1})Y}]
\\&=e^{-\frac12\|w_j-w_i\|^2}.
\end{align}
We apply \cref{normsqB} and expand the square to obtain:
%so in particular,
\begin{equation}\label{GaussianDdiag}
\Xnorm{\aa_\w}^2=e^{-\|\w\|^2}\det\big((e^{w_i\cdot w_j})_{ij}\big).
\end{equation}
%

\section{Determinant bound}
To obtain an upper bound on \cref{GaussianDdiag} we decompose the matrix $(e^{w_i\cdot w_j})_{ij}$ into a sum $\sum_{k=0}^\infty Q_k$, bounding the ranks and operator norms of the terms $Q_k$. For $L=\sum_{k=1}^{p-1}\rank Q_k$ we can then bound the $L$-th eigenvalue as the tail sum $\sum_{k=p}^\infty\|Q_k\|$. Taking the product of the eigenvalues yields a bound on the determinant and therefore on the norms of anti-symmetrized plane waves $\aa_\w$.%(\cref{sec:det_upper_bound})



\begin{restatable}{lemma}{lemmalowranksum}
    \label{lem:rank_one_terms}
Let $\v=(v_1,\ldots,v_n)^T\in\RR^{n\times d}$ and $\w=(w_1,\ldots,w_n)^T\in\RR^{n\times d}$. Then $(e^{v_i\cdot w_j})_{ij}=\sum_{k=0}^\infty Q_k$ where
\begin{equation}\opn{rank}Q_k\le\binom{k+d-1}{d-1},\qquad\|Q_k\|\le \frac{n(\|v\|_\infty\|w\|_\infty d)^k}{k!}.\label{Qk}\end{equation}
\end{restatable}
%\lemmalowranksum*
\begin{proof}
Let $(c_1,\ldots,c_d)$ and $(\tilde c_1,\dots,\tilde c_d)$ be the columns of $v$ and $w$ and let $\entwise$ denote elementwise operations (exponentiation and product, respectively). Then,
\begin{equation}
    \label{columndecomp}
(e^{v_i\cdot w_j})_{ij}=e^{\entwise\sum_{i=1}^dc_i\tilde c_i^T}=\entp_{i=1}^d e^{\entwise c_i\tilde c_i^T}.
\end{equation}
We first consider each factor $e^{\entwise c_i\tilde c_i^T}$ separately. Elementwise multiplication of rank-one matrices given as outer products corresponds to elementwise multiplication of the vectors, $ab^T\entp \tilde a\tilde b^T=(a\entp\tilde a)(b\entp\tilde b)^T$. Therefore, applying the Taylor expansion entrywise,
\begin{equation}
    \label{singlecolumn}
e^{\entwise c\tilde c^T}
=
\sum_{k=0}^\infty\frac{(c\tilde c^T)^{\entwise k}}{k!}
=
\sum_{k=0}^\infty\frac{(c^{\entwise k})(\tilde c^{\entwise k})^T}{k!},
\end{equation}
where $c=c_i$, $\tilde c=\tilde c_i$ are column vectors. Apply \eqref{singlecolumn} to each factor of \eqref{columndecomp} and expand the sums,
\begin{align}
\entp_{i=1}^de^{\entwise c_i\tilde c_i^T}
&=
\sum_{k_1,\ldots,k_d=0}^\infty\frac{(\entp_{i=1}^d c_i^{\entwise k_i})(\entp_{i=1}^d \tilde c_i^{\entwise k_i})^T}{\prod_{i=1}^dk_i!}.
\\&=
\sum_{k=0}^\infty
\sum_{k_1+\ldots+k_d=k}^\infty\frac1{k!}\binom{k}{k_1,\ldots,k_d}(\entp_{i=1}^d c_i^{\entwise k_i})(\entp_{i=1}^d \tilde c_i^{\entwise k_i})^T\label{doublesum}
%\\&=\sum_{k=0}^\infty Q_k,
\end{align}
Let $Q_k$ be the innermost sum of \eqref{doublesum}. We estimate the maximum over the entries,  
\[\|Q_k\|\submax\le\frac{\|\v\|_\infty^{k}\|\w\|_\infty^k}{k!}\sum_{k_1+\ldots+k_d=k}^\infty\binom{k}{k_1,\ldots,k_d}=\frac{\|\v\|_\infty^k\|\w\|_\infty^kd^k}{k!}\]
and apply the inequality $\|Q_k\|\le n\|Q_k\|\submax$.  
\end{proof}

\begin{lemma}\label{lem:eigbound}
Let $\lambda_0\ge\lambda_1\ge\ldots$ be the absolute values of the eigenvalues of $(e^{v_i\cdot w_j})_{ij}$ and let $\mu=\|\v\|_\infty\|\w\|_\infty d$. Then $\lambda_0\le ne^\mu$, and for $\mu\le1/2$ and $p\in\NN$,
\begin{equation}
    \label{binomkd}
    \lambda_L\le\frac{2n}{p!}\mu^p,\qquad L=\binom{p+d-1}{d},
\end{equation}
where the case $p=0$ of \eqref{binomkd} holds with the interpretation $L=\binom{d-1}{d}=0$, $\lambda_0\le ne^{1/2}\le 2n$.
\end{lemma}
\begin{proof}
From the identity
\[
\binom{p+d-1}{d}=1+d+\binom{d+1}{d-1}+\cdots+\binom{p+d-2}{d-1},
\]
there are 
\[1+d+\binom{d+1}{d-1}+\cdots+\binom{p+d-2}{d-1}\ge\rank Q_0+\cdots+\rank Q_{p-1}\]
eigenvalues in front of $\lambda_L$ where $L=\binom{p+d-1}{d}$, and we have used \cref{lem:rank_one_terms}. By the min-max principle,
\[\lambda_L\le\left\|\sum_{k=p}^\infty Q_k\right\|\le n\sum_{k=p}^\infty\frac{\mu ^k}{k!}=\frac{n}{p!}\sum_{k=p}^\infty\mu^k=\frac{n}{p!}\frac{\mu^p}{1-\mu}\le \frac{2n}{p!}\mu^p.\]
\end{proof}

\begin{restatable}{proposition}{propdetbound}\label{prop:detbound}
Let $\gamma=\frac1{2\sqrt{d}}$ and let $p$ be any integer such that $\binom{p+d-1}{d}\le n/2$ and $p!\ge 4n^2$. Then,
\begin{equation}
\det((e^{w_i\cdot w_j})_{ij})\le \Big(\frac{\|\w\|_\infty}{2\gamma}\Big)^{pn}  \quad\text{ for }\quad  \|\w\|_\infty\le\gamma.
\end{equation}
\end{restatable}
%\propdetbound*
\begin{proof}
Let $\v=\w$. By \cref{lem:eigbound} and the assumptions on $p$ we have $\lambda_{\lfloor n/2\rfloor}\le\frac{2n}{p!}\mu^p\le\frac{\mu^p}{2n}$ and $\lambda_0\le 2n$ where $\mu=d\|\w\|_\infty^2$, so it follows that 
$|\det((e^{w_i\cdot w_j})_{ij})|\le\lambda_0^{n/2}\lambda_{n/2}^{n/2}\le(\mu^p)^{n/2}=(\|\w\|_\infty\sqrt d)^{pn}=(\frac12\frac{\|\w\|_\infty}{\gamma})^{pn}$. This holds when $d\|\w\|_\infty^2=\mu\le1/4$, i.e., when $\|\w\|_\infty^2\le\gamma^2$.
%=(\nu/2)^{pn}$. Set $\v=\w$.
\end{proof}





\cref{prop:detbound} suffices to give a fine-grained bound on the norms of the anti-symmetrized plane waves $\aa_\w$.
Let $p=\Theta(n^{1/d})$ and apply the bound to \cref{GaussianDdiag} to obtain:
\begin{equation}\label{Dttlowbound}
\Xnorm{\aa_{\w}}^2\le\Big(\frac{\|\w\|_\infty}{2\gamma}\Big)^{\Omega(n^{1+1/d})}
\end{equation}
for $\|\w\|_\infty\le\gamma$, where $\gamma=\frac1{2\sqrt d}$.



\section{Bound on the infra-red truncation error}
In the limit as the infra-red cutoff $\epsilon\to0$, \cref{FdefA} provides an expansion of an anti-symmetrized ridge function into functions $\aa_\w$. We need to bound the error when evaluating \cref{FdefA} with a finite infra-red truncation. We denote the anti-symmetric functions defined with or without such a truncation as follows:
\begin{definition}
Let $A_{\w,b}=\AS\sigmaa_{\w,b}$, and for $\gamma>0$, let
\begin{equation}
A_{\w,b;\gamma}=\frac1{\sqrt{2\pi}}\int_{|\theta|\ge\gamma}e^{ib\theta}\hat\sigma(\theta)\aa_{\theta\w}d\theta.
\label{At}
\end{equation}
To bound the error incurred from the infrared truncation we use the following triangle inequality:
\end{definition}
\begin{lemma}[Triangle inequality]
\begin{equation}
\Xnorm{A_{\w,b;\gamma}-A_{\w,b}}\vphantom{\frac11}
\le
\frac1{\sqrt{2\pi}}\int_{0<|\theta|<\gamma}|\hat\sigma(\theta)|\Xnorm{\aa_{\theta\w}}d\theta,
\end{equation}
where the right-hand side is interpreted as the limit \cref{limeps} below.
\end{lemma}
\begin{proof}
We use \cref{FdefA} to write $A_{\w,b}$ as a limit. Then,
\begin{align}\label{triangle}
&\Xnorm{A_{\w,b;\gamma}-A_{\w,b}}\vphantom{\frac11}
\\=&\lim_{\epsilon\to0}\Xnorm{A_{\w,b;\gamma}-A_{\w,b;\epsilon}}\vphantom{\frac11}
\\\le&\lim_{\epsilon\to0}\frac1{\sqrt{2\pi}}\int_{\epsilon<|\theta|<\gamma}|\hat\sigma(\theta)|\Xnorm{\aa_{\theta\w}}d\theta\label{limeps}
%\\=&\frac1{\sqrt{2\pi}}\int_{|\theta|<t}|\hat\sigma(\theta)|\Xnorm{\aa_{\theta\w}}d\theta.
\end{align}
\end{proof}
Combining the triangle inequality with the bound from \cref{prop:detbound} yields the following error bound, as shown in \ref{ests}.
\begin{corollary}\label{AAbound}
Suppose $\|\w\|_\infty=1$ and let $\gamma=\frac1{2\sqrt{d}}$. Then,
\begin{equation}
\Xnorm{A_{\w,b;\gamma}-A_{\w,b}}\le 2^{-\Omega(n^{1+1/d})}.
\end{equation}
\end{corollary}


\section{Expanding a Barron function}

%\subsection{Applying the bounds on $\alpha_\w$}


Given an anti-symmetric Barron function $\psi$, let $\rho$ be a Barron measure, i.e.,
\begin{equation}\label{fullexpand}
\psi=\AS f_\rho=\int a \: A_{\w,b} \: d\rho(a,b,\w).
\end{equation}
We say that $\rho$ is \emph{canonical} if $\|\w\|_\infty=1$ for all $(a,b,\w)$ in the support of $\rho$. As the next lemma shows we may assume without loss of generality that $\rho$ is canonical.

\begin{lemma}
Fix $p\in[1,\infty]$. In the definition of the Barron norm we may restrict to measures $\rho$ such that $\|\w\|_p=1$ for all $(a,b,\w)$ in the support of $\rhoB$. The resulting definition is equivalent with the original one.
In particular,
\begin{equation}
\ABnorm\psi=\inf\{\varphi(\rho)\:|\:\rho\text{ is canonical and }\AS f_\rho=\psi\}.
\end{equation}
\end{lemma}
\begin{proof}
Given $\rho$ such that $f_\rho=f$, define $\tilde\rho=h(\rho)=\rho\circ h$ where $h(a,b,\w)=(\tilde a,\tilde b,\tilde\w)$ where
\[\tilde a=\|\w\|_p a,\quad
\tilde b=b/\|\w\|_p,\quad
\tilde\w=\w/\|\w\|_p.\]
Then $\varphi(\tilde\rho)=\varphi(\rho)$, and $f_{\tilde\rho}=f_\rho$ due to the homogeneity of ReLU.
\end{proof}
For any canonical $\rho$ we define
\begin{equation}\label{truncexpand}
\psi_\gamma^{(\rho)}=\int a\:A_{\w,b;\gamma}\:d\rho(a,b,\w).
\end{equation}
We apply the usual triangle inequality to the integral over $\rho$ and then apply \cref{AAbound} to obtain
\begin{align}\label{prelimit}
\Xnorm{\psi_\gamma^{(\rho)}-\psi}
&\le\int |a|\Xnorm{A_{\w,b;\gamma}-A_{\w,b}} d\rho(a,b,\w)\\
&=2^{-\Omega(n^{1+1/d})}\int |a| d\rho(a,b,\w)\\
&=2^{-\Omega(n^{1+1/d})}\varphi(\rho),
\end{align}
where $\gamma=\frac1{2\sqrt{d}}$.
By expanding $A_{\w,b;\gamma}$ we can write $\psi_\gamma^{(\rho)}$ in the following form:
\begin{definition}\label{defcomplexmeasure}
Given a canonical Barron measure $\rho$ and threshold $\gamma>0$, define the complex measure
\begin{equation}
d\mu(\theta,a,b,\w)=\frac{-1_{|\theta|\ge\gamma}\:e^{ib\theta}}{{2\pi}\theta^2}\:d\theta\:\times\:a\:d\rho(a,b,\w)
\end{equation}
Then,
\begin{equation}\label{Adefasint}
\psi_{\gamma}^{(\rho)}=\iint \aa_{\theta\w} d\mu(\theta,a,b,\w).
\end{equation}
\end{definition}


\section{Proof of the main theorem}
Variants of the following fact are attributed to Maurey by Pisier \cite{pisier_remarques_1980,Barron1993} and widely used in the literature \cite{Barron1993,e_barron_2022}.
%\cref{convexcombi}
Its statement follows from lemma 1 (page 934) of \cite{Barron1993} when $\psi\neq0$ and is vacuously true when $\psi=0$.
\begin{lemma}[Maurey]
\label{convexcombi}
Let $\mathcal F$ be a subset of a Hilbert space with inner product $\bracket{f}{g}$. Suppose $\Xnorm{f}\le 1$ for all $f\in\mathcal F$, and let 
\begin{equation}
    \psi=\int f d\mu(f),
\end{equation}
where $\mu$ is a complex-valued measure on $\mathcal F$.
Then for each $m\in\NN$ there exists a complex linear combination $\psi_m=\frac1m\sum_{k=1}^ma_k f_k$ of $m$ elements of $\mathcal F$ such that
\[\|\psi_m-\psi\|\le\frac{\|\mu\|}{\sqrt{m}},\]
where $\|\mu\|=\int 1 d|\mu|$ is the total variation of the complex measure $\mu$.
\end{lemma}

Let $\mu$ be the complex measure from \cref{defcomplexmeasure}. Then the absolute value of $\mu$ is
\begin{equation}
d|\mu|(\theta,a,b,\w)=\frac{-1_{|\theta|\ge\gamma}}{{2\pi}\theta^2}\:d\theta\:\times\:|a|\:d\rho(a,b,\w),
\end{equation}
which is a product measure. So its total variation is
\begin{align}\label{TV}
\|\mu\|&=\iint 1d|\mu|(\theta,\w)\\
&=-\frac1{2\pi}\int_{|\theta|\ge\gamma}\frac{1}{\theta^2}\:d\theta\:\times\varphi(\rho)=\frac{\varphi(\rho)}{\pi\gamma}.
\end{align}
We are now ready to finish the proof of the main theorem.
\begin{proof}[Proof of \cref{mainthm}]
Applying the definition of the Barron norm, pick a canonical Barron measure $\rho$ such that $\AS f_\rho=\psi$ and such that
\begin{equation}\label{nearby}
\varphi(\rho)\le(1+\epsilon)\ABnorm{\psi},
\end{equation}
where $\epsilon=2^{-n^2}$. We set the truncation at level $\gamma=\frac1{2\sqrt d}$ as in \cref{AAbound}.
By \cref{prelimit} we can truncate the infra-red part, resulting in an error of order
\begin{align}\label{apply1}
\Xnorm{\psi_\gamma^{(\rho)}-\psi}
&=2^{-\Omega(n^{1+1/d})}\varphi(\rho)
\end{align}
We now approximate $\psi_\gamma^{(\rho)}$ with a finite sum. \cref{Adefasint} decomposes $\psi_\gamma^{(\rho)}$ as an integral over functions $\aa_{\theta\w}$, $\Xnorm{\aa_{\theta\w}}\le1$, against the complex measure $\mu$. We then apply \cref{convexcombi} to obtain a linear combination $\psi_m$ of $m$ terms such that
\begin{equation}\label{applycomplexlemma}
\Xnorm{\psi_m-\psi_\gamma^{(\rho)}}\le\|\mu\|/\sqrt m=C\varphi(\rho)/\sqrt m,
\end{equation}
where $C=2\sqrt d/\pi$. Here, the last equality is by \cref{TV}. Combine \cref{apply1,applycomplexlemma} using the triangle inequality to obtain
\begin{align}
\Xnorm{\psi_m-\psi}
&\le\varphi(\rho)(\tfrac{C}{\sqrt m}+2^{-\Omega(n^{1+1/d})})\\
&\le(1+\epsilon)\ABnorm{\psi}(\tfrac{C}{\sqrt m}+2^{-\Omega(n^{1+1/d})})\\
&=\ABnorm{\psi}(\tfrac{C}{\sqrt m}+2^{-\Omega(n^{1+1/d})})\\
&+\epsilon\ABnorm{\psi}(\tfrac{C}{\sqrt m}+2^{-\Omega(n^{1+1/d})}).
\end{align}
Finally, note that we can absorb the $\epsilon$-term because
\[2^{-n^2}\ABnorm{\psi}(\tfrac{C}{\sqrt m}+2^{-\Omega(n^{1+1/d})})=\ABnorm{\psi}2^{-\Omega(n^{1+1/d})}.\]
\end{proof}


\section{Experiments}    
\cref{fig:thmplot} shows the relation between the anti-symmetric Barron norm $\ABnorm{\psi}$ and the optimal approximation error $\Xnorm{\psi_m-\psi}$ by Slater sums $\psi_m$ as in \cref{mainthm,maincor1}.
Given different target functions $\psi$ we minimized $\Xnorm{\psi_m-\psi}$ over Slater sums to estimate the optimal LHS in \cref{mainthm}. We compared the optimal value with an estimate of the antisymmetric Barron norm $\ABnorm{\psi}$ obtained by a constrained minimization of the network weights over anti-symmetrized neural networks. To construct different target states $\psi$ we used the ground state of a fermionic quantum harmonic oscillator restricted to a sliding window of varying size and location. %In \cref{fig:thmplot} we have $n=6$ and $d=3$, and the determinant Ansatz $\psi_m$ has $m=4096$.
%See \cref{numericsdetails} for more details on the estimation.
\cref{fig:thmplot} illustrates that the anti-symmetric Barron norm provides an upper bound on the complexity of approximation by determinant-based Ansatz as in \cref{mainthm,maincor1}. 

\begin{figure}[h]
    \centering
    \includegraphics[width=.6\textwidth]{figures/comparison.pdf}
    \caption{Relation between the anti-symmetric Barron norm $\ABnorm{\psi}$ and the approximation error $\Xnorm{\psi_m-\psi}$ of \cref{mainthm}. Here, $n=6$, $d=3$, and $m=4096$. The opacity is $(1-\epsilon)^{10}$ where $\epsilon$ is the approximation error in the $\epsilon$-smooth anti-symmetric Barron norm $\ABnorm{\psi}^{(\epsilon)}$ (estimates with $\epsilon\approx0.1$ or worse show up as a light color).}
\label{fig:thmplot}
\end{figure}
\subsection{Estimating the anti-symmetric Barron norm}
The Barron norm of \cref{Bspdef} is defined in terms of the ReLU activation, but we can similarly define a Barron norm $\Bnorm{f}^{[\sigma]}$ with any other activation function $\sigma$. Then $\Bnorm{f}=\Bnorm{f}^{[\opn{ReLU}]}$ by definition. We observe that the definition is not overly sensitive to the choice of activation function: 
\begin{lemma}\label{eitherac}
\[\Bnorm{f}^{[\opn{ReLU}]}\le\Bnorm{f}^{[\opn{softplus}]},
\]
where $\opn{softplus}(y)=\log(1+e^y)$. Similarly, $\ABnorm{f}^{[\opn{ReLU}]}$ is bounded by $\ABnorm{f}^{[\opn{softplus}]}$.
\end{lemma}
\begin{proof}
It suffices to show that $\softplus$ is in the closed convex hull of translates of $\relu$. That is, it suffices to write it as a convolution
\begin{equation}
\relu*\phi=\softplus
\end{equation}
for some probability distribution $\psi$ on $\RR$. But we can solve for $\phi$ by differentiating twice:
\begin{align}
\relu'*\phi&=\softplus'=\sigma,\\
\phi=\delta*\phi=\relu''*\phi&=\sigma'=\sigma(1-\sigma).
\end{align}
where $\sigma$ is the logistic sigmoid function. clearly $\phi=\sigma(1-\sigma)$ is positive and $\int\phi=\int\sigma'=1$, so $\softplus$ is in the closed convex hull of translates of $\relu$, 
\end{proof}

\begin{remark}\label{rem:softplussharp}
\textup{Conversely to \cref{eitherac} softplus can approximate ReLU by using weights $a/t,tb,t\w$ with $t\to\infty$ without changing $\varphi(\rho)$. }
\end{remark}


To numerically estimate the Barron norm of a function we make use of \cref{eitherac} and define
\begin{equation}
f_{\a,\b,W}(x)=\sum_{k=1}^ma_k\softplus(\w^{(k)}\cdot x+b_k),
\end{equation}
where $\w^{(k)}\in\mathbb R^{nd}$ and $b_k,a_k\in\mathbb R$ for each $k=1,\ldots,m$. The smoother activation function allows us to use fewer neurons to approximate smooth features, and is justified by \cref{eitherac}. Sharper features can be approximated as in \cref{rem:softplussharp} without changing the norm estimate. We define the corresponding anti-symmetric Ansatz
\begin{equation}
\psi_{\a,\b,W}=\AS f_{\a,\b,W}.
\label{NN}\end{equation}
For functions of this form we have
%\begin{equation}
\begin{align*}
\ABnorm{\psi_{\a,\b,W}}\le\Bnorm{\AS f_{\a,\b,W}}\le&\tilde\varphi(\a,\b,W),\\
\tilde\varphi(\a,\b,W):=&\sum_{k=1}^m|a_k|(\|\w^{(k)}\|_1+b_k).
\end{align*}
We then minimize over $a,b,W\in\RR^{m}\times\RR^{m}\times\RR^{mnd}$ to estimate $\ABnorm{\psi_{\a,\b,W}}$.
Given a target function $\psi$ we define its $\epsilon$-smooth anti-symmetric Barron norm $\ABnorm{\psi}^{(\epsilon)}=\inf\{\ABnorm{\psi'}\:|\:\|\psi'-\psi\|^2\le\epsilon\}$. To estimate $\ABnorm{\psi}^{(\epsilon)}$ we then implement a supervised (SGD) learning procedure for $\psi_{\a,\b,W}$ with a loss function $L$ consisting of two penalties:%\LL{ typo below?} 
\[L(\psi')=\tau_\epsilon(\|\psi_{\a,\b,W}-\psi\|^2)+\lambda\varphi(\a,\b,W),\]
where $\lambda$ is a small constant, and $\tau_\epsilon(y)=\max\{y-\epsilon,0\}$. After the first penalty has converged we then use the optimal value of $\varphi(\a,\b,W)$ as our estimate of $\ABnorm{\psi}^{(\epsilon)}$. %(\cref{fig:B_eps}).


%\begin{figure}[h]
%    \centering
%    \includegraphics[width=.9\textwidth]{figures/eps.pdf}\\
%    \includegraphics[width=.8\textwidth]{figures/Btrain.pdf}
%    \caption{
%    Training process to estimate the anti-symmetric %\LL{ anti-symmetric?}  
%    Barron norm. A neural network Ansatz with one hidden layer is trained to approximate the function $\psi$ using the $L^2$ loss (top) with a small penalty added corresponding to the norm $\varphi(\a,\b,W)$ on the weights.}\label{fig:B_eps}
%\end{figure}







\section{Conclusion}
We have shown that anti-symmetric functions in the Barron space can be efficiently approximated by determinant-based neural network architectures, and the number of determinants depends on the anti-symmetric Barron norm of the function. Compared to existing bounds for neural network approximations, we obtain a factorially improved error bound. Our result illustrates the importance of choosing an Ansatz which reflects the known symmetries of the problem. It is an open question whether the anti-symmetric Barron norm is a useful characterization of certain challenging quantum states in practice. 

\section*{Acknowledgment}
 (N. A.) was supported by the NSF Quantum Leap Challenge Institute (QLCI) program through grant number OMA-2016245 and by the Simons Foundation under Award No. 825053. This material is also based upon work supported by the U.S. Department of Energy, Office of Science, Office of Advanced Scientific Computing Research and Office of Basic Energy Sciences, Scientific Discovery through Advanced Computing (SciDAC) program (L.L.).  L.L. is a Simons Investigator. 


%%%%%%%%%%%%%%%%%%%%%%%%%%%%%%%%%%%%%%%%%%%%%%%%%%%%%%%%%%%%%%%%%%%%%%%%%%%%%%%%%%%%%%%%%%%%%%%%%%%%

%\bibliography{na_ref_antisym,lin_ref}


\bibliographystyle{elsarticle-num}
\bibliography{refs}

%\onecolumn
\appendix

% %\newpage
\section{Alternative Definitions}\label{sec:other-definitions-short}
In this section, we discuss other potential definitions of Leximin approximation that might be considered intuitive.
\eden{removed ack. for anonymous submission}
% \footnote{We thank Sylvain Bouveret for suggesting definitions \ref{altDef:5} and \ref{altDef:6}.}.
For each alternative, we provide an example that illustrates why we believe it is inappropriate and a conclusion based on that example.
It should be noted that in order to avoid confusion, the error parameter $\gamma \in (0,1)$ is used in the alternative definitions (instead of $\beta$), to emphasize that these are only alternatives we do not use.


\begin{potentialDefinition}\label{altDef:2}
    A solution $x$ is a $(1-\gamma)$-approximately optimal if given a Leximin-optimal solution, $x^*$, there exists an integer $k \in [n]$ such that: 
    \begin{align*}
    \forall j < k: & \valBy{j}{x} \geq (1-\gamma) \cdot \valBy{j}{x^*}\\
    & \valBy{k}{x} > \valBy{k}{x^*}
    \end{align*}
\end{potentialDefinition}

\paragraph{Bad example def. \ref{altDef:2}:} Consider the following example with three objectives:
\begin{align*}
    \max \quad &\{f_i(x) = x_i \mid \forall 1 \leq i \leq 3 \} \\ \tag{E1}\label{eq:alt-def-eaxmple-1}
    s.t. \quad  & 99 x_1 + x_2 \leq 100\\
    &  x_3 \leq 100\\
    & x \in \mathbb{R}^3_{+}
\end{align*}
The Leximin optimal solution $x^*$ is $(1,1, 100)$ and therefore, by taking $k$ to be $2$, we get that any solution that its minimum objective value is at least $(1-\gamma)$ and its second-smallest objective value is more than $1$ is considered $(1-\gamma)$-approximately optimal Leximin solution.
For instance, consider $\gamma = 0.1$, the solution $(0.9, 1.1, 1.1)$ should be considered a $0.9$-approximately optimal according to this definition.
However, it is easy to see that this solution is quite bad for $f_2$ who can achieve $10.9$ (higher by a factor $> 9$) and very bad for $f_3$ who can achieve $100$ (higher by a factor $>90$).
And so, it seem reasonable to require that a good definition will consider as many objectives as possible.
% \erel{What objective values exactly? Do you mean: as many objectives as possible?}
% \eden{yes. To myself: this comment might be relevant to other places..}

\paragraph{Conclusion def. \ref{altDef:2}:} An appropriate definition should take into account as many objectives as possible.

\begin{potentialDefinition}\label{altDef:1}
    A solution $x$ is a $(1-\gamma)$-approximately optimal if for a Leximin-optimal solution, $x^*$, and for each $j = 1, \dots, n$ the following holds: 
    \begin{align*}
        \valBy{j}{x} \geq (1-\gamma) \cdot \valBy{j}{x^*} 
    \end{align*}
\end{potentialDefinition}

\paragraph{Bad example def. \ref{altDef:1}:} 
% An error in the first objective value might cause the other values to increase significantly.
Consider example \eqref{eq:alt-def-eaxmple-1} again.
Here, as the optimal solution is $(1,1, 100)$, any solution that yields at least $(1-\gamma,1-\gamma, (1-\gamma)\cdot 100)$.
However, considering $\gamma = 0.1$, $f_2$ can again achieve $9.1$ which is higher by a factor $> 100$ than the value it got $0.1$.

\paragraph{Conclusion def. \ref{altDef:1}:} An appropriate definition should consider the fact that an error in one objective might change the optimal value of other objectives.
As a consequence, another conclusion is that an appropriate definition should not consider the optimal solution at all.



\begin{potentialDefinition}\label{altDef:3}
    A solution $x$ is a $(1-\gamma)$-approximately optimal 
    if it satisfies the following requirements:
    \begin{enumerate}
        \item The objective-function with the smallest objective value achieves at least its maximum value times $(1-\gamma)$:
        \begin{align*}
            \valBy{1}{x} \geq (1-\gamma) \cdot \valBy{1}{x^*} 
        \end{align*}
        
        \item Given all the solutions that satisfies the first condition, let $m_2$ be the highest second-smallest objective value.
        The objective-function with the second-smallest objective value achieves at least the $m_2$ times $(1-\gamma)$.
        
        \item Given all the solutions that satisfies the former conditions, let $m_3$ be the highest third-smallest objective value.
        The objective-function with the third-smallest objective value achieves at least the $m_3$ times $(1-\gamma)$.
        
        \item and so on.
    \end{enumerate}
\end{potentialDefinition}

\paragraph{Bad example def. \ref{altDef:3}:}
Consider the following example with only two objectives:
\begin{align*}
    \max \quad &\{f_1(x) = x_1, f_2(x)=x_2\} \\
    s.t. \quad  & 99 x_1 + x_2 \leq 100\\
    & x \in \mathbb{R}^2_{+}
\end{align*}
The Leximin-optimal solution is $(1,1)$. Consider $\gamma = 0.1$, according to part (1) of this definition, all solutions in which the smallest objective value is at least $(1-\gamma)=0.9$ should be considered in order to determine $m_2$.
So, in this case, $m_2$ is determined to be $100 - 0.9 \cdot 99 = 10.9$.
Then, according to part (2), in order to be considered a $0.9$-approximately optimal, the second value must be at least $0.9 \cdot 10.9 = 9.81$.
But, even the exact Leximin optimal solution does not satisfy this requirement, so this cannot be considered an approximation to Leximin optimal.

In general, this definition has the disadvantage of favoring solutions that give the lowest bounds to the objective functions considered in the earlier steps,  since this may enable to increase the values of the higher objectives.
According to the Leximin nature, the most important thing is to make the worst-off player as happy as possible (and then the second worst-off and so on), therefore, we emphasize the importance of this characteristic also in the definition of the approximated version.

\paragraph{Conclusion def. \ref{altDef:3}:} An appropriate definition should also capture the Leximin optimal solutions, and maintain the Leximin nature whenever possible.

% \eden{I think this definition is actually equivalent to our current... need to think about it again}
% \begin{potentialDefinition}\label{altDef:4}
%     A solution $x$ is a $\gamma$-approximately-optimal Leximin solution if it can be viewed as the result of this process:
%     \begin{enumerate}
%         \item Choose a solution in which the objective-function with the smallest objective value achieves at least the maximum value minus $\gamma$:
%         \begin{align*}
%             \valBy{1}{x} \geq \valBy{1}{x^*} - \gamma
%         \end{align*}
%         Let $z_1$ be the value it achieves (i.e., $\valBy{1}{x})$.
        
%         \item Consider all the solutions in which the objective-function with the smallest objective value achieves at least $z_1$ and let $m_2$ be the highest second-smallest objective value.
%         Then, choose a solution in which the objective-function with the second-smallest objective value achieves at least the $m_2$ minus $\gamma$.
%         Let $z_2$ be the value it achieves.
%         \item Consider all the solutions in which the objective-function, the smallest objective value achieves at least $z_1$ and the second-smallest objective value achieves at least $z_2$, and let $m_3$ be the highest third-smallest objective value.
%         Then, choose a solution in which the objective-function with the third-smallest objective value achieves at least the $m_3$ minus $\gamma$.
        
%         \item and so on...
%     \end{enumerate}
% \end{potentialDefinition}

% \paragraph{Bad example def. \ref{altDef:3}:} Although in this definition, the Leximin optimal solution is also approximately-optimal as we wanted, another issue arises.

% \begin{itemize}
%     \item \textbf{Bad example:} two solutions that meet this definition, but one of them is strictly better (by more than $\gamma$) than the other from some point.
%     \item \textbf{Conclusion:} an appropriate (good?) definition should determine between two solutions if possible. 
% \end{itemize}

% -----------------------------------
% \subsection{others}
% (From the correspondence of Erel with Lemaitre and Bouveret)

%----------------------------------
% need to think about a corresponding def for mult...
\begin{potentialDefinition}\label{altDef:5}
    A solution $x$ is a $(1-\gamma)$-approximately optimal if for a Leximin-optimal solution, $x^*$, and for each $j = 1, \dots, n$: 
    % in the addive version was $$$
    \begin{align*}
        % |\valBy{j}{x} - \valBy{j}{x^*}| \leq \gamma\\
         \max\{\valBy{j}{y},\valBy{j}{x}\}  \leq \frac{1}{1-\gamma} \cdot \min\{\valBy{j}{y},\valBy{j}{x}\}
    \end{align*}
\end{potentialDefinition}

\paragraph{Bad example and conclusion def. \ref{altDef:5}:}
This definition is close to definition \ref{altDef:1} but weaker, still the same example and conclusion apply.

\begin{potentialDefinition}\label{altDef:6}
    A solution $x$ is a $(1-\gamma)$-approximately optimal if given a Leximin-optimal solution, $x^*$, there exists an integer $k \in [n]$ such that: 
    \begin{align*}
    \forall j < k: & \valBy{j}{x} = \valBy{j}{x^*}\\
    & \valBy{k}{x} > (1-\gamma) \cdot \valBy{k}{x^*}
    \end{align*}
\end{potentialDefinition}
% \eden{I'm not sure it is well defined, since by decreasing $\gamma$ (for example) the second value might become smaller than the first.}

\paragraph{Bad example def. \ref{altDef:6}:} As in the case of definition \ref{altDef:2}, by taking a small $k$, we cannot distinguish between two solutions that satisfy this definition, but one of them should be definitely preferred.
Consider again the following example with three objectives, where:
\begin{align*}
    \max \quad &\{f_i(x) = x_i \mid \forall 1 \leq i \leq 3 \} \\
    s.t. \quad  & 9 x_1 + x_2 \leq 10\\
    &  x_3 \leq 100\\
    & x \in \mathbb{R}^3_{+}
\end{align*}
The Leximin optimal solution $x^*$ is $(1,1, 100)$ and therefore, by taking $k$ to be $2$, we get that any solution that its minimum value is $1$ and its second-smallest objective value is more than $(1-\gamma)$ is considered $(1-\gamma)$-approximately optimal.
As an example, the solution $(1, 1, 1)$ is considered $(1-\gamma)$-approximately-optimal Leximin solution (as $(1-\gamma) < 1$).
But it is easy to see that this solution is quite bad for $f_3$ (who can achieve $100$).

\paragraph{Conclusion def. \ref{altDef:6}:} Same as for def. \ref{altDef:2}, an appropriate definition should take into account as many objectives as possible.

% \begin{potentialDefinition}
%     OWA.
% \end{potentialDefinition}

%--------------------------

\begin{potentialDefinition}\label{altDef:7}
    A solution $x$ is a $(1-\gamma)$-approximately optimal if there is no other solution $y$ that is $(1-\gamma)$-Leximin preferred over it, where this relation is defined as follows: $y$ is preferred over $x$ if  there exists an integer $k \in [n]$ such that:
    \begin{align*}
        \forall j < k \colon \quad &   \max\{\valBy{j}{y},\valBy{j}{x}\}  \leq \frac{1}{(1-\gamma)} \cdot \min\{\valBy{j}{y},\valBy{j}{x}\}\\
        & \valBy{k}{y} > \frac{1}{(1-\gamma)} \cdot 
\valBy{k}{x}
    \end{align*}
     [This relation is related to a one suggested in \cite{kalai_lexicographic_2012}, it is described in more detail in the Related work Section]. 
\end{potentialDefinition}

\paragraph{Bad example and conclusion def. \ref{altDef:7}:} As with definition \ref{altDef:3}, here also, the Leximin optimal solution is not optimal according to this relation and it might favor solutions with lower smallest objective values. 
Consider again the following example:
\begin{align*}
    \max \quad &\{f_1(x) = x_1, f_2(x)=x_2\} \\
    s.t. \quad  & 99 x_1 + x_2 \leq 100\\
    & x \in \mathbb{R}^2_{+}
\end{align*}
Assume that $\gamma = 0.1$, the Leximin-optimal solution is $(1,1)$, but the solution $(0.9,10.9)$ is preferred over it according to this relation (since for $k=2$ we get that $\max\{0.9,1\} \leq \frac{1}{0.9}\cdot\min\{0.9,1\}$ and $10.9 > \frac{1}{0.9} \cdot 1$) and therefore, it is not approximately-optimal.



\section{The Approximate Leximin Order}\label{sec:approx-order-is-strict-partial}

Unlike the leximin order, $\leximinPreferred$, which is a strict \textbf{total} order, the approximate leximin order, $\alphaBetaPreferred$ for $\DEFmultApprox\in (0,1]$ and $\DEFadditiveApprox \geq 0$ is a strict \textbf{partial} order.
The difference is that in partial orders, not all vectors are comparable.
Consider for example the sorted vectors $(1,2)$ and $(1, 3)$. 
According to the leximin order, $(1,3)$ is clearly preferred (as $3>2$), but according to many approximate leximin orders neither one is preferred over the other, for example according to the orders $\alphaBetaPreferredParams{0.6}{0}$,$ \alphaBetaPreferredParams{1}{1}$ or $\alphaBetaPreferredParams{0.8}{0.5}$.
% (irreflexive, asymmetric and transitive).

An order is a strict partial order if it is irreflexive, transitive and asymmetric.
Lemma \ref{lemma:order-is-irreflexive} proves that the order is irreflexive, Lemma \ref{lemma:order-is-transitive} proves it is transitive, and Lemma \ref{lemma:order-is-asymmetric} proves that it is asymmetric.
% \erel{It would be good to show an example why this is not a total order.}

% need to prove irreflexive, asymmetric (we have already proved that it is transitive).

% ***[I thought it would be better to prove it on vectors (rather than "solutions") to make it as general as possible]\\

Let $\DEFmultApprox\in (0,1]$ and $\DEFadditiveApprox \geq 0$. 

\begin{lemma}\label{lemma:order-is-irreflexive}
    The approximate leximin order $\alphaBetaPreferred$ is irreflexive.
\end{lemma}

\begin{proof}
    % \eden{I used $x$ only to remind the reader what irreflexive is, maybe it should simply be in the lemma description}
    Let $x$ be a solution. We will show that $x \nAlphaBetaPreferred x$.
    As the definition requires that one component be \emph{strictly greater} than the other, it is trivial.
\end{proof}

\begin{lemma}\label{lemma:order-is-transitive}
    The approximate leximin order $\alphaBetaPreferred$ is transitive.
\end{lemma}

\begin{proof}
    Let $x,y$ and $z$ be solutions such that $x \alphaBetaPreferred y$ and $y \alphaBetaPreferred z$.
    We will prove that $x \alphaBetaPreferred z$.

    
    Since $x \alphaBetaPreferred y$, there exists an integer $ k_1 \in [n]$ such that:
    \begin{align*}
        \forall j<k_1 \colon &  \valBy{j}{x} \geq \valBy{j}{y}\\
            & \valBy{k_1}{x} > \frac{1}{\DEFmultApprox} \left( \valBy{k_1}{y} + \DEFadditiveApprox \right)
    \end{align*}
    And since $y \alphaBetaPreferred z$, there exists an integer $k_2 \in [n]$ such that:
    \begin{align*}
        \forall j<k_2 \colon &  \valBy{j}{y} \geq \valBy{j}{z}\\
            & \valBy{k_2}{y} > \frac{1}{\DEFmultApprox} \left( \valBy{k_2}{z} + \DEFadditiveApprox \right) 
    \end{align*}

    As $\DEFmultApprox \in (0,1]$ and $\DEFadditiveApprox \geq 0$, it follows that:
    \begin{align}\label{eq:trans-k-s}
        \valBy{k_1}{x} > \valBy{k_1}{y}, \Hquad \valBy{k_2}{y} >  \valBy{k_2}{z}
    \end{align}

    % Accordingly, if $k_1=k_2$, then this integer, denoted by $k$, allows us to conclude that $x \alphaBetaPreferred z$. 
    % By the definitions of $k_1$ and $k_2$, for any $j<k_1=k_2$ the required holds as $\valBy{j}{x} \geq \valBy{j}{y} \geq \valBy{j}{z}$.
    % In addition, $\valBy{k_1}{x}> \valBy{k_1}{y}$ by equation \ref{eq:trans-k-s}
    % $ > \frac{1}{\DEFmultApprox} \left( \valBy{k_1}{y} + \DEFadditiveApprox \right)$ and nd  and 
    
    Let $k = \min\{k_1,k_2\}$.
    
    If $k = k_1$, by the definition of $k_1$, $\valBy{k}{x} > \frac{1}{\DEFmultApprox} \left( \valBy{k}{y} + \DEFadditiveApprox \right)$.
    However, $\valBy{k}{y} \geq \valBy{k}{z}$, by definition if $k<k_2$ and by equation \ref{eq:trans-k-s} if $k=k_2$. \ref{eq:transitive-k}
    Therefore, $\valBy{k}{x} > \frac{1}{\DEFmultApprox} \left( \valBy{k}{z} + \DEFadditiveApprox \right)$.
    
    Otherwise, if $k=k_2$, by the definition of $k_2$, $\valBy{k}{y} > \frac{1}{\DEFmultApprox} \left( \valBy{k}{z} + \DEFadditiveApprox \right)$. But, $\valBy{k}{x} \geq \valBy{k}{y}$, by definition if $k<k_1$ and by equation \ref{eq:trans-k-s} if $k=k_1$. Again, we can conclude that $\valBy{k}{x} > \frac{1}{\DEFmultApprox} \left( \valBy{k}{z} + \DEFadditiveApprox \right)$.

     In addition, for each $j<k$, since $j< k_1$ and $j < k_2$, by definition the following holds:
    \begin{align}\label{eq:transitive-k}
        \valBy{j}{x} \geq \valBy{j}{y} \geq \valBy{j}{z}
    \end{align}
    So, $k$ is an integer that satisfy all the requirements, and so, $x \alphaBetaPreferred z$.
    \end{proof}
    

    
    \begin{lemma}\label{lemma:order-is-asymmetric}
        The approximate leximin order $\alphaBetaPreferred$ is asymmetric.
    \end{lemma}
    
    \begin{proof}
        Let $x$ and $y$ be solutions such that $x \alphaBetaPreferred y$. We will show that $y \nAlphaBetaPreferred x$. 
        Assume by contradiction that $y \alphaBetaPreferred x$. 
        From Lemma \ref{lemma:order-is-transitive}, this relation is transitive. Therefore, since $x \alphaBetaPreferred y$ and $y \alphaBetaPreferred x$, also $x \alphaBetaPreferred x$.
        But, from Lemma \ref{lemma:order-is-irreflexive}, this relation is irreflexive --- a contradiction.
    \end{proof}
\section{Proof of Theorem \ref{th:main}}\label{sec:algo-sec-proofs}
\eden{should probably change the title}

This section is dedicated to proving Theorem \ref{th:main}.
To this end, we use another equivalent representation of \eqref{eq:sums-OP}, which was also introduced by \cite{Ogryczak_2006} 
(we provide the proof of equivalence in Appendix \ref{sec:equivalent-proofs}). 
\erel{Can't we just use it directly instead of P2?}
% (here also, the variables are $\ztVar{x}$ and $x$, and $z_1, \ldots z_{t-1}$ are constants)
\begin{align*}
    \max \quad &z_t \tag{P2-compact}\label{eq:compact-OP} \;\;
        s.t. &\quad  & (1) \quad x \in S\\
                    &&& (\Tilde{2}) \quad \sum_{i=1}^{\ell} \valBy{i}{x} \geq \sum_{i=1}^{\ell}  z_i && \ell = 1,\ldots, t-1 \nonumber\\
                    &&& (\Tilde{3}) \quad \sum_{i=1}^{t} \valBy{i}{x} \geq \sum_{i=1}^{t}  z_i
\end{align*}
In this problem, constraints $(\hat{2})$ and $(\hat{3})$ are replaced by  $(\Tilde{2})$ and $(\Tilde{3})$, respectively.  
The difference is that
$(\hat{2})$ gives, for each $\ell$, a lower bound on the sum for \emph{any} set of $\ell$ objective functions; whereas $(\Tilde{2})$ only considers the sum of the $\ell$ \emph{smallest} such values.  
% However, since the constraints set the same lower bound on this sum, the constraints are equivalent.  
Similarly for $(\hat{3})$ and $(\Tilde{3})$. 
Since  the problems are equivalent, a solver, either exact or approximate, for one can be used as a solver, with the same level of accuracy, for the other (Lemma \ref{lemma:solver-equivalent-prob}). 
Therefore, as \eqref{eq:compact-OP} is equivalent to \eqref{eq:sums-OP}, which, in turn, is equivalent to \eqref{eq:vsums-OP}, in proving the theorem we may assume that \textsf{OP} is an approximation procedure for \eqref{eq:compact-OP}.  
This will simplify the proofs. \eden{I added the line from the comment back, isn't it important to explain why we need this representation?}

% \erel{*** I do not understand. We say that P1 and P2 are equivalent with an exact solver, but not with an approximate solver. Here, we claim that P3 and P2-compact are equivalent, but this is true only with an exact solver. Don't we have to prove that they are equivalent also with an approximate solver? ***}

We denote $\retSol := x_n$ = the solution $x$ attained at the last iteration ($t=n$) of the algorithm. 

Following are some observations regarding the set of feasible solutions in each iteration, their objective values, and the solution $\retSol$ that will be useful later on.

% For any constants $z_1,\ldots, z_{t-1}$,
% any vector $x \in S$ that satisfies constraint $(\Tilde{2})$ of \eqref{eq:compact-OP} 
% is feasible to this problem.
% This is because any solution $x \in S$ can satisfy constraint $(\Tilde{3})$ with a small enough assignment to the variable $z_t$. \eden{I'm not sure how to explain it....}
\begin{observation}\label{obs:feasi-and-constraint2}
For any constants $z_1,\ldots, z_{t-1}$,
any vector $x \in S$ that satisfies constraint $(\Tilde{2})$ of \eqref{eq:compact-OP} 
can be a part of a feasible solution $(x,z_t)$ for any $z_t \leq \sum_{i=1}^{t} \valBy{i}{x} - \sum_{i=1}^{t-1} z_i$.
\end{observation}

Since $\retSol$ is a feasible solution of \eqref{eq:compact-OP} in iteration $n$, and as each
iteration only adds new constraints to $(\Tilde{2})$, it follows that $\retSol$ is also a feasible solution of \eqref{eq:compact-OP} in any iteration $1 \leq t\leq n$. 
\begin{observation}\label{obs:retSol-solves-any-t}
$\retSol$ is a feasible solution of \eqref{eq:compact-OP} in any iteration $1 \leq t\leq n$.
\end{observation}

Now, consider the problem \eqref{eq:compact-OP} that was solved in iteration $t$.
Here, $z_t$ is a \emph{variable} and $z_1, \ldots z_{t-1}$ are constants.
The objective of this problem is $\max z_t$, and the only constraint that includes the variable $z_t$ is  $(\Tilde{3})$.
Therefore, rearranging it to $\sum_{i=1}^{t} \valBy{i}{x} - \sum_{i=1}^{t-1}  z_i\geq z_t$, allows us to conclude that the objective value is determined by the left side of this inequality (as $z_t$ is maximized when the inequality turns to equality).
\begin{observation}\label{obs:obj-value}
The objective value obtained by a feasible solution $x$ to the problem \eqref{eq:compact-OP} that was solved in iteration $t$ is $\sum_{i=1}^{t} \valBy{i}{x} - \sum_{i=1}^{t-1}  z_i$.
\end{observation}

Lastly, as the value obtained as a $(\multApprox, \additiveApprox)$-approximation for this problem is the \emph{constant} $z_t$, the optimal value is at most $\frac{1}{\multApprox} (z_t+\additiveError)$. 
Consequently, the objective value of any feasible solution is at most this value.
Since $\retSol$ is feasible for any iteration $t$ (Observation \ref{obs:retSol-solves-any-t}) and since its objective is $\sum_{i=1}^t \valBy{i}{\retSol} - \sum_{i=1}^{t-1} z_i$ (Observation \ref{obs:obj-value}), we can conclude:

\begin{observation}\label{obs:obj-xt-to-zt}
    The objective value obtained by $\retSol$ to the problem \eqref{eq:compact-OP} that was solved in iteration $t$ is at most $\frac{1}{\multApprox} (z_t+\additiveError)$. That is:
    \begin{align*}
        \sum_{i=1}^t \valBy{i}{\retSol} - \sum_{i=1}^{t-1} z_i \leq \frac{1}{\multApprox} \left(z_t+\additiveError \right).
    \end{align*}
\end{observation}

% This conclusion also implies that for any $1 \leq t \leq n$, the solution $(x_t, z_t)$ that that was outputted for \eqref{eq:compact-OP} in iteration $t$, satisfies constraint $(\Tilde{3})$ as equality. That is:
% \begin{observation}\label{obs:equality-xt-zt}
% For any $1 \leq t \leq n$,  $\sum_{i=1}^{t} \valBy{i}{x_t} = \sum_{i=1}^{t}  z_i$.
% \end{observation}



%%%
% OVERALL EXPLANATION 
We start with Lemmas \ref{lemma:beta-vk}-\ref{lemma:fk-to-all}, which establish a relationship between the $k$-th least objective value obtained by $\retSol$ 
% ($\valBy{k}{\retSol}$) 
and the difference between the sum of the $(k-1)$ least objective values obtained by $\retSol$ and the sum of the $(k-1)$ first $z_i$ values.
% ($\sum_{i=1}^{k-1}\valBy{k}{\retSol} - \sum_{i=1}^{k-1}z_i$). 
Theorem \ref{th:main} then uses this relation to prove that the existence of another solution that would be $\left(\frac{\multApprox^2}{1-\multApprox + \multApprox^2}, \frac{\multApprox(2-\multApprox)\additiveApprox}{1-\multApprox +\multApprox^2}\right)$-preferred over $\retSol$ would lead to a contradiction.

For clarity, throughout the proofs, we denote the multiplicative error factor by $\multError = 1-\multApprox$.

% LEMMAS.
% BLAH BLAH.

\begin{lemma}\label{lemma:beta-vk}
    For all $k\in[n]$, 
    \begin{align*}
        \multError \valBy{k}{\retSol} \geq \left(\sum_{i=1}^k \valBy{i}{\retSol} - \sum_{i=1}^k z_i\right) -\multError \left(\sum_{i=1}^{k-1} \valBy{i}{\retSol} - \sum_{i=1}^{k-1} z_i\right) -\additiveError
    \end{align*}
\end{lemma}

\begin{proof}
By Observation \ref{obs:obj-xt-to-zt},
    \begin{align*}
         &\sum_{i=1}^k \valBy{i}{\retSol} - \sum_{i=1}^{k-1} z_i \leq \frac{1}{\multApprox} \left(z_k + \additiveError \right) = \frac{1}{1-\multError} \left(z_k + \additiveError \right)\\
         &\Rightarrow z_k +\additiveError \geq (1-\multError) \left(\sum_{i=1}^{k} \valBy{i}{\retSol} - \sum_{i=1}^{k-1}  z_i\right)\\
        &\Rightarrow z_k +\additiveError\geq \left(\sum_{i=1}^{k} \valBy{i}{\retSol} - \sum_{i=1}^{k-1}  z_i\right) - \multError \left(\sum_{i=1}^{k} \valBy{i}{\retSol} - \sum_{i=1}^{k-1}  z_i\right)\\
        &\Rightarrow \multError \valBy{k}{\retSol} \geq \left(\sum_{i=1}^k \valBy{i}{\retSol} - \sum_{i=1}^k z_i\right) -\multError \left(\sum_{i=1}^{k-1} \valBy{i}{\retSol} - \sum_{i=1}^{k-1} z_i\right) -\additiveError.
        \qedhere
    \end{align*}
\end{proof}


\begin{lemma}\label{lemma:beta-sums-to-diff}
    For all $k\in[n]$, 
    \begin{align*}
        \sum_{i=1}^k \multError^{i} \valBy{k-i+1}{\retSol} \geq \sum_{i=1}^k \valBy{i}{\retSol} - \sum_{i=1}^{k} z_i -\additiveError
    \end{align*}
\end{lemma}

\begin{proof}
    The proof is by induction on $k$.
    For $k=1$ the claim follows directly from Lemma \ref{lemma:beta-vk}.
    Assuming the claim is true for $1,\ldots k-1$, we show it is true for $k$:
    \begin{align*}
        &\sum_{i=1}^k \multError^{i} \valBy{k-i+1}{\retSol} = \multError \valBy{k}{\retSol} + \sum_{i=2}^k \multError^{i} \valBy{k-i+1}{\retSol}\\
        &= \multError \valBy{k}{\retSol} + \sum_{i=1}^{k-1} \multError^{i+1} \valBy{k-(i+1)+1}{\retSol} \\
        &= \multError \valBy{k}{\retSol} + \multError \sum_{i=1}^{k-1} \multError^{i} \valBy{(k-1) -i+1}{\retSol}\\
        &= \multError \valBy{k}{\retSol} + \multError \left(\sum_{i=1}^{k-1} \valBy{i}{\retSol} - \sum_{i=1}^{k-1} z_i\right) && \text{(by induction assumption)}\\
        &\geq \left(\sum_{i=1}^k \valBy{i}{\retSol} - \sum_{i=1}^k z_i\right) -\multError \left(\sum_{i=1}^{k-1} \valBy{i}{\retSol} - \sum_{i=1}^{k-1} z_i\right)-\additiveError  \\
        & \quad +  \multError \left(\sum_{i=1}^{k-1} \valBy{i}{\retSol} - \sum_{i=1}^{k-1} z_i\right) && \text{(by Lemma \ref{lemma:beta-vk})} \\
        &= \sum_{i=1}^k \valBy{i}{\retSol} - \sum_{i=1}^{k} z_i -\additiveError.
        \qed
    \end{align*}
\end{proof}


\begin{lemma}\label{lemma:fk-to-all}
    For all $1<k \leq n$, 
    \begin{align*}
        \frac{\multError}{1-\multError} \valBy{k}{\retSol} \geq \sum_{i=1}^{k-1}\valBy{i}{\retSol} - \sum_{i=1}^{k-1}z_i - \additiveError
    \end{align*}
\end{lemma}

\begin{proof}
    First, notice that since $k \geq (k-1)-i+1$ for any $1\leq i \leq k$ and as the function $\valBy{i}$ represents the $i$-th smallest objective value, also:
    \begin{align}\label{eq:increase-by-obj-size}
        \forall 1\leq i \leq k \colon \quad \valBy{k}{\retSol} \geq \valBy{(k-1)-i+1}{\retSol}
    \end{align}
    In addition, consider the geometric series with a first element $1$, a ratio $\multError$, and a length $(k-1)$. 
    As $\multError < 1$, its sum can be bounded in the following way:
    \begin{align}\label{eq:geometric-series-beta}
        \sum_{i=1}^{k-1} \multError^{i-1} = \frac{1-\multError^{k-1}}{1-\multError} < \lim_{k \to \infty}\frac{1-\multError^{k-1}}{1-\multError} = \frac{1}{1-\multError}
    \end{align}
    
    Now, the claim can be concluded as follows:
    \begin{align*}
        & \frac{\multError}{1-\multError}\valBy{k}{\retSol} = \multError \left(\frac{1}{1-\multError} \valBy{k}{\retSol} \right)\\
        & > \multError \left(\sum_{i=1}^{k-1} \multError^{i-1} \valBy{k}{\retSol} \right) && \text{(by Equation \eqref{eq:geometric-series-beta})}\\
        & \geq  \multError \left(\sum_{i=1}^{k-1} \multError^{i-1} \valBy{(k-1)-i+1}{\retSol} \right) && \text{(by Equation \eqref{eq:increase-by-obj-size})}\\
        &= \sum_{i=1}^{k-1} \multError^{i} \valBy{(k-1)-i+1}{\retSol} \\
        &\geq \sum_{i=1}^{k-1}\valBy{i}{\retSol} - \sum_{i=1}^{k-1}z_i - \additiveError && \text{(by Lemma \ref{lemma:beta-sums-to-diff})}
\end{align*}
\erel{Formally, Lemma \ref{lemma:beta-sums-to-diff} is for $k\geq 1$, and we apply it for $k-1$, which might be $0$.}\eden{I tried to fixed it, is it better?}
\end{proof}



%------
% thm.

We are now ready to prove the Theorem \ref{th:main}.
\begin{proof}[Proof of Theorem \ref{th:main}]
% \eden{I'm not sure if we should write again about the claim with $\multApprox$}
Recall that the claim is that $\retSol$ is a $\left(\frac{\multApprox^2}{1-\multApprox + \multApprox^2}, \frac{\multApprox(2-\multApprox)\additiveApprox}{1-\multApprox +\multApprox^2}\right)$-approximation.

For brevity, we define the following constants:
\begin{align*}
    \Delta^{mult} = \frac{\multApprox}{1-\multApprox + \multApprox^2}, \quad  \Delta^{add} = \frac{\multApprox(2-\multApprox)}{1-\multApprox +\multApprox^2}
\end{align*}
Accordingly, we need to prove that $\retSol$ is a $\left(\Delta^{mult} \cdot \multApprox, \Delta^{add}\cdot\additiveApprox\right)$-approximation.

We prove the following equation, that will be helpful later:
\begin{align}\label{equ:mu}
\frac{1}{\Delta^{mult} \cdot \multApprox} = \frac{1-\multError +\multError^2}{(1-\multError)^2}
\end{align}
This is true because
\begin{align*}
    &\Delta^{mult} \cdot \multApprox =   \frac{\multApprox^2}{1-\multApprox + \multApprox^2} && \text{(Definition of $\Delta^{mult}$)} \\
    &= \frac{(1-\multError)^2}{\multError +(1-\multError)^2} = \frac{(1-\multError)^2}{1-\multError +\multError^2} &&\text{(since $\multApprox = 1-\multError$)}\\
    & \Rightarrow \frac{1}{\Delta^{mult} \cdot \multApprox} = \frac{1-\multError +\multError^2}{(1-\multError)^2}
    \end{align*}
    Another equation that will be useful later is:
    \begin{align}\label{eq:additive-error}
        \frac{\Delta^{add}}{\Delta^{mult}\cdot \multApprox}  = \frac{1+\multError}{1-\multError}.
    \end{align}
    The reason for this is that
    \begin{align*}
        &\frac{\Delta^{add}}{\Delta^{mult}\cdot \multApprox} =\frac{1-\multApprox + \multApprox^2}{\multApprox^2} \cdot \frac{\multApprox(2-\multApprox)}{1-\multApprox +\multApprox^2}&& \text{(Definitions of $\Delta^{mult}$ and $\Delta^{add}$)}\\
        &=\frac{\multApprox(2-\multApprox)}{\multApprox^2} = \frac{(1-\multError)(1 + \multError)}{(1-\multError)^2} =\frac{1+\multError}{1-\multError}  &&\text{(since $\multApprox = 1-\multError$)}
    \end{align*}

    Now, suppose by contradiction that $\retSol$ is \emph{not} $\left(\Delta^{mult} \cdot \multApprox, \Delta^{add}\cdot\additiveApprox\right)$-approximately-optimal.
    By definition, this means there exists a solution $y \in S$  that is $\left(\Delta^{mult} \cdot \multApprox, \Delta^{add}\cdot\additiveApprox\right)$-preferred over it.
    That is, there exists an integer $1 \leq k \leq n$ such that:
    \begin{align*}
        \forall j < k \colon &\valBy{j}{y} \geq \valBy{j}{\retSol};\\
        & \valBy{k}{y} > \frac{1}{\Delta^{mult} \cdot\multApprox} \left(\valBy{k}{\retSol} + \Delta^{add} \cdot\additiveError \right).
    \end{align*}

    Since $\retSol$ was obtained in \eqref{eq:compact-OP} that was solved in the last iteration $n$, it is clear that $\sum_{i=1}^k \valBy{i}{\retSol} \geq \sum_{i=1}^{k} z_i$ (by constraint $(\Tilde{2})$ if $k<n$ and $(\Tilde{3})$ otherwise).
    Which implies:
    \begin{align}\label{eq:fk-to-zk}
        \sum_{i=1}^k \valBy{i}{\retSol} - \sum_{i=1}^{k-1} z_i \geq z_k
    \end{align}

    Now, consider \eqref{eq:compact-OP} that was solved in iteration $k$.
    By Observation \ref{obs:retSol-solves-any-t}, $\retSol$ is feasible to this problem.
    As the $(k-1)$ smallest objective values of $y$ are at least as high as those of $\retSol$, it is easy to conclude that $y$ also satisfies constraints $(\Tilde{2})$ of this problem; since, for any $\ell < k$:
    \begin{align*}
        \sum_{i=1}^{\ell} \valBy{i}{y} \geq\sum_{i=1}^{\ell} \valBy{i}{\retSol} \geq \sum_{i=1}^{\ell} z_i
    \end{align*}
    Therefore, by Observation \ref{obs:feasi-and-constraint2}, $y$ is also feasible to this problem. 

    If $k=1$, the objective value $y$ in this problem is $\valBy{1}{y}$ (Observation \ref{obs:obj-value}).
    In addition, $\valBy{1}{\retSol} \geq z_1$ by equation \ref{eq:fk-to-zk}. As $\Delta^{mult}\geq 0$ and $\Delta^{add}\geq 0$, it follows that:
    \begin{align*}
        \valBy{1}{y}> \frac{1}{\Delta^{mult} \cdot\multApprox} \left(\valBy{1}{\retSol} + \Delta^{add} \cdot\additiveError \right)\geq \frac{1}{\multApprox} \left(z_1 + \additiveError \right)
    \end{align*}
    But, $z_1$ was obtained as an approximation for this problem, therefore the optimal value is at most $\frac{1}{\multApprox}\left(z_1 + \additiveError \right)$ --- a contradiction.

    
    Otherwise, $k>1$, we shall now see that in this case $y$ also satisfies the following:
    \begin{align}\label{eq:yk-to-sum}
        \valBy{k}{y} > \frac{1}{1-\multError} \valBy{k}{\retSol} + \frac{\multError}{1-\multError}\sum_{i=1}^{k-1}\valBy{i}{\retSol} - \frac{\multError}{1-\multError} \sum_{i=1}^{k-1}z_i  +\frac{1}{1-\multError}\cdot\additiveError
    \end{align}
    this is true because
    \begin{align*}
        &\valBy{k}{y} > \frac{1}{ \Delta^{mult} \cdot\multApprox} \left(\valBy{k}{\retSol} + \Delta^{add}\cdot \additiveError \right) && \text{(Definition of $y$ for $k$)}\\
        &= \frac{1-\multError +\multError^2}{(1-\multError)^2} \valBy{k}{\retSol}+ \frac{\Delta^{add}}{\Delta^{mult} \multApprox}\cdot\additiveError && \text{(by Equation \ref{equ:mu})}\\
        &= \frac{1-\multError +\multError^2}{(1-\multError)^2} \valBy{k}{\retSol}+ \frac{1+\multError}{1-\multError}\cdot\additiveError && \text{(by Equation \ref{eq:additive-error})} \erel{???}\\
        &\geq\frac{1}{1-\multError} \valBy{k}{\retSol} + \frac{\multError}{1-\multError}\left(\sum_{i=1}^{k-1}\valBy{i}{\retSol} - \sum_{i=1}^{k-1}z_i-\additiveError\right) +\frac{1+\multError}{1-\multError}\cdot\additiveError && \text{(by Lemma \ref{lemma:fk-to-all} for $k>1$)}\\
        & = \frac{1}{1-\multError} \valBy{k}{\retSol} +\frac{\multError}{1-\multError}\sum_{i=1}^{k-1}\valBy{i}{\retSol} - \frac{\multError}{1-\multError} \sum_{i=1}^{k-1}z_i +\frac{1}{1-\multError}\cdot\additiveError &&\erel{???}\text{\eden{is it more clear?}}
    \end{align*}    
    
    We compute the objective value of $y$, which is $\sum_{i=1}^k \valBy{i}{y} - \sum_{i=1}^{k-1} z_i$ (by Observation \ref{obs:obj-value}):  
    \begin{align*}
        &\sum_{i=1}^k \valBy{i}{y} - \sum_{i=1}^{k-1} z_i=\sum_{i=1}^{k-1} \valBy{i}{y} - \sum_{i=1}^{k-1} z_i + \valBy{k}{y}\\
        &\geq \sum_{i=1}^{k-1} \valBy{i}{\retSol} - \sum_{i=1}^{k-1} z_i + \valBy{k}{y} && \text{(Definition of $y$ for $j<k$)}\\
        &> \sum_{i=1}^{k-1} \valBy{i}{\retSol} - \sum_{i=1}^{k-1} z_i + \frac{1}{1-\multError} \valBy{k}{\retSol} \\
        & \quad + \frac{\multError}{1-\multError}\sum_{i=1}^{k-1}\valBy{i}{\retSol} - \frac{\multError}{1-\multError}\sum_{i=1}^{k-1}z_i +\frac{1}{1-\multError}\cdot\additiveError && \text{(by Equation \ref{eq:yk-to-sum})}\\
        & = \frac{1}{1-\multError} \left(\sum_{i=1}^k \valBy{k}{\retSol} - \sum_{i=1}^{k-1}z_i + \additiveError\right) &&\text{(since  $1+\frac{\multError}{1-\multError} = \frac{1}{1-\multError}$)}\erel{???}\text{\eden{is it more clear?}}
        \\
        &\geq \frac{1}{1-\multError} \left(z_k +\additiveError\right) && \text{(by Equation \ref{eq:fk-to-zk}) }
    \end{align*}
    \eden{I'm not sure why to comment the lines, shouldn't we explain why it is a contradiction?how is the following?}
    % \emark{However, the approximately-optimal solution obtained for this problem during the algorithm run is $z_k$, so the optimal value is at most $\frac{1}{(1-\multError)}\left(z_k+\additiveError\right)$.
    % But, as we shall see, the objective $y$ yields in this problem, $\sum_{i=1}^k \valBy{i}{y} - \sum_{i=1}^{k-1} z_i$ (by Observation \ref{obs:obj-value}), is higher than this value, which is of course a contradiction:}
    However, the approximately-optimal value obtained for this problem during the algorithm run is $z_k$, so the optimal value is at most $\frac{1}{(1-\multError)}\left(z_k+\additiveError\right)$, which is, again, a contradiction.
    
\end{proof}

\section{Proof of Theorem \ref{th:app-main}}\label{sec:app-sec-proofs}
% \eden{should probably change the title}

% Agents are assumed to care only about their own share (allowing us to use the following abuse of notation in which $u_j$ takes a bundle $b$ of items), their utilities are assumed to be normalized ($u_j(\emptyset) = 0$), monotone ($u_j(b_1) \leq u_j(b_2)$ if $b_1 \subseteq b_2$), and submodular ($u_j(b_1) + u_j(b_2) \geq u_j(b_1 \cup b_2) + u_j(b_1 \cap b_2)$ for any bundles $b_1,b_2$).
% It is assumed that each agent assigns a positive utility to the set of all items.
% The utilities $(u_i)_{i=1}^n$ are assumed to be given in the \emph{value oracle model}, meaning that we do not have a direct access to them, but only to an oracle that indicates the value of an agent from a given simple allocation.
% % \eden{z1 > 0}

This section proves Theorem \ref{th:app-main}:
suppose we are given a randomized algorithm that returns a simple allocation that approximates the utilitarian welfare with multiplicative error $\multError$ (with success probability $p$).
Then, Algorithm \ref{alg:basic-ordered-Outcomes} can be used to obtain a stochastic allocation that approximates leximin with a multiplicative error of at most $\frac{\multError}{1-\multError +\multError^2}$ (with the same probability).

% title: the specific problem as P3
As we saw in Section \ref{sec:algo-short}, an approximation to leximin can be obtained by providing a procedure \textsf{OP} to approximate \eqref{eq:vsums-OP}  (Theorem \ref{th:main}), which, under these particular settings, becomes:
% \erel{Why do you call it "configuration LP"? I think this term refers to something else: \url{https://en.wikipedia.org/wiki/Configuration_linear_program}}
\begin{align}
&\max \quad z_t \quad s.t. \tag{\progAppFirst}\label{eq:app-vsums-OP}\\
& (\text{\progAppFirst.1.1}) \Hquad \sum_{A \in \mathcal{A}} p_d(A) = 1 \nonumber\\
& (\text{\progAppFirst.1.2}) \Hquad p_d(A) \geq 0  && \forall A \in \mathcal{A} \nonumber\\
& (\text{\progAppFirst.2}) \Hquad \ell y_{\ell} - \sum_{j=1}^n m_{\ell,j}\geq \sum_{i=1}^{\ell}  z_i && \forXinY{\ell}{t-1} \nonumber \\
& (\text{\progAppFirst.3}) \Hquad t y_t - \sum_{j=1}^{n} m_{t,j} \geq \sum_{i=1}^{t}  z_i \nonumber \\
& (\text{\progAppFirst.4}) \Hquad m_{\ell,j} \geq y_{\ell} - \sum_{A \in \mathcal{A}}p_d(A) \cdot u_j(A)  && \forXinY{\ell}{t},\Hquad \forXinY{j}{n} \nonumber \\
& (\text{\progAppFirst.5}) \Hquad m_{\ell,j} \geq 0  && \forXinY{\ell}{t},\Hquad \forXinY{j}{n} \nonumber
\end{align}
Here the variables are $p_d(A)$ for any simple allocation $A \in \mathcal{A}$, $\ztVar{}$, and $y_{\ell}$ and $m_{\ell,j}$ for all $\ell \in [t]$ and $ j\in [n]$; and the values $z_1, \ldots z_{t-1}$ are constants.
Notice that it is a \emph{linear program} that has a polynomial number of constraints thanks to \eqref{eq:vsums-OP} representation, but an exponential number of variables (since there is a variable $p_d(A)$ for each simple allocation).
So, it is unclear how to approach it directly in polynomial time.
% \eden{here?}
In addition, it means that the output size is exponential in $n$.
To deal with this issue, the solutions are considered in \emph{sparse form} --- a list of the variables with positive values, along with their values.
Accordingly, if a solution has only a polynomial number of variables with positive values it can be represented by a polynomial size.
We will later see that the procedure described in this section returns such a solution in polynomial time.
% \eden{should write something about the output size, as \cite{kawase_max-min_2020}}

% title: baseline
% \erel{I would move the following paragraph upwards}
With $t=1$, \eqref{eq:app-vsums-OP} can be viewed as the problem of egalitarian welfare maximization, indeed, Kawase and Sumita \cite{kawase_max-min_2020} who studied this problem, considered a slightly simpler representation. 
% After proving that approximating the optimal value to a factor better than $(1-\frac{1}{e})$ is NP-hard, they present a dual-based algorithm that achieves this accuracy \er{w.h.p (?)}.
We now show how their dual-based technique can be applied to approximate \eqref{eq:app-vsums-OP} for any $t\geq 1$ while maintaining the same approximation factor.


To begin, consider the following program \eqref{eq:app-ver2-vsums-OP}, which is the result of modifying \eqref{eq:app-vsums-OP} in three ways. 
First, changing the objective-function to $\min 1/z_t$ instead of $\max z_t$. 
Second, replacing all the original variables and constants, except $z_t$, with new ones that are smaller by a factor $z_t$ (that is, $p'_A = p_d(A)/z_t$ for all $A \in \mathcal{A}$, $,y'_{\ell} = y_{\ell}/z_t,m'_{\ell,j} = m_{\ell,j}/z_t$ for $\ell \in [t]$ and $ j\in [n]$,  and $z'_i = z_i/z_t$ for $i \in [t-1]$).
And third, dividing all the constraints by $z_t$ ($z_t > 0$ since $z_t \geq z_1$ for any $t \geq 1$ and  $z_1 >0$).
\eden{to myself: maybe to explain why $z_1>0$}

\begin{align}
& \min \quad 1/z_t \quad s.t. \tag{\progAppSecond}\label{eq:app-ver2-vsums-OP}\\
& (\text{\progAppSecond.1.1}) \Hquad \sum_{A \in \mathcal{A}} p'_A = 1/z_t \nonumber\\
& (\text{\progAppSecond.1.2}) \Hquad p'_A \geq 0  && \forall A \in \mathcal{A} \nonumber\\
& (\text{\progAppSecond.2}) \Hquad \ell y'_{\ell} - \sum_{j=1}^n m'_{\ell,j}\geq \sum_{i=1}^{\ell}  z'_i && \forXinY{\ell}{t-1} \nonumber \\
& (\text{\progAppSecond.3}) \Hquad t y'_t - \sum_{j=1}^{n} m'_{t,j} \geq \sum_{i=1}^{t-1}  z'_i + 1 \nonumber \\
& (\text{\progAppSecond.4}) \Hquad m'_{\ell,j} \geq y'_{\ell} - \sum_{A \in \mathcal{A}}p'_A \cdot u_j(A)  && \forXinY{\ell}{t},\Hquad \forXinY{j}{n} \nonumber \\
& (\text{\progAppSecond.5}) \Hquad m'_{\ell,j} \geq 0  && \forXinY{\ell}{t},\Hquad \forXinY{j}{n} \nonumber
\end{align}
The programs \eqref{eq:app-vsums-OP} and \eqref{eq:app-ver2-vsums-OP} are related in the following way:
% \erel{I would make this a lemma:}
\begin{lemma}\label{lemma:bijection}
There exists a bijection mapping each solution of 
\eqref{eq:app-vsums-OP} with objective value $V$ to a unique solution of 
\eqref{eq:app-ver2-vsums-OP} with objective value $1/V$.
\end{lemma}
\begin{proof}
Let $p_d(A)$ for $A \in \mathcal{A}$, $\ztVar{}$, and $y_{\ell}$ and $m_{\ell,j}$ for all $\ell \in [t]$ and $ j\in [n]$ be a feasible solution to the program \eqref{eq:app-vsums-OP} with objective value $V$.
It can be easily verified that $p'_A = p_d(A)/z_t$ for $A \in \mathcal{A}$, $z_t$, and $y'_{\ell} = y_{\ell}/z_t$ and $m'_{\ell,j} = m_{\ell,j}/z_t$ for all $\ell \in [t]$ and $ j\in [n]$ is a feasible solution to the program \eqref{eq:app-ver2-vsums-OP} with objective value $1/V$.
\end{proof}
% \eden{maybe to write something about why it is a bijection (or to write that it is straightforward)}

Denote this bijection by $\Psi$, this also implies the following:
\begin{lemma}\label{lemma:approx-acc-by-bijection}
    If a solution approximates the program \eqref{eq:app-ver2-vsums-OP} with a multiplicative error of $\frac{\multError}{1-\multError}$. Then the corresponding solution to \eqref{eq:app-vsums-OP} according to the bijection $\Psi$ approximates this program with a multiplicative error of $\multError$.
\end{lemma}

\begin{proof}
    Let $V^*$ be the optimal objective value of \eqref{eq:app-vsums-OP}. 
    By Lemma \ref{lemma:bijection}, there exists a solution to \eqref{eq:app-ver2-vsums-OP} with value $1/V^{*}$.
    This solution yields the optimal value for \eqref{eq:app-ver2-vsums-OP} --- if there was a solution that had a value \emph{lower} than $1/V^{*}$ (\eqref{eq:app-ver2-vsums-OP} is a minimization problem), then the corresponding solution to \eqref{eq:app-vsums-OP} (by the bijection $\Psi$) would have a value higher than the optimal value $V^*$.
    Now, let the value of the solution that approximates the program \eqref{eq:app-ver2-vsums-OP} with a multiplicative error of $\frac{\multError}{1-\multError}$ be $1/V$. 
    Since \eqref{eq:app-ver2-vsums-OP} is a minimization problem, assuming that $1/V$ approximates $1/V^*$ with a multiplicative error of $\frac{\multError}{1-\multError}$ means that:
    \begin{align*}
        \frac{1}{V} \leq \left(1+\frac{\multError}{1-\multError}\right)\frac{1}{V^*},
    \end{align*}
 which implies that $V \geq (1-\multError)V^*$.
    As \eqref{eq:app-vsums-OP} is a maximization problem, this means that $V$ approximates this problem with multiplicative error $\multError$.
    By Lemma \ref{lemma:bijection}, $V$ is the value of the corresponding solution to \eqref{eq:app-vsums-OP} by the bijection $\Psi$.
\end{proof}

Notice that the only constraint of \eqref{eq:app-ver2-vsums-OP} that includes the variable $z_t$, (\progAppSecond.1.1), says that $\sum_{A \in \mathcal{A}}p'_A = 1/z_t$, and also that its objective function is $\min 1/z_t$.
As a result, we can reduce the need for the variable $z_t$ by removing constraint (\progAppSecond.1.1) and changing the objective function to $\min \sum_{A \in \mathcal{A}}p'_A$.
This change makes \eqref{eq:app-ver2-vsums-OP} a \emph{linear} program.
This will allow us to approximate it using its dual, as we will see.

The following observation will be useful later:
\begin{observation}\label{obs:c2-to-c1-in-poly-time}
    If a solution to \eqref{eq:app-ver2-vsums-OP} is given in a sparse form --- a list of the variables with nonzero value and their values, then the corresponding solution to \eqref{eq:app-vsums-OP} in a sparse form can be computed in time polynomial to the number of nonzero variables.
\end{observation}
\noindent For completeness, we briefly outline the process. 
When given a list of variables with nonzero values, we first iterate the list and sum all variables of the form $p'_A$, and then set $z_t$ to be $1$ divided by this sum. 
After, for each variable $\nu'$ in the list, we set the corresponding variable, $\nu$, to $z_t \cdot \nu'$.


% title: dual 
Now, let us consider the dual program of \eqref{eq:app-ver2-vsums-OP}, which can be described as follows:
% \erel{When you present an LP, it can help the reader if you mention what exactly the variables of the LP are.}
\begin{align}
    \max &&& \sum_{\ell=1}^{t-1} q_{\ell} \sum_{i=1}^{\ell} z_i + q_t (\sum_{i=1}^{t-1} z_i +1) \tag{\progAppDual}\label{eq:app-dual}\\
        s.t. &&& (\text{\progAppDual.1}) \Hquad \sum_{j=1}^n u_j(A) \sum_{\ell=1}^t v_{\ell,j} \leq 1  && \forall A \in \mathcal{A} \nonumber\\
                    &&& (\text{\progAppDual.2}) \Hquad \ell q_{\ell} - \sum_{j=1}^n v_{\ell,j} = 0 && \forXinY{\ell}{t} \nonumber \\
                    &&& (\text{\progAppDual.3}) \Hquad q_{\ell} - v_{\ell,j} \leq 0  && \forXinY{\ell}{t},\Hquad \forXinY{j}{n} \nonumber \\
                    &&& (\text{\progAppDual.4}) \Hquad v_{\ell,j} \geq 0  && \forXinY{\ell}{t},\Hquad \forXinY{j}{n} \nonumber \\
                    &&& (\text{\progAppDual.5}) \Hquad q_{\ell} \geq 0  && \forXinY{\ell}{t} \nonumber
\end{align}
Here, the variables are $q_{\ell}$ and $v_{\ell,j}$ for any $\ell \in [t]$ and $j \in [n]$; and the constants are (as before) $z_i$ for $i \in [t-1]$.
Recall that $u_j(A)$ is the utility that agent $j$ assigns to simple allocation $A$, as given by the value oracle.
% title: ellipsoid variant
This problem has an exponential number of constraints --- a constraint for each allocation (in line (\progAppDual.1)) but only a polynomial number of variables.
Using the ellipsoid method \cite{grotschel_ellipsoid_1981}, it could be solved in polynomial time 
if we had a \emph{separation oracle} ---
an oracle that given a vector $\upsilon$ either determines that $\upsilon$ is infeasible and returns a violated constraint, or asserts that $\upsilon$ is feasible.
Unfortunately, as we shall now see, it is NP-hard to compute a separation oracle to this problem.
\begin{lemma}
    Computing a separation oracle to \eqref{eq:app-dual} is NP-hard.
\end{lemma}

% very similar to what they did in yonatan's paper..
\begin{proof}
We prove that a separation oracle for \eqref{eq:app-dual} would allow us to compute a leximin optimal stochastic allocation.
    As discussed previously, computing such an allocation is NP-hard, so the same applies for computing a separation oracle for \eqref{eq:app-dual}.

    First, we prove that such a separation oracle can be used to extract an optimal solution to \eqref{eq:app-ver2-vsums-OP}.
    Assume that the ellipsoid method was operated with the given oracle to solve \eqref{eq:app-dual}.
    Let $C$ be the set of constraints that the oracle determined as being violated.
    Since the ellipsoid method operates in polynomial time, the size of the set $C$ is also polynomial.
    Let $V_C$ be the set of variables of \eqref{eq:app-ver2-vsums-OP} associated with the constraints in $C$.
    By complementary slackness, the variables in $V_C$ are the only ones that may get a \emph{positive} value in the corresponding optimal solution to \eqref{eq:app-ver2-vsums-OP}.
    Therefore, the program \eqref{eq:app-ver2-vsums-OP} with only the variables in $V_C$ (and the other variables equal to zero) has a polynomial size, and therefore can be solved exactly.


    But, by Observation \ref{obs:c2-to-c1-in-poly-time}, this would allow us to find the corresponding optimal solution to \eqref{eq:app-vsums-OP} in polynomial time.
    % \erel{Did we say that $\psi$ can be computed in polynomial time?}\eden{in the way it is written now is not, it iterate over each variable of \eqref{eq:app-vsums-OP} and there are exponential number of them. I need to think how to write it appropriately. maybe "that can be computed in time equals to the number of positive variables"?}
    % \erel{If it is not polynomial, then the reduction is not polynomial, so it does not imply NP-hardness}
    This means the described process can be used as an approximation procedure to \eqref{eq:vsums-OP} (that became \eqref{eq:app-vsums-OP} under the settings of this problem) with $\multError = \additiveError = 0$.
    Therefore, by Theorem \ref{th:main}, this means we can use Algorithm \ref{alg:basic-ordered-Outcomes} to obtain a leximin optimal solution\footnote{Actually, Theorem \ref{th:main} says that Algorithm \ref{alg:basic-ordered-Outcomes} will output a $(1,0)$-leximin-approximation; But Lemma \ref{lemma:absence-of-errors} says that such a solution is, indeed, a leximin optimal solution.}.
\end{proof}


% --- it would allow us to compute a leximin optimal stochastic allocation, which is, as discussed previously, NP-hard.

In Appendix \ref{sec:mult-variant-ellipsoid}, we present another variant of the ellipsoid method, which allows us to approximate the program \eqref{eq:app-ver2-vsums-OP} given a \emph{half-randomized approximate separation oracle} to \eqref{eq:app-dual}.
That is, an oracle that, given a multiplicative error $\multError$, a success probability $p$, and a vector $\upsilon$, either determines that $\upsilon$ is infeasible and returns a violated constraint; or determines that $\upsilon$ is $\multError$-\textit{approximately-feasible}, which means that for any constraint $a \cdot x \leq b$, the vector $\upsilon$ satisfies $a \cdot \upsilon \leq (1+\multError)\cdot b$.
When the oracle says that $\upsilon$ is $\multError$-approximately-feasible, it is correct with probability at least $p$.
Given such an oracle for the dual program, the ellipsoid method variant can be used to output a solution to the primal, that approximates it to the same factor with probability at least $p^I$, where $I$ is an upper bound on the number of iterations in any execution of the ellipsoid method variant on the dual (if it is given a deterministic oracle).
We can therefore conclude the following result:
\begin{lemma}\label{lemma:approx-sep-oracle-to-goal}
    Given a half-randomized approximate separation oracle to the problem \eqref{eq:app-dual}, with a multiplicative error of $\frac{\beta}{1-\beta}$ and a success probability $p$, a stochastic allocation that approximates leximin to a multiplicative error $\frac{\multError}{1-\multError+\multError^2}$ can be obtained with probability $p^{nI}$.
\end{lemma}

\begin{proof}
    % To begin, assume that we are given a deterministic approximate separation oracle (i.e., with failure probability $p=0$).
    As described above, we can use the ellipsoid method variant of Appendix \ref{sec:mult-variant-ellipsoid} with the given oracle to \eqref{eq:app-dual} to obtain a solution to \eqref{eq:app-ver2-vsums-OP},  that approximates it with a multiplicative error of $\frac{\multError}{1-\multError}$ with probability $p^I$.
    Then, by Observation \ref{obs:c2-to-c1-in-poly-time}, this would allow us to find the corresponding solution to \eqref{eq:app-vsums-OP}, that, with probability $p^I$, approximates it with a multiplicative error of $\multError$.
    That is, the described process can be used as a randomized approximation procedure to \eqref{eq:vsums-OP} (that became \eqref{eq:app-vsums-OP} under the settings of this problem).
    % with $\multError = \additiveError = 0$.
    Therefore, by Theorem \ref{th:main}, Algorithm \ref{alg:basic-ordered-Outcomes} can be used to obtain a leximin approximation to the original problem with only a multiplicative error of $\frac{\multError}{1-\multError+\multError^2}$ with probability $p^{nI}$ (Corollary \ref{corollary:main-with-probability}).
\end{proof}

Now, we show that such an oracle can be designed given a randomized approximation algorithm for computing a simple allocation that approximates the utilitarian welfare. Specifically, 

\begin{lemma}\label{lemma:alg-for-utilitarian-to-sep-oracle}
    Given a randomized approximation algorithm for computing a simple allocation that approximates the utilitarian welfare with multiplicative error $\multError$ and a success probability $p$, a half-randomized approximate separation oracle to \eqref{eq:app-dual} can be designed with a multiplicative error of $\frac{\beta}{1-\beta}$ and a success probability at least $\left(1-\frac{1}{nI}(1-p)\right)$.
\end{lemma}

% \eden{should say somewhere that the oracle is polynomial time and therefore everything is?...}
% FROM HERE: https://tex.stackexchange.com/a/675333/20929
\algdef{SE}[REPEATN]{REPEATN}{ENDREP}[1]{\algorithmicrepeat\ #1 \textbf{times}}{\algorithmicend\ \algorithmicrepeat}
\begin{algorithm}[!tbp]
\caption{A Half-Randomized Approximate Separation Oracle to \eqref{eq:app-dual}}
\label{alg:sep-oracle}
INPUT: variables $q_{\ell}$ and $v_{\ell,j}$ for any $\ell \in [t]$ and $j \in [n]$, an $\multApprox$-approximation algorithm for the utilitarian welfare problem (\eqref{eq:utilitarian}) with success probability $p$.
\begin{algorithmic}[1] %[1] enables line numbers
\STATE Iterate over constraints (\progAppDual.2)-(\progAppDual.5). If one of them is  violated, stop and return it.
\STATE \textbf{If} $p=1$ then set $T:=1$; \textbf{else} set $T := 1 + \lceil-\log_{(1-p)}(nI)\rceil$.

\REPEATN{$T$}
    \STATE Operate the algorithm for the utilitarian welfare problem on $n,m,(u'_j)_{j=1}^n$ to obtain an allocation $\Tilde{A}$ with value $\nu$.
    \IF{$\nu > 1$}  
        \STATE Return the corresponding violated constraint $\sum_{j=1}^n u_j(\Tilde{A}) \sum_{\ell=1}^t v_{\ell,j} > 1$
    \ENDIF
\ENDREP
\STATE Return "the assignment is approximately-feasible".

\end{algorithmic}
\end{algorithm}


Algorithm \ref{alg:sep-oracle} describes the oracle.
It accepts as input an assignment to the variables of \eqref{eq:app-dual}, that is, $q_{\ell}$ and $v_{\ell,j}$ for any $\ell \in [t]$ and $j \in [n]$, and an algorithm for approximating the maximum utilitarian welfare.
It starts by verifying constraints (\progAppDual.2)-(\progAppDual.5) one by one (this is possible as their number is polynomial in $n$ and $m$). 
If a violated constraint was found, the oracle simply returns it. Otherwise, it proceeds to check constraints (\progAppDual.1).
Although the number of constraints described by (\progAppDual.1) is exponential in $n$, they can be treated collectively in polynomial time (as in \cite{kawase_max-min_2020}).
% \eden{here maybe to say something about the randomness}.\erel{Maybe mention that \textcite{kawase_max-min_2020} ignored this issue.}
First, notice that in order to determine whether the expression $\sum_{j=1}^n u_j(A) \sum_{\ell=1}^t v_{\ell,j}$ is at most $1$ for all simple allocations ($A \in \mathcal{A}$), it is sufficient to check the allocation that maximizes this expression and compare it to $1$.
Define new utility functions for all $j \in [n]$ and $A \in \mathcal{A}$, 
\begin{align*}
u'_j(A) := \sum_{\ell=1}^t v_{\ell,j} \cdot u_j(A) 
\end{align*}
The above expression can be simplified to $\sum_{j=1}^n u'_j(A)$. An allocation that maximizes this expression is an allocation that maximizes the utilitarian welfare (i.e., the sum of utilities) when the same sets of agents and items is considered but with different utilities%
\footnote{Notice that the utilities $u'_j$ are  normalized, monotone, submodular, and can be computed using $t\leq n$ calls to the value oracle of $u_j$}
($u'_j$ instead of $u_j$ for $j \in [n]$).
Such an allocation cannot be found in polynomial time since approximating the utilitarian welfare up to a factor better than $(1-\frac{1}{e})$ in the case of submodular utilities is known to be NP-hard \cite{khot_inapproximability_2008}.
However, the oracle is given an approximation algorithm to the utilitarian welfare problem as input.
Therefore, an allocation $\Tilde{A}$ with utilitarian value at least $(1-\multError)$ of the optimal can be obtained with probability $p$.
We shall now see that it is enough.


\begin{proof}[Proof of Lemma \ref{lemma:alg-for-utilitarian-to-sep-oracle}]
First, observe that when Algorithm \ref{alg:sep-oracle} returns a violated constraint, it is always correct.
This is obvious for constraints described by (\progAppDual.2)-(\progAppDual.5), since these constraints have been verified directly.
For constraints described by (\progAppDual.1), it means that the algorithm found an allocation $\Tilde{A}$ that satisfies $\sum_{j=1}^n u'_j(\Tilde{A}) > 1$.
    By the definition of $u'$, the constraint corresponding to this allocation is, indeed, violated:
    \begin{align*}
         \sum_{j=1}^n u_j(\Tilde{A}) \sum_{\ell=1}^t v_{\ell,j} = \sum_{j=1}^n u'_j(\Tilde{A}) > 1.
    \end{align*}
Let us assume that the given algorithm for the utilitarian welfare problem is deterministic (i.e., $p=1$) and then revisit the case $p<1$.
    Assume that the oracle said that the assignment is approximately-feasible.
    This means that the algorithm for the utilitarian welfare problem found an allocation $\Tilde{A}$ with value at most $1$.
    Since $\Tilde{A}$ is approximately-optimal, the optimal utilitarian value is at most $1/(1-\multError)\cdot 1$.
    As this is an upper bound of the utilitarian value of any allocation, it follows that all the constraints described bu (\progAppDual.1) are $\frac{\multError}{1-\multError}$-approximately maintained --- that is, for any allocation $A \in \mathcal{A}$ the following holds:
    \begin{align*}
        \sum_{j=1}^n u'_j(A) = \sum_{j=1}^n u_j(A) \sum_{\ell=1}^t v_{\ell,j} \leq \frac{1}{1-\multError}\cdot 1 = \left(1+\frac{\multError}{1-\multError}\right)\cdot1
    \end{align*}
    We get that, in this case, the oracle is also deterministic, and that the success probability is at least $\left(1-\frac{1}{nI}(1-p)\right) = 1$ for $p=1$.

    Assume now that $p<1$. Then, the oracle may be incorrect when it says the assignment is approximately feasible, but only if the algorithm for the utilitarian welfare problem did not return an appropriate approximation in all $T = \lceil-\log_{(1-p)}(nI)\rceil + 1$ operations, that is, with probability at most $(1-p)^T$.
    % as each operation of the oracle is independent
    Notice that $T>1$ since $\log_{(1-p)}(nI) < 0$\footnote{
    % The fact that  $\log_{(1-p)}(nI) < 0$ can be easily concluded 
    Since $(1-p)\in(0,1)$ and $nI>1$ by change of base: $\log_{(1-p)}(nI) = \log(nI)/\log(1-p)$, the numerator is positive and the denominator is negative.}.
    Now, as $T \geq -\log_{(1-p)}(nI) + 1$ and $(1-p)<1$ we get that:
    \begin{align*}
        &(1-p)^T \leq (1-p)\cdot(1-p)^{-\log_{(1-p)}(nI)} = (1-p)(nI)^{-1}
    \end{align*}
    So, the success probability is at least $\left(1-\frac{1}{nI}(1-p)\right)$.
\end{proof}

We can now prove Theorem \ref{th:app-main}.

\begin{proof}[Proof of Theorem \ref{th:app-main}]
    Assume we are given an algorithm that returns a simple allocation that approximates the utilitarian welfare with multiplicative error $\multError$ with success probability $p$.
    By Lemma \ref{lemma:alg-for-utilitarian-to-sep-oracle} this algorithm can be used to obtain an half-randomized approximate separation oracle to \eqref{eq:app-dual} with a multiplicative error $\frac{\multError}{1-\multError}$ with success probability $\left(1-\frac{1}{nI}(1-p)\right)$.
    By Lemma \ref{lemma:approx-sep-oracle-to-goal}, with such an oracle a stochastic allocation that approximates leximin to a multiplicative error of $\frac{\multError}{1-\multError+\multError^2}$ can be obtained with probability $\left(1-\frac{1}{nI}(1-p)\right)^{nI}$.
    If $p=1$ then the success probability is $1$ too (at least $\left(1-\frac{1}{nI}(1-p)\right)^{nI}= 1$).
    However, if $p<1$, then $\frac{1}{nI}(1-p) \in (0,1)$ and therefore the success probability is at least $p$\footnote{For any $\epsilon \in (0,1)$ and $k \in \mathbb{Z}_{+} \colon \Hquad (1 - \epsilon)^k \geq 1 - k \cdot \epsilon$}:
    \begin{align*}
        \left(1-\frac{1}{nI}(1-p)\right)^{nI} \geq \left(1-nI\cdot\frac{1}{nI}(1-p)\right) = p.   \end{align*}
\end{proof}

\section{Equivalent Single-objective Optimization Problems in the Presence of Errors}\label{sec:equivalent-proofs}

Many times, when referring to two optimization problems\footnote{In this section,  we consider only single-objective optimization problems.} as equivalent, one means that they have the same optimal value.
When two problems satisfy this relation, it is clear that in order to obtain an optimal \emph{value}, a solver\footnote{It is assumed that a solver (either approximate or exact) for a single-objective optimization problem returns a solution and its objective value.\eden{maybe to explain it better..}} for one can be used as a solver for the other. 
However, if we are interested in an optimal \emph{solution} that yields this value, a solver that returns an optimal solution for another problem with the same optimal value is not enough\eden{reduction to feasibility problem}.
Moreover, when it comes to approximation, even if we are only concerned about the objective value, an approximate solver for one can no longer be used for the other.
To illustrate, consider the following problems:
\begin{align*}
    (E1) \Hquad &\max\quad x                         &&& (E2)\Hquad &  \max\quad x\\
    &\Hquad s.t.\quad  x \in \{0.9,1\}       &&&& \Hquad s.t.\quad x \in \{0.95,1\} 
\end{align*}    
Both problems have the same optimal objective value $1$.
Now, assume that a multiplicative error of $0.1$ is acceptable.
An approximate solver for the problem $(E1)$ may return the objective value $0.9$, which is not a possible value of $(E2)$; similarly, an approximate solver for the problem $(E2)$ may return the objective value $0.95$, which is not a possible value of $(E2)$.
% Thus, although both problems have the same optimal value, an approximate solver for one problem \emph{cannot} be used as an approximate solver for the other.

In this appendix, we present a new definition of equivalent optimization problems, which requires a stronger relationship.
We prove that, according to our definition, when two optimization problems are equivalent, a solver for one, either exact or approximate, can also be used for the other.

\paragraph{Equivalent problems definition} We say that two (single-objective) optimization problems, $OP1 = (S_1,f_1)$ and $OP2 = (S_2,f_2)$, are \emph{equivalent} if they are from the same type --- either both are maximization problems or both are minimization problems; and there exists a bijection, $B \colon S_1 \to S_2$, mapping each solution of $OP1$, $x \in S_1$, to a unique solution of $OP2$, $B(x) \in S_2$, and they have the same objective value $f_1(x) = f_2(B(x))$.

% It is easy to conclude that this relation is symmetric, reflexive and transitive and therefore it is, indeed, an equivalence relation.
The following observation can be easily concluded by the definition:
\begin{observation}
    The equivalent relation between problems is transitive, reflexive and symmetric.
\end{observation}
% \begin{observation}
%     The equivalent relation between problems is transitive.
% \erel{maybe also argue that it is reflexive and symmetric, so it is an equivalence relation.}
% \end{observation}
% \eden{to explain }

If we are only concerned with the objective value, then the following lemma ensures that an approximate solver for one problem can be applied, as is, to the other (it is not necessary to know what the bijection is):

\begin{lemma}\label{lemma:approx-value-equivalent-prob}
    Let $OP1 = (S_1,f_1)$ and $OP2 = (S_2,f_2)$ be equivalent optimization problems, and let $v_1 \in \mathbb{R}$ be an $(\multApprox,\additiveApprox)$-approximation of the optimal objective value of $OP1$.
    Then, $v_1$ is also an $(\multApprox,\additiveApprox)$-approximation of the optimal objective value of $OP2$.
% \erel{Add that the approximation ratios ($\alpha$,$\epsilon$) is the same}
\end{lemma}

\begin{proof}
    For brevity, we prove the claim only for maximization problems, the proof for minimization problems is similar.
    
    Let $x^*\in S_1$ and $y^*\in S_2$ be optimal solutions of the problems $OP1$ and $OP2$ respectively.
    In order to prove that $v_1$ is an $(\multApprox,\additiveApprox)$-approximation of the optimal objective value of $OP2$, we will show that there is a solution $y \in S_2$ with objective value $v_1$, and also that $v_1 \geq \multApprox f_2(y^*) - \additiveApprox$.

    First, since $v_1$ is an $(\multApprox,\additiveApprox)$-approximation of the optimal objective value of $OP1$, there exists a solution $x \in S_1$ such that $f_1(x) = v_1$ and also $v_1 \geq \multApprox f_1(x^*) - \additiveApprox$.
    By definition of equivalent problems, the corresponding solution to $OP2$ by the bijection, $B(X) \in S_2$, has the same objective value $f_2(B(x)) = v_1$.
    
    In addition, we shall now see that both problems have the same optimal objective value.
    Let $B: S_1\to S_2$ be a bijection as described in the definition of equivalent problems.    
    So $f_1(x^*)=f_2(B(x^*))$, and $f_2(B(x^*))\leq f_2(y^*)$ by optimality of $y^*$, so $f_1(x^*)\leq f_2(y^*)$. By analogous arguments $f_2(y^*)\leq f_1(x^*)$, so in fact $f_1(x^*) = f_2(y^*)$.
    
    Therefore, we can conclude that:
    \begin{align*}
        f_2(B(x)) = v_1 \geq \multApprox f_1(x^*) - \additiveApprox =  \multApprox f_2(y^*) - \additiveApprox
    \end{align*}
    as required.
\end{proof}

% \begin{lemma}\label{lemma:solver-equivalent-prob}
%     Let $OP1 = (S_1,f_1)$ and $OP2 = (S_2,f_2)$ be equivalent optimization problems. Then, in order to approximate the optimal value, an $(\multApprox,\additiveApprox)$-approximate solver for one can be used as an $(\multApprox,\additiveApprox)$-approximate solver for the other.
% % \erel{Add that the approximation ratios ($\alpha$,$\epsilon$) is the same}
% \end{lemma}

% \begin{proof}
%     For brevity, we prove the claim only for maximization problems, the proof for minimization problems is similar.
%     Let $x^*\in S_1$ and $y^*\in S_2$ be optimal solutions of the problems $OP1$ and $OP2$ respectively.
%     Let $B: S_1\to S_2$ be a bijection as described in the definition of equivalent problems.    
%     So $f_1(x^*)=f_2(B(x^*))$, and $f_2(B(x^*))\leq f_2(y^*)$ by optimality of $y^*$, so $f_1(x^*)\leq f_2(y^*)$. By analogous arguments $f_2(y^*)\leq f_1(x^*)$, so in fact $f_1(x^*) = f_2(y^*)$.
%     % , since otherwise one of them is higher, and therefore the bijection can be used to obtain a solution to the second problem with value higher than optimal. \eden{need to rewrite it..}
%     Now, 
%     % without loss of generality, 
%     assume that we have an $(\multApprox, \additiveApprox)$-approximate solver for $OP1$, for some $\multApprox\in(0,1]$ and $\additiveError\geq 0$.
%     That is, the solver returns a solution $x \in S_1$ such that $f_1(x) \geq \multApprox \cdot f_1(x^*) - \additiveError$. 
%     Consider the corresponding solution to $OP2$ by the bijection, $B(X) \in S_2$, we know that $f_1(x_1) = f_2(B(x_1))$.
%     It follows that $B(x)$ is an $(\multApprox, \additiveApprox)$-approximation to $OP2$:
%     \begin{align*}
%         f_2(B(x)) = f_1(x) \geq \multApprox \cdot f_1(x^*) - \additiveError = \multApprox \cdot f_2(y^*) - \additiveError
%     \end{align*}
% \end{proof}

Notice that the approximation value is obtained by the corresponding solution ($B(X)$), and therefore, we can also conclude the following result:
% Therefore, if we also have access to procedures to calculate the bijection and its inverse, then we can use a solver for one problem to find the solution to the other, that is:
\begin{corollary}\label{corollary:solver-equivalent-prob}
    Let $OP1 = (S_1,f_1)$ and $OP2 = (S_2,f_2)$ be equivalent optimization problems, and let $P_{1\to 2}$ be a procedure that, given a solution to $OP1$, returns the corresponding solution to $OP2$.
    Then, an $(\multApprox, \additiveApprox)$-approximate solver for $OP1$ can be used to obtain a \emph{solution} that is an $(\multApprox, \additiveApprox)$-approximation for $OP2$.
\end{corollary}

If the procedure from $OP1$ to $OP2$ operates in polynomial time we say that $OP1$ is \emph{polynomial-time equivalent} to $OP2$. 

\eden{how is the name "polynomial-time equivalent"?}

% \eden{If we will have time: "Further, if the bijection is given and can be calculated in polynomial time, then ....}



\subsection{Relationships Between Single-Objective Problems for Leximin Optimization}
\eden{I'm not sure which title to give}

For clarity, descriptions of all the problems are provided here as well (table \ref{table:prob-des}).

\begin{table}[h!]
\begin{tabular}{l}
\hline
\\
$\begin{aligned}
     \text{(P1)}\Hquad \max \quad &\ztVar{x}  \;\;
        s.t. &\quad  & (1) \quad x \in S \\
              &     & & (2) \quad \valBy{\ell}{x}\geq z_{\ell} & \ell = 1,\ldots,t-1\nonumber \\
               &    & & (3) \quad \valBy{t}{x} \geq \ztVar{x} \nonumber  \\\\
    \text{(P2)}\Hquad\max \quad &\ztVar{x}  \;\;
        s.t. &\quad  & (1) \quad x \in S  \\
        &&& (\hat{2}) \quad \sum_{i \in F'} f_i(x) \geq \sum_{i=1}^{|F'|}  z_i & \forall F' \subseteq [n], |F'| < t \\
        &&& (\hat{3}) \quad \sum_{i \in F'} f_i(x) \geq \sum_{i=1}^{t}  z_i  & \forall F' \subseteq [n], |F'| = t\\\\
     \text{(P3)}\Hquad \max \quad &\ztVar{x}  \;\;
        s.t. &\quad  & (1) \quad x \in S  \\
                    &&& (2) \quad \ell y_{\ell} - \sum_{j=1}^n m_{\ell,j}\geq \sum_{i=1}^{\ell}  z_i & \ell = 1, \ldots,t-1 \nonumber \\
                    &&& (3) \quad t y_t - \sum_{j=1}^{n} m_{t,j} \geq \sum_{i=1}^{t}  z_i  \nonumber \\
                    &&& (4) \quad m_{\ell,j} \geq y_{\ell} - f_j(x)  & \ell = 1, \ldots,t,\Hquad j = 1, \ldots,n \nonumber \\
                    &&& (5) \quad m_{\ell,j} \geq 0  & \ell = 1, \ldots,t,\Hquad j = 1, \ldots,n \nonumber\\\\
    \text{(P2-compact)}& \\
    \max \quad &z_t  \;\;
        s.t. &\quad  & (1) \quad x \in S \\
                    &&& (\Tilde{2}) \quad \sum_{i=1}^{\ell} \valBy{i}{x} \geq \sum_{i=1}^{\ell}  z_i & \ell = 1,\ldots, t-1 \nonumber\\
                    &&& (\Tilde{3}) \quad \sum_{i=1}^{t} \valBy{i}{x} \geq \sum_{i=1}^{t}  z_i\\
\end{aligned}$\\
\\
\hline
\end{tabular}
\caption{Summary description of the problems.}
\label{table:prob-des}
\end{table}


\subsubsection{Equivalence of The Problems \eqref{eq:sums-OP} and \eqref{eq:compact-OP}}\label{sec:prob-sums-and-comp}
we prove that the \emph{identity function} is an appropriate bijection between \eqref{eq:sums-OP} and \eqref{eq:compact-OP}. Therefore, they are polynomial-time equivalent to each other. 

We start by proving the following lemma:
\begin{lemma}\label{lemma:sums-to-comp-constrants}
    For any $x \in S$, any $\ell \in [n]$ and a constant $c \in \mathbb{R}$ the following two conditions are equivalent:
    \begin{align}\label{eq:sums-to-comp-constrants}
         \forall F' \subseteq [n], |F'| = \ell \colon \sum_{i \in F'} f_i(x) &\geq c 
         \\
         \sum_{i=1}^{\ell} \valBy{i}{x}&\geq c 
    \end{align}
\end{lemma}

\begin{proof}
    For the first direction, recall that the values $ \valBy{1}{x}, \dots,  \valBy{\ell}{x}$ were obtained from $\ell$ objective functions (those who yield the smallest value).
    By the assumption, the sum of any set of function with size $\ell$ is at least $c$; therefore, it is true in particular for the functions corresponding to the values $ (\valBy{1}{x})_{i=1}^{\ell}$.
    For the second direction, assume that $\sum_{i=1}^{\ell} \valBy{i}{x}\geq c$.
    Since $ \valBy{1}{x}, \dots,  \valBy{\ell}{x}$ are the $\ell$ smallest values in $\allValues{x}$, we get that:
    \begin{align*}
       \forall F' \subseteq [n],\Hquad |F'| = \ell \colon \quad \sum_{i \in F'}f_i(x) \geq \sum_{i=1}^s \valBy{i}{x}\geq c.
    \end{align*}
\end{proof}

    Now, let $(x,z_t)$ be a solution to \eqref{eq:sums-OP}. 
    As $x$ satisfies constraint (1) of \eqref{eq:sums-OP}), it is also satisfies constraint (1) of \eqref{eq:compact-OP} (as both constraints are the same, $x \in S$).
    In addition, as $x$ satisfies constraint $(\hat{2})$ of \eqref{eq:sums-OP}, for any $\ell \in [t-1]$, 
    \begin{align*}
        \forall F' \subseteq [n], |F'| = \ell \colon \sum_{i \in F'} f_i(x) \geq \sum_{i=1}^{\ell} z_i
    \end{align*}
    by Lemma \ref{lemma:sums-to-comp-constrants}, also $\sum_{i=1}^{\ell} \valBy{i}{x} \geq \sum_{i=1}^{\ell} z_i$. Therefore, $x$ satisfies constraint $(\Tilde{2})$ of \eqref{eq:compact-OP}.
    Lastly, as $x$ and $z_t$ satisfy constraint $(\hat{3})$ of \eqref{eq:sums-OP}, 
    \begin{align*}
        \forall F' \subseteq [n], |F'| = t \colon \sum_{i \in F'} f_i(x) \geq \sum_{i=1}^{t} z_i
    \end{align*}
    again, by Lemma \ref{lemma:sums-to-comp-constrants} also $\sum_{i=1}^{t} \valBy{i}{x} \geq \sum_{i=1}^{t} z_i$.
    So, $x$ ans $z_t$   satisfy constraint $(\Tilde{3})$ of \eqref{eq:compact-OP}.
    Since we saw that $x$ ans $z_t$ satisfy all the constraints of \eqref{eq:compact-OP}, it is feasible to this problem.

    As in both problems the objective value is determined by $z_t$, it is clear that $(x,z_t)$ obtains the same objective value from both \eqref{eq:sums-OP} and \eqref{eq:compact-OP}.

    Therefore, the identity function (i.e., $B((x,z_t)) = (x,z_t)$) is an appropriate bijection and so, the problems are equivalent.



%--------------------------------------------------
\subsubsection{Equivalence of The  problems \eqref{eq:compact-OP} and \eqref{eq:vsums-OP}} We prove that these problems are equivalent by describing an appropriate bijection.
We will also see that this bijection and its inverse can be calculated in polynomial time and therefore, each problem is polynomial-time equivalent to the other.

We start with the following lemma:
\begin{lemma}\label{lemma:comp-to-p3-m-sums}
    For any $x \in S$ and any constant $c \in C$,
    \begin{align*}
        \sum_{j=1}^n \max(0, c - f_j(x) ) = \sum_{j=1}^n \max(0, c - \valBy{j}{x} )
    \end{align*}
\end{lemma}
\begin{proof}
     Let $(\pi_1, \ldots, \pi_n)$ be a permutation of $\{1,\ldots,n\}$ such that $f_{\pi_i}(x) = \valBy{i}{x}$ for any $i \in [n]$ (notice that such permutation exists by the definition of $\valBy{}{}$).
     That is, the value that $f_{\pi_i}$ obtains is the ${\pi_i}$-th smallest one in the multiset of all values $\allValues{x}$.
    Since each element in the sum $\sum_{j=1}^n \max(0, c - f_j(x))$ is affected by $j$ only through $f_j(x)$, the permutation $\pi$ allows us to conclude the following:
    \erel{This argument is not clear}\eden{better?}
    \begin{align*}
        \sum_{j=1}^n \max(0, c -f_j(x)) &= \sum_{j=\pi_1}^{\pi_n} \max(0,c -f_j(x)\\ 
        &= \sum_{j=1}^n \max(0,c -f_{\pi_i}(x)  = \sum_{j=1}^{n} \max(0,c -\valBy{j}{x})
    \end{align*}
\end{proof}



Following is Lemma \ref{lemma:comp-to-p3-mapping}, which describes a function $B$ and proves that it is a mapping from the feasible region of the problem \eqref{eq:compact-OP} to the feasible region of the problem \eqref{eq:vsums-OP}.
Then, Lemma \ref{lemma:comp-to-p3-is-bij} proves that this mapping is a bijection.
Lastly, Lemma \ref{lemma:comp-to-p3-obj} shows that the same objective value is obtained.

\begin{lemma}\label{lemma:comp-to-p3-mapping}
    Let $(x,z_t)$ be a feasible solution  to \eqref{eq:compact-OP}. Then $B((x,z_t)) = (x, z_t, (y_1,\ldots,y_n), (m_{1,1},\ldots,m_{n,n}))$ is a feasible solution to \eqref{eq:vsums-OP}, where
    \begin{align*}
        \quad y_{\ell} &:= \valBy{\ell}{x} \Hquad\forall \ell \in [n], 
        \\
        m_{\ell,j} &:= \max(0,y_{\ell} -f_j(x)) \Hquad \forall \ell \in [n], \Hquad \forall 1 \leq j \leq n 
    \end{align*}
\end{lemma}

\begin{proof}
    First, since $x$ satisfies constraint (1) of \eqref{eq:compact-OP}, it is also satisfies constraint (1) of \eqref{eq:vsums-OP} (as both constraints are the same).
    Also, as $m_{\ell,j} \geq 0$ and $m_{\ell,j} \geq y_{\ell} - f_j(x)$ for any $\ell \in [n]$ and $j \in [n]$, this assignment satisfies constraints (4) and (5) of \eqref{eq:vsums-OP}.
    
    To show that this assignment also satisfies constraints (2) and (3) of problem \eqref{eq:vsums-OP}, we first prove that for any $\ell \in [n]$ and any constant $c \in \mathbb{R}$ this assignment satisfies the following:
    \begin{align}\label{eq:comp-to-p3}
        \sum_{i=1}^{\ell} \valBy{i}{x}\geq c \Hquad \Longrightarrow \Hquad \ell y_{\ell} - \sum_{j=1}^n m_{\ell,j}\geq c
    \end{align}
    As $y_{\ell} = \valBy{\ell}{x}$, also  $m_{\ell,j} = \max(0,\valBy{\ell}{x} -f_j(x))$.
    % in this way it is easy to see that $j$ affects $m$ only through $f_j(x)$.
    And so, by Lemma \ref{lemma:comp-to-p3-m-sums}, it can also be described as $\sum_{j=1}^{n} \max(0,\valBy{\ell}{x} -\valBy{j}{x})$.
    Since $\valBy{\ell}{x}$ is the $\ell$-th smallest objective, it is clear that $\valBy{\ell}{x} - \valBy{j}{x} \leq 0$ for any $j > \ell$, and $\valBy{\ell}{x} - \valBy{j}{x} \geq 0$ for any $j \leq \ell$.
    We can now conclude that $\ell y_{\ell} - \sum_{j=1}^n m_{\ell,j}\geq c$:
    \begin{align*}
        &\ell y_{\ell} - \sum_{j=1}^n m_{\ell,j} = \ell \cdot \valBy{\ell}{x} - \sum_{j=1}^n \max(0,\valBy{\ell}{x} -\valBy{j}{x}) \\
        &= \ell \cdot \valBy{\ell}{x} - \sum_{j=1}^{\ell} \max(0,\valBy{\ell}{x} -\valBy{j}{x}) - \sum_{j=\ell+1}^n \max(0,\valBy{\ell}{x} -\valBy{j}{x}) \\
        &= \ell \cdot \valBy{\ell}{x} - \sum_{j=1}^{\ell} \left(\valBy{\ell}{x} -\valBy{j}{x}\right) - \sum_{j=\ell+1}^n 0 = \ell \cdot \valBy{\ell}{x} - \ell \cdot\valBy{\ell}{x} + \sum_{j=1}^{\ell} \valBy{j}{x}\\
        &= \sum_{j=1}^{\ell} \valBy{j}{x} \geq  c \text{~~~by assumption.}
    \end{align*}

    Now, since $x$ satisfies constraint $(\Tilde{2})$ of \eqref{eq:compact-OP}, for any $\ell \in [t-1]$, $\sum_{i=1}^{\ell} \valBy{i}{x} \geq \sum_{i=1}^{\ell} z_i$ and so by equation \ref{eq:comp-to-p3}, $\ell y_{\ell} - \sum_{j=1}^n m_{\ell,j}\geq  \sum_{i=1}^{\ell} z_i$
    Therefore, this assignment constraint (2) of problem \eqref{eq:vsums-OP}.
    In addition, as $x$ and $z_t$ satisfy constraint $(\Tilde{3})$ of \eqref{eq:compact-OP}, $\sum_{i=1}^{t} \valBy{i}{x} \geq \sum_{i=1}^{t} z_i$ and so by equation \ref{eq:comp-to-p3}, \ref{eq:comp-to-p3}, $t y_{t} - \sum_{j=1}^n m_{t,j}\geq  \sum_{i=1}^{t} z_i$.
    This means that also satisfies constraints (3) of problem \eqref{eq:vsums-OP}.
\end{proof}

\begin{lemma}\label{lemma:comp-to-p3-is-bij}
    The mapping $B$ is a bijection.
\end{lemma}

\begin{proof}
    Injective ($B(a) = B(b) \Rightarrow a = b$) is trivial since $x$ and $z_t$ are part of the solution.
    
    To prove that the mapping is surjective, we will show that for any feasible solution to \eqref{eq:vsums-OP}, that is,
    \begin{align*}
        (x \in S, z_t, y_1, \ldots, y_t, m_{1,1}, \ldots, m_{1,n}, m_{2,1}, \ldots, m_{2,n},\ldots, m_{t,1}, \ldots, m_{t,n})
    \end{align*}
    there is a feasible solution to \eqref{eq:compact-OP} that is  mapped to this solution.
    In fact, we prove that $(x,z_t)$ does.

    It is easy to see that since $x$ satisfies constraint (1) of \eqref{eq:vsums-OP}, it is also satisfies constraint (1) of \eqref{eq:compact-OP} (as both are the same).
    To show that it also satisfies constraints $(\Tilde{2})$ and $(\Tilde{3})$ of \eqref{eq:compact-OP}, we start by proving that for any $\ell \in [n]$ and any constant $c \in \mathbb{R}$:
    \begin{align}\label{eq:p3-to-comp}
         \ell y_{\ell} - \sum_{j=1}^n m_{\ell,j}\geq c
         \Hquad \Longrightarrow \Hquad \sum_{i=1}^{\ell} \valBy{i}{x}\geq c
    \end{align}
    Notice that, for any $j\in [n]$ and any $\ell \in [n]$, $m_{\ell,j} \geq y_{\ell} - f_j(x)$ by constraint (4) of \eqref{eq:vsums-OP}, and also $m_{\ell,j} \geq 0$ by constraint (5) of \eqref{eq:vsums-OP}.
    Therefore, $m_{\ell,j} \geq \max(0,y_{\ell} -f_j(x))$.
    And so, by Lemma \ref{lemma:comp-to-p3-m-sums}:
    \begin{align}\label{eq:p3-to-conp-m-sum}
        \sum_{j=1}^n m_{\ell,j} \geq  \sum_{j=1}^n \max(0,y_{\ell} -f_j(x)) = \sum_{j=1}^n \max(0,y_{\ell} -\valBy{j}{x})
    \end{align}
    Now, suppose by contradiction that $\ell y_{\ell} - \sum_{j=1}^n m_{\ell,j}\geq c$ but at the same time $\sum_{i=1}^{\ell} \valBy{i}{x}< c$ (equation \ref{eq:p3-to-comp}).
    Since $\ell y_{\ell} - \sum_{j=1}^n m_{\ell,j}\geq c$, by equation \ref{eq:p3-to-conp-m-sum} also:
    \begin{align*}
        c \leq \ell y_{\ell} - \sum_{j=1}^n m_{\ell,j} \leq \ell y_{\ell} - \sum_{j=1}^n \max(0,y_{\ell} -\valBy{j}{x})
    \end{align*}
    But, as $\sum_{i=1}^{\ell} \valBy{i}{x}< c$ this lead to contradiction:
\begin{align*}
       &\sum_{i=1}^{\ell} \valBy{i}{x} < c \leq \ell y_{\ell} - \sum_{j=1}^n \max(0,y_{\ell} -\valBy{j}{x})\\
       \Rightarrow \Hquad & \ell y_{\ell} - \sum_{i=1}^{\ell} \valBy{i}{x} - \sum_{j=1}^n \max(0,y_{\ell} -\valBy{j} {x}) > 0\\
       \Rightarrow \Hquad & \sum_{i=1}^{\ell} y_{\ell} - \sum_{i=1}^{\ell} \valBy{i}{x} - \sum_{j=1}^n \max(0,y_{\ell} -\valBy{j} {x}) > 0\\
       \Rightarrow \Hquad & \sum_{i=1}^{\ell}\left( y_{\ell} - \valBy{i}{x} \right) - \sum_{j=1}^{\ell} \max(0,y_{\ell} -\valBy{j} {x}) - \sum_{j=\ell+1}^n \max(0,y_{\ell} -\valBy{j} {x}) > 0\\
        \Rightarrow \Hquad &  \sum_{j=1}^{\ell} \underbrace{\left((y_{\ell} - \valBy{j}{x}) - \max(0,y_{\ell} -\valBy{j}{x})\right)}_{\text{each element } \leq 0} - \sum_{j=\ell+1}^n \underbrace{\max(0,y_{\ell} -\valBy{j}{x})}_{\text{each element } \geq 0} >  0\\
     \Rightarrow \Hquad & 0 > 0
   \end{align*}

    Now, as constraint (2) of problem \eqref{eq:vsums-OP} is satisfied, for any $\ell \in [t-1]$,  $\ell y_{\ell} - \sum_{j=1}^n m_{\ell,j}\geq  \sum_{i=1}^{\ell} z_i$, and so by equation \ref{eq:p3-to-comp}, also $\sum_{i=1}^{\ell} \valBy{i}{x} \geq \sum_{i=1}^{\ell} z_i$.
    This implies that $x$ satisfies constraint $(\Tilde{2})$ of \eqref{eq:compact-OP}.
    Similarly, as constraint (3) of problem \eqref{eq:vsums-OP} is satisfied,  $t y_{t} - \sum_{j=1}^n m_{t,j}\geq  \sum_{i=1}^{t} z_i$, and so by equation \ref{eq:p3-to-comp}, also $\sum_{i=1}^{t} \valBy{i}{x} \geq \sum_{i=1}^{t} z_i$.
    This implies that $x$ and $z_t$ satisfy constraint $(\Tilde{3})$ of \eqref{eq:compact-OP}.
\end{proof}


\begin{lemma}\label{lemma:comp-to-p3-obj}
    $(x,z_t)$ and $B((x,z_t))$ obtain the same objective value from the problems \eqref{eq:compact-OP} and \eqref{eq:vsums-OP} respectively.
\end{lemma}

\begin{proof}
    As in both problems the objective value is determined by $z_t$, by the definition of $B$ (the variable $z_t$ is mapped to itself), it is clear that $(x,z_t)$ and $B((x,z_t))$ obtains the same objective value from \eqref{eq:compact-OP} and \eqref{eq:vsums-OP} respectively.
\end{proof}


%--------------------------------------------------
\subsubsection{Relationship Between the Problems \eqref{eq:basic-OP} and \eqref{eq:sums-OP}} 
% Both problems are depended on a set of constants $z_1, \ldots, z_{t-1}$, 
We shall now prove Lemma \ref{lemma:alg-1-can-use-sums-exact} (Section \ref{sec:algo-short}), which says that in Algorithm \ref{alg:basic-ordered-Outcomes}, a solver for \eqref{eq:sums-OP} can be used (instead of for \eqref{eq:basic-OP}), and the algorithm will still output a leximin optimal solution.

\begin{proof}[Proof of Lemma \ref{lemma:alg-1-can-use-sums-exact}]
    Contrariwise, suppose that the returned solution, $x^*$, is not leximin optimal.
    This means that there exists a solution, $y \in S$, that leximin-preferred over it.
    That is, there exists an integer $k \in [n]$ such that:
    \begin{align*}
        \forall j < k \colon &\valBy{j}{y} = \valBy{j}{\retSol};\\
        & \valBy{k}{y} > \valBy{k}{\retSol}.
    \end{align*}
    In addition, since $x^*$ is the returned solution, it is the solution of \eqref{eq:sums-OP} that was solved in the last iteration and therefore $\sum_{i=1}^{s} \valBy{i}{s} \geq \sum_{i=1}^{s} z_i$ for any $s \in [n]$ (by constraint  $(\hat{2})$ for $s<n$ and constraint  $(\hat{3})$ for $s=n$).
    Now, consider \eqref{eq:sums-OP} that was solved in iteration $t$.
    Since $y$ is a solution ($y \in S$) it satisfies constraint (1).
    It is also easy to see that $y$ satisfies constraint $(\hat{2})$ --- for any $s \in [k-1]$:
    \begin{align*}
        &\sum_{i=1}^s \valBy{i}{y} = \sum_{i=1}^s \valBy{i}{\retSol}
        && \text{since } i\leq s<k \text{ and $y$'s def.}\\
        & \geq \sum_{i=1}^s z_i
    \end{align*}
    Moreover, since $z_t$ is a variable in this problem, it satisfies constraint $(\hat{3})$ with any $z_t \geq \sum_{i=1}^t \valBy{i}{y} - \sum_{i=1}^{t-1} z_i$.
    Therefore, it is feasible to this problem. 
    But, the objective value obtained by $y$ is higher than the optimal value, $z_t$, which is a contradiction:
    \begin{align*}
        \sum_{i=1}^t \valBy{i}{y} - \sum_{i=1}^{t-1} z_i > \sum_{i=1}^t \valBy{i}{x^*} - \sum_{i=1}^{t-1} z_i \geq \sum_{i=1}^t z_t - \sum_{i=1}^{t-1} z_i = z_t
    \end{align*}
\end{proof}
% We start by proving that, for $t \in [n]$, when the constants $z_1, \ldots, z_{t-1}$ represent the optimal values of \eqref{eq:basic-OP} in iterations $1, \ldots t$ respectively, the programs \eqref{eq:basic-OP} and \eqref{eq:sums-OP} are equivalent.

\eden{alternative: is this better?
\begin{proof}
    In Section \ref{sec:algo-sec-proofs}, it was proven that if Algorithm \ref{alg:basic-ordered-Outcomes} uses an $(\multApprox, \additiveApprox)$-approximate solver for \eqref{eq:compact-OP} as \textsf{OP}, then the returned solution is an $(\multApprox, \additiveApprox)$-approximation to leximin. 
    This means that, given an exact solver to \eqref{eq:compact-OP}, the algorithm will output a leximin optimal solution.
    However, we saw that \eqref{eq:sums-OP} and \eqref{eq:compact-OP} are equivalent and that the identity function is an appropriate bijection (Section \ref{sec:prob-sums-and-comp}).
    Therefore, in each iteration, a solver for \eqref{eq:sums-OP} will output the same solution and the same result will be obtained.
\end{proof}
}

\eden{in the next version we can also prove it in a maybe more interesting way.. that when the constants $z_1, \ldots, z_{t-1}$ represent the optimal values of \eqref{eq:basic-OP} the programs are equivalent}

\section{Ellipsoid Method Variant for Approximation}\label{sec:mult-variant-ellipsoid}
This Appendix describes a variant of the ellipsoid method that can be used to approximate  LPs that cannot be solved directly due to a large number of variables.
% It requires an approximate separation oracle for the dual program.
The method combines techniques presented in \cite{grotschel_geometric_1993,grotschel_ellipsoid_1981,karmarkar_efficient_1982}.

\subsection{Using Approximate Separation Oracles (multiple error)}
Our goal is to solve the following linear program (the primal):
\begin{align}
\tag{P}
\begin{split}
\min \quad &c^T \cdot x \\
s.t. \quad &A \cdot x \geq b, \quad x\geq 0;
\end{split}
\end{align}
We assume that (P) has a small number of constraints, but may have a huge number of variables, so we cannot solve (P) directly. We consider its \emph{dual}:
\begin{align}
\tag{D}
\begin{split}
\max \quad & b^T \cdot y \\
s.t. \quad &A^T \cdot y \leq c,\quad y\geq 0.
\end{split}
\end{align}
Assume that both problems have optimal solutions and denote the optimal solutions of (P) and (D) by $x^{*}$ and $y^{*}$ respectively. By the strong duality theorem:
\begin{align}
    c^T \cdot x^{*} = b^T \cdot y^{*}
\end{align}

While (D) has a small number of variables, it has a huge number of constraints, so
we cannot solve it directly either. 
In this Appendix, we show that it can be approximately using the following tool:

\begin{definition}
An \emph{approximate separation oracle} with multiplicative error (MASO) for the dual LP is an efficient function parameterized by a constant $\multError \geq 0$.
Given a vector $y$  it returns one of the following two answers:
\begin{enumerate}
\item "$y$ is infeasible". In this case, is returns a violated constraint, that is, a row $a_i^T \in A^T$ such that $a_i^T  y > c_i$.
\item "$y$ is \emph{approximately feasible}". 
That means that $A^T y \leq (1+\multError) \cdot c$
\end{enumerate}

\end{definition}
Given the MASO, we apply the ellipsoid method as follows (this is just a sketch
to illustrate the way we use the MASO; it omits some technical details):
\begin{itemize}
    \item Let $E_0$ be a large ellipsoid, that contains the entire feasible region, that is, all $y \geq 0$ for which $A^T y \leq c$.

    \item For $k = 0,1,\dots, K$ (where $K$ is a fixed constant, as will be explained later):
    \begin{itemize}
        \item Let $y_k$ be the centroid of ellipsoid $E_k$.
        
        \item Run the MASO on $y_k$.
        
        \item If the MASO returns "$y_k$ is infeasible" and a violated constraint $a_i^T$, then make a \emph{feasibility cut} --- keep in $E_{k+1}$ only those $y \in E_k$ for which $a_i^T y \leq c_i$.
        
        \item If the MASO returns "$y$ is approximately feasible", then make an \emph{optimality cut} --- keep in $E_{k+1}$ only those $y \in E_k$ for which $b^T y \geq b^T y_k$.
    \end{itemize}
    
    \item From the set $y_0, y_1, \dots, y_K$, choose the point with the highest $b^T \cdot y_k$ among all the approximately-feasible points.
\end{itemize}
Since both cuts are through the center of the ellipsoid, the ellipsoid dilates by a factor of at least $\frac{1}{r}$ at each iteration, where $r > 1$ is some constant (see \cite{grotschel_ellipsoid_1981} for computation of $r$). Therefore, by choosing $K := \log_2 r \cdot L$, where $L$ is the
number of bits in the binary representation of the input, the last ellipsoid $E_K$ is so small that all points in it can be considered equal (up to the accuracy of the binary representation).


The solution $y'$ returned by the above algorithm satisfies the following two conditions:
\begin{equation} \label{mult:y-star-is-approximetly-feasible}
     A^T y' \leq (1+\multError)\cdot c
\end{equation}
\begin{equation} \label{mult:y-star-obj-geq-opt}
     b^T y' \geq b^T y^{*}
\end{equation}
Inequality \ref{mult:y-star-is-approximetly-feasible} holds since, by definition, $y'$ is approximately-feasible.

To prove \ref{mult:y-star-obj-geq-opt}, suppose by contradiction that $b^T y^{*} > b^T y'$. 
Since $y^{*}$ is feasible for (D), it is in the initial ellipsoid. 
It remains in the ellipsoid throughout the algorithm: it is removed neither by a feasibility cut (since it is
feasible), nor by an optimality cut (since its value is at least as large as all values used for optimality cuts).
Therefore, it remains in the final ellipsoid, and it is chosen as the highest-valued feasible point rather than $y'$ --- a contradiction.

Now, we construct a reduced version of (D), where there are only at most $K$ constraints --- only the constraints used to make feasibility cuts.
Denote the reduced constraints by $A_{red}^T \cdot y \leq c_{red}$, where $A_{red}^T$ is a matrix containing a subset of at most $K$ rows of of $A^T$, and $c_{red}$ is a vector containing the corresponding subset of the elements of $c$. The reduced-dual LP is:
\begin{equation}
\tag{RD}
\begin{split}
\max  \quad & b^T y \\
s.t. \quad & A_{red}^T \cdot y \leq c_{red}, \quad y\geq 0
\end{split}
\end{equation}
Notice that it has the same number of variables as the program (D). Further, if we had run this ellipsoid method variant on (RD) (instead of (D)), then the result would have been exactly the same --- $y'$.
Therefore, (\ref{mult:y-star-obj-geq-opt}) holds for the (RD) too:
\begin{equation} \label{mult:y-star-to-y-redopt}
    b^T y' \geq b^T y^{*}_{red}
\end{equation}
where $y^{*}_{red}$ is the optimal value of (RD).


As $A_{red}^T$ contains a subset of at most $K$ rows of $A^T$, the matrix $A_{red}$ contains a subset of \emph{columns} of $A$.
Therefore, the dual of (RD) has only at most $K$ variables, which are those who correspond to the remaining columns of $A$:
\begin{equation}
	\tag{RP}
    \begin{split}
     \min \quad &c_{red}^T \cdot x_{red} \\
            s.t. \quad &A_{red} \cdot x_{red} \geq b, \quad x_{red}\geq 0
    \end{split}
\end{equation}
%  reduced-primal
%\er{Note that $A_{red}$ is a matrix with the same number of rows as $A$, but only at most $K$ columns.}
Since (RP) has a polynomial number of variables  and constraints, it can be solved exactly by any LP solver (not necessarily the ellipsoid method).
Denote the optimal solution by $x^{*}_{red}$. 

Let $x'$ be a vector which describes an assignment to the variables of (P), in which all variables that exist in (RP) have the same value as in $x^{*}_{red}$, and all other variables are set to $0$.
It follows that $A \cdot x' = A_{red} \cdot x^{*}_{red}$, therefore, since $x^{*}_{red}$ is feasible to RD, also $x'$ is a feasible solution to (P).
\erel{
In second reading, I think this should be made more formal.
Let $x'$ be a solution to (P), in which all variables that exist in (RP) have the same value as in $x^{*}_{red}$, and all other variables are set to 0.
We have to prove that 
(1) $x'$ is feasible for (P);
(2) $c^T x' \leq (1+\epsilon)\cdot c^T\cdot x^{*}$.
}
\eden{better?}
Similarly, $c^T \cdot x' = c^T_{red} \cdot x^{*}_{red}$.
We shall now see that this implies that the objective obtained by $x'$ approximates the objective obtained by $x^{*}$:
\begin{align*} 
&c^T \cdot x' = c^T_{red} \cdot x^{*}_{red} \\
&=  b^T \cdot y^{*}_{red} & \text{(by strong duality for the reduced LPs)} \\
                     &\leq  b^T\cdot y' & \text{(By (\ref{mult:y-star-to-y-redopt}))}\\
                     &\leq  (A \cdot x^{*})^T y' & \text{(definition of (P))} \\
                     &=  (x^{*})^T (A^T\cdot y') & \text{(properties of transpose and associativity of multiplication)} \\
                     &\leq  (x^{*})^T ((1+\multError)\cdot c) & \text{(by \ref{mult:y-star-is-approximetly-feasible})} \\
                     & = (1+\multError) \cdot (c^T x^{*}) & \text{(properties of transpose)}
\end{align*}
So, $x'$ ($x^{*}_{red}$ with all missing variables set to $0$) is an approximate solution to the primal LP (P) --- as required.

\subsection{Using Half-Randomized Approximate Separation Oracles}
Here, we allow the oracle to also be \emph{half-randomized}, that is, when it says that a solution is infeasible, it is always correct; however, when it says that a solution is approximately feasible, it is only correct with some probability $p \in [0,1]$.

Since the ellipsoid method variant is iterative, and since the oracle calls are independent, there is a probability $p^T$ that the oracle answers correctly in each iteration, and so, the overall process performs as before. 
We shall now explain why, using a half-randomized oracle, this ellipsoid method variant \emph{always} returns a feasible solution to the primal (even if the oracle was incorrect).

First, notice that the oracle is always correct when it determines that a solution is infeasible.
In addition, the construction of RD is only depended by these set of constraints.
Therefore, by the same arguments, $x'$ would still be a feasible solution to P (but not necessarily with an approximately-optimal objective value).

This means that given a half-randomized approximate separation oracle for the dual with error $\multError$ and success probability $p$, this ellipsoid method variant can be used as a randomized approximation algorithm for the primal with the same error and success probability $p^I$ (where $I$ is an upper bound on the number of iteration of the method on the given input). 
% \section{Saturation Algorithm}\label{sec:saturation-algorithm}
The following algorithm was independently proposed by different researchers for different problems \cite{willson,airiau_portioning_2019,nace_max-min_2008}.
% --- by Willson \cite{willson} for the  problem of fair allocation of divisible items, Airiau et al. \cite{airiau_portioning_2019} for the problem of portioning with ordinal preferences, Bei at el. \cite{bei_truthful_2022} for a variant of cake cutting that they called cake sharing and Nace and Pioro \cite{nace_max-min_2008} for multi-commodity flow problem 
But it can be generalized to capture the following case:
\begin{enumerate}
    \item The feasible region $S$ is \textit{convex}: for any two solutions $x, y \in S$ and for any $\lambda \in [0,1]$, the convex combination of $x$ and $y$ in relation to $\lambda$ is also a solution:
    \begin{align*}
        \forall x, y \in S, \quad \forall \lambda \in [0,1] \colon \quad  
        \bigl(\lambda x + (1-\lambda)y\bigl)\in S
    \end{align*}

    \item The size of the feasible region $S$ is polynomial with respect to $n$.
    % \eden{To myself: to check if this is accurate: i.e., it can be described with a number of variables and constraints that is polynomial to $n$.}

    \item The objective-functions are \textit{additive}: let $x,y,z \in S$ be solutions for which $\alpha,\beta \in \mathbb{R}$ exist such that $z = \alpha x + \beta y$, then for each objective-function $f_i \in \allObjFunc$:
    \begin{align*}
        f_i(z) &= f_i(\alpha x + \beta y) =\\
        &= \alpha f_i(x) + \beta f_i(y)
    \end{align*}
    \erel{ 
    For this condition, we must say that the solutions are vectors (we did not say this so far). Otherwise there is no meaning to adding or multiplying by scalars.
    }

    \item The objective-functions are \textit{concave}: for any objective-function $f_i \in \allObjFunc$ the set $\{f(x) \mid x \in S\}$ is concave (equivalently, the set $\{-f(x) \mid x \in S\}$ is convex). 

    \item There is a black-box for finding  the \textit{next maximin} value (denote by $OP1$): given a subset of objective-functions ($\mathcal{A}\subset \allObjFunc$) for which lower bounds have been set ($\forall f_i \in \mathcal{A} \colon z_i \in \mathbb{R}$), finds the highest value that all other objective functions can achieve simultaneously:
    \begin{align*}
        \max \quad &z\\
        s.t. \quad  & x \in S\\
                    & f_i(x) = z_i   & f_i \in \mathcal{A}\\
                    & f_i(x) \geq z   & f_i \notin \mathcal{A}
    \end{align*}

    \item There is a black-box for solving a saturation test (denote by $OP2$):
    % \eden{I think we should name this process, but I'm not sure if it is the best name...}: 
    For each objective-function $f_k \in \allObjFunc$, a single-objective optimization version of the problem with lower bounds on the values of the other objectives ($\forall f_i \in \mathcal{A} \colon z_i \in \mathbb{R}$ and $z \in \mathbb{R}$):
    \begin{align*}
    \max \quad &f_i(x)\\
    s.t. \quad  & x \in S\\
                    & f_i(x) = z_i   & f_i \in \mathcal{A}\\
                    & f_i(x) \geq z   & f_i \notin \mathcal{A}
    \end{align*}
\end{enumerate}
The algorithm is described in detail (in our terms and notations) in Algorithm \ref{alg:willson-leximin}. 


\begin{algorithm}[!htbp]
\caption{Saturation Algorithm--- for finding the Leximin optimal solution}
\label{alg:willson-leximin}
% \textbf{Input}: A black-box for OP1 and a black-box for OP2\\
% \textbf{Output}: The Lexical optimal solution
\begin{algorithmic}[1] %[1] enables line numbers 
\STATE Initialize the set of \textit{saturated} objective-functions $\mathcal{A} = \{\}$ and initialize $t=0$ (a step counter).

\STATE increase $t$ ($t = t+1$).

\STATE Use the black-box for $OP1$ to solve the following  problem, where the variables are $x$ (a vector) and $v$ (a scalar): 
\begin{align*}
\max \quad &v\\
        s.t. \quad  & x \in S\\
                    & f_i(x) \geq z_i   & f_i \in \mathcal{A}\\
                    & f_i(x) \geq v   & f_i \notin \mathcal{A}
\end{align*}
Let $x_t$ and $v_t$ be the optimal solution. 
    
\FOR{$f_k \notin \mathcal{A}$}
    \STATE Use the black-box for $OP2$ to solve the following problem, where the variables are $x$ (a vector) and $v$ (a scalar):
    \begin{align*}
    \max \quad & v\\
            s.t. \quad  & x \in S\\
                        & f_i(x) \geq z_i   & f_i \in \mathcal{A}\\
                        & f_i(x) \geq v_t   & f_i \notin \mathcal{A}\\
                        & f_k(x) \geq v
    \end{align*}
    Let $x_t^k$ and $v_t^k$ be the optimal solution. 

    \STATE \textbf{if} $v_t^k = v_t$ \textbf{then} set $f_k$ as \textit{saturated}: add it to $\mathcal{A}$ ($\mathcal{A} = \mathcal{A} \cup \{f_k\}$) and set its value to $v_t$ ($z_k = v_t$).
    % \IF{$z_{max}^k = z_{max}$}
        % \STATE Set $f_k$ as saturated: add it to $\mathcal{A}$ ($\mathcal{A} = \mathcal{A} \cup \{f_k\}$) and set its value to $z_{max}$ ($z_k = z_{max}$).
    % \ENDIF
\ENDFOR
\STATE \textbf{if} $|\mathcal{A}| = n$ \textbf{then} return $x_t$ \textbf{else} Goto line 2.
% \IF{$|\mathcal{A}| = n$}
    % \STATE return $x$ \eden{To myself: the return part of all algorithms should be explain better}
% \ENDIF
\end{algorithmic}
\end{algorithm}

The algorithm keeps a set of objective-functions that are saturated ($\mathcal{A}$) and lower bounds on their values ($\forall f_i \in \mathcal{A} \colon z_i \in \mathbb{R}$). 
The set is initially empty. 
At each iteration, at least one function becomes saturated and its lower bound is set.
When all functions become saturated, the algorithm terminates.
Each iteration of the algorithm can be divided into two parts.
In the first part, the first black-box is used to find the \textit{next max-min} value, which is the maximum value that all functions outside of $\mathcal{A}$ can achieve at the same time, given that all functions within $\mathcal{A}$ achieve their lower bounds.
In the second part, \textit{a saturation test} is made.
% \eden{I think we should name this process, but I'm not sure if it is the best name...}
For every function not in $\mathcal{A}$, the second black-box is used to find the maximum value of this function when all saturated functions ($f_i \in \mathcal{A}$) achieve their lower bounds and all other functions (outside of $\mathcal{A}$) achieve at least the max-min value from the first part.
This value is used to determine if this objective function is saturated, that is, if its maximal value from the saturation test is equal to the max-min value obtained in the first fart.
If so, we add it to the set of saturated objective-functions ($\mathcal{A}$) and set its lower bound to this value.

% \section{Additive Variant}\label{sec:additive}

\begin{theorem}\label{thm:leximin-approx-alg-leximin-opt}
    Let $\epsilon \in [0,1]$ and \textsf{OP} be a procedure that outputs a $\epsilon$ \emph{additive} approximation to \eqref{eq:vsums-OP}. Then Algorithm \ref{alg:basic-ordered-Outcomes} outputs a $\epsilon$ additive-approximate Leximin solution.  
\end{theorem}

\begin{proof}
    Contrariwise, suppose that $\retSol$ is not an $\epsilon$-approximately Leximin-optimal solution.
    This means that there exists a solution $y$ that is $\epsilon$-preferred over $\retSol$.
    That is, there exists an integer $k \in [n]$ such that:
    \begin{align*}
        \forall j < k \colon &\valBy{j}{y} \geq \valBy{j}{\retSol};\\
        & \valBy{k}{y} > \valBy{k}{\retSol} + \epsilon.
    \end{align*}
    We get that for all $s \in [k-1]$:
    \begin{align*}
         &\sum_{i=1}^s \valBy{i}{y} \geq \sum_{i=1}^s \valBy{i}{\retSol}
        && \text{since } i\leq s<k \text{ and $y$'s def.}\\
        & \geq \sum_{i=1}^s z_i && \text{constraint (2) for $t=n$.}
    \end{align*}
    Therefore, $y$ is a solution to the OP that was solved when $t = k$.
    \erel{You proved that $y$ satisfies constraint (2), but what about constraint (3)?}
    \eden{I'm not sure how to explain that constraint (3) is not a \textbf{standard} constraint. It determines the objective value $z$, so although it is not always optimal, it is always valid.}
    
    In addition, either $k<n$ or $k=n$. 
    If $k<n$ then constraint (2) for $t=n$ says that:
    \begin{align}\label{equ:approx-sum-k-geq-z-1}
        \sum_{i=1}^k \valBy{i}{\retSol} \geq \sum_{i=1}^k z_i
    \end{align}
    If $k=n$ then since $z=z_n$ constraint (3) says it.
    In both cases, we know that equation \ref{equ:approx-sum-k-geq-z-1} holds.
    
    And so, we get that:
    \begin{align*}
         \sum_{i=1}^k \valBy{i}{y} &= \sum_{i=1}^{k-1} \valBy{i}{y} + \valBy{k}{y}\\
         &  \geq\sum_{i=1}^{k-1}\valBy{i}{\retSol} + \valBy{k}{y} &&  \text{since } i \leq k-1 < k \text{ and $y$'s def.}\\
        & > \sum_{i=1}^{k-1}\valBy{i}{\retSol} + \valBy{k}{\retSol} + \epsilon &&  \text{$y$'s def. for } k
        \\
        & = \sum_{i=1}^{k}\valBy{i}{\retSol} + \epsilon \\
        & \geq \sum_{i=1}^{k} z_i + \epsilon &&  \text{equation } \ref{equ:approx-sum-k-geq-z-1}
    \end{align*}
    Which simply means that:
    \begin{align}\label{equ:sum-y-geq-sum-z-plus-eps}
         \sum_{i=1}^k \valBy{i}{y} > \sum_{i=1}^{k} z_i +\epsilon
    \end{align}
    That means that the $z$ achieved by the solution $y$ in the OP that was solved when $t = k$ is strictly more than the value we achieved $z_k$ plus $\epsilon$:
    \begin{align*}
        &\sum_{i=1}^k \valBy{i}{y} - \sum_{i=1}^{k-1} z_i && \text{insulated } z\\
        &> \sum_{i=1}^{k} z_i + \epsilon - \sum_{i=1}^{k-1} z_i  && \text{equation } \ref{equ:sum-y-geq-sum-z-plus-eps} \\
        &= z_k + \epsilon
    \end{align*}
    But we know that the error in this OP is at most $\epsilon$ --- a contradiction.
\end{proof}

\end{document}
