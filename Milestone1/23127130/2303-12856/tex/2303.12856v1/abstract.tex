A fundamental problem in quantum physics is to encode functions that are completely anti-symmetric under permutations of identical particles. The Barron space consists of  high-dimensional functions that can be parameterized by infinite neural networks with one hidden layer. By explicitly encoding the anti-symmetric structure, we prove that the anti-symmetric functions which belong to the Barron space can be efficiently approximated with sums of determinants. This yields a factorial improvement in complexity compared to the standard representation in the Barron space and provides a theoretical explanation for the effectiveness of determinant-based architectures in ab-initio quantum chemistry.
