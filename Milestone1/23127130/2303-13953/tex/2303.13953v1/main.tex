\documentclass[10pt,twocolumn,letterpaper]{article}

% If you comment hyperref and then uncomment it, you should delete
% egpaper.aux before re-running latex.  (Or just hit 'q' on the first latex
% run, let it finish, and you should be clear).
\usepackage[pagebackref=true,breaklinks=true,letterpaper=true,colorlinks,bookmarks=false]{hyperref}

\usepackage{iccv}
\usepackage{times}
\usepackage{epsfig}
\usepackage{graphicx}
\usepackage{amsmath}
\DeclareMathOperator*{\argmin}{argmin}
\usepackage{amssymb}

\usepackage{caption}
\usepackage{booktabs}
\usepackage{gensymb}
\usepackage[normalem]{ulem}
\usepackage{bbding}

\newcommand{\copyrightgoogle}{\href{https://www.google.com/help/terms_maps/}{\textcopyright 2023 Google}\xspace}


% Support for easy cross-referencing
\usepackage[capitalize]{cleveref}
\crefname{section}{Sec.}{Secs.}
\Crefname{section}{Section}{Sections}
\Crefname{table}{Table}{Tables}
\crefname{table}{Tab.}{Tabs.}

% Include other packages here, before hyperref.
\definecolor{purple}{rgb}{1, 0, 1}

\newcommand{\ie}{\emph{i.e.,}\xspace}
\newcommand{\eg}{\emph{e.g.,}\xspace}
\newcommand{\abr}{\emph{abbr.}\xspace}
\newcommand{\ea}{\emph{et al.}\xspace}
\newcommand{\gensync}{\emph{GenSync}\xspace}
\newcommand{\colosseum}{\emph{Colosseum}\xspace}
\newcommand{\srep}{\emph{SREP}\xspace} % Set Reconciliation Enhances
\newcommand{\srepsim}{\emph{SREPSim}\xspace}
% Propagation
\newcommand{\esrep}{\emph{E-SREP}\xspace}
\newcommand{\epsrep}{\emph{EP-SREP}\xspace}
\newcommand{\mesrep}{\emph{ME-SREP}\xspace}
\newcommand{\mempoolsync}{\emph{MempoolSync}}

\newcommand{\fref}[1]{Fig.~\ref{#1}}
\newcommand{\tref}[1]{Table~\ref{#1}}
\newcommand{\aref}[1]{Algorithm~\ref{#1}}
\newcommand{\procref}[1]{Procedure~\ref{#1}}
\newcommand{\sref}[1]{Section~\ref{#1}}
\newcommand{\lineref}[1]{line~\ref{#1}}
\newcommand{\appref}[1]{Appendix~\ref{#1}}

% Change \eqref
\LetLtxMacro{\originaleqref}{\eqref}
\renewcommand{\eqref}{Eq.~\originaleqref}

% Theorems and corollaries
\newcounter{theoremcount}
\setcounter{theoremcount}{0}
\DeclareRobustCommand{\theorem}[1]{%
  \refstepcounter{theoremcount}%
  \noindent\textit{\textbf{Theorem \thetheoremcount\label{theorem:#1}: }}%
}
\DeclareRobustCommand{\theoremref}[1]{Theorem~\ref{theorem:#1}}

\DeclareRobustCommand{\proof}{\emph{Proof:}\xspace}
\DeclareRobustCommand{\qqed}{\hfill$\blacksquare$}

\newcounter{corollcount}
\setcounter{corollcount}{0}
\DeclareRobustCommand{\coroll}[1]{%
  \refstepcounter{corollcount}%
  \noindent\textit{\textbf{Corollary \thecorollcount\label{coroll:#1}: }}%
}
\DeclareRobustCommand{\corollref}[1]{Corollary~\ref{coroll:#1}}

\newcounter{lemmacount}
\setcounter{lemmacount}{0}
\DeclareRobustCommand{\lemma}[1]{%
  \refstepcounter{lemmacount}%
  \noindent\textit{\textbf{Lemma \thelemmacount\label{lemma:#1}: }}%
}
\DeclareRobustCommand{\lemmaref}[1]{Lemma~\ref{lemma:#1}}

\newcounter{definitioncount}
\setcounter{definitioncount}{0}
\DeclareRobustCommand{\definition}[1]{%
  \refstepcounter{definitioncount}%
  \noindent\textit{\textbf{Definition \thedefinitioncount\label{definition:#1}: }}%
}
\DeclareRobustCommand{\defref}[1]{Definition~\ref{definition:#1}}

%notes of different authors
\newif\ifnotes
\notestrue
\notesfalse

\newif\ifdiff
\difftrue
\difffalse

\newcommand{\anote}[1]{\ifnotes $\ll$\textsf{\textcolor{purple}{Ari: {#1}}}$\gg$ \fi}
\newcommand{\nnote}[1]{\ifnotes $\ll$\textsf{\textcolor{orange}{Novak: {#1}}}$\gg$ \fi}
\newcommand{\diff}[1]{\ifdiff\textcolor{orange}{#1}\else#1\fi}

%%% Local Variables:
%%% mode: latex
%%% TeX-master: "main"
%%% End:




 \iccvfinalcopy % *** Uncomment this line for the final submission

\def\iccvPaperID{5711} % *** Enter the ICCV Paper ID here
\def\httilde{\mbox{\tt\raisebox{-.5ex}{\symbol{126}}}}

% Pages are numbered in submission mode, and unnumbered in camera-ready
\ificcvfinal\pagestyle{empty}\fi

\begin{document}

%%%%%%%%% TITLE
\title{AssetField: Assets Mining and Reconfiguration \\ in \Representation}
\author{Yuanbo Xiangli$^1$$^*$ , Linning Xu$^1$$^*$ , Xingang Pan$^{3}$, Nanxuan Zhao$^{4}$, Bo Dai$^{2}$\Envelope, Dahua Lin$^{1,2}$\\
	$^{1}$ The Chinese University of Hong Kong \quad
	$^{2}$ Shanghai AI Laboratory \\
	$^{3}$ Max Planck Institute for Informatics \quad 
	$^{4}$ Adobe Research \\
	{\tt\small \{xy019,xl020,dhlin\}@ie.cuhk.edu.hk}~
	{\tt\small xpan@mpi-inf.mpg.de}~ \\
	{\tt\small nanxuanzhao@gmail.com}~
	{\tt\small daibo@pjlab.org.cn}
}

%\maketitle
% Remove page # from the first page of camera-ready.


\twocolumn[{%
	\renewcommand
	\twocolumn[1][]{#1}%
	\maketitle
	\begin{center}
		\centering
		\vspace{-25pt}
		\includegraphics[width=0.95\textwidth]{figs/teaser-new_0308_v2.pdf}
		\vspace{-10pt}
		\captionof{figure}{\small
			Man-made environments are often characterized by repetitive scene objects,~\eg tables, chairs, and trees. \emph{AssetField} represents these environments with a set of informative ground feature planes aligning with the physical ground, from which neural representations of scene objects are extracted and grouped into categories. The proposed mechanism allows users to manipulate and compose assets directly on the ground feature plane and produces high-quality rendering on novel scene configurations.
		}
		\vspace{-5pt}
		\label{fig:teaser}
	\end{center}
}]

\ificcvfinal\thispagestyle{empty}\fi

\input{sections/0_abs.tex}
\section{Introduction}
\label{sec:intro}
\begin{figure}[t]
\begin{center}
    \includegraphics[width=1\linewidth]{figures/teaser.pdf}
\end{center}
\vspace{-0.1in}
\caption{\textbf{{\em Foggy} vs {\em Clear} NeRF.} Our \ournerf gets rid of reconstruction errors manifested as foggy ``floaters" in the density volume without additional input or significant computational overhead. 
%
Below are density profiles along a given ray before and after our geometry correction procedure, where we discard density peaks corresponding to floaters.
}
\label{fig:teaser}
\vspace{-0.2in}
\end{figure}



%The emergence of 
Neural Radiance Fields (NeRFs)~\cite{mildenhall2020nerf}  %and its variants 
have made revolutionary contributions in %photo-realistic 
novel view synthesis~\cite{barron2021mip,barron2022mip}, 
autonomous driving~\cite{rematas2022urban,tancik2022block}, digital human~\cite{hong2022headnerf,zhao2022humannerf}, and 3D content generation~\cite{eg3d,poole2022dreamfusion,lin2022magic3d}.
%by leveraging a multi-layer perceptron (MLP) to implicitly model the mapping from input 5D coordinates (i.e., 3D coordinates $\mathbf{x} = (x,y,z)$ and 2D viewing directions $\mathbf{d}=(\theta,\phi)$) to volume density $\sigma$ and view-dependent emitted radiance color $\mathbf{c} = (r,g,b)$. 
%
%They then use traditional volume rendering mechanisms on the obtained continuous 5D function (i.e., MLP) to generate novel views. 
To date, unfortunately, most NeRF-based methods encounter challenges when tackling large-scale cluttered scenes (e.g., Fig.~\ref{fig:teaser}):
\begin{enumerate}[leftmargin=0.16in, topsep=2pt,itemsep=-1ex,partopsep=1ex,parsep=1ex]
\item Input observations used for NeRF are often too sparse  compared to forward-facing or synthetic looking-inward scenes;
%\item Recovering fine-grained objects within a large volume is challenging for NeRF; %in capturing details accurately.
\item View-dependent visual effects give rise to ambiguity, resulting in a ``foggy" density field as shown in Fig.~\ref{fig:teaser}. 
%
Such artifacts are particularly pronounced in indoor scenes strewn with view-dependent appearances, such as specular highlights, glossy surface reflections from man-made objects. 
\end{enumerate}

Despite attempts to enhance NeRF's rendering quality given suboptimal input, such as using 3D conical frustums~\cite{barron2021mip,barron2022mip}, physically-grounded augmentations~\cite{chen2022aug}, and misalignment correction~\cite{jiang2022alignerf},  these challenges have yet to be fully resolved.
%
Depth supervision~\cite{deng2022depth, wei2021nerfingmvs} or proxy geometry~\cite{xu2021scalable,wu2022scalable} images can help alleviate the challenges in handling large-scale with sparse input, at the expense of %but they come at the cost of requiring 
expensive pre-processing or additional input.
%
Another line of work~\cite{wang2021neus, oechsle2021unisurf, wang2022neuris} achieves better reconstruction of surface geometry by using signed distances instead of volume density as scene representation. However, they sacrifice the ability to synthesize photo-realistic novel views.

%We observe that NeRF has been suffering from foggy ``floater" artifacts in large-scale cluttered scenes.
%
%Such artifacts are particularly pronounced in indoor scenes strewn with view-dependent appearances from man-made objects. 
%
To address the above issues, we propose an extension to NeRF, dubbed as {\bf \ournerf}, which enforces effective {\em appearance} and {\em geometry} constraints conducive to accurate colors and 3D densities estimation. We believe \ournerf can contribute beyond novel view synthesis, such as NeRF object detection~\cite{hu2022nerf}, NeRF object segmentation~\cite{zhi2021place, liu2022unsupervised, fan2022nerf,ren2022neural}, and NeRF registration~\cite{goli2022nerf2nerf}, where the rooms for improvement are substantial if more accurate color and density estimation are available.

Correspondingly, there are two steps in \ournerf. First, for appearance correction, the view-independent and view-dependent color components are predicted from the underlying 3D scene, which is combined to produce the final color estimation (Fig.~\ref{fig:toaster}).
%
The view-independent component (diffuse color and shading) captures the overall scene color, while the view-dependent component (highlights or reflections) captures color variations due to changes in viewing angle.
%
\ournerf then discards these view-dependent appearances in the training views to prevent them from interfering with the density estimation.
%
Second, a simple and effective geometry correction procedure will be performed to further eliminate the foggy ``floaters" or density errors. This geometry correction procedure is based on an assumption in line with traditional ray tracing in computer graphics.
\begin{comment}
% xh: basically copying method
On the other hand, ClearNeRF performs a geometric correction procedure performed on each traced ray during inference to refine the density estimation and better tackle the floater artifacts. 
%
The geometry correction procedure assumes that there should only be one salient peak along each traced ray during NeRF inference. 
Only the salient peak closest to the ray origin (the camera center) corresponds to  true geometry while the others will be manifested as foggy floaters hovering in the density volume. 
%
This assumption is in line with traditional ray tracing in computer graphics where in the absence of noise, only one intersection per ray should be returned to indicate the closest ray-object intersection.
%
\end{comment}
%%%%%%%%%%%
%As shown in Fig.~\ref{fig:teaser}, when reconstructing an indoor scene with sparse input and highly view-dependent objects, NeRF produces severe floating artifacts due to its attempt to explain view-dependent appearances.
%
Experiments verify that our proposed \ournerf can effectively get rid of floater artifacts without additional input.% or significant computational overhead. 


In summary, our contributions include the following:
\begin{itemize}[leftmargin=0.16in, topsep=2pt,itemsep=-1ex,partopsep=1ex,parsep=1ex]
    \item We propose a concise method for decomposing view-independent and view-dependent appearance during NeRF training and eliminate the interference of view-dependent appearance.
    \item We propose a geometric correction procedure performed on each traced ray during inference to refine the density estimation and better tackle the floater artifacts.
    \item Extensive experiments and ablations verify the effectiveness of our core designs and results in improvements over the vanilla NeRF and other state-of-the-art alternatives.
    %without additional computational resources or other inputs.
\end{itemize}




\section{Related work}

In recent years, large language models have improved significantly in various NLP areas, especially in generative tasks.
A lot of new concepts were introduced, starting from attention mechanism~\cite{bahdanau2014neural}, transformers~\cite{vaswani2017attention} to multitask, learning from instructions~\cite{wang2022super} and human feedback~\cite{wang2021putting}.
The last becomes extremely popular in the generative context including machine translation. 
% new architectures were proposed~\cite{radford2019language,brown2020language}, and, 
Consequently, the usage of machine translation tools has become a necessary compound for understanding a foreign language. 
Unfortunately, like other neural network-based algorithms, these tools are vulnerable to adversarial examples~\cite{DBLP:journals/corr/GoodfellowSS14}. 
Starting from text classification \cite{li-etal-2020-bert-attack,DBLP:conf/acl/EbrahimiRLD18,Li2018TextBuggerGA}, vulnerability and robustness received a lot of attention in the NLP community. 
For MT systems one of the pioneering works was~\cite{ebrahimi2018adversarial}, where authors proposed a character-level approach to generate adversarial examples.
% that neural MT systems are vulnerable to character-level perturbations, where only a few symbols in an input query are subject to change. 
Inheriting HotFlip~\cite{ebrahimi-etal-2018-hotflip} there were considered white-box and black-box settings, where only a few symbols in an input query are subject to change imitating typos.

While white-box optimization may yield stronger adversarial perturbations it implies access to the model's architecture and weights which is impractical in the case of online MT tools. 
In~\cite{wallace} there was considered a white-box universal approach to a targeted attack on conditional text generation. 
The authors modeled perturbation as an insertion of a trigger, a token sequence of small length, that results in a generated sequence similar to the target set of sentences. 
While during experiments certain triggers cause a model to produce sensitive racist output, they are generally meaningless and similarly to character-level attacks are easy to detect. 
Authors of~\cite{guo-etal-2021-gradient,9747475} reported high attack transferability making this approach promising for black-box setup, however,  the research is limited only to the GPT-2 model for generation task. 
The above papers use greedy techniques to walk through the searching space during the optimization, on the other hand, attacks on NLP models could be found via projection onto embeddings~\cite{wallace}, and for MT task this was discovered in~\cite{Seq2Sick,Sadrizadeh2023TargetedAA,sadrizadeh2023transfool}. 
In~\cite{zhang2021crafting}, it was shown that black-box optimization may yield transferable word-level attack that fools online translation tools, for example Baidu and Bing translators. 
This work proposed to use the word saliency as the measure of uncertainty. 
Masking candidates the saliency was estimated via additional BERT model~\cite{devlin2018bert}  which lead to strong readable and imperceptible adversaries, however, neither human evaluation was performed nor quantities results for online tools were given. In~\cite{wan2022paeg}, a gradient-based approach to generate phrase-level adversarial examples for neural MT systems was proposed. Similarly to~\cite{zhang2021crafting}, it is proposed to estimate the vulnerable word positions are estimated in an input phrase with the use of gradient information and replace corresponding words by the candidates computed with an auxiliary model.

% \mynote{actually we may underline that we do not generate adversarial examples per se (we arent aimed at misclassification), but rather generate inputs that are been translated though they should not}

% \mynote{TODO: Maybe add more criticism of zhang2021crafting and point out the differences in our approach.}

% \todopa{}{}{
% https://www.semanticscholar.org/paper/AdvAug\%3A-Robust-Adversarial-Augmentation-for-Neural-Cheng-Jiang/1e7d3a9846da556bc7b84ae1410d257b89448c30
% }

%\todopa{}{}{
%https://www.semanticscholar.org/paper/A-Targeted-Attack-on-Black-Box-Neural-Machine-with-Xu-Wang/2a46eb47e8742be29b16a5b83dc1a38616b24ce6
%}

%\todopa{}{}{https://www.semanticscholar.org/paper/PAEG\%3A-Phrase-level-Adversarial-Example-Generation-Wan-Yang/a6dd2a8debb5d5324c4f2be7fb7bb52ce109cbaf}

% \todopa{}{}{
% https://download.huan-zhang.com/events/srml2022/accepted/bhandari22lost.pdf
% }

%\todopa{}{}{http://fan-yao.com/paper/2021_SEED_nmtstroke.pdf}

% \todopa{}{}{https://arxiv.org/pdf/2303.01068v1.pdf}

%\todopa{kosinski2023theory}
%    {Theory of mind may have spontaneously emerged in large language models}
%    {https://arxiv.org/pdf/2302.02083.pdf}
%    {We can say that large language models are very clever now, etc...}

% \todopa{ebrahimi2018adversarial}
%     {On adversarial examples for character-level neural machine translation}
%     {https://arxiv.org/pdf/1806.09030.pdf}
%     {Very related work (see beamsearch in the text also)...}

% \todopa{zhang2021crafting}
%     {Crafting adversarial examples for neural machine translation}
%     {https://github.com/JHL-HUST/AdvNMT-WSLS}
%     {Very related work. See: ``Besides, WSLS exhibits strong transferability on attacking Baidu and Bing online translators.''}

% \todopa{sadrizadeh2023transfool}
%     {TransFool: An Adversarial Attack against Neural Machine Translation Models}
%     {https://arxiv.org/pdf/2302.00944.pdf}
%     {Very related work!}
\section{Method}
\label{sec: method}
% This section introduces the rendering pipeline of our proposed hierarchical compositional scene. 
% our pipeline consists of three processes, including decomposing the text into editable 3D layout, rendering the compositional views with local (object) NeRFs and global (scene) NeRF and the joint optimization on these hierarchical 3D representations.

% Note that the transformation between the object and the scene frame is defined by ${p}_o$ and ${D}_o$. 
%
% Next, we build a residual connection to add ${\sigma}_o$ and the referenced global color, and the rendering result will be used to calculate the SDS loss based on the global text.  
% Fig.~\ref{fig:framework} illustrates our pipeline, which consists of three main components, including the editable 3D scene layout based on multi-object text (Sec.~\ref{ssec:layout}), the scene rendering pipeline that composites the predictions from all local NeRFs (Sec.~\ref{ssec:render}), and the joint optimization on both local and global representation models (Sec.~\ref{sec:optimization}).
% To elaborate, our editable 3D scene layout represents a global frame of the scene by decomposing it into a set of local frames, where each is parameterized by a local NeRF, a 3D bounding box, and a corresponding local text prompt.
% For instance, the text prompt `A teddy bear and a stuffed monkey sit side by side' is interpreted as a 3D scene layout, as shown in Fig.~\ref{fig:framework}.  
% The whole 3D layout, \ie, scene frame, consists of two 3D bounding boxes, \ie local frames \#1 and \#2, with specific local text prompts, \ie, `a teddy bear' and `a stuffed monkey'. 
% %
% To render the scene view, we first calculate the ray-box intersections between the boxes and rays $({\boldsymbol{r}}_o, \boldsymbol{\phi}_d, {\boldsymbol{\theta}}_d)$, where the ${\boldsymbol{r}}_o$ is the ray origin and the $({\boldsymbol{r}}_o, \boldsymbol{\phi}_d)$ is its direction.
% Then, to infer each object's properties in local NeRFs, we sample the global points $({\boldsymbol{x}}_g, {\boldsymbol{y}}_g, {\boldsymbol{z}}_g)$ in the global frame within the ray-box intersection intervals and project them into the normalized local location $({\boldsymbol{x}}_l, {\boldsymbol{y}}_l, {\boldsymbol{z}}_l)$ in the local frame.
% %
% Given the local sampling points $({\boldsymbol{x}}_l, {\boldsymbol{y}}_l, {\boldsymbol{z}}_l)$, the implicit local NeRF ${\boldsymbol{\theta}}_l$ outputs four pseudo-color channels ${\boldsymbol{C}}_l$ and density $\boldsymbol{\sigma}$, which can be used to render a local view of the local frame to match its local text prompt.
% %
% We further calibrate the predicted pseudo-color $\boldsymbol{C}_l$ from local frames by adding the global embeddings ${\boldsymbol{emb}}_g$ to improve the global view consistency.
% Then, the calibrated predictions after composition are used to reconstruct the scene view by volumetric rendering along the rays.
% %
% Lastly, the rendered views based on local and global frames are guided by score distillation sampling loss $\nabla \mathcal{L}_{\text{SDS}}$~\cite{poole2022dreamfusion} to optimize all the learnable parameters. 
To resolve the issue of guidance collapse, our principal strategy is to \textit{decompose the scene into reusable components and compose/recompose them into a unified and consistent one}.
This enables flexible control over the generated content with direct use of prompts and box layouts, as illustrated in \cref{fig:teaser}.
%
Our proposed CompoNeRF confers several key benefits:
1) \textbf{Semantic Coherence}: It reliably creates 3D objects with detailed textures and global consistency, exemplified by authentic light interactions, such as reflections on the bed surface.
2) \textbf{Modularity and Reusability}: CompoNeRF functions as an ensemble of independently trained NeRF models. These can be efficiently stored and later retrieved from a cached dataset, enabling their reuse in various cases.
3) \textbf{Editability}: Our approach allows for flexible scene modification, such as interchanging the lamp for a vase filled with sunflowers or altering its scale, by simply adjusting the box dimensions for later finetuning. This feature enhances flexibility and creative possibilities. 


% Furthermore, the usage of layout boxes enables more flexible control over the generated content compared with the intricate sketch shape in Latent-NeRF\cite{metzer2022latent}. 
\begin{figure*}[t]
    \centering
    \includegraphics[width=0.9\linewidth]{figures/method.pdf}
    % \vspace{-12pt}
    \caption{\textbf{Framework Overview}.
The CompoNeRF model unfolds in three stages: 1) Editing 3D scene, which initiates the process by structuring the scene with 3D boxes and textual prompts; 2) Scene rendering, which encapsulates the composition/recomposition process, facilitating the transformation of NeRFs to a global frame, ensuring cohesive scene construction. Here, we specify design choices between density-based or color-based(without refining density) composition; 3) Joint Optimization, which leverages textual directives to amplify the rendering quality of both global and local views, while also integrating revised text prompts and NeRFs for refined scene depiction.
  % The model is structured into three components: Composition, Decomposition, and Recomposition. Composition deals with the foundational setup, detailed with choices for density-based and color-based composition. Decomposition utilizes the modularity of the CompoNeRF feature, caching each NeRF module offline for efficient recalibration. Recomposition reuses these cached NeRFs and adjusts the semantic context, providing a revised output with the inclusion of the offline NeRF enhancements.
    % Our model consists of two branches where the upper part is individual NeRFs, and the lower part denotes global calibration with our tailored composition model. The specific designs for density-based and color-based composition modules are highlighted. 
    % CompoNeRF consists of three parts: 1). The editable 3D scene layout configures the scene representations with 3D boxes and text prompts; 2).  The scene rendering includes the global calibration and the compositional process; 3). The joint optimization applies global and local text guidance on global and local render views.
    % The global frame (scene space) contains a set of local frames. Each is  represented by a local NeRF associated with a 3D box and text prompt defined by the editable 3D layout.
    % The scene view is volumetric rendered by sampling the points $({\boldsymbol{x}}_g, \boldsymbol{y}_g, \boldsymbol{z}_g)$ intersected with any local frame along the ray $(\boldsymbol{r}_o, {\boldsymbol{\phi}}_d, \boldsymbol{\theta}_d)$.
    % The sampling points are first inferred through the local NeRF with the local frame locations $({\boldsymbol{x}}_l, \boldsymbol{y}_l, \boldsymbol{z}_l)$ projected from the global location $({\boldsymbol{x}}_g, \boldsymbol{y}_g, \boldsymbol{z}_g)$.
    % And then, all the local predictions are calibrated by a global MLP with conditional input to render the scene view.
    % During the optimization, the text guidance is applied to both local views predicted by local frames only and global views predicted by the composition of all local frame predictions.
    }
    \label{fig:framework}
    % \vspace{-8pt}
\end{figure*}

\subsection{Preliminaries}
Defining individual object bounding boxes as \textit{local frames} and the overall scene coordinate system as the \textit{global frame}, we build the foundation of NeRF and diffusion processes.

\label{sec:background}
\noindent \textbf{3D Representation in Latent Space.}
Our methodology capitalizes on the state-of-the-art text-to-image generative model—Stable Diffusion as described by Rombach et al\cite{rombach2022high}.
We build upon the Latent-NeRF framework~\cite{metzer2022latent}, which computes latent colors for individual objects by considering their sample positions within a localized frame. Specifically, it maps a three-dimensional point in local coordinates \(\boldsymbol{x}_l = (x_l, y_l, z_l)\) to a volumetric density \(\boldsymbol{\sigma}_l\) and an associated color \(\boldsymbol{C}_l\), expressed as \((\boldsymbol{C}_l, \boldsymbol{\sigma}_l) = f_{\boldsymbol{\theta}_l}(x_l, y_l, z_l)\). Here, \(f\) represents a Multi-Layer Perceptron (MLP) characterized by parameters \(\boldsymbol{\theta}_l\).
 This NeRF-generated color is then assessed in the context of the Stable Diffusion model, using text prompts to guide NeRF toward spatially coherent inference with intricate context.
% to infer pseudo-color for each object using local NeRF.
% Specifically, the representation maps a point $\boldsymbol{x}_l = \left({x}_l, {y}_l, {z}_l\right)\in [-1, 1]$ in the local frame to its corresponding volumetric density $\boldsymbol{\sigma}_l$ and emitted color $\boldsymbol{C}_l$, \ie,  $\left(\boldsymbol{C}_l, {\boldsymbol{\sigma}_l}\right)=\boldsymbol{\theta}_{_l}\left({x_l}, {y}_l, {z}_l\right)$.
% The predicted pseudo-color is fed forward into the decoder of the Stable Diffusion model to obtain the final rendering result.

\noindent \textbf{Volume Rendering with Multiple Objects.}
% For each local frame $j$ with NeRF parameterized as $\theta_j$, we follow original NeRF design\cite{nerf} to integrate $(\boldsymbol{C}_l, \boldsymbol{\sigma}_l)$ of   sampled points from any hit ray $r_l=(\boldsymbol{o}_l, \boldsymbol{d}_l)$ by,
% For consistent scene rendering, object transmittance $T_k$ must be recalculated in the global frame based on independent properties inferred from local NeRFs. Hence, we sort predictions according to their distance to $\boldsymbol{o}_g$. 
% Similar to \cref{eq:volrend}, global color $\hat{\boldsymbol{C}}_g$ of ray $\boldsymbol{r}_g=(\boldsymbol{o}_g, \boldsymbol{d}_g)$ is predicted by the volumetric rendering integrating over $m$ objects,
We extend the volume rendering process to accommodate multiple objects by assigning each a local frame, denoted as $j$, with NeRF parameters $\boldsymbol{\theta}_{l, j}$. Drawing from the foundational NeRF approach \cite{nerf}, in each local frame, we integrate the color $\boldsymbol{C}_l$ and density $\boldsymbol{\sigma}_l$ for points $\boldsymbol{x}_l$ sampled along a ray $\boldsymbol{r}_l$, emanates from the camera origin $\boldsymbol{o}_l$ in direction $\boldsymbol{d}_l$. This is formalized in the predicted color integration for $\hat{\boldsymbol{C}}_l$ as:
{\setlength\abovedisplayskip{2pt}
\setlength\belowdisplayskip{2pt}
\begin{equation}
\label{eq:volrend}
{\hat{\boldsymbol{C}}_l}({\boldsymbol{r}_l})=\sum_{k=1}^{N} T_{l, k} \left(1-\exp \left(-\sigma_{l, k} \delta_k\right) \right) {\boldsymbol{C}}_{l,k},
\end{equation}}where $T_{l, k}=\exp \left(-\sum_{j=1}^{k-1} \sigma_{l,j} \delta_j\right)$ represents the transmittance to the $k$-th of total $N$ sample, calculated exponentially over the cumulative density along $\boldsymbol{r}_l$, and $\delta_k$ is the interval between adjacent samples.
%
To synthesize a coherent scene, we transition from processing individual local frames to a collective global frame. Within this global context, we reconcile object attributes inferred from their individual local NeRFs for refined $\boldsymbol{\sigma}_g, \boldsymbol{C}_g$ along with $T_{g, k}$. The samples $\boldsymbol{x}_g$ are ordered based on their spatial distances from the origin $\boldsymbol{o}_g$ following the coordinate transformation. We then express the volumetric rendering of a ray $\boldsymbol{r}_g$ integrating $m$ objects within the global frame as follows:
{
\setlength\abovedisplayskip{2pt}
\setlength\belowdisplayskip{2pt}
\begin{equation}
\label{eq:multi_volrend}
{\hat{\boldsymbol{C}}_g}({\boldsymbol{r}_g})=\sum_{k=1}^{m*N} T_{g, k} \left(1-\exp \left(-\sigma_{g, k} \delta_k\right) \right) {\boldsymbol{C}}_{g,k}. 
\end{equation}}

\noindent \textbf{Score Distillation Sampling.}
% During the SDS process, a noise image $\boldsymbol{X}_t$ is first generated by adding a sampled noise $\epsilon \sim \mathcal{N}(0, I)$ in noise level $t$ into a rendered view $\boldsymbol{X}$ from a NeRF.
To facilitate the conversion from text descriptions to 3D models, DreamFusion~\cite{poole2022dreamfusion} utilizes Score Distillation Sampling (SDS), leveraging the generative capabilities of a diffusion model, denoted as $\phi$, to guide the optimization of NeRF parameters, symbolized as $\boldsymbol{\theta}$.
%
Initially, SDS creates a noisy image $\boldsymbol{X}_t$ by infusing a randomly sampled noise $\epsilon$, which follows a normal distribution $\mathcal{N}(0, I)$, into a NeRF-rendered image $\boldsymbol{X}$ at a given noise level $t$.
The diffusion model $\phi$ then estimates the noise $\epsilon_\phi\left(\boldsymbol{X}_t, t, T\right)$ from this noisy image, conditioned by the noise level $t$ and an optional text prompt $T$. 
The key step in SDS involves calculating the gradient of the loss function, which measures the discrepancy between the estimated noise and the originally added noise:
{\setlength\abovedisplayskip{2pt}
\setlength\belowdisplayskip{2pt}
\begin{equation}
\label{eq:sds_loss}
\nabla_\theta \mathcal{L}_{\text{SDS}}(\boldsymbol{X}_t, T)=  w(t)\left(\epsilon_\phi\left(\boldsymbol{X}_t, t, T\right)-\epsilon\right),
\end{equation}}where $w(t)$ is a weighting function that adjusts the influence of the gradient based on the noise level. 
The gradients across all rendered views direct the update of $\boldsymbol{\theta}$, ensuring that the NeRF-generated images align with the text descriptions. Additionally, we incorporate the 'perturb and average' technique from SJC for more robust $\mathcal{L}_{\text{SDS}}$. For a comprehensive understanding of these methods, the reader is directed to the detailed explanations provided in \cite{poole2022dreamfusion,wang2022score}.

%
%
% \subsection{Editable 3D Scene Layout}
% \label{ssec:layout}
% The 3D scene layout explicitly combines language structures with 3D layouts in an editable way.
% Given the input text prompt $T$, the attribute-object pairs can be easily obtained based on user control.
% Note that the text prompt indicates the multi-object text prompt by default.
% % available for free in many structured representations, such as the constituency tree.
% As shown in Fig.~\ref{fig:framework}, we can extract multiple noun phrases with their binding attributes and map these local text prompts into corresponding regions.
% Specifically, we define the scene structure with $m$ local frames, each employs a local NeRF $\boldsymbol{\theta}_l$ as representation, the local text prompt $T_{l} \subseteq{T}$ and its spatial layout with 3D boxes $\mathbf{b} = \{\mathbf{p}, \mathbf{s}\} \in  \mathbb{R}^6$ of each object entity, where $\mathbf{p}=\{p_x, p_y, p_z\}$ refers to the center point and $\mathbf{s}=\{s_x, s_y, s_z\}$ denotes the box scale. 
% \textit{Our editable 3D layout is easy to be collected and edited with its simplicity, allowing for versatile and interactive user control by modifying the box's or text's properties to define a new scene}.
% Moreover, as depicted in Fig.~\ref{fig:teaser}, each component in a 3D scene layout can be replaced or re-composited with other trained local NeRFs, which is more friendly for flexible user editions compared with using only text prompts.
% We fine-tuned the new layout by global rendering, which enables scalable re-editing.
% Each relationship $r_k \in R$ is a triplet in a <subject-predictive object> format, where a subject node is. After we generate the scene graph from the complex prompts, we can sample the closest relationship with the 2d spatial layout as the initial 3D position. fine-tuned the new layout by global rendering, which enables scalable re-editing
%
% \subsection{Scene Rendering Pipeline}
% \label{ssec:render}
% In CompoNeRF, the scene images are rendered by a ray-casting approach following the design of NeRF.
% % Each ray to be cast is generated based on the camera pose, intrinsic, and transformation.
% The camera is defined by a pinhole camera model, casting a set of rays $(\boldsymbol{r}_o, \boldsymbol{\phi}_d, {\boldsymbol{\theta}}_d)=\boldsymbol{o}+t\boldsymbol{d}$ through each pixel on the frame of size $H \times W$, where the $\boldsymbol{r}_o \in  \mathbb{R}^3$ is the origin and the $(\boldsymbol{\phi}_d, \boldsymbol{\theta}_d)$ is the viewing direction.
% Along this ray, we sample all the points intersected with any layout box of local frames.
% For each hit sampled point, the color and volumetric density are computed through the local NeRF of the hit local frame.
% The ray color perdition is calculated by the differentiable integration applied on all the point-predicted colors and volumetric density along the ray.
%
% \noindent \textbf{Ray-box Intersection with Local Frames.}
% Given a ray $\boldsymbol{r}_i$, each box $\boldsymbol{b}_j$ of the local frame is applied with the AABB ray intersection test algorithm to check the intersections.
% When the ray $r_i$ is hit with a box $\boldsymbol{b}_j$ of the local frame, we use the entrance and exit points as near $\boldsymbol{t}_{in}$ and far $\boldsymbol{t}_{out}$ bounds to sample $N$ equidistant quadrature points, $
% \boldsymbol{t}_{i,j,n}=\frac{n-1}{N-1}\left(\boldsymbol{t}_{out}-\boldsymbol{t}_{in}\right)+\boldsymbol{t}_{in} , n \in \left[1, N\right]$
% % Despite each local frame only having a small number of hit rays compared to the scene, we observe that it is enough to represent each object accurately while maintaining short rendering times.
% Note that the coordinates of sampled points are first projected into normalized coordinates using the box scale of local frames to enable each local NeRF to learn the scale-independent representation.
% The bounding box $\mathbf{b}$ of the local frame in global coordinate can be transformed into a canonical bounding box by ${(\mathbf{b}} - \boldsymbol{p}) / \mathbf{s}$.
% Considering the rendering efficiency, we only calculate the valid points, interacted with the boxes, and set all the empty points with a constant background color.
%
% The appearance of a set object representations depends on its interaction with the scene and illumination which should be decided by the local frame location.
% To ensure the volumetric consistency, we only calibrate the emitted color with scene location, while the gradient still can be propagated.
% Since the overall color depends on both the global  positions $({x}_w, {y}_w, {z}_w)$ and ray directions $({\phi}_d, {\theta}_d)$, the global color embedding is learned based on both the positions and ray directions.
% Since the overall color depends on both the global  positions $({x}_w, {y}_w, {z}_w)$ and ray directions $({\phi}_d, {\theta}_d)$, the global color embedding is learned based on both the positions and ray directions.
% \subsection{The Proposed CompoNeRF}
% \subsubsection{Composition Module}
% CompoNeRF aims to composite multiple NeRFs to reconstruct multi-object scenes with both box and prompt guidance.
% %
% Our framework, as shown in \cref{fig:framework}, applies the AABB ray intersection test algorithm to check for intersections on each box in the global frame. We then samples $\boldsymbol{x}_g$ within the ray box intervals, and project them to $\boldsymbol{x}_l$ to infer  $\left(\boldsymbol{C}_l, {\boldsymbol{\sigma}_l}\right)$ in separate NeRF models. 
% %
% We then utilize volume rendering to obtain rendered views for each local frame respectively. 
% %
% After that, they would be passed on to our tailored composition Module to infer 
% $\left(\boldsymbol{C}_g, {\boldsymbol{\sigma}_g}\right)$
% for global rendering. 
% Next, we match local and global texts with their corresponding image outputs by SDS losses. 
% We also support recomposition by passing samples from cached models into $\boldsymbol{x}_l$ to continue the above process.
\begin{figure}[t!]
    \centering
    \includegraphics[width=\linewidth]{figures/abls.pdf}
    % \vspace{-22pt}
    % \caption{Ablation study on text guidance. (a) without local SDS losses. (b) without global SDS losses. (c) vanilla SDS losses without perturb and average scoring~\cite{wang2022score}. (d) full model.}
    \caption{\textbf{Design Impact Comparison: Density vs. Color-based Methods.} The top row illustrates the density-based approach's detailed rendering and quick convergence in the 'table wine' scene. The bottom row highlights the color-based method's enhancements and its drawbacks, such as geometric and shadow inaccuracies, particularly in close-up views and slow convergence.
    % \textbf{(a)} global text guidance(integrating local frames by \cref{eq:multi_volrend}) and global calibration(integrating local frames, then aligning the rendering result directly with the full text). 
    }
    \label{fig:abls}
    % \vspace{-20pt}
\end{figure}
\subsection{The Proposed CompoNeRF}
\subsubsection{Composition Module}
CompoNeRF is designed to composite multiple NeRFs to reconstruct scenes featuring multiple objects, utilizing guidance from both bounding boxes and textual prompts. Within our framework, depicted in \cref{fig:framework}, the Axis-Aligned Bounding Box (AABB) ray intersection test algorithm is applied to ascertain intersections across each box in the global frame. Subsequently, we sample points \(\boldsymbol{x}_g\) within the intervals of the ray-box and project them to \(\boldsymbol{x}_l\) to deduce the corresponding color \(\boldsymbol{C}_l\) and density \(\boldsymbol{\sigma}_l\) within individual NeRF models.
%
These properties are processed through our composition module to infer the global color \(\boldsymbol{C}_g\) and density \(\boldsymbol{\sigma}_g\), crucial for the global rendering.
%
Volume rendering techniques~\cite{kajiya1984ray} are then employed to procure the rendered views for both local and global frames. We propose dual SDS losses to ensure coherence between the image outputs and their corresponding textual descriptions. Additionally, our approach facilitates recomposition by channeling samples from cached models back into local frames along with the text revision, thereby streamlining the integration.

% As shown in \cref{fig:abls}(a), we verify its necessity by dropping $\nabla \mathcal{L}_{\text{SDS}_g}$. 
% %
% Compared with our full model, its layout does not fit our shared sense of a room, \ie, \emph{nightstand} is usually lower than \emph{bed}; \emph{lamp} needs a base to support it. Additionally,  it lacks global consistency, such as light reflection, to make it more realistic. 
% %
% Therefore, we leverage the full text semantics to ensure consistent global rendering across local frames. 
% %
% Instead of conditioning the global rendering view with the full prompt directly, we note that global calibration is necessary for geometry and color to be learned sufficiently.
% For example, we observe that geometric completeness and texture of \emph{nightstand} are not ideal. Although reflection appears around \emph{nightstand}, \emph{bed} is stripped of the light. 
% %
% Therefore, we opt to leverage the correlation between the rendering output of the combined NeRFs and the overall semantics to perform multi-object scene reconstruction.  
%

\noindent\textbf{Global Composition.}
The independent optimization of each local frame may inadvertently result in a lack of global coherence within the scene. To address this, our scene composition process is designed to integrate these frames, thereby achieving a more consistent result.
%
Before exploring the specifics of the module, it is imperative to discuss two critical design decisions within the composition module, as depicted in \cref{fig:framework}.
%
Upon integrating the properties inferred from \(\boldsymbol{x}_g\) into the composition module, they are fine-tuned through gradients derived from the global SDS loss.  This process leads to a critical consideration: the necessity and implications of refining the global density \(\boldsymbol{\sigma}_g\). This can be divided into two approaches: \textbf{1) Density-based:} The advantage of adjusting \(\boldsymbol{\sigma}_g\) is that it can adjust geometry, thus yielding a scene more congruent with the global text prompt. 
However, this comes at the cost of potentially compromising the optimal color \(\boldsymbol{C}_g\), as calibrating \(\boldsymbol{\sigma}_g\) introduces more uncertainty for subsequent color refinement as it requires prior density features $\boldsymbol{h}$ as shown at \cref{fig:compo}. 
\textbf{2) Color-based:} Conversely, directly employing \(\boldsymbol{\sigma}_l\) mitigates this uncertainty but at the expense of reduced geometric control, presenting a challenging balance to strike in the pursuit of precise scene composition.
% , which may lead to suboptimal outcomes.
%
After thorough experiments, exemplified in \cref{fig:abls}, we have opted for the density-based approach to refine \(\boldsymbol{\sigma}_g\)  prioritizing both \textbf{accuracy and efficiency}. The test revealed that it excels in rendering intricate details, such as enhanced wood grain textures and more naturally contoured 'salad', as accentuated by boxes. This method also demonstrated a swifter convergence rate. Conversely, while the color-based improved reflections and reduced flickering on the 'wine cup', it was plagued by issues such as sparse density, which adversely brings holes at the base of the 'cup' and the corner of the 'table'.
Furthermore, upon close examination, it becomes evident that shadow artifacts of 'wine' on the 'table' are pronounced, suggesting that its disadvantages outweigh its advantages.
%  in this context
% \textbf{Global Composition.}
% Each local frame is optimized independently, causing a lack of global connections for scene composition.
% Before delving into module details, there are two choices (see \cref{fig:framework}) on the composition module design we need to elaborate on first. 
% %
% In \cref{fig:framework}, by taking $\boldsymbol{x}_g$ into the composition module, their inferred properties are calibrated with gradients propagated from the global SDS loss. 
% However, it remains unclear whether $\boldsymbol{\sigma}_g$ should be refined or not. 
% %
% The trade-off on its usage is the density adjustment bringing a more reasonable layout and more geometric details that fit the global text prompt. While its potential downside is that $\boldsymbol{C}_g$ may not be optimal as $\boldsymbol{\sigma}_g$ has more uncertainty compared to $\boldsymbol{\sigma}_l$, bringing sub-optimal rendering results. 

% We choose the density-based method after comparing them with the experiment shown in \cref{fig:abls}. 
% %
% Specifically, we test both designs on the scene \emph{table wine} and discover that the density-based design provides more intrinsic details(as indicated by green boxes), \eg, enriched wood grains, and a more natural shape for \emph{salad} and has much faster convergence speed. In contrast, the color-based method enhances the reflection and smooths flickering on \emph{wine cup}, (as indicated by red boxes), but it suffers from 1) sparse density, resulting in poorly generated geometry at the base of  \emph{cup} and the wood \emph{table} corner. Additionally, shadow artifacts appeared on \emph{table} when viewed up close, outweighing benefits of the color-based method.

\begin{figure}[t!]
    \centering
    \includegraphics[width=\linewidth]{figures/compo_module.pdf}
    % \vspace{-24pt}
    % \caption{Ablation study on text guidance. (a) without local SDS losses. (b) without global SDS losses. (c) vanilla SDS losses without perturb and average scoring~\cite{wang2022score}. (d) full model.}
    \caption{\textbf{Detail of Composition module}: density-based design. 
    }
    \label{fig:compo}
    % \vspace{-18pt}
\end{figure}
\noindent\textbf{Network Design.}
The compositional framework of our network, as delineated in \cref{fig:compo}, is predicated on an architecture that employs a suite of MLPs, represented as \(\{\boldsymbol{\theta}_l\}_{l=1}^{m}\),  each dedicated to a distinct local frame. To harmonize \(\boldsymbol{\sigma}_l\) and \(\boldsymbol{C}_l\), we incorporate global MLPs, including density calibrator $f_{\boldsymbol{\theta}_{g_d}}$ and color calibrator $f_{\boldsymbol{\theta}_{g_c}}$.
%
A transformation module complements this system, tasked with maintaining the spatial coherence between the global and local frames. It governs the transformation of sampling points $\boldsymbol{x}$, ray directions $\boldsymbol{d}$, and adjacent sampling distances $\delta$. This module also orders the points $\{\boldsymbol{x}_{g,j}\}_j$ by their distance to the global camera origin $\boldsymbol{o}_g$, ensuring that each local point $\boldsymbol{x}_l$ is accurately matched with its corresponding global point $\boldsymbol{x}_g$ for subsequent volume rendering. 
%
The network design is:
{
\setlength\abovedisplayskip{4.5pt}
\setlength\belowdisplayskip{4.5pt}
\begin{align}
\label{eq:g_c_d}
{\boldsymbol{\sigma}_g}  &= \alpha_d f_{\boldsymbol{\theta}_{g_d}}({\boldsymbol{x}_g}) + \boldsymbol{\sigma}_l, \\
{\boldsymbol{C}_g}  &= \alpha_c f_{\boldsymbol{\theta}_{g_c}}(\boldsymbol{h}, {\boldsymbol{d}_g}) + \boldsymbol{C}_l. 
\end{align}}In contrast to the local frames, the global frame's color output $\boldsymbol{C}_g$ is inferred based on $\boldsymbol{h}$ and conditional on $\boldsymbol{d}_g$ to enable a view-dependent lighting effect.
% Denote the density features as $\boldsymbol{h}$. 
%
%
Residual learning is leveraged here, where \(\boldsymbol{\sigma}_l, \boldsymbol{C}_l\) serve as foundational elements that support the learning of global density \(\boldsymbol{\sigma}_g\) and color \(\boldsymbol{C}_g\). The parameters \(\alpha_d, \alpha_c\) are adjustable, allowing fine-tuning of the influence that local components exert on the global outputs.
%
It is imperative to acknowledge that in our color-based method, density calibration is intentionally excluded to concentrate solely on the refinement of color dynamics as shown at \cref{fig:framework}. This is achieved by conditioning the process on both spatial and directional global inputs \((\boldsymbol{x}_g, \boldsymbol{d}_g)\), as demonstrated in the following equations:
\begin{align}
\setlength\abovedisplayskip{4.5pt}
\setlength\belowdisplayskip{4.5pt}
\label{eq:g_c_c}
\boldsymbol{\sigma}_g = \boldsymbol{\sigma}_l, \quad
{\boldsymbol{C}_g} = \alpha_c f_{\boldsymbol{\theta}_{g_c}}({\boldsymbol{x}_g}, {\boldsymbol{d}_g}) + \boldsymbol{C}_l.
\end{align}
The integration of extra $\boldsymbol{x}_g$ aims to facilitate a fair comparison under same inputs with the density-based. It enhances the visual appeal of effects like the wine cup's reflection, as demonstrated in \cref{fig:abls}. However, this method is not without its compromises. It tends to produce artifacts and is characterized by a slower convergence rate. Additionally, this approach limits the ability to precisely control density, subsequently impacting the intricate geometric details.


\begin{figure*}[t!]
    \centering
    \includegraphics[width=\linewidth]{figures/sota.pdf}
    % \vspace{-24pt}
    \caption{\textbf{Qualitative comparison with other text-to-3D methods using multi-object text prompts}. Cases 1-3 demonstrate simpler settings characterized by compositions involving two objects. In contrast, Cases 4-8 delve into more intricate scenarios featuring compositions with more than two objects. Smaller images are presented to illustrate the generated local NeRFs(partially shown in Cases 4-8).}
    \label{fig:sota}
    % \vspace{-5pt}
\end{figure*}
%
% \begin{table*}[t!]
% \centering
% \resizebox{\textwidth}{!}
% {
% \begin{tabular}{cccccccc}
% \toprule
% Method            & \rotatebox{60}{table wine}  & \rotatebox{60}{teddy monkey} & \rotatebox{60}{computer mouse} & \rotatebox{60}{bed room}  & \rotatebox{60}{chess} & \rotatebox{60}{pisa tower} & \rotatebox{60}{astronaut} & \rotatebox{60}{tesla}  \\ \midrule
% LatentNeRF  & 21.55 & 27.38 & 17.13 & 21.86 & 31.19 & 24.31 & 27.07 & 25.16 \\
% SJC & 23.33 & 27.37 & 18.00 & 22.54 & 30.53 & \textbf{26.18 }& 27.84 & 23.55 \\
% CompoNeRF & \textbf{32.68} & \textbf{28.57}	 &\textbf{ 22.34} &\textbf{ 28.65} & \textbf{31.45} & \textbf{28.96} & 25.82 & 25.95 & 24.42 & \textbf{32.71} & \textbf{26.13 }& \textbf{26.38} & \textbf{30.98} & \textbf{33.37} \\
% \bottomrule
% \end{tabular}
% }
% \vspace{-10pt}
% \caption{Performance of our CompoNeRF in different 3D scenes. We use CLIP score \cite{parmar2023zero,zhang2023sine,wang2023imagen} as our evaluation metric, which is a common evaluation metric in text-to-image generation tasks to evaluate the similarity of the generated image to the text prompt. }
% \label{perclass}
% \end{table*}
%
\begin{table*}[t!]
% \scalebox{0.8}
\renewcommand{\arraystretch}{1.2}
\fontsize{4pt}{4pt}
\selectfont 
\centering
% \vspace{-8pt}
\resizebox{\textwidth}{!}
{
% \begin{tabular}{lcccccccc}
% \hline
% Method     & table\_wine    & tesla          & pyramid        & chess          & apple and banana      & astronaut      & glass\_balls   & Eiffel\_tower    \\ \hline
% LatentNeRF & 21.55          & 25.16          & 27.43          & 31.19          & 27.69          & 27.07          & 29.51          & 26.32          \\
% SJC        & 23.33          & 23.55          & 25.62          & 30.53          & 28.21          & 27.84          & 28.76          &27.41 \\
% \textbf{CompoNeRF(Ours)}     & \textbf{32.68} & \textbf{26.13} & \textbf{28.96} & \textbf{31.45} & \textbf{33.37} & \textbf{32.71} & \textbf{30.98} & \textbf{28.44}          \\ \hline
% \end{tabular}
\begin{tabular}{lcccccccc}
\hline
Method                   & Case 1         & Case 2         & Case 3         & Case 4         & Case 5         & Case 6         & Case 7         & Case 8         \\ 
\hlineB{1.1}
LatentNeRF               & 25.16          & 27.07          & 27.69          & 31.19          & 21.55          & 26.32          & 27.43          & 29.51          \\
SJC                      & 23.55          & 27.84          & 28.21          & 30.53          & 23.33          & 27.41          & 25.62          & 28.76          \\
\textbf{CompoNeRF (Ours)} & \textbf{26.13} & \textbf{32.71} & \textbf{33.37} & \textbf{31.45} & \textbf{36.06} & \textbf{28.44} & \textbf{28.96} & \textbf{30.98} \\ \hlineB{1.1}
\end{tabular}
}

% \vspace{-6pt}
\caption{\textbf{Performance comparison of our CompoNeRF in different 3D scenes}. For our evaluation metric, we utilize the average of CLIP scores~\cite{parmar2023zero,zhang2023sine,wang2023imagen} across different views, which serve to assess the similarity between the generated images and the global text prompt. }
\label{tb:perclass}
\end{table*}
% \cref{fig:framework} depicts the network architecture of the composition module. Denote $m$ as local MLP $\{\boldsymbol{\theta}_l\}_{l=1}^{m}$ for each local frame. Then, we introduce the global MLPs including density $\boldsymbol{\theta}_{g_d}$ and $\boldsymbol{\theta}_{g_c}$ calibrators to refine $\boldsymbol{\sigma}_l$ and $\boldsymbol{C}_l$. 
% %
% In detail, the network design is, 
% {
% % \setlength\abovedisplayskip{4.5pt}
% % \setlength\belowdisplayskip{4.5pt}
% \begin{align}
% \label{eq:g_c_d}
% {\boldsymbol{\sigma}_g}  &= \alpha_d \boldsymbol{\theta}_{g_d}({\boldsymbol{\sigma}_l}) + \boldsymbol{\sigma}_l, \\  
% {\boldsymbol{C}_g}  &= \alpha_c \boldsymbol{\theta}_{g_c}({\boldsymbol{C}_l},  {\boldsymbol{d}_g}) + \boldsymbol{C}_l, 
% \end{align}}
% %
% where residual $\boldsymbol{\sigma}_l, \boldsymbol{C}_l$ assist in learning $\boldsymbol{\sigma}_g$ and $\boldsymbol{C}_g$, while $\alpha_d, \alpha_c$ balance their contribution as learnable parameters.
% %
% Note that the color-based omits density calibration, and simply uses the shared color refinement.



% The 3D boxes are only used for the spatial configuration of local NeRFs, while the implicit representation of local NeRFs is inferred by the canonical samples inside the local frame without considering the global relationship across different objects.
% To relieve such location-dependent effects, we further calibrate the output color and density from the local NeRF with global coordinates $({\boldsymbol{x}}_g, {\boldsymbol{y}}_g, {\boldsymbol{z}}_g)$ and ray directions $\left({\boldsymbol{\phi}}_{d}, {\boldsymbol{\theta}}_{d}\right)$ as the conditional input.
% % to inject the global visual clues.
% %
% %
% Specifically, we adopt a shared MLP $\boldsymbol{\theta}_{g}$ to calibrate all the predicted object colors, that is,
% {\setlength\abovedisplayskip{4.5pt}
% \setlength\belowdisplayskip{4.5pt}
% \begin{align}
% \label{eq:MLP_dyn_2}
% {\boldsymbol{C}_g} = {\boldsymbol{C}_l} + \boldsymbol{emb}_{g} &= {\boldsymbol{C}_l} + \boldsymbol{\theta}_{g}({\boldsymbol{x}}_g, {\boldsymbol{y}}_g, {\boldsymbol{z}}_g, {\boldsymbol{\phi}}_{d}, {\boldsymbol{\theta}}_{d}),
% \end{align}}
% where ${\boldsymbol{C}_l}$ is the color predicted by the local NeRF.
% Therefore, the scene color can preserve the view-consistent behavior from the original architecture and add consistency across poses for the volumetric density.
% Since the color and density values share the same latent expression in $({\boldsymbol{x}}_l, {\boldsymbol{y}}_l, {\boldsymbol{z}}_l)$, we only calibrate the emitted scene color explicitly with the scene location, as the densities of local NeRFs also are implicitly adjusted during optimization.

% \noindent \textbf{Global and Local Volumetric Rendering.}
% After compositing all the interacted points, each ray $\boldsymbol{r}_i$ collects a set sampling points by $\{\boldsymbol{t}_{i,j,n} \}_{j=1, n=1}^{m_j, N}$, where $m_j$ is the number of the hit object.
% For each sampling point, the inference results with the respective 3D representations are the local color $\boldsymbol{c}_{l}$, global color $\boldsymbol{c}_{g}$, and density $\sigma$.

% In fact, the local view $\hat{C}_{l,j}$ of single object $j$ also can be rendered by the sampled points  belongs to the same local frames as shown at Fig.~\ref{fig:framework}.

\subsubsection{Recomposition}
Our architecture advances scene reconstruction by providing an intuitive interface for layout manipulation.  This capability is crucial for the reconfiguration of scene elements into novel scenes, as depicted in \cref{fig:framework}. Here, the input panel allows for adjustments in the attributes of bounding boxes, such as modifying the position and scale of the 'apple' bounding box prior to composition. The refinement process further involves sampling ray-box intervals from the global frame, leading to transformed coordinates with the corresponding ray samples that are then incorporated into the pipeline, as demonstrated in \cref{fig:compo}.
%
Each bounding box represents an individual NeRF, providing the flexibility to move, scale, or remove elements as needed. CompoNeRF's capabilities also extend to textual edits, exemplified by the transformation of 'wine' into 'juice'.
%
Since NeRFs have been well trained, we only finetune \(\theta_g, \theta_l\) to align text prompts to promote consistency of both local and global views.
%
Moreover, the NeRFs once retrained within the edited scene, are also structured to be decomposable and cacheable in future scene compositions.
% Our CompoNeRF architecture facilitates the seamless reconstruction of scenes leveraging existing models. It enables precise editing of bounding boxes parameterized by \(\{\boldsymbol{\theta}_l\}_{l=1}^{m}\), allowing for their reconfiguration into new layouts. Refer to \cref{fig:framework}, the input panel permits the modification of attributes such as the position and scale of the 'apple' node's bounding box prior to composition. The process is further refined by sampling from the updated ray-box intervals within the global frame, which are then projected onto \(\boldsymbol{x}_l\), ensuring a streamlined reconstruction that integrates the 'apple' effectively. This addition is executed with careful attention to color consistency, positioning the 'apple' adjacent to the 'French bread' to complement the scene's overall palette. Each bounding box represents an individual NeRF, which means they can be manipulated through moving, scaling, and removal operations. CompoNeRF also extends its editing prowess to textual modifications, as evidenced by the 'wine cup' now appearing filled with juice—a change propagated through both subtexts and the global test. 
% %
% Since NeRFs have been well trained, we only finetune $\theta_g, \theta_l$ to align text prompts to promote consistency of both local and global views . 
% %
% Moreover, the NeRFs, once retrained within the reimagined scene, are also structured to be decomposable and cacheable for subsequent scene compositions.

% , as shown in Fig.~\ref{fig:framework}.
% For each scene described by the multi-object text prompt $T$, we
% To enhance the guidance of local representations, we use the local text prompt $T_l \subseteq T$ of a single object to optimize the local NeRFs in local views.
% The scene views $\hat{\boldsymbol{X}}_g=\{\hat{\boldsymbol{C}}_{g,i}\}_{i=1}^{H\times W}$ is obtained from the predicted pixel values of $H \times W$ rays by compositing all the ray-box interaction values.
% Similarly, the rendered view $\hat{\boldsymbol{X}}_{l,j}$ of the local frame $\boldsymbol{\theta}_j$ without compositing other objects can be calculated by $\hat{\boldsymbol{C}}_{l,j}$, as depicted in Sec.~\ref{ssec:render}.
% We use the local color instead of the globally calibrated color to obtain a local view because the local NeRF should learn the object identity unrelated to its placed position, as the position can be different during user edition.
% % Compared to cropping the local region from a global view for training, separate rendering can avoid the undesired information from other objects brought by the occlusion and resolution adjustments.
% Formally, we employ the following loss as the learning objective,
\begin{figure*}[t!]
    \centering
    \includegraphics[width=\linewidth]{figures/editing.pdf}
    % \vspace{-23pt}
    \caption{\textbf{Scene Editing Outcome:} Demonstrated here are the stages of our recomposition, utilizing cached source scenes. Each NeRF is individually identified by colorful labels. These decomposed nodes are then positioned in the initial layout and subsequently calibrated to form the final composition. The detailed description of the ambient environment is underscored, enhancing the scene's realism.}
    \label{fig:app}
    % \vspace{-12pt}
\end{figure*} 
\subsubsection{Optimization}
\label{sec:optimization}
During optimization, our method employs dual text guidance to align rendering results with both global and local textual descriptions. The optimization objective is:
{
\small
\setlength\abovedisplayskip{2pt}
\setlength\belowdisplayskip{2pt}
\begin{equation}
\label{eqn:loss_f}
\mathcal{L}= {\alpha_g}\nabla\mathcal{L}_{\text{SDS}}(\hat{\boldsymbol{X}}_{g}, T) + {\alpha_l}\sum_{j=1}^{m} \nabla\mathcal{L}_{\text{SDS}}(\hat{\boldsymbol{X}}_{l,j}, T_{l,j}) + \beta\mathcal{L}_{\text{sparse}},\nonumber
\end{equation}
}where $T$ signifies the global text prompt, while $T_{l}$ pertains to a specific object within the global context. The hyperparameters $\alpha_{g}, \alpha_{l}$, and $\beta$ modulate the respective loss weights. 
% $\nabla \mathcal{L}_{\text{SDS}}$ is the score distillation sampling loss, as described in Sec.~\ref{sec:background}.
As suggested in~\cite{metzer2022latent}, we use $L_{\text{sparse}}$ included to penalize the binary entropy of local NeRFs' densities, thereby mitigating the issue of extraneous floating radiance.
Additionally, incorporating directional cues such as "front view" or "side view" into the input text, as suggested by \cite{poole2022dreamfusion,metzer2022latent} proves beneficial in specifying camera poses during the training phase, further enhancing the alignment of our generated scenes with the intended perspectives.
% Note that the global calibration in the scene frame can adaptively revise both $({C}_l, {\sigma})$ in local NeRF with $\nabla \mathcal{L}_{SDS}$ along with the back-propagating gradient.

\section{Experiments}
\label{sec:exp}
\begin{figure*}
\begin{center}
\includegraphics[width=.95\linewidth]{img/novelview.pdf}
\end{center}
\vspace{-0.4cm}
   \caption{\textbf{Qualitative comparison with baseline methods on Novel View Synthesis. } We compare our novel view synthesis on transparent objects with the methods that we identify as most relevant to ours, NeRF~\cite{mildenhall2020nerf}, Eikonal Field~\cite{bemana2022eikonal}, IDR~\cite{yariv2020multiview}, and PhySG~\cite{zhang2021physg}. Our method outperforms the others on the high-frequency details caused by ray refraction. }
   \vspace{-0.4cm}
\label{fig:novelviewsyn}
\end{figure*}

\subsection{Synthetic Data Evaluation}
\vspace{-0.2cm}
\medskip
\noindent\textbf{Datasets. }We use the 4 mesh objects of kitty, cow, bear, and key-mouse from~\cite{Xing2022drot, zhang2021physg}, and render each object with the smooth dielectric BSDF model with Mitsuba 3 \cite{jakob2022mitsuba3} under an environmental light emitter. For synthetic dataset evaluations we set interior IOR to 1.4723 for glass and exterior IOR to 1.00028 for air. We also create datasets with interior IOR set to 1.2, and 2.4 for ablation studies. We uniformly sample 200 camera poses on the upper hemisphere around each object following the Fibonacci lattice and randomly assign 100 each for training and testing. We obtain object masks through data pre-processing~\cite{remove_bg}.    

\vspace{-0.2cm}
\medskip
\noindent\textbf{Baseline. }As discussed in Sec.~\ref{sec:related} and shown in Tab.~\ref{tab:baseline}, no other work studies the same problem as ours, i.e., modeling refraction for transparent objects with complex geometry by neural networks for novel view and relighting synthesis. We, therefore, classify our baselines based on different tasks: \textbf{NeRF}~\cite{mildenhall2020nerf} and \textbf{Eikonal Fields}~\cite{bemana2022eikonal} on novel view synthesis; \textbf{IDR}~\cite{yariv2020multiview} on novel view synthesis and geometry reconstruction; \textbf{PhySG}~\cite{zhang2021physg} on novel view and relighting synthesis, and geometry reconstruction.  As geometry is not our aimed task to improve, extracted mesh quality is only to show that RBN effectively disentangles geometry and ray refraction on appearance. We do not include comparisons with volume-based neural relighting methods~\cite{boss2021nerd, verbin2022refnerf} as they share the same appearance model with PhySG.

\begin{figure}
\begin{center}
\includegraphics[width=0.85\linewidth]{img/relight-cow.pdf}
\end{center}
\vspace{-0.4cm}
   \caption{\textbf{Qualitative results on Relighting for synthetic datasets. } We show that our network can faithfully relight the object with unseen environment illumination, unlike PhySG~\cite{zhang2021physg}. }
\label{fig:relightsyn}
\vspace{-0.3cm}
\end{figure}

\vspace{-0.2cm}
\medskip
\noindent\textbf{Novel View Synthesis.} 
A qualitative comparison of our method and baseline methods is shown in Fig.~\ref{fig:novelviewsyn}. NeRF and Eikonal Fields model object appearance as MLP-based volume and cannot distill radiance properly around the object surface. However, when modeling refractive objects with complex geometry, it is important to locate the surface for accurate refraction direction estimation. Eikonal fields relies on user-defined bounding boxes, resulting in failure cases where the opaque scene and the refractive part cannot be separated. Meanwhile, IDR and PhySG are surface-based methods, but IDR models appearance as a light field~\cite{wood2000surface} and cannot correctly interpolate the high-frequency change of the refracted environment illumination on object appearance. PhySG uses Disney BSDF~\cite{burley2012physically} which does not work for non-opaque objects and therefore fails to correctly disentangle geometry and appearance. 

We report quantitative evaluation on novel view synthesis with metrics including PSNR, SSIM, and LPIPS~\cite{zhang2018perceptual} through testing on held-out images in Tab.~\ref{tab:quanti}. Our method significantly outperforms all of our baselines on synthesizing novel views for accurately modeling the refraction direction of each ray intersected with geometry. 

% \vspace{-0.2cm}
\medskip
\noindent\textbf{Relight Synthesis.} 
We provide a qualitative comparison of relighting synthesis in Fig.~\ref{fig:relightsyn}. As the environment map used during training is natural and unstructured unlike in prior works~\cite{lyu2020differentiable, wu2018full}, many pixels share similar radiance, but our learned refractions are not overfitted on the training illumination; they are aligned with the true refractions. We relight each scene with an unseen environment map to test the correctness of the object refraction. PhySG fails on this task as it does not model refractive material, resulting in incorrect appearance decomposition~\cite{burley2012physically}.  We report quantitative evaluation w.r.t ground truth relighting in Tab.~\ref{tab:quanti}. 



\begin{table}
  \centering
  \scalebox{0.9}{
  \begin{tabular}{@{}lcccc@{}}
    \toprule
    Synthetic & \multicolumn{4}{c}{$\downarrow$Chamfer $L_1(10^{-3})$} \\
    \midrule
    Method & Kitty & Bear & Key Mouse & Cow\\
    \cmidrule{2-5}
    IDR \cite{yariv2020multiview}& 4.30 & 3.66 & 3.70 & 11.66 \\
    PhySG \cite{zhang2021physg} & 87.67 & 67.43 & 31.61 & 52.17 \\
    \textbf{NEMTO} & \textbf{2.22} & \textbf{1.71} & \textbf{2.27} & \textbf{2.60} \\
    \bottomrule
    \vspace{-0.2cm}
  \end{tabular}
    }
  \caption{\textbf{Quantitative evaluation on recovered meshes of synthetic datasets.} We report the chamfer distance metric~\cite{cdcode} on g.t. mesh versus extracted meshes as a quantitative measure for reconstructed geometry quality. NEMTO achieves better results than baseline methods that models object surfaces. }
  \label{tab:chamfersyn}
  \vspace{-0.4cm}
\end{table}


\vspace{-0.2cm}
\medskip
\noindent\textbf{Disentanglement on Geometry and Appearance.}
We evaluate our extracted geometry on synthetic datasets with ground truth mesh through the Chamfer distance metric and compare our geometry with those of surfaced-based methods. In Fig~\ref{fig:relightsyn}, the geometry of PhySG is entangled with surface appearance, i.e. the appearance under the original illumination is imprinted on the surface and raised geometry. Tab.~\ref{tab:chamfersyn} shows that IDR does better than PhySG, though still worse than ours. Our geometry and refracted appearance are better separated due to our modeling of ray refraction and optimizations. 

\begin{figure}[t]
\begin{center}
   \includegraphics[width=0.8\linewidth]{img/ablation-ior.pdf}
\end{center}
   \caption{\textbf{Experiments on different transparent media. }We show that NEMTO works for different transparent media other than glass. The learned $\eta_\mathbf{t}$ is adaptive to different media and allows our model to synthesize faithful results.}
\label{fig:ablation-ior}
\vspace{-0.4cm}
\end{figure}



\vspace{-0.2cm}
\medskip
\noindent\textbf{Robustness to different refractive indices.}
We conducted experiments on transparent objects rendered with various IORs to showcase the robustness of our framework to IOR changes. Our approach is suitable for different types of refractive materials, as demonstrated in Fig.~\ref{fig:ablation-ior}. Note that our predicted $\eta_\mathbf{t}$ for the blending of ray refraction and reflection is also adaptive to different IOR, as shown in the case for IOR = 2.4, the reflected radiance is adequately brighter than in IOR = 1.4723 and 1.2. 

\begin{table}
\scalebox{0.75}{
  \centering
  \begin{tabular}{@{}lcccccc@{}}
  
    \toprule
     & \multicolumn{3}{c}{ Novel View } 
     & \multicolumn{3}{c}{Relighting} \\
     
    \cmidrule(lr){2-4}\cmidrule(lr){5-7}
    
    Method 
    & PSNR $\uparrow$ & SSIM $\uparrow$ & LPIPS $\downarrow$ 
    & PSNR $\uparrow$ & SSIM $\uparrow$ & LPIPS $\downarrow$ \\
    
    \cmidrule(lr){2-4}\cmidrule(lr){5-7}
    
    NeRF~\cite{mildenhall2020nerf} 
    & 21.274 & 0.837 & 0.171
    & - & - & - 
    \\ 

    Eikonal~\cite{bemana2022eikonal} 
    & 15.866 & 0.452 & 0.589
    & - & - & - 
    \\ 
    
    
    IDR~\cite{yariv2020multiview} 
    & 22.695 & 0.851 & 0.152
    & - & - & - 
    \\ 
    
    PhySG~\cite{zhang2021physg}
    & 19.981 & 0.791 & 0.203
    & 15.412 & 0.749 & 0.237
    \\ 
        
    \midrule
    
    SDF-A 
    & 21.758 & 0.828 & 0.145
    & 17.846 & 0.787 & 0.192
    \\ 

    w/o $\mathcal{L}_{\textrm{rg}}$
    & 15.659 & 0.746 & 0.221
    & 14.585 & 0.713 & 0.238
    \\
    
    w/o $\mathcal{L}_{\textrm{rs}}$
    & 21.623 & 0.811 & 0.163
    & 19.026 & 0.823 & 0.149
    \\
    
    \textbf{NEMTO} 
    & \textbf{26.582} & \textbf{0.924} & \textbf{0.083}
    & \textbf{25.147} & \textbf{0.918} & \textbf{0.098}
    \\
    
    \bottomrule
  \end{tabular}
  }
  \vspace{0.05cm}
  \caption{\textbf{Quantitative Evaluations. } We present the average result on all synthetic datasets. The first three methods are not capable of relighting. Our method performs significantly better on both novel view and relighting synthesis than all of our baseline methods and ablation experiments. }
  \label{tab:quanti}
  \vspace{-0.4cm}
\end{table}



\begin{figure}[t]
\begin{center}
   \includegraphics[width=0.9\linewidth]{img/sdf-a.pdf}
\end{center}
\vspace{-0.3cm}
   \caption{\textbf{Qualitative ablation on SDF-A}. SDF-A shows that jointly optimizing refraction and geometry is prone to error. Our approach performs significantly better than this naive approach.}
\label{fig:ablation-sdfa}
\vspace{-0.4cm}
\end{figure} 




\begin{figure*}
\begin{center}
\includegraphics[width=.95\linewidth]{img/rw.pdf}
\end{center}
   \caption{\textbf{Qualitative results on image synthesis and extracted geometry for real-world data. } We compare our extracted geometry, novel view synthesis, and relighting with the extracted geometry and rendering layer of TLG~\cite{li2020through}, which restricts the light bounce within transparent media to only two bounces. }
\label{fig:relightrw}
\vspace{-0.5cm}
\end{figure*}


\begin{figure}[t]
\begin{center}
   \includegraphics[width=\linewidth]{img/ablation-loss.pdf}
\end{center}
\vspace{-0.4cm}
   \caption{\textbf{Ablation on losses for ray refraction optimizations.} Each experiment is trained with a frozen geometry network to demonstrate the effect of each loss term on ray bending. }
\label{fig:ablation-loss}
\vspace{-0.4cm}
\end{figure}

\vspace{-0.2cm}
\subsection{Ablation Studies}
\vspace{-0.2cm}
We perform the following ablation studies to demonstrate the effectiveness of Our RBN, $\mathcal{L}_{\textrm{rg}}$, and $\mathcal{L}_{\textrm{rs}}$. 

\vspace{-0.2cm}
\medskip
\noindent\textbf{Ablation on ray bending network.} We implemented a naive version of our method~\textbf{SDF-A} without using RBN. It renders transparent objects with \textit{analytical} refraction to demonstrates the effectiveness of our RBN and neural environment matting method over the use of a physically-based differentiable renderer on transparent objects. As shown in Fig.~\ref{fig:ablation-sdfa}, our method synthesizes more accurate results when jointly optimizing for geometry and light refraction, which are better disentangled. This is evident from the smoother surfaces of our method due to $\mathcal{L}_{\textrm{rs}}$. NEMTO estimated smoother surface normal than SDF-A, and gives much more faithful ray refractions.%SDF-A shows that our method is able to disentangle geometry and appearance during joint optimization of the networks.

\vspace{-0.2cm}
\medskip
\noindent\textbf{Ablation on $\mathcal{L}_{\textrm{rg}}$ and $\mathcal{L}_{\textrm{rs}}$.} For experiments on the refraction guiding and refraction smoothness loss, we fix the optimized geometry and only show different optimization results for refraction prediction. The lower part of Tab.~\ref{tab:quanti} shows quantitative evaluation that our complete architecture performs better than without each of these two loss terms. Fig.~\ref{fig:ablation-loss} compares the learned refraction from each ablation experiment: in column (b) without $\mathcal{L}_{\textrm{rg}}$, the model cannot learn the correct direction; in column (c) without $\mathcal{L}_{\textrm{rs}}$, the optimized ray refraction is around the true scope but shows wrong wave-patterned artifacts.

\vspace{-0.2cm}
\subsection{Real World Data Results}
\noindent\textbf{Datasets.} For real-world evaluation, we use 4 sets of captured images and environment maps on dog, monkey, pig, and mouse shapes from TLG ~\cite{li2020through}. As TLG only provides 10-12 images for each real-world object, we render training data with the ground truth CT-scanned meshes following the steps of synthetic datasets generation detailed in the supplementary. We do evaluations on released real-world images. 

\vspace{-0.3cm}
\medskip
\noindent\textbf{Edited Scene Synthesis. }Fig.~\ref{fig:relightrw} shows our synthesis for real-world images. Our method is able to predict accurate ray refraction for transparent objects and produces a smoother surface normal prediction on geometry extraction than TLG. TLG designed a novel differentiable rendering layer for physically-based transparent object modeling, but it only renders up to two bounces of refraction, whereas our method does not pose an upper bound on the number of ray bounces. Moreover, TLG does not work with an unknown IOR for transparent objects. Note that, although TLG claims to require only 10-12 images for testing, it requires rendering a large-scale synthetic dataset with 1.5k HDR (High Dynamic Range) environment maps for training, which is unnecessary in our case.% However, the generalization ability of TLG remains an isse.
\vspace{-0.2cm}

\section{Discussion and Limitations}

Although we can ablate concepts efficiently for a wide range of object instances, styles, and memorized images, our method is still limited in several ways. First, while our method overwrites a target concept, this does not guarantee that the target concept cannot be generated through a different, distant text prompt. We show an example in \reffig{limitation} (a), where after ablating {\menlo Van Gogh}, the model can still generate {\menlo starry night painting}. However, upon discovery, one can resolve this by explicitly ablating the target concept {\menlo starry night painting}. Secondly, when ablating a target concept, we still sometimes observe slight degradation in its surrounding concepts, as shown in \reffig{limitation} (c). 

\nupur{Our method does not prevent a downstream user with full access to model weights from re-introducing the ablated concept~\cite{ruiz2022dreambooth,kumari2022multi,gal2022image}. Even without access to the model weights, one may be able to iteratively optimize for a text prompt with a particular target concept. Though that may be much more difficult than optimizing the model weights, our work does not guarantee that this is impossible.}

Nevertheless, we believe every creator should have an ``opt-out'' capability. We take a small step towards this goal, creating a computational tool to remove copyrighted images and artworks from large-scale image generative models.




{\small
\bibliographystyle{ieee_fullname}
\bibliography{egbib}
}

\end{document}