%\documentclass[12pt,lineno,linenumbers]{aastex631}
%\documentclass[12pt,twocolumn]{aastex631}
\documentclass[12pt,lineno]{aastex631}

\usepackage{color}
\usepackage{hyperref}
\usepackage{epstopdf}
\epstopdfsetup{update}
\usepackage{graphicx}
\usepackage[FIGTOPCAP]{subfigure}
\usepackage{CJKutf8}
\usepackage{amsmath}
\usepackage{longtable}
%\shortauthors{Wang et al.}
\newcommand{\xzq}[1]{{\textcolor{magenta} {#1}}}



\begin{document}
%\linenumbers 
%\title{Significant Evolution of Photosphere Emission and a broken ``$\alpha$-intensity" relation in a possible long-short burst GRB~230307A}
\title{Significant Evolution of Photosphere Emission and a broken ``$\alpha$-intensity" relation of GRB 230307A}

\correspondingauthor{Yi-Zhong Fan}
\email{yzfan@pmo.ac.cn}

\author[0000-0002-8385-7848]{Yun Wang}
\affiliation{Key Laboratory of Dark Matter and Space Astronomy, Purple Mountain Observatory, Chinese Academy of Sciences, Nanjing 210034, China}
\affiliation{School of Astronomy and Space Science, University of Science and Technology of China, Hefei, Anhui 230026, China}

\author[0000-0003-4963-7275]{Zi-Qing Xia}
\affiliation{Key Laboratory of Dark Matter and Space Astronomy, Purple Mountain Observatory, Chinese Academy of Sciences, Nanjing 210034, China}
 
\author[0000-0001-6076-9522]{Tian-Ci Zheng}
\affiliation{Key Laboratory of Dark Matter and Space Astronomy, Purple Mountain Observatory, Chinese Academy of Sciences, Nanjing 210034, China}
\affiliation{School of Astronomy and Space Science, University of Science and Technology of China, Hefei, Anhui 230026, China}

\author[0000-0002-9037-8642]{Jia Ren}
\affiliation{School of Astronomy and Space Science, Nanjing University, Nanjing 210093, China}
\affiliation{Key Laboratory of Modern Astronomy and Astrophysics (Nanjing University), Ministry of Education, China}

\author[0000-0002-8966-6911]{Yi-Zhong Fan}
\affiliation{Key Laboratory of Dark Matter and Space Astronomy, Purple Mountain Observatory, Chinese Academy of Sciences, Nanjing 210034, China}
\affiliation{School of Astronomy and Space Science, University of Science and Technology of China, Hefei, Anhui 230026, China}

\begin{abstract}
%Shortly after the brightest gamma-ray burst (GRB) in history was observed by human beings, the second brightest GRB followed.
%This is undoubtedly an excellent natural laboratory for studying extreme physics.
%However, GRB~230307A has no pile-up effect on the instrument like GRB~221009A, so it is an excellent sample for finely analyzing the prompt emission of GRB. 
GRB~230307A is one of the birghtest gamma-ray bursts detected so far.
Its prompt emission has been analyzed by the high-energy detection data comprehensive analysis tool ({\tt HEtools}) developed by us.
%Previously, an $\alpha$-intensity relation has been interpreted as a manifestation of subphotospheric heating in a flow with a varying entropy, or as the change of the magnetic field in the radiation regime under the synchrotron radiation scenario.
%Therefore, the origin of this relation is still unclear.
With the excellent observation of GRB~230307A by Fermi-GBM in keV-MeV energy range, we can reveal the details about the evolution of this relation.
As found in the high-time-resolution spectral analysis, the early low-energy spectral indices ($\alpha$) of this burst exceed the limit of synchrotron radiation ($\alpha=-2/3$), and gradually decreases with the energy flux ($F$).
%Such evolution pattern of these two parameters with intensity is called ``double tracking". 
A tight $E_{\rm p}\propto F^{0.44}$ correlation anyhow holds within the whole duration of the burst, where $E_{\rm p}$ is the spectral peak energy.
Such evolution pattern of $\alpha$ and $E_{\rm p}$ with intensity is called ``double tracking".
In addition, for the relation between the $\alpha$ and $F$, we find a log Bayes factor $\sim$ 160 in favor of a smoothly broken power-law function over a linear function in log-linear space.
%In addition, for the relation between the low-energy spectral index ($\alpha$) and the energy flux we find a Bayesian odds ratio $\sim$ 160 in favor of a smoothly broken power-law function over a linear function in log-linear space.
%log Bayes factor $\sim$ 160. 
We call this particular $\alpha-F$ relation as broken ``$\alpha$-intensity", and interpret it as the evolution of the ratio of thermal and non-thermal components, which is also the evolution of the photosphere.


%These results suggest that GRB~230307A is a peculiar revealing the transition of the emission components from thermal to non-thermal.

%These results suggest that GRB~230307A is a typical lesson of photosphere evolution in the GRB prompt emission.
%Therefore, we revealed the significant photosphere emission evolution through this excellent sample, and made an analogy with other typical GRBs according to the basic characteristics of its prompt emission.

\end{abstract}
\keywords{Gamma-ray bursts (629)}

\section{Introduction} \label{sec:intro}
At the early time when gamma-ray burst (GRB) were recognized as the origin of cosmology, the prompt emission was predicted to be quasi-thermal \citep{paczynski1986gamma,goodman1986gamma}.
This is due to the huge energy ($\gtrsim$ 10$^{53}$ erg) released in a very small space ($r \sim 10^7 - 10^8$ cm), which is bound to produce a huge optical depth.
However, in the BATSE era, most of the GRB spectra are non-thermal, which can be well fitted by the Band function \citep{band1993batse}.
The main theory is to explain the non-thermal component as synchrotron radiation and synchrotron self-Compton scattering from relativistic electrons \citep{tavani1996shock,lloyd2000synchrotron,zhang2002analysis,daigne2011reconciling,zhang2010internal,uhm2014fast}.
Although the synchrotron radiation model has been well developed in GRBs, there are still some problems. For example, the synchrotron radiation or synchrotron self-Compton scattering cannot explain the steep low-energy spectral index in some observations \citep{crider1997evolution,preece1998synchrotron,preece2002consistency,ghirlanda2003extremely}, and cannot explain the correlation between the peak energy and luminosity ($E_{\rm p}$ - $L$) without introducing additional assumptions \citep{golenetskii1983correlation,amati2002intrinsic}, and the efficiency of energy dissipation in the internal shock model \citep{mochkovitch1995internal,kobayashi1997can,panaitescu1999power,lazzati1999constraints,kumar1999gamma,spada2000analysis,guetta2001efficiency,maxham2009modeling}. Hybrid models of thermal and non-thermal components can address these issues \citep{pe2017photospheric}.

So far, the most direct evidence of photosphere emission is the observation of GRB 090902B by Fermi satellite \citep{abdo2009fermi}.
\cite{ryde2010identification} confirmed that the spectrum of GRB~090902B is a broadened Planck function superimposed with a power-law component extending to the high energy band, such geometric broadening \citep{pe2008temporal,lundman2013theory,deng2014low} can be described as a multi-color blackbody model \citep{ryde2010identification,hou2018multicolor}.
Events like GRB090902 are very rare due to the identification of thermal components usually requires sufficient photon counts for time-resolved spectrum analysis.
%And too many photons will cause pile-up effect in the instrument, such as GRB~221009A, of course, such events are extremely rare.
The currently detected GRB~230307A, which may be the second brightest GRB \citep{GCN33414}, has a high-quality observation through the Fermi Gamma-ray Burst Monitor (GBM).
It thus provides a valuable opportunity to study the time-resolved spectrum of GRB.

The paper is organized as follows:
In Section \ref{sec:obs_ana}, we present the data analysis of GRB~230307A by the self-developed {\tt HEtools} package.
In Section \ref{sec:2}, we further characterize GRB~230307A based on the Bayesian inference results for time-resolved spectra.
In Section \ref{sec:3}, we discuss some of these results and compare this burst with some typical GRBs.
In Section \ref{sec:4}, we summarize results of our analysis.

\section{Observation and Data analysis}\label{sec:obs_ana}
Shortly after the last monster swept by, Fermi-GBM reported another bright GRB~230307A \citep[trigger 699896651/ 230307656;][]{GCN33405}.
From the current observations and some preliminary analysis \citep{GCN33406,GCN33411,GCN33414,GCN33418,GCN33427}, its energy flux is second only to GRB~221009A.

We performed further analysis on the GBM data by using {\tt HEtools} (see in APPENDIX. \ref{app:1}).
Before this tool was named, we have already applied it in some GRB data analysis \cite[e.g.,][]{jin2023detection,ren2022very,wang2022grb}.
The Fermi-GBM \citep{meegan2009fermi} payload has two types of detectors, including the 12 sodium iodide (NaI) detectors and 2 bismuth germanate (BGO) detectors.
The selection of detector is usually based on the count rate and pointing direction, here we selected a NaI (na) and a BGO (b1) for data analysis.
As shown in Figure. \ref{fig:LC_para}, the top two panels show the light curve and duration $T_{90}$ ($\sim$ 39.88 s) of GBM (na: 50 -300 keV), respectively. The red solid line is the time interval given by the Bayesian block technique \citep{scargle2013studies} that can be used as a reference for time-resolved spectral time intervals.
%In addition, we plotted the light curves of GRB~211211A in the same energy band together with a purple solid line. 
%Both of them contain the main emission (ME) phase and a weaker extended emission (EE) phase.
\begin{figure}[!h]
    \centering
    \includegraphics[width=0.8\textwidth]{LC_whithout_LSB.pdf}
    \caption{Observational data and parameter evolution of GRB~220307A.
    The first panel shows the light curves of GRB~220307A, and the vertical red dashed lines indicate the intervals of ME and EE, respectively. %and GRB~211211A, respectively. 
    The second panel is the photon accumulation curve, in which the blue dashed lines represent the interval at which photons accumulate 5\% to 95\%. The next three panels are the evolution of $\alpha$, $E_{\rm p}$, and energy Flux (1 - 10,000 keV), where orange represents the Band model results, and blue represents the CPL model results.
    The slashed region in the third panel is the interval that the low-energy spectral index exceeds the synchrotron radiation limit (red dashed line).
    %and the black dashed line is the broken point in the $\alpha-flux$ relation.
    }
    \label{fig:LC_para}
\end{figure}

\subsection{Spectral Analysis}\label{sec:spec_ana}
Here we analyze the time-integrated and time-resolved spectra of GRB~230307A, respectively. 
The intervals of time-integrated spectrum analysis are $T_{90}$ ($T_0+ 1.79 - T_0 + 41.67$ s), main emission (ME; $T_0 + 0.00 - T_0 + 18.50$ s), extended emission (EE; $T_0+ 18.50 - T_0 + 84.14$ s), respectively.
Due to the extremely abundant photon count of this burst, we can perform very high-time-resolution spectrum analysis. 
A total of 106 intervals were selected for time-resolved spectral analysis, and the specific time intervals are shown in Table \ref{tab:1}.

We used two type of photon spectrum model in our spectral inference, the first is a smoothly joined broken power-law function (the so-called ``Band” function; \citealt{band1993batse}).
The Band function is written as
%\begin{footnotesize}
\begin{equation}
    N(E)=
    \begin{cases}
        A \big(\frac{E}{100\,{\rm keV}}\big)^{\alpha}{\rm exp}{\big(-\frac{E}{E_0}\big)}, \mbox{if $E<(\alpha-\beta)E_{0}$ }\\
        A\big[\frac{(\alpha-\beta)E_0}{100\,{\rm keV}}\big]^{(\alpha-\beta)}{\rm exp}{\big(\beta-\alpha\big)}\big(\frac{E}{100\,{\rm keV}}\big)^{\beta},
        \mbox{if $E > (\alpha-\beta)E_{0}$}
    \end{cases}
    \label{eq:band}
\end{equation}
%\end{footnotesize}
where \emph{A} is the normalization constant, \emph{E} is the energy in unit of keV, $\alpha$ is the low-energy spectral index, $\beta$ is the high-energy photon spectral index, and \emph{E$_{0}$} is the broken energy in the spectrum.
The peak energy in the $\nu F_\nu$ spectrum $E_{\rm p}$ is equal to $E_{0}\times(2+\alpha)$.
When the detection energy range is narrow or the high-energy photon count rate is low, the $\beta$ of the Band function is often not well constrained, so another model is a cutoff power-law function (CPL), written as
\begin{equation}
    { N(E)=A(\frac{E}{100\,{\rm keV}})^{\alpha}{\rm exp}(-\frac{E}{E_{\rm c}}) },
\end{equation}
where \emph{$\alpha$} is the power law photon spectral index, \emph{E$_{\rm c}$} is the broken energy in the spectrum,
and the peak energy $E_{\rm p}$ is equal to $E_{\rm c}\times(2+\alpha)$.
For the possible components of the photosphere, we consider a multicolor blackbody (mBB) model \citep{ryde2010identification,hou2018multicolor} to describe it, which is
\begin{equation}
    N(E)=\frac{8.0525(m+1)K}{\Big[\big(\frac{T_{\rm max}}{T_{\rm min}}\big)^{m+1}-1\Big]}\Big(\frac{kT_{\rm min}}{\rm keV}\Big)^{-2}I(E),\label{N(E)}
\end{equation}
where
\begin{equation}
    I(E)=\Big(\frac{E}{kT_{\rm min}}\Big)^{m-1}\int_{\frac{E}{kT_{\rm max}}}^{\frac{E}{kT_{\rm min}}}\frac{x^{2-m}}{e^x-1}dx,\label{I(E)}
\end{equation}
where $x=E/kT$, the temperature range from $kT_{\rm min}$ to $kT_{\rm max}$, and the index $m$ of the temperature determines the shape of spectra.
In addition, the thermal component is usually accompanied by a non-thermal component, which is a power-law (PL) model with $\gamma$ index, written as $N(E) = A~E^{\hat{\gamma}}$.


Then we employ the Bayesian inference \citep{thrane2019introduction,van2021bayesian} for parameter estimation and model comparison.
We use {\tt Dynesty} \citep{speagle2020dynesty,skilling2006nested,higson2019dynamic} from the {\tt Bilby} \citep{ashton2019bilby} package as the posterior parameter sampler.
Usually for GBM data, the likelihood function used in Bayesian inference is {\tt pgstat}\footnote{\url{https://heasarc.gsfc.nasa.gov/xanadu/xspec/manual/XSappendixStatistics.html}}.
When considering different hypotheses with the same prior volume, model selection can be done by comparing Bayes factors. 
The Bayes factor (BF) is the ratio of the Bayesian evidence ($\mathcal{Z} = \int \mathcal{L}(d|\theta) \pi(\theta) d\theta$) for different models. 
The log of Bayes factor can be written as
\begin{equation}
    \ln\text{BF}^\text{A}_\text{B} = \ln({\cal Z}_\text{A}) - \ln({\cal Z}_\text{B}) .
    \label{eq:7}
\end{equation}
When $\ln{\rm BF} > 8$, we can say that there is a ``strong evidence" in favor of one hypothesis over the other \citep{thrane2019introduction}.	

The posterior parameters and model selection of each model are shown in Table \ref{tab:1}.
As shown in Figure \ref{fig:LC_para}, the three panels at the bottom are the evolution of model parameters and energy flux ($1 - 10,000$ keV) over time, in which the blue points are the CPL model parameters, and the orange points are the Band model parameters.
It is worth noting that in the early phase (before $\sim$ 7 s) of this burst, the low-energy spectral index obtained by both the Band model and the CPL model exceed the synchrotron limit, also known as the ``Line of Death" \citep{preece1998synchrotron,preece2002consistency}.
Beside, all high-energy photon spectral indexes also exceed typical values ($\beta \sim-2$) \citep{preece2000batse},
which most likely corresponds to the exponential decay of the Planck function at the highest temperature or insufficient high-energy photon count rate.

And the evolution of $E_{\rm p}$ and $\alpha$ over the entire outburst simultaneously shows the pattern of intensity tracking \citep{lu2012comprehensive,Ryde2019intensity}, also known as ``Double-tracking" \citep{li2019double}.
Even more peculiarly, we found a broken behavior in the $\alpha-F$ relation of GRB~230307A and called it broken ``$\alpha$-intensity".
As shown in Figure \ref{fig:th_veo}, we performed additional spectral analyzes for the $\alpha$-hardest interval ($T_0$ + [1.84 - 1.97] s), the broken-$\alpha$ interval ($T_0$ + [21.10 - 21.68] s), and a late interval ($T_0$ + [79.47 - 84.14] s). The $E_{\rm p}$ of the first time interval may come from the maximum temperature ($kT_{\rm max} \sim 480$ keV) of the mBB spectrum, and is accompanied by a non-thermal PL component ($\hat{\gamma} \sim -1.72$).
In the second time interval, the ratio of these two components (thermal vs. non-thermal) changed significantly compared to before. By the final time interval, the superposition of these two components has become indistinguishable, and the slope of the energy spectrum ($\alpha \sim -1.59$) is represented by the non-thermal emission.


%while $\alpha$ behaves abnormally in the early phase (region filled with black slashes), and then changes to the same evolution pattern, that is, the so-called ``double tracking" \citep{li2019double}.
%Such an anomaly of low-energy spectral index $\alpha$ and intensity tracking pattern also appears in the thermal dominated GRB~220426A \citep{wang2022grb}, but it does not show a clear broken point.
%We will continue to be analyzed and discussed in the following chapters.
\begin{figure}[t]
    \centering
    \includegraphics[width=0.32\textwidth]{eeuf_1.pdf}
    \includegraphics[width=0.32\textwidth]{eeuf_2.pdf}
    \includegraphics[width=0.32\textwidth]{eeuf_3.pdf}
    \caption{The $\nu F_{\nu}$ spectra of GRB~230307A at three special time intervals. From left to right, the photon spectrum used in the first two time intervals is the mBB+PL model, and the last one is the CPL model.}
    \label{fig:th_veo}
\end{figure}


\section{Characteristics}\label{sec:2}
\subsection{The $\alpha-F$ and $F-E_{\rm p}$ relations}

Based on results obtained in Section \ref{sec:spec_ana}, we performed a statistical analysis on the two relations (i.e., $\alpha-F$, $F-E_{\rm p}$) of GRB~230307A for the CPL-model and Band-model samples (listed in the Table~\ref{tab:1}), respectively.
In order to be conservative, we consider 20\% uncertainty of the GBM effective area as the systematic error of energy flux $F$~\citep{2009ApJ...702..791M}.

As for the $\alpha-F$ relation,
\cite{Ryde2019intensity} has analyzed the sample in \cite{yu2019Bayesian} and organized this relation as the log-linear (LL) function:
\begin{equation}
\label{eq:ll}
    F(\alpha) =  N~e^{k\alpha},
\end{equation}
where $N$ is normalization factor, and they found the parameter $k$ was about 3.
Here we fit the log-linear relation function for the GRB~230307A and we obtain the best-fit parameter $k$ of 4.18/4.18 for the CPL/Band-model samples with the log of Bayesian evidence $\ln({\cal Z}) = $ -572.83/-579.08, respectively.
However, the $\alpha-F$ samples of GRB~230307A exhibit an obvious broken behavior as shown in the left panel of the Figure \ref{fig:afreation}.
In our work, we take two other relation function to fit the $\alpha-F$ samples:
One is the broken log-linear (BLL) function given as
\begin{equation}
\label{eq:bll}
    F(\alpha) = 
    \begin{cases}
    N~{\rm e}^{k_1 \alpha}, ~~\mbox{if $\alpha~<~\alpha_b$ }\\
    N'~{\rm e}^{k_2 \alpha}, ~~\mbox{if $\alpha~>~\alpha_b$ }\\
    \end{cases}
\end{equation}
where $N' = N~{\rm e}^{(k_1-k_2)\alpha_b}$, 
and we get that the best-fit broken point $\alpha_b$ is $-1.05/-1.04$, the best-fit first index $k_1$ is $6.10/6.12$, and the best-fit second index $k_2$ is 2.59/2.52, corresponding to $\ln({\cal Z}) =-413.42/-415.13$ for the CPL/Band-model samples, respectively.
The other is the smoothly broken power-law (SBPL) function written as
\begin{equation}
\label{eq:sbpl}
    F(\alpha) =  N~\bigg[\bigg(\frac{\alpha}{\alpha_b}\bigg)^{\gamma_1} + \bigg(\frac{\alpha}{\alpha_b}\bigg)^{\gamma_2} \bigg]^{-1}.
\end{equation}
Based the CPL/Band-model samples, the best-fit broken point $\alpha_b$ we obtained is $-1.10/-1.10$, the best-fit first index $\gamma_1$ is $9.85/9.98$, and the best-fit second index $\gamma_2$ is $1.25/1.24$, with the evidence of $\ln({\cal Z})= $  $-410.24/-414.81$.
Compared with the previous log-linear relation function, the broken log-linear and smoothly broken power-law functions exhibit large Bayes factors with $\ln{\rm BF}_{\rm LL}^{\rm BLL} = 159.41$ and $\ln{\rm BF}_{\rm LL}^{\rm SBPL} = 162.59$ for the CPL-model samples ($\ln{\rm BF}_{\rm LL}^{\rm BLL} = 163.95$ and $\ln{\rm BF}_{\rm LL}^{\rm SBPL} = 164.27$ for the Band-model samples), which indicates that there is a strong evidence of the broken ``$\alpha$-intensity" relation in GRB~230307A.
The results of the $\alpha-F$ relation for the CPL-model sample are displayed in the left panel of the Figure \ref{fig:afreation}.

The $F-E_{\rm p}$ relation is found in a large fraction of GRBs and also in the time-resolved spectrum of single GRB \citep{wei2003there,yonetoku2004gamma,liang2004luminosity,lu2012comprehensive}, which can be naturally explained by the photosphere model \citep{fan2012photospheric}.
As for GRB~230307A , the $F-E_{\rm p}$ relation exhibits a linear relation in logarithmic space,  as shown in the right panel of Figure \ref{fig:afreation}.
This behavior coincides with the intensity trace in the fourth panel of Figure \ref{fig:LC_para}, which can be described as a single power-law function:
\begin{equation}
    %f(E_{\rm p}) =  N~{E_{\rm p}}^{\gamma}.
    E_{\rm p} = N~F^{\gamma}.
\end{equation}
where $N$ is the normalization factor and $\gamma$ is the flux index. 
The best-fit index for GRB~230307A is $\gamma = 0.43/0.44$ based on the CPL/Band-model samples, respectively. 
Our result on the $F-E_{\rm p}$ relation for the CPL-model samples is displayed in the right panel of the Figure \ref{fig:afreation}.

\begin{figure}[!htb]
    \centering
    \includegraphics[width=0.49\textwidth]{cpl_af_paper.pdf}
    \includegraphics[width=0.49\textwidth]{cpl_fEp_paper.pdf}
    \caption{The $\alpha-F$ relation (left panel) and the $F-E_{\rm p}$ relation (right panel) for the CPL-model samples of GRB~230307A.
    In the left panel: The samples with the $\alpha$ exceeding the synchrotron limit ($\alpha=-2/3$, the grey dashed line) are marked in purple, and the others are in blue. The black/red/green line corresponds to the best-fit Log-linear/Broken Log-linear/Smoothly broken power-law model for the $\alpha-F$ relation, respectively.
    In the right panel: The black line corresponds to the best-fit Power-law model we obtained for the $\alpha-F$ relation.}
    \label{fig:afreation}
\end{figure}


% \begin{figure}[!htb]
%     \centering
%     \includegraphics[width=0.49\textwidth]{cpl_fEp.pdf}
%     \includegraphics[width=0.49\textwidth]{band_fEp.pdf}
%     \caption{
%     The $F-E_{\rm p}$ relation for the CPL-model samples (left panel) and the Band-model samples (right panel). The black line in the two figures correspond to the best-fit Power-law relation we obtained.}
%     \label{fig:fepreation}
% \end{figure}


\subsection{Spectral lag}\label{sec:spec_lag}
Another important characteristic of GRBs is the spectral lag, in the energy range below 10 MeV, the high-energy photons arrive earlier than the low-energy photons \citep{norris2000connection,norris2002implications,norris2005long}.
The delay of pulse peaks in different energy bands can be quantified using the cross-correlation function (CCF), which is widely used in the calculation of GRB spectral lag \citep{band1997gamma,ukwatta2010spectral}.
We calculated the CCF of the GRB~230307A time series in the energy band (100 - 150 keV) and (200 - 250 keV) with a time interval of ($T_0$ - 1, $T_0$ + 40 s) , and the peak values of CCF were calculated via polynomial fitting.
We estimated the uncertainty of the lag by Monte Carlo simulation \citep{ukwatta2010spectral}, and get the spectral lag as $\tau = 0.65 \pm 0.03$ s finally.
\begin{figure}[!htb]
    \centering
    \includegraphics[width=0.5\textwidth]{lag.pdf}
    \caption{The spectral lag of the light curves in different energy ranges of the GBM n2 detector compared with the lowest energy range (10 - 20 keV).}
    \label{fig:lag}
\end{figure}
And Spectral lag usually evolves with energy \citep{lu2018comprehensive}. As shown in Figure \ref{fig:lag}, the spectral lag of GRB~230307A increases with the energy band compared to the lowest energy band ($10-20$ keV).

\subsection{$E_{\rm p,z}$ - $E_{\gamma,\rm iso}$ relation}\label{sec:amati}
The relation of $E_{\rm p,z}-E_{\gamma,\rm iso}$ \citep{amati2002intrinsic} is often used in the judgment of GRBs classification \citep[e.g.,][]{gehrels2006new}, where $E_{\rm p,z} = (1+z)E_{\rm p}$ is the rest frame peak energy, $E_{\gamma,\rm iso}$ is the isotropic bolometric emission energy, written as 
\begin{equation}
    E_{\gamma,\text{iso}} = \frac{4 \pi d_L^2 k S_{\gamma}}{1+z},
    \label{eq:E_gamma_iso}
\end{equation}
where $d_L$ is the luminosity distance, $S_{\gamma}$ is the energy fluence in the gamma-ray band, and $k$ is the correction factor, which can correct the energy range of the observer frame to the energy range of 1 - 10,000 keV in the rest frame. The correction factor $k$ \citep{bloom2001prompt} writes as
\begin{equation}
    k = \frac{\int_{1/(1+z)}^{10^4/(1+z)}  E N(E) {\rm d} E }{\int_{e_1}^{e_2} EN(E) {\rm d} E},
    \label{eq:k_cor}
\end{equation}
where $e_1$ and $e_2$ correspond to the energy range of the detector.
%Due to the lack of exact redshift measurements, we calculate $E_{\gamma,\rm iso}$ and $E_{p,z}$ of GRB~230307A through the assumed redshift range (from 0.01 to 5), and the assumed cosmological parameters are \emph{H$_{0}$} = $\rm 69.6 ~k ms^{-1}~Mpc^{-1}$, $\Omega_{\rm m}= 0.29$, and $\Omega_{\rm \Lambda}= 0.71$, as shown by the red trajectory in Figure \ref{fig:amati}.
The redshift of GRB~230307A is assumed to be $z \sim 0.065$ \citep{GCN33485}, and the adopted cosmological parameters are \emph{H$_{0}$} = $\rm 69.6 ~k ms^{-1}~Mpc^{-1}$, $\Omega_{\rm m}= 0.29$, and $\Omega_{\rm \Lambda}= 0.71$. 
And the data of Type I and Type I GRBs with redshifts are adopted from \cite{minaev2020p}. Unlike GRB 060614 and GRB 211211A, GRB 230307A seems to be more consistent with being a long GRB rather than a long-short burst. This is in particular the case if the redshift is larger than 0.1. 

%, the data of GRB~211211A comes from \cite{yang2022long}, and the data of GRB~221009A comes from \cite{yang2023synchrotron}.
\begin{figure}[!h]
    \centering
    \includegraphics[width=0.75\textwidth]{amati.pdf}
    \caption{The $E_{\rm p,z}$ - $E_{\gamma,\rm iso}$ diagram.
    %The red trajectory represents the result of the assumed redshift (from 0.01 to 5) of GRB~230307A.
    The blue and gray points are the data of Type I and Type II gamma-ray bursts with known redshifts, and the corresponding dashed lines represent the 2$\sigma_{\rm cor}$ correlation regions, respectively \citep{minaev2020p,minaev2020grb}. 
    Green and gold markers represent the different phases of GRB 060614, the first long duration burst (lasting about 100 seconds) with identified kilonova emission \citep{Yang2015,Jin2015}, and GRB~211211A \citep{yang2022long}, and blue stars indicate the brightest GRB~221009A \citep{an2023insight,yang2023synchrotron}.
   %{\bf I do not think this plot is very helpful for this work. If we really want to have such a plot, better to have all the events, not just select two recent ones.}
    }
    \label{fig:amati}
\end{figure}

\section{Discussion}\label{sec:3}
%	\subsection{the broken in $\alpha-F$ relation}

\subsection{The broken ``$\alpha$-intensity" relation}
%For GRB prompt emission, a part of case exhibits an uncommon intensity-tracking low-energy spectral index $\alpha$.
For the GRB prompt emission, the low-energy spectral index manifested as intensity-tracking evolution pattern occur in a small number of samples.
\cite{Ryde2019intensity}  has organized the $\alpha-F$ relation as the log-linear ($F \propto e^{k\alpha}$) function, which is believed to be a manifestation of subphotospheric heating in a flow with a varying entropy \citep{Ryde2017Emission,Ryde2019intensity}.
However, $\alpha -$intensity was also expected to be picked from non-thermal, i.e., synchrotron radiation, dominated GRBs, such as GRB~131231A \citep{li2019double} and GRB~211211A \citep{yang2022long}.
The evolution of $\alpha$ in the synchrotron radiation scenario may originate from the change of the magnetic field at the emission region \citep[e.g.,][]{uhm2014fast}.
%Therefore, one naturally expects that there is a broken point in the $\alpha-flux$ relation when the transition of the emission component from thermal to non-thermal occurs.
%Therefore, the origin of this relation is still unclear, and 
There are no clear observations and studies pointing out the impact on the $\alpha-F$ relationship when the proportion of thermal and non-thermal components evolve.

Thanks to the sufficient photon count GRB~230307A, we discovered an interesting phenomenon through the high-time-resolution spectrum analysis.
In Figure \ref{fig:afreation}, by counting the parameters of 106 time-resolved spectra, we found that there is an obvious broken behavior exhibited in the $\alpha-F$ relation.
In the left panel of Figure \ref{fig:th_veo}, the energy spectrum of the $\alpha$-hardest time interval can be fitted by the mBB+PL model,%and the early $flux-E_{\rm p}$ relation appears more compact.
and there is a correlation between $F-E_{\rm p}$.
Although the synchrotron radiation model can predict the $F-E_{\rm p}$ relation, this relation is more natural in the photosphere model \citep{fan2012photospheric}, and we consider that the thermal component is dominant in the early phase of GRB~230307A.

When the proportion of thermal components gradually decreases (as shown in the middle panel in Figure \ref{fig:th_veo}), the $\alpha-F$ relation begins to show an broken behavior.
The weaker emission and the broader spectrum may come from the entropy decreases the photosphere secedes from the saturation radius \citep{Ryde2019intensity}.
%As green dashed line shown in the third panel of Figure \ref{fig:LC_para}, 
And the broken point $\alpha_b \sim -1.05 $ is consistent with the typical low-energy spectral index of GRBs \citep[$\alpha \sim -1$;][]{kaneko2006complete,preece2000batse}, %and may hint an internal magnetization $\epsilon_B > 10^{-3}$ \citep{Vurm2016Radiative}.
which may arise from different physical situations \citep[e.g., transfer simulations of magnetized jets with $\epsilon_B > 10^{-3}$;][]{Vurm2016Radiative}, and can be summarized as unsaturated comptonization of low-energy photons.
One of the time-resolved spectra of this phase is shown in the middle panel of Figure \ref{fig:th_veo}, which may explain the statistically significant hidden black body components found on the left side of the Band model energy peak in some previous studies \citep{guiriec2011detection,axelsson2012grb110721a,guiriec2013evidence}.

While at a later phase, the spectral index of the CPL model $\alpha \sim -1.59$, and the emission may be dominated by non-thermal components (shown in the right panel of Figure \ref{fig:th_veo}).
%the $\alpha-flux$ relation evolves and the $flux-E_{\rm p}$ relation becomes more diffuse. 
%We predict that there may be a second ``broken point", but cannot confirm this due to observational limitations.
Hence, GRB~230307A is a peculiar case which is recorded the detailed transition of the emission component from thermal to non-thermal, and characterizes a special broken ``$\alpha$-intensity" behavior.


%\subsection{a possible long-short burst vs. GRB~211211A}	
%\subsection{the second brightest burst vs. GRB~221009A}
%\subsection{Even rate of bright GRB\,230307}
\subsection{Comparison with other GRBs}% with GRB~211211A and GRB~221009A}
First we compare GRB~230307A with some long-short (also known as ``hybrid") bursts, i.e., the bursts with a duration much longer than 2 seconds but from the merger of the compact objects. The light curve of GRB~230307A is similar to that of GRB~060614 and GRB~211211A, which also have ME and EE phases respectively. 
But the spectral lag of GRB~230307A is $\tau = 0.65 \pm 0.03$ s, which is 
%higher than the delay of GRB~211211A \citep{yang2022long} and GRB~060614 \citep{gehrels2006new} by about a few millisecondse
consistent with that of long GRBs \citep[one exception is GRB 060505, a burst with a time lag of $\sim 0.35$ s but hosts a kilonova signal;][]{Jin2021}.
%And the spectral lag increases with time, which is the opposite of GRB~130427A \citep{preece2014first}.
%If GRB~230307A is located on the fitting line of the Type I burst in the $E_{p,z}$ - $E_{\gamma,\rm iso}$ relation, the redshift $z \sim 0.016$ can be roughly estimated by the black trajectory in Figure \ref{fig:amati}.
And GRB~230307A would have a projected offset of $\sim$40 kpc from the possible host galaxy \citep{GCN33485}.
Of course, if it is a long-short burst, the most definitive evidence would require the discovery of Li-Paczynski macronova \citep[also known as the kilonova;][]{li1998transient}, which would be a ``smoking gun" for compact-binary origin.

In the energy spectrum analysis, we found that there may be a photosphere emission component in the early phase of GRB~230307A, which is similar to that of the also very bright GRB~130427A \citep{preece2014first}. 
However, the Brightest Of All Time (the BOAT, GRB~221009A) is believed to be dominated by non-thermal synchrotron radiation throughout its duration \citep{yang2023synchrotron}.
So these bright GRBs clearly have different emission components.
%The broken point in $\alpha-flux$ relation may come from the evolution of thermal emission components. 
A typical example of the evolution of emission components is GRB~160625B, which has three episodes dominated by different components \citep{zhang2018transition}. 
Since GRB~230307A is bright enough, and the transition from thermal to non-thermal emission is clear in the time-resolved spectra.

The brightest GRB~221009A has been reported to be a recurrence timescale about 10,000 years \citep{Burns2023BOAT}.
%GRB~230307, as the secondary brightest GRB \citep{GCN33414}
For a given GRB with observed fluence $S$, its annual rate can be estimated with $R_{\rm GRB} = 1.037\times 10^{-5} \times S^{-3/2}$,
and the corresponding recurrence timescale $\tau (S)= R_{\rm GRB}^{-1}$ \citep{Burns2023BOAT}.
GRB~230307A has a duration of $T_{90} \sim 39.88 $ s and a integral fluence $S = 2.63\times 10^{-3} ~{\rm erg~cm^{-2}}$ (10 - 1,000 keV), also exceeds that of GRB~130427A \cite[$2.5\times 10^{-3} ~{\rm erg~cm^{-2}}$;][]{an2023insight}, and the derived $\tau (S) \sim 13.01$ yrs.
Although its event rate is much higher than GRB~221009A, as the possible second brightest GRB, it is also very rare in Fermi's entire orbital career.

\section{SUMMARY}\label{sec:4}
In this work, we use {\tt HEtools} to analyze the Fermi-GBM data of the possibly second brightest GRB~230307A, and its features can be summarized as follows:
\begin{itemize}
\item {GRB~230307A has a very high count rate and also a long duration for the high-time-resolution spectral analysis. 
Through the parameter statistics inferred from up to 106 resolved spectra, we found a clear broken behavior in the $\alpha-flux$ relation with a corresponding log Bayes factor of $\sim$ 160, which may be interpreted as the transition from thermal emission to non-thermal emission, and the significant evolution of photosphere throughout its duration.}
    
\item {The light curve of GRB~230307A is similar to the GRBs of compact-binary merger origin (for instance, GRB~060614 and GRB~211211A). It may also have a projected offset of $\sim$40 kpc from its host galaxy \citep{GCN33485}.
Anyhow, the spectral lag of GRB~230307A is not consistent with that of long-short bursts \citep[one exception is GRB 060505, a burst with a time lag of $\sim 0.35$ s but hosts a kilonova signal, as reported in][]{Jin2021}. The physical parameters of main outburst and the tail component are also very different from that of GRB 060614 and GRB 211211A in the $E_{\rm p,z}-E_{\gamma, {\rm iso}}$ plot. Thus no conclusion can be drawn on its origin without more observations.}
    
\item{Although recurrence timescale ($\tau (S)=13.01$ yrs) of this tiny monster is lower than that of the BOAT GRB~221009A, it remained very rare throughout the Fermi's lifetime.
}
    
\end{itemize}

\section*{Acknowledgments}
We acknowledge the use of the Fermi archive's public data.
This work is supported by the Natural Science Foundation of China (NSFC) under grant No. 11921003,  No. 11933010 and No. 12003069.

\software{\texttt{Matplotlib} \citep{Hunter}, \texttt{Numpy} \citep{harris2020array}, %\texttt{scikit-learn} %\citep{scikit-learn},
	\texttt{bilby} \citep{ashton2019bilby}, \texttt{GBM Data Tools} \citep{GbmDataTools}}%, \texttt{Fermitools}}

\bibliography{bibtex}
%%%%%%%%%%%%%%%%%%%%%%%%%%%%%%%%%%%%%%%%%%%%%%%%%%%%
\appendix
\section{HEtools}
The development of high-energy detection data analysis tools ({\tt HEtools}) is the preparation for the future Very Large Area gamma-ray Space Telescope \citep[VLAST;][]{fan2022very}.
Its original intention is to provide reasonable observational references in order to optimize and expand the scientific output of VLAST, e.g. to make reasonable observational arrangements for the cascading radiation of GRB veay high energy (VHE) photons under the influence of intergalactic magnetic field \citep{2022arXiv221013052X}.
In order to test the feasibility of this tool, we apply it to the study of GRB.
As shown in Figure \ref{fig:HEtools}, {\tt HEtools} has separate modules for data analysis of different instruments, currently including but not limited to Swift-BAT, Fermi-GBM/LAT.
Python-based frameworks are very friendly to future instrument support, such as VLAST.
The corresponding light curve and energy spectrum can be generated through online data retrieval, and the preset spectrum inference or time series analysis can be performed, and finally a briefing is generated and sent to the user's mailbox.
Based on friendly expansion capabilities, by combining with A Standard Gamma-ray burst Afterglow Radiation Diagnoser \citep[{\tt ASGARD};][]{ren2022very}, it is used to moudle afterglow light curve and spectrum by considering various situations, various environments, and various physical processes.
It is possible to constrain the microscopic physical parameters of GRB by using a sampling method (e.g. {\tt MCMC}, {\tt Nested}, etc) for Bayesian inference.
And more features are under continuous development.

\label{app:1}
\begin{figure}[!h]
	\centering
	\includegraphics[width=0.8\textwidth]{HEtools.png}
	\caption{Framework of {\tt HEtools}.}
	\label{fig:HEtools}
\end{figure}
%%%%%%%%%%%%%%%%%%%%%%%%%%%%%%%%%%%%%%%%%%%%%%%%%%%%

\section{Spectral inference result}
\begin{longtable}{ccccccccc}
\label{tab:1}\\
%\centering
%\small
\caption{Spectral inference result}\\
\toprule
{Time Interval} & & {CPL Model} & & & {Band Model} & & & {Favored}\\
{[s]} & $\alpha$ & $E_{\rm p}$ [keV]& $\ln{\cal Z}$  & $\alpha$ & $\beta$ & $E_{\rm p}$ [keV] &  $\ln{\cal Z}$& \\
\endfirsthead
\caption{continued}\\
\toprule
{Time Interval} & & {CPL Model} & & & {Band Model} & & & {Favored}\\
{[s]} & $\alpha$ & $E_{\rm p}$ [keV]& $\ln{\cal Z}$  & $\alpha$ & $\beta$ & $E_{\rm p}$ [keV] &  $\ln{\cal Z}$& \\
\hline
\endhead
\hline
\endlastfoot
\hline
{$T_{90}$  [1.79 - 41.67]} & -0.989 $\pm$ 0.001 & 918.65 $\pm$ 1.82 & -3415.21 & -0.989 $\pm$ 0.001 & -9.88 $\pm$ 0.80 & 918.68 $\pm$ 1.85 & -3420.80 & CPL \\ 
{ME [0.00 - 18.50]} & -0.779 $\pm$ 0.001 & 951.83 $\pm$ 1.71 & -2206.84 & -0.778 $\pm$ 0.001 & -6.42 $\pm$ 0.66 & 951.08 $\pm$ 1.78 & -2208.51 & CPL \\ 
{EE [18.50 - 84.14]} & -1.464 $\pm$ 0.003 & 377.58 $\pm$ 2.91 & -1812.11 & -1.464 $\pm$ 0.003 & -9.18 $\pm$ 1.66 & 377.61 $\pm$ 2.91 & -1823.45 & CPL \\ 
\hline
{[-0.02 - 0.75]} & -0.989 $\pm$ 0.019 & 205.66 $\pm$ 3.41 & -348.83 & -0.884 $\pm$ 0.028 & -2.93 $\pm$ 0.09 & 180.66 $\pm$ 4.87 & -333.77 & Band \\ 
{[0.75 - 0.88]} & -0.495 $\pm$ 0.021 & 646.15 $\pm$ 12.81 & -227.72 & -0.485 $\pm$ 0.022 & -4.33 $\pm$ 1.49 & 636.80 $\pm$ 13.28 & -228.82 & CPL \\ 
{[0.88 - 0.94]} & -0.608 $\pm$ 0.035 & 633.67 $\pm$ 24.19 & -197.83 & -0.613 $\pm$ 0.034 & -8.89 $\pm$ 1.75 & 637.09 $\pm$ 23.71 & -200.55 & CPL \\ 
{[0.94 - 1.07]} & -0.410 $\pm$ 0.022 & 646.51 $\pm$ 11.65 & -216.47 & -0.397 $\pm$ 0.022 & -4.68 $\pm$ 1.58 & 637.84 $\pm$ 11.72 & -217.46 & CPL \\ 
{[1.07 - 1.46]} & -0.601 $\pm$ 0.014 & 534.75 $\pm$ 7.34 & -249.33 & -0.597 $\pm$ 0.014 & -5.14 $\pm$ 1.62 & 532.07 $\pm$ 7.30 & -251.61 & CPL \\ 
{[1.46 - 1.52]} & -0.427 $\pm$ 0.022 & 1179.50 $\pm$ 25.82 & -240.47 & -0.404 $\pm$ 0.025 & -3.92 $\pm$ 0.72 & 1140.25 $\pm$ 28.56 & -239.12 & Band \\ 
{[1.52 - 1.65]} & -0.269 $\pm$ 0.016 & 992.74 $\pm$ 11.49 & -308.71 & -0.269 $\pm$ 0.015 & -7.38 $\pm$ 1.21 & 993.53 $\pm$ 11.33 & -310.53 & CPL \\ 
{[1.65 - 1.84]} & -0.353 $\pm$ 0.012 & 1339.53 $\pm$ 13.87 & -319.24 & -0.353 $\pm$ 0.012 & -6.02 $\pm$ 1.23 & 1340.09 $\pm$ 14.22 & -320.75 & CPL \\ 
{[1.84 - 1.97]} & -0.238 $\pm$ 0.015 & 1216.99 $\pm$ 13.41 & -296.16 & -0.228 $\pm$ 0.016 & -5.07 $\pm$ 0.92 & 1205.26 $\pm$ 14.47 & -295.39 & Band \\ 
{[1.97 - 2.29]} & -0.250 $\pm$ 0.012 & 814.15 $\pm$ 6.74 & -392.14 & -0.245 $\pm$ 0.012 & -5.46 $\pm$ 1.07 & 811.38 $\pm$ 7.22 & -391.65 & Band \\ 
{[2.29 - 2.67]} & -0.258 $\pm$ 0.012 & 722.73 $\pm$ 6.40 & -266.76 & -0.258 $\pm$ 0.012 & -9.97 $\pm$ 1.19 & 722.41 $\pm$ 6.34 & -268.30 & CPL \\ 
{[2.67 - 3.12]} & -0.520 $\pm$ 0.015 & 513.65 $\pm$ 6.82 & -265.82 & -0.517 $\pm$ 0.015 & -4.32 $\pm$ 1.05 & 510.48 $\pm$ 7.03 & -265.63 & Band \\ 
{[3.12 - 3.18]} & -0.448 $\pm$ 0.024 & 812.11 $\pm$ 17.89 & -209.68 & -0.444 $\pm$ 0.024 & -8.69 $\pm$ 1.48 & 809.51 $\pm$ 17.89 & -212.39 & CPL \\ 
{[3.18 - 3.50]} & -0.346 $\pm$ 0.008 & 1177.94 $\pm$ 8.14 & -503.61 & -0.345 $\pm$ 0.008 & -9.64 $\pm$ 0.94 & 1176.45 $\pm$ 8.19 & -506.08 & CPL \\ 
{[3.50 - 3.57]} & -0.365 $\pm$ 0.023 & 929.07 $\pm$ 18.28 & -245.53 & -0.357 $\pm$ 0.023 & -5.13 $\pm$ 1.59 & 922.41 $\pm$ 17.83 & -246.64 & CPL \\ 
{[3.57 - 3.70]} & -0.402 $\pm$ 0.019 & 733.25 $\pm$ 12.16 & -241.11 & -0.404 $\pm$ 0.019 & -6.59 $\pm$ 1.37 & 734.88 $\pm$ 11.93 & -243.53 & CPL \\ 
{[3.70 - 3.76]} & -0.564 $\pm$ 0.031 & 786.43 $\pm$ 26.46 & -184.06 & -0.565 $\pm$ 0.031 & -6.64 $\pm$ 1.68 & 786.91 $\pm$ 25.55 & -186.87 & CPL \\ 
{[3.76 - 3.82]} & -0.431 $\pm$ 0.027 & 725.72 $\pm$ 17.46 & -175.04 & -0.429 $\pm$ 0.029 & -5.10 $\pm$ 1.74 & 726.57 $\pm$ 18.95 & -176.87 & CPL \\ 
{[3.82 - 4.14]} & -0.553 $\pm$ 0.013 & 706.37 $\pm$ 9.43 & -255.65 & -0.548 $\pm$ 0.013 & -4.63 $\pm$ 1.72 & 698.26 $\pm$ 9.98 & -257.39 & CPL \\ 
{[4.14 - 4.40]} & -0.538 $\pm$ 0.012 & 1009.16 $\pm$ 13.33 & -260.69 & -0.536 $\pm$ 0.012 & -6.49 $\pm$ 1.37 & 1007.10 $\pm$ 13.19 & -263.53 & CPL \\ 
{[4.40 - 4.53]} & -0.612 $\pm$ 0.019 & 775.85 $\pm$ 16.83 & -224.92 & -0.589 $\pm$ 0.021 & -3.89 $\pm$ 1.60 & 748.23 $\pm$ 20.45 & -226.03 & CPL \\ 
{[4.53 - 4.72]} & -0.498 $\pm$ 0.015 & 698.58 $\pm$ 9.65 & -238.00 & -0.486 $\pm$ 0.016 & -4.45 $\pm$ 1.54 & 687.74 $\pm$ 11.35 & -240.34 & CPL \\ 
{[4.72 - 4.98]} & -0.529 $\pm$ 0.010 & 904.62 $\pm$ 9.39 & -286.61 & -0.529 $\pm$ 0.010 & -6.78 $\pm$ 1.25 & 903.83 $\pm$ 9.26 & -289.24 & CPL \\ 
{[4.98 - 5.42]} & -0.557 $\pm$ 0.007 & 939.10 $\pm$ 6.78 & -349.13 & -0.553 $\pm$ 0.007 & -5.30 $\pm$ 1.00 & 934.38 $\pm$ 6.75 & -349.47 & CPL \\ 
{[5.42 - 5.68]} & -0.633 $\pm$ 0.010 & 836.96 $\pm$ 9.83 & -281.84 & -0.631 $\pm$ 0.010 & -5.21 $\pm$ 1.40 & 834.22 $\pm$ 9.80 & -284.53 & CPL \\ 
{[5.68 - 5.81]} & -0.562 $\pm$ 0.012 & 1160.26 $\pm$ 16.10 & -262.79 & -0.561 $\pm$ 0.012 & -5.75 $\pm$ 1.49 & 1158.36 $\pm$ 15.98 & -265.01 & CPL \\ 
{[5.81 - 6.90]} & -0.546 $\pm$ 0.004 & 1205.71 $\pm$ 5.05 & -794.40 & -0.545 $\pm$ 0.004 & -6.53 $\pm$ 1.15 & 1204.96 $\pm$ 5.21 & -797.14 & CPL \\ 
{[6.90 - 6.96]} & -0.673 $\pm$ 0.021 & 882.81 $\pm$ 24.01 & -192.76 & -0.672 $\pm$ 0.021 & -9.30 $\pm$ 1.52 & 884.26 $\pm$ 23.52 & -195.88 & CPL \\ 
{[6.96 - 7.22]} & -0.725 $\pm$ 0.013 & 712.50 $\pm$ 11.61 & -212.19 & -0.720 $\pm$ 0.013 & -4.64 $\pm$ 1.47 & 706.40 $\pm$ 12.10 & -214.66 & CPL \\ 
{[7.22 - 7.28]} & -0.583 $\pm$ 0.020 & 898.18 $\pm$ 20.53 & -204.75 & -0.567 $\pm$ 0.021 & -4.01 $\pm$ 1.09 & 874.17 $\pm$ 23.02 & -204.98 & CPL \\ 
{[7.28 - 7.47]} & -0.718 $\pm$ 0.013 & 755.90 $\pm$ 12.06 & -211.25 & -0.710 $\pm$ 0.014 & -4.27 $\pm$ 1.38 & 742.98 $\pm$ 13.37 & -212.98 & CPL \\ 
{[7.47 - 7.66]} & -0.789 $\pm$ 0.016 & 516.51 $\pm$ 10.44 & -232.59 & -0.761 $\pm$ 0.021 & -3.59 $\pm$ 1.68 & 492.78 $\pm$ 14.99 & -234.84 & CPL \\ 
{[7.66 - 7.98]} & -0.762 $\pm$ 0.009 & 742.30 $\pm$ 9.09 & -294.85 & -0.760 $\pm$ 0.009 & -5.16 $\pm$ 1.46 & 738.18 $\pm$ 9.50 & -297.19 & CPL \\ 
{[7.98 - 8.62]} & -0.682 $\pm$ 0.005 & 1208.32 $\pm$ 8.13 & -416.09 & -0.677 $\pm$ 0.005 & -4.95 $\pm$ 0.46 & 1198.17 $\pm$ 9.22 & -414.64 & Band \\ 
{[8.62 - 8.88]} & -0.681 $\pm$ 0.010 & 888.19 $\pm$ 11.28 & -274.44 & -0.679 $\pm$ 0.010 & -5.25 $\pm$ 1.46 & 883.35 $\pm$ 11.45 & -276.92 & CPL \\ 
{[8.88 - 9.33]} & -0.741 $\pm$ 0.008 & 819.98 $\pm$ 8.74 & -291.80 & -0.735 $\pm$ 0.009 & -4.65 $\pm$ 1.21 & 811.19 $\pm$ 10.20 & -294.20 & CPL \\ 
{[9.33 - 9.52]} & -0.708 $\pm$ 0.011 & 783.93 $\pm$ 11.14 & -267.81 & -0.703 $\pm$ 0.012 & -4.50 $\pm$ 1.66 & 775.44 $\pm$ 12.79 & -270.09 & CPL \\ 
{[9.52 - 9.71]} & -0.789 $\pm$ 0.011 & 1045.08 $\pm$ 17.27 & -235.41 & -0.781 $\pm$ 0.012 & -4.31 $\pm$ 1.87 & 1028.52 $\pm$ 19.43 & -239.10 & CPL \\ 
{[9.71 - 9.97]} & -0.768 $\pm$ 0.010 & 1062.86 $\pm$ 16.12 & -275.58 & -0.766 $\pm$ 0.010 & -4.38 $\pm$ 1.57 & 1051.19 $\pm$ 17.40 & -278.17 & CPL \\ 
{[9.97 - 10.10]} & -0.743 $\pm$ 0.013 & 1131.57 $\pm$ 21.65 & -262.97 & -0.743 $\pm$ 0.013 & -5.82 $\pm$ 1.63 & 1127.88 $\pm$ 21.93 & -266.25 & CPL \\ 
{[10.10 - 10.35]} & -0.711 $\pm$ 0.009 & 1120.10 $\pm$ 13.90 & -281.95 & -0.699 $\pm$ 0.009 & -4.06 $\pm$ 0.24 & 1092.27 $\pm$ 15.49 & -275.88 & Band \\ 
{[10.35 - 10.54]} & -0.720 $\pm$ 0.013 & 850.01 $\pm$ 14.60 & -248.45 & -0.712 $\pm$ 0.014 & -4.07 $\pm$ 1.37 & 831.56 $\pm$ 16.84 & -250.31 & CPL \\ 
{[10.54 - 10.67]} & -0.759 $\pm$ 0.012 & 975.23 $\pm$ 17.78 & -224.87 & -0.758 $\pm$ 0.013 & -4.40 $\pm$ 1.78 & 967.61 $\pm$ 19.52 & -228.18 & CPL \\ 
{[10.67 - 10.99]} & -0.796 $\pm$ 0.010 & 820.59 $\pm$ 11.60 & -250.28 & -0.795 $\pm$ 0.010 & -5.49 $\pm$ 1.55 & 817.07 $\pm$ 11.40 & -254.16 & CPL \\ 
{[10.99 - 11.18]} & -0.841 $\pm$ 0.015 & 673.09 $\pm$ 14.68 & -268.75 & -0.837 $\pm$ 0.015 & -4.31 $\pm$ 1.86 & 669.09 $\pm$ 15.55 & -272.80 & CPL \\ 
{[11.18 - 11.44]} & -0.837 $\pm$ 0.010 & 841.01 $\pm$ 12.79 & -282.62 & -0.836 $\pm$ 0.010 & -4.99 $\pm$ 1.59 & 837.96 $\pm$ 13.08 & -286.08 & CPL \\ 
{[11.44 - 11.76]} & -0.894 $\pm$ 0.012 & 599.32 $\pm$ 10.22 & -300.34 & -0.888 $\pm$ 0.012 & -4.18 $\pm$ 1.62 & 590.97 $\pm$ 11.69 & -303.59 & CPL \\ 
{[11.76 - 11.82]} & -0.980 $\pm$ 0.032 & 513.66 $\pm$ 26.26 & -186.13 & -0.982 $\pm$ 0.032 & -9.22 $\pm$ 1.76 & 516.04 $\pm$ 25.98 & -190.74 & CPL \\ 
{[11.82 - 12.08]} & -0.891 $\pm$ 0.009 & 944.59 $\pm$ 15.19 & -317.53 & -0.889 $\pm$ 0.009 & -5.16 $\pm$ 1.67 & 942.72 $\pm$ 15.91 & -321.37 & CPL \\ 
{[12.08 - 12.66]} & -0.935 $\pm$ 0.007 & 907.54 $\pm$ 11.30 & -320.10 & -0.933 $\pm$ 0.007 & -5.02 $\pm$ 1.62 & 904.34 $\pm$ 11.32 & -324.15 & CPL \\ 
{[12.66 - 12.91]} & -0.870 $\pm$ 0.011 & 759.83 $\pm$ 12.37 & -277.93 & -0.862 $\pm$ 0.012 & -3.94 $\pm$ 1.25 & 744.56 $\pm$ 15.95 & -279.45 & CPL \\ 
{[12.91 - 13.04]} & -0.910 $\pm$ 0.012 & 780.91 $\pm$ 16.11 & -276.48 & -0.888 $\pm$ 0.014 & -3.62 $\pm$ 0.34 & 740.75 $\pm$ 19.34 & -274.54 & Band \\ 
{[13.04 - 13.36]} & -0.972 $\pm$ 0.008 & 868.19 $\pm$ 13.29 & -294.04 & -0.969 $\pm$ 0.008 & -4.54 $\pm$ 1.50 & 862.00 $\pm$ 13.89 & -297.40 & CPL \\ 
{[13.36 - 13.49]} & -1.014 $\pm$ 0.014 & 859.05 $\pm$ 23.65 & -223.34 & -1.006 $\pm$ 0.014 & -4.24 $\pm$ 1.89 & 844.43 $\pm$ 24.92 & -227.54 & CPL \\ 
{[13.49 - 14.00]} & -1.053 $\pm$ 0.009 & 740.81 $\pm$ 12.80 & -353.83 & -1.048 $\pm$ 0.009 & -4.19 $\pm$ 1.85 & 730.31 $\pm$ 14.15 & -357.77 & CPL \\ 
{[14.00 - 14.38]} & -1.045 $\pm$ 0.008 & 902.13 $\pm$ 15.62 & -324.21 & -1.041 $\pm$ 0.009 & -4.05 $\pm$ 1.29 & 886.72 $\pm$ 18.23 & -327.35 & CPL \\ 
{[14.38 - 14.51]} & -1.089 $\pm$ 0.022 & 511.97 $\pm$ 21.33 & -237.81 & -1.071 $\pm$ 0.026 & -3.18 $\pm$ 2.23 & 485.06 $\pm$ 26.48 & -242.08 & CPL \\ 
{[14.51 - 14.96]} & -1.161 $\pm$ 0.010 & 595.94 $\pm$ 12.24 & -362.61 & -1.151 $\pm$ 0.011 & -3.66 $\pm$ 1.64 & 578.34 $\pm$ 14.68 & -366.03 & CPL \\ 
{[14.96 - 15.15]} & -1.042 $\pm$ 0.012 & 718.04 $\pm$ 17.20 & -262.66 & -1.037 $\pm$ 0.012 & -4.05 $\pm$ 1.95 & 707.72 $\pm$ 17.91 & -266.76 & CPL \\ 
{[15.15 - 15.79]} & -1.100 $\pm$ 0.008 & 610.22 $\pm$ 9.94 & -387.46 & -1.098 $\pm$ 0.008 & -4.67 $\pm$ 1.72 & 608.71 $\pm$ 9.94 & -391.12 & CPL \\ 
{[15.79 - 16.69]} & -1.110 $\pm$ 0.007 & 773.09 $\pm$ 11.78 & -418.04 & -1.104 $\pm$ 0.007 & -3.88 $\pm$ 0.67 & 757.67 $\pm$ 13.70 & -419.30 & CPL \\ 
{[16.69 - 17.46]} & -1.177 $\pm$ 0.009 & 542.63 $\pm$ 10.64 & -333.94 & -1.178 $\pm$ 0.009 & -9.52 $\pm$ 1.57 & 541.82 $\pm$ 10.65 & -339.17 & CPL \\ 
{[17.46 - 17.65]} & -1.131 $\pm$ 0.023 & 437.91 $\pm$ 17.85 & -225.09 & -1.134 $\pm$ 0.024 & -4.07 $\pm$ 1.99 & 436.03 $\pm$ 17.93 & -230.21 & CPL \\ 
{[17.65 - 17.90]} & -1.207 $\pm$ 0.033 & 255.46 $\pm$ 11.56 & -232.97 & -1.193 $\pm$ 0.035 & -2.97 $\pm$ 1.62 & 244.57 $\pm$ 12.66 & -236.99 & CPL \\ 
{[17.90 - 18.67]} & -1.510 $\pm$ 0.026 & 211.49 $\pm$ 10.72 & -246.95 & -1.499 $\pm$ 0.027 & -3.04 $\pm$ 2.26 & 204.69 $\pm$ 10.93 & -253.03 & CPL \\ 
{[18.67 - 18.99]} & -1.383 $\pm$ 0.023 & 361.83 $\pm$ 20.30 & -262.07 & -1.383 $\pm$ 0.023 & -4.82 $\pm$ 1.94 & 364.25 $\pm$ 19.76 & -268.33 & CPL \\ 
{[18.99 - 19.12]} & -1.297 $\pm$ 0.031 & 399.53 $\pm$ 28.47 & -206.72 & -1.173 $\pm$ 0.056 & -2.39 $\pm$ 0.79 & 284.87 $\pm$ 41.62 & -209.25 & CPL \\ 
{[19.12 - 19.82]} & -1.251 $\pm$ 0.009 & 546.93 $\pm$ 12.20 & -374.74 & -1.248 $\pm$ 0.010 & -3.84 $\pm$ 1.98 & 539.93 $\pm$ 12.87 & -379.88 & CPL \\ 
{[19.82 - 20.40]} & -1.215 $\pm$ 0.008 & 808.29 $\pm$ 17.91 & -393.03 & -1.214 $\pm$ 0.008 & -4.34 $\pm$ 1.94 & 799.98 $\pm$ 18.51 & -398.04 & CPL \\ 
{[20.40 - 21.10]} & -1.157 $\pm$ 0.010 & 601.17 $\pm$ 12.79 & -331.86 & -1.157 $\pm$ 0.010 & -5.14 $\pm$ 1.72 & 599.88 $\pm$ 12.97 & -337.31 & CPL \\ 
{[21.10 - 21.68]} & -1.144 $\pm$ 0.009 & 616.20 $\pm$ 12.88 & -343.02 & -1.128 $\pm$ 0.011 & -3.35 $\pm$ 0.21 & 587.33 $\pm$ 14.78 & -339.28 & Band \\ 
{[21.68 - 23.22]} & -1.195 $\pm$ 0.007 & 513.65 $\pm$ 7.77 & -414.16 & -1.195 $\pm$ 0.007 & -7.97 $\pm$ 1.42 & 513.74 $\pm$ 7.62 & -420.12 & CPL \\ 
{[23.22 - 23.54]} & -1.182 $\pm$ 0.026 & 241.79 $\pm$ 7.65 & -209.84 & -1.179 $\pm$ 0.027 & -3.59 $\pm$ 2.12 & 238.17 $\pm$ 7.84 & -215.21 & CPL \\ 
{[23.54 - 25.01]} & -1.285 $\pm$ 0.007 & 523.19 $\pm$ 9.34 & -413.35 & -1.286 $\pm$ 0.007 & -6.52 $\pm$ 1.64 & 523.90 $\pm$ 9.33 & -419.18 & CPL \\ 
{[25.01 - 25.20]} & -1.191 $\pm$ 0.029 & 312.28 $\pm$ 13.61 & -239.49 & -1.185 $\pm$ 0.029 & -4.60 $\pm$ 1.90 & 310.79 $\pm$ 13.34 & -245.16 & CPL \\ 
{[25.20 - 25.65]} & -1.336 $\pm$ 0.014 & 379.73 $\pm$ 10.97 & -288.55 & -1.285 $\pm$ 0.030 & -2.74 $\pm$ 2.26 & 326.93 $\pm$ 27.96 & -294.32 & CPL \\ 
{[25.65 - 25.78]} & -1.454 $\pm$ 0.030 & 332.57 $\pm$ 24.93 & -192.48 & -1.453 $\pm$ 0.031 & -4.43 $\pm$ 2.07 & 331.63 $\pm$ 25.57 & -199.54 & CPL \\ 
{[25.78 - 25.97]} & -1.305 $\pm$ 0.022 & 342.05 $\pm$ 13.91 & -226.09 & -1.304 $\pm$ 0.022 & -7.41 $\pm$ 1.87 & 342.16 $\pm$ 14.04 & -231.78 & CPL \\ 
{[25.97 - 26.16]} & -1.407 $\pm$ 0.030 & 221.10 $\pm$ 10.52 & -222.82 & -1.394 $\pm$ 0.031 & -3.41 $\pm$ 2.13 & 217.30 $\pm$ 11.40 & -228.83 & CPL \\ 
{[26.16 - 26.54]} & -1.460 $\pm$ 0.026 & 157.06 $\pm$ 5.34 & -226.38 & -1.374 $\pm$ 0.052 & -2.70 $\pm$ 2.05 & 133.85 $\pm$ 12.03 & -231.84 & CPL \\ 
{[26.54 - 26.74]} & -1.539 $\pm$ 0.024 & 303.73 $\pm$ 20.87 & -258.59 & -1.502 $\pm$ 0.037 & -2.50 $\pm$ 1.24 & 261.14 $\pm$ 29.33 & -263.53 & CPL \\ 
{[26.74 - 27.95]} & -1.562 $\pm$ 0.011 & 354.38 $\pm$ 12.06 & -390.35 & -1.561 $\pm$ 0.011 & -9.30 $\pm$ 1.73 & 353.78 $\pm$ 11.88 & -398.72 & CPL \\ 
{[27.95 - 28.46]} & -1.434 $\pm$ 0.011 & 568.87 $\pm$ 21.29 & -286.64 & -1.434 $\pm$ 0.011 & -3.93 $\pm$ 1.93 & 569.16 $\pm$ 20.82 & -293.03 & CPL \\ 
{[28.46 - 28.72]} & -1.443 $\pm$ 0.019 & 601.31 $\pm$ 43.14 & -215.41 & -1.434 $\pm$ 0.019 & -3.29 $\pm$ 2.21 & 579.88 $\pm$ 43.22 & -221.65 & CPL \\ 
{[28.72 - 30.19]} & -1.435 $\pm$ 0.010 & 509.42 $\pm$ 16.94 & -330.77 & -1.436 $\pm$ 0.010 & -4.27 $\pm$ 1.95 & 508.38 $\pm$ 17.27 & -337.68 & CPL \\ 
{[30.19 - 31.02]} & -1.368 $\pm$ 0.021 & 300.46 $\pm$ 12.67 & -274.87 & -1.367 $\pm$ 0.021 & -9.62 $\pm$ 1.92 & 299.86 $\pm$ 12.86 & -281.80 & CPL \\ 
{[31.02 - 31.98]} & -1.455 $\pm$ 0.023 & 169.70 $\pm$ 5.69 & -259.46 & -1.457 $\pm$ 0.024 & -5.83 $\pm$ 1.85 & 169.84 $\pm$ 5.76 & -265.57 & CPL \\ 
{[31.98 - 32.82]} & -1.604 $\pm$ 0.029 & 119.93 $\pm$ 4.98 & -265.19 & -1.603 $\pm$ 0.030 & -9.04 $\pm$ 1.84 & 120.21 $\pm$ 5.09 & -273.58 & CPL \\ 
{[32.82 - 34.74]} & -1.526 $\pm$ 0.013 & 261.60 $\pm$ 7.58 & -350.06 & -1.525 $\pm$ 0.013 & -9.74 $\pm$ 1.84 & 262.29 $\pm$ 7.53 & -358.65 & CPL \\ 
{[34.74 - 35.63]} & -1.503 $\pm$ 0.016 & 343.03 $\pm$ 14.96 & -276.88 & -1.500 $\pm$ 0.016 & -3.46 $\pm$ 2.13 & 338.91 $\pm$ 15.19 & -285.07 & CPL \\ 
{[35.63 - 37.49]} & -1.531 $\pm$ 0.015 & 196.96 $\pm$ 5.36 & -302.39 & -1.529 $\pm$ 0.015 & -5.57 $\pm$ 1.86 & 196.63 $\pm$ 5.37 & -310.48 & CPL \\ 
{[37.49 - 39.02]} & -1.426 $\pm$ 0.022 & 193.36 $\pm$ 6.32 & -247.76 & -1.417 $\pm$ 0.022 & -3.25 $\pm$ 2.17 & 189.18 $\pm$ 6.46 & -255.64 & CPL \\ 
{[39.02 - 40.37]} & -1.577 $\pm$ 0.025 & 214.94 $\pm$ 12.43 & -288.73 & -1.576 $\pm$ 0.024 & -7.22 $\pm$ 1.90 & 215.57 $\pm$ 11.82 & -301.05 & CPL \\ 
{[40.37 - 41.65]} & -1.651 $\pm$ 0.042 & 89.60 $\pm$ 5.44 & -241.34 & -1.658 $\pm$ 0.054 & -4.64 $\pm$ 1.94 & 89.33 $\pm$ 5.80 & -255.13 & CPL \\ 
{[41.65 - 44.66]} & -1.745 $\pm$ 0.018 & 137.80 $\pm$ 6.08 & -274.03 & -1.745 $\pm$ 0.018 & -8.25 $\pm$ 1.81 & 138.73 $\pm$ 6.10 & -290.60 & CPL \\ 
{[44.66 - 45.94]} & -1.657 $\pm$ 0.052 & 82.54 $\pm$ 6.50 & -215.06 & -1.659 $\pm$ 0.052 & -3.94 $\pm$ 2.03 & 81.62 $\pm$ 6.22 & -220.89 & CPL \\ 
{[45.94 - 47.98]} & -1.601 $\pm$ 0.021 & 202.42 $\pm$ 10.01 & -270.32 & -1.593 $\pm$ 0.022 & -3.03 $\pm$ 2.19 & 195.88 $\pm$ 10.22 & -285.71 & CPL \\ 
{[47.98 - 51.31]} & -1.506 $\pm$ 0.024 & 151.10 $\pm$ 5.15 & -303.35 & -1.488 $\pm$ 0.025 & -2.97 $\pm$ 2.21 & 145.00 $\pm$ 5.70 & -310.94 & CPL \\ 
{[51.31 - 52.66]} & -1.616 $\pm$ 0.084 & 53.97 $\pm$ 5.52 & -242.21 & -1.521 $\pm$ 0.109 & -2.81 $\pm$ 2.29 & 51.41 $\pm$ 4.89 & -250.67 & CPL \\ 
{[52.66 - 55.15]} & -1.529 $\pm$ 0.030 & 132.77 $\pm$ 5.37 & -262.53 & -1.527 $\pm$ 0.030 & -8.32 $\pm$ 1.95 & 133.19 $\pm$ 5.47 & -271.82 & CPL \\ 
{[55.15 - 59.44]} & -1.697 $\pm$ 0.045 & 51.68 $\pm$ 3.60 & -310.38 & -1.695 $\pm$ 0.051 & -9.11 $\pm$ 1.92 & 52.17 $\pm$ 3.61 & -316.46 & CPL \\ 
{[59.44 - 62.58]} & -1.497 $\pm$ 0.034 & 117.84 $\pm$ 4.67 & -292.39 & -1.497 $\pm$ 0.034 & -8.05 $\pm$ 1.85 & 117.75 $\pm$ 4.71 & -303.78 & CPL \\ 
{[62.58 - 65.20]} & -1.772 $\pm$ 0.088 & 38.41 $\pm$ 12.19 & -267.84 & -1.766 $\pm$ 0.091 & -3.47 $\pm$ 2.09 & 39.29 $\pm$ 9.84 & -272.76 & CPL \\ 
{[65.20 - 68.02]} & -1.597 $\pm$ 0.041 & 166.11 $\pm$ 14.93 & -271.10 & -1.601 $\pm$ 0.040 & -6.06 $\pm$ 2.03 & 167.10 $\pm$ 14.28 & -284.43 & CPL \\ 
{[68.02 - 72.50]} & -1.539 $\pm$ 0.043 & 127.30 $\pm$ 7.95 & -298.35 & -1.543 $\pm$ 0.044 & -7.68 $\pm$ 2.01 & 126.77 $\pm$ 7.99 & -310.06 & CPL \\ 
{[79.47 - 84.14]} & -1.590 $\pm$ 0.068 & 99.18 $\pm$ 10.05 & -306.67 & -1.591 $\pm$ 0.076 & -5.57 $\pm$ 2.12 & 98.78 $\pm$ 9.75 & -322.78 & CPL \\ 
\end{longtable}

%%%%%%%%%%%%%%%%%%%%%%%%%%%%%%%%%%%%%%%%%%%%%%%%%%%%
\end{document}
