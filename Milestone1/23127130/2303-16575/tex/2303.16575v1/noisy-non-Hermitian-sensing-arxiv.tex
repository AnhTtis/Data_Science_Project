
% ****** Start of file apssamp.tex ******
%
%   This file is part of the APS files in the REVTeX 4.1 distribution.
%   Version 4.1r of REVTeX, August 2010
%
%   Copyright (c) 2009, 2010 The American Physical Society.
%
%   See the REVTeX 4 README file for restrictions and more information.
%
% TeX'ing this file requires that you have AMS-LaTeX 2.0 installed
% as well as the rest of the prerequisites for REVTeX 4.1
%
% See the REVTeX 4 README file
% It also requires running BibTeX. The commands are as follows:
%
%  1)  latex apssamp.tex
%  2)  bibtex apssamp
%  3)  latex apssamp.tex
%  4)  latex apssamp.tex
%
\documentclass[%
%
reprint,
%superscriptaddress,
%groupedaddress,
%unsortedaddress,
%runinaddress,
%frontmatterverbose,
%preprint,
%showpacs,preprintnumbers,
%nofootinbib,
%nobibnotes,
%bibnotes,
 amsmath,amssymb,
 aps,
pra,
%prb,
%rmp,
%prstab,
%prstper,
%floatfix,
]{revtex4-1}
\usepackage{amsmath}
%\usepackage{caption}


\usepackage{graphicx}% Include figure files
\usepackage{subfigure}
\usepackage{dcolumn}% Align table columns on decimal point
\usepackage{bm}
\usepackage{epstopdf}
\usepackage{epsfig}% bold math
\usepackage[colorlinks,
            linkcolor=blue,
            anchorcolor=blue,
            citecolor=blue
            ]{hyperref}
\usepackage{color}

\hyphenpenalty=5000
\tolerance=1000
%\usepackage{hyperref}% add hypertext capabilities
%\usepackage[mathlines]{lineno}% Enable numbering of text and display math
%\linenumbers\relax % Commence numbering lines

%\usepackage[showframe,%Uncomment any one of the following lines to test
%%scale=0.7, marginratio={1:1, 2:3}, ignoreall,% default settings
%%text={7in,10in},centering,
%%margin=1.5in,
%%total={6.5in,8.75in}, top=1.2in, left=0.9in, includefoot,
%%height=10in,a5paper,hmargin={3cm,0.8in},
%]{geometry}

\begin{document}

%\preprint{APS/123-QED}

\title{Exponential Sensitivity Revival and Robust Stability of Noisy \\non-Hermitian Quantum Sensing}% Force line breaks with \\
%\thanks{qibo@amss.ac.cn}%
\author{Liying Bao$^{1,2,3}$, Bo Qi$^{1,2\star}$,  Franco Nori$^{4,5,6}$, Daoyi Dong$^{7}$
\\
$^{1}$\textit{Key Laboratory of Systems and Control, Academy of Mathematics and Systems Science, Chinese Academy of Sciences, Beijing 100190, People's Republic of China}\\
$^{2}$\textit{University of Chinese Academy of Sciences, Beijing 100049, People's Republic of China}\\
$^{3}$\textit{Civil Aviation University of China, Tianjin 300300, China}\\
$^{4}$\textit{Theoretical Quantum Physics Laboratory, RIKEN, Saitama, 351-0198, Japan}\\
$^{5}$\textit{Quantum Computing Center, RIKEN, Saitama, 351-0198, Japan}\\
$^{6}$\textit{Physics Department, The University of Michigan, Ann Arbor, Michigan 48109, USA}\\
$^{7}$\textit{School of Engineering and Information Technology, University of New South Wales, Canberra ACT 2600, Australia}\\
$^\star$\textit{qibo@amss.ac.cn}}
%\collaboration{CLEO Collaboration}%\noaffiliation

\date{\today}% It is always \today, today,
             %  but any date may be explicitly specified

\begin{abstract}
Unique properties of multimode non-Hermitian (NH) lattice dynamics can be utilized to construct  exponentially sensitive sensors. The impact of noise however remains unclear, which may greatly degrade the ability to detect small parameter changes. We analytically characterize and highlight the impact of the structure of loss and gain on the sensitivity and stability of NH quantum sensors. Counter-intuitively, we find that by only tuning the loss structure properly, the exponential sensitivity can be surprisingly regained when the sensing dynamics is stable. Furthermore, we prove that the gain is crucial to ensure the stability of the NH sensor by making a balanced loss and gain.  For unbalanced noise, we demonstrate that there is a striking tradeoff between the enhancement of the sensitivity  and the exponential decrement of the robust stability. This work demonstrates a clear signature about the impact of noise on the sensitivity and stability of NH quantum sensors, and has potential applications in quantum sensing and quantum engineering.
\end{abstract}

%\pacs{Valid PACS appear here} % PACS, the Physics and Astronomy
                             % Classification Scheme.
%\keywords{Suggested keywords}%Use showkeys class option if keyword
                              %display desired
\maketitle

%\tableofcontents
\textit{Introduction.---}
High precision sensors are ubiquitous and vitally important in both science and technology. Due to the high susceptibility of the complex energy  spectra of non-Hermitian (NH) Hamiltonians in response to small perturbations, NH sensors have been attracting increasing attention. Various unconventional properties  of NH systems have been studied to theoretically propose high precision sensors \cite{Peng2014,ZhangJing2018,Leykam2017,G.-Q.Zhang2021,El-Ganainy2018,Bliokh2019,Chu2020,ma2011,Liu2019,Bensa2021,Lee2023,Okuma2020,Scheibner2020,Kawabata2020}, and some  architectures  have already been experimentally realized \cite{Park2021,Chen2021,xu2022}. In this work, we investigate the sensitivity revival and stability of NH quantum sensing in noisy environments.

%Non-Hermitian systems \cite{ZhangJing2018, Roberts2021, G.-Q.Zhang2021, Xiao2021, Bensa2021, Budich2020, Bliokh2019, Pan2020, Cao2020, Liu2019, Chu2020, Sun2020, Koch2021, Rao2021, Park2021} have received increasing attention in theoretical analysis and technology because they can better describe the real dynamic evolution. Some peculiar properties, such as the, exceptional points \cite{Wiersig2014, Leykam2017, Zhang2019, Wiersig2016, Chen2019, Smith2020, WangGao2020, Rui2019, Lange2020, Pap2021}, skin effect \cite{Baoli2021, McDonald2020, Okuma2020, Scheibner2020, Kawabata2020} and non-reciprocity \cite{Tzuang2014, Metelmann2015, Sounas2017, Lau2018, Bao2021}, have been increasingly studied in various fields, such as quantum sensing, condensed matter, and optomechanical systems.

Recent progress has shown that the intriguing degeneracy property of NH systems can be employed to enhance the sensitivity of sensors operating at the exceptional point (EP), where the coalesced energies have a diverging susceptibility to small perturbations \cite{Wiersig2014,ozdemir2019, Zhang2019, Wiersig2016, Chen2019, WangGao2020, Rui2019, Jiang2022, Pap2021}. To assess the performance of EP sensors, the effect of the coalesced eigenstates should also be taken into account \cite{Wiersig2016, Chen2019}. Other distinct properties of NH systems have also been harnessed to enhance the sensitivity of  NH sensors, which do not necessarily work at EPs. As studied in  \cite{Lau2018},  nonreciprocity \cite{Tzuang2014,TangJ2022, Tang2022, Lai2020,Sounas2017, Lau2018, Bao2021} can be identified as a powerful resource for sensing. In \cite{Bao2021}, by introducing two coherent excitations, a unified limit was set up for both reciprocal and nonreciprocal sensors.   Remarkably, a class of sensors having exponential sensitivity were proposed \cite{McDonald2020,Budich2020,Qin2018,Koch2021,Baoli2021}. The drastic enhancements rely upon the strikingly anomalous sensitivity to the boundary conditions of NH systems. Furthermore, the implication of optimizing controllable parameters in attaining an exponential enhancement was investigated in \cite{Baoli2021,Chen2021}. However, in practical applications the existence of noise is unavoidable, which may severely degrade  the performance,   such as the sensitivity and stability of NH sensors. Stability is an essential requirement for  high precision sensing. This work focuses on understanding the impact of noise on  the sensitivity and stability of NH sensing and finding out how to engineer the noise structure to achieve exponentially enhanced and stable NH quantum sensing.



%Exceptional points are concerned in the field of quantum sensing because near the exceptional points, the eigenenergies will produce divergent sensitivity on small parameter changes. However, the coalescence of eigenstates may offset the divergence of the eigenvalue susceptibility, thereby making the sensitivity to disturbance disappear, so the actual effect needs to be carefully considered \cite{Wiersig2016, Chen2019, Smith2020}. Therefore, it is necessary to be very careful when evaluating the actual measurement precision of sensors with exceptional points. The skin effect refers to the sensitivity of the system to changes in boundary conditions, which can achieve the presicion of exponential amplification with the size of the system, and can also greatly improve the measurement precision under nonlinear response \cite{Baoli2021}. Another recognizable characteristic, non-reciprocity, proved to be a powerful resource for quantum sensing. Even in the presence of quantum noise, the accuracy of non-reciprocal sensors can still be arbitrarily improved compared to reciprocal sensors \cite{Lau2018}. Reciprocal sensors with two excitation signals can effectively simulate non-reciprocal sensors, which still have advantages in the selection of system parameters \cite{Bao2021}.

We consider loss and gain in NH sensors. {\it Loss} noise has an essential impact on the attainable sensitivity in quantum sensing \cite{Escher2011, Datta2011,Demkowicz2012,Zhou2018}. For conventional sensors, it is well-known that by using quantum strategies, the precision can be scaled as $1/N$ in terms of the number $N$ of resources for noiseless processes \cite{Giovannetti2006,Napolitano2011,Thomas2011,Huelga1997,Hou2019,Hou2020,Yuan2015}. However, it was demonstrated in  \cite{Escher2011, Datta2011,Demkowicz2012,Zhou2018} that even a weak loss noise can degrade the precision from $1/N$  to $1/\sqrt{N}$,  independently of the initial state of the probes and even regardless of the use of adaptive feedback.
%Previous papers have introduced a non-reciprocal lattice model as a sensing sensor to perceive small disturbances in the Hamiltonian, and exponential amplification can be achieved when there is no environmental noise \cite{McDonald2020}. However, real applied systems, especially the multi-body systems, must consider the influence of environmental noise. Many studies have shown that as long as a little decoherent noise or detection noise makes the sensor lose its quantum advantage \cite{Escher2011, Kacprowicz2010, Datta2011, Thomas2011, Demkowicz2012}. For example, in \cite{Escher2011}, the relationship between the estimation error $\delta x=|\Delta x-x|$ ($x$ is the actual value; $\Delta x$ is the estimator of $x$) and the number of quantum resources $N$ is as follows
%\begin{equation}
%\begin{aligned}
%\delta x \geq \frac{1+\sqrt{1+\frac{(1-\eta)}{\eta}N}}{2N},
%\end{aligned}
%\end{equation}
%where $\eta$ quantifies the photon losses (from $\eta=1$, lossless case, to $\eta=0$, complete absorption). It can be seen that as long as $\eta<1$, that is, the detection is imperfect, the estimation error changes from the Heisenberg limit $\delta x\propto \frac{1}{N}$ to the standard quantum limit $\delta x\propto \frac{1}{\sqrt{N}}$.
{\it Gain} has been demonstrated to be a necessary ingredient to have an enhanced signal power in NH sensing \cite{Lau2018}, whereas  too much gain may result in an unstable sensing dynamics. Hence, it is of vital importance to understand the impact of loss and gain on the sensitivity and stability of NH sensing. The loss and gain generally induce two effects: one is the diffusion noise that may be further amplified during the sensing and then severely degrade the sensitivity; the other one is the dissipative drift that may lead to system instability.


In this work,  we find conditions to achieve the revival of the best sensitivity for noisy NH quantum sensing in both the perturbation regime and the case beyond linear response. Specifically, we discover  that it is the coupling structure of the loss rather than the gain that plays a pivotal role in obtaining exponential sensitivity revival in our setting.  Counter-intuitively, it is demonstrated that by only tuning the coupling coefficients of the loss properly, e.g., by adding a loss coupling which is exponentially large (or small) in terms of the amplification factor, \textit{an exponential signal-to-noise ratio} (SNR)  \textit{can be surprisingly regained when the sensing dynamics is stable}. We then further point out that the gain is vital to ensure the stability of the NH sensing dynamics by making the loss and gain balanced. Actually, we demonstrate that for unbalanced noise, there is  a striking tradeoff between the \textit{enhancement of the sensitivity and the exponential decrement of  the robust stability}. The feasibility of tuning the loss and gain couplings is supported by the current capability in optics of engineering the loss and gain of photons in a controlled manner, especially by spatially manipulating loss and gain structures \cite{Feng2014,Liuyl2017,Ren2022}.
%Thus, it is natural to investigate whether the exponentially enhanced NH sensing is still attainable if loss and gain are involved. is Thus, additional design of the gain and loss structures is required to construct a highly sensitive sensor by canceling out the total effect of the noise power owing to the gain. It is demonstrated that if, an exponential enhancement of the sensitivity can be attained with appropriately designed gain and loss coupling structures. This  property persists in the  regime beyond linear response, i.e., with properly engineered gain and loss structures, the best sensitivity obtained in the absence of gain and loss noise can be regained.we demonstrate how to engineer the gain and loss coupling structures to  exponentially improve the Hamiltonian parameter estimation .
%The paper is organized as follows. In Section II, we describe the generic NH lattice setup with noise. In Section III, we analysed the stability of the system. In section IV, we analysed the measurement precision of stable non-Hermitian sensors. Section V concludes the paper.



%\begin{figure}[htbp]
%\centering
%\subfigure[]{
%\includegraphics[scale=0.52]{1-1}
%}
%\quad
%\subfigure[]{
%\includegraphics[scale=0.52]{2-1}
%}
%\caption{A general multimode noisy NH setup.  (a) The setup consists of a 1D chain of $N$  bosonic modes. The parameter to be detected is $\epsilon$, which represents a small change in the resonance frequency of the last site. To detect $\epsilon$, a coherent drive $\beta$ accompanied by  quantum noise $\hat{B}^{\textsf{in}}$ is injected into the chain at mode 1 through an input-output waveguide with coupling rate $\kappa$. The reflected field  is measured by homodyne detection.  The modes are coupled via nearest neighbour hopping $w$ and coherent two-photon drive $\Delta$. To account for the noise, couplings between the modes and the loss/gain baths (blue solid/red dashed) are included.  The coupling rate between the $i$th mode and the $j$th loss (gain) bath is described by $Z_{ij}$  ($Y_{ij}$). (b) The nonreciprocal amplification between modes can be described by  two $N$-site non-Hermitian Hatano-Nelson chains \cite{Hatano1997} with effective hopping amplitude $J$ and amplification factor $A$. For the top (bottom) \textsf{X} (\textsf{P}) chain, hopping to the right is a factor of $e^{2A}$ larger (smaller) than hopping to the left. The last modes of the two chains are coupled due to the presence of small tunneling with amplitude $\epsilon$, allowing the signal to be transmitted between the two chains.}
%\end{figure}


\begin{figure}[htbp]
\centering
\includegraphics[scale=0.57]{fig1}
\vspace{-0.5em}
\caption{A general multimode noisy NH setup.  (a) The setup consists of a 1D chain of $N$  bosonic modes. The parameter to be detected is $\epsilon$, which represents a small change in the resonance frequency of the last site. To detect $\epsilon$, a coherent drive $\beta$ accompanied by  quantum noise $\hat{B}^{\textsf{in}}$ is injected into the chain at mode 1 through an input-output waveguide with coupling rate $\kappa$. The reflected field  is measured by homodyne detection.  The modes are coupled via nearest neighbour hopping $w$ and coherent two-photon drive $\Delta$. To account for the noise, couplings between the modes and the loss/gain baths (blue solid/red dashed) are included.  The coupling rate between the $i$th mode and the $j$th loss (gain) bath is described by $Z_{ij}$  ($Y_{ij}$). (b) The nonreciprocal amplification between modes can be described by  two $N$-site non-Hermitian Hatano-Nelson chains \cite{Hatano1997,Hatano1996} with effective hopping amplitude $J$ and amplification factor $A$. For the top (bottom) \textrm{X} (\textrm{P}) chain, hopping to the right is a factor of $e^{2A}$ larger (smaller) than hopping to the left. The last modes of the two chains are coupled due to the presence of small tunneling with amplitude $\epsilon$, allowing the signal to be transmitted between the two chains.}
\end{figure}







\textit{Setup of noisy NH sensors.---} A generic multimode noisy NH  setup is illustrated in Fig.~1a.  Consider  an odd $N$-site 1-dimensional  (1D) cavity array, and let  $\hat{a}_i$ denote the mode annihilation operator on the $i$th site. Our aim is to sense a small perturbation $\epsilon$ of a perturbation Hamiltonian $\epsilon \hat{V}$,  where $\hat{V}$ is a system operator. In Fig.~1a and Fig.~1b,   $\hat{V}=\hat{a}_N^\dag\hat{a}_N$,  and thus the aim is to estimate a small change $\epsilon$ in the resonance of the last site. A general measurement strategy is to couple mode 1 to an input-output waveguide with rate $\kappa$, and then inject a coherent drive with amplitude $\beta$ at the resonant frequency of the mode. The reflected signal is measured by a homodyne detection \cite{Moiseyev2011} to infer $\epsilon$ \cite{Lau2018,Bao2021,Baoli2021,McDonald2020}.
In the rotating frame set by the drive frequency, the system Hamiltonian reads $$\hat{H}_S=\sum^{N-1}_{n=1}(i w \hat{a}^\dagger_{n+1}\hat{a}_n+i \Delta \hat{a}^\dagger_{n+1}\hat{a}^\dagger_n+\textit{h.c.}),$$ where  $\omega$ depicts the hopping of neighbor modes and $\Delta$ depicts the  coherent two-photon drive.  We assume $w>\Delta>0$.
Up to now this is the ideal model considered in \cite{McDonald2020}.

To fully account for the  noise effect, we couple the modes to $N_Z$ loss  and $N_Y$ gain baths, with the coupling rates characterized by the matrices $Z$ and $Y$, respectively.  Without loss of generality, we assume that $Z$ and $Y$ are both real.
Using the standard input-output theory \cite{Clerk2010}, the total effective Hamiltonian reads (see \cite{SM} for details)
\begin{equation*}\label{fullHamiltonian}
\begin{aligned}
\hat{H}[\epsilon]=\hat{H}_S+\epsilon\hat{V}+\hat{H}_\kappa+\hat{H}_{G}+\hat{H}_{L}-i\sqrt{\kappa}(\hat{a}_1^\dagger \beta-h.c.),
\end{aligned}
\end{equation*}
where $\hat{H}_{\kappa}$ describes the damping of mode 1 due to the coupling with the  waveguide,  while $\hat{H}_{G}$ and $\hat{H}_L$ describe the damping owing to the coupling with the gain  and loss baths, respectively.   The Heisenberg-Langevin (HL) equations read  \cite{SM}
\begin{equation*}\label{anheisenberg}
\begin{aligned}
\dot{\hat{a}}_n&=w \hat{a}_{n-1}-w\hat{a}_{n+1}+\Delta\hat{a}^\dagger_{n+1}+\Delta\hat{a}^\dagger_{n-1}-i\epsilon[\hat{a}_n,\hat{V}]\\
&+\sum_{j=1}^{N_{Y}}\sum_{i=1}^{N}Y_{n,j}Y_{i,j}\hat{a}_i-\sum_{j=1}^{N_{Z}}\sum_{i=1}^{N}Z_{n,j}Z_{i,j}\hat{a}_i-\frac{\kappa}{2}\hat{a}_1\delta_{n,1}\\
&-\sqrt{\kappa}(\hat{B}^{\textsf{in}}+\beta)\delta_{n,1}-\sqrt{2}(\sum_{j=1}^{N_{Y}}Y_{n,j}\hat{C}_{j}^{\textsf{in}\dagger}+\sum_{j=1}^{N_{Z}}Z_{n,j}\hat{D}_{j}^{\textsf{in}}).
\end{aligned}
\end{equation*}
Here, $\hat{B}^{\textsf{in}}$ denotes the quantum noise entering from the waveguide, and $\hat{C}_{j}^{\textsf{in}}$ ($\hat{D}_{j}^{\textsf{in}}$) are quantum noises arising from the gain (loss) process of the baths. To ensure  Markovian dynamics, $Q$ is assumed to be quantum Gaussian white noise,
where $Q\in \{ \hat{B}^{\textsf{in}}, \hat{C}^{\textsf{in}}_j, \hat{D}^{\textsf{in}}_j \}$ \cite{Lau2018,Gardiner2000}, and there are no correlations between different noise operators. Hereafter, we focus on the vacuum noise.


To see clearly how the signal is amplified, we turn to the picture of canonical quadratures $\hat{x}_n$ and $\hat{p}_n$ related with $\hat{a}_n$ via $\hat{a}_n=(\hat{x}_n+i\hat{p}_n)/\sqrt{2}$. Define quadrature vectors $\hat{\mathbf{X}}=(\hat{x}_1,\hat{x}_2,\ldots,\hat{x}_N)^\top$ and $\hat{\mathbf{P}}=(\hat{p}_1,\hat{p}_2,\ldots,\hat{p}_N)^\top$. Then the HL equations can be described in terms of the quadratures \cite{SM}
\begin{equation}\label{main}
\begin{aligned}
 \begin{pmatrix}
 \dot{ \hat{\mathbf{X}} } \\
  \dot{\hat{\mathbf{P}}}  \\
 \end{pmatrix}=&\begin{pmatrix}
  h^\mathbb{X}+YY^\top-ZZ^\top &0 \\
  0 & h^\mathbb{P}+YY^\top-ZZ^\top  \\
 \end{pmatrix}\begin{pmatrix}
  \hat{\mathbf{X}}  \\
  \hat{\mathbf{P}}  \\
 \end{pmatrix}\\
 &-i\epsilon\begin{pmatrix}
 [\hat{\mathbf{X}},\hat{V}]\\
 [\hat{\mathbf{P}},\hat{V}]
 \end{pmatrix}-\vec{\beta}-\hat{\Omega}^{\textsf{in}}.
\end{aligned}
\end{equation}
Here, $h^\mathbb{X}$ and $h^\mathbb{P}$ represent the ideal noise-free dynamical matrices of the quadratures $\hat{\mathbf{X}}$ and $\hat{\mathbf{P}}$, respectively, and read
\begin{equation*}
\begin{aligned}
h^\mathbb{X}&=-\frac{\kappa}{2}|1\rangle\langle1|+\sum^{N-1}_{n=1}\Big(Je^{A}|n+1\rangle\langle n|-Je^{-A}|n\rangle\langle n+1|\Big),\\
h^\mathbb{P}&=-\frac{\kappa}{2}|1\rangle\langle1|+\sum^{N-1}_{n=1}\Big(Je^{-A}|n+1\rangle\langle n|-Je^{A}|n\rangle\langle n+1|\Big),
\end{aligned}
\end{equation*}
where $J= \sqrt{w^2-\Delta^2}$ denotes the hopping amplitude and the amplification factor
 $A$ is defined via $e^{2A}= \frac{w+\Delta}{w-\Delta}$. Due to the dynamical matrices $h^\mathbb{X}$ and $h^\mathbb{P}$,  for the  top \textrm{X} (bottom \textrm{P}) chain in Fig.~1b, hopping to the right is a factor of $e^{2A}$ larger (smaller) than hopping to the left.  The commutators with the perturbation $\hat{V}$ are defined in an element-wise way, e.g.,  $[\hat{\mathbf{X}},\hat{V}]=([\hat{x}_1, \hat{V}],\cdots,[\hat{x}_N, \hat{V}])^{\top}$.
%The commutator with the perturbation
%$\begin{pmatrix}
% [\hat{\mathbf{X}},\hat{V}]\\
% [\hat{\mathbf{P}},\hat{V}]
% \end{pmatrix}^\top =([\hat{x}_1, \hat{V}],\cdots,[\hat{x}_N, \hat{V}],[\hat{p}_1, \hat{V}],\cdots,[\hat{p}_N, \hat{V}])$.
The coherent input vector
$\vec{\beta}=(\sqrt{2\kappa}\beta,0,0,\ldots,0)^\top$, and  $\hat{\Omega}^{\textsf{in}}$ denote the quantum noise vector \cite{SM}.


\textit{SNR per photon.---} We now introduce the figure of merit that evaluates the performance of sensing.


From the input-output theory, the output field $\hat{B}^{\textsf{out}}(t) $  reads $\hat{B}^{\textsf{out}}(t)=\beta+\hat{B}^{\textsf{in}}(t)+\sqrt{\kappa}\hat{a}_1(t).$
To estimate a small $\epsilon$, we should integrate the output field over a long time period $[0, \tau]$. The corresponding temporal mode is defined by $\hat{\mathcal{B}}=\frac{1}{\sqrt{\tau}}\int^\tau_0\hat{B}^{\textsf{out}}(t)dt$, which is a canonical bosonic annihilation operator.
For the perturbation $\epsilon\hat{V}=\epsilon\hat{a}^\dag_N\hat{a}_N$,  if  the drive $|\beta|\gg 1$, then the optimal observable is
$\hat{\mathcal{M}}=\frac{1}{\sqrt{2}i}\Big{(}\hat{\mathcal{B}}-\hat{\mathcal{B}}^\dagger\Big{)},$
which is $\hat{p}$-quadrature of the temporal output field $\hat{\mathcal{B}}$ \cite{McDonald2020}.

Let us first consider the case when $\epsilon$ is infinitesimal. Define the signal power in terms of the optimal observable $\hat{\mathcal{M}}$ as
$\mathcal{S}(\epsilon)=|\langle\hat{\mathcal{M}}\rangle_\epsilon-\langle\hat{\mathcal{M}}\rangle_0|^2,$
and the noise power as $
\mathcal{N}(\epsilon)=\langle\hat{\mathcal{M}}^2\rangle_{\epsilon}-\langle\hat{\mathcal{M}}\rangle_{\epsilon}^2. $
Here, the average $\langle\cdot\rangle_\epsilon$ represents the mean with  the steady state whose dynamics is governed by $\hat{H}[\epsilon]$. The  SNR is defined by $$\textrm{SNR}(\epsilon)=\frac{\mathcal{S}(\epsilon)}{\mathcal{N}(0)}.$$ Note that since $\epsilon$ is infinitesimal, we can only consider the zeroth order of $\epsilon$ for the noise power.
Since the dominant term of the SNR with respect to $\epsilon$ is the same as that of the quantum Fisher information when $|\beta|\gg1$ \cite{Bao2021,McDonald2020,Baoli2021,Lau2018}, below we use the SNR to evaluate the performance of NH sensors.

To make a fair comparison,  the resources used in the measurement should be constrained. Following \cite{Baoli2021,McDonald2020} we take the SNR per photon denoted by $$\overline{\textrm{SNR}}(\epsilon)=\frac{\textrm{SNR}(\epsilon)}{\bar{n}_{\textsf{tot}}(0)}$$ as the figure of merit, where the total average photon number is $\bar{n}_{\textsf{tot}}(0)= \sum_{n} \langle\hat{a}_n^\dagger\hat{a}_n\rangle_0\simeq \sum_n \langle\hat{a}_n^\dagger\rangle_0\langle\hat{a}_n\rangle_0$ in the large-drive limit. Following the same reasoning as that of the noise power, only the zeroth order of $\bar{n}_{\textsf{tot} }$  in $\epsilon$ is concerned.
%According to Eqs. \eqref{bout}, \eqref{bputfield}, \eqref{mtau} and \eqref{signaldefinition}, the dominate term of signal power is
%\begin{equation}\label{signal1}
%\begin{aligned}
%\mathcal{S}_\tau(\epsilon)=2\kappa\tau\Big{|}\textrm{Re}[e^{-i\phi}\delta\langle\hat{a}_1\rangle]\Big{|}^2,
%\end{aligned}
%\end{equation}
%where
%\begin{equation}
%\begin{aligned}
%\delta\langle\hat{a}_1\rangle\equiv\epsilon \mathop{\textrm{lim}}\limits_{\epsilon\rightarrow0} \frac{\langle\hat{a}_1\rangle_\epsilon-\langle\hat{a}_1\rangle_0}{\epsilon}
%\end{aligned}
%\end{equation}
%depends on the specific form of the perturbation Hamiltonian $\hat{V}$.

\textit{$\overline{\textrm{SNR}}$ of noisy NH sensors.---} Take the perturbation $\hat{V}$ in Eq.~(\ref{main}) as $\hat{V}=\hat{a}_N^\dagger\hat{a}_N$. It can be derived that the signal power $\mathcal{S}$, noise power $\mathcal{N}$, and the total average photon number $\bar{n}_{\textsf{tot}}$ are described by (see \cite{SM} for detailed derivation)
\begin{equation}\label{SNn}
\begin{aligned}
\mathcal{S}(\epsilon)
%=&2\kappa\tau\epsilon^2\kappa \beta^2\Big{|}(h^\mathbb{X})^{-1}_{N,1}\Big{|}^2\Big{|}(h^\mathbb{P})^{-1}_{1, N}\Big{|}^2\\
=&2\epsilon^2\kappa^2\beta^2\tau\cdot|{Q_{N,1}^\mathbb{X}}|^2\cdot|{Q_{1,N}^\mathbb{P}}|^2,\\
\mathcal{N}(0)=&\frac{1}{2}(1+\kappa Q_{1,1}^\mathbb{P})^2+\kappa\big{[} Q^\mathbb{P}(YY^\top+ZZ^\top){Q^\mathbb{P}}^{\top}\big{]}_{1,1} ,\\
\bar{n}_{\textsf{tot}}(0)=&\kappa\beta^2\big{[}{Q^\mathbb{X}}^{\top}Q^\mathbb{X}\big{]}_{1,1},
\end{aligned}
\end{equation}
with information matrices $Q^{\mathbb{X}}=(h^{\mathbb{X}}+YY^\top-ZZ^\top)^{-1}$  and $Q^{\mathbb{P}}=(h^{\mathbb{P}}+YY^\top-ZZ^\top)^{-1}$.
%
%Here $$h=-\frac{\kappa}{2}|1\rangle\langle1|+\sum^{N-1}_{n=1}\Big(-J|n\rangle\langle n+1|+J|n+1\rangle\langle n|\Big)$$
%and $h^{-1}_{1,1}=h^{-1}_{1,N}=h^{-1}_{N,1}=-\frac{2}{\kappa}$.

When there is no loss and gain ($Y=Z=0$),  it was demonstrated in \cite{McDonald2020} that $\overline{\textrm{SNR}}(\epsilon)\propto e^{2A(N-1)}$, i.e.,  an exponentially enhanced sensitivity can be obtained. The basic idea is illustrated in Fig.~1b.  To sense  $\epsilon$,  a real drive is injected at  site 1 to excite the \textrm{X} chain, then the wavepacket propagates  rightwards
with amplitude amplifying. When it  reaches the last site, the amplitude grows a factor of $e^{A(N-1)}$. Then at site $N$, owing to the perturbation, the wavepacket can scatter off the boundary and change to  $\hat{p}_N$ quadrature. It then propagates backwards to site 1 amplifying the signal. If the $\hat{p}$-quadrature of the output field is measured, then a net amplification factor $e^{2A(N-1)}$ is obtained.

\textit{Pivotal role of loss couplings.---}Now we investigate how the loss and gain impact the performance of NH sensors.

From Eq.~(\ref{SNn}), it is clear to see that $\mathcal{S}$, $\mathcal{N}$ and $\bar{n}_{\textsf{tot}}(0)$ all depend on the loss ($Z$) and gain ($Y$). Under the nonreciprocal dynamics governed by  $(Q^\mathbb{X})^{-1}$ and $(Q^\mathbb{P})^{-1}$, not only is the signal amplified, but also the loss and gain are significantly amplified. In general, the noise power $\mathcal{N}\propto e^{2A(N-1)}$, causing the vanishing of the (ideally) exponential sensitivity. Moreover, from Eq.~(\ref{main}), the net noise matrix $YY^\top-ZZ^\top$  may result in the sensing dynamics becoming unstable, implying that at least one of the eigenvalues of the noisy NH dynamical matrices  $h^\mathbb{X}+YY^\top-ZZ^\top$ and $h^\mathbb{P}+YY^\top-ZZ^\top$ sits in the right half plane \cite{Franklin2019}.

To address the sensitivity revival problem in the presence of loss and gain, we find that the loss structure $Z$ plays a pivotal role. First consider the case
where there is only loss, i.e., $Y=0$. It can be proven that under the condition
$$\text{(\textbf{C1})}: Z=h^{\mathbb{P}}\cdot C,~\text{and} ~C_{1,j}=0,~\text{for}~ j=1,2,\ldots,N_{Z}~~~~$$
the signal power $\mathcal{S}$, noise power $\mathcal{N}$ and the total average photon number $\bar{n}_{\textsf{tot}}(0)$ are the same as those of the ideal noise-free case \cite{SM}.  Condition (\textbf{C1}) implies that each column of $Z$ should be represented as a linear combination of the second column through the last column of the dynamical matrix $h^{\mathbb{P}}$. It is remarkable as the exponential sensitivity  can be regained by only tuning the coupling rates of the loss when the sensing dynamics is stable. For example, we can increase or even add some loss coupling rates to be exponentially large (or small) to make $\overline{\textrm{SNR}} \propto e^{2A(N-1)}.$  This is quite different from the general belief that the loss will induce a vanishing quantum advantage in high precision sensing.

If the sensing dynamics is unstable, we can further tune $Y$ such that the balanced condition
 $$\text{(\textbf{C2})}:~YY^\top=ZZ^\top~~~~~~~~~~~~~~~~~~~~~~~~~~~~~~~~~~~~~~~~~~~~~~~~~~~~~~~~$$
holds.  Note that there are many freedoms to choose $Y$ to make  (\textbf{C2}) hold. It is verified that under (\textbf{C1}) and (\textbf{C2}), an exponential enhancement of noisy NH sensing  can be revived, that is $\overline{\textrm{SNR}} \propto e^{2A(N-1)}$  \cite{SM}. The feasibility of tuning $Z$ and $Y$  is well supported by the current capability of spatially engineering the loss and gain structures in optics in a controlled manner \cite{Feng2014,Liuyl2017,Ren2022}. Here we provide an illustrative example.


{\bf Example}: Consider a 3-site NH sensor. If there is no loss and gain, the ideal $\overline{\textrm{SNR}}$ is  $ e^{4A}$ which grows exponentially with the amplification factor $A$. The ideal $\overline{\textrm{SNR}}$ is depicted in Fig.~2a by the black line.  Let us consider two loss structures, that is, $Z_1=\alpha\left(
 \begin{array}{ccc}
 -e^A & -e^A & 0 \\
 0 & 1 & 0 \\
 e^{-A} & 0 & 0 \\
 \end{array}
 \right)$ and $Z_2=\alpha\left(
 \begin{array}{ccc}
 -e^A & 0 & 0 \\
 0 & 1 & 0 \\
 e^{-A} & e^{-A} & 0 \\
 \end{array}
 \right)$, where $\alpha>0$. It is clear that $Z_1$ and $Z_2$ do not obey  \text{(\textbf{C1})}. With parameters $\alpha=0.5$, $\kappa=10$, $\omega=10^5$ and $J=\omega\frac{2e^A}{e^{2A}+1}$,   the
 $\overline{\textrm{SNR}}$ under $Z_1$ (dashed blue) and $Z_2$ (dotted green) are depicted in Fig.~2a, respectively. Here, we only care about the amplification factor $A$ that makes the dynamics stable. Note that under $Z_1$ the  $\overline{\textrm{SNR}}$ approaches a constant as $A$ increases, while under $Z_2$ the $\overline{\textrm{SNR}}$ increases as $A$, but is still much smaller than the ideal case. Now we add an exponentially small (large) coupling $e^{-A}$ ($-e^{A}$) to $Z_1$ ($Z_2$), such that the tuned loss structure to be $Z=\alpha\left(
 \begin{array}{ccc}
 -e^A & -e^A & 0 \\
 0 & 1 & 0 \\
 e^{-A} & e^{-A} & 0 \\
 \end{array}
 \right)$, which meets   (\textbf{C1}).  The
 $\overline{\textrm{SNR}}$ under $Z$ (red circle) is illustrated in Fig.~2a, where we also only consider $\textit{A}\textrm{s}$ that make the sensing stable. It is clear that the ideal exponential sensitivity $e^{4A}$ is regained in the stable region and the best $\overline{\textrm{SNR}}$ obtained under $Z$ is much better than those attained under $Z_1$ and $Z_2$. To further improve the sensitivity and increase the range of $A$ that guarantees stability, we can tune $Y$ to meet  ($\mathbf{C2}$). Then the ideal exponential sensitivity can be fully regained, and the range of $A$ can, in principle, be arbitrarily large, which of course depends on the parameters of the real setup.  In Fig.~2b, we illustrate the $\overline{\textrm{SNR}}$ of $Z_1$ with different $\alpha$.  If $\alpha=0$, it is the ideal case (black). After tuning $Z_1$ to be $Z$ to meet  ($\mathbf{C1}$),  we regain the ideal sensitivity with different ranges of $A$ for different $\alpha$. We indicate the corresponding best sensitivity that can be attained by colored dots on the ideal black line for different $\alpha$'s.  It is clear that as $\alpha$ increases, the best sensitivity that can be revived decreases. However, if we can further tune $Y$ to make ($\mathbf{C2}$) hold, the $\overline{\textrm{SNR}}$ becomes the same as the ideal $\alpha=0$ case.

\begin{figure}[htbp]
\centering
\includegraphics[scale=0.6]{fig2}
\caption{The performance $\log(\frac{\overline{\textrm{SNR}}}{\tau\epsilon^2})$ versus $A$ under different loss structures. The range of $A$ is identified to guarantee a stable dynamics.  Here, $\kappa=10,~\omega=10^5$ and $J=\omega\frac{2e^A}{e^{2A}+1}$. (a) The dashed blue corresponds to the loss structure $Z_1$, and the  dotted green line depicts the loss $Z_2$, the red circle demonstrates the loss $Z$  which satisfies  ($\mathbf{C1}$), while the black line corresponds to both the ideal case and the tuned case where both  ($\mathbf{C1}$) and ($\mathbf{C2}$) are met. Here $\alpha=0.5$. (b) The performance $\log(\frac{\overline{\textrm{SNR}}}{\tau\epsilon^2})$ versus $A$ under  losses $Z_1$ and $Z$, for different $\alpha$.  Black solid: $\alpha=0$, blue dashed: $\alpha=0.5$, magenta dot-dashed: $\alpha=1$, green dotted: $\alpha=2$. The best sensitivity that can be revived under loss $Z$ for different values of $\alpha$ are shown by the colored dots on the ideal black solid line.}
\end{figure}


%\begin{figure}[htbp]
%\centering
%\subfigure[]{
%\includegraphics[scale=0.6]{3}
%}
%\quad
%\subfigure[]{
%\includegraphics[scale=0.6]{4}
%}
%\caption{The performance $\log(\frac{\overline{\textrm{SNR}}}{\tau\epsilon^2})$ versus $A$ under different loss structures. The range of $A$ is identified to guarantee a stable dynamics.  Here, $\kappa=10,~\omega=10^5$ and $J=\omega\frac{2e^A}{e^{2A}+1}$. (a) The dotted blue corresponds to the loss structure $Z_1$, and the  dashed green line depicts the loss $Z_2$, the red circle demonstrates the loss $Z$  which satisfies  ($\mathbf{C1}$), while the black line corresponds to both the ideal case and the tuned case where both  ($\mathbf{C1}$) and ($\mathbf{C2}$) are met. Here $\alpha=0.5$. (b) The performance $\log(\frac{\overline{\textrm{SNR}}}{\tau\epsilon^2})$ versus $A$ under  losses $Z_1$ and $Z$, for different $\alpha$.  Black solid: $\alpha=0$, blue dashed: $\alpha=0.5$, magenta pecked: $\alpha=1$, green dotted: $\alpha=2$. The best sensitivity that can be revived under loss $Z$ for different values of $\alpha$ are shown by the colored dots on the ideal black solid line.}
%\end{figure}


%For balanced gain and loss, it can be verified that the signal power and the total average photon number for noisy NH sensors are the same as NH sensors without noise. However, from the second term of the noise power  $\mathcal{N}_\tau$ in Eq.~\eqref{SNn}, it is clear that the gain noise $YY^\top$ still has great impact on the noise power. Moreover, since $(h^\mathbb{P})^{-1}_{i,j}=h^{-1}_{i,j}e^{A(j-i)}$, generally the noise power $\mathcal{N}_\tau$ is in the order of $e^{2A(N-1)}$. Hence, in the general case, the exponentially enhanced sensitivity of noisy NH sensing may vanish. Therefore, to make the SNR per photon grow exponentially in terms of the product of the size $N$ and the amplification factor $A$, the key is to properly design the gain coupling matrix $Y$.

%From the definition of $h^\mathbb{P}$, it can be calculated that
%\begin{equation}
%[(h^\mathbb{P})^{-1}YY^\top(h^\mathbb{P})^{-1\top}]_{1,1}=\sum^{N_Y}_{j=1}(\sum_{i=1}^Nh^{-1}_{1,i}Y_{i,j}e^{A(i-1)})^2,
%\end{equation}
%where
%\begin{equation}
%h^{-1}_{1,i}=\left\{
%\begin{matrix}
%-\frac{2}{\kappa}&~~ i\text{ is odd},\\
%0 &~~ i\text{ is even.}
%\end{matrix}
%\right.
%\end{equation}

%\textit{$\overline{\textrm{SNR}}$ of noisy NH sensor.---} Now we provide a condition ({\bf C1}) that can achieve the sensitivity revival of noisy NH sensing: (i) $Y=h^\mathbb{P}\cdot C$ with $C\in\mathbb{R}^{N\times N_{Y}}$ and $C_{1,j}=0$ for $j=1,2,\ldots,N_{Y}$; (ii) $Z=h^\mathbb{P}\cdot C\cdot U$ with $C\in\mathbb{R}^{N\times N_{Y}}$,  $U\in\mathbb{R}^{N_Y\times N_{Z}}$,  $C_{1,j}=0$ for $j=1,2,\ldots,N_{Y}$, and $UU^{\top}=I_{N_Y\times N_Y}$. Under {\bf C1}, the noise power  $\mathcal{N}_\tau(0)=\frac{1}{2}$, and the effects of all the gain noise on the first mode cancel out and the dominate term of the SNR per photon is
%\begin{equation}
%\overline{\textrm{SNR}}_{\tau,\textsf{D}}(\epsilon)=\frac{\textrm{SNR}_\tau(\epsilon)}{\bar{n}_{\textsf{tot,D}}(0)}= 16\tau \frac{\epsilon^2}{\kappa}e^{2A(N-1)}.
%\end{equation}
%Thus, an exponential enhancement of the SNR per photon has been revived.
%\begin{itemize}
%\item Condition 1:
%$$Y=h^\mathbb{P}\cdot C,$$
%$Y_{i,j}=0,$$
%where $i$ is odd and $j=1,2,3,\ldots,N_Y$,
%\end{itemize}


%Obviously, condition 2 is a special case of condition 1. These two conditions are also in line with physical intuition. The Condition 1 is to cut off the channel of bath noise to mode 1, and the Condition 2 is that all bath noises are passed to mode 1 and cancel each other out.

\textit{Essential role of balanced loss and gain.---} From the above analysis and illustrative example, it is clear that although the loss structure is key to revive exponential sensitivity, just employing it  may result in  unstable sensing. Thus this greatly limits the potential capability of NH sensing. To address this, the gain structure should be further tuned to make the loss and gain balanced. Actually we will demonstrate that balanced loss and gain does play an essential role in ensuring the NH sensing dynamics stable.


%When $\epsilon$ is infinitesimal, it is clear that the stability of the dissipative dynamics of Eq.~(\ref{main})  is equivalent to the stability of both $h^\mathbb{X}+YY^\top-ZZ^\top$ and $h^\mathbb{P}+YY^\top-ZZ^\top$.
%For balanced loss and gain, it has been  shown that the sensitivity can be exponentially enhanced in terms of the product of $A$ and $N$.
We now consider the case where the loss and gain are unbalanced, i.e., $YY^\top\neq ZZ^\top$. We find that  Eq.~(\ref{main}) becomes extremely unstable when the nondiagonal elements of the net noise matix $YY^\top-ZZ^\top$ are not zero.

We illustrate this by the following two most sensitive cases \cite{SM}: (i) If $ YY^\top-ZZ^\top=\gamma|1\rangle\langle N-1|+\gamma| N-1\rangle\langle1|$, to ensure the stability of the dissipative dynamics of Eq.~(\ref{main}),  a necessary condition for $\gamma$ is $|\gamma|<\frac{N+1}{2}Je^{-A(N-2)}$; (ii) If $ YY^\top-ZZ^\top=\gamma|1\rangle\langle N|+\gamma| N\rangle\langle1|$, to ensure the stability, a necessary  condition for $\gamma$ is   $|\gamma|< \kappa e^{-A(N-1)}$.
%\begin{itemize}
%\item If $ YY^\top-ZZ^\top=\gamma|1\rangle\langle N-1|+\gamma| N-1\rangle\langle1|$, to ensure the stability of the dissipative dynamics of Eq.~(\ref{main}),  a necessary (but not sufficient) condition for $\gamma$ is $|\gamma|<\frac{N+1}{2}Je^{-A(N-2)}$;
%\item If $ YY^\top-ZZ^\top=\gamma|1\rangle\langle N|+\gamma| N\rangle\langle1|$, to ensure the stability of the dissipative dynamics, a necessary (but not sufficient) condition for $\gamma$ is   $|\gamma|< \kappa e^{-A(N-1)}.$
%\end{itemize}
Therefore, to make the sensing dynamics stable, the coupling of  the first mode to the last one, and the coupling between the first and penultimate modes should be both exponentially small in terms of the product of the amplification factor $A$ and the size $N$ of the device.
%{\color{blue} In general, when the gain and loss are unbalanced, the strength of the $(i,j)$th ($i<j$) element of the net noise matrix $YY^\top-ZZ^\top$  should be exponentially small in terms of the product of $A$ and $(j-i)$. }
Note that to obtain a higher level of sensitivity, a natural way is to increase the amplification factor $A$ and/or the size of $N$. This implies that  there is a remarkable tradeoff between the enhancement of the sensitivity and the decrement of the robust stability of  the NH sensor if the  loss and gain are unbalanced.
%{\color{blue} Note that here the impacts of the elements of the net noise matrix $YY^\top-ZZ^\top$ are considered separately.  In Appendix C, their impacts are considered together in the case where $N=3$. It is found that to ensure the dynamics stable, the scaling of each parameter is the same as that being considered separately.}
Thus, in order to construct a stable and highly sensitive NH sensor in a noisy environment, the loss and gain structures should be finely tuned to the balanced working condition, i.e., $YY^\top=ZZ^\top.$
%To further investigate how to construct a highly sensitive NH sensor in presence of gain and loss,  the gain and loss structures should be finely tuned to make
%\begin{description}
%  \item[C0] $ YY^\top=ZZ^\top.$
%\end{description}


 \textit{Regime beyond linear response.---} We now consider the case where the parameter to be sensed, $\epsilon_0$, is not infinitesimally small. Thus, not only the linear response of $\epsilon_0$, but all orders in $\epsilon_0$ of the output field should be calculated.

 As in  \cite{Baoli2021,McDonald2020}, we focus on the most interesting case where $\epsilon_0/\kappa\ll1$. Take $$\overline{\textrm{SNR}}(\epsilon_0)=\frac{\mathcal{S}(\epsilon_0)}{\mathcal{N}\cdot\bar{n}_{\textsf{tot}}}$$
as the figure of merit, which quantifies the distinguishability between Gaussian homodyne current distributions under parameter $\epsilon=0$ and $\epsilon=\epsilon_0$. Here,  the noise power
$\mathcal{N}=[\mathcal{N}(0)+\mathcal{N}(\epsilon_0)]/2,$
and the total average number of photons
$\bar{n}_{\textsf{tot}}=[\bar{n}_{\textsf{tot}}(0)+\bar{n}_{\textsf{tot}}(\epsilon_0)]/2.$

%With the perturbation Hamiltonian $\hat{V}=\hat{a}_N^\dagger\hat{a}_N$ and the optimal measurement phase  $\phi=\frac{\pi}{2}$, the corresponding signal power, noise power and the total average photon number are (see Appendix E for detailed derivation)
% \begin{equation}\label{SNnBeyond}
%\begin{aligned}
%\mathcal{S}_\tau(\epsilon_0)=& 2\tau\kappa^2\beta^2\bigg(\mathbb{H}[\epsilon_0]^{-1}_{N+1,1}-\mathbb{H}[0]^{-1}_{N+1,1}\bigg)^2,\\
%%=&2\tau\kappa^2\beta^2\frac{\epsilon_0^2}{(\frac{\kappa^2}{4}+\epsilon_0^2)^2}e^{4A(N-1)},\\
%\mathcal{N}_\tau=&\frac{1}{4}\Bigg\{\kappa^2(\mathbb{H}[\epsilon_0]^{-1}_{N+1,1})^2+(1+\kappa\mathbb{H}[\epsilon_0]^{-1}_{N+1,N+1})^2\\
%&+\!(1\!+\!\kappa(h^{\mathbb{P}})^{-1}_{1,1})^2\!+\!4\kappa\big[(h^{\mathbb{P}})^{-1}YY^\top (h^{\mathbb{P}})^{-1\top}\big]_{1,1}\\
%&+\!4\kappa\!\bigg[\mathbb{H}[\epsilon_0]^{-\!1}\!\!\begin{pmatrix}
%                                                                                                                                                                     YY^\top & \! 0 \\
%                                                                                                                                                                    \! 0 & \! YY^\top
%                                                                                                                                                                   \end{pmatrix}\!\!(\mathbb{H}[\epsilon_0]^{-\!1}\!)^\top\!\bigg]_{\!N\!+\!1,N\!+\!1}\Bigg\},\\
%\bar{n}_{\textsf{tot}}=&\frac{1}{2}\kappa\beta^2\!\sum_{n=1}^{N}\!(\mathbb{H}[0]^{-1}_{n,1}\!)^2\!+\!(\mathbb{H}[\epsilon_0]^{-1}_{n,1})^2\!+\!(\mathbb{H}[\epsilon_0]^{-1}_{N+n,1})^2.
%\end{aligned}
%\end{equation}
With the perturbation Hamiltonian $\epsilon\hat{V}=\epsilon\hat{a}_N^\dagger\hat{a}_N$ and the optimal measurement phase  $\phi=\frac{\pi}{2}$, we can obtain the corresponding signal power, noise power, and the total average photon number \cite{SM}.
Similar to the case in the linear response, we verify \cite{SM} that under the balanced noise condition (\textbf{C2}) and the following two conditions,
$$\text{(\textbf{C3})}:  Z=h^\mathbb{P}\cdot C,~ \text{and}~  C_{1,j}=C_{N,j}=0,~ j=1,2,\ldots,N_{Z},$$
$$\text{(\textbf{C4})}: [(h^\mathbb{X})^{-1}\cdot Z]_{N,j}=0,~ j=1,2,\ldots,N_{Z},~~~~~~~~~~~~~~~~$$
we can revive and achieve the best sensitivity as that in \cite{McDonald2020} where there is no noise. Note that in the regime beyond linear response, to revive the sensitivity,
 (\textbf{C3}) and \text{(\textbf{C4})} are much stricter than (\textbf{C1}) in the linear response for constraining the  loss structures. Condition \text{(\textbf{C3})} means that each column of $Z$ should be represented as a linear combination of the second column through the $(N-1)$-th column of the dynamical matrix $h^{\mathbb{P}}$, while  \text{(\textbf{C4})} implies that the columns of $Z$ should be orthogonal to the $N$-th row of the ideal information matrix  $(h^\mathbb{X})^{-1}$.
% the signal power and the total average photon number do not change when the gain and loss are balanced. Thus, to regain the best sensitivity as obtained in the NH sensing without noise, a useful strategy is to design the gain coupling $Y$ to cancel its effect out on the first mode.

%We find the following condition ({\bf C2}) that can achieve the sensitivity revival of noisy NH sensing beyond linear response regime: (i)
%It can be verified that the contribution of the gain in the noise power is 0, if $Y$ and $Z$ are designed as following:
%\begin{description}
%  \item[C2(Y)] $Y=h^\mathbb{P}\cdot C$ with $C\in\mathbb{R}^{N\times N_{Y}}$, $C_{1,j}=C_{N,j}=0$ and
%$[(h^\mathbb{X})^{-1}h^\mathbb{P}C]_{N,j}=0$ for $j=1,2,\ldots,N_{Y}$.
%\end{description}
%\begin{description}
%  \item[C2(Z)] $Z=h^\mathbb{P}\cdot C\cdot U$ with $C\in\mathbb{R}^{N\times N_{Y}}$, $U\in\mathbb{R}^{N_Y\times N_{Z}}$, $C_{1,j}=C_{N,j}=0$,  $[(h^\mathbb{X})^{-1}h^\mathbb{P}C]_{N,j}=0$ for $j=1,2,\ldots,N_{Y}$,  and $UU^{\top}=I_{N_Y\times N_Y}$.
%\end{description}
%$Y=h^\mathbb{P}\cdot C$ with $C\in\mathbb{R}^{N\times N_{Y}}$, $C_{1,j}=C_{N,j}=0$ and
%$[(h^\mathbb{X})^{-1}h^\mathbb{P}C]_{N,j}=0$ for $j=1,2,\ldots,N_{Y}$;
%(ii) $Z=h^\mathbb{P}\cdot C\cdot U$ with $C\in\mathbb{R}^{N\times N_{Y}}$, $U\in\mathbb{R}^{N_Y\times N_{Z}}$, $C_{1,j}=C_{N,j}=0$,  $[(h^\mathbb{X})^{-1}h^\mathbb{P}C]_{N,j}=0$ for $j=1,2,\ldots,N_{Y}$,  and $UU^{\top}=I_{N_Y\times N_Y}$.
%Under {\bf C2}, the contribution of the gain in the noise power is 0 and the noise power is the same as the case without noise. Thus, w

In conclusion, we have investigated the sensitivity revival and the robust stability of noisy NH quantum sensing. We presented a strategy to properly engineer the loss and gain coupling structures to construct a stable NH sensor achieving an exponential sensitivity. We found that  the loss structure is crucial to revive the sensitivity, and that the balanced loss and gain is essential to ensure a stable NH sensor and to improve the capability of NH sensing no matter if the parameter is infinitesimal or in the regime beyond linear response.  When the loss and gain are unbalanced or there are errors in manipulating loss and gain couplings, there is a  striking tradeoff between the enhancement of the sensitivity and the robust stability of the dynamics.



\section*{Acknowledgments}
B.Q. acknowledges the support of the National Natural Science Foundation of China (No.~61833010 and No.~61773370), and D.D. acknowledges the support of the Australian Research Council's Future Fellowship funding scheme under project FT220100656 and the U.S. Office of Naval Research Global under Grant No. N62909-19-1-2129. F.N. is supported in part by: Nippon Telegraph and Telephone Corporation (NTT) Research, the Japan Science and Technology Agency (JST) [via the Quantum Leap Flagship Program (Q-LEAP), and the Moonshot R$\&$D Grant Number JPMJMS2061], the Asian Office of Aerospace Research and Development (AOARD) (via Grant No. FA2386-20-1-4069), and the Foundational Questions Institute Fund (FQXi) via Grant No. FQXi-IAF19-06.


\clearpage
\begin{widetext}

\section*{Appendices}
This supplementary material is organized as follows. We first describe the total Hamiltonian of the sensor and derive the Heisenberg-Langevin equations in Section I. Then we derive the signal-to-noise ratio (SNR) per photon in Section II. The real matrix $Z~(Y)$ describes the coupling between the system and the loss (gain) bath. In Section III we prove that when the gain coupling $Y=0$, the loss coupling $Z$ satisfies condition $\mathbf{(C1)}$, and if the dynamics is stable, the signal power $\mathcal{S}$, noise power $\mathcal{N}$ and the total average photon number $\bar{n}_{\textsf{tot}}(0)$ are the same as those of the ideal noise-free case. The SNR per photon under conditions $(\mathbf{C1})$ and $(\mathbf{C2})$ are given in Section IV. In Section V we derive the necessary conditions to ensure the stability of the dynamics if the loss and gain are unbalanced. In Section VI we calculate the SNR per photon in the regime beyond linear response. The calculation of  the elements of $\mathbb{H}[\epsilon]^{-1}$ and $\mathbb{H}[\epsilon_0]^{-1}$ is shown in Section VII and Section VIII, respectively.




\appendix

\section{The non-Hermitian sensor and the Heisenberg-Langevin equations}
The total Hamiltonian of the sensor is described by
\begin{equation}
\begin{aligned}
\hat{H}_{\text{tot}}=&\hat{H}_S+\hat{H}_{\epsilon}+\hat{H}_{\textrm{input}}+\hat{H}_{\textrm{wave}}+\hat{H}_{\textrm{gain}}+\hat{H}_{\textrm{loss}}+\hat{H}_{S,\textrm{wave}}+\hat{H}_{S,\textrm{gain}}+\hat{H}_{S,\textrm{loss}},
\end{aligned}
\end{equation}
with the perturbation Hamiltonian
\begin{equation}
\begin{aligned}
\hat{H}_{\epsilon}=\epsilon\hat{V},
\end{aligned}
\end{equation}
input Hamiltonian
\begin{equation}
\begin{aligned}
\hat{H}_{\textrm{input}}=-i\sqrt{\kappa}~(\hat{a}_1^\dagger \beta-\hat{a}_1 \beta^\dagger),
\end{aligned}
\end{equation}
waveguide Hamiltonian
\begin{equation}
\begin{aligned}
\hat{H}_{\textrm{wave}}=\int d k ~(k \hat{b}_k^\dagger \hat{b}_k),
\end{aligned}
\end{equation}
the $j$th gain bath Hamiltonian
\begin{equation}
\begin{aligned}
\hat{H}_{\textrm{gain}}=\int d k ~(k \hat{c}_{j,k}^\dagger \hat{c}_{j,k}),
\end{aligned}
\end{equation}
the $j$th loss bath Hamiltonian
\begin{equation}
\begin{aligned}
\hat{H}_{\textrm{loss}}=\int d k ~(k \hat{d}_{j,k}^\dagger \hat{d}_{j,k}),
\end{aligned}
\end{equation}
the interaction Hamiltonian between the chain and the waveguide
\begin{equation}
\begin{aligned}
\hat{H}_{S,\textrm{wave}}=\int dk~ \frac{1}{\sqrt{\pi}}\sqrt{\frac{\kappa}{2}}~(\hat{a}_1 \hat{b}_k^\dagger +\hat{a}_1^\dagger \hat{b}_k),
\end{aligned}
\end{equation}
the interaction Hamiltonian between the chain and the $j$th gain bath
\begin{equation}
\begin{aligned}
\hat{H}_{S,\textrm{gain}}=\sum_{i=1}^N\sum_{j=1}^{N_Y}\int dk~ \frac{1}{\sqrt{\pi}}Y_{i,j}(\hat{a}_i\hat{c}_{j,k}+\hat{a}_i^\dagger\hat{c}_{j,k}^\dagger),
\end{aligned}
\end{equation}
and
the interaction Hamiltonian between the chain and the $j$th loss bath
\begin{equation}
\begin{aligned}
\hat{H}_{S,\textrm{loss}}=\sum_{i=1}^N\sum_{j=1}^{N_Z}\int dk~ \frac{1}{\sqrt{\pi}}Z_{i,j}(\hat{a}_i\hat{d}^\dagger_{j,k}+\hat{a}_i^\dagger\hat{d}_{j,k}).
\end{aligned}
\end{equation}
Here, $\hat{a}_i$ denotes the mode annihilation operator on site $i$, $\hat{b}_k$ is the annihilation operator of the mode with wave number $k$ in the waveguide, $\hat{c}_{j,k}$ is the annihilation operator of the $j$th gain bath with wave number $k$, and $\hat{d}_{j,k}$ is the annihilation operator of the $j$th loss bath mode with wave number $k$. The real matrix $Z$ ($Y$) depicts the coupling between the system and the loss (gain) bath.

The Heisenberg equations of the motion for the cavity modes and the field modes are
\begin{equation}\label{abcdmotion}
\begin{aligned}
\frac{d\hat{a}_n}{dt}=&w \hat{a}_{n-1}+\Delta\hat{a}^\dagger_{n+1}+\Delta\hat{a}^\dagger_{n-1}-w\hat{a}_{n+1}-i\epsilon[\hat{a}_n,\hat{V}]-\sqrt{\kappa}\beta\delta_{n,1}\\
&-i\delta_{n,1}\int dk~ (\frac{1}{\sqrt{\pi}}\sqrt{\frac{\kappa}{2}} \hat{b}_k)-i\sum_{j=1}^{N_Y}\int dk~(\frac{1}{\sqrt{\pi}}Y_{n,j}\hat{c}_{j,k}^\dagger)-i\sum_{j=1}^{N_Z}\int dk~(\frac{1}{\sqrt{\pi}}Z_{n,j}\hat{d}_{j,k}),\\
\frac{d\hat{b}_k}{dt}=&-ik\hat{b}_k-i\frac{1}{\sqrt{\pi}}\sqrt{\frac{\kappa}{2}} \hat{a}_1,\\
\frac{d\hat{c}_{j,k}}{dt}=&-i k\hat{c}_{j,k}-i\sum_{i=1}^N\frac{1}{\sqrt{\pi}}Y_{i,j}\hat{a}_i^\dagger,\\
\frac{d\hat{d}_{j,k}}{dt}=&-i k\hat{d}_{j,k}-i\sum_{i=1}^N\frac{1}{\sqrt{\pi}}Z_{i,j}\hat{a}_i.
\end{aligned}
\end{equation}
The solutions of the last three equations in Eq. \eqref{abcdmotion} are
\begin{equation}\label{bcdmotion}
\begin{aligned}
\hat{b}_k=&e^{-ik(t-t_0)}\hat{b}_k(t_0)-i\frac{1}{\sqrt{\pi}}\sqrt{\frac{\kappa}{2}}\int_{t_0}^t dt' e^{-i k (t-t')}\hat{a}_1(t'),\\
\hat{c}_{j,k}=&e^{-ik(t-t_0)}\hat{c}_{j,k}(t_0)-i\frac{1}{\sqrt{\pi}}\sum_{i=1}^N\int_{t_0}^t dt' e^{-i k (t-t')}Y_{ij}\hat{a}_i^\dagger(t'),\\
\hat{d}_{j,k}=&e^{-ik(t-t_0)}\hat{d}_{j,k}(t_0)-i\frac{1}{\sqrt{\pi}}\sum_{i=1}^N\int_{t_0}^t dt' e^{-i k (t-t')}Z_{ij}\hat{a}_i(t').
\end{aligned}
\end{equation}
Substituting Eq. \eqref{bcdmotion} into the first equation of  Eq.~\eqref{abcdmotion} yields
\begin{equation}
\begin{aligned}
\frac{d\hat{a}_n}{dt}=&w \hat{a}_{n-1}+\Delta\hat{a}^\dagger_{n+1}+\Delta\hat{a}^\dagger_{n-1}-w\hat{a}_{n+1}-i\epsilon[\hat{a}_n,\hat{V}]-\sqrt{\kappa}\beta\delta_{n,1}\\
&-i\sqrt{\frac{\kappa}{2\pi}}\delta_{n,1}\int dk e^{-ik(t-t_0)}\hat{b}_k(t_0)-\delta_{n,1}\frac{\kappa}{2\pi}\int dk \int_{t_0}^t dt' e^{-i k (t-t')}\hat{a}_1\\
&-i\sum_{j=1}^{N_Y}\int dk \frac{1}{\sqrt{\pi}}Y_{n,j}e^{ik(t-t_0)}\hat{c}_{j,k}^\dagger(t_0)+\frac{1}{\pi}\sum_{j=1}^{N_Y}\sum_{i=1}^NY_{n,j}Y_{i,j}\int dk\int ^t_{t_0}dt'e^{ik(t-t')}\hat{a}_i\\
&-i\sum_{j=1}^{N_Z}\int dk \frac{1}{\sqrt{\pi}}Z_{n,j}e^{-ik(t-t_0)}\hat{d}_{j,k}(t_0)+\frac{1}{\pi}\sum_{j=1}^{N_Z}\sum_{i=1}^NZ_{n,j}Z_{i,j}\int dk\int ^t_{t_0}dt'e^{-ik(t-t')}\hat{a}_i\\
=&w \hat{a}_{n-1}+\Delta\hat{a}^\dagger_{n+1}+\Delta\hat{a}^\dagger_{n-1}-w\hat{a}_{n+1}-i\epsilon[\hat{a}_n,\hat{V}]
-\sqrt{\kappa}\beta\delta_{n,1}\\
&-\frac{\kappa}{2}\hat{a}_1(t)\delta_{n,1}+\sum_{i=1}^N(YY^\top-ZZ^\top)_{n,i}\hat{a}_i-\sqrt{\kappa}\hat{B}^{\textsf{in}}\delta_{n,1}-\sqrt{2}~\Bigg(\sum_{j=1}^{N_{Y}}Y_{n,j}\hat{C}_{j}^{\textsf{in}\dagger}+\sum_{j=1}^{N_{Z}}Z_{n,j}\hat{D}_{j}^{\textsf{in}}\Bigg),
\end{aligned}
\end{equation}
where we have defined $$\hat{B}^{\textsf{in}}=i\sqrt{\frac{1}{2\pi}}\int dk e^{-ik(t-t_0)}\hat{b}_k(t_0),$$ $$\hat{C}^{\textsf{in}}_j=i\sqrt{\frac{1}{2\pi}}\int dk e^{-ik(t-t_0)}\hat{c}_{j,k}(t_0),$$ $$\hat{D}^{\textsf{in}}=i\sqrt{\frac{1}{2\pi}}\int dk e^{-ik(t-t_0)}\hat{d}_{j,k}(t_0),$$
and used the equations $\int dk e^{-ik(t-t')}=2\pi \delta(t-t')$ and $\int_{t_0}^t dt' \delta(t-t')\hat{a}_i(t')=\frac{1}{2}\hat{a}_i(t)$. To ensure the Markovian nature of the entire dynamics, $\hat{B}^{\textsf{in}}$, $\hat{C}_{j}^{\textsf{in}}$ and $\hat{D}_{j}^{\textsf{in}}$ are assumed to be quantum Gaussian white noise: $\langle Q(t)Q^\dagger(t')\rangle=(\bar{n}^{\textsf{th}}_Q+1)\delta(t-t')$, $\langle Q^\dagger(t)Q(t')\rangle=\bar{n}^{\textsf{th}}_Q\delta(t-t')$, and $\langle Q(t)Q(t')\rangle=0$,
where $Q\in \{ \hat{B}^{\textsf{in}},~\hat{C}^{\textsf{in}}_j,~\hat{D}^{\textsf{in}}_j \}$, and there are no correlations between different noise operators. Here, $\bar{n}^{\textsf{th}}_Q$ is the number of thermal quanta in the input field. Therefore, the Heisenberg-Langevin equations can be expressed as
%Eq. \eqref{anheisenberg}.
\begin{equation}\label{anheisenberg}
\begin{aligned}
\frac{d\hat{a}_n}{dt}=&w \hat{a}_{n-1}-w\hat{a}_{n+1}+\Delta\hat{a}^\dagger_{n+1}+\Delta\hat{a}^\dagger_{n-1}-i\epsilon[\hat{a}_n,\hat{V}]-\frac{\kappa}{2}\hat{a}_1\delta_{n,1}\\
&+\sum_{j=1}^{N_{Y}}\sum_{i=1}^{N}Y_{n,j}Y_{i,j}\hat{a}_i-\sum_{j=1}^{N_{Z}}\sum_{i=1}^{N}Z_{n,j}Z_{i,j}\hat{a}_i-\sqrt{\kappa}(\hat{B}^{\textsf{in}}+\beta)\delta_{n,1}-\sqrt{2}\;\Bigg(\sum_{j=1}^{N_{Y}}Y_{n,j}\hat{C}_{j}^{\textsf{in}\dagger}+\sum_{j=1}^{N_{Z}}Z_{n,j}\hat{D}_{j}^{\textsf{in}}\Bigg).
\end{aligned}
\end{equation}

To see  how the signal is amplified, it is better to turn to the picture of canonical quadratures $\hat{x}_n$ and $\hat{p}_n$ defined  via $\hat{a}_n=\frac{\hat{x}_n+i\hat{p}_n}{\sqrt{2}}$.
Then the corresponding  Heisenberg-Langevin equations in terms of $\hat{x}_n$ and $\hat{p}_n$ read
\begin{equation}
\begin{aligned}
\frac{d\hat{x}_n}{dt}=&-(w-\Delta)\hat{x}_{n+1}+(w+\Delta)\hat{x}_{n-1}-i\epsilon[\hat{x}_n,\hat{V}]-\frac{\kappa}{2}\hat{x}_1\delta_{n,1}\\
&+\sum_{j=1}^{N_{Y}}\sum_{i=1}^{N}Y_{n,j}Y_{i,j}\hat{x}_i-\sum_{j=1}^{N_{Z}}\sum_{i=1}^{N}Z_{n,j}Z_{i,j}\hat{x}_i\\
&-\sqrt{\kappa}\frac{\hat{B}^{\textsf{in}}+\hat{B}^{\textsf{in}\dagger}}{\sqrt{2}}\delta_{n,1}-\sqrt{2\kappa}\beta\delta_{n,1}-\sqrt{2}~\Bigg(\sum_{j=1}^{N_{Y}}Y_{n,j}\frac{\hat{C}_{j}^{\textsf{in}\dagger}+\hat{C}_{j}^{\textsf{in}}}{\sqrt{2}}+\sum_{j=1}^{N_{Z}}Z_{n,j}\frac{\hat{D}_{j}^{\textsf{in}}+\hat{D}_{j}^{\textsf{in}\dagger}}{\sqrt{2}}\Bigg),\\
\frac{d\hat{p}_n}{dt}=&(w-\Delta)\hat{p}_{n-1}-(w+\Delta)\hat{p}_{n+1}-i\epsilon[\hat{p}_n,\hat{V}]-\frac{\kappa}{2}\hat{p}_1\delta_{n,1}\\
&+\sum_{j=1}^{N_{Y}}\sum_{i=1}^{N}Y_{n,j}Y_{i,j}\hat{p}_i-\sum_{j=1}^{N_{Z}}\sum_{i=1}^{N}Z_{n,j}Z_{i,j}\hat{p}_i\\
&-\sqrt{\kappa}\frac{\hat{B}^{\textsf{in}}-\hat{B}^{\textsf{in}\dagger}}{\sqrt{2}i}\delta_{n,1}-\sqrt{2}~\Bigg(\sum_{j=1}^{N_{Y}}Y_{n,j}\frac{\hat{C}_{j}^{\textsf{in}\dagger}-\hat{C}_{j}^{\textsf{in}}}{\sqrt{2}i}+\sum_{j=1}^{N_{Z}}Z_{n,j}\frac{\hat{D}_{j}^{\textsf{in}}-\hat{D}_{j}^{\textsf{in}\dagger}}{\sqrt{2}i}\Bigg).
\end{aligned}
\end{equation}
By defining $\hat{B}^{\textsf{in}}=\frac{\hat{X}^{\textsf{in}}+i\hat{P}^{\textsf{in}}}{\sqrt{2}}$, $\hat{C}_{j}^{\textsf{in}}=\frac{\hat{C}^{\textsf{in}}_{j,X}+i\hat{C}^{\textsf{in}}_{j,P}}{\sqrt{2}}$, $\hat{D}_{j}^{\textsf{in}}=\frac{\hat{D}^{\textsf{in}}_{j,X}+i\hat{D}^{\textsf{in}}_{j,P}}{\sqrt{2}}$, and let $J= \sqrt{w^2-\Delta^2}$ and $\exp\{2A\}= \frac{w+\Delta}{w-\Delta}$, the above equation can be described by
\begin{equation}
\begin{aligned}
\frac{d\hat{x}_n}{dt}=&-Je^{-A}\hat{x}_{n+1}+Je^A\hat{x}_{n-1}-i\epsilon[\hat{x}_n,\hat{V}]-\frac{\kappa}{2}\hat{x}_1\delta_{n,1}\\
&+\sum_{j=1}^{N_{Y}}\sum_{i=1}^{N}Y_{n,j}Y_{i,j}\hat{x}_i-\sum_{j=1}^{N_{Z}}\sum_{i=1}^{N}Z_{n,j}Z_{i,j}\hat{x}_i\\
&-\sqrt{\kappa}\hat{X}^{\textsf{in}}\delta_{n,1}-\sqrt{2\kappa}\beta\delta_{n,1}-\sqrt{2}~\Big(\sum_{j=1}^{N_{Y}}Y_{n,j}\hat{C}^{\textsf{in}}_{j,X}+\sum_{j=1}^{N_{Z}}Z_{n,j}\hat{D}^{\textsf{in}}_{j,X}\Big),\\
\frac{d\hat{p}_n}{dt}=&Je^{-A}\hat{p}_{n-1}-Je^A\hat{p}_{n+1}-i\epsilon[\hat{p}_n,\hat{V}]-\frac{\kappa}{2}\hat{p}_1\delta_{n,1}\\
&+\sum_{j=1}^{N_{Y}}\sum_{i=1}^{N}Y_{n,j}Y_{i,j}\hat{p}_i-\sum_{j=1}^{N_{Z}}\sum_{i=1}^{N}Z_{n,j}Z_{i,j}\hat{p}_i\\
&-\sqrt{\kappa}\hat{P}^{\textsf{in}}\delta_{n,1}-\sqrt{2}~\Big(-\sum_{j=1}^{N_{Y}}Y_{n,j}\hat{C}^{\textsf{in}}_{j,P}+\sum_{j=1}^{N_{Z}}Z_{n,j}\hat{D}^{\textsf{in}}_{j,P}\Big).
\end{aligned}
\end{equation}

By defining the quadrature vectors $\hat{\mathbf{X}}=(\hat{x}_1,\hat{x}_2,\ldots,\hat{x}_N)^\top$ and $\hat{\mathbf{P}}=(\hat{p}_1,\hat{p}_2,\ldots,\hat{p}_N)^\top$, we can convert the Heisenberg-Langevin equations into a compact form:
\begin{equation}\nonumber
\begin{aligned}
 \begin{pmatrix}
 \dot{ \hat{\mathbf{X}} } \\
  \dot{\hat{\mathbf{P}}}  \\
 \end{pmatrix}=&\begin{pmatrix}
  h^\mathbb{X}+YY^\top-ZZ^\top &0 \\
  0 & h^\mathbb{P}+YY^\top-ZZ^\top  \\
 \end{pmatrix}\begin{pmatrix}
  \hat{\mathbf{X}}  \\
  \hat{\mathbf{P}}  \\
 \end{pmatrix}-i\epsilon\begin{pmatrix}
 [\hat{\mathbf{X}},\hat{V}]\\
 [\hat{\mathbf{P}},\hat{V}]
 \end{pmatrix}-\vec{\beta}-\hat{\Omega}^{\textsf{in}}.
\end{aligned}
\end{equation}
Here, the dynamical matrices $h^\mathbb{X}$ and $h^\mathbb{P}$ are
\begin{equation*}
\begin{aligned}
h^\mathbb{X}&=-\frac{\kappa}{2}|1\rangle\langle1|+\sum^{N-1}_{n=1}\Big(Je^{A}|n+1\rangle\langle n|-Je^{-A}|n\rangle\langle n+1|\Big),\\
h^\mathbb{P}&=-\frac{\kappa}{2}|1\rangle\langle1|+\sum^{N-1}_{n=1}\Big(Je^{-A}|n+1\rangle\langle n|-Je^{A}|n\rangle\langle n+1|\Big),
\end{aligned}
\end{equation*}
the commutation with $\hat{V}$ is
\begin{equation*}
\begin{pmatrix}
  [\hat{\mathbf{X}},\hat{V}] \\
  [\hat{\mathbf{P}},\hat{V}]
\end{pmatrix}=([\hat{x}_1,\hat{V}],\ldots,[\hat{x}_N,\hat{V}],[\hat{p}_1,\hat{V}],\ldots,[\hat{p}_N,\hat{V}])^\top,
\end{equation*}
the coherent input vector $\vec{\beta}$ is
\begin{equation*}
\vec{\beta}=(\sqrt{2\kappa}\beta,0,0,\ldots,0)^\top,\\
\end{equation*}
and the quantum noise vector $\hat{\Omega}^{\textsf{in}}$ are
\begin{equation*}
\begin{aligned}
\hat{\Omega}^{\textsf{in}}_{i}&=\sqrt{\kappa}\hat{X}^{\textsf{in}}\delta_{i,1}+\sqrt{2}~\Bigg(\sum_{j=1}^{N_Y}Y_{i,j}\hat{C}_{j,X}^{\textsf{in}}+\sum_{j=1}^{N_Z}Z_{i,j}\hat{D}_{j,X}^{\textsf{in}}\Bigg),\\
\hat{\Omega}^{\textsf{in}}_{i+N}&=\sqrt{\kappa}\hat{P}^{\textsf{in}}\delta_{i,1}+\sqrt{2}~\Bigg(-\sum_{j=1}^{N_Y}Y_{i,j}\hat{C}_{j,P}^{\textsf{in}}+\sum_{j=1}^{N_Z}Z_{i,j}\hat{D}_{j,P}^{\textsf{in}}\Bigg),
\end{aligned}
\end{equation*}
for $i=1, \cdots, N$.

\section{Derivations of the SNR per photon}

In this section, we calculate the signal power, noise power, and the total average photon number when $\epsilon$ is infinitesimal.

According to the Heisenberg-Langevin equations and the definition of  the perturbation Hamiltonian $\epsilon\hat{V}=\epsilon\hat{a}^\dag_N\hat{a}_N$, we have
\begin{equation}
\begin{aligned}
\frac{d\hat{x}_n}{dt}=&Je^A\hat{x}_{n-1}-Je^{-A}\hat{x}_{n+1}+\epsilon\hat{p}_N\delta_{n,N}-\sqrt{2\kappa}\beta\delta_{n,1}-\frac{\kappa}{2}\hat{x}_1\delta_{n,1}\\
&+\sum_{j=1}^{N_{Y}}\sum_{i=1}^{N}Y_{n,j}Y_{i,j}\hat{x}_i-\sum_{j=1}^{N_{Z}}\sum_{i=1}^{N}Z_{n,j}Z_{i,j}\hat{x}_i\\
&-\sqrt{2}~\Bigg(\sum_{j=1}^{N_{Y}}Y_{n,j}\hat{C}_{j,X}^{\textsf{in}}+\sum_{j=1}^{N_{Z}}Z_{n,j}\hat{D}_{j,X}^{\textsf{in}}\Bigg)-\sqrt{\kappa}\hat{X}^{\textsf{in}}\delta_{n,1},\\
\frac{d\hat{p}_n}{dt}=&Je^{-A}\hat{p}_{n-1}-Je^{A}\hat{p}_{n+1}-\epsilon\hat{x}_N\delta_{n,N}-\frac{\kappa}{2}\hat{p}_1\delta_{n,1}\\
&+\sum_{j=1}^{N_{Y}}\sum_{i=1}^{N}Y_{n,j}Y_{i,j}\hat{p}_i-\sum_{j=1}^{N_{Z}}\sum_{i=1}^{N}Z_{n,j}Z_{i,j}\hat{p}_i\\
&-\sqrt{2}~\Bigg(-\sum_{j=1}^{N_{Y}}Y_{n,j}\hat{C}_{j,P}^{\textsf{in}}+\sum_{j=1}^{N_{Z}}Z_{n,j}\hat{D}_{j,P}^{\textsf{in}}\Bigg)-\sqrt{\kappa}\hat{P}^{\textsf{in}}\delta_{n,1}.
\end{aligned}
\end{equation}
Since the system is stable, for sufficiently long time, $\hat{x}_n$ and $\hat{p}_n$ can be described as
\begin{equation}\label{appendixC2}
\begin{aligned}
\hat{x}_n=&\mathbb{H}[\epsilon]^{-1}_{n,1}(\sqrt{2\kappa}\beta+\sqrt{\kappa}\hat{X}^{\textsf{in}}) +\mathbb{H}[\epsilon]^{-1}_{n,N+1}\sqrt{\kappa}\hat{P}^{\textsf{in}}\\
&+\sqrt{2}\sum^N_{i=1}\mathbb{H}[\epsilon]^{-1}_{n,i}\Big(\sum^{N_Y}_{j=1}Y_{i,j}\hat{C}^{\textsf{in}}_{j,X}
+\sum^{N_Z}_{j=1}Z_{i,j}\hat{D}^{\textsf{in}}_{j,X}\Big)\\
&+\sqrt{2}\sum^N_{i=1}\mathbb{H}[\epsilon]^{-1}_{n,N+i}\Big(-\sum^{N_Y}_{j=1}Y_{i,j}\hat{C}^{\textsf{in}}_{j,P}
+\sum^{N_Z}_{j=1}Z_{i,j}\hat{D}^{\textsf{in}}_{j,P}\Big),\\
\hat{p}_n=&\mathbb{H}[\epsilon]^{-1}_{N+n,1}(\sqrt{2\kappa}\beta+\sqrt{\kappa}\hat{X}^{\textsf{in}}) +\mathbb{H}[\epsilon]^{-1}_{N+n,N+1}\sqrt{\kappa}\hat{P}^{\textsf{in}}\\
&+\sqrt{2}\sum^N_{i=1}\mathbb{H}[\epsilon]^{-1}_{N+n,i}\Big(\sum^{N_Y}_{j=1}
Y_{i,j}\hat{C}^{\textsf{in}}_{j,X}+\sum^{N_Z}_{j=1}Z_{i,j}\hat{D}^{\textsf{in}}_{j,X}\Big)\\
&+\sqrt{2}\sum^N_{i=1}\mathbb{H}[\epsilon]^{-1}_{N+n,N+i}\Big(-\sum^{N_Y}_{j=1}
Y_{i,j}\hat{C}^{\textsf{in}}_{j,P}+\sum^{N_Z}_{j=1}Z_{i,j}\hat{D}^{\textsf{in}}_{j,P}\Big),
\end{aligned}
\end{equation}
where $\mathbb{H}[\epsilon]=\mathbb{H}_1[0]+\mathbb{H}_N[\epsilon]$,
$
\mathbb{H}_1[0]=\begin{pmatrix}
  (Q^{\mathbb{X}})^{-1} &0 \\
  0 & (Q^{\mathbb{P}})^{-1} \\
 \end{pmatrix},
$
$Q^\mathbb{X}=(h^\mathbb{X}+YY^\top-ZZ^\top)^{-1}$, $Q^\mathbb{P}=(h^\mathbb{P}+YY^\top-ZZ^\top)^{-1}$
and $\mathbb{H}_N[\epsilon]=\epsilon |N\rangle\langle2N|-\epsilon|2N\rangle\langle N|$.

Using Dyson's equation and keeping it up to the first order in $\epsilon$, we have
\begin{equation}
\begin{aligned}
\mathbb{H}[\epsilon]^{-1}=(\mathbb{H}_1[0] +\mathbb{H}_N[\epsilon])^{-1} = \mathbb{H}_1[0]^{-1}-\mathbb{H}_1[0]^{-1}\mathbb{H}_N
[\epsilon]\mathbb{H}_1[0]^{-1}.
\end{aligned}
\end{equation}
The detailed calculation of the elements of $\mathbb{H}[\epsilon]^{-1}$ can be found in Section VII.

It can be computed that the steady state of mode 1 is
\begin{equation}\label{x1p1}
\begin{aligned}
\hat{x}_1=&Q^{\mathbb{X}}_{1,1}(\sqrt{2\kappa}\beta+\sqrt{\kappa}\hat{X}^{\textsf{in}}) -\epsilon Q^{\mathbb{X}}_{1,N}Q^{\mathbb{P}}_{N,1}\sqrt{\kappa}\hat{P}^{\textsf{in}}\\
&+\sqrt{2}\sum^N_{i=1}Q^{\mathbb{X}}_{1,i}\Big(\sum^{N_Y}_{j=1}Y_{i,j}\hat{C}^{\textsf{in}}_{j,X}+\sum^{N_Z}_{j=1}Z_{i,j}\hat{D}^{\textsf{in}}_{j,X}\Big)\\&-\sqrt{2}\epsilon\sum^N_{i=1}Q^{\mathbb{X}}_{1,N}Q^{\mathbb{P}}_{N,i}\Big(-\sum^{N_Y}_{j=1}Y_{i,j}\hat{C}^{\textsf{in}}_{j,P}+\sum^{N_Z}_{j=1}Z_{i,j}\hat{D}^{\textsf{in}}_{j,P}\Big),\\
\hat{p}_1=&\epsilon Q^{\mathbb{X}}_{N,1}Q^{\mathbb{P}}_{1,N}(\sqrt{2\kappa}\beta+\sqrt{\kappa}\hat{X}^{\textsf{in}}) +Q^{\mathbb{P}}_{1,1}\sqrt{\kappa}\hat{P}^{\textsf{in}}\\
&+\sqrt{2}\epsilon\sum^N_{i=1}Q^{\mathbb{X}}_{N,i}Q^{\mathbb{P}}_{1,N}\Big(\sum^{N_Y}_{j=1}Y_{i,j}\hat{C}^{\textsf{in}}_{j,X}+\sum^{N_Z}_{j=1}Z_{i,j}\hat{D}^{\textsf{in}}_{j,X}\Big)\\&+\sqrt{2}\sum^N_{i=1}Q^{\mathbb{P}}_{1,i}\Big(-\sum^{N_Y}_{j=1}Y_{i,j}\hat{C}^{\textsf{in}}_{j,P}+\sum^{N_Z}_{j=1}Z_{i,j}\hat{D}^{\textsf{in}}_{j,P}\Big),
\end{aligned}
\end{equation}
where we have used $\mathbb{H}[0]^{-1}=\begin{pmatrix}
  Q^{\mathbb{X}} &0 \\
  0 & Q^{\mathbb{P}}  \\
 \end{pmatrix}$.

From the definition of  $\hat{\mathcal{M}}$, we have
\begin{equation}
\begin{aligned}
\langle\hat{\mathcal{M}}\rangle_{\epsilon}&=\frac{1}{\sqrt{2}i}\Big{(}\langle\hat{\mathcal{B}}\rangle_{\epsilon}-\langle\hat{\mathcal{B}}^\dagger\rangle_{\epsilon}\Big{)}\\
&=\frac{1}{\sqrt{2\tau}i}\int_{0}^{\tau}\sqrt{\kappa}~\Big(\langle\hat{a}_1\rangle_{\epsilon}-\langle\hat{a}_1^\dagger\rangle_{\epsilon}\Big)\;dt\\
&=\frac{1}{\sqrt{2\tau}i}\int_{0}^{\tau}\sqrt{\kappa}\sqrt{2}i\langle\hat{p}_1\rangle_\epsilon \;dt\\
&=\frac{1}{\sqrt{2\tau}i}\int_{0}^{\tau}\sqrt{\kappa}\sqrt{2}i\epsilon Q^{\mathbb{X}}_{N,1}Q^{\mathbb{P}}_{1,N}\sqrt{2\kappa}\beta \;dt\\
&=\sqrt{2\kappa\tau}\epsilon\sqrt{\kappa}\beta Q^{\mathbb{X}}_{N,1}Q^{\mathbb{P}}_{1,N}.
\end{aligned}
\end{equation}

According to the definition of the signal power, we have
\begin{equation}\label{App2signal}
\begin{aligned}
\mathcal{S}(\epsilon)&=\Big|\langle\hat{\mathcal{M}}\rangle_\epsilon-\langle\hat{\mathcal{M}}\rangle_0\Big|^2=
2\epsilon^2\kappa^2\beta^2\tau\Big{|}Q^\mathbb{X}_{N,1}\Big{|}^2\Big{|}Q^\mathbb{P}_{1, N}\Big{|}^2.
\end{aligned}
\end{equation}

For the noise power, recall that only the zeroth order in $\epsilon$ is related to the SNR. Thus,
\begin{equation}
\begin{aligned}
&\hat{\mathcal{M}}\Big|_{\epsilon=0}-\langle\hat{\mathcal{M}}\rangle\Big|_{\epsilon=0}\\ =&\frac{1}{\sqrt{2\tau}i}\int^\tau_0\big{(}\hat{B}^{\textsf{in}}+\sqrt{\kappa}(\hat{a}_1-\langle\hat{a}_1\rangle)
\big|_{\epsilon=0}\big{)}-\big{(}\hat{B}^{\textsf{in}\dagger}+\sqrt{\kappa}(\hat{a}^\dagger_1
-\langle\hat{a}^\dagger_1\rangle)\big|_{\epsilon=0}\big{)}dt\\ =&\frac{1}{\sqrt{2\tau}i}\int^\tau_0 \Bigg[\Big(\frac{1}{\sqrt{2}}+\frac{\kappa}{\sqrt{2}}Q^\mathbb{X}_{1,1}\Big)\hat{X}^{\textsf{in}}+\sqrt{\kappa}\sum^N_{i=1}Q^\mathbb{X}_{1,i}\Big(\sum^{N_Y}_{j=1}Y_{i,j}\hat{C}^{\textsf{in}}_{j,X}+\sum^{N_Z}_{j=1}Z_{i,j}\hat{D}^{\textsf{in}}_{j,X}\Big)\\
&~~~~~~+i\Big(\frac{1}{\sqrt{2}}+\frac{\kappa}{\sqrt{2}}Q^\mathbb{P}_{1,1}\Big)\hat{P}^{\textsf{in}}+i\sqrt{\kappa}\sum^N_{i=1}Q^\mathbb{P}_{1,i}\Big(\sum^{N_Y}_{j=1}Y_{i,j}\hat{C}^{\textsf{in}}_{j,P}+\sum^{N_Z}_{j=1}Z_{i,j}\hat{D}^{\textsf{in}}_{j,P}\Big)\Bigg]\\
&~~~~~~-\Bigg[\Big(\frac{1}{\sqrt{2}}+\frac{\kappa}{\sqrt{2}}Q^\mathbb{X}_{1,1}\Big)\hat{X}^{\textsf{in}}+\sqrt{\kappa}\sum^N_{i=1}Q^\mathbb{X}_{1,i}\Big(\sum^{N_Y}_{j=1}Y_{i,j}\hat{C}^{\textsf{in}}_{j,X}+\sum^{N_Z}_{j=1}Z_{i,j}\hat{D}^{\textsf{in}}_{j,X}\Big)\\
&~~~~~~-i\Big(\frac{1}{\sqrt{2}}+\frac{\kappa}{\sqrt{2}}Q^\mathbb{P}_{1,1}\Big)\hat{P}^{\textsf{in}}-i\sqrt{\kappa}\sum^N_{i=1}Q^\mathbb{P}_{1,i}\Big(\sum^{N_Y}_{j=1}Y_{i,j}\hat{C}^{\textsf{in}}_{j,P}+\sum^{N_Z}_{j=1}Z_{i,j}\hat{D}^{\textsf{in}}_{j,P}\Big)\Bigg]dt\\
=&\frac{1}{\sqrt{2\tau}}\int^\tau_0\sqrt{2}\Big(1+\kappa Q^\mathbb{P}_{1,1}\Big)\hat{P}^{\textsf{in}}+2\sqrt{\kappa}\sum^N_{i=1}Q^\mathbb{P}_{1,i}\Big(\sum^{N_Y}_{j=1}Y_{i,j}\hat{C}^{\textsf{in}}_{j,P}+\sum^{N_Z}_{j=1}Z_{i,j}\hat{D}^{\textsf{in}}_{j,P}\Big)
dt.
\end{aligned}
\end{equation}
Hence, it can be computed that the noise power is
\begin{equation}\label{appendixC7}
\begin{aligned}
\mathcal{N}(0)=\frac{1}{2}(1+\kappa Q_{1,1}^\mathbb{P})^2+\kappa\big{[} Q^\mathbb{P}(YY^\top+ZZ^\top){Q^\mathbb{P}}^{\top}\big{]}_{1,1},
\end{aligned}
\end{equation}
where we have focused on the vacuum noise, i.e., $\bar{n}^{\textsf{th}}_Q=0$.

Following a similar reasoning to the noise power, we are only concerned about $\bar{n}_{\textsf{tot}}(0)$, the zeroth order term of the total average photon number with respect to $\epsilon$. In the large-drive limit $|\beta|\geq 1$, we have
\begin{equation}\label{app2ntot}
\begin{aligned}
\bar{n}_{\textsf{tot}}(0)= &\sum_{n=1}^N \langle\hat{a}_n^\dagger\rangle_0\langle\hat{a}_n\rangle_0=\frac{1}{2}\sum_{n=1}^N\langle\hat{x}_n\rangle^2_0+\langle\hat{p}_n\rangle^2_0
=\kappa\beta^2\sum_{n=1}^N\Big|Q^\mathbb{X}_{n,1}\Big|^2
=\kappa\beta^2[{Q^{\mathbb{X}}}^\top Q^\mathbb{X}]_{1,1}.
\end{aligned}
\end{equation}

\section{The SNR under condition (\textbf{C1}) }

In this section, we demonstrate that when the coupling between the system and the gain bath satisfies $Y=0$, the loss structure $Z$ satisfies condition (\textbf{C1}), and the sensing dynamics is stable, the signal power $\mathcal{S}$, noise power $\mathcal{N}$, and the total average photon number $\bar{n}_{\textsf{tot}}(0)$ are the same as those in the ideal noise-free case.

First of all, we prove $(h^\mathbb{P}-ZZ^\top)^{-1}_{1,j}=(h^\mathbb{P})^{-1}_{1,j}$ and $(h^\mathbb{X}-ZZ^\top)^{-1}_{j,1}=(h^\mathbb{X})^{-1}_{j,1}$, for $j=1,\cdots,N$. Indeed, according to the matrix inverse formula,
\begin{equation}\nonumber
\begin{aligned}
(h^\mathbb{P}-ZZ^\top)^{-1}
&=(h^\mathbb{P}-h^\mathbb{P}CC^\top{ h^{\mathbb{P}}}^{\top})^{-1}\\
&=(h^\mathbb{P})^{-1}+(h^\mathbb{P})^{-1}h^\mathbb{P}CC^\top(-{ h^{\mathbb{P}}}^{\top}(h^{\mathbb{P}})^{-1}
h^\mathbb{P}CC^\top+I)^{-1}{ h^{\mathbb{P}}}^{\top}(h^\mathbb{P})^{-1}\\
&=(h^\mathbb{P})^{-1}+CC^\top(I-{ h^{\mathbb{P}}}^{\top}CC^\top)^{-1}{ h^{\mathbb{P}}}^{\top}(h^\mathbb{P})^{-1}.
\end{aligned}
\end{equation}
Thus, for $j=1,\cdots,N$, we have
\begin{equation}\nonumber
(h^\mathbb{P}-ZZ^\top)^{-1}_{1,j}
=(h^\mathbb{P})^{-1}_{1,j}+\sum_{k=1}^{N}(CC^\top)_{1,k}[(I-h^{\mathbb{P}\top}CC^\top)^{-1}
{ h^{\mathbb{P}}}^{\top}(h^\mathbb{P})^{-1}]_{k,j}=(h^\mathbb{P})^{-1}_{1,j}.
\end{equation}

Note that since the span of the second through  the $N$-th column of $h^\mathbb{P}$ is the same  as that of ${h^{\mathbb{X}}}^{\top}$, there exists $D\in R^{N_Z\times N}$ with $D_{j,1}=0$ for $j=1,\cdots,N_Z$, such that $Z={h^{\mathbb{X}}}^{\top} D^\top$. Then according to the matrix inverse formula,
\begin{equation}\nonumber
\begin{aligned}
(h^\mathbb{X}-ZZ^\top)^{-1}
&=(h^\mathbb{X}-{h^{\mathbb{X}}}^{\top}D^\top D h^{\mathbb{X}})^{-1}\\
&=(h^\mathbb{X})^{-1}+(h^\mathbb{X})^{-1}{h^{\mathbb{X}}}^{\top}(I-D^\top Dh^{\mathbb{X}}(h^{\mathbb{X}})^{-1}{h^{\mathbb{X}}}^{\top})^{-1}D^\top D h^{\mathbb{X}}(h^\mathbb{X})^{-1}\\
&=(h^\mathbb{X})^{-1}+(h^\mathbb{X})^{-1}{h^{\mathbb{X}}}^{\top}(I-D^\top D{h^{\mathbb{X}}}^{\top})^{-1}D^\top D.
\end{aligned}
\end{equation}
Thus, for $j=1, \cdots,N$, we have
\begin{equation}\nonumber
\begin{aligned}
(h^\mathbb{X}-ZZ^\top)^{-1}_{j,1}
&=(h^\mathbb{X})^{-1}_{j,1}+\sum_{k=1}^{N}[(h^\mathbb{X})^{-1}{h^{\mathbb{X}}}^{\top}
(I-D^\top D{h^{\mathbb{X}}}^{\top})^{-1}]_{j,k}(D^\top D)_{k,1}=(h^\mathbb{X})^{-1}_{j,1}.
\end{aligned}
\end{equation}

Therefore, specifically, we have $(h^\mathbb{P}-ZZ^\top)^{-1}_{1,N}=(h^\mathbb{P})^{-1}_{1,N}$ and $(h^\mathbb{X}-ZZ^\top)^{-1}_{N,1}=(h^\mathbb{X})^{-1}_{N,1}$.
Recall that the signal power is in the form of Eq.~\eqref{App2signal}, we have proved that under $Y=0$, (\textbf{C1}), and if the system is stable,
the signal power of the noisy sensor  is the same as that of the ideal noise-free case.

As for the noise power, according to  Eq.~\eqref{appendixC7}, we only need to prove  $$[(h^\mathbb{P}-ZZ^\top)^{-1}ZZ^\top(h^\mathbb{P}-ZZ^\top)^{-1\top}]_{1,1}=0.$$ In fact,
\begin{equation}\nonumber
\begin{aligned}
&~~[(h^\mathbb{P}-ZZ^\top)^{-1}ZZ^\top(h^\mathbb{P}-ZZ^\top)^{-1\top}]_{1,1}\\
=&\sum_{k=1}^{N_Z}[(h^\mathbb{P}-ZZ^\top)^{-1}Z]_{1,k}\cdot[(h^\mathbb{P}-ZZ^\top)^{-1}Z]_{1,k}\\
=&\sum_{k=1}^{N_Z}\bigg[\sum_{m=1}^{N}(h^\mathbb{P}-ZZ^\top)^{-1}_{1,m}Z_{m,k}\bigg]\cdot\bigg[\sum_{m=1}^{N}(h^\mathbb{P}-ZZ^\top)^{-1}_{1,m}Z_{m,k}\bigg]\\
=&\sum_{k=1}^{N_Z}\bigg[\sum_{m=1}^{N}(h^\mathbb{P})^{-1}_{1,m}Z_{m,k}\bigg]\cdot\bigg[\sum_{m=1}^{N}(h^\mathbb{P})^{-1}_{1,m}Z_{m,k}\bigg]\\
=&[(h^\mathbb{P})^{-1}ZZ^\top(h^\mathbb{P})^{-1\top}]_{1,1}\\
=&(CC^\top)_{1,1}=0.
\end{aligned}
\end{equation}
Thus, under $Y=0$, (\textbf{C1}), and if the system is stable,
the  noise power is the same as that of the ideal noise-free case.

As for  the total average photon number $\bar{n}_{\textsf{tot}}(0)$, since $Q^{\mathbb{X}}_{j,1}=(h^\mathbb{X}-ZZ^\top)^{-1}_{j,1}=(h^\mathbb{X})^{-1}_{j,1}$, along with Eq.~\eqref{app2ntot}, it is clear that  $\bar{n}_{\textsf{tot}}$ is also the same as that of the ideal noise-free case.

\section{The SNR per photon under conditions \text{(\textbf{C1})} and \text{(\textbf{C2})}}

Now we demonstrate that under conditions (\textbf{C1}) and (\textbf{C2}), an exponential enhancement of noisy non-Hermitian sensing  can be revived, that is $\overline{\textrm{SNR}} \propto \exp\{2A(N-1)\}$.

Under  (\textbf{C1}) and (\textbf{C2}), the sensing dynamics is stable.  From  Eqs.~\eqref{App2signal} and \eqref{app2ntot}, it is clear that
\begin{equation}
\begin{aligned}
\mathcal{S}(\epsilon)&=2\epsilon^2\kappa^2\beta^2\tau\Big{|}(h^\mathbb{X})^{-1}_{N,1}\Big{|}^2\Big{|}(h^\mathbb{P})^{-1}_{1, N}\Big{|}^2
=2\epsilon^2\kappa^2\beta^2\tau(\frac{2}{\kappa})^4e^{4A(N-1)},\\
\bar{n}_{\textsf{tot}}(0)&=\kappa\beta^2[(h^{\mathbb{X}})^{-1\top} (h^\mathbb{X})^{-1}]_{1,1},
\end{aligned}
\end{equation}
where we have used $(h^\mathbb{X})^{-1}_{N,1}=-\frac{2}{\kappa} \exp\{A(N-1)\}$, and $(h^\mathbb{P})^{-1}_{1, N}=-\frac{2}{\kappa} \exp\{A(N-1)\}$ (see Section VII).

As for the noise power,
\begin{equation}
\begin{aligned}
\mathcal{N}(0)&=\frac{1}{2}(1+\kappa Q_{1,1}^\mathbb{P})^2+\kappa\big{[} Q^\mathbb{P}(YY^\top+ZZ^\top){Q^\mathbb{P}}^{\top}\big{]}_{1,1}\\
&=\frac{1}{2}(1+\kappa (h^\mathbb{P})^{-1}_{1,1})^2+2\kappa\big{[} (h^\mathbb{P})^{-1}ZZ^\top{(h^\mathbb{P}})^{-1\top}\big{]}_{1,1}\\
&=\frac{1}{2},
\end{aligned}
\end{equation}
where $(h^\mathbb{P})^{-1}_{1,1}=-\frac{2}{\kappa}$ has been used (see Section VII).
Thus we now have  $\overline{\textrm{SNR}} \propto \exp\{2A(N-1)\}$.



\section{Stability analysis under $YY^{\top}\neq ZZ^{\top}$}
We  now employ Routh’s stability criterion to derive the necessary conditions of the elements of $(YY^\top-ZZ^\top)$ to ensure the stability of the system.
Note that the dynamical matrix $(h^\mathbb{X}+YY^\top-ZZ^\top)$ being stable is equivalent to that all the roots of its characteristic polynomial sit on the left half plane.

\begin{figure}[htbp]
\centering
\subfigure[]{
\includegraphics[scale=0.7]{fig3}
}
\quad
\subfigure[]{
\includegraphics[scale=0.7]{fig4}
}
\caption{A schematic of an $N$-site non-Hermitian setup. (a) There is an effective coupling $\gamma$ between the first and the $(N-1)$-th modes.
(b)  There is an effective coupling $\gamma$ between the first and the $N$-th modes.}
\end{figure}

\textbf{Case 1}: $YY^\top-ZZ^\top=\gamma|1\rangle\langle N-1|+\gamma| N-1\rangle\langle1|$.


The noisy setup is illustrated in Fig.~1(a), where there is  an effective coupling $\gamma$ between the first and the $(N-1)$-th modes.
The characteristic polynomial of $(h^\mathbb{X}+YY^\top-ZZ^\top)$  can be calculated in a recursive way as
\begin{equation}\label{characteristic1}
\begin{aligned}
\Big|\lambda I-(h^\mathbb{X}+YY^\top-ZZ^\top)\Big|=&(\lambda+\frac{\kappa}{2})D_{N-1}+J^2D_{N-2}-\gamma^2\lambda D_{N-3}\\
&+\gamma\lambda J^{N-2}e^{-A(N-2)}-\gamma\lambda J^{N-2}e^{A(N-2)},
\end{aligned}
\end{equation}
where
\begin{equation}
\begin{aligned}
D_{N}=\Bigg|\lambda I-\sum^{N-1}_{n=1}\Big(-Je^{-A}|n\rangle\langle n+1|+Je^{A}|n+1\rangle\langle n|\Big)\Bigg|=\sum_{k=0}^{[\frac{N}{2}]}\begin{pmatrix}
N-k\\
k
\end{pmatrix}\lambda^{N-2k}J^{2k},
\end{aligned}
\end{equation}
and $[x]$ is the integer function which returns the largest integer not larger than $x$.

According to Routh's stability criterion, a necessary (but not sufficient) condition for stability is that all the coefficients of $\lambda$  are positive. It can be computed that the coefficient of the first order with respect to $\lambda$  in Eq. \eqref{characteristic1} is
\begin{equation}\label{B3}
\begin{aligned}
&\begin{pmatrix}
\frac{N-1}{2}\\
\frac{N-1}{2}
\end{pmatrix}J^{N-1}+\begin{pmatrix}
\frac{N-1}{2}\\
\frac{N-3}{2}
\end{pmatrix}J^{N-1}-\gamma^2\begin{pmatrix}
\frac{N-3}{2}\\
\frac{N-3}{2}
\end{pmatrix}J^{N-3}+\gamma J^{N-2}e^{-A(N-2)}-\gamma J^{N-2}e^{A(N-2)}\\
&=J^{N-3}\Bigg[\frac{N+1}{2}J^2+\Big(e^{-A(N-2)}-e^{A(N-2)}\Big)J\gamma-\gamma^2\Bigg].
\end{aligned}
\end{equation}
It can be verified that when $e^{-A(N-2)}$ is sufficiently small, to ensure Eq.~(\ref{B3}) be positive, $\gamma$
should satisfy $\gamma_1<\gamma<\gamma_2$, where
\begin{equation}
\begin{aligned}
\gamma_1&=\frac{(e^{-A(N-2)}-e^{A(N-2)})-\sqrt{(e^{-A(N-2)}-e^{A(N-2)})^2+2(N+1)}}{2}J\\ &=-Je^{A(N-2)}+O(e^{-A(N-2)}),\\
\gamma_2&=\frac{(e^{-A(N-2)}-e^{A(N-2)})+\sqrt{(e^{-A(N-2)}-e^{A(N-2)})^2+2(N+1)}}{2}J\\ &=\frac{N+1}{2}Je^{-A(N-2)}+O(e^{-3A(N-2)}).
\end{aligned}
\end{equation}
Thus, if $e^{-A(N-2)}$ is sufficiently small, to ensure the stability of $(h^\mathbb{X}+YY^\top-ZZ^\top)$,  a necessary condition for $\gamma$ is
\begin{equation}\label{l1}
\begin{aligned}
-Je^{A(N-2)}<\gamma<\frac{N+1}{2}Je^{-A(N-2)}.
\end{aligned}
\end{equation}

After a similar analysis with the characteristic polynomial of $(h^\mathbb{P}+YY^\top-ZZ^\top)$, it can be verified that if $e^{-A(N-2)}$ is sufficiently small, to ensure the stability of $(h^\mathbb{P}+YY^\top-ZZ^\top)$, a necessary condition is
\begin{equation}\label{r1}
\begin{aligned}
-\frac{N+1}{2}Je^{-A(N-2)}<\gamma<Je^{A(N-2)}.
\end{aligned}
\end{equation}
Combining Eqs.~(\ref{l1}) and (\ref{r1}), if $YY^\top-ZZ^\top=\gamma|1\rangle\langle N-1|+\gamma| N-1\rangle\langle1|$, a necessary (but not sufficient) condition for the stability of the  system is
\begin{equation}\label{B7}
\begin{aligned}
|\gamma|<\frac{N+1}{2}Je^{-A(N-2)}.
\end{aligned}
\end{equation}


\textbf{Case 2}: $YY^\top-ZZ^\top=\gamma|1\rangle\langle N|+\gamma| N\rangle\langle1|$.

The noisy setup is illustrated in Fig.~1(b), where there is  an effective coupling $\gamma$ between the first and the $N$-th modes. The characteristic polynomial of $(h^\mathbb{X}+YY^\top-ZZ^\top)$ is
\begin{equation}\label{characteristic2}
\begin{aligned}
&\Big|\lambda I-(h^\mathbb{X}+YY^\top-ZZ^\top)\Big|\\=&(\lambda+\frac{\kappa}{2})D_{N-1}+J^2D_{N-2}-\gamma^2 D_{N-2}-\gamma J^{N-1}e^{-A(N-1)}-\gamma J^{N-1}e^{A(N-1)}\\
=&\lambda^N+c_{N-1}\lambda^{N-1}+c_{N-2}\lambda^{N-2}+\ldots+c_1\lambda+c_0,
\end{aligned}
\end{equation}
where
\begin{equation}
\begin{aligned}
c_0=\frac{\kappa}{2}J^{N-1}-\gamma J^{N-1}(e^{-A(N-1)}+e^{A(N-1)}).
\end{aligned}
\end{equation}
According to Routh's stability criterion, a necessary condition for $(h^\mathbb{X}+YY^\top-ZZ^\top)$ to be stable is that all the coefficients of $\lambda$  of the characteristic polynomial  are positive. Thus, from $c_0>0$, we have
\begin{equation}\label{r2}
\begin{aligned}
\gamma<\frac{\kappa}{2}\frac{1}{e^{-A(N-1)}+e^{A(N-1)}}.
\end{aligned}
\end{equation}
%noting that $\frac{\frac{\kappa}{2}}{e^{-A(N-1)}+e^{A(N-1)}}=\frac{\kappa}{2}e^{-A(N-1)}+O(e^{-3A(N-1)})$.
Note that a system is stable if and only if all the elements in the first column of the Routh array are positive. It can be verified that calculating to the fourth row of the Routh array implies that
\begin{equation}\label{l2}
\gamma>-\kappa e^{-A(N-1)}.
\end{equation}
Combining Eqs.~(\ref{r2}) and (\ref{l2}), we obtain that if $YY^\top-ZZ^\top=\gamma|1\rangle\langle N|+\gamma| N\rangle\langle1|$, a necessary (but not sufficient) condition for the  stability of the system is
\begin{equation}\label{B12}
\begin{aligned}
|\gamma|<\kappa e^{-A(N-1)}.
\end{aligned}
\end{equation}

From Eqs.~(\ref{B7}) and (\ref{B12}), it can be seen that if the gain and loss are unbalanced, then  to ensure a stable non-Hermitian sensor, the $(1, N-1)$th and $(1, N)$th elements of the net noise matrix $(YY^\top-ZZ^\top)$ should be exponentially small in terms of the product of $A$ and $N$.

\section{Derivations of the SNR beyond linear response}
In this section, we consider the case where the parameter to be detected, $\epsilon_0$ , is not infinitesimally small.
Thus, as opposed to the infinitesimal case, not only the linear response
of $\epsilon_0$ , but all orders in $\epsilon_0$ of the output field should be
calculated.


The signal power and the total average photon number can be  straightforwardly  calculated  from Eq.~\eqref{appendixC2}.
From the definition of  $\hat{\mathcal{M}}$, we have
\begin{equation}
\begin{aligned}
\langle\hat{\mathcal{M}}\rangle_{\epsilon_0}&=\frac{1}{\sqrt{2}i}\Big{(}\langle\hat{\mathcal{B}}\rangle_{\epsilon_0}-\langle\hat{\mathcal{B}}^\dagger\rangle_{\epsilon_0}\Big{)}\\
&=\frac{1}{\sqrt{2\tau}i}\int_{0}^{\tau}\sqrt{\kappa}\sqrt{2}i\langle\hat{p}_1\rangle_{\epsilon_0} dt\\
&=\frac{1}{\sqrt{2\tau}i}\int_{0}^{\tau}\sqrt{\kappa}\sqrt{2}i\mathbb{H}[\epsilon_0]^{-1}_{N+1,1}\sqrt{2\kappa}\beta dt\\
&=\frac{1}{\sqrt{\tau}}\int_{0}^{\tau}\sqrt{\kappa}\mathbb{H}[\epsilon_0]^{-1}_{N+1,1}\sqrt{2\kappa}\beta dt\\
&=\sqrt{2\kappa\tau}\sqrt{\kappa}\beta\mathbb{H}[\epsilon_0]^{-1}_{N+1,1}.
\end{aligned}
\end{equation}
According to the definition of the signal powers, we have
\begin{equation}
\begin{aligned}
\mathcal{S}(\epsilon_0)&=2\tau\kappa^2\beta\Big|\mathbb{H}[\epsilon_0]^{-1}_{N+1,1}-\mathbb{H}[0]^{-1}_{N+1,1}\Big|^2.
\end{aligned}
\end{equation}
It is clear that under (\textbf{C2}),  $\mathbb{H}[\epsilon_0]$ and  $\mathbb{H}[0]$ are the same as those of the  noise-free case; so is the
signal power accordingly.

The total average photon number $\bar{n}_{\textsf{tot}}$ can be calculated as
\begin{equation}
\begin{aligned}
\bar{n}_{\textsf{tot}}&=\frac{\bar{n}_{\textsf{tot}}(0)+\bar{n}_{\textsf{tot}}(\epsilon_0)}{2}\\
&=\frac{\sum_{n=1}^N\langle\hat{x}_n\rangle^2_0+\langle\hat{p}_n\rangle^2_0}{4}+\frac{\sum_{n=1}^N\langle\hat{x}_n\rangle^2_{\epsilon_0}+\langle\hat{p}_n\rangle^2_{\epsilon_0}}{4}\\
&=\frac{\kappa\beta^2\sum_{n=1}^{N}|(h^{\mathbb{X}})^{-1}_{n,1}\big|^2}{2}+\frac{\kappa\beta^2\sum_{n=1}^{N}\Big(\big|\mathbb{H}[\epsilon_0]^{-1}_{n,1}\big|^2+\big|\mathbb{H}[\epsilon_0]^{-1}_{N+n,1}\big|^2\Big)}{2}.
\end{aligned}
\end{equation}
Similar to the signal power, we can see that under (\textbf{C2}), the total average photon number is the same as that of the noise-free case.


We now calculate the noise power. Since $\epsilon_0$ is no longer infinitesimal, we have to compute all the orders of the output field with respect to $\epsilon_0$. According to the definition of $\hat{\mathcal{M}}$, we have
\begin{equation}
\begin{aligned}
&\hat{\mathcal{M}}\Big|_{\epsilon=\epsilon_0}-\langle\hat{\mathcal{M}}\rangle\Big|_{\epsilon=\epsilon_0}\\
=&\frac{1}{\sqrt{\tau}}\int^{\tau}_{0}\Bigg{\{}\kappa\mathbb{H}[\epsilon_0]^{-1}_{N+1,1}\hat{X}^{\textsf{in}}+\kappa\mathbb{H}[\epsilon_0]^{-1}_{N+1,N+1}\hat{P}^{\textsf{in}}+\hat{P}^{\textsf{in}}+\sum_{i=1}^{N}\mathbb{H}[\epsilon_0]^{-1}_{N+1,i}\sqrt{2\kappa}~\Big(\sum_{j=1}^{N_Y}Y_{i,j}\hat{C}^{\textsf{in}}_{j,X}+\sum_{j=1}^{N_Z}Z_{i,j}\hat{D}^{\textsf{in}}_{j,X}\Big)\\
&~~~~~+\sum_{i=1}^{N}\mathbb{H}[\epsilon_0]^{-1}_{N+1,N+i}\sqrt{2\kappa}~\Big(-\sum_{j=1}^{N_Y}Y_{i,j}\hat{C}^{\textsf{in}}_{j,P}+\sum_{j=1}^{N_Z}Z_{i,j}\hat{D}^{\textsf{in}}_{j,P}\Big)\Bigg{\}}dt.
\end{aligned}
\end{equation}
Then the noise power $\mathcal{N}(\epsilon_0)$ can be calculated as
\begin{equation}
\begin{aligned}
\mathcal{N}(\epsilon_0)&=\langle\hat{\mathcal{M}}^2\rangle_{\epsilon_0}-\langle\hat{\mathcal{M}}\rangle_{\epsilon_0}^2\\
&=\frac{1}{2}\Bigg\{\kappa^2(\mathbb{H}[\epsilon_0]^{-1}_{N+1,1})^2+(1+\kappa\mathbb{H}[\epsilon_0]^{-1}_{N+1,N+1})^2+2\kappa\Big[\mathbb{H}[\epsilon_0]^{-1}\begin{pmatrix}
                                                                                                                                                                    Y \\
                                                                                                                                                                    0
                                                                                                                                                                  \end{pmatrix}\cdot\begin{pmatrix}
                                                                                                                                                                                      Y^\top & 0
                                                                                                                                                                                    \end{pmatrix}\mathbb{H}[\epsilon_0]^{-1\top}\Big]_{N+1,N+1}\\
&~~~+2\kappa\Big[\mathbb{H}[\epsilon_0]^{-1}\begin{pmatrix}
                                                                                                                                                                    Z \\
                                                                                                                                                                    0
                                                                                                                                                                  \end{pmatrix}\cdot\begin{pmatrix}
                                                                                                                                                                                      Z^\top & 0
                                                                                                                                                                                    \end{pmatrix}\mathbb{H}[\epsilon_0]^{-1\top}\Big]_{N+1,N+1}
+2\kappa\Big[\mathbb{H}[\epsilon_0]^{-1}\begin{pmatrix}
                                                                                                                                                                    0 \\
                                                                                                                                                                    Y
                                                                                                                                                                  \end{pmatrix}\cdot\begin{pmatrix}
                                                                                                                                                                                     0 & Y^\top
                                                                                                                                                                                    \end{pmatrix}\mathbb{H}[\epsilon_0]^{-1\top}\Big]_{N+1,N+1}\\
&~~~+2\kappa\Big[\mathbb{H}[\epsilon_0]^{-1}\begin{pmatrix}
                                                                                                                                                                    0 \\
                                                                                                                                                                    Z
                                                                                                                                                                  \end{pmatrix}\cdot\begin{pmatrix}
                                                                                                                                                                                      0 & Z^\top
                                                                                                                                                                                    \end{pmatrix}\mathbb{H}[\epsilon_0]^{-1\top}\Big]_{N+1,N+1}
\Bigg\}\\
&=\frac{1}{2}\Bigg\{\!\kappa^2(\mathbb{H}[\epsilon_0]^{-1}_{\!N\!+\!1,1})^2\!+\!(1\!+\!\kappa\mathbb{H}[\epsilon_0]^{-1}_{\!N\!+\!1,N\!+\!1})^2\!+\!2\kappa\bigg[\mathbb{H}[\epsilon_0]^{-1}\!\begin{pmatrix}
                                                                                                                                                               YY^\top\!\!+\!ZZ^\top & 0 \\
                                                                                                                                                               0 & YY^\top\!\!+\!ZZ^\top
                                                                                                                                                             \end{pmatrix}\!\mathbb{H}[\epsilon_0]^{-1\top}\bigg]_{\!N\!+\!1,N\!+\!1}
\!\Bigg\}\\
&=\frac{1}{2}\Bigg\{\!\kappa^2(\mathbb{H}[\epsilon_0]^{-1}_{N\!+\!1,1})^2\!+\!(1\!+\!\kappa\mathbb{H}[\epsilon_0]^{-1}_{N\!+\!1,N\!+\!1})^2\!+\!4\kappa\bigg[\mathbb{H}[\epsilon_0]^{-1}\!\begin{pmatrix}
                                                                                                                                                               ZZ^\top & 0 \\
                                                                                                                                                               0 & ZZ^\top
                                                                                                                                                             \end{pmatrix}\!\mathbb{H}[\epsilon_0]^{-1\top}\bigg]_{N\!+\!1,N\!+\!1}
\!\Bigg\}.
\end{aligned}
\end{equation}
Combining with Eq.~\eqref{appendixC7}, the total noise power $\mathcal{N}$ beyond linear response reads
\begin{equation}
\begin{aligned}
\mathcal{N}=&\frac{\mathcal{N}(0)+\mathcal{N}(\epsilon_0)}{2}\\
=&\frac{1}{4}\Bigg\{\kappa^2(\mathbb{H}[\epsilon_0]^{-1}_{N+1,1})^2+(1+\kappa\mathbb{H}[\epsilon_0]^{-1}_{N+1,N+1})^2+(1+\kappa(h^{\mathbb{P}})^{-1}_{1,1})^2+4\kappa\big[(h^{\mathbb{P}})^{-1}ZZ^\top (h^{\mathbb{P}})^{-1\top}\big]_{1,1}\\
&+4\kappa\bigg[\mathbb{H}[\epsilon_0]^{-1}\begin{pmatrix}
                                                                                                                                                                     ZZ^\top & 0 \\
                                                                                                                                                                    0 & ZZ^\top
                                                                                                                                                                   \end{pmatrix}(\mathbb{H}[\epsilon_0]^{-1})^\top\bigg]_{N+1,N+1}\Bigg\}.
\end{aligned}
\end{equation}
Note that in the above equations the first three terms do not depend on the noise.  To regain the SNR per photon in the ideal case, it is necessary to design the loss structure $Z$ such that the total effect of the noise on the noise power $\mathcal{N}$ is canceled out. The last term in the above equation can be expressed as
\begin{equation}
\begin{aligned}
&\bigg[\mathbb{H}[\epsilon_0]^{-1}\begin{pmatrix}
                                                                                                                                                                     ZZ^\top & 0 \\
                                                                                                                                                                    0 & ZZ^\top
                                                                                                                                                                   \end{pmatrix}(\mathbb{H}[\epsilon_0]^{-1})^\top\bigg]_{N+1,N+1}\\
=&\sum_{j=1}^{N_Z}\Bigg(\sum_{i=1}^{N}\mathbb{H}[\epsilon_0]^{-1}_{N+1,i}Z_{i,j}\Bigg)^2+\sum_{j=1}^{N_Z}\Bigg(\sum_{i=1}^{N}\mathbb{H}[\epsilon_0]^{-1}_{N+1,N+i}Z_{i,j}\Bigg)^2\\
=&\sum_{j=1}^{N_Z}\Bigg(\sum_{i=1}^{N}-\frac{2\epsilon }{\kappa} h^{-1}_{N,i}\frac{1}{1+\epsilon^2\frac{4}{\kappa^2}}e^{A(2N-1-i)}Z_{i,j}\Bigg)^2+\sum_{j=1}^{N_Z}\Bigg( \sum_{i=1}^{N}\bigg(h^{-1}_{N,i}(\frac{1}{1+\epsilon^2\frac{4}{\kappa^2}}-1)+h^{-1}_{1,i}\bigg)e^{A(i-1)}Z_{i,j} \Bigg)^2\\
=&\frac{\frac{4\epsilon^2}{\kappa^2}}{1+\epsilon^2\frac{4}{\kappa^2}}e^{2A(N-1)}\sum_{j=1}^{N_Z}\sum^{N}_{i=1}\Bigg(\sum_{i=1}^{N}(h^{\mathbb{X}})^{-1}_{N,i}Z_{i,j}\Bigg)^2
+\Big(\frac{1}{1+\epsilon^2\frac{4}{\kappa^2}}-1\Big)^2e^{2A(N-1)}\sum_{j=1}^{N_Z}\Bigg(\sum_{i=1}^{N}(h^{\mathbb{P}})^{-1}_{N,i}Z_{i,j}\Bigg)^2\\
&+\sum^{N_Z}_{j=1}\Bigg(\sum_{i=1}^{N}(h^{\mathbb{P}})^{-1}_{1,i}Z_{i,j}\Bigg)^2+2\Big(\frac{1}{1+\epsilon^2\frac{4}{\kappa^2}}-1\Big)e^{A(N-1)}\sum_{j=1}^{N_Z}\Bigg(\sum_{i=1}^{N}(h^{\mathbb{P}})^{-1}_{N,i}Z_{i,j}\Bigg)\Bigg(\sum_{i=1}^{N}(h^{\mathbb{P}})^{-1}_{1,i}Z_{i,j}\Bigg)\\
=&\frac{\frac{4\epsilon^2}{\kappa^2}}{(1+\epsilon^2\frac{4}{\kappa^2})^2}e^{2A(N-1)}\Big[(h^{\mathbb{X}})^{-1}ZZ^\top (h^{\mathbb{X}})^{-1\top}\Big]_{N,N}+\Big(\frac{1}{1+\epsilon^2\frac{4}{\kappa^2}}-1\Big)^2e^{2A(N-1)}\Big[(h^{\mathbb{P}})^{-1}ZZ^\top (h^{\mathbb{P}})^{-1\top}\Big]_{N,N}\\
&+\Big[(h^{\mathbb{P}})^{-1}ZZ^\top (h^{\mathbb{P}})^{-1\top}\Big]_{1,1}+2\Big(\frac{1}{1+\epsilon^2\frac{4}{\kappa^2}}-1\Big)e^{A(N-1)}\Big[(h^{\mathbb{P}})^{-1}ZZ^\top (h^{\mathbb{P}})^{-1\top}\Big]_{N,1},
\end{aligned}
\end{equation}
where the details of calculating the elements of  $\mathbb{H}[\epsilon_0]^{-1}$ can be found in Section VIII.

It can be seen that under  (\textbf{C3}) and (\textbf{C4}),
\begin{equation}
\begin{aligned}
\Big[(h^{\mathbb{P}})^{-1}ZZ^\top(h^{\mathbb{P}})^{-1\top}\Big]_{1,1}&=\big[CC^\top\big]_{1,1}=0,\\
\Big[(h^{\mathbb{P}})^{-1}ZZ^\top(h^{\mathbb{P}})^{-1\top}\Big]_{N,N}&=\big[CC^\top\big]_{N,N}=0,\\
\Big[(h^{\mathbb{P}})^{-1}ZZ^\top(h^{\mathbb{P}})^{-1\top}\Big]_{N,1}&=\big[CC^\top\big]_{N,1}=0,\\
\Big[(h^{\mathbb{X}})^{-1}ZZ^\top(h^{\mathbb{X}})^{-1\top}\Big]_{N,N}&=0.
\end{aligned}
\end{equation}
Hence, under (\textbf{C2}), (\textbf{C3}) and (\textbf{C4}), the  noise power is the same as that of the ideal case.  Thus, we have regained the best sensitivity, which is the same as when there is no noise.


\section{Calculation of the elements of  $\mathbb{H}[\epsilon]^{-1}$}

In this section, we calculate elements of $\mathbb{H}[\epsilon]^{-1}$ when $\epsilon$ is infinitesimal. Using Dyson's equation and keeping it up to the first order in $\epsilon$, we have
\begin{equation}\label{HH1HN}
\begin{aligned}
\mathbb{H}[\epsilon]^{-1}=(\mathbb{H}_1[0] +\mathbb{H}_N[\epsilon])^{-1} = \mathbb{H}_1[0]^{-1}-\mathbb{H}_1[0]^{-1}\mathbb{H}_N
[\epsilon]\mathbb{H}_1[0]^{-1},
\end{aligned}
\end{equation}
where $\mathbb{H}_1[0]=\begin{pmatrix}
  h^{\mathbb{X}}+YY^\top-ZZ^\top &0 \\
  0 & h^{\mathbb{P}}+YY^\top-ZZ^\top  \\
 \end{pmatrix}$, and $\mathbb{H}_N
[\epsilon]=\epsilon|N\rangle\langle2N|-\epsilon|2N\rangle\langle N|$.

We only compute $\mathbb{H}[\epsilon]^{-1}_{N+1,1}$ as an illustration, and the other elements can be computed in a similar way.
Multiplying Eq.~\eqref{HH1HN} from the left by $\langle N+1|$ and from the right by $|1\rangle$ yields
\begin{equation}
\begin{aligned}
\mathbb{H}[\epsilon]^{-1}_{N+1,1}&= \mathbb{H}_1[0]^{-1}_{N+1,1}+\epsilon\mathbb{H}_1[0]^{-1}_{N+1,2N}\mathbb{H}_1[0]^{-1}_{N,1}\\
&=\epsilon (h^\mathbb{P}+YY^\top-ZZ^\top)^{-1}_{1,N}(h^\mathbb{X}+YY^\top-ZZ^\top)^{-1}_{N,1}\\
&=\epsilon Q^{\mathbb{X}}_{N,1}Q^{\mathbb{P}}_{1,N}.
\end{aligned}
\end{equation}


Under (\textbf{C2}), we only need to compute the elements of $(h^{\mathbb{X}})^{-1}$ and $(h^{\mathbb{P}})^{-1}$. Defining $T=\text{diag}\{1,e^A,e^{2A},\ldots,e^{A(N-1)}\}$, it can be verified that
\begin{equation}
\begin{aligned}
(h^\mathbb{X})^{-1}=Th^{-1}T^{-1},\\
(h^\mathbb{P})^{-1}=T^{-1}h^{-1}T,
\end{aligned}
\end{equation}
where
\begin{equation}
h=-\frac{\kappa}{2}|1\rangle\langle1|+\sum^{N-1}_{n=1}\Big(-J|n\rangle\langle n+1|+J|n+1\rangle\langle n|\Big).
\end{equation}
Then the elements of $(h^\mathbb{X})^{-1}$ are related to the elements of $h^{-1}$ as
\begin{equation}
\begin{aligned}
(h^\mathbb{X})^{-1}_{i,j}=h^{-1}_{i,j}e^{A(i-j)};
\end{aligned}
\end{equation}
while the elements of $(h^\mathbb{P})^{-1}$ relate to the elements of $h^{-1}$ as
\begin{equation}
\begin{aligned}
(h^\mathbb{P})^{-1}_{i,j}=h^{-1}_{i,j}e^{A(j-i)}.
\end{aligned}
\end{equation}
We now  introduce how to compute the elements of $h^{-1}$. According to the definition of $h$, we have
\begin{equation}
\begin{aligned}
\mathbb{I}=\left(J\sum^{N-1}_{n=1} \left(|n+1\rangle\langle n|-|n \rangle\langle n+1| \right)-\frac{\kappa}{2}|1\rangle\langle 1|\right)h^{-1},
\end{aligned}
\end{equation}
where $\mathbb{I}$ is the $N\times N$ identity matrix. Multiplying this equation from the left by $\langle i|$ and the right by $|1\rangle$, for $i=1, \cdots, N$, yields
\begin{equation}
\begin{aligned}
1=&-J\langle2|h^{-1}|1\rangle-\frac{\kappa}{2}\langle1|h^{-1}|1\rangle,\\
0=&(J\langle1|-J\langle3|)h^{-1}|1\rangle,\\
0=&(J\langle2|-J\langle4|)h^{-1}|1\rangle,\\
\vdots\\
0=&(J\langle N-2|-J\langle N|)h^{-1}|1\rangle,\\
0=&J\langle N-1|h^{-1}|1\rangle.
\end{aligned}
\end{equation}
Simplifying the above recursive formula, we have
\begin{equation}
\begin{aligned}
h^{-1}_{1,1}&=h^{-1}_{3,1}=\cdots=h^{-1}_{N,1}=-\frac{2}{\kappa},\\
h^{-1}_{2,1}&=h^{-1}_{4,1}=\cdots=h^{-1}_{N-1,1}=0.
\end{aligned}
\end{equation}
The other elements can be computed in a similar way.




\section{Calculation of the elements  of $\mathbb{H}[\epsilon_0]^{-1}$ }

In this section, we calculate the elements of $\mathbb{H}[\epsilon_0]^{-1}$ under (\textbf{C2}). Since the parameter  $\epsilon_0$ is not infinitesimal, we have to consider all orders of  $\epsilon_0$.

Define
\begin{equation}
\begin{aligned}
\tilde{\mathbb{H}}_1[0]&=\begin{pmatrix}
                           T & 0 \\
                           0 & T^{-1}
                         \end{pmatrix}^{-1}\mathbb{H}_1[0]\begin{pmatrix}
                           T & 0 \\
                           0 & T^{-1}
                         \end{pmatrix}=\begin{pmatrix}
                          h & 0 \\
                          0 & h
                        \end{pmatrix},\\
\tilde{\mathbb{H}}_N[\epsilon_0]&=\begin{pmatrix}
                           T & 0 \\
                           0 & T^{-1}
                         \end{pmatrix}^{-1}\mathbb{H}_N[\epsilon_0]\begin{pmatrix}
                           T & 0 \\
                           0 & T^{-1}
                         \end{pmatrix}=\epsilon_0e^{-2A(N-1)}|N\rangle\langle2N|-\epsilon_0e^{2A(N-1)}|2N\rangle\langle N|.
\end{aligned}
\end{equation}
 It can be seen that \begin{equation}
\begin{aligned}
\big(\mathbb{H}_1[0]+\mathbb{H}_N[\epsilon_0]\big)^{-1}_{i,j}=\big(\tilde{\mathbb{H}}_1[0]+\tilde{\mathbb{H}}_N[\epsilon_0]\big)^{-1}_{i,j}e^{A(i-j)},   ~\text{for}~ i,~j=1,\cdots, N.
\end{aligned}
\end{equation} The other elements have similar relationships. Thus, to calculate the elements of $\mathbb{H}[\epsilon_0]^{-1}=\big(\mathbb{H}_1[0]+\mathbb{H}_N[\epsilon_0]\big)^{-1}$, we can calculate the elements of $\big(\tilde{\mathbb{H}}_1[0]+\tilde{\mathbb{H}}_N[\epsilon_0]\big)^{-1}$. It can be verified that
\begin{equation}
\begin{aligned}
\big(\tilde{\mathbb{H}}_1[0]+\tilde{\mathbb{H}}_N[\epsilon_0]\big)^{-1}=&\sum_{n=1}\Big\{(-1)^n\epsilon_0^{2n}\tilde{\mathbb{H}}_1[0]^{-1}|N\rangle(h^{-1}_{N,N})^{2n-1}\langle N|\tilde{\mathbb{H}}_1[0]^{-1}\\
&+(-1)^n\epsilon_0^{2n}\tilde{\mathbb{H}}_1[0]^{-1}|2N\rangle(h^{-1}_{N,N})^{2n-1}\langle 2N|\tilde{\mathbb{H}}_1[0]^{-1}\Big\}+\tilde{\mathbb{H}}_1[0]^{-1}\\
&+\sum_{n=0}\Big\{(-1)^{n+1}\epsilon_0^{2n+1}\tilde{\mathbb{H}}_1[0]^{-1}|N\rangle(h^{-1}_{N,N})^{2n}\langle 2N|\tilde{\mathbb{H}}_1[0]^{-1}e^{-2A(N-1)}\\
&+(-1)^n\epsilon_0^{2n+1}\tilde{\mathbb{H}}_1[0]^{-1}|2N\rangle(h^{-1}_{N,N})^{2n}\langle N|\tilde{\mathbb{H}}_1[0]^{-1}e^{2A(N-1)}\Big\}.
\end{aligned}
\end{equation}
Multiplying this equation from the left by $\langle 1|$ and the right by $|1\rangle$, yields
\begin{equation}
\begin{aligned}
\big(\tilde{\mathbb{H}}_1[0]+\tilde{\mathbb{H}}_N[\epsilon_0]\big)^{-1}_{1,1}=\sum_{n=1}(-1)^n\epsilon_0^{2n}(-\frac{2}{\kappa})^{2n+1}+(-\frac{2}{\kappa})=-\frac{2}{\kappa}\cdot\frac{1}{1+\epsilon_0^2\frac{4}{\kappa^2}},
\end{aligned}
\end{equation}
and then \begin{equation}\big(\mathbb{H}_1[0]+\mathbb{H}_N[\epsilon_0]\big)^{-1}_{1,1}=\big(\tilde{\mathbb{H}}_1[0]+\tilde{\mathbb{H}}_N[\epsilon_0]\big)^{-1}_{1,1}e^{A(1-1)}=-\frac{2}{\kappa}\cdot\frac{1}{1+\epsilon_0^2\frac{4}{\kappa^2}}.\end{equation}
The other elements can be computed in a similar way.

\end{widetext}


\begin{thebibliography}{1}

%1-15



\bibitem{Peng2014}
B. Peng, $\c{\textrm{S}}$. K. $\ddot{\textrm{O}}$zdemir, F. Lei, \textit{et al.}, Parity–time-symmetric whispering-gallery microcavities, \href{https://www.nature.com/articles/nphys2927}{Nat. Phys. \textbf{10}, 394-398 (2014).}

\bibitem{ZhangJing2018}
J. Zhang, B. Peng, $\c{\textrm{S}}$. K. $\ddot{\textrm{O}}$zdemir, \textit{et al.}, A phonon laser operating at an exceptional point, \href{https://www.nature.com/articles/s41566-018-0213-5/#citeas}{Nat. Photon. \textbf{12}, 479-484 (2018).}

\bibitem{El-Ganainy2018}
R. El-Ganainy, K. G. Makris, M. Khajavikhan, Z. H. Musslimani, S. Rotter and D. N. Christodoulides, Non-Hermitian physics and PT symmetry, \href{https://www.nature.com/articles/nphys4323#citeas}{Nat. Phys. \textbf{14}, 11-19 (2018).}



\bibitem{Bliokh2019}
K. Bliokh, D. Leykam, M. Lein, \textit{et al.}, Topological non-Hermitian origin of surface Maxwell waves,  \href{https://www.nature.com/articles/s41467-019-08397-6#citeas}{  Nat. Commun. \textbf{10}, 580 (2019).}


\bibitem{ma2011}
J. Ma, X. Wang, C. P. Sun, \textit{et al.}, Quantum spin squeezing, \href{https://www.sciencedirect.com/science/article/abs/pii/S0370157311002201}{Phys. Rep. \textbf{509}, 89-165 (2011).}

\bibitem{Liu2019}
T. Liu, Y. R. Zhang, Q. Ai, \textit{et al.},
Second-order topological phases in non-Hermitian systems,
\href{https://journals.aps.org/prl/abstract/10.1103/PhysRevLett.122.076801}{ Phys. Rev. Lett. \textbf{122}, 076801 (2019).}

\bibitem{Chu2020}
Y. Chu, Y. Liu, H. Liu and J. Cai, Quantum sensing with a single-qubit pseudo-Hermitian system, \href{https://journals.aps.org/prl/abstract/10.1103/PhysRevLett.124.020501}{Phys. Rev. Lett. \textbf{124}, 020501 (2020).}

%\bibitem{Sun2020}
%L. Sun, X. He, C. You, C. Lv, B. Li, S. Lloyd and X. Wang, Exponential enhancement of quantum metrology using continuous variables, arXiv:2004.01216 (2020).











%\bibitem{Pan2020}
%L. Pan, X. Chen, Y. Chen and H. Zhai, Non-Hermitian linear response theory, \href{https://www.nature.com/articles/s41567-020-0889-6}{Nat. Phys. \textbf{16}, 767-771 (2020).}

%\bibitem{Cao2020}
%W. Cao, X. Lu, X. Meng, J. Sun, H. Shen and Y. Xiao, Reservoir-mediated quantum correlations in non-Hermitian optical system, \href{https://journals.aps.org/prl/abstract/10.1103/PhysRevLett.124.030401}{Phys. Rev. Lett. \textbf{124}, 030401 (2020).}


%\bibitem{Roberts2021}
%D. Roberts, A. Lingenfelter and A. A. Clerk, Hidden time-reversal symmetry, quantum detailed balance and exact solutions of driven-dissipative quantum systems, \href{https://journals.aps.org/prxquantum/abstract/10.1103/PRXQuantum.2.020336}{PRX Quantum \textbf{2}, 020336 (2021).}

\bibitem{G.-Q.Zhang2021}
G.-Q. Zhang, Z. Chen, D. Xu, \textit{et al.}, Exceptional point and cross-relaxation effect in a hybrid quantum system, \href{https://journals.aps.org/prxquantum/abstract/10.1103/PRXQuantum.2.020307}{PRX Quantum \textbf{2}, 020307 (2021).}




%\bibitem{Xiao2021}
%L. Xiao, D. Qu, K. Wang, H.-W. Li, J.-Y. Dai, B. D$\acute{\textrm{o}}$ra, M. Heyl, R. Moessner, W. Yi and P. Xue, Non-Hermitian Kibble-Zurek Mechanism with tunable complexity in single-photon interferometry, \href{https://journals.aps.org/prxquantum/abstract/10.1103/PRXQuantum.2.020313}{PRX Quantum \textbf{2}, 020313 (2021).}

\bibitem{Bensa2021}
J. Bensa and M. $\breve{\textrm{Z}}$nidari$\breve{\textrm{c}}$, Fastest local entanglement scrambler, multistage thermalization, and a non-Hermitian phantom, \href{https://journals.aps.org/prx/abstract/10.1103/PhysRevX.11.031019}{Phys. Rev. X \textbf{11}, 031019 (2021).}


%\bibitem{Rao2021}
%J. Rao, Y. Zhao, Y. Gui, X. Fan, D. Xue and C.-M. Hu, Controlling microwaves in non-Hermitian metamaterials, \href{https://journals.aps.org/prapplied/abstract/10.1103/PhysRevApplied.15.L021003}{Phys. Rev. Applied \textbf{15}, L021003 (2021).}





\bibitem{Lee2023}
K. Y. Lee, J. D. Lin, A. Miranowicz, \textit{et al.}, Steering-enhanced quantum metrology using superpositions of noisy phase shifts, \href{https://journals.aps.org/prresearch/abstract/10.1103/PhysRevResearch.5.013103}{Phys. Rev. Research \textbf{5}, 013103 (2023).}


\bibitem{Okuma2020}
N. Okuma, K. Kawabata, K. Shiozaki and M. Sato, Topological origin of non-Hermitian skin effects, \href{https://journals.aps.org/prl/abstract/10.1103/PhysRevLett.124.086801}{Phys. Rev. Lett. \textbf{124}, 086801 (2020).}

\bibitem{Scheibner2020}
C. Scheibner, W. T. M. Irvine and V. Vitelli, Non-Hermitian band topology and skin modes in active elastic media, \href{https://journals.aps.org/prl/abstract/10.1103/PhysRevLett.125.118001}
{Phys. Rev. Lett. \textbf{125}, 118001 (2020).}

\bibitem{Kawabata2020}
K. Kawabata, N. Okuma and M. Sato, Non-Bloch band theory of non-Hermitian Hamiltonians in the symplectic class, \href{https://journals.aps.org/prb/abstract/10.1103/PhysRevB.101.195147}{Phys. Rev. B \textbf{101}, 195147 (2020).}

\bibitem{Leykam2017}
D. Leykam, K. Y. Bliokh, C. Huang, \textit{et al.}, Edge modes, degeneracies, and topological numbers in non-Hermitian systems,  \href{https://journals.aps.org/prl/abstract/10.1103/PhysRevLett.118.040401}{ Phys. Rev. Lett. \textbf{118}, 040401 (2017).}
%1-15



%
\bibitem{Park2021}
S. Park, J. Lee and S. Kim, Robust wireless power transfer with minimal field exposure using parity-time symmetric microwave cavities, \href{https://journals.aps.org/prapplied/abstract/10.1103/PhysRevApplied.16.014022}{Phys. Rev. Applied \textbf{16}, 014022 (2021).}

\bibitem{Chen2021}
Y. Chen, W. Qin, X. Wang, \textit{et al.}, Shortcuts to adiabaticity for the quantum Rabi model: efficient generation of giant entangled cat states via parametric amplification, \href{https://journals.aps.org/prl/abstract/10.1103/PhysRevLett.126.023602}{Phys. Rev. Lett. \textbf{126}, 023602 (2021). }

\bibitem{xu2022}
K. Xu, Y. R. Zhang, Z. H. Sun, \textit{et al.}, Metrological characterization of non-Gaussian entangled states of superconducting qubits, \href{https://journals.aps.org/prl/abstract/10.1103/PhysRevLett.128.150501}{Phys. Rev. Lett. \textbf{128}, 150501 (2022).}

%\bibitem{Kacprowicz2010}
%M. Kacprowicz, R. Demkowicz-Dobrza$\acute{n}$ski, W. Wasilewski, K. Banaszek and I. A. Walmsley, Experimental quantum-enhanced estimation of a lossy phase shift, \href{https://www.nature.com/articles/nphoton.2010.39}{Nat. Photon. \textbf{4}, 357-360 (2010).}







\bibitem{Wiersig2014}
J. Wiersig, Enhancing the sensitivity of frequency and energy splitting detection by using exceptional points: application to microcavity sensors for
single-particle detection, \href{https://journals.aps.org/prl/abstract/10.1103/PhysRevLett.112.203901}{ Phys. Rev. Lett. \textbf{112}, 203901 (2014).}



\bibitem{Wiersig2016}
J. Wiersig, Sensors operating at exceptional points: general theory, \href{https://journals.aps.org/pra/abstract/10.1103/PhysRevA.93.033809}{ Phys. Rev. A \textbf{93}, 033809 (2016).}



\bibitem{Chen2019}
C. Chen, L. Jin and R.-B. Liu, Sensitivity of parameter estimation near the exceptional point of a non-Hermitian system, \href{https://iopscience.iop.org/article/10.1088/1367-2630/ab32ab/meta}{ New J. Phys. \textbf{21}, 083002 (2019).}

\bibitem{Zhang2019}
M. Zhang, W. Sweeney, C. W. Hsu, L. Yang, A. D. Stone and L. Jiang, Quantum noise theory of exceptional point amplifying sensors, \href{https://journals.aps.org/prl/abstract/10.1103/PhysRevLett.123.180501}{ Phys. Rev. Lett. \textbf{123}, 180501 (2019).}

\bibitem{ozdemir2019}
$\c{\textrm{S}}$. K. $\ddot{\textrm{O}}$zdemir, S. Rotter, F. Nori, \textit{et al.}, Parity–time symmetry and exceptional points in photonics, \href{https://www.nature.com/articles/s41563-019-0304-9#citeas}{Nat. Mater. \textbf{18}, 783-798 (2019).}















\bibitem{Rui2019}
W. Rui, Y. Zhao and A. Schnyder, Topology and exceptional points of massive Dirac models with generic non-Hermitian perturbations, \href{https://journals.aps.org/prb/abstract/10.1103/PhysRevB.99.241110}{Phys. Rev. B \textbf{99}, 241110 (2019).}





\bibitem{WangGao2020}
X. Wang, G. Guo and J. Berakdar, Steering magnonic dynamics and permeability at exceptional points in a parityšCtime symmetric waveguide, \href{https://www.nature.com/articles/s41467-020-19431-3}{Nat. Commun. \textbf{11}, 5663 (2020).}







\bibitem{Pap2021}
E. J. Pap, D. Boer and H. Waalkens, A unified view on geometric phases and exceptional points in adiabatic quantum mechanics, \href{https://www.emis.de/journals/SIGMA/2022/003/}{SIGMA. \textbf{18}, 003 (2022).}

\bibitem{Jiang2022}
H. Jiang, W. Zhang, G. Lu, L. Ye, H. Lin, J. Tang, Z. Xue, Z. Li, H. Xu and Q. Gong, Exceptional points and enhanced nanoscale sensing with a plasmon-exciton hybrid system, \href{https://opg.optica.org/prj/fulltext.cfm?uri=prj-10-2-557&id=469085}{Photon. Res. \textbf{10}, 557-563 (2022).}



\bibitem{Lau2018}
H.-K. Lau and A. A. Clerk, Fundamental limits and non-reciprocal approaches in non-{H}ermitian quantum sensing, \href{https://www.nature.com/articles/s41467-018-06477-7#citeas}{  Nat. Commun. \textbf{9}, 4320 (2018).}

\bibitem{Bao2021}
L. Bao, B. Qi, D. Dong, \textit{et al.}, Fundamental limits for reciprocal and nonreciprocal non-Hermitian quantum sensing, \href{https://journals.aps.org/pra/abstract/10.1103/PhysRevA.103.042418}{ Phys. Rev. A \textbf{103}, 042418 (2021).}

\bibitem{Tzuang2014}
L. D. Tzuang, K. Fang, P. Nussenzveig, S. Fan and M. Lipson, Non-reciprocal phase shift induced by an effective magnetic flux for light, \href{https://www.nature.com/articles/nphoton.2014.177}{Nat. Photon. \textbf{8}, 701-705 (2014).}

\bibitem{Sounas2017}
D. L. Sounas and A. Al\`{u}, Non-reciprocal photonics based on time modulation, \href{https://www.nature.com/articles/s41566-017-0051-x}{ Nat. Photon. \textbf{11}, 774-783 (2017).}

\bibitem{Lai2020}
D. G. Lai, J. F. Huang, X. L. Yin, \textit{et al.}, Nonreciprocal ground-state cooling of multiple mechanical resonators,\href{https://journals.aps.org/pra/abstract/10.1103/PhysRevA.102.011502}{
Phys. Rev. A \textbf{102}, 011502 (2020).}

\bibitem{Tang2022}
L. Tang, J. Tang, M. Chen, \textit{et al.}, Quantum squeezing induced optical nonreciprocity, \href{https://journals.aps.org/prl/abstract/10.1103/PhysRevLett.128.083604}{Phys. Rev. Lett. \textbf{128}, 083604 (2022).}

\bibitem{TangJ2022}
J. S. Tang, W. Nie, L. Tang, \textit{et al.}, Nonreciprocal single-photon band structure, \href{https://journals.aps.org/prl/abstract/10.1103/PhysRevLett.128.203602}{Phys. Rev. Lett. \textbf{128}, 203602 (2022).}

\bibitem{Qin2018}
W. Qin, A. Miranowicz, P. B. Li, \textit{et al.}, Exponentially enhanced light-matter interaction, cooperativities, and steady-state entanglement using parametric amplification, \href{https://journals.aps.org/prl/abstract/10.1103/PhysRevLett.120.093601}{Phys. Rev. Lett. \textbf{120}, 093601 (2018).}

\bibitem{McDonald2020}
A. McDonald and A. A. Clerk, Exponentially-enhanced quantum sensing with non-Hermitian lattice dynamics, \href{https://www.nature.com/articles/s41467-020-19090-4}{ Nat. Commun. \textbf{11}, 5382 (2020).}

\bibitem{Budich2020}
J. C. Budich and E. J. Bergholtz, Non-Hermitian topological sensors, \href{https://journals.aps.org/prl/abstract/10.1103/PhysRevLett.125.180403}{Phys. Rev. Lett. \textbf{125}, 180403 (2020).}



\bibitem{Koch2021}
F. Koch and J. C. Budich, Quantum non-Hermitian topological sensors, \href{https://journals.aps.org/prresearch/abstract/10.1103/PhysRevResearch.4.013113}{Phys. Rev. Research \textbf{4}, 013113 (2022).}







\bibitem{Baoli2021}
L. Bao, B. Qi and D. Dong, Exponentially enhanced quantum non-Hermitian sensing via optimized coherent drive, \href{https://journals.aps.org/prapplied/abstract/10.1103/PhysRevApplied.17.014034}{ Phys. Rev. Applied \textbf{17}, 014034 (2022).}




\bibitem{Zhou2018}
S. Zhou, M. Zhang, J. Preskill and L. Jiang, Achieving the Heisenberg limit in quantum metrology using quantum error correction, \href{https://www.nature.com/articles/s41467-017-02510-3}{
Nat. Commun. \textbf{9}, 78 (2018).}



\bibitem{Escher2011}
B. M. Escher, R. L. de Matos Filho and L. Davidovich, General framework for estimating the ultimate precision limit in noisy quantum-enhanced metrology, \href{https://www.nature.com/articles/nphys1958}{Nat. Phys. \textbf{7}, 406-411 (2011).}



\bibitem{Demkowicz2012}
R. Demkowicz-Dobrza$\acute{\textrm{n}}$ski, J. Kolody$\acute{\textrm{n}}$ski and M. Gu\c{t}\u{a}, The elusive Heisenberg limit in quantum-enhanced metrology, \href{https://www.nature.com/articles/ncomms2067}{Nat. Commun. \textbf{3}, 1063 (2012).}

\bibitem{Datta2011}
A. Datta, L. Zhang, N. Thomas-Peter, U. Dorner, B. J. Smith and I. A. Walmsley, Quantum metrology with imperfect states and detectors, \href{https://journals.aps.org/pra/abstract/10.1103/PhysRevA.83.063836}{Phys. Rev. \textbf{A} 83, 063836 (2011).}

\bibitem{Giovannetti2006}
V. Giovannetti, S. Lloyd and L. Maccone, Quantum metrology, \href{https://journals.aps.org/prl/abstract/10.1103/PhysRevLett.96.010401}{Phys. Rev. Lett. \textbf{96}, 010401 (2006).}

\bibitem{Napolitano2011}
M. Napolitano, M. Koschorreck, B. Dubost, N. Behbood, R. J. Sewell and M. W. Mitchell, Interaction-based quantum metrology showing scaling beyond the Heisenberg limit, \href{https://www.nature.com/articles/nature09778}{Nature \textbf{471}, 486-489 (2011).}


\bibitem{Huelga1997}
S. F. Huelga, C. Macchiavello, T. Pellizzari, A. K. Ekert, M. B. Plenio and J. I. Cirac, Improvement of frequency standards with quantum entanglement, \href{https://journals.aps.org/prl/abstract/10.1103/PhysRevLett.79.3865}{Phys. Rev. Lett. \textbf{79}, 3865 (1997).}

\bibitem{Hou2019}
Z. Hou, R.-J. Wang, J.-F. Tang, H. Yuan, G. Xiang, C. Li and G.-C. Guo, Control-enhanced sequential scheme for general quantum parameter estimation at the Heisenberg limit, \href{https://journals.aps.org/prl/abstract/10.1103/PhysRevLett.123.040501}{Phys. Rev. Lett. \textbf{123}, 040501 (2019).}

\bibitem{Hou2020}
Z. Hou, Z. Zhang, G.-Y. Xiang, C. Li, G. C. Guo, H. Chen, L. Liu and H. Yuan, Minimal tradeoff and ultimate precision limit of multiparameter quantum magnetometry under the parallel scheme, \href{https://journals.aps.org/prl/abstract/10.1103/PhysRevLett.125.020501}{ Phys. Rev. Lett. \textbf{125}, 020501 (2020).}

\bibitem{Yuan2015}
H. Yuan and C.-H. F. Fung, Optimal feedback scheme and universal time scaling for Hamiltonian parameter estimation, \href{https://journals.aps.org/prl/abstract/10.1103/PhysRevLett.115.110401}{Phys.
Rev. Lett. \textbf{115}, 110401 (2015).}

\bibitem{Thomas2011}
N. Thomas-Peter, B. J. Smith, A. Datta, L. Zhang, U. Dorner and I. A. Walmsley, Real-world quantum sensors: evaluating resources for precision measurement, \href{https://journals.aps.org/prl/abstract/10.1103/PhysRevLett.107.113603}{Phys. Rev. Lett. \textbf{107}, 113603 (2011).}






%36-40


















%16-25



%26-30









%41



\bibitem{Feng2014}
L. Feng, Z. J. Wong, R. M. Ma, Y. Wang and X. Zhang, Single-mode laser by parity-time symmetry breaking, \href{https://www.science.org/doi/full/10.1126/science.1258479}{Science \textbf{346}, 972-975 (2014).}





\bibitem{Liuyl2017}
Y.-L. Liu, R. Wu, J. Zhang, \textit{et al.}, Controllable optical response by modifying the gain and loss of a mechanical resonator and cavity mode in an optomechanical system, \href{https://journals.aps.org/pra/abstract/10.1103/PhysRevA.95.013843}{Phys. Rev. A \textbf{95}, 013843 (2017).}

\bibitem{Ren2022}
Z. Ren, D. Liu, E. Zhao, C. He, K. Pak, J. Li and Gyu-Boong Jo, Chiral control of quantum states in non-Hermitian spin–orbit-coupled fermions, \href{https://www.nature.com/articles/s41567-021-01491-x}{Nat. Phys. \textbf{16}, 5 (2022).}







%\bibitem{Wiseman2010}
%H. M. Wiseman and G. J. Milburn, \emph{Quantum Measurement and Control} (Cambridge University Press, U.K., 2010).

\bibitem{Moiseyev2011}
N. Moiseyev, \emph{Non-Hermitian Quantum Mechanics} (Cambridge University Press, Cambridge, England, 2011).

%\bibitem{Pinel2013}
%O. Pinel, P. Jian, N. Treps, C. Fabre and D. Braun, Quantum parameter
%estimation using general single-mode Gaussian states, \href{https://journals.aps.org/pra/abstract/10.1103/PhysRevA.88.040102}{ Phys. Rev. A \textbf{88}, 040102 (2013).}

%\bibitem{Banchi2015}
%L. Banchi, S. L. Braunstein and S. Pirandola, Quantum fidelity for arbitrary Gaussian states, \href{https://journals.aps.org/prl/abstract/10.1103/PhysRevLett.115.260501}{ Phys. Rev. Lett. \textbf{115}, 260501 (2015).}






\bibitem{Clerk2010}
A. A. Clerk, M. H. Devoret, S. M. Girvin, F. Marquardt and R. J. Schoelkopf, Introduction to quantum noise, measurement, and amplification, \href{https://journals.aps.org/rmp/abstract/10.1103/RevModPhys.82.1155}{ Rev. Mod. Phys. \textbf{82}, 1155-1208 (2010).}




\bibitem{SM}
Supplementary Material: Exponential sensitivity revival and robust stability of noisy  non-Hermitian quantum sensing.

\bibitem{Hatano1996}
N. Hatano and D. R. Nelson, Localization transitions
in non-Hermitian quantum mechanics, \href{https://journals.aps.org/prl/abstract/10.1103/PhysRevLett.77.570}{Phys. Rev. Lett. \textbf{77}, 570–573 (1996).}

\bibitem{Hatano1997}
N. Hatano and D. R. Nelson, Vortex pinning and non-Hermitian quantum mechanics, \href{https://journals.aps.org/prb/abstract/10.1103/PhysRevB.56.8651}{Phys. Rev. B \textbf{56}, 8651 (1997).}

\bibitem{Gardiner2000}
C. W. Gardiner and P. Zoller, \textit{Quantum noise: A handbook of Markovian and non-Markovian quantum stochastic methods with applications to quantum optics, 2nd ed}. (Springer-Verlag, Berlin, 2000).





%\bibitem{Liu2017}
%J. Liu and H. Yuan, Quantum parameter estimation with optimal
%control, \href{https://journals.aps.org/pra/abstract/10.1103/PhysRevA.96.012117}{Phys. Rev. A \textbf{96}, 012117 (2017).}


%\bibitem{OH2019}
%C. Oh, C. Lee, C. Rockstuhl, H. Jeong, J. Kim, H. Nha and S. Lee, Optimal Gaussian measurements for phase estimation in single-mode Gaussian metrology, \href{https://www.nature.com/articles/s41534-019-0124-4#citeas}{npj Quantum Inf. \textbf{5}, 10 (2019).}


\bibitem{Franklin2019}
G. F. Franklin, J. D. Powell and A. Emami-Naeini, \emph{Feedback Control of Dynamic Systems} (Pearson, New York, 2019).



\end{thebibliography}

\end{document}
%
% ****** End of file apssamp.tex ******
