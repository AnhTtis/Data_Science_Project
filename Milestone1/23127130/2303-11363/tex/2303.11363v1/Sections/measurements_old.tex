\begin{figure}[!t]
    \centering
    %\includegraphics[width=0.49\linewidth]{Figures/Lambda0.pdf}
    %\hfill
    \includegraphics[width=0.75\linewidth]{Figures/proton.pdf}
    \caption{%{\it Left:} Comparison between the different measurements of the proton scaled momentum ($x_p = |p|/|p_{\rm beam}|$). Here, we show the results from \textsc{Aleph} (red), \textsc{Delphi} (blue and green), \textsc{Opal} (purple), and \textsc{Sld} (orange and cyan). 
    Comparison between the different measurements of the proton scaled momentum ($x_p = |p|/|p_{\rm beam}|$). Here, we show the results from \textsc{Aleph} (red), \textsc{Delphi} (blue), \textsc{Opal} (green), and \textsc{Sld} (orange). In both panels, the errors correspond to statistical and systematic uncertainties summed up in quadrature. Data is taken from \cite{Barate:1996fi, Akers:1994ez, Abreu:1995cu, Abreu:1998vq, Abe:1998zs, Abe:2003iy}.}
    \label{fig:comparisons}
\end{figure}    

%%%%%%%%%%%%%%%%%%%%%%%%%%%%%%%%%%%%%%%%%%%%%%%%%%%%%%%%%%%
\section{Experimental measurements}
\label{sec:measurements}
%%%%%%%%%%%%%%%%%%%%%%%%%%%%%%%%%%%%%%%%%%%%%%%%%%%%%%%%%%%


There are some inconsistencies between the \textsc{Opal} and the other experiments in the measurement of the scaled momentum of $p/\bar{p}$. This issue occurs for $x_p = |p|/|p|_{\rm beam} > 0.1$ as can be seen Fig. \ref{fig:comparisons}. We discuss, below, the methods used by the different collaborations to measure the (scaled)-momentum of $p/\bar{p}$ and $\Lambda/\bar{\Lambda}$.

%%%%%%%%%%%%%%%%%%%%%%%%%%%%%%%%%%%%%%%%%%%%%%
\subsection{ALEPH 1996--2000}
%%%%%%%%%%%%%%%%%%%%%%%%%%%%%%%%%%%%%%%%%%%%%%

\paragraph{{\bf ALEPH\_1996\_S3486095} \cite{Barate:1996fi}.} The \textsc{Aleph} collaboration has carried a systematic analysis of QCD at the $Z$-boson pole including measurements of meson and baryon spectra. More details about the analysis strategy/systematic erros and unfolding can be found in \cite{Buskulic:1994ft} for $p/\bar{p}$ and in \cite{Buskulic:1994ny} for $\Lambda/\bar{\Lambda}$. The measurement of inclusive identified particle spectra is based on $520000$ events recorded by the \textsc{Aleph} detector in 1992. Among the identified particle spectra, those corresponding to $p/\bar{p}$ and $\Lambda$ have been measured for kinematics ranges of $x_p \in [1.2 \times 10^{-2},~0.9]$ for $\Lambda$ and $x_p \in [1.0 \times 10^{-2},~0.8]$ for $p/\bar{p}$ ($x_p$ is defined by $x_p = |p_{\rm hadron}|/|p_{\rm beam}|$). The differential rates of stable hadrons -- $\pi^\pm, K^\pm,~{\rm and}~p/\bar{p}$ -- are determined by a simultaneous measurement of their momentum $p$ and ionization energy loss (${\rm d}E/{\rm d}x$) in the Time Projection Chamber (TPC). On the other hand, the differential cross sections for $\Lambda$ are measured from their decays into $p\pi$. Events are selected if they contain at least five tracks with a total energy of at least $20\%$ of the total center-of-mass energy. Further requirements on the polar angles of the selected tracks and their transverse momenta were imposed to improve the momentum resolution. Since the bands for the energy loss of different species ($\pi^\pm, K^\pm,~{\rm and}~p/\bar{p}$) overlap in the in the regions $x_p \in [1.8, 2.4] \times 10^{-2}$ and $x_p \in [2.8, 7] \times 10^{-2}$, the rates of protons can not be determined in these regions. Correction factors have been applied to the measured rates of identified particles, {\it i.e.} $X_{\rm corrected} = C \cdot X_{\rm measured}$ with $C$ being the correction factor which is defined as 
\begin{eqnarray}
C =\frac{X_{\rm generator}}{X_{\rm MC+detector}},
\end{eqnarray}
where $X_{\rm generator}$ is estimated from generated Monte Carlo (MC) samples with the QED initial state radiation switched off and all the particles with mean lifetime of $\langle \tau \rangle < 10^{-9}$ seconds are forced to decay. $X_{\rm MC+detector}$ was estimated using \textsc{Hvfl} \cite{Barate:2000ab}, the hard-scattering matrix elements and the initial state radiation was generated using \textsc{Dymu} \cite{Campagne:1988fj} and passed to \textsc{JetSet} version 7.3 \cite{Sjostrand:1993yb} for parton-showering and hadronisation. These MC samples were passed to a dedicated detector simulation program.

A few comments about the corrections are in order here:
\begin{itemize}
    \item The effects of the individual selection cuts do not agree in MC and data for the distribution of the numbers of samples per track (MC predicts fewer number of wire measurements per track than data). This has been corrected by applying a momentum-dependent adjustment on the cut on the number of wire measurements per track. This implies a correction that varies smoothly from $2\%$ for small $x_p$ to $10\%$ in the highest bins of $x_p$ without inducing a bias on the relative proportions of hadrons.
    \item The reconstructed momentum of protons and kaons are lower than the original momentum in the $1/\beta^2$ region (small $x_p$). A correction has been applied to proton momentum for $x_p < 0.018$ by exploiting a study of the mitigation to low momentum using MC.  
    \item The acceptance after the selection cuts is around $50\%$ and decreases to about $35\%$ in the high $x_p$ region (due to overlapping). For $x_p < 7 \times 10^{-3}$, the acceptance drops dramatically to $10\%$ due to the high ionization. 
    \item The analysis of the systematic uncertainties on $\Lambda$ spectrum has been studied in details by \textsc{Aleph} \cite{Buskulic:1994ny}. The uncertainties can be categorised into three components: track reconstruction efficiencies, efficiencies of the $\Lambda^0$ reconstruction and background calibration. Most of these features induce quite small errors ranging from $0.4\%$ for track efficiencies, $1\%$--$4\%$ for $\Lambda^0$ finding, an overall renormalization error of $3\%$ for $\Lambda^0$, and background calibration errors of $0.6\% \oplus 1.2\% \oplus 4\%$.
\end{itemize}

From the above, one can conclude that in the process of tuning, one may need to remove the bins of $x_p < 0.018$ in the proton spectrum and (possibly) give a smaller weight to the higher energy bins of that same distribution. For $\Lambda^0$ spectrum, the agreement between \textsc{JetSet} and data was found to be good, the experimental errors are not large, and therefore the correction factors should not be large. We may keep the $\Lambda^0$ spectrum without removing any bins.

\paragraph{{\bf ALEPH\_2000\_I507531} \cite{Barate:1999gb}.} Using $3.7$ million hadronic events at LEP, the \textsc{Aleph} collaboration has performed a measurement of $\Lambda$ spectrum in the all-hadronic, $2$-jet and $3$-jet events. This analysis was done for similar selection cuts as in a previous \textsc{Aleph} measurement \cite{Barate:1996fi, Buskulic:1994ny}. The results were compared to the predictions of \textsc{JetSet} version 7.4 \cite{Sjostrand:1993yb}, \textsc{Ariadne} version 4.08 \cite{Lonnblad:1992tz} and \textsc{Herwig} version 5.8 \cite{Marchesini:1991ch}. The $\Lambda/\bar{\Lambda}$ baryons are reconstructed from their decay into a proton and a charged pion with a branching ratio of $63.9\%$. All pairs of oppositely charged tracks in an event are tested for the hypothesis that they originate from a common secondary vertex and their measured specific ionizations are required to be consistent with those expected for the decay particles (same as in \cite{Barate:1996fi, Buskulic:1994ny}). The invariant masses of the pairs of charged tracks are calculated in intervals of the scaled momentum $x_p$. In each interval the invariant mass distribution is fitted with the sum of a signal function and a back- ground function: a Breit-Wigner distribution is used for the $\bar{\Lambda}$ while the background is described by a linear function. The reconstruction efficiency has a strong dependence on the $\Lambda/\bar{\Lambda}$ momentum. Typically, it is $50\%$ for $\Lambda \to p\pi^-$ at $8$ GeV/c and drops below $30\%$ for momenta smaller than $1.5$ GeV/c or greater than $15$ GeV/c. The systematic errors for $\Lambda$ are dominated by the $\Lambda$ selection, the choice of signal- and background functions and the errors on the $\Lambda$ branching ratios. The systematic errors due to $\Lambda$ selection are studied by successively varying the cuts on: the specific ionization ${\rm d}E /{\rm d}x$, the quality of the vertex fit $\chi^2_{\rm vertex~ fit}$, the decay length and the decay angle $\cos\theta^*$. The decay angle $\theta^*$ is defined here as the angle between the $\Lambda$'s direction of flight and one of its daughter particles in the $\Lambda$'s rest frame. The systematic error due to the modeling of the signal and background spectra was estimated by fitting the mass spectra with various combinations of signal (sum of two Gaussian functions, a Breit–Wigner distribution) and background (first and second order polynomials) functions. The error is taken to be the maximum difference between the nominal multiplicity and the ones obtained using the different fit functions. The uncertainties on the branching ratios for the decays $\Lambda \to p \pi^-$ are propagated into an error on the $\Lambda$ multiplicity, respectively. These uncertainties range from $0.7\%$--$1.2\%$.

%%%%%%%%%%%%%%%%%%%%%%%%%%%%%%%%%%%%%
\subsection{DELPHI 1993--1998}
%%%%%%%%%%%%%%%%%%%%%%%%%%%%%%%%%%%%%

\paragraph{{\bf DELPHI\_1993\_I360638} \cite{Abreu:1993mm}.} \textsc{Delphi} has performed a detailed analysis of both the scaled momentum as well as the correlations of $\Lambda$-baryons using $993,000$ hadronic events collected in the period 1991--1992. The $\Lambda$ baryons are reconstructed using their decay in flight into $p\pi^-$\footnote{The particle momenta have been determined from the measured Cherenkov ionization angles.} where these decays have been checked to be secondary to distinguish them from the primary $Z\to q\bar{q}$ decays. All the tracks with opposite charge have been considered to construct the secondary vertex and the pair with a minimum $\chi^2$ was retained (the $\chi^2$ is computed from the distances of the vertex to the extrapolated tracks). A number of conditions have to be satisfied by the $\Lambda$ vertex candidates and additional cuts were imposed to remove the ambiguities with the $K_S^0 \to \pi \pi$ candidates. The systematic errors mainly arise from the parametrization of the backgrounds. No details were given about how they have modeled the errors neither about the way the correction factors have been applied. We must stress that a comparison between the experimental measurement of $\Lambda$ scaled momentum and of the predictions from \textsc{Herwig} 5.4 \cite{Marchesini:1987cf} and \textsc{JetSet} 7.3 \cite{Sjostrand:1993yb} has been done and both the generators fail to describe the data at high $x_p$.  

\paragraph{{\bf DELPHI\_1995\_I394052} \cite{Abreu:1995cu}.} \textsc{Delphi} has performed a measurement of the proton spectrum for momenta in the range $[1.4,~5.0]~{\rm GeV}$ based on $17 \times 10^4$ hadronic events from $Z^0$ decays. Contrarily to \textsc{Aleph}, the reconstruction of the particle momenta in \textsc{Delphi} is based on the measurement of the ionization angle in the Ring Imaging CHerenkov (RICH) detector. A number of selection cuts have been applied to improve the quality of particle identification in RICH (see Section 3 of Ref.~\cite{Abreu:1995cu} for more details). We note that the efficiencies of the tracks' selection increases with increasing particle momenta. The fraction of $p/\bar{p}$ particles were determined from a fit for the Cherenkov angle distribution in specific momentum ranges\footnote{Note that the Cherenkov angle has a dependence on the particle mass and the refractive index of the radiatior -- two radiators have been used in this analysis --. The probability density of observing the measured Cherenkov angle $\theta_C^i$ for a track "$i$" depends on various parameters and was fitted taking into account three particle species $\pi$ (which also includes electrons and muons since they cannot be distinguished from pions with this method), $K^\pm$ and $p/\bar{p}$. Finally, the Likelihood function includes an additional constant term which depends on the noise.}. Several correction factors have been applied to data. A bin-by-bin correction factors have been applied to measurements so that all the $K_S^0$ and $\Lambda^0$ decay products were included and all the $K_L^0$ were stable particles. Heavier particles are expected to have lower number of photo-electrons than lighter ones which results in a lower ring detection efficiency in some lower momentum bins for protons. A correction factor has been applied to account for this effect and which varies between $1.09 \pm 0.08$ for $\langle P \rangle = 2.75~{\rm GeV}/c$ to $1.96 \pm 0.15$ for $\langle P \rangle = 1.55~{\rm GeV}/c$ (no correction factors associated to ring detection efficiencies have been applied for proton momenta $P > 3.0~{\rm GeV}/c$). 

\paragraph{{\bf DELPHI\_1998\_I473409} \cite{Abreu:1998vq}.} \textsc{Delphi} has performed measurements of identified particle momentum for various hadrons covering the full momentum range; $P \in~ ]0.7,~45.6[~{\rm GeV}/c$ and utilizing $\simeq 1,400,000$ hadronic events. The reconstruction of the particle momenta has been performed by measuring the Cherenkov ionization angles in the RICH detector. Details about the event selection and the experimental techniques can be found in Sections $2$--$4$ in Ref.~\cite{Abreu:1998vq}. The acceptance times efficiency after the full selection is about $76.3\%$~($85.2\%$) for experimental~(MC) data. A fairly detailed analysis of systematic uncertainties has been given in this analysis. One categorizes these uncertainties into four categories: {\it (i)} Errors due to the analysis method which have been estimated by comparing the results of the $3\times 3$ matrix inversion method (the entries of the tagging efficiency matrix correspond to the probabilities of selecting a particle $i$ as $j$; $i, j \in \{\pi^\pm, K^\pm, p\bar{p}\}$) to the results of the $2 \times 2$ matrix inversion method (where $K^\pm$ and $p\bar{p}$ are combined as ${\rm Hv}^\pm$); {\it (ii)} Errors due to PID which have been estimated using the \textsc{NewTag} software \cite{} by comparing results from different levels of tagging quality and of RICH track quality selections -- very loose, loose, or standard --, {\it (iii)} the errors on the event selection were found to be small (except errors for the event selection in the flavour tagging) and {\it (iv)} Finally, errors due to correction factors which we are going to discuss shortly afterwards. A nice summary of the errors have been presented as a function of the particle momentum $P$ in Fig. 6 of Ref.~\cite{Abreu:1998vq}: For $p\bar{p}$, the total errors vary between $1\%$ for $P \simeq 0.7~{\rm GeV}/c$ to about $6\%$ for $P \simeq 45.6~{\rm GeV}/c$. There are two types of corrections: {\it (i)} Local corrections and {\it (ii)} global corrections. Local corrections depend on the TPC and RICH detectors and their particle identification inducing bin-by-bin corrections from the track selection efficiency and from the tagging efficiency. The correction from the track selection efficiency was of order $1$--$2\%$. The correction from the tagging efficiency was of order $\simeq 3\%$~($5\%$)  in the gas~(liquid) radiator. On the other hand, global corrections are connected to detector effects and event selection criteria. They are derived from Monte Carlo simulations, using \textsc{JetSet} version 7.3, by comparing the rates before and after full detector simulations; the correction factor derived on a bin-by-bin basis $c_i = N_i^{\rm generator}/N_i^{\rm detector}$ was applied to the measured rates. 

%%%%%%%%%%%%%%%%%%%%%%%%%%%%%%%%%%%%%%
\subsection{OPAL 1994--1997}
%%%%%%%%%%%%%%%%%%%%%%%%%%%%%%%%%%%%%%

\paragraph{{\bf OPAL\_1994\_S2927284} \cite{Akers:1994ez}.} \textsc{Opal} has measured the spectra of charged hadrons using $\mathcal{L} =24.9~{\rm pb}^{-1}$  of data collected in 1992 selecting multi-hadronic data. The charged particle energies were measured using the energy loss measurement in the jet chamber and taking into account tracks with $p_\perp > 0.15~{\rm GeV}/c^2$. Due to overlap between different particle species, the rate of protons, for $x_p \in [3.16, 9] \times 10^{-2}$ was not determined. The selected tracks are required to satisfy a number of selection criteria (similar to the ones used by e.g. \textsc{Aleph}). After all the selection criteria, about $22.1\%$ of the tracks are retained while keeping tracks from decays of short-lived particles with mean life-time satisfying $\langle \tau \rangle < 3 \times 10^{-10}$ seconds -- with this definition, $K^0_S$, $\Lambda$, $\Sigma^\pm$, $\Xi$, and $\Omega$ are forced to decay --. The identification of charged hadrons cannot be done unambiguously -- since depending on the momentum region, the energy loss of e.g. protons may not be distinguished from that of charged Kaons --. Therefore, \textsc{Opal} collaboration used a statistical method to fit the numbers measured in the data. Momentum-dependent efficiency corrections which take into account the effects of geometrical and kinematical acceptance, nuclear corrections, and decays in flights have been applied to data. The corrections are obtained from Monte Carlo simulations; They found roughly; $C \equiv X_{\rm generator}/X_{\rm MC+detector} = 20$--$30\%$ after the full selection cuts. \textsc{Opal} made a comprehensive description of systematic uncertainties. The systematic errors depend on the momentum region where we identify two distinct regions; $p < 1.4$ GeV/c and $p > 2$ GeV/c. The uncertainties are linked to track cuts, muon subtractions (does not have an effect on proton spectra), momentum-corrections, nuclear interactions, decay in flights, fit methods, double Gaussian functions and relative resolution. The last three errors contribute only for momentum values $> 2~{\rm GeV}/c$. The total systematic uncertainty on proton spectra is $4.1\%~(15.7\%)$ for $p<1.4~(p >2)$ GeV/c.  Fewer protons were observed than what was seen by \textsc{Aleph} in \cite{Buskulic:1994ft, Barate:1996fi} and \textsc{Delphi} \cite{Abreu:1998vq} for $x_p = p/|p_{\rm beam}| > 0.1$. In fact this issue was reported on by \textsc{Aleph} \cite{Buskulic:1994ft}.

%\begin{itemize}
%    \item A minimum $p_\perp > 0.15$ GeV/c.
%    \item Maximum transverse distance to the interaction vertex of $|d_0| < 5~{\rm cm}$.
%    \item Maximum distance along the beam axis to be $|z_0| < 40~{\rm cm}$.
%    \item The polar angle to be $|\cos\theta^*| < 0.7$
%\end{itemize}

%In addition, they require that each track should have a minimum of 100 hits in the jet chamber used for the determination of the energy loss ${\rm d}E/{\rm d}x$. 


\paragraph{{\bf OPAL\_1997\_S3396100} \cite{Alexander:1996qj}.} A study of the strange baryon production has been performed by \textsc{Opal} collaboration utilizing approximately $3.65$ millions hadronic events collected between 1990--1994. The $\Lambda$ baryons have been reconstructed using their decays into $p\pi^-$. Two methods were applied to select $\Lambda$ baryons. The first method was optimised to have good mass and momentum resolution and was used to measure the $\Lambda$ cross section due to the fact that it has small systematic uncertainties. The second method was optimised to give a higher efficiency over a broader $\Lambda$ momentum range, and therefore, yields a more consistent differential cross section in the whole $x_E$-range\footnote{$x_E$ is defined as 
$$
x_E \equiv 2 \frac{E_{\Lambda}}{\sqrt{s}}.
$$
Note that there is an error in the \textsc{Rivet} analysis since they computed $x_p$ instead of $x_E$. I have checked this information in \textsc{HepData}. The effect of this is very important for small $x$ as can be seen in Fig. \ref{fig:Lambda0:OPAL}.}. More details about the methods can be found in Section 3.1 of Ref.~\cite{Alexander:1996qj}. A detailed study of the systematic uncertainties on $\Lambda$ spectrum have been performed in their paper. The total systematic uncertainty is about $2.7\%$~($3.3\%$) for the first~(second) method. Statistical uncertainties are very small $\simeq 0.3$--$0.4\%$. These systematic uncertainties arose from background determinations, signal mass resolution, simulation of selection cut distributions, $x_E$ range corrections, and Breit-Wigner resonance tails. The most important correction in our study is the $x_E$ range corrections. This has been performed in order to calculate the integrated production cross sections, and was accounted for by a correction for the unobserved momentum regions. The corrections has been performed by using \textsc{JetSet} 7.3 \cite{Sjostrand:1993yb} and are certainly going to be large because of the absence of data in that regions. Besides, the method to determine the uncertainty from these corrections consist in comparing the predictions of \textsc{Herwig} 5.6 \cite{Marchesini:1987cf} and \textsc{JetSet} 7.3 \cite{Sjostrand:1993yb} and taking the half of their differences. This results in an additional systematic uncertainty of $0.2\%$ on the measured rate which itself yields about $3\%$ of the total $\Lambda$ rate.

%%%%%%%%%%%%%%%%%%%%%%%%%%%%%%%%%%%%%%%%%%%%%%%%%%%%%%%%
\subsection{Conclusions about the included measurements}
%%%%%%%%%%%%%%%%%%%%%%%%%%%%%%%%%%%%%%%%%%%%%%%%%%%%%%%%

From the discussion in section \ref{sec:physics}, it is clear that the modeling of anti-protons will be improved if one includes all the relevant measurements performed at LEP. Besides, improvements on the quality of the fits can be made if the following measurements are included:
\begin{itemize}
    \item {\it{$\Lambda^0$ scaled momentum}}: $\Lambda^0$ is the dominant source of secondary protons at LEP -- about $22\%$ --. The mean multiplicity of $\Lambda^0$ was measured to be $\Lambda^0 = 0.32$. On the other hand, $\Lambda^0$ decays with $63.8\%$ into $p\pi^-$ \cite{} and therefore we expect a strong correlation between the scaled momentum of $\Lambda^0$ and of $p/\bar{p}$ since $\Lambda^0$ is reconstructed using tracks identified with charged pions and protons.
    
    \item {\it{$\Delta^{++}$ scaled momentum}}: $\Delta^{++}$ contributes about $11\%$ to the proton rate at LEP. $\Delta^{++}$ decays with $100\%$ into $p\pi$. The measurement of the rate of $\Delta^{++}$ scaled momentum has been performed by \textsc{Delphi} and \textsc{Opal}.

    \item {\it{$\Sigma^\pm + (c.c.)$ scaled momentum}}: About $5\%$ of protons at LEP are coming from the decays of $\Sigma^+$. The corresponding branching ratio is ${\rm BR}(\Sigma^+ \to p\pi^0) = 51.57\%$ \cite{}. There are two measurements of $\Sigma^+$ at LEP and they will be used in our tunes.
\end{itemize}