%%%%%%%%%%%%%%%%%%%%%%%%%%%%%%%%%%%%%%%%%%%%%%%%%%%%%%%%%%
\section{Application to dark-matter indirect detection experiments}
\label{sec:DMfit}
%%%%%%%%%%%%%%%%%%%%%%%%%%%%%%%%%%%%%%%%%%%%%%%%%%%%%%%%%%
In this section we quantify the effects of QCD uncertainties on two DM observables: the velocity-weighted effective cross section, $\langle \sigma v \rangle$, and the DM mass $M_\chi$. We first discuss the general methodology for determining the DM uncertainties arising from QCD uncertainties. We then discuss the results for an antiproton final state. This is followed by a discussion of electron antineutrinos and photons final states as a proof of principle.

%%%%%%%%%%%%%%%%%%%%%%%%%%%%%%%%%%%%%%%%%%%%%%%%%%%%%%%%%%
\subsection{Dark Matter uncertainties}
\label{subsec:DMunc}
In DM indirect searches, there are generally two important parameters: the velocity-weighted effective cross section, $\langle \sigma v \rangle$\footnote{In principle, the DM density can also be used instead of the velocity-weighted effective cross section.}, and the DM mass, $M_\chi$. The inclusion of QCD uncertainties will translate into uncertainties in determination of these two DM observables. The effect on $\langle \sigma v \rangle$ is simply an overall shift in the height of the spectrum; varying $\langle \sigma v \rangle$ is identical to changing the normalization of the spectrum. The uncertainty concerning the DM mass is quite straightforward, since a change in the DM mass changes the spectrum including the peak position. To quantify the effects of the QCD uncertainties on the DM mass, we consider two different approaches. First, for a spectrum coming from a DM particle with mass $M_\chi$, which different masses have spectra including QCD uncertainties that can match the spectrum of $M_\chi$, i.e., which masses can look the same as $M_\chi$. Or, the spectra of which masses can match the spectra of $M_\chi$ including QCD uncertainties, i.e., with which masses can $M_\chi$ look similar. We will call these two different approaches to finding the DM mass uncertainty $\Delta M$ and $\Delta S$, respectively. To quantify which spectra agree when QCD uncertainties are included, we require that for the upper and lower bounds of the DM uncertainties $\chi^2 / N_{\text{d.f.}} \approx 1$ holds. Although not proven, we consider it safe to assume that $\chi^2 / N_{\text{d.f.}}$ strictly increases for larger deviating values of DM mass or spectral height.

We have written a public code to both interpolate spectra for a given DM mass, channel, and final state, and to compute the upper and lower bounds for the three aforementioned DM uncertainties: $\Delta \langle \sigma v \rangle$, $\Delta M$, and $\Delta S$. Additionally, the considered fit region can be user supplied. The code can be found in this \href{https://github.com/ajueid/qcd-dm.github.io.git}{github} repository.
 
For both electron antineutrino and photon final states the spectrum does not need to be propagated, as these particles can propagate freely. Thus the DM uncertainties can be determined directly from the annihilation spectrum. However, for the antiproton final state, the spectrum must first be propagated before the upper and lower bounds of the uncertainties can be determined. Our code emulates cosmic-ray propagation by diffusing each bin using tabulated diffusion values, if the bin is not tabulated the diffusion values are linearly interpolated. By performing the diffusion for every bin, the complete propagated spectrum is obtained. We use \textsc{Dragon}~2~\cite{Evoli:2017, Evoli:2018} to compute the tabulated diffusion values by inserting an antiproton spectrum for the DM annihilation spectrum that is peaked at a specific bin. The propagation parameters are determined by fitting the AMS-02~\cite{AMS:2021} proton $p$, antiproton over proton $\bar{p}/p$ and Boron over Carbon ratio $B/C$ spectra using an artificial bee colony as the optimization algorithm~\cite{Dervis:2005, Akay:2012}. The resulting propagation parameters can be found in the README of the code. This fit resulted in a DM mass of $\mathcal{O}(100\text{GeV})$, which may affect results for other DM masses. 

To assess the effects of the different propagation parameters, we cross-compared a set of parameters from~\cite{CR_parameters:Di_Bernardo_2010} and the default settings from \textsc{Dragon} 2. The variation in the DM uncertainties between these sets is about 1-5\%. We consider these discrepancies to be small enough that it is safe to use our fitted parameters for the following results. 

Our default tabulated diffusion values do not account for solar modulation; only a low-energy correction factor for the diffusion coefficient is implemented. To account for this, we have fitted antiproton final-state spectra down to 1 GeV. While solar modulation is relevant at these energies, QCD uncertainties are typically small in this range, so we do not expect the inclusion of solar modulation to significantly change our results. Regardless, the tabulated diffusion values can be input by the user to meet situation-specific requirements. In particular, positron propagation is not implemented, so all results for positron final states were obtained using the annihilation spectrum and should therefore be treated with caution. In addition, the results of our code may be unreliable in certain cases for very small values of differential flux or QCD uncertainties due to numerical problems. We recommend that all results be checked for accuracy.


%%%%%%%%%%%%%%%%%%%%%%%%%%%%%%%%%%%%%%%%%%%%
\subsection{Results for antiprotons}
\label{subsec:antiprot_DMunc}
%%%%%%%%%%%%%%%%%%%%%%%%%%%%%%%%%%%%%%%%%%%%
In order to showcase the impact of the QCD uncertainties on the antiproton spectrum and consequently the DM observables, we show $\Delta S$ for all channels in figure \ref{fig:antiproton_DM_unc} for a DM mass of 200 GeV. The QCD uncertainties are shown as the ratio to nominal, and the fitted upper and lower boundaries as a green and blue line respectively. We use a 200 GeV DM particle such that all channels can be produced on shell without having to account for kinematic boundary effects. We use $\Delta S$ as the DM uncertainty simply for visualization purposes; $\Delta S$ only has one set of QCD uncertainties, namely those of the nominal spectrum, as opposed to $\Delta M$ which considers the QCD uncertainties of both the upper and lower limits. The uncertainty on $\langle \sigma v \rangle$ is of course simply a vertical shift of the spectrum. \\
\begin{figure}[h]
    \centering
    \includegraphics[width=.9\textwidth]{Figures/DM/all_AntiP.pdf}
    \caption{The $\Delta S$ uncertainty of various channels for a nominal DM mass of 200 GeV. The QCD uncertainties and the spectra of the upper and lower bound of $\Delta S$ are shown as the ratio to the spectrum of the 200 GeV DM particle. The shaded region from 1 GeV up to the DM mass of 200 GeV is considered for the fit.}
    \label{fig:antiproton_DM_unc}
\end{figure}
In figure~\ref{fig:antiproton_DM_unc} the shaded region, $x \in [1 / M_\chi, 1]$, is the energy range that is used for the fit, i.e. 1 GeV up to the DM mass of 200 GeV. Due to the requirement on the upper and lower bounds of $\Delta S$ that $\chi^2 / N_{\text{df}}\approx 1$ some averaging over the spectra is to be expected, which indeed occurs mainly in the high-$x$ regions. This is most striking for $b\bar{b}$, where the spectra fits comfortably within the QCD uncertainties, for low $x$, thereby compensating for a poorer fit at high $x$. The relevance of the high-$x$ regions is of course dependent on a case-by-case scenario. In order to make no assumptions about the relevance of the high-$x$ region we fit the entire spectrum up to $x=1$. However, not considering the high-$x$ regions increases the size of the DM uncertainties for most channels, most significantly for $b\bar{b}$. 

The relative DM uncertainties for higher masses DM must be calculated specifically for a given mass and channel, but some general observations can be made from the figure \ref{fig:antiproton_DM_unc}. For different masses, the influence of the low-$x$ regions becomes more important; the relative uncertainties of the spectra do not change significantly for higher DM masses, so the low-$x$ regions can significantly affect the relative magnitude of the upper and lower limits. This is because if one consistently considers the fitting range down to 1 GeV, the fraction of low $x$ regions increases. We have checked the following statements for both $M_\chi = 500$ GeV and $M_\chi = 1000$ GeV. For $u\bar{u}$, $d\bar{d}$, $c\bar{c}$, and $gg$, no significant change occurs in the relative size of $\Delta S$, which is evident when considering figure \ref{fig:antiproton_DM_unc}. The $W^+W^-$, $ZZ$, and $hh$ obtain relatively smaller DM uncertainties, as is also expected since the fitted spectra for these channels diverge in the low-$x$ regions. A relative increase in $\Delta S$ is seen for $b\bar{b}$, the upper limit of $s\bar{s}$, and the upper limit of $t\bar{t}$, as is also expected. 

While these results refer to $\Delta S$, the conclusions can easily be applied to $\Delta M$, since in both cases it is a matter of shifting the mass. The main difference, of course, is that the QCD uncertainties are different for different DM masses, so the numerical values for $\Delta S$ and $\Delta M$ may be different. This will be made explicit in the following section dealing with electron antineutrino and photon final states.

%%%%%%%%%%%%%%%%%%%%%%%%%%%%%%%%%%%%%%%%%%%%%%%%%%
\subsection{Results for electron antineutrinos and photons}
\label{subsec:neutrino_DMunc}
%%%%%%%%%%%%%%%%%%%%%%%%%%%%%%%%%%%%%%%%%%%%%%%%%%
In the following, we consider only electron antineutrinos and photons to show the effects of QCD uncertainties. The results for muon and tau antineutrinos may be significantly different from those for electron antineutrinos because their spectra differ considerably, as can be seen in Figure~\ref{fig:spectra:mDM:100:2}. The DM uncertainties for antineutrinos and photons can be determined directly from their annihilation spectra without the need for CR propagation. We show the three different DM uncertainties, $\Delta \langle \sigma v \rangle$, $\Delta M$, $\Delta S$, for a DM mass of 200 GeV for an electron antineutrino and photon final state in Table~\ref{tab:neutrino_and_photon_DM_unc}. We consider the energy range down to $x=10^{-5}/M_\chi$ for antineutrinos and $x=10^{-3}/M_\chi$ for photons, since for lower values of $x$ the QCD uncertainties become erratic. For both final states, we additionally perform a fit that includes and excludes the high- $x$ region of $x\in(0.5,1]$ to quantify the effects of these regions\\
\begin{table}[!t]
    \centering
    \begin{tabular}{l?cc|cc|cc?cc|cc|cc}
        \toprule
        \toprule
        & \multicolumn{6}{c?}{$\chi \chi \rightarrow XX \rightarrow \bar{\nu}_e$} & \multicolumn{6}{c}{$\chi \chi \rightarrow XX \rightarrow \gamma$} \\
        \midrule
        & \multicolumn{2}{c|}{$\Delta\langle\sigma v\rangle$ [\%]} & \multicolumn{2}{c|}{$\Delta M_\chi$ [GeV]} & \multicolumn{2}{c?}{$\Delta S$ [GeV]} & \multicolumn{2}{c|}{$\Delta\langle\sigma v\rangle$ [\%]} & \multicolumn{2}{c|}{$\Delta M$ [GeV]} & \multicolumn{2}{c}{$\Delta S$ [GeV]}\\
         & $A$ & $B$& $A$ & $B$& $A$ & $B$& $C$ & $D$& $C$ & $D$& $C$ & $D$\\
        \midrule
        $q\bar{q}$ &$^{+7}_{-7}$& $^{+7}_{-7}$ & $^{+33}_{-15}$& $^{+34}_{-28}$ & $^{+17}_{-30}$   & $^{+30}_{-30}$ & $^{+6}_{-6}$& $^{+6}_{-6}$ & $^{+20}_{-3}$ & $^{+23}_{-26}$& $^{+17}_{-25}$   & $^{+36}_{-29}$\\
        \rowcolor{Gray}
        $b\bar{b}$ &$^{+8}_{-7}$& $^{+7}_{-7}$ & $^{+31}_{-14}$& $^{+44}_{-27}$ & $^{+11}_{-29}$  & $^{+29}_{-30}$ & $^{+5}_{-6}$ & $^{+5}_{-6}$& $^{+14}_{-11}$& $^{+14}_{-11}$ & $^{+11}_{-20}$   & $^{+12}_{-20}$\\
        $t\bar{t}$ &$^{+3}_{-3}$& $^{+3}_{-3}$ & $^{+0}_{-0}$ & $^{+0}_{-0}$  & $^{+0}_{-0}$ & $^{+0}_{-0}$ & $^{+1}_{-3}$& $^{+1}_{-3}$ & $^{+10}_{-0}$& $^{+10}_{-0}$ & $^{+0}_{-0}$   & $^{+0}_{-8}$\\
        \rowcolor{Gray}
        $W^+W^-$ &$^{+2}_{-1}$& $^{+4}_{-4}$ & $^{+0}_{-0}$ & $^{+8}_{-9}$ & $^{+0}_{-0}$   & $^{+10}_{-7}$ & $^{+4}_{-2}$& $^{+4}_{-2}$ & $^{+0}_{-0}$& $^{+21}_{-10}$ & $^{+0}_{-0}$   & $^{+10}_{-10}$\\
        $ZZ$ &$^{+3}_{-0}$& $^{+4}_{-0}$ & $^{+0}_{-0}$& $^{+7}_{-5}$ & $^{+0}_{-0}$   & $^{+0}_{-6}$ & $^{+4}_{-4}$& $^{+4}_{-4}$ & $^{+8}_{-8}$ & $^{+10}_{-10}$& $^{+9}_{-7}$   & $^{+11}_{-10}$\\
        \rowcolor{Gray}
        $HH$ & $^{+4}_{-4}$ & $^{+5}_{-5}$& $^{+0}_{-0}$ & $^{+15}_{-12}$ & $^{+0}_{-0}$  & $^{+13}_{-12}$ &$^{+4}_{-2}$ & $^{+4}_{-3}$& $^{+0}_{-0}$ & $^{+11}_{-8}$& $^{+3}_{-0}$   & $^{+9}_{-10}$\\
        $gg$ &$^{+7}_{-7}$ & $^{+7}_{-7}$& $^{+30}_{-20}$ & $^{+43}_{-26}$& $^{+14}_{-28}$   & $^{+29}_{-28}$ & $^{+6}_{-6}$& $^{+6}_{-6}$ & $^{+14}_{-20}$ & $^{+14}_{-20}$& $^{+13}_{-20}$   & $^{+13}_{-29}$\\
        \bottomrule
        \bottomrule
    \end{tabular}
    \caption{The three DM uncertainties, $\Delta \langle \sigma v \rangle$, $\Delta M$, and $\Delta S$, for a DM mass of $M_\chi=200$ GeV for electron antineutrino and photon final states. The uncertainties are fitted for four different ranges: $A = [10^{-5}/M_\chi, 1]$, $B = [10^{-5}/M_\chi, 0.5]$, $C=[10^{-3}/M_\chi,1] $, and $D=[10^{-3}/M_\chi, 0.5]$. The lower bounds of the fit range has been chosen such the QCD uncertainties do not become erratic.}
    \label{tab:neutrino_and_photon_DM_unc}
\end{table}
From the table~\ref{tab:neutrino_and_photon_DM_unc} it can be seen that for the $W^+W^-$, $ZZ$, and $HH$ channels the inclusion of the high-$x$ region is very important, especially for a $\bar{\nu}_e$ final state. Figure~\ref{fig:spectra:mDM:100:1} clearly shows the small QCD uncertainties in the high-$x$ regions for both $W^+W^-$ and $ZZ$, and Figure~\ref{fig:spectra:mDM:1000:2} includes $HH$, where the small QCD uncertainties naturally translate into small DM uncertainties. While for most channels the spectrum falls off smoothly at high $x$ values, this is not the case for $W^+W^-$ or $ZZ$ when the final state is an antineutrino. Thus, while for many channels the inclusion or exclusion of high-$x$ values can generally be justified due to the low differential flux in this region, for both $W^+W^-$ and $ZZ$ this must be judged on a case-by-case basis depending on the specific analysis.

The uncertainties of DM for all non-top quarks and gluons are comparable to an antiproton final state: The upper and lower bounds are approximately at the edges of the QCD uncertainties. Thus, in general, the uncertainties of DM do not change significantly with or without the high-$x$ region. There are some exceptions though, e.g., the upper limit of $\Delta M$ for $\chi\rightarrow gg\rightarrow \bar{\nu}_e$. For a $t\bar{t}$-mediated $\bar{\nu}_e$ or $\gamma$ final state, the QCD uncertainties are more comparable to the $VV$ or $HH$ channels than to the $q\bar{q}$ or $gg$ channels, as can be seen from figure~\ref{fig:spectra:mDM:1000:2}. This is reflected in the DM uncertainties, which are indeed more comparable to $VV$ and $HH$.

In general, the magnitude of the DM uncertainties for both $\bar{\nu}_e$ and $\gamma$ final states depends strongly on the channel and the considered fitting region, even more so than for an antiproton final state. The impact of the QCD uncertainties on the DM observables can be as large as 20\% in certain scenarios and therefore must be considered in future DM indirect detection studies.