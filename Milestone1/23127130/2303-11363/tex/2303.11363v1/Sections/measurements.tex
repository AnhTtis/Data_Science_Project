%%%%%%%%%%%%%%%%%%%%%%%%%%%%%%%%%%%%%%%%%%%%%%%%%%%%%%%%%%%
\section{Experimental measurements and generator predictions}
\label{sec:measurements}
%%%%%%%%%%%%%%%%%%%%%%%%%%%%%%%%%%%%%%%%%%%%%%%%%%%%%%%%%%%
%%%%%%%%%%%%%%%%%%%%%%%%%%%%%%%%%%%%%
\subsection{Introduction}
%%%%%%%%%%%%%%%%%%%%%%%%%%%%%%%%%%%%%%
From the discussion in section \ref{sec:physics}, it is clear that the modeling of the spectra of anti-protons will be improved if one includes all the relevant measurements of proton spectrum performed at LEP. Besides the measurements of the proton spectrum itself, one may expect some improvements from measurements of the spectra of the following baryons:
\begin{itemize}
    \item {\it{$\Lambda^0$}}: $\Lambda^0$ is the dominant source of secondary protons at LEP (about $22\%$ of the total protons are coming from $\Lambda^0$ baryons). The mean multiplicity of $\Lambda^0$ was measured by several collaborations at LEP: $\langle n_{\Lambda^0} \rangle = 0.357\pm 0.017$ \cite{Abreu:1993mm}, $\langle n_{\Lambda^0} \rangle = 0.348\pm 0.013$ \cite{Abreu:1996na} performed in $1993$ and $1996$ respectively. We note that $\Lambda^0$ baryons decay with $63.8\%$ branching ratios into $p\pi^-$ \cite{ParticleDataGroup:2020ssz} and therefore we expect a strong correlation between the scaled momentum of $\Lambda^0$ and of $p/\bar{p}$ since most of $\Lambda^0$ baryons are reconstructed using tracks identified with charged pions and protons. 
    
    \item {\it{$\Delta^{++}$}}: About $11\%$ of protons at the $Z$-pole are produced from the decay of the $\Delta^{++}$ which decays with $100\%$ branching ratio into $p\pi$. Given that $\Delta^{++}$ and $p$ are members of the same multiplet, we may expect important constraints on $p/\bar{p}$ due to isospin. However, there are only two measurements of $\Delta^{++}$ performed by \textsc{Delphi} \cite{DELPHI:1995ysj} and \textsc{Opal} \cite{OPAL:1995otk}. These measurements suffer from large uncertainties due to the difficulty in isolating the $\Delta^{++}$ signal from the overwhelming backgrounds. We, therefore, do not include these measurements in the fits.

    \item {\it{$\Sigma^\pm$}}: About $5\%$ of protons at LEP are coming from the decays of $\Sigma^+$. The corresponding branching ratio is ${\rm BR}(\Sigma^+ \to p\pi^0) = 51.57\%$ \cite{ParticleDataGroup:2020ssz}. There are two measurements of $\Sigma^+$ scaled momentum at LEP by \textsc{Delphi} \cite{DELPHI:2000oqt} and \textsc{Opal} \cite{OPAL:1996dbo} collaborations. These measurements will not be used in the fit for the same arguments we used to exclude $\Delta^{++}$ measurements.
\end{itemize}


Therefore, the main constraining observables in this study will consist of a set of measurements of $\Lambda$ and $p/\bar{p}$ energy--momentum distributions. To guarantee a good agreement with the results of the previous study \cite{Amoroso:2018qga}, we also include measurements of meson spectra, event shapes and particle multiplicities. Before going into a discussion of the setup used in this study, we discuss briefly the various measurements performed by LEP at the $Z$--pole using data collected between 1992 and 1999\footnote{The measurements reported on by the experimental collaborations at LEP of the $\Lambda$ or $p/\bar{p}$ spectra correspond the measured variables being either $x_p = |p_{\rm hadron}|/|p_{\rm beam}|$ (\textsc{Aleph} \cite{Barate:1996fi} and \textsc{Delphi} \cite{Abreu:1993mm, Abreu:1995cu, Abreu:1998vq}), $x_E = E_{\rm hadron}/|p_{\rm beam}|$ (\textsc{Opal} \cite{Alexander:1996qj}), $\xi = \log(1/x_p)$ (\textsc{Aleph} \cite{Barate:1999gb}) or $|p_{\rm hadron}|$ (\textsc{Delphi} \cite{Abreu:1995cu, Abreu:1998vq} and \textsc{Opal} \cite{Akers:1994ez}).}. 

%variables are defined as 
%\begin{eqnarray}
%x_p = \frac{|p_{\rm hadron}|}{|p_{\rm beam}|},~{\rm %ALEPH}~\cite{Barate:1996fi},~{\rm DELPHI}~  \nonumber \\
%x_E = \frac{E_{\rm hadron}}{|p_{\rm beam}|}, \\
%\xi = \log(1/x_p). \nonumber
%\end{eqnarray}}

%%%%%%%%%%%%%%%%%%%%%%%%%%%%%%%%%%%%%%%%%%%%%%
\subsection{Note about the relevant measurements}
%%%%%%%%%%%%%%%%%%%%%%%%%%%%%%%%%%%%%%%%%%%%%%

\paragraph{{\bf ALEPH~(1996--2000).}} The \textsc{Aleph} collaboration has reported on three measurements of $\Lambda$ scaled momentum -- $x_p$ in \cite{Barate:1996fi} and $\xi$ in \cite{Barate:1999gb} -- and one measurement of $p/\bar{p}$ scaled momentum \cite{Barate:1996fi}. The first measurements rely on the initial data which contain 520,000 inclusive hadronic events \cite{Barate:1996fi}. In a second paper published in 2000, the \textsc{Aleph} collaboration used a more complete dataset consisting of 3.7 million hadronic events \cite{Barate:1999gb}. The measurement of identified particle spectra by \textsc{Aleph} can be made by a simultaneous measurement of the hadronic momentum and ionisation energy loss ${\rm d}E/{\rm d}x$ in the Time Projection Chamber (TPC). There are holes in the reported measurement of $p/\bar{p}$ spectrum in the regions $x_p \in [1.8, 2.4] \times 10^{-2} \cup [2.8, 7] \times 10^{-2}$ due to the strong overlap between the bands of $p/\bar{p}$ and of the other hadrons ($\pi^\pm, K^\pm$). Most of the systematic uncertainties can be roughly categorised into three components:  track reconstruction efficiencies, efficiencies in the $\Lambda^0$ reconstruction and the background calibration. These errors have been corrected for by using the predictions of \textsc{JetSet} \cite{Sjostrand:1993yb}. We note that these correction factors can be large in some kinematics regions.

\begin{figure}[!t]
    \centering
    \includegraphics[width=0.49\linewidth]{Figures/Lambda0_Linear.pdf} \hfill
    \includegraphics[width=0.49\linewidth]{Figures/Lambda0_Log.pdf}
    \vfill
    \includegraphics[width=0.49\linewidth]{Figures/proton_Linear.pdf}
    \hfill
    \includegraphics[width=0.49\linewidth]{Figures/proton_Log.pdf}
    \caption{{\it Upper panels:} Comparison between the different measurements of the $\Lambda^0$ scaled momentum $x_p$~({\it left}) and $\log(1/x_p)$~({\it right}). Here, we show the results from \textsc{Aleph} (red and violet), \textsc{Delphi} (blue) and  \textsc{Opal} (green). {\it Bottom panels}: Same as in the upper panels but for the $p/\bar{p}$ scaled momentum ({\it left}) and its logarithm ({\it right}). For the proton case, we show the results from \textsc{Aleph} (red), \textsc{Delphi} (blue and green) and \textsc{Opal} (violet). In both panels, the errors correspond to statistical and systematic uncertainties summed up in quadrature. Data is taken from \cite{Abreu:1993mm, Barate:1996fi, Akers:1994ez, Abreu:1995cu, Abreu:1998vq, Abe:1998zs, Barate:1999gb, Abe:2003iy}.}
    \label{fig:comparisons}
\end{figure}  

\paragraph{{\bf DELPHI~(1993--1998).}} The \textsc{Delphi} collaboration has performed a detailed analysis of the scaled momentum of $\Lambda^0$ and $p/\bar{p}$ using data collected in the period of 1991--1998 \cite{Abreu:1993mm, Abreu:1995cu, Abreu:1998vq}. In a first measurement of $\Lambda^0$ spectrum and $\Lambda^0$--$\bar{\Lambda}^0$ correlations, 993,000 hadronic events were used \cite{Abreu:1993mm}. In \cite{Abreu:1995cu}, a measurement of proton momentum for $x_p \in [0.03, 0.1]$ has been carried using 17000 hadronic events. This measurement has been superseded by a more recent one which relied on a larger statistical sample consisting of 1,400,000 hadronic events and covering a wider range of proton momenta, {\it i.e.} $x_p \in [1.53 \times 10^{-2}, 1]$ \cite{Abreu:1998vq}. Contrarily to \textsc{Aleph}, the reconstruction of the particle momenta in \textsc{Delphi} is based on the measurement of the ionization angle in the Ring Imaging CHerenkov (RICH) detector\footnote{Note that the Cherenkov angle has a dependence on the particle mass and the refractive index of the radiator -- two radiators have been used in this analysis --. The probability density of observing the measured Cherenkov angle $\theta_C^i$ for a track "$i$" depends on various parameters and was fitted taking into account three particle species $\pi$ (which also includes electrons and muons since they cannot be distinguished from pions with this method), $K^\pm$ and $p/\bar{p}$. Finally, the Likelihood function includes an additional constant term which depends on the noise.} . A number of selection cuts have been applied to improve the quality of particle identification in RICH (see section 3 of Ref.~\cite{Abreu:1995cu} for more details). The fraction of $p/\bar{p}$ particles were determined from a fit to the Cherenkov angle distribution in specific momentum ranges. The systematic errors mainly arise from the parametrization of the backgrounds. To account for these errors, correction factors ranging from $1\%$ to $10\%$ have been applied using the MC simulations of \textsc{JetSet} event generator. 

%\footnote{Note that the Cherenkov angle has a dependence on the particle mass and the refractive index of the radiatior -- two radiators have been used in this analysis --. The probability density of observing the measured Cherenkov angle $\theta_C^i$ for a track "$i$" depends on various parameters and was fitted taking into account three particle species $\pi$ (which also includes electrons and muons since they cannot be distinguished from pions with this method), $K^\pm$ and $p/\bar{p}$. Finally, the Likelihood function includes an additional constant term which depends on the noise.} 
 
\paragraph{{\bf OPAL~(1994--1997).}} \textsc{Opal} has measured the spectra of charged hadrons using $\mathcal{L} =24.9~{\rm pb}^{-1}$  of data collected in 1992 \cite{Akers:1994ez} and of strange baryons using approximately 3.65 million events collected between 1990--1994 \cite{Alexander:1996qj}. The determination of the hadron yields has been done from the simultaneous measurement of the track momentum and differential energy loss. Since the identification of charged hadrons cannot be done unambiguously, the \textsc{Opal} collaboration has used a statistical method to fit the number of particles measured in the data. Correction factors of order $20$--$30\%$ have been applied to account for effects of geometrical and kinematical acceptance, nuclear corrections and decay in flight \cite{Akers:1994ez}. The $\Lambda^0$ baryons have been reconstructed from the tracks associated to their decay products ($p\pi$) using two methods optimised to either have a good mass and momentum resolution or optimised to give a higher efficiency over a broader $\Lambda^0$ momentum range. The total systematic uncertainty is about $2.7\%~ (3.3\%)$ for the first (second) method while the statistical uncertainties are subleading. 



%%%%%%%%%%%%%%%%%%%%%%%%%%%%%%%%%%%%%%%%%%%%%%%%%%%%%%%%
\subsection{Conclusions about the included measurements}
%%%%%%%%%%%%%%%%%%%%%%%%%%%%%%%%%%%%%%%%%%%%%%%%%%%%%%%%

In Fig.~\ref{fig:comparisons}, we show the comparisons between the different experimental measurements of $\Lambda^0$~spectrum and $p/\bar{p}$ spectrum. As was pointed out previously, the measurements of baryon spectra were presented for different variables. In order to be able to easily compare between the different measurements, we scale all the normalized distributions to be either in $x_p$ or in $\log(1/x_p)$ in case they depend on a different variable, {\it i.e.} under the change of variable $x_E \to x_p$, the differential normalized cross section changes as $1/\sigma {\rm d}\sigma/{\rm d}x_E \to |J|^{-1} 1/\sigma {\rm d}\sigma/{\rm d}x_p$ with $J = \partial x_E/\partial x_p$ being the Jacobian of the transformation.  The normalized cross sections in $x_p$ are shown in the left panels of Fig. \ref{fig:comparisons} to identify differences in the tails of the scaled momentum. On the other hand, the normalized cross sections in $\log(1/x_p)$ are very useful to display the differences in the bulk and the peak regions  (right panels of Fig. \ref{fig:comparisons}). We can see the following differences between the different measurements:
\begin{itemize}
    \item There are some tensions between the measurement of the scaled momentum of $p/\bar{p}$ performed by \textsc{Opal} and the other experiments for $x_p > 0.1$ (the \textsc{Opal} result is below all the others).
    \item The old \textsc{Delphi} measurement (blue) of $p/\bar{p}$ momentum is inconsistent with the new one (green) for few bins of $\xi \simeq 3$--$3.2$. Note that both these \textsc{Delphi}~measurements cover the hole left by \textsc{Aleph}--1996. Furthermore, the trend of the data seems to be more consistent with \textsc{Delphi}--1998 rather than \textsc{Delphi}--1995 (very large corrections have been applied to the proton momentum in the \textsc{Delphi}--1995 measurement). 
    \item The \textsc{Delphi}--1993 measurement of $\Lambda^0$ scaled momentum seems to be inconsistent with the others for for $\xi < 1.1$ (the discrepancy is mild as compared to the proton case).
\end{itemize}

\begin{figure}[!t]
\centering
\includegraphics[width=0.49\linewidth]{Figures/Comparison/ALEPH_1996_S3486095-d33-x01-y01.pdf}
\hfill
\includegraphics[width=0.49\linewidth]{Figures/Comparison/OPAL_1997_S3396100-d01-x01-y01.pdf}
\vfill 
\includegraphics[width=0.49\linewidth]{Figures/Comparison/ALEPH_2000_I507531-d17-x01-y01.pdf}
\hfill
\includegraphics[width=0.49\linewidth]{Figures/Comparison/ALEPH_2000_I507531-d19-x01-y01.pdf}
\caption{Comparison between theory predictions and experimental measurements of $\Lambda^0$ (scaled)-momentum distribution at LEP. Here we show the $\Lambda^0$ spectrum (left upper panel) \cite{Barate:1996fi}, $\Lambda^0$ scaled energy (right upper panel) \cite{Alexander:1996qj}, Log of scaled momentum for $\Lambda^0$ in all events (left lower panel) and in $2$--jet event (right lower panel) \cite{Barate:1999gb}. The \textsc{Pythia}~8 prediction is shown in red, the \textsc{Sherpa}~2 predictions are shown in blue for the cluster (solid) and Lund (dashed) models while the \textsc{Herwig}~7 is shown in green for dipole (solid) and angular-based (dashed) shower algorithms.}
\label{fig:generators:comparison:Lambda}
\end{figure}


Based on these observations, the tunes of the fragmentation function are not expected to find a robust best-fit point for $\texttt{StringZ:aExtraDiquark}$ due to the mutual tensions between some of the measurements unless some of them are discarded in our fits. We, first, tune individually to various measurements and display the results as they are. In the more general tune that includes also the measurements of the meson scaled momenta, event shapes and mean multiplicities, we do not include the measurements of $p/\bar{p}$ and $\Lambda^0$ scaled momenta performed by \textsc{Aleph}--1996 \cite{Barate:1996fi} since the former has a hole in the peak region while the latter is superseded by \textsc{Aleph}--2000 measurement \cite{Barate:1999gb}. On the other hand, we do not include \textsc{Opal}--1994 measurement as it is inconsistent with all the other measurements for $x_p > 0.1$ (\textsc{Pythia}~8 cannot find a good agreement for this region for any choice of fragmentation function parameters). Finally, \textsc{Delphi}--1995 measurement of proton spectrum is not included in the four-dimensional parameter space tune as it was superseded by the \textsc{Delphi}--1998 measurement. 



%%%%%%%%%%%%%%%%%%%%%%%%%%%%%%%%%%%%%%%%%%%%%%%%%%%%%%%%%%%%%%%%%%%
\subsection{How good are the current theory predictions?}
\label{sec:MCgenerators}
%%%%%%%%%%%%%%%%%%%%%%%%%%%%%%%%%%%%%%%%%%%%%%%%%%%%%%%%%%%%%%%%%%%
We discuss in this section the level of agreement between the theory predictions of the three commonly used multi-purpose Monte Carlo event generators and the experimental measurements of the baryon spectra at LEP (proton and $\Lambda^0$). For this task, we use the \textsc{Pythia}~8 version 307 with the baseline Monash tune \cite{Skands:2014pea}, \textsc{Sherpa} version 2.2.12 \cite{Gleisberg:2008ta} with a shower model based on the Catani-Seymour subtraction (CSS) method \cite{Schumann:2007mg} and two hadronisation models: the cluster model which is provided by \textsc{Ahadic}++ \cite{Winter:2003tt} (the default in \textsc{Sherpa}) and the Lund string model based on \textsc{Pythia}~6 \cite{Sjostrand:2006za} and \textsc{Herwig} version 7.2.3 \cite{Bellm:2015jjp} with two radiation models: the angular-based parton-shower algorithms \cite{Gieseke:2003rz} and the dipole-based algorithm \cite{Platzer:2009jq, Platzer:2011bc}. It is found that the different parton-shower algorithms in the three MC event generators yield similar predictions in several collider observables within the theory uncertainties (see chapter V.I. of ref. \cite{Andersen:2016qtm}). In ref. \cite{Amoroso:2018qga}, we have found that the three MC event generators agree pretty well in various experimental measurements of event shapes, pion and photon spectra.  \\ 

In figures \ref{fig:generators:comparison:Lambda} and \ref{fig:generators:comparison:p}, we show the comparison between the aforementioned generators for a set of selected measurements of proton and $\Lambda^0$ spectra at LEP. We can see that the generators based on the cluster model {\it i.e.} \textsc{Herwig}~7 and \textsc{Sherpa}~2 do not agree with data and have discrepancy of more than $40\%$ in some regions with respect to the experimental measurement. On the other hand, \textsc{Sherpa}~2 with the Lund model agrees quite well with \textsc{Pythia}~8 as well as with data.

\begin{figure}[!t]
    \centering
    \includegraphics[width=0.49\linewidth]{Figures/Comparison/ALEPH_1996_S3486095-d27-x01-y01.pdf}
    \hfill
    \includegraphics[width=0.49\linewidth]{Figures/Comparison/DELPHI_1998_I473409-d23-x01-y01.pdf}
%    \vfill
    \caption{Same as figure \ref{fig:generators:comparison:Lambda} but for $p$ spectrum. Here we show the $p$ spectrum (left panel) \cite{Barate:1996fi} and $p,\bar{p}$ scaled momentum (right panel) \cite{Abreu:1998vq}.}
    \label{fig:generators:comparison:p}
\end{figure}
