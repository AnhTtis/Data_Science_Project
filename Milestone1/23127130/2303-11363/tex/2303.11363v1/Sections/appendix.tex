\appendix
%%%%%%%%%%%%%%%%%%%%%%%%%%%%%%%%%%%%%%%%%%%%%%%%%%%%%%%%%%%%%%%
\section{Measurements: Complete list}
\label{sec:appendix1}
%%%%%%%%%%%%%%%%%%%%%%%%%%%%%%%%%%%%%%%%%%%%%%%%%%%%%%%%%%%%%%%

\begin{table*}[!h]
\setlength\tabcolsep{18pt}
\begin{center}
\begin{adjustbox}{max width=\textwidth}
%\parbox{0.55\textwidth}{
\begin{tabular}{l  l  l l}
\toprule
Dataset & {Measurement} & $N_{\rm bins}$ & Reference \\
\toprule
\multicolumn{4}{l}{\textit{$p/\bar{p}$ momentum}} \\
%\textsc{Aleph} & $x_p \equiv |{\bf{p}}|/|p|_{\rm beam}$ & $26$ & ALEPH\_1995\_I382179 \cite{Buskulic:1994ft}   \\
\textsc{Aleph}~\textcolor{red}{(*)} & $x_p = |{\bf{p}}|/|p|_{\rm beam}$ & $26$ & ALEPH\_1996\_S3486095 \cite{Barate:1996fi}  \\
%\textsc{Delphi} & $x_p$  &  $8$ & DELPHI\_1995\_I394052 \cite{Abreu:1995cu} \\
\textsc{Delphi}~\textcolor{red}{(*)} & $|\bf{p}|$  & $8$ & DELPHI\_1995\_I394052 \cite{Abreu:1995cu}  \\
%\textsc{Delphi} & $x_p$ & $23$ &  DELPHI\_1998\_I473409 \cite{Abreu:1998vq}  \\
\textsc{Delphi} & $|\bf{p}|$ & $23$ & DELPHI\_1998\_I473409 \cite{Abreu:1998vq}  \\
\textsc{Delphi} & $N_{p/\bar{p}}/N_{\rm charged}$ & $23$ & DELPHI\_1998\_I473409 \cite{Abreu:1998vq}  \\
\textsc{Opal}~\textcolor{red}{(*)} & $|{\bf{p}}|$  & $37$ &  OPAL\_1994\_S2927284 \cite{Akers:1994ez}   \\
\midrule
\multicolumn{4}{l}{\textit{Baryon spectra}} \\
\textsc{Aleph}~\textcolor{red}{(*)} & $\Lambda^0:~x_p$ & $25$ & ALEPH\_1996\_S3486095 \cite{Barate:1996fi}  \\
\textsc{Aleph} & $\Lambda^0:~\xi \equiv \log(1/x_p)$~(all events) & $22$ & ALEPH\_2000\_I507531 \cite{Barate:1999gb} \\
\textsc{Aleph} & $\Lambda^0:~\xi \equiv \log(1/x_p)$~($2$-jet events) & $22$ & ALEPH\_2000\_I507531 \cite{Barate:1999gb} \\
\textsc{Delphi} & $\Lambda^0$ scaled momentum & $11$ & DELPHI\_1993\_I360638 \cite{Abreu:1993mm} \\
\textsc{Opal} & $\Lambda^0$ scaled energy $x_E$ & $15$ & OPAL\_1997\_S3396100 \cite{Alexander:1996qj} \\ 
\bottomrule
\end{tabular}
\hspace{0.2cm}
\end{adjustbox}
\end{center}
\caption{\label{tab:measurements:protons} 
Measurements used in the optimisation process, and the corresponding number of bins for $p/\bar{p}$ and $\Lambda/\bar{\Lambda}$ momenta. The data is taken from \textsc{Aleph} \cite{Barate:1996fi, Barate:1999gb}, \textsc{Delphi} \cite{Abreu:1993mm, Abreu:1995cu, Abreu:1998vq}, and \textsc{Opal} \cite{Akers:1994ez, Alexander:1996qj}. The measurements marked by \textcolor{red}{(*)} are not used in the four-dimensional tunes (see the text for more details).}
\end{table*}

\begin{table*}[!h]
\setlength\tabcolsep{20pt}
\begin{center}
\begin{adjustbox}{max width=\textwidth}
%\parbox{0.55\textwidth}{
\begin{tabular}{l  l  l l}
\toprule
Dataset & {Measurement} & $N_{\rm bins}$ & Reference \\
\toprule
\multicolumn{4}{l}{\textit{Charged multiplicity}} \\
%\textsc{Aleph} & $N_{\rm ch}$ & $25$ & ALEPH\_1991\_S2435284 \cite{Decamp:1991uz}   \\
%\textsc{Aleph} & $N_{\rm ch}$ & $25$ & ALEPH\_1996\_S3486095 \cite{Barate:1996fi}  \\
%\textsc{Delphi} & $N_{\rm ch}$,~both hemispheres  &  $25$ & DELPHI\_1991\_I324035 \cite{Abreu:1991yc} \\
\textsc{Delphi} & $N_{\rm ch}$,~2-jets, $y_{\text{cut}}=0.01$ & $19$ & DELPHI\_1992\_I334948 \cite{Abreu:1992gp}  \\
\textsc{Delphi} & $N_{\rm ch}$,~2-jets, $y_{\text{cut}}=0.02$ & $19$ & DELPHI\_1992\_I334948 \cite{Abreu:1992gp}  \\
%\textsc{Delphi} & $N_{\rm ch}$,~2-jets, $y_{\text{cut}}=0.04$ & $23$ & DELPHI\_1992\_I334948 \cite{Abreu:1992gp}  \\
\textsc{L3} & $N_{\rm ch}$ & $28$ & L3\_2004\_I652683 \cite{Achard:2004sv}  \\
%\textsc{L3} & $N_{\rm ch}$,~udsc events & $28$ & L3\_2004\_I652683 \cite{Achard:2004sv}  \\
%\textsc{Opal} & $N_{\rm ch}$ & $27$ & OPAL\_1992\_I321190 \cite{Acton:1991aa}  \\
\toprule
\multicolumn{4}{l}{\textit{Charged Particle Momentum}} \\
%\textsc{Delphi} & $x = |p|/|p_{\rm beam}|$  & $23$ & DELPHI\_1996\_S3430090 \cite{Abreu:1996na}  \\
\textsc{Delphi} & $\log(1/x_p)$  & $27$ & DELPHI\_1996\_S3430090 \cite{Abreu:1996na}  \\
\textsc{Delphi} & $|\bf{p}|$ (all events)  & $27$ & DELPHI\_1998\_I473409 \cite{Abreu:1998vq}  \\
%\textsc{Aleph} & $x_p = |p|/|p_{\rm beam}|$ (charged)  & $46$ & ALEPH\_1996\_S3486095 \cite{Barate:1996fi}  \\
\textsc{Aleph} & $\log(1/x_p)$ (charged)  & $42$ & ALEPH\_1996\_S3486095 \cite{Barate:1996fi}  \\
\textsc{L3} & $\log(1/x_p)$ & $40$ & L3\_2004\_I652683 \cite{Achard:2004sv}  \\
\textsc{L3} & $\log(1/x_p)$, $udsc$ events & $40$ & L3\_2004\_I652683 \cite{Achard:2004sv}  \\
\textsc{Opal} & All events $\log(1/x_p)$  & $29$ & OPAL\_1998\_S3780481 \cite{Ackerstaff:1998hz}  \\
\bottomrule
\end{tabular}
\hspace{0.2cm}
\end{adjustbox}
\end{center}
\caption{\label{tab:measurements:charged} 
Same as for Table \ref{tab:measurements:protons} but for the charged multiplicity distributions. Data is taken from \cite{Abreu:1992gp, Achard:2004sv}.}
\end{table*}


\begin{table*}[!t]
\setlength\tabcolsep{9pt}
\begin{center}
\begin{adjustbox}{max width=\textwidth}
%\parbox{0.55\textwidth}{
\begin{tabular}{l  l  l l}
\toprule
Dataset & {Measurement} & $N_{\rm bins}$ & Reference \\
\toprule
\multicolumn{4}{l}{\textit{Mean charged multiplicity}} \\
\textsc{Aleph} & $\langle N_{\rm ch} \rangle$ & $1$ & ALEPH\_1996\_S3486095 \cite{Barate:1996fi}  \\
\textsc{Aleph} & $\langle N_{\rm ch} \rangle$~for $|Y| < 0.5$ & $1$ & ALEPH\_1996\_S3486095 \cite{Barate:1996fi}  \\
\textsc{Aleph} & $\langle N_{\rm ch} \rangle$~for $|Y| < 1.0$ & $1$ & ALEPH\_1996\_S3486095 \cite{Barate:1996fi}  \\
\textsc{Aleph} & $\langle N_{\rm ch} \rangle$~for $|Y| < 1.5$ & $1$ & ALEPH\_1996\_S3486095 \cite{Barate:1996fi}  \\
\textsc{Aleph} & $\langle N_{\rm ch} \rangle$~for $|Y| < 2.0$ & $1$ & ALEPH\_1996\_S3486095 \cite{Barate:1996fi}  \\
\textsc{Delphi} & $\langle N_{\rm ch} \rangle$  & $1$ &  DELPHI\_1996\_S3430090 \cite{Abreu:1996na}   \\
\textsc{Delphi} & $\langle N_{\rm ch} \rangle$~(all events)  & $1$ &  DELPHI\_1998\_I473409 \cite{Abreu:1998vq}   \\
\textsc{Opal} & Mean charged multiplicity  & $1$ &  OPAL\_1992\_I321190 \cite{Acton:1991aa}   \\
\textsc{Opal} & All events mean charged multiplicity  & $1$ &  OPAL\_1998\_S3780481 \cite{Ackerstaff:1998hz}   \\
\toprule
\multicolumn{4}{l}{\textit{Identified Particle multiplicities}} \\
\textsc{Delphi} & Mean $\Lambda^0, \bar{\Lambda}^0$ multiplicity  & $1$ &  DELPHI\_1993\_I360638 \cite{Abreu:1993mm}   \\
\textsc{Delphi} & Mean $\pi^+/\pi^-$ multiplicity  & $1$ &  DELPHI\_1996\_S3430090 \cite{Abreu:1996na}   \\
\textsc{Delphi} & Mean $\pi^0$ multiplicity  & $1$ &  DELPHI\_1996\_S3430090 \cite{Abreu:1996na}   \\
\textsc{Delphi} & Mean $\rho$ multiplicity  & $1$ &  DELPHI\_1996\_S3430090 \cite{Abreu:1996na}   \\
\textsc{Delphi} & Mean $\Lambda^0$ multiplicity  & $1$ &  DELPHI\_1996\_S3430090 \cite{Abreu:1996na}   \\
\textsc{Delphi} & $\langle N_{\pi^\pm} \rangle$~(all events)  & $1$ &  DELPHI\_1998\_I473409 \cite{Abreu:1998vq}   \\
\textsc{Delphi} & $\langle N_{p/\bar{p}} \rangle$~(all events)  & $1$ &  DELPHI\_1998\_I473409 \cite{Abreu:1998vq}   \\
\bottomrule
\end{tabular}
\hspace{0.2cm}
\end{adjustbox}
\end{center}
\caption{\label{tab:measurements:mean} 
Same as for Table \ref{tab:measurements:protons} but for the mean multiplicity of charged particles $\langle N_{\rm ch} \rangle$ and of identified mesons and baryons. Data is taken from \cite{Barate:1996fi, Decamp:1991uz, Abreu:1990cc}. }
\end{table*}

\begin{table*}[!t]
\setlength\tabcolsep{13pt}
\begin{center}
\begin{adjustbox}{max width=\textwidth}
%\parbox{0.55\textwidth}{
\begin{tabular}{l  l  l l}
\toprule
Dataset & {Measurement} & $N_{\rm bins}$ & Reference \\
\toprule
\multicolumn{4}{l}{\textit{Identified particle spectra}} \\
\textsc{Delphi} & $\pi^\pm$ momentum (all events)  & $23$ & DELPHI\_1998\_I473409 \cite{Abreu:1998vq}  \\
\textsc{Aleph} & $\pi^\pm$ momentum (charged)  & $39$ & ALEPH\_1995\_I382179 \cite{Buskulic:1994ft}  \\
\textsc{Opal} & $\pi^\pm$ momentum & $51$ & OPAL\_1994\_S2927284 \cite{Akers:1994ez}  \\
\textsc{Aleph} & $\pi^\pm$ spectrum  & $8$ & ALEPH\_1996\_S3486095 \cite{Barate:1996fi}  \\
\textsc{Delphi} & $\pi^0$ scaled momentum, all events  & $24$ & DELPHI\_1996\_I401100 \cite{Adam:1995rf}  \\
\textsc{Aleph} & $\pi^0$ spectrum  & $23$ & ALEPH\_1996\_S3486095 \cite{Barate:1996fi}  \\
%\textsc{Opal} & $\pi^0$ scaled momentum  & $20$ & OPAL\_1998\_S3749908 \cite{Ackerstaff:1998ap}  \\
\textsc{Opal} & $\pi^0$ scaled momentum, $\log(1/x_p)$  & $20$ & OPAL\_1998\_S3749908 \cite{Ackerstaff:1998ap}  \\
\bottomrule
\end{tabular}
\hspace{0.2cm}
\end{adjustbox}
\end{center}
\caption{\label{tab:measurements:mesons} 
Same as for table \ref{tab:measurements:protons} but for the spectrum of charged and neutral pions. Data is taken from \cite{Abreu:1998vq, Buskulic:1994ft, Akers:1994ez, Barate:1996fi, Adam:1995rf, Ackerstaff:1998ap}.}
\end{table*}

\begin{table*}[!t]
\setlength\tabcolsep{11pt}
\begin{center}
\begin{adjustbox}{max width=1.1\textwidth}
%\parbox{0.55\textwidth}{
\begin{tabular}{l  l  l l}
\toprule
Dataset & {Measurement} & $N_{\rm bins}$ & Reference \\
\toprule
\multicolumn{4}{l}{\textit{$C$-parameter}} \\
\textsc{Aleph} & $C$ parameter (charged) & $24$ & ALEPH\_1996\_S3486095 \cite{Barate:1996fi}  \\
\textsc{Aleph} & $C$-parameter~($E_{\rm CMS} = 91.2~{\rm GeV}$) & $50$ & ALEPH\_2004\_S5765862 \cite{Heister:2003aj}   \\
\textsc{Delphi} & $C$ parameter & $23$ & DELPHI\_1996\_S3430090 \cite{Abreu:1996na}  \\
\textsc{L3} & $C$-parameter, $udsc$ events & $20$ & L3\_2004\_I652683 \cite{Achard:2004sv}  \\
\textsc{Opal} & $C$-parameter at $91$~GeV & $12$ & OPAL\_2004\_S6132243 \cite{Abbiendi:2004qz}  \\
\toprule
\multicolumn{4}{l}{\textit{Thrust}} \\
\textsc{Aleph} & $1-T$ (charged) & $21$ & ALEPH\_1996\_S3486095 \cite{Barate:1996fi}  \\
\textsc{Aleph} & Thrust (charged) & $42$ & ALEPH\_2004\_S5765862 \cite{Heister:2003aj}  \\
\textsc{Delphi} & Thrust, $1-T$ & $20$ & DELPHI\_1996\_S3430090 \cite{Abreu:1996na}  \\
\textsc{L3} & Thrust, $udsc$ events ($91.2~{\rm GeV}$) & $17$ & L3\_2004\_I652683 \cite{Achard:2004sv}  \\
\textsc{Opal} & Thrust, $1-T$, at $91~{\rm GeV}$ & $11$ & OPAL\_2004\_S6132243 \cite{Abbiendi:2004qz}  \\
\bottomrule
\end{tabular}
\hspace{0.2cm}
\end{adjustbox}
\end{center}
\caption{\label{tab:measurements:eventshapes} 
Same as in table \ref{tab:measurements:protons} but for the Thrust and the $C$--parameter. Data is taken from \cite{Barate:1996fi, Heister:2003aj, Abreu:1996na, Achard:2004sv, Abbiendi:2004qz}.}
\end{table*}

%%%%%%%%%%%%%%%%%%%%%%%%%%%%%%%%%%%%%%%%%%%%%%%%%%%%%%%%%%%%%%%%%%%%%%%
\section{QCD uncertainties in anti-matter spectra: Additional plots}
%%%%%%%%%%%%%%%%%%%%%%%%%%%%%%%%%%%%%%%%%%%%%%%%%%%%%%%%%%%%%%%%%%%%%%%

In this section, we should the spectra of antimatter and photons in dark-matter annihilation for $M_\chi = 100$ GeV (figures \ref{fig:spectra:mDM:100:1}--\ref{fig:spectra:mDM:100:2}) and for $M_\chi = 1000$ GeV (figures \ref{fig:spectra:mDM:1000:1}--\ref{fig:spectra:mDM:1000:2}).

\begin{figure}
    \centering
    \includegraphics[width=0.49\linewidth]{Figures/Spectra/AntiP-DM-100GeV.pdf}
    \hfill
    \includegraphics[width=0.49\linewidth]{Figures/Spectra/Posit-DM-100GeV.pdf}
    \vfill
    \includegraphics[width=0.49\linewidth]{Figures/Spectra/AntivE-DM-100GeV.pdf}
    \hfill
    \includegraphics[width=0.49\linewidth]{Figures/Spectra/Gamma-DM-100GeV.pdf}
    \caption{The scaled kinetic energy distribution of anti-protons (left upper panel), positrons (right upper panel), electron antineutrinos (left bottom panel) and photons (right bottom panel) in dark matter annihilation into $q\bar{q}$ (red), $gg$ (green) and $VV$ (blue). Here, the dark matter mass is chosen to be $100$ GeV. For each pane, the dark shaded band corresponds to the parton-shower uncertainties while the light shaded band corresponds to hadronisation uncertainties.}
    \label{fig:spectra:mDM:100:1}
\end{figure}

\begin{figure}[!t]
\centering 
\includegraphics[width=0.49\linewidth]{Figures/Spectra/AntivM-DM-100GeV.pdf}
\hfill 
\includegraphics[width=0.49\linewidth]{Figures/Spectra/AntivT-DM-100GeV.pdf}
\caption{Same as for figure \ref{fig:spectra:mDM:100:1} but for muon antineutrinos (left) and tau antineutrinos (right).}
\label{fig:spectra:mDM:100:2}
\end{figure}

\begin{figure}[!h]
    \centering
    \includegraphics[width=0.49\linewidth, height=10.5cm]{Figures/Spectra/AntivM-DM-1000GeV.pdf}
    \hfill 
    \includegraphics[width=0.49\linewidth, height=10.5cm]{Figures/Spectra/AntivT-DM-1000GeV.pdf}
    \caption{Same as for figure \ref{fig:spectra:mDM:100:2} but for $M_\chi = 1000$ GeV.}
    \label{fig:spectra:mDM:1000:1}
\end{figure}

\begin{figure}[!h]
    \centering
    \includegraphics[width=0.49\linewidth, height=10.5cm]{Figures/Spectra/AntiP-DM-1000GeV.pdf}
    \hfill
    \includegraphics[width=0.49\linewidth, height=10.5cm]{Figures/Spectra/Posit-DM-1000GeV.pdf}
    \vfill
    \includegraphics[width=0.49\linewidth, height=10.5cm]{Figures/Spectra/AntivE-DM-1000GeV.pdf}
    \hfill
    \includegraphics[width=0.49\linewidth, height=10.5cm]{Figures/Spectra/Gamma-DM-1000GeV.pdf}
    \caption{The scaled kinetic energy distribution of anti-protons (left upper panel), positrons (right upper panel), electron antineutrinos (left bottom panel) and photons (right bottom panel) in dark matter annihilation into $q\bar{q}$ (red), $gg$ (green), $VV$ (blue), $HH$ (purple) and $t\bar{t}$ (turquoise). Here, the dark matter mass is chosen to be $1000$ GeV. For each pane, the dark shaded band corresponds to the parton-shower uncertainties while the light shaded band corresponds to hadronisation uncertainties.}
    \label{fig:spectra:mDM:1000:2}
\end{figure}


\clearpage