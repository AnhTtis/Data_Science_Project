%%%%%%%%%%%%%%%%%%%%%%%%%%%%%%%%%%%%%%%%%%%%%%%%
\section{Results}
\label{sec:tunes}
%%%%%%%%%%%%%%%%%%%%%%%%%%%%%%%%%%%%%%%%%%%%%%%%

\begin{figure}[!tb]
    \centering
    \vspace{-0.5cm}
    \includegraphics[width=0.90\linewidth]{Figures/best_fit_1D.pdf}
    \caption{Best fit point for $\texttt{StringZ:aExtraDiquark}$, the corresponding $68\%$ errors and the associated $\chi^2/N_{\rm df}$ for the different observables and their combinations. Here, we show $\Lambda$ scaled momentum/energy (light green), combination of all $\Lambda$ measurements (olive), $p/\bar{p}$ momenta (purple), combination of all $p/\bar{p}$ measurements (red) and the combined measurements of $\Lambda$ and $p/\bar{p}$ momenta (turquoise). More details are can be found in the main text.}
    \label{fig:tunes1D:results}
\end{figure}

%value \textsc{Opal}--1997, we have removed the last two bins in the $x_E$ distribution corresponding to $x_E \geq 0.5$ since including those bins give rise to an unacceptably large value for \texttt{StringZ:aExtraDiquark} of order $1.95$.
The results of the one-parameter fits are displayed in figure \ref{fig:tunes1D:results} which are shown in the form of horizontal bar plots. For the one-dimensional tunes, we show the best-fit point for tunes to ({\it i}) $\Lambda^0$ scaled momentum/energy (green bar) including data from \textsc{Aleph}--1996 \cite{Barate:1996fi}, \textsc{Aleph}--2000 \cite{Barate:1999gb}, \textsc{Delphi}--1993 \cite{Abreu:1993mm}, and \textsc{Opal}--1997 \cite{Alexander:1996qj}  and to ({\it ii}) proton (scaled)-momentum (blue bar) which includes data from \textsc{Aleph}--1996 \cite{Barate:1996fi}, \textsc{Delphi}--1995 \cite{Abreu:1995cu}, \textsc{Delphi}--1998 \cite{Abreu:1998vq} and \textsc{Opal}--1994 \cite{Akers:1994ez}. The tune of \texttt{StringZ:aExtraDiquark} resulting from a combination of $\Lambda^0$~($p/\bar{p}$) measurements is shown in olive~(red) bar while the one resulting from  a combination of all the measurements is shown in turquoise. Finally, a combination procedure described in section \ref{sec:uncertainty} is also shown with the label Tune--A. We first discuss the impact of the individual measurements on the best-fit point of \texttt{StringZ:aExtraDiquark} and we close this section by a discussion of the combinations. First, we can see that all the individual tunes but the ones using data from \textsc{Aleph}--1996, \textsc{Delphi}--1993, \textsc{Delphi}--1995 and \textsc{Opal}--1994 give results that are consistent with each other; {\it i.e.} the best-fit point floats around $0.8$--$0.93$ with uncertainties of about $0.20$--$0.61$. The tune the \textsc{Opal}--1997 measurement prefers large values of \texttt{StringZ:aExtradDiquark}, of $1.97$\footnote{This is due to the fact that the last two bins, corresponding to $x_E \geq 0.5$, of the $x_E$ distribution forces \texttt{StringZ:aExtraDiquark} to take the largest possible. Removing these two bins will reduce the best-fit point from $1.95$ to $0.91$.}. We note that the consistency between the theory and data for this measurement is almost independent of whether we include or not these two bins in the tune. Finally, we must note that the result of the fit of the proton momentum performed by \textsc{Delphi}--1995 prefers very small values of \texttt{StringZ:aExtraDiquark}. 
%Measurements of the proton low-momentum bins prefer the lowest value used in the prior of \texttt{StringZ:aExtraDiquark} -- here $0.018$. 
\begin{table}[t!]
\setlength\tabcolsep{3pt}
  \begin{center}
    \begin{tabular}{lcccccc}
      \toprule
      Tune     & \verb|aLund| & \verb|avgZLund| & \verb|sigma| & \verb|aExtraDiquark| & \verb|bLund| &  $\chi^2/N_{\rm df}$\\
      \midrule
      \textsc{Aleph}  & $0.758\substack{+0.074\\ -0.074}$ & $0.541\substack{+0.007\\ -0.007}$ & $0.297\substack{+0.005\\-0.005}$ & $1.218\substack{+0.358\\-0.358}$ & $1.040$ & $116.22/296$ \\
      \textsc{Delphi} & $0.358_{-0.054}^{+0.054}$ & $0.497_{-0.007}^{+0.007}$ & $0.287_{-0.006}^{+0.006}$ & $0.782_{-0.298}^{+0.298}$ & $0.533$
      &  $144.37/268$ \\
      \textsc{L3}     &  $0.478_{-0.063}^{+0.063}$  & $0.557_{-0.006}^{+0.006}$ &  $0.315_{-0.007}^{+0.007}$ & $1.998_{-0.049}^{+0.049}$ & $0.897$ & $84.70/140$ \\
      \textsc{Opal}   & $0.588_{-0.086}^{+0.086}$ & $0.536_{-0.005}^{+0.005}$ &     $0.300_{-0.005}^{+0.005}$    &  $1.998_{-0.204}^{+0.204}$ & $0.872$ &     $53.54/136$ \\
      \midrule
      \textsc{Combined} &       $0.601_{-0.038}^{+0.038}$ & $0.540_{-0.004}^{+0.004}$ &  $0.307_{-0.002}^{+0.002}$ & $1.671_{-0.196}^{+0.196}$ & $0.897$ &     $676.69/852$ \\
      \bottomrule
    \end{tabular}
  \end{center}
    \caption{\label{tab:experiments} Results of the tunes performed 
    separately to all the considered measurements from a given experiment.} %The COMBINED result corresponds to the 
    %T2 tune given in Table \ref{tab:T2tune}. }
\end{table}

\begin{figure}[!h]
    \centering
    \vspace{-1cm}
    \includegraphics[width=0.80\linewidth]{Figures/contours_combined_frequentist.pdf}
    \caption{Results of tunes performed to all the data shown in Table  \ref{tab:measurements:protons}-\ref{tab:measurements:eventshapes}. The results are projected on different parameters we used in the four-dimensional parameter space tune. The contours corresponding to $68\%$, $95\%$ and $99.5\%$ confidence levels are shown in yellow, orange, and red respectively.}
    \label{fig:tune:results}
\end{figure}

We turn now to a discussion about the combination procedure. First, we dot not include the measurement of $p/\bar{p}$ spectrum by \textsc{Opal}--1994 since it is inconsistent with the others for $x_p > 0.1$~(see figure \ref{fig:comparisons})\footnote{We have checked that the value of $\texttt{StringZ:aExtraDiquark}$ at the minimum is almost unaffected if the OPAL\_1994\_S2927284 measurement of the proton ($p/\bar{p}$) momentum is added to the fit. One must note that the net effect of including that measurement is worsening the quality of the fit since $\chi^2/N_{\rm df}$ increases by almost a factor of one. This is unsurprising due to the fact that the OPAL measurement itself is inconsistent with the other experimental measurements of proton (scaled)-momentum.}. The measurements of $p/\bar{p}$ and $\Lambda^0$ scaled momenta performed by \textsc{Aleph}--1996 have been removed from the combination as well. There are two reasons for this choice: first, the tuning of \texttt{StringZ:aExtraDiquark} to the corresponding $p/\bar{p}$ measurement prefers small value of about $0.42$ (see figure \ref{fig:tunes1D:results}) and the agreement between theory and data at the best-fit point is not good enough ($\chi^2 \simeq 44$ even after including the $5\%$ theory uncertainty). Second, the measurement of $\Lambda^0$ scaled momentum has already been superseded by a more recent one performed by \textsc{Aleph}--2000 \cite{Barate:1999gb} which includes more data. On the other hand, the tuning of \texttt{StringZ:aExtraDiquark} to \textsc{Aleph}--1996 measurement of $\Lambda^0$ scaled momentum prefers large value of about $1.72$ as shown in figure \ref{fig:tunes1D:results}. The measurements of $p/\bar{p}$ (scaled) momentum by \textsc{Delphi}--1995 \cite{Abreu:1995cu} have been removed since these measurements have already been superseded by the more recent \textsc{Delphi}--1998 \cite{Abreu:1998vq} measurement which covers a wider range of momenta $p \in [0.5, 45]~{\rm GeV}/c^2$.

After all these considerations, two measurements of $p/\bar{p}$ and four measurements of $\Lambda$ (scaled) momenta have been used in the combinations. The result of the tune from a combination of $\Lambda$ measurement only is about $1.26\pm 0.13$ with a $\chi^2/N_{\rm df}$ of order $1.2$. The combination of the $p/\bar{p}$ scaled momentum and $N_{p/\bar{p}}/N_{\rm charged}$ measurements gives a best-fit point of $0.84\pm 0.39$ consistent with the previous combination within the quoted uncertainties. Finally, we note that the combination of all the six mentioned measurements give a best-fit point of $1.22\pm 0.12$ with a good value of $\chi^2/N_{\rm df} \sim 0.91$.

%\begin{figure}[!t]
%    \centering
%    \includegraphics[width=0.49\linewidth]{Figures/BestFit_aExtraDiquark_Limit.pdf}
%    \includegraphics[width=0.49\linewidth]{Figures/BestFit_GoF_Limit.pdf}
%    \caption{Dependence of the best-fit point of $\texttt{StringZ:aExtraDiquark}$ on the maximum prior value of the $\texttt{StringZ:aExtraDiquark}$ parameter used in the fit. The results are shown for tunes including data from \textsc{Aleph} (blue), \textsc{Delphi} (red), \textsc{L3}  (green), \textsc{Opal} (turquoise) and their combination (gray). The error bars correspond to the MIGRAD errors on the best-fit points. The goodness-of-fit is mildly dependent on the prior of $\texttt{StringZ:aExtraDiquark}$ used in the fit. The MC response has been modeled with a fourth order interpolation polynomial.}
%    \label{fig:dependence:fit}
%\end{figure}

Now, we discuss the results of the four-dimensional parameter space tuning. The results of the fits are shown in table \ref{tab:experiments} for individual experiments and their combinations. We can see that the best-fit points of \texttt{StringZ:aLund}, \texttt{StringZ:avgZLund} and \texttt{StringPT:sigma} are consistent with the results of a previous study \cite{Amoroso:2018qga}. On the other hand, the best-fit point of \texttt{StringZ:aExtraDiquark} is larger than what we found in the one-dimensional parameter space tune. Note that this value is driven by the results from two experiments: \textsc{L3} and \textsc{Opal}. The correlations are found to be 
\begin{eqnarray}
C_{ij} =
\left(
%\begin{pmatrix}
\begin{array}{rrrr}
1.000 & 0.718 & 0.057 & 0.415 \\
0.718 & 1.000 & -0.270 & 0.816 \\
0.057 & -0.270 & 1.000 & -0.204 \\
0.415 & 0.816 & -0.204 & 1.000
\end{array}
%\end{pmatrix},
\right),
\label{eq:correlation}
\end{eqnarray}

\begin{figure}[!tbp]
    \centering
    \includegraphics[width=0.80\textwidth]{Figures/Tunes-combined_corner.pdf}
    \caption{One- and two-dimensional marginalized posterior distributions for the uni-modal four-dimensional parameter space fit. Here, the contours show the $68\%$ and $95\%$ Bayesian credible intervals. The results include all the measurements listed in Tables \ref{tab:measurements:protons}-\ref{tab:measurements:eventshapes}.}
    \label{fig:bayesian:results:combined}
\end{figure}

where the coefficients are ordered as follow $\{a, \langle z_\rho \rangle, \sigma_\perp, a_{\rm Diquark}\}$. It is clear from equation \ref{eq:correlation} that the correlation between the fragmentation function parameters are small except that \texttt{StringZ:avgZLund} is extremely highly correlated with both \texttt{StringZ:aLund} and \texttt{StringZ:aExtraDiquark}. In figure \ref{fig:tune:results} we show the $68\%$, $95\%$ and $99.5\%$ CL intervals projected on the different fragmentation function parameters. This is a clear visualization of the results of equation \ref{eq:correlation} and Table \ref{tab:experiments}. As was pointed in the previous subsection, we have found that different measurements prefer different values of the \texttt{StringZ:aExtraDiquark} parameter. We have also checked that these results do not correspond to flat directions as the best-fit point of \texttt{StringZ:aExtraDiquark} does not significantly depend on the choice of the prior (unless the prior is too close to the best-fit point). We finally note that these tune results give fairly good agreement with data and the results are competitive with the baseline \textsc{Monash} tune.

We close this section by showing the results of the Bayesian fit which constitutes a very good cross-check of the results of the frequentist analysis. These results are shown in figure \ref{fig:bayesian:results:combined} where we can see a very good agreement with those of the frequentist fit shown in figure \ref{fig:tune:results}. The results of the Bayesian fit along with the $95\%$ credible levels are shown below:
\begin{eqnarray*}
   \verb|StringZ:aLund| &=&  0.60 \pm 0.07, \\
   \verb|StringZ:avgZLund| &=& 0.54 \pm 0.01, \\
   \verb|StringPT:sigma| &=& 0.31 \pm 0.01, \\
   \verb|StringZ:aExtraDiquark| &=& 1.68^{+0.28}_{-0.38},\\
\end{eqnarray*}
which agrees very well with the results of the frequentist fit.