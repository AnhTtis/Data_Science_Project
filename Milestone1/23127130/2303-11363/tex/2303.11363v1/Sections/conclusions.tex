%%%%%%%%%%%%%%%%%%%%%%%%%%%%%%%%%%%%%%%%%%%%%%%%
\section{Conclusions}
\label{sec:conclusions}
%%%%%%%%%%%%%%%%%%%%%%%%%%%%%%%%%%%%%%%%%%%%%%%%

In this work, we have studied the QCD uncertainties on antimatter spectra from dark-matter annihilation and therefore completing the series of analyses related to QCD uncertainties in dark-matter indirect searches where we studied gamma-ray spectra in ref. \cite{Amoroso:2018qga} and antiproton spectra in ref. \cite{Jueid:2022qjg}.  After studying the general features of antimatter production in dark-matter annihilation within \textsc{Pythia}~8, we studied in detail the origin of antimatter for various annihilation channels taking $M_\chi = 1000$ GeV as an example. We found that the spectra of antimatter can be modeled correctly if the spectrum of charged pions, antiprotons and hyperons measured at LEP is modeled properly. We have performed a detailed analysis of baryon production at LEP (especially antiprotons and hyperons). We have found some tensions between the different measurements at LEP and selected the most reliable measurements for our tunes. Then, we have compared between the predictions of the state-of-art MC event generators, {\it i.e.}, \textsc{Herwig}~7, \textsc{Pythia}~8 and \textsc{Sherpa}~2. We found that MC event generators based on the string hadronisation model have a good agreement with data while the MC event generators based on cluster hadronisation model have very poor agreement with the experimental measurements with disagreement reaching up to $50\%$ in some kinematic regions. This comparison also suggests that the envelope spanned between the different MC event generators cannot define a faithful estimate of the QCD uncertainties since for photons and charged pions they agree pretty well in the peak region (see \cite{Amoroso:2018qga}) while for protons and hyperons they have very large differences. Therefore, we studied the alternative scenario where we estimate the QCD uncertainties within \textsc{Pythia}~8. \\

First we performed several returnings of the fragmentation-function parameters in \textsc{Pythia}~8 with the \textsc{Monash} 2013 tune as our baseline and using a set of constraining measurements at LEP totalling 48 measurements and 856 bins. The resulting tune yields very good agreement with the experimental measurements with $\chi^2/N_{\rm df} \simeq 1$ for most of the observables. We then estimated the QCD uncertainties that arise from parton-shower scale evolution variation and from hadronisation modeling using some parametric variations around the best-fit points of the hadronisation model. We studied the impact of these uncertainties on antimatter and photon spectra. We found that the QCD uncertainties are highly dependent on the annihilation channel, the DM mass,  the particle specie and the energy region. A notable example is the antiproton spectrum where the hadronisation uncertainties dominate the particle-physics error budgets and can reach $10\%$--$20\%$ in the bulk and the peak of spectra and up to $50\%$ in the high-$x$ region. The QCD uncertainties on the other antimatter species are highly dependent on the annihilation final state but are around $10\%$--$15\%$ depending on the annihilation channel and DM mass. \\ 

We finally analysed the impact of these QCD uncertainties on the DM indirect detection fits using realistic CR propagation models for antiprotons, electron antineutrinos and photons. We have considered various annihilation channels that lead to hadronic final states -- $u\bar{u}$, $d\bar{d}$, $s\bar{s}$, $c\bar{c}$, $b\bar{b}$, $t\bar{t}$, $W^+ W^-$, $ZZ$, $hh$, and $gg$ --. For antiprotons, we found that the QCD uncertainties impact the DM masses by up to $\Delta M_\chi = 18$--$32$ GeV depending on the annihilation channel. For the electron antineutrinos and photons the effects are much more different and can go anywhere between $0$ GeV to $43$ GeV depending on the annihilation channel and the kinematic region used in the fits. The effects on $\langle \sigma v \rangle$ and $\Delta S$ were also studied where we found important consequences. The size of QCD uncertainties are negligibly dependent on the choice of the diffusion parameters. Further measurements of the antimatter spectra at the Large Hadron Collider can be very important as they will deliver additional information that are necessary to improve the theory predictions and reduce the associated uncertainties. Therefore, we recommend the DM groups to start using these results for their future analyses. For this purpose, we provide the spectra in tabulated form including QCD uncertainties and some code snippets to perform fast DM fits (can be found in this \href{https://github.com/ajueid/qcd-dm.github.io.git}{github} repository). The tables can also be found in the latest releases of \textsc{DarkSusy}~6 \cite{Bringmann:2018lay},  \textsc{MicrOmegas}~5 \cite{Belanger:2018ccd} and \textsc{MadDM} \cite{Ambrogi:2018jqj}.