

%%%%%%%%%%%%%%%%%%%%%%%%%%%%%%%%%%%%%%%%%%%%%%%%%%%%%%%%%%
\section{Antimatter from dark-matter annihilation}
\label{sec:physics}
%%%%%%%%%%%%%%%%%%%%%%%%%%%%%%%%%%%%%%%%%%%%%%%%%%%%%%%%%%

%%%%%%%%%%%%%%%%%%%%%%%%%%%%%%%%%%%%%%%%%%%%%%%%%%%%%%%%%%
\subsection{General features of stable antiparticle production in \textsc{Pythia}~8}
%%%%%%%%%%%%%%%%%%%%%%%%%%%%%%%%%%%%%%%%%%%%%%%%%%%%%%%%%%

To properly assess the QCD uncertainties on antimatter spectra in DM indirect searches, we first study their production in a generic annihilation or decay process of a dark-matter candidate with mass in the GeV--TeV range. For this, let us consider the following generic annihilation process\footnote{This discussion applies to the case of decaying dark-matter as well. For instance, $\chi \to {\rm SM}~{\rm SM}$ is theoretically possible in models breaking the $Z_2$ symmetry through {\it e.g.} nonminimal interaction of dark matter with gravity (see {\it e.g.} \cite{Cata:2016dsg, Azri:2020bzl, Shaposhnikov:2020aen} for more details).}
\begin{equation}
    \chi \chi \to X_1 X_2 \ldots X_N \to \bigg(\displaystyle\prod_{i=1}^{a_1} Y_{1i} \bigg) \bigg(\displaystyle\prod_{j=1}^{a_2} Y_{2j} \bigg) \ldots \bigg(\displaystyle\prod_{z=1}^{a_N} Y_{Nz} \bigg).
    \label{eq:annihilation}
\end{equation}
We have assumed the narrow-width approximation which enables us to factorise the process into a production part $\chi \chi \to \prod_{i=1}^{N} X_i$ and a decay part $X_i \to \prod_{k=1}^{a_i} Y_{ik}$. In equation \ref{eq:annihilation}, $X_i$ refers to any parton-level SM particle which could be a resonance such as the Higgs boson, $W/Z$-bosons, or the top quark or a non-resonant state like gluons or light quarks. The $X_i$ states are assumed to produce $a_i$ states, through $X_i \to \prod_{k=1}^{a_i} Y_{ik}$, after a complex sequence of processes such as QED bremsstrahlung, QCD parton showering, hadronisation and hadron decays. The produced antiparticles are, therefore, part of this final state and can be detected in indirect detection experiments. Unlike the case of gamma rays, both the total rate as well as the shape of the antiparticle spectra are slightly affected by QED bremsstrahlung. This process occurs if $X_i$ or $Y_i$ contains electrically charged particles or photons and will lead to production of additional photons and/or charged fermions. Besides QED bremsstrahlung, coloured fermions produced in DM annihilation, either promptly or through the decay of heavy resonances, will undergo QCD bremsstrahlung wherein additional coloured particles are produced. This phenomenon is characterised by an enhancement of probabilities for emissions of soft and/or collinear gluons, with the latter being suppressed if the produced fermions are heavy. Furthermore, the rates of $g\to q\bar{q}$ are enhanced at low gluon virtualities: $Q^2 = (p_q + p_{\bar{q}})^2 \to 0$. The rate of QCD radiation processes is mainly controlled by the effective value of the strong coupling constant $\alpha_S$ which is evaluated at a scale proportional to the shower evolution variable\footnote{In \textsc{Pythia}~8, the shower evolution scale is the transverse momentum of the branching parton.}. We note that the value of $\alpha_S(M_Z)$ in \textsc{Pythia}~8 is larger than $\alpha_S(M_Z)^{\overline{{\rm MS}}}$ %from that derived in the $\overline{\mathrm{MS}}$-scheme by about $20\%$ 
for two reasons: {\it (i)} the so-called Catani-Webber-Marchesini (CMW) scheme involves a set of universal corrections in the soft limit \cite{Catani:1990rr} which is equivalent to a net $10\%$ increase in the value of $\alpha_S(M_Z)$ and {\it (ii)} an agreement between data and theory in experimental measurements of $3$-jets observables in $e^+ e^-$ collisions at LEP energies is reached if $\alpha_S(M_Z)$ is increased by a further $\sim10\%$ \cite{Skands:2010ak,Skands:2014pea}.  

Finally, at a scale $Q_{\rm IR} \simeq \mathcal{O}(1)~{\rm GeV}$, any coloured particle must hadronise to produce a set of colourless hadrons. This process, called fragmentation is modeled within \textsc{Pythia}~8 with the Lund string model \cite{Andersson:1983ia, Sjostrand:1982fn, Sjostrand:1984ic}. The longitudinal part of the description of the hadronisation process is given by the left-right symmetric {\it fragmentation function}, $f(z)$, which gives the probability for a hadron to take a fraction $z \in [0, 1]$ of the remaining energy at each step of the (iterative) string fragmentation process. The general form can be written as 
\begin{equation}
        f(z,m_{\perp h}) \propto N \frac{(1-z)^a}{z}\exp\left(\frac{-b m_{\perp h}^2}{z}\right)~,
\label{eq:fz}
\end{equation}
where $N$ is a normalisation constant that guarantees the distribution to be normalised to unit integral, and $m_{\perp h}= \sqrt{m_h^2 + p_{\perp h}^2}$ is called the ``transverse mass'', with $m_h$ the mass of the produced hadron and $p_{\perp h}$ its momentum transverse to the string direction, and $a$ and $b$ are tunable parameters which are denoted in \textsc{Pythia}~8 by \texttt{StringZ:aLund} and \texttt{StringZ:bLund} respectively. As was pointed in a previous work \cite{Amoroso:2018qga}, the $a$ and $b$ parameters are highly correlated (in the context of fits to measurements) since the former controls the high-$z$ tail, while the latter mainly controls the low-$z$ one, while most of the data is sensitive to the average $z$ which is given by a combination of the two which does not have a simple analytical expression. A new reparametrisation of the fragmentation function exists for which the $b$ parameter is replaced by $\langle z_\rho \rangle$ which represents the average longitudinal momentum fraction taken by mainly the $\rho$ mesons, {\it i.e.} 
\begin{equation}
\left<z_\rho\right> = \int_0^1 \mathrm{d}z \ z f(z,\left<m_{\perp\rho}\right>)~.
\label{eq:zrho}
\end{equation}

This equation is solved numerically for $b$ at the initialisation stage (which requires that the option \texttt{StringZ:deriveBLund = on}) where the following parameters:
\begin{eqnarray}
\left<m_{\perp\rho}\right>^2 & = & 
m^2_\rho + 2( \mbox{\texttt{StringPT:sigma}})^2~,
\\
\left<z_\rho\right> & = &\mbox{\texttt{StringZ:avgZLund}}~,
\end{eqnarray}
are used. 

 In the string-fragmentation picture, baryons are produced similarly to mesons, by allowing the breaking of strings by the production of diquarks-antidiquark pairs; these can be thought as bound states of two quarks (in an antiduqark) or two antiquarks (in a diquark). This basic picture entails a very strong (anti)correlation of the produced baryons in both flavour and phase space, due to the fact that a baryon originating from a diquark produced in a string breaking is associated with an anti-baryon as the new end-point of the residual string. Experimental measurements of $\Lambda^0 \bar{\Lambda}^0$ correlations by the \textsc{Opal} collaboration \cite{Abbiendi:1998ux} do not find such strong correlations. To address this, the so-called \emph{pop-corn} mechanism was suggested \cite{Andersson:1984af, Eden:1996xi}, in which baryons are produced such that the string breaking occurs with the production of one or more $q\bar{q}$ pairs ``in between'' the diquark-antidiquark pair. This picture enables the production of one or more mesons between two baryons and therefore decrease the correlations between them. Note that in the context of DM indirect searches, the correlation between baryons is not relevant as the produced protons travel for long distances before they reach the detector. In \textsc{Pythia8}, baryon production is controlled by an additional parameter denoted by $a_{\rm Diquark}$ such that the $a$ parameter in $f(z)$ is modified as $a \to a + a_{\rm Diquark}$. The extra parameter relevant for baryon production is denoted in \textsc{Pythia}~8 by \verb|StringZ:aExtraDiquark|. 


%%%%%%%%%%%%%%%%%%%%%%%%%%%%%%%%%%%%%%%%%%%%%%%%%%%%%%%%%%%%%%%%%%%%%%%%%%%
\subsection{The origin of positrons, antineutrinos and antiprotons in a generic dark-matter annihilation process}
%%%%%%%%%%%%%%%%%%%%%%%%%%%%%%%%%%%%%%%%%%%%%%%%%%%%%%%%%%%%%%%%%%%%%%%%%%%
After discussing the general features of antiparticle production in a generic dark-matter annihilation process in \textsc{Pythia}~8, we turn into a detailed investigation of the origin of these particle species. For this task, we consider a generic dark matter with mass of $1000$ GeV and focus on four annihilation channels: $q\bar{q}:~q=u,d,s,b$, $t\bar{t}$, $VV:~V=W,Z$ and $HH$. The reason is that at this mass value, the universality of the fragmentation function implies that all these annihilation channels have approximately the same features. In  the following we assume that $\sigma(\chi\chi \to WW) = \sigma(\chi\chi \to ZZ)$ without any loss of generality.

\begin{figure}[!t]
\centering
\includegraphics[width=0.9\linewidth]{Figures/Source-ElectronAntineutrinos-MDM-1000GeV.pdf}
\caption{The mean contribution to $\bar{\nu}_e$ production in DM annihilation with $M_{\chi} = 10000~{\rm GeV}$. From the left to the right we show the $q\bar{q}$, $t\bar{t}$, $VV$ and $HH$ channels. One shows $\bar{\nu}_e$ produced from the decay of muons (red), $K_L$ (turquoise), $n, \bar{n}$ (olive), $D^0$ (dark blue), $D^+$ (violet), $B^0$ (hot pink), $Z^0$ (light blue) and $W^\pm$ (cyan).}
\label{fig:origin:nue} 
\end{figure}

\begin{figure}[!t]
\centering
\includegraphics[width=0.9\linewidth]{Figures/Source-MuonAntineutrinos-MDM-1000GeV.pdf}
\caption{Same as figure \ref{fig:origin:nue} but for $\bar{\nu}_\mu$. Here, $\bar{\nu}_\mu$ are produced from the decay of muons (red), $\pi^\pm$ (turquoise), $D^0$ (olive), $B^+$ (dark blue), $Z^0$ (purple) and $W^\pm$ (hot pink).}
\label{fig:origin:num}
\end{figure}

\begin{figure}[!h]
\centering
\includegraphics[width=0.9\linewidth]{Figures/Source-TauAntineutrinos-MDM-1000GeV.pdf}
\caption{Same as figure \ref{fig:origin:nue} but for $\bar{\nu}_\tau$. The $\bar{\nu}_\tau$ produced from the decay of $\tau^\pm$, $B^+$, $B_s^0$, $B^0$, $D_s^+$, $W^\pm$ and $Z^0$ are shown in red, turquoise, olive, dark blue, purple, hot pink, and cyan respectively.}
\label{fig:origin:nuta}
\end{figure}



%%%%%%%%%%%%%%%%%%%%%%%%%%%%%%%%%%%%
\subsubsection{Antineutrinos}
%%%%%%%%%%%%%%%%%%%%%%%%%%%%%%%%%%%%

We start with the spectra of antineutrinos and we split them into electron antineutrinos ($\bar{\nu}_e$), muon antineutrinos ($\bar{\nu}_\mu$) and tau antineutrinos ($\bar{\nu}_\tau$) and they are shown in figures \ref{fig:origin:nue}--\ref{fig:origin:nuta}. 

First, most of the $\bar{\nu}_e$ are coming from the decay of $\mu^\pm$ in $\mu^- \to e^- \bar{\nu}_e \nu_\mu$. The contribution of the muons to the rate of $\bar{\nu}_e$ is $90\%$ irrespective of the annihilation channel. This is followed by the contribution of (anti)-neutrons through $n \to p e^- \bar{\nu}_e$ which is about $4\%$--$5\%$ depending on the annihilation channel. The other contributions are small (below $\simeq 2\%$); one note among others $K_L$, $D^0$, $D^+$, $B^0$, $W^- \to e^- \bar{\nu}_e$ and $Z^0 \to \nu_e \bar{\nu}_e$. The last two contributions ($Z\to \nu_e$ and $W^+ \to \nu_e$) are possible in the $t\bar{t}$, $VV$ and $HH$ channels. We note that most of the muons that leads to the $\bar{\nu}_e$ are coming from the decays of charged pions, {\it i.e.}, $\pi^- \to \mu^- \bar{\nu}_\mu$. Charged pions are produced in abundance through fragmentation of quarks/gluons and/or decay of heavier hadrons \cite{Amoroso:2018qga}. Therefore, one can find a direct connection between the modeling of gamma rays and electron antineutrinos. 

The situation is slightly different for $\bar{\nu}_\mu$ where can see that both $\mu^+$ and $\pi^+$ contribute to about $47$--$50\%$ of the total rate while the other sources give negligible contributions (see figure \ref{fig:origin:num}). Similar to $\bar{\nu}_e$, most of the muons that give rise to $\bar{\nu}_\mu$ are coming from the decay of $\pi^+$. Finally, one note that $\bar{\nu}_\tau$ have extremely different type of sources -- the total rate of $\bar{\nu}_\tau$ is negligibly small as compared to $\bar{\nu}_e$ and $\bar{\nu}_\mu$ --. $\bar{\nu}_\tau$ are produced from $\tau^+$, $D_s^0$, $B_s^0$, $B^+$, $B^0$, $W^\pm$ and $Z^0$ with average contributions of about $1\%$--$50\%$ depending on the annihilation channel (see figure \ref{fig:origin:nuta}).

\begin{figure}[!t]
\centering
\includegraphics[width=0.95\linewidth]{Figures/Source-Positrons-MDM-1000GeV.pdf}
\caption{Same as figure \ref{fig:origin:nue} but for $e^+$. Here, $e^+$ are produced from the decay of muons (red), $K_L$ (turquoise), $n$ (olive), $D^0$ (dark blue), $D^+$ (purple), $B^0$ (hot pink), $Z^0$ (light blue) and $W^\pm$ (cyan).}
\label{fig:origin:e}
\end{figure}

%%%%%%%%%%%%%%%%%%%%%%%%%%%%%%%%%
\subsubsection{Positrons}
%%%%%%%%%%%%%%%%%%%%%%%%%%%%%%%%%

In figure \ref{fig:origin:e}, we show the mean contribution of different particles to the rate of $e^+$ in generic DM annihilation for DM mass of $1000$~GeV. We can see that, similarly to $\bar{\nu}_e$ (see figure \ref{fig:origin:nue}), most of $e^+$ are produced from the decay of $\mu^+$ independently of the annihilation channel. The other contributions are similar to the case of $\bar{\nu}_e$ production. 

%%%%%%%%%%%%%%%%%%%%%%%%%%%%%%%%%%%
\subsubsection{Antiprotons}
%%%%%%%%%%%%%%%%%%%%%%%%%%%%%%%%%%%

\begin{figure}[!t]
    \centering
    \includegraphics[width=0.95\linewidth]{Figures/Source-Antiprotons-MDM-1000GeV.pdf}
    \caption{Same as in figure \ref{fig:origin:nue} but for $\bar{p}$. Here, one shows $p/\bar{p}$ produced from QCD fragmentation (red), decay of $\Lambda^0$ (turquoise), $n$ (olive), $\Sigma^+$ (dark blue), $\Delta^+$ (purple), $\Delta^{++}$ (hot pink) and $\Delta^0$ (light blue).}
    \label{fig:origin:p}
\end{figure}

In general (anti-)protons can be split into two categories: \emph{(i)} primary (anti-)protons produced directly from the string fragmentation of quarks and gluons and \emph{(ii)} secondary (anti-)protons produced from the decay of heavier baryons. In figure \ref{fig:origin:p}, we display the mean contributions to the (anti-)proton yields in DM annihilation for $M_\chi = 1000$ GeV. We can see that about $22\%$ of the produced (anti)protons are emerging from $q/g$-fragmentation. On the other hand, the dominant large fraction ($\simeq 50\%$) of (anti)-protons is originated from the decay of (anti-)neutrons in decays $\bar{n} \to \bar{p} e^+ \nu_e$ as can be clearly seen in figure \ref{fig:origin:p}. This is followed by the contribution of five baryons which mainly contribute to secondary protons: $\Lambda^0$, $\Sigma^+$ and $\Delta(1232)$ -- summing contributions from $\Delta^+, \Delta^{0}$ and $\Delta^{++}$ -- contribute to about $2\%$--$7\%$ of the produced (anti)-protons respectively. \\

We conclude that the spectra of antiparticles are correlated to each other with the dominant hadrons to be modeled properly are charged pions, protons and $\Lambda^0$ baryons. We note that in ref. \cite{Amoroso:2018qga}, we have discussed in detail the modeling of pion spectra at LEP and therefore we will not discuss it anymore here (although we do include these measurements in our fits). In the next section, we discuss in detail the measurements of baryon spectra at LEP and assess the agreement between the corresponding theory predictions and experimental measurements for three multi-purpose MC event generators.  