%%%%%%%%%%%%%%%%%%%%%%%%%%%%%%%%%%%%%%%%%%%%%%%%
\section{Introduction}
\label{sec:intro}
%%%%%%%%%%%%%%%%%%%%%%%%%%%%%%%%%%%%%%%%%%%%%%%%

Various theories beyond the Standard Model (SM) predict the annihilation of weakly interacting massive particles (WIMPs) into SM particles that produce stable final-state particles such as photons, anti-protons, neutrinos, or positrons as a result of complex sequences of processes (for a review see {\it e.g.}, \cite{Jungman:1995df,Bergstrom:2000pn,Bertone:2004pz,Feng:2010gw}). Contributions of dark-matter (DM) annihilation products  to the cosmic-ray (CR) fluxes may leave footprints in various experiments like Fermi Large Area Telescope (\textsc{Fermi--LAT}), or the Alpha Magnetic Spectrometer~(AMS). 
%Since the establishment of these experiments, a number of excesses with respect to the astrophysical backgrounds have been observed by \textsc{Fermi--LAT} \cite{Fermi-LAT:2011baq}, \textsc{Pamela} \cite{PAMELA:2013vxg} and \textsc{AMS}--02 \cite{AMS:2016oqu,AMS:2019rhg} experiments.   \\
Secondary CR antiprotons and positrons are important tools to study the nature and the properties of various sources in the galactic region and beyond. The idea of using antiprotons in DM indirect detection is not recent\footnote{In fact, an excess over the astrophysical predictions was reported right after the first evidence for the existence of antiprotons in 1979 \cite{Golden:1979bw} (further confirmation was done in 1981 \cite{Buffington:1981zz}). An attempt to explain this excess was performed shortly after this discovery wherein a massive photino DM with mass of $m_{\tilde{\gamma}} \sim 3~{\rm GeV}$ can reproduce both the correct relic density and the total antiproton flux \cite{Silk:1984zy, Stecker:1985jc}.}. Recently, a measurement of the antiproton flux and the $\bar{p}/p$ flux ratio has been performed by the AMS--02 collaboration at the International Space Station over the rigidity range $1$--$450$~GV  \cite{AMS:2016oqu}. With data collected between 2011 and 2015, about $3.49\times 10^{5}$ antiproton events have been reported which render the statistical uncertainties a very subleading contribution to the total errors in most rigidity regions. Interestingly an excess over the expected background has been reported by several analyses in the rigidity range of $10$--$20$ GV \cite{Cuoco:2016eej,Cui:2016ppb,Cuoco:2017rxb,Reinert:2017aga,Cui:2018klo,Cuoco:2019kuu,Cholis:2019ejx,Lin:2019ljc,Abdughani:2021pdc,Hernandez-Arellano:2021bpt,Biekotter:2021ovi}. It was pointed out that DM with mass of about $m_X \sim 50$--$100$ GeV annihilating predominantly into hadronic final states can explain this excess, with most of the analyses preferring a DM mass of roughly $m_X \sim 60$~GeV. Moreover, similar DM properties (mass range, and thermal annihilation cross sections) have been considered in the so-called gamma ray Galactic Center Excess (GCE) found in data reported by the \textsc{Fermi}--LAT collaboration \cite{Goodenough:2009gk, Vitale:2009hr, Hooper:2010mq, Gordon:2013vta, Hooper:2011ti, Daylan:2014rsa, Calore:2014xka, Abazajian:2014fta,Zhou:2014lva, Caron:2015wda, vanBeekveld:2016hbo, Butter:2016tjc, Karwin:2016tsw, Achterberg:2017emt}. 

While the statistical uncertainties on the antiproton flux are now very small, a proper treatment of systematic uncertainties and their correlations can be very important in DM analyses \cite{diMauro:2014zea,Kappl:2014hha,Kachelriess:2015wpa,Winkler:2017xor,Korsmeier:2018gcy}. It was pointed out that proper treatment of systematic errors may not only reduce the antiproton excess but can even completely exclude it \cite{Heisig:2020nse}. On the other hand, the AMS--02 collaboration has released measurements of both the $e^+/e^-$ ratio and the positron flux that both pointed toward an excess with respect to the astrophysical backgrounds in the region $\simeq 10$--$300$ GeV \cite{AMS:2019rhg} which is consistent with the previous measurements but with smaller error bars. The fact that the observed spectra are not expected in astrophysics has prompted several explanations including DM (see e.g. \cite{Bergstrom:2008gr,Cirelli:2008pk,Bergstrom:2009fa} for details about the different DM possibilities).\\
%As systematic errors being in the few percent level have caused an important shift on the antiproton AMS--02 excess, Quantum Chromodynamics (QCD) uncertainties on the antiproton flux are usually overlooked while they can dominate the particle physics error budget and therefore imply significant impact on DM fits. \\ 

For DM masses above a few GeV, QCD jet fragmentation can be the leading source of antimatter production in DM annihilation. A large number of antineutrinos, positrons and antiprotons can be produced from complex sequences of physical processes that can include resonance decays (if the DM annihilate to intermediate resonances like $W/Z/H$ bosons or the top quark), QED and QCD bremsstrahlung, hadronisation, and hadron decays. The modeling of hadronisation in particle production from DM annihilation is usually done using multi-purpose Monte Carlo (MC) event generators \cite{Buckley:2011ms} which are either based on the string \cite{Artru:1974hr,Andersson:1983ia} or the cluster \cite{Webber:1983if,Winter:2003tt} models. This is due to the fact that hadronisation cannot be solved from first principles in QCD but only using phenomenological models typically involving several free parameters. Proper estimates of QCD uncertainties that stem from hadronisation are seldom rigorously addressed in DM literature. We note that  comparisons between different MC event generators such as \textsc{Herwig} and \textsc{Pythia} have been done in \cite{Cirelli:2010xx,Cembranos:2013cfa}. Using gamma rays as an example, it was found that the differences in particle spectra predicted in different MC event generators can be observed in the tails of the spectra while there is a high level of agreement between them in the bulk of the spectra \cite{Cembranos:2013cfa}. This finding has been confirmed in a previous study where we have used the most recent and widely used MC event generators \cite{Amoroso:2018qga}. This level of agreement is due to the fact that the default parameter sets for the MC models are being optimised to essentially provide ``central'' fits to roughly the same set of constraining data, comprised mostly of LEP measurements \cite{Buckley:2009bj,Buckley:2010ar,Skands:2010ak,Platzer:2011bc,Karneyeu:2013aha,Skands:2014pea,Fischer:2014bja,Fischer:2016vfv,Reichelt:2017hts,Kile:2017ryy}. Therefore, the envelope of different MC event generators does not reliably span the theory uncertainty allowed by data in the bulk of the spectra while it can overestimate the uncertainty in both the high- and low-energy tails of the spectra. To address this issue, we have provided for the {\it first time} a conservative estimate of QCD uncertainties within \textsc{Pythia}~8 on gamma-ray \cite{Amoroso:2018qga} and antiproton \cite{Jueid:2022qjg} production from DM annihilation  (see also \cite{Amoroso:2020mjm, Jueid:2021dlz,Jueid:2022pzm} for short summaries of these studies). In the present work, we aim of to complete these studies by performing a comprehensive analysis of the QCD uncertainties on antimatter production from DM annihilation. 

We first revisit the constraints from LEP measurements on the parameters of the Lund fragmentation function while this time thoroughly discussing the differences between the various measurements of baryon spectra at the $Z$--pole. We then perform several (re)tunings based on the baseline \textsc{Monash} tune \cite{Skands:2014pea} of \textsc{Pythia}~8.244 event generator \cite{Sjostrand:2014zea}. In this paper, we perform for the first time a Bayesian analysis of the fragmentation-function parameters, finding very good agreement with the results of the frequentist fit. We then estimate the various QCD uncertainties both connected to parton-shower modeling and hadronisation. This paper provides also a first unified model for the production of gamma rays, positrons, antineutrinos and antiprotons in DM annihilation within \textsc{Pythia}~8. The uncertainties are found to be important and range from a few percent to about $50\%$ depending on  particle species,  DM mass,  annihilation channel, and particle energy. Therefore, we recommend  DM groups to use the results of this work in their analyses\footnote{The new data tables and code snippets to perform fast DM fits can be found in \href{https://github.com/ajueid/qcd-dm.github.io.git}{github}.}. \\


The remainder of this paper is organised as follows. In section \ref{sec:physics}, we discuss the physics modeling of antimatter production in a generic DM annihilation process. The last part of section \ref{sec:physics} is essential to determine the relevant constraining data at LEP. In section \ref{sec:measurements}, we discuss the relevant experimental measurements of baryon spectra at LEP  and the consistency between the theory predictions and experimental measurements using three state-of-art MC event generators.  The fitting procedure of this work is discussed in section \ref{sec:setup}. In section \ref{sec:tunes} we discuss the results of the various tunings. A comprehensive discussion of the different types of uncertainties and their estimates is done in section \ref{sec:uncertainty} where we also study quantitatively the impact on the energy spectra for a few selected DM masses, annihilation channels and particle species. Section \ref{sec:DMfit} is devoted to the impact of QCD uncertainties on DM indirect detection experiments for the spectra of antiprotons, electron antineutrinos and photons. We draw our conclusions in section \ref{sec:conclusions}.
