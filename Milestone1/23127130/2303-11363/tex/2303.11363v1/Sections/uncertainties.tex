%%%%%%%%%%%%%%%%%%%%%%%%%%%%%%%%%%%%%%%%%%%%%%%%
\section{Uncertainty estimates}
\label{sec:uncertainty}
%%%%%%%%%%%%%%%%%%%%%%%%%%%%%%%%%%%%%%%%%%%%%%%%
In this section, we discuss the different sources of QCD uncertainties that may affect the particle spectra from dark-matter annihilation. We start with a discussion of the parton-shower uncertainties and how they are estimated within \textsc{Pythia}~8. The formalism of the shower uncertainty estimates used here is based on the method developed in reference \cite{Mrenna:2016sih}. We then move to the discussion of the hadronisation uncertainties and we estimate their size. The estimate of these uncertainties is done with two different methods: Hessian variations that are widely used in the estimate of parton distribution functions (PDFs) uncertainties (see  {\it e.g.} \cite{Pumplin:2001ct} for more details about the Hessian method) and a manual method which we used in a previous analysis \cite{Amoroso:2018qga}. 
%%%%%%%%%%%%%%%%%%%%%%%%%%%%%%%%%%%%%%%%%
\subsection{Perturbative uncertainties}
%%%%%%%%%%%%%%%%%%%%%%%%%%%%%%%%%%%%%%%%%

\begin{figure}[!t]
    \centering
    \includegraphics[width=0.8\textwidth]{Figures/1D_QCD_uncertainties_WeightedFit.pdf}
    \caption{The tunes results of the \texttt{StringZ:aExtraDiquark} parameter performed separately to each of the eight measurements of proton or $\Lambda$ spectra. The weighted average of the tunes to the individual measurements is shown with a black line while the green shaded area corresponds to the $68\%$ CL interval on \texttt{StringZ:aExtraDiquark}.}
    \label{fig:weightedfit:result}
\end{figure}

The perturbative uncertainties are split into catergories: scale-variation uncertainties and non-singular uncertainties. Before digging into the details of the estimate of these uncertainties, we first remind the reader that the parton showers in \textsc{Pythia}~8 are based on a dipole type $p_\perp$--evolution which has been available since \textsc{Pythia}~6.3 \cite{Sjostrand:2004ef} and is used for both QED and QCD emissions. There are strong arguments that the renormalisation scale, at which the parton branchings are estimated, should be equal to the transverse momentum of the branching parton. This scale choice is accompanied by a universal factor that absorbs the leading second-order corrections in the soft-limit \cite{Amati:1980ch, Catani:1990rr}. This usually tends to increase the value of the strong coupling $\alpha_S(M_Z^2)$ by about $10\%$. Therefore, it makes complete sense to estimate uncertainties that are originated from the shower scale variations which define the most important source of perturbative uncertainties. This first class of the shower uncertainties are estimated by varying the renormalisation scale by a factor of two in each direction with respect to the nominal scale choice. Let us consider a variation defined by $\mu_R \equiv p_\perp \to k p_\perp$ with $k = 1/2, 2$. Under this variation, the gluon-emission probability
\begin{eqnarray}
P(t, z) = \frac{\alpha_S(p_\perp)}{2\pi} \frac{P(z)}{t},
\end{eqnarray}
changes into 
\begin{eqnarray}
P'(t, z) = \frac{\alpha_S(k p_\perp)}{2\pi} \bigg(1 + \frac{\alpha_S(\mu)}{2\pi} \beta_0 \ln k\bigg) \frac{P(z)}{t},
\label{eq:splitting}
\end{eqnarray}
where $t = p_\perp^2, z$ is the longitudinal momentum fraction, $P(z)$ is the Dokshitzer-Gribov-Lipatov-Altarelli-Parisi (DGLAP) splitting kernel for the $a\to bc$ branching, $\beta_0 = (11 N_c - 2 n_f)/3$ is the one-loop beta function with $N_c = 3$ being the number of colour degrees of freedom and $n_f$ is the number of active quarks with mass $m_q$ below $\mu_R = p_\perp$. To guarantee that scale variations are as  conservative as possible, a number of modifications to equation \ref{eq:splitting} are in order. First, the scale $\mu$ at which $\alpha_S$ is evaluated (the second term inside the parenthesis) is chosen to be the maximum of the dipole mass $m_{\rm dip}$ and $k p_\perp$; {\it i.e.} $\mu \equiv \mu_{\rm max} = \max({m_{\rm dip}, k p_\perp}$). Second, another factor $\zeta$ that depends on the singularity of the splitting is introduced.  In this case, the second term inside the parenthesis of equation \ref{eq:splitting} is multiplied by $(1 - \zeta)$ so that the correction factor vanishes linearly outside the soft limit. We note that in \textsc{Pythia}~8, a limit on the allowed value of $\alpha_S$ that changes under the scale variation is imposed {\it i.e.} $|\Delta \alpha_S| \leq 0.2$ in order to prevent branchings near the cut-off scale from generating important changes to the event weights. The variations of the non-universal hard components of the DGLAP kernels are also possible with the new formalism. Under these variations, the shower splitting kernels transform as 
\begin{eqnarray}
P(z) \frac{{\rm d}Q^2}{Q^2} \to \bigg(P(z) + \frac{c_{\rm NS} Q^2}{m_{\rm dip}^2}\bigg) \frac{{\rm d}t}{t},
\end{eqnarray}
where $Q^2$ is the virtuality of the parent branching parton, and $c_{\rm NS}$ is a factor that corresponds to the variations -- by default $c_{\rm NS} = \pm 2$ is used but the user is free to change it. We close this discussion by noting that Matrix-Element Corrections (MECs), switched on by default in \textsc{Pythia}~8, lead to very small variations of the non-singular terms of the DGLAP splittings. It was found that switching off these corrections would lead to comprehensively larger envelopes \cite{Mrenna:2016sih}. We have checked that this the case but the error band from non-singular variations without MECs strongly overlaps with those from scale variations. Therefore, if the total perturbative uncertainty is taken as the envelope of the variations and not their sum in quadrature then there is no major difference between switching MECs on or off.

%%%%%%%%%%%%%%%%%%%%%%%%%%%%%%%%%%%%%%
\subsection{Fragmentation function uncertainties}
%%%%%%%%%%%%%%%%%%%%%%%%%%%%%%%%%%%%%%
\subsubsection{Weighted fit} 
%%%%%%%%%%%%%%%%%%%%%%%%%%%%%%%%%%%%%%
This method consists of taking the values of the best-fit points of the $a_{\rm Diquark}$ parameter from eight different measurements -- shown in figure \ref{fig:tunes1D:results} and estimate a new combined best-fit point (weighted best-fit point) and the associated error. In what follows, we do not take the results from the fits to the measurement of the (anti-)proton momentum by \textsc{Delphi}--1995 \cite{Abreu:1995cu} and \textsc{Opal}--1994 \cite{Akers:1994ez}. Let us denote the best-fit point of \texttt{StringZ:aExtraDiquark} from a measurement $i$~(here $i=1,\cdots,8$) by $x_i \pm \sigma_i$ (with $\sigma_i$ being the \texttt{MIGRAD} error on $x_i$). If one assumes a Gaussian probability distribution function for $x_i$ and no correlation between the different measurements, we define a global $\chi^2$ of the variable $x$ as 
\begin{eqnarray}
\chi^2 = \sum_{i=1}^8 \bigg(\frac{x - x_i}{\sigma_i}\bigg)^2.
\label{eq:chi2:combo}
\end{eqnarray}
The combined best-fit point value of \texttt{StringZ:aExtraDiquark} is obtained by minimizing the $\chi^2$ measure defined in equation \ref{eq:chi2:combo}:
\begin{eqnarray}
\frac{\partial \chi^2}{\partial x}\bigg|_{x = \hat{x}} = 2 \sum_{i=1}^8 \bigg(\frac{\hat{x} - x_i}{\sigma_i^2}\bigg) = 0.
\end{eqnarray}
Solving this equation will give
\begin{eqnarray}
\hat{x} = \frac{\sum_i x_i \sigma^{-2}_i}{\sum_i \sigma_i^{-2}} = 1.03,
\end{eqnarray}
where the numerical value is obtained by using the results shown in figure \ref{fig:tunes1D:results}. The error on $\hat{x}$ is obtained by differentiating $\chi^2$ two times with respect to $\hat{x}$. In this case, we obtain
\begin{eqnarray}
\hat{\sigma}^2 = \frac{1}{\sum_i \frac{1}{\sigma_i^2}}.
\end{eqnarray}
This basic approach leads to a small value of $\hat{\sigma}$ that will be mainly controlled by the measurement with the smallest $\sigma_i$. In principle, this approach does not define a conservative estimate of $\hat{\sigma}$ as we can see clearly that the best-fit points $x_i$ are not stable and strongly depends on the measurement being used. Therefore, in order to improve this basic approach we inflate the error $\hat{\sigma}$ by a reasonable factor which we choose to be five. The result of this approach is depicted in figure \ref{fig:weightedfit:result} where we can see the green shaded area which corresponds to the combined errors passes through most of the best-fit points. Therefore, the new combined result along with the corresponding uncertainty is expected to provide a very good agreement with almost all the measurements within the error bands.  The final result is given by
\begin{eqnarray}
\texttt{StringZ:aExtraDiquark} = 1.03 \pm 0.59.
\label{eq:aDi:weighted}
\end{eqnarray}
To get out a comprehensive uncertainty bands from the variations of the fragmentation function parameters we consider all the possible variations obtained from the errors depicted in equations \ref{eq:tune2018} and \ref{eq:aDi:weighted} and not only the correlated ones: $\{p_1, p_2, p_3, p_4\} = (++++), (----)$. For a parameter space of dimension four, the total number of variations, excluding the nominal value, is given by $N = 4^3 - 1 = 63$ variations. However, given that the sum of \texttt{StringZ:aLund} and \texttt{StringZ:aExtraDiquark} may be thought as a new effective $a_{\rm eff}$ parameter, we can consider their correlated variations, {\it i.e.} they will be changed in the same direction, we end up with $3^3 - 1 = 26$ possible variations. 

\begin{table}[!h]
 \setlength\tabcolsep{4pt}
 \begin{center}
\begin{adjustbox}{max width=\textwidth}
  \centering
    \begin{tabular}{l c c c c}
    \toprule
    \toprule
Tune	& \texttt{StringZ:aLund} &	\texttt{StringZ:avgZLund} & \texttt{StringPT:sigma}	& \texttt{StringZ:aExtraDiquark} \\
\toprule
Central	& $0.601$	& $0.540$ &	$0.307$	& $1.671$ \\
\toprule
\multicolumn{5}{l}{\textit{$1\sigma$ eigentunes}} \\
\toprule
Variation $1^+$	& $0.608$ &  $0.542$ &  $0.307$ &  $1.771$ \\
Variation $1^-$	& $0.592$ & $0.538$ &  $0.307$ &	$1.568$ \\
Variation $2^+$	& $0.498$ &	$0.535$ &  $0.306$ &	$1.679$ \\
Variation $2^-$	& $0.701$ & $0.544$ &  $0.309$ &    $1.662$ \\
Variation $3^+$	& $0.599$ &	$0.575$ &  $0.321$ &	$1.671$ \\
Variation $3^-$	& $0.602$ &	$0.506$ &  $0.295$ &	$1.671$ \\
Variation $4^+$	& $0.601$ & $0.511$ &  $0.384$ &	$1.671$ \\
Variation $4^-$	& $0.600$ & $0.563$	& $0.245$  &    $1.671$ \\
\toprule
\multicolumn{5}{l}{\textit{$2\sigma$ eigentunes}} \\
\toprule
Variation $1^+$	& $0.609$ & $0.542$ &  $0.307$ &    $1.775$ \\
Variation $1^-$	& $0.591$ & $0.538$ &  $0.307$ &	$1.558$ \\
Variation $2^+$	& $0.501$ &	$0.535$ &  $0.306$ &	$1.679$ \\
Variation $2^-$	& $0.700$ & $0.544$ &  $0.308$ &    $1.662$ \\
Variation $3^+$	& $0.597$ &	$0.609$ &  $0.333$ &	$1.670$ \\
Variation $3^-$	& $0.603$ &	$0.474$ &  $0.283$ &	$1.671$ \\
Variation $4^+$	& $0.601$ & $0.478$ &  $0.475$ &	$1.672$ \\
Variation $4^-$	& $0.600$ & $0.581$	&  $0.197$ &    $1.669$ \\
\toprule
\multicolumn{5}{l}{\textit{$3\sigma$ eigentunes}} \\
\toprule
Variation $1^+$	& $0.609$ & $0.542$ &  $0.307$ &    $1.780$ \\
Variation $1^-$	& $0.590$ & $0.538$ &  $0.307$ &	$1.543$ \\
Variation $2^+$	& $0.500$ &	$0.535$ &  $0.306$ &	$1.679$ \\
Variation $2^-$	& $0.700$ & $0.544$ &  $0.309$ &    $1.662$ \\
Variation $3^+$	& $0.595$ &	$0.642$ &  $0.345$ &	$1.669$ \\
Variation $3^-$	& $0.605$ &	$0.447$ &  $0.272$ &	$1.672$ \\
Variation $4^+$	& $0.602$ & $0.445$ &  $0.562$ &	$1.673$ \\
Variation $4^-$	& $0.599$ & $0.595$	& $0.158$  &    $1.669$ \\
\toprule
\bottomrule
\end{tabular}
\hspace{0.2cm}
\end{adjustbox}
\end{center}
    \caption{The Hessian variations (eigentunes) for the nominal tune including all the measurements performed by \textsc{Aleph}. The variations correspond to $\Delta \chi^2 = 1$ ($68\%$ CL), $\Delta \chi^2 = 4$~($95\%$ CL) and $\Delta \chi^2 = 9$~($99\%$ CL) with $\Delta \chi^2$ is defined as $\Delta \chi^2 \equiv \chi^2_{\rm var} - \chi^2_{\rm min}$.}
    \label{tab:eigentunes}
\end{table}

%%%%%%%%%%%%%%%%%%%%%%%%%%%%%%%%%%%%%%%%%%%%%%%%%%%%%
\subsubsection{Hessian errors (eigentunes)} 
%%%%%%%%%%%%%%%%%%%%%%%%%%%%%%%%%%%%%%%%%%%%%%%%%%%%%
The \textsc{Professor} toolkit provides an estimate of the uncertainties on the fitted parameters through the Hessian method (also known as eigentunes). This method consists of a diagonalisation of the $\chi^2$ covariance matrix near the minimum
\begin{eqnarray}
\Delta \chi^2 = \sum_i \sum_j H_{ij}(x_i, x_j) (x_i - x_i^0)(x_j - x_j^0),
\end{eqnarray}
with $H_{ij}$ being the Hessian matrix which consists of second-order derivatives of the covariance matrix with respect to the parameters near the minimum , {\it i.e} $H_{ij} = \partial^2 \chi^2/\partial x_i \partial x_j$. The problem then consists of diagonalising the $H_{ij}$ matrix in the space of the optimised parameters, {\it i.e.} finding the principal directions or eigenvectors and the corresponding eigenvalues. This results in building a set of $2 \cdot N_{\rm params}$ variations. These variations are then obtained as corresponding to a fixed change in the goodness-of-fit measure which is found by imposing a constraint on the maximum variation, defined as a hypersphere with maximum radius of $T$ (defined as the tolerance), i.e. $\Delta \chi^2 \leq T$. Therefore one can define the $\Delta \chi^2$ to match a corresponding confidence level interval; {\it i.e.} one-sigma variations are obtained by requiring that $\Delta \chi^2 \simeq N_{\rm df}$ where $N_{\rm df}$ is the number of degrees-of-freedom defined in equation \ref{eq:NDF}. This approach defines a conservative estimate of the uncertainty if the event generator being used has a good agreement with data (which is usually quantified by $\chi^2_{\rm min}/N_{\rm df} \leq 1$) and the resulting uncertainties provide a good coverage of the errors in the experimental data. To enable for this situation, we have added an extra $5\%$ uncertainty to the MC predictions for all the observables and bins (see section \ref{sec:setup}) which already implied a $\chi^2/N_{\rm df} \leq 1$ in our fits as depicted in Table \ref{tab:experiments}. On the other hand, we enable for large uncertainties by considering not only the one-sigma eigentunes but also the two-sigma eigentunes (correspond to $\Delta \chi^2/N_{\rm df} = 4$) and the three-sigma eigentunes (correspond to $\Delta \chi^2/N_{\rm df} = 9$). The three-sigma eigentunes provide a very good coverage of all the experimental uncertianties in the data for meson and baryon spectra but results in unreasonably large uncertainties that overshot the experimental errors for {\it e.g.} event shapes or jet rates.  The variations corresponding to the one-sigma, two-sigma, and three-sigma eigentunes are shown in Table \ref{tab:eigentunes}. \\

\begin{figure}[!t]
    \centering
    \includegraphics[width=0.49\linewidth]{Figures/ALEPH_1996_proton_xP_uncertainties.pdf}
    \hfill
    \includegraphics[width=0.49\linewidth]{Figures/OPAL_1997_Lambda_xE_uncertainties.pdf}
    \caption{Comparison between the theory predictions and the experimental measurement of $p$ spectrum ({\it left}) and the $\Lambda^0$ scaled energy ({\it right}). The nominal predictions correspond to the fit results shown in Table \ref{tab:experiments} (red, green, and blue) and equations \ref{eq:tune2018}, \ref{eq:aDi:weighted} (cyan). The uncertainty envelopes corresponding to the one-sigma, two-sigma, and three-sigma eigentunes are shown in red, green and blue while the uncertainty envelope from the $26$ variations around the best fit points resulting from the weighted fit is shown as cyan shaded area. Data is taken from \cite{Alexander:1996qj, Barate:1996fi}}.
    \label{fig:comparison}
\end{figure}

\begin{figure}[!t]
    \centering
    \includegraphics[width=0.49\linewidth]{Figures/Spectra/AntiP-DM-10GeV.pdf}
    \hfill
    \includegraphics[width=0.49\linewidth]{Figures/Spectra/Posit-DM-10GeV.pdf}
    \vfill
    \includegraphics[width=0.49\linewidth]{Figures/Spectra/AntivE-DM-10GeV.pdf}
    \hfill
    \includegraphics[width=0.49\linewidth]{Figures/Spectra/Gamma-DM-10GeV.pdf}
    \caption{The scaled kinetic energy distribution of anti-protons (left upper panel), positrons (right upper panel), electron antineutrinos (left bottom panel) and photons (right bottom panel) in dark matter annihilation into $q\bar{q}$ (red) and $gg$ (green) and dark matter mass of $10$ GeV. For each pane, the dark shaded band corresponds to the parton-shower uncertainties while the light shaded band corresponds to hadronisation uncertainties.}
    \label{fig:spectra:mDM:10:1}
\end{figure}

\begin{figure}[!t]
    \centering
    \includegraphics[width=0.49\linewidth]{Figures/Spectra/AntivM-DM-10GeV.pdf}
    \hfill 
    \includegraphics[width=0.49\linewidth]{Figures/Spectra/AntivT-DM-10GeV.pdf}
    \caption{Same as for figure \ref{fig:spectra:mDM:10:1} but for the spectra of muon antineutrinos (left) and tau antineutrinos (right).}
    \label{fig:spectra:mDM:10:2}
\end{figure}
In figure \ref{fig:comparison}, we show the comparison between the theory predictions at the best-fit points (including the envelope from theory uncertainties) and the experimental data for the $p$ spectrum ({\it left}) and $\Lambda^0$ scaled energy ({\it right}). We can see that all the theory predictions agree pretty well with data within the uncertainty envelopes where a large uncertainty coverage clearly visible in the case of two-sigma and three-sigma eigentunes. On the other hand, the envelope spanned from the variations around the best-fit point of the weighted fit (shown in cyan) is somehow located as in between the one-sigma and the two-sigma eigentunes. There are, however, some regions in the $p$ spectrum where all the variations seem to cancel each other (specifically in the few bins between $x_p = 0.1$ and $x_p = 0.3$). The resulting uncertainty can range from about $5\%$ in the low $x_E$ region to about $40\%$ in the high $x_E$ region. In order to be as conservative as possible while in the same do not allow for very large variations we can define either the envelopes from the weighted fit or from the two-sigma eigentunes as our estimate of the uncertainty on the baryon spectra. As the uncertainty estimate from the weighted fits is computationally more expensive, we will use the two-sigma eigentunes in the rest of the paper and also for the new data tables that can be found in this \href{https://github.com/ajueid/qcd-dm.github.io.git}{github} repository\footnote{To obtain the anti-matter spectra along with the hadronisation uncertainties from the weighted fit one needs to generate twenty seven MC samples for each dark matter mass. Therefore, we recommend the user who is interested in producing the spectra himself to use the variations provided in Table \ref{tab:eigentunes} which will require nine runs instead of twenty seven and thus significantly reduce the running time.}.

%%%%%%%%%%%%%%%%%%%%%%%%%%%%%%%%%%%%%%%%%%%%%%%%%%%%%%%%%%%%%%%%%%%%
\subsection{Assessing QCD uncertainties on anti-matter spectra}
%%%%%%%%%%%%%%%%%%%%%%%%%%%%%%%%%%%%%%%%%%%%%%%%%%%%%%%%%%%%%%%%%%%%

In this section, we quantify the impact of QCD uncertainties on particle spectra from dark-matter annihilation for few dark matter masses and annihilation channels. The results will be shown in the $x$ variable defined by 
\begin{eqnarray}
x \equiv \frac{E_{\rm kin}}{M_{\chi}} = \frac{E - m}{M_{\chi}},
\end{eqnarray}
with $E_{\rm kin}$ is the kinetic energy of the particle specie (antiproton, positron, neutrino and photon), $m$ is its mass and $M_{\chi}$ is the dark matter mass. We study the following annihilation channels: 
\begin{table}[!h]
    \centering
    \begin{tabular}{l l l}
    $M_\chi$          \hspace{2cm} &   $\chi \chi \to XX$ \hspace{2cm} & Spectra  \\
    $10~{\rm GeV}$     &   $q\bar{q}$, $gg$ & Figures \ref{fig:spectra:mDM:10:1}--\ref{fig:spectra:mDM:10:2} \\
    $100~{\rm GeV}$   &   $q\bar{q}$, $gg$, $VV$ & Figures \ref{fig:spectra:mDM:100:1}--\ref{fig:spectra:mDM:100:2} \\
    $1000~{\rm GeV}$   &   $q\bar{q}$, $gg$, $VV$, $HH$, $t\bar{t}$ & Figures \ref{fig:spectra:mDM:1000:1}--\ref{fig:spectra:mDM:1000:2} \\
    \end{tabular}
%    \caption{Caption}
%    \label{tab:my_label}
\end{table} \\
For the $q\bar{q}$ annihilation channel, we assume that the dark matter is annihilated to all the quarks except the top quark with ${\rm BR}(\chi \chi \to q\bar{q}) = 0.2, q=u,d,s,c,b$. For the $VV$ channel, we include both the $ZZ$ and $W^+ W^-$ channels with equal probabilities: {\it i.e.} ${\rm BR}(\chi \chi\to W^+ W^-) = {\rm BR}(\chi \chi \to ZZ) = 0.5$. The impact of QCD uncertainties on the particle spectra are shown in figures \ref{fig:spectra:mDM:10:1}--\ref{fig:spectra:mDM:10:2} for $M_\chi = 10$ GeV, figures \ref{fig:spectra:mDM:100:1}--\ref{fig:spectra:mDM:100:2}  for $M_\chi = 100$ GeV and figures \ref{fig:spectra:mDM:1000:1}--\ref{fig:spectra:mDM:1000:2} for $M_\chi = 1000$ GeV for the above mentioned annihilation channels. We can see that the QCD uncertainties resulting from parton-shower variations are subleading for dark matter mass of $10$ GeV and especially in the anti-matter spectra. As far as we go to high dark matter masses, for example $1000$ GeV, these uncertainties become more competitive with the hadronisation uncertainties and reach up to $15\%$ in the peak region. The hadronisation uncertainties on the anti-proton spectra are very important and can reach up to $20\%$ in the low energy region and about $10\%$ in the peak region. In the high energy region, both the perturbative and hadronisation uncertainties are important with the latter are dominant with respect to the former and can reach up to $50\%$. Note that the position of the peak changes for some particle species. There are regions where all the variations result in no uncertainty at all, {\it i.e.} $x \simeq 0.2$ in the anti-proton spectra in the $q\bar{q}$ final state. 