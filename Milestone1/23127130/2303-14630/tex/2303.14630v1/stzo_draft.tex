\documentclass[twocolumn,showpacs,superscriptaddress,amsmath,amssymb,aps,prb]{revtex4-2}
\usepackage{CJK}
\usepackage{lipsum}
\usepackage{times}
%\usepackage{caption}
%\usepackage{subcaption}
\usepackage{graphicx}% Include figure files
\usepackage{subfigure}
\usepackage{dcolumn}% Align table columns on decimal point
\usepackage{mathrsfs}% Math symbols
\usepackage{amsmath,bm,amsfonts}
\usepackage{latexsym}
\usepackage{textcomp}
\usepackage{pifont}
\usepackage{gensymb}
\usepackage{epstopdf}
\usepackage[perpage]{footmisc} %  for customizing footnotes
\usepackage[breaklinks,colorlinks = true,linkcolor = blue,urlcolor  = blue,citecolor = red,anchor color = green,bookmarks=true]{hyperref}
\usepackage{url} % url connections------------------------------------------------------------------------------------
\usepackage{hyperref}%add hypertex capabilities

%\newcommand{\gev}{~\mathrm{GeV}}
\newcommand{\tev}{~\mathrm{TeV}}

\newcommand\refeq[1]{Eq.~(\ref{#1})}
\newcommand\refeqs[1]{Eqs.~(\ref{#1})}
\newcommand\refta[1]{Tab.~\ref{#1}}
\newcommand\refse[1]{Sect.~\ref{#1}}
\newcommand\refses[1]{Sects.~\ref{#1}}
\newcommand\citere[1]{Ref.~\cite{#1}}
\newcommand\citeres[1]{Refs.~\cite{#1}}
\newcommand\refap[1]{App.~\ref{#1}}
\def\reffi#1{\mbox{Fig.~\ref{#1}}}

\newcommand{\sigCMS}{2.9}
\newcommand{\muCMS}{0.35}
\newcommand{\dmuCMSpl}{0.12}
\newcommand{\dmuCMSmi}{0.12}

\newcommand\plane[2]{$(#1, #2)$ plane}
%\newcommand\plane[2]{plane $\{#1,#2\}$}


%\newcommand{\SH}[1]{{\color{orange}#1}}
\newcommand{\SH}[1]{{\color{black}#1}}
\newcommand{\SHc}[1]{{\color{red}[SH: #1]}}
%\newcommand{\TB}[1]{{\color{magenta}#1}}
\newcommand{\TB}[1]{{\color{black}#1}}
\newcommand{\TBc}[1]{{\color{red}[TB: #1]}}
\newcommand{\TBalt}[1]{{\color{green}#1}}
%\newcommand{\GW}[1]{{\color{blue}#1}}
\newcommand{\GW}[1]{{\color{black}#1}}
\newcommand{\GWc}[1]{{\color{red}[GW: #1]}}

\newcommand{\myfigwidth}{0.4\paperwidth}

\begin{document}

\setlength{\parskip}{0pt}

\title{Temperature Dependent Phonon Properties in Tetragonal SrTiO$_3$ and Orthorhombic SrZrO$_3$ Perovskites}
\author{P. K. Verma}
\email{pkverma.physics@gmail.com}
\affiliation{Department of Physics, Indian Institute of Science Bangalore, Bangalore 560012, India}


\begin{abstract}
Using first-principles calculations, we have investigated temperature-dependent phonon properties in tetragonal SrTiO$_3$ (STO) and orthorhombic SrZrO$_3$ (SZO) perovskites. Within the quasiharmonic approximation, we have calculated the mode Gr\"uneisen parameters, thermal expansion, and frequency shifts for several optical modes. These shifts show a downward trend with increasing temperature for almost all the modes in both systems and thus add to the normal behavior of frequencies with temperature. Next, within the third-order lattice anharmonic effect, we have investigated the temperature dependence of linewidths and lineshifts for several phonon modes. Except for a few modes, the lineshifts for most of the modes also show a downward trend with temperature and thus further leading to the normal temperature-dependent behavior of phonon modes. However, these lineshifts for a few optical modes in both the systems show an upward trend with increasing temperature, which, in principle, could lead to the anomalous temperature-dependent behavior of the frequencies of these modes; but when we add all the corrections including the quasiharmonic and third-order anharmonic shifts to the frequencies as obtained within the harmonic approximation, the frequencies of almost all the modes behave normally with temperature, i.e. they display blue shifts with cooling. 



\end{abstract}


\maketitle

\section{\label{sec:intro} Introduction}
Perovskites with the ABO$_3$ crystal structure display a wide variety of interesting properties of fundamental as well as technological importance such as ferroelectricity\cite{KMRabePF2007}, superconductivity\cite{JGBednorz1988} or metal-insulator transitions\cite{APRamirezJPCM1997}. Among the perovskite oxide family, Strontium titanate (SrTiO$_3$, from here referred to as STO) has been a center of attraction due to its novel electronic and vibrational properties\cite{TTadanoJCSJ2019}. The cubic phase of STO exhibits a soft phonon mode at $R-$ point of the Brillouin zone (BZ), arising from the TiO$_6$ octahedral antiphase rotation. The freezing of this mode below 110 K, results in a tetragonal lattice distortion (known as anti-Ferro distortion (AFD)), along with a slight stretching of a unit-cell ($c/a > 1$)\cite{UnokiJPSJ1967,JFScottRMP1974,SaiPRB2000}. Such lattice instabilities of cubic STO at zero Kelvin are related to the appearance of the imaginary phonon frequencies\cite{EBlokhinIF2011}. The low-temperature STO structure has the ferroelectric transition around 37 K\cite{KAMullerZPB1991,AKTagantsevPRB2001,VVLemanovFerro2002}. However, despite showing anomalous dielectric properties at low temperatures, it continues to be paraelectric down to zero Kelvin, due to the AFD phase transition and quantum effects\cite{OEKvyatkovskiiPSS2001}. In literature, there exist several studies related to the phonon properties in cubic STO\cite{SaiPRB2000,RWahlPRB2008,CLasotaFerroel1997,ALebedevPSS2009,TTrautmannJPCM2004,NChoudhuryPRB2008,YXieJPCM2008,ABoudaliPRA2009}, where there has been inconsistency on the soft modes at different BZ points (and, if present, whether the calculated frequencies are real or imaginary). On the other hand, concerning the tetragonal STO, there are a few theoretical studies on the phonon properties\cite{SaiPRB2000,EvarestovPRB2011,UAschauerJPCM2014}. Here again, there is no consistency related to the presence of soft modes (if present, whether real or imaginary). 

Strontium zirconate (SrZrO$_3$, from here referred to as SZO) is a material of great technological importance due to its promising mechanical, thermal, chemical, and electrical properties such as high dielectric constant, high breakdown strength, low leakage current density, and wide band gap. These promising characteristics have led SZO to be extensively studied. Thus, SZO finds several practical applications, e.g. solid oxide fuel cells and hydrogen sensors\cite{YAJIMA1992101,LING2002170}, high-voltage and high-reliability capacitors\cite{KathauserNanoTech2004}, optical devices\cite{MOREIRA2009293}, etc. Studies of high-temperature phase transitions have shown that SZO displays three-phase transitions: from the orthorhombic $Pnma (Pbnm)$ to orthorhombic $Cmcm$ at 970-1041 K, then to the tetragonal $I4/mcm$ at 1020-1130 K and finally to cubic $Pm\Bar{3}m$ at 1360-1443 K\cite{CarlssonAC1967,AhteeACSB1976,LingyPRB1996,KennedyPRB1999}. To the best of our knowledge, only a few first-principles calculations deal with the phonon properties in the orthorhombic SZO, where it has been found that this system does not exhibit any imaginary frequencies\cite{AmisiPRB2012,VALI2008497}.

The role of the lattice-anharmonic effects is significant for several physical properties of solids, including temperature-dependent phonon frequencies, lattice thermal expansion, and phase stability of solids\cite{DCWallace1972}. The presence of these effects leads the phonons with a finite lifetime and materials with finite thermal conductivity. The magnitude of the anharmonicity is material specific. Thus, the covalently bonded materials display almost negligible anharmonic effects and consequently show large thermal conductivities\cite{WardPRB2009, LindsayPRB2010}. In contrast, thermoelectric and ferroelectric materials generally exhibit large anharmonicity, obtained from their inelastic neutron scattering spectra and ultra-low thermal conductivity values\cite{ZhaoNature2014, ToshiroRMP2014}.

Generally, the ab initio calculations employing the density functional theory (DFT) approach considers only the lowest-order terms of the phonon self-energies\cite{MaradudinPR1962}. In this paper, we have included only the quasiharmonic and the third-order anharmonic corrections in our calculations to study the temperature-dependent phonon properties in STO and SZO perovskites. We note that the STO exhibits the tetragonal phase only up to 110 K, so we have considered all the temperature-dependent studies only within the temperature range of 0-110 K. However, for SZO, a temperature range of 0-300 K is considered.

 

\section{Computational Details}
\label{sec:computational}
In this paper, we have employed the density functional perturbation theory approach \cite{HohenbergPR1964, KohnPR1965, BaroniRMP2001} as implemented in the QUANTUM ESPRESSO\cite{PaoloJPCM2009}, a PW basis set-based code, to study the lattice dynamical properties in STO and SZO. The optimized norm-conserving Vanderbilt pseudopotentials \cite{VanderbiltPRB1990} were used to represent the interaction between the ionic cores and valence electrons. The LDA method\cite{CeperleyPRL1980}, parameterized by the Perdew-Zunger, was used for the exchange-correlation energy functional. The energy cutoff of 80 Ry was used to describe the Kohn-Sham wave functions in both systems. The k-points summation of the electronic energy calculations over the BZ was carried out using the Monkhorst-Pack method with special k-points mesh\cite{PackPRB1977} of $4\times 4\times 4$ in STO and $3\times 3\times 3$ in SZO. We note that these special k-point grids were selected based on the total energy convergence in both systems. For phonon dispersion calculations, $\bm q$-grids of $4\times 4\times 4$ and $2\times 2\times 2$ were used for STO and SZO, respectively. The lattice constants were optimized to minimize the total energy, interatomic forces, and unit-cell stresses. Within the level of cubic anharmonic approximation, the most important and perhaps computationally challenging as well is the ab-initio calculation of the third-order interatomic force constant matrix elements (IFC3s)\cite{DebernardiPRL1995, DeinzerPRB2003, GonzePRB1989, DebernardiSSC1994}. Here, the $D3Q$ code\cite{PaulattoPRB2013} was employed to evaluate these IFC3s. The advantage of using the $D3Q$ code is that only the linear response of the electronic density to the atomic displacements is enough to get the information of the third-order interatomic force constants. The $\bm q$-grids of $4\times 4\times 4$ and $2\times 2\times 2$ were used to evaluate IFC3s in reciprocal space in STO and SZO, respectively. Using the Fourier transformation, these were then converted into real space. Finally, using these real-space IFC3s and a $\bm q$ mesh of $30\times 30\times 30$, the finer grid-based IFC3S were obtained. The same $\bm q$ grid of $30\times 30\times 30$ was also used in the quasiharmonic approximation to calculate the temperature-dependent linear thermal expansion coefficient.


%
%\vspace{0.05cm}
\section{Structural Details}
\label{sec:structural_details}
%
\subsection{STO structure}
\label{subsec:sto_structure}
As mentioned in sec~\ref{sec:intro}, STO exists in two phases, namely the cubic and tetragonal\cite{LytleJAP1964}. The cubic phase of STO (space group: Pm$\bar{3}$m) is observed above the transition temperature of 110 K\cite{LytleJAP1964, SaifiPRB1970, MullerPRL1971}. Below 110 K, STO exhibits a phase transition from cubic to a non-polar antiferrodistortive (AFD) tetragonal phase (I4/mcm, SG: 140). The primitive cell of tetragonal STO is body-centered and contains only 10 atoms (two formula units of STO)\cite{EvarestovPRB2011}. The I4/mcm space group has three degrees of freedom, one of which is the rotation of the oxygen octahedra, and the other two are the two distinct lattice parameters. The Wyckoff positions for the tetragonal STO phase are shown in Table.~\ref{table:tetra_wyckoff_pos}. 
The strontium atom is located at the 4b site (0, 1/2, 1/4), whereas the titanium atom is located at the 4c site (0, 0, 0). There are two inequivalent positions for the oxygen atom, i.e. $4a$ (0, 0, 1/4) and $8h$ ($u, u+1/2, 0$). The structural parameter $u$ defines the oxygen $8h$ position. The primitive unit cell of the body-centered tetragonal lattice contains two Sr, Ti, and O$_{4a}$ atoms and four $O_{8h}$ atoms. The crystallographic (quadruple) unit cell of the body-centered tetragonal STO is shown in Fig.~\ref{fig:sto_structure_tetragonal}.

\begin{table}[h]
\begin{center}
\caption{Wyckoff positions for tetragonal STO (space group I4/mcm D$^{18}_{4h}$ (SG 140) ).} 
\begin{tabular}{c c c }
\hline
Atom  &  Position   &   Coordinates  \\
\hline
Sr   & 4b  &  ($0,\frac{1}{2},\frac{1}{4}$ ) ($\frac{1}{2},0,\frac{1}{4}$) \\
Ti   & 4c  &  ($0,0,0$) ($0,0,\frac{1}{2}$)\\
O        & 4a  &  ($0,0,\frac{1}{4}) (0,0,\frac{3}{4}$)\\ 
O        & 8h  &  ($u,u+\frac{1}{2},0$) ($-u,-u+\frac{1}{2},0$) \\
         &     &  ($-u+\frac{1}{2},u,0$) ($u+\frac{1}{2},-u,0$) \\
\hline
\end{tabular}\label{table:tetra_wyckoff_pos}
\end{center}
\end{table} 

%
\begin{figure}[h!]
 \centering
 \includegraphics[width=0.3\textwidth]{figs/sto_tconventional}
 \caption{(Color online) Crystal structure of tetragonal STO (quadruple). (Green balls) Sr atoms, (cyan balls) Ti, and (red balls) O atoms.}
 \label{fig:sto_structure_tetragonal}
 \end{figure}

 

\subsection{SZO structure}
\label{subsec:szo_structure}
As discussed in sec~\ref{sec:intro}, the SZO is known to display three phase transitions: Orthorhombic ($Pbnm$) $\rightarrow$ Orthorhombic ($Cmcm$)  $\rightarrow$ Tetragonal ($I4/mcm$) $\rightarrow$ Cubic ($Pm\Bar{3}m$) at 970 K, 1100 K, and 1440 K, respectively. In this paper, our interest is in the low-temperature orthorhombic phase of SZO with the space group symmetry $Pbnm$. The Wyckoff positions of the atoms are shown in Table~\ref{table:szo_ortho_wyckoff_pos}. The Zr atom resides at the origin with the Wyckoff site 4(a) and the Sr atom at $u, 1/2+v, 1/4$ with the Wyckoff site 4(c). There are two inequivalent positions of oxygen atoms, i.e. O(1) at $u, v, 1/4$ with Wyckoff site 4(c) and O(2) at $1/4 - u, 1/4 + v, w$ with Wyckoff site 8(d). The unit cell of the orthorhombic SZO is shown in Fig.~\ref{fig:szo_structure_ortho}.

\begin{table}[h]
\begin{center}
\caption{Atomic displacements $u, v, w$ in orthorhombic SZO according to the space group $Pbnm$ with origin on the Zr atoms. } 
\begin{tabular}{c c c c c}
\hline
 Atom & Wyckoff Position  &  $u$  & $v$   &  $w$  \\
  \hline
  Sr  &  4(c)  &  $u_{Sr}$  &  $\frac{1}{2}+v_{Sr}$  &  $\frac{1}{4}$ \\
  Zr  &  4(a)  &   0   &   0   &   0   \\
  O(1)  &  4(c)  &   $u_{O(1)}$   &  $v_{O(1)}$    &   $\frac{1}{4}$   \\
  O(2)  &  8(d)  &   $\frac{1}{4} - u_{O(2)}$   &  $\frac{1}{4} + v_{O(2)}$   &   $w_{O(2)}$   \\  
\hline
\end{tabular}\label{table:szo_ortho_wyckoff_pos}
\end{center}
\end{table} 

\begin{figure}[h!]
 \centering
 \includegraphics[width=0.3\textwidth]{figs/szo_ortho_structure.png}
 \caption{(Color online) Bulk unit cell of orthorhombic SZO. Cyan spheres represent Sr atoms, Zr by grey, and O by red.}
 \label{fig:szo_structure_ortho}
 \end{figure}
 

\section{Lattice Anharmonic Details}
\label{sec:self-energy}
The temperature dependence of the frequency of $j^{th}$ phonon mode is obtained as
\begin{equation}
\omega_j(T) = \omega_j(0) + \Delta_j^{(qh)}(\omega,T) + \Delta_j^{(3)}(\omega,T) + \dots
\label{eqn:freq_T}
\end{equation}
where $\Delta_j^{(qh)}(\omega,T)$ and $\Delta_j^{(3)}(\omega, T)$ are the quasiharmonic and cubic anharmonic shifts of frequency of $j^{th}$ phonon mode $\omega_j(0)$. 

The quasiharmonic shift $\Delta_j^{(qh)}$ in Eq.~\ref{eqn:freq_T} accounts for an expansion/contraction of the lattice leading to a change in the harmonic force constant without changing the phonon population and is obtained as
\begin{equation}
\Delta_j^{(qh)}(\omega,T) = \omega_j(0)[e^{-3\gamma_j\int_{0}^{T}\alpha(T')dT'} - 1]
\label{eqn:qh}
\end{equation}
where $\gamma_j$ is Gr\"uneisen parameter of the $j^{th}$ phonon mode, which describes the variations of the vibrational properties of a crystal lattice with respect to its volume, and, as a consequence, the effect that changing temperature has on the size or dynamics of the lattice. 

The linear thermal expansion coefficient in Eq.~\ref{eqn:qh} is calculated as\cite{PKVermaPRB2022}
\begin{equation}
\alpha = \frac{1}{3BV}\sum_{\bm q,j}\left[\gamma_j(\bm q)\frac{[\hbar\omega_j(\bm q)]^2}{k_BT^2}\frac{\exp[\hbar\omega_j(\bm q)/k_BT]}{(\exp[\hbar\omega_j(\bm q)/k_BT]-1)^2}\right]
\label{eqn:alpha}
\end{equation}
where B is the zero-temperature and zero-pressure value of the bulk modulus and $\gamma_j(\bm q)$ is the wavevector-dependent mode Gr\"uneisen parameter of the $jth$ phonon mode.

The third-order shift in Eq.~\ref{eqn:freq_T}, $\Delta_j^{(3)}(\omega, T)$ is the real part of the phonon self-energy within the cubic-anharmonic approximation ($\Sigma_j(\omega, T) = \Delta_j^{(3)}(\omega, T) + i \Gamma_j^{(3)}(\omega, T)$, where the linewidth $\Gamma_j^{(3)}(\omega, T)$ is the imaginary part of self-energy of $j^{th}$ mode) and is obtained by utilizing the Kramers-Kr\"{o}nig relation:
  
\begin{equation}
\Delta_j^{(3)}(\omega_j, T) = -\frac{1}{\pi}\mathcal{P} \int_{-\infty}^{\infty}d\omega'\frac{\Gamma^{(3)}(\omega', T)}{\omega' - \omega_j} 
\label{eqn:anh}
\end{equation}  
where $\mathcal{P}$ stands for the principal value of the integral.

The phonon linewidth in Eq.~\ref{eqn:anh} is obtained as\cite{PKVermaPRB2022} 
\begin{eqnarray}
\Gamma_j^{(3)}(\omega, T) &=& \frac{18\pi}{\hbar^2}\sum_{\bm{q},j_1,i_2}|V^{(3)}(\bm 0,j;\bm{q},j_1;-\bm{q},j_2)|^2 \nonumber \\
 & &  \times \{[n(\omega_{j_1}(\bm{q}))+ n(\omega_{j_2}(\bm {-q})) + 1] \nonumber \\
 & & \times \delta[\omega - \omega_{j_1}(\bm{q}) - \omega_{j_2}(\bm{-q})] \nonumber \\
& &+ 2[n(\omega_{j_2}(\bm{-q})) -  n(\omega_{j_1}(\bm {q}))] \times \nonumber \\
& & \delta[\omega - \omega_{j_1}(\bm{q}) + \omega_{j_2}(\bm{-q})]  \}
\label{eqn:Gamma}
\end{eqnarray} 

Where $V^{(3)}(\bm 0,j;\bm{q},j_1;-\bm{q},j_2)$ is the three phonon coupling constant\cite{PaulattoPRB2013}. The first term in curly brackets on the right-hand side of Eq.\ref{eqn:Gamma}, is responsible for the non-equilibrium phonon decaying into two low-energy phonons, whereas the second term describes the process in which a non-equilibrium phonon is destroyed together with a thermal phonon and a phonon of higher energy with respect to the initial one is created.

\paragraph*{}    
The factor $ n(\omega_j(\bm q))$ in Eq.~\ref{eqn:Gamma} is the occupation number of the $j^{th}$ phonon mode with wave-vector $\bm q$ and is given by,  
\begin{equation}
n(\omega_j(\bm q)) = \frac{1}{e^{\hbar\omega_j(\bm q)/k_B T} - 1}
\end{equation}   

The Cauchy principal value integral in Eq.~\ref{eqn:anh} and the Dirac delta functions in Eq.~\ref{eqn:Gamma} were evaluated using the Lorentzian broadening function with a broadening parameter of 3 cm$^{-1}$.


\section{Results and Discussion}
\label{sec:results-discussion}

%\section{Structural and vibrational properties}
%\label{sec:stru_vib}

\subsection{Structural Properties}
\label{subsec:structural}

\subsubsection{STO results}
\label{subsubsec:sto_structural}
The basic structural properties of the tetragonal STO using the PW basis sets and the LDA exchange-correlation functional is shown in Table~\ref{table:tetra_sto_props}. Our theoretical values of the lattice constants $a (= b), c$, the internal positional parameter $u$, and the TiO$_6$-rotation angle are compared with the available literature data. As can be seen from Table~\ref{table:tetra_sto_props},  our calculated values of the lattice constants $a\sim 5.426$ \text{\normalfont\AA} and $c\sim 7.673$ \text{\normalfont\AA}  are in reasonable good agreement with the experimentally obtained values of 5.507 \text{\normalfont\AA} and 7.796 \text{\normalfont\AA}, respectively\cite{JauchPRB1999}. We note that the lattice constants calculated by Evarestov et al.\cite{EvarestovPRB2011} within the GGA method are slightly larger whereas our LDA values are slightly smaller than the experimentally obtained values. This trend in the lattice constants is a known fact in the DFT calculations, where GGA slightly overestimates and LDA underestimates the lattice constants. Next, as is clear from our calculations and in the calculations by Evarestov et al.\cite{EvarestovPRB2011}, the octahedra rotation angle is strongly overestimated by PW. This overestimation in angle arises due to the limitations in capturing tiny structural changes using the PW approach. We note that the octahedra rotation angle is related to the internal structure parameter value $u$. The $u$ value of 0.221 obtained from our PW-LDA method is smaller by  8$\%$ than the experimentally measured value of 0.241. 


\begin{table}[ht]
\centering
\caption{Calculated results of lattice constants ($a$ and $c$), cubic-tetragonal distortion ($c/(\sqrt{2}a)$), O-atom positional parameter ($u$), and TiO$_6$-rotation angle (arctan $(1 - 4u)^\circ$) of STO along with the available literature data.}
\begin{tabular}{c c c c c  }
\hline
\hline
         &  PW-LDA  & LCAO-GGA & PW-GGA & Expt. \\
         & (This Paper) & Ref.\cite{EvarestovPRB2011} & Ref.\cite{EvarestovPRB2011} & Ref.\cite{JauchPRB1999} \\ 
\hline
$a$ (\AA)  & 5.426 & 5.594 & 5.566 & 5.507 \\
\hspace{0.1 cm} $c$ (\AA)  & 7.673 & 7.922 & 7.908 & 7.796  \\ 
$c/(\sqrt{2}a)$ & 1.0  & 1.0014 & 1.0046 & 1.0010  \\
$u$  & 0.221 & 0.245 & 0.228 & 0.241 \\
arctan $(1 - 4u)^\circ$ & 6.6 & 1.1 & 4.9 & 2.0 \\         
\hline
\hline
\end{tabular}
\label{table:tetra_sto_props}
\end{table}
%

We further note that the cubic-tetragonal distortion is characterized by the ratio $c/(\sqrt{2}a)$, where $a$ and $c$ correspond to the lattice parameters of the tetragonal STO. As is clear from Table~\ref{table:tetra_sto_props}, our PW-LDA value of 1.0 is slightly smaller than the experimental value of 1.001. Thus, our theoretical calculations do not capture the cubic-tetragonal distortion quite well. Perhaps, this could be due to the limitation of the PW-LDA approach. The PW-GGA calculations by Evarestov et al. perform slightly better in capturing this distortion. However, PW-GGA slightly overestimates it with 1.0046 from the experimental value of 1.0010. On the other hand, the LCAO-GGA calculations by Evarestov et al. capture the distortion quite well with the value of 1.0014, which is quite close to the experimental value. Thus, it is clear that the low-temperature phase of STO is quite complex in terms of minor structural details, which perhaps makes it challenging to theoretically study its physical properties, especially the lattice dynamical ones. As already discussed in section~\ref{sec:intro} and shown in section~\ref{subsec:vibrational}, there is no consistency in describing the phonon properties of tetragonal STO. It could be related to the inability of theoretical methods to precisely capture the minor structural details of the low-temperature phase of STO. 

\subsubsection{SZO results}
\label{subsubsec:szo_structural}
Calculated lattice constants together with the available literature data are shown in Table~\ref{table:szo_latt_const}. In the literature, there exist several theoretical works describing the structural properties of the orthorhombic SZO\cite{LIU2012425,LIU20102032,Evarestov2005R11,VALI2008497,Longo2009FirstPC,LONGO20082191}. The calculated equilibrium lattice constants are in accordance with the available theoretical and experimental data\cite{PhysRevB.59.4023,CAVALCANTE20071020,Yamanaka20051496}. Next, the calculated values of the atomic positions of orthorhombic SZO are shown in Table~\ref{table:szo_atom_pos_cal}. Here again, the results are quite close to the literature data. The pressure-volume curve was constructed by optimizing the unit cell at each hydrostatic pressure in the range of 0-30 GPa and the bulk modulus was obtained by fitting it with the non-linear third-order Birch-Murnaghan equation of state (BM3). The obtained bulk modulus 172 GPa is very close to the previously calculated value of 170 GPa\cite{LIU2012425}.


\begin{table}[ht]
\centering
\caption{Calculated lattice constants of orthorhombic SZO (in \AA) together with the available theoretical\cite{LIU2012425,LIU20102032,Evarestov2005R11,VALI2008497,Longo2009FirstPC,LONGO20082191} and experimental data\cite{PhysRevB.59.4023,CAVALCANTE20071020,Yamanaka20051496}.}
\begin{tabular}{c c c c }
\hline
\hline
       &  a      &     b    &   c \\ 
\hline
This work (LDA)  &  5.7025  &  5.7619  &  8.0814 \\
Others calculated results  &   &   &  \\
Theory A\cite{VALI2008497}        &  5.652    &  5.664    &  7.995  \\
Theory B\cite{LIU2012425}  &  5.7077   &  5.7669   & 8.0875  \\
Theory C\cite{LIU20102032}        &  5.8118   &  5.8701   &  8.2426 \\
Theory D\cite{Evarestov2005R11}        &  5.847    &  5.911    &  8.295  \\
Theory E\cite{Longo2009FirstPC}        &  5.78     &  5.79     &  8.15  \\
Theory F\cite{LONGO20082191}        &  5.776    &  5.788    &  8.154  \\
Experimental results   &   &    &   \\
Expt1\cite{PhysRevB.59.4023}   &  5.7963   &   5.8171   &  8.2048 \\
Expt2\cite{CAVALCANTE20071020}   &  5.7910   &   5.8108   &  8.1964  \\
Expt3\cite{Yamanaka20051496}   &  5.816    &   5.813    &  8.225  \\  
\hline
\hline
\end{tabular}
\label{table:szo_latt_const}
\end{table}
%
%---------------------------------------
\begin{table*}[ht]
\centering
\caption{Calculated atomic positions of orthorhombic SZO (in \AA), compared with the available theoretical\cite{LIU2012425,VALI2008497,Evarestov2005R11,LiuSSC2010} and experimental data\cite{PhysRevB.59.4023}.}
\begin{tabular}{c c c c c}
\hline
\hline
       &  Sr      &     Zr    &   O(1)   &  O(2) \\ 
\hline
This work (LDA)  &  0.008, 0.535, 0.25  & 0, 0, 0  &  -0.081, -0.022, 0.25  &  0.212, 0.288, 0.043 \\
Others calculated results  &   &   &  & \\
Theory A\cite{VALI2008497}       &  0.007, 0.534, 0.25  &  0, 0, 0  &  -0.107, -0.036, 0.25  &  0.199, 0.301, 0.056 \\
Theory B\cite{LIU2012425}  &  0.008, 0.535, 0.25  &  0, 0, 0  &  0.920, 0.979, 0.25  &  0.213, 0.287, 0.043  \\
Theory C\cite{Evarestov2005R11}       &  0.007, 0.533, 0.25  &  0, 0, 0  &  -0.077, -0.021, 0.25  &  0.213, 0.287, 0.041 \\
Theory D\cite{LiuSSC2010}       &  0.007, 0.519, 0.25  &  0, 0, 0  &  -0.076, -0.0201, 0.25  &  0.2142, 0.2856, 0.0399 \\

Experimental results   &   &    &   & \\
Expt\cite{PhysRevB.59.4023}   &  0.0040, 0.5242, 0.25  &  0, 0, 0  & -0.0687, -0.0133, 0.25  &  0.2154, 0.2837, 0.0363 \\
\hline
\hline
\end{tabular}
\label{table:szo_atom_pos_cal}
\end{table*}

%
\subsection{Harmonic Lattice Dynamical Properties}
\label{subsec:vibrational}
\subsubsection{STO Results}
\label{subsubsec:ph_sto_results}
In the tetragonal-STO, two modes are acoustic ($A_{2u} + E_u$), eight modes are infrared (IR) active (3$A_{2u} + 5E_u$), and seven modes are Raman active ($A_{1g}, B_{1g}, 2B_{2g}, 3E_g$). The silent modes have the symmetry $A_{1u}, B_{1u}$, and $2A_{2g}$. The $A_{1g}$ and $B_{1g}$ Raman active modes involve the vibrations of only the O atoms. The Raman active modes with the symmetry $B_{2g}$ and $E_{g}$ are Sr-O vibrational modes. The vibrations associated with Ti atom displacements are active only in IR spectra ($A_{2u}, 5E_u$ modes).

\begin{table}[ht]
\centering
\caption{Calculated LDA values of frequencies of Raman active, silent, and IR active modes in tetragonal STO structure. The results are compared with the available theoretical and experimental data.}
\begin{tabular}{c c c c c c }
\hline
    &  LDA   &  LCAO-PBE  & PW-PBE  &  PW-LDA  & Expt. \\
    & (This work) & Ref.\cite{EvarestovPRB2011} & Ref.\cite{EvarestovPRB2011} &  Ref.\cite{SaiPRB2000} & \\
\hline
Raman & & & & & \\
$A_{1g}$ & 60 & 29 & 98 &  & 48\cite{FleuryPRL1968} \\
$B_{2g}$ & 61 & 48 & 17i &  & 15\cite{FleuryPRL1968} \\
$E_{g}$ & 146 & 137 & 183 &  & 143\cite{FleuryPRL1968} \\
$E_{g}$ & 152 & 152 & 140 &  & 235\cite{FleuryPRL1968} \\
$B_{2g}$ & 438 & 441 & 421 &  &  \\
$E_{g}$ & 439 & 444 & 425 &  & 460\cite{FleuryPRL1968} \\
$B_{1g}$ & 507 & 438 & 437 &  &  \\
& & & & & \\
Silent & & & & & \\ 
$B_{1u}$ & 263 & 252 & 245 &  &  \\ 
$A_{1u}$ & 435 & 430 & 410 &  &  \\
$A_{2g}$ & 497 & 440 & 434 &  &  \\
$A_{2g}$ & 847 & 806 & 793 &  &  \\
& & & & & \\
IR & & & & & \\
$A_{2u}$  & 67 & 2 & 4i & 90i &  \\
$E_{u}$  & 66 & 1 & 28i & 96i &  \\
$E_{u}$  & 173 & 163 & 183 &  &  \\
$A_{2u}$  & 182 & 180 & 158 & 157 &  185\cite{GalzeraniSSC1982} \\
$E_{u}$  & 248 & 252 & 239 & 240 &  \\
$E_{u}$  & 442 & 433 & 411 & 419 & 450\cite{GalzeraniSSC1982} \\
$E_{u}$  & 543 & 523 & 504 & 515 &  \\
$A_{2u}$  & 548 & 526 & 510 &  &  \\ 
\hline
\end{tabular}
\label{table:tetra_stro_freqs}
\end{table}

Theoretically calculated values of the frequencies of Raman, silent, and IR active modes of tetragonal STO, along with the available theoretical and experimental values are presented in Table.~\ref{table:tetra_stro_freqs}. As can be seen, the frequencies of some of the low-frequency modes are inconsistent among different theoretical approaches. The calculations by Evarestov et al.\cite{EvarestovPRB2011} are based on GGA-PBE using the LCAO as well as PW basis functions. On the other hand, the results of Sai et al.\cite{SaiPRB2000} are using the PW-LDA. Interestingly, two low-frequency Raman and IR phonon modes display inconsistency among the theoretical approaches. Our calculated frequencies of these Raman active $A_{1g}$ and $B_{2g}$ modes are 60 and 61 cm$^{-1}$, respectively. However, Evarestov et al.\cite{EvarestovPRB2011} report them to be 29 and 48 cm$^{-1}$ using LCAO PBE and 98 and $17i$ cm$^{-1}$ using PW PBE, respectively. Next, our calculated frequencies of two-low frequency IR active $A_{2u}$ and $E_u$ modes are 67 and 66 cm$^{-1}$, respectively, which are significantly different from the calculated values of 2 and 1 cm$^{-1}$ by Evarestov et al. using LCAO-PBE and $4i$ and $28i$ cm$^{-1}$ using PW-PBE. On the other hand, PW-LDA calculations by Sai et al.\cite{SaiPRB2000} report them to be around $90i$ and $96i$ cm$^{-1}$. Despite these discrepancies in different theoretical calculations, the frequencies of most of the other modes are in reasonably good agreement with each other. We further note that our calculated frequencies of most of the modes are quite close to the experimentally measure values\cite{FleuryPRL1968}.


Next, the phonon dispersion curve of the tetragonal STO phase, along with the phonon density of states (DOS), is shown in Fig~\ref{fig:sto_dispersion_phdos}. We note that the frequency of most of the modes lies in the range of 0-550 cm$^{-1}$. 
%
\begin{figure}[ht]
 \centering
 \includegraphics[width=0.45\textwidth]{figs/sto_phdos}
 \caption{(Color online) Phonon dispersion curve and density of states of tetragonal STO.}
 \label{fig:sto_dispersion_phdos}
 \end{figure}
 
 
Furthermore, in Fig.~\ref{fig:displacement_vectors}, we also present the phonon eigenvector displacement patterns of ten optical phonon modes of tetragonal STO. We note that, among these ten modes, only $E_{u}$(66 cm$^{-1}$), $A_{1u}$(435 cm$^{-1}$), and $E_{u}$(442 cm$^{-1}$) modes involve the vibrations of Ti atoms. The Sr atoms do not participate in the vibrations of any of these modes. Except for the mode $E_{u}$(442 cm$^{-1}$) where they have very small vibrational amplitude, O atoms contribute in almost all the remaining 9 modes.

% %
\begin{figure}[htp]
  \subfigure[Displacement pattern of mode $E_{u}$(66 cm$^{-1}$).]{\includegraphics[width=.35\linewidth]{figs/sto_Eu_66.png}}
  \subfigure[Displacement pattern of mode $E_{g}$(152 cm$^{-1}$).]{\includegraphics[width=.35\linewidth]{figs/sto_Eg_152.png}} \\
  \subfigure[Displacement pattern of mode $E_{u}$(248  cm$^{-1}$).]{\includegraphics[width=.35\linewidth]{figs/sto_Eu_248.png}} 
  \subfigure[Displacement pattern of mode $A_{1u}$(435 cm$^{-1}$).]{\includegraphics[width=.35\linewidth]{figs/sto_A1u_435.png}} \\
  \subfigure[Displacement pattern of mode $E_{u}$(442 cm$^{-1}$).]{\includegraphics[width=.35\linewidth]{figs/sto_Eu_442.png}}
  \subfigure[Displacement pattern of mode $A_{2g}$(497 cm$^{-1}$).]{\includegraphics[width=.31\linewidth]{figs/sto_A2g_497.png}} 
  \subfigure[Displacement pattern of mode $B_{1g}$(507 cm$^{-1}$).]{\includegraphics[width=.35\linewidth]{figs/sto_B1g_507.png}} 
  \subfigure[Displacement pattern of mode $E_{u}$(543 cm$^{-1}$).]{\includegraphics[width=.35\linewidth]{figs/sto_Eu_543.png}} \\
  \subfigure[Displacement pattern of mode $A_{2u}$(548 cm$^{-1}$).]{\includegraphics[width=.35\linewidth]{figs/sto_A2u_548.png}}
  \subfigure[Displacement pattern of mode $A_{2g}$(847 cm$^{-1}$).]{\includegraphics[width=.31\linewidth]{figs/sto_A2g_847.png}} 
\caption{(Color online) Displacement patterns of the optical modes of STO. The Ti atoms are represented by the spheres of grey color, the Sr atoms by the spheres of cyan color, and the O atoms by the spheres of red color.  }
\label{fig:displacement_vectors}
\end{figure}



\subsubsection{SZO Results}
\label{subsubsec:ph_szo_results}

As discussed in sec~\ref{subsec:szo_structure}, orthorhombic SZO belongs to the Pbnm space group with four formula units in its primitive unit cell, leading to a total of 60 normal modes. The irreducible representations at the BZ center are represented as
\begin{equation}
    \Gamma = 7 A_g + 5B_{1g} + 7B_{2g} + 5B_{3g} + 8A_u + 10B_{1u} + 8B_{2u} + 10B_{3u} \nonumber
    \label{eq:Gamma_irrep}
\end{equation}
Out of these, 24 modes are Raman active ($7 A_g + 5B_{1g} + 7B_{2g} + 5B_{3g}$), 25 modes are IR active ($9B_{1u} + 7B_{2u} + 9B_{3u}$), 3 modes are translational ($B_{1u} + B_{2u} + B_{3u}$), and remaining 8 modes ($8A_u$) are non-active.

Our calculated frequencies of the Raman active modes of the orthorhombic SZO together with the literature data are listed in Table~\ref{table:ortho_szo_raman}. We note that our mode assignment and the calculated frequencies of most of the modes are well in accordance with the previous theoretical results\cite{AmisiPRB2012,VALI2008497}, particularly with the GGA-based calculations by Amisi et al.\cite{AmisiPRB2012}. However, we notice significant differences from previous LDA data by Vali et al.\cite{VALI2008497}, where frequencies of some of the modes disagree by more than 100 cm$^{-1}$. These discrepancies are probably related to the differences in the parameters used in the two calculations. Next, our calculated frequencies of most of the modes are in good agreement with respect to the experimental measurements\cite{OKamishima_1999}. We note that previous experiments did not observe all $B_{1g}, B_{2g}$, and $B_{3g}$ modes and also because experimental measurements were limited to frequencies below 600 cm$^{-1}$, the three modes at the calculated frequencies of 730, 770, and 796 cm$^{-1}$ were not observed.


\begin{table}[ht]
\centering
\caption{Calculated frequencies (cm$^{-1}$) of the Raman active modes of the orthorhombic SZO together with the available theoretical and experimental data.}
\begin{tabular}{c c c c c}
\hline
 Mode & This work (LDA) &  GGA\cite{AmisiPRB2012} & LDA\cite{VALI2008497}  &  Expt.\cite{OKamishima_1999} \\
%      &  (This work) & Ref1(GGA)\cite{AmisiPRB2012} & Ref2(LDA)\cite{VALI2008497}  &  Ref.\cite{OKamishima_1999}  \\
 \hline
 $A_{g}$  &  103  & 94  & 135  &  96 \\
 $A_{g}$  &  112  &  110 & 214  & 107  \\
 $B_{2g}$  & 125   &  119 & 161  &  117 \\
 $B_{1g}$  & 132   & 128  & 181  &  133 \\
 $B_{2g}$  & 147   & 136  & 202  &   \\
 $B_{3g}$  & 151   & 140  & 192  & 138  \\
 $B_{2g}$  & 158   & 145  & 314  & 146  \\
 $B_{3g}$  & 159   & 155  & 333  &   \\
 $A_{g}$  &  187  &  174 & 287  &  169 \\
 $A_{g}$  &  264  & 254  & 356  &  242 \\
 $B_{1g}$  & 285   & 281  & 534  &   \\
 $A_{g}$  &  299  & 278  & 261  &  278 \\
 $B_{3g}$  & 327   & 310  & 522  & 315  \\
 $B_{2g}$  & 331   & 319  & 413  &   \\
 $B_{2g}$  & 385   & 376  & 650  & 392  \\
 $B_{1g}$  & 401   & 390  & 526  &   \\
 $A_{g}$  &  409  & 398  & 632  & 413  \\
 $B_{2g}$  & 439   & 426  & 672  & 441  \\
 $B_{3g}$  & 564   & 528  & 649  &   \\
 $B_{1g}$  & 566   & 530  & 593  & 547  \\
 $A_{g}$  & 571   & 537  & 672  & 556  \\
 $B_{3g}$  & 730   & 708  & 760  &   \\
 $B_{2g}$  & 770   & 750  & 751  &   \\
 $B_{1g}$  & 796   & 777  & 709  &   \\ 
\hline
\end{tabular}
\label{table:ortho_szo_raman}
\end{table}

In Table~\ref{table:ortho_szo_ir} we report the frequencies of the transverse optic IR active phonon modes. No IR measurements on orthorhombic SZO were found in the literature, so our results are compared with previous GGA\cite{AmisiPRB2012} and LDA\cite{VALI2008497}-based results. Here again, our calculated frequencies are in good agreement with the GGA results by Amisi et al.\cite{AmisiPRB2012}. However, we notice significant differences from previous LDA data by Vali et al.\cite{VALI2008497}, where frequencies of some of the modes disagree by more than 100 cm$^{-1}$. 

In Table~\ref{table:ortho_szo_silent} we report frequencies of optically inactive phonon modes. Here, only LDA data by Vali et al.\cite{VALI2008497} are available for comparison. Here again, significant differences in frequencies are noticed.

\begin{table}[ht]
\centering
\caption{Calculated frequencies $\omega$ (cm$^{-1}$) of IR modes of the orthorhombic SZO.}
\begin{tabular}{c c c c }
\hline
 Mode & This work (LDA) & GGA\cite{AmisiPRB2012} & LDA\cite{VALI2008497} \\
 \hline
 $B_{1u}$  &  100  & 94    & 54 \\
 $B_{3u}$  &  106  & 102   & 108 \\
 $B_{2u}$  &  113  & 111   & 141  \\
 $B_{2u}$  &  137  & 129   & 186 \\
 $B_{3u}$  &  153  & 140   & 205 \\
 $B_{1u}$  &  155  & 149   & 204 \\
 $B_{3u}$  &  193  & 184   & 230  \\
 $B_{1u}$  &  204  & 188   & 270  \\
 $B_{3u}$  &  209  & 192   & 236  \\
 $B_{2u}$  &  211  & 199   & 315  \\
 $B_{1u}$  &  213  & 203   & 298  \\
 $B_{2u}$  &  244  & 225   & 311  \\
 $B_{3u}$  &  256  & 240   & 352  \\
 $B_{1u}$  &  260  & 252   & 339  \\
 $B_{3u}$  &  279  & 265   & 373  \\
 $B_{1u}$  &  329  & 308   & 372  \\
 $B_{2u}$  &  333  & 317   & 416  \\
 $B_{1u}$  &  339  & 320   & 450  \\
 $B_{1u}$  &  364  & 356   & 637  \\
 $B_{3u}$  &  386  & 374   & 457  \\
 $B_{3u}$  &  462  & 448   & 649  \\
 $B_{2u}$  &  492  & 459   & 456  \\
 $B_{3u}$  &  505  & 474   & 666  \\
 $B_{2u}$  &  512  & 481   & 621  \\
 $B_{1u}$  &  521  & 491   & 668  \\
 \hline
\end{tabular}
\label{table:ortho_szo_ir}
\end{table}


\begin{table}[ht]
\centering
\caption{Calculated frequencies $\omega$ (cm$^{-1}$) of silent modes of the orthorhombic SZO.}
\begin{tabular}{c c c}
\hline
 Mode & This work (LDA) & LDA\cite{VALI2008497} \\
 \hline
 $A{u}$  &  98   & 82   \\
 $A{u}$  &  131  & 149   \\
 $A{u}$  &  148  & 161   \\
 $A{u}$  &  195  & 259  \\
 $A{u}$  &  245  & 297   \\
 $A{u}$  &  328  & 420   \\
 $A{u}$  &  487  & 434  \\
 $A{u}$  &  509  & 636   \\ 
 \hline
\end{tabular}
\label{table:ortho_szo_silent}
\end{table}

\begin{figure}[htp]
  \subfigure[Displacement pattern of mode $B_{1u}$(100 cm$^{-1}$).]{\includegraphics[width=.26\linewidth]{figs/szo_B1u_100.png}}
  \subfigure[Displacement pattern of mode $A_g$(103 cm$^{-1}$).]{\includegraphics[width=.24\linewidth]{figs/szo_Ag_103.png}}
  \subfigure[Displacement pattern of mode $A_g$(106 cm$^{-1}$).]{\includegraphics[width=.24\linewidth]{figs/szo_Ag_106.png}}  \\
  \subfigure[Displacement pattern of mode $B_{2u}$(113 cm$^{-1}$).]{\includegraphics[width=.26\linewidth]{figs/szo_B2u_113.png}}
  \subfigure[Displacement pattern of mode $B_{2g}$(125 cm$^{-1}$).]{\includegraphics[width=.26\linewidth]{figs/szo_B2g_125.png}}
  \subfigure[Displacement pattern of mode $B_{2g}$(439 cm$^{-1}$).]{\includegraphics[width=.26\linewidth]{figs/szo_B2g_439.png}} \\
  \subfigure[Displacement pattern of mode $B_{3u}$(505 cm$^{-1}$).]{\includegraphics[width=.26\linewidth]{figs/szo_B3u_505.png}}
  \subfigure[Displacement pattern of mode $B_{2u}$(512 cm$^{-1}$).]{\includegraphics[width=.26\linewidth]{figs/szo_B2u_512.png}}
  \subfigure[Displacement pattern of mode $B_{1u}$(521 cm$^{-1}$).]{\includegraphics[width=.26\linewidth]{figs/szo_B1u_521.png}}  \\
  \subfigure[Displacement pattern of mode $B_{3g}$(730 cm$^{-1}$).]{\includegraphics[width=.26\linewidth]{figs/szo_B3g_730.png}}
  \subfigure[Displacement pattern of mode $B_{2g}$(770 cm$^{-1}$).]{\includegraphics[width=.26\linewidth]{figs/szo_B2g_770.png}}
  \subfigure[Displacement pattern of mode $B_{1g}$(796 cm$^{-1}$).]{\includegraphics[width=.26\linewidth]{figs/szo_B1g_796.png}}
\caption{(Color online) Displacement patterns of the optical modes of orthorhombic SZO. The Zr atoms are represented by the spheres of grey color, the Sr atoms by the spheres of cyan, and the O atoms by the spheres of red color. }
\label{fig:szo_displacement_vectors}
\end{figure}

The displacement patterns of twelve optical modes of orthorhombic SZO are shown in Fig~\ref{fig:szo_displacement_vectors}. For clarity, the atomic bonds have been removed. We notice that among these modes, the first five low-frequency modes involve the vibrations of cations along with the anions, whereas the remaining high-frequency modes have only the vibrations of anions. Next, the phonon dispersion curve of orthorhombic SZO is shown in Fig.~\ref{fig:szo_dispersion}, where the frequencies of all the modes are found to be real-valued implying that the system is dynamically stable.

\begin{figure}[ht]
 \centering
 \includegraphics[width=0.4\textwidth]{figs/szo_phdisp.pdf}
 \caption{(Color online) Phonon dispersion spectrum of orthorhombic SZO.}
 \label{fig:szo_dispersion}
 \end{figure}



\subsection{Temperature Dependent Phonon Anharmonic Contributions}
\label{subsec:anh-contributions}

\subsubsection{STO results}

The mode Gr\"uneisen parameters are calculated by first fitting the frequency-volume curves for each mode with the cubic spline polynomials and then taking the first derivative of those obtained smooth functions with respect to unit-cell volume. In Table~\ref{table:sto_gamma_delta3}, we report calculated $\gamma_j$ for all the optical modes in STO. Interestingly, $\gamma_j$ of the first two low-frequency modes $A_{1g}$ (60 cm$^{-1}$) and $B_{1g}$ (61 cm$^{-1}$) are negative, each having a value of -7.3. Next, $\gamma_j$ is quite large for the mode $E_{u}$ (66 cm$^{-1}$) with a value of 29.1 and relatively small for the modes $A_{2u}$ (182 cm$^{-1}$) and $E_u$ (248 cm$^{-1}$) each with the value of 0.2. The $\gamma_j$ values for all the remaining modes lie in the range of 0.7-3.2. 

\begin{table*}[htp]
\centering
\caption{STO data of the calculated values of phonon frequencies at $T = 0 $ K, mode Gr\"uneisen parameter ($\gamma_j$),  linewidths ($\Gamma_j^{(3)}(T)$) and lineshifts ($\Delta_j^{(3)}(T)$) calculated at $T = 0$ and $T = 100$ K and their differences, i.e. $d\Gamma^{(3)}_j = \Gamma_j^{(3)}(100 \text{K}) - \Gamma_j^{(3)}(0 \text{K})$ and $d\Delta^{(3)}_j = \Delta_j^{(3)}(100 \text{K}) - \Delta_j^{(3)}(0 \text{K})$. Here, frequencies, linewidths, and lineshifts are in cm$^{-1}$.}
\begin{tabular}{c c c c c c c c c}
\hline
Mode & $\omega$ & $\gamma_j$ &$\Gamma_j^{(3)}$ (0 K) & $\Gamma_j^{(3)}$ (100 K)  & $d\Gamma^{(3)}_j$  & $\Delta_j^{(3)}$ (0 K)  & $\Delta_j^{(3)}$ (100 K)  &  $d\Delta^{(3)}_j$ \\
\hline
$A_{1g}$       & 60 & -7.3 & 0.03  & 1.26  & 1.23  & -4.70  & -7.84  & -3.14  \\
$B_{1g}$       & 61 & -7.3 & 0.02  & 1.19  & 1.17  & -4.67  & -7.70  & -3.03  \\
$E_{u}$  & 66 & 29.1 & 0.02  & 1.83  & 1.81  & -14.73  & -19.23  & -4.51  \\
$A_{2u}$       & 67 & 2.0 & 0.03  & 1.69  & 1.66  & -14.21  & -18.26  & -4.05  \\
$E_{g}$        & 146 & 2.1 & 0.02  & 0.16  & 0.14  & -0.69  & -0.87  & -0.17  \\
$E_{g}$   & 152 & 2.2 & 1.02  & 2.48  & 1.46  & -4.70  & -5.63  & -0.93  \\
$E_{u}$        & 173 & 3.1 & 0.11  & 0.37  & 0.26  & -1.57  & -1.97  & -0.40  \\
$A_{2u}$       & 182 & 0.2 & 0.10  & 0.36  & 0.26  & -1.32  & -1.64  & -0.32  \\
$E_{u}$   & 248 & 0.2 & 0.44  & 0.84  & 0.40  & -1.81  & -2.08  & -0.27  \\
$B_{1u}$       & 263 & 1.0 & 0.59  & 1.06  & 0.47  & -2.85  & -3.31  & -0.46  \\
$A_{1u}$  & 435 & 1.1 & 1.17  & 3.09  & 1.92  & -5.30  & -5.41  & -0.11  \\
$B_{2g}$       & 438 & 0.8 & 0.78  & 0.97  & 0.19  & -0.71  & -0.57  & 0.13  \\
$E_{g}$        & 439 & 0.7 & 0.64  & 0.79  & 0.15  & -0.60  & -0.54  & 0.06  \\
$E_{u}$  & 442 & 1.3 & 1.28  & 3.72  & 2.44  & -5.26  & -4.97  & 0.29  \\
$A_{2g}$  & 497 & 3.2 & 4.60  & 6.20  & 1.60  & -14.13  & -15.13  & -1.00  \\
$B_{1g}$ & 507 & 3.1 & 5.89  & 7.94  & 2.05  & -14.44  & -15.60  & -1.16  \\
$E_{u}$  & 543 & 2.1 & 1.44  & 1.68  & 0.24  & -3.97  & -4.15  & -0.18  \\
$A_{2u}$  & 548 & 1.7 & 1.64  & 1.89  & 0.25  & -3.78  & -3.97  & -0.19  \\
$A_{2g}$  & 847 & 1.5  & 4.29  & 4.35  & 0.06  & -0.23  & 0.17  & 0.40  \\
\hline
\end{tabular}
\label{table:sto_gamma_delta3}
\end{table*}

Next, the calculation of the linear thermal expansion coefficient in Eq.~\ref{eqn:alpha} requires the evaluation of the bulk modulus, which was obtained by fitting the calculated pressure-volume curve in the range of 0-30 GPa with the third-order Birch-Murnaghan (BM3) equation of state (EOS) (see Fig.~\ref{fig:sto_vol_press}). The calculated bulk modulus for the tetragonal STO is 200 GPa, slightly larger than the previously calculated value of 170 GPa\cite{BHANDARI201827}. Furthermore, the evaluation of $\alpha$ in Eq.~\ref{eqn:alpha} also requires the calculation of the first derivatives of phonon frequencies $\omega_j(\bm q)$ with respect to the volume. Here again, we have used the cubic spline polynomials as an interpolation to obtain the smooth functions of frequencies with respect to the volume. A $\bm q$-grid of $30\times 30\times 30$ was used for the wavevector summation in Eq.~\ref{eqn:alpha}. Finally, the calculated temperature dependence of the linear thermal expansion coefficient is shown in Fig.~\ref{fig:sto_alpha_vs_temp}. We notice that the thermal expansion coefficient is temperature invariant in the range of 0-20 K and increases afterward with increasing temperature. 

\begin{figure}[h!]
 \centering
 \includegraphics[width=0.4\textwidth]{figs/sto_vol_vs_press.pdf}
 \caption{(Color online) Calculated pressure-volume curve of tetragonal STO fitted with the non-linear third-order Birch-Murnaghan equation of state (BM3).}
 \label{fig:sto_vol_press}
 \end{figure}

\begin{figure}[htp]
 \centering
 \includegraphics[width=0.45\textwidth]{figs/sto_alpha_vs_temp.pdf}
 \caption{Temperature dependence of the linear thermal expansion coefficient of tetragonal STO.}
 \label{fig:sto_alpha_vs_temp}
 \end{figure}
 %
 
 %
\begin{figure}[htp]
 \centering
 \includegraphics[width=0.45\textwidth]{figs/sto_delta2_vs_temp.pdf}
 \caption{(Color online) Temperature dependence of the Quasiharmonic contributions to the optical phonon modes of tetragonal STO.}
 \label{fig:sto_delta2}
 \end{figure}
 %

Temperature dependence of QHA shift, $\Delta_j^{(qh)}(T)$, as calculated using Eq.~\ref{eqn:qh} for a total of ten optical modes is shown in Fig.~\ref{fig:sto_delta2}. Almost all the modes show a downward shift with increasing temperature and have a negligible contribution in the temperature range of 0-30 K. An interesting behavior is found for the mode $E_u$ (248 cm$^{-1}$) where $\Delta_j^{(qh)}(T)$ remains quite small across the temperature range of 0-110 K. This behavior is related to its low value of the mode Gr\"uneisen parameter. On the other hand, a large change in $\Delta_j^{(qh)}(T)$ is found for the low-frequency mode $E_u$ (66 cm$^{-1}$), which is related to its large value of $\gamma_j$.

The linewidths of all the optical modes in STO calculated at 0 and 100 K are also shown in Table~\ref{table:sto_gamma_delta3}. We note that the linewidths at 0 K are quite small for several low-frequency modes with many modes having values around 0.02-0.03 cm$^{-1}$; whereas a few high-frequency modes have slightly higher values around 4-6 cm$^{-1}$. However, when the temperature increases to 100 K, the change in the linewidths $d\Gamma^{(3)}_j = \Gamma_j^{(3)}(100 \text{K}) - \Gamma_j^{(3)}(0 \text{K})$ is large for the low-frequency modes compared to several high-frequency modes; the largest change being for the modes $A_{1u}$ (248 cm$^{-1}$), $E_u$ (442 cm$^{-1}$), and $B_{1g}$ (507 cm$^{-1}$), with their respective values of 1.92, 2.44, and 2.05 cm$^{-1}$. 

\begin{figure}[htp]
 \centering
 \includegraphics[width=0.45\textwidth]{figs/sto_linewidth_vs_temp.pdf}
 \caption{(Color online) Temperature dependence of phonon linewidths of optical phonon modes of tetragonal STO.}
 \label{fig:sto_lw}
 \end{figure} 
 
Temperature dependence of the linewidths of a total of ten optical modes in tetragonal STO is shown in Fig.~\ref{fig:sto_lw}. Since the linewidth is directly proportional to the phonon occupation number $n(\omega_j(\bm q), T)$, they increase with increasing temperature. We note that the invariance of the linewidths in the temperature range of 0-30 K arises due to the smaller values of $n$ at lower temperatures.

%\subsection{Third-order lineshifts}
%\label{sec:cubic-anh}

\begin{figure}[h!]
  \centering
  \includegraphics[width=0.45\textwidth]{figs/sto_delta3_vs_temp.pdf}
  \caption{(Color online) Temperature dependence of the lineshifts of optical phonon modes of tetragonal STO. }
  \label{fig:sto_delta3}
  \end{figure}
 % 
\begin{figure}[h!]
 \centering
 \includegraphics[width=0.5\textwidth]{figs/sto_freq_vs_temp.pdf}
 \caption{(Color online) Temperature dependence of frequencies of the optical phonon modes of tetragonal STO.}
 \label{fig:sto_wt}
 \end{figure} 
The lineshifts calculated at 0 and 100 K and $d\Delta^{(3)}_j = \Delta_j^{(3)}(100 \text{K}) - \Delta_j^{(3)}(0 \text{K})$ are given in Table~\ref{table:sto_gamma_delta3} for all the optical modes and temperature dependence of the lineshifts of a total of ten optical modes is shown in Fig.~\ref{fig:sto_delta3}. We note that except for the low-frequency mode $E_{u} (66 \text{cm}^{-1})$, the change in the lineshifts is quite small for several phonon modes in the temperature range of 0-110 K. We further note that the non-zero lineshifts at zero Kelvin arise due to the quantum corrections. Large lineshifts can be seen for the modes $E_{u}$ (66 cm$^{-1})$, $A_{2g}$ (497 cm$^{-1})$, and $B_{1g}$ (507 cm$^{-1})$. Next, from Table~\ref{table:sto_gamma_delta3}, the positive $d\Delta^{(3)}_j$ are found for the modes $B_{2g}$ (438 cm$^{-1})$, $E_{g}$(439 cm$^{-1})$, $E_{u}$(442 cm$^{-1})$, and $A_{2g}$(847 cm$^{-1})$, which, in principle, could lead to the anomalous temperature-dependent behavior of frequencies, i.e. a red shift with cooling. However, the actual temperature dependence of the frequency of any mode also depends on the temperature dependence of the quasiharmonic contributions. Thus, when both the quasiharmonic and cubic anharmonic corrections are added to the frequencies obtained from the harmonic approximation at zero Kelvin (see Eq.~\ref{eqn:freq_T}), except for $E_{u}$(442 cm$^{-1})$, frequencies of the other modes show a normal behavior with temperature, i.e. they redshift with increasing temperature (see Fig.~\ref{fig:sto_wt}). Though the change is quite small across the temperature range, the $E_{u}$(442 cm$^{-1})$ mode behaves interestingly with temperature, i.e. its frequency first softens with heating until 60 K and then starts hardening. Another interesting point to note is that frequencies of almost all the modes remain invariant in the temperature range of 0-30 K, which is due to the fact that the thermal phonon occupation number remains invariant at lower temperatures. 
%

%

\subsubsection{SZO results}
\label{subsubsec:szo_results_anh}
Here, we present the temperature-dependent phonon properties in orthorhombic SZO in the temperature range of 0-300 K.
Our calculated values of the mode Gr\"uneisen parameters, linewidths, and lineshifts of Raman active modes are shown in Table~\ref{table:szo_gamma_delta3_raman}. The differences in linewidths and lineshifts, i.e. $d\Gamma^{(3)}_j = \Gamma_j^{(3)}(300 \text{K}) - \Gamma_j^{(3)}(0 \text{K})$ and $d\Delta^{(3)}_j = \Delta_j^{(3)}(300 \text{K}) - \Delta_j^{(3)}(0 \text{K})$ are also given. We notice the negative values of the mode Gr\"uneisen parameters for the modes $A_{g} (103 \text{cm}^{-1})$, $B_{3g} (151 \text{cm}^{-1})$, $B_{2g} (158 \text{cm}^{-1})$, and $A_{g} (264 \text{cm}^{-1})$ with the values of -1.29, -1.0, -0.96, and -0.31, respectively. The zero Kelvin values of the linewidths, $\Gamma_j^{(3)}$ (0 K), are much smaller for the low-frequency modes compared to the high-frequency modes, whereas the change in the linewidths of the low-frequency modes from 0 K to 300 K, $d\Gamma^{(3)}_j$, is comparable with the high-frequency modes. Furthermore, the lineshifts, $\Delta_j^{(3)}$, are found to be non-zero at zero Kelvin, which are due to the zero-point corrections in energies. Another interesting point to note is that the change in the lineshifts from zero Kelvin to 300 K, $d\Delta^{(3)}_j$, is found to be large for the low-frequency modes compared to the high-frequency ones. Next, positive values of $d\Delta^{(3)}_j$ are seen for some of the high-frequency modes $B_{2g} (439 \text{cm}^{-1})$, $B_{3g} (730 \text{cm}^{-1})$, $B_{2g} (770 \text{cm}^{-1})$, and $B_{1g} (796 \text{cm}^{-1})$ with their values of 0.21, 2.33, 0.97, and 0.48 cm$^{-1}$, respectively. 

Next, the the mode Gr\"uneisen parameters, linewidths, and lineshifts of IR active and optically in-active modes are shown in Tables~\ref{table:szo_gamma_delta3_ir} and ~\ref{table:szo_gamma_delta3_silent}, respectively. Here again, we notice negative values of the mode Gr\"uneisen parameters for some of the low-frequency modes. To the best of our knowledge, there is no literature data on any of the above mentioned quantities to compare with our results. 



\begin{table*}[htp]
\centering
\caption{SZO data of Raman frequencies at $T = 0 $ K, mode Gr\"uneisen parameter ($\gamma_j$),  linewidths ($\Gamma_j^{(3)}(T)$) and lineshifts ($\Delta_j^{(3)}(T)$) calculated at $T = 0$ and $T = 300$ K and their differences, i.e. $d\Gamma^{(3)}_j = \Gamma_j^{(3)}(300 \text{K}) - \Gamma_j^{(3)}(0 \text{K})$ and $d\Delta^{(3)}_j = \Delta_j^{(3)}(300 \text{K}) - \Delta_j^{(3)}(0 \text{K})$. Here, frequencies, linewidths, and lineshifts are in cm$^{-1}$.}
\begin{tabular}{c c c c c c c c c}
\hline
Mode & $\omega$ & $\gamma_j$ &$\Gamma_j^{(3)}$ (0 K) & $\Gamma_j^{(3)}$ (300 K)  & $d\Gamma^{(3)}_j$  & $\Delta_j^{(3)}$ (0 K)  & $\Delta_j^{(3)}$ (300 K)  &  $d\Delta^{(3)}_j$ \\
\hline
$A_{g}$   & 103   & -1.29  &  0.02   & 3.66   & 3.65   &  -1.68   &  -8.77   & -7.09   \\
$A_{g}$   & 112   &  2.30  &  0.18   & 4.55   & 4.36   & -1.92   & -8.69   & -6.77  \\
$B_{2g}$  & 125   &  1.19  &  0.04   & 1.15   & 1.12   & -0.78   & -3.09   & -2.31  \\
$B_{1g}$  & 132   &  2.72  &  0.08   & 1.81   & 1.73   & -0.87   & -3.47   & -2.60  \\
$B_{2g}$  & 147   &  0.62  &  0.06   & 1.23   & 1.17   & -0.70   & -2.74   & -2.03  \\
$B_{3g}$  & 151   & -1.00  &  0.25   & 4.03   & 3.77   & -1.93   & -6.71   & -4.78  \\
$B_{2g}$  & 158   & -0.96  & 0.15    & 1.27   & 1.11   & -0.45   & -1.46   & -1.01  \\
$B_{3g}$  & 159   &  0.62  & 0.23    & 3.38   & 3.15   & -1.05   & -4.67   & -3.62  \\
$A_{g}$   &  187  &  0.33  & 0.23    & 2.83   & 2.60   & -1.48   & -4.48   & -3.00  \\
$A_{g}$   &  264  & -0.31  & 0.44    & 3.33   & 2.89   & -2.09   & -4.57   & -2.48  \\
$B_{1g}$  &  285  &  0.15  & 0.70    & 4.07   & 3.37   & -2.17   & -4.05   & -1.88  \\
$A_{g}$   &  299  &  0.15  & 1.41    & 6.22   & 4.80   & -2.13   & -2.30   & -0.17  \\
$B_{3g}$  &  327  &  0.82  &  0.86   & 5.34   & 4.47   & -2.12   & -3.83   & -1.72  \\
$B_{2g}$  &  331  &  0.53  & 0.67    & 3.19   & 2.53   & -2.20   & -4.25   & -2.04  \\
$B_{2g}$  &  385  &  0.34  & 1.02    & 4.82   & 3.80   & -2.52   & -4.31   & -1.79  \\
$B_{1g}$  &  401  &  0.50  & 0.98    & 3.37   & 2.39   & -1.36   & -1.48   & -0.13  \\
$A_{g}$   &  409  &  0.77  & 0.85    & 3.54   & 2.69   & -1.45   & -1.56   & -0.11  \\
$B_{2g}$  &  439  &  0.67  &  1.45   & 4.38   & 2.93   & -1.30   & -1.08   & 0.21  \\
$B_{3g}$  &  564  &  2.78  & 3.19    & 9.48   & 6.29   & -4.74   & -7.31   & -2.57  \\
$B_{1g}$  &  566  &  2.52  & 3.61    & 9.78   & 6.17   & -4.57   & -5.99   & -1.42  \\
$A_{g}$   &  571  &  2.74  & 3.23    & 8.70   & 5.47   & -4.06   & -5.50   & -1.44  \\
$B_{3g}$  &  730  &  1.85  & 5.05    & 10.16   & 5.12   & -3.36   & -1.04   & 2.33  \\
$B_{2g}$  &  770  &  1.79  & 5.30    & 8.88   & 3.58   & -5.67   & -4.69   & 0.97  \\
$B_{1g}$  &  796  &  1.73  & 5.47    & 8.29   & 2.82   & -5.69   &  -5.20  & 0.48  \\
\hline
\end{tabular}
\label{table:szo_gamma_delta3_raman}
\end{table*}

\begin{table*}[htp]
\centering
\caption{SZO data of the calculated IR frequencies at $T = 0 $ K, mode Gr\"uneisen parameter ($\gamma_j$),  linewidths ($\Gamma_j^{(3)}(T)$) and lineshifts ($\Delta_j^{(3)}(T)$) calculated at $T = 0$ and $T = 300$ K and their differences, i.e. $d\Gamma^{(3)}_j = \Gamma_j^{(3)}(300 \text{K}) - \Gamma_j^{(3)}(0 \text{K})$ and $d\Delta^{(3)}_j = \Delta_j^{(3)}(300 \text{K}) - \Delta_j^{(3)}(0 \text{K})$. Here, frequencies, line widths, and line shifts are in cm$^{-1}$.}
\begin{tabular}{c c c c c c c c c}
\hline
Mode & $\omega$ & $\gamma_j$ &$\Gamma_j^{(3)}$ (0 K) & $\Gamma_j^{(3)}$ (300 K)  & $d\Gamma^{(3)}_j$  & $\Delta_j^{(3)}$ (0 K)  & $\Delta_j^{(3)}$ (300 K)  &  $d\Delta^{(3)}_j$ \\
\hline
$B_{1u}$   & 100  &  1.92  & 0.01    &  1.19   & 1.18   & -0.70   & -3.71   & -3.00   \\
$B_{3u}$   & 106  &  1.82  &  0.03   & 2.44   & 2.41   &  -1.41  & -5.84   & -4.43  \\
$B_{2u}$   & 113  &  2.11  & 0.30    & 4.90   & 4.60   & -1.50   & -5.52   & -4.02  \\
$B_{2u}$   & 137  &  1.09  &  0.44   & 8.20   & 7.76   & -2.92   & -10.75   & -7.83   \\
$B_{3u}$   & 153  &  2.15  &  0.30   & 5.40   & 5.09   & -2.01   & -7.31   & -5.31  \\
$B_{1u}$   & 155  &  0.64  & 0.45    & 7.05   & 6.60   & -2.80   & -9.05   & -6.25  \\
$B_{3u}$   & 193  &  0.30  &  0.62   & 6.11   &  5.49  & -1.79   & -4.32   & -2.54  \\
$B_{1u}$   & 204  &  0.92  & 0.68    & 5.32   & 4.64   & -2.27   & -6.30   & -4.03  \\
$B_{3u}$   & 209  & -0.33  & 0.75    & 5.87   & 5.12   & -2.10   & -5.19   & -3.09  \\
$B_{2u}$   & 211  &  0.46  &  0.75   & 6.95   & 6.21   & -2.04   & -6.34   & -4.30  \\
$B_{1u}$   & 213  & -0.17  & 0.29    & 2.62   & 2.33   & -1.47   & -4.83   & -3.36  \\
$B_{2u}$   & 244  &  0.88  &  0.80   & 5.50   & 4.71   & -2.34   & -5.33   & -2.98  \\
$B_{3u}$   & 256  &  1.31  &  0.60   & 4.47   & 3.87   & -2.14   & -5.07   & -2.93  \\
$B_{1u}$   & 260  &  0.48  &  0.64   & 4.61   & 3.97   & -1.92   & -4.26   & -2.34  \\
$B_{3u}$   & 279  &  1.66  & 0.55    & 3.28   & 2.73   & -1.06   & -2.22   & -1.16  \\
$B_{1u}$   & 329  &  0.89  &  0.48   & 4.12   & 3.64   & -2.60   & -4.46   & -1.86  \\
$B_{2u}$   & 333  &  1.73  & 0.66    & 4.02   & 3.36   & -2.64   & -4.75   & -2.12  \\
$B_{1u}$   & 339  &  1.09  & 1.12    & 4.85   & 3.73   & -2.92   & -4.88   & -1.96  \\
$B_{1u}$   & 364  &  0.55  & 1.51    & 5.36   & 3.85   & -1.80   & -2.92   & -1.12  \\
$B_{3u}$   & 386  &  0.65  & 1.02    & 3.62   & 2.60   & -1.10   & -1.34   & -0.24  \\
$B_{3u}$   & 462  &  0.48  & 1.48    & 6.65   & 5.17   & -2.20   & -3.61   & -1.40  \\
$B_{2u}$   & 492  &  3.22  & 1.04    & 5.69   & 4.65   & -3.32   & -4.42   & -1.10  \\
$B_{3u}$   & 505  &  2.73  & 0.91    & 4.52   & 3.61   & -3.31   & -4.24   & -0.93  \\
$B_{2u}$   & 512  &  2.66  & 0.98    & 4.59   & 3.61   & -3.16   & -3.34   & -0.18  \\
$B_{1u}$   & 521  &  2.55  & 1.12    & 3.71   & 2.59   & -2.93   & -3.65   & -0.73  \\
\hline
\end{tabular}
\label{table:szo_gamma_delta3_ir}
\end{table*}

\begin{table*}[htp]
\centering
\caption{SZO data of the calculated Silent frequencies at $T = 0 $ K, mode Gr\"uneisen parameter ($\gamma_j$),  linewidths ($\Gamma_j^{(3)}(T)$) and lineshifts ($\Delta_j^{(3)}(T)$) calculated at $T = 0$ and $T = 300$ K and their differences, i.e. $d\Gamma^{(3)}_j = \Gamma_j^{(3)}(300 \text{K}) - \Gamma_j^{(3)}(0 \text{K})$ and $d\Delta^{(3)}_j = \Delta_j^{(3)}(300 \text{K}) - \Delta_j^{(3)}(0 \text{K})$. Here, frequencies, line widths, and line shifts are in cm$^{-1}$.}
\begin{tabular}{c c c c c c c c c}
\hline
Mode & $\omega$ & $\gamma_j$ &$\Gamma_j^{(3)}$ (0 K) & $\Gamma_j^{(3)}$ (300 K)  & $d\Gamma^{(3)}_j$  & $\Delta_j^{(3)}$ (0 K)  & $\Delta_j^{(3)}$ (300 K)  &  $d\Delta^{(3)}_j$ \\
\hline
$A_{u}$   &  98   & 0.98   &  1.03   & 29.74   & 28.71   &  -5.51  &  -31.37  & -25.86   \\
$A_{u}$   &  131  & 1.09   & 0.04    & 1.63   & 1.59   & -1.07   & -4.10   & -3.03  \\
$A_{u}$   &  148  & 0.46   &  0.35   & 3.66   & 3.32   & -1.89   & -6.29   & -4.40  \\
$A_{u}$   &  195  & -0.29  &  0.64   & 7.09   & 6.46   & -1.62   & -5.31   & -3.69  \\
$A_{u}$   &  245  &  0.33  & 1.57    & 8.00   & 6.43   & -2.80   & -5.45   & -2.64  \\
$A_{u}$   &  328  &  1.24  & 0.59    & 3.63   & 3.04   & -2.76   & -5.49   & -2.73  \\
$A_{u}$   &  487  &  3.46  & 0.92    & 9.59   & 8.67   & -4.04   & -5.46   & -1.42  \\
$A_{u}$   &  509  &  2.84  & 0.87    & 4.47   & 3.60   & -3.58   &  -4.15  & -0.58  \\
\hline
\end{tabular}
\label{table:szo_gamma_delta3_silent}
\end{table*}

Within the QHA, the temperature dependence of the linear thermal expansion coefficient and the frequency shifts in SZO are shown in Figs.~\ref{fig:szo_alpha_vs_temp} and ~\ref{fig:szo_delta2}, respectively. The $\Delta_j^{(qh)}(T)$ for most of the modes shows a downward trend with increasing temperature. On the other hand, an upward shift in $\Delta_j^{(qh)}(T)$ for the mode $A_{g} (103 \text{cm}^{-1})$ can be seen, which is related with its negative mode Gr\"uneisen parameter value. 

\begin{figure}[h]
 \centering
 \includegraphics[width=0.45\textwidth]{figs/szo_alpha_vs_temp.pdf}
 \caption{(Color online) Temperature dependence of the linear thermal expansion coefficient of orthorhombic SZO.}
 \label{fig:szo_alpha_vs_temp}
 \end{figure}

\begin{figure}[h]
  \centering
  \includegraphics[width=0.45\textwidth]{figs/szo_delta2_vs_temp.pdf}
  \caption{(Color online) Temperature dependence of the quasiharmonic shifts of twelve optical modes in SZO. }
  \label{fig:szo_delta2}
\end{figure}

Temperature dependence of the linewidths, $\Gamma^{(3)}_j(T)$, for several optical phonon modes in SZO is shown in Fig.~\ref{fig:szo_linewidth}. We can see that $\Gamma^{(3)}_j(T)$ is temperature independent for most of the modes in the range of 0-30 K and increases with increasing temperature. An increase in $\Gamma^{(3)}_j(T)$ with temperature is related to the increase in phonon occupation number. Next, the temperature dependence of the lineshifts, $\Delta^{(3)}_j(T)$, for several modes in SZO is shown in Fig.~\ref{fig:szo_delta3}. Non-zero values of $\Delta^{(3)}_j$(0 K) at zero Kelvin can be seen for all the modes, which arise due to quantum corrections in energies at lower temperatures.
Furthermore, except for a few, the lineshifts, $\Delta^{(3)}_j(T)$, for most of the modes show a downward trend with increasing temperature. An interesting behavior is found for some of the high-frequency modes $B_{2g}$ (439 cm$^{-1}$), $B_{3g}$ (730 cm$^{-1}$), $B_{2g}$ (770 cm$^{-1}$), and $B_{1g}$ (796 cm$^{-1}$) for which lineshifts increase with increasing temperature, though starting from the negative values. These upward shifts in $\Delta^{(3)}_j(T)$ could in principle lead to the anomalous behavior of the frequencies of these modes with temperature (softening with cooling). However, the overall nature of temperature dependence of the frequency of any mode depends on the contributions coming from all the perturbative corrections to the harmonic level. Thus, after adding the contributions of $\Delta^{(qh)}_j(T)$ and $\Delta^{(3)}_j(T)$ to $\omega_j$, the frequencies of all the modes show a downward behavior with increasing temperature (see Fig.~\ref{fig:szo_freq}), and hence no anomalous temperature-dependent behavior of any of the modes in SZO is found.

\begin{figure}[htp]
  \centering
  \includegraphics[width=0.45\textwidth]{figs/szo_lw_vs_temp.pdf}
  \caption{(Color online) Temperature dependence of the linewidths of twelve optical modes in SZO. }
  \label{fig:szo_linewidth}
\end{figure}

\begin{figure}[htp]
  \centering
  \includegraphics[width=0.45\textwidth]{figs/szo_delta3_vs_temp.pdf}
  \caption{(Color online) Temperature dependence of the cubic anharmonic shifts of several optical modes in SZO. }
  \label{fig:szo_delta3}
\end{figure}

\begin{figure}[htp]
  \centering
  \includegraphics[width=0.5\textwidth]{figs/szo_freq_vs_temp.pdf}
  \caption{(Color online) Temperature dependence of frequencies of twelve optical modes in SZO. }
  \label{fig:szo_freq}
\end{figure}




   
%------------------------------------------------------------
\section{Conclusions}
\label{sec:conclusions}
In this paper, we have studied temperature-dependent phonon properties in the tetragonal STO  and orthorhombic SZO using the first-principles density functional theory together with the cubic lattice anharmonic calculations. The calculated structural properties of both systems were in accordance with the available literature data. Within the harmonic lattice approximation, frequencies of most of the modes in both systems were compared well with previous theoretical and experimental results. However, there were some discrepancies in the frequencies of some of the modes in both systems with regard to the previous theoretical data. In the case of tetragonal STO, our calculations based on PW-LDA showed the frequencies of all the modes to be real-valued. However, the PW-LDA-based calculations by Sai et al.\cite{SaiPRB2000} and PW-PBE-based calculations by Evarestov et el.\cite{EvarestovPRB2011} found a few modes with imaginary frequencies. This could be due to the different parameters used in their calculations. As discussed earlier in the paper, perhaps the most important reason for these frequency discrepancies could be related to the complex nature of tetragonal STO. Next, in the case of orthorhombic SZO, our theoretical values of frequencies of modes of the modes are in accordance with the previous theoretical data by Amisi et al.\cite{AmisiPRB2012} and by Vali et al.\cite{VALI2008497}. However, there were some discrepancies in frequencies from the calculations by Vali et al., where a change in frequency by more than 100 cm$^{-1}$ was reported for some of the modes. Our calculated frequencies of the Raman active modes were found to be in reasonably good agreement with the available experimental results\cite{OKamishima_1999}.

Next, within the quasiharmonic approximation, the mode G\"uneisen parameters, linear thermal expansion coefficient, and frequency shifts were discussed. Except for one mode in STO, almost all the modes showed a downward shift with increasing temperature. Furthermore, within the third-order lattice anharmonic effect, the temperature dependence of the phonon linewidths and lineshifts were presented in detail. Finally, the temperature dependence of phonon frequencies was also discussed.


\section*{ACKNOWLEDGMENTS}
P.K.V. greatly acknowledges Professors H R Krishnamurthy and Ajay Kumar Sood, Department of Physics, IISc for the discussion and guidance. The author is also grateful to Professor Umesh V Waghmare, Theoretical Sciences Unit, JNCASR for helpful discussions. 
The author is thankful to the Council of Scientific and Industrial Research, India for financial support. The calculations were performed using the Cray XC40 at the Supercomputer Education and Research Centre IISc.

\newpage
%\setcounter{footnote}{0} 

\cleardoublepage
\phantomsection
\addcontentsline{toc}{section}{\refname}

% REMOVE in  % \bibitem{}:
	% accented characters (acute or grave accent) [´`], 
	% comma [,], 
	% cedilla: a̧, b̧, ç, etc.
	% German Eszett: ß
	% letters with an umlaut mark or diaeresis [¨]
	% letters with breve: ğ
	% letters with breve diacritic mark [˘]
	% letters with circumflex diacritic (chevron-shaped) or caron [ˆˇ], 
	% letters with ogonek ą
	% letters with stroke: ł 
	% letters with the addition of a dot above [◌̇]
	% Scandinavian vowel ø Ø
	% these quotation mark: “ ”

% Regola sui cognomi composti. Se la prima parola è scritta in minuscolo (da, de, 't, van, von, etc.), si segue l'ordine alfabetico della seconda parola. Se la prima parola è scritta in maiuscolo oppure se c'è l'apostrofo, anche con singola lettera minuscola (d', De, Dell', Di, Van, etc.), si segue l'ordine alfabetico della prima parola/lettera.

\begin{thebibliography}{0}\small 
\thispagestyle{empty}
\markboth{Bibliography}{Bibliography}

\bibitem{Besov Il'In Nikol'skii "Integral Representations of Functions and Imbedding Theorems I"} O.V. Besov, V.P. Il'In, S.M. Nikol'skii, \textit{Integral Representations of Functions and Imbedding Theorems, Vol. I}, Ed. by M.H. Taibleson, J. Wiley \& Sons, Washington, 1978.
\bibitem{Besov Il'In Nikol'skii "Integral Representations of Functions and Imbedding Theorems II"} ------, \textit{Integral Representations of Functions and Imbedding Theorems, Vol. II}, Ed. by M.H. Taibleson, J. Wiley \& Sons, Washington, 1979.

\bibitem{Chen Cruzeiro "Stochastic geodesics and forward-backward stochastic differential equations on Lie groups"} X. Chen, A.B. Cruzeiro, \textit{Stochastic geodesics and forward-backward stochastic differential equations on Lie groups}, Discrete Contin. Dyn. Syst., Vol. 2013 (Supplement), pp. 115-121.

\bibitem{Cruzeiro "Hydrodynamics Probability and the Geometry of the Diffeomorphisms Group"} A.B. Cruzeiro, \textit{Hydrodynamics, Probability and the Geometry of the Diffeomorphisms Group}, in R.C. Dalang, M. Dozzi, F. Russo (Eds.), \textit{Seminar on Stochastic Analysis, Random Fields and Applications VI}, Centro Stefano Franscini, Ascona, May 2008, Birkhäuser · Springer Basel \textsc{ag}, Basel, 2011, pp. 83-93.

\bibitem{Cruzeiro and Zambrini "Feynman's Functional Calculus and Stochastic Calculus of Variations"} A.B. Cruzeiro and J.-C. Zambrini, \textit{Feynman's Functional Calculus and Stochastic Calculus of Variations}, in A.B. Cruzeiro, J.-C. Zambrini (Eds.), \textit{Stochastic Analysis and Applications}, Proceedings of the 1989 Lisbon Conference, Springer Science+Business Media, New York, 1991, pp. 82-95.

\bibitem{Dankel Jr. "Mechanics on Manifolds and the Incorporation of Spin into Nelson's Stochastic Mechanics"} T.G. Dankel Jr., \textit{Mechanics on Manifolds and the Incorporation of Spin into Nelson's Stochastic Mechanics}, Arch. Rational Mech. Anal., Vol. 37, № 3, 1970, pp. 192-221.

\bibitem{Dieudonne Schwartz "La dualite dans les espaces (F) et (LF)"} J. Dieudonné, L. Schwartz, \textit{La dualité dans les espaces $(\mathscr{F})$ et $(\mathscr{LF})$}, Ann. Inst. Fourier, Vol. 1, 1949, pp. 61-101.

\bibitem{Dohrn and Guerra "Nelson's stochastic mechanics on Riemannian manifolds"} D. Dohrn and F. Guerra, \textit{Nelson's stochastic mechanics on Riemannian manifolds}, Lett. Nuovo Cimento, Vol. 22, № 4, 1978, pp. 121-127.

\bibitem{Dohrn and Guerra "Geodesic correction to stochastic parallel displacement of tensors"} ------, \textit{Geodesic correction to stochastic parallel displacement of tensors}, in G. Casati, J. Ford (Eds.), \textit{Stochastic Behavior in Classical and Quantum Hamiltonian Systems}, Volta Memorial Conference, Como 1977, Springer-Verlag, Berlin, Heidelberg, 1979, pp. 241-249.

\bibitem{Dohrn Guerra Ruggiero "Spinning Particles and Relativistic Particles in the Framework of Nelson's Stochastic Mechanics"} D. Dohrn, F. Guerra, P. Ruggiero, \textit{Spinning Particles and Relativistic Particles in the Framework of Nelson's Stochastic Mechanics}, in S. Albeverio, Ph. Combe, R. Høegh-Krohn, G. Rideau, M. Sirugue-Collin, M. Sirugue and R. Stora (Eds.), \textit{Feynman Path Integrals}, Proceedings of the International Colloquium, Held in Marseille, May 1978, Springer-Verlag, Berlin, Heidelberg, 1979, pp. 165-181. 

\bibitem{Gliklikh and Vinokurova "The Newton-Nelson Equation on Fiber Bundles with Connections"} \textcyrillic{Ю.Е. Гликлих, Н.В. Винокурова}, \textcyrillic{\textit{Уравнение Ньютона–Нельсона на расслоениях со связностями}}, \textcyrillic{Фундаментальная и прикладная математика, том 20, № 3, 2015, pp. 61-81}.\footnote{ 
	{} An En. version, made by the authors themselves, Y.E. Gliklikh and N.V. Vinokurova, is in J. Math. Sci., Vol. 225, № 4, 2017, pp. 575-589, under the title \textit{The Newton–Nelson Equation on Fiber Bundles with Connections}.
	}

\bibitem{Grothendieck "Sur les espaces (F) et (DF)"} A. Grothendieck, \textit{Sur les espaces $(\mathscr{F})$ et $(\mathscr{DF})$}, Summa Bras. Math., Vol. 3, № 6, 1954, pp. 57-123. Fr. version is not found; version consulted: Ru. transl. by D.A. Raikov, Matematika, Vol. 2, № 3, 1958, pp. 81-127.
\bibitem{Grothendieck "Recoltes et Semailles. Reflexions et temoignage sur un passe de mathematicien"} ------, \textit{Récoltes et Semailles. Réflexions et témoignage sur un passé de mathématicien} [Juin 1983-Avril 1986], text in a free, online version: the number of pages refers to the page numbering of the manuscript (otm = of the manuscript).

\bibitem{Guerra and Ruggiero "A Note on Relativistic Markov Processes"} F. Guerra and P. Ruggiero, \textit{A Note on Relativistic Markov Processes}, Lett. Nuovo Cimento, Vol. 23, № 15, 9 Dic. 1978, pp. 529-534.

\bibitem{Ito "The Brownian motion and tensor fields on Riemannian manifold"} K. Itô, \textit{The Brownian motion and tensor fields on Riemannian manifold}, in  \textit{Proceedings of the International Congress of Mathematicians}, 15-22 August 1962, Stockholm, Institut Mittag-Leffler, Djursholm, Almqvist \& Wiksell, Uppsala, 1963, pp. 536-539.
\bibitem{Ito "Stochastic parallel displacement"} ------, \textit{Stochastic parallel displacement}, in M.A. Pinsky (Ed.), \textit{Probabilistic Methods in Differential Equations}, Proceedings of the Conference held at the University of Victoria, August 19-20, 1974, pp. 1-7.

\bibitem{"On the logarithmic normal distribution of particle sizes under grinding"} A.N. Kolmogorov, \textit{On the logarithmic normal distribution of particle sizes under grinding} (1941), in A.N. Shiryayev, \textit{Selected Works of A.N. Kolmogorov, Vol. II. Probability Theory and Mathematical Statistics}, Springer Science+Business Media, Dordrecht, 1992, pp. 281-284.

\bibitem{Littlewood Paley "Theorems on Fourier Series and Power Series"} J.E. Littlewood, R.E.A.C. Paley, \textit{Theorems on Fourier Series and Power Series}, J. Lond. Math. Soc. (1), Vol. 6, № 3, 1931, pp. 230-233.
\bibitem{Littlewood Paley "Theorems on Fourier Series and Power Series II"} ------, \textit{Theorems on Fourier Series and Power Series (II)}, J. Lond. Math. Soc. (2), Vol. 42, № 1, 1937, pp. 52-89.
\bibitem{Littlewood Paley "Theorems on Fourier Series and Power Series III"} ------, \textit{Theorems on Fourier Series and Power Series (III)}, J. Lond. Math. Soc. (2), Vol. 43, № 1, 1938, pp. 105-126.

\bibitem{Mandelbrot "Les objets fractals. Forme hasard et dimension"} B. Mandelbrot, \textit{Les objets fractals. Forme, hasard et dimension}, Flammarion, 1975, Paris (quatrième édition, 1995).
\bibitem{Mandelbrot "Measures of fractal lacunarity: Minkowski content and alternatives"} ------, \textit{Measures of fractal lacunarity: Minkowski content and alternatives}, in C. Bandt, S. Graf, M. Zähle (Eds.), \textit{Fractal Geometry and Stochastics}, Birkhäuser · Springer Basel \textsc{ag}, Basel, 1995, pp. 15-42.

\bibitem{Nelson "Derivation of the Schrodinger Equation from Newtonian Mechanics"} E. Nelson, \textit{Derivation of the Schrödinger Equation from Newtonian Mechanics}, Phys. Rev., Vol. 150, № 4, 1966, pp. 1079-1085. 
\bibitem{Nelson "Construction of Quantum Fields from Markoff Fields"} ------, \textit{Construction of Quantum Fields from Markoff Fields}, J. Funct. Anal., Vol. 12, № 1, 1973, pp. 97-112.
\bibitem{Nelson "Quantum fluctuations"} ------, \textit{Quantum fluctuations}, Princeton Univ. Press, Princeton (\textsc{nj}), 1985. 
\bibitem{Nelson "Dynamical Theories of Brownian Motion"} ------, \textit{Dynamical Theories of Brownian Motion}, re-typesetted second edition as \TeX{} file by J. Suzuki and revised by Nelson in 2001 (originally published by Princeton Univ. Press, 1967).

\bibitem{Niccolai "Notes in Pure Mathematics and Mathematical Structures in Physics"} E. Niccolai, \textit{Notes in Pure Mathematics \& Mathematical Structures in Physics}, \href{https://arxiv.org/abs/2105.14863}{arXiv:2105.14863} [math-ph], 2023 [v9]; the latest revision: \emph{Download} page at \href{https://www.edoardoniccolai.com}{https://www.edoardoniccolai.com}.
\bibitem{Niccolai "Spin and Torsion Tensors on Gauge Gravity: a Re-examination of the Einstein-Cartan Spatio-Temporal Theory"} ------, \textit{Spin \& Torsion Tensors on Gauge Gravity: a Re-examination of the Einstein–Cartan Spatio-Temporal Theory}, doi:\href{https://doi.org/10.5281/zenodo.7775360}{10.5281.7775360}, 2023 [v5], or \href{https://hal.science/hal-03948127}{hal-03948127} (HAL Id), 2023 [v3].

\bibitem{Nottale "Scale Relativity Fractal Space-Time and Quantum Mechanics"} L. Nottale, \textit{Scale Relativity, Fractal Space-Time and Quantum Mechanics}, Chaos Solitons Fractals, Vol. 4, № 3, 1994, pp. 361-388.
\bibitem{Nottale "Scale Relativity and Fractal Space-Time: Applications to Quantum Physics Cosmology and Chaotic Systems"} ------, \textit{Scale Relativity and Fractal Space-Time: Applications to Quantum Physics, Cosmology and Chaotic Systems}, Chaos Solitons Fractals, Vol. 7, № 6, 1996, pp. 877-938.
\bibitem{Nottale "Scale Relativity and Fractal Space-Time: Theory and Applications"} ------, \textit{Scale Relativity and Fractal Space-Time: Theory and Applications}, Found. Sci., Vol. 15, № 2, 2010, pp. 101-152; \href{https://arxiv.org/abs/0812.3857}{arXiv:0812.3857} [physics.gen-ph], 2008 [v1].
\bibitem{Nottale "Scale Relativity and Fractal Space-Time. A New Approach to Unifying Relativity and Quantum Mechanics"} ------, \textit{Scale Relativity and Fractal Space-Time. A New Approach to Unifying Relativity and Quantum Mechanics}, World Scientific, Singapore, 2011.

\bibitem{Nottale Celerier and T. Lehner Non-Abelian gauge field theory in scale relativity"} L. Nottale, M.-N. Célérier, and T. Lehner, \textit{Non-Abelian gauge field theory in scale relativity}, J. Math. Phys. 47, № 3, 2006, pp. 032303-1-19; \href{https://arxiv.org/abs/hep-th/0605280}{arXiv:hep-th/0605280}, 2006 [v1].

\bibitem{Nottale and Lehner "Turbulence and Scale Relativity"} L. Nottale and T. Lehner, \textit{Turbulence and Scale Relativity}, Phys. Fluids, Vol. 31, № 10, 2019, pp. 105109-1-22; \href{https://arxiv.org/abs/1807.11902}{arXiv:1807.11902} [physics.gen-ph], 2018 [v1].\bibitem{Ruggiero and Tartaglia "Einstein-Cartan theory as a theory of defects in space-time"} M.L. Ruggiero and A. Tartaglia, \textit{Einstein–Cartan theory as a theory of defects in space–time}, Amer. J. Phys., Vol. 71, № 12, 2003, pp. 1303-1313.

\bibitem{Paley and Wiener "Fourier Transforms in the Complex Domain"} R.E.A.C. Paley and N. Wiener, \textit{Fourier Transforms in the Complex Domain}, American Mathematical Society, New York, 1934.

\bibitem{Paley Wiener and Zygmund "Notes on random functions"} R.E.A.C. Paley, N. Wiener and A. Zygmund, \textit{Notes on random functions}, Math. Z., Vol. 37, № 1, 1933, pp. 647-668.

\bibitem{Sawano "Theory of Besov Spaces"} Y. Sawano, \textit{Theory of Besov Spaces}, Springer Nature, Singapore, 2018.

\bibitem{Schwartz "Produits tensoriels topologiques d'espaces vectoriels topologiques. Espaces vectoriels topologiques nucleaires"} L. Schwartz, \textit{Produits tensoriels topologiques d'espaces vectoriels topologiques. Espaces vectoriels topologiques nucléaires}, Séminaire Schwartz, Tome 1, 1953-1954, exp. № 1-24.

\bibitem{Sobolev "Sur un theoreme d'analyse fonctionnelle"} S.[L.] Sobolev (Soboleff), \textcyrillic{\textit{Об одной теореме функционального анализа}} (\textit{Sur un théorème d'analyse fonctionnelle}), Mat. Sb., Vol. 4(46), № 3, 1938, pp. 471-497.
\bibitem{Sobolev "Some Applications of Functional Analysis in Mathematical Physics"} ------, \textit{Some Applications of Functional Analysis in Mathematical Physics}, transl. from the Ru. by H.H. McFaden, American Mathematical Society (\textsc{ams}), Providence (\textsc{ri}), 2008\textsuperscript{re}.

\bibitem{Uhlenbeck and Ornstein "On the Theory of the Brownian Motion"} G.E. Uhlenbeck and L.S. Ornstein, \textit{On the Theory of the Brownian Motion}, Phys. Rev., Vol. 36, № 5, 1930, pp. 823-841.

\bibitem{Wiener "Nonlinear Problems in Random Theory"} N. Wiener, \textit{Nonlinear Problems in Random Theory}, The Technology Press of The Massachusetts Institute of Technology and J. Wiley \& Sons, New York, Chapman \& Hall, London, 1958.

\bibitem{Zastawniak "A Relativistic Version of Nelson's Stochastic Mechanics"} T. Zastawniak, \textit{A Relativistic Version of Nelson's Stochastic Mechanics}, Europhys. Lett., Vol. 13, № 1, 1990, pp. 13-17.
\end{thebibliography}

 
 \bibliography{bibl}  

\end{document}
%
% ****** End of file apssamp.tex ******