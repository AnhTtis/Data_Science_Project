\documentclass{aastex631} % twocolumn

%% The default is a single spaced, 10 point font, single spaced article.
%% There are 5 other style options available via an optional argument. They
%% can be invoked like this:
%%
%% \documentclass[arguments]{aastex631}
%% 
%% where the layout options are:
%%
%%  twocolumn   : two text columns, 10 point font, single spaced article.
%%                This is the most compact and represent the final published
%%                derived PDF copy of the accepted manuscript from the publisher
%%  manuscript  : one text column, 12 point font, double spaced article.
%%  preprint    : one text column, 12 point font, single spaced article.  
%%  preprint2   : two text columns, 12 point font, single spaced article.
%%  modern      : a stylish, single text column, 12 point font, article with
%% 		  wider left and right margins. This uses the Daniel
%% 		  Foreman-Mackey and David Hogg design.
%%  RNAAS       : Supresses an abstract. Originally for RNAAS manuscripts 
%%                but now that abstracts are required this is obsolete for
%%                AAS Journals. Authors might need it for other reasons. DO NOT
%%                use \begin{abstract} and \end{abstract} with this style.
%%
%% Note that you can submit to the AAS Journals in any of these 6 styles.
%%
%% There are other optional arguments one can invoke to allow other stylistic
%% actions. The available options are:
%%
%%   astrosymb    : Loads Astrosymb font and define \astrocommands. 
%%   tighten      : Makes baselineskip slightly smaller, only works with 
%%                  the twocolumn substyle.
%%   times        : uses times font instead of the default
%%   linenumbers  : turn on lineno package.
%%   trackchanges : required to see the revision mark up and print its output
%%   longauthor   : Do not use the more compressed footnote style (default) for 
%%                  the author/collaboration/affiliations. Instead print all
%%                  affiliation information after each name. Creates a much 
%%                  longer author list but may be desirable for short 
%%                  author papers.
%% twocolappendix : make 2 column appendix.
%%   anonymous    : Do not show the authors, affiliations and acknowledgments 
%%                  for dual anonymous review.
%%
%% these can be used in any combination, e.g.
%%
%% \documentclass[twocolumn,linenumbers,trackchanges]{aastex631}
%%
%% AASTeX v6.* now includes \hyperref support. While we have built in specific
%% defaults into the classfile you can manually override them with the
%% \hypersetup command. For example,
%%
%% \hypersetup{linkcolor=red,citecolor=green,filecolor=cyan,urlcolor=magenta}
%%
%% will change the color of the internal links to red, the links to the
%% bibliography to green, the file links to cyan, and the external links to
%% magenta. Additional information on \hyperref options can be found here:
%% https://www.tug.org/applications/hyperref/manual.html#x1-40003
%%
%% Note that in v6.3 "bookmarks" has been changed to "true" in hyperref
%% to improve the accessibility of the compiled pdf file.
%%
%% If you want to create your own macros, you can do so
%% using \newcommand. Your macros should appear before
%% the \begin{document} command.
%%
\newcommand{\vdag}{(v)^\dagger}
\newcommand\aastex{AAS\TeX}
\newcommand\latex{La\TeX}
\newcommand{\objname}{2010 LH$_{15}$}
\newcommand{\objnameFull}{2010 LH$_{15}$ (alternately designated 2010 TJ$_{175}$)} % 3/11/2023 COC

%% Reintroduced the \received and \accepted commands from AASTeX v5.2
%\received{March 1, 2021}
%\revised{April 1, 2021}
%\accepted{\today}

%% Command to document which AAS Journal the manuscript was submitted to.
%% Adds "Submitted to " the argument.
%\submitjournal{PSJ}

%% For manuscript that include authors in collaborations, AASTeX v6.31
%% builds on the \collaboration command to allow greater freedom to 
%% keep the traditional author+affiliation information but only show
%% subsets. The \collaboration command now must appear AFTER the group
%% of authors in the collaboration and it takes TWO arguments. The last
%% is still the collaboration identifier. The text given in this
%% argument is what will be shown in the manuscript. The first argument
%% is the number of author above the \collaboration command to show with
%% the collaboration text. If there are authors that are not part of any
%% collaboration the \nocollaboration command is used. This command takes
%% one argument which is also the number of authors above to show. A
%% dashed line is shown to indicate no collaboration. This example manuscript
%% shows how these commands work to display specific set of authors 
%% on the front page.
%%
%% For manuscript without any need to use \collaboration the 
%% \AuthorCollaborationLimit command from v6.2 can still be used to 
%% show a subset of authors.
%
%\AuthorCollaborationLimit=2
%
%% will only show Schwarz & Muench on the front page of the manuscript
%% (assuming the \collaboration and \nocollaboration commands are
%% commented out).
%%
%% Note that all of the author will be shown in the published article.
%% This feature is meant to be used prior to acceptance to make the
%% front end of a long author article more manageable. Please do not use
%% this functionality for manuscripts with less than 20 authors. Conversely,
%% please do use this when the number of authors exceeds 40.
%%
%% Use \allauthors at the manuscript end to show the full author list.
%% This command should only be used with \AuthorCollaborationLimit is used.

%% The following command can be used to set the latex table counters.  It
%% is needed in this document because it uses a mix of latex tabular and
%% AASTeX deluxetables.  In general it should not be needed.
%\setcounter{table}{1}

%%%%%%%%%%%%%%%%%%%%%%%%%%%%%%%%%%%%%%%%%%%%%%%%%%%%%%%%%%%%%%%%%%%%%%%%%%%%%%%%
%%
%% The following section outlines numerous optional output that
%% can be displayed in the front matter or as running meta-data.
%%
%% If you wish, you may supply running head information, although
%% this information may be modified by the editorial offices.
%\shorttitle{AASTeX v6.3.1 Sample article}
%\shortauthors{Schwarz et al.}
%%
%% You can add a light gray and diagonal water-mark to the first page 
%% with this command:
%% \watermark{text}
%% where "text", e.g. DRAFT, is the text to appear.  If the text is 
%% long you can control the water-mark size with:
%% \setwatermarkfontsize{dimension}
%% where dimension is any recognized LaTeX dimension, e.g. pt, in, etc.
%%
%%%%%%%%%%%%%%%%%%%%%%%%%%%%%%%%%%%%%%%%%%%%%%%%%%%%%%%%%%%%%%%%%%%%%%%%%%%%%%%%
%\graphicspath{{./}{figures/}}
%% This is the end of the preamble.  Indicate the beginning of the
%% manuscript itself with \begin{document}.

\begin{document}

\title{New Recurrently Active Main-belt Comet 2010 LH15} % 3/13/2023 COC
% \title{Active Asteroids Citizen Science: New Main-belt Comet 2010 LH15} % 3/10/2023 COC
% testing Texifier refresh via Dropbox 3/11/2023 COC

%% LaTeX will automatically break titles if they run longer than
%% one line. However, you may use \\ to force a line break if
%% you desire. In v6.31 you can include a footnote in the title.

%% A significant change from earlier AASTEX versions is in the structure for 
%% calling author and affiliations. The change was necessary to implement 
%% auto-indexing of affiliations which prior was a manual process that could 
%% easily be tedious in large author manuscripts.
%%
%% Use \affiliation for affiliation information. The old \affil is now aliased to \affiliation. AASTeX v6.31 will automatically index these in the header. When a duplicate is found its index will be the same as its previous entry.
%%
%% The new \altaffiliation can be used to indicate some secondary information such as fellowships. This command produces a non-numeric footnote that is set away from the numeric \affiliation footnotes.  NOTE that if an \altaffiliation command is used it must come BEFORE the \affiliation call, right after the \author command, in order to place the footnotes in the proper location.
%%
%% Use \email to set provide email addresses. Each \email will appear on its own line so you can put multiple email address in one \email call. A new \correspondingauthor command is available in V6.31 to identify the corresponding author of the manuscript. It is the author's responsibility to make sure this name is also in the author list.
%%

%\correspondingauthor{August Muench}
%\email{greg.schwarz@aas.org, gus.muench@aas.org}

\correspondingauthor{Colin Orion Chandler}
\email{coc123@uw.edu}

\author[0000-0001-7335-1715]{Colin Orion Chandler}
\affiliation{Dept. of Astronomy \& the DiRAC Institute, University of Washington, 3910 15th Ave NE, Seattle, WA 98195, USA}
\affiliation{LSST Interdisciplinary Network for Collaboration and Computing, 933 N. Cherry Avenue, Tucson, AZ 85721, USA}
\affiliation{Dept. of Astronomy \& Planetary Science, Northern Arizona University, PO Box 6010, Flagstaff, AZ 86011, USA}

\author[0000-0001-5750-4953]{William J. Oldroyd}
\affiliation{Dept. of Astronomy \& Planetary Science, Northern Arizona University, PO Box 6010, Flagstaff, AZ 86011, USA}

\author[0000-0001-7225-9271]{Henry H. Hsieh}
\affiliation{Planetary Science Institute, 1700 East Fort Lowell Rd., Suite 106, Tucson, AZ 85719, USA}
\affiliation{Institute of Astronomy and Astrophysics, Academia Sinica, P.O.\ Box 23-141, Taipei 10617, Taiwan}

\author[0000-0001-9859-0894]{Chadwick A. Trujillo}
\affiliation{Dept. of Astronomy \& Planetary Science, Northern Arizona University, PO Box 6010, Flagstaff, AZ 86011, USA}

\author[0000-0002-6023-7291]{William A. Burris}
\affiliation{Dept. of Physics, San Diego State University, 5500 Campanile Drive, San Diego, CA 92182, USA}
\affiliation{Dept. of Astronomy \& Planetary Science, Northern Arizona University, PO Box 6010, Flagstaff, AZ 86011, USA}

\author[0000-0001-8531-038X]{Jay K. Kueny}
\altaffiliation{National Science Foundation Graduate Research Fellow}
\affiliation{University of Arizona Dept. of Astronomy and Steward Observatory, 933 North Cherry Avenue Rm. N204, Tucson, AZ 85721, USA}
\affiliation{Lowell Observatory, 1400 W Mars Hill Rd, Flagstaff, AZ 86001, USA}
\affiliation{Dept. of Astronomy \& Planetary Science, Northern Arizona University, PO Box 6010, Flagstaff, AZ 86011, USA}
% \affiliation{Wyant College of Optical Sciences, University of Arizona, 1630 E. University Blvd., Tucson, AZ 85721, USA}

% JAD contributions: forum moderating, monitoring project completion status, observing at VATT + LDT
\author[0000-0002-7489-5893]{Jarod A. DeSpain}
\affiliation{Dept. of Astronomy \& Planetary Science, Northern Arizona University, PO Box 6010, Flagstaff, AZ 86011, USA}

% KAF contributions: observing at VATT + LDT
\author[0000-0003-2521-848X]{Kennedy A. Farrell}
\affiliation{Dept. of Astronomy \& Planetary Science, Northern Arizona University, PO Box 6010, Flagstaff, AZ 86011, USA}

\author[0000-0002-2204-6064]{Michele T. Mazzucato} % @nilium (#1 classifier!)
\altaffiliation{Active Asteroids Citizen Scientist}
\affiliation{Royal Astronomical Society, Burlington House, Piccadilly, London, W1J 0BQ, UK}

\author[0000-0002-9766-2400]{Milton K. D. Bosch} % @mboschmd
% \affiliation{}
\altaffiliation{Active Asteroids Citizen Scientist}

\author{Tiffany Shaw-Diaz} % @Tiffany_Shaw-Diaz
\altaffiliation{Active Asteroids Citizen Scientist}

\author{Virgilio Gonano} % @Wirg78 2/15/2023 COC
\altaffiliation{Active Asteroids Citizen Scientist}


% \collaboration{20}{(AAS Journals Data Editors)}

% \author{F.X Timmes}
% \affiliation{Arizona State University}
% \affiliation{AAS Journals Associate Editor-in-Chief}

% \author{Amy Hendrickson}
% \altaffiliation{AASTeX v6+ programmer}
% \affiliation{TeXnology Inc.}

% \author{Julie Steffen}
% \affiliation{AAS Director of Publishing}
% \affiliation{American Astronomical Society \\
% 1667 K Street NW, Suite 800 \\
% Washington, DC 20006, USA}

%% Mark off the abstract in the ``abstract'' environment. 
\begin{abstract}
% NOTE: RNAAS 150 word limit!
% 
We announce the discovery of a main-belt comet (MBC), \objnameFull{}. MBCs are a rare type of main-belt asteroid that display comet-like activity, such as tails or comae, caused by sublimation. Consequently, MBCs help us map the location of solar system volatiles, providing insight into the origins of material prerequisite for life as we know it. However, MBCs have proven elusive, with fewer than 20 found among the 1.1 million known main-belt asteroids. This finding derives from Active Asteroids, a NASA Partner Citizen Science program we designed to identify more of these important objects. After volunteers classified an image of \objname{} as showing activity, we carried out a follow-up investigation which revealed evidence of activity from two epochs spanning nearly a decade. This discovery is timely, with \objname{} inbound towards its 2024 March perihelion passage, with potential activity onset as early as late 2023.
% cycles to record potential reactivation and to a priori acquire, e.g.,  colors of the nucleus prior to activity-induced obfuscation.
\end{abstract}

%% Keywords should appear after the \end{abstract} command. 
%% The AAS Journals now uses Unified Astronomy Thesaurus concepts: https://astrothesaurus.org
%% You will be asked to selected these concepts during the submission process but this old "keyword" functionality is maintained in case authors want to include these concepts in their preprints.
\keywords{
Asteroid belt (70), 
Asteroids (72), 
Comae (271), 
Comet tails (274)
% Classical Novae (251) --- Ultraviolet astronomy(1736) --- History of astronomy(1868) --- Interdisciplinary astronomy(804)
}


%%%%%%%%%%%%%%%%%%%%%%%%%%%%%%%%%%%%%%%%%%%%%%%%%%%%%%%%%%%%%%%%
\section{Introduction} \label{sec:intro}

Main-belt comets (MBCs) are rare, with fewer than 20 found among the 1.1 million known main-belt asteroids. MBCs represent a subpopulation of the active asteroids, which are small solar system bodies that exhibit comet-like activity (i.e., tails, comae) but have asteroidal orbits \citep{jewittActiveAsteroids2015}. The MBCs are active asteroids that are specifically found in the main asteroid belt, and whose activity is attributed to sublimation \citep{hsiehMainbeltCometsPanSTARRS12015}. Knowledge of these objects and their composition help us map the location of solar system volatiles, thereby improving understanding of the origins of the ingredients for life as we know it.% With so few objects discovered, much remains unknown about the MBCs and the broader active asteroid population. 


%%%%%%%%%%%%%%%%%%%%%%%%%%%%%%%%%%%%%%%%%%%%%%%%%%%%%%%%%%%%%%%%
\section{Methods}
\label{sec:methods}

% To improve our understanding of these populations, 
To find more of these remarkable objects we created the Citizen Science program \textit{Active Asteroids}\footnote{\url{http://activeasteroids.net}}, a NASA Partner. Participants classify images of known minor planets, which we extracted from the Dark Energy Camera (DECam) public archive \citep{chandlerSAFARISearchingAsteroids2018,chandlerSixYearsSustained2019,chandlerCometaryActivityDiscovered2020b,chandlerRecurrentActivityActive2021,chandlerMigratoryOutburstingQuasiHilda2022}, as either active or inactive. We investigate activity candidates by conducting archival image searches and follow-up telescope observations, then report our confirmed discoveries \citep[e.g.,][]{oldroydCometlikeActivityDiscovered2023,chandlerNewActiveAsteroid2023,chandlerDiscoveryDustEmission2023}.



%%%%%%%%%%%%%%%%%%%%%%%%%%%%%%%%%%%%%%%%%%%%%%%%%%%%%%%%%%%%%%%%
\section{Results}
\label{sec:results}

One DECam image (Figure \ref{fig:activity}) of \objname{} ($a=2.74$~au, eccentricity $e=0.35$, inclination $i=10.9^\circ$, perihelion distance $q=1.77$~au, aphelion distance $Q=3.72$~au, Tisserand's parameter with respect to Jupiter $T_\mathrm{J}=3.23$, retrieved UT 2023 March 11 from JPL Horizons; \citealt{giorginiJPLOnLineSolar1996}) originally acquired UT 2019 September 30, was unanimously classified as showing activity by \textit{Active Asteroids} volunteers. Our archival investigation revealed additional images (examples provided in Figure \ref{fig:activity}) unambiguously showing activity from two separate orbital epochs: $\sim$10 images from UT 2010 September 27 -- October 10; (true anomaly range of $20.5^\circ<\nu<27.6^\circ$) and $>10$ between UT 2019 August 10 -- November 3 ($-14.2^\circ<\nu<26.5^\circ$). All images of activity were taken when \objname{} was approximately near perihelion passage ($\nu=0^\circ$). When considered with the recurrent activity, this indicates that sublimation is the probable underlying activity mechanism. Hence, given its main-belt orbit, \objname{} is an MBC.

\begin{figure}[h]
    \centering
    \begin{tabular}{ccc}
    	\includegraphics[width=0.32\linewidth]{2010_LH15_2010-09-27_11.30.06.322000_rings.v3.skycell.0970.085.wrp.i.55466_47858_chip1_126arcsec_NuEl.png} & \includegraphics[width=0.32\linewidth]{2010_LH15_2019-08-31_08.53.55.963000_ztf_20190831370440_000293_zr_c02_o_q4_sciimg_chip0-4_126arcsec_NuEl.png} & \includegraphics[width=0.32\linewidth]{2010_LH15_2019-09-30_02.35.59.092218_c4d_190930_023514_opi_i_v1_chip16-S17_126arcsec_NuEl.png}\\
    \end{tabular}
    \caption{\objname\ (green dashed arrows) with a tail (white arrows) oriented on sky roughly towards the anti-solar (yellow -$\odot$) direction, and counter-clockwise from the anti-motion (red $-v$) vector. The FOV of each image is $126'' \times 126''$. 
    \textbf{Left:} UT 2020 September 27 45~s $i$-band GigaPixel1 exposure on the 1.8~m Pan-STARRS~1 (Haleakala). 
    \textbf{Center:} UT 2019 August 31 Zwicky Transient Facility (ZTF) camera on the 48'' Samuel Oschin telescope (Mt. Palomar) 30~s $r$ band exposure.  
    \textbf{Right:} UT 2019 September 30 90~s Dark Energy Camera (DECam) on the 4~m Blanco telescope (Cerro Tololo Inter-American Observatory, Chile); Prop. ID 2019B-1014, PI Olivares, observers F. Olivares, I. Sanchez. This was the image classified as active by \textit{Active Asteroids} volunteers.% Felipe Olivares, Ignacio Sanchez; delve-wide program
    }
    \label{fig:activity}
\end{figure}

\objname{} is currently observable (as of 2023 March 15; $\nu\sim240^{\circ}$) and currently inbound to its 2024 March 26 perihelion passage, and may become active again as early as 2023 October when it reaches $\nu=290^\circ$, the earliest activity onset point observed to date for an MBC \citep{hsiehObservationalCharacterizationMainBelt2023}, just before its current observing window ends.  It is also important to observe the target prior to its possible reactivation in order to measure properties of the nucleus (e.g., color, absolute magnitude, rotation rate) in the absence of activity, which will impede these measurements. After 2023 October, the object will next be observable from 2024 July to 2025 May ($45^{\circ}\lesssim\nu\lesssim130^{\circ}$).  Observations of the object during these available windows to characterize its expected activity are highly encouraged.



%%%%%%%%%%%%%%%%%%%%%%%%%%%%%%%%%%%%%%%%%%%%%%%%%%%%%%%%%%%%%%%%
\clearpage
\section*{Acknowledgements}
\begin{acknowledgments}
% IMPORTANT: leave a blank line after this one!

\textbf{General:} We thank Dr.\ Mark Jesus Mendoza Magbanua (University of California San Francisco) for his ongoing, timely feedback on the project and observing accompaniment.

\textbf{Citizen Science:} We thank Elizabeth Baeten (Belgium) for moderating the Active Asteroids forums. We thank our NASA Citizen Scientists that examined \objname{}: 
% I just sent these requests evening of 3/10/2023 COC
% @ACTverotayl
% @Cool_ESA_Stargirl
Al Lamperti (Royersford, USA), % @lamperti
Angela Hoffa (Greenfield, USA), % @ThoseWhoWander
Carl L. King (Ithaca, USA), % @ck12074
Jayanta Ghosh (Purulia, India), % @prime23
Konstantinos Dimitrios Danalis (Athens, Greece), % @MrAgent99
Lydia Yvette Solis (Nuevo, USA), % @ACTlysolisy
Michele T. Mazzucato (Florence, Italy), % @nilium
Milton K. D. Bosch MD (Napa, USA), % @mboschmd
Panagiotis J. Ntais (Philothei, Greece), % @GreekHunter49
and 
Virgilio Gonano (Udine, Italy). % @Wirg78
% @actchavewcl
% @gonzaopa
%
% %
We also thank super-classifiers 
Ivan A. Terentev (Petrozavodsk, Russia) % @Dolorous_Edd
and 
Marvin W. Huddleston (Mesquite, USA)% @kc5lei
% and 
% Milton K D Bosch, MD (Napa, USA) % @miltonbosch
. 
Many thanks to Cliff Johnson (Zooniverse) and Marc Kuchner (NASA) for their ongoing Citizen Science guidance.

\textbf{Funding:} This material is based upon work supported by the NSF Graduate Research Fellowship Program under grant No.\ 2018258765 and grant No.\ 2020303693. % removing second sentence 1/31/2023 COC (too long) Any opinions, findings, and conclusions or recommendations expressed in this material are those of the author(s) and do not necessarily reflect the views of the National Science Foundation.  
C.O.C., H.H.H., and C.A.T.\ acknowledge support from NASA grant 80NSSC19K0869. 
W.J.O. and C.A.T.\ acknowledge support from NASA grant 80NSSC21K0114. 
This work was supported in part by NSF awards 1950901 (NAU REU program in astronomy and planetary science). % 7/12/2022 COC -- per DET, for WAB?
Computational analyses were run on Northern Arizona University's Monsoon computing cluster, funded by Arizona's Technology and Research Initiative Fund.% This work was made possible in part through the State of Arizona Technology and Research Initiative Program. 

\textbf{Software \& Services:} 
% ``GNU's Not Unix!'' (GNU) Astro \textit{astfits} \citep{Akhlaghi:2015gv} provided command-line FITS file header access.
% C Flexible Image Transport System Input Output (CFITSIO) enabled FITS compression and more \cite{Pence:1999us}. 
World Coordinate System corrections facilitated by \textit{Astrometry.net} \citep{langAstrometryNetBlind2010}. 
This research has made use of 
NASA's Astrophysics Data System, 
%. 
%This research has made use of 
the NASA/IPAC Infrared Science Archive, % clipped remainder 3/12/2023 COC which is funded by the National Aeronautics and Space Administration and operated by the California Institute of Technology.
% This research has made use of 
the Institut de M\'ecanique C\'eleste et de Calcul des \'Eph\'em\'erides SkyBoT Virtual Observatory tool \citep{berthierSkyBoTNewVO2006}, 
and 
data and/or services provided by the International Astronomical Union's Minor Planet Center, 
SAOImageDS9, developed by Smithsonian Astrophysical Observatory \citep{joyeNewFeaturesSAOImage2006}. 
% This work made use of the Lowell Observatory Asteroid Orbit Database \textit{astorbDB} \citep{bowellPublicDomainAsteroid1994,moskovitzAstorbDatabaseLowell2021}. 
% COC: already cited in software: This work made use of the \textit{astropy} software package \citep{robitailleAstropyCommunityPython2013}.

\textbf{Facilities \& Instrumentation:} This project used data obtained with the Dark Energy Camera (DECam), which was constructed by the Dark Energy Survey (DES) collaboration. % Cutting off here to comply with AAS word limits 1/31/2023 COC
%Funding for the DES Projects has been provided by the US Department of Energy [and $\sim$20 others].%, the US National Science Foundation, the Ministry of Science and Education of Spain, the Science and Technology Facilities Council of the United Kingdom, the Higher Education Funding Council for England, the National Center for Supercomputing Applications at the University of Illinois at Urbana-Champaign, the Kavli Institute for Cosmological Physics at the University of Chicago, Center for Cosmology and Astro-Particle Physics at the Ohio State University, the Mitchell Institute for Fundamental Physics and Astronomy at Texas A\&M University, Financiadora de Estudos e Projetos, Fundação Carlos Chagas Filho de Amparo à Pesquisa do Estado do Rio de Janeiro, Conselho Nacional de Desenvolvimento Científico e Tecnológico and the Ministério da Ciência, Tecnologia e Inovação, the Deutsche Forschungsgemeinschaft and the Collaborating Institutions in the Dark Energy Survey. The Collaborating Institutions are Argonne National Laboratory [and 25 others].%, the University of California at Santa Cruz, the University of Cambridge, Centro de Investigaciones Enérgeticas, Medioambientales y Tecnológicas–Madrid, the University of Chicago, University College London, the DES-Brazil Consortium, the University of Edinburgh, the Eidgenössische Technische Hochschule (ETH) Zürich, Fermi National Accelerator Laboratory, the University of Illinois at Urbana-Champaign, the Institut de Ciències de l’Espai (IEEC/CSIC), the Institut de Física d’Altes Energies, Lawrence Berkeley National Laboratory, the Ludwig-Maximilians Universität München and the associated Excellence Cluster Universe, the University of Michigan, NSF’s NOIRLab, the University of Nottingham, the Ohio State University, the OzDES Membership Consortium, the University of Pennsylvania, the University of Portsmouth, SLAC National Accelerator Laboratory, Stanford University, the University of Sussex, and Texas A\&M University. 
This research uses services or data provided by the Astro Data Archive at NSF's NOIRLab. % Trimming duplicate text of 2nd sentence 1/31/2023 COC: NOIRLab is operated by the Association of Universities for Research in Astronomy (AURA), Inc. under a cooperative agreement with the National Science Foundation. 
Based on observations at Cerro Tololo Inter-American Observatory, NSF’s NOIRLab (NOIRLab Prop. ID 2019B-1014; PI: F. Olivares), % updated 3/12/2023 COC % clipping remainder for word count 1/31/2023 COC, which is managed by the Association of Universities for Research in Astronomy under a cooperative agreement with the National Science Foundation.
%
the Pan-STARRS1 Surveys (PS1) and the PS1 public science archive \citep{chambersPanSTARRS1Surveys2016},
% The Pan-STARRS1 Surveys (PS1) and the PS1 public science archive have been made possible through contributions by the Institute for Astronomy, the University of Hawaii, the Pan-STARRS Project Office, the Max-Planck Society and its participating institutes, the Max Planck Institute for Astronomy, Heidelberg and the Max Planck Institute for Extraterrestrial Physics, Garching, The Johns Hopkins University, Durham University, the University of Edinburgh, the Queen's University Belfast, the Harvard-Smithsonian Center for Astrophysics, the Las Cumbres Observatory Global Telescope Network Incorporated, the National Central University of Taiwan, the Space Telescope Science Institute, the National Aeronautics and Space Administration under Grant No. NNX08AR22G issued through the Planetary Science Division of the NASA Science Mission Directorate, the National Science Foundation Grant No. AST-1238877, the University of Maryland, Eotvos Lorand University (ELTE), the Los Alamos National Laboratory, and the Gordon and Betty Moore Foundation. 
the Zwicky Transient Facility \citep{bellmZwickyTransientFacility2019},
%
% Magellan observations made use of the Inamori-Magellan Areal Camera and Spectrograph (IMACS) instrument \citep{dresslerIMACSInamoriMagellanAreal2011}; PI C. Trujillo.
%
% added this but it is long 1/29/2023 COC
% TODO IPAC (IRSA?) DOI 3/12/2023 COC
%The Legacy Surveys consist of three individual and complementary projects: the DECam Legacy Survey (DECaLS; Proposal ID \#2014B-0404; PIs: David Schlegel and Arjun Dey), the Beijing-Arizona Sky Survey (BASS; NOAO Prop. ID \#2015A-0801; PIs: Zhou Xu and Xiaohui Fan), and the Mayall z-band Legacy Survey (MzLS; Prop. ID \#2016A-0453; PI: Arjun Dey). % Commenting remainder 1/31/2023 COC: DECaLS, BASS and MzLS together include data obtained, respectively, at the Blanco telescope, Cerro Tololo Inter-American Observatory, NSF's NOIRLab; the Bok telescope, Steward Observatory, University of Arizona; and the Mayall telescope, Kitt Peak National Observatory, NOIRLab. The Legacy Surveys project is honored to be permitted to conduct astronomical research on Iolkam Du'ag (Kitt Peak), a mountain with particular significance to the Tohono O'odham Nation. BASS is a key project of the Telescope Access Program (TAP), which has been funded by the National Astronomical Observatories of China, the Chinese Academy of Sciences (the Strategic Priority Research Program ``The Emergence of Cosmological Structures'' Grant \# XDB09000000), and the Special Fund for Astronomy from the Ministry of Finance. The BASS is also supported by the External Cooperation Program of Chinese Academy of Sciences (Grant \# 114A11KYSB20160057), and Chinese National Natural Science Foundation (Grant \# 11433005). The Legacy Survey team makes use of data products from the Near-Earth Object Wide-field Infrared Survey Explorer (NEOWISE), which is a project of the Jet Propulsion Laboratory/California Institute of Technology. NEOWISE is funded by the National Aeronautics and Space Administration. The Legacy Surveys imaging of the DESI footprint is supported by the Director, Office of Science, Office of High Energy Physics of the U.S. Department of Energy under Contract No. DE-AC02-05CH1123, by the National Energy Research Scientific Computing Center, a DOE Office of Science User Facility under the same contract; and by the U.S. National Science Foundation, Division of Astronomical Sciences under Contract No. AST-0950945 to NOAO.
%
%The LBT is an international collaboration among institutions in the United States, Italy and Germany. LBT Corporation Members are: The University of Arizona on behalf of the Arizona Board of Regents; Istituto Nazionale di Astrofisica, Italy; LBT Beteiligungsgesellschaft, Germany, representing the Max-Planck Society, The Leibniz Institute for Astrophysics Potsdam, and Heidelberg University; The Ohio State University, representing OSU, University of Notre Dame, University of Minnesota and University of Virginia. 
%
%The VATT referenced herein refers to the Vatican Observatory’s Alice P. Lennon Telescope and Thomas J. Bannan Astrophysics Facility. We are grateful to the Vatican Observatory for the generous time allocations (Proposal ID S165, PI Chandler).
and the CADC Solar System Object Information Search \citep{gwynSSOSMovingObjectImage2012}.
\end{acknowledgments}

\vspace{5mm}
\facilities{
% HST(STIS), 
% Swift(XRT and UVOT), 
% AAVSO, 
% CTIO:1.3m, 
% CTIO:1.5m, 
CTIO:4m (DECam), 
IRSA\footnote{\url{https://www.ipac.caltech.edu/doi/irsa/10.26131/IRSA539}}, % required if we have anything from ZTF, PTF, WISE, etc. 3/13/2023 COC
%LBT (LBCB, LBCR), 
% Magellan:Baade (IMACS), 
PO:1.2 m (PTF, ZTF), % PO:1.2 m (PTF, ZTF)
PS1
%VATT (VATT4K), 
% CXO
}

%% Similar to \facility{}, there is the optional \software command to allow authors a place to specify which programs were used during the creation of  the manuscript. Authors should list each code and include either a citation or url to the code inside ()s when available.

\software{
        astropy \citep{robitailleAstropyCommunityPython2013}, 
        % {\tt astrometry.net} \citep{langAstrometryNetBlind2010}, % already in acks 1/31/2023 COC
        % {\tt JPL Horizons} \citep{giorginiJPLOnLineSolar1996}, % already in acks, text 1/31/2023 COC
        {\tt Matplotlib} \citep{hunterMatplotlib2DGraphics2007},
        {\tt NumPy} \citep{harrisArrayProgrammingNumPy2020},
        {\tt pandas} \citep{rebackPandasdevPandasPandas2022}, % limiting to 1 citation, removing older mckinneyDataStructuresStatistical2010 1/31/2023 COC
        {\tt SAOImageDS9} \citep{joyeNewFeaturesSAOImage2006},
        {\tt SciPy} \citep{virtanenSciPyFundamentalAlgorithms2020}
        % {\tt Siril}\footnote{\url{https://siril.org}}, % actually not used in this work (but will be for the full paper) so removing 1/31/2023 COC
        % {\tt SkyBot} \citep{berthierSkyBoTNewVO2006} % cited in acks already
          }

%% Appendix material should be preceded with a single \appendix command. There should be a \section command for each appendix. Mark appendix subsections with the same markup you use in the main body of the paper.

%% Each Appendix (indicated with \section) will be lettered A, B, C, etc. The equation counter will reset when it encounters the \appendix command and will number appendix equations (A1), (A2), etc. The Figure and Table counter will not reset.

% \appendix

% \section{Appendix information}

% Appendices can be broken into separate sections just like in the main text. The only difference is that each appendix section is indexed by a letter (A, B, C, etc.) instead of a number.  Likewise numbered equations have the section letter appended.  Here is an equation as an example.

% \begin{equation}
% I = \frac{1}{1 + d_{1}^{P (1 + d_{2} )}}
% \end{equation}
% Appendix tables and figures should not be numbered like equations. Instead
% they should continue the sequence from the main article body.


% \section{Author publication charges} \label{sec:pubcharge}

% In April 2011 the traditional way of calculating author charges based on the number of printed pages was changed.  The reason for the change was due to a recognition of the growing number of article items that could not be represented in print. Now author charges are determined by a number of digital ``quanta''.  A single quantum is 350 words, one figure, one table, and one enhanced digital item.  For the latter this includes machine readable tables, figure sets, animations, and interactive figures.  The current cost for the different quanta types is available at \url{https://journals.aas.org/article-charges-and copyright/#author_publication_charges}. Authors may use the online length calculator to get an estimate of the number of word and float quanta in their manuscript. The calculator is located at \url{https://authortools.aas.org/Quanta/newlatexwordcount.html}.


\clearpage
\bibliography{zotero}{}
\bibliographystyle{aasjournal}

%% This command is needed to show the entire author+affiliation list when the collaboration and author truncation commands are used.  It has to go at the end of the manuscript.
%\allauthors

%% Include this line if you are using the \added, \replaced, \deleted commands to see a summary list of all changes at the end of the article.
%\listofchanges

\end{document}
