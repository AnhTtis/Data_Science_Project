\section{Testing for non-Gaussian noise}
\label{sec:testing_gaussian}

The noise distribution of HCI residuals can vary within and between observations. To confirm that the noise is sufficiently normal for the t-test to be applicable, various tests are carried out prior to the calculation of the detection limits. For example \cite{mawetFUNDAMENTALLIMITATIONSHIGH2014} and \cite{ottenONSKYPERFORMANCEANALYSIS2017} compute histograms based on pixel values within annuli of $1 \lambda / D$ width. A visual inspection of histograms can provide first insights about the general distribution shape. A direct comparison with the normal distribution, however, is difficult. This is especially the case for the tails of the distribution, i.e. the occurrence rate of bright events. These events are of special importance for the calculation of the detection limits. Q-Q plots are a good alternative to histograms as they allow to directly compare the noise with the normal distribution. In this way, insights about a possible over- or underestimation of the contrast can be obtained.

\begin{figure}[t!]
	\epsscale{1.15}
	\plotone{14_Residual_Shapiro.pdf}
	\caption{Sensitivity and false positive rate (FPR) of the Shapiro-Wilk test as a function of separation from the star. For each separation, $10^5$ samples are drawn, each containing as many values as independent resolution elements are available. We sample from a Gaussian distribution to measure the FPR (blue) and from a Laplacian distribution to measure the sensitivity (orange). The left and right plots show the results for two different thresholds $p < 0.01$ and $p < 0.2$.}
	\label{fig:shapiro}
\end{figure}

The Shapiro-Wilk test \citep{shaphiroAnalysisVarianceTest1965} can be used as a quantitative measure for normality. The test can be used to reject the null hypothesis that a sample was drawn from a normal distribution. For the test to be applicable, it is assumed that the values of the sample are independent. Due to the spatial size of the speckles, neighboring pixel in the HCI residuals are never independent. Thus, the test has to be calculated on pixel which are at least \mbox{1 FWHM} apart. Results computed on non-independent values \cite[see e.g.][]{absilSearchingCompanionsAU2013} are likely biased. The output of the test is a p-value corresponding to the remaining risk that the sample is compatible with the null hypothesis. For example, if we observe $p=0.01$, we know that $1\%$ of the observations under Gaussian noise have a value of the test statistic equal or more extreme than ours. By setting a threshold, we can specify the accepted error by which Gaussian noise is misclassified as non-Gaussian noise. Although the Shapiro-Wilk test has shown superior performance compared to other tests \citep{razaliPowerComparisonsShapirowilk2011}, it lacks power in detecting non-Gaussian noise at close separation to the star. We study this effect in \mbox{Figure \ref{fig:shapiro}}.

The highest true positive rate is reached for the threshold $p<0.2$ at 8 FWHM and is just about $60\%$. The results get worse for close separations. This is problematic as speckle noise is expected to be the dominant noise source close to the star. While sensitivity is one problem, preliminary testing comes with a "logical problem" \citep{scholzKSampleAndersonDarling1987}: "Because insufficient evidence exists to reject normality, normality will be considered true" \citep{rochonTestNotTest2012}. If we cannot reject the null hypothesis, we must not conclude that our data is normal, i.e., combined with the low sensitivity for small sample sizes, we can neither prove nor disprove normality. As concluded in \cite{rochonTestNotTest2012} "support for the assumption of normality must come from extra-data sources" \citep{easterlingEffectPreliminaryNormality1978}.
