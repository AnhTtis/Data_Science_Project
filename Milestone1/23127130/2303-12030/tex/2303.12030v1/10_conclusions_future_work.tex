\section{Summary and Conclusions}
\label{sec:conclusion}
In this paper we presented a new framework to quantify exoplanet detections in high-contrast imaging. Through the use of parametric bootstrapping, our new metric is able to estimate robust detection limits for any type of residual noise. Our metric, however, assumes that the noise distribution is known. A comparison of the detection limits under the assumption of Gaussian and Laplacian noise revealed that commonly used metrics, such as the t-test, might be too optimistic in case of speckle dominated observations. 
The occurrence rate of large noise values is higher in case of Laplacian noise compared to Gaussian noise. This results in a higher risk to obtain a high signal-to-noise value originating from the noise. For example, the risk that the noise produces a false detection at 2 FWHM distance from the star with a signal-to-noise ratio of 5 is about 250 times higher under Laplacian than under Gaussian noise. Therefore, the signal-to-noise ratio should not be considered as a direct measure for the detection uncertainty. Only if we take the sample size and the correct noise distribution into account the signal-to-noise ratio becomes interpretable. 

The link between the detection limit and the noise characteristics makes a fair comparison between HCI observations difficult. This is especially the case if the noise distributions differ between the datasets we want to compare. An example for this is the development of new post-processing algorithms. If we compare algorithms that produce residuals with different noise, the comparison of the methods is likely biased. The same applies for any comparison between observations taken under different circumstances: e.g. ground-based vs. space-based observations, different instruments or observing strategies such as ADI and RDI. If we want to compare inhomogeneous data, we have to take possible biases arising from statistics into account.
%
We recommend to compute detection limits under different assumptions to set optimistic and conservative bounds for the achieved contrast. Studies on the residual noise using for example Q-Q plots can provide valuable insights about the noise. But they can never prove that the noise is sufficiently normal to use a t-test. Apart from graphical tools, quantitative tests such as the Shaphiro-Wilk test can only reject normality. They can never proof that the data is actually normal.

Future work should seek to better understand the speckle statistics of the HCI residuals. If we knew the true distribution of noise as a function of observing conditions, instrument, and data post-processing, we could use the bootstrapping algorithm presented in this paper to determine the true contrast. Alternatively, we can use non-parametric methods such as the non-parametric bootstrap to estimate the statistics directly from the data. Future work should further investigate the origin of non-Gaussian noise and develop new methods to account for the spatial dependencies of speckle noise.
