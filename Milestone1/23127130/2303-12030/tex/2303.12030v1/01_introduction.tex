\section{Introduction}
\label{sec:intro}
% We need a good metric to quantify our observations
In order to translate the information of exoplanet high-contrast imaging (HCI) observations into quantitative scientific results, an objective metric to quantify the brightness contrast between planet and host star is indispensable. In this context, two scientific questions are of particular interest:
% What a Standard is needed for - Two key research questions:
%
% Q1 - Is this real?
First, in case a new planet candidate is found, we want to know how confident we can be that the discovery is real. If several candidates are found during a survey, the calculated confidence can be used as a guideline for the management of follow-up time. 
%
% Q2 - What can we rule out?
Second, we want to know which planets we can safely rule out i.e. we are sure that, given some confidence level, we should have seen them in our data. This second question is the subject of studies about planet occurrence rates and completeness in large surveys such as SHINE \citep{desideraSPHEREInfraredSurvey2021,langloisSPHEREInfraredSurvey2021,viganSPHEREInfraredSurvey2021} or GPIES \citep{nielsenGeminiPlanetImager2019}.
%
Any metric used to answer these two questions must be able to deal with the large diversity of datasets available in HCI. Over the last years, HCI has gained increasing scientific importance as its discovery space complements other exoplanet detection and characterization techniques, such as radial velocity or transit observations. As a result, existing ground-based observatories feature, and continue to be upgraded with, dedicated HCI instruments \citep[e.g.,][]{kenworthyHighContrastImaging2018}. Furthermore, HCI exoplanet science is driving the development of future instruments for the upcoming 30-m class telescopes \citep[e.g,][]{quanzDirectDetectionExoplanets2015,bowensExoplanetsELTMETISEstimating2021,brandlMETISMidinfraredELT2021, kasperPCSRoadmapExoearth2021}. Also, the recently launched \emph{James Webb Space Telescope (JWST)} is equipped with a suite of HCI modes\footnote{\url{https://jwst-docs.stsci.edu/methods-and-roadmaps/jwst-high-contrast-imaging}} and a dedicated \emph{Early Release Science (ERS) program} was accepted for early execution \citep{hinkleyJWSTEarlyRelease2022}. In the long run, highly optimized space missions of large scale will be needed for the direct detection and characterization of a statistically relevant sample of temperate, terrestrial exoplanets \citep[e.g.,][]{gaudiHabitableExoplanetObservatory2018, quanzLargeInterferometerExoplanets2022, theluvoirteamLUVOIRMissionConcept2019}.

% Create the family picture
\begin{table*}[t]
	\caption{Standards for contrast quantification used in high-contrast imaging  \label{tab:family}}
	\begin{center}
	\begin{tabular}{p{4cm}P{0.5cm}P{0.5cm}P{1.5cm}P{2cm}P{1cm}P{2cm}P{3cm}}
	\hline
 	Method \& Reference & 
 	Q1 &
 	Q2&
 	Input type &
 	What is Noise? &
 	SSS &
 	Adresses Completeness & 
 	Accounts for non-Gaussian noise\\
 	\hline
 	\hline
 	 Estimation of noise PDF\tablenotemark{a} 
 		& \cmark & \cmark & RI & pixel & \xmark  & \xmark & \cmark\tablenotemark{y} \\
 		
 	 SNR \tablenotemark{b} 
 		& \cmark & \cmark & RI & both & \xmark  & \xmark &\xmark \\
 		
 	 t-Test (small sample)\tablenotemark{c}  
 		& \cmark & \cmark & RI & apertures & \cmark  & \xmark &\xmark \\
 		
 	Parametric P-Map \tablenotemark{d}   
 		& \xmark & \cmark & RI & apertures & \cmark & \cmark &\xmark \\
 		 		
 	Non-Parametric P-Map \tablenotemark{d}  
 		& \xmark & \cmark & RI & apertures & n.a.  & \cmark &\cmark\tablenotemark{z}  \\
 	
 	STIM \tablenotemark{e} 
 		& \cmark & \xmark & RI & pixel & \xmark   & \xmark & \cmark\tablenotemark{y} \\

 	ANDROMEDA\tablenotemark{f} 
 		& \cmark & \cmark & RS & pixel & \cmark  & \xmark & \xmark \\
 		
 	FMMF \tablenotemark{g} 
 		& \cmark & \cmark & RS  & pixel & \xmark  & \cmark & \cmark\tablenotemark{v} \\
 	
 	ROC curves \tablenotemark{h} 
 		& \xmark & \xmark & detection map \& RI & apertures or pixel & n.a.  & \xmark & \cmark\tablenotemark{z} \\
 		
 	Supervised ML \tablenotemark{i} 
 		& \cmark & \xmark & multiple RIs & pixel patches & n.a.  & \xmark  & \cmark\tablenotemark{u}  \\
 		
 	RSM \tablenotemark{j} 
 		& \cmark & \xmark & RS & apertures or pixel & n.a.  & \cmark  & \cmark\tablenotemark{w}  \\
 		
  	\hline
  	This paper
  		& \cmark & \cmark & RI  & spaced pixel & \cmark & \cmark\tablenotemark{x} & \cmark \\
  	\hline
\end{tabular}
	\end{center}
	\tablecomments{This table does not claim to be complete and is meant as an overview. We further note that the classification used here only considers a selection of the relevant aspects. A \cmark displays that the aspect is addressed in the cited paper, not including followup work. Abbreviations: %\newline
		\textbf{Q1}: Quantifies the uncertainty of a new detection,
		\textbf{Q2}: Provides detection limits,
		\textbf{SSS}: Accounts for small sample statistics,
		\textbf{RI}: Residual image,
		\textbf{RS}: Sequence of residuals along time,
		\textit{\textbf{u}}: Through learning typical noise pattern, \textit{\textbf{v}}: Estimates the distribution of the SNR over a whole survey, \textit{\textbf{w}}: Gaussian and Laplacian noise, \textit{\textbf{x}}: Future work, \textit{\textbf{y}}: Estimates the noise probability density function (PDF) based on pixel values, \textit{\textbf{z}}: No assumption about the noise.}
	\tablerefs{\quad 
		\mbox{\textit{a}: \cite{maroisConfidenceLevelSensitivity2008},}
		\mbox{\textit{b}: \cite{meshkatFURTHEREVIDENCEPLANETARY2013},}
	 	\mbox{\textit{c}: \cite{mawetFUNDAMENTALLIMITATIONSHIGH2014},}
	 	\mbox{\textit{d}: \cite{jensen-clemNewStandardAssessing2017},}
	 	\mbox{\textit{e}: \cite{pairetSTIMMapDetection2019},}
	 	\mbox{\textit{f}: \cite{cantalloubeDirectExoplanetDetection2015},}
	 	\mbox{\textit{g}: \cite{ruffioImprovingAssessingPlanet2017},}
	 	\mbox{\textit{h}: \cite{gonzalezLowrankSparseDecomposition2016},}
	 	\mbox{\textit{i}: \cite{gomezgonzalezSupervisedDetectionExoplanets2018},}
	 	\mbox{\textit{j}: \cite{dahlqvistRegimeswitchingModelDetection2020}}
	 	}
\end{table*}

Finding a metric that provides comparable estimates for contrast across current and future instruments is challenging because any change in instrumentation, observing conditions or data post-processing influences the characteristics of the data. Currently used standards often rely on several explicit or implicit assumptions about the noise. For example, it is often assumed that the noise in the post-processed residual image is independent identically distributed (i.i.d.) and Gaussian \cite[e.g.][]{mawetFUNDAMENTALLIMITATIONSHIGH2014,cantalloubeDirectExoplanetDetection2015}. While being an inseparable part of the metric, such assumptions are rarely verified. In fact, previous work has shown that the noise in ground-based HCI observations often deviates from Gaussian noise due to the presence of systematic speckle noise (compare \cite{perrinStructureHighStrehl2003, aimeUsefulnessLimitsCoronagraphy2004, soummerSpeckleNoiseDynamic2007} for the noise statistics in raw images and \cite{pairetSTIMMapDetection2019, dahlqvistRegimeswitchingModelDetection2020} for the noise statistics of post-processed residual images). 
%This special type of noise is caused by residual wavefront aberrations which remains uncorrected by the telescopes adaptive optics. 
If one computes detection limits under the assumption of Gaussian noise, but the actual noise is dominated by speckles, the results will be biased. The severity of this error depends on the extent to which the assumptions are violated. This makes it hard, or even impossible, to compare results across datasets and instruments with different noise characteristics.

% This work is about & Outline
In this paper, we revisit the fundamental question of \emph{"how to quantify detection limits"} in HCI. Compared to previous work \mbox{(Section \ref{sec:current_standards})}, we focus our analysis on the error budget caused by violated assumptions (\mbox{Section \ref{sec:critical}}). For this purpose we propose a new metric based on \emph{bootstrapping}, which generalizes the widely used standard by \cite{mawetFUNDAMENTALLIMITATIONSHIGH2014} \mbox{(Section \ref{sec:what_is_a_detection})} to non-Gaussian noise \mbox{(Section \ref{sec:beyond_gaussian_noise})}. This allows us to study the systematic error of our detection uncertainty as a function of the noise characteristics. We further highlight the important difference between the signal-to-noise ratio (SNR) and the detection uncertainty. In Section \ref{sec:quantify_detections_with_bs} we demonstrate how to quantify detections with our new bootstrapping metric. For this purpose, we present a new approach which is consistent with previous work but universally applicable irrespective of the algorithm used during data post-processing. We call this method the contrast grid. Finally, we analyze the error budget caused by non-Gaussian noise w.r.t. the detection limits in Section \ref{sec:results_and_discussion}.

The code of the metrics presented in this paper is public available as a \texttt{python} package called \texttt{applefy} on GitHub \url{https://github.com/markusbonse/applefy}. A detailed documentation page on how to use the package, including many tutorials, can be found at ReadtheDocs \url{https://applefy.readthedocs.io/}.