\section{Analytical Contrast Curves}
\label{sec:analytical_contast_curves}
%
\begin{figure}[b!]
	\epsscale{1.1}
	\plotone{11_Fake_planet_residuals.pdf}
	\caption{The top two panels show example residuals for the $\beta$-Pictoris dataset with and without fake planets. The bottom left image gives their difference. The bottom right image shows the result of the same experiment but for a fake planet closer to the star and with 50 PCA components. For 20 PCA components the presence of the planet only affects values of the residual in the direct neighborhood of the signal. For 50 PCA components the whole residual image changes. This is potentially problematic as discussed in the text.}
	\label{fig:fake_planets}
\end{figure}
%
\begin{figure}[t!]
	\epsscale{1.2}
	\plotone{12_Throughput.pdf}
	\caption{Throughput of the $\beta$ Pictoris dataset for 30 PCA components. Every value in the plot is the average of six experiments with artificial planets inserted at 6 different azimuthal positions. }
	\label{fig:throughputs_beta_pic}
\end{figure}

The calculation of one contrast grid presented in \mbox{Section \ref{sec:detection_limits}} requires to process several hundred datasets with inserted fake planets. This computation can be very time consuming. Under some mild assumptions, it is possible to reduce this computation time considerably. This is the commonly used approach of how contrast curves are calculated in packages like \texttt{PynPoint} \cite{stolkerPynPointModularPipeline2019} or \texttt{VIP} \cite{gonzalezVIPVortexImage2017}. We start by rearranging \mbox{Equation \ref{eq:SNR}}

\begin{equation}
	\label{eq:needed_flux_residual}
	Y_1 = T_{\text{obs}} \cdot \hat{\sigma}_{\mathcal{X}} \sqrt{1 + \frac{1}{n}} + \hat{\mu}_\mathcal{X} \, .
\end{equation}
The value of $Y_1$ is the brightness of the signal required in the residual to reach $T_{\text{obs}}$. Given a detection threshold specified as a FPF we can derive the $T_{\text{obs}}$ needed to be counted as a detection. For Gaussian noise this can be done by solving \mbox{Equation \ref{eq:FPF_ttest}} for $T_{\text{obs}}$ were $p(T = t| H_0)$ is given by the t-distribution. For non-Gaussian noise, larger values of $T$ might be required to reach the same FPF (see \mbox{Figure \ref{fig:requried_SNR}}). We re-use the sorted bootstrap results \mbox{$T_{(1)}^* \leq ... \leq T_{(B)}^*$}, discussed in step 6 of \mbox{Section \ref{sec:para_bs}}, and estimate $T_{\text{obs}}$ by linear interpolation:
	 
\begin{equation}
\label{eq:constrain_T}
	 	T_{\text{obs}} = T^*_{(\lfloor{a}\rfloor)} (a - \lfloor{a}\rfloor) + T^*_{(\lceil{a}\rceil)} (\lceil{a}\rceil - a)
\end{equation}
where $a = (B - 1) (1 - \text{FPF})$ gives the index of the two bootstrap results $T^*_{(\lfloor{a}\rfloor)}$ and $T^*_{(\lceil{a}\rceil)}$, which are closest to the required FPF. This step is the inverse of the linear interpolation explained in step 7 of \mbox{Section \ref{sec:para_bs}}. Due to planet over- and self-subtraction during the data post-processing, the flux of the planet in the residual is attenuated. We can describe this effect by

\begin{equation}
	\label{eq:residual_signal}
	Y_1 = f_p \cdot \kappa(f_p, s) + X_{n+1}
\end{equation}
were $X_{n+1}$ is the speckle noise at the position of the planet and $\kappa(f_p, s) \in [0, 1]$ is the throughput accounting for the attenuation of the data post-processing and potential coronagraphs. The separation is denoted as $s$. The throughput can be computed using the following procedure:
\begin{itemize}
	\item As for the contrast grid in \mbox{Section \ref{sec:detection_limits}}, insert artificial planets at different separations $s$ with different contrast $c = f_p / f_*$ into the raw data. Run the data post-processing to compute their residuals. In order to account for azimuthal variations we insert six planets, one at a time, for each separation and contrast.
	\item Compute one residual without fake planets. 
	\item Subtract the planet-free residual from every fake planet residual and estimate the flux at the position of the fake planet. Thanks to the linearity of PCA this gives us $Y_1 - X_{n+1}$. Since our statistic is based on pixel spaced by one FWHM, we integrate the flux within an area of one pixel around the position of the planet (compare \mbox{Section \ref{sec:how_to_detection}}). We note, that this step is only valid for sufficiently faint planets which do not affect the PCA component matrix. It further does not hold for all existing post-processing techniques.
	\item Use \mbox{Equation \ref{eq:residual_signal}} to compute the throughput.
\end{itemize}
Examples for residuals with and without fake planets are shown in \mbox{Figure \ref{fig:fake_planets}}. The throughput is summarized in \mbox{Figure \ref{fig:throughputs_beta_pic}}. It depends on the brightness of the inserted fake planet $f_p$ as well as its separations from the star. For faint planets, the PCA basis is not changed by the presence of the planet and the throughput converges. Bright planets, on the other hand, influence the PCA basis and cause additional signal loss. Planets which are close to the detection limit are usually faint. That is, under the assumption that these planets do not affect the PCA basis we can simplify $\kappa(f_p, s) = \kappa(s)$ and use the convergence throughput. That is, we only need to compute the last row of \mbox{Figure \ref{fig:throughputs_beta_pic}}. The limit for the planet to star contrast can be calculated by:

\begin{equation}
\label{eq:contrast}
	c = \frac{T_{\text{obs}} \cdot \hat{\sigma}_{\mathcal{X}} \sqrt{1 + 1/n} + \hat{\mu}_\mathcal{X} - X_{n+1}}{\kappa(s) \cdot f_*}
\end{equation}
The values of $X_{n+1}$, $\hat{\sigma}_{\mathcal{X}}$ and $\hat{\mu}_\mathcal{X}$ are based on noise observations from the planet free residual. They are again dependent on where the noise gets extracted. We propose the following procedure to account for this effect:
%
\begin{enumerate}
	\item Estimate $f_*$ on the unsaturated PSF by integration of the flux within an area of one pixel around the star.
	\item Choose a detection threshold, for example $5 \sigma_{\mathcal{N}} = 2.87 \times 10^{-7} \text{FPF}$.
	\item For each separation extract noise values  spaced by 1 FWHM from the planet free residual. Use one value as $X_{n+1}$ and the rest as $\mathcal{X}$.
	\item Make an assumption about the noise to constrain $T_{\text{obs}}$. For Gaussian noise solve \mbox{Equation \ref{eq:FPF_ttest}} for $T_{\text{obs}}$. For non-Gaussian noise use \mbox{Equation \ref{eq:constrain_T}}.
	\item Use \mbox{Equation \ref{eq:contrast}} to compute the contrast.
	\item Repeat the steps 3.-5. with different noise positions (compare \mbox{Figure \ref{fig:rotation}}). Report the median contrast over all repetitions. The mean absolute deviation from the median can be used as a measure for the uncertainty introduced by the placement of the noise positions.
\end{enumerate}
The given procedure assumes that the planet signal has no effect on the noise sample $\mathcal{X}$. As shown in the bottom right plot in \mbox{Figure \ref{fig:fake_planets}} this is not necessary the case for high number of PCA components. Under such circumstances, the calculation of a complete contrast grid is favorable. A comparison of the analytical contrast curve with the results of the contrast grid is shown in \mbox{Figure \ref{fig:contrast_curve_vs_map}}.
%
\begin{figure}[t!]
	\epsscale{1.1}
	\plotone{13_Contrast_grid_vs_curve.pdf}
	\caption{Comparison of the analytical contrast curve (based on the approximation that the throughput is only a function of separation) with the complete contrast grid presented in the main paper \mbox{(Section \ref{sec:detection_limits})}. The orange line gives the contrast grid thresholded at $5 \sigma_{\mathcal{N}}$. The results are based on the assumption of Gaussian noise.}
	\label{fig:contrast_curve_vs_map}
\end{figure}
As shown in the figure, both methods are consistent. In other words, the detection uncertainty of the artificial planets inserted for the contrast grid agrees with the contrast curve. At small separations, the contrast curve reaches a regime where the PCA basis is changed by the planet. For these separations no $5 \sigma_{\mathcal{N}}$ contrast exist, meaning that no planet, no matter how bright, will ever give a $5 \sigma_{\mathcal{N}}$ detection. This effect can only be identified with the contrast grid.
