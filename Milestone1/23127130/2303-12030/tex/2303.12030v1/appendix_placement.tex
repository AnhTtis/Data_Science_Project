\newpage
\section{Aperture placement}
\label{sec:rotation}
% Aspect 1: Implementation Details --------------------------------------------------------------

The first step of the hypothesis test illustrated in \mbox{Figure \ref{fig:hypothesis_testing}} is the extraction of the signal and the noise. The position of the signal can be estimated by maximizing the flux inside the signal aperture. The positions for the noise are commonly selected depending on the final signal position. This choice, however, is arbitrary and any other arrangement of non-overlapping apertures should yield similar results.
%
In order to check if $T_{\text{obs}}$ is invariant to the positioning of the apertures we insert fake planets into the \mbox{$\beta$ Pictoris} dataset and perform a PCA-based PSF subtraction. We insert one fake planet at a separation of \mbox{$2 \text{ FWHM}$} and one at \mbox{$3 \text{ FWHM}$}. 
%
We measure the flux of the inserted fake planets $\overline{Y_1}$ in each residual image and extract values for the noise $\overline{X_1}, ..., \overline{X_n}$ by using apertures with the same separation from the star. At \mbox{$2 \text{ FWHM}$} we have $n = 12$ and at \mbox{$3 \text{ FWHM}$} we have $n = 18$ apertures available. Apertures close to the planet signal are excluded in order to account for self-subtraction artefacts next to the planet. This reduces $n$ by about 2 apertures. Next, we compute $T_{\text{obs}}$ and the FPF as explained in \mbox{Section \ref{sec:what_is_a_detection}}. We repeat the calculation several times for slightly rotated aperture positions. In total we rotate all apertures by 60 degrees which corresponds to a displacement by $60/360 \cdot n = 2$ apertures at \mbox{$2 \text{ FWHM}$} and 3 apertures at \mbox{$3 \text{ FWHM}$}. The results are summarized in \mbox{Figure \ref{fig:rotation}}.
%

\begin{figure}[t!]
	\epsscale{1.15}
	\plotone{02_Rotation.pdf}
	\caption{The effect of the aperture placement on the detection uncertainty. The top two images show the residuals with the inserted fake planets at 2 and 3 $\text{FWHM}$. The white circles indicate the initial positions of the apertures used to extract the noise. Below we show the computed test statistic $T_{\text{obs}}$ (orange) and detection uncertainty / FPF (blue) for different rotations of the initial noise positions. The difference between the values of $T_{\text{obs}}$ and the FPF are due to the number of apertures changing by $\pm 1$ during the rotation.}
	\label{fig:rotation}
\end{figure}
%
As shown in the plot the computed FPF changes significantly depending on the rotation of the aperture positions $\Delta_{\theta}$. Especially at $2 \text{ FWHM}$ the values of $T_{\text{obs}}$ vary by a factor of 2. The corresponding confidence levels span from $3.4 \sigma_{\mathcal{N}}$ to $5.0 \sigma_{\mathcal{N}}$. At \mbox{$3 \text{ FWHM}$} the effect is smaller but still not negligible.
%
The position of the apertures influences whether speckles end up between or inside the apertures. If a bright speckle is located between two apertures, its flux will be split and $\hat{\sigma}_{\mathcal{X}}$ decreases. Vice versa, if the speckle is in the center of an aperture, $\hat{\sigma}_{\mathcal{X}}$ will be large. At small separations only a few apertures are available and a single speckle can already have a large effect on $T_{\text{obs}}$.

We recommend to estimate $T_{\text{obs}}$ several times for slightly rotated aperture positions. Instead of 60 degrees total rotation a rotation by $360 / (2 \pi r)$ degrees is sufficient for a displacement by one aperture (one oscillation; compare \mbox{Figure \ref{fig:rotation}}). The results can be summarized by reporting the median value of $T_{\text{obs}}$ over all rotations and the median absolute deviation (MAD) as a measure for the influence of the aperture placement. The minimum and maximum value of $T_{\text{obs}}$ give the worst- and best-case result. The same procedure should be used if spaced pixel are used instead of apertures (see discussion in \mbox{Section \ref{sec:independence}}.