\section{Results and Discussion}
\label{sec:results_and_discussion}

The parametric bootstrap presented in \mbox{Section \ref{sec:beyond_gaussian_noise}} quantifies planet detections for different types of residual noise. This way we can calculate the FPF of a potential planet candidate or, conversely, fix the FPF in order to determine detection limits. But, how accurate is the calculated FPF for non-Gaussian noise? Can we afford to make the wrong assumption? 

% Setup of the MC experiment
To answer these questions we run four Monte Carlo simulations considering Gaussian noise with ($\mu = 5$, $\sigma = 5$), Laplacian noise with ($\mu = 5$, $b = 5$) at 2 FWHM and 8 FWHM separation from the star. The separation in this simulation only influences the number of available noise values. Each Monte Carlo simulation is based on $10^9$ experiments for which we sample $n$ values from the respective noise distribution representing the planet signal under $H_0$ and the $n-1$ noise values available at 2 FWHM ($n = 11$) and 8 FWHM ($n=49$) separation. We run three tests to calculate the FPF for each experiment of the Monte Carlo simulation: The t-test, a parametric bootstrap under the assumption of Gaussian noise and a parametric bootstrap under the assumption of Laplacian noise. For the parametric bootstrap we use precomputed lookups as discussed in \mbox{Section \ref{sec:pivot}} with $B=10^8$. 

We set different detection thresholds and count the number of experiments for which the tests are \emph{fooled} by the noise, that is the calculated FPF is smaller than the chosen threshold. This way we obtain the true FPF of the tests for a given detection threshold. A comparison of the true FPF with the FPF promised by the tests is shown in  \mbox{Figure \ref{fig:mc_sim_para_bs}}.
% Figure of convergence distribution for both Laplacian and Gaussian noise
\begin{figure}[t!]
	\epsscale{1.2}
	\plotone{08_Monte_Carlo_Parametric_BS.pdf}
	\caption{Effect of different noise distributions on the detection uncertainty. The plots give the relationship between the selected detection threshold and the actually observed FPF for the t-test and the parametric bootstrap tests presented in this paper. Any deviation from the diagonal corresponds to an over- or underestimation of the FPF. The results are based on a Monte Carlo simulation described in the text.}
	\label{fig:mc_sim_para_bs}
\end{figure}
%
We observe that if the simulated noise is Laplacian the obtained confidence for the t-test is substantially lower than the desired confidence. For example at 2 FWHM, a chosen confidence of $5 \sigma_{\mathcal{N}} = 2.867 \times 10^{-7} \quad \text{FPF}$ results in $3.8 \sigma_{\mathcal{N}} = 7.235 \times 10^{-5} \quad \text{FPF}$. This means that we underestimate the number of false positives by a factor of 250. Unlike the t-test, the Laplacian parametric bootstrap computes the correct FPF. If we suspect that our residual noise distribution is better described by a Laplacian we should choose the parametric bootstrap over the t-test. 
%
If the noise is Gaussian the t-test is well calibrated and provides accurate FPF results. The Laplacian parametric bootstrap, however, overestimates the FPF and the results become too pessimistic. This means that accurate knowledge about the underlying noise distribution is critical to determine the FPFs. The parametric bootstrap under the assumption of Gaussian noise is consistent with the t-test. This result demonstrates that bootstrapping is able to account for non-Gaussian noise and at the same time for the small sample statistics. 

\paragraph{Required Signal-to-Noise Ratio}
% How much SNR is needed for a 5 sigma detection Plot. 
\begin{figure}[t!]
	\epsscale{1.1}
	\plotone{09_Required_SNR.pdf}
	\caption{Relationship between the value of the test statistic $T_{\text{obs}}$ and the detection uncertainty specified as \mbox{$5 \sigma_{\mathcal{N}} = 2.87 \times 10^{-7} \quad \text{FPF}$} and \mbox{$3 \sigma_{\mathcal{N}} = 1.35 \times 10^{-3} \quad \text{FPF}$}. The t-distribution shown in blue converges towards a Gaussian for large separations (black dashed line) while the Laplacian parametric bootstrap remains heavier-tailed.}
	\label{fig:requried_SNR}
\end{figure}
For many applications, such as the computation of contrast curves, we are not interested in the FPF of a potential planet. Instead, we want to fix the FPF to constrain which value of the test statistic $T_{\text{obs}}$ (signal-to-noise ratio) is needed to achieve the desired confidence (FPF). For Gaussian noise this can be done by solving \mbox{Equation \ref{eq:FPF_ttest}} for $T_{\text{obs}}$. For Laplacian noise we use the procedure explained in \mbox{Appendix \ref{sec:analytical_contast_curves}}. The results for different separations are summarized in \mbox{Figure \ref{fig:requried_SNR}}.
%
% Final discussion of the result
As shown in the plot a significantly larger value of $T_{\text{obs}}$ is required under Laplacian noise. This is because the Laplacian distribution has heavier-tails and the occurrence rate of large noise values is higher.
If we aim for a detection confidence of $5 \sigma_{\mathcal{N}}$ the signal of the planet needs to be more than two times brighter compared to the limits of Gaussian noise. The effect is important irrespective of the separation from the star. This means that even if we have a large sample size, we are not robust to non-Gaussian noise. This is due to the noise at the position of the planet and the fact that the sample of the signal contains only one observation (see discussion in \mbox{Section \ref{sec:what_is_a_detection}}). For a detection threshold of $3 \sigma_{\mathcal{N}}$ the difference between Gaussian and Laplacian noise is less important but still not negligible. At large separations the t-test converges to the classical $5 \sigma$ or $3 \sigma$ limits. Both tests account for the effect of small sample statistics at separations close to the star.

\paragraph{Detection Limits}
% Goal: Compare the effect of non-Gaussian noise on detection limits
In order to investigate the effect of non-Gaussian noise on the detection limits, we compute contrast curves under the assumption of Gaussian and Laplacian noise. 
%
% Approach: Compute the exemplary for the Beta Pic dataset
For this purpose, we compare the results of the classical t-test with those of the Laplacian parametric bootstrap test. We follow the procedure discussed in \mbox{Appendix \ref{sec:analytical_contast_curves}} to calculate our contrast curves. We overcome the limitations discussed in \mbox{Section \ref{sec:critical}} i.e. we use spaced pixel instead of apertures to guarantee independent noise observations and take the median contrast over several different noise positions.
%
% Results are shown in the Figure
The results for the $\beta$-Pictoris dataset are shown in \mbox{Figure \ref{fig:contrast_curves}}.
\begin{figure}[t]
	\epsscale{1.1}
	\plotone{10_Contrast_curves.pdf}
	\caption{Comparison of the detection limits for the $\beta$-Pictoris datasets, once under the assumption of Gaussian (blue line) and once for Laplacian (orange line) residual noise. The solid line is the median contrast over 360 different placements of the noise values in the residual. The shaded area gives the mean absolute deviation from the median. We used 30 PCA components for all separations.}
	\label{fig:contrast_curves}
\end{figure}
%
% The key results: We are off by one magnitude
For separations $>2$ FWHM the contrast of the t-test is about one magnitude deeper compared to the contrast of the parametric bootstrap. This difference corresponds to a factor 2.5 in planetary brightness. At separations close to the star (1-2 FWHM) the difference between the two limits is smaller. At these separations the effect of small sample statistics becomes relevant. The t-test accounts for this effect through the heavier tails of the t-distribution. These heavier tails also partially mitigate the problem caused by non-Gaussian noise.
%
% Implications for the field in general
If the true noise is heavy-tailed, the contrast curve of the t-test is too optimistic. In such situations direct imaging surveys might have ruled out regions of the parameter space where we might still find planets. Note, non-Gaussian noise has a systematic effect on the results. That is, the error does not average out if several datasets are combined within a survey. If the datasets within one survey are all affected by heavy tailed speckle noise the whole survey is biased. 
%
% We do not know the true distribution of the noise
In reality, the true distribution of the noise is influenced by many factors and therefore often unknown. Hence, we cannot decide which of the two detection limits is actually correct.
%
Based on the related work discussed in Section \ref{sec:non_gaussian_noise} we would expect the noise in the speckle dominated regime, i.e. close to the star, to be better described by a Laplacian. At these separations, the true contrast is likely closer to the contrast curve of the parametric bootstrap under the assumption of Laplacian noise. At larger separations, the noise becomes more Gaussian and the results of the t-test or Gaussian parametric bootstrap are likely accurate.
%
It is important to note that we can not prove if the noise is sufficiently normal. A discussion about this problem is given in \mbox{Appendix \ref{sec:testing_gaussian}}.
%
Thus, as long as no additional knowledge about the noise is available, we have to accept that our contrast curves are potentially inaccurate by about one magnitude.

