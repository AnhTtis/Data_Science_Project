\section{Speckle Noise Correlation}
\label{sec:Correlation}

A key assumption of the hypothesis tests discussed in this paper is that the observations in $\mathcal{Y}$ and $\mathcal{X}$ are independent. This assumption is fundamental for both the t-test as well as the parametric bootstrap. In \mbox{Section \ref{sec:critical}}, we motivate that pixel values spaced by 1 FWHM are approximately independent. This, however, is only the case if the actual correlations in the data follow the shape of the unsaturated PSF. In the following, we compute the spatial correlations of the noise in the $\beta$ Pictoris dataset processed with PCA \citep{amaraPYNPOINTImageProcessing2012,soummerDETECTIONCHARACTERIZATIONEXOPLANETS2012} and cADI \citep{maroisAngularDifferentialImaging2006}:
\begin{enumerate}
	\item We start the analysis with the pre-processed stack of images in which the star is located in the center of the frames. For cADI we subtract the median frame from every image in the stack. For PCA we subtract the first 30 PCA components. We de-rotate the frames according to their parallactic angles. The result is the sequence of residual images, which we average to obtain the final residual image.
	\item We compute the correlation of every pixel along the temporal domain w.r.t. all other pixel. For this we use the sequence of de-rotated residuals after PSF-subtraction. This way we obtain a single two-dimensional correlation map for each pixel.
	\item We crop the correlation maps such that the pixel they correspond to is in the center of the cropped map. The resulting maps show the local correlations for each pixel in the residual frame. 
	\item We average the cropped correlation maps for all pixel within an annulus at \mbox{2 FWHM} and \mbox{6 FWHM} distance from the star.
\end{enumerate}
The results and a comparison with the unsaturated PSF of the star is shown in \mbox{Figure \ref{fig:noise_correlation}}.
%
\begin{figure}[b!]
	\epsscale{1.13}
	\plotone{15_Spatial_Correlation_of_Speckles.pdf}
	\caption{Estimates of the spatial noise correlations of the $\beta$-Pictoris dataset. The image on the top left is an example of a correlation map. It shows how much the pixel marked with the white cross is correlated w.r.t. all other pixel in the image. We compute one correlation map for every pixel in the residual. The top right images show the local averages of the correlation maps at \mbox{2 FWHM} and \mbox{6 FWHM} for cADI and PCA. The regions used to compute the averages are marked in the correlation map. The bottom two plots give the one-dimensional profile of the local correlation maps (top right plots) in comparison to the unsaturated PSF of the star.}
	\label{fig:noise_correlation}
\end{figure}
%
We observe that the true correlations in the data do not always follow the shape of the PSF. Further, they are influenced by the separation from the star as well as the data post-processing used. For cADI the FWHM of the local correlation map at \mbox{2 FWHM} is about 60 \% larger compared to the PSF-FWHM. Similar results were previously observed by \cite{golombPlanetEvidencePlanetNoise2021} for data from the Gemini Planet Imager. If the spatial correlation length is large, a spacing of 1 PSF-FWHM might not be sufficient to ensure that the noise observations in $\mathcal{X}$ are independent. 
%
At \mbox{6 FWHM} the FWHM of the local correlations is slightly smaller than the PSF-FWHM. If we use PCA, the local correlations at \mbox{2 FWHM} change while they stay the same at \mbox{6 FWHM}. Future work should further investigate this behavior to better understand the spatial correlations of speckle noise in real data. A possible path for this could be bootstrapping for dependent observations \citep{zoubirBootstrapTechniquesSignal2004}.