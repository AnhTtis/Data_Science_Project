\appendix

\setcounter{page}{1}

\renewcommand{\thefigure}{S\arabic{figure}}
\renewcommand{\thetable}{S\arabic{table}}
\setcounter{figure}{0}  
\setcounter{table}{0}  

\twocolumn[
\centering
\Large
\textbf{\titlemethod} \\
\vspace{0.5em}Supplementary Material \\
\vspace{1.0em}
] %< twocolumn





\section{Datasets}

\subsection{1D Dataset}

Fig.~\ref{fig:1d-dataset} shows an example of a ground truth 1D function $f^*$, its self-similarity maps and its regions of attraction. We show visual results for the ablation study of Fig.~5. Details on the generation process are given in the main paper (4.1). A jupyter notebook for generating these datasets and running our experiments will be provided in an open source repository.

\begin{figure}[h]
    \centering
    \includegraphics[width=\linewidth]{fig/1d_dataset.pdf}
    \caption{(Top) Example of ground truth function $f^*$ and visual results for the ablation study (Fig.~5).
    (Bottom) Self-similarity maps and regions of attraction. The self-similarity map shows local minima on the line $\theta^*=\theta+\pi$. With the equivalence relation $\relation=\relation_2$, the regions of attraction cover the entire quotient set $\Omega/\relation$.
    }
    \label{fig:1d-dataset}
\end{figure}

\subsection{3D Datasets}

Table~\ref{tab:datasets} summarizes the properties of the datasets used for 3D inverse rendering. Fig.~\ref{fig:sph-dataset} shows sample views of the RGB-MELON datasets. These datasets and the script for generating them in Blender \cite{blender} will be provided upon publication.



\bgroup
\setlength{\tabcolsep}{1.3mm}
\begin{tabular}{lrrrrrcl}
    \toprule
    \cthead{Dataset}                                           & \cthead{\( n \)} & \cthead{\( v \)} & \cthead{\( k \)} & \cthead{\( n_\text{small} \)} & \cthead{\( n_\text{big} \)} & \cthead{Dim.\ }                    & \cthead{Licence} \\ \cmidrule(lr){1-1} \cmidrule(lr){2-8}
    NoisyMNIST~\cite{lecunGradientbasedLearningApplied1998}   & \( 70000 \)     & \( 2 \)         & \( 10 \)        & \( 6313 \)                   & \( 7877 \)                 & \( (28 \times 28)^{2} \)          & CC BY-SA 3.0 \\
    NoisyFashion~\cite{xiaoFashionMNISTNovelImage2017}        & \( 70000 \)     & \( 2 \)         & \( 10 \)        & \( 7000 \)                   & \( 7000 \)                 & \( (28 \times 28)^{2} \)          & MIT \\
    EdgeMNIST~\cite{lecunGradientbasedLearningApplied1998}    & \( 70000 \)     & \( 2 \)         & \( 10 \)        & \( 6313 \)                   & \( 7877 \)                 & \( (28 \times 28)^{2} \)          & CC BY-SA 3.0 \\
    EdgeFashion~\cite{xiaoFashionMNISTNovelImage2017}         & \( 70000 \)     & \( 2 \)         & \( 10 \)        & \( 7000 \)                   & \( 7000 \)                 & \( (28 \times 28)^{2} \)          & MIT \\
    COIL-20~\cite{neneColumbiaObjectImage1996}               & \( 480 \)       & \( 3 \)         & \( 20 \)        & \( 24 \)                     & \( 24 \)                   & \( (64 \times 64)^{3} \)          & None \\
    Caltech7~\cite{fei-feiLearningGenerativeVisual2007}       & \( 1474 \)      & \( 6 \)         & \( 7 \)         & \( 34 \)                     & \( 798 \)                  & \( 48, 40, 254, 1984, 512, 928 \) & CC BY 4.0 \\
    Caltech20~\cite{fei-feiLearningGenerativeVisual2007}      & \( 2386 \)      & \( 6 \)         & \( 20 \)        & \( 33 \)                     & \( 798 \)                  & \( 48, 40, 254, 1984, 512, 928 \) & CC BY 4.0 \\
    PatchedMNIST~\cite{lecunGradientbasedLearningApplied1998} & \( 21770 \)     & \( 12 \)        & \( 3 \)         & \( 6903 \)                   & \( 7877 \)                 & \( (28 \times 28)^{12} \)         & CC BY-SA 3.0 \\
    \bottomrule
\end{tabular}

\egroup



\begin{figure*}[t]
  \centering
  \includegraphics[width=\linewidth]{fig/sph_dataset.pdf}
   \caption{Sample views from the RGB-MELON datasets. The red--green texture is symmetric along the azimuthal direction. In order to make the pose estimation problem well-posed, the symmetry is broken by three red/green/blue squares. All the views are taken from the equator plane. The self-similarity maps and quotient self-similarity maps are plotted on the right for various replication orders $\reporder$.
   }
   \label{fig:sph-dataset}
\end{figure*}

\section{Baselines}

\subsection{COLMAP}

We ran COLMAP with the default configuration for performance evaluations. To ensure fair comparison, we swept parameters, running with different patch window radii (\verb|window_radius|) of 5, 3, 7, 9, 11, and different filtering thresholds of photometric consistency cost (\verb|filter_min_ncc|) of 0.1, 0.01, 0.2, 0.5, and 1.0. We tested all combinations of each.

COLMAP fails when it can not find enough corresponding points between frames. When it finds enough points, it does not always solve all the poses. We report the mean angular error over solved poses in the main text. For fair comparison to MELON and GNeRF, which always predict pose, we also compute a mean over all poses where the error of an unsolved pose is counted as a random guess (\textit{i.e.} $73^\circ$ when the elevation range is $[0^\circ, 90^\circ]$ and $90^\circ$ when it is $[-90^\circ, 90^\circ]$). Results are reported in Table~\ref{tab:colmap-means}.

In general, we note that the high mean angular error of COLMAP is due to large errors on few images. The distribution of angular errors (Fig.~\ref{fig:histo-colmap-errors}) shows a large discrepancy between the median and the mean.

\begin{figure}[!h]
  \centering
  \includegraphics[width=0.6\linewidth]{fig/histo_colmap_errors.pdf}
   \caption{
   Histogram of angular errors obtained with COLMAP on the Lego dataset. 100 images rendered at $128\times128$.
   }
   \label{fig:histo-colmap-errors}
\end{figure}

\begin{table}[h]
  \centering
  \resizebox{\linewidth}{!}{
  \begin{tabular}{l|cccc}
  \toprule
  
   & Mean over& Number of  & Elevation & Mean over \\
  Scene & Solved Poses & Solved Poses & Range &  All Poses \\
  
  \midrule
  %[0.229, 0.328, 9.712, 0.467, 0.860, 0.292, 1.053]
  %[100, 91, 100, 21, 15, 10, 17]
%   [0.229,
%  6.868479999999997,
%  9.712,
%  57.76807,
%  62.178999999999995,
%  65.7292,
%  74.87901000000001]
  Lego & 0.229 & 100/100 & $[0^\circ,90^\circ]$ & 0.229 \\
  Hotdog & 0.328 & 91/100 & $[0^\circ,90^\circ]$ & 6.87 \\
  Chair & 9.71 & 100/100 & $[0^\circ,90^\circ]$ & 9.71 \\
  Drums & 0.467 & 21/100 & $[0^\circ,90^\circ]$ & 57.8 \\
  Mic & 0.860 & 15/100 & $[0^\circ,90^\circ]$ & 62.2 \\
  Ship & 0.292 & 10/100 & $[0^\circ,90^\circ]$ & 65.7 \\
  Materials & 1.05 & 17/100 & $[-90^\circ,90^\circ]$ & 74.9 \\
  Ficus & fails & --- & $[-90^\circ,90^\circ]$ & ---  \\
  
  \bottomrule 
  \end{tabular}
  } % tabular
  
  \caption{Mean angular error of COLMAP poses on the ``NeRF-Synthetic'' scenes.}
  \label{tab:colmap-means}
\end{table}

\subsection{GNeRF}

We run the public version of GNeRF, available at \url{https://github.com/quan-meng/gnerf} with the default parameters. When optimizing the poses in $\text{SO}(3)$, we use $6$-dimensional embeddings to parameterize the camera-to-world rotation matrices $R_i$ in the training and test sets. Based on the value of $R_i$ we set the camera location $t_i$ to
\begin{equation}
    t_i = -R_i\cdot T_i,
\label{eqn:translation-camera-to-world}
\end{equation}
where $T_i$ is the published 3D vector representing the location of the origin in the reference frame of camera $i$. The 3x4 matrix ($R_i$, $t_i$) represents an element of $\text{SE}(3)$.

\subsection{VMRF}
\begin{table*}[t]
  \centering
  \resizebox{\linewidth}{!}{
  \begin{tabular}{lc|ccc|cccc|cccc|cccc}
  \toprule
  & & \multicolumn{3}{c|}{Pose Estimation}
  & \multicolumn{12}{c}{Novel View Synthesis} \\
  
  & & \multicolumn{3}{c|}{Rotation ($^\circ$)~$\downarrow$} 
  & \multicolumn{4}{c|}{PSNR~$\uparrow$} 
  & \multicolumn{4}{c|}{SSIM~$\uparrow$} 
  & \multicolumn{4}{c}{LPIPS~$\downarrow$} \\
  
  \cmidrule(r){3-5} \cmidrule(r){6-9} \cmidrule(r){10-13} \cmidrule(r){14-17}
  
  Scene  & $\reporder$
  & GNeRF & VMRF  & \textbf{\methodname} 
  & GNeRF & VMRF & \textbf{\methodname} & NeRF
  & GNeRF & VMRF & \textbf{\methodname} & NeRF
  & GNeRF & VMRF & \textbf{\methodname} & NeRF \\
  
  \midrule
  
  % Best of 4 runs from 40b, 40c, 40d, 40g
  % /cns/oe-d/home/synthx/users/mjmatthews/experiments/ab_initio_vmrf_comparison_40c
  Lego & 2
  & 3.315 & 1.394  & \textbf{0.059}
  & 22.95 & 25.23 & \textbf{30.22} & 30.44 
  & 0.8493 & 0.8865 & \textbf{0.9525} & 0.9526
  & 0.1630 & 0.1215 & \textbf{0.0627} & 0.0591 \\

  % /cns/oe-d/home/synthx/users/mjmatthews/experiments/ab_initio_vmrf_comparison_40d
  Hotdog & 2
  & --- & ---  & \textbf{3.703}
  & --- & --- & \textbf{27.52} & 35.97 
  & --- & --- & \textbf{0.9428} & 0.9782 
  & --- & --- & \textbf{0.0951} & 0.0384  \\
  
%   % /cns/oe-d/home/synthx/users/mjmatthews/experiments/ab_initio_vmrf_comparison_40b
%   Chair & 2
%   & 6.078 & \textbf{4.853}  & 36.11
%   & 25.01 & \textbf{26.05} & 12.57 & 32.45
%   & 0.8940 & \textbf{0.9083} & 0.6719 & 0.9673
%   & 0.1526 & \textbf{0.1397} & 0.4582 & 0.0558 \\
  
  % /cns/oe-d/home/synthx/users/mjmatthews/experiments/ab_initio_comparison_98b_chair_rep=4
  Chair & 4
  & 6.078 & 4.853  & \textbf{2.215}
  & 25.01 & 26.05 & \textbf{30.96} & 32.45
  & 0.8940 & 0.9083 & \textbf{0.9598} & 0.9673
  & 0.1526 & 0.1397 & \textbf{0.0587} & 0.0558 \\
  
  
  % /cns/oe-d/home/synthx/users/mjmatthews/experiments/ab_initio_vmrf_comparison_40d
  Drums    & 2
  & 2.769 & 1.284  & \textbf{0.053}
  & 20.63 & 23.07 & \textbf{24.55} & 24.43
  & 0.8628 & 0.8917 & \textbf{0.9121} & 0.9120
  & 0.2019 & 0.1605 & \textbf{0.1096} & 0.1095 \\
  
  % /cns/oe-d/home/synthx/users/mjmatthews/experiments/ab_initio_vmrf_comparison_40g
  Mic & 2
  & 3.021 & 1.394  & \textbf{0.045}
  & 23.68 & 27.63 & \textbf{32.40} & 31.83 
  & 0.9332 & 0.9483 & \textbf{0.9727} & 0.9705
  & 0.1095 & 0.0803 & \textbf{0.0413} & 0.0465 \\
  
  % /cns/oe-d/home/synthx/users/mjmatthews/experiments/ab_initio_vmrf_comparison_40g
  Ship  & 2
  & 31.56 & 16.89  & \textbf{0.210}
  & 17.91 & 21.39 & \textbf{28.27} & 27.71
  & 0.7626 & 0.7998 & \textbf{0.8607} & 0.8553 
  &  0.3628 & 0.2933 & \textbf{0.1704} & 0.1719  \\

%   \midrule
  
%   Mean  & 
%   & 0.000 & 0.000 & 00.000 
%   & 00.00 & 00.00 & 00.00 
%   & 0.000 & 0.0000 & 0.0000 
%   & 0.000 & 0.0000 & 0.0000 \\
  
  \bottomrule 
  \end{tabular}
  } % tabular
  
  \caption{Comparison of \methodname to GNeRF and VMRF (results from~\cite{Zhang2022vmrf}). 
  Experiments run with an elevation range of $[-30^\circ,90^\circ]$ as done in VMRF.
  We report the best of five runs.
  $\reporder$ is the replication order used with \methodname.
  }
  \label{tab:ab_initio_vmrf_comparison}
\vspace{-3mm}
\end{table*}

% GT Nerf from: /cns/oe-d/home/synthx/users/mjmatthews/experiments/gt_nerf_60a_viewdirs

We compare \methodname to VMRF's reported results~\cite{Zhang2022vmrf} by running with the same configuration as in our GNeRF comparison, but with an an expanded elevation range of $[-30^\circ,90^\circ]$, the smallest range reported by VMRF. Results are show in Table \ref{tab:ab_initio_vmrf_comparison}.

\subsection{SAMURAI}

We run the public implementation of SAMURAI, available at \url{https://github.com/google/samurai}. In the fixed initialization scheme, we set all the initial directions to [Center, Above, Center] (North pole). When optimizing the poses in $\text{SO}(3)$, we overwrite the predicted camera locations using~\eqref{eqn:translation-camera-to-world}.

With a manual initialization, each view direction is specified with a triplet [Left/Center/Right, Above/Center/Below, Front/Center/Back]. [Center, Center, Center] is disallowed, giving a total of 26 possible initial directions.

\section{Architecture}

\subsection{Encoder}

Fig.~\ref{fig:encoder} summarizes the architecture of the encoder mapping 2D images to poses. For the 1D experiments, the encoder contains five 1D convolution layers of size $5$ interlaced with SiLU activation functions~\cite{hendrycks2016gaussian}, max-pooling layers and group normalization layers~\cite{wu2018group}.

\section{Decoupled Regression as an Universal Encoding}
\label{sec:UE}

In this section, we propose a universal encoding model for point clouds.
Building on this encoder, in~\cref{sec:UCM}, we introduce Universal Continuous Mapping.

\subsection{Decoupled Gaussian Process Regression as Encoder-Decoder}% one page
\label{sec:encoder}

Inspired by DI-Fusion~\cite{huang2021di}, which introduces latent feature maps for continuous surface prediction, our goal is to generate a latent vector for each local patch.
%
% introduce the gaussian process
First, a universal encoder design comes from the Gaussian Process Regression (GPR), which is a widely used technique for low-dimensional regression.
It has been used in various mapping models~\cite{yuan2018fast,martens2016geometric}.
%
Given a set of $N$ observation points $\{(\V x_n, \V y_n)\}_{n=1}^{N}$ where point positions are $\V x_n \in \mathbb R^d$ ($d=3$ in this paper) and point properties are $\V y_n \in \mathbb R^c$, GPR is used to regress a function $f$ that best explains the data.
The $N$ points are aggregated in the $N\times d$ matrix $\V X$, and the targets are collected in the $N\times c$ matrix $\V Y$.
%The regression model estimation is $f(\V x)=E[\V y | \V X=\V x]$.
%based on the observation set. %Which is $\V y = f(\V x) + \espilon$ where $\epsilon$ is Gaussian noise. 

The Gaussian process assumes that the data is sampled from a multivariate Gaussian distribution, i.e.,
\begin{equation}
\begin{bmatrix}
\V Y\\
\V Y_{*}
\end{bmatrix}
	 \sim \mathcal{N}(\V 0, 
\begin{bmatrix}
k(\V X, \V X)\ k(\V X, \V X_{*})\\
k(\V X_{*}, \V X)\ k(\V X_{*}, \V X_{*})
\end{bmatrix}	 
	 ).
\end{equation}
where $(\V X, \V Y)$ and $(\V X_*, \V Y_*)$ represent the observation and inference pairs. 
For simplicity, we denote the matrix $\V K=k(\V X, \V X)$, $\V K_{*}=k(\V X, \V X_*)$, $\V K_{**}=k(\V X_*, \V X_*)$.

%
With the derivative in~\cite{williams2006gaussian}, we obtain 
\begin{equation}
	\V Y_{*} | \V X, \V Y, \V X_{*} \sim \mathcal{N} (\V K_{*}^T\V K^{-1}\V Y, \V K_{**}-\V K_{*}^T\V K^{-1}\V K_{*}).
\end{equation}
%
When an additional Gaussian error is introduced as $\V y=f(\V x)+\epsilon$, the covariance of $\V X$ is rewritten as $\V K+\delta^2_n\V I$.
%
The regression result is typically considered to be the mean
\begin{equation}
	\label{eq:mean}
	\V Y_{*} = \V K_{*}^T(\V K+\delta^2_n\V I)^{-1}\V Y.
\end{equation}
This well illustrates the challenge of using Gaussian process regression directly in large-scale reconstructions due to its $\mathcal{O}(n^3)$ time complexity, which is not feasible.
% with approxmiation
Furthermore, the formula (\cref{eq:mean}) is impractical since it requires the large input point cloud data to be maintained for the $ \V K_{*}$ computation. 

To address the issue of high complexity and avoid the need to maintain the entire point cloud, we propose to decouple the GPR to obtain the low-dimensional latent vectors for local regions.
%
The decoupling process involves approximating the kernel function as
\begin{equation}
	k(\V X,\V X_{*} ) \approx f_{posi}(\V X) f_\text{posi}(\V X_{*})^T
	\label{eq:k_approx}
\end{equation}
where $f_\text{posi}: \mathbb{R}^3\rightarrow \mathbb{R}^l$.
We refer to the function $f_\text{posi}$ as the position encoding function, consistent with the Neural Implicit Maps model, Di-Fusion~\cite{huang2021di}.
%
Thus, \cref{eq:mean} is rewritten as
\begin{equation}
	\label{eq:approx_K_mean}
	\begin{split}
	\V y_{*} %&= f_\text{posi}(\V X_*)^Tf_{posi}(\V X)(f_\text{posi}(\V X)^Tf_\text{posi}(\V X) +\delta^2_n\V I)^{-1}\V y\\
	%&
	=f_\text{posi}(\V X_*)f_\text{enc}(\V X, \V Y)
	\end{split}
\end{equation}
%With $X\in \mathbb{R}^{N\times 3}$, $y\in\mathbb{R}^{N\times c}$, we
where the content encoder function is
\begin{equation}
	\label{eq:encode}
	f_\text{enc}(\V X, \V Y) = f_\text{posi}(\V X)^T(f_\text{posi}(\V X)f_\text{posi}(\V X)^T +\delta^2_n\V I)^{-1}\V Y \in \mathbb{R}^{l \times c}.
\end{equation}

The encoded feature is denoted by $ \V F_{(\V X, \V Y)} = f_{enc}(\V X, \V Y) \in \mathbb{R}^{l\times c}$.
which serves as the basis for the construction of latent maps in the subsequent universal continuous mapping model (\cref{sec:UCM}).
Specifically, for geometry encoding we set $l=20$ and $c=1$, resulting in a $20$-dimensional vector feature.
Similarly, for color encoding, we have $l=20$ and $c=3$.

The decoding value for the inferred point $\V x_*$ is expressed as
\begin{equation}
	\label{eq:decode}
	f_{dec}(\V x_{*},\V F_{(\V X, \V Y)}) = f_{posi}(\V x_{*}) \V F_{(\V X, \V Y)}.
\end{equation}
Thus, a signed distance field or surface property field is approached.

In the following we derive the approximation function.

\begin{figure}[]
	\centering
    \psfragfig[width=1\linewidth]{im/eps/encoder1}{
	\psfrag{y1}{$\V y_1$}
	\psfrag{x1}{$\V x_1$}
	\psfrag{yn}{$\V y_n$}
	\psfrag{xn}{$\V x_n$}
	\psfrag{z1}{$\V z_1$}
	\psfrag{zn}{$\V z_n$}
	\psfrag{fp}{$f_{posi}$}
	\psfrag{fd}{$f_{dec}$}
	\psfrag{fe}{\color{mybronze}{$f_{enc}$}}
	\psfrag{ec}{\color{mybronze}{{Encoder}}}
	\psfrag{zm}{$\V Z_m$}
	\psfrag{ym}{$\V Y_m$}
	\psfrag{Fm}{$\V F_m$}
	\psfrag{x}{$\V x_{*}$}
	\psfrag{y}{$\V y_{*}$}
	\psfrag{f}{\tiny $\V Z_m^T(\V Z_m\V Z_m^T+\sigma_n\V I)^{-1}\V Y_m$}
}
	
	%\includegraphics[width=1\linewidth]{im/encoder}
	\caption{Interpreting formula with a graph that is coherent to the Encoder-decoder structure in Neural Implicit Maps~\cite{huang2021di}. }
	\label{fig:encoder}
	\vspace{-.3cm}
\end{figure}



\subsection{Position Encoding with Approximated Kernel Function} % one page
\label{sec:posi_encode}

Considering that our mapping needs to encode the local geometry \& property, the encoding function only requires to touch points in a limited region ($[-.5,.5]^3$ in our case).

As we discussed in related work, Nytr\"om methods offer greater accuracy than RFFs, because they depend on the given points. This property is well-suited to our application.

% introduce the Nytrom methods
The Nystr\"om method for kernel approximation begins with the use of eigenfunctions according to Mercer's theorem:
\begin{equation}
	 k(\V x_1, \V x_2) = \sum_{i\ge 1} \mu_i \psi_i(\V x_1)^T\psi_i(\V x_2)
	\label{eq:eigen}
\end{equation}
where $\psi_i$ and $\mu_i\ge 0$ are eigenfunctions and eigenvalues of kernel function $k$ with respect to the probability measure $q$.% With the top-$k$ eigenpairs, Nystr\"om method approximate kernel with $ k(\V x, \V y) = \sum^k_{i\ge 1} \alpha_i \varphi_i(\V x)\varphi_i(\V y)$.

Given a set of anchor samples $\hat{\V X}=\{\hat{\V x}_1,\cdots,\hat{\V x}_N\}$, 
% how we use it
we perform eigen-decomposition on the matrix $k(\hat{\V X}, \hat{\V X})$ to obtain its eigenpairs $\{(\lambda_i,\V u_i)\}_{i\in\{1,\cdots,l\}}$ with rank $l$.

Subsequently, Nystr\"om method produces
\begin{equation}
\label{eq:nytrom_1}
\psi_i(\V x) = \sum_{n}k(\V x, \hat{\V x}_n)\V u_{i,n}, i= 1,\cdots,l.
\end{equation}

% can be obtained with eigendecomposition of matrix $K(\hat{\V X}, \hat{\V X})$. 

%To approach
%$k(\V x, \V y) = \sum^{k}_{i= 1} \hat{\mu_i} \hat{\varphi_i(x)}\hat{\varphi_i(y)}$ 

%The \cref{eq:nytrom_1} is then rewriten as 
%\begin{equation}
%	\label{eq:nytrom_2}
%	\hat{\varphi}_i(\V x) = \frac{1}{N\hat{\mu_i}}\sum_{n=1}{N}k(\V x, \V x_n)\hat{\varphi}(\V x_n), i\in{1,\cdots,k}.
%\end{equation}


To simplify, we express the eigenfunction as 
\begin{equation}
\psi_i(\V x) = k(\V x, \hat{\V X})\V u_i .
\label{eq:nytrom_2}
\end{equation}
Similarly, the eigenvalue is written as $\mu_i=\frac{1}{\lambda_i}$.

For clarity, we introduce the notation $ \pmb \mu=[{\mu_1},\cdots,{\mu_l}]$ and
${\Psi}=[\psi_1,\cdots,{\psi_l}]^T$
 based on \cref{eq:eigen} where
${\Psi}:\mathbb{R}^3\rightarrow \mathbb{R}^l$.

To maintain consistency with \cref{eq:k_approx}, we set 
\begin{equation}
	f_{posi}(\V x) = diag(\sqrt {\pmb\mu})\Psi(\V x) 
	\label{eq:f_posi}
\end{equation}
where $diag$ produces diagonal matrix.
$f_{posi}$ above refers to the position encoder in \cref{eq:encode}.

In this paper, we employ the Mat\'ern kernel function~\cite{genton2001classes}\footnote{https://en.wikipedia.org/wiki/Mat\'ern\_covariance\_function}:
\begin{equation}
\label{eq:matern}
k(\V x_1, \V x_2) = \sigma^2 \frac{2^{1-\nu}}{\Gamma(\nu)}(\sqrt{2\nu}\frac{dist(\V x_1, \V x_2)}{\rho})K_{\nu}(\sqrt{2\nu}\frac{dist(\V x_1, \V x_2)}{\rho})
\end{equation}
where $\Gamma$ presents the gamma function, $K_{\nu}$ is the modified Bessel function of the second kind, $dist$ denotes the Euclidean distance, and $\sigma$ and $\rho$ are hyperparameters of the kernel function.
We utilize the half integer $\nu=3+\frac{1}{2}$, which results in the specific function:
\begin{equation}
\label{eq:matern_specific}
k(d) = \sigma^2(1+\frac{\sqrt{7}d}{\rho}+\frac{2}{5}(\frac{\sqrt{7}d}{\rho})^2+\frac{1}{15}(\frac{\sqrt{7}d}{\rho})^3)exp(-\frac{\sqrt{7}d}{\rho})
\end{equation}
where $d=dist(\V x_1, \V x_2)$ for short.

To approximate the above kernel function, anchor points are required.
We sample $N_a=256$ points uniformly 
from $[-.5,.5]^3$ cube (as $\hat{\V X}$) to compute the kernel matrix $\V K_{a}=k(\hat{\V X},\hat{\V X})$.
Subsequently, we perform an eigendecomposition on this kernel matrix, resulting in $\V K_{a}=\V U \Lambda \V U^T$. % with $U$ a $256\times $.
Lastly, $\V U$ and $\Lambda$ are utilized in \cref{eq:f_posi} and~\cref{eq:k_approx} to form the kernel function approximation.

It is important to note that the dimension of the encoded feature, $l$, used in \cref{sec:encoder}, depends on $\V U$. 
It further determines the size of the map in the next section (~\cref{sec:UCM}).% creating a trade-off between the approximation and map size.
%Therefore, the $l$ cannot be too large.
\begin{figure}[htbp]
	\centering
	\includegraphics[width=1.\linewidth]{im/eigenvalue}
	\caption{Sorted eigenvalues for $\V K_{a}$'s eigendecomposition.}
	\label{fig:eigenvalues}
\end{figure}

The eigenvalues of $\Lambda$ are plotted in \cref{fig:eigenvalues}, revealing that the matrix is primarily influenced by a small number of pairs with significant eigenvalues.
Most of the eigenvalues are less than 1. 
Therefore, we choose $l=20$ which is about $0.8$ in this plot to approximate the kernel while maintaining a compact feature dimension.

Note that we perform a single sampling and decomposition step. Subsequently, the encoding-decoding process in \cref{fig:encoder} only requires loading and reusing the parameters for $f_{posi}$, $f_{enc}$ and $f_{dec}$. In contrast, related works often involve pre-training on large datasets of objects~\cite{huang2021di,li2022bnv} or indoor scenes~\cite{zhu2022nice}. 
For our model, however, \textbf{no training is needed}.

%For the convenience of use in the next section, this section also encapsulate functions into API as in~\cref{alg:encode-decode}.
%\begin{algorithm}[H]
%	\caption{Encoding and decoding}\label{alg:encode-decode}
%	\begin{algorithmic}
%		\STATE 
%		\STATE {\textsc{Encode}}$(\V X,\ \V y)$
%		\STATE \hspace{0.5cm}$\V Z \gets f_{posi}(\V X) $ \hspace{1in} \# \cref{eq:f_posi}
%		\STATE \hspace{0.5cm}$ \V F \gets \V Z(\V Z^T\V Z+\sigma_n\V I)^{-1}\V y$ \hspace{1cm} \# \cref{eq:encode}
%		\STATE \hspace{0.5cm}\textbf{return}  $\V F$
%		\STATE 
%		\STATE {\textsc{Decode}}$(\V {X_*},\ \V F )$
%		\STATE \hspace{0.5cm}$\V y_*=f_{posi}(\V X_*)^T\V F$ \hspace{2cm}\# \cref{eq:decode}
%		\STATE \hspace{0.5cm}\textbf{return}  $\V y_*$
%	\end{algorithmic}
%\end{algorithm}


\subsection{Encoder Ablations}
\label{sec:encoder-ablation}
We perform a series of ablations to explore encoder architecture. We run \methodname on all training views of ``lego'', ``drum'', ``ficus'' and ``ship'' at $400\times400$ resolution and $\reporder=2$. We observe in our primary experiments that when poses converge, they tend to do so very quickly, often in the first thousand steps. We therefore train each configuration for 10k steps, but repeat 10 times to find min, max and mean metrics. We compute metrics as in other experiments over all test views. We choose a primary configuration of $L=64$, $F=32$, $d=4$ varying one parameter at a time in our experiments.

\begin{figure}
    \centering
    \includegraphics[width=\linewidth]{fig/encoder_ablation.pdf}
    \caption{PSNR of novel view synthesis vs. encoder depth $d$, number of features $F$, and input resolution $L$, for four ``NeRF-Synthetic'' scenes. We report min, max and mean values over 10 runs.}
    \label{fig:encoder-ablation}
\end{figure}

Fig.~\ref{fig:encoder-ablation} shows the result of sweeping feature count $F$ from 8 to 256 and encoder depth $d$ from 2 to 6 layers. We observe optimal values at 4 layers, and a feature count of 32. 

% \input{figref/encoder_resolution_sweep}
Fig.~\ref{fig:encoder-ablation} also shows performance sweeping encoder input resolutions $L$ from $16\times16$ to $256\times256$. We observe that certain datasets appear to work better at different resolutions, with optimal values typically ranging from $32\times32$ to $128\times128$. We therefore choose per-scene values in our primary experiments. 


\subsection{Replication Order}
\begin{figure}[ht]
    \centering
    \includegraphics[width=\linewidth]{fig/replication_order.png}
    \caption{Scatter plot of the mean angular error vs. replication order $\reporder$ for various ``NeRF-Synthetic'' scenes. We run each scene 10 times to 10k steps.}
    \label{fig:replication-order}
\end{figure}
To determine the effect of varying the replication order, we run with the same base configuration as Sec.~\ref{sec:encoder-ablation}, but varying replication order over $\reporder = \{1, 2, 4\}$. We run \methodname 10 times to 10k steps on five scenes and plot the results in Fig.~\ref{fig:replication-order}. We observe a clear bi-modal behavior of convergence / non-convergence, and likelier convergence with higher $\reporder$, consistent with our analysis of Sec.~3.4. The scenes ``ficus'' and ``ship'' typically take longer than 10k steps to converge, and thus are not well illustrated by this study.

\subsection{Neural Radiance Field}
We use the standard NeRF architecture of \cite{Mildenhall20eccv_nerf}, an MLP with 8 layers of 256 features, and one skip connection to layer 4. Our density and RGB branches use a single layer of 128 channels. We use a single MLP, sampled at 96 stratified points, plus 32 additional importance sampled points, guided by the initial stratified samples. We keep the $t_f - t_n$ span fixed at 3.0 for baseline NeRF runs with ground truth cameras. For \methodname runs, we linearly interpolate from 1.0 to 3.0 during the first 10k steps, which we found to aid convergence. We ignore view dependence for the first 50k steps by providing random view directions to the network. After 50k steps, we anneal a maximum of 4 frequency bands of directional encoding, until training completion at 100k steps. We use the Adam optimizer \cite{kingma2014adam} with a learning rate of $10^{-4}$, $\beta_1=0.9, \beta_2=0.999$ and $128$ pixels per batch.

\section{Additional Results}

\subsection{Lego Scene}

We show the self-similarity map and the regions of attraction of the Lego scene for a fixed elevation and a variable reference pose $z^*$ in Fig.~\ref{fig:ssm-roa-lego-fixed-elevation}.

\begin{figure}[ht]
  \centering
  \includegraphics[width=\linewidth]{fig/method_3d_new_3.pdf}
   \caption{
   We generate 100 views of the ``lego'' scene with from an elevation $\phi\simeq 7^\circ$ and an azimuth $\theta\in[0^\circ,360^\circ]$. The (normalized) self-similarity map $\degen_{f^*}$ shows that the views taken from opposite sides look similar, due to the presence of rotationally symmetric elements in the bulldozer. The regions of attraction $\conv_{f^*}(\theta^*)$ are colored in green. They only partially cover the latent space ($D(f^*)=0.55$, higher is better, see~\eqref{eq:difficulty}). With the equivalence relation $\relation=\relation_2$, the regions of attraction cover the quotient space $\latentspace/\relation$ almost entirely ($D_\relation(f^*)=0.87$, see~\eqref{eq:difficulty-rep}). Black dotted lines correspond to $\theta^*\in[\theta]$.
   }
   \label{fig:ssm-roa-lego-fixed-elevation}
\vspace{-3mm}
\end{figure}

\subsection{Comparison to VMRF}

In our comparison to VMRF, the predicted elevation is constrained to the range $[-30^\circ, 90^\circ]$, as done in~\cite{Zhang2022vmrf}. \methodname achieves higher reconstruction metrics on all the ``NeRF-Synthetic'' scenes (Table~\ref{tab:ab_initio_vmrf_comparison}).

\subsection{Number of Views}

We show qualitative reconstructions obtained on the ``lego'' dataset with a varying number of views in the training set in Fig.~\ref{fig:num-views-qual}.

\subsection{NeRF in the Wild}

We report quantitative metrics obtains on the real datasets in Table~\ref{tab:nerf_in_the_wild_qualitative} and qualitative reconstructions in Fig.~\ref{fig:nerf_in_the_wild_so3_se3}.

\begin{table}[ht]
  \centering
  \resizebox{\linewidth}{!}{
  \begin{tabular}{l|cc|cc|cc|cc|c}
  \toprule
  
  &
  \multicolumn{2}{c|}{GNeRF} &
  \multicolumn{2}{c|}{SAMURAI*} &
  \multicolumn{2}{c|}{\textbf{\methodname}} &
  \multicolumn{2}{c|}{SAMURAI} &
  NeRF \\
  
  Scene &
  Rot. ($^\circ$) & PSNR &
  Rot. ($^\circ$) & PSNR &
  Rot. ($^\circ$) & PSNR &
  Rot. ($^\circ$) & PSNR &
  PSNR \\
  
  \midrule
  
  Cape &
  56 & 16.3 &
  56 & 12.9 &
  \textbf{1.5} & \underline{19.5} &
  5.1 & \textbf{20.6} &
  19.1\\
  
  Head &
  42 & 14.5 &
  46 & 19.0 &
  \textbf{2.0} & 20.6 &
  7.1 & \underline{23.5} &
  \textbf{24.3} \\
  
  Toytruck &
  45 & 13.6 &
  54 & 15.6 &
  \textbf{2.5} & \underline{23.5} &
  3.4 & 22.2 &
  \textbf{26.2} \\
  
  \bottomrule 
  \end{tabular}
  } % tabular
  
  \caption{Mean angular error on predicted poses (training set) and quality of novel view synthesis (test set). SAMURAI* uses a fixed initialization of the poses at the North pole while SAMURAI uses a manual rough initialization. ``NeRF'' is our implementation using published poses, provided by COLMAP. SAMURAI can achieve a better reconstruction thanks to a more flexible 3D representation (shape, BRDF and per-camera illumination), in spite of less accurate predicted poses. As ground truth poses are provided by COLMAP, angular errors may be noisy estimates of the true pose accuracy.}
  \label{tab:nerf_in_the_wild_qualitative}
\end{table}

\subsection{Noise Sensitivity}

We compare \methodname to COLMAP and GNeRF on noisy datasets. We add pixel-independent Gaussian noise of variance $\sigma^2$ to the training images and evaluate the mean angular error and novel view synthesis accuracy of competing methods in Table~\ref{tab:ab_initio_vmrf_comparison}. We show a qualitative comparison between \methodname and GNeRF in Fig.~\ref{fig:noise-sweep-gnerf}.

\begin{table}[h]
  \centering
  \resizebox{\linewidth}{!}{
  \begin{tabular}{l|ccc|cc}
  \toprule
  \multicolumn{1}{c|}{Noise} & \multicolumn{3}{c|}{Pose Estimation}        & \multicolumn{2}{c}{Novel View Synthesis} \\
  \multicolumn{1}{c|}{Std. Dev.} & \multicolumn{3}{c|}{Rotation ($^\circ$)$\downarrow$} & \multicolumn{2}{c}{PSNR$\uparrow$} \\
  \cmidrule(r){2-4} \cmidrule(r){5-6}
  $\sigma$ & COLMAP & GNeRF &   \textbf{\methodname} & GNeRF & \textbf{\methodname}\\
  \midrule

% /cns/oe-d/home/synthx/users/mjmatthews/experiments/pixel_noise_sweep_52c_interval_anneal  
%   0.005   &
%   51.3 & 0.000 & 0.092 &  
%   00.00 & 32.90 & 
%   0.0000 & 0.9670 &  
%   0.0000 & 0.0621 \\
  
% /cns/oe-d/home/synthx/users/mjmatthews/experiments/pixel_noise_sweep_52c_interval_anneal
%   0.0125   &
%   58.5 & 0.000 & 0.094 &  
%   00.00 & 32.21 \\

% /cns/oe-d/home/synthx/users/mjmatthews/experiments/pixel_noise_sweep_52b_interval_anneal    
%   0.025    & 
%   51.3 & 0.000 & 0.098 &  
%   00.00 & 32.09 \\
  
% /cns/oe-d/home/synthx/users/mjmatthews/experiments/pixel_noise_sweep_52c_interval_anneal  
%   0.05    & 
%   fails & 1.252 & \textbf{0.096} &  
%   \textbf{31.77} & \textbf{31.77} \\

% /cns/oe-d/home/synthx/users/mjmatthews/experiments/pixel_noise_sweep_52c_interval_anneal  
  0.125    & 
   fails & 2.351 & \textbf{0.119} &  
  30.15 & \textbf{30.25} \\

% /cns/oe-d/home/synthx/users/mjmatthews/experiments/pixel_noise_sweep_52c_interval_anneal
  0.25    & 
   fails & 4.271 & \textbf{0.190} &
  27.32 & \textbf{28.62} \\  

%   /cns/oe-d/home/synthx/users/mjmatthews/experiments/pixel_noise_sweep_52c_interval_anneal  
  0.5    & 
   fails & 5.883 & \textbf{0.363} &  
  24.95 & \textbf{26.66} \\
  
%   /cns/oe-d/home/synthx/users/mjmatthews/experiments/pixel_noise_sweep_52b_interval_anneal
  1.0    & 
   fails & 7.557 & \textbf{1.784} &  
  22.90 & \textbf{24.20} \\
  
%   /cns/oe-d/home/synthx/users/mjmatthews/experiments/pixel_noise_sweep_52b_interval_anneal
  2.0    & 
   fails & 31.06 & \textbf{8.416} &  
  17.12 & \textbf{19.99} \\
  
  \bottomrule 
  \end{tabular}
  } % tabular
  
    \caption{Pose estimation accuracy and novel view synthesis quality of competing methods on the noisy ``lego'' scene.
    We report the best of three runs.
  }
  \label{tab:image_noise_sweep}
\end{table}

\begin{figure}[h]
  \centering
  \includegraphics[width=\linewidth]{fig/noise_sweep_gnerf.png}
   \caption{Novel view synthesis from noisy unposed images ($128\times 128$). Comparison between \methodname and GNeRF.}
   \label{fig:noise-sweep-gnerf}
\end{figure}

\begin{figure*}
  \centering
  \includegraphics[width=\linewidth]{fig/num_views.pdf}
   \caption{Qualitative reconstructions for various numbers of training views on the ``lego'' scene.}
   \label{fig:num-views-qual}
\end{figure*}

\begin{figure*}
  \centering
  \includegraphics[width=0.8\linewidth]{fig/nerf_in_the_wild_so3_se3.pdf}
   \caption{Additional qualitative results on the real datasets. Ground truth (GT) poses are provided by COLMAP. SAMURAI* uses a fixed initialization of the poses at the North pole while SAMURAI uses a manual coarse initialization. With the $\sothree$ parameterization, the object-to-camera distances and the camera in-plane translations are assumed to be known.}
   \label{fig:nerf_in_the_wild_so3_se3}
\end{figure*}
