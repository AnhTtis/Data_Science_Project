\begin{figure*}[t]
  \centering
  \includegraphics[width=\linewidth]{fig/method_3d_new_2.pdf}
   \caption{
   \textbf{(a)} One-dimensional quasi-periodic loss $\mathcal{L}$ defined on $\Omega=\sotwo$, minimized by $z^*$, and modulo loss $\mathcal{L}_\relation$ defined on $\Omega/\relation$. The region of attraction is the set of $z$'s that can be connected to $z^*$ with a path where $\mathcal{L}$ always decreases. In this example, the modulo loss is convex for $\relation=\relation_3$.
   \textbf{(b)} Self-similarity map and region of attraction on the Lego scene for all azimuths and elevations, fixed in-plane roll and a fixed reference camera pose $z^*$.
   The self-similarity map shows the photometric distance to the reference image with respect to the rendering pose $z\in\Omega$.  The region of attraction covers all the quotient set $\Omega/\mathcal{R}$ when $\mathcal{R}=\mathcal{R}_2$. See supplements for the self-similarity-map of the Lego scene with a fixed elevation and a variable reference $z^*$.
   }
   \label{fig:method-3d}
\vspace{-3mm}
\end{figure*}