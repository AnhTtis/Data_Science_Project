\begin{figure*}[t]
  \centering
  \includegraphics[width=\linewidth]{fig/method_3d.pdf}
   \caption{\textbf{(a)} The quotient set of a space for a given equivalence relation is the set of equivalence classes. In this example where $\Omega$ is a segment, each equivalence class contains three elements and the quotient set is three times smaller than $\Omega$. \textbf{(b)} We generate 100 views of the ``lego'' scene with from an elevation $\phi\simeq 7^\circ$ and an azimuth $\theta\in[0^\circ,360^\circ]$. The (normalized) self-similarity map $\degen_{f^*}$ shows that the views taken from opposite sides look similar, due to the presence of rotationally symmetric elements in the bulldozer. The regions of attraction $\conv_{f^*}(\theta^*)$ are colored in green. They only partially cover the latent space ($D(f^*)=0.55$, higher is better, see~\eqref{eq:difficulty}). With the equivalence relation $\relation=\relation_2$, the regions of attraction cover the quotient space $\latentspace/\relation$ almost entirely ($D_\relation(f^*)=0.87$, see~\eqref{eq:difficulty-rep}). Black dotted lines correspond to $\theta^*\in[\theta]$.
   }
   \label{fig:method-3d}
\vspace{-3mm}
\end{figure*}