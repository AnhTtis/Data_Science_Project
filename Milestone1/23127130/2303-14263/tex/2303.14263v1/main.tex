\documentclass[12pt]{article}
% Import basic packages
\usepackage{amsmath, amsfonts, amssymb, amsthm, graphicx, enumerate, setspace, color, colortbl, multirow, verbatim, array, float, longtable, booktabs, tablefootnote, plain, dsfont, color, makecell}

% Set layout
\usepackage[margin=1in]{geometry}
\usepackage[labelfont=bf]{caption}
\usepackage[title]{appendix}

% Import and format url
\usepackage{url}
\urlstyle{same}
\def\UrlBreaks{\do\/\do-}

% hyperref setup
\usepackage[breaklinks]{hyperref}
\hypersetup{
    colorlinks = true,
    linkcolor = {blue},
    citecolor = blue,
    urlcolor = {blue}
}

% Natbib setup for author-year style
\usepackage{natbib}
 \bibpunct[, ]{(}{)}{,}{a}{}{,}%
 \def\bibfont{\small}%
 \def\bibsep{\smallskipamount}%
 \def\bibhang{24pt}%
 \def\newblock{\ }%
 \def\BIBand{and}%

% Define theorems and commands
\newtheorem{thm}{Theorem}[]
\newtheorem{lemma}{Lemma}
\newtheorem{proposition}{Proposition}
\newtheorem{corollary}{Corollary}
\newtheorem{specialcase}{Special Case}
\newtheorem{observation}{Observation}
\newcommand{\fix}[1]{\textcolor{monkey}{\textbf{#1}}}
\newcommand{\fixb}[1]{\textcolor{gorilla}{[* #1 *]}}
\DeclareTextFontCommand{\emph}{\em}



% Title spacing
\usepackage{titlesec}
\def\endpf{\hfill \vrule depth-1pt height7pt width6pt}
\titlespacing\section{0pt}{10pt plus 4pt minus 2pt}{0pt plus 2pt minus 2pt}
\titlespacing\subsection{0pt}{10pt plus 4pt minus 2pt}{0pt plus 2pt minus 2pt}
\titlespacing\subsubsection{0pt}{10pt plus 4pt minus 2pt}{0pt plus 2pt minus 2pt}

% Color
\usepackage{pstricks}
\newrgbcolor{monkey}{1 0.0 0.0}
\newrgbcolor{gorilla}{0.0 0.0 0.75}


\begin{document}
\begin{titlepage}

\vspace{-1cm}

\title{\vspace{-1cm}The Dark Side of Algorithms? The Effect of Recommender Systems on Online Investor Behaviors}

		
\author{Ruiqi Rich Zhu\footnote{Scheller College of Business, Georgia Institute of Technology, Atlanta, Georgia 30308. E-mail: $\textrm{rich.zhu@scheller.gatech.edu}.$},
$\ $ Cheng He\footnote{Wisconsin School of Business, University of Wisconsin-Madison, Madison, Wisconsin 53706. E-mail: $\textrm{cheng.he@wisc.edu}.$},
$\ $ Yu Jeffrey Hu\footnote{Scheller College of Business, Georgia Institute of Technology, Atlanta, Georgia 30308. E-mail: $\textrm{jeffrey.hu@scheller.gatech.edu}.$}}


\vspace{-1.5cm}
\date{}
\maketitle
\vspace{-1cm}
\thispagestyle{empty}%

\setstretch{1.5}
% \begin{center}
% {\noindent\large{Version: 02/05/2023}}
% \end{center}

\begin{abstract}
\noindent Despite the widespread adoption of recommender systems by online investment platforms, empirical research into their impact on online investors’ behaviors is scarce. Using data from a global e-commerce platform, the authors of this study adopt a regression discontinuity design to causally examine the effects of recommender systems on online investor behaviors, specifically in a mutual fund investment context. The results show that funds featured by recommender systems prompt significantly more purchases. This effect is especially salient among unsophisticated investors, who appear more likely to follow system-provided recommendations. Further analysis also reveals that these investors tend to suffer significantly worse investment performance after purchasing the recommended funds. Thus, recommender systems threaten to amplify wealth inequality among investors in financial markets.
\\ \\

\noindent \emph{Keywords:} Recommender Systems, Fintech, Behavioral Finance, Mutual Fund
\end{abstract}
\end{titlepage}

\newpage
\setstretch{1.6}

%%%%%%%%%%%%%%%%
\section{Introduction}

The number of online retail investors is growing by leaps and bounds. In 2020 alone, the average daily net buying by online investors of U.S. listed equities has grown to \$1 billion, a fivefold increase compared with its pre-2020 level \citep{retailinvest}. Online retail investors also appear willing to make aggressive moves: In January 2021 for example, they took on institutional traders by backing GameStop's stock, increasing the share price by more than 1,700\% compared with its level in December 2020 \citep{eraofretail}. Credit Suisse estimates that as of Q3 2021, retail flow in the equity market is nearly double its level in 2010 \citep{creditsuisse}. Meanwhile, the popularity of online investment platforms such as Robinhood and SoFi has continued to grow rapidly, as measured by the number of investors and revenues \citep{fitzgerald}. These platforms allow online investors to purchase financial products using tactics that mimic their shopping experiences on other e-commerce platforms, including ubiquitous e-commerce features such as recommender systems. 

Recommender systems aim to shape investor behaviors and increase sales by featuring financial products with high historical returns. In consumer goods contexts, recommendations frequently lead to increased sales \citep{de2010technology, brynjolfsson2011goodbye, hosanagar2014will}; most recommender systems are designed to feature products' ratings and sales \citep{adomavicius2005toward}, which provide strong quality signals to consumers \citep{pathak2010empirical}. However, when it comes to financial products, the relationship between being recommended and higher sales is not obvious. Historical returns do not reliably predict future returns, so rational investors should not treat historical performance as a reliable quality signal when purchasing financial products \citep{kahn1995does, clifford2021salience, barber2021attention}. While recommender systems affect consumer behaviors and increase consumer goods sales by providing a reliable quality signal, it is unclear whether recommender systems based on an unreliable quality signal in the context of financial products shape investors' behaviors and enhance product sales.

Furthermore, consumers usually share similar perceptions of quality signals for consumer goods  \citep{zhao2013modeling}. Such consensus is not the case for financial products, because investors with varying knowledge and experience establish dramatically different evaluations of quality signals, such as historical returns \citep{havakhor2021tech}. For example, unsophisticated investors often chase financial products with high historical returns and tend to suffer losses due to their trend chasing \citep{barber2021attention}. With this study, we investigate specifically whether unsophisticated investors might be more likely to adopt an online investment platform's recommendations, as well as whether they suffer diminished outcomes after purchasing the recommended products. Recommender systems ideally should help reduce knowledge gaps among consumers, by providing a reliable quality signal to less-educated consumers \citep{oestreicher2012recommendation}; we consider instead whether, in financial markets, recommender systems actually might amplify wealth inequality among investors, as a dark-side implication.

To address these possibilities, we collect a novel data set from a global e-commerce platform that sells financial products directly to investors.\footnote{The platform ranks among the top five global e-commerce companies, according to GMV \citep{ecommercerank}.} Our data set contains investor-level mutual fund purchases and redemptions, along with investor profile information, which is rare in prior literature.\footnote{Most studies of mutual funds focus on market-level mutual fund inflows and outflows, without considering information about individual investors who purchase the fund.} The platform from which we collect these data relies on recommender systems that are similar to content-based recommendations \citep{adomavicius2005toward} and use a ranking algorithm. Top-performing funds, ranked by annual return, get recommended to investors; similar recommendations are common on investment platforms such as Alipay, Tenpay, and TD Ameritrade, where top-performing products based on historical returns are recommended to investors. When an investor views a focal fund on a page, the recommender system features, on that same page, the top five funds in the same category as the focal fund with the highest annual returns, as of the viewing date. Using a sharp regression discontinuity design (RDD), we compare sales of funds that just reach the annual return cutoff to be recommended with non-recommended funds that nearly reach this cutoff. In so doing, we can look for discontinuous jumps in fund purchases that follow discontinuous changes in whether the fund is recommended. The empirical evidence indicates that recommender systems affect investors' behaviors, even though they are not a reliable quality signal from a rational investor's perspective. On average, being featured as a recommended fund increases purchases by 3.4\%. 

Next, we investigate the heterogeneous effects of recommender systems that arise due to varying investor characteristics. The recommendation effect emerges as more salient for investors with low-level financial sophistication (i.e., lower education and lower income level). In follow-up analyses, we calculate profits (losses) experienced by the investors over their purchase and redemption histories. With a similar RDD approach, we find that investors of low-level financial sophistication tend to suffer significantly worse performance after purchasing recommended funds. However, we do not observe the same effect for investors with high-level financial sophistication. Thus, investors of low-level financial sophistication, who arguably should be better protected in financial markets, instead tend to have worse investment outcomes when they follow the suggestions of recommender systems provided by online investment platforms. Rather than helping investors with low-level financial sophistication, recommender systems might be amplifying wealth inequality between investors with low versus high-level financial sophistication. 

Finally, we propose explanations along with empirical evidence for why purchasing recommended funds lead to worse performance for investors of low-level financial sophistication. First, our study examines purchase and redemption behavior differences between investors who purchase recommended funds and those who purchase non-recommended funds. Our results show that investors of low-level financial sophistication tend to spend significantly less effort on searching when purchasing recommended funds than when purchasing non-recommended funds. Such a difference is not statistically significant for investors of high-level financial sophistication. With a simulation analysis, we also demonstrate that investors of low-level financial sophistication who purchase recommended funds tend to redeem mutual funds at a worse time than those who purchase non-recommended funds. There are no such results for investors of high-level financial sophistication. Second, we find that, on average, recommended funds in our observation window have a worse return gain in three months than non-recommended funds, which could be due to market overreaction and the herding effect \citep{de1985does}.

With these findings, we make several contributions. First, we extend the information systems literature related to the economic effect of recommender systems, by noting that recommender systems affect investor behaviors even though the quality signal is unreliable. Second, we establish that recommender systems can hurt investors and that those with the lowest level of financial sophistication suffer the most, revealing a potential dark side of recommender systems in financial market contexts. Our results suggest that recommender systems in the financial platform can amplify investors' irrational behaviors and further intensify economic inequality \citep{adomavicius2013recommender, adomavicius2018effects, walsh2020algorithms, zhang2021welfare}. Third, we explore the underlying mechanisms for why investors have worse performance when purchasing recommended funds and show their behavior differences in the purchase and redemption stages. These findings add to the literature on behavioral finance and provide empirical evidence of how investor behaviors change when technology enables financial information to be more accessible and salient to investors \citep{barber2002online, clifford2021salience, havakhor2021tech}. Fourth, our study has real-world implications for online investment platforms, which need to consider how their recommender systems might prompt unexpected online investor behaviors. The amplification of investors' irrational behaviors and intensification of economic inequality among investors can create systematic risks in the financial market, where many investors follow similar recommendations and are simultaneously affected by losses that propagate throughout the financial system \citep{benoit2017risks}.  

\section{\label{literature section}Literature Review}
In reviewing existing literature related to the study of recommender systems for financial products, we consider two main streams, pertaining to (1) the economic effects of recommender systems and (2) behavioral finance.

\subsection{\label{economic recommend}Economic Effects of Recommender Systems}
Existing information systems literature on the economic effects of recommender systems suggests that recommender systems increase firm sales and revenues \citep{de2010technology, pathak2010empirical, brynjolfsson2011goodbye, hosanagar2014will, lee2021product}. \cite{de2010technology} show that recommender systems can increase sales of both promoted and non-promoted products. \cite{pathak2010empirical} point out that the strength of the recommendation is relevant to the positive effect on sales. Recommendations also have a volume effect, such that they encourage consumers to consume more and increase both views and final conversion rates \citep{hosanagar2014will, lee2021product}. 

Whereas most previous work focuses on interactions between recommender systems and consumer goods such as books, clothes, music, furniture, or toys, we seek to extend the purview to financial products. In doing so, we acknowledge that recommender systems serve as a reliable quality signal for consumer goods, but they can be unreliable for financial products, because of their reliance on historical returns. Recommender systems promote financial products that have attained superior historical returns, but the quality of financial products depends on future returns, which are unobserved and cannot be reliably predicted using historical returns \citep{kahn1995does, jain2000truth, choi2020carhart}. When they recommend consumer goods though, recommender systems identify those goods with the highest historical ratings and sales, which represent stable, observable quality signals. Because the relationship between being recommended and higher sales is uncertain for financial products, unlike for consumer goods, we explicitly seek to study the effect of recommender systems on online investor behaviors.

Some researchers have used analytical models, lab experiments, and field data to examine the potentially undesirable impacts of recommender systems. For example, information technology (IT) powered by recommender systems and search engines can enhance communications access and filtering, but it also might produce more fragmented intellectual and social interactions \citep{van2005global}. According to \cite{van2005global}, this negative outcome results from bounded rationality, which leads to specialization and smaller ranges of overlapping activities. By satisfying  preferences, despite the friction of geography, such IT also might amplify individual preferences and create virtual, specialized communities across geographic boundaries. 

Another concern involves the potentially negative effects of recommender systems on consumer judgment and decision-making when purchasing consumer goods. \cite{adomavicius2013recommender} suggest that ratings presented by recommender systems serve as anchors for consumers' constructed preferences, with the assumption that their preference ratings are malleable and can be significantly affected by recommender systems. In another study, \cite{adomavicius2018effects} find that after seeing personalized recommendations, even in completely random or perturbed conditions (i.e., introducing recommendation error), individual consumers' willingness-to-pay judgments shift significantly. Using a large-scale field experiment, \cite{zhang2021welfare} also note that consumers are less price-elastic toward more salient recommended products, such that a system can limit consumer surplus and welfare relative to a system that is designed to maximize welfare. To expand our understanding of such implications of recommender systems, we focus on their potentially undesirable impacts on investment behaviors rather than consumer goods purchases. 

\subsection{\label{behavioral finance}Behavioral Finance}
Prior behavioral finance literature identifies various factors that can influence investment decisions. For example, investors might purchase financial products that capture their attention. As \cite{barber2008all} demonstrate, individual investors are net buyers of attention-grabbing stocks, and their attention-driven purchases stem from the difficulty of searching among the thousands of stocks they could buy. Purchase decisions also appear influenced by salient, attention-grabbing information, such as ``in-your-face" fees, exceptional performance, or marketing or advertising \citep{barber2005out}. Likewise, \cite{sirri1998costly} determine that purchases of mutual funds relate directly to the current media attention received by the fund, which lowers consumers’ search costs. \cite{clifford2021salience} suggest that mutual fund investors are more likely to purchase and redeem funds with high idiosyncratic volatility because their more extreme returns increase the salience of the fund and attract investor attention. We move beyond the influences of attention-grabbing features (e.g., extreme returns, fees) and news media on investment choices \citep{tumarkin2001news, antweiler2004all, tetlock2007giving, chen2014wisdom} to investigate the possibly similar influences of recommender systems. In particular, these systems can shift investor attention; through them, the platform also can choose which products to bring to investors' attention and update those recommendations instantly and frequently. In support of this effort, we use individual-level investor transaction data that are more fine-grained than, for example, aggregate-level mutual fund flows. In turn, we can analyze the effect of a recommendation on each investor, whereas most prior work pertains to the overall investor population level. Our individual-level analysis can reveal heterogeneous effects according to investors’ characteristics. Meanwhile, individual-level transaction data provide us with information we can use to investigate investors’ fund selection process and profits (losses), which rarely have been studied in prior literature.

Furthermore, several behavioral finance studies note investor characteristics and their effects on investment decisions. \cite{barber2001boys} argue that men are more overconfident than women and trade more frequently, leading to reductions of 2.65 percentage points per year, compared with 1.72 percentage points for women. \cite{grinblatt2016iq} demonstrate that cognitive ability influences mutual fund choice: High-IQ investors avoid funds with high management fees. In addition, \cite{anderson2013trading} argues that lower-income, poorer, younger, and less well-educated investors invest a greater proportion of their wealth in individual stocks, hold more highly concentrated portfolios, trade more, and exhibit worse performance. With our rich data set, we can track multiple investor characteristics, including income, liabilities, education, risk tolerance, and investment goals. Accordingly, we extend prior research by establishing that recommender systems have a more salient effect on investors with low-level financial sophistication, who in turn suffer the highest investment losses because they follow the system recommendations. 

Another stream of literature looks at the impact of expert recommendations on fund flow. One notable example is the paper by \cite{cookson2021best} that analyzes mutual fund recommendations curated by experts on investment platforms. As the expert recommendation is perceived as a reliable quality signal, it is no surprise that this literature finds a significant effect of the expert recommendation on sales. However, the recommender system studied in our context is based on historical return and without any human input, which is not a reliable quality signal from the perspective of rational investors. As a result, the effect of such a recommender system remains unclear. While most studies use descriptive analysis to show the relationship between expert recommendation and sales, we apply regression discontinuity design to study the causal effect of recommendation on investor behaviors, which will be discussed in more detail in Section~\ref{main effect}.

Another relevant work is \cite{barber2021attention} that establishes how the Robinhood app’s unique “Top Mover” list drives investors to herd and buy attention-grabbing stocks. As the “Top Mover” list displays stocks with high price fluctuations, it sends a different signal from the recommender systems in our study which is based on historical return. Most importantly, these prior studies focus on the aggregate effect of the app feature on sales using aggregate-level order flow, rather than the more granular effect of recommender systems on individual investors in our paper. Our individual-level data with the history of purchase and redemption allows us to evaluate the effects on both purchase behavior and investors' performance, as well as to study heterogeneous influences in terms of investor characteristics, thereby revealing whether such systems might affect the profits (losses) of investors with varying levels of financial sophistication.


\section{\label{data}Data and Empirical Context}

\subsection{\label{institutional background}Institutional Background}
We collect a unique data set from a top-five global e-commerce platform that features a business-to-consumer (B2C) business model. On the platform, finance companies can sell various financial products, such as mutual funds and insurance, and provide services, including crowdfunding and loans, directly to investors through the platform. 

The algorithm the platform uses is similar to those that inform content-based recommendations. When investors view a particular fund on the platform, it recommends five additional funds, on the right side of the screen. These recommended funds all are of the same type as the focal fund; they appear in the order of their annual returns from highest to lowest on the current viewing date. Funds that rank in a fifth or higher place, in terms of their annual returns, are featured to investors, as the example in Figure~\ref{fig:RSIllustration} shows. If the focal fund is a hybrid fund, then fund 4 represents the hybrid fund with the fourth highest annual return among all hybrid funds on the view date. On the basis of this clear recommendation scheme, we can infer which funds were recommended to investors during the sample period. 
% Recommender Systems Illustration
\begin{figure}[ht]
    \centering
    \caption{Recommender Systems Illustration}
    \includegraphics[scale=0.5]{figures/RS_illustration.png}
    \label{fig:RSIllustration}
\end{figure}

In informal interviews with employees of the platform, we learned that investors seek financial products with high historical returns. Furthermore, the recommender systems are designed to encourage investors to trade frequently, which increases the profits for the platforms. Using historical returns as input for recommender systems is common among online investment platforms; Tenpay, Alipay, and TD Ameritrade, three of the most popular online investment platforms in the world, feature similar designs which highlight financial products with high historical returns.

\subsection{\label{investor behavior}Investor Transactions}
Our data set includes a random sample of 10,000 individual investors. They are identified by their encrypted investor ID so we can gather their fund purchase and redemption histories between January 2015 and December 2016. That is, we observe the date, fund code, investor ID, and order amount for each transaction. We also obtain associated investor characteristics and fund attributes, which we describe in more detail in the following subsections.


\subsection{\label{fund attributes}Mutual Fund Attributes}
A rich panel data set pertaining to 2,999 funds on the platform includes detailed fund attribute information from January 2015 to December 2016. These data consist of the fund code, name,  type (i.e., stock, hybrid, bond, and money market), and  fees (which comprise management and custodial fees), as well as its establish time. Regarding fund performance, the data set lists weekly, monthly, three-month, annual, and lifetime returns.\footnote {For the analyses that follow, we exclude funds listed as unknown type; very few investors purchase funds of this type.} Table~\ref{tab:summaryStats} shows the summary statistics of key variables in this study.

% Summary Statistics
\begin{table}[ht!]
  \footnotesize
  \caption{Summary Statistics}
  \centering
    \begin{tabular}{llccccc}
    \midrule
    Variables & Description & N & Mean & Std & Min & Max \\
    \midrule
    ManagementFee & \makecell[lt]{Percentage of fees charged by \\ fund companies as operating cost} & 14,556 & 0.946 & 0.474 & 0.150 & 1.850 \\
    CustodialFee & \makecell[lt]{Percentage of fees charged by \\ banks to safeguard the securities} & 14,556 & 0.203 & 0.075 & 0.040 & 0.350 \\
    AnnualReturn & \makecell[lt]{Past annual return of a fund} & 14,556 & 0.513 & 0.738 & -0.071 & 9.510 \\
    LifetimeReturn & \makecell[lt]{Past lifetime return of a fund} & 14,556 & 0.736 & 0.950 & -3.489 & 9.461 \\
    Type & \makecell[lt]{General type of the fund} \\
    FundSize & \makecell[lt]{Size of the fund, measured in \\ 100 million RMB} & 14,556 & 10.838 & 14.258 & 0.000 & 158.472 \\
    EstablishTime & \makecell[lt]{Number of days the fund is \\ established} & 14,556 & 41.005 & 25.963 & 1.000 & 174.000 \\
    NumberOfPurchases & \makecell[lt]{Number of fund purchases} & 14,556 & 0.340 & 1.627 & 0.000 & 38.000 \\
    NumberOfInvestors & \makecell[lt]{Number of unique investors \\ making the purchase} & 14,566 & 0.322 & 1.555 & 0.000 & 35.000 \\
    PurchaseAmount & \makecell[lt]{Units of funds purchased} & 14,556 & 10.735 & 66.881 & 0.000 & 3761.000 \\
    NetAssetValue & \makecell[lt]{Price of a single unit of the fund} & 14,043 & 1.518 & 0.708 & 0.178 & 20.117 \\
    \midrule
    \end{tabular}
  \label{tab:summaryStats}%
\end{table}%


\subsection{\label{investor characteristics}Investor Characteristics}
We also obtain demographic information for each investor from a mandatory questionnaire when they build the account, as summarized in Table~\ref{tab:investorSummary} in Appendix~\ref{A}. This information includes investors’ levels of income, liabilities, education, risk tolerance, and investment goals. Most investors in this sample indicate an unstable income and a high school education or less. They appear extremely risk-averse and want to keep their capital safe. It closely resembles the mutual fund investor demographics in China\footnote{https://edu.efunds.com.cn/c/2022-01-14/497432.shtml} though with a slightly higher percentage of unsophisticated investors. We use these variables to segment investors and thereby identify heterogeneous recommender system effects in terms of investor characteristics. We encode the investor demographics variables as ordinal values, such that higher values represent higher incomes, smaller liabilities, higher education, risk tolerance, and motives to obtain larger returns. With these encoded values, we segment the pool of investors, as detailed in Section~\ref{investor level effect}.

\section{\label{main effect}Effect of Recommender Systems on Fund Purchases}
In this section, we examine the effect of recommender systems on purchases of recommended funds. A key empirical concern when trying to identify the causal effects of recommender systems is the possibility that recommended funds and non-recommended funds might differ inherently. Therefore, comparing the purchases of recommended and non-recommended funds directly would likely be misleading. Instead, we apply a regression discontinuity design (RDD) to exploit the method by how recommender systems display funds. Several robustness checks are also conducted to show the consistency of the results.

\subsection{\label{RDD}Empirical Strategy: Regression Discontinuity Design}
Recall that in this study, funds that rank fifth or higher in their annual returns are recommended to investors. We can exploit this design and restrict the sample to observations in which the funds rank fourth or fifth -- that is, they just reach the cutoff to be recommended. Then we consider funds that rank sixth and seventh, such that they just miss the recommendation cutoff. This bandwidth is optimal in mean squared error; furthermore, funds that fall within this window exhibit better covariate balance and overlap. We also conduct a robustness check with a narrower bandwidth in the following robustness checks and have consistent results. As the graphical illustration of our RDD in Figure~\ref{fig:RDDesign} reveals, a fund that ranks fourth or fifth gets recommended to the investor; a fund that ranks sixth or seventh does not. This variation in whether the fund is recommended is exogenous to fund attributes and performance.

% RDD Illustration
\begin{figure}[ht]
    \centering
    \caption{RDD Illustration}
    \includegraphics[scale=0.5]{figures/RDD.png}
    \label{fig:RDDesign}
\end{figure}

An identifying assumption of the RDD is that other determinants of fund purchases, such as annual returns, fund fees, or fund size, vary smoothly, but being recommended to investors varies discontinuously at the fifth rank in annual returns. A smoothness assumption appears to be reasonable if the treatment depends on annual returns because, unlike other factors such as fees, annual returns cannot be directly controlled by the funds. Thus, funds cannot easily manipulate the recommendation algorithm. Formally, to examine the effect of recommender systems, we estimate the following log-linear model:

\begin{eqnarray} \label{eq:mainEquation}
\begin{aligned}
\log{(NumberOfPurchases_{it})}= &\beta\ast Recommended_{it}+ f(AnnualReturn_{it}) + \\
& \theta\ast X_{it}+\alpha_{s(i)t}+\epsilon_{it}
\end{aligned}
\end{eqnarray}

We define the binary variable $Recommended_{it}$:
$$
Recommended_{it}= \begin{cases}1, & \text { if fund } i \text { rank in } 4^{\text {th }} \text { or } 5^{\text {th }} \text { place on day } t \\ 0, & \text { if fund } i \text { rank in } 6^{\text {th }} \text { or } 7^{\text {th }} \text { place on day } t\end{cases}
$$

The unit of analysis is fund ($i$) – day ($t$) (e.g., Agricultural Bank Green Energy Hybrid Fund – Jan. 25, 2015). The $NumberOfPurchases_{it}$ is the number of purchase transactions for fund $i$ on day $t$. Then $f(AnnualReturn_{it})$ is a local polynomial function of the fund’s annual return, up to the degree of 3\footnote{The choice of this degree is based on obtaining the highest BIC.}, that allows for the smooth change of annual return around the cutoff. Next, $\alpha_{s(i)t}$ refers to day-fund type fixed effects, which constitute experiments for each fund type on each transaction date. Thus we can compare the purchase number of recommended funds with that of non-recommended funds in each experiment. These fixed effects not only capture fund type-specified and day-specified unobserved factors but also capture daily shocks in the market that are common to each fund type. With $X_{it}$, we summarize the control variables, including fund attributes, fees, establishment time, and size, each of which captures a potential impact of fund attributes on the number of fund purchases. Finally, $\beta$ is the parameter of interest. With the identifying assumption that other determinants of fund purchases are continuous at the cutoff, we use $\beta$ to capture recommendation effects on fund purchases \citep{thistlethwaite1960regression, lee2010regression, cattaneo2022regression}. 


\subsection{\label{purchase positive}Estimation Results}
Table~\ref{tab:mainEffect} contains the estimated effect of recommendations on online investors’ purchase behaviors. 
% Effect of Fund Recommendation on Fund Purchases
\begin{table}[ht!]
\footnotesize
\caption{Effect of Recommendation on Fund Purchases}
  \centering
    \begin{tabular}{lcccc}
    \midrule
    \multicolumn{1}{l}{Dependent Variable:} & \multicolumn{1}{c}{(1)} & \multicolumn{1}{c}{(2)} & \multicolumn{1}{c}{(3)} \\
    \multicolumn{1}{l}{log(NumberOfPurchases)} & \multicolumn{1}{c}{} & \multicolumn{1}{c}{} & \multicolumn{1}{c}{} \\
    \midrule
    \multicolumn{1}{l}{Recommended} & \multicolumn{1}{c}{0.033***} & \multicolumn{1}{c}{0.033***} & \multicolumn{1}{c}{0.021***} \\
    \multicolumn{1}{r}{} & \multicolumn{1}{c}{(0.006)} & \multicolumn{1}{c}{(0.006)} & \multicolumn{1}{c}{(0.007)} \\
    \multicolumn{1}{l}{Polynomial Degree} & \multicolumn{1}{c}{3} & \multicolumn{1}{c}{3} & \multicolumn{1}{c}{3} \\
    \multicolumn{1}{l}{Day Dummies * Fund Type Dummies} & \multicolumn{1}{c}{Yes} & \multicolumn{1}{c}{Yes} & \multicolumn{1}{c}{No} \\
    \multicolumn{1}{l}{Control Variables} & \multicolumn{1}{c}{Yes} & \multicolumn{1}{c}{No} & \multicolumn{1}{c}{No} \\
    \multicolumn{1}{l}{Number of Observations} & \multicolumn{1}{c}{14,556} & \multicolumn{1}{c}{14,556} & \multicolumn{1}{c}{14,556} \\
    \multicolumn{1}{l}{\textcolor[rgb]{ .133,  .133,  .133}{$R^2$}} & \multicolumn{1}{c}{0.386}  & \multicolumn{1}{c}{0.364} & \multicolumn{1}{c}{0.015} \\
    \midrule
    \multicolumn{4}{p{30em}}{\scriptsize Notes: {***} $p<0.01$, {**} $p<0.05$, {*} $p<0.1$. Control variables include fund fees, fund size, and fund establish time. Robust standard errors are in parentheses.} \\
    \end{tabular}
  \label{tab:mainEffect}
\end{table}

The estimates in Column (1) suggest that funds featured by recommender systems prompt more purchases (0.033), and the effect is statistically significant ($p < 0.01$). The log-linear model indicates a 3.4\%\footnote{Because  we use a log-linear model, we interpret the coefficient as 100*(exp(0.033) - 1)\% change.} increase in the number of purchases of featured funds, compared with non-recommended funds. As shown in Columns (2) and (3), this result is consistent when we relax our model specification by removing control variables and fixed effects gradually. This finding suggests that recommender systems that use historical returns exert influences in financial product domains. However, as prior literature has established, historical returns are not reliable quality signals in financial markets \citep{kahn1995does, jain2000truth, choi2020carhart}. Therefore, even though funds featured by recommender systems are not more likely to realize superior returns in the future, investors accept the influence of recommender systems and purchase the recommended funds.



\subsection{\label{robustness}Robustness Checks}
To strengthen the causal interpretation of our results, we conduct several robustness checks, including (1) a falsification exercise, (2) alternative bandwidths, (3) alternative polynomial functions, (4) alternative measurements, (5) lagged effect, (6) a different model specification, and (7) a sub-sample analysis. The results from these analyses align with our main findings, as we detail next. See tables in Appendix~\ref{B} for details.

\subsubsection{\label{falsification}Falsification Exercise}
To rule out an alternative explanation that any funds with higher rankings prompt more purchases, we perform a falsification exercise in which funds that rank fourth are the recommended funds, and funds that rank fifth represent the non-recommended funds. Both funds are recommended in reality, so arguably, funds that rank fourth should not account for significantly more purchases than funds that rank fifth. The results, as presented in Column (1) of Table~\ref{tab:falsificationAlternateBandwidth} in Appendix~\ref{B}, indicate that this placebo recommendation effect is statistically insignificant. We perform a similar placebo assessment of funds rank sixth and seventh, and as the results in Column (2) of Table~\ref{tab:falsificationAlternateBandwidth} in Appendix~\ref{B} show, it again has a statistically insignificant effect. The falsification test thus strengthens our causal interpretation of the main effect estimation.


\subsubsection{\label{bandwidth}Alternative Bandwidth}
Because funds that are closer in rank might be more similar than funds that are farther in the rankings, we check the robustness of our results with a narrower bandwidth. That is, we include only the fifth-ranked funds as recommended and the sixth-ranked ones as non-recommended. The estimation result, provided in Column (3) of Table~\ref{tab:falsificationAlternateBandwidth} in Appendix~\ref{B}, shows that the recommendation effect on the number of purchases remains positive and statistically significant. Thus, our results are robust to the use of a narrower bandwidth.

\subsubsection{\label{polymonial}Alternative Polynomial Functions}
Our primary model uses the polynomial function with a degree of 3 which achieves the highest BIC. To test whether our results are sensitive to the degree of the polynomial function, we instead include degree 1 and degree 2 polynomial functions in our primary model. Columns (1) and (2) of Table~\ref{tab:differentPolynomials} in Appendix B demonstrate the estimation results. The coefficients of $Recommended_{it}$ are consistently positive and significant.

In our primary model, we assume the relationships between annual return and fund purchases are the same on both sides of the cutoff. To relax this assumption, we include the interaction between $Recommended_{it}$ and the polynomial function, which allows for separate relationships between annual return and fund purchases on either side of the cutoff. Column (3) of Table~\ref{tab:differentPolynomials} in Appendix B shows the estimation results which are consistent with the findings in our primary model.

\subsubsection{\label{moredv}Alternative Measurements}
Instead of the number of purchases, we gauge the number of unique investors and total fund amount purchased as alternative measures of the effect of recommender systems. Our approach is similar to that represented by Equation~\ref{eq:mainEquation}, except that as the dependent variable, we use $\log{(NumberOfInvestors_{it})}$ and $\log(PurchaseAmount_{it})$. The $\log{(NumberOfInvestors_{it})}$ is the log of the number of unique investors who purchased fund $i$ on the day $t$; the $\log(PurchaseAmount_{it})$ is the log of the total units of fund $i$ purchased on the day $t$. The estimation results are in Table~\ref{tab:altDV} in Appendix~\ref{B}. In Column (1), we observe that more unique investors purchase featured funds. The effect is positive (0.033) and statistically significant ($p < 0.01$), amounting to 3.4\% more investors purchasing recommended rather than non-recommended funds. In Column (2), the coefficient of recommendation is positive (0.100) and statistically significant ($p < 0.01$), which suggests that recommended funds enjoy 10.5\% more purchased units than non-recommended funds.

\subsubsection{\label{lagged}Lagged Effect}
In our primary model, we investigate the sales change on the date when funds are recommended. One argument is that investors may not immediately purchase recommended funds. To mitigate this concern, we redefine $Recommended_{it}$ as being recommended at least once in the past three days and did not appear as non-recommended funds during those three days. Non-recommended funds remain the same as in our primary model. We then conduct a similar approach in Section~\ref{main effect}. Estimation results are presented in Column (3) of Table~\ref{tab:altDV} in Appendix~\ref{B}. The coefficient for $Recommended_{it}$ remains positive and statistically significant. 

\subsubsection{\label{poisson}Different Model Specification}
The dependent variable in our study is the number of purchases, which is count data. We use the Poisson regression model, which is widely adopted for count data, to check whether our results remain consistent \citep{consul1992generalized}. We present the results in Column (1) of Table~\ref{tab:altModelSubsample} in Appendix~\ref{B} where we use the same formula as in Equation~\ref{eq:mainEquation}. The coefficient for $Recommended_{it}$ remains positive and statistically significant. 


\subsubsection{\label{matching}Sub-sample Analysis}
In our main model, we include several fund attributes to capture investors' potential preferences. These attributes largely rule out the impact of fund-specific features on the number of purchases, but we also acknowledge that we cannot collect all fund attributes. To alleviate this concern, we perform a sub-sample analysis, in which we identify 433 distinct funds that rank in fourth to seventh place, of which 279 appear as recommended or non-recommended funds on different days in our sample period. These 279 funds also appear in each category roughly the same number of times. Therefore, we create a sub-sample of these 279 funds and include fund fixed effects in the model to capture all fund-specific attributes, which are directly comparable and differ only in terms of whether they are featured as recommended or not on a particular day. The results in Column (2) of Table~\ref{tab:altModelSubsample} in Appendix~\ref{B} confirm that the recommendation effect remains positive and significant.



\section{\label{investor level effect}Heterogeneity Analysis by Investor Characteristics}
The results thus far show that on average, recommender systems affect investors' purchase behaviors. In this section, using investor-level analyses, we examine how such an effect might vary across investors, depending on their characteristics. In detail, we divide investors into groups according to their income, liabilities, education, risk tolerance, and investment goals. Investors in different groups might perceive quality signals provided by recommender systems differently, which would lead to varying influences on their behaviors. To test these predictions, we first establish the investor segmentation using K-means clustering, then examine the heterogeneous effects of recommender systems across the identified investor groups.

\subsection{\label{segmentation}Investor Segmentation}
We segment our investors into three groups on the basis of their income, liabilities, education, risk tolerance, and investment goals. 
% Investor Segmentation
\begin{table}[ht!]
\footnotesize
\caption{Investor Segmentation}
  \centering
    \begin{tabular}{cccc}
    \midrule
    Investors Group & \multicolumn{1}{c}{\# of Investors} & \multicolumn{1}{c}{Investor Characteristics} & \multicolumn{1}{c}{Cluster Center} \\
    \midrule
    \makecell{Group 1: \\ Low Level \\Financial Sophistication}  & 8,508 & \makecell{Low Income \\ Low Education \\ Low Risk Tolerance} & \makecell{Income: 0.12 \\ Liabilities: 0.16 \\ Education: 0.15 \\ Risk Tolerance: 0.06 \\ Investment Goals: 0.00} \\
    \midrule
    \makecell{Group 2: \\ Medium Level \\Financial Sophistication} & 1,008 & \makecell{Medium Income \\ Medium Education \\ Medium Risk Tolerance} & \makecell{Income: 0.97 \\ Liabilities: 1.23 \\ Education: 1.68 \\ Risk Tolerance: 2.08 \\ Investment Goals: 1.08} \\
    \midrule
   \makecell{Group 3: \\ High Level \\Financial Sophistication} & 484 & \makecell{High Income \\ High Education \\ High Risk Tolerance} & \makecell{Income: 2.00 \\ Liabilities: 3.00 \\ Education: 1.80 \\  Risk Tolerance: 2.18 \\ Investment Goals: 1.18} \\
    \midrule
    \multicolumn{4}{p{45em}}{\scriptsize Notes: The cluster centers are based on K-means clustering results using investors' income, liabilities, education, risk tolerance, and investment goals. All investor characteristics are encoded as ordinal variables and normalized. Detailed value encoding is presented in Table~\ref{tab:investorSummary} in Appendix~\ref{A}.} \\
    \end{tabular}
  \label{tab:investorSegmentation}
\end{table}

These features can be encoded as standardized ordinal variables and are used in K-means clustering. Specifically, we use education level as a proxy for investors’ financial knowledge and ability. Income level, liabilities, and risk tolerance should inform investors’ fund selection process. The investment goals also guide their preferences for certain types and attributes of funds. All these variables are important factors determining investors' financial sophistication level \citep{calvet2009measuring}. A summary of the three groups, defined by investor characteristics, appears in Table~\ref{tab:investorSegmentation}. Group 1 investors represent low-level financial sophistication investors, with low income, education, and risk tolerance levels. Group 2 investors represent medium-level financial sophistication investors, with medium income, medium education, and medium risk tolerance levels. Then Group 3 represents investors with high-level financial sophistication, with the highest income level, education, and risk tolerance levels. These segmentation results match real-world categorizations, in that investors usually are characterized by their (financial) ability to invest and the risk that they are willing to tolerate.


\subsection{\label{investor heterogeneous effect}Heterogeneous Effects of Recommendation by Investor Groups}
We perform the analyses indicated by Equation~\ref{eq:mainEquation} for each investor group using sub-samples by investor groups. The estimation results are presented in Table~\ref{tab:recommendByGroup}. Columns (1) - (3) show the estimation results for the sub-sample of investors with low-, medium-, and high-level financial sophistication respectively. The results suggest that the purchase behaviors of all three groups are significantly influenced by the recommendation provided by the platform despite varying degrees of magnitude. The recommendation effect is strongest for unsophisticated investors (i.e., those with low-level financial sophistication), suggesting that they are more likely to purchase recommended funds. On average, unsophisticated investors engage in  2.6\% more recommended fund purchases, by number, than non-recommended funds. This number is more than triple compared to investors with medium- and high-level financial sophistication. Moreover, using Chow tests, we find the effect of recommendation on low-level financial sophistication investors to be significantly higher than those with medium-level (difference $= 0.012; p = 0.021$) and high-level financial sophistication investors (difference $= 0.011; p =0.046$). Considering the characteristics of low-level financial sophistication investors, they are least likely to possess enough knowledge to choose funds and make rational investment decisions. As a result, compared with investors with medium- and high-level financial sophistication, who note their higher income and education levels, investors of low-level financial sophistication are more likely to perceive a recommendation by the platform as a strong quality signal that seemingly should inform their purchase decision.

% Effect of Fund Recommendation by Investor Group
\begin{table}[ht!]
\footnotesize
\caption{Heterogeneous Effects of Recommendation by Investor Groups}
  \centering
    \begin{tabular}{cccc}
    \midrule
    \multicolumn{1}{l}{Dependent Variable:} & \multicolumn{1}{c}{(1)} & \multicolumn{1}{c}{(2)} & \multicolumn{1}{c}{(3)} \\
    \multicolumn{1}{l}{log(NumberOfPurchases)} & \multicolumn{1}{c}{\makecell{Financial \\ Sophistication: Low}} & \multicolumn{1}{c}{\makecell{Financial \\ Sophistication: Medium}} & \multicolumn{1}{c}{\makecell{Financial \\ Sophistication: High}} \\
    \midrule
    \multicolumn{1}{l}{Recommended} & \multicolumn{1}{c}{0.026***} & \multicolumn{1}{c}{0.006***} & \multicolumn{1}{c}{0.008***} \\
    \multicolumn{1}{r}{} & \multicolumn{1}{c}{(0.005)} & \multicolumn{1}{c}{(0.002)} & \multicolumn{1}{c}{(0.003)} \\
    \multicolumn{1}{l}{Polynomial Degree} & \multicolumn{1}{c}{3} & \multicolumn{1}{c}{3} & \multicolumn{1}{c}{3} \\
    \multicolumn{1}{l}{Day Dummmies * Fund Type Dummies} & \multicolumn{1}{c}{Yes} & \multicolumn{1}{c}{Yes} & \multicolumn{1}{c}{Yes} \\
    \multicolumn{1}{l}{Control Variables} & \multicolumn{1}{c}{Yes} & \multicolumn{1}{c}{Yes} & \multicolumn{1}{c}{Yes} \\
    \multicolumn{1}{l}{Number of Observations} & \multicolumn{1}{c}{14,556} & \multicolumn{1}{c}{14,556} & \multicolumn{1}{c}{14,556} \\
    \multicolumn{1}{l}{\textcolor[rgb]{ .133,  .133,  .133}{$R^2$}} & \multicolumn{1}{c}{0.393} & \multicolumn{1}{c}{0.300} & \multicolumn{1}{c}{0.298} \\
    \midrule
    \multicolumn{4}{p{48em}}{\scriptsize Notes: {***} $p<0.01$, {**} $p<0.05$, {*} $p<0.1$. Control variables include fund fees, fund size, and fund establish time. Columns (1)-(3) show the estimation results for the sub-sample of investors with low, medium, and high financial sophistication levels respectively. Robust standard errors are in parentheses.} \\
    \end{tabular}
  \label{tab:recommendByGroup}
\end{table}


\section{\label{effect on return}Effect of Recommender Systems on Investment Return}
Substantial literature has established that historical returns cannot reliably predict future returns \citep{kahn1995does, clifford2021salience, barber2021attention}. Therefore, we must ask whether purchasing a recommended fund leads to better or worse outcomes for investors. In this section, we examine the effect of recommended fund purchases on investment returns.

\subsection{\label{negative effect}Investment Return Difference Between Recommended and Non-recommended Funds}
For this analysis, we calculate the alpha of a mutual fund using compound returns to represent investment returns, such that we can take both the holding period and profits (losses) into account and obtain a risk-adjusted measure of how a mutual fund performs in comparison to the overall market average return. We perform the analysis at the investment level and our sample consists only of individual investors' purchases of recommended and non-recommended funds. Once an investor purchases a fund that is either recommended or non-recommended at the date of purchase, we obtain the purchase price, $PurchasePrice_{ijt}$. For each fund, we keep track of the investor’s holding amount history. During the holding period, each investor can increase or decrease holdings of fund $i$, through additional purchases or redemptions. Once the holding amount reaches 0, we mark that day as the final redemption date, then obtain the final redemption price, $RedemptionPrice_{ijt}$. The number of days between the fund purchase date and the final redemption date represents the holding duration, $HoldingDuration_{ijt}$. Using the calculated holding duration and observed purchase and redemption prices, we apply a compound return formula that calculates the interest rate to represent actual investment return. 
$ActualInvestReturn_{ijt}$ = $\text{exp}(\frac{{\text{log}(RedemptionPrice_{ijt} / PurchasePrice_{ijt})}}{HoldingDuration_{ijt}+1})-1$, taking accounts for both time and profit. After that, we calculate the benchmark investment return by using the same formula as the actual investment return but replace $PurchasePrice$ with $AvgPurchasePrice$, which is the average price of all the funds on the purchase date $t$ that share the same fund type as fund $i$. $AvgRedemptionPrice$ is the average price of all the funds that share the same fund type as fund $i$ on the redemption date of this investment. Then we calculate the benchmark return as $BenchmarkReturn_{ijt}$ = $\text{exp}(\frac{{\text{log}(AvgRedemptionPrice_{ijt} / AvgPurchasePrice_{ijt})}}{HoldingDuration_{ijt}+1})-1$. Finally, we calculate the risk-adjusted investment return as $InvestReturn_{ijt} = ActualInvestReturn_{ijt} - BenchmarkReturn_{ijt}$.  All returns are represented in basis points.

With these investment returns as our dependent variable, we apply the RDD approach again to investigate the return difference between recommended and non-recommended funds with the following model: 

\begin{eqnarray} \label{eq:invest_return}
\begin{aligned}
InvestReturn_{ijt}= &\beta\ast Recommended_{it}+\delta_j * Group_j+ \\ 
& f(AnnualReturn_{it}) + \theta\ast X_{it}+\alpha_{s(i)t}+\epsilon_{ijt}
\end{aligned}
\end{eqnarray}
We perform this analysis at the fund($i$)-investor($j$)-day($t$) level. $InvestReturn_{ijt}$ is the risk-adjusted investment return that investor $j$ achieve from purchasing the fund $i$ on the day $t$. The fixed effects and control variables are the same as in Equation~\ref{eq:mainEquation}.

The estimation results, presented in Column (1) of Table~\ref{tab:recommendInvestReturn}, suggest that purchasing recommended funds is associated with 0.04 lower basis points in returns. Such a difference is marginally significant. On average, investors appear to perform worse when they purchase recommended funds. 

% Effect of Fund Recommendation on Investment Return
\begin{table}[ht!]
\footnotesize
  \centering
  \caption{Effect of Recommendation on Investment Return}
    \begin{tabular}{lllll}
    \midrule
    \multicolumn{1}{l}{Dependent Variable:} & \multicolumn{1}{c}{(1)} & \multicolumn{1}{c}{(2)} & \multicolumn{1}{c}{(3)} & \multicolumn{1}{c}{(4)} \\
    \multicolumn{1}{l}{InvestReturn} & \multicolumn{1}{c}{\makecell{All \\ Investors}} & \multicolumn{1}{c}{\makecell{Financial \\ Sophistication: Low}} & \multicolumn{1}{c}{\makecell{Financial \\ Sophistication: Medium}} & \multicolumn{1}{c}{\makecell{Financial \\ Sophistication: High}}\\
    \midrule
    \multicolumn{1}{l}{Recommended} & \multicolumn{1}{c}{-0.036*} & \multicolumn{1}{c}{-0.055***} & \multicolumn{1}{c}{0.019} & \multicolumn{1}{c}{-0.024} \\
    \multicolumn{1}{r}{} & \multicolumn{1}{c}{(0.020)} & \multicolumn{1}{c}{(0.015)} & \multicolumn{1}{c}{(0.098)} & \multicolumn{1}{c}{0.039} \\
    \multicolumn{1}{l}{Polynomial Degree} & \multicolumn{1}{c}{3} & \multicolumn{1}{c}{3} & \multicolumn{1}{c}{3} & \multicolumn{1}{c}{3} \\
    \multicolumn{1}{l}{Investor Group Dummies} & \multicolumn{1}{c}{Yes} & \multicolumn{1}{c}{No} & \multicolumn{1}{c}{No} & \multicolumn{1}{c}{No} \\
    \multicolumn{1}{l}{Fund Type Dummies} & \multicolumn{1}{c}{Yes} & \multicolumn{1}{c}{Yes} & \multicolumn{1}{c}{Yes} & \multicolumn{1}{c}{Yes} \\
    \multicolumn{1}{l}{Control Variables} & \multicolumn{1}{c}{Yes} & \multicolumn{1}{c}{Yes} & \multicolumn{1}{c}{Yes} & \multicolumn{1}{c}{Yes} \\
    \multicolumn{1}{l}{Number of Observations} & \multicolumn{1}{c}{5,638} & \multicolumn{1}{c}{4,155} & \multicolumn{1}{c}{894} & \multicolumn{1}{c}{589} \\
    \multicolumn{1}{l}{\textcolor[rgb]{ .133,  .133,  .133}{\textit{$R^2$}}} & \multicolumn{1}{c}{0.015} & \multicolumn{1}{c}{0.018} & \multicolumn{1}{c}{0.018} & \multicolumn{1}{c}{0.170} \\
    \midrule
    \multicolumn{5}{p{48em}}{\scriptsize Notes: {***} $p<0.01$, {**} $p<0.05$, {*} $p<0.1$. InvestReturn is calculated using risk-adjusted compound investment return. Control variables include fund fees, fund size, and fund establish time. Column (1) shows the estimation results for the sample of all investors. Columns (2)-(4) show the estimation results for the sub-samples of investors with low-, medium-, and high-level financial sophistication respectively. Robust standard errors are in parentheses.} \\
    \end{tabular}%
  \label{tab:recommendInvestReturn}
\end{table}%


\subsection{\label{investor performance}Heterogeneous Effect of Recommendation on Investment Return by Investor Groups}
Our previous results consistently indicate that the recommendation effect is heterogeneous across investor groups. These heterogeneous effects in turn might lead to heterogeneous investment returns across investor groups. We examine such heterogeneous effects using equation~\ref{eq:invest_return} and perform sub-sample analysis by dividing investors into groups based on their level of financial sophistication which is similar to the approach in Section~\ref{investor heterogeneous effect}. 

The estimation results are presented in Columns (2) - (4) of Table~\ref{tab:recommendInvestReturn}. The results show that only investors with low-level financial sophistication suffer significantly worse performance when purchasing recommended funds. For investors of medium and high-level financial sophistication, purchasing recommended funds is not associated with significantly worse returns than purchasing non-recommended ones. Along with our prior results that unsophisticated investors are more likely to purchase recommended funds, these findings imply that unsophisticated investors rely most on recommender systems and are also hurt the most by the recommendations. This dark side of recommender systems is worrisome, in that investors who need recommendations the most also suffer the most, which could amplify the problematic wealth inequality among investors in financial markets.

\subsection{\label{return explain}Potential Explanations on Worse Performance when Purchasing Recommended Funds}
We explore three potential explanations, along with empirical evidence, for why purchasing recommended funds tends to lead to poorer performance, especially for investors of low-level financial sophistication. First, we examine behavior differences between investors who purchase recommended funds and those who purchase non-recommended funds. We investigate behavior differences in both the purchase and redemption stages. Second, we study the performance differences between recommended funds and non-recommended funds.

\subsubsection{\label{fund research}Behavior Differences in the Purchase Stage}
To explore whether investors who purchase recommended funds have different behaviors in the purchase stage, we specifically examine the differences in search effort investors spent before they make a purchase decision. With more effort exerted, investors likely make more informed and rational investment decisions \citep{barberis2003survey}. But if they have access to recommender systems, investors might reduce their effort and make a poorer purchasing decision. We test this hypothesis by using the total number of clicks an investor makes 7 days before each fund purchase, as a proxy for the search effort. Results are presented in Table~\ref{tab:clicksBeforePurchase}. 

% Number of Clicks Before Fund Purchases
\begin{table}[ht!]
  \footnotesize
  \caption{Number of Clicks Before Fund Purchases}
  \centering
    \begin{tabular}{lccc}
    \midrule
    Investor Group & Recommended Funds & Non-Recommended Funds & Difference\\
    \midrule
    All Investors & 67.103 & 73.546  & -6.443**   \\
    Financial Sophistication: Low  & 53.813 & 62.438 & -8.625*** \\
    Financial Sophistication: Medium & 138.965 & 126.296 & 12.669 \\
    Financial Sophistication: High & 40.560 & 46.853 & -6.293 \\
    \midrule
    \multicolumn{4}{p{40em}}{\scriptsize Notes: {***} $p < 0.01$, {**} $p < 0.05$, {*} $p < 0.1$. The significance level is based on Paired t-test.} 
    \\
    \end{tabular}
    \vspace{1em}
  \label{tab:clicksBeforePurchase}%
\end{table}%

We find significantly more clicks before investors purchase non-recommended funds than recommended funds (73.55 vs. 67.10, $p = 0.027$), so investors generally expend more effort before purchasing a non-recommended rather than a recommended fund. We then explore the search effort difference across investor groups. Columns (2) to (4) in Table~\ref{tab:clicksBeforePurchase} show such differences for investors of low-, medium-, and high-level financial sophistication. Our results demonstrate that only investors with low-level financial sophistication spent significantly less effort on searching before purchasing recommended funds compared to purchasing non-recommended funds. There are no significant differences in the search effort between purchasing recommended funds and purchasing non-recommended funds for investors with medium- and high-level financial sophistication. \cite{havakhor2021tech} show that access to tech-enabled raw financial data can evoke gambling-like behaviors among less sophisticated investors. They argue that access to additional performance data increases unsophisticated investors' overconfidence and excessive trading. Unsophisticated investors may be least capable of processing relevant financial information. When they encounter recommender systems that suggest financial products with high historical returns, they might be strongly influenced by this highlighted information and become more behaviorally biased, overconfident, and ready to trade in a gambling-like fashion compared with more sophisticated investors. This also aligns with our previous results that investors of low-level financial sophistication are more likely to follow recommender systems. As a result, spending less search effort when purchasing recommended funds could be one potential reason that leads to worse investment performance.  


\subsubsection{\label{fund redemption timing}Behavior Differences in the Redemption Stage}
Fund redemption timing is also critical to investment returns. Selling funds at relatively higher prices produce higher returns. If investors are more informed before purchasing non-recommended funds, perhaps the relatively uninformed investors who buy recommended funds are less confident in their decisions, so they redeem the funds irrationally at a less optimal time, resulting in diminished investment return. We test this conjecture by simulating investors' potential investment returns if they were to redeem their funds a few months after their current redemption date. We first simulate investors' returns had they sold the fund 6 months\footnote{We also use 1 month and 3 months to conduct the simulation. Results are consistent.} after their current fund redemption date. Then we calculate the difference between simulated investment return and realized investment return and define the difference as hypothetical excess return. A larger value of hypothetical excess return means a worse time of redemption. In Table~\ref{tab:fundReturnCompareGroup}, the first two columns show the hypothetical excess return for investors who purchase recommended funds and non-recommended funds respectively, and the third column shows the difference in hypothetical excess return between investors who purchase recommended funds and those who purchase non-recommended funds. The first row of Table~\ref{tab:fundReturnCompareGroup} shows that on average, investors who purchase recommended funds tend to have a larger hypothetical excess return than those who purchase non-recommended funds, suggesting that investors redeem mutual funds at a worse time when purchasing recommended funds than when purchasing non-recommended funds. However, the difference is not significant. 

We then conduct a similar simulation for investors of different groups and present the results in the last three rows of Table~\ref{tab:fundReturnCompareGroup}. The results show that only investors of low-level financial sophistication redeem mutual funds at a significantly worse time when purchasing recommended funds as the difference between hypothetical excess returns is positive and significant (0.186 vs 0.081; difference = $0.105; p < 0.01$). Another interesting result is that investors of high-level financial sophistication tend to have the worst redemption timing regardless of recommended vs non-recommended funds. This finding is consistent with conclusions in the past literature that investors with higher education and experience tend to have a higher self-attribution bias and be more overconfident \citep{mishra2015study}, which can lead to excessive trading and worse investment performance \citep{barber2000trading, hoffmann2016does}. Our results also suggest that investors of low-level and high-level financial sophistication tend to time fund redemption poorly after purchasing recommended funds. Thus, a worse timing in the redemption stage could be one potential reason for a worse investment return when purchasing recommended funds.


% Fund return simulation by group using alpha
\begin{table}[ht!]
\footnotesize
  \caption{Hypothetical Excess Return Comparison Between Recommended and Non-Recommended Funds}
  \centering
    \begin{tabular}{cccc}
    \midrule
    \multicolumn{1}{c}{Investor Group} & \multicolumn{1}{c}{Recommended Funds} &  \multicolumn{1}{c}{Non-Recommended Funds} & \multicolumn{1}{c}{Difference} \\
    \midrule
    All Investors & 0.175 & 0.098 & 0.076 \\
    Financial Sophistication: Low & 0.186 & 0.081 & 0.105*** \\
    Financial Sophistication: Medium & 0.115 & 0.114 & 0.001 \\
    Financial Sophistication: High & 0.197 & 0.187 & 0.010 \\
    \midrule
    \multicolumn{4}{p{45em}}{\scriptsize Notes: {***} $p < 0.01$, {**} $p < 0.05$, {*} $p < 0.1$.  The significance level is based on an independent sample t-test.} \\
    \end{tabular}
  \label{tab:fundReturnCompareGroup}%
\end{table}%


\subsubsection{\label{fund reserversal}Fund Future Performance Reversal}
As our study only focuses on funds that just reach and just miss the recommendation cutoff, the future performance of recommended funds might not be any better than that of non-recommended ones because their annual returns are so close on the fund purchase date. To test this conjecture, we investigate the relationship between current annual returns and future annual returns for selected recommended and non-recommended funds. Specifically, we calculate the average value of a fund's current annual return and its annual return three months later, using funds that rank between fourth and seventh place. The three-month future return measure resonates with investors' average holding period in our data sample, namely, 3.3 months. 

As shown in Table~\ref{tab:fundFutureReturn}, recommended funds (funds in the fourth and fifth places) have a larger annual return than non-recommended funds (funds in the sixth and seventh places) on the day of being recommended. However, such a gap dramatically declines after three months. From another perspective, on average, investors who purchase recommended funds achieve a worse return after three months than those who purchased non-recommended funds (-0.192 vs -0.153; difference $=-0.039; p<0.01$). Price reversals are also observed in financial markets, due to market overreaction and the herding effect \citep{de1985does, atkins1990price, huang2010return}. Thus, besides investor behavior differences, the performance difference between recommended funds and non-recommended funds could also be one potential explanation for why investors perform worse when purchasing recommended funds. 

\begin{table}[ht!]
  \footnotesize
  \caption{Relation Between Current and Future Annual Return}
  \centering
    \begin{tabular}{lccc}
    \midrule
    Variable & CurrentAnnualReturn & ThreeMonthAnnualReturn  & Gain\\
    \midrule
    Recommended Funds  & 0.522 & 0.330 & -0.192*** \\
    Non-Recommended Funds & 0.463 & 0.310  & -0.153*** \\
    Difference & 0.059*** & 0.020** & -0.039*** \\
    \midrule
    \multicolumn{4}{p{40em}}{\scriptsize Notes: {***} $p < 0.01$, {**} $p < 0.05$, {*} $p < 0.1$. CurrentAnnualReturn is the current annual return of the fund. ThreeMonthAnnualReturn is the annual return of the fund after three months. Gain is the difference between ThreeMonthAnnualReturn and CurrentAnnualReturn. The significance level is based on Paired t-test.} 
    \\
    \end{tabular}
    \vspace{1em}
  \label{tab:fundFutureReturn}%
\end{table}%


\section{\label{conclusion}Conclusion}
As the number of retail investors trading on online investment platforms continues to soar, their increasing influences are propelling shifts in financial markets. These investors rely on online investment platforms, which intentionally provide e-commerce features like recommender systems. Whereas recommender systems can provide reliable quality signals for consumer goods, using past ratings and sales, their quality signaling appears inaccurate for financial products when they rely on historical returns. To clarify this difference, as well as address some gaps in the extant literature, we examine the effect of recommender systems in the context of financial products.

We first show that recommender systems influence investor behaviors, even though they are not a reliable quality signal from a rational investor's perspective. In particular, recommender systems can increase sales for recommended products. To gain further insights, we explore the heterogeneous effects of recommender systems across investor groups. These results indicate that the recommendation effect is more salient for investors with low-level financial sophistication, who have low education and income levels. In addition, investors with low-level financial sophistication tend to experience significantly worse returns after purchasing recommended funds. This outcome does not arise for investors with high-level financial sophistication. We can explain the negative effect on investment returns; as our results reveal, recommended funds do not perform as well as non-recommended funds. Furthermore, these investors expend more decision effort to select funds before purchasing non-recommended funds and time their redemptions much better when redeeming non-recommended funds. Thus, recommender systems appear to push unsophisticated investors to become more behaviorally biased.

Our article thus extends the information systems literature on the economic effects of recommender systems by providing detailed evidence of how they might amplify investors' irrational behaviors and further intensify economic inequality. Investors who represent the lowest level of financial sophistication follow recommender systems most closely when making investment decisions, so they suffer the largest investment losses. We also study how recommender systems function in financial markets, rather than in relation to consumer goods. The different quality signals for consumer goods versus financial products make it possible for us to identify the quality signal as an essential mechanism that defines the influence of recommender systems on online investor behaviors. As a contribution to behavioral finance literature, we investigate more fine-grained, investor-level mutual fund investment behaviors, unlike prior work that has been limited to using aggregate mutual fund inflows and outflows. This study is among the first to examine how the features of online investment platforms actually influence investors' behaviors. 

Accordingly, the findings have real-world implications. First, we reveal that investors rely on recommender systems to make investment decisions even though the recommendation based on historical returns cannot offer reliable quality signals. Our results also show that they have worse performance when purchasing recommended funds. Therefore, financial platforms should clearly inform investors that historical returns cannot predict future returns and such recommendation is only for reference. Platforms should provide more educational resources to help these investors make better investment judgments by emphasizing other information beyond historical returns such as fund portfolio and risk rating. Second, recommender systems tend to have the strongest effect on unsophisticated investors, who rely on them the most and tend to have the worst performance when purchasing recommended funds. These findings suggest that recommender systems in the financial platform could intensify wealth inequality among investors as unsophisticated investors who are supposed to be protected are instead hurt the most because of the existence of recommender systems. Meanwhile, recommender systems using historical returns could amplify investors' irrational behaviors as our findings reveal that investors spend less effort on researching recommended funds and have a worse time redeeming the recommended funds. This can potentially create systematic risks in the financial market, where many investors make similar investment decisions by following recommendations and are simultaneously affected by losses that then propagate throughout the financial system. To avoid such detrimental outcomes for investors and potentially the larger financial market, investment platforms, along with policymakers, should develop effective strategies to monitor the deployment of recommender systems as well as suggest best practices to support investors' decision-making process. 

In this early effort to identify and specify the effects of recommender systems in a financial market context, we focus on mutual funds. We hope continued research will explore other popular financial products, such as futures, options, and insurance. The recommender systems included in our research rely on a ranking-based algorithm, similar to content-based recommendations, which is more common and widely adopted. More sophisticated recommender systems, using state-of-the-art and AI-based algorithms, might have different effects on online investor behaviors and thus require further research consideration too.


\bibliographystyle{informs2014}
\bibliography{reference}

\newpage
\begin{appendices}

\section{\label{A} Investor Characteristics}
\setcounter{table}{0}
\renewcommand{\thetable}{A\arabic{table}}

% Summary Statistics of Investor Characteristics
\begin{table}[ht!]
\footnotesize
 \caption{Summary Statistics of Investor Characteristics}
  \centering
    \begin{tabular}{ccccc}
    \midrule
    \multicolumn{1}{c}{Characteristics} & \multicolumn{1}{c}{Description}  & \multicolumn{1}{c}{\# of Investors} & \multicolumn{1}{c}{Value} & \multicolumn{1}{c}{Percentage} \\
    \midrule
    \multirow{3}[1]{*}{Income} & Not Stable & 7,731 & 0 & 77.30\% \\
          & Stable & 1,559 & 1 & 15.60\% \\
          & Stable and High & 710 & 2  & 7.10\% \\
    \midrule
    \multirow{5}[0]{*}{Liabilities} & Borrowed From Friends & 7,731 & 0 & 77.30\% \\
          & Financial Distress & 191 & 0  & 1.90\% \\
          & Long-term Debt & 573 & 1  & 5.70\% \\
          & Short-Term Debt & 795 & 2  & 8.00\% \\
          & Positive Cash flow & 710 & 3  & 7.10\% \\
    \midrule
    \multirow{4}[0]{*}{Education} & High School or Below & 8,006 & 0 & 80.10\% \\
          & Junior College & 546 & 1  & 5.50\% \\
          & Bachelor & 1,076 & 2 & 10.80\% \\
          & Graduate Degree & 372 & 3 & 3.70\% \\
    \midrule
    \multirow{4}[0]{*}{Risk Tolerance} & Extremely Risk Averse & 7,928 & 0 & 79.30\% \\
          & Risk Averse & 580 & 1  & 5.80\% \\
          & Take Certain Risk & 1,327 & 2 & 13.30\% \\
          & Risk Taking & 165 & 3 & 1.70\% \\
    \midrule
    \multirow{3}[1]{*}{Investment Goals} & Keep Capital Safe & 8,508 & 0 & 85.08\% \\
          & Make Some Return & 1,327 & 1 & 13.27\% \\
          & Make Large Return & 165 & 2 & 1.65\% \\
    \midrule
    \multicolumn{5}{p{36em}}{\scriptsize Notes: Investor characteristics are encoded into categorical values for investor segmentation. They are ordinal variables where higher values represent higher income, education, risk tolerance, more risk-seeking in investment goals, and fewer liabilities.} \\
    \end{tabular}
  \label{tab:investorSummary}
\end{table}

\newpage
\section{\label{B} Robustness Checks}
\setcounter{table}{0}
\renewcommand{\thetable}{B\arabic{table}}
\begin{table}[ht!]
\caption{Robustness Check: Falsification Exercise \& Alternative Bandwidth}
\footnotesize
  \centering
\begin{tabular}{lccc}
    \midrule
    \multicolumn{1}{l}{Dependent Variable:} & \multicolumn{1}{c}{(1)} & \multicolumn{1}{c}{(2)} & \multicolumn{1}{c}{(3)} \\
    \multicolumn{1}{l}{log(NumberOfPurchases)} & \multicolumn{1}{c}{} & \multicolumn{1}{c}{}  & \multicolumn{1}{c}{}\\
    \midrule
    \multicolumn{1}{l}{Recommended} & \multicolumn{1}{c}{-0.007} & \multicolumn{1}{c}{0.011} & \multicolumn{1}{c}{0.028***} \\
    \multicolumn{1}{l}{} & \multicolumn{1}{c}{(0.009)} & \multicolumn{1}{c}{(0.008)} & \multicolumn{1}{c}{(0.009)} \\
    \multicolumn{1}{l}{Day Dummies * Fund Type Dummies} & \multicolumn{1}{c}{Yes} & \multicolumn{1}{c}{Yes} & \multicolumn{1}{c}{Yes} \\
    \multicolumn{1}{l}{Control Variables} & \multicolumn{1}{c}{Yes} & \multicolumn{1}{c}{Yes} & \multicolumn{1}{c}{Yes} \\
    \multicolumn{1}{l}{Number of Observations} & \multicolumn{1}{c}{7,278} & \multicolumn{1}{c}{7,278} & \multicolumn{1}{c}{7,278} \\
    \multicolumn{1}{l}{\textcolor[rgb]{ .133,  .133,  .133}{$R^2$}} & \multicolumn{1}{c}{0.621} & \multicolumn{1}{c}{0.593} & \multicolumn{1}{c}{0.583} \\
    \midrule
    \multicolumn{4}{p{30em}}{\scriptsize Notes: {***} $p<0.01$, {**} $p<0.05$, {*} $p<0.1$. Control variables include fund fees, fund size, and fund establish time. Columns (1) and (2) show the results of the falsification exercise. Column (1) compares funds in fourth  and fifth places (both are recommended). Column (2) compares funds in the sixth and seventh places (both are non-recommended). Column (3) uses a smaller bandwidth that includes only funds in the fifth (recommended) and sixth (non-recommended) places in the sample. Robust standard errors are in parentheses.}\\
    \end{tabular}
  \label{tab:falsificationAlternateBandwidth}
\end{table}


% Robustness Check : Polymonial Degrees Check
\begin{table}[ht!]
\caption{Robustness Check: Alternative Polynomial Functions}
\footnotesize
  \centering
\begin{tabular}{lccc}
    \midrule
    \multicolumn{1}{l}{Dependent Variable:} & \multicolumn{1}{c}{(1)} & \multicolumn{1}{c}{(2)} & \multicolumn{1}{c}{(3)} \\
    \multicolumn{1}{l}{log(NumberOfPurchases)} & \multicolumn{1}{c}{} & \multicolumn{1}{c}{} & \multicolumn{1}{c}{} \\
    \midrule
    \multicolumn{1}{l}{Recommended} & \multicolumn{1}{c}{0.027***} & \multicolumn{1}{c}{0.032***} & \multicolumn{1}{c}{0.020***} \\
    \multicolumn{1}{l}{} & \multicolumn{1}{c}{(0.006)} & \multicolumn{1}{c}{(0.006)} & \multicolumn{1}{c}{(0.007)} \\
    \multicolumn{1}{l}{Polynomial Degree} & \multicolumn{1}{c}{1} & \multicolumn{1}{c}{2} & \multicolumn{1}{c}{3} \\
    \multicolumn{1}{l}{Day Dummies * Fund Type Dummies} & \multicolumn{1}{c}{Yes} & \multicolumn{1}{c}{Yes} & \multicolumn{1}{c}{Yes} \\
    \multicolumn{1}{l}{Control Variables} & \multicolumn{1}{c}{Yes} & \multicolumn{1}{c}{Yes} & \multicolumn{1}{c}{Yes} \\
    \multicolumn{1}{l}{Number of Observations} & \multicolumn{1}{c}{14,556} & \multicolumn{1}{c}{14,556} & \multicolumn{1}{c}{14,556} \\
    \multicolumn{1}{l}{\textcolor[rgb]{ .133,  .133,  .133}{$R^2$}} & \multicolumn{1}{c}{0.385} & \multicolumn{1}{c}{0.386} & \multicolumn{1}{c}{0.387} \\
    \midrule
    \multicolumn{4}{p{31em}}{\scriptsize Notes: {***} $p<0.01$, {**} $p<0.05$, {*} $p<0.1$. Control variables include fund fees, fund size, and fund establish time. Columns (1) and (2) include polynomials of annual fund returns up to the degree of 1 and 2 respectively. Column (3) shows results with the polynomial function allowing for a separate relationship between fund annual return and fund purchases for recommended and non-recommended funds.  Robust standard errors are in parentheses.}\\
    \end{tabular}
  \label{tab:differentPolynomials}
\end{table}

% Robustness Check : Alternative Measurements
\begin{table}[ht!]
\footnotesize
\caption{Robustness Check: Alternative Measurements \& Lagged Effect}
  \centering
    \begin{tabular}{p{8em}llll}
    \midrule
    \multicolumn{1}{l}{} & \multicolumn{1}{c}{(1)} & \multicolumn{1}{c}{(2)} & \multicolumn{1}{c}{(3)} \\
    \multicolumn{1}{l}{} & \multicolumn{3}{c}{}\\
    \midrule
    \multicolumn{1}{l}{Recommended} & \multicolumn{1}{c}{0.033***} & \multicolumn{1}{c}{0.100***} & \multicolumn{1}{c}{0.020***} \\
    \multicolumn{1}{r}{} & \multicolumn{1}{c}{(0.006)} & \multicolumn{1}{c}{(0.020)} & \multicolumn{1}{c}{(0.006)} \\
    \multicolumn{1}{l}{Day Dummies * Fund Type Dummies} & \multicolumn{1}{c}{Yes} & \multicolumn{1}{c}{Yes} & \multicolumn{1}{c}{Yes} \\
    \multicolumn{1}{l}{Control Variables} & \multicolumn{1}{c}{Yes} & \multicolumn{1}{c}{Yes} &  \multicolumn{1}{c}{Yes} \\
    \multicolumn{1}{l}{Number of Observations} & \multicolumn{1}{c}{14,556} & \multicolumn{1}{c}{14,556} & \multicolumn{1}{c}{15,983} \\
    \multicolumn{1}{l}{\textcolor[rgb]{ .133,  .133,  .133}{$R^2$}} & \multicolumn{1}{c}{0.391} & \multicolumn{1}{c}{0.380} & \multicolumn{1}{c}{0.360} \\
    \midrule
    \multicolumn{4}{p{32em}}{\scriptsize Notes: {***} $p<0.01$, {**} $p<0.05$, {*} $p<0.1$. Control variables include fund fees, fund size, and fund establish time. The dependent variable in Column (1) is log(NumberOfInvestors) and it is log(PurchaseAmount) in Column (2). Columns (1) and (2) use alternative measurements to operationalize the effect of recommendation on purchase behavior. The dependent variable in Column (3) is log(NumberOfPurchases). $Recommended$ in Column (3) is defined as being recommended in the past three days. Robust standard errors are in parentheses.} \\
    \end{tabular}
  \label{tab:altDV}
\end{table}

% Robustness Check : Alternative Model Specification and Sub-sample Analysis
\begin{table}[ht!]
\footnotesize
\caption{Robustness Check: Alternative Model Specification and Sub-sample Analysis}
  \centering
    \begin{tabular}{lcc}
    \midrule
    \multicolumn{1}{l}{} & \multicolumn{1}{c}{(1)} & \multicolumn{1}{c}{(2)} \\
    \multicolumn{1}{l}{} & \multicolumn{1}{c}{} & \multicolumn{1}{c}{} \\
    \midrule
    \multicolumn{1}{l}{Recommended} & \multicolumn{1}{c}{0.196**} & \multicolumn{1}{c}{0.013**} \\
    \multicolumn{1}{r}{} & \multicolumn{1}{c}{(0.079)} & \multicolumn{1}{c}{(0.006)} \\
     \multicolumn{1}{l}{Day Dummies} & \multicolumn{1}{c}{No}  & \multicolumn{1}{c}{Yes} \\
      \multicolumn{1}{l}{Fund Dummies} & \multicolumn{1}{c}{No}  & \multicolumn{1}{c}{Yes} \\
    \multicolumn{1}{l}{Control Variables} & \multicolumn{1}{c}{Yes}  & \multicolumn{1}{c}{Yes}\\
    \multicolumn{1}{l}{Number of Observations} & \multicolumn{1}{c}{14,556} & \multicolumn{1}{c}{13,722} \\
    \multicolumn{1}{l}{\textcolor[rgb]{ .133,  .133,  .133}{$R^2$}} & \multicolumn{1}{c}{0.175} & \multicolumn{1}{c}{0.557} \\
    \midrule
    \multicolumn{3}{p{30em}}{\scriptsize Notes: {***} $p<0.01$, {**} $p<0.05$, {*} $p<0.1$. Control variables include fund fees, fund size, and fund establish time. NumberOfPurchases is the dependent variable in Column (1). log(NumberOfPurchases) is the dependent variable in Column (2). Column (1) displays the result using a Poisson regression model. Column (2) shows the result of the sub-sample that includes only funds that have appeared as both recommended and non-recommended funds during the sample period. Robust standard errors in parentheses in Column (1). Robust standard errors are in parentheses.} \\
    \end{tabular}
  \label{tab:altModelSubsample}
\end{table}

\end{appendices} 



%%%%%%%%%%%%%%%%%
\end{document}
%%%%%%%%%%%%%%%%%
