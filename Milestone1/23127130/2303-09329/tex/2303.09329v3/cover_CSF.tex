\documentclass[11pt]{article}


\usepackage[pdftex]{color}
\usepackage{amsmath}
\usepackage{amssymb}
\usepackage{amsthm}
\usepackage{epsfig} 
\usepackage{graphics} 
\usepackage{setspace}
%\usepackage{times}


\setlength{\topmargin}{-1.8cm}
\setlength{\textheight}{25.5cm}
\setlength{\oddsidemargin}{0cm}
\setlength{\textwidth}{16.5cm}
%\setlength{\columnsep}{0.6cm}

\thispagestyle{empty}

\begin{document}

\begin{flushleft}
$\begin{array}{l}

\textrm{Department of Mathematics}\\

\textrm{State University of New York at New Paltz}\\

\textrm{1 Hawk Drive, New Paltz, NY 12561}\\

\textrm{phone: (845) 257-3532, \: fax: (845) 257-3571}\\

\end{array}$
\end{flushleft}


\vspace{.5cm}
\noindent Dear Editorial Office,

\vspace{.5cm}

\noindent We would like to submit our paper entitled \emph{Asymptotic dynamics in systems of two coupled quadratic maps} for publication in \emph{Chaos, Solitons and Fractals}. In this work, we return to study the dynamics of two coupled nodes with complex quadratic dynamics, as a foundational case for contextualizing prior results in higher-dimensional complex quadratic networks.\\

\noindent  The two-node case acts as a fundamental testing ground: its analytical simplicity allows for the clear identification of key behaviors and their relationship to coupling, while shedding light on broader underlying principles. In particular, we examine the structure and limitations of the equi-M set (the parameter region with bounded postcritical dynamics). We describe the relationship between the equi-M set and the parameter domains where the critical orbit converges to periodic attractors (pseudo-bulbs). Through a combination of analytical and numerical methods, we show that even minimal coupling introduces significant complexity—such as the breakdown of classical one-dimensional correspondence between critical orbits and global dynamics, and the emergence of nontrivial synchronization thresholds. Our results provide foundational insight into how local node interactions shape global behavior in low-dimensional networks, laying the groundwork for understanding more complex systems.\\


\noindent We believe that our paper is well suited for publication in \emph{Chaos, Solitons and Fractals}.  Both the nature of the fractal objects we study (equi-M sets), as well as the potential applications of this framework to modeling of natural networks (as shown in prior work) are expected to be of interest for the audience of this journal. Our manuscript has not been published, and is not under consideration elsewhere.  A related manuscript had been previously placed on the preprint archive (arXiv:2303.09329). We would like to kindly ask that our submission be handled by Dr. Adilson Motter. For suggestions of potential reviewers, please consider the list included below.


\vspace{1cm}


\noindent Thank you for your consideration,

\vspace{0.1cm}

\noindent The authors

\clearpage
\subsection*{Reviewer suggestions}

\hspace{6mm}{\bf Leonid Rubchinsky}

Department of Mathematical Sciences

Indiana University Purdue University Indianapolis

Email: \emph{lrubchin@iupui.edu}
\\

{\bf Scott Kaschner}

Department of Mathematics

Butler University

Email: \emph{skaschne@butler.edu}
\\

{\bf Bruce Peckham}

Department of Mathematics

University of Minnesota Duluth

Email: \emph{bpeckham@d.umn.edu}
\\

{\bf Christopher Staniszewski}

Department of Mathematics

Framingham State University

Email: \emph{cstaniszewski@framingham.edu}
\\




\end{document}
