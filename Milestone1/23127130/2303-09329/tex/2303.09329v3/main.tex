\documentclass[10pt]{article}
\usepackage{iftex}
\usepackage{xparse}

% {{{ packages

% enable support for unicode fonts
\ifpdftex
    \usepackage[T1]{fontenc}
\else
    \usepackage{fontspec}
\fi

\usepackage{geometry}

% math
\usepackage{amsmath}
\usepackage{amssymb}
\usepackage{amsthm}
\usepackage{xfrac}

% colors
\usepackage{xcolor}
\usepackage{graphicx}
\usepackage[colorlinks=true, citecolor=blue]{hyperref}

% misc
\usepackage{wrapfig}
\usepackage{enumitem}

% }}}

% {{{ formatting

% TODO: remove this when switching to a journal template
\geometry{top=2.5cm, left=2.5cm, right=2.5cm, bottom=4cm}
\setlength{\parindent}{0pt}
\setlength{\parskip}{0.5\baselineskip}

% TODO: remove these if the journal already defines them
\newtheorem{thm}{Theorem}[section]
\newtheorem{prop}[thm]{Proposition}
\newtheorem{lemma}[thm]{Lemma}
\newtheorem{corol}[thm]{Corollary}
\newtheorem{defn}[thm]{Definition}
\newtheorem{conj}[thm]{Conjecture}
\newtheorem{example}[thm]{Example}

% }}}

% {{{ commands

\newcommand{\ds}{\displaystyle}
\newcommand{\ben}{\begin{equation}}     %equation
\newcommand{\eeqn}{\end{equation}}
\newcommand{\bey}{\begin{eqnarray}}
\newcommand{\eey}{\end{eqnarray}}

\newcommand{\nno}{\nonumber \\} % no number at end of equation

% NOTE: use these for adding notes and whatnot! They can be turned off completely
% by setting \reviewfalse below, without the need to hunt them down through the text.
\newif\ifreview
\reviewtrue
\definecolor{NoteRed}{HTML}{DC143C}
\definecolor{NoteBlue}{HTML}{007BA7}
\ifreview
    \usepackage[textwidth=1.25\linewidth]{todonotes}

    \newcounter{todocounter}
    \NewDocumentCommand \GenericNote { m m m m } {\stepcounter{todocounter} \todo[
        #1,
        size=\normalsize,
        backgroundcolor=white,
        bordercolor=NoteBlue,
        linecolor=NoteBlue,
        textcolor=#2]{\textbf{[\thetodocounter] @#3}: #4}}

    \NewDocumentCommand \Note { m m } {\GenericNote{inline}{NoteBlue}{#1}{#2}}
    \NewDocumentCommand \Todo { m } {\textbf{\textcolor{NoteRed}{§#1§}}}
\else
    \NewDocumentCommand \Note { m m } {}
    \NewDocumentCommand \Todo { m } {}
\fi

% }}}

\begin{document}

% {{{ metadata

\begin{flushleft}
{\Large
\textbf{Asymptotic dynamics in systems of two coupled quadratic maps}
}
\\
\vspace{4mm}
Anca R\u{a}dulescu$^{*,}$\footnote{Assistant Professor, Department of Mathematics, State University of New York at New Paltz; New York, USA; Phone: (845) 257-3532; Email: radulesa@newpaltz.edu}, Eva Kaslik$^{2,3}$, Alexandru Fikl$^3$\\

\indent $^1$ Department of Mathematics, SUNY New Paltz, USA

\indent $^2$ Department of Computer Science, West University of Timi\c{s}oara, Romania

\indent $^3$ Institute for Advanced Environmental Research, West University of Timisoara, Romania
\end{flushleft}

% }}}

\begin{abstract}
This paper examines the structure and limitations of equi-M sets in two-dimensional Complex Quadratic Networks (CQNs). In particular, we aim to describe the relationship between the equi-M set and the parameter domains where the critical orbit converges to periodic attractors (pseudo-bulbs). The two-node case serves as a foundational testbed: its analytical tractability enables the identification of critical phenomena and their dependence on coupling, while offering insight into more general principles. The two-node case is also simple enough to allow for explicit coupling conditions that govern phase transitions between synchronized and desynchronized behavior.

Using a combination of analytical and numerical methods, the study reveals that while the period-1 pseudo-bulb closely tracks the boundary of the equi-M set near its main cusp, this correspondence breaks down for higher periods and in regions supporting coexisting attractors. These discrepancies highlight key differences between single-map and coupled dynamics, where equi-M sets no longer provide a full encoding of system combinatorics. These findings clarify the topological and dynamical behavior of low-dimensional CQNs and point toward a sharp increase in complexity as the number of nodes grows, laying the groundwork for future studies of high-dimensional dynamics.
\end{abstract}

%\Note{Alex}{\url{https://doi.org/10.1142/S0129167X18500477} (and related) seems like an interesting read. It's about Hénon maps in $\mathbb{C}^2$.. talks about Julia sets and the like.}

\section{Introduction}

In our previous work~\cite{Ariel,Simone,gender,AC2,DTI}, we introduced complex quadratic networks (CQNs) as means to study the evolution of high-dimensional coupled systems. Their simplified dynamics are described by discrete quadratic iterations in the complex plane that are coupled through a real weight matrix. As mathematical objects, CQNs exhibit a much wider range of behaviors compared to standard univariate quadratic maps. These behaviors are governed by the interplay between the properties of individual nodes, the strength of the coupling between them and the topology of the network. For instance, studying the effect of edge placement and weight strength is of crucial interest in the context of understanding and classifying the long-term behavior of the system~\cite{iniguez2020bridging,seabrook2021evaluating,faskowitz2022edges}. Existing results for CQNs have revealed that even simple changes to the edge weights can lead to large-scale changes in the system's dynamics. This has significant implications not only for network science, but also for other fields such as physics~\cite{robinson2022physics}, biology, and engineering, where understanding the dynamics of complex systems is a critical component. 

However, many questions in network sciences are extremely difficult to address in a realistic model of a natural system, and are deemed analytically, and often also computationally, intractable. The aim of our is to develop a canonical modeling framework for quantifying the impact of hard-wiring on the emerging network behavior. Rephrasing such a question around coupled dynamics of complex iterated maps presents a mathematically tractable and canonical approach that may lead to ground-breaking results, in terms of both mathematical novelty and understanding of biological systems. 


\subsection{Combinatorics in CQNs} 

More specifically, we study the properties of a general network model in which the nodes are complex quadratic maps with coupling specified by the weighted adjacency matrix $A \in \mathbb{R}^{d \times d}$~\cite{AC2,Simone,Ariel}. The overall evolution of this discrete system takes the form of an iteration in $\mathbb{C}^d$, given component-wise by
\[
z_k(n + 1) = \left(\sum_{j = 1}^d A_{kj} z_j(n)\right)^2 + c_k.
\]

Over the decades, the study of the dynamics of discrete iterated maps has developed into a rich research field~\cite{carleson1996complex}, with univariate quadratic polynomials taking a central position~\cite{mcmullen1994complex}. For these systems, real quadratic iterations have first been associated with cascades of period-doubling bifurcations and universality. On the other hand, complex quadratic iterations in the family $f_c(z) = z^2 + c$, with $c \in \mathbb{C}$, provide textbook recipes for creating fractal asymptotic sets. Together, they represent a simple, yet incredibly rich, family of maps and have seen many practical applications.

For univariate quadratic maps, one of the most striking results is that the orbit of the critical point captures global information on all other accessible orbits. Therefore, we can focus on the parameters $c \in \mathbb{C}$ for which the critical orbit of $f_c$ is bounded, also known as the Mandelbrot set. It is known that for parameters $c$ in the Mandelbrot set, the prisoner set of $f_c$ (i.e., the set of points with bounded orbits under $f_c$) forms a connected set. If $c$ is not in the Mandelbrot set, then the prisoner set is totally disconnected. This result is known as the Fatou--Julia Theorem~\cite{milnor2011dynamics}. Furthermore, the topological properties of the Mandelbrot set in the parameter plane classify all possible combinatoric behaviors that orbits can have via a ``hyperbolic bulb'' structure. Specifically, the main cardioid of the Mandelbrot set bounds the set of points whose orbits converge to an attracting fixed point~\cite{brucks2004topics}. The remaining ``hyperbolic bulbs'' correspond to all other higher-order periodic attractor combinatorics~\cite{ivancevic2007high}. %The possibility of other ``ghost components'' \Todo{briefly explain what this is} remains an open question~\cite{douady1984exploring,jung2002homeomorphisms}. 
Establishing  connections between quadratic dynamics and the topology of the Mandelbrot set has been highly nontrivial. 
While many interesting properties have been established (such as connectedness~\cite{douady1984etude,douady1985dynamics} and full Hausdorff dimension~\cite{shishikura1998hausdorff}), there are open problems that persist (such as local connectedness~\cite{hubbard1992local} and density of hyperbolicity~\cite{douady1984exploring,jung2002homeomorphisms}). Even so, the Mandelbrot set remains a staple in discrete dynamical systems and a canonical representation for more general phenomena.

In our work on CQNs, we track asymptotic behavior of multi-dimensional orbits, and use it to quantify how systemic dynamics emerge under different specifications for network architecture. To do this, we defined for any given network with identical nodes $c$  the extension of the Mandelbrot set as the postcritically bounded locus for $c$ (and called it the equi-M set of the network). We noted that equi-M sets are combinatorially more complex than the traditional Mandelbrot set, due to the interactions between the multiple nodes of the network. In particular, since the system is no longer generated by a single unimodal map, we can no longer expect that the critical orbit encapsulates the behavior of all orbits of the system. In our previous work~\cite{Ariel,Simone,AC2,DTI}, we explored post-critical dynamics for CQNs
using the equi-M set of a network, as well as node-wise versions of it. 

Our observations suggested that the equivalence of the post-critically bounded and connectedness loci does not hold in the same form for CQNs as it did for univariate iterated maps. For one-dimensional systems, this property allows the Mandelbrot set to provide an atlas of all possible periodic dynamics of iterated maps in the complex quadratic family (aside form the hypothetical ``ghost components''). In our preliminary work~\cite{Simone}, we suggest potential extensions to capture the fact that the critically bounded locus for CQNs is only related in a weaker sense to the connectedness locus in $\mathbb{C}^d$. Here, we revisit the question and show the mechanisms that shape this weaker version the of the Fatou--Julia relationship through the underlying combinatorics of CQNs. For example, we expect that there may be CQNs which have periodic attractors, but for which the critical orbit escapes, breaking one of the most basic theorems for the dynamic family defined by univariate quadratic iterated maps. %Because of this, equi-M sets cannot be written as disjoint unions of hyperbolic bulbs. 

\subsection{Dynamics of two coupled maps}

In this paper, we focus on setting the groundwork for a few questions that go along these lines. One consists of finding an effective atlas that would extend the idea of hyperbolic components as a combinatorial bookkeeping method. This organization would have to go beyond the traditional partition in hyperbolic bulbs, and would have to allow for more complex possibilities, such as parametric regions in which multiple period attractors coexist. We study to what extent the equi-M set offers a roughly approximate (but computationally more inexpensive) representation of this atlas, and which features of this set remain useful for characterizing and classifying the asymptotic behavior of the system. Finally, we want to understand how behavior differs between individual nodes, and how the different node-wise components of the critical orbit contribute to shaping the intersection (which is the equi-M set).


We focus our study on systems consisting of two coupled nodes, as a first step to investigate the complexity of network combinatorics and the relationship between the combinatoric atlas and the shape of the equi-M set. We focus on two aspects of the equi-M set: topology and synchronization. First, we use analytical and computational tools to identify low-period attractors, and sketch the skeleton of the combinatoric atlas for different types of two-dimensional systems. Then, we compare the geometry of these structures with that of the corresponding equi-M sets, generated numerically. Second, we examine equi-M sets from the perspective of node-wide contributions, identifying the factors that make nodes ``synchronize'', i.e. exhibit the same parameter locus for which their individual components of the critical orbit are bounded. In our previous work~\cite{Simone,AC2}, we determined that network nodes may cluster into groups that have the same node-wise equi-M set, based on the architectural profile of the network. However, the contribution of specific hard-wired factors to this synchronization profile remained unclear in large oriented networks. The study of two-dimensional coupled systems will allow us to identify possible basic mechanisms for synchronization in this simple case, that can be further extended to more general systems.

In our previous work, we have approached CQNs from a network science standpoint~\cite{AC2,gender}, and have focused primarily on the contribution of the network architectural aspect to the emerging dynamics. In particular, we worked to identify the contribution of the coupling to the topological properties of the equi-M sets, and to the grouping and shapes of the synchronization clusters. However, in order to have nontrivial architecture, the network has to be reasonably large. This leap to access higher-dimensional, interesting ``connectivity patterns'' omitted a specific analysis of two-dimensional networks, which can only have virtually trivial architecture schemes between two nodes. This case is nonetheless extremely important to our analytical understanding of how coupling affects the equi-M sets and their synchronization. In this paper, we focus precisely on this aspect and embrace a bottom-up, constructive approach. In future work we will build upon results on interaction between two nodes taken in isolation, to determine how mutual coupling acts when it is more generally embedded into higher-dimensional networks. 

The paper is organized as follows: In Section~\ref{combinatorics}, we introduce the general concepts and notation for systems of two coupled maps, including the definition of equi-M set in this case. In Subsection~\ref{escape_rad} also establish the existence of escape radius for two coupled maps (a crucial piece supporting reliability of numerical simulations based on direct iterations of the system). In Subsection~\ref{cardioid} we define and describe the ``main equi-cardioid'' of the equi-M set in this context, in analytic and numerical terms, in general and for a few example families. Regions with higher-order period attractors are computed numerically and illustrated in Subsection~\ref{higher_order}. Section~\ref{synchronization} describes synchronization in coupled systems and analyses phase transitions between clustering and de-clustering of the two nodes. The results are then reviewed and contextualized in the Discussion section.

\section{Combinatorics of coupled maps}
\label{combinatorics}

We consider a system of two coupled complex variables $(z_1, z_2)$ that are evolved discretely according to quadratic complex dynamics with linear coupling specified by a matrix $A$. The two-dimensional map describing the coupled dynamics is given by:
\begin{equation} \label{mother_sys}
\begin{cases}
z_1(n+1) = [a z_1(n) + b z_2(n)]^2 + c, \\
z_2(n+1) = [d z_1(n) + f z_2(n)]^2 + c,
\end{cases}
\qquad \text{where} \qquad
A =
\begin{bmatrix}
a & b \\
d & f
\end{bmatrix}
\in \mathbb{R}^{2 \times 2}
\end{equation}
and $c \in \mathbb{C}$. We refer to this system as a 2D-CQN to differentiate it from the the general $d$-dimensional case considered in~\cite{AC2,Simone}. For this family of maps, we consider the following definition of the equi-M set.

\begin{defn} \label{def:equi_m_set}
The equi-M set of 2D-CQN~\eqref{mother_sys} is defined as
\[
\mathcal{M}_{2} = \{
    c \in \mathbb{C} \mid
    |z_1(n)|^2 + |z_2(n)|^2 < \infty, \forall n \in \mathbb{N},
    \text{where } z_1(0) = z_2(0) = 0
\}.
\]
\end{defn}

As in the case of univariate quadratic maps, the equi-M set is defined as the subset of the parameter space where the critical orbits remain bounded in the $\ell_2$ norm. This represents only a slice of a higher-dimensional parameter object $(c_1,c_2)\in \mathbb{C}^2$, with structure that depends on the coupling between nodes. When discussing the geometry of the equi-M set and its relationship to the combinatorics of~\eqref{mother_sys}, we also define two extensions of traditional objects from the study of univariate iterated maps. For any given 2D CQN:

\begin{defn} \label{eq:pseudo_bulb}
The pseudo-bulb ${\cal B}_2^k$ of period $k \geq 1$ is the region of $c \in \mathbb{C}$ where the critical point converges to an attractor of period $k$.
The combinatorial region ${\cal C}_2^k$, for $k \geq 1$, is the region of $c\in \mathbb{C}$ for which the system has an attractor of period $k$. In particular, we will refer to ${\cal C}_2^1$ as the main equi-cardioid of the coupled system.
\end{defn}

While clearly ${\cal B}_2^k \subset {\cal C}_2^k$, we note that this inclusion may be strict. The critical orbit no longer encompasses the global fate of the system, as it did in single iterated maps, hence ${\cal B}_2^k$ and ${\cal C}_2^k$ are often distinct objects. In fact, we will show that combinatorial regions ${\cal C}_2^k$ are not necessarily disjoint, since periodic attractors may coexist in coupled systems, leading to a more complex combinatorial partition in the parameter plane. These properties and relationships will be further examined in the next sections.


\subsection{Escape radius}
\label{escape_rad}

In this section, we will show that 2D-CQN~\eqref{mother_sys} has an escape radius, under minimal assumptions on the entries of the connectivity matrix $A$. First, the following result shows that once a pair $(z_1, z_2)$ leaves the ball of radius $M$, its subsequent iteration will be outside of the ball of radius $2M$.
 
\begin{lemma} \label{case1}
Let the connectivity matrix $A$ be such that $\Delta = |a f| - |bd| \neq 0$. Then, there exists a large enough $M > 0$ and an $n \ge 0$, such that
\[
\max\{|z_1(n)|, |z_2(n)|\} > M \implies \max\{|z_1(n + 1)|, |z_2(n + 1)|\} > 2 M.
\]

This is true in particular for the critical orbit of the system.
\end{lemma}
 

\begin{proof}
Assuming that $\Delta = |af | - |bd | \neq 0$, we define
\begin{equation} \label{case1.m1m2}
M_1 = K_1 + \sqrt{K_1 (K_1 + |c|)}
\text{ and }
M_2 = K_2 + \sqrt{K_2(K_2 + |c|)},
\end{equation}
where
\[
K_1 = \left(\frac{|b| + |f|}{|\Delta|}\right)^2
\text{ and }
K_2 = \left(\frac{|a| + |d|}{|\Delta|}\right)^2.
\]


Let $M > \max\{M_1, M_2\}$. Suppose that, for some $n \geq 0$, we have $\lvert z_1(n) \rvert > M$ or $\lvert z_2(n) \rvert> M$. We will show that $\lvert z_1(n + 1) \rvert > 2 M$ or $\lvert z_2(n + 1) \rvert > 2 M$ using a standard proof by contradiction. To simplify the notation, we will abbreviate $z_1(n)$ and $z_2(n)$ to $z_1$ and $z_2$, when there is no danger of confusion. Suppose both $z_1(n + 1) \leq 2M$ and $z_2(n + 1) \leq 2M$. Then, starting from $|z_1(n + 1)| \le 2 M$, we have that
\[
2 M
\geq |z_1(n + 1)| = |(a z_1 + b z_2)^2 + c|
\geq |a z_1 + b z_2|^2 - |c|
\geq (|a z_1| - |b z_2|)^2 - |c|,
\]
by repeated use of the reverse triangle inequality. Applying the analogous steps to $|z_2(n + 1)| \le 2 M$, we obtain the following inequalities
\begin{equation} \label{case1.proof1}
    |\xi_1| = \big||a z_1| - |b z_2|\big| \leq \sqrt{2M + |c|}
    \quad \text{and} \quad
    |\xi_2| = \big||d z_1| - |f z_2|\big| \leq \sqrt{2M + |c|}.
\end{equation}

Then, we can write
\begin{equation*}
\big||f| \xi_1 - |b| \xi_2\big|
    = \big|(|a f| - |b d|) z_1\big|
    \leq |f| |\xi_1| + |b| |\xi_2|
    \leq (|b| + |f|) \sqrt{2M + |c|},
\end{equation*}
which can be expressed as a bound on $z_1$, under the assumption that $\Delta \ne 0$, i.e.
\begin{equation} \label{ineq_z1}
|z_1| \leq \frac{|b| + |f|}{|\Delta|} \sqrt{2 M + |c|} = \sqrt{K_1 (2 M + |c|)}.
\end{equation}

However, we have also assumed that $|z_1| > M$, so the right side of \eqref{ineq_z1} cannot be larger than $M$. Squaring both sides, this is equivalent to $h_1(M) = M^2 - K_1 (2 M +|c|) > 0$, which is guaranteed by taking $M > M_1$, the larger of the roots of $h_1$. In conclusion, we obtain that $|z_1(n)| \leq M$. Analogously, we obtain
\begin{equation*}
    z_2(n) \leq \sqrt{K_2 (2 M + |c|)} \leq M.
\end{equation*}

As before, the second part of the inequality follows from the condition that $M > M_2$, where $M_2$ is the larger of the roots of $h_2(M) = M^2 - K_2 (2 M + |c|)$. Since $z_1(n)$ and $z_2(n)$ cannot be simultaneously smaller than $M$, the contradiction follows.
\end{proof}

\begin{thm}
Let the connectivity matrix $A$ be such that $\Delta = |a f| - |bd| \neq 0$. Then $M = \max\{M_1, M_2\}$ is an escape radius for~\eqref{mother_sys}, where $M_1$ and $M_2$ are given by~\eqref{case1.m1m2}.
\end{thm}

As an extension of this result, suppose that $a f = b d$ (the opposite sign case is similar). We want to see if, in this case, there still exists a large enough $M > 0$ such that, for an iteration step $n \geq 0$, we have $|z_1(n)| > M$ or $|z_2(n)| > M$, then the next iterates $|z_1(n + 1)| > 2 M$ or $|z_2(n + 1)| > 2 M$. We analyze all the possible cases in the following.\\


\begin{description}
    \item[Case 1 $(a, b \neq 0)$.] Since $a f = b d$, we can define $\ds k = \frac{d}{a} = \frac{f}{b}$ and consider a fixed iteration $n$, for which $|z_1(n)| > M$ and $|z_2|  > M$. It follows that $|a z_1 + b z_2| > g M$ or $|a z_1 - b z_2| > g M$, for $g = \min\{|a|, |b|\}$. Indeed, assume by contradiction that both $|a z_1 + bz_2| \leq g M$ and $|a z_1 - b z_2| \leq g M$. Then
\begin{equation*}
2|az_1 | \leq |az_1 + bz_2 | + |az_1-bz_2 | \leq 2gM < 2 |a | M
\end{equation*}
implying that $z_1 \leq M$. Analogously,
\begin{equation*}
2|bz_1 | \leq |az_1 + bz_2 | + |bz_2- az_1 | \leq 2gM < 2 |b | M
\end{equation*}
implies that $z_2 \leq M$. Since we cannot simultaneously have $|z_1| \leq M$ and $|z_2| \leq M$, the initial assumption must be false. Then, if we assume that $|a z_1 + b z_2| > g M$, we have that
\[
\begin{aligned}
& a z_1(n + 1) + b z_2(n + 1) = (a + b k^2) (a z_1 + b z_2)^2 + c (a + b) \\
\implies & |a z_1(n + 1) + b z_2(n + 1)| \geq g^2 M^2 |a + b k^2| - |c (a + b)|.
\end{aligned}
\]

\begin{description}
    \item[Case (i).] If $a + b k^2 \neq 0$, we can consider the quadratic function $f(M) = g^2M^2 |a+bk^2| -2(|a | + |b |)M - |c(a+b) |$. The larger root $M_1 > 0$ of this function is defined explicitly as:
    \begin{equation*}
    M_3 = \frac{|a| + |b| + \sqrt{(|a| + |b|)^2 + g^2 |c| (|a| + |b|) |a + b k^2|}}{g^2 |a + b k^2|}.
    \end{equation*}

    If $M > M_3$, then $f(M) > 0$, and $|a z_1(n+1) + b z_2(n+1)| > 2(|a| + |b|)M$. It follows that either $|z_1(n + 1)| > 2M$ or $|z_2(n + 1)| > 2M$.
    \item[Case (ii).] If $a + b k^2 = 0$, then $a = -b k^2$, $d = -b k^3$ and $f = b k$. Therefore, the system becomes
    \[
    \begin{cases}
    z_1(n+1) = b^2 (-k^2 z_1 + z_2)^2 + c, \\
    z_2(n+1) = b^2 k^2 (-k^2 z_1 + z_2)^2 + c.
    \end{cases}
    \]

    We can then show that $-k^2 z_1(n) + z_2(n) = c (1 - k^2)$ is constant for $n \geq 1$. It follows that $z_1(n + 1)$ and $z_2(n + 1)$ are also constant for $n \geq 1$.
\end{description}

    \item[Case 2 $(a = 0, b \ne 0)$.] Then necessarily $d = 0$ and the system becomes
    \[
    \begin{cases}
    z_1(n+1) = (bz_2)^2 + c, \\
    z_2(n+1) = (fz_2)^2 + c.
    \end{cases}
    \]

    In this case, we can see that the second equation is decoupled. If $f \neq 0$, one can make the change of variables $\xi = f^2 z_2$, and rewrite the equation as $\xi(n+1) = \xi^2 + c_f$, where $c_f = f^2 c$. It follows from the traditional theory on escape radius for single quadratic maps that $M = \max\{2 / f^2, |c|\}$ is an escape radius for $z_2$. Hence the whole system has an escape radius, since $z_1$ only depends on $z_2$. If $f = 0$, the system is trivial as both $z_2$ and $z_1$ are constant.

    \item[Case 3 $(a \ne 0, b = 0)$.] Then necessarily $f = 0$ and the system becomes
    \[
    \begin{cases}
    z_1(n+1) = (az_1)^2+c, \\
    z_2(n+1)=(dz_1)^2+c.
    \end{cases}
    \]

    The proof follows similarly with Case 2.

    \item[Case 4 $(a = b = 0)$.] Then $z_1(n) = c$ for all $n \geq 0$ (constant sequence) and $z_2(n + 1)= (d c +f z_2)^2 + c$. With the change of variable $\xi = f^2 z_2 + d c$, the second equation again reduces to a standard quadratic map $\xi(n + 1) = \xi^2 + c (f^2 + d)$.
\end{description}

Altogether, these cases imply that the escape radius property remains valid when $|a f| = |b d|$, as long as $a + b k^2 \neq 0$. One can easily find singular matrices $A$ (i.e., $af-bd=0$) with $a + b k^2 = 0$, for which the corresponding system does not satisfy the escape radius property. Take for example $a = 1$, $b = -1$, $k = 1$, in which case $z_1(n + 1) = z_2(n + 1) =(z_2 - z_1)^2 + c$. Initial conditions with arbitrarily large $z_1(0) = z_2(0)$ immediately collapse to $z_1(n) = z_2(n) = c$, for all $n \geq 1$, showing that there is no radius $M>0$ past which orbits automatically escape to infinity. Computational approaches to identifying whether orbits are bounded lead to a somewhat more general and difficult question: whether (or for which networks) it is possible that, given any large $M$, there exist $c$ values for which the critical orbit can grow larger than $M$, but remain asymptotically bounded. This can inform us for what types of networks one can test computationally whether certain values of $c$ can be unquestionably excluded from the equi-M set, based on a finite number of iterations (in the same manner as in the single map case). Investigating this statement is not within the scope of this paper.

\subsection{Stable fixed point and main cardioid}
\label{cardioid}

In this section, we examine the main equi-cardioid for 2D-CQNs, that is the region in the $c$-plane where the system has a stable fixed point. Starting with the fixed point equations
\[
\begin{cases}
    z_1=(az_1+bz_2)^2+c \\
    z_2=(dz_1+fz_2)^2+c,
\end{cases}
\]
we must ensure that for at least one of the  fixed points, the Jacobian matrix 
\[
\begin{bmatrix}
2a(az_1+bz_2) & 2b(az_1+bz_2)\\
2d(dz_1+fz_2) & 2f(dz_1+fz_2)
\end{bmatrix}
\]
has both eigenvalues within the unit disk. With the notations $u_1=az_1+bz_2$, $u_2=dz_1+fz_2$, this translates to solving the system 
\begin{equation}\label{eq.fixed.point}
   \begin{cases}
    c(a+b)+au_1^2+bu_2^2=u_1\\c(d+f)+du_1^2+fu_2^2=u_2
\end{cases}
\end{equation}
such that both roots of the characteristic equation
\begin{equation} \label{eq.char}
\lambda^2 - 2 (a u_1 + fu_2) \lambda + 4 \delta u_1 u_2=0,
\end{equation}
where $\delta = \det(A) = af - bd$, are within the unit disk.

The following lemma is a consequence of the classical Schur-Cohn stability test \cite{henrici1974computational}.

\Note{Eva}{Am lasat o varianta mai simpla ce rezulta un pic mai direct din Schur-Cohn, sa nu mai fie radicalul acela urat. Si am pus si o referinta.}


\begin{lemma}\label{lem.quadratic.poly}
Let $\alpha, \beta\in\mathbb{C}$. Both roots of the quadratic polynomial $x^2-2\alpha x+\beta$ belong to the open unit disk if and only if
\begin{equation} \label{ineg.lemma}
    |\beta| < 1
    \quad \text{and} \quad
    2 |\alpha-\overline{\alpha}\beta| < 1-|\beta|^2.
\end{equation}
\end{lemma}



We assume that the matrix $A$ is non-singular, and hence, without loss of generality, $a+b\neq 0$. We remark that eliminating $c$ from system \eqref{eq.fixed.point} gives
\begin{equation} \label{eq.u1.u2}
    (d + f) u_1 - (a + b) u_2 = \delta (u_1^2 - u_2^2),
\end{equation}
Based on \eqref{eq.char} and \eqref{eq.u1.u2}, applying  Lemma \ref{lem.quadratic.poly} for $\alpha=au_1+fu_2$ and $\beta=4\delta u_1u_2$, we obtain the following:

\Note{Eva}{I replaced $\alpha$ and $\beta$ in the system below.}

\begin{prop}\label{prop.cardioid}
Consider the region $\mathcal{R}$ of points $(u_1,u_2)\in\mathbb{C}^2$ which simultaneously verify 
\[
\begin{cases}
    (d + f) u_1 - (a + b) u_2 = \delta (u_1^2 - u_2^2), \\
    4|\delta u_1u_2| < 1, \\
    2|a u_1 + f u_2-4\delta(a|u_1|^2u_2+fu_1|u_2|^2)| <1 - 4|\delta u_1 u_2|^2.
\end{cases}
\]
The set of complex values $c$ for which the coupled system of quadratic maps has at least one stable fixed point is given by $g(\mathcal{R})$, where
\[
g(u_1,u_2)=\frac{u_1-au_1^2-bu_2^2}{a+b}.
\]
\end{prop}

To provide specific illustrations of the properties of the main cardioid in particular cases, we worked out direct computations for two different families of coupled maps, as detailed in the next section.

\subsubsection{Example 1: Feed-forward networks}

Consider the family of feed-forward two-dimensional coupled maps given by $b = 0$ and $f > 0$, additionally requiring $a = d = 1$ (which allows computations to remain tractable). In other words:
\begin{equation}  \label{family1}
\begin{aligned}
z_1(n+1) & = z_1^2 + c \\
z_2(n+1) & = (z_1 + f z_2)^2 + c
\end{aligned}
\end{equation}

We want to establish the boundary of the the region in the $c$-plane where the two-dimensional map has at least one stable fixed point $(z_1^*,z_2^*)$. This encompasses fixed point conditions:
\begin{align} 
z^*_1 &= {z^*_1}^2+c
\tag{F1} \label{fixed1}\\
z^*_2 &= (z^*_1+fz^*_2)^2+c
\tag{F2} \label{fixed2}
\end{align}
and stability conditions on the magnitude of the eigenvalues:
\begin{equation}
2f |z^*_1+fz^*_2 | =1 \quad \text{and} \quad |2z^*_1 | \leq 1
\tag{S1} \label{stab1}
\end{equation}
\begin{equation}
|2z^*_1 | = 1 \quad \text{and} \quad 2f|z^*_1+fz^*_2 | \leq 1
\tag{S2} \label{stab2}
\end{equation}

\Note{Alex}{What's up with this ridiculous hardcoded equation numbering? :( And the (Ia) an (IIb') and so on below.}

\Note{Alex}{Maybe we want to re-order this example in terms of some propositions or corollaries (or whatever) to give it more structure.}

Similarly with the traditional case for single iterated maps (where this region is the interior of the main cardioid), we aim to give a parametric description of the boundary of the region in polar form. For simplicity, for the rest of this computation we will drop the $(*)$ and refer to the fixed point as $(z_1,z_2)$, since there is no danger of confusion. If we eliminate $c$ from the fixed point equations, we get that $z_2-z_1 = 2fz_1z_2+f^2z_2^2$. Hence one can express

\begin{eqnarray*}
& & z_1
    = \frac{z_2 (1 - f^2 z_2)}{1 + 2 f z_2}, \\
& & z_1 + f z_2
    = \frac{z_2 (1 - f^2 z_2)}{1 + 2 f z_2} + f z_2
    = \frac{z_2 (1 + f + f^2 z_2)}{1 + 2 f z_2}, \\
& & c = z_1 - z_1^2
    = \frac{z_2 (1 - f^2 z_2)}{1 + 2 f z_2}
    - \frac{z_2^2 (1 - f^2 z_2^2)^2}{(1 + 2 f z_2)^2}
    = \frac{z_2 (1 - f^2 z_2)[1 + (2 f - 1) z_2 + f^2 z_2^2]}{(1 + 2 f z_2)^2}.
\end{eqnarray*}

The first condition in~\eqref{stab1}  can then be rewritten as:
\[
2 f |z_2| |1 + f + f^2 z_2| = |1 + 2 f z_2|.
\]

If one considers $z_2$ in its polar form $z_2 = \rho e^{i\theta}$, the condition becomes:
\[
\begin{aligned}
&
    2 f \rho |1 + f + f^2 \rho e^{i \theta}| = |1 + 2 f \rho e^{i\theta}| \\
\iff\,\, &
    4 f^2 \rho^2 [(1 + f)^2 + 2 f^2 (1 + f) \rho \cos \theta + f^4 \rho^2]
    = 1 + 4 f \rho \cos \theta + 4 f^2 \rho^2.
\end{aligned}
\]

Solving for $\theta$, we obtain
\begin{equation} \label{S1a}
\cos(\theta) =
    \frac{1 - 8 f^3 \rho^2 - 4 f^4 \rho^2 - 4 f^6 \rho^4}{8 f^4 \rho^3 (1 + f) - 4 f \rho}
     \tag{$S1a$}
\end{equation}

\Note{Alex}{This $(S1a)$ tag is italic while the ones before are upright.}

In turn, the second condition in~\eqref{stab1} becomes
\[
\begin{aligned}
&
    2 \rho |1 - f^2 \rho e^{i\theta} |
    \leq |1 + 2 f \rho e^{i\theta}| \\
\iff\,\, &
    4 \rho^2 (1 - 2 f^2 \rho \cos \theta + f^4 \rho^2)
    \leq 1 + 4 f \rho \cos \theta + 4 f^2 \rho^2.
\end{aligned}
\]

If one calls
\begin{equation}
h(\rho) = \frac{4 f^4 \rho^4+4\rho^2(1-f^2)-1}{4f\rho(2f\rho^2+1)},
\end{equation}
then condition in polar form becomes:
\begin{equation}
\cos \theta \geq h(\rho). \tag{$S1b$} \label{S1b}
\end{equation}

To interpret conditions~\eqref{S1a} and~\eqref{S1b} together, we distinguish the following three cases:

\begin{description}
\item[Case I:] $h(\rho) > 1$, so that there is no angle $\theta$ that satisfies~\eqref{S1b}. The condition on $h$ is equivalent to
\[
    4 \rho^2 (f^2 \rho - 1)^2 > (2 f \rho + 1)^2.
\]

\begin{itemize}
    \item If $f^2 \rho > 1$, then this is further equivalent to asking that $2 f^2 \rho^2 - 2 \rho (f + 1) - 1 > 0$. Since the two quadratic roots
\[
K^{\pm}(f) = \frac{1 + f \pm \sqrt{(f + 1)^2+2f^2}}{2f^2}
\]
are such that $K^-(f) < 0 < f^{-2} < K^+(f)$ for all $f > 0$, we get that this scenario occurs when $\rho > K^+(f)$.

    \item If $f^2 \rho < 1$, then the condition becomes $f^2\rho + 2\rho(f-1)+1<0$. This has no solutions for $\rho$ when $f>\sqrt{2}-1$. When $f<\sqrt{2}-1$, the quadratic function has two roots
\[
M^{\pm}(f) = \frac{1-f \pm \sqrt{(1-f)^2-2f^2}}{2f^2}.
\]

Since $f < 1$, the roots are such that $0 < M^-(f) < M^+(f) < f^{-2}$, hence this scenario occurs only for $f < \sqrt{2}-1$, when $M^-(f) < \rho < M^+(f)$.
\end{itemize}

Notice that the two intervals for $\rho$ (where there is no angle $\theta$ that satisfies ~\eqref{S1b}) are disjoint, since $M^+(f)<\frac{1}{f^2}< K^+(f)$. 


\item[Case II:] $h(\rho) < -1$, hence~\ref{S1b} generates no additional conditions on $\theta$. The condition is equivalent to
\[
4\rho^2(f^2\rho +1)^2 < (2f\rho-1)^2.
\]

\begin{itemize}
    \item If $2 f \rho > 1$, then this is further equivalent to $2f^2\rho^2+2(1-f)\rho+1<0$. When $f>\sqrt{2}-1$, the quadratic roots are complex, and there are no solutions for $\rho$. When $f<\sqrt{2}-1$, both roots
\[
    R^{\pm}(f) = \frac{f - 1 \pm \sqrt{(1 - f)^2 - 2 f^2}}{2 f^2} < 0,
\]
hence there is no solution here, either.

    \item If $2 f \rho < 1$, then the condition becomes $2f^2\rho^2+2(1+f)\rho-1<0$, so that there are always two real roots
\[
S^{\pm}(f) = \frac{-(f+1)\pm \sqrt{(f+1)^2+2f^2}}{2f^2}.
\]

In addition, we notice that $S^-<0<S^+<\frac{1}{2f}$.
\end{itemize}

We continue investigating the $2 f \rho < 1$ case that allows valid solutions.
If $0<\rho<S^+$, the additional restriction coming from~\eqref{S1b} is that $-1 \leq \cos \theta \leq 1$, hence:
\[
-1 \leq 1-8f^3 \leq 1
\]
and
\[
\ds \rho < \frac{-2+\sqrt{6}}{2}
    \iff 2\rho^2+4\rho-1<0
    \iff 2\rho^2-1<-4\rho
    \iff h(\rho)<-1.
\]

Hence in this case condition~\eqref{S1b} is satisfied by all values of $\theta$.

Since $\ds \rho < \frac{-2+\sqrt{6}}{2} <\frac{1}{2}$, then $4\rho^2-1<0$. Using the expression in~\eqref{S1a} for $\cos \theta$ we get that the only condition on the angle is that $-1 \leq \cos \theta \leq 1$, which can be rewritten as:
\begin{equation}
-16\rho^4+4\rho \geq 1-12\rho^2-4\rho^4.
\tag{IIa} \label{IIa}
\end{equation}
and
\begin{equation}
1-12\rho^2-4\rho^4 \geq 16\rho^3-4\rho.
\tag{IIb} \label{IIb}
\end{equation}

One can calculate that~\eqref{IIa} is equivalent to:
\begin{equation*}
4\rho^4-16\rho^3+12\rho^2+4\rho-1 \geq 0 \iff 4\rho^2(\rho-2)^2 \geq (2\rho-1)^2.
\end{equation*}

Since in this case $\rho<1/2<2$, it follows that~\eqref{IIa} is equivalent with the quadratic inequality $2\rho(2-\rho) \geq 1-2\rho$, which is satisfied when
\begin{equation}
\frac{3-\sqrt{7}}{2} < \rho <\frac{3+\sqrt{7}}{2}
\tag{IIa$'$} \label{IIa'}
\end{equation}

The second inequality~\eqref{IIb} is equivalent to:
\begin{equation*}
4\rho^4+16\rho^3+12\rho^2-4\rho-1 \leq 0 \iff 4\rho^2(\rho+2)^2 \geq (2\rho+1)^2
\end{equation*}

This is equivalent with the quadratic inequality $2\rho(2+\rho) \geq 1+2\rho$, which is satisfied when
\begin{equation}
0 < \rho <\frac{-1+\sqrt{3}}{2}
\tag{IIb$'$} \label{IIb'}
\end{equation}

Since $\ds \frac{-1+\sqrt{3}}{2}>\frac{-2+\sqrt{6}}{2}$ and $\ds \frac{3+\sqrt{7}}{2}>\frac{-\sqrt{6}}{2}$, the two conditions~\eqref{IIa'} and~\eqref{IIb'} combined lead to the interval:
 \begin{equation}
\frac{3-\sqrt{7}}{2} < \rho <\frac{-2+\sqrt{6}}{2}
\tag{II$'$} \label{II$'$}
\end{equation}

\item[Case III:] $\ds \frac{-2+\sqrt{6}}{2} \leq \rho \leq \frac{2+\sqrt{6}}{2} \iff -1 \leq h(\rho) \leq 1$. In this case we have nontrivial solutions for $\theta$, so we need to consider both conditions~\eqref{IIa} and~\eqref{IIb} when computing the domain for $\rho$, as follows:
\begin{equation*}
h(\rho) = \frac{2\rho^2-1}{4\rho} \leq \frac{1-12\rho^2-4\rho^4}{4\rho(4\rho^2-1)} \leq 1
\end{equation*}

We distinguish two subcases:

\begin{itemize}
\item If $\ds \rho < \frac{1}{2}$. Then we have~\eqref{IIb'} and also (since $4\rho^2-1<0$)
\[
(2\rho^2-1)(4\rho^2-1) \geq 1-12\rho^2-4\rho^4 \iff 12\rho^4+6\rho^2 \geq 0
\]
which is satisfied by all $\rho$. Since $\ds \frac{-1+\sqrt{3}}{2}<\frac{1}{2}$, in this case the conditions for $\rho$ will summarize as:
\[
\frac{-2+\sqrt{6}}{2} \leq \rho \leq \frac{-1+\sqrt{3}}{2}.
\]

\item If $\ds \rho > \frac{1}{2}$. Then $4\rho^2-1>0$ and we have
\begin{equation*}
(2\rho^2-1)(4\rho^2-1) < 1-12\rho^2-4\rho^4 < 16\rho^3-4\rho \iff 12\rho^2 + 6\rho^2  <0
\end{equation*}
which has no solution.
\end{itemize}

In conclusion, Case III allows for
\[
\frac{-2+\sqrt{6}}{2} \leq \rho \leq \frac{-1+\sqrt{3}}{2}.
\]

\end{description}

This closes the discussion of the joint conditions~\eqref{S1a} and~\eqref{S1b}, so that~\eqref{stab1} can be altogether translated into the following polar curve, defined parametrically as:
\begin{equation}
\cos \theta = \frac{1-12\rho^2-4\rho^4}{4\rho(4\rho^2-1)}, \text{ for } \frac{3-\sqrt{7}}{2} \leq \rho \leq \frac{-1+\sqrt{3}}{2}
\tag{S1$'$} \label{S1'}
\end{equation}

It is relatively easy to see that no additional boundary points result from conditions~\eqref{stab2}. Indeed, considering again the polar form for $z_2 = \rho e^{i\theta}$ we can rewrite the first part of~\eqref{stab2} as:
\begin{equation*}
|z_1 | = \left |\frac{z_2(1-z_2)}{1+2z_2}  \right | = \frac{1}{2}\\ \\ \iff 2\rho \sqrt{1-2\rho \cos \theta + \rho^2}  = \sqrt{1+4\rho \cos \theta + 4\rho^2}
\end{equation*}
leading to
\begin{equation}
\cos \theta = \frac{4\rho^4-1}{4\rho (2\rho^2+1)} = \frac{2\rho^2-1}{4\rho}
\tag{S2a} \label{S2a}
\end{equation}

The second part of~\eqref{stab2} becomes:
\begin{equation*}
2|z_1+z_2 | = \frac{2|z_2 | |2+z_2 |}{\vert 1+2z_2 |} =1 \iff 2\rho \sqrt{4+4 \rho \cos\theta+\rho^2} \leq \sqrt{1+4\rho \cos\theta+4\rho^2}
\end{equation*}
from which it follows that
\begin{equation*}
1-12\rho^2-4\rho^4 \geq 4\rho (4\rho^2-1) \cos \theta = 4\rho (4\rho^2-1) \frac{2\rho^2-1}{4\rho} = 8\rho^4-6\rho^2+1
\end{equation*}

Hence $12\rho^4+6\rho^2 \leq 0$, which produces no valid solutions for $\rho$. This concludes the proof that the conditions in~\eqref{stab2} are empty, and the bounding curve of the region with one locally attracting fixed point is indeed given by the expression in~\eqref{S1'}, as represented in Figure~\ref{fig:cardio.feedforward.1}.

\begin{figure}[htbp]
    \centering
    \includegraphics[width=\linewidth]{figures/fig_ex1.png}
    \caption{Cardioids for the feedforward case  with the coupling matrix $A=\begin{bmatrix}
        1 & 0 \\ 1 & f
    \end{bmatrix}$.}
\label{fig:cardio.feedforward.1}
\end{figure}

\subsubsection{Example 2: Equal row sum networks}

With the simplifying assumption $a+b=d+f:=\omega$, equation \eqref{eq.u1.u2} provides:
\[(u_1-u_2)\left[(a-d)(u_1+u_2)-1\right]=0,\]
and hence, we have two cases:\\ 

\noindent\textbf{Case I:} $u_1=u_2$. In this case, system \eqref{eq.fixed.point} reduces to: 
\begin{equation}\label{eq.fixed.point.w.case1}
c\omega +\omega u_1^2=u_1
\end{equation}
and the characteristic equation becomes: 
\[\lambda^2-2\tau u_1\lambda+4\delta u_1^2=0,\quad\text{where }\tau=\text{Tr}(A)=a+f.\]
or equivalently 
\[\left(\frac{\lambda}{2u_1}\right)^2-\tau\left(\frac{\lambda}{2u_1}\right)+\delta=0,\]
which means that $\lambda/(2u_1)\in\sigma(A)=\{\omega,\tau-\omega\}$. Therefore, eliminating $u_1$ from \eqref{eq.fixed.point.w.case1}, it follows that:
\[\lambda^2 -2\frac{\mu}{\omega}\lambda +4c\mu^2 =0,\quad\text{where }\mu\in\{\omega,\tau-\omega\}.\]

If $\mu=\omega$, from Vieta's formulas we deduce that $\lambda_1+\lambda_2=2$. Consequently, at least one eigenvalue has absolute value larger than one, and the corresponding fixed point is unstable.  

On the other hand, if $\mu=\tau-\omega$, then $\alpha=\frac{\tau}{\omega}-1$ and $\beta=4c(\tau-\omega)^2$, and the inequalities from Lemma \ref{lem.quadratic.poly} become: 
\begin{equation}\label{ineq.cardio.1}
|c|<\frac{1}{4(\tau-\omega)^2}\quad\text{and}\quad 2\left|\frac{\tau}{\omega}-1\right|<\frac{1-16(\tau-\omega)^4|c|^2}{\sqrt{1+16(\tau-\omega)^4|c|^2-8(\tau-\omega)^2\Re(c)}}.\end{equation}
These inequalities define a part of the cardioid only if  $\left|\frac{\tau}{\omega}-1\right|<1$. Otherwise, the second inequality from \eqref{ineq.cardio.1} cannot hold, and consequently, both fixed points for which $u_1=u_2$ are unstable.\\

\noindent\textbf{Case II:} $u_1+u_2=(a-d)^{-1}=(\tau-\omega)^{-1}$. 

Let us write 
\[
q:=u_{1}-u_{2},\qquad 
u_{1}=\frac{(\tau-\omega)^{-1}+q}{2}, \qquad 
u_{2}=\frac{(\tau-\omega)^{-1}-q}{2},\qquad q\in\mathbb C .
\]
Substituting the above expressions into the first equation of~\eqref{eq.fixed.point} we obtain
\begin{equation}\label{eq.c.case2}
c(q)=\frac{-\,a\,(q+1)^{2}-b\,(q-1)^{2}+2(\tau-\omega)\,(q+1)}
           {4\omega(\tau-\omega)} .
\end{equation}
With the same parametrization we have
\[
\alpha(q)=a\,u_{1}+f\,u_{2}
         =\frac{(\tau-\omega)q+d-\omega}{2(\tau-\omega)},\qquad
\beta(q)=4\delta\,u_{1}u_{2}
        =\frac{\delta}{(\tau-\omega)^{2}}\,(1-q^{2}).
\]
Based on Lemma~\ref{lem.quadratic.poly}, we define
\[
\mathcal Q:=
\biggl\{\,q\in\mathbb{C}:
            |\beta(q)|<1\;\text{ and }\;
            2|\alpha(q)-\overline{\alpha(q)}\beta(q)|<
            1-|\beta(q)|^{2}
     \biggr\}.
\]
The set of complex parameters $c$ for which the coupled quadratic map has a
stable fixed point satisfying $u_{1}+u_{2}=(\tau-\omega)^{-1}$ is
\[
c(\mathcal Q)
               :=\bigl\{\,c(q)\;:\;q\in\mathcal Q\bigr\},
\]
where $c(q)$ is given by~\eqref{eq.c.case2}. While the general case becomes intractable quickly, the analysis for special subfamilies is more approachable. We present two interesting special cases below.\\\\

\noindent \fbox{\textbf{\it{Example 2(a).}}} We first consider the feed-forward case: $b=0$. Hence, $a=\omega$, $d=2\omega-\tau$ and $f=\tau-\omega$. Let us denote $\rho=\tau/\omega-1$. The characteristic equation \eqref{eq.char} simplifies to:
\[\lambda^2-2(au_1+fu_2)\lambda+4afu_1u_2=0,\]
and hence, the roots are $\lambda_1=2a u_1=2\omega u_1$ and $\lambda_2=2f u_2=2(1-(\tau-\omega)u_1)=2(1-\rho\omega u_1)$, where $u_1$ is a root of 
\[c\omega+\omega u_1^2=u_1.\]
The boundary of the cardioid is given by either $|\lambda_1|=1$ and $|\lambda_2|\leq 1$, or $|\lambda_2|=1$ and $|\lambda_1|\leq 1$.

On the one hand, if $\lambda_1=e^{i\theta}$, then $u_1=e^{i\theta}/(2\omega)$ and we must require that 
\[\left|2-\rho e^{i\theta}\right|\leq 1\quad\Leftrightarrow\quad 4\rho\cos\theta\geq 3+|\rho|^2\quad\Rightarrow\quad 1\leq |\rho|\leq 3. \]
In this case, the parametric equation of a part of the boundary of the cardioid is 
\[c=\frac{u_1(1-\omega u_1)}{\omega}=\frac{e^{i\theta}(2-e^{i\theta})}{4\omega^2},\quad\text{where }\text{sign}(\rho)\cos\theta\geq \frac{3+|\rho|^2}{4|\rho|}.\]

On the other hand, if $\lambda_2=e^{i\theta}$, then $u_2=\frac{e^{i\theta}}{2(\tau-\omega)}=\frac{e^{i\theta}}{2\rho\omega}$ and hence $u_1=\frac{2-e^{i\theta}}{2\rho\omega}$. We must ensure that 
\[\left|2-e^{i\theta}\right|\leq|\rho|\quad\Leftrightarrow\quad \cos\theta\geq\frac{5-|\rho|^2}{4}\quad\Rightarrow\quad |\rho|\geq 1. \]
In this case, we have the parametric equation \[c=\frac{u_1(1-\omega u_1)}{\omega}=\frac{2-e^{i\theta}}{2\omega(\tau-\omega)}\left(1-\frac{\omega(2-e^{i\theta})}{2(\tau-\omega)}\right)=\frac{(2-e^{i\theta})(2\rho-2+e^{i\theta})}{4\rho^2\omega^2},\quad\text{where }\cos\theta\geq\frac{5-|\rho|^2}{4}.\]

In conclusion, if $|\rho|\geq 1$, the cardioid is bounded by the union of the parametric curves given above. On the other hand, if $|\rho|<1$, we fall back to Case I, and hence, the cardioid is defined  by \eqref{ineq.cardio.1}. These two cases can be visualized in Figure \ref{fig:cardio.feedforward}.

\begin{figure}[htbp]
    \centering    \includegraphics[width=\linewidth]{figures/fig_ex2a.png}
    \caption{Cardioids for the feedforward case  with the coupling matrix $A=\begin{bmatrix}
        \omega & 0 \\ 2\omega-\tau & \tau-\omega
    \end{bmatrix}$ (with equal row sum $\omega$). }
\label{fig:cardio.feedforward}
\end{figure}

\vspace{5mm}
\noindent
\fbox{\textbf{\it{Example 2(b).}}} Considering $a=b=\omega/2$, $d=1$ and $f=\omega-1$, we obtain more complicated cardioid structures shown in Figure \ref{fig:cardio.w}. It is worth noting that in some cases, the cardioid is the union of three disjoint connected components. We also note an interesting property of the maps in this family. Due to the fact that the coupling matrix $A$ has equal row sum $\omega$, the symmetric subspace $z_1=z_2$ is an invariant set, and the dynamics on this subspace reduces to the one-dimensional quadratic map $z\mapsto (\omega z)^2+c$. For $c=-2/\omega^2$, this quadratic map is topologically conjugate, through the linear homeomorphism $\phi(z)=-\omega^2 z/4+1/2$, to the logistic map $z\mapsto 4z(1-z)$, which is known to be chaotic. Therefore, the trajectory of the 2D quadratic system starting from $(0,0)$ is chaotic. However, the 2D system has non-symmetric asymptotically stable fixed points for certain ranges of $\omega$ (e.g. $0.674942 < \omega < 0.737465$).

\begin{figure}[htbp]
    \centering    \includegraphics[width=\linewidth]{figures/fig_ex2b.png}
    \caption{Cardioids for coupled quadratic maps with the coupling matrix $A=\begin{bmatrix}
        \omega/2 & \omega/2 \\ 1 & \omega-1
    \end{bmatrix}$.}
    \label{fig:cardio.w}
\end{figure}

\subsection{Higher order periods and pseudo-bulb structure}
\label{higher_order}

While one can still aim to classify higher order periodic combinatorics for systems of two coupled nodes, this extension is readily more complex than the similar question for single maps, rising from the significantly richer behavior of coupled nodes in $\mathbb{C}^2$.

\begin{figure}[h!]
\centering
\fbox{\includegraphics[width=0.4\textwidth]{figures/figure-equal-row-03-periods.png}}
\quad \quad
\fbox{\includegraphics[width=0.4\textwidth]{figures/figure-equal-row-04-periods.png}}
\caption{Postcritically periodic regions of periods up to 10 are shown as subsets of the equi-M set, for two example networks: {\bf A.} $z_1(n+1)= z_1^2+c$ and $z_2(n+1) =(z_1+z_2)^2+c$ (Example 1); {\bf B.} $z_1(n+1)= (0.4z_1+0.4z_2)^2+c$ and $z_2(n+1) =(z_1-0.2z_2)^2+c$ (Example 2, for $\omega=0.8$). The colors correspond to the critical orbit being attracted respectively to a fixed point (purple region), period two orbit (cyan region), period three orbit (green regions), period four orbit (orange regions), etc. }
\label{higher_periods1}
\end{figure}

To start with, our numerical simulations suggest that the equi-M set replicates in some sense the hyperbolic bulb structure found in the traditional Mandelbrot set. More precisely, one can identify disjoint subsets of the equi-M set characterized by different postcritical combinatorics (which we called in previous work \emph{pseudo-bulbs}). Figure~\ref{higher_periods1} illustrates low period pseudo-bulbs for two of our working examples. One is the simple forward network in  Example 1, with iterations given by $z_1(n+1)= z_1^2+c$ and $z_2(n+1) =(z_1+z_2)^2+c$. The other is the network in Example 2(b) with $w=0.4$. We used numerical computations to map out in each case the regions of $c \in \mathbb{C}$ for which the critical point is attracted to a periodic obit of certain period $1 \leq k \leq 10$ (color coded based on the period $k$) (the Mathematica code is included in the supplementary material). We know that pseudo-bulbs of all periods do not form an exhaustive partition for the equi-M (since we found examples of chaotic critical dynamics). A more complex question to raise is whether the density of hyperbolicity hypothesis becomes easier to disprove in this multi-dimensional context (starting with two coupled nodes, but also for higher-dimensional networks). While we are considering this line of questioning in our current work, it is not within the scope of this paper.


For both example systems, the boundary of the main cardioid was already computed analytically and illustrated in the previous section. Based on this existing information, one can easily notice that, unlike in the case of single quadratic map iterations, the main cardioid does not simply identify in either case with the boundary of the period $k=1$ pseudo-bulb (purple region in the  Figure~\ref{higher_periods1} panels). For the system in Example 1, there are points where an attracting fixed point exists (i.e., $c$ is inside of the main cardioid, shown in Figure~\ref{higher_periods2}a as the blue shaded region), yet the critical point escapes (i.e., $c$ is outside of the black equi-M contour). The situation is even more interesting in Example 2, where the $c$-locus inside the main cardioid (where the system has an attracting fixed point, shaded in blue in Figure~\ref{higher_periods2}b)  is composed of three connected components: the rightmost (and largest) lies completely inside the equi-M contour; the leftmost is completely outside of the equi-M contour; the middle one is in the equi-M set as the patch where an attracting fixed point and attracting period two coexist.

This phenomenon is important, since it indicates a departure from the results in the single iterated map case. It is well-known that, for polynomial families, the critical orbits encompass global information on all other possible orbits. In particular, this is true for the complex quadratic family, with the  Mandelbrot set offering a comprehensive atlas of all combinatorics. Our two examples clarify that this is no longer the case 2D-CQNs, where one cannot expect the postcritical behavior to be accurately and completely descriptive of the whole system dynamics. More implications of this extension are contextualized in the Discussion section.

\begin{figure}[h!]
\centering
\fbox{\includegraphics[width=0.4\textwidth]{figures/figure-equal-row-03.pdf}}
\quad \quad
\fbox{\includegraphics[width=0.4\textwidth]{figures/figure-equal-row-04.pdf}}
\caption{\small \emph{{\bf Regions with higher order periods for two example systems:} {\bf A.} The feed-forward network $z_1 = z_1^2+c$; $z_2=(z_1+z_2)^2+c$ and {\bf B.} The system in Example 2(b), for $\omega = 0.4$. In each panel, the shaded regions correspond to the system having an attracting fixed point (blue), and attracting period 2 orbit (yellow) and period 3 orbit (brown). In the right panel, the green region represents coexistence of an attracting fixed point and an attracting period two orbit. In both cases, the boundary of the equi-M set is shown as a black contour, for comparison.}}
\label{higher_periods2}
\end{figure}




%%%%%%%%%%%%%%%%%%%%%%%%%%
\section{Synchronization in systems of two coupled maps}
\label{synchronization}

The ultimate structure of the equi-M set is determined by the intersection of node-wise equi-M sets (defined as the parameter regions where individual components of the critical orbit remain independently bounded, as the multi-dimentional orbit is iterated). In previous work on CQNs, we showed that nodes can independently have different behaviors, with some nodes remaining bounded, while other nodes in the same coupled network escaping asymptotically to infinity. While it is likely that the number and formation of these clusters are related to network architectural structures like reaches or strong components, the exact correspondence is harder to address in general networks. Two-dimensional CQNs are an excellent toy family for anchoring basic conditions for synchronization, when the network structure is the simplest. It represents a stepping stone for more general results in higher dimensions. In this section, we build a sequence of results that track the impact of coupling parameters on the synchronization and de-synchronization of the two nodes (and of the phase transitions between them).

\begin{lemma}
If $z_j$ has a nontrivial input to $z_k$, and if $z_j(n)$ is bounded in $\mathbb{C}$ as $n \to \infty$, then $z_k(n)$ is also bounded.
\label{basic_lemma}
\end{lemma}

\proof{We can assume without loss of generality that $z_1$ is bounded, and that the input $b$ from $z_2$ to $z_1$ is nonzero, and prove that $z_2$ is also bounded. Since $z_1$ is bounded, there exists $M>0$ such that $|z_1(n) | \leq M$, for all $n \geq 0$. Since
$|z_1(n+1) | \geq |az_1 + b z_2 |^2 - |c |$, it follows that $|b z_2 | - |az_1 |  \leq \sqrt{|z_1(n+1) | + |c |}$. Then:
$$|b z_2 | \leq \sqrt{|z_1(n+1) | + |c |} + |a z_1 |\leq \sqrt{M + |c |} + |a | M$$
Then, if $b \neq 0$, we have, for all $n \geq 0$:
$$|z_1(n) | \leq K = \frac{\leq \sqrt{M + |c |} + |f | M}{|d |}$$

insuring that $z_2(n)$ is bounded.

\qed}

\begin{prop}
    If $z_1$ and $z_2$ are nontrivially interconnected (i.e., both $b,d \neq 0$), then they are simultaneously bounded.
    \label{z1z2_bounded}
\end{prop}

\vspace{3mm}
In previous work, we have looked in particular at the possible behaviors of the components of the critical orbit of the network, and their ``synchronization'' into different clusters with identical bounded versus unbounded behavior. More precisely:

\begin{defn}
For each node $z_k$, $1 \leq k \leq n$, we define the \textbf{node-wise equi-M set} ${\cal M}_k$ as the parameter locus $c \in \mathbb{C}$ for which the component corresponding to the node $z_k$ of the critical multi-orbit is bounded. We say that two network nodes $z_i$ and $z_j$ are \textbf{M-synchronized}, if their equi-M sets ${\cal M}(z_i)$ and ${\cal M}(z_j)$ are identical. We also say that the nodes belong to the same \textbf{M-synchronization cluster} (or simply \textbf{M-cluster}) of the network.
\end{defn}

In this context, we can reinterpret the results in this sections as

\begin{corol}
If in the network described by \eqref{mother_sys} the nodes $z_1$ and $z_2$ are nontrivially interconnected (i.e., $b,d \neq 0$), then the nodes are simultaneously bounded. In particular, ${\cal M}(z_1) = {\cal M}(z_2)$.
\end{corol}

\begin{figure}[h!]
\begin{center}
\includegraphics[width=0.55\textwidth]{figures/d_all.png}
\end{center}
\caption{\emph{\small {\bf Evolution of the equi M set as the value of $d$ increases.} The other entries of the matrix $A$ were fixed to $a=0.6$, $b=0.8$, $f=0.4$. The panel illustrates the equi-M contour (identical between the two nodes), using different colors for different values of $d$, as follows: $d=-0.2$ (grey); $d=0.01$ (cyan); $d=0.1$ (orange); $d=0.3$ (blue, scaling of the traditional Mandelbrot set, since $d$ is the critical values for which $\det(A)=0$); $d=1$ (pink); $d=2$ (green); $d=3$ (brown).}}
\label{d_all}
\end{figure}

The proof follows directly from Proposition~\ref{z1z2_bounded}. Notice that if in addition the matrix $A$ is singular, then $k=\frac{d}{a} = \frac{f}{b}$ and, with the notation $\xi = (a+bk^2)(az_1+bz_2)$, we have $\xi(n+1) = \xi^2+(a+bk^2)(a+b)c$. The values of $c$ for which $z_1$ and $z_2$ are (simultaneously) bounded are the same as the values of $c$ for which $(a+bk^2)(a+b)c$ is in the traditional Mandelbrot set. Hence the equi M set (identical between the two nodes) is in this case a rescaling of the traditional set. Figure~\ref{d_all} illustrates how the shape of the equi M set of the two synchronized nodes evolves as the value of $d$ increases through the negative and positive range as the other three entries of $A$ are fixed, crossing through the critical value which makes $A$ singular. Notice that, as $d$ is increasing from negative values, the shape slowly approaches that of the traditional Mandelbrot set (achieved for the critical $d=0.3$), and then slowly degrades away from this shape as $d$ continues to increase.

This sheds some light on the effects of the coupling when the nodes are interconnected ($b,d \neq 0$). At the other end of the spectrum, we have the case when the nodes both act independently. This case is trivial, as described in the following lemma:

\begin{lemma}
    If $b=d=0$ and $a,f \neq 0$, then the two node-wise equi M sets are both scaled versions of the traditional Mandelbrot set.
\end{lemma}

\proof{We have $z_1(n+1) = (az_1)^2+c$, and $z_2(n+1) = (fz_2)^2+c$. Then $c \in {\cal M}(z_1) \iff ac \in {\cal M} \iff {\cal M}(z_1) = \frac{1}{a^2}{\cal M}$. Similarly, ${\cal M}(z_2) = \frac{1}{f^2}{\cal M}$.
}
\hfill \qed

\vspace{3mm}
A natural question to ask next is whether the nodes need to necessarily be nontrivially interconnected in order to be M-synchronized. The answer to that is negative. The following lemma gives a weaker condition that insures M-synchronization of the two nodes (although it does not deliver the stronger result in Corollary~\ref{z1z2_bounded}).

\begin{lemma}
    Suppose $z_1(n+1) = (az_1)^2 + c$ (decoupled), and $z_2(n+1)=(dz_1+fz_2)^2+c$. Then, for $|f |$ sufficiently small, if the $z_1$ component of the critical orbit is bounded, then the $z_2$ component is also bounded.
    \label{weak_f}
    \end{lemma}


\proof{ Since $z_1$ is bounded, there exists a large enough $M>0$ such that $z_1(n) \leq M$, for all $n \geq 0$. (Note: the value of $M$ depends on $a$ and $|c |$.) Assume $|f | $ is small enough so that
\begin{equation}
4M |f d | + 4|f |^2 |c | <1
\end{equation}
Then the discriminant of the quadratic function in $X$
$$h(X) = |f |^2X^2 + (2M|d f | -1)X + M^2 |d |^2 + |c |$$
is $\Delta = 1 - 4M |f d | - 4|f |^2 |c | > 0$. If we call $X_1<X_2$ the two distinct roots of $h(X)=0$, then the larger root $X_2$is positive. Consider $\max\{X_1,0\}<K<X_2$. Clearly, we have that $h(K)<0$. We will show inductively that $K$ is an upper bound for $z_2(n)$, for all $n \geq 0$. Indeed, $z_2(0)=0<K$. Now suppose $|z_2 | = |z_2(n) | < K$, and consider
\begin{eqnarray}
|z_2(n+1) | &=& |(d z_1 + f z_2)^2 + c | \leq |d z_1 + f z_2 | ^2 + |c |
\leq (|d | M + |f z_2 | )^2 + |c | \nonumber\\
\nonumber\\
&=& |d |^2 M^2 + 2|f d | M |z_2 | + |f |^2 |z_2 |^2 + |c | \leq |d |^2 M^2 + 2|f d | M K + |f |^2 K^2 + |c | \nonumber\\
\nonumber\\
&=& h(K)+K < K
\end{eqnarray}

\vspace{2mm}
Hence $|z_2(n+1) | <K$, and the induction is complete. In conclusion, $K$ is an upper bound for the critical component $z_2$.\\
\hfill \qed


}


    \begin{corol}
If $z_1(n+1) = (az_1)^2 + c$, and $z_2(n+1)=(dz_1+fz_2)^2+c$, with $a,d \neq 0$ and $|f |$ sufficiently small, then ${\cal M}(z_1) = {\cal M}(z_2)$.
\label{f_threshold}
    \end{corol}

    \proof{Follows easily from Lemmas~\ref{basic_lemma} and~\ref{weak_f}.}
    \hfill \qed

\vspace{3mm}
{\bf Remark.} Corollary~\ref{f_threshold} states that, if node $z_2$ depends on $z_1$, but $z_1$ does not depend on $z_2$, then a sufficiently weak self-dependence of $z_2$ will still insure M-synchronization. What weak means in this context, and how this bound depends on the other system parameters will become obvious in the proof of the lemma.\\


Figure~\ref{evolution_f} illustrates the behavior of the equi M contours for positive values of $f$, as $|f |$ decreases, when the first node is decoupled ($b=0$), for different fixed values of $a \neq 0$ and $d \neq 0$. Notice that the nodes eventually synchronize in each case for small enough values of $|f |$. In addition, as expected from the proof of Corollary~\ref{f_threshold},  the threshold depends on the values of $a$ and $d$.\\

\begin{figure}[h!]
\begin{center}
\includegraphics[width=\textwidth]{figures/evolution_f.png}
\caption{\small \emph{{\bf Equi-M Evolution of M synchrinization in a 2D CQN when changing $f$.} Each panel shows the two contours of the node-wise equi M sets, in green (for $z_1$) and in blue (for $z_2$). Top-down, the panels illustrate, respectively:.}}
\label{evolution_f}
\end{center}
\end{figure}

\section{Discussion}

In our previous work, we investigated various aspects of emerging dynamics in Coupled Quadratic Networks (CQNs), with a primary focus on understanding the relationship between network architecture and temporal behavior (as captured by the equi-M set). Through a combination of analytical and numerical approaches, we sought to characterize how different features of the network coupling influence the system’s asymptotic behavior, as reflected in the topology of the equi-M set, across both low- and high-dimensional networks.

The current paper marks a departure from that prior focus. We returned to the foundational aspects of the CQN framework and revisited more basic questions. In particular, we explored the extent to which the equi-M set captures meaningful information about the global asymptotic dynamics of the system. This is easier to do in low-dimensional coupled systems, which offer higher analytical tractability. Systems with fewer nodes exhibit simpler architectures, making it possible to derive more comprehensive results that highlight general dynamical principles in CQNs—rather than the dependence of behavior on specific architectural features. This contrasts with our previous studies, where we prioritized understanding the effects of network structure, even when working with low-dimensional systems.

In this paper, we focused specifically on two-node coupled networks and investigated the dynamical phenomena that arise in systems where architectural detail is minimal—limited to a four-parameter weight matrix. This minimalist setup enabled analytical exploration of the onset and transitions in the topological and synchronization properties of the equi-M set, expressed in terms of the four coupling parameters.

One central question we addressed was the extent to which the equi-M set of 2D CQNs replicates the hyperbolic bulb structures seen in single-variable iterated maps. To this end, we defined the pseudo-bulb of period $k \geq 1$ as the parameter region in which the critical orbit converges to an attractor of period 
$k$. To identify these periodic attractors and track the behavior of the critical orbit, we use analytical techniques for the period-1 case and numerical methods for higher periods. We then assessed whether the dynamics of the critical orbit remain representative of the full range of admissible combinatorics in the system.

Our findings show that, unlike the simple family of one-dimensional quadratic maps, coupled systems may support coexisting local attractors of identical or differing periods. Remarkably, even in the presence of such attractors, the critical orbit can escape to infinity in $\mathbb{C}^2$. This undermines any attempt to establish a one-to-one correspondence between postcritical combinatorics and pseudo-bulb structure, a correspondence that is foundational to the one-dimensional setting.

To further investigate this phenomenon, we analyzed more carefully the parameter regions where the critical orbit converges to an attracting fixed point -- what we termed the \emph{main equi-cardioid}. We conducted explicit computations across a broad range of one-parameter families, showing how specific parameter choices influence the geometry of the equi-cardioid and shape key topological features such as connectedness and the presence of a right-hand cusp. Numerical computation of the period-1 pseudo-bulb reveals that, while it aligns with the equi-cardioid boundary near the rightmost point on the real axis (typically corresponding to the cusp), it fails to include regions inside the equi-cardioid where attracting fixed points exist but the critical orbit still escapes. This discrepancy persists for higher-period attractors as well. Nevertheless, the close alignment of the period-1 pseudo-bulb and the equi-cardioid near the cusp suggests that the boundary of the equi-M set serves as a reasonable approximation for both these structures in this region. As one moves away from the cusp, however, this alignment breaks down: the postcritically bounded set no longer provides a faithful geometric approximation of the loci corresponding to higher-period dynamics. While this result is somewhat disappointing, it is not unexpected, given the greater complexity introduced by coupling.

Importantly, we find that the geometry of the equi-M set near the main cusp offers a better predictor of the system’s global combinatorics than other regions of the set. This is an encouraging find, as our earlier work linked variations in network coupling to the position and orientation of the main cusp. For instance, in our analysis of large brain networks, we observed that the main cusp could occur on either side of the equi-M set~\cite{Simone}, and follow-up studies confirmed that such shifts can be observed in theoretical networks of size greater than three~\cite{AC2}, suggesting that the shift is influenced by the number and distribution of negative weights in the network. The current result reinforces the idea that such geometric features of the equi-M set reflect meaningful aspects of the network’s underlying architecture and its associated combinatorics.

The study of two-dimensional systems also provides new insight into the synchronization of node-wise equi-M sets, a topic we explored in earlier work. While synchronization is likely a key factor in the function of complex dynamical networks, it is difficult to analyze directly in large-scale systems such as brain networks. In our previous studies, we proposed general principles for how synchronization may arise, but it remained challenging to identify specific architectural patterns that promote it. The two-node case offers a tractable model in which only two dynamical outcomes are possible, allowing us to identify key coupling parameters and analytically compute the conditions for transitions between synchronized and desynchronized behavior. These findings are valuable not only in their own right but also as a demonstration of how low-dimensional systems can yield foundational insights that are both scalable and generalizable. Our ongoing work aims to identify necessary and sufficient conditions that govern synchronization in three-node systems, where the complexity increases significantly due to added degrees of freedom and a richer combinatorial structure.

\bibliographystyle{plain}
\bibliography{references}

\end{document}
