\revres{\section{Appendix} \label{sec:appendix}}
\revres{\begin{defn} [Connected Undirected Graph] A graph is a pair $\mathcal G = (\mathcal V,\mathcal E)$ where $\mathcal V$ is is a finite set of $n$ nodes or vertices  $\{v_1,\dots,v_n\}$,  and $\mathcal E \subset \{1,\dots,n\} \times \{1,\dots,n\}$ is a set of edges. The pair $(i,j)$ denotes the edge that links vertex $v_i$ to $v_j$. If $(i,j) \in \mathcal E \Leftrightarrow (j,i) \in \mathcal E$, the graph is said to be \emph{undirected}. An undirected graph is \emph{connected} if there exists a path (set of edges) between any two nodes. The undirected graph is \textit{connected} if there exists a path (set of edges) between any two nodes; the connected egde set is denoted with $\mathcal E_c$
\end{defn}}

\revres{\begin{defn}[Adjacency and Laplacian Matrices] A (weighted) graph is fully described by the adjacency matrix, $\Lambda \in \mathbb{R}^{n \times n}$ where
\begin{align}
\Lambda_{i,j} = 
\begin{cases}\lambda_{i,j} & \text{if } (i,j) \in \mathcal E \\
0& \text{otherwise};  \end{cases}
\end{align} 
where $\lambda_{i,j}$ is the \textit{weight} assigned to any of its edges $(i,j)$. Consider the diagonal matrix $D$ whose diagonal entry $D_{i,i}=\sum_{j=1}^{n}\lambda_{i,j}$. $D$ contains the  total egdes' weight that is incident to each vertex, and is termed the \textit{degree matrix}. The graph Laplacian is defined as $L = D-\Lambda$.
\end{defn}}

\revres{\begin{defn}[Integral Graph] \label{def:int_graph} Given a parameter-varying graph $\mathcal G(\rho)=$ $(\mathcal V, \mathcal E(\rho))$ for some time-dependent parameter $\rho$, the integral graph of $\mathcal G(\rho)$ on $[0, \infty)$ is a constant graph $\bar{\mathcal G}_{[0, \infty)}:=(\mathcal V, \bar{\mathcal E})$ where $\mathcal V$ is the same vertex set of $\mathcal G(\rho)$, and the adjacency matrix is defined by
$$
\bar{\Lambda}_{i,j}= \begin{cases}1, & \text { if } \int_0^{+\infty} \lambda_{i,j}(\rho) d t=\infty \\ 0, & \text { if } \int_0^{+\infty} \lambda_{i,j}(\rho) d t<\infty\end{cases}
$$
\end{defn}}

\revres{Consider the system where the dynamics of each node $x_i$ is given by
\begin{align}
    \dot{x}_i &=\sum_{j \in \mathcal{N}(i)} \lambda_{i,j}(\rho)\left(x_j - x_i\right) ~~ i\in\{1, \dots, n\}, \label{eq:sys_graph}
\end{align}
and $\lambda_{i,j}(\rho)$ is the weight on the vertex if $(i,j) \in \mathcal{E}$ (otherwise, $\lambda_{i,j}=0$). This system corresponds to a weighted graph and can be represented in state-space form as
\begin{align}
    \dot{x}&= -L(\rho)x, \label{eq:sys}
\end{align}
where $L$ is the Laplacian. Lemma \ref{lemma:sys_cons}  holds for the system \cite{cao_consensus_2011}.}

\revres{\begin{lemma}[\cite{cao_consensus_2011}]\label{lemma:sys_cons}
The dynamics of \eqref{eq:sys} implements consensus, that is, $\lim _{t \rightarrow+\infty} x(t) \in \operatorname{span}\{\mathbf{1}\}$ if and only if $\bar{\mathcal G}_{[0, \infty)}$ is connected.
\end{lemma}}

\revres{Lemma \ref{lemma:sys_cons2}  follows directly from Definition \ref{def:int_graph}.}

\revres{\begin{lemma}\label{lemma:sys_cons2} \eqref{eq:sys} achieves consensus if its corresponding parameter-varying graph $\mathcal G(\rho)$ is connected $\forall t$, that is, $\lambda_{i,j}(\rho) \geq \lambda_l >0$ for $(i,j)\in \mathcal E_c(\rho) ~\forall t$.
\end{lemma}}

\revres{Suppose the system in \eqref{eq:sys} has a piecewise constant output $y_{{k}}\in \mathbb{R}^{m}$ over time interval $\Delta t = t_{{k+1}} - t_{k}$, the discrete-time formulation of the system may be written as}
\revres{\begin{subequations}\label{eq:sys_disc}
\begin{align}
        x_{{{k}}+1} &= \Phi_{k}x_{{k}} \\
    y_{{k}} &= Cx_{{k}}
\end{align}
\end{subequations}}
%$t_k$, $k = 1, 2,\dots$ $\Delta t = t_{k+1} - t_k$,
\revres{where $x_{k} \in \mathbb{R}^{n}$ is the state vector, $\Phi_{k} \in \mathbb{R}^{n\times n}$  and $C \in \mathbb{R}^{m \times n}$are the system state and output matrices respectively. Further, for two positive successive integers $k, {l}$ (${l}\geq k$), let the \textit{state transition matrix} be defined as $\Phi_{{l}|k} = \Phi_{{l}|{l}-1}\Phi_{{l}-1|k}$ where $\Phi_{k|k} = I_n$ and $\Phi_{k+1|k} = \Phi_{k}$. Then the system is said to be uniformly detectable if Definition \ref{def:detectability} holds.} 

\revres{\begin{defn}[Uniform Detectability \cite{anderson_detectability_1981}] \label{def:detectability} The pair $(A_{k},C)$ is uniformly detectable if there exist integers $p$, $q \geq 0$, and some $0 \leq a<1$, $b>0$ such that whenever
\begin{align}
\left\|\Phi_{k+q|k}\zeta\right\| & \geq  a\left\|\zeta\right\|
\end{align}
{for some} $\zeta \in \mathbb{R}^{n}$ and $k$, then
\begin{align}
\zeta^TW_{k+r|k}\zeta \geq b\zeta^T\zeta,
\end{align}
where $W_{k+r|k}$  is the discrete-time observability gramian given by
\begin{align}\label{eq:gramian_disc}
   W_{k+r|k} \coloneqq \sum^{k+q}_{i = k}\Phi_{{i}|k}^{T}C^TC\Phi_{{i}|k}.
\end{align}
\end{defn}}

\revres{\begin{lemma}
$(\Phi_{k}, C)$ is uniformly detectable if $C$ has at least one non-zero row sum.
\end{lemma}}

\revres{\begin{proof}
Following Definition \ref{def:detectability}, we need to evaluate {for what $v$} {
\begin{align}
    \frac{\left\|\Phi_{k+q|k}v\right\|}{\left\|v\right\|} \geq a ,  ~ a \in [0,1)\label{eq:detect_norm}
\end{align}}
 and show that for such $v$,
%\begin{align}
    %\frac{\left\|\Phi_{k+q|k}v\right\|}{\left\|v\right\|} \nless 1 , \label{eq:detect_norm}
%\end{align}
 %and show that for such $v$,
 \begin{align}
v^TW_{k+r|k}v \geq bv^Tv ~~ (b>0)\label{eq:detect_2norm}
\end{align}
for {some integer} $q$ {$\geq 0$}.
Since $\mathcal G(\rho)$ implements consensus, that is, $\lim _{t \rightarrow+\infty} x(t) = x^*\in \operatorname{span}\{\mathbf{1}_n\}$; for any $t_{k}\geq0$ ($t_{k} = k\Delta t$ ) where $ x(t_{k}) \notin \operatorname{span}\{\mathbf{1}_n\}, ~\exists$ ${a \in [0,1)}$ and $t_{{p+k}}$ with $p>0$ such that
\begin{align} \label{eq:x_ineq}
    \left\|x(t_{{p+k}}) - x^*\right\| < {a}\left\|x(t_{k}) - x^*\right\| .
\end{align}
Without loss of generality, let $v = x(t_{k}) - x^*$. \eqref{eq:x_ineq} implies $\exists ~\Phi_{k+q|k} $ such that 
\begin{align}
    \frac{\left\|\Phi_{k+q|k}v\right\|}{\left\|v\right\|} < {a} , ~~\forall ~ v \notin \operatorname{span}\{\mathbf{1}_n\} 
\end{align}
Thus, we only need to show that \eqref{eq:detect_2norm} is satisfied with $v= \mathbf{1}_n$. Since $C$ has at least one non-zero row sum, $\exists$ {some}  $b {>0} $ such that
\begin{align}
v^TW_{k+r|k}v  &= v^T\sum^{k+q}_{i = k}\Phi_{{i}|k}^{T}C^TC\Phi_{{i}|k}v \nonumber \\&= {\begin{cases} 
\mathbf{1}_n^TC^TC\mathbf{1}_n
~ (q=0)\\q \mathbf{1}_n^TC^TC\mathbf{1}_n\end{cases}} \geq b>0.
\end{align}
\end{proof}}

%$\bar{A}=\left(\bar{a}_{i j}\right)_{n \times n}$ 