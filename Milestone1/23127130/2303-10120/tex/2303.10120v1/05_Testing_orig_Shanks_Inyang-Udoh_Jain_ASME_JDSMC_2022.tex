\section{Testing and Validation}\label{sec:testing}
The objective of the following case studies is to validate the SDRE filter by demonstrating that the errors in the estimated temperatures and state of charge converge toward zero and remain bounded.  
%Although the SDRE filter is capable of real-time estimation on the experimental testbed, we perform experimental validation offline so that multiple filter configurations can be tested on the same data.
We first collect a dataset containing temperature measurements from the seven thermocouples and a measurement of the mass flow rate on the experimental testbed.  The mass flow rate through the TES is controlled by manually opening and closing the flow control valves.
%, so $\dot{m}$ must be measured; if a feedback controller were used, $\dot{m}$ would be the control input. 
Fig.\ \ref{fig:mass_flow_rate} shows the measured mass flow rate through the TES.  Fig.\ \ref{fig:fluid_in_out} shows the temperatures of the fluid entering and leaving the TES measured by TC0 and TC1, respectively. Note that periods of zero flow rate were intentionally introduced as it would be reasonable for the TES to only be charged or discharged intermittently.  Conductive heat transfer still occurs inside the TES even when the mass flow rate is zero, so the state estimator must still work during these periods of inactivity to track the changing temperature distribution.  In the subsequent figures, blue shading denotes periods of zero mass flow rate when this information is relevant to the result presented in the figure.
%The performance of the SDRE filter with three different sampling rates is investigated.
\begin{figure}[tpb]
    %\centering
    \includegraphics[width=3.1in]{Figures/Exp Results/mass_flow_rate.pdf}
    \caption{Measured mass flow rate through the TES. Blue shading in this figure and subsequent figures highlights periods of zero mass flow rate.}
    \label{fig:mass_flow_rate}
\end{figure}
\begin{figure}[tpb]
    %\centering
    \includegraphics[width=3.1in]{Figures/Exp Results/fluid_in_out.pdf}
    \caption{Selected fluid temperature measurements.}
    \label{fig:fluid_in_out}
\end{figure}

As it is not possible to validate the SOC estimate for the experimental TES module with only five thermocouples embedded in the PCM, we first run a simulation with the high-fidelity simulation model validated in \cite{gohil_reduced-order_2020} and \cite{shanks_design_2022}, which has the same finite volume structure as the state estimator model with a finer spatial discretization of $n_x = 21$ and $n_y = 22$.  The simulation is executed in MATLAB using the \texttt{ode23} solver \cite{shampine_matlab_1997}.  The simulation receives only the experimentally-measured mass flow rate and fluid inlet temperature (TC0) as inputs to generate a simulated dataset containing the SOC and temperatures of all control volumes.  We then test the state estimator on the simulated dataset, using simulated values of the thermocouple measurements, to establish convergence of the SOC estimate. 

The reason for conducting a simulation-based test of the SDRE filter using experimental inputs to the model is to enable a comparison with the experimental validation and draw conclusions about those estimation errors that are due to modeling error.  To analyze convergence of the temperature estimates experimentally, select thermocouple measurements are withheld from the state estimator to be used for validation.  However, we expect that using fewer thermocouples will degrade the accuracy of the estimates, so we first use the simulation dataset to quantify the estimation error when selected thermocouple measurements are withheld.  We expect the error between the state estimates and the simulated dataset to be comparable to the error between the state estimates and the experimental measurements.  Any additional error in the experimental results would therefore be due to modeling inaccuracies.  

%Let $\mathcal{Y} = \left\{Y_{TC0},\,Y_{TC1},\,\dots,Y_{TC4b}\right\}$ denote the set of measurements from all thermocouples for all sampling times $k$.  , for example $\mathcal{Y}\setminus Y_{TC1}$,

The simulation-based case study is followed by validation of the estimator's convergence and boundedness on the experimental testbed as discussed in Section XXX. Note that although the SDRE filter is capable of real-time estimation on the testbed, we perform experimental validation offline so that multiple filter configurations can be tested on the same data. 

\subsection{Simulation-Based Testing}

To test the state estimator offline, we first collect a dataset containing temperature measurements from the seven thermocouples and a measurement of the mass flow rate on the experimental testbed.  The mass flow rate through the TES is controlled by manually opening and closing the flow control valves, so $\dot{m}$ must be measured; if a feedback controller were used, $\dot{m}$ would be the control input. Fig.\ \ref{fig:mass_flow_rate} shows the measured mass flow rate through the TES.  Fig.\ \ref{fig:fluid_in_out} shows the temperatures of the fluid entering and leaving the TES measured by TC0 and TC1, respectively. Note that periods of zero flow rate were intentionally introduced as it would be reasonable for the TES to only be charged or discharged intermittently. Conductive heat transfer still occurs inside the TES even when the mass flow rate is zero, so the state estimator must still work during these periods of inactivity to track the changing temperature distribution.
% \begin{figure}[tpb]
%     %\centering
%     \includegraphics[width=3.1in]{Figures/Exp Results/mass_flow_rate.pdf}
%     \caption{Measured mass flow rate through the TES.}
%     \label{fig:mass_flow_rate}
% \end{figure}
% \begin{figure}[tpb]
%     %\centering
%     \includegraphics[width=3.1in]{Figures/Exp Results/fluid_in_out.pdf}
%     \caption{Selected fluid temperature measurements.}
%     \label{fig:fluid_in_out}
% \end{figure}

Using the high-fidelity simulation as a validation dataset, we compare the estimated states to the simulated states to establish convergence of the estimates and determine sources of estimation error.  The simulation calculates the temperature distribution inside the TES, converts the high-fidelity temperature distribution to a set of temperatures corresponding to the state estimator model's control volumes, and generates ``thermocouple measurements'' at discrete time steps by adding Gaussian noise to the temperatures of those control volumes that map to the measurement locations on the experimental TES module. These simulated thermocouple measurements comprise the measurement vector $y_k$ for the state estimator; the thermocouple locations and the labels of the corresponding control volumes are given in Fig.\ \ref{fig:TC_CV_labels}.  
%To maintain consistency with future case studies, the simulation takes the measured fluid inlet temperature and mass flow rate from the experimental dataset as inputs to the model.  Because it does not use measurements of the PCM temperature or the fluid outlet temperature, we call this an open-loop simulation (OLS).  Values from the open-loop simulation are denoted ``OLS", and values estimated by the state estimator (SE) are denoted ``SE".  All estimate errors are calculated as $E = x_{SE} - x_{OLS}$.

% \subsubsection{SOC Convergence}
% In this section, we analyze the SOC estimate and establish that it converges to the same SOC calculated by the open-loop simulation. 

%%%  Turned off track changes and moved SOC discussion to after the temperature estimation error analysis

%\subsubsection{Temperature Estimate Error}
First, we verify that the temperature estimates of unmeasured control volumes converge toward the temperatures calculated by the simulation, and the estimation errors remain bounded.  In this case study, the state estimator receives the full measurement output vector at a rate of 10 S/s.  Fig.\ \ref{fig:TCall_sim_a} compares the estimated temperatures of the control volumes labeled CV5 and CV6 (see Fig.\ \ref{fig:TC_CV_labels} for the locations of these control volumes) to the corresponding simulated temperatures.  The gray shading represents the range of temperatures over which the latent heat of fusion is approximated by a large increase in specific heat. Since it is difficult to distinguish the estimates from the simulated values in Fig.\ \ref{fig:TCall_sim_a}, we show the estimation errors in Fig.\ \ref{fig:TCall_sim_b}.  Note that all estimation errors are calculated as $e(t_k)=\hat{x}(t_k)-x_{sim}(t_k)$ where $x_{sim}$ is a state variable calculated by the simulation. 

Even though the temperatures of CV5 and CV6 are not included in the output vector $y_k$, the SDRE filter is able to track these temperatures with an estimation error magnitude less than 0.6 K. During periods when the temperatures are changing slowly, the estimation error magnitude is less than 0.05 K, which is less than one standard deviation of the measurement noise ($\sqrt{0.0035\textup{ K}^2} = 0.059\textup{ K}$).  In fact, the largest estimation errors only occur during phase changes; the large negative transient estimation errors correspond to periods when the PCM is melting, suggesting that the SDRE filter introduces a small phase lag between the estimates and the true values.  This claim is supported by the observation that the estimation error has a larger negative value when the temperatures change rapidly, such as at $t = 1050$ seconds; a faster transient change results in more phase lag, which increases the transient estimation error.  There are fewer periods of large positive error because the PCM only fully solidifies at the start ($t=50$ seconds) and end ($t=1650$ seconds) of the dataset; this means that the largest transient errors are expected when the PCM completes a phase change, but not during a partial phase change.
\begin{figure}[tb]
    \centering
    \begin{subfigure}[b]{\columnwidth}
        %\centering
        \includegraphics[width=3.1in]{Figures/Exp Results/TCall_sim.pdf}
        \caption{}
        \label{fig:TCall_sim_a}
    \end{subfigure}
    \begin{subfigure}[b]{\columnwidth}
        %\centering
        \includegraphics[width=3.1in]{Figures/Exp Results/TCall_sim_err.pdf}
        \caption{}
        \label{fig:TCall_sim_b}
    \end{subfigure}
    \caption{(a) Estimated temperatures (red) and simulated temperatures (black) at CV5 and CV6; Gray shading denotes the range of temperatures corresponding to the latent region, and the black line is the melting point of the PCM. (b) Estimation errors.}
    \label{fig:TCall_sim}
\end{figure}

Next, we quantify the temperature estimation error when the simulated measurement from (i) TC1 or (ii) TC3 is withheld from the state estimator; when testing the state estimator on experimental data, these thermocouples are withheld to be used for validation.  Fig.\ \ref{fig:TC1_sim_a} shows the temperature at CV1 calculated by the simulation and estimated by the SDRE filter when TC1 is withheld, and Fig.\ \ref{fig:TC1_sim_b} shows the corresponding estimation error.  As expected, the estimation error is more substantial when an output is removed; the error magnitude tends to be less than 0.5 K when the control volume temperature is changing slowly, but it increases to nearly 2 K when the temperature changes suddenly, like at $t=1050$ seconds or $t=1200$ seconds.  The steady-state errors from $t=850$ to $t=1000$ seconds and $t=1350$ to $t=1550$ seconds are due to the zero mass flow rate.  Since advection is the dominant form of heat transfer in the fluid channel, when advection is removed, heat transfer in the fluid control volumes is greatly reduced, and it takes longer for the temperature estimate to converge. %Since CV1 is a fluid control volume, its estimate error is expected to be large when the mass flow rate is zero because heat transfer between fluid control volumes is dominated by advection.
\begin{figure}[tb]
    \centering
    \begin{subfigure}[b]{\columnwidth}
        %\centering
        \includegraphics[width=3.1in]{Figures/Exp Results/TC1_sim.pdf}
        \caption{}
        \label{fig:TC1_sim_a}
    \end{subfigure}
    \begin{subfigure}[b]{\columnwidth}
        %\centering
        \includegraphics[width=3.1in]{Figures/Exp Results/TC1_sim_err.pdf}
        \caption{}
        \label{fig:TC1_sim_b}
    \end{subfigure}
    \caption{(a) Simulated temperature at CV1 and the estimated temperature for CV1. (b) Estimation error.}
    \label{fig:TC1_sim}
\end{figure}

Fig.\ \ref{fig:TC3_sim_a} shows that the estimated temperature for the control volume CV3 still tracks the simulated temperature for this control volume closely when TC3 is withheld from the estimator; Fig.\ \ref{fig:TC3_sim_b} shows that the error tends to be largest near the melting point and during periods of zero mass flow rate.  The estimation error magnitude stays within 1 K, but it only converges to zero outside the latent temperature range.  
\begin{figure}[tb]
    \centering
    \begin{subfigure}[b]{\columnwidth}
        %\centering
        \includegraphics[width=3.1in]{Figures/Exp Results/TC3_sim.pdf}
        \caption{}
        \label{fig:TC3_sim_a}
    \end{subfigure}
    \begin{subfigure}[b]{\columnwidth}
        %\centering
        \includegraphics[width=3.1in]{Figures/Exp Results/TC3_sim_err.pdf}
        \caption{}
        \label{fig:TC3_sim_b}
    \end{subfigure}
    \caption{(a) Simulated temperature at CV3 and the estimated temperature for CV3. (b) Estimation error.}
    \label{fig:TC3_sim}
\end{figure}

Fig.\ \ref{fig:RMSE_sim} shows the root-mean-square error (RMSE) averaged over all $n$ control volumes at time step $t_k$, given in Eqn.\ \eqref{eq:RMSE}.  In this figure, the RMSE is shown for the state estimator with full access to all measurements, the state estimator with access to all measurements except TC1, and the state estimator with access to all measurements except TC3.  The RMS errors of the estimator with all measurements and the estimator without TC1 remain below 0.4 K (except at 0 seconds, when the state estimates are initialized) and are nearly indistinguishable; this indicates that the measurement from TC1 has a negligible effect on the estimates of the other control volumes in the model.  Removing TC3 slightly increases the RMSE during phase changes and periods of zero mass flow rate.  This slight increase in error should also happen when testing on the experimental dataset.
\begin{equation}
    e_{rms}(t_k)=\left[\sum_{i=1}^{n}\frac{\left(\hat{x}_i(t_k)-x_i(t_k)\right)^2}{n}\right]^\frac{1}{2}\label{eq:RMSE}
\end{equation}
\begin{figure}[tb]
    %\centering
    \includegraphics[width=3.1in]{Figures/Exp Results/RMSE_sim.pdf}
    \caption{RMS errors of all control volumes for the state estimator with all available measurements, all measurements except TC1, and all measurements except TC3. }
    \label{fig:RMSE_sim}
\end{figure}

%%% Moved text from earlier here.  Now using track changes for editing

Although it is important to analyze the error of individual temperature estimates, the overarching goal of the proposed estimator is to estimate the SOC of the TES for online control decision-making. Fig.\ \ref{fig:SOC_sim_a} compares the simulated and estimated SOC values calculated from the temperature estimates of the SDRE filter with access to all outputs. The shading in Fig.\ \ref{fig:SOC_sim_a} represents the range of SOC in which all or part of the PCM is undergoing a phase change.  
%Shading in Fig.\ \ref{fig:OLS_temp} indicates the temperature range over which the latent heat is applied to simulate the phase change; the black line is the melting point of the PCM.  Near the melting point, small changes in temperature result in large changes in SOC;  
Compare Fig.\ \ref{fig:SOC_sim_a} to Fig.\ \ref{fig:OLS_temp}, which shows simulated PCM temperatures for three control volumes over the duration of the simulation; refer to Fig.\ \ref{fig:TC_CV_labels} for the location of these control volumes. Note that the SOC is near 1 when the PCM temperatures are low and near 0 when the PCM temperatures are high. In Fig.\ \ref{fig:OLS_temp}, the black line is the melting point of the PCM near which small changes in temperature result in large changes in SOC.

This sensitivity near the melting point is also evident in Fig.\ \ref{fig:SOC_sim_b}, which shows the SOC estimation error.  The error is maximized when the PCM temperatures are near the melting point and the SOC is within the latent heat range.  Despite the nonlinear relationship between SOC and temperature, the estimated SOC tracks the simulation well, and the estimate error remains within $\pm$0.02.  Outside the latent range when the system dynamics are approximately linear (between $t=50$ and $t=380$ seconds, for example), the SOC error converges to zero (although there is some noise with amplitude less than 0.001).  The SOC error tends to be positive when the SOC is decreasing and negative when the SOC is increasing, which indicates that there is a small phase lag between the SOC estimate and the true SOC.  Additionally, the error does not converge to zero when the mass flow rate through the TES is zero and the PCM temperatures are within the latent range (for example, between $t=850$ and $t=1000$ seconds).  This is because heat transfer along the length of the module is dominated by advection. When the mass flow rate is zero, it takes longer for the PCM to reach an equilibrium temperature, which means it takes longer for temperature estimation errors to converge to zero.  Small temperature errors near the phase change temperature result in large errors in SOC.  Outside the latent temperature range, the SOC error converges to zero when the mass flow rate is zero---see the period between $t=1050$ and $t=1200$ seconds for an example.

\begin{figure}[tb]
    \centering
    \begin{subfigure}[b]{\columnwidth}
        \includegraphics[width=3.1in]{Figures/Exp Results/SOC_sim.pdf}
        %\centering
        \caption{}
        \label{fig:SOC_sim_a}
    \end{subfigure}
    \begin{subfigure}[b]{\columnwidth}
        \includegraphics[width=3.1in]{Figures/Exp Results/SOC_sim_err.pdf}
        %\centering
        \caption{}
        \label{fig:SOC_sim_b}
    \end{subfigure}
    \caption{(a) Simulated and estimated SOC; gray shading represents the latent SOC region where errors are expected to be large. (b) SOC estimation error; blue shading represents periods when the mass flow rate is zero. }
    \label{fig:SOC_sim}
\end{figure}

\begin{figure}[tb]
    %\centering
    \includegraphics[width=3.1in]{Figures/Exp Results/sim_PCM_T.pdf}
    \caption{Simulated temperature values from select PCM control volumes.}%  Shading denotes the range of temperatures corresponding to the latent region, and the black line is the melting point of the PCM.}
    \label{fig:OLS_temp}
\end{figure}

\subsection{Experimental Validation}\label{sec:exp_testing}
In this case study, we test the SDRE filter on experimental data.  First, we investigate convergence of the SDRE filter estimates when (i) thermocouple TC1 is withheld for validation and (ii) thermocouple TC3 is withheld for validation.  We also investigate how the SDRE filter performs with different sample rates.  All estimation errors are calculated as $e(t_k) = \hat{x}(t_k)- y_{k,TCj}$ where $y_{k,TCj}$ is a measurement from an excluded thermocouple.

%\subsubsection{Temperature Estimate Validation}
When TC1, the fluid outlet thermocouple measurement, is excluded from the measurement set, the state estimator tracks the fluid outlet temperature closely over long time periods, as shown in Fig.\ \ref{fig:TC1_exp_a}. However, the estimation error increases when the temperature changes rapidly, as shown in Fig.\ \ref{fig:TC1_exp_b} at $t=1000$ seconds, $t=1200$ seconds, and $t=1550$ seconds. For most of the duration of the dataset, the magnitude of the estimation error at CV1 is less than 1 K.  This error behavior is similar to that of the simulated results shown in Fig.\ \ref{fig:TC1_sim_b}, but the larger magnitude of the error when implementing the estimator on the experimental testbed is due to modeling error, which includes inaccuracies in the reduced-order finite volume model \cite{gohil_reduced-order_2020, shanks_design_2022}, simplifications of the phase-change dynamics such as neglecting undercooling and hysteresis \cite{sgreva_thermo-physical_2022, barz_paraffins_2021}, and exogenous disturbances in the physical system like heat transfer with the surroundings.
\begin{figure}[tb]
    \centering
    \begin{subfigure}[b]{\columnwidth}
        \includegraphics[width=3.1in]{Figures/Exp Results/TC1_exp.pdf}
        \caption{}
        \label{fig:TC1_exp_a}
    \end{subfigure}
    \begin{subfigure}[b]{\columnwidth}
        \includegraphics[width=3.1in]{Figures/Exp Results/TC1_exp_error.pdf}
        \caption{}
        \label{fig:TC1_exp_b}
    \end{subfigure}
    \caption{(a) Estimated temperature of CV1 compared to the fluid outlet temperature measurement TC1. (b) Estimate error for CV1.}
    \label{fig:TC1_exp}
\end{figure}

Excluding TC3 from the measurement set results in degradation of the estimator's performance, as expected based on the simulated results. The estimate of CV3 still tracks TC3 closely over long time periods, as Fig.\ \ref{fig:TC3_exp_a} shows.  When the TES is experiencing rapid changes in temperature, Fig.\ \ref{fig:TC3_exp_b} shows that the magnitude of the error tends to increase; the convergence rate of the estimate error of CV3 is slower than the transient dynamics, resulting in larger errors up to 6 K at $t=1050$ seconds and 3 K at $t=1550$ seconds.  During the period between $t=450$ and $t=900$ seconds and the period between $t=1200$ and $t=1400$ seconds, the PCM undergoes partial phase change.  Estimation errors during these periods of time are due to the phase change hysteresis and undercooling effects which delay solidification when the PCM is cooled rapidly \cite{barz_paraffins_2021}.  The system model does not account for these effects, so the estimator incorrectly predicts that CV3 (and other unmeasured control volumes) begins to solidify at 293.5 K (the top boundary of the gray shaded region in Fig.\ \ref{fig:TC3_exp_a}) whereas the PCM in the experimental TES begins to solidify at a lower temperature.  Over sufficiently long time periods during which the PCM temperatures reach equilibrium, such as $t=750$ to $t=1000$ seconds, errors caused by hysteresis and undercooling are corrected and the estimate error returns to zero.
\begin{figure}
    \centering
    \begin{subfigure}[b]{\columnwidth}
        \includegraphics[width=3.1in]{Figures/Exp Results/TC3_exp.pdf}
        \caption{}
        \label{fig:TC3_exp_a}
    \end{subfigure}
    \begin{subfigure}[b]{\columnwidth}
        \includegraphics[width=3.1in]{Figures/Exp Results/TC3_exp_err.pdf}
        \caption{}
        \label{fig:TC3_exp_b}
    \end{subfigure}
    \caption{(a) Estimated temperature of CV3 compared to the measurement of TC3. (b) Estimate error for CV3.}
    \label{fig:TC3_exp}
\end{figure}

\subsection{Effect of Sample Rate}\label{sec:sample_rate}
In this section, we investigate how the SDRE filter performs when measurements are provided at three different rates: 80 samples per second (S/s), 10 S/s, and 1 S/s.  All previous experimental and simulation results used a sample rate of 10 S/s.  The continuous-discrete SDRE formulation allows for more flexibility with sample rates because the state estimates and the estimate covariance matrix are propagated with the state transition matrix of the continuous-time system; estimation errors introduced by discretizing the nonlinear system are mitigated by propagating the state estimates and error covariance with a time step smaller than the interval between samples.

Fig.\ \ref{fig:TC1_exp_SR} shows (a) the estimated temperatures at CV1 and (b) the estimation errors when TC1 is excluded from the measurement set during a validation test with the experimental dataset.  As discussed in the previous section, the state estimator can track the temperature at CV1 well; this is also the case at the faster and slower sample rates.  In Fig.\ \ref{fig:TC1_exp_SR_b}, none of the three sample rates appears to have a smaller error than the others.  To more easily compare the performance of the three sample rates, we calculate the root-mean-square errors for the temperature at CV1 averaged over all time steps using Eqn.\ \eqref{eq:RMSE_t} where $N_s$ is the total number of time steps in the dataset. Results for the three sample rates are included in Table \ref{tab:RMSE_TC1}.  Note that this RMSE definition represents an average over all time steps, but the previous RMSE definition in Eqn.\ \eqref{eq:RMSE} averages over all control volumes.  As expected the 80 S/s rate results in the smallest estimation error, but only by a small margin; 10 S/s results in a similar RMSE.
\begin{equation}\label{eq:RMSE_t}
    e_{rms,CVj}=\left[\sum_{k=1}^{N_{s}}\frac{\left(\hat{x}_{j}(t_k)-y_{k,TCj}\right)^2}{N_{s}}\right]^\frac{1}{2}
\end{equation}
\begin{table}[htbp]
\centering
\caption{RMS errors at CV1 for three sample rates}
\label{tab:RMSE_TC1}
\begin{tabular}{r|l}
Sample Rate  & $e_{rms,CV1}$ (K) \\ \hline
80 S/s         & 0.2436     \\
10 S/s         & 0.2617     \\
1 S/s          & 0.3811    
\end{tabular}
\end{table}

Fig.\ \ref{fig:SOC_exp_TC1_SR} shows the estimated SOC for the three sample rates.  All three estimates are similar with almost no difference outside the latent range. During the periods of partial phase change, a noticeable difference in the peak of the SOC estimate can be seen at $t=500$ seconds, $t=650$ seconds, and $t=1250$ seconds; the faster sample rates reach a sharper and higher peak SOC, suggesting that the faster sample rate results in a more precise estimate and a faster response time to transient changes.

\begin{figure}[tb]
    \centering
    \begin{subfigure}[b]{\columnwidth}
        \includegraphics[width=3.1in]{Figures/Exp Results/TC1_exp_SR.pdf}
        \caption{}
        \label{fig:TC1_exp_SR_a}
    \end{subfigure}
    \begin{subfigure}[b]{\columnwidth}
        \includegraphics[width=3.1in]{Figures/Exp Results/TC1_exp_err_SR.pdf}
        \caption{}
        \label{fig:TC1_exp_SR_b}
    \end{subfigure}
    \caption{(a) Estimated temperatures of CV1 compared to the fluid outlet temperature measurement TC1. (b) Estimation errors for CV1.}
    \label{fig:TC1_exp_SR}
\end{figure}
\begin{figure}[tb]
    %\centering
    \includegraphics[width=3.1in]{Figures/Exp Results/SOC_exp_TC1.pdf}
    \caption{Estimated SOC for the three sample rates when TC1 is withheld.}
    \label{fig:SOC_exp_TC1_SR}
\end{figure}

Finally, we compare the performance of the three sample rates when TC3 is excluded from the measurement set.  Fig.\ \ref{fig:TC3_exp_SR} shows (a) the estimated temperatures at CV3 and (b) the estimate errors.  In Fig.\ \ref{fig:TC3_exp_SR_b}, it is clear that the state estimator receiving 80 S/s has a slightly smaller error most of the time.  This is noticeable at $t=500$ seconds, $t=1100$ seconds, and $t=1600$ seconds.  The RMSE of CV3 averaged over all time steps is given in Table \ref{tab:RMSE_TC3}. The state estimator with a sample rate of 80 S/s again performs the best, but by a small margin.  Note that these RMS errors are larger than those in Table \ref{tab:RMSE_TC1}; this indicates that the measurement from TC3 is more critical to estimate the SOC accurately than that of TC1.  This makes sense given that TC3 is located near the center of the TES; without the measurement from TC3, only measurements of the temperatures near the edge of the TES are available. Estimation errors in CV3 and the control volumes above CV3 can be large for long time periods (greater than 2 K for more than 20 seconds, according to Fig.\ \ref{fig:TC3_exp_SR_b}) before the state estimator compensates for the errors.
\begin{figure}[tb]
    \centering
    \begin{subfigure}[b]{\columnwidth}
        \includegraphics[width=3.1in]{Figures/Exp Results/TC3_exp_SR.pdf}
        \caption{}
        \label{fig:TC3_exp_SR_a}
    \end{subfigure}
    \begin{subfigure}[b]{\columnwidth}
        \includegraphics[width=3.1in]{Figures/Exp Results/TC3_exp_err_SR.pdf}
        \caption{}
        \label{fig:TC3_exp_SR_b}
    \end{subfigure}
    \caption{(a) Estimated temperature of CV3 compared to the measurement of TC3. (b) Estimation errors for CV3.}
    \label{fig:TC3_exp_SR}
\end{figure}
\begin{table}[htbp]
\centering
\caption{RMS errors at CV3 for three sample rates}
\label{tab:RMSE_TC3}
\begin{tabular}{r|l}
Sample Rate & $E_{rms,CV3}$ (K) \\ \hline
80 S/s         & 1.0022     \\
10 S/s         & 1.1046     \\
1 S/s          & 1.2048    
\end{tabular}
\end{table}

Fig.\ \ref{fig:SOC_exp_TC3_SR} shows the estimated SOC for the three sample rates.  The results shown in this figure are almost identical to those in Fig.\ \ref{fig:SOC_exp_TC1_SR}, except there is less difference between the 80 S/s results and the 10 S/s results.  These results demonstrate an important advantage of the continuous-discrete SDRE filter over other discrete-time state estimators.  Since the state estimates are propagated with a small time step, accuracy of the prediction step is not degraded as the time interval between update steps is increased.
\begin{figure}[tb]
    \centering
    \includegraphics[width=3.1in]{Figures/Exp Results/SOC_exp_TC3.pdf}
    \caption{Estimated SOC for the three sample rates when TC3 is withheld.}
    \label{fig:SOC_exp_TC3_SR}
\end{figure}