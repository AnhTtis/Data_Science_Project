\section{State-Dependent Riccati Equation Filter} \label{sec:SDRE}

In \cite{mracek_new_1996}, Mracek et al.\ discuss the continuous-time formulation of the SDRE filter in which the observer gain is obtained by solving the continuous-time state-dependent Riccati equation \cite{mracek_new_1996}. In \cite{jaganath_sdre-based_2005}, Jaganath et al.\ present the discrete-time SDRE filter in which the discrete-time Riccati equation is solved recursively with prediction and update steps resembling the EKF algorithm.  For systems with continuous dynamics and discrete sampling, Berman et al.\ \cite{berman_comparisons_2014} and Inyang-Udoh et al.\ \cite{inyang-udoh_sampling_2022} describe the continuous-discrete SDRE filter in which the state estimates and covariance matrix are propagated in discrete time steps using a state-dependent state transition matrix. In this section we outline the continuous-discrete SDRE filter algorithm and describe its application to the latent TES state estimation problem. As we will show in Section \ref{sec:sample_rate}, the continuous-discrete formulation can be particularly advantageous when the sample rate is limited.

Let the input $u = \dot m(t)$ be constant over each interval $\Delta t_k = t_{k+1} - t_k$. Further assume some process noise ${w}_k \sim \mathcal{N}({0}, W_k)$ and measurement noise ${v}_{k} \sim \mathcal{N}({0}, V_{k})$ are present, where $W_k \in \mathbb{R}^{n\times n}$, $V_{k} \in\mathbb{R}^{p\times p}$ are covariance matrices.  If the system dynamics change little during the interval $\Delta t_k$, the linear parameter-varying state-space model can be ``frozen-in-time'' (see \cite{inyang-udoh_strongly_2022}) and discretized using the exact solution for the state transition matrix of a linear dynamic system.  The  discrete-time system is given by %can be approximated in discrete time as
\begin{subequations}\label{eq:sys_nonlin}
    \begin{align} 
        x_{k+1} &= \Phi_kx_k + \Gamma_{k}u_k + w_k \\
        y_{k} &= Cx_{k} + v_{k},    \end{align}\label{eq:discrete_sys}
\end{subequations}
where 
\begin{subequations}
    \begin{align}
        \Phi_k &= e^{A(x_k)\Delta t_k},\\
        \Gamma_k &= \int_{0}^{ \Delta t_{k}}e^{A(x_k)\tau}B(x_k)d\tau.
\end{align}\label{eq:discrete_mat}
\end{subequations}

To estimate the unmeasured temperature states of the TES, we use the two-step recursive formulation of the SDRE filter given in \cite{jaganath_sdre-based_2005} and \cite{berman_comparisons_2014} where the predicted state $\hat{x}_{k+1|k}$ and estimation error covariance $P_{k+1|k}$ are given by
\begin{subequations}
    \begin{align}
        \hat{x}_{k+1|k} &= \Phi_k\hat{x}_k + \Gamma_{k}{u}_k,\\
        P_{k+1|k} &= \Phi_k P_{k|k}\Phi_{k}^\top + W_k,
    \end{align}\label{eq:prop}
\end{subequations}
the Kalman gain $K_{k}$ is given by
\begin{equation}
    K_{k} = P_{k+1|k}C^\top\left(CP_{k+1|k}C^\top+V_k\right)^{-1},
    \label{eq:kalman_gain}
\end{equation}
and the updated state, output and covariance estimates are 
\begin{subequations}
    \begin{align}
        \hat{x}_{k+1|k+1} &= \hat{x}_{k+1|k} + K_k\left(y_k-\hat{y}_k\right),\\ 
        \hat{y}_k &= C\hat{x}_k, \\
        P_{k+1|k+1} &= \left(I-K_kC\right)P_{k+1|k}.
    \end{align} \label{eq:update}
\end{subequations}

Using a smaller time interval $\Delta t_k$ in the discrete-time model results in a better approximation of the continuous nonlinear dynamics.  If measurements are not available at every time step $t_k$, the prediction step in Eqn.\ \eqref{eq:prop} can be performed multiple times between each update step; we call this method the \textit{continuous-discrete} SDRE filter \cite{berman_comparisons_2014, inyang-udoh_sampling_2022} although it still uses a discrete approximation of the continuous dynamics.  

Now we establish the boundedness of the SDRE filter when applied to the latent TES under consideration.   
%The following proposition proves that the nonlinear finite-volume model is uniformly detectable; see Inyang-Udoh et al.\ \cite{inyang-udoh_strongly_2022} for the proof.  Since the graph-based representation of the TES defined in Eqn.\ \ref{eq:TES_ss} represents a connected graph (see Proposition \ref{prop:graph}), the SDRE error covariance for the TES state estimation will be bounded.

\begin{proposition}\label{prop:bounded} The estimation error covariance of the SDRE filter for the TES model given by Eqn.\ \eqref{eq:TES_ss} will remain bounded.
\end{proposition}

\begin{proof}
For the discrete-time system of Eqn.\ \eqref{eq:discrete_sys}, the SDRE error covariance is bounded if the pair $\left(\Phi_k, C\right)$ is uniformly detectable \cite{anderson_detectability_1981}. The TES model given by Eqn.\ \ref{eq:TES_ss} represents a \revres{continuously connected undirected graph since $A_{i,j}(x) >0 ~\forall t$ for each existing edge  $(i,j)$ where $i\neq j $. If a parameter-varying undirected graph is continuously connected $\forall t$, the graph implements consensus, and is, hence, uniformly detectable from any node (see the Appendix). Furthermore, the pair $\left(\Phi_k, C\right)$ is uniformly detectable for any $C$ with at least
one non-zero row sum (see Lemma 3 in the Appendix).} Therefore, it follows that the error covariance of the state estimate will remain bounded.
%as given by Proposition \ref{prop:graph}
%Furthermore, for a connected undirected graph with weighted Laplacian $A(x)$ and binary output matrix $C$, for some $\Delta t_k$, the discrete-time, linear parameter-varying system $\left(\Phi_k = e^{A(x_k)\Delta t_k},\,C\right)$ is uniformly detectable \textup{\cite{inyang-udoh_strongly_2022}}.
\end{proof}

\begin{remark}
\textup{Proposition \ref{prop:bounded} can be generalized to show that any thermal energy storage device that can be spatially discretized into a contiguous lattice of control volumes (that is, the thermal resistance between any two control volumes is finite) is uniformly detectable given that at least one state measurement exists.  In other words, when the SDRE filter is used for state estimation, boundedness of the state estimate error covariance is guaranteed.}
\end{remark}


%It can be shown that the discrete-time system is marginally stable for any $x_k$ by noting that the Laplacian matrix $L(x)$ is symmetric and diagonally dominant and therefore positive semidefinite. Moreover, the capacitance matrix $M(x)$ is diagonal and positive definite; therefore, $A(x) = -M(x)^{-1}L(x)$ has no eigenvalues in the open right half-plane, and $\Phi_k$ has no eigenvalues with magnitude greater than 1.