\section{Introduction} \label{sec:intro}

There is growing interest in the integration of thermal energy storage (TES) devices with a variety of thermal-fluid systems for improving performance. However, new control strategies are needed to take full advantage of the potential benefits of TES devices in fast-timescale transient thermal management systems (TMSs) \cite{shanks_control_2022, shafiei_model_2015, pangborn_hierarchical_2020, barz_state_2018, zsembinszki_evaluation_2020}. Control strategies for many forms of TMSs with integrated latent thermal storage, or \emph{hybrid} thermal management systems, have been studied in both experimental and simulation environments.  In \cite{shanks_control_2022}, Shanks et al.\ demonstrate a logic-based controller for state of charge management in a simulated single-phase hybrid TMS for aircraft electronics cooling.  In \cite{shafiei_model_2015}, Shafiei and Alleyne develop a model predictive controller in simulation for a two-phase hybrid TMS in a transport refrigeration system. Similarly, Pangborn et al.\ develop a model predictive controller for a vehicle TMS with distributed TES and demonstrate it in simulation \cite{pangborn_hierarchical_2020}.  In each of these examples, state estimation and measurement are neglected because the controllers are incorporated into the simulation model, and all system states are known.  However, in practice, real-time controllers for TES require accurate state estimation, including real-time estimation of the internal temperatures and the state of charge (SOC) \cite{barz_state_2018, zsembinszki_evaluation_2020}. Experimental testing of control strategies for fast-timescale hybrid TMS, such as vehicle TMS, is limited; therefore, development and experimental validation of real-time model-based state estimation techniques for latent TES devices will contribute to filling this gap in the literature.   

For latent TES containing phase-change materials (PCM), the SOC is typically defined as a function of either (i) the fraction of PCM in either the liquid or solid phase (phase fraction) or (ii) the amount of energy stored \cite{zsembinszki_evaluation_2020}.  Methods for direct measurement of the phase fraction definition have been developed for various types of TES \cite{zsembinszki_evaluation_2020, paberit_detecting_2016}, but direct measurement of the stored energy is difficult when the stored energy can be in the form of \emph{both} latent and sensible heat. Instead, methods for determining this SOC definition must rely on indirect measurements and model-based estimators \cite{beyne_estimating_2022}.  Dynamic state estimators such as the Kalman filter or Luenberger observer are commonly used to estimate unmeasured or unmeasurable states in dynamic systems given limited measurements \cite{kalman_new_1960, luenberger_observers_1966}.  In latent TES systems, temperature and SOC estimation is particularly difficult because the temperature of the PCM is nearly constant during the phase change process when the PCM exchanges most of its energy.  In addition, the nonlinear or hybrid dynamics associated with the phase change increase the complexity of the dynamic model, making observability, stability, and convergence difficult to guarantee in advance. 

% Building cooling systems with ice storage is another form of hybrid TMS which has been thoroughly studied \cite{tam_development_2019, henze_experimental_2005, candanedo_model-based_2013}. These systems typically operate on time scales of hours or days and can be modeled as quasi-steady-state or lumped parameter systems, as opposed to vehicle TMS which operate on time scales of seconds or minutes.  

%Control strategies for many forms of transient thermal management systems with integrated thermal storage, or hybrid thermal management systems, have been studied in both experimental and simulation environments.  State estimation is often neglected in simulation-based studies because the controllers are incorporated into the simulation model, and all system states are known.  In \cite{shanks_control_2022}, Shanks et al.\ demonstrate a logic-based controller for state of charge management in a single-phase hybrid TMS for aircraft electronics cooling.  Controller testing is performed in simulation only, and the SOC of the TES device is assumed to be known to the controller.  In \cite{shafiei_model_2015}, Shafiei and Alleyne develop a model predictive controller in simulation for a two-phase hybrid TMS in a transport refrigeration system. However, the SOC is directly calculated from other known states which would need to be estimated or measured in an experimental setting.  Similarly, Pangborn et al.\ develop a model predictive controller for a vehicle TMS with distributed TES \cite{pangborn_hierarchical_2020}.  Each TES device is modeled in the simulation using a single state variable---the state of charge---which is assumed to be known to the controller.  Building cooling systems with ice storage is another form of hybrid TMS which has been thoroughly studied in both simulation and experimental environments \cite{tam_development_2019, henze_experimental_2005, candanedo_model-based_2013}. These systems typically operate on time scales of hours or days and can be modeled as quasi-steady-state or lumped parameter systems, as opposed to vehicle TMS which operate on time scales of seconds or minutes.  Experimental testing of control strategies for fast-timescale hybrid TMS, such as vehicle TMS, is limited; therefore, development and experimental validation of model-based state estimation techniques for TES devices will contribute to filling this gap in the literature.

%Control of transient thermal management systems with integrated thermal storage has been extensively studied in simulation environments.  State estimation is often neglected in these studies because the controllers are incorporated into the simulation model, and all system states are known.  In \cite{shanks_control_2022}, Shanks et al.\ demonstrate a logic-based controller for state of charge management in a single-phase TMS with integrated TES (called a hybrid TMS), but the SOC of the TES device is assumed to be known to the controller.  In \cite{shafiei_model_2015}, Shafiei and Alleyne develop a model predictive controller for a two-phase hybrid TMS in a transport refrigeration system. However, the SOC is directly calculated from other known states which would need to be estimated or measured in an experimental setting.  Similarly, Pangborn et al.\ develop a model predictive controller for a vehicle TMS with distributed TES \cite{pangborn_hierarchical_2020}.  Each TES device is modeled in the simulation using a single state variable---the state of charge---which is assumed to be known to the controller.  Conversely, 

\subsection{Related Work}
In experimental environments, state of charge measurement or estimation methods vary extensively and are often application-specific. Zsembinszki et al.\ present an overview of various measurement methods for determining the phase fraction \cite{zsembinszki_evaluation_2020}.  These include (i) displacement or level sensors for measuring the volume change as the PCM melts and solidifies \cite{henze_experimental_2005}, (ii) pressure sensors for measuring the volume change in enclosed PCM containers \cite{paberit_detecting_2016, steinmaurer_development_2014}, (iii) digital cameras or image sensors for locating the phase change front \cite{charvat_visual_2017}, and (iv) electrical conductivity sensors for determining the phase at selected locations in electrically conductive PCMs \cite{paberit_detecting_2016, ezan_ice_2011}.  Locating the phase change front using distributed temperature sensors is possible, but calculating the stored energy from temperature measurements introduces large uncertainties because of hysteresis, undercooling, and the isothermal nature of phase change \cite{beyne_estimating_2022}.  Beyne et al.\ discuss additional methods for estimating the stored energy; one method involves measuring the heat transfer rate between the working fluid and the PCM with temperature and mass flow rate sensors in the fluid only and then integrating the heat transfer rate over time to calculate the stored energy.  This method requires no temperature sensors embedded in the PCM, but heat transfer between the TES and the surroundings must be negligible for the method to be valid \cite{beyne_estimating_2022}. 

Research on model-based dynamic state estimation for thermal energy storage is limited.  Barz et al.\ design and implement an extended Kalman filter (EKF) for temperature and SOC estimation in a shell-and-tube TES device using temperature measurements.  The authors find that the SOC estimated by the EKF tracks that of a high-fidelity simulation more closely than the SOC calculated directly from the temperature measurements \cite{barz_state_2018}. Similarly, Pernsteiner et al.\ implement an EKF for state and parameter estimation of a reduced-order PCM-based TES simulation model \cite{pernsteiner_state_2021}.  Jaccoud et al.\ investigate a particle filter, a type of Monte Carlo method for state estimation, for estimating the temperature and phase change front location in a 1-dimensional heat transfer problem \cite{jaccoud_state_2018}.  However, a limitation of these works is the absence of guarantees on the convergence of the state estimates. Although Barz et al.\ \cite{barz_state_2018} verify local observability, none of these authors attempt to prove uniform observability or uniform detectability of the nonlinear system model or guarantee convergence of the state estimates. One exception is Morales Sandoval et al.\ who utilize a nonlinear Luenberger observer for temperature estimation in a \emph{sensible} TES device, a hot water thermal storage tank.  The authors check observability of their five-state model but find that the system is only observable when all five states are measured \cite{morales_sandoval_design_2021}.  Hence, the thermal storage device must be designed with observability and state estimation in mind to ensure there are a sufficient number of sensors to fully observe the system. Increasing the number of states in a thermodynamic model would improve model accuracy, but the higher order model could be unobservable, leading to divergence of the state estimate \cite{barz_state_2018, morales_sandoval_design_2021}.  In many \emph{latent} TES architectures, including a sensor for each state in the model is not feasible because of manufacturability or cost constraints, so the estimation model must be constructed such that the system is observable, or at least detectable, given limited measurements.

\subsection{Contribution}
We fill this gap in the literature by designing and experimentally validating a state-dependent Riccati equation (SDRE) filter \cite{mracek_new_1996, jaganath_sdre-based_2005, berman_comparisons_2014} for a PCM-based thermal energy storage device integrated with a single-phase cooling loop.  In contrast to the ubiquitous nonlinear state estimators---the extended Kalman filter \cite{the_analytic_sciences_corporation_applied_1974} and the unscented Kalman filter \cite{julier_new_1997,wan_unscented_2001}---the SDRE filter uses a linear parameter-varying model parameterized by the state of the system with which uniform detectability and boundedness of the estimation error can be guaranteed in advance for many types of dynamic systems \cite{beikzadeh_exponential_2012}.  A key difference between the SDRE filter and the EKF is that the SDRE filter does not require the Jacobian matrix for linearization.  For latent TES, the highly nonlinear nature of phase change makes using estimation methods requiring linearization especially undesirable; derivation of the Jacobian matrix can be difficult, and its calculation for higher order models can be too computationally intensive for real-time estimators.  Additionally, in highly nonlinear systems, linearization can lead to poor performance, instability, and loss of observability. Fortunately, these issues can be avoided by using the SDRE filter \cite{ewing_analysis_2000}.

% Mracek et al.\ discuss the continuous-time formulation of the SDRE filter in which the observer gain is obtained by solving the continuous-time state-dependent Riccati equation \cite{mracek_new_1996}. In \cite{jaganath_sdre-based_2005}, Jaganath et al.\ present the discrete-time SDRE filter in which the discrete-time Riccati equation is solved recursively with prediction and update steps resembling the EKF algorithm.  For systems with continuous dynamics and discrete sampling, Berman et al.\ describe the continuous-discrete SDRE filter in which the state estimates and covariance matrix are propagated in discrete time steps using a state-dependent state transition matrix \cite{berman_comparisons_2014}.

We leverage a finite-volume model of a PCM-based TES and limited temperature measurements to estimate the temperature distribution inside the TES using the continuous-discrete SDRE formulation \cite{berman_comparisons_2014}.  A state of charge metric is defined that is directly calculated from the estimated temperatures. We show that the nonlinear finite-volume model, when parameterized as a linear parameter-varying model for the SDRE filter, is uniformly detectable, thereby guaranteeing boundedness of the state estimation error.  Furthermore, we show that the SDRE approach can be generalized to any phase-change TES architecture by constructing the finite-volume heat transfer model with a strongly connected weighted graph, which will be uniformly detectable given at least one state measurement. \revres{ While there are other nonlinear state estimators (particle and Kalman-type) \cite{simon_optimal_2006} which, like the SDRE, do not require the Jacobian, this graph-structured model for which we can directly show detectability and error boundedness with limited sensing, makes the SDRE a natural choice.}  Through a series of simulated and experimental case studies, we demonstrate the boundedness and accuracy of the SDRE filter applied to the nonlinear estimation problem of a latent TES device. 

This paper is organized as follows.  In Section~\ref{sec:model} we present the TES architecture as well as the system model---both a higher fidelity one used for simulation and a reduced-order model used for estimation. Section~\ref{sec:SDRE} describes the SDRE filter formulation and proves that the estimation model is uniformly detectable. The experimental setup is described in Section~\ref{sec:setup}.  Section~\ref{sec:testing} presents a series of case studies demonstrating the estimator's accuracy and convergence in both simulation and experiments.
