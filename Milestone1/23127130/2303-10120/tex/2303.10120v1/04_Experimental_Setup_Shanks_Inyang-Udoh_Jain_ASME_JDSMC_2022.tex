\section{Experimental Testbed} \label{sec:setup}
%\subsection{Experimental Setup}
The TES is integrated into the experimental single-phase thermal-fluid loop testbed shown in Fig.\ \ref{fig:testbed_photo}.  The testbed has four TES modules arranged in series, each with a capacity of approximately 25.6 kJ.  %Although each module is designed to be identical, the manufacturing tolerances resulted in slight variations in the mass of PCM and other dimensions. Therefore, e
Each module uses a separate state estimator with a model tuned to the specific module's parameters, but in this work, we will show validation of the state estimator on only the first module in the series. 
\begin{figure}[tbp]
    \centering
    \begin{subfigure}{\columnwidth}
        %\centering
        \includegraphics[width=3.1in]{Figures/TMS_front_side.jpg}
        \caption{}
        \label{fig:front_panel}
    \end{subfigure}
    
    \begin{subfigure}{\columnwidth}
        %\centering
        \includegraphics[width=3.1in]{Figures/TES_Modules.jpg}
        \caption{}
        \label{fig:back_panel}
    \end{subfigure}
    \caption{Images of the experimental single-phase thermal-fluid testbed with thermal energy storage.}
    \label{fig:testbed_photo}
\end{figure}

Figure \ref{fig:tms_diag} shows a simplified diagram of the thermal management system; a pump circulates the working fluid (water) though the primary fluid loop with a variable flow rate up to 0.183 kg/s.  Heat is added through a 6 kW 600 VDC electrical resistive heater mounted to an Advanced Thermal Solutions ATS-CP-1002 cold plate. The primary mode of heat rejection is a shell-and-tube heat exchanger cooled by a secondary chilled water loop from a Neslab HX300-DD 10 kW chiller.  A pair of Burkert Type 2875 variable position solenoid valves control the mass flow rate through the TES. The four modules are custom designed and fabricated for the testbed, and together they provide up to 2 kW of heat rejection. 
\begin{figure}[tbp]
    \centering
    \includegraphics[width=3.1in]{Figures/System_Diagram.pdf}
    \caption{Simplified schematic of the experimental thermal management system.}
    \label{fig:tms_diag}
\end{figure}

Temperature measurements of the PCM layer in the TES are provided by five type T thermocouples bonded to the PCM side of the plate, labeled TC2a, TC2b, TC3, TC4a, and TC4b, in the pattern shown in Fig.\ \ref{fig:TES_3D_labeled}. TC2a and TC2b are placed at the same distance along the length of the module, and likewise for TC4a and TC4b.  Type T thermocouples labeled TC0 and TC1 in the fluid channel measure the inlet and outlet fluid temperature, respectively.  Not shown in Fig.\ \ref{fig:TES_3D_labeled} is an Omega FTB-1313 turbine flow meter upstream of the TES that measures the mass flow rate of the working fluid.
\begin{figure}[tbp]
    \centering
    \includegraphics[width=3.1in]{Figures/TES_3D_diagram_labeled.pdf}
    \caption{Section view of a 3D model of the TES module.  Thermocouples TC0 and TC1 measure the fluid inlet and outlet temperatures, respectively, and TC2a, TC2b, TC3, TC4a, and TC4b are embedded in the PCM.}
    \label{fig:TES_3D_labeled}
\end{figure}

The thermocouples are sampled at regular intervals to produce the measurement output vector $y_k$.  Thermocouple TC0 measures $T_{in}$, which is not a state but a parameter in the input matrix $B(x)$. Thermocouple TC1 measures the output from control volume CV1; the fluid outlet temperature is assumed to be equal to the temperature of the fluid in CV1.  The five thermocouples in the PCM layer lie within three control volumes (see Fig.\ \ref{fig:TC_CV_labels}). Thermocouples TC2a and TC2b lie within the same control volume, so measurements from these thermocouples are averaged to provide one output measurement, TC2, for the control volume labeled CV2. Thermocouple TC3 measures the output for control volume CV3. Measurements from thermocouples TC4a and TC4b are also averaged to provide a single output measurement, TC4, for control volume CV4. 
\begin{figure}[tpb]
    \centering
    \includegraphics[width=3.1in]{Figures/TES_CV_Diagram_Labeled.pdf}
    \caption{Locations of thermocouples TC0-4 relative to the control volume grid, denoted by $\times$ marks. Selected control volumes used for validation are also labeled CV1-6.}
    \label{fig:TC_CV_labels}
\end{figure}

During experimental tests, the SDRE filter is deployed on a National Instruments PXIe-8820 embedded controller with an Intel Celeron 1020E 2.20 GHz dual-core processor.  Data aquisition peripherals include a NI PXI-6225 16-bit analog input module for reading the flow meter and a NI PXIe-4353 24-bit thermocouple module.  Additionally, experimental data is recorded so the state estimator can be simulated offline for validation.  A custom LabVIEW VI manages the data acquisition, executes the state estimator algorithm, and translates user input into voltage signals to operate the valves, heaters, and pump.
%; the control logic updates at a rate of 10 Hz.  Unless otherwise stated, the sample rate of the measurements and the update rate of the state estimator is 10 Hz. 

%In Section \ref{sec:sample_rate}, we investigate the effect of increased and reduced sample rates by comparing 80 samples per second (S/s), 10 S/s, 1 S/s, and 0.2 S/s.  To make an unbiased comparison, the prediction equations given in Eqn.\ \eqref{eq:prop} are executed with the time step $\Delta t_k = 0.0125$~s so that the state estimator model is identical in all three cases.  The update step given in Eqns.\ \eqref{eq:kalman_gain} and \eqref{eq:update} is performed whenever measurements are available, once every 1, 8, 80, or 400 time steps depending on whether the sample rate is 80 S/s, 10 S/s, 1 S/s or 0.2 S/s, respectively.  Note that the filters receiving measurements at 10 S/s, 1 S/s, and 0.2 S/s are continuous-discrete filters, but the filter receiving measurements at 80~S/s is a discrete filter since the prediction and update steps are executed at the same rate.

\revres{The thermocouple sensor noise variance is determined by repeatedly sampling a nearly constant temperature, fitting a linear regression model to the sampled data, and calculating the residual variance.}  Each thermocouple is assumed to have independent Gaussian white noise.  \revres{The sample variance of each thermocouple is calculated from a dataset containing 4000 samples, which yields a median value of $\sigma^2=0.0068 K^2$. Given the similarity in noise variance across the thermocouples (which is expected given that each is identical and purchased from the same manufacturer), a single value of $\sigma^2=0.007 K^2$ (to the nearest thousandth) is assumed for each thermocouple.}  Averaging the measurements from laterally adjacent thermocouples reduces the variance of these outputs to $\text{Var}\!\left[\frac{y_{a} + y_{b}}{2}\right]=\frac{\sigma^2}{2}=0.0035\,\textup{K}^2$.  Assuming the output is $y_k = \begin{bmatrix}y_{k,TC1} & y_{k,TC2} & y_{k,TC3} & y_{k,TC4}\end{bmatrix}^\top$, the measurement noise covariance matrix is given in Eqn.\ \eqref{eq:meas_noise}.  The process noise covariance is not as easily characterized, so we assume it is negligible; Eqn.\ \eqref{eq:proc_noise} defines the matrix $W_k$ just large enough to ensure that $P_{k+1|k}$ remains positive definite.  
\begin{subequations}
\begin{gather}
    V_k = \begin{bmatrix}
        0.007 & 0 & 0 & 0\\ 0 & 0.0035 & 0 & 0\\0 & 0 & 0.007 & 0 \\ 0 & 0 & 0 & 0.0035
    \end{bmatrix}\label{eq:meas_noise}\allowdisplaybreaks\\
    W_k = 10^{-7}I_{n}\label{eq:proc_noise}
\end{gather}
\end{subequations}
