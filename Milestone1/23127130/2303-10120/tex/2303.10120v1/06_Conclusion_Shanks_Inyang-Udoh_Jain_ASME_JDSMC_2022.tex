\section{Conclusion} \label{sec:conclusion}

In this paper we designed and experimentally demonstrated a state-dependent Riccati equation (SDRE) filter for nonlinear state and state of charge estimation in a phase-change thermal energy storage device.  Unlike other state estimators for nonlinear systems, the SDRE filter uses a linear parameter-varying system model which offers several advantages.  First, the LPV model can be computed quickly using graph-based methods, and linearization is not required.  Second, the SDRE filter is generalizable to many TES architectures because the finite-volume heat transfer model can be applied to any heat conduction problem, even those with highly nonlinear dynamics and material properties.  Finally, the structure of the LPV model remains the same, so boundedness and detectability of the SDRE filter can be guaranteed in advance.

The SDRE filter is first tested in simulation to verify that the SOC estimate converges to the true SOC.  The SOC estimation error remains bounded within $\pm0.02$, even when the TES experiences rapid temperature changes or the PCM undergoes phase change.  In experimental tests, the SDRE filter is shown to accurately estimate temperatures at two locations in the TES.  The largest sources of error are model inaccuracies such as (i) the assumption that there is no heat transfer with the surroundings and (ii) a simplified phase-change model which does not account for hysteresis or undercooling during solidification.  Additionally, when the mass flow rate through the TES is zero, estimate accuracy is reduced because heat transfer along the length of the TES is typically dominated by mass transfer.  Despite the limitations of the finite volume model, the accuracy of the estimator is not significantly affected by the sample rate of the measurements; sample rates of 80 samples/s and 1 sample/s yield similar root-mean-square errors.  

Experimental validation of model-based state estimation techniques for fast-timescale thermal storage devices advances the continuing research on control strategies for hybrid transient thermal management systems. \revres{In future work, the boundedness and convergence of other Kalman-type and particle filters suitable for SOC estimation may be explored.} \revres{Additionally, investigation of the effects of model uncertainty and the previously discussed model inaccuracies has the potential to decrease estimation error.}    


