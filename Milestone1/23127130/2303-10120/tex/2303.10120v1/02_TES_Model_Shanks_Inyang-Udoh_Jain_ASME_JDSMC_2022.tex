\section{TES Thermodynamic Model} \label{sec:model}

The TES device considered in this work is designed for use in a vehicle thermal management system to provide supplementary heat rejection capability during periods of large transient heat loading. The TES design consists of a paraffin phase change material, hexadecane, with a melting point of 289.5 K \cite{hale_phase_1971} embedded in a rectangular fin heat sink.  The working fluid enters the device and flows along a flat metal separator plate below the PCM layer, as shown in Fig.\ \ref{fig:TES_3D}.  The fins in the PCM layer increase the thermal conductivity of the PCM/fin composite.  The entire TES module is contained in an insulating enclosure to reduce heat transfer with the surroundings. 
\begin{figure}[tpb]
    \centering
    \includegraphics[width=3.1in]{Figures/TES_3D_diagram.pdf}
    \caption{Section view of a 3D model of the TES module.  The fluid outlet is not visible, but the device is symmetric and the outlet geometry is identical to the inlet.}
    \label{fig:TES_3D}
\end{figure}

Experimental testing of the TES design has been used to validate a high-fidelity nonlinear simulation model, which is briefly discussed in Section~\ref{sec:fvm}.  More detail about the simulation model can be found in \cite{gohil_reduced-order_2020} and \cite{shanks_design_2022}. The TES model used for state estimation, discussed in Section~\ref{sec:SE_model}, is based on this validated simulation model.

\subsection{Finite Volume Method} \label{sec:fvm}

To model heat transfer in the TES, we derive a model based on finite-volume discretization as described in \cite{gohil_reduced-order_2020}.  Here we summarize the model for the benefit of the reader. We discretize the cross-section of the device into a grid of $n_x\times n_y$ control volumes, as shown in Fig.\ \ref{fig:TES_diag}, and derive an energy balance equation for each control volume.  The fluid channel and metal plate each comprise one layer of control volumes, and the remaining $n_y-2$ layers contain the PCM and metal fins. The set of all control volumes in the fluid channel is denoted as $\mathcal{F}$, and the fluid control volumes are numbered $j=1$ to $j=n_x$ from the inlet to the outlet\footnote{In Eqn.\ \eqref{eq:advection}, when $j=1$, the term $T_{j-1}=T_{0}$ is defined as the temperature of the fluid entering the TES, or $T_{in}$.}.  The set of metal plate control volumes, comprising $j=n_x+1$ to $j=2n_x$, is denoted as $\mathcal{P}$, and the set of PCM/fin control volumes ($j=2n_x+1$ to $j=n_xn_y$) is denoted as $\mathcal{C}$.

Eqn.\ \eqref{eq:energy_bal} defines an energy balance for control volume $j$.  Heat transfer along the width of the TES ($z$ direction) is assumed to be negligible, and temperatures across the width are assumed to be uniform.  The $\dot{Q}^{adv}_j$ term, defined in Eqn.\ \eqref{eq:advection}, represents heat transfer in fluid control volumes due to mass transfer, or advection; $\dot{m}$ is the mass flow rate, and $c_{p,f}$ is the specific heat of the working fluid.  Conductive heat transfer in the fluid channel is neglected because it is negligible compared to the advective and convective heat transfer rates.  The  $\dot{Q}_{i\rightarrow j}$ term, defined in Eqn.\ \eqref{eq:heat_transfer}, represents the conductive (solid-solid) or convective (fluid-solid) heat transfer rate from $i$ to $j$, where $i\in\mathcal{A}(j)$, and $\mathcal{A}(j)$ is the set of up to four control volumes adjacent to $j$.  Heat transfer between adjacent control volumes is modeled using the thermal resistance, given in Eqn.\ \eqref{eq:resistance}, which is either conductive or convective. In Eqn.\ \eqref{eq:resistance}, $d_{i,j}$ is the distance between the centers of control volumes $i$ and $j$, $\kappa_j$ is the thermal conductivity of $j$, $a_{i,j}$ is the area of the boundary between $i$ and $j$, and $U_j$ is the convective heat transfer coefficient of fluid control volume $j$.  
\begin{figure}[tpb]
    \centering
    \includegraphics[width=3.1in]{Figures/TES_CV_Diagram.pdf}
    \caption{Cross section of the TES module considered in this work (not to scale).  A finite volume heat transfer model is constructed by discretizing the cross section into a grid of rectangular control volumes.  In this figure, the dashed lines demarcate the grid of control volumes for the model used for state estimation.}
    \label{fig:TES_diag}
\end{figure}
\begin{subequations}
\begin{gather}
    m_{j}c_{p,j} \frac{d T_j}{d t}=\dot{Q}^{adv}_j +    \sum_{i \in \mathcal{A}(j)} \hspace{-4pt}\dot{Q}_{i\rightarrow j}\label{eq:energy_bal} \allowdisplaybreaks\\
    \dot{Q}^{adv}_j = \begin{cases}
        \dot{m}c_{p,f}\left(T_{j-1} - T_j\right)&j\in\mathcal{F}\\
        0&j\in\mathcal{P}\cup\mathcal{C}
    \end{cases} \label{eq:advection}\allowdisplaybreaks\\
    \dot{Q}_{i\rightarrow j} = \frac{T_{i}-T_{j}}{R_{j,i} + R_{i,j}} \label{eq:heat_transfer} \allowdisplaybreaks\\
    R_{j, i} = 
    \begin{cases}
        \dfrac{1}{U_j a_{i,j}} & i\in\mathcal{P};\:j \in\mathcal{F} \vspace{3pt}\\
        \dfrac{d_{i, j}}{2 \kappa_j a_{i,j}}& j \in{\mathcal{P}\cup\mathcal{C}}
    \end{cases} \label{eq:resistance}
\end{gather}
\end{subequations} 

The PCM/fin layer is modeled as a composite material composed of metal fins and PCM, which we call a composite PCM or CPCM. In Eqns.\ \eqref{eq:energy_bal} and \eqref{eq:resistance}, the thermal conductivity $\kappa_j$ and specific heat $c_{p,j}$ are temperature-dependent composite properties for control volumes in the CPCM layer.  Thermal properties of the metal plate and fluid layers are assumed constant. Equations for the composite properties can be found in \cite{shamberger_cooling_2018} and \cite{tamraparni_design_2021}.  Validation of the composite assumption for closely-spaced parallel rectangular fins is given in \cite{tamraparni_design_2021}. 

The phase change dynamics are modeled using the effective specific heat function given in Eqn.\ \eqref{eq:cp_eff}, which is validated in \cite{gillis_numerical_2021} and \cite{yangSolvingHeatTransfer2010}.  Near the phase change temperature $T_{pc}$, the effective specific heat of CPCM is greatly increased so that the latent heat is modeled as sensible heat over the temperature range $T_{pc}\pm \frac{\Delta T_{pc}}{2}$, with $\Delta T_{pc}=8$ K, as shown in Fig.\ \ref{fig:sp_heat}.  The width of this temperature range is determined by the parameter $\alpha = \frac{8}{\Delta T_{pc}}$.  Differential scanning calorimetry (DSC) of hexadecane has shown that the phase change occurs over this range of temperatures, although hexadecane also exhibits undercooling during the solidification process and thermal hysteresis between the melting and solidifying temperatures \cite{sgreva_thermo-physical_2022}.  These additional phase-change phenomena are neglected in the simulation model. Outside the latent temperature range, the specific heat approaches the liquid specific heat $c_{p,liq}$ for $T_j > T_{pc}$ or the solid specific heat $c_{p,sol}$ for $T_j < T_{pc}$.  The specific enthalpy of fusion, $h_{fus}$, and the solid and liquid specific heats, $c_{p,sol}$ and $c_{p,liq}$, represent properties of the CPCM. %Letting $\mathcal{C}$ be the set of all CPCM control volumes yields 
\begin{multline}\label{eq:cp_eff}
c_{p,j}(T_j)
= c_{p,sol}+\frac{\left(c_{p,liq}-c_{p,sol}\right)}{1+e^{-\alpha\left(T_j-T_{pc}\right)}} \\
+\frac{h_{fus}  \alpha}{2+e^{-\alpha\left(T_j-T_{pc}\right)}+e^{\alpha\left(T_j-T_{pc}\right)}} \;\forall j\in \mathcal{C}
\end{multline} 
%for all $j\in\mathcal{C}$.

During phase change, a control volume will contain both solid and liquid CPCM. The partially melted CPCM is then treated as an isotropic composite material \cite{yangSolvingHeatTransfer2010}; the volume fraction of liquid CPCM is given in Eqn.\ \eqref{eq:mf} \cite{gillis_numerical_2021}.
\begin{equation}\label{eq:mf}
f_{m,j}=\frac{1}{1+e^{-\alpha\left(T_j-T_{pc}\right)}}\;\forall j\in \mathcal{C}
\end{equation}
\begin{figure}[tpb]
    \centering
    \includegraphics[width=3.1in]{Figures/sp_heat_curve.pdf}
    \caption{Effective specific heat function for simulating phase change in the TES device.  Shading represents the approximate range of temperatures over which the phase change occurs.  The black line is the true melting point, 289.5 K.}
    \label{fig:sp_heat}
\end{figure}

\subsection{Model Used for State Estimation}\label{sec:SE_model}
The number of states in the finite-volume model is a function of the level of discretization chosen by the user.  For a higher fidelity model, the user may choose $n_x$ and $n_y$ to be on the order of tens of control volumes.  However, this would results in a model that is not suitable for state estimation.  Instead, for the purpose of state estimation, we choose a grid of $n_x=3$ by $n_y=7$ control volumes for a total of $n=21$ states. Defining an energy balance for each control volume yields a system of differential equations that can be represented using a linear parameter-varying state-space model.  This model can be quickly derived using graph-based methods by representing the finite-volume model as the thermal resistance network in Fig.\ \ref{fig:res_network}.  

\begin{proposition}\label{prop:graph} The thermal resistance network in Fig.\ \ref{fig:res_network} forms a connected undirected weighted graph. The control volumes constitute the vertices of the graph, and the thermal resistances between control volumes are the graph's edges.\end{proposition}
\begin{proof}
If a path, or a set of contiguous edges, exists between any two vertices, an undirected graph is said to be connected. Proposition \ref{prop:graph} follows from this definition.
\end{proof}
The state-space model in Eqn.\ \eqref{eq:TES_ss} is derived by calculating the graph's Laplacian matrix, $L(x)$, in addition to a diagonal capacitance matrix, $M(x)$, and an input matrix, $B(x)$, where the state vector
$x=\begin{bmatrix}
T_1 & \cdots & T_n
\end{bmatrix}^\top\in\mathbb{R}^n$ contains the temperatures of all $n$ control volumes; this includes the fluid control volumes, metal plate control volumes, and CPCM control volumes. 
% \begin{equation}\label{eq:state_vec}
% \small x = \begin{bmatrix}
%      T^{\mathcal{F}}_1\cdots T^{\mathcal{F}}_{n_x} & T^{\mathcal{P}}_{n_x+1}\cdots T^{\mathcal{P}}_{2n_x} & T^{\mathcal{C}}_{2n_x+1}\cdots T^{\mathcal{C}}_{n_xn_y}
% \end{bmatrix}^\top\normalsize
% \end{equation}
The notation $x_k$ refers to the value of $x$ at time $t_k$, or $x_k = x(t_k)$. The control input $u\in \mathbb{R}^1$ is the mass flow rate, and $y_k\in \mathbb{R}^p$ is the (output) vector of states that are measured at time step $k$.  
\begin{subequations}\label{eq:TES_ss}
\begin{align}
\begin{split}
    \dot{x}&=-M^{-1}(x)L(x) x + B(x)u 
    \\ &= A(x)x + B(x)u 
\end{split}\\
    y_k &= Cx_k \label{eq:output}
\end{align}
\end{subequations}
\begin{figure}[tpb]
    \centering
    \includegraphics[width=3.1in]{Figures/Graph_Network.pdf}
    \caption{The finite-volume model forms a resistance network and can be analyzed as a connected graph. Each edge consists of two series resistances, one for each control volume connected by the edge.  The fluid control volumes are not connected by edges because conduction in the fluid is neglected. }
    \label{fig:res_network}
\end{figure}

The Laplacian matrix $L(x)\in\mathbb{R}^{n\times n}$ for the weighted graph is defined in Eqn.\ \eqref{eq:laplacian}\footnote{The Laplacian of a graph is more formally defined as the difference between the degree matrix and the adjacency matrix; see \cite{inyang-udoh_strongly_2022}}. The matrix $M(x)\in\mathbb{R}^{n\times n}$ is defined in Eqn.\ \eqref{eq:cap_mat} and contains the thermal capacitance $m_jc_{p,j}$ of each control volume; for CPCM control volumes, the specific heat is defined using Eqn.\ \eqref{eq:cp_eff}.  The input matrix $B(x)\in \mathbb{R}^{n\times1}$ is defined in Eqn.\ \eqref{eq:input_mat} and contains the advection terms. Finally, $T_{in}$ is the temperature of the fluid at the inlet, which is not a state. Recall that $T_1$, $T_2$, \dots, $T_{n_x}$ are the fluid control volume temperatures, which are the first $n_x$ states in $x$.  

\begin{subequations}
\begin{gather}
    L_{i,j} = 
    \begin{cases}
    \dfrac{-1}{R_{j,i} + R_{i,j}} &  i\neq j\textup{ and } \exists\textup{ edge } (i,j) \\
    0& i\neq j\textup{ and }\nexists\textup{  edge } (i,j) \\
    \displaystyle\underset{l\in\mathcal{A}(j)}{\sum}\hspace{-4pt}\left(-L_{l,j}\right)&  i=j 
\end{cases}\label{eq:laplacian} \allowdisplaybreaks\\
    M(x) = \begin{bmatrix}
    m_1c_{p,1} & 0 & \cdots  & 0 \\
    0 & m_2c_{p,2}   & \cdots  & 0  \\
    \vdots & \vdots & \ddots & \vdots \\
    0 & 0 & \cdots & m_nc_{p,n}  
\end{bmatrix}\label{eq:cap_mat}\allowdisplaybreaks\\
B (x)= c_{p,f}M^{-1}\begin{bmatrix}
  T_{in} - T_1 \\
  T_1 - T_2\\
   \vdots  \\
  T_{n_x-1} - T_{n_x}\\
  0 \\
   \vdots  \\
  0 
\end{bmatrix}\in\mathbb{R}^{n\times1}\label{eq:input_mat}
\end{gather}
\end{subequations}

Suppose measurements of the temperature of certain control volumes are available at discrete time instances $t_k$, $k=1,2,\dots$; then  $C\in\mathbb{R}^{p\times n}$ is a binary output matrix where $C_{i,j} = 1$ if $y_{k,i} = x_{k,j}$, and $C_{i,j} = 0$ otherwise.  Thus, the output equation (Eqn.\ \ref{eq:output}) is discrete-time although the state dynamics are continuous.

\subsection{State of Charge}
We define the state of charge (SOC) of the TES in Eqn.\ \eqref{eq:soc} as a piecewise linear function of the energy stored in the device, or the enthalpy of the CPCM, $H$ \cite{shanks_control_2022}.  Following the convention described in \cite{zsembinszki_evaluation_2020}, minimum stored energy corresponds to a maximum SOC of 1, and maximum stored energy corresponds to a minimum SOC of 0. Note that this definition of SOC considers both latent and sensible energy storage within the TES.
\begin{equation}
x_{soc}=\begin{cases}
1 & H<H_{min } \\
\dfrac{H_{max}-H}{H_{max }-H_{min }} & H_{min } \leq H \leq H_{max } \\
0 & H>H_{max}
\end{cases}
\label{eq:soc}
\end{equation}

With the finite volume model, the total stored energy, given in Eqn.\ \eqref{eq:total_enth}, is the sum of the enthalpies of each CPCM control volume.  The specific enthalpy of the PCM/fin composite in Eqn.\ \eqref{eq:abs_enth} is found by integrating the effective specific heat function, Eqn.\ \eqref{eq:cp_eff}, with respect to temperature.  Fig.\ \ref{fig:sp_enth} shows the shape of the specific enthalpy curve; the specific enthalpy is defined such that it is zero at the phase change temperature, $T_{pc}$. The maximum enthalpy $H_{max}$ and minimum enthalpy $H_{min}$ in Eqn.\ \eqref{eq:soc} are calculated for the PCM at a uniform temperature $T_{max} = 308$ K or $T_{min} = 278$ K, which are user-defined parameters corresponding to the maximum and minimum expected temperatures, respectively.  Thus, the SOC of the TES can be calculated given the temperature of each CPCM control volume.
\begin{gather}
\begin{multlined}
\label{eq:total_enth}
    H=\sum_{j\in\mathcal{C}}m_{j} h(T_{j})
    \end{multlined}\allowdisplaybreaks\\
\begin{multlined}\label{eq:abs_enth}
    h(T_{j\in\mathcal{C}})= \int_{T_{pc}}^{T_j}c_{p,j}(x)dx \\ 
    = \frac{h_{fus}}{2} \tanh \left[\frac{\alpha}{2}\left(T_j-T_{pc}\right)\right] + 
    (T_j-T_{pc})c_{p,sol} \\
    +\frac{c_{p,liq}-c_{p,sol}}{\alpha}\ln{\left[\frac{1+e^{\alpha\left(T_j-T_{pc}\right)}}{2}\right]} 
\end{multlined}
\end{gather}
\begin{figure}[tpb]
    \centering
    \includegraphics[width=3.1in]{Figures/sp_enth_curve.pdf}
    \caption{Specific enthalpy function for the CPCM derived by integrating the effective specific heat function.  Shading represents the approximate range of temperatures over which the phase change occurs.  The black line is the phase change temperature, 289.5 K.}
    \label{fig:sp_enth}
\end{figure}