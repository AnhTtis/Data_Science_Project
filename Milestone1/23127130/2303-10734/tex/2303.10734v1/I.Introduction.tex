\section{Introduction}

It is currently understood that, at an elementary level, all matter is
made up of leptons and quarks. Quarks are subject to all four fundamental forces, as they have mass and electrical charge, which subject them to gravitational and electromagnetic force; they have a color charge, which subject them to the strong force, and can change flavour in decays, which is due to the weak force. Quantum chromodynamics (QCD) is the theory that explains the interaction between quarks through the strong force. 

The ordinary matter, however, is formed by hadrons. That is, under normal conditions, quarks are always confined inside hadrons. The reason behind confinement remains unknown and a one million dollar prize will be awarded for the correct explanation \cite{premio_confinamento}.
QCD has also a property called asymptotic freedom, %discovered and described 
proposed in 1973 by Frank Wilczek and David Gross \cite{gross1973ultraviolet}, and independently by David Politzer in the same year \cite{politzer1973reliable}, which says that the binding to which the quarks are subject decreases in intensity at small distances or high energies. This means that in extreme situations of high energy, such as the ones found in particle colliders or in the early universe, or at very high densities, such as the ones found inside compact stars, quarks can be deconfined. In fact, since the discovery of asymptotic freedom, the idea of the existence of a quark-gluon plasma (QGP) has been increasingly reinforced. The first experimental corroboration of the existence of the QGP occurred at RHIC in 2005 \cite{adams2005experimental}.

From the theoretical point of view, there are still many limitations for the QCD to be solved, which led physicists to attack the problem from two different perspectives: through QCD calculations on the lattice, the so called lattice QCD (LQCD), and through effective models. Due to computational constraints and numerical difficulties, such as the sign problem, the LQCD method only achieves results for low chemical potentials \cite{boricci2004reweighting,alexandru2005lattice,mendes2007lattice,bhattacharya2014qcd,goy2017sign}. This makes effective models currently the only source to generate results that cover the QCD phase diagram at higher chemical potentials.

The LQCD predicts the existence of a smooth crossover around a temperature of $160-170$~MeV at low chemical potentials, while the effective models predict a first order phase transition at higher densities \cite{aoki2006order,bellwied2015qcd,luostudy}. Moreover, the first order phase transition must end at an unique point where a second order phase transition takes place, the critical end point (CEP), even though its existence and exact location are not yet well established \cite{bazavov2017skewness,bazavov2017qcd}.

In a previous work \cite{lopes2021modified_pt2}, we obtained an estimation of the QCD phase diagram based only on the extensions of the MIT bag model. Towards a more realistic description, in this work we use two relativistic effective models to describe the hadronic and quark phases of matter. 
For the quark phase, we use again an extended version of the MIT bag model, as introduced in ref.~\cite{lopes2021modified}, while for the hadronic phase, the quantum hadrodynamics (QHD)  model with non-linear terms and $\omega-\rho$ interaction is used, since it reconciles the theory with experimental results \cite{boguta/bodmer,PhysRevC.82.055803}. In both models the relative strength of the vector channel is fixed by symmetry group arguments, which allows us to build an unified scheme for the strong interaction. The Gibbs' conditions are used to establish the crossing points of the pressure as a function of the chemical potentials obtained in both phases \cite{landau2013course,glendenning2012compact}.  

To reproduce reliable results, some restrictions must be imposed when choosing the models mentioned above. The MIT based models are used only with constant values that don't allow stable u-d matter, and  QHD based models, in turn, are restricted to parameterizations that satisfy well known nuclear and astrophysical properties \cite{PhysRevC.90.055203,PhysRevC.93.025806}.

We first obtain phase diagrams for two-flavour symmetric matter, i.e., we take the hadronic matter to be constituted only by nucleons with equal chemical potentials $\mu_p=\mu_n$ and the quark matter constituted only by quarks $u$ and $d$ with $\mu_u=\mu_d$. We then confront these results with the chemical freeze-out line obtained by Cleymans \cite{cleymans2006comparison}, which is known to be a pure hadronic process, and, therefore, the freeze-out line needs to be in the hadronic phase. The liquid-gas phase transition \cite{finn1982nuclear}, which is also a pure hadronic process that takes place at low temperatures, must be confined to the hadronic part of the QCD phase diagram.

At $T=0$ we also expect the phase transition to occur at chemical potential values at least higher than $\mu=1050$~MeV, as shown in \cite{fukushima2010phase} using Polyakov loop formalism. In \cite{buballa2005njl,ruester2005phase}, using NJL-based models, it is shown that the chemical potential of the phase transition at $T=0$ occurs in the range $\mu=1080 - 1100$~MeV. In \cite{klahn2017simultaneous}, using a chiral bag model, the obtained value is $\mu=1250$~MeV. Although there is no experimental evidence of the maximum chemical potential which preserves the hadron phase, in \cite{annala2020evidence} it is pointed out that quark matter inside neutron stars is not only possible, but probable. For $\beta$-stable matter, using a relativistic density functional approach, \cite{ayriyan2018robustness} indicates that the transitions should occur around $\mu=1200$~MeV. As done in our previous work \cite{lopes2021modified_pt2}, for two-flavour symmetric matter we assume a maximum value of $\mu=1400$~MeV as a more conservative estimate.

Next we obtain the phase diagram for stellar matter, i.e. $\beta$-stable and charge neutral matter, and compare two different prescriptions that affect the quark phase, one where we impose flavour conservation during the phase transition, so that the quark matter is totally determined from the hadronic matter \cite{bombaci2017quark}
and the other where the quark matter is also $\beta$-stable and charge neutral, the so called Maxwell prescription, usually taken to build hybrid stars \cite{voskresensky2002charge,paoli2010importance}.

%Here we will sometimes refer to this maximum temperature as critical temperature, but one has to keep in mind that we don’t mean that this is the critical temperature associated with the CEP. By critical temperature here we mean merely the highest temperature where we were able to satisfy the phase transition conditions we impose here. In the same way, when we refer here to critical chemical potential we mean the chemical potential where there occurs a phase transition at $T=0$, which corresponds to the maximum chemical potential of the phase diagram. %talvez não precise desse parágrafo

It is important to stress that the prescriptions we have just mentioned can only provide first order phase transitions since they depend on relativistic models within mean field approximations (MFA) and quantum fluctuations are completely disregarded. 
Another aspect is that one should be aware of the limitations imposed by extrapolations of stellar matter conditions to high temperatures. For completeness we analyse the dependence of the results on the models used at both ends of the curve, i.e, high temperatures and low chemical potentials and low temperatures and high chemical potentials, but our results at low chemical potentials and high temperatures serve as a guide only and should be regarded with a grain of salt. 

Finally, we calculate the latent heat based on two different simple expressions to try to estimate the position (temperature and corresponding chemical potential) of the critical end point. 
