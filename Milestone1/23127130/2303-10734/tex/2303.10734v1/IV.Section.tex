
\section{Different matter hypotheses}\label{sec:Conditions}
Here we discuss the conditions that we impose to the hadronic matter before it goes through a first order phase transition and also the sort of conservation that we impose when the transition occurs.

\subsection{Based on symmetric matter}\label{subsec:Symmetric_matter}

The first situation considered in the present work is a two-flavour symmetric matter where the nucleons have equal chemical potentials ($\mu_n=\mu_p$) as well as the quarks ($\mu_u=\mu_d$). 

The main reason to consider this scenario is to adjust our parametrizations in order to satisfy the constraints imposed by the chemical freeze-out line \cite{cleymans2006comparison}, which, as being a pure hadronic process, needs to be in the hadronic phase.


\subsection{Based on stellar matter}\label{subsec:Stellar_matter}

Stable neutron and quark stars are normally described by zero temperature matter, but temperatures up to $T=50$ MeV can be important in the early moments of the star as the cooling of a newborn neutron star by neutrino diffusion takes a few seconds \cite{menezes2004warm}. For higher temperatures we do not expect $\beta$-stable matter \cite{gupta2003study,cavagnoli2008warm}, nevertheless here we extend our study beyond the temperature of $T=50$ MeV for educational reasons.


For a description of neutral and stable stellar matter we include a non-interacting lepton gas to both the hadronic matter and the quark matter. For that, we add the following Lagrangian density to Eqs.~(\ref{NLWM}) and (\ref{mit}):

\begin{equation}\label{leptons}
    \mathcal{L}_l=\sum_l \overline{\psi}_l (i \gamma^\mu \partial_\mu - m_l)\psi_l,
\end{equation}
where the sum extends over the leptons $e^-$ and $\mu^-$ with mass $m_{e^-}=0.511$ MeV and $m_{\mu^-}=105.66$ MeV, respectively \cite{tanabashi2018review}.

Furthermore, as neutron stars present internal densities that can be up to 10 times higher that the nuclear saturation density, the onset of hyperons is expected because their appearance is energetically favorable as compared with the inclusion of more nucleons in the system \cite{menezes2021neutron}, and so, here we add the six lightest hyperons to hadron matter so that the sum in Eq.~(\ref{NLWM}) extends over the baryon octet. Nevertheless, at the end of this work we also obtain diagrams for stellar matter without strangeness so that, in this case, the sum in Eq.~(\ref{NLWM}) extends only over the nucleons.

We also have to impose $\beta$-equilibrium and electric charge neutrality to the hadronic matter:

\begin{equation}
    \mu_B=\mu_n-q_B ~\mu_e \quad \textrm{and} \quad \mu_{e^-}=\mu_{\mu^-},
\end{equation}

\begin{equation}
    n_p + n_{\Sigma^+} = n_{e^-} + n_{\mu^-} + n_{\Sigma^-} + n_{\Xi^-}.
\end{equation}

We next consider two scenarios. In the first one we impose flavour conservation at the point of the phase transition, so that the quark phase is completely determined from the initial hadronic matter through the bond:

\begin{equation}\label{eq:charge_conser}
    y_q=\frac{1}{3}\sum_i n_{qi} y_i,
\end{equation}
where $i=n,p,\Lambda,\Sigma^-,\Sigma^0,\Sigma^+,\Xi^-,\Xi^0$, $y_i=n_i/n_B$, being $n_i$ the baryon density of baryon $i$ and $n_B$ the total baryon density, and $n_{qi}$ the number of quarks with flavour $q$ that constitute baryon $i$ \cite{olesen1994nucleation}. %So, the phase transition preserves the total baryonic mass and lepton number which consequently preserves the charge neutrality. 
Thus, it is generally assumed that under certain circumstances, the electrically neutral and in chemical equilibrium hadronic matter is metastable and can be converted into an energetically favored deconfined quark phase. Due to the imposition shown in Eq.~\ref{eq:charge_conser}, this matter will not be in $\beta$-equilibrium \cite{JCAP12(2017)028_Kauan,bombaci2017quark}.

Within this scenario we also briefly analyze two sub-scenarios. One where the lepton fraction, defined as the fraction between lepton density and the baryonic density of each phase is preserved at the point of the phase transition and the other where we have the same lepton density, and consequently same energy density and pressure, for both phases at the point of the hadron-quark phase transition.
%{\bf Tem necessidade? naum esta claro qual deve ser o melhor caminho? aquele que mantem a neutralidade...} {\color{red} Ninguem nunca mostrou...vamos fazer isso.}


%One where the lepton fraction, defined as $y_l=n_l^H/n_B^H$ in the hadron phase and $y_l=n_l^q/n_B^q$ in the quark phase, where $n_l^H$ and $n_l^q$ are the lepton density, $n_B^H$ and $n_B^q$ are the baryon densities of the hadron phase and the quarks phase, respectively, are also conserved at the point of the phase transition. And the other where ...

We also consider a second scenario, where we do not impose flavour conservation, but charge neutrality and chemical equilibrium to both phases, so that we have, for the quark matter:

\begin{equation}
    \mu_s=\mu_d=\mu_u+\mu_e \quad \textrm{and} \quad \mu_{e^-}=\mu_{\mu^-},
\end{equation}

\begin{equation}
    n_e+n_\mu=\frac{1}{3}(2n_u-n_d-n_s).
\end{equation}

This second scenario is the so called Maxwell construction or prescription, most commonly used to construct the EoSs to describe hybrid stars, as done, for instance, in~\cite{lopes2022hypermassive}. 

%{\color{red} Acho essa frase a seguir estranha. A ideia seria comparar construcoes de Maxwell e Gibbs? Melhor retirar, nao?} Both constructions result in very similar mass radius curves, being that the differences in the hadronic EoSs dominate over the differences in the quark EoSs so that the maximum mass is mostly determined by the hadronic part. More details about this discussion can be found in~\cite{menezes2021neutron} and the references therein. 









