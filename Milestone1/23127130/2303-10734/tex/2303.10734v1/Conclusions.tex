\section{Conclusions}

In this work, we sought to obtain the QCD phase diagram using two effective models. For the description of quark matter, a simple relativistic model, the MIT bag model and a modification thereof, were used. And for the description of hadronic matter, a model of quantum hadrodynamics, the Walecka Model with non-linear terms.

The Gibbs' conditions were used to establish the crossing points of the pressure as a function of the chemical potentials obtained in both phases.

Some restrictions were imposed when choosing the models mentioned above. We used temperature-dependent Bag B(T) values with the same parametrizations already used in our previous work \cite{lopes2021modified_pt2}, with one set of constants within the limits the stability window, i.e., $B_0^{1/4}=148$~MeV and another set outside this window , i.e., $B_0^{1/4}=165$~MeV, both sets with $G_V$ varying from zero to $G_V=0.3$~fm$^2$. As for the QHD based model, we used the NL3$^*\omega \rho$ and eL3$\omega \rho$ parametrizations, which satisfy several nuclear and astrophysical properties.

We first obtained the phase diagrams for two-flavour symmetric matter. As the parametrizations for the B(T) were adjusted so that the $T_{max}$ would satisfy the restrain imposed by the LQCD and freeze-out results, the $T_{max}$s are all around 168 MeV, but only the results for $B_0^{1/4}=165$~MeV fit the Cleymans line entirely inside the confined (hadron) phase. The results for low temperatures depend on all parametrizations, as already concluded in \cite{biesdorf2022qcd}. The influences of the different MIT based models follow the conclusions of \cite{lopes2021modified_pt2}, but here the results are more sensible to the different values of $G_V$s. Only a few combination of parametrizations resulted in maximum chemical potentials $\mu_{max}$ within the range $1050 \leq \mu \leq 1400$~MeV. Ultimately only NL3$^*\omega \rho$ + $B_0^{1/4}=165$~MeV and NL3$^*\omega \rho$ + $B_0^{1/4}=165$~MeV + $G_V=0.05$~fm$^2$ satisfy all restrains we imposed.

We also calculated the phase diagrams for stellar matter using two different prescriptions, one where we impose flavour conservation at the point of the phase transition and the other where both phases are $\beta$-stable and charge neutral, being this last one the Maxwell prescription. Between the two prescriptions, the of flavour conservation results in higher chemical potentials at low temperatures, but for both prescriptions the $T_{max}$s are the same and depend only on the value of $B_0$. Comparing this results to the symmetric matter, we conclude that the stellar matter favors the quark phase, being that the $T_{max}$s and $\mu_{max}$s are higher for symmetric matter. 

In all cases a higher Bag pressure and $G_V$ value increases the value of the chemical potentials at low temperatures, but this increase is lower for the stellar matter in the flavour conservation prescription and even lower in the Maxwell prescription than in the two-flavour symmetric matter. As for the QHD model parametrization, the eL3$\omega \rho$ always favors the hadron phase when compared to the NL3$^*\omega \rho$ one.

We also discussed the results for stellar matter without strange matter end concluded that those results are very similar to the ones for two-flavour symmetric matter.

Finally, at the last section we discussed briefly the results for latent heat and showed that for stellar matter with strange matter it always grows with the temperature. For two-flavour symmetric matter and stellar matter without strange matter, though, the behavior is more of a sinusoid. 