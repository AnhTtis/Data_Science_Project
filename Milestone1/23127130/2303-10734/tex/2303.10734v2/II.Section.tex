
\section{Effective models}\label{sec:the_effective_models}


\subsection{Hadronic matter - Relativistic Mean Field QHD models}

To describe hadronic matter we use the Walecka Model \cite{NLWM} with non-linear terms \cite{boguta/bodmer}. 
In this model four types of mesons are included to describe the interactions between baryons; so we have the scalar $\sigma$, isoscalar-vector $\omega^\mu$, isovector-vector $\vec{\rho}^\mu$  and strange isoscalar-vector $\phi^\mu$ meson fields. As in \cite{PhysRevC.84.065810,PhysRevLett.95.122501,PhysRevC.82.025805,PhysRevC.82.055803} we also consider the $\omega$-$\rho$ meson coupling terms as this term influences the symmetry energy and its slope, resulting in EoSs that can satisfy all important nuclear matter saturation properties and observational constraints. The inclusion of the $\phi$ meson does not affect the properties of the nuclear matter as it does not couple to the nucleons, but it stiffens the EoSs that include hyperons.  The Lagrangian density is as follows:

\begin{widetext}
\begin{center}

\begin{align}\label{NLWM}
    \mathcal{L}_{NLWM}& = \sum_{B}\overline{\psi}_B [\gamma_\mu(i\partial^\mu - g_{B \omega}\omega^\mu - g_{B \rho} \frac{\vec{\tau}_B}{2} \vec{\rho}^\mu - g_{B \phi} \phi^\mu)%\nonumber \\
    -m^*_B]\psi_B%\nonumber \\
    +\frac{1}{2}\partial_\mu \sigma \partial^\mu \sigma - \frac{1}{2} m_\sigma^2 \sigma^2 - \frac{1}{3!} \kappa \sigma^3 - \frac{1}{4!} \lambda \sigma^4 + \nonumber \\
    &+\frac{1}{2} m_\phi^2 \phi_\mu \phi^\mu -\frac{1}{4} \Omega^{\mu\nu}\Omega_{\mu\nu} + \frac{1}{2}m_\omega^2 \omega_\mu \omega^\mu %+ \frac{1}{4!}\xi g_{N \omega}^4 (\omega_\mu \omega^\mu)^2 
    - \frac{1}{4}\Phi^{\mu \nu} \Phi_{\mu \nu} %\nonumber \\
     - \frac{1}{4} \vec{R}_{\mu \nu} \vec{R}^{\mu \nu} + \frac{1}{2}m_\rho^2 \vec{\rho}_\mu \vec{\rho}^\mu + \Lambda_v g_{N \omega}^2 g_{N \rho}^2\omega_\mu \omega^\mu \vec{\rho}_\mu \vec{\rho}^\mu, 
\end{align}
\end{center}
\end{widetext}
where the Dirac spinor $\psi_B$ represents the baryons with the effective mass $m_B^*=m_B-g_{B \sigma} \sigma$, $\vec{\tau}_B$ are the corresponding Pauli matrices, $g_{Bi}$ are the coupling constants of the mesons $i=\sigma, \omega, \rho, \phi$ with the baryon $B$, $m_i$ is the mass of the meson $i$, $\Omega_{\mu \nu}=\partial_\mu \omega_\nu - \partial_\nu \omega_\mu$, $\vec{R}_{\mu \nu}=\partial_\mu\vec{\rho}_\nu - \partial_\nu\vec{\rho}_\mu - g_\rho(\vec{\rho}_\mu \times \vec{\rho}_\nu)$ and $\Phi_{\mu \nu}=\partial_\mu \phi_\nu - \partial_\nu \phi_\mu$. $\kappa$ and $\lambda$ are scalar self-interaction constants introduced by \cite{boguta/bodmer}%, $\xi$ {\color{red} Nao usamos isso. Que tal retirar? Vamos economizar uma coluna na tabela e ela vai caber melhor no texto.}is the coupling of the quartic isoscalar-vector self interaction 
and $\Lambda_v$ is the coupling constant of the mixed quartic isovector-vector interaction. The $B$ sum extends over the octet of the lightest baryons $\{n, p, \Lambda, \Sigma^-, \Sigma^0, \Sigma^+, \Xi^-, \Xi^0\}$ or only over the nucleons, depending on the scenario considered (see section \ref{sec:Conditions}).

%Omega-meson self-interactions, as described by the parameter ζ, soften the equation of state at high density and can be tuned to reproduce the maximum mass of a neutron star. 

After applying the mean field approximation (MFA), the EoS can be easily obtained from Eq.~(\ref{NLWM}) and detailed calculations can be seen in \cite{PhysRevC.90.055203}, for instance. 

As it is well known, it is easy to find in the literature several parametrizations 
for this model, but not all of them satisfy the constraints imposed by experimental results for nuclear matter, as shown in \cite{PhysRevC.90.055203}. Moreover, it is desirable that the EoSs reproduce neutron star masses that are heavier than $2$~M$_\odot$, as imposed by the NICER results for PSR J$0740 + 6620$ ($2.08 \pm 0.07$M$_\odot$ M$_\odot$) \cite{miller2021radius} and the recently detected PSR J0952-0607, with a mass of $2.35 \pm 0.17$M$_\odot$\cite{romani2022psr}, yet to be confirmed.

In the present work we choose two parametrizations: eL3$\omega \rho$ and NL3$^*\omega \rho$. The eL3$\omega \rho$ is the parametrization proposed in \cite{lopes2021hyperonic} with a slightly modification to adjust the symmetry energy and its slope accordingly to \cite{essick2021astrophysical} and the NL3$^*\omega \rho$ is the NL3$^*$ parametrization proposed in \cite{lalazissis2009effective} with the addition of the $\omega \rho$-channel as done in \cite{lopes2022nature}. All of them satisfy the symmetric nuclear matter properties at the saturation density and also reproduce maximum star masses above $2$~M$_\odot$ even when hyperons are included. The main parameters of these parametrizations are presented in Table~\ref{tab:parametros} and the main nuclear properties are presented in Table~\ref{tab:propriedades}.


\begin{widetext}
\begin{center}
\begin{table}[]
\caption{
Parameter sets for the two models discussed in the text. The meson masses $m_\sigma$, $m_\omega$, and $m\rho$ are all given in MeV. The nucleon and the $\phi$ meson masses were fixed at $M=939$ MeV and $m_\phi=1020$~MeV, respectively, in both models.}
%\begin{tabular}{@{}ccccccccccc@{}}
\begin{tabular}{p{1.5cm}p{1.2cm}p{1.2cm}p{1.2cm}p{1.5cm}p{1.5cm}p{1.5cm}p{1.9cm}p{2.2cm}p{1.2cm}}
\toprule[0.6pt]
 Model  & $m_\sigma$  & $m_\omega$  & $m_\rho$  & $g_{N \sigma}$  & $g_{N \omega}$  & $g_{N \rho}$  & $\kappa$  & $\lambda$  & $\Lambda_v$  \\ \midrule[1.5pt]
 eL3$\omega \rho$ & $512.000$  & $783.000$  & $770.000$  & $9.0286$  & $10.5970$  & $9.4381$  & $0.008280 \cdot g_\sigma^3$  & $-0.023400 \cdot g_\sigma^4$  & $0.0283$  \\
 NL3$^*\omega \rho$ & $502.574$  & $782.600$  & $763.000$  & $10.0944$  & $12.8065$  & $14.4410$  & $0.004417\cdot g_\sigma^3$  & $-0.017422\cdot g_\sigma^4$  & $0.045$ \\ \bottomrule[0.6pt]
\end{tabular}\label{tab:parametros}
\end{table}
\end{center}
\end{widetext}



\begin{table}[]
\caption{Properties at saturation of the models eL3$\omega \rho$ and NL3$^*\omega \rho$. We present the saturation density (n$_0$), energy per particle (E/A), compressibility (K), and effective nucleon mass (M$^*$/M) in symmetric nuclear matter, and also the symmetry energy (E$_{sym}$) and slope of the symmetry energy (L) at n$_0$.}
%\begin{tabular}{@{}ccccccccccc@{}}
\begin{tabular}{p{2.0cm}p{1.4cm}p{1.4cm}}
\toprule[0.6pt]
 Model  & eL3$\omega \rho$  & NL3$^*\omega \rho$    \\ \midrule[1.5pt]
 n$_0$~(fm$^{-3}$)  & 0.156  & 0.150   \\ 
 E/A (MeV)    & 16.2   &    16.3       \\
 K (MeV)  & 256 & 258     \\
 M$^*$/M & 0.69  & 0.59    \\
 E$_{sym}$~(MeV) & 32.1  & 30.7   \\ 
 L (MeV)    &   66   &  42     \\\bottomrule[0.6pt]
\end{tabular}\label{tab:propriedades}
\end{table}


We consider the hyperon masses to be $m_\Lambda=1116$~MeV, $m_{\Sigma}=1193$~MeV and $m_{\Xi}=1318$~MeV. The couplings of the hyperons to the vector mesons are related to the nucleon couplings, $g_{N \omega}$ and $g_{N \rho}$, by assuming SU(6)-flavour symmetry, according to the ratios \cite{dover1984hyperon,schaffner1996hyperon,banik2014new,tolos2017equation,lopes2014hypernuclear}:

\begin{align}
    &g_{\Lambda \omega} : g_{\Sigma \omega} : g_{\Xi \omega} : g_{N \omega} = \frac{2}{3} : \frac{2}{3} : \frac{1}{3} : 1, \nonumber \\
    &g_{\Lambda \rho} : g_{\Sigma \rho} : g_{\Xi \rho} : g_{N \rho} = 0:2:1:1,\\
    &g_{\Lambda \phi} : g_{\Sigma \phi} : g_{\Xi \phi} : g_{N \omega} = -\frac{\sqrt{2}}{3} : -\frac{\sqrt{2}}{3} : -\frac{2\sqrt{2}}{3} : 1 \nonumber,
\end{align}
noting that $g_{N \phi}=0$. The coupling of each hyperon to the $\sigma$ field is adjusted to reproduce the hyperon potential in strange nuclear matter (SNM) derived from hypernuclear observables. We fix this potentials as $U_\Lambda(n_0)=-28$~MeV, $U_\Sigma(n_0)=+30$~MeV and $U_\Xi(n_0)=-4$~MeV and obtain the coupling constants presented in Table~\ref{tab:acoplamentos}.


\begin{table}[]
\caption{Hyperon-$\sigma$ coupling constants adjusted to reproduce the hyperon potential in SNM derived from hypernuclear observables.}
%\begin{tabular}{@{}ccccccccccc@{}}
\begin{tabular}{p{1.5cm}p{1.4cm}p{1.4cm}p{1.4cm}}
\toprule[0.6pt]
 Model  & $g_{\Lambda \sigma}/g_{N \sigma}$  & $g_{\Sigma \sigma}/g_{N \sigma}$  & $g_{\Xi \sigma}/g_{N \sigma}$  \\ \midrule[1.5pt]
 eL3$\omega \rho$  & $0.610$  & $0.406$  & $0.269$ \\ 
 NL3$^*\omega \rho$ & $0.613$  & $0.461$  & $0.279$  \\ \bottomrule[0.6pt]
\end{tabular}\label{tab:acoplamentos}
\end{table}


%%%%%%%%%%%%%%%%%%%%%%%%%%%%%%%%%%%%%%%%%%%%%%%%%%%%%%%%%%%%%%%%%%%%%%%%%%%%%

\subsection{Quark matter - MIT bag based models}

To describe the quark matter we choose the simple MIT bag model \cite{PhysRevD.9.3471_mit_original} and a modification of that model that includes a vector field as presented in \cite{lopes2021modified}. In the original MIT bag model the quarks are free inside the bag and confined inside it. All the information about the strong force relies on the bag constant $B$, which mimics the vacuum pressure. With the inclusion of a vector field $V_\mu$, the quark interaction inside the bag is mediated by the $\omega$ meson, in a similar way as in the QHD model with the baryons. In \cite{lopes2021modified} a self-interacting vector field which allows more malleability on the stiffness of the EoSs has also been added. However, in the present study, we opted not to include this term as its influence is barely noticeable for relatively low densities. 
%However this term has a crucial role at very high densities. 
The Lagrangian density of the model follows:

\begin{align}\label{mit}
    \mathcal{L}_{MIT}&=\sum_q \Big\{ \overline{\psi}_q \Big[\gamma^\mu(i \partial_\mu -g_{qqV} V_\mu) - m_q\Big]\psi_q   \nonumber \\
    &+ \frac{1}{2}m_V^2 V_\mu V^\mu - B\Big\}\Theta(\overline{\psi}_q \psi_q) - \frac{1}{2}\overline{\psi}_q \psi_q \delta_S,
\end{align}
where the Dirac spinor $\psi_q$ represents the quark with mass $m_q$, $g_{qqV}$ the coupling constant, and $m_V$ the mass of the meson. $\Theta(\overline{\psi}_q \psi_q)$ is a Heaviside function that ensures that the quarks are confined inside the bag and $\delta_S$ is a Dirac function that guarantees continuity of the fields of the quarks on the surface of the bag. If we take $g_{qqV}=m_V=0$ we obtain the original MIT bag model. Using mean field approximation we easily obtain the EoS. For more details, the interested reader can see reference \cite{lopes2021modified}.

In all cases we choose the quark masses as being $m_u=m_d=4$~MeV and $m_s=95$~MeV \cite{tanabashi2018review}. The meson mass $m_V$ %considered 
is the same as the $\omega$ meson mass $m_\omega$ considered in the QHD models.

As we have done in the hadronic phase, for the relation between the coupling constants we opt to use the ones obtained via symmetry group, which allows us to build an unified scheme for the strong interaction. In such approach we have 
$$g_{ssV}=\frac{2}{5} g_{uuV}=\frac{2}{5} g_{ddV},$$
as done in \cite{lopes2021modified}. We also define $G_V=(g_{uuV}/m_V)^2$ and choose here $G_V=0.05$~fm$^2$, $G_V=0.1$~fm$^2$ and $G_V=0.3$~fm$^2$.

As for the bag pressure value, we choose a temperature-dependent bag model in order to be able to obtain higher transition temperatures at low chemical potentials and at the same time maintain the transition chemical potentials at low temperatures within a certain range. More details on this discussion can be seen in \cite{lopes2021modified_pt2} and \cite{biesdorf2022qcd}. So, the $B$ in Eq. \ref{mit} is substituted by:

\begin{equation}\label{eq:B(T)}
    B(T)=B_0 \Bigg[1+\left(\frac{T}{T_0} \right)^4 \Bigg],
\end{equation}
where $T_0$ is adjusted to reproduce the LQCD and freeze-out (pseudo) critical temperature at zero chemical potential. Thus we use $T_0=131$~MeV for $B_0^{1/4}=148$~MeV and $T_0=155$~MeV for $B_0^{1/4}=165$~MeV. As for the values of $B_0$, we choose $B_0^{1/4}=148$~MeV, which is the lowest value within the stability window \cite{torres2013quark,lopes2021modified} that satisfies the Bodmer-Witten conjecture \cite{bodmer1971collapsed,witten1984cosmic}, and $B_0^{1/4}=165$~MeV, which is outside the stability window, but allows us to satisfy another constraint that will become clear latter on.




