\section{Latent Heat and Latent Energy}\label{sec:latent_heat}

The latent heat is an important quantity in the study of phase transitions and we investigate two different expressions found in the literature.

We  first take a look at the relativistic latent heat $L|_\epsilon$, as given 
%byEq.~\ref{eq:energia_latente}
in ~\cite{agasian2008quark,lope2022maximum} 
\begin{equation}\label{eq:energia_latente}
    L|_\epsilon=P^H \frac{\epsilon^Q-\epsilon^H}{\epsilon^Q \epsilon^H},
\end{equation}
where $\epsilon^Q$ and $\epsilon^H$ are the energy densities at the point of the phase transition for the quark and hadronic matter, respectively. We call it latent energy next, since it yields results that are not zero at zero temperature. Notice that this quantity has no dimension. According to \cite{agasian2008quark}, when the latent energy becomes zero, the critical end point, where the first order phase transition turns into a crossover, is reached.

\begin{figure}[ht] 
\begin{centering}
\includegraphics[angle=0,width=0.5\textwidth]{figuras/energia_latente-1x1.pdf}
\caption{Results for latent energy for different matter hypothesis and prescriptions for phase transition used in this work considering the NL3$^* \omega \rho$ parametrization for the hadronic phase and $B_0^{1/4}=165$~MeV + $G_V=0.1$~fm$^2$ for the quark phase.}\label{fig:energia_latente}
\end{centering}
\end{figure}


We also analyse the usual expression found in textbooks and investigated in \cite{roark2019hyperons} as well:
\begin{equation}\label{eq:calor_latente}
    L|_S= \left( S^Q - S^H \right) T,
\end{equation}
where $S^Q$ and $S^H$ are the entropy densities,  defined in Eq.~\ref{eq:entropia}, at the point of the phase transition for the quark and hadronic matter, respectively.

\begin{align}\label{eq:entropia}
    S&=- \gamma \sum_i \int \frac{d^3k}{(2 \pi)^3} \Bigg[f_{i+} \ln \left(\frac{f_{i+}}{1-f_{i+}} \right) + \\ \nonumber 
    & + \ln (1-f_{i+}) + f_{i-} \ln \left(\frac{f_{i-}}{1-f_{i-}} \right) + \ln (1-f_{i-}) \Bigg],
\end{align}
where $\gamma$ is degeneracy of spin (and color in the case of the quarks) and $f_{i \pm }$ are the Fermi-Dirac distribution functions of the particles and anti-particles, respectively.

\begin{figure}[ht] 
\begin{centering}
\includegraphics[angle=0,width=0.5\textwidth]{figuras/calor_latente-1x1.pdf}
\caption{Results for latent heat for different matter hypothesis and prescriptions for phase transition used in this work considering the NL3$^* \omega \rho$ parametrization for the hadronic phase and $B_0^{1/4}=165$~MeV + $G_V=0.1$~fm$^2$ for the quark phase.}\label{fig:calor_latente}
\end{centering}
\end{figure}

In Figs.~\ref{fig:energia_latente} and ~\ref{fig:calor_latente} we present the results obtained with the latent energy and latent heat, respectively as a function of the temperature at the point of the hadron-quark phase transition. We choose only one parametrization since they yield qualitatively 
similar results. One can see that our results for the latent energy at zero temperature are of the same order of magnitude as the ones obtained in \cite{pelicer2022phase}, but never reach the values shown in \cite{lope2022maximum} and are always positive. 
As far as the latent heat is concerned, the behaviour we find is more similar to the result shown in \cite{roark2019hyperons} and we do find a maximum point, but at much lower temperatures. Our curves then cross the zero value, which, according to \cite{agasian2008quark} is an indication of the end of the first order phase transition. Assuming this interpretation is correct, our curves depicted in the QCD phase diagrams are only valid until the temperatures where $L|_S$ is zero and the corresponding values are shown as dots also in Fig.\ref{fig:all_in_one}. 



%%%%%%%%%%%%%%%%%%%%%%%%%%%%%%%%%%%%%%%%%%%%%%%%%%%%%%%

-----------------------------------------

%The latent energy quantifies the intensity of the phase transition, i.e., the discontinuity in the energy density between phases.

----

%A seguir só coisa antiga onde foi usado a definição apenas para energia latente.


%In Fig.~\ref{fig:stellar_matter_latent-1x1} we present the relation between latent heat, as given by Eq.~\ref{eq:energia_latente}, and temperature at the point of the hadron-quark phase transition. We chose only some parametrizations to illustrate its behavior as they are qualitatively the same for all the different parametrizations. As we see, the latent heat grows with the temperature. This is a little different from what happens for the two-flavour symmetric matter, where the $L_{|\epsilon}$ starts with values a little higher at T=0, decreases and then increases again, but not as much as here, reaching values around 0.14 only.  Smaller $B_0$ results in higher $L_{|\epsilon}$ values, and the Maxwell prescription also results in higher $L_{|\epsilon}$ values than the flavour conservation prescription. As the latent heat reflects the discontinuity in the energy density between phases, we conclude that, in general, this discontinuity is bigger when using the Maxwell prescription and even more so when we use $B_0^{1/4}=148$~MeV with this prescription. 



%\begin{figure}[ht] 
%\begin{centering}
%\includegraphics[angle=0,width=0.5\textwidth]{figuras/mat_estelar-calor-1x1.pdf}
%\caption{Results for the latent heat throughout the entire phase diagram for strange stellar matter. The solid lines stand for the flavour conservation prescription and the dashed ones for the Maxwell prescription.}\label{fig:stellar_matter_latent-1x1}
%\end{centering}
%\end{figure}

%As for the latent heat for stellar matter without strange matter, its behavior is somewhere in between the of two-flavour symmetric matter and strange stellar matter, with higher $L_{|\epsilon}$ values than the two-flavour symmetric matter, but smaller than the strange stellar matter ones. 
