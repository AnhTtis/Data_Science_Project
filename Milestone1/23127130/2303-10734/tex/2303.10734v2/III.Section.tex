
\section{Conditions for phase coexistence - Gibbs' conditions}

Two states of matter which can exist simultaneously in equilibrium with each other and in contact are described as different phases. The equilibrium conditions can be used to determine when the transition from one phase to another occurs. For the two phases to be in equilibrium it is necessary that, first of all, their temperatures are the same. Also, as the forces exerted by the two phases on each other at their surface of contact must be equal and opposite, the pressures have also to be equal. And, finally, the chemical potentials have to be identical \cite{landau2013course}. 

As in our case the two phases considered are the hadronic phase ($H$) and the quark phase ($Q$), we can write:

\begin{align}\label{condicoes_transicao_HQ}
	& T^{H} = T^{Q} = T,\nonumber\\
	& P^{H} = P^{Q} = P_0 \\
	& \mu^{H}(P_0,T) = \mu^{Q}(P_0,T) = \mu_0,\nonumber
\end{align}
%with

%\begin{equation}
%    \mu^{(f)}=\frac{\epsilon^{(f)}+P^{(f)}-s^{(f)}T}{n_B^{(f)}},
%\end{equation}
%where $\epsilon^{(f)}$, $P^{(f)}$, $s^{(f)}$ and $n_B^{(f)}$ are the total energy density, pressure, entropy density and number density of the phase $f=\{H,Q\}$. {\color{red}esta eq. só é válida quando há equilíbrio beta, ou seja, é preciso deixar essa equação completamente de fora ou também adicionar MUQ=(yu*MUu+yd*MUd+ys*MUs)*3.d0+ye*mue+ymu*mue)}

These are the necessary conditions for thermodynamic equilibrium of the hadronic and quark phases, also called Gibbs' conditions.  The detailed calculation of the relevant physical quantities can be easily  found in the literature for both the hadronic (ref.~\cite{NLWM,PhysRevC.90.055203,shao2011phase}) and the quark (ref.~\cite{lopes2021modified,lopes2021modified_pt2} phases. The total baryonic chemical potential for the hadron ($\mu^H$) and the quark ($\mu^Q$) phases are given in terms of its individual constituents~\cite{lopes2022hypermassive}:

\begin{eqnarray}
\mu^H_B = \frac{(\sum_H n_H \mu_H + \sum_l n_l \mu_l)}{\sum_H n_H}, \nonumber \\
\mu^Q_B = \frac{3(\sum_Q n_Q \mu_Q + \sum_l n_l \mu_l)}{\sum_Q n_Q},  \label{chempot}
\end{eqnarray}
where $H$, $l$ and $Q$ stand for hadrons, leptons and quarks, respectively.
It is also worth emphasizing that due to the model limitations and the prescriptions we use, Eq.~\ref{condicoes_transicao_HQ} gives rise to
a first order phase transition, while a more realistic approach should reproduce a smooth crossover at low chemical potentials~\cite{aoki2006order,bellwied2015qcd}. 
In the last section of this paper, we analyse possible ways to limit the temperature and chemical potentials at which the critical end point is reached and hence, define the limits of our calculation.

%Nevertheless, in this work we are more interested in the physical quantities as the (pseudo) critical  temperature and chemical potentials, instead of analysing the order of the phase transition. }

%Phase transitions can be of first or second order. If the first order derivative of the Gibbs free energy with respect to some intensive variable, such as entropy, for example, shows a discontinuity, the phase transition will be of first order. If, however, only some derivative of second order is discontinuous, we say that the phase transition is of second order.


\begin{figure}[ht] 
\begin{centering}
 \includegraphics[angle=0,width=0.45\textwidth]{figuras/cruz.pdf}
\caption{Relation between pressure and chemical potential for the hadron (dashed lines) and quark (solid lines) phases, respectively, described by the eL3$\omega \rho$ parametrization and $B_0^{1/4}=165$~MeV considering three different temperatures. Both phases are of symmetric matter. The red dots are the points where the Gibbs' conditions are satisfied.}  \label{fig:cruzamento}
\end{centering}
\end{figure}


In Fig.~\ref{fig:cruzamento} we show examples where the Gibbs' conditions are met for various temperatures. In this case we use symmetric matter for both phases, with $\mu_n=\mu_p$ for the hadron phase and $\mu_u=\mu_d$ for the quark phase. But in this work we will also explore constructions using stellar matter, i.e., a charge neutral matter in $\beta$-equilibrium, as already explained in the Introduction of the paper.
%First we will impose charge neutrality and $\beta$-equilibrium to the hadron phase and then consider flavour conservation so that the quark phase is also charge neutral but not yet $\beta$-stable, and second we will impose charge neutrality and $\beta$-equilibrium to both phase, in the so called Maxwell construction. 
In all cases, the phase transition points are found as shown in Fig.~\ref{fig:cruzamento}.

%In the present work, due to limitations of the effective models employed, we only deal with first order phase transitions. In this kind of transition, the absorption or loss of energy in the form of latent heat is always possible \cite{greiner_thermodynamics} and this quantity will be discussed later on, in a future section. Esse parágrafo agora se tornou redundante.

%In addition, due to the conditions of phase coexistence, a mixed phase will appear during the process {\bf Rever isso ai.... Maxwell construction eh de primeira ordem e naum tem coexistencia de phase}. In a second order transition, the thermodynamic properties vary continuously during the transition throughout the system and the appearance of a mixed phase is not allowed . 
%In the QCD phase diagram there may be second order transitions due to the chiral transition that occurs at low temperatures and high densities \cite{Nuclear_Physics_B399}. However, in the present work, because the effective models employed do not contain information about chirality, we only deal with first order phase transitions.




