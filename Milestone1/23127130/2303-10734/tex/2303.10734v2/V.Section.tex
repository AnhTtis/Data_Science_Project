\section{Phase Diagrams}

In this section we present the results first for two-flavour symmetric matter and then for stellar matter.

\subsection{Symmetric Matter}\label{sec:results-Sym_matter}




In Fig.~\ref{fig:symmetric_matter} we present the phase diagrams obtained considering two-flavour symmetric matter for both phases, with $\mu_n=\mu_p$ and $\mu_d=\mu_u$. In each graphic we present three curves, one where we use the NL3$^* \omega \rho$ parametrization (blue line) and other where we use the eL3$\omega \rho$ parametrization (purple line) in combination with the  MIT bag model and we also present the curve obtained using only the MIT bag model (red line), i.e., instead of constructing the phase diagram with two models, quark matter pressure is simply set to be zero (see ref. \cite{lopes2021modified_pt2}, for details). The graphics at the top are obtained using a bag pressure value $B_0^{1/4}=148$~MeV and for the ones at the bottom we use $B_0^{1/4}=165$~MeV. From left to right we vary the value of the vector channel of the MIT bag model where we use $G_V=0.0$, $G_V=0.05$~fm$^2$ and $G_V=0.1$~fm$^2$.

We first take a look at the region at low chemical potentials. As can be seen, all curves converge more or less at the same point, so that the maximum temperatures (by maximum temperature $T_{max}$ we mean merely the highest temperature where we are able to satisfy the phase transition conditions at $\mu \rightarrow 0$) obtained are above the freeze-out line. This happens because we choose the values of $T_0$ exactly to satisfy the restrain imposed by the LQCD and freeze-out results, which states that the temperatures at very low chemical potentials should be around  $T=168$~MeV. In \cite{biesdorf2022qcd} we have already verified that the maximum temperature obtained depends only on the bag pressure value: the higher the $B^{1/4}$, the higher the maximum temperature. The same conclusion can be drawn here. %because $B(T)^{1/4}$ has more or less the same value, namely, around 205 MeV, at the maximum temperature for both parametrizations. So, as both maximum  values of $B(T)^{1/4}$ are pretty much the same, so are the values of the maximum temperatures.
Following, it is interesting to note that the low chemical potential
results obtained using the combination of the two models are the same as the results using only the quark model. The explanation for this is that the Gibbs' conditions at this region are met at very low pressures ($P_0 \leq 5$~MeV/fm$^3$), as can be seen in Fig. \ref{fig:cruzamento}, and so, as the condition for phase transition when using only the MIT bag model is $P=0$, it comes with no surprise that this results are very much alike. The same is true when we use a bag pressure value with no dependence on the temperature, as done in \cite{biesdorf2022qcd}.

Now we look at the region of the diagrams at high chemical potentials, i.e., $T \rightarrow 0$. We see that the results are different for each parametrization. First we analyze the effect of the different MIT bag based model parametrizations. The most evident effect is that a higher bag pressure value results in a higher chemical potential. This follows what happens when using only the MIT model, a case easy to understand: in the equation for the pressure, the value of the Bag subtracts from the value of the pressure of the quarks (likewise, the meson $\omega$ when $G_V \neq 0$) and so, the higher the Bag value the higher must the contribution for the pressure coming from the quarks be in order to meet the condition for phase transition ($P=0$) and, consequently, the chemical potential is also higher. When we use two models, however, the increase in the chemical potential is even greater and this happens because the Gibbs' conditions are met at pressures higher than zero ($P_0 > 0$) so that the pressure realized by the quarks and $\omega$ meson is even higher, which also increases the chemical potential at the transition point $\mu_0$.

The inclusion and increase of the $G_V$ value also follows what happens when we use only one model, the higher its value, the higher the chemical potential. As already stated in \cite{lopes2021modified_pt2}, the vector field causes an additional repulsion between the quarks, increasing the chemical potential of the phase transition. And again, as $P_0 > 0$, the effect when using two models is increased, so much so that here we use smaller values for $G_V$ than used in our previous work \cite{lopes2021modified_pt2}.

As for the influence of the different parametrizations of the QHD based model, we can see that the eL3$\omega \rho$ parametrization always results in higher chemical potentials, and, as the NL3$^* \omega \rho$ is a stiffer EoS, we conclude the same as already noticed in \cite{biesdorf2022qcd}, a stiffer hadron EoS results in a lower chemical potential. 

The behavior of the diagrams at low temperatures is not so smooth, as we can observe an abrupt increase of the chemical potential, especially when using a bag pressure value of 148 MeV. This happens because the lower the temperature, the smaller the change in the graphs of $\mu$ x P for different temperatures. And more so, at lower temperatures the curves for the hadron and the quark phases are very close to each another. As a result, a small change in the temperature makes the crossing point of the lines take a big leap resulting in an abrupt change in the chemical potential on the phase diagram.

%The differences for the maximum chemical potentials between different QHD parametrizations for a fixed bag value also increase when the bag value is higher.

In Table \ref{tab:pot_max} we present the values of the maximum chemical potential obtained for each combination of parametrizations used in Fig.~\ref{fig:symmetric_matter}. By maximum chemical potential $\mu_{max}$ we mean the chemical potential where a phase transition at $T=0$ takes place.
%, which corresponds to the maximum chemical potential of the phase diagram.  
Only a few combinations of parametrizations result in a chemical potential within the range $1050 \leq \mu \leq 1400$~MeV, namely: NL3$^*\omega \rho$ in combination with $B^{1/4}=148$~MeV + $G_V=0.05$~fm$^2$ and $G_V=0.1$~fm$^2$ and $B^{1/4}=165$~MeV + $G_V=0.0$ and $G_V=0.05$~fm$^2$ and eL3$\omega \rho$ in combination with $B^{1/4}=148$~MeV. When using only the MIT based model the results for $B^{1/4}=165$~MeV + $G_V=0.05$~fm$^2$ and $G_V=0.1$~fm$^2$ are also within this range.

\begin{widetext}
\begin{center}
\begin{table}[]
\caption{Chemical potentials ($\mu_{max}$) at $T=0$ for all combinations of parametrizations considering \textbf{symmetric} matter. The first line corresponds to the results obtained with the MIT based bag models only, where the phase transition criterion is just the value of the chemical potential where the pressure goes to zero, as done in \cite{lopes2021modified_pt2}.
All values are given in MeV }
%\begin{tabular}{@{}ccccccccccc@{}}
\begin{tabular}{p{1.5cm}|p{2.5cm}|p{2.5cm}|p{2.5cm}|p{2.5cm}|p{2.5cm}|p{2.5cm}}
\toprule[0.6pt]
 Models  & $B_0^{1/4}=148$~MeV  & $B_0^{1/4}=148$~MeV + $G_V=0.05$~fm$^2$  & $B_0^{1/4}=148$~MeV + $G_V=0.1$~fm$^2$ & $B_0^{1/4}=165$~MeV  &   $B_0^{1/4}=165$~MeV + $G_V=0.05$~fm$^2$    &   $B_0^{1/4}=165$~MeV + $G_V=0.1$~fm$^2$   \\ \midrule[1.5pt]
 - & $\mu_{max}=936$  & $\mu_{max}=948$  & $\mu_{max}=959$  & $\mu_{max}=1043$ &   $\mu_{max}=1060$ &   $\mu_{max}=1075$  \\
 NL3$^*\omega \rho$ & $\mu_{max}=976$  & $\mu_{max}=1084$  & $\mu_{max}=1212$  & $\mu_{max}=1260$   &   $\mu_{max}=1345$    &   $\mu_{max}=1441$  \\
 eL3$\omega \rho$  & $\mu_{max}=1197$  & $\mu_{max}=1434$  & $\mu_{max}=1655$  & $\mu_{max}=1431$    &   $\mu_{max}=1592$    &   $\mu_{max}=1783$ \\\bottomrule[0.6pt]
\end{tabular}\label{tab:pot_max}
\end{table}
\end{center}
\end{widetext}


As we chose $T_0$ in order to obtain higher temperatures at low chemical potentials, we are able to obtain a maximum temperature around $T=168$~MeV in all combination of parametrizations, but we are only able to fit the freeze-out line entirely inside the confined (hadron) phase for the combinations that include a $B_0^{1/4}=165$~MeV and a QHD based model. Therefore we have only obtained two combinations of parametrizations that satisfy both constraints for low and high temperatures: NL3$^*\omega \rho$ + $B_0^{1/4}=165$~MeV and NL3$^*\omega \rho$ + $B_0^{1/4}=165$~MeV + $G_V=0.05$~fm$^2$. 



%\newpage
%%%%%%%%%%%%%%%%%%%%%%%%%%%%%%%%%%%%%%%%%%%%%%%%%%%%%%%%%%%%%%%%%%%%%%%%%%%%%%%%%%%%%%%%%%%%%%%

\subsection{Stellar Matter}\label{sec:results-Stel_matter}

Here we present the results for charge neutral matter in $\beta$-equilibrium for two scenarios, one considering flavour conservation, which results in quark matter that is not in $\beta$-equilibrium, and another with no flavour conservation where both phases are in $\beta$-equilibrium, a typical Maxwell prescription used to construct hybrid stars, as done in \cite{lopes2021hyperonic}, for example. But first we analyze the results for a sub-scenario within the scenario of flavour conservation, where we analyze different prescriptions for the lepton matter during the phase transition. In the first one, the lepton fraction, defined as the fraction between lepton density and the baryonic density of each phase is preserved at the point of the phase transition and in the other one, the lepton density is taken to be the same, and consequently the same energy density and pressure are obtained for both phases at the point of the hadron-quark phase transition. In Fig.~\ref{fig:lepton_conser} we compare the two prescriptions. %The solid lines represent the prescription of lepton fraction conservation and the dashed lines represent the prescription of lepton density conservation at the point of the phase transition. 

%\subsubsection{flavour conservation}

\begin{figure*}[ht] 
\begin{centering}
 \includegraphics[angle=0,width=1.0\textwidth]{figuras/cons_lept-2.pdf}
\caption{Comparison between two different prescriptions for the lepton matter in the quark phase for strange stellar matter considering flavour conservation. Here we present the results for the NL3$^*\omega \rho$ and different parametrizations for the MIT based model. The solid lines represent the prescription of lepton fraction conservation and the dashed lines represent the prescription of lepton density conservation. }\label{fig:lepton_conser}
\end{centering}
\end{figure*}


As the different prescriptions only affect the quark phase, we choose here to present the results of only one of the QHD based model, namely, the NL3$^* \omega \rho$, and vary the MIT based model parametrizations. As can be seen, the differences are very small. They are greater for the bag pressure value of $B_0^{1/4}=148$~MeV (blue lines) than for the $B_0^{1/4}=165$~MeV (red lines), and also decrease with the increase of the $G_V$ constant. But for all parametrizations the prescription considering the lepton fraction equal at both phases at the transition point (solid lines) is the one where the chemical potentials are always slightly higher. Because of the imposition of flavour conservation and the Gibbs' conditions as criteria for the phase transition, there is an increase of the baryon density at the point of the phase transition, so, in the sub-scenario where we impose the lepton fraction conservation this means also an increase, in the same proportion, of the lepton density, which, in turn, means an increase in the chemical potential. As the contribution to the chemical potential from the leptons is very small, these differences are also very small.

From this sub-scenario we choose the one where there is a conservation of the lepton fraction at the point of the phase transition because, in this case, the charge neutrality is also preserved, This approach is contrary to what was done in \cite{pelicer2022phase}. We find that this is more compelling to be true than the case where we have the same lepton density at the point of the phase transition, but no charge neutrality.
%{\color{red} Sera que devemos mexer nesse palheiro? Estamos lidando com materia infinita...nao seria melhor deixar a discussao sobre o volume so entre nos?
%As there is an increase of the baryon density at the phase transition there may be a decrease in the volume occupied by the quark phase in comparison to the one occupied by the hadron phase and, as the leptons accompany the quarks to this new phase, they may also be in this smaller volume. To summarize, in this sub-scenario, at the quark phase we have the same quarks that were (confined) in the hadron phase and also the same leptons, changing only the volume from one phase to another.} 


In the following we present in Fig.~\ref{fig:stellar_matter} the phase diagrams for stellar matter and compare the results considering flavour conservation (top) and the Maxwell prescription (bottom). The blue lines are the ones where we use the NL3$^*\omega \rho$ parametrization, the purple ones are for the eL3$\omega \rho$ parametrization and the red ones we use to display again also the results obtained using only the MIT based models where, in this case, the quark matter is always $\beta$-stable. Furthermore, the solid lines stand for the results using a bag pressure value $B_0^{1/4}=148$~MeV and the dashed lines for $B_0^{1/4}=165$~MeV. The value of G$_V$ increases from 0 to 0.3 fm$^2$ from left to right.


\begin{figure*}[ht] 
\begin{centering}
 \includegraphics[angle=0,width=1.0\textwidth]{figuras/mat_estelar-2.pdf}
\caption{Phase diagrams for strange stellar matter considering the NL3$^*\omega \rho$ and eL3$\omega \rho$ parametrizations for the hadronic matter and different MIT bag based models for two temperature dependent bag $B(T)$ values for the quark matter. At the top we use the prescription of the flavour conservation and at the bottom the of Maxwell.}  \label{fig:stellar_matter}
\end{centering}
\end{figure*}


As expected, the general behavior is the same as observed in Fig.~\ref{fig:symmetric_matter} for symmetric matter. For both prescriptions, only the bag pressure value dictates the maximum temperature obtained for $\mu \rightarrow 0$. At low temperatures,  the maximum chemical potential depends largely on the bag pressure value, but also on the inclusion or not of the vector field to the bag model and the parametrization chosen for the QHD model.

As done above, we analyse first the region of low chemical potentials. At this region we have $T_{max}=129$~MeV for $B_0^{1/4}=148$~MeV and $T_{max}=138$~MeV for $B_0^{1/4}=165$~MeV no matter the QHD parametrization, if we add the vector channel to the MIT model or not or even if we use the prescription of Maxwell or of flavour conservation. Which is different from what happened in Fig.~\ref{fig:symmetric_matter}, where the results for $T_{max}$ are more or less the same, i.e. around $T_{max}=168$~MeV, no matter the value of $B_0$, but that is because we used the symmetric matter to chose the values of $T_0$ in Eq.~\ref{eq:B(T)} exactly so that the maximum temperatures where around this value%, which means that the values of $B(T)$ for both parametrizations are more or less the same at its $T_{max}$
. Here, as we are not able to reach the same $T_{max}$, the maximum values of $B(T)$ are different for each value of $B_0$, which, in turn, means that the maximum temperatures here are also different and depend on the value of $B_0$. Also differently from what happened with symmetric matter, here the results for $T_{max}$ do not match the results obtained via MIT based model alone. This is so, for both prescriptions, because the crossing of the curves $\mu \times P$ occurs at pressures that are not so close to zero as in the symmetric matter case. At the same time, these pressures are low enough so that they fall in a region where the hadronic EoSs for stellar matter are stiffer than the ones for symmetric matter, and so we have again that a softer hadronic EoS favors the hadron phase.

Now we analyse the region at low temperatures. Here, as for the symmetric matter, the results are different for each combination of parametrizations and they also change depending on the prescription used, either flavour conservation or Maxwell construction. The maximum chemical potentials obtained for each combination of parametrizations are presented in Table~\ref{tab:pot_max_est}. 

\begin{widetext}
\begin{center}
\begin{table}[]
\caption{Chemical potentials ($\mu_{max}$) at $T=0$ for all combinations of parametrizations considering strange \textbf{stellar} matter. The first line corresponds to the results using the MIT based bag models only, where the phase transition criterion is just the value of the chemical potential where the pressure goes to zero, as done in \cite{lopes2021modified_pt2}. All values are given in MeV.}
%\begin{tabular}{@{}ccccccccccc@{}}
\begin{tabular}{p{1.5cm}|p{2.5cm}|p{2.5cm}|p{2.5cm}|p{2.5cm}|p{2.5cm}|p{2.5cm}}
\toprule[0.6pt]
 Models  & $B_0^{1/4}=148$~MeV  & $B_0^{1/4}=148$~MeV + $G_V=0.1$~fm$^2$  & $B_0^{1/4}=148$~MeV + $G_V=0.3$~fm$^2$ & $B_0^{1/4}=165$~MeV  &   $B_0^{1/4}=165$~MeV + $G_V=0.1$~fm$^2$    &   $B_0^{1/4}=165$~MeV + $G_V=0.3$~fm$^2$   \\ \midrule[1.5pt]
 - & $\mu_{max}=867$  & $\mu_{max}=884$  & $\mu_{max}=915$  & $\mu_{max}=962$ &   $\mu_{max}=986$ &   $\mu_{max}=1027$  \\\midrule[1.5pt]
 \multicolumn{7}{c}{flavour conservation}\\
\bottomrule[0.6pt]
 NL3$^*\omega \rho$ & $\mu_{max}=960$  & $\mu_{max}=1089$  & $\mu_{max}=1268$  & $\mu_{max}=1192$   &   $\mu_{max}=1253$    &   $\mu_{max}=1462$  \\
 eL3$\omega \rho$  & $\mu_{max}=959$  & $\mu_{max}=1181$  & no crossing  & $\mu_{max}=1225$    &   $\mu_{max}=1324$    &   no crossing \\\midrule[1.5pt]
 \multicolumn{7}{c}{Maxwell prescription}\\
\bottomrule[0.6pt]
NL3$^*\omega \rho$ & $\mu_{max}=867$  & $\mu_{max}=884$  & $\mu_{max}=915$  & $\mu_{max}=967$   &   $\mu_{max}=1012$    &   $\mu_{max}=1142$  \\
 eL3$\omega \rho$  & $\mu_{max}=867$  & $\mu_{max}=884$  &  $\mu_{max}=915$ & $\mu_{max}=967$    &   $\mu_{max}=1013$    & $\mu_{max}=1727$   \\\bottomrule[0.6pt]
\end{tabular}\label{tab:pot_max_est}
\end{table}
\end{center}
\end{widetext}

Let us first scrutinize the effect of the different MIT bag based model
parametrizations at this region of low temperatures. %{\color{red} The first thing that we can point out is that the results for $\mu_{max}$ here are not so sensible to the inclusion of the vector field to the MIT model, in fact, when using $G_V=0.05$~fm$^2$ its influence is barely noticeable and all results for $\mu_{max}$ are very similar. Because of this we don't even present those results here and use higher values for $G_V$ in order to analyse the influence of this term. Nevertheless, although we need higher values here, we have the same effect as for the symmetric matter: the higher the values for $G_V$ the higher are the results for $\mu_{max}$. }
The effect of $G_V$ here, in stellar matter, causes smaller changes in the $\mu_{max}$, which allows us to increase the values up to 0.3 fm$^2$. Nevertheless the qualitative results are the same: the higher the values for $G_V$ the higher are the results for $\mu_{max}$.

We can still compare the results between symmetric matter and stellar matter when using the same parametrizations, namely, when $G_V=0$ and $G_V=0.1$~fm$^2$. In doing so we can see very clearly that there is a reduction in the values of chemical potentials at $T=0$ here, being this reduction even greater when we use the Maxwell prescription. This reduction also grows with the value of $G_V$ and $B_0^{1/4}$ and is also greater when using the eL3$\omega \rho$ than when using the NL3$^*\omega \rho$ parametrization. 

Being that the quark EoSs for symmetric matter are always stiffer than the ones for stellar matter and that the ones for stellar matter within the flavour conservation prescription are stiffer that the $\beta$-stable and charge neutral ones, in general, stiffer quark EoSs favor the hadron phase. The same is true when we compare the results for different values of $G_V$, as higher values of $G_V$ result in higher values for $\mu_{max}$ and also increase the stiffness of the EoS, the same behaviour found in symmetric matter. The increase of the bag pressure value $B_0^{1/4}$ also results in higher values of $\mu_{max}$, but it softens the EoS, as we already pointed out in \cite{biesdorf2022qcd} for symmetric matter. So, to summarize, stiffer quark EoSs always favor the hadron phase, unless we change the value of the Bag constant $B_0^{1/4}$ and then the opposite is true, a softer quark EoS favors the hadron phase.


An interesting aspect to analyse when comparing the results at low temperatures obtained from the two different prescriptions is their behavior in relation to the results obtained via MIT bag based model alone. As it is very clear from Fig.~\ref{fig:stellar_matter} and Table~\ref{tab:pot_max_est}, when we use the prescription of flavour conservation the results are quite different, whereas when using the Maxwell prescription the diagrams generally converge to the same $\mu_{max}$s. First we have to point out that the diagrams obtained via MIT bag based model alone (red lines) presented at the top and at the bottom of Fig.~\ref{fig:stellar_matter} are the same, i.e., of $\beta$-stable and charge neutral quark and lepton matter. Now, when we use a QHD+MIT models, with the flavour conservation the hadronic phase completely determines the quark phase through the bond given by Eq.~\ref{eq:charge_conser}. The quark matter at the phase transition point is not yet $\beta$-stable, and, adding to that, in this case, we have $P_0$s that are not so close to zero, so, it is not surprising that the results for this prescription are different from the ones obtain only with the MIT based model. When using the Maxwell prescription, however, we have a quark matter at the phase transition point that is $\beta$-stable, and, when the the Gibbs' conditions at this region are met at very low pressures ($P_0 \leq 5$~MeV/fm$^3$), we get $\mu_{max}$s very similar to the ones obtained via MIT alone. When the $P_0$s start to get higher, also the $\mu_{max}$s grow. 

As for the effects of the different QHD model parametrizations at this region of low temperatures, we can see that they are not so preponderant here as for the symmetric matter and are the smallest when using the Maxwell prescription. But, whenever there is a difference, the $\mu_{max}$s for the eL3$\omega \rho$ are the highest and in some cases, we were not able to find crossings at $T=0$, as for example, the case with $G_V=0.3$~fm$^2$ and this QHD parametrization. The reason for this behaviour is the fact that the onset of the hyperons softens the EoS in the hadronic phase, while the $G_V$ monotonically stiffens the EoS of the quark phase. At very high density we expect that the quark phase becomes non-interacting, therefore the contribution of the Dirac sea can no longer be neglected.
%{\color{red}So, we conclude once again, that stiffer hadronic EoSs result in lower chemical potentials at the transition point. But when we compare the stiffness of the hadronic EoSs of symmetric matter with the ones of stellar matter this statement is not always true. The hadronic EoSs for stellar matter are stiffer than the ones for symmetric matter up until pressures around 50 MeV/fm$^3$, when the symmetric ones become stiffer, and we have $P_0$s for stellar matter above and below this value and the same for symmetric matter. In general we have that, the higher the $\mu_{max}$ the higher is the value of $P_0$, no matter if symmetric or stellar matter.}{

To finalize the analysis of the results in this region of low temperatures we point out that we obtain results for $\mu_{max}$ within a range of 1000 MeV from the $\mu=1200$~MeV indicated in \cite{ayriyan2018robustness} for $\beta$-stable matter and various combinations of parametrizations when using the prescription of flavour conservation, namely, for NL3$^*\omega \rho$ in combination with $B_0^{1/4}=148$ + $G_V=0.3$~fm$^2$, $B_0^{1/4}=165$ + $G_V=0$ and $B_0^{1/4}=165$ + $G_V=0.1$~fm$^2$ and for eL3$\omega \rho$ in combination with $B_0^{1/4}=148$ + $G_V=0.1$~fm$^2$ and $B_0^{1/4}=165$ + $G_V=0$, but only for one combination of parametrizations when using the Maxwell prescription, namely, for NL3$^*\omega \rho$ in combination with $B_0^{1/4}=165$ + $G_V=0.3$~fm$^2$. So, considering the MIT parametrizations, only two, the ones with $B_0^{1/4}=148$, are within the stability window and are able to describe stable strange matter \cite{lopes2021modified}. However, in order to construct hybrid stars, parametrizations outside the stability window are preferable to avoid that once the phase transition starts, the entire star converts to a strange quark star. In \cite{lopes2022hypermassive} we constructed hybrid stars using the original L3$\omega \rho$ parametrization (a slight different paramaetrization than the one used here) for the QHD model and the MIT based model. 

We could improve the results at this region by adding also a quartic term to the MIT vector model, as done in \cite{lopes2021modified,lopes2021modified_pt2}. Using the very small value of $b_4=2$ we are able to obtain crossings at T=0 for the two cases pointed out in table~\ref{tab:pot_max_est} where we did not obtain results without this term, although the values for the $\mu_{max}$, and also the pressure, would still be very high. Choosing higher values for $b_4$ we can adjust the values of $\mu_{max}$ without altering the values of $T_{max}$ and, the higher the value of $G_V$ the higher is the affect of this term, as already shown in \cite{lopes2021modified_pt2}. %The sensibility of the results to this term would be much higher in the present work.

%{\bf ficou estranho... voce fala que o termo b4 pode fazer varias coisas, e os resultados sao muito sensiveis a esse termo... mas a gente simplesmente ignora? Tipo.. isso aqui pode afetar mas deixa pra la.... nao sei se gosto do jeito que esta apresentado}




%%%%%%%%%%%%%%%%%%%%%%%%%%%%%%%%%%%%%%%%%%%%%%%%%%%%%%%%%%%%%%%%%%

\subsection{Testing different hypotheses}

%A better comparison between the different matter hypothesis and prescriptions and stellar matter without strange matter}


In this section we take a better look at the differences occurring due to the different matter hypothesis and prescription imposed along this work. In order to do so we choose only two different combination of parametrizations and plot all results using the same combination in  Fig.~\ref{fig:all_in_one}. We  also include the results for stellar matter with no strange matter, i.e., with no hyperons and strange quarks (dashed lines).


\begin{figure}[ht] 
\begin{centering}
\includegraphics[angle=0,width=0.5\textwidth]{figuras/todos--2x1.pdf}
\caption{Phase diagrams for all different matter hypothesis and prescriptions for phase transition used in this work considering the NL3$^* \omega \rho$ parametrization for the hadronic phase and $B_0^{1/4}=148$~MeV + $G_V=0.1$~fm$^2$ (top) and $B_0^{1/4}=165$~MeV + $G_V=0.1$~fm$^2$ (bottom) for the quark phase. We also include the results for stellar matter with no strange matter (dashed lines). The points presented on the bottom diagrams correspond to the ones where the latent heat $L|_S$ goes to zero (see section~\ref{sec:latent_heat}) and may correspond to the critical end points. } \label{fig:all_in_one}
\end{centering}
\end{figure}

As already pointed out above and is more evident here, the maximum temperature for strange %{\color{blue} quark} 
stellar matter depends only on the bag pressure value and is the same for both prescriptions, the one with flavour (solid red line) conservation as well as the one with Maxwell construction (solid blue line). Its also more evident here that at low temperatures the prescription of flavour conservation favors the hadronic phase. What is completely new here, though, are the diagrams for stellar matter with no hyperons in the hadronic matter and no strange quark in the quark matter. We chose only to present those results for this two combination of parametrizations because its general behavior follows the ones analysed until now, i.e., the different parametrizations have the same affect to this matter as for the previous cases.

As could be expected, the results for the diagrams for stellar matter without strangeness are very close to the ones for two-flavour symmetric matter. This because the contribution to the chemical potential coming from the leptons is very small.

At very low temperatures we have obtained an interesting feature for the flavour conservation prescription. For the NL3$^* \omega \rho$ + $B_0^{1/4}=148$~MeV + $G_V=0.1$~fm$^2$ parametrization, the phase transition for the case where the strange matter is allowed to appear occurs at chemical potentials where there is no hyperon onset, so, at this region, the results for stellar matter with (solid red line) and without (dashed red lines) strange matter are the same. And at higher temperatures, where the hyperons start to appear at the point of the phase transition, the two lines start to differentiate and the line without strange matter approximates itself from the line of two-flavour symmetric matter. For the other parametrization, i.e., NL3$^* \omega \rho$ + $B_0^{1/4}=165$~MeV + $G_V=0.1$~fm$^2$, however, the phase transition for strange matter occurs at chemical potentials where there are already hyperons present and so, the results for stellar matter with and without strange matter are different throughout the entire phase diagram. As for the Maxwell prescription, despite the phase transition for strange matter at low temperatures for both parametrizations occurring at chemical potentials where there are no hyperons, the results for strange matter (solid blue lines) and no strange matter (dashed blue lines) are always very different. This happens because for strange quark stellar matter the strange quarks are always present, as can be seen in \cite{lopes2021modified}.  


In Table \ref{tab:pot_max_est-ns} we present the results for the maximum chemical potentials at the phase transition for stellar matter without strange matter. There we also include the results when using only the MIT based model and, as can be noted, its results are only similar to the ones when combining models when we use $B_0^{1/4}=148$~MeV and $G_V=0$. This because in this cases $P_0 \rightarrow 0$ and for all the other parametrizations the phase transition happens at increasing pressures. Yet about the results for stellar matter without strange matter using only the MIT based model, its $T_{max}$s reach around only T=140 MeV, i.e., very below the ones obtained for the same matter but combining models. That's because when combining models the crossing of the $\mu \times P$ occurs at pressures considerably high, and higher than for strange matter.

\begin{widetext}
\begin{center}
\begin{table}[]
\caption{Chemical potentials ($\mu_{max}$) at $T=0$ for all combinations of parametrizations considering \textbf{stellar} matter \textbf{without strange matter}. All values are given in MeV }
%\begin{tabular}{@{}ccccccccccc@{}}
\begin{tabular}{p{1.5cm}|p{2.5cm}|p{2.5cm}|p{2.5cm}|p{2.5cm}|p{2.5cm}|p{2.5cm}}
\toprule[0.6pt]
 Models  & $B_0^{1/4}=148$~MeV  & $B_0^{1/4}=148$~MeV + $G_V=0.1$~fm$^2$  & $B_0^{1/4}=148$~MeV + $G_V=0.3$~fm$^2$ & $B_0^{1/4}=165$~MeV  &   $B_0^{1/4}=165$~MeV + $G_V=0.1$~fm$^2$    &   $B_0^{1/4}=165$~MeV + $G_V=0.3$~fm$^2$   \\ \midrule[1.5pt]
 - & $\mu_{max}=953$  & $\mu_{max}=976$  & $\mu_{max}=1017$  & $\mu_{max}=1063$ &   $\mu_{max}=1094$ &   $\mu_{max}=1149$  \\\midrule[1.5pt]
 \multicolumn{7}{c}{flavour conservation}\\
\bottomrule[0.6pt]
 NL3$^*\omega \rho$ & $\mu_{max}=960$  & $\mu_{max}=1089$  & $\mu_{max}=1750$  & $\mu_{max}=1259$   &   $\mu_{max}=1431$    &   $\mu_{max}=1947$  \\
 eL3$\omega \rho$  & $\mu_{max}=959$  & $\mu_{max}=1568$  & no crossing  & $\mu_{max}=1405$    &   $\mu_{max}=1751$    &   no crossing \\\midrule[1.5pt]
 \multicolumn{7}{c}{Maxwell prescription}\\
\bottomrule[0.6pt]
NL3$^*\omega \rho$ & $\mu_{max}=955$  & $\mu_{max}=1042$  & $\mu_{max}=1671$  & $\mu_{max}=1229$   &   $\mu_{max}=1391$    &   $\mu_{max}=1886$  \\
 eL3$\omega \rho$  & $\mu_{max}=957$  & $\mu_{max}=1450$  &  no crossing & $\mu_{max}=1350$    &   $\mu_{max}=1681$    & no crossing   \\\bottomrule[0.6pt]
\end{tabular}\label{tab:pot_max_est-ns}
\end{table}
\end{center}
\end{widetext}





