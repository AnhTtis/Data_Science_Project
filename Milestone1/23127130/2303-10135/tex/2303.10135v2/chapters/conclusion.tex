\section{Conclusion}
In this work, we addressed the RASP problem, which is aimed at deriving an order of operations, according to which a target product can
be assembled step-by-step on a robotic system. 
Given a geometric description of a final product, we predict if a target robot system is able to assemble it.
More concretely, we propose a graph representation, called Assembly Graphs for the aluminum profile assemblies, which is flexible to represent different 2d structures and meanwhile agnostic to rotation and mirroring.
Based on this, a novel Graph Assembly Processing Network (GRACE) is introduced to extract meaningful information for assembly sequence prediction within the IL framework.
We conducted extensive experiments to evaluate the capability of our model to transfer knowledge between different assembly tasks, as previous methods lacked
this capacity. 
As the results show, our method can generalize knowledge gained on larger assemblies and then apply it to smaller ones. 
We believe this is the case since the relevant constraints guiding the assembly of the smaller structures are contained in larger
assemblies.

We propose two directions for future research. 
Instead of considering 2d assemblies laying on a flat surface in our setting, we suggest to generalize our graph representation to support 3d objects as well.
Secondly, our method predicts feasible assembly sequences without considering their optimality.
As this problem attracted considerable research in the past~\cite{jimenez2013survey}, we suggest to incorporate it into our method.
