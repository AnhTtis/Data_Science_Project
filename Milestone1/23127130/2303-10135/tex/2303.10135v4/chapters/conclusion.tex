\section{Conclusion}
In this work, we addressed the RASP problem with a learning-based framework. %, which is aimed at deriving an order of operations, according to which a target product can be assembled step-by-step on a robotic system. 
%Given a geometric description of a final product, we predict if a target robot system is able to assemble it.
Concretely, we propose a graph representation, called Assembly Graphs for the aluminum profile assemblies, which is flexible to represent different 2d structures and meanwhile agnostic to rotation and mirroring.
Based on this, a novel policy network -- GRACE is introduced to extract meaningful information for assembly sequence prediction. %within the IL framework.
Extensive experiments in simulation verify the capability of transferring knowledge between different assembly tasks, on which previous methods fall short. 
Further, our method can generalize knowledge gained on larger assemblies and then apply it to smaller ones. 
Last but not least, it is worth to mention, though only validated in simulation, our method should address the challenges during the real-world deployment like not finding a valid motion or a feasible grasping point if these cases are enclosed in the training data and learned to reject by GRACE.
% We believe this is the case since the relevant constraints guiding the assembly of the smaller structures are contained in larger assemblies.
% We propose two directions for future research. 
% Instead of considering 2d assemblies, laying on a flat surface and comprised of simple profiles, 
Meanwhile encouraged by the superior results on objects with simple geometries, our holistic graphical method lays a solid basis for handling complex 3d objects like curve blocks in the future. % by considering the cues of more complex spatial structures.
% Secondly, our method predicts feasible assembly sequences without considering their optimality.
% As this problem attracted considerable research in the past~\cite{jimenez2013survey}, we suggest to incorporate it into our method.
% All of these challenges are enclosed in the previous experience of the training data. Problems like not finding a valid motion or a feasible grasping point affects the solutions which can be fully assembled by the roboic system.