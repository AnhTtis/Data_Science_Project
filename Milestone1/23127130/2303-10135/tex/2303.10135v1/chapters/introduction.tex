\section{Introduction}
Aiming for high flexibility, manufacturers around the globe are introducing automation for \gls{rasp} at a greater pace to respond to rapid changes in market needs for customization of novel product variants~\cite{Shih2020}. 
These changes cause often modifications in assembly lines, requiring time-consuming and resource-intensive re-planning,
because of the NP-hard combinatorial characteristic~\cite{Rashid2011} of \gls{asp}, where the number of possible solutions grows with the factorial of the amount of parts involved.
Also, to check whether a certain assembly sequence can actually be executed on a specific robotic system is computationally expensive.
For example in~\cite{suarez2018can}, 11 minutes were required for the assembly motion planning of an IKEA chair.
This is more time than the actual execution of the plan, not to mention the case of product variants whose assembly sequence space itself must be explored, easily leading to a search of hours or days instead of minutes.
\begin{figure}[t]
	\centering
	\includegraphics[width=1.\linewidth, height=8.6cm]{figures/teaser/GRACE_teaser.pdf}
	\caption{\textbf{Workflow of our proposed graphical ASP method} on a dual-armed robotic system (used in our setting): an aluminum assembly specification is first represented as an Assembly Graph and then fed into our policy network GRACE, designed to flexibly and efficiently generate assembly sequences in a \textit{step-by-step} manner that can be executed by the robot. Best viewed in color.}
	\label{fig:teaser}
\end{figure}

Several existing works attempt to improve the tedious \gls{asp} process by predicting feasible assembly sequences~\cite{zhao2019aspw,watanabe2020search} or inferring the underlying rules guiding their creation~\cite{rodriguez2019iteratively, rodriguez2020pattern}.
Although those works already facilitate the assembly planning, they still lack desirable attributes such as generalization across varying product types and sizes as well as run-time efficiency.
In this work, we address this problem with a graphical learning-based approach, that is able to automatically generate sequences for \textit{unknown} assembly variants in an \emph{efficient} way.

In a nutshell of our main idea, inspired by~\cite{lin2022efficient}, we formulate \gls{rasp} as a sequential decision-making problem with a \gls{mdp}, in order to break the restriction of combinatorial complexity wrt. the number of parts and thus, boost generalization performance.
Hence the sequence is generated step-by-step based on the current assembly state.
Meanwhile we exploit the idea of distilling previous knowledge acquired for assembling products to predict the next feasible actions with a designed policy architecture~\cite{rodriguez2020pattern}, that is optimized to imitate the demonstration sequences collected in simulation which are interpreted as expert demonstrations~\cite{lin2022efficient}. 

Specifically, to put the aforementioned ideas into practice, we propose to use a graphical representation to faithfully describe the structure of assemblies.
Our so-called Assembly Graph is adapted from and more fine-grained than the one in~\cite{rodriguez2020pattern} by representing the assembly as a heterogeneous graph whose edges denote geometrical relations between the assembly part surfaces.
Based on this, we further develop a policy architecture based on \gls{gnn}, called \textbf{GR}aph \textbf{A}ssembly pro\textbf{C}essing n\textbf{E}tworks, for short GRACE, to extract useful information from the Assembly Graph and predict actions determining which parts should be assembled next.
Apart from this, false predicted sequences and infeasible assemblies pose a severe problem for efficiency of the assembly robots, \eg an incorrect sequence might require the robot to perform time-consuming re-planning.
Therefore, it is highly desirable and beneficial to detect these beforehand, hence we further develop and analyze various schemes to enhance the performance of feasibility prediction. 

It is worthwhile to note that there are several advantages for the proposed graphical representation and the policy architecture, GRACE:
(1) Invariance to number of parts: contrary to previous works such as~\cite{rodriguez2020pattern} restricted by a fixed number of parts, ours is free from this limitation, as %\gls{gnn}s% 
GRACE is capable of handling varying number of input graph nodes.
(2) Memory efficient learning: GRACE %\gls{gnn}s 
employs shared weights across all nodes in the graph, further alleviating the burden of the aforementioned complexity.
(3) Generalization: GRACE trained on assemblies of one size is able to generalize to those of smaller sizes (see results in Tab.~\ref{exp:tab_knowledge_transfer_inter_sized_one2many}). 
(4) Multiple solutions: GRACE predicts several feasible sequences (in contrast to~\cite{lin2022efficient}), allowing greater flexibility and resilience during execution.

We validate the proposed method with comprehensive experiments based on a dataset of assemblies made of different numbers of aluminum parts created in simulation of a dual-armed robotic system.
This setting can be mapped to various tasks in the industry~\cite{rodriguez2020pattern} as it allows for construction of numerous product variations.
Moreover, as shown in Fig.~\ref{fig:asp_exp_aluminum_assemblies}, it requires a deeper understanding of several complex relations (\eg distances between parts, physical part characteristics). 
The results show that our approach is able to efficiently predict feasible assembly sequences across product variants (with few millisecond to predict the next step (\ref{implementation})).

To summarize, our contribution is three-fold:
\begin{itemize}
	\item We introduce Assembly Graph, a heterogeneous graphical representation for the \gls{rasp} task, which is a more fine-grained and flexible representation than our previous one in~\cite{rodriguez2020pattern} by including part surfaces and parts in the same graph. 
	\item We develop a policy architecture GRACE to process the Assembly Graph and predict feasible assembly sequences in a step-by-step manner as well as the feasibility for a given assembly specification.
	\item We conduct comprehensive experiments to validate the proposed approach including failure analysis and ablation studies on design choices.
\end{itemize}

\begin{table*}[t]
	\centering
	\resizebox{\textwidth}{!}{
	\begin{tabular}{l c c c c }
	\toprule
	 & \shortstack{Efficient generalization \\across assembly sizes?}& \shortstack{Robotic constraints \\ involved?}  & \shortstack{Direct sequences \\generation?} & \shortstack{Fine-grained \\graph representation?} \\ 	
	 \midrule
	ASPW-DRL \cite{zhao2019aspw}	& \xmark & \xmark & \cmark & \xmark \\ 
	LEGO-GRAPH \cite{ma2022planning}	& \cmark & \xmark & \cmark & \xmark \\		
	KT-RASP \cite{rodriguez2020pattern}	& \xmark & \cmark & \xmark & \xmark \\ 	
	The proposed method	& \cmark & \cmark & \cmark & \cmark \\ 	
	\bottomrule
	\end{tabular}
	}
	\caption{Comparison between our proposed method and other relevant works.}\label{tab:comparison}
\end{table*}