\documentclass[%
aps,
prl,
reprint,
amsmath,
amssymb,
amsfonts,
superscriptaddress,
]{revtex4-2}
\usepackage{graphicx}
\usepackage{xcolor}
\usepackage{bm}
\usepackage[%
hidelinks,
colorlinks=true,
citecolor=blue,
urlcolor=blue,
linkcolor=blue
]{hyperref}
\usepackage[inline]{enumitem}

\definecolor{red}{HTML}{C74431}
\newcommand\red[1]{{\color{red}#1}}
\definecolor{green}{HTML}{3C9455}
\newcommand\green[1]{{\color{green}#1}}

%%%%%%%%%%%%%%%%%%%%%%%%%%%%%%%%%%%%%%%%%%%%%%%%%%%%%%%%%%%%%%%%%%%%%%%%%%%%%%%%%%%%%%%%%%%%%%%%%%%%
\begin{document}
%%%%%%%%%%%%%%%%%%%%%%%%%%%%%%%%%%%%%%%%%%%%%%%%%%%%%%%%%%%%%%%%%%%%%%%%%%%%%%%%%%%%%%%%%%%%%%%%%%%%
\author{Roland Wiese}
\email{wiese@itp.uni-leipzig.de}
\author{Klaus Kroy}
\affiliation{Institute for Theoretical Physics, Leipzig University, 04103 Leipzig, Germany}
\author{Demian Levis}
\email{levis@ub.edu}
\affiliation{Departement de F{\'i}sica de la Materia Condensada, Facultat de F{\'i}sica, Universitat de Barcelona, Mart{\'i} i Franqu{\`e}s 1, 08028 Barcelona, Spain}
\affiliation{University of Barcelona Institute of Complex Systems (UBICS), Facultat de F{\'i}sica,  Universitat de Barcelona, Mart{\'i} i Franqu{\`e}s 1, 08028 Barcelona, Spain}
\date{\today}
% \title{Shear Rheology of Dense Dry Active Matter: from linear to glassy response}
% \title{From linear to glass rheology in Dense Dry Active Matter}
\title{Fluid-Glass-Jamming Rheology of Soft Active Brownian Particles}
\begin{abstract}
 We study the shear rheology of a binary mixture of soft Active Brownian Particles, from the fluid to the disordered solid regime. At low shear rates, we find a Newtonian regime, where a Green-Kubo relation with an effective temperature provides the linear viscosity. It is followed by a shear-thinning regime at larger shear rates. A solid regime at high densities is signalled by the emergence of a finite yield stress.  %From the yield stress, 
We construct a ``fluid-glass-jamming'' phase diagram, in which the temperature axis is replaced by activity. %We find that activity suppresses the glass sector of the yield stress curves, and 
 Although activity, like temperature, increases the amount of fluctuations in the system, it plays a different role, as it changes the exponent characterizing the decay of the diffusivity close to the glass transition and the shape of the yield stress surface. The dense disordered active solid  appears to be mostly dominated by athermal jamming rather than glass rheology.
\end{abstract}
% \keywords{active matter; linear response; rheology; yielding}
%%%%%%%%%%%%%%%%%%%%%%%%%%%%%%%%%%%%%%%%%%%%%%%%%%%%%%%%%%%%%%%%%%%%%%%%%%%%%%%%%%%%%%%%%%%%%%%%%%%%

\maketitle
%%%%%%%%%%%%%%%%%%%%%%%%%%%%%%%%%%%%%%%%%%%%%%%%%%%%%%%%%%%%%%%%%%%%%%%%%%%%%%%%%%%%%%%%%%%%%%%%%%%%

%GlassJamming
%Upon compression, a collection 
Ensembles of repulsive particles %, e.g. hard colloids,  
commonly undergo a phase transition  from a fluid to a solid state. If the tendency to crystallize is frustrated (e.g. by size polydispersity), the system exhibits a glass transition to a disordered solid \cite{pusey1986phase-behaviour-of-concentrated-suspensions, mewis2011colloidal-suspension-rheology}. 
Solidity can also emerge upon compression in athermal  systems of non-Brownian particles, such as foams or grains: The so-called jamming transition \cite{nagel2010jamming-transition-and-marginally-jammed-solid}. Both transitions share the existence of a critical density beyond which solidity emerges, characterized by a dramatic slowing down of the dynamics and the emergence of a yield stress \cite{berthier2011theoretical, bonn2017yield-stress-materials}: Thus a unified picture in terms of a ``jamming'' phase diagram was proposed \cite{liu1998jamming-is-not-just-cool-any-more, trappe2001jamming-phase-diagram-for-attractive-particles}. The  yield stress surface  has since  been quantified with the help of idealized  particle models and rheological experiments \cite{ciamarra2010recent-results-on-the-jamming-phase-diagram, liu2011universal-jamming-phase-diagram-in-the-hard-sphere-limit, ikeda2012unified, ikeda2013disentangling-glass-and-jamming}, helping to decipher the mechanisms responsible for the emergence of rigidity in diverse soft materials \cite{bonn2017yield-stress-materials}. 

% is controlled by jamming or glass physics (usually a mix of both). 
%Understanding the emergence of solidity is leaving 
A resurgence of interest in understanding the emergence of solidity is recently observed in an \emph{a priori} completely new context, namely dense disordered active matter (DDAM).  Indeed, cell assemblies display a fluid to solid transition, key to  understand biological processes such as morphogenesis. Again, this phenomenology has been rationalized in terms of a jamming phase diagram  \cite{mongera2018fluid-to-solid-jamming-transition,manning2021jamming-and-arrest-of-cell-motion-in-biological-tissues, lenne2022sculpting-tissues-by-phase-transitions, campas2022nuclear-jamming-transition-in-vertebrate-organogenesis}, where temperature is replaced by activity, usually in the form of motility; see Fig. \ref{fig:phasediagram}(a), where $\mathrm{Pe}$ quantifies activity. 
Dense assemblies of cells \cite{angelini2011glass-like-dynamics-of-collective-cell-migration, manning2013glassy-dynamics-in-3d-embryonic-tissues} and synthetic active colloids \cite{klongvessa2019active-glass-ergodicity-breaking} feature collective dynamics reminiscent of supercooled liquids approaching a glass transition. However, as thermal fluctuations can usually be neglected in active systems, and fluctuations are  non-thermal, it is conceptually unclear whether the emergence of a disordered solid has to be attributed to a jamming rather than a glass transition. 
The question of how activity, in the form of self-propulsion, affects the glass transition, has been addressed in numerous  works using model  systems  \cite{berthier2013non-eq-glass-transitions-in-driven-and-active-matter, ni2013pushing-the-glass-transition, berthier2014nonequilibrium-glassy, levis2015single-particle-to-collective-effective-temperatures, berthier2017how-active-forces-influence-noneq-glass-transitions, manning2016motility-driven-glass-and-jamming-transitions-in-biological-tissues, sussman2018anomalous-glassy-dynamics-dense-biological-tissue,  janssen2019active-glasses, manning2019glassy-dynamics-in-models-of-confluent-tissue,  henkes2020dense,sadhukhan2021glassy-dynamics-in-cellular-potts-model, paoluzzi2022mips-to-glassiness-in-dense-active-matter, berthier2022disordered-collective-motion-in-dense-assemblies-of-persistent-particles}. However, the rheology of active matter \cite{henkes2017cell-division-and-death-inhibit-glassy-behaviour, amiri2022yielding-transition-in-epithelial-tissues} and its relation with jamming, remains poorly understood. 

%rheology of active fluids 


\begin{figure}[t]
  \centering
  \includegraphics[width=\linewidth]{fig_phase_diagram_shear_profile.pdf}
  \caption{(a)``Glass-Jamming" phase diagram of harmonic ABPs at reduced temperature $T=10^{-4}$ in terms of the yield stress surface.  The black dotted line corresponds to the athermal ($T=0$) passive jamming limit. The black symbols in the P{\'e}clet number versus volume fraction plane locate the active glass transition. (b) Cross section of the $xy$-plane of the three dimensional simulation box, showing the imposed velocity profile. % $\dot{\gamma}y$. 
  } \label{fig:phasediagram}
\end{figure}



%and grains \cite{} have also been studied from these perspective, sharing again similar collective dynamics with passive colloidal and granular matter at large densities. 

In this Letter we investigate the shear rheology of self-propelled soft  particles by means of computer simulations. In previous studies  with micro-swimmer suspensions, both hydrodynamic  and particle-wall interactions are likely to be crucial \cite{rafai2010effective, gachelin2013non, lopez2015turning, saintillan2018rheology, martinez2020combined-rheometry-and-imaging-study-of-viscosity-reduction-in-bacterial-suspensions}. Here, in order to decipher the role played by self-propulsion alone,  we consider a simplified model where none of these two ingredients are at play \cite{shaebani2020computational}. More precisely, we consider a dry model system of  harmonic Active Brownian Particles (ABP) in three dimensions \cite{wysocki2014cooperative-motion, turci2021phase-separation-and-multibody-effects-in-3D-ABP}  with periodic boundary conditions (see Fig. \ref{fig:phasediagram}(b)). 
This simplified starting point allows us to cover the dilute and dense regimes  in our numerical simulations, and to explore both  linear and  non-linear response (across eight orders of magnitude in the shear rate). 
The advantages of choosing harmonic spheres are threefold: \begin{enumerate*}[label=\roman*)]\item Harmonic spheres have been largely studied as models of foams \cite{durian1995foam-mechanics-at-the-bubble-scale}, and its active version  provides a useful  model of cell tissues \cite{matoz2017nonlinear-rheology-in-a-model-biological-tissue, thirumalai2018cell-growth-rate-dictates-the-onset-of-glass}; \item the rheology of the equilibrium limit of the model has been studied in detail \cite{ikeda2012unified, ikeda2013disentangling-glass-and-jamming}, allowing for a smooth connection with previous results and thus helping for the identification of the new features brought about by activity; \item computational speed-up compared to hard spheres. \end{enumerate*}

The rheology of model active particles has received recent interest, both for individual ABPs \cite{loewen2011self-propelled-shear-flow}, and for interacting many-body systems  \cite{asheichyk2019response, reichert2021transport-coefficients-MCT, mandal2021shear-induced-orientational-ordering,bayram2023motility}. An analogy has been drawn between shear (a global drive) and activity (a local drive) \cite{mo2020rheological-similarities-between-dense-self-propelled-and-sheared-particulate-systems}, and indeed, in infinite spatial dimensions in the infinite-persistence limit, their formal equivalence has been established \cite{morse2021link-active-matter-sheared-granular, agoritsas2021mean-field-dynamics-of-infinite-dimensional-particle-systems}. In two dimensions, the mechanisms that govern yielding in the respective systems were found to be different, though \cite{villarroel2021critical-yielding-rheology}. 
Active particles under shear orientationally order in the presence of hard walls \cite{mandal2021shear-induced-orientational-ordering} and trigger shear thickening as a result of clustering %at large activities 
 \cite{bayram2023motility}. 
However,  the rheology of dense disordered ABPs with a finite persistence has not yet been explored.

 %model definition and implementation
 We study sheared assemblies of $N$ soft repulsive ABPs, located in positions $\{\bm r_i\}_{i=1}^N$ in a $V=L^3$ cubic box with PBC. Their  dynamics is overdamped and follows
\begin{align}
  \begin{split}
    \dot{\bm r}_i &= \mu\sum_{j\neq i}\bm F_{ij} + v_0\bm n_i + \dot{\gamma}y_i\bm e_x + \sqrt{2D_t}\bm \xi_i\,,\\
    \dot{\bm n}_i &= \sqrt{2D_r}\bm n_i\times\bm \nu_i\,.\\
  \end{split}
  \label{eq:eom}
\end{align}
Particles are self-propelled along their orientations $\bm n_i$ (with $|\bm n_i|=1$), with a speed $v_0$. Interaction forces derive from a harmonic repulsive pair potential $V(r)~=~\epsilon~(1-r/a)^2~\Theta(a-r)$, $\Theta(r)$ being the Heaviside step function. 
We consider a 50:50 bidisperse mixture of $N=10^3$  particles with diameter ratio $a_B/a_A=\sqrt{2}$ to suppress crystallization \cite{ikeda2012unified, olsson2007critiical-scaling-of-shear-viscosity-at-the-jamming-transition}.
Both $\bm \xi_i$ and $\bm \nu_i$ are Gaussian white noises of zero mean and unit variance,  $D_t=\mu k_BT$ is the (bare) translational diffusion coefficient and $D_r$  the rotational diffusivity fixed at $D_r=3D_t/a_A^2$. With this choice, $T\to 0$ corresponds to  infinite persistence. To further explore the athermal case, $T=0$, we fix $D_t=0$ and vary $D_r$ independently. 
 The term $\dot{\gamma}y_i\bm e_x$, together with Lees-Edwards boundary conditions imposes a linear velocity profile to the particles with slope $\dot{\gamma}$, the shear rate (see Fig. \ref{fig:phasediagram}(b))  \cite{lees1972computer, allen-tildesley2017computer-simulation-of-liquids}. 
The translational part of Eq.\eqref{eq:eom} is integrated by an Euler-Mayurama scheme and the rotational part using the algorithm described in \cite{raible2004langevin-equation-rotation} %, which provides a convenient way of  integrating a Langevin equation on the unit sphere 
(see \cite{SM} for details). %Relaxation to steady-conditions are guranteed

Lengths are measured in units of the small particle diameter $a_A$, time in units of  $\hat{t}=a_A^2/(\mu\epsilon)$ and temperature in units of $\hat{T}=\epsilon/k_B$. In the following, all observables will be given in these units.
From Eq.\eqref{eq:eom}, one can identify a set of non-dimensional control parameters: 
 the volume fraction $\phi$, the P{\'e}clet number $\mathrm{Pe}=v_0/a_A D_r$, quantifying activity, and the dimensionaless shear rate $\dot{\gamma}$. %$\hat{\dot{\gamma}}=\dot{\gamma}\hat{t}^{-1}$ . 
  We  study the system at $T=10^{-6}...10^{3}$ (most of the results presented are for $T=10^{-4}$), $\mathrm{Pe}=0,1, 3,10,30$, and a wide range of shear rates $\dot{\gamma}=10^{-7}...\, 10$  including both the linear and nonlinear response regimes.
Keeping the activity at the aforementioned values and $D_r$ constant, ensures that the repulsive force is always several orders of magnitude larger than the self-propulsion, avoiding any possibility of a reentrant gas phase, as seen in \cite{fily2014freezing-and-phase-separation}.
It is known that monodisperse hard ABPs above a critical $\mathrm{Pe}\approx30$ exhibit Motility-Induced Phase Separation (MIPS) \cite{wysocki2014cooperative-motion, turci2021phase-separation-and-multibody-effects-in-3D-ABP, omar2021phase-diagram-of-active-brownian-spheres}. Here we explore a parameter regime where our system remains homogeneous. 

%Phase behavior (STATICS, no shear)
In equilibrium ($\mathrm{Pe}=0$), as $\phi$ is increased, the system exhibits a dramatic slowing down of the dynamics that one identifies with a glass transition at $\phi_G$, characterized by the divergence of the viscosity and the emergence of a yield stress for $\phi>\phi_G(0)$. As we show below, similar behavior is observed in the presence of activity, although in this case,  the location of the glass transition is shifted to higher densities $\phi_G(\mathrm{Pe})$ (as previously reported for different models  \cite{berthier2013non, ni2013pushing-the-glass-transition, marchetti2011active-jamming, berthier2014nonequilibrium-glassy, levis2015single-particle-to-collective-effective-temperatures, berthier2017how-active-forces-influence-noneq-glass-transitions, paoluzzi2022mips-to-glassiness-in-dense-active-matter}). 

\begin{figure}[t]
  \centering
  \includegraphics[width=\linewidth]{fig_diff-coeff.pdf}
  \caption{\label{fig:diff-coeff}
    (a) Mean-square displacement at fixed $T=10^{-4}$,  $\mathrm{Pe}=0$ and $10$, for $\phi=0.66,\,0.72$, with and without shear, showing the melting of the  glass by activity (green curves) and by shear (red curves).
    The dashed line indicates the initial diffusive regime.% with $\Delta^2(t)=6D_tt$. 
   (b) Diffusion coefficients $D_s$ normalized by their ideal gas value $D_a$ (open green circles correspond to $T=10^{-3}$ and $\mathrm{Pe}=0$),  as a function of the distance to the critical density $\phi_G - \phi$.
    Dashed lines are power law fits $D_s\propto (\phi_G - \phi)^\alpha$. %, from which the critical density $\phi_G$ of the glass transition and exponent $\alpha$ are estimated.
  }
\end{figure}
We start our analysis by investigating the dynamics in the absence of shear by means of the Mean-Square Displacement (MSD), defined as  $\Delta^2(t)=N^{-1}\sum_{i=1}^N\langle (\bm r_i(t) - \bm r_i(0))^2\rangle$, where the average is taken over different noise realisations. 
As shown in Fig. \ref{fig:diff-coeff}(a), at $\phi=0.66$ the passive system exhibits caged dynamics, evidenced by the sub-diffusive (plateau) regime in the MSD. For $\mathrm{Pe}=10$, particles diffuse in the same time window, showing that activity is able to fluidize the glass. 
From the long time {MSD} we measure the diffusion coefficient $D_s \equiv \lim_{t\to\infty}\Delta^2(t) / 6t$ in the range of parameters for which a diffusive regime, $\Delta^2(t)\propto t$, is observed. At high densities, $D_s(\phi)$ can be fitted by a power law  $D_s\propto (\phi_G - \phi)^\alpha$ that we use to locate the glass transition density reported in Fig. \ref{fig:phasediagram}(a) \cite{berthier2014nonequilibrium-glassy}.
Figure \ref{fig:diff-coeff}(b) shows $D_s(\phi, \mathrm{Pe})$ as a function of $(\phi_G - \phi)$ normalized by the (active) ideal gas diffusion coefficient $D_a = D_t + \mathrm{Pe}^2D_r/6$, for different $\mathrm{Pe}$. 
As the activity primarily enhances diffusion, it moves the glass transition  from $\phi_G(0)=0.62$ at $\mathrm{Pe}=0$ to ever higher densites, up to $\phi_G(30)=0.72$ at $\mathrm{Pe}=30$ (see Fig. \ref{fig:phasediagram}(a)). It might be tempting to simply interpret such a shift as resulting from an increase of the single particle effective temperature, defined by $T_{\text{eff}}=\mu D_a$, since the equilibrium glass transition density would also be shifted upon increasing $T$ \cite{berthier2009glass}. 
However, not only $\phi_G$ is affected by activity, but also the exponent $\alpha$ increases from $1.79$ at $\mathrm{Pe}=0$, % , $1.97$ at $\mathrm{Pe}=3$, $2.70$ at $\mathrm{Pe}=10$ 
up to $4.42$ at $\mathrm{Pe}=30$. While in equilibrium,  a slightly lower value ($\alpha=1.67$) is measured for higher temperature  $T=10^{-3}$, $\alpha$ significantly changes with $\mathrm{Pe}$, showing that activity cannot be simply reduced to an effective temperature in this regime. 

Applying shear provides another route to fluidize the disordered solid state. As shown in Fig. \ref{fig:diff-coeff}(a), particles exhibiting caged dynamics in an active system at $\phi=0.72>\phi_G(10)$, become mobile when shear is turned on. The MSD displays super-diffusive, then diffusive behavior at long times.  The obtained long time diffusion is sensitive to finite-size effects: Lees-Edwards boundary conditions introduce a discontinuity in the shearing profile which becomes apparent in the MSD after a sufficiently long time (see \cite{SM} for details).

To characterize the flow properties %at different Pe (and $T$) 
we measure the $xy$-component of the stress tensor, using the Irving-Kirkwood expression \cite{irving1950statistical-mechanical-hydrodynamics} 
$
\sigma_{xy}(t) = -(2V)^{-1}\sum_{j\neq k}x_{ij}(t)F^y_{ij}(t)\,,
$
from which we get  the shear viscosity $\eta = \langle\sigma_{xy}\rangle/\dot{\gamma}$.
%The kinetic contribution to the IK tensor is omitted for this overdamped model.
Activity contributes to the stress tensor with a self-term $\sigma^s_{\alpha\beta}=-V^{-1}\sum_ir^\alpha_iv_0n^\beta_i(t)$,  %\cite{takatori2014swim-pressure, solon2015pressure-phase-equilibria},
%, winkler2015virial, falasco2016mesoscopic}, 
 but as the orientations ${\bm n}_i$ are decoupled from the shear flow, $\sigma^s_{xy}$ fluctuates around zero  \cite{mandal2021shear-induced-orientational-ordering}.
The flow curves characterizing the rheology of the system at $T=10^{-4}$ and $\mathrm{Pe}=0,10$ are depicted in Fig. \ref{fig:flow-curves}   (see \cite{SM} for more parameter values). 
In the passive case, $\mathrm{Pe}=0$, we reproduce the flow curves reported for the same system in \cite{ikeda2012unified}. Then, we explore the rheology in the presence of activity. 
In  dense two-dimensional active assemblies, shearing was observed to lead to orientational order at large persistence time \cite{mandal2021shear-induced-orientational-ordering}.
However, we did not find orientational correlations in our three-dimensional model for the parameter range explored (see \cite{SM}).


\begin{figure}[h]
  \centering
  \includegraphics[width=\linewidth]{fig_flow_curves.pdf}
  \caption{\label{fig:flow-curves}
    Flow curves showing the dependence of the shear stress $\langle\sigma_{xy}\rangle$ (top row) and viscosity $\eta$ (bottom row) on  $\dot{\gamma}$ (points, color code  encoding  $\phi$). The thick red lines correspond to $\phi_G$ (as estimated via the diffusivity). Thinner lines represent fits of the form $\sigma_{xy}(\dot{\gamma}) = \sigma_Y + (k\dot{\gamma})^n$, used to determine the yield stress $\sigma_Y$.
  }
\end{figure}


%The rheology of this system was studied for the $\mathrm{Pe}=0$ case and varying temperatures in 
At densities below $\phi_G(\mathrm{Pe})$, we find $\sigma_{xy}\propto \dot{\gamma}$ for small enough applied shear. This corresponds to the Newtonian fluid regime, defining a linear viscosity $\eta_0 = \lim_{\dot{\gamma}\to 0}\langle\sigma_{xy}\rangle/\dot{\gamma}$. For higher values,  $\dot{\gamma} \gtrsim 10^{-2}$, in the non-linear regime, we find shear-thinning in all cases, meaning that the shear flow reduces the  system's viscosity.  In this regime, we find a decay of the viscosity compatible with the  $\eta\sim 1/|\dot{\gamma}|$ scaling expected from Mode-Coupling Theory for  Brownian suspensions \cite{fuchs2002non-newtonian-viscosity-of-interacting-brownian-particles}.
%We find that it equally applies for all considered P{\'e}clet values in our study, as the flow curves all reduce to the same behaviour in this limit (see \cite{SM}).
At $\phi>\phi_G$, a finite yield stress $\sigma_Y=\lim_{\dot{\gamma}\to 0} \sigma_{xy}(\dot{\gamma})$ appears, identified by a plateau in the stress flow curves. This results in a divergent viscosity, signalling the emergence of solidity. 
Since activity melts the solid, the system yields at higher densities at $\mathrm{Pe}=10$ compared to $\mathrm{Pe}=0$.

\begin{figure}[t]
  \centering
  \includegraphics[width=\linewidth]{fig_GK_viscosity.pdf}
  \caption{\label{fig:GK-viscosity}
  Linear shear viscosities $\eta_0$, extracted from the flow curves (symbols) and $\eta_{GK}/(\mathrm{Pe}+1)$, from Green-Kubo, (lines) as a function of $\phi$. 
    Dashed vertical lines indicate $\phi_G(\mathrm{Pe})$.
    Inset: $\eta_{GK}\sim\phi^2$ at low densities .
  }
\end{figure}
  
%The liquid
In the fluid regime, one can access the linear  viscosity by applying small enough shear and, in equilibrium, it should correspond to the one given by the Green-Kubo (GK) relation
$
  \eta_{GK} = \frac{V}{k_BT}\int_0^\infty\mathrm dt\,\langle\sigma_{\alpha\beta}(t)\sigma_{\alpha\beta}(0)\rangle_0\,,
$
for $\alpha\neq\beta$, where $\langle*\rangle_0$ denotes an average over the unperturbed ($\dot{\gamma}=0$) equilibrium distribution. As shown in  Fig. \ref{fig:GK-viscosity}, the shear viscosity $\eta_0$ extracted from the low $\dot{\gamma}$ plateau in the flow curves in Fig. \ref{fig:flow-curves}(c) matches $\eta_{GK}$,  measured from the GK relation by direct integration of the equilibrium stress correlation function. In the presence of activity,  GK relations do not need to hold anymore, although  extensions of linear response theory to active systems have recently been proposed  \cite{cengio2019linear-response, solon2019response, cengio2021fluctuation-dissipation, dadhichi2022time}, providing GK relations involving  steady-state correlation functions \cite{cengio2019linear-response, cengio2021fluctuation-dissipation}. Here we apply the same procedure in the presence of activity, thus replacing  equilibrium by  steady-state stress correlations, and find a  good agreement between $\eta_0$ and $\eta_{GK}$ if we replace $T$ in the GK expression by an effective temperature $T_{\mathrm{eff}}=T(\mathrm{Pe}+1)$, see Fig. \ref{fig:GK-viscosity}. In all cases, $\eta_0$ increases with $\phi$ and eventually diverges at $\phi_G$, providing yet another estimate for the onset of solidity. % for $\phi \gtrsim \phi_G$, with $ \phi_G$ pushed to higher values as Pe increases.  
 %
In dilute conditions, we find that the GK viscosity  grows like $\sim\phi^2$ (inset of Fig. \ref{fig:GK-viscosity}), as predicted in  \cite{batchelor1972determination-of-bulk-stress-in-a-suspension-of-spherical-particles}  for dilute Brownian suspensions: $\eta/\eta_0 = 1 + 2.5\phi + 7.6\phi^2$. %This extends  Einstein's expression $\eta/\eta_0 = 1 + 2.5\phi$ \cite{einstein1911berichtigung-neue-bestimmung-der-molekueldimensionen} by including pair collisions. 
We only observe the $\phi^2$ contributions, as there is no solvent in our system. 
%the solvent, responsible for the other terms, is absent here. 

Above $\phi_G$ the viscosity diverges and the system acquires a yield stress $\sigma_Y$ that we measure  by fitting a Herschel-Bulkley law, $\sigma_{xy}(\dot{\gamma}) = \sigma_Y + (k\dot{\gamma})^n$ \cite{herschel-bulkley1926konsistenzmessungen-von-gummi-benzolloesungen}, to the flow curves, see Fig. \ref{fig:flow-curves}. We report  the obtained  yield stress  $\sigma_Y(\phi)$ as a function of volume fraction for different values of $\mathrm{Pe}$ and $T$ in  Fig. \ref{fig:yield-stress}.
%
At $\mathrm{Pe}=0$ and finite $T$,  the emergence of solidity is controlled by the glass transition, at a density $\phi_G(T)$ that increases with $T$. At $T=0$, it is instead controlled by the jamming transition, at $\phi_J\approx 0.648$. 
Both  glass and jamming physics  affect the behavior of $\sigma_Y (\phi,T)$. 
%By decreasing $T$, $\sigma_Y$ is lowered and a wider branch of glassy dynamics is observed  before $\sigma_Y$ follows the power of jammed systems
At $T\leqslant 10^{-4}$, for which $\phi_G<\phi_J$, $\sigma_Y$ increases gently with $\phi$ and $T$ up to $\phi\approx\phi_J$ (given by the dotted line in Fig. \ref{fig:yield-stress}). Above this value,  the behavior of $\sigma_Y$ changes qualitatively: it grows faster with $\phi$ close to $\phi_J$, with little $T$-dependence, following $\sigma_Y\sim(\phi-\phi_J)^{\alpha'}$ \cite{ikeda2012unified}, see  Fig. \ref{fig:yield-stress}(a). 
Such crossover allows us to differentiate  glass- and jamming-dominated regimes.  
%

\begin{figure}[t]
  \centering
  \includegraphics[width=\linewidth]{fig_yield_stress.pdf}
  \caption{\label{fig:yield-stress}
   Yield stress $\sigma_Y$ as a function of  $\phi$ for different $\mathrm{Pe}$ and $T$.
    The dashed line  given by $(\phi - \phi_J)^{\alpha'}$ with $\phi_J= 0.648$ and $\alpha'= 1.04$, marks the athermal jamming limit.
  }
\end{figure}


%At finite Pe and $T$,  $\sigma_Y$ decreases with Pe at a given $\phi>\phi_G$. 
At $\mathrm{Pe}=1$ and finite $T$, $\sigma_Y$ displays a $T$-sensitive glass-like branch for $\phi<\phi_J$ followed by the $T$-insensitive jamming branch, see  Fig. \ref{fig:yield-stress}(a). 
 The yield stress curves do not follow the trend one would expect if activity could be subsumed into an extra source of noise and encoded by an increased effective temperature. The yield stress is smaller for $\mathrm{Pe}=1$ than for $\mathrm{Pe}=0$, at a given $\phi$ and $T$, up to $\phi_J$. A higher $T$ would result in a larger $\sigma_Y$. Moreover, at $T=0$, we  recover again the athermal jamming behavior, despite the presence of  random (active) forces. 
As we further increase the activity at fixed $T=10^{-4}$, $\sigma_Y$ quickly collapses onto the jamming branch. 
%Glassy behaviour is strongly suppressed by activity:  
For $\mathrm{Pe}>3$,  $\phi_G(\mathrm{Pe})>\phi_J$, and  a finite yield stress can only emerge in the regime controlled by  jamming \cite{olsson2007critiical-scaling-of-shear-viscosity-at-the-jamming-transition}, where $\sigma_Y\sim(\phi-\phi_J)^{\alpha'}$ universally applies, independently of $\mathrm{Pe}$. The crossover between glass and jamming rheology can thus be tuned by activity and is eventually lost, as  it pushes $\sigma_Y$ towards the $T=0$ behavior.  
The separation between $\phi_G$ and $\phi_J$ progressively vanishes as activity increases. 
An overview over the impact of activity on the yield stress at $T=10^{-4}$ is represented in the fluid-glass-jamming phase diagram in Fig. \ref{fig:phasediagram},  in terms of the shear stress, density and activity. 

We have studied  ABPs under shear, from its fluid to disordered solid regime. 
%Let us first discuss the fluid. 
In the fluid, taking the zero shear limit, the Green-Kubo viscosities are compatible with the ones extracted from the flow curves  in the Newtonian regime, once $T$ is rescaled by $\mathrm{Pe}$. Such effective temperatures $T_{\text{eff}}$ have been introduced earlier to quantify the violations of the fluctuation-dissipation theorem in active systems. In the dilute limit $T_{\text{eff}}\sim \mathrm{Pe}^2$. Its $\mathrm{Pe}$ dependence at finite densities is weaker and  more complex and generically dependent on the observables used to define it \cite{levis2015single-particle-to-collective-effective-temperatures, cengio2019linear-response, petrelli2020effective}. 
%In the non-linear regime, ABPs exhibit shear-thinning, just as passive Brownian particles.  
As the packing fraction is increased towards the fluid-solid transition,  the diffusivity decays $D_s\sim(\phi_G-\phi)^{\alpha}$, with  $\alpha$ increasing from $\alpha\approx 1.8$ for $\mathrm{Pe}=0$, to $\alpha\approx 4.4$ for $\mathrm{Pe}=30$, a behavior hardly interpretable on the grounds of an effective temperature. 
In the solid regime, the  glass-jamming phase diagram (Fig. \ref{fig:phasediagram}) reveals that ABPs rheology is mainly controlled by jamming. Although both $T$ and $\mathrm{Pe}$ push $\phi_G$ to higher values, activity, as opposed to temperature, eases the yielding. %, 
%$T$ increases the yield stress, while Pe reduces it, 
%suppressing the glass sector. 
%This work provides the first quantitative 
Our quantitative results for the relationship between the fluid-solid transition in dense disordered active matter and the glass and jamming transitions %, showing that its rheology in a way that cannot be simply reduced to a higher $T_{\text{eff}}$.
%assessment of the results about the fluid-solid transition in active systems in terms of jamming vs. glassy physics. 
%Such results provide the first quantitative glass-jamming phase diagram of an active system, showing that activity affects the rheology of DDAM in a way that cannot be simply understood by means of an effective temperature. %: the Pe-axis of the phase diagram cannot be readily mapped into the usual $T$-axis. 
%The non-equilibrium fluctuations introduced by activity play a fundamentally different role. 
should serve as a helpful reference for future studies. A possible application could be in interpreting observations in dense assemblies of cells. We also hope that they stimulate theoretical work to better understand the fundamental role played by   non-equilibrium fluctuations introduced by activity, in dense disordered systems. 

\paragraph{Acknowledgments}
We warmly thank T. Voigtmann, A. Ikeda and L. Berthier for useful exchanges. 
D.L. acknowledges MCIU/ AEI/FEDER for financial support under Grant Agreement No. RTI2018-099032-J-I00. 

\bibliographystyle{apsrev4-2}
\bibliography{refs}

\end{document}
