\documentclass[amsmath,amssymb]{revtex4}
\input{epsf}
\renewcommand{\baselinestretch}{1.0}
\usepackage{gensymb}
\usepackage{graphicx}% Include figure files
\usepackage{float}
\usepackage{dcolumn}% Align table columns on decimal point
\usepackage{bm}% bold math
\usepackage{epstopdf}
\usepackage{array}
\usepackage{bigstrut}
\usepackage{longtable}
\usepackage{rotating,booktabs}
\usepackage{booktabs,threeparttable}
\usepackage{color}
\usepackage[colorlinks,linkcolor=blue, citecolor=green, urlcolor=blue, anchorcolor=blue]{hyperref}
\bibliographystyle{apsrev4-2}


\begin{document}

\preprint{APS/123-QED}

\title{Supplemental Material for \\
	``Measurement of hyperfine structure and the Zemach radius in $\rm^6Li^+$ using optical Ramsey technique"}





\date{\today}% It is always \today, today,
             %  but any date may be explicitly specified
\maketitle


\section{The Fano-Voigt fitting function for quantum interference}

 

Quantum interference between atomic transitions causes spectral line-pulling and possible systematic errors in high precision measurements
\cite{Low1952,Horbatsch2010,Udem2019}. A rigorous treatment requires solving the optical Bloch equations, which is computationally intensive for $\rm^6Li^+$ since there are 36 magnetic sublevels and thus 1296 differential equations to be solved.  In addition, the geometry is not very well defined. The purpose of this Supplementary Material is to provide additional information on a fitting procedure used in place of a rigorous analysis to reduce the uncertainty due to quantum interference and spectral line pulling.

The key point is that the the Ramsey fringe pattern consists of an ``envelope" multiplied by an oscillating sinusoid.  Although the line pulling (quantum interference) distorts the symmetry of the resonant line, its effect is primarily reflected in the envelope portion, not in the sinusoidal part, since the latter is a geometric factor. In the case of well-separated lines,  Udem \emph{et al.}~\cite{Udem2019} derived that the emission spectrum of an atom in a monochromatic light field exhibits a so-called Fano-Voigt line shape. Accordingly, in this work we use the Fano-Voigt function times a sinusoid as the fitting function, which takes the form
\begin{equation}\label{eq:1}
P(\omega_L) = A({\rm Re}[w(z)]+2\eta{\rm Im}[w(z)]){\rm cos}[2(\omega_L-\omega_0)T]+C,
\end{equation}
\noindent
where the ${\rm Re}[w(z)]+2\eta{\rm Im}[w(z)]$ part represents the Fano–Voigt line shape, and the ${\rm cos}[2(\omega_L-\omega_0)T]$ part is the oscillating term caused by Ramsey interference. The function $w(z)$ in Eq.~(\ref{eq:1}) denotes the Faddeeva function $w(z) = \exp(-z^2){\rm erfc}(-iz)$ of argument
\begin{equation}\label{eq:2}
z = 2\sqrt{{\rm ln2}}\,[(\omega_L-\omega_0)+i\Gamma/2]/\Gamma_G\,.
\end{equation}
Equations~(\ref{eq:1}) and~(\ref{eq:2}) involve 7 fitting parameters: the center frequency $\omega_0$, the amplitude \textit{A}, a constant background \textit{C}, the Lorentzian and Gaussian widths, $\Gamma$ and $\Gamma_G$, the asymmetry parameter $\eta$ and the free-evolution time \textit{T} of a $\rm^6Li^+$ ion between adjacent laser interactions. A nonlinear curve fitting is employed to determine the line center $\omega_0$. In our fitting procedure, the initial value of \textit{A} is chosen as the difference of the maximum and minimum value of the measured intensities divided by 2; the initial value of \textit{C} is chosen as the averaged value of all measured intensities at different laser frequencies; the initial value of $\Gamma$ is chosen as the natural linewidth (2$\pi$$\times$3.7 MHz); the initial value of $\Gamma_G$ is chosen as the full width at half maximum (FWHM) of the Lamb dip; the initial value of \textit{T} is chosen as the interval between adjacent laser fields divided by the velocity of the ions; and the initial value of $\eta$ is set to zero.

As shown in Fig.\ 4 of the main text, large oscillations with respect to the angle of laser polarization are reduced to near the noise level, when the above fitting procedure is used.  
%\bibliography{apssamp}
\providecommand{\noopsort}[1]{}\providecommand{\singleletter}[1]{#1}%
\begin{thebibliography}{3}%
\makeatletter
\providecommand \@ifxundefined [1]{%
 \@ifx{#1\undefined}
}%
\providecommand \@ifnum [1]{%
 \ifnum #1\expandafter \@firstoftwo
 \else \expandafter \@secondoftwo
 \fi
}%
\providecommand \@ifx [1]{%
 \ifx #1\expandafter \@firstoftwo
 \else \expandafter \@secondoftwo
 \fi
}%
\providecommand \natexlab [1]{#1}%
\providecommand \enquote  [1]{``#1''}%
\providecommand \bibnamefont  [1]{#1}%
\providecommand \bibfnamefont [1]{#1}%
\providecommand \citenamefont [1]{#1}%
\providecommand \href@noop [0]{\@secondoftwo}%
\providecommand \href [0]{\begingroup \@sanitize@url \@href}%
\providecommand \@href[1]{\@@startlink{#1}\@@href}%
\providecommand \@@href[1]{\endgroup#1\@@endlink}%
\providecommand \@sanitize@url [0]{\catcode `\\12\catcode `\$12\catcode
  `\&12\catcode `\#12\catcode `\^12\catcode `\_12\catcode `\%12\relax}%
\providecommand \@@startlink[1]{}%
\providecommand \@@endlink[0]{}%
\providecommand \url  [0]{\begingroup\@sanitize@url \@url }%
\providecommand \@url [1]{\endgroup\@href {#1}{\urlprefix }}%
\providecommand \urlprefix  [0]{URL }%
\providecommand \Eprint [0]{\href }%
\providecommand \doibase [0]{https://doi.org/}%
\providecommand \selectlanguage [0]{\@gobble}%
\providecommand \bibinfo  [0]{\@secondoftwo}%
\providecommand \bibfield  [0]{\@secondoftwo}%
\providecommand \translation [1]{[#1]}%
\providecommand \BibitemOpen [0]{}%
\providecommand \bibitemStop [0]{}%
\providecommand \bibitemNoStop [0]{.\EOS\space}%
\providecommand \EOS [0]{\spacefactor3000\relax}%
\providecommand \BibitemShut  [1]{\csname bibitem#1\endcsname}%
\let\auto@bib@innerbib\@empty
%</preamble>
\bibitem [{\citenamefont {Low}(1952)}]{Low1952}%
  \BibitemOpen
  \bibfield  {author} {\bibinfo {author} {\bibfnamefont {F.}~\bibnamefont
  {Low}},\ }\href {https://doi.org/10.1103/PhysRev.88.53} {\bibfield  {journal}
  {\bibinfo  {journal} {Phys. Rev.}\ }\textbf {\bibinfo {volume} {88}},\
  \bibinfo {pages} {53} (\bibinfo {year} {1952})}\BibitemShut {NoStop}%
\bibitem [{\citenamefont {Horbatsch}\ and\ \citenamefont
  {Hessels}(2010)}]{Horbatsch2010}%
  \BibitemOpen
  \bibfield  {author} {\bibinfo {author} {\bibfnamefont {M.}~\bibnamefont
  {Horbatsch}}\ and\ \bibinfo {author} {\bibfnamefont {E.~A.}\ \bibnamefont
  {Hessels}},\ }\href {https://doi.org/10.1103/PhysRevA.82.052519} {\bibfield
  {journal} {\bibinfo  {journal} {Phys. Rev. A}\ }\textbf {\bibinfo {volume}
  {82}},\ \bibinfo {pages} {052519} (\bibinfo {year} {2010})}\BibitemShut
  {NoStop}%
\bibitem [{\citenamefont {Udem}\ \emph {et~al.}(2019)\citenamefont {Udem},
  \citenamefont {Maisenbacher}, \citenamefont {Matveev}, \citenamefont
  {Andreev}, \citenamefont {Grinin}, \citenamefont {Beyer}, \citenamefont
  {Kolachevsky}, \citenamefont {Pohl}, \citenamefont {Yost},\ and\
  \citenamefont {Hänsch}}]{Udem2019}%
  \BibitemOpen
  \bibfield  {author} {\bibinfo {author} {\bibfnamefont {T.}~\bibnamefont
  {Udem}}, \bibinfo {author} {\bibfnamefont {L.}~\bibnamefont {Maisenbacher}},
  \bibinfo {author} {\bibfnamefont {A.}~\bibnamefont {Matveev}}, \bibinfo
  {author} {\bibfnamefont {V.}~\bibnamefont {Andreev}}, \bibinfo {author}
  {\bibfnamefont {A.}~\bibnamefont {Grinin}}, \bibinfo {author} {\bibfnamefont
  {A.}~\bibnamefont {Beyer}}, \bibinfo {author} {\bibfnamefont
  {N.}~\bibnamefont {Kolachevsky}}, \bibinfo {author} {\bibfnamefont
  {R.}~\bibnamefont {Pohl}}, \bibinfo {author} {\bibfnamefont {D.~C.}\
  \bibnamefont {Yost}},\ and\ \bibinfo {author} {\bibfnamefont {T.~W.}\
  \bibnamefont {Hänsch}},\ }\href {https://doi.org/10.1002/andp.201900044}
  {\bibfield  {journal} {\bibinfo  {journal} {Ann. Phys. (Berlin)}\ }\textbf
  {\bibinfo {volume} {531}},\ \bibinfo {pages} {1900044} (\bibinfo {year}
  {2019})}\BibitemShut {NoStop}%
\end{thebibliography}%

\end{document}
%
% ****** End of file apssamp.tex ******
