%%%%%%%%%%%%%%%%%%%%%%%%%%%%%%%%%%%%%%%%%%%%%%%%%%%%%%%%%%%%%%%
% Preamble version 08.09.22
%%%%%%%%%%%%%%%%%%%%%%%%%%%%%%%%%%%%%%%%%%%%%%%%%%%%%%%%%%%%%%%

\documentclass[11pt,a4paper,oneside]{amsart}
% \usepackage[a4paper]{geometry} % Changing page shape
% \geometry{left=2.5cm,right=2.5cm,top=3cm,bottom=3cm}

% Comments

\newcounter{commentcounter}
\newcommand{\commentSH}[1]{\stepcounter{commentcounter}\textbf{Comment \arabic{commentcounter} (by Sam)}: {\textcolor{blue}{#1}}}
\newcommand{\commentMK}[1]{\stepcounter{commentcounter}\textbf{Comment \arabic{commentcounter} (by Monika)}: {\textcolor{red}{#1}}}

%%%%%%%%%%%%%%%%%%%%%%%%%%%%%%%%%%%%%%%%%%%%%%%%%%%%%%%%%%%%%%%
% Packages
\usepackage{amsmath} % Lots of maths functionality
\usepackage{amssymb} % Maths symbols
\usepackage{amsthm} % Maths environments: \begin{proof}, etc.
\usepackage{stmaryrd} % [[ brackets
%\usepackage[toc,page]{appendix} % Nice formatting for appendices
\usepackage[english]{babel} % Language and hyphenation.
\usepackage[font=small,justification=centering]{caption} % More flexibility for captioning figures
%\usepackage{csquotes} % Quotation environment
\usepackage[nodayofweek]{datetime}
%\usepackage{empheq} % Allows grouping of equations with empheq environment
\usepackage[shortlabels]{enumitem} % Change enumeration labelling with \begin{enumerate}[a)] etc.
%\usepackage{float} % For placing graphics, allows \begin{figure}[H] etc.
\usepackage[T1]{fontenc} % For font encoding, to allow accents, copy & paste, inequality signs, etc. to all work nicely.
%\usepackage{graphicx} % Add images to the document
\usepackage[utf8]{inputenc} % To be loaded after fontenc, also for encoding.
\usepackage{ifthen} % For Dani's \begin{com} environment and \numberedtheorem
\usepackage{mathabx} % Contains \pnest symbol and more. Clashes with accents for \ring
\usepackage{mathtools} % Uses amsmath, fixes quirks and adds functionality.
\usepackage[dvipsnames]{xcolor} % Allow links to have colour. Needs to be before hyperref and tikz-cd
\usepackage[pdftex,  colorlinks=true]{hyperref} % Makes references and citations into links
    \hypersetup{urlcolor=RoyalBlue, linkcolor=RoyalBlue,  citecolor=black}
\usepackage{setspace} % Allows \onehalfspacing etc. for changing gaps between lines
% \onehalfspacing
\usepackage{tikz-cd} % Commutative diagrams
\usepackage{xfrac} % Nicer quotients 
\usepackage[capitalize]{cleveref} % use \Cref{} for instead of X~\ref{} 
%%%%%%%%%%%%%%%%%%%%%%%%%%%%%%%%%%%%%%%%%%%%%%%%%%%%%%%%%%%%%%%

% Subject class
\makeatletter
\@namedef{subjclassname@1991}{Mathematical subject classification 1991}
\@namedef{subjclassname@2000}{Mathematical subject classification 2000}
\@namedef{subjclassname@2010}{Mathematical subject classification 2010}
\@namedef{subjclassname@2020}{Mathematical subject classification 2020}
\makeatother


% Theorem Counters
\newtheorem{thm}{Theorem}[section]
\newtheorem{lemma}[thm]{Lemma}
\newtheorem{corollary}[thm]{Corollary}
\newtheorem{prop}[thm]{Proposition}
\newtheorem{conjecture}[thm]{Conjecture}
\newtheorem{claim}[thm]{Claim}
\newtheorem{addendum}[thm]{Addendum}
\newtheorem{assumption}[thm]{Assumption}
\newtheorem{question}[thm]{Question}



% "letter-numbered" theorems
\newtheorem{thmx}{Theorem}
\newtheorem{corx}[thmx]{Corollary}
\newtheorem{propx}[thmx]{Proposition}
\renewcommand{\thethmx}{\Alph{thmx}}
\renewcommand{\thecorx}{\Alph{corx}}


% Definition environment style 
\theoremstyle{definition}
\newtheorem{defn}[thm]{Definition}
\newtheorem{remark}[thm]{Remark}
\newtheorem{remarks}[thm]{Remarks}
\newtheorem{construction}[thm]{Construction}
\newtheorem{setup}[thm]{Setup}
\newtheorem{example}[thm]{Example}
\newtheorem{examples}[thm]{Examples}

\theoremstyle{plain}
\newtheorem*{ConjSinger}{The Singer Conjecture}


    \newtheoremstyle{TheoremNum}
        {\topsep}{\topsep} %%% space between body and thm
        {\itshape} %%% Thm body font
        {-0.25cm} %%% Indent amount (empty = no indent)
        {\bfseries} %%% Thm head font
        {.} %%% Punctuation after thm head
        { }  %%% Space after thm head
        {\thmname{#1}\thmnote{ \bfseries #3}}%%% Thm head spec
    \theoremstyle{TheoremNum}
    \newtheorem{duplicate}{}



\newcommand*{\claimproofname}{My proof}
\newenvironment{claimproof}[1][\claimproofname]{\begin{proof}[#1]\renewcommand*{\qedsymbol}{\(\blacksquare\)}}{\end{proof}}


%%%%%%%%%%%%%%%%%%%%%%%%%%%%%%%%%%%%%%%%%%%%%%%%%%%%%%%%%%%%%%%

% Large asterisks created by Ian Leary with some help from Boris Okun
\DeclareMathOperator*{\Aster}{\text{\LARGE{\textasteriskcentered}}}
\DeclareMathOperator{\aster}{\text{\LARGE{\textasteriskcentered}}}


% Sets
\DeclareMathOperator{\Aut}{\mathrm{Aut}}
\DeclareMathOperator{\Out}{\mathrm{Out}}
\DeclareMathOperator{\Inn}{\mathrm{Inn}}
\DeclareMathOperator{\Ker}{\mathrm{Ker}}
\DeclareMathOperator{\coker}{\mathrm{Coker}}
\DeclareMathOperator{\Coker}{\mathrm{Coker}}
\DeclareMathOperator{\Hom}{\mathrm{Hom}}
\DeclareMathOperator{\Ext}{\mathrm{Ext}}
\DeclareMathOperator{\Tor}{\mathrm{Tor}}
\DeclareMathOperator{\tr}{\mathrm{tr}}
\DeclareMathOperator{\im}{\mathrm{im}}
\DeclareMathOperator{\Fix}{\mathrm{Fix}}
\DeclareMathOperator{\orb}{\mathrm{Orb}}
\DeclareMathOperator{\End}{\mathrm{End}}
\DeclareMathOperator{\Irr}{\mathrm{Irr}}
\DeclareMathOperator{\Comm}{\mathrm{Comm}}
\DeclareMathOperator{\Isom}{\mathrm{Isom}}
\DeclareMathOperator{\Min}{\mathrm{Min}}
\DeclareMathOperator{\Core}{\mathrm{Core}}
\DeclareMathOperator{\bigset}{Big}
\DeclareMathOperator{\cay}{Cay}
\DeclareMathOperator{\diam}{diam}
\DeclareMathOperator{\hull}{hull}
\DeclareMathOperator{\link}{Lk}
\DeclareMathOperator{\map}{Map}
\DeclareMathOperator{\sym}{Sym}
\DeclareMathOperator{\lat}{\mathrm{Lat}}
\DeclareMathOperator{\dom}{\mathrm{dom}}
\DeclareMathOperator{\ord}{\mathrm{ord}}

% mathcal
\newcommand{\cala}{{\mathcal{A}}}
\newcommand{\calb}{{\mathcal{B}}}
\newcommand{\calc}{{\mathcal{C}}}
\newcommand{\cald}{{\mathcal{D}}}
\newcommand{\cale}{{\mathcal{E}}}
\newcommand{\calf}{{\mathcal{F}}}
\newcommand{\calg}{{\mathcal{G}}}
\newcommand{\calh}{{\mathcal{H}}}
\newcommand{\cali}{{\mathcal{I}}}
\newcommand{\calj}{{\mathcal{J}}}
\newcommand{\calk}{{\mathcal{K}}}
\newcommand{\call}{{\mathcal{L}}}
\newcommand{\calm}{{\mathcal{M}}}
\newcommand{\caln}{{\mathcal{N}}}
\newcommand{\calo}{{\mathcal{O}}}
\newcommand{\calp}{{\mathcal{P}}}
\newcommand{\calq}{{\mathcal{Q}}}
\newcommand{\calr}{{\mathcal{R}}}
\newcommand{\cals}{{\mathcal{S}}}
\newcommand{\calt}{{\mathcal{T}}}
\newcommand{\calu}{{\mathcal{U}}}
\newcommand{\calv}{{\mathcal{V}}}
\newcommand{\calw}{{\mathcal{W}}}
\newcommand{\calx}{{\mathcal{X}}}
\newcommand{\caly}{{\mathcal{Y}}}
\newcommand{\calz}{{\mathcal{Z}}}

% mathfrak
\newcommand*{\frakg}{\mathfrak{G}}
\newcommand*{\fraks}{\mathfrak{S}}
\newcommand*{\frakt}{\mathfrak{T}}

% mathbb
\newcommand*{\bbG}{\mathbb{G}}

% underline
\newcommand{\ulE}{\underline{E}}
\newcommand{\ulEG}{\underline{E}\Gamma}
\newcommand{\Efin}{E_{\mathcal{FIN}})}
\newcommand{\EfinG}{E_{\mathcal{FIN}}\Gamma}

% Categories
\newcommand{\grp}{\mathbf{Grp}}
\newcommand{\set}{\mathbf{Set}}
\newcommand{\gtop}{\mathbf{Top}_\Gamma}
\newcommand{\orbf}{\mathbf{Or}_\calf(\Gamma)}
\newcommand{\bfE}{\mathbf{E}}
\newcommand{\Top}{\mathbf{Top}}

% Spectra
\newcommand{\spectra}{\mathbf{Spectra}}
\newcommand{\ko}{\mathbf{ko}}
\newcommand{\KO}{\mathbf{KO}}

% Families
\newcommand{\TRV}{\mathcal{TRV}}
\newcommand{\FIN}{\mathcal{FIN}}
\newcommand{\VC}{\mathcal{VC}}
\newcommand{\ALL}{\mathcal{ALL}}

% Functors
\newcommand{\hocolim}{{\rm hocolim}}
\newcommand{\colim}{{\rm colim}}
\newcommand{\ind}{{\rm ind}}
\newcommand{\res}{{\rm res}}
\newcommand{\coind}{{\rm coind}}

% Groups
\newcommand{\GL}{\mathrm{GL}}
\newcommand{\PGL}{\mathrm{PGL}}
\newcommand{\PGammaL}{\mathrm{P\Gamma L}}
\newcommand{\SL}{\mathrm{SL}}
\newcommand{\PSL}{\mathrm{PSL}}
\newcommand{\GU}{\mathrm{GU}}
\newcommand{\PGU}{\mathrm{PGU}}
\newcommand{\UU}{\mathrm{U}}
\newcommand{\SU}{\mathrm{SU}}
\newcommand{\PSU}{\mathrm{PSU}}
\newcommand{\Sp}{\mathrm{Sp}}
\newcommand{\PSp}{\mathrm{PSp}}
\newcommand{\OO}{\mathrm{O}}
\newcommand{\SO}{\mathrm{SO}}
\newcommand{\PSO}{\mathrm{PSO}}
\newcommand{\PGO}{\mathrm{PGO}}
\newcommand{\AGL}{\mathrm{AGL}}
\newcommand{\ASL}{\mathrm{ASL}}
\newcommand{\He}{\mathrm{He}}

\newcommand{\PSLp}{\mathrm{PSL}_2(\ZZ[\frac{1}{p}])}
\newcommand{\SLp}{\mathrm{SL}_2(\ZZ[\frac{1}{p}])}
\newcommand{\SLm}{\mathrm{SL}_2(\ZZ[\frac{1}{m}])}
\newcommand{\GLp}{\mathrm{GL}_2(\ZZ[\frac{1}{p}])}

\newcommand{\LM}{\mathrm{LM}}
\newcommand{\HLM}{\mathrm{HLM}}
\newcommand{\CAT}{\mathrm{CAT}}

% HHG relations
\newcommand*{\lhalf}[1]{\overleftarrow{#1}}
\newcommand*{\rhalf}[1]{\overrightarrow{#1}}
\newcommand*{\sgen}[1]{\langle#1\rangle}
\newcommand*{\nest}{\sqsubseteq}
\newcommand*{\pnest}{\sqsubset}
\newcommand*{\conest}{\sqsupset}
\newcommand*{\pconest}{\sqsupsetneq}
\newcommand*{\trans}{\pitchfork}

%Misc
\DeclareMathOperator{\Ad}{\mathrm{Ad}}
\DeclareMathOperator{\id}{id}
\newcommand{\onto}{\twoheadrightarrow}
\def\iff{if and only if }
\newcommand{\TAP}{\mathsf{TAP}}

% Symmetric spaces
\newcommand{\EE}{\mathbb{E}}
\newcommand{\KH}{\mathbb{K}\mathbf{H}} % Hyperbolic space over K
\newcommand{\RH}{\mathbb{R}\mathbf{H}} % Real hyperbolic space
\newcommand{\CH}{\mathbb{C}\mathbf{H}} % Complex hyperbolic space
\newcommand{\HH}{\mathbb{H}\mathbf{H}} % Quaternion hyperbolic space
\newcommand{\OH}{\mathbb{O}\mathbf{H}^2} % Cayley hyperbolic space
\newcommand{\Ffour}{\mathrm{F}_4^{-20}} % The other rank one group

% Projective spaces
\newcommand{\KP}{\mathbb{K}\mathbf{P}} % Projective plane over K
\newcommand{\RP}{\mathbb{R}\mathbf{P}} % Real projective plane
\newcommand{\CP}{\mathbb{C}\mathbf{P}} % Complex projective plane
\newcommand{\OP}{\mathbb{O}\mathbf{P}^2} % Cayley projective plane


% Invariants
\newcommand{\Covol}{\mathrm{Covol}}
\newcommand{\Vol}{\mathrm{Vol}}
\newcommand{\rank}{\mathrm{rank}}
\newcommand{\gd}{\mathrm{gd}}
\newcommand{\cd}{\mathrm{cd}}
\newcommand{\vcd}{\mathrm{vcd}}
\newcommand{\hd}{\mathrm{hd}}
\newcommand{\vhd}{\mathrm{vhd}}
\newcommand{\betti}{b^{(2)}}
\newcommand{\MC}{\mathrm{MC}}
\DeclareMathOperator{\lcm}{\mathrm{lcm}}
\DeclareMathOperator{\Char}{\mathrm{Char}}

% Spaces
\newcommand{\flag}{{\rm Flag}}
\newcommand{\wtX}{\widetilde{X}}
\newcommand{\wtXj}{\widetilde{X_J}}
\newcommand{\ulG}{\underline{\Gamma}}

% Rings
\newcommand{\MM}{\mathbf{M}}
\newcommand{\repr}{\calr_\RR}
\newcommand{\repc}{\calr_\CC}
\newcommand{\reph}{\calr_\HH}
%\newcommand{\gg}{\mathfrak{g}}
%\newcommand{\hh}{\mathfrak{h}}
%\newcommand{\gl}{\mathfrak{gl}}
%\renewcommand{\sl}{\mathfrak{sl}}
%\newcommand{\so}{\mathfrak{so}}
%\newcommand{\nov}[3]{{\mathrm{Nov}({#1 #2, #3}})}
\newcommand{\nov}[3]{{\widehat{#1 #2}^{#3}}}
\newcommand{\cgr}[2]{#1 \llbracket #2 \rrbracket} %completed group ring
\def\Z{\mathbb{Z}}
\newcommand{\cgrZ}{\widehat{\Z}\llbracket t^{\widehat{\Z}}\rrbracket}


% Fields
\newcommand{\NN}{\mathbb{N}}
\newcommand{\ZZ}{\mathbb{Z}}
\newcommand{\CC}{\mathbb{C}}
\newcommand{\RR}{\mathbb{R}}
\newcommand{\QQ}{\mathbb{Q}}
\newcommand{\FF}{\mathbb{F}}
\newcommand{\KK}{\mathbb{K}}

%ODEs and PDEs
\newcommand{\ode}{\mathrm{d}}
\newcommand*{\pde}[3][]{\ensuremath{\frac{\partial^{#1} #2}{\partial #3}}}


%tikz
\usepackage{tikz}
\usetikzlibrary{arrows,quotes}
\tikzstyle{blackNode}=[fill=black, draw=black, shape=circle]


\title[On profinite rigidity amongst free-by-cyclic groups]{On profinite rigidity amongst free-by-cyclic groups I: the generic case}

\usepackage[foot]{amsaddr}
\author{Sam Hughes}
\email{sam.hughes@maths.ox.ac.uk}
\author{Monika Kudlinska}
\email{kudlinska@maths.ox.ac.uk}
\address{Mathematical Institute, Andrew Wiles Building, Observatory Quarter, University of Oxford, Oxford OX2 6GG, UK}

\date{\today}
\subjclass[2020]{20E36; 20E18; 20E26 (primary) 20J05; 20J06; 57M07; 20F67; 20F65 (secondary)}

% Primary:
% 20E18 Limits; profinite groups
% 20E26 Residual  properties  and  generalizations;  residually finite groups
% 20E36 Automorphisms of infinite groups

% Secondary:
% 20J05 Homological methods in group theory
% 20J06 Cohomology of groups
% 57M07 Topological methods in group theory
% 20F65 Geometric group theory
% 20F67 Hyperbolic groups and nonpositively curved groups

%%%%%%%%%%%%%%%%%%%%%%%%%%%%%%%%%%%%%%%%%%%%%%%%%%%%%%%%%%%%%%%
\begin{document}
\maketitle

\begin{abstract}

We prove that amongst the class of free-by-cyclic groups, Gromov hyperbolicity is an invariant of the profinite completion. We show that whenever $G$ is a free-by-cyclic group with first Betti number equal to one, and $H$ is a free-by-cyclic group which is profinitely isomorphic to $G$, the ranks of the fibres and the characteristic polynomials associated to the monodromies of $G$ and $H$ are equal. We further show that for hyperbolic free-by-cyclic groups with first Betti number equal to one, the stretch factors of the associated monodromy and its inverse is an invariant of the profinite completion. We deduce that irreducible free-by-cyclic groups with first Betti number equal to one are almost profinitely rigid amongst irreducible free-by-cyclic groups.  We use this to prove that generic free-by-cyclic groups are almost profinitely rigid amongst free-by-cyclic groups. We also show a similar results for \{universal Coxeter\}-by-cyclic groups.
\end{abstract}


\section{Introduction}
Two finitely generated groups $G$ and $H$ are said to be \emph{profinitely isomorphic} if they share the same isomorphism types of finite quotient groups.  It is a classical result that if two groups are profinitely isomorphic then they have the same profinite completion \cite{DixonFormanekPolandRibes1982}.  For a class $\calc$ of finitely generated residually finite groups, a group $G\in\calc$ is \emph{profinitely rigid in $\calc$} if any group $H$ in $\calc$ profinitely isomorphic to $G$ is in fact isomorphic to $G$.  Similarly, we say $G$ is \emph{almost profinitely rigid in $\calc$} if there are at most finitely isomorphism types of groups $H$ profinitely isomorphic to $G$.

There has been a large volume of work investigating profinite rigidity of $3$-manifold groups.  For example, deep work of Bridson--McReynolds--Reid--Spitler shows that there are hyperbolic $3$-manifolds which are profinitely rigid amongst all finitely generated residually finite groups \cite{BridsonMcReynoldsReidSpitler2020} with more examples constructed in \cite{CheethamWest2022}.  On the other hand, the existence of Anosov torus bundles and periodic closed surface bundles with non-isomorphic but profinitely isomorphic fundamental groups constructed in \cite{Stebe1972,Funar2013,Hempel2014} shows that profinite almost rigidity does not hold for all 3-manifold groups.  

Significant progress has been made on the problem of profinite rigidity \emph{within} the class $\mathcal{C}$ of 3-manifolds. For example, Wilkes has shown Seifert fibre spaces and graph manifolds are almost profinitely rigid in $\mathcal{C}$ \cite{Wilkes2017,Wilkes2018b,Wilkes2019}. Wilton--Zalesskii have shown that the JSJ decomposition and the geometries of the pieces are profinite invariants \cite{WiltonZalesskii2010,WiltonZalesskii2017,WiltonZalesskii2017b,WiltonZalesskii2019}.  Jaikin-Zapirain showed that being fibred is a profinite invariant of a $3$-manifold group \cite{Jaikin2020} (generalised to LERF groups in \cite{HughesKielak2022}).  Using Jaikin-Zapirain's result, Liu proved the spectacular theorem that finite volume hyperbolic $3$-manifold groups are almost profinitely rigid \cite{Liu2023}.  Other results have also been obtained, e.g. \cite{BridsonReidWilton2017,Wilkes2018,BoileauFriedl2020,Zalesskii2022,Liu2023b}.

A group $G$ is said to be \emph{free-by-cyclic} if it contains a normal subgroup $N \trianglelefteq G$ which is isomorphic to a free group of finite rank $F_n$, and such that $G/ N \cong \Z$. We will almost always think of a free-by-cyclic group as a pair $(G, \varphi)$, where $\varphi \in \mathrm{Hom}(G; \Z)$ and there's a short exact sequence
\[1 \to F_n \to G \xrightarrow{\varphi}  \Z \to 1.\]
Since any such short exact sequence splits, one can realise a free-by-cyclic group as the semi-direct product $G \cong F_n \rtimes_{\Phi} \Z$, for some $\Phi \in \mathrm{Out}(F_n)$ which we refer to as the \emph{monodromy} of the splitting. Conversely, given a semi-direct splitting $G \cong F_n \rtimes_{\Phi} \Z$ there's an associated character $\varphi \colon G \to \Z$ which maps the normal free factor to zero, and the stable letter (with respect to any choice of representative of $\Phi$) to the generator 1 of $\Z$. We call this the \emph{induced character} of the splitting $F_n \rtimes_{\Phi} \Z$.

Free-by-cyclic groups form a well-studied class which has been shown to exhibit many key properties; these include residual finiteness \cite{Baumslag1971}, quadratic isoperimetric inequality \cite{BridsonGroves2010}, and the property of being large \cite{Button2013}. Further, it is known that hyperbolic free-by-cyclic groups are cubulable \cite{HagenWise2015} and thus virtually compact special in the sense of Haglund--Wise \cite{HaglundWise2008}, and more generally that all free-by-cyclic groups which do not virtually split as a direct product admit non-elementary acylindrical actions on hyperbolic spaces \cite{GenevoisHorbez2021}. Despite this, there are still many open questions in this area, most notably on the subject of rigidity, even when one considers only rigidity amongst the class of free-by-cyclic groups.

Our goal in writing this paper is to investigate profinite rigidity amongst free-by-cyclic groups. Although we draw inspiration from the results in the $3$-manifold setting, the problem for free-by-cyclic groups is significantly subtler. This stems in part from the lack of a sufficient $\mathrm{Out}(F_n)$-analogue of the Nielsen--Thurston decomposition for homeomorphisms of finite-type surfaces. One artefact of this is that we frequently have to restrict our attention to the class of \emph{irreducible} free-by-cyclic groups, that is free-by-cyclic groups which admit irreducible monodromy. Recall that an outer automorphism $\Phi \in \mathrm{Out}(F_n)$ is \emph{irreducible} if there does not exist a free splitting \mbox{$F_n=A_1 \ast \ldots \ast A_k \ast B$,} where $A_1 \ast \ldots \ast A_k$ is non-trivial, and such that $\Phi$ permutes the conjugacy classes of the factors $A_i$. By the work of Mutanguha \cite{Mutanguha2021}, for any two realisations of $G$ as a free-by-cyclic group, $G \cong F_n \rtimes_{\Phi} \Z \cong F_m \rtimes_{\Psi} \Z$, the monodromy $\Phi$ is irreducible if and only if $\Psi$ is. 

Our first result is analogous to Liu's theorem with the additional hypotheses that $b_1(G)=1$ and restricting to the class of irreducible free-by-cyclic groups.  The first hypothesis is due to the fact that we do not have a method to establish $\widehat{\Z}$-regularity (see \Cref{sec:regularity} for a definition) without an analogous result to the main theorems in \cite{FriedlVidussi2008,FriedlVidussi2011annals} --- this is one of the main technical steps in Jaikin-Zapirain's and Liu's results.  The second hypothesis arises since, although we can show that hyperbolicity of free-by-cyclic groups is a profinite invariant, we are currently unable to show the same holds true for irreducibility.

\begin{thmx}\label{thmx:Irr}
    Let $G$ be an irreducible free-by-cyclic group.  If $b_1(G)=1$, then $G$ is almost profinitely rigid amongst irreducible free-by-cyclic groups.
\end{thmx}

\subsection{Profinite invariants}
The next theorem is somewhat more technical.  We will not include definitions of the invariants in the introduction, but many of them will be familiar to experts and they are scattered throughout the paper.  Note that the result actually holds in the more general setting of a $\widehat{\Z}$-regular isomorphism (the specific results stated throughout the paper comprising \Cref{thmx:invariants} are stated in this generality). 

% Let $\mathcal{C}$ denote the class of free-by-cyclic groups. 

\begin{thmx}\label{thmx:invariants}
    Let $G=F\rtimes_{\Phi} \Z$ be a free-by-cyclic group with induced character $\varphi \colon G \to \Z$. If $b_1(G)=1$, then the following properties are determined by the profinite completion $\widehat{G}$ of $G$: \begin{enumerate}
        \item the rank of $F$;
        \item the homological stretch factors $\{\nu^+_G,\nu^-_G\}$;
        \item the characteristic polynomials $\{\Char{\Phi^+},\Char{\Phi^-}\}$ of the action of $\Phi$ on $H_1(F;\QQ)$;
        \item for each representation $\rho\colon G\to \GL(n, \QQ)$ factoring through a finite quotient, the twisted Alexander polynomials $\{\Delta^{\varphi,\rho}_n,\Delta^{-\varphi,\rho}_n\}$ and the twisted Reidemeister torsions $\{\tau^{\varphi,\rho},\tau^{-\varphi,\rho}\}$ over $\QQ$.
    \end{enumerate}
    Moreover, if $G$ is conjugacy separable, (e.g. if $G$ is hyperbolic), then $\widehat G$ also determines the Nielsen numbers and the homotopical stretch factors $\{\lambda^+_G,\lambda^-_G\}$.
\end{thmx}

We point out the general fact that the first Betti number of any discrete group is an invariant of its profinite completion.

The reason for obtaining a set of invariants corresponding to $\Phi$ and $\Phi^{-1}$ is that the dynamics of $\Phi$ and $\Phi^{-1}$ can be different.  Indeed, this is somewhat a feature of free-by-cyclic groups rather than a bug.  A large technical hurdle in this work was overcoming this phenomena which cannot occur for $3$-manifolds.

We also obtain a complete geometric picture \'a la Wilton--Zalesskii in the case of hyperbolic free-by-cyclic groups.

\begin{thmx}\label{thmx:hyperbolicity}
     Let $G_A$ and $G_B$ be profinitely isomorphic free-by-cyclic groups. Then $G_A$ is Gromov hyperbolic if and only if $G_B$ is Gromov hyperbolic.
\end{thmx}


\subsection{Almost profinite rigidity and applications}
We will now explain how to apply \Cref{thmx:Irr}, \Cref{thmx:invariants}, and \Cref{thmx:hyperbolicity} to various classes of free-by-cyclic groups to obtain strong profinite rigidity phenomena.  


\subsubsection{Super irreducible free-by-cyclic groups}
We say that a free-by-cyclic group $G$ is \emph{super irreducible}, if $G \cong F_n \rtimes_{\Phi} \Z$ and the matrix $M\colon H_1(F_n;\QQ)\to H_1(F_n;\QQ)$ representing the action of $\Phi$ on $H_1(F_n;\QQ)$ satisfies the property that no positive power of $M$ maps a proper subspace of $H_1(F_n;\QQ)$ into itself.  Note that this implies $G$ is irreducible by \cite[Theorem~2.5]{GerstenStallings1991} and that $b_1(G)=1$. 

An example of a super irreducible free-by-cyclic group is  whenever the characteristic polynomial of $M$ is a \emph{Pisot--Vijayaraghavan polynomial}, namely, it is monic, it has exactly one root (counted with multiplicity) with absolute value strictly greater than one, and all other roots have absolute value strictly less than one \cite{GerstenStallings1991}.

\begin{corx}\label{corx:PV}
Let $G$ be a super irreducible free-by-cyclic group.  Then, every free-by-cyclic group profinitely isomorphic to $G$ is super irreducible.  In particular, $G$ is almost profinitely rigid amongst free-by-cyclic groups.
\end{corx}
\begin{proof}
    Let $H$ be a free-by-cyclic group and suppose $\widehat{H}\cong\widehat{G}$. As explained in \cite[Section~2]{GerstenStallings1991} $G$ being super irreducible is a property of the characteristic polynomial of the matrix $M\colon H_1(F_n;\QQ)\to H_1(F_n;\QQ)$ representing the action of $\Phi$ on $H_1(F_n;\QQ)$.  Thus, by \Cref{thmx:invariants} we see $H$ is super irreducible.  The result follows from \Cref{thmx:Irr}.
\end{proof}


\subsubsection{Random free-by-cyclic groups}

Fix $n \geq 2$ and let $S$ be a finite generating set of $\mathrm{Out}(F_n)$. For any $l \geq 1$, define $\mathcal{G}_l$ to be the set of all free-by-cyclic groups 
$G$ which admit a splitting $G \cong F_n \rtimes_{\Phi} \Z$, where $\Phi$ can be expressed as a word of length at most $l$ in $S$. We say that for a random free-by-cyclic group the property $P$ holds \emph{asymptotically almost surely}, or that a \emph{generic free-by-cyclic group satisfies property $P$}, if 
\[ \frac{\#\{G \in \mathcal{G}_l \mid G \text{ satisfies property }P\} }{\#\mathcal{G}_l} \to 1 \text{ as }l\to \infty.\]

We now state the result alluded to in the title of the paper.

\begin{corx}\label{corx:generic}
    Let $G$ be a random free-by-cyclic group.  Then, asymptotically almost surely $G$ is almost profinitely rigid amongst free-by-cyclic groups.
\end{corx}
\begin{proof}
    By \Cref{generic_properties}, every generic free-by-cyclic group $G$ is super-irreducible and has $b_1(G)=1$.  The result follows from  \Cref{corx:PV}.
\end{proof}


\subsubsection{Low rank fibres}
When the fibre of the free-by-cyclic group has rank two or three we are able to obtain rigidity statements within the class of all free-by-cyclic groups.

\begin{corx}\label{corx:F3}
Let $G=F_3\rtimes\Z$.  If $G$ is hyperbolic and $b_1(G)=1$, then $G$ is almost profinitely rigid amongst free-by-cyclic groups.
\end{corx}
\begin{proof}
   We first prove $G$ is irreducible.  Suppose that this is not the case. Then $G$ has a subgroup isomorphic to either $\Z\rtimes \Z$ or $F_2\rtimes \Z$. But both possibilities would imply $G$ contains a $\Z^2$ subgroup contradicting hyperbolicity.  Now let $H$ be a free-by-cyclic group and suppose that $\widehat{H}\cong\widehat{G}$.  By \Cref{thmx:hyperbolicity} we see $H$ is hyperbolic and by \Cref{thmx:invariants} we see that $H$ splits as $F_3\rtimes\Z$.  Thus, the previous paragraph implies $H$ is irreducible.  The result follows from \Cref{thmx:Irr}.
\end{proof}

Note in the next statement we see that $G$ is uniquely determined.

\begin{corx}\label{corx:F2}
     Let $G=F_2\rtimes\Z$.  If $b_1(G)=1$, then $G$ is profinitely rigid amongst free-by-cyclic groups.
\end{corx}
\begin{proof}
    Let $H$ be a free-by-cyclic group and suppose $\widehat{H}\cong\widehat{G}$.  By \Cref{thmx:invariants} we see that $H\cong F_2\rtimes \Z$.   But each $F_2\rtimes \Z$ is profinitely rigid amongst groups of the form $F_2\rtimes \Z$ by \cite{BridsonReidWilton2017}.
\end{proof}

\subsubsection{Procongruent conjugacy}
Our next result investigates conjugacy in $\Out(\widehat{F}_n)$ and is somewhat analogous to \cite[Theorem~1.2]{Liu2023b}.  We say two outer automorphisms $\Psi$ and $\Phi$ of $F_n$ are \emph{procongruently conjugate} if they induce a conjugate pair of outer automorphisms in $\Out(\widehat{F}_n)$.  In this setting we have no assumption on the action of $\Psi$ or $\Phi$ on the homology of $F_n$.

\begin{thmx}\label{thmx:procongruence}
Let $\Psi\in\Out(F_n)$ be atoroidal.  If $\Phi\in\Out(F_n)$ is procongruently conjugate to $\Psi$, then $\Phi$ is atoroidal and $\{\lambda_\Psi,\lambda_{\Psi^{-1}}\}=\{\lambda_\Phi,\lambda_{\Phi^{-1}}\}$.  In particular, if $\Psi$ is additionally irreducible, then there are only finitely many $\Out(F_n)$-conjugacy classes of irreducible automorphisms which are conjugate with $\Psi$ in $\Out(\widehat{F}_n)$.
\end{thmx}

\subsubsection{Automorphisms of universal Coxeter groups} 
Finally, we extend our results to the setting of universal Coxeter groups. A group $G$ is \emph{\{universal Coxeter\}-by-cyclic}, or \emph{\{universal Coxeter\}-by-cyclic} for short, if it splits as a semi-direct product $W_n \rtimes \Z$ where $W_n = \bigast_{i=1}^n \Z /2$ is the free product of $n$ copies of $\Z / 2$. A \emph{free basis} of $W_n$ is a generating set for $W_n$ such that each element has order 2.

After fixing a free basis for $W_n$, the kernel of the homomorphism $W_n \to \Z/2$ which maps every free generator of $W_n$ to 1, is a characteristic index-two subgroup $K$ of $W_n$ isomorphic to the free group of rank $n-1$. We say an outer automorphism of a free group is \emph{induced by an element of $\mathrm{Out}(W_n)$}, if it is the restriction $\Phi|_K \colon K \to K$, for some $\Phi \in \mathrm{Out}(W_n)$ and $K = \mathrm{ker}(W_n \to \Z/2)$ as above.

\begin{thmx}\label{thmx:CoxeterStretchInvariance}
    Let $G = W \rtimes \Z$ be a \{universal Coxeter\}-by-cyclic group. Then the rank of the fibre $W$ is an invariant of $\widehat{G}$. 

    Suppose that all free-by-cyclic groups with monodromy induced by an element of $\Out(W_n)$ are conjugacy separable. Then $\widehat{G}$ determines the the stretch factors $\{
    \lambda^{+}, \lambda^{-}\}$ associated to the monodromy of the splitting $W \rtimes \Z$.
\end{thmx}

\begin{thmx}\label{thmx:IrredCoxeterAlmostProfinite}
    Let $G$ be an irreducible \{universal Coxeter\}-by-cyclic group. Then $G$ is almost profinitely rigid amongst irreducible \{universal Coxeter\}-by-cyclic groups.  
\end{thmx}

\subsection{Some unanswered questions}
While we began in earnest to transport the programme of profinite rigidity amongst $3$-manifold groups to free-by-cyclic groups we have perhaps raised as many questions as answers.  We will highlight some key questions that we have encountered and hope to answer in the future.
Perhaps the most pressing issue is that of $\widehat{\Z}$-regularity. 

\begin{question}\label{q.Zregular}
    Is every profinite isomorphism of free-by-cyclic groups $\widehat{\Z}$-regular?
\end{question}

 One may hope to answer the previous question as in \cite{Liu2023}, but using the agrarian polytope \cite{HennekeKielak2021,Kielak2020polytopes} in place of the Thurston polytope.  The key issue is that we do not have the $\TAP_1$ property for free-by-cyclic groups (for $3$-manifolds this is a deep result of Friedl--Vidussi \cite{FriedlVidussi2008,FriedlVidussi2011annals}).  The reader is referred to  \cite[Definition~3.1]{HughesKielak2022} for the definition due to its technical nature.
 
\begin{question}\label{q.TAP1}
    Is every free-by-cyclic group $G$ in $\mathsf{TAP}_1(\FF)$ for $\FF\in\{\QQ,\FF_p\}$ with $p$ prime? 
\end{question}

The other somewhat obvious question is whether irreducibility is a profinite invariant.  We expect this to be the case (at least amongst hyperbolic free-by-cyclic groups) and so leave this as one final question.

\begin{question}\label{q.irreducibility}
    Is being irreducible a profinite invariant amongst free-by-cyclic groups?
\end{question}


\subsection{Structure of the paper}
In \Cref{prelim:autos} we recall the necessary background on free group automorphisms and free-by-cyclic groups and prove a number of results we will need throughout the paper.  

In \Cref{TopRep} we recall the definition of a topological representative of a free group automorphism, its stretch factor and the various definitions of irreducibility we will need.  We include a proof that there are at most finitely many equivalence classes of irreducible topological representatives such that the graph has rank $n$ and the stretch factor is at most some positive real number $C>1$ (\Cref{min}).  

In \Cref{sec:generic} we study generic outer automorphisms of free groups and prove that a generic free-by-cyclic group has first Betti number equal to one and is super irreducible (\Cref{generic_properties}).  

In \Cref{sec:Nielson} we relate the Nielsen numbers of an outer automorphism of a free group to the stretch factor of the outer automorphism.

In \Cref{sec:hyp} we study certain subgroup separability properties of free-by-cyclic groups.  In particular, we show that every abelian and every free-by-cyclic subgroup is separable (\Cref{fully_separable}).  We combine this with results of Wilton--Zalesskii \cite{WiltonZalesskii2017} to prove \Cref{thmx:hyperbolicity} from the introduction.

In \Cref{sec:AP} we recall the definitions of twisted Alexander polynomials and twisted Reidemeister torsions.  We establish a number of facts about twisted Alexander polynomials which we will use later in the paper.  Our main new contribution is a complete calculation of the zeroth twisted Alexander polynomials over $\QQ$ for any path-connected topological space (\Cref{zeroth AP palindromic}) 
% --- part of the proof requires reproducing a lemma of Dunfield--Friedl--Jackson which has not appeared in print before and we thank them for letting us do this (\Cref{DFJ Lemma}). 
As well as formula for the twisted Reidemeister torsion of a free-by-cyclic group in terms of the twisted Alexander polynomials.

In \Cref{sec:regularity} we recall the notion of a matrix coefficient module and a $\widehat{\Z}$-regular isomorphism.  The main reason for this section is to allow us to work in the generality of a $\widehat{\Z}$-regular isomorphism, thus, if one established a positive answer to \Cref{q.Zregular} then one could apply the results in this paper without any further modifications.  

At this stage we establish some notation.  Let $G_A$ be a free-by-cyclic group with character $\psi$ and fibre subgroup $F_A$.  Also let $G_B$ be a free-by-cyclic group with character $\varphi$ and fibre subgroup $F_B$.  Let $\Theta\colon \widehat{G}_A\to\widehat{G}_B$ be a $\widehat{\Z}$-regular isomorphism (see \Cref{defn:Zhatreg}).  Our final result of the section is that $F_A\cong F_B$ (\Cref{fibre iso}).

In \Cref{sec:Rtorsion} we set out to prove profinite invariance of Reidemeister torsion over $\QQ$ twisted by representations of finite quotients for $G_A$ and $G_B$.  Our strategy is parallel to that of Liu \cite[Section~7]{Liu2023}, however due to the extra complexity of free-by-cyclic groups we have to invoke extra results about twisted Alexander polynomials of free-by-cyclic groups established in \Cref{sec:AP}.  In \Cref{sec:AP.profinite} we prove profinite invariance of the twisted Alexander polynomials although we work in the more general setting of $\{$good type $\mathsf{F}\}$-by-$\Z$ groups and $\widehat{\Z}$-regular isomorphisms.  In \Cref{sec:profRtorsion} we establish the profinite invariance of twisted Reidemeister torsion for $G_A$ and $G_B$.  In \Cref{sec.homostretch} we prove that the homological stretch factors $\{\nu_A,\nu_{A}^{-}\}$ and $\{\nu_B,\nu_{B}^{-}\}$ are equal.

In \Cref{sec:profNielson}, under the assumption of conjugacy separability of $G_A$ and $G_B$ we prove that the homotopical stretch factors $\{\lambda_A,\lambda_A^{-}\}$ and $\{\lambda_B,\lambda_{B}^{-}\}$ are equal.  Again our strategy is largely motivated by \cite[Section~8]{Liu2023}.  The key difference is that for a fibred character $\chi$ on a finite volume hyperbolic $3$-manifold the stretch factors of $\chi$ and $\chi^{-1}$ are the same.  This is not true for free-by-cyclic groups and so our main work is resolving this issue.

Combining the major results up to this point proves \Cref{thmx:invariants}.

In \Cref{sec.proofmain} we prove \Cref{thmx:Irr}.  In the hyperbolic case this is a corollary of \Cref{thmx:invariants} and the fact that hyperbolic free-by-cyclic groups are virtually special and hence conjugacy separable.  In the general case we apply a result of Mutanguha \cite{Mutanguha2021} and train track theory to deduce the conjugacy separability we require.

In \Cref{sec.proconjugacy} we prove \Cref{thmx:procongruence}.  This is really an easy consequence of \Cref{thmx:invariants} once we transport a result of Liu \cite[Proposition~3.7]{Liu2023b} on procongruence conjugacy of mapping class groups to the $\Out(F_n)$ setting.

Finally, in \Cref{sec:Wn} we prove results on profinite invariants and profinite almost rigidity of \{universal Coxeter\}-by-cyclic groups. To do so, we start by establishing notation and recalling background on morphisms of graphs of groups in \Cref{sec:graphofgroups}. The purpose of \Cref{sec:traintrackWn} is to relate the theory of train track representatives of elements in $\mathrm{Out}(W_n)$ with Nielsen fixed point theory. We also prove a lemma on irreducibility of covers of directed graphs and use this to relate the stretch factor of an outer automorphism $\Phi \in \mathrm{Out}(W_n)$ with the stretch factor of the free group automorphism obtained by restricting $\Phi$ to a free characteristic subgroup of $W_n$. In the final \Cref{sec:ProfiniteCoxeter} we combine results from previous sections to prove \Cref{thmx:CoxeterStretchInvariance} and  \Cref{thmx:IrredCoxeterAlmostProfinite}.

\newpage

\subsection{Notation}
We include a table of notation for the aid of the reader.

\begin{table}[h]
    \centering
    \begingroup
    \renewcommand{\arraystretch}{1.2}
\begin{tabular}{|c|p{10cm}|}
\hline
Symbol & Definition\\
\hline
    $F_n$ & Free group of rank $n$\\
    $W_n$ & Universal Coxeter group of rank $n$, that is, $\aster_{i=1}^n\Z/2$\\
    \hline
    $\Gamma$ & Graph \\
    $(\Gamma, \mathcal{G})$, $\mathcal{G}$ & Graph of groups \\
    $X_{\mathcal{G}}$ & Graph of spaces \\
    $(f, f_X), f$ & Morphism of graphs of groups \\
     \hline
    $\psi$, $\varphi$, $\psi_A$, $\varphi_B$ & Character of a free-by-cyclic group\\
    $(G_A,\psi$), $(G_B,\varphi)$ & Free-by-cyclic group \\
    $F$, $F_A$, $F_B$ & Fibre subgroup \\
    $\Psi$, $\Phi$ & Outer automorphisms (of $F_n$ or $W_n$)\\
    $f$, $f_A$, $f_B$ & Train track \\
    $A$ & Incidence matrix \\
    $\mathrm{Orb}_m(f)$  & Set of $m$-periodic orbits of $f$\\
    $N_m(f)$ & $m$th Nielsen number of $f$\\
    $\lambda$, $\lambda_f$, $\lambda_\Psi$ & Homotopical stretch factor (of $f$ or $\Psi$) \\
    $\nu$, $\nu_f$, $\nu_\Psi$ & Homological stretch factor (of $f$ or $\Psi$)\\
    \hline
    $R$ & Unique factorisation domain\\
    $R^\times$ & Units of $R$\\
    $\Delta_{R,n}^{\varphi,\alpha}$ & $n$th Alexander polynomial of $\varphi$ twisted by $\alpha$ over $R$\\
    $\tau_R^{\varphi,\alpha}$ & Reidemeister torsion of $\varphi$ twisted by $\alpha$ over $R$\\
    \hline
    \multicolumn{2}{|p\textwidth|}{In some contexts we will drop the $R$ from the previous notations and replace it with a group $G$ for clarity, that is, $\Delta_{G,n}^{\varphi,\alpha}$ and $\tau_G^{\varphi,\alpha}$ or even $\tau_{G,R}^{\varphi,\alpha}$}\\
    \hline
    $\alpha$, $\beta$, $\gamma$ & Finite quotients \\
    $Q$ & Image of a finite quotient \\
    $\rho$, $\sigma$ & Representation of a group\\
    $\chi_\rho$ & Character of the representation $\rho$\\
    $\gamma^\ast(\sigma)$ & Pullback representation of $\sigma$ along $\gamma$\\
    $\mathbf{1}$ & The trivial representation\\
    \hline
    $\Theta$ & Profinite isomorphism\\
    $\MC(\Theta)$ & Mapping coefficient module\\
     $\Theta_\ast^\epsilon$, $\Theta^\ast_\epsilon$ & $\epsilon$-specialisation of $\Theta$\\
     $\mu$ & Unit of $\widehat{\Z}$\\
     \hline
\end{tabular}
\endgroup
    \caption{Table of notation.}
    \label{tab:notation}
\end{table}

\pagebreak

\subsection*{Acknowledgements}
We would like to extend a big thank you to Naomi Andrew, Martin Bridson, and especially Dawid Kielak for a number of helpful conversations.  
% We would also like to thank Nathan Dunfield, Stefan Friedl, and Nicholas Jackson for kindly letting us reproduce \Cref{DFJ Lemma} here.

This work has received funding from the European Research Council (ERC) under the European Union's Horizon 2020 research and innovation programme (Grant agreement No. 850930). The second author was supported by an Engineering and Physical Sciences Research Council studentship (Project Reference 2422910). 

\section{Preliminaries on free group automorphisms}\label{prelim:autos}

\subsection{Topological representatives}\label{TopRep}

Let $n \geq 2$ and $\Phi \in \mathrm{Out}(F_n)$ be an outer automorphism of $F_n$. A \emph{topological representative} of $\Phi$ is a tuple $(f, \Gamma)$, where $\Gamma$ is a connected graph with $\pi_1(\Gamma)\cong F_n$, and $f \colon \Gamma \to \Gamma$ is a homotopy equivalence which induces the outer automorphism $\Phi$. Furthermore, $f$ preserves the set of vertices of $\Gamma$ and it is locally injective on the interiors of the edges of $\Gamma$. Fix an ordering of the edges of $\Gamma$. The \emph{incidence matrix} $A$ of $f$ is the matrix with entries $a_{ij}$, which is the number of occurrences of the unoriented edge $e_j$ in the edge-path $f(e_i)$.

Recall that a non-negative integral $n$-by-$n$ square matrix $M$ is said to be \emph{irreducible}, if for any $i,j \leq n$, there exists some $k\geq 1$ such that the $(i,j)$-th entry of $M^k$ is positive.

Let $(f, \Gamma)$ be a topological representative. A \emph{filtration of length l} of $(f, \Gamma)$ is a sequence of subgraphs
\begin{equation}\label{filtration} \emptyset = \Gamma_0 \subseteq \Gamma_1 \subseteq \ldots \subseteq \Gamma_l = \Gamma,\end{equation}
so that $f(\Gamma_i) \subseteq \Gamma_i$ for all $i$. The closure $S_i = \mathrm{Cl}(\Gamma_i \setminus \Gamma_{i-1})$ is called the \emph{i}th \emph{stratum} of the filtration. Re-order the edges of $\Gamma$ so that whenever $i < j$, the edges in $\Gamma_i$ precede the edges in $\Gamma_j$.  The filtration is said to be \emph{maximal} if the square submatrix $A_i$ of the incidence matrix $A$ which corresponds to the $i$-th stratum is either the zero matrix, or it is irreducible. It is a standard fact that any topological representative admits a maximal filtration which is unique up to reordering of the strata. If $(f, \Gamma)$ admits a maximal filtration of length one then we say that $(f, \Gamma)$ is \emph{irreducible.}

By the Perron--Frobenius theorem (see Chapter~2 in \cite{Seneta2006}), if $A_i$ is the submatrix of the incidence matrix $A$ of $(f, \Gamma)$ which corresponds to an irreducible stratum $S_i$, then the spectral radius $\rho(A_i)$ of $A_i$ is an eigenvalue of $A_i$, which is known as the \emph{Perron--Frobenius eigenvalue} of $A_i$. Furthermore, $\rho(A_i) \geq 1$ and equality holds exactly when $A_i$ is a permutation matrix. We call $S_i$ an \emph{exponentially-growing stratum} if its Perron--Frobenius eigenvalue is strictly greater than one. For a topological representative $(f, \Gamma)$, we write $\lambda_f$, (or $\lambda$ if there is no potential for confusion), to denote the maximal Perron--Frobenius eigenvalue taken over all the non-zero strata of the maximal filtration of $(f, \Gamma)$, and we call it the \emph{(homotopical) stretch factor} of $f$.

Let $(f, \Gamma)$ be a topological representative. The \emph{mapping torus}  $M_f$ of $f$ is the quotient space 
\[M_f = \frac{\Gamma \times [0,1]}{(f(x), 0) \sim (x, 1)}.\]
We define an equivalence relation $\sim$ on the set of topological representatives of elements of $\mathrm{Out}(F_n)$, so that $(f_1, \Gamma_1) \sim (f_2, \Gamma_2)$ whenever there exists an isomorphism $\pi_1(M_{f_1}) \cong \pi_1(M_{f_2})$.

The following lemma is well known to the experts. We reproduce the proof here for completeness and in order to refer to it in \Cref{sec:traintrackWn}.

\begin{lemma}\label{min}
Let $n \geq 2$ and $C > 1$. There exist at most finitely many equivalence classes of topological representatives $(f, \Gamma)$ of elements in $\mathrm{Out}(F_n)$, such that $f$ is irreducible and the stretch factor $\lambda$ of $f$ satisfies $\lambda \leq C$. 
\end{lemma}

\begin{proof}
    Let $(f, \Gamma)$ be an irreducible topological representative with $\pi_1(\Gamma) \cong F_n$. By composing valence-one and valence-two homotopies, and collapsing maximal invariant forests (see \cite[Section 1]{BestvinaHandel1992}), we obtain an irreducible representative $(f', \Gamma')$ which is equivalent to $(f,\Gamma)$, and such that $\pi_1(\Gamma') \cong F_n$ and $\Gamma'$ has no valence-one and valence-two vertices. We now argue as in \cite[Theorem~1.7]{BestvinaHandel1992}. By a simple Euler characteristic argument, the number of edges in $\Gamma'$ is bounded above by $3n - 3$. Furthermore, it is known that the Perron--Frobenius eigenvalue of an irreducible matrix $A$ is bounded below by the minimum of the sums of the rows of $A$. Hence the transition matrix of $(f', \Gamma')$ can take on at most finitely many values. 
\end{proof}

A subgraph is \emph{non-trivial} if it has a component which is not a vertex. An outer automorphism $\Phi \in \mathrm{Out}(F_n)$ is \emph{irreducible}, if every topological representative $(f, \Gamma)$ of $\Phi$, where $\Gamma$ has no valence-one vertices and no non-trivial $f$-invariant forests, is irreducible. A free-by-cyclic group $G$ is \emph{irreducible}, if $G$ admits a splitting $G \cong F_n \rtimes_{\Phi} \mathbb{Z}$, with $\Phi \in \mathrm{Out}(F_n)$ irreducible. Note that by \cite{Mutanguha2021}, if $G$ is irreducible then the monodromy associated to every fibred splitting of $G$ is an irreducible outer automorphism.


The \emph{stretch factor} of an irreducible outer automorphism $\Phi$ is the minimum of the stretch factors of the irreducible topological representatives of $\Phi$. Such a minimum exists by \Cref{min}. Any train track representative $(f, \Gamma)$ of $\Phi$ realises the stretch factor of $\Phi$. For a general outer automorphism $\Phi$, we define the stretch factor of $\Phi$ to be the maximum of the set of stretch factors of the non-zero strata of any relative train track representative. 


\subsection{Generic elements of \texorpdfstring{$\mathrm{Out}(F_n)$}{Out(Fn)}}\label{sec:generic}

Fix a finite generating set $S$ of $\mathrm{Out}(F_n)$. For each $l \geq 1$, let $\mathcal{W}_l$ denote the subset of elements $\Phi \in \mathrm{Out}(F_n)$ such that the shortest word representative of $\Phi$ has length $l$. We say that a \emph{random element of $\mathrm{Out}(F_n)$ satisfies property $P$ with probability $p$,} if
\[\frac{\#\{\Phi \in \mathcal{W}_l \mid \Phi \text{ satisfies }P\}}{\#\mathcal{W}_l} \to p \text{ as }l \to \infty.\]
We say that a \emph{generic element in $\mathrm{Out}(F_n)$ has property $P$}, if a random element satisfies property $P$ with probability $p = 1$.

The following theorem is a consequence of the results in Section~7 of \cite{Rivin2008}, which hold verbatim after replacing $\mathrm{SL}(n, \Z)$ by $\mathrm{GL}(n, \mathbb{Z})$ in all the statements.

\begin{thm}[{\cite{Rivin2008}}]\label{Rivin}
A generic element in $\mathrm{Out}(F_n)$ is super irreducible.
\end{thm}

\begin{prop}\label{genericBetti}
    For a generic element $\Phi \in \mathrm{Out}(F_n)$, the first Betti number of $F_n \rtimes_{\Phi} \Z$ is equal to one.
\end{prop}

\begin{proof}Write $\Phi_{\mathrm{ab}}$ to denote the image of $\Phi$ under the natural map induced by the abelianisation of $F_n$,
\[ \begin{split} \mathrm{Out}(F_n) &\to \mathrm{GL}(n, \Z)  \\ \Phi &\mapsto \Phi_{\mathrm{ab}}.\end{split}\]
    The free abelianisation of $F_n \rtimes_{\Phi} \Z$ is isomorphic to $\Z$ if and only if $\Phi_{\mathrm{ab}}$ has no eigenvalue equal to 1 \cite[Theorem~2.4]{BogopolskiMartinoVentura2007}. By \Cref{Rivin}, for a generic element $\Phi$ in $\mathrm{Out}(F_n)$, $\Phi_{\mathrm{ab}}$ has irreducible characteristic polynomial. Hence the result follows. 
\end{proof}

Write $\mathcal{G}_l$ to denote the set of all (isomorphism types of) free-by-cyclic groups $G$ such that $G \cong F_n \rtimes_{\Phi} \Z$ and $\Phi \in \mathcal{W}_l$. We say that a \emph{random $F_n$-by-cyclic group satisfies property $P$ with probability $p$}, if 
\[\frac{\# \{G \in \mathcal{G}_l \mid G \text{ satisfies }P\}}{\# \mathcal{G}_l} \to p \text{ as }l \to \infty.\]

An outer automorphism $\Phi \in \mathrm{Out}(F_n)$ is said to be \emph{super irreducible} if no positive power of the induced map $\Phi_{\mathrm{ab}} \in \mathrm{GL}(n, \QQ)$ maps a proper subspace of $H_1(F_n;\QQ)$ into itself.  A free-by-cyclic group $G$ is \emph{super irreducible} if there exists some splitting $G \cong F_n \rtimes_{\Phi} \Z$ such that $\Phi$ is super irreducible.

\begin{prop}\label{generic_properties} A generic $F_n$-by-cyclic group has first Betti number equal to one and is super irreducible. 
\end{prop}

\begin{proof}

Let $G \cong F_n \rtimes_{\Phi} \Z$ and suppose that $b_1(G) = 1$ and $G \in \mathcal{G}_l$. Then the monodromies associated to $G$ are exactly $\Phi$ and $\Phi^{-1}$. Both of these have length $l$ in $S$, and if $G$ is super irreducible then at least one of them is super irreducible. Thus for every $l \geq 1$, 
\[ \begin{split} &\#\{G \in \mathcal{G}_l \mid b_1(G) = 1 \text{ and $G$  super irreducible}\}\\
&\geq  \#\{ \Phi \in \mathcal{W}_l \mid b_1(F_n \rtimes_{\Phi} \mathbb{Z}) = 1\text{ and $\Phi$ is super irreducible}\}. \end{split}\]
Note also that the obvious map $\mathcal{W}_l \to \mathcal{G}_l$ which sends an element $\Phi \in \mathcal{W}_l$ to the associated free-by-cyclic group $F_n \rtimes_{\Phi} \Z$ is surjective. Hence
\[\begin{split}
    &\#\{G \in \mathcal{G}_l \mid b_1(G) = 1 \text{ and $G$ is super irreducible} \} / \#\mathcal{G}_l \\
    &\geq \#\{ \Phi \in \mathcal{W}_l \mid b_1(F_n \rtimes_{\Phi} \Z) = 1 \text{ and $\Phi$ super irreducible} \} / \#\mathcal{W}_l 
    \end{split}
\]
By \cref{Rivin} and \cref{genericBetti}, the latter tends to $1$ as $l \to \infty$. The result follows. \end{proof}

\subsection{Nielsen fixed point theory}\label{sec:Nielson}

Let $X$ be a connected topological space and $f \colon X \to X$ a self-map. An \emph{m-periodic point} $p \in X$ is a fixed point of the map $f^m$. Let $\mathrm{Orb}_m(f)$ be the set of orbits of $m$-periodic points of $X$ under the action of $f$. Each orbit $\mathcal{O} \in \mathrm{Orb}_m(f)$ determines a free homotopy class of loops in the mapping torus $M_f$, and thus a conjugacy class in $\pi_1(M_f)$, which we denote by $\mathrm{cd}(\mathcal{O})$. Furthermore, every $\mathcal{O} \in \mathrm{Orb}_m(f)$ admits an index $\mathrm{ind}_m(f; \mathcal{O}) \in \mathbb{Z}$, which is the fixed point index of $f^m$ at any point $p \in \mathcal{O}$ (see \cite[Section~1.3]{Jiang1996}). A periodic orbit of a point in $X$ under the action of $f$ is said to be \emph{essential} if it has non-zero index. Note that if $X$ is a graph then an isolated fixed point $x$ has index zero, if and only if $f$ is not locally injective at $x$.  

\begin{defn} The \emph{$m$-th Nielsen number} of $f$, denoted by $N_m(f)$, is the number of essential $m$-periodic orbits of $f$. 
\end{defn}

It is a standard fact from Nielsen fixed point theory (see e.g. \cite[Chapter 1]{Jiang1983} and \cite{Jiang1996}), that each Nielsen number is independent of the choice of topological representative of $\Phi$. Hence, we may write $N_{\infty}(\Phi)$ to denote 
\[ N_{\infty}(\Phi) = \mathrm{lim}\,\mathrm{sup}_{m\to \infty}N_m(f)^{1/m},\]
where $(f, \Gamma)$ is any topological representative of $\Phi$.

\begin{prop}\label{train_track} Let $\Phi \in \mathrm{Out}(F_n)$. Then $N_{\infty}(\Phi)$ is equal to the stretch factor $\lambda$ of $\Phi$.
\end{prop}

\begin{proof}

By \cite{BestvinaFeighnHandel2000}, there exists a positive integer $k$ such that $\Phi^k$ admits a topological representative $(f, \Gamma)$ which is an improved relative train track. Let $A$ be the corresponding incidence matrix and fix a maximal filtration of $\Gamma$. Let $\{S_i\}_{i\in I}$ be the set of exponentially-growing strata of $\Gamma$ and write $A^m_i$ to denote the submatrix of $A^m$ spanned by the edges of $S_i$. Let $\lambda_i$ be the stretch factor of $S_i$. 

For each exponentially-growing stratum $S_i$, there exists a length assignment $L \colon E(\Gamma) \to \mathbb{R}_{\geq 0}$ on the edges of $\Gamma$, such that $L(e) > 0$ and $L(f(e)) = \lambda_i \cdot L(e)$, for every edge $e$ in $S_i$. Hence the number of fixed points of $f^m$ contained in the interior of the edge $e$ of $\Gamma_i$ is given by the number of times the edge path $f^m(e)$ crosses the edge $e$ in either direction. This is precisely the element on the diagonal of the matrix $A^m$ corresponding to the edge $e$. 

By definition of improved relative train tracks, every periodic Nielsen path has period one, and for each exponentially growing stratum $S_i$, there is at most one indivisible Nielsen path which intersects $S_i$. Furthermore, the number of Nielsen fixed point classes which intersect the non-exponentially-growing strata non-trivially or which contain a vertex is uniformly bounded as $m$ goes to infinity.  Hence, there exists some constant $C$ such that 
\[\begin{split} \mathrm{lim}\,\mathrm{sup}_{m\to \infty}N_m(f)^{1/m} &=  \mathrm{lim}\,\mathrm{sup}_{m\to \infty}\left(C + \sum_{i\in I} \mathrm{tr}(A^m_i) \right)^{1/m}\\ &= \mathrm{max}\{\lambda_i\}_{i\in I} = \lambda_f.\end{split}\]
Thus, $N_{\infty}(\Phi^k)$ is equal to the stretch factor of $\Phi^k$.

By \cite[Corollary~7.14]{FrancavigliaMartino2021}, if $\lambda$ is the maximal stretch factor of a relative train track representative of $\Phi^k$, then $\lambda^{1/k}$ is the maximal stretch factor associated to $\Phi$. Note also that $N_{\infty}(\Phi^k) = N_{\infty}(\Phi)^k$. The result follows by combining the arguments in the previous paragraphs. \end{proof}


\subsection{Detecting atoroidal monodromy}\label{sec:hyp}
In this section we will prove that hyperbolicity (equivalently the property of admitting atoroidal monodromy) is determined by the profinite completion.  The strategy is to show finitely generated abelian subgroups of free-by-cyclic groups are fully separable and the use work of Brinkmann \cite{Brinkmann2000} and Wilton--Zalesskii \cite{WiltonZalesskii2017}.

Recall a subgroup $H\leqslant G$ is \emph{separable} if for every $g\in G\backslash H$ there exists a finite quotient $\rho\colon G\onto Q$ such that $\rho(g)\not\in\rho(H)$.  A subgroup $H$ is \emph{fully separable} if every finite index subgroup of $H$ is separable in $G$.

\begin{prop}\label{sep reducible}
Let $G$ be a free-by-cyclic group and let $H\leqslant G$ be a subgroup. If $H$ is free-by-cyclic or cyclic, then $H$ is separable in $G$. 
\end{prop}

\begin{proof}
Fix a fibred character $\varphi \colon G \to \mathbb{Z}$ of $G$. Let $F = \mathrm{ker}\varphi$ be the fibre, $t \in \varphi^{-1}(1)$ and $\Phi \in \mathrm{Aut}(F)$ the automorphism corresponding to the conjugation action of $t$ on $F$ in $G$. 

Let $H \leq G$ be a free-by-cyclic subgroup of $G$. By \cite[Proposition~2.3]{FeighnHandel1999}, there exist a finitely generated subgroup $A \leq F$, an element $y\in F$ and a positive integer $k$ such that $\Phi^k(A) = yAy^{-1}$ and $H = A \rtimes \langle t^ky \rangle$. Let $g \in G \setminus H$. Then $g = bt^m$, for some $b \in F$ and $m \in \mathbb{Z}$. Suppose that $m$ is not a multiple of $k$. The subgroup $G' = \langle F, t^ky \rangle \cong F \rtimes_{ \iota_y \Phi^k } \mathbb{Z}$, where $\iota_y \in \mathrm{Aut}(F)$ denotes the inner automorphism $\iota_y(x) = y^{-1}xy$ for all $x \in F$, is a finite index subgroup of $G$, and it is free-by-cyclic. Thus, using normal forms it is easy to see that $H \leq G'$ and $g \not\in G'$. 

Suppose now that $m = kl$ for some $l \in \mathbb{Z}$. Then $g = b' (t^ky)^l$, for some $b' \in F$, and since $g \not \in H$ it must be that $b' \not \in A$. The usual Marshall--Hall argument gives a finite-index subgroup $F' \leq F$ such that $b' \not\in F'$ and $A \leq F'$. Let $N = [F : F']$. Since $\iota_y \cdot \Phi^k \colon F \to F$ is an automorphism, it permutes the (finite) set of subgroups of $F$ of index $N$. Hence there exists some positive integer $M$ such that $(\iota_y \cdot \Phi^k)^M(F') = F'$. Let $F'' = \bigcap_{i=0}^{M-1} (\iota_y \cdot \Phi^k)^i(F')$. Then $\iota_y \cdot \Phi^k (F'') = F''$ and $A \leq F''$. Furthermore, since $F'' \leq F'$, we have that $b' \not \in F''$. Thus $G' = \langle F'', t^ky \rangle \cong F'' \rtimes \langle t^ky \rangle$ is a finite index subgroup of $G$ which is free-by-cyclic. Again, by normal forms it follows that $H \leq G'$ and $g \not \in H$. \end{proof}

%A group $G$ is said to be \emph{locally abelian extended, residually finite}, or \emph{LAERF} for short, if every finitely generated abelian subgroup is separable in $G$. 

%\begin{corollary}\label{LAERF}
%    If $G$ is free-by-cyclic then it is LAERF.
%\end{corollary}

%\begin{proof}
%    Let $H$ be a finitely generated abelian subgroup of a free-by-cyclic group $G$. Then by \cite{FeighnHandel1999}, $H\cong \Z$ or $H \cong \Z^2$. Hence the result follows by \cref{sep reducible}.
%\end{proof}

\begin{corollary}\label{fully_separable}
Let $G$ be a free-by-cyclic group.  If $H\leq G$ is a free-by-cyclic or abelian subgroup, then $H$ is fully separable in $G$.  In particular $\bar H$, the closure of $H$ in $\widehat{G}$, is isomorphic to $\widehat{H}$.
\end{corollary}
\begin{proof}
Every finite-index subgroup of $H$ is free-by-cyclic or abelian.  It follows from \Cref{sep reducible} that every finite-index subgroup of $H$ is separable (including $H$ itself).  The result now follows from \cite[Lemma~4.6]{Reid2015}.
\end{proof}

We have everything we need to prove \Cref{thmx:hyperbolicity} from the introduction.

\medskip

\begin{duplicate}[\Cref{thmx:hyperbolicity}]
Let $G_A$ and $G_B$ be profinitely isomorphic free-by-cyclic groups. Then $G_A$ is Gromov hyperbolic if and only if $G_B$ is Gromov hyperbolic.
\end{duplicate}

\smallskip

\begin{proof}
Let $G_A$ and $G_B$ be free-by-cyclic groups such that $\widehat G_A\cong \widehat G_B$. Suppose that $G_A$ is Gromov hyperbolic. By \cite{HagenWise2015}, $G_A$ is a cocompactly cubulated and thus virtually special.  Hence we may apply \cite[Theorem~D]{WiltonZalesskii2017} to deduce that $\widehat\ZZ^2$ is not a subgroup of $\widehat G_A$.  By \Cref{fully_separable}, the $\Z^2$ subgroups of $G_B$ are fully separable and since $\widehat G_B$ contains no $\widehat\ZZ^2$ subgroups, it follows $G_B$ contains no $\ZZ^2$ subgroups.  In particular, by \cite[Theorem~1.2]{Brinkmann2000} $G_B$ is Gromov hyperbolic.

Suppose conversely that $G_A$ is not Gromov hyperbolic. Then by \cite{Brinkmann2000}, $G_A$ has a $\ZZ^2$ subgroup and so by \Cref{fully_separable}, $\widehat{G}_A$ contains a $\widehat{\ZZ}^2$ subgroup.  Suppose now $G_B$ is not Gromov hyperbolic, then by the argument in the previous paragraph $\widehat{G}_B$ does not contain $\widehat{\ZZ}^2$ subgroups.  This contradiction completes the proof.
\end{proof}

We will need the following proposition later.

\begin{prop}\label{fully separable fibres}
Let $G$ be a finitely generated, residually finite group and $\varphi \colon G \to \Z$ an epimorphism. Suppose that $N = \ker\varphi$ is finitely generated. Then $N$ is fully separable in $G$.
\end{prop}

\begin{proof}
By \cite[Lemma 4.6]{Reid2015}, it suffices to prove that every finite index subgroup $N' \leq_f N$ of $N$ is closed in the profinite topology on $G$. Note that since $N$ is finitely generated, for any such subgroup $N' \leq_f N$ there exists a finite index subgroup $G' \leq_f G$ such that $\varphi$ induces an epimorphism $\varphi' \colon G' \to \Z$ with $N' = \ker\varphi'$. Since $G'$ has finite index in $G$, every subset of $G'$ which is closed in the profinite topology on $G'$ is also closed in the profinite topology on $G$. Hence it suffices to show that $N$ is closed in the profinite topology on $G$. To that end, fix $t \in \varphi^{-1}(1)$ and let $g \in G \setminus N$. Then $\varphi(g) = k$, for some $k \neq 0$. Let $C_{|k|+1}$ denote the cyclic group of order $|k| + 1$. Define a homomorphism $\pi \colon G \to  C_{|k|+1}$ which sends $t$ to a generator of $C_{|k|+1}$ and all the elements of $N$ to $0$. Then $\pi(g) \neq 0$, whereas $N \leq \ker\pi$. 
\end{proof}





\section{Some properties of twisted Alexander polynomials and Reidemeister torsion}\label{sec:AP}
In this section we will collect a number of facts about twisted Alexander polynomials and twisted Reidemeister torsion that we will use later on.  Our main contribution is a complete computation of the zeroth Alexander polynomials twisted by representations factoring through finite groups over characteristic zero fields (\Cref{zeroth AP palindromic}).

\begin{defn}[Alexander modules and polynomials]
Let $R$ be a unique factorisation domain and let $G$ be a finitely generated group.  Let $\varphi$ be a non-trivial primitive class in $H^1(G;\Z)$ considered as a homomorphism $G\onto \Z$ and let $\rho\colon G\to \GL_n(R)$ be a representation. Consider $R^n[t^{\pm 1}]$ equipped with the $RG$-bimodule structure given by 
\[g.x = t^{\varphi(g)}\rho(g)x, \quad x.g = xt^{\varphi(g)}\rho(g) \]
 for $g \in G, x \in  R^n[t^{\pm 1}]$.  For $n\in\ZZ$, we define the \emph{$k$th twisted Alexander module of $\varphi$ and $\rho$} to be $H_k(G;R^n[t^{\pm 1}])$, where $R^n[t^{\pm 1}]$ has the right $RG$-module structure described above. Observe that $H_k(G;R^n[t^{\pm 1}])$ also has the structure of a left 
 $R[t^{\pm 1}]$-module.
If $G$ is of type $\mathsf{FP}_k(R)$, then the $k$th twisted Alexander module is a finitely generated $R[t^{\pm 1}]$-module.  Moreover, it is zero whenever $k<0$ or $k$ is greater than the cohomological dimension of $G$ over $R$.

Since $R$ is UFD so is $R[t^{\pm1}]$.  Let $M$ be an $R[t^{\pm1}]$-module.  The \emph{order} of $M$ is the greatest common divisor of all maximal minors in a presentation matrix of $M$ with finitely many columns.  The order of $M$ is well-defined up to a unit of $R[t^{\pm1}]$ and depends only on the isomorphism type of $M$. 

Suppose that $G$ is of type $\mathsf{FP}_k(R)$.  The \emph{$k$th twisted Alexander polynomial} $\Delta_{k,R}^{\varphi,\rho}(t)$ over $R$ with respect to $\varphi$ and $\rho$ is defined to be the order of the $k$th twisted (homological) Alexander module of $\varphi$ and $\rho$, treated as a left $R[t^{\pm 1}]$-module.
\end{defn}

We will now collect a number of facts about twisted Alexanbder polynomials. The following lemma is a triviality.

\begin{lemma}\label{AP conjugacy}
    Let $X$ be a path-connected non-empty topological space.  Let $G=\pi_1 X$ be a finitely generated group, let $\varphi\colon G\twoheadrightarrow\ZZ$, and let $\rho,\sigma\colon G\to\GL_n(R)$ be representations of $G$ over a UFD $R$.  If $\rho$ and $\sigma$ are conjugate representations, then
\[\Delta_n^{\varphi,\rho}(t)\doteq \Delta_n^{\varphi,\sigma}(t). \]
\end{lemma}

\begin{lemma}\label{AP direct sum formula}
Let $X$ be a path-connected non-empty topological space.  Let $G=\pi_1 X$ be a finitely generated group, let $\varphi\colon G\twoheadrightarrow\ZZ$, and let $\rho,\sigma\colon G\to\GL_n(R)$ be representations of $G$ over a UFD $R$.  Then,
\[\Delta_n^{\varphi,\rho\oplus\sigma}(t)\doteq \Delta_n^{\varphi,\rho}(t)\times\Delta_n^{\varphi,\sigma}(t). \]
\end{lemma}
\begin{proof}
This follows from the fact that homology commutes with taking direct sums of coefficient modules. 
\end{proof}

The next lemma will be a key step in proving profinite rigidity of twisted Reidemeister torsion for our class of free-by-cyclic groups.  Recall for a $G$-module $M$ being acted on via $\alpha\colon G\times M\to M$ we write $M_\alpha$ when we wish to make clear the $G$-module structure.

\begin{lemma}\label{zeroth AP palindromic}
Let $X$ be a topological space.  Let $G=\pi_1 X$ be a finitely generated group, let $\varphi\colon G\twoheadrightarrow\ZZ$ be algebraically fibred, and let $\rho\colon G\twoheadrightarrow Q\to\GL_k(\QQ)$ be a representation factoring through a finite group. Then,
\[\Delta^{\varphi,\rho}_0(t)\doteq (1-t)^n P(t),\] 
where $n\geq 0$ and $P(t)$ is a product of cyclotomic polynomials, up to multiplication by monomials with coefficients in $\QQ^\times$.  In particular,
\[\Delta^{\varphi,\rho}_{0}(t)\doteq\Delta^{\varphi,\rho}_{0}(t^{-1}). \]
\end{lemma}
\begin{proof}
Let $F$ denote the kernel of $\varphi$.  We need to compute $M\coloneq H_0(G;\QQ^k[t^{\pm1}])$ which is naturally isomorphic to $(\QQ^k[t^{\pm1}])_G$.  %This latter group is isomorphic to $(\QQ^k)_Q$ by \cite[II.6.2 and III.8.2]{Brown1982}.

By Maschke's Theorem we may write the representation $\rho$ of $Q$ as a sum $\oplus_{i=1}^\ell\rho_i\colon Q\to \prod_{i=1}^\ell \GL_{k_i}(\QQ)$, where $\sum_{i=1}^\ell k_i =k$, of irreducible $\QQ$-representations of $L$.   We may now write \[M = \bigoplus_{i=1}^\ell (\QQ^{k_i}[t^{\pm1}])_G.\]

For each $i$ there are three possibilities:

\paragraph{\underline{\textbf{Case 1:}}} $\rho_i(Q)\neq \{1\}$ but $\rho_i(F)=\{1\}$.

In this case $\rho_i$ has image a non-trivial finite cyclic group $L$.  We quickly recap the $\QQ$-representation theory of $\Z/n$ for $n\geq 2$.  The irreducible representations of $\Z/n$ are exactly the $1$-dimensional trivial representation and for each prime $p$ dividing $n$ the representation given by $\Z/n\twoheadrightarrow \Z/p\to \GL_{p-1}(\QQ)$ where $\Z/p$ acts faithfully on $\QQ^{p-1}$.  The latter representation may be constructed by taking $\QQ[\Z/p]$ and quotienting out by the $\Z/p$-invariant subspace $\langle \sum_{j=0}^{p-1} g^j \rangle$ where $g$ is any non-trivial element of $\Z/p$.  Note that in this case the characteristic polynomial of $\rho(g)$ is the cyclotomic polynomial $\sum_{j=0}^{p-1} t^j$.

Since $\rho_i$ is irreducible it follows that we are in the situation of a faithful representation of $\Z/p$ on $\QQ^{p-1}$ for some prime $p$.  Consider the tail end of the standard resolution for $\Z$ over $\Z G$
\[\begin{tikzcd}
C_1 \arrow[r,"\partial"] & C_0 \quad =\quad a_0\Z G\oplus\dots\oplus a_{m-1}\Z G \oplus t\Z G \arrow[r,"\partial"] & \Z G \end{tikzcd}\]
where $a_0,\dots,a_{m-1}$ is a generating set for $F$, where $t$ is the generator of $\Z$ viewing $G=F\rtimes\Z$, and where 
\begin{equation}\label{eqn.partial0}
\partial = \begin{bmatrix} 1-a_0,\dots,1-a_{m-1},1-t \end{bmatrix}.
\end{equation}

We need to compute the order of the presentation matrix 
\[
\partial\otimes_{\Z G}\id_{\QQ^{p-1}[t^{\pm1}]}=\begin{bmatrix} 0,\dots,0,\id-\rho_i(t)t \end{bmatrix}.
\]
But this is the same as computing an order of the square matrix $\id-\rho_i(t)t$.  
% Moreover, we may conjugate $\rho_i(t)$ to be a matrix of the following form
% \[\rho_i(t)=\begin{bmatrix}
%     0       & 0           & \cdots     & 0         & -1 \\
%     1       & 0           & \cdots     & 0         & -1 \\
%     0       & 1           & \cdots     & 0         & -1 \\
%     \vdots  & \vdots      & \ddots    & \vdots    &\vdots \\
%     0       & 0           & \cdots     &1          &-1
% \end{bmatrix} \]
% so
% \[\id_{p-1}-\rho_i(t)t=\begin{bmatrix}
%     1       & 0           & \cdots     & 0         & -t \\
%     -t       & 1           & \cdots     & 0         & -t \\
%     0       & -t           & \cdots     & 0         & -t \\
%     \vdots  & \vdots      & \ddots    & \vdots    &\vdots \\
%     0       & 0           & \cdots     &-t          &1-t
% \end{bmatrix} \]
Now,
\begin{equation}\label{eqn.ordrhot}
    \ord(\id-\rho_i(t)t)\doteq \det(\id t^{-1}-\rho_i(t)t\cdot t^{-1})t^{p-1} \doteq \det(\id t^{-1}-\rho_i(t))
\end{equation}
but this is exactly the characteristic polynomial of $\rho_i(t)$ with respect to $t^{-1}$.  Namely, it is the cyclotomic polynomial $\chi_p(t^{-1})\doteq \sum_{j=0}^{p-1}t^j$.  So we have that $\Delta_0^{\varphi,\rho_i}(t)\doteq\sum_{j=0}^{p-1}t^j$. \hfill$\blackdiamond$

\paragraph{\underline{\textbf{Case 2:}}} $\rho_i(F)\neq\{1\}$.

We start by again by viewing $G$ as $F\rtimes \Z$.  In particular, we have a differential $\partial$ as in \eqref{eqn.partial0} such that $\Delta_0^{\varphi,\rho_i}$ is given by an order of 
\[D_i\coloneqq \partial\otimes_{\Z G}\id_{\QQ^{k_i}[t^{\pm1}]}=[\id - \rho_i(a_0),\dots,\id-\rho_i(a_{m-1}),\id-\rho_i(t)].\]
To this end we define $D$ to be the set of cofactors of $D_i$.  So  $\Delta_0^{\varphi,\rho_i}\doteq \gcd D$.

We first conjugate $\rho_i$ so that $\rho_i(t)$ is in block diagonal form.  Since the image of $t$ is cyclic, say of order $n$, we obtain a block structure where the non-identity blocks are matrices corresponding to non-trivial $\QQ$-representations of various subgroups $H\leqslant \Z/n$.  Thus, arguing as in \eqref{eqn.ordrhot} we see that \[(1-t)^{n'}\cdot \prod_{j=1}^\ell\chi_{n_j}(t)\in D,\] where $n'$ is dimension of the fixed subspace of $\rho_i(t)$ and $\chi_{n_j}(t)$ is the cyclotomic polynomial of order ${n_j}$ such that $n_j$ divides $n$.

Now, $\Delta_0^{\varphi,\rho_i}$ divides every element of $D$ and is a polynomial defined over $\QQ[t]$ (up to multiplication by $t^\ell$ for some $\ell\geq 0$).  Now, $\chi_{n_j}(t)$ is the minimal polynomial for all primitive $n_j$th roots of unity.
% Now, \[\chi_{n_j}(t)=\prod_{1\leq q\leq n_j,\\ \gcd(q,n_j)=1} \left( t-e^{2i\pi\frac{q}{n_j}}\right).\]
In particular, any non-trivial polynomial dividing and not equal to $\chi_{n_j}(t)$ is not defined over $\QQ[t^{\pm1}]$.  It follows that $\Delta_0^{\varphi,\rho_i}=P_i(t)\cdot(1-t)^{n''}$ where $P_i(t)$ is a product of cyclotomic polynomials and $n''$ is a non-negative integer less than or equal to $k_i$. \hfill$\blackdiamond$

% In this case we have $(M_i)_G=0$ and $\Delta_0^{\varphi,\rho_i}\doteq 1$ by \Cref{DFJ Lemma}. 



\paragraph{\underline{\textbf{Case 3:}}} $\rho_i(G)=\{1\}$.

In this case we are computing $(\QQ[t^{\pm1}])_G$ where $G$ acts trivially on $\QQ$.  Clearly, this is isomorphic to $\QQ[t^{\pm1}]/(1-t)$ which is additively isomorphic to $\QQ$.\hfill $\blackdiamond$

By \Cref{AP direct sum formula} we have that $\Delta^{\varphi,\rho}_0(t)\doteq \prod_{i=1}^\ell \Delta^{\varphi,\rho_i}_0(t)\doteq (1-t)^nP(t)$ where $n$ is some non-negative integer and $P(t)$ is a product of cyclotomic polynomials.
\end{proof}

\begin{remark}
The previous lemma easily generalises to any field $\FF$ of characteristic  zero with the modified conclusion that $\Delta_0^{\varphi,\rho}(t)\doteq Q(t)P(t)$, where $Q(t)$ is a product of polynomials $(1-\zeta_i t)$ such that $\zeta_i$ is some root of unity in $\FF$, and where $P(t)$ is a product of cyclotomic polynomials whose roots do not lie in $\FF$.
\end{remark}

Let $R$ be a unique factorisation domain.  A polynomial $c(t)\in R[t^{\pm1}]$ is \emph{palindromic} if $c(t)=\sum_{i=0}^r c_it^i$ and $c_i=c_{r-i}$.  Given any polynomial $c(t)\in R[t^{\pm1}]$ where $c(t)=\sum_{i=0}^r c_it^i$ we write $c^\bigstar(t)$ for the polynomial $\sum_{i=0}^r c_{r-i}t^i$.  Note that $c(t)\cdot c^\bigstar(t)$ is palindromic.

The following lemma is a triviality

\begin{lemma}\label{AP -phi recip}
let $R$ be a UFD.  Let $G$ be a group, let $\varphi\colon G\twoheadrightarrow\ZZ$, and let $\rho\colon G\to\GL_k(R)$ be a representation. Then,
\[(\Delta^{\varphi,\rho}_{n})^\bigstar(t)\doteq\Delta^{-\varphi,\rho}_{n}(t)\doteq\Delta^{\varphi,\rho}_{n}(t^{-1})  \]
up to monomial factors with coefficients in $R^\times$.
\end{lemma}

The following lemma is well known to experts.  We include a proof for completeness.

\begin{lemma}\label{ordAP is Bno}
Let $\FF$ be a field.  Let $G$ be a group of type $\mathsf{FP}_n(\FF)$.  If $\varphi\colon G\to \Z$ is an $\mathsf{FP}_n(\FF)$-fibring, then 
\[
\deg \Delta_{G,n}^{\varphi,\mathbf{1}}(t)= b_n(\ker\varphi;\FF),
\]
where the Alexander polynomial is taken over $\FF$.
\end{lemma}
\begin{proof}
We may write $G$ as $\ker\varphi\rtimes\langle t\rangle$ and $\Delta_{G,n}^{\varphi,\mathbf{1}}(t)$ as the characteristic polynomial of the $\FF$-linear transformation $T_n\colon H_n(\ker\varphi;\FF)\to H_n(\ker\varphi;\FF)$ and $T^n\colon H^n(\ker\varphi;\FF)\to H^n(\ker\varphi;\FF)$, where $T$ is the induced map of $t$ on (co)homology.  Hence, \[H^1(\ker\varphi;\FF)\cong \FF[t^{\pm1}]/(\Delta_{G,n}^{\varphi,\mathbf{1}}(t)).\qedhere\]
\end{proof}

Recall that for a discrete group $G$, the \emph{geometric dimension} of $G$, denoted $\gd(G)$, is the minimal $n\in\NN\cup\{\infty\}$ such that $G$ admits an $n$-dimensional model for $K(G,1)$.

\begin{lemma}\label{top dim AP}
Let $R$ be a UFD.  Let $G$ be a group of type $\mathsf{F}$ with $\gd(G)=n$, let $\varphi\colon G\twoheadrightarrow\ZZ$, and let $\rho\colon G\to\GL_k(R)$ be a representation.  If $\Delta^{\varphi,\rho}_{n}\neq 0$ over $R$, then $\Delta^{\varphi,\rho}_{n}\doteq 1$.
\end{lemma}
\begin{proof}
Consider the head end of the cellular chain complex for $G$, namely,
\[\begin{tikzcd}
0 \arrow[r] & C_n \arrow[r,"\partial_{n-1}"] & C_{n-1} \arrow[r] & \cdots
\end{tikzcd}\]
tensoring with $R^k[t^{\pm1}]$ and taking homology we see that
$H_n(G;R^k[t^{\pm1}])=\ker \partial_{n-1}\otimes \id_{R^k[t^{\pm1}]}$.  In particular, it is a submodule of a free $R[t^{\pm1}]$-module and so cannot be $R[t^{\pm1}]$-torsion unless it is $0$.  But since $\Delta^{\varphi,\rho}_{n}\neq 0$ by assumption, we have that $H_n(G;R^k[t^{\pm1}])$ is $R[t^{\pm1}]$-torsion.  The result follows.
\end{proof}

We now wish to define \emph{the twisted Reidemeister torsion} $\tau^{\varphi,\rho}_{G,R}(t)$ of $\varphi$ twisted by $\rho$ over $R$.  Rather than give the original definition which we will not need, we instead use the following lemma which recasts the invariant in terms of twisted Alexander polynomials as our definition.  The lemma can be deduced by standard methods, for example, it is an immediate corollary of \cite[Lemma 2.1.1]{Turaev1986}.


\begin{lemma}\label{defn RT}
Let $R$ be a UFD.  Let $G$ be a group of type $\mathsf{F}$, let $\varphi\colon G\twoheadrightarrow\ZZ$ have kernel of type $\mathsf{F}$, and let $\rho\colon G\to\GL_k(R)$ be a representation.  Then,
\[\tau^{\varphi,\rho}_{G,R}(t)\doteq \prod_{n\geq 0}\left(\Delta_{G,n}^{\varphi,\rho}(t)\right)^{(-1)^{n+1}} \]
up to monomial factors with coefficients in $\mathrm{Frac}(R)^\times$.
\end{lemma}

This allows us to easily compute the Reidemeister torsion of free-by-cyclic groups.

\begin{prop}\label{defn RT FbyZ}
Let $R$ be a UFD.  Let $(G,\varphi)$ be a free-by-cyclic group and let $\rho\colon G\to\GL_k(R)$ be a representation.  Then,
\[\tau^{\varphi,\rho}_{G,R}(t)=\frac{\Delta_{G,1}^{\varphi,\rho}(t)}{\Delta_{G,0}^{\varphi,\rho}(t)} \]
up to monomial factors with coefficients in $\mathrm{Frac}(R)^\times$.
\end{prop}
\begin{proof}
This follows from \Cref{top dim AP} and \Cref{defn RT}.
\end{proof}

The final well known lemma is elementary.

\begin{lemma}\label{conjugacy invariance of RT}
Let $R$ be a UFD.  Let $G$ be a group of type $\mathsf{F}$ admitting a character $\varphi\colon G\twoheadrightarrow\Z$ which has kernel of type $\mathsf{F}$.  If $\rho_1$ and $\rho_2$ are conjugate representations of $G$ into $\GL_k(R)$, then $\tau^{\varphi,\rho_1}_{G,R}(t)\doteq\tau^{\varphi,\rho_2}_{G,R}(t)$.
\end{lemma}



\section{Regularity}\label{sec:regularity}
In this section we will introduce the definition of a $\widehat{\Z}$-regular isomorphism.  We will prove that in the case where $G$ has $b_1(G)=1$ every profinite isomorphism is $\widehat{\Z}$-regular and deduce some consequences.

\begin{defn}[Matrix coefficient modules]
Let $H_A$ and $H_B$ be a pair of finitely generated $\ZZ$-modules.  Let $\Theta \colon \widehat{H_A}\to\widehat{H_B}$ be a continuous homomorphism of the profinite completions.  We define the \emph{matrix coefficient module} 
\[\MC(\Theta;H_A,H_B)\]
 (or simply 
$\MC(\Theta)$ if there is no chance of confusion) for $\Theta$ with respect to $H_A$ and $H_B$ to be the smallest $\Z$-submodule $L$ of $\widehat{\Z}$ such that $\Theta(H_A)$ lies in the submodule $H_B\otimes_\Z L$ of $\widehat{H_B}$.  We denote by 
\[\Theta^\MC\colon H_A\to H_B\otimes_\ZZ\MC(\Theta)\]
 the homomorphism uniquely determined by the restriction of $\Theta$ to $H_A$.

For a finitely generated group $G$ let $G^{\mathrm fab}$ denote the free part of the abelianisation $G^{\mathrm ab}$.  That is, the quotient of the abelianisation of $G$ by its torsion elements.

Given groups $G_A$ and $G_B$ and a continuous homomorphism $\Theta\colon \widehat{G}_A\to\widehat{G}_B$ we have an induced continuous homorphisms $\Theta_\ast\colon\widehat{G}_A^{\mathrm{fab}}\to\widehat{G}_B^{\mathrm{fab}}$ and $\Theta^{*} \colon H^1(G_B, \Z) \to H^1(G_A, \Z)$.  We define $\MC(\Theta)\coloneqq\MC(\Theta_\ast,G_A^{\mathrm{fab}},G_B^{\mathrm{fab}})$.
\end{defn}

\begin{defn}[$\widehat \Z$-regular isomorphism] \label{defn:Zhatreg}
The isomorphism $\Theta \colon \widehat{G}_A \to \widehat{G}_B$ is \emph{$\widehat \Z$-regular}, if there exists a unit $\mu \in \widehat{\Z}^{\times}$ and 
an isomorphism $\Xi\colon G_A^{\mathrm{fab}} \to G_B^{\mathrm{fab}}$ such that $\Theta_{*}$ is the profinite completion of the map given by the composite
\begin{equation}\label{composite}
\begin{tikzcd}
G_A^{\mathrm{fab}} \arrow[r, "\Xi"] & G_B^{\mathrm{fab}} \arrow[r, "\cdot\times\mu"] &\widehat{G_B^{\mathrm{fab}}}.
\end{tikzcd}
\end{equation}
We sometimes write $\Theta_{*}^{1/\mu} \colon G_A^{\mathrm{fab}} \to G_B^{\mathrm{fab}}$ to denote the map $\Xi$ in \eqref{composite} and $\Theta^{*}_{1/\mu} \colon H^1(G_B, \Z) \to H^1(G_A, \Z)$ to denote its dual. 

For any $\varphi \in H^1(G_B; \Z)$ and $\psi \in H^1(G_A; \Z)$, we say $\psi$ is the \emph{pullback of $\varphi$ via $\Theta$}, if $\psi = \Theta^{*}_{1/\mu}(\varphi)$.  

Suppose $L_A$ is a finite index normal subgroup of $G_A$ and let $L_B$ be the corresponding normal subgroup of $G_B$ under $\Theta$.  If $\psi \in H^1(G_A; \Z)$ is the pullback of $\varphi$ via $\Theta$, then we say $\psi|_{L_A}$ is the pullback of $\varphi|_{L_B}$ via $\Theta|_{\widehat{L}_A}$.
\end{defn}

We say a pair $(G,\psi)$ is a $\calp$-by-$\Z$ group for some group property $\calp$ if $G$ admits an epimorphism $\psi\colon G\to \Z$ such that the kernel has property $\calp$.

\begin{prop}[$\widehat{\Z}$-regularity]\label{mu unit}
    Let $G_A$ and $G_B$ be $\{$type $\mathsf{FP_\infty}\}$-by-$\Z$ groups satisfying $b_1(G_A)=b_1(G_B)=1$.  If $\Theta\colon \widehat{G}_A\to\widehat{G}_B$ is an isomorphism, then there exists a unit $\mu\in\widehat\Z^\times$ such that $\MC(\Theta)=\mu\Z$.
\end{prop}
\begin{proof}
    By \cite[Proposition 3.2(1)]{Liu2023}, the $\Z$-module $\MC(\Theta)$ is a non-zero finitely generated free $\Z$-module spanned by the single entry of the $1\times1$ matrix $(\mu)$ over $\widehat{\Z}$. By \cite[Proposition 3.2(2)]{Liu2023} we obtain a homomorphism $\Xi \colon G_A^{\mathrm{fab}}\to G_B^{\mathrm{fab}}$ such that $\Psi_\ast=\mu\widehat{\Xi}$.  Moreover, $\mu$ is a unit because $\Theta$ is an isomorphism.  Hence, $\MC(\Theta_\ast)=\mu\Z$.
\end{proof}




\begin{prop}[Fibre closure isomorphisms]\label{fibre closure iso}
Let $(L_A,\psi)$ and $(L_B,\varphi)$ be $\{$type $\mathsf{FP}_\infty\}$-by-$\Z$ groups.   Suppose $\Theta\colon \widehat{L}_A\to\widehat{L}_B$ is an isomorphism and $\psi$ is the pullback of $\varphi$ via $\Theta$ with unit $\mu$.  If $F_A$ is the fibre subgroup of $L_A$, then $\overline{F}_A$ projects isomorphically onto $\overline{F}_B$, the closure of the fibre subgroup of $L_B$, under $\Theta$.
\end{prop}
\begin{proof}
There are two cases, the first is when $\Theta$ is a $\widehat{\Z}$-regular isomorphism, and the second is when $L_A$ and $L_B$ are finite index subgroups of groups $G_A$ and $G_B$ respectively such that there is $\widehat{\Z}$-regular isomorphism $\tilde\Theta\colon\widehat{G}_A\to\widehat{G}_B$ and $\psi$ is the pullback of $\varphi$ via $\tilde\Theta$.

We first prove the case where $\Theta$ is $\widehat{\Z}$-regular.  Our proof in this case essentially follows \cite[Corollary 6.2]{Liu2023}.  Write $L_A = F_A\rtimes_{\Psi} Z_A$ and $G_B = F_B\rtimes_{\Phi} Z_B$ with $Z_A\cong Z_B\cong \Z$.  Identify, $H_A$ with $G_A^{\mathrm fab}$ and $H_B$ with $G_B^{\mathrm fab}$.  By hypothesis the map $\Theta_\ast$ is the completion of an isomorphism $\Theta_\mu\colon H_A\to H_B$ followed by multiplication by $\mu$ in $\widehat{H}_B=H_B\otimes_Z \widehat{\Z}$.  Thus, $\psi$ is the composite
\[\begin{tikzcd}
H_A \arrow[r,"\Theta_\mu\otimes\mu"] & H_B\otimes_\Z\mu\Z \arrow[r, "1\otimes \mu^{-1}"] & H_B\otimes_\Z \Z \arrow[r, "="] & H_B \arrow[r, "\varphi|_{Z_B}"] & \Z.
\end{tikzcd}\]
We obtain that $\Theta_\ast(\ker\psi_\ast)=\mu F_\mu (\ker\varphi_\ast)=\mu\ker(\varphi_\ast)$ in $\widehat{H}_B$.  Since $\ker\varphi_\ast$ is a $\Z$-submodule of $H_B$, the closure of $\widehat{H}_B$ is invariant under multiplication by a unit.  Hence, $\Theta_\ast\overline{\ker\psi_\ast} = \overline{\mu \ker\varphi_\ast}= \mu \overline{\ker\varphi_\ast}=\overline{\ker\varphi_\ast}$.   This completes the proof of the first case.

We now prove the second case.  We may assume $G_A$ is a finite index overgroup of $H_A$ admitting a finite quotient $\alpha$ such that $\ker\alpha=H_A$.  Note that $\overline{F}_A$ is equal to the intersection of a finite index normal subgroup $\ker\widehat\alpha$ with $\ker\widehat{\tilde{\psi}}$ in $\widehat{G}_A$, where $\tilde\psi$ is the lift of $\psi$ to $G_A$.  Similarly, $\overline{F}_B=\ker\widehat\alpha\cap\ker\widehat{\tilde{\varphi}}$.  The result now follows from the $\widehat{\Z}$-regular case applied to $\tilde\Theta\colon G_A\to G_B$.
\end{proof}

This following lemma is a special case of \cite[Corollary~2.9]{Lorensen2008}.

\begin{lemma}\label{Serre good}
Let $G$ be a free-by-cyclic group.  Then $G$ is cohomologically good.
\end{lemma}


\begin{prop}[Isomorphism of fibre subgroups]\label{fibre iso}
Let $(G_A,\psi)$ and $(G_B,\varphi)$ be free-by-cyclic groups.  Suppose $\Theta\colon \widehat{G}_A\to\widehat{G}_B$ is an isomorphism.  If $\psi$ is the pullback of $\varphi$ via $\Theta$,  then the fibre subgroup $F_A$ of $G_A$ and the fibre subgroup $F_B$ of $G_B$ are isomorphic.
\end{prop}
\begin{proof}
We will show that the degree of the first Alexander polynomials of $G_A$ and $G_B$ are equal.  By \Cref{ordAP is Bno} this computes the rank of the $\FF_p$-homology of $F_A$ and $F_B$ which determines their rank.  Since $F_A$ and $F_B$ are free groups this determines them up to isomorphism.  

Let $\psi_n$ and $\varphi_n$ denote the modulo $n$ reduction of $\psi\colon G_A\twoheadrightarrow\Z$ and $\varphi\colon G_B\twoheadrightarrow \Z$ respectively, namely the composites
\[\begin{tikzcd}
G_A\arrow[r, two heads, "\psi"] & \Z \arrow[r, two heads] & \Z/n \quad \quad \text{and} \quad \quad G_B\arrow[r, two heads, "\varphi"] & \Z \arrow[r, two heads] & \Z/n.
\end{tikzcd}\]
We endow $M_{A,n}\coloneqq\FF_p[\Z/n]$ with the $G_A$-module structure given by $\psi_n$ and $M_{B,n}\coloneqq \FF_p[\Z/n]$ with the $G_B$-module given by $\varphi_n$.  Since $G_A$ and $G_B$ are cohomologically good (\Cref{Serre good}), by \cite[Proposition~4.2]{BoileauFriedl2020} we have isomorphisms
$H_k(G_A;M_{A,n})\cong H_k(G_B;M_{B,n})$
for all $k,n\geq 0$.  In particular, $\dim_{\FF_p} H_k(G_A;M_{A,n}) = \dim_{\FF_p} H_k(G_B;M_{B,n})$.  Now, by applying \cite[Proposition~3.4]{BoileauFriedl2020} twice we get
\begin{align*}
    \deg\Delta_{G_A,1}^{\psi,\mathbf{1}}(t) &= \max_{n\in\NN}\left\{\dim_{\FF_p}H_1(G_A;M_{A,n}) - \dim_{\FF_p}H_0(G_A;M_{A,n}), \right\}\\
    &=\max_{n\in\NN}\left\{\dim_{\FF_p}H_1(G_B;M_{B,n}) - \dim_{\FF_p}H_0(G_B;M_{B,n}), \right\}\\
    &=\deg\Delta_{G_B,1}^{\varphi,\mathbf{1}}(t).\qedhere
\end{align*}
\end{proof}




\section{Profinite invariance of twisted Reidemeister torsion}\label{sec:Rtorsion}
The goal of this section is to establish profinite invariance of twisted Reidemeister torsion (\Cref{profinite R Torsion}) for free-by-cyclic groups with first Betti number equal to one.  We do this by first establishing invariance of the twisted Alexander polynomials in a more general setting.  Finally, in \Cref{sec.homostretch} we establish profinite invariance of homological stretch factors.

\subsection{Twisted Alexander polynomials}\label{sec:AP.profinite}
\begin{defn}[Corresponding quotients]\label{corresponding quotients}
Let $G_A$ and $G_B$ be residually finite groups. Suppose there exists an isomorphism $\Theta \colon \widehat{G}_A \to \widehat{G}_B$. Let $Q$ be a finite group. A pair of quotients $\gamma_A \colon G_A \twoheadrightarrow Q$ and $\gamma_B\colon G_B \to Q$ is said to be \emph{$\Theta$-corresponding}, if $\gamma_A$ is given by the composite
\begin{equation}
    \begin{tikzcd}
        G_A \arrow[r, "i"] & \widehat{G}_A \arrow[r, "\Theta"] & \widehat{G}_B \arrow[r, "\widehat{\gamma}_B"] & Q
    \end{tikzcd}
\end{equation}
Here, $i \colon G_A \to \widehat{G}_A$ denotes the natural inclusion and $\widehat{\gamma_B}$ denotes the (profinite) completion of $\gamma_B$.
\end{defn}

\begin{prop}[Profinite invariance of twisted Alexander polynomials]\label{profinite AP}
Let $(G_A,\psi_A)$ and $(G_B,\varphi_B)$ be residually finite $\{$good type $\mathsf{F}\}$-by-$\Z$ groups. Let $\Theta\colon\widehat{G}_A\to\widehat{G}_B$ be an isomorphism and suppose $\psi_A$ is the pullback of $\varphi_B$ via $\Theta$ with unit $\mu$. Let $\psi_B \in H^1(G_B, \Z)$ be a primitive fibred class. Let $\psi_A \in H^1(G_A, \Z)$ be the fibred class $\Theta^{\ast}_{\mu}(\psi_B)$. Fix a $\Theta$-corresponding pair of finite quotients  $\gamma_A \colon G_A \to Q$ and $\gamma_B \colon G_B \to Q$. Suppose $\rho \colon Q \to \mathrm{GL}(k, \mathbb{Q})$ is a representation and $\rho_A \colon G_A \to \mathrm{GL}(k, \mathbb{Q})$ and $\rho_B \colon G_B \to \mathrm{GL}(k, \mathbb{Q})$ the pullbacks. Then,
\[\Delta_{G_A,n}^{\psi_A,\rho_A}(t)\cdot\Delta_{G_A,n}^{\psi_A,\rho_A}(t^{-1})\doteq\Delta_{G_B,n}^{\varphi_B,\rho_B}(t)\cdot\Delta_{G_B,n}^{\varphi_B,\rho_B}(t^{-1}) \]
holds in $\QQ[t^{\pm1}]$ up to monomial factors with coefficients in $\QQ^\times$.
\end{prop}

Before proving \Cref{profinite AP} we will collect a number of facts.

The following criterion is due to Ueki \cite[Lemma~3.6]{Ueki2018}.

\begin{thm}[Ueki]\label{Ueki criterion}
Let $a(t),b(t)\in \Z[t]$ be a be a pair of palindromic polynomials and $\mu\in\widehat{\Z}$ be a unit.  If the principal ideals $(a(t^\mu))$ and $(b(t))$ of the completed group algebra $\cgrZ$ are equal, then $a(t)\doteq b(t)$ holds in $\Z[t^{\pm1}]$.
\end{thm}

\begin{defn}[$\mu$-powers]
Let $G$ be a profinite group, let $g\in G$, and let $\mu\in\widehat{\Z}$.  We define the  \emph{$\mu$-power} of $g$ to be $g^\mu=\varprojlim_N g^n\mod N$ where $N$ ranges over the inverse system of open normal subgroups of $G$ and $n\in\Z$ is congruent to $\mu$ modulo $|G/N|$.  Note that $hg^\nu h^{-1}=(hgh^{-1})^\mu$ for all $h\in G$.
\end{defn}

\begin{lemma}\emph{\cite[Lemma~7.6]{Liu2023}}\label{integralising finite rep}
Let $L$ be a finite group.  If $\rho\colon L\to \GL_k(\QQ)$ is a representation, then $\rho$ is conjugate to the induced representation $\sigma_\QQ$ over $\QQ$ of some representation $\sigma\colon L\to \GL_k(\Z).$
\end{lemma}


\begin{remark}\label{reduction to integral reps}
Combining \Cref{integralising finite rep} and \Cref{AP conjugacy} we may assume without loss of generality that the representation $\gamma$ is equal to the induction of some integral representation $\sigma\colon L\to\GL_k(\Z)$.  We denote by $\sigma_A\colon G_A\to\GL_k(\Z)$ the pullback $\gamma_A^\ast(\sigma)$ and similarly write $\sigma_B$ for $\gamma_B^\ast(\sigma)$.
\end{remark}

By \Cref{fibre closure iso} and \Cref{fully separable fibres} we have a commutative diagram with exact rows
\begin{equation}\label{eqn BIG diagram}
\begin{tikzcd}
1 \arrow[r, no head, tail] & F_A \arrow[r, tail] \arrow[d, tail]               & G_A \arrow[r, "\psi_A", two heads] \arrow[d, tail]                       & \Z \arrow[d, tail] \arrow[r, two heads] & 1 \\
1 \arrow[r, tail]          & \widehat{F}_A \arrow[r, tail] \arrow[d, "\Theta_F"] & \widehat{G}_A \arrow[r, "\widehat{\psi}_A", two heads] \arrow[d, "\Theta"] & \widehat{\Z} \arrow[d, "\mu"] \arrow[r] & 1 \\
1 \arrow[r, tail]          & \widehat{F}_B \arrow[r, tail]                     & \widehat{G}_B \arrow[r, "\widehat{\varphi}_B", two heads]                & \widehat{\Z} \arrow[r]                  & 1 \\
1 \arrow[r, tail]          & F_B \arrow[u, tail] \arrow[r, tail]               & G_B \arrow[r, "\varphi_B", two heads] \arrow[u, tail]                    & \Z \arrow[u, tail] \arrow[r]            & 1,
\end{tikzcd}
\end{equation}
where $\Theta_F=\Theta|_{\overline{F}_A}$ and $\Theta_F$, $\Theta$, and $\mu$ are isomorphisms.

We now write $G_A=F_A\rtimes\langle t_A\rangle$ with $\psi_A(t_A)=1$ and $G_B=F_B\rtimes \langle t_B\rangle$ with $\varphi_B(t_B)=1$.  Now \eqref{eqn BIG diagram} is equivalent to the fact that $\Theta(t_A)$ is conjugate to the $\mu$-power $t^\mu_B$ of $t_B$ in $\widehat{G}_B$.  Let $M_A$ be $\Z^k$ equipped with the $F_A$-module structure given by $\sigma_A|_{F_A}$ and similarly for $M_B$.  Note that $\psi_A$ and $\varphi_B$ induce automorphisms $\Psi_A$ of $F_A$ and $\Phi_B$ of $F_B$ respectively.  Moreover, $\Psi_A$ induces a $\Z$-linear isomorphism $\psi_{A,n}\colon H_n(F_A;M_A)\to H_n(F_A;M_A)$.  We note that the choices made here for picking group automorphisms $\Psi_A$ and $\Phi_B$ only depend on the choices up to conjugacy in $\Aut(F_A)$ and $\Aut(F_B)$.  It follows that $\psi_{A,n}$ only depends on $\sigma$ and $\psi_A$.  We obtain a commutative diagram of $\Z$-modules with exact rows
\begin{equation}\label{eqn Z module splitting}
    \begin{tikzcd}
0 \arrow[r, tail] & H_n(F_A;M_A)_{\mathrm{tors}} \arrow[d, "{\psi_{A,n}^{\mathrm{tors}}}"] \arrow[r, tail] & H_n(F_A;M_A) \arrow[d, "{\psi_{A,n}}"] \arrow[r, two heads] & H_n(F_A;M_A)_{\mathrm{free}} \arrow[d, "{\psi_{A,n}^{\mathrm{free}}}"] \arrow[r, two heads] & 0 \\
0 \arrow[r, tail] & H_n(F_A;M_A)_{\mathrm{tors}} \arrow[r, tail]                                           & H_n(F_A;M_A) \arrow[r, two heads]                           & H_n(F_A;M_A)_{\mathrm{free}} \arrow[r, two heads]                                           & 0.
\end{tikzcd}
\end{equation}
Note that after fixing bases we may consider $\psi_{A,n}^{\mathrm{free}}$ as a matrix in $\GL(H_n(F_A;M_A)_{\mathrm{free}})$.  Define
\begin{equation}\label{eqn PAn}
    P_{A,n}(t)\coloneqq \det_{\Z[t^{\pm1}]}\left(\mathbf{1}-t\cdot \psi_{A,n}^{\mathrm{free}}\right)
\end{equation}
and
\begin{equation}\label{eqn PBn}
    P_{B,n}(t)\coloneqq \det_{\Z[t^{\pm1}]}\left(\mathbf{1}-t\cdot \varphi_{B,n}^{\mathrm{free}}\right).
\end{equation}

The following lemma is \cite[Lemma~7.7]{Liu2023}.  The proof goes through verbatim once one assumes the kernels of $\psi_A$ and $\varphi_B$ are type $\mathsf{F}$.

\begin{lemma}\emph{\cite[Lemma~7.7]{Liu2023}}\label{Liu7.7}
Adopt the notation from \Cref{profinite AP}, \Cref{reduction to integral reps}, \eqref{eqn PAn}, and \eqref{eqn PBn}. We have $\Delta^{\psi_A,\rho_B}_{G_A,n}(t)\doteq P_{A,n}(t)$ and $\Delta^{\varphi_B,\rho_B}_{G_B,n}(t)\doteq P_{B,n}(t)$ in $\QQ[t^{\pm1}]$ up to monomials with coefficients in $\QQ^\times$.
\end{lemma}

The following lemma is \cite[Lemma~7.8]{Liu2023}.  The proof goes through verbatim once one assumes that the kernels of $\psi_A$ and $\varphi_B$ are type $\mathsf{F}$, that $F_A$ and $F_B$ are fully separable in $G_A$ and $G_B$ respectively (this is given by \Cref{fully separable fibres}), and that $F_A$ and $F_B$ are good.

\begin{lemma}\emph{\cite[Lemma~7.8]{Liu2023}}\label{Liu7.8}
Adopt the notation from \Cref{profinite AP}, \Cref{reduction to integral reps}, \eqref{eqn PAn}, and \eqref{eqn PBn}. For all $n$ we have an equality of principal ideals $(P_{A,n}(t^\mu))=(P_{B,n}(t))$ in $\cgrZ$.
\end{lemma}


\begin{proof}[Proof of \Cref{profinite AP}]
This follows from \Cref{Liu7.7,Liu7.8} and \Cref{Ueki criterion} after observing that the polynomials $\Delta_{G_A,n}^{\psi_A,\rho_A}(t)\cdot\Delta_{G_A,n}^{\psi_A,\rho_A}(t^{-1})$ and $\Delta_{G_B,n}^{\varphi_A,\rho_B}(t)\cdot\Delta_{G_B,n}^{\varphi_B,\rho_B}(t^{-1})$ are palindromic by \Cref{AP -phi recip}.
\end{proof}

\subsection{Twisted Reidemeister torsion}\label{sec:profRtorsion}
We now prove profinite invariance of twisted Reidemeister torsion for free-by-cyclic groups with first Betti number equal to one.

\begin{corollary}[Profinite invariance of twisted Reidemeister torsion]\label{profinite R Torsion}
Let $(G_A,\psi_A)$ and $(G_B,\varphi_B)$ be free-by-cyclic groups. Let $\Theta\colon\widehat{G}_A\to\widehat{G}_B$ be an isomorphism. Let $\varphi_B \in H^1(G_B; \Z)$ be a primitive fibred class and suppose $\psi_A$ is the pullback of $\varphi_B$ via $\Theta$.   Fix a $\Theta$-corresponding pair of finite quotients  $\gamma_A \colon G_A \to Q$ and $\gamma_B \colon G_B \to Q$. Suppose $\rho \colon Q \to \GL(k, \QQ)$ is a representation and $\rho_A \colon G_A \to \GL(k, \QQ)$ and $\rho_B \colon G_B \to \GL(k, \QQ)$ the pullbacks.  Then,
\[\{\tau^{\psi_A,\rho_A}_{G_A}(t),\tau^{-\psi_A,\rho_A}_{G_B} \}=\{\tau^{\varphi_B,\rho_B}_{G_B}(t),\tau^{-\varphi_B,\rho_B}_{G_B} \}. \]
\end{corollary}
\begin{proof}
By \Cref{profinite AP}, unique factorisation in $\QQ[t^{\pm1}]$, and \Cref{AP -phi recip} we obtain
\[S_{A,n}=\{\Delta_{G_A,n}^{\psi_A,\rho_A}(t),\Delta_{G_A,n}^{-\psi_A,\rho_A}(t)\}=\{\Delta_{G_B,n}^{\varphi_A,\rho_B}(t), \Delta_{G_B,n}^{-\varphi_B,\rho_B}(t)\}=S_{B,n}. \]
By \Cref{defn RT FbyZ} the relevant Alexander polynomials are concentrated in degree $0$ and $1$.  By \Cref{zeroth AP palindromic} the sets $S_{A,0}$ and $S_{B,0}$ contain exactly one element up to $\doteq$-equivalence.  Finally, the result follows from \Cref{defn RT FbyZ}.
\end{proof}

\subsection{Profinite invariance of homological stretch factors}\label{sec.homostretch}

\begin{thm}[Profinite invariance of homological stretch factors]\label{thm:homostretch}
Let $(G_A,\psi)$ and $(G_B,\varphi)$ be free-by-cyclic groups.  If $\Theta\colon\widehat{G}_A\to\widehat{G}_B$ is an isomorphism and $\psi$ is the pullback of $\varphi$ via $\Theta$, then $\{\nu_{\psi}^+,\nu_{\psi}^-\}=\{\nu_{\varphi}^+,\nu_{\varphi}^-\}$.
\end{thm}
\begin{proof}
Denote the non-trivial primitive characters of $G_A$ by $\psi_A^{\pm}$ and the non-trivial primitive characters of $G_B$ by $\varphi_B^{\pm}$.  By \Cref{profinite AP} we have \[\Delta_{G_A,1}^{\psi_A^+,\mathbf{1}}(t)\cdot\Delta_{G_A,1}^{\psi_A^-,\mathbf{1}}(t)\doteq\Delta_{G_B,1}^{\varphi_A^+,\mathbf{1}}(t)\cdot\Delta_{G_B,1}^{\varphi_B^-, \mathbf{1}}(t)\] over $\QQ[t^{\pm1}]$.  Normalise the polynomials so that every term is a non-negative power of $t$ and the lowest term is $1$, and note that each of the four terms has the same degree.  Now, by unique factorisation in $\QQ[t^{\pm1}]$ we obtain the equality of sets \[S_A=\{\Delta_{G_A,1}^{\psi_A^+,\mathbf{1}}(t),\Delta_{G_A,1}^{\psi_A^-,\mathbf{1}}(t)\}=\{\Delta_{G_B,1}^{\varphi_A^+,\mathbf{1}}(t),\Delta_{G_B,1}^{\varphi_B^-, \mathbf{1}}(t)\}=S_B.\]
Now, since we are working over $\QQ$ the set $S_A$ [resp. $S_B$] is the set of characteristic polynomials for $(\psi^{\pm}_{A})_1$ [resp. $(\varphi_B^{\pm})_1$], that is, the set of characteristic polynomials for the induced maps on degree $1$ homology of the respective fibres.  In particular, the sets \[\{\nu_{\psi}^+,\nu_{\psi}^-\}\text{ and }\{\nu_{\varphi}^+,\nu_{\varphi}^-\}\]
can be computed by taking the modulus of the largest root of the Alexander polynomials in $S_A$ and $S_B$.  The desired equality follows.
\end{proof}


\section{Profinite invariance of Nielsen numbers}\label{sec:profNielson}

Let $X$ be a connected, compact topological space that is homeomorphic to a finite-dimensional cellular complex, with a finite number of cells in each dimension, and let $f \colon X \to X$ be a self-map. Recall from \cref{sec:Nielson} the definitions of the fixed point index $\mathrm{ind}_m(f; \mathcal{O})$ of $f^m$ at any point $p \in \mathcal{O}$, and the $m$-th Nielsen number $N_m(f)$ of $f$.

We will write $M_f$ to denote the mapping torus \[M_f = \frac{X\times [0,1]}{(f(x), 0) \sim (x, 1)}.\] Let $x_0 \in X$ and fix a path $\alpha$ from $f(x_0)$ to $x_0$ in $X$. We identify $X$ with the fibre $X \times \{0\}$ in $M_f$ and write $\bar{x}_0$ denote the image of $x_0$ in $M_f$. We define $t \in \pi_1(M_f, \bar{x}_0)$ to be the loop obtained by concatenation of paths $\eta \cdot \alpha$, where 
$\eta_s = (x_0, s)$ for $s \in [0,1]$. The \emph{induced character} $\varphi \colon \pi_1(M_f) \to \Z$ maps every loop in $X$ based at $x_0$ to zero, and $\varphi(t) = 1$. 

Let $\zeta \colon \pi_1(M_f) \to \mathbb{Q}$ be any map that is constant on conjugacy classes. Then the \emph{$m$-th twisted Lefschetz number} of $f$ with respect to $\zeta$ is
\begin{equation}\label{eq:1} L_m(f; \zeta) = \sum_{\mathcal{O} \in \mathrm{Orb}_m(f)} \zeta(\mathrm{cd}(\mathcal{O}))\cdot \mathrm{ind}_m(f; \mathcal{O}).\end{equation}

For a finite-dimensional representation $\rho \colon \pi_1(M_f) \to \mathrm{GL}(k, R)$ of $\pi_1(M_f)$, let $\chi_{\rho} \colon \pi_1(M_f) \to R$ denote the trace map. We write $\mathrm{exp}(\cdot)$ to denote the formal power series,
\[\mathrm{exp}(x) = \sum_{k=0}^{\infty}\frac{x^k}{k!}.\]

\begin{thm}[{\cite{Jiang1996}, \cite[Lemma~8.2]{Liu2023}}]\label{Jiang} Let $\varphi \colon \pi_1(M_f) \to \Z$ denote the induced character. Suppose that $\mathbb{F}$ is a commutative field of characteristic 0 and $\rho \colon \pi_1(M_f) \to \mathrm{GL}(k, \mathbb{F})$ a finite-dimensional linear representation of $\pi_1(M_f)$. Then
\[\tau^{\rho, \varphi}_{\pi_1(M_f), \mathbb{F}[t^{\pm 1}]^k} \doteq \mathrm{exp} \sum_{m\geq 1} L_m(f; \chi_{\rho}) \frac{t^m}{m},\]
where the equality holds as rational functions in $t$ over $\mathbb{F}$, up to multiplication by monomial factors with coefficients in $\mathbb{F}^{\times}$.
\end{thm}

Let $Q$ be a finite group. %Given $\nu \in \widehat{\mathbb{Z}}^{\times}$ we define $g^{\nu}$ to be $g^n$, where $n$ is congruent to $\nu$ modulo $|Q|$. We say two elements $g_1$ and $g_2$ of $Q$ are \emph{$\widehat{\mathbb{Z}}$-conjugate} if there exists some $\nu \in \widehat{\mathbb{Z}}$ such that $g_1^{\nu}$ and $g_2$ are conjugate in $Q$. 
We say two elements $g_1$ and $g_2$ in $Q$ are \emph{$\widehat{\mathbb{Z}}$-conjugate} if the cyclic groups $\langle g_1 \rangle$ and $\langle g_2 \rangle$ are conjugate in $Q$ (note that this is equivalent to the notion of $\widehat{\Z}$-conjugacy defined in \cite{Liu2023}).
This gives rise to an equivalence relation on the set $\mathrm{Orb}(Q)$ of conjugacy classes of $Q$. We write $\Omega(Q)$ to denote the resulting set of equivalence classes. For $\omega \in \Omega(Q)$, we let $\chi_{\omega} \colon \mathrm{Orb}(Q) \to \mathbb{Q}$ denote the characteristic function of $\omega$.

\begin{lemma}[{\cite[Lemma~8.5]{Liu2023}}]\label{Nielsen_counting} Fix $m \in \mathbb{N}$. Let $\gamma \colon \pi_1(M_f) \to Q$ be a quotient of $\pi_1(M_f)$ onto a finite group $Q$. Then,
\[N_m(f) \geq \# \{\omega \in \Omega(Q) \mid L_m(f; \gamma^{*}\chi_{\omega}) \neq 0\}.\] 
\end{lemma}

Note that by \eqref{eq:1}, for every $\omega \in \Omega(Q)$ such that $L_m(f, \gamma^{*}\chi_{\omega}) \neq 0$, there exists some $\mathcal{O} \in \mathrm{Orb}_m(f)$ such that $\mathrm{ind}_m(f, \mathcal{O}) \neq 0$ and \[ \begin{split}\gamma^{*}\chi_{\omega}(\mathrm{cd}(\mathcal{O})) &= \chi_{\omega}  \circ \gamma(\mathrm{cd}(\mathcal{O}))  \\
&\neq 0,
\end{split}
\]
which holds if and only if $\gamma(\mathrm{cd}(\mathcal{O})) \in \omega$. Hence the number of such elements in $\Omega(Q)$ is bounded above by the number of essential $m$-periodic orbits of $f$, which is exactly $N_m(f)$.

The following lemma is a strengthening of Lemma~8.6 in \cite{Liu2023}, however the proof follows from Liu's proof with only a slight modification. We provide a sketch for the convenience of the reader.

\begin{lemma}\label{Nielsen_equiv} Suppose that $\pi_1(M_f)$ is conjugacy separable. Then, for any $m \in \mathbb{N}$ there exists a finite quotient $Q_m$ of $\pi_1(M_f)$ such that 
\begin{equation} \label{Nielsen_equalities}
\begin{split}  N_m(f) &= \{\omega \in \Omega(Q_m) \mid L_m(f; \gamma^{*}\chi_{\omega}) \neq 0\}, \text{ and }\\ N_m(f^{-1}) &= \{\omega \in \Omega(Q_m) \mid L_m(f^{-1}; \gamma^{*}\chi_{\omega}) \neq 0\}.\end{split}
\end{equation}
\end{lemma}

\begin{proof} 
Let $G = \pi_1(M_f)$ and write $\varphi \colon G \to \Z$ to denote the induced character, $t \in G$ the stable letter and $K = \mathrm{ker}\varphi$ the fibre subgroup as before. Since $G$ is conjugacy separable, for each $m \geq 1$ there exists a finite quotient $\tilde{\pi}_m \colon G \to \tilde{Q}_m$, such that for all $m$-periodic orbits of $f$ and $f^{-1}$, the corresponding distinct conjugacy classes in $G$ are mapped to distinct conjugacy classes in $\tilde{Q}_m$. 

By the discussion directly following the statement of \cref{Nielsen_counting}, the inequality provided by \cref{Nielsen_counting} is achieved when the conjugacy classes corresponding to the essential $m$-periodic orbits of $f$ are mapped to distinct $\widehat{\Z}$-conjugacy classes in the finite quotient. Hence, it suffices to find a finite quotient $\pi_m \colon G \to Q_m$ such that $\tilde{\pi}_m$ factors through $\pi_m$, and which satisfies the following property. If $x_1$ and $x_2$ are two elements of $G$ which correspond to $m$-periodic orbits of $f$, or of $f^{-1}$, and if $\langle \pi_m(x_1) \rangle$ and $\langle \pi_m(x_2) \rangle$ are conjugate in $Q_m$, then in fact the elements $\pi_m(x_1)$ and $ \pi_m(x_2)$ are  conjugate in $Q_m$. This will then imply that $\tilde{\pi}_m(x_1)$ and $ \tilde{\pi}_m(x_2)$ are conjugate in $\tilde{Q}_m$, since $\tilde{\pi}_m$ factors through $\pi_m$. Hence $x_1$ and $x_2$ are conjugate in $G$, showing that the required property holds for $\pi_m$.

To construct $Q_m$ note that the $m$-periodic orbits of $f$ correspond to elements in the coset $Kt^m$ of $G$, and the $m$-periodic orbits of $f^{-1}$ to the elements in the coset $Kt^{-m}$. If $\widebar{K}$ and $\bar{t}$ are the images of $K$ and $t$ in a finite quotient of $G$, then the coset $\widebar{K}\bar{t}^m$ is invariant under conjugation by elements in the quotient group. Hence, it suffices to find $Q_m$ such that that the cyclic subgroups generated by $\bar{x}_1$ and $\bar{x}_2$, for any $x_1, x_2 \in Kt^m$, intersect $\widebar{K}\bar{t}^m$ exactly at $\bar{x}_1$ and $\bar{x}_2$, respectively. It will then follow that if $\langle\bar{x}_1 \rangle$ and $\langle \bar{x}_2 \rangle$ are conjugate, then $\bar{x}_1$ and $\bar{x}_2$ are conjugate. The details of this construction are spelled out in the proof of Lemma~8.6 in \cite{Liu2023}. \end{proof}

We will also need the following proposition from representation theory of finite groups (see e.g. \cite[Section~12.4]{Serre1977}). We refer the reader to \cite[Lemma~8.4]{Liu2023} for the proof of statement rephrased in the language of $\widehat{\Z}$-conjugacy classes.

\begin{prop}\label{basis}
Let $K$ be a finite group. The set of irreducible finite-dimensional characters of $K$ over $\mathbb{Q}$ forms a basis for the space of maps $\mathrm{Orb}(K) \to \mathbb{Q}$ which are constant on $\widehat{\Z}$-conjugacy classes of $K$.
\end{prop}

Let $X_A$ and $X_B$ be topological spaces as before, with self-maps $f_A \colon X_A \to X_A$ and $f_B \colon X_B \to X_B$. We write $G_A = \pi_1(M_{f_A})$ and $G_B = \pi_1(M_{f_B})$, and let $\psi_A \colon G_A \to \Z$ and $\varphi_B \colon G_B \to \Z$ be the induced characters. 

\begin{lemma}\label{Nielsen number invariance}
    Suppose that $G_A$ and $G_B$ are conjugacy separable. Let $\Theta \colon \widehat{G}_A \to \widehat{G}_B$ be an isomorphism such that for any finite group $Q$, a $\Theta$-corresponding pair of quotients $\gamma_B \colon G_B \twoheadrightarrow Q$ and $\gamma_A \colon G_A \twoheadrightarrow Q$ (see \Cref{corresponding quotients}), and a representation $\rho \colon Q \to \mathrm{GL}(k, \mathbb{Q})$, we have
    \[\{\tau^{\psi_A,\rho \gamma_A}_{G_A},\tau^{-\psi_A,\rho \gamma_A}_{G_B} \}=\{\tau^{\varphi_B,\rho\gamma_B}_{G_B},\tau^{-\varphi_B,\rho \gamma_B}_{G_B} \}. \] 
    Then, for every $m \in \mathbb{N}$,
    \[ \{N_m(f_A) , N_m(f_A^{-1}) \} = \{ N_m(f_B), N_m(f_B^{-1})\}.\]
\end{lemma}

\begin{proof}
Let $m \in \mathbb{N}$. Invoke \Cref{Nielsen_equiv} to obtain a finite quotient $\gamma_B \colon G_B \to Q_m$ such that 
\[N_m(f_B^{\pm}) = \#\{\omega \in \Omega(Q_m) \mid L_m(f_B^{\pm}; \gamma_B^{*}\chi_{\omega}) \neq 0\}.\]
By \Cref{basis}, for every $\omega \in \Omega(Q_m)$, $\chi_{\omega}$ can be expressed uniquely as a $\mathbb{Q}$-linear combination $\chi_{\omega} = \sum_i \lambda_i \chi_{\rho_i}$, where each $\rho_i \colon Q_m \to \mathrm{GL}(k_i, \mathbb{Q})$ is an irreducible representation, and $\lambda_i \in \mathbb{Q}$. Hence 
\[L_m (f_B; \gamma_B^{\ast}\chi_{\omega}) = \sum_i \lambda_i L_m(f_B; \gamma_B^{*}\chi_{\rho_i}).\]
Let $\gamma_A$ be the map obtained by composing 
\[ G_A \xrightarrow{\iota} \widehat{G}_A  \xrightarrow{\widehat{\gamma_B}} Q,\]
where $\iota \colon G_A \to \widehat{G}_A$ is the natural inclusion. In particular, $\gamma_A$ and $\gamma_B$ are $\Theta$-corresponding, and thus by our assumption, for every representation $\rho_i \colon Q_m \to \mathrm{GL}(k_i, \mathbb{Q})$ we have that
    \[\{\tau^{\psi_A,\rho_i \gamma_A}_{G_A},\tau^{-\psi_A,\rho_i \gamma_A}_{G_A} \}=\{\tau^{\varphi_B,\rho_i\gamma_B}_{G_B},\tau^{-\varphi_B,\rho_i \gamma_B}_{G_B} \}. \] 
\smallskip

By \cref{Jiang} it follows that, up to multiplication by monomials in $t$,
\[\tau^{\psi_A,\rho_i \gamma_A}_{G_A} (t) \doteq  1 + L_1(f_A; \gamma_A^{*}\chi_{\omega})t + \sum_{i=2}^{\infty} a_i t^i,\]
where for every $i \geq 2$, the coefficient $a_i$ is of the form \[a_i = \frac{1}{i} L_i(f_A; \gamma_A^{*}\chi_{\omega}) + C_i,\]
with $C_i$ a constant term obtained from the numbers $L_k(f_A; \gamma_A^{*}\chi_{\omega})$, $k < i$. Similarly, 
\[\begin{split} \tau^{-\psi_A,\rho_i \gamma_A}_{G_A} (t) &\doteq  1 + L_1(f_A^{-1}; \gamma_A^{*}\chi_{\omega})t + \sum_{i=2}^{\infty} b_i t^i, \\  
b_i &= \frac{1}{i} L_i(f^{-1}_A; \gamma_A^{*}\chi_{\omega}) + D_i,\end{split} \]
and each $D_i$ is a constant term which only depends on the numbers $L_k(f_A^{-1}; \gamma_A^{*}\chi_{\omega})$, $k < i$. Note that the coefficients $a_i$ and $b_j$ are non-zero for only finitely many values of $i$ and $j$. Furthermore, the analogous equalities hold true for $\tau^{\varphi_B,\rho_i\gamma_B}_{G_B}$ and $\tau^{-\varphi_B,\rho_i \gamma_B}_{G_B}$. 
\smallskip 

Hence, by comparing the coefficients of the powers of $t$ in the expansions of the Redemeister torsions, it follows that for each $\rho_i$, \[\{L_m(f_B ; \gamma_B^{*}\chi_{\rho_i}), L_m(f_B^{-1} ; \gamma_B^{*}\chi_{\rho_i})\} = \{L_m(f_A ; \gamma_A^{*}\chi_{\rho_i}), L_m(f_A^{-1} ; \gamma_A^{*}\chi_{\rho_i})\}. \] Thus, \begin{gather*}L_m(f_B ; \gamma_B^{*}\chi_{\omega}) +L_m(f_B^{-1} ; \gamma_B^{*}\chi_{\omega}) = L_m(f_A ; \gamma_A^{*}\chi_{\omega}) + L_m(f_A^{-1} ; \gamma_A^{*}\chi_{\omega}), \text{ and}\\ 
L_m(f_B ; \gamma_B^{*}\chi_{\omega})L_m(f_B^{-1} ; \gamma_B^{*}\chi_{\omega}) = L_m(f_A ; \gamma_A^{*}\chi_{\omega})L_m(f_A^{-1} ; \gamma_A^{*}\chi_{\omega}).
\end{gather*}
Solving the above equations, we obtain
\[\{L_m(f_B ; \gamma_B^{*}\chi_{\omega}), L_m(f_B ; \gamma_B^{*}\chi_{\omega})\} = \{L_m(f_A ; \gamma_A^{*}\chi_{\omega}), L_m(f_A^{-1} ; \gamma_A^{*}\chi_{\omega})\}. \]
Now, 
\[\begin{split}
       N_m(f_B) + N_m(f^{-1}_B) &= \#\{\omega \in \Omega(Q_m) : L_m(f_B, \gamma_B^{*}\chi_{\omega}) \neq 0\}\\
      &     +\#\{\omega \in \Omega(Q_m) : L_m(f_B^{-1}, \gamma_B^{*}\chi_{\omega}) \neq 0\} \\
     &= \#\{\omega \in \Omega(Q_m) : L_m(f_A, \gamma_A^{*}\chi_{\omega}) \neq 0\}\\ 
     &    + \#\{\omega \in \Omega(Q_m) : L_m(f_A^{-1}, \gamma_A^{*}\chi_{\omega}) \neq 0\} \\ 
     &\leq N_m(f_A) + N_m(f_A^{-1}),
\end{split} \]
where the last inequality follows from Lemma~\ref{Nielsen_counting}. The same argument shows that $N_m(f_A) + N_m(f_A^{-1}) \leq N_m(f_B) + N_m(f_B^{-1})$. Hence $N_m(f_A) + N_m(f_A^{-1}) = N_m(f_B) + N_m(f_B^{-1})$. Similarly, we get that $N_m(f_B)\cdot N_m(f_B^{-1}) = N_m(f_A)\cdot N_m(f_A^{-1}).$ It follows that \[\{N_m(f_A), N_m(f_A^{-1})\} = \{N_m(f_B), N_m(f_B^{-1})\}.\qedhere\]\end{proof}

Combining Corollary~\ref{profinite R Torsion} with Lemma~\ref{Nielsen number invariance} and Proposition~\ref{train_track}, we obtain the following theorem.


\begin{thm}[Profinite invariance of Nielson numbers and stretch factors]\label{Nielsen numbers equality}
Let $G_A$ and $G_B$ be conjugacy separable free-by-cyclic groups with an isomorphism $\Theta \colon \widehat{G}_A \to \widehat{G}_B.$ Let $\varphi_B \in H^1(G_B, \mathbb{Z})$ be primitive and fibred, and let $\psi_A \in H^1(G_A, \mathbb{Z})$ be the primitive fibred class which is the pullback of $\varphi_B$ via $\Theta$. Let $(f^{\pm}_A, \Gamma_A)$ and $(f_B^{\pm}, \Gamma_B)$ be the corresponding relative train track representatives with stretch factors $\lambda_{f_A^{\pm}}$ and $\lambda_{f_B^{\pm}}$, respectively. Then, for all $m \in \mathbb{N}$, 
\begin{gather*}\{N_m(f_A) , N_m(f_A^{-1}) \} = \{ N_m(f_B), N_m(f_B^{-1})\}, \text{ and }\\ \{\lambda_{f_A}, \lambda_{f_A^{-1}} \}= \{\lambda_{f_B}, \lambda_{f_B^{-1}}\}.\end{gather*}
\end{thm}


We now have everything we need to prove \Cref{thmx:invariants}.

\medskip

\begin{duplicate}[\Cref{thmx:invariants}]
    Let $G=F\rtimes_{\Phi} \Z$ be a free-by-cyclic group with induced character $\varphi \colon G \to \Z$. If $b_1(G)=1$ then following properties are determined by the profinite completion $\widehat{G}$ of $G$: \begin{enumerate}
        \item the rank of $F$; \label{thmB1}
        \item the homological stretch factors $\{\nu^+_G,\nu^-_G\}$; \label{thmB2}
        \item the characteristic polynomials $\{\Char{\Phi^+},\Char{\Phi^-}\}$ of the action of $\Phi$ on $H_1(F;\QQ)$; \label{thmB3}
        \item for each representation $\rho\colon G\to \GL(n, \QQ)$ factoring through a finite quotient, the twisted Alexander polynomials $\{\Delta^{\varphi,\rho}_n,\Delta^{-\varphi,\rho}_n\}$ and the twisted Reidemeister torsions $\{\tau^{\varphi,\rho},\tau^{-\varphi,\rho}\}$ over $\QQ$. \label{thmB4}
    \end{enumerate}
    Moreover, if $G$ is conjugacy separable, (e.g. if $G$ is hyperbolic), then $\widehat G$ also determines the Nielsen numbers and the homotopical stretch factors $\{\lambda^+_G,\lambda^-_G\}$.
\end{duplicate}
\begin{proof}
    Note that any free-by-cyclic group $G_A$ profinite isomorphic to $G=G_B$ has $b_1(G_A)=1$ and so the monodromy $\psi$ is the pullback of $\varphi$ or $\varphi^{-1}$ via some isomorphism $\Theta$.  With this setup we:
    \begin{enumerate}
        \item is given by \Cref{fibre iso};
        \item is given by \Cref{thm:homostretch};
        \item follows from (4) and the fact that we can identify $\Char{\Phi^\pm}$ with $\Delta^{\pm\varphi,\mathbf{1}}_1$;
        \item is given by \Cref{profinite AP}.
    \end{enumerate} 
Finally, the moreover is given by \Cref{Nielsen numbers equality}.
\end{proof}

%\begin{proof}
%By Proposition~\ref{hyperbolicity is profinite}, $G_A$ is hyperbolic. Hence by \cite{HagenWise2015}, $G_A$ and $G_B$ are compact special. It follows by \cite{MinasyanZalesskii2016}, that $G_A$ and $G_B$ are conjugacy separable. Combining these facts with Corollary~\ref{profinite R Torsion}, we get that $G_A$ and $G_B$ satisfy the hypotheses of Lemma~\ref{Nielsen number invariance}. Thus, 
%\[\{N_m(f_A) , N_m(f_A^{-1}) \} = \{ N_m(f_B), N_m(f_B^{-1})\}.\]
%Combining this with Proposition~\ref{train_track} yields the equality of the sets of stretch factors. 
%\end{proof}

\section{Almost profinite rigidity for free-by-cyclic groups}\label{sec.proofmain}

The aim of this section is to prove \Cref{thmx:Irr}.

\medskip

\begin{duplicate}[\Cref{thmx:Irr}]\label{main}
    Let $G$ be an irreducible free-by-cyclic group.  If $b_1(G)=1$, then $G$ is almost profinitely rigid amongst irreducible free-by-cyclic groups.
\end{duplicate}

\medskip

The following proposition is folkore. A careful proof in the more general setting of expanding free group endomorphisms can be found in the paper of Mutanguha \cite[Theorem~A.4]{Mutanguha2021}. An outer automorphism $\Phi \in \mathrm{Out}(F_n)$ is said to be \emph{atoroidal} if there does not exist a non-trivial element $x \in F_n$ and $n \geq 1$ such that $\Phi^n$ preserves the conjugacy class of $x$. 

\begin{prop}\label{geometric}
    Let $\Phi \in \mathrm{Out}(F_n)$ be an outer automorphism of $F_n$. Suppose that $\Phi$ is irreducible and not atoroidal. Then $\Phi$ is induced by a pseudo-Anosov homeomorphim of a punctured surface.
\end{prop}

\begin{proof}[Proof of \Cref{thmx:Irr}]
    Let $G_A$ be a free-by-cyclic group with $b_1(G_A) = 1$ and irreducible monodromy $\Phi$. Let $G_B$ be another free-by-cyclic group with irreducible monodromy $\Psi$ and suppose that $\widehat{G}_A \cong \widehat{G}_B$.
    
    If $\Phi$ not atoroidal then $\Psi$ is not atoroidal by \Cref{thmx:hyperbolicity}. Hence, by \Cref{geometric}, both $\Phi$ and $\Psi$ are induced by pseudo-Anosov homeomorphisms of compact surfaces. Thus $G_A$ and $G_B$ are fundamental groups of compact hyperbolic 3-manifolds and the result holds by \cite[Theorem~9.1]{Liu2023}.     

    Suppose that $\Phi$ is atoroidal. Then by \Cref{thmx:hyperbolicity}, $\Psi$ is also atoroidal. Hence $G_A$ and $G_B$ are hyperbolic free-by-cyclic groups. By \cite{HagenWise2015}, $G_A$ and $G_B$ are virtually compact special, and thus by \cite{Minasyan2006} they are conjugacy separable. Furthermore, $b_1(G_B) = 1$ since Betti numbers are invariants of profinite completions. Thus by \Cref{mu unit}, the isomorphism $\widehat
{G}_A \to \widehat{G}_B$ is $\widehat
\Z$-regular. Hence by \Cref{Nielsen numbers equality}, the sets of stretch factors $\{\lambda_{\Phi}, \lambda_{\Phi^{-1}}\}$ of $\Phi^{\pm 1}$ and $\{\lambda_{\Psi}, \lambda_{\Psi^{-1}}\}$ of $\Psi^{\pm 1}$  are equal. The result now follows by \Cref{min}.
\end{proof}


\section{Procongruent conjugacy in \texorpdfstring{$\Out(F_n)$}{Out(Fn)}}\label{sec.proconjugacy}

In this section we show that the stretch factors of atoroidal elements of $\Out(F_n)$ are procongruent conjugacy invariants.

\begin{defn}[Procongruently conjugate]
    Let $\Psi,\Phi\in\Out(F_n)$.  We say $\Psi$ and $\Phi$ are \emph{procongruently conjugate} if they induce a pair of conjugate outer automorphisms in $\Out(\widehat{F}_n)$.
\end{defn}

\smallskip

\begin{duplicate}[\Cref{thmx:procongruence}]
    Let $\Psi\in\Out(F_n)$ be atoroidal.  If $\Phi\in\Out(F_n)$ is procongruently conjugate to $\Psi$, then $\Phi$ is atoroidal and $\{\lambda_\Psi,\lambda_{\Psi^{-1}}\}=\{\lambda_\Phi,\lambda_{\Phi^{-1}}\}$.  In particular, if $\Psi$ is additionally irreducible, then there are only finitely many $\Out(F_n)$-conjugacy classes of irreducible automorphisms which are conjugate with $\Psi$ in $\Out(\widehat{F}_n)$
\end{duplicate}

\smallskip 

\begin{proof}
    The first result follows from applying \Cref{thmx:hyperbolicity}, \Cref{thmx:invariants}, and \Cref{prop:procongruency}.  The latter of which is proved below.  The ``in particular'' then follows from \Cref{min}.
\end{proof}

\begin{defn}[Aligned isomorphism]
    Let $\Psi,\Phi\in\Out(F_n)$. Write $G_A = F_n \rtimes_{\Psi} \Z$ and $G_B = F_n \rtimes_{\Phi} \Z$ and let $\psi \colon G_A \to \Z$ and $\psi \colon G_B \to \Z$ be the induced characters.  We say that an isomorphism $\Theta\colon\widehat{G}_A\to \widehat{G}_B$ is \emph{aligned} if the following diagram commutes
    \[\begin{tikzcd}
        \widehat{G}_A \arrow[r,"\widehat\psi"] \arrow[d,"\Theta"] & \widehat{\Z} \arrow[d,"\id"]\\
        \widehat{G}_B \arrow [r,"\widehat\varphi"] & \widehat{\Z}.
    \end{tikzcd}\]
    Note that an aligned isomorphism realises $\psi$ as the pullback of $\varphi$ with respect to $\Theta$ with unit $1$ in the sense that $\Theta_\ast(\varphi)=\psi$.
\end{defn}

The following proposition patterns \cite[Proposition~3.7]{Liu2023b}.

\begin{prop}\label{prop:procongruency}
    Let $\Phi,\Psi\in\Out(F_n)$.  The following are equivalent:
    \begin{enumerate}
        \item the free-by-cyclic groups $G_A=F_n\rtimes_\Psi\Z$ and $G_B=F_n\rtimes_\Phi\Z$ are aligned isomorphic; \label{prop:procongruency.1}
        \item the outer automorphisms $\Phi$ and $\Psi$ are procongruently conjugate. \label{prop:procongruency.2}
    \end{enumerate}
\end{prop}
\begin{proof}
    In constructing $G_A$ and $G_B$ we have implicitly picked lifts of $\Phi$ and $\Psi$ to $\Aut(F_n)$ which abusing notation we have also denoted by $\Phi$ and $\Psi$.  Write $G_A=F_n\rtimes_\Psi\langle t_A\rangle$ and $G_B=F_n\rtimes_\Phi\langle t_B\rangle$.  Denote the images of $t_A$ and $t_B$ is $\Out(F_n)$ by $\tau_A$ and $\tau_B$.  Note $\widehat{G}_A=\widehat{F}_n\rtimes\widehat{\langle t_A\rangle}$ and similarly for $G_B$.  Denote the images of $\tau_A$ and $\tau_B$ in $\Aut(\widehat{F}_n)$ by $\widehat\tau_A$ and $\widehat\tau_B$ respectively.
    
    We now prove that \eqref{prop:procongruency.1} implies \eqref{prop:procongruency.2}.  Suppose there is an aligned isomorphism $\Theta\colon \widehat{G}_A\to\widehat{G}_B$ and denote its restriction to $\widehat{F}_n$ by $\Theta_F$.  We have $\Theta(t_A)=t_Bh$ for some $h\in\widehat{F}_n$.  Since $gt_A=t_At_A^{-1}gt_A=t_A\widehat\tau_A(g)$ we have $\Theta_F(g)t_Bh=t_Bh\Theta_0(\widehat\tau_A(g))$.  Let $I_h$ denote the inner automorphism given by conjugation by $h$.  We have $\Theta_F(g)t_B=t_BI_h(\Theta_F(\widehat\tau_A(g))$, and hence, $t_B\widehat\tau_B(\Theta_F(g))=t_BI_h(\Theta_F(\widehat\tau(g)))$ for all $g\in\widehat{F}_n$.  Hence, $\widehat\tau_B=I_h\Theta_F\widehat\tau_A\Theta^{-1}$.  It follows that $\widehat\tau_A$ and $\widehat\tau_B$ are conjugate when projected to $\Out(\widehat{F}_n)$.  Hence, $\Phi$ and $\Psi$ are procongruently conjugate.

    To show \eqref{prop:procongruency.2} implies \eqref{prop:procongruency.1} we reverse the previous calculation to obtain a group isomorphism $\widehat{G}_A\to\widehat{G}_B$.
\end{proof}

\section{Automorphisms of universal Coxeter groups}\label{sec:Wn}

Let $n \geq 2$ be an integer. The \emph{universal Coxeter group of rank $n$} is the free product $W_n$ of $n$ copies of $\Z /2$, 
\[W_n = \bigast_{i=1}^n \Z /2.\] 
A \emph{free basis} of $W_n$ is a collection of $n$ elements $a_1 ,\ldots, a_n$ of $W_n$ of order 2, such that 
\[W_n \cong \langle a_1 \rangle \ast \ldots \ast \langle a_n \rangle.\]

\subsection{Graphs of groups}\label{sec:graphofgroups} 

For further detail and careful proofs of the claims made in this section, the interested reader is referred to \cite{Lyman2022a}. We closely follow the notation established there.

A \emph{graph of groups $(\Gamma, \mathcal{G})$ with trivial edge groups} consists of a connected graph $\Gamma$ and an assignment of a group $\mathcal{G}_v$ to every vertex $v$ of $\Gamma$. The vertex $v$ is said to be \emph{essential} if $\mathcal{G}_v$ is non-trivial. To every graph of groups $(\Gamma, \mathcal{G})$ we associate a graph of spaces $X_{\mathcal{G}}$ constructed by attaching a $K(\mathcal{G}_v, 1)$ with a unique vertex $v_0$ to the corresponding vertex $v$ of $\Gamma$. For the sake of brevity, we will sometimes write $\mathcal{G}$ to denote the graph of groups $(\Gamma, \mathcal{G})$. After fixing a basepoint and a spanning tree in $\mathcal{G}$, and immediately suppressing their notation, we write $\pi_1(\mathcal{G})$ to denote the fundamental group of the graph of groups $\mathcal{G}$.

A \emph{morphism} between graphs of groups $(\Gamma, \mathcal{G})$ and $(\Lambda, \mathcal{H})$ consists of a pair of maps $(f, f_X)$ with the following properties. The first map $f \colon \Gamma \to \Lambda$ sends vertices to vertices, and edges to edge paths. The second map $f_X \colon X_{\mathcal{G}} \to X_{\mathcal{H}}$ is a map of spaces such that the following diagram commutes,
\begin{equation*}\label{}\begin{tikzcd}
    X_{\mathcal{G}} \arrow[r, "f_X"] \arrow[d] & X_{\mathcal{H}} \arrow[d] \\
    \Gamma \arrow[r, "f"] & \Lambda
\end{tikzcd}
\end{equation*}
The vertical maps are the retractions obtained by collapsing the vertex spaces. Again we will sometimes write $f$ to mean the pair $(f, f_X)$.

A \emph{homotopy} from the morphism $(f, f_X) \colon (\Gamma, \mathcal{G} ) \to (\Lambda, \mathcal{H})$ to $(f', f'_X) \colon (\Gamma, \mathcal{G} ) \to (\Lambda, \mathcal{H})$ is a collection of morphisms \[\{(f_s, f_{X,s}) \colon \mathcal{G} \to \mathcal{H} : s\in [0,1]\},\] such that $\{f_s\}$ is a homotopy from $f$ to $f'$, and $\{f_{X,s}\}$ is a homotopy from $f_{X}$ to $f_{X}'$.

A morphism $f \colon \mathcal{G} \to \mathcal{H}$ is a \emph{homotopy equivalence,} if there exists a morphism $g \colon \mathcal{H} \to \mathcal{G}$ 
such that $fg$ and $gf$ are homotopic to the identity morphisms. Any homotopy equivalence $f \colon \mathcal{G} \to \mathcal{H}$ induces an isomorphism $f_{*} \colon \pi_1(\mathcal{G}) \to \pi_1(\mathcal{H})$.

We will use the term \emph{combinatorial graph} when we want to emphasise that we are considering a graph with no extra structure.

\subsection{Topological representatives of \texorpdfstring{$\mathrm{Out}(W_n)$}{Out(Wn)} and Nielsen numbers}\label{sec:traintrackWn}

Let $\Phi \in \mathrm{Out}(W_n)$. A \emph{topological representative} of $\Phi$ is a tuple $(f, \mathcal{G})$ where $\mathcal{G}$ is a graph of groups with $\pi_1(\mathcal{G}) \cong W_n$ and $f\colon \mathcal{G} \to \mathcal{G} $ is homotopy equivalence of graphs of groups, which induces the outer automorphism $\Phi$. We assume that $\mathcal{G}$ has trivial edge groups and finite vertex groups (in particular, each vertex group is isomorphic to $\Z /2$). We further assume that $f$ is locally injective on the interiors of the edges of $\Gamma$. When we talk of the \emph{transition matrix, maximal filtration} and \emph{exponential strata} of $(f, \mathcal{G})$, we are referring to those objects associated to the underlying graph map $(f, \Gamma)$ (see Section~\ref{TopRep}). In particular, the topological representative $(f, \mathcal{G})$ is said to be \emph{irreducible} if the filtration of the underlying graph map $(f, \Gamma)$ has length one. 

Given a topological representative $(f, \mathcal{G})$ of an element in $\mathrm{Out}(W_n)$, we define the mapping torus $M_f$ to be the quotient space 
\[M_f = \frac{X_{\mathcal{G}} \times [0,1]}{(f_X(x),0) \sim (x, 1)}.\]
We define an equivalence relation $\sim$ on the set of topological representatives of elements of $\mathrm{Out}(W_n)$, so that $(f_1, \mathcal{G}) \sim (f_2, \mathcal{H})$ whenever there is an isomorphism $\pi_1(M_{f_1}) \cong \pi_1(M_{f_2})$. The proof of the following lemma is a slight modification of the proof of Lemma~\ref{min}.

%The transition matrix $A$ of a topological representative $(f, \mathcal{G})$ is the transition matrix of the underlying graph map $f \colon G \to G$. We say $(f, \bbG)$ is \emph{irreducible}, if the matrix $A$ is an irreducible matrix. The \emph{stretch factor} of $(f, \bbG)$ is the Perron--Frobenius eigenvalue of the transition matrix (see Section~\ref{TopRep}). 

\begin{lemma}\label{finite}
    Let $n \geq 2$ and $C > 1$. There exists at most finitely many equivalence classes of topological representatives $(f, \mathcal{G})$ of elements in $\mathrm{Out}(W_n),$ such that $f$ is irreducible and the stretch factor $\lambda$ of $f$ satisfies $\lambda \leq C$.
\end{lemma}

\begin{proof}
    Fix $n \geq 2$. Since the vertex groups of $\mathcal{G}$ are finite, each topological representative $(f, \mathcal{G})$ of an element in $\mathrm{Out}(W_n)$ contains exactly $n$ essential vertices and the underlying graph $\Gamma$ is simply-connected. Furthermore, as in the proof of Lemma~\ref{min}, every equivalence class of irreducible topological representatives contains an irreducible representative $(f, \mathcal{G})$ with no inessential valence-one and valence-two vertices.  In particular, the total number of edges of $\Gamma$ is uniformly bounded in terms of $n$. Hence, the transition matrix associated to $(f, \mathcal{G})$ can take on at most finitely many values. Since the vertex groups are finite, there are finitely many equivalence classes of topological representatives $(f, \mathcal{G})$ with a given transition matrix. The result follows. 
\end{proof}

The outer automorphism $\Phi \in \mathrm{Out}(W_n)$ is said to be \emph{irreducible}, if every topological representative $(f, \mathcal{G})$ of $\Phi$, where the underlying graph $\Gamma$ has no inessential valence-one vertices and no invariant non-trivial forests, is irreducible.  The \emph{stretch factor} of $\Phi$ is the infimum of the stretch factors of irreducible topological representatives of $\Phi$. By Lemma~\ref{finite}, the infimum is realised.

There exists a theory of (improved) relative train track representatives for elements of $\mathrm{Out}(W_n)$ \cite{Lyman2022b} (see also \cite{CollinsTurner1994}, \cite{FrancavigliaMartino2018} and  \cite{Lyman2022a} for earlier results on train tracks on graphs of groups), which is completely analogous to that for elements in $\mathrm{Out}(F_n)$. As in the case of $\mathrm{Out}(F_n)$, the stretch factor of an irreducible outer automorphism $\Phi \in \mathrm{Out}(W_n)$, as defined in the previous paragraph, coincides with the stretch factor of any train track representative. The stretch factor of a general element $\Phi \in \mathrm{Out}(W_n)$ is defined to be the stretch factor of any relative train track representative. 

The proof of the following lemma is completely analogous to the proof of Proposition~\ref{train_track}.

\begin{lemma} Let $\Phi \in \mathrm{Out}(W_n)$ be an outer automorphism of $W_n$ with stretch factor $\lambda$. Let $(f, \mathcal{G})$ be a topological representative of $\Phi$. Then 
\[ \lambda = \mathrm{lim}\,\mathrm{sup}_{m \to \infty} N_m(f)^{1/m}.\]
\end{lemma}

Before proceeding further, we take a detour to discuss irreducibility of matrices and graphs. 

Let $A \in M_n(\mathbb{Z})$ be a matrix with non-negative integer entries $a_{ij}$. We construct a directed graph $\Gamma_A$ associated to $A$, so that $\Gamma_A$ has $n$ vertices $\{v_1, \ldots, v_n \}$ and there exist $a_{ij}$ directed edges from $v_i$ to $v_j$, for every $i,j \leq n$. The directed graph $\Gamma_A$ is said to be \emph{irreducible}, if for any two vertices $u$ and $v$ of $\Gamma_A$, there exists a directed path from $u$ to $v$. The following is an elementary exercise.

\begin{lemma} The non-negative integer matrix $A$ is irreducible if and only if the associated graph $\Gamma_A$ is irreducible.
\end{lemma}

We now prove a crucial lemma on the irreducibility of degree-two covers of directed graphs. In what follows, when we say \emph{path} from $u$ to $v$, we will always mean a directed path. Given an oriented edge $e$ in an oriented graph $\Gamma$, we write $i(e)$ to denote the initial vertex of $e$ in $\Gamma$ and $t(e)$ the terminal vertex.

\begin{lemma}\label{index_2_irred}
Let $\Gamma$ be a directed graph on $n$ vertices, and let $\Gamma'$ be a degree-two cover of $\Gamma$. If $\Gamma$ is irreducible then either $\Gamma'$ is irreducible, or it has two connected components and each is isomorphic to $\Gamma$.

Furthermore, if $\Gamma'$ is irreducible then the Perron--Frobenius eigenvalues of $A_{\Gamma'}$ and $A_\Gamma$ are equal.
\end{lemma}

\begin{proof} Let $\{v_1, \ldots, v_n\}$ be the vertex set of $\Gamma$. Let $v_i^1$ and $v_i^2$, be the two lifts of $v_i$ in $\Gamma'$, and write $V_1 = \{v_i^1 \mid 1 \leq i \leq n\}$ and $V_2 = \{v_i^2 \mid 1 \leq i \leq n\}$. Let $N$ be the number of edges $e$ in $\Gamma'$ such that $i(e) \in V_1$ and $t(e) \in V_2$. We call such edges \emph{special}. We prove our result by induction on $N$.

If $N=0$ then the lemma is clearly true, since $\Gamma'$ has two connected components and each is isomorphic to $\Gamma$. 

Let $N\geq 1$ and suppose the lemma is true whenever the number of special edges is at most $N-1$. Let $\Gamma' \to \Gamma$ be a degree-two cover with $N$ special edges. Note that since $\Gamma$ is irreducible, for any vertices $v_i$ and $v_j$ of $\Gamma$, there exists a path $\gamma$ from $v_i$ to $v_j$. This path has two lifts $\gamma_1$ and $\gamma_2$ in $\Gamma'$ such that either 
\begin{enumerate}[label=\roman*)]
    \item $\gamma_1$ joins $v_i^1$ to $v_j^1$ and $\gamma_2$ joins $v_i^2$ to $v_j^2$; or
    \item $\gamma_1$ joins $v_i^1$ to $v_j^2$ and $\gamma_2$ joins $v_i^2$ to $v_j^1$.
\end{enumerate}
Hence to prove the lemma it suffices to show that there exists a path in $\Gamma'$ from $v_k^1$ to $v_k^2$, for all $k$. 

Let $e_1$ be a special edge and suppose that $i(e_1) = v_i^1$ and $t(e_1) = v_j^2$, for some $i$ and $j$. Then $\Gamma'$ contains an edge $e_2$ such that  $i(e_2) = v_i^2$ and $t(e_2) = v_j^1$. Construct a graph $\Gamma''$ from $\Gamma'$ by replacing $e_1$ with the edge $e_1'$ which joins $v_i^1$ to $v_j^1$, and replacing $e_2$ with the edge $e_2'$ which joins $v_i^2$ to $v_j^2$. Note that $\Gamma''$ is a degree -two cover of $\Gamma$ with $N-1$ special edges. 

Suppose first that $N = 1$ and fix index $k \leq n$. Since $\Gamma$ is irreducible, there exists a path in $\Gamma$ from $v_k$ to $v_i$. Let $\gamma$ be a shortest such path. Then $\gamma$ has two lifts $\gamma_1$ and $\gamma_2$ in $\Gamma''$. Since $\Gamma''$ has zero special edges, $\gamma_1$ only crosses edges with both endpoints in $V_1$ and $\gamma_2$ only crosses edges with both endpoints in $V_2$ (possibly after swapping $\gamma_1$ and $\gamma_2$). Also by minimality of the length of $\gamma$, the lifts of $\gamma$ do not cross the edges $e_1'$ and $e_2'$. Hence the path $\gamma_1$ descends to a path in $\Gamma'$ joining $v_k^1$ to $v_i^1$. Similarly one constructs a path from $v_j^2$ to $v_{k}^2$ in $\Gamma'$. The concatenation of these two paths and the edge $e_1$ gives a path from $v_k^1$ to $v_k^2$. 

Now assume $N\geq 2$. Then $\Gamma''$ is irreducible and thus there exists a shortest path $\eta_1$ in $\Gamma''$ from $v_k^1$ to $v_i^1$, and a shortest path $\eta_2$ from $v_j^2$ to $v_k^2$. Since $i(e_1') = v_i^1$, any shortest path from $v_k^1$ to $v_i^1$ does not contain $e_1'$. Similarly, any shortest path from $v_j^2$ to $v_k^2$ does not contain $e_2'$. Hence $\eta_1$ and $\eta_2$ descend to paths in $\Gamma'$. The concatenation of these paths, together with the edge $e_1$ give rise to a path from $v_k^1$ to $v_k^2$. Hence statement holds for $\Gamma'$. This proves the first part of the lemma.

To prove the statement about equality of Perron--Fobenius eiganvalues, suppose that $\Gamma'$ is irreducible. Let $a_{ij}$ and $a'_{ij}$ denote the $ij$th elements of $A_{\Gamma}$ and $A_{\Gamma'}$, respectively. Since $\Gamma'$ is a degree-two cover of $\Gamma$, it follows that for every $i,j \leq n$,
\begin{equation} \label{entries} a_{ij} = a'_{ij} + a'_{i(j+n)} = a_{(i+n)j}' + a_{(i+n)(j+n)}'. \end{equation}
Let $v_{pf}$ denote the Perron--Frobenius eigenvector of $A_{\Gamma}$ and let $\lambda$ be the Perron--Frobenius eigenvalue. Let $v_{pf}'$ be the vector obtained by concatenating two copies of $v_{pf}$. Then by \eqref{entries},
\[A_{\Gamma'}v_{pf}' = \lambda \cdot v_{pf}'.\] 
Hence the Perron--Frobenius eigenvalue of $A_{\Gamma'}$ is $\lambda$.\end{proof}

Let $W_n$ be the universal Coxeter group with a free basis $\{a_1, \ldots, a_n\}.$ There exists a homomorphism $W_n \twoheadrightarrow \mathbb{Z}/2$ which maps each generator $a_i$ to the generator of $\mathbb{Z}/2$. The kernel $K \leq W_n$ is an index-two characteristic subgroup of $W_n$, which is isomorphic to the free group of rank $n-1$, $F_{n-1}$. Any outer automorphism $\Phi \in \mathrm{Out}(W_n)$ induces an outer automorphism of $K$, which we denote by $\Phi_K \in \mathrm{Out}(F_{n-1})$. Note that $K$ depends on the choice of free basis of $W_n$.

\begin{prop}\label{stretch factors of W_n auts}
    Let $n \geq 3$ and $\Phi \in \mathrm{Out}(W_n)$. Then, for some subgroup $K \leq W_n$ as above, the stretch factor of the map $\Phi_K \colon K \to K$ is equal to the stretch factor of $\Phi$.

    Furthermore, if $\Phi$ is irreducible then so is $\Phi_K$. 
\end{prop}

\begin{proof}
    Let $(f, \mathcal{G})$ be a relative train track representative of $\Phi \in \mathrm{Out}(W_n)$, where $\mathcal{G} = (\Gamma, \mathcal{G})$ is a graph of groups as before. Let $\{a_1, \ldots, a_n\}$ be a free basis of $W_n$ so that each vertex of the underlying graph $\Gamma$ of $\mathcal{G}$ is labelled by some $\langle a_i \rangle \cong \Z / 2$ or the trivial group. Note that $\Gamma$ is simply-connected. Let $K = \langle a_1a_2, a_1a_3, \ldots, a_1a_n \rangle$. 
    
    As before, let $X_{\mathcal{G}}$ denote the graph of spaces associated to $\mathcal{G}$. Let $\pi \colon Y \to X_{\mathcal{G}}$ be the cover of $X_{\mathcal{G}}$ corresponding to the subgroup $K$. Since $K$ is a characteristic subgroup, there's a lift of the map $f_X$ to a map $f_Y \colon Y \to Y$. Note that $a_i \not\in K$  for every $i$, and thus the edge loops in $X_{\mathcal{G}}$ which correspond to the elements $a_i \in \pi_1(\mathcal{G})$ lift to edges in $Y$ with distinct endpoints. Let $Y'$ be the space obtained from $Y$ by collapsing all such edges and the lifts of the 2-cells. Then $Y'$ is homotopy equivalent to $Y$, and there is a map $f_{Y'} \colon Y'\to Y'$ which is homotopic to $f_Y$. Note that $Y'$ is a (combinatorial) graph which is obtained by doubling the underlying graph $\Gamma$ of $\mathcal{G}$ along the essential vertices. In particular, $(f_{Y'}, Y')$ is a topological representative of $\Phi_K$.

    The relative train track structure of $f$ lifts to a relative train track structure of $f_{Y'}$. If $S$ is a non-zero stratum of $\mathcal{G}$ with stretch factor $\lambda$, then by Lemma~\ref{index_2_irred}, its lift to $Y'$ is either an irreducible stratum with stretch factor $\lambda$ or two irreducible strata, each with stretch factor $\lambda$. This proves the claim about equality of stretch factors.

 %   Now suppose that $f$ is reducible. We will show that $\Phi_K$ admits a topological representative with no valence-one vertices and no invariant forests, and an invariant non-trivial proper subgraph. To that end, start with a relative train track $(f, \mathcal{G})$ of $\Phi$ and let $(f_{Y'}, Y')$ be defined as above. Note that $Y'$ has no valence-one vertices since $\Gamma$ has no inessential valence-one vertices. Let $\Gamma_k$ be the first term of the filtration of $(f, \mathcal{G})$ such that $\Gamma_k$ contains at least two essential vertices of degree one. 
    
    
%    the union $\Gamma_k^1 \cup \Gamma_k^2$ of the lifts of $\Gamma_k$ to $Y'$ is non-simply-connected. Such a $k$ exists since $Y'$ is not simply-connected and furthermore $\Gamma_k \subsetneq \Gamma$ is a proper subgraph since . Hence $\Gamma_k^1 \cup \Gamma_k^2 \subsetneq Y'$ is a proper subgraph.
    
%    Let $S$ be the 
 %   and write $S_1$ and $S_2$ to denote the lifts of the subgraph $S$ of $Y$ to $Y'$. Note that since $f$ is reducible, $S_1 \cup S_2$ is a proper subgraph of $Y'$. 
    
 %   Suppose first that $f_{Y'}$ preserves $S_2$. Let $Y''$ be the graph obtained from $Y'$ by collapsing the subgraph $S_2 \subseteq Y'$ to a point. Since $S_2$ is simply-connected, $Y''$ is homotopy equivalent to $Y'$ and there's an induced relative train track map $f_{Y''} \colon Y'' \to Y''$ which is homotopic to $f_{Y'}$.  Let $\bar{S}_1$ be the image of $S_1$ in $Y''$. Since $Y'$ is a double of $\Gamma$ that $\bar{S}_1$ is not simply-connected. Now construct $Y'''$ from $Y''$ by contracting to a point each $f_{Y''}$-invariant simply-connected subgraph of $Y''$. Hence the image $\bar{\bar{S}}_1$ of $S_1$ in $Y'''$ is not simply-connected and the resulting map $f_{Y'''} \colon Y''' \to Y'''$ is a topological representative of $\Phi_K$ with the required properties. 
    
 %   Suppose now that $S_1 \cup S_2$ is an irreducible stratum of $f_{Y'}$. We have that $S_1 \cup S_2$ is not simply-connected. Let $Y''$ be the graph obtained from $Y'$ by contracting to a point all the maximal $f_{Y'}$-invariant forests, and let $f_{Y''} \colon Y'' \to Y''$ be the induced map. Then $(f_{Y''}, Y'')$ is a topological representative of $\Phi_K$ as required.
    
    Suppose now that $f$ is irreducible. Then either $f_{Y'}$ is irreducible, or $Y' = \Gamma \cup \Gamma^{*}$ is the double of a simply-connected graph $\Gamma$, and $f_{Y'}$ preserves $\Gamma$ and $\Gamma^{*}$. Let $Y''$ be the graph obtained from $Y'$ by collapsing the subgraph $\Gamma^{*} \subseteq Y'$ to a single point. Then $Y''$ is homotopy equivalent to $Y'$ and there is an induced map $f_{Y''} \colon Y'' \to Y''$ which is homotopic to $f_{Y'}$. Then since $f_{Y''}$ is a train track map, it follows that $\Phi$ is irreducible.
\end{proof}


 \subsection{Profinite invariants and almost rigidity of \{universal Coxeter\}-by-cyclic groups}\label{sec:ProfiniteCoxeter}

A group $G$ is said to be $\{$\emph{universal Coxeter$\}$-by-cyclic} if it fits into the short exact sequence
\[1 \to W_n \to G \to \Z \to 1.\]

For the remainder of this section, we let $(G_A, \varphi)$ and $(G_B, \psi)$ denote \{universal Coxeter\}-by-cyclic groups with fibred characters $\varphi \colon G_A \to \Z$ and $\psi \colon G_B \to \Z$.  We write $G_A = W_n \rtimes_{\Phi} \Z$ and $G_B = W_m \rtimes_{\Psi} \Z$ to denote the splittings of $G_A$ and $G_B$ induced by the characters, and let $K_A \leq W_n$ and $K_B \leq W_m$ be the index-two characteristic free subgroups associated to $\Phi$ and $\Psi$, respectively, as obtained in \Cref{stretch factors of W_n auts}. In particular, the stretch factors of $\Phi_{K_A} = \Phi|_{K_A}$ and $\Psi_{K_B} = \Psi|_{K_B}$ are the same as those of $\Phi$ and $\Psi$, respectively.  We fix some $t \in \varphi^{-1}(1)$ and $s \in \psi^{-1}(1)$, and let 
\begin{equation}\label{index_2_free_by_z}\begin{split} H_A &= \langle K_A , t \rangle_{G_A} \cong K_A \rtimes_{\Phi_{K_A}} \Z,  \\ H_B &= \langle K_B , s \rangle_{G_B} \cong K_B \rtimes_{\Psi_{K_B}} \Z.\end{split}\end{equation}
We write $\bar{\varphi}$ to denote the character $\varphi \colon G_A \to \Z$ restricted to the subgroup $H_A$, and define $\bar{\psi}$ similarly. We note that the characters $\bar{\varphi}$ and $\bar{\psi}$ induce the splittings \eqref{index_2_free_by_z}.

For a group $G$ and prime $p$ we denote its \emph{pro-$p$ completion} by $\widehat{G}^\mathbf{p}$.  Note this is exactly the inverse limit over all quotients of order a power of $p$.


\begin{prop}\label{prop.profinite.Wn}
    Let $(G_A,\varphi)$ and $(G_B,\psi)$ be \{universal Coxeter\}-by-cyclic groups, and suppose $\Theta\colon \widehat{G}_A\to\widehat{G}_B$ is an isomorphism.  The following conclusions hold:
    \begin{enumerate}
        \item $\Theta$ is $\widehat{\Z}$-regular;\label{prop.propfinite.Wn.1}
        \item $G_A$ and $G_B$ have isomorphic fibres; \label{prop.propfinite.Wn.2}
        \item the free-by-cyclic groups $(H_A,\bar{\varphi})$ and $(H_B,\bar{\psi})$ satisfy that $\bar{\varphi}$ is the pullback of $\bar{\psi}$ via $\Theta|_{\widehat{H}_A}$; \label{prop.propfinite.Wn.3}
        \item $G_A$ and $G_B$ are good. \label{prop.propfinite.Wn.4}
    \end{enumerate}
\end{prop}
\begin{proof}
    It is easy to see that $G_A$ and $G_B$ satisfy $b_1(G_A)=b_1(G_B)=1$.  Thus, \eqref{prop.propfinite.Wn.1} follows from \Cref{mu unit}.  Note that $b_1(W_n;\FF_2)=n$.  We may prove \eqref{prop.propfinite.Wn.2} by an identical argument to \Cref{fibre iso} but taking the twisted Alexander polynomials over $\FF_2$ instead of an arbitrary prime.  
    
    The fact that $G_A$ has an index two free-by-cyclic group $H_A$ is \cite[Section~2]{Healy2021} (see also \cite{PiggottRuane2010}).  Since goodness passes to finite index overgroups this also proves \eqref{prop.propfinite.Wn.4}.
    
    Now, the group $H_A$ is the kernel of a map $\alpha\colon G_A\onto\Z/2$.  We see that $H_A$ is torsion-free and so its pro-$2$ completion has finite cohomological dimension, whereas $G_A$ has $2$-torsion so $\cd_2(\widehat{G}_A^{\mathbf 2})=\infty$ (see \cite[Section 1.1. and Proposition~11.1.5]{Wilson1998} for the definition of $\cd_2$ and the relevant facts).  Completing the map $\alpha$ to $\widehat{G}_A$ we obtain an induced map $\widehat{G}_B\onto\Z/2$ and hence a map $\beta\colon G_B\onto \Z/2$.  Now $\ker\beta$ is torsion-free since $\ker \widehat\beta\cong\ker\widehat\alpha$ and $\cd_2(\ker\widehat\alpha^\mathbf{2})$ is finite.  We have shown that $H_A$ and $H_B$ are profinitely isomorphic free-by-cyclic groups with monodromies $\bar\varphi$ and $\bar\psi$ respectively.  Since $\Theta$ is $\widehat{Z}$-regular by \eqref{prop.propfinite.Wn.1}, it follows that $\bar\varphi$ is the pullback of $\bar\psi$ via $\Theta|_{\widehat H_A}$.
\end{proof}



\begin{duplicate}[\Cref{thmx:CoxeterStretchInvariance}]
    Suppose all free-by-cyclic groups with monodromy induced by an element of $\Out(W_n)$ are conjugacy separable.  Let $(G_A,\varphi)$ and $(G_B,\psi)$ be profinitely isomorphic \{universal Coxeter\}-by-cyclic groups. Let $\{\lambda_{A}^+,\lambda_{A}^-\}$ and $\{\lambda_{B}^+,\lambda_{B}^-\}$ be the stretch factors of $(G_A, \varphi)$ and $(G_B, \psi)$, respectively. Then
    \[ \{\lambda_{A}^+,\lambda_{A}^-\}=\{\lambda_{B}^+,\lambda_{B}^-\}. \]
\end{duplicate}

\begin{proof}
    The groups $(G_A, \varphi)$ and $(G_B, \psi)$ have isomorphic fibres by \cref{prop.profinite.Wn} \cref{prop.propfinite.Wn.1}, and by \cref{prop.profinite.Wn} \cref{prop.propfinite.Wn.3}, the character $\bar{\varphi} \colon H_A \to \Z$ is the pullback of $\bar{\psi} \colon H_B \to Z$ under a profinite isomorphism $\widehat{H}_A \to \widehat{H}_B$. Also, by assumption, $(H_A, \bar{\varphi})$ and $(H_B, \bar{\psi})$ are conjugacy separable free-by-cyclic groups. Hence by \Cref{Nielsen numbers equality}, the stretch factors associated to $(H_A, \bar{\varphi})$ and $(H_B, \bar{\psi})$ are equal. Thus by construction of $H_A$ and $H_B$, the stretch factors of $(G_A, \varphi)$ and $(G_B, \psi)$ are equal.
\end{proof}


 A \{universal Coxeter\}-by-cyclic group is said to be \emph{irreducible} if it admits irreducible monodromy.

\smallskip

\begin{duplicate}[\Cref{thmx:IrredCoxeterAlmostProfinite}]
    Let $G$ be an irreducible \{universal Coxeter\}-by-cyclic group. Then $G$ is almost profinitely rigid amongst the class of irreducible \{universal Coxeter\}-by-cyclic groups.
\end{duplicate}

\begin{proof}
    Let $G_A$ and $G_B$ be irreducible \{universal Coxeter\}-by-cyclic groups with index-two free-by-cyclic groups $H_A \leq G_A$ and $H_B \leq G_B$ as before. Suppose that there exists a profinite isomorphism $\widehat{G}_A \to \widehat{G}_B$, and for a fixed character $\psi \colon G_B \to \Z$, let $\varphi \colon G_A \to Z$ be the pul-back of $\psi$. By \cref{prop.profinite.Wn}, there exists an isomorphism $\widehat{H}_A \cong \widehat{H}_B$, and the induced character $\bar{\varphi} \colon H_A \to \Z$ is the pull-back of $\bar{\psi} \colon H_B \to \Z$. Hence by \Cref{thmx:hyperbolicity}, $H_A$ is hyperbolic if and only if $H_B$ is hyperbolic. 
    
    Suppose first that $H_A$ and $H_B$ are hyperbolic. Then they are virtually special by \cite{HagenWise2015} and thus conjugacy separable by \cite{Minasyan2006}. Now arguing as in the proof of \Cref{thmx:CoxeterStretchInvariance}, the sets of stretch factors of $(G_A, \varphi)$ and $(G_B, \psi)$ coincide. Thus by \Cref{finite}, $G_B$ can take on at most finitely many different isomorphism types. 

    Assume now that $H_A = K_A \rtimes_{\Phi_{K_A}} \Z$, and thus also $H_B = K_B \rtimes_{\Psi_{K_B}} \Z$, are not hyperbolic. Then $\Phi_{K_A}$ and $\Psi_{K_B}$ are not atoroidal. By \Cref{stretch factors of W_n auts}, $\Phi_{K_A}$ and $\Psi_{K_B}$ are irreducible. Hence by \Cref{geometric},  $\Phi_{K_A}$ and $\Psi_{K_B}$ are realised by a pseudo-Anosov homeomorphisms of punctured surfaces. It follows that $H_A$ and $H_B$ are the fundamental groups of compact hyperbolic 3-manifolds. The result now follows by \cite{Liu2023}. 
\end{proof}


\bibliographystyle{alpha}
\bibliography{refs.bib}

\end{document}
