\section{Experiments}\label{sec:experiments}
\subsection{Dataset}
We evaluate our method on various public available document datasets from journal articles, tables, to magazines.

\textbf{PubLayNet}~\cite{zhong2019publaynet} consists of 330K document layouts by matching XML representations of public PDF articles from PubMed Centra\textsuperscript{TM}. It has 5 semantic categories including \textit{Text, Title, List, Figure,} and \textit{Table}. Semantic elements on each layout are annotated by categorized bounding box in COCO~\cite{lin2014microsoft} format. Typically there is no overlapping between semantic units. We utilized its official splits: 335,703 for training, 11,245 for validation.

\textbf{DocBank}~\cite{li2020docbank} consists of 500K document layouts by weak supervision of articles available on the arXiv.com. It contains 12 categories including \textit{Abstract, Author, Caption, Equation, Figure, Footer, List, Paragraph, Reference, Section, Table,} and \textit{Title}. Its annotated bounding boxes are created by merging extraction results of text lines. In general, this dataset has more (number) and fragmented (size) annotated bounding boxes compared to PubLayNet~\cite{zhong2019publaynet}. And overlapping exists between semantic units (\textit{Paragraph} within \textit{Figure} etc.). We use 399,811 layouts for training, and 49,980 layouts for validation.

% \textbf{RICO}~\cite{deka2017rico} contains 66K mobile application UI designs spanning 27 semantic categories. The dataset is created by software coding hierarchy structure, thus it naturally has severe amount of overlapping among semantic bounding boxes.

\textbf{Magazine}~\cite{zheng2019content} has 4K magazine layouts classified into 6 semantic categories including \textit{Text, Image, Headline, Text-over-image, Headline-over-image,} and \textit{Background.} The dataset itself holds natural overlaps between categories.


\subsection{Comparisons}
We compare our approach with 3 related methods on document layout generation task, including LayoutVAE~\cite{jyothi2019layoutvae},  Gupta~\etal~\cite{gupta2021layouttransformer}, and VTN~\cite{arroyo2021variational}. We utilize the code provided by the author's repository of~\cite{gupta2021layouttransformer} for the first two approaches. In particular, we privilege the ground truth bounding box count to LayoutVAE and only train its BBoxVAE portion. For Gupta~\etal~\cite{gupta2021layouttransformer}, the original inference code use the first bounding box as input prior and sample the top-$k$ ($k=5$) predicted layout bounding boxes to enable diversity. We modify the inference code to generate from initial token rather than the first bounding box. During inference, we find that results' diversity vanishes fast with $k<5$. Thus the top-5 sampling is kept to ensure fairness. For VTN~\cite{arroyo2021variational}, we add the variational training scheme to the code of \cite{gupta2021layouttransformer}. For fairness, all methods are trained for the same epochs on each dataset. Their models are based on default settings in the code or recommended settings in the original papers.

\subsubsection{Quantitative Results} \label{sec: quantitative}
For all three datasets we list, we generate 1000 ( 391 for Magazine dataset since its validation dataset is small. ) document layouts by ours and other competitors. Then we compare the generated results with the same amount of real document layouts by series of metrics. Specifically, the LayoutVAE~\cite{jyothi2019layoutvae} is conditioned on an input bounding box count. Gupta~\etal~\cite{gupta2021layouttransformer} and VTN~\cite{arroyo2021variational} generation is fully random from initial token. 

Our set-by-set comparison is evaluated by 4 quantitative metrics. We already discuss the capability of \textit{Doc-EMD} in Section~\ref{intro docemd}, and we also include \textit{DocSim} as reference. In theory, these two metrics should indicate reverse pattern (Smaller \textit{DocSim} corresponds to larger \textit{Doc-EMD}. And our results illustrate this pattern.). \textit{Overlap} is the percentage of total overlapping area among the generated layout bounding boxes. Generally, less overlap indicates better performance a method achieves. However, we notice that there is a certain amount of reasonable overlapping existing in DocBank ( Legend texts in a figure are recognized as paragraph. Thus its bounding box overlaps with figure's bounding box, etc. ) and Magazine dataset ( The categories of \textit{Text-over-image}, and \textit{Headline-over-image} naturally overlap with \textit{Image} category. ). Thus reasonable amount ($<2\%$, etc.) of overlapping area will not affect the realism of generated layouts in terms of DocBank and Magazine dataset. Finally, \textit{Coverage} is the percentage of total bounding box area over the document extent area. The closer value to the real data one, the better performance a method achieves.

As shown in Table~\ref{tabpubley}, ~\ref{tabdocbank}, ~\ref{tabmagazine}, our method outperforms competitors in most metrics on 3 datasets. Specifically, our method achieves the best performance in \textit{DocSim} and \textit{Doc-EMD}, and the second in other two metrics. For DocBank, our method achieves the best performance in \textit{Coverage} and \textit{Doc-EMD}, and the second in \textit{DocSim}. For Magazine, our method dominates the \textit{DocSim}, \textit{Doc-EMD}, and \textit{Overlap}. Moreover, our \textit{Doc-EMD} metric keeps stable scalability and capability across three datasets, which verify the contributions in Section~\ref{intro docemd}.



\begin{table}[t]
\begin{center}
\caption{Benchmark performance on PubLayNet Dataset}\label{tabpubley}
\begin{tabular}{|l|c|c|c|c|}
\hline
Approaches &  \textit{DocSim} $\uparrow$ & \textit{Doc-EMD} $\downarrow$  & Overlap$\downarrow$ & Coverage\\
\hline
Layoutvae~\cite{jyothi2019layoutvae} & 0.129 & 0.191 &  2.02\% & \textbf{56.21\%}\\
 Gupta~\etal~\cite{gupta2021layouttransformer} & 0.137 & \underline{0.063} & \textbf{0.065\%} & 51.62\%\\
 VTN~\cite{arroyo2021variational} & \underline{0.141} & 0.068 & 0.083\% & 53.49\% \\
 Ours & \textbf{0.163} & \textbf{0.053}  & \underline{0.062\%} & \underline{55.30\%} \\
\hline
 Real Data & & & 0.026\% & 56.09\% \\
\hline
\end{tabular}
\end{center}
\end{table}





\begin{table}[!htbp]
\begin{center}
\caption{Benchmark performance on DocBank Dataset}\label{tabdocbank}
\begin{tabular}{|l|c|c|c|c|}
\hline
Approaches &  \textit{DocSim}$\uparrow$ & \textit{Doc-EMD}$\downarrow$ &  Overlap$\downarrow$ & Coverage\\
\hline
Layoutvae~\cite{jyothi2019layoutvae} & 0.087 & 0.592 & 2.02\% & 56.21\%\\
 Gupta~\etal~\cite{gupta2021layouttransformer} & 0.078 & 0.518 &  \textbf{0.56\%} & 44.01\%\\
 VTN~\cite{arroyo2021variational} & \textbf{0.096} & \underline{0.353} & \underline{0.61\%} & \underline{44.27\%} \\
 Ours & \underline{0.093} & \textbf{0.319} & 2.04\% & \textbf{45.49\%} \\
\hline
 Real Data & & & 0.45\% & 46.20\% \\
\hline
\end{tabular}
\end{center}
\end{table}


\begin{table}[!htbp]
\begin{center}
\caption{Benchmark performance on Magazine Dataset}\label{tabmagazine}
\begin{tabular}{|l|c|c|c|c|}
\hline
Approaches &  \textit{DocSim}$\uparrow$ & \textit{Doc-EMD}$\downarrow$ & Overlap & Coverage\\
\hline
Layoutvae~\cite{jyothi2019layoutvae} & \underline{0.260} & 0.143 &  \underline{2.23\%} & \underline{80.93\%}\\
 Gupta~\etal~\cite{gupta2021layouttransformer} & 0.176 & 0.227 &  12.6\% & 81.64\%\\
 VTN~\cite{arroyo2021variational} & 0.232 & \underline{0.138} &  5.29\% & \textbf{79.88\%} \\
 Ours & \textbf{0.302} & \textbf{0.117} &  \textbf{1.23\%} & 70.55\% \\
\hline
 Real Data & & & 1.36\% & 76.00\% \\
\hline
\end{tabular}
\end{center}
\end{table}


% \begin{table}
% \begin{center}
% \caption{Benchmark performance on RICO Dataset}\label{tabRICO}
% \begin{tabular}{|l|l|l|l|l|}
% \hline
% Approaches &  DocSim\uparrow & Doc-EMD\downarrow & Overlap & Coverage\\
% \hline
% Layoutvae~\cite{jyothi2019layoutvae} & 0.122 & 0.100 & \textbf{25.95\%} & \textbf{38.47\%}\\
%  Gupta et al.~\cite{gupta2021layouttransformer} & 0.165 & \textbf{0.023} & 33.27\% & 43.33\%\\
%  VTN~\cite{arroyo2021variational} & 0.145 & \underline{0.041} & 49.50\% & 46.47\% \\
%  Ours & 0.077 & 0.056 & 3.97\% & \underline{42.16\%} \\
% \hline
%  Real Data & & & 28.14\% & 36.64\% \\
% \hline
% \end{tabular}
% \end{center}
% \end{table}



\subsubsection{Qualitative Results}
Figure~\ref{fig:publaynet},~\ref{fig:docbank}, and~\ref{fig:magazine} show qualitative comparison results in PubLayNet, DocBank, and Magazine dataset, respectively. The visual quality indicates that our method is able to produce diverse and realistic document layouts across three datasets. 

For PubLayNet (Figure~\ref{fig:publaynet}), LayoutVAE~\cite{jyothi2019layoutvae} generates layouts that is poor in alignment, and containing noticeable overlaps. There are significant amount of "list" categorized bounding boxes appearing in most of the generated layouts (Figure~\ref{fig:publaynet}, the second to the rightmost column in LayoutVAE row), which is not realistic to open access academic papers that PubLayNet sampled from. The similar unrealistic pattern also happens in Gupta~\etal~\cite{gupta2021layouttransformer} (Figure~\ref{fig:publaynet}, the second column in Gupta row) and VTN~\cite{arroyo2021variational} (Figure~\ref{fig:publaynet}, the rightmost column in VTN row). This abnormity will not only reduce the performance of similarity evaluation in Section~\ref{sec: quantitative}, but also negatively impact the downstream tasks using the generated layouts such as the detection task we discussed in Section~\ref{sec: downstream task}. For Gupta~\etal~\cite{gupta2021layouttransformer} and VTN~\cite{arroyo2021variational}, their performance are similar quantitatively and qualitatively. We find that if we generate layouts from the initial token rather than inputting the first bounding box (default setting by original codes), the diversity and realism of the generated results are rather repeating the same pattern (Figure~\ref{fig:publaynet}, the leftmost, the second, and the forth column in VTN~\cite{arroyo2021variational} row), or poor in alignment and overlap (Figure~\ref{fig:publaynet}, Gupta~\etal~\cite{gupta2021layouttransformer} row). However, our method outperforms other methods and generates both realistic and well-aligned document layouts without category abnormity. Our method illustrates plausible capability in both single and double column document layouts. The quality of our results will also support better performance in downstream detection task.


\begin{figure*}[t]
\centering
\setlength{\tabcolsep}{3pt}
\tiny
\begin{tabular}{p{0.08mm}cccccc}
    \rotatebox[origin=l]{90}{\textbf{LayoutVAE~\cite{jyothi2019layoutvae}}}  &
    \includegraphics[width=0.15\textwidth]{figures/publaynet_competitors/layoutvae_40_result.png} &
    \includegraphics[width=0.15\textwidth]{figures/publaynet_competitors/layoutvae_1_result.png} &
    \includegraphics[width=0.15\textwidth]{figures/publaynet_competitors/layoutvae_85_result.png} &
    \includegraphics[width=0.15\textwidth]{figures/publaynet_competitors/layoutvae_38_result.png} &
    \includegraphics[width=0.15\textwidth]{figures/publaynet_competitors/layoutvae_48_result.png} &\\


    \rotatebox[origin=l]{90}{\textbf{Gupta~\etal~\cite{gupta2021layouttransformer}}}  &
    \includegraphics[width=0.15\textwidth]{figures/publaynet_competitors/gupta_sample_random_757.png} &
    \includegraphics[width=0.15\textwidth]{figures/publaynet_competitors/gupta_sample_random_661.png} &
    \includegraphics[width=0.15\textwidth]{figures/publaynet_competitors/gupta_sample_random_671.png} &
    \includegraphics[width=0.15\textwidth]{figures/publaynet_competitors/gupta_sample_random_717.png} &
    \includegraphics[width=0.15\textwidth]{figures/publaynet_competitors/gupta_sample_random_733.png} &\\

    \rotatebox[origin=l]{90}{\textbf{VTN~\cite{arroyo2021variational}}}  &
    \includegraphics[width=0.15\textwidth]{figures/publaynet_competitors/sample_random_190_0.png} &
    \includegraphics[width=0.15\textwidth]{figures/publaynet_competitors/sample_random_265_0.png} &
    \includegraphics[width=0.15\textwidth]{figures/publaynet_competitors/sample_random_489_0.png} &
    \includegraphics[width=0.15\textwidth]{figures/publaynet_competitors/sample_random_497_0.png} &
    \includegraphics[width=0.15\textwidth]{figures/publaynet_competitors/sample_random_508_0.png} &\\

    
    \rotatebox[origin=l]{90}{\textbf{Ours}} &
    \includegraphics[width=0.15\textwidth]{figures/publaynet_diffusion/15_result.png} &
    \includegraphics[width=0.15\textwidth]{figures/publaynet_diffusion/32_result.png} &
    \includegraphics[width=0.15\textwidth]{figures/publaynet_diffusion/41_result.png} &
    \includegraphics[width=0.15\textwidth]{figures/publaynet_diffusion/66_result.png} &
    \includegraphics[width=0.15\textwidth]{figures/publaynet_diffusion/20_result.png} &
    \includegraphics[width=0.08\textwidth]{figures/publaynet_diffusion/publaynet_legend.png} \\
     % &  & Generated PubLayNet & &
\end{tabular}%
    \caption{\textbf{Qualitative Results for PubLayNet.} Competitors are worse in alignment, repeat similar patterns, contain abnormal bounding box categories, or have noticeable overlaps. Our method appears to represent PubLayNet better.}
  \label{fig:publaynet}
\end{figure*}


DocBank is the largest and the most complex dataset (more categories, diverse-sized bounding boxes) in our comparison. In this case, LayoutVAE~\cite{jyothi2019layoutvae} is unable to provide reasonable document layouts. For Gupta~\etal~\cite{gupta2021layouttransformer} and VTN~\cite{arroyo2021variational}, most of generated results are unreal and repeating patterns of permuted "equation" and "paragraph" categories (Figure~\ref{fig:docbank}, the leftmost, middle, and right most column in Gupta~\etal~\cite{gupta2021layouttransformer} row; the middle column in VTN~\cite{arroyo2021variational} row). There is also improper permutations of "reference" and 'paragraph' bounding boxes (Figure~\ref{fig:docbank}, the second column in VTN~\cite{arroyo2021variational} row). Our method shows reasonable patterns of permuted "equation" and "paragraph" categories (Figure~\ref{fig:docbank}, the leftmost, and the rightmost column in Ours row). Moreover, our method is able to handle complex design of the cover page (Figure~\ref{fig:docbank}, the second column in Ours row). And all generated layouts achieve plausible alignment.



\begin{figure*}[!htbp]
\centering
\setlength{\tabcolsep}{3pt}
\tiny
\begin{tabular}{p{0.08mm}cccccc}
    \rotatebox[origin=l]{90}{\textbf{LayoutVAE~\cite{jyothi2019layoutvae}}}  &
    \includegraphics[width=0.15\textwidth]{figures/docbank_competitors/70_result.png} &
    \includegraphics[width=0.15\textwidth]{figures/docbank_competitors/89_input.png} &
    \includegraphics[width=0.15\textwidth]{figures/docbank_competitors/96_input.png} &
    \includegraphics[width=0.15\textwidth]{figures/docbank_competitors/97_result.png} &
    \includegraphics[width=0.15\textwidth]{figures/docbank_competitors/98_result.png} & \\

    \rotatebox[origin=l]{90}{\textbf{Gupta~\etal~\cite{gupta2021layouttransformer}}}  &
    \includegraphics[width=0.15\textwidth]{figures/docbank_competitors/gupta_sample_random_715.png} &
    \includegraphics[width=0.15\textwidth]{figures/docbank_competitors/gupta_sample_random_798.png} &
    \includegraphics[width=0.15\textwidth]{figures/docbank_competitors/gupta_sample_random_894.png} &
    \includegraphics[width=0.15\textwidth]{figures/docbank_competitors/gupta_sample_random_947.png} &
    \includegraphics[width=0.15\textwidth]{figures/docbank_competitors/gupta_sample_random_991.png} & \\

    \rotatebox[origin=l]{90}{\textbf{VTN~\cite{arroyo2021variational}}}  &
    \includegraphics[width=0.15\textwidth]{figures/docbank_competitors/vtn_sample_random_897_0.png} &
    \includegraphics[width=0.15\textwidth]{figures/docbank_competitors/vtn_sample_random_910_0.png} &
    \includegraphics[width=0.15\textwidth]{figures/docbank_competitors/vtn_sample_random_968_0.png} &
    \includegraphics[width=0.15\textwidth]{figures/docbank_competitors/vtn_sample_random_991_0.png} &
    \includegraphics[width=0.15\textwidth]{figures/docbank_competitors/vtn_sample_random_993_0.png} & 
    \multirow{2}{0.09\textwidth}{\includegraphics[width=0.08\textwidth]{figures/docbank_diffusion/docbank_legend.png}}\\

    \rotatebox[origin=l]{90}{\textbf{Ours}}  &
    \includegraphics[width=0.15\textwidth]{figures/docbank_diffusion/10_result.png} &
    \includegraphics[width=0.15\textwidth]{figures/docbank_diffusion/33_result.png} &
    \includegraphics[width=0.15\textwidth]{figures/docbank_diffusion/49_result.png} &
    \includegraphics[width=0.15\textwidth]{figures/docbank_diffusion/82_result.png} &
    \includegraphics[width=0.15\textwidth]{figures/docbank_diffusion/9_result.png} &\\     

\end{tabular}%

    \caption{\textbf{Qualitative Results for DocBank.} Competitors are poor in alignment, repeating similar patterns, containing abnormity of bounding box categories. Our method outperforms other competitors}
  \label{fig:docbank}
\end{figure*}

Magazine dataset is the smallest dataset. But it has the most diverse alignment among bounding boxes (1 to 4 text columns within a single page), because the dataset is based on magazine design rather than academic papers. In this case, our competitors are unable to handle complex alignment and result in severe unreasonable overlaps between bounding boxes (Figure~\ref{fig:magazine}, the first three rows). Note that there are natural overlaps in magazine dataset since two of the categories are defined to be on top of the image category. But most overlaps are between the same category (Figure~\ref{fig:magazine}, the middle column in VTN~\cite{arroyo2021variational} row, "text-over-image" category, etc.), or unreasonable overlaps (Figure~\ref{fig:magazine}, the second column in VTN~\cite{arroyo2021variational} row, several "text" rather than "text-over-image" bounding boxes are on top of the "image" bounding boxes, etc.). However, our method is able to handle complex text alignment (Figure~\ref{fig:magazine}, the forth and the rightmost column in Ours row), and also prevent unreasonable overlaps. Our method also outperforms quantitatively in Table~\ref{tabmagazine}.


\begin{figure*}[t]
\centering
\setlength{\tabcolsep}{3pt}
\tiny
\begin{tabular}{p{0.08mm}cccccc}
    \rotatebox[origin=l]{90}{\textbf{LayoutVAE~\cite{jyothi2019layoutvae}}}  &
    \includegraphics[width=0.15\textwidth]{figures/magazine_competitors/layoutvae_55_result.png} &
    \includegraphics[width=0.15\textwidth]{figures/magazine_competitors/layoutvae_59_result.png} &
    \includegraphics[width=0.15\textwidth]{figures/magazine_competitors/layoutvae_87_result.png} &
    \includegraphics[width=0.15\textwidth]{figures/magazine_competitors/layoutvae_92_result.png} &
    \includegraphics[width=0.15\textwidth]{figures/magazine_competitors/layoutvae_93_result.png} &\\

    \rotatebox[origin=l]{90}{\textbf{Gupta~\etal~\cite{gupta2021layouttransformer}}}  &
    \includegraphics[width=0.15\textwidth]{figures/magazine_competitors/gupta_sample_random_357.png} &
    \includegraphics[width=0.15\textwidth]{figures/magazine_competitors/gupta_sample_random_367.png} &
    \includegraphics[width=0.15\textwidth]{figures/magazine_competitors/gupta_sample_random_385.png} &
    \includegraphics[width=0.15\textwidth]{figures/magazine_competitors/gupta_sample_random_387.png} &
    \includegraphics[width=0.15\textwidth]{figures/magazine_competitors/gupta_sample_random_389.png} & \\

    \rotatebox[origin=l]{90}{\textbf{VTN~\cite{arroyo2021variational}}}  &
    \includegraphics[width=0.15\textwidth]{figures/magazine_competitors/vtn_wedding_0236.png_0_sample_random.png} &
    \includegraphics[width=0.15\textwidth]{figures/magazine_competitors/vtn_wedding_0288.png_0_sample_random.png} &
    \includegraphics[width=0.15\textwidth]{figures/magazine_competitors/vtn_wedding_0369.png_0_sample_random.png} &
    \includegraphics[width=0.15\textwidth]{figures/magazine_competitors/vtn_wedding_0496.png_0_sample_random.png} &
    \includegraphics[width=0.15\textwidth]{figures/magazine_competitors/vtn_wedding_0605.png_0_sample_random.png} & \\

    \rotatebox[origin=l]{90}{\textbf{Ours}} &
    \includegraphics[width=0.15\textwidth]{figures/magazine_diffusion/49_result.png} &
    \includegraphics[width=0.15\textwidth]{figures/magazine_diffusion/10_result.png} &
    \includegraphics[width=0.15\textwidth]{figures/magazine_diffusion/89_result.png} &
    \includegraphics[width=0.15\textwidth]{figures/magazine_diffusion/92_result.png} &
    \includegraphics[width=0.15\textwidth]{figures/magazine_diffusion/59_result.png} &
    \includegraphics[width=0.14\textwidth]{figures/magazine_diffusion/magazine_legend.png} \\
\end{tabular}%

    \caption{\textbf{Qualitative Results for Magazine.} Competitors are poor in alignment, or with unreasonable overlaps. Our method outperforms other competitors.}
  \label{fig:magazine}
\end{figure*}




\subsection{Layout Detection Task} \label{sec: downstream task}
The best way to illustrate benefits of a generative networks is to utilize its results for downstream tasks, especially for data augmentation. Document layout detection task is a subdomain of Optical Character Recognition (OCR). The detection network is trained to segment and label each layout bounding box in the given input document images. Practically, the annotation of document layouts is tedious and time-consuming, and the accuracy of the ground truth annotation contains inevitable ambiguity. To solve these problems, we can augment the dataset by generating document images with our generated document layout designs and labels. In this way, the quality of the annotation is ensured. And we will have unlimited amount of augmented dataset for better subtask model training and evaluation.

Similar to~\cite{arroyo2021variational}, as shown in Figure~\ref{fig:detection}, we develop a synthetic document image generation framework. First, we utilize our pretrained diffusion networks to randomly generate the same amount of document layouts as original training dataset (PubLayNet). Second, for each bounding box we generate, we find the nearest matching bounding box in the training dataset by matching categories, aspect ratio, size, etc. Then we slice the corresponding group of pixels from the original images in PubLayNet dataset, to synthetically mosaic a document image. Finally, we train a faster R-CNN model~\cite{ren2015faster} as our document layout detector. We compare the detection performance with the one trained by our competitor results and the original dataset for evaluation.

In Table~\ref{tabdetection}, we show the mean average precision (mAP) at IoU = 0.5. The values for VTN~\cite{arroyo2021variational} and PubLayNet are reported by its original paper. Our diffusion-based generation method enable plausible detection accuracy and outperforms VTN.


\begin{figure*}[t]
  \includegraphics[width=\linewidth]{figures/downstream_task.png}
  \caption{\textbf{Synthetic generation of document images.} Utilize our method to create a training dataset based on PubLayNet images for a layout detector. We use our generated layouts, find the most similar bounding box in real dataset (by matching category, aspect ratio, size, etc.). Then crop pixels in the original images and render a new image dataset.} 
   \label{fig:detection}
\end{figure*}


\begin{table}[!htbp] 
\begin{center}
\caption{Detection accuracy comparison between the detector trained by synthetic generated layouts, and by original PubLayNet.}\label{tabdetection}
\begin{tabular}{|l|c|c|c|}
\hline
& Ours & VTN~\cite{arroyo2021variational} & PubLayNet\\
\hline
mAP (IoU=0.5) & 0.795&  0.769 & 0.9646\\
\hline
\end{tabular}
\end{center}
\end{table}




\begin{table}[h]
\begin{center}
\caption{Ablations on Magazine Dataset}\label{ablations}
\begin{tabular}{|l|c|c|c|c|}
\hline
Ablations &  \textit{DocSim}$\uparrow$ & \textit{Doc-EMD}$\downarrow$ & Overlap & Coverage\\
\hline
 lr = 0.0001, steps = 500 & 0.156 & 0.315 & 5.34\% & 79.80\% \\
 lr = 0.0001, steps = 1000 & 0.203 & 0.245 & 3.67\% & \textbf{75.42\%} \\
 lr = 0.0002, steps = 2000 & 0.282 & 0.172 & 1.56\% & 72.34\% \\
 lr = 0.00001, steps = 2000 & 0.274 & 0.133 & 2.33\% & 71.21\% \\
 Ours (lr = 0.0001, steps = 2000) & \textbf{0.302} & \textbf{0.117} &  \textbf{1.23\%} & 70.55\% \\
\hline
 Real Data & & & 1.36\% & 76.00\% \\
\hline
\end{tabular}
\end{center}
\end{table}



\subsection{Ablations}
In our model training, we conduct an ablation study on both learning rate and diffusion steps, as shown in Table~\ref{ablations}. In general, more diffusion steps will improve network performance. But more diffusion steps also lead to extremely longer training time. We found that more diffusion steps above 2000 have no explicit benefits to the performance. Thus we choose diffusion steps as 2000 for all our trainings on Magazine and other datasets. Meanwhile, a proper learning rate is also the key to good performance. Though we found that diffusion model performance is quite robust to different learning rate. An arbitrary learning rate may still negatively influence the network performance.
