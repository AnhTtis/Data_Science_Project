% This is samplepaper.tex, a sample chapter demonstrating the
% LLNCS macro package for Springer Computer Science proceedings;
% Version 2.20 of 2017/10/04
%
% \RequirePackage{amsthm}
\documentclass[runningheads]{llncs}
%
% \usepackage{amsthm}
\usepackage{amsmath}
\usepackage{graphicx}
\usepackage{textcomp}
\usepackage[dvipsnames]{xcolor}
\usepackage{amssymb}
\usepackage{comment}
\usepackage{multirow}
\usepackage{adjustbox}
\usepackage{authblk}
% Used for displaying a sample figure. If possible, figure files should
% be included in EPS format.
%
% If you use the hyperref package, please uncomment the following line
% to display URLs in blue roman font according to Springer's eBook style:
% \renewcommand\UrlFont{\color{blue}\rmfamily}
\newcommand{\etal}{\textit{et al.}}

\begin{document}
%
\title{Diffusion-based Document Layout Generation}
%
%\titlerunning{Abbreviated paper title}
% If the paper title is too long for the running head, you can set
% an abbreviated paper title here
%
\author{Liu He\inst{1}\thanks{Work done while a research intern at Microsoft Cloud and AI.} \and
Yijuan Lu\inst{2}\and
John Corring\inst{2}\and
Dinei Florencio\inst{2}\and
Cha Zhang\inst{2}}
%
\authorrunning{He et al.}
% First names are abbreviated in the running head.
% If there are more than two authors, 'et al.' is used.
%
\institute{Purdue University, West Lafayette IN 47906, USA\\
\email{he425@purdue.edu}\\
\and Microsoft Cloud and AI, Bellevue WA 98004, USA\\
\email{\{yijlu,jocorrin,dinei,chazhang\}@microsoft.com}}
%
\maketitle              % typeset the header of the contribution
%



\begin{abstract}
% \vspace{-1em}
The diffusion-based generative models have achieved remarkable success in text-based image generation. However, since it contains enormous randomness in generation progress, it is still challenging to apply such models for real-world visual content editing, especially in videos. 
In this paper, we propose \texttt{FateZero}, a zero-shot text-based editing method on real-world videos without per-prompt training or use-specific mask. 
\RM{Specifically, different from a pipeline of two independent inversion and then generation stages, we find the intermediate attention maps during inversions store better structure and motion information. We thus reform them to temporally casual attention and replace them in the generation progress. To further reduce the unnecessary semantic leakage of source video and enhance the editing quality, we then remix the temporally casual attentions via the cross-attention features of the source prompt as the mask.}
To edit videos consistently, we propose several techniques based on the pre-trained models. Firstly, in contrast to the straightforward DDIM inversion technique, our approach captures intermediate attention maps during inversion, which effectively retain both structural and motion information. These maps are directly fused in the editing process rather than generated during denoising. To further minimize semantic leakage of the source video, we then fuse self-attentions with a blending mask obtained by cross-attention features from the source prompt. Furthermore, we have implemented a reform of the self-attention mechanism in denoising UNet by introducing spatial-temporal attention to ensure frame consistency.
Yet succinct, our method is the first one to show the ability of zero-shot text-driven video style and local attribute editing from the trained text-to-image model. We also have a better zero-shot shape-aware editing ability based on the text-to-video model~\cite{tuneavideo}. \RM{Besides video, our unified method also achieves state-of-the-art performance in zero-shot image editing.\chenyang{Need exp or remove the zero-shot image}} Extensive experiments demonstrate our superior temporal consistency and editing capability than previous works.
% The code will be released.
% \chenyang{emphasize: our observation at inversion time} \xiaodong{replacing the bold part to the actual pipeline: \textbf{Specifically, we work on replacing and mixing the attention maps between the inversion and generation since the self-attention map keeps the structure of the original natural image and the cross-attention is semantic-related, after remixing, we replace them in the corresponding generation steps for denoising.}}
% \footnote{Since there is no general video diffusion model is publicly available, we use one-shot video generation method~(Tune-A-Video~\cite{tuneavideo}) as the pretrained video diffusion model for zero-shot video editing\xiaodong{can be removed if we actually zero-shot on video}.}.
\end{abstract}
\section{Introduction}

The ability to reason about plans is critical for performing long-horizon tasks \citep{erol1996hierarchical, sohn2018hierarchical, sharma-etal-2022-skill}, compositional generalization \citep{corona-etal-2021-modular} and generalization to unseen tasks and environments \citep{shridhar2020alfred}.
Consider a simple long-horizon planning scenario where a robot is tasked with preparing a meal and serving it on the table. 
This presents a non-trivial planning problem since the agent needs to understand the sequence of operations required to perform the task and search for the relevant objects in the unfamiliar environment by interacting with various objects. %



Large language models have been recently shown to possess commonsense knowledge about the world such as object affordances and physical dynamics \citep{ouyang2022training,chowdhery2022palm}.
Early approaches considered text based environments and fine-tuned PLMs to predict actions given the history of past observations and actions \citep{jansen-2020-visually,micheli-fleuret-2021-language,yao-etal-2020-keep}.
Recent work has used this ability to reason about plans from text instructions in simulated household environments with simplifying assumptions such as text-only environment observations or feedback \citep{huang2022language,ahn2022can,li2022pre,logeswaran-etal-2022-shot}.


We focus on \emph{visually grounded planning} with PLMs --- the ability to adapt plans based on interaction and visual feedback from the environment.
While PLMs have strong planning commonsense priors, predictions from a PLM may not be directly realizable in the environment since the observation and action spaces are unknown.
This requires \emph{grounding} the PLM in the environment and adapting it to observe visual feedback, which is highly non-trivial.
Some prior works assume the availability of a pre-trained affordance function \citep{ahn2022can} or a success detector \citep{mirchandani2021ella}.
Notably, SayCan \citep{ahn2022can} completely decouples the PLM from observation information by selecting actions that have both high affordability (through a pre-trained affordance model) and high PLM likelihood.
Although this partially addresses the grounding problem, the use of visual feedback for action affordance alone is limited.
Often an agent must choose one of many affordable actions using information from observations.
For example, a driving agent should re-navigate and possibly turn around when encountering a ``road closed'' sign, but both turning around and driving forward are indistinguishable to SayCan because they are both affordable and the PLM is blind to observations.

Another workaround explored in prior work is translating the information in the visual observations to text using a pre-trained captioning system \citep{shridhar2021alfworld,huang2022language}.
However, it can be difficult to faithfully describe an image in words and information is lost in this inherently noisy process, which limits the information available to the planner.



Recent work shows that PLMs can be adapted for various natural language tasks by inserting tunable embeddings or soft prompts at the input of the PLM (also called prompt tuning or prefix tuning)~\citep{li-liang-2021-prefix,lester-etal-2021-power}.
This approach also extends to multi-modal understanding tasks such as image captioning \citep{mokady2021clipcap} and VQA \citep{tsimpoukelli2021multimodal} where images are encoded as soft prompts and finetuned for the target task.
Transformer based architectures have also been successfully applied to offline Reinforcement Learning in recent work \citep{chen2021decision,janner2021offline,li2022pre,reid2022can}.

Taking inspiration from these works, we propose the simple approach of embedding visual observations (`visual prompts') and \textit{directly inserting them as PLM input embeddings}.
The visual encoder and PLM are jointly trained for the target task, an approach we call \textbf{\oursfull}~(\ours).
By teaching the PLM to use observations for planning in an end to end manner, we remove the dependency on external data such as captions and affordability information that was used in prior work.
We show that this simple approach performs better than prior PLM-based planning approaches on two embodied planning benchmarks based on ALFWorld~\citep{shridhar2021alfworld} and Virtualhome~\cite{puig2018virtualhome}.



\section{Terminology}
 Already in 1936, Merton~\cite{merton1936unanticipated} coined the term \emph{unanticipated  consequences} to describe unforeseen, desirable or undesirable outcomes of one's action. Today, most researchers refer to unforeseen outcomes (the results of policies, technologies, or other ``purposive social actions''~\cite{merton1936unanticipated}) as \emph{unintended consequences}, though some have suggested that this term conflation has caused a loss of nuance~\cite{huntington1971change, parvin2020unintended}. As we show in ~\autoref{fig:term}, Merton's original term of \emph{unanticipated consequences} suggests that such consequences are always unintended. In contrast, \emph{unintended consequences} can be either unanticipated or anticipated. Parvin and Pollock have therefore argued that this lack of precision in terminology may lead people ``to abdicate responsibility for the perfectly foreseeable consequences of particular decisions''~\cite[p.323]{parvin2020unintended}. Hence, the use of the term \emph{unintended consequences} may reduce accountability: ``Phenomena described as unintended consequences are deemed too difficult, too out of scope, too out of reach, or too messy to have been dealt with at any point in time before they created problems for someone else. The descriptive approach works as a defensive and dismissive strategy''~\cite[p.322]{parvin2020unintended}. 
 

 \label{terminology}
 \begin{figure}[h!]
    \centering
    \includegraphics[width=0.4\textwidth]{figures/terminology.pdf}
        \caption{Terminology of anticipated, intended, unintended, and unanticipated consequences. \rr{Unintended consequences can be either unanticipated or anticipated. }}
    \label{fig:term}
    \Description{A Venn diagram with two intersecting circles labeled "unanticipated" and "intended." The intersection is labeled "anticipated". The area including the circle labeled "unanticipated" but excluding the area of the "anticipated" intersection is labeled "unintended."}
\end{figure}
 
 In this paper, we use the term \emph{unintended consequences (UCs)} to purposefully broaden the discussion to include both anticipated and unanticipated, positive or negative unintended side effects of technology on society. Our definition includes consequences that the instigators of an action (i.e., researchers and/or technology innovators) may not have addressed but could have foreseen. While these consequences can be positive or negative in nature (and oftentimes have different effects on a population), our work is inherently oriented towards considering negative UCs more than positive ones. In the remainder of this paper, we use ``technology'' for digital technology, such as hardware devices or software systems. We broadly refer to ``society''  at a regional, national, or international level.
 

\section{Related Work}
 
 Our work draws upon prior work studying values and ethics in digital technology~\cite{Shilton2018ValuesAE} and is informed by discussions about the effects of digital technology on society in fields such as philosophy~\cite{moor1985computer, johnson1985computer}, STS~\cite{winner2017artifacts, winner2010whale, Klein2002TheSC}, social informatics~\cite{kling1996computerization}, feminism~\cite{Bardzell2010FeministHT,haimson2016constructing} and postcolonial theories~\cite{irani2010postcolonial}. We start by showing how researchers in these fields have long discussed various societal effects of technology before outlining the methods and approaches researchers and practitioners have developed for mitigating unintended consequences. 
 
 \paragraph{Critiques of Technology}
 
 Prior work in STS, and later in HCI, has provided critical analyses of the risks and benefits of technology in society since at least the 1960s~\cite{sveiby2009unintended, Shilton2018ValuesAE}. According to STS scholar Winner, technology ``embodies specific forms of power and authority''~\cite{winner2017artifacts} and technologists should ``pay attention not only to the making of physical instruments and processes [...], but also the production of psychological, social, and political conditions as a part of any significant technical change''~\cite{winner2010whale}. 

 \rr{Work on the risks of technology has been published on a broad range of topics, including the Internet ~\cite{kraut1998internet}, health care information technologies~\cite{harrison2007unintended,ash2007categorizing}, mobile phones~\cite{reyns2013unintended,Moser:2016}, smart technologies~\cite{machidon2018societal}, machine learning~\cite{cabitza2017unintended}, and social media ~\cite{del2016spreading, starbird2019disinformation,starbird2017examining}.} 
 

 The examples above provide a critical lens of the role of technology in society and caution about the unknown and differential effects on societies. Historically, however, most innovation research has focused on desirable and intended consequences~\cite{Rogers1976NewPA, sveiby2009unintended}. In 2009, Sveiby and colleagues suggested that this focus could potentially be due to a ``pro-innovation bias among researchers and vested interests of funding agencies''~\cite{sveiby2009unintended}. Our literature review did not reveal whether this bias has changed in the years since. 

 However, we found many recent calls for more accountability for research innovations~\cite{metcalf2019owning,friedman2019value,Hecht2021ItsTT}. Several prominent computing conferences --- such as the Conference on Neural Information Processing System (NeurIPS ~\cite{NeurIps2021}), Annual Meetings of the Association for Computational Linguistics (ACL) ~\cite{ACL2022}, and the ACM Conference on Intelligent User Interfaces (IUI)~\cite{IUI2022} --- have begun to experiment with ways to encourage or even require researchers to state both the positive and negative potential implications of their work in all paper submissions. \rr{Recent work has made several suggestions for such broader impact statements based on an analysis of these statements in NeurIPS conference proceedings~\cite{nanayakkara2021unpacking, Liu2022ExaminingRA, Ashurst2022AIES}}. After requiring all submissions to contain a section describing the impact of the work, NeurIPS has since transitioned towards a checklist system that offers additional guidance and adaptability ~\cite{NeurIps2021}. 
%
% In 2018, the ACM Code of Ethics and Professional Conduct was revised for the first time since 1992 to address the significant advances in computing technology and the growing pervasiveness of computing in all aspects of society~\cite{Gotterbarn2018ACMCO}. There are also calls for researchers within different computing communities to accurately report the design considerations of their datasets and models~\cite{gebru2021datasheets,mitchell2019model,bender2018data,rogers2021changing} as well as the tasks that they are working on~\cite{Lindley2017ImplicationsFA,mohammad2021ethics}. 

 The HCI community has been raising awareness of UCs of computing research through dedicated publication tracks (e.g., Critical Computing at CHI) and workshops~\cite{conseuqences, unethically}. In particular, a CHI 2021 workshop explored how HCI researchers might think about and report potential negative consequences stemming from their research~\cite{conseuqences}. HCI researchers have also advocated for changes to the peer review process to reduce negative impacts of research innovations, suggesting that reviewers should routinely require that papers and proposals discuss potential adverse effects~\cite{Hecht2021ItsTT}. 
%
 Given these calls for examining the societal impacts of technology, our work explores whether researchers adopt any methods for anticipating the UCs of their own work. 

\paragraph{Anticipating and Mitigating Unintended Consequences of Technology}
 UCs are often dismissed as unavoidable because anticipating what may happen in the future can be hard~\cite{parvin2020unintended} \rr{and uncertain~\cite{Nanayakkara2020AnticipatoryEA}.} However, HCI researchers have developed ethics-focused design methods to ensure the inclusion of various stakeholders in the design process (for an overview  see~\cite{chivukula2021surveying}). One prominent example is the value-sensitive design (VSD) approach by Friedman, Kahn, and Borning~\cite{friedman2008value}, which can aid in understanding technology, its human value, and its context of use. The process aims to help product teams and researchers identify alternative approaches that better uphold their chosen values while accommodating the same constraints. 
 A number of recent proposals have sought to bridge the gap between theory and implementation by creating toolkits meant for brainstorming about a product's potential societal impacts. For example, the Envisioning Cards~\cite{EnvisioningCards} present the VSD concepts in a clear and modular fashion~\cite{Nathan2008}. In addition, stakeholder tokens also support a VSD stakeholder analysis~\cite{Yoo:2018}. Prior work in HCI has also recommended the use of Tarot Cards of Tech~\cite{Menking:2019} and the Value Cards~\cite{shen2021value} for anticipating potential UCs of specific design choices. 
 
 \rrr{Another approach for considering possible societal impacts is through design fiction~\cite{bleecker2022design, Baumer2018WhatWY}. As a form of speculative design~\cite{Lindley2016PushingTL, Dunne2013SpeculativeED},  design fiction creates a fictional future world to think through sociotechnical issues that have relevance and implications for the present~\cite{wong2017eliciting, lindley2017implications}. This practice has been used to reflect on potential downsides of public data~\cite{fiesler2019ethical}, technology design~\cite{Harrington2022AllTY}, and research prototypes~\cite{soden2019chi4evil}. More recent work developed the design fiction memos method to explore how UX practitioners engage with ethical issues and social impact in their work~\cite{wong2021using}.} 
 
 \rrr{The existing approaches to consider societal implications, however, were often assumed to be effective in practice~\cite{gray2019ethical} and might be difficult to evaluate ~\cite{Baumer2018WhatWY}. While the toolkits often target designers and practitioners as users~\cite{gray2019ethical}, applying them for research projects may pose additional complexities. We extend this line of work by inquiring into whether computer science researchers are aware of and proactively incorporate these tools in their research process. Our work also explores future design implications to support researchers to consider UCs in their research process.
 }
 %These existing approaches are not without shortcomings.} For one, they are primarily designed for practitioners; applying them to research projects \rr{in practice} may be challenging~\cite{gray2019ethical}. Nevertheless, they could serve as important tools for considering UCs. In the present work, we explore whether computer science researchers are aware of and incorporate any of these tools in their research process.}  
 
\paragraph{Reacting to Unintended Consequences of Technology}
 Not much work has investigated how practitioners and researchers react to UCs in practice. Kling's book on ``Computerization and Controversy'' shows how the power dynamics between programmers and their employers can prevent discussions of potential ethical issues in the products they work on~\cite{kling1996computerization}. As a result, computer science professionals may feel discouraged when reacting to potential or known UCs. 
 \rr{Recently, an interview study showed that the Deepfake open source contributors felt unable to control downstream uses of their software, given the core principle of open source~\cite{widder}.}
 %More recently, some researchers were found to be ``nonchalant'' toward the broader impact statement and perceive it as ``burden''~\cite{Abuhamad2020LikeAR}. 
%
 %\rr{It could well be that similar power dynamics prevent anticipating and reacting to UCs in academic settings.} 
%
 Researchers have occasionally written public posts in response to public backlash or negative press after deploying a research project~\cite{jiang_2021, openai_2022}, but it is unclear whether they also do so when an incident is less public or when it has only been anticipated (but has not materialized). We fill this gap in prior work by studying whether and how academic computer scientists react if they discover that their work may have UCs. 

\begin{figure*}[t!]
\includegraphics[width=1.0\linewidth, trim={0 0.3cm 0 0.1cm}, clip]{figures/architecture/architecture.pdf}
\vspace{-15pt}
\caption{
\textbf{Point2Vec pre-training.}
Our model divides the input point cloud into %
point patches using farthest point sampling (FPS) and $k$-NN aggregation.
We obtain patch embeddings by applying a mini-PointNet\,\colorsquare{m_pointnet} to each point patch (\emph{right}).
The teacher Transformer encoder\,\colorsquare{m_green} infers a contextualized %
representation for all patch embeddings which, after normalization and averaging over the last $K$ Transformer layers, serve as training targets.
The student's input is a masked view on the input data, \ie we randomly mask out a ratio of patch embeddings and only pass the remaining embeddings into the student Transformer encoder\,\colorsquare{m_blue}.
After applying a shallow decoder\,\colorsquare{m_red} on the outputs of the student, padded with learned mask embeddings\,\protect\maskembedding{}, we train the student and decoder to predict the latent teacher representation of the patch embeddings.
\vspace{-10pt}
}
\label{fig:model}
\end{figure*}
\section{Method}

The aim of this work is to unlock the full potential of data2vec-like\,\cite{baevski2022data2vec} pre-training on point clouds by addressing point cloud specific challenges.
To achieve this, we first summarize the technical concepts of data2vec (\refsec{method_d2v}) and show how to learn rich representations on point clouds using data2vec pre-training (\refsec{method_d2v_pcl}).
Finally, we propose \name{}, which accounts for the point cloud specific limitations of data2vec (\refsec{method_p2v}).

\subsection{Data2vec}\label{sec:method_d2v}
Data2vec\,\cite{baevski2022data2vec} is designed to pre-train Transformer-based models, which involve a feature encoder that maps the input data to a sequence of embeddings.
These embeddings are subsequently passed to a standard Transformer encoder to generate the final latent representations.
During pre-training, two versions of the Transformer encoder are kept: a \emph{student} and a \emph{teacher}.
The teacher is a momentum encoder, \ie its parameters $\Delta$ track the student's parameters $\theta$ by being updated after each training step according to an exponential moving average (EMA) rule\,\cite{caron2021dino, baevski2022data2vec, grill2020BYOL, he2020moco}: $\Delta \leftarrow \tau \Delta + (1-\tau)\theta$,
where $\tau \in [0,1]$ is the EMA decay rate.
The teacher provides the training targets, which the student predicts given a corrupted version of the same input.

In a first step, the teacher encodes the uncorrupted input sequence.
The training targets are then constructed by averaging the outputs of the last $K$ blocks of the teacher, which are normalized beforehand to prevent a single block from dominating the sum.
Due to the self-attention layers, these targets are \emph{contextualized}, \ie they incorporate global information from the whole input sequence.
This is an important difference to other masked-prediction methods such as BERT\,\cite{devlin2018bert} and MAE\,\cite{he2022mae}, where the targets only comprise local information, \eg a word or an image patch. %

The student is given a masked version of the same input, where some of the embeddings in the input sequence are substituted by a special learned \emph{mask embedding}. %
The student's task is to predict the targets corresponding to the masked parts of the input.
The model is trained by optimizing a Smooth L1 loss on the regressed targets. %







\subsection{Data2vec for Point Clouds}\label{sec:method_d2v_pcl}

To apply data2vec to point clouds, we utilize the same underlying model as Point-BERT\,\cite{yu2021pointbert} and Point-MAE\,\cite{pang2022pointmae}.
This model is well suited for data2vec pre-training: it extracts a sequence of patch embeddings from the input point cloud and feeds it to a standard Transformer encoder.
For downstream tasks, we append a task-specific head to the Transformer encoder (\refsec{experiments}).
Next, we describe the point cloud embedding and the Transformer in detail and conclude with a summary of data2vec for point clouds.


\parag{Point Cloud Embedding.}
First, we sample $n$ center points from the input point cloud using farthest point sampling (FPS)\,\cite{qi2017pointnetplusplus}.
Grouping the center points' $k$-nearest neighbors ($k$-NN) in the point cloud yields $n$ contiguous \emph{point patches}, \ie sub-clouds of $k$ elements.
Next, we normalize the point patches by subtracting the corresponding center point from the patch's points.
This untangles the positional and the structural information.
To account for the permutation-invariant property of point clouds, we employ a mini-PointNet\,\cite{qi2016pointnet} (\reffig{model}, \emph{right}) that maps each normalized point patch to a \emph{patch embedding}.

The mini-PointNet involves the following steps:
First, we map each point of a patch to a feature vector using a shared MLP.
Then, we concatenate max-pooled features to each feature vector.
The resulting feature vectors are then passed through a second shared MLP and a final max-pooling layer to obtain the patch embedding.

\paragraph{Transformer Encoder.}
The central component of the model is a standard Transformer encoder.
The patch embeddings form the input sequence to the Transformer encoder.
Since the point patches are normalized, the patch embeddings carry no positional information;
therefore, a two-layer MLP maps each center point to a position embedding, which is then added to the corresponding patch embedding.
Due to the special importance of positional information in point clouds, the position embeddings are added again before each subsequent Transformer block to ensure that the positional information is incorporated at every step of the encoding process.

\paragraph{\emakefirstuc{\datavec{}}.}

To establish a baseline, we apply the unmodified data2vec approach to the previously described underlying model of Point-BERT and Point-MAE.
Going forward, we will refer to this approach as \datavec{}.


\subsection{\emakefirstuc{\name{}}}\label{sec:method_p2v}
In \reffig{model}, we present the complete pipeline of our \name{} model.
Directly applying data2vec to point cloud data without modifications is not optimal, as the position embeddings are also added to the mask embeddings, revealing the overall shape of the point cloud to the student.
As positions are the only features for point clouds, this makes the masking far less effective, as noted by Pang \etal \cite{pang2022pointmae} in the context of masked autoencoders.

To solve this issue, we adopt an approach inspired by MAE\,\cite{he2022mae}, where we only feed the non-masked embeddings to the student\,\colorsquare{m_blue}.
A separate decoder\,\colorsquare{m_red}, implemented as a shallow Transformer encoder, takes the output of the student and the previously held-back masked embeddings\,\maskembedding{} as input and predicts the training targets.
In contrast to \datavec{}, this approach does not suffer from leaking positional information from the masked-out point patches to the student.
Moreover, utilizing an MAE-inspired setup provides additional benefits:
First, the student is more computationally efficient, as it only needs to process the non-masked embeddings.
Second, the model's inputs during fine-tuning are more similar to those during pre-training because the inputs during pre-training are no longer dominated by masked embeddings which are absent during fine-tuning.
This likely makes the learned representations more transferable to downstream tasks.

\section{Experiments}
\label{sec:experiments}

\subsection{Setup}
\textbf{Datasets.} We evaluate RFFR with four challenging datasets specifically designed for deepfake detection. We adopt the high quality (HQ) version of Faceforensics++ (FF)~\cite{ff} for training our deepfake detector. Faceforensics++ includes videos of real faces as well as four subsets of fake faces, each manipulated with a different algorithm, namely Deepfakes (DF), Face2Face (F2F), FaceSwap (FSW) and NeuralTextures (NT). We also utilize the test set of Celeb-DF~\cite{celeb-df} and DFDC~\cite{dfdc} for evaluating the cross-dataset performance of our model. Finally, in addition to real faces of Faceforensics++, we adopt the real face images from ForgeryNet (FN)~\cite{forgerynet} for learning RFFR, which helps improve representation learning with additional data.

\textbf{Implementation Details.} We extract the frames from all video datasets and use RetinaFace~\cite{retinaface} to detect and align the faces. All images are scaled to the size of $224 \times 224$. For our RFFR model, we adopt a base version of Masked Autoencoder (MAE)~\cite{mae} and train it on real faces with a batch size of $128$. Following MAE, we set the learning rate at $7.5 \times 10^{-5}$ and adjust it with a schedule with warmup and cosine decay. By default, we train this model with the real faces from both FF~\cite{ff} and FN~\cite{forgerynet}. 

For training the deepfake detector, we divide each image with $k = 4$ (Refer to Appendix for the motivation of choosing $k$). Each block enters the classifier with a probability of $p = 0.25$, and the residual images are amplified by $\alpha=4$. No data augmentation is applied to the images. We initialize both branches of Vision Transformer with ImageNet-pretrained weights and train them with a learning rate of $2 \times 10^{-5}$. During testing, we iteratively mask and restore all blocks to obtain a full residual image for the detector to process. We evaluate the testing results with AUC (Area Under Curve). 

\subsection{Cross-domain performance evaluation}
In this section, we test the performance of our RFFR-based deepfake detector with cross-manipulation and cross-dataset evaluations. 

\textbf{Cross-manipulation evaluations.} We train our deepfake detector on each subset of Faceforensics++ and test on all four subsets to demonstrate our model's ability to identify different manipulations, including those not seen during training. \emph{We adopt the HQ version of FF for both training and testing, and only use one frame every video for testing.} We compare our results with state-of-the-art image-based methods Multi-Attention~\cite{multiatt}, DCL~\cite{dcl}, RECCE~\cite{recce} and UIA-ViT~\cite{uia}. We ran the public code of RECCE and UIA-ViT to produce results under the same setting.

In~\cref{tab:cross-manipulation}, we show that our method outperforms the state-of-the-art methods under most settings, with a maximum improvement of $10.25\%$ (F2F $\rightarrow$FSW). Meanwhile, our model remains effective under the four intra-domain settings, which are shown in gray. The method tends to slightly underperform when trained on NeuralTextures, likely because its manipulation patterns only exist in certain small regions, and may be neglected during our block sampling. Nevertheless, compared to existing methods, our deepfake detector yields much better overall performances. 

\begin{table}[t]
\setlength\tabcolsep{4.5pt} 
\caption{Cross-manipulation performances in terms of AUC(\%) compared with previous methods. Classifiers are trained on one subset of FF and tested on all four subsets. Intra-domain results are marked in gray. We ran the public code of methods marked with "*" to produce results under identical settings \emph{(HQ for training and single frames for testing).}}
\vspace{-1.5em}
\label{tab:cross-manipulation}
\begin{center}  
\scalebox{0.80}{
\begin{tabular}{c|l|cccc|c}
\toprule
Training &\multirow{2}*{Method} & \multicolumn{4}{c|}{Test data} & \multirow{2}*{Avg} \\
\cmidrule(lr){3-6}
     data  &            ~                   & DF    & F2F   & FSW   & NT    & ~   \\
     
\midrule
\multirow{5}*{DF}
& MultiAtt~\cite{multiatt} & \cellcolor{Gray}99.92 & 75.23 & 40.61 & 71.08 & 71.71                \\ 
& DCL~\cite{dcl}       & \cellcolor{Gray}\textbf{99.98} & \textbf{77.13} & 61.01 & 75.01 & 78.28              \\
& RECCE*~\cite{recce}     & \cellcolor{Gray}99.19 & 74.39 & 57.42 & \textbf{85.04} & 79.01                \\ 
& UIA-ViT*~\cite{uia}  & \cellcolor{Gray}99.39      &   74.44    &   53.89    &   70.92    & 74.66 \\ 
& Ours  & \cellcolor{Gray}99.19 & 76.61 & \textbf{68.96} & 74.83 & \textbf{79.90}            \\ 
       
\midrule
\multirow{5}*{F2F}
        & MultiAtt~\cite{multiatt}       & 86.15 & \cellcolor{Gray}99.13 & 60.14 & 64.59 & 77.50 \\
        & DCL~\cite{dcl}       & 91.91 & \cellcolor{Gray}99.21 & 59.58 & 66.67 & 79.34 \\
       & RECCE*~\cite{recce}       & 88.04 & \cellcolor{Gray}98.93 & 67.35 & 74.16 & 82.12 \\
       & UIA-ViT*~\cite{uia}       & 83.39 & \cellcolor{Gray}98.32 & 68.37 & 67.17 & 79.31 \\
       & Ours                                  & \textbf{93.75} & \cellcolor{Gray}\textbf{99.61} & \textbf{78.62} & \textbf{79.56} & \textbf{87.81} \\

\midrule
\multirow{5}*{FSW}
& MultiAtt~\cite{multiatt} & 64.13 & 66.39 & \cellcolor{Gray}99.67 & 50.10 & 70.07              \\
& DCL~\cite{dcl}           & 74.80 & 69.75 & \cellcolor{Gray}99.90 & 52.60 & 74.26              \\
& RECCE*~\cite{recce}       & 66.66 & 73.66 & \cellcolor{Gray}\textbf{99.76} & \textbf{57.46} & 74.39               \\

& UIA-ViT*~\cite{uia}       &   81.02    &   66.30    & \cellcolor{Gray}99.04      &   49.26    & 73.91 \\ 
& Ours                                           & \textbf{87.46} & \textbf{75.96} & \cellcolor{Gray}99.42 & 55.87 & \textbf{79.68}            \\ 

\midrule
\multirow{5}*{NT}
& MultiAtt~\cite{multiatt} & 87.23 & 75.33 & 48.22 & \cellcolor{Gray}98.66 & 77.36                \\
& DCL~\cite{dcl}      & 91.23 & 79.31 & 52.13 & \cellcolor{Gray}\textbf{98.97} & 80.41                \\
& RECCE*~\cite{recce}    & \textbf{90.20}  & 76.65 & \textbf{58.06} & \cellcolor{Gray}97.17 & \textbf{80.52}                \\
 & UIA-ViT*~\cite{uia}  &    79.37   &   67.98    &   45.94    &\cellcolor{Gray}94.59       & 71.97 \\
 & Ours     & 84.31 & \textbf{81.04} & 54.67 & \cellcolor{Gray}96.19 & 79.05          \\
       
\bottomrule
\end{tabular}}
\vspace{-2em}
\end{center}
\end{table}

\textbf{Cross-dataset evaluations.} We train our model on the Faceforensics++ dataset and evaluate its performance on the test sets of Celeb-DF\cite{celeb-df} and DFDC~\cite{dfdc}. Specifically, following the previous practice in~\cite{lip}, we validate the model on Celeb-DF and use the selected model to test on DFDC.  \emph{We adopt the HQ version of FF for training, and only use one frame every video for testing.} Under the same setting, we ran the public code of RECCE~\cite{recce}, UIA-ViT~\cite{uia} and SBI~\cite{sbi} to produce corresponding results. In Table~\ref{tab:cross-dataset}, we show a competitive performance with existing image-based methods, signaling satisfying adaptability of RFFR to different datasets, especially high quality datasets like Celeb-DF. 
  
SBI~\cite{sbi} is a recent powerful deepfake detection method. By utilizing a hand-crafted blending algorithm to produce diverse fake samples, it achieves highly competitive performances on datasets including Celeb-DF. We show that by training on fake samples generated by SBI, our approach can further improve upon their state-of-the-art result. 

\begin{table}[]
\setlength\tabcolsep{4.5pt} 
\caption{Cross-dataset performances in terms of AUC(\%) compared with previous methods. Classifiers are trained on FF and tested on Celeb-DF and DFDC. We ran the public code of methods marked with "*" to produce results under identical settings \emph{(HQ for training and single frames for testing).}}
\vspace{-1em}
\label{tab:cross-dataset}
\begin{center}  
\scalebox{0.90}{
\begin{tabular}{l|cc}
\toprule
\multirow{2}*{Method} & \multicolumn{2}{c}{Test data}\\
\cmidrule{2-3}
        ~                           &     Celeb-DF         &  DFDC \\
\midrule
      Xception~\cite{xception}  &     65.30       &    -  \\
      Face X-ray~\cite{xray}          &     74.20       &     70.00 \\
      MultiAtt~\cite{multiatt}        &     67.44       &     67.34 \\
      SPSL~\cite{SPSL}                &     76.88        &   -  \\
      SOLA~\cite{sola}                &       76.02         &  -    \\
      SLADD~\cite{sladd}              &    79.70       &  -  \\
      RECCE*~\cite{recce}             &     68.94       &   68.34   \\
      UIA-ViT*~\cite{uia}             &     80.31      &   67.93   \\
      SBI*~\cite{sbi}                       &       86.46     &   66.60     \\
\midrule
 	Ours                                      &   81.97  & \textbf{72.08}  \\
    Ours + SBI~\cite{sbi}                  &  \textbf{88.98}           &    67.84   \\
\bottomrule
\end{tabular}}
\vspace{-2.5em}
\end{center}
\end{table}

\subsection{Ablation Study}
\label{ablation}

In this section, we analyze the effect of our implementations for RFFR learning and deepfake detection. 

\textbf{Effect of the training data for RFFR.} The effectiveness of deepfake detection with RFFR depends on the quality of representation learning, where the real faces plays an important role. In this experiment, we examine the effect of scaling the real face dataset for representation learning. As a baseline, we learn RFFR with only real faces from Faceforensics++ (FF), the same data we use for the downstream classification tasks. Meanwhile, another model is supplemented with real faces from both FF and ForgeryNet (FN), a significantly larger and more diverse dataset. We train deepfake detectors on the F2F subset of FF with residual images produced by these two models. In Table~\ref{tab:data}, we demonstrate that including the extra dataset of ForgeryNet for learning RFFR consistently improves the performances of the deepfake detector in all tests, creating a maximum performance gain of $9.57\%$  in terms of AUC (F2F $\rightarrow$ NT).

We note that learning RFFR with FF already allows our deepfake detector to outperform the state-of-the-arts. Nevertheless, learning with extra data enhances the efficacy of our real face foundation representations, and further improves the downstream task of deepfake detection. Therefore, refining the representation learning of real faces, especially with large-scale datasets, could be a viable path for further improving generalized deepfake detection. 

In addition, we examine the scalability of RECCE under the same setting, considering that RECCE~\cite{recce} also involves learning to reconstruct real samples for deepfake detection. However, their performance gain is less significant than ours. Although the reconstruction branch of RECCE~\cite{recce} is able to highlight forgery cues with residual images, they tend to involve more background noise caused by imperfect reconstructions, as depicted in~\cref{fig:unet_comparison},. This undermines the ability of residual images to expose artifacts for deepfake detection. 

\begin{table}[t]
\setlength\tabcolsep{4.5pt} 
\caption{Deepfake detection performances of RECCE~\cite{recce} and our method with different real face dataset, namely the real faces from Faceforensics++ (FF) alone, and FF combined with ForgeryNet (FF + FN). Classifiers are trained on F2F and tested on four subsets of FF. We present the results in AUC (\%).  }
\vspace{-1.5em}
\label{tab:data}
\begin{center}  
\scalebox{0.90}{
\begin{tabular}{c|c|cccc|c}
\toprule
\multirow{2}*{Method} & Real face  & \multicolumn{4}{c|}{Test data} & \multirow{2}*{Avg} \\
\cmidrule(lr){3-6}
&dataset  &      DF    & F2F   & FSW   & NT    & ~   \\
    \midrule
\multirow{2}*{RECCE~\cite{recce}}&FF           & 88.04          & 98.93          & 67.35          & 74.16          & 82.12          \\
&FN + FF &  90.12       & 99.24       & 69.89    & 79.59     & 84.71		\\
    \midrule
\multirow{2}*{Ours}&FF           & 90.16          & 98.56          & 74.10          & 69.99          & 83.20          \\
&FN + FF & \textbf{93.44}       & \textbf{99.61}        & \textbf{78.62}       & \textbf{79.56}        & \textbf{87.81}		\\
\bottomrule
\end{tabular}}
\vspace{-1em}
\end{center}
\end{table}

\textbf{Effect of masked image modeling for RFFR.} We analyze the effect of using MIM-based residual images for deepfake detection. We train a UNet-based autoencoder (AE) to learn the reconstruction of real faces and obtain residual images. Our MIM-trained inpainting model and the AE are compared on the quality of reconstruction in~\cref{fig:unet_comparison}. Note that despite being trained with real faces, the AE "generalizes" well to fake images, preserving delicate details, including the artifacts caused by manipulations. Such generalization leaves the residual images empty with little information. 

\begin{figure}
\centering
  \includegraphics[width=0.9\columnwidth]{figs/compare_ICCV_Final.pdf}
  \vspace{-1em}
   \caption{Reconstruction results and residual images of the autoencoder (AE), RECCE~\cite{recce} and our inpainting model. AE reconstructs both images perfectly, leaving no information in residual images. RECCE~\cite{recce} suffers from insufficient training. Our model successfully highlights potential artifacts in the residual image of only the fake face, and therefore can best facilitate deepfake detection. }
\vspace{-1em}
\label{fig:unet_comparison}
\end{figure}

Masked image modeling enables our model to learn better real face representations and inpaint fake faces with real textures instead of artifacts. In the downstream task of deepfake detection,  our classifier generalizes significantly better than the AE-based classifier, which performs only marginally better than learning with no residuals (detailed in Appendix). Both the reconstruction results and the downstream performance confirm the validity of our choice to learn RFFR with MIM instead of direct reconstruction. 


\textbf{Effect of classifier backbone.} In Table~\ref{tab:backbone}, we present the deepfake detection results of vanilla Xception~\cite{xception} and Vision Transformer (ViT)~\cite{vit}, both trained with full original images. The models are trained with the F2F subset of FF and tested on all four subsets. While a larger backbone increases a deepfake detector's generalization performance in some cases, it is not the primary factor of our performance improvement. Instead, it is the residual input aided by RFFR that leads the performance gain.

\begin{table}[t]
\setlength\tabcolsep{4.5pt} 
\caption{Comparing ours results with vanilla backbones. We present the results in AUC (\%).  }
\label{tab:backbone}
\vspace{-1.5em}
\begin{center}  
\scalebox{0.90}{
\begin{tabular}{c|c|cccc|c}
\toprule
Training  &  \multirow{2}*{Method}    &   \multicolumn{4}{c|}{Test Data} & \multirow{2}*{Avg} \\
\cmidrule(lr){3-6}
 data  &   ~  &   DF    & F2F   & FSW   & NT    & ~   \\
    \midrule
\multirow{3}*{F2F} & Xception~\cite{xception} & 84.94          & 99.26          & 58.82          & 71.19          & 78.55          \\
                                   & ViT~\cite{vit}      & 84.25          & 97.89          & 65.53          & 65.18          & 78.21          \\
                                   & Ours     & \textbf{93.44} & \textbf{99.61} & \textbf{78.62} & \textbf{79.56} & \textbf{87.81} \\
\bottomrule
\end{tabular}}
\vspace{-1.5em}
\end{center}
\end{table}

\textbf{Effect of classifier design.} We compare different variants of our classifier design. Specifically, we analyze the performance gains brought by the introduction of two branches and the random input mechanism. We test six variants of our classifier by training them with the F2F subset of FF and testing with the FSW subset. The settings of these variants are specified by the input data they accept, as shown in~\cref{tab:classifier}. 

\begin{table}[t]
\caption{Deepfake detection performances with classifiers of different inputs in terms of AUC (\%). We train the classifiers on F2F and test on FSW.}
\label{tab:classifier}
\vspace{-1.5em}
\begin{center}
\begin{tabular}{c|c|c|c|c}
\toprule
\multicolumn{2}{c|}{Original Image} & \multicolumn{2}{c|}{Residual Image} & \multirow{2}*{AUC (\%)} \\
\cline{1-4}
               Full        &             Random           &          Full          &          Random          &   ~\\
 \hline
\checkmark        &                                       &                            &                                   &  65.53\\
% \hline
                              &                                      &   \checkmark    &                                   &  66.30  \\
 %\hline
\checkmark        &                                      &   \checkmark    &                                   &  71.48  \\
 %\hline
                             &       \checkmark          &                             &                                   &  70.76  \\
%\hline
                             &                                       &                             &      \checkmark       &  68.10  \\
 %\hline
                             &        \checkmark         &                             &      \checkmark       &  \textbf{78.62}  \\
\bottomrule
\end{tabular}
\vspace{-2em}
\end{center}
\end{table}

\begin{table*}[t]
\setlength\tabcolsep{4.5pt} 
\caption{Deepfake detection performances of validated and non-validated models. Classifiers are trained on F2F and tested on four subsets of FF. We present the results and the performance gaps in AUC (\%). Second best results are underlined. }
\label{tab:validation}
\vspace{-1em}
\begin{center}  
\scalebox{0.90}{
\begin{tabular}{c|c|llll|l}
\toprule
\multirow{2}*{Method}  & \multirow{2}*{Validated} & \multicolumn{4}{c|}{Test Data} & \multirow{2}*{Avg} \\
\cmidrule(lr){3-6}
~                   &                      ~                   &      DF               & F2F                    & FSW                 & NT                    & ~   \\
    \midrule
\multirow{2}*{Xception\cite{xception}} &   \checkmark    & 84.94                 & 99.26                & 58.82                 & 71.19                & 78.55            \\
~ &                                             -                              & 83.08   (- 1.86) & 99.12   (- 0.14) & 46.63   (- 12.19) & 64.93   (- 6.26)  & 73.44   (- 5.11)  \\
 \hline
 \multirow{2}*{RECCE\cite{recce}} &\checkmark               & 88.04                & 98.93                 & 67.35                & 74.16                & 82.12            \\
 ~&                                                -                  & 74.51   (- 8.57) & 99.22   (+ 0.29)  & 50.17   (- 17.18) & 59.46   (- 14.70)  & 70.84   (- 11.28) \\
 \hline
\multirow{2}*{Ours} &    \checkmark  & \textbf{93.44}            & \textbf{99.61}            & \textbf{78.62}            & \textbf{79.56}            & \textbf{87.81}            \\
 ~&  - & \underline{91.56} (- 1.88) & \underline{99.39}   (- 0.22) & \underline{76.00}   (- 2.62)  & \underline{76.41} (- 3.15) & \underline{85.84}   ( - 1.97)    \\
\hline
\end{tabular}}
\vspace{-2em}
\end{center}
\end{table*}

We treat the vanilla ViT with full original image input as a baseline, which achieves an AUC of $65.53\%$. By switching to accept the full residual images, we obtain a $0.77\%$ performance gain. Combining the two modalities to form a dual-branch classifier further increases our result to $71.48\%$. This demonstrates that the artifacts are better exploited when both the original and the residual images enter the classifier, and are used as references to each other. Therefore, both modalities should be considered for classification. 

In addition, we improve on the test by merely modifying the baseline ViT to accept randomly selected original image blocks. This results in a $5.23\%$ increase in performance. Similarly, changing full residual input to random residual blocks also results in a $1.8\%$ improvement. These observations confirm our hypothesis in \cref{sec:method_deepfake_detection} that models benefit from learning with random inputs, which prevents the model from only focusing on the most prominent features in an image, and forces it to learn from subtle artifacts. 

Finally, bringing in the random input mechanism for the dual-branch classifier completes our full implementation, which maximally exploits the artifacts exposed by RFFR and achieves the best performance of $78.62\%$. 



\subsection{Validation-free Model Selection}
\label{sec:validation-free}

\begin{figure}
\centering
  \includegraphics[width=0.5\textwidth]{figs/validation-free_ICCV_Final.png}
  \vspace{-1.5em}
   \caption{Comparing the validation curves of RFFR-based deepfake detector and previous methods. Detectors are trained on the F2F subset of FF for $15k$ iterations and validated on four different subsets. (a) to (d) correspond to experiments on DF, F2F, FSW and NT.  Results are reported in AUC (\%). All three methods perform well when validated on F2F. However, under cross-manipulation settings, only our method avoids overfitting during training. The curves are smoothed for better visibility.}
\label{fig:validation-free}
\vspace{-1em}
\end{figure}

Models expected to generalize to other domains benefit from target domain validations~\cite{domainbed}. By frequently performing model validation, we can select the model  that best suits the detection of target manipulation, resulting in high performance on the test set. While using such an \textit{oracle} could be acceptable for the early development of cross-domain algorithms~\cite{domainbed}, it is not ideal for applications, as labeled data of unseen manipulation is usually not available. 

In this section, we demonstrate the potential of our deepfake detector to circumvent this practice and therefore avoid the need for extra validation data. As shown in \cref{tab:validation}, we train our classifier on F2F for 15k iterations and directly use the final model for testing. Simultaneously, we employ four validation sets to select the models with the best validation performances on target data. All validated and non-validated models are tested under the same conditions. We report all results on the target test sets in Table~\ref{tab:validation}. The performance gaps between validated and non-validated models are reported along with the test results. Although our non-validated models are not performing as well as those selected with a validation set, we show that our model remains effective on target data, with a maximum performance drop of $3.15\%$ and an average drop of $1.97\%$. However, previous methods~\cite{xception, recce} suffer from significantly larger performance drops when evaluated under the same procedure. 

To take a closer look at how the cross-manipulation performances vary during training, we train the deepfake detectors again with F2F. We test the AUC performances on all target subsets every 50 iterations to produce validation curves in \cref{fig:validation-free}. Our RFFR-based deepfake detector consistently maintains a high performance long after its peaks without serious overfitting. On the contrary, both previous methods compared here overfit quickly after reaching their highest target domain performances. In addition, compared methods exhibit large fluctuations across different evaluations, while our model remains stable. This suggests that with RFFR, our model focuses exclusively on generalizable features which fall outside the distribution of RFFR. Such resistance to overfitting guarantees our model a satisfying performance even when labeled validation sets are not available, which is generally expected in practice. We present more results on validation-free evaluations in Appendix.
\section{Conclusion, Limitations, and Future Work}
\label{sec:future}
We presented \ours, a NeRF editing method conditioned on text and sketch. Using novel loss functions, our framework allows for local editing of neural fields.
\begin{wrapfigure}{r}{0.2\textwidth} 
\vspace{-10pt}
  \begin{center}
    \includegraphics[width=0.2\textwidth]{figs/failures_Ali.jpg}
  \end{center}
    \vspace{-15pt}
 \vspace{1pt}
\end{wrapfigure} 
Similar to previous works \cite{poole2022dreamfusion, lin2022magic3d, metzer2022latent}, our approach utilizes the SDS Loss and may be vulnerable to the well-known "multiface issue" (inset figure) depending on the choice of diffusion model and prompt. Our method supports a single set of prompt and sketch views at a time. A simple workaround is to apply our method multiple times progressively (Fig.~\ref{fig:progressive}). 
Our results rely on the publicly available Stable-Diffusion model \cite{rombach2021highresolution}, which is less amenable to directional text prompts and produces lower quality 3D generated outputs compared to commercial diffusion models used by previous works~\cite{poole2022dreamfusion, lin2022magic3d}. In Fig~\ref{fig:diff_diff} we show that it is possible to get better results by using the Deepfloyd-IF model \cite{deepfloyd}.


Future directions may expand our method to better support for non-opaque materials, or condition on other modalities, possibly through the diffusion model. More research may further extend the usage of sketch scribbles for animation, similar to \cite{dvoro2020monstermash}. 



% \orrc{In addition, the interface of our method may further close the gap with non data-driven methods, through allowing inflated single sketch views or other primitive based sketch interfaces. Mention also we didn't explore half-transparent objects enough

% \orr{
% Limitations: 1. Janus effect / multiface problem (cat with santa hat), 2. sketching multiple disjoint regions at once. 3. mention that quality presented in this work depend on the diffusion model used? (we can't compete with the larger IMAGINE / e-diffi).

% Notes: remember thanking people: Andrey for SGMT code and mention mesh sources. (the cat, the plate, the horse)
% }




% \noindent Displayed equations are centered and set on a separate
% line.
% \begin{equation}
% x + y = z
% \end{equation}
% Please try to avoid rasterized images for line-art diagrams and
% schemas. Whenever possible, use vector graphics instead (see
% Fig.~\ref{fig1}).

% \begin{figure}
% \includegraphics[width=\textwidth]{fig1.eps}
% \caption{A figure caption is always placed below the illustration.
% Please note that short captions are centered, while long ones are
% justified by the macro package automatically.} \label{fig1}
% \end{figure}






%
% ---- Bibliography ----
%
% BibTeX users should specify bibliography style 'splncs04'.
% References will then be sorted and formatted in the correct style.
%
\bibliographystyle{splncs04}
\bibliography{mybibliography}


\end{document}
