\documentclass[12pt]{amsart}
\usepackage{graphicx}
\usepackage{amssymb}
\usepackage{amsthm}
\usepackage{listings}
\usepackage{lineno}
\usepackage[margin=3cm]{geometry}
\usepackage[all,cmtip, color,matrix,arrow]{xy}
\usepackage{amsaddr}
\usepackage{tikz-cd}
\usepackage{amsmath}%To use \text 
\usepackage[utf8]{inputenc}
\usepackage{hyperref}
\usepackage[capitalize]{cleveref}
\crefname{thm}{Theorem}{Theorems}
%\usepackage{bbold}
\usepackage[export]{adjustbox}
\usepackage{todonotes}
\usepackage{bm}
\usepackage{wrapfig}
\usepackage{float}
\usepackage{mathtools}
\usepackage{aliascnt}
\newaliascnt{eqfloat}{equation}
\newfloat{eqfloat}{h}{eqflts}
\floatname{eqfloat}{Equation}
%\usepackage{dirtytalk}
\usepackage[mathscr]{euscript}

\newcommand*{\ORGeqfloat}{}
\let\ORGeqfloat\eqfloat
\def\eqfloat{%
  \let\ORIGINALcaption\caption
  \def\caption{%
    \addtocounter{equation}{-1}%
    \ORIGINALcaption
  }%
  \ORGeqfloat
}
\newcommand{\raul}[1]{\todo[color=green!30,inline]{Raul: #1}}
\newcommand{\carlo}[1]{\todo[color=red!30,inline]{Carlo: #1}}


\makeatletter
\providecommand*{\shuffle}{%
  \mathbin{\mathpalette\shuffle@{}}%
}
\newcommand*{\shuffle@}[2]{%
  % #1: math style
  % #2: unused
  \sbox0{$#1\vcenter{}$}%
  \kern .15\ht0 % side bearing
  \rlap{\vrule height .25\ht0 depth 0pt width 2.5\ht0}%
  \raise.1\ht0\hbox to 2.5\ht0{%
    \vrule height 1.75\ht0 depth -.1\ht0 width .17\ht0 %
    \hfill
    \vrule height 1.75\ht0 depth -.1\ht0 width .17\ht0 %
    \hfill
    \vrule height 1.75\ht0 depth -.1\ht0 width .17\ht0 %
  }%
  \kern .15\ht0 % side bearing
}
\makeatother


%\def\shuffle{\sqcup\mathchoice{\mkern-7mu}{\mkern-7mu}{\mkern-3.2mu}{\mkern-3.8mu}\sqcup\,}
\newcommand{\qshuffle}{\,\overline{\shuffle}\,}


\theoremstyle{definition}
\newtheorem{thm}{Theorem}[section]
\newtheorem{prop}[thm]{Proposition}
\newtheorem{lm}[thm]{Lemma}
\newtheorem{cor}[thm]{Corollary}
\newtheorem{obs}[thm]{Observation}
\newtheorem{defin}[thm]{Definition}
\newtheorem{smpl}[thm]{Example}
\newtheorem{quest}[thm]{Question}
\newtheorem{prob}[thm]{Problem}
\newtheorem{conj}[thm]{Conjecture}
\newtheorem{rem}[thm]{Remark}
\crefname{lm}{Lemma}{Lemmas}
\crefname{thm}{Theorem}{Theorems}
\crefname{prop}{Proposition}{Propositions}
\crefname{defin}{Definition}{Definitions}
\crefname{rem}{Remark}{Remarks}

\newcommand{\parpi}{\boldsymbol{\pi}}
\newcommand{\partau}{\boldsymbol{\tau}}
\newcommand{\makepar}{\boldsymbol{\lambda}}
\newcommand{\uhsm}{\boldsymbol{\Psi}}
\newcommand{\uHsm}{\boldsymbol{\Psi}}
\newcommand{\sqbinom}{\genfrac{[}{]}{0pt}{}}
\newcommand{\x}{\boldsymbol{x}}


\DeclareMathOperator{\Alg}{\mathrm{Alg}}
\DeclareMathOperator{\im}{im}
\DeclareMathOperator{\id}{id}
\DeclareMathOperator{\comu}{comu}
\DeclareMathOperator{\Orth}{Orth}
\DeclareMathOperator{\rk}{\mathrm{rk}}
\DeclareMathOperator{\cano}{cano}
\DeclareMathOperator{\dpt}{\mathbf{depth}}
\DeclareMathOperator{\Func}{\mathrm{Func}}
\DeclareMathOperator{\spn}{\mathrm{span}}
\DeclareMathOperator{\cone}{\mathrm{cone}}
\DeclareMathOperator{\inc}{\mathrm{inc}}
\newcommand{\eval}{\mathrm{ev}} 

\newcommand{\len}{l} % for length (degree) of a composition or partition

%signatures
\DeclareMathOperator{\Dsign}{\mathscr{S}}
\DeclareMathOperator{\sign}{\sigma}

%algebras
\newcommand{\T}{\mathcal{T}}
\newcommand{\e}{\vec{\mathbf{e}}}
\newcommand{\polymond}{\mathcal M_{\bullet}(d)}
\newcommand{\polymondk}{\mathcal M_{\bullet}(d, k)}
\newcommand{\K}{\mathbb{K}}
\newcommand{\R}{\mathbb{R}}
\newcommand{\Z}{\mathbb{Z}}
\newcommand{\polyd}{\mathfrak A_d}
\newcommand{\polydh}{\mathfrak A_{d}^{\leq h}}
\newcommand{\XX}{\mathbf{X}}
\newcommand{\KK}{\mathtt{K}}

%set partitions
\newcommand{\splamb}{\vec{\boldsymbol{\lambda}}}
\newcommand{\sptau}{\vec{\boldsymbol{\tau}}}
\newcommand{\spmu}{\vec{\boldsymbol{\mu}}}


%set compositions
\newcommand{\oPi}{\mathbf{C}}
\newcommand{\opi}{\vec{\boldsymbol{\pi}}}
\newcommand{\otau}{\vec{\boldsymbol{\tau}}}
\newcommand{\olambda}{\vec{\boldsymbol{\gamma}}}

%numbers as characters of a word
\newcommand{\one}{\mathtt{1}}
\newcommand{\two}{\mathtt{2}}
\newcommand{\thr}{\mathtt{3}}

%varieties
\newcommand{\V}{\mathcal V}
\newcommand{\U}{\mathcal U}
\newcommand{\W}{\mathcal W}
\newcommand{\cL}{\mathcal L}

\newcommand{\Hiso}{\mathbf{\Phi}_*}
\newcommand{\tens}{\otimes}

%%vectors
\newcommand{\w}{\mathsf{w}}
\newcommand{\bv}{\mathsf{b}}
\newcommand{\vv}{\mathsf{v}}
\newcommand{\vw}{\mathsf{w}}
\newcommand{\lv}{\mathfrak{l}}

%characters
\newcommand{\el}{\mathsf{e}}
\newcommand{\fl}{\mathsf{f}}
\newcommand{\il}{\mathsf{i}}
\newcommand{\jl}{\mathsf{j}}


\newcommand{\MS}{\mathrm{MS}}
\newcommand{\he}{\mathrm{ht}}
\newcommand{\ot}{\otimes}
\newcommand{\op}{\oplus}
\newcommand{\tms}{\times}
\newcommand{\Mat}{\mathtt{Mat}}
\newcommand{\dg}{\mathrm{deg}}
\newcommand{\LL}{\mathtt{L}}
\newcommand{\uw}[1]{#1}
\newcommand{\blt}{\bullet}
\newcommand{\bltS}{\hspace{-.7mm}\bullet\hspace{-.7mm}}
\newcommand{\blts}{\hspace{-.2mm}\bullet\hspace{-.2mm}}

\newcommand{\citefriz}{\cite[Theorem 7.28]{frizbook}}

%\pagecolor[rgb]{0,0,0}
%\color{yellow}
\begin{document}

%% Title, authors and addresses
\title{Discrete signature varieties} % Subtitle
\author{Carlo Bellingeri}
\address{Technische Universität Berlin, Germany}
\email{bellinge@math.tu-berlin.de}

\author{Raul Penaguiao}
\address{Max-Planck-Institut für Mathematik in den Naturwissenschaften, Germany}
\email{raul.penaguiao@mis.mpg.de}


\keywords{Free Lie algebra, Signatures, Algebraic varieties, Quasi-shuffle product}
\subjclass[2020]{14Q15, 60L70, 17B45}
\date{\today} % Date

\begin{abstract}
Discrete signatures are invariants computed from time series that correspond to the discretised version of the signature of paths. We study the algebraic varieties arising from their images, the discrete signature variety. We introduce them and compute their dimension in many cases. From the analysis of a particular subclass of these varieties, we also derive a partial solution to the Chen-Chow theorem in an algebraically closed setting. %ents that play a crucial role in computing the dimension of the associated Lie algebra.

\end{abstract}

\maketitle
%\tableofcontents
\section{Introduction}


Signatures of smooth curves were introduced in \cite{chen1957integration}, leading to revolutionary applications on stochastic analysis with contributions by Lyons, Friz, Hairer, and others, see \cite{lyons2007differential,frizbook,friz2020course}.

The discrete counterpart of this invariant is the \textbf{discrete signature} of a time-series, or iterated sum signature, introduced by \cite{Tapia20}.
When focusing on data that is naturally discrete, several interesting applications of this new invariant arise, of which we remark signal compression \cite{bandeira2017estimation}.
These applications arise from the fact that discrete signatures are invariant under time warping.
That is, data is allowed to ``stutter'', leading nonetheless to the same signatures.

More concretely, if we are given a time-series $x\colon \{1, \ldots, N\}\to \KK^d $ in some field $\KK$ of characteristic $0$, the discrete signature (denoted by $\Dsign(x)$)  is an element in a space of tensor series.  This tensor series space is spanned by infinite sums indexed on words of monomials. Given a word $p_1 \blts \cdots \blts\, p_l$ of non-constant monomials $p_i$ in $\KK[X_1, \ldots , X_d]$  we can define the corresponding coefficient in the signature of a time-series $x = (x_1, \ldots, x_N)$ as follows:
\begin{equation*}\label{eq:dss}
\langle\Dsign(x) , p_1\bullet\cdots \bullet p_l \rangle = \sum_{1 \leq i_1 < \dots < i_l < N} p_1(x_{i_1+1} - x_{i_1}) \cdots p_l(x_{i_l+1} - x_{i_l}) \, .
\end{equation*}
For example, using the notation, $x_j= (x^1_j\,, \ldots \,, x_j^d)$ and the identification $X_i=\mathtt{i} $  we have the identities
\begin{align*}
\langle\Dsign(x),\mathtt{i}\rangle =\sum_{k=1}^{N-1} (x^i_{k+1} - x^i_{k})= x_N^i- x_1^i\,,\\\quad \langle\Dsign(x),\mathtt{i}\mathtt{j}\rangle =\sum_{k=1}^{N-1}  (x^i_{k+1} - x^i_{k})(x^j_{k+1} - x^j_{k})\,,\\ \langle\Dsign(x),\mathtt{i}\bullet \mathtt{j} \rangle=\sum_{1 \leq k < l < N} (x_{k+1}^i - x_{k}^i)(x_{l+1}^j - x_{l}^j)\,.
\end{align*} 

We argue that the set in tensor space where discrete signatures live is  a very interesting  from the point of view of algebraic geometry.
To state it in a precise way, we fix an integer $h\geq 0$ and consider the projection of $\Dsign(x)$ on the space of degree $h$. From this perspective, the discrete signature becomes naturally a polynomial map $x\mapsto \Dsign(x)$ on the $Nd$ parameters of the time-series with values in a specific finite dimensional vector space. 
Our goal is to describe the image of $\Dsign$. 

Since the image of a polynomial map is not always an algebraic set we consider the Zariski closure of this set, the \textbf{variety of discrete signatures $\V_{d, h, N}$}.  This variety is an irreducible algebraic variety that arises from a very difficult implicitization problem. The main goal of this paper is to introduce this family of varieties and describe in particular the limit of these spaces when $N$ is very large, which gives us the \textbf{universal variety $\V_{d, h}$}. 
%Due to the presence of homogeneous polynomials in the truncation of $\Dsign(x)$, we will mainly consider the case when $\KK=\mathbb{C}$ and the polynomial map $\Dsign$ is seen as a rational map among projective spaces. However,  we will also introduce several variations of these varieties by choosing the other reasonable choice $\KK=\mathbb{R}$ or the affine variety from which we define $\V_{d, h, N}$.
In many aspects, our results can be related with \cite{amendola2019varieties}, as it takes advantage of the algebraic properties underlying signatures, and will be understood as an extension of the results obtained with Chen's signatures. 
%Nevertheless, 
The use of an algebraic geometry perspective produces different results from the case of smooth paths. 


A difference between the discrete signature presented here and the classical signature of paths from \cite{chen1957integration}, is that the \textbf{shuffle relations} become \textbf{quasi-shuffle relations} (see \cite{hoffman2000}).
%inherited from the ring of quasi-symmetric functions (see \cite{hazewinkel2001algebra, Malv_reu95}) when $d=1$ and more general the ring of quasi-symmetric functions of level $d$ (see \cite{novelli_thibon10}). 
In our truncated setting we use the \textbf{Hoffman isomorphism} \cite{hoffman2000} and simple results on free Lie algebras \cite{reutenauer1993free} to connect these relations and give a description in terms of a Lie Group (denoted by $\hat{\mathcal{G}}^{\leq h}(\polyd)$) containing all the affine version of our variety of interest $\mathcal{V}_{d,h, N}$. An equivalent study of the same relations was recently given in \cite{Bellingeri2022}. From the knowledge of quasi-shuffle relations we will be able to define then a proper family of algebraic varieties $\hat{\mathcal{G}}_{d,h}$ containing  $\mathcal{V}_{d,h}$.

A second difference between Chen's signatures and discrete signatures is related to the absence of a \textbf{Chow theorem} for discrete signatures (see \cite[Section 3.2]{diehl_tapia_EF_2020}).  
In few words, given an element $g$ in $ \hat{\mathcal{G}}^{\leq h}(\polyd)$ does there exist a time-series whose discrete signature is equal to $g$? A general positive result to this question will be able to describe completely $\mathcal{V}_{d,h}$ but  there is no clear answer. To better tackle this open problem, we introduce the equivalent notion of \textbf{reachability} and we prove this condition in the simplest case $h=2$. 
Fundamental for the combinatorics of this formulation are \textbf{Lyndon words}.
Specifically, we extend the original concept from \cite{chen1958free} by skewing them with a height function. 
These combinatorially skewed Lyndon words arise in several places of the Hopf algebra landscape (see \cite{borga2020feasible, penaguiao2022pattern,penaguiao2020algebraic}).

Our main theorem, stated below  in  \cref{thm:dimension,lm:h2constructionX}, establishes the following:
\begin{thm}
Fix $d, h$ integers $\geq 1$. The dimension of the variety $\hat{\mathcal{G}}_{d,h}$ is given by 
\[\sum_{l=1}^{h}\sum_{k|l} \frac{k}{l} \mu\left(\frac{l}{k} \right) \sum_{\alpha\in C(k)} \frac{1}{\ell (\alpha)} \prod_i \binom{\alpha_i + d - 1}{d - 1} -1\,,\]
where $\mu $ is the M\"obius function on integer. Moreover  one has $\hat{\mathcal{G}}_{d,h}=\mathcal{V}_{d,h}$ if $h=2$.
\end{thm}


%\begin{equation}\label{eq:exp}
%\exp(\w) = \Phi_H^*( \Dsign (x))\, ,
%\end{equation}
%where $\Phi_H$ is the Hofmann isomorphism, a linear map introduced below.
%Such timeseries $x$ are called \textbf{primary elements}, due to their role in spanning the entire space of discrete signatures, \textit{vide} \cref{lm:reachable}.






%Finally, we discuss the \textbf{degree problem} from the point of view of path recovery.
%This traces back to the study in \cite{pfeffer2019learning}, where paths are recovered only knowing its signature of order three.
%We establish bounds for the degree of low order signature varieties.
%This means that for a set of linear constraints, there is a bounded number of time-series that have a specific signature.



We now state the organisation of the paper. 
Preliminaries and main definitions will be presented in \cref{sec:prelim}.
After introducing the main properties of quasi-shuffle algebras in \cref{sec:computations}, we define the varieties of discrete signatures. 
For some specific values $h$ and $d$, we leverage computational algebra tools to present a \textbf{ Gr\"obner basis} of the vanishing ideals of $\V_{d, h, N}$, drawing inspiration from algebraic statistics (see \cite{drton2008lectures}). 
From the explicit knowledge of a Gr\"obner basis we compute first invariants like dimension and degree. 
Then, in \cref{sec_Lie}, we present an enumerative result for the number of Lyndon words with some fixed height.
We conjecture that these also count the dimension of the universal variety.
Finally, in \cref{sec:conj}, we prove a characterisation of the universal variety under the notion of reachability, and prove the dimension conjecture for $h=2$.
%\carlo{recheck this paragraph}


In this article, we want to develop connections between applied algebraic geometry and rough analysis. 
In particular, we mention a general formula for the degree of the universal varieties and possible extensions of the theory in new contexts, such as non-zero characteristic or the study of varieties defined from stochastic time-series.



%\begin{itemize}
%
%\item Given an element $\Dsign \in \hat{\mathcal G}(\polyd)$, can we compute all posible time  series of a fixed size that have $\Dsign ( x ) = \Dsign$?
%This is related with the degree of the variety $\hat{\mathcal G}(\polyd)$, which is not easy to compute in full generality.
%
%\item Does the dimension change when the field characteristic is non-zero?
%\item stochastic 
%
%\end{itemize}

\section{Preliminaries}
\label{sec:prelim}

\subsection*{Combinatorics}
For an integer $n\geq 1$, write $[n] \coloneqq \{ 1, \dots , n\}$ and $\mathbb{N}= \{1\,, 2\,, \ldots\}$. Given a countable alphabet $I$, a \textbf{word} $w = (i_1, \dots , i_n)$ is an $n$-tuple of elements in $I$, for $n\geq 0$ integer.
We may write $w = i_1 \bltS  \cdots \bltS  i_n $ for simplicity, and $|w | = n$ is  called the length of $w$. We denote by  $\mathcal W (I)$ the set of words in $I$, including the empty word, which we denote $\varepsilon$. 
Furthermore, we abuse notation and write $ i$to refer to the length one word $(i)$.

Two words $w $ and $v$ may be concatenated, which we represent by $w \,\bltS \,v$.
%If $\el \in I$, we may abuse notation and interpret $\el \in \mathcal W(I)$. 
An alphabet $I$ may be equipped with a \textbf{grading} map $\text{grad} \colon I \to \mathbb{Z}_{> 0}$. 
We call the sum $\sum_{j=1}^n \text{grad}(i_j)$ the \textbf{height}  of a word $w = i_1 \bullet  \cdots \bullet   i_n$ and we  denote it by $||w||_{\he}$. One has that $|w|\leq ||w||_{\he}$ on any non-empty word $w$.


A \textbf{composition} $\alpha$ of an integer $k$ is an $n$-tuple of positive integers $\alpha = (\alpha_1, \dots , \alpha_m)$ such that $\sum_{i=1}^m \alpha_i = k$.
We denote the set of all composition of $k$ by $C(k)$, and write $\ell(\alpha) = m$ for the \textbf{length} of the composition. 
We use the following statistics on compositions $\alpha! \coloneqq \prod_{i=1}^m \alpha_i !$ and $\Pi \alpha \coloneqq \prod_{i=1}^m \alpha_i$.

Assume now that $I$ is an alphabet with an associative and commutative operation, which we denote by juxtaposition. 
If $\alpha \in C(k) $ and $w = i_1 \bullet\cdots \bullet i_k$ is a word such that $\ell(\alpha) = |w| = k$, then we define the \textbf{contracted word} $(w)_{\alpha}$ by multiplying  the letters in $w$ according to $\alpha$, i.e.
\[(w)_{\alpha} = \tau_1 \blt \cdots \bltS \tau_{\ell(\alpha)}\,,  \quad \text{with} \quad \tau_j \coloneqq i_{s_j + 1} \cdots i_{s_j+ \alpha_j}\, ,\] 
where $s_j = \sum_{i=1}^{j-1} \alpha_i$ for $j = 1, \dots , \ell(\alpha)$.

Given a countable set $A$, a \textbf{multiset} $S \subset A$ is a collection of elements in $A$, allowing for repetitions.
We denote the family of multisets of elements in $A$ by $\MS_A^0$.
This includes the empty multiset. 
We denote $\MS_A \coloneqq \MS_A^0 \setminus \{\emptyset \}$. 
When $A = [d]$, we denote $\MS_A^0, \MS_A$ by $\MS_d^0, \MS_d$, respectively.

The alphabet $\MS_d$ as a multiset has an associative and commutative operation, the union. Note that in the context of multisets, the union counts multiplicity of elements. Moreover it admits many total order preserving the usual ordering of $[d]$.
This can be easily contructed by taking a bijection between multisets of $[d]$ and monomials on $d$ variables, where monomial orders such as the \textit{graded lexicographic} are defined in \cite{miller2005combinatorial}.

\subsection*{Tensor algebras}

Let $V$ be a vector space over a charateristic zero field $\KK$. 
We define $T(V)$, the \textbf{tensor algebra}, and $T((V))$, the \textbf{tensor series} over $V$ as follows:
\begin{equation}\label{tensor}
T(V) \coloneqq \bigoplus_{k=0}^{+ \infty}V^{\otimes k} \quad \quad \quad T((V)) \coloneqq \prod_{k=0}^{+ \infty}V^{\otimes k}  \,,
\end{equation}
with $ V^{\otimes 0}:=\KK $.
By fixing a basis $\mathcal{B}=\{\el_i\colon i \in I\} $  of $V$ we can write elements of $T(V)$ and $T((V))$ as finite linear combinations and formal series of elements in $\mathcal W (I)$, respectively, via the identification
\[i_1 \bltS \cdots \bltS i_k= \el_{i_1} \ot\cdots \ot \el_{i_k}\,.\]
More generally, the $\bullet$ operation is identified with the algebraic tensor product $(w ,v) \mapsto w\ot v$, which is well defined on both $T(V)$ and $T((V))$, because it is locally finite. Both $T(V)$ and $T((V))$ form an algebra under $\bullet$. We define a bilinear and non-singular duality bracket $\langle \--, \-- \rangle: T((V)) \times T(V) \to \KK$ as
\begin{equation}\label{scalar}
 \left\langle \sum_{w \in \mathcal W(I)} \alpha_{w} w, v \right\rangle = \alpha_{v} \, , \end{equation}
extended linearly to $T(V)$.
In this way, we can identify $ T((V))$ with the algebraic dual of $T(V)$.

In what follows, we will always equip the indexing set $I$ of the basis of $V$ with a grading, such that there are finitely many elements of $I$ with a given grading.
Then $V$ becomes a graded vector space, and we can write $V = \oplus_{h \geq 0} V^h $ for the corresponding grading.
The vector space $T(V)$ also inherits a grading using the height of words by setting
\[T^h(V) \coloneqq \spn \{  w \in \mathcal W(I) \colon\,  ||w||_{\he} = h\} \,, \quad T(V)= \bigoplus_{h=0}^{+ \infty}T^h(V)\,, \]
\begin{equation*}
\begin{split}
T^{\leq h}(V)= \spn \{  w \in \mathcal W(I) \colon\,  ||w||_{\he} \leq  h\}\, ,  &\quad  T^{>h}(V)=\spn \{  w \in \mathcal W(I) \colon\, \, ||w||_{\he} >h\}\,.
\end{split}
\end{equation*}


The vector space $T^{\leq h}(V) \cong T(V)/{T^{>h}(V)}$ is called the space of \textbf{truncated tensors}. 
Since $T^{>h}(V)$ is a $\bullet$ ideal, the corresponding truncated tensor product is well defined on the quotient.
We write $\bullet_h$ for the product on the quotient. We always have the isomorphism
$$T^{\leq h}(V)\cong T^{\leq h}\left(V^{\leq h}\right)\, ,$$
where $V^{\leq h} = \oplus_{k=0}^h V^k$. For $k \in \KK$, let $T_k^{\leq h}(V) \coloneqq \{ \vv \in T^{\leq h}(V) \colon  \langle \vv, \varepsilon \rangle = k\}$. These vector spaces are all finite dimensional.

The same truncation procedure applies to $T((V))$ and we obtain a vector space which is isomorphic to $T^{\leq h}(V)$. 
For sake of simplicity, we denote this vector space with the same notation $T^{\leq h}(V)$.
This identification will also be used latter in the context of Hopf algebras. 


\subsection*{Shuffle and quasi-shuffle Hopf algebras}
Fix $d\geq 1$ integer, and consider henceforth 
\[V =\polyd =  \KK[X_1, \dots, X_d]/\{P\in \KK[X_1, \dots, X_d]\colon \deg(P)\leq0\}\] as our vector space of interest, on which we study the tensor algebra and tensor series.

The vector space $\polyd$ has a basis of non-constant monomials.
These are identified with $\MS_d$.
We abuse notation and refer to a multiset $p \in \MS_d$ as a monomial in the variables $X_1, \dots, X_d$. 
For example, we identify the monomial $X_1^2X_2$ with the multiset $\mathtt{112}$. 
This abuse of notation extends to the evaluation of a monomial, thus writing $\mathtt{112}(y_1, y_2, y_3) = y_1^2y_2$.
To distinguish elements of the alphabet $\MS_d$ and scalars in $\mathbb{Z}$, we use typewritter typeset for multisets.
The alphabet $\MS_d$ is graded, with degree given by the polynomial degree of the associated monic polynomial, or equivalently by the set size. 
Given any $h\geq 0$ we use the notation $\polydh:=(\polyd)^{\leq h} $.


Two products can be defined on $T(\polyd)$: the \textbf{shuffle} $\shuffle$ and the \textbf{quasi-shuffle} $\qshuffle$ products. 
We define them here recursively.
For any $p,q \in \MS_d$ and $w,v\in \mathcal{W}(\MS_d)$, 
\begin{equation}\label{shuffle_qshuffle}
\begin{split}
w =&\varepsilon \qshuffle w = w \qshuffle \varepsilon= \varepsilon \shuffle w = w \shuffle \varepsilon\, \\
w \blt p\shuffle v \blt q =& (w \shuffle  v \blt q)\blt p + (w \blt p\shuffle v) \blt q\\
w \blt p \qshuffle v \blt q =& (w \qshuffle  v \blt q)\blt p + (w \blt i\qshuffle v) \blt q+ (w \qshuffle v) \bltS pq
\end{split}
\end{equation}
These relations define two commutative algebras on $T(\polyd)$ which are compatible with the grading of $T(\polyd)$ given above (see \cite[Theorem 2.1]{hoffman2000} for a proof of this fact).

The tensor algebra $T(\polyd)$ can be further equipped with two structures of \textbf{Hopf algebras} by introducing the deconcatenation coproduct $\delta\colon T(\polyd) \to  T(\polyd) \otimes T(\polyd)$ and a matching counit  $\eta^*\colon T(\polyd) \to \KK$.
For a word  $w = p_1 \blt \cdots \blt p_k$, let
\[\delta (\uw{w})=\uw{\varepsilon}\otimes \uw{w} + \uw{w} \otimes \uw{\varepsilon} + \sum_{l=1}^{k-1} \uw{p_1 \blt \cdots \blt p_l} \otimes \uw{p_{l+1} \blt \cdots \blt p_k} ,\quad \eta^*(w): =\left\{
	\begin{array}{ll}
    1 & \mbox{if } w=\varepsilon\,, \\
	0 & \mbox{otherwise.}
	\end{array}
	\right.\]
We define as well the reduced coproduct map $\tilde{\delta} = \delta - \id\ot 1 - 1 \ot \id$. 
This endows $(T(\polyd),\shuffle, \delta)$ and   $(T(\polyd), \qshuffle, \delta)$ with graded Hopf algebra structures.
We expect there to be no confusion between elements in $T(V)$ and $T(V)\otimes T(V)$.

These Hopf algebras were shown to be isomorphic.
Explicit algebra morphisms $\Phi_H, \Psi_H:T(\polyd) \to T(\polyd)$ were constructed in \cite{hoffman2000}, which are inverses of each other.
Specifically, the maps $\Phi_H$ and $\Psi_H$ are the linear maps that act on words as follows:
\begin{align}\label{Hoff_exp_log}
\Phi_H(\uw{w}) &\coloneqq \sum_{\alpha \in C( |w |)} \frac{1}{\alpha !} \uw{(w)_{\alpha}}\,, \quad\quad
\Psi_H(\uw{w}) \coloneqq \sum_{\alpha \in C( |w |)} \frac{(-1)^{|w| - \ell(\alpha)}}{\Pi \alpha} \uw{(w)_{\alpha}}\,.
\end{align}
For instance, we have the following identities
\begin{align*}
\Phi_H(\uw{\mathtt{1} \blt \mathtt{2}}) =& \uw{\mathtt{1} \blt \mathtt{2}} + \frac{1}{2} \uw{\mathtt{1}\mathtt{2}} \,, \quad\quad
\Psi_H(\uw{\mathtt{1} \blt \mathtt{2}}) = \uw{\mathtt{1} \blt \mathtt{2}} - \frac{1}{2} \uw{\mathtt{1}\mathtt{2}}\,, \\
\Phi_H(\uw{\mathtt{1} \blt \mathtt{2} \blt \mathtt{3}}) =& \uw{\mathtt{1} \blt \mathtt{2} \blt \mathtt{3}} + \frac{1}{2} \uw{\mathtt{1}\mathtt{2} \blt \mathtt{3}} + \frac{1}{2} \uw{\mathtt{1}\blt \mathtt{2}\mathtt{3}} + \frac{1}{6}\uw{\mathtt{1}\mathtt{2}\mathtt{3}}\,,\\
\Psi_H(\uw{\mathtt{1} \blt \mathtt{2} \blt \mathtt{3}}) =& \uw{\mathtt{1} \blt \mathtt{2} \blt \mathtt{3}} - \frac{1}{2} \uw{\mathtt{1}\mathtt{2} \blt \mathtt{3}} - \frac{1}{2} \uw{\mathtt{1}\blt \mathtt{2}\mathtt{3}} + \frac{1}{3}\uw{\mathtt{1}\mathtt{2}\mathtt{3}}\,.
\end{align*}
The map $\Phi_H$ is a graded isomorphism from $(T(\polyd),\shuffle, \delta)$ to $(T(\polyd),\qshuffle, \delta)$.

The adjoints of $\Phi_H, \Psi_H$ with respect to $\langle \--, \-- \rangle$ are also explicitly described in \cite[Section 4.2]{hoffman2000}. Specifically, $\Phi_H^*, \Psi_H^* : T((\polyd)) \to T((\polyd))$  are the maps defined by the following identities:
\begin{equation}\label{eq:phistar}
\begin{split}
\Phi^*_H(\uw{p_1\blt \cdots \blt p_k}) \coloneqq \Phi^*_H(\uw{p_1}) \blt \cdots \blt \Phi^*_H(\uw{p_k})\,,  \;
&\Psi^*_H(\uw{p_1\blt \cdots \blt p_k}) \coloneqq \Psi^*_H(\uw{p_1}) \blt \cdots  \blt \Psi^*_H(\uw{p_k})\\
 \Phi^*_H(\uw{p}) \coloneqq \sum_{n \geq 1}\frac{1}{n!}\sum_{p_1 \cdots p_n = p} p_1 \blt \cdots \blt p_n\,,  \;
&\Psi^*_H(p) \coloneqq \sum_{n \geq 1}\frac{(-1)^{n-1}}{n}\sum_{p_1 \cdots p_n = p} \uw{p_1 \blt \cdots \blt p_n}\, . 
\end{split}
\end{equation}
for any $p\in \MS_d$. 
Moreover one has $p_1, \dots, p_k , i\in \MS_d$.
For instance, one has
\begin{align*}
\Phi^*_H(\uw{\mathtt{1}\mathtt{2}}) =& \uw{\mathtt{1}\mathtt{2}} + \frac{1}{2} \uw{\mathtt{1}\blt \mathtt{2}} + \frac{1}{2} \uw{\mathtt{2}\blt \mathtt{1}}\,, \quad \quad
\Psi^*_H(\uw{\mathtt{1}\mathtt{2}}) = \uw{\mathtt{1}\mathtt{2}} - \frac{1}{2} \uw{\mathtt{1}\blt \mathtt{2}} - \frac{1}{2} \uw{\mathtt{2}\blt \mathtt{1}}\,, \\
\Phi^*_H(\uw{\mathtt{1}\mathtt{2}\mathtt{3}}) =& \uw{\mathtt{1}\mathtt{2}\mathtt{3}} + \frac{1}{2} \uw{\mathtt{1}\mathtt{2} \blt \mathtt{3}} + \frac{1}{2} \uw{\mathtt{1}\mathtt{3} \blt \mathtt{2}} + \frac{1}{2} \uw{\mathtt{2}\mathtt{3} \blt \mathtt{1}} + \frac{1}{2} \uw{\mathtt{1} \blt \mathtt{2}\mathtt{3}} \\&+ \frac{1}{2} \uw{\mathtt{2} \blt \mathtt{1}\mathtt{3}} + \frac{1}{2} \uw{\mathtt{3} \blt \mathtt{1}\mathtt{2}} + \frac{1}{6}\sigma (\uw{\mathtt{1}\blt \mathtt{2}\blt \mathtt{3}})\\
\Psi^*_H(\uw{\mathtt{1}\mathtt{2}\mathtt{3}}) =&  \uw{\mathtt{1}\mathtt{2}\mathtt{3}} - \frac{1}{2} \uw{\mathtt{1}\mathtt{2} \blt \mathtt{3}} - \frac{1}{2} \uw{\mathtt{1}\mathtt{3} \blt \mathtt{2}} - \frac{1}{2} \uw{\mathtt{2}\mathtt{3} \blt \mathtt{1}} - \frac{1}{2} \uw{\mathtt{1} \blt \mathtt{2}\mathtt{3}} \\&- \frac{1}{2} \uw{\mathtt{2} \blt \mathtt{1}\mathtt{3}} - \frac{1}{2} \uw{\mathtt{3} \blt \mathtt{1}\mathtt{2}} + \frac{1}{3} \sigma (\uw{\mathtt{1}\blt \mathtt{2}\blt \mathtt{3}}) \, , 
\end{align*}
where $ \sigma (\uw{\mathtt{1}\blt \mathtt{2}\blt \mathtt{3}}) = \uw{\mathtt{1}\blt \mathtt{2}\blt \mathtt{3}} + \uw{\mathtt{2}\blt \mathtt{1}\blt \mathtt{3}} + \uw{\mathtt{3}\blt \mathtt{2}\blt \mathtt{1}} + \uw{\mathtt{1}\blt \mathtt{3}\blt \mathtt{2}} + \uw{\mathtt{2}\blt \mathtt{3}\blt \mathtt{1}} + \uw{\mathtt{3}\blt \mathtt{1}\blt \mathtt{2}}$.

Let $H(\polyd), H((\polyd))$ be the Hopf algebras dual to $(T(\polyd), \shuffle, \delta)$ and $(T(\polyd), \qshuffle, \delta)$ respectively.
Note that the product structure of $H(\polyd), H((\polyd))$ is $\bullet$.
Let us denote the respective coproducts by $\delta_{\shuffle}$, $\delta_{\qshuffle}$.
These were given explicitly in \cite{reutenauer1993free}.
In this way, the adjoint maps $\Phi_H^*, \Psi^*_H$ are \textbf{graded Hopf algebra isomorphism} between $H(\polyd)$ and $H((\polyd))$, see \textit{e.g.} \cite[Proposition 3.5]{Bellingeri2022}. 

The maps $\Phi, \Psi, \Phi^*$ and $\Psi^*$ are graded.
Therefore, for any $h\geq 0$ these maps restrict to isomorphisms of vector spaces.
For instance, $\Phi^*$ is an isomorphism  between the algebras $(T^{\leq h}(\polyd),\bullet_h)$ and $(T^{\leq h}(\polyd),\bullet_h)$. We display these maps in \cref{cd:level_h_lie}, in the context of two important subspaces of $T((\polyd))$, that we introduce in the next section.

In what follows, it is also useful to introduce the following shorthand notation for the combinatorial coefficients 
\[n_{h,d}= \sum_{\alpha\in C(h)}\binom{d-1 +\alpha_1}{\alpha_1}\cdots\binom{d-1 +\alpha_{|\alpha|}}{\alpha_{|\alpha|}}\,.\]
As shown in \cite[Remark 2.3]{Tapia20}, these coefficients describe the dimension  of the vector spaces  $T^h(\polyd)$,  the ambient space  for the varieties we define below.

  
%\raul{How do they prove in this paper if Thpoly was only defined in our paper? I think we will need to help the reader connect the dots more on this one}


\section{Discrete signatures and their varieties \label{sec:computations}}
We now recover the definition of discrete signatures from \cite{Tapia20}. 
In what follows, we consider $\KK$ to be the real or the complex field of numbers.
\begin{defin}
 Fix integers $d, N\geq 1$ and let $x = (x_1, \dots , x_N)$ be a finite time-series of elements in $\KK^d$. Denoting by $\Delta x_i = x_{i+1} - x_i \in \KK^d$, we define the \textbf{discrete signature} $\Dsign(x)\in T((\polyd))$ as the tensor series such that $\langle\Dsign(x), \varepsilon\rangle =1$ and
\begin{equation}\label{eq:dsign_unrep}
\langle \Dsign(x) , p_1 \blt \cdots \blt p_k \rangle \coloneqq \sum_{1\leq i_1 < \cdots < i_k < N} p_1(\Delta x_{i_1}) \cdots p_k(\Delta x_{i_k}) \, , 
\end{equation}
for any word $p_1 \blt \cdots \blt p_k \in \mathcal W  (\MS_d)$.
We use the convention that $\langle \Dsign(x) , p_1 \blt \cdots \blt p_k \rangle  = 0$ if the summing set in \eqref{eq:dsign_unrep} is empty. 
Given $h\geq 0$, we denote the projections of  $\Dsign(x)$ onto $T^{\leq h}(\polyd)$ and $T^{h}(\polyd)$ by $\Dsign^{\leq h}(x)$ and $\Dsign^{h}(x)$, respectively,  refered as \textbf{truncated discrete signature of height} $h$ and \textbf{discrete signature of height} $h$.
\end{defin}


We note that $\langle\Dsign(x), w\rangle= 0$ whenever  $|w|\geq N$, therefore $\Dsign(x)$ and its projection on the vector space $\spn \{  w \in \mathcal W(\MS_d) \colon\,  |w| <  N\}$ are the same. 

\begin{smpl}\label{ex_1}
By identifying a time-series $x = (x_1, \ldots , x_N)$  with a $d\times N$ matrix whose $i$-th column $x_i$ has coefficients $(x_i^1, \ldots , x_i^d)^T$, we can efficiently describe truncated signatures. For instance, one has 
\begin{align*}
\Dsign^{\leq 2}\left(\begin{bmatrix}
1&2&3\\
2&3&2
\end{bmatrix}\right) = \uw{\varepsilon} &+ 2 \, \uw{\mathtt{1}} + 0\, \uw{\mathtt{2}} + 2\, \uw{\mathtt{1}\mathtt{1}} + 0\, \uw{\mathtt{1}\mathtt{2}} + 2\, \uw{\mathtt{2}\mathtt{2}}\\
& + \, \uw{\mathtt{1}\blts \mathtt{1}} -\, \uw{\mathtt{1}\bltS \mathtt{2}} -\, \uw{\mathtt{2}\bltS \mathtt{1}} + \,\uw{\mathtt{2}\blts \mathtt{2}} \, .
\end{align*}
More generally,  we have the formula
\begin{align*}
\Dsign^{\leq 2}\left(\begin{bmatrix}
x_1^1&x_2^1&x_3^1\\
x_1^2&x_2^2&x_3^2
\end{bmatrix}\right) =& \uw{\varepsilon} + (x_3^1-x_1^1) \, \uw{\mathtt{1}} + (x_3^2-x_1^2)\, \uw{\mathtt{2}} + \\&\, ((x_2^1-x_1^1)^2+(x_3^1-x_2^1)^2) \,\uw{\mathtt{1}\mathtt{1}} + ((x_2^2-x_1^2)^2+(x_3^2-x_2^2)^2)\, \uw{\mathtt{2}\mathtt{2}}\\&+ ((x_2^2-x_1^2)(x_2^1-x_1^1)+(x_3^2-x_2^2)(x_3^1-x_2^1))\, \uw{\mathtt{1}\mathtt{2}} + \\
& + (x_2^1-x_1^1)(x_3^1-x_2^1)\, \uw{\mathtt{1}\blts \mathtt{1}} + (x_2^1-x_1^1)(x_3^2-x_2^2)\, \uw{\mathtt{1}\bltS \mathtt{2}} \\&+ (x_2^2-x_1^2)(x_3^1-x_2^1)\, \uw{\mathtt{2}\bltS \mathtt{1}} + (x_2^2-x_1^2)(x_3^2-x_2^2)\,\uw{\mathtt{2}\blts \mathtt{2}} \, .
\end{align*}
\end{smpl}

\begin{smpl}[The Canonical Axis time-series]
For any given $d\geq 1$ we set $N=d+1$ and we consider the following time-series  $x^{\text{axis}}$   defined by $x^{\text{axis}}_1=0$  and the recursive condition $x^{\text{axis}}_{i+1}= x_{i}^{\text{axis}}+ e_i$
where $e_i$ for  $i=1, \ldots, d$ is the canonical basis  of $\R^d$. In few words,  $x^{\text{axis}}$ consists of $d$ steps from the zero vector $0$ to the unit vector $(1, \ldots, 1)^T$. Since the evaluation of a monomial $X_{j_1}\cdots X_{j_l}$ on the vector $e_i$ is non-zero if and only if $j_i= \ldots =j_l=i$,  the entry $\langle \Dsign(x^{\text{axis}}), w\rangle$  is zero unless the word $w$ has the form  $w=i_1\ldots i_{1}\blt  i_2\ldots i_2 \blt \cdots  \blt  i_l\ldots i_l$  with $i_1\leq i_2 < \cdots < i_l$.  In that case, it equals to $1$. For instance, when $d=2$ one has 
\[\langle\Dsign(x^{\text{axis}}), \uw{\mathtt{1}\mathtt{1}}\rangle= \langle\Dsign(x^{\text{axis}}),\uw{\mathtt{2}\mathtt{2}}\rangle=\langle\Dsign(x^{\text{axis}}), \uw{\mathtt{1}\blt\mathtt{2}}\rangle=1\,,\]
\[\langle\Dsign(x^{\text{axis}}), \uw{\mathtt{1}\mathtt{2}}\rangle= \langle\Dsign(x^{\text{axis}}), \uw{\mathtt{2}\blt\mathtt{1}}\rangle=\langle\Dsign(x^{\text{axis}}), \uw{\mathtt{2}\mathtt{2}}\rangle= \langle\Dsign(x^{\text{axis}}), \uw{\mathtt{1}\blt\mathtt{1}}\rangle=0\,,\]
as one can also check from Example \ref{ex_1}.
\end{smpl}

 \begin{rem}\label{rm_lite}
The notion of discrete signature presented here coincides with the evaluation of the \textbf{iterated-sums signature} $\text{ISS}(x)_{n,m}\in  T((\polyd))$ (see \cite[Definition 3.1]{Tapia20}) on an infinite time-series $x\colon \mathbb{N}\to \KK^d$ defined by
\begin{equation}\label{eq:ISS}
\langle \text{ISS}(x)_{n,m} , p_1 \blt \cdots \blt p_k \rangle \coloneqq \sum_{n\leq i_1 < \cdots < i_k < m} p_1(\Delta x_{i_1}) \cdots p_k(\Delta x_{i_k}) \, , 
\end{equation}
for any $n<m$.
In particular, one has that $\text{ISS}(x)_{n,m} = \Dsign(x_n, x_{n+1}, \ldots, x_m)$.
%If one identifies a finitte time-series of $N$ elements in $\KK^d$ with an infinite time-series $x\colon \mathbb{N}\to \KK^d$ with $x_M = x_N$ for $M>N$, one has $ \Dsign(x)=\text{ISS}(x)_{0,N}$.  
\end{rem}

Discrete signatures can be easily computed thanks to some general  algebraic identities  among their coefficients, which justify the use of quasi-shuffle Hopf algebras. To state them precisely, we define concatenation of time-series.
\begin{defin}
Let  $ x=(x_1, \dots , x_N)$ and $ y=(y_1, \dots , y_M)$  be two time-series, we define the \textbf{concatenated time-series} $x|y\colon \{1, \ldots, M+N\}\to \KK^d$ by the conditions
\[(x|y)_k:=\begin{cases}x_k &k \in \{1, \ldots,N\}\\ y_{k-N} - y_1 +x_N&k \in \{N+1, \ldots,N+M\}\end{cases} \]
\end{defin}

\begin{thm}\label{thm_prop_disc_sig}
The discrete signature has the following properties:

\begin{itemize}
\item  (Time-warping invariance) for any integer $M\geq 1$ one has 
\begin{equation}\label{eq_time_warp}
\Dsign (x_1, \dots , \underbrace{x_N, \dots, x_N}_{M \; \text{times}}) = \Dsign (x_1, \dots , x_N)\,.
\end{equation}






\item (Quasi-shuffle identity)
for any two words $w, v$ in $\MS_d$ we have
\begin{equation} \label{thm:qsrels}
\langle \Dsign (x), w \qshuffle v \rangle = \langle \Dsign (x), w \rangle \langle \Dsign (x), v \rangle \, .
\end{equation}
\item (Chen's identity) For any two time-series  $ x=(x_1, \dots , x_N)$ and $ y=(y_1, \dots , y_M)$ one has 
\begin{equation}\label{eq_chen_equation}
\Dsign(x) \bullet\Dsign (y) = \Dsign(x|y)\,,
\end{equation}
\end{itemize}
\end{thm}

\begin{proof}
The first two properties follow directly from \cite[Section 4]{Tapia20} and \cite[Theorem 3.4, Part 1]{Tapia20}.  Concerning the Chen identity, this one follows directly from the Chen property on iterated-sums signatures (see \cite[Theorem 3.4, Part 2]{Tapia20}). Indeed, using the the notation of Remark \ref{rm_lite} with this result, we have 
\[\Dsign (x|y)=\text{DS}(x|y)_{1,N+M}=\text{DS}(x|y)_{1,N} \bullet\text{DS}(x|y)_{N,N+M}= \Dsign (x) \bullet \Dsign (y)\,. \]
\end{proof}
\begin{rem}\label{rem_trun_char}
By projecting the identity \eqref{eq_chen_equation} on $T^{\leq h}(\polyd)$ we also obtain the truncated identity
\begin{equation}\label{trunc_chen}
\Dsign^{\leq h}(x) \bullet_h\Dsign^{\leq h} (y) = \Dsign^{\leq h}(x|y)\,.
\end{equation}
Moreover  one has 
\[
\langle \Dsign^{\leq h} (x), w \qshuffle v \rangle = \langle \Dsign ^{\leq h} (x), w \rangle \langle \Dsign^{\leq h}  (x), v \rangle \, .
\]
for all words $w$ $v$ such that $||w||_{\he} + ||v||_{\he}  \leq h$.
\end{rem}
We now introduce the related varieties associated to the discrete signature. For any  fixed integer $h\geq 1$ we consider the discrete signature of height $h$ as a \textbf{polynomial mapping}  of the underlying time-series $x$, i.e.
\[\Dsign^h\colon \KK^{d\times   N}\to T^h(\polyd)\simeq \KK^{n_{d,h}}\,.\]
%Our primary goal is to fully describe  $\text{Im}(\Dsign^h)$, and we wish to do it algebraically. 
%A classic way to overcome this difficulty is to consider the \textbf{Zariski closure} of this set, namely we take the smallest algebraic variety containing $\text{Im}(\Dsign^h)$. 

Any coordinate of $\Dsign^h$ is a homogeneous polynomial of degree $h$. Therefore  $\Dsign^h$ induces a well-defined rational map among projective spaces 
\[(\Dsign^h)'\colon  \mathbb{P}^{dN-1}(\KK) \dashrightarrow \mathbb{P}^{n_{d,h}-1}(\KK). \]

This is a rational map, as there can be some values $\bar{x}\in \KK^{d \times N}\setminus \{0\}$  such that $\Dsign^h(\bar{x})=0 $. 
For any element $v$ of a vector space $V$ we denote by $[v]$ its equivalence class in the associated projective space. Every subset $U\subset V$  is associated to a  subset of the projective space $\mathbb{P}(V)$ by setting
\[[U]\coloneqq \{[u] | u \in U\setminus \{0\}\}\,.\]


\begin{defin}
Fix $d, h, N$ integers $\geq 1$ and set $\KK=\mathbb{C}$. We define the \textbf{affine discrete signature variety} $\V_{d, h, N}^{\text{af}}$ as the Zariski closure of the image of $\Dsign^h$ and the \textbf{discrete signature variety} is defined by setting $\V_{d, h, N}:=[\V_{d, h, N}^{\text{af}}]$. 
In the case $\KK=\mathbb{R}$, we refer to the associated varieties as \textbf{ real affine discrete signature variety} and \textbf{real discrete signature variety}, respectively.
\end{defin}

\begin{rem}
Over the complex field, $\V_{d, h, N}$ is a homogeneous variety, so we can consider its projectivisation (see \cite[Section 5.2]{shafarevich1994basic}), which is the image of $(\Dsign^h)'$. 
In this paper we do not work with manifolds but we define the manifold structure of the real discrete signature variety, for the application of time-series and for future works.
\end{rem}

\begin{rem}\label{rk_par}
An alternative way to parametrise $\V_{d, h, N}$ is obtained by expressing $\Dsign$ using the \textbf{difference time-series}. 
That is, for any $y\in \KK^{d(N-1)}$ and any word $w = p_1 \blt \cdots \blt p_k$, consider the map $ \Dsign_d$ such that:
\begin{equation}\label{eq:dsign_rep}
\langle \Dsign_d(y) , \uw{w} \rangle \coloneqq \sum_{1 \leq i_1 < \cdots < i_k <N} p_1(y_{i_1}) \cdots p_k(y_{i_k}) \, . 
\end{equation}
Note that $\im \Dsign = \im \Dsign_d$.
This reparametrisation allows for a more tractable computer assisted calculation, as it reduces de input space of $\Dsign$ and it simplifies implementation. 
We will use this  parametrisation only in the explicit description of the varieties with height $1$, $2$ and $3$.
\end{rem}



For any $h$ and $d$, let $N =1$ and note that $\V_{d, h, 1}= \emptyset$. 
We see that our variety of interest behaves well with the variation of the index $N$.



\begin{thm}\label{thm:chain}
For any pair of integers  $1\leq N\leq M$ one  has $\V_{d, h, N}\subseteq \V_{d, h, M} $. Moreover, there exists an integer $N'$ depending on $d$ and $h$ such that the following ascending chain of varieties stabilizes.
\[
 \V_{d, h, 1}\subseteq \V_{d, h, 2} \subseteq \cdots  \subseteq  \V_{d, h, N'}= \V_{d, h, N'+1}= \cdots \, \, .
\]
We call $ \V_{d, h, N'}$ the 
  \textbf{universal discrete signature variety} and we denote it by $\mathcal{V}_{d,k}$.
\end{thm}

%\raul{I dont understand why we change letter here from V to U}

\begin{proof}
From the time-warping invariance equation in \eqref{eq_time_warp}, the image of the $\Dsign^h$ on time-series in $\KK^{d\times   N}$ is included in the image of the $\Dsign^h$ on time-series in $\KK^{d\times M}$. 
This concludes the chain claim. To note that the variaties stabilise, we use the classical fact that the vanishing ideal of a parametrised variety is a prime ideal.
Indeed, if $\phi: \KK [\mathbf{y} ] \to \KK [\mathbf{x}]$ is a parametrisation of a variety $V$, its vanishing ideal $I(V) = \ker \phi$ is the kernel of a ring homomorphism on an integral domain, so $I(V)$ is prime.

By considering the associated family of ideals (see for instance \cite{cox2006using}), we obtain a descending chain of prime ideals in a polynomial ring on finite variables.  
This stabilises from Krull's principal ideal theorem.
\end{proof}

We describe some general facts about $\V_{d, h, N}$.
%with different choices of the parameters $d, h, N$.

%We observe that when using the new parametrisation, adding zero-columns to a matrix does not change $\Dsign^{\leq 2}$.
%For instance we have:
%\begin{align*}
%\Dsign^{\leq 2}\left(\begin{bmatrix}
%1&0&2&3&0\\
%2&0&3&2&0
%\end{bmatrix}\right) = \uw{\varepsilon} + &6 \, \uw{\mathtt{1}} + 7\, \uw{\mathtt{2}} + 14\, \uw{\mathtt{1}\mathtt{1}} + 14\, \uw{\mathtt{1}\mathtt{2}} + 17\, \uw{\mathtt{2}\mathtt{2}}\\
%& + 11\, \uw{\mathtt{1}\bltS \mathtt{1}} + 9\, \uw{\mathtt{1}\bltS \mathtt{2}} + 19\, \uw{\mathtt{2}\bltS \mathtt{1}} + 16\, \uw{\mathtt{2}\blt \mathtt{2}} \, . 
%\end{align*}









%This means that $\Dsign^h (x) \in \hat{\mathcal G}^h(\polyd)$.
%In particular, $\V_{d, h} \subseteq \hat{\mathcal G}^h(\polyd)$.
%This was shown in \cite[Theorem 3.4]{Tapia20}.
%In \cref{conj:dim} we conjecture that this inclusion is tight.

\subsection{The height one varieties}

For completeness sake we include this case here. For $h=1$ and any field $\KK$ of $0$ characteristic, the map $\Dsign^1$  is a linear map  with values in  $ T^1(\polyd)= \langle \mathtt{1},\ldots,\mathtt{d}\rangle $. In this case, for any $N\geq 2$, we have trivially  $\V^{\text{af}}_{d, 1, N}\cong \KK^d$ and therefore $\V_{d, 1, N} \cong  \mathbb{P}^{d-1}(\KK) $.



\subsection{The height two varieties}
We consider now discrete  signature varieties of height $2$ when $\KK= \mathbb{C}$. 
By definition of $T^2(\polyd)$, we can actually write  the direct sum
\[T^2(\polyd)= \langle \{\mathtt{i} \bullet \mathtt{j} \colon \mathtt{i} ,\mathtt{j}  \in [d] \}\rangle\oplus \langle \{\mathtt{i}  \mathtt{j} \colon \mathtt{i} \leq \mathtt{j} \in [d] \}\rangle\,.\]
Therefore we can easily identify every element $ T^2(\polyd)$ as a sum  of a  $d\times d$ matrix and a  symmetric $d\times d$ matrix. 
For any time-series $x$ we can describe this decomposition with the coefficients
\[\Dsign_{\mathtt{i}\bullet\mathtt{j}}\coloneqq \langle\Dsign(x),\mathtt{i}\bullet \mathtt{j} \rangle\,, \quad \Dsign_{\mathtt{i}\mathtt{j}}\coloneqq\langle\Dsign(x),\mathtt{i}\mathtt{j}\rangle\,.\]

Thanks to the quasi-shuffle identities \eqref{thm:qsrels}, we deduce that these coefficients
satisfy the relation
\[ \Dsign_{\mathtt{j}\bullet\mathtt{i}}+\Dsign_{\mathtt{i}\bullet\mathtt{j}}+\Dsign_{\mathtt{i}\mathtt{j}}=\langle\Dsign(x),\mathtt{i} \rangle\langle\Dsign(x), \mathtt{j} \rangle =(x_N^i- x_1^i) (x_N^j- x_1^i)\,.\]
Since the matrix with coefficients $(x_N^i- x_1^i) (x_N^j- x_1^j) $  has rank $\leq 1$, we obtain that $\V_{d, 2, N}$ is always included in the variety of the projective equivalence class of $d\times d$ matrices $A=S+M$ with $S$ symmetric and  $M$ generic such that 
\begin{equation}\label{eq:rk1cond}
\rk (S+ M+ M^\top)\leq 1.
\end{equation}
% This type of variety have indeed a very specific structure and very well known in the literature. \carlo{add reference}.
In \cref{sec:conj} we show the converse. 
That is, for any element $s \in T^2(\polyd)$ that satisfies \eqref{eq:rk1cond} there exists a time-series $x$ such that $\Dsign^2(x) = s$.
For instance, when $d=2$ this is equivalent to impose that the matrix
$$\begin{bmatrix}
2\Dsign_{\mathtt{1}\bullet \mathtt{1}} + \Dsign_{\mathtt{1} \mathtt{1}} &\Dsign_{ \mathtt{1} \mathtt{2}} + \Dsign_{\mathtt{1}\bullet  \mathtt{2}}+\Dsign_{ \mathtt{1}\bullet \mathtt{2}}\\
\Dsign_{ \mathtt{1} \mathtt{2}} + \Dsign_{\mathtt{1}\bullet  \mathtt{2}}+\Dsign_{ \mathtt{1}\bullet \mathtt{2}}& 2\Dsign_{\mathtt{2}\bullet \mathtt{2}} + \Dsign_{\mathtt{2} \mathtt{2}}
\end{bmatrix}\, .$$
is non-singular.
%, i.e. we search the solution $[\Dsign_{\mathtt{1} \mathtt{1}}: \Dsign_{\mathtt{1} \mathtt{2}}: \Dsign_{\mathtt{2} \mathtt{2}}: \Dsign_{\mathtt{1} \bullet \mathtt{1}}:\Dsign_{\mathtt{1} \bullet\mathtt{2}}:\Dsign_{\mathtt{2}\bullet \mathtt{1}}:\Dsign_{\mathtt{2}\bullet \mathtt{2}}]\in \mathbb{P}^6(\mathbb{C})$ of the quadratic equation 
%\[
%\left(2\Dsign_{\mathtt{1}\bullet \mathtt{1}} + \Dsign_{\mathtt{1} \mathtt{1}}\right)\left(2\Dsign_{\mathtt{2}\bullet \mathtt{2}} + \Dsign_{\mathtt{2} \mathtt{2}} \right) =  \left(\Dsign_{ \mathtt{1} \mathtt{2}} + \Dsign_{\mathtt{1}\bullet  \mathtt{2}}+\Dsign_{ \mathtt{1}\bullet \mathtt{2}}\right)^2\, .
%\]

%By dimension counting, together with \cref{dim:cord2} we get the following generator result:

%Moreover, For each fixed value of $d$, we have the following chain of  varieties in $\mathbb{P}^{n_{2,d}-1}(\mathbb{C})$:
%\begin{equation}
%\label{eq:Mchain}
%\mathcal{M}_{d,1} \subset \mathcal{M}_{d,2} \subset \mathcal{M}_{d,3} \subset \,
%\cdots \,\subset \mathcal{M}_{d,d} = \mathcal{M}_{d,d+1}
%= \mathcal{M}_{d,d+2} = \cdots
%\end{equation}


%coinThe quadratic equations given in \eqref{eq:h2gens} generate the variety .
%Furthermore, because the equations in \eqref{eq:h2gens} are all transversal, the degree of this variety is $2^{\binom{d}{2}}$





%This gives us the following equation on height two discrete signatures




\subsection{Paths on the line}

We now look at the case where $d = 1$ and $h$ is generic. In this setting, $\MS_1^0$ contains only multisets of the singleton $\{\mathtt{1}\}$ and  $T^{h}(\polyd)$  has the  dimension $n_{1,h}=|C(h)|= 2^{h-1}$. For any $h\geq 1$ this results comes also with a natural identification  between any $w  \in \mathcal{W}(\MS_1)$ such that $ \Vert w\Vert_{\text{ht}}=h$ with a  compositions of $h$.   For instance, when $h=5$, we identify $w = 11 \bullet 1 \bullet 11$ with the composition $(2, 1, 2)$ of $5$.

Using this notation, we can easily reinterpret the map $\Dsign^h$ using the theory of quasi-symmetric functions over the field $\KK$, see e.g. \cite{hazewinkel2001algebra}. A classic basis of this subring of formal power series in the  variables $y_{1},y_{2},y_{3},\ldots$ is given for any composition  $\alpha=(\alpha_1, \dots, \alpha_k)$ by the monomial formal series 
$$M_{(\alpha_1, \dots, \alpha_k)} = \sum_{1 \leq i_1 < 
\dots < i_k} y_{i_1}^{\alpha_1} \cdots y_{i_k}^{\alpha_k} \, . $$ 
Denoting by $M_w $ the monomial quasi-symmetric function indexed by the composition of $w\in \mathcal{W}(\MS_1)$, we immediately see that the coefficient  $\langle \Dsign^h (x), w \rangle $ is the evaluation of $M_w$ on $y=\Delta x$ appended with zeroes.

%For instance, if we consider $w = 1\bltS 11$, this corresponds to the composition $\alpha = (1, 2)$.If $x$ is the time-series $(1, 4, 2, 3)$ in $\R^1$, then$$\langle \Dsign (x), w \rangle = M_{(1, 2)}(1, 4, 2, 3, 0, 0, 0, \dots ) = 1\cdot 4^2 + 1\cdot 2^2 +1\cdot 3^2 +4\cdot 2^2 +4\cdot 3^2 +2\cdot 3^2 \, .$$



We recall that \cite[Theorem 8.1]{hazewinkel2001algebra} has shown that, the algebra of quasi-symmetric functions, is freely generated over the integers, and indeed gives a generating set for this algebra. From this  structure we will see that any variety $\V_{1, h, N}$ is embedded 
in a class of varieties whose dimension can be explicitly computed. We add that this result is independent of the characteristic of the underlying field.

%
%\begin{prop}
%The variety $\V_{1, h, N}$ has the dimension expected in \cref{conj:dim}.
%\end{prop}
\begin{smpl}
We analyse here further particular cases with $d = 1$ using the software   \texttt{Macaulay2} \cite{Mac2}.  

For $h = 3$ and $N= 4$, the signature variety is embedded in $\mathbb{P}^3(\mathbb{C})$. Using the parametrisation given in Remark \ref{rk_par}, the variety $ \mathcal{V}_{1,3,4}$  is associated to the rational map $\mathbb{P}^2(\mathbb{C}) \dashrightarrow \mathbb{P}^3(\mathbb{C})$ explicitly given by
$$ [y_1: y_2: y_3] \to [y_1  y_2   y_3: y_1^2  y_2+y_1^2  y_3+y_2^2 y_3: y_1  y_2^2 +y_1  y_3^2 +y_2  y_3^2: y_1^3+ y_2^3+ y_3^3] \, .$$
Using the notation $s_{(1,1,1)}, s_{(1,2)}, s_{(2,1)}, s_{3}$ for the coordinates of $\mathbb{C}^4$, we can see that the variety $\V_{1, 3, 3}$ is generated by one  equation of degree nine. Here is an excerpt of this equation:
\begin{align*}
81 &s_{(1,1,1)}^9+162 s_{(1,1,1)}^8 s_{(1,2)}+351 s_{(1,1,1)}^7 s_{(1,2)}^2+333 s_{(1,1,1)}^6 s_{(1,2)}^3\\&+72 s_{(1,1,1)}^5 s_{(1,2)}^4-63 s_{(1,1,1)}^4 s_{(1,2)}^5-30 s_{(1,1,1)}^3 s_{(1,2)}^6\\&+6 s_{(1,1,1)}^2 s_{(1,2)}^7+
6 s_{(1,1,1)} s_{(1,2)}^8+s_{(1,2)}^9+162 s_{(1,1,1)}^8 s_{(2,1)}-30 s_{(1,1,1)}^2 s_{(1,2)} s_{(2,1)}^6\\&+ \cdots +4 s_{(1,1,1)}^3 s_{(2,1)}^2 s_{3}^4+4 s_{(1,1,1)}^2 s_{(1,2)} s_{(2,1)}^2 s_{3}^4+s_{(1,1,1)} s_{(1,2)}^2 s_{(2,1)}^2 s_{3}^4\, .
\end{align*}
The dimension of the resulting variety $\mathcal{V}_{1,3,4}$ is two.
%\carlo{check again the computations (there was a mistake in the polynomial map)}




For $h = 4$ and $N= 4$, the variety $\V_{1, 4, 4}$ is embedded in $\mathbb{P}^7(\mathbb{C})$ and is generated by $20$ polynomials, whose degree counts are the following:

\begin{center}
\begin{tabular}{l|c|c|c|c}
Degree & 1 & 2 & 3 & 4\\
Quantity &1& 1& 12& 6
\end{tabular}
\end{center}

%The degree one polynomial is $s_{000}$, the degree two polynomial is:
%\begin{align*}
%s_{001}^2&+2 s_{001} s_{010}+s_{010}^2+2 s_{001} s_{011}+2 s_{010} s_{011}+s_{011}^2+2 s_{001} s_{100}+2 s_{010} s_{100}+2 s_{011} s_{100}\\
%&+s_{100}^2-4 s_{001} s_{101}-4 s_{010} s_{101}-4 s_{100} s_{101}-2 s_{101}^2+2 s_{001} s_{110}+2 s_{010} s_{110}+2 s_{011} s_{110}\\
%&+2 s_{100} s_{110}+s_{110}^2-2 s_{001} s_{111}-2 s_{010} s_{111}-2 s_{100} s_{111}-s_{101} s_{111}\, .
%\end{align*}
\end{smpl}
\subsection{Three steps}

Let us now focus on the case $N = 4$, $d = 2$ and $h = 3$. Using the parametrisation given in Remark \ref{rk_par}, the variety $ \mathcal{V}_{2,3,3}$, we rename some coordinates of $T^3(\polyd)$ as follows:

\begin{itemize}
\item For $w = \el_1 \bltS \el_2 \bltS \el_3$, we write $s_{\el_1, \el_2, \el_3}$ for $\langle \Dsign (y), w \rangle$.

\item For $w = \el_1 \el_2 \bltS \el_3$, we write $t_{\el_1, \el_2, \el_3}$ for $\langle \Dsign (y), w \rangle$.

\item For $w = \el_1 \bltS \el_2 \el_3$, we write $u_{\el_1, \el_2, \el_3}$ for $\langle \Dsign (y), w \rangle$.

\item For $w = \el_1 \el_2 \el_3$, we write $v_{\el_1, \el_2, \el_3}$ for $\langle \Dsign (x), w \rangle$.
\end{itemize}

If $y = \begin{bmatrix}
a_{\mathtt{1}} & b_{\mathtt{1}} & c_{\mathtt{1}}\\
a_{\mathtt{2}} & b_{\mathtt{2}} & c_{\mathtt{2}}
\end{bmatrix}$, then we have the following parametric equations:
\begin{align*}
s_{i, j, k} =& a_ib_jc_k\\
t_{i, j, k} =& a_ia_jb_k+ a_ia_jc_k+ b_ib_jc_k\\
u_{i, j, k} =& a_ib_jb_k+ a_ic_jc_k+ b_ic_jc_k\\
v_{i, j, k} =& a_ia_ja_k+ b_ib_jb_k+ c_ic_jc_k\, .
\end{align*}


There are 226 minimal generators of the ideal associated to $\mathcal \V_{2, 3, 4}$ of degree at most four. These break down into $58$ quadrics, $74$ cubic and $134$ quartic generators. Here is one of the cubics generating $\mathcal I(\V_{2, 3, 3})$.
\begin{align*}
&s_{121}s_{222}v_{222} - s_{121}t_{222}u_{222}-s_{122}s_{222}v_{122} + s_{122}t_{222}u_{212}\\
&-s_{221}s_{222}v_{122} + s_{221}t_{122}u_{222}+s_{222}^2v_{112}-s_{222}t_{122}u_{212}\, .
\end{align*}


\subsection{Overview table}

We sum up some dimensions, degrees and also display some of the generators obtained with the software \texttt{Macaulay2}. We mention that a general description of the varieties $\V_{d, h, N}$ using only computational tools is a computationally hard question, as even for small sizes like in $\V_{2, 2, 4}$, computing a Gr\"obner basis is too expensive.



\begin{center}
\begin{tabular}{|l| c | c | l|}
\hline
Variety & Dimension & Degree & Some generators\\
\hline
$\V_{1, 3, 3}$ & $2$ & $81$ & $ 81s_{00}^9+162s_{00}^8s_{01}+351s_{00}^7s_{01}^2+333s_{00}^6s_{01}^3 \cdots  $\\
$\V_{1, 4, 3}$ & $2$ & $65536$ & $ s_{001}s_{011}^2-12s_{001}s_{100}^2-16s_{010}s_{100}^2-5s_{011}s_{100}^2 \cdots $\\
$\V_{2, 2, 3}$ & $5$ & $128$ & $s_{12}^2+2s_{12}s_{21}+s_{21}^2-4s_{11}s_{22}-2s_{22}t_{11}$\\
& & & \quad $ -2s_{11}t_{22} -t_{11}t_{22}+2s_{12}t_{12}+2s_{21}t_{12}+t_{12}^2 $\\
$\V_{2, 3, 2}$ & $3$ & $282429536481$ & $u_{212}v_{112}-u_{211}v_{122}-u_{112}v_{122}+u_{111}v_{222}$\\
$\V_{2, 3, 3}$ & $7$ & $28$ & $ s_{12}^2+2s_{12}s_{21}+s_{21}^2-4s_{11}s_{22}-2s_{22}t_{11} $ \\
 & & & \quad $-2s_{11}t_{22}-t_{11}t_{22}+2s_{12}t_{12}+2s_{21}t_{12}+t_{12}^2$ \\
$\V_{3, 2, 2}$ & $5$ & $32768$ & $t_{23}t_{32}-t_{22}t_{33}, t_{13}t_{32}-t_{12}t_{33}, \cdots $ \\
\hline
\end{tabular}
\end{center}



\section{Lie polynomials and Lie groups}\label{sec_Lie}



In this section, we introduce some fundamental subsets of $T^{\leq h}(\polyd)$: the space of Lie polynomials and the variety of group-like elements in the shuffle and quasi-shuffle context. These will play a role in understanding discrete signatures and, as the name suggests, they will form pairs of Lie algebra and Lie group. Most of results in this section will from general facts on the theory of free Lie algebras \cite{Reutenauer89} which will be adapted to take in account our intrinsic notion of height of words. The underlying field $\KK$ will just be of characteristic zero.

We define the \textbf{truncated Lie bracket} $[ \--, \-- ]_h : T^{\leq h}(\polyd) \otimes T^{\leq h}(\polyd) \to T^{\leq h}(\polyd)$ by setting 
 \[[v, w]_h \coloneqq w\bullet_h v - w\bullet_h v\,,\]
and extending it linearly. 
We define  $\mathcal L^{\leq h}(\polyd)$ as the Lie subalgebra of $T^{\leq h}(\polyd)$ generated by  $\polyd^{\leq h}$.  Equivalently, $\mathcal L^{\leq h}(\polyd)$ is the space of iterated Lie brackets starting from the finite dimensional vector space $\polyd^{\leq h}$. We refer to this Lie algebra as the set of \textbf{height-$h$ Lie polynomials}.  

This Lie algebra  will be finitely generated by an explicit family of elements.   In what follows, we introduce for any $w\in  \mathcal{W}(\MS_d)$ the family of Lie polynomials $\lv_{w}\in T((\polyd))$ defined by $\lv_{\varepsilon} = 0$ and  the  recursive condition
 \begin{equation}\label{eq:prims}
 \lv_{w} = [p_1, \lv_{u}]\,, 
 \end{equation}
 where $w= p_1\blt u$ and $[\,, ]$ is  the Lie bracket without truncation on $T((\polyd))$
\[[v, w] \coloneqq w\bullet v - w\bullet v\, ,.\]
To describe $\mathcal L^{\leq h}(\polyd)$, we introduce an intermediary Lie algebra that sits between $\mathcal L^{\leq h}(\polyd)$ and $T((\polyd))$. We denote by $\mathfrak{L}^{h}$ the \textbf{free $h$-step nilpotent Lie algebra} with respect to $[\,,]$. More explicitly, one has 
 \[\mathfrak{L}^{h}= \polydh\oplus [\polydh, \polydh]\oplus\cdots \oplus \underbrace{ [\polydh ,[ \polydh \ldots, [\polydh ,  \polydh ] }_{\text{$h-1$ times}}\, ,\]
where we set that any iterated sequence of $h$ brackets vanishes, see 
\cite[Definition 7.25]{frizbook}.
Using $\mathfrak{L}^{h}$, we can actually write $\mathcal L^{\leq h}(\polyd)$ as a quotient.
\begin{lm}\label{prop_quotient}
The space $\mathcal L^{\leq h}(\polyd)$ is a Lie algebra which arises as the quotient 
\begin{equation}\label{trunc_pol}
\mathcal L^{\leq h}(\polyd)=\mathfrak{L}^{ h}/ \left(T^{>h}(\polyd )\cap\mathfrak{L}^h \right)\,.
\end{equation}
\end{lm}



\begin{proof}
By definition of $\bullet_h$,  the Lie product $[\,,]_h$ is $h$-step nilpotent. Therefore we can use the fundamental property of $\mathfrak{L}^h$ as the free $h$-step nilpotent Lie algebra (see \cite[Remark 7.26]{frizbook}). This gives us an explicit Lie algebra morphism $\mathfrak{i}\colon \mathfrak{L}^h\to \mathcal L^{\leq h}(\polyd)$ lifting the identity map on $\polydh $ and sending iterated $[\,, ]$ Lie polynomials into $[\,,]_h$ polynomials. By definition of  $\mathcal L^{\leq h}(\polyd)$, the map $\mathfrak{i}$ is surjective  and one can see that its kernel is $T^{>h}(\polyd )\cap \mathfrak{L}^h $. 
\end{proof}
%Notice how  $\mathcal L^{\leq h}(\polyd) \cap \polyd = \polyd^{\leq h}$, so the principal elements of $\mathcal L^{\leq h}(\polyd) $ are $\polyd^{\leq h}$, and $\mathcal L^{\leq h}(\polyd) $ is finitely generated by iterated bracket.

%As a consequence, by projecting onto $T^{\leq h}(\polyd)$ any generating set or basis of $\mathfrak{L}^h$ one has a corresponding basis or generating set for $\mathcal{L}^{\leq h}(\polyd)$.
  From the definition of $\mathfrak{L}^h$ one has immediately that $\{ \lv_w : ||w||_{\he} \leq h\}$ is a generating set for $\mathcal L^{\leq h}(\polyd)$. Additional properties of $\mathfrak{L}^h$ allows us to write down an explicit basis for $\mathcal L^{\leq h}(\polyd)$. Using the total order on $\MS_d$,  we can order all words $ \mathcal{W} (\MS_d)$ with lexicographic order $\geq_{lex}$. A word $I\in \mathcal{W} (\MS_d)$ is said a \textbf{Lyndon word} if it is strictly smaller than all of its rotations. For any Lyndon word $I$ we can uniquely associate an iterated Lie bracket $b(I)\in T((\polyd))$ called \textbf{Lyndon bracket}. The definition of $b(I)$ follows by induction on $|I|$ by setting $b(I)=I$ if $|I|=1$ and  $b(I)= [b(I_1), b(I_2)]$ where $I= I_1\bullet I_2$ and $I_2$ is the longest Lyndon word appearing as a proper right factor of $I$.

\begin{lm}\label{dim_lie}
The family of Lyndon brackets $b(I)$ where $I$ is a Lyndon word of height smaller than $h$ is a basis for $\mathcal L^{\leq h}(\polyd) $. The dimension of this Lie algebra equals to 
\[\Lambda_{d,l}:=\sum_{l=1}^h\lambda_{d,l}\,,\]
where $\lambda_{d,l}$ denotes the number of Lyndon words in $\mathcal W (\MS_d)$ of heigth $l$.
\end{lm}
\begin{proof}
Following \cite[Proposition 4.7]{amendola2019varieties},  we know that the the family of Lyndon brackets $b(I)$ where $I$ is a Lyndon words of length smaller than $h$  is a basis for $\mathfrak{L}^{h}$. Since any Lyndon bracket $b(I)$ preserves the height of the underlying word $I$, the result follows from the the quotient \eqref{trunc_pol}.
\end{proof}


%The dimension of the universal variety $\dim \V_{d, h}$ arises as the number of Lyndon words with a specific height.
The tilting of the usual grading on word algebras with the introduction of a degree function is uncommon in the study of the coefficients $\lambda_{d,l}$. For instance, Lyndon words arise in the study of words on the alphabet $\{1, \dots, d\}$, taken with the degree function constant equal to one. There, words of length $h$ correspond to words of height $h$.
In that context, the number of Lyndon words of \textbf{height} $h$ in such alphabets is given  by
$$ \frac{1}{h}\sum_{k | h} \mu \left(\frac{h}{k}\right)d^k \, ,  $$
where $\mu $ is the M\"obius function on integers, see \cite[Section 0]{reutenauer1993free}. This reflects the fact that the height and the length play the same role for this alphabet. 

In our context, we do not have the luxury of having the same height and length on most words but we can still derive a general formula.

\begin{thm}\label{thm:enum}
The number of Lyndon words in $\mathcal W (\MS_d)$ of heigth $h$ is
\begin{equation}\label{eq_lyndon_words-height}
\lambda_{d, h} = \sum_{k|h} \frac{k}{h} \mu\left(\frac{h}{k} \right) \sum_{\alpha\in C(k)} \frac{1}{\ell (\alpha)} \prod_i \binom{\alpha_i + d - 1}{d - 1} \, . 
\end{equation}
\end{thm}

\begin{proof}
First we note that the height grading on $T(\polyd)$ gives us the following power series
$$H(x) = \sum_{w \in \mathcal W(\MS_d)} x^{\he (w) } = \frac{1}{1 - \sum_{I \in \MS_d} x^{\text{grad}(I)}} \, ,$$
where $\text{grad}(I)$ is the degree of the monomial associated to $I$. Furthermore, for any $h \geq 1 $ there are $\binom{h+d-1}{d-1}$ multisets on $\{1, \dots, d\}$ size $h$, so 
$$ \sum_{I \in \MS_d} x^{\text{grad}(I)} = \sum_{k\geq 1} \binom{k + d -1 }{d-1} x^ k = (1 - x)^{-d} - 1\, ,$$
thereby obtaining $H(x) = [1 - ( (1-x)^{-d} - 1)]^{-1}$.

On the other hand, the \textbf{Lyndon unique factorization theorem} (see \cite{chen1958free}) guarantees that each word $w \in \mathcal W (\MS_d)$ can be written uniquely as 
$$ w = \tau_1 \blt \cdots \blt \tau_j \, , $$
where $\tau_i $ are Lyndon words with $\tau_1 \geq_{lex} \dots \geq_{lex} \tau_k$. 
Therefore
\begin{align*}
H(x) &= \sum_{w \in \mathcal W(\MS_d)} x^{\he (w) } = \prod_{\substack{\tau \text{ Lyndon word} \\ \tau  \in \mathcal W(\MS_d )}}\left( 1 + x^{\he (\tau)} + x^{2\he (\tau)} + x^{3\he (\tau)} + \cdots  \right)  \\
&= \prod_{\substack{\tau \text{ Lyndon word} \\ \tau  \in \mathcal W(\MS_d }} ( 1 - x^{\he(\tau)} )^{-1} = \prod_{k\geq 1} ( 1 - x^{k} )^{-\lambda_{d, k}},
\end{align*}
where we recall that $\lambda_{d, k} $ is the number of Lyndon words $\tau \in \mathcal W(\MS_d)$ of length $k$.

Putting it together, applying $\log $ on both sides and using  the expansion
\[-\log(1-f(x) ) = \sum_{n\geq 1} \frac{1}{n}f(x)^ n\,,\] 
we get the equivalent series expansions
\begin{align*}
-\log(H(x) ) &= \sum_{k\geq 1} - \lambda_{d, k} \log ( 1 - x^{k} )  = \sum_{j, k\geq 1}\frac{1}{j} x^{j k} \lambda_{d, k}  = \\
&= \sum_{n\geq 1} x^n \sum_{k | n} \frac{1}{n/k} \lambda_{d, k}  = \sum_{n\geq 1} \frac{1}{n}x^n \sum_{k | n} k \lambda_{d, k} \\
\\
-\log(H(x) ) &= -\log [ 1 - ( (1-x)^{-d} - 1 ) ]  = \\
&= \sum_{k\geq 1} \frac{1}{k}\left(\sum_{t\geq 1} \binom{t + d -1 }{d-1} x^t \right)^k\\
&= \sum_{n\geq 0} x^n\sum_{\alpha \models n} \frac{1}{\ell(\alpha)}\prod_i \binom{\alpha_i + d - 1}{d-1}
\end{align*}

Equating both sides it follows that for each $n$ we have
$$ \sum_{k | n} k \lambda_{d, k} =   n \sum_{\alpha \models n} \frac{1}{\ell(\alpha)}\prod_i \binom{\alpha_i + d - 1}{d-1} \, .$$

Summing both sides for all $n$ divisors of $h$ and multiplying by $\mu\left(\frac{h}{n}\right)$ we get throught M\"obius inversion that
$$h \lambda_{d, h} = \sum_{n|h}  n\, \, \mu\left( \frac{h}{n} \right) \sum_{\alpha \models n} \frac{1}{\ell(\alpha)}\prod_i \binom{\alpha_i + d - 1}{d-1}\, ,$$
from which the theorem follows.
\end{proof}

\begin{rem}
It is somewhat surprising that the formula \eqref{eq_lyndon_words-height} always yields integers values. It is a corollary of the proof below that the intermediary terms
$$k \sum_{\alpha\in C(k)} \frac{1}{\ell (\alpha)} \prod_i \binom{\alpha_i + d - 1}{d - 1} $$
are also integers.
\end{rem}
%
This allows us to create the following values of $\lambda_{d, h} $. 

%Code created to generate this table can be found in \cite{M2_MATHREPO}.
\bigskip

\begin{center}
\begin{tabular}{|l|| c | c | c |c |c |c|c | c | c |}

\hline
$h$ & 1 & 2 & 3 & 4 & 5 & 6 & 7 & 8 &9 \\
\hline
$d = 1$&  1& 1& 2& 3& 6& 9& 18& 30& 56\\
$d = 2$&  2& 4& 12& 31& 92& 256& 772& 2291& 7000 \\
$d = 3$&  3 & 9& 36& 132& 534& 2140& 8982& 38031& 164150 \\
$d = 4$&  4& 16& 80& 380& 1960& 10228& 55352& 304223& 1700712 \\
$d = 5$&  5& 25& 150& 875& 5500& 35335& 234530& 1584845& 10885640 \\
$d = 6$&  6& 36& 252& 1743& 12936& 98686& 776412& 6226008& 50732712  \\
$d = 7$&  7& 49& 392& 3136& 26852& 237160& 2158156& 20028764& 188856934   \\
\hline
\end{tabular}
\end{center}
\bigskip

To pass from Lie algebras to Lie groups, we introduce the polynomial maps
\[\exp_{\bullet_h}\colon T^{\leq h}_0(\polyd) \to T^{\leq h}_1(\polyd) \, \quad \text{and} \quad \log_{\bullet_h}:  T^{\leq h}_1(\polyd ) \to T^{\leq h}_0(\polyd )\] defined by
\begin{equation}\label{exp_log_bullet}
\exp_{\bullet_h}(\vv)\coloneqq\sum_{n \geq 0 } \frac{1}{n!} \vv^{\bullet_h \,  n} \,, \quad \log_{\bullet_h}( v) \coloneqq \sum_{n\geq 0} \frac{(-1)^{n-1}}{n} (\vv- \varepsilon)^{ \bullet_h\, n }\, ,
\end{equation}
where $\vv^{\blt_h \, n} = \vv \blt_h \cdots \blt_h \vv$ stands for the $n$-th truncated bullet product. It is known in the literature, see e.g. \cite[Chapter 3]{reutenauer1993free},  that the equivalent definition of $\exp_{\bullet_h}$ and $\log_{\bullet_h}$ with the product $\blt$ are well defined map from $T_0((\polyd ))$ to $T_1((\polyd ))$ with the log being the inverse of the exponential. By simple quotienting, we obtain that  $\exp_{\blt_h}$ and $\log_{\blt_h}$ are well defined maps such that  $\log_{\bullet_h}= \exp_{\bullet_h}^{-1}$.

We define the \textbf{height-$h$ free nilpotent Lie group} $\mathcal{G}^{\leq h}(\polyd)$ as the image
\[\mathcal{G}^{\leq h}(\polyd)\coloneqq\exp_{\bullet_h}(\mathcal L^{\leq h}(\polyd))\subset T^{\leq h}_1(\polyd)\,. \]
Similarly as $\mathcal{L}^{\leq h}(\polyd)$, we can indeed describe $\mathcal{G}^{\leq h}(\polyd)$ as an explicit quotient of a more known Lie group. We denote by  $\mathfrak{G}^{h}$ the free nilpotent group of step $h$ over $\polydh$, see \cite[Theorem 7.30]{frizbook}, which is defined as
\[\mathfrak{G}^{h}= \exp_h( \mathfrak{L}^h)\]
where $\exp_h\colon T((\polyd))\to T((\polyd))$ is defined by
$\exp_h(\vv)\coloneqq\sum_{n \geq 0 }^h \frac{1}{n!} \vv^{\bullet \,  n}\,$, where $\vv^{\blt \, n} = \vv \blt \cdots \blt \vv$.

\begin{lm}
$\mathcal{G}^{\leq h}(\polyd)$ with the operation $\bullet_h$ is a Lie group  which arises as the quotient 
\begin{equation}\label{trunc_pol2}
\mathcal G^{\leq h}(\polyd)=\mathfrak{G}^{ h}/ \left(T^{>h}(\polyd )\cap\mathfrak{G}^h \right)\,.
\end{equation}
\end{lm}
\begin{proof}
Since we know already
 from the literature that $\mathfrak{G}^h$ is a simply connected Lie group with Lie algebra equal to $\mathfrak{L}^h$, see \cite[Theorem 7.30]{frizbook}, using the identification in \eqref{trunc_pol}, it is sufficient to check   that the following diagram
\begin{center}
\begin{tikzcd}
\mathfrak{L}^h \arrow{d}{\pi} \arrow{r}{\exp_h}
& \mathfrak{G}^h \arrow{d}{\pi} \\
\mathfrak{L}^{ h}/ \left(T^{>h}(\polyd )\cap\mathfrak{L}^h \right) \arrow{r}{\exp_{\bullet_h}}
& \mathfrak{G}^{ h}/ \left(T^{>h}(\polyd )\cap\mathfrak{G}^h \right)
\end{tikzcd}
\end{center}
commutes, where we denote by $\pi$ the projections on the quotient. This follows trivially from simple consideration regarding  the projection on $T^{>h}(\polyd )$.
\end{proof}

Thanks to these identifications, we can transfer the fundamental  properties of $ \mathfrak{G}^h$ and $\mathfrak{L}^h$ at the level of $\mathcal{G}^{\leq h}(\polyd)$ and $\mathcal{L}^{\leq h}(\polyd)$. The first one is the the Chow theorem, which expresses elements of $\mathcal G^{\leq h}(\polyd )$ as concatenation of simpler element. See \cite[Theorem 7.28]{frizbook} for the proof on $\mathfrak{G}^{\leq h}(\polyd)$.
\begin{thm}\label{thm:chow}
For any height $h\geq 1$ and $g \in \mathcal G^{\leq h}(\polyd ) $ there exists an integer $m$ and $\vv_1, \ldots , \vv_m \in  \polydh $ such that 
$$ g = \exp_{\bullet_h}(\vv_1) \bullet_h \cdots \bullet_h \exp_{\bullet_h}(\vv_m) \, . $$
\end{thm}



The second property we mention is also related to a representation of $\mathcal G^{\leq h}(\polyd )$ as an  affine variety. This result follows by applying the classical properties on  free nilpoltent Lie algebras and then the quotient. See \cite[Theorem 3.1, Theorem 3.2]{reutenauer1993free} and  \cite[Lemma 4.1, Lemma 4.2]{amendola2019varieties}.

\begin{thm}\label{shuffle_Lie}
The elements of $\mathcal G^{\leq h}(\polyd )$ and $\mathcal L^{\leq h}(\polyd)$ are characterized by the following relations
\begin{equation}\label{lie_shuffle}
\mathcal L^{\leq h}(\polyd)=\{\vv\in T^{\leq h}_0(\polyd) \colon \quad \langle \vv,u\shuffle k\rangle=0 \quad \text{for all} \;||u||_{\he} + ||k||_{\he}  \leq h \}\,.
\end{equation}
\begin{equation}\label{trunc_groups}
\mathcal G^{\leq h}(\polyd ) =\left\{ \vv \in T^{\leq h}_1(\polyd ) \colon \, \, 
\langle \vv, \uw{w }\shuffle \uw{u }\rangle = \langle \vv, \uw{w }\rangle \langle \vv,\uw{ u}\rangle   \, \, \text{for} \;||w||_{\he} + ||u||_{\he}  \leq h \right\}\,. 
\end{equation}
\end{thm}

%Remark that these equations are all polynomial (quadratic) equations on the entries of $\vv$. The equation \eqref{lie_shuffle} follows from  \cite[Theorem 3.1]{reutenauer1993free}.








We now turn our attention to the quasi-shuffle relations, and define an analogous space to the one presented in \cref{shuffle_Lie}, in the $\qshuffle$ context.

\begin{defin}\label{def_quasi-shuffle}
For any integer  $h\geq 1$ we define the \textbf{height-$h$ free quasi-shuffle Lie group} $\hat{\mathcal G}^{\leq h}(\polyd )$ and  \textbf{height-$h$ quasi-shuffle Lie polynomials} $\hat{\mathcal{L}}^{\leq h}(\polyd)$ as 
\[
\hat{\mathcal G}^{\leq h}(\polyd ) \coloneqq \left\{ \vv \in T^{\leq h}_1(\polyd ) \colon \, \, 
\langle \vv, \uw{w }\qshuffle \uw{u }\rangle = \langle \vv, \uw{w }\rangle \langle \vv,\uw{ u }\rangle   \, \, \text{for all} \;||w||_{\he} + || u||_{\he}  \leq h \right\}\,,\]
\[\hat{\mathcal{L}}^{\leq h}(\polyd)\coloneqq\log_{\bullet_h}(\hat{\mathcal G}^{\leq h}(\polyd ))\subset T^{\leq h}_0(\polyd)\,.\]
\end{defin}



Recall that $\Phi_H$ and $ \Psi_H$, defined in \eqref{Hoff_exp_log}, and their adjoints, are Hopf algebra isomorphisms. Summing up the results in \cite[Theorem 4.2]{hoffman2000} and \cite{Bellingeri2022} one has the following properties:

\begin{thm}\label{prop:phi_iso}
$\hat{\mathcal G}^{\leq h}(\polyd )$ with the operation $\bullet_h$ is a Lie group with $\hat{\mathcal{L}}^{\leq h}(\polyd)$ as  Lie algebra. Moreover, the function $\Phi^*_H$ maps isomorphically $\hat{\mathcal G}^{\leq h}(\polyd) $ to $\mathcal G^{\leq h}(\polyd )$ and $\hat{\mathcal L}^{\leq h}(\polyd) $ to $\mathcal L^{\leq h}(\polyd )$.
The maps $\Phi^*_H, \Psi_H^*$ commute with $\exp_{\bullet_h}, \log_{\bullet_h} $ on these domains.
\end{thm}

The importance of this last Lie group/ Lie algebra pair is important  because  the signature of a time-series is an element of $\hat{\mathcal G}^{\leq h}(\polyd)$, according to \cref{rem_trun_char}. Summing up the relation in a commutative diagram, we can describe the properties of $\Phi^*_H$ and $\Psi^*_H$ in Figure \ref{cd:level_h_lie} below.

%\vspace*{-0.5cm}

\begin{figure}

\begin{center}

\begin{tikzcd}
& & \hat{\mathcal{L}}^{\leq h}(\polyd) \arrow[rr, bend left=15, "\Phi^{ *}_H" description] \arrow[dd, bend left=30, "\exp_{\blt_h}" description] & & \mathcal{L}^{\leq h}(\polyd ) \arrow[ll, bend left=15, "\Psi^{*}_H" description] \arrow[dd, bend left=30, "\exp_{\bullet_h}" description] \\ \KK^{d\times N} \arrow[rrd, bend right=15, "\mathscr{S}^{\leq h}" description] & & & &\\
& & \hat{\mathcal G}^{\leq h}(\polyd )  \arrow[rr, bend left=15, "\Phi^{*}_H" description] \arrow[uu, bend left=30, "\log_{\bullet_h} " description] & & \mathcal G^{\leq h}(\polyd )  \arrow[ll, bend left=15, "\Psi^{*}_H" description] \arrow[uu, bend left=30, "\log_{\bullet_h}" description] 
\end{tikzcd}
\end{center}


\caption{The height $h$ free Lie algebras and the height $h$ Lie groups are connected via $\Phi^{*}$ and $\Psi^{*}$.\label{cd:level_h_lie}}
\end{figure}


From an algebraic geometry point of view, we can reinterpret $\hat{\mathcal G}^{\leq h}(\polyd )$ as an affine variety. Using the notation $\Dsign_{w} $ with $w\in \mathcal{W}(\MS_d)$ $||w||_{\he}\leq h$ to denote the coordinates of $T^{\leq h}(\polyd)$, we consider the following ideal of polynomial functions on $T^{\leq h}(\polyd)$
\begin{equation}\label{ideal_affine}
\hat{G}_{d,\leq h}\coloneq \langle \Dsign_{w}\Dsign_{v}- \Dsign_{w\qshuffle v}\;\colon \text{for all words  $w$ and $v$ s.t.} \;||w||_{\he} + || v||_{\he}  \leq h\rangle\,.
\end{equation}
The zero set of this ideal is exactly $\hat{\mathcal G}^{\leq h}(\polyd )$. Combining the Lie group structure with this affine representation we can indeed give a first description of this variety by computing its dimension. 


\begin{thm}\label{thm_dim_1}
The ideal $\hat{G}_{d,\leq h}$ is prime. Its irreducible variety coincide with $\hat{\mathcal G}^{\leq h}(\polyd )$ and it has dimension equal to $\Lambda_{d,l} $ .
\end{thm}

\begin{proof}
The affine  variety $\hat{\mathcal G}^{\leq h}(\polyd )$ is irreducible because it is the image of the linear space $\mathcal L^{\leq h}(\polyd ) $  under the polynomial map $ \exp_{\bullet_h}\circ\Psi_H^*$. This map has a polynomial inverse, namely $\log_{\bullet_h}\circ\Phi_H^* $. Hence the dimension of $\hat{\mathcal G}^{\leq h}(\polyd )$ agrees with that of $\mathcal L^{\leq h}(\polyd )$, which is given from Lemma \ref{dim_lie}.
\end{proof}







We conclude the section by defining an explicit adjoint of $\log_{\blt_h}$. 
These are the \textbf{truncated shuffle eulerian map} and \textbf{truncated quasi-shuffle eulerian map}, the maps $e_1^{\shuffle}$, $e_1^{\qshuffle}: T^{\leq h}(\polyd) \to T^{\leq h}(\polyd)$ defined as
\begin{align*}
e_1^{\shuffle} = \sum_{n\geq 1}\frac{(-1)^{n-1}}{n} \shuffle^{\circ (n-1)} \circ \tilde{\delta}^{\circ (n-1)} \, , \quad
e_1^{\qshuffle} = \sum_{n\geq 1}\frac{(-1)^{n-1}}{n} \qshuffle^{\circ (n-1)} \circ \tilde{\delta}^{\circ (n-1)}\, ,
\end{align*}
where we use the convention that $\qshuffle^{\circ (0)} \circ \tilde{\delta}^{\circ (0)}=\shuffle^{\circ (0)} \circ \tilde{\delta}^{\circ (0)}$ is the projection from $T^{\leq h}(\polyd) $ to $\varepsilon \KK$ and  the symbols $\qshuffle^{\circ (k)}$, $\shuffle^{\circ (k)}$ $\tilde{\delta}^{\circ (k)}$ describe  iterated products and reduced coproducts. For instance, one has
\begin{align*}
e_1^{\shuffle}(\uw{\varepsilon}) &= e_1^{\qshuffle}(\uw{\varepsilon}) = 0\,, \quad \quad 
e_1^{\shuffle}(\mathtt{i}) = e_1^{\qshuffle}(\mathtt{i}) = \mathtt{i} \, ,\\
e_1^{\shuffle}(\uw{\mathtt{i} \blt \mathtt{j}}) &= \frac{1}{2}(\uw{\mathtt{i} \blt \mathtt{j}}-\uw{\mathtt{j} \blt \mathtt{i}})\,, \quad e_1^{\qshuffle}(\uw{\mathtt{i} \blt \mathtt{j}}) = \frac{1}{2}(\uw{\mathtt{i} \blt \mathtt{j}} - \uw{\mathtt{j} \blt \mathtt{i}}) + \frac{1}{2}\uw{\mathtt{i} \mathtt{j}}\,.
\end{align*}
These two maps allow to compute the adjoint of $\log_{\bullet_h}$ on group-like elements.


\begin{prop}\label{lm:adjoint_log}
Let $\vv\in \mathcal G^{\leq h}(\polyd)$ and $\vw\in \hat{\mathcal G}^{\leq h}(\polyd)$. 
For any word $w$ with $||w||_{\he}\leq h$ one has:
\begin{equation}\label{log_eul}
\begin{split}
\langle \log_{\bullet_h} (\vv), w\rangle=\langle \vv, e_1^{\shuffle}(w)\rangle \, ,\quad 
\langle \log_{\bullet_h} (\vw), w\rangle=&\langle \vw, e_1^{\shuffle}(w)\rangle \, .
\end{split}
\end{equation}
\end{prop}

\begin{proof}
To prove the result, we use the following fact, called the duality between product and coproduct associated to $\blt $ and $\delta$
$$\langle (\vv - \varepsilon)^{\blt_h n}, \uw{w} \rangle = \langle \vv^{\otimes n}, \tilde{\delta}^{\circ ( n-1)} \uw{w} \rangle_{T(\polyd)^{\otimes n}} \, ,$$
This identity is a standard result in the Free lie algebra literature. Using the duality above, and applying \cref{shuffle_Lie} we have:
\begin{align*}
\langle\log_{\bullet_h}(\vv) , \uw{w} \rangle &= 
\sum_{n\geq 1} \frac{(-1)^{n-1}}{n} \langle (\vv - \uw{\varepsilon})^{\bullet_h n}, \uw{w} \rangle =
\sum_{n\geq 1} \frac{(-1)^{n-1}}{n} \langle \vv^{\otimes n}, \tilde{\delta}^{\circ (n-1)} \uw{w} \rangle_{T(\polyd)^{\otimes n}}\\
&=\sum_{n\geq 1} \frac{(-1)^{n-1}}{n} \langle \vv, \shuffle^{\circ (n-1)}\tilde{\delta}^{\circ (n-1)} \uw{w} \rangle \\
&=\langle \vv, \sum_{n\geq 1} \frac{(-1)^{n-1}}{n} \shuffle^{\circ (n-1)}\tilde{\delta}^{\circ (n-1)} \uw{w} \rangle = \langle \vv, e_1^{\shuffle} (\uw{w})\rangle \, ,
\end{align*}
which concludes one equality.
The remaning follows via the same computations. 
\end{proof}


%
%\begin{proof}
%If $\vv \in \hat{\mathcal G}(\polyd )$, consider $w, \tau $ words in $\mathcal W(\MS_d)$.
%Then 
%
%\begin{align*}
%\langle \Phi_H^*(\vv) , \uw{w} \shuffle \uw{\tau }\rangle &= \langle \vv , \Phi_H(\uw{w }\shuffle \uw{\tau}) \rangle \\
%&= \langle \vv , \Phi_H(\uw{w}) \qshuffle \Phi_H(\uw{\tau}) \rangle \\
%&= \langle \vv , \Phi_H(\uw{w}) \rangle \langle \vv , \Phi_H(\uw{\tau}) \rangle \\
%&= \langle \Phi_H^*(\vv) , \uw{w }\rangle \langle \Phi_H^*(\vv) , \uw{\tau }\rangle 
%\end{align*}
%showing that $\Phi_H^*(\vv) \in \mathcal G(\polyd)$.
%Because $\Phi^*_H$ is isomorphism, we get the desired result.
%An equivalent computation shows that $\Phi_H^*$ also maps group-like elements on $\shuffle$ to group-like elements on $\qshuffle$.
%\end{proof}





%We underscore a particular family of such Lie polynomials: given $i \in I$ and $w \in \mathcal W (I)$, we define recursively $\lv_{i} =i$ and $\lv_{i \bullet w } = [i, \lv_{w}]$. Using  the convention $\lv_{\varepsilon} = 0$ we defineand their quasi shuffle equivalent elements and the \textbf{Lie polynomials}
%Lie polynomials are characterized by the vanishing of all shuffle
%linear forms:
%\begin{align*}
% \\
%\mathcal \mathcal L^{\leq h}(\polyd) &= \spn \left\{\lv_{w} \Big|  w \in \mathcal W(I), \, \, ||w||_{\he} \leq h \right\}.
%\end{align*}
%It is a straightforward fact that any Lie polynomial can be obtained by a suitable linear combination of the elements in $\{\lv_{w}\}_{w \in \mathcal W(I)}$.

%The following projections $\pi^{\leq h}: T(V) \to T^{\leq h}(V)$ and $\pi^{\leq h}: T((V)) \to T^{\leq h}((V))$ are defined naturally.The projections of the products $\ot$ in $T(V)$ and $T((V))$ are well defined and give us algebra structures $\bullet_h$ on $T^{\leq h}(V)$ and $T^{\leq h}((V))$.The projection of the product $\shuffle$ is well defined and gives us an algebra structure $\shuffle_h$ on $T^{\leq h}(V)$. We will write $\pi^h$ for the projections $T(V) \to V^{= h}$ and $T((V)) \to V^{= h}$.


%\begin{align*}
%\hat{\mathcal G} (\polyd) &\coloneqq \left\{\vv \in T((V)) \Big| \langle \vv, \uw{w} \qshuffle \uw{\tau } \rangle  = \langle \vv, \uw{w }\rangle \langle \vv, \uw{\tau }\rangle \quad \forall w, \tau \in \mathcal W(I) \right\} \\
%\hat{\mathcal L} (\polyd) &\coloneqq \spn_{\infty} \left\{\Psi^*_H(\lv_{w})  \right\}_{w \in \mathcal W(I)}
%\end{align*}
%
%
%
%
%
%In this section, we will be discussing the exponential and logarithm maps between $\mathcal G(\polyd)$, $\hat{\mathcal G}(\polyd)$, $\mathcal G^{\leq h}(\polyd)$, $\hat{\mathcal G}^{\leq h}(\polyd)$ and $\mathcal L(\polyd)$, $\hat{\mathcal L}(\polyd)$, $\mathcal L^{\leq h}(\polyd)$, $\hat{\mathcal L}^{\leq h}(\polyd)$, respectively.



%\begin{align*}
%\exp_{\ot}:& T((\polyd))_0 \to \mathcal T((\polyd))_1 \text{ and }& \log_{\ot}:  T((\polyd))_1 \to T((\polyd))_0 \text{ defined as}\\
%\exp_{\ot}& ( \vv ) \coloneqq \sum_{n \geq 0 } \frac{1}{n!} \vv^{\bulletn} & \log_{\ot}( \vv + \uw{\varepsilon}) \coloneqq \sum_{n\geq 0} \frac{(-1)^{n-1}}{n} \vv^{ \bulletn}\\
%\end{align*}


%\begin{thm}[]\label{thm:lie_strucutre}
%The maps $\exp_{\ot}, \exp_{\bullet_h}, \log_{\ot}, \log_{\bullet_h}$ define the Lie structure maps for the following pairs of Lie groups and Lie algebras.
%These are Lie groups for the operations $\bullet$ and $\bullet_h$, and Lie algebras for the operations $[, ]$ and $[, ]_h$.
%\begin{align*}
%&\mathcal G(\polyd) \text{ and } \mathcal L(\polyd)\, , \\
%&\mathcal G^{\leq h}(\polyd) \text{ and } \mathcal L^{\leq h}(\polyd)\, , \\
%&\hat{\mathcal G}(\polyd) \text{ and } \hat{\mathcal L}(\polyd)\, , \\
%&\hat{\mathcal G}^{\leq h}(\polyd) \text{ and } \hat{\mathcal L}^{\leq h}(\polyd)\, .
%\end{align*}
%\end{thm}


%
%Once we descend to the Lie algebras and Lie groups of height $h$, everyting is finite dimensional.
%Our goal in \cref{conj:dim} is to nail down this specific dimension.
%We do this with the help of the following theorem. [Chow theorem ]
%
%
%\begin{align*}
%\mathcal G^{ > h }(V) &\coloneqq \mathcal G(V) \cap T^{>h}((V))\, \, \, \, \, \quad \quad \quad \quad \mathcal G^{\leq h}(V) \coloneqq \mathcal G(V) /_{\mathcal G^{>h}(V)} \\
%\mathcal L^{ > h }(V) &\coloneqq \mathcal L(V) \cap T^{>h}((V))\quad \quad \, \, \,  \, \,\quad \quad \mathcal L^{\leq h}(V) \coloneqq \mathcal L(V) /_{\mathcal L^{>h}(V)}
%\end{align*}






\section{Universal varieties and Chen-Chow Theorem for discrete signatures}\label{sec:conj}
We pass now to introduce the natural varieties associated to each  homogeneous component of the variety 
$\hat{\mathcal G}^{\leq h}(\polyd )$, by projecting on a specific height. In what follows,  we denote by  $\pi^h: T^{\leq h}(\polyd) \to T^h(\polyd)$ the canonical projection on this finite dimensional vector space.  We denote by $\Dsign_{w} $ with $w\in \mathcal{W}(\MS_d)$ $||w||_{\he}= h$  the coordinates of $T^{\leq h}(\polyd)$ and we set by $\KK[\Dsign^{(h)}]$ the ring of polynomial function on $ T^h(\polyd)$ over a generic field $\KK$ of characteristic $0$.
\begin{defin}
Fix $d, h$ integers $\geq 1$. We define the \textbf{quasi-shuffle variety}
 $\hat{\mathcal{G}}_{d,h}$ to be the zero set of the ideal $\hat{G}_{d,h}= \hat{G}_{d,\leq h}\cap\mathbb{C}[\Dsign^{(h)}]$  in the projective space $\mathbb{P}^{n_{d,h}-1}(\mathbb{C})$.
\end{defin}
This is trivially a homogeneous prime ideal in the polynomial ring over $T^h(\polyd)$. Moreover its  dimension can be explicitly computed.






\begin{thm}\label{thm:dimension}
For any $d, h$  the dimension of the variety $ \hat{\mathcal{G}}_{d, h}$ is $\Lambda_{d, h}-1$.
\end{thm}

\begin{proof}
We follow the strategy laid out in \cite[Theorem 6.1]{amendola2019varieties}.
Specifically, we show that at the level of affine varieties $ \pi^h$ is a generically $h$-to-$1$ map on $\hat{\mathcal G}^{\leq h}(\polyd)$, thereby obtaining that the dimension of the affine zero set of the ideal $  \hat{G}_{d, h}$ is given by the dimension of $\hat{\mathcal G}^{\leq h}(\polyd)$ and the result will follow from Theorem \ref{thm_dim_1}. To show that $\pi^h$ is generically $h$-to-$1$, let $\vv \in \hat{\mathcal{G}}_{d, h}$ such that $\langle \vv, \uw{\mathtt{1}}^{\qshuffle h}\rangle \neq 0$, and write
$$\vv = \sum_{\substack{w \in \mathcal W (\MS_d) \\ ||w ||_{\he} = h}} \alpha_{w} \uw{w}\, .$$
If $\vv = \pi^h(\vw)$ for $\vw \in \hat{\mathcal G}^{\leq h}(\polyd)$, then 
$$\langle \vw,  \uw{\mathtt{1}}\rangle^h = \langle \vw,  \uw{\mathtt{1}}^{\qshuffle h}\rangle = \langle \vv, \uw{\mathtt{1}}^{\qshuffle h}\rangle \, ,$$
because all elements in $\mathtt{1}^{\qshuffle h}$ have heigth $h$.
This equation determines a non-zero value for $\langle \vw,  \uw{\mathtt{1}}\rangle$ up to an $h$-root of unity of $1$.


Now for any $w $ of height $l < h$, note that
$$\langle \vv, \uw{ w} \qshuffle \uw{\mathtt{1}}^{\qshuffle h-l}\rangle  = \langle \vw,  \uw{w} \qshuffle \uw{\mathtt{1}}^{\qshuffle h-l}\rangle =
\langle \vw,  \uw{w} \rangle \langle \vw,\uw{\mathtt{1}}\rangle^{ h-l} \, .$$
This determines 
$$\langle \vw,  \uw{w }\rangle = \langle \vv,\uw{  w} \qshuffle \uw{\mathtt{1}}^{\qshuffle h-l}\rangle/_{\langle \vw,\uw{\mathtt{1}}\rangle^{ h-l}}$$
We conclude that for $\vv$ in an open set of $\hat{\mathcal{G}}_{d, h}$, there are $h$ many values of $\vw$ that map to $\vv$. 
\end{proof}

Coming back to the varieties discrete signatures and looking at the equivalent version of the quasi-shuffle varieties for signatures of paths, see \cite[Section 4]{amendola2019varieties}, it turns out that the family of quasi-shuffle varieties $\hat{\mathcal{G}}_{d, h}$ should coincide with the family  $\mathcal U_{d, h}$  of universal varieties but this equality seems much more harder to prove. The big reason behind this difficulty is the actual lack of a Chen-Chow theorem in the context  discrete signatures, relating the image of $x\to \Dsign^{\leq h}(x)$ with $\hat{\mathcal G}^{\leq h}(\polyd )$.


\begin{thm}[Chen-Chow theorem for discrete signatures]\label{thm_chen_chow}
For any $d$,$h$ integers $\geq 1$ and $ g\in \hat{\mathcal G}^{\leq h}(\polyd )$ there exists a time-series  $x\in \KK^{d\times N} $ for some $N\geq 1$ such that $\Dsign^{\leq h}(x)= g $.
\end{thm}
As it was already pointed out in \cite[Section 3]{Tapia20}, in the case $\KK= \mathbb{R}$ Theorem \ref{thm_chen_chow} does not hold because of trivial counterexamples.  However, if we are able to establish Theorem \ref{thm_chen_chow} in the complex field we should derive a fundamental equality of varieties.


%\raul{no period at the end, sentence does not seem to have been finished. Also explanation on the first sentence is not saying much (it's not true because we can find a counter-example), I would prefer to have no explanation at all}

\begin{prop}\label{conj:dim}
For any  $d, h$ integers $\geq 1$ one has $\V_{d, h}\subset \hat{\mathcal{G}}_{d, h}$. Moreover, if Theorem \ref{thm_chen_chow} holds true when $\KK= \mathbb{C}$ then  $\V_{d, h}=\hat{\mathcal{G}}_{d, h}$.
\end{prop}
\begin{proof}
To prove the first inclusion it is sufficient to show that $\V_{d, h, N}\subset \hat{\mathcal{G}}_{d, h}$
for any integer $N\geq 1$. However, by projecting at height $h$ the quasi-shuffle identity  in Theorem \ref{thm_prop_disc_sig} we immediately see that for any integer $N\geq 1$ one has 
\[\text{Im}\Dsign^h\subset \pi^h( \hat{\mathcal G}^{\leq h}(\polyd ))\]
from which we obtain the first inclusion. From Theorem \ref{thm_chen_chow} together with  Definition \ref{def_quasi-shuffle} we deduce immediately that there exists  an integer $\bar{N}\geq 1$ such that the ideal of the variety $\V_{d, h, \bar{N}}^{\text{af}}$ denoted by $I_{d, h, \bar{N}}$ satisfies 
\[I_{d, h, \bar{N}} \subset \hat{G}_{d,h}\,.\]
By passing to projective varieties, we obtain $\hat{\mathcal{G}}_{d,h}\subset \V_{d, h, \bar{N}}\subset \mathcal{V}_{d,h}$ and we conclude.
\end{proof}




In what follows, we will illustrate a possible strategy to prove Theorem \ref{thm_chen_chow}, through which we will prove the case $h=2$. Using  the results in section \ref{sec_Lie}  we can show  the Theorem \ref{thm_chen_chow} is equivalent to prove a simpler property on the finite dimensional vector space $\polyd^{\leq h} $.
%We show that for length at most two, the reachablility conditions are satistied.


%In the rest of this section we present some important consequences of this conjecture.Specifically, we present some dimension results and an enumeration result.





%The following definition and lemma outline a strategy for showing \cref{conj:dim}.
\begin{defin}\label{defin:reachability}
A given  $\vv \in \polyd^{\leq h}$ is said \textbf{reachable} if  there exists a time-series $x \in \KK^{d \times N}$ for some $N>1$ such that 
\begin{equation}\label{reac_eq_1}
\log_{\bullet_h}\circ\, \Phi^{*}_H \Dsign^{\leq h}(x) = \vv\,.
\end{equation}
\end{defin}

Due to the properties of applications $\Phi^{*}$ and $\log_{\bullet_h}$, the notion of reachability is equivalent to an explicit system of polynomial equations describing the image of $\Dsign^{\leq h}$.



\begin{lm}\label{lm:reachable}
Fix $d, h$ integers $\geq 1$. an element $\vv\in \polyd^{\leq h}$ is \textbf{reachable} if and only if there exists a time-series $x \in \KK^{d \times N}$ for some $N>1$ that satisfies the following system of equations 
\begin{equation}\label{eq_reach_eq}
\begin{cases}
\langle \Dsign (x), \uw{I}\rangle = \langle \vv, \uw{I}\rangle \\
\langle \Dsign (x), \Phi_H\circ  e_1^{\shuffle} \, (b(J))\rangle = 0
\end{cases}
\end{equation}
for all monomials $I \in \MS_d$ of height at most  $h$ and all Lyndon words  $J $ of height at most  $h$ such that $|J|\geq 2$. We call the equations in \eqref{eq_reach_eq} the \textbf{reachability equations}. 
\end{lm}
\begin{proof}
Looking at the diagram in Figure \ref{cd:level_h_lie}, we observe that $\log_{\bullet_h}\circ \,\Phi^{*}_H \Dsign^{\leq h}(x) = \vv$ if and only if $\langle \Phi^{*}_H \log_{\bullet_h}\Dsign(x), \bv\rangle =\langle \vv, \bv \rangle $ for all $\bv$ running over a basis of $\mathcal L^{\leq h}(\polyd)$. Thanks to Lemma \ref{dim_lie} one has
\[ \langle \Phi^{*}_H \log_{\bullet_h}\Dsign(x), b(K)\rangle =\langle \vv, b(K) \rangle\]  
for all Lyndon words  $K $ of height at most  $h$. By using \cref{lm:adjoint_log} the left-hand side in the equation above becomes
\begin{align*}
\langle \log_{\bullet_h}\circ \,\Phi^{*}_H \Dsign(x),b(K) \rangle =&
\langle  \Phi^{*}_H\Dsign(x),e_1^{\shuffle} (b(K)) \rangle\\
 =&
\langle  \Dsign(x),\Phi_H\circ  e_1^{\shuffle}(b(K)) \rangle\, .
\end{align*}
We obtain the desired equations once we note that whenever $|K | = 1$, $e_1^{\shuffle}K  =K$. Furthermore, the map $\Phi_H e_1^{\shuffle}$ is graded, so $\langle  \vv,\Phi_H\circ  e_1^{\shuffle}(b(K)) \rangle = 0 $ for $|K| > 1$.
\end{proof}
\begin{rem}
The choice of the basis given by lyndon words to derive the reachability equations  is completely arbitrary. Other possible basis for $\mathcal L^{\leq h}(\polyd)$ are studied in \cite[Chapter 4]{reutenauer1993free}.
%General set of equations We call the equations in \cref{defin:reachability} the \textbf{reachability} equations.These are split into \textbf{levels} according to the length of $w$.In this way, the reachability equations of level $k$ correspond to all equations where $w$ words have length $k$.For instance, the reachability equations of lenght $1$ are the non-homogeneous equations.Our strategy is to show that all $\vv $ of degree at most $h$ are reachable: the following lemma establishes that this is enough to infer \cref{conj:dim}.
\end{rem}

We know link the notion of reachability with \cref{thm_chen_chow}.




\begin{prop}\label{lm:strategy}
Fix $d, h$ integers $\geq 1$.  \cref{thm_chen_chow} holds with parameters $h$ and $d$ if and only if every element of $\vv \in \polyd^{\leq h}$ is reachable.
%Consider $\Dsign^{\leq h} $ as a map $\Dsign^{\leq h} :(\KK^d)^m \to \hat{\mathcal G}^{\leq h}(\polyd)$.Then, for some integer $m$, we have $\im \Dsign^{\leq h} = \hat{\mathcal G}^{\leq h}(\polyd)$.
\end{prop}

%If $a \in (\KK^d)^{N}, b\in (\KK^d)^{M}$ are two time-series, we denote its concatenation by $a|b$.Specifically, it denotes the time s-series in $\KK^d$ with $M+N$ vectors resulting from appending $b$ to $a$. For the proof of this lemma we use the following fact without proof:



\begin{proof}
Supposing \cref{thm_chen_chow} true we can fix $ \vv\in \polyd^{\leq h}$
and apply  \cref{thm_chen_chow} to the element $\exp_{\bullet_h}\circ \Psi_H^*(\vv)$, which by construction belongs to $\hat{\mathcal G}^{\leq h}(\polyd )$. By inverting the maps $\exp_{\bullet_h}$ and $\Psi_H^*$ we obtain \eqref{reac_eq_1}.

On the other hand, supposing that every element of $ \vv \in \polyd^{\leq h}$ is reachable we derive  \cref{thm_chen_chow}. We fix $g \in \hat{\mathcal G}^{\leq h}(\polyd)$ and consider $ \Phi_H^*(g)\in \mathcal G^{\leq h}(\polyd)$. By applying \cref{thm:chow}  we can find $\vv^1, \dots , \vv^m \in \polyd$ such that
$$\Phi_H^*(g) = \exp_{\bullet_h}(\vv^1) \bullet_h \cdots \bullet_h\exp_{\bullet_h}(\vv^m)\, . $$
Since every element $\vv^j $,  $j=1\,, \ldots\,, m$ is reachable there exists a time-series $x(\vv^j)$  such that $\log_{\bullet_h}\circ \Phi^{*}_H \Dsign^{\leq h}(x(\vv^j)) = \vv^j$. Plugging this relation in the equation above, we obtain
$$
\Phi^*_H(g) = \Phi^*_H\Dsign^{\leq h} (x(\vv^1)) \bullet_h\cdots  \bullet_h \Phi^*_H\Dsign^{\leq h} (x(\vv^m))\,. $$
Using the homomorphism property of $\Phi^*_H$ and \eqref{trunc_chen} one has 
$$
\Phi^*_H(g) = \Phi^*_H(\Dsign^{\leq h} (x(\vv^1)|\cdots  | (x(\vv^m)))\,. $$
Because $\Phi^{*}_H$ is an isomorphism, we conclude.
\end{proof}
We display the equations necessary to solve the reachability problem for height two. Writing every element of $\vv\in \polyd^{\leq 2}$ as a sum
\[\vv= \sum_{\mathtt{i}} \vv^{\mathtt{i}}\mathtt{i} + \sum_{\mathtt{i}\mathtt{j}}\vv^{\mathtt{i}\mathtt{j}} \mathtt{i}\mathtt{j}\,,\]
and using the properties of $e_1^{\shuffle}$, the equations \eqref{eq_reach_eq} become
\begin{equation}\label{level_2eq}
\begin{cases}
\langle \Dsign (x), \mathtt{i}\rangle = \vv^{\mathtt{i}} & \mathtt{i}=\mathtt{1}\,, \ldots\,, \mathtt{d} \\
\langle \Dsign (x),\mathtt{ij}\rangle = \vv^{\mathtt{ij}} & \mathtt{i}, \mathtt{j}\in [d]\,,\; \;\mathtt{i}\leq \mathtt{j} \\
\langle \Dsign (x), \mathtt{i}\blt \mathtt{j}-\mathtt{j}\blt \mathtt{i}\rangle = 0 & \mathtt{i}, \mathtt{j}\in [d]\,,\;\;\mathtt{i}< \mathtt{j}
\end{cases}
\end{equation}
%Consider the timeseries $x = x(\vv^1) | \cdots | x(\vv^m) $ in.
%Then \eqref{eq_chen_equation}, together with the fact that $\Phi^*_H$ is an algebra homomorphism (see \cref{eq:phistar}) gives us:
%\[\Phi^*_H\Dsign^{\leq h} (x) = \Phi^*_H\Dsign^{\leq h} (x(\vv^1)) \bullet_h\cdots  \bullet_h \Phi^*_H\Dsign^{\leq h} (x(\vv^m)) = \vw\,.\]
%This shows that the map $\Phi^{*}_H \circ \Dsign^{\leq h}$ is surjective.

\begin{thm}\label{lm:h2constructionX}
Assume $\KK= \mathbb{C}$. Then every element of $\vv \in \polyd^{\leq h}$ is reachable when $h=2$ and $d$ is generic. 
\end{thm}



\begin{proof}
%By invariance of translation it is sufficient to solve the equations
%\[
%\begin{cases}
%\langle \Dsign (x), \mathtt{i}\rangle = 0 & \mathtt{i}=\mathtt{1}\,, \ldots\,, \mathtt{d} \\
%\langle \Dsign (x),\mathtt{ij}\rangle = \vv^{\mathtt{ij}} & \mathtt{i}, \mathtt{j}\in [d]\,,\; \;\mathtt{i}\leq \mathtt{j} \\
%\langle \Dsign (x), \mathtt{i}\blt \mathtt{j}-\mathtt{j}\blt \mathtt{i}\rangle = 0 & \mathtt{i}, \mathtt{j}\in [d]\,,\;\;\mathtt{i}< \mathtt{j}
%\end{cases}
%\]
We argue as follows: First we construct a time-series $x = x(\vv)$ that satisfies the first two lines of \eqref{level_2eq}. From this solutions we construct another  solutions to keep also the last equation, which is amenable to algebraic manipulations on the level one.

Let us find the time-series $x(\vv)$ solving the first two lines of \eqref{level_2eq} by a dimension argument.  Let $\U$ be a basis of $\polyd^{\leq 2}$. 
Let $Y$ be the variety in $\mathbb{C}^{|\U|}$ given by the image of the following map from $\mathbb{C}^d$
$$ z \mapsto (p(z))_{p \in \U} \, . $$
This is an irreducible variety, as observed in the proof of \cref{thm:chain}.

Say it has dimension $e$. Fix $N = \lceil |\U|/{e} \rceil$. The linear relations $\langle \Dsign (x), \mathtt{i}\rangle = x_N^i- x_1^i=\vv^{\mathtt{i}}$ for $\mathtt{i}=\mathtt{1}\,, \ldots\,, \mathtt{d}   $ trace out an affine space $\LL$ of codimention $|\U|$ in $\KK^N$.
Given that the $N$ cartesian product $Y^N$ has dimension $eN \geq |\U|$, and since $Y^N$ is not contained in any hyperplane, the intersection $\LL \cap Y^N$ is non-empty.
The time  series $x$ in this intersection space is the desired $x(\vv)$.

To establish the equations on level two, for a time  series $x$ let $
\overleftarrow{x}$ be the reversed-time  time-series of $x$.
Then, we show that for any time-series $x$, the time-series $ g(x) = x |\overleftarrow{x} $  satisfies the equations of height two, while having an amenable level one result, that is:
\[
\begin{cases}
\langle \Dsign (g(x)), \mathtt{i}\rangle = 2\vv^{\mathtt{i}} & \mathtt{i}=\mathtt{1}\,, \ldots\,, \mathtt{d} \\
\langle \Dsign (g(x)),\mathtt{ij}\rangle =2^2 \vv^{\mathtt{ij}} & \mathtt{i}, \mathtt{j}\in [d]\,,\; \;\mathtt{i}\leq \mathtt{j} \\
\langle \Dsign (g(x)), \mathtt{i}\blt \mathtt{j}-\mathtt{j}\blt \mathtt{i}\rangle = 0 & \mathtt{i}, \mathtt{j}\in [d]\,,\;\;\mathtt{i}< \mathtt{j}
\end{cases}
\]
It follows that $g( 2^{-1}x(\vv) )$ satisfies the desired equations, and the theorem is proven.
\end{proof}
%\subsection{Length three equations\label{sec:len_three_eqs}}
\begin{rem}
We note that the dimension bound argument presented in the proof above works for any fixed height, and displays the intuition that given a large enough time-series size, a simple dimension argument allows us to construct a suitable solution.
The construction of a solution satisfying all equations in \eqref{level_2eq} does not follow from a dimension argument in the general case, and so we require a suitable construction.
\end{rem}
\begin{rem}

%First we construct a time-series $x = x(\vv)$ that satisfies the equations of length one.That is, for any element $\vv \in \polyd$ of degree at most $h$, the time-series satisfies $$\langle \Dsign(x), I \rangle = \langle \vv, I\rangle  \text{ for all $I$ monomial.}$$ This concludes the level one.From these solutions we construct solutions to the level two reachability equations, which are amenable to algebraic manipulations on the level one. This allows us to construct the desired solution.


It can be checked computationally that $e_1^{\shuffle}(K)=K$ for all Lyndon words  $K $ of height at most  $3$. 
From this fact, we would derive a set of reachablility equations that looks more tractable, even if one needs to develop more sophisticated tools to have accessible algebraic manipulations. 
However,  for level four and above one has $e_1^{\shuffle}(K)\neq K$ and even writing down a general formula for the reachability equation seems to be a challenging combinatorial problem.
%We remark that the map $e_1^{\shuffle}$ is a projection in $\mathcal L (\polyd)$ for any element of $\mathcal L^{\leq h}(\polyd)$ of length one, two and three.
%
%However, it is not true that $e_1^{\shuffle}$ is the identity in $\mathcal L(\polyd)$, and counter examples arise for level four and above.
\end{rem}
%
%We conclude the section by displaying the equations necessary to solve the reachability problem for height three. As someone can check by direct computations one has  that $e_1^{\shuffle}(K)=K$ is a projection in $\mathcal L (\polyd)$ for any element of $\mathcal L^{\leq h}(\polyd)$ of length one, two and three.
%This is why the equations of level one and two, as well as the equations of level three, as we will see below, are tractable.
%We display the equations necessary to solve the reachability problem for length three.
%Here we present \cref{defin:reachability} for $w = \mathtt{1} \blt\mathtt{2} \blt\mathtt{3}$ and $w = \mathtt{12}\blt \mathtt{3} \blt \mathtt{4}$, which is a generic element of length three.
%That is, we compute $\Phi^*_H e_1^{\shuffle} (\lv_w)$.
%From the same token as in \cref{lm:h2constructionX}, finding a time-series that has
%$$\langle \Dsign (x) ,  \Phi^*_H e_1^{\shuffle} (\lv_w) \rangle = 0 \, , $$
%implies \cref{conj:dim} for $h = 3$.

%From the remark above we have that $e_1^{\shuffle} (\lv_w) = \lv_w$.
%Therefore, by using the examples given after \cref{eq:phistar} we get:
%\begin{align*}
%&\text{For } w = \mathtt{1} \blt\mathtt{2} \blt\mathtt{3}\\
%\Phi^*_H e_1^{\shuffle} (\lv_w) =& \Phi^*_H (\lv_w) \\
%=& \Phi^*_H (\mathtt{1} \blt\mathtt{2} \blt\mathtt{3} - \mathtt{1} \blt\mathtt{3} \blt\mathtt{2} - \mathtt{2} \blt\mathtt{3} \blt\mathtt{1} + \mathtt{3} \blt\mathtt{2} \blt\mathtt{1} )\\
%=& \mathtt{1} \blt\mathtt{2} \blt\mathtt{3} - \mathtt{1} \blt\mathtt{3} \blt\mathtt{2} - \mathtt{2} \blt\mathtt{3} \blt\mathtt{1} + \mathtt{3} \blt\mathtt{2} \blt\mathtt{1} = \lv_w\, . \\
%&\text{For } w = \mathtt{12}\blt \mathtt{3} \blt \mathtt{4}\\
%\Phi^*_H e_1^{\shuffle} ( \lv_w) =& \Phi^*_H (\lv_w) \\
%=& \Phi^*_H (\mathtt{12} \blt\mathtt{3} \blt\mathtt{4} - \mathtt{12} \blt\mathtt{4} \blt\mathtt{3} - \mathtt{3} \blt\mathtt{4} \blt\mathtt{12} + \mathtt{4} \blt\mathtt{3} \blt\mathtt{12} )\\
%=& (12 + \frac{1}{2}\, \mathtt{1}\blt \mathtt{2}+ \frac{1}{2}\, \mathtt{2}\blt \mathtt{1})\blt\mathtt{3} \blt\mathtt{4}  - (12 + \frac{1}{2}\, \mathtt{1}\blt \mathtt{2}+ \frac{1}{2}\, \mathtt{2}\blt \mathtt{1})\blt\mathtt{4} \blt\mathtt{3}\\
%& \quad -\blt\mathtt{3} \blt\mathtt{4} \blt (12 + \frac{1}{2}\, \mathtt{1}\blt \mathtt{2}+ \frac{1}{2}\, \mathtt{2}\blt \mathtt{1}) + \blt\mathtt{4} \blt\mathtt{3} \blt (12 + \frac{1}{2}\, \mathtt{1}\blt \mathtt{2}+ \frac{1}{2}\, \mathtt{2}\blt \mathtt{1})\\
%=& \lv_w + \frac{1}{2} \left( \lv_{ \mathtt{12} \blt  \mathtt{3} \blt  \mathtt{4}} +\lv_{ \mathtt{12} \blt  \mathtt{3} \blt  \mathtt{4}}  \right) \, .
%\end{align*}
%A general equation was, however, not found.

\subsection*{Aknowledgments}
The first author is supported in part by the DFG Research Unit FOR2402. The  second author is supported by the Max Planck institute for the mathematics in the sciences. Both authors would like to thank fruitful conversations with Sylvie Paycha and Bernd Sturmfels. We would also like to thank  \'Angel R\'ios Ortiz, Pierpaola Santarsiero and Nikolas Tapia for all the suggestions and comments on discrete signatures, tensor algebras and algebraic varieties.

%
%
%
%\begin{thm}\label{thm:enum}
%The number of Lyndon words in $\mathcal W (\MS_d)$ of heigth $h$ is
%$$ \lambda_{d, h} = \sum_{k|h} \frac{k}{h} \mu\left(\frac{h}{k} \right) \sum_{\alpha\in C(k)} \frac{1}{\ell (\alpha)} \prod_i \binom{\alpha_i + d - 1}{d - 1} \, . $$
%\end{thm}
%

%\begin{proof}
%First we note that the height grading on $T(V)$ gives us the following power series
%$$H(x) = \sum_{w \in \mathcal W(\MS_d)} x^{\he (w) } = \frac{1}{1 - \sum_{I \in \MS_d} x^{\text{grad}(I)}} \, .$$
%Furthermore, for $h \geq 1 $ there are $\binom{h+d-1}{d-1}$ multisets on $\{1, \dots, d\}$ size $h$, so 
%$$ \sum_{I \in \MS_d} x^{\text{grad}(I)} = \sum_{k\geq 1} \binom{k + d -1 }{d-1} x^ k = (1 - x)^{-d} - 1\, ,$$
%so we have $H(X) = [1 - ( (1-x)^{-d} - 1)]^{-1}$.
%
%On the other hand, the \textbf{Lyndon unique factorization theorem} (see \cite{chen1958free}) guarantees that each word $w \in \mathcal W (\MS_d)$ can be written uniquely as 
%$$ w = \tau_1 \blt \cdots \blt \tau_j \, , $$
%where $\tau_i $ are Lyndon words with $\tau_1 \geq_{lex} \dots \geq_{lex} \tau_k$. 
%Therefore
%\begin{align*}
%H(x) &= \sum_{w \in \mathcal W(\MS_d)} x^{\he (w) } = \prod_{\substack{\tau \text{ Lyndon word} \\ \tau  \in \mathcal W(\MS_d )}}\left( 1 + x^{\he (\tau)} + x^{2\he (\tau)} + x^{3\he (\tau)} + \cdots  \right)  \\
%&= \prod_{\substack{\tau \text{ Lyndon word} \\ \tau  \in \mathcal W(\MS_d }} ( 1 - x^{\he(\tau)} )^{-1} = \prod_{k\geq 1} ( 1 - x^{k} )^{-\lambda_{d, k}},
%\end{align*}
%where we recall that $\lambda_{d, k} $ is the number of Lyndon words $\tau \in \mathcal W(\MS_d)$ of length $k$.
%
%Putting it together, applying $\log $ on both sides and using 
%
%\[-\log(1-f(x) ) = \sum_{n\geq 1} \frac{1}{n}f(x)^ n\,,\] 
%we get 
%\begin{align*}
%-\log(H(x) ) &= \sum_{k\geq 1} - \lambda_{d, k} \log ( 1 - x^{k} )  = \sum_{j, k\geq 1}\frac{1}{j} x^{j k} \lambda_{d, k}  = \\
%&= \sum_{n\geq 1} x^n \sum_{k | n} \frac{1}{n/k} \lambda_{d, k}  = \sum_{n\geq 1} \frac{1}{n}x^n \sum_{k | n} k \lambda_{d, k} \\
%\\
%-\log(H(x) ) &= -\log [ 1 - ( (1-x)^{-d} - 1 ) ]  = \\
%&= \sum_{k\geq 1} \frac{1}{k}\left(\sum_{t\geq 1} \binom{t + d -1 }{d-1} x^t \right)^k\\
%&= \sum_{n\geq 0} x^n\sum_{\alpha \models n} \frac{1}{\ell(\alpha)}\prod_i \binom{\alpha_i + d - 1}{d-1}
%\end{align*}
%
%Equating both sides it follows that for each $n$ we have
%$$ \sum_{k | n} k \lambda_{d, k} =   n \sum_{\alpha \models n} \frac{1}{\ell(\alpha)}\prod_i \binom{\alpha_i + d - 1}{d-1} \, .$$
%
%Summing both sides for all $n$ divisors of $h$ and multiplying by $\mu\left(\frac{h}{n}\right)$ we get throught M\"obius inversion that
%$$h \lambda_{d, h} = \sum_{n|h}  n\, \, \mu\left( \frac{h}{n} \right) \sum_{\alpha \models n} \frac{1}{\ell(\alpha)}\prod_i \binom{\alpha_i + d - 1}{d-1}\, ,$$
%from which the theorem follows.
%\end{proof}
%
%This allows us to create the following values of $\lambda_{d, h} $. %Code created to generate this table can be found in \cite{M2_MATHREPO}.
%
%\begin{center}
%\begin{tabular}{|l|| c | c | c |c |c |c|c | c | c |}
%\hline
%$h$ & 1 & 2 & 3 & 4 & 5 & 6 & 7 & 8 &9 \\
%\hline
%$d = 1$&  1& 1& 2& 3& 6& 9& 18& 30& 56\\
%$d = 2$&  2& 4& 12& 31& 92& 256& 772& 2291& 7000 \\
%$d = 3$&  3 & 9& 36& 132& 534& 2140& 8982& 38031& 164150 \\
%$d = 4$&  4& 16& 80& 380& 1960& 10228& 55352& 304223& 1700712 \\
%$d = 5$&  5& 25& 150& 875& 5500& 35335& 234530& 1584845& 10885640 \\
%$d = 6$&  6& 36& 252& 1743& 12936& 98686& 776412& 6226008& 50732712  \\
%$d = 7$&  7& 49& 392& 3136& 26852& 237160& 2158156& 20028764& 188856934   \\
%\hline
%\end{tabular}
%\end{center}
%



\bibliographystyle{alpha}
\bibliography{bibli}



\end{document}
