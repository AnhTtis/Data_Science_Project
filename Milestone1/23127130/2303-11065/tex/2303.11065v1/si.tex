
\section{Supporting Information} \label{sec:sup_methods}
\setcounter{figure}{0}  
\setcounter{table}{0}  


\subsection*{Supporting Methods}


\begin{table}
\footnotesize
\centering
\begin{tabular}{c|c|c|c|c|c}
\hline
             $\lambda$   &  $\lambda_1$  & $\lambda_2$  & $\alpha$  &  $u_0$  &  $w_0$\\

\hline
 0.00  & 0.00 & 0.00 & 0.10 & 110 & 0 \\
 0.05  & 0.00 & 0.10 & 0.10 & 110 & 0 \\
 0.10  & 0.00 & 0.20 & 0.10 & 110 & 0 \\
 0.15  & 0.00 & 0.30 & 0.10 & 110 & 0 \\
 0.20  & 0.00 & 0.40 & 0.10 & 110 & 0 \\
 0.25  & 0.00 & 0.50 & 0.10 & 110 & 0 \\
 0.30  & 0.10 & 0.50 & 0.10 & 110 & 0 \\
 0.35  & 0.20 & 0.50 & 0.10 & 110 & 0 \\
 0.40  & 0.30 & 0.50 & 0.10 & 110 & 0 \\
 0.45  & 0.40 & 0.50 & 0.10 & 110 & 0 \\
 0.50  & 0.50 & 0.50 & 0.10 & 110 & 0 \\


\end{tabular}
\caption{\label{tab:params_table}Alchemical schedule of the Solftplus Alchemical Potential for the two legs for the alchemical transformations. $\alpha$ values are in (kcal/mol)$^{-1}$ and $u_0$ and $w_0$ are in kcal/mol }
\end{table}

\begin{figure}
\centering
\includegraphics[width=0.7\textwidth]{SI/compare_prots_all_1.pdf}
\caption{\label{fig:comparison_corrsSI_1} Scatterplots for the calculated $\Delta\Delta$G estimated against the experimental measurements and compared to other methodologies for MCL-1, TYK2, PTP1B and BACE systems. The first column represents calculations performed with ATM. The other columns contain data from benchmark studies of FEP+\cite{wang2015accurate}, Amber\cite{lee2020alchemical} and pmx\cite{gapsys2020large}. }
\end{figure}

\begin{figure}
\centering
\includegraphics[width=0.7\textwidth]{SI/compare_prots_all_2.pdf}
\caption{\label{fig:comparison_corrsSI_2} Scatterplots for the calculated $\Delta\Delta$G estimated against the experimental measurements and compared to other methodologies for JNK1, CDK2, Thrombin and p38 systems. The first column represents calculations performed with ATM. The other columns contain data from benchmark studies of FEP+\cite{wang2015accurate}, Amber\cite{lee2020alchemical} and pmx\cite{gapsys2020large}. }
\end{figure}

\begin{figure}
\centering
\includegraphics[width=0.7\textwidth]{SI/Ref_ligands_all.pdf}
\caption{\label{fig:ref_alignment} Reference atoms selected for the applied restraining potential in every system. selected atoms of each molecule define a cartesian coordinate system with the orange atom at the origin, a \textit{z} axis along the orange to green direction, and the purple atom oriented at the \textit{xz} plane}
\end{figure}

\begin{figure}
\centering
\includegraphics[width=\linewidth]{Figures/RMSE_comparison_new.pdf}
\caption{\label{fig:comparison_RMSE} Root Mean Square Error (RMSE) for each protein-ligand system calculated with ATM and reported through other methodologies such as FEP+\cite{wang2015accurate}, Amber\cite{lee2020alchemical} and pmx\cite{gapsys2015pmx}.}
\end{figure}

\begin{figure}
\centering
\includegraphics[width=\linewidth]{SI/MCL1_Rplots.pdf}
\caption{\label{fig:perturbation_MCL1} Distribution of the perturbation energies for a series of ligand pairs coresponding to the MCL-1 system. Each line corresponds to a $\lambda$. Top row refers to the first leg an bottom row to the second one. }
\end{figure}

\begin{figure}
\centering
\includegraphics[width=\linewidth]{SI/TYK2_Rplots.pdf}
\caption{\label{fig:perturbation_TYK2} Distribution of the perturbation energies for a series of ligand pairs coresponding to the TYK2 system. Each line corresponds to a $\lambda$. Top row refers to the first leg an bottom row to the second one. }
\end{figure}

\begin{figure}
\centering
\includegraphics[width=\linewidth]{SI/JNK1_Rplots.pdf}
\caption{\label{fig:perturbation_JNK1} Distribution of the perturbation energies for a series of ligand pairs coresponding to the JNK1 system. Each line corresponds to a $\lambda$. Top row refers to the first leg an bottom row to the second one.}
\end{figure}

\begin{figure}
\centering
\includegraphics[width=\linewidth]{SI/compare_correlations_methods.pdf}
\caption{\label{fig:correlation_methods} Pearson, spearman and kendall correlations for all data points of ATM against experimental data and the other compared methodologies: FEP+\cite{wang2015accurate}, Amber\cite{lee2020alchemical} and pmx\cite{gapsys2020large}}
\end{figure}

\begin{figure}
\centering
\includegraphics[width=\linewidth]{SI/compare_correlations_methos_proteins_1.pdf}
\caption{\label{fig:correlation_methods_prot_1} Pearson, spearman and kendall correlations for the MCL1, TYK2, JNK1 and PTP1B systems of ATM against experimental data and the other compared methodologies: FEP+\cite{wang2015accurate}, Amber\cite{lee2020alchemical} and pmx\cite{gapsys2020large}}
\end{figure}

\begin{figure}
\centering
\includegraphics[width=\linewidth]{SI/compare_correlations_methos_proteins_1.pdf}
\caption{\label{fig:correlation_methods_prot_2} Pearson, spearman and kendall correlations for the BACE, p38, CDK2 and Throbmin systems of ATM against experimental data and the other compared methodologies: FEP+\cite{wang2015accurate}, Amber\cite{lee2020alchemical} and pmx\cite{gapsys2020large}}
\end{figure}

\begin{figure}
\centering
\includegraphics[width=\linewidth]{SI/Problematic_ligands_examples.pdf}
\caption{\label{fig:problematic_ligands} Structures of some of the ligands that showed poor correlation with experimental values in two or more instances}
\end{figure}

\begin{figure}
\centering
\includegraphics[width=\linewidth]{SI/Converge_longer.pdf}
\caption{\label{fig:Converge_longer}  Free energy convergence as a function of time for a series of ligand pairs of CDK2, MCL-1, PTP1B and JNK1 with a greater sampling than 50ns. The red line corresponds to the experimental $\Delta \Delta G$ value. }
\end{figure}