\documentclass[journal=jctcce,manuscript=article,layout=twocolumn]{achemso}


% Language setting
% Replace `english' with e.g. `spanish' to change the document language
\usepackage[english]{babel}

% Set page size and margins
% Replace `letterpaper' with `a4paper' for UK/EU standard size
%\usepackage[letterpaper,top=2cm,bottom=2cm,left=3cm,right=3cm,marginparwidth=1.75cm]{geometry}

% Useful packages
\usepackage{amsmath}
\usepackage{graphicx}
\usepackage{natbib}

\usepackage[colorlinks=true, allcolors=blue]{hyperref}




\title{Validation of the Alchemical Transfer Method for the Estimation of Relative Binding Affinities of Molecular Series}
\author{Francesc Saban\'es Zariquiey}
\affiliation{Computational Science Laboratory, Universitat Pompeu Fabra, Barcelona Biomedical Research Park (PRBB), C Dr. Aiguader 88, 08003, Barcelona, Spain}
\author{Adri\`a P\'erez }
\affiliation{Acellera Labs, C Dr Trueta 183, 08005, Barcelona, Spain}
\author{Maciej Majewski}
\affiliation{Acellera Labs, C Dr Trueta 183, 08005, Barcelona, Spain}
\author{Emilio Gallicchio}
\email{egallicchio@brooklyn.cuny.edu}
\affiliation[NYC]{Department of Chemistry, Brooklyn College of the City University of New York;  PhD Program in Chemistry, Graduate Center of the City University of New York; PhD Program in Biochemistry, Graduate Center of the City University of New York, USA}
\author{Gianni De Fabritiis}
\email{gianni.defabritiis@upf.edu}
\affiliation{Computational Science Laboratory, Universitat Pompeu Fabra, Barcelona Biomedical Research Park (PRBB), C Dr. Aiguader 88, 08003, Barcelona, Spain}
\alsoaffiliation{Acellera, Devonshire House 582 Honeypot Lane Stanmore Middlesex,
HA7 1JS United Kingdom}
\alsoaffiliation{Instituci\'o Catalana de Recerca i Estudis Avan\c{c}ats (ICREA), Passeig Lluis Companys 23, 08010 Barcelona, Spain}


\begin{document}
\maketitle

\begin{abstract}
The accurate prediction of protein-ligand binding affinities is crucial for drug discovery. Alchemical free energy calculations have become a popular tool for this purpose. However, the accuracy and reliability of these methods can vary depending on the methodology. In this study, we evaluate the performance of a relative binding free energy protocol based on the alchemical transfer method (ATM), a novel approach based on a coordinate transformation that swaps the positions of two ligands. The results show that ATM matches the performance of more complex free energy perturbation (FEP) methods in terms of Pearson correlation, but with marginally higher mean absolute errors. This study shows that the ATM method is competitive compared to more traditional methods in speed and accuracy and offers the advantage of being applicable with any potential energy function.
\end{abstract}

\section{Introduction}

The ability to accurately predict the binding free energy of a ligand to a protein can provide crucial information for drug discovery, as it allows for the identification of compounds that have a higher likelihood of binding to a target. Alchemical free energy calculations have become the leading tools in this field. \cite{jorgensen2004many,abel2017advancing,armacost2020novel}  Free energy approaches are especially relevant in hit-to-lead and lead optimization stages of drug design while dealing with a series of similar ligands. Both commercial and free tools for free energy calculations have been developed over the past few years, with extensive use in both academia and the pharmaceutical industry.\cite{zou2019blinded,gapsys2015pmx,bieniek2021ties,wang2015accurate,kuhn2020assessment} 

One of the most common approaches to alchemical calculations is Free Energy Perturbation (FEP). This method involves many distinct equilibrium MD simulations for all states along a $\lambda$ coordinate that modifies a ligand A into a ligand B alchemically. Commonly these simulations are split between 12 or more $\lambda$-intermediates where the two ligands are interchanged.\cite{mey_best_2020} One of the most common methodologies that use this approach is Schrödinger's FEP+.\cite{wang2015accurate} Another way to approach alchemical calculations is via Thermodynamic Integration (TI), as in the Amber implementation.\cite{lee2020alchemical} The main difference with FEP is that TI calculates the free energy difference by integrating the derivative of the Hamiltonian with respect to the alchemical progress parameter $\lambda$. The pmx\cite{gapsys2015pmx} protocol implements a similar strategy based on non-equilibrium trajectories. Although different, FEP and TI share a few common traits such as the adoption of a double-decoupling process that obtains the relative binding free energy from the difference of the alchemical free energies from separate solution and receptor legs, and the requirement of softcore potentials to avoid clashes and instabilities\cite{beutler1994avoiding,zacharias1994separation,gapsys2012new}. Custom alchemical topologies and the need for multiple simulations of distinct systems (receptor complex and ligand in solvent) tend to require more user expertise. Furthermore, they are usually not suited for ligand pairs with different net charges\cite{chen2018accurate}, leading to potential issues with the treatment of long-range electrostatic interactions and artifacts in the free energy estimates, unless complex correction factors are introduced.\cite{rocklin2013calculating}

Recently, a novel approach to performing alchemical calculations has been proposed. The Alchemical Transfer Method (ATM) is a protocol for the estimation of relative binding free energies based on a coordinate transformation that swaps the positions of two ligands. The method performs the calculation in a single solvent box and, unlike double-decoupling free energy perturbation approaches,\cite{Gilson:Given:Bush:McCammon:97,wang2015accurate,lee2020alchemical} avoids the split of the binding free energy calculation into receptor and solvation legs. Furthermore, ATM does not require the implementation of softcore pair potentials. ATM is implemented in the free and open-source OpenMM\cite{eastman2017openmm} molecular simulation package, allowing a simple and easy route to large-scale automated deployments and flexibility to employ  any potential energy function. 
In spirit, ATM is similar to the separated topologies method \cite{rocklin2013separated} with the difference that the latter achieves the transfer by decoupling the first ligand while coupling the second by modifying the force field parameters. Whereas in ATM, the perturbation is implemented as a coordinate displacement that swaps the position of the two ligands. Recently, this approach was reintroduced to be used in GROMACS.\cite{baumann2023broadening}
ATM can handle both relative (ATM-RBFE) and absolute (ATM-ABFE) binding free energy calculations. In this work we focus on testing the accuracy and feasibility of the RBFE approach.

All the different free energy estimation methods have their pros and cons and can vary in their accuracy and reliability. Therefore it is important to rigorously evaluate their performance against large and diverse benchmark datasets.

In this work, we aim to evaluate the performance of ATM,\cite{wu2021alchemical,azimi2022relative} using the dataset of Wang \textit{et al.},\cite{wang2015accurate} one of the most popular benchmarks for evaluating relative binding free energy protocols. We use ATM to calculate the difference in binding free energies for 330 ligand pairs across 8 different protein systems. We also compared our results with state-of-the art methodologies such as FEP+\cite{wang2015accurate}, Amber\cite{lee2020alchemical} and pmx\cite{gapsys2020large}. We show that ATM, a methodology that requires less expertise and preparation than alternative protocols, performs as well as other existing tools and even better from a correlation point of view. 


\section{Methods}

The aim of this study is to further expand the benchmarking of ATM to a series of targets tested in other similar methodologies and evaluate whether it can provide accurate and reliable estimates of relative binding free energies for these systems. To address this question, we conducted a computational study in which we applied ATM to the dataset of Wang \textit{et al.}.\cite{wang2015accurate} This benchmark includes eight targets relevant to pharmaceutical research (MCL-1, TYK2, MCL-1, JNK1, PTP1B, BACE, Thrombin and p38) with a total of 330 ligand pairs.

ATM is based on a displacement coordinate transformation that swaps the positions of two ligands, one of which is initially placed in the binding site of the receptor and the other into the solvent bulk.\cite{azimi2022relative} The potential energies of the system before and after the displacement are combined into a $\lambda$-dependent potential function, such that the system is progressively transformed from the state in which the first ligand is bound to the receptor and the second is in solution, to the reversed situation in which the second ligand is bound to the receptor and the first is not. ATM protocol does not require soft-core pair potentials or modifications of the energy routines of the molecular dynamics engine, and it does not require splitting the binding free energy calculation into receptor and solvation legs.  ATM is implemented as an OpenMM plugin.\cite{ATMMetaForce-OpenMM-plugin} Further details of the methodology can be found in previous work.\cite{gallicchio2015asynchronous,wu2021alchemical,azimi2022relative} 

\begin{figure}[h!]
\centering
\includegraphics[width=\columnwidth]{Figures/Workflow_ATM.pdf}
\caption{ The AToM-OpenMM workflow used in this work. (1) Starting from the protein-ligand complexes from the Wang dataset\cite{wang2015accurate} the ligands are parametrised with GAFF2 and protein tropologies prepared in the Amber ff4SB forcefield. (2) System complexes are build with Ambertools and ligand B is displaced based on a vector. (3) Energy minimisation and equilibration is performed. Later an annealing and equilibration at $\lambda$=1/2 is performed. (4) Aysncronous replica exchange simulations are performed until a total sampling of at least 50ns is achieved. (5) After the simulations were finished, these were analyzed with the UWHAM package to obtain the calculated $\Delta\Delta G$ estimates.}
\label{fig:Workflow_ATM}
\end{figure}

We used the AToM-OpenMM package\citep{AToM-OpenMM} to set up and run the alchemical calculations. The AToM-OpenMM workflow (Fig. \ref{fig:Workflow_ATM}) prepares the complex systems for simulation using the LEaP program in AmberTools19.\cite{AmberTools19} Amber ff14SB parameters\cite{zou2019blinded,maier2015ff14sb} were assigned to the receptors while GAFF2/AM1-BCC\cite{wang2006automatic,he2020fast} were used for the ligands. Each complex system built in  LEaP consists of the receptor and a pair of aligned ligands. One of the ligands is selected to be translated along the diagonal of the solvent box so it is placed outside the receptor, ensuring at least three layers of water molecules in between. A restraining potential is also introduced to maintain geometrical  alignment between the two ligands aimed at enhancing the rate of convergence of the free energy estimate. The alignment restraints are based on the relative position and orientation  of the coordinate frames of the two ligands  defined by three chosen reference atoms.\cite{azimi2022relative} More information on the reference atoms selected for each system can be found in the Supporting Information (Supporting Figure \ref{fig:ref_alignment}).

Each complex system was solvated with a 10 \AA\ solvent buffer and with sufficient sodium and chloride ions to neutralize the system. The solvated complexes are minimized and thermalized at 300 K. Next, the system was annealed from the bound state ($\lambda$ = 0) to the symmetric alchemical intermediate ($\lambda$ = 0.5) for 250 ps. This step facilitates the creation of an initial configuration of the system without strong repulsive interactions at the alchemical intermediate state, which serves as the starting point for the subsequent Hamiltonian replica exchange\cite{gallicchio2015asynchronous} molecular dynamics that computes the free energies of the two ATM legs connecting the alchemical intermediate to physical end states at $\lambda=0$ and $\lambda=1$. C$\alpha$ atoms of the protein receptor were restrained using a flat-bottom harmonic restraining potential with a tolerance of 1.5 \AA.Additionally, we apply binding site restraints as a flat-bottom distance restraint between the geometrical centres of sets of receptor atoms that surround the binding site and ligand atoms. This defines the binding site volume as required by the quasi-chemical statistical mechanics formulation of molecular binding.\cite{Gilson:Given:Bush:McCammon:97} The restraints of the receptor and the ligands are optional and employed here to limit the conformational space that needs to be explored to reach convergence of the binding free energy estimate.
\begin{figure*}
\centering
\includegraphics[width=\linewidth]{Figures/Equation.pdf}
\caption{ Top: Free energy diagram for an ATM-RBFE calculation, consisting of two independent legs connected to single alchemical intermediate state. The first leg starts at $\lambda$=0 where ligand A is bound to the receptor R's binding site, and ligand B is present in the solvent bulk. Leg 1 ends at $\lambda$=1/2 where both ligands A and B are simultaneously present at 50\% both in the binding site and in the solvent bulk. The second leg starts with ligand B bound to the binding site and ligand A in the solvent bulk and ends in the same alchemical intermediate. Bottom: Graphical representations for the described events.}
\label{fig:Scheme_ATM}
\end{figure*}

The softplus alchemical potential\cite{khuttan2021alchemical} was used for all calculations with 11 $\lambda$-states distributed between $\lambda$ = 0 and 0.5 for each of the two ATM legs (Supporting Table \ref{tab:params_table}). AToM-OpenMM performs asynchronous Hamiltonian replica exchanges in $\lambda$-space using the method described by Gallicchio \textit{et al.} \cite{gallicchio2015asynchronous} Exchanges were performed every 10 ps. To maintain a temperature of 300 K, a Langevin thermostat with a time constant of 2 ps was employed. Each ligand pair was simulated for a minimum of 50 ns per $\Delta\Delta G$ estimate. The sampling time has been chosen in order to be comparable to the aforementioned works, FEP+ studies sample between 36 and 60 ns per $\Delta\Delta G$ estimate whereas Lee \textit{et al.}\cite{lee2020alchemical} employed a total of 48 ns per ligand pair in Amber. In the case of pmx,\cite{gapsys2020large} calculations were carried out for 50 ns per pair for two force fields, GAFF and CGenFF. Since we performed ATM calculations on wall time rather than simulation time, as at the time of performing these calculations there was no support for simulation time-based runs, the sampled simulation time is similar but not identical for all ligand pairs.  Binding free energies and their corresponding uncertainties were calculated from the perturbation energy samples using the UWHAM method.\cite{tan2012theory} The obtained relative binding free energies ($\Delta \Delta G$) were compared to experimental measurements in terms of the mean absolute error (MAE), root mean square error (RMSE), and Pearson correlation coefficient. The obtained values are compared to the corresponding values reported in the literature.\cite{wang2015accurate,lee2020alchemical,gapsys2020large}

The parallel replica exchange alchemical molecular dynamics simulations were performed with the OpenMM 7.7\cite{eastman2017openmm} MD engine and the ATM Meta Force plugin\cite{ATMMetaForce-OpenMM-plugin} using the CUDA platform on NVIDIA RTX 2080 Ti cards.

\subsection{Results}

We conducted a comparison of ATM's relative binding free energies ($\Delta \Delta G$) estimates for the benchmarking dataset of Wang \textit{et al.}\cite{wang2015accurate} with those of commercial and open-source alchemical approaches: FEP+,\cite{wang2015accurate} Amber.\cite{lee2020alchemical} and pmx.\cite{gapsys2020large} In addition to differences in the methodology, these comparisons test variations in protein force fields, ligand parameterization techniques, and differences in the behavior of MD packages as potential sources of deviation in the obtained results. In this study, we have chosen to maintain the parameters described in the previous ATM publications as we believe it would provide a fair and consistent comparison. 


\begin{figure}[h!]
\centering
\includegraphics[width=\columnwidth]{Figures/corr_mae_vertical.pdf}
\caption{ (Top) Pearson correlation (r) and (bottom) Mean Absolute Error (MAE) for each protein-ligand system calculated with ATM and reported estimates using the alternative methodologies FEP+\cite{wang2015accurate}, Amber\cite{lee2020alchemical} and pmx\cite{gapsys2015pmx}.}
\label{fig:comparison_corrs}
\end{figure}

\begin{figure*}
\centering
\includegraphics[width=\linewidth]{Figures/compare_prots_ATM.pdf}
\caption{ Performance of the Alchemical Transfer Method (ATM) for each protein-ligand system studied. The calculated $\Delta\Delta G$ estimates are plotted against their corresponding experimental values. MAE is in kcal/mol, $r$ is Pearson correlation and values refer to the number of ligand pairs evaluated for each system.}
\label{fig:prots_ATM}
\end{figure*}

The results of the simulations are displayed in Figures \ref{fig:comparison_corrs} and \ref{fig:prots_ATM}, which highlight the relative (Pearson correlation) and absolute (MAE) performance of the method. Table \ref{tab:result_table} contains a comparison against other free energy methods. Comparison against the other mentioned methodologies can be found in Supporting Figures \ref{fig:comparison_corrsSI_1}, \ref{fig:comparison_corrsSI_2} and \ref{fig:comparison_RMSE}. We observe that ATM performs similarly to the other approaches in overall Pearson correlation (0.59), with values for specific systems ranging from 0.42 to 0.71. ATM's Pearson correlation coefficients are particularly  good  for the MCL-1, JNK1, and Thrombin datasets, where it outperforms the other methods albeit only by relatively small margins. For the other protein targets, Pearson correlation metrics fall within those of the other methodologies. For instance, in the case of p38, the observed correlation is 0.71 for ATM, which is the lowest when compared to the other approaches. However, the difference is not significant as the results for all methods fall within the measurement error. Despite these positive observations, we did encounter some difficulties with the BACE dataset, as we obtained the lowest correlation (0.42) among the methods which is significantly lower than the best correlation value obtained using the FEP+ methodology (0.61). It is worth mentioning that the range of $\Delta\Delta G$ values of the BACE pairs is quite narrow and covers only 3.5 kcal/mol, while comparable in-size datasets cover a wider range of values of at least 5 kcal/mol. In effect inaccuracies of the method, as well as experimental measurements, get amplified.

\begin{table*}
\footnotesize
\centering
\begin{tabular}{|c|cc|cc|cc|cc|}
\hline
         &    ATM  &   &  FEP+\cite{wang2015accurate}    &    & Amber\cite{lee2020alchemical} &     &     pmx\cite{gapsys2020large} &  \\
         & r & MAE  & r & MAE   & r & MAE  & r & MAE  \\
\hline
\hline
MCL1     & \textbf{0.58$\pm$0.10} & 1.7$\pm$0.2 & 0.51$\pm$0.10 &  \textbf{1.4$\pm$0.2} & 0.51$\pm$0.10 & 1.3$\pm$0.2 & 0.32$\pm$0.11 & 1.2$\pm$0.2 \\
TYK2     & 0.63$\pm$0.17 & 0.9$\pm$0.2  & \textbf{0.70$\pm$0.15} & \textbf{0.7$\pm$0.2} & 0.58$\pm$0.17 & 0.9$\pm$0.2 & 0.64$\pm$0.16 & 1.0$\pm$0.2 \\
JNK1     & \textbf{0.69$\pm$0.13} & 0.6$\pm$0.1 & 0.59$\pm$0.15 & 0.8$\pm$0.1 & 0.59$\pm$0.15 & 0.7$\pm$0.2 & 0.65$\pm$0.14 & \textbf{0.5$\pm$0.1} \\
PTP1B    & 0.66$\pm$0.11 & 0.9$\pm$0.2 & 0.66$\pm$0.11 & 0.9$\pm$0.2 & 0.72$\pm$0.10 & 0.8$\pm$0.2 & \textbf{0.74$\pm$0.10} & \textbf{0.8$\pm$0.2} \\
CDK2     & 0.58$\pm$0.17 & 1.0$\pm$0.2 & 0.39$\pm$0.19 & 0.9$\pm$0.2 & 0.46$\pm$0.19 & 0.9$\pm$0.2 & \textbf{0.61$\pm$0.17} & \textbf{0.7$\pm$0.2} \\
Thrombin & \textbf{0.61$\pm$0.21} & 0.8$\pm$0.2 & 0.42$\pm$0.24 & 0.8$\pm$0.2 & 0.31$\pm$0.25 & \textbf{0.4$\pm$0.2} & -0.02$\pm$0.27 & 0.5$\pm$0.2\\
p38      & 0.71$\pm$0.09 & 1.0$\pm$0.2 & 0.79$\pm$0.08 & 0.8$\pm$0.2 & \textbf{0.80$\pm$0.08} & 0.8$\pm$0.2 & 0.78$\pm$0.08 & \textbf{0.7$\pm$0.2} \\
BACE     & 0.42$\pm$0.12 & 1.2$\pm$0.2  & \textbf{0.61$\pm$0.11} & \textbf{0.8$\pm$0.2} & 0.54$\pm$0.11 & 0.9$\pm$0.2 & 0.49$\pm$0.12 & 0.9$\pm$0.2  \\
\hline
ALL      & 0.59$\pm$0.11 & 1.0$\pm$0.1 & \textbf{0.60$\pm$0.11} & 0.9$\pm$0.1 & 0.58$\pm$0.11 & 0.9$\pm$0.1 & 0.56$\pm$0.11 & \textbf{0.8$\pm$0.1}\\
\hline
\end{tabular}
\caption{\label{tab:result_table} Comparison of the performance of free energy methods. Pearson correlation (r) and Mean Absolute Error (MAE) in kcal/mol for the 8 tested Protein Targets. }
\end{table*}

When considering absolute deviations from the experimental references, ATM displayed consistently poorer performance than the other methodologies. ATM's MAE metric is the highest among the three methods considered in the comparison. However, the difference between ATM and other methodologies is not very high in most cases. The exception to this is BACE, where, consistent with the previously mentioned results, the differences are the highest. 

In terms of convergence, we observed that 50 to 60 ns per $\Delta \Delta G$ estimate (2.3-2.8 ns per $\lambda$) tends to be sufficient. Convergence analysis over time shows good convergence for most cases as illustrated in Figure (Figure \ref{fig:convergence_plots}). The variance of the predicted $\Delta \Delta G$ tends to level off for a majority of cases around 50 ns per $\Delta \Delta G$ estimate. We also performed longer simulations for a series of ligand pairs to evaluate if that extended sampling time was causing any drift on the predicted $\Delta \Delta G$ (Supporting Figure \ref{fig:Converge_longer}) where we observed that obtained values were stable.  Furthermore, we analysed the perturbation energy distributions for every $\lambda$-state (Supporting Figures \ref{fig:perturbation_MCL1},\ref{fig:perturbation_TYK2},\ref{fig:perturbation_JNK1}). This analysis helps to determine if the system has converged or will converge in a reasonable amount of time. A poor overlap between perturbation energy distributions are indicative of unreliable relative free energy estimates that could require the design of alternative alchemical routes. We have observed that in some of the ligand pairs that showed poor correlation with both experimental and calculated values from other approaches, there tends to be a poor overlap between perturbation energy distributions of nearby $\lambda$-states, even after conducting multiple replicates. From these results we can say that we obtain an analogous convergence performing a similar sampling time than the other mentioned methodologies in this work. The issue of convergence of binding free energy calculations is a very complex topic that we intend to investigate in future work.

\begin{figure}
\centering
\includegraphics[width=\columnwidth]{Figures/convergence_plots.pdf}
\caption{\label{fig:convergence_plots} Free energy convergence as a function of time for a series of ligand pairs of MCL-1, TYK2 and JNK1. The red line corresponds to the experimental $\Delta \Delta G$ value. }
\end{figure}


We have observed that ATM's performance metrics are significantly skewed by poor relative binding free energy predictions involving a relatively small number of problematic ligands (Supporting Figure \ref{fig:problematic_ligands}). These ligands might be affected by force field parameterization issues or some specific aspects of the ATM methodology. In terms of ligand force fields, OPLS3 was used for FEP+, GAFF2 was used in two of the approaches (ATM and Amber), and a consensus of GAFF2 with CGenFF in pmx. Given that all methods perform similarly, the accuracy of the forcefield is of the same order as the precision of the methods. In relation to this aspect, Merck published a series of guidelines for FEP calculations with the requirement of an RMSE  lower than 1.3 kcal/mol in the validation phase.\cite{schindler2020large}. As we can observe in Table \ref{tab:result_table} ATM fulfills this requirement for most of the analyzed systems in this study. 

One major difference between ATM and the other methods is that ATM models explicitly binding/unbinding processes in the alchemical space. For example, a flexible ligand with different conformational propensities when bound vs when in solution, will undergo an actual conformational transition. In double-decoupling instead, the transformation is applied to the bound and solution conformations individually and a conformational transition is not necessarily required to reach convergence. We believe that this is both a strength and a weakness of ATM. When conformational changes are important, especially if there are differences in the conformational transition between the ligands, ATM is expected to be superior to other methods as it explicitly models the transitions. When considering rigid ligands or ligands with similar transitions, the conformational rearrangements will cancel out and the extra work ATM needs to do is unnecessary and might hurt convergence. We intend to study these aspects in more detail in future work.


\section{Conclusion}

In this study, we evaluated the performance of the Alchemical Transfer Method (ATM), a novel approach for predicting protein-ligand binding affinities. We benchmarked it against the dataset of Wang \textit{et al.}\cite{wang2015accurate}, one of the most popular data sets on the evaluation of free binding energy methodologies. Our results showed that ATM is a competitive approach for predicting binding affinities, matching or even surpassing the performance of other state-of-the-art methods in terms of Pearson correlation. While mean absolute errors were slightly higher compared to other methods, ATM is a promising approach for the estimation of relative binding free energies.

Unlike other methods, ATM does not require splitting of binding free energy calculations into receptor and solvation legs or the use of softcore pair potentials. Furthermore, ATM is implemented in the open-source OpenMM MD engine, which is freely available. Its flexibility opens up the possibility for further improvement of the method through the use of new force fields, such as neural network potentials.

\section{Data and software availability}
The calculated free energy values, ligand and protein structures as well as preparation scripts are available at: \url{https://github.com/compsciencelab/ATM_benchmark}

\section{Acknowledgement}
This project has received funding from
the European Union’s Horizon 2020 research and innovation programme under grant agreement No. 823712;
and the project PID2020-116564GB-I00 has been funded by MCIN / AEI / 10.13039/501100011033; the Torres-Quevedo Programme from the Spanish National Agency for Research (PTQ2020-011145 / AEI / 10.13039/501100011033).
Research reported in this publication was supported by the National Institute of General Medical Sciences (NIGMS) of the National Institutes of Health under award number GM140090. {E. G.} acknowledges support from the United States' National Science Foundation (NSF CAREER 1750511).
The content is solely the responsibility of the authors and does not necessarily represent the official views of the National Institutes of Health.
\clearpage
\onecolumn
\documentclass[journal=jacsat,manuscript=article]{achemso}
\setkeys{acs}{maxauthors=299}
\usepackage[utf8]{inputenc}
\usepackage{physics}
\usepackage{braket}
\usepackage{bm}% bold math
\usepackage[hidelinks]{hyperref}% add hypertext capabilities
\SectionNumbersOn 
\usepackage{multirow}
\usepackage{xcolor}
\usepackage{graphicx}
\usepackage{qcircuit}
%\usepackage{quantikz}
\usepackage{tikz}
\usetikzlibrary{positioning}
% smaller spacing in itemize
\usepackage{enumitem}
\setlist{nosep}
% lstlisting font
\usepackage{courier}
\usepackage[version=4]{mhchem}

\usepackage{listings}
% Code annotations
\definecolor{codegreen}{rgb}{0,0.6,0}
\definecolor{codegray}{rgb}{0.5,0.5,0.5}
\definecolor{codepurple}{rgb}{0.58,0,0.82}
\definecolor{backcolour}{rgb}{0.95,0.95,0.92}

\lstdefinestyle{mystyle}{
    backgroundcolor=\color{backcolour},   
    commentstyle=\color{codegreen},
    keywordstyle=\color{magenta},
    numberstyle=\tiny\color{codegray},
    stringstyle=\color{codepurple},
    basicstyle=\linespread{0.9}\ttfamily\footnotesize,
    breakatwhitespace=false,         
    breaklines=true,                 
    captionpos=b,                    
    keepspaces=true,                 
    numbers=left,                    
    numbersep=5pt,                  
    showspaces=false,                
    showstringspaces=false,
    showtabs=false,                  
    tabsize=2,
    basewidth=0.55em,
}

\lstset{style=mystyle}
% change listing to code snippet
\renewcommand{\lstlistingname}{Code Snippet S\hspace{-0.14cm}}

\newcommand{\bp}[1]{\vspace{\baselineskip}\noindent\textbf{#1}}
\definecolor{bkgd}{RGB}{240,242,246}
\definecolor{orange-red}{rgb}{1.0, 0.27, 0.0}
\newcommand{\fancylink}[2]{\colorbox{bkgd}{\color{orange-red}\href{#1}{\sf {#2}}}}

\newcommand{\tc}{\textsc{TensorCircuit}}
\newcommand{\tcc}{\textsc{TenCirChem}}
\newcommand{\jax}{\textsc{JAX}}
\newcommand{\numpy}{\textsc{NumPy}}
\newcommand{\cupy}{\textsc{CuPy}}
\newcommand{\pyscf}{\textsc{PySCF}}
\newcommand{\scipy}{\textsc{SciPy}}
\newcommand{\reno}{\textsc{Renormalizer}}
\newcommand{\openfermion}{\textsc{OpenFermion}}
\newcommand{\opteinsum}{\textsc{opt\_einsum}}
\newcommand{\qiskit}{\textsc{QisKit}}
\newcommand{\qiskitnature}{\textsc{QisKit-Nature}}
\newcommand{\pennylane}{\textsc{PennyLane}}
\newcommand{\tequilla}{\textsc{Tequilla}}
\newcommand{\mindquantum}{\textsc{MindQuantum}}
\newcommand{\qschem}{\textsc{Q$^2$ Chemistry}}
\newcommand{\qforte}{\textsc{QForte}}
\newcommand{\nex}{N_{\rm{ex}}}
\newcommand{\nshots}{N_{\rm{shots}}}


\newcommand{\jon}[1]{\textcolor{red}{#1}}
\newcommand{\jpm}[1]{\textcolor{blue}{#1}}
\newcommand{\wtli}[1]{\textcolor{cyan}{#1}}
\newcommand{\lc}[1]{\textcolor{purple}{#1}}

\makeatletter
\renewcommand \thesection{S\@arabic\c@section}
\renewcommand\thetable{S\@arabic\c@table}
\renewcommand \thefigure{S\@arabic\c@figure}
\renewcommand*{\thepage}{S\arabic{page} }
\makeatother

\title{Supplementary Materials for TenCirChem: An Efficient Quantum Computational Chemistry Package for the NISQ Era}

\author{Weitang Li}
\email{liw31@gmail.com}
\affiliation{Tencent Quantum Lab, Shenzhen, 518057, China}


\author{Jonathan Allcock}
%\email{jonallcock@tencent.com}
\affiliation{Tencent Quantum Lab, Hongkong, 999077, China}

\author{Lixue Cheng}
\affiliation{Tencent Quantum Lab, Shenzhen, 518057, China}

\author{Shi-Xin Zhang}
\affiliation{Tencent Quantum Lab, Shenzhen, 518057, China}

\author{Yu-Qin Chen}
\affiliation{Tencent Quantum Lab, Shenzhen, 518057, China}

\author{Jonathan P. Mailoa}
\affiliation{Tencent Quantum Lab, Shenzhen, 518057, China}

\author{Zhigang Shuai}
\affiliation{Department of Chemistry, Tsinghua University, Beijing, 100084, China}
\alsoaffiliation{School of Science and Engineering, The Chinese University of Hong Kong, Shenzhen, 518172, China}

\author{Shengyu Zhang}
\email{shengyzhang@tencent.com}
\affiliation{Tencent Quantum Lab, Hongkong, 999077, China}


\begin{document}

\maketitle
%We don't need TOC here
%\tableofcontents

\section{Review of UCC ansatz}
We start with the general expression
\begin{equation}
\label{eq:ucc2}
    \ket{\Psi(\theta)}_{\text{UCC}} := \prod_{k=\nex{}}^1 e^{\theta_k G_k} \ket{\phi}. \
\end{equation}

\subsection{UCCSD}
For the most common case of single and double excitations, the $G_k$ has the form
\begin{equation}
\label{eq:g-single-double}
    G_k = \begin{cases}
        a^\dagger_p a_q - \text{h.c.} , \\
        a^\dagger_p a^\dagger_q a_r a_s - \text{h.c.}
        \end{cases}
\end{equation}

In this case, the disentangled UCC ansatz -- which is known as the UCCSD ansatz -- has the form
\begin{equation}
\label{eq:uccsd}
    \ket{\Psi(\theta)}_{\text{UCCSD}} := \prod_{pqrs}e^{\theta_{pqrs}(a^\dagger_p a^\dagger_q a_r a_s - \text{h.c.})}\prod_{pq}e^{\theta_{pq}(a_p^\dagger a_q - \text{h.c.})}\ket{\phi} \ .
\end{equation}
%In the remainder of this paper, we will refer to the first-order Trotterized version of the UCCSD ansatz as the the ``UCCSD ansatz".
If the number of spin-orbitals is $N$, then the number of gates in the corresponding circuit scales as $\order{N^4}$~\cite{o2019generalized}.
% if have time, explain the traditional approach in more detail, such as the scaling of the number of gates.

\subsection{$k$-UpCCGSD}
This is a variant of the UCC ansatz which satisfies both high accuracy and low ($\order{N^2}$) gate count requirements.~\cite{lee2018generalized}
Here ``G'', ``p'' and ``$k$'' stand for generalized excitation, paired double excitations, and repeating the ansatz $k$ times, respectively.
By pairing double excitations, the number of double excitations is reduced from $\order{N^4}$ to $\order{N^2}$.
More specifically, the ansatz has the form
\begin{equation}
\label{eq:kupccgsd}
    \ket{\Psi(\theta)}_{k\text{-UpCCGSD}} := \prod_{l=1}^k \prod_{pq}e^{\theta_{kpq}^{(2)}(a^\dagger_{p\alpha} a^\dagger_{p\beta} a_{q\beta} a_{q\alpha} - \text{h.c.})}\prod_{rs}e^{\theta_{krs}^{(1)}(a_r^\dagger a_s - \text{h.c.})}\ket{\phi} \ .
\end{equation}
Here $a^\dagger_{p\alpha} a^\dagger_{p\beta} a_{q\beta} a_{q\alpha}$ means spin-paired double excitations from the $q$-th spatial orbital to the $p$-th spatial orbital. 



\subsection{pUCCD}
This is an efficient ansatz requiring only $\order{N}$ circuit depth and half as many qubits as other UCC ans\"{a}tze~\cite{henderson2015pair, elfving2021simulating, o2022purification, zhao2022orbital}.
pUCCD allows only paired double excitations, 
which enables one qubit to represent one spatial orbital instead of one spin orbital,
and removes the need to perform the fermion-qubit mapping.
Thus, the $\order{N^2}$ excitations can be executed on a quantum computer efficiently using a compact circuit with a linear depth of Givens-SWAP gates~\cite{elfving2021simulating}.
The Hamiltonian also takes a simpler form in this case, with only $N^2$ terms:
\begin{equation}
    H = \sum_p h_p c^\dagger_p c_p + \sum_{pq} v_{pq} c^\dagger_p c_q + \sum_{p \neq q} w_{pq} c^\dagger_p c_p c^\dagger_q c_q + E_{\rm{nuc}} \ ,
\end{equation}
where $h_p = 2h_{pp}$, $v_{pq} = (pq|pq)$ and $\omega_{pq} = 2 (pp|qq) - (pq|pq)$.
Here $p$ and $q$ are indices for spatial orbitals.
A drawback of this ansatz is its compromised accuracy, which is comparable to the doubly occupied configuration interaction (DOCI) method \cite{weinhold1967reduced} in quantum chemistry.

\section{Conventions}
\label{sec:preliminary}
In this section, we introduce the conventions used in \tcc{}, particularly for static electronic structure problems. 

\bp{Excitations. }\tcc{} uses tuples to denote unitary fermionic excitations:
\begin{itemize}
    \item $(p,q)$: denotes single excitations $a_p^\dagger a_q- a_q^\dagger a_p$
    \item $(p,q,r,s)$:  denotes double excitations $a_p^\dagger a_q^\dagger a_r a_s - a_s^\dagger a_r^\dagger a_q a_p$
\end{itemize} 
Higher-order excitations are handled similarly. e.g., a $k$-th order excitation is represented by a tuple of length $2k$, with the first half corresponding to the excitation operators, and the second half corresponding to the annihilation operators, and Hermitian conjugation is implied, e.g., $(p,q,r,s,t,u)$ denotes $a_p^\dagger a_q^\dagger a_r^\dagger a_s a_t a_u - \text{h.c.}$


\bp{Spin-orbital indexing.} Spin orbitals are indexed from 0 and ordered according to the following rules:
\begin{itemize}
    \item Beta (down) spins first, followed by alpha (up) spins
    \item Low energy orbitals first, followed by high energy orbitals
\end{itemize}


\bp{Qubit indexing.} Qubits are numbered from 0, with multi-qubit registers numbered  with the zeroth qubit on the left, e.g. $\ket{0}_{q_0}\ket{1}_{q_2}\ket{0}_{q_2}$. Unless necessary, we will omit subscripts and use a more compact notation e.g.\ $\ket{010}$ to denote multi-qubit states.  In quantum circuit diagrams qubit numbers increase downwards, starting from qubit zero at the top. Throughout, we assume the Jordan-Wigner encoding where the occupancy of spin-orbital $i$  (using the above ordering) corresponds  to the state of qubit $N - 1 - i$ where $N$ is the total number of qubits.
In other words, the qubit ordering is reversed spin-orbitals ordering.

The conventions described above are summarized in Fig.~\ref{fig:excitations}, 
taking a 4-electron and 4-orbital system as an example.
Following the qubit indexing convention above, the restricted Hartree--Fock (HF) state for such a system is represented as 00110011 in bitstring form.
The highest orbital with $\alpha$-spin comes first in the bitstring, and the lowest orbital with $\beta$-spin comes last.
In quantum chemistry language, 00110011 refers to a configuration with orbitals 5, 4, 1, and 0 occupied,
while in quantum computing language, 00110011 refers to a direct product state
with the 2nd, 3rd, 6th, and 7th qubits in state $\ket{1}$ and the rest in state $\ket{0}$.
Upon application of the excitation operator (6, 2, 0, 4), the HF state transforms to 01100110.

\begin{figure}[htp]
    \centering
    \includegraphics[width=0.65\textwidth]{excitations.png}
\caption{An example of the conventions used for excitations, orbital indexing, and qubit indexing. The HF state of a 4 electron, 4 orbital system is represented as bitstring 00110011. The state is excited to 01100110 via the excitation operator (6, 2, 0, 4).}
    \label{fig:excitations}
\end{figure}

\section{\lstinline{print_summary} output}
In this section, we provide an overview of the output by the \lstinline{print_summary} command (see Code Snippet~\ref{lst:print_summary}).  This summary is divided into a number of blocks:

\begin{lstlisting}[caption={UCCSD ansatz applied to the \ce{H_2} molecule: output of the \lstinline{print_summary} command.}, label={lst:print_summary}]
################################ Ansatz ###############################
 #qubits  #params  #excitations initial condition
       4        2             3               RHF
############################### Circuit ###############################
 #qubits  #gates  #CNOT  #multicontrol  depth #FLOP
       4      15     10              1      9  2160
############################### Energy ################################
       energy (Hartree)  error (mH) correlation energy (%)
HF            -1.116706   20.568268                 -0.000
MP2           -1.129868    7.406850                 63.989
CCSD          -1.137275   -0.000165                100.001
UCCSD         -1.137274    0.000000                100.000
FCI           -1.137274    0.000000                100.000
############################# Excitations #############################
     excitation configuration     parameter  initial guess
0        (3, 2)          1001  1.082849e-16       0.000000
1        (1, 0)          0110  1.082849e-16       0.000000
2  (1, 3, 2, 0)          1010 -1.129866e-01      -0.072608
######################### Optimization Result #########################
            e: -1.1372744055294384
          fun: array(-1.13727441)
     hess_inv: <2x2 LbfgsInvHessProduct with dtype=float64>
   init_guess: [0.0, -0.07260814651571333]
          jac: array([-9.60813938e-19, -1.11022302e-16])
      message: 'CONVERGENCE: NORM_OF_PROJECTED_GRADIENT_<=_PGTOL'
         nfev: 6
          nit: 4
         njev: 6
     opt_time: 0.02926015853881836
 staging_time: 4.5299530029296875e-06
       status: 0
      success: True
            x: array([ 1.08284918e-16, -1.12986561e-01])
\end{lstlisting}


\bp{Ansatz (lines 1-3).}  From here we see that the UCCSD ansatz is mapped (via the Jordan-Wigner transformation) to a variational quantum circuit on $4$ qubits, with two tunable parameters.  The circuit corresponds to $3$ excitation operators, which are detailed in the \lstinline{Excitations} block of the summary.  The fact that there are only two tunable parameters for three excitation operators is due to symmetry.

\bp{Circuit (lines 4-6).}  This block details the number of quantum gates that the Jordan-Wigner transformed ansatz is compiled to.  By default, \tcc{} uses the efficient circuit decomposition of \cite{yordanov2020efficient} for circuit compilation. 
The circuit depth is estimated using \qiskit{} and, as the running time of computing this estimate can be non-negligible for large circuits, the \lstinline{include_circuit=True} option
is by default False. 
The floating point operation count (FLOP) required to calculate the circuit statevector via tensor network contraction 
is estimated by the \opteinsum{} package \cite{daniel2018opt} using the default greedy contraction path-finding algorithm.

Note that, in the absence of noise, computing the output of the circuit does not require the circuit to first be decomposed explicitly into gates.  Instead, \tcc{} makes use of \textit{UCC factor expansion} (See Sec.~\ref{sec:ucc-theory}) to perform this computation more efficiently. However, if one wishes to simulate noisy quantum circuits -- corresponding, for instance, to realistic quantum hardware -- then an explicit gate decomposition is necessary. 

\bp{Energy (lines 7-13).}  This block includes the final UCCSD energy corresponding to the optimized parameter values, as well as other benchmark energies computed by \pyscf{} at Hartree--Fock, second-order Møller–Plesset perturbation theory (MP2), CCSD, and FCI levels of theory.  


\bp{Excitations (lines 14-18).}  This block gives further details on the ansatz, which takes the form
\begin{align*}
e^{\theta_2 G_2}  e^{\theta_1 G_1}  e^{\theta_0 G_0}\ket{\text{HF}},
\end{align*}
where $\theta_0$ is set to be equal to $\theta_1$ due to symmetry, and where
\begin{align*}
    G_0 &= a_2^\dagger a_3 - a_3^\dagger a_2, \\
    G_1 &= a_0^\dagger a_1 - a_1^\dagger a_0, \\
    G_2 &= a_0^\dagger a_2^\dagger a_3a_1 - a_1^\dagger a_3^\dagger a_2a_0.
\end{align*}
The convention for spin-orbital indices 
is described in Sec.~\ref{sec:preliminary}.
The minimum energy found corresponds to parameter values of $\theta_0 = \theta_1 = 1.05\times 10^{-16}$ and $\theta_2 = -1.13\times 10^{-1}$.  These final values were obtained from initial guesses of $0,0, -0.0726$ respectively, which correspond to the MP2 excitation amplitudes.  The configuration bitstrings obtained after applying each excitation to $\ket{\text{HF}}$ are also given here. 
Note that the Hermitian conjugation part of the excitation 
operator annihilates $\ket{\text{HF}}$ and thus has no effect.
For comparison, the configuration bitstring corresponding to the Hartree--Fock state is 0101.

\bp{Optimization Result (lines 19-33).}  This block gives details of the procedure used to optimize the ansatz parameters, including the number of iterations required (\lstinline{nit: 3}) as well as the final energy obtained for the UCCSD ansatz (\lstinline{e: -1.137...}). 


\section{The UCC classes: advanced features}
\subsection{User-specified UCC ans\"{a}tze}
UCCSD, $k$-UpCCGSD, and pUCCD are all special cases of UCC, where the excitations are restricted to take a certain form.  No setup is required for these three kinds of ansatz from users: once imported they can be used directly, e.g., via \lstinline{PUCCD(h2).kernel()}.


More general UCC ans\"{a}tze can be defined by directly specifying the corresponding excitations. 
This could be helpful, for instance, in investigating new quantum computational chemistry algorithms.  
% based on UCC, as was the case for the
% ADAPT-VQE algorithm which relies heavily on customized UCC ~\cite{grimsley2019adaptive}.
In Sec.~\ref{sec:adapt-vqe}, we will illustrate the power of this approach by implementing the ADAPT-VQE~\cite{grimsley2019adaptive} using \tcc{}, and we first provide a simpler example and implement the PUCCD ansatz from scratch (see Code Snippet~\ref{lst:custom_ucc}).
\begin{lstlisting}[language=Python, caption={Implementing the PUCCD ansatz from scratch by using a custom UCC ansatz.}, label={lst:custom_ucc}]
import numpy as np
from tencirchem import UCC, PUCCD
from tencirchem.molecule import h4

puccd = PUCCD(h4)
ucc = UCC(h4)

# only paired excitations are included
ucc.ex_ops = [
    (6, 2, 0, 4),
    (7, 3, 0, 4),
    (6, 2, 1, 5),
    (7, 3, 1, 5),
]

e1 = puccd.kernel()
# evaluate UCC energy using the PUCCD circuit parameter
e2 = ucc.energy(puccd.params)
# e1 is the same as e2
np.testing.assert_allclose(e1, e2, atol=1e-6)
\end{lstlisting}
Here, a \lstinline{UCC} class with only paired excitations is used to reproduce the \lstinline{PUCCD} class
for the \ce{H4} molecule.
The configuration is done by directly setting the \lstinline{ex_ops} attribute with tuples for the excitations.
\lstinline{param_ids} and \lstinline{initial_guess} can be set similarly 
and the full \fancylink{https://github.com/tencent-quantum-lab/TenCirChem/blob/master/example/custom_excitation.py}{Python script} is available online.
Although internally the \lstinline{UCC} class uses a quantum circuit of 8 qubits 
for simulation and the \lstinline{PUCCD} class uses only 4 qubits because of the restriction of paired excitations,
%\jon{(Comment) how can the user further set only 4 qubits to be used when using custom UCC?}
the corresponding energies with the same set of circuit parameters are exactly the same.
If custom UCC with paired excitation is desired, the \lstinline{UCC} class can be initialized with the \lstinline{hcb=True} argument.
For the index of the spin-orbitals and excitation operators, Fig.~\ref{fig:excitations} is a helpful reference.


\subsection{Active Space Approximation}
\label{sec:active-space}
As NISQ quantum hardware is limited in the number of qubits available, the active space approximation 
is frequently adopted to reduce the problem size~\cite{reiher2017elucidating, takeshita2020increasing, mizukami2020orbital, mccaskey2019quantum, o2022purification, huang2022variational}.
In many cases, inner shell molecular orbitals can be treated at the mean-field level without significant loss of accuracy, and this approximation is thus also sometimes dubbed the frozen core approximation.
Denote the set of frozen occupied spin-orbitals by $\Omega$.
The frozen core provides an effective repulsion potential $V^{\rm{eff}}$  to the remaining electrons
\begin{equation}
\label{eq:effective-pot}
    V^{\rm{eff}}_{pq} = \sum_{m \in \Omega} \left ( [mm|pq] - [mp|qm] \right).    \ 
\end{equation}
The frozen core also bears the mean-field core energy which effectively modifies
the core energy $E_{\rm{nuc}}$
\begin{equation}
    E_{\rm{core}} = E_{\rm{nuc}} + \sum_{m \in \Omega} h_{mm} + \frac{1}{2}\sum_{m, n \in \Omega} \left ( [mm|nn] - [mn|nm] \right ). \
\end{equation}
Thus, the \textit{ab initio} Hamiltonian is rewritten as
\begin{equation}
\label{eq:ham-abinit-as}
    H = \sum_{pq}(h_{pq} + V^{\rm{eff}}_{pq})a^\dagger_p a_q + \sum_{pqrs}h_{pqrs}a^\dagger_p a^\dagger_q a_r a_s + E_{\rm{core}},  \
\end{equation}
where $p$, $q$, $r$ and $s$ refer to spin-orbitals not included in $\Omega$.

%The following code performs a (2e, 2o) active space calculation of the \ce{H8} chain.
Making use of an active space approximation in \tcc{} is straightforward 
%\jon{(Comment) do we need to explain the (2e, 2o) notation?} \lc{I don't think we need to explain, it's just the common practice in chemistry}
: 
\begin{lstlisting}[language=Python, caption={UCCSD calculation of \ce{H8} with (2e, 2o) active space approximation. Note that the electron integrals in the active space are readily 
accessible via \lstinline{uccsd.int1e} and \lstinline{uccsd.int2e}.}]
from tencirchem import UCCSD
from tencirchem.molecule import h8
# (2e, 2o) active space
uccsd = UCCSD(h8, active_space=(2, 2))
uccsd.kernel()
uccsd.print_summary()
\end{lstlisting}

% \subsubsection{UCCSD-CASSCF}
% The complete active-space self-consistent field (CASSCF) method~\cite{roos1980complete}
% is one of the most common methods to treat multi-reference systems.
% CASSCF rotates the molecular orbitals in addition to solving the active space using the FCI kernel.
% In practice, the exponential scaling of the FCI kernel limits the size of the active space to systems with at most 16 electrons and 16 orbitals.
% In order to treat systems with strong correlations spreading over more than (16e, 16o),
% alternative solvers to FCI, such as density matrix renormalization group (DMRG)~\cite{ghosh2008orbital,nakatani2017density}, 
% and FCI quantum Monte Carlo (FCIQMC)~\cite{li2016combining} have been used to implement CASSCF in larger active spaces.
% Recently, orbital-optimized UCC and using UCCSD as a CASSCF solver have also been proposed~\cite{sokolov2020quantum, mizukami2020orbital, takeshita2020increasing}.
% The following code snippet shows how to carry out UCCSD-CASSCF calculation by interfacing \tcc{} with \pyscf{}~\cite{sun2017general}. 
% Each \lstinline{UCC} class can be transformed to a \pyscf{} FCI solver
% and the CASSCF algorithm implemented within the \pyscf{} package is then called 
% with the \lstinline{UCC} class as the kernel.
% Another example for orbital-optimized pUCCD calculation is available as a \fancylink{https://github.com/tencent-quantum-lab/TenCirChem/blob/master/example/oo_puccd.py}{Python script} online.

% \begin{lstlisting}[language=Python, caption={UCCSD-CASSCF calculation of \ce{H8} with (2e, 2o) active space.}]
% from pyscf.mcscf import CASSCF
% from tencirchem import UCCSD
% from tencirchem.molecule import h8
% # normal PySCF workflow
% hf = h8.HF()
% hf.kernel()
% casscf = CASSCF(hf, 2, 2)
% # set the FCI solver for CASSCF to be UCCSD
% casscf.fcisolver = UCCSD.as_pyscf_solver()
% casscf.kernel()
% \end{lstlisting}
% %\jon{(Comment) do we need to explain more what we mean by using uccsd as an fci solver for casscf? e.g. what equation is actually being solved? Or is this section aimed at people already familiar with what this is?}
% By interfacing with \pyscf{}, it is also straightforward to calculate nuclear gradients for the UCC ansatz,
% as demonstrated in the online \fancylink{https://github.com/tencent-quantum-lab/TenCirChem/blob/master/example/nuc_grad.py}{Python script}.


\subsection{Engines, Backends, and GPU}
Different engines can be specified by the \lstinline{engine} argument, e.g.,
\begin{lstlisting}[language=Python]
from tencirchem import UCCSD
from tencirchem.molecule import h4
uccsd = UCCSD(h4, engine="civector-large")
print(uccsd.kernel())
print(uccsd.energy(engine="tensornetwork"))
\end{lstlisting}
and, similar to \tc{}, computational backends can be switched at runtime as follows:
\begin{lstlisting}[language=Python]
from tencirchem import set_dtype, set_backend
set_dtype("complex64")
set_backend("cupy")
\end{lstlisting}
There are two ways to use GPUs with \tcc{}. The first is to set the backend to \lstinline{"cupy"}.
The second is to set the backend to \lstinline{"jax"} and make sure that CUDA support for \jax{} is \href{https://github.com/google/jax#installation}{properly configured}.


\subsection{Code Example: Implementing ADAPT-VQE}
\label{sec:adapt-vqe}

In this section, 
we illustrate how to use \tcc{} to build novel algorithms by implementing ADAPT-VQE~\cite{grimsley2019adaptive}.
The complete \fancylink{https://github.com/tencent-quantum-lab/TenCirChem/blob/master/docs/source/tutorial_jupyter/adapt_vqe.ipynb}{Jupyter Notebook} tutorial is available online.
The first step of the algorithm is to construct an excitation operator pool, as shown below.
Here we use all of the single and double excitations as described in Eq.~\ref{eq:g-single-double}.
Operators with the same parameter are grouped together in the operator pool so as to prevent spin contamination.
\begin{lstlisting}[language=Python, caption=Construction of an operator pool for the ADAPT-VQE algorithm.]
from tencirchem import UCC
from tencirchem.molecule import h4

ucc = UCC(h4)

# get all single and double excitations
# param_id maps operators to parameters (some operators share the same parameter)
ex1_ops, ex1_param_ids, _ = ucc.get_ex1_ops()
ex2_ops, ex2_param_ids, _ = ucc.get_ex2_ops()

# group the operators to form an operator pool
from collections import defaultdict
op_pool = defaultdict(list)
for ex1_op, ex1_id in zip(ex1_ops, ex1_param_ids):
    op_pool[(1, ex1_id)].append(ex1_op)
for ex2_op, ex2_id in zip(ex2_ops, ex2_param_ids):
    op_pool[(2, ex2_id)].append(ex2_op)
op_pool = list(op_pool.values())
\end{lstlisting}

Once the operator pool is formed, ADAPT-VQE constructs the ansatz by selecting operators in the pool iteratively, with new operators chosen to maximize the absolute energy gradient.  That is, suppose the ansatz wavefunction at a certain point in the process is $\ket{\psi}$ and the operator selected is $G_k$, then
the new ansatz, defined to be
\begin{equation}
    \ket{\Psi} = e^{\theta_k G_k} \ket{\psi}, \
\end{equation}
has a corresponding energy gradient 
\begin{equation}
    \pdv{\braket{E}}{\theta_k} = 2 \braket{\Psi|HG_k|\Psi}. \
\end{equation}
ADAPT-VQE selects $G_k$ from the operator pool such that
\begin{equation}
    \pdv{\braket{E}}{\theta_k} \bigg{|}_{\theta_k=0} = 2 \braket{\psi|HG_k|\psi} 
\end{equation}
is maximized. If multiple operators sharing the same parameter are added at the same time,
then their gradients are added together. The iterative process by which the ansatz is constructed terminates when the norm of the gradient vector falls below a predefined threshold $\epsilon$.

In the following, $\ket{\psi}$ is obtained by \lstinline{ucc.civector()} as a vector in the configuration interaction space.
\lstinline{ucc.hamiltonian(psi)} applies $H$ onto $\ket{\psi}$ and
\lstinline{ucc.apply_excitation} applies $G_k$ onto $\ket{\psi}$.

\begin{lstlisting}[language=Python, caption={Implementation of the ADAPT-VQE iteration. The complete tutorial is available online as a  \fancylink{https://github.com/tencent-quantum-lab/TenCirChem/blob/master/docs/source/tutorial_jupyter/adapt_vqe.ipynb}{Jupyter Notebook}.}]
import numpy as np

ucc.ex_ops = []
ucc.params = []
ucc.param_ids = []

MAX_ITER = 100
EPSILON = 1e-3
for i in range(MAX_ITER):
    # calculate gradient of each operator from the pool
    op_gradient_list = []
    psi = ucc.civector()
    bra = ucc.hamiltonian(psi)
    for op_list in op_pool:
        grad = bra.conj() @ ucc.apply_excitation(psi, op_list[0])
        if len(op_list) == 2:
            grad += bra.conj() @ ucc.apply_excitation(psi, op_list[1])
        op_gradient_list.append(2 * grad)
    if np.linalg.norm(op_gradient_list) < EPSILON:
        break
    chosen_op_list = op_pool[np.argmax(np.abs(op_gradient_list))]
    # update ansatz and run calculation
    ucc.ex_ops.extend(chosen_op_list)
    ucc.params = list(ucc.params) + [0]
    ucc.param_ids.extend([len(ucc.params) - 1] * len(chosen_op_list))
    ucc.init_guess = ucc.params
    ucc.kernel()
\end{lstlisting}


\section{Efficient UCC circuit simulation algorithms}
We now introduce the algorithm behind efficient UCC circuit simulation that makes \tcc{} capable of accurately simulating deep UCC circuits with more than 32 qubits.
We will first introduce the UCC factor expansion technique,  which forms the backbone of the \lstinline{"civector"} engine. 
We then move on to an efficient algorithm to evaluate energy gradients with respect to circuit parameters and other implementation details.
\subsection{UCC factor expansion}
\label{sec:ucc-theory}
In this section, we introduce the UCC factor expansion technique used to achieve efficient simulations of UCC circuits in \tcc{}.
Benchmark results are included in the main text. 
We emphasize that the
techniques described in this section are specific for efficiently simulating an ideal circuit \textit{classically}, 
but are not applicable (without large overhead costs) for real hardware or noisy circuit simulations.

\subsubsection{Traditional approach.}
Recall that the general UCC ansatz can be written as $\ket{\Psi} = \prod_{k=\nex{}}^1 e^{\theta_k G_k} \ket{\phi}$.
To compile each UCC factor $e^{\theta_k G_k}$ into a quantum circuit, 
the traditional approach is to transform
$G_k$ (in the remainder of this section we will refer to $G_k$ simply as $G$) into commutable Pauli strings,
and then simulate each Pauli string via Hamiltonian evolution.
Another approach is the YAB method described in the main text, 
which relies on multi-qubit controlled rotations.
The \lstinline{"tensornetwork"} engine follows the YAB method to simulate the UCC circuit.

\subsubsection{Factor expansion approach.}
In the more efficient \lstinline{"civector"} engine, 
we use a different approach to classically simulate the UCC circuit, which involves expanding each UCC factor into a polynomial form, implementable only on a classical computer.
While this formalism has been published multiple times in a variety of contexts~\cite{chen2021quantum, kottmann2021feasible, rubin2021fermionic}, to the best of our knowledge, \tcc{} is the first package to use such a technique for large-scale UCC circuit simulation.

In the most general case, $G$ can be written as $G = g - g^\dagger$, with
\begin{equation}\label{eq:general-G}
    g^{i_1 \dots i_m}_{j_1 \dots j_m} = a^\dagger_{i_1} \cdots a^\dagger_{i_m} a_{j_1} \cdots a_{j_m}, \
\end{equation}
where $m$ is the order of the excitation.
$g^\dagger g$ is a projector onto the space that is not annihilated by $g$~\cite{rubin2021fermionic}:
\begin{equation}
    \left (g^{i_1 \dots i_m}_{j_1 \dots j_m} \right )^\dagger g^{i_1 \dots i_m}_{j_1 \dots j_m} = (1-n_{i_1}) \cdots (1 - n_{i_m}) n_{j_1} \cdots n_{j_m}, \
\end{equation}
where $n$ is the occupation number operator.
It follows that $g^\dagger g g^\dagger = g^\dagger$.
Using this equation, together with $gg = g^\dagger g^\dagger = 0$,
it is straightforward to show that $G$ has the property
\begin{equation}
    G^3 = g^\dagger - g = -G.
\end{equation}

The corresponding UCC factor can then be expanded as
\begin{equation}
\label{eq:expanding-ucc}
\begin{aligned}
    e^{\theta G} %& = \sum_{j=0} \frac{(\theta G)^j}{j!} \\
    & = 1 + \theta G + \frac{\theta^2 G^2}{2} - \frac{\theta^3 G}{3!} 
    - \frac{\theta^4 G^2}{4!} + \frac{\theta^5 G}{5!} + \cdots \\
    & = 1 + G^2 + \sum_{j=0} \frac{(-1)^j \theta^{2j+1}}{(2j+1)!} G  
    - \sum_{j=0} \frac{(-1)^j \theta^{2j}}{(2j)!} G^2  \\
    & = 1 + \sin{\theta}G + ( 1 - \cos{\theta} ) G^2 
\end{aligned}
\end{equation}
In the special case of $G^2=-I$, the famous formula $e^{\theta G}=\cos\theta + \sin\theta G$ is recovered.
Supposing $\ket{\psi}$ is any intermediate state during circuit execution,
multiplying  $\ket{\psi}$  with Eq.~\ref{eq:expanding-ucc} yields
\begin{equation}\label{eq:ucc-poly}
    e^{\theta G} \ket{\psi} = \ket{\psi} + \sin{\theta} G \ket{\psi} + ( 1 - \cos{\theta} ) G^2  \ket{\psi}.
\end{equation}
Thus, to evaluate $e^{\theta G} \ket{\psi}$ it is sufficient to evaluate  $ G \ket{\psi} $ and $ G^2 \ket{\psi} $.

The advantages of using Eq.~\ref{eq:ucc-poly} compared to the traditional approaches are twofold.
First, Eq.~\ref{eq:expanding-ucc} significantly saves on the number of matrix multiplications required to simulate $e^{\theta G} \ket{\psi}$.
Second, as both $G$ and $G^2$ conserve particle numbers in the up and down spin sectors,
{\tcc} is able to store $\ket{\psi}$ in the particle-number conserving space instead of the whole Fock space,
just as in the standard FCI calculation.
Thus, in the following, we use configuration interaction space 
to denote this particle-number conserving sector of the Fock space. 
In traditional UCC circuit simulations,  where each quantum gate is realized by matrix multiplication,
storing wavefunction in configuration interaction space
is not possible because a single $H$ gate or $X$ gate is able to destroy the particle number conserving property.

Representing the quantum state in configuration interaction space greatly reduces the memory requirements for UCC simulation.
A closed-shell molecule with $N$ spatial orbitals and $M$ electrons has Fock space of dimension $2^{2N}$.
In contrast, the number of possible configurations in each spin sector is $C^{N}_{M/2}$, and the dimension of its configuration interaction space is only $\left ( C^{N}_{M/2} \right)^2$ .
In addition, because $G$ and $\ket{\phi}$ are both real, $\ket{\psi}$ can be represented by real numbers rather than complex numbers in classical computers. 
In Table~\ref{tab:ci-memory}, we list the memory requirements to represent wavefunctions in configuration interaction spaces and Fock spaces for several values of $N$ and $M$.
While the required memory using configuration interaction space still scales exponentially, in practice the memory saving compared with using Fock space is significant, particularly so when $|N-M|$ is large.

\begin{table}[h]
\caption{\label{tab:ci-memory}
Memory requirements to represent system wavefunctions in configuration interaction space and Fock space.
In configuration interaction space each amplitude is stored by a 64-bit floating-point number
and in Fock space each amplitude is stored by a 128-bit complex number.
}
\begin{tabular}{ccrrrrc}
\hline
\multirow{2}{*}{$N$}  & \multirow{2}{*}{$M$} & \multicolumn{2}{c}{Configuration Interaction Space} & \multicolumn{2}{c}{Fock Space}     & \multirow{2}{*}{Memory Saving}    \\
                    &                    & \multicolumn{1}{c}{Dimension} & \multicolumn{1}{c}{Memory}  & \multicolumn{1}{c}{Dimension} & \multicolumn{1}{c}{Memory} & \\ 
                    \hline
2 & 2 & 4 & 32 B & 16 & 256 B & 8x \\
\multirow{2}{*}{4}  & 2    &   16    &   128 B & \multirow{2}{*}{256}  &  \multirow{2}{*}{4.1 kB} &  32x   \\
                    & 4    &   36     &   288 B   &        &    &   14x  \\ 
\multirow{2}{*}{8}  & 4    &   784    &   6.3 kB & \multirow{2}{*}{65,536}  &  \multirow{2}{*}{1.0 MB} &  167x   \\
                    & 8    &   4,900     &   39.2 kB   &        &    &   27x  \\ 
\multirow{2}{*}{16}  & 8    &   3,312,400 &   26.5 MB & \multirow{2}{*}{4,294,967,296}  &  \multirow{2}{*}{68.7 GB} &  2,593x   \\
                    & 16    &   165,636,900  &   1.3 GB   &        &    &   52x  \\ 
48 & 4 & 1,272,384 & 10.1 MB &$ 7.9 \times 10^{28}$ & 1.3 QB &  $1.2\times 10^{23}$x \\
\hline
\end{tabular}
\end{table}

\subsection{Gradients with respect to circuit parameters}
\label{sec:ucc-gradient}
\tcc{} uses an efficient algorithm to calculate the energy gradient with respect to the parameters~\cite{luo2020yao}.
The algorithm is applicable to the \lstinline{"civector"} engine, and for the \lstinline{"tensornetwork"} engine traditional auto-differentiation is implemented.
For $j=1,2,\ldots, N_{ex}$ define
\begin{equation}
    \ket{\psi_j} = \prod_{k=j}^1 e^{\theta_k G_k} \ket{\phi}, \
\end{equation}
and  $\ket{\psi'_j}$
\begin{equation}
    \langle \psi'_j |  = \langle \psi | \hat H \prod_{k=\nex}^{j} e^{\theta_k G_k}, \
\end{equation}
where $H$ is the system Hamiltonian.
The energy expectation value can be written as
\begin{equation}
    \langle E \rangle = \langle \psi'_{j+1} | \psi_j \rangle, \
\end{equation}
and the energy gradient as
\begin{equation}
\label{eq:grad-numerical}
        \frac{\partial \langle E \rangle}{\partial \theta_j} =
    2\langle \psi'_{j+1} |  G_j | \psi_j \rangle. \
\end{equation}

Once $| \psi \rangle$ is obtained, $\langle \psi^{(1)}_{\nex+1} |=\langle \psi | \hat H$
and $| \psi_{\nex} \rangle = |\psi \rangle$
are calculated. The remaining $\langle \psi'_j |$ and $| \psi_{j-1} \rangle$ are then obtained by the recurrence relation
\begin{equation}
    \begin{aligned}
    \langle \psi'_j | & = \langle \psi'_{j+1} | e^{\theta_j G_j}, \ \\
    | \psi_{j-1} \rangle & = e^{-\theta_{j} G_{j}}  | \psi_{j} \rangle, \
    \end{aligned}
\end{equation}
and, for each $\langle \psi'_{j} |$ and $| \psi_{j-1} \rangle$ pair,
$\frac{\partial \langle E \rangle}{\partial \theta_j}$ is evaluated by Eq.~\ref{eq:grad-numerical}.
This algorithm thus computes all gradients in one iteration over $j$, 
requiring only a constant amount of memory.

\subsection{Other techniques for efficient simulation}

With UCC factor expansion and efficient gradient computation, \tcc{} is able to simulate much larger molecular systems compared to traditional simulation packages. 
% In general, the computational cost is essentially at the same level as FCI calculations \jon{(Comment) any justification for this claim? Isn't FCI very expensive? If we don't beat FCI why not just use FCI?}.
\tcc{} also use several other techniques to accelerate calculations. In particular (i)
  initial values of $\theta_k$ are set to the corresponding $t_2$ amplitudes obtained by MP2, to enable faster convergence; (ii)
%For closed-shell molecules, configurations with spin-flip symmetry look, when ask also ask ci space
double excitation operators with $t_2$ very close to zero are screened out by default
because the excitation is likely prohibited by molecular point-group symmetry~\cite{cao2022progress}; (iii)
double excitation operators are also sorted by $t_2$ amplitudes 
to avoid ambiguous ans\"{a}tze~\cite{grimsley2019trotterized}.

\bibliography{refs}

\end{document}


%\bibliographystyle{achemso}
\bibliography{ref}

\end{document}