\documentclass[12pt]{article}
\sloppy


\usepackage{newpxtext}
\usepackage{amsthm,amsmath,amssymb}
\usepackage[normalem]{ulem}
\usepackage[bookmarks=true,hypertexnames=false,pagebackref]{hyperref}
\hypersetup{colorlinks=true, citecolor=blue, linkcolor=red, urlcolor=blue}
\usepackage{thm-restate}
\usepackage{cleveref}
\usepackage{fullpage}
\usepackage{bm}
\usepackage[ruled,linesnumbered]{algorithm2e}
\usepackage{enumerate}
\usepackage{xcolor}

\newtheorem{theorem}{Theorem}[section]
\newtheorem{claim}[theorem]{Claim}
\newtheorem{fact}[theorem]{Fact}
\newtheorem{corollary}[theorem]{Corollary}
\newtheorem{lemma}[theorem]{Lemma}
\newtheorem{definition}[theorem]{Definition}
\newtheorem{conjecture}[theorem]{Conjecture}
\newtheorem{problem}[theorem]{Problem}
\newtheorem{example}[theorem]{Example}
\newtheorem{observation}[theorem]{Observation}


\def\E{\mathbb{E}}
\def\bits{\{0,1\}}
\newcommand{\F}[1]{\mathcal{F}^{#1}}
\newcommand{\Fsize}[2]{\mathcal{F}_{\text{size}}^{#1}[#2]}
\newcommand{\disprob}[2]{\mu_{\text{prob}}^{#1}[#2]}
\newcommand{\dissize}[2]{\mu_{\text{size}}^{#1}[#2]}
\newcommand{\s}{\mathfrak{s}}

\newcommand{\cF}{\mathcal{F}}
\newcommand{\rv}[1]{\bm{#1}}    % random variable
\newcommand{\rX}{\rv{X}}
\newcommand{\rY}{\rv{Y}}
\newcommand{\rx}{\rv{x}}
\newcommand{\ry}{\rv{y}}
\newcommand{\rz}{\rv{z}}
\newcommand{\rW}{\rv{W}}
\newcommand{\rj}{\rv{j}}
\newcommand{\rd}{\rv{d}}
\newcommand{\rU}{\rv{U}}
\newcommand{\re}{\rv{e}}
\newcommand{\rV}{\rv{V}}
\newcommand{\rR}{\rv{R}}
\newcommand{\rb}{\rv{b}}
\newcommand{\rS}{\rv{S}}
\newcommand{\rA}{\rv{A}}
\newcommand{\rF}{\rv{F}}
\newcommand{\rN}{\rv{N}}
\newcommand{\rpi}{\rv{\pi}}
\newcommand{\rlambda}{\rv{\lambda}}
\newcommand{\rSigma}{\rv{\Sigma}}
\def\locs{\text{locs}}
\def\next{\text{next}}
\newcommand{\cT}{\mathcal{T}}
\newcommand{\val}{\text{val}}
\newcommand{\length}{\text{length}}
\newcommand{\unif}{\text{Uniform}}
\newcommand{\distinct}{\text{Distinct}}
\newcommand{\planted}{\text{Planted}}
\newcommand{\stoch}{\text{Stochastic}}
\newcommand{\set}{\text{Sets}}
\newcommand{\size}{\text{Sizes}}
\newcommand{\embed}[1]{\text{Embed}[#1]}
\newcommand{\alg}{\mathcal{ALG}}
\newcommand{\one}[1]{\textbf{1}[#1]}
\newcommand{\cH}{\mathcal{H}}
\newcommand{\tr}{\operatorname{Tr}}
\newcommand{\ip}[2]{\langle #1 , #2\rangle}
\newcommand{\bigip}[2]{\bigl\langle #1, #2 \bigr\rangle}
\newcommand{\Bigip}[2]{\Bigl\langle #1, #2 \Bigr\rangle}
\newcommand{\biggip}[2]{\biggl\langle #1, #2 \biggr\rangle}
\newcommand{\Biggip}[2]{\Biggl\langle #1, #2 \Biggr\rangle}

\newcommand{\Op}{\operatorname}
\newcommand{\wt}[1]{\widetilde{#1}}
\newcommand{\wh}[1]{\widehat{#1}}

\newcommand{\fid}[2]{\operatorname{F}(#1,#2)}
\newcommand{\h}{\mathcal{H}}
\newcommand{\norm}[1]{\left\lVert\tinyspace#1\tinyspace\right\rVert}
\newcommand{\bra}[1]{\ensuremath{\left\langle#1\right|}}
\newcommand{\ket}[1]{\ensuremath{\left|#1\right\rangle}}
\newcommand{\op}[2]{|#1\rangle \langle #2|}
\newcommand{\Tr}{\mathop{\mathrm{tr}}\nolimits}
\newcommand{\bracket}[2]
	{\ensuremath{\left\langle#1 \vphantom{#2}
	\right| \left. #2 \vphantom{#1}\right\rangle}}
\def\eps{\epsilon}

\newcommand{\restate}[2]{\medskip \noindent {\bf #1 (restated).} {\sl #2}}
\newcommand{\restatethm}[3]{\medskip \noindent {\bf #1} (#2). \textsl{#3}\\}


\title{Learning marginals suffices!}

\begin{document}
\author{
Nengkun Yu
\thanks{Computer Science Department,
 Stony Brook University}
\and
Tzu-Chieh Wei
\thanks{CN Yang Institute for Theoretical Physics and Department of Physics and Astronomy,
 Stony Brook University}
}
\maketitle

\begin{abstract}
Beyond computer science, quantum complexity theory can potentially revolutionize multiple branches of physics, ranging from quantum many-body systems to quantum field theory. In this paper, we investigate the relationship between the sample complexity of learning a quantum state and the circuit complexity of the state. The circuit complexity of a quantum state refers to the minimum depth of the quantum circuit necessary to implement it. We show that learning its marginals for the quantum state with low circuit complexity suffices for state tomography, thus breaking the exponential barrier of the sample complexity for quantum state tomography. Our proof is elementary and overcomes difficulties characterizing short-range entanglement by bridging quantum circuit complexity and ground states of gapped local Hamiltonians. Our result, for example, settles the quantum circuit complexity of the multi-qubit GHZ state exactly.

\end{abstract}

\newpage
\section{Introduction}
\label{sec:intro}
The complexity of the physical systems provides an additional dimension to our understanding of the world~\cite{PhysRevLett.50.1946,PhysRevLett.43.1754}. The computational complexity theory offers a tool for studying the computational power of quantum computers and the resources required to solve computational problems. It also grants a new perspective on the nature of physical systems and how they function in quantum many-body physics. The circuit model offers a convenient way to quantify the complexity of a quantum state, which is the minimum depth of a quantum circuit that produces the state.


The overwhelming majority of quantum states have complexities close to the maximum possible value through a counting argument~\cite{Nielsen:2011:QCQ:1972505}. However, despite being a long-standing open problem in quantum information theory~\cite{NLTS,https://doi.org/10.48550/arxiv.2206.13228}, it is highly challenging to establish a lower bound on quantum complexity for a given state because of the potential cancellation of gates~\cite{Haferkamp_2022}, and quantum entanglement~\cite{RevModPhys.81.865}. Canceling gates means that gates applied early in a quantum circuit can partially cancel out the effect of the gates afterward. Quantum entanglement is a phenomenon that describes the correlation between subsystems. The presence of entanglement poses considerable challenges to the powerful tools of characterizing, simulating, and manipulating quantum systems~\cite{Shi_2006}, such as quantum state tomography. This technique allows the reconstruction of the quantum state of a system from measurements and is one of the most indispensable tools for the development and verification of quantum technology. The sample complexity of quantum state tomography describes the protocol's efficiency, which refers to the number of measurements or samples required to reconstruct a quantum state using tomography techniques accurately~\cite{BBMR04,Keyl06,GJK08,FlammiaGrossLiuEtAl2012,KRT14,HHJ+16,https://doi.org/10.48550/arxiv.2206.05265,compressed,Flammia_2012,Gu__2020,https://doi.org/10.48550/arxiv.2009.04610,doi:10.1137/1.9781611977554.ch47}. 
The sample complexity generically grows exponentially as the number of qubits increases~\cite{Holevo73,HHJ+16,OW16,OW17}. 

We are currently in the Noisy Intermediate-Scale Quantum (NISQ) era of quantum computing. The available intermediate-scale quantum devices have limited coherence times, which makes it challenging to execute quantum circuits with considerable depth. Certifying the performance and accuracy of various near-term applications of quantum computers has become a critical challenge in quantum technologies. In the NISQ era, we are most likely facing quantum states as the output states of shallow quantum circuits. Intuitively, the description of a shallow quantum circuit only requires polynomial parameters. Therefore, it might be a manageable amount of information for a complete characterization.
Nevertheless, there is an exponential bottleneck in resource consumption in quantum state tomography. Will this obstacle prevent us from learning quantum systems in principle? The answer so far is indecisive, even for circuits with small depth. On the one hand, we can rigorously verify the intuition of efficient learning for a depth $1$ quantum circuit because the output state is the tensor product of two-qubit states. On the other hand, the output state dramatically and quickly becomes unfathomable even for the quantum circuit of depth $2$, $3$ or $4$~\cite{https://doi.org/10.48550/arxiv.quant-ph/0205133}.

%\newpage
This paper breaks the exponential barrier of quantum state tomography for quantum states with low circuit complexity. We prove that the output state of a depth $D$ circuit, on general interaction graphs, is uniquely determined by its $2^D$-local reduced density matrices. Employing the geometrical locality can improve the bound to $2D$ on a 1-dimensional chain, and $\gamma(D)$~\footnote{We leave the definition of $\gamma_2(D)$ in Section 4.} on the square lattices with a combinatorial function $\gamma_2(D)\leq D^2+(D+1)^2$. This dependence aligns well with a light cone intuition. Our result is robust against perturbation in the following sense: any quantum state $\rho$, which has similar local reduced density matrices to those of a low-complexity quantum state $\ket{\psi}$, must be close to $\ket{\psi}$. In other words, we can treat the set of reduced density matrices as a robust fingerprint of quantum states with low complexity. Furthermore, we can test whether a given quantum state is close to some low-complexity state or far from any low-complexity state.
The strength of our results
is that it relies on no assumption other than state complexity.

The intuition behind our findings originates from the observation which connects perhaps the two most important classes of quantum states: (1) the ground states of local Hamiltonian and (2) the output states of quantum circuits. The output state of a shallow circuit is always the unique ground state of a local frustration-free Hamiltonian. Moreover, this local frustration-free Hamiltonian always has a unique ground state and a constant lower bound on the spectral gap. We further characterize the set of unique ground states of local Hamiltonians as quantum states uniquely determined by their local reduced density matrices among all mixed states (UDA).~\footnote{This is not true if we use "among all pure states" (UDP), instead of "among all mixed states."} This determination is robust against both statistical fluctuations from measurements and perturbation in the Hamiltonian as long as the gap is maintained throughout an entire short-range entangled phase of matter.



 Our findings justify  quantum state tomography from the viewpoint  of reduced density matrices ~\cite{PhysRevLett.89.277906,PhysRevLett.89.207901,Xin_2017,Cramer_2010},   showing that it is precise and reliable against statistical fluctuations. Our results also concur with the recent active research of the {quantum overlapping tomography}  \cite{Cotler_2020,https://doi.org/10.48550/arxiv.2009.04610}, where the goal is to output the classical representation of local reduced density matrices rather than the entire quantum state, as well as classical shadow tomography~\cite{10.1145/3188745.3188802,Huang_2020,Huang_2022,evans2019scalable,10.1145/3406325.3451109}, which aims to predict functions of a quantum state using only a logarithmic number of measurements. 





As an application, our result also leads to a lower-bound method of the quantum state complexity. For example, the circuit complexity of 
$n$-qubit GHZ state is at least $\lceil \log n \rceil $ on general interaction graphs. This can be improved to $\lceil \frac{n}{2} \rceil $ on $1$-D chain, and $\max\limits_{D: \gamma_2(D)\leq n}\lceil D \rceil $ on square lattice. Interestingly, these are the exact complexity of generating $n$-qubit GHZ state. Our result also provides a lower-bound method for the complexity of the unitary through the correspondence between Choi states and unitaries, where the complexity of a unitary is the smallest number of the circuit depth among all the circuits implementing the unitary.
\newpage

\section{Preliminaries}

\subsection{Basic quantum mechanics}
An isolated physical system is associated with a
Hilbert space, called the {\it state space}. A {\it pure state} of a
quantum system is a normalized vector in its state space, denoted by the Dirac notation $\ket{\varphi}$. A
{\it mixed state} is represented by a density operator on the state
space. Here, a density operator $\rho$ on $d$-dimensional Hilbert space $\cH$ is a
semi-definite positive linear operator such that $\tr(\rho)=1$.

The state space of a composed quantum system is the tensor product of the state spaces of its component systems.
A Hilbert space $\bigotimes_{k=1}^{n}\cH_{k}$ is the tensor product of Hilbert spaces $\cH_{k}$.
The quantum state on the multipartite system $\bigotimes_{k=1}^{n}\cH_{k}$ is a semi-definite positive linear operator such that $\tr(\rho)=1$.


\subsection{Local Hamiltonian}

A Hamiltonian is an operator representing a quantum system's total energy in quantum mechanics. A local Hamiltonian refers to a Hamiltonian operator expressed as a sum of terms, where each term involves only a few qubits in the system. More precisely, a $k$ local Hamiltonian of $n$ qubit system is of form
\begin{align*}
H=\sum_s H_s\otimes I_{\bar{s}}
\end{align*}
where $s$ may ranges over all subset of $\{0,\cdots,n-1\}$ with $|s|\leq k$ and $0\leq H_s \leq I_s$ is an operator acting on the sub-system $s$ (whose complement is denoted as $\bar{s}$). 
We call the set of $s$'s as the interaction graph.
For simplicity, we will also use 
\begin{align*}
H=\sum_s H_s,
\end{align*}
without explicitly writing out $I_{\bar{s}}$ in each term.

The ground energy of $H$ is the smallest eigenvalue of $H$. Ground space is the linear subspace spanned by the eigenvectors corresponding to the smallest eigenvalue. We call the normalized eigenvector a unique ground state if the ground space has only one dimension. The spectrum gap of Hamiltonian $H$ is the difference between the smallest eigenvalue and the second smallest eigenvalue.

A local Hamiltonian is called frustration-free if its ground energy is 0. The ground space of a local Hamiltonian $H=\sum_s H_s$ is the intersection of the null space of each local term $H_s$.

%\newpage
\subsection{Quantum Circuits}

The most popular and well-established model of quantum computing is the quantum circuit model, where quantum algorithms consist of a sequence of quantum gates applied to a set of qubits. In the \textit{quantum circuit model}, a quantum program consists of an instruction sequence $U_1 \cdots U_k$ and operates on an $n$ qubit register. For $1\leq i \leq k$, $U_i=\otimes_j V_{i,j}$ is the tensor product of two-qubit unitaries $V_{i,j}$ such that $V_{i,j_1}$ and $V_{i,j_2}$ applies on different qubits when $j_1\neq j_2$.


We take the initial state as $\ket{0}^{\otimes n}$. The meaning of the circuit is the matrix product $U_{k} \cdots U_{1}$ applies on the initial state  $\ket{0}^{\otimes n}$. We can regard $U_{i}$ as
a unitary matrix that applies to the entire $n$ qubit register. It is difficult to determine a quantum program's behavior using classical methods due to the state space's exponential size. This fact is known as the ``exponential wall" in quantum computing and is one of the main challenges in quantum program verification.

Quantum circuit depth refers to the number of sequential layers of quantum gates required to implement a quantum algorithm or operation on a quantum computer. Each layer consists of a set of quantum gates acting in different qubits. The depth of a quantum circuit is an essential metric for measuring the efficiency of a quantum algorithm, as it can impact the time required to execute the algorithm on a quantum computer.


\subsection{Uniquely determined (UD) by reduced density matrices}
Quantum state tomography via reduced
density matrices is an especially
promising resource-saving approach. It requires
the global state
has to be the only state compatible with its reduced density matrices. In other words,
it must be uniquely determined (UD) by its reduced density matrices \cite{PhysRevLett.89.277906,PhysRevLett.89.207901,Xin_2017}.



The UD criterion can be further classified into two categories: uniquely determined among all states (UDA) and
uniquely determined among pure states by local
reduced density matrices~\cite{PhysRevA.88.012109}.

A quantum state $\psi=\op{\psi}{\psi}$ is UDP by its reduced density matrices $\psi_{s_i}$ (supported on a set $s_i$ of sites, each with a finite Hilbert space) for $1\leq i\leq m$ if
\begin{align*}
\phi=\op{\phi}{\phi},\ \&  \ \phi_{s_i}=\psi_{s_i}, \forall 1\leq i\leq m\  \Longrightarrow \  \phi=\psi.
\end{align*}

A quantum state $\psi=\op{\psi}{\psi}$ is UDA by its reduced density matrices $\psi_{s_i}$ for $1\leq i\leq m$ if
\begin{align*}
 \rho_{s_i}=\psi_{s_i}, \forall 1\leq i\leq m\  \Longrightarrow \  \rho=\psi.
\end{align*}

In other words, UDP means no other pure state shares the same set of reduced density matrices;  UDA means no other state shares the same set of reduced density matrices.
It is important to note that UDP does not imply UDA~\cite{Xin_2017}.

\section{A characterization of unique ground state}

In this subsection, we prove the following characterization of the unique ground state of the local Hamiltonian.

\begin{theorem}\label{Hamiltonian}
A quantum state $\ket{\psi}$ is the unique ground state of some $k$-local Hamiltonian if and only if it is UDA by its $k$-local reduced density matrices on the interaction graph of the Hamiltonian.
\end{theorem}
\begin{proof}
Let $G=\{s_1,\cdots,s_m\}$ be the interaction graph of a $k$-local Hamiltonian $H=\sum_{s_i} H_{s_i}$.
We denote the $k$-reduced density matrices of $\ket{\psi}$ as $\psi_{s_i}$ for $1\leq i\leq m$.
If $\ket{\psi}$ is the unique ground state of $H$, then
\begin{align*}
\tr(H \psi)=\sum_{s_i} \tr(H_{s_i} \psi_{s_i})
\end{align*}
is the ground-state energy.
For $\rho=\sum_{j} p_j \op{\phi_j}{\phi_j}$ such that
\begin{align*}
\rho_{s_i}=\psi_{s_i}, \ \ \ \forall 1\leq i\leq m,
\end{align*}
we have ${\rm Tr}(H\rho)={\rm Tr}(H\psi)$, i.e., $\rho$ has the same energy as the ground-state energy. Because of the uniqueness of the ground state, this further implies that $\rho=\psi$.

In other words, the ground state $\ket{\psi}$ is UDA by its $k$-local reduced density matrices.

To prove the other direction, we begin with $\ket{\psi}$, which is UDA by its $k$-local reduced density matrices.  We will show the existence of a $k$-local Hamiltonian $H$ such that $\ket{\psi}$ is its unique ground state.

Let us look at the set of all $n$-qubit mixed states 
\begin{align*}
\mathcal{D}=\{\rho| \rho^\dagger=\rho,\,\rho\geq 0,\ \tr \rho=1\},
\end{align*}
and a linear map $L$ which maps the set $\mathcal{D}$ into the set of tuples of reduced density matrices
\begin{align*}
\mathcal{R}=\{(\sigma_{s_1},\cdots,\sigma_{m})| \forall \sigma\in \mathcal{D}\}.
\end{align*}
We note that $\mathcal{R}$ is convex.  

To construct a $k$-local Hamiltonian $H$ with $\ket{\psi}$ being its unique ground state, we first prove $(\psi_{s_1},\cdots,\psi_{m})$
is an extreme point of $\mathcal{R}$.
For any two points $(\sigma_{s_1},\cdots,\sigma_{s_{m}})$ and $ (\tau_{s_1},\cdots,\tau_{s_{m}})\in \mathcal{R}$ such that
\begin{align*}
\frac{(\sigma_{s_1},\cdots,\sigma_{s_{m}}) + (\tau_{s_1},\cdots,\tau_{s_{m}})}{2}=(\psi_{s_1},\cdots,\psi_{s_{m}}),
\end{align*}
we first examine  pre-images of $(\sigma_{s_1},\cdots,\sigma_{s_{m}})$ and $(\tau_{s_1},\cdots,\tau_{s_{m}})$, denoted as $\sigma$ and $\tau$, respectively.
We know that $\frac{\sigma+\tau}{2}\in \mathcal{D}$, and it shares the same set of $k$-local reduced density matrices with $\psi$, as the map $L$ is linear. 
Because $\ket{\psi}$ is UDA by its $k$-local reduced density matrices, we thus infer that
\begin{align*}
\frac{\sigma+\tau}{2}=\op{\psi}{\psi}.
\end{align*}
Because $\op{\psi}{\psi}$ is an extreme point of $\mathcal{D}$, we conclude that
\begin{align*}
\sigma=\tau=\op{\psi}{\psi}.
\end{align*}
This leads to
\begin{align*}
(\sigma_{s_1},\cdots,\sigma_{s_{m}})=(\tau_{s_1},\cdots,\tau_{s_{m}})=(\psi_{s_1},\cdots,\psi_{s_{m}}).
\end{align*}
Therefore, $(\psi_{s_1},\cdots,\psi_{s_{m}})$ is an extreme point of the convex set $\mathcal{R}$.
There exists a hyperplane that contains $(\psi_{s_1},\cdots,\psi_{s_{m}})$ while keeping all other points in $\mathcal{R}$ at one side on this plane. This hyperplane corresponds to a linear function $f:\mathcal{R}\mapsto \mathbb{R}$ such that
\begin{align*}
&f(\rho_{s_1},\cdots,\rho_{s_{m}})\geq 0, \ \forall (\rho_{s_1},\cdots,\rho_{s_{m}})\in\mathcal{R}\\
&f(\psi_{s_1},\cdots,\psi_{s_{m}})= 0.
\end{align*}
Any linear function on $\mathcal{R}$ is of form
\begin{align*}
f(\rho_{s_1},\cdots,\rho_{s_{m}})=\sum_{s_i} \tr(H_{s_i}\rho_{s_i}),
\end{align*}
for a set of Hermitian operators $\{H_{s_i}\}$. Hence,
$\ket{\psi}$ is the unique ground state of the $k$-local Hamiltonian $H=\sum_{s_i} H_{s_i}$. 

\end{proof}

We remark that the geometry of reduced-density matrices has been extensively studied in previous works. From the geometric picture,  the authors observed that, for an interacting spin system, ``the most extreme points in the convex set of reduced density operators uniquely characterize a state'' in~\cite{Verstraete_2006}. Reference~\cite{https://doi.org/10.48550/arxiv.1606.07422} studied the geometry of reduced density matrices by projecting the set to $\mathbb{R}^3$ from the joint numerical range point of view~\cite{Pucha_a_2011,Gawron_2010}. Theorem~\ref{Hamiltonian} completes the other direction of the observation in \cite{Verstraete_2006}.

Theorem \ref{Hamiltonian} indicates that the tuple of reduced density matrices is a fingerprint of the unique ground state of the local Hamiltonian. Here, we do not explicitly construct $H$, which depends on the detailed structure of $\mathcal{R}$. Moreover, $H$ is generally not frustration-free. The following observation shows this fingerprint is robust for the gapped local Hamiltonian.

\begin{lemma}\label{ground}
Let $\ket{\psi}$ be the unique ground state of a $k$-local Hamiltonian $H$ with gap $\Delta>0$ and interaction graph $G=\{s_1,\cdots,s_m\}$, for any state $\rho$, one of the following conditions must be satisfied:
\begin{enumerate}
\item $||\psi-\rho||_1<\eps$;
\item  $||\psi_{s_i}-\rho_{s_i}||_1>\frac{\Delta \eps^2}{4 {m}}$ for some $i$.
\end{enumerate}
where $\psi\equiv\op{\psi}{\psi}$ denotes the density matrix of state $|\psi\rangle$ and $\psi_{s_i}$ denotes the corresponding reduced density matrix supported on the subsystem $s_i$.
\end{lemma}

\begin{proof} Let the ground energy of $H$ be $\lambda$, then
we have
\begin{align*}
    H\geq \lambda \op{\psi}{\psi}+(\lambda+\Delta)(I-\op{\psi}{\psi})=(\lambda+\Delta)I-\Delta \op{\psi}{\psi}.
\end{align*}
Let us assume that scenario 1 above does not hold. Then by the relation between quantum fidelity and trace distance~\cite{Nielsen:2011:QCQ:1972505}, we know that
\begin{align*}
||\psi-\rho||_1>\eps \Longrightarrow \tr (\op{\psi}{\psi}\rho)\leq 1- \frac{||\psi-\rho||_1^2}{2^2}\leq 1-\frac{\eps^2}{4}.
\end{align*}
We can obtain the following bound
\begin{align*}
\tr(H\rho)\geq \tr [((\lambda+\Delta)I-\Delta \op{\psi}{\psi})\rho]=(\lambda+\Delta)-\Delta \tr (\op{\psi}{\psi}\rho).
\end{align*}
Then 
\begin{align*}
\tr[H(\rho-\psi)]\geq (\lambda+\Delta)-\Delta \tr (\op{\psi}{\psi}\rho)-\lambda=\Delta(1-\tr (\op{\psi}{\psi}\rho))\geq \frac{\Delta \eps^2}{4}.
\end{align*}
Therefore,
\begin{align*}
\tr[H(\rho-\psi)]=\sum_{s_i} \tr[H_{s_i}(\rho_{s_i}-\psi_{s_i})]\geq \frac{\Delta \eps^2}{4}.
\end{align*}
This means that there must exist some $s_i$ such that
\begin{align*}
\tr[H_{s_i}(\rho_{s_i}-\psi_{s_i})]\geq \frac{\Delta \eps^2}{4 m}.
\end{align*}
Since $0\leq H_{s_i}\leq I_{s_i}$, we have
\begin{align*}
||\rho_{s_i}-\psi_{s_i}||_1\geq \tr[H_{s_i}(\rho_{s_i}-\psi_{s_i})]\geq \frac{\Delta \eps^2}{4 m}.
\end{align*}
\end{proof}

We note that a pioneering work \cite{Cramer_2010} studied the MPS tomography by focusing on the frustration-free local Hamiltonian and
obtained a similar bound on the unique ground state. 

 A direct application of this result is the following corollary. 
\begin{corollary}
It is sufficient to perform tomography of all the $k$-local reduced density matrices with precision $\frac{\Delta \eps^2}{4 m}$ for trace distance to determine the unique ground state of some $k$-local Hamiltonian up to precision $\eps$ for trace distance. 
\end{corollary}
We employ a specific  overlapping tomography protocol in \cite{https://doi.org/10.48550/arxiv.2009.04610}, which uses $\mathcal{O}\Big(\frac{10^k \log m}{\delta^2}\Big)$ samples for the tomography of $m$ different $k$-qubit reduced density matrices accurate up to a trace distance parameter $\delta$. Here, we only consider about a successful probability greater than  a constant greater than $1/2$, says $2/3$.

From this, we can obtain different sample complexities based on different settings of our knowledge about the connectivity of the local Hamiltonian by choosing $\delta=\frac{\Delta \eps^2}{4m}$. 

\begin{enumerate}
    \item If we do not assume any knowledge of the connectivity of the local Hamiltonian, but only being $k$ local and a gap, we only know that $m\le {n\choose k}$ and hence $\mathcal{O}\Big(\frac{10^k {n \choose k}^2 \log{n \choose k}}{\Delta^2\eps^4}\Big)$ samples suffice for the unique ground state tomography of a $k$-local Hamiltonian $H$ with gap $\Delta>0$ and error parameter $\eps$. 
    \item If we know $m$ but not the interaction graph $G=\{s_1,\cdots,s_m\}$, we must perform tomography on every $k$-local reduced density matrices with error 
parameter $\delta$. Therefore, $\mathcal{O}\Big(\frac{10^k m^2 \log{n \choose k}}{\Delta^2\eps^4}\Big)$ samples suffice.
\item If we know the interaction graph $G=\{s_1,\cdots,s_m\}$, then $\mathcal{O}\Big(\frac{10^k m^2 \log m}{\Delta^2\eps^4}\Big)$ samples suffice.

\end{enumerate}
It may also happen that by merging several $k$-local terms, the resultant $(k+a)$-local terms may have a fewer number, yielding some improvement, where $(k+a)$ is the locality of the merged terms.

\section{Learning the output state of a shallow quantum circuit}
In this section, we study the output state of a quantum circuit and its corresponding parent Hamiltonian.

Before presenting the result, we define $\gamma_2(D)$ for the square lattice.
\begin{definition}
 $\gamma_2(D)$ denotes the largest cardinality of a set of points obtained from a single point set $S_0=\{(0,0)\}$ of the square lattice in $D$ steps, where at each step $j\leq D$, we could get an $S_i$ by adding at most one neighbor point, if it is not in $S_{i-1}$, for each $p\in S_{i-1}$. 
\end{definition}
Through a simple counting argument, one can find the sequence of $\gamma_2(D)$ is $$2,4,8,16,30,\cdots$$ and $$\gamma_2(D)\leq (D+1)^2+D^2\approx 2D^2.$$
We note that this counting does not consider that no two gates can act on the same point in the same layer; hence, it overestimates $\gamma_2(D)$. The following observation builds the connection between the circuit output and the ground state of a local Hamiltonian.
\begin{lemma}\label{circuit-to-Hamiltonian}
The output state $|\psi_{{D}}\rangle$ of $n$ qubit quantum circuit with depth $D\geq 1$ is the unique ground state of a $k$-local frustration-free Hamiltonian with a gap at least $1$.  Moreover, $k=2^D$, if the gates are not geometrically local; $k=2D$ for a chain; $k=\gamma_2(D)$ for the square lattice. 
\end{lemma}


\begin{proof}
Let us denote the circuit as\,  $\mathcal{U}\equiv U_{D} \cdots U_1$ and its action on the initial $n$-qubit product state $\ket{0}^{\otimes n}$ gives $ |\psi_{{D}}\rangle=\mathcal{U}\ket{0}^{\otimes n}$, with $U_\alpha$ being the $\alpha$-th layer of unitaries that are composed of non-overlapping two-qubit gates. 

The initial state $\ket{0}^{\otimes n}$ is the unique ground state of 
\begin{align*}
H_0=\sum_i H_{0,i},
\end{align*}
with $H_{0,i}=\op{1}{1}_i$, where the subscript $i$ indicates the qubit site. 
The frustration-free Hamiltonian is $1$-local, including $n$ terms, and has a gap of $1$.

We define the Hamiltonian as follows,
\begin{align*}
 {H_{{D}}}=\mathcal{U} H_0 \mathcal{U}^{\dag}=U_{D} \cdots U_1 H_0 U_{1}^{\dag} \cdots U_D^{\dag}=\sum_i U_{D} \cdots U_1 H_{0,i} U_{1}^{\dag} \cdots U_D^{\dag}.
\end{align*}
$ {H_{{D}}}$ shares the same spectrum as $H_0$. Therefore, $ {H_{{D}}}$ is a frustration-free Hamiltonian with the spectral gap of 1 and the ground state $ |\psi_{{D}}\rangle$. 


We observe that each term $H_{D,i}:=U_{D} \cdots U_1 H_{0,i} U_{1}^{\dag} \cdots U_D^{\dag}\geq 0$ is nontrivial on at most $k$ qubits, where
\begin{enumerate}
    \item $k\leq 2^D$, if the gates are not geometrically local;
    \item $k\leq \gamma_2(D)$ on the square  lattice;
    \item $k\leq 2D$ on a 1-dimensional chain.
\end{enumerate}
The above bound on $k$ can be seen from a light cone argument. 


If different local terms are nontrivial on a different set of qubits, $ {H_{{D}}}$ is already the local Hamiltonian
with the spectral gap of 1 and the ground state $ |\psi_{{D}}\rangle$. 
Otherwise, there exist some local terms in $\sum_{j=1}^n H_{D,j}:=U_{D} \cdots U_1 H_{0,j} U_{1}^{\dag} \cdots U_D^{\dag}\geq 0$ which are nontrivial on the same set of qubits. In other words, the interaction graph (i.e., generally a hypergraph) of $H_D$ may contain less than $n$ elements. We divide $\{1,\cdots,n\}=\cup_{j} S_j$ such that $i,l\in S_j$ iff $H_{D,i}$ and $H_{D,l}$ are nontrivial on the same set of qubits, and
 define the revised initial Hamiltonian as follows,
\begin{align*}
\widetilde{H}_0=\sum_j \widetilde{H}_{0,j},
\end{align*}
where $\widetilde{H}_{0,j}=I_{S_j}-\op{0\cdots 0}{0\cdots 0}_{S_j}$   with $0$ on qubits in $S_j$.


Further, we let
\begin{align*}
\widetilde {H}_{{D}}=\mathcal{U} \widetilde{H}_0 \mathcal{U}^{\dag}=U_{D} \cdots U_1 \widetilde{H}_0 U_{1}^{\dag} \cdots U_D^{\dag}=\sum_j U_{D} \cdots U_1 \widetilde{H}_{0,i} U_{1}^{\dag} \cdots U_D^{\dag}.
\end{align*}
$\widetilde {H}_{{D}}$ shares the same spectrum as $\widetilde{H}_0$. Therefore, $\widetilde {H}_{{D}}$ is a frustration-free Hamiltonian with a spectral gap of $1$ and the ground state $ |\psi_{{D}}\rangle$ defined above. 

One can verify that different local terms $\widetilde{H}_{D,j}:=U_{D} \cdots U_1 \widetilde{H}_{0,j} U_{1}^{\dag} \cdots U_D^{\dag}\geq 0$ are nontrivial on a different set of qubits.  Furthermore, for $i\in S_j$, $\widetilde{H}_{D,j}$ and ${H_{D,i}}$ apply nontrivially on the same set of qubits. In other words, the locality of $\widetilde{H}_{D}$ is that same as that of  ${H_{{D}}}$, i.e.,
\begin{enumerate}
    \item $k\leq 2^D$, if the gates are not geometrically local;
    \item $k\leq 2D$ on a 1-dimensional chain;
        \item $k\leq \gamma_2(D)$ on the square lattice.
\end{enumerate}
\end{proof}

According to Lemma \ref{ground}, we have
\begin{theorem}\label{output}
$\ket{\psi}$ has circuit complexity at most $D$. For any state $\rho$, one of the following conditions must be satisfied:
\begin{enumerate}
\item $||\psi-\rho||_1<\eps$;
\item  $||\psi_s-\rho_s||_1>\frac{ \eps^2}{4 n}$ for some $s\subseteq \{0,\cdots, n-1\}$ with $|s|=k$.
\end{enumerate}
where $\psi=\op{\psi}{\psi}$ and we list several scenarios:
\begin{enumerate}
    \item $k= 2^D$, if the gates are not geometrically local;
        \item $k= 2D$ on a 1-dimensional chain;
    \item $k= \gamma_2(D)$ on the square lattice.
\end{enumerate}
\end{theorem}

\subsection{Tomography}
According to Theorem \ref{output}, we can accomplish the tomography of the quantum circuit outcome as long as we know that the circuit is of depth at most $D$.
\begin{theorem}
To accomplish the quantum state tomography for depth-$D$ circuit output with precision $\eps$,
\begin{enumerate}
\item $\mathcal{O}\Big(\frac{n^2\cdot10^{k}\log{n \choose k}}{\eps^4}\Big)$ suffice, if we do not know the circuit structure;

\item $\mathcal{O}\Big(\frac{n^2\cdot 10^{k} \log{n}}{\eps^4}\Big)$ suffice, if we know the circuit structure.
\end{enumerate}
Note that, same as above, $k= 2^D$, if the gates are not geometrically local; $k= \gamma_2(D)$ on the square lattice; $k= 2D$ on a 1-dimensional chain.
\end{theorem}
\begin{proof}
If we do not know the circuit structure, it is sufficient to do tomography on all the $k$-local reduced density matrices with precision $\frac{\eps^2}{4 n}$.

If we know the circuit structure, we can compute the $k$-local Hamiltonian, which has at most $n$ terms and at least $1$ spectral gap. 
It is sufficient to perform the tomography of these corresponding $n$ terms of the $k$-local reduced density matrices with precision $\frac{\eps^2}{4 n}$.
\end{proof}

The sample complexity is $poly(n,\frac{1}{\eps})$ for circuits with depth $D$ and, similarly, we list several scenarios:
\begin{enumerate}
\item $D\leq \log\log n+O(1)$ when gates are not geometrically local;
\item $D\leq O(\sqrt{\log n})$ on the square lattice;
\item $D\leq O({\log n})$ on a 1-dimensional chain.
\end{enumerate}

\subsection{Testing the circuit complexity of states}
Using our results, one can test the circuit complexity of an unknown state. This is manifested in the following theorem.
\begin{theorem}\label{complexity}
For an unknown quantum state $\rho$, and a given $D$, 
 $\mathcal{O}\Big(\frac{n^2\cdot10^{k}\log{n \choose k}}{\eps^4}\Big)$ samples suffice to distinguish between the two cases:
 \begin{enumerate}
 \item $||\psi-\rho||<\frac{\eps^2}{12n}$ for some quantum state $\ket{\psi}$ with circuit complexity $\leq D$;
 \item $||\psi-\rho||>\eps$ for any quantum state $\ket{\psi}$ with circuit complexity $\leq D$;
 \end{enumerate}
$k= 2^D$ in the above sample complexity if the gates are not geometrically local; $k= \gamma_2(D)$ on the square lattice; $k= 2D$ on a 1-dimensional chain.
\end{theorem}

\begin{proof}
We prove the correctness of Algorithm \ref{alg:cap}:

\begin{algorithm}
\KwIn{$\rho$ and $D$\;}
\KwOut{``Yes'' if $||\psi-\rho||<\frac{\eps^2}{12n}$ for some quantum state $\ket{\psi}$ with circuit complexity $\leq D$\;
\ \ \ \ \ \ \ \ \ \ \ \ \ \ \ \ \ ``No'' if $||\psi-\rho||>\eps$ for any quantum state $\ket{\psi}$ with complexity $\leq D$;}

Do the overlapping tomography up to precision $\frac{\eps^2}{12n}$ and obtain $\tilde{\rho_{s}}$ for each $|s|=k$\;

Compute a quantum state $\ket{\psi}$ with circuit complexity $\leq D$ such that $||\psi_s-\tilde{\rho_{s}}||_1<\frac{\eps^2}{6n}$ for every $|s|=k$\;

\If {such $\ket{\psi}$ does not exist}
{
Return ``No''\;
}
\Else
{Return ``Yes''.}
\caption{Testing circuit complexity}\label{alg:cap}
\end{algorithm}

If $||\psi-\rho||<\frac{\eps^2}{12n}$ for some quantum state $\ket{\psi}$ with circuit complexity $\leq D$, then
\begin{align*}
||\psi_s-\tilde{\rho_{s}}||\leq ||\psi_s-\rho_s||+||\tilde{\rho_{s}}-\rho_s||<||\psi-\rho||+||\tilde{\rho_{s}}-\rho_s||<\frac{\eps^2}{6n}.
\end{align*}
Step 5 shall find a $\ket{\psi}$, and the algorithm will return ``Yes''.

If $||\psi-\rho||>\eps$ for any quantum state $\ket{\psi}$ with circuit complexity $\leq D$, we need to show that the algorithm shall not find a $\ket{\phi}$ at Step 5. Otherwise, for each $|s|=k$, we have
\begin{align*}
||\phi_s-\rho_{s}||\leq ||\phi_s-\tilde{\rho_s}||+||\tilde{\rho_{s}}-\rho_s||<\frac{\eps^2}{12n}+\frac{\eps^2}{6n}=\frac{\eps^2}{4n}.
\end{align*}
Theorem \ref{output} implies 
\begin{align*}
||\rho-\phi||<\eps,
\end{align*}
which leads to a contradiction!



\end{proof}
It is not hard to see a lower bound of $\Omega(\frac{n^2}{\eps^2})$ samples is needed, e.g., by considering  depth-1 quantum circuits consisting of one-qubit unitaries and by studying the tomography of a tensor product state. The intriguing question is whether the $\frac{1}{\eps^4}$ is necessary. 

We note that the classical computation in Step 2 is not necessarily easy. It would be interesting to know this problem's precise complexity class. We know at least that this problem for low-depth circuits is in QCMA (i.e., Quantum Merlin Arthur with a classical string as the proof) \cite{Watrous2009}. To see this, we only need to use the classical description of a quantum circuit as proof. Then, the verifier can use the overlapping tomography protocol to verify the correctness.

\section{Lower bound of the quantum state complexity}
 In this section, we will employ our results to study the lower bound of the circuit complexity of quantum states.
\begin{definition}
For a quantum state $\ket{\psi}$, its circuit complexity is defined as the minimum depth of quantum circuit $C$ such that
\begin{align*}
\ket{\psi}=C\ket{0}^{\otimes n}.
\end{align*}
\end{definition}

Theorem \ref{Hamiltonian} and Lemma \ref{circuit-to-Hamiltonian} implies the following
\begin{theorem}\label{bound}
If $\ket{\psi}$ is not UDA by its $r$ local reduced density matrices, its circuit complexity is at least: $\log (r+1)$ for non-geometrical circuits, $\lceil \frac{r+1}{2} \rceil $ on $1$-D chain, and $\max\limits_{D:\gamma_2(D)\leq r+1}\lceil D \rceil $ on the square lattice.
\end{theorem}
\begin{proof}
Suppose $\ket{\psi}$ is the output state of a depth $D$ circuit. According to Lemma \ref{circuit-to-Hamiltonian}, $\ket{\psi}$ is the unique ground state of a $k$-local Hamiltonian where $k=2^D$ for non-geometrical circuits, $k=\gamma_2(D)$ for the square lattice and $2D$ for $1$-D chain. Theorem \ref{Hamiltonian} implies that it is UDA by its $k$-local reduced density matrices. That means
\begin{align*}
k>r.
\end{align*}
This proves our statement.
\end{proof}



\subsection{Examples: GHZ state, long-range entangled and short-range  states}
Take the GHZ state $\ket{\psi}=\frac{1}{\sqrt{2}}(\ket{0}^{\otimes n}+\ket{1}^{\otimes n})$ as an example, it is not UDA by its $n-1$ reduced density matrices. To see this, we observe its $n-1$ reduced density matrices are all 
\begin{align*}
\frac{1}{2}(\otimes_{i=1}^{n-1}\op{0}{0}+\otimes_{i=1}^{n-1}\op{1}{1}).
\end{align*}
The following state has the same $n-1$ reduced density matrices as $\ket{\psi}$
\begin{align*}
\frac{1}{2}(\otimes_{i=1}^{n}\op{0}{0}+\otimes_{i=1}^{n}\op{1}{1}).
\end{align*}
Theorem \ref{bound} implies that the circuit complexity of GHZ is at least
$\lceil \log n\rceil$ for non-geometrical circuits, $\lceil \frac{n}{2} \rceil $ on $1$-D chain, and $\max\limits_{D: \gamma_2(D)\leq n}\lceil D \rceil $ on the square lattice.

All these bounds are tight. The upper bound for non-geometrical circuits follows from the following arguments: In the first layer, a single two-qubit gate can generate a Bell state $\frac{1}{\sqrt{2}}\ket{00}+\ket{11}$; In the second layer, we can use each qubit of the Bell state as a control to generate a $4$-qubit GHZ state; and we continue in this fashion to grow the system size.  In the $\lceil \log n \rceil$-th layer, we can use all the previous qubits of the GHZ state to generate $L$-qubit GHZ state for any $L\leq 2^{\lceil \log n \rceil}$.

We obtain the upper bounds of $\lceil \frac{n}{2} \rceil $ on $1$-D chain and $\max\limits_{D: \gamma_2(D)\leq n}\lceil D \rceil $ on the square lattice similarly. One only needs to have the light cone argument to see that the maximally possible size of a light cone is $2D$ on a $1$-D chain and 
$\gamma_2(D)$ on the square lattice, respectively.

On the other hand, if we allow doing quantum measurement during the quantum circuit, we can generate GHZ state more efficiently: 
In \cite{verresen2021efficiently}, the authors provide a protocol to generate $n$-qubit GHZ state. In the first step, they use a depth 2 circuit to create a graph state on a circle of $2n$. In the second step, they perform single qubit measurements on half of them. The last action corrects the potential phase change according to the measurement outcomes.

The GHZ state has long-range entanglement. There are other long-range entangled states~\cite{chen2010local} that are topologically ordered, such as the toric code~\cite{kitaev2003fault} or string-nets~\cite{levin2005string}. It is known that they cannot be created from a product state by a finite-depth quantum circuit with geometrically local gates~\cite{chen2010local} but can be created with a linear depth in the system size. But this does not lead to any specific useful lower bound on the locality of reduced density matrices that enable their unique determination. It is also known that circuits can create them with $\mathcal{O}(\log(N))$-depth, but with long-range gates, in a reverse real-space renormalization procedure~\cite{aguado2008entanglement,konig2009exact}, where $N$ is the total number of qudits. For example,  Ref.~\cite{aguado2008entanglement} considers disentangling the toric code state and gives a scheme in which an operation of  7 (non-local) CNOT depths can reduce the system size by a factor of 4. This gives that the number of depth $D$ to disentangle all qubits is $D=(7/2)\log_2 N$, leading $k\ge 2^D=N^{(7/2)}$, not a useful lower bound. One would expect reduced density matrices to be proportional to the system size (more precisely, of logical code distance)  to determine which degenerate ground state is produced uniquely. Although from the circuit complexity perspective, one cannot directly lower bound the range of the reduced density matrices so that a long-range entangled state is UDA, it may be more appropriate from the perspective of topological code structure and its distance~\cite{kitaev2003fault}. In particular, we expect that the code distance provides a lower bound on $k$. 

On the other hand, short-range entangled states can be created from geometrically-local gates with a constant depth $D$~\cite{chen2010local} without any symmetry constraint. (Even with symmetry, geometrically non-local gates can disentangle symmetry-protected topologically ordered states with finite-depth circuits~\cite{stephen2022non-local}.) These states can be uniquely determined by their local reduced density matrices with locality $k\ge 2^D$ in the worst-case scenario, assuming 2-local  gates (note geometrically local gates will yield smaller locality $k$). 
Such a property of UDA for short-range entangled states extends to the entire gapped phase (except at the phase boundary), as a finite depth of local gates can connect any  two points inside the same phase. Thus, UDA can be a useful property in short-range gapped phases.


\section{Conclusion and Discussion}
This paper shows that the sample complexity of tomography is low for quantum states with low circuit complexity because learning marginals suffices for state tomography. Our result aligns with the intuition that the lower the complexity of the quantum state, the fewer samples are needed for learning.
The previous exponential lower bound seems to originate from considering the general quantum states with exponential circuit complexity~\cite{holevo-book,Hayashi_1998}. Our findings thus pave the way for studying the relationship between sample complexity of learning and circuit complexity.  
One exciting question is to complete the picture in the intermediate regime, i.e., to determine the sample complexity of quantum state tomography for quantum states with \textit{polynomial} circuit complexity.

From a software perspective, data structures are essential for programmers: which enable efficient data storage and retrieval, algorithm design, resource management, and performance optimization.
Our results provide a promising choice as a data structure for quantum computing. The quantum state vector is the primary data structure used in quantum computing, which exhibits the exponential wall in the cost of classical description. Our work shows that the tuple of reduced density matrices is a potential candidate since it is economical and precise in many essential scenarios, including for shallow circuits most relevant to NISQ devices. Additionally, our lower bound describes the bottleneck of quantum gate synthesis and quantum circuit optimization, which will be beneficial to understand the performance and feasibility of quantum algorithms in comprehensive quantum advantages.

Certification of quantum computation is a timely challenge in quantum technologies because it is essential for developing practical quantum applications, particularly in the NISQ era. Our approach provides a solid theoretical justification and vindication of methods by reduced density matrices. Moreover, potential testing schemes on various NISQ devices suffice to perform only a few local Pauli measurements.



Understanding complex condensed matter systems and facilitating quantum computation relies on the fundamental concept of many-body entanglement. Our characterization of the unique ground state of local Hamiltonians may offer a new perspective to study many-body quantum phases from the perspective of UDA, as we have seen in our discussion of short-range entangled gapped phases. It is also interesting to explore the relationship between quantum phase 
transitions and the geometry of the reduced density matrices~\cite{Verstraete_2006,PhysRevA.93.012309,PhysRevA.106.012434}, as well as to understand the unique determinism of general tensor network states~\cite{Cirac_2021,https://doi.org/10.48550/arxiv.0707.2260}. Moreover, extending our framework to study long-range entangled, topologically ordered states would be desirable. 


We have shown that a state with UDA (of finite range) is a unique ground state of a local Hamiltonian. However, an explicit construction from reduced density matrices is generally nontrivial unless it is a ground state of some frustration-free local Hamiltonian. Even in the latter case, it becomes challenging with statistical fluctuations from measurement. For states that MPS or PEPS can approximately describe, 
one possible approach is to compute a parent Hamiltonian~\cite{https://doi.org/10.48550/arxiv.quant-ph/0608197,Cramer_2010}. It is even more challenging when we want to reconstruct a quantum circuit that has the same/close output state. The reasons include the non-uniqueness of local Hamiltonians and the unclear method of transforming local Hamiltonians into quantum circuits.
 
We end with  potentially exciting directions on extending our results to Hamiltonian with degenerate ground states and to quantum circuits with mid-circuit measurements. The development along these directions may provide a new perspective on topological order. 






\bibliographystyle{alpha}
\bibliography{opt-tomo.bib}

\end{document}
