\section{Conclusion}
% 
In this work, we proposed an efficient Dynamic Multi-scale Voxel Flow Network (DMVFN) that excels previous video prediction methods on dealing with complex motions of different scales.
%
With the designed routing module, our DMVFN adaptively activates different sub-networks based on the inputs, improving the prediction performance while reducing the computation costs.
% 
%
Experiments on diverse benchmark datasets demonstrated that our DMVFN achieves \sota ~performance with greatly reduced computation. We believe DMVFN will provide general insights for long-term prediction, video frame synthesis and representation learning~\cite{hafner2023mastering,ha2018world}. We hope our DMVFN will inspire further research in light-weight video processing and make video prediction more accessible for downstream tasks such as CODEC for streaming video.

Our DMVFN can be improved at two aspects. Firstly, iteratively inferring future frames suffers from accumulate errors. We has shown DMVFN is perceptive to the time interval. This issue may be alleviated by further using explicit temporal modeling~\cite{qvi,TMNet2021,ifrnet,rife}. Secondly, DMVFN simply selects the nodes in a chain network topology, which can be improved by exploring more complex topology. For example, our routing module can be extended to automatically determine the scaling factors for parallel branches~\cite{darts}.
% 

\noindent
\textbf{Acknowledgements}. We sincerely thank Wen Heng for his exploration on neural architecture search at Megvii Research and Tianyuan Zhang for meaningful suggestions. This work is supported in part by the National Natural Science Foundation of China (No. 62002176 and 62176068).