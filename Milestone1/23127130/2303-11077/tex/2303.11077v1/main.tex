\documentclass[aps,twocolumn,groupedaddress,superscriptaddress,amsmath,amssymb,pra]{revtex4-2}
\usepackage{amsmath}
\usepackage{amssymb}
\usepackage{amsthm}
\usepackage{physics}
\usepackage{dsfont}
\usepackage{graphicx}% Include figure files
\usepackage{dcolumn}% Align table columns on decimal point
\usepackage{bm}% bold math
\usepackage{bbold}
\usepackage[usenames,dvipsnames]{color}
\usepackage[colorlinks,citecolor=Blue,linkcolor=Red, urlcolor=Blue]{hyperref}

\hypersetup{breaklinks=true}

\usepackage{tikz}
\usepackage{pgfplots}
\usetikzlibrary{quantikz,angles,quotes}
\usetikzlibrary{decorations.pathmorphing}
\usetikzlibrary{arrows.meta}
\usepackage{braket}
\usepackage[normalem]{ulem}
\usepackage{subcaption}
\usepackage{float}

\section{Complexity Analysis}
\label{sec:complexity_analysis}

{\bf Size bounds.} For a join query $Q$, its hypergraph $H(Q)$ has one node per variable in $Q$ and one hyperedge per relation in $Q$.  Figures~\ref{fig:example_intro_varorder} depicts a query hypergraph.

An edge cover is a subset of (hyper)edges of $H(Q)$ such that each node appears in at least one edge. Edge cover can be formulated as an integer programming problem by assigning to each edge $R_i$ a weight $w_{R_i}$ that can be $1$ if $R_i$ is part of the cover and $0$ otherwise. The size of an edge cover upper bounds the size of the query result, since the Cartesian product of the relations in the cover includes the
query result: $|Q(\db)| \leq |R_1|^{w_{R_1}}\cdot\ldots\cdot|R_n|^{w_{R_n}}$, where the database $\db$ is $(R_1,\ldots,R_n)$. By minimizing the size of the edge cover, we can obtain a lower upper bound on the size of the query result. This bound becomes tight for fractional weights~\cite{AGM:2013}.  Minimizing the sum of the weights thus becomes the objective of a linear program.

\begin{definition}[\cite{AGM:2013}]\label{def:agm}
Given a join query $Q$ over a database $(R_1,\ldots,R_n)$, the {\em fractional edge cover number} $\rho^*(Q)$ is the cost of an optimal solution to the linear program with variables $(w_{R_i})_{i\in[n]}$ representing weights of $(R_i)_{i\in[n]}$:
\begin{flalign*}
\textrm{minimize} &\prod_{i\in[n]} |R_i|^{w_{R_i}}\\
\textrm{subject to} &\sum_{R\textrm{ is relation of } X} w_R \geq 1~~\textrm{for each variable } X \\
&~~~\forall i\in[n]: \omega_{R_i}\geq 0.
\end{flalign*}
\end{definition}

\begin{example}
\em
Consider the triangle query:
\begin{align*}
Q_{\vartriangle} = R(A,B), S(B,C), T(C,A)
\end{align*}
Figure~\ref{fig:triangle_hypergraph_viewtree} gives the hypergraph of $Q_{\vartriangle}$. The linear program is: 
\begin{flalign*}
\textrm{\em minimize} & \quad |R|^{w_{R}} \cdot |S|^{w_{S}} \cdot |T|^{w_{T}} \\ 
\textrm{\em subject to} & \quad
\begin{tabular}[t]{@{}c@{\hspace*{.5em}}c@{\hspace*{.25em}}c@{\hspace*{.25em}}c@{\hspace*{.25em}}c@{\hspace*{.25em}}c@{\hspace*{.25em}}c}
$A:$ & $w_{R}$ & & & $+$& ${w_{T}}$& $\geq 1$ \\
$B:$ & $w_{R}$ & $+$ & ${w_{S}}$ & & & $\geq 1$ \\
$C:$ & & & $w_{S}$ & $+$ & ${w_{T}}$ & $\geq 1$
\end{tabular}
\end{flalign*}
% 
For $|R|=|S|=|T|=N$, setting $w_{R}=w_{S}=w_{T}=1/2$ gives the optimal solution $\rho^*(Q_{\vartriangle}) = N^{3/2}$. Consequently, the query result has $\bigO{N^{3/2}}$ tuples. This bound is tight in the sense that there exist classes of databases for which the result size is at least $\Omega(N^{3/2})$. For the acyclic query $Q$ in Section~\ref{sec:introduction}, setting the weights $1$ to each of the three relations gives $\rho^*(Q)=N^3$ if all relations have size $N$.
\punto
\end{example}

\nop{
Cardinality constraints can be used to lower the size bounds of query results. For instance, if the number of distinct $A$-values in $R(A,B)$ is $k \ll N$, then we can refine  $Q_\vartriangle$ as $R(A,B),S(B,C),T(C,A),U(A)$ with the new size bound $\rho^*(Q_\vartriangle) = N \cdot k$, where $w_{S}=1$ and $w_{U}=1$.

Join selectivities can also be incorporated to obtain a size {\em estimate} (in contrast to an upper bound). For instance, assume the selectivity of the join on $A$ between $R$ and $T$ is very low: $sel(A) = \frac{|R(A,B),T(C,A)|}{|R|\cdot|T|} = \frac{k}{N}$. Then, we consider a relation $U(A,B,C)=R(A,B),T(C,A)$ whose size estimate is $k \cdot N$ and use this as a cardinality constraint to obtain an estimate of $k \cdot N$ for $Q_\vartriangle$'s size since the size of the join of $S$ and $U$ cannot exceed the size of $U$.
}

Similarly to $\rho^*(Q)$, the {\em factorization width} $\fw(Q)$ governs the sizes of the factorized results of a join query $Q$~\cite{Olteanu:FactBounds:2015:TODS}. In a factorized join over a variable order $\omega$, the values of a variable $X$  depend on the tuples of values of its $\mathit{key}(X)$ variables and are independent of the values for other variables. A tight bound on this number is then given by the size of a join query that covers the variables in $\mathit{key}(X)\cup\{X\}$. We denote this restriction of $Q$ by $Q_{\mathit{key}(X)\cup\{X\}}$. An upper bound on the size of the factorization is then given by the maximum over all variables in $\omega$ of their number of values. This can be improved by going over all possible variable orders of $Q$ and taking the minimum upper bound. This is the factorization width of the query.

\begin{definition}\label{def:fw}
Given a join query $Q$, the {\em factorization width} of $Q$ is  $\fw(Q) = \min_{\omega\in\Omega(Q)} \max_{v\in\mathit{vars}(Q)} \rho^*(Q_{\mathit{key}(X)\cup\{X\}})$.
\end{definition}

\begin{example}\em
For acyclic queries $Q$ over relations $R_1,\ldots,R_n$, $\fw(Q)=\max_{i\in[n]}(|R_i|)$, while $\rho^*(Q)$ can be as much as $\prod_{i\in[n]}|R_i|$ as in our running example. Here are examples of restrictions of our natural join $Q$ in Section~\ref{sec:intro_example}: $\mathit{key}(D)\cup\{D\}=\{C,D\}$ is covered by the query restriction $Q_{\{C,D\}}$ that is the relation $T$; $\mathit{key}(C)\cup\{C\}=\{A,C\}$ is covered by the query restriction $Q_{\{A,C\}}$ that is the relation $S$. For the triangle query $Q_\vartriangle$ and variable order $A-B-C$: $\mathit{key}(C)\cup\{C\}=\{A,B,C\}$ is covered by $Q_\vartriangle$, while $\mathit{key}(B)\cup\{B\}=\{A,B\}$ is covered by relation $R$.
\punto
\end{example}

For any join query $Q$, its factorization width is the fractional hypertree width~\cite{Olteanu:FactBounds:2015:TODS}, a parameter that captures tracta\-bility for a host of computational problems~\cite{FAQ:PODS:2016}.

\begin{proposition}
\label{prop:factorization}
Given a join query $Q$, for every database $\db$, the result $Q(\db)$ admits:
\begin{itemize}
\item a flat representation of size $\bigO{\rho^*(Q)}$~{\em\cite{AGM:2013}};
\item a factorized representation of size $\bigO{\fw(Q)}$~{\em\cite{Olteanu:FactBounds:2015:TODS}}.
\end{itemize}

There are classes of databases $\db$ for which the above size bounds are tight. The flat and factorized representations of $Q(\db)$ can be computed worst-case optimally{\em~\cite{Ngo:SIGREC:2013,Olteanu:FactBounds:2015:TODS}}.
\end{proposition}


\subsection{Dynamic Factorization Width}
\label{sec:dynamic_width}

\milos{Doesn't consider other rings (only LR), indicator projections, and factorizable updates}

As in the non-incremental case, different variable orders may lead to wildly different performance of our IVM approach. In this section, we settle the question of which variable orders can best support IVM under updates to a given set of relations and thereby pinpoint the complexity of maintaining query results under updates. This is captured by a novel notion called {\em dynamic factorization width}, which is a refinement of the factorization width.

We first recall the complexities in the non-incremental case. There, we only materialize the root view of a view tree over a variable order with the smallest factorization width, and we thus have the time data complexity $\bigO{\fw(Q)}$ for computing factorized joins~\cite{Olteanu:FactBounds:2015:TODS} and aggregates over them~\cite{BKOZ:PVLDB:2013,FAQ:PODS:2016}; for cofactor matrices over factorized joins, there is an additional $\bigO{m^2}$ factor, since the sizes of these matrices can be quadratic in the number $m$ of variables (features)~\cite{SOC:SIGMOD:2016}. The space complexity is $\bigO{1}$ or $\bigO{m^2}$ to store the aggregate or cofactor matrix in addition to the database (modulo logarithmic factors in the data size for data iterators).

We next discuss the IVM case. Let $Q$ be any join query. For any variable order $\omega \in \Omega(Q)$, let $\tau(\omega)$ be the view tree inferred from $\omega$. This view tree has exactly one leaf for each relation symbol in $Q$.

We consider updates to relations whose relation symbols in $Q$ form a set ${\mathcal{U}}$; a relation may have several relation symbols  if it is involved in self-joins in $Q$, in which case all of them are in ${\mathcal{U}}$. For a relation symbol $R\in{\mathcal{U}}$, let $\Upsilon_{\tau(\omega)}(R)$ be the set of views that are ancestors of the leaf $R$ in $\tau(\omega)$, i.e., it consists of all the views (recursively) defined using $R$. 

The time needed to compute the delta for a view $\VIEW[keys]{V^{@X}_{rels}}$ is upper bounded by that of a join query $Q^{\sf rels}_{\sf keys \cup \{X\}-\sigma(R)}$ over relations in {\sf rels} that cover $X$ and the variables in {\sf keys} but excluding the variables in $R$. The reason for the exclusion is that a single-tuple update to $R$ binds the variables in $R$ to constants. The overall time to compute the deltas of all views in $\Upsilon_{\tau(\omega)}(R)$ is then
\begin{align*}
T(\omega,R) = \sum_{\VIEW[keys]{V^{@X}_{rels}}\in\Upsilon_{\tau(\omega)}(R)} \rho^*(Q^{\sf rels}_{{\sf keys\cup\{x\}}-\sigma(R)}).
\end{align*}

We are now ready to define the dynamic factorization width that captures the time complexity of incremental maintenance of $Q$ under updates to relations in ${\mathcal{U}}$.

\begin{definition}
Given a join query $Q$ and a set of relation symbols ${\mathcal{U}}$ in $Q$. Then, the {\em dynamic factorization width} of $Q$ and ${\mathcal{U}}$ is $\dfw(Q,{\mathcal{U}}) = \min_{\omega\in\Omega(Q)}\max_{R\in\mathcal{U}} T(\omega,R).$
\end{definition}

\begin{theorem}
Given a query $Q$ with $m$ variables, database $\db$, a payload ring $\RING$, and a set of relations ${\mathcal{U}}$ in $\db$. The time complexity of incrementally maintaining the result of $Q$ over the ring $\RING$ under single-tuple updates to relations in ${\mathcal{U}}$ is $\bigO{\dfw(Q,{\mathcal{U}})\cdot T_\RING}$, where $T_\RING$ is $\bigO{1}$ for rings of numbers and $\bigO{m^2}$ for the degree-$1$ matrix ring.
\end{theorem}

\begin{example}\label{ex:time-complexity}
\em
For our query $Q$ in Section~\ref{sec:intro_example} and database $\db$, the (static) factorization width is $\fw(Q)=O(|R|+|S|+|T|)$. Under single-tuple updates to relations in a set ${\mathcal{U}}_1\subseteq\{R,S\}$, the dynamic factorization width is $\dfw(Q,{\mathcal{U}}_1)=1$ since there are no free variables of the views over $R$ or $S$ in the variable order in Figure~\ref{fig:example_intro_varorder}. This means that we can maintain the result of a sum aggregate over $Q$ in $\bigO{1}$ time under ${\mathcal{U}}_1$ updates. The same holds for ${\mathcal{U}}_2\subseteq\{S,T\}$, i.e., $\dfw(Q,{\mathcal{U}}_2)=1$, as supported by the variable order $C-\{ D, A - \{ B, E \}\}$. However, $\dfw(Q,{\mathcal{U}}_3)=\bigO{|\db|}$ for ${\mathcal{U}}_3=\{R,S,T\}$ since there is no variable order without free variables above all three relations and some variable orders have one free variable above at least one of the three relations. Under the variable order in Figure~\ref{fig:example_intro_varorder}, $\dfw(Q,{\mathcal{U}}_3)=\min(|R|,|S|)$.

The triangle query $Q_\vartriangle$ has the (static) factorization width $\fw(Q_\vartriangle)= \rho^*(Q_\vartriangle)$. For any relation $U \in \{ R, S, T \}$, the dynamic factorization width is $\dfw(Q,\{ U \})=1$ as supported by a path variable order that has the variables in $U$ as prefix. We can thus maintain an aggregate over the triangle query in $\bigO{1}$ under single-tuple updates to exactly one of its three relations. For updates to at least two relations ${\mathcal{U}}_4$, $\dfw(Q,{\mathcal{U}}_4)=O(|\db|)$. For instance, assume a variable order $A-B-C$. We need to cover: no variable under updates to $R$; one of the variables $A$ or $B$ under updates to $S$ or $T$ respectively (the case for other permutations of this variable order is analog). Maintenance has thus lower time cost than recomputation.
\punto
\end{example}

We next analyze the space complexity $S(Q)$ of our approach. This is the sum of the sizes of the views in a view tree. The space needed by the keys of a view $\VIEW[keys]{V^{@X}_{rels}}$ is given by the fractional edge cover of a join query built using relation symbols {\sf rels} to cover the variables in {\sf keys}. To obtain the minimum size, we go over all variable orders of $Q$:
\begin{align*}
 S(Q) = \min_{\omega\in\Omega(Q)}\sum_{\VIEW[keys]{V^{@X}_{rels}}\in\tau(\omega)} \rho^*(Q^{\sf rels}_{\sf keys}).
\end{align*}

\begin{theorem}
Given a query $Q$ with $m$ variables, database $\db$, a payload ring $\RING$. The space complexity required by the materialization of a view tree for $Q$ over the ring $\RING$ is $\bigO{S(Q)\cdot T_\RING}$, where $T_\RING$ is $\bigO{1}$ for the sum ring and $\bigO{m^2}$ for the degree-$m$ matrix ring.
\end{theorem}

There are three differences between the formula $S(Q)$ and Definition~\ref{def:fw} of the factorization width $\fw(Q)$: (1) the use of summation vs. maximum, though the gap between them is linear in $m$ and thus independent of the database size; (2) the cover for $S(Q)$ can only use relation symbols of the view; (3) for $S(Q)$, we only need to cover $\sf keys$ and not also the variable at the view as in the case of $\fw(Q)$. The interplay of (2) and (3) can in fact make $S(Q)$ larger than $\fw(Q)$.
For acyclic queries, both complexities are linear if all relations have the same size and $S(Q)$ can be smaller than $\fw(Q)$ in case some relations are asymptotically smaller than others. 
For cyclic queries, however, $S(Q)$ can be larger than $\fw(Q)$. We show this for the triangle query $Q_\vartriangle$ and relations of the same size $N$. Under any variable order, there is a view of size $\bigO{N^2}$, whereas $\fw(Q_\vartriangle)=N^{3/2}$. For instance, for the variable order $A-B-C$, we materialize the view $\VIEW[A,B]{V^{@C}_{ST}} = \VSUM_{C} \VIEW[B,C]{S} \VPROD \VIEW[C,A]{T} \VPROD \VIEW[C]{\VLIFT_{C}}$, which may create $\bigO{N^2}$ pairs $(A,B)$ as we need both $S$ and $T$ to cover the variables $A$ and $B$. To avoid the large intermediate result, we join all three relations at the same time~\cite{Ngo:SIGREC:2013}, so as to cover $(A,B)$ using $R$. That would, however, require recomputation of this 3-way join for each update. This takes $\bigO{N}$ time since only two of the three variables are bound to constants. In contrast, our IVM approach trades off space for time: We need $\bigO{N^2}$ space but then support $\bigO{1}$ updates to one of the three relations (Example~\ref{ex:time-complexity}).

%%%%%%%%%%%%%%%%%%%%%%%%%%%%%%%%%%%%%%%%%%%%

\documentclass[letterpaper,twocolumn,10pt]{quantumarticle}

\pdfoutput=1
\usepackage[utf8]{inputenc}
\usepackage[english]{babel}
\usepackage[T1]{fontenc}
\usepackage{amsmath}

\usepackage{hyperref}

%\usepackage{tikz}
%\usepackage{lipsum}
\usepackage{mathtools}
\usepackage{amssymb}
\usepackage{graphicx}% Include figure files
%\usepackage{dcolumn}% Align table columns on decimal point
\usepackage{bm}% bold math
%\usepackage[mathlines]{lineno}% Enable numbering of text and display math
%\linenumbers\relax % Commence numbering lines
%\usepackage{newtxtext,newtxmath}
%\usepackage{anyfontsize}

\usepackage{subfigure}
\usepackage{array}
\usepackage{multirow}
\newcolumntype{P}[1]{>{\centering\arraybackslash}p{#1}}

\newtheorem{theorem}{Theorem}
\newtheorem{acknowledgement}[theorem]{Acknowledgement}
\newtheorem{algorithm}[theorem]{Algorithm}
\newtheorem{axiom}[theorem]{Axiom}
%\newtheorem{case}[theorem]{Case}
\newtheorem{claim}[theorem]{Claim}
\newtheorem{conclusion}[theorem]{Conclusion}
\newtheorem{condition}[theorem]{Condition}
\newtheorem{conjecture}[theorem]{Conjecture}
\newtheorem{corollary}[theorem]{Corollary}
\newtheorem{criterion}[theorem]{Criterion}
\newtheorem{definition}[theorem]{Definition}
\newtheorem{example}[theorem]{Example}
\newtheorem{exercise}[theorem]{Exercise}
\newtheorem{lemma}[theorem]{Lemma}
\newtheorem{notation}[theorem]{Notation}
\newtheorem{problem}[theorem]{Problem}
\newtheorem{proposition}[theorem]{Proposition}
\newtheorem{remark}[theorem]{Remark}
\newtheorem{solution}[theorem]{Solution}
\newtheorem{summary}[theorem]{Summary}
\newcommand{\qipeb}{QIP$_{\operatorname{EB}}(2)$}

\newenvironment{proof}[1][Proof]{\noindent\textbf{#1.} }{\ \rule{0.5em}{0.5em}}

\newcommand{\ap}[1]{{\color{magenta} AP: #1}}
\newcommand{\sr}[1]{{\color{blue!95!black} SR: #1}}
\newcommand{\vr}[1]{{\color{blue!95!black} VR: #1}}
\newcommand{\mmw}[1]{{\color{purple} MMW: #1}}

\allowdisplaybreaks

\pdfstringdefDisableCommands{\def\eqref#1{(\ref{#1})}}


\begin{document}

%\newcolumntype{P}[1]{>{\centering\arraybackslash}p{#1}}
%\title{Quantum Steering Algorithm for Estimating Fidelity of Separability}
\title{\texorpdfstring{Schr\"odinger}{Schrodinger} as a Quantum Programmer:\newline
Estimating Entanglement via Steering}

%\title{\texorpdfstring{Schr\"odinger}{Schroedinger} as a quantum programmer:
%Estimating entanglement via steering}

\author{Aby Philip}
\orcid{0000-0002-4608-7522}
\author{Soorya Rethinasamy}
\orcid{0000-0002-8849-3748}
\affiliation{School of Applied and Engineering Physics,
Cornell University, Ithaca, New York 14850, USA}

\author{Vincent Russo}
\affiliation{Unitary Fund}

\author{Mark M. Wilde}
\affiliation{School of Electrical and Computer Engineering,
Cornell University, Ithaca, New York 14850, USA}
\affiliation{School of Applied and Engineering Physics,
Cornell University, Ithaca, New York 14850, USA}
\orcid{0000-0002-3916-4462}

%\date{\today}

\begin{abstract}
    Quantifying entanglement is an important task by which the resourcefulness of a quantum state can be measured. Here, we develop a quantum algorithm that tests for and quantifies the separability of a general bipartite state by using the quantum steering effect, the latter initially discovered by Schr\"odinger. Our separability test consists of a distributed quantum computation involving two parties: a computationally limited client, who prepares a purification of the state of interest, and a computationally unbounded server, who tries to steer the reduced systems to a probabilistic ensemble of pure product states. To design a practical algorithm, we replace the role of the server with a combination of parameterized unitary circuits and classical optimization techniques to perform the necessary computation. The result is a variational quantum steering algorithm (VQSA), a modified separability test that is implementable on quantum computers that are available today.
    %This VQSA has an additional interpretation as a distributed variational quantum algorithm (VQA) that can be executed over a quantum network, in which each node is equipped with classical and quantum computers capable of executing a VQA.
    We then simulate our VQSA on noisy quantum simulators and find favorable convergence properties on the examples tested. We also develop semidefinite programs, executable on classical computers, that benchmark the results obtained from our VQSA. Thus, our findings provide a meaningful connection between steering, entanglement, quantum algorithms, and quantum computational complexity theory. They also demonstrate the value of a parameterized mid-circuit measurement in a VQSA.% and represent a first-of-its-kind application for a distributed VQA.
    %Finally, the whole framework  generalizes to the case of multipartite states and entanglement. 
\end{abstract}


\maketitle
\tableofcontents

\section{Introduction}
    Entanglement is a unique feature of quantum mechanics, initially brought to light by Einstein, Podolsky, and Rosen   \cite{Einstein1935}. Many years later, the modern definition of entanglement was given \cite{W89}, which we recall now.
    A bipartite quantum state $\sigma_{AB}$ of two spatially separated systems $A$ and $B$ is separable (unentangled) if it can be written as a probabilistic mixture of product states \cite{W89}:
    \begin{equation}
    \label{eqn:sep-state-informal}
    \sigma_{AB}= \sum_{x \in \mathcal{X}} p(x)\, \psi^{x}_{A} \otimes \phi^{x}_{B}    ,
    \end{equation}
     where $\{p (x) \}_{x\in \mathcal{X}}$ is a probability distribution and $\psi^{x}_{A}$ and $\phi^{x}_{B} $ are pure states. The idea here is that the correlations between~$A$ and $B$ can be fully attributed to a classical, inaccessible random variable with probability distribution $\{p (x)\}_{x\in \mathcal{X}}$.
      
     The definition above is straightforward to write down, but it is a different matter to formulate an algorithm to decide if a general state is separable; in fact, it has been proven to be computationally difficult in a variety of frameworks \cite{G03, Gharibian2010, HMW13, HMW14, GHMW15}. Intuitively, deciding the answer requires performing a search over all possible probabilistic decompositions of the state, and there are too many possibilities to consider. Regardless, determining whether a general state $\rho_{AB}$ is separable or entangled, known as the separability problem, is a fundamental problem of interest relevant to various fields of physics, including condensed matter \cite{RevModPhys.80.517,cramer2011measuring, laflorencie2016quantum},  quantum gravity \cite{Takayanagi_2012,bose2017spin,marletto2017gravitationally,Qi2018,Swingle18}, quantum optics \cite{RevModPhys.92.035005}, and quantum key distribution \cite{Ekert91,PhysRevLett.113.140501}. In quantum information science, entanglement is the core resource in several basic quantum information processing tasks \cite{Ekert91,bennett1992communication,bennett1993teleporting}, making the separability problem essential in this field as well.

    Part of the challenge in using entangled states for various tasks is that they are hard to produce and maintain faithfully on any physical platform. The utility of entangled states drops off dramatically the further they are from being perfectly or maximally entangled. Therefore, assessing the quality of entangled states produced becomes an important task, thus motivating the problem of quantifying entanglement \cite{BDSW96, VPRK97, VP98, Horodecki2009}, in addition to deciding whether entanglement is present. 
    
    
    To check whether a state is entangled and to quantify its entanglement content experimentally, a rudimentary approach employs state tomography to reconstruct the density matrix and check whether the matrix represents a state that is entangled \cite{Home2006, Steffen2006}. However, the computational complexity of this method scales exponentially with the number of qubits, thus prohibiting its use on larger states of interest. With the rapid development of quantum computers of increasing size, it is already infeasible to perform tomography to estimate the density matrices describing the states of these computers. It is even more daunting to address the separability problem using various well-known one-sided entanglement tests  \cite{Peres1996, Horodecki1996, W89a, Doherty2004}. This leaves us to seek out alternative methods for addressing the separability problem, and one forward-thinking direction is to employ a quantum computer to do so \cite{HMW13, HMW14, GHMW15, LRW21}. 

    An approach for addressing the separability problem, which we employ here, involves the quantum steering effect, originally discovered by Schr\"odinger~\cite{Schrodinger1935, Schroedinger1935}. The idea of steering is that if two distant systems are entangled, distinct probabilistic ensembles of states can be prepared on one system by performing distinct measurements on the other system. To describe this phenomenon more precisely, we can employ some elementary notions from quantum mechanics. Let $\psi_{CD}$ be a pure state of two distant quantum systems~$C$ and~$D$, and let $\rho_C = \operatorname{Tr}_D[\psi_{CD}]$ be the reduced state of the system~$C$. Then by performing a measurement on the system $D$, it is possible to realize a probabilistic ensemble $\{(p (z),\psi^z_C)\}_z$ of pure states on the system $C$ that satisfies $\rho_C = \sum_z p (z)\psi^z_C$. Moreover, for each possible probabilistic decomposition of $\rho_C$, a measurement acting on~$D$ can realize this decomposition. Steering has been a topic of interest in recent years, with applications to quantum key distribution \cite{CS17, UCNG20}, quantum optics \cite{PhysRevLett.128.200401, PhysRevA.106.042414}, and the foundations of quantum mechanics~\cite{wittmann2012loophole, PhysRevA.91.012112}.

    As suggested above, we can make a non-trivial link between the separability problem and steering, which offers a quantum mechanical method for approaching the former. To see it, recall that
    a purification of the separable state $\sigma_{AB}$ in \eqref{eqn:sep-state-informal} 
      is a pure state $\varphi_{RAB} $ that satisfies $\operatorname{Tr}_R[\varphi_{RAB}] = \sigma_{AB}$, and consider that one such  choice of the state vector $|\varphi\rangle_{RAB}$ in this case is as follows:
    \begin{equation}
        |\varphi\rangle_{RAB}=\sum_{x\in \mathcal{X}} \sqrt{p (x)}\,|x\rangle_R\otimes |\psi^{x}\rangle_{A} \otimes |\phi^{x}\rangle_{B},
        \label{eq:purif-sep-state}
    \end{equation}
    where $\{ |x\rangle_R\}_{x\in \mathcal{X}}$ is an orthonormal basis. Purifications are not unique, but all other purifications of $\sigma_{AB}$ are related to the one in \eqref{eq:purif-sep-state} by the action of a unitary operation on the reference system $R$ \cite{NC00}.
    By inspecting~\eqref{eq:purif-sep-state}, we see that the systems $A$ and $B$ can be steered into the probabilistic ensemble $\{(p(x),\psi^{x}_{A} \otimes \phi^{x}_{B})\}_{x\in \mathcal{X}}$ of product states by performing the projective measurement $\{|x\rangle\!\langle x|_R\}_{x\in \mathcal{X}}$ on the reference system~$R$ of $\varphi_{RAB}$. This leads to an idea for testing separability in the general case. If purification of a general state $\rho_{AB}$ is available and the state $\rho_{AB}$ is indeed separable, then one can a) try to find the unitary that realizes the purification in~\eqref{eq:purif-sep-state} and b) perform the measurement $\{|x\rangle\!\langle x|_R\}_{x\in \mathcal{X}}$ on the reference system~$R$. After receiving the outcome~$x$,  one can finally test whether the reduced state is a product state.
    
    As we will see in more detail later, the basic idea outlined above is at the heart of our method to test whether a state is separable. Additionally, this approach leads to a quantum algorithm and complexity-theoretic statements for quantifying the amount of entanglement in a state. We thus provide a meaningful connection between steering, entanglement, quantum algorithms, and quantum computational complexity theory, which has not been observed hitherto. 
    
    In this paper, we expand on the abovementioned idea to develop various separability tests using the quantum steering effect. Our separability test for mixed states consists of a distributed quantum computation involving two parties: a computationally unbounded server, called a prover, which can, in principle, perform any quantum computation imaginable, and a computationally limited client, called a verifier, which can perform time-efficient quantum computations (see Figure~\ref{fig:Max_Sep_Fidelity_QIP_EB}). We show in Theorems~\ref{theorem:qip-eb_msf} and~\ref{theorem:global_cost_func} that the acceptance probabilities of our algorithms, in the ideal case, are directly related to a bonafide entanglement measure, the fidelity of separability. We also employ concepts from quantum computational complexity theory \cite{watrous2009complexity, VW15} to understand how difficult this test is to perform. Our second contribution results from a modification of our separability test. In an attempt to design a practical algorithm, we replace the prover with a combination of parameterized unitary circuits and classical optimization techniques to perform the necessary computation.  This results in a variational quantum steering algorithm (VQSA) that approximates the aforementioned separability test (see Figure~\ref{fig:Max_Sep_Fidelity_VQA}). The concept of quantum steering is again at the heart of our VQSA, just like the test for separability that it approximates. Interestingly, we prove that the acceptance probability of both tests is related to an entanglement measure called fidelity of separability \cite{VPRK97, VP98}. We also generalize our separability test and VQSA to the multipartite setting using appropriate definitions of multipartite separability. 
    
    Next, we report the results of simulations of the VQSA on a quantum simulator and find that they show favorable convergence properties. In light of the limited scale and error tolerance of near-term quantum computers, we develop semidefinite programs (SDPs) to approximate the fidelity of separability using positive-partial-transpose (PPT) conditions \cite{Peres1996, Horodecki2009} and $k$-extendibility \cite{W89a, Doherty2004} to benchmark the results obtained from our VQSA. As variational quantum algorithms (VQAs), in general, are prone to encountering barren plateaus \cite{McClean2018}, we also explore how we can mitigate this issue for our algorithms by making use of the ideas presented in \cite{Cerezo2021a}.

    Our approach is distinct from recent work on quantum algorithms for estimating entanglement.
    For example, VQAs have been used to address this problem by estimating the Hilbert--Schmidt distance \cite{Consiglio2022}, by creating a zero-sum game using parameterized unitary circuits~\cite{Yin2022}, by employing symmetric extendibility tests \cite{LRW21}, by estimating logarithmic negativity \cite{arxiv.2012.14311}, and using the positive map criterion \cite{arxiv.2012.14311}. VQAs have also been used to estimate the geometric measure of entanglement of multiqubit pure states \cite{PhysRevApplied.18.024048}.
    The work of \cite{AM22} is the closest related to ours, but the test used there requires two copies of the state of interest and controlled swap operations, while our VQSA does not require either.
    In contrast, we introduce a paradigm for VQAs involving parameterized mid-circuit measurements, which is the core of our method for estimating entanglement, and we suspect that this approach will be helpful in future work for a wide variety of VQAs. Furthermore, as we show in Theorems~\ref{theorem:qip-eb_msf} and~\ref{theorem:global_cost_func}, the acceptance probabilities of our algorithms, in the ideal case, are directly related to a bonafide entanglement measure, the fidelity of separability.
     
     
\section{Results} 

\subsection{Quantum Interactive Proof for Fidelity of Separability}

\label{sec:qip-FoS}
    
    We first introduce our test for the separability of mixed states. Recall that a bipartite state is separable or unentangled if it can be written in the form given in \eqref{eqn:sep-state-informal}, where  $\left\vert\mathcal{X}\right\vert\leq\text{rank}(\sigma_{AB})^{2}$ \cite{W89,watrous_2018}. 

    \begin{figure}
        \includegraphics[width=\columnwidth]{figures/fig_QIP_EB_new.pdf}
        \caption{Test for separability of mixed states. 
        The verifier uses a unitary circuit $U^\rho$ to produce the state $\psi_{RAB}$, which is a purification of $\rho_{AB}$. The prover (indicated by the dotted box) applies an entanglement-breaking channel $\mathcal{E}_{R\rightarrow A^{\prime}}$ on $R$ by measuring the rank-one POVM $\{\mu^{x}_{R}\}_{x}$ and then, depending on the outcome $x$, prepares a pure state from the set $\{\phi^{x}_{A^{\prime}}\}_{x}$. The final state is sent to the verifier, who performs a swap test. Theorem~\ref{theorem:qip-eb_msf} states that the maximum acceptance probability of this interactive proof is equal to $\frac{1}{2}(1 + F_{s}(\rho_{AB}))$, i.e., a simple function of the fidelity of separability.}
        \label{fig:Max_Sep_Fidelity_QIP_EB}
    \end{figure}
    
    Our separability test for mixed states consists of a distributed quantum computation involving a prover and a verifier.
    The computation (depicted in Figure~\ref{fig:Max_Sep_Fidelity_QIP_EB}) begins with the verifier preparing a purification $\psi_{RAB}$\ of~$\rho_{AB}$. The verifier sends the system $R$ to a quantum prover, whom, in our model, we restrict to performing entanglement-breaking channels. The prover thus performs an entanglement-breaking channel on the reference system $R$ and sends a system $A^{\prime}$ to the verifier. An entanglement-breaking channel $\mathcal{E}_{R\rightarrow A^{\prime}}$ can always be written as a measure-and-prepare channel \cite{HSR03}, as follows:
    \begin{equation}\label{eqn:ent_breaking}
        \mathcal{E}_{R\rightarrow A^{\prime}}(\cdot)=\sum_{x\in \mathcal{X}}\operatorname{Tr}\!\left[\mu^{x}_{R}(\cdot)\right]\phi^{x}_{A^{\prime}},
    \end{equation}
    where $\{\mu^{x}_{R}\}_{x\in \mathcal{X}}$ is a rank-one positive operator-valued measure (POVM) and $\{\phi^{x}_{A^{\prime}}\}_{x\in \mathcal{X}}$ is a set of pure states. (Due to the above measure-and-prepare decomposition of an entanglement-breaking channel, we can alternatively think of the prover as being split into two provers, a first who is allowed to perform a general quantum operation, followed by the communication of classical data to a second prover, who then is allowed to perform a general operation before communicating quantum data to the verifier. However, we proceed with the single-prover terminology in what follows.) By performing the measurement portion of the entanglement-breaking channel, the prover has, in essence, steered the verifier's systems $A$ and $B$ to a certain probabilistic ensemble of pure states. After steering the verifier's system, the prover sends system $A^\prime$ to the verifier using the preparation portion of the entanglement-breaking channel. The verifier finally performs a swap test on system $A$ and $A^\prime$ and accepts if and only if the measurement outcome of the swap test is zero. The standard model in quantum computational complexity theory \cite{watrous2009complexity, VW15} is that the prover is always trying to get the verifier to accept the computation: in this scenario, the prover steers the verifier's systems $A$ and $B$ to an ensemble that has maximum overlap with a product-state ensemble and then sends an appropriate state to pass the swap test with the highest probability possible.
    
    The maximum acceptance probability of the distributed quantum computation detailed above is equal to
    \begin{equation}
    \label{eqn:qip-eb_accept_prob}
        \max_{\mathcal{E}\in\operatorname{EB}_{R\to A^\prime}}\operatorname{Tr}\!\left[\left(\Pi_{A^{\prime}A}^{\operatorname{sym}}\otimes I_{RB}\right)\mathcal{E}_{R\rightarrow A^{\prime}}\left(\psi_{RAB}\right)\right],
    \end{equation}
    where $\Pi_{A^{\prime}A}^{\operatorname{sym}}$ is the projector onto the symmetric subspace of the $A^{\prime}$ and $A$ systems, and  $\operatorname{EB}_{R\to A^\prime}$ denotes the set of all entanglement-breaking channels with input system $R$ and output system $A^\prime$.  We find in Theorem~\ref{theorem:qip-eb_msf} below that the maximum acceptance probability in~\eqref{eqn:qip-eb_accept_prob} can be expressed as a simple function of the fidelity of separability of $\rho_{AB}$, the latter defined as \cite{VPRK97, VP98}
    \begin{equation}
    \label{eq:max-sep-fid-def}
        F_{s}(\rho_{AB}) \coloneqq \max_{\sigma_{AB}\in \operatorname{SEP}(A:B)} F(\rho_{AB},\sigma_{AB}) ,
    \end{equation}
    where $\operatorname{SEP}\!\left(A\!:\!B\right)$ denotes the set of separable states shared between Alice and Bob and $F(\rho,\sigma) \coloneqq \left\|\sqrt{\rho}\sqrt{\sigma}\right\|_1^2$ is the fidelity of the states $\rho$ and $\sigma$ \cite{Uhlmann1976}. The fidelity of separability is also known as the maximum separable fidelity \cite{HMW13, HMW14, GHMW15}. With this definition, we state the first key theoretical result of our paper: 
        
    \begin{theorem}
    \label{theorem:qip-eb_msf}
        For a pure state $\psi_{RAB}$, the following holds%
        \begin{multline}
        \label{eqn:qip-eb_msf}
            \max_{\mathcal{E}\in\operatorname{EB}_{R\to A^\prime}}\operatorname{Tr}\!\left[\left(\Pi_{A^{\prime}A}^{\operatorname{sym}}\otimes I_{RB}\right)\mathcal{E}_{R\rightarrow A^{\prime}}\left(\psi_{RAB}\right)\right]\\
            =\frac{1 + F_{s}(\rho_{AB})}{2},
        \end{multline}
        where $F_{s}(\rho_{AB})$ is the fidelity of separability of the state $\rho_{AB} = \operatorname{Tr}_{R}[\psi_{RAB}]$.
    \end{theorem}
    
    See the first part of Section~\ref{sec:methods} for a brief overview of the proof and Appendix~\ref{appendix:proof_swap-test-eb-channel} for a detailed proof. Appendices~\ref{appendix:proof_alt_streltsov} and \ref{appendix:proof_max-sep-fid-inf-norm} recall some auxiliary results that support the proof in Appendix~\ref{appendix:proof_swap-test-eb-channel}. With this theorem, we have established a separability test for mixed states.
        
    %\textit{VQSA to test and quantify separability}.---

\subsection{Variational Quantum Steering Algorithm for Fidelity of Separability}
    
    We want to point out two important aspects of our separability test from Section~\ref{sec:qip-FoS}. First, note that the swap test at the end of the computation essentially leads to a measure of overlap between the state of the verifier's system and the state provided by the prover. The other important point is that, in the real world, no computationally unbounded quantum prover is available to provide the ideal states required for the tests. 

    Taking both these points into consideration, we modify the computational scenario in Figure~\ref{fig:Max_Sep_Fidelity_QIP_EB} to a)  measure the necessary overlaps directly and b) make use of quantum variational techniques \cite{Cerezo2021} (parameterized unitary circuits and classical optimization of parameters) to approximate the actions of a computationally unbounded prover. The resulting procedure also tests and quantifies the separability of a given state by estimating its fidelity of separability. This procedure is a different quantum variational technique called a variational quantum steering algorithm (VQSA). As can be seen in Figure~\ref{fig:Max_Sep_Fidelity_VQA}, quantum steering is at the core of the VQSA via the use of a parameterized mid-circuit measurement. 

    \begin{figure}
        \includegraphics[width=\columnwidth]{figures/fig_VQA_new.pdf}
        \caption{Quantum part of the VQSA to estimate the fidelity of separability $F_s(\rho_{AB})$. 
        The unitary circuit $U^\rho$ prepares the state $\psi_{RAB}$, which is a purification of $\rho_{AB}$. The parameterized circuit $W_R(\Theta)$ acts on $R$ to evolve $\psi_{RAB}$ to another purification of $\rho_{AB}$. The following measurement, labeled ``steering measurement,'' steers the systems $AB$ to be in a pure state $\psi_{AB}^{x}$ if the measurement outcome~$x$ occurs. Conditioned on the outcome~$x$, the final parameterized circuit~$U^{x}_{A}(\Theta^x)$ and the subsequent measurement accepts with a maximum probability of $F_s(\rho_{AB})$.} 
        \label{fig:Max_Sep_Fidelity_VQA}
    \end{figure}
    %~$\left\|\psi_{A}^{x}\right\|_{\infty}$.
    %Our VQSA can be understood by noting that  \eqref{eq:convex-decomp-max-sep-fid} has two maximization problems: first is the maximization to compute $F_{s}(\psi_{AB}^{x})$ and the other is the maximization over all pure-state decompositions of $\rho_{AB}$, such that $\sum_{x}p(x)\psi_{AB}^{x}=\rho_{AB}$.

    Our VQSA is structured as follows. Let $\rho_{AB}$ denote the state for which we want to estimate the fidelity of separability, and let $\psi_{RAB}$ be a purification of it, which results from the action of the unitary operator $U^\rho$ on the all-zeros pure state $|0\rangle\!\langle 0|$. 
    Once we have $\psi_{RAB}$, we can attempt to access all possible pure-state decompositions $\{(p(x),\psi_{AB}^{x})\}_{x\in \mathcal{X}}$ of $\rho_{AB}$ by acting on system $R$ with unitary operations. We use the first parameterized unitary $W_R(\Theta)$ to do so.  To ensure that we have a sufficient number of measurement outcomes (to cover the possible case when $|\mathcal{X}| = \text{rank}(\rho_{AB})^2$), we can prepare some ancilla qubits in the all-zeros state, for a system $R'$, and act with $W$ on $R$ and $R'$. However, without loss of generality, these extra qubits can be grouped as part of an overall reference system, relabeled as $R$. 
    
    After the action of $W_R(\Theta)$, the reference system is measured in the standard basis, and based on the outcome $x$, the post-measurement state of the system $AB$ is a pure state $\psi^x_{AB}$. We then estimate the maximum eigenvalue of the reduced state~$\psi^x_{A}$: this can be accomplished by performing a parameterized unitary $U^{x}_{A}(\Theta^x)$, based on the outcome $x$, on the reduced state~$\psi_A^x$, measuring all qubits of $A$ in the computational basis, and accepting if the all-zeros outcome occurs.
    
    Using a hybrid quantum-classical optimization loop, we can maximize the acceptance probability to estimate the value of the fidelity of separability. The quantum part of this VQSA is summarized in Figure~\ref{fig:Max_Sep_Fidelity_VQA}.

    \begin{theorem}
    \label{theorem:global_cost_func}
        If the parameterized unitary circuits involved in the quantum part of the VQSA, summarized in Figure~\ref{fig:Max_Sep_Fidelity_VQA}, can express all possible unitary operators of their respective systems, then the maximum acceptance probability of the quantum circuit is equal to $F_s(\rho_{AB})$.
    \end{theorem} 
    
    See Appendix~\ref{appendix:step_by_step} for a detailed proof.

\subsection{Benchmarking Semidefinite Programs and Examples}

   \begin{figure}
        \includegraphics[width=\columnwidth]{figures/plot_75MixturePhi_state.pdf}
        \caption{Fidelity of separability calculated for a ($3/4$,$1/4$) classical mixture of $|\Phi^+\rangle$ and $|\Phi^-\rangle$ using our VQSA (blue line). The algorithm converges to 0.93, which agrees with the value obtained using the benchmarks $\widetilde{F}_s^1$ and $\widetilde{F}_s^2$.}
        \label{fig:Max_Sep_Fidelity_Bell}
    \end{figure}

 

    Since our algorithms will be running on near-term quantum computers with limited scale and error tolerance, we develop semidefinite programs (SDPs) to benchmark the results from our VQSA because the ideal outcomes can be estimated classically for small numbers of qubits. Our benchmarks $\widetilde{F}_{s}^{1}(\rho_{AB}, k)$ and $\widetilde{F}_{s}^{2}(\rho_{AB}, k)$ are based on the positive partial transpose (PPT) and $k$-extendibility hierarchy. See details in Appendices~\ref{appendix:ppt-k-state-sdp} and \ref{appendix:proof_swap-test-ppt-k-channel-sdp}. 


    We now present an example simulation of our VQSA to demonstrate that it can estimate the fidelity of separability. For our first example, we take the state of interest $\rho_{AB}$ to be a ($3/4$,$1/4$) probabilistic mixture of two maximally entangled states, $|\Phi^+\rangle = \sqrt{1/2}(|00\rangle+|11\rangle)$ and $|\Phi^-\rangle = \sqrt{1/2}(|00\rangle-|11\rangle)$, so that
    \begin{equation}
        \rho_{AB} = \frac{3}{4} |\Phi^+\rangle\!\langle \Phi^+| + \frac{1}{4} |\Phi^-\rangle\!\langle \Phi^-|.
    \end{equation}
    Systems $R$, $A$, and $B$ of the purification of $\rho_{AB}$ contain one qubit each. See Figure~\ref{fig:Max_Sep_Fidelity_Bell} for the results. We use the benchmarks and VQSA to estimate the fidelity of separability as $\approx$ 0.93. We evaluate these benchmarks for different levels of the $k$-extendibility hierarchy. See Appendix~\ref{appendix:simulations} for more examples and Appendix~\ref{appendix:software} for details about the code we developed.

   \begin{figure}
        \includegraphics[width=\columnwidth]{figures/plot_depolarization_p_0_7.pdf}
        \caption{Fidelity of separability calculated for the state $\tilde{\rho}_{AB}$ as specified in \eqref{eqn:rho-tilde} using our VQSA (blue line) and $\widetilde{F}_s^1$ (orange line).}
        \label{fig:Max_Sep_Fidelity_Depol}
    \end{figure}
    
    As a second example, we consider a state consisting of four qubits. Let us consider the four qubit state $|\psi \rangle$ defined as follows:
    \begin{equation}
        \frac{1}{\sqrt{2}}\left(|0\rangle_{A_1} |0\rangle_{A_2} |0\rangle_{B_1} |0\rangle_{B_2} +|1\rangle_{A_1} |1\rangle_{A_2} |1\rangle_{B_1} |1\rangle_{B_2}\right),
    \end{equation} 
    where $A$ consists of two qubits $A_1$ and $A_2$ and $B$ consists of two qubits $B_1$ and $B_2$. We then pass $A_1$ and $A_2$ through a qubit depolarizing channel defined as $\mathcal{D}_p(\rho)\coloneqq (1-p)\rho+p\mathbb{I}/2$ where $p=0.7$. So, the final state under consideration can be written as
    \begin{equation}\label{eqn:rho-tilde}
        \tilde{\rho}_{AB}\coloneqq (\mathcal{D}_{p, A_1}\otimes\mathcal{D}_{p,A_2}\otimes\mathbb{I}_{B_1}\otimes\mathbb{I}_{B_2})\left(|\psi \rangle\!\langle\psi|\right).
    \end{equation}
    We can then use our VQSA to estimate the fidelity of separability for $\tilde{\rho}_{AB}$ and compare the result against the previous SDP benchmarks. See Figure~\ref{fig:Max_Sep_Fidelity_Depol} for the results.

    %The jitters in the value of fidelity between iterations of the VQSA can be attributed to the shot noise in estimating the acceptance probability, using the Qiskit Aer simulator and the fact that we use Qiskit's Simultaneous Perturbation Stochastic Approximation (SPSA) optimizer to do the classical optimization. We provide more examples in Appendix~\ref{appendix:simulations}.
    
 

\subsection{Generalization to Multipartite Fidelity of Separability}

    We also generalize our VQSA to measure the fidelity of separability of multipartite states
    in the following fashion.
    %using the appropriate definition of separability in the multipartite scenario and the multipartite version of \eqref{eq:max-sep-fid-def}.
    %, as given in \cite{Streltsov2010}. See Appendix~\ref{appendix:multipartite} for details.  
    %See Figure~\ref{fig:Max_Sep_Fidelity_VQSA_Multi} in Appendix~\ref{appendix:multipartite} for this multipartite algorithm.
%Multipartite Generalizations starts here.
    %\textit{Multipartite Generalizations, should this be in results?}.---We now discuss the multipartite generalization of the tests of separability of mixed states. 
    A multipartite state $\rho_{A_1\cdots A_M}\in \mathcal{D}(\mathcal{H}_{A_1\cdots A_M})\equiv \mathcal{D}(\mathcal{H}_{A_1}\otimes\cdots\otimes\mathcal{H}_{A_M})$ is  separable if it can be written as
        \begin{equation}
            \rho_{A_1\cdots A_M}=\sum_{x\in\mathcal{X}}p(x)\psi^{x,1}_{A_1}\otimes\cdots\otimes\psi^{x,M}_{A_M}
        \end{equation}
        where $\psi_{A_i}^{x,i}$ is a pure state for every $x\in\mathcal{X}$ and $i\in\{1,\ldots,M\}$.
    Let $M\text{-SEP}$ denote the set of all $\rho_{A_1\cdots A_M}\in \mathcal{D}(\mathcal{H}_{A_1\cdots A_M})$ such that $\rho_{A_1\cdots A_M}$ is separable. 
    
    For the multipartite case of the distributed quantum computation, the verifier prepares a purification $\psi^{\rho}_{RA_1\cdots A_M}$\ of $\rho_{A_1\cdots A_M}$. The prover applies a multipartite entanglement-breaking channel on $R$, which can be written as: 
    \begin{multline}
    \label{eqn:ent-break-mult-1}
        \mathcal{E}_{R\rightarrow A^{\prime}_1\cdots A^{\prime}_{M-1}}(\cdot)\\
        =\sum_{x\in \mathcal{X}}\operatorname{Tr}[\mu^{x}_{R}(\cdot)]\left(\phi^{x,1}_{A^{\prime}_1}\otimes\cdots\otimes\phi^{x,M-1}_{A^{\prime}_{M-1}}\right),
    \end{multline}
     where $\{\mu^{x}_{R}\}_{x}$ is a rank-one POVM and $\{\phi^{x,i}_{A^{\prime}_i}\}_{x,i}$ is a set of pure states. The prover sends systems $(A^{M-1})^{\prime}\equiv  A^{\prime}_1\cdots A^{\prime}_{M-1}$ to the verifier. Finally, the verifier performs a collective swap test on these systems and the systems $A_1\cdots A_M$, as depicted in Figure~\ref{fig:Max_Sep_Fidelity_QIP_EB_Multi-1}. The acceptance probability of this distributed quantum computation is given by
    \begin{equation}
        \max_{\mathcal{E}\in\operatorname{EB}_{M-1}}\operatorname{Tr}[\Pi_{(A^{M-1})^{\prime}A^{M-1}}^{\operatorname{sym}}\mathcal{E}_{R\rightarrow (A^{M-1})^{\prime}}(\psi_{RA^{M-1}})],
    \end{equation}
    where $\Pi_{(A^{M-1})^{\prime}A^{M-1}}^{\operatorname{sym}}$ is the projection onto the symmetric subspace of systems $(A^{M-1})^{\prime}$ and $A^{M-1}$ and $\operatorname{EB}_{M-1}$ denotes the set of multipartite entanglement-breaking channels defined in~\eqref{eqn:ent-break-mult-1}.
    This leads to the following theorem:
    \begin{theorem}\label{theorem:qip-eb_msf_multi}
        For a pure state $\psi_{RA^{M}} \equiv \psi_{RA_1\cdots A_{M}}$, the following equality holds:%
        \begin{multline}
            \label{eq:mult-part-fid-sep-test-acc-prob-1}
            \max_{\mathcal{E}\in\operatorname{EB}_{M-1}} \operatorname{Tr}[ \Pi_{(A^{M-1})^{\prime}(A^{M-1})}^{\operatorname{sym}} \mathcal{E}_{R\rightarrow A^{\prime}_1\cdots A^{\prime}_{M-1}}(\psi_{RA^{M}})]
            \\
            =\frac{1}{2}\left(1 + F_s(\rho_{A_1\cdots A_M})\right),
        \end{multline}
        where the multipartite fidelity of separability is defined as
        \begin{multline}
            F_s(\rho_{A_1\cdots A_M})\coloneqq \\ \max_{\sigma_{A_1\cdots A_M}\in M-\operatorname{SEP}}F(\rho_{A_1\cdots A_M},\sigma_{A_1\cdots A_M}).
        \end{multline}
    \end{theorem}
    
    \begin{figure}
        \includegraphics[width=\columnwidth]{figures/fig_QIP_EB_new3.pdf}
        \caption{Test for separability of multipartite mixed states. The verifier uses the unitary circuit $U^\rho$ to prepare the state $\psi_{RA_1 A_2 A_3 A_4}$, which is a purification of $\rho_{A_1 A_2 A_3 A_4}$. The prover (indicated by the dotted box) applies an entanglement-breaking channel $\mathcal{E}_{R\rightarrow A^{\prime}_1 A^{\prime}_2 A^{\prime}_3}$ on $R$ by measuring the rank-one POVM $\{\mu^{x}_{R}\}_{x\in \mathcal{X}}$ and then, depending on the outcome $x$, prepares a state from the set $\{\phi^{x,1}_{A^{\prime}_1}\otimes\phi^{x,2}_{A^{\prime}_2}\otimes\phi^{x,3}_{A^{\prime}_3}\}_{x\in \mathcal{X}}$. The final state is sent to the verifier, who performs a collective swap test. Theorem~\ref{theorem:qip-eb_msf_multi} states that the maximum acceptance probability of this interactive proof is equal to $\frac{1}{2}(1 + F_{s}(\rho_{A_1 A_2 A_3 A_4}))$, i.e., a simple function of the fidelity of separability.}
        \label{fig:Max_Sep_Fidelity_QIP_EB_Multi-1}
    \end{figure}
     
    See Appendix~\ref{appendix:multipartite} for a proof. We can then use the generalized test of separability of mixed states to develop a VQSA for the multipartite case. See Figure~\ref{fig:Max_Sep_Fidelity_VQSA_Multi-1}. This involves replacing the collective swap test in Figure~\ref{fig:Max_Sep_Fidelity_QIP_EB_Multi-1} with an overlap measurement, similar to how we got Figure~\ref{fig:Max_Sep_Fidelity_VQA} from Figure~\ref{fig:Max_Sep_Fidelity_QIP_EB}.

    \begin{figure}
        \includegraphics[width=\columnwidth]{figures/fig_VQA_new_multi3.pdf}
        \caption{VQSA to estimate the multipartite fidelity of separability $F_s(\rho_{A_1A_2A_3A_4})$. 
        The unitary circuit $U^\rho$ prepares the state $\psi_{RA_1A_2A_3A_4}$, which is a purification of $\rho_{A_1A_2A_3A_4}$. The parameterized circuit $W_R(\Theta)$ acts on $R$ to evolve the state to another purification of $\rho_{A_1A_2A_3A_4}$. The following measurement, labeled ``steering measurement,'' steers the remaining systems to be in a state $\psi_{A_1A_2A_3A_4}^{x}$ if the measurement outcome $x$ occurs. Conditioned on the outcome $x$, the final parameterized circuits $U^{x,1}_{A_1}(\Theta^x_1)$, $U^{x,2}_{A_2}(\Theta^x_2)$, and $U^{x,3}_{A_3}(\Theta^x_3)$ are applied and the subsequent measurement accepts with a maximum probability of $F_s(\rho_{A_1A_2A_3A_4})$.} 
        \label{fig:Max_Sep_Fidelity_VQSA_Multi-1}
    \end{figure}
%Multipartite Generalizations ends here.

    %{\color{red} Soorya : Okay, I think that we need to show the computational complexity result in this section. Broad overview here. Two sentences about this in the next section and the rest in the methods.}

\subsection{Quantum Computational Complexity Considerations}

    Our final result is regarding the computational complexity of estimating the fidelity of separability $F_s(\rho_{AB})$. The complexity-theoretic approach allows us to classify the separability problem based on its computational difficulty. Analyses of this form can be effectively conducted within the framework of quantum computational complexity theory~\cite{watrous2009complexity, VW15}.

     In the paradigm of complexity theory \cite{arora_barak_2009}, a complexity class is a set of computational problems that require similar resources to solve. If a complexity class~$A$ is contained within another class $B$, then some problems in~$B$ could require more computational resources than problems in $A$. To effectively characterize the difficulty of a class of problems, we pick a problem that is representative of the class or complete for the class. A problem $h$ is considered complete for a complexity class $A$ if $h$ is contained in the class and the ability to solve problem $h$ can be extended efficiently to solve every other problem in~$A$. 

    To tackle the question posed about the computational complexity of estimating the fidelity of separability, we define \qipeb\ to be the complexity class containing problems that can be solved using a prover restricted to using only entanglement-breaking channels, which processes a quantum message received from the verifier and sends back a quantum message to the verifier. Thus, estimating the fidelity of separability of a given state then falls within \qipeb, as seen from Figure~\ref{fig:Max_Sep_Fidelity_QIP_EB}. To fully characterize this novel complexity class, we provide a complete problem for it. We establish that, given quantum circuits to generate a channel $\mathcal{N}_{A\rightarrow B}$ and a state $\rho_{B}$, estimating the following quantity is complete for \qipeb:%
        \begin{multline}
           \max_{\substack{\{  (p(x),\psi^{x})\}  _{x},\left\{  \varphi^{x}\right\}_{x},\\
           \rho_{B} = \sum_{x}p(x)\psi_{B}^{x}
           }}   \sum_{x}p(x)F(\psi_{B}^{x},\mathcal{N}_{A\rightarrow B}(\varphi_{A}^{x}))  .
        \end{multline}
    See Appendix~\ref{appendix:qipeb} for details and an interpretation of this quantity. 

    By placing the problem of estimating the fidelity of separability in the class \qipeb, we establish results that link quantum steering and the separability problem to quantum computational complexity theory. Furthermore, we show that the complexity class \qipeb\ 
    %is contained in QIP \cite{watrous2003pspace,kitaev2000parallelization}, and 
    contains QAM \cite{marriott2004quantum} and QSZK \cite{watrous2006zero}. It also follows, as a direct generalization of the hardness results from \cite{HMW13, HMW14}, that the problem of estimating the fidelity of separability is hard for QSZK and NP. All of the aforementioned complexity classes are considered to be, in the worst case, out of reach of the capabilities of efficient quantum computers. See Appendix~\ref{appendix:complexity_placements} for proofs and Figure~\ref{fig:placement} for a detailed diagram. However, following the approach of~\cite{210808406}, we can try to solve some instances of problems in these classes using parameterized circuits and VQAs. 

    \begin{figure}
    \centering
        \includegraphics[width=0.8\columnwidth]{figures/placement_diagram_new.pdf}
        \caption{Placement of \qipeb\ relative to other known complexity classes. The complexity classes are organized such that if a class is connected to a class above it, the complexity class placed lower is a subset of the class above. For example, \qipeb\ is a superset of both QSZK and QAM.}
        \label{fig:placement}
    \end{figure}
    
\section{Conclusion and Discussion}  

    In this paper, we detailed a distributed quantum computation to test the separability of a quantum state that, at its core, uses quantum steering. This test demonstrated a link between quantum steering and the separability problem. The acceptance probability of this distributed quantum computation is directly related to an entanglement measure known as the fidelity of separability. 
    Using the test's structure, we also showed computational complexity-theoretic results and established a link between quantum steering, quantum algorithms, and quantum computational complexity. 
    By replacing the prover with a parameterized circuit, we modified this distributed quantum computation to develop our VQSA, a novel kind of variational quantum algorithm that uses quantum steering to address the problem of estimating the fidelity of separability. This algorithm allows for the direct estimation of the fidelity of separability without the need for state tomography and subsequent approximate tests on separability. Our algorithm is not unitary due to the mid-circuit measurement 
    on system $R$ and the consequent conditional operation applied on system $A$. This is an important distinction from most VQAs, which do not use a parameterized mid-circuit measurement. We also discuss multipartite generalizations of both our separability test and VQSA. Finally, we simulated our VQSA  using the noisy Qiskit Aer simulator, which showed favorable convergence trends and was compared against two classical SDP benchmarks. 

    %{\color{red} Soorya : I don't think this belongs in this section. As mentioned above, I think this needs to be mentioned in the previous section and the proof moved to the methods section.}
    

    Our VQSA has applications beyond entanglement quantification on a single quantum computer. We can also think of our VQSA as a distributed variational quantum algorithm for measuring the entanglement of a bipartite state. See \cite{zhao2021practical,DBKC23,arxiv.2208.00450} for previous instances of distributed VQAs. Indeed, our algorithm can be executed over a quantum network in which each node has quantum and classical computers capable of performing VQAs. The initial part of the algorithm distributes $R$ to Rob, $A$ to Alice, and $B$ to Bob, who are all in distant locations. Then, Rob performs the parameterized measurement and sends the outcome over a classical channel to Alice, who performs another parameterized measurement. Then, they can repeat this process to assess the quality of the entanglement between Alice and Bob. This interpretation is even more interesting regarding quantum networks for the multipartite case, in which the classical data gets broadcast from Rob to all the other nodes except the last one.
    
    VQSAs can tackle other problems involving quantum steering, like maximizing the pure-state decompositions of quantum states. This technique may also be helpful for estimating other entanglement measures that involve optimizing the set of separable states. By applying the insights of \cite[Appendix~A]{Streltsov2010} and our approach here, it is clear that VQSAs will also be helpful for estimating maximal fidelities associated with other resource theories, such as the resource theory of coherence~\cite{BCP14}. More broadly, we suspect that the paradigm of parameterized mid-circuit measurements and distributed variational quantum algorithms will be helpful in addressing other computational problems of interest in quantum information science and physics, given recent advances in experimental implementations~\cite{Cramer2016, Egan2021, Acharya2023,graham2023midcircuit}.

    Going forward from here, we consider it an important open question in quantum computational complexity theory to place a non-trivial upper bound on the class \qipeb. As indicated in Remark~\ref{rem:de-finetti-qip}, an approach using the known quantum de Finetti theorem from \cite[Theorem~II.7']{christandl2007one} does not appear to be helpful for this task.

    
\section{Methods}

\label{sec:methods}


    This section briefly overviews the techniques used to prove Theorem~\ref{theorem:qip-eb_msf}, our main result, a brief description of SDP benchmarks, and essential details about our simulations.
    
    To gain intuition about the separability test for mixed states, let us formulate a simple test for the separability of pure states. From \eqref{eqn:sep-state-informal}, we can see that a pure bipartite state~$\varphi_{AB}$ is separable  if it can be written in product form, as
    \begin{equation}
     \label{eq:pure-sep-state}
     \varphi_{AB}=\psi_{A} \otimes \phi_{B},
    \end{equation}
    where $\psi_{A}$ and $ \phi_{B}$ are pure states.
    The test we developed below is important because it will reappear as part of the test for separability in the general case, along with quantum steering. Additionally, our approach slightly differs from the standard approach for testing entanglement of pure states, which employs two copies of the state in a swap test \cite{Brennen03, HM10, GHMW15}. Instead, our approach requires only a single copy of the state.

    Our pure-state separability test consists of a distributed quantum computation involving a prover and a verifier (see Figure~\ref{fig:swaptest_pure}). 
    The computation starts with the verifier preparing the pure state $\psi_{AB}$. The prover sends the verifier the pure state $\phi_{A^\prime}$ in register $A^\prime$. (We note that the prover can send a mixed state; however, the maximum acceptance probability of the test is achieved by a pure state. Hence, without loss of generality, the prover must send a pure state.) 
    The verifier then performs the standard swap test \cite{BBD+97, BCWW01} on $A$ and $A^\prime$ and accepts if the measurement outcome is zero. In the standard model of quantum computational complexity \cite{watrous2009complexity, VW15}, the prover attempts to get the verifier to accept the swap test with as high a probability as possible. Thus, in this scenario, the prover selects  $\phi_{A^\prime}$ to maximize the overlap between the reduced stated $\psi_{A} \coloneqq \operatorname{Tr}_B[\psi_{AB}]$ and $\phi_{A^\prime}$. The maximum acceptance probability is then equal to
    \begin{align}
    & \max_{\phi}\operatorname{Tr}[(\Pi_{A^{\prime}A}^{\operatorname{sym}}\otimes I_B)(\phi_{A^\prime} \otimes \psi_{AB})]\notag \\
    & =
    \frac{1}{2}\left(1+\max_{\phi}\operatorname{Tr}[F_{A'A}(\phi_{A'} \otimes \psi_A)]\right) \\
    & =
    \frac{1}{2}\left(1+\max_{\phi}\operatorname{Tr}[\phi_{A}  \psi_A]\right) 
     = \frac{1}{2}\left(1+\left\|\psi_{A}\right\|_{\infty}\right),    
    \label{eq:initial-swap-test}
    \end{align}
    where $F_{A'A}$ is the unitary swap operator acting on systems $A'$ and $A$, the projector $\Pi_{A^{\prime}A}^{\operatorname{sym}} \coloneqq  \frac{1}{2} \left( I_{A'A} + F_{A'A}\right)$ projects onto the symmetric subspace of $A^{\prime}$ and $A$, and $\left\|\psi_{A}\right\|_{\infty}$ is the spectral norm of the reduced state $\psi_{A}$ (equal to its largest eigenvalue). Since $\left\|\psi_{A}\right\|_{\infty} = 1$ if and only if $\psi_{A}$ is a pure state and this occurs if and only if $\psi_{AB}$ is a product state, it follows that the maximal acceptance probability is equal to one if and only if~$\psi_{AB}$ is a product state.
    \begin{figure}
        \centering
        \includegraphics[width=0.7\columnwidth]{figures/swaptest_pure_new.pdf}
        \caption{Pure-state separability test: The verifier has the pure state $\psi_{AB}$ of interest. The prover (indicated by the dotted box) sends the verifier a pure state $\phi_{A^\prime}$, who then performs the standard swap test on systems~$A'$ and~$A$. 
        As mentioned in \eqref{eq:initial-swap-test}, the acceptance probability is equal to $ \frac{1}{2}(1+\left\|\psi_{A}\right\|_{\infty})$.} 
        \label{fig:swaptest_pure}
    \end{figure}

     Now we outline the proof of Theorem~\ref{theorem:qip-eb_msf}, which relies on two important facts. The first is that the fidelity of separability can be written in terms of a convex roof as follows \cite[Theorem~1]{Streltsov2010}:
        \begin{equation}
        \label{eq:convex-decomp-max-sep-fid}
            F_{s}(\rho_{AB})=  \max_{\substack{\{(p(x),\psi_{AB}^{x})\}_{x},\\\rho_{AB}=\sum_{x}p(x)\psi_{AB}^{x}}}\sum_{x}p(x)F_{s}(\psi_{AB}^{x}),
        \end{equation}
        where $\{p(x)\}_x$ is a probability distribution and each $\psi_{AB}^{x}$ is a pure state.
        See also \cite[Lemma~1]{Regula2018}.
        The second fact is that, for a pure bipartite state, $F_{s}(\psi_{AB})$ can be rewritten as \cite[Section~6.2]{Streltsov2010}
        \begin{equation}
            F_{s}(\psi_{AB})= \left\Vert \psi_{A}\right\Vert _{\infty}.
            \label{eq:max-sep-fid-inf-norm}%
        \end{equation}      
    Along with these facts, we also note that the optimization over all entanglement-breaking channels in \eqref{eqn:qip-eb_accept_prob} is the same as optimizing over all pure-state decompositions of $\rho_{AB}$ and the rest of the proof follows. For completeness, we provide proofs of \eqref{eq:convex-decomp-max-sep-fid} and \eqref{eq:max-sep-fid-inf-norm} in Appendices~\ref{appendix:proof_alt_streltsov} and \ref{appendix:proof_max-sep-fid-inf-norm}, respectively. It follows from \eqref{eqn:sep-state-informal} and \eqref{eq:max-sep-fid-inf-norm}  that $\sum_{x}p(x)\left\Vert \psi_{A}^{x}\right\Vert _{\infty} = 1$ for a separable state, which is the maximum possible value of $F_{s}(\rho_{AB})$. Hence, the distributed quantum computation in Figure~\ref{fig:Max_Sep_Fidelity_QIP_EB} tests and quantifies the separability of a state by estimating its fidelity of separability. 
    %However, we note that there are complexity-theoretic subtleties associated with the parallel repetition of this algorithm, which we discuss in detail in Appendix~\ref{app:par-rep}. 
    Finally, note that the computation in Figure~\ref{fig:Max_Sep_Fidelity_QIP_EB} can be reduced to that in Figure~\ref{fig:swaptest_pure} if the purifying system $R$ is trivial, implying that the verifier only prepares a pure state on systems $A$ and $B$ in this case. 

    %\textit{Quantum Complexity Classes}.---{\color{red} Add a citation to some book maybe, also needs improvement}
   
    
    \textit{Benchmarking via semidefinite programs}---Here we briefly explain  the derivation of the SDP benchmarks $\widetilde{F}_{s}^{1}(\rho_{AB}, k)$ and $\widetilde{F}_{s}^{2}(\rho_{AB}, k)$.
    
    First, let us recall that the fidelity between two quantum states has an SDP formulation \cite{Wat13}. Since there is no semidefinite constraint that directly corresponds to optimizing over the set of separable states \cite{Fawzi2021}, we can approximate the fidelity of separability of a state by maximizing its fidelity with positive partial transpose (PPT) states \cite{Peres1996, Horodecki1996}  and $k$-extendible states \cite{W89a, Doherty2004}. Further noting that the PPT and $k$-extendibility constraints are positive semidefinite constraints, we obtain our first benchmark $\widetilde{F}_s^1(\rho_{AB}, k)$, defined in Appendix~\ref{appendix:ppt-k-state-sdp}, and which is proven there to satisfy the following bounds:
    \begin{multline}
    \label{eq:first-sdp-bounds}
        F_{s}(\rho_{AB}) \leq
        \widetilde{F}_{s}^{1}(\rho_{AB}, k)
    \\
    \leq 1- \left[\sqrt{1-F_{s}(\rho_{AB})} -2\sqrt{\frac{\left\vert B\right\vert ^{2}}{k}\left(  1-\frac{\left\vert B\right\vert ^{2}}{k}\right)  } \right]^2,
    \end{multline}
    where $\left\vert B\right\vert$ is the dimension of system $B$. By inspection of the above inequalities, observe that 
    \begin{equation}
    \lim_{k\to \infty} \widetilde{F}_{s}^{1}(\rho_{AB}, k) = F_{s}(\rho_{AB}).    
    \end{equation}
    
    
    The second benchmark can be obtained using \eqref{eqn:qip-eb_accept_prob}. Just like PPT and $k$-extendible states were used to approximate separable states for the first benchmark, we use PPT channels \cite{Rai99, Rai01} and $k$-extendible channels \cite{PBHS13, Kaur2018, KDWW21, BBFS18} to approximate entanglement-breaking channels, leading to our second benchmark $\widetilde{F}_s^2(\rho_{AB}, k)$. We show that $\widetilde{F}_s^2(\rho_{AB}, k)$ is an SDP and approximates the fidelity of separability in the following fashion: 
        \begin{equation}
        \label{eqn:swap-test-ppt-k-channel-sdp}
         F_{s}(\rho_{AB}) \leq \widetilde{F}_s^2(\rho_{AB}, k) \leq 
            F_{s}(\rho_{AB})+\frac{4  \left\vert A\right\vert^{3} \left\vert B\right\vert}{k}.
        \end{equation}
    where $\left\vert A\right\vert$ and $\left\vert B\right\vert$ is the dimension of systems $A$ and $B$, respectively. See Appendix~\ref{appendix:proof_swap-test-ppt-k-channel-sdp} for a proof. Again, observe that
    \begin{equation}
    \lim_{k\to \infty} \widetilde{F}_{s}^{2}(\rho_{AB}, k) = F_{s}(\rho_{AB}).    
    \end{equation}
    
    

    %\textit{Simulations}.---We now present a few simulations of our VQSA, which demonstrate that it can obtain an estimate of the fidelity of separability. We have selected two known examples of states to indicate the working of the VQSA and the benchmarks we have developed. The first example is  a random product state, shown in Figure~\ref{fig:Max_Sep_Fidelity_Product}. The fidelity of separability of a product state is equal to one, and the figure shows that our VQSA converges to the correct value. We also evaluate our benchmarks for different levels of the $k$-extendibility hierarchy. We repeat these exercises for a ($\sqrt{3/4}$,$\sqrt{1/4}$) superposition of $|00\rangle$ and $|11\rangle$. See Figure~\ref{fig:Max_Sep_Fidelity_Bell} for the results. The fidelity of separability of this state is equal to 3/4 and the benchmarks and VQSA converge to this value. 
    \textit{Simulations and Reward Functions}---For our simulations, we use the Qiskit Aer simulator and Qiskit's Simultaneous Perturbation Stochastic Approximation (SPSA) optimizer to perform the classical optimization. The jitters in the fidelity values between iterations of the VQSA can be attributed to the shot noise in estimating the acceptance probability using the Qiskit Aer simulator, as well as the fact that the SPSA optimizer we have used to perform the classical optimization is itself a stochastic algorithm. We provide more examples in Appendix~\ref{appendix:simulations}.
    
    An essential issue with variational quantum techniques, such as VQAs, is the emergence of barren plateaus or vanishing gradients as the number of qubits increases~\cite{McClean2018}. However, recent results have shown that this problem can be mitigated by switching from a global reward function to a local reward function \cite{Cerezo2021a}. In our case, a global reward function is one in which we measure all the qubits that constitute system~$A$, as done in the approach discussed in  Theorem~\ref{theorem:global_cost_func}. An example of a local reward function involves selecting a qubit in the system $A$ at random to measure in the computational basis and recording the outcome, accepting if the result is equal to zero. Our proposed local reward function can be used to obtain upper and lower bounds on our initial global reward function, following the approach of \cite[Appendix C]{Khatri2019quantumassisted} and discussed for completeness in Appendix~\ref{appendix:local_cost}. Local functions have also been used to avoid barren plateaus in VQAs to determine the geometric measure of entanglement for pure states \cite{zambrano2023avoiding}. We provide simulations of the local reward function in Appendix~\ref{appendix:simulations}, indicating that the local reward function can also be used to estimate the fidelity of separability of a given state.

    \section*{Acknowledgments}
        We are especially grateful to Gus Gutoski, for providing the main idea of the quantum interactive proof detailed in Figure~\ref{fig:Max_Sep_Fidelity_QIP_EB}, back in September 2013. We also thank Paul Alsing, Eric Chitambar, Zoe Holmes, and Wilfred Salmon for insightful discussions, and Ludovico Lami, Bartosz Regula, and Alexander Streltsov for the same, as well as pointing us to \cite{Regula2018}. AP, SR, and MMW acknowledge support from the National Science Foundation under Grant No.~1907615.
    

\bibliographystyle{quantum}
\bibliography{Ref}

\onecolumn
\appendix


%\section{Summary of Contents}

%This supplementary material provides proofs, generalizations, and benchmarks that were introduced or mentioned in the main text. We first give proofs for the theorems. After that, we provide more details on the multipartite generalization of the fidelity of separability. Next, we expand upon the introduced complexity class \qipeb\ and the relative placement of this class in the complexity hierarchy. After that, we formulate two classical SDPs to benchmark our VQSA for small inputs. Lastly, we define a modification to our algorithm to address the barren plateau issue and provide more simulations.  Please note that equation numbers, sections, and other links that are in plaintext refer to the main text.

\large 

\section{Proof of Theorem~\ref{theorem:qip-eb_msf}}
\label{appendix:proof_swap-test-eb-channel}
In this appendix, we prove Theorem~\ref{theorem:qip-eb_msf}, showing that the acceptance probability of the first test of separability for mixed states is equal to $\frac{1}{2}\left(1+ F_{s}(\rho_{AB})\right)$.
\bigskip 

    \begin{proof}[Proof of Theorem~\ref{theorem:qip-eb_msf}]
    Recall that an entanglement-breaking channel can be rewritten as
    \begin{equation}
        \mathcal{E}_{R\rightarrow A^{\prime}}(\cdot)=\sum_{x}\operatorname{Tr}[\mu^{x}_{R}(\cdot)]\phi^{x}_{A^{\prime}},
        \label{eq:eb-proof-1}
    \end{equation}
    where $\{\mu^{x}_{R}\}_{x}$ is a rank-one POVM and $\{\phi^{x}_{A^{\prime}}\}_{x}$ is a set of pure states. Then we find, for fixed~$\mathcal{E}_{R\rightarrow A^{\prime}}$, that%
    \begin{align}
         \operatorname{Tr}[\Pi_{A^{\prime}A}^{\operatorname{sym}}\mathcal{E}_{R\rightarrow A^{\prime}}(\psi_{RAB})] 
        &  =\frac{1}{2}\operatorname{Tr}[(I_{A^{\prime}A}+F_{A^{\prime}A})\mathcal{E}_{R\rightarrow A^{\prime}}(\psi_{RAB})]\\
        &=\frac{1}{2}\left(  1+\operatorname{Tr}[F_{A^{\prime}A}\mathcal{E}_{R\rightarrow A^{\prime}}(\psi_{RAB})]\right).
    \end{align}
    So let us work with the expression $\operatorname{Tr}[F_{A^{\prime}A}\mathcal{E}_{R\rightarrow A^{\prime}}(\psi_{RAB})]$. Consider that%
    \begin{align}
          \operatorname{Tr}[F_{A^{\prime}A}\mathcal{E}_{R\rightarrow A^{\prime}}(\psi_{RAB})]
        &  =\operatorname{Tr}\!\left[  F_{A^{\prime}A}\sum_{x}\operatorname{Tr}_{R}[\mu^{x}_{R}\psi_{RAB}]\otimes\phi^{x}_{A^{\prime}}\right]  \\
        &  =\operatorname{Tr}\!\left[  F_{A^{\prime}A}\sum_{x}p(x)\psi_{AB}^{x}\otimes\phi^{x}_{A^{\prime}}\right]  \\
        &  =\operatorname{Tr}\!\left[  F_{A^{\prime}A}\sum_{x}p(x)\psi_{A}^{x}\otimes\phi^{x}_{A^{\prime}}\right]  \\
        &  =\sum_{x}p(x)\langle\phi^{x}|_{A}\psi_{A}^{x}|\phi^{x}\rangle_{A},
    \end{align}
    where
    \begin{align}
        p(x) &  \coloneqq \operatorname{Tr}[\mu^{x}_{R}\psi_{RAB}], \label{eq:eb-proof-2} \\
        \psi_{AB}^{x} &  \coloneqq \frac{1}{p(x)}\operatorname{Tr}_{R}[\mu^{x}_{R}\psi_{RAB}].
        \label{eq:eb-proof-3}
    \end{align}
    Thus, the acceptance probability for a fixed entanglement-breaking channel is given by
    \begin{equation}
        \operatorname{Tr}[\Pi_{A^{\prime}A}^{\operatorname{sym}}\mathcal{E}_{R\rightarrow A^{\prime}}(\psi_{RAB})]=\frac{1}{2}\left(  1+\sum_{x}p(x)\langle\phi^{x}|_{A}\psi_{A}^{x}|\phi^{x}\rangle_{A}\right)  .
    \end{equation}
    After optimizing over every element of $\operatorname{EB}_{R\to A^\prime}$, which denotes the set of all entanglement-breaking channels with input system $R$ and output system $A^\prime$, and realizing that optimizing over measurements in $\mathcal{E}_{R\rightarrow A^{\prime}}$ induces a pure-state decomposition of $\rho_{AB}$ and optimizing over preparation channels in $\mathcal{E}_{R\rightarrow A^{\prime}}$ gives the spectral norm of $\psi _{A}^{x}$, we find the claimed formula for the acceptance probability, when combined with the development in Appendices~\ref{appendix:proof_alt_streltsov} and \ref{appendix:proof_max-sep-fid-inf-norm}:%
    \begin{equation}\label{eqn:accept_prob_supp}
            \max_{\mathcal{E}\in\operatorname{EB}_{R\to A^\prime}}\operatorname{Tr}[(\Pi_{A^{\prime}A}^{\operatorname{sym}}\otimes I_{RB})\mathcal{E}_{R\rightarrow A^{\prime}}(\psi_{RAB})]=\frac{1 + F_{s}(\rho_{AB})}{2}.
        \end{equation}
    This concludes the proof.
    \end{proof}


\section{Alternative 
Proof of Equation~\eqref{eq:convex-decomp-max-sep-fid}}
\label{appendix:proof_alt_streltsov}

    This appendix provides an alternative proof for Theorem 1 in \cite{Streltsov2010}. This proof relies on Uhlmann's theorem \cite{Uhlmann1976}, the triangle inequality, and the Cauchy--Schwarz inequality. See also \cite[Lemma~1]{Regula2018}.
    
    \begin{theorem}[\cite{Streltsov2010}]\label{theorem:Streltsov}
        The following formula holds 
        \begin{equation}
            F_{s}(\rho_{AB})= \max_{\left\{(p(x),\psi_{AB}^{x})\right\}  _{x}}\left\{\sum_{x}p(x)F_{s}(\psi_{AB}^{x}):\rho_{AB}=\sum_{x}p(x)\psi_{AB}^{x}\right\},
        \end{equation}
        where $\{(p(x),\psi_{AB}^{x})\}  _{x}$ satisfies $\sum_{x}p(x)\psi_{AB}^{x}=\rho_{AB}$, all $\psi_{AB}^{x}$ are pure, and
        \begin{equation}
        F_{s}(\psi_{AB}) = \max_{|\phi\rangle_{A} , |\varphi\rangle_{B}} |\langle\psi|_{AB} |\phi\rangle_{A}\otimes|\varphi\rangle_{B}|^2. 
        \end{equation}
        
    \end{theorem}
    
    \begin{proof}
    Since the definition in \eqref{eq:max-sep-fid-def} requires an optimization over all separable states, we take $\left\vert \mathcal{X}\right\vert =\left(\left\vert A\right\vert \left\vert B\right\vert\right)^{2}$. The separable state in \eqref{eqn:sep-state-informal} is purified by%
    \begin{equation}
    |\psi^{\sigma}\rangle_{RAB}=\sum_{x\in\mathcal{X}}\sqrt{p(x)}|x\rangle_{R}|\psi^{x}\rangle_{A}|\phi^{x}\rangle_{B}.\label{eq:sep-purify}%
    \end{equation}
    Now consider a generic purification $|\psi^{\rho}\rangle_{R^{\prime}AB}$\ of $\rho_{AB}$. Recall that the dimension of the purifying system $R^{\prime}$ satisfies rank$(\rho_{AB})\leq\left\vert R^{\prime}\right\vert $ and so we can simply set $\left\vert R^{\prime}\right\vert =\left\vert A\right\vert\left\vert B\right\vert $. Taking $R^{\prime\prime}$ to be a system of dimension $\left\vert A\right\vert \left\vert B\right\vert $, we then have that%
    \begin{equation}
        |\psi^{\rho}\rangle_{R^{\prime}AB}|0\rangle_{R^{\prime\prime}}%
    \end{equation}
    purifies $\rho_{AB}$. Applying Uhlmann's theorem \cite{Uhlmann1976}, the maximum separable root fidelity can be written as%
    \begin{multline}
        \max_{\sigma_{AB}\in\operatorname{SEP}(A:B)}\sqrt{F}(\rho_{AB},\sigma_{AB})
        = \\
        \max_{\substack{U,\\\left\{\left(  p(x),\psi^{x}{}_{A},\phi_{B}^{x}\right)  \right\}_{x}}} \left\vert \left(  \sum_{x^{\prime}} \sqrt{p(x^{\prime})}\langle x^{\prime}|_{R} \langle \psi^{x^{\prime}}|_{A} \langle \phi^{x^{\prime}} |_{B}\right)  \left(  U_{R^{\prime}R^{\prime\prime}\rightarrow R}\otimes I_{AB}\right) |\psi^{\rho}\rangle_{R^{\prime}AB}|0\rangle_{R^{\prime\prime}}\right\vert ,
    \end{multline}
    where the maximization is over every unitary $U_{R^{\prime}R^{\prime\prime}\rightarrow R}$. Expanding $U_{R^{\prime}R^{\prime\prime}\rightarrow R}|\psi^{\rho}\rangle_{R^{\prime}AB}|0\rangle_{R^{\prime\prime}}$ in terms of the standard basis $|x\rangle$ as%
    \begin{equation}
        U_{R^{\prime}R^{\prime\prime}\rightarrow R}|\psi^{\rho}\rangle_{R^{\prime}AB}|0\rangle_{R^{\prime\prime}}=\sum_{x\in\mathcal{X}}\sqrt{q(x)}|x\rangle_{R}|\varphi^{x}\rangle_{AB},
    \end{equation}
    we note that $U$ followed by a measurement in the standard basis induces a convex decomposition of $\rho_{AB}$ in terms of the ensemble $\{(q(x),\varphi_{AB}^{x})\}_{x}$. We can write the root fidelity as 
    \begin{align}
        \max_{\sigma_{AB}\in\operatorname{SEP}(A:B)}\sqrt{F}(\rho_{AB},\sigma_{AB})\nonumber
        &=\max_{\substack{\left\{  \left(  p(x),\psi^{x}{}_{A},\phi_{B}^{x}\right)\right\}  _{x},\\\{(q(x),\varphi_{AB}^{x})\}_{x}}} \left\vert \sum_{x,x^{\prime}}\-\sqrt{q(x)p(x^{\prime})}\langle x|x^{\prime}\rangle_{R}\langle\varphi^{x}|_{AB}|\psi^{x^{\prime}}\rangle_{A}|\phi^{x^{\prime}}\rangle_{B}\right\vert \\
        &=\max_{\substack{\left\{  \left(  p(x),\psi^{x}{}_{A},\phi_{B}^{x}\right)\right\}  _{x},\\\{(q(x),\varphi_{AB}^{x})\}_{x}}}\left\vert \sum_{x}\sqrt{q(x)p(x)}\langle\varphi^{x}|_{AB}|\psi^{x}\rangle_{A}|\phi^{x}\rangle_{B}\right\vert \\
        &=\max_{\substack{\left\{  \left(  p(x),\psi^{x}{}_{A},\phi_{B}^{x}\right)\right\}  _{x},\\\{(q(x),\varphi_{AB}^{x})\}_{x}}}\left\vert \sum_{x}\sqrt{q(x)p(x)}\langle\varphi^{x}|_{AB}|\psi^{x}\rangle_{A}|\phi^{x}\rangle_{B}\right\vert .
    \end{align}
    Next, for fixed $\left\{  \left(  p(x),\psi^{x}{}_{A},\phi_{B}^{x}\right)\right\}  _{x}$ and $\{(q(x),\varphi_{AB}^{x})\}_{x}$, we bound the objective function in the optimization above as follows:%
    \begin{align}
        \left\vert \sum_{x}\sqrt{q(x)p(x)}\langle\varphi^{x}|_{AB}|\psi^{x}\rangle_{A}|\phi^{x}\rangle_{B}\right\vert &\leq\sum_{x}\sqrt{p(x)q(x)}\left\vert \langle\varphi^{x}|_{AB}|\psi
    ^{x}\rangle_{A}|\phi^{x}\rangle_{B}\right\vert \\
    &\leq\sqrt{\sum_{x}p(x)}\sqrt{\sum_{x}q(x)\left\vert \langle\varphi
    ^{x}|_{AB}|\psi^{x}\rangle_{A}|\phi^{x}\rangle_{B}\right\vert ^{2}}\\
    &=\sqrt{\sum_{x}q(x)\left\vert \langle\varphi^{x}|_{AB}|\psi^{x}\rangle
    _{A}|\phi^{x}\rangle_{B}\right\vert ^{2}}.
    \end{align}
    The first inequality follows from the triangle inequality, and the second from an application of Cauchy--Schwarz.\ We see that equality is achieved in the second inequality by choosing
    \begin{equation}
        p(x)=\frac{q(x)\left\vert \langle\varphi^{x}|_{AB}|\psi^{x}\rangle_{A}|\phi^{x}\rangle_{B}\right\vert ^{2}}{\sum_{x}q(x)\left\vert \langle\varphi^{x}|_{AB}|\psi^{x}\rangle_{A}|\phi^{x}\rangle_{B}\right\vert ^{2}}.
    \end{equation}
    We can achieve equality in the first inequality by tuning a global phase for the state $|\psi^{x}\rangle_{A}$, which amounts to a relative phase in \eqref{eq:sep-purify}. Putting everything together, we conclude that
    \begin{equation}
        \max_{\sigma_{AB}\in\operatorname{SEP}}F(\rho_{AB},\sigma_{AB})=\max_{\{(q(x),\varphi_{AB}^{x})\}_{x}}\sum_{x}q(x)\max_{\left(  |\psi^{x}\rangle_{A}|\phi^{x}\rangle_{B}\right)  _{x},}\left\vert \langle\varphi^{x}|_{AB}|\psi^{x}\rangle_{A}|\phi^{x}\rangle_{B}\right\vert ^{2},
    \end{equation}
    which is equivalent to the desired equality in \eqref{eq:convex-decomp-max-sep-fid}.
\end{proof}

\section{Proof of Equation~\eqref{eq:max-sep-fid-inf-norm}}
\label{appendix:proof_max-sep-fid-inf-norm}
    
In this appendix, we show that the fidelity of separability of a bipartite state can be written in terms of the spectral norm, which was also observed in \cite[Section~6.2]{Streltsov2010}.
    \begin{proposition}
    \label{prop:max-sep-fid-inf-norm}
        For a bipartite state, the following equality holds
        \begin{equation}
            F_{s}(\rho_{AB})= \max_{\left\{(p(x),\psi_{AB}^{x})\right\}  _{x}}\left\{\sum_{x}p(x)\left\Vert \psi_{A}^{x}\right\Vert _{\infty}:\rho_{AB}=\sum_{x}p(x)\psi_{AB}^{x}\right\}.
        \end{equation}
    \end{proposition}
    
    \begin{proof}
    Consider that the following holds for a pure bipartite state $\psi_{AB}$:
    \begin{align}
        F_{s}(\psi_{AB}) &  =\max_{|\phi\rangle_{A},|\varphi\rangle_{B}}\left\vert\langle\psi|_{AB}|\phi\rangle_{A}\otimes|\varphi\rangle_{B}\right\vert ^{2}\\
        &=\max_{|\phi\rangle_{A},|\varphi\rangle_{B}}\left\vert \langle\phi|_{A}\otimes\langle\varphi|_{B}|\psi\rangle_{AB}\right\vert ^{2}\\
        &=\max_{|\phi\rangle_{A}}\left\Vert \langle\phi|_{A}\otimes I_{B}|\psi\rangle_{AB}\right\Vert _{2}^{2}\\
        &=\max_{|\phi\rangle_{A}}\operatorname{Tr}[(|\phi\rangle\!\langle\phi|_{A}\otimes I_{B})\psi_{AB}]\\
        &=\max_{|\phi\rangle_{A}}\operatorname{Tr}[|\phi\rangle\!\langle\phi|_{A}\psi_{A}]\\
        &=\left\Vert \psi_{A}\right\Vert _{\infty}.
    \end{align}
    The first two equalities follow from the definition and a rewriting. The third equality follows from the variational characterization of the Euclidean norm of a vector. The fourth equality follows because
    \begin{align}
         \left\Vert \langle\phi|_{A}\otimes I_{B}|\psi\rangle_{AB}\right\Vert _{2}^{2} 
        & =\left(\langle\psi|_{AB}|\phi\rangle_{A}\otimes I_{B}\right)\left(\langle\phi|_{A}\otimes I_{B}|\psi\rangle_{AB}\right)  \\
        &=\langle\psi|_{AB}|\phi\rangle\!\langle\phi|_{A}\otimes I_{B}|\psi\rangle_{AB}\\
        & =\operatorname{Tr}[(|\phi\rangle\!\langle\phi|_{A}\otimes I_{B})\psi_{AB}].
    \end{align}
    The next step follows by taking a partial trace and the final equality from the variational characterization of the spectral norm. So this implies the desired equality after applying \eqref{eq:convex-decomp-max-sep-fid}.
    \end{proof}
    
\section{Proof of Theorem~\ref{theorem:global_cost_func}}
\label{appendix:step_by_step}
    In this appendix, we show that the acceptance probability of our VQSA is indeed equal to $F_s(\rho_{AB})$ if the parameterized unitary circuits can express all possible unitary operators of their respective systems. For this, let us track the state of the VQSA at the points indicated in Figure~\ref{fig:Max_Sep_Fidelity_costfun_proof}. 
    
    \begin{itemize}
        \item At Step $(1)$, the unitary $U^\rho$ prepares the pure state $\psi_{RAB}$. This is a specific initial purification of $\rho_{AB}$.
        
        \item At Step $(2)$, we apply the parameterized unitary circuit $W_R(\Theta)$ to $\psi_{RAB}$. Expanding $W_R(\Theta)|\psi^{\rho}\rangle_{RAB}$ in terms of the standard basis $\{|x\rangle\}_x$ leads to
            \begin{equation}
                W_R(\Theta)|\psi\rangle_{RAB}=\sum_{x\in\mathcal{X}}\sqrt{q(x)}|x\rangle_{R}|\varphi^{x}\rangle_{AB}.
            \end{equation}
            
        \item At Step $(3)$, the measurement outcome $x$ occurs with probability $q(x)$, and the state vector of registers $A$ and $B$ becomes  $|\varphi^{x}\rangle_{AB}$.
        
        \item At Step $(4)$, depending on the measurement outcome $x$, we apply the parameterized unitary circuit $U^x_A(\Theta^x)$ to register $A$. The state vector is now $U^x_A(\Theta^x)|\varphi^{x}\rangle_{AB}$.
        
        \item At Step $(5)$, we trace over $B$ and measure $A$ in the standard basis. We accept when we get the all-zeros outcome. The acceptance probability is then equal to
        \begin{equation}
            \sum_{x\in\mathcal{X}}q(x)\, \langle0|\,U^x_A(\Theta^x)\varphi^{x}_{A}\left(U^x_A\right)^\dagger|0\rangle
            =\sum_{x\in\mathcal{X}}q(x)\,\langle\phi^x|_A\varphi^{x}_{A}|\phi^x\rangle_A,
        \end{equation}
        where we have defined $|\phi^x\rangle_A \coloneqq \left(U^x_A\right)^\dagger|0\rangle$.
        
        \item Maximizing the acceptance probability corresponds to maximization over the parameters of $W_R(\Theta)$ and~$U^x_A(\Theta^x)$. 
        
        \item Maximization over the parameters of $W_R$ is a maximization over all possible pure-state decompositions of~$\rho_{AB}$. 
        
        \item Maximization over the parameters of $U^x_A(\Theta^x)$ is a maximization of $\langle\phi^x|\varphi^{x}_{A}|\phi^x\rangle$, which yields the value of~$\left\Vert \varphi_{A}^{x}\right\Vert _{\infty}$.
        
        \item The maximum acceptance probability is equal to
        \begin{equation}
             \max_{\left\{(p(x),\psi_{AB}^{x})\right\}  _{x}}\left\{\sum_{x}p(x)\left\Vert \varphi_{A}^{x}\right\Vert _{\infty}:\rho_{AB}=\sum_{x}p(x)\psi_{AB}^{x}\right\} ,
        \end{equation}
        which is in turn equal to $F_s(\rho_{AB})$, by Proposition~\ref{prop:max-sep-fid-inf-norm}.
    \end{itemize} 
    
    This proves that if the parameterized unitary circuits can express all possible unitary operators of their respective systems, the maximum acceptance probability equals $F_s(\rho_{AB})$. However, we note that any ansatz employed for the parameterized unitary circuits has limited expressibility. As such, the maximum acceptance probability obtained via the VQSA will also be closer to the actual value of $F_s(\rho_{AB})$ if we use a more expressive ansatz. 

    \begin{figure}
        \centering
        \includegraphics[width=0.6\columnwidth]{figures/fig_VQA_proof_new.pdf}
        \caption{VQSA to estimate the fidelity of separability $F_s(\rho_{AB})$. 
        The unitary circuit $U^\rho$ produces the state $\psi_{RAB}$, which is a purification of $\rho_{AB}$. The parameterized circuit $W_R(\Theta)$ acts on $R$ to evolve $\psi_{RAB}$ to another pure-state decomposition of $\rho_{AB}$. The following measurement steers the system $AB$ to be in a pure state $\psi_{AB}^{x}$ if the measurement outcome $x$ occurs. Conditioned on the outcome $x$, the final parameterized circuit $U^{x}_{A}(\Theta^x)$ and the subsequent measurement estimates~$\left\|\psi_{A}^{x}\right\|_{\infty}$.}
        \label{fig:Max_Sep_Fidelity_costfun_proof}
    \end{figure}

\section{First Benchmarking SDP \texorpdfstring{$\widetilde{F}_s^1$}{Lg} and Proof of Equation~\eqref{eq:first-sdp-bounds}}

\label{appendix:ppt-k-state-sdp}

     This appendix details the derivation of our first benchmarking SDP $\widetilde{F}_s^1$, based on the SDP for fidelity \cite{Wat13}. Let $\rho_{AB}$ and $\sigma_{AB}$ be bipartite states. The SDP for the root fidelity $\sqrt{F}(\rho_{AB},\sigma_{AB})$, which makes use of Uhlmann's theorem~\cite{Uhlmann1976}, is as follows:
    \begin{equation}
    \label{eqn:fidelity_sdps}
        \sqrt{F}(\rho_{AB},\sigma_{AB}) = \max_{\substack{X_{AB} \in\mathcal{L}(\mathcal{H}_{AB})}}
        \left\{\begin{array}
                [c]{c}%
                \operatorname{Re}[\operatorname{Tr}[X_{AB}]]:
                \begin{bmatrix}
                \rho_{AB} & X_{AB}\\
                X_{AB}^{\dag} & \sigma_{AB}%
                \end{bmatrix}
                \geq0 
        \end{array}\right\},
    \end{equation}
    where $\mathcal{L}(\mathcal{H}_{AB})$ is the set of all linear operators acting on the Hilbert space $\mathcal{H}_{AB}$.

    We would ideally like to include a maximization over the set of all separable states, but it is well known to be computationally challenging to optimize over this set \cite{G03, Gharibian2010}. Note that it is not generally possible to characterize the set of separable states using semi-definite constraints \cite{Fawzi2021}. Instead, we can only approximate the set by constraining $\sigma_{AB}$ to have a positive partial transpose (PPT) \cite{Peres1996, Horodecki1996} and be $k$-extendible \cite{W89a, Doherty2004}, since all separable states satisfy these constraints. Let $\widetilde{F}_s^1(\rho_{AB})$ denote the resulting quantity, the square root of which is defined as follows:
    \begin{equation}
        \sqrt{\widetilde{F}_s^1}(\rho_{AB}, k) \coloneqq 
        \max_{\substack{X_{AB} \in\mathcal{L}(\mathcal{H}_{AB}),\\\sigma_{AB^{k}}\geq0}}
        \left\{\begin{array}
                [c]{c}
                \operatorname{Re}[\operatorname{Tr}[X_{AB}]]:\\%
                \begin{bmatrix}
                \rho_{AB} & X_{AB}\\
                X_{AB}^{\dag} & \sigma_{AB_{1}}%
                \end{bmatrix}
                \geq0,\\
                \operatorname{Tr}[\sigma_{AB^{k}}]=1,\\
                \sigma_{AB^{k}}=\mathcal{P}_{B^{k}}(\sigma_{AB^{k}}),\\ 
                T_{B_{1\cdots j}}(\sigma_{AB_{1\cdots j}})\geq 0 \quad \forall j\leq k
        \end{array}\right\} ,
        \label{eq:benchmark-sdp-1}
    \end{equation}
    where $B^{k}\equiv B_1 \cdots B_k$, the notation $T_R$ denotes the partial transpose map acting on system $R$, and $\mathcal{P}_{B^{k}}$ denotes the channel that performs a uniformly random permutation of systems $B_1$ through~$B_k$.
    
    We prove the inequalities in \eqref{eq:first-sdp-bounds}.
    Due to the containment discussed above, we note that
    \begin{equation}
        F_s(\rho_{AB}) \leq \widetilde{F}_s^1(\rho_{AB}, k).
    \end{equation}
    An opposite bound on $\widetilde{F}_s^1(\rho_{AB}, k)$ in terms of $F_s(\rho_{AB})$ is as follows:
    \begin{equation}
\sqrt{1-F_{s}(\rho_{AB})}\leq
\sqrt{1-\widetilde{F}_{s}^{1}(\rho_{AB}, k)}
+2\sqrt{\frac{\left\vert B\right\vert ^{2}}{k}\left(  1-\frac{\left\vert
B\right\vert ^{2}}{k}\right)  },
\label{eq:sdp-benchmark-1-good-approx}
\end{equation}
which can be rewritten as
\begin{equation}
    \widetilde{F}_{s}^{1}(\rho_{AB}, k)
    \leq 1- \left[\sqrt{1-F_{s}(\rho_{AB})} -2\sqrt{\frac{\left\vert B\right\vert ^{2}}{k}\left(  1-\frac{\left\vert
B\right\vert ^{2}}{k}\right)  } \right]^2.
\end{equation}
It is a consequence of \cite[Theorem~II.7]{christandl2007one}, the triangle inequality for sine distance \cite{R06}, and  the Fuchs-van-de-Graaf inequalities \cite{FvG99}.
    Indeed, consider that%
\begin{align}
\widetilde{F}_{s}^{1}(\rho_{AB}, k)  & =\max_{\sigma_{AB}\in\text{EXT-PPT}_{k}%
}F(\rho_{AB},\sigma_{AB})
\label{eq:relaxed-benchmark-1a}\\
& \leq\max_{\sigma_{AB}\in\text{EXT}_{k}}F(\rho_{AB},\sigma_{AB}),
\label{eq:relaxed-benchmark-1}
\end{align}
where EXT-PPT$_{k}$ denotes the set being optimized over in \eqref{eq:benchmark-sdp-1} and
EXT$_{k}$ is the set of $k$-extendible states. Now recall that for all $\omega^k_{AB} \in \operatorname{EXT}_{k}$ \cite[Theorem~II.7']{christandl2007one}
\begin{equation}
\min_{\sigma_{AB}\in\operatorname{SEP}(A:B)}\frac{1}{2}\left\Vert \omega
_{AB}^{k}-\sigma_{AB}\right\Vert _{1}\leq\frac{2\left\vert B\right\vert ^{2}%
}{k},
\end{equation}
the sine distance obeys the triangle inequality \cite{R06}:%
\begin{equation}
\sqrt{1-F(\omega,\tau)}\leq\sqrt{1-F(\omega,\xi)}+\sqrt{1-F(\xi,\tau)},
\end{equation}
and the Fuchs-van-de-Graaf inequality \cite{FvG99}:%
\begin{equation}
1-\sqrt{F}(\omega,\tau)\leq\frac{1}{2}\left\Vert \omega-\tau\right\Vert _{1},
\end{equation}
where $\omega$, $\tau$, and $\xi$ are states. If $\frac{1}{2}\left\Vert
\omega-\tau\right\Vert _{1}\leq\varepsilon$, the latter implies that
\begin{equation}
1-\sqrt{F}(\omega,\tau)   \leq\varepsilon 
\ \Leftrightarrow \ \sqrt{1-F(\omega,\tau)}  \leq\sqrt{\varepsilon\left(
2-\varepsilon\right)  } .
\end{equation}
Letting $\sigma_{AB}^{k}$ be an optimal choice in \eqref{eq:relaxed-benchmark-1} and $\sigma_{AB}'$ an
optimal choice for $\min_{\sigma_{AB}\in\operatorname{SEP}(A:B)}\frac{1}%
{2}\left\Vert \omega_{AB}^{k}-\sigma_{AB}\right\Vert _{1}$, this implies that%
\begin{align}
\min_{\sigma_{AB}\in\operatorname{SEP}(A:B)}\sqrt{1-F(\rho_{AB},\sigma
_{AB})}
& \leq\sqrt{1-F(\rho_{AB},\sigma_{AB}^{\prime})}\\
& \leq\sqrt{1-F(\rho_{AB},\sigma_{AB}^{k})}+\sqrt{1-F(\sigma_{AB}^{\prime
},\sigma_{AB}^{k})}\\
& \leq\sqrt{1-F(\rho_{AB},\sigma_{AB}^{k})}+2\sqrt{\frac{\left\vert
B\right\vert ^{2}}{k}\left(  1-\frac{\left\vert B\right\vert ^{2}}{k}\right)
}.
\end{align}
Rearranging and applying \eqref{eq:relaxed-benchmark-1a}--\eqref{eq:relaxed-benchmark-1}, we arrive at the claimed inequality in \eqref{eq:sdp-benchmark-1-good-approx}.

\section{Second Benchmarking SDP \texorpdfstring{$\widetilde{F}_s^2$}{Lg}
and Proof of Equation \eqref{eqn:swap-test-ppt-k-channel-sdp}}
\label{appendix:proof_swap-test-ppt-k-channel-sdp}

    In this appendix, we detail the derivation of our second benchmark SDP $\widetilde{F}_s^2$, which is an SDP that approximates \eqref{eqn:qip-eb_accept_prob} in the main text. Consider a version of the distributed quantum computation that led to \eqref{eqn:qip-eb_accept_prob} where, instead of restricting the prover to only entanglement-breaking channels, we insist that the prover sends back $k$ systems labeled as $A_{1}\cdots A_{k}$. Then, the verifier randomly selects one of the $k$ systems and performs a swap test on the $A$ system of the state $\psi_{RAB}$. This random selection is conducted so that the prover output is effectively reduced to that of an approximate entanglement-breaking channel. Note that the resulting interactive proof is in QIP(2). More specifically, the acceptance probability of this interactive proof system is given by
    \begin{equation}\label{eqn:part_prep_channel}
       \max_{\mathcal{P}_{R\rightarrow A_{1}^{\prime}\cdots A_{k}^{\prime}}}\operatorname{Tr}[\Pi_{A^{\prime}A}^{\operatorname{sym}}\overline{\mathcal{P}}_{R\rightarrow A^{\prime}}(\psi_{RAB})],
    \end{equation}
    where
    \begin{equation}\label{eqn:whole_prep_channel}
        \overline{\mathcal{P}}_{R\rightarrow A^{\prime}}\coloneqq \frac{1}{k}\sum_{i=1}^{k}\operatorname{Tr}_{A_{1}^{k\prime}\backslash A_{i}}\circ\mathcal{P}_{R\rightarrow A_{1}^{\prime}\cdots A_{k}^{\prime}},
    \end{equation}
    and $\mathcal{P}$ is a preparation channel. Observing that $\overline{\mathcal{P}}_{R\rightarrow A^{\prime}}$ is a $k$-extendible channel \cite{PBHS13, Kaur2018, KDWW21, BBFS18}, it follows that%
    \begin{align}
    \label{eqn:part_prep_chan_k-ext}
        \max_{\mathcal{P}_{R\rightarrow A_{1}^{\prime}\cdots A_{k}^{\prime}}}\operatorname{Tr}[\Pi_{A^{\prime}A}^{\operatorname{sym}}\overline{\mathcal{P}}_{R\rightarrow A^{\prime}}(\psi_{RAB})] 
        &=\max_{\mathcal{E}_{R\rightarrow A^{\prime}}^{k}\in\text{EXT}_{k}}\operatorname{Tr}[\Pi_{A^{\prime}A}^{\operatorname{sym}}\mathcal{E}_{R\rightarrow A^{\prime}}^{k}(\psi_{RAB})]
        \\
        &\eqqcolon \frac{1}{2} (1+\widetilde{F}_s^2(\rho_{AB}, k)),
    \end{align}
    where EXT$_{k}$ denotes the set of $k$-extendible channels. These are defined by $\mathcal{E}_{R\rightarrow A^{\prime}}^{k}(\rho_{SR})\in$ EXT$_{k}(S\!:\!A^{\prime})$ for every input state $\rho_{SR}$, where EXT$_{k}(S\!:\!A^{\prime})$ denotes the set of $k$-extendible states. Hence, we estimate \eqref{eqn:qip-eb_accept_prob} using $\widetilde{F}_s^2(\rho_{AB}, k)$ and it is given by the following SDP: 
    \begin{equation}
    \frac{1}{2}(1+\widetilde{F}_s^2(\rho_{AB}, k)) =
    \max_{\Gamma^{\mathcal{E}^{k}}_{RA^{\prime k}}\geq 0}
    \left\{\begin{array}
            [c]{c}%
            \operatorname{Tr}[\Pi_{A^{\prime}A}^{\operatorname{sym}} \operatorname{Tr}_{R}[T_R(\psi_{RAB})\Gamma^{\mathcal{E}^{k}}_{RA^{\prime}_1}]]:\\%
            \operatorname{Tr}_{A^{\prime k}}[\Gamma^{\mathcal{E}^{k}}_{RA^{\prime k}}]=I_R,\\
            \Gamma^{\mathcal{E}^{k}}_{RA^{\prime k}}=\mathcal{P}_{A^{\prime k}}(\Gamma^{\mathcal{E}^{k}}_{RA^{\prime k}}),\\ 
            T_{A^{\prime}_{1\cdots j}}(\Gamma^{\mathcal{E}^{k}_{RA^{\prime k}}}) \geq 0 \quad \forall j\leq k
    \end{array}\right\} ,
    \end{equation}
    where $\Gamma^{\mathcal{E}^{k}}_{RA^{\prime}}$ is the Choi operator of $\mathcal{E}^{k}$, the map $T_R$ is the partial transpose map acting on system~$R$, and $\mathcal{P}_{A^{\prime k}}$ is the channel that randomly permutes the systems $A^{\prime k}$.
    
    The following theorem indicates how $\widetilde{F}_s^2$ approximates $F_{s}(\rho_{AB})$.

    \begin{proposition}
    \label{prop:swap-test-ppt-k-channel-sdp}
    The following bound holds for a bipartite state~$\rho_{AB}$:
        \begin{equation}
            F_{s}(\rho_{AB}) \leq \widetilde{F}_s^2(\rho_{AB}, k) \leq 
            F_{s}(\rho_{AB})+\frac{4  \left\vert A\right\vert^{3} \left\vert B\right\vert}{k}.
        \end{equation}
    \end{proposition}
    \begin{proof}
    Since every entanglement-breaking channel is $k$-extendible, we trivially find that%
    \begin{align}
        \frac{1+F_{s}(\rho_{AB})}{2} &=\max_{\mathcal{E}\in\operatorname{EB}}\operatorname{Tr}[(\Pi_{A^{\prime}A}^{\operatorname{sym}}\otimes I_{RB})\mathcal{E}_{R\rightarrow A^{\prime}}(\psi_{RAB})]\\
        & \leq\max_{\mathcal{E}_{R\rightarrow A^{\prime}}^{k}\in\text{EXT}_{k}}\operatorname{Tr}[(\Pi_{A^{\prime}A}^{\operatorname{sym}}\otimes I_{RB})\mathcal{E}_{R\rightarrow A^{\prime}}^{k}(\psi_{RAB})]
        \\
        & = \frac{1+\widetilde{F}_s^2(\rho_{AB}, k)}{2}.
    \end{align}
    Consider the following bound for a $k$-extendible state $\omega_{AB}^{k}$ \cite[Theorem~II.7']{christandl2007one}:%
    \begin{equation}
        \min_{\sigma_{AB}\in\operatorname{SEP}(A:B)}\frac{1}{2}\left\Vert\omega_{AB}^{k}-\sigma_{AB}\right\Vert _{1}\leq\frac{2\left\vert B\right\vert ^{2}}{k}.
    \end{equation}
    We can use it and the result of \cite[Lemma~7]{Wallman_2014} to conclude that
    \begin{equation}
        \min_{\mathcal{E}\in\operatorname{EB}}\frac{1}{2}\left\Vert \mathcal{E}^{k}-\mathcal{E}\right\Vert _{\diamond}\leq\frac{2\left\vert R\right\vert\left\vert A^{\prime}\right\vert ^{2}}{k}.
    \end{equation}
    Then consider, for every fixed choice of $\mathcal{E}_{R\rightarrow A^{\prime}}^{k}$, there exists an entanglement-breaking channel $\mathcal{E}$ satisfying%
    \begin{equation}
        \frac{1}{2}\left\Vert \mathcal{E}^{k}-\mathcal{E}\right\Vert _{\diamond}\leq\frac{2\left\vert R\right\vert \left\vert A^{\prime}\right\vert ^{2}}{k}.
    \end{equation}
    Then we find that%
    \begin{align}
        & \operatorname{Tr}[(\Pi_{A^{\prime}A}^{\operatorname{sym}}\otimes I_{RB})\mathcal{E}_{R\rightarrow A^{\prime}}^{k}(\psi_{RAB})]\notag\\
        &\leq\operatorname{Tr}[(\Pi_{A^{\prime}A}^{\operatorname{sym}}\otimes I_{RB})\mathcal{E}_{R\rightarrow A^{\prime}}(\psi_{RAB})]+\frac{2\left\vert R\right\vert \left\vert A^{\prime}\right\vert ^{2}}{k}\\
        &\leq\max_{\mathcal{E}\in\operatorname{EB}}\operatorname{Tr}[(\Pi_{A^{\prime}A}^{\operatorname{sym}}\otimes I_{RB})\mathcal{E}_{R\rightarrow A^{\prime}}(\psi_{RAB})]+\frac{2\left\vert R\right\vert \left\vert A^{\prime}\right\vert ^{2}}{k}\\
        &=\frac{1+F_{s}(\rho_{AB})}{2}+\frac{2\left\vert R\right\vert \left\vert A^{\prime}\right\vert ^{2}}{k}.
    \end{align}
    Since the inequality holds for every $\mathcal{E}_{R\rightarrow A^{\prime}}^{k}\in$ EXT$_{k}$, it follows that%
    \begin{equation}
        \max_{\mathcal{E}_{R\rightarrow A^{\prime}}^{k}\in\text{EXT}_{k}}\operatorname{Tr}[(\Pi_{A^{\prime}A}^{\operatorname{sym}}\otimes I_{RB})\mathcal{E}_{R\rightarrow A^{\prime}}^{k}(\psi_{RAB})]
        \leq\frac{1+F_{s}(\rho_{AB})}{2}+\frac{2\left\vert R\right\vert \left\vert A^{\prime}\right\vert ^{2}}{k}.
    \end{equation}
    This concludes the proof after recalling that $|R| \leq |A| |B|$, observing that $|A|=|A'|$, and performing some simple algebra.
    \end{proof}

    \begin{remark}
    \label{rem:de-finetti-qip}
        Although the correction term in the upper bound in Proposition~\ref{prop:swap-test-ppt-k-channel-sdp} decreases with increasing $k$, it is clear that, for it to become arbitrarily small,  $k$ needs to be larger than $|A|^3|B|$, which is exponential in the number of qubits for the state $\rho_{AB}$. Thus, this approach does not lead to an efficient method for placing the fidelity of separability estimation problem in QIP(2) or even QIP.
    \end{remark}



\section{Further Simulations and Details}
\label{appendix:simulations}

%\begin{figure}
        %\centering
        %\subfigure[Fidelity of separability calculated for a three-qubit GHZ state using our VQSA (green line) converging to the true value of 0.5.]{\includegraphics[width=0.6\columnwidth]{figures/plot_ghz_state.pdf}\label{fig:Max_Sep_Fidelity_GHZ}}
        %\subfigure[Fidelity of separability calculated for a three-qubit W state using our VQSA (green line) converging to the true value of 4/9.]{\includegraphics[width=0.6\columnwidth]{figures/plot_w_state.pdf}\label{fig:Max_Sep_Fidelity_W}}
        %\caption{Fidelity of separability estimated for multipartite states using the VQSA and benchmarked known values of fidelity of separability.}
    %\end{figure}

    \begin{figure}
        \centering
        \subfigure[Fidelity of separability calculated for a random product state using the local reward function of the VQSA and benchmarked by $\widetilde{F}_s^1$.]{\includegraphics[width=0.48\columnwidth]{figures/plot_random_product_state.pdf}\label{fig:local_cost_function1}}
            \hspace*{\fill}
        \subfigure[Fidelity of separability calculated for a random entangled state using the local reward function of the VQSA and benchmarked by $\widetilde{F}_s^1$.]{\includegraphics[width=0.48\columnwidth]{figures/plot_random_entangled_state.pdf}\label{fig:local_cost_function2}}
        \caption{Fidelity of separability estimated using the local reward function of the VQSA and benchmarked by~$\widetilde{F}_s^1$.} 
    \end{figure}
    
    For all the simulations in our work, the input states and the parameterized unitaries are generated using the hardware efficient ansatz (HEA)  \cite{KMTTBCG17}. The HEA consists of several layers, where each layer consists of two parameters per qubit per layer, specifying rotations about the $x$- and $y$-axes. After each layer of rotations is a series of neighboring qubit CNOT gates. When using the HEA to generate the input states, we keep the rotation angles fixed, thus leading to a fixed input state. For the parameterized unitaries, the rotation angles are parameters and are optimized over.
    
    In Figure~\ref{fig:local_cost_function1}, we report simulation results after generating a random bipartite product state, with each partition containing two qubits. To guarantee a product state, we remove all the CNOT gates from the HEA that generates the input state $\rho$. We calculated the fidelity of separability using both the local reward function of the VQSA and the benchmark $\widetilde{F}^1_s$, the latter discussed in Appendix~\ref{appendix:ppt-k-state-sdp}. 
    
    In Figure~\ref{fig:local_cost_function2}, we do the same for a random bipartite state with the partitions $A$ and $B$ containing two qubits and one qubit, respectively, and three qubits in the reference system.  
    
    We generated all parameterized unitary circuits in the following fashion. We used the Qiskit Aer simulator and Qiskit's Simultaneous Perturbation Stochastic Approximation (SPSA) optimizer to perform the classical optimization. All other details can be found in Table~\ref{tab:details_about_fig}. The local reward function of the VQSA requires more classical processing (like picking a qubit at random to measure) and seems to require more iterations to reach the right value. However, these downsides are outweighed by the fact that it is less susceptible to the emergence of barren plateaus. More details about the local cost function can be found in Appendix~\ref{appendix:local_cost}.
    
    \renewcommand{\arraystretch}{1.25}
    \begin{table*}
    \centering
    \begin{tabular}{|c|c|c|c|}
    \hline
    Figure & No. of Qubits & State $\rho_{AB}$ & Layer Count \\
    \hline\hline
    \multirow{1}{*}[-0.8em]{\ref{fig:Max_Sep_Fidelity_Bell}} & \multirow{1}{*}[-0.8em]{$R=2$, $A=1$, $B=1$} & \multirow{1}{*}[-0.8em]{$(3/4)|\Phi^+\rangle\!\langle\Phi^+|+(1/4)|\Phi^-\rangle\!\langle\Phi^-|$} & $W_R$ no.~of layers = 2\\
    & & & $U^x_A$ no.~of layers = 2\\
    \hline
    \multirow{1}{*}[-0.8em]{\ref{fig:Max_Sep_Fidelity_Depol}} & \multirow{1}{*}[-0.8em]{$R=4$, $A=2$, $B=2$} & $(\mathcal{D}_{p, A_1}\otimes\mathcal{D}_{p,A_2}\otimes\mathbb{I}_{B})\left(|\psi \rangle\!\langle\psi|\right)$ & $W_R$ no.~of layers = 4\\
    & & $|\psi \rangle=\frac{1}{\sqrt{2}}\left(|0\rangle_{A_1} |0\rangle_{A_2} |00\rangle_{B} +|1\rangle_{A_1} |1\rangle_{A_2} |11\rangle_{B}\right)$ & $U^x_A$ no.~of layers = 4\\
    \hline
    \multirow{1}{*}[-0.8em]{\ref{fig:local_cost_function1}} & \multirow{1}{*}[-0.8em]{$R=3$, $A=2$, $B=2$} & \multirow{1}{*}[-0.8em]{Random product state using HEA \cite{KMTTBCG17}} & $W_R$ no.~of layers = 4\\
    & & & $U^x_A$ no.~of layers = 4\\
    \hline
    \multirow{1}{*}[-0.8em]{\ref{fig:local_cost_function2}} & \multirow{1}{*}[-0.8em]{$R=3$, $A=2$, $B=2$} & \multirow{1}{*}[-0.8em]{Random entangled state using HEA \cite{KMTTBCG17}} &$W_R$ no.~of layers = 4\\
    & & & $U^x_A$ no.~of layers = 4\\
    \hline
    
    \end{tabular}
    \caption{Details of all VQSA simulations.}
    \label{tab:details_about_fig}
\end{table*}
\renewcommand{\arraystretch}{1.0}

\section{Software}
\label{appendix:software}
All of our Python source files are available with the arXiv posting of this paper. We performed all simulations using the noisy Qiskit Aer simulator. The Picos Python package~\cite{sagnol2012picos} was used to invoke the CVXOPT solver~\cite{vandenberghe2010cvxopt} for solving the SDPs, and the toqito Python package~\cite{russo2021toqito} was used for carrying out specific operations on the matrices representing quantum systems.      

\section{Multipartite Scenarios}
\label{appendix:multipartite}

    In this appendix, we discuss a multipartite generalization of mixed states' separability tests.

    \begin{definition}
        A state $\rho_{A_1\cdots A_M}\in \mathcal{D}(\mathcal{H}_{A_1\cdots A_M})=\mathcal{D}(\mathcal{H}_{A_1}\otimes\cdots\otimes\mathcal{H}_{A_M})$ is  separable if it can be written as
        \begin{equation}
            \rho_{A_1\cdots A_M}=\sum_{x\in\chi}p(x)\psi^{x,1}_{A_1}\otimes\cdots\otimes\psi^{x,M}_{A_M} ,
        \end{equation}
        where $\psi_{A_i}^{x,i}$ is a pure state for every $x\in\mathcal{X}$ and $i\in\{1,\ldots,M\}$.
    \end{definition}
    
    Let $M\text{-SEP}$ denote the set of all $\rho_{A_1\cdots A_M}\in \mathcal{D}(\mathcal{H}_{A_1\cdots A_M})$ such that $\rho_{A_1\cdots A_M}$ is separable. The following theorem is important for the rest of this analysis.
    
    \begin{theorem}[\cite{Streltsov2010}]\label{theorem:Multi-Streltsov}
        The following formula holds 
        \begin{equation}
        \label{eq:convex-decomp-multi-max-sep-fid}
            \max_{\sigma_{A_1\cdots A_M}\in M-\operatorname{SEP}}F(\rho_{A_1\cdots A_M},\sigma_{A_1\cdots A_M})
            =\max_{\{(q(x),\varphi_{A_1\cdots A_M}^{x})\}_{x}}\sum_{x}q(x) F_s(\varphi^{x}_{A_1\cdots A_M}),
        \end{equation}
        where the optimization is over every pure-state decomposition $\{(q(x),\varphi_{A_1\cdots A_M}^{x})\}_{x}$   of $\rho_{A_1\cdots A_M}$ (similar to those in Theorem~\ref{theorem:Streltsov}) and 
        \begin{equation}
        \label{eqn:multi_fid_sep}
            F_{s}(\varphi_{A_1\cdots A_M}^{x}) =
            \max_{\left\{|\phi^{x,i}\rangle_{A_i}\right\}_{i=1}^{M}}\left\vert\langle\varphi^{x}|_{A_1\cdots A_M}|\phi^{x,1}\rangle_{A_1}\otimes\cdots\otimes |\phi^{x,M}\rangle_{A_M}\right\vert^2.
        \end{equation}
    \end{theorem}
    
    For the multipartite case of the distributed quantum computation, the verifier prepares a purification $\psi^{\rho}_{R A_1\cdots A_M}$\ of $\rho_{A_1\cdots A_M}$. The prover applies a multipartite entanglement-breaking channel on system~$R$, which can be written as
    \begin{equation}
    \label{eqn:ent-break-mult}
        \mathcal{E}_{R\rightarrow A^{\prime}_1\cdots A^{\prime}_{M-1}}(\cdot)=\sum_{x}\operatorname{Tr}[\mu^{x}_{R}(\cdot)]\left(\phi^{x,1}_{A^{\prime}_1}\otimes\cdots\otimes\phi^{x,M-1}_{A^{\prime}_{M-1}}\right),
    \end{equation}
     where $\{\mu^{x}_{R}\}_{x}$ is a rank-one POVM and $\{\phi^{x,i}_{A^{\prime}_i}\}_{x,i}$ is a set of pure states. The prover sends systems $(A^{M-1})^{\prime}\equiv  A^{\prime}_1\cdots A^{\prime}_{M-1}$ to the verifier. Now, the verifier performs a collective swap test of these systems with $A_1\cdots A_M$, as depicted in Figure~\ref{fig:Max_Sep_Fidelity_QIP_EB_Multi}. The acceptance probability of this distributed quantum computation is given by
    \begin{equation}
        \max_{\mathcal{E}\in\operatorname{EB}_{M-1}}\operatorname{Tr}[\Pi_{(A^{M-1})^{\prime}A^{M-1}}^{\operatorname{sym}}\mathcal{E}_{R\rightarrow (A^{M-1})^{\prime}}(\psi_{RA^{M-1}})],
    \end{equation}
    where $\mathcal{E}\in\operatorname{EB}_{M-1}$ denotes the set of entanglement-breaking channels defined in \eqref{eqn:ent-break-mult}.
    This leads to the following theorem:
    \begin{theorem}\label{theorem:qip-eb_msf_multi-app}
        For a pure state $\psi_{RA^{M}} \equiv \psi_{RA_1\cdots A_{M}}$, the following equality holds:%
        \begin{equation}
            \label{eq:mult-part-fid-sep-test-acc-prob}
            \max_{\mathcal{E}\in\operatorname{EB}_{M-1}} \operatorname{Tr}[ \Pi_{(A^{M-1})^{\prime}(A^{M-1})}^{\operatorname{sym}} \mathcal{E}_{R\rightarrow A^{\prime}_1\cdots A^{\prime}_{M-1}}(\psi_{RA^{M}})]
            =\frac{1}{2}\left(1 +\max_{\sigma_{A_1\cdots A_M}\in M-\operatorname{SEP}}F(\rho_{A_1\cdots A_M},\sigma_{A_1\cdots A_M})\right).
        \end{equation}
    \end{theorem}
    
    \begin{figure}
        \centering
        \includegraphics[width=0.7\columnwidth]{figures/fig_QIP_EB_new3.pdf}
        \caption{Test for separability of multipartite mixed states. The verifier uses a unitary circuit $U^\rho$ to produce the state $\psi_{RA_1 A_2 A_3 A_4}$, which is a purification of $\rho_{A_1 A_2 A_3 A_4}$. The prover (indicated by the dotted box) applies an entanglement-breaking channel $\mathcal{E}_{R\rightarrow A^{\prime}_1 A^{\prime}_2 A^{\prime}_3}$ on $R$ by measuring the rank-one POVM $\{\mu^{x}_{R}\}_{x}$ and then, depending on the outcome $x$, prepares a state from the set $\{\phi^{x,1}_{A^{\prime}_1}\otimes\phi^{x,2}_{A^{\prime}_2}\otimes\phi^{x,3}_{A^{\prime}_3}\}_{x}$. The final state is sent to the verifier, who performs a collective swap test. Theorem~\ref{theorem:qip-eb_msf_multi-app} states that the maximum acceptance probability of this interactive proof is equal to $\frac{1}{2}(1 + F_{s}(\rho_{A_1 A_2 A_3 A_4}))$, i.e., a simple function of the multipartite fidelity of separability.}
        \label{fig:Max_Sep_Fidelity_QIP_EB_Multi}
    \end{figure}
    
    \begin{proof}
        The circuit diagram is given in Figure~\ref{fig:Max_Sep_Fidelity_QIP_EB_Multi}. The verifier prepares a purification $\psi^{\rho}_{RA_1\cdots A_M}$\ of $\rho_{A_1\cdots A_M}$. The prover applies a multipartite entanglement-breaking channel on $R$, which can be written as
    \begin{equation}
        \mathcal{E}_{R\rightarrow A^{\prime}_1\cdots A^{\prime}_{M-1}}(\cdot)=\sum_{x}\operatorname{Tr}[\mu^{x}_{R}(\cdot)]\left(\phi^{x,1}_{A^{\prime}_1}\otimes\cdots\otimes\phi^{x,M-1}_{A^{\prime}_{M-1}}\right),
    \end{equation}
     where $\{\mu^{x}_{R}\}_{x}$ is a rank-one POVM and $\{\phi^{x,i}_{A^{\prime}_i}\}_{x,i}$ is a set of pure states. The prover sends the systems $(A^{M-1})^{\prime}= A^{\prime}_1\cdots A^{\prime}_{M-1}$ to the verifier. Now, the verifier performs a collective swap test on $A_1\cdots A_M$, as depicted at the final part of the circuit diagram in Figure~\ref{fig:Max_Sep_Fidelity_QIP_EB_Multi}. The acceptance probability of this interactive proof system is thus given by
    \begin{equation}\label{eqn:qipeb-mult}
        \max_{\mathcal{E}\in\text{EB}}\operatorname{Tr}[\Pi_{(A_1\cdots A_{M-1})^{\prime}A_1\cdots A_{M-1}}^{\text{sym}}\mathcal{E}_{R\rightarrow (A_1\cdots A_{M-1})^{\prime}}(\psi_{RA_1\cdots A_{M}})],
    \end{equation}
    where
    \begin{equation}
    \Pi^{\text{sym}}_{(A_1\cdots A_{M-1})^{\prime}A_1\cdots A_{M-1}}:=\frac{1}{2}\left(I_{(A_1\cdots A_{M-1})^{\prime}A_1\cdots A_{M-1}}+F_{(A_1\cdots A_{M-1})^{\prime}A_1\cdots A_{M-1}}\right)
    \end{equation}
    is the projector onto the symmetric subspace of $A^{\prime}$ and $A$ and $F_{(A_1\cdots A_{M-1})^{\prime}A_1\cdots A_{M-1}}$ is a tensor product of individual swaps $F_{A'_i A_i}$, for $i\in\{1,\ldots,M\}$. That is,
    \begin{equation}
        F_{(A_1\cdots A_{M-1})^{\prime}A_1\cdots A_{M-1}} = \bigotimes_{i=1}^n F_{A'_i A_i}.
    \end{equation}
    Then we find, for fixed $\mathcal{E}_{R\rightarrow A^{\prime}_1\cdots A^{\prime}_{M-1}}$, that
    \begin{align}
        &\operatorname{Tr}[\Pi_{(A_1\cdots A_{M-1})^{\prime}(A_1\cdots A_{M-1})}^{\text{sym}}\mathcal{E}_{R\rightarrow A^{\prime}_1\cdots A^{\prime}_{M-1}}(\psi_{RA_1\cdots A_{M}})]\notag \\
        &=\frac{1}{2}\operatorname{Tr}[(I_{(A_1\cdots A_{M-1})^{\prime}(A_1\cdots A_{M-1})}+F_{(A_1\cdots A_{M-1})^{\prime}(A_1\cdots A_{M-1})})
        \mathcal{E}_{R\rightarrow A^{\prime}_1\cdots A^{\prime}_{M-1}}(\psi_{RA_1\cdots A_{M}})]\\
        &=\frac{1}{2}+\frac{1}{2}\operatorname{Tr}[F_{(A_1\cdots A_{M-1})^{\prime}(A_1\cdots A_{M-1})}\mathcal{E}_{R\rightarrow A^{\prime}_1\cdots A^{\prime}_{M-1}}(\psi_{RA^{M}})]\\
        &=\frac{1}{2}+\frac{1}{2}\operatorname{Tr}\!\left[F_{(A_1\cdots A_{M-1})^{\prime}(A_1\cdots A_{M-1})}\sum_{x}\operatorname{Tr}[\mu^{x}_{R}(\psi_{RA^{M}})]   \phi^{x,1}_{A^{\prime}_1}\otimes\cdots\otimes\phi^{x,M-1}_{A^{\prime}_{M-1}}\right],\\
        &=\frac{1}{2}+\frac{1}{2}\operatorname{Tr}\!\left[F_{(A_1\cdots A_{M-1})^{\prime}(A_1\cdots A_{M-1})}\sum_{x}p(x)(\psi^{x}_{A_1\cdots A_{M}})\phi^{x,1}_{A^{\prime}_1}\otimes\cdots\otimes\phi^{x,M-1}_{A^{\prime}_{M-1}}\right],\\
        &=\frac{1}{2}+\frac{1}{2}\sum_{x}p(x)\operatorname{Tr}\!\left[\left(\phi^{x,1}_{A_1}\otimes\cdots\otimes \phi^{x,M-1}_{A_{M-1}}\right)\psi^{x}_{A_1\cdots A_{M-1}}\right]\label{eqn:mult-fid-sep-intermediate},
    \end{align}
    where
    \begin{align}
        p(x) &  \coloneqq \operatorname{Tr}[\mu^{x}_{R} \psi_{RA^{M}} ],\\
        \psi^{x}_{A_1\cdots A_{M}} &  \coloneqq \frac{1}{p(x)}\operatorname{Tr}_R[\mu^{x}_{R}\psi_{RA^{M}}].
    \end{align}
    For a given $x$, let us  simplify $F_s(\varphi_{A_1\cdots A_M})$ as defined in~\eqref{eqn:multi_fid_sep}, 
   \begin{align} 
        F_s(\varphi_{A_1\cdots A_M})&=\max_{\left\{|\phi^{i}\rangle_{A_i}\right\}_{i=1}^M}\left\vert\langle\varphi|_{A_1\cdots A_M}|\phi^{1}\rangle_{A_1}\otimes\cdots\otimes |\phi^{M}\rangle_{A_M}\right\vert^2 \\
        &=\max_{\left\{|\phi^{i}\rangle_{A_i}\right\}_{i=1}^M}\left\vert\langle\phi^{1}|_{A_1}\otimes\cdots\otimes \langle\phi^{M}|_{A_M}|\varphi\rangle_{A_1\cdots A_M}\right\vert^2\\
        &=\max_{\left\{|\phi^{i}\rangle_{A_i}\right\}_{i=1}^{M-1}}\left\Vert \langle\phi^{1}|_{A_1}\otimes\cdots\otimes\langle\phi^{M-1}|_{A_{M-1}}\otimes I_{A_M}|\varphi\rangle_{A^M}\right\Vert _{2}^{2}\\
        &=\max_{\left\{|\phi^{i}\rangle_{A_i}\right\}_{i=1}^{M-1}}\operatorname{Tr}[(|\phi^{1}\rangle\!\langle\phi^{1}|_{A_1}\otimes\cdots \otimes|\phi^{M-1}\rangle\!\langle\phi^{M-1}|_{A_{M-1}}\otimes I_{A_M})\varphi_{A_1\cdots A_M}]\\
        &=\max_{\left\{|\phi^{i}\rangle_{A_i}\right\}_{i=1}^{M-1}}\operatorname{Tr}[(|\phi^{1}\rangle\!\langle\phi^{1}|_{A_1}\otimes\cdots \otimes|\phi^{M-1}\rangle\!\langle\phi^{M-1}|_{A_{M-1}})\varphi_{A_1\cdots A_{M-1}}]\label{eqn:mult-fid-sep-pure}.
    \end{align}
    The first two equalities are from the definition and a rewriting. The third equality follows from the variational characterization of the Euclidean norm of a vector. Noting the form in  \eqref{eqn:mult-fid-sep-pure} and applying the maximization over entanglement-breaking channels of the form described in \eqref{eqn:ent-break-mult} to \eqref{eqn:mult-fid-sep-intermediate}, we arrive at the desired claim in \eqref{eq:mult-part-fid-sep-test-acc-prob}.
    \end{proof}

    \begin{figure}
        \centering
        \includegraphics[width=0.7\columnwidth]{figures/fig_VQA_new_multi3.pdf}
        \caption{VQSA to estimate the multipartite fidelity of separability $F_s(\rho_{A_1A_2A_3A_4})$. 
        The unitary circuit $U^\rho$ prepares the state $\psi_{RA_1A_2A_3A_4}$, which is a purification of $\rho_{A_1A_2A_3A_4}$. The parameterized circuit $W_R(\Theta)$ acts on $R$ to evolve the state to another purification of $\rho_{A_1A_2A_3A_4}$. The following measurement, labeled ``steering measurement,'' steers the remaining systems to be in a state $\psi_{A_1A_2A_3A_4}^{x}$ if the measurement outcome $x$ occurs. Conditioned on the outcome $x$, the final parameterized circuits $U^{x,1}_{A_1}(\Theta^x_1)$, $U^{x,2}_{A_2}(\Theta^x_2)$, and $U^{x,3}_{A_3}(\Theta^x_3)$ are applied and the subsequent measurement estimates the quantity~$\operatorname{Tr}\!\left[\left(\phi^{x,1}_{A_1}\otimes\phi^{x,2}_{A_2}\otimes \phi^{x,3}_{A_{3}}\right)\psi^{x}_{A_1 A_2 A_{3}}\right]$.} 
        \label{fig:Max_Sep_Fidelity_VQSA_Multi}
    \end{figure}
    
    \bigskip 
    We can then use the generalized test of separability of mixed states to develop a VQSA for the multipartite case. See Figure~\ref{fig:Max_Sep_Fidelity_VQSA_Multi}. This involves replacing the collective swap test in Figure~\ref{fig:Max_Sep_Fidelity_QIP_EB_Multi} with an overlap measurement, similar to how we got Figure~\ref{fig:Max_Sep_Fidelity_VQA} in the main text from Figure~\ref{fig:Max_Sep_Fidelity_QIP_EB} in the main text.
    
\section{Complexity Class \texorpdfstring{\qipeb}{Lg}} \label{appendix:qipeb}

    In this appendix, we establish a complete problem for \qipeb, and then we interpret this problem in Remark~\ref{rem:interp-QIP-EB2}. See \cite{watrous2009complexity,VW15} for further background on quantum computational complexity theory. Let us first define the complexity class \qipeb. Let $A=\left(A_{\text{yes}},A_{\text{no}}\right)$ be a promise problem, and let $a,b:\mathbb{N}\to [0,1]$ and $p$ be polynomial functions. The verifier $V$ is described by a polynomial-time generated family of quantum circuits. The prover $P$ is a family of arbitrary entanglement-breaking channels that interface with a given verifier naturally. Then $A\in$ \qipeb$(a,b)$ if there exists a two-message verifier with the following properties:
    \begin{enumerate}
        \item Completeness: For all $x\in A_{\text{yes}}$, there exists a prover $P$ that causes the verifier $V$ to accept $x$ with probability at least $a(|x|)$.
        \item Soundness: For all $x\in A_{\text{no}}$, every prover $P$  causes the verifier $V$ to accept $x$ with probability at most $b(|x|)$.
    \end{enumerate}
    In the above, acceptance is defined as obtaining the outcome one upon measuring the decision-qubit register.

\begin{problem}\label{prob:eb-2}
Given are circuits to generate a channel $\mathcal{N}_{G\rightarrow S}$ and a state $\rho_{S}$. Fix $\alpha$ and $\beta$ such that $0 \leq \alpha < \beta \leq 1$. Decide which of the following holds:
\begin{align}
    \text{Yes:} \quad f(\mathcal{N}_{G\rightarrow S},\rho_{S}) & \geq \beta, \\
    \text{No:} \quad f(\mathcal{N}_{G\rightarrow S},\rho_{S}) & \leq \alpha,
\end{align}
where
        \begin{equation}
            f(\mathcal{N}_{G\rightarrow S},\rho_{S}) \coloneqq  \max_{\substack{\{  (p(x),\psi^{x})\}  _{x},
            \left\{  \varphi^{x}\right\}_{x}
            }} \left\{  \sum_{x}p(x)F(\psi_{S}^{x},\mathcal{N}_{G\rightarrow S}(\varphi_{G}^{x})) : \sum_{x}p(x)\psi_{S}^{x}=\rho_{S}  \right\}
            \label{eq:accept-prob-qip_eb-2}
        \end{equation}
        with the optimization being over every pure-state decomposition of $\rho_{S}$ as $\sum_{x}p(x)\psi_{S}^{x}=\rho_{S}$. Also, $\left\{  \varphi^{x}\right\}_{x}$ is a set of pure states.
\end{problem}
    
    \begin{theorem}
         Problem~\ref{prob:eb-2} is a complete problem for $\operatorname{QIP}_{\operatorname{EB}}(2)$.
    \end{theorem}

    \begin{proof}
    The main idea behind the proof is to show that the acceptance probability of a general \qipeb\ problem can precisely be written as $f(\mathcal{N}_{G\rightarrow S},\rho_{S})$. This implies that an arbitrary \qipeb\ problem can be reduced to an instance of Problem~\ref{prob:eb-2}, and we argue at the end how this also implies that Problem~\ref{prob:eb-2} can be reduced to an instance of a problem in \qipeb.

    Consider a general interactive proof system in \qipeb\ that begins with the verifier preparing a bipartite pure state $\psi_{RS}$, followed by the system $R$ being sent to the prover, which subsequently performs an entanglement-breaking channel. The verifier then performs a unitary $V_{R^{\prime}S\rightarrow DG}$ and projects onto the $|1\rangle\!\langle 1|$ state of the decision qubit. Indeed, the acceptance probability is given by
    \begin{equation}
        \max_{\mathcal{E}\in\operatorname{EB}}\operatorname{Tr}[(|1\rangle\!\langle1|_{D}\otimes I_{G})\mathcal{V}_{R^{\prime}S\rightarrow DG}(\mathcal{E}_{R\rightarrow R^{\prime}}(\psi_{RS}))],
    \end{equation}
    where $\mathcal{V}_{R^{\prime}S\rightarrow DG}$ is the unitary channel corresponding to the unitary operator $V_{R^{\prime}S\rightarrow DG}$.
    By the reasoning similar to that in \eqref{eq:eb-proof-1}, \eqref{eq:eb-proof-2}, and \eqref{eq:eb-proof-3}, we find that
    \begin{equation}
        \mathcal{E}_{R\rightarrow R^{\prime}}(\psi_{RS})=\sum_{x}p(x)\phi_{R^{\prime}}^{x}\otimes\psi_{S}^{x},
    \end{equation}
    so that the acceptance probability is equal to
    \begin{multline}
          \max_{\substack{\{  (p(x),\psi^{x})\}  _{x},\\
          \left\{  \phi^{x}\right\}_{x},\\
          \sum_{x}p(x)\psi_{S}^{x}=\psi_{S}}}
            \operatorname{Tr}\!\left[  (|1\rangle\!\langle1|_{D}\otimes I_{G})\mathcal{V}\left(  \sum_{x}p(x)\phi_{R^{\prime}}^{x}\otimes\psi_{S}^{x}\right)  \right]
          \\
          =\max_{\substack{\{  (p(x),\psi^{x})\}  _{x},\\
          \left\{  \phi^{x}\right\}_{x},\\
          \sum_{x}p(x)\psi_{S}^{x}=\psi_{S}}}
            \sum_{x}p(x)\operatorname{Tr}\!\left[  (|1\rangle\!\langle 1|_{D}\otimes I_{G})\mathcal{V}
            \left(\phi_{R^{\prime}}^{x}\otimes\psi_{S}^{x}\right)  \right] ,
    \end{multline}
    where we have used the shorthand $\mathcal{V} \equiv \mathcal{V}_{R^{\prime}S\rightarrow DG}$.
    Consider that%
    \begin{align}
        \operatorname{Tr}\!\left[  (|1\rangle\!\langle1|_{D}\otimes I_{G})\mathcal{V} \left(  \phi_{R^{\prime}}^{x}\otimes\psi_{S}^{x}\right)  \right] 
        &=\left\Vert \langle1|_{D}\otimes I_{G}) V|\phi^{x}\rangle_{R^{\prime}}\otimes|\psi^{x}\rangle_{S}\right\Vert _{2}^{2}\\
        &=\max_{|\varphi^{x}\rangle_{G}}\left\vert \langle1|_{D}\otimes\langle\varphi^{x}|_{G}) V |\phi^{x}\rangle_{R^{\prime}}\otimes|\psi^{x}\rangle_{S}\right\vert ^{2}\\
        &  =\max_{|\varphi^{x}\rangle_{G}}\operatorname{Tr}\!\left[  V^{\dag}(|1\rangle\!\langle1|_{D}\otimes|\varphi^{x}\rangle\!\langle\varphi^{x}|_{G})  V\phi^{x}_{R^{\prime}}\otimes|\psi^{x}\rangle\!\langle\psi^{x}|_{S}\right]  \\
        &  =\max_{|\varphi^{x}\rangle_{G}}\operatorname{Tr}\!\left[  \mathcal{W}_{G\rightarrow R^{\prime}S}(|\varphi^{x}\rangle\!\langle\varphi^{x}|_{G})\phi^{x}_{R^{\prime}}\otimes|\psi^{x}\rangle\!\langle\psi^{x}|_{S}\right],
    \end{align}
    where the isometric channel $\mathcal{W}_{G \to R'S}$ is defined as 
    \begin{equation}
        \mathcal{W}_{G\rightarrow R^{\prime}S}(\cdot)\coloneqq (V_{R^{\prime}S\rightarrow DG})^{\dag}(|1\rangle\!\langle1|_{D}\otimes(\cdot)_{G})V_{R^{\prime}S\rightarrow DG}.
    \end{equation}
    Then, the acceptance probability is given by
    \begin{equation}
    \max_{\substack{\{  (p(x),\psi^{x})\}  _{x},\\\left\{\phi^{x}\right\}  _{x},\left\{  \varphi^{x}\right\}  _{x}}}
    \left\{\begin{array}
            [c]{c}%
            \sum_{x}p(x)\operatorname{Tr}\!\left[\mathcal{W}_{G\rightarrow R^{\prime}S}(|\varphi^{x}\rangle\!\langle\varphi^{x}|_{G})\phi^{x}_{R^{\prime}}\otimes|\psi^{x}\rangle\!\langle\psi^{x}|_{S}\right]  :\\
            \sum_{x}p(x)\psi_{S}^{x}=\psi_{S}%
        \end{array}\right\}  .
    \end{equation}
    Since the optimization over $\phi^{x}_{R^{\prime}}$ is arbitrary, we can also write%
    \begin{align}
        &\max_{|\phi^{x}\rangle_{R^{\prime}}}\operatorname{Tr}\!\left[  \mathcal{W}_{G\rightarrow R^{\prime}S}(|\varphi^{x}\rangle\!\langle\varphi^{x}|_{G})\phi^{x}_{R^{\prime}}\otimes|\psi^{x}\rangle\!\langle\psi^{x}|_{S}\right] \notag \\
        &= \max_{|\phi^{x}\rangle_{R^{\prime}}}\left\vert \langle\phi^{x}|_{R^{\prime}}\otimes\langle\psi^{x}|_{S}W_{G\rightarrow R^{\prime}S}|\varphi^{x}\rangle_{G}\right\vert ^{2} \\
        &  =\left\Vert I_{R^{\prime}}\otimes\langle\psi^{x}|_{S}W_{G\rightarrow R^{\prime}S}|\varphi^{x}\rangle_{G}\right\Vert _{2}^{2} \\
        &  =\left(  \langle\varphi^{x}|_{G}\left(  W_{G\rightarrow R^{\prime}S}\right)  ^{\dag}I_{R^{\prime}}\otimes|\psi^{x}\rangle_{S}\right)   \left(I_{R^{\prime}}\otimes\langle\psi^{x}|_{S} W_{G\rightarrow R^{\prime}S}|\varphi^{x}\rangle_{G}\right)  \\
        &  =\langle\varphi^{x}|_{G}\left(W_{G\rightarrow R^{\prime}S}\right)^{\dag}\left(  I_{R^{\prime}}\otimes|\psi^{x}\rangle\!\langle\psi^{x}|_{S}\right)  W_{G\rightarrow R^{\prime}S}|\varphi^{x}\rangle_{G}\\
        &  =\operatorname{Tr}\!\left[  \left(  I_{R^{\prime}}\otimes|\psi^{x}\rangle\!\langle\psi^{x}|_{S}\right)  W_{G\rightarrow R^{\prime}S}|\varphi^{x}\rangle\!\langle\varphi^{x}|_{G}\left(  W_{G\rightarrow R^{\prime}S}\right)^{\dag}\right]  \\
        &  =\operatorname{Tr}[|\psi^{x}\rangle\!\langle\psi^{x}|_{S}\mathcal{N}_{G\rightarrow S}(|\varphi^{x}\rangle\!\langle\varphi^{x}|_{G})],
    \end{align}
    where we define the channel $\mathcal{N}_{G\rightarrow S}$ as
    \begin{equation}
        \mathcal{N}_{G\rightarrow S}(\cdot)\coloneqq \operatorname{Tr}_{R^{\prime}}[(V_{R^{\prime}S\rightarrow DG})^{\dag}(|1\rangle\!\langle1|_{D}\otimes(\cdot)_{G})V_{R^{\prime}S\rightarrow DG}].
        \label{eq:eb2-channel-def}
    \end{equation}
    Then, we find that the acceptance probability is given by
    \begin{equation}
          \max_{\substack{\{  (p(x),\psi^{x})\}  _{x},\\
          \left\{  \varphi^{x}\right\}_{x},\\
          \sum_{x}p(x)\psi_{S}^{x}=\psi_{S}}}  \sum_{x}p(x)\operatorname{Tr}[|\psi^{x}\rangle\!\langle\psi^{x}|_{S}\mathcal{N}_{G\rightarrow S}(|\varphi^{x}\rangle\!\langle\varphi^{x}|_{G})]
         =\max_{\substack{\{  (p(x),\psi^{x})\}  _{x},\left\{  \varphi^{x}\right\}_{x}\\
         \sum_{x}p(x)\psi_{S}^{x}=\psi_{S}
         }}  \sum_{x}p(x)F(\psi_{S}^{x},\mathcal{N}_{G\rightarrow S}(\varphi_{G}^{x}))  .
    \end{equation}
    This concludes the proof of the first part. 

    To see how this implies that Problem~\ref{prob:eb-2} can be realized in \qipeb, note that the circuit preparing the state $\rho_S$ prepares a purification and traces over the reference system, and the circuit to generate $\mathcal{N}_{G\to S}$ is realize by adjoining an environment system in the state $|0\rangle\!\langle 0|$, performing a unitary, and tracing over the environment. So we let the verifier prepare the purification of $\rho_S$ and this plays the role of $\psi_{RS}$ above, and the channel $\mathcal{N}_{G\to S}$ can be realized precisely as in \eqref{eq:eb2-channel-def} with appropriate substitutions.
    \end{proof}

    \begin{remark}
    \label{rem:interp-QIP-EB2}
    The quantity in \eqref{eq:accept-prob-qip_eb-2} can be interpreted as follows: Given a channel $\mathcal{N}$ and a source state~$\rho$, calculate the largest average ensemble fidelity attainable in reproducing the source at the output of the channel. This means it is necessary to find the ensemble decomposition $\{  (p(x),\psi^{x})\}  _{x}$ of $\rho$ as well as a set $\left\{  \varphi^{x}\right\}_{x}$ of encoding states 
    that lead to the largest ensemble fidelity (and this is what is left to the prover). This criterion is similar to one used in Schumacher data compression~\cite{PhysRevA.51.2738}, but this seems more similar to the setting of the source-channel separation theorem \cite{6287590}, in which the goal is to transmit an information source over a quantum channel. The channel $\mathcal{N}$ here could consist of a fixed encoding $\mathcal{E}$, noisy channel $\mathcal{M}$, and fixed decoding $\mathcal{D}$, (i.e., $\mathcal{N} = \mathcal{D} \circ \mathcal{M} \circ \mathcal{E}$) and then the goal is to test how well a given fixed scheme $(\mathcal{E},\mathcal{D})$ can communicate a source $\rho$ over a channel $\mathcal{M}$, according to the ensemble fidelity criterion.
    \end{remark}

    \begin{remark}
    \label{rem:concave-closure-QIPEB2}
        We can write the expression in \eqref{eq:accept-prob-qip_eb-2} alternatively as
        \begin{equation}
            \text{\rm Eq.~\eqref{eq:accept-prob-qip_eb-2}} = \max_{\substack{\{  (p(x),\psi^{x})\}  _{x},\\
            \sum_{x}p(x)\psi_{S}^{x}=\rho_{S}
            }}   \sum_{x}p(x)\left\|(\mathcal{N}_{G\rightarrow S})^{\dag}(\psi_{S}^{x})\right\|_\infty,
            \label{eq:concave-closure}
        \end{equation}
        where $(\mathcal{N}_{G\rightarrow S})^{\dag}$ is the Hilbert--Schmidt adjoint of the channel $(\mathcal{N}_{G\rightarrow S})^{\dag}$.
        Employing the abbreviations $\psi^{x}_{S}\equiv |\psi^{x}\rangle\!\langle\psi^{x}|_{S}$ and $\varphi^{x}_{G}\equiv |\varphi^{x}\rangle\!\langle\varphi^{x}|_{G}$, this follows because
        \begin{align}
        & \max_{\substack{\{  (p(x),\psi^{x})\}  _{x},\\
          \left\{  \varphi^{x}\right\}_{x},\\
          \sum_{x}p(x)\psi_{S}^{x}=\psi_{S}}}  \sum_{x}p(x)\operatorname{Tr}[\psi^{x}_{S} \mathcal{N}_{G\rightarrow S}(\varphi^{x}_{G})] \notag \\
          & = \max_{\substack{\{  (p(x),\psi^{x})\}  _{x},\\
          \left\{  \varphi^{x}\right\}_{x},\\
          \sum_{x}p(x)\psi_{S}^{x}=\psi_{S}}}  \sum_{x}p(x)\operatorname{Tr}[(\mathcal{N}_{G\rightarrow S})^{\dag}(\psi^{x}_{S})\varphi^{x}_{G}] \\
          & = \max_{\substack{\{  (p(x),\psi^{x})\}  _{x},
          \\
          \left\{  \varphi^{x}\right\}_{x},\\
          \sum_{x}p(x)\psi_{S}^{x}=\psi_{S}}}  \sum_{x}p(x)
          \max_{\left\{  \varphi^{x}\right\}_{x}}
          \operatorname{Tr}[(\mathcal{N}_{G\rightarrow S})^{\dag}(\psi^{x}_{S})\varphi^{x}_{G}] \\
          & = \max_{\substack{\{  (p(x),\psi^{x})\}  _{x},\\
            \sum_{x}p(x)\psi_{S}^{x}=\rho_{S}
            }}   \sum_{x}p(x)\left\|(\mathcal{N}_{G\rightarrow S})^{\dag}(\psi_{S}^{x})\right\|_\infty.
        \end{align}
        If we define the function
        \begin{equation}
            g_{\mathcal{N}}(\rho) \coloneqq \left\|(\mathcal{N}_{G\rightarrow S})^{\dag}(\rho_S)\right\|_\infty,
        \end{equation}
        then the function in \eqref{eq:concave-closure} is known as the concave closure of $g_{\mathcal{N}}(\rho)$ and has been studied in other contexts in quantum information theory \cite[Section~2]{AFKW04}. It has an interesting dual formulation, as demonstrated in \cite[Eq.~(15)]{AFKW04}. Given the observation in \eqref{eq:concave-closure}, we can thus conclude that, given circuits to realize the channel $\mathcal{N}$ and state $\rho$, estimating the concave closure of the function $g_{\mathcal{N}}(\rho) $ within additive error is a complete problem for {\rm \qipeb}.
    \end{remark}

    \begin{remark}
        Employing the reasoning from Remark~\ref{rem:concave-closure-QIPEB2}, we find that the acceptance probability in~\eqref{eqn:accept_prob_supp} is equal to the concave closure of the following function:
        \begin{equation}
            f(\rho_{AB}) \coloneqq \left \| \Pi^{\operatorname{sym}}_{AA'} (\rho_{AB} \otimes I_{A'}) \Pi^{\operatorname{sym}}_{AA'}\right \|_{\infty},
            \label{eq:concave-closure-sep-prob-accept-prob}
        \end{equation}
        where we used the fact that the state $\rho_S$ from Remark~\ref{rem:concave-closure-QIPEB2} is $\rho_{AB}$ and the map $\mathcal{N}_{G \to S}$ from Remark~\ref{rem:concave-closure-QIPEB2} is
        \begin{equation}
         \mathcal{N}(\sigma_{AA'B}) = \operatorname{Tr}_{A'}[\Pi^{\operatorname{sym}}_{AA'} \sigma_{AA'B}\Pi^{\operatorname{sym}}_{AA'}],   
        \end{equation}
          with adjoint
          \begin{equation}
          \mathcal{N}^{\dag}(\omega_{AB}) = \Pi^{\operatorname{sym}}_{AA'} (\omega_{AB} \otimes I_{A'}) \Pi^{\operatorname{sym}}_{AA'}.    
          \end{equation}
        Observe that the map $\rho_{AB} \mapsto \Pi^{\operatorname{sym}}_{AA'} (\rho_{AB} \otimes I_{A'}) \Pi^{\operatorname{sym}}_{AA'}$ is proportional to that used in a $1\to 2$ universal cloning machine \cite[Eq.~(17)]{SIGA05}.
        If $\rho_{AB}$ is pure, so that we write it as $\psi_{AB}$, then the following inequality holds:
        \begin{equation}
            \left \| \Pi^{\operatorname{sym}}_{AA'} (\psi_{AB} \otimes I_{A'}) \Pi^{\operatorname{sym}}_{AA'}\right \|_{\infty}
             \leq \left \| \Pi^{\operatorname{sym}}_{AA'} \right \|_{\infty}
            \left \|
            \psi_{AB} \otimes I_{A'} \right \|_{\infty}
            \left \|\Pi^{\operatorname{sym}}_{AA'}
            \right \|_{\infty}
             \leq 1,
        \end{equation}
        where we applied the multiplicativity of the spectral norm. 
        Thus, the concave closure of $f(\rho_{AB})$ satisfies $f(\rho_{AB}) \in [0,1]$. Furthermore, from Lemma~\ref{lem:alt-accept-prob-pure-states} below, we know that
        \begin{equation}
            \left \| \Pi^{\operatorname{sym}}_{AA'} (\psi_{AB} \otimes I_{A'}) \Pi^{\operatorname{sym}}_{AA'}\right \|_{\infty} 
        = 
            \frac{1}{2}\left(  1+\left\Vert \psi_{A} \right\Vert _{\infty}\right),
        \end{equation}
        showing the consistency of the claim just above \eqref{eq:concave-closure-sep-prob-accept-prob} with Theorem~\ref{theorem:qip-eb_msf} and Eqs.~\eqref{eq:convex-decomp-max-sep-fid} and \eqref{eq:max-sep-fid-inf-norm} in the main text.
        If $\rho_{AB}$ is a pure product state, so that we can write it as $\rho_{AB} = \phi_A \otimes \varphi_B$, then we have that
        \begin{equation}
            \left \| \Pi^{\operatorname{sym}}_{AA'} (\phi_A \otimes \varphi_B \otimes I_{A'}) \Pi^{\operatorname{sym}}_{AA'}\right \|_{\infty} = 
            \left \| \Pi^{\operatorname{sym}}_{AA'} (\phi_A  \otimes I_{A'}) \Pi^{\operatorname{sym}}_{AA'}\right \|_{\infty},
            \label{eq:spec-norm-analysis}
        \end{equation}
        and the spectral norm on the right-hand side of \eqref{eq:spec-norm-analysis} is achieved by choosing the vector $|\phi\rangle_A \otimes |\phi\rangle_{A'}$, so that
        \begin{align}
            & (\langle\phi|_A \otimes \langle\phi|_{A'})\Pi^{\operatorname{sym}}_{AA'} (\phi_A  \otimes I_{A'}) \Pi^{\operatorname{sym}}_{AA'}(|\phi\rangle_A \otimes |\phi\rangle_{A'}) \notag 
            \\
            &  = (\langle\phi|_A \otimes \langle\phi|_{A'}) (\phi_A  \otimes I_{A'}) (|\phi\rangle_A \otimes |\phi\rangle_{A'})\\
            &  =1.
        \end{align}
    \end{remark}

    \begin{lemma}
    \label{lem:alt-accept-prob-pure-states}
        For a pure state $\psi_{AB}$, the following equality holds:
        \begin{equation}
        \left \| \Pi^{\operatorname{sym}}_{AA'} (\psi_{AB} \otimes I_{A'}) \Pi^{\operatorname{sym}}_{AA'}\right \|_{\infty} 
        = 
            \frac{1}{2}\left(  1+\left\Vert \psi_{A} \right\Vert _{\infty}\right),
        \end{equation}
        where $\psi_{A}\equiv \operatorname{Tr}_B[\psi_{AB}]$.
    \end{lemma}

    \begin{proof}
Consider that
\begin{multline}
\left\Vert \left(  \Pi_{AA^{\prime}}^{\text{sym}}\otimes I_{B}\right)  \left(
|\psi\rangle\!\langle\psi|_{AB}\otimes I_{A^{\prime}}\right)  \left(
\Pi_{AA^{\prime}}^{\text{sym}}\otimes I_{B}\right)  \right\Vert _{\infty}
\\
=\left\Vert \left(  |\psi\rangle\!\langle\psi|_{AB}\otimes I_{A^{\prime}%
}\right)  \left(  \Pi_{AA^{\prime}}^{\text{sym}}\otimes I_{B}\right)  \left(
|\psi\rangle\!\langle\psi|_{AB}\otimes I_{A^{\prime}}\right)  \right\Vert
_{\infty}.%
\end{multline}
Now consider that
\begin{align}
& \left(  |\psi\rangle\!\langle\psi|_{AB}\otimes I_{A^{\prime}}\right)  \left(
\Pi_{AA^{\prime}}^{\text{sym}}\otimes I_{B}\right)  \left(  |\psi
\rangle\!\langle\psi|_{AB}\otimes I_{A^{\prime}}\right)  \nonumber\\
& =\left(  |\psi\rangle\!\langle\psi|_{AB}\otimes I_{A^{\prime}}\right)  \left(
\frac{I_{AA^{\prime}}+F_{AA^{\prime}}}{2}\otimes I_{B}\right)  \left(
|\psi\rangle\!\langle\psi|_{AB}\otimes I_{A^{\prime}}\right)   \\
& =\frac{1}{2}\left(  |\psi\rangle\!\langle\psi|_{AB}\otimes I_{A^{\prime}%
}\right)+\frac{1}{2}\left(  |\psi\rangle\!\langle\psi|_{AB}\otimes
I_{A^{\prime}}\right)  \left(  F_{AA^{\prime}}\otimes I_{B}\right)  \left(
|\psi\rangle\!\langle\psi|_{AB}\otimes I_{A^{\prime}}\right)  .
\end{align}
Then writing the Schmidt decomposition of $|\psi\rangle_{AB}$ as $|\psi
\rangle_{AB}=\sum_{i}\sqrt{\lambda_{i}}|i\rangle_{A}|i\rangle_{B}$, we find that
\begin{align}
    & \left(  |\psi\rangle\!\langle\psi|_{AB}\otimes I_{A^{\prime}}\right)  \left(
    F_{AA^{\prime}}\otimes I_{B}\right)  \left(  |\psi\rangle\!\langle\psi
    |_{AB}\otimes I_{A^{\prime}}\right)  \nonumber\\
    &=\sum_{i,i^{\prime},j,j^{\prime}}\sqrt{\lambda_{i}\lambda_{i^{\prime}}%
    }\left(  |\psi\rangle\!\langle i|_{A}\langle i|_{B}\otimes|j\rangle\!\langle
    j|_{A^{\prime}}\right)  \left(  F_{AA^{\prime}}\otimes I_{B}\right) \left(
    |i^{\prime}\rangle_{A}|i^{\prime}\rangle_{B}\langle\psi|_{AB}\otimes
    |j^{\prime}\rangle\!\langle j^{\prime}|_{A^{\prime}}\right)  \\
    &=\sum_{i,i^{\prime},j,j^{\prime}}\sqrt{\lambda_{i}\lambda_{i^{\prime}}}\left(  |\psi\rangle\!\langle i|_{A}\langle i|_{B}\otimes|j\rangle\!\langle
    j|_{A^{\prime}}\right) \left(  |j^{\prime}\rangle_{A}|i^{\prime}\rangle_{B}\langle\psi|_{AB}\otimes|i^{\prime}\rangle\!\langle j^{\prime}|_{A^{\prime}%
    }\right)  \\
    &=\sum_{i,i^{\prime},j,j^{\prime}}\sqrt{\lambda_{i}\lambda_{i^{\prime}}}%
|\psi\rangle\!\langle i|j^{\prime}\rangle_{A}\langle i|i^{\prime}\rangle
_{B}\langle\psi|_{AB}\otimes|j\rangle\!\langle j|i^{\prime}\rangle\!\langle
j^{\prime}|_{A^{\prime}}\\
& =\sum_{i}\lambda_{i}|\psi\rangle\!\langle\psi|_{AB}\otimes|i\rangle\!\langle
i|_{A^{\prime}} \\
& =|\psi\rangle\!\langle\psi|_{AB}\otimes\sum_{i}\lambda_{i}|i\rangle\!\langle
i|_{A^{\prime}} \\ & =|\psi\rangle\!\langle\psi|_{AB}\otimes\psi_{A^{\prime}}.
\end{align}
Then%
\begin{align}
&\left(  |\psi\rangle\!\langle\psi|_{AB}\otimes I_{A^{\prime}}\right)  \left(
\Pi_{AA^{\prime}}^{\text{sym}}\otimes I_{B}\right)  \left(  |\psi
\rangle\!\langle\psi|_{AB}\otimes I_{A^{\prime}}\right)  \nonumber\\
& =\frac{1}{2}\left(  |\psi\rangle\!\langle\psi|_{AB}\otimes I_{A^{\prime}%
}\right)  +|\psi\rangle\!\langle\psi|_{AB}\otimes\frac{1}{2}\psi_{A^{\prime}}\\
& =|\psi\rangle\!\langle\psi|_{AB}\otimes\frac{1}{2}\left(  I_{A^{\prime}}%
+\psi_{A^{\prime}}\right)  ,
\end{align}
and we conclude that%
\begin{align}
& \left\Vert \left(  \Pi_{AA^{\prime}}^{\text{sym}}\otimes I_{B}\right)
\left(  |\psi\rangle\!\langle\psi|_{AB}\otimes I_{A^{\prime}}\right)  \left(
\Pi_{AA^{\prime}}^{\text{sym}}\otimes I_{B}\right)  \right\Vert _{\infty
}\nonumber\\
& =\left\Vert |\psi\rangle\!\langle\psi|_{AB}\otimes\frac{1}{2}\left(
I_{A^{\prime}}+\psi_{A^{\prime}}\right)  \right\Vert _{\infty}\\
& =\frac{1}{2}\left(  1+\left\Vert \psi_{A^{\prime}}\right\Vert _{\infty
}\right)  
 \\
 &=\frac{1}{2}\left(  1+\left\Vert \psi_{A}\right\Vert _{\infty}\right)  .
\end{align}
This concludes the proof.
    \end{proof}

    
    
\section{Placement of \texorpdfstring{\qipeb}{Lg}}
\label{appendix:complexity_placements}

    In this appendix, we establish the following containments:
    \begin{equation}
    \text{QAM}, \text{QSZK} \subseteq \text{\qipeb}.    
    \end{equation}
    See Figure~\ref{fig:placement} for a detailed diagram.
    
    
    \subsection{QAM \texorpdfstring{$\subseteq$}{Lg} \texorpdfstring{\qipeb}{Lg}}

    First, recall that QAM consists of the verifier selecting a classical letter $x$ uniformly at random, sending the choice to the prover, who then sends back a pure state $\psi_x$ to the verifier, who finally performs an efficient measurement to decide whether to accept the computation \cite{marriott2004quantum}. Note that QAM contains QMA~\cite{marriott2004quantum}.
    
    To see the containment QAM $\subseteq$ \qipeb, consider that the verifier's first circuit in \qipeb\ can consist of preparing a random classical bitstring in a system $R$. The verifier sends system $R$ to the prover. Then, the prover's action amounts to preparing some state that gets returned to the verifier. The rest of the protocol then simulates a QAM protocol.

% \begin{figure}
%         \centering
%         \includegraphics[width=0.4\columnwidth]{figures/placement_diagram_new.pdf}
%         \caption{Placement of \qipeb\ relative to other known complexity classes. The complexity classes are organized such that if a class is connected to a class above it, the complexity class placed lower is a subset of the class above. For example, \qipeb\ is a superset of both QSZK and QAM.}
%         \label{fig:placement1}
%     \end{figure}


    \subsection{QSZK \texorpdfstring{$\subseteq$}{Lg} \texorpdfstring{\qipeb}{Lg}}

    Quantum statistical zero-knowledge (QSZK) consists of all problems that can be solved by the interaction between a quantum verifier and a quantum prover, such that the verifier accumulates statistical evidence about the answer to a decision, but does not learn anything other than the answer by interacting with the prover \cite{W02,watrous2006zero}. A complete problem for this class is quantum state distinguishability, in which the goal is to decide whether two states $\rho_0$ and $\rho_1$, generated by quantum circuits, are far or close in trace distance \cite{W02}. This is a nice problem for understanding the basics of the QSZK complexity class: the interaction begins with the verifier picking one of the states uniformly at random, recording the choice as a bit $x$, and then sending the chosen state~$\rho_x$ to the prover over a quantum channel. The prover can then perform the optimal Helstrom measurement \cite{H67,H69} to distinguish the states which has success probability equal to
    \begin{equation}
        p_{\text{succ}}\coloneqq \frac{1}{2}\left(1 + \frac{1}{2}\left \|  \rho_0 - \rho_1\right\|_1\right).
    \end{equation}
    The Helstrom measurement leads to a decision bit $y$, which the prover sends back to the verifier over a quantum channel (here, a single classical bit channel would suffice). The verifier then accepts if $x = y$, and the probability that this happens is equal to $p_{\text{succ}}$. By repeating this protocol a polynomial number of times and invoking the error-reduction protocol from \cite{W02}, it follows that the verifier can make the completeness and soundness probabilities exponentially close to one and zero, respectively, to have essentially zero error probability in the final decision about whether the states are near or far in trace distance. Finally, the interaction has a ``zero knowledge'' aspect because the verifier only learns the bit of the prover and nothing about how to distinguish the states. 

    Since quantum state distinguishability is a complete problem for QSZK and the interaction described above can be performed in \qipeb, QSZK $\subseteq$ \qipeb follows.

    %\subsection{\texorpdfstring{\qipeb}{Lg} \texorpdfstring{$\subseteq$}{Lg} QIP}

    %\label{app:qipeb-2-in-qip}
    
    %{\color{red} Appendix \qipeb $\subseteq$ QIP commented out for now. }
    
    %To see the containment \qipeb\ $\subseteq$ QIP, we need to simulate \qipeb\ in QIP. If we can simulate \qipeb\ in QIP(4), then we know from the general result of \cite{kitaev2000parallelization} that QIP(4) = QIP(3) = QIP. 
    
    %The first step begins with the verifier acting with the first circuit from the original \qipeb\ protocol. Then the verifier sends some of the output qubits to the QIP prover.

    %Now, we need to simulate a general entanglement-breaking channel from \qipeb\ with, say, $p(n)$ input qubits and $q(n)$ output qubits, where $p(n)$ and $q(n)$ are polynomials in the complexity parameter $n$. What we need is a handle on the size of the classical register in between. In general, this could be arbitrarily large. However, the Choi state of this channel is separable, and the bound $\left\vert\mathcal{X}\right\vert\leq\text{rank}(\sigma_{AB})^{2}$ \cite{watrous_2018} stated in the main text implies that the number of terms in the decomposition need not be any larger than $2^{2 p(n) q(n)}$. This means that the classical register for the channel need not be any larger than $2 p(n)  q(n)$ qubits.

    %So the rest of the QIP simulation consists of the prover applying a general unitary, sending back $2 p(n) q(n)$ qubits, the verifier decohering / measuring these qubits by attaching to each an ancilla in the all-zeros state $|0\rangle\!\langle 0|$, applying CNOTs, and keeping the second qubits in each pair locally. Then the verifier sends the decohered qubits to the prover, who acts with a general unitary, and sends back some of the outputs to the verifier. The verifier then acts with the final unitary from the original \qipeb\ protocol and measures the decision qubit.

    %This concludes the proof, as all entanglement-breaking channels can be simulated in this way, and the number of classical bits needed in the middle, by the above argument, need not be larger than twice  the product of the number of input and output qubits, which is still polynomial in the input length. 

    %\subsection{Error Reduction for \texorpdfstring{\qipeb}{Lg} via Parallel Repetition}

    %\label{app:par-rep}

    %{\color{red} Appendix Error Reduction for \qipeb via Parallel Repetition commented out for now. }
    
    %An important question that arises when defining a complexity class is the issue of repetition, since the verifier cannot make a decision from just one trial, but rather needs to accumulate evidence before doing so \cite{watrous2009complexity,VW15}. For complexity classes like BPP (efficient probabilistic classical computing) or BQP (efficient quantum computing), this issue is a simple matter of repeating the algorithm a sufficient number of times and taking a majority vote of the outcomes. By an application of the Chernoff bound, it follows that a polynomial number of repetitions implies an exponential decrease in the error probability of making an incorrect decision.

    %This issue becomes more subtle when interacting with a quantum prover, because the prover has the ability to correlate his actions across multiple trials \cite{watrous2009complexity,VW15}. Thus, when defining the complexity class \qipeb\ and performing a \qipeb\ algorithm multiple times in parallel, this issue arises because the prover can act with a joint entanglement-breaking channel across all of the systems received from the verifier and then return a state that is entangled across all systems for the final measurements of the verifier. Thus, it is necessary to analyze a method for reducing errors in this case.

    %Fortunately, for the case of analyzing this problem with \qipeb, we can make direct use of a prior analysis \cite[Section~3.2]{JUW09} for the parallel repetition and error reduction for the class QIP(2), essentially verbatim. We note that other approaches for error reduction with QIP(2) are available~\cite{molina2012parallel,hornby2018concentration}, but we do not make use of them here.

    %To summarize the idea from \cite[Section~3.2]{JUW09}, starting from a fixed \qipeb\ protocol, which consists of an initial verifier circuit $V_1$, a final verifier circuit $V_2$ with a measurement on a single decision qubit, and predetermined bounds $a$ and $b$ on the completeness and soundness probabilities, we define  a new \qipeb\ parallelized protocol, which consists of an initial verifier circuit given by $V_1^{\otimes st}$ and a final verifier circuit given by $V_2^{\otimes st}$, where $s$ and $t$ are positive integers that are no more than polynomial in the problem size. The verifier then receives $st$ measurement outcomes (bits) after measuring all of the decision qubits of $V_2^{\otimes st}$ and has to decide whether to accept or reject based on them. The verifier groups the measurement outcomes into $s$ groups of $t$ measurement outcomes each. For the $i$th group, let $z_i$ be equal to one if the number of times the protocol accepted in the $i$th group, divided by $t$, is greater than a threshold equal to the average $(a+b)/2$, and let it be equal to zero otherwise. The parallelized verifier then accepts if and only if every $z_i$ is equal to one.
    
    %By the same analysis given in \cite[Section~3.2]{JUW09}, if we are in a yes instance in which the verifier should accept, then the prover can act with a tensor product $\mathcal{E}^{\otimes n}$ of the entanglement-breaking channel $\mathcal{E}$ that leads to the highest acceptance probability in an individual trial. By doing so, it then follows from the analysis in \cite[Section~3.2]{JUW09} that the acceptance probability of the parallelized protocol in this case is exponentially close to one.

    %Now consider the case that we are in a no instance in which the verifier should not accept. By the containment \qipeb~$\subseteq$~QIP discussed in Section~\ref{app:qipeb-2-in-qip}, it follows that there is a QIP simulation of the \qipeb\ protocol, with no change in the completeness and soundness probabilities. Thus, following the same reasoning from \cite[Section~3.2]{JUW09}, the $t$ parallel repetitions of the protocol in the $i$th group (call this the $i$th protocol) can be precisely simulated by a QIP, and the general result from \cite{kitaev2000parallelization} on parallel repetition of QIP applies here. Thus, the acceptance probability of the overall parallelized protocol is no greater than the product of the acceptance probabilities for the individual protocols, and the parameters $s$ and $t$ can be chosen in such a way that the acceptance probability of the overall protocol decays exponentially fast to zero.

    %Thus, for the parallelized protocol with the classical postprocessing discussed above and with $st$ no more than a polynomial in the problem size, the acceptance probability increases to one exponentially fast in the case of a yes instance, and it decays to zero exponentially fast in the case of a no instance, as desired.
    
\section{Local Reward Function}
\label{appendix:local_cost}

    In this appendix, we develop a local reward function as an alternative to the global reward function considered in the main text, i.e., the acceptance probability in  Theorem~\ref{theorem:global_cost_func}. 
    The acceptance probability in Theorem~\ref{theorem:global_cost_func} can be considered a global reward function because it corresponds to the probability of measuring zero in every register. As indicated in \cite{Cerezo2021a}, it is helpful to employ a local reward function to mitigate the barren plateau problem \cite{McClean2018}, which plagues all variational quantum algorithms. 
    
    Let us define the local and global reward functions. Let $Z_i$ be the event of measuring zero in the $i$th register. We then set the local reward function to be the probability of measuring zero in a register chosen uniformly at random; that is, it is given by the following:
    \begin{equation}
        L \equiv \frac{1}{n} \sum_i \Pr\!\left(Z_i\right).
    \end{equation}
    The event of measuring all zeros is given by $\bigcap_i Z_i$, and the probability that this event occurs is $G \equiv \Pr\!\left(\bigcap_i Z_i\right)$, which is what we used in the main text as the global reward function.
    
    We are interested in determining inequalities related to the global and local reward functions, and the following analysis employs the same ideas used in \cite[Appendix C]{Khatri2019quantumassisted}. Using DeMorgan's laws, we find that
    \begin{equation}
        \Pr\!\left(\bigcap_i Z_i\right) = \Pr\!\left(\left(\bigcup_i Z_i^c\right)^c\right) = 1- \Pr\!\left(\bigcup_i Z_i^c\right).
    \end{equation}
    We can then use the union bound to conclude that
    \begin{equation}
        \Pr\!\left(\bigcap_i Z_i\right) = 1- \Pr\!\left(\bigcup_i (Z_i)^c\right) \geq 1- \sum_i \Pr\!\left((Z_i)^c\right).
    \end{equation}
    Finally, consider that
    \begin{align}
    \label{eqn:lower_bnd_local_cost}
        G & = \Pr\!\left(\bigcap_i Z_i\right) \\
        & \geq 1- \sum_i \Pr\!\left(Z_i^c\right) \\
        & = \sum_i \Pr\!\left(Z_i\right) - (n-1)\\
        & = n L - (n-1)\\
        & = n (L-1) + 1.
    \end{align}

    We can also derive an upper bound on the global reward function in terms of the local reward function. Recall the following inequality, which holds for every set $\{A_1,A_2,\ldots,A_n\}$ of events:
    \begin{equation}
        \Pr\!\left(\bigcup_i A_i\right) \geq \frac{1}{n} \sum_i \Pr\!\left(A_i\right).
    \end{equation}
    Setting $A_i=Z_i^c$, we get
    \begin{equation}
        \Pr\!\left(\bigcup_i Z_i^c\right) \geq \frac{1}{n} \sum_i \Pr\!\left(Z_i^c\right).
    \end{equation}
    Using DeMorgan's laws, we obtain the desired upper bound as follows:
    \begin{align}
        \Pr\!\left(\bigcap_i Z_i\right) & \leq 1-\frac{1}{n} \sum_i \left(1-\Pr\!\left(Z_i\right)\right)\\
        &= \frac{1}{n} \sum_i \Pr\!\left(Z_i\right).
    \end{align}

    In summary, we have established the following bounds:
    \begin{equation}
          n (L-1) + 1 \leq  G \leq L,
    \end{equation}
    so that $G = 1 $ if and only if $L=1$. Since we always have $G \in [ 0,1]$, the lower bound is only nontrivial if $L$ is sufficiently large, i.e., if $L \geq 1 - \frac{1}{n}$.    
\end{document}

\input{preamble/theoremsetup}

\bibliographystyle{apsrev4-2}

\begin{document}

\title{Quantum Signal Processing, Phase Extraction, and Proportional Sampling}

\author{Lorenzo Laneve}
\email{lorenzo.laneve@usi.ch}
\affiliation{Faculty of Informatics — Universit\`a della Svizzera Italiana, 6900 Lugano, Switzerland}


%\keywords{quantum signal processing, quantum algorithms, quantum state preparation}


\begin{abstract}
\noindent Quantum Signal Processing (QSP) is a technique that can be used to implement a polynomial transformation $P(x)$ applied to the eigenvalues of a unitary $U$, essentially implementing the operation $P(U)$, provided that $P$ satisfies some conditions that are easy to satisfy. A rich class of previously known quantum algorithms were shown to be derived or reduced to this technique or one of its extensions. In this work, we show that QSP can be used to tackle a new problem, which we call \emph{phase extraction}, and that this can be used to provide quantum speed-up for \emph{proportional sampling}, a problem of interest in machine-learning applications and quantum state preparation. We show that, for certain sampling distributions, our algorithm provides an almost-quadratic speed-up over classical sampling procedures. Then we extend the result by constructing a sequence of algorithms that increasingly relax the dependence on the space of elements to sample.
\end{abstract}
\maketitle

\twocolumngrid 

\section{Introduction}
\label{sec:introduction}
% \begin{itemize}
%     % Diffusion of FL
%     \item {\st{Diffusion of FL}}
%     % Security threats to FL
%     \item {\st{Security threats to FL with particular focus on model poisoning}}
%     % Limitations of existing countermeasures
%     \item {\st{Current countermeasures (e.g., KRUM) and their limitations}}
%     % Proposed method and its advantages
%     \item {\st{Intuitive description of the proposed method and its difference (i.e., advantages) w.r.t. state of the art}}
%     % Main contributions
%     \item {\st{Summary of the main contributions of this work}}
%     % Paper's structure and organization
%     \item {\st{Paper's structure and organization}}
% \end{itemize}

% Diffusion of FL
Recently, {\em federated learning} (FL) has emerged as the leading paradigm for training distributed, large-scale, and privacy-preserving machine learning (ML) systems~\cite{mcmahan2017googleai,mcmahan2017aistats}. 
The core idea of FL is to allow multiple edge clients to collaboratively train a shared, global model without disclosing their local private training data.
%Specifically, an FL system consists of a central server and many edge clients; 
A typical FL round involves the following steps: {\em(i)} the server randomly picks some clients and sends them the current, global model; {\em(ii)} each selected client locally trains its model with its own private data; then, it sends the resulting local model to the server;\footnote{Whenever we refer to global/local model, we mean global/local model {\em parameters}.} {\em(iii)} the server updates the global model by computing an \emph{aggregation function}, usually the average (FedAvg), on the local models received from clients.
% \begin{enumerate}
%     \item[{\em(i)}] the server sends the current, global model to the clients and appoints some of them for training;
%     \item[{\em(ii)}] each selected client locally trains its copy of the global model with its own private data; then, it sends the resulting local model back to the server;\footnote{Whenever we refer to global/local model, we mean global/local model {\em parameters}.}
%     \item[{\em(iii)}] the server updates the global model by computing an \emph{aggregation function} on the local models received from clients (by default, the average, also referred to as FedAvg~\cite{mcmahan2017aistats}).
% \end{enumerate}
This process goes on until the global model converges. %(e.g., after a certain number of rounds or other similar stopping criteria).
%\\
% The advantages of FL over the traditional, centralized learning paradigm are undoubtedly clear in terms of flexibility/scalability (clients can join/disconnect from the FL network dynamically), network communications (only model weights\footnote{We will use \textit{parameters} and \textit{weights} interchangeably.} are exchanged between clients and server), and privacy (each client's private training data is kept local at the client's end and not uploaded to the server).
\\
% Security threats to FL
%However, the growing adoption of FL also raises security concerns~\cite{costa2022covert}, particularly about its confidentiality, integrity, and availability.
Although its advantages over standard ML, FL also raises security concerns~\cite{costa2022covert}. %, particularly about its confidentiality, integrity, and availability~\cite{costa2022covert}.
% OLD, LONG VERSION
% Indeed, some work deals with privacy leakage that may expose the local data of some clients~\cite{melis2019sp}. 
% A large body of work, instead, investigates attacks that usually aim to detriment the predictive accuracy of the learned global model. For instance, \emph{data poisoning} attacks achieve this goal by letting an adversary pollute the training set of some corrupt FL clients with maliciously crafted examples~\cite{jagielski2018sp}.
% Similarly, in \emph{model poisoning} the attacker attempts to tweak the global model weights~\cite{bhagoji2019pmlr} by directly perturbing the local model's weights of some infected FL clients before these are sent to the central server for aggregation, usually via so-called Byzantine attacks. 
% It turns out that Byzantine model poisoning attacks severely impact standard FedAvg; therefore, more robust aggregation functions must be designed to make FL systems secure.
Here, we focus on \emph{untargeted model poisoning} attacks~\cite{bhagoji2019pmlr}, where an adversary attempts to tweak the global model weights %\footnote{We will use the terms \textit{parameters} and \textit{weights} interchangeably.} 
by directly perturbing the local model's parameters of some infected clients before these are sent to the central server for aggregation.
In doing so, the adversary aims to jeopardize the global model \textit{indiscriminately} at inference time.
Such model poisoning attacks severely impact standard FedAvg; therefore, more robust aggregation functions must be designed to secure FL systems.
\\
% In this paper, we focus on designing a novel robust aggregation scheme at the server's end to contrast the effect of Byzantine model poisoning attacks.
%
% Current countermeasures and their limitations
%Several countermeasures have been proposed in the literature to combat model poisoning attacks on FL systems.
% Some methods use simple statistics more robust than plain average to smooth the impact of malicious updates (e.g., Trimmed Mean and FedMedian~\cite{yin2018icml}). 
% Other defenses implement outlier detection techniques to discard malicious updates from the aggregation performed at the server's end. Those are either based on heuristics (e.g., Krum/Multi-Krum~\cite{blanchard2017nips} and Bulyan~\cite{mhamdi2018pmlr}) or data-driven approaches (e.g., K-means clustering~\cite{shen2016acm} or DnC via spectral analysis~\cite{shejwalkar2021ndss}). 
% Finally, some strategies rely on a centralized ``source of trust'' to spot potential malicious updates (e.g., FLTrust~\cite{cao2020fltrust}).
% Several countermeasures have been proposed in the literature to combat model poisoning attacks on FL systems, i.e., to discard possible malicious local updates from the aggregation performed at the server's end. 
% These techniques range from simple statistics more robust than plain average (e.g., Trimmed Mean and FedMedian~\cite{yin2018icml}) to outlier detection heuristics (e.g., Krum/Multi-Krum~\cite{blanchard2017nips} and Bulyan~\cite{mhamdi2018pmlr}) or data-driven approaches (e.g., spectral analysis via K-means clustering~\cite{shen2016acm} or spectral analysis), or methods based on ``source of trust'' (e.g., FLTrust~\cite{cao2020fltrust}).
% OLD, LONG VERSION
%Several countermeasures have been proposed in the literature to combat Byzantine model poisoning attacks on FL systems.
% Descriptive statistics
% For example, Trimmed Mean and FedMedian aggregate local model updates using more robust statistics than standard average~\cite{yin2018icml}.
%
% % Heuristics for outlier detection
% Many existing Byzantine-resilient strategies implement some outlier detection heuristics to discard the model updates sent by potentially malicious clients from the input of the aggregation function.
% One of the most popular heuristics is Krum~\cite{blanchard2017nips}.
% This strategy tries to mitigate the impact of Byzantine attacks by selecting as a global model the local model with the smallest sum of Euclidean distances to {\em all} the other local models.
% Although powerful, Krum requires the server to know (or, at least, estimate) the number of malicious FL clients upfront, which is generally impossible in a realistic attack scenario. %
% Moreover, Krum may become ineffective for complex, high-dimensional model parameter spaces due to the curse of dimensionality.
% Bulyan~\cite{mhamdi2018pmlr} tries to overcome this issue by combining Krum with a variant of Trimmed Mean.
% % Data-driven outlier detection
% Other strategies use data-driven outlier detection techniques -- e.g., via K-means clustering~\cite{shen2016acm} -- to spot potential malicious local model updates. 
% %For instance, Shen et al. propose to cluster local model updates with K-means and thus identify outliers.
%
% % Other techniques
% As far as the server is concerned, any local model received can be from a potential malicious client. 
% FLTrust~\cite{cao2020fltrust} assumes the server acts as a client, i.e., trains a local model on an additional {\em trustworthy} dataset at the server's end and compares it against all the local models from other clients. 
% This way, the server can rely on some ``source of trust'' when discarding potentially malicious clients.
%\\
% Limitations of existing Byzantine-resilient strategies
Unfortunately, existing defense mechanisms either rely on simple heuristics (e.g., Trimmed Mean and FedMedian by~\cite{yin2018icml}) or need strong and unrealistic assumptions to work effectively (e.g., foreknowledge or estimation of the number of malicious clients in the FL system, as for Krum/Multi-Krum~\cite{blanchard2017nips} and Bulyan~\cite{mhamdi2018pmlr}, which, however, cannot exceed a fixed threshold).
Furthermore, outlier detection methods using K-means clustering~\cite{shen2016acm} or spectral analysis like DnC~\cite{shejwalkar2021ndss} do not directly consider the temporal evolution of local model updates received.
Finally, strategies like FLTrust~\cite{cao2020fltrust} require the server to collect its own dataset and act as a proper client, thereby altering the standard FL protocol.
\\
% OLD, LONG VERSION
% Overall, existing Byzantine-resilient strategies are either simple heuristics (e.g., FedMedian) or, if they are more complex, they rely on strong and unrealistic assumptions to work effectively (e.g., knowing the number of malicious clients in the FL system in advance, as for Krum and alike).
% Furthermore, data-driven outlier detection methods do not consider the temporary evolution of local model updates received (e.g., K-means clustering). 
% Finally, strategies like FLTrust requires the server to collect its own dataset and act as a proper client, thereby altering the standard FL protocol.
%
% Description of the proposed method
This work introduces a novel pre-aggregation \textit{filter} robust to untargeted model poisoning attacks. Notably, this filter $(i)$ operates without requiring prior knowledge or constraints on the number of malicious clients and $(ii)$ inherently integrates temporal dependencies. 
The FL server can employ this filter as a preprocessing step before applying \textit{any} aggregation function, be it standard like FedAvg or robust like Krum or Bulyan.
Specifically, we formulate the problem of identifying corrupted updates as a multidimensional (i.e., matrix-valued) time series anomaly detection task. 
The key idea is that legitimate local updates, resulting from well-calibrated iterative procedures like stochastic gradient descent (SGD) with an appropriate learning rate, show \textit{higher predictability} compared to malicious updates. This hypothesis stems from the fact that the sequence of gradients (thus, model parameters) observed during legitimate training exhibit regular patterns, as validated in Section~\ref{subsec:intuition}. %until convergence. 
%This regularity may be more pronounced for smooth convex loss functions, but it can still be captured within an appropriate time window, even for more complex and convoluted loss surfaces. 
%We provide evidence of this claim in Appendix~B, where we show that the average mutual information (i.e., ``predictability''), calculated over pairs of legitimate model updates sent at different FL rounds, is significantly higher than the corresponding computation for a malicious client.
\\
Inspired by the matrix autoregressive (MAR) framework for multidimensional time series forecasting~\cite{chen2021je}, we propose the FLANDERS ({\em \textbf{F}ederated \textbf{L}earning meets \textbf{AN}omaly \textbf{DE}tection for a \textbf{R}obust and \textbf{S}ecure}) filter.
The main advantages of FLANDERS over existing strategies like FLDetector~\cite{zhao2020multivariate} are its resilience to large-scale attacks, where $50\%$ or more FL participants are hostile, and the capability of working under realistic non-iid scenarios.
We attribute such a capability to two key factors: $(i)$ FLANDERS works without knowing a priori the ratio of corrupted clients, and $(ii)$ it embodies temporal dependencies between intra- and inter-client updates, quickly recognizing local model drifts caused by evil players. Below, we summarize our main contributions:

\begin{itemize}
\item[{\em(i)}]
We provide empirical evidence that the sequence of models sent by legitimate clients is more predictable than those of malicious participants performing untargeted model poisoning attacks.
\\
\item[{\em(ii)}] 
We introduce FLANDERS, the first pre-aggregation filter for FL robust to untargeted model poisoning based on multidimensional time series anomaly detection.
\\
\item[{\em(iii)}] 
We integrate FLANDERS into Flower,\footnote{\scriptsize{\url{https://flower.dev/}}} a popular FL simulation framework for reproducibility.
\\
\item[{\em(iv)}] 
We show that FLANDERS improves the robustness of the existing aggregation methods under multiple settings: different datasets, client's data distribution (non-iid), models, and attack scenarios.
\\
\item[{\em(v)}] 
We publicly release all the implementation code of FLANDERS along with our experiments.\footnote{\scriptsize{\url{https://anonymous.4open.science/r/flanders_exp-7EEB}}}
\end{itemize}

% Paper's structure and organization
The remainder of the paper is structured as follows. %some related work and the current state-of-the-art solutions to security issues that FL entails. 
Section~\ref{sec:background} covers background and preliminaries. 
In Section~\ref{sec:related}, we discuss related work.
Section~\ref{sec:problem} and Section~\ref{sec:method} describe the problem formulation and the method proposed. % to tackle it. 
Section~\ref{sec:experiments} gathers experimental results. %, and Section~\ref{sec:limitations} discusses some limitations of this work.
Finally, we conclude in Section~\ref{sec:conclusion}.
 %discusses the limitations of this work and draws future research directions.
%reports conclusions and draws perspectives for future research directions.

%%%%%%% OLD %%%%%%%
%to overcome the resilience of Byzantine failures in distributed Stochastic Gradient Descent computations. 
% The strength of Krum is its time complexity, which is linear in the gradient dimension. 
% However, the robustness of the approach is guaranteed for gradient-based learning applications only when the majority of the clients are not compromised. 
% Besides, the aggregation mechanism of Krum, as well as that of similar methods, is robust from a coarse-grained perspective and does not provide solutions to errors and perturbations that may occur at inference time.
%A related approach to~\cite{blanchard2017nips} is the work of Su et al.~\cite{su2016dc}. Here, the authors propose an iterated approximate agreement to tackle a multi-layer scenario attacked by Byzantine agents. 
%However, the method works efficiently on the sole discrete context and it is inapplicable to continuous state environments.
%\gabri{Maybe, we should just talk about the main limitations of existing countermeasures without digging into their details (or, we can just mention Krum as this is the most popular one). I will move the description of all these methods to the Related Work section.}
\section{Phase Extraction}
\label{sec:phase-extraction-problem}

In this section we define a new problem, which we call \emph{phase extraction}. In some sense, this problem can be seen as the inverse of the simulation problem: while, in order to simulate an Hamiltonian, we need to construct the complex exponential of the given matrix, in the phase extraction problem we essentially want to extract its complex logarithm.

\begin{problem}[Phase Extraction]
    \label{def:phase-extraction}
    Let $H$ be an Hermitian matrix satisfying $||H|| < 1$. Given a controlled version of the unitaries $U = e^{i \pi H}$ and $U^\dag = e^{-i \pi H}$ as oracles and $\varepsilon > 0$, construct a quantum circuit $C_U$ such that:
    $$||\bra{0}_A C_U \ket{0}_A - H|| \le \varepsilon,$$
    where $A$ is the subsystem containing ancilla registers.
\end{problem}
\noindent In other words, if we initially have a quantum circuit acting on its eigenbasis $\{ \ket{\phi_k} \}_k$ as
\begin{align*}
    U \ket{\phi_k} = e^{i \pi \varphi_k} \ket{\phi_k}
\end{align*}
for $\varphi_k \in [0, 1)$, we would like $C_U$ to act on the same eigenbasis as
\begin{align*}
    C_U \ket{0} \ket{\phi_k} = \varphi_k \ket{0} \ket{\phi_k} + \ket{1} \ldots \ .
\end{align*}
Thus, $C_U$ will contain a so-called \emph{block encoding} of the matrix $H$. The reader might argue that the stated problem is ill-defined: a global phase on $U$ would change our desired output, but not the input. However, it is important to note that the unitaries $U, U^\dag$ have to be given in their \emph{controlled} versions, where a global phase on $U$ would be seen as a controlled phase kickback. Moreover, it is worth remarking that this problem is substantially different from the well-known \emph{phase-estimation} problem, where one wants to achieve classical information about~$\varphi_k$, given $\ket{\phi_k}$.

Now we would like to use the quantum eigenvalue transform to apply a polynomial transformation of the eigenvalues of $U$, using the oracles we have. In particular, Haah's work tells us that any Laurent polynomial transformation $F: U(1) \rightarrow SU(2)$ of degree $n$ can be implemented using the QSP construction with only $\bigO(n)$ calls to $U, U^\dag$, provided that $F$ has definite parity (see Appendix~\ref{apx:haah-construction}). It is sufficient for us to have a desired transformation $f: U(1) \rightarrow [-1, 1]$ on one entry of this matrix (from now on we will assume without loss of generality that this entry is the top-left one, i.e., $f(z) = \bra{0} F(z) \ket{0}$).

Therefore, all we need now is to design a class of polynomials uniformly approximating a function of the phase. An interesting observation is that Laurent polynomials on the unit circle can be constructed using Fourier series.

\begin{observation}
    \label{thm:phase-extraction-poly}
    Let $\phi : \R \mapsto \R$ be a $2\pi$-periodic function whose Fourier series uniformly converges, and let $f : U(1) \mapsto \R$ be such that, for any real $x$,
    \begin{align*}
        f(e^{i x}) = \phi(x)
    \end{align*}
    holds. Then, there is a sequence of complex (Laurent) polynomials $P_n : \C \mapsto \C$ which uniformly converges to $f$ on the unit circle.
\end{observation}
This theorem tells us that we can construct a sequence of Laurent polynomials approximating any function of the eigenphase, and this will inherit all the strong convergence properties of Fourier sequences in our domain of interest. From now on, we use $\bar{\phi}(z)$ to denote the Laurent polynomial (or Laurent series) that satisfies $\bar{\phi}(e^{ix}) = \phi(x)$ for any real $x$. Notice that such function is unique since polynomials are fully defined by their behaviour on an infinite set.
\begin{proof}
    If $\phi(x) = \sum_{k \in \Z} c_k e^{i k x}$ is the Fourier series of $\phi(x)$, then
    \begin{align*}
        \bar{\phi}(z) = \sum_{k \in \Z} c_k z^k
    \end{align*}
    and this function is equal to $f$ on the unit circle. If $P_n(x) = \sum_{k = -n}^n c_k e^{ikx}$, then it is sufficient to see:
    \begin{align*}
        ||P_n - \phi||_{\R} = ||\bar{P}_n - \bar{\phi}||_{U(1)} = ||\bar{P}_n - f||_{U(1)},
    \end{align*}
    and the first norm tends to zero as $n \rightarrow \infty$.
\end{proof}
This result can be used to solve an extension of Problem~\ref{def:phase-extraction}. Indeed, we can compute a block encoding of $\phi(H)$ for an arbitrary real function $\phi$, provided it is sufficiently `well-behaved'.

Let us solve Problem~\ref{def:phase-extraction} using this technique: we would like that, for $\theta \in (-\pi, \pi]$, the eigenvalue $e^{i \theta}$ is mapped to $\theta/\pi$. Realistically, we cannot approximate this function in the whole interval, since there is a discontinuity in $\theta = \pm \pi$ and, because of this, the Fourier sequence could take too long to converge, or it could not uniformly converge at all. Therefore, we give up a small portion of the interval: for some small $\delta > 0$, we design a function $\phi_\delta(x)$ that is equal to $\phi(x) = x/\pi$ for every $x \in (-\pi + \delta, \pi - \delta)$, while in the neighbourhoods of $\pm \pi$ we replace its derivative with something continuous and piecewise linear that preserves the $2\pi$-periodicity of $\phi_\delta(x)$ (see Figure~\ref{fig:function-approx}). This implies that $\phi_\delta \in C^1$ by construction and, intuitively, the associated Fourier sum will converge faster. As a consequence, we will have to restrict Problem~\ref{def:phase-extraction} to instances with $||H|| \le 1 - \delta$.

\begin{figure}
    \centering
    \begin{subfigure}[b]{\columnwidth}
         \centering
         \begin{tikzpicture}
\begin{axis}[
    width=250pt,height=100pt,
    xmin=-4,xmax=4,
    ymin=-1.2,ymax=1.2,
    samples=50,
    xtick={-pi, -pi/2, 0, pi/2, pi},
    xticklabels={$-\pi$, $-\frac{\pi}{2}$, $0$, $\frac{\pi}{2}$, $\pi$},
    grid style={line width=.1pt, draw=gray!10},
    axis line style={latex-latex}]
    
    \addplot[blue, ultra thick, domain=-pi:pi] (x, x/pi);
    \addplot[blue, ultra thick, domain=-4:-pi] (x, x/pi + 2);
    \addplot[blue, ultra thick, domain=pi:4] (x, x/pi - 2);

    \addplot[mark=*] coordinates {(-pi,1)};
    \addplot[mark=o] coordinates {(-pi,-1)};
    \addplot[mark=*] coordinates {(pi,1)};
    \addplot[mark=o] coordinates {(pi,-1)};

    \draw [dashed] (axis cs:{-pi},-2) -- (axis cs:{-pi},2);
    \draw [dashed] (axis cs:{pi},-2) -- (axis cs:{pi},2);

    \node at (axis cs:-1.5,0.6) {$\phi(x)$};
  
\end{axis}
\end{tikzpicture}
    \end{subfigure}
    \hfill
    \begin{subfigure}[b]{\columnwidth}
        \centering
        \def\dlt{0.4}
\begin{tikzpicture}
\begin{axis}[
    width=250pt,height=110pt,
    xmin=-4,xmax=4,
    ymin=-1.2,ymax=1.45,
    samples=50,
    xtick={-pi, -pi/2, 0, pi/2, pi},
    xticklabels={$-\pi$, $-\frac{\pi}{2}$, $0$, $\frac{\pi}{2}$, $\pi$},
    grid style={line width=.1pt, draw=gray!10}]
    
    \addplot[red, ultra thick, domain=-pi+\dlt:pi-\dlt] (x, x/pi);
    \addplot[red, ultra thick, domain=-4:-pi-\dlt] (x, x/pi + 2);
    \addplot[red, ultra thick, domain=pi+\dlt:4] (x, x/pi - 2);

    \addplot[red, ultra thick, domain=-pi:-pi+\dlt] (x, x/pi + x*x/\dlt/\dlt + 2*pi*x/\dlt/\dlt - 2*x/\dlt + 1 + pi*pi/\dlt/\dlt - 2*pi/\dlt);
    \addplot[red, ultra thick, domain=-pi-\dlt:-pi] (x, 1 - 2*pi/\dlt - 2*x/\dlt + x/pi - pi*pi/\dlt/\dlt - pi*x/\dlt/\dlt - pi*x/\dlt/\dlt - x*x/\dlt/\dlt);
    \addplot[red, ultra thick, domain=pi-\dlt:pi] (x, -1 + 2*pi/\dlt - 2*x/\dlt + x/pi - pi*pi/\dlt/\dlt + pi*x/\dlt/\dlt + pi*x/\dlt/\dlt - x*x/\dlt/\dlt);
    \addplot[red, ultra thick, domain=pi:pi+\dlt] (x, -1 + 2*pi/\dlt - 2*x/\dlt + x/pi + pi*pi/\dlt/\dlt - pi*x/\dlt/\dlt - pi*x/\dlt/\dlt + x*x/\dlt/\dlt);

    \draw [dashed] (axis cs:{pi-\dlt},-2) -- (axis cs:{pi-\dlt},2);
    \draw [dashed] (axis cs:{pi+\dlt},-2) -- (axis cs:{pi+\dlt},2);
    \draw [dashed] (axis cs:{-pi-\dlt},-2) -- (axis cs:{-pi-\dlt},2);
    \draw [dashed] (axis cs:{-pi+\dlt},-2) -- (axis cs:{-pi+\dlt},2);

    \draw[|<->|] (axis cs:{-pi-\dlt},1) -- node[above] {$2\delta$} (axis cs:{-pi+\dlt},1);
    \draw[|<->|] (axis cs:{pi-\dlt},1) -- node[above] {$2\delta$} (axis cs:{pi+\dlt},1);

    \node at (axis cs:-1.5,0.85) {$\phi_\delta(x)$};
    
\end{axis}
\end{tikzpicture}
\undef\dlt
    \end{subfigure}
    \caption{Construction of $\phi_\delta(x)$ from $\phi(x)$. The two functions are identical except in the $\delta$-neighbourhood of the odd multiples of $\pi$. We have that $\phi_\delta \in C^1$ for any $\delta > 0$, so its Fourier series will converge faster than the series of $\phi(x)$.}
    \label{fig:function-approx}
\end{figure}

\begin{theorem}
    \label{thm:phase-extraction-jackson-rate}
    The function $\bar{\phi}_\delta : U(1) \rightarrow \R$ defined as
    \begin{align*}
        \bar{\phi}_\delta(e^{ix}) = \phi_\delta(x)
    \end{align*}
    for every $x \in \R$ can be $\epsilon$-approximated on the unit circle using a polynomial of degree
    $$d = \Tilde{\bigO}\left(\frac{1}{\delta} \sqrt{\frac{1}{\epsilon}}\right).$$
\end{theorem}
\begin{proof}
    Let $S_{\delta, d}(x) = \sum_{k = -d}^{d} c_k e^{ikx}$ be the Fourier sum of $\phi_\delta$ up to terms of degree $d$. By a result of Jackson~\cite[pp.\ 20--25]{jacksonTheoryApproximation1930a} we know that, since $\phi_\delta \in C^1$ and $\phi'_\delta$ is $2/\delta^2$-Lipschitz, the approximation error of the $d$-th degree Fourier sum is bounded by
    \begin{align*}
        ||S_{\delta, d} - \phi_\delta||_{\R} \le K \frac{\log d}{\delta^2 d^2} = K \frac{1}{\delta^2 d^{2-o(1)}}
    \end{align*}
    for some absolute constant $K$. The above is bounded by $\epsilon$ when $d^{2-o(1)} = \bigO\left(\frac{1}{\epsilon \delta^2}\right)$ or, in other words
    \begin{align*}
        d = \Tilde{\bigO}\left( \frac{1}{\delta} \sqrt{\frac{1}{\epsilon}} \right).
    \end{align*}
    This concludes the proof since we can replace $e^{ix} = z$ in $S_{\delta, d}$ to obtain the sequence of Laurent polynomials $\bar{S}_{\delta, d}$ uniformly converging to $\bar{\phi}_\delta$ on the unit circle with the same rate.
\end{proof}
It is interesting to point out that, if one only cares about a constant $\delta$ (e.g., $\delta = \pi/2$ so we get our good approximation only on the right semicircle), one could increase the smoothness of $\phi_\delta$ to get a better dependence from $1/\epsilon$. The Lipschitz constant increases, but it would only depend on $\delta$. We investigate this improvement in Section~\ref{sec:inductive-smoothening}.

Now we have a Laurent polynomial approximating the function $\bar{\phi}_\delta$ that we would like to achieve. We need to take care of one last problem: remember that Haah's construction requires the Laurent polynomial to be either even or odd. Keep in mind that $\bar{\phi}_\delta(z)$ does not have definite parity (and so do not its approximating polynomials), even though $\phi_\delta$ is odd.

A simple workaround is to split $\bar{S}_{\delta, d}(z)$ into even and odd polynomials $\bar{S}^0_{\delta, d}(z), \bar{S}^1_{\delta, d}(z)$, implement them separately, and then add them up using a simple block encoding (see Figure~\ref{fig:block-encoding-sum-circuit}). The phase extraction function $\phi(x)$ can be split into $\phi_0, \phi_1$ such that $\phi_0(x) = \phi_0(x + \pi)$ and $\phi_1(x) = -\phi_1(x + \pi)$ (the reader can check that the Laurent polynomials associated to their Fourier series are even and odd, respectively). Moreover, both of them are bounded by~$1/2$ in absolute value everywhere, and the sum of their Laurent polynomials gives exactly the Laurent polynomial of $\phi$ (see Figure~\ref{fig:function-approx-even-odd}).

\begin{figure}
    \centering
    \begin{subfigure}[b]{\columnwidth}
         \centering
         \begin{tikzpicture}
\begin{axis}[
    width=250pt,height=100pt,
    xmin=-4,xmax=4,
    ymin=-0.6,ymax=0.6,
    samples=50,
    xtick={-pi, -pi/2, 0, pi/2, pi},
    xticklabels={$-\pi$, $-\frac{\pi}{2}$, $0$, $\frac{\pi}{2}$, $\pi$},
    grid style={line width=.1pt, draw=gray!10},
    axis line style={latex-latex}]
    
    \addplot[green, ultra thick, domain=-pi:0] (x, x/pi + 1/2);
    \addplot[green, ultra thick, domain=0:pi] (x, x/pi - 1/2);
    \addplot[green, ultra thick, domain=pi:4] (x, x/pi - 3/2);
    \addplot[green, ultra thick, domain=-4:-pi] (x, x/pi + 3/2);

    \addplot[mark=*] coordinates {(-pi,1/2)};
    \addplot[mark=o] coordinates {(-pi,-1/2)};
    \addplot[mark=*] coordinates {(0,1/2)};
    \addplot[mark=o] coordinates {(0,-1/2)};
    \addplot[mark=*] coordinates {(pi,1/2)};
    \addplot[mark=o] coordinates {(pi,-1/2)};

    \draw [dashed] (axis cs:{-pi},-2) -- (axis cs:{-pi},2);
    \draw [dashed] (axis cs:0,-2) -- (axis cs:0,2);
    \draw [dashed] (axis cs:{pi},-2) -- (axis cs:{pi},2);

    \node at (axis cs:-1.5,0.35) {$\phi_0(x)$};
  
\end{axis}
\end{tikzpicture}
    \end{subfigure}
    \hfill
    \begin{subfigure}[b]{\columnwidth}
        \centering
        \input{figures/function-approx-odd.tex}
    \end{subfigure}
    \caption{Plots of $\phi^0(x)$ and $\phi^1(x)$. One can see that their sum will give exactly $\phi(x)$. Since $\phi(x + \pi) = \bar{\phi}(e^{i(x + \pi)}) = \bar{\phi}(-e^{ix})$, this determines the parity of the approximating Laurent polynomials.}
    \label{fig:function-approx-even-odd}
\end{figure}

In the end, this procedure has a certain failure probability due to non-unitarity of the matrix we are extracting, which can be mitigated by classical repetition or amplitude amplification techniques~\cite{brassardQuantumAmplitudeAmplification2002, berryExponentialImprovementPrecision2014a, berrySimulatingHamiltonianDynamics2015}. It is now important to estimate the initial probability of measuring $\ket{00}$ (i.e., our success probability), in order to bound the multiplicative factor that the amplifying procedure gives to our overall complexity. However, this strictly depends on the unitary $U$ whose phases we want to extract: for example, if $U$ has all the eigenphases close to $0$, then all of them will be mapped to very small amplitudes by $\bar{\phi}_\delta(z)$, giving us a low probability of success, which is harder to amplify. In the next section we will see a concrete example.

\begin{figure}
    \centering
    \begin{quantikz}
    \lstick{$\ket 0_A$} & \gate{H} & \octrl{1} & \ctrl{1} & \gate{H} & \qw \\
    \lstick{$\ket 0_B$} & \qw & \gate[wires=2]{
    \begin{bmatrix}
        \bar S^0(U) & \cdot \\
        \cdot & \cdot
    \end{bmatrix}
    } & \gate[wires=2]{
    \begin{bmatrix}
        \bar S^1(U) & \cdot \\
        \cdot & \cdot
    \end{bmatrix}
    } & \qw & \qw \\
    \lstick{$\ket\phi$} & \qw & & & \qw & \qw
\end{quantikz}
    \caption{Implementation of the block encoding for summing two matrices. The two big gates represent a quantum eigenvalue transform on the unitary $U$ implementing the polynomials $\bar S^0, \bar S^1$. If $\ket{00}$ is measured on the two control qubits, then $\ket{\phi}$ will be transformed by the matrix $\bar S^0(U) + \bar S^1(U) \equiv \bar S(U)$, implementing the original Fourier approximation.}
    \label{fig:block-encoding-sum-circuit}
\end{figure}
\section{Application to Proportional Sampling}
\label{sec:prop-sampling}

\begin{figure*}
    \centering
    \begin{quantikz}
    \lstick{$\ket{x}$} & \qwbundle[alternate]{} & \gate[4, nwires=3]{\bigO'_c}\qwbundle[alternate]{} & \qwbundle[alternate]{} & \qwbundle[alternate]{} & \qwbundle[alternate]{} & \gate[4, nwires=3]{(\bigO'_c)^\dag}\qwbundle[alternate]{} & \qwbundle[alternate]{}\rstick{$e^{i\pi \sqrt{\Tilde{c}(x)}/2}\ket{x}$} \\
    \lstick[wires=3]{$\ket{0}^{\otimes m'}$} & \qw & & \ctrl{3} & \ \qw\ldots \ & \qw & & \qw\rstick[wires=3]{$\ket{0}^{\otimes m'}$} \\
    & \qdots & & & \ddots & & & \qdots \\
    & \qw & & \qw & \ \qw\ldots \ & \ctrl{1} & & \qw \\
    \lstick{$\ket{1}$} & \qw & \qw & \gate{e^{i\pi/{2^2}}} & \ \qw\ldots \ & \gate{e^{i\pi/2^{m+1}}} & \qw & \qw\rstick{$\ket{1}$}
\end{quantikz}

    \caption{Implementation of a phase oracle for $\sqrt{\Tilde{c}(x)}$ using one call to $\bigO'_c, (\bigO'_c)^\dag$. The total phase applied to the quantum state will be $e^{i \pi \sqrt{\Tilde{c}(x)} / 2}$ where $|\sqrt{\Tilde{c}(x)} - \sqrt{c(x)}| \le 2^{-m}$. The two ancilla registers can then be discarded.}
    \label{fig:phase-oracle}
\end{figure*}

\noindent We consider the following problem.
\begin{problem}
    \label{def:prop-sampling}
    Consider a set $X$ of elements. We are given a function $c(x) : X \mapsto [0, 1)$ as an oracle, i.e.,
    \begin{align*}
        \bigO_c \ket{x} \ket{0}^{\otimes m} = \ket{x} \ket{c(x)} \ ,
    \end{align*}
    where $\ket{c(x)}$ contains the $m$ bits of $c(x)$ after the decimal point. Sample a value $x \in X$ such that $P(x) \propto c(x)$ up to some error $\epsilon > 0$. In other words:
    \begin{align*}
        \left| P(x) - \frac{c(x)}{\sum_{x \in X} c(x)} \right| \le \epsilon \ .
    \end{align*}
\end{problem}
For ease of exposition, we assume that the elements of $X$ are encoded as integers, i.e., $X = \{ 0, \ldots, N - 1 \}$, where $N \le 2^n \le 2N$. Moreover, we assume to know the \emph{average oracle value} $\bar{c} := \frac{1}{N} \sum_{x \in X} c(x)$.

In pratical applications, one usually wants to bound the so-called \emph{total variation distance} between the target and implemented probability distributions (see Section 4.1 of~\cite{levinMarkovChainsMixing2017} for a nice introduction to this metric). In order to bound this distance by $\epsilon$ it is sufficient to solve Problem~\ref{def:prop-sampling} within $\epsilon/N$ error. Quantum speed-up for this stronger version of the problem can be achieved using the improvements proposed in Section~\ref{sec:inductive-smoothening}.

\subsection{Constructing a phase oracle}
Since we want $c(x)$ as a probability, the idea is to extract the phases using the transformation $\bar{\phi}(z)$ designed in Section~\ref{sec:phase-extraction-problem} on the unitary
\begin{align*}
    U \ket{x} = e^{i \pi \sqrt{c(x)} / 2} \ket{x} \ .
\end{align*}
The factor $\pi/2$ will be clear later. Of course we cannot reproduce an arbitrary square root with an infinite number of digits as our eigenphases, but here we show a simple construction for the following unitary
\begin{align*}
    U' \ket{x} = e^{i \pi \sqrt{\Tilde{c}(x)} / 2} \ket{x} \ ,
\end{align*}
where $\sqrt{\Tilde{c}(x)}$ is the $m'$-bit truncation of $\sqrt{c(x)}$. Using the bit oracle $\bigO_c$ given in Problem~\ref{def:prop-sampling}, we can construct a bit oracle $\bigO'_c$ behaving as
\begin{align*}
    \bigO'_c \ket{x} \ket{0}^{\otimes m'} = \ket{x} \ket{\sqrt{\Tilde{c}(x)}} \ ,
\end{align*}
by decorating one call to the original oracle with a square root algorithm of $m'$-bit precision.

Now, using a standard construction (depicted in Figure~\ref{fig:phase-oracle}), we can implement $U'$ using $\bigO'_c$ and $(\bigO'_c)^\dag$. Only two copies of the original oracles $\bigO_c, \bigO^\dag_c$ are needed to construct $U'$.

\subsection{Bounding the approximation error}
By transforming $U'$ using $\bar{\phi}_\delta(z)$ we obtain an Hermitian block-encoded matrix $H_c$ acting as
\begin{align}
    H_c \ket{x} & = \bar\phi_\delta(e^{i \pi \sqrt{\Tilde{c}(x)} / 2}) \ket{x}\nonumber \\
    & = \phi_\delta \left(\frac{\pi}{2} \sqrt{\Tilde{c}(x)} \right) \ket{x} = \frac{1}{2} \sqrt{\Tilde{c}(x)} \ket{x} \label{eq:prop-sampling-ideal-transform}
\end{align}
provided that $|\frac{\pi}{2} \sqrt{\Tilde{c}(x)}| \le \pi - \delta$. It should now be evident why we added the $\pi/2$ factor, as we can fix $\delta$ as high as $\pi/2 = \Theta(1)$. If we applied this ideal transformation to the uniform superposition, we would obtain
\begin{align*}
    H_c \ket{+}^{\otimes n} = \frac{1}{2 \sqrt{N}} \sum_{x \in X} \sqrt{\Tilde{c}(x)} \ket{x}
\end{align*}
We have two sources of error on the final sampling probabilities: the first one, $|c(x) - \Tilde{c}(x)|$, is given by the $m'$-bit truncation and can be bounded easily
\begin{align*}
    |c(x) - \Tilde{c}(x)| & = |\sqrt{c(x)} + \sqrt{\Tilde{c}(x)}| \cdot |\sqrt{c(x)} - \sqrt{\Tilde{c}(x)}| \\
    & \le 2 \cdot \frac{1}{2^{m'}} = \frac{1}{2^{m'-1}} \stackrel{!}{\le} \frac{\epsilon'}{2} \ ,
\end{align*}
where the last inequality holds for $m' = \bigO(\log \epsilon')$. The second noise comes from the approximation of $\phi_\delta$ by $S_{\delta, d}$: if the approximation is up to $\epsilon'/16$ then
\begin{align*}
    \left| S^2_{\delta, d}\left(\frac{\pi}{2} \sqrt{c} \right) - \frac{1}{4} c \right| & = \left| S^2_{\delta, d}\left(\frac{\pi}{2} \sqrt{c} \right) - \phi^2_\delta\left(\frac{\pi}{2} \sqrt{c} \right) \right| \\
    & \le ||S_{\delta, d} + \phi_\delta||_{\R} \cdot ||S_{\delta, d} - \phi_\delta||_{\R} \\
    & \le 2 \cdot \frac{\epsilon'}{16} = \frac{\epsilon'}{8}
\end{align*}
for any $c \in [0, 1)$ (the $1/4$ factor comes from the $1/2$ factor in Eq.~(\ref{eq:prop-sampling-ideal-transform})). This is guaranteed by Theorem~\ref{thm:phase-extraction-jackson-rate} if we take
\begin{align*}
    d = \Tilde{\bigO}\left( \frac{1}{\delta} \sqrt{\frac{8}{\epsilon'}} \right) = \Tilde{\bigO}\left( \sqrt{\frac{1}{\epsilon'}} \right)
\end{align*}
as $\delta$ was already fixed to be constant. Therefore, the distance between the ideal oracle $c(x)$ and our implementation $s(x) := 4 S^2_{\delta, d}(\frac{\pi}{2} \sqrt{\Tilde{c}(x)})$ is
\begin{align*}
    \left|s(x) - c(x)\right| & \le \left|s(x) - \Tilde{c}(x)\right| + |\Tilde{c}(x) - c(x)| \\
    & \le 4 \frac{\epsilon'}{8} + \frac{\epsilon'}{2} \le \epsilon' \ .
\end{align*}
This is needed to bound the error induced in the sampling probabilities, but it is not sufficient, as this error bound will now be amplified by the amplitude amplification scheme.

\subsection{Amplifying the state}
The state $H_c \ket{+}^{\otimes n}$ is sub-normalized, because $H_c$ is not a unitary transformation, but only a block-encoded Hermitian matrix. By looking again at Figure~\ref{fig:block-encoding-sum-circuit}, we can see that we have two control qubits, $A$ and $B$. We measure both these qubits in the computational basis and, if we measure $\ket{00}$, then we picked the correct block. The probability of this happening is:
\begin{align*}
    || H_c \ket{+}^{\otimes n} ||^2 & = \frac{1}{N} \sum_{x \in X} S^2_{\delta, d} \left(\frac{\pi}{2} \sqrt{\Tilde{c}(x)} \right) \\
    & \ge \frac{1}{4 N} \sum_{x \in X} (c(x) - \epsilon') =: \frac{1}{4}(\bar{c} - \epsilon')\ ,
\end{align*}
and we take $\epsilon' = \frac{\bar{c} \epsilon}{2} \le \frac{\bar{c}}{2}$. Since an upper bound $\frac{1}{4}(\bar{c} + \epsilon')$ can be derived in a similar fashion, the initial amplitude is $\Theta(\sqrt{\bar{c}})$ and we can employ an \emph{amplitude amplification} procedure (see Appendix~\ref{apx:amplitude-amplification}) to normalize the state with $\bigO\left(\frac{1}{\sqrt{\bar{c}}}\right)$ repetitions. Using $c^*, s^*$ as shorthands for $\sum_x c(x), \sum_x s(x)$, we get
\begin{align*}
    \left| \frac{c(x)}{c^*} - \frac{s(x)}{s^*}\right| & = \frac{1}{s^* c^*} \left|c(x) s^* - s(x) c^* \right| \\
    & = \frac{s(x)}{s^* c^*} \left| s^* - c^* \right| + \frac{1}{c^*} |c(x) - s(x)| \\
    & \le \frac{1}{c^*} \left| s^* - c^* \right| + \frac{1}{c^*} |c(x) - s(x)| \\
    & \le \frac{N \epsilon'}{c^*} + \frac{\epsilon'}{c^*} = \frac{\epsilon'}{\bar{c}} + \frac{\epsilon'}{N \bar{c}} \\
    & \le \frac{2 \epsilon'}{\bar{c}}  \stackrel{!}{\le} \epsilon \ ,
\end{align*}
and one can see that our choice of $\epsilon'$ always satisfies the last inequality. To sum up, we obtained the following algorithm:
\begin{algorithm}[Proportional Sampling by QSP]
\label{alg:prop-sampling-qsp}
Let $\bar{c} = \frac{1}{N} \sum_{x \in X} c(x)$ be the average oracle value, and fix $\delta = \frac{\pi}{2}$.
\begin{enumerate}
        \item Use Quantum Eigenvalue Transform on the eigenvalues of $U'$ using $\bar{S}_{\delta, d}(z)$, obtaining a block-encoding of $\bar{S}_{\delta, d}(U')$. This requires $d$ calls to $U'$ or, equivalently, $2d$ calls to $\bigO_c$.

        \item Compute the state $\bar{S}_{\delta, d}(U') \ket{+}^{\otimes n}$.

        \item Use OAA to amplify the above state. This requires $\bigO(\frac{1}{\sqrt{\bar{c}}})$ copies of $\bar{S}_{\delta, d}(U')$.
    \end{enumerate}
    The number of total calls to $\bigO_c$ will be
    \begin{align*}
        \bigO \left( \frac{1}{\sqrt{\bar{c}}} \cdot d \right) = \Tilde{\bigO} \left( \frac{1}{\bar{c}} \sqrt{\frac{1}{\epsilon}} \right)
    \end{align*}
    to achieve error up to $\epsilon$ with high probability.
\end{algorithm}
We remark that Algorithm~\ref{alg:prop-sampling-qsp} is a so-called \emph{Las Vegas} algorithm: whenever it fails (i.e., we pick the wrong block of the encoding), we immediately know, because we measure $\neq 00$ on the control qubits. Therefore, if this check fails, we repeat the whole algorithm, and the number of repetitions is constant in expectation, as the probability of success is lower bounded by a constant after the amplification scheme.

\subsection{Separation proof}
In this subsection, we show that Algorithm~\ref{alg:prop-sampling-qsp} gives actual speed-up over any classical algorithm. Consider the following instance: we assume $N$ to be even and divide $\{ 0, \ldots, N - 1 \}$ into two equally sized sets $A, B$, where
\begin{align*}
    c(x) =
    \begin{cases}
        \frac{1}{4} & x \in A \\
        \frac{1}{8} & x \in B
    \end{cases}
\end{align*}
One can see that, in this case, $\bar{c} = 3/16$ and if we want to sample up to error $\epsilon = 1/100 N$, Algorithm~\ref{alg:prop-sampling-qsp} samples correctly with $\Tilde{\bigO}(\sqrt{N})$ queries to the oracle.

\begin{theorem}
    No classical algorithm can solve Problem~\ref{def:prop-sampling} with less than $N - 1$ queries to the oracle.
\end{theorem}
\begin{proof}
    Let $P_c(\cdot) = \frac{c(x)}{\sum_y c(y)}$ be the probability distribution to approximate for instance $c$. Assume a classical algorithm $\mathcal{A}$ doing at most $N - 2$ queries to the oracle, and let $x_1, x_2$ be two of the values not queried by $\mathcal{A}$. If we take any instance $c'$ by only modifying $c(x_1), c(x_2)$ (in such a way that $\bar{c}' = \bar{c}$), then $\mathcal{A}$ will return the same probability distribution also for these two values, call it $P_{\mathcal{A}}(\cdot)$. Let us consider two separate cases: if one of $x_1, x_2 \in A$ (w.l.o.g. $x_1$), then $P_c(x_1) = 4/3N$, and the intervals of admitted values for correctness are at most $2/100 N$ long. Therefore, if we choose $c'(x_1) = 0$ (and $c'(x_2)$ so that $\bar{c}' = \bar{c}$), then $P_{c'}(x_1) = 0$, and $P_\mathcal{A}(x_1)$ cannot be $\epsilon$-close to both these probabilities.

    If $x_1, x_2 \in B$ setting $c'(x_1) = 0, c'(x_2) = \frac{1}{4}$ gives $P_{c'}(x_1) = 0 < P_{c}(x_1) - \frac{2}{100 N}$, therefore even in this case, $P_{\mathcal{A}}(x_1)$ cannot well-approximate both $c, c'$. We conclude that $\mathcal{A}$ is not correct.
\end{proof}
\section{Faster and Faster Approximations}
\label{sec:inductive-smoothening}
In this section we would like to improve on the rate of convergence of Theorem~\ref{thm:phase-extraction-jackson-rate}, which we can plug into Algorithm~\ref{alg:prop-sampling-qsp} to achieve further speed-up. We restate an important result about Fourier series here, which we also used in the proof of Theorem~\ref{thm:phase-extraction-jackson-rate}:
\begin{theorem}[Jackson~\cite{jacksonTheoryApproximation1930a}, Corollary 3, p.\ 22]
    \label{thm:jackson-rate}
    If $f : \R \rightarrow \R$ is a $2\pi$-periodic function such that its $p$-th derivative is $K$-Lipschitz continuous, then its Fourier sum $S_d(x)$ of $d$-th degree satisfies
    \begin{align*}
        ||S_d - f||_\R \le K \frac{A_p \log d}{d^{p+1}}
    \end{align*}
    where $A_p$ is a constant depending only on $p$.
\end{theorem}
In our example we took $p = 1$, we smoothed out $\phi_\delta$ so that its first derivative is continuous and we noticed that it is $\frac{2}{\delta^2}$-Lipschitz. This led to $d = \Tilde\bigO\left(\frac{1}{\delta} \sqrt{\frac{1}{\epsilon}}\right)$ to bound the approximation error by $\epsilon$ on the unit circle. However in the application for proportional sampling we could do more: we could smooth out all the derivatives up to some constant $p$, taking advantage of the fact that $\delta = \pi/2$.

\begin{figure*}
    \centering
    \def\dlt{0.4}
\begin{tikzpicture}
\begin{axis}[
    width=250pt,height=110pt,
    xmin=pi-0.5,xmax=pi+0.5,
    ymin=-1.2,ymax=1.45,
    samples=50,
    xtick={pi-\dlt, pi, pi+\dlt},
    xticklabels={$\pi - \delta$, $\pi$, $\pi + \delta$},
    ytick={-1, 0, 1},
    yticklabels={$-\frac{2}{\delta^2}$, $0$, $\frac{2}{\delta^2}$},
    grid style={line width=.1pt, draw=gray!10}]

    \addplot[red, ultra thick, domain=0:pi-\dlt] (x, 0);
    \addplot[red, ultra thick, domain=pi+\dlt:4] (x, 0);
    \addplot[red, ultra thick, domain=pi-\dlt:pi] (x, -1);
    \addplot[red, ultra thick, domain=pi:pi+\dlt] (x, 1);

    \draw [dashed] (axis cs:{pi-\dlt},-2) -- (axis cs:{pi-\dlt},2);
    \draw [dashed] (axis cs:{pi},-2) -- (axis cs:{pi},2);
    \draw [dashed] (axis cs:{pi+\dlt},-2) -- (axis cs:{pi+\dlt},2);
    
    \node at (axis cs:2.9,1.1) {$g^{(2)}_1 \equiv \phi''_\delta$};

    \addplot[mark=*] coordinates {(pi-\dlt,0)};
    \addplot[mark=o] coordinates {(pi-\dlt,-1)};
    \addplot[mark=*] coordinates {(pi,-1)};
    \addplot[mark=o] coordinates {(pi,1)};
    \addplot[mark=*] coordinates {(pi+\dlt,1)};
    \addplot[mark=o] coordinates {(pi+\dlt,0)};
\end{axis}
\end{tikzpicture}
\undef\dlt
    \def\dlt{0.4}
\begin{tikzpicture}
\begin{axis}[
    width=250pt,height=110pt,
    xmin=pi-0.5,xmax=pi+0.5,
    ymin=-1.2,ymax=1.45,
    samples=50,
    xtick={pi-\dlt, pi, pi+\dlt},
    xticklabels={$\pi - \delta$, $\pi$, $\pi + \delta$},
    ytick={-1, 0, 1},
    yticklabels={$-\frac{4}{\delta^2}$, $0$, $\frac{4}{\delta^2}$},
    grid style={line width=.1pt, draw=gray!10}]

    \addplot[blue, ultra thick] coordinates {
        (0, 0) (pi-\dlt, 0) (pi-\dlt/2, -1) (pi+\dlt/2, 1) (pi+\dlt, 0) (4,0)
    };

    \draw [dashed] (axis cs:{pi-\dlt},-2) -- (axis cs:{pi-\dlt},2);
    \draw [dashed] (axis cs:{pi},-2) -- (axis cs:{pi},2);
    \draw [dashed] (axis cs:{pi+\dlt},-2) -- (axis cs:{pi+\dlt},2);
    
    \node at (axis cs:2.9,1.1) {$g^{(2)}_2$};
\end{axis}
\end{tikzpicture}
\undef\dlt
    \def\dlt{0.4}
\begin{tikzpicture}
\begin{axis}[
    width=250pt,height=110pt,
    xmin=pi-0.5,xmax=pi+0.5,
    ymin=-1.2,ymax=1.45,
    samples=50,
    xtick={pi-\dlt, pi, pi+\dlt},
    xticklabels={$\pi - \delta$, $\pi$, $\pi + \delta$},
    ytick={-1, 0, 1},
    yticklabels={$-\frac{8}{\delta^3}$, $0$, $\frac{8}{\delta^3}$},
    grid style={line width=.1pt, draw=gray!10}]

    \addplot[blue, ultra thick, domain=0:pi-\dlt] (x, 0);
    \addplot[blue, ultra thick, domain=pi-\dlt:pi-\dlt/2] (x, -1);
    \addplot[blue, ultra thick, domain=pi-\dlt/2:pi+\dlt/2] (x, 1);
    \addplot[blue, ultra thick, domain=pi+\dlt/2:pi+\dlt] (x, -1);
    \addplot[blue, ultra thick, domain=pi+\dlt:4] (x, 0);

    \draw [dashed] (axis cs:{pi-\dlt},-2) -- (axis cs:{pi-\dlt},2);
    \draw [dashed] (axis cs:{pi-\dlt/2},-2) -- (axis cs:{pi-\dlt/2},2);
    \draw [dashed] (axis cs:{pi},-2) -- (axis cs:{pi},2);
    \draw [dashed] (axis cs:{pi+\dlt/2},-2) -- (axis cs:{pi+\dlt/2},2);
    \draw [dashed] (axis cs:{pi+\dlt},-2) -- (axis cs:{pi+\dlt},2);
    
    \node at (axis cs:2.9,1.1) {$g^{(3)}_2$};

    \addplot[mark=*] coordinates {(pi-\dlt,0)};
    \addplot[mark=o] coordinates {(pi-\dlt,-1)};
    \addplot[mark=*] coordinates {(pi-\dlt/2,-1)};
    \addplot[mark=o] coordinates {(pi-\dlt/2,1)};
    \addplot[mark=*] coordinates {(pi+\dlt/2,1)};
    \addplot[mark=o] coordinates {(pi+\dlt/2,-1)};
    \addplot[mark=*] coordinates {(pi+\dlt,-1)};
    \addplot[mark=o] coordinates {(pi+\dlt,0)};
\end{axis}
\end{tikzpicture}
\undef\dlt
    \def\dlt{0.4}
\begin{tikzpicture}
\begin{axis}[
    width=250pt,height=110pt,
    xmin=pi-0.5,xmax=pi+0.5,
    ymin=-1.2,ymax=1.45,
    samples=50,
    xtick={pi-\dlt, pi, pi+\dlt},
    xticklabels={$\pi - \delta$, $\pi$, $\pi + \delta$},
    ytick={-1, 0, 1},
    yticklabels={$-\frac{16}{\delta^3}$, $0$, $\frac{16}{\delta^3}$},
    grid style={line width=.1pt, draw=gray!10}]

    \addplot[green, ultra thick] coordinates {
        (0, 0) (pi-\dlt, 0) (pi-3*\dlt/4, -1) (pi-\dlt/4, 1) (pi, 0) (pi+\dlt/4, 1) (pi+3*\dlt/4, -1) (pi+\dlt, 0) (4,0)
    };

    \draw [dashed] (axis cs:{pi-\dlt},-2) -- (axis cs:{pi-\dlt},2);
    \draw [dashed] (axis cs:{pi-\dlt/2},-2) -- (axis cs:{pi-\dlt/2},2);
    \draw [dashed] (axis cs:{pi},-2) -- (axis cs:{pi},2);
    \draw [dashed] (axis cs:{pi+\dlt/2},-2) -- (axis cs:{pi+\dlt/2},2);
    \draw [dashed] (axis cs:{pi+\dlt},-2) -- (axis cs:{pi+\dlt},2);
    
    \node at (axis cs:2.9,1.1) {$g^{(3)}_3$};
\end{axis}
\end{tikzpicture}
\undef\dlt
    \def\dlt{0.4}
\begin{tikzpicture}
\begin{axis}[
    width=250pt,height=110pt,
    xmin=pi-0.5,xmax=pi+0.5,
    ymin=-1.2,ymax=1.45,
    samples=50,
    xtick={pi-\dlt, pi, pi+\dlt},
    xticklabels={$\pi - \delta$, $\pi$, $\pi + \delta$},
    ytick={-1, 0, 1},
    yticklabels={$-\frac{64}{\delta^4}$, $0$, $\frac{64}{\delta^4}$},
    grid style={line width=.1pt, draw=gray!10}]

    \addplot[green, ultra thick, domain=0:pi-\dlt] (x, 0);
    \addplot[green, ultra thick, domain=pi-\dlt:pi-3*\dlt/4] (x, -1);
    \addplot[green, ultra thick, domain=pi-3*\dlt/4:pi-\dlt/2] (x, 1);
    \addplot[green, ultra thick, domain=pi-\dlt/2:pi-\dlt/4] (x, 1);
    \addplot[green, ultra thick, domain=pi-\dlt/4:pi] (x, -1);
    \addplot[green, ultra thick, domain=pi:pi+\dlt/4] (x, 1);
    \addplot[green, ultra thick, domain=pi+\dlt/4:pi+\dlt/2] (x, -1);
    \addplot[green, ultra thick, domain=pi+\dlt/2:pi+3*\dlt/4] (x, -1);
    \addplot[green, ultra thick, domain=pi+3*\dlt/4:pi+\dlt] (x, 1);
    \addplot[green, ultra thick, domain=pi+\dlt:4] (x, 0);

    \draw [dashed] (axis cs:{pi-\dlt},-2) -- (axis cs:{pi-\dlt},2);
    \draw [dashed] (axis cs:{pi-3*\dlt/4},-2) -- (axis cs:{pi-3*\dlt/4},2);
    \draw [dashed] (axis cs:{pi-\dlt/2},-2) -- (axis cs:{pi-\dlt/2},2);
    \draw [dashed] (axis cs:{pi-\dlt/4},-2) -- (axis cs:{pi-\dlt/4},2);
    \draw [dashed] (axis cs:{pi},-2) -- (axis cs:{pi},2);
    \draw [dashed] (axis cs:{pi+\dlt/4},-2) -- (axis cs:{pi+\dlt/4},2);
    \draw [dashed] (axis cs:{pi+\dlt/2},-2) -- (axis cs:{pi+\dlt/2},2);
    \draw [dashed] (axis cs:{pi+3*\dlt/4},-2) -- (axis cs:{pi+3*\dlt/4},2);
    \draw [dashed] (axis cs:{pi+\dlt},-2) -- (axis cs:{pi+\dlt},2);
    
    \node at (axis cs:2.8,1.1) {$g^{(4)}_3$};

    \addplot[mark=*] coordinates {(pi-\dlt,0)};
    \addplot[mark=o] coordinates {(pi-\dlt,-1)};
    \addplot[mark=*] coordinates {(pi-3*\dlt/4,-1)};
    \addplot[mark=o] coordinates {(pi-3*\dlt/4,1)};
    \addplot[mark=*] coordinates {(pi-\dlt/4,1)};
    \addplot[mark=o] coordinates {(pi-\dlt/4,-1)};
    \addplot[mark=*] coordinates {(pi,-1)};
    \addplot[mark=o] coordinates {(pi,1)};
    \addplot[mark=*] coordinates {(pi+\dlt/4,1)};
    \addplot[mark=o] coordinates {(pi+\dlt/4,-1)};
    \addplot[mark=*] coordinates {(pi+3*\dlt/4,-1)};
    \addplot[mark=o] coordinates {(pi+3*\dlt/4,1)};
    \addplot[mark=*] coordinates {(pi+\dlt,1)};
    \addplot[mark=o] coordinates {(pi+\dlt,0)};
\end{axis}
\end{tikzpicture}
\undef\dlt
    \def\dlt{0.4}
\begin{tikzpicture}
\begin{axis}[
    width=250pt,height=110pt,
    xmin=pi-0.5,xmax=pi+0.5,
    ymin=-1.2,ymax=1.45,
    samples=50,
    xtick={pi-\dlt, pi, pi+\dlt},
    xticklabels={$\pi - \delta$, $\pi$, $\pi + \delta$},
    ytick={-1, 0, 1},
    yticklabels={$-\frac{128}{\delta^4}$, $0$, $\frac{128}{\delta^4}$},
    grid style={line width=.1pt, draw=gray!10}]

    \addplot[yellow, ultra thick] coordinates {
        (0, 0) (pi-\dlt, 0) (pi-7*\dlt/8, -1) (pi-5*\dlt/8, 1) (pi-\dlt/2, 0) (pi-3*\dlt/8, 1) (pi-\dlt/8, -1) (pi+\dlt/8, 1) (pi+3*\dlt/8, -1) (pi+\dlt/2, 0) (pi+5*\dlt/8, -1) (pi+7*\dlt/8, 1) (pi+\dlt, 0) (4,0)
    };

    \draw [dashed] (axis cs:{pi-\dlt},-2) -- (axis cs:{pi-\dlt},2);
    \draw [dashed] (axis cs:{pi-3*\dlt/4},-2) -- (axis cs:{pi-3*\dlt/4},2);
    \draw [dashed] (axis cs:{pi-\dlt/2},-2) -- (axis cs:{pi-\dlt/2},2);
    \draw [dashed] (axis cs:{pi-\dlt/4},-2) -- (axis cs:{pi-\dlt/4},2);
    \draw [dashed] (axis cs:{pi},-2) -- (axis cs:{pi},2);
    \draw [dashed] (axis cs:{pi+\dlt/4},-2) -- (axis cs:{pi+\dlt/4},2);
    \draw [dashed] (axis cs:{pi+\dlt/2},-2) -- (axis cs:{pi+\dlt/2},2);
    \draw [dashed] (axis cs:{pi+3*\dlt/4},-2) -- (axis cs:{pi+3*\dlt/4},2);
    \draw [dashed] (axis cs:{pi+\dlt},-2) -- (axis cs:{pi+\dlt},2);
    
    \node at (axis cs:2.8,1.1) {$g^{(4)}_4$};
\end{axis}
\end{tikzpicture}
\undef\dlt
    \caption{Construction of the fourth derivative of $g_4 \in C^4$. To construct the derivative of the next function, we linearize the `last' derivative in order to make it continuous, and we do so by replacing the rectangles with triangles of the same area, so that the integral over $I = (\pi - \delta, \pi + \delta)$ is preserved. In order to obtain $g_4$, we will integrate four times, keeping in mind that every derivative has value $0$ at the origin, except for the first one, which has value $1/\pi$.}
    \label{fig:function-smoothening-1}
\end{figure*}

\begin{figure}
    \centering
    \def\dlt{0.4}
\begin{tikzpicture}
\begin{axis}[
    width=250pt,height=190pt,
    xmin=-0.5,xmax=+0.5,
    ymin=-0.55,ymax=0.05,
    samples=50,
    xtick={},
    xticklabels={},
    ytick={0},
    yticklabels={$0$},
    grid style={line width=.1pt, draw=gray!10},
    legend style={at={(0.39,0.9)}, anchor=north west}]

    \addplot[red, ultra thick] coordinates {
        (-0.5, 0) (0, -0.5) (0.5,0)
    };

    \addplot[blue, ultra thick, domain=-0.5:-0.25] (x, {-4*x*x-4*x-1});
    \legend{$g^{(p)}_p$, $g^{(p)}_{p+1}$}
    \addplot[blue, ultra thick, domain=-0.25:0.25] (x, {4*x*x-0.5});
    \addplot[blue, ultra thick, domain=0.25:0.5] (x, {-4*x*x+4*x-1});
    

    \draw [dashed] (axis cs:{0},-2) -- (axis cs:{0},2);
    
    \node at (axis cs:2.9,1.1) {$g^{(3)}_3$};
\end{axis}
\end{tikzpicture}
\undef\dlt
    \caption{Comparison between a triangle and its first-order smoothing. If this is the $j$-th triangle of $g^{(p)}_{p+1}$, then the interval of the plot is $I^{p-1}_j$, which is split into the two intervals $I^p_{2j}, I^p_{2j+1}$. One can see that the difference of the two functions is odd in each of the two sub-intervals, and even in $I^{p-1}_j$.}
    \label{fig:function-smoothening-parabola}
\end{figure}

We generalize the smoothing procedure shown in Section~\ref{sec:phase-extraction-problem}: one can see that $\phi'_\delta$ is a piecewise linear function.
\begin{claim}
    For $p \ge 1$, we can construct $g_p \in C^p$ that is $2\pi$-periodic, and
    \begin{align}
        \label{eq:general-construction-identity-condition}
        g_p(x) = \frac{x}{\pi}
    \end{align}
    for every $x \in (-\pi + \delta, \pi - \delta)$.
\end{claim}

\begin{proof}
    Note that $g_1 \equiv \phi_\delta$ gives a base for our inductive construction. First of all, Eq.~(\ref{eq:general-construction-identity-condition}) holds iff the derivatives of $g_p$ satisfy:
    \begin{align}
        \label{eq:smoothening-derivative-values}
        g^{(k)}_p(x) & = 
        \begin{cases}
            \frac{1}{\pi} & k = 1 \\
            0 & k \neq 1
        \end{cases}
    \end{align}
    for $x \in (-\pi + \delta, \pi - \delta)$ and every $k \ge 0$. By this condition, we are only allowed to change the behaviour of the function outside of this domain, thus from now on we will only consider the interval $I = (\pi-\delta, \pi+\delta)$.
    
    Starting from $g^{(1)}_1$, which is continuous and piecewise linear, its derivative, $g^{(2)}_1$ will be piecewise constant, i.e., its integral will be given by a sequence of rectangles (Figure~\ref{fig:function-smoothening-1}, top left). We construct the second derivative of the next function $g^{(2)}_2$ by replacing the rectangles of $g^{(1)}_2$ by triangles of the same area (Figure~\ref{fig:function-smoothening-1}, top right). Then $g^{(3)}_2$ will be again piecewise constant, where the number of intervals in which this function is subdivided is twice the intervals of our starting function $g^{(2)}_1$. In general, $g^{(p)}_{p+1}$ will be piecewise constant, $g^{(p+1)}_{p+1}$ is obtained with the above procedure, and its pieces divide $I$ into $2^p$ equal segments. At this point, $g_p$ is easily defined by integrating the constructed derivatives $p+1$ times, with boundary conditions given by Eq.~(\ref{eq:smoothening-derivative-values}):
    \begin{align*}
        g_{p+1}(x) = \int_0^x \left( \frac{1}{\pi} + \underbrace{\int_0^x \cdots \int_0^x}_{p} g^{(p+1)}_{p+1}(x) \ dx^p \right) dx\ .
    \end{align*}
    Figure~\ref{fig:function-smoothening-1} shows the construction of the fourth derivative of $g_4$. We now prove all the claimed properties: $g_p \in C^p$ by construction, and we never changed the derivatives within $\pm (\pi - \delta)$, thus Eq.~(\ref{eq:smoothening-derivative-values}) is preserved (and so is also (\ref{eq:general-construction-identity-condition})).
    We only need to prove periodicity: the key here is to notice that $g^{(2)}_p$ satisfies
    \begin{align*}
        \int_I g^{(2)}_p(t) \ dt = 0 \ ,
    \end{align*}
    implying that $g^{(1)}_p$ is $\frac{1}{\pi}$ at the boundaries of $I$. This because $g^{(2)}_p$ has the same form as $g^{(2)}_1$ in Figure~\ref{fig:function-smoothening-1}, except for the fact that, instead of having two opposite rectangles, we have two opposite (smoother) shapes which will cancel out with one another in the integral (in the case of $p = 2$, these shapes are the triangles as in Figure~\ref{fig:function-smoothening-1}, top right). This shows that $g^{(1)}_p$ is $2\pi$-periodic. It is now sufficient to prove
    \begin{align}
        \label{eq:smoothening-ultimate-integral-condition}
        g_p(2\pi) - g_p(0) = \int_0^{2\pi} g^{(1)}_p(t) \ dt = 0
    \end{align}
    in order to prove periodicity of $g_p$.
    This is already true for $p = 1$, by design: in this case the shape of $g^{(1)}_1$ in $I$ is one big triangle pointing downwards, whose area was chosen in order to satisfy (\ref{eq:smoothening-ultimate-integral-condition}).

    For $p > 1$, this triangle is replaced by a smoother shape, and it is sufficient to prove that such shape has the same area as the original triangle in order to preserve the above integral. Thus we prove the following by induction:
    \begin{align*}
        \int_I g^{(1)}_{p+1}(t) - g^{(1)}_p(t) \ dt = 0
    \end{align*}
    so that by telescoping and linearity of integral the claim will follow. We introduce some notation: for $0 \le j < 2^k$, $I^k_j$ is the $j$-th slice of $I$ (starting from its left endpoint $\pi - \delta$) after dividing it into $2^k$ sub-intervals. Notice that it holds that $I^{k-1}_j = I^k_{2j} \cup I^k_{2j+1}$, for $0 \le j < 2^{k-1}$. When we say that a function is even (odd) in some interval, we mean that the function restricted to that interval is symmetric (anti-symmetric) with respect to the middle point.
    
    As a base for an inductive argument, one can see that the function $g^{(p)}_{p+1} - g^{(p)}_p$ is odd in $I^p_j$, for every $j$: this because $g^{(p)}_p$ is the side of a triangle in $I^p_{j}$, $g^{(p)}_{p+1}$ is the side of a parabolic bell and we can directly check that the difference is odd in $I^p_{j}$ (see Figure~\ref{fig:function-smoothening-parabola}).
    
    Assuming that $g^{(k)}_{p+1} - g^{(k)}_p$ is odd in $I^k_j$ for every $j$ and some $1 < k \le p$, we have that
    \begin{align*}
        \int_{L(I^k_j)}^{R(I^k_j)} g^{(k)}_{p+1}(t) - g^{(k)}_p(t) \ dt = 0 \ ,
    \end{align*}
    where $L(J), R(J)$ are, respectively, the left and right endpoints of an interval $J$. The function $g^{(k)}_p$ is a sequence of (smoothed) triangles in $I$, but this shape must be equal to $0$ at the endpoints: this because these shapes are symmetric and all equal up to sign by construction and the left endpoint of the leftmost shape must preserve continuity with the constant function outside of $I$, in both $g^{(k)}_p$ and its smoothed version $g^{(k)}_{p+1}$. Hence, at the base of the shapes the difference is $0$.
    One of $L(I^k_j), R(I^k_j)$, corresponds to one of the base points of the shape, while the other is the peak (which one depends on whether $I^k_j$ contains the first or second half of the shape, i.e., the parity of $j$). By oddness in this interval, also at the peak the difference is $0$. Considering the leftmost shape, the function $g^{(k)}_{p+1} - g^{(k)}_p$ is even in $I^{k-1}_0 = I^k_0 \cup I^k_1$. Therefore,
    \begin{align*}
        g^{(k-1)}_{p+1} - g^{(k-1)}_p & = \int_{0}^x g^{(k)}_{p+1}(t) - g^{(k)}_p(t) \ dt \\
        & = \int_{L(I^k_0)}^x g^{(k)}_{p+1}(t) - g^{(k)}_p(t) \ dt \\
        & = \int_{R(I^k_0)}^x g^{(k)}_{p+1}(t) - g^{(k)}_p(t) \ dt \ ,
    \end{align*}
    and the last expression is clearly odd in $I^{k-1}_0$, since $R(I^k_0)$ is the middle point.
    
    By a simple induction on $j$ the result can be extended to every shape, implying that $g^{(k-1)}_{p+1} - g^{(k-1)}_p$ is odd in $I^{k-1}_j$ for every $j$, and in particular, $g^{(1)}_{p+1} - g^{(1)}_p$ is odd in $I^1_0, I^1_1$, and both integrals cancel out, giving zero also on their union, which is the whole interval $I$.
\end{proof}
All we need now is to bound the Lipschitz constant for $g^{(p)}_p$: we already know that $g^{(1)}_1$ is $K_1$-Lipschitz with
\begin{align*}
    K_1 = \frac{2}{\delta^2}
\end{align*}
since its derivative $g^{(2)}_1$ is bounded by this number in absolute value. If $K_p$ is the Lipschitz constant of $g^{(p)}_p$, then
$$K_p = \sup |g^{(p+1)}_p| \ .$$
In order to obtain the same integral when we transform the rectangles into triangles, we need twice the height, thus the triangles of $g^{(p+1)}_{p+1}$ are as high as $2 K_p$. The slopes of these triangles are then $2 K_p$ over the length of a single segment which is $\frac{2 \delta}{2^{p+1}}$. Therefore, the Lipschitz constant for $g^{(p+1)}_{p+1}$ is
\begin{align*}
    K_{p+1} = \frac{2 K_p 2^{p+1}}{2 \delta} = \frac{K_p 2^{p+1}}{\delta} \ ,
\end{align*}
and this recurrence relation has unique solution
\begin{align}
    \label{eq:lipschitz-constant-recurrence-solution}
    K_p = \frac{\sqrt{2}^{p(p+1)}}{\delta^{p+1}} \ .
\end{align}
In conclusion, we can now use Theorem~\ref{thm:jackson-rate} to extend Theorem~\ref{thm:phase-extraction-jackson-rate} with a similar argument.
\begin{theorem}
    \label{thm:phase-extraction-jackson-rate-enhanced}
    Let $p > 0$ be a fixed constant. The function $\bar{g}_p : U(1) \rightarrow \R$ defined as
    \begin{align*}
        \bar{g}_p(e^{ix}) = g_p(x)
    \end{align*}
    for every $x \in \R$ can be $\epsilon$-approximated on the unit circle using a polynomial of degree
    \begin{align*}
        d = \Tilde{\bigO}\left(\frac{A_p \sqrt{2}^p}{\delta} \left(\frac{1}{\epsilon}\right)^{\frac{1}{p+1}} \right).
    \end{align*}
\end{theorem}
As a Corollary, for a constant $p$ Algorithm~\ref{alg:prop-sampling-qsp} can be done in only
\begin{align*}
    \Tilde{\bigO}\left( \frac{1}{\bar{c}^{\frac{1}{2} + \frac{1}{p}}} \sqrt[p]{\frac{1}{\epsilon}} \right)
\end{align*}
total oracle queries.

\section{Conclusions}
We consider the phase-extraction problem, and we showed that, given a unitary $U = e^{i\pi H}$ and its inverse $U^{\dag}$, we could implement a block-encoding of $\phi(H)$ for some smooth function $\phi(x)$. The word `smooth' here means existence and continuity of the derivatives: the higher the number of continuous derivatives that a function has, the faster its Fourier sum (and thus the Laurent polynomial on the eigenphases) uniformly converges to that function. We are confident this can have many more applications beyond what is shown in this work. It is also worth remarking that Jackson showed that the convergence rate of a Fourier series is almost-optimal, in the sense that no trigonometric (or, equivalently, complex exponential) series can approximate the desired function faster, up to that $\log d$ factor~\cite[p.\ 21]{jacksonTheoryApproximation1930a}. Also remember that `smoothing' a function, i.e., replacing its derivative with a continuous function, does not give faster convergence for free in general, as its derivative will become steep in the points where we smooth out discontinuities, and this translates to a high Lipschitz constant: a~clear example is given by Eq.~\ref{eq:lipschitz-constant-recurrence-solution}, but in that case, fortunately, nothing depends on the size of the input $N$, and thus does not influence the asymptotic query complexity of Algorithm~\ref{alg:prop-sampling-qsp}, although the constant factor can become large even for $p = 20$. From a theoretical point of view, this work shows that, for any $\eta > 0$, there is an algorithm with query complexity 
$$\Tilde{\bigO}\left(\frac{1}{\bar{c}^{\frac{1}{2} + \eta}} \frac{1}{\epsilon^\eta} \right)$$
solving the proportional-sampling problem. This statement seems to suggest there exists an algorithm which directly solves the problem with $\eta = 0$, and an open question would be to find such algorithm.


It is also interesting to remark that Theorems~\ref{thm:haah-construction},~\ref{thm:haah-completion} indeed allow the construction for any $\phi$, even complex-valued, provided that its absolute value is reciprocal.

One could think that, in Section~\ref{sec:prop-sampling}, instead of using the linear function in the phase-extraction subroutine, we could approximate the square root and then apply the transformation directly on $e^{i \pi c(x)}$. However, in the case of proportional sampling this would be inconvenient, as the derivative of the square root function has a discontinuity with an infinite jump around 0, and we could not choose a constant $\delta$ if we had values of the oracle that are too close to $0$.

\section*{Acknowledgements}
I would like to thank William Schober and Stefan Wolf for insightful feedback and discussions. This work was supported by the Swiss National Science Foundation (SNF), grant No. \texttt{200020\_182452}.

\appendix

\section{Haah's Construction of QSP Polynomials}
\label{apx:haah-construction}

Here we briefly review the Quantum Signal Processing technique, in its Laurent-polynomial formulation. We suggest~\cite{martynGrandUnificationQuantum2021, haahProductDecompositionPeriodic2019, chaoFindingAnglesQuantum2020} for a more comprehensive discussion on the topic. Given an arbitrary unitary $W$, consider the following matrix:
\begin{align*}
    \Tilde{W} = \begin{bmatrix}
        W & 0 \\
        0 & W^\dag
    \end{bmatrix} .
\end{align*}
This corresponds to a circuit that applies $W$ or its inverse $W^\dag$, depending on the state of a control qubit. If we feed an eigenstate $\ket{\theta}$ with associated eigenvalue $w = e^{i\theta}$, by a phase kickback the control qubit undergoes a unitary transformation of the form
\begin{align*}
    \Tilde{w} = \begin{bmatrix}
        w & 0 \\ 0 & w^{-1}
    \end{bmatrix} .
\end{align*}
This is the idea behind a technique called \emph{qubitization}~\cite{lowHamiltonianSimulationQubitization2019}, where one can apply Quantum Signal Processing polynomials simultaneously on all the eigenvalues of a unitary given as a black-box (in our case $W$), giving birth to the so-called \emph{quantum eigenvalue transform}. From now on we consider $w$ as an arbitrary unitary eigenvalue, and we want to understand which transformations can be realized. One can see that a unitary of the form
\begin{align}
    \label{eq:haah-primitive-decomposition}
    U(\Tilde{w}) = Q_n \cdot \Tilde{w} \cdot Q_n^\dag \cdots Q_1 \cdot \Tilde{w} \cdot Q_1^\dag \cdot Q_0
\end{align}
gives a $2$-by-$2$ matrix of \emph{Laurent polynomials} (or equivalently, a Laurent polynomial with $2$-by-$2$ matrices as coefficients), and the maximum degree of this polynomial is $n$. Roughly speaking, we construct a circuit where the $Q_i$'s act on the control qubit and $\Tilde{w}$ is replaced with the controlled unitary $\Tilde{W}$. We are mainly interested in two sub-algebras of the matrices above.
\begin{definition}
    The \emph{Haah algebra} $H$ is the sub-algebra of the ring of Laurent polynomials over $2$-by-$2$ complex matrices of the form
    \begin{align*}
        a(\Tilde{w}) + b(\Tilde{w}) \cdot iX + c(\Tilde{w}) \cdot iY + d(\Tilde{w}) \cdot iZ
    \end{align*}
    where $a, b, c, d$ are polynomials with real coefficients.
\end{definition}
\noindent In other words, the Haah algebra is the polynomial ring $H \simeq \R[iX, iZ, \Tilde{w}]$. An important sub-algebra of the Haah algebra, which is also the most interesting one for practical implementations is the \emph{Low algebra}.
\begin{definition}
    The \emph{Low algebra} $L$ is the sub-algebra of the ring of Laurent polynomials over $2$-by-$2$ complex matrices of the form
    \begin{align*}
        a(\Tilde{w}) + b(\Tilde{w}) \cdot iX
    \end{align*}
    where $a, b$ are Laurent polynomials with real coefficients.
\end{definition}
\noindent Analogously as before we have $L \simeq \R[iX, \Tilde{w}]$.

\begin{theorem}[Haah~\cite{haahProductDecompositionPeriodic2019}]
    \label{thm:haah-construction}
    An element $U(w) \in H$ that is unitary and has definite parity (i.e.\ $a, b, c, d$ are either all even or all odd Laurent polynomials) is always decomposable as in (\ref{eq:haah-primitive-decomposition}), in a unique way (up to global phase).

    Furthermore, if $U(w) \in L$ then all $Q_i \in L$. In particular, they will be $X$-rotations.
\end{theorem}
\noindent The Low algebra is useful exactly because it is practical, as we only have to define the phases of the $X$-rotations. The decomposition using the Low algebra thus simply reduces to the $W_z$-convention of the traditional formulation of the QSP~\cite{lowMethodologyResonantEquiangular2016,martynGrandUnificationQuantum2021}. On the other hand, while Low algebra only allows the implementation of Laurent polynomials with real coefficients, Haah algebra can even represent polynomials with complex coefficients.

In applications, we are usually interested in a single transformation of the unitary, i.e., given $\Tilde{W}$ we would like to implement $f(W)$ for some Laurent polynomial $f$ that satisfies $|f|^2 \le 1$ on the unit circle. In order to do this we need an element $F(\Tilde{w})$ of the Haah algebra containing $f$ as a particular entry, which determines the initial state and the state to post-select on the control qubit. Here we will only use the top-left entry, $f(w) = \bra{0} F(\Tilde{w}) \ket{0}$, i.e.\ we will start and post-select $\ket{0}$. The question now is whether we can find the polynomials $a, b, c, d$ such that we obtain $f$ in the top-left entry.






\iffalse
\begin{theorem}[Haah~\cite{haahProductDecompositionPeriodic2019}]
    \label{thm:haah-construction}
    Let $F(t) : U(1) \rightarrow SU(2)$ be a Laurent polynomial of degree $n$ with definite parity. Such function can be decomposed into a product of $n$ single-qubit unitaries of the form
    \begin{align*}
        E_{P_i}(t) = t P_i + t^{-1} (\id - P_i) = Q_i
        \begin{bmatrix}
            t & 0 \\
            0 & t^{-1}
        \end{bmatrix} Q_i^{\dag}.
    \end{align*}
    where $P_i$ is a rank-1 matrix, and $Q_i \ketbra{0}{0} Q_i^\dag = P_i$.
\end{theorem}
This result by Haah gives a simple and general construction implementing Quantum Signal Processing polynomials: in particular, the diagonal matrix above can be seen as the phase kickback induced by the application of a controlled unitary
\begin{align*}
    \Tilde{W} = \begin{bmatrix}
        W & 0 \\
        0 & W^\dag
    \end{bmatrix}
\end{align*}
when we apply an eigenstate associated with eigenvalue $t$. This technique is called qubitization~\cite{lowHamiltonianSimulationQubitization2019, haahProductDecompositionPeriodic2019}, and a nice exposition of this circuit decomposition can be found in~\cite{chaoFindingAnglesQuantum2020}. In applications, one generally wants to implement a polynomial $f(t) : U(1) \rightarrow \C$, given as a particular entry of the $2$-by-$2$ unitary matrix, so that we can implement $f(W)$. Hence, our next question is about how we can embed our desired polynomial $f(t)$ into a $SU(2)$-valued polynomial as defined in Theorem~\ref{thm:haah-construction}, and which conditions on $f$ are necessary and/or sufficient in order for this embedding to be possible.

An important observation made by Haah is that an element $F$ of $SU(2)$ can be rewritten as a real linear combination
\begin{align*}
    F = a \id + b i X + c i Y + d i Z
\end{align*}
with $a^2 + b^2 + c^2 + d^2 = 1$. In particular, if $F(t)$ is such that $F(t) \in SU(2)$ for $t \in U(1)$, then $a(t), b(t), c(t), d(t)$ can be seen as polynomials taking real values on the same domain. This structure was called \emph{Haah algebra}~\cite{chaoFindingAnglesQuantum2020}.

If we restrict this algebra to the so-called \emph{Low sub-algebra} (i.e.\ $c \equiv d \equiv 0$), there are already results telling us that a class of functions can be achieved.

\begin{theorem}[Completion]
    Let $a(t)$ be such that $a^2(t)$ is reciprocal, has real coefficients and satisfies $a^2(t) < 1$ for every $t \in U(1)$. There exists a polynomial $b(t)$ such that:
    \begin{align*}
        a(t) \cdot \id + b(t) \cdot i X \in SU(2)
    \end{align*}
    for every $t \in U(1)$.
\end{theorem}

\begin{proof}
    In other words we need to satisfy the condition $a(t) \cdot a^*(t) + b(t) \cdot b^*(t) \equiv 1$, or
    \begin{align}
        \label{eq:low-completion-su-condition}
        b(t) \cdot b^*(t) = 1 - a(t) \cdot a(1/t) = 1 - a^2(t)
    \end{align}
    The first equality comes from the fact that $a$ has real coefficients, while the second one is guaranteed by reciprocity. No roots can lie on the unit circle by hypothesis, therefore all of its roots must be either inside or outside of the unit circle. Moreover, by reciprocity, if $a^2(t) = 1$ implies $a^2(1/t) = 1$. Let $\mathcal{D}$ be the multiset of the roots of $1 - a^2$ within the unit circle. Therefore, our polynomial can be rewritten as:
    \begin{align}
        \label{eq:low-completion-decomposition}
        1 - a^2(t) = \alpha \prod_{r \in \mathcal{D}} (t - r) (1/t - r)
    \end{align}
    where $\alpha$ is some proportionality constant. Reality of the coefficients also imply that the roots $r$ in Eq. (\ref{eq:low-completion-decomposition}) either are real, or come in complex conjugate pairs. Moreover, one can see that
    \begin{align*}
        1 - a^2(1) = \alpha \prod_{r \in \mathcal{D}} (1 - r)^2
    \end{align*}
    and, since $\mathcal{D}$ is closed under complex conjugation, we can conclude that the product is real and positive, implying that $\alpha$ is real and positive as well. If we take, for example
    \begin{align*}
        b(t) = \sqrt{\alpha} \prod_{r \in \mathcal{D}} (t - r)
    \end{align*}
    one can see that this has real coefficients as well, and $b(t) \cdot b^*(t) = b(t) \cdot b(1/t)$ satisfies Eq. (\ref{eq:low-completion-su-condition}).
\end{proof}
Let us remark an important fact about the condition on the reciprocity of $a^2(t)$: $a^2$ being reciprocal with real coefficients means that either $a(t)$ is also reciprocal (with real coefficients), or anti-reciprocal (with imaginary coefficients). The first case comes from a cosine series, while the second case is yielded by a sine series. Similar (but not equivalent) to the original QSP constraints, the function $\theta \mapsto a(e^{i \theta})$ has to be of definite parity for this completion to succeed. On the other hand, it is worth highlighting that the constraint $a^2(t) < 1$ is never a problem in applications, as we can always multiply all the coefficients of the polynomial by a constant $c < 1$, incurring only a multiplicative constant on the failure probability, which can be easily re-amplified.

We now try to give a more general completion procedure, including polynomials with complex coefficients.
\fi

\begin{theorem}[Haah's completion~\cite{haahProductDecompositionPeriodic2019}, restated]
    \label{thm:haah-completion}
    Given real-valued polynomials $a(w), d(w) : U(1) \rightarrow \R$, let 
    $$f(w) = a(w) + i d(w)$$
    be a function such that the polynomial $|f|^2 = a^2 + d^2$ is reciprocal (i.e.\ $|f(w)|^2 = |f(w^{-1})|^2$), has real coefficients and satisfies $|f(w)|^2 < 1$ on the unit circle.
    There exist polynomials $b(w), c(w)$ such that
    \begin{align*}
        a(\Tilde{w}) + b(\Tilde{w}) \cdot iX + c(\Tilde{w}) \cdot iY + d(\Tilde{w}) \cdot iZ \in SU(2)
    \end{align*}
    for every $t \in U(1)$.
\end{theorem}
Note that, with the given construction, $f(w)$ is present on the top-left corner of the matrix, while the bottom-right entry contains $(f(w^{-1}))^*$.
\begin{proof}
    On the unit circle, this translates to the condition
    \begin{align*}
        a^2(w) + b^2(w) + c^2(w) + d^2(w) \equiv 1.
    \end{align*}
    The polynomial $p(w) = 1 - a^2(w) - d^2(w)$ has degree $n' \le 2n$ (in Laurent polynomials leading terms could cancel out). Moreover, by the conditions on $|f|^2$, it has real coefficients, all its roots are outside the unit circle, and by reciprocity for any root $r$ also $1/r$ is a root. Therefore, using $\mathcal{D}$ to denote the multiset of roots within the unit circle, the expression becomes
    \begin{align}
        \label{eq:haah-construction-roots}
        p(t) = \alpha \prod_{r \in \mathcal{D}} (w - r) (1/w - r)
    \end{align}
    for some proportionality constant $\alpha$. Keep in mind that $|\mathcal{D}| = n'$. Let us now define
    \begin{align*}
        e(w) := w^{-\lfloor n'/2 \rfloor} \prod_{r \in \mathcal{D}} (w - r) \ .
    \end{align*}
    In this way, $\alpha \cdot e(w) \cdot e(1/w) = p(w)$. Note that the factor in front of the product is used to ‘center' the exponents of $e(w)$, so that its degree is $\lceil n'/2 \rceil \le n$. This however, does not affect the equality we just stated.
    Plugging $t = 1$ in Eq.\ (\ref{eq:haah-construction-roots}) gives real and positive expressions on both sides (complex roots cannot not give negative contributions, since they all come in complex conjugate pairs, by reality of the coefficients). Hence we conclude that $\alpha$ is positive.
    \begin{align*}
        p(t) & = \alpha \cdot e(w) \cdot e(1/w) \\
        & = \left( \frac{e(w) + e(1/w)}{2} \sqrt{\alpha} \right)^2 + \left( \frac{e(w) - e(1/w)}{2i} \sqrt{\alpha} \right)^2 \ ,
    \end{align*}
    and we can choose the two expressions in the tuples as $b(w), c(w)$, for example. They will have both degree $\le \lceil n'/2 \rceil \le n$.
\end{proof}
It is interesting to remark that the reciprocity of $|f(w)|^2$ implies the reciprocity of $a^2(w) + d^2(w)$. Under this assumption, we have that $a^2$ is reciprocal if and only if $d^2$ is reciprocal. If the square of a polynomial is reciprocal with real coefficients, then the original polynomial is either reciprocal itself with real coefficients (cosine transform), or anti-reciprocal with imaginary coefficients (sine transform). On the other hand, if $a^2$ is not reciprocal, then in order to maintain reciprocity of the sum we need $a^2(w) = d^2(1/w)$, or simply $a(w) = \pm d(1/w)$. In this last case, $a(w)$ and $d(w)$ do not need to have real nor imaginary coefficients as long as they are real-valued for any $w \in U(1)$. Also as pointed out by Haah, (anti-)reciprocity is not a severe restriction, since any function can be decomposed into a sum of reciprocal and anti-reciprocal components, which can be implemented separately and then summed using a circuit like in Figure~\ref{fig:block-encoding-sum-circuit}, incurring on at most a constant factor on the failure probability.
\section{Semi-oblivious amplitude amplification}
\label{apx:amplitude-amplification}

\noindent In Section~\ref{sec:phase-extraction-problem} we have the following unitary:
\begin{align*}
    S \ket{00} \ket{\psi}^{\otimes n} = \ket{00} H_c \ket{\psi} + \ket{\Phi}
\end{align*}
\smallskip

\noindent where $\braket{00}{\Phi} = 0$, and $H_c \ket{x} = \frac{1}{2} \sqrt{\Tilde{c}(x)} \ket{x}$. Unfortunately, we cannot apply an oblivious amplitude amplification~\cite{berryExponentialImprovementPrecision2014a} procedure, not even in its robust version~\cite{berrySimulatingHamiltonianDynamics2015}. This because we would need $H_c$ to be (almost-)unitary, but it can even have zero eigenvalues. On the other hand, however, we only need to transform one particular state, i.e.\ $\ket{+}^{\otimes n}$. We use the Quantum Singular Value Transform (QSVT)~\cite{gilyenQuantumSingularValue2019a, gilyenQuantumSingularValue2019c} in a similar way it was used in~\cite{martynGrandUnificationQuantum2021} for Grover's search. We introduce some notation: let $\ket{\Psi} = \ket{00} \ket{+}^{\otimes n}$ denote our initial state, while $\ket{w} = \ket{00} \frac{H_c \ket{+}^{\otimes n}}{||H_c \ket{+}^{\otimes n}||}$ is our (normalized) target state. If $\Pi' = \ketbra{00}{00} \otimes \id, \Pi = \ketbra{\Psi}{\Psi}$. Then
\begin{align*}
    \Pi' S \Pi \ket{\Psi} = \ket{00} H_c \ket{+}^{\otimes n} = \sigma \ket{w}
\end{align*}
where $\sigma = ||H_c \ket{+}^{\otimes n}|| = \Theta(\sqrt{\bar{c}})$. Since any state orthogonal to $\ket{\Psi}$ is canceled out by $\Pi$, then it means that a singular value decomposition is
\begin{align*}
    \Pi' S \Pi = \sigma \ketbra{w}{\Psi}.
\end{align*}
Notice that, in our case, reflections and rotations of $\Pi', \Pi$ can be easily implemented and we can use the polynomial that approximates $\erf(k[x - c])$ (see, e.g.\,~\cite[Section III]{martynGrandUnificationQuantum2021}), having constant success probability with only
$$\bigO\left(\frac{1}{\sigma}\right) = \bigO\left(\frac{1}{\sqrt{\bar{c}}}\right)$$
calls to $S$.


\bibliography{refs}

\end{document}
