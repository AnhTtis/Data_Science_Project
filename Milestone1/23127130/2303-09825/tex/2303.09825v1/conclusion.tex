\section{Conclusion}
\label{sec:conclusion}
% This paper proposed an automatic checkerboard-based approach to calibrate extrinsics between a LiDAR and a frame/event camera. This method uses a checkerboard as the calibration target.
% It contains several desirable features, including automatic checkerboard detection and tracking from point clouds, image reconstruction from event streams, and extrinsics estimation with a globally optimal solver.
% Extensive experiments on simulated sensors and real-world devices were conducted for evaluation.
% Our approach yielded accuracy in centimeters in translation and decidegrees in rotation, outperforming a SOTA checkerboard-based method.

% In future work, the time undistortion effect of cameras and LiDARs should be considered.
% It would also be beneficial to advance the LCE configuration by taking advantage of each sensor to address several robotics challenges, such as ego-motion estimation or object detection in complex scenarios.

This paper presents a novel automatic checkerboard-based approach to calibrate extrinsics between a LiDAR and a frame/event camera, which offers several desirable features such as automatic checkerboard detection and tracking, image reconstruction from event streams, and globally optimal extrinsics estimation. The proposed method has been extensively evaluated on both simulated sensors and real-world devices, demonstrating its superior performance in terms of accuracy in translation and rotation compared to a SOTA checkerboard-based method.
Moving forward, future research could explore the effect of time undistortion on camera and LiDAR sensors, as well as further advancing the LCE configuration to address various robotics challenges, such as ego-motion estimation or object detection in complex scenarios.
% Overall, this research provides a valuable contribution to the field of robotics and has the potential to enhance the capabilities of LiDAR-camera fusion systems.