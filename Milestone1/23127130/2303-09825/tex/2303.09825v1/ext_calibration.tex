\section{Extrinsic Calibration with Multi-Frame Measurements}
\label{sec:ext_calibration}

\begin{figure}[t]
  \centering
  \subfigure[LiDAR-Frame camera sensor]
  {\label{fig:exp_real_device_1}\centering\includegraphics[width=0.21\textwidth]{figure/experiment/real_device_1_compressed.png}}
  \subfigure[LiDAR-Event camera sensor]
  {\label{fig:exp_real_device_2}\centering\includegraphics[width=0.219\textwidth]{figure/experiment/real_device_2_compressed.png}}
  \caption{The RLFS and RLES devices for calibration tests.}
  \label{fig:exp_real_device}
\end{figure}

\begin{figure}[t]
  \centering
  \subfigure[]
  {\label{fig:exp_simu_camera}\centering\includegraphics[width=0.210\textwidth]{figure/experiment/simu/simu_camera}}
  \
  \subfigure[]
  {\label{fig:exp_simu_lidar}\centering\includegraphics[width=0.21\textwidth]{figure/experiment/simu/simu_lidar}}
  \caption{Simulated sensor data: (a) an image and (b) a point cloud.}
  \label{fig:exp_simu_sensor}
\end{figure}


%%%%%%%%%%%%%%%%%%%%%%%%%%%%%%%%%%%%%%%%%%%%%%%%%%%%%%
Section \ref{sec:measure_process} has explained how checkerboard's normal vectors: $\bm{n}^{c}$, $\bm{n}^{e}$, planar points: $\mathcal{P}^{l}_{pl}$, and edge points: $\mathcal{P}^{l}_{edge}$, $\mathcal{P}^{c}_{edge}$, $\mathcal{P}^{e}_{edge}$ are extracted from each frame of LiDARs, frame cameras, and event cameras.
This section introduces how to use these features to constrain extrinsics.
We take the LiDAR-frame camera extrinsic calibration as an example without loss of generality.
At first, we can easily construct a set of point-plane pairs.
But the point-edge association is ambiguous without the initial extrinsics.
In other words, we cannot determine which boundaries a point belongs to since the board's shape is symmetric.
Thus, we follow the \textit{initialization-refinement} philosophy \cite{jiao2021robust} to estimate extrinsics in a coarse-fine manner, as summarized in Algorithm \ref{alg:ext_refinement}.

The first phase initializes the extrinsics $\bm{T}^{c}_{l,ini}$
(i.e., transformation from $()^{c}$ to $()^{l}$).
We formulate the point-to-plane optimization objective \eqref{equ:qpep-ptpl}
by utilizing planar features: $\bm{n}^{c}$ and $\mathcal{P}^{l}_{pl}$
from several frames of measurements.
The \textit{QPEP-PtPL} algorithm is applied to solve the problem.

The second phase further refines the initial solution by jointly utilizing the planar and edge points that offer stronger geometric constraints.
This phase is executed with multiple computations.
Specifically, three steps are done at line $6$-$7$:
\begin{enumerate}
  \item With $\bm{T}^{c}_{l,ini}$, we first transform $\mathcal{P}^{l}_{edge}$ into the camera frame. For each point from transformed point set, we find the nearest point in $\mathcal{P}^{c}_{edge}$ as its corresponding edge point, and then construct the point-to-line objective.
  \item From \eqref{equ:qpep-ptpl}, we propagate the covariance of each point-to-plane distance as
        $\sigma^{2}=(\bm{R}^{c}_{l}\bm{p}_{i}^{l}+\bm{t}^{c}_{l})^{\top}\bm{\Sigma}_{n^{c}}(\bm{R}^{c}_{l}\bm{p}_{i}^{l}+\bm{t}^{c}_{l})$. We filter out $10\%$ error terms if their variances are large.
  \item The overall optimization objective for the refinement is defined as the sum of all weighted point-to-plane and point-to-line residuals as
        \begin{equation}
          \label{equ:obj-refinement}
          \mathcal{L}(\bm{R}^{c}_{l},\bm{t}^{c}_{l})
          =
          w_{ptpl}\mathcal{L}_{ptpl}(\bm{R}^{c}_{l},\bm{t}^{c}_{l})
          +
          w_{ptl}\mathcal{L}_{ptl}(\bm{R}^{c}_{l},\bm{t}^{c}_{l}),
        \end{equation}
        where $w_{ptpl}$ and $w_{ptl}$ are tunned parameters which balance these two constraints.
        Similar to initialization, we can also use the \textit{QPEP-PtPL} solver.
\end{enumerate}

After getting a set of candidate extrinsics, we select one with the minimum geometric error as our resulting extrinsics.
Two examples are shown in Fig. \ref{fig:exp_cloud_align}, where LiDAR's planar points and edge points are aligned with the board plane of images with the estimated extrinsiscs.
