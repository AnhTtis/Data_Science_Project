\subsubsection{Extrinsic Calibration of RLES}
\label{sec:exp_rles}

We further verify the complete LCE-Calib method with RLES.
Three groups of calibration data are collected in an indoor room under different lighting conditions, which are called \textit{RLES01}, \textit{RLES02}, and \textit{RLES03} respectively.
In contrast to the last section, we additionally compare calibration results of using frame images (\textit{X-LF}) and reconstructed images (\textit{X-LE}) from event streams, respectively.
Fig. \ref{fig:exp_rles_error} plots calibration errors of the proposed method and Table \ref{tab:rles_calibration_results} reports quantitative results.
% \textit{RLFS01} and \textit{RLFS02}/\textit{RLFS03} are collected on different days.
% We provide two ``fake'' ground-truth extrinsics for them.

The calibration errors consistently decrease with more frames.
Though, compared with Table \ref{tab:rlfs_calibration_results}, the calibration errors are larger.
We consider that this is mainly caused by the low resolution of the event camera, affecting the inner pattern detection accuracy.
Cameras' low resolution also makes \textit{MGE} values not consistent with \textit{EGT} in some cases since boards' positions are not detected very precisely.
But we can still use this metric for selecting good extrinsics.
% In most cases, the proposed method outperforms the baseline algorithm.
% The proposed method has comparable
From Fig. \ref{fig:exp_rles_error}, we observe that extrinsics converge if around $\bm{15}$ frames are given. More data can further enforce accuracy.
For event cameras, inner patterns and boundaries of the board are reconstructed clearly. Thus, the calibration accuracy does not have much degradation.
A sufficient number of frames are used in calibration, whether using frame images or images reconstructed from event streams, and the extrinsics are successfully recovered.
Fig. \ref{fig:exp_rles_calibration_image} shows the back-projection results of LiDAR points using extrinsics in Table \ref{tab:rles_calibration_results}.
The checkerboard detection results from LiDAR points are also shown.
In \textit{RLES03-LF}, extrinsics from our method can help LiDAR points better align on the board's boundaries than \textit{Zhou-MATLAB}.
