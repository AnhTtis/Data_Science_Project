% biography section
% 
% If you have an EPS/PDF photo (graphicx package needed) extra braces are
% needed around the contents of the optional argument to biography to prevent
% the LaTeX parser from getting confused when it sees the complicated
% \includegraphics command within an optional argument. (You could create
% your own custom macro containing the \includegraphics command to make things
% simpler here.)
%\begin{IEEEbiography}[{\includegraphics[width=1in,height=1.25in,clip,keepaspectratio]{mshell}}]{Michael Shell}
% or if you just want to reserve a space for a photo:

\begin{IEEEbiography}[{\includegraphics[width=0.9in,height=1.25in,clip,keepaspectratio]{figure/bio/jjh.png}}]{Jianhao Jiao}
    received the B.Eng. degree in instrument science from Zhejiang University, Hangzhou, China, in 2017, and the Ph.D. from the Department of Electronic and Computer Engineering, the Hong Kong University of Science and Technology, Hong Kong, China, in 2021, supervised by Prof. Ming Liu.
    He is now a research associate in the same university.
    His research interests include state estimation, SLAM, dense mapping, sensor fusion, and computer vision.
\end{IEEEbiography}
\vspace{-0.7cm}

\begin{IEEEbiography}[{\includegraphics[width=0.9in,height=1.15in,clip,keepaspectratio]{figure/bio/cfy_color}}]{Feiyi Chen}
    received the B.Eng. degree from Huazhong University of Science and Technology, Wuhan, China, in 2019 and Master degree from Hong Kong University of Science and Technoloay in 2022, under the supervision of Prof.Ming Liu. His research intersests include multi-sensor fusion, calibration and computer vision.
\end{IEEEbiography}
\vspace{-0.7cm}

\begin{IEEEbiography}[{\includegraphics[width=0.9in,height=1.25in,clip,keepaspectratio]{figure/bio/whx}}]{Hexiang Wei}
    received B.Eng. degree in Mechanical Engineering from Northeastern University, Shenyang,  China, in 2020. He is currently a Ph.D. student in the Intelligent and Autonomous Driving Center at Hong Kong University of Science and Technology led by Prof. Ming Liu. His research interests include SLAM, sensor fusion and robotics.
\end{IEEEbiography}
\vspace{-0.7cm}

\begin{IEEEbiography}[{\includegraphics[width=0.9in,height=1.25in,clip,keepaspectratio]{figure/bio/wujin_color}}]{Jin Wu}
    received the B.S. Degree from University of Electronic Science and Technology of China, Chengdu, China.
    He has been a research assistant since 2018, and now a Ph.D. student in Department of Electronic and Computer Engineering, Hong Kong University of Science and Technology. He was with Tencent Robotics X, Shenzhen, China from 2019 to 2020. During 2012 to
    2018, he was in the UAV industry and has brought up two companies.
\end{IEEEbiography}
\vspace{-0.7cm}

\begin{IEEEbiography}[{\includegraphics[width=0.9in,height=1.25in,clip,keepaspectratio]{figure/bio/ming_color.png}}]{Ming Liu}
    received the Ph.D. degree from the Department of Mechanical and Process Engineering, ETH Zurich, Switzerland, in 2013, supervised by Prof. Roland Siegwart.
    He is currently with the Robotics and Autonomous Systsms, The Hong Kong University of Science and Technology (Guangzhou), the director of Intelligent Autonomous Driving Center, as an Associate Professor.
    He has published many popular papers in top journals including IEEE Transactions on Robotics and International Journal of Robotics Research. Dr. Liu is currently an Associate Editor for IEEE Robotics and Automation Letters, IET Cyber-Systems and Robotics, International Journal of Robotics and Automation.
    His research interests include dynamic environment modeling, deep-learning for robotics, 3D mapping, machine learning, and visual control.
\end{IEEEbiography}

