\section{Problem Statement}
\label{sec:problem_statement}

The extrinsic calibration problem is divided into three subproblems:
PnP, PtPL, and Point-to-Line (or Edge) (PtL) registration problems.
All of them are generally formulated as quadratic pose estimation problems (QPEPs) \cite{wu2022quadratic}.
Before delving into the details of LCE-Calib, we first introduce basic concepts here.
Section \ref{sec:ps_notation} presents notations.
Section \ref{sec:ps_qpep} introduces the definition of QPEP and general idea to solve QPEPs based on our previous work.

\subsection{Notations and Definitions}
\label{sec:ps_notation}
We consider a sensor suite that consists of a LiDAR, a frame camera, and an event camera.
Frames of these sensors are defined as $()^{l}$, $()^{c}$, and $()^{e}$ respectively.
We also define $()^{b}$ as the frame of the checkerboard, where the origin stays at the board's center, and the $z$-axis is perpendicular to the board's plane.
We use $\bm{t}\in\mathbb{R}^{3}$ and $\bm{R}\in SO(3)$ to represent the 3D translation and rotation. Especially, the rotation matrix is from the Lie group $SO(3)$ where $\bm{R}^{\top}\bm{R}=\bm{I}$, $\det\bm{R}=1$.
For any real 3D vector $\bm{\phi}\in \mathbb{R}^{3}$, its skew-symmetric matrix is
\begin{equation}
  \bm{\phi}^{\wedge}
  =
  \begin{bmatrix}
    0         & -\phi_{3} & \phi_{2}  \\
    \phi_{3}  & 0         & -\phi_{1} \\
    -\phi_{2} & \phi_{1}  & 0         \\
  \end{bmatrix}
  \in
  \mathfrak{so}(3),
\end{equation}
which is an element from the Lie algebra $\mathfrak{so}(3)$.
We use the exponential operator to associate an element from $SO(3)$ with an element from $\mathfrak{so}(3)$:
$\bm{R}=\exp(\bm{\phi}^{\wedge})$.
From the derivation in \cite{barfoot2014associating}, we can model the Gaussian uncertainty of rotation using the right perturbation as $\bm{R}=\bar{\bm{R}}\exp(\delta\bm{\phi}^{\wedge})\approx\bar{\bm{R}}(\bm{I}+\delta\bm{\phi}^{\wedge})$,
where $\bar{\bm{R}}$ is the noise-free rotation and $\delta\bm{\phi}\sim\mathcal{N}(\bm{0},\bm{\Sigma}_{\delta\bm{\phi}})$.
If $\bm{R}$ and $\bm{t}$ are considered simultaneously, we also use the 3D transformation matrix $\bm{T}\in SE(3)$ from the Lie group:
$\bm{T}=SE_{3}(\bm{R},\bm{t})=
  \begin{bmatrix}
    \bm{R} & \bm{t} \\
    \bm{0} & 1      \\
  \end{bmatrix}
$ to represent poses.

\begin{figure}[t]
  \centering
  \includegraphics[width=0.47\textwidth]{figure/methodology/geometric_constraint-crop.pdf}
  \caption{Visualization of geometric constraints of the PnP, PtPL, and PtL problems. The red dot is the reference point, while the green dot is the corresponding corner or edge feature.}
  \label{fig:qpep-geometric-constraint}
\end{figure}

\subsection{Quadratic Pose Estimation Problem}
\label{sec:ps_qpep}

As first proposed in our previous work \cite{wu2022quadratic}, QPEPs define a series of $SE(3)$-differentiable optimization problems as
\begin{equation}
  \underset{\bm{R}\in SO(3),\ \bm{t}\in\mathbb{R}^{3}}{\arg\min} \
  \mathcal{L}(\bm{R},\bm{t}),
\end{equation}
where $\mathcal{L}(\cdot)$ is represented as a \textit{quadratic objective} in terms of quadratic products of elements in $\bm{R}$ and $\bm{t}$. The PnP, PtPL, and PtL problems are three QPEPs since their objective are quadratic.
Fig. \ref{fig:qpep-geometric-constraint} visualizes the geometry of these problems.
Regarding the PnP problem, our goal is to estimate the transformation between the world frame and the camera frame given a set of 3D-2D point correspondences $<\bm{p}_{i}, \bm{u}_{i}>$:
\begin{equation}
  \label{equ:qpep-pnp}
  \mathcal{L}_{pnp}(\bm{R},\bm{t})
  =
  \sum||\bm{u}_{i} - \pi(\bm{R}\bm{p}_{i}+\bm{t})||^{2},
\end{equation}
where $\pi(\cdot)$ projects a 3D point onto the image plane.
The PtPL problem obtains the optimal transformation between two 3D point clouds by minimizing the point-to-plane residuals:
\begin{equation}
  \label{equ:qpep-ptpl}
  \mathcal{L}_{ptpl}(\bm{R},\bm{t})
  =
  \sum[\bm{n}_{i}^{\top}(\bm{g}_{i} - \bm{R}\bm{p}_{i} - \bm{t})]^{2},
\end{equation}
where $\bm{g}_{i}$ and $\bm{p}_{i}$ are $i$-th corresponding points from two point clouds and $\bm{n}^{i}$ is the $i$-th normal vector of a plane on which $\bm{g}_{i}$ stay.
For the PtL problem \cite{jiao2021robust}, we define the edge residual as a planar residual using \eqref{equ:qpep-ptpl},
where $\bm{n}_{i}$ coincides with the projection direction from the line to $\bm{p}_{i}$ (see Fig. \ref{fig:qpep-geometric-constraint}).
This formulation allows us to address both the PtPL and PtL problems with one solution.
Our previous work has proposed a unified quaternion-based globally optimal solution to solve general QPEPs \cite{wu2022quadratic}.
% \begin{equation}
%   \label{equ:qpep-ptl}
%   \mathcal{L}_{ptl}(\bm{R},\bm{t})
%   =
%   \sum\sum_{j=1,2}[\bm{n}_{i,j}^{\top}(\bm{g}_{i} - \bm{R}\bm{p}_{i} - \bm{t})]^{2},   
% \end{equation}
% \begin{equation}
%   \label{equ:qpep-ptl}
%   \mathcal{L}_{ptl}(\bm{R},\bm{t})
%   =
%   \sum\sum_{j=1,2}[\bm{n}_{i,j}^{\top}(\bm{g}_{i} - \bm{R}\bm{p}_{i} - \bm{t})]^{2},   
% \end{equation}
% where $\bm{n}_{i,1}$ and $\bm{n}_{i,2}$ are normal vectors of two planes. 
% $\bm{n}_{i,1}$ coincides with the projection direction from the line to $\bm{p}_{i}$ and 
% $\bm{n}_{i,2}$ is perpendicular to both $\bm{n}_{i,1}$ and the line.
The mathematical derivation, solution strategies, and analysis of uncertainty are already detailed in this paper \cite{wu2022quadratic}.
In subsequent sections, we introduce how to extract features from raw sensor measurements and apply the \textit{QPEP-PnP} and \textit{QPEP-PtPL} algorithms to address the extrinsic calibration.

