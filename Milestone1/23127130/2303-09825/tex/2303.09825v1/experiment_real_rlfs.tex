\subsection{Calibration Results on Real-World Data}
\label{sec:exp_calib_realworld}

This section tests the proposed method with two real-world sensor suites.
Similarly, we move a checkerboard before sensors slowly to collect several groups of calibration data. Experimental results are shown in the following sections.

%%%%%%%%%%%%%%%%%%%%%%%%%%%%%%%%%%%%%%%%%%
\subsubsection{Extrinsic Calibration of RLFS}
\label{sec:exp_rlfs}
We first verify the calibration method with the LiDAR-frame camera setup.
We collect three groups of data called \textit{RLFS01}, \textit{RLFS02}, and \textit{RLFS03} to test our approach.
\textit{RLFS01} and \textit{RLFS02} were collected in an indoor corridor. Besides the checkerboard, other objects such as grounds, walls, and square pillars appear as planar surfaces.
\textit{RLF03} were collected in an outdoor garden. Several noisy objects such as buildings, poles, and lawns exist, affecting the checkerboard detection from point clouds.

We firstly employ the MATLAB camera calibration toolbox\footnote{\url{https://www.mathworks.com/help/vision/camera-calibration.html}} to calibrate the camera's intrinsics with these collected data and then calibrate extrinsics.
Our method successfully recovers the extrinsics.
Fig. \ref{fig:exp_rlfs_error} plots errors with increasing frames of measurements and Table \ref{tab:rlfs_calibration_results} gives quantitative sample results.
$N_{\text{detect}}/N$ shows the portion of successfully detecting the checkerboard from images or point clouds.
\textit{Zhou-MATLAB} only uses a portion of measurements since it fails to detect checkerboards from several LiDAR frames and correctly detect features, thus obtaining inaccurate extrinsics.
These results are consistent with those in Section \ref{sec:exp_calib_simulate}, where our method outperforms the baseline and reaches up to $<0.3deg$ and $<1.5cm$ calibration error as well as $<0.9cm$ \textit{MGE} in both indoor and outdoor environments.

Fig. \ref{fig:exp_rlfs_calibration_image} shows the back-projection results using extrinsics in Table \ref{tab:rlfs_calibration_results}.
Circular points indicate detected edge points.
Our method can provide better results since the projections of edge points nearly stay on the the checkerboard's boundary.
We additionally show the calibration error along with the number of frames used in
calibration. The error gradually decreases since more constraints help to enforce the accuracy. One example on \textit{RLFS01} is shown in Fig. \ref{fig:calibration_error_frame_rlfs}.
