
\begin{table}[t]
	\centering
	\caption{Calibration results on \textit{RLFS} with our proposed method by disabling some modules. $\downarrow$/$\uparrow$ indicates that the lower/higher the value, the better the score. The first two results are marked as bold.}
	\renewcommand\arraystretch{1.0}
	\renewcommand\tabcolsep{6.0pt}
	\scriptsize
	\begin{tabular}{ccccc}
		\toprule
		\multirow{1}{*}{Dataset}                              &
		\multirow{1}{*}{Method}                               &
		\multirow{1}{*}{$EGT_{\mathbf{R}}\ [deg,\downarrow]$} &
		\multirow{1}{*}{$EGT_{\mathbf{t}}\ [m,\downarrow]$}   &
		\multirow{1}{*}{\textit{MGE}$\ [m,\downarrow]$}
		\\
		\midrule[0.03cm]

		\multirow{7}{*}{\textit{RLFS01}}

		                                                      & \textit{WO. PP}   & $0.517$      & $0.023$      & $0.015$      \\
		                                                      & \textit{WO. UM}   & $\bm{0.119}$ & $0.010$      & $\bm{0.008}$ \\
		                                                      & \textit{WO. PtPL} & $0.292$      & $0.011$      & $\bm{0.008}$ \\
		                                                      & \textit{WO. PtL}  & $0.556$      & $0.018$      & $\bm{0.008}$ \\
		                                                      & $I=1$             & $0.195$      & $\bm{0.007}$ & $0.009$      \\
		                                                      & $I=5$             & $0.262$      & $0.012$      & $\bm{0.008}$ \\
		                                                      & \textit{Ours}     & $\bm{0.130}$ & $\bm{0.009}$ & $0.009$      \\
		\midrule

		\multirow{7}{*}{\textit{RLFS02}}

		                                                      & \textit{WO. PP}   & $0.369$      & $0.021$      & $0.015$      \\
		                                                      & \textit{WO. UM}   & $0.338$      & $0.015$      & $0.008$      \\
		                                                      & \textit{WO. PtPL} & $0.314$      & $\bm{0.012}$ & $0.008$      \\
		                                                      & \textit{WO. PtL}  & $0.653$      & $0.021$      & $\bm{0.007}$ \\
		                                                      & $I=1$             & $\bm{0.277}$ & $\bm{0.014}$ & $0.008$      \\
		                                                      & $I=5$             & $0.342$      & $0.015$      & $0.008$      \\
		                                                      & \textit{Ours}     & $\bm{0.293}$ & $\bm{0.014}$ & $\bm{0.007}$ \\
		\midrule

		\multirow{7}{*}{\textit{RLFS03}}

		                                                      & \textit{WO. PP}   & $0.378$      & $0.018$      & $0.016$      \\
		                                                      & \textit{WO. UM}   & $\bm{0.207}$ & $\bm{0.007}$ & $\bm{0.008}$ \\
		                                                      & \textit{WO. PtPL} & $0.338$      & $0.016$      & $0.009$      \\
		                                                      & \textit{WO. PtL}  & $0.289$      & $0.019$      & $\bm{0.007}$ \\
		                                                      & $I=1$             & $0.475$      & $0.024$      & $\bm{0.008}$ \\
		                                                      & $I=5$             & $0.324$      & $0.017$      & $\bm{0.008}$ \\
		                                                      & \textit{Ours}     & $\bm{0.136}$ & $\bm{0.008}$ & $\bm{0.008}$ \\
		\bottomrule[0.03cm]
		\multicolumn{5}{l}{WO.: without. PP: point projection. UM: uncertainty modelling of normal vectors.}                   \\
		% \multicolumn{5}{l}{}                                                       \\
		\multicolumn{5}{l}{PtPL: point-to-plane constriants. PtL: point-to-line constriants.}                                  \\
		\multicolumn{5}{l}{$I$: the computation number in Algorithm \ref{alg:ext_refinement}.}                                 \\
	\end{tabular}
	\label{tab:ablation_study}
\end{table}

\subsection{Ablation Study}
\label{sec:exp_ablation_study}

Our method proposes several steps to handle the uncertainty of sensor data and calibration results, such as the point projection (Section \ref{sec:methodology_fsl_feature_extraction}), uncertainty modelling of normal vectors (Section \ref{sec:methodology_fsc}), and calibration with multiple computations (Section \ref{sec:ext_calibration}).
To show the impact of these individual modules, we conduct the following ablation study.
We disable one of these modules and repeatedly test the proposed method with \textit{RLFS}.
Table \ref{tab:ablation_study} reports the calibration accuracy.

Disabling the point projection \textit{(WO. PP)} or PtL constraints \textit{(WO. PtL)} significantly degrades the calibration performance.
This is because the point projection reduces noise of LiDAR points with the fitted planar parameters and the checkerboard's edges (the source of PtL constraints) offer strong constraints to extrinsics.
We also observe that setting $I=1$ or $I=5$ may sometimes induce unreliable results (see \textit{RLFS03}).
To minimize the error bound of the results, the value of $I$ cannot be small in practice.


