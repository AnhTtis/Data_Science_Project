% =========================================================================
% SciPost LaTeX template
% Version 2021-08
%
% Submissions to SciPost Journals should make use of this template.
%
% INSTRUCTIONS: simply look for the `TODO:' tokens and adapt your file.
%
% You can also make use of our empty "skeleton" templates for each Journal,
% e.g. SciPostPhys_skeleton.tex
% =========================================================================


% TODO: uncomment ONE of the class declarations below

% Class declaration format: \documentclass[submission, {DOI label of journal}]{SciPost}
% where the DOI label of the journal should be one of:
% Phys          (for SciPost Physics)
% PhysCore      (for SciPost Physics Core)
% PhysLectNotes (for SciPost Physics Lecture Notes)
% PhysProc      (for SciPost Physics Proceedings -> !! Please use the conference-specific template which you will find on our website !!
% PhysCodeb     (for SciPost Physics Codebases)
% Astro         (for SciPost Astronomy)
% Bio           (for SciPost Biology)
% Chem          (for SciPost Chemistry)
% CompSci       (for SciPost Computer Science)
% Math          (for SciPost Mathematics)


%% PHYSICS:
% If you are submitting a paper to SciPost Physics: uncomment next line
\documentclass[submission, Phys]{SciPost}
% If you are submitting a paper to SciPost Physics Core: uncomment next line
%\documentclass[submission, PhysCore]{SciPost}
% If you are submitting a paper to SciPost Physics Lecture Notes: uncomment next line
%\documentclass[submission, PhysLectNotes]{SciPost}
% If you are submitting a paper to SciPost Physics Proceedings: uncomment next line
%\documentclass[submission, PhysProc]{SciPost}
% If you are submitting a paper to SciPost Physics Codebases: uncomment next line
%\documentclass[submission, PhysCodeb]{SciPost}

%% ASTRONOMY:
% If you are submitting a paper to SciPost Astronomy: uncomment next line
% \documentclass[submission, Astro]{SciPost}

%% BIOLOGY:
% If you are submitting a paper to SciPost Biology: uncomment next line
% \documentclass[submission, Bio]{SciPost}

%% CHEMISTRY:
% If you are submitting a paper to SciPost Chemistry: uncomment next line
% \documentclass[submission, Chem]{SciPost}

%% COMPUTER SCIENCE:
% If you are submitting a paper to SciPost Computer Science: uncomment next line
% \documentclass[submission, CompSci]{SciPost}

%% MATHEMATICS:
% If you are submitting a paper to SciPost Mathematics: uncomment next line
% \documentclass[submission, Math]{SciPost}



% Prevent all line breaks in inline equations.
\binoppenalty=10000
\relpenalty=10000

\hypersetup{
    colorlinks,
    linkcolor={red!50!black},
    citecolor={blue!50!black},
    urlcolor={blue!80!black}
}

\usepackage[bitstream-charter]{mathdesign}
\urlstyle{sf}

% Fix \cal and \mathcal characters look (so it's not the same as \mathscr)
\DeclareSymbolFont{usualmathcal}{OMS}{cmsy}{m}{n}
\DeclareSymbolFontAlphabet{\mathcal}{usualmathcal}

\begin{document}

% TODO: write your article's title here.
% The article title is centered, Large boldface, and should fit in two lines
\begin{center}{\Large \textbf{
Antisymmetric Breaking of Voltage Gauge Invariance due to Majorana States in Magnetic Topological Insulators \\
}}\end{center}

% TODO: write the author list here. Use first name (+ other initials) + surname format.
% Separate subsequent authors by a comma, omit comma and use "and" for the last author.
% Mark the corresponding author with a superscript star.
\begin{center}
Daniele Di Miceli\textsuperscript{1,2$\star$},
Eduárd Zsurka\textsuperscript{2,3,4},
Julian Legendre\textsuperscript{2}, \\
Kristof Moors\textsuperscript{3,4},
Thomas L. Schmidt\textsuperscript{2,5} and
Llorenç Serra\textsuperscript{1,6}
\end{center}

% TODO: write all affiliations here.
% Format: institute, city, country
\begin{center}
{\bf 1} Institute for Cross-Disciplinary Physics and Complex Systems IFISC (CSIC-UIB), E-07122 Palma, Spain
\\
{\bf 2} Department of Physics and Materials Science, University of Luxembourg, 1511 Luxembourg, Luxembourg
\\
{\bf 3} Peter Grünberg Institute  (PGI-9), Forschungszentrum Jülich, 52425 Jülich, Germany
\\
{\bf 4} JARA-Fundamentals of Future Information Technology, Jülich-Aachen Research Alliance, Forschungszentrum Jülich and RWTH Aachen University, Germany
\\
{\bf 5}
School of Chemical and Physical Sciences, Victoria University of Wellington, P.O. Box 600, Wellington 6140, New Zealand
\\
{\bf 6}
Department of Physics, University of the Balearic Islands, E-07122 Palma, Spain
\\
% TODO: provide email address of corresponding author
${}^\star$ {\small \sf daniele@ifisc.uib-csic.es}
\end{center}

\begin{center}
\today
\end{center}

% For convenience during refereeing (optional),
% you can turn on line numbers by uncommenting the next line:
%\linenumbers
% You should run LaTeX twice in order for the line numbers to appear.

\section*{Abstract}
{\bf
% TODO: write your abstract here.
We theoretically discuss how Majorana bound states and Majorana chiral propagating states break voltage gauge invariance of electric transport in specific ways.
While the breaking of voltage gauge invariance can be generically related to Andreev processes at interfaces, an \emph{antisymmetric} conductance with respect to the point of equally split bias across a normal-superconductor-normal (NSN) junction is a more specific signal of topologically-protected Majorana modes.
These electric signatures are discussed in NSN junctions made with narrow (wire-like) or wide (film-like) magnetic topological insulator slabs with a central proximitized superconducting sector.
}


% TODO: include a table of contents (optional)
% Guideline: if your paper is longer that 6 pages, include a TOC
% To remove the TOC, simply cut the following block
\vspace{10pt}
\noindent\rule{\textwidth}{1pt}
\tableofcontents\thispagestyle{fancy}
\noindent\rule{\textwidth}{1pt}
\vspace{10pt}


%%%%%%%%%%%%%%%%%%%%%%%%%%%%%%%%%%
%%%%%%%%%%% INTRODUCTION %%%%%%%%%
%%%%%%%%%%%%%%%%%%%%%%%%%%%%%%%%%%
\section{Introduction}
\label{sec:intro}

Majorana modes in solid state physics are zero-energy quasiparticle excitations with the unusual property of being their own antiparticles \cite{Majorana, Majorana_returns}, which emerge in 1D and 2D topological superconductors (TSCs) over boundaries and vortices \cite{Majorana_solid_state}. 
These fascinating states can be distinguished into Majorana chiral propagating states (MCPS) in two-dimensional superconducting phases \cite{Majoranas_TSC, TSC_2D_and_3D, Topological_Ins_Sc}, and zero-energy Majorana bound states (MBS) in spinless $p$-wave superconducting chain \cite{Kitaev_chain}.
The former are dispersive modes analogous to quantum anomalous Hall and quantum spin Hall edge states in superconducting materials \cite{TSC_2D_and_3D, Topological_Ins_Sc}, while the latter are localized in-gap modes emerging at the ends of gapped phases of 1D topological superconducting wires.
Both allow non-abelian braiding operations, which makes them promising for fault-tolerant topological quantum computing \cite{Non-abelyan_anyons, QC_anyons, QC_Chiral_Majoranas}.
Magnetic topological insulators \cite{magnetic_TIs}, i.e., 3D topological insulators (TIs) with topological surface states and ferromagnetic ordering, are outstanding candidates for the realization of such robust platforms for quantum computation, since in presence of proximity coupling to an ordinary $s$-wave superconductor they realize different TSCs with either propagating or localized Majorana modes \cite{Chiral_TSC, Chiral_TSC_Half-Integer_Plateau, QAH_Majorana_Platform, Quasi-1D-QAH-Majorana}.


Despite the growing interest in proximitized MTIs \cite{Progress_MTIs}, the experimental detection of Majorana modes is still a matter of concern \cite{Retracted_MCPS, Editorial_Retraction, Absence_Evidence}.
In this paper, we highlight a characteristic feature of Majorana modes that can be used in their detection. 
Through theoretical analysis and numerical simulations, we find that both types of Majoranas can be detected in NSN junctions between normal (N) and proximitized (S) magnetic topological insulators when
the bias between the two N sections is split asymmetrically with respect to the central S lead.
Without Majorana modes, or any trivial Andreev bound states (ABSs) which can also be found in \emph{non-topological} 1D superconductors \cite{ABS_review, ABS_introduction}, the proposed setup is gauge invariant, in the sense that the electric current flowing through the junction is independent of how the bias is split between left and right terminals. 
Due to the emergence of Andreev processes at the interfaces of the junction, Majorana states and ABSs break the voltage gauge invariance, making the current on the two normal leads depend on the fraction of the bias that is applied to each side of the junction.
In these conditions, charge conservation implies the existence of an electric current going to ground from the superconductor, defining a nonzero total differential conductance.
In presence of nontrivial MBSs or MCPSs, this differential conductance is \emph{antisymmetric} with respect to the splitting of the bias, since the probability of Andreev reflection and transmission is equal on both N sides. 
Conversely, a trivial ABS gives rise to a conductance that is not antisymmetric with the bias split, due to different Andreev processes on the two sides of the junction.
Observing how the total conductance varies with  the bias split across the junction provides a robust criterion to identify topologically-protected Majorana modes in MTIs.
Similar criteria to detect MBSs on the ends of proximitized semiconducting wires have been discussed in recent works \cite{Ros18,Ans19,Dan20,Men20,Pan21,prot}, without addressing however the gauge invariance breaking as a main underlying principle.




%%%%%%%%%%%%%%%%%%%%%%%%%%%%%%%%%%
%%%%%%%% MODEL HAMILTOMNIAN %%%%%%
%%%%%%%%%%%%%%%%%%%%%%%%%%%%%%%%%%
\section{Model Hamiltonian}

To start, we consider the Hamiltonian of a 3D TI in presence of ferromagnetic ordering. 
In the basis
$\phi_{k\sigma}^\tau = ( c_{k \uparrow}^{+}, c_{k \uparrow}^{-}, c_{k \downarrow}^{+}, c_{k \downarrow}^{-})^T$,
where $c_{k \sigma}^{\tau} \equiv c_{k \sigma}^{\tau}(y,z)$ annihilates an electron with longitudinal wave number $k \equiv k_x$, spin $\sigma=\uparrow, \downarrow$ and orbital index $\tau=\pm$,
the effective 3D Hamiltonian for magnetic TIs takes the following form \cite{Zhang_TI_Model, Zhang_Hamiltonian}
\begin{gather}\label{eq:MTI_Hamiltonian}
    \mathcal{H}_0 (\mathbf{k}) = \epsilon(\mathbf{k}) + M(\mathbf{k}) \tau_z + A(\mathbf{k}) \tau_x + \Lambda \sigma_z \,,
\end{gather}
where
\begin{equation}
\begin{split}
    \epsilon(\mathbf{k}) & = \mu - C_\perp \left( k_x^2+\hat{k}_y^2 \right) - C_z \hat{k}_z^2 \,, \\
    %
    M(\mathbf{k}) & = M_0 - M_\perp \left( k_x^2+\hat{k}_y^2 \right) - M_z \hat{k}_z^2 \,, \\
    %
    A(\mathbf{k}) & = A_\perp \left( k_x \sigma_x + \sigma_y \hat{k}_y \right) + A_z \sigma_z\hat{k}_z \,.
\end{split}
\end{equation}
Here $\mathbf{k}=(k_x,\hat{k}_y,\hat{k}_z)$ and the transverse momentum operators are given by $\hat{k}_{y (z)}=-i \hbar \frac{\partial}{\partial y (z)}$.
The Pauli matrices $\sigma_i$ ($\tau_i$) with $i=x,y,z$ act on spin (orbital) subspaces, the magnetization, assumed along $z$, is represented by the Zeeman term $\Lambda  \sigma_z$ and $\mu$ is the chemical potential.
This Hamiltonian is suitable to describe TIs such as $\text{Bi}_2\text{Se}_3$, $\text{Bi}_2\text{Te}_3$ and $\text{Sb}_2\text{Te}_3$ through proper choice of parameters \cite{Zhang_TI_Model}.
In our simulations, we used the values given in the ``toy model'' in Ref. \cite{Magnetotransport_signatures_TIs}.
When placed in proximity to an ordinary $s$-wave superconductor, the system is described by the Bogoliubov-de Gennes (BdG) Hamiltonian \cite{BdG_Theory} 
\begin{equation}\label{eq:BdG_Hamiltonian}
    \mathcal{H}_{\text{BdG}}(\mathbf{k})     = 
    \begin{pmatrix}
        \mathcal{H}_0(\mathbf{k}) & \Delta^\star \\
        \Delta & -\sigma_y \mathcal{H}_0^\star(-\mathbf{k}) \sigma_y
    \end{pmatrix}
    \,,
\end{equation}
expressed in the basis \cite{BdG_basis}
\begin{equation}\label{eq:BdG_wavefunction}
    \Phi_{k \sigma}^\tau = 
    \begin{pmatrix}
        c_{k \uparrow}^{+}, c_{k \uparrow}^{-}, c_{k \downarrow}^{+}, c_{k \downarrow}^{-},
        %
        -c_{-k \downarrow}^{+ \dagger}, -c_{-k \downarrow}^{- \dagger}, c_{-k \uparrow}^{+ \dagger}, c_{-k \uparrow}^{- \dagger}
    \end{pmatrix}^T
    \,,
\end{equation}
where $\Delta \equiv \Delta(y,z)$ is the superconducting pairing field induced by proximity.
In the following, we fixed the thickness of the slab to $d=4$ nm and considered a wire-like geometry with width $L_y=20$ nm and a film-like one with $L_y=160$ nm.
We assumed a constant pairing field along $y$, and modelled the proximity coupling on the upper surface of the magnetic TI through a stepwise function $\theta$ of the form
\begin{equation}
    \Delta(y,z) = \Delta\, \theta\left( z - d/2 \right ) \,.
\end{equation}

All the numerical results below are obtained with $\Delta=5$ meV for the wire ($L_y=20 $ nm) and $\Delta=10$ meV for the film ($L_y=160 $ nm). Despite these values are a bit unrealistic, qualitatively similar results can be obtained through smaller pairings and rescaled systems.
Indeed, the decay length of Majorana edge states is inversely proportional to the pairing potential $\xi \propto 1/\Delta$, meaning that, in order to guarantee well-separated MBSs at the ends of a MTI wire, a smaller pairing can be compensated by a greater length $L_x$ as long as the ratio $\xi/L_x$ is unchanged.
Similarly, a smaller gap requires the thin film to be wider to maintain the ratio $\xi/L_y$ unchanged and ensure decoupled edge modes in the phase with MCPSs.
In this way, a larger pairing allows us to reduce the computational effort using smaller systems and, at the same time, gives us the opportunity to enhance the energy gap for MBSs and  increase the width of the region with MCPSs.
A similar scaling has already been proposed in graphene \cite{Scalable_Graphene}. 


%Majorana States in Film and Wire
\begin{figure}
    \centering
    \includegraphics[width=0.9\linewidth]{Figures/Energy-Bands.pdf}
    \caption{\label{fig:Energy-Bands}
    (a) $k=0$ low energy states and (b)-(c) full energy spectrum for an infinitely-long thin film with $\mu=0$. The black dashed line in (a) stands for the bulk gap, while the band structures are computed with (b) $\Lambda=15$ meV and (c) $\Lambda=30$ meV.
     (d) $k=0$ energy gap and (e)-(f) band structures for an infinite wire with $\mu=10$ meV.
    The band structures are obtained with (e) $\Lambda=10$ meV and (f) $\Lambda=30$ meV.
    Red and blue colours represent electron and hole modes, respectively, purple is a superposition of the two states.
    }
\end{figure}


Since the Hamiltonian for a 2D system with particle-hole symmetry belongs to the D symmetry class, the different phases in an effective two-dimensional slab can be labelled by an integer topological invariant $\mathcal{N}$ \cite{Topological_Table, Topological_classification_symmetries}.
A chiral TSC with odd Chern invariant and unpaired Majorana modes can be realized in a 2D thin film from the quantum anomalous Hall (QAH) phase, which is routinely achieved in MTIs \cite{QAHE_in_MTIs, Observation_QAH_MTI, QAH_intrinsic_MTI}.
For $\mu=0$, the proximity pairing induces a novel region between the $\mathcal{N}=0$ trivial superconductor and the $\mathcal{N}=2$  QAH state \cite{Majorana_backscattering}.
In this intermediate region, the MTI thin film realizes a $\mathcal{N}=1$ TSC with \emph{unpaired} chiral Majorana modes on the edges \cite{Chiral_TSC, Chiral_TSC_Half-Integer_Plateau}.
The occurrence of this chiral TSC region can be observed in Fig. \ref{fig:Energy-Bands}(a), which displays the $k=0$ low-energy eigenvalues of Eq. \eqref{eq:BdG_Hamiltonian} solved in the thin film geometry as a function of $\Lambda$.
The black dashed line represents the bulk energy gap, showing the existence of two distinct critical points where topological phase transitions occur with the emergence of gapless edge modes within the bulk gap.
Figs. \ref{fig:Energy-Bands}(b)-(c) display the full band structure of these phases: the first shows a single crossing of unpaired MCPSs which characterizes the $\mathcal{N}=1$ chiral topological superconductor, the second corresponds to the BdG quasiparticle spectrum of a $\mathcal{N}=2$ proximitized QAH system. 


While a thin film with $\mu=0$ can realize different 2D topological superconducting states, a narrow MTI wire with $\mu \neq 0$ can be exploited to achieve quasi one-dimensional TSCs with end-localized MBSs \cite{QAH_Majorana_Platform}.
Since the effective BdG Hamiltonian of a QAH/SC heterostructure in a 1D geometry fits in the BDI symmetry class \cite{Quasi-1D-QAH-Majorana}, the topological properties of the system are characterized by an integer invariant \cite{Topological_Table, Topological_classification_symmetries}, which discriminates between trivial $N_{BDI}=0$ and topological $N_{BDI}=1$ states with unpaired Majorana edge modes in finite-length systems. 
In principle, even higher topological states with $N_{BDI} \ge 2$ could be realized, but they are not relevant for our discussion as in presence of disorder a pair of co-located MBSs (i.e., at the same end of the ribbon) couples into a trivial fermion \cite{Kitaev_chain}.
The $k=0$ gap for a $\mu=10$ meV infinitely-long wire is shown in Fig. \ref{fig:Energy-Bands}(d), where the closing and reopening of the energy gap signals a phase transition between trivial and topological states. The full band structures of the two distinct phases are depicted in Figs. \ref{fig:Energy-Bands}(e)-(f).
It can be noted in Fig.\ \ref{fig:Energy-Bands}(f) that the normal order of the energy bands around $k=0$ is inverted, indicating a nontrivial topology of the bulk and, as a consequence of the bulk-boundary correspondence, the presence of topologically protected MBSs on 
the extremities of wires with finite length \cite{Colloquium_TIs}.




%%%%%%%%%%%%%%%%%%%%%%%%%%%%%%%%%%
%%%% GAUGE-INVARIANCE BREAKING %%%
%%%%%%%%%%%%%%%%%%%%%%%%%%%%%%%%%%
\section{Gauge-Invariance Breaking}

\begin{figure}
    \centering
    \includegraphics[width=0.9\linewidth]{Figures/Setup.pdf}
    \caption{\label{fig:Setup}
    (a) Experimental setup proposed for the detection of topologically-protected Majorana modes. The potential $V_0=-e \mu$ is set by the back-gade electrode.
    (b) Sketches of the transmission processes at the interfaces of the junction for different superconducting phases in the central sector. Red and blue colors stand for electron and hole currents.
    }
\end{figure}

We propose to detect Majorana modes in topological superconductors through a NSN junction with an \emph{asymmetric} bias drop between left and right leads. The experimental setup is schematically shown in Fig. \ref{fig:Setup}(a).
The electric current $I_i$ in the normal terminal $i=1,2$ of a double junction can be computed through the Lambert formalism as \cite{Lambert, Transport_Majorana_Nanowires}
\begin{equation}
    I_i = \int_0^{+\infty} dE \sum_a s_a\, \lbrack J_i^a (E) - K_i^a (E) \rbrack \,,
\end{equation}
where 
\begin{equation}
\begin{aligned}
    J_i^a (E) &= \frac{e}{h} N_i^a(E)\, f_i^a(E) \,, \\[7pt]
    %
    K_i^a (E) &= \frac{e}{h} \sum_{jb} P_{ij}^{ab}(E)\, f_j^b(E) \,, 
\end{aligned}   
\end{equation}
are the in-going and out-going fluxes of quasiparticles of type $a=e,h$ ($s_{e,h}=\pm$) in the normal lead $i=1,2$.
The electric current is expressed in terms of the number of propagating modes in each terminal $N_i^a$, the Fermi distribution function $f_i^a$ and the transmission amplitudes $P_{ij}^{ab}$.
The latter indicates the probability of transmission of a quasiparticle $b$ in lead $j$ to a quasiparticle $a$ in lead $i$, such that both normal and Andreev reflection $i=j$ and transmission $i \neq j$ are taken into account.
We define a differential conductance in the normal terminals of the double junction as
\begin{equation}
    G_{i} = \frac{\partial I_i}{\partial V} \,,
\end{equation}
where $V=V_1-V_2$ is the \emph{total} bias across the junction and $V_i$ is the voltage drop between the $i$-th lead and the central sector.
Here, we assume an asymmetric bias $V_1=\alpha V$ and $V_2=-\beta V$ with $0 \leq \alpha \leq 1$ and $\alpha+\beta=1$, such that the total bias between left and right terminals is fixed.
With this assumption, we can derive the following expressions for the conductance in the normal leads
\begin{eqnarray}
    G_1(V) & = & \alpha \frac{e^2}{h} \left\lbrack N_1^e(\alpha V) - P_{11}^{ee}(\alpha V) + P_{11}^{he}(\alpha V) \right\rbrack \nonumber\\
    &+& \beta \frac{e^2}{h} \left\lbrack P_{12}^{hh}(\beta V) - P_{12}^{eh}(\beta V) \right\rbrack \,, 
    \label{eq:G_1}\\   
    G_2(V) & = & \beta \frac{e^2}{h}  \left\lbrack  -N_2^h(\beta V) - P_{22}^{eh}(\beta V) + P_{22}^{hh}(\beta V) \right\rbrack  \nonumber \\
    &+& \alpha \frac{e^2}{h} \left\lbrack P_{21}^{he}(\alpha V) - P_{21}^{ee}(\alpha V) \right\rbrack \,.    
\label{eq:G_2}
\end{eqnarray}



%Conductances Table G1,G2,Gs 
\begin{table}
\begin{center}
\begin {tabular}{c|c|c|c}
\hline
\hline
S-Phase     & $G_1$ & $G_2$ & $G_t$  \\
\hline
$\mathcal{N}=0$     & 0                 & 0                     & 0 \\
$\mathcal{N}=1$     & $\alpha e^2/h$    & $(\alpha-1) e^2/h$    & $(2\alpha-1) e^2/h$  \\
$\mathcal{N}=2$     & $e^2/h$           & $-e^2/h$              & 0 \\
%
$N_{BDI}=0$         & 0                 & 0                     & 0 \\
$N_{BDI}=1$         & $2\alpha e^2/h$   & $2(\alpha-1) e^2/h$   & $2(2\alpha-1) e^2/h$ \\
ABS                 & $2\alpha e^2/h$   & 0                     & $2\alpha e^2/h$ \\
\hline
\hline
\end{tabular}
\end{center}
\caption{\label{tab:conductances}
Low-bias conductances $G_1, G_2$ and $G_t=G_1+G_2$ computed through Eqs. \eqref{eq:G_1}-\eqref{eq:G_2}. 
The first column summarizes all the possible phases in the central S lead of the junction. In the last row, we considered a trivial ABS on the left side of the junction.
The conductances are given for $\beta=\alpha-1$.
}
\end{table}


Without Andreev processes, the system conductance is gauge invariant, in the sense that the current flowing across the junction depends only on the total voltage drop $V$ and is not affected by the way the bias is distributed on left and right leads.
The emergence of Majorana edge states or trivial ABSs in the superconductor \emph{breaks} such an invariance, so that the currents in the two terminals acquire different intensities proportional to the fraction $\alpha, \beta$ of the total bias applied in the two sides of the junction.
According to the different bulk topology of the proximitized sector, different electrical responses are expected, as sketched in Fig. \ref{fig:Setup}(b) for all the possible cases.
At low bias, the trivial superconductor has an insulating behaviour, preventing electrical current flow between the normal terminals of the system. Conversely, the $\mathcal{N}=2$ topological superconductor allows perfect transmission of electrons and holes across the junction, due to topologically-protected fermionic edge modes within the bulk gap.
More complex scattering processes are allowed in presence of Majorana states: in a $N_{BDI}=1$ proximitized wire with MBSs, electrons and holes undergo perfect Andreev reflection \cite{Andreev_reflection, Majorana_Andreev_Reflection}, while equal probabilities of normal reflection, Andreev reflection, normal transmission and Andreev transmission characterize the interaction with unpaired MCPSs in a $\mathcal{N}=1$ topological superconducting thin film \cite{Majorana_backscattering}.
Similarly to MBSs, a trivial ABS allows Andreev reflection  on only one interface of the double junction.


Choosing appropriate values for the transmission probabilities $P_{ij}^{ab}$ in order to recover the scenarios above, the conductance on the two terminals of the junction can be easily computed from Eqs. \eqref{eq:G_1}-\eqref{eq:G_2}.
Their values are summarized in Table \ref{tab:conductances} for the different phases in the proximitized MTI and for a trivial ABS on the left side of the junction in a wire geometry. 
It can be noted that \emph{only} in presence of Majorana modes or ABSs, i.e., when gauge invariance is broken, the conductance on the two terminals depends on $\alpha$ and the total conductance $G_t = G_1 + G_2 \neq 0$.
However, the behaviour of the total conductance $G_t$ as a function of $\alpha$ distinguishes topologically-protected Majorana states from trivial ABSs.
For the topological case, the total conductance $G_t$ is \emph{antisymmetric} around $\alpha=0.5$  (equal bias splitting), while for the trivial one, the total conductance is antisymmetric around $\alpha=0$ (completely unbalanced bias splitting).
Therefore, the gauge invariance breaking is generically related to the emergence of Andreev processes in the superconductor, while the antisymmetric behaviour of $G_t$ around $\alpha=0.5$ is a more specific signal of MBSs or MCPSs.

The total conductance $G_t \neq 0$ is related to the existence of an electric current going to ground from the superconductor, which ensures charge conservation when gauge invariance is broken and the current entering on the left is different from the one flowing out on the right.
Since this current can be easily detected through electric measurements, the detection of an antisymmetric conductance $G_t$ with respect to the bias split parameter $\alpha$ gives a robust criterion to identify Majorana quasiparticles in MTI slabs.
%
We point out that, in our model, the only current flowing through the $s$-wave superconductor is due to Cooper pairs originating in the proximitized MTI. 
Indeed, for low bias, no higher modes can be activated, preventing unintended transmissions between the terminals of the junction. 
We also neglected scattering processes occurring between the normal leads and the $s$-wave superconductor, since the presence of a physical interface between the two distinct materials make them less favourable than the scattering events which take place within the MTI slab. 




%%%%%%%%%%%%%%%%%%%%%%%%%%%%%%%%%%
%%%%%%%% NUMERICAL RESULTS %%%%%%%
%%%%%%%%%%%%%%%%%%%%%%%%%%%%%%%%%%
\section{Numerical Results}

%Conductances Gs
\begin{figure}
    \centering
    \includegraphics[width=0.9\linewidth]{Figures/Conductances.pdf}
    %
    \caption{\label{fig:Conductances}
    Conductance $G_t$ computed in the NSN junction as a function of (a) magnetization and (b) bias split. In the left panel $\alpha=0.25$ and the blue (green) line stands for the wire (thin film) geometry.
    The ABS is modelled adding a barrier on the right side of the junction with the proximitized sector in the $N_{BDI}=1$ phase.
    (c)-(d) Transmission amplitudes $P_{ij}^{ab}$ for the left terminal of the junction as a function of the length $L_x$ of the central proximitized sector. 
    The probabilities are computed for a (c) $N_{BDI}=1$ superconductor with MBSs and a (d) $\mathcal{N}=1$ superconductor with MCPSs.
    In all the pictures, the total bias $V=0.1$ meV is chosen within the bulk gap.
    The values of $\Lambda$ in panels (b),(c) and (d) are chosen according to Fig. \ref{fig:Energy-Bands} to reproduce the different TSCs.
    }
\end{figure}

We computed numerically the conductances $G_1, G_2$ and the sum $G_t=G_1+G_2$ in the NSN junction with a magnetic TI in the wire and thin film configurations, reproducing the physics of 1D and 2D topological superconductors, respectively.
Fig. \ref{fig:Conductances}(a) displays $G_t$ versus the magnetization of the MTI for an asymmetric bias $\alpha=0.25$. In the thin film geometry, a region with $G_t \neq 0$ distinguishes the $\mathcal{N}=1$ chiral TSC from the $\mathcal{N}=0$ trivial superconductor and the  $\mathcal{N}=2$ QAH phase, where $G_t=0$ denotes that the electric currents in the two terminals are equal and independent of the bias split.
For the chosen $\alpha$, the conductance for $\mathcal{N}=1$ was expected to be quantized at $G_t=-e^2/2h$, which is roughly the value reached in the nontrivial region with MCPSs.
Similarly, in the wire geometry a plateau  $G_t=-e^2/h$ characterizes the $N_{BDI}=1$ nontrivial phase, while the $N_{BDI}=0$ gapped superconductor exhibits $G_t=0$.
%
Fig. \ref{fig:Conductances}(b) shows $G_t$ as a function of the split parameter $\alpha$ 
for all the  nontrivial phases realized by the proximitized MTI. 
A trivial ABS is also simulated in the wire geometry through a $N_{BDI}=1$ superconductor with an insulating barrier on the right side of the junction.
Here, the values of the conductance are in perfect agreement with our prediction in Table \ref{tab:conductances}: the QAH system displays $G_t=0$ independently of $\alpha$, while Majorana states and trivial ABSs break the gauge invariance, resulting in the $\alpha$ dependence of the total conductance $G_t$. However, the different symmetry around $\alpha=0.5$ distinguishes trivial and topological cases.
We emphasize here that a symmetrically distributed bias $\alpha=0.5$ is \emph{never} able to discriminate the phases with Majorana modes from the trivial superconductor and the proximitized QAH state, since $G_t=0$ for all these distinct phases. 


The lower panels in Fig. \ref{fig:Conductances} display the probabilities $P_{ij}^{ab}$ for all the scattering processes occurring on the left interface of the junction, i.e., normal reflection $R_N$, Andreev reflection $R_A$, normal transmission $T_N$ and Andreev transmission $T_A$.
The figures correspond to a $N_{BDI}=1$ topological superconducting wire with unpaired MBSs in (c) and a $\mathcal{N}=1$ TSC thin film with MCPSs in (d).
The former shows that, when the junction is sufficiently large to prevent transmission by evanescent modes, the injected electron undergoes perfect Andreev reflection $R_A=1$ in presence of MBSs.
The latter indicates that due to MCPSs, normal and Andreev transmission and reflection occur with equal probability $R_N=R_A=T_N=T_A=0.25$.
Oscillations around the expected plateaus are due to the interference between back-scattered chiral modes from the two interfaces of the double junction, resulting in an interferometric behaviour. Such oscillations are expected for a transverse width $L_y$ in the micrometer range or smaller \cite{Conductance_oscillations_MCPS}.
%
For all the above results, the total bias across the junction is $V=0.1$ meV, which is always smaller than the bulk energy gap of the different phases. 
Such low bias ensures that no bulk modes are activated in the proximitized sector and that the injected electrons and holes interact in the condensate only with topologically-protected Majorana boundary states.
Indeed, the proposed framework does not hold for higher bias, which implies interaction with multiple active modes in the superconductor.




%%%%%%%%%%%%%%%%%%%%%%%%%%%%%%%%%%
%%%%%%%%%%% CONCLUSIONS %%%%%%%%%%
%%%%%%%%%%%%%%%%%%%%%%%%%%%%%%%%%%
\section{Conclusion}

In summary, we showed that voltage gauge invariance is a robust criterion to detect Majorana excitations in magnetic topological insulator slabs with a central proximitized section. 
A characteristic dependency on how the total bias is split, namely the antisymmetry of the conductance with respect to the point of equal bias splitting ($\alpha=0.5$) is obtained in presence of Majorana modes.
Detailed model calculations for a narrow (wire-like) and a wide (film-like) slab, hosting MBSs and MCPSs respectively, are shown to support our conclusions. 
Our results will be useful for the experimental detection of the elusive Majorana quasiparticles, contributing to the progress towards a solid platform for quantum computing.






\section*{Acknowledgements}
%Acknowledgements should follow immediately after the conclusion.


% TODO: include author contributions
%\paragraph{Author contributions}
%This is optional. If desired, contributions should be succinctly described in a single short paragraph, using author initials.

% TODO: include funding information
\paragraph{Funding information}
%Authors are required to provide funding information, including relevant agencies and grant numbers with linked author's initials. Correctly-provided data will be linked to funders listed in the \href{https://www.crossref.org/services/funder-registry/}{\sf Fundref registry}.
This project is supported by the QuantERA grant MAGMA, by the National Research Fund, Luxembourg, under the grant INTER/QUANTERA21/16447820/MAGMA, by the German Research Foundation under grant 491798118, by MCIN/AEI/10.13039/501100011033 under project PCI2022-132927, and by the European Union NextGenerationEU/PRTR.
L.S.\ acknowledges support from Grants No.\ PID2020-117347GB-I00, funded by MCIN/AEI/10.13039/501100011033, and No. PDR2020-12 funded by GOIB.
K.M.\ acknowledges the financial support by the Bavarian Ministry of Economic Affairs, Regional Development and Energy within Bavaria’s High-Tech Agenda Project "Bausteine für das Quantencomputing auf Basis topologischer Materialien mit experimentellen und theoretischen Ansätzen" (grant allocation no.\ 07 02/686 58/1/21 1/22 2/23).














\begin{appendix}

%%%%%%%%%%%%%%%%%%%%%%%%%%%%%%%%%%
%%%%%% CONDUCTANCE DERIVATION %%%%
%%%%%%%%%%%%%%%%%%%%%%%%%%%%%%%%%%
\section{Derivation of the Differential Conductances}

We derive here the equations for the conductances $G_1$ and $G_2$ given in the main article.
The nonlocal differential conductance in the NSN double junction between normal and proximitized MTIs can be defined as 
\begin{equation}
    G_{i}(E) = \frac{\partial I_i}{\partial V} \,,
\end{equation}
where $I_i$ is the current in the $i$-th terminal and $V$ is the total voltage drop across the junction.
The electric current can be computed with Lambert's \cite{Lambert} formalism as 
\begin{equation}
    I_i = \int_0^{+\infty} dE \sum_a a\, \lbrack J_i^a (E) - K_i^a (E) \rbrack \,,
\end{equation}
where 
\begin{equation}\label{eq:flux}
    J_i^a (E) = \frac{e}{h} N_i^a(E) f_i^a(E) \,, \qquad 
    K_i^a (E) = \frac{e}{h} \sum_{jb} P_{ij}^{ab}(E) f_j^b(E) \,,    
\end{equation}
are the in-going and out-going fluxes of the quasiparticle of type $a,b=e,h$ in the lead $i=1,2$.
Here the electric current is expressed in terms of the number of propagating modes in each terminal $N_i^a(E)$,  the Fermi distribution function 
\begin{equation}
    f_i^a(E) =
    \begin{cases}
    \frac{1}{1+e^{\left( E - e V_i \right)/k_B T}}  & \text{if } a=e\;, \\
    %
    \frac{1}{1+e^{\left( E + e V_i \right)/k_B T}}  & \text{if } a=h\; ,  
    \end{cases}
\end{equation}
and the transmission amplitudes $P_{ij}^{ab}(E)$ indicating the probability of transmission of a quasiparticle $b$ in lead $j$ to a quasiparticle $a$ in lead $i$. The bias drop applied between the $i$-th normal lead and the central superconducting sector is represented by $V_i$. 
%
By making explicit the sum over the quasiparticle types and using Eq. \eqref{eq:flux}, the electric current can be rewritten as
\begin{equation}
\begin{split}
    I_i & = \frac{e}{h} \int_0^{+\infty} dE \left \lbrack J_i^e - K_i^e - J_i^h + K_i^h \right \rbrack 
    = \frac{e}{h} \int_0^{+\infty} dE \left \lbrack 
    N_i^e f_i^e -\sum_{jb} P_{ij}^{eb} f_j^b - N_i^h f_i^h + \sum_{jb} P_{ij}^{hb} f_j^b 
    \right \rbrack \\
    %
    & = \frac{e}{h} \int_0^{+\infty} dE\, \frac{e}{h} \left \lbrack 
    N_i^e f_i^e -\sum_j \left( P_{ij}^{ee} f_j^e + P_{ij}^{eh} f_j^h  \right)
    - N_i^h f_i^h + \sum_j \left( P_{ij}^{he} f_j^e + P_{ij}^{hh} f_j^h \right)
    \right \rbrack \,,
\end{split}
\end{equation}
where for simplicity we omitted the energy dependence.
Expanding the sum over the terminals $j=1,2$ we can write the electric current into the two leads of the junction as 
\begin{equation}
\begin{split}
    I_1 = \frac{e}{h} \int_0^{+\infty} dE \Biggl\{
    & \left\lbrack  N_1^e - P_{11}^{ee} + P_{11}^{he}  \right\rbrack f_1^e +
    \left\lbrack -N_1^h - P_{11}^{eh} + P_{11}^{hh}  \right\rbrack f_1^h \\
    %
    & \left\lbrack P_{12}^{he} - P_{12}^{ee} \right\rbrack  f_2^e + 
    \left\lbrack P_{12}^{hh} - P_{12}^{eh} \right\rbrack  f_2^h
    \Biggr\} \,,
\end{split}
\end{equation}
and
\begin{equation}
\begin{split}
    I_2 = \frac{e}{h} \int_0^{+\infty} dE \Biggl\{
    & \left\lbrack  N_2^e - P_{22}^{ee} + P_{22}^{he}  \right\rbrack f_2^e +
    \left\lbrack -N_2^h - P_{22}^{eh} + P_{22}^{hh}  \right\rbrack f_2^h \\
    %
    & \left\lbrack P_{21}^{he} - P_{21}^{ee} \right\rbrack  f_1^e + 
    \left\lbrack P_{21}^{hh} - P_{21}^{eh} \right\rbrack  f_1^h
    \Biggr\} \,.
\end{split}
\end{equation}
We assume that the bias is asymmetrically distributed as $V_1=\alpha V$ and $V_2=-\beta V$ with $0 \leq \alpha \leq 1$ and $\beta = 1-\alpha$ such that the total voltage drop across the junction is fixed $V_1-V_2 = V$, and we recall that in the zero-temperature limit the Fermi function takes the form of a stepwise function 
\begin{equation}
\begin{split}
    f_1^{e} & = \frac{1}{1+e^{\left( E - e \alpha V \right)/k_B T}} \xrightarrow[T \to 0]{} \Theta( E - \alpha e V ) \,, \\
    %
    f_1^{h} & = \frac{1}{1+e^{\left( E + e \alpha V \right)/k_B T}} \xrightarrow[T \to 0]{} \Theta( E + \alpha e V ) \,,
\end{split}
\end{equation}
for the left terminal of the junction and
\begin{equation}
\begin{split}
    f_2^{h} & = \frac{1}{1+e^{\left( E + e \beta V \right)/k_B T}} \xrightarrow[T \to 0]{} \Theta( E + \beta e V ) \,, \\
    %
    f_2^{h} & = \frac{1}{1+e^{\left( E - e \beta V \right)/k_B T}} \xrightarrow[T \to 0]{} \Theta( E - \beta e V ) \,,
\end{split}
\end{equation}
for the right one. The expressions of the currents in the two terminals can thus be simplified as
\begin{equation}\label{eq:I_1}
\begin{split}
    I_1 & = \frac{e}{h} \int_0^{+\infty} dE \biggl\{
    \left\lbrack  N_1^e - P_{11}^{ee} + P_{11}^{he}  \right\rbrack \Theta( E - \alpha e V )
    + \left\lbrack P_{12}^{hh} - P_{12}^{eh} \right\rbrack  \Theta( E - \beta e V )
    \biggr\} \\
    %
    & = \frac{e}{h} \int_0^{\alpha e V} dE  \left\lbrack  N_1^e - P_{11}^{ee} + P_{11}^{he}  \right\rbrack +
    \int_0^{\beta e V} dE \left\lbrack P_{12}^{hh} - P_{12}^{eh} \right\rbrack \,,
\end{split}
\end{equation}
and
\begin{equation}\label{eq:I_2}
\begin{split}
    I_2 & = \frac{e}{h} \int_0^{+\infty} dE \biggl\{
    \left\lbrack -N_2^h - P_{22}^{eh} + P_{22}^{hh}  \right\rbrack \Theta( E - \beta e V ) 
    + \left\lbrack P_{21}^{he} - P_{21}^{ee} \right\rbrack \Theta( E - \alpha e V )
    \biggr\} \\
    %
    & = \frac{e}{h} \int_0^{\beta e V} dE  \left\lbrack  -N_2^h - P_{22}^{eh} + P_{22}^{hh} \right\rbrack +
    \int_0^{\alpha e V} dE \left\lbrack P_{21}^{he} - P_{21}^{ee} \right\rbrack \,,
\end{split}
\end{equation}
and the differential conductance can be computed as the derivative of Eqs. \ref{eq:I_1}-\ref{eq:I_2} with respect to the total bias $V$ across the junction, leading to 
\begin{equation}\label{app-eq:G_1}
    G_1(V) = \frac{\partial I_1}{\partial V} = \alpha \frac{e^2}{h} \left\lbrack N_1^e(\alpha V) - P_{11}^{ee}(\alpha V) + P_{11}^{he}(\alpha V) \right\rbrack + 
    \beta \frac{e^2}{h} \left\lbrack P_{12}^{hh}(\beta V) - P_{12}^{eh}(\beta V) \right\rbrack \,,
\end{equation}
and
\begin{equation}\label{app-eq:G_2}
    G_2(V) = \frac{\partial I_2}{\partial V} = \beta \frac{e^2}{h}  \left\lbrack  -N_2^h(\beta V) - P_{22}^{eh}(\beta V) + P_{22}^{hh}(\beta V) \right\rbrack +
    \alpha \frac{e^2}{h} \left\lbrack P_{21}^{he}(\alpha V) - P_{21}^{ee}(\alpha V) \right\rbrack \,.
\end{equation}

%Table for parameters of the conductance G1
\begin{table}
\begin{center}
\begin {tabular}{l|c|c|c|c|c||c}
\hline
\hline
S-Phase     & $N_1^e$ & $P_{11}^{ee}$ & $P_{11}^{he}$ & $P_{12}^{hh}$ & $P_{12}^{eh}$ & $G_1$ \\
\hline
$N_{BDI}=0$             & 1     & 1     & 0    & 0    & 0     & 0               \\
$N_{BDI}=1$ (MBS)       & 1     & 0     & 1    & 0    & 0     & $2\alpha e^2/h$ \\
%
$\mathcal{N}=0$         & 1     & 1     & 0    & 0    & 0     & 0               \\
$\mathcal{N}=1$ (MPCS)  & 1     & 0.25  & 0.25 & 0.25 & 0.25  & $\alpha e^2/h$  \\
$\mathcal{N}=2$ (QAH)   & 1     & 0     & 0    & 1    & 0     & $e^2/h$ \\
\hline
\hline
\end{tabular}
\end{center}
\caption{\label{tab:G_1}
Transmission amplitudes and number of electronic modes required to compute the conductance $G_1$ through Eq. \eqref{app-eq:G_1}. The values are given for all the possible topological phases which can be found in the central superconducting sector of the NSN junction.
}
\end{table}

%Table for parameters of the conductance G2
\begin{table}
\begin{center}
\begin {tabular}{l|c|c|c|c|c||c}
\hline
\hline
S-Phase     & $N_2^h$ & $P_{22}^{eh}$ & $P_{22}^{hh}$ & $P_{21}^{he}$ & $P_{21}^{ee}$ & $G_2$ \\
\hline
$N_{BDI}=0$             & 1     & 0     & 1    & 0    & 0     & 0                   \\
$N_{BDI}=1$ (MBS)       & 1     & 1     & 0    & 0    & 0     & $2(\alpha-1) e^2/h$ \\
%
$\mathcal{N}=0$         & 1     & 0     & 1    & 0    & 0     & 0                   \\
$\mathcal{N}=1$ (MPCS)  & 1     & 0.25  & 0.25 & 0.25 & 0.25  & $(\alpha-1) e^2/h$  \\
$\mathcal{N}=2$ (QAH)   & 1     & 0     & 0    & 0    & 1     & $-e^2/h$            \\
\hline
\hline
\end{tabular}
\end{center}
\caption{\label{tab:G_2}
Transmission amplitudes and number of hole modes required to compute the conductance $G_2$ through Eq. \eqref{app-eq:G_2}. The values are given for all the possible topological phases which can be found in the central superconducting sector of the NSN junction. The conductances are given making explicit $\beta=\alpha-1 $.
}
\end{table}


%Table for resuming the conductances G1, G2 and their sum G_t 
\begin{table}
\begin{center}
\begin {tabular}{l|c|c|c}
\hline
\hline
S-Phase     & $G_1$ & $G_2$ & $G_t$  \\
\hline
$N_{BDI}=0$             & 0                 & 0                  & 0 \\
$N_{BDI}=1$ (MBS)       & $2\alpha e^2/h$   & $2(\alpha-1) e^2/h$  & $2(2\alpha-1) e^2/h$ \\
%
$\mathcal{N}=0$         & 0                 & 0                  & 0 \\
$\mathcal{N}=1$ (MPCS)  & $\alpha e^2/h$    & $(\alpha-1) e^2/h$   & $(2\alpha-1) e^2/h$  \\
$\mathcal{N}=2$ (QAH)   & $e^2/h$           & $-e^2/h$           & 0 \\
\hline
\hline
\end{tabular}
\end{center}
\caption{\label{app-tab:conductances}
Conductances $G_1, G_2$ and their sum $G_t=G_1+G_2$ computed through Eqs. \eqref{app-eq:G_1}-\eqref{app-eq:G_2} using the transmission probabilities given in Tabs.\ref{tab:G_1}-\ref{tab:G_2}. The sum of the conductance on the two terminal is non-zero only in presence of topologically-protected Majorana modes. Furthermore, the value of $G_t$ discriminates between end-localized MBSs and dispersive MPCSs. 
}
\end{table}

The left-terminal conductance, Eq. \eqref{app-eq:G_1}, is given by the number of injected electrons $N_1^e$, the normal $P_{11}^{ee}$ and Andreev $P_{11}^{he}$ reflection amplitudes for electrons injected in lead 1, and the normal $P_{12}^{hh}$ and Andreev $P_{12}^{eh}$ transmission amplitudes for holes injected in lead 2.
Similarly, the right-terminal conductance Eq. \eqref{app-eq:G_2} is given by the number of injected holes $N_2^h$, normal $P_{22}^{hh}$ and Andreev $P_{22}^{eh}$ reflection amplitudes for holes injected in lead 2, and normal $P_{21}^{ee}$ and Andreev $P_{21}^{he}$ transmission amplitudes for electrons injected in lead 1.
The number of injected quasiparticles $N_i^a$ and the values of the transmission amplitudes $P_{ij}^{ab}$ 
in the low-bias scenario described in the main article are given in Tabs. \ref{tab:G_1}-\ref{tab:G_2} for all the topological phases of the superconducting sector.
Tab. \ref{app-tab:conductances} summarizes the corresponding values of the conductances $G_1,G_2$ and their sum $G_t=G_1+G_2$.





%%%%%%%%%%%%%%%%%%%%%%%%%%%%%%%%%%
%%%%%%%% ROLE OF A BARRIER %%%%%%%
%%%%%%%%%%%%%%%%%%%%%%%%%%%%%%%%%%
\section{Role of an Interface Barrier}

%Table for parameters of the conductance G1
\begin{table}
\begin{center}
\begin {tabular}{l|c|c|c|c|c||c|c|c}
\hline
\hline
S-Phase     & $N_1^e$ & $P_{11}^{ee}$ & $P_{11}^{he}$ & $P_{12}^{hh}$ & $P_{12}^{eh}$ & $G_1$ & $G_2$ & $G_t$ \\
\hline
$N_{BDI}=1$ (MBS)       & 1     & 0     & 1    & 0    & 0     & $2\alpha e^2/h$ & 0 & $2\alpha e^2/h$ \\
$\mathcal{N}=1$ (MPCS)  & 1     & 0.5   & 0.5  & 0    & 0     & $\alpha e^2/h$  & 0 & $\alpha e^2/h$ \\
$\mathcal{N}=2$ (QAH)   & 1     & 1     & 0    & 0    & 0     & 0               & 0 & 0 \\
\hline
\hline
\end{tabular}
\end{center}
\caption{\label{tab:G_barrier}
Transmission amplitudes for the scattering processes on the left interface and conductances $G_1,G_2$ and $G_t$ assuming an insulating barrier on the right side of the system.
The values are given for all the topologically nontrivial cases.
}
\end{table}

We consider in this section the role of an interface barrier between the central proximitized (S) sector  and the right (N) lead. That is, an NSN'N  structure where N' represents a slab of a normal MTI material without any propagating modes. The presence of N' breaks the left-right symmetry 
with respect to the central sector and, depending on the 
barrier transparency, it will affect the electrical connection to the right side. A small barrier length 
mimicks some interface disorder, while a large barrier length corresponds to the complete electrical insulation.


We show here that the presence of a barrier does \emph{not} change our conclusions about the breaking of the gauge invariance. 
Indeed, despite the value of the conductance $G_1$ and $G_2$ may change, the total conductance keeps its meaning, with $G_t \neq 0$ as long as Andreev processes take place in the junction.
The different cases can be clearly understood when a completely insulating barrier is introduced, for instance, on the right side of the system: as the right lead is electrically disconnected from the proximitized MTI, $G_2=0$ regardless of the topological phase realized in the proximitized sector.
In a $N_{BDI}=1$ superconducting wire, perfect Andreev reflections occurs on the left interface of the junction due to interaction with MBS.
In a $\mathcal{N}=1$ TSC film, the electron is completely reflected, since the transmission to the right side is prevented by the barrier. Normal and Andreev processes take place with same probability.
In an analogous way, in a $\mathcal{N}=2$ TSC, the electrons are perfectly reflected from the barrier, and no Andreev processes take place in the junction.
The conductances $G_1$ and $G_2$ can be easily computed through Eqs. \eqref{eq:G_1}-\eqref{eq:G_2}. Their values, together with the transmission amplitudes for the left interface of the junction, are summarized in Tab. \ref{tab:G_barrier}.

%Barrier
\begin{figure}
    \centering
    \includegraphics[width=\linewidth]{Figures/SM_Barrier.pdf}
    \caption{\label{fig:Barrier}
    Total conductance $G_t$ for a NSN'N junction with a proximitized sector in the (a) $N_{BDI}=1$ phase and (b) $\mathcal{N}=1$ TSC with an insulating barrier on the right side of the system. The conductance is computed as a function of the length $L_x$ of the barrier. The plots are obtained with $\alpha=0.25$.
    }
\end{figure}


%Our conclusion about the gauge invariance breaking are not affected by the presence of an insulating barrier on one side of the system, with $G_t \neq 0$ in presence of Andreev processes in the junction.
A numerical simulation for the conductance in the NSN'N junction with $\alpha=0.25$ is shown in Fig. \ref{fig:Barrier} for (a) a wire geometry with a proximitized sector in the $N_{BDI}=1$ state and (b) a film geometry with a $\mathcal{N}=1$ TSC in the central sector. We focus on the dependence 
on $L_x$, the length of the intermediate barrier N'.
For a completely transparent barrier ($L_x \approx 0$) the total conductance 
for $\alpha=0.25$ is $G_t=-e^2/h$ in presence of MBS and $G_t=-e^2/2h$ in presence of MCPS.
An increasingly opaque barrier ($L_x\to\infty$) changes these values, keeping $G_t \neq 0$ as long as Andreev processes occur in the proximitized MTI.
For the case of MBSs, a barrier with $L_x \gtrsim 50\, {\rm nm}$ is long enough to prevent the electric transmission on the right side, leading to $G_t=e^2/2h$.
Remarkably, Fig.\
\ref{fig:Barrier}(b) shows a very different decay length along $x$ for MCPSs.
Indeed, a larger barrier $L_x \gtrsim 5\; \mu{\rm m}$ is required to prevent completely the electric transmission in presence of MCPSs, changing the total conductance to $G_t=e^2/4h$. 
Both limiting values are in  agreement with Tab. \ref{tab:G_barrier} for the selected bias split parameter $\alpha=0.25$.
Focusing on the short barrier limit, which represents interface disorder effects, Fig.\ \ref{fig:Barrier} suggests that the antisymmetric breaking of gauge invariance is robust for barriers with $L_x\lesssim 5$ nm for the MBS and $L_x\lesssim 0.5$ $\mu$m for the MCPS, since $G_t$ is almost unaffected by the barrier in these cases. 





\end{appendix}


\bibliography{main.bib}

\nolinenumbers

\end{document}
