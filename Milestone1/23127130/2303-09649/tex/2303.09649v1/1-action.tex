In the following sections, we use $C^k$ to represent the space of continuous functions with $k$ continuous derivatives, where the domain and range can be inferred by the context. We model a neuronal process as a regular 3D curve $c:[0,L] \rightarrow \mathbb{R}^3$, $c \in C^k$, and $\vert \dot c \vert > 0$. When neurons are traced, they are typically stored as a sequence of points  $\{x_i=c(t_i): t_i < t_{i+1}\}_{i=1}^n$, where the independent variables $t_i$ are not specified and, from a geometric point of view are arbitrary \cite{younes2010shapes}. When there is a diffeomorphism between coordinate systems $\phi: \mathbb{R}^3\rightarrow \mathbb{R}^3$, these traces are mapped via the group action:

\begin{align*}
    \phi \cdot \{x_i\}_{i=1}^n &= \{\phi(x_i)\}_{i=1}^n
\end{align*}

We want to extend the space of traces, and the associated action, to include derivatives of the underlying curve denoted $\partial_t c$. This can be done using the jet space $J^k$. In our setting, $J^k=[0,L] \times X^{(k)}$, where an element of $X^{(k)}$ is a $k+1$-tuple $(x^0,x^1,...,x^k) \in (\mathbb R^3)^{k+1}$ representing a position and first $k$ derivatives of a curve in $\mathbb R^3$. A $C^k$ curve $c: [0,L] \to \mathbb R^3$ can be extended to a curve $\hat c: [0,L] \to X^{(k)}$ simply by adding derivatives, with $\hat c(t) = (c(t), \partial_t c(t), \ldots, \partial_t^k c(t)) \in X^{(k)}$ \cite{olver1995equivalence}. 
%In our setting, we will define the jet space $J^k=[0,L]\times (\mathbb{R}^3)^{k+1}$ which consists of $t$, the one dimensional independent variable, and $x^{(k)}$, the $3(k+1)$ dimensional collection of $c$ and its first $k$ derivatives evaluated at $t$.

The $C^k$ diffeomorphisms have a natural group action on the jet space $J^k$, ensuring the commutation between the standard action of diffeomorphisms on curves, $(\phi, c) \mapsto \phi\circ c$ and their extensions, such that the identity $\phi\cdot \hat c(t) = \widehat{\phi\circ c}(t)$ holds for all curves $c$ and times $t$, defining the left-hand side. For example, for $k=2$, this provides
\[
\phi\cdot (t, x^{0}, x^{1}, x^{2}) = (t, \phi(x^{0}), D\phi(x^{0}) x^{1}, D\phi(x^{0}) x^{2} + D^2\phi(x^{0}) (x^{1}, x^{1}))
\]

%Since $\phi \circ c$ is a composition of $k$-times differentiable functions and is therefore $k$-times differentiable. In other words, we can compute not just the transformed positions $\phi \circ c(t_i)$, but also the transformed derivatives $\partial_t (\phi\circ c)(t_i)$ without knowing the full curve $c(t)$. 

Neuron traces, as mentioned before, involve a sequence of samples with time-stamps $\{(t_i,x^{(k)}_i)\}_{i=1}^n$, identified as elements of $(J^k)^n$, the $n$-fold Cartesian product of $J^k$. Our diffeomorphisms will act on such a sequence as follows:

\begin{statement}
For a sequence of time-stamped elements on the jet space, 
$T = \{(t_i, x_i^{(k)})\}_{i=1}^n$ in $(J^k)^n$, we define the action of diffeomorphisms 


% where $c:[0,L] \rightarrow \mathbb{R}^3$ is a $\mathbf{C}^k$ regular curve, identify a set of $\mathbf{C}^k$ functions $\{f_i\}$ that map from $[0,L]$ to $\mathbb{R}^3$ and satisfy $f_i(t_i)=c(t_i)$ and $\partial_t^j f_i(t_i)=\partial_t^j c(t_i) \; j=1,...,k$. Then the following defines a group action of a $k$-times differentiable diffeomorphism $\phi$ on $T$. 

\begin{align}
    \phi \cdot T = \{(t_i, \phi\cdot x_i^{(k)})\}_{i=1}^n \label{eq:action}
\end{align}
% For a sequence of elements on the jet space, 
% $T = \{(t_i, x_i^{(k)}) : x_i^{(k)} = (c(t_i),\partial_t c(t_i),...,\partial_t^k c(t_i)) \}_{i=1}^n$, where $c:[0,L] \rightarrow \mathbb{R}^3$ is a $\mathbf{C}^k$ regular curve, identify a set of $\mathbf{C}^k$ functions $\{f_i\}$ that map from $[0,L]$ to $\mathbb{R}^3$ and satisfy $f_i(t_i)=c(t_i)$ and $\partial_t^j f_i(t_i)=\partial_t^j c(t_i) \; j=1,...,k$. Then the following defines a group action of a $k$-times differentiable diffeomorphism $\phi$ on $T$. 

% \begin{align}
%     \phi \cdot T = \{(t_i, \bar x_i^{(k)}) : \bar x_i^{(k)} = (c(t_i),\partial_t (\phi \circ f_i)(t_i),...,\partial_t^k (\phi \circ f_i)(t_i))\}_{i=1}^n \label{eq:action}
% \end{align}

\end{statement}


The fact that this operation provides an action is is an established result \cite{olver1995equivalence}, and the proof is provided in the Supplement. We will define \textit{$k$'th order discrete mapping} to be the action in Equation \ref{eq:action} of a diffeomorphism on a curve sampling that includes $k$ derivatives. The axioms that define group actions are important to verify because they ensure that applying the identity transformation does not change the object, and that applying a composition of transformations is equivalent to applying the individual transformations successively \cite{boothby2003introduction}. Further, group actions can exchange mathematical structure between the acting group and the set being acted upon, and they are at the core of several important theorems \cite{suksumran2016gyrogroup}.

The $k$'th order discrete mapping method allows us to compute the first $k$ derivatives of the transformed curve. We will interpolate the transformed curve using splines of order $2k+1$ that satisfy the derivative values. For example, zeroth order mapping will produce a first order spline and first order mapping will produce a cubic Hermite spline \cite{kincaid2009numerical}.
