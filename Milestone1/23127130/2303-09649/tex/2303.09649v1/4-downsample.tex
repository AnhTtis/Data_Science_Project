Heretofore, we were focused on curves that were known only by a discrete sampling of knots. In this section, we will examine a couple examples where the underlying curve $c$ is smooth, and known. The positions and derivatives of the curve are then discretely sampled. It is natural to assume that the first order mapping method should be more accurate in this setting, since it uses derivative information that the zeroth order method does not. For interpolation, derivative information has proven useful in previous research comparing Lagrange interpolation to Hermite interpolation \cite{devia2019comparison}. However, the only well-known theoretical advantages of using derivative information is in the limiting behavior at the knots \cite{kincaid2009numerical}. In the two examples shown in Figure \ref{fig:toy}g-j, the first order method matches ground truth better than the zeroth order method.

