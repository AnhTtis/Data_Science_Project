In this paper we examine the ``naive'' approach to mapping discretely sampled one-dimensional structures by simply transforming the positions of the knots, i.e. mapping line segments to line segments. We show that this method can be inaccurate when the Jacobian of the transformation is non-constant. We describe how to preserve derivative information which will lead to more accurate mappings in neighborhoods of the knots. We offer an implementation of a first-order mapping technique which, empirically, is more accurate on discretely sampled differentiable curves. We also apply our method to real neuron reconstructions and show that it has similar error to zeroth order mapping in the original reconstructions, but can be more accurate when the reconstructions are highly downsampled.

In our experiment with real neuron reconstructions, it is important to note what we are considering ground truth. Since the original reconstructions are in SWC format, only the knot positions are known, and the neurons are typically represented as piecewise linear structures. Real neuron morphologies are not piecewise linear, and instead are continuously curving as they pass through dense brain tissue. Nonetheless, because we have no further information about the neuron trajectories, we consider the original reconstructions to be piecewise linear, and generate the ground truth mappings by transforming the straight lines between the knots. 

Since the registration transformation had a minor LDDMM component (Figure \ref{fig:map-1}b), the straight lines were generally transformed into other straight lines, and the zeroth order method was therefore sufficiently accurate. We believe this result is largely driven by the different scales of the neuron reconstruction segments, and the brain registration. Neuron reconstructions were composed of knots that were spaced, typically, at the tens-of-microns scale. In contrast, brain registration was done at the scale of brain regions, which, for mice, are generally at the hundreds-of-microns scale. Indeed, the default resolution at which CloudReg performs registration is one hundred microns \cite{chandrashekhar2021cloudreg}. This means that the registration was not likely to distort reconstruction segments, making the zeroth order mapping adequate.

When neuron branches were downsampled by a factor of 100 however, larger differences in error emerged. After downsampling, the length of trace segments were long enough that the registration could warp them, as reflected by our error bounds \ref{eq:error}, \ref{eq:error2}. If image resolution and computing power continues to improve, it is likely that registration methods will incorporate higher resolution image information. In this case, the spatial scale of registration transformations  will approach the scale of neuron morphologies, rendering the original problem in Figure \ref{fig:toy}a,b more relevant, and perhaps making first order mapping useful at even smaller sampling periods.

Conversely our results can be used to make manual tracing and storage of trace data more efficient. If the registration transformation, $\phi$, is know a priori, and there are stretches where a neuronal branch is straight, then it is possible to compute the minimum sampling rate while still controlling the amount of error introduced during mapping to atlas coordinates. For example, under the registration computed in our real data experiment, one could sample a straight branch only once every $88$ microns while maintaining zeroth order mapping error below $1$ micron. This is a lower sampling rate than is usually used in practice, meaning our results could make neuron tracing more efficient.

It may be tempting to use our ``ground-truth'' mapping method, i.e. upsampling a linear interpolation then performing zeroth order mapping, as a neuron mapping method. While this may be appropriate in some settings, this approach has two primary disadvantages. First, as stated before, neurons are not piecewise linear structures so, while the knot positions can be generally regarded as lying on the neuron, the linear interpolation cannot. Therefore, it would be necessary to keep track of which knots are from the original trace, and which knots are from the upsampling in order to preserve the original trace information. This would require existing file formats to expand their metadata conventions. Secondly, for large traces, the upsampled data could become computationally cumbersome to store.

Lastly, we want to highlight work in the adjacent field of neuron reconstruction where algorithms such as \cite{li2020brain} can convert reconstruction knots into dense image segmentations which capture neuron trajectories at finer resolutions. Algorithms to automatically trace images of single neurons have been under development for decades \cite{peng2015diadem, athey2022hidden}. They could be adapted to generate both denser neuron samplings, and more accurate derivative estimates at the sampled points. These methods could improve both zeroth and first order mapping methods, so weighing these effects alongside the accuracy required for the given scientific goal would help determine which mapping method is appropriate.