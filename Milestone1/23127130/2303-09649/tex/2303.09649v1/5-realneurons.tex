We applied our method to reconstructed neurons in SWC format from a whole mouse brain image from the Janelia MouseLight project \cite{winnubst2019reconstruction}. Neurons have a tree-like  structure, and we split them into non-branching curves in order to apply our mapping methods. We follow a method introduced previously \cite{athey2021spline} where the root to leaf path with the longest arc length is recursively removed until the tree is reduced to non-bifurcating ``branches''. The whole-brain image was registered to the Allen Reference atlas \cite{wang2020allen} using CloudReg \cite{chandrashekhar2021cloudreg} (Figure \ref{fig:map-1}). CloudReg computes a transformation that is decomposed into an affine component, and nonlinear large deformation diffeomorphic metric mapping (LDDMM) component. Zeroth and first order mapping under the LDDMM component was applied to the reconstructed neurons (Figure \ref{fig:map-2}). The registration transformation for this sample had only a small LDDMM component, as evidenced by the small displacement values in Figure \ref{fig:map-1}b. As in Figure \ref{fig:toy}, we consider ground truth to be the ``continuous mapping'' of the original piecewise linear curve where each line segment is densely sampled, and all the points are mapped into that atlas space.

\begin{figure}[ht]
\centering
\includegraphics[width=\textwidth]{images/fig2-part1.jpg}
\caption{Registration of whole-brain image from Janelia MouseLight project to Allen Reference Atlas (ARA) \cite{wang2020allen}. \textbf{a}, The whole brain image (shown under ``Sample'' with reconstructed neurons in color) was registered to the ARA using CloudReg \cite{chandrashekhar2021cloudreg} (\textbf{a}). \textbf{b} Histograms of position displacement magnitudes and displacement magnitude of the tangent vector $(1,1,1)^T$ by the LDDMM component of the registration across a regular 3D grid over the sample with $100$ micron spacing.}
\label{fig:map-1}
\end{figure}

\begin{figure}[ht]
\centering
\includegraphics[width=\textwidth]{images/fig2-part2.jpg}
\caption{Deformation of reconstructed mouse neurons by a transformation derived from image based registration. \textbf{a-b} Single branch of a neuron reconstruction before (\textbf{a}) and after (\textbf{b}) applying the LDDMM component of the registration. \textbf{a-b} Full neuron reconstruction before (\textbf{a}) and after (\textbf{b}) applying the LDDMM component of the registration. Axes units are in microns.}
\label{fig:map-2}
\end{figure}

In our experiment, we observed that both zeroth and first order mapping agreed with our ground truth. The overlap of the curves in Figure \ref{fig:neurons}b is an example of both methods being accurate. However, when we downsampled the neuron branch traces 100 times, then first order mapping was more similar to ground truth, as in Figure \ref{fig:neurons}d-e. Both zeroth and first order mapping methods were given the same amount of information, the positions of the knots.

We compared discrete Frechet errors between the two discrete mapping methods and ground truth across all $22,984$ neuron branches in this brain sample, and found the same trend. Zeroth and first order mapping had similar deviation from ground truth using the original, complete sampling of the neurons, while there were several instances of first order mapping being superior when the neuron reconstructions were downsampled (Figure \ref{fig:neurons}f-g). We downsampled the branches by retaining both end knots, along with every one out of 100 knots in between. There was a statistically significant correlation between the error difference and average sampling period in the downsampled case (Pearson's correlation test, $\alpha=0.01$), but the correlation was very weak.

Further, we applied Eq. \ref{eq:error2} and found that the error from zeroth order mapping for a line segment of length $L$ microns under this transformation is bounded by, approximately $0.011 L + 0.022$. In particular, the error for a line segment of length $10$ microns is at most $\sim 0.13$ microns and of length $1$mm is at most $\sim 11$ microns. The observed errors in Figure \ref{fig:neurons}f-g obeyed these bounds.

\begin{figure}[ht]
\centering
\includegraphics[width=0.85\textwidth]{images/fig3.jpg}
\caption{Application of discrete mapping methods to original and downsampled neuron traces. \textbf{a-b}, Applying the brain registration vector field to a neuron branch trace \textbf{(a)} where both zeroth order and first order mapping remain close to ground truth \textbf{(b)}. \textbf{c-e}, Applying the same vector field to the trace after downsampling \textbf{(c)} where the first order mapping is more accurate along a long segment of the branch (\textbf{d}, black arrow) which is more visible in a closer view (\textbf{e}, black arrow). \textbf{f-g}, Comparison of discrete Frechet error between zeroth and first order mapping methods from ground truth for 22,984 neuron branches (positive values indicate zeroth order mapping had more error). On the original neuron traces, zeroth order and first order mapping had very similar error \textbf{(f)}. When the traces were downsampled by 100 times, first order mapping often performed better with a weak positive correlation between difference of error and average sampling period \textbf{(g)}. Axes units in \textbf{a-e} are microns.}
\label{fig:neurons}
\end{figure}