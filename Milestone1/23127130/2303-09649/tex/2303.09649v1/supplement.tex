
\section*{Mathematical Proofs}

We use $\vert \cdot \vert$ to denote the Euclidean norm for elements of $\mathbb{R}^d$, and the spectral norm for matrices.

\begin{statement}
For a sequence of time-stamped elements on the jet space, 
$T = \{(t_i, x_i^{(k)})\}_{i=1}^n$ in $(J^k)^n$, we define the action of diffeomorphisms 


\begin{align}
    \phi \cdot T = \{(t_i, \phi\cdot x_i^{(k)})\}_{i=1}^n \label{sup-eq:action}
\end{align}
\end{statement}
\begin{proof}
For a regular differentiable curve $c:[0,L]\rightarrow \mathbb{R}^3$ with extension $\hat{c}:[0,L]\rightarrow X^{(k)}$, we defined $\phi \cdot \hat{c}$ as the extension, $\widehat{\phi \circ c}$. In practice, only the finite sampling $\{x^{(k)}_i\}_{i=1}^n$ is accessible. However, it is always possible to define a curve $c$ that agrees with this sampling, such as with a polynomial \cite{olver1995equivalence}. Then, we can apply $\phi$ to this curve, and compute the transformed positions and derivatives, defining the action on the finite sampling. Now, we verify the axioms of a group action. 
% First, we need to show that the functions $f_i$ exist. A natural choice of $f_i$ is the $k$-th order polynomial which is infinitely differentiable:

% \begin{align*}
%     f_i(t) = c(t_i) + \sum_{j=1}^k \frac{1}{j!} (t-t_i)^j \partial_t^j c(t_i)
% \end{align*}

% The derivatives in the right side of \ref{sup-eq:action} exist since $f_i$ and $\phi$ are $k$-times differentiable \cite{constantine1996multivariate}. Now we turn to the two axioms of a group action, action of the identity element, and action of a composition.

First, the identity element ($\phi_{Id}$) in the diffeomorphism group should leave a sampling unchanged. Assume that $c$ is a curve that agrees with the sampling $T$:

\begin{align*}
    \phi_{Id} \cdot T &= \{(t_i, \bar x_i^{(k)}) : \bar x_i^{(k)} = ((\phi_{Id} \circ c)(t_i),\partial_t (\phi_{Id} \circ c)(t_i),...,\partial_t^k (\phi_{Id} \circ c)(t_i))\}_{i=1}^n \\
    &= \{(t_i, \bar x_i^{(k)}) : \bar x_i^{(k)} = (c(t_i),\partial_t c(t_i),...,\partial_t^k c(t_i))\}_{i=1}^n \\
    &= T
\end{align*}

Second, a composition of diffeomorphisms ($\phi_1 \circ \phi_2$) should act successively on a sampling:

\begin{align*}
    (\phi_1 \circ \phi_2) \cdot T &=  \{(t_i, \bar x_i^{(k)}) : \bar x_i^{(k)} = ... \\
    &((\phi_1 \circ \phi_2 \circ c)(t_i),\partial_t (\phi_1 \circ \phi_2 \circ c)(t_i),...,\partial_t^k (\phi_1 \circ \phi_2 \circ c)(t_i))\}_{i=1}^n \\
    &=  \{(t_i, \bar x_i^{(k)}) : \bar x_i^{(k)} = ... \\
    &((\phi_1 \circ f)(t_i),\partial_t (\phi_1 \circ f)(t_i),...,\partial_t^k (\phi_1 \circ f)(t_i))\}_{i=1}^n & f \triangleq \phi_2 \circ c \\
    &= \phi_1 \cdot \{(t_i, y_i^{(k)}) : y_i^{(k)} = (f(t_i),\partial_t f(t_i),...,\partial_t^k f(t_i)) \}_{i=1}^n \\
    &= \phi_1 \cdot \{(t_i, y_i^{(k)}) : y_i^{(k)} = ... \\
    &((\phi_2 \circ c)(t_i),\partial_t (\phi_2 \circ c)(t_i),...,\partial_t^k (\phi_2 \circ c)(t_i))\}_{i=1}^n \\
    &= \phi_1 \cdot \phi_2 \cdot \{(t_i, x_i^{(k)}) : x_i^{(k)} = (c(t_i),\partial_t c(t_i),...,\partial_t^k c(t_i)) \}_{i=1}^n \\
    &= \phi_1 \cdot \phi_2 \cdot \{(t_i, x_i^{(k)}) : x_i^{(k)} = (c(t_i),\partial_t c(t_i),...,\partial_t^k c(t_i)) \}_{i=1}^n \\
    &= \phi_1 \cdot \phi_2 \cdot T
\end{align*}
\end{proof}




\begin{proposition}
\textbf{[Zeroth Order Mapping Error Bound 1]} Say $\phi: \mathbb{R}^3 \rightarrow \mathbb{R}^3$ is a $C^1$ diffeomorphism and $c: [0,L] \rightarrow \mathbb{R}^3$ is a continuous, piecewise $C^1$ curve parameterized by arc length with knots $\{t_i: t_1=0, t_n=L, t_{i-1} < t_i\}_{i=1}^n$. For the transformed curve $f=\phi \circ c$, the zeroth order mapping defines a first order spline $g$ which satisfies:

\begin{align}
    \max_{t \in [0,L]} \vert f(t)-g(t)\vert &\leq \delta \sqrt{3} \max_{t \in [0,L]} \vert D\phi \circ c(t) \vert \label{sup-eq:error}
\end{align}

\noindent
where $\delta \triangleq \max_{2 \leq i\leq n} \vert t_i-t_{i-1}\vert$, and $D\phi \circ c(t)$ is the Jacobian of $\phi$ evaluated at $c(t)$.
\end{proposition}
\begin{proof}
We will define the modulus of continuity for a function $h:[t_0,t_n] \rightarrow \mathbb{R}$ as in \cite{kincaid2009numerical}:

\begin{align*}
    \omega(h;\delta) &= \max_{\vert s-t\vert \leq \delta} \vert h(s)-h(t)\vert
\end{align*}

and note that if $\dot h$ exists, then $\vert \dot h \vert \leq M$ implies $\omega(h;\delta) \leq M\delta$ by the mean value theorem.

The theorem on spline function approximation in \cite{kincaid2009numerical} states that the first order spline $l$ that interpolates the function $h:[t_0,t_n] \rightarrow \mathbb{R}$ satisfies:

\begin{align*}
    \max_{t \in [t_0,t_n]} \vert l(t)-h(t)\vert &\leq \omega(h,\delta)
\end{align*}

\noindent where $\delta=\max_{0\leq i \leq n} \vert t_i-t_{i-1}\vert$.

Now consider the curve $f=\phi \circ c$. First we will consider the curve restricted to the interval $(t_i,t_{i+1})$, denoted $f_i$. The derivative of $f_i$ exists everywhere and is equal to $(D\phi \circ c(t)) \cdot \dot c(t)$, which is bounded by $\max_{t} \vert D\phi \circ c(t) \vert$ since $\vert \dot c(t) \vert=1$ ($c$ is parameterized by arc length).

If we split the 3D curve $f_i$ into its coordinate functions $f(t)=(f_{i,1}(t), f_{i,2}(t), f_{i,3}(t))^T$, we note $\vert f_{i,k}(t) \vert \leq \vert f_i(t) \vert$ and therefore that $\omega(f_{i,k}, \delta) \leq \delta \max_{t} \vert D\phi \circ c(t) \vert$ for $k=1,2,3$. Then for each $f_{i,k}$, we can construct with a first order spline $g_{i,k}$ from the knots $\{f_{k}(t_i), f_{k}(t_{i+1})\}$. The composite function $g_i(t)=(g_{i,1}(t), g_{i,2}(t), g_{i,3}(t))^T$ satisfies, for all $t \in (t_i,t_{i+1})$:

\begin{align*}
    \vert f_i(t) - g_i(t) \vert &= \sqrt{\vert f_{i,1}(t) - g_{i,1}(t) \vert^2 + \vert f_{i,2}(t) - g_{i,2}(t)  \vert^2 + \vert f_{i,3}(t) - g_{i,3}(t)  \vert^2} \\
    &\leq \sqrt{ \omega^2(f_{i,1},\delta) + \omega^2(f_{i,2},\delta) + \omega^2(f_{i,3},\delta) } \\
    &\leq \delta \sqrt{3} \max_{t} \vert D\phi \circ c(t) \vert
\end{align*}

with $\delta=\vert t_{i+1}-t_i \vert$ since we are looking at a single segment. We can construct the full zeroth order mapping, $g$, by performing the same procedure over all segments indexed by $i$. We then maximize over all segments to get:

\begin{align*}
    \max_{t \in [0,L]} \vert f(t)-g(t) \vert \leq \delta \sqrt{3} \max_{t \in [0,L]} \vert D\phi \circ c(t) \vert
\end{align*}
\end{proof}



\begin{proposition}
\textbf{[Zeroth Order Mapping Error Bound 2]} Say $\phi: \mathbb{R}^3 \rightarrow \mathbb{R}^3$ is a $C^1$ diffeomorphism and $c: [0,L] \rightarrow \mathbb{R}^3$ is a continuous, piecewise linear curve parameterized by arc length with knots $\{t_i: t_1=0, t_n=L, t_{i-1} < t_i\}_{i=1}^n$. For the transformed curve $f=\phi \circ c$, the zeroth order mapping defines a first order spline $g$ which satisfies:

\begin{align}
    \max_{t \in [0,L]} \vert f(t)-g(t)\vert &\leq \max_{i \in \{0,...,n\}, t \in [t_{i-1}, t_i]} \frac{1}{2} \left( \vert D\phi \circ c(t) - I \vert \vert t_i-t_{i-1} \vert + \vert \epsilon_i - \epsilon_{i-1}\vert \right) \label{sup-eq:error2}
\end{align}

\noindent
where $\epsilon_i\triangleq c(t_i)-\phi(c(t_i))$ and $D\phi \circ c(t)$ is the Jacobian of $\phi$ evaluated at $c(t)$.
\end{proposition}

\begin{proof}
We will focus on a single line segment $c_i=c\vert_{[t_{i-1},t_i]}$, then maximize over all such segments. $c_i$ is a function from $[t_{i-1},t_i]$ to $\mathbb{R}^3$. Denote the endpoints of $c_i$ as $c_{i,0}=c_i(t_{i-1})$ and $c_{i,1}=c_i(t_i)$. The zeroth order mapping of $c_i$ defines the first order spline $g_{c_i}(t)=\phi(c_{i,0}) + \frac{(t-t_{i-1})}{t_i-t_{i-1}} (\phi(c_{i,1})-\phi(c_{i,0}))$.

For simplicity, we will reparameterize the problem using $\sigma(t)=t_{i-1}+t(t_i-t_{i-1}):[0,1]\rightarrow [t_{i-1},t_i]$ and define $c'=c_i \circ \sigma$ which is defined on $[0,1]$. The zeroth order mapping of $c'$ defines the spline $g_{c'}(t)=\phi(c_{i,0}) + t (\phi(c_{i,1})-\phi(c_{i,0}))$. Note that the zeroth order mapping errors are the same in both parameterizations i.e. for $f_{c_i}=\phi \circ c_i, f_{c'}=\phi \circ c'$ we have:

\begin{align*}
    \max_{t \in [t_{i-1},t_i]} \vert f_{c_i}(t)-g_{c_i}(t)\vert &=\max_{t \in [0,1]} \vert f_{c'}(t)-g_{c'}(t)\vert
\end{align*}

since, for every $t \in [0,1]$, $\vert f_{c'}(t)-g_{c'}(t)\vert = \vert f_{c_i}(\sigma(t))-g_{c_i}(\sigma(t))\vert$ and for every $t \in [t_{i-1},t_i]$, $\vert f_{c_i}(t)-g_{c_i}(t)\vert = \vert f_{c'}(\sigma^{-1}(t))-g_{c'}(\sigma^{-1}(t))\vert$. So, we have converted the problem to bounding:

\begin{align*}
    \max_{t \in [0,1]} \vert f_{c'}(t)-g_{c'}(t)\vert
\end{align*}

We have

\begin{align*}
    f_{c'}(t)-g_{c'}(t) &= \phi(c'(t)) - \left[\phi(c_{i,0}) + t (\phi(c_{i,1})-\phi(c_{i,0})) \right]
\end{align*}

and since $f_{c'}(t)-g_{c'}(t)$ vanishes at both $t=0$ and $t=1$, the following argument, which uses the fundamental theorem of calculus, applies both going forward from $t=0$ and backward from $t=1$. So, without loss of generality, we consider $0 \leq t \leq \frac{1}{2}$:

\begin{align*}
    f_{c'}(t)-g_{c'}(t) &= \phi(c'(t)) - \left[\phi(c_{i,0}) + t (\phi(c_{i,1})-\phi(c_{i,0})) \right]\\
    &= \int_0^t \partial_{\tau} \left(\phi(c'(\tau)) - \left[\phi(c_{i,0}) + \tau (\phi(c_{i,1})-\phi(c_{i,0})) \right] \right) d\tau \\
    &= \int_0^t D\phi \circ c'(\tau)\cdot \dot c'(\tau) - (\phi(c_{i,1})-\phi(c_{i,0})) d \tau \\
    &\leq \max_{t \in [0,1]} \vert D\phi \circ c'(t) \cdot \dot c'(t) - (\phi(c_{i,1})-\phi(c_{i,0})) \vert \int_0^t d\tau \\
    &\leq \frac{1}{2} \max_{t \in [0,1]} \vert D\phi \circ c'(t) \cdot \dot c'(t) - (\phi(c_{i,1})-\phi(c_{i,0})) \vert &t\leq \frac{1}{2} \\
    &\text{Define: }\epsilon_i\triangleq c_{i,1} - \phi(c_{i,1}),\epsilon_{i-1}\triangleq c_{i,0} - \phi(c_{i,0}) \\
    &= \frac{1}{2} \max_{t \in [0,1]} \vert D\phi \circ c'(t) \cdot (c_{i,1}-c_{i,0}) - (c_{i,1}-c_{i,0}) + (\epsilon_i - \epsilon_{i-1}) \vert  \\
    &\leq \max_{t \in [0,1]} \frac{1}{2}\left( \vert D\phi \circ c'(t) - I \vert \vert c_{i,1}-c_{i,0} \vert  +\vert \epsilon_i - \epsilon_{i-1} \vert \right) \\
    &= \max_{t \in [t_{i-1},t_i]} \frac{1}{2}\left(\vert D\phi \circ c(t) - I \vert \vert c_{i,1}-c_{i,0} \vert  +\vert \epsilon_i - \epsilon_{i-1} \vert \right)\\
    &= \max_{t \in [t_{i-1},t_i]} \frac{1}{2}\left(\vert D\phi \circ c(t) - I \vert \vert t_i-t_{i-1} \vert  +\vert \epsilon_i - \epsilon_{i-1} \vert\right)
\end{align*}

where the last equality comes from the fact that $c$ is parametrized by arc length, so $\vert t_i-t_{i-1} \vert=\vert c_{i,1}-c_{i,0} \vert$. In summary we have:

\begin{align*}
    \max_{t \in [t_{i-1},t_i]} \vert f_{c_i}(t)-g_{c_i}(t)\vert &\leq \max_{t \in [t_{i-1},t_i]} \frac{1}{2}\left(\vert D\phi \circ c(t) - I \vert \vert t_i-t_{i-1} \vert  +\vert \epsilon_i - \epsilon_{i-1} \vert\right)
\end{align*}

Finally, we maximize over all segments to get:

\begin{align*}
    \max_{t \in [0,L]} \vert f(t)-g(t)\vert &\leq \max_{i \in \{0,...,n\}, t \in [t_{i-1},t_i]} \frac{1}{2}\left(\vert D\phi \circ c(t) - I \vert \vert t_i-t_{i-1} \vert  +\vert \epsilon_i - \epsilon_{i-1} \vert\right)
\end{align*}

where $f=\phi \circ c$ and $g$ is the first order spline defined by the zeroth order mapping of $c$. 

\end{proof}


\begin{proposition}
\textbf{[Comparable Bounds for Zeroth and First Order Mapping]} Say $\phi: \mathbb{R}^3 \rightarrow \mathbb{R}^3$ is a $C^4$ diffeomorphism and $c: [a,b] \rightarrow \mathbb{R}^3$ is a continuous, piecewise $C^4$ curve parameterized with knots $\{t_i: t_1=a, t_n=b, t_{i-1} < t_i\}_{i=1}^n$. For the transformed curve $f=\phi \circ c$ defined by coordinate functions $f=(f^0,f^1,f^2)^T$, the zeroth order mapping defines a first order spline $g_0$ which satisfies:

\begin{align}
    \max_{t \in [a,b]} \vert f(t)-g_0(t)\vert &\leq \frac{\sqrt{3}}{4} \max_{t \in [a,b], j \in \{0,1,2\}} \vert \partial^{(4)}_t f^j(t)\vert \left(\frac{\delta}{2}\right)^4 + \nonumber \\
    &\frac{\sqrt{3}}{2} \left(\frac{\delta}{2}\right)^2 \max_{i \in \{1...n\}, j \in \{0,1,2\}} \vert \partial^{(3)}_t f^j(t_i)\vert \left(\frac{\delta}{2}\right) + \nonumber \\
    & \frac{\sqrt{3}}{2} \left(\frac{\delta}{2}\right)^2 \max_{i \in \{1...n\}, j \in \{0,1,2\}} \vert \partial^{(2)}_t f^j(t_i)\vert \label{sup-eq:compare0}
\end{align}

\noindent 
where $\delta\triangleq \max_{2 \leq i\leq n} \vert t_i-t_{i-1}\vert$ and $\partial^{(k)}_t f^j(t)$ is the $k$'th derivative of $f^j$ evaluated at $t$. Also, the first order mapping defines a third order spline $g_1$, which satisfies

\begin{align}
    \max_{t \in [a,b]} \vert f(t)-g_1(t)\vert &\leq \frac{\sqrt{3}}{4!} \max_{t \in [a,b], j\in \{0,1,2\}} \vert \partial^{(4)}_t f^j(t)\vert \left(\frac{\delta}{2}\right)^4 \label{sup-eq:compare1}
\end{align}

\noindent

and we note that the bound in \ref{eq:compare1} is tighter than the bound in \ref{eq:compare0}.
Further, there exists a transformed curve $f$ and a set of knots $\{t_i\}_{i=1}^n$ that achieves both bounds exactly.

\end{proposition}
\begin{proof}
    We will prove both bounds starting with a single segment of $c$, then extending to the entire piecewise curve.
    
    The bound in \ref{sup-eq:compare1} comes from the error estimate of Hermite interpolation for one dimensional functions, Theorem 2 in Section 6.3 of \cite{kincaid2009numerical}. For a single segment between knots $t_{i-1},t_i$, this theorem states that if $p$ is the polynomial of degree at most $3$ which agrees with $h$ and $\partial_t h$ at the knots, then, for each $t$, there exists a point $\xi \in (t_{i-1},t_i)$ such that:

    \begin{align*}
        h(t)-p(t) &= \frac{\partial_t^{(4)} h(\xi)}{4!}(t-t_{i-1})^2(t-t_i)^2 \\
        &\text{therefore, for all }t \\
        \vert h(t)-p(t) \vert &\leq \frac{1}{4!} \max_{t \in [t_{i-1},t_i]} \vert \partial_t^{(4)}h(t) \vert \left( \frac{t_i-t_{i-1}}{2}\right)^4
    \end{align*}

    The first order mapping is indeed the third order spline that matches function and derivative values at the knots, so the above bound applies to all three coordinate functions of $f=(f^0,f^1,f^2)^T$ and $g_1=(g_1^0, g_1^1, g_1^2)^T$. To accommodate all three dimensions, we maximize over all dimensions and add a $\sqrt{3}$ term:

    \begin{align*}
        \max_{t \in [t_{i-1},t_i]} \vert f(t)-g_1(t) \vert &\leq \frac{\sqrt{3}}{4!} \max_{t \in [t_{i-1},t_i], j\in \{0,1,2\}} \vert \partial_t^{(4)}f^j(t) \vert \left( \frac{t_i-t_{i-1}}{2}\right)^4
    \end{align*}

    Then if we maximize both sides over all the segments, we get

    \begin{align*}
        \max_{t \in [a,b]} \vert f(t)-g_1(t) \vert &\leq \frac{\sqrt{3}}{4!} \max_{t \in [a,b], j\in \{0,1,2\}} \vert \partial_t^{(4)}f^j(t) \vert \left( \frac{\delta}{2}\right)^4
    \end{align*}

    The bound in \ref{sup-eq:compare0} comes from the error estimate of polynomial interpolation, Theorem 2 in Section 6.1 of \cite{kincaid2009numerical}. This theorem states that, for a single segment between knots $t_{i-1},t_i$, if $p$ is the line that agrees with a one dimensional function $h$ at the knots, then there exists a point $\xi \in (t_{i-1},t_i)$ such that:

    \begin{align}
        h(t)-p(t)&=\frac{\partial_t^{(2)}h(\xi)}{2} (t-t_{i-1})(t-t_i) \nonumber \\
        &\text{therefore} \nonumber \\
         \max_{t \in [t_{i-1,t_i}]}\vert h(t)-p(t) \vert &\leq \frac{1}{2} \max_{t \in [t_{i-1}, t_i]} \vert \partial_t^{(2)}h(t) \vert \left(\frac{t_i-t_{i-1}}{2}\right)^2 \label{sup-eq:polybound}
    \end{align}

    Our remaining task is to relate the maximum second derivative to the maximum fourth derivative. We start by using the fundamental theorem of calculus twice to get (for $t\in [t_{i-1},t_i]$):

    \begin{align}
        \partial_t^{(2)}h(t)&= \int_{t_{i-1}}^t \int_{t_{i-1}}^\tau h^{(4)}(\upsilon) d\upsilon d\tau + \partial_t^{(3)}h(t_{i-1})(t-t_{i-1}) + \partial_t^{(2)} h(t_{i-1}) \nonumber \\
        &\text{therefore} \nonumber \\
        \vert \partial_t^{(2)}h(t) \vert &\leq \max_{t \in [t_{i-1},t_i]} \vert \partial_t^{(4)}h(t) \vert \int_{t_{i-1}}^t \int_{t_{i-1}}^\tau  d\upsilon d\tau + \nonumber \\
        &\vert t-t_{i-1}\vert \max_{t \in \{t_{i-1}, t_i\}} \vert\partial_t^{(3)}h(t) \vert + \max_{t \in \{t_{i-1}, t_i\}} \vert \partial_t^{(2)} h(t) \vert \nonumber \\
        &= \frac{1}{2}(t-t_{i-1})^2 \max_{t \in [t_{i-1},t_i]} \vert \partial_t^{(4)}h(t) \vert  + \nonumber \\
        &\vert t-t_{i-1}\vert \max_{t \in \{t_{i-1}, t_i\}} \vert\partial_t^{(3)}h(t) \vert + \max_{t \in \{t_{i-1}, t_i\}} \vert \partial_t^{(2)} h(t) \vert \label{sup-eq:bound1}
    \end{align}
    Similarly,
    \begin{align}
        \partial_t^{(2)}h(t)&= \int_{t_{i}}^t \int_{t_{i}}^\tau h^{(4)}(\upsilon) d\upsilon d\tau + \partial_t^{(3)}h(t_{i})(t-t_{i}) + \partial_t^{(2)} h(t_{i}) \nonumber \\
        &\text{therefore} \nonumber \\
        \vert \partial_t^{(2)}h(t) \vert &\leq \max_{t \in [t_{i-1},t_i]} \vert \partial_t^{(4)}h(t) \vert \int_{t_{i}}^t \int_{t_{i}}^\tau  d\upsilon d\tau + \nonumber \\
        &\vert t-t_{i}\vert \max_{t \in \{t_{i-1}, t_i\}} \vert\partial_t^{(3)}h(t) \vert + \max_{t \in \{t_{i-1}, t_i\}} \vert \partial_t^{(2)} h(t) \vert \nonumber \\
        &= \frac{1}{2}(t-t_{i})^2 \max_{t \in [t_{i-1},t_i]} \vert \partial_t^{(4)}h(t) \vert  + \nonumber \\
        &\vert t-t_{i}\vert \max_{t \in \{t_{i-1}, t_i\}} \vert\partial_t^{(3)}h(t) \vert + \max_{t \in \{t_{i-1}, t_i\}} \vert \partial_t^{(2)} h(t) \vert \label{sup-eq:bound2}
    \end{align}

    So, we have two bounds for $\vert \partial_t^{(2)}h(t) \vert$, given by \ref{sup-eq:bound1} and \ref{sup-eq:bound2}. Both bounds are in the form of a second order polynomial of $t$, so it is straightforward to show that the combined bound is maximal at the midpoint between $t_{i-1}$ and $t_i$, where the bounds also happen to intersect. Thus, we have:

    \begin{align}
        \max_{t \in [t_{i-1},t_i]} \vert \partial_t^{(2)}h(t) \vert &\leq \frac{1}{2}\left(\frac{t_i-t_{i-1}}{2}\right)^2 \max_{t \in [t_{i-1},t_i]} \vert \partial_t^{(4)}h(t) \vert  + \nonumber \\
        &\left( \frac{t_i-t_{i-1}}{2}\right) \max_{t \in \{t_{i-1}, t_i\}} \vert\partial_t^{(3)}h(t) \vert + \max_{t \in \{t_{i-1}, t_i\}} \vert \partial_t^{(2)} h(t) \vert \label{sup-eq:2dbound}
    \end{align}
    
    Combining \ref{sup-eq:2dbound} with \ref{sup-eq:polybound}, we get

    \begin{align*}
        \max_{t \in [t_{i-1,t_i}]}\vert h(t)-p(t) \vert &\leq\frac{1}{4} \max_{t \in [t_{i-1}, t_i]} \vert \partial_t^{(4)}h(t) \vert \left(\frac{t_i-t_{i-1}}{2}\right)^4 + \\
        & \frac{1}{2} \left( \frac{t_i-t_{i-1}}{2}\right)^3 \max_{t \in \{t_{i-1}, t_i\}} \vert\partial_t^{(3)}h(t) \vert + \\
        &\frac{1}{2} \left( \frac{t_i-t_{i-1}}{2}\right)^2 \max_{t \in \{t_{i-1}, t_i\}} \vert \partial_t^{(2)} h(t) \vert 
    \end{align*}

    Again, to extend this result from a function that takes values in $\mathbb{R}$, to one that takes values in $\mathbb{R}^3$, we need to maximize over all dimensions and include a factor of $\sqrt{3}$. After maximizing over all segments, we get, for a piecewise $C^4$ curve $f=\phi \circ c:[a,b] \rightarrow \mathbb{R}^3$, and the zeroth order mapping $g_0$ which linearly interpolates the mapped knots:

    \begin{align*}
        \max_{t \in [a,b]}\vert f(t)-g_0(t) \vert &\leq \frac{\sqrt{3}}{4} \max_{t \in [a, b], j\in \{0,1,2\}} \vert \partial_t^{(4)}f^j(t) \vert \left(\frac{\delta}{2}\right)^4 + \\
        & \frac{\sqrt{3}}{2} \left( \frac{\delta}{2}\right)^2  \max_{i \in \{0...n\}, j\in \{0,1,2\}} \vert\partial_t^{(3)}f^j(t_i) \vert \left( \frac{\delta}{2}\right) + \\
        & \frac{\sqrt{3}}{2} \left( \frac{\delta}{2}\right)^2 \max_{i \in \{0...n\}, j\in \{0,1,2\}} \vert \partial_t^{(2)} f^j(t_i) \vert
    \end{align*}
    Finally, we note that if the mapped curve is $f(t)=(1-t^2, 1-t^2, 1-t^2)^T$ defined on $[-1,1]$ with two knots $t_0=-1,t_1=1$, then the zeroth order mapping is $g_0(t)=(0,0,0)$ and the first order mapping is $g_1(t)=(1-t^2, 1-t^2, 1-t^2)$. The first order mapping bound satisfies the bound from \ref{sup-eq:compare0} of 0, because $f=g_1$. The zeroth order bound is $\sqrt{3}$ which is achieved by the zeroth order mapping at $t=0$.
\end{proof}

% \subsection*{Proposition \ref{sup-prop} as a Tight Bound}

% In the case where $\phi(c_0)=c_0$ and $\phi(c_1)=c_1$, Proposition \ref{sup-prop} says:

% \begin{align*}
%     \max_{t \in [0,1]} \dv c_\phi(t)- \tilde c_\phi(t)\dv &\leq \frac{1}{2} \max_{t \in [0,1]} \dv D\phi(c(t)) - I \dv \dv c(1) - c(0) \dv
% \end{align*}

% So our bound above depends on the distance of the Jacobian of $\phi$ to the identity matrix $I$, which is the Jacobian of the identity diffeomorphism.  Say $c(0)=(1,-1,0)$ and $c(1)=(1,1,0)$ and

% \begin{align*}
%     \phi(x_1,x_2,x_3) = \begin{cases}
%     (x_1,x_2,x_3) & \vert x_2 \vert> 1\\
%     (x_1+x_2 \cdot s(x_2)-1, x_2, x_3) & o.w.
%     \end{cases}
% \end{align*}

% If $s(x)=sgn(x)$ then we have $c_\phi(0)=c(0)$ and $c_\phi(1)=c(1)$ and:

% \begin{align*}
%     \dv D\phi\vert_{c} - I \dv &= 1 \\
%     \dv c(1)-c(0)\dv &= 2
% \end{align*}

% So the right hand side of the bound is 1. However, halfway along this line segment, we have:

% \begin{align*}
%     c(0.5)&=\tilde c_\phi(0.5)=(1,0,0) \\
%     c_\phi(0.5)&=\phi((1,0,0)) =(0,0,0) \\
%     \dv c_\phi(0.5)- \tilde c_\phi(0.5)\dv &= 1
% \end{align*}

% so we have achieved equality. Of course, $s(x)=sgn(x)$ is not differentiable, so the associated $\phi$ is not a diffeomorphism, but we can construct a sequence of smooth functions $s_n(x)$ that approach $sgn(x)$ pointwise, and also approach equality in this bound in the limit, proving the bound is tight.
