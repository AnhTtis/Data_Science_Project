\subsection*{Software Implementation}

In order to implement a first order action method that transforms neuronal curves, we needed to address two questions. The first is how to estimate derivatives in the original discretely sampled curve. The second is, once the knot positions and derivatives are transformed, how can they be used to generate a continuous curve in the new space.

\subsubsection*{One Sided Derivatives from First Order Splines}

The original trace points are assumed to represent the knots in a first order spline, i.e. the points are linearly interpolated. In this representation, derivatives do not necessarily exist at the knots, but one-sided derivatives do and can be easily computed using the difference of consecutive knot positions. Both one-sided derivatives ($\dot c(t_i^-),\dot c(t_i^+)$) are transformed according to Eq. \ref{eq:action} and used to generate the transformed curve.

% Say our curve is decomposed into its three coordinate functions $c(t)=[c_1(t),c_2(t),c_3(t)]^T$, and say all the coordinate functions are discretely sampled at knots $\{t_k\}_{k=1}^K$, where the knots are not necessarily evenly spaced. In order to estimate derivatives of $c_i$ at the knots, we use a central difference method that is adapted to the nonuniform sampling of the function and is second order accurate \cite{sundqvist1970simple}. For all knots except the first and last knot, the approximation of the derivative of a function $c_i$ at knot $t_k$ (denoted $\hat{c_i}'(t_k)$) is given as:

% \begin{align*}
%     \hat{c_i}'(t_k) \propto  c_i(t_{k+1})-\left( \frac{t_{k+1}-t_k}{t_k-t_{k-1}}\right)^2 c_i(t_{k-1})-\left[ 1-\left( \frac{t_{k+1}-t_k}{t_k-t_{k-1}}  \right)^2 \right] c_i(t_k)
% \end{align*}

% The derivative estimates at the end knots are given as:

% \begin{align*}
%     \hat{c_i}'(t_1) &\propto c_i(t_2)-c_i(t_1) &\text{The ``forward'' difference} \\
%     \hat{c_i}'(t_1) &\propto c_i(t_K)-c_i(t_{K-1}) &\text{The ``backward'' difference} \\
% \end{align*}

% Then, all derivatives are scaled so that $\vert \vert \nabla \hat{c}(t_k)\vert \vert=1$ for all $k$.

\subsubsection*{Fitting Curve to Transformed Positions and Derivatives}

For the $i$'th curve segment, the positions $c(t_{i-1}), c(t_i)$ and one sided derivatives $c(t_{i-1}^+), c(t_i^-)$
present four constraints for the interpolating curve. We use these constraints to define a cubic polynomial between each pair of knots, which is known as Hermite interpolation \cite{kincaid2009numerical}. The result is a third order spline. Specifically, we use the scipy implementation of cubic Hermite splines \cite{2020SciPy-NMeth}. It is important to note that this spline is still not necessarily differentiable at the knots.

\subsection*{Quantitatively Comparing Curves}

As described in the Results, the ground truth was considered to be the zeroth order mapping of sampling every $2$ microns along the piecewise linear trace. The splines defined by zeroth and first order mapping were sampled at the same values of the independent variable as the ground truth (every $2$ microns of arc length before the transformation) then compared with ground truth. We used the package from Jekel et al. to compute discrete frechet distance \cite{Jekel2019}. Discrete Frechet distance is an approximation of, and upper bound to Frechet distance \cite{eiter1994computing}. We used nGauge to compute morphometric quantities and SciPy to perform Kolmogorov-Smirnov statistics \cite{walker_ngauge_2022, 2020SciPy-NMeth}.

Further details about our implementation can be found in our open-source Python package brainlit: \\
http://brainlit.neurodata.io/.
