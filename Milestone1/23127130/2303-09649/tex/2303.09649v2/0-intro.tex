% Consider the differentiable curve in 3D space $c:[a,b] \rightarrow \mathbb{R}^3$ and the diffeomorphism $\phi : \mathbb{R}^3 \rightarrow \mathbb{R}^3$. We are interested in how $\phi$ transforms, or acts on, $c$, particularly when $c$ is stored as a discrete sampling of points, which we will henceforth call \textit{knots}.

The brain functions as a network of chemical and electrical activity, so identifying how neurons connect across brain regions is central to understanding how the brain works, and how to treat brain diseases. Modern neuroscience techniques can image single neuron morphology at scale \cite{economo2016platform}, and subsequent neuron tracing can help discover new morphological subtypes \cite{winnubst2019reconstruction}. Due to anatomical variation, and deformations that may have occurred during tissue preparation, neuron traces need to be mapped between coordinate spaces to compare morphologies from different brain samples. Brain registration software often includes neuron mapping implementations, but these implementations have not been thoroughly characterized from a numerical analysis perspective. 

This question is relevant to the ongoing work of the international neuroscience community, including the Brain Initiative Cell Census Network (BICCN), to establish comprehensive neuronal atlases of the mammalian brain \cite{brain2021multimodal}. This effort has produced many images of stained or fluorescently labeled brains, which are being used to generate digital neuron traces for morphological analysis. The traces are commonly stored as a set of connected 3D coordinates, or knots, such as in the SWC format \cite{stockley1993system, cannon1998line}. The connections between the knots are classically represented as cylinders \cite{cannon1998line}, or conical frustums \cite{o2020module}, but here we ignore radius information, since it is not generated by all neuron tracing methods. Consequently, the whole neuron trace is considered to be a tree of piecewise linear curves.

In order to assemble these traces into a complete picture of the various neuron morphologies in the brain, scientists need a way to map neuron traces into common coordinate systems. Several popular software applications exist for this task and are used to assemble atlases of neuron morphology. For example, Peng et al. \cite{peng2021morphological} used mBrainAligner \cite{qu2022cross}, Gao et al. \cite{gao2022single} used the Computational Morphometry Toolkit, and the MouseLight project \cite{winnubst2019reconstruction} used displacement fields from Fedorov et al. \cite{fedorov20123d}. Existing methods use what we call \textit{zeroth order} curve mapping in that they only map the positions of the knots. However, depending on the nonlinearity of the mapping, and the continuous representation of the neuron trace, zeroth order mapping is sensitive to different samplings of the original neuronal curve (Fig. \ref{fig:summary}a,b). In other words, sampling the same curve different ways while tracing in the original image may lead to different mapped morphologies. It is critical that neuron mapping methods preserve the geometry of digital neuron traces in order to build reliable atlases of neuron morphology, and to accurately identify deviations in diseased brains.

In this work, we introduce a method to preserve derivative information when mapping neuronal curves, and investigate the conditions under which this technique is advantageous to existing methods (Figure \ref{fig:summary}). We applied our method to both simulated data and real neuron traces from a whole mouse brain image, and the code used developed in this work is freely available in our Python package brainlit.


\begin{figure}[ht]
\centering
\includegraphics[width=\textwidth]{images/summary.jpg}
\caption{Neglecting the action of a nonlinear mapping on a curve's derivatives can introduce errors. \textbf{a-b} Different samplings of a curve can lead to different results under nonlinear deformations, such as only sampling the endpoints (\textbf{a}) versus sampling several times along the curve (\textbf{b}). \textbf{c-d} Large distances between control points can contribute to mapping inaccuracies. The green line segment following cortical layers 2/3 in a synthetic mouse brain image (\textbf{c}) is defined only by its endpoints. Transforming only the positions of the endpoints (zeroth order mapping, \textbf{d}), is less accurate than incorporating the action on the derivatives as well (first order mapping, \textbf{d}). \textbf{e-f} Quantitative descriptions of the mapping from target to atlas via the logarithm of the Jacobian determinant, which quantifies expansion and compression (\textbf{e}), and the spectral norm of the displacement field, which plays a role in an error bound of zeroth order mapping (\textbf{f}).}
\label{fig:summary}
\end{figure}