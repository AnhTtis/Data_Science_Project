We implemented a first order discrete mapping method in our our open-source
Python package brainlit. In accordance with original SWC formulation \cite{stockley1993system, cannon1998line}, we compute one-sided derivatives at the knots of the curve from first order splines. Then, once the knot positions and derivatives are transformed, we generate a continuous curve in the new space using Hermite interpolation. Further details of our implementation can be found in the Methods.

Figure \ref{fig:toy} shows examples of our method on simulated data, compared to the zeroth order method, and the ``ground truth'' where we map a dense sampling of points along the first order spline of the original curve.

\begin{figure}[ht]
\centering
\includegraphics[width=\textwidth]{images/fig1-v3.jpg}
\caption{Preserving derivative information can mitigate errors when transforming discretized curves. \textbf{a-b} Applying a nonlinear deformation field to a single line segment (\textbf{a}) using zeroth and first order mapping (\textbf{b}). \textbf{c-d} Applying a nonlinear deformation field to a piecewise linear curve (\textbf{c}) using zeroth and first order mapping (\textbf{d}). Zeroth and first order discrete mapping methods are shown relative to ground truth considered to be the application of the vector field to a dense sampling of the original curves.}
\label{fig:toy}
\end{figure}