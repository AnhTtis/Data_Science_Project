\def\sigone{80}
\def\sigtwo{160}
\def\sigthree{320}
\def\sigfour{640}

We applied our method to $20$ reconstructed neurons in SWC format from a whole mouse brain image from the Janelia MouseLight project \cite{winnubst2019reconstruction}. We selected the first $20$ SWC files that successfully downloaded from MouseLight's NeuronBrowser repository and did not have repeat trace nodes. Neurons have a tree-like  structure, and we split them into non-branching curves in order to apply our mapping methods. We follow a method introduced previously \cite{athey2021spline} where the root to leaf path with the longest arc length is recursively removed until the tree is reduced to non-bifurcating ``branches''. 

We generate random transformations using the Large Deformation Diffeomorphic Metric Mapping (LDDMM) framework described in Miller et al. and applied in Tward and Miller \cite{miller_geodesic_2006, tward_complexity_2017}. We generate an initial momentum field by sampling Gaussian noise with zero mean and varying standard deviation, $\sigma$. The momentum is smoothed to construct a velocity field, and integrated in time according to the conservation laws established in Miller et al. to generate a diffeomorphic transformation \cite{miller_geodesic_2006}. We generated four diffeomorphisms with $\sigma$ levels of $\sigone$, $\sigtwo$, $\sigthree$ and $\sigfour$ $\mu m/\text{time}$. The position and tangent displacement profiles of these four diffeomorphisms are shown in Figure \ref{fig:results}a. We centered the neuron traces at the origin then applied the random diffeomorphisms to compare zeroth and first order mapping to ground truth (Fig. \ref{fig:results}b-g). Ground truth was generated by upsampling the original traces to a maximum node spacing of $2 \mu m$ followed by zeroth order mapping.

\begin{figure}[ht]
\centering
\includegraphics[width=\textwidth]{images/combined.png}
\caption{Application of zeroth and first order mapping of neuron traces under diffeomorphisms derived from random Gaussian initial momenta. \textbf{a} Different values of $\sigma$ produced diffeomorphisms with different position and tangent vector displacement profiles. The positions and tangent vectors sampled in the histogram were distributed as a uniform grid with a spacing of $500 \mu m$. \textbf{b-g} Two examples of the diffeomorphism with $\sigma=\sigfour$ applied to neuron traces to produce zeroth and first order mappings, along with ground truth. Both examples show the original trace and the transformation \textbf{(b,e)}, the results of the different transformation methods \textbf{(c,f)}, and a zoomed in view of the region outlined by the dotted line to show discrepancies between the methods. Plot axes are in units of microns.}
\label{fig:results}
\end{figure}

For each neuron trace, we computed the discrete frechet error from ground truth (Fig. \ref{fig:stats}a). We also wanted to measure which mapping method better matched the ground truth with respect to a neuron's distribution of common morphometric quantities, such as path angle, branch angle, tortuosity, and segment length. We used the Kolmogorov-Smirnov test statistic to measure how much the distribution of these quantities differed from ground truth (Fig. \ref{fig:stats}b). We performed two-sided Wilcoxon signed-rank tests for each comparison and used a Bonferroni correction across the different $\sigma$ values (Fig. \ref{fig:stats}b). Lastly, we compared the discrete frechet errors the average sampling period of the trace i.e. the average distance between trace nodes (Figure \ref{fig:stats}c).


\begin{figure}[ht]
\centering
\includegraphics[width=\textwidth]{images/stats.png}
\caption{Comparison of zeroth and first order mapping of neuron traces under random diffeomorphisms. \textbf{a} Discrete Frechet error was computed between the different order mappings, and ground truth. \textbf{b} Distributions of common morphometric quantities were compared to that of ground truth using the Kolmogorov-Smirnov test statistic. Differences between zeroth and first order methods were tested using Wilcoxon signed-rank test with Bonferroni correction across different values of $\sigma$ ($\ast: p \leq 0.05, \ast \ast: p \leq 0.01, \ast \ast \ast: p \leq 0.001, \ast\ast\ast\ast\: p\leq 0.0001$). Box plots show median, upper and lower quartiles and whiskers have a maximum length of 1.5x the interquartile range with other outlier data marked with points. \textbf{c} Relationship between discrete frechet error and average sampling period (distance between trace points) under the random diffeomorphisms.}
\label{fig:stats}
\end{figure}

To explore the effect of downsampling neuron traces on mapped morphologies, we identified non-branching nodes in straight portions of the trace, and measured the impact of removing those nodes from the trace. Specifically, we performed first order mapping on the segment with the node removed, and compared it to the ground truth mapping of the original segment. We determined which fraction of nodes maintained a discrete frechet error less than one micron, serving as an estimate for the fraction of nodes which are not necessary to maintained the mapped morphology (Fig. \ref{fig:removal}).

\begin{figure}[ht]
\centering
\includegraphics[width=\textwidth]{images/removal.png}
\caption{Counting how many nodes in MouseLight neuron traces can be removed without affecting the mapped morphology. \textbf{a} For each non-branching node with path angle above 170 degrees, we generated a line segment with that node removed. \textbf{b} We performed first order mapping on the downsampled line segment and compared the result with the ground truth mapping of the original curve. \textbf{c} For each neuron trace, we determined the fraction of nodes where the discrete frechet error is less than or equal to one micron under the four random diffeomorphisms. Box plots show median, upper and lower quartiles and whiskers have a maximum length of 1.5x the interquartile range with other outlier data marked with points.}
\label{fig:removal}
\end{figure}