In this section we settle the complexity of the satisfiability problem for HyperLTL:
given a \hyltl sentence, determine whether it has a model.

\begin{thm}\label{thm:hyltl-sat}
  \hyltl satisfiability is $\Sigma^1_1$-complete.
\end{thm}

We should contrast this result with the $\Sigma^0_1$-completeness of
\hyltl satisfiability restricted to
\emph{finite} sets of ultimately periodic traces~\cite[Theorem 1]{FHH18}. The $\Sigma^1_1$-completeness
of \hyltl satisfiability in the general case implies that, in particular, the set of
satisfiable \hyltl sentences is neither recursively enumerable nor co-recursively
enumerable. A semi-decision procedure, like the one introduced in \cite{FHH18}
for finite sets of ultimately periodic traces, therefore cannot exist in general.


\subsection{HyperLTL satisfiability is in $\Sigma_1^1$}
The $\Sigma^1_1$ upper bound relies on the fact that every satisfiable \hyltl
formula has a countable model~\cite{FZ17}. 
This allows us to represent these models, and Skolem functions on them, by sets of natural numbers, which are 
type~$1$ objects. In this encoding, trace assignments are type~$0$ objects, as traces in a countable set can be identified by natural numbers. With some more existential type~$1$ quantification one can then express the existence of a function witnessing that every trace assignment consistent with the Skolem functions satisfies the quantifier-free part of the formula under consideration. 

\begin{lem}
\label{lemma:hyltl_membership}
\hyltl satisfiability is in $\Sigma_1^1$.	
\end{lem}

\begin{proof}
Let $\varphi$ be a \hyltl formula, let $\Phi$ denote the set of quantifier-free subformulas of~$\varphi$, and let $\Pi$ be a trace assignment whose domain contains the variables of $\varphi$. The expansion of $\varphi$ on $\Pi$ is the function~$\expansion_{\varphi, \Pi} \colon \Phi \times \nats \rightarrow \set{0,1}$ with
\[
\expansion_{\varphi, \Pi}(\psi, j)= \begin{cases}
1 &\text{if $\suffix{\Pi}{j} \models \psi$, and}\\
0 &\text{otherwise.}
\end{cases} 
\]
The expansion is completely characterised by the following consistency conditions:
\begin{itemize}
	\item $\expansion_{\varphi, \Pi} (a_\pi,j)=1$ if and only if $a \in \Pi(\pi)(j)$.
	\item $\expansion_{\varphi, \Pi} (\neg \psi,j)=1$ if and only if  $\expansion_{\varphi, \Pi} (\psi,j)=0$.
	\item $\expansion_{\varphi, \Pi} (\psi_1 \vee \psi_2,j)=1$ if and only if $\expansion_{\varphi, \Pi} (\psi_1,j)=1$ or $\expansion_{\varphi, \Pi} (\psi_2,j)=1$.
	\item $\expansion_{\varphi, \Pi} (\X\psi,j)=1$ if and only if $\expansion_{\varphi, \Pi} (\psi,j+1)=1$.
	\item $\expansion_{\varphi, \Pi} (\psi_1 \U \psi_2,j)=1$ if and only if there is a $j' \ge j$ such that $\expansion_{\varphi, \Pi} (\psi_2,j')=1$ and $\expansion_{\varphi, \Pi} (\psi_2,j'')=1$ for all $j''$ in the range~$j \le j'' < j'$.
\end{itemize}

Every satisfiable \hyltl sentence has a countable model~\cite{FZ17}.
Hence, to prove that the \hyltl satisfiability problem is in $\Sigma_1^1$, we express, for a given \hyltl sentence encoded as a natural number, the existence of the following type~$1$ objects (relying on the fact that there is a bijection between finite sequences over $\nats$ and $\nats$ itself):
\begin{itemize}

	\item A countable set of traces over the propositions of $\varphi$ encoded as a function~$T$ from~$\nats \times \nats$ to $\nats$, mapping trace names and positions to (encodings of) subsets of the set of propositions appearing in $\varphi$.
	
	\item A function~$S$ from $\nats \times \nats^*$ to $\nats$ to be interpreted as Skolem functions for the existentially quantified variables of $\varphi$, i.e.\ we map a variable (identified by a natural number) and a trace assignment of the variables preceding it (encoded as a sequence of natural numbers) to a trace name.

	\item A function~$E$ from $\nats \times \nats \times \nats$ to $\nats$, where, for a fixed~$a \in\nats$ encoding a trace assignment~$\Pi$, the function~$x,y\mapsto E(a, x, y)$ is interpreted as the expansion of $\varphi$ on $\Pi$, i.e.\ $x$ encodes a subformula in $\Phi$ and $y$ is a position.

\end{itemize}
Then, we express the following properties using only type~$0$ quantification: For every trace assignment of the variables in $\varphi$, encoded by $a \in \nats$, if $a$ is consistent with the Skolem function encoded by $S$, then the function~$x,y\mapsto E(a, x, y)$ satisfies the consistency conditions characterising the expansion, and we have $E(a,x_0, 0) = 1$, where $x_0$ is the encoding of the maximal quantifier-free subformula of $\varphi$.
 We leave the tedious, but standard, details to the industrious reader. 
\end{proof}

\subsection{HyperLTL satisfiability is $\Sigma_1^1$-hard} To prove a matching lower bound, we reduce from the recurrent tiling problem~\cite{Harel85}, a standard $\Sigma_1^1$-complete problem.

\begin{lem}
\label{lemma:hyltl_hardness}
\hyltl satisfiability is $\Sigma_1^1$-hard.	
\end{lem}
\begin{proof}
A \emph{tile} is a function $\tau\colon\{\east,\west,\north,\south\}\to \mathcal{C}$
that maps directions into a finite set $\mathcal{C}$ of colours.
Given a finite set $\tileset$ of tiles, a \emph{tiling of the positive quadrant}
with $\tileset$ is a function $\tiling\colon\mathbb{N}\times\mathbb{N}\to \tileset$ with the property that:
\begin{itemize}
	\item if $\tiling(i,j)=\tau_1$ and $\tiling(i+1,j)=\tau_2$, then $\tau_1(\east)=\tau_2(\west)$ and
	\item if $\tiling(i,j)=\tau_1$ and $\tiling(i,j+1)=\tau_2$ then $\tau_1(\north)=\tau_2(\south)$.
\end{itemize}
The \emph{recurring tiling problem} is to determine, given a finite set $\tileset$ of tiles and a designated $\tau_0\in \tileset$, whether there is a tiling~$\tiling$ of the positive quadrant with $\tileset$ such that there are infinitely many $j\in\mathbb{N}$ such that $\tiling(0,j)=\tau_0$. 
%
This problem is known to be $\Sigma_1^1$-complete~\cite{Harel85}, so reducing it to \hyltl 
satisfiability will establish the desired hardness result. 

In our reduction, each $x$-coordinate in the positive quadrant will be represented by a trace, and each $y$-coordinate by a point in time.\footnote{Note that this means that if we were to visually represent this construction, traces would be arranged vertically.} In order to keep track of which trace represents which $x$-coordinate, we use one designated atomic proposition $x$ that holds on exactly one time point in each trace: $x$ holds at time $i$ if and only if the trace represents $x$-coordinate $i$.

For this purpose, let $\tileset$ and $\tau_0$ be given, and define the following formulas over $\ap = \set{ x } \cup \tileset$:

\begin{itemize}
  \item Every trace has exactly one point where $x$ holds:
  \[\varphi_1 = \forall \pi.\ (\neg x_\pi \U (x_\pi\wedge \X\G\neg x_\pi))\]
  \item   For every $i\in\mathbb{N}$, there is a trace with $x$ in the $i$-th position:
  \[\phi_2 = (\exists \pi.\ x_\pi) \wedge (\forall \pi_1.\ \exists \pi_2.\ \F(x_{\pi_1}\wedge \X x_{\pi_2}))\]
  \item
  If two traces represent the same $x$-coordinate, then they contain the same tiles:
  \[\varphi_3 = \forall \pi_1,\pi_2.\ (\F(x_{\pi_1}\wedge x_{\pi_2})\rightarrow \G(\bigwedge_{\tau\in \tileset}(\tau_{\pi_1}\leftrightarrow \tau_{\pi_2})))\]
  \item Every time point in every trace contains exactly one tile:
  \[\varphi_4=\forall \pi.\ \G\bigvee_{\tau\in \tileset}(\tau_\pi\wedge \bigwedge_{\tau'\in \tileset\setminus \{\tau\}}\neg (\tau')_\pi)\]
  \item Tiles match vertically:
  \[\varphi_5=\forall \pi.\ \G\bigvee_{\tau\in \tileset}(\tau_\pi\wedge \bigvee\nolimits_{\tau'\in\{\tau'\in \tileset\mid \tau(\north)=\tau'(\south)\}}\X (\tau')_\pi)\]
  \item Tiles match horizontally:
  \[\varphi_6=\forall \pi_1,\pi_2.\ (\F(x_{\pi_1}\wedge \X x_{\pi_2})\rightarrow \G\bigvee_{\tau\in \tileset}(\tau_{\pi_1}\wedge \bigvee\nolimits_{\tau'\in \{\tau'\in \tileset\mid \tau(\east)=\tau'(\west)\}}(\tau')_{\pi_2}))\]
  \item Tile~$\tau_0$ occurs infinitely often at $x$-position $0$:
  \[\varphi_7=\exists \pi.\ (x_\pi \wedge \G\F \tau_0)\]
\end{itemize}

Finally, take $\varphi_{\tileset} = \bigwedge_{1\leq n \leq 7}\varphi_n$. Technically $\varphi_{\tileset}$ is not a \hyltl formula, since it is not in prenex normal form, but it can be trivially transformed into one. Collectively, subformulas $\varphi_1$--$\varphi_3$ are satisfied in exactly those sets of traces that can be interpreted as $\mathbb{N}\times\mathbb{N}$. Subformulas $\varphi_4$--$\varphi_6$ then hold if and only if the $\mathbb{N}\times\mathbb{N}$ grid is correctly tiled with $\tileset$. Subformula $\varphi_7$, finally, holds if and only if the tiling uses the tile $\tau_0$ infinitely often at $x$-coordinate $0$. Overall, this means $\varphi_{\tileset}$ is satisfiable if and only if $\tileset$ can recurrently tile the positive quadrant.

The $\Sigma^1_1$-hardness of \hyltl satisfiability therefore follows from the $\Sigma_1^1$-hardness of the recurring tiling problem~\cite{Harel85}.
\end{proof}

The $\Sigma^1_1$-completeness of \hyltl satisfiability still holds if we restrict to ultimately periodic traces.

\begin{thm}
\label{thm:hyltlsatup-copmleteness}
  \hyltl satisfiability restricted to sets of ultimately periodic traces is $\Sigma^1_1$-complete.
\end{thm}

\begin{proof}
The problem of whether there is a tiling of $\{(i,j)\in \mathbb{N}^2\mid i\geq j\}$, i.e.\ the part of $\mathbb{N}\times\mathbb{N}$ below the diagonal, such that a designated tile $\tau_0$ occurs on every row, is also $\Sigma_1^1$-complete~\cite{Harel85}.\footnote{The proof in \cite{Harel85} is for the part \emph{above} the diagonal with $\tau_0$ occurring on every column, but that is easily seen to be equivalent.} We reduce this problem to \hyltl satisfiability on ultimately periodic traces.

The reduction is very similar to the one discussed above, with the necessary changes being: (i) every time point beyond $x$ satisfies the special tile ``null'', (ii) horizontal and vertical matching are only checked at or before time point $x$ and (iii) for every trace~$\pi_1$ there is a trace~$\pi_2$ such that $\pi_2$ has designated tile~$\tau_0$ at the time where $\pi_1$ satisfies $x$ (so $\tau_0$ holds at least once in every row).

Membership in $\Sigma_1^1$ can be shown similarly to the proof of \autoref{lemma:hyltl_membership}. So, the problem is $\Sigma^1_1$-complete.
\end{proof}

Furthermore, a careful analysis of the proof of Theorem~\ref{thm:hyltlsatup-copmleteness} shows that we can restrict ourselves to ultimately periodic traces of the form~$x\cdot\emptyset^\omega$, i.e.\ to essentially finite traces.