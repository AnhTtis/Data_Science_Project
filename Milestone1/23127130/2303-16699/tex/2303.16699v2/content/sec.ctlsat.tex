Here, we consider the \hyctlstar satisfiability problem: given a \hyltl sentence, determine whether it has a model $\tsys$ (of arbitrary size). We prove that it is much harder than \hyltl satisfiability.
As a key step of the proof, which is interesting in its own right, we also prove that every satisfiable sentence
admits a model of cardinality at most $\cont$ (the cardinality of the continuum). Conversely, we exhibit
a satisfiable \hyctlstar sentence whose models are all of cardinality at least~$\cont$.

\begin{thm}\label{thm:hyctl-sat}
  \hyctlstar satisfiability is $\Sigma^2_1$-complete.
\end{thm}

\subsection{An Upper Bound on the Size of HyperCTL$^*$ Models}
Before we begin proving membership in $\Sigma^2_1$, we obtain a bound on the size of minimal models of satisfiable \hyctlstar sentences. For this, we use an argument based on Skolem functions, which is a transfinite generalisation of the proof that all satisfiable \hyltl sentences have a countable model~\cite{FZ17}.
Later, we complement this upper bound by a matching lower bound, which will be applied in the hardness proof.

In the following, we use $\omega$ and $\omega_1$ to denote the first infinite and the first uncountable ordinal, respectively, and write $\aleph_0$ and $\aleph_1$ for their cardinality. 
As $\cont$ is uncountable, we have $\aleph_1 \le \cont$.

\begin{lem}\label{lemma:hyctl-model}
Each satisfiable HyperCTL$^*$ sentence~$\phi$ has a model of size at most~$\cont$.
\end{lem}
% \begin{proof}[Proof sketch]
% Suppose $\phi$ has a model $\tsys$ of arbitrary size, and fix Skolem functions witnessing this satisfaction. We then create a transfinite sequence of transition systems~$\tsys_\alpha$. We start by taking $\tsys_0$ to be any single path from $\tsys$ starting in the initial vertex, and obtain $\tsys_{\alpha+1}$ by adding to $\tsys_\alpha$ all vertices and edges of the paths that are the outputs of the Skolem functions when restricted to inputs from $\tsys_\alpha$. If $\alpha$ is a limit ordinal we take $\tsys_\alpha$ to be the union of all previous transition systems.

% This sequence does not necessarily stabilise at $\omega$, since $\tsys_{\omega}$ may contain a path~$\rho$ such that
% $\rho(i)$ was introduced in $\tsys_i$. This would result in $\tsys_{\omega}$ containing a path that was not present in any earlier model $\tsys_i$ with $i<\omega$, and therefore we could have $\tsys_{\omega+1}\not = \tsys_{\omega}$.

% The sequence does stabilise at $\omega_1$, however. This is because every path $\rho$ contains only countably many vertices, so if every element $\rho(i)$ of $\rho$ is introduced at some countable $\alpha_i$, then there is a countable $\alpha$ such that all of $\rho$ is included in $\tsys_\alpha$. It follows that $\tsys_{\omega_1}$ does not contain any ``new'' paths that were not already in some $\tsys_\alpha$ with $\alpha<\omega_1$, and therefore the Skolem function $f$ does not generate any ``new'' outputs either.

% In each step of the construction at most $\cont$ new vertices are added, so $\tsys_{\omega_1}$ contains at most $\cont$ vertices. Furthermore, because $\tsys_{\omega_1}$ is closed under the Skolem functions, the satisfaction of $\phi$ in $\tsys$ implies its satisfaction in $\tsys_{\omega_1}$.
% \end{proof}

  The proof of \autoref{lemma:hyctl-model} uses a Skolem function to create a model. Before giving this proof, we should therefore first introduce Skolem functions for \hyctlstar.

Let $\phi$ be a \hyctlstar formula. A quantifier in $\phi$ occurs \emph{with polarity 0} if it occurs inside the scope of an even number of negations, and \emph{with polarity 1} if it occurs inside the scope of an odd number of negations. We then say that a quantifier \emph{occurs existentially} if it is an existential quantifier with polarity 0, or a universal quantifier with polarity 1. Otherwise the quantifier \emph{occurs universally}. A Skolem function will map choices for the universally occurring quantifiers to choices for the existentially occurring quantifiers.

For reasons of ease of notation, it is convenient to consider a single Skolem function for all existentially occurring quantifiers in a \hyctlstar formula $\phi$, so the output of the function is an $l$-tuple of paths, where $l$ is the number of existentially occurring quantifiers in~$\phi$. The input consists of a $k$-tuple of paths, where $k$ is the number of universally occurring quantifiers in $\phi$, plus an $l$-tuple of integers. The reason for these integers is that we need to keep track of the time point in which the existentially occurring quantifiers are invoked.

Consider, for example, a \hyctlstar formula of the form $\forall \pi_1.\ \G \exists \pi_2.\ \psi$. This formula states that for every path $\pi_1$, and for every future point $\pi_1(i)$ on that path, there is some $\pi_2$ starting in $\pi_1(i)$ satisfying $\psi$. So the choice of $\pi_2$ depends not only on $\pi_1$, but also on $i$. For each existentially occurring quantifier, we need one integer to represent this time point at which it is invoked. A \hyctlstar Skolem function for a formula $\phi$ on a transition system~$\tsys$ is therefore a function~$f$ of the form~$f\colon\mathit{paths}(\tsys)^k\times \mathbb{N}^l\rightarrow \mathit{paths}(\tsys)^l$, where $\mathit{paths}(\tsys)$ is the set of paths over $\tsys$, $k$ is the number of universally occurring quantifiers in $\phi$ and $l$ is the number of existentially occurring quantifiers. 
Note that not every function of this form is a Skolem function, but for our upper bound it suffices that every Skolem function is of that form.

Now, we are able to prove that every satisfiable \hyctlstar formula has a  model of size~$\cont$.

\begin{proof}[Proof of \autoref{lemma:hyctl-model}]
If $\phi$ is satisfiable, let $\tsys$ be one of its models, and let $f$ be a Skolem function witnessing the satisfaction of $\phi$ on $\tsys$. We create a sequence of transition systems $\tsys_\alpha$ as follows.
\begin{itemize}
	\item $\tsys_0$ is a single, arbitrarily chosen, path of $\tsys$ starting in the initial vertex.
	\item $\tsys_{\alpha+1}$ contains exactly those vertices and edges from $\tsys$ that are (i) part of $\tsys_{\alpha}$ or (ii) among the outputs of the Skolem function $f$ when restricted to input paths from $\tsys_\alpha$.
	\item if $\alpha$ is a limit ordinal, then $\tsys_\alpha= \bigcup_{\alpha'<\alpha} \tsys_{\alpha'}$.
\end{itemize}
Note that if $\alpha$ is a limit ordinal then $\tsys_\alpha$ may contain paths~$\rho(0)\rho(1)\rho(2)\cdots$ that are not included in any $\tsys_{\alpha'}$ with $\alpha' <\alpha$, as long as each finite prefix $\rho(0) \cdots \rho(i)$ is included in some $\alpha'_i<\alpha$.

First, we show that this procedure reaches a fixed point at $\alpha = \omega_1$. Suppose towards a contradiction that $\tsys_{\omega_1+1}\not =\tsys_{\omega_1}$. Then there are $\vec{\rho}=(\rho_1,\ldots,\rho_k)\in \mathit{paths}(\tsys_{\omega_1})^k$ and $\vec{n}\in \mathbb{N}^l$ such that $f(\vec{\rho},\vec{n})\not \in \mathit{paths}(\tsys_{\omega_1})^l$. Then for every $i\in\mathbb{N}$ and every $1\leq j \leq k$, there is an ordinal $\alpha_{i,j}<\omega_1$ such that the finite prefix $\rho_j(0)\cdots \rho_j(i)$ is contained in $\tsys_{\alpha_{i,j}}$. The set $\{\alpha_{i,j}\mid i\in \mathbb{N}, 1\leq j \leq k\}$ is countable, and because $\alpha_{i,j}<\omega_1$ each $\alpha_{i,j}$ is also countable. A countable union of countable sets is itself countable, so $\sup \{\alpha_{i,j}\mid i\in \mathbb{N}, 1 \leq j \leq k\}=\bigcup_{i\in\mathbb{N}}\bigcup_{1\leq j \leq k}\alpha_{i,j}=\beta< \omega_1$. 

But then the $\vec{\rho}$ are all contained in $\tsys_\beta$, and therefore $f(\vec{\rho},\vec{n})\in \mathit{paths}(\tsys_{\beta+1})^l$. But $\beta+1< \omega_1$, so this contradicts the assumption that $f(\vec{\rho},\vec{n})\not \in \mathit{paths}(\tsys_{\omega_1})^l$. From this contradiction we obtain $\tsys_{\omega_1+1}=\tsys_{\omega_1}$, so we have reached a fixed point. Furthermore, because $\tsys_{\omega_1}$ is contained in $\tsys$ and closed under the Skolem function and $\tsys$ satisfies $\phi$, we obtain that $\tsys_{\omega_1}$ also satisfies $\phi$.

Left to do, then, is to bound the size of $\tsys_{\omega_1}$, by bounding the number of vertices that get added at each step in its construction. We show by induction that $\size{\tsys_\alpha}\leq \cont$ for every $\alpha$. So, in particular, we have $\tsys_{\omega_1} \le \cont$, as required.

As base case, we have $\size{\tsys_{0}}
\le \aleph_0 < \cont$, since it consists of a single path. 
Consider then $\size{\tsys_{\alpha+1}}$. For each possible input to $f$, there are at most $l$ new paths, and therefore at most $\size{\nats \times l}$ new vertices in $\tsys_{\alpha+1}$. Further, there are $\size{\mathit{paths}(\tsys_\alpha)}^k\times \size{\mathbb{N}}^l$ such inputs. By the induction hypothesis, $\size{\tsys_\alpha}\leq \cont$, which implies that $\size{\mathit{paths}(\tsys_\alpha)}\leq \cont$. As such, the number of added vertices in each step is limited to $\cont^k\times \aleph_0^l \times \aleph_0\times l= \cont$. So $\size{\tsys_{\alpha+1}}\leq \size{\tsys_\alpha}+\cont=\cont$.

If $\alpha$ is a limit ordinal, $\tsys_\alpha$ is a union of at most $\aleph_1$ sets, each of which has, by the induction hypothesis, a size of at most $\cont$. Hence $\size{\tsys_\alpha}\leq \aleph_1\times \cont = \cont$.
\end{proof}


\subsection{HyperCTL$^*$ satisfiability is in $\Sigma^2_1$}

With the upper bound on the size of models at hand, we can place \hyctlstar satisfiability in $\Sigma_1^2$, as the existence of a model of size~$\cont$ can be captured by quantification over type~$2$ objects.

\begin{lem}\label{lemma:hyctl-membership}
\hyctlstar satisfiability is in $\Sigma_1^2$.
\end{lem}
\begin{proof}
As every \hyctlstar formula is satisfied in a model of size at most $\cont$, these models can be represented by objects of type~$2$. Checking whether a formula is satisfied in a transition system is equivalent to the existence of a winning strategy for Verifier in the induced model checking game. Such a strategy is again a type~$2$ object, which is existentially quantified. Finally, whether it is winning can be expressed by quantification over individual elements and paths, which are objects of types~$0$ and $1$.
Checking the satisfiability of a \hyctlstar formula $\phi$ therefore amounts to existential third-order quantification (to choose a model and a strategy) followed by a second-order formula to verify that $\phi$ holds on the model (i.e.\ that the chosen strategy is winning). Hence \hyctlstar satisfiability is in $\Sigma^2_1$.


Formally, we encode the existence of a winning strategy for Verifier in the \hyctlstar model checking game~$\mcgame(\tsys, \varphi)$ induced by a transition system~$\tsys$ and a \hyctlstar sentence~$\varphi$. 
This game is played between Verifier and Falsifier, one of them aiming to prove that $\tsys \models \varphi$ and the other aiming to prove $\tsys \not\models \varphi$. It is played in a graph whose positions correspond to subformulas which they want to check (and suitable path assignments of the free variables): each vertex (say, representing a subformula $\psi$) belongs to one of the players who has to pick a successor, which represents a subformula of $\psi$. A play ends at an atomic proposition, at which point the winner can be determined.

Formally, a vertex of the game is of the form~$(\Pi, \psi,b)$ where $\Pi$ is a path assignment, $\psi$ is a subformula of $\varphi$, and $b \in \set{0,1}$ is a flag used to count the number of negations encountered along the play; the initial vertex is $(\Pi_\emptyset, \varphi,0)$.
Furthermore, for until-subformulas~$\psi$, we need auxiliary vertices of the form~$(\Pi,\psi, b, j)$ with $j \in \nats$.
The vertices of Verifier are 
\begin{itemize}
	\item of the form $(\Pi, \psi, 0)$ with $\psi = \psi_1 \vee \psi_2$, $\psi = \psi_1 \U \psi_2$, or $\psi = \exists \pi.\ \psi'$, 
	\item of the form $(\Pi, \forall \pi.\ \psi', 1)$, or
	\item of the form $(\Pi, \psi_1 \U \psi_2, 1, j)$.
\end{itemize}
The moves of the game are defined as follows:
\begin{itemize}
	
	\item A vertex~$(\Pi, a_\pi, b)$ is terminal. It is winning for Verifier if $b = 0$ and $a \in \lambda(\Pi(\pi)(0))$ or if $b=1$ and $a \notin \lambda(\Pi(\pi)(0))$, where $\lambda$ is the labelling function of $\tsys$.
	
	\item A vertex~$(\Pi, \neg \psi, b)$ has a unique successor~$(\Pi,\psi, b+1 \bmod 2)$.
	
	\item A vertex~$(\Pi, \psi_1\vee \psi_2, b)$ has two successors of the form~$(\Pi, \psi_i, b)$ for $i \in \set{1,2}$.
	
	\item A vertex~$(\Pi, \X \psi, b)$ has a unique successor~$(\suffix{\Pi}{1}, \psi, b)$.
	
	\item A vertex~$(\Pi, \psi_1\U \psi_2, b)$ has a successor~$(\Pi, \psi_1\U \psi_2, b,j)$ for every $j \in \nats$.
	
	\item A vertex~$(\Pi, \psi_1\U \psi_2, b,j)$ has the  successor~$(\suffix{\Pi}{j}, \psi_2, b)$ as well as successors $(\suffix{\Pi}{j'}, \psi_1, b)$ for every $0 \le j' <j$.
	
	\item A vertex~$(\Pi, \exists \pi.\ \psi, b)$ has successors~$(\Pi[\pi\mapsto \rho],\psi, b)$ for every path~$\rho$ of $\tsys$ starting in $\last(\Pi)$.
	
	\item A vertex~$(\Pi, \forall \pi.\ \psi, b)$ has successors~$(\Pi[\pi\mapsto \rho],\psi, b)$ for every path~$\rho$ of $\tsys$ starting in $\last(\Pi)$.
\end{itemize}

A play of the model checking game is a finite path through the graph, starting at the initial vertex and ending at a terminal vertex. It is winning for Verifier if the terminal vertex is winning for her. 
Note that the length of a play is bounded by~$2d$, where $d$ is the depth\footnote{The depth is the maximal nesting of quantifiers, Boolean connectives, and temporal operators.} of $\varphi$, as the formula is simplified during at least every other move.

A strategy~$\sigma$ for Verifier is a function mapping each of her vertices~$v$  to some successor of~$v$. 
A play~$v_0 \cdots v_k$ is consistent with $\sigma$, if $v_{k'+1} = \sigma(v_{k'})$ for every $0 \le k' < k$ such that $v_{k'}$ is a vertex of Verifier.
A straightforward induction shows that Verifier has a winning strategy for $\mcgame(\tsys,\varphi)$ if and only if $\tsys\models\varphi$.
 
Recall that every satisfiable \hyctlstar sentence has a model of cardinality~$\contcard$~(\autoref{lemma:hyctl-model}). 
Thus, to place \hyctlstar satisfiability in $\Sigma_1^2$, we express, for a given natural number encoding a \hyctlstar formula~$\varphi$, the existence of the following type~$2$ objects (using suitable encodings):
\begin{itemize}
	\item A transition system~$\tsys$ of cardinality~$\contcard$. 
	\item A function~$\sigma$ from $V$ to $V$, where $V$ is the set of vertices of $\mcgame(\tsys,\varphi)$. Note that a single vertex of $V$ is a type~$1$ object. 
\end{itemize}
Then, we express that $\sigma$ is a strategy for Verifier, which is easily expressible using quantification over type~$1$ objects. 
Thus, it remains to express that $\sigma$ is winning by stating that every play (a sequence of type~$1$ objects of bounded length) that is consistent with $\sigma$ ends in a terminal vertex that is winning for Verifier. 
Again, we leave the tedious, but standard, details to the reader.
\end{proof}

	

