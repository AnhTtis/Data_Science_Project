\subsection{HyperCTL$^*$ satisfiability is $\Sigma^2_1$-hard}

Next, we prove a matching lower bound.
We first describe a satisfiable \hyctlstar sentence $\phiset$ that does not have any model of
cardinality less than $\cont$ (more precisely, the initial vertex must have
uncountably many successors), thus matching the upper bound from
Lemma~\ref{lemma:hyctl-model}.
We construct $\phiset$ with one particular model~$\Kset$ in mind,
defined below, though it also has other models.

The idea is that we want all possible subsets of $A \subseteq \Nat$ to
be represented in $\Kset$ in the form of paths
$\rho_A$ such that $\rho_A(i)$ is labelled by $1$ if $i \in A$, and by $0$ otherwise.
By ensuring that the first vertices of these paths are pairwise distinct, we obtain the desired lower bound on the cardinality.
We express this in \hyctlstar as follows: First, we express that there is a part of the model (labelled by $\fbt$) where every reachable vertex has two successors, one labelled with $0$ and one labelled with $1$, i.e.\ the unravelling of this part contains the full binary tree.
Thus, this part has a path~$\rho_A$ 
as above for every subset~$A$, but their initial vertices are not necessarily distinct.
Hence, we also express that there is another part (labelled by $\pset$) that contains a copy of each path in the $\fbt$-part, and that these paths indeed start at distinct successors of the initial vertex.

\begin{figure}
    \centering
\def\dist{2.0cm}
\begin{tikzpicture}[->,
>=stealth', 
level/.style={sibling distance = 3cm/#1, level distance = \dist}, 
scale=0.7,
transform shape,
grow=left,thick]

\node (init) [state] {\phantom{0}}
child {
    node (t0) [state] {0} 
    child {
        node (t00) [state] {0} 
        child {edge from parent[dashed] }
        child {edge from parent[dashed] }
    }
    child {
        node (t01) [state] {1} 
        child {edge from parent[dashed] }
        child {edge from parent[dashed] }
    }
}
child {
    node (t1) [state] {1} 
    child {
        node (t10) [state] {0} 
        child {edge from parent[dashed] }
        child {edge from parent[dashed] }
    }
    child {
        node (t11) [state] {1} 
        child {edge from parent[dashed] }
        child {edge from parent[dashed] }
    }
}
;

{\color{red}
\node[state,right=6cm of t00] (s00) {0}; 
\node[state,right=1cm of s00] (s000) {0}; 
\node[state,right=1cm of s000] (s0000) {0}; 
\draw (init) -- (s00);
\draw (s00) -- (s000);
\draw (s000) -- (s0000);
\draw[dotted] (s0000) -- +(\dist,0);

\draw[dotted] (init) -- (2.5,1.33);
\draw[-,dotted] (2.5,1.33) -- +(5,0);
\draw[dotted] (init) -- (2.5,1.66);
\draw[-,dotted] (2.5,1.66) -- +(5,0);

\draw[dotted] (init) -- (2.5,0.25);
\draw[-,dotted] (2.5,0.25) -- +(5,0);

\draw[dotted] (init) -- (2.5,-0.25);
\draw[-,dotted] (2.5,-0.25) -- +(5,0);

\draw[dotted] (init) -- (2.5,-1.33);
\draw[-,dotted] (2.5,-1.33) -- +(5,0);
\draw[dotted] (init) -- (2.5,-1.66);
\draw[-,dotted] (2.5,-1.66) -- +(5,0);

\node[state,right=6cm of t01] (s01) {0}; 
\node[state,right=1cm of s01] (s010) {1}; 
\draw (init) -- (s01);
\draw (s01) -- (s010);
\draw[dotted] (s010) -- +(3.5cm,0);


\node[state,right=6cm of t10] (s10) {1}; 
\node[state,right=1cm of s10] (s100) {0}; 
\draw (init) -- (s10);
\draw (s10) -- (s100);
\draw[dotted] (s100) -- +(3.5,0);

\node[state,right=6cm of t11] (s11) {1}; 
\node[state,right=1cm of s11] (s111) {1}; 
\node[state,right=1cm of s111] (s1111) {1}; 
\draw (init) -- (s11);
\draw (s11) -- (s111);
\draw (s111) -- (s1111);
\draw[dotted] (s1111) -- +(\dist,0);

}
\draw[->] (0,1cm) -- (init);

\end{tikzpicture}
    \caption{A depiction of $\Kset$. Vertices in black (on the left including the initial vertex) are labelled by $\fbt$, those in red (on the right, excluding the initial vertex) are labelled by $\pset$.
}%

    \label{fig:phiset}
\end{figure}


We let $\Kset = (V_\cont,E_\cont,t_\varepsilon,\lambda_\cont)$ (see \autoref{fig:phiset}), where
\begin{itemize}
	\item $V_\cont = \{t_u \mid u \in \{0,1\}^\ast\}
      \cup \{s^i_A \mid i \in \Nat \land A \subseteq \Nat\}$,
      \item $E_\cont
    = \{(t_u,t_{u0}),(t_u,t_{u1}) \mid u \in \{0,1\}^\ast\} \cup {} 
      \{ (t_\varepsilon, s^0_A) \mid A \subseteq \Nat \} \cup
      \{ (s^i_A,s^{i+1}_A) \mid A \subseteq \Nat, i \in \Nat \}$,
      \item and the labelling~$\lambda_\cont$ is defined as 
      \begin{itemize}
      	\item $\lambda_\cont(t_\varepsilon)  = \{\fbt\}$
      	\item $\lambda_\cont(t_{u \cdot 0}) = \{\fbt,0\}$
      	\item $\lambda_\cont(t_{u \cdot 1}) = \{\fbt,1\}$, and 
		\item $\lambda_\cont(s^i_A) = \begin{cases}
    \{\pset,0\} & \text{if } i \notin A, \\
    \{\pset,1\} & \text{if } i \in A.
  \end{cases}$      	
      \end{itemize} 
\end{itemize}

\begin{lem}\label{lem:phiset}
There is a satisfiable \hyctlstar sentence~$\phiset$ that has only models of cardinality at least $\contcard$.
\end{lem}

\begin{proof}
  The formula $\phiset$ is defined as the conjunction of the formulas below:
  \begin{enumerate}
  \item The label of the initial vertex is $\{\fbt\}$ and the labels of non-initial vertices are $\set{\fbt, 0}$, $\set{\fbt, 1}$, $\set{\pset, 0}$, or $\set{\pset, 1}$:
    \[
      \forall \pi.\ ( \fbt_\pi \land \lnot 0_\pi \land \lnot 1_\pi \land
      \lnot \pset_\pi )\wedge \X \G \big(
      (\pset_\pi \leftrightarrow \lnot \fbt_\pi) \land 
      (0_\pi \leftrightarrow \lnot 1_\pi)      \big)
    \]

  \item All $\fbt$-labelled vertices have a successor with label $\{\fbt,0\}$
    and one with label $\{\fbt,1\}$, and all $\fbt$-labelled vertices that are additionally labelled by $0$ or $1$ have no $\pset$-labelled successor:
    \[
      \forall \pi.\ \G\big(\fbt_\pi \rightarrow (
      (\exists \pi_0.\ \X (\fbt_{\pi_0} \land 0_{\pi_0}))
      \land (\exists \pi_1.\ \X (\fbt_{\pi_1} \land  1_{\pi_1})) \land
      ( (0_\pi \vee 1_{\pi}) \rightarrow \forall \pi'.\ \X \fbt_{\pi'}))
      \big)
    \]
   \item From $\pset$-labeled vertices, only $\pset$-labeled vertices are reachable:  
   \[\forall \pi.\ \G (\pset \rightarrow \G \pset) \] 
    
  \item For every path of $\fbt$-labelled vertices starting at a successor
    of the initial vertex, there is a path of $\pset$-labelled vertices
    (also starting at a successor of the initial vertex) with the same
    $\{0,1\}$ labelling:
    \[
      \forall \pi.\  \big( (\X \fbt_\pi) \rightarrow
      \exists \pi'.\ \X( \pset_{\pi'} \land \G(0_\pi \leftrightarrow 0_{\pi'}))
      \big)
    \]
  \item Any two paths starting in the same $\pset$-labelled vertex have the same sequence of labels:
    \[
      \forall \pi.\ \G \big(\pset_\pi \rightarrow
      \forall \pi'.\ \G(0_\pi \leftrightarrow 0_{\pi'})
      \big) \, .
    \]
  \end{enumerate}

  It is easy to check that $\Kset \models \phiset$.
  Note however that it is not the only model of $\phiset$: for instance,
  some paths may be duplicated, or merged after some steps if their label
  sequences share a common suffix.
  So, consider an arbitrary transition system $\tsys = (V,E,v_\initmark, \lambda)$
  such that $\tsys \models \phiset$.
  By condition 2, for every set $A \subseteq \Nat$,  there is a path
  $\rho_A$ starting at a successor of $v_\initmark$
  such that $\lambda(\rho_A(i)) = \{\fbt,1\}$ if $i \in A$ and
  $\lambda(\rho_A(i)) = \{\fbt,0\}$ if $i \notin A$.
  Condition 3 implies that there is also a $\pset$-labelled path $\rho'_A$
  such that $\rho'_A$ starts at a successor of $v_\initmark$, and has the same
  $\{0,1\}$ labelling as $\rho_A$.
  Finally, by condition 4, if $A \neq B$ then $\rho_A'(0) \neq \rho_{B}'(0)$.
  
  So, the initial vertex has at least as many successors as there are subsets of $\nats$, i.e., at least $\cont$ many.
\end{proof}


Before moving to the proof that \hyctlstar satisfiability is $\Sigma^2_1$-hard,
we introduce one last auxiliary formula that will be used in the reduction,
showing that addition and multiplication can be defined in \hyctlstar,
and in fact even in \hyltl, as follows:
  Let $\AP = \{\argl,\argr,\res,\add,\mult\}$ and let $\Top$ be the set of all traces $t \in {(2^{\AP})}^\omega$ such that 
  \begin{itemize}
  \item there are unique $n_1,n_2,n_3 \in \nats$ with $\argl \in t(n_1)$,
    $\argr \in t(n_2)$, and $\res \in t(n_3)$, and
  \item either $\add \in t(n)$ and $\mult \notin t(n)$  for all $n$ and $n_1+n_2 = n_3$,
    or $\mult \in t(n)$ and $\add \notin t(n)$  for all $n$ and $n_1 \cdot n_2 = n_3$.
  \end{itemize}

\begin{lem}\label{lem:phiop}
  There is a \hyltl sentence $\phiop$ which has $\Top$ as unique model.
\end{lem}

\begin{proof}
Consider the conjunction of the following \hyltl sentences:
\begin{enumerate}

	\item For every trace~$t$ there are unique $n_1,n_2,n_3 \in \nats$ with $\argl \in t(n_1)$,
    $\argr \in t(n_2)$, and $\res \in t(n_3)$:
\[
		\forall \pi.\ \bigwedge_{a \in \set{\argl,\argr,\res}}(\neg a_\pi)\U (a_\pi \wedge \X\G\neg a_\pi)
\]

	\item Every trace~$t$ satisfies either $\add \in t(n)$ and $\mult\notin t(n)$ for all $n$ or $\mult \in t(n)$ and $\add \notin t(n)$ for all $n$:
	\[
	\forall \pi.\ \G(\add_\pi\wedge\neg\mult_\pi) \vee \G(\mult_\pi\wedge\neg\add_\pi)
	\]

\end{enumerate}
	In the following, we only consider traces satisfying these formulas, as all others are not part of a model. Thus, we will speak of addition traces (if $\add$ holds) and multiplication traces (if $\mult $ holds).
	Furthermore, every trace encodes two unique arguments (given by the positions $n_1$ and $n_2$ such that $\argl\in t(n_1)$ and $\argr\in t(n_2)$) and a unique result (the position~$n_3$ such that $\res\in t(n_3)$). 
	
Next, we need to express that all possible arguments are represented in a model, i.e.\ for every $n_1$ and every $n_2$ there are two traces~$t$ with $\argl \in t(n_1)$ and $\argr \in t(n_2)$, one addition trace and one multiplication trace.	
	We do so inductively. 
	\begin{enumerate}
\setcounter{enumi}{2}
	
	\item\label{itemcomplstart} 
     There are two traces with both arguments being zero (i.e.\ $\argl$ and $\argr$ hold in the first position), one for addition and one for multiplication:
\[\bigwedge_{a \in \set{\add,\mult}}\exists \pi.\ a_\pi \wedge \argl_\pi \wedge \argr_\pi\]


\item\label{itemcomplstep} Now, we express that for every trace, say encoding the arguments~$n_1$ and $n_2$, the argument combinations~$(n_1+1, n_2)$ and $(n_1, n_2+1)$ are also represented in the model, again both for addition and multiplication (here we rely on the fact that either $\add$ or $\mult$ holds at every position, as specified above):
\begin{align*}
\forall \pi.\ \exists\pi_1, \pi_2.\ \left( \bigwedge_{i \in\set{1,2}} \add_\pi \leftrightarrow \add_{\pi_i} \right)\wedge 
&\F(\argl_\pi \wedge \X\argl_{\pi_1}) \wedge \F(\argr_\pi \wedge \argr_{\pi_1}) \wedge \\
&
\F(\argl_\pi \wedge \argl_{\pi_2}) \wedge \F(\argr_\pi \wedge \X\argr_{\pi_2})	
\end{align*}  
\end{enumerate}	
Every model of these formulas contains a trace representing each possible combination of arguments, both for addition and multiplication. 

To conclude, we need to express that the result in each trace is correct. We do so by capturing the inductive definition of addition in terms of repeated increments (which can be expressed by the next operator) and the inductive definition of multiplication in terms of repeated addition.
Formally, this is captured by the next formulas:	
	\begin{enumerate}
	  \setcounter{enumi}{4}
	\item\label{itemcorrfirst} For every trace~$t$: if $\set{\add,\argl} \subseteq t(0)$ then $\argr$ and $\res$ have to hold at the same position (this captures $0 + n = n$):
	\[
	\forall \pi.\ (\add_\pi\wedge \argl_\pi ) \rightarrow \F(\argr_\pi \wedge \res_\pi)
	\]
	\item For each trace~$t$ with $\add\in t(0)$, $\argl \in t(n_1)$,
    $\argr \in t(n_2)$, and $\res \in t(n_3)$ such that $n_1 > 0$ there is a trace~$t'$ such that $\add\in t'(0)$, $\argl \in t'(n_1-1)$,
    $\argr \in t'(n_2)$, and $\res \in t'(n_3-1)$ (this captures $n_1 + n_2 = n_3 \Leftrightarrow n_1-1 + n_2 = n_3-1$ for $n_1 > 0$):
    \[
    \forall \pi.\ \exists\pi'.\ (\add_\pi \wedge \neg \argl_\pi) \rightarrow (\add_{\pi'} \wedge  \F( \X\argl_{\pi} \wedge \argl_{\pi'} ) \wedge \F(\argr_\pi \wedge \argr_{\pi'}) \wedge \F(\X\res_{\pi} \wedge \res_{\pi'}) )
    \]

	\item For every trace~$t$: if $\set{\mult,\argl} \subseteq t(0)$ then also $\res \in t(0)$ (this captures $0 \cdot n = 0$):
	\[
	\forall \pi.\ (\mult_\pi\wedge \argl_\pi ) \rightarrow  \res_\pi
	\]
	\item\label{itemcorrlast} Similarly, for each trace~$t$ with $\mult \in t(0)$, $\argl \in t(n_1)$,
    $\argr \in t(n_2)$, and $\res \in t(n_3)$ such that $n_1 > 0$ there is a trace~$t'$ such that $\mult \in t'(0)$, $\argl \in t'(n_1-1)$,
    $\argr \in t'(n_2)$, and $\res \in t'(n_3-n_2)$.
   The latter requirement is expressed by the existence of a trace~$t''$ with $\add\in t''(0)$, $\argr \in t''(n_2)$, $\res \in t''(n_3)$, and $\argl$ holding in $t''$ at the same time as $\res$ in $t'$, which implies $\res \in t'(n_3-n_2)$. Altogether, this captures $n_1 \cdot n_2 = n_3 \Leftrightarrow (n_1-1) \cdot n_2 = n_3-n_2$ for $n_1 > 0$.
    \begin{align*}
    \forall \pi.\ \exists\pi',\pi''.\ &(\mult_\pi \wedge  \neg \argl_\pi) \rightarrow   \left(\mult_{\pi'} \wedge \add_{\pi''} \wedge\right.  \\ 
&    
    \F(\X\argl_{\pi} \wedge \argl_{\pi'} ) \wedge 
    \F(\argr_\pi \wedge \argr_{\pi'} \wedge \argr_{\pi''}) \wedge \\ 
&  \left.    \F(\res_{\pi'} \wedge \argl_{\pi''}) \wedge 
  \F(\res_{\pi} \wedge \res_{\pi''}
    )\right)
    \end{align*}
\end{enumerate}

Now, $\Top$ is a model of the conjunction~$\phiop$ of these eight formulas. Conversely, every model of $\phiop$ contains all possible combinations of arguments (both for addition and multiplication) due to Formulas~(\ref{itemcomplstart}) and (\ref{itemcomplstep}).
Now, Formulas~(\ref{itemcorrfirst}) to (\ref{itemcorrlast}) ensure that the result is \emph{correct} on these traces. Altogether, this implies that $\Top$ is the unique model of $\phiop$.
\end{proof}

To establish $\Sigma_1^2$-hardness, we give an encoding of formulas of
existential third-order arithmetic into \hyctlstar, i.e.\ every formula of the form~$\exists \xt_1.\ \ldots \exists \xt_n.\ \psi$ where
  $\xt_1, \ldots, \xt_n$ are third-order variables
  and $\psi$ is a sentence of second-order arithmetic can be translated into a \hyctlstar sentence.

As explained in Section~\ref{sec:definitions}, we can (and do for the remainder
of the section) assume that first-order (type 0) variables range over natural
numbers,
second-order (type 1) variables range over sets of natural numbers,
and third-order (type~$2$) variables range over sets of sets of natural numbers.


\begin{lem}\label{lem:hyctlstar-reduction}
One can effectively translate sentences~$\varphi$ of existential third-order arithmetic into \hyctlstar sentences~$\varphi'$ such that
  $(\Nat, +, \cdot, <,\in)$ is a model of $\varphi$ if and only if
  $\varphi'$ is satisfiable.
\end{lem}

\begin{proof}  
  The idea of the proof is as follows. We represent sets of natural numbers
  as infinite paths with labels in $\{0,1\}$, so that quantification over
  sets of natural numbers in $\psi$ can be replaced by \hyctlstar path
  quantification.
    First-order quantification is handled in the same way, but using
  paths where exactly one vertex is labelled $1$.
In particular we encode first- and second-order variables~$x$ of $\varphi$ as path variables~$\pix{x}$ of $\varphi'$.
  For this to work, we need to make sure that every possible set
  has a path representative in the transition system (possibly several
  isomorphic ones).
  This is where formula $\phiset$ defined in Lemma \ref{lem:phiset} is used.
  For arithmetical operations, we rely on the formula $\phiop$ from
  Lemma~\ref{lem:phiop}.
  Finally, we associate with every existentially quantified third-order
  variable $\xt_i$ an atomic proposition $\ai i$, so that
  for a second-order variable $\ys$, the atomic formula~$\ys \in \xt_i$ is interpreted as the
  atomic proposition $\ai i$ being true on the second vertex of $\pix y$.
  This is all explained in more details below.
  \smallskip
  
  Let $\varphi = \exists \xt_1.\ \ldots \exists \xt_n.\ \psi$ where
  $\xt_1, \ldots, \xt_n$ are third-order variables
  and $\psi$ is a formula of second-order arithmetic. 
  We use the atomic propositions 
  \[\AP = \{a_1,\ldots,a_n,0,1,\pset,\fbt,\argl,\argr,\res,\mult,\add\}.\]
  Given an interpretation
  $\inter : \{\xt_1,\ldots,\xt_n\} \rightarrow 2^{(2^\Nat)}$
  of the third-order variables of $\varphi$, we denote by $\tsysI$ the
  transition system over $\AP$ obtained as follows:
  We start from $\Kset$, and extend it with an $\{a_1,\ldots,a_n\}$-labelling
  by setting $\ai i \in \lambda(\rho_A(0))$ if $A \in \inter(\xt_i)$;
  then, we add to this transition system all traces in $\Top$ as disjoint
  paths below the initial vertex.
  
  
  From the formulas $\phiset$ and $\phiop$ defined in Lemmas~\ref{lem:phiset}
  and \ref{lem:phiop}, it is not difficult to construct a formula
  $\phiz$ such that:
  \begin{itemize}
  \item For all $\inter : \{\xt_1,\ldots,\xt_n\} \rightarrow 2^{(2^\Nat)}$,
    the transition system $\tsysI$ is a model of $\phiz$.
  \item Conversely, in any model $\tsys = (V,E,v_\initmark, \lambda)$ of $\phiz$, the following conditions
    are satisfied:
  \begin{enumerate}
    \item
      For every path $\rho$ starting at a $\pset$-labelled successor of
      the initial vertex $v_\initmark$, the vertex~$\rho(0)$ has a label of the form $\lambda(\rho(0)) = \{\pset,b\} \cup \ell$
      with $b \in \{0,1\}$ and $\ell \subseteq \{\ai 1, \ldots, \ai n\}$,
      and every vertex $\rho(i)$ with $i > 0$ has a label
      $\lambda(\rho(i)) = \{\pset,0\}$ or $\lambda(\rho(i)) = \{\pset,1\}$.
      
    \item For every $A \subseteq \Nat$, there exists a $\pset$-labelled path
      $\rho_A$ starting at a successor of $v_\initmark$ 
      such that $1 \in \lambda(\rho_A(i))$ if  $i \in A$,
      and $0 \in \lambda(\rho_A(i))$ if  $i \notin A$.
      Moreover, all such paths have the same $\{\ai 1,\ldots,\ai n\}$ labelling;
      this can be expressed by the formula
      \[
        \forall \pi, \pi'.\
        \X\left( \Big(\G (\pset_\pi \land \pset_{\pi'} \land
        (1_\pi \leftrightarrow 1_{\pi'}))\Big) \rightarrow
        \bigwedge\nolimits_{\ax \in \{\ai 1, \ldots, \ai n\}}
        \ax_\pi \leftrightarrow \ax_{\pi'}\right)
        \, .
      \]
    \item For every path $\rho$ starting at an $\add$- or $\mult$-labelled
      successor of the initial vertex, the label sequence
      $\lambda(\rho(0)) \lambda(\rho(1)) \cdots$ of $\rho$ is in $\Top$.

    \item Conversely, for every trace $t \in \Top$, there exists a path
      $\rho$ starting at a successor of the initial vertex such that
      $\lambda(\rho(0)) \lambda(\rho(1)) \cdots = t$.
    \end{enumerate}
  \end{itemize}

  We then let $\varphi' = \phiz  \land \tr  \psi$, where $\tr  \psi$ is defined inductively from the second-order body $\psi$
  of $\varphi$ as follows:
  \begin{itemize}
  \item $\tr {\psi_1 \lor \psi_2} = \tr {\psi_1} \lor \tr {\psi_2}$ and
  \item $\tr {\lnot \psi_1} = \lnot \tr {\psi_1}$.
  \item If $x$ ranges over sets of natural numbers,
    \[\tr {\exists x.\ \psi_1} =
    \exists \pix x.\ ((\X \pset_{\pix x}) \land \tr  {\psi_1}),\]
    and
    \[\tr {\forall x.\ \psi_1} =
    \forall \pix x.\ ((\X \pset_{\pix x}) \rightarrow \tr  {\psi_1}).\]

  \item If $x$ ranges over natural numbers, 
    \[\tr {\exists x.\ \psi_1} = \exists \pix x.\ ((\X \pset_{\pix x}) \land
    \X (0_{\pix x} \U (1_{\pix x} \land \X\G 0_{\pix x})) \land \tr  {\psi_1}),\]
    and
    \[\tr {\forall x.\ \psi_1} = \forall \pix x.\ ((\X \pset_{\pix x}) \land
    \X (0_{\pix x} \U (1_{\pix x} \land \X\G 0_{\pix x})) \rightarrow \tr  {\psi_1}).\] 
    Here, the subformula~$0_{\pix x} \U (1_{\pix x} \land \X\G 0_{\pix x})$ expresses that there is a single $1$ on the trace assigned to $\pix x$, i.e.\ the path represents a singleton set.
  \item If $\ys$ ranges over sets of natural numbers,
    $\tr {\ys \in \xt_i} = \X (\ai i)_{\pix \ys}$.
  \item If $\xo$ ranges over natural numbers and $\yt$ over sets of natural
    numbers, $\tr {\xo \in \yt} = \F(1_{\pix \xo} \land 1_{\pix \yt})$.
  \item $\tr {\xo < \yo} = \F(1_{\pix \xo} \land \X \F 1_{\pix \yo})$.
  \item $\tr {\xo \cdot \yo = \zo} = \exists \pi.\ (\X \add_\pi) \land \F(\argl_\pi \land 1_{\pix \xo}) \land
    \F(\argr_{\pi} \land 1_{\pix \yo}) \land \F(\res_\pi \land 1_{\pix \zo})$,
    and $\tr {\xo+\yo = \zo} =
    \exists \pi.\ (\X \mult_\pi) \land \F(\argl_\pi \land 1_{\pix \xo}) \land
    \F(\argr_{\pi} \land 1_{\pix \yo}) \land \F(\res_\pi \land 1_{\pix \zo})$.
  \end{itemize}

  If $\psi$ is true under some interpretation $\inter$ of $\xt_1, \ldots, \xt_n$
  as sets of sets of natural numbers,
  then the transition system $\tsysI$ defined above is a model of $\varphi'$.
  Conversely, if $\tsys \models \varphi'$ for some transition system $\tsys$,
  then for all sets $A \subseteq \Nat$ there is a path $\rho_A$ matching $A$ in $\tsys$,
  and all such paths have the same $\{\ai 1,\ldots,\ai n\}$-labelling, so we
  can define an interpretation $\inter$ of $\xt_1,\ldots,\xt_n$ by taking
  $A \in \inter(\xt_i)$ if and only if $\ai i \in \lambda(\rho_A(0))$.
  Under this interpretation $\psi$ holds, and thus $\varphi$ is true, as first- and second-order quantification in   $(\Nat, +, \cdot, <,\in)$ is mimicked by path quantification in $\tsys$.
\end{proof}

Now, we have all the tools at hand to prove the lower bound on the \hyctlstar satisfiability problem.

\begin{lem}
  \hyctlstar satisfiability is $\Sigma^2_1$-hard.
\end{lem}
\begin{proof}
  Let $N$ be a $\Sigma^2_1$ set, i.e.\
  $N = \set{x \in \nats \mid \exists x_0.\ \cdots \exists x_k.\ \psi(x, x_0, \ldots, x_k)}$ for some second-order arithmetic formula $\psi$ with existentially
  quantified third-order variables $x_i$.
  For every $n \in \Nat$, we define the sentence
  \[\varphi_n = \exists x_0.\ \cdots \exists x_k.\ \psi(n,x_0,\ldots,x_k)\, .\]
	Recall that every fixed natural number~$n$ is definable in first-order arithmetic, which is the reason we can use $n$ in $\psi$.
  
  Then $\varphi_n$ is true  if and only if  $n \in N$.
  Combining this with Lemma~\ref{lem:hyctlstar-reduction}, we obtain a
  computable function that maps any $n \in \Nat$ to a \hyctlstar formula
  $\phi'_n$ such that $n \in N$ if and only if $\phi'_n$ is satisfiable.
\end{proof}

\subsection{Variations of HyperCTL$^*$ Satisfiability}

The general \hyctlstar satisfiability problem, as studied above, asks for the existence of a model of arbitrary size. 
In the $\Sigma^2_1$-hardness proof we relied on uncountable models with infinite branching. 
Hence, it is natural to ask whether satisfiability is easier when we consider restricted classes of transition systems.
In the remainder of this section, we study the following variations of satisfiability.

\begin{itemize}
\item The \hyctlstar \emph{finite satisfiability problem}: given a \hyctlstar sentence, determine whether it has a finite model.
\item The \hyctlstar \emph{finitely-branching satisfiability problem}: given a \hyctlstar sentence, determine whether it has a finitely-branching model.\footnote{A transition system is finitely-branching, if every vertex has only finitely many successors.}
\item The \hyctlstar \emph{countable satisfiability problem}: given a \hyctlstar sentence, determine whether it has a countable model.
\end{itemize}

Let us begin with finite satisfiability. 
In contrast to general satisfiability, it is much simpler, but still undecidable.

\begin{thm}
\hyctlstar finite satisfiability is $\Sigma_1^0$-complete.
\end{thm}

\begin{proof}
The upper bound follows from \hyctlstar model checking being decidable~\cite{ClarksonFKMRS14} (therefore, the finite satisfiability problem is recursively enumerable and thus in $\Sigma_1^0$) while the matching lower bound is inherited from \hyltl~\cite{FinkbeinerH16}.	
\end{proof} 

Next, we show that the complexity of \hyctlstar finitely-branching satisfiability
and countable satisfiability lies between that of finite satisfiability
and general satisfiability:
both are equivalent to \emph{truth in second-order arithmetic}, that is,
the problem of deciding whether a given sentence of second-order arithmetic is satisfied
in the standard model~$(\nats, 0,1,+,\cdot, <,\in)$ of second-order arithmetic.

\begin{thm}\label{thm:hyctlvars}
All of the following problems are effectively interreducible:
  \begin{enumerate}
  \item\label{thm:hyctlvars:count} \hyctlstar countable satisfiability.
  \item\label{thm:hyctlvars:finbranch} \hyctlstar finitely-branching satisfiability.
  \item\label{thm:hyctlvars:sotruth} Truth in second-order arithmetic.
  \end{enumerate}
\end{thm}

To prove \autoref{thm:hyctlvars}, we show the implication (\ref{thm:hyctlvars:count}) $\Rightarrow$ (\ref{thm:hyctlvars:sotruth}) in \autoref{lem:hyctlvars:count} and the  implication (\ref{thm:hyctlvars:finbranch}) $\Rightarrow$ (\ref{thm:hyctlvars:sotruth}) in \autoref{lem:hyctlvars:finbranch}.
Then, in \autoref{lem:hyctlvars:rev} we show both converse implications simultaneously. 

We start by showing that countable satisfiability can be effectively reduced to truth in second-order arithmetic.
As every countable set is in bijection with the natural numbers, countable satisfiability asks for the existence of a model whose set of vertices is the set of natural numbers. 
This can easily be expressed in second-order arithmetic, leading to a fairly
straightforward reduction to truth in second-order arithmetic.

\begin{lem}\label{lem:hyctlvars:count}
There is an effective reduction from \hyctlstar countable
  satisfiability to truth in second-order arithmetic.
\end{lem}

\begin{proof}
  Let $\phi$ be a \hyctlstar sentence. We construct a sentence~$\phicount$ of second-order
  arithmetic such that
  $(\nats,0,1,+,\cdot,<,\in) \models \phicount$ if and only if $\phi$
  has a countable model, or, equivalently, if and only if $\phi$ has a model
  of the form $\tsys = (\nats,E,0,\lambda)$ with vertex set~$\nats$, which
 implies that
  the set~$E$ of edges is a subset of $\nats \times \nats$.
  Note that we assume (w.l.o.g.) that the initial
  vertex is~$0$. The labeling function $\lambda$ maps each natural number
  (that is, each vertex) to a set of atomic propositions.
  We assume a fixed encoding of valuations in $2^\AP$ as natural numbers in
  $\{0,\ldots,|2^{\AP}|-1\}$, so that we can equivalently view $\lambda$ as a
  function $\lambda : \nats \rightarrow \nats$ such that $\lambda(n) < \size{2^{\AP}}$
  for all $n \in \nats$. Note that binary relations over $\nats$ can be encoded by functions from natural numbers to natural numbers, and the encoding can be implemented in first-order arithmetic.

  The formula $\phicount$ is defined as
  \[
    \phicount = \exists E.\, \exists \lambda.\,
    (\forall x.\, \lambda(x) < |2^\AP|) \land 
    \phi'(E,\lambda,0) \, ,
  \]
  where $E$ is a second-order variable ranging over subsets of
  $\nats \times \nats$,
  $\lambda$ a second-order variable ranging over functions from $\nats \to  \nats$,
  and $\phi'(E,\lambda,i)$, defined below, expresses the fact that
  the transition system $(\nats,E,0,\lambda)$ is a model of $\phi$. 
  
  We use the following abbreviations:
  \begin{itemize}
  \item Given a second-order variable $f$ ranging over functions from
    $\nats$ to $\nats$, the formula
    $\mathit{path}(f,E) = \forall n.\ (f(n),f(n+1)) \in E$
    expresses the fact that $f(0) f(1) f(2) \ldots$ is
    a path in $(\nats,E,0,\lambda)$.

  \item Given second-order variables $f$ and $f'$ ranging over functions from
    $\nats$ to $\nats$ and a first-order variable $i$ ranging over natural
    numbers, we let
    \[
      \mathit{branch}(f,f',i,E) = \mathit{path}(f,E) \land \mathit{path}(f',E) \land \forall j \le i.\ 
      f(j) = f'(j) \, .
    \]
    This formula is satisfied by paths $f$ and $f'$ if $f$ and $f'$ coincide up to (and including) position $i$. We will use to restrict path quantification to those that start at a given position of a given path.
  \end{itemize}

  We define $\phi'$ inductively from $\phi$, therefore considering in general
  \hyctlstar formulas with free variables~$\pi_1,\ldots,\pi_k$, in which case the formula $\phi'$ has free variables
  $E,\lambda,f_{\pi_1},\ldots,f_{\pi_k},i$.
  The variable $i$ is interpreted as the current time point.
  If $\phi$ is a sentence, $i$ is not free in $\phi'$, as we use $0$ in that case.
Also, the translation depends on an ordering of the free variables of $\phi$, i.e.\  quantified paths start at position~$i$ of the largest variable, as path quantification depends on the context of a formula with free variables.
In the following, we indicate by ordering by the naming of the variables, i.e.\ we have  $\pi_1 <\cdots <\pi_k$.
  \begin{itemize}
   
    \item $a'_{\pi_j}(E,\lambda,f_{\pi_1},\ldots, f_{\pi_k} ,i) =
    \bigvee_{\{v \in 2^\AP \mid a \in v\}} \lambda(f_{\pi_j}(i)) = [v]$,
    where $[v]$ is the encoding of $v$ as a natural number.
     
  \item If $\phi(\pi_1, \ldots, \pi_k) = \lnot \psi$ then $\phi'(E,\lambda,f_{\pi_1},\ldots, f_{\pi_k} ,i) = \neg (\psi'(E,\lambda,f_{\pi_1},\ldots, f_{\pi_k} ,i))$.
     
  \item If $\phi(\pi_1, \ldots, \pi_k) = \psi_1 \lor \psi_2 $ then \[\phi'(E,\lambda,f_{\pi_1},\ldots, f_{\pi_k} ,i) = (\psi_1'(E,\lambda,f_{\pi_1},\ldots, f_{\pi_k} ,i))\lor(\psi_2'(E,\lambda,f_{\pi_1},\ldots, f_{\pi_k} ,i)).\]
   
  \item If $\phi(\pi_1,\ldots,\pi_k) = \X \psi$,  then we define
    \[
      \phi'(E,\lambda,f_{\pi_1},\ldots,f_{\pi_k},i) =
      \psi'(E,\lambda,f_{\pi_1},\ldots,f_{\pi_k},i+1) \, .
    \]
     
  \item If $\phi(\pi_1,\ldots,\pi_k) = \psi_1 \U \psi_2$,  then we define
    \begin{align*}
      \phi'(E,\lambda,_{\pi_1},\ldots,f_{\pi_k},i) =
      \exists j.\,
      & j \ge i \land 
       \psi_2'(E,\lambda,f_{\pi_1},\ldots,f_{\pi_k},j) \land {} \\
      & \forall j'.\  (i \le j' < j \rightarrow
        \psi_1'(E,\lambda,f_{\pi_1},\ldots,f_{\pi_k},j')) \, .
    \end{align*}
   
  \item
    If $\phi = \exists \pi_1. \psi(\pi_1)$ is a sentence, then we define
    \[
      \phi'(E,\lambda) = \exists f_{\pi_1}.\ 
      \mathit{path}(f_{\pi_1},E) \land f_{\pi_1}(0) = 0 \land
      \psi'(E,\lambda,f_{\pi_1},0) \, .
    \]
    Recall that~$f_{\pi_1}$ ranges over functions from $\nats$ to $\nats$ and note that the formula requires $f$ to encode a path and to start at the initial vertex~$0$.
    
    If $\phi(\pi_1,\ldots,\pi_k) = \exists \pi_{k+1}.\ \psi(\pi_1,\ldots,\pi_k,\pi_{k+1})$
    with $k > 0$, then we define
    \[
      \phi'(E,\lambda,f_{\pi_1},\ldots,f_{\pi_k},i) = 
        \exists f_{\pi_{k+1}}.\  \mathit{branch}(f_{\pi_{k+1}},f_{\pi_k},i,E) \land
        \psi'(E,\lambda,f_{\pi_1},\ldots,f_{\pi_k},f_{\pi_{k+1}},i) \, .
      \]
      Here, we make use of the ordering of the free variables of $\phi$, as the translated formula requires the function assigned to $f_{\pi_{k+1}}$ to encode a path branching of the path encoded by the function assigned to $f_{\pi_{k}}$.
      

    \item
      If $\phi = \forall \pi_1.\ \psi(\pi_1)$ is a sentence, then we define
      \[
        \phi'(E,\lambda) = \forall f_{\pi_1}.\
        \left(\mathit{path}(f_{\pi_1},E) \land f_{\pi_1}(0) = 0 \right) \rightarrow
        \psi'(E,\lambda,f_{\pi_1},0) \, .
      \]
      If $\phi(\pi_1,\ldots,\pi_k) = \forall \pi_{k+1}.\ \psi(\pi_1,\ldots,\pi_k,\pi_{k+1})$
      with $k > 0$,  then we define
    \[
      \phi'(E,\lambda,f_{\pi_1},\ldots,f_{\pi_k},i) = 
        \forall f_{\pi_{k+1}}.\  \mathit{branch}(f_{\pi_{k+1}},f_{\pi_k},i,E) \rightarrow
        \psi'(E,\lambda,f_{\pi_1},\ldots,f_{\pi_k},f_{\pi_{k+1}},i)
      \]
  \end{itemize}
  
Now, $\phi$ has a countable model if and only if the second-order sentence~$\phicount$ is true in $(\nats,0,1,+,\cdot,<,\in)$.
\end{proof}

Since every finitely-branching model has countably many vertices that are reachable from the initial vertex, the previous proof can
be easily adapted for the case of finitely-branching satisfiability.

\begin{lem}\label{lem:hyctlvars:finbranch}
There is an effective reduction from \hyctlstar finitely-branching
  satisfiability to truth in second-order arithmetic.
\end{lem}

\begin{proof}
  Let $\phi$ be a \hyctlstar sentence. We construct a second-order
  arithmetic formula~$\phifb$ such that
  $(\nats,0,1,+,\cdot,<,\in) \models \phifb$ if and only if $\phi$ has a
  finitely-branching model, which we can again assume without loss of
  generality to be of the form $\tsys = (\nats,E,0,\lambda)$, where the set of
  vertices is $\nats$, the set~$E$ of edges is a subset of
  $\nats \times \nats$, the initial vertex is $0$, and the labeling function
  $\lambda$ is encoded as a function from  $\nats$ to $ \nats$.

  The formula $\phifb$ is almost identical to $\phicount$, only adding the
  finite branching requirement:
  \[
    \phifb = \exists E.\ \exists \lambda.\
    (\forall x.\ \lambda(x) < |2^\AP|) \land
    (\forall x.\ \exists y.\ \forall z.\ (x,z) \in E \rightarrow z < y) \land
    \phi'(E,\lambda,0) \, . \qedhere
  \]
\end{proof}


Now, we consider the converse, i.e.\ that truth of second-order arithmetic can be reduced to countable and finitely-branching satisfiability. To this end, we adapt the $\Sigma_1^2$-hardness proof for \hyctlstar. 
Recall that we constructed a formula whose models contain all $\set{0,1}$-labelled paths, which we used to encode the subsets of $\nats$. 
In that proof, we needed to ensure that the initial vertices of all these paths are pairwise different in order to encode existential third-order quantification, which resulted in uncountably many successors of the initial vertex.
Also, we used the traces in $\Top$ to encode arithmetic operations. 

Here, we only have to encode first- and second-order quantification, so we can drop the requirement on the initial vertices of the paths encoding subsets, which simplifies our construction and removes one source of infinite branching.
However, there is a second source of infinite branching, i.e.\ the infinitely many traces in $\Top$ which all start at successors of the initial vertex. This is unavoidable: To obtain formulas that always have finitely-branching models, we can no longer work with $\Top$.
We begin by explaining the reason for this and then explain how to adapt the construction to obtain the desired result. 

Recall that we defined $\Top$ over $\AP = \{\argl,\argr,\res,\add,\mult\}$ as the set of all traces $t \in {(2^{\AP})}^\omega$ such that 
  \begin{itemize}
  \item there are unique $n_1,n_2,n_3 \in \nats$ with $\argl \in t(n_1)$,
    $\argr \in t(n_2)$, and $\res \in t(n_3)$, and
  \item either $\add \in t(n)$ and $\mult \notin t(n)$  for all $n$ and $n_1+n_2 = n_3$,
    or $\mult \in t(n)$ and $\add \notin t(n)$  for all $n$ and $n_1 \cdot n_2 = n_3$.
  \end{itemize}

An application of Kőnig's Lemma~\cite{konig} shows that there is no finitely-branching transition system whose set of traces is $\Top$. 
The reason is that $\Top$ is not (topologically) closed (see definitions below), while the set of traces of a finitely-branching transition system is always closed.


Let $\prefs{t} \subseteq (\pow{\ap})^*$ denote the set of finite prefixes of a trace~$t \in (\pow{\ap})^\omega$.
Furthermore, let $\prefs{T} = \bigcup_{t \in T}\prefs{t}$ be the set of finite prefixes of a set~$T \subseteq (\pow{\ap})^\omega$ of traces.
The closure~$\closure{T} \subseteq (\pow{\ap})^\omega$ of such a set~$T$ is defined as 
\[
\closure{T} = \set{t \in (\pow{\ap})^\omega \mid \prefs{t} \subseteq \prefs{T}}.
\]
For example, $\set{\add}^\omega \in \closure{\Top}$ and $\set{\mult}^*\set{\argr,\mult}\set{\mult}^\omega \subseteq \closure{\Top}$.
Note that we have $T \subseteq \closure{T}$ for every $T$. 
As usual, we say that $T$ is closed if $T = \closure{T}$.

Let $\ap$ be finite and let $T \subseteq (\pow{\ap})^\omega$ be closed.
Furthermore, let $\tsys(T)$ be the finitely-branching transition system~$(\prefs{T}, E, \epsilon, \lambda)$ with \[E = \set{(w,wv) \mid wv \in \prefs{T} \text{ and } v \in \pow{\ap} },\]
$\lambda(\epsilon) = \emptyset$, and $\lambda(wv) = v$ for all $wv \in \prefs{T}$ with $v \in \pow{\ap}$.


\begin{rem}
The set of traces of paths of $\tsys(T)$ starting at the successors of the initial vertex~$\epsilon$ is exactly $T$.
\end{rem}


In the following, we show that we can replace the use of $\Top$ by $\closure{\Top}$ and still capture addition and multiplication in \hyltl.
We begin by characterising the difference between $\Top$ and $\closure{\Top}$ and then show that $\closure{\Top}$ is also the unique model of some \hyltl sentence~$\phiopc$.

Intuitively, a trace is in $\closure{\Top} \setminus \Top$ if at least one of the arguments (the propositions~$\argl$ and $\argr$) are missing. 
In all but one case, this also implies that $\res$ does not occur in the trace, as the position of $\res$ is (almost) always greater than the positions of the arguments. 
The only exception is when $\mult$ holds and $\res$ holds at the first position, i.e.\ in the limit of traces encoding $0\cdot n = n$ for $n$ tending towards infinity.

Let $\diff$ be the set of traces~$t$ over $\AP = \{\argl,\argr,\res,\add,\mult\}$ such that
\begin{itemize}
    \item for each $a \in \set{\argl,\argr,\res}$ there is at most one	~$n$ such that $a \in t(n)$, and
	\item either $\add \in t(n)$ and $\mult \notin t(n)$  for all $n$,
    or $\mult \in t(n)$ and $\add \notin t(n)$  for all $n$, 
    \item there is at least one $a \in \set{\argl,\argr}$  such that $a \notin t(n)$ for all $n$.
    \item Furthermore, if there is an $n$ such that $\res \in t(n)$, then $\mult \in t(0)$, $n = 0$, and either $\argl\in t(0)$ or $\argr\in t(0)$. 
\end{itemize}

\begin{rem}
$\closure{\Top} \setminus \Top = D$.	
\end{rem}

Now, we show the analogue of Lemma~\ref{lem:phiop} for $\closure{\Top}$.

\begin{lem}\label{lem:phiopc}
  There is a \hyltl sentence $\phiopc$ which has $\closure{\Top}$ as unique model.
\end{lem}

\begin{proof}
We adapt the formula~$\phiop$ presented in the proof of Lemma~\ref{lem:phiop} having $\Top$ as unique model.
Consider the conjunction of the following \hyltl sentences:
\begin{enumerate}

	\item For every trace~$t$ and every $a \in \set{\argl,\argr,\res}$ there is at most one $n$ such that $a \in t(n)$:
\[
		\forall \pi.\ \bigwedge_{a \in \set{\argl,\argr,\res}}(\G\neg a_\pi) \vee (\neg a_\pi)\U (a_\pi \wedge \X\G\neg a_\pi)
\]
\item For all traces $t$: If both $\argl$ and $\argr$ appear in $t$, then also $\res$ (this captures the fact that the position of $\res$ is determined by the positions of $\argl$ and $\argr$):

\[
\forall \pi.\ (\F\argl_\pi \wedge \F\argr_\pi) \rightarrow \F\res_\pi\]

	\item Every trace~$t$ satisfies either $\add \in t(n)$ and $\mult\notin t(n)$ for all $n$ or $\mult \in t(n)$ and $\add \notin t(n)$ for all $n$:
	\[
	\forall \pi.\ \G(\add_\pi\wedge\neg\mult_\pi) \vee \G(\mult_\pi\wedge\neg\add_\pi)
	\]	
	
	\item For all traces~$t$: If there is an $a \in \set{\argl,\argr}$ such that $a \notin t(n)$ for all $n$, but $\res \in t(n_3)$ for some $n_3$, then $\set{\mult, \res} \subseteq t(0)$ and $\set{\argl,\argr} \cap t(0) \neq \emptyset$:
	\[
	\forall \pi.\ \left(\F\res_\pi \wedge \bigvee_{a \in \set{\argl,\argr}}\G\neg a_\pi\right)\rightarrow \left(\mult_\pi \wedge \res_\pi \wedge \bigvee_{a \in \set{\argl,\argr}}a\right)
	\]

\end{enumerate}
 We again only consider traces satisfying these formulas in the remainder of the proof, as all others are not part of a model. Also, we again speak of addition traces (if $\add$ holds) and multiplication traces (if $\mult $ holds).
 
	Furthermore, if a trace satisfies the (guard) formula~$\phi_g = \F \arg_1 \wedge \F\argr$, then it encodes two unique arguments (given by the unique positions $n_1$ and $n_2$ such that $\argl\in t(n_1)$ and $\argr\in t(n_2)$. As the above formulas are satisfied, such a trace also encodes a result via the unique position~$n_3$ such that $\res \in t(n_3)$. 
	
As before, we next express that every combination of inputs is present:
\begin{enumerate}
\setcounter{enumi}{4}
	
	\item\label{itemcomplstartcl}
     There are two traces with both arguments being zero, one for addition and one for multiplication:
\[\bigwedge_{a \in \set{\add,\mult}}\exists \pi.\ a_\pi \wedge \argl_\pi \wedge \argr_\pi\]

\item\label{itemcomplstepcl} For every trace encoding the arguments~$n_1$ and $n_2$, the argument combinations~$(n_1+1, n_2)$ and $(n_1, n_2+1)$ are also represented in the model, again both for addition and multiplication (here we rely on the fact that either $\add$ or $\mult$ holds at every position, as specified above). Note however, that not every trace will encode two inputs, which is why we have to use the guard~$\phi_g$.
\begin{align*}
\forall \pi.\ \phi_g \rightarrow  \exists\pi_1, \pi_2.\ \left( \bigwedge_{i \in\set{1,2}} \add_\pi \leftrightarrow \add_{\pi_i} \right)\wedge 
&\F(\argl_\pi \wedge \X\argl_{\pi_1}) \wedge \F(\argr_\pi \wedge \argr_{\pi_1}) \wedge \\
&
\F(\argl_\pi \wedge \argl_{\pi_2}) \wedge \F(\argr_\pi \wedge \X\argr_{\pi_2})	
\end{align*}  
\end{enumerate}	
Every model of these formulas contains a trace representing each possible combination of arguments, both for multiplication and addition. 

To conclude, we need to express that the result in each trace is correct by again capturing the inductive definition of addition in terms of repeated increments and the inductive definition of multiplication in terms of repeated addition.
The formulas differ from those in the proof of Lemma~\ref{lem:phiop} only in the use of the guard~$\phi_g$.

	\begin{enumerate}
	  \setcounter{enumi}{6}
	\item\label{itemcorrfirstcl} For every trace~$t$: if $\set{\add,\argl} \subseteq t(0)$ and $\argr$ appears in $t$ then $\argr$ and $\res$ have to hold at the same position (this captures $0 + n = n$):
	\[
	\forall \pi.\ (\phi_g\wedge \add\wedge \argl_\pi ) \rightarrow \F(\argr_\pi \wedge \res_\pi)
	\]
	\item For each trace~$t$ with $\add\in t(0)$, $\argl \in t(n_1)$,
    $\argr \in t(n_2)$, and $\res \in t(n_3)$ such that $n_1 > 0$ there is a trace~$t'$ such that $\add\in t'(0)$, $\argl \in t'(n_1-1)$,
    $\argr \in t'(n_2)$, and $\res \in t'(n_3-1)$ (this captures $n_1 + n_2 = n_3 \Leftrightarrow n_1-1 + n_2 = n_3-1$ for $n_1 > 0$):
    \[
    \forall \pi.\ \exists\pi'.\ (\phi_g\wedge\add_\pi \wedge \neg \argl_\pi) \rightarrow (\add_{\pi'} \wedge  \F( \X\argl_{\pi} \wedge \argl_{\pi'} ) \wedge \F(\argr_\pi \wedge \argr_{\pi'}) \wedge \F(\X\res_{\pi} \wedge \res_{\pi'}) )
    \]

	\item For every trace~$t$: if $\set{\mult,\argl} \subseteq t(0)$ then also $\res \in t(0)$ (this captures $0 \cdot n = 0$):
	\[
	\forall \pi.\ (\mult\wedge \argl_\pi ) \rightarrow  \res_\pi
	\]
	\item\label{itemcorrlastcl} Similarly, for each trace~$t$ with $\mult \in t(0)$, $\argl \in t(n_1)$,
    $\argr \in t(n_2)$, and $\res \in t(n_3)$ such that $n_1 > 0$ there is a trace~$t'$ such that $\mult \in t'(0)$, $\argl \in t'(n_1-1)$,
    $\argr \in t'(n_2)$, and $\res \in t'(n_3-n_2)$.
   The latter requirement is expressed by the existence of a trace~$t''$ with $\add\in t''(0)$, $\argr \in t''(n_2)$, $\res \in t''(n_3)$, and $\argl$ holding in $t''$ at the same time as $\res$ in $t'$, which implies $\res \in t'(n_3-n_2)$. Altogether, this captures $n_1 \cdot n_2 = n_3 \Leftrightarrow (n_1-1) \cdot n_2 = n_3-n_2$ for $n_1 > 0$.
    \begin{align*}
    \forall \pi.\ \exists\pi',\pi''.\ &(\phi_g\wedge\mult_\pi \wedge  \neg \argl_\pi) \rightarrow   \left(\mult_{\pi'} \wedge \add_{\pi''} \wedge\right.  \\ 
&    
    \F(\X\argl_{\pi} \wedge \argl_{\pi'} ) \wedge 
    \F(\argr_\pi \wedge \argr_{\pi'} \wedge \argr_{\pi''}) \wedge \\ 
&  \left.    \F(\res_{\pi'} \wedge \argl_{\pi''}) \wedge 
  \F(\res_{\pi} \wedge \res_{\pi''}
    \right)
    \end{align*}
\end{enumerate}

Now, $\closure{\Top}$ is a model of the conjunction~$\phiopc$ of these ten formulas. Conversely, every model of $\phiopc$ contains all possible combinations of arguments (both for addition and multiplication) due to Formulas~(\ref{itemcomplstartcl}) and (\ref{itemcomplstepcl}).
Now, Formulas~(\ref{itemcorrfirstcl}) to (\ref{itemcorrlastcl}) ensure that the result is \emph{correct} on these traces.
Furthermore, all traces in $\diff$, but not more, are also contained due to the first four formulas. 
 Altogether, this implies that $\closure{\Top}$ is the unique model of $\phiopc$.
\end{proof}


We are now ready to prove the lower bounds for \hyctlstar countable and
finitely-branching satisfiability.


\begin{lem}\label{lem:hyctlvars:rev}
  There is an effective reduction from truth in second-order arithmetic to
  \hyctlstar countable and finitely-branching satisfiability.
\end{lem}

\begin{proof}
We proceed as in the proof of Lemma~\ref{lem:hyctlstar-reduction}.
  Given a sentence~$\phi$ of second-order arithmetic we construct a \hyctlstar formula $\varphi'$ such that
  $(\Nat, +, \cdot, <,\in)$ is a model of $\varphi$ if and only if
  $\varphi'$ is satisfied by a countable and finitely-branching model.

As before, we represent sets of natural numbers
  as infinite paths with labels in $\{0,1\}$, so quantification over sets of natural numbers and natural numbers is captured by path quantification.
The major difference between our proof here and the one of Lemma~\ref{lem:hyctlstar-reduction} is that we do not need to deal with third-order quantification here.
This means we only need to have every possible $\{0,1\}$-labeled path in our models, but not with pairwise distinct initial vertices.
In particular, the finite (and therefore finitely-branching) transition system~$\tsys_f$ depicted in Figure~\ref{fig:phisetc} has all such paths.

\begin{figure}
    \centering
\def\dist{2.0cm}
\begin{tikzpicture}[->,
>=stealth', 
level/.style={sibling distance = 3cm/#1, level distance = \dist}, 
scale=0.7,
transform shape,
grow=right,thick]

\node (init) [state] {\phantom{0}}
child {
    node (t0) [state] {1} 
}
child {
    node (t1) [state] {0} 
}
;

\draw[->] (-1,0) -- (init);
\path[->] 
(t0) edge[bend left] (t1)
(t1) edge[bend left] (t0)
(t0) edge[loop right] ()
(t1) edge[loop right] ()
;

\end{tikzpicture}
    \caption{A depiction of $\tsys_f$. All vertices but the initial one are labelled by $\fbt$.
}%

    \label{fig:phisetc}
\end{figure}

For arithmetical operations, we rely on the \hyltl sentence~$\phiopc$ from
  Lemma~\ref{lem:phiopc}, with its unique model~$\closure{\Top}$, and the transition system~$\tsys(\closure{\Top})$, which is countable, finitely-branching, and whose set of traces starting at the successors of the initial vertex is exactly $\closure{\Top}$.
  We combine $\tsys_f$ and $\tsys(\closure{\Top})$ by by identifying their respective initial vertices, but taking the disjoint union of all other vertices.
  The resulting transition system~$\tsysc$ contains all traces encoding the subsets of the natural numbers as well as the traces required to model arithmetical operations.
  Furthermore, it is still countable and finitely-branching.
  
  \smallskip
  
  Let $\AP = \{0,1,\fbt,\argl,\argr,\res,\mult,\add\}$.
  Using parts of the formula~$\phiset$ defined in Lemma~\ref{lem:phiset}
  and the formula~$\phiopc$ defined in Lemma~\ref{lem:phiopc}, it is not difficult to construct a formula
  $\phizc$ such that:
  \begin{itemize}
  \item The transition system $\tsysc$ is a model of $\phizc$.
  \item Conversely, in any model $\tsys = (V,E,v_\initmark, \lambda)$ of $\phizc$, the following conditions
    are satisfied:
  \begin{enumerate}
    \item
      For every path $\rho$ starting at a $\fbt$-labelled successor of
      the initial vertex $v_\initmark$, every vertex $\rho(i)$ with $i \ge 0$ has a label
      $\lambda(\rho(i)) = \{\fbt,0\}$ or $\lambda(\rho(i)) = \{\fbt,1\}$.
      
    \item For every $A \subseteq \Nat$, there exists a $\fbt$-labelled path
      $\rho_A$ starting at a successor of $v_\initmark$ 
      such that $1 \in \lambda(\rho_A(i))$ if  $i \in A$,
      and $0 \in \lambda(\rho_A(i))$ if  $i \notin A$.
     \item For every path $\rho$ starting at an $\add$- or $\mult$-labelled
      successor of the initial vertex, the label sequence
      $\lambda(\rho(0)) \lambda(\rho(1)) \cdots$ of $\rho$ is in $\closure{\Top}$.

    \item Conversely, for every trace $t \in \closure{\Top}$, there exists a path
      $\rho$ starting at a successor of the initial vertex such that
      $\lambda(\rho(0)) \lambda(\rho(1)) \cdots = t$.
    \end{enumerate}
  \end{itemize}

  We then let $\varphi' = \phizc  \land \tr  \psi$, where $\tr  \phi$ is defined inductively from $\varphi$ as in the proof of Lemma~\ref{lem:hyctlstar-reduction}:
  \begin{itemize}
  \item $\tr {\psi_1 \lor \psi_2} = \tr {\psi_1} \lor \tr {\psi_2}$ and
  \item 
    $\tr {\lnot \psi_1} = \lnot \tr {\psi_1}$.
  \item If $x$ ranges over sets of natural numbers,
    \[\tr {\exists x.\ \psi_1} =
    \exists \pix x.\ ((\X \fbt_{\pix x}) \land \tr  {\psi_1}),\]
    and
    \[\tr {\forall x.\ \psi_1} =
    \forall \pix x.\ ((\X \fbt_{\pix x}) \rightarrow \tr  {\psi_1}).\]

  \item If $x$ ranges over natural numbers, 
    \[\tr {\exists x.\ \psi_1} = \exists \pix x.\ ((\X \fbt_{\pix x}) \land
    \X (0_{\pix x} \U (1_{\pix x} \land \X\G 0_{\pix x})) \land \tr  {\psi_1}),\]
    and
    \[\tr {\forall x.\ \psi_1} = \forall \pix x.\ ((\X \fbt_{\pix x}) \land
    \X (0_{\pix x} \U (1_{\pix x} \land \X\G 0_{\pix x})) \rightarrow \tr  {\psi_1}).\] 
    Here, the subformula~$0_{\pix x} \U (1_{\pix x} \land \X\G 0_{\pix x})$ expresses that there is a single $1$ on the trace assigned to $\pix x$, i.e.\ the path represents a singleton set.
  \item If $\xo$ ranges over natural numbers and $\yt$ over sets of natural
    numbers, $\tr {\xo \in \yt} = \F(1_{\pix \xo} \land 1_{\pix \yt})$.
  \item $\tr {\xo < \yo} = \F(1_{\pix \xo} \land \X \F 1_{\pix \yo})$.
  \item $\tr {\xo \cdot \yo = \zo} = \exists \pi.\ (\X \add_\pi) \land \F(\argl_\pi \land 1_{\pix \xo}) \land
    \F(\argr_{\pi} \land 1_{\pix \yo}) \land \F(\res_\pi \land 1_{\pix \zo})$,
    and $\tr {\xo+\yo = \zo} =
    \exists \pi.\ (\X \mult_\pi) \land \F(\argl_\pi \land 1_{\pix \xo}) \land
    \F(\argr_{\pi} \land 1_{\pix \yo}) \land \F(\res_\pi \land 1_{\pix \zo})$.
  \end{itemize}

  If $\varphi$ is true in $(\Nat, +, \cdot, <,\in)$,
  then the countable and finitely-branching transition system~$\tsysc$ defined above is a model of $\varphi'$.
  Conversely, if $\tsys \models \varphi'$ for some transition system $\tsys$,
  then for all sets $A \subseteq \Nat$ there is a path $\rho_A$ matching $A$ in $\tsys$ and trace quantification in $\tsys$ mimics first- and second-order in $(\Nat, +, \cdot, <,\in)$.
  Thus, $\varphi$ is true in $(\Nat, +, \cdot, <,\in)$.
\end{proof}

Note that the preceding proof shows that even \hyctlstar bounded-branching satisfiability is equivalent to truth in second-order arithmetic, i.e., the question whether a given sentence is satisfied by a transition system where each vertex has at most $k$ successors, for some uniform $k \in \nats$.
