Most classical temporal logics like \ltl and \ctlstar refer to a single execution trace at a time while information-flow properties, which are crucial for security-critical systems,
require reasoning about multiple executions of a system. 
Clarkson and Schneider~\cite{ClarksonS10} coined the term \emph{hyperproperties} for such properties which, structurally, are sets \emph{of sets} of traces.
Just like ordinary trace and branching-time properties, hyperproperties can be specified using temporal logics, e.g.\ \hyltl and \hyctlstar~\cite{ClarksonFKMRS14}, expressive, but intuitive specification languages that are able to express typical information-flow properties such as noninterference, noninference, declassification, and input determinism.
Due to their practical relevance and theoretical elegance, hyperproperties and their specification languages have received considerable attention during the last decade~\cite{DBLP:conf/atva/AbrahamBBD20,DBLP:conf/qest/AbrahamB18,DBLP:conf/vmcai/BartocciFHNC22,DBLP:conf/cav/BaumeisterCBFS21,BerardHH18,DBLP:conf/concur/BeutnerF21,DBLP:conf/lics/BozzelliPS21,DBLP:conf/concur/BozzelliPS22,ClarksonFKMRS14,ClarksonS10,CoenenFHH19,DimitrovaFT20,FZ17,GutsfeldMO20,DBLP:journals/pacmpl/GutsfeldMO21,HO2020104639,DBLP:conf/mfcs/KrebsMV018,DBLP:conf/fsttcs/VirtemaHFK021}.

\hyltl is obtained by extending \ltl~\cite{Pnueli77}, the most influential specification language for linear-time properties, by trace quantifiers to refer to multiple executions of a system. 
For example, the \hyltl formula
\[\forall \pi, \pi'.\ \G( i_\pi \leftrightarrow i_{\pi'}) \rightarrow \G (o_\pi \leftrightarrow o_{\pi'}) \]
expresses input determinism, i.e.\ every pair of traces that always has the same input (represented by the proposition~$i$) also always has the same output (represented by the proposition~$o$). 
Similarly, \hyctlstar is the extension of the branching-time logic \ctlstar~\cite{EmersonH86} by path quantifiers. 
\hyltl only allows formulas in prenex normal form while \hyctlstar allows arbitrary quantification, in particular under the scope of temporal operators. 
Consequently, \hyltl formulas are evaluated over sets of traces while \hyctlstar formulas are evaluated over transition systems, which yield the underlying branching structure of the traces. 

All basic verification problems, e.g.\ model checking~\cite{DBLP:conf/cav/BeutnerF22,DBLP:journals/corr/abs-2301-11229,Finkbeiner21,FinkbeinerRS15}, runtime monitoring~\cite{AgrawalB16,BonakdarpourF16,BrettSB17,DBLP:conf/atva/CoenenFHHS21,FinkbeinerHST18}, and synthesis~\cite{BonakdarpourF20,FinkbeinerHHT20,FinkbeinerHLST20}, have been studied. Most importantly, \hyctlstar model checking over finite transition systems is TOWER-complete\cite{FinkbeinerRS15}, even for a fixed transition system~\cite{MZ20}. 
However, for a small number of alternations, efficient algorithms have been developed and were applied to a wide range of problems, e.g.\ an information-flow analysis of an I2C bus master~\cite{FinkbeinerRS15}, the symmetric access to a shared resource in a mutual exclusion protocol~\cite{FinkbeinerRS15}, and to detect the use of a defeat device to cheat in emission testing~\cite{BartheDFH16}.

But surprisingly, the exact complexity of the satisfiability problems for \hyltl and \hyctlstar is still open. 
Finkbeiner and Hahn proved that \hyltl satisfiability is undecidable~\cite{FinkbeinerH16}, a result which already holds when only considering finite sets of ultimately periodic traces
 and $\forall\exists$-formulas. 
In fact, Finkbeiner et al.\ showed that \hyltl satisfiability restricted to finite sets of ultimately periodic traces is $\Sigma_1^0$-complete~\cite{FHH18} (i.e.\ complete for the set of recursively enumerable problems).
Furthermore, Hahn and Finkbeiner proved that the $\exists^*\forall^*$-fragment has decidable satisfiability~\cite{FinkbeinerH16} while Mascle and Zimmermann studied the \hyltl satisfiability problem restricted to bounded sets of traces~\cite{MZ20}. The latter work implies that \hyltl satisfiability restricted to finite sets of traces (even non ultimately periodic ones) is also $\Sigma_1^0$-complete.
Following up on the results presented in the conference version of this article~\cite{DBLP:conf/mfcs/FortinKT021}, Beutner et al.\ studied satisfiability for safety and liveness fragments of \hyltl~\cite{DBLP:conf/lics/BeutnerCFHK22}.
Finally, Finkbeiner et al.\ developed tools and heuristics~\cite{DBLP:journals/corr/abs-2301-11229,FHH18,FinkbeinerHS17}.

As every \hyltl formula can be turned into an equisatisfiable \hyctlstar formula, \hyctlstar satisfiability is also undecidable. 
Moreover, Rabe has shown that it is even $\Sigma_1^1$-hard~\cite{Rabe16}, i.e.\ it is not even arithmetical.
However, both for \hyltl and for \hyctlstar satisfiability, only lower bounds, but no upper bounds, are known. 

\paragraph{Our Contributions.}
In this paper, we settle the complexity of the satisfiability problems for \hyltl and \hyctlstar
by determining exactly how undecidable they are.
That is, we provide matching lower and upper bounds in terms of the analytical hierarchy and beyond,
where decision problems (encoded as subsets of $\nats$) are classified based on their definability by formulas of higher-order arithmetic, namely by the type of objects one can quantify over and by the number of alternations of such quantifiers.
We refer to Roger's textbook~\cite{Rogers87} for fully formal definitions.
For our purposes, it suffices to recall the following classes.
$\Sigma_1^0$ contains the sets of natural numbers of the form
\[
\set{x \in \nats \mid \exists x_0.\  \cdots \exists x_k.\ \psi(x, x_0, \ldots, x_k)}
\] 
where quantifiers range over natural numbers and $\psi$ is a quantifier-free arithmetic formula.
The notation~$\Sigma_1^0$ signifies that there is a single block of existential quantifiers (the subscript~$1$) ranging over natural numbers (type~$0$ objects, explaining the superscript~$0$).
Analogously, $\Sigma_1^1$ is induced by arithmetic formulas with existential quantification of type~$1$ objects (functions mapping natural numbers to natural numbers) and arbitrary (universal and existential) quantification of type~$0$ objects.
Finally, $\Sigma_1^2$ is induced by arithmetic formulas with existential quantification of type~$2$ objects (functions mapping type~$1$ objects to natural numbers) and arbitrary  quantification of type~$0$ and type~$1$ objects.
So, $\Sigma_1^0$ is part of the first level of the arithmetic hierarchy, $\Sigma_1^1$ is part of the first level of the analytical hierarchy, while $\Sigma_1^2$ is not even analytical.

In terms of this classification, we prove that \hyltl satisfiability is $\Sigma_1^1$-complete while \hyctlstar satisfiability is $\Sigma_1^2$-complete, thereby settling the complexity of both problems
% Our results show that both problems are highly undecidable.
and showing that they are highly undecidable.
In both cases, this is a significant increase of the lower bound and the first upper bound.

First, let us consider \hyltl satisfiability.
The $\Sigma_1^1$ lower bound is a straightforward reduction from the recurrent tiling problem, a standard $\Sigma_1^1$-complete problem asking whether $\nats\times\nats$ can be tiled by a given finite set of tiles.
So, let us consider the upper bound: $\Sigma_1^1$ allows to quantify over type~$1$ objects: functions from natural numbers to natural numbers, or, equivalently, over sets of natural numbers, i.e.\ countable objects.
On the other hand, \hyltl formulas are evaluated over sets of infinite traces, i.e.\ uncountable objects. 
Thus, to show that quantification over type~$1$ objects is sufficient, we need to apply a result of Finkbeiner and Zimmermann proving that every satisfiable \hyltl formula has a countable model~\cite{FZ17}.
Then, we can prove $\Sigma_1^1$-membership by expressing the existence of a model and the existence of appropriate Skolem functions for the trace quantifiers by type~$1$ quantification. 
We also prove that the satisfiability problem remains $\Sigma_1^1$-complete when restricted to ultimately periodic traces, or, equivalently, when restricted to finite traces.

Then, we turn our attention to \hyctlstar satisfiability.
Recall that \hyctlstar formulas are evaluated over (possibly infinite) transition systems, which can be much larger than type~$2$ objects, as the cardinality of type~$2$ objects is bounded by $\contcard$, the cardinality of the continuum.
Hence, to obtain our upper bound on the complexity we need, just like in the case of \hyltl, an upper bound on the size of minimal models of satisfiable \hyctlstar formulas.
To this end, we generalise the proof of Finkbeiner and Zimmermann to \hyctlstar, showing that every satisfiable \hyctlstar formula has a model of size~$\contcard$. 
We also exhibit a satisfiable \hyctlstar formula~$\varphi_\contcard$ whose models all have at least cardinality~$\contcard$, as they have to encode all subsets of $\nats$ by disjoint paths.
Thus, our upper bound~$\contcard$ is tight.

With this upper bound on the cardinality of models, we are able to prove $\Sigma_1^2$-membership of \hyctlstar satisfiability by expressing with type~$2$ quantification the existence of a model and the existence of a winning strategy in the induced model checking game.
The matching lower bound is proven by directly encoding the arithmetic formulas inducing $\Sigma_1^2$ as instances of the \hyctlstar satisfiability problem.
To this end, we use the formula~$\varphi_\contcard$ whose models have for each subset $A \subseteq \nats$ a path encoding $A$.
Now, quantification over type~$0$ objects (natural numbers) is simulated by quantification of a path encoding a singleton set, quantification over type~$1$ objects (which can be assumed to be sets of natural numbers) is simulated by quantification over the paths encoding such subsets, and existential quantification over type~$2$ objects (which can be assumed to be subsets of $\pow{\nats}$) is simulated by the choice of the model, i.e.\ a model encodes $k$ subsets of $\pow{\nats}$ if there are $k$ existential type~$2$ quantifiers.
Finally, the arithmetic operations can easily be implemented in \hyltl, and therefore also in \hyctlstar.

Using variations of these techniques, we also show that \hyctlstar satisfiability restricted to countable or to finitely branching models is equivalent to truth of second-order arithmetic, i.e.\ the question whether a given sentence of second-order is satisfied in the structure~$(\nats, 0,1,+,\cdot,<)$. 
Restricting the class of models makes the problem simpler, but it is still highly-undecidable.

After settling the complexity of satisfiability, we turn our attention to the \hyltl quantifier alternation hierarchy and its relation to satisfiability. 
Rabe remarks that the hierarchy is strict~\cite{Rabe16}.
On the other hand, Mascle and Zimmermann show that every \hyltl formula has a polynomial-time computable equi-satisfiable formula with one quantifier alternation~\cite{MZ20}.
Here, we present a novel proof of strictness by embedding the \fol alternation hierarchy, which is also strict~\cite{CohenB71,Thomas81}. 
We use our construction to prove that for every $n > 0$, deciding whether a given formula is equivalent to a formula with at most $n$ quantifier alternations is $\Pi_1^1$-complete ($\Pi_1^1$ is the co-class of $\Sigma_1^1$, i.e.\ containing the complements of sets in $\Sigma_1^1$).




