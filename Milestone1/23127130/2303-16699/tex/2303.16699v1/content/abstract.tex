Temporal logics for the specification of information-flow properties are able to express relations between multiple executions of a system.
The two most important such logics are HyperLTL and HyperCTL*, which generalise LTL and CTL* by trace quantification. 
It is known that this expressiveness comes at a price, i.e.\ satisfiability is undecidable for both logics. 

In this paper we settle the exact complexity of these problems, showing that both are in fact highly undecidable:
we prove that HyperLTL satisfiability is $\Sigma_1^1$-complete and HyperCTL* satisfiability is $\Sigma_1^2$-complete. 
These are significant increases over the previously known lower bounds and the first upper bounds.
To prove $\Sigma_1^2$-membership for HyperCTL*, we prove that every satisfiable HyperCTL* sentence has a model that is equinumerous to the continuum, the first upper bound of this kind. We also prove this bound to be tight.
Furthermore, we prove that both countable and finitely-branching satisfiability for HyperCTL* are as hard as truth in second-order arithmetic, i.e.\ still highly undecidable.

Finally, we show that the membership problem for every level of the HyperLTL quantifier alternation hierarchy is $\Pi_1^1$-complete.
