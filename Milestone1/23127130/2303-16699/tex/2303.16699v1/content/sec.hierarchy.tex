The number of quantifier alternations in
a formula is a crucial parameter in the complexity of \hyltl model-checking
\cite{FinkbeinerRS15,Rabe16}.
A natural question is then to understand \emph{which} properties
can be expressed with $n$ quantifier alternations, that is, given
a sentence~$\varphi$, determine if there exists an equivalent one with at
most $n$ alternations.
In this section, we show that this problem is in fact exactly as hard as the
\hyltl unsatisfiability problem (which asks whether a \hyltl sentence has no model),
and therefore $\Pi^1_1$-complete. Here, $\Pi_1^1$ is the co-class of $\Sigma_1^1$, i.e.\ it contains the complements of the $\Sigma_1^1$ sets.


\subsection{Definition and strictness of the hierarchy}

Formally, the \hyltl quantifier alternation hierarchy is defined as follows.
Let $\varphi$ be a \hyltl formula. We say that $\varphi$ is a $\Sigma_0$- or
a $\Pi_0$-formula if it is quantifier-free. It is a $\Sigma_n$-formula
if it is of the form~$\varphi = \exists \pi_1.\  \cdots \exists \pi_k.\ \psi$
and $\psi$ is a $\Pi_{n-1}$-formula. It is a $\Pi_n$-formula if
it is of the form~$\varphi = \forall \pi_1.\ \cdots \forall \pi_k.\ \psi$ and
$\psi$ is a $\Sigma_{n-1}$-formula.
We do not require each block of quantifiers to be non-empty, i.e.\ we may have
$k=0$ and $\varphi = \psi$.
Note that formulas in $\Sigma_0 = \Pi_0$ have free variables.
As we are only interested in sentences, we disregard $\Sigma_0 = \Pi_0$ in the following and only consider the levels~$\Sigma_n $ and $ \Pi_n$ for $n>0$.


By a slight abuse of notation, we also let $\Sigma_n$ denote the set of hyperproperties
definable by a $\Sigma_n$-sentence, that is, the set of all 
$L(\varphi) = \{ T \subseteq (2^\AP)^\omega \mid T \models \varphi\}$
such that $\varphi$ is a $\Sigma_n$-sentence of \hyltl.

\begin{thm}[{\cite[Corollary 5.6.5]{Rabe16}}]
  The quantifier alternation hierarchy of \hyltl is strict:
  for all $n > 0$, $\Sigma_n \subsetneq \Sigma_{n+1}$.
\end{thm}

The strictness of the hierarchy also holds if we restrict our attention to
sentences whose models consist of finite sets of traces that end in the suffix~$\emptyset^\omega$, i.e.\ that are essentially finite.

\begin{thm}\label{thm:strictness-finite}
  For all $n > 0$, there exists a $\Sigma_{n+1}$-sentence $\varphi$ of \hyltl
  that is not equivalent to any $\Sigma_{n}$-sentence, and such that for all
  $T \subseteq {(2^{\AP})}^\omega$, if $T \models \varphi$ then $T$ contains
  finitely many traces and $T \subseteq {(2^{\AP})}^\ast\emptyset^\omega$.
\end{thm}

This property 
is a necessary ingredient for our argument that membership
at some fixed level of the quantifier alternation hierarchy is $\Pi_1^1$-hard.
It could be derived from a small adaptation of the proof in~\cite{Rabe16},
and we provide for completeness an alternative proof by exhibiting a connection between the \hyltl quantifier alternation hierarchy and the quantifier alternation hierarchy for first-order logic over finite
words, which is known to be strict~\cite{CohenB71,Thomas82}.
The remainder of the subsection is dedicated to the proof of \autoref{thm:strictness-finite}.



The proof is organised as follows. We first define an encoding of finite words
as sets of traces. We then show that every first-order formula can be
translated into an equivalent (modulo encodings) HyperLTL formula with the
same quantifier prefix.
Finally, we show how to translate back HyperLTL formulas into $\FOw$ formulas
with the same quantifier prefix, so that if the HyperLTL alternation quantifier
hierarchy collapsed, then so would the hierarchy for $\FOw$.


\paragraph{First-Order Logic over Words.}
Let $\AP$ be a finite set of atomic propositions. A finite word over $\AP$
is a finite sequence $w = w(0)w(1) \cdots w(k)$ with $w(i) \in 2^\AP$ for all
$i$. We let $|w|$ denote the \emph{length} of $w$, and
$\pos(w) = \{0,\ldots,|w|-1\}$ the set of \emph{positions} of $w$.
The set of all finite words over $\AP$ is ${(2^\AP)}^\ast$.

Assume a countably infinite set of variables $\varfo$.
The set of $\FOw$ formulas is given by the grammar
\[
  \varphi ::= a(x) \mid x \le y \mid\lnot \varphi
  \mid \varphi \lor \varphi \mid  \exists x.\ \varphi \mid \forall x.\ \varphi \, ,
\]
where $a \in \AP$ and $x,y \in \varfo$.
The set of free variables of $\varphi$ is denoted $\Free(\varphi)$.
A sentence is a formula without free variables.

The semantics is defined as follows, $w \in {(2^\AP)}^\ast$ being a finite word
and $\nu : \Free(\varphi) \to \pos(w)$ an interpretation mapping variables to
positions in $w$:
\begin{itemize}
\item $(w, \nu) \models a(x)$ if $a \in w(\nu(x))$.
\item $(w, \nu) \models x \le y$ if $\nu(x) \le \nu(y)$.
\item $(w, \nu) \models \lnot \varphi$ if $w, \nu \not\models \varphi$.
\item $(w, \nu) \models \varphi \lor \psi$ if $w, \nu \models \varphi$ or
  $(w, \nu) \models \psi$.
\item $(w, \nu) \models \exists x.\ \varphi$ if there exists a position
  $n \in \pos(w)$ such that $(w,\nu[x \mapsto n]) \models \varphi$.
  \item $(w, \nu) \models \forall x.\ \varphi$ if for all positions
  $n \in \pos(w)$: $(w,\nu[x \mapsto n]) \models \varphi$.
\end{itemize}
If $\varphi$ is a sentence, we write
$w \models \varphi$ instead of $(w,\nu) \models \varphi$.

As for \hyltl, a $\FOw$ formula in prenex normal form is a $\Sigma_n$-formula if its quantifier
prefix consists of $n$ alternating blocks of
quantifiers (some of which may be empty), starting with a block of
existential quantifiers.
We let $\Sfo n$ denote the class of languages of finite words definable
by $\Sigma_n$-sentences.


\begin{thm}[\cite{Thomas82,CohenB71}]\label{thm:strictness-FO}
  The quantifier alternation hierarchy of $\FOw$ is strict: for all
  $n \ge 0$, $\Sfo {n} \subsetneq \Sfo {n+1}$.
\end{thm}


\paragraph{Encodings of Words}
The idea is to encode a word $w \in {(2^\AP)}^\ast$ as a set of traces
$T$ where each trace in $T$ corresponds to a position in $w$;
letters in the word are reflected in the label of the first
position of the corresponding trace in $T$, while the total order $<$ is
encoded using a fresh proposition $o \notin \ap$. More precisely, each trace
has a unique position labelled~$o$, distinct from one trace to another,
and traces are ordered according to the order of appearance of
the proposition $o$.
Note that there are several possible encodings for a same word, and we may fix
a canonical one when needed.
This is defined more formally below.

A \emph{stretch function} is a monotone funtion~$f: \Nat \to  \Nat\setminus\set{0}$, i.e.\ it satisfies $0 < f(0) < f(1) < \cdots$.
For all words $w \in {(2^{\AP})}^\ast$ and stretch functions~$f$, we define the set of traces~$\enc w f = \{t_n \mid n \in \pos(w)\} \subseteq (2^{\AP \cup \{o\}})^\ast\emptyset^\omega$
as follows: for all $i \in \Nat$,
\begin{itemize}
\item for all $a \in \AP$,
  $a \in t_n(i)$ if and only if $i = 0$ and $a \in w(n)$
\item $o \in t_n(i)$ if and only if $i = f(n)$.
\end{itemize}

It will be convenient to consider encodings with arbitrarily large spacing
between $o$'s positions.
To this end, for every $N \in \Nat$, we define a particular encoding
\[
  \encn w N = \enc w {n \mapsto N(n+1)} \, .
\]
So in $\encn w N$, two positions with non-empty labels are at distance at
least $N$ from one another.

Given $T = \enc w f$ and a trace assignment
$\Pi \colon \var \rightarrow T$,
we let $\tn T N = \encn w N$, and
$\nun \Pi N \colon \var \rightarrow \tn T N$ the trace assignment defined
by shifting the $o$ position in each $\Pi(\pi)$ accordingly, i.e.
\begin{itemize}
	\item $o \in \Pi^{(N)}(\pi) (N(i+1))$ if and only if $o \in \Pi(\pi)(f(i))$ and
	\item for all $a \in \ap$: $a \in \Pi^{N}(\pi)(0)$ if and only if $a \in \Pi(\pi)(0)$.
\end{itemize}


\paragraph{From FO to HyperLTL}
We associate with every $\FOw$ formula $\varphi $ in prenex normal form
a \hyltl formula $\encPhi \varphi$ over $\ap \cup \{o\}$ by replacing
in $\varphi$:
\begin{itemize}
\item $a(x)$ with $a_x$, and
\item $x \le y$ with $\F(o_x \land \F o_y)$. 
\end{itemize}
In particular, $\encPhi \varphi$ has the same quantifier prefix as $\varphi$, which means that we treat variables of $\varphi$ as trace variables of $\encPhi{\varphi}$.

\begin{lem}\label{lem:FO-to-hyltl}
  For every $ \FOw$ sentence $\varphi $ in prenex normal form,
  $\varphi$ is equivalent to $\encPhi \varphi$ in the following sense:
  for all $w \in {(2^{\AP})}^\ast$ and all stretch functions~$f$,
  \[
    w \models \varphi \quad\text{if and only if}\quad
    \enc w f \models \encPhi \varphi \, .
  \]
\end{lem}

\begin{proof}
By induction over the construction of $\varphi$, relying on the fact that traces in $\enc w f$ are in bijection with positions in $w$.
\end{proof}

In particular, note that the evaluation of $\encPhi \varphi$ on $\enc w f$ does
not depend on $f$. We call such a formula \emph{stretch-invariant}:
a \hyltl sentence  $\varphi$ is \emph{stretch-invariant} if for all finite words $w$ and all stretch functions~$f$ and $g$,
\[
  \enc w f \models \varphi \quad\text{ if and only if }\quad
  \enc w g \models \varphi \, .
\]


\begin{lem}\label{lem:stretchinv}
  For all $\varphi \in \FOw$, $\encPhi \varphi$ is stretch-invariant.
\end{lem}

\begin{proof}
By induction over the construction of $\encPhi{\varphi}$, relying on the fact that the only temporal subformulas of $\encPhi{\varphi}$ are of the form~$\F(o_x \land \F o_y)$.
\end{proof}


\paragraph{Going Back From HyperLTL to FO}
Let $\simpleHyperLTL$ denote the fragment of \hyltl consisting of
all formulas $\encPhi \varphi$, where $\varphi$ is an $\FOw$ formula in
prenex normal form.
Equivalently, $\psi \in \simpleHyperLTL$ if it is a HyperLTL formula of the
form $\psi = Q_1x_1 \cdots Q_kx_k.\ \psi_0$, where $\psi_0$ is a Boolean
combination of formulas of the form  $a_x$ or $\F(o_x \land \F o_y)$.

Let us prove that every \hyltl sentence is equivalent, over
sets of traces of the form~$\enc w f$, to a sentence in $\simpleHyperLTL$
with the same quantifier prefix.
This means that if a \hyltl sentence $\encPhi \varphi$ is equivalent to a \hyltl
sentence with a smaller number of quantifier alternations, then it is also
equivalent over all word encodings to one of the form $\encPhi \psi$, which
in turns implies that the $\FOw$ sentences $\varphi$ and $\psi$ are equivalent.

The \emph{temporal depth} of a quantifier-free formula in \hyltl
is defined inductively as
\begin{itemize}\label{defdepth}
\item 	$\depth(a_\pi) = 0$,
\item  $\depth(\lnot \varphi) = \depth(\varphi)$,
\item $\depth(\varphi \lor \psi) = \max(\depth(\varphi),\depth(\psi))$,
\item $\depth(\X \varphi) = 1 + \depth(\varphi)$, and
\item $\depth(\varphi \U \psi) = 1 + \max(\depth(\varphi,\psi))$.
\end{itemize}
For a general \hyltl formula $\varphi = Q_1 \pi_1 \cdots Q_k \pi_k.\ \psi$,
we let $\depth(\varphi) = \depth(\psi)$.

\begin{lem}\label{lem:QFsimpl}
  Let $\psi$ be a quantifier-free formula of \hyltl.
  Let $N = \mathit{depth}(\psi)+1$.
  There exists a quantifier-free formula $\simpl \psi \in \simpleHyperLTL$
  such that for all $T = \enc w f$ and trace assignments $\Pi$,
  \[
    (\tn T N, \nun \Pi N) \models \psi \quad\text{if and only if}\quad (T,\Pi) \models \simpl \psi
     \, .
  \]
\end{lem}

\begin{proof} 
  Assume that $\Free(\psi) = \{\pi_1, \ldots, \pi_k\}$ is the set of free variables of $\psi$.
  Note that the value of $(\tn T N, \nun \Pi N) \models \psi$ depends
  only on the traces $\nun \Pi N(\pi_1),\ldots,\nun \Pi N(\pi_k)$.
  We see the tuple $(\nun \Pi N(\pi_1),\ldots,\nun \Pi N(\pi_k))$ as a single
  trace $w_{T,\Pi,N}$ over the set of propositions
  $\AP' = \{a_{\pi} \mid a \in \AP\cup\set{o} \land \pi \in \Free(\psi)\}$,
  and $\psi$ as an LTL formula over $\AP'$.
  
  We are going to show that the evaluation of $\psi$ over words $w_{T,\Pi,N}$ is
  entirely determined by the ordering of $o_{\pi_1}, \ldots, o_{\pi_n}$ in
  $w_{T,\Pi,N}$ and the label of $w_{T,\Pi,N}(0)$, which we can both
  describe using a formula in $\simpleHyperLTL$.
  The intuition is that non-empty labels in $w_{T,\Pi,N}$ are
  at distance at least $N$ from one another, and  a temporal formula of depth
  less than $N$ cannot distinguish between $w_{T,\Pi,N}$ and other words
  with the same sequence of non-empty labels and sufficient spacing between
  them.
  More generally, the following can be easily proved via
  Ehrenfeucht-Fra\"iss\'e games:

  \begin{clm}\label{claim:EF}
    Let $m,n \ge 0$, $(a_i)_{i \in \Nat}$ be a sequence of letters in
    $2^{\AP'}$, and
    \[
      w_1, w_2 \in
      \emptyset^m a_0 \emptyset^{n}\emptyset^\ast
      a_1 \emptyset^{n}\emptyset^\ast a_2 \emptyset^{n}\emptyset^\ast \cdots
    \]
    Then for all LTL formulas $\varphi$ such that $\depth(\varphi) \le n$,
    $w_1 \models \varphi$ if and only if $w_2 \models \varphi$.
  \end{clm}


  Here we are interested in words of a particular shape.
  Let $L_N$ be the set of infinite words $w \in {(2^{\AP'})}^\omega$ such that:
  \begin{itemize}
  \item For all $\pi \in \{\pi_1,\ldots,\pi_k\}$, there is a unique
    $i \in \Nat$ such that $o_{\pi} \in w(i)$. Moreover, $i \ge N$.
  \item If $o_{\pi} \in w(i)$ and $o_{\pi'} \in w(i') $,
    then $|i-i'| \ge N$ or $i = i'$.
  \item If $a_{\pi} \in w(i)$ for some $a \in \AP$ and
    $\pi \in \{\pi_1,\ldots,\pi_k\}$, then $i = 0$.
  \end{itemize}
  Notice that $w_{T,\Pi,N} \in L_N$ for all $T$ and all $\Pi$.
  
  For $w_1, w_2 \in L_N$, we write $w_1 \sim w_2$ if $w_1$ and $w_2$ differ
  only in the spacing between non-empty positions, that is, if there are
  $\ell \le k$ and $a_0, \ldots, a_\ell \in 2^{\AP'}$ such that
  $w_1,w_2 \in a_0 \emptyset^\ast a_1 \emptyset^\ast \cdots a_\ell \emptyset^\omega$.    
  Notice that $\sim$ is of finite index.
  Moreover, we can distinguish between its equivalence classes using formulas
  defined as follows. For all
  $A \subseteq \{a_{\pi} \mid a \in \AP \land \pi \in \{\pi_1,\ldots,\pi_k\}\}$
  and all total preorders $\preceq$ over $\{\pi_1, \ldots, \pi_k\}$,\footnote{
    i.e.\ $\preceq$ is required to be transitive and for all
    $\pi,\pi' \in \{\pi_1, \ldots, \pi_k\}$, we have $\pi \preceq \pi'$
    or $\pi' \preceq \pi$ (or both)} we let
  \[
    \varphi_{A,\preceq} =
    \bigwedge_{a \in A} a \land \bigwedge_{a \notin A} \lnot a
    \land \bigwedge_{\pi_i \preceq \pi_j} \F(o_{\pi_i} \land \F o_{\pi_j})
    \, .
  \]
  Note that every word $w \in L_N$ satisfies exactly one formula
  $\varphi_{A,\preceq}$, and that all words in an equivalence class
  satisfy the same one.
  We denote by $L_{A,\preceq}$ the equivalence class of $L_N/{\sim}$
  consisting of words satisfying $\varphi_{A,\preceq}$.
  So we have $L_N = \biguplus L_{A,\preceq}$.

  Since $\psi$ is of depth less than $N$, by \autoref{claim:EF}
  (with $n = N-1$ and $m = 0$), for all $w_1 \sim w_2$ we have
  $w_1 \models \psi$ if and only if $w_2 \models \psi$.
  Now, define $\simpl \psi$ as the disjunction of all $\varphi_{A,\preceq}$
  such that $\psi$ is satisfied by elements in the class $L_{A,\preceq}$.
  Then $\simpl \psi \in \simpleHyperLTL$, and
  \[
    \text{for all } w \in L_N, \quad w \models \simpl \psi
    \text{ if and only if } w \models \psi \, .
  \]
  In particular, for every $T$ and every $\Pi$, we have
  $(\tn T N,\nun \Pi N) \models \simpl \psi$ if and only if 
  $(\tn T N,\nun \Pi N) \models \psi$.
  Since the preorder between propositions $o_\pi$ and the label of the initial
  position are the same in $(\tn T N,\nun \Pi N)$ and $(T,\Pi)$,
  we also have $(T,\Pi) \models \simpl \psi$ if and only if
  $(\tn T N,\nun \Pi N) \models \simpl \psi$.
  Therefore,
  \[
(\tn T N, \nun \Pi N) \models \psi \quad\text{if and only if}\quad     (T,\Pi) \models \simpl \psi
     \, . \qedhere
  \]
\end{proof}

For a quantified \hyltl sentence $\varphi = Q_1\pi_1 \cdots Q_k\pi_k.\ \psi$, we let
$\simpl \varphi = Q_1\pi_1 \ldots Q_k\pi_k.\ \simpl \psi$, where $\simpl \psi$ is
the formula obtained through \autoref{lem:QFsimpl}.

\begin{lem}\label{lem:Qsimpl}
  For all \hyltl formulas~$\varphi $,
  for all $T = \enc w f$ and trace assignments~$\Pi$,
  \[
 (\tn T N, \nun \Pi N) \models \varphi    \text{ if and only if }
    (T, \Pi) \models \simpl \varphi \, ,
  \]
  where $N = \mathit{depth}(\varphi)+1$.
\end{lem}
\begin{proof}
  We prove the result by induction. Quantifier-free formulas are covered by \autoref{lem:QFsimpl}. 
  We have
  \begin{align*}
    (T, \Pi) \models \exists \pi.\ \simpl \psi
    & \quad\Leftrightarrow\quad
      \exists t \in T \text{ such that }
      (T,\Pi[\pi \mapsto t]) \models \simpl \psi \\
    & \quad\Leftrightarrow\quad
      \exists t \in T \text{ such that }
      (\tn T N, \nun {(\Pi[\pi \mapsto t])} N)
      \models \psi && \text{(IH)} \\
    & \quad\Leftrightarrow\quad
        \exists t \in \tn T N \text{ such that }
      (\tn T N, \nun {\Pi} N[\pi \mapsto t])
      \models \psi \\
    & \quad\Leftrightarrow\quad
        (\tn T N, \nun \Pi N) \models \exists \pi.\ \psi \, ,
  \end{align*}
  and similarly,
  \begin{align*}
    (T, \Pi) \models \forall \pi.\ \simpl \psi
    & \quad\Leftrightarrow\quad
      \forall t \in T, \text{we have }
      (T,\Pi[\pi \mapsto t]) \models \simpl \psi \\
    & \quad\Leftrightarrow\quad
      \forall t \in T, \text{we have }
      (\tn T N, \nun {(\Pi[\pi \mapsto t])} N)
      \models \psi && \text{(IH)} \\
    & \quad\Leftrightarrow\quad
        \forall t \in \tn T N, \text{we have }
      (\tn T N, \nun \Pi N [\pi \mapsto t])
      \models \psi \\
    & \quad\Leftrightarrow\quad
        (\tn T N, \nun \Pi N) \models \forall \pi.\ \psi \, . && \qedhere
  \end{align*}
\end{proof}

As a consequence, we obtain the following equivalence.

\begin{lem}\label{lem:eqsimpl}
  For all stretch-invariant \hyltl sentences $\varphi$ and
  for all $T = \enc w f$,
  \[
    T \models \varphi \quad\text{if and only if}\quad
    T \models \simpl \varphi \, .
  \]
\end{lem}

\begin{proof}
  By definition of $\varphi$ being stretch-invariant, we have
  $T \models \varphi$ if and only if
  $\tn T N \models \varphi$, which by \autoref{lem:Qsimpl} is
  equivalent to $T \models \simpl \varphi$.
\end{proof}


We are now ready to prove the strictness of the \hyltl quantifier alternation
hierarchy.

\begin{proof}[Proof of \autoref{thm:strictness-finite}.]
  Suppose towards a contradiction that the hierarchy collapses at
  level~$n > 0$, i.e.\ every \hyltl $\Sigma_{n+1}$-sentence is equivalent to some
  $\Sigma_n$-sentence.
  Let us show that the $\FOw$ quantifier alternation hierarchy also
  collapses at level $n$, a contradiction with \autoref{thm:strictness-FO}.

  Fix a $\Sigma_{n+1}$-sentence $\varphi$ of $\FOw$.
  The \hyltl sentence $\encPhi \varphi$ has the same quantifier prefix as
  $\varphi$, i.e.\ is also a $\Sigma_{n+1}$-sentence.
  Due to the assumed hierarchy collapse, there exists a  \hyltl
  $\Sigma_n$-sentence $\psi$ that is equivalent to $\encPhi \varphi$,
  and is stretch-invariant by \autoref{lem:stretchinv}.
  Then the \hyltl sentence $\simpl \psi$ defined above is also a $\Sigma_n$-sentence.
  Moreover, since $\simpl \psi \in \simpleHyperLTL$, there exists
  a $\FOw$ sentence $\varphi'$ such that $\simpl \psi = \encPhi {\varphi'}$,
which has the same quantifier prefix as $\simpl \psi$, i.e.\
  $\varphi'$ is a $\Sigma_n$-sentence of $\FOw$.
  For all words $w \in (2^\AP)^\ast$, we now have
  \begin{align*}
    w \models \varphi
    & \quad\text{if and only if}\quad \enc w f \models \encPhi \varphi
    && \text{(\autoref{lem:FO-to-hyltl})} \\
    & \quad\text{if and only if}\quad \enc w f \models \psi
    && \text{(assumption)} \\
    & \quad\text{if and only if}\quad \enc w f \models \simpl \psi
    && (\text{\autoref{lem:eqsimpl} and \autoref{lem:stretchinv}}) \\
    & \quad\text{if and only if}\quad \enc w f \models \encPhi {\varphi'}
    && (\text{definition}) \\
    & \quad\text{if and only if}\quad w \models \varphi'
    && \text{(\autoref{lem:FO-to-hyltl})}
  \end{align*}
  for an arbitrary stretch function~$f$.
  Therefore, $\Sfo {n+1} = \Sfo n$,  yielding the desired contradiction.

  This proves not only that for all $n > 0$, there is a \hyltl $\Sigma_{n+1}$-sentence
  that is not equivalent to any $\Sigma_n$-sentence,
  but also that there is one of the form $\encPhi \varphi$.
  Now, the proof still goes through if we replace $\encPhi \varphi$ by
  any formula equivalent to $\encPhi \varphi$ over all $\enc{w}{f}$,
  and in particular if we replace $\encPhi \varphi$ by
  $\encPhi \varphi \land \psi$, where the sentence
  \[
    \psi = \exists \pi.\ \forall \pi'.\
    (\F \G \emptyset_\pi) \land
    \G( \G \emptyset_\pi \rightarrow  \G \emptyset_{\pi'})  \]
    with $
    \emptyset_\pi = \bigwedge_{a \in \AP} \lnot a_\pi$
  selects models that contain finitely many traces, all in
  $(2^\AP)^\ast \cdot \emptyset^\omega$.
  Indeed, all $\enc w f$ satisfy~$\psi$.
  Notice that $\psi$ is a $\Sigma_2$-sentence, and since $n+1 \ge 2$,
  (the prenex normal form of) $\encPhi \varphi \land \psi$ is still a $\Sign {n+1}$-sentence.
\end{proof}












\subsection{Membership problem}

In this subsection, we investigate the complexity of the membership problem for the \hyltl quantifier alternation hierarchy.
Our goal is to prove the following result.

\begin{thm}\label{thm:Pi11}
  Fix $n > 0$. The problem of deciding whether a given \hyltl sentence is equivalent
  to some $\Sigma_n$-sentence is $\Pi^1_1$-complete.
\end{thm}

The easier part of the proof will be the upper bound, since a corollary of
\autoref{thm:hyltl-sat} is that the problem of deciding whether two
\hyltl formulas are equivalent is $\Pi^1_1$-complete.

The lower bound will be proven by a reduction from the \hyltl unsatisfiability problem.
The proof relies on Theorem~\ref{thm:strictness-finite}: given a sentence
$\varphi$, we are going to combine $\varphi$ with some $\Sigma_{n+1}$-sentence
$\varphi_{n+1}$ witnessing the strictness of the hierarchy, to construct a sentence $\psi$
such that $\varphi$ is unsatisfiable if and only if $\psi$ is equivalent to
a $\Sigma_n$-sentence.
Intuitively, the formula $\psi$ will describe models consisting of the
``disjoint union'' of a model of $\varphi_{n+1}$ and a model of~$\varphi$.
Here ``disjoint'' is to be understood in a strong sense: we split both
the set of traces and the time domain into two parts, used respectively to
encode the models of $\varphi_{n+1}$ and those of $\varphi$.

\begin{figure}
\centering
  \begin{tikzpicture}[yscale=0.6,font=\small]
    \node (03) at (0,3) {$\{a,b\}$};
    \node (13) at (1,3) {$\{a\}$};
    \node (23) at (2,3) {$\{a\}$};
    \node (02) at (0,2) {$\{b\}$};
    \node (12) at (1,2) {$\emptyset$};
    \node (22) at (2,2) {$\{a\}$};
    \begin{pgfonlayer}{background}
      \draw[teal,fill,fill opacity = 0.2,semithick]
      (03.north west) rectangle (8.5,1.6);
    \end{pgfonlayer}
    \node[teal] at (-1,2.5) {\large $T_\ell$} ;
    
    \node (31) at (3,1) {$\{a\}$};
    \node (41) at (4,1) {$\{a\}$};
    \node (51) at (5,1) {$\{a,b\}$};
    \node (61) at (6,1) {$\emptyset$};
    \node (71) at (7,1) {$\{a\}$};
    \node (81) at (8,1) {${\cdots}$};
    \node (30) at (3,0) {$\{b\}$};
    \node (40) at (4,0) {$\{a\}$};
    \node (50) at (5,0) {$\{b\}$};
    \node (60) at (6,0) {$\{a,b\}$};
    \node (70) at (7,0) {$\{a\}$};
    \node (80) at (8,0) {${\cdots}\vphantom{\{a\}}$};
    \begin{pgfonlayer}{background}
      \draw[brown,fill, fill opacity = 0.2,semithick]
      (31.north west) rectangle (8.5,-0.4) ;
      \node[brown] at (9,0.5) {\large $T_r$} ;
    \end{pgfonlayer}

    \foreach \i in {0,...,2} {
      \foreach \j in {0,1} {
        \node (ij) at (\i,\j) {$\{ \$ \}$} ;
      }
    }
    \foreach \i in {3,...,7} {
      \foreach \j in {2,3} {
        \node (ij) at (\i,\j) {$\{ \$ \}$} ;
      }
    }
    \node at (8,3) {${\cdots}$};
    \node at (8,2) {${\cdots}$};
  \end{tikzpicture}
  \caption{Example of a split set of traces where
  each row represents a trace and $b=3$.
      \label{fig:split}}
\end{figure}


\smallskip

To make this more precise, let us introduce some notations.
We assume a distinguished symbol $\$ \notin \AP$.
We say that a set of traces $T \subseteq {(2^{\AP\cup \{\$\}})}^\omega$ is
\emph{bounded} if there exists $b \in \Nat$ such that
$T \subseteq {(2^{\AP})}^b \cdot \{\$\}^\omega$.


\begin{lem}\label{lemma:bounded}
  There exists a  $\Pi_1$-sentence $\phib$ 
  such that for all $T \subseteq {(2^{\AP\cup \{\$\}})}^\omega$, we have
  $T \models \phib$ if and only if $T$ is bounded.
\end{lem}

\begin{proof}
  We let
    \begin{align*}
    \phib =
    \forall \pi, \pi'.\
    & (\lnot \$_\pi \U \G \$_\pi) \land
      \bigwedge_{a \in \AP} \G(\lnot (a_\pi \land \$_\pi)) \land
      \F \left(
      \lnot \$_\pi \land \lnot \$_{\pi'} \land \X \$_\pi \land \X \$_{\pi'}
      \right) \, .
    \end{align*}
    The conjunct $(\lnot \$_\pi \U \G \$_\pi) \land
    \bigwedge_{a \in \AP} \G(\lnot (a_\pi \land \$_\pi))$
    ensures that every trace is in ${(2^{\AP})}^\ast \cdot \{\$\}^\omega$,
    while
    $\F \left(
      \lnot \$_\pi \land \lnot \$_{\pi'} \land \X \$_\pi \land \X \$_{\pi'}
    \right)$
    ensures that the $\$$'s in any two traces $\pi$ and $\pi'$ start at the same
    position.
\end{proof}

We say that a nonempty set~$T$ of traces is \emph{split} if there exist a $b \in \Nat$ and $T_1$, $T_2$
such that $T = T_1 \uplus T_2$,
$T_1 \subseteq {(2^{\AP})}^b \cdot \{\$\}^\omega$, and
$T_2 \subseteq \{\$\}^b \cdot {(2^{\AP})}^\omega$.
Note that $b$ as well as $T_1$ and $T_2$ are unique then.
Hence, we define the left and right part of $T$ as $\Tl = T_1$ and
$\Tr  = \{t \in {(2^{\AP})}^\omega \mid \{\$\}^b \cdot t \in T_2\}$, respectively
(see \autoref{fig:split}).


It is easy to combine \hyltl specifications for the left and right part
of a split model into one global formula.

\begin{lem}
    \label{lem:dupsi}
  For all \hyltl sentences~$\varphi_\ell, \varphi_r$, one can construct
  a sentence $\psi$ such that for all split 
  $T \subseteq {(2^{\AP\cup \{\$\}})}^\omega$,
  it holds that   $\Tl \models \varphi_\ell$ and $\Tr  \models \varphi_r$ if and only if $T \models \psi$.
\end{lem}

\begin{proof}
  Let $\widehat {\varphi_r}$ denote the formula obtained from $\varphi_r$
  by replacing:
  \begin{itemize}
  \item every existential quantification $\exists \pi.\ \varphi$ with
    $\exists \pi.\ ((\F\G \lnot \$_\pi) \land \varphi)$;
  \item every universal quantification $\forall \pi.\ \varphi$ with
    $\forall \pi.\ ((\F\G \lnot \$_\pi) \rightarrow \varphi)$;
  \item the quantifier-free part $\varphi$ of $\varphi_r$ with
    $\$_\pi \U (\lnot \$_\pi \land \varphi)$,
    where $\pi$ is some free variable in $\varphi$.
  \end{itemize}
 Here, the first two replacements restrict quantification to traces in the right part while the last one requires the formula to hold at the first position of the right part. 
  We define $\widehat {\varphi_\ell}$ by similarly relativizing quantifications
  in $\varphi_\ell$.
  The formula $\widehat {\varphi_\ell} \land \widehat {\varphi_r}$ can then
  be put back into prenex normal form to define $\psi$.
\end{proof}


Conversely, any \hyltl formula that only has split models can be decomposed
into a Boolean combination of formulas that only talk about the left or right
part of the model. This is formalised in the lemma below.

\begin{lem}\label{lem:split}
  For all \hyltl $\Sigma_n$-sentences $\varphi$
  there exists a finite family $(\varphi^i_\ell, \varphi^i_r)_i$ of $\Sigma_n$-sentences 
  such that for all 
  split $T \subseteq {(2^{\AP\cup \{\$\}})}^\omega$: 
  $T \models \varphi$ if and only if there is an $i$ with
  $\Tl \models \varphi^i_\ell$ and $\Tr  \models \varphi^i_r$.
\end{lem}

\begin{proof}
  To prove this result by induction, we need to strengthen the statement to make it dual and allow for formulas with free variables.
  We let $\Free(\varphi)$ denote the set of free variables of a formula
  $\varphi$. 
  We prove the following result, which implies \autoref{lem:split}.

\begin{clm}\label{lem:splitgen}
  For all HyperLTL $\Sigma_n$-formulas (resp.\ $\Pin n$-formulas) $\varphi$,
  there exists a finite family of $\Sigma_n$-formulas
  (resp.\ $\Pi_n$-formulas)
  $(\varphi^i_\ell, \varphi^i_r)_{i}$
  such that for all $i$,
  $\Free(\varphi) = \Free(\varphi^i_\ell) \uplus \Free(\varphi^i_r)$,
  and for all split $T$ and $\Pi$: $(T, \Pi) \models \varphi$
  if and only if there exists $i$ such that
  \begin{itemize}
  \item For all $\pi \in \Free(\varphi)$, $\Pi(\pi) \in \Tl$ if and only if
    $\pi \in \Free(\varphi^i_{\ell})$
    (and thus $\Pi(\pi) \in T \setminus \Tl$ if and only if
    $\pi \in \Free(\varphi^i_{r})$).
  \item $(\Tl, \Pi) \models \varphi^i_\ell$;
  \item $(\Tr , \Pi') \models \varphi^i_r$,
    where $\Pi'$ maps every $\pi \in \Free(\varphi^i_r)$ to the trace in
    $\Tr $ corresponding to $\Pi(\pi)$ in $T$
    (i.e.\ $\Pi(\pi) = \set{\$}^b \cdot \Pi'(\pi)$ for some $b$).
  \end{itemize}
\end{clm}

  To simplify, we can assume that the partition of the free variables of
  $\varphi$ into a left and right part is fixed, i.e.\
  we take $\Vl \subseteq \Free(\varphi)$
  and $\Vr = \Free(\varphi) \setminus \Vl$, and we restrict our attention
  to split $T$ and $\Pi$ such that $\Pi(\Vl) \subseteq \Tl$
  and $\Pi(\Vr) \subseteq T \setminus \Tl$.
  The formulas $(\varphi^i_\ell, \varphi^i_r)_{i}$ we are looking
  for should then be such that $\Free(\varphi^i_\ell) = \Vl$ and
  $\Free(\varphi^i_r) = \Vr$.
  If we can define sets of formulas $(\varphi^i_\ell, \varphi^i_r)$ for each
  choice of $\Vl, \Vr$, then the general case is solved by taking the union
  of all of those.
  So we focus on a fixed $\Vl,\Vr$, and prove the result by induction on
  the quantifier depth of $\varphi$.
  
  \paragraph{Base case}
  If $\varphi$ is quantifier-free, then it can be seen as an LTL formula
  over the set of propositions
  $\{a_\pi, \$_\pi \mid \pi \in \Free(\varphi), a \in \AP\}$,
  and any split model of $\varphi$ consistent with $\Vl,\Vr$
  can be seen as a word in $\Sigma_\ell^\ast \cdot \Sigma_r^\omega$,
  where
  \begin{align*}
    \Sigmal & = \big\{ \alpha \cup \{\$_\pi \mid \pi \in \Vr\} \mid 
    \alpha \subseteq {\{a_\pi \mid \pi \in \Vl \land a \in \AP\}} \big\} \text{ and }\\
    \Sigmar & = \big\{ \alpha \cup \{\$_\pi \mid \pi \in \Vl\} \mid 
    \alpha \subseteq {\{a_\pi \mid \pi \in \Vr \land a \in \AP\}} \big\} \, .
  \end{align*}
  Note in particular that $\Sigmal \cap \Sigmar = \emptyset$.
  We can thus conclude by applying the following standard result of
  formal language theory:
  
  \begin{clm}\label{lem:splitLTL}
    Let $L \subseteq \Sigma_1^\ast \cdot \Sigma_2^\omega$, where
    $\Sigma_1 \cap \Sigma_2 = \emptyset$.
    If $L = L(\varphi)$ for some LTL formula $\varphi$, then there exists a finite family~$(\varphi^i_1, \varphi^i_2)_{i}$  of LTL formulas 
    such that $L = \bigcup_{1 \le i \le k} L(\varphi^i_1) \cdot L(\varphi^i_2)$
    and for all $i$, $L(\varphi^i_1) \subseteq \Sigma_1^\ast$
    and $L(\varphi^i_2) \subseteq \Sigma_2^\omega$.
  \end{clm}

  \begin{proof}
    A language is definable in LTL if and only if it is accepted by some
    counter-free automaton~\cite{DG08,Thomas81}.
    Let $\mathcal{A}$ be a counter-free automaton for $L$.
    For every state~$q$ in~$\mathcal{A}$, let
    \begin{align*}
      L^q_1 & = \{w \in \Sigma_1^\ast \mid q_0 \xrightarrow{w} q
              \text{ for some initial state $q_0$}\} \text{ and }\\
      L^q_2 & = \{w \in \Sigma_2^\omega \mid \text{there is an accepting run
              on $w$ starting from $q$}\} \, .
    \end{align*}
    We have $L = \bigcup_q L^q_1\cdot  L^q_2$.
    Moreover, $L^q_1$ and $L^q_2$ are still recognisable by counter-free
    automata, and therefore LTL definable.
  \end{proof}
  
  \paragraph{Case $\varphi = \exists \pi.\ \psi$}
  Let $(\psi^i_{\ell,1},\psi^i_{r,1})$ and $(\psi^i_{\ell,2},\psi^i_{r,2})$
  be the formulas constructed respectively for
  $(\psi,\Vl \cup \{\pi\},\Vr)$ and $(\psi,\Vl,\Vr \cup \{\pi\})$.
  We take the union of all $(\exists \pi.\ \psi^i_{\ell,1},\psi^i_{r,1})$
  and $(\psi^i_{\ell,2},\exists \pi.\ \psi^i_{r,2})$.

  \paragraph{Case $\varphi = \forall \pi.\ \psi$}
  Let $(\xi^i_{\ell},\xi^i_{r})_{1 \le i \le k}$ be the formulas obtained for
  $\exists \pi.\ \lnot \psi$.
  We have ${(T,\Pi) \models \varphi}$ if and only if for all $i$,
  $(\Tl,\Pi) \not\models \xi^i_{\ell}$ or $(\Tr ,\Pi') \not\models \xi^i_r$;
  or, equivalently, if there exists $h: \{1,\ldots,k\} \to  \{\ell,r\}$ such
  that $(\Tl,\Pi) \models \bigwedge_{h(i) = \ell} \lnot \xi^i_{\ell}$ and
  $(\Tr ,\Pi') \models \bigwedge_{h(i) = r} \lnot \xi^i_{r}$.
  Take the family $(\varphi^h_{\ell},\varphi^h_{r})_h$, where
  $\varphi^h_{\ell} = \bigwedge_{h(i) = \ell} \lnot \xi^i_{\ell}$ and
  $\varphi^h_{r} = \bigwedge_{h(i) = r} \lnot \xi^i_{r}$.
  Since $\varphi = \forall \pi.\ \psi$ is a $\Pi_n$-formula,
  the formula $\exists \pi.\ \lnot \psi$ and by induction all
  $\xi^i_{\ell}$ and $\xi^i_{r}$ are $\Sigma_n$-formulas.
  Then all $\lnot \xi^i_{r}$ are $\Pi_n$-formulas, and since
  $\Pi_n$-formulas are closed under conjunction
  (up to formula equivalence), all $\varphi^h_{\ell}$ and $\varphi^h_{r}$
  are $\Pi_n$-formulas as well.
\end{proof}


We are now ready to prove \autoref{thm:Pi11}. 

\begin{proof}[Proof of \autoref{thm:Pi11}]
  The upper bound is an easy consequence of \autoref{thm:hyltl-sat}: Given a \hyltl sentence~$\varphi$, we express the existence of a $\Sigma_n$-sentence~$\psi$ using first-order quantification and encode equivalence of $\psi$ and $\varphi$
via the formula~$(\lnot \varphi \land \psi) \lor (\varphi \land \lnot \psi)$, which is unsatisfiable if and only if $\varphi$ and $\psi$ are equivalent. Altogether, this shows membership in $\Pi_1^1$, as $\Pi_1^1$ is closed under existential first-order quantification (see, e.g.~\cite[Page 82]{hinman}).

  We prove the lower bound by reduction from the unsatisfiability problem for
  \hyltl. So given a \hyltl sentence $\varphi$, we want to
  construct $\psi$ such that $\varphi$ is unsatisfiable if and only if
  $\psi$ is equivalent to a $\Sigma_n$-sentence.

  \smallskip
  
  We first consider the case $n > 1$.
  Fix a $\Sigma_{n+1}$-sentence~$\varphi_{n+1}$ that is in not
  equivalent to any $\Sigma_n$-sentence, and such that every model of
  $\varphi_{n+1}$ is bounded.
  The existence of such a formula is a consequence of
  \autoref{thm:strictness-finite}.
  By \autoref{lem:dupsi}, there exists a computable $\psi$ such that for all split models
  $T$, we have $T \models \psi$ if and only if $\Tl \models \varphi_{n+1}$
  and $\Tr  \models \varphi$.

  First, it is clear that if $\varphi$ is unsatisfiable, then $\psi$ is
  unsatisfiable as well, and thus equivalent to
  $\exists \pi.\ a_\pi \land \lnot a_\pi$, which is a $\Sigma_n$-sentence
  since $n \ge 1$.
  
  Conversely, suppose towards a contradiction that $\varphi$ is satisfiable
  and that $\psi$ is equivalent to some $\Sigma_n$-sentence.
  Let $(\psi^i_\ell,\psi^i_r)_{i}$ be the finite family of $\Sigma_n$-sentences given
  by Lemma~\ref{lem:split} for $\psi$.
  Fix a model $T_\varphi$ of $\varphi$.
  For a bounded $T$, we let $\overline T$ denote the unique split set of traces such that
  $\overline \Tl = T$ and $\overline \Tr  = T_\varphi$.
  For all $T$, we then have $T \models \varphi_{n+1}$ if and only if $T$ is
  bounded and $\overline T \models \psi$.
  Recall that the set of bounded models can be defined by a
  $\Pi_1$-sentence~$\phib$ (\autoref{lemma:bounded}), which is also
  a $\Sigma_n$-sentence since $n > 1$.
  We then have $T \models \varphi_{n+1}$ if and only if $T \models \phib$ and
  there exists $i$ such that $T \models \psi^i_\ell$ and
  $T_\varphi \models \psi^i_r$.
  So $\varphi_{n+1}$ is equivalent to
  \[
    \phib \land \bigvee\nolimits_{i \text{ with } T_{\varphi} \models \psi^i_r}  \psi^i_\ell \, ,
  \]
  which, since $\Sigma_n$-sentences are closed (up to logical equivalence)
  under conjunction and disjunction, is equivalent to a $\Sigma_n$-sentence.
  This contradicts the definition of $\varphi_{n+1}$.

  \smallskip

  We are left with the case $n = 1$.
  Similarly, we construct $\psi$ such that $\varphi$ is unsatisfiable
  if and only if $\psi$ is unsatisfiable, and if and only if
  $\psi$ is equivalent to a $\Sigma_1$-sentence.
  However, we do not need to use bounded or split models here.
  Every satisfiable $\Sigma_1$-sentence has a model with finitely
  many traces.
  Therefore, a simple way to construct $\psi$ so that it is not equivalent to any
  $\Sigma_1$-sentence (unless it is unsatisfiable) is to ensure that every model
  of $\psi$ contains infinitely many traces.

  Let $x \notin \AP$, and
  $T_\omega = \{ \emptyset^n \{x\} \emptyset^\omega \mid n \in \Nat \}$.
  As seen in the proof of \autoref{lemma:hyltl_hardness},
  $T_\omega$ is definable in \hyltl:  There is a sentence
  $\varphi_\omega$ such that $T\subseteq (\pow{\ap \cup\set{x}})^\omega$ is a model of  $\varphi_\omega$ if and only if
  $T = T_\omega$.
  By relativising quantifiers in $\varphi_\omega$ and $\varphi$ to traces
  with or without the atomic proposition~$x$, one can construct a \hyltl
  sentence $\psi$ such that $T \models \psi$ if and only if
  $T_\omega \subseteq T$ and $T \setminus T_\omega \models \varphi$.

  Again, if $\varphi$ is unsatisfiable then $\psi$ is unsatisfiable and
  therefore equivalent to $\exists \pi.\ a_\pi \land \lnot a_\pi$,
  a $\Sigma_1$-sentence.
  Conversely, all models of $\psi$ contain infinitely many traces and therefore,
  if $\psi$ is equivalent to a $\Sigma_1$-sentence then it is unsatisfiable,
  and so is $\varphi$.
\end{proof}
