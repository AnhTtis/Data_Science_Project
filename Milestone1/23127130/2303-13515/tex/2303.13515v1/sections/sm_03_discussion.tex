\balance


\section{Discussion}\label{sec:sm_discussion}

A limiting factor of our method is the reliance on a volume rendering operation to decode the 2D layout feature grid into a 3D feature at each sampled point along the ray. Due to this operation, the rendered output $\ImageLR$ can only be trained at low resolution (32x32), and does not learn to generate detailed textures. (In contrast to NeRF-style models which can use per-ray supervision, we must render a complete image as an input for the discriminator.) We rely on a refinement module to upsample the result and add additional textures, but any refinement in image space is prone to losing 3D consistency.  Our extended triplane variation reduces the computational expense of volume rendering by reducing the capacity of the decoder MLP and increasing the capacity of the feature representation, thus allowing for neural rendering at 64x64 resolution (we find that geometry degrades at higher resolutions) and decreasing reliance on the upsampler. While we did not find improvements when training on rendered patches, improved patch sampling techniques could help in adding more detail to the rendered result~\cite{shorokhodov2022epigraf}.

As our model does not have explicit 3D or aerial supervision, we find that it may generate unnatural or repeating geometry. This can appear as thin mountains, sloping water, or hills of a similar shape but different appearance when sampling from different random noise codes.  

\begin{figure*}[ht!] %
\centering
\begin{subfigure}[b]{0.45\textwidth}
      \includegraphics[width=\textwidth]{img/cam1k_ctrain.png}
      \caption{1K training cameras; training poses}
\end{subfigure}
\begin{subfigure}[b]{0.45\textwidth}
      \includegraphics[width=\textwidth]{img/cam05_ctrain.png}
      \caption{5 training cameras; training poses}
\end{subfigure}
\begin{subfigure}[b]{0.45\textwidth}
      \includegraphics[width=\textwidth]{img/cam1k_ctest.png}
      \caption{1K training cameras; independent test poses}
\end{subfigure}
\begin{subfigure}[b]{0.45\textwidth}
      \includegraphics[width=\textwidth]{img/cam05_ctest.png}
      \caption{5 training cameras; independent test poses}
\end{subfigure}
\vspace{-5pt}
\caption{\small \emph{Adjusting the set of training cameras.} We plot disparity maps corresponding to training with one thousand cameras, and five cameras. (a) With our default setting of one thousand training cameras with camera origins uniformly sampled over the layout feature grid, we find that the model can learn repeating geometry, such that the disparity map generated from the same pose but different latent codes tends to look similar (each row corresponds to the same pose), despite the RGB colors appearing different. (b) With fewer training cameras, the models learns more diversity in the rendered geometry, where again each row corresponds to the same camera pose. (c \& d) However, the model trained with one thousand cameras generalizes better to an independent set of cameras, whereas the model trained with five cameras has a greater frequency to put holes in the decoded landscape (evidenced by completely black disparity maps, or disparity maps that have no nearby content and thus are darker overall) or regions of solid content without sky (evidenced by disparity maps that do not fade to black near the top of each image). We use one thousand training cameras as our default setting, but a more optimal setting may involve fewer training cameras, while still ensuring adequate coverage over the feature grid.
}
\label{fig:sm_expt_cameras}\vspace{-5pt}
\end{figure*}











