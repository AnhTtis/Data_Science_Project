\section{Extended Triplane Variation}\label{sec:sm_triplane}


Instead of decoding the scene from a 2D layout feature grid and height of a 3D point above this layout plane, we also experiment with a model variation that adds vertical support planes parallel to the XY and YZ planes. Thus, the layout features are described by a 2D extended XZ layout feature grid, and sets of orthogonal support planes shown in pink in Fig.~\ref{fig:sm_expt_triplane}-left. Decoding a given 3D point projects the point to the XZ plane, the four nearest vertical planes (two parallel to XY and two parallel to YZ, which are weighted linearly according to the distance of the point from each plane). 

Qualitatively, the triplane model achieves more geometry diversity, with more mountainous terrain compared to the feature layout model. We attribute this to the additional support provided from the vertical feature planes. Additionally, the vertical feature planes allow for a lighter decoding network with higher neural rendering resolution, allowing for faster video rendering and improved temporal consistency (lower one-step consistency error) due to less reliance on a 2D upsampling operation. We show qualitative examples in Fig.~\ref{fig:sm_expt_triplane_qualitative} with video results on our project page, and quantitative evaluations in Tab.~\ref{tab:triplane_1}. Quantitatively, while this extended triplane variation does not output perform the layout model in terms of FID, we hypothesize that the FID may be impacted by two possible factors: first, this model requires inference-time camera height adjustment to avoid intersecting with increased complexity of the generated geometry, and second, interpolation between vertical feature planes qualitatively produces more muted colors compared to the real image distribution.
\begin{figure}[ht!] %
\centering
\includegraphics[width=\linewidth]{img/triplane.pdf}
\caption{\small \emph{Diagram of Extended Triplane Representation.} The extended triplane representation adds a sequence of orthogonal vertical feature planes outlined in pink in addition to the ground plane features outlined in white \textbf{(left)}. Each unit consists of a triplane representation~\cite{chan2022eg3D} generated from three independent generators -- $\Generator_{XY}$, $\Generator_{XZ}$, and $\Generator_{YZ}$ -- tied to the same latent code and mapping network \textbf{(right)}. At inference time, the features of each generator are stitched along the appropriate dimensions using the SOAT procedure~\cite{chong2021stylegan}.}
\label{fig:sm_expt_triplane}\vspace{-3pt}
\end{figure}


We also investigate the impact of using a projected 3D noise pattern as input into the extended triplane upsampler, with results in Tab.~\ref{tab:triplane_2}. While this improves FID and consistency in the layout representation, we find that the benefits of the projected noise are more limited in the extended triplane setting. Adding projected noise offers improvements in FID, but also a small increase in consistency error. Qualitatively, the model outputs are similar with and without the projected noise, perhaps attributed to decreased reliance on the upsampling operation.

\begin{table}[t!]
  \centering
  \resizebox{1.0\linewidth}{!}{
  \begin{tabular}{lccccc}
    \toprule
     \multirow{2}{*}{\bf Model} & 
     \multicolumn{3}{c}{\bf FID} & 
     \multirow{2}{*}{\bf Consistency} & 
     \multirow{2}{*}{\bf \shortstack[c]{\bf Render\\\textbf{Time (s)}}} \\
     \cmidrule(lr){2-4}
     &  $C_\mathrm{train}$ & $C_\mathrm{forward}$ & $C_\mathrm{random}$ &  & \\
    \midrule
    Extended Layout & \bf 21.42 & \bf 26.67 & \bf 23.39 & 3.56 &  8.49 \\ 
    Extended Triplane & 24.47 & 34.89 & 34.76 & \bf 2.29 & \bf 0.16 \\
    \bottomrule
  \end{tabular}
  }
\caption{\small \emph{Extended Layout vs. Extended Triplane}
While the extended layout representation presented in the main paper attains better image quality (lower FID scores), the extended triplane representation offers improved consistency (lower one-step consistency error) and dramatically faster video rendering (as the layout model requires supersampling for video smoothness). We hypothesize that inference-time camera adjustments and interpolation between vertical feature planes may negatively impact FID for the extended triplane model, despite its ability to generate more complex and diverse landscape geometry.
}
\label{tab:triplane_1}
\end{table}

\begin{table}[t!]
  \centering
  \resizebox{0.9\linewidth}{!}{
  \begin{tabular}{lcccc}
    \toprule
     \multirow{2}{*}{\bf Model} & 
     \multicolumn{3}{c}{\bf FID} & 
    \multirow{2}{*}{\bf Consistency} \\
     \cmidrule(lr){2-4}
     &  $C_\mathrm{train}$ & $C_\mathrm{forward}$ & $C_\mathrm{random}$ &  \\
    \midrule
    Without Noise & \bf 24.47 &	34.89 & 34.76 & \bf 2.29 \\ 
    With 3D noise & 25.31 &	\bf 33.30 & \bf 33.28 & 3.06 \\
    \bottomrule
  \end{tabular}
  }
\caption{\small \emph{Effect of 3D Projected Noise} Adding projected noise into the upsampler of the extendable triplane representation offers improvements in FID but is slightly more inconsistent, but still more consistent than the layout model.}
\label{tab:triplane_2}
\end{table}


\begin{figure*}[ht!] %
\centering
\includegraphics[width=\linewidth]{img/triplane_qualitative.pdf}
\caption{\small \emph{Extendable Triplane Visualization.} Qualitative examples of rendering from the extendable triplane representation. This representation results in larger scene and geometry diversity compared to the layout feature representation, with improved 3D consistency. }
\label{fig:sm_expt_triplane_qualitative}\vspace{-3pt}
\end{figure*}


