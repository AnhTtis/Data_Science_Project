\begin{figure*}[ht!] %
\centering
\begin{subfigure}[b]{0.45\textwidth}
      \includegraphics[width=\textwidth]{img/cam1k_ctrain.png}
      \caption{1K training cameras; training poses}
\end{subfigure}
\begin{subfigure}[b]{0.45\textwidth}
      \includegraphics[width=\textwidth]{img/cam05_ctrain.png}
      \caption{5 training cameras; training poses}
\end{subfigure}
\begin{subfigure}[b]{0.45\textwidth}
      \includegraphics[width=\textwidth]{img/cam1k_ctest.png}
      \caption{1K training cameras; independent test poses}
\end{subfigure}
\begin{subfigure}[b]{0.45\textwidth}
      \includegraphics[width=\textwidth]{img/cam05_ctest.png}
      \caption{5 training cameras; independent test poses}
\end{subfigure}
\vspace{-5pt}
\caption{\small \emph{Adjusting the set of training cameras.} We plot disparity maps corresponding to training with one thousand cameras, and five cameras. (a) With our default setting of one thousand training cameras with camera origins uniformly sampled over the layout feature grid, we find that the model can learn repeating geometry, such that the disparity map generated from the same pose but different latent codes tends to look similar (each row corresponds to the same pose), despite the RGB colors appearing different. (b) With fewer training cameras, the models learns more diversity in the rendered geometry, where again each row corresponds to the same camera pose. (c \& d) However, the model trained with one thousand cameras generalizes better to an independent set of cameras, whereas the model trained with five cameras has a greater frequency to put holes in the decoded landscape (evidenced by completely black disparity maps, or disparity maps that have no nearby content and thus are darker overall) or regions of solid content without sky (evidenced by disparity maps that do not fade to black near the top of each image). We use one thousand training cameras as our default setting, but a more optimal setting may involve fewer training cameras, while still ensuring adequate coverage over the feature grid.
}
\label{fig:sm_expt_cameras}\vspace{-5pt}
\end{figure*}