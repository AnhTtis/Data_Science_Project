\begin{figure*}[ht!] %
\centering
\begin{subfigure}[b]{0.45\textwidth}
      \includegraphics[width=\textwidth]{img/accumulation_without_skydome.png}
      \caption{Accumulated ray density with separate skydome}
\end{subfigure}
\begin{subfigure}[b]{0.45\textwidth}
      \includegraphics[width=\textwidth]{img/accumulation_with_skydome.png}
      \caption{Accumulated ray density without separate skydome}
\end{subfigure}
\vspace{-5pt}
\caption{\small \emph{Training without a separate skydome.} We supervise the sky content with zero inverse-depth (infinite distance) to ensure that the camera does not intersect the sky as the layout features are extended. As such, we model content at infinite distances with a separate skydome model, such that the terrain model treats sky regions as empty space (left). Without the separated skydome, the model is forced to put sky content at finite distances leading to foggy, semi-transparent content near the camera (right). }
\label{fig:sm_expt_skydome}\vspace{-5pt}
\end{figure*}