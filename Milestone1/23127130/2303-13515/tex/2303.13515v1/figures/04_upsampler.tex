\begin{figure*}[ht!] %
\centering
\includegraphics[width=1\textwidth]{img/upsampler_compressed.pdf}\vspace{-0.25cm}
\caption{\small \emph{Qualitative comparison of model variations.} Each row shows a model variant, visualizing generated geometry (as a rendered scene filled with a checkerboard pattern), sky mask, rendered terrain, and final image composite. 
(Top) Without geometry regularization, the model produces semi-transparent terrain. (Middle) Adding geometry regularization (Eqn.~\ref{eqn:geometry}) makes the terrain more solid, but there are inconsistencies between the terrain and mask prediction. (Bottom) Our full model uses geometry regularization and also adds a upsampler that operates on inverse-depth and sky mask inputs in addition to RGB (Eqn.~\ref{eqn:upsampler}) to discourage boundary effects between the terrain and sky. 
}
\label{fig:upsampler}\vspace{-10pt}
\end{figure*}
