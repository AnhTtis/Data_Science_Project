\chapter{Introduction}
%\addcontentsline{toc}{chapter}{Introduction}

Starting with the discovery of the charmed quark and the $\tau$ lepton during the 1974 and 1975
%``November Revolution''
~\cite{Galison:1992us}, the results
from low-energy $e^+e^-$ collider experiments with a center-of-mass energy (CME) in the 2$\sim$6 GeV ($\tau$--charm threshold)
region have played a key role in elucidating the properties of these intriguing particles. Historically, there have
been several generations of $\tau$--charm facilities (TCFs) in the world, including
the Mark II and Mark III detectors~\cite{Abrams:1989cm,Bernstein:1983wk}, DM2~\cite{Augustin:1980ad}, CLEO-c~\cite{Asner:2004yu},
and BEPC/BES \cite{Bai:1994zm}, which have produced numerous critical contributions to the establishment of the
SM and to searches for new physics beyond the Standard Model~(SM). Of these, the BEPC/BES facility in Beijing, China, is no doubt one of the
most prolific TCFs.
This program, which started in the late 1980s, has produced many interesting physics results, such as precision
measurements of the $\tau$-lepton mass~\cite{Bai:1992bu,Ablikim:2020tau-mass} and the $e^+e^-$ annihilation cross section~\cite{Bai:2001ct}, the first observations of purely leptonic charmed
meson decays~\cite{Bai:1998cg}, the discovery of the $X(1835)$ as a baryonium state candidate~\cite{Bai:2003sw,Ablikim:2005um,BESIII:2011aa,besiii-ppetap}, and clear elucidation of the $\sigma$
($f_0(500)$)~\cite{Ablikim:2004qna} and $\kappa$ ($K_0(700)$)~\cite{Ablikim:2010ab}, the lowest-lying scalar mesons.

The currently operating BEPCII/BESIII~\cite{bepcii,Ablikim:2009aa} complex, which is a major upgrade of BEPC/BESII~\cite{Bai:1994zm} that includes separate electron
and positron magnet rings as part of the highest-ever-luminosity TCF and a completely new, state-of-the-art detector,
is the only facility in the world that can address the physics opportunities in this interesting energy range. BEPCII/BESIII's
unique capabilities and excellent performance have attracted a large collaboration of researchers from all over the
world that has been very successful in producing numerous high-quality, frequently cited physics results. After
ten years of operation, BEPCII is operating reliably at its designed luminosity of $10^{33}$~cm$^{-2}$s$^{-1}$ at
$\sqrt{s}= 3.77$~GeV. A continuous injection system has recently been implemented that increases its integrated luminosity
by 30\%, and its CME upper limit has been extended from 4.6 to 4.9~GeV, thereby providing access to charmed baryon thresholds.

In 2019, BESIII achieved one of its main data-taking goals with the successful accumulation of 10 billion $J/\psi$ events
for studies of light hadron physics. An early payoff from this unprecedentedly large data sample was the discovery of an
isospin-singlet $\eta\eta^{\prime}$ meson resonance with manifestly exotic $J^{PC}=1^{-+}$ quantum numbers~\cite{BESIII:2022riz, BESIII:2022iwi}. This is best
explained as a ``smoking-gun'' candidate for a Quantum Chromodynamics~(QCD) hybrid meson comprising a quark--antiquark pair plus a valence gluon,
a hadronic substructure that was predicted over forty years ago~\cite{Horn:1977rq} but has only recently started to
emerge experimentally thanks to the availability of enormous datasets such as the BESIII 10~billion $J/\psi$ event sample. Other notable light
hadron physics results from BESIII include the discoveries of an anomaly in the $X(1835)\to\pi^+\pi^-\eta^{\prime}$ line
shape at the $p\bar{p}$ mass threshold~\cite{besiii-ppetap, BESIIICollaboration:2022kwh}, an anomalously large partial width for the isospin-violating
$\eta(1405)\to f_0(980) \pi^0$ decay~\cite{BESIII:2012aa}, the first observation of $a_0(980)\leftrightarrow f_0(980)$
mixing~\cite{Ablikim:2018pik} (another forty-year-old prediction~\cite{Achasov:1979xc} that was eventually confirmed by
BESIII) and precise measurements of the hyperon decay parameters and tests of charge-conjugate and parity~($CP$) invariance in
$J/\psi\to\Lambda\bar{\Lambda}$ decays~\cite{BESIII:2018cnd, BESIII:2022qax} and $J/\psi\to\Xi^{-}\bar{\Xi}^{-}$~\cite{BESIII:2021ypr}.


For studies of charmed mesons and baryons, BESIII has accumulated samples of 1.7 million tagged
$D^+D^-$ events and 2.8~million tagged $D^0\bar{D}^0$ events produced via $\psi(3770)\to D\bar{D}$ decays, 0.30~million
tagged $D_s^+D_s^{*-}$ events from $\psi(4160)\to D_s^+D_s^{*-}$ decays and 90~thousand tagged $\Lambda_c^+\bar{\Lambda}_c^-$
events from $e^+e^-\to \Lambda_c^+\bar{\Lambda}_c^-$ with CMEs above 4.6~GeV. Measurements of purely leptonic and
semileptonic decays of $D$ and $D_s$ mesons produced the world's best measurements of the Cabibbo-Kobayashi-Maskawa~(CKM)
matrix elements $|V_{cs}|$ and $|V_{cd}|$~\cite{Ablikim:2017lks,Ablikim:2015ixa,Ablikim:2018junl,Ablikim:2016duz}. Absolute $\Lambda_c$ branching
fraction measurements based on the tagged $\Lambda_c$ baryon sample~\cite{Ablikim:2015flg}, including measurements for a number of
previously unseen modes, dominate the Particle Data Group (PDG)~\cite{ParticleDataGroup:2022pth} listings for this state. In addition, the large samples of $CP$-tagged
$D^0$-meson decays have been used to make precise measurements of final-state strong interaction phase decays, which are critical
inputs to the LHCb and Belle (II) measurements of the $CP$-violating angle $\gamma$ of the CKM unitary
triangle~\cite{Ablikim:2014gvw,Ablikim:2020lpk,Ablikim:2020cfp}.

Measurements of $e^+e^-$ annihilations for CME values between 2.0 and 3.67~GeV have provided $R$ measurements, defined as the ratio of cross section at lowest-order between the inclusive hadronic process $e^{+}e^{-}\to hadrons$ and the Quantum Electrodynamics~(QED) process $e^{+}e^{-}\to\mu^{+}\mu^{-}$, with an unprecedented precision of
$\sim$3\%~\cite{BESIII:2021wib}, which were critical inputs to SM calculations of $\alpha_{\rm QED}(m_Z^2)$~\cite{Davier:2017zfy}
that were used in fits to the electroweak sector of the model that provided accurate predictions of the Higgs
boson mass that were spectacularly confirmed in 2012 by LHC experiments. BESIII R measurements for $e^+e^-\to\pi^+\pi^-$
at CMEs below 1~GeV, extracted from $e^+e^-\to\gamma_{\rm ISR}\pi^+\pi^-$ events~\cite{Ablikim:2015orh}, where $\gamma_{\rm ISR}$ is an
initial-state radiation, offer significant improvements in accuracy over previous results and will enable future SM calculations
of $(g-2)_{\mu}$ matching the higher precision that is expected for imminent measurements from currently operating
experiments at Fermilab~\cite{Grange:2015fou,Muong-2:2021ojo} and JPARC~\cite{Mibe:2010zz}. In addition, this data sample is being used for numerous low-energy QCD
studies, including measurements of the proton, neutron and $\Lambda$ time-like form factors with improved
precision~\cite{Ablikim:2019eau, BESIII:2021tbq, Ablikim:2017pyl, Ablikim:2019vaj}; first measurements of the $\Sigma$ and $\Xi$ form
%Editor: Please ensure that the intended meaning has been maintained in the above edit.
factors~\cite{Ablikim:2020kqp,Ablikim:2020rwi};
and studies of the enigmatic $Y(2175)$ resonance~\cite{Ablikim:2020pgw,Ablikim:2020coo,Ablikim:2018iyx,Ablikim:2019tpp}.

A data sample of $\sim$20~fb$^{-1}$ integrated luminosity accumulated at a variety of CME values between 4.0 and 4.6~GeV
support detailed studies of charmonium-like $XYZ$ states, including some of BESIII's most remarkable results, such as
the discoveries of the charged charmonium-like states $Z_c(3900)$ and $Z_c(4020)$~\cite{Ablikim:2013mio,Ablikim:2013wzq};
the $Z_{cs}(3985)$, the first example of a charmonium-like state with a nonzero strangeness~\cite{Ablikim:2020Zcs-discovery};
an anomalous line shape for the $Y(4260)$ resonance\footnote{also known as $\psi(4230)$, and was $\psi(4260)$}~\cite{Ablikim:2016qzw}; and a large partial decay width for the
radiative process $Y(4260)\to\gamma X(3872)$~\cite{Ablikim:2013dyn}.

The measurement of the $\tau$-lepton mass by the original BES experiment~\cite{Bai:1992bu} in 1992 yielded a result that was 7~MeV ($\sim$2$\sigma$)
lower than the average of all previous measurements. It clarified what was the major discrepancy with the SM at that
time~\cite{Marciano:1991pr}. Since then, the BES program has made further improved $\tau$ mass measurements; the latest BESIII result
is in good agreement with the original BES measurement but with an order of magnitude better
precision~\cite{Ablikim:2020tau-mass}.

The BEPC/BES program followed by its upgrades has
significantly advanced our understanding of elementary
particle physics.
The primary task of the particle physics community during the next two decades will be to mount a comprehensive challenge
to the SM and to develop an understanding of the laws of nature at a more fundamental level. This will require a
coordinated multidimensional program including
precise predictions for measurable quantities in the framework of the SM.
These predictions, in turn, will need to be compared with experimental
measurements with state-of-the-art sensitivities and well-controlled systematic errors. The physics potential of the
current BEPCII/BESIII program is limited by its luminosity and CME range. Higher luminosities will be crucial for
investigating many of the key questions that can be uniquely addressed in the $\tau$--charm threshold
energy region, such as more precise measurements of the SM's free parameters, a better understanding of the internal
compositions of exotic hadron states such as the $XYZ$ and other charmed mesons and baryons as well as quark--gluon states
and their underlying dynamics, measurements of $CP$ violation~(CPV) in hyperon decays and other systems, $\tau$ physics and
probes for possible new physics beyond the SM. Next-generation studies of charmed baryons, especially the newly
discovered doubly charmed baryon states~\cite{Aaij:2017ueg}, will require an increased CME range. Because of strict spatial
constraints, there is insufficient space on the IHEP campus in Beijing to accommodate an upgrade of BEPCII that would meet
these luminosity and energy goals. As a result, after BEPCII/BESIII completes its mission in the near future, there
will be a need for a new collider with two orders of magnitude higher luminosity and a much broader (by a factor of 2) energy range in order
to continue pursuing and extending the scientific opportunities in the $\tau$--charm region.


The proposed STCF~\cite{Peng:2022loi} is an electron--positron collider with separated electron and positron rings and symmetric beam energy to be constructed in China.
It is designed to have a CME range spanning from 2~to~$\sim$7 GeV, with a peak luminosity of at least
$0.5\times 10^{35}$ cm$^{-2}$s$^{-1}$ optimized for a CME of 4~GeV. In addition to the boost in luminosity, the extended accessible energy
region will provide opportunities to study the recently discovered doubly charmed baryons~\cite{Aaij:2017ueg}. The
proposed design leaves space for higher-luminosity upgrades and for the implementation of a polarized $e^-$
beam in a future phase II~\cite{Luo:2019gri} of the project. To achieve such a high luminosity, several advanced technologies, such as
the introduction of a crabbed-waist beam-crossing scheme with a large-Piwinski-angle interaction region~\cite{crabbedwaist}, will be
implemented in the machine.



Some of the physics that such an STCF could access can also be investigated by the Belle II~\cite{Adachi:2018qme} and LHCb~\cite{Alves:2008zz}
experiments. Detailed descriptions of the physics programs of Belle II and LHCb can be found in Refs.~\cite{Kou:2018napp,Bediaga:2018lhg}, respectively. Both of these experiments can produce more $\tau$ leptons and charmed hadrons and mesons than the STCF. However, STCF
data samples will have distinctly lower backgrounds, near-100\% detection efficiencies, almost full detector-acceptance, better full-event reconstruction
rates, well-controlled systematic uncertainties, etc. The STCF will also have several unique features that are not available at
Belle II and LHCb, including the direct production of $1^{--}$ resonances such as charmonium ($J/\psi$, $\psi(3686)$
and $\psi(3770)$) and nonstandard charmonium-like mesons, such as $Y(4260)$, $Y(4320)$ and $Y(4660)$, as well as operation near
particle--antiparticle thresholds, thus providing the ability to fully reconstruct events with final-state neutrinos,
neutrons/antineutrons or $K_L$ mesons with high efficiency.

%The STCF project is still in its research and development (R\&D) stage.
To achieve these goals, a sophisticated, machine-compatible detector will be required to maximize the physics potential.
The detector is expected to exhibit considerably improved performance in each subsystem compared to the BESIII detector.
%Currently, the STCF detector design, shown in Fig.~\ref{stcf},
%as visualized using DD4hep~\cite{Petric:2017psf}, 
%is still under research and development.
%Several features must be considered for this detector.

%\begin{figure}[htbp]
%\begin{center}
%\includegraphics[width=0.7\textwidth]{Figs_01_Introduction/detector_stcf.jpg}
%\caption{
%    \label{stcf}
%Schematic view of the STCF detector.
%}
%\end{center}
%\end{figure}

%***{\color{red} modify the following to make sure statements are correct}*****

The STCF detector features large solid-angle coverage, low noise, high detection efficiency and resolution and excellent particle identification capabilities.
It is also required to have fast trigger response, high rate capability and high levels of radiation tolerance.
From the interaction point outward, the STCF detector consists of a tracking system, a particle identification (PID) system, an electromagnetic calorimeter (EMC), a superconducting solenoid~(SCS) and a muon detector (MUD), where the tracking system is composed of both inner and outer trackers. 

Among all the subdetectors, the inner tracker is the closest to the interaction point and hence is exposed to the highest level of radiation.
To withstand the high radiation background,
a novel micropattern gaseous detector, based on $\mu$RWELL technology and consisting of three cylindrical layers located 6, 11 and 16 cm away from the interaction point, is proposed as a baseline option for the inner tracker.
A low-mass silicon pixel detector based on the complementary metal-oxide-semiconductor~(CMOS) monolithic active pixel sensor~(MAPS) technology is also considered as an alternative option.
A large cylindrical drift chamber with ultralow material,
%Editor: Please consider replacing 'ultralow material' with 'ultralow-background material', 'ultralow-density material', or another description that better conveys your intended meaning.
spanning from 200 to 820~mm in radius and operating with a helium-based gas mixture, is proposed as the outer tracker. The momentum resolution in a 1 T magnetic field is expected to be better than 0.5\% for charged tracks with a momentum of 1~$\gevc$, and the dE/dx resolution should be better than 6\%, which can be exploited for particle identification for low-momentum charged particles.
The PID system uses two different Cherenkov detector technologies, one in the barrel region and one in the endcap region, to achieve a 3$\sigma$ separation between kaons and pions with a momentum up to 2~$\gevc$. 
%A separation capability of 3$\sigma$ between muons and pions with a momentum between 0.2 and 0.6~$\gevc$ is also available with the PID system.
A homogeneous electromagnetic calorimeter composed of trapezoid-shaped pure CsI crystal scintillators is proposed for the EMC to achieve an excellent energy resolution ($\sim$2.5\% at an energy of 1~$\gev$) and a good position resolution ($\sim$5~mm at an energy of 1~$\gev$) in a high radiation background. The EMC also has timing capability allowing the effective separation of photons from neutrons and $K_{L}^{0}$ in the energy region of interest. 
A SCS magnet surrounding the EMC provides the tracking system with a magnetic field of 1~T.
A hybrid of resistive plate chamber (RPC) with 3 inner layers and plastic scintillator (7 outer layers) detectors is proposed as the MUD,
which is expected to provide an excellent capability to efficiently separate muons from pions with an efficiency of 95\% and a misidentification rate of less than 3\% or even better.
Advanced data acquisition (DAQ) and trigger systems are required to handle a high event rate in the range from 60~kHz to 400~kHz.



%The baseline detector concept has been studied in depth through simulation, and the results demonstrate that the system can deliver the performance necessary to achieve the physics goals of the STCF. Critical R\&D tasks have already been under development, with preliminary and promising results achieved.

The rest of the document is organized as follows: The physics opportunities at the STCF are discussed in Chapter~\ref{CDR_phys}. In Chapter~\ref{CDR_det}, the conceptual designs of the STCF detector system are described. 
In Chapter~\ref{chap_phyper}, performances of several benchmark 
physics processes are introduced, and Chapter~\ref{chap_plan} is
the future plans and R\&D prospects.


%In Section V, several topics related to QCD, such as $R$-value and Collins effect measurements, the $Q^2$ behavior of
%baryon form factors, precision tests of rare/forbidden decays and CP violation in $\eta/\eta'$ and hyperon decays, and
%studies of glueballs and hybrids are discussed. In Section VI, the discovery potentials for new, beyond-the-SM light particles
%are presented. Section VII is a summary.


