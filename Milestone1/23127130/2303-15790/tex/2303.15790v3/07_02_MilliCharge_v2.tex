\subsection{Millicharged particles}



Particles with an electric charge that is significantly
smaller than that of an electron are often referred to as millicharged
particles.
A variety of BSM models predict millicharged
particles;
for example, millicharged fermions in the hidden sector
can naturally arise via kinetic mixing
\cite{Holdom:1985ag, Holdom:1986eq, Foot:1991kb}
or Stueckelberg mass mixing
\cite{Kors:2004dx, Cheung:2007ut, Feldman:2007wj}.
Millicharged
particles have previously been searched for
at various mass scales both at terrestrial laboratories
and in cosmological/astrophysical processes
(see, e.g.\,  \cite{Jaeckel:2010ni} for a review).
Electron colliders operating at the GeV scale can probe
the previously allowed millicharged
particles parameter space for masses
in the MeV--GeV range \cite{Liu:2018jdi, Liang:2019zkb}.
At the MeV--GeV energy scale,
the existing laboratory constraints
on millicharged
particles include constraints from colliders \cite{Davidson:1991si},
the SLAC electron beam dump experiment \cite{Prinz:1998ua},
and neutrino experiments \cite{Magill:2018tbb}.



\begin{figure}[htbp]
\begin{center}
\includegraphics[width=0.45 \columnwidth]{Figs_07_NewPhys/singal-background.pdf}
\caption{
Monophoton cross sections
for millicharged
particles (solid) and for the SM irreducible BG (dashed)
versus the collision energy $\sqrt{s}$.
These cross sections are computed with the
following detector preselection cuts:
$E_\gamma > 25$ MeV for $\cos\theta_\gamma<0.8$
and
$E_\gamma > 50$ MeV for $0.86 < \cos\theta_\gamma< 0.92$.
The model parameters
$\epsilon=0.001$ and $m_\chi=0.1$ GeV
are used for the millicharged
particles model.
Taken from Ref. \cite{Liang:2019zkb}.
}
\label{Fig-SB-compare}
\end{center}
\end{figure}



A small fraction of the dark matter (DM) can be
millicharged in nature.
Recently, the EDGES experiment
detected an anomalous absorption signal in the global 21 cm background
near a redshift of $z=17$ \cite{Bowman:2018yin}.
Millicharged dark matter models have been invoked to
provide sufficient cooling of cosmic hydrogens
\cite{Munoz:2018pzp, Berlin:2018sjs, Barkana:2018qrx};
because the interaction cross section between
millicharged DM and baryons increases
as the universe cools, constraints from the early
universe can be somewhat alleviated.

Because the ionization signals from millicharged
particles are
so weak that typical detectors in particle colliders
are unable to detect millicharged
particles directly,
millicharged
particles can be searched for at electron colliders via
the monophoton final state
\cite{Liu:2018jdi, Liang:2019zkb}.
%
%MCPs can be searched for at the electron colliders via
%the monophoton final state
%\cite{Liu:2018jdi, Liang:2019zkb}.
%This is because the ionization signals from MCPs is
%so weak that typical detectors in particle colliders
%are unable to detect MCPs directly.
%Searches
The analysis of searching for millicharged
particles via the monophoton state at the STCF can be
easily extended to a variety of invisible
particles in the hidden sector.
In millicharged
particle models, monophoton events can be produced via
$e^+ e^- \to \bar \chi \chi \gamma$,
where $\chi$ is the millicharged
particle.
The irreducible monophoton background
processes have the form $e^+ e^- \to \bar \nu \nu \gamma$,
where $\nu$ is a neutrino. There are also
reducible monophoton backgrounds
due to the limited coverage of the detectors.
There are two types of reducible backgrounds:
the ``bBG'' background, which occurs when
all other visible final-state particles are emitted along the beam directions,
and the ``gBG'' background, which
is due to visible particles escaping the detectors via gaps
\cite{Liang:2019zkb}.


Fig.~\ref{Fig-SB-compare} shows the monophoton cross sections
for millicharged
particles and for the SM irreducible background,
where the analytical differential cross sections for these
processes are taken from Ref.\ \cite{Liu:2018jdi}.
%%
The monophoton cross section for millicharged
particles
increases as the collision energy decreases, as shown in Fig.~\ref{Fig-SB-compare}.
In contrast, the monophoton irreducible
background grows with increasing collision energy.
Thus, an electron collider with a lower collision energy
has a better sensitivity to kinematically accessible millicharged
particles.





\begin{figure}[htbp]
\begin{center}
\includegraphics[width=0.45 \columnwidth]{Figs_07_NewPhys/milli-charge-limit-combine-2.pdf}
\caption{
The expected 95\% C.L.\ upper bounds on millicharged
particles
from the STCF as well as from Belle II, BESIII, and BaBar.
The dot-dashed curves are obtained with
the bBG cut for the STCF, BESIII, and Belle II,
while gBG is neglected \cite{Liang:2019zkb}.
Taken from Ref. \cite{Liang:2019zkb}.
}
\label{fig:exclusion}
\end{center}
\end{figure}












%%
To analyze the sensitivity of the proposed STCF experiment to millicharged particles,
the STCF detector is assumed to have the same
acceptance as the BESIII detector.
The STCF sensitivity to millicharged
particles in the MeV--GeV mass range is shown
in Fig.~\ref{fig:exclusion},
under the assumption of 20 ab$^{-1}$ of data collected at $\sqrt{s}= 4$ GeV.
The STCF can probe a large parameter space below that of the SLAC
electron beam dump experiment for millicharged
particles,
from $\sim$4 MeV to 0.1 GeV.
Millicharged
particles with $\epsilon \lesssim (0.8-3) \times 10^{-4}$
and masses from $\sim$4 MeV to 1 GeV
can be probed by the STCF with
20 ab$^{-1}$ of data at $\sqrt{s}= 4$ GeV.
This also eliminates a significant portion of the
parameter space in which the 21 cm anomaly
observed by the EDGES experiment can be
explained \cite{Munoz:2018pzp}.
%
The expected constraints on millicharged
particles from the STCF
analyzed on the basis of 20 ab$^{-1}$ of data collected at $\sqrt{s}= 4$ GeV
are better than those from Belle II with 50 ab$^{-1}$ of data
for millicharged
particles from 1 MeV to 1 GeV.
The increase in sensitivity is largely due to the
fact that the collision energy of the STCF
is lower than that of Belle II, which is $\sim 10.6$ GeV.
Thus,
the STCF has unprecedented sensitivity to the millicharged parameter space
for the MeV--GeV mass scale that has not been explored by current experiments.



\begin{figure}[htbp]
\begin{center}
\includegraphics[width=0.45 \columnwidth]{Figs_07_NewPhys/STCF-limit.pdf}
\caption{
The expected 95\% C.L.\ upper bounds on millicharged
particles
with 10 ab$^{-1}$ of data assumed for each of the three
STCF $\sqrt{s}$ values.
The solid curves are analyzed with the bBG cut.
Taken from Ref. \cite{Liang:2019zkb}.
}
\label{F-STCF-limit}
\end{center}
\end{figure}


For simplicity, a single collision energy of $\sqrt{s}= 4$ GeV
with 20 ab$^{-1}$ is assumed to obtain the limits in Fig.~\ref{fig:exclusion}.
However, because the STCF will be operated at various energy points,
as shown in Table~\ref{tablelumi},
the actual limit should be analyzed by considering
various collision energies and detailed detector simulations.
%
The STCF sensitivities to millicharged
particles at three different collision
energies are compared in Fig.~\ref{F-STCF-limit},
where 10 ab$^{-1}$ of data is assumed for each collision
energy.
%
Although the low-energy mode loses sensitivity to heavy millicharged
particles,
it has better sensitivity than the high-energy mode for
probing light millicharged
particles.
For example, 10 ab$^{-1}$ of data with $\sqrt{s}=2$ GeV can be used to probe millicharged
particles
down to $\sim 4 \times 10^{-5}$ for a 10 MeV mass, as shown
in Fig.~\ref{F-STCF-limit}, outperforming the
$\sqrt{s}=7$ GeV mode by a factor of $\sim 5$.
The constraints on millicharged
particles at various collision energies are also shown in
Table~\ref{tab:mcp} for several benchmark points.
%
\begin{table}[htbp]
\begin{center}
\begin{tabular}{|c|c|c|c|c|c|c|}
\hline 
$\sqrt{s}$ (GeV)  & 2  & 2 & 4 & 4 & 7 & 7 \\ \hline
$m$ (MeV) & 1 & 100 & 1 & 100 & 1 & 100 \\ \hline
$\epsilon \lesssim$ 
& $3 \times 10^{-5}$ &  $7\times 10^{-5}$
& 9 $\times 10^{-5}$ &  $1\times 10^{-4}$
& 2 $\times 10^{-4}$ &  $3\times 10^{-4}$ \\ \hline
\end{tabular}
\caption{
The expected 95\% C.L.\ upper bounds on $\epsilon$ for millicharged
particles 
with 10 ab$^{-1}$ of data for three
STCF $\sqrt{s}$ values, namely, 2 GeV, 4 GeV, and 7 GeV, as 
analyzed with the bBG cut \cite{Liang:2019zkb}.}
\label{tab:mcp}
\end{center}
\end{table}

