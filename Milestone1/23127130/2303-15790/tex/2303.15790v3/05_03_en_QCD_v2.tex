
The formation of the observed hadrons from QCD partons is still not understood. Experimentally, an $e^+e-$ collider is a suitable place to
study hadronization because its initial states are leptons, whereas such studies at a hadron collider will suffer from uncertainties due to the presence of initial hadrons.
At the STCF, such a study can be performed by measuring the $R$ value for a totally inclusive cross section and by measuring the inclusive production
of one or two hadrons. The latter will provide important information about various parton fragmentation functions.
In addition to inclusive processes, exclusive processes will also be studied at the STCF. There are interesting near-threshold phenomena
in $e^+e^-\to B\bar B$, where $B$ is a baryon. Because the STCF will run at center-of-mass energies of up to 7 GeV, it will be possible to exclusively produce two charmonia. The study of the exclusive and inclusive
production of quarkonia will provide important tests of theoretical predictions of nonrelativistic QCD.

In addition to the abovementioned processes, which can be theoretically studied via perturbative QCD at a certain level, many phenomena at the STCF are totally nonperturbative.
These nonperturbative processes can also be well studied at the STCF to provide more insights
into nonperturbative QCD and even new physics beyond the SM.



\subsection{QCD Physics}
\label{subsec:qcd}
\subsubsection{$R$ value}
The $R$ value is defined as
\begin{equation}
   R(s) = \frac{\sigma_{\rm tot} ( e^+ e^- \to \gamma^* \to  {\rm hadrons)}}{\sigma (e^+e^-\to \gamma^* \to \mu^+ \mu^-)},
\end{equation}
which is a function of $s$. An early measurement of $R$ was made at BES \cite{BES-R-2000, BES-R-2002}. Recently, it has also been measured by the KEDR and BESIII~\cite{KEDR, BESIII:2021wib}.

From experimental measurements of $R$, one can determine the running of the electroweak coupling and conduct precision
tests of the SM, as demonstrated in a recent study of the global SM fit \cite{Gfit}.
Precise measurements of $R$ enable the determination of the coupling constants in the SM.
Currently, the possible deviation of $(g-2)_\mu$ of the $\mu$ lepton has motivated many efforts to improve
the precision of the theoretical predictions and to explain this deviation as an effect of new physics beyond the SM. The newest result indicates that there are 4.2 standard deviations between the experimentally measured and theoretically predicted $(g-2)_\mu$ values \cite{NewM}.
An important contribution to the uncertainty of $(g-2)_\mu$ is the contribution from hadronic vacuum polarization. This contribution can be extracted from $R$ as measured in experiments.
Therefore, a precise measurement of $R$ can play an important role in precision tests of the SM.
It is clear that more precise results for the $R$ value will be obtained at the STCF.

\subsubsection{Inclusive production of a single hadron}

For a sufficiently large $\sqrt{s}$, the inclusive production of a single hadron in $e^+e^- \to h +X$ can be predicted from QCD
via the QCD factorization theorem \cite{Book}:
\begin{eqnarray}
\frac{d\sigma (e^+ e^- \rightarrow h +X)} {d z}  &=& \sum_{a=q,\bar q ,g} \int \frac{d\xi}{\xi} H_a (\frac{z}{\xi},Q^2,\mu^2) D_{a\rightarrow h} (\xi,\mu^2)
\nonumber\\
 &=& \sum_q \sigma (e^+ e^- \rightarrow q\bar q) \biggr ( D_{q\rightarrow h} (z) +   D_{\bar q\rightarrow h} (z) \biggr ) + {\mathcal O}(\alpha_s),
\label{FF}
\end{eqnarray}
where $z$ is the fraction of the energy carried by the observed hadron $h$, the functions $H_a$ ($a=q$, $\bar q$, and $g$) can be calculated
via perturbation theory, and $D_{a\rightarrow h}$ denotes parton fragmentation functions describing the hadronization of a parton $a$ to $h$.
Eq. (\ref{FF}) is the expression from QCD for collinear factorization.
The fragmentation functions are universal for any process in which QCD factorization is applicable. Extracting fragmentation functions
at rather low energy, such as the energy region of the STCF near 4--5~GeV, is especially important because with these extracted fragmentation
functions, it is possible to test their energy evolution from a rather low energy scale to high energy scales.

\subsubsection{The Collins effect in the inclusive production of two hadrons}

If two hadrons in the final state are observed in the kinematic region such that the two hadrons are almost back to back, collinear factorization cannot be used.
However, there is another type of factorization, called transverse-momentum-dependent~(TMD) factorization, that holds in this region~\cite{TMDEP}. The angular distributions in this kinematic region are determined by TMD quark fragmentation functions. These functions describe the fragmentation of an initial parton into the observed hadron, where the hadron has a small transverse momentum with respect to the momentum of the initial parton.
The general form of these angular distributions can be found in~\cite{TMDFF}. Studies of the production
in this region are expected to yield many interesting results regarding TMD parton fragmentation functions. Among them, one, called the Collins function,
is of particular interest. This function describes how a transversely polarized quark fragments into a hadron~\cite{Collins}. Its value is zero
if there is no $T$-odd effect. Belle, operating at $\sqrt{s}=10.6$~GeV, has performed a study of the Collins function~\cite{ColBelle}. It will be interesting to see whether the
Collins function can be measured at the STCF. Theoretical predictions concerning the Collins effect in the energy region of $\sqrt{s}\sim 4$~GeV have been presented in~\cite{SY}.
In general, by studying the angular correlations of the two produced hadrons in the kinematic region, one can extract various TMD quark fragmentation functions. These functions contain information on how quarks are hadronized into a hadron. Studies of TMD parton fragmentation
functions will be important not only for understanding hadronization but also for exploring the inner structure of hadrons
in semi-inclusive DIS, for which one needs to know the TMD parton fragmentation functions in order to extract the TMD parton distribution functions.

\subsubsection{Form factors of hadrons}


Measuring the electromagnetic~(EM) form factors of nucleons has played an important role in exploring the inner structure
of nucleons. At present, these form factors are still the simplest structure observables for testing QCD predictions.
In the past, these form factors have been studied mostly in the space-like region. Recently, however,
such studies have been extended to the time-like region. In the time-like region, it is also possible to measure
EM form factors for baryons other than nucleons.
Currently available
time-like experiments demonstrate several puzzling features of EM form factors. Enhancement has been observed by BES \cite{BES1} near the threshold of the $p\bar p$ system. BaBar has also reported enhancement in the
$e^+ e^- \to p\bar p$, $\Lambda \bar \Lambda$, and $\Sigma^0 \bar \Sigma^0$ processes \cite{BaB}.
In the space-like region, the EM form factors of the proton and neutron have been measured with precision at the $1\sim 2\%$ level. In the time-like region, the EM form factors of the proton have a precision o $3.4\%$ \cite{Ablikim:2019eau}, while those of the neutron have an error at the $20\%$ level\cite{BESIII:2021tbq}.
At the STCF with the suggested luminosity, it will be possible to measure the EM form factors in the time-like region with a precision
%Editor: Please ensure that the intended meaning has been maintained in the above edit.
of $0.4\%$ for the
proton and $2\%$ for the neutron, comparable to that in the space-like region. Moreover,
it will be possible to study the enhancement in the production of baryon--antibaryon pairs near the threshold more precisely
to extract information about the interaction between a baryon and an antibaryon,
and it will also be possible to extend this study to the $\Lambda_c \bar \Lambda_c$ system to see whether enhancement occurs in the heavy baryon--heavy antibaryon system.
%Editor: Please ensure that the intended meaning has been maintained in the above edit.


Processes such as $\gamma\gamma\to {\rm hadrons}$ can be studied at the STCF. In addition to the general interest in photon--photon physics, particular quantities of interest are the transition form factors of mesons.
These form factors determine the leading contributions to hadronic light--light scattering, which is closely related
to the precise prediction of the interesting quantity $(g-2)_\mu$. Precise results for these form factors will greatly help to reduce the uncertainties in the contribution to $(g-2)_\mu$ from hadronic light--light scattering.




\subsubsection{Production of charmonia}

The inclusive production of a charmonium has been observed at Belle. The ratio has been measured as \cite{Belle09}
\begin{equation}
   R_{c\bar c} = \frac{\sigma (e^+ e^- \to J/\psi + c +\bar c +X)}
       {\sigma(e^+ e^- \to J/\psi  +X_{non. c\bar c} ) }   \approx 0.63.
\end{equation}
This is in conflict with theoretical expectations. Some progress in theoretically explaining
this result has been made by including various higher-order corrections. Although the experimental result can be explained by adding one-loop corrections \cite{MZC1,MZC2,BGJW1,BGJW2}, the outcome may be not consistent. If one includes the so-called color-octet contributions estimated
from the hadroproduction of $J/\psi$, there is still conflict between experiment and theory (see also \cite{REVW}).
Belle has also observed the exclusive production of double charmonia, $e^+ + e^-
\to J/\psi + \eta_c$\cite{Belle02}. Theoretically, the measured cross section is still not well explained,
even with the inclusion of two-loop predictions in the theory \cite{FJS}.
\par
With the STCF running at $\sqrt{s}$ larger than 6~GeV, it will be possible to experimentally study these production processes more precisely. This will be helpful for gaining a better understanding of production. This energy range offers a unique opportunity to study physics related to the production of two $c\bar c$ pairs. The production cross sections for $e^+e^-\to J/\psi\eta_c$ and $e^+e^-\to J/\psi c\bar c$ based on the NRQCD calculations in Refs.~\cite{ZGC1,ZGC2} are shown in Fig.~\ref{fig:xsec_NRQCD}. These cross sections can be tested at the STCF.

%%%%%%%%%%%%%%%%%%% Fig 2 %%%%%%%%%%%%%%%%%%%%%%%%
\begin{figure*}[t]
	\centering
	\includegraphics[width=0.45\textwidth]{Figs_02_CharmoniumXYZ/ee2psietac}~~
	\includegraphics[width=0.45\textwidth]{Figs_02_CharmoniumXYZ/ee2psicc}
\vspace{0cm}
\caption{Cross sections for $e^+e^-\to J/\psi\eta_c$ (left) and $e^+e^-\to J/\psi c\bar c$ (right) as calculated using NRQCD with the charm quark mass fixed at 1.5~GeV. The solid and dashed curves represent the results from the next-to-leading-order and leading-order calculations, respectively.
		\label{fig:xsec_NRQCD}}
\end{figure*}
%%%%%%%%%%%%%%%%%%%%%%%%%%%%%%%%%%%%%%%%%%%%%%%%%%

% *****{\color{red} Figure 5 already appear early in section 2.3. Need to make coherent display. May be combine the materials for this two small sections?}*****

%\input{06_ref_QCD}

