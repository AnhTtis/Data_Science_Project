Offline software is designed and developed for Monte Carlo simulation and offline data processing. It mainly consists of a software framework, detector simulation, calibration and reconstruction as well as physics analysis tools. In the prestudy phase of the STCF experiment, offline software is deployed to optimize the detector options and study the detector performance as well as the physics potentials. After the experiment is running, it will be used to conduct complicated offline data processing on the data collected with the detector and to convert them into physics results.

The STCF, with a peak luminosity of $\stcflum$ or higher, producing a data sample approximately 100 times larger than current $\tau$-charm factories, presents a very large challenge for offline software and computing in terms of both rate and complexity; therefore, a specific offline software needs to be redesigned and developed with the state-of-art technologies to meet the STCF requirements. The main tasks are listed below.

\begin{itemize}

\item \textbf{Development of a high-performance software framework} \\
One of the great challenges of the STCF offline software is the management and processing of a higher volume of data (approximately several petabytes per day) than the present $\tau$-charm factories; the software framework, providing common functions for offline data processing and integrating all the applications into the unified software platform, plays a very important role. Therefore, one of the most crucial tasks is to design and develop a new framework that supports heterogeneous computing, including algorithms, data models and workflows running in heterogeneous environments, and provides interfaces to new toolkits, such as machine learning toolkits, IO systems, and simulation engines.

\item \textbf{Development of fast and accurate detector simulation suites} \\
Detector simulations serve many purposes at each point in the lifecycle of the STCF facility. The new toolkit DD4hep is adopted for the detector description, including the ITK, MDC, RICH, DTOF, EMC and MUD, as well as the MDI and support systems. A detector geometry management system is needed to manage different versions of subdetector options, to support fast iterations of detector performance studies and to provide consistent geometry information for different applications, such as detector simulation, reconstruction and visualization. Further study on the accuracy of detector description, the physics interaction of the different types of particles with the detector medium, and a realistic electronics response is the part of the detector simulation most crucial to achieving a high degree of compliance between the simulation and the data. Additionally, it is necessary to explore emerging technologies, such as parallel computing, heterogeneous computing and machine learning toolkits, to speed up the detector simulation and improve its performance.


\item \textbf{Development of the calibration methods and algorithms}\\
The main task is to study the calibration methods for the key measurements from each subdetector, such as the relationship between drift distance and drift time, event start time, energy loss of the MDC, refractive index of the PID radiator, energy and position of the shower from the EMC, and noise level of the MUD, to develop the corresponding calibration algorithms and establish the complete calibration system to perform accurate conversions between electronic readouts and physical quantities, minimizing the influence of external factors of the experiment and the operating status of the detector itself on the physical measurements.


\item \textbf{Development of event reconstruction methods and algorithms}\\
Event reconstruction is a very complicated and challenging task in offline data processing, including reconstruction of the charged tracks, electromagnetic showers and particle identifications to produce the momentum, energy and type of the particles for further physics analysis. For the charge tracks, we develop a track finding method with conformal transformation and Hough transformation, a track fitting method based on the deterministic annealing filter (DAF) and track extrapolation based on Geant4. The likelihood-based PID methods are also studied for the RICH and DTOF detectors. The procedure for the EMC shower is also built up from clustering, seed finding, cluster splitting and the correction of the energy. For the MUD, one algorithm is developed based on the BDT method. Further study of these methods is crucial to achieving the design specifications of the detector hardware.

\item \textbf{Development of physics simulation software}\\
In the prestudy stage of the STCF, a parameterized (fast) simulation toolkit is necessary for detector optimization and determining the physics potential capabilities of the STCF. The fast simulation takes as inputs the response of physical objects in each subsystem of the detector, including the resolution, efficiency and related variables for the kinematic fit and the secondary vertex reconstruction algorithm. Therefore, the physics signal significance can be used as a metric to evaluate the detector options and the physics reach studies. The further optimization of the current fast simulation tool according to the new requirements of the physics study and detector design is one of the key tasks of offline software.

\end{itemize}
