\subsection{Machine Parameters}
The STCF is an electron-positron collider with one interaction point~(IP) and two symmetrical storage rings with a circumference of approximately 600 meters.
For the current design, the CMEy ranges from $2$ to $7\gev$, and the target luminosity is over \stcflum at the optimized CME of $4\gev$. Motivated by the wide range of physics programs at a CME approximately 4~GeV, {\it e.g.}, charm physics and $XYZ$ physics, experimental conditions with peak luminosity at a CME of 4~GeV are considered in the following discussion. The luminosity changes moderately within a few hundred MeVs around a CME of 4~GeV.

The STCF will carry out collision with flat beams, in which case the luminosity can be calculated by:
\begin{eqnarray}
  \mathcal{L}&=&\frac{\gamma f_0 N_b}{2r_e\beta^*_y}\xi_y,
  \label{eq:Luminosity_MachineParameters}
\end{eqnarray}
where $\gamma$ is the relativistic factor, $f_0$ is the collision frequency, $r_e$ is the classical radius of an electron, $N_b$ is the number of particles per bunch, $\beta^*_y$ is the vertical betatron function at the IP and $\xi_y$ is the vertical beam-beam parameter.

To inhibit the hourglass effect with a small $\beta_y*$, a large Piwinski angle collision is adopted, and the crossing angle is 60~mrad to obtain a large Piwinski angle, which results in a larger boost than that of the BEPCII and slightly limits the coverage of the detector. Additionally, the synchro-betatron coupling resonances are suppressed with the crab waist scheme.
The machine parameters are listed in Table~\ref{tab:MachineParameters}. 
More details on the design of the STCF accelerator can be found in Chapter~\ref{CDR_phys}.

\begin{table*}[htb]
    \caption{The designed machine parameters for the STCF.}
    \label{tab:MachineParameters}
    \centering
    \begin{tabular}{c|c}
      \hline
      Parameter                                                              &  Value \\ \hline
      Circumference (m)                                                   &  600       \\
      Beam energy range (GeV)                          &  $1\sim3.5     $        \\
      Optimized beam energy (GeV)                                   &  2                       \\
      Current (A)                                                              &  2                        \\
      Crossing angle $2\theta$ (mrad)                                & 60                    \\
      Natural energy spread                                               & $4.0\times10^{-4}$  \\
      Bunch length (mm)                                                   & 12                          \\
      Luminosity ($\times 10^{35}$~cm$^{-2}$s$^{-1}$)                &  $>0.5$                 \\
      \hline
    \end{tabular}
\end{table*}



\subsection{Machine Detector Interface}
The final focus magnet QD0 is a crucial element in the machine detector interface~(MDI) and is currently set 0.9~m away from the IP to balance the requirements of the accelerator and detector. QD0 is a double aperture magnet because the distance of electron and positron beams 0.9~m from the IP exceeds the aperture limit, with a considerable requirement for the magnetic field gradient. The electron and positron beams are 0.5~m from the IP when their distance is large enough to smoothly transition to two beam pipes.
The layout of the MDI structure considering the crucial issues above is shown in Fig.~\ref{fig:MDI2D}. The layout includes a beam pipe and a cryostat that contains the magnets near the IP and a helium channel to cool the magnet. In order of distance from the IP, the magnets consist of an anti-solenoid, a final defocus magnet (QD0), a correcting magnet and a final focus magnet (QF0). From innermost to outermost, the beam pipe is composed of an inner pipe, a Y-shaped pipe and a separated pipe.

Considering the reconstruction precision and the background level, the inner beam pipe is designed to have an inner radius of 30~mm and thickness of 1~mm. The precision and background level define the boundary conditions of the radius of the ITK.
With a narrower radius of 10~mm, the expected precision of the reconstructed position would be improved by $15\%$ only for low-momentum tracks ($p=0.1\gev$/c), with little difference for high-momentum tracks. However, the background from lost particles would increase by approximately nine-fold.
The angle between the marginal line of the MDI structure and the midline of the two beams is $15^\circ$, which limits the corresponding angular acceptance of the detector.

\begin{figure}[hbtp]
  \begin{center}
    \begin{overpic}[width=15.cm,angle=0]{Figures/Figs_05_MDI_LuminosityMonitoring/MDI2D.png}
    \end{overpic}
    \caption{The MDI structure layout includes an inner pipe (white), Y-shaped pipe (orange and green), separated pipe (pink), stainless shield (yellow), copper shield (blue), tungsten shield (dark red) and magnets (red).}
    \label{fig:MDI2D}
  \end{center}
\end{figure}

\subsection{Beam Background}
\label{sec:mdi_bkg}

With high luminosity and narrow beam design, the estimation and inhibition of background is a crucial issue for the STCF. The main sources of background are the luminosity-related background, such as radiative Bhabha scattering and two-photon processes, and beam-related background, including the Touschek and beam-gas effects. All of these background sources are simulated by different source generators and then transmitted to {\sc Geant4} for full simulations.
The background level at the beam energy of 2.0~GeV is estimated from simulation with a full luminosity of $1\times10^{35}$~cm$^{-2}$s$^{-1}$ since the luminosity is the largest at this beam energy.
However, the background level at other beam energies does not exceed that at the optimized energy with a decrease in luminosity.


\subsubsection{Background Sources}
\paragraph{Luminosity-related Background}
\begin{itemize}
\item Radiative Bhabha scattering: In the process of $e^{+}e^{-} \to e^{+}e^{-}{(n)\gamma}$, both electron-positron pairs and photons are potential background signals and are considered in the simulation. The {\sc Geant4} source particles come from the {\sc BBBrem} generator for $|\cos\theta|<0.9$ and {\sc BabaYaga} for $|\cos\theta|>0.9$.

%\subparagraph{Two Photon processes}
%\quad\\
\item Two-photon processes: In the reaction of $e^{+}e^{-} \to \gamma^{*} \gamma^{*} \to e^{+}e^{-}e^{+}e^{-}$, the original electron-positron pair cannot be considered due to the extreme forward focus of the momentum after the reaction. Thus, only the low-energy electron-positron pair is considered and simulated by {\sc DIAG 36}, while the calculated particle rate is listed in Table~\ref{tab:5.2.01}.
\end{itemize}

\paragraph{Beam-related Background}
\begin{itemize}
%\subparagraph{Thouschek Effect}
%\quad\\
\item Thouschek effect: The Touschek effect is caused by the collision of particles in a beam bunch, which transforms the transverse momentum to longitudinal momentum, causing bunch spread and particle loss. The Touschek scattering rate can be calculated by the Touschek lifetime from the $\rm{Br\ddot{u}ck}$ model~\cite{BKG:bkg_bruckmodel}. The Touschek effect mostly occurs at the IP, where the beam is most compressed. However, the background is more affected by the Touschek effect from upstream because the original particles are transported for a distance in the ring instead of being lost immediately. %as elaborated in Sec.\ref{section:SAD}.

%\subparagraph{Beam-gas Effect}
%\quad\\
\item Beam-gas effect: The beam-gas effect mainly includes Coulomb scattering and bremsstrahlung, both of which are caused by the reaction between particles and residue gas in the ring and are highly influenced by the gas pressure in the vacuum chamber. The Coulomb scattering rate is approximately proportional to the $\beta$ function. Therefore, Coulomb scattering from upstream of the IP is very dangerous to the detector.

%\subparagraph{Transport in the ring}
%\label{section:SAD}
%\quad\\
\item Transport in the ring: The particles lost from the beam due to the Touschek effect and the beam-gas effect transport a distance in the ring, which is simulated by the {\sc SAD} program developed by KEK~\cite{bkg_SAD}. The aperture in the {\sc SAD} simulation is described according to the MDI structure.
The lost particles near the IP are input to {\sc Geant4} for further simulation, and the corresponding distribution is shown in Fig.~\ref{fig:distribution_SAD}.
\end{itemize}

\begin{figure*}[bt]
  \begin{center}
    \begin{overpic}[width=12.cm,angle=0]{Figures/Figs_05_MDI_LuminosityMonitoring/Zloss.pdf}
    \end{overpic}
    \caption{Particle distribution near the IP if it hits the vacuum chamber for the electron beam, which is supposed to transfer from left to right. The positron beam is considered lost at the symmetrical position from the IP.The dark blue line and light blue line means the radius of the beamline in X and Y direction, respectively. The yellow line means the bias distance between the center of the electron/positron cluster and the center of the beam pipe in the shared pipe section.}
    \label{fig:distribution_SAD}
  \end{center}
\end{figure*}

\paragraph{Other Sources of Background}
\begin{itemize}
\item Synchrotron radiation (SR): The synchrotron radiation generated in the upstream pipe may influence the inner part of the detector system and should be carefully considered in the accelerator design to prevent it from directly entering the detector.

%\subparagraph{Injection}
%\quad\\
\item Injection: During injection, the background maybe increases by one or two orders of magnitude higher than the normal level, which is determined by the method of injection and accelerator status. In Belle II and BESIII, the injection background is not the major background source; thus, the detector response is not simulated in this step.
\end{itemize}

%%%%%%%%%%%%%%%%%  TABLE  %%%%%%%%%%%%%%%%%%%%%%%%
\begin{table*}[htb]
    \caption{Calculated particle rates for various background sources. The thresholds for the scattering angle and radiated photon energy for radiative Bhabha scattering are set to 4.47~mrad and 1~MeV, respectively. The average number of radiated photons in a radiative Bhabha event is denoted by $\bar{n}_\gamma$. No threshold is set for the two photon process. The particle rate of the beam-related background is calculated by the theoretical lifetime.}
    \label{tab:5.2.01}
    \vspace{0pt}
    \centering
    \begin{tabular}{cccc}
        \hline
        \thead[c]{Luminosity-related} & \thead[c]{RBB $e^{+}e^{-}$}& \thead[c]{RBB photon} &\thead[c]{Two photon process} \\
        \hline
        Cross-section (mb) &2.99 &2.99, $\bar{n}_\gamma$=1.3573 &5.15 \\
        Luminosity (/cm$^{2}$/s)& & $1\times10^{35}$  & \\						
        Particle rate (Hz) &$5.98\times10^{8}$ &$1.07\times10^{8}$ &$1.03\times10^{9}$ \\
        \hline
        \end{tabular}
        \begin{tabular}{cccc}
        \hline
        \thead[c]{Beam-related} &\thead[c]{Touschek effect}  &\thead[c]{Coulomb scattering}  &\thead[c]{Bremsstrahlung}\\
        \hline				
        Particle rate (Hz) &$1.12\times10^{9}$ &$2.09\times10^{8}$ &$2.1\times10^{6}$ \\
        \hline
    \end{tabular}
\end{table*}
%%%%%%%%%%%%%%%%%%%%%%%%%%%%%%%%%%%%%%%%%%%%%%%%%%

\subsubsection{Background Simulation Results}
\label{sec:bkg_sim}
%\paragraph{Simulation results}
\quad\\
As mentioned before, the luminosity-related background, Touschek background and beam-gas background are the main background components, which are fully simulated with {\sc Geant4}, and the physics list QGSP$\textunderscore$BERT$\textunderscore$HP is chosen. In the full simulation, 6 kinds of background signals are set as the primary particles in {\sc Geant4}, the electron-positron pairs generated by radiative Bhabha scattering, two photon processes, the Touschek effect, Coulomb scattering and bremsstrahlung and the photons generated by radiative Bhabha scattering. Each is simulated by $1\times10^{6}$ particles to obtain accurate background estimation. After the simulation, the detector responses to the 6 kinds of background signals are weighted and summed according to the particle generation rate shown in Table~\ref{tab:5.2.01}. In this case, the background influences on the detector system can be estimated well. In the simulation, there are three main parameters that we are most concerned with: the total ionizing dose (TID), nonionizing energy loss (NIEL) damage and background count rate of the STCF detector system.


Fig.~\ref{fig:5.3.01} (left) shows the TID value distribution in the RZ plane, which is divided into 1 cubic cm pixels. The average value of TID in various STCF detector subsystems and electronic systems, as well as the maximum TID value, are calculated. These results indicate that the first layer of the silicon-based ITK has the highest TID, with a value of 1170~Gy/y. For detectors and electronics other than the ITK, the TID is less than 20 Gy/y, which is tolerable for the current technologies. Fig.~\ref{fig:5.3.01}~(middle) displays the NIEL damage distribution in the RZ plane. The simulated result shows that the NIEL damage of all of the important detector and electronic systems is below $10^{11}$ 1~MeV neutron/cm$^{2}$/y (for silicon). For the plastic scintillator detector in the MUD, the NIEL damage is on the order of $10^{11}$ 1~MeV neutron/cm$^{2}$/y because many more low-energy protons are produced via the neutron elastic reaction, which has an extremely high equivalent neutron coefficient.
The background count rate distribution is shown in Fig.~\ref{fig:5.3.01}~(right). For different detectors, suitable thresholds are used. For the gaseous detector, the deposition energy threshold of the background count is set to 100~eV. For scintillator detectors, the deposition energy threshold of background hits is 0.15-0.5~MeV. For Cherenkov detectors, the kinetic energy of incident electrons is 0.185~MeV, corresponding to the Cherenkov light generation demand. It is shown that the MDC has the highest background count level because this detector is composed of 48 layer wires. The maximum background rate per cm$^{2}$ occurs in the first layer of the ITK, necessitating a good detector design to realize an acceptable occupancy level.


%%%%%%%%%%%%%%%%%%% Fig %%%%%%%%%%%%%%%%%%%%%%%%%%
\begin{figure*}[htb]
	\centering
	\includegraphics[width=50mm]{Figures/Figs_05_MDI_LuminosityMonitoring/TID.png}
\hspace{3 mm}
    \includegraphics[width=50mm]{Figures/Figs_05_MDI_LuminosityMonitoring/NIEL_damage.png}
\hspace{3 mm}
    \includegraphics[width=50mm]{Figures/Figs_05_MDI_LuminosityMonitoring/TOTALCOUNT.png}
\vspace{0cm}
\caption{The TID value (left), NIEL damage (middle) and background count rate (right) distributions of the STCF detector system.}
    \label{fig:5.3.01}
\end{figure*}
%%%%%%%%%%%%%%%%%%%%%%%%%%%%%%%%%%%%%%%%%%%%%%%%%%


%\quad\\
Fig.~\ref{fig:5.3.04} shows the contributions of the luminosity-related background and beam-related background to the TID, NIEL damage and background count. Tables~\ref{tab:TIDNIEL_mean}-\ref{tab:TIDNIEL_eletronic} list the TID, NIEL and count rate of the detector subsystems and electronic subsystems, respectively.
These simulated results indicate that the beam-related background sources, especially Coulomb scattering, are the major contributors under full luminosity conditions in the inner detector subsystems. Radiative Bhabha scattering is the main influence on the outer detector subsystems, while the two-photon process has less influence. Coulomb scattering is the major component of the beam-related background because the Touschek- and Bremsstrahlung-produced electron-positron pairs have large probabilities of appearing downstream of the beam pipe, while Coulomb scattering-generated particle loss appears around the IP.

%%%%%%%%%%%%%%%%%%% Fig %%%%%%%%%%%%%%%%%%%%%%%%%%
\begin{figure*}[htbp!]
	\centering
	\includegraphics[width=50mm]{Figures/Figs_05_MDI_LuminosityMonitoring/TIDcontribution.png}
\hspace{3 mm}
    \includegraphics[width=50mm]{Figures/Figs_05_MDI_LuminosityMonitoring/NIELcontribution.png}
\hspace{3 mm}
    \includegraphics[width=50mm]{Figures/Figs_05_MDI_LuminosityMonitoring/TOTALCOUNTcontribution.png}
\vspace{0cm}
\caption{The contribution of the background to the TID, NIEL damage and count.}
    \label{fig:5.3.04}
\end{figure*}
%%%%%%%%%%%%%%%%%%%%%%%%%%%%%%%%%%%%%%%%%%%%%%%%%%

%%%%%%%%%%%%%%%%%%% Table %%%%%%%%%%%%%%%%%%%%%%%%%%
\begin{table*}[htbp!]
    \caption{{\sc Geant4} simulated TID and NIEL in the STCF subdetectors. The numbers are given as the mean values along the beam direction for each subdetector. For the ITK, the results are given for two different design options, the silicon pixel-based and the $\mu$RWELL-based designs.}
    \label{tab:TIDNIEL_mean}
    \vspace{0pt}
    \centering
    \begin{tabular}{llll}
        \hline
        \thead[l]{Detector} & \thead[l]{TID \\value (Gy/y)}& \thead[l]{NIEL damage\\ (1~MeV neutron/cm$^{2}$/y)}& \thead[l]{Total count\\ rate (Hz)}\\
        \hline
        Silicon-inner-1	&1170	&$2.71\times10^{10}$ & $3.90\times10^{8}$  \\
        Silicon-inner-2	&243	&$1.02\times10^{10}$ & $3.59\times10^{8}$  \\
        Silicon-inner-3	&64.9	&$1.71\times10^{10}$ & $2.92\times10^{8}$  \\
        \hline
        $\mu$RWELL-inner-1 &10.9    &$9.95\times10^{9}$ & $5.35\times10^{8}$  \\
        $\mu$RWELL-inner-2 &4.55    &$1.15\times10^{10}$ & $4.75\times10^{8}$  \\
        $\mu$RWELL-inner-3 &4.66    &$1.44\times10^{10}$ & $6.81\times10^{8}$  \\
        \hline
        \end{tabular}
    \begin{tabular}{llll}
        \hline
        \thead[l]{Detector} & \thead[l]{TID \\value (Gy/y)}& \thead[l]{NIEL damage\\ (1~MeV neutron/cm$^{2}$/y)}& \thead[l]{Total count\\ rate (Hz)}\\
        \hline
        MDC	&11.0	&$4.27\times10^{10}$ & $7.27\times10^{8}$ \\
        PID-Barrel~(RICH)	&2.96	&$8.67\times10^{9}$ & $4.50\times10^{8}$  \\
        PID-Endcap~(DTOF)	&1.34	&$4.65\times10^{9}$ & $8.30\times10^{8}$  \\
        EMC-Barrel	&0.35	&$1.41\times10^{10}$ & $2.64\times10^{9}$  \\
        EMC-Endcap	&0.32	&$7.26\times10^{9}$ & $9.38\times10^{8}$  \\
        MUD-Barrel-RPC	&0.028	&$3.23\times10^{8}$ & $5.58\times10^{6}$  \\
        MUD-Barrel-Scintillator	&0.040	&$3.89\times10^{11}$ & $1.06\times10^{7}$  \\
        MUD-Endcap-RPC	&0.017	&$7.03\times10^{7}$ & $3.53\times10^{6}$  \\
        MUD-Endcap-Scintillator	&0.027	&$1.86\times10^{11}$ & $1.22\times10^{7}$  \\
        \hline
        \end{tabular}
\end{table*}


\begin{table*}[htbp!]
    \caption{{\sc Geant4} simulated TID and NIEL in the STCF subdetectors. The numbers are given as the maximum values along the beam direction for each subdetector. For the inner tracker, the results are given for two different design options, the silicon pixel-based and the $\mu$RWELL-based designs.}
    \label{tab:TIDNIEL_max}
    \vspace{0pt}
    \centering
        \begin{tabular}{llll}
        \hline
        \thead[l]{Detector} & \thead[l]{Highest TID value\\ per pixel (Gy/y)}& \thead[l]{Highest NIEL \\damage per pixel \\(1~MeV neutron/cm$^{2}$/y)} & \thead[l]{Highest count rate\\ per channel (Hz/channel)} \\
        \hline
        Silicon-inner-1	&3490	&$1.75\times10^{11}$ &  $2.61\times10^{2}$ \\
        Silicon-inner-2	&320	&$3.72\times10^{10}$ &  $2.74\times10^{1}$ \\
        Silicon-inner-3	&150	&$2.68\times10^{10}$ &  $8.51\times10^{0}$ \\
        \hline
        $\mu$RWELL-inner-1 &118    &$1.12\times10^{10}$ & $3.35\times10^{5}$  \\
        $\mu$RWELL-inner-2 &61.8    &$1.46\times10^{10}$ & $1.63\times10^{5}$  \\
        $\mu$RWELL-inner-3 &38.6    &$5.67\times10^{10}$ & $1.61\times10^{5}$  \\
        \end{tabular}
        \begin{tabular}{llll}
        \hline
        \thead[l]{Detector} & \thead[l]{Highest TID value\\ per pixel (Gy/y)}& \thead[l]{Highest NIEL \\damage per pixel \\(1~MeV neutron/cm$^{2}$/y)} & \thead[l]{Highest count rate\\ per channel (Hz/channel)} \\
        \hline
        MDC	 &60.5	&$4.87\times10^{10}$ &  $4.00\times10^{5}$ \\
        PID-Barrel~(RICH)	&4.25	&$1.07\times10^{10}$ &  $3.3\times10^{3}$ \\
        PID-Endcap~(DTOF)	 &44.3	&$1.98\times10^{10}$ &  $1.20\times10^{5}$ \\
        EMC-Barrel	&21.1	&$1.76\times10^{10}$ &  $9.00\times10^{5}$ \\
        EMC-Endcap	&45.1	&$1.88\times10^{10}$ &  $1.50\times10^{6}$ \\
        MUD-Barrel-RPC	&0.093	&$3.74\times10^{11}$ &  $1.76\times10^{3}$ \\
        MUD-Barrel-Scintillator &0.047	&$4.88\times10^{11}$ & $1.15\times10^{3}$ \\
        MUD-Endcap-RPC	&0.37	&$1.22\times10^{10}$ &  $2.83\times10^{4}$ \\
        MUD-Endcap-Scintillator	&0.24	&$2.79\times10^{12}$ &  $9.8\times10^{4}$ \\
        \hline
        \end{tabular}
\end{table*}


\begin{table*}[htbp!]
    \caption{{\sc Geant4} simulated TID and NIEL values in the STCF electronic subsystems.}
    \label{tab:TIDNIEL_eletronic}
    \vspace{0pt}
    \centering
        \begin{tabular}{lllllll}
        \hline
        \thead[l]{Electronic component} & \thead[l]{TID \\value (Gy/y)}& \thead[l]{NIEL damage\\ (1~MeV neutron/cm$^{2}$/y)}& & \thead[l]{Highest TID value\\ per pixel (Gy/y)}& \thead[l]{Highest NIEL \\damage per pixel \\(1~MeV neutron/cm$^{2}$/y)} \\
        \hline
        Inner-1-electronic	&1420	&$5.09\times10^{10}$ & &1460	&$5.94\times10^{10}$ & \\
        Inner-2-electronic	&238	&$2.22\times10^{10}$ & &250	&$2.35\times10^{10}$ & \\
        Inner-3-electronic	&95.9	&$2.95\times10^{10}$ & &97.2	&$3.24\times10^{10}$ & \\
        MDC-electronic	&5.2	&$6.44\times10^{9}$ & &7.4	&$2.20\times10^{10}$ & \\
        PID-Barrel-electronic	&2.45	&$6.87\times10^{9}$ & &2.95	&$8.37\times10^{9}$ & \\
        PID-Endcap-electronic	&1.02	&$2.70\times10^{9}$ & &6.81	&$3.96\times10^{9}$ & \\
        EMC-Barrel-electronic	&0.046	&$1.51\times10^{9}$ & &1.03	&$3.88\times10^{9}$ & \\
        EMC-Endcap-electronic	&0.67	&$9.44\times10^{8}$ & &60.5	&$1.78\times10^{10}$ & \\
        MUD-Barrel-electronic	&0.020	&$1.45\times10^{8}$ & &0.065	&$3.42\times10^{11}$ & \\
        MUD-Endcap-electronic	&0.28	&$1.87\times10^{8}$ & &3.56	&$1.79\times10^{9}$ & \\
        \hline
    \end{tabular}
\end{table*}
%%%%%%%%%%%%%%%%%%%%%%%%%%%%%%%%%%%%%%%%%%%%%%%%%%

\subsubsection{Comparison and Validation}
Since the beam energy range of the STCF is similar to that of BESIII, it is helpful to compare the background results with those of BESIII. The count rate of the BESIII MDC layer at radius = 20 cm is approximately 100 Hz/cm$^2$ with a beam current of 0.45 A and a luminosity of 0.3$\times$10$^{33}$ /cm$^2$/s at BESIII. The luminosity-related background is proportional to the luminosity. The beam-related background dominated by Touschek is roughly proportional to the beam current (4.4) and inversely proportional to the beam size (35). Thus, the expected count rate under ideal experimental conditions is approximately 1.35 MHz/channel for the innermost STCF MDC layer. The simulated data is 450 kHz/channel, indicating a 3 times of difference. Considering the uncertainty of the extrapolation, the 3 times of difference is acceptable and the simulation result is reasonable. Also, the high luminosity in STCF requires for a much better MDI design compared with that in BESIII, which would further suppress the background interference. Thus, the predicted background level is close to the simulation result, and the simulated data would be used to evaluate the detector spectrometer performance in this step.



\subsection{Conclusion}
The basic MDI geometry is designed with limits on the radius of the ITK larger than 33~mm and acceptance larger than $15^{\circ}$.
Several kinds of luminosity-related and beam-related background sources are generated by optimal generators and fully simulated with {\sc Geant4}. The simulation shows that the TIDs of the detectors and electronics are below 1500~Gy/y. Additionally, the NIEL damage in all of the detector and electronic volumes is below $10^{11}$ 1~MeV neutron/cm$^{2}$/y.
The background count simulation result indicates that in most detector subsystems, the occupancy and s/n ratio problems caused by the background count are moderate. With a well-designed detector and electronics, the STCF detector is expected to operate safely under these conditions. Although the current background level is acceptable, MDI upgrades, such as adding shieldings and collimators, will be the next stage of study to ensure safety in a real machine.
