\newpage
\chapter{Physics Performance}
\label{chap_phyper}

\section{Fast Simulation}

To facilitate physics potential studies, a fast simulation package~\cite{Fast_simu_ref2} has been developed and used to produce physics signals and background samples to investigate the physics potential capabilities of the STCF.
The basic idea of this fast simulation package is to model the detector responses, including the efficiencies,
resolutions, particle identification and other responses needed in data analysis, instead of simulating all details
and physical interactions with {\sc Geant4}. To more accurately simulate the detector's response, these model
shapes are extracted from the full simulation of the BESIII detector and parameterized based on empirical
formulas or extracted from histograms, which are defined with different momentum and polar angle (with
with respect to the beam direction) regions.

In this package, the modeling of responses in each subdetector is individually implemented for the charged
particles $e$, $\mu$, $\pi$, $K$ and $p$, as well as the neutral particles $\gamma$, $n$, $\bar{n}$ and $K_{L}^{0}$. The detector efficiency is simulated by a sampling according to its curves as a function of a two-dimensional variable,
{\it i.e.}, momentum versus $\cos\theta$, where $\theta$ is the polar angle of objects in the laboratory frame. For the
observables with measurement uncertainty, such as energy, momentum, space and time, the expected
value is the overlay of the detection resolution on top of the MC truth value, where the corresponding
detection resolution is extracted by a sampling according to the distribution. The reliability and stability
of the fast simulation package have been validated in terms of many aspects. At the object level, input and output
checks for all the observations of different types of particles with various input parameters have been
performed. In addition, full physics analyses for some interesting physical processes are performed in
in the fast simulation package by setting the same parameters as those of the BESIII detector and comparing the
results from BESIII physical programs, {\it e.g.} event selection efficiency and distribution of some physical
variables of interest. Good consistency is found in the above validation. In addition, an interface
is provided for users to flexibly adjust the responses to easily estimate the detector's performance. With
this fast simulation package, many physics analysis processes are ongoing, as introduced in the next
Section.

\section{Selected Physics Performance}
As discussed in Chapter~\ref{CDR_phys}, the physics highlights of the STCF can be grouped into three categories:
QCD and hadronic physics, flavor physics and CP violation and the search for new physics. 
The statistics is a crucial factor for learning the properties of the exotic
particles. Figure~\ref{fig1} shows the number of expected samples
at STCF under 0.2~ab$^{-1}$ and 1~ab$^{-1}$ integrated luminosity. 
STCF can collect world leading statistics for the charmonium and
charmonium-like samples. Moreover, the clean environment of STCF will
provide an ideal platform for the tau physics and study of charmed hadrons. 


\begin{figure}[htbp]
\begin{center}
\begin{overpic}[width=15cm, height=5.cm, angle=0]{Figures/Figs_07_PredictedPhysicalPerformance/fig1-eps-converted-to.pdf}
\end{overpic}
\end{center}
\caption{Number of expected samples at STCF under 0.2~ab$^{-1}$ and 1~ab$^{-1}$ integrated luminosity, compared with current BESIII statistics and Belle II 50~ab$^{-1}$ expected. }
\label{fig1}
\end{figure}

With the expected luminosity collected at STCF, the key parameters
from EW test and new physics probe are shown in Fig.~\ref{fig2} and Fig~\ref{fig3}. The statistical sensitivity for flavor and CP violation test can be significantly improved compared with current world best result. However, the systematic uncertainty will be an 
essential limitation of the precision by then. The discussion 
of systematic uncertainty is presented in Sec.~\ref{sec:sysunc}.
For the new physics probe, an increased statistics will no doubt help to test various models beyond SM. The sensitivity of various rare of forbidden decay can be improved with a magnitude factor of 2 to 3, and is lying in the range of beyond SM model predictions.


\begin{figure}[htbp!]
\begin{center}
\begin{overpic}[width=15cm, height=6.5cm, angle=0]{Figures/Figs_07_PredictedPhysicalPerformance/fig2-eps-converted-to.pdf}
\end{overpic}
\end{center}
\caption{Precision of various measurements to test SM, such as muon g-2, tau mass, CKM matrix and $CPV$, from current precision and STCF expected with 0.2~ab$^{-1}$ and 1~ab$^{-1}$ integrated luminosity. The uncertainties of STCF expected consider the sources from statistics (sta.), reducible systematic (sys.) such as tracking, PID, and other selection criteria, irreducible systematic from theoretic input and instrument effects such as beam energy and beam spread.  }
\label{fig2}
\end{figure}

\begin{figure}[htbp!]
\begin{center}
\begin{overpic}[width=17cm, height=6.5cm, angle=0]{Figures/Figs_07_PredictedPhysicalPerformance/fig3-eps-converted-to.pdf}
\end{overpic}
\end{center}
\caption{Sensitivity of processes that are forbidden or rare in SM prediction, from current results and STCF expect with 0.2~ab$^{-1}$ and 1~ab$^{-1}$, and compared with predictions from theoretical models beyond the SM.}
\label{fig3}
\end{figure}

Table~\ref{phyper} lists the statistical sensitivities for several
benchmark processes and compared with current operating BESIII and
Belle II experiments.

\begin{table}[htbp!]
\caption{Summary of the statistical sensitivities for some benchmark physics processes, not inclusive yet.}
\label{phyper}
\footnotesize
\begin{center}
\begin{spacing}{1.3}
\begin{tabular}{cccc}
\hline
\hline
\vspace{0.2cm}
Observable                       &  BESIII (2020)~~~~~     & Belle II (50~ab$^{-1}$)~~~~~              &  STCF (1~ab$^{-1}$)~~~~~   \\           
\hline
{\it Charmonium(like) spectroscopy:}\\
Luminosity between 4-5~GeV        &     20~fb$^{-1}$          &      0.23~ab$^{-1}$                       &    1~ab$^{-1}$  \\            
\hline
{\it Collins fragmentation functions:}\\
Asymmetry in $e^{+}e^{-}\to KK+X$  &      0.3~\cite{BESIII:2015fyw}                 &        -                                  &  $<0.002$~\cite{collins_stcf} \\       
\hline 
{\it CP violations:} \\
$A_{cp}$ in hyperon              & 0.014~\cite{BESIII:2018cnd}          &  -                                  &      $0.00023$    \\
$A_{cp}$ in $\tau$                  &   -               &  $\mathcal{O}(10^{-3})/\sqrt{70}$~\cite{Chen:2020uxi}      &      0.0009~\cite{Sang:2020ksa}    \\
\hline
{\it Leptonic decays of $D(_{s})$:}  \\
$V_{cd}$                          &    0.03~\cite{BESIII:2019vhn}            &     -          & 0.0015     \\
$f_{D}$ 		                  &    0.03            &     -          & 0.0015  \\ 
$\frac{\mathcal{B}(D\to\tau\nu)}{\mathcal{B}(D\to\mu\nu)}$  &   0.2  &     -   &  0.005 \\
$V_{cs}$                          &    0.02~\cite{BESIII:2018hhz}            &   0.005        & 0.0015  \\
$f_{D_{s}}$                       &     0.02           &    0.005       & 0.0015  \\
$\frac{\mathcal{B}(D_{s}\to\tau\nu)}{\mathcal{B}(D_{s}\to\mu\nu)}$  &   $0.04$  &     $0.009$   &  0.0038 \\
\hline
{\it D mixing parameter:}  \\
$x$                          &     -         & 0.03    &   0.05~\cite{Cheng:2022tog}  \\
$y$                          &     -         & 0.02    &   0.05 \\
\hline
{\it $\tau$ properties: }\\
$m_{\tau}$ (MeV/c$^{-2}$)                 &   0.12~\cite{BESIII:2014srs}   &  -    &   0.012  \\
$d_{\tau}$ (e~cm)             & -          &  $2.02\times10^{-19}$   &   $5.14\times10^{-19}$ \\
\hline
{\it cLFV decays of $\tau$(U.L at 90\% C.L.):}  \\
$\tau\to lll$        &              -                 &  $1\times10^{-9}$                &   $1.4\times10^{-9}$  \\  
$\tau\to \gamma\mu$  &              -                 &  $5\times10^{-9}$                 &   $1.8\times10^{-8}$  \\
$J/\psi\to e\tau$     &                $7.5\times10^{-8}$      &    -                                     &  $7.1\times10^{-10}$  \\
\hline 
\hline
\end{tabular}
\end{spacing}
\end{center}
\end{table}

\subsection{Prospects of the Collins Fragmentation Function}
The measurement of the Collins fragmentation function~(FF) represents an important test for understanding strong interaction dynamics and thus is of fundamental interest in understanding QCD.
There have been several semi-inclusive deep inelastic scattering (SIDIS) measurements of Collins asymmetry from HERMES~\cite{Airapetian:2004tw, Airapetian:2010ds}, COMPASS~\cite{Adolph:2012sn} and JLab~\cite{Qian:2011py}. Direct information on the Collins FF can be observed from $\ee$ annihilation experiments such as Belle~\cite{Abe:2005zx, Seidl:2008xc} and BaBar~\cite{TheBABAR:2013yha}, which give consistent nonzero asymmetries. However, the $\ee$ Collins asymmetry obtained from Belle and BaBar corresponds to a $Q^2(\approx 100\GeV^2)$ higher than the typical energy scale of the existing SIDIS data. The STCF studies $\ee$ at a moderate energy scale ($Q^2$ in 4$\sim$49~GeV$^2$).
The results can be directly connected to future SIDIS experiments, such as EIC and EicC. In addition, it is crucial to explore the $Q^2$ evolution of the Collins FF and to improve our knowledge of strong interaction dynamics.

To study the Collins FF, we introduce the $2\phi_0$ normalized ratio,
$R=\frac{N(2\phi_0)}{\langle N_0\rangle}$, where the azimuthal angle $\phi_{0}$ is defined as in Fig.~\ref{kpimisidbackground}(a),
$N(2\phi_0)$ is the dipion yield in each $2\phi_0$ subdivision, and $\langle N_0\rangle$ is the averaged bin content. The normalized ratios are built for unlike signs ($\pi^\pm \pi^\mp $) and like signs ($\pi^\pm \pi^\pm $), defined as $R^U$, $R^L$, where different combinations of favored FFs and disfavored FFs are involved. A favored fragmentation process refers to the fragmentation of a quark into a hadron containing the valence quark of the same flavor, {\it e.g.}, $u(\bar d)\to \pi^+$, while the corresponding $u(\bar d)\to \pi^-$ is disfavored. A double ratio $R^U/R^L$ is used to extract the azimuthal asymmetries since $R$ is strongly affected by detector acceptance. In the double ratio, charge-independent instrumental effects cancel out, while the charge-dependent Collins asymmetries are kept. $R^U/R^L$ follows the expression $\frac{R^U}{R^L} = 1 + A_{UL}\cos(2\phi_0)$, while $A_{UL}$ denotes the asymmetry for the UL ratio.

The measured asymmetry of $KK+X$ is significantly diluted by the $K\pi$ background. We use $A^{KK}_{\rm true}$ and $A^{K\pi}_{\rm true}$ to express the true Collins asymmetries for two processes $KK+X$ and $K\pi+X$, respectively, with $f_{K\pi}$ and $f_{\rm flat}$ for the level of the $K\pi$ background and the other background contributing zero asymmetry. Then, we can obtain the Collins measurement by unfolding
\begin{equation}
  A^{KK}_{\rm meas} = (1-f_{\rm flat}-f_{K\pi}) A^{KK}_{\rm true} + f_{K\pi}A^{K\pi}_{\rm true}.
\end{equation}
The $f_{K\pi}$ levels for $K\pi$ mis-ID levels from $0.001$ to $0.01$ are shown in Fig.~\ref{kpimisidbackground}(b). $f_{K\pi}$ can reach up to 8\% for $K\pi$ mis-ID at the 1\% level. Since $A^{KK}_{\rm meas}$ and $A^{\pi\pi}_{\rm meas}$ are at the (0.1, 0.2) level, the systematic background coming from $f_{K\pi}$ is approximately 0.02.

\begin{figure}[htbp]
\begin{center}
\begin{overpic}[width=7.5cm, height=6.cm, angle=0]{Figures/Figs_07_PredictedPhysicalPerformance/collins-fig/phi0.pdf}
\put(20,70){\small{(a)}}
\end{overpic}
\begin{overpic}[width=7.5cm, height=6.cm, angle=0]{Figures/Figs_07_PredictedPhysicalPerformance/collins-fig/kpimisidbackground.pdf}
\put(20,70){\small{(b)}}
\end{overpic}
\end{center}
\vspace{-0.5cm}
\caption{(a) Definition of $\phi_{0}$ as the angle between the plane spanned by the beam axis and the momentum of the second hadron ($P_2$) and the plane spanned by the transverse momentum $p_t$ of the first hadron relative to the second hadron.
(b) Different $K\pi$ mis-IDs versus the background level $f_{K\pi}$. }
\label{kpimisidbackground}
\end{figure}


With the $K/\pi$ misidentification rate required to be 1\% up to 2~GeV, the statistical uncertainty for $\pi\pi+X$ is $(2\sim7)\times10^{-4}$, and the statistical uncertainty for $KK+X$ is $(7\sim20)\times10^{-4}$ for $1~\rm {ab}^{-1}$ luminosity at $\sqrt{s}=7$~GeV. The momentum dependent precision 
for the Collins effect is described in Ref.~\cite{Wang:2021pus}, and it is found the precision of Collins effect is better than 0.2\% if the azimuthal asymmetries in the inclusive production of $\pi^{+}\pi^{-}$ is over 0.02 and that of $K^{+}K^{-}$ is over 0.07 in the kinematics bin $z=E/E_{\rm beam}\in [0.5, 0.9]$. 

\subsection{Prospects of $CP$ Violations in the Lepton/Baryon Sector}


Within the SM, there is no direct $CP$ violation in hadronic $\tau$ decays at the tree level in weak
interaction; however, the well-measured $CP$ asymmetry in $K_{L}\to\pi^{\mp}l^{\pm}\nu$ produces a difference between $\Gamma(\tau^{+}\to K_{L}\pi^{+}\bar{\nu})$
and $\Gamma(\tau^{-}\to K_{L}\pi^{-}{\nu})$ due to the $K^0-\bar K^0$ oscillation.
The same asymmetry also appears between $\Gamma(\tau^{+}\to K_{S}\pi^{+}\bar{\nu})$ and $\Gamma(\tau^{-}\to K_{S}\pi^{-}{\nu})$ and is calculated to be $(0.36\pm0.01)$\%~\cite{taucp5,taucp6}.

The BaBar experiment has found evidence for $CP$ violation in $\tau\to K_{S}\pi^{-}\nu[\geq 0\pi^{0}]$ with an asymmetry rate of $(-0.36\pm0.23\pm0.11)$\%~\cite{CPbabar},
which is $2.8$ standard deviations from the theoretical prediction.
Apart from the above BaBar measurement, a collaboration of CLEO~\cite{CPcleo} and Belle~\cite{CPbelle} focuses on the $CP$ violation that could arise from a charged scalar boson exchange~\cite{theocp}; this type of $CP$ violation can be detected as a difference in the $\tau$ decay angular distributions. The results are found to be compatible with zero with a precision of $\mathcal{O}(10^{-3})$ in each mass bin~\cite{CPbelle}.
In all these experimental measurements, the statistical uncertainty is at the level of $\mathcal{O}(10^{-3})$, and the current experimental sensitivity cannot yield a conclusion regarding the $CP$ from $\tau$ decay. Therefore, a higher-precision result is strongly required for clarifying the new physics features.

MC samples normalized to 1~ab$^{-1}$ luminosity at $\sqrt{s}=4.26$~GeV are simulated.
The $e^{+}e^{-}\to \tau^{+}\tau^{-}$ process is generated with {\sc KKMC}~\cite{KKMC}, which implements {\sc Tauola} to describe the production of the $\tau$ pair.
Passage of the particles through the detector is simulated by the fast simulation software.
We select signal events with one $\tau^{+}$ decay to leptons, $\tau^{+}\to l^{+}\nu_{l}\bar{\nu}_{\tau},(l=e,\mu)$,
denoted as the tag side.
The other is $\tau^{-}\to K_{S}\pi^{-}\nu_{\tau}$ with $K_{S}\to\pi^{+}\pi^{-}$, denoted as the signal side.
The charge conjugate decays are implied.
The distribution of $M_{K_{s}\pi^{-}}$ after selection is shown in Fig.~\ref{tautokspinu}(a), and the selection
efficiency for the signal is 32.8\% after detector optimization including improving the low-momentum tracking
efficiency by 10\% and requiring the $\pi/\mu$ misidentification to be 3\%.

The efficiency corrected numbers for $\tau^{-}\to K_{S}\pi^{-}\nu_{\tau}$ and $\tau^{+}\to K_{S}\pi^{+}\bar{\nu}_{\tau}$ are well consistent with the input value.
The statistical sensitivity of $CP$ asymmetry with decay rate is calculated to be $9.7\times10^{-4}$.
In addition, the selection efficiency for this process with CME from $\sqrt{s}=4.0$~GeV to 5.0~GeV, where the cross-section for $e^{+}e^{-}\to\tau^{+}\tau^{-}$ is the
maximum, is studied, as shown in Fig.~\ref{tautokspinu}(b).
It is found that the efficiency varies from 32.6\% to 33.5\% (without a likelihood requirement), which is very stable in this energy region.
Therefore, the statistics of the signal process can be increased linearly with more data collected.
Since this process is not free of a background, the sensitivity of the $CP$ asymmetry is proportional to $1/\sqrt{L}$.
With 10~ab$^{-1}$ integrated luminosity collected at the STCF, the sensitivity of the $CP$ asymmetry is estimated
to be $3.1\times10^{-4}$, comparable to the uncertainty for theoretical prediction.\\


\begin{figure}[htbp]
\begin{center}
\begin{overpic}[width=7.5cm, height=5.cm, angle=0]{Figures/Figs_07_PredictedPhysicalPerformance/mass1.png}
\put(26,58){\small{(a)}}
\end{overpic}
\begin{overpic}[width=7.5cm, height=5.cm, angle=0]{Figures/Figs_07_PredictedPhysicalPerformance/energy_eff.pdf}
\put(26,58){\small{(b)}}
\end{overpic}
\end{center}
\vspace{-0.5cm}
\caption{(a) Invariant mass of $K_{S}\pi^{-}$ with combined $e$-tag and $\mu$-tag from $\tau^{+}$ decay. (b) Selection efficiencies for the signal process at different CMEs from $\sqrt{s}=4.0$~GeV to 5.0~GeV.}
\label{tautokspinu}
\end{figure}


According to the CKM mechanism, the $CP$ violation of $\Lambda$ was predicted to be on the order of $\mathcal{O}(10^{-4}) \sim \mathcal{O}(10^{-5})$ ~\cite{SM27}. Recently, BESIII has measured this $CP$ violation by studying the spin correlation
in $J/\psi\to\Lambda\bar{\Lambda}$ with 1.3~billion $J/\psi$~\cite{BESIII:2018cnd}. The sensitivity of the $CP$ violation is $\mathcal{O}(10^{-2})$,
which is far from the sensitivity of theoretical prediction. At the STCF, more than 1 trillion $J/\psi$ are expected in
one year, and the $CP$ violation of hyperons can be studied with high precision.

For the process $J/\psi \to \Lambda \bar{\Lambda} \to p \pi^{-} \bar{n} \pi^{0}$, the selection efficiency is 13.1\%,
which is increased by a factor of 22.1\% after a series of detector response optimizations. With the fitting of the joint angular distribution with the spin-correlation functions, the precision of the $\alpha$ asymmetry is estimated to be $8\times10^{-4}$ with a 1 trillion $J/\psi$ MC sample.
The $CP$ violation is constructed with the difference of $\alpha$ from the decay of $\Lambda$ and $\bar{\Lambda}$.
From this study, we can also estimate the sensitivity of $CP$ violation for the process $J/\psi\to\Lambda\bar{\Lambda}\to p\bar{p}\pi^{+}\pi^{-}$; by improving the tracking efficiency for low-momentum tracks by 10\%, the sensitivity of $CP$ study for this process is approximately $3\times10^{-4}$, which is comparable to the sensitivity from theoretical prediction.

It should be noted that, 
if the electron beam is polarized, the polarization translates nearly 100\% into a well understood polarization of the two final-state tau leptons. 
Thus, the STCF operation with a polarized electron beam just above the $tau$-pair threshold would enable a high sensitivity search for CP-violating asymmetries in many $\tau$ decay
modes, such as $\tau\to\pi\pi^{0}\nu$~\cite{Tsai:1994rc}. Besides, the polarization of electron beams can also improve the sensitivity of CP in hyperon decays with a factor of three~\cite{Salone:2022lpt}. 


\subsection{Prospects of Leptonic Decay $D_{s}^{+}\to l^{+}\nu_{l}$}

In the SM, pure leptonic decay $D_{s}^{+} \to l^{+} \nu~(l=e,~\mu,~ \tau)$ is
described by the annihilation of the initial quark-antiquark pair into a virtual $W^{+}$ that materializes as a $l^+ \nu$ pair. Since there is no strong
interaction present in the leptonic final state $l^+ \nu$, this decay provides a clean way to probe the complex strong interaction hiding the quark and antiquark within the initial-state meson, where the strong interaction effects can be parameterized by the decay constants $\fdsp$ and $|V_{cs}|$.

For a given value of $|V_{cs}|$, we can obtain $f_{D_s^+}$, and vice versa. Improved measurements of $f_{D_s^+}$ are important for testing and calibrating lattice QCD calculations, and $|V_{cs}|$ is important for testing the CKM matrix unitarity.
In addition, the ratio of the decay rates of $D_s^+ \to \tau^+ \nu_{\tau}$ and $D_s^+ \to \mu^+ \nu_{\mu}$ can be
used to test lepton flavor universality (LFU).
The current experimental results of $D_{s}\to l^{+}\nu$ are limited by statistics; therefore, it is important
to study the $D_{s}$ leptonic decay with high statistics.


We study $D_{s}^{+}\to l^{+}\nu~(l=\mu,\tau)$ at $\sqrt{s}=4.009$~GeV, where the $D_s^{+} D_s^{-}$ pairs are produced with no other accompanying particles, which provides extremely clean and pure $D_s^{\pm}$ signals. The cross-section
for $e^{+}e^{-}\to D_s^{+} D_s^{-}$ is approximately 0.3~nb~\cite{crosssection_DsDs_cleo}.
The double tag~(DT) technique is used to perform absolute measurement of the branching fraction.
The single tag~(ST) candidate events are selected by reconstructing a $D_s^-$ in the following 11 tag modes: $\dstoksk$, $\kkpi$, $\kkpipiz$, $\kskpipi$, $\kskpipim$, $\pipipi$, $\pieta$, $\pipizeta$, $\pietapgam$, $\pietaprho$, and $\kpipi$.
Figure~\ref{Ds}(a) shows the distribution of $M_{BC}$ for the $D_{s}$ candidates from the $K^{+}K^{-}\pi^{-}$ mode.


To reconstruct $D_{s}^{+}\to\tau^{+}\nu$, the decay $\tau^{+}\to e^{+}\nu\bar{\nu}$ is studied. Only one good charged track is required; this track is identified as a positron and has a charge opposite that of the ST $D_s^-$ meson. The variable $\etot$ is used to demonstrate the $D_s^+$ candidate, which is defined as the total energy of good showers in the EMC, excluding those used in the tag side and photons from the positron.
The distribution of $\etot$ is shown in Fig.~\ref{Ds}(b).
The signal events form a peak at 0 on the $\etot$ spectrum. The background sources can be categorized into three classes. The class I background is from the events with the wrong tag $D_s^-$, which is represented by the events in the $M_{\rm BC}$ sideband regions; the class II background arises from the peak of $D_s^+ \to K_L^0 e^+ \nu_e$ decay with little or no energy from $K_L^0$ deposited in the EMC; the class III background is dominant and includes other $D_s^+$ semileptonic decays. The class II and class II backgrounds are depicted with MC simulations. The signal region of $\etot$ is defined to be $\etot<0.4$ GeV to maximize the signal significance. To mitigate the signal shape dependency on the tag mode, we use the cut-count method, {\it i.e.}, fit $\etot$ in the high region of $\etot>0.6$ GeV, and then subtract the extrapolated background from the number of observed events in the signal of $\etot<0.4$ GeV to obtain the DT yield and DT efficiencies.
To reconstruct $D_{s}\to\mu\nu$, one charged track is selected and identified as $\mu$, and the missing mass of neutrinos is used to distinguish signals from background.

\begin{figure}[htbp]
\begin{center}
\begin{overpic}[width=7.5cm, height=5.cm, angle=0]{Figures/Figs_07_PredictedPhysicalPerformance/mbc.pdf}
\put(23,50){\small{(a)}}
\end{overpic}
\begin{overpic}[width=7.5cm, height=5.cm, angle=0]{Figures/Figs_07_PredictedPhysicalPerformance/eextra.pdf}
\put(23,50){\small{(b)}}
\end{overpic}
\end{center}
\vspace{-0.5cm}
\caption{(a) Distribution of $M_{bc}$ from signal tag $D_{s}^{-}\to K^{+}K^{-}\pi^{-}$.
(b) Distribution of extra energy for $D_{s}\to\tau\nu$ with $\tau\to e\nu\nu$, where the open dots come from class I background,
the shaded green comes from class II background, and the dashed blue comes from class III background. The plots are depicted with
0.1~ab$^{-1}$ simulated cocktail MC. }
\label{Ds}
\end{figure}



With the 0.1~ab$^{-1}$ cocktail MC at 4.009~GeV, the statistical uncertainty for $D_{s}\to\mu\nu$ is estimated to be 0.89\%,
and the statistical uncertainty for $D_{s}\to\tau\nu$ is 1.3\%. This is comparable with the expected precision
at Belle II, with 50~ab$^{-1}$ luminosity.
Moreover, at the STCF, with 1~ab$^{-1}$ luminosity collected in one year, the uncertainty can
be reduced to 0.28\% and 0.41\% for $D_{s}\to\mu \nu$ and $D_{s}\to\tau\mu$ with $\tau\to e\nu\nu$, respectively.
The obtained $\fdsp$ and $|V_{cs}|$ have an uncertainty of less than 0.2\%, which is comparable to the uncertainty of the theoretical calculation.
Figure~\ref{expectvcs1} and Fig.~\ref{expectvcs2} shows the expected $|V_{cs}|$ and $f_{D_{s}}$ obtained with 1~ab$^{-1}$ luminosity collected at 4.009~GeV at the future STCF.

\begin{figure}[htbp]
\begin{center}
\includegraphics[width=14.5cm,height=10.cm,angle=0]{Figures/Figs_07_PredictedPhysicalPerformance/draw_com_Vcs.pdf}
\end{center}
\vspace{-0.5cm}
\caption{Comparison of $|V_{cs}|$.}
\label{expectvcs1}
\end{figure}


\begin{figure}[htbp]
\begin{center}
\includegraphics[width=14.5cm,height=10.cm,angle=0]{Figures/Figs_07_PredictedPhysicalPerformance/draw_com_fds.pdf}
\end{center}
\vspace{-0.5cm}
\caption{Comparison of $f_{D_{s}}$.}
\label{expectvcs2}
\end{figure}


\subsection{Prospect of cLFV in $\tau$ Decay}

The charged lepton flavor violation (cLFV) process is forbidden in the SM and is highly suppressed even when taking neutrino oscillation into account since the rate of the cLFV process is suppressed by $(m_{\nu}/m_{W})^{4}$, where $m_\nu$ and $m_W$ are the masses of neutrinos and $W$ bosons, respectively. Any observation of cLFV in experiment would be an unambiguous signature of new physics. On the other hand, lepton flavor conservation, differing from other conservation laws in SM, is not associated with an underlying conserved current symmetry. Therefore, many BSM models naturally introduce lepton flavor violation processes, and some of them predict branch fractions that are almost as high as the current experimental sensitivity~\cite{tauto3lepton2,tauto3lepton3,tauto3lepton4, tauto3lepton5, tauto3lepton6, tauto3lepton7}.
Of all the possible cLFV processes, $\tau \to \gamma \mu$ and $\tau\to lll$ are two golden channels as
the most likely decay modes in a wide variety of theoretical predictions.
Currently, the most stringent upper limits of the branch ratios of the two channels are given by B-factories \SI{4.4e-8} at \SI{90}{\percent} C.L. \cite{tautogammuBABAR} and \SI{4.5e-8} at \SI{90}{\percent} C.L. \cite{tautogammuBelle} for $\tau\to\gamma\mu$ and  $(4-8)\times10^{-8}$ at 90\% C.L. for $\tau\to lll$. The next generation of electron-position colliders, such as Belle II and the STCF, aim to push the sensitivity down another several orders of magnitude into the prediction range of BSM models.


We use the DT technique for cFLV decays of $\tau$, in which the tag of $\tau^{+}$ is the large branching fraction
processes, including one charged track
($\tau^{+}\to\pi^{+}/e^{+}/\mu^{+}+n\pi^{0}+n\nu)$. For $\tau^{-}\to
l^{+}l^{-}l^{-}$, 82.62\% of the total $\tau^{+}$ branching fraction is tagged, and $54\%$ of the total $\tau^{+}$ branching fraction is tagged for $\tau^{-}\to\gamma\mu$ (excluding the channels $\tau\to\mu\nu\mu$ and $\tau\to\pi \pi^{0}\pi^{0}\nu$). The tag side can be reconstructed with missing energy from neutrinos, while the signal side can be fully reconstructed.
For the process $\tau^{-}\to l^{+}l^{-}l^{-}$, there are six modes ($e^{-}e^{+}e^{-}$, $\mu^{+}e^{-}e^{-}$, $\mu^{-}e^{+}e^{-}$, $e^{+}\mu^{-}\mu^{-}$, $e^{-}\mu^{+}\mu^{-}$, $\mu^{-}\mu^{+}\mu^{-}$).
The number of surviving events $N_{BG}$ using 1~ab$^{-1}$ cocktail MC at $\sqrt{s}=4.26~$GeV is obtained, with scaling of the $\pi/\mu$ mis-ID rate to 10\%, 3\%,
and 1\%, as illustrated in Fig.~\ref{tauto3leptonresults}. The MC selection efficiencies are also obtained by
varying the $\pi/\mu$ mis-ID rate from 10\% to 1\%. From the plot, we find that the detection efficiency
of $\tau^{-}\to\mu^{-}\mu^{+}\mu^{-}$ is most sensitive to the $\pi/\mu$ mis-ID rate and is increased from
15.4\% to 22.5\%, with the corresponding background rate decreasing from 237 to 8.
With $3.5\times10^{9}$ $\tau^{+}\tau^{-}$ pairs collected at the STCF per year, under the assumption that the $\pi/\mu$ mis-ID rate is 1\%, the upper limit is predicted to be $1.4\times10^{-10}$.



\begin{figure}[htbp]
\begin{center}
\begin{overpic}[width=7.5 cm, height=6.0 cm, angle=0]{Figures/Figs_07_PredictedPhysicalPerformance/result_3leptonv1.pdf}
\end{overpic}
\begin{overpic}[width=7.5 cm, height=6.0 cm, angle=0]{Figures/Figs_07_PredictedPhysicalPerformance/result_3leptonv2.pdf}
\end{overpic}
\caption{(a) The number of surviving background events $N_{BG}$ and (b) the selection efficiency with different $\pi/\mu$ mis-ID rates for $\tau\to lll$. }
\label{tauto3leptonresults}
\end{center}
\end{figure}



For $\tau\to\gamma\mu$, there are two dominant backgrounds, from $e^{+}e^{-}\to\mu^{+}\mu^{-}$
and $e^{+}e^{+}\to\tau^{+}\tau^{-}$ with $\tau^{+}\to\pi \pi^{0}\nu$ and $\tau^{-}\to\mu\nu\nu$.
Stringent selection criteria should be applied to remove the background.
With the $\pi/\mu$ misidentification required to be 1\% and
the position resolution of photons required to be 4~mm, we obtain a background-free sample
with a selection efficiency of approximately 8\%.
The upper limit for
$\tau\to\gamma\mu$ is calculated to be around $1.8\times10^{-8}$ with 1~ab$^{-1}$
at $\sqrt{s}=4.26$~GeV~\cite{Xiang:2023mkc}.



\section{Discussion of Systematic Uncertainties}
\label{sec:sysunc}
To shed light on the physics programs at the STCF, especially those that require
precise measurements, it is essential to have precise knowledge about the systematic
uncertainty sources.
However, a full systematic uncertainty study requires both experimental data and MC simulation,
which are not possible for this conceptual design report.
Here, we only provide a general discussion of the leading systematic uncertainty sources in physics analyses.
A more precise estimation of systematic uncertainty is expected when the design and construction of the detector is completed.

\subsection{Luminosity Measurement}
The precise measurement of the production cross-section at the STCF
requires a high-precision measurement of the luminosity.
Usually, the luminosity measurement utilizes the Bhabha scattering process
$e^{+}e^{-}\to (\gamma)e^{+}e^{-}$ due to its clear signature and large production
cross-section. A cross check of the luminosity can be performed using the di-photon process $e^{+}e^{-}\to\gamma\gamma$.
At a tau-charm factory, large-angle Bhabha scattering events are selected.
The integrated luminosity is calculated with
$\mathcal{L}=N_{\rm obs}/(\sigma\times\varepsilon)$, where $N_{\rm obs}$ is the
number of observed signal events, $\sigma$ is the cross-section of the process
$e^{+}e^{-}\to (\gamma)e^{+}e^{-}$ or $e^{+}e^{-}\to\gamma\gamma$, and
$\varepsilon$ is the detection efficiency.

Sources of systematic uncertainties of luminosity measurements necessitate requirements
regarding event selection, MC statistics, trigger efficiency and the MC generator.
Regarding the requirements for event selection, {\it e.g.}, the tracking efficiency,
the EMC energy and track acceptance requirements can
be studied by an alternative selection criterion using only information from the EMC,
varying these parameters, respectively. The MC
statistics can be negligible with the generation of enough MC samples. The trigger efficiency
will be discussed after the design of the detector is complete.
For the MC generator, the observed cross-sections $\sigma$ for the two processes are currently provided by the {\sc Babayaga@NLO} generator~\cite{Balossini:2006wc,Balossini:2008xr} with a precision of 0.1\%.
An even higher-precision QED calculation of the MC generator is expected with more accurate luminosity measurements required in the future~\cite{Balossini:2006wc}.

\subsection{Tracking/PID Uncertainty}

Tracking and PID play a key role in physics analyses, and comprehensive study of the data/MC difference for the tracking and PID efficiencies is important.
With the high luminosity of the STCF, large samples of charged and neutral final states
can be selected from the QED processes, $J/\psi$ decay, and charm meson decay.
The tracking and PID efficiencies can be studied in different kinematic regions with
pure samples. These process-independent efficiencies can be used to
correct the data/MC differences and estimate the corresponding systematic uncertainties.

A large sample of electrons and muons can be selected from radiative QED processes
$e^{+}e^{-}\to\gamma e^{+}e^{-}$ and $e^{+}e^{-}\to\gamma \mu^{+}\mu^{-}$.
Charged pions and kaons can be selected from $J/\psi$ decay, {\it e.g.}, $J/\psi\to\pi^{+}\pi^{-}\pi^{0}$ and $J/\psi\to K_{s}K^{\mp}\pi^{\pm}$,
or charm meson decay, $D^{0}\to K^{-}\pi^{+}$.
Protons can be selected from $J/\psi\to p\bar{p}\pi^{+}\pi^{-}$, while this process can also
be used to study the tracking and PID efficiencies of pions.
The detection efficiency of photons can be selected from processes $e^{+}e^{-}\to\gamma\mu^{+}\mu^{-}$, $J/\psi\to \pi^{+}\pi^{-}\pi^{0}$, $\psi(2S)\to\pi^{0}\pi^{0} J/\psi$, and $D^{0}\to K^{-}\pi^{+}\pi^{0}$.
The detection efficiency of neutrons can be studied from $J/\psi\to p\bar{n}\pi^{-}+c.c$.
The selection efficiency for the intermediate states $K_{s}$ and $\Lambda$ can be studied
with $J/\psi\to K_{s}\pi^{\pm}K^{\mp}$ and $J/\psi\to\Lambda\bar{\Lambda}$, respectively.



\subsection{Radiative Correction}

The radiative correction includes both initial-state radiative~(ISR) and final-state radiative~(FSR) corrections, and the current precision for the radiative corrections is model dependent.
For models used in {\sc Phokhara}, the precision is 0.5\%~\cite{Rodrigo:2001kf}.
A comprehensive discussion of radiation corrections is presented in Ref.~\cite{WorkingGrouponRadiativeCorrections:2010bjp}.
With increasing experimental accuracy, better modeling with radiative corrections is necessary.


\subsection{Vacuum Polarization}
In the calculation of the Born cross-section or the bare cross-section, the vacuum polarization~(VP) effect should be taken into account, which affects the EM fine structure constant $\alpha(s)$ as follows:
\begin{equation}
\sigma^{\rm bare} = \frac{\sigma^{\rm dressed}}{|1-\Pi(s)|^{2}} = \sigma^{\rm dressed}\cdot \left(\frac{\alpha(0)}{\alpha(s)}\right)^{2}
\end{equation}
The leptonic VP contributions to $\Pi(s)$ are calculated to four-loop accuracy~\cite{Sturm:2013uka},
the dominant uncertainty comes from the hadronic VP contribution that relies on the hadronic cross-sections,
which is currently within 0.2\% ~\cite{Jegerlehner:2008rs}. With more precise cross-sections obtained in experiments, especially at low CMEs, the accuracy of the VP effect will be further improved.



\subsection{Others}
In addition to the above systematic uncertainty sources widely involved in physics analyses,
there are other uncertainty sources depending on the physics processes involved.
In $\tau$ mass measurements, minimizing the uncertainties in the beam energy scale and energy spread is
so essential that a dedicated detector for these measurements is needed.
In the leptonic decay of charm mesons, radiative processes, for example, $D_{s}^{+}\to\gamma l^{+} \nu_{l}$,
are considered to have 100\% uncertainty on the branching fraction, which yields 1\% uncertainty
in the measurement of $D_{s}^{+}\to l^{+} \nu_{l}$. With the large sample collected at the STCF,
radiative processes can be measured with high precision, and its effect on the leptonic
decay will be limited to within 0.1\%.
Other uncertainty sources include those from background estimation, fitting procedures,
trigger efficiency, etc. Only when each uncertainty source is carefully examined and understood can the experimental accuracy match the high statistical precision achieved at the STCF.

\subsection{Conclusion}
Among the various physics highlights discussed at the beginning of this
section, the precision frontier needs careful systematic uncertainty 
studies. These uncertainties can be categorized into three aspects,
the first is the reducible uncertainties, that comes from tracking, PID, and other selection criteria, which will be scaled down 
according to the statistics of control sample. 
The second aspects from theoretical input as discussed above, such as initial radiative correction, vacuum polarization correction, luminosity measurement etc. The third comes from instrumental effects,  such as beam energy
measurement, energy spread etc. The latter two are irreducible 
uncertainties that should be studied in detail during R\&D. 