\subsection{DTOF Conceptual Design}
%{Conceptual Design of DTOF}

The proposed DTOF detector consists of two identical endcap discs positioned at $\sim \pm 1400$~mm away from the collision point along the beam direction. Each disc is made up of several quadrantal sectors, as shown in Fig.~\ref{layout}, with an inner radius of $\sim 560$~mm and an outer radius of $\sim 1050$~mm, covering in polar angles of $\sim 22^\circ - 36^\circ$. The sensitive regions of the the DTOF and RICH detectors overlap, leaving no dead areas between the barrel and the endcap. In each sector, a synthetic fused silica plate is used as a radiator to generate Cherenkov photons. The supporting structure between sectors occupies $\sim 15~mm$ of space, approximately $1.25 \%$ of the total sensitive area. Considering the effect of the magnetic field on the photon sensors, an array of
multianode MCP-PMTs are optically coupled to the radiator along the outer side. Fig.~\ref{layout} also shows an example of the light path from a photon directly hitting the MCP-PMT. Note that there
are also alternative paths for photons reflecting off the lateral-side mirror.


\begin{figure}[!htb]
	\centering
	\includegraphics[width=0.45\textwidth]{Figures/Figs_04_00_DetectorSubSystems/Figs_04_03_ParticleIdentification/Disc.png}
\caption{An example of the radiator sector for the DTOF detector and the light path of the radiator.}
	\label{layout}
\end{figure}


\subsubsection{DTOF Time Resolution}

Time resolution is a key indicator of the performance of a DTOF detector.
It is necessary to analyze the factors that affect the timing uncertainty,
and the relative importance of the various factors must be investigated to optimize the time performance.
The main sources contributing to the timing uncertainty of the DTOF detector can be expressed by
\begin{equation}
\sigma^{2}_{\rm tot} \approx \sigma^{2}_{\rm trk} +\sigma^{2}_{\rm T_{0}} + (\frac{\sigma_{\rm elec}}{\sqrt{N_{PE}}})^{2} + (\frac{\sigma_{\rm TTS}}{\sqrt{N_{PE}}})^{2}  + (\frac{\sigma_{\rm det}}{\sqrt{N_{PE}}})^{2} 
\label{uncertainty}
\end{equation}
where $\sigma_{\rm trk}$ is the uncertainty caused by track reconstruction, which is $\sim 10$~ps;
$\sigma_{\rm T_{0}}$ is the event reference time (i.e. the time of $e^{+}e^{-}$ collision, $T_{0}$) uncertainty, which is $\sim 40$~ps (it is closely related to the bunch length, which is $\sim 12 mm$ in the current design, as depicted in the STCF accelerator conceptual design report);
$\sigma_{\rm elec}$ is the electronic timing accuracy;
$\sigma_{\rm TTS}$ is the single-photon transit time spread~(TTS) of the MCP-PMT;
and $\sigma_{\rm det}$ is the time reconstruction uncertainty of the DTOF detector.

The contribution from $\sigma_{\rm elec}$, $\sigma_{\rm TTS}$ and $\sigma_{\rm det}$
decreases with increasing $N_{PE}$, and the timing uncertainty of these three effects
can be estimated from the single photoelectron~(SPE) time resolution of
the photon sensor and the electronics, called $\sigma_{SPE}$.
The calculation results for the SPE parameter are shown in Fig.~\ref{FIG:SPE_Resolution} as a function of the photon transmission distance $D$. The timing jitter of the photon sensor plays a major role when $D$ is relatively short ($<1$~m), whereas the dispersion effect gradually becomes the
dominant factor when the distance from the incident point to the
photon sensor $D$ is large ($>1.5$~m). A proper optical design can be used; then, the dispersion effect can be corrected by position information if a very precise timing is maintained.
For the DTOF detector, a typical $D$ value is approximately $0.5-1$~m; hence, the timing uncertainty due to the dispersion effect is
smaller than the timing jitter of the photon sensor. This means that a compact design for the
DTOF with no focusing component is desirable. In addition, the spatial
resolution, including the thickness of fused silica and the pixel size of the photon sensor, has little
effect on the time uncertainty of the DTOF. Therefore, a photon sensor with a large pixel size can be
used to reduce the number of electronic readout channels. Notably, $N_{PE}$ may also
increase with the thickness of the radiator. However, this can cause an increase in the material
budget, although it has little influence on the SPE time resolution.

From the simulation, the average number of photons detected by the MCP-PMT arrays is $N_{PE}\approx 17$.
By applying a TOP-position calibration, where the average track length collected by each
sensor pixel and the average velocity of photons are used to calculate the TOP
(no dispersion effects are accounted for), a timing resolution of $\sim 20$~ps or better
for the latter three effects expressed in Eq.~\ref{uncertainty} can be obtained.
To further study the time resolution of the DTOF detector, a reconstruction algorithm is
required, which is presented in Sec.~\ref{DTOF_performance}.

\begin{figure}[!htb]
	\centering
	\includegraphics[width=0.6\textwidth]{Figures/Figs_04_00_DetectorSubSystems/Figs_04_03_ParticleIdentification/SPE_resolution.jpg}
\caption{Main DTOF timing error factors and their dependences on the distance from the incident point of the particle (kaon
at $p=1$~GeV/c) to the photon detector}
	\label{FIG:SPE_Resolution}
\end{figure}

\subsubsection{DTOF Detector Layout}
%{Layout of the DTOF}
\label{sec:dtof_layout}
Based on the results from both the simulation study and the experimental test, we developed the conceptual design of the DTOF at the STCF, as shown in Fig. \ref{FIG:DTOF_Conceptual_Design}. The detailed structure inside a sector is also depicted in Fig.~\ref{FIG:DTOF_Conceptual_Design}. The planar synthetic fused silica radiator is fan shaped and can be viewed as a composite structure of 3 trapezoidal units, each $\sim 295$~mm (inner side) $/$ $\sim 533$~mm (outer side) wide, $\sim 470$~mm high and $15$~mm thick. An array of $3 \times \sim 14-16$ multi-anode MCP-PMTs are directly coupled to the radiator along the outer side. The whole sector is enclosed in a light-tight black box made of $5$~mm thick carbon fiber, occupying $\sim 200$~mm space along the beam (Z) direction.

\begin{figure}[!htb]
	\centering
	\includegraphics[width=0.75\textwidth]{Figures/Figs_04_00_DetectorSubSystems/Figs_04_03_ParticleIdentification/DTOF_Conceptual_Structure1.jpg}
	\includegraphics[width=0.20\textwidth]{Figures/Figs_04_00_DetectorSubSystems/Figs_04_03_ParticleIdentification/DTOF_Conceptual_Structure2.jpg}
\caption{The conceptual design of the DTOF detector.}
	\label{FIG:DTOF_Conceptual_Design}
\end{figure}

\subsection{DTOF Performance Simulation}
%{DTOF Performance}
\label{DTOF_performance}

\begin{comment}
\subsubsection{Simulation}
Geant4 simulations are performed to study the expected performance of the designed DTOF detector. A $20 mm$ thick aluminum plate is added at a distance of $100 mm$ before the DTOF detector to simulate the material budget of the MDC endcap. When tracking photon propagation, the inner and outer sides of the DTOF radiator are set to be absorptive, while the two lateral sides are set to be reflective (reflection factor $\sim 92\%$). The surface roughness of the radiator is simulated by randomizing the normal direction of the facet by $\sigma = 0.1^\circ$ (corresponding to an average reflection factor of $\sim 97\%$). Pion and kaon particles are emitted from the interaction point at different momenta and directions. Different polar angles, azimuth angles and particle momenta are tested. A typical Cherenkov photon hit pattern for pions at $p = 1~GeV/c$, $\theta = 23.66^\circ$ and $\phi = 15^\circ$ is displayed in Fig. \ref{FIG:DTOF_TOP_Resolution}. A clear correlation between the TOP and the hit position is demonstrated by the simulation. There are two bands in the figure: the lower left band represents direct photons without reflection, and the upper right band represents indirect photons with one reflection off the lateral side. %Editor: Please ensure that the intended meaning has been maintained in this edit.
Obviously, good separation between the two bands is obtained, except for from a few sensors close to the edge. The mean number of photons detected by the MCP-PMT arrays is $\sim 17$. By applying a TOP-position calibration, where the average track length collected by each sensor pixel and the average velocity of photons are used to calculate the TOP (no dispersion effects are accounted for), a timing resolution of $\sim 20 ps$ or better is obtained (as shown in Fig. \ref{FIG:DTOF_TOP_Resolution}).

\begin{figure}[!htb]
	\centering
	\includegraphics[width=0.45\textwidth]{Figures/Figs_04_00_DetectorSubSystems/Figs_04_03_ParticleIdentification/DTOF_TOP_1.jpg}
	\includegraphics[width=0.45\textwidth]{Figures/Figs_04_00_DetectorSubSystems/Figs_04_03_ParticleIdentification/DTOF_TOPReso_1.jpg}
\caption{The simulated TOP vs. hit position pattern of the DTOF detector and the timing resolution after calibration.}
	\label{FIG:DTOF_TOP_Resolution}
\end{figure}




The expected timing resolution in different DTOF regions is deduced according to Equation (~\ref{EQ:DIRC-Time-Reso}), as shown in Fig. \ref{FIG:DTOF_TReso_NPE} for pions (kaons are very similar) at $2~GeV/c$. The black numbers represent the total time resolution, while the red numbers represent the intrinsic detector timing resolution $\sigma_{det}$. It is clear that $\sigma_{det}$ becomes worse at lower $\theta$, which corresponds to hits closer to the inner side of the DTOF radiator, and slightly improves at larger $\phi$ (best at $\phi = 45^{\circ}$), which corresponds to hits farther from the lateral side of the DTOF radiator. This $\theta$ and $\phi$ dependence is mainly reflected by the total number of detected Cherenkov photons, as also shown in Fig. \ref{FIG:DTOF_TReso_NPE}. It is also evident that the event start time resolution $\sigma_{T_{0}} = 40 ps$ dominates over all timing errors, so an optimized STCF bunch size is crucial. Furthermore, we find that the constraint on $\sigma_{elec}$ ($10 ps$ in this case) can be relaxed since the single-photon response of an MCP-PMT $\sigma_{TTS}$ dominates the terms with $\frac{1}{NPE}$ scaling.

\begin{figure}[!htb]
	\centering
	\includegraphics[width=0.45\textwidth]{Figures/Figs_04_00_DetectorSubSystems/Figs_04_03_ParticleIdentification/DTOF_TReso_All_1.jpg}
	\includegraphics[width=0.45\textwidth]{Figures/Figs_04_00_DetectorSubSystems/Figs_04_03_ParticleIdentification/DTOF_NPE_1.jpg}
\caption{The expected overall DTOF timing resolution (left) and the number of detected direct photon hits for $\pi^{\pm}$ particles at $2~GeV/c$.}
	\label{FIG:DTOF_TReso_NPE}
\end{figure}

With the expected timing performance obtained, the PID power, i.e., the $K/\pi$ separation power, is defined by the following formula:
\begin{equation}
\label{eq::DTOF-PID-Power}
\Delta T=\frac{L}{\beta c} \frac{\Delta m^2}{2p^2}\sim\frac{L}{c}\frac{\Delta m^2}{2p^2},
PID power = \frac{\Delta T}{\sigma_{tot}},
\end{equation}
where $\Delta T$ is the TOF difference between pions and kaons with the same momentum $p$, $L$ is the flight path length, and $\sigma$ is the total time resolution of the DTOF. The results are shown in Fig. \ref{FIG:DTOF_PIDPower} for $2~GeV/c$ $\pi/K$ in different regions of the DTOF detector. Due to the longer $L$ and better timing resolution at larger $\phi$, the DTOF PID capabilities are better for tracks hitting the radiator closer to its outer side. The region that performs the worst is located at the corner of the inner and lateral edges, where the separation power is as high as $3.1$ and still acceptable.

\begin{figure}[!htb]
	\centering
	\includegraphics[width=0.6\textwidth]{Figures/Figs_04_00_DetectorSubSystems/Figs_04_03_ParticleIdentification/DTOF_PID_Power_1.jpg}
\caption{The expected PID capabilities under the given timing performance for $\pi/K$ separation at $2~GeV/c$.}
	\label{FIG:DTOF_PIDPower}
\end{figure}
\end{comment}

%% The following paragraphs are based on the work by BinBin Qi
\subsubsection{Reconstruction Algorithm}
{\sc Geant4} simulations are performed to predict the performance of the DTOF detector.
A $20$~mm thick aluminum plate is added at a distance of $100$~mm in front of the DTOF detector to simulate the material budget of the MDC endcap. When tracking photon propagation, the inner and outer sides of the DTOF radiator are set to be absorptive, while the two lateral sides are set to be reflective (reflection factor $\sim 92\%$, typical for reflective coatings). The surface roughness of the radiator is simulated by randomizing the normal direction of the facet by $\sigma = 0.1^\circ$ (corresponding to a conservative average reflection factor of $\sim 97\%$). The wavelength dependence of the refractive index, absorption length for the radiator, and quantum efficiency of MCP-PMTs are carefully taken into account. Pion and kaon particles are emitted from the IP at different momenta and directions. Different polar angles, azimuth angles and particle momenta are tested. A typical Cherenkov photon hit pattern of pions at $p = 1$~GeV/c, $\theta = 23.66^\circ$ and $\phi = 15^\circ$ is displayed in Fig.~\ref{FIG:DTOF_TOP_Resolution}. A clear correlation between the time of propagation (TOP) and the hit position is demonstrated by the simulation. There are two bands in the figure: the lower left band represents direct photons without reflection, and the upper right band represents indirect photons with one reflection off the lateral side. %Editor: Please ensure that the intended meaning has been maintained in this edit.
Obviously, good separation between the two bands is obtained, except from a few sensors close to the edge.
% (as shown in Fig. \ref{FIG:DTOF_TOP_Resolution}).


\begin{figure}[!htb]
	\centering
	\includegraphics[width=0.6\textwidth]{Figures/Figs_04_00_DetectorSubSystems/Figs_04_03_ParticleIdentification/DTOF_TOP_1.jpg}
\caption{The simulated TOP vs. hit position pattern of the DTOF detector.}
	\label{FIG:DTOF_TOP_Resolution}
\end{figure}

The DTOF reconstruction is performed in the coordinates shown in Fig.~\ref{FIG:DTOF_Reco_Coor} for one DTOF quadrant. According to the Cherenkov angle relation
\begin{equation}
\cos(\bar{\theta_{c}}) = \frac{1}{n_{p}\beta} = \frac{{\vec{v_{t}}} \cdot {\vec{v_{p}}}}{|\vec{v_{t}}| \cdot |\vec{v_{p}}|},
\end{equation}
where $\vec{v_{t}}=(a,b,c)$ is the incident particle velocity vector when the particle hits the radiator, $\vec{v_{p}}$ is the velocity vector of the emitted Cherenkov photons, $n_{p}$ is the refractive index of the radiator, and $\beta$ is the reduced speed of the particle.
The directional components of $\vec{v_{p}}$ can be expressed as $(\Delta{X},\Delta{Y},\Delta{Z})$, representing
the 3D spatial difference between the photon sensor pixel and the incident position of the particle on the radiator surface, as depicted in Fig. \ref{FIG:DTOF_Reco_Coor} (right). Although the 2D (X and Y) difference can be readily obtained, $\Delta{Z}$ must be deduced with a certain particle species hypothesis. If $V = \cos(\bar{\theta_{c}})$ is known, the equation for $\Delta{Z}$ can be expressed as
\begin{equation}
\label{eq::DTOF-DZ-Solver}
(c^{2}-V^{2})\Delta{Z}^{2} + 2c(a\Delta{X}+b\Delta{Y})\Delta{Z} + (a\Delta{X}+b\Delta{Y})^{2} - V^{2}(\Delta{X}^{2}+\Delta{Y}^{2}) = 0.
\end{equation}
By solving this equation, we find $\Delta{Z}=\frac{-B\pm\sqrt{B^{2}-4AC}}{2A}$, with $A=c^{2}-V^{2}$, $B=2c(a\Delta{X}+b\Delta{Y})$ and $C=(a\Delta{X}+b\Delta{Y})^{2} - V^{2}(\Delta{X}^{2}+\Delta{Y}^{2})$. To obtain a real solution, $\Delta = B^{2}-4AC \ge 0$ is required. Furthermore, after the physical cuts $V > 0$ (Cherenkov photons are forwardly emitted) and $\frac{\Delta{X}^{2}+\Delta{Y}^{2}}{\Delta{X}^{2}+\Delta{Y}^{2}+\Delta{Z}^{2}} \ge \frac{1}{n_{p}^{2}}$ (internal total reflection is ensured) are applied, the minimal solution of Eq.~\ref{eq::DTOF-DZ-Solver}, ($\Delta{Z} = \min{|\Delta{Z}_{1}|,|\Delta{Z}_{2}|}$) is taken as the optimum.

\begin{figure}[!htb]
	\centering
	\includegraphics[width=0.55\textwidth]{Figures/Figs_04_00_DetectorSubSystems/Figs_04_03_ParticleIdentification/DTOF_Reco_Coor.jpg}
	\includegraphics[width=0.35\textwidth]{Figures/Figs_04_00_DetectorSubSystems/Figs_04_03_ParticleIdentification/DTOF_Reco_Dir.jpg}
\caption{The coordinate system used in DTOF reconstruction (left) and the direction of Cherenkov photons (deep blue line).}
	\label{FIG:DTOF_Reco_Coor}
\end{figure}

The timing error of such an approach is estimated by adding up the possible factors, such as the dispersion effect, the finite photon sensor size and the propagation length of photons inside the radiator. The expected timing uncertainty for a pion of $p \sim 1 - 2$~GeV/c crossing the radiator perpendicularly, with $H = 0.5$~m (in the coordinate system defined in Fig.~\ref{FIG:DTOF_Reco_Coor}), is shown in Fig.~\ref{FIG:DTOF_Reco_TLerr} for sensors at different positions. The pitch of the photon sensor is $5.5$~mm. No multiple Coulomb scattering (MCS) effects are accounted for. It is obvious that the intrinsic detector timing uncertainty is no more than $40$~ps with this DTOF structure. Furthermore, we find that the reconstructed length of propagation (LOP) of light inside the radiator agrees with the MC truth to a precision of $\sim 3.3$~mm, as also shown in Fig.~\ref{FIG:DTOF_Reco_TLerr}. The reconstruction algorithm works well for most photon sensors independent of the incident position of the particles, except for a few sensors near the lateral side.

\begin{figure}[!htb]
	\centering
	\includegraphics[width=0.45\textwidth]{Figures/Figs_04_00_DetectorSubSystems/Figs_04_03_ParticleIdentification/DTOF_Reco_Terr.jpg}
	\includegraphics[width=0.45\textwidth]{Figures/Figs_04_00_DetectorSubSystems/Figs_04_03_ParticleIdentification/DTOF_Reco_Lerr.jpg}
\caption{The expected timing error and propagation length uncertainty of Cherenkov photons in a DTOF quadrant.}
	\label{FIG:DTOF_Reco_TLerr}
\end{figure}

By applying the formula below
\begin{equation}
\label{eq::DTOF-TOF}
TOF = T - TOP - T_{0} = T - \frac{LOP}{v_{g}} - T_{0} = = T - \frac{LOP\times\bar{n_{g}}}{c} - T_{0},
\end{equation}
where $v_{g}$ is the group velocity of Cherenkov light in the radiator, the TOF information is obtained. Figure~\ref{FIG:DTOF_Reco_TOFreso} shows the time resolution of the DTOF detector for a SPE and the average of all photons without taking into account the timing jitter of the MCP-PMTs and electronics. For an SPE, the intrinsic time resolution of the DTOF is $\sim 41$~ps. Averaging the timing information over $\sim$~18 detected photons, the timing jitter shrinks to $\sim$~10~ps. It is also noted that in the TOF distribution plot, a low (but visible) long tail appears on both sides of the main peak. The tail is mainly caused by secondary particles along the primary pion, mostly $\delta$-electrons.

\begin{figure}[!htb]
	\centering
	\includegraphics[width=0.45\textwidth]{Figures/Figs_04_00_DetectorSubSystems/Figs_04_03_ParticleIdentification/DTOF_Reco_TOFspe.jpg}
	\includegraphics[width=0.45\textwidth]{Figures/Figs_04_00_DetectorSubSystems/Figs_04_03_ParticleIdentification/DTOF_Reco_TOFall.jpg}
\caption{The TOF resolution of the DTOF detector for a single photoelectron and the average of all photons.}
	\label{FIG:DTOF_Reco_TOFreso}
\end{figure}

The TOF information is deduced and compared to the expectation of each particle hypothesis. Figure~\ref{FIG:DTOF_Reco_TOFPID} shows the reconstructed intrinsic TOF distributions of both pions and kaons at $2~$GeV/c. We can easily find if the particle hypothesis is correct the reconstructed TOF peak is at the correct position, with a resolution of $\sim$~10~ps. However, if the hypothesis is not correct, the reconstructed TOF peak is shifted with respect to the expectation. The shift makes the separation between the pion and kaon TOF peaks even larger, which may benefit the PID power. When convoluting all contributing factors, the overall reconstructed TOF time resolution is $45\sim50$~ps, as shown in Fig. \ref{FIG:DTOF_Reco_TOFPID}. Directly comparing the TOF information shows that a $3.0\sigma$ separation power for $\pi/K$ at $2$~GeV/c is achieved. Furthermore, the separation power becomes stronger if we compare the reconstructed TOFs of various hypotheses for the same set of particles, mainly due to the beneficial time shift under the incorrect hypothesis (as in Fig. \ref{FIG:DTOF_Reco_TOFPID}).

\begin{figure}[!htb]
	\centering
	\includegraphics[width=0.45\textwidth]{Figures/Figs_04_00_DetectorSubSystems/Figs_04_03_ParticleIdentification/DTOF_Reco_TOFPID_1.jpg}
	\includegraphics[width=0.45\textwidth]{Figures/Figs_04_00_DetectorSubSystems/Figs_04_03_ParticleIdentification/DTOF_Reco_TOFPID_2.jpg}
\caption{The TOF PID capabilities of the DTOF detector for $\pi/K$ separation at $2$~GeV/c, without (left) and with (right) contributions from other timing uncertainties.}
	\label{FIG:DTOF_Reco_TOFPID}
\end{figure}

\subsubsection{Expected Performance}

To evaluate the PID capabilities of the DTOF detector, we apply a likelihood method. The likelihood function is constructed by
\begin{equation}
\label{eq::DTOF-Likelihood-Function}
\mathcal{L}_{h} = \Pi^{i}_{i=1}f_{h}(TOF^{h}_{i}), ~~~~~
\Delta\mathcal{L} = \mathcal{L}_{\pi} - \mathcal{L}_{K},
\end{equation}
where $h$ denotes hadron species (in our case, $\pi$ and $K$) and $i$ accounts for each detected photon. The probability density function $f_{h}$ is taken as a Gaussian fit to the expected TOF distribution (as in Fig. \ref{FIG:DTOF_Reco_TOFPID} (right)), plus a constant background of 0.05. Fig.~\ref{FIG:DTOF_Reco_LHPID} (left) shows an example of the reconstructed $\Delta\mathcal{L}$ for $2$~GeV/c $\pi$ and $K$, where a $\pi/K$ separation power better than $4\sigma$ is demonstrated. Fig.~\ref{FIG:DTOF_Reco_LHPID} (right) shows the corresponding identification efficiency under this condition. When a kaon misidentification efficiency of $<2\%$ is required, a pion identification efficiency of $>98\%$ can be achieved, fulfilling the STCF PID requirement. Furthermore, with improved PID algorithm that includes also the spatial hit pattern information, the PID performance of DTOF can extend to even higher momentum range. Fig.~\ref{FIG:DTOF_Reco_LHPID_Scan} shows the likelihood PID capabilities of the DTOF detector for $\pi/K$ separation in different directions and different momenta. Despite the very different particle directions and momenta, a separation power of $\sim4\sigma$ or better over the full DTOF sensitive area is achieved.

\begin{figure}[!htb]
	\centering
	\includegraphics[width=0.45\textwidth]{Figures/Figs_04_00_DetectorSubSystems/Figs_04_03_ParticleIdentification/DTOF_Reco_LHPID_2.jpg}
	\includegraphics[width=0.45\textwidth]{Figures/Figs_04_00_DetectorSubSystems/Figs_04_03_ParticleIdentification/DTOF_Reco_LHPID_3.jpg}
\caption{The likelihood PID capabilities of the DTOF detector for $\pi/K$ separation at $2$~GeV/c emitted at different angles.}
	\label{FIG:DTOF_Reco_LHPID}
\end{figure}

\begin{figure}[!htb]
	\centering
	\includegraphics[width=0.95\textwidth]{Figures/Figs_04_00_DetectorSubSystems/Figs_04_03_ParticleIdentification/DTOF_Reco_LHPID_scan_new.jpg}
\caption{The likelihood PID capabilities of the DTOF detector for $\pi/K$ separation in different directions and at different momenta.}
	\label{FIG:DTOF_Reco_LHPID_Scan}
\end{figure}

\subsubsection{$T_0$ Determination}
\label{sec:dtof_t0}
The DTOF detector's excellent timing performance can help determine the $e^{+}e^{-}$ collision time ($T_{0}$). According to Eq.~\ref{eq::DTOF-TOF}, the TOF of incident particles and the TOP of Cherenkov photons can be obtained from simulations or calculations with different particles and different $T_{0}$ hypotheses, provided that the track information is known. To evaluate different hypotheses, we define a likelihood function,

\begin{equation}
	\label{eq::T0-Likelihood-Function}
	\mathcal{L}_{h_{1}, h_{2}} = \Pi^{i}_{i=0}f_{h_{1}, h_{2}} (ch_{i}, T_{i}), 
\end{equation}

where $h_{1}$ denotes hadron species (in our case, $\pi$ and $K$) and $h_{2}$ denotes different $T_{0}$ (the bunches collide every $4 ns$, i.e., $T_{0} = 0$, $\pm 4 ns$, $\pm 8 ns$,...). The probability density function $f_{h_{1}, h_{2}} (ch_{i}, T_{i})$ is the photon arrival-time distribution of different channels. The ${L}_{h_{1}, h_{2}}$ with different particle and $T_{0}$ hypotheses is compared, and the $T_{0}$ candidate with the maximum likelihood is determined.
Fig.~\ref{FIG:DTOF_T0_Reco} shows the $T_{0}$ determination efficiency using the DTOF detector for $\pi$ samples in different directions and at different momenta. Despite the very different particle directions or momenta, $T_{0}$ can be determined correctly with an efficiency $>99\%$. Note that if the particle velocity is below the Cherenkov threshold, the corresponding $T_{0}$ cannot be correctly pinpointed. With decreasing momentum, the time resolution of the DTOF detector worsens, resulting in a decrease in the $T_{0}$ determination efficiency.

\begin{figure}[!htb]
	\centering
	\includegraphics[width=0.95\textwidth]{Figures/Figs_04_00_DetectorSubSystems/Figs_04_03_ParticleIdentification/DTOF-T0sep.png}
\caption{The rate at which $T_{0}$ is determined correctly using the DTOF detector for $\pi$ samples in different directions and at different momenta.}
	\label{FIG:DTOF_T0_Reco}
\end{figure}


\subsection{DTOF Structure Optimization}

Different geometric parameters of the DTOF detector are tested to study their effects on PID. The $\pi$/K separation powers of different geometry configurations are compared with the reconstruction algorithm and likelihood method described above. The geometry configurations studied are listed in Table~\ref{TAB:DTOF_GEO_CONFIG}. We study the effects of three main factors: radiator shape/size, radiator thickness and the setting of mirrors.

%%%%%%%%%%%%%%%%%  TABLE  %%%%%%%%%%%%%%%%%%%%%%%%
\begin{table*}[hptb]
	\small
	\caption{Description of the different DTOF geometry configurations, where A stands for absorber and R for reflective mirror.}
	\label{TAB:DTOF_GEO_CONFIG}
	\vspace{0pt}
	\centering
	\begin{tabular}{llllllll}
	\hline
	\thead[l]{Configuration/Geometry ID} & \thead[l]{0} & \thead[l]{1} & \thead[l]{2} & \thead[l]{3} & \thead[l]{4} & \thead[l]{5} & \thead[l]{6} \\
	\hline
	Radiator shapes (sector number)	&4        &12	&24	&4	&4	    &4  &4 \\
	Radiator thickness (mm)	    &15	    &15   	&15      &10     &20    &15   	&15   \\
	Outer side surface	&A	&A    	&A    	&A      	&A     	&R   &$45^{\circ}$ R  \\
	Inner side surface	 &A	&A    	&A    	&A      	&A     	&A   &A  \\
	Lateral side surface	&R	&R    	&R    	&R      	&R    	&R   &R  \\
	\hline	\end{tabular}
\end{table*}
%%%%%%%%%%%%%%%%%%%%%%%%%%%%%%%%%%%%%%%%%%%%%%%%%%

In Table~\ref{TAB:DTOF_GEO_CONFIG}, three different radiator shapes (and sizes) correspond to Geometries 0, 1 and 2, where the DTOF disc of Geometry 0 is made up of 4 quadrant sectors, as in Sec.~\ref{sec:dtof_layout}, and for Geometries 1 and 2, the DTOF disc is made up of 12 and 24 trapezoidal sectors, respectively. Each sector includes readouts from 18 and 8 MCP-PMTs for Geometries 1 and 2, respectively. The effect of the radiator thickness is studied by comparing Geometries 0, 3 and 4. The radiator thicknesses are 15 mm, 10 mm and 20 mm, respectively. For all the geometric configurations listed in Table~\ref{TAB:DTOF_GEO_CONFIG}, the inner/lateral side surfaces of the radiator are covered by an absorber/reflective mirror, labeled A/R, respectively. For the outer side surface, a mirror can extend the acceptance of Cherenkov light, which increases the number of detected photons. As shown in Fig.~\ref{FIG:DTOF_OPTIMIZATION}, in Geometry 5, a mirror is attached to the outer side surface of the radiator, which is equivalent to putting mirror MCP-PMTs parallel to the real tubes, and in Geometry 6, a mirror is placed on the $45^{\circ}$ chamfer, which is equivalent to putting mirror MCP-PMTs perpendicular to the real tubes.

\begin{figure}[!htb]
	\centering
	\includegraphics[width=0.75\textwidth]{Figures/Figs_04_00_DetectorSubSystems/Figs_04_03_ParticleIdentification/DTOF_Geo_Optimization_New.jpg}
\caption{Three different configurations on the outer surface of the radiator. An absorber (left) or mirror (middle) on the outer surface and a mirror on the $45^\circ$ chamber of the outer side surface (right).}
	\label{FIG:DTOF_OPTIMIZATION}
\end{figure}

The key results regarding the DTOF performance for different geometry configurations are listed in Table ~\ref{TAB:DTOF_GEO_PERFORM} for $\pi/K$ mesons at $p = 2$~GeV/c, $\theta = 24^{\circ}$ and $\phi = 45^{\circ}$.

%%%%%%%%%%%%%%%%%  TABLE  %%%%%%%%%%%%%%%%%%%%%%%%
\begin{table*}[hptb]
	\small
	\caption{Performance of different geometries at $p = 2$~GeV/c, $\theta = 24^{\circ}$ and $\phi = 45^{\circ}$.}
	\label{TAB:DTOF_GEO_PERFORM}
	\vspace{0pt}
	\centering
	\begin{tabular}{llllllll}
		\hline
		\thead[l]{Configuration/Geometry ID} & \thead[l]{0} & \thead[l]{1} & \thead[l]{2} & \thead[l]{3} & \thead[l]{4} & \thead[l]{5} & \thead[l]{6} \\
		\hline
		$N_{pe}$ for pions	&21.8       &21.9	&17.0	&15.5	&25.7	    &33.2  &38.7 \\
		Accumulated charge density on     &10.8	    &10.5   	&9.6      &8.8     &11.8    &17.0   	&25.6   \\
		MCP-PMT anode ($C/{\rm cm}^{2}$)	    &	    &   	&      &     &    &   	&   \\
		$\pi$/K separation power	&$4.17 \sigma$	&$4.08 \sigma$	&$3.66 \sigma$	&$3.99 \sigma$	&$4.27 \sigma$	&$4.26 \sigma$   &$4.19 \sigma$  \\
		\hline	\end{tabular}
\end{table*}
%%%%%%%%%%%%%%%%%%%%%%%%%%%%%%%%%%%%%%%%%%%%%%%%%%


The effect of radiator shape/size is studied by comparing the DTOF performance with Geometries 0, 1 and 2. As the radiator becomes smaller, the reflection time of Cherenkov light off the lateral-side mirror increases, which causes more photon losses and, more importantly, ``confusion'' in LOP reconstruction. Geometry 0 has the best $\pi/K$ separation power, $4.17 \sigma$, while Geometry 2 has the worst, $3.66 \sigma$, indicating that a larger radiator is preferred.

Geometries 3 and 4 have different radiator thicknesses than Geometry 0, which affects the photon yield. Although more detected photons mean better time resolution and PID performance, we favor the 15 mm thick radiator, as it offers the best balance between reducing the impact of the material budget on EMC and providing performance redundancy, e.g., for reducing the influence of the detector aging effect in long-term operation.

To increase the photon yield, mirrors are attached to the outer side surface of the radiator in different ways in Geometries 5 and 6. The numbers of p.e. received in these two geometries are $\sim33$ and $\sim39$, which are much higher than that in Geometry 0. However, the mirror also increases the number of possible light paths, which causes ``confusion'' similar to the effect of multiple reflections off the lateral-side mirror and degrades the time resolution. Therefore, even with more detected photons, the $\pi/K$ separation powers of Geometries 5 and 6 are similar to that of Geometry 0. In addition, the accumulated charge densities of these two geometry configurations are much higher, which affects the lifetime of the MCP-PMTs. Thus, options with mirrors attached to the outer-side surface of the radiator are not favorable.

According to the above comparison, the optimum geometry configuration of the DTOF detector is Geometry 0.
It is worth noting that the large radiator of Geometry 0 may result in inefficiency when two tracks hit the same radiator in one event. The rate of inefficiency due to such pile-ups is studied in two kinds of MC samples, $J/\psi\to anything$ and $\psi(3770)\to D^{0}\bar{D}^{0}\to anything$. By requiring the polar angle
of the charged tracks to be within $(22^{\circ}, 36^{\circ})$, the multiplicity of these endcap tracks is examined, where
one track is dominant and three tracks are negligible. For the case of two tracks, if the difference in their
azimuth angles is within $(-45^{\circ}, 45^{\circ})$, it is treated as a pile-up event and causes inefficiency.
These studies show that the rates of inefficiency for these two MC samples are 0.36\% and 0.63\%, respectively.
It can be concluded that the large radiator of the DTOF design does not affect the reconstruction efficiency for most
physics programs and is acceptable at the STCF.
Therefore, Geometry 0 can be chosen as the baseline design.

\subsection{Background Simulation}

The distributions of background particles obtained by the dedicated MDI and background study, as discussed in Sec.~\ref{sec:mdi_bkg}, are used as input in a DTOF {\sc Geant4} simulation and then to simulate the effect of background on DTOF performance. The background particles influencing the DTOF detector are mainly secondary gammas and electrons. Their hit rates are approximately $7\times10^{9}$~Hz and $6\times10^{7}$~Hz, respectively, including two main parts: the beam-induced background ($75\%$) and the physical background ($25\%$). The probability that the background particle generates single photon-electron signal is very low,  and the simulated background value is approximately 10 hits per quadrantal sectors in a 100~ns time window, based on the following simulations. This is consistent with data shown in Table~\ref{tab:TIDNIEL_mean} and Table~\ref{tab:TIDNIEL_max}. The beam-induced background is uniformly distributed in time, while the physical background is related to collisions during bunch crossing (once per 8 ns), exhibiting a characteristic time structure. With Monte Carlo sampling of the physical background time distribution combined with the uniform time distribution of the beam-induced background, the overall time distribution of background hits on the DTOF detector can be obtained, as shown in Fig.~\ref{FIG:DTOF_ALLBKG_T}.


\begin{figure}[!htb]
	\centering
	\includegraphics[width=0.65\textwidth]{Figures/Figs_04_00_DetectorSubSystems/Figs_04_03_ParticleIdentification/DTOF_AllBkg_New.jpg}
\caption{Overall time distribution of background particles hitting the DTOF detector.}
	\label{FIG:DTOF_ALLBKG_T}
\end{figure}


$\pi$ and K particles are generated in the {\sc Geant4} simulation along with the background samples to study the effect of the background. In the {\sc Geant4} simulation, the time window of the signal acquisition is 100 ns and is placed within the interval [-40 ns, 60 ns] so that the real signal is in the middle of the time window. In this time window, the number of background particles is given by a Poisson distribution, and the time distribution is sampled according to Fig~\ref{FIG:DTOF_ALLBKG_T}. The background may greatly increase the number of photoelectrons detected by the DTOF detector for a single event, resulting in an increased possibility of multiple hits in a single channel. The correction of multiple hits is applied, which means that in the time window of [-40 ns, 60 ns], only the first arriving photoelectron signal is taken, and all other hits are dismissed. With the background hits taken into account and assuming an MCP-PMT gain of $10^{6}$, the average accumulated charge density on the MCP-PMT anode is 12~C/cm$^{2}$ over 10 years of STCF operation ($50\%$ run time). Under such a background level, the radiation effects of the quartz radiator and the MCP-PMTs should be small according to studies of other DIRC and DIRC-like detectors.


Fig.~\ref{FIG:DTOF_XTHIT_BKG} shows the 2D time-position map of DTOF hits. Hits by the background particles are uniformly distributed throughout the phase space, while the real signal hits are concentrated as bands. After the time reconstruction, the TOF distribution of a single photoelectron signal can be obtained, as shown in Fig.~\ref{FIG:DTOF_XTHIT_BKG}. The reconstructed TOF of the real signal is a Gaussian distribution ($\sigma \sim100$ ps), while the TOFs of background particles are uniformly distributed. Some single-photon electrons with zero TOF do not meet the reconstruction conditions and are taken as background. Due to the uniform distribution of the reconstructed background signal, the influence of the background can be eliminated by using the maximum likelihood method. The $\pi$/K resolution is found to be $4.1 \sigma$, as shown in Fig.~\ref{FIG:DTOF_PID_BKG}.

\begin{figure}[!htb]
	\centering
	\includegraphics[width=0.45\textwidth]{Figures/Figs_04_00_DetectorSubSystems/Figs_04_03_ParticleIdentification/DTOF_XTHit_BKG_Cor.jpg}
	\includegraphics[width=0.45\textwidth]{Figures/Figs_04_00_DetectorSubSystems/Figs_04_03_ParticleIdentification/DTOF_SPETOF_BKG_Cor.jpg}
\caption{2-D time-position map of DTOF hits (left) and reconstructed TOF distribution of a single photoelectron signal (right), with multiple-hit correction.}
	\label{FIG:DTOF_XTHIT_BKG}
\end{figure}

\begin{figure}[!htb]
\centering
\includegraphics[width=0.65\textwidth]{Figures/Figs_04_00_DetectorSubSystems/Figs_04_03_ParticleIdentification/DTOF_PID_BKG_Cor.jpg}
\caption{$\pi/K$ identification capabilities (at $2$~GeV/c) of DTOF with multiple-hit correction.}
\label{FIG:DTOF_PID_BKG}
\end{figure}

Considering the Poisson fluctuation of the background count, the $\pi/K$ resolution remains at
4.12$\sigma$ even under an extreme condition of three standard deviations above the average background
level, {\it i.e.}, the background hit rate is increased to $2.1\times10^{8}$~Hz per single DTOF disc.
 Although the number of photoelectrons from the background increases dramatically, the final effect on $\pi/K$ separation is fairly small.

\subsection{Readout Electronics}
\label{FTOF_ELECTRONICS}

In MCP-PMT-based DTOF detectors, particle identification relies on the Cherenkov photon arrival time rather than the position information of the MCP-PMTs. Therefore, high timing resolution is an essential feature of DTOF detectors. Furthermore, DTOF detectors usually have a large channel number and require a track timing resolution below 30~ps, which is a great challenge for front-end electronics.

The preliminary structure of the DTOF readout electronics is shown in Fig.~\ref{FTOF_electronics_architecture}. The readout electronics consist of the front-end board and the data-control board. The front-end board utilizes a time over multithreshold scheme to extract timing information from analog signals. The signals from the MCP-PMTs are preamplified first. The gains of the amplifiers are set independently to compensate for the gain variations of the individual MCP-PMT channels. The amplified signals are fed into high-performance comparators, each of which has a different threshold. The outputs of these comparators are fed into the FPGA. The time-to-digital converter~(TDC) module is implemented in the FPGA. The TDC measures the arrival time of both edges of the comparator outputs with high accuracy. Then, the front-end board passes the resulting binary data stream to the data-control board. The data-control board not only collects data streams from front-end boards but also sends out a high-performance clock and control signal to the front-end boards. According to the structure described above, we can briefly estimate the amount of data that the readout electronic system feeds into the subsequent DAQ system. Assuming that the event rate of the DTOF detector is 80M hits/s, the data rate output to the subsequent DAQ system can be simply calculated to be 960 MB/s without considering the effects of the detector background noise and crosstalk.

For high time resolution, it is crucial that the front-end electronics have high bandwidth and a high SNR. The time resolution of the FPGA-based TDC was found to be 3.9~ps in a previous work of ours~\cite{fastTDC}. We therefore implement the TDC that can measure narrow pulse widths, of which the time resolution will be better than 5~ps. The TDC performance in the radiation environment will be evaluated. The results of the evaluation will affect our system structure design. A stable, low-jitter detector-wide clock distribution network also needs to be developed. Its long-term stability, short-term stability, and temperature stability will be carefully evaluated. The jitter of the clock distribution network, which is better than 15~ps, satisfies the design requirements. To further improve the measurement resolution of the leading-edge timing, the multithreshold TOT 
method may be used.

\begin{figure}[!htb]
	\centering
	\includegraphics[width=0.8\textwidth]
{Figures/Figs_04_00_DetectorSubSystems/Figs_04_03_ParticleIdentification/Figs_FTOF_Electronics/FTOF_Electronics_Struture.jpg}
\caption{The preliminary structure of the DTOF readout electronic system.}
	\label{FTOF_electronics_architecture}
\end{figure}

The DTOF readout electronic system consists of many front-end boards that need to be synchronized, so the jitter of the distribution clock network is a key parameter of the electronic system. It is necessary to develop a clock distribution network that meets the requirements of the DTOF detector based on existing technology. The conventional White Rabbit network
can provide subnanosecond accuracy and tens of picosecond precision for synchronization.
To achieve a high synchronization accuracy, a new optical fiber
link and synchronization scheme based on optical circulators and
serial transceivers embedded in FPGAs is proposed~\cite{clock}.
Through a prototype implementation and performance evaluation, the distributed clock
has shown a synchronization accuracy better than 15 ps, which
can meet the requirements of most current large-scale physics
experiments.

Finally, the readout electronics must sustain the radiation loads during the operational lifetime of the STCF. Among them, the radiation influence on the FPGA-based TDC needs to be regarded from two aspects, i.e., the damage to the FPGA-based TDC from a single event and the long-term changes in the FPGA-based TDC performance under the radiation environment.


