\subsection{Introduction}
\label{sec:rich_intro}

The particle identification for the full momentum range is essential for charm physics studies and fragmentation function studies. In particular, the precise measurement of the collision fragmentation function requires a $\pi/K$ misidentification rate less than $2\%$ up to $p=2.0\gevc$, with the corresponding identification efficiency being larger than 97\%. In addition, studies of $XYZ$ physics, $\tau$ physics and (semi)leptonic decays of charmed mesons require a good suppression power for $\mu/\pi$.

The identification of hadrons in the low momentum range is achieved through measurements of the specific energy loss rate ($dE/dx$) by the MDC. The identification of leptons and neutral particles is provided by the EMC and the MUD. The PID system of the STCF focuses on charged hadrons with a high momentum, from approximately 0.7\,$\gevc$ up to 2\,$\gevc$. To cover this range, the Cherenkov detector is one of the technologies that can fulfill those requirements.

The PID system of the STCF is placed between the EMC and MDC. The solid angle coverage $\cos\theta$ of the barrel PID is 0.83 and that of the end-cap PID ranges from $0.81$ to $0.93$.

Cherenkov radiation can be used for PID in a wide momentum range in modern high-energy physics experiments through the measurement of the characteristic radiation angle, which depends on the refractive index of the medium and the particle velocity. According to the PID requirements of various particle species and momentum ranges, different kinds of media (commonly called Cherenkov radiators) with different refractive indexes can be chosen to identify particles with momenta ranging from $\gevc$ to several hundred $\gevc$. The methods for realizing the measurement of Cherenkov light, ({\it e.g.}, the radiation angle or spatial-time hit pattern) are numerous. Two main types, namely, the RICH and the detection of internal total-reflected Cherenkov light (DIRC), are commonly employed in high luminosity experiments. Due to the space limit, the RICH detector is chosen as the baseline candidate for the PID barrel region and is described below.

Time of flight (TOF) is also a common PID technology. For the barrel region, the minimum flight distance is $\sim85$~cm. An overall TOF time resolution of $\sim 20$~ps is needed to effectively separate $\pi/K$ at $p \ge 2$\,$\gevc$. It is thus very challenging to apply the TOF technique to barrel PID detectors. However, for the end-cap region, where the flight distance is approximately $>1.5$~m, a TOF resolution of $\sim 35$~ps is adequate. The details is discussed in Sec.~\ref{sec:dtof}.
