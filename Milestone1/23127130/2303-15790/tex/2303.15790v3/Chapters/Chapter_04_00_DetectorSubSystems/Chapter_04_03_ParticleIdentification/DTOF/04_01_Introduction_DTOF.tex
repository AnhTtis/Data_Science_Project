\subsection{Introduction}

As discussed in Sec.~\ref{sec:rich_intro}, for the PID detector in the endcap region, a technology based on the detection of internal total-reflected Cherenkov light (DIRC) is adopted, and the conceptual design is described in this section.

The concept of DIRC was first introduced by the BaBar experiment~\cite{BabarTDR}. Cherenkov lights generated in long-fused silica bars are propagated to the ends through total internal reflections and then projected to an
array of photo sensors via a water expansion volume. The fused silica is taken as both a Cherenkov radiator and light guide. The angles of Cherenkov photons are maintained through hundreds of reflections, and the spatial pattern of the Cherenkov ring can be recognized for PID purposes.
An excellent $\pi/K$ separation can be achieved up to $p=4~\gevc$.
It is worth noting that with BaBar DIRC, the time resolution of a single photon is approximately $1$~ns, which is mainly used to suppress uncorrelated background through the setting of a proper time window.


With improved timing resolution, better PID capability is expected for new generation DIRC detectors, such as those proposed in the future PANDA experiment~\cite{PandaTDR} at FAIR or EID at EIC~\cite{EiC}. The 3-D measurements of $(x, y, t)$ are achieved by a multianode PMT with high-precision timing performance, namely, a microchannel plate photomultiplier tube (MCP-PMT). The volume of such a detector is rather compact, with typical thickness $\le 5$~cm (excluding the optical focusing part). The radiation resistance and mechanical robustness are good, as well as the rate capacity, all of which make it a suitable detector in high luminosity experiments.
The excellent timing capabilities of the new DIRC detectors has also led to the direct
application of the DIRC technology to high time resolution and high rate TOF measurement.
For example, assuming that the time resolution of a single photoelectron is $\sim100$~ps, then for $N_{PE} = 10$, the total timing resolution is $\sim30$~ps, where $N_{PE}$ is the number of photoelectrons.
Such a DIRC-like TOF detector was first proposed for the superB project~\cite{SuperB} and recently adopted
for the LHCb upgrade, namely, the Time Of internally Reflected CHerenkov light (TORCH) detector~\cite{torch}.
For TORCH, a single photon timing error of $\le 70$~ps and a light yield of $N_{PE} \sim 25 $ are predicted, which is expected to achieve a high precision with a time resolution of $\le 15$~ps.

According to the above discussion, a DIRC-like time-of-flight~(DTOF) detector
is supposed to meet the PID requirement for the endcap region of the STCF
owing to the
extended distance between the interaction point and the endcap PID detectors.
From physics requirements, a $\pi/K$ misidentification rate less than 2\% with corresponding identification efficiency larger than 97\% at $p=2$~GeV/c is needed. This is equivalent to a $4\sigma$ deviation
of two probability distributions. As a consequence,
with a flight length of 1.5~m for the hadrons,
a total time resolution of 35~ps is needed for the
TOF measurements.
However, the time measurement with DTOF benefits from both the time of flight of a
hadron and the time of propagation~(TOP) of a photon, which has the potential to provide a 40~ps time resolution, for the DTOF at the STCF.
In the following, the conceptual
design of the DTOF detector at the STCF and its geometry optimization are presented.
