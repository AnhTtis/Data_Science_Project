\section{Main Drift Chamber (MDC)}
\label{sec:mdc}

\subsection{Introduction}
The MDC is the main part of the tracking system of the STCF detector, providing important functions including the following:
\begin{itemize}
\item Reconstructing charged tracks together with the ITK.
\item Measuring the momentum (~$p$~)/transverse momentum ($p_T$) of charged particles (with a resolution of $\sigma_{p_T}/p_{T} < 0.5$\%@1~GeV/c).
\item Measuring the energy loss of charged particles in each cell ($dE/dx$) and providing the $dE/dx$ information (with $dE/dx$ resolution $\sim$ 6\%) to facilitate particle identification, especially for low-momentum charged particles.
\item Providing critical input to the trigger decision of the STCF detector system.
\end{itemize}
With the advantages of robustness, low cost and low material budget, drift chambers have been used as the main tracking system in many particle physics experiments, for example, BESIII~\cite{besiii}, Belle II~\cite{belle2} and GlueX~\cite{gluex}.
The inner radius of the MDC is determined to be 200~mm, a compromise between the count rate capabilities and tracking performance, and the outer radius is 850~mm.

\subsection{Conceptual Design of the MDC}
Figure~\ref{fig:4.2.01.a} shows the main geometric parameters of the MDC.
The STCF MDC adopts a square cell and a superlayer wire configuration similar to those used at BESIII and Belle II.
There are six layers of drift cells in each superlayer. Each wire layer contains a field wire layer (with all field wires) and a sense wire layer (with alternating sense and field wires).
The sense wire is 20~$\mu$m-diameter and 0.5~$\mu$m-thick gold-coated tungsten wire, and the field wire is 100~$\mu$m-diameter and 0.5~$\mu$m-thick gold-coated aluminum wire.
The cell size increases gradually from the innermost layer to the outermost layer, approximately $9.8\times9.8$~mm$^2$ to $12.5\times 12.5$~mm$^2$ in the first superlayer and $13.3\times13.3$~mm$^2$ to $14.5\times 14.5$~mm$^2$ in the outermost superlayer, as shown in Table~\ref{tab:4.2.01}.
The innermost and outermost superlayers contain axial (``A'') layers to match the shape of the inner and outer cylinders. The intervening superlayers alternate between stereo (``U'' or ``V'') and axial layers. The stereo angles are listed in Table~\ref{tab:4.2.01}.
In total, there are 8 superlayers (AUVAUVAA) and 48 layers.
The working gas is He/C$_{3}$H$_{8}$~(60/40), together with the use of other low-mass materials, to minimize the effect of multiple scattering.
The main parameters of the MDC are summarized in Table~\ref{tab:4.2.01}. Figure~\ref{fig:4.2.01.b} shows a schematic of the MDC wire structure.

%%%%%%%%%%%%%%%%%  TABLE  %%%%%%%%%%%%%%%%%%%%%%%%
\begin{table*}[htb]
\small
    \caption{The main parameters of the STCF MDC conceptual design.}
    \label{tab:4.2.01}
    \vspace{0pt}
    \centering
    \begin{tabular}{llllll}
        \hline
        \thead[l]{Superlayer} & \thead[l]{Radius (mm)}&\thead[l]{Num. of Layers} & \thead[l]{Stereo angle (mrad)}&\thead[l]{Num. of Cells}&\thead[l]{Cell size (mm)}\\
        \hline
            A & 200.0 & 6 & 0 & 128 & 9.8 to 12.5 \\
            U & 271.6 & 6 & 39.3 to 47.6 & 160 & 10.7 to 12.9 \\
            V & 342.2 & 6 & -41.2 to -48.4 & 192 & 11.2 to 13.2 \\
            A & 419.2 & 6 & 0 & 224 & 11.7 to 13.5 \\
            U & 499.8 & 6 & 50.0 to 56.4 & 256 & 12.3 to 13.8 \\
            V & 578.1 & 6 & -51.3 to -57.2 & 288 & 12.6 to 14.0 \\
            A & 662.0 & 6 & 0 & 320 & 13.0 to 14.3 \\
            A & 744.0 & 6 & 0 & 352 & 13.3 to 14.5 \\
            total & 200 to 827.3 & 48 & & 11520 & \\
        \hline
    \end{tabular}
\end{table*}
%%%%%%%%%%%%%%%%%%%%%%%%%%%%%%%%%%%%%%%%%%%%%%%%%%

%%%%%%%%%%%%%%%%%%% Fig %%%%%%%%%%%%%%%%%%%%%%%%%%
\begin{figure*}[htb]
    \centering
      \includegraphics[width=0.5\textwidth]{Figures/Figs_04_00_DetectorSubSystems/Figs_04_02_MainDriftChamber/fig01_1.pdf}
\vspace{0cm}
\caption{The schematic structure of the MDC.}
    \label{fig:4.2.01.a}
\end{figure*}
%%%%%%%%%%%%%%%%%%%%%%%%%%%%%%%%%%%%%%%%%%%%%%%%%%

%%%%%%%%%%%%%%%%%%% Fig %%%%%%%%%%%%%%%%%%%%%%%%%%
\begin{figure*}[htb]
    \centering
\subfloat[][]{\includegraphics[width=0.5\textwidth]{Figures/Figs_04_00_DetectorSubSystems/Figs_04_02_MainDriftChamber/fig01_2.pdf}}
\hspace {5 mm}
\subfloat[][]{\includegraphics[width=0.15\textwidth]{Figures/Figs_04_00_DetectorSubSystems/Figs_04_02_MainDriftChamber/fig01_3.png}}
\vspace{0cm}
\caption{(a) Cross-section view of the layout of wire layers and superlayers. (b) Enlarged cross-section view of the wire layout. Open circles represent field wires, and dots represent sense wires. The square-shaped drift cell structure can be seen, with a sense wire in the center and field wires forming a square.}
    \label{fig:4.2.01.b}
\end{figure*}
%%%%%%%%%%%%%%%%%%%%%%%%%%%%%%%%%%%%%%%%%%%%%%%%%%





On each side of the MDC, there is a 15~mm-thick aluminum endcap flange. All the sense wires and field wires are fixed between the two flanges and fastened. The inner and outer surfaces of the MDC are made of cylindrical carbon fiber composite material, with thicknesses of 1~mm and 10~mm, respectively. The drift chambers in BESIII and Belle II demonstrated that this type of material has enough mechanical strength and a low material budget, which meets the requirements of the STCF MDC design.

\FloatBarrier

\subsection{MDC Simulation and Optimization}
An extensive simulation and optimization study is performed for the conceptual design of the MDC, including the wire material and diameter, cell structure and layout, and working gas choice.
The detector simulation is based on {\sc Geant4} with $\pi$ incident particles, and track fitting is then applied to evaluate the MDC performance.

\subsubsection{Drift wires}
%\quad\\
Aluminum wire and tungsten wire are widely used in the manufacturing of MWDCs due to their low material budget and high robustness, respectively. It is demonstrated that adding a coating of gold or silver on the surface can mitigate aging effects and enhance the conductivity of wires. Thus, gold-coated aluminum wires are used as sense wires, and gold/silver-coated tungsten wires are chosen as field wires, with a coating thickness of 0.5~$\mu$m.

The wire diameter can influence the performance of the MDC. A thicker wire has better robustness but with the disadvantages of a larger material budget and a higher required working voltage.
In BESIII, the sense wire diameter is 25~$\mu$m and the field wire diameter is 110~$\mu$m with a working gas of He/C$_3$H$_8$~(60/40).
The drift chamber used in MEG-II~\cite{mdc1} indicates that the diameter of the field wire can be decreased to 50~$\mu$m with a gas mixture He/iC$_4$H$_{10}$~(90/10), resulting in a lower material budget and better momentum resolution.
A 20~$\mu$m sense wire diameter was proposed for a drift chamber for a future $e^{+}e^{-}$ collider experiment~\cite{idea}.
A compromise needs to be made between the low material budget and robustness.
Fig.~\ref{fig:4.2.04} shows the dependence of the transverse momentum resolution on the wire diameter from the simulation.
The simulation is performed with the combined inner tracker (Sec.~\ref{sec:itk}) and MDC tracking system.
Different configurations of the sense wire diameter (20~$\mu$m and 25~$\mu$m) and field wire diameter (100~$\mu$m and 110~$\mu$m) are investigated.
It can be seen that a thinner wire setting benefits the tracker system. It is decided to use a diameter of 20~$\mu$m for the sense wires and 100~$\mu$m for the field wires.

%%%%%%%%%%%%%%%%%%% Fig %%%%%%%%%%%%%%%%%%%%%%%%%%
\begin{figure*}[htb]
    \centering
{
\subfloat[][]{\includegraphics[width=70 mm]{Figures/Figs_04_00_DetectorSubSystems/Figs_04_02_MainDriftChamber/MDC_dia_pt0p0.png}}
}
\hspace{2 mm}  
{
\subfloat[][]{\includegraphics[width=70 mm]{Figures/Figs_04_00_DetectorSubSystems/Figs_04_02_MainDriftChamber/MDC_dia_pt0p5.png}}
}
\vspace{0cm}
\caption{The simulated resolution of the transverse momentum of the MDC-only tracking system with different wire diameter settings, with polar angles of (a) cos$\theta$= 0 and (b) cos$\theta$ = 0.5.}
    \label{fig:4.2.04}
\end{figure*}
%%%%%%%%%%%%%%%%%%%%%%%%%%%%%%%%%%%%%%%%%%%%%%%%%%


\subsubsection{Drift Cells}
%\quad\\
Square cells have been used for many small-cell drift chambers in particle physics experiments. The cell aspect ratio, {\it i.e.} ratio of cell width to height is around 1 in this case. 
The electric field inside a square drift cell is more symmetric and homogeneous than that in those cells with an aspect ratio other than 1, and square cells in the same layer can be easily arranged at the same radius. The square shape is adopted for drift cells in the MDC baseline design. 

%, and this is the main reason why it is selected as the baseline design.

The cell aspect ratio affects the MDC drift time distribution directly. Fig.~\ref{fig:4.2.05}  compares the drift time simulated for two different cell aspect ratios. In the simulation, the incident particle passed a drift cell at 45 degree polar angle and a distance of half of the cell width from the sense wire. The drift time for the cell aspect ratio of 1 has sizably smaller spread compared to that for a cell aspect ratio of 1.1, suggesting a better spatial resolution. 



%%%%%%%%%%%%%%%%%%% Fig %%%%%%%%%%%%%%%%%%%%%%%%%%
\begin{figure*}[htb]
    \centering
{
        \includegraphics[width=70mm]{Figures/Figs_04_00_DetectorSubSystems/Figs_04_02_MainDriftChamber/fig05_2.png}
}
\hspace{2 mm}    
{
        \includegraphics[width=70mm]{Figures/Figs_04_00_DetectorSubSystems/Figs_04_02_MainDriftChamber/fig05_1.png}
}
\vspace{0cm}
\caption{ The simulated drift time for  particles entering the drift cell at a distance of half cell width from the sense wire and at a polar angle of 45 degree, with the  cell aspect ratio of 1 (left) and 1.1 (right), respectively. }
    \label{fig:4.2.05}
\end{figure*}
%%%%%%%%%%%%%%%%%%%%%%%%%%%%%%%%%%%%%%%%%%%%%%%%%%


\subsubsection{Layer Arrangement}
%{Optimization of the Layers Layout}
%\quad\\
In the design of the layer layout, the primary parameter is the number of layers. At the STCF, the inner radius of the MDC is approximately 200~mm, and the outer radius is approximately 850~mm, which allows 40 to 52 layers of cells to achieve the required spatial resolution and momentum resolution with an acceptable detector complexity.
Fig. \ref{fig:4.2.06} illustrates the transverse momentum resolution with various layer layouts at different incident polar angles, indicating that the range of 40 to 52 layers in the MDC exhibits tiny differences. Considering the radial/azimuthal cell width ratio of 1, a larger layer number results in a smaller cell size, which is beneficial, with higher count rate tolerance and decreased drift time. A greater number of layers could also improve the $dE/dx$ performance, with a larger number of hits. However, the cost will also be higher, and the increase in the material budget would degrade the momentum resolution of charged particles, especially for low-momentum tracks.
As a compromise, it is decided to use 48 layers.

%%%%%%%%%%%%%%%%%%% Fig %%%%%%%%%%%%%%%%%%%%%%%%%%
\begin{figure*}[htb]
    \centering
{
        \includegraphics[width=70mm]{Figures/Figs_04_00_DetectorSubSystems/Figs_04_02_MainDriftChamber/MDC_mat_p0p0.png}
}
\hspace{2 mm}   
{
        \includegraphics[width=70mm]{Figures/Figs_04_00_DetectorSubSystems/Figs_04_02_MainDriftChamber/MDC_mat_p0p5.png}
}
\vspace{0cm}
\caption{The simulated transverse momentum resolution with different numbers of layer, with polar angles of cos$\theta$=0 (left) and cos$\theta$=0.5 (right).}
    \label{fig:4.2.06}
\end{figure*}
%%%%%%%%%%%%%%%%%%%%%%%%%%%%%%%%%%%%%%%%%%%%%%%%%%



\subsubsection{Working Gas Choices}
%\quad\\
The choice of working gas is essential for the performance of the MDC, affecting the time resolution, maximum count rate and other detector performance aspects.
The ideal working gas of the MDC should have a low material budget, fast drift velocity of ionized electrons and strong primary ionization for penetrating charged particles.
Mixtures of He/C$_{2}$H$_{6}$ (50/50) and He/C$_{3}$H$_{8}$ (60/40) were used in the drift chambers of BESIII and Belle II, respectively.
To find the optimal working gas choice, the characteristic parameters of different gas mixtures are calculated and compared, as summarized in Table~\ref{tab:4.2.02}.
It can be seen that the argon-based gas mixture has a small radiation length and large primary ionization power due to the high density, which leads to a larger material budget. The helium-based working gas has more balanced parameters, while the components of the gas mixture have a significant influence on the performance.
Fig. \ref{fig:4.2.03} illustrates the simulated transverse momentum resolution with different choices of working gas. The simulation is performed with the combined inner tracker-MDC tracking system at a polar angle of cos$\theta$=0.5. The result indicates that with He/C$_{2}$H$_{6}$ (50/50), the MDC has slightly better momentum resolution. However, He/C$_{3}$H$_{8}$ (60/40) has a significantly better particle stopping ability, resulting in a higher signal amplitude and better SNR. Additionally, other simulations demonstrate that the MDC with He/C$_{3}$H$_{8}$ (60/40) has a better position resolution and $dE/dx$ resolution than that with He/C$_{2}$H$_{6}$ (50/50).
As a compromise, the He/C$_{3}$H$_{8}$ (60/40) gas mixture is chosen as the working gas for the conceptual design of the STCF MDC.

%%%%%%%%%%%%%%%%%  TABLE  %%%%%%%%%%%%%%%%%%%%%%%%
\begin{table*}[htb]
\small
    \caption{The main parameters of several kinds of gas mixtures, pressure = 1 atm, temperature = 20 Celsius, magnetic field strength = 1~T .}
    \label{tab:4.2.02}
    \vspace{0pt}
    \centering
    \begin{tabular}{llllll}
        \hline
        \thead[l]{Gas Mixture} & \thead[l]{Ar/CO$_{2}$/CH$_{4}$\\(89/10/1)}&\thead[l]{He/CH$_{4}$\\(60/40)} & \thead[l]{He/C$_{2}$H$_{6}$\\(50/50)}&\thead[l]{He/C$_{3}$H$_{8}$\\(60/40)}&\thead[l]{He/iC$_{4}$H$_{10}$\\(80/20)}\\
        \hline
            Drift velocity of an electron & 5.0 & 3.7 & 4.0 & 3.8 & 3.4 \\
            v$_{d}$ (cm/$\mu$s) &&&&&\\
            Transverse diffusion coefficient & 233 & 191 & 170 &154 & 159 \\
            $\sigma$$_{L}$ ($\mu$m/$\sqrt {cm}$) @E=760 V/cm &&&&&\\
            Lorentz angle &41 &28 &29 &24 &21 \\
            $\theta$$_{L}$ (degree) @E=760 V/cm &&&&&\\
            Primary ionizing power & 30 & 10 & 23 & 30 & 21 \\
            (i.p./cm) &&&&&\\
            Radiation length& 124 & 808 & 640 & 550 & 807 \\
            (m) &&&&&\\
        \hline
    \end{tabular}
\end{table*}
%%%%%%%%%%%%%%%%%%%%%%%%%%%%%%%%%%%%%%%%%%%%%%%%%%

%%%%%%%%%%%%%%%%%%% Fig %%%%%%%%%%%%%%%%%%%%%%%%%%
\begin{figure*}[htb]
    \centering
{
        \includegraphics[width=0.6\textwidth]{Figures/Figs_04_00_DetectorSubSystems/Figs_04_02_MainDriftChamber/MDC_gas_pt0p5.png}
}
\vspace{0cm}
\caption{The simulated resolution of the transverse momentum of the MDC with different working gases, with the polar angle of cos$\theta=0.5$.}
    \label{fig:4.2.03}
\end{figure*}
%%%%%%%%%%%%%%%%%%%%%%%%%%%%%%%%%%%%%%%%%%%%%%%%%%

\FloatBarrier

\subsection{Expected Performance}
\subsubsection{Momentum and Spatial Resolution}
%\subsubsection{Expected performance of baseline tracking system}
%\quad\\
The expected performance of the combined tracking system, with the 3-layer ITK and the 48-layer MDC, is evaluated, especially the momentum resolution at different polar angles.
The expected tracking performance results are obtained using a Geant4 simulation, without considering the background contribution, detector signal readout digitization or track reconstruction, and track fitting is then performed.
%A comparison of the MDC only and ITK-MDC tracking system is also performed.
Fig.~\ref{fig:4.2.07} shows the simulated results on the resolution of impact parameters and momentum resolution.
A minimum $\sigma$$_{p}/p$ can reach approximately 0.35\% for $\pi$ particles at $p = 0.2$~GeV/c with a polar angle of cos$\theta=0$.



%%%%%%%%%%%%%%%%%%% Fig %%%%%%%%%%%%%%%%%%%%%%%%%%
\begin{figure*}[htb]
	\centering
\subfloat[][]{\includegraphics[height=50 mm]{Figures/Figs_04_00_DetectorSubSystems/Figs_04_02_MainDriftChamber/MDC_ITK_the_D0_0p8.png}}
\subfloat[][]{\includegraphics[height=50 mm]{Figures/Figs_04_00_DetectorSubSystems/Figs_04_02_MainDriftChamber/MDC_ITK_the_Z0_0p8.png}} \\
\subfloat[][]{\includegraphics[height=50 mm]{Figures/Figs_04_00_DetectorSubSystems/Figs_04_02_MainDriftChamber/MDC_ITK_the_p0p8.png}}
\subfloat[][]{\includegraphics[height=50 mm]{Figures/Figs_04_00_DetectorSubSystems/Figs_04_02_MainDriftChamber/MDC_ITK_the_pt0p8.png}}
\vspace{0cm}
\caption{The simulated resolution of the impact parameters (a) $d_0$ and (b) $z_0$ and (c) momentum $p$ and (d) transverse momentum $p_T$ as a function of $p_T$. The results with different polar angles of incident particles, with cos$\theta$=0, 0.2 and 0.8, are compared.
}
    \label{fig:4.2.07}
\end{figure*}
%%%%%%%%%%%%%%%%%%%%%%%%%%%%%%%%%%%%%%%%%%%%%%%%%%


%\subsubsection{Comparison of the performance of the ITK designs}
As described in Sec.~\ref{sec:itk}, the baseline design of the STCF ITK is a cylindrical $\mu$RWELL-based detector, and an alternative design using CMOS pixel sensors is also considered.
To evaluate the performance of different designs, the entire tracking system, with the MDC and the ITK combined, should be considered in the {\sc Geant4} simulation and track fitting.
Single hit position resolutions of $100\times400$~$\mu$m and $30\times75$~$\mu$m are assumed for the $\mu$RWELL detector and PXD, respectively.
A material budget of 0.25\%$X_0$ is assumed for both ITK designs.
Fig.~\ref{fig:4.2.08} shows the comparison of the expected performance of the two different tracking system designs.
As expected, for the spatial resolution, the PXD ITK gives much better performance than the $\mu$RWELL-based ITK, while for the momentum resolution, the two different ITK designs give similar detector performance in the low momentum range since the material budget is the limiting factor.
%
The results also indicate that the inclusion of the inner tracker improves the momentum resolution at p$_{T}$ $>$ 0.3~GeV/c. When the momentum of the incident particle is below 0.2~GeV/c and the polar angle is close to the zenith direction, the inner tracker exerts a negative influence due to the impact of the additional material budget. In general, the addition of the inner tracker can benefit the momentum resolution of the tracker system.


%%%%%%%%%%%%%%%%%%% Fig %%%%%%%%%%%%%%%%%%%%%%%%%%
\begin{figure*}[htb]
	\centering
\subfloat[][]{\includegraphics[height=50 mm]{Figures/Figs_04_00_DetectorSubSystems/Figs_04_02_MainDriftChamber/diff_ITK_D0_0p0.png}}
\subfloat[][]{\includegraphics[height=50 mm]{Figures/Figs_04_00_DetectorSubSystems/Figs_04_02_MainDriftChamber/diff_ITK_Z0_0p0.png}} \\
\subfloat[][]{\includegraphics[height=50 mm]{Figures/Figs_04_00_DetectorSubSystems/Figs_04_02_MainDriftChamber/diff_ITK_p0p0.png}}
\subfloat[][]{\includegraphics[height=50 mm]{Figures/Figs_04_00_DetectorSubSystems/Figs_04_02_MainDriftChamber/diff_ITK_pt0p0.png}}
\vspace{0cm}
\caption{The simulated resolution of the impact parameters (a) $d_0$ and (b) $z_0$ and (c) momentum $p$ and (d) transverse momentum $p_T$ as a function of the $p_T$ of an incident particle with a polar angle of cos$\theta$=0. The results with different ITK designs are compared.
For the comparison of the $z_0$ resolution, the results for the MDC-only option are not shown since the design of the MDC alone cannot provide precise $z_0$ measurements.
}
    \label{fig:4.2.08}
\end{figure*}
%%%%%%%%%%%%%%%%%%%%%%%%%%%%%%%%%%%%%%%%%%%%%%%%%%

\subsubsection{$dE/dx$ Resolution and PID Performance}
The $dE/dx$ measurement provided by the MDC can be used for particle identification, especially for low momentum charged particles.
To explore the PID potential of the MDC, the $dE/dx$ measurement of the MDC is simulated for various types of particles.
In the simulation, the {\sc GEANT4 PAI} model is used to model the primary ionization process of high-energy charged particles with the gas medium in the MDC.
The {\sc HEED} module of the {\sc GARFIELD++} package~\cite{heed} is then invoked to produce the secondary ionization caused by the very energetic electrons produced in the primary ionization process. All ionization electrons are fed into the {\sc GARFIELD++} for simulation of the electron avalanche multiplication process.
Given the computationally intensive nature of this simulation, a fast approach is adopted here instead of simulating electron avalanche multiplication using {\sc GARFIELD++} for every ionization electron.
In this approach, the distribution of the total charge induced on a sense wire due to the avalanche multiplication process of a single electron was first obtained by performing the full {\sc GARFIELD++} simulation for many single electrons.
The charge of the induced signal due to each of the ionization electrons is then generated by sampling this distribution.
The simulated $dE/dx$ measurement by a drift cell is finally taken as the sum of the induced charge on the sense wire for all ionization electrons produced in the cell and normalized by the track length in the cell.
The fluctuations of the avalanche multiplication process are fully included in the simulation by this approach.


Fig.~\ref{fig:4.2.14}~(a) shows the $\dedx$ distribution within a single MDC cell by $\pi$ particles at $p = 0.5$~GeV/c.
A truncated-average method~\cite{dedx} is used in $dE/dx$ estimation to reduce the impact of the Landau tail of the ionization energy loss on the $dE/dx$ estimation, hence improving the $dE/dx$ resolution.
In this method, 25\% of the cell hits with the highest energy deposition are removed, and only the remaining 75\% are used to calculate the average $dE/dx$ of the track.
%Figure~\ref{fig:4.2.13} shows the $dE/dx$ distribution for $\pi$ particles at $p = 0.5$~GeV/c, with and without the truncated average method.
The simulation results indicate that the truncated average method can effectively improve the $dE/dx$ resolution.
The simulated $dE/dx$ resolution is approximately 5.89\% and smaller than the requirement of 6\%, as shown in Fig.~\ref{fig:4.2.14}.



%%%%%%%%%%%%%%%%%%% Fig %%%%%%%%%%%%%%%%%%%%%%%%%%
\begin{figure*}[htb]
    \centering
{
\subfloat[][]{\includegraphics[width=0.45\textwidth]{Figures/Figs_04_00_DetectorSubSystems/Figs_04_02_MainDriftChamber/edepincell_trunc.png}}
\subfloat[][]{\includegraphics[width=0.45\textwidth]{Figures/Figs_04_00_DetectorSubSystems/Figs_04_02_MainDriftChamber/truncatedmean.pdf}}
}
\vspace{0cm}
\caption{(a) The distribution of the original $\dedx$ in one MDC cell, with penetrating $\pi$ particles with $p = 0.5$~GeV/c.
(b) The calculated $dE/dx$ resolution of $p = 0.5$~GeV/c $\pi$ with the truncated average method.}
    \label{fig:4.2.14}
\end{figure*}
%%%%%%%%%%%%%%%%%%%%%%%%%%%%%%%%%%%%%%%%%%%%%%%%%%

The truncated mean of $dE/dx$ is simulated as a function of the momentum for different particle species, as shown in Fig.~\ref{fig:4.2.16}~(left), while the $dE/dx$ PID separation power is also extracted from the simulation, as shown in Fig.~\ref{fig:4.2.16}~(right).
The $dE/dx$ PID separation power for two particles A and B is defined as follows:
\begin{equation}
S_{AB} = \frac{\dedx_{A}-\dedx_{B}}{\sigma_{\dedx(AB)}},
\end{equation}
where $\dedx_{A}$($\dedx_{B}$) is the $\dedx$ for particle A(B) and $\sigma_{\dedx(AB)}$ is the average resolution (defined as the RMS of the $\dedx$ distribution for a given type of particle) of the two particles.
Fig.~\ref{fig:dedxresolution_p} presents that the $dE/dx$ resolution of most of the particles is below 6\%, except for MIP particles.
Fig.~\ref{fig:4.2.16} shows the simulated PID performance of the MDC, with five scenarios of hypothesized signals and background particles.
It can be seen that in the low momentum region, the MDC can achieve good particle separation power.
For example, for $K/\pi$ ($p/\pi$), the MDC PID separation power is over 3 $\sigma$ up to 700 MeV/c (1300 MeV/c).


%%%%%%%%%%%%%%%%%%%% Fig %%%%%%%%%%%%%%%%%%%%%%%%%%

%%%%%%%%%%%%%%%%%%%%%%%%%%%%%%%%%%%%%%%%%%%%%%%%%%%

%%%%%%%%%%%%%%%%%%% Fig %%%%%%%%%%%%%%%%%%%%%%%%%%
\begin{figure*}[htb]
    \centering
\subfloat[][]{\includegraphics[width=0.45\textwidth]{Figures/Figs_04_00_DetectorSubSystems/Figs_04_02_MainDriftChamber/by1.png}}
\subfloat[][]{\includegraphics[width=0.45\textwidth]{Figures/Figs_04_00_DetectorSubSystems/Figs_04_02_MainDriftChamber/by3.png}}
\vspace{0cm}
\caption{The simulated relationship between $dE/dx$ and momentum with various particles (left) and the simulated PID performance of the MDC (right).}
    \label{fig:4.2.16}
\end{figure*}
%%%%%%%%%%%%%%%%%%%%%%%%%%%%%%%%%%%%%%%%%%%%%%%%%%

\begin{figure*}[htb]
    \centering
    {
        \includegraphics[height=60mm]{Figures/Figs_04_00_DetectorSubSystems/Figs_04_02_MainDriftChamber/relations_dedxresolution_p.png}
    }
    \vspace{0cm}
	\caption{The simulated relationship between momentum and $dE/dx$ resolution for various particles.}
    \label{fig:dedxresolution_p}
\end{figure*}

\FloatBarrier
\subsection{Pileup and Radiation Effects}
\label{sec:mdc_pileup_effect}
Given that the STCF detector operates with very high luminosity, additional challenges in the design of the MDC detector must be taken into account,
such as pileup and radiation effects.
% pileup
\subsubsection{Pileup Effects}
From Table~\ref{tab:TIDNIEL_max}, the expected hit rate of the MDC is approximately $4\times 10^5$~Hz/channel for the innermost layer. This is an extremely high hit rate for the MDC given the maximum drift time in its drift cell being about 250 ns and the induced signal spreading over 500 ns (as shown in Fig.~\ref{fig:4.2.18}). There is a high probability of signals overlapping in one channel. This poses big challenges to readout electronics. The MDC counting rate may have to be reduced by modifying the MDC design depending on further studies.  

%%%%%%%%%%%%%%%%%%% Fig %%%%%%%%%%%%%%%%%%%%%%%%%%
\begin{figure*}[htb]
    \centering
{
        \includegraphics[height=60mm]{Figures/Figs_04_00_DetectorSubSystems/Figs_04_02_MainDriftChamber/drift_time_distribution_for_all_electrons.png}
}
\vspace{0cm}
\caption{Drift time distribution in the STCF MDC.}
    \label{fig:4.2.18}
\end{figure*}
%%%%%%%%%%%%%%%%%%%%%%%%%%%%%%%%%%%%%%%%%%%%%%%%%%


The STCF detector is expected to operate at an event rate up to 400~kHz (Sec.~\ref{sec:tdaq}). At such a high event rate, the probability of events piling up is approximately 8 (18)\% within a time window of 200 (500) ns. This poses a severe problem for the MDC which could have drift time spread over a few hundred ns (as shown in Fig.~\ref{fig:4.2.05}). Such a time spread implies a rather large integration time window for the MDC and hence a significantly high probability of tracks from different events overlapping in the MDC. One way to resolve the overlapping tracks is to exploit the timing capability of the STCF detector. In the current STCF detector design, both the CMOS-ITK and EMC have timing capability at different levels of precision, with the former being able to reach a few ns time resolution for charged particles~\cite{aplide-improved} ~\cite{malta2}  and the latter a level of a few hundred ps for energy deposits of ~100 MeV (see discussions in Sec.~\ref{sec:emc_perf}). The time measurements by the two subdetectors can be associated to each track recorded by the MDC by spatial matching of hits between the MDC and either the MAPS-ITK or EMC or both depending on availability of time measurements on the two subdetectors. The MDC tracks can then be assigned to different events according to their associated time measurements. As a result, the overlapping events recorded by the MDC are resolved. Further studies are needed to investigate the event overlapping or pileup problem with the MDC.    



% radition and aging effects
\subsubsection{Radiation Effects}
The detailed background radiation simulation is described in Sec.~\ref{sec:mdi_bkg}, and the expected radiation levels in each subdetector are given in
Table~\ref{tab:TIDNIEL_mean} and \ref{tab:TIDNIEL_max}.
From Table~\ref{tab:TIDNIEL_max}, the MDC needs to withstand a TID of approximately 60~Gy/y and a NIEL of $4.9\times 10^{10}$ n/cm$^{2}$/y~(1 MeV neutron equivalent). Also, the highest accumulated charge can be calculated as approximately 24.2 mC/cm/y in the innermost layer. 
Therefore, aging effects in wire chambers, a permanent degradation of the operating characteristics under sustained irradiation, must be considered.
The classical aging effects are the results of chemical reactions occurring in avalanche plasma near anodes in wire chambers, leading to the formation of deposits on electrode surfaces~\cite{gaseousaging}.
For the MDC, the aging effect includes those from both anode aging and cathode aging, expected a further study in the future.

\FloatBarrier
\subsection{Readout Electronics}
%\quad\\
In the conceptual design, the MDC contains 11520 cells, each with a sense wire in the center. The total number of MDC readout electronic channels is 11520 as well. The readout electronics system is arranged in the endcap that is made of a 15 mm-thick aluminum plate for mechanical support.
The background simulation (Sec.~\ref{sec:bkg_sim}) indicates that the first (innermost) layer of the MDC is subjected to the highest background count rate, approximately 200 to 400 kHz, leading to severe interference in the measurement of charged tracks from signal events. The {\sc Garfield}~\cite{garfield} simulation demonstrates that the signal in each cell lasts for 200 to 500~ns due to the electrons distributed by the uncertainty of the distance from the particle track to the wire. As a consequence, the readout electronics must have a fast shaping time, and sufficient background shielding in the MDC endcap is necessary.


High precision time (approximately 0.5~ns RMS) and charge measurement are required for the MDC readout electronics, with an input signal amplitude up to 1.8~pC.
The MDC readout electronics are composed of FEE modules and RUs, with the schematic structure shown in Fig. \ref{fig:4.2.09}.
%%%%%%%%%%%%%%%%%%% Fig %%%%%%%%%%%%%%%%%%%%%%%%%%
\begin{figure*}[htb]
    \centering
{
        \includegraphics[width=120mm]{Figures/Figs_04_00_DetectorSubSystems/Figs_04_02_MainDriftChamber/fig_ele_1.png}
}
\vspace{0cm}
\caption{Block diagram of the MDC electronics.}
    \label{fig:4.2.09}
\end{figure*}
%%%%%%%%%%%%%%%%%%%%%%%%%%%%%%%%%%%%%%%%%%%%%%%%%%


The FEE are responsible for analog signal manipulation, A/D conversion, and time \& charge measurement.
The circuit structure of the FEE is shown in Fig.~\ref{fig:4.2.10}.
The signals from the MDC detector are first amplified by fast trans-impedance amplifiers (TIAs) to ensure a fast response for time measurement and to provide charge measurement. The output signal from the TIA is then split into two paths: one connects to the charge measurement circuits and the other connects to the time measurement circuits.
%\quad\\
For the time measurement, considering the characteristic shape of the signal, the threshold of the discriminator is set to a low value. To achieve a high time precision, an amplifier located before the discriminator is used to enhance the signal slew rate. To filter out the situations when the noise crosses this low threshold, another high-threshold discriminator is used (after shaping), as shown in Fig.~\ref{fig:4.2.10}. Then, the time-to-digital converter ~(TDC) is used to digitize the time of the leading edge of the low-threshold discriminator.
%\quad\\
The charge measurement circuitry uses a shaping circuit to enhance the SNR, the output waveforms of which are digitized and sent to an FPGA chip for charge calculation. As mentioned above, the output signal of the shaper is fed to a discriminator with a high threshold. The output from this discriminator is used as the flag signal to start the charge calculation process and as a ``valid'' condition for time measurement from the output from the low-threshold discriminator. To suppress the effect of pileup, baseline restoration and digital processing on the signal waveform after digitization will be applied in the electronics design.


%%%%%%%%%%%%%%%%%%% Fig %%%%%%%%%%%%%%%%%%%%%%%%%%
\begin{figure*}[htb]
    \centering
{
        \includegraphics[width=120mm]{Figures/Figs_04_00_DetectorSubSystems/Figs_04_02_MainDriftChamber/fig_ele_2.png}
}
\vspace{0cm}
\caption{Time and charge measurement circuit for the MDC.}
    \label{fig:4.2.10}
\end{figure*}
%%%%%%%%%%%%%%%%%%%%%%%%%%%%%%%%%%%%%%%%%%%%%%%%%%


Outputs from the FEE are collected by the RUs, which further assemble these data and transfer them to the DAQ through optical fibers. The MDC readout electronics are required to be synchronized with a system clock signal, which is received by the RUs and then fanned out to the FEE. The RUs also receive the global trigger signal and fan it to the FEE to perform trigger matching to read out valid data.


\subsection{Conclusion}
In summary, an MWDC-based MDC is the baseline for the conceptual design of the main tracking system for the STCF, ensuring the robustness and stability of the whole tracker system.
The MDC consists of 48 layers of drift cells with an inner radius of 200~mm and an outer radius of 850~mm.
To improve its performance, several detector design parameters, including the working gas component, wire parameters, cell structure and layer layout, are optimized via simulation.
The study indicates that the baseline MDC can satisfy the physics requirements of the STCF, with a transverse momentum resolution of $\sigma_{p_T}/p_{T} < 0.5$\%@1~GeV/c. According to the experience of the BESIII experiment, a $dE/dx$ resolution of $\sim$ 6\% can be achieved for this MDC design.

