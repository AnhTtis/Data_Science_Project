\section{ElectroMagnetic Calorimeter (EMC)}
\label{sec:emc}

\subsection{Introduction}
\label{sec:emcintro}


The electromagnetic calorimeter (EMC) of the STCF detector is a cylindrical array of scintillating crystals that provides energy and position measurements for photons with high resolution and 4$\pi$ coverage. It can also identify particles including photons, electrons and hadrons. The primary detector requirements leading to the conceptual design of the EMC are listed as follows:

\begin{itemize}
\item Energy resolution of approximately 2.5\% for 1 GeV photons and good energy linearity from 25 MeV to 3.5 GeV.
\item Position resolution of about 5~mm for 1 GeV photons.
\item Fast response to cope with the expected high rate environment.
\item Time resolution of about 300 ps for 1 GeV energy deposits to suppress background and identify particles.
\item Good radiation resistance against the radiation dose anticipated of 10 years of operation.
\item Providing critical input to the trigger system.
\item Precise luminosity measurement. 
\end{itemize}



%%%%%%%%%%%%%%%%%%%%%%%%%%%%%%%%%%%%%%%%%%%%%%%%%%%%%%%%%%%%%%%%%%%%%%

\subsection{EMC Conceptual Design}

\subsubsection{Crystal and Photo-Detector Choices}

%\quad\\
With the advantage of high light-yield, CsI(Tl) crystal based calorimeters have been widely used in collider experiments in the $\tau$-charm energy region, for example, in BESIII~\cite{bes}, Belle~\cite{belle} and Belle II~\cite{belle2}. However, the decay time of this crystal is so long that it cannot meet the requirements of a high luminosity experiment. In the STCF, it is necessary to select a faster crystal for the EMC. Pure CsI (pCsI) crystals, with a short decay time and excellent radiation resistance, are promising candidates. The light yield of pCsI crystal is approximately 2\% of that from CsI(Tl)~\cite{1} and would further decrease by 20\% when the received irradiation dose reaches 100~krad (the total irradiation dose of the STCF EMC endcap in 10 years would be approximately 45 krad (see Table~\ref{tab:TIDNIEL_max})).

For the readout electronics of a crystal scintillator, semiconductor photodetectors are favored to operate in a strong magnetic field of 1~T. Mature commercial semiconductor photodetectors mainly include photodiodes~(PDs), avalanch photodiodes~(APDs) and silicon photomultiplifiers~(SiPMs). Considering the gain performance, large dynamic response range, linear output signal amplitude, sensitive area and light collection efficiency of APDs, the photodetectors are more suitable as light collection devices of pCsI crystals.

A pCsI crystal scintillator with an APD readout is chosen as a possiable candidate  satisfying the physics requirements of the STCF EMC.Two of the most important factors regarding the performance of the EMC are the energy resolution and position resolution. The relevant considerations for the conceptual design of the EMC are discussed below.


\iffalse
\newcommand{\tabincell}[2]
{\begin{tabular}{@{}#1@{}}#2\end{tabular}}
\begin{tiny}
\begin{table}
\begin{center}
\caption{The Parameters of Crystals.}\label{Tab:List-1st}
\begin{tabular}{|c|c|c|c|c|c|c|}
\hline Crystal&pCsI&LYSO&PWO&YAP&GSO&BaF:Y\\
\hline Density ($g/cm^{3}$)&4.51&7.40&8.30&5.37&6.71&4.89\\
\hline Melting Point (${}^{o}C$)&621&2050&1123&1872&1950&1280\\
\hline Radiation Length (cm)&1.86&1.14&0.89&2.70&1.38&2.03\\
\hline Moliere Radius (cm)&3.57&2.07&2.00&4.50&2.23&3.10\\
\hline Reflective index&1.95&1.82&2.20&1.95&1.85&1.50\\
\hline Hygroscopicity&Slight&No&No&No&No&No\\
\hline Luminosity (nm)&310&402&\tabincell{c}{425\\420}&370&430&\tabincell{c}{300\\220}\\
\hline Decay Time (ns)&\tabincell{c}{30\\6}&40&\tabincell{c}{30\\10}&30&60&\tabincell{c}{600\\1.2}\\
\hline Light Yield(\%)&\tabincell{c}{3.6\\1.1}&85&\tabincell{c}{0.3\\0.1}&65&20&\tabincell{c}{1.7\\4.8}\\
\hline Dose Rate Dependent&No&No&Yes&TBA&TBA&No\\
\hline D(LY)/dT ($\%/{}^{o}C$)&-1.4&-0.2&-2.5&TBA&-0.4&TBA\\
\hline Experiment&\tabincell{c}{kTeV\\Mu2e}&-&\tabincell{c}{CMS\\ALICE\\PANDA}&-&-&-\\
\hline
\end{tabular}
\end{center}
\end{table}
\end{tiny}
\fi
%--------------------------------------------------------%
%\quad\\
The energy resolution of a calorimeter is affected by the shower leakage, which is mainly determined by the total radiation length ($X_{0}$) of the EMC. Figure~\ref{Fig4:ECAL-TotalRadiationLength} shows the longitudinal development of the shower of a 3.5\,GeV photon (close to the most energetic photon in the STCF) in a pure CsI crystal. Approximately 95\% of the shower energy is deposited within 15 $X_{0}$ (28\,cm). Then, with increasing total radiation length, the total energy deposition increases slowly, and the improvement in the energy resolution is very small. Considering the total crystal cost and performance of the EMC, we choose the crystal radiation length of 15 $X_{0}$.


\begin{figure*}[htb]
 \centering
 \mbox{
  %\vskip -1.5cm
  \begin{overpic}[width=0.5\textwidth, height=0.33\textwidth]{Figures/Figs_04_00_DetectorSubSystems/Figs_04_04_ElectromagneticCalorimeter/ECAL-TotalRL.pdf}
  \end{overpic}
 }
\caption{The shower longitudinal distribution.}
\label{Fig4:ECAL-TotalRadiationLength}
\end{figure*}
%%%%%%%%%%%%%%%%%%%%%%%%%%%%%%%%%%%%%%%%%%%%%%%%%%%%%%%%%%%%%%%%%%%%%%%%%%%%%%
%\quad\\
The position resolution (angular resolution) is affected by the shower transverse development and the calorimeter segmentation and is an important parameter for the reconstruction of $\pi^{0}$ particles. According to the simulation, the minimum angle between two photons from the decay of a 1.5 GeV $\pi^{0}$ (about the most energetic $\pi^{0}$ in the STCF experiment) is about 10 degrees. To separate these two photons, the maximum coverage of each crystal should be $\sim$ 3 degrees.  In principle, finer dimensions improve the angular resolution. However, the packaging and supporting materials of the crystal increase, and the lateral leakage increases, which worsens the energy resolution. Optimization of the crystal size is performed using simulation and is discussed in Sec.~\ref{sec:emc_opt}.


\iffalse
\begin{tiny}
\begin{table}
\begin{center}
\caption{The photodetector parameters.}\label{Tab:List-2nd}
\begin{tabular}{|c|c|c|c|}
\hline Device&Typical Gain&Typical QE or PDE (\%)&Typical Size ($cm^2$)\\
\hline Photodiode&1&80&1\\
\hline Avalanche Photodiode&50&80&0.25\\
\hline SiPM&$10^{5}$&25&0.36\\
\hline
\end{tabular}
\end{center}
\end{table}
\end{tiny}
\fi
%%%%%%%%%%%%%%%%%%%%%%%%%%%%%%%%%%%%%%%%%%%%%%%%%%%%%%%%%%%%%%%%%%%%%%%%%%%%%%%%%%%
\subsubsection{Crystal Size Optimization}
\label{sec:emc_opt}

Figure~\ref{Fig4:ECAL-CrystalCellSize} shows the schematic arrangement of crystals in the $X-Y$ plane of the barrel EMC. The inner radius R of the calorimeter is 105 cm, and the length of the crystal is 28 cm. The size of the cyrstal, denoted by the dimension of the front face where particles enter, is a key design parameter to be optimized.

\begin{figure*}[htb]
 \centering
 \mbox{
  \begin{overpic}[width=0.5\textwidth, height=0.33\textwidth]{Figures/Figs_04_00_DetectorSubSystems/Figs_04_04_ElectromagneticCalorimeter/ECAL-CellSize.pdf}
  \end{overpic}
 }
\caption{The schematic arrangement of crystals.}
\label{Fig4:ECAL-CrystalCellSize}
\end{figure*}

\iffalse
\subparagraph{Enenrgy Response and Material Effects}
The response of the calorimeter is studied with Geant4 simulation using 1 GeV photons. As shown in Fig.\ref{Fig4:ECAL EneRes-1stfactor}(a), the intrinsic energy resolution is about 1.52\% by only considering the interaction between photons and the crystal. Here, the energy reconstruction is in the range of 5 $\times$ 5 array and each crystal has a cross-section of $5\times 5$~cm$^2$ at the front face. The energy deposition fluctuation is mostly caused by backscattering and leakage of the shower tail. The energy resolution is simulated by taking into account of the several main factors. The influence of these factors on the energy resolution is given in Table \ref{Tab:List-3rd}. A 200~$\mu$m thick carbon fiber material is introduced as the support unit, and the energy resolution becomes 1.96\%. When the crystal non-uniformity of fluorescence collection is 5\%, the energy resolution is 2.06\%. Considering that the secondary particles generated in the shower process will hit the APD and produce electron-hole pairs, which are superimposed on the signals, the energy resolution is 2.11\%. When adding electronic noise of 1 MeV, the final energy resolution is 2.15\%, as shown in Fig.\ref{Fig4:ECAL EneRes-1stfactor}(b). Here the light yield is set to be 100 pe/MeV, and its influence at 1 GeV is less than 1\%, which can be ignored.

 \begin{figure*}[htbp]
 \centering
  \subfloat[][]{\includegraphics[width=0.4\textwidth]{Figures/Figs_04_00_DetectorSubSystems/Figs_04_04_ElectromagneticCalorimeter/ECAL-EneRes-Intrinsic.pdf}}
  \subfloat[][]{\includegraphics[width=0.4\textwidth]{Figures/Figs_04_00_DetectorSubSystems/Figs_04_04_ElectromagneticCalorimeter/ECAL-EneRes-Noise-1MeV.pdf}}
\caption{The Energy Resolution of EMC. Intrinsic (a), with several main factors (b).}
\label{Fig4:ECAL EneRes-1stfactor}
\end{figure*}

\begin{tiny}
\begin{table}[htb]
\begin{center}
\caption{The Energy Resolution consider different effects.}\label{Tab:List-3rd}
\begin{tabular}{|c|c|c|c|c|c|}
\hline Condition&Intri&Carbon Fiber (200 um)&Uni (5\%)&APD&Noise (1 MeV)\\
\hline EneRes @1GeV (\%)&1.52&1.96&2.06&2.11&2.15\\
\hline
\end{tabular}
\end{center}
\end{table}
\end{tiny}
\fi
\subparagraph{Energy Resolution}
The influence of the front face size of the crystal on the energy resolution of the EMC is also investigated. In principle, the angular resolution of the EMC can be improved by reducing the crystal front face size, but more packaging and supporting materials would be introduced, which would degrade the energy resolution of the EMC. Figure~\ref{Fig4:ECAL AngRes-1stfactor} shows the energy resolution of 1~GeV gamma ray reconstruction, comparing the energy deposition in two layouts: a 3 $\times$ 3 crystal array with a 5~cm crystal front face size and a $5\times5$ array with a 3~cm crystal front face size. The reconstructed energies are 0.911~GeV and 0.905~GeV, respectively, where a smaller crystal gives a lower reconstructed energy, but the difference is not significant.
To obtain the energy resolution, a Crystal Ball function is fit to the reconstructed energy distribution, and the FWHM divided by 2.35 is used as the energy resolution.
For the two layouts, the energy resolutions are found to be 2.46\% and 2.56\%, respectively, and the smaller crystal results in a larger energy uncertainty. The total effective areas of the EMC used in both cases are the same. However, more packaging materials with a smaller crystal size are needed. The packaging materials here are 300 $\mu$m Teflon film and 200 $\mu$m carbon fiber as crystal support materials.
\begin{figure*}[htbp]
 \centering
  \subfloat[][]{\includegraphics[width=0.4\textwidth]{Figures/Figs_04_00_DetectorSubSystems/Figs_04_04_ElectromagneticCalorimeter/ECAL-Granularity-5cm-EneRes.pdf}}
  \subfloat[][]{\includegraphics[width=0.4\textwidth]{Figures/Figs_04_00_DetectorSubSystems/Figs_04_04_ElectromagneticCalorimeter/ECAL-Granularity-3cm-EneRes.pdf}}
\caption{The energy resolution of the EMC. The pCsI front face size is 5~cm  (a) and 3~cm (b).}
\label{Fig4:ECAL AngRes-1stfactor}
\end{figure*}

%%%

\subparagraph{Position Resolution and $\pi^{0}$ Reconstruction Efficiency}
\quad
The position resolution of incident photons is directly dependent on the crystal front face size. The shower position measurement is the key to reconstrcuting the incident angle, where the barycenter method is commonly used, with the following formula:
\begin{equation}
\begin{array}{cll}
x_{c} = \sum\limits_{j}^{N} w_{j}(E_{j}) x_{j}/\sum\limits_{j}^{N} w_{j}(E_{j}),
\end{array}
\end{equation}
\noindent
where $x_{c}$ is the reconstruction position, $x_{j}$ is the spatial coordinate of the $j-th$ crystal, and $w_{j}$ ($E_{j}$) is the weight of the $j-th$ crystal involved in the position reconstruction. The weight is related to the energy deposition in that crystal.
The angle can be determined by reconstructing the incident point on the crystal front surface and the collision point. Figure~\ref{Fig4:ECAL PosRes-2-2ndfactor} compares the position resolution of 1~GeV $\gamma$ reconstructed by the logarithmic weight of energy in crystals with two different front face sizes of (a) 5~cm $\times$ 5~cm and (b) 3~cm $\times$ 3~cm. The position resolutions are 4.8~mm and 3.8~mm, respectively, clearly showing that the position or angular resolution can be improved by decreasing the crystal size.\\

\begin{figure*}[htbp]
 \centering
  \subfloat[][]{\includegraphics[width=0.4\textwidth]{Figures/Figs_04_00_DetectorSubSystems/Figs_04_04_ElectromagneticCalorimeter/9_2_4_a.pdf}}
  \subfloat[][]{\includegraphics[width=0.4\textwidth]{Figures/Figs_04_00_DetectorSubSystems/Figs_04_04_ElectromagneticCalorimeter/9_2_4_b.pdf}}
\caption{The EMC position resolution based on the logarithmic energy weighting method, with (a) 5 cm $\times$ 5 cm and (b) 3 cm $\times$ 3 cm crystal front face sizes.}
\label{Fig4:ECAL PosRes-2-2ndfactor}
\end{figure*}

With decreasing crystal front face size, the proportion of packaging materials between crystals increases, which leads to degraded reconstruction efficiency for $\pi^{0}$. The reconstruction efficiencies of $\pi^{0}$ with different crystal front face sizes are compared, as shown in Fig.~\ref{Fig4:ECAL AngRes-2-5thfactor}. Here, it is required that the two photons generated by $\pi^{0}$ decay are both within the effective acceptance of EMC, and to reconstruct the $\pi^{0}$ mass.
It is clear that the reconstruction efficiency of $\pi^{0}$ is higher for larger crystal front face size. The average opening angle of the crystal with the front  face size of 5 cm relative to the collision point is about 2 degrees, which can distinguish the two photons generated from $\pi^{0}$ ($\pi^{0}$ of 1.5~GeV, where the minimum included angle of photons is about 10 degrees). Moreover, the energy resolution of the 5 cm crystal is better than that of the small-sized crystal, so the reconstruction efficiency of $\pi^{0}$ in the same mass width range is slightly higher.

\subparagraph{Fake Photon Discrimination}
In contrast with the normal photons from the collision point, most of the fake photons are generated at other positions, and their direction of incidence on the EMC deviates greatly from the axial direction of the crystal, which causes the hit number and secondary moment of the shower in the EMC to be different from that of real photons. Another important difference is that the time information is different from the real signal. This information can be used to identify fake photons. In addition, whether the direction reconstruction of photons can be realized by longitudinal sampling of the EMC is still an interesting topic, but considering the importance of energy resolution, the scheme of longitudinal sampling needs to be carefully designed.

In conclusion, considering the energy resolution, position resolution, reconstruction efficiency and total number of readout channels, based on the premise of meeting the STCF EMC requirements, the end face size of $\sim$5~cm is selected as the typical size of the crystal.



\begin{figure*}[htbp]
 \centering
 \mbox{
  \begin{overpic}[width=0.5\textwidth, height=0.33\textwidth]{Figures/Figs_04_00_DetectorSubSystems/Figs_04_04_ElectromagneticCalorimeter/ECAL-Pi0Efficiency.pdf}
  \end{overpic}
 }
\caption{The efficiency of the EMC for $\pi^{0}$.}
\label{Fig4:ECAL AngRes-2-5thfactor}
\end{figure*}

%%%%%%%%%%%%%%%%%%%%%%%%%%%%%%%%%%%%%%%%%%%%%%%%%%%%%%%%%%%%%%
%%%%%%%%%%%%%%%%%%%%%%%%%%%%%%%%%%%%%%%%%%%%%%%%%%%%%%%%%%%%%%%%%%%%%%%%%%%%%%%%%
\subsubsection{EMC Layout}
%{The Layout Design}
The calorimeter is composed of a barrel part and an endcap cover, and the crystal arrangement diagram is shown in Fig.~\ref{Fig4:ECAL Layout}. The barrel part covers polar angles from $33.85^{\circ}$ to $146.15^{\circ}$, with an inner radius of 105~cm and length of 320~cm along the beam direction. The endcap part is located at 160~cm $\le \left|z\right| \le$ 190~cm, with an inner radius of 56~cm and an outer radius of 105~cm. The endcap covers polar angles from 19.18$^{\circ}$ to 33.64$^{\circ}$ and 156.15$^{\circ}$ to 160.82$^{\circ}$. The entire EMC provides a 94.45\% solid angle coverage of 4 $\pi$.

There are 51 circles of crystals in the barrel along the beam (Z) direction, 132 crystals per circle, and 6732 crystals in total.
The crystal size and shape in the same circle are the same, and the crystal shapes in different circles are similar but different. The crystal is in an irregular trapezoid platform shape, in which the sizes of the front and back faces are about $5\times5$ and $6.5\times6.5$~cm$^2$ respectively, and the longitudinal length is 28~cm. The endcap is divided into two identical parts, located to the left and right side of the barrel part, each with 969 crystals, with a total of 1938 crystals. The crystal in the endcap is also in an irregular trapezoid platform shape. The sizes of the front and back faces are about $4.5\times4.5$~cm$^2$ and $5.2\times5.2$~cm$^2$ respectively, and the longitudinal length is also 28~cm.

To reduce the probability of secondary particles escaping from the gap between crystals, a defocus design is added to the geometric structure of the EMC to improve the detection efficiency. In the direction of $\theta$, except for the middle circle of crystals in the barrel, each circle of crystals points to $\pm2.5$~cm away from the collision point, and each circle of crystals in the endcap points to $\pm10$~cm away from the collision point, which can also be seen in Fig.~\ref{Fig4:ECAL Layout}. In the azimuth direction ($\phi$), the crystal in each circle deflects 1.36$^{\circ}$ in the direction of phi. In this way, the crystal points to the circumference with a radius of 2.5~cm centered on the beam line, as shown Fig.~\ref{Fig4:ECAL Defocus-phi}.
\begin{figure*}[htbp]
 \centering
 \mbox{
  %\vskip -1.5cm
  \begin{overpic}[width=0.8\textwidth]{Figures/Figs_04_00_DetectorSubSystems/Figs_04_04_ElectromagneticCalorimeter/ECAL-Layout.pdf}
  \end{overpic}
 }
\caption{The EMC layout design.}
\label{Fig4:ECAL Layout}
\end{figure*}

\begin{figure*}[htbp]
 \centering
 \mbox{
  %\vskip -1.5cm
  \begin{overpic}[width=0.5\textwidth]{Figures/Figs_04_00_DetectorSubSystems/Figs_04_04_ElectromagneticCalorimeter/ECAL-Defocus-phi.pdf}
  \end{overpic}
 }
\caption{The EMC defocus design.}
\label{Fig4:ECAL Defocus-phi}
\end{figure*}
%%%%%%%%%%%%%%%%%%%%%%%%%%%%%%%%%%%%%%%%%%%%%%%%%%%%%%%%%%%%%%%%%%%%%%


%%%%%%%%%%%%%%%%%%%%%%%%%%%%%%%%%%
\iffalse
\subparagraph{Intrinsic Resolution}
\quad\\
Taking 1 GeV $\gamma$ rays as an example, the response of the calorimeter is studied. The energy deposition is shown in Fig.~\ref{Fig4:ECAL EneRes-1stfactor},reconstructed by 5 $\times$ 5 crystal array. Figure~\ref{Fig4:ECAL EneRes-1stfactor}(a) depicts the ideal energy deposition of $\gamma$ rays in pure crystal ECAL, and Fig.~\ref{Fig4:ECAL EneRes-1stfactor}(b) further introduces the reflection and structural support materials of crystal (Teflon : 240 $\mu m$ thick, Mylar : 150 $\mu m$ thick, carbon fiber support : 200 $\mu m$ thick) into account. The results show that after the introduction of packaging materials, the energy resolution increased from 1.52\% to 1.92\%, and the peak energy decreased from 0.97 GeV to 0.96 GeV. This is because part of the energy deposition cannot be measured in the ``dead material''.\\
\begin{figure*}[htbp]
 \centering
  \subfloat[][]{\includegraphics[width=0.4\textwidth]{Figures/Figs_04_00_DetectorSubSystems/Figs_04_04_ElectromagneticCalorimeter/ECAL-EneRes-Intrinsic.pdf}}
  \subfloat[][]{\includegraphics[width=0.4\textwidth]{Figures/Figs_04_00_DetectorSubSystems/Figs_04_04_ElectromagneticCalorimeter/ECAL-EneRes-ReflectorFilm.pdf}}
\caption{The Energy Resolution of EMC. Intrinsic (a), with reflective materials (b).}
\label{Fig4:ECAL EneRes-1stfactor}
\end{figure*}
\fi
%%%%%%%%%%%%%%%%%%%%%%%%%%%%%%%%%%%%%%%%%%%%%%%%%%%%%%%%%%%%%%%%%%%%%%%
\iffalse
\subparagraph{Light Yield}
\quad\\
Because the light yield of pCsI crystal is only about 1/50 of that of CsI(Tl) crystal, the influence of the number of photoelectrons (p.e.) collected by APD on the energy resolution has to be considered carefully. Figure~\ref{Fig4:ECAL EneRes-2ndfactor} shows the energy resolution obtained by setting the fluorescence yield of pCsI crystal to 20 p.e./MeV, 50 p.e./MeV and 100 p.e./MeV respectively, on the basis of Fig.~\ref{Fig4:ECAL EneRes-1stfactor}(b). It's found that for 1 GeV $\gamma$ rays, the energy resolution under three settings are 2.15\%, 2.02\% and 1.96\% respectively. Compared with the result without considering the photoelectron fluctuation (1.91\%, Fig.~\ref{Fig4:ECAL EneRes-1stfactor}(b)), the effect introduced by the fluorescence yield of 100 p.e./MeV is reasonably small.\\
\begin{figure*}[htbp]
 \centering
  %\vskip -1.5cm
  \subfloat[][]{\includegraphics[width=0.3\textwidth]{Figures/Figs_04_00_DetectorSubSystems/Figs_04_04_ElectromagneticCalorimeter/ECAL-EneRes-LY-20pe.pdf}}
  \subfloat[][]{\includegraphics[width=0.3\textwidth]{Figures/Figs_04_00_DetectorSubSystems/Figs_04_04_ElectromagneticCalorimeter/ECAL-EneRes-LY-50pe.pdf}}
  \subfloat[][]{\includegraphics[width=0.3\textwidth]{Figures/Figs_04_00_DetectorSubSystems/Figs_04_04_ElectromagneticCalorimeter/ECAL-EneRes-LY-100pe.pdf}}
\caption{The Energy Resolution of EMC. Light yield is 20 p.e./MeV (a), Light yield is 50 p.e./MeV (b), Light yield is 100 p.e./MeV (c).}
\label{Fig4:ECAL EneRes-2ndfactor}
\end{figure*}
%%%%%%%%%%%%%%%%%%%%%%%%%%%%%%%%%%%%%%%%%%%%%%%%%%%%%%%%%%%%%%%%%%%%%
\subparagraph{Light Collection Uniformity}
\quad\\
The non-uniformity of light collection along the longitudinal length (28 cm) of the crystal arises from its self-absorption effect in the transmission of fluorescence, the wedge-shaped structure, and the inconsistency in the growth process of pCsI. This will lead to nonlinear energy reconstruction. The reconstruction energy resolution is compared under different non-uniformity parameters. The results are as shown in the Fig.~\ref{Fig4:ECAL EneRes-3rdfactor}. With the increase of non-uniformity, the energy resolution gradually becomes worse. Under the conditions of complete uniformity, non-uniformity of 5\% and 10\%, the energy resolution of 1 GeV is 2.06\% and 2.28\%. Because the energy resolution is required to be better than 2.5\%, the non-uniformity of fluorescence collection should be better than 10\%. In the following simulation study, the non-uniformity parameter is set to 5\%.
\begin{figure*}[htbp]
 \centering
 \mbox{
  %\vskip -1.5cm
  \begin{overpic}[width=0.6\textwidth]{Figures/Figs_04_00_DetectorSubSystems/Figs_04_04_ElectromagneticCalorimeter/ECAL-EneRes-pCsIUniformity.pdf}
  \end{overpic}
 }
\caption{The Relationship Between Non-uniformity and Energy Resolution of EMC.}
\label{Fig4:ECAL EneRes-3rdfactor}
\end{figure*}
%%%%%%%%%%%%%%%%%%%%%%%%%%%%%%%%%%%%%%%%%%%%%%%%%%%%%%%%%%%%%%%%%%%%%%
\subparagraph{Secondary Particles Incident on APD}
\quad\\
APD, as a photoelectric device, is used to collect fluorescent photons emitted by crystals and convert them into photoelectrons. Photoelectrons cause avalanches to produce a large number of electron-hole pairs, which are finally input into readout electronics. However, high-energy photons shower in EMC, producing a large number of secondary photons and e$^\pm$. These secondary particles may directly hit the APD and may also causes the avalanche signal to be collected, which will be added on the fluorescent signal produced by the crystal. This means that the measured signal amplitude may be larger than its true value. Figure~\ref{Fig4:ECAL EneRes-4thfactor} shows the energy spectrum of 1 GeV $\gamma$ rays simulated considering this effect. The energy resolution is 2.11\%, a little worse than the value 2.06\% without considering this effect.\\
\begin{figure*}[htbp]
 \centering
 \mbox{
  %\vskip -1.5cm
  \begin{overpic}[width=0.6\textwidth]{Figures/Figs_04_00_DetectorSubSystems/Figs_04_04_ElectromagneticCalorimeter/ECAL-EneRes-APD.pdf}
  \end{overpic}
 }
\caption{The EMC Energy Resolution after introducing APD.}
\label{Fig4:ECAL EneRes-4thfactor}
\end{figure*}
%%%%%%%%%%%%%%%%%%%%%%%%%%%%%%%%%%%%%%%%%%%%%%%%%%%%%%%%%%%%%%%%%%%%%%%%%%%%%%%%%
\subparagraph{Electronics Noise}
\quad\\
The noise of readout electronics will be added on the signal amplitude, which will affect the resolution of the signal. In order to ensure the energy resolution of the EMC, it is required that the readout electronic noise of the calorimeter is less than 1 MeV. Figure~\ref{Fig4:ECAL EneRes-5thfactor} shows the energy spectrum of 1 GeV $\gamma$ rays obtained by introducing 1 MeV electronic noise. The value is 2.15\%.
\begin{figure*}[htbp]
 \centering
 \mbox{
  %\vskip -1.5cm
  \begin{overpic}[width=0.6\textwidth]{Figures/Figs_04_00_DetectorSubSystems/Figs_04_04_ElectromagneticCalorimeter/ECAL-EneRes-Noise-1MeV.pdf}
  \end{overpic}
 }
\caption{The EMC Energy Resolution after introducing Electronics Noise.}
\label{Fig4:ECAL EneRes-5thfactor}
\end{figure*}
\fi
%%%%%%%%%%%%%%%%%%%%%%%%%%%%%%%%%%%%%%%%%%%%%%%%%%%%%%%%%%%%%%%%%%%%%%%%%%%%%%%%%
%%%%%%%%%%%%%%%%%%%%%%%%%%%%%%%%%%%%%%%%%%%%%%%%%%%%%%%%%%%%%%%%%%%%%%%%%%%%%%

%%%%%%%%%%%%%%%%%%%%%%%%%%%%%%%%%%%%%%%%%%%%%%%%%%%%%%%%%%%%%%%%%%%%%%%%%%%%%%%%%%%%%%%%%%%%%
\iffalse
\subsubsection{Angular Resolution}
\quad\\
The angular resolution is mainly related to the crystal front face size and its distance to the collision point. It can be described as follows:
\begin{equation}
\begin{array}{cll}
\sigma_{\theta} = \frac{c}{\sqrt{E}} \oplus d
\end{array}
\end{equation}
The parameters $c$ and $d$ are used to describe the statistical fluctuation term and systematic error term, respectively. The angular resolution is studied by comparison between crystals with 3 cm $\times$ 3 cm and 5 cm $\times$ 5 cm front face size, and mainly consider the following aspects:
\begin{itemize}
\item Energy Response
\item Position Resolution
\item Gamma Efficiency from $\pi^{0}$
\item Total Readout Channels
\end{itemize}
\fi
%%%%%%%%%%%%%%%%%%%%%%%
\iffalse
Figure~\ref{Fig4:ECAL AngRes-2-2ndfactor} compares the angular resolution of 1 GeV $\gamma$ reconstructed by the logarithm weight of energy in crystals with two different front face sizes of (a) 5 cm $\times$ 5 cm and (b) 3 cm $\times$ 3 cm. The angular resolutions are 0.24 degree and 0.19 degree clearly showing showing that the angular resolution can be improved by decreasing the crystal size.\\
It's noticed that when directly using energy as the weight, we can select an appropriate function to fit the $S-curve$ in Fig.~\ref{Fig4:ECAL AngRes-2-3rdfactor}(a). By using the fitting curve for calibration, we can also get a relatively good position resolution. As shown in Fig.~\ref{Fig4:ECAL AngRes-2-3rdfactor}(b), the angular resolution by using 5 cm crystal size is 0.23 degree, which is consistent with the algorithm weight. Figure~\ref{Fig4:ECAL AngRes-2-4thfactor} shows the angular resolution at different $\gamma$ energies.\\

\begin{figure*}[htbp]
 \centering
 \mbox{
  %\vskip -1.5cm
  \begin{overpic}[width=0.4\textwidth]{Figures/Figs_04_00_DetectorSubSystems/Figs_04_04_ElectromagneticCalorimeter/ECAL-Granularity-AngRes-5cm-lnEi.pdf}
  \end{overpic}
  \begin{overpic}[width=0.4\textwidth]{Figures/Figs_04_00_DetectorSubSystems/Figs_04_04_ElectromagneticCalorimeter/ECAL-Granularity-AngRes-3cm-lnEi.pdf}
  \end{overpic}
  }
\caption{The EMC Angle Resolution Based on logarithm energy weighting method. with (a) 5 cm $\times$ 5 cm , and (b) 3 cm $\times$ 3 cm crystal front face sizes.}
\label{Fig4:ECAL AngRes-2-2ndfactor}
\end{figure*}

\begin{figure*}[htbp]
 \centering
 \mbox{
  %\vskip -1.5cm
  \begin{overpic}[width=0.4\textwidth]{Figures/Figs_04_00_DetectorSubSystems/Figs_04_04_ElectromagneticCalorimeter/ECAL-Granularity-AngRecTruFit-5cm-Ei.pdf}
  \end{overpic}
  \begin{overpic}[width=0.4\textwidth]{Figures/Figs_04_00_DetectorSubSystems/Figs_04_04_ElectromagneticCalorimeter/ECAL-Granularity-AngRes-5cm-Ei.pdf}
  \end{overpic}
  }
\caption{(a) A fit to the correlation between the reconstructed angle and the real incident angle with a linear energy weight, and (b) the angle resolution of EMC based on linear energy weight, with 5cm*5cm crystal front face size.}
\label{Fig4:ECAL AngRes-2-3rdfactor}
\end{figure*}

\begin{figure*}[htbp]
 \centering
 \mbox{
  %\vskip -1.5cm
  \begin{overpic}[width=0.6\textwidth, height=0.45\textwidth]{Figures/Figs_04_00_DetectorSubSystems/Figs_04_04_ElectromagneticCalorimeter/ECAL-AngResCurve.pdf}
  \end{overpic}
 }
\caption{The Angular Resolution Curve of EMC.}
\label{Fig4:ECAL AngRes-2-4thfactor}
\end{figure*}
\fi
%%%%%%%%%%%%%%%%%%%%%%%%%%%%%%%%%%%%%%%%%%%%%%%%%%%%%%%%%%%%%%%%%%%%%%%%%%%%%%%%%
%%%%%%%%%%%%%%%%%%%%%%%%%%%%%%%%%%%%%%%%%%%%%%%%%%%%%%%%%%%%%%%%%%%%%%%%%%%%%%%%%
\subsection{Expected Performance of the EMC}
\label{sec:emc_perf}
\subsubsection{Energy Response}
Based on the conceptual design, the response of the EMC to photons with different energies is studied via {\sc Geant4} simulation using 1~GeV photons.
As shown in Fig.~\ref{Fig4:ECAL EneRes-1stfactor}(a), the intrinsic energy resolution is about 1.52\% with only considering the interaction between photons and the crystal. Here, the energy reconstruction is in the range of the $5\times5$ array and each crystal has a cross-section of $5\times 5$~cm$^2$ at the front face. The energy deposition fluctuation is mostly caused by backscattering and leakage of the shower tail. The energy resolution is simulated by taking into account several main factors. The influence of these factors on the energy resolution is given in Table~\ref{Tab:List-3rd}. When the light yield of the crystal is set to 100~pe/MeV (preliminary measurement shows that the light yield of the pCsI crystal can reach about 150 pe/MeV, see Sec.~\ref{sec:emc_cosmic}), the simulated energy resolution is 1.52\%. When a 200~$\mu$m thick carbon fiber material is introduced as the support unit, the energy resolution becomes 1.96\%. The light collection nonuniformity of large crystals can generally reach a few percent. For example, in the BESIII experiment, the average nonuniformity is about 3\%-4\%. After taking into account a 5\% nonuniformity, the simulated energy resolution increases to 2.06\%. Considering that the secondary particles generated in the shower process hit the APD and produce electron-hole pairs, which are superimposed on the signals, the energy resolution is 2.11\%. When electronic noise of 1 MeV is added, the final energy resolution is 2.15\%, as shown in Fig.~\ref{Fig4:ECAL EneRes-1stfactor}(b).
 \begin{figure*}[htbp]
 \centering
  \subfloat[][]{\includegraphics[width=0.4\textwidth]{Figures/Figs_04_00_DetectorSubSystems/Figs_04_04_ElectromagneticCalorimeter/ECAL-EneRes-Intrinsic.pdf}}
  \subfloat[][]{\includegraphics[width=0.4\textwidth]{Figures/Figs_04_00_DetectorSubSystems/Figs_04_04_ElectromagneticCalorimeter/ECAL-EneRes-Noise-1MeV.pdf}}
\caption{The expected energy resolution of the EMC, (a) the intrinsic performance without considering material effects and (b) with several main factors.}
\label{Fig4:ECAL EneRes-1stfactor}
\end{figure*}

\begin{tiny}
\begin{table}[htb]
\begin{center}
\caption{The energy resolution considering different effects.}\label{Tab:List-3rd}
\begin{tabular}{|c|c|c|c|c|c|}
\hline Condition&Intri&Carbon Fiber (200 um)&Uni (5\%)&APD&Noise (1 MeV)\\
\hline EneRes @1GeV (\%)&1.52&1.96&2.06&2.11&2.15\\
\hline
\end{tabular}
\end{center}
\end{table}
\end{tiny}


Figure~\ref{Fig4:EMC Energy Response Curve}(a) shows the photon energy response from 50 MeV to 3.5 GeV. The results show excellent linearity, with a nonlinearity of about 1\%. Figure~\ref{Fig4:EMC Energy Response Curve}(b) shows the energy resolution curve. The results show that from 50 MeV - 2 GeV, the energy resolution gradually improves with increasing energy, but it begins to deteriorate slightly at 2 GeV because of energy leakage in the EMC back end. At 1 GeV, the energy resolution is better than 2.5 \%, which meets the requirement of the EMC.
\begin{figure*}[htbp]
 \centering
  %\vskip -1.5cm
  \subfloat[][]{\includegraphics[width=0.4\textwidth]{Figures/Figs_04_00_DetectorSubSystems/Figs_04_04_ElectromagneticCalorimeter/ECAL-ELCurve.pdf}}
  \subfloat[][]{\includegraphics[width=0.4\textwidth]{Figures/Figs_04_00_DetectorSubSystems/Figs_04_04_ElectromagneticCalorimeter/ECAL-ERCurve.pdf}}
\caption{The expected performance of (a) the energy linearity, and (b) the energy resolution.}
\label{Fig4:EMC Energy Response Curve}
\end{figure*}

\subsubsection{Position Resolution}
Figure~\ref{Fig4:EMC Position and Angular Reslution Curve}(a) shows the performance of the EMC position resolution. The results show that the position resolution gradually improves with increasing energy. At 1~GeV, the expected positon resolution is about 5~mm, which meets the design requirements. With the position resolution, the angular resolution can be obtained, as shown in Fig.~\ref{Fig4:EMC Position and Angular Reslution Curve}(b). The angular resolution is 4~mrad at 1~GeV.
\begin{figure*}[htbp]
 \centering
  %\vskip -1.5cm
  \subfloat[][]{\includegraphics[width=0.4\textwidth]{Figures/Figs_04_00_DetectorSubSystems/Figs_04_04_ElectromagneticCalorimeter/ECAL-PosRCurve.pdf}}
  \subfloat[][]{\includegraphics[width=0.4\textwidth]{Figures/Figs_04_00_DetectorSubSystems/Figs_04_04_ElectromagneticCalorimeter/ECAL-ARCurve.pdf}}
\caption{The expected (a) position resolution and (b) angular resolution for 1~GeV photons.}
\label{Fig4:EMC Position and Angular Reslution Curve}
\end{figure*}

\subsubsection{Time Resolution}
The full response of the EMC including the optical processes, the response of APD and readout electronics is simulated with {\sc Geant4}. The output waveform from the simulation was processed using the template fitting method (see Sec.~\ref{sec:emc_waveformfitting}) to extract the arrival time of the output signal (corresponding to the hit time of incident particles up to a certain delay). Assuming that the light yield is 100 pe/MeV, the distribution of the hit time on the seed crystal in a shower for 100 MeV photons is shown in Fig.~\ref{Fig4:EMC Time Reslution Curve}(a). The width of this distribution representative of the EMC time resolution is 318.8~ps (no electronic noise is considered here). Figure~\ref{Fig4:EMC Time Reslution Curve}(b) shows the time resolution for different energy deposits. The time resolution improves as the deposited energy increases and can reach 200~ps at 1 GeV. Our preliminary study has demonstrated a light yield beyond 100 pe/MeV for the pCsI unit of the EMC. In view of the enhanced light yield, the time resolution of the EMC could be even better. For muons or charged hadrons that penetrate the EMC without shower, the  deposited energy is about 100 MeV, and hence the EMC time resolution for these particles would be about 300 ps. If either charged or neutral hadrons produced shower in the EMC, the time resolution for these particles may be subject to the uncertainty of the shower starting point.

\begin{figure*}[htbp]
 \centering
  %\vskip -1.5cm
  \subfloat[][]{\includegraphics[width=0.4\textwidth]{Figures/Figs_04_00_DetectorSubSystems/Figs_04_04_ElectromagneticCalorimeter/ECAL-TimeRes.pdf}}
  \subfloat[][]{\includegraphics[width=0.4\textwidth]{Figures/Figs_04_00_DetectorSubSystems/Figs_04_04_ElectromagneticCalorimeter/ECAL-TimeCurve.pdf}}
\caption{(a) The time resoluiton of 100 MeV gamma rays and (b) time resolution curve.}
\label{Fig4:EMC Time Reslution Curve}
\end{figure*}

\subsubsection{Impact of Upstream Materials}
%{Influence of upstream materials}
Before arriving at the EMC, photons have a certain probability of interacting with the beam pipe or other subdetectors in front of the EMC. To study the influence of the upstream materials on the performance of the EMC, materials with different thicknesses are added in front of the EMC, and the resulting EMC performance is compared. Four cases of radiation length for materials in front of the EMC, 23\% $X_{0}$, 27\% $X_{0}$, 31\% $X_{0}$ and 35\% $X_{0}$ are considered. The equivalent mass of aluminum is placed at the RICH detector position, about 10 cm in front of the EMC. The simulated energy resolution is shown in Fig.~\ref{Fig4:EMC-ER-UpStreamMaterial}, and the energy resolution changes from 2.25\% to 2.36\%. The photon detection efficiency curve is shown in Fig.~\ref{Fig4:EMC-Effi-UpStreamMaterial}, and it can be seen that in the low energy region, below 1~GeV, with additional material, the detection efficiency decreases significantly, and the impact is found to be very small in the higher energy region.

\begin{figure*}[htbp]
 \centering
  %\vskip -1.5cm
  \subfloat[][]{\includegraphics[width=0.4\textwidth]{Figures/Figs_04_00_DetectorSubSystems/Figs_04_04_ElectromagneticCalorimeter/ECAL-ERwith23X0-1GeV.pdf}}
  \subfloat[][]{\includegraphics[width=0.4\textwidth]{Figures/Figs_04_00_DetectorSubSystems/Figs_04_04_ElectromagneticCalorimeter/ECAL-ERwith27X0-1GeV.pdf}}

  \subfloat[][]{\includegraphics[width=0.4\textwidth]{Figures/Figs_04_00_DetectorSubSystems/Figs_04_04_ElectromagneticCalorimeter/ECAL-ERwith31X0-1GeV.pdf}}
  \subfloat[][]{\includegraphics[width=0.4\textwidth]{Figures/Figs_04_00_DetectorSubSystems/Figs_04_04_ElectromagneticCalorimeter/ECAL-ERwith35X0-1GeV.pdf}}
\caption{The energy resolution of EMC with different thicknesses of upstream materials. (a) 23\% $X_{0}$, (b) 27\% $X_{0}$, (c) 31\% $X_{0}$ and (d) 35\% $X_{0}$.}
\label{Fig4:EMC-ER-UpStreamMaterial}
\end{figure*}

\begin{figure*}[htbp]
 \centering
 \mbox{
  %\vskip -1.5cm
  \begin{overpic}[width=0.5\textwidth, height=0.33\textwidth]{Figures/Figs_04_00_DetectorSubSystems/Figs_04_04_ElectromagneticCalorimeter/9_3_5.pdf}
  \end{overpic}
 }
\caption{The Reconstruction Efficiency curve of the EMC with upstream materials.}
\label{Fig4:EMC-Effi-UpStreamMaterial}
\end{figure*}


For the baseline design of the STCF detector system, the radiation length of each subdetector in front of the EMC is studied via simulation, as shown in Fig.~\ref{Fig4:STCF-Materials}. In the barrel, the total radiation length is about $0.3 X_0$, while in the endcap, the radiation length can reach $0.8 X_0$, which is mainly contributed by the MDC. To evaluate the effect of upstream materials, a Geant4 simulation study is carried out to preliminarily study the energy resolution of the EMC under the existing structural design. A full STCF detector simulation is performed, and the performance of the EMC is compared to that of an EMC-only simulation. The results are shown in Fig.~\ref{Fig4:EMC-ERCurve-FullSim}. Figure~\ref{Fig4:EMC-ERCurve-FullSim}(a) shows that upstream materials have little effect on the energy resolution of the EMC when considering only the barrel EMC. Figure~\ref{Fig4:EMC-ERCurve-FullSim}(b) shows the energy resolution at different polar angles of incident photon with an energy of 1~GeV. In the endcap region, due to material effects of inner subdetectors, the expected energy resolution decreases from about 2.5\% to about 3.0\%. The significant degradation of the energy resolution around $\theta=10^{\circ}$ and $\theta=60^{\circ}$ is due to the large energy leakage at the transition region of the barrel and the endcap.


\begin{figure*}[htbp]
 \centering
 \mbox{
  %\vskip -1.5cm
  \begin{overpic}[width=0.5\textwidth, height=0.33\textwidth]{Figures/Figs_04_00_DetectorSubSystems/Figs_04_04_ElectromagneticCalorimeter/STCF-Materials.pdf}
  \end{overpic}
 }
\caption{The radiation length of the inner subdetectors in front of the EMC.}
\label{Fig4:STCF-Materials}
\end{figure*}

\begin{figure*}[htbp]
 \centering
  %\vskip -1.5cm
  \subfloat[][]{\includegraphics[width=0.4\textwidth]{Figures/Figs_04_00_DetectorSubSystems/Figs_04_04_ElectromagneticCalorimeter/9_3_7_a.pdf}}
  \subfloat[][]{\includegraphics[width=0.4\textwidth]{Figures/Figs_04_00_DetectorSubSystems/Figs_04_04_ElectromagneticCalorimeter/9_3_7_b.pdf}}
\caption{(a) Comparison of the energy resolution of the EMC (barrel) as a function of the energy with full detector simulation and EMC-only simulation. (b) The energy resolution as a function of the incident polar angle for 1~GeV photons.}
\label{Fig4:EMC-ERCurve-FullSim}
\end{figure*}

\subsection{Pileup Mitigation}

\subsubsection{Challenges of High Background}
In Sec.~\ref{sec:emc_perf}, when studying the expected performance of the EMC, the background contribution is normally not considered in the simulation. The photon reconstruction is based on a simple clustering algorithm used in BESIII. The reconstruction algorithm searches the related crystals in a shower, adds their energies together, and calculates the hit position.

Considering the high background at the STCF, the EMC background is studied by Monte Carlo simulation. The simulated background energy distribution of the EMC is shown in Fig.~\ref{Fig4:ECAL Background-Energy-CR}(a). Figure~\ref{Fig4:ECAL Background-Energy-CR}(b) shows the background counting rate at each position of the calorimeter (with a threshold value of 1~MeV). The background counting rate is close to about 1~MHz. This result is consistent with the simulation data shown in Table~\ref{tab:TIDNIEL_max}, which uses a 0.5~MeV threshold.

Such a high background counting rate has a great impact on the energy measurement of the calorimeter. As shown in Fig.~\ref{Fig4:ECAL EnergyRes Before and After BG}, the energy spectrum of the 1~GeV gamma-ray is reconstructed without and with considering the background. The results show that the energy resolution values are 2.15\% and 5.05\%, respectively. After the background is introduced, the energy resolution is degraded by more than a factor of two.

\begin{figure*}[htbp]
 \centering
  %\vskip -1.5cm
  \subfloat[][]{\includegraphics[width=0.4\textwidth]{Figures/Figs_04_00_DetectorSubSystems/Figs_04_04_ElectromagneticCalorimeter/9_4_1_a.pdf}}
  \subfloat[][]{\includegraphics[width=0.4\textwidth]{Figures/Figs_04_00_DetectorSubSystems/Figs_04_04_ElectromagneticCalorimeter/9_4_1_b.pdf}}
\caption{The background simulation in the EMC: (a) the deposited energy distribution and (b) the background counting rate.}
\label{Fig4:ECAL Background-Energy-CR}
\end{figure*}

\subsubsection{Waveform Fitting}
\label{sec:emc_waveformfitting}
%{Waveform Sampling and Fitting Method}
To correct the uncertainty of the energy measurement caused by a high intensity background, one feasible scheme is to reconstruct the amplitude and time information by using a waveform fitting algorithm; this was used in the electromagnetic calorimeter of the CMS experiment at the LHC~\cite{CMS-CAL}. The CMS template fitting technique, named ``multifit'', was motivated by the reduction of the LHC bunch spacing from 50 to 25~ns and by the higher instantaneous luminosity of Run II, which led to a substantial increase in both the in-time and out-of-time pileup, where the latter refers to the overlapping signals from neighboring bunch crossings. It has been demonstrated by CMS that, with the multifit method, the contribution of out-of-time pileup to the signal reconstruction is found to be negligible, both in data and in simulated samples. The energy resolution and response are improved with respect to those of the Run I method where the amplitude reconstructed as a weighted sum of the ten digitized samples.

For the STCF EMC, considering the design parameters of the readout circuit (see Sec.~\ref{sec:emc_elec}), the total width of the signal after shaping is about 500~ns, and the leading edge is about 50~ns. To sample the leading edge and consider that the total amount of data is controllable, we use a sampling rate of 40 MHz as the baseline design here. In the multiwaveform fitting algorithm, the time window of the waveform fitting is set to 1000~ns (-250~ns - 750~ns, and the signal start time is set to 0~ns). The pulse template, $p_j$, for both the signal and the background is built by convoluting the pCsI fluorescence signal with the CSA impulse response function. The total template, $P$ is shown in Eq.~\ref{eq:multifit}; this template is an overlay of the signal template and the background template but shifted in time, in which the normalizations are free parameters ($A_j$). The total template is then used to fit the waveform readout by the electronics. Referring to the CMS multifit method, $N=40$ templates with a repetition period of 12.5~ns (CMS uses 25~ns, half of which is temporarily taken here) are used to fit our results. Starting at -250~ns, one template is placed every 12.5~ns, and a total of 40 templates are used for a multitemplate fitting.
\begin{equation}
P = \sum_{j=0}^{N} A_j p_j,\label{eq:multifit}
\end{equation}


A set of toy events are generated to test the performance of the multiwaveform fitting.
In Fig.~\ref{Fig4:ECAL EnergyRes with Multifit}(a), the dotted green curve is the superposition of the signal and background spectra of one simulated event. The signal peak is around 120 ns (the integration time is set as 40 ns), and the rest represent the backgrounds. The background event rate is assumed to be $\sim$~MHz, and the energy is obtained by sampling according to Fig.~\ref{Fig4:ECAL Background-Energy-CR}(a). Because of the background, the peak value of the spectrum is larger than the real value. With multiwaveform fitting, the original energy can be reconstructed more precisely. Additionally as shown in Fig.~\ref{Fig4:ECAL EnergyRes with Multifit}(a), the red curve represents the signal template, and blue represents the fitting results of the background templates. Figure~\ref{Fig4:ECAL EnergyRes with Multifit}(b) shows the energy resolution of 1~GeV gamma-ray events, obtained by multiwaveform fitting. The result is 2.47\%, which is 50\% better than that without waveform fitting. This result is close to the result without considering the background.

Figure~\ref{Fig4:ECAL EnergyRes with BG and Multi-Fitting}(a) shows the EMC energy resolution curve from the Geant4 simulation for photons reconstructed in the barrel only. The expected energy resolution is compared for three scenarios: the traditional clustering reconstruction algorithm without the influence of background, that with the influence of the background, and the multifit method with the background included. The result shows that the pileup background has a significant impact on the energy resolution, especially in the low energy region (\textless 100~MeV), and the energy resolution changes from $\sim$ 4.6\% to $\sim$ 22\% at the energy of 100~MeV when the background effect is included. With increasing energy, the impact of the background  gradually decreases, which is mainly due to the relatively low energy of the background. By using the waveform sampling and fitting method, the energy resolution is greatly improved, from $\sim$ 22\% to $\sim$ 5.2\%, which is close to the result without considering the background. The result for the endcap, which suffers from an even higher pileup background of about 10~MHz, is shown in Fig.~\ref{Fig4:ECAL EnergyRes with BG and Multi-Fitting}(b). Even for a 10~MHz background rate, the multiwaveform fitting method works well and can greatly improve the energy resolution.


\iffalse
In order to evaluate the influence of the background on signal measurement, sampling is carried out according to the simulated counting rate, energy and angular distribution, and superimposed with the real signal. The pulse of readout can be described by inverse Laplace transform, as shown as follows.
\begin{equation}
\begin{array}{cll}
A(t) = -\frac{e^{-at}}{(a-b)^{3}}+\frac{e^{-bt}}{(a-b)^{3}}-\frac{e^{-bt}t}{(a-b)^{2}}+\frac{e^{-bt}t^{2}}{2(a-b)}
\end{array}
\end{equation}
Where A(t) is the pulse output, a is the reciprocal of decay time of pCsI crystal, and b is the reciprocal of shaping time. As shown in Fig.~\ref{Fig4:ECAL Background-Pulse}, they are the pulse spectra within the width of 1500 ns of the acquisition time window, in which the red line is the signal, and the blue lines are the background obtained by sampling according to the background spectra. The signal is superimposed with the background to get the actual measurement result. Figure~\ref{Fig4:ECAL Background-EnergySpectrum} shows the 1~GeV $\gamma$ ray spectrum obtained after considering the background, and its energy resolution is about 3\%. Compared with Fig. \ref{Fig4:ECAL EneRes-5thfactor}, the energy resolution became worse by about 20\%. This needs our attention in the future.
\fi


\iffalse
\begin{figure*}[htbp]
 \centering
 \mbox{
  %\vskip -1.5cm
  \begin{overpic}[width=0.6\textwidth, height=0.45\textwidth]{Figures/Figs_04_00_DetectorSubSystems/Figs_04_04_ElectromagneticCalorimeter/ECAL-Background-CountingRate.pdf}
  \end{overpic}
 }
\caption{The Counting Rate of Background in EMC.}
\label{Fig4:ECAL Background-CR}
\end{figure*}
\fi

\begin{figure*}[htbp]
 \centering
  %\vskip -1.5cm
  \subfloat[][]{\includegraphics[width=0.4\textwidth]{Figures/Figs_04_00_DetectorSubSystems/Figs_04_04_ElectromagneticCalorimeter/ECAL-EneRes-Noise-1MeV.pdf}}
  \subfloat[][]{\includegraphics[width=0.4\textwidth]{Figures/Figs_04_00_DetectorSubSystems/Figs_04_04_ElectromagneticCalorimeter/ECAL-ER-AftBG.pdf}}
\caption{The EMC energy response for 1~GeV photons (a) without background and (b) with background included in the simulation.}
\label{Fig4:ECAL EnergyRes Before and After BG}
\end{figure*}

\begin{figure*}[htbp]
 \centering
  %\vskip -1.5cm
  \subfloat[][]{\includegraphics[width=0.4\textwidth]{Figures/Figs_04_00_DetectorSubSystems/Figs_04_04_ElectromagneticCalorimeter/ECAL-Background-Pulse.pdf}}
  \subfloat[][]{\includegraphics[width=0.4\textwidth]{Figures/Figs_04_00_DetectorSubSystems/Figs_04_04_ElectromagneticCalorimeter/ECAL-ER-AftMFit.pdf}}
\caption{(a) An example output pulse of the EMC with multiwaveform fitting. The dotted green curve is a simulated waveform, which is a superposition of the signal and background spectra. The red curve represents the signal template, and the blue represents the fitting results of the background, (b) The energy resolution of 1~GeV $\gamma$ rays in the EMC using the multifit method.}
\label{Fig4:ECAL EnergyRes with Multifit}
\end{figure*}

\iffalse
\begin{figure*}[htbp]
 \centering
 \mbox{
  %\vskip -1.5cm
  \begin{overpic}[width=0.6\textwidth, height=0.45\textwidth]{Figures/Figs_04_00_DetectorSubSystems/Figs_04_04_ElectromagneticCalorimeter/ECAL-Background-Pulse.pdf}
  \end{overpic}
 }
\caption{An example output pulse of the EMC with multiwaveform fitting. The dotted green curve is a simulated waveform, which is a superposition of the signal and background spectra. The red curve represents the signal template, and the blue represents the fitting results of the background.}
\label{Fig4:ECAL Background-Pulse}
\end{figure*}

\begin{figure*}[htbp]
 \centering
 \mbox{
  %\vskip -1.5cm
  \begin{overpic}[width=0.6\textwidth, height=0.45\textwidth]{Figures/Figs_04_00_DetectorSubSystems/Figs_04_04_ElectromagneticCalorimeter/ECAL-ER-AftMFit.pdf}
  \end{overpic}
 }
\caption{The energy resolution of 1~GeV $\gamma$ rays in the EMC using the multifit method.}
\label{Fig4:ECAL EnergyRes After MultiFit with BG}
\end{figure*}
\fi


\begin{figure*}[htbp]
 \centering
  %\vskip -1.5cm
  \subfloat[][]{\includegraphics[width=0.4\textwidth]{Figures/Figs_04_00_DetectorSubSystems/Figs_04_04_ElectromagneticCalorimeter/ECAL-ERCurve-Barrel-wBG.pdf}}
  \subfloat[][]{\includegraphics[width=0.4\textwidth]{Figures/Figs_04_00_DetectorSubSystems/Figs_04_04_ElectromagneticCalorimeter/ECAL-ERCurve-Endcap-wBG.pdf}}
\caption{The expected EMC energy resolution with the multifit method for (a) the barrel region and (b) the endcap region.}
\label{Fig4:ECAL EnergyRes with BG and Multi-Fitting}
\end{figure*}

%%%%%%%%%%%%%%%%%%%%%%%%%%%%%%%%%%%%%%%%%%%%%%%%%%%%%%%%%%%%%%%%%%%%%%%%%%%%%%%%%
%%%%%%%%%%%%%%%%%%%%%%%%%%%%%%%%%%%%%%%%%%%%%%%%%%%%%%%%%%%%%%%%%%%%%%%%%%%%%%%%%
\FloatBarrier


%%%%%%%%%%%%%%%%%%%%%%%%%%%%%%%%%%%%%%%%%%%%%%%%%%%%%%%%%%%%%%%%%%%%%%%%%%%%%%%%%
%%%%%%%%%%%%%%%%%%%%%%%%%%%%%%%%%%%%%%%%%%%%%%%%%%%%%%%%%%%%%%%%%%%%%%
\subsection{Readout Electronics}
\label{sec:emc_elec}
The electronics system of the EMC provides the photon detector signal readout, analog-to-digital conversion and data acquisition. According to the physics requirements and detector characteristics, some demands placed to the electronics system, which are listed below:
\begin{itemize}
\item The energy deposition on each crystal ranges from 2.5~MeV to 2500~MeV. Considering that the light yield of pCsI is about 100 p.e./MeV and the gain of the photodetector is about 50, the dynamic range of each electronics channel should range from 2~fC to 2000~fC.
\item The high luminosity of the STCF results in a high event rate in the detector. It is estimated that the event rate in the barrel can reach as high as hundreds kHz, and it can be even higher in the endcap. Therefore, the deadtime of the readout system should be shorter than 1~$\mu s$.
\item In addition to energy measurement, time measurement is needed and the precision should be better than 200 ps at 1 GeV.
\end{itemize}
%%%%%%%%%%%%%%%%%%%%%%%%%%%%%%%%%%%%%%%%%%%%%%%%%%%%%%%%%%%%%%%%%%%%%%%%%%%%%%%
%\subsubsection{CSA-based Design}
According to the requirements discussed above, a charge-sensitive amplifier (CSA) based design is proposed. The structure of the CSA-based readout electronics is shown in Fig.~\ref{Fig4:ECAL-electronics-1st}, and it mainly consists of a front-end board (FEB) and a back-end board (BEB). The FEB is placed at the outer end of a pCsI crystal with 4 APDs on it to receive the fluorescence light. Multiple APDs are used to improve the light yield and the system robustness. On the other side of the FEB, 4 CSAs read out the signals of 4 APDs. Then, the outputs of the CSAs are added by two adders, providing in a dual-gain (high/low) outputs. One BEB can connect several FEBs via cables. The BEB provides power and high voltage to FEBs and obtains high-gain and low-gain signals from FEBs. Signals pass through the CR-RC$^2$ shaping circuits and are then digitized by ADCs on the BEB. In addition, there is one comparator corresponding to each channel that can compare the input signal with the threshold and generate a hit for the FPGA-TDC for time measurement. All data produced by the ADCs and the FPGA-TDCs are aggregated, packaged and transmitted by the FPGA. Considering the high event rate, we use the waveform sampling readout method to suppress the high background. The waveform sampling can retain the original waveform information, which is also necessary for the waveform fitting mentioned in the previous background study.


Due to the relatively large size ($\sim6.5\times6.5$~cm) of the crystal end, it is favorable to couple the APD with a large sensitive area to improve the light yield. Presently, there are only a few commercial models with large areas available. The HAMAMATSU company mainly provides two models: S8664-55 (effective area of $5\times5$~mm) and S8664-1010 (effective area of $10\times10$~mm).\\
\begin{figure*}[htb]
 \centering
 \mbox{
  %\vskip -1.5cm
  \begin{overpic}[width=0.5\textwidth, height=0.33\textwidth]{Figures/Figs_04_00_DetectorSubSystems/Figs_04_04_ElectromagneticCalorimeter/ECAL-Electronics-1st.pdf}
  \end{overpic}
 }
\caption{The structure of CSA-based readout electronics.}
\label{Fig4:ECAL-electronics-1st}
\end{figure*}

%%%%%%%%%%%%%%%%%%%%%%%%%%%%%%%%%%%%%%%%%%%%%%%%%%%%%%%%%%%%%%%%%%%%%%%%%%%%%%%%%
%%%%%%%%%%%%%%%%%%%%%%%%%%%%%%%%%%%%%%%%%%%%%%%%%%%%%%%%%%%%%%%%%%%%%%%%%%%%%%%%%
\subsection{EMC R\&D}
\subsubsection{pCsI Crystal}
To understand the properties of the pCsI crystal, relevant tests are carried out in the laboratory. The crystals, as shown in Fig.~\ref{Fig4:ECAL CR-1st}, are produced at the Shanghai Institute of Ceramics, Chinese Academy of Sciences (SIC CAS). The wavelength of emission spectrum spreads from $\sim$ 260 nm to $\sim$ 700 nm, with its main component around 310 nm accounting for $\sim$ 3/4 of the total light intensity.

\begin{figure*}[htbp]
 \centering
 \mbox{
  %\vskip -1.5cm
  \begin{overpic}[width=0.5\textwidth, height=0.33\textwidth]{Figures/Figs_04_00_DetectorSubSystems/Figs_04_04_ElectromagneticCalorimeter/ECAL-pCsICrystal.pdf}
  \end{overpic}
 }
\caption{The pCsI crystal for the EMC.}
\label{Fig4:ECAL CR-1st}
\end{figure*}

\iffalse
\begin{figure*}[htbp]
 \centering
  %\vskip -1.5cm
  \subfloat[][]{\includegraphics[width=0.4\textwidth]{Figures/Figs_04_00_DetectorSubSystems/Figs_04_04_ElectromagneticCalorimeter/ECAL-pCsIEmissionSpec.pdf}}
  \subfloat[][]{\includegraphics[width=0.4\textwidth]{Figures/Figs_04_00_DetectorSubSystems/Figs_04_04_ElectromagneticCalorimeter/ECAL-pCsITransmission.pdf}}
\caption{The fluorescence properties of the pCsI crystal, including (a) the emission spectrum and (b) the longitudinal light transmission.}
\label{Fig4:ECAL pCsI-1st}
\end{figure*}
\fi

\subsubsection{Reflective Material}
The efficiency of light collection is an important factor in achieving a high-precision energy resolution in crystal calorimeters. Since the fluorescence emitted by pCsI is mainly in the ultraviolet wavelength range (about 310 nm) and is very easily to be absorbed. Considering the excellent reflection coefficient of Teflon, which is basically independent of the wavelength, we chose Teflon with a thickness of 300 um to package the crystal.
\iffalse
\begin{figure*}[htbp]
 \centering
 \mbox{
  %\vskip -1.5cm
  \begin{overpic}[width=0.6\textwidth, height=0.45\textwidth]{Figures/Figs_04_00_DetectorSubSystems/Figs_04_04_ElectromagneticCalorimeter/ECAL-ReflectorEff.png}
  \end{overpic}
 }
\caption{The reflectivity of different materials.}
\label{Fig4:ECAL-ReflectorEff}
\end{figure*}
\fi

\iffalse
The pCsI crystal is wrapped by Teflon film (BC-642, produced by Saint Gobain company) before the test of the light yield. A photomultiplier tube (PMT) of XP22020Q produced by PHOTONIS is used in the test. The diameter of the PMT photocathode is 50.8 mm, and its quantum efficiency can reach $\sim$20\% at 310 nm. The crystal and PMT are placed in a ``darkroom'', and the radioactive source is Cs-137 662 keV $\gamma$ rays. The PMT output signal from the anode is processed through a preamplifier (Ortec 113), then through a shaping amplifier (Ortec 671), and finally by a multichannel (Ortec easy MCA) acquisition.\\
The light yield with 1-6 layers of BC642 (thickness $\sim$ 80 $\mu$m  per layer) is measured. The test result is shown in Fig.~\ref{Fig4:ECAL pCsI-4th}. The result shows that the signal amplitude increases gradually with an increasing number of packaging layers at the beginning. When the number of film layers is increased to 4 layers, the signal amplitude does not change much. In later tests, the Teflon thickness is set at 320 $\mu$m.
\fi
\iffalse
\begin{figure*}[htbp]
 \centering
 \mbox{
  %\vskip -1.5cm
  \begin{overpic}[width=0.6\textwidth, height=0.45\textwidth]{Figures/Figs_04_00_DetectorSubSystems/Figs_04_04_ElectromagneticCalorimeter/ECAL-pCsILYTest.pdf}
  \end{overpic}
 }
\caption{The pCsI Crystal Light Yield.}
\label{Fig4:ECAL pCsI-4th}
\end{figure*}
\fi

\subsubsection{Cosmic Ray Test}
\label{sec:emc_cosmic}
The performance of the sensitive unit (pCsI + APD) is tested with the cosmic rays. A typical-sized pCsI crystal is coupled with four APDs. In the test, the crystal is wrapped with three layers of BC642 material. Two large-area APD models from HAMAMATSU are used, S8664-55 and S8664-1010, with sensitive areas of 5~mm~$\times$~5~mm and 10~mm~$\times$ 10~mm, respectively. The APD is coupled with silicone (EJ-550, 310 nm wavelength transmission is greater than 90\%) at the back end of the crystal. The deposition energy of Minimum Ionization Particles (MIPs, $\mu$) passing through the pCsI crystal is about 30 MeV. The results are shown in Fig.~\ref{Fig4:ECAL CR-2nd}. The light yield of pCsI is calculated to be 54 p.e./MeV (with S8664-55 APD) and 156~p.e./MeV (with S8664-1010 APD).\\

\begin{figure*}[htbp]
 \centering
  %\vskip -1.5cm
  \subfloat[][]{\includegraphics[width=0.4\textwidth,angle=-90]{Figures/Figs_04_00_DetectorSubSystems/Figs_04_04_ElectromagneticCalorimeter/ECAL-MIPs-55.pdf}}
  \subfloat[][]{\includegraphics[width=0.4\textwidth,angle=-90]{Figures/Figs_04_00_DetectorSubSystems/Figs_04_04_ElectromagneticCalorimeter/ECAL-MIPs-1010.pdf}}
\caption{The energy deposition of MIPs. (a) S8664-55 result and (b) S8664-1010 result.}
\label{Fig4:ECAL CR-2nd}
\end{figure*}
%%%%%%%%%%%%%%%%%%%%%%%%%%%%%%%%%%%%%%%%%%%%%%%%%%%%%%%%%%%%%%%%%%%%%%%%%%%%%%%%%


%%%%%%%%%%%%%%%%%%%%%%%%%%%%%%%%%%%%%%%%%%%%%%%%%%%%%%%%%%%%%%%%%%%%%%%%%%%%%%%%%
\subsubsection{Readout Electronics}

%At present, the CSA has made a preliminary attempt.
A prototype readout out electronics system for the CSA-based method, described in Sec.~\ref{sec:emc_elec}, has been implemented, as shown in Fig.~\ref{Fig4:ECAL-electronics-3rd}. The dynamic range and noise performance is studied based on a CSA with a 3-JFET as the input stage with different APDs: Hamamatsu S8664-0505 and S8664-1010.
\begin{figure*}[htbp]
 \centering
 \mbox{
  %\vskip -1.5cm
  \begin{overpic}[width=0.8\textwidth, height=0.3\textwidth]{Figures/Figs_04_00_DetectorSubSystems/Figs_04_04_ElectromagneticCalorimeter/ECAL-Electronics-3rd.pdf}
  \end{overpic}
 }
\caption{Prototype electronics of the CSA-based method (FEE on the left and BEU on the right).}
\label{Fig4:ECAL-electronics-3rd}
\end{figure*}


%%%%%%%%%%%%%%%%%%%%%%%%%%%%%%%%%%%%%%%%%%%%%%%%%%%%%%%%%%%%%%%%%%%%%%%%%%%
%\subsubsection{The Noise and Dynamic Range of Electronics}
The electronic noise of the readout system with two types of APDs at different shaping times is measured, with an APD gain of 50. Type S8664-1010, which creates twice as much noise as S8664-0505, achieves similar performance considering the size of the area. The detailed experimental results are shown in {Fig.~\ref{Fig4:ECAL-electronics-4th}}. The noise of S8664-1010 is lower than 0.4~fC when the shaping time is 100~ns, which means the noise performance would be better than 0.8~fC when using 4 APDs and 4 CSAs. The equivalent noise energy is 1 MeV when the light yield reaches 100~p.e./MeV.\\
Given that the upper limit of the charge measurement of the high-gain channel is 120 fC, the dynamic range of the high-gain channel can cover the range of 3~MeV (2.4~fC, 3 times noise) - 150~MeV (120~fC). Considering that the gain ratio of the high- and low-gain channels is 20 and the low-gain noise is close to 2~fC, the dynamic range of the low-gain channel is 10~MeV (6~fC, 3 times noise) - 3000 MeV (2400 fC). The dynamic range that can be realized by this dual gain design is shown in Table~\ref{Tab:ECAL Dynamic Range}.
\begin{figure*}[htbp]
 \centering
 \mbox{
  %\vskip -1.5cm
  \begin{overpic}[width=0.5\textwidth, height=0.33\textwidth]{Figures/Figs_04_00_DetectorSubSystems/Figs_04_04_ElectromagneticCalorimeter/ECAL-Electronics-4th.pdf}
  \end{overpic}
 }
\caption{Noise of the readout system at different shaping times.}
\label{Fig4:ECAL-electronics-4th}
\end{figure*}

\begin{tiny}
\begin{table}[htb]
\begin{center}
\caption{The Dynamic Range of EMC}
\label{Tab:ECAL Dynamic Range}
\begin{tabular}{|c|c|c|}
\hline Channel&Low Limit (MeV)&High Limit (MeV)\\
\hline High Gain&3&150\\
\hline Low Gain&10&3000\\
\hline
\end{tabular}
%\end{scriptsize}
\end{center}
\end{table}
\end{tiny}

%%%%%%%%%%%%%%%%%%%%%%%%%%%%%%%%%%%%%%%%%%%%%%%%%%%%%%%%%%%%%%%%%%%%%%%%%%%
\iffalse
The time resolution of EMC is of great significance to eliminating the background and distinguishing photons from neutral hadrons. The time measurement of the readout electronics (including the APD) is carried out using an LED test, and the test results are shown in Fig.~\ref{Fig4:ECAL-electronics-TR-PF}. The results show that with increasing charge (the LED light intensity enhancement), the time resolution gradually improves, and the corresponding time resolution at 200 fc ($\sim$ 1 GeV energy deposition) is 150 ps.
\begin{figure*}[htbp]
 \centering
 \mbox{
  %\vskip -1.5cm
  \begin{overpic}[width=0.5\textwidth, height=0.33\textwidth]{Figures/Figs_04_00_DetectorSubSystems/Figs_04_04_ElectromagneticCalorimeter/ECAL-Electronics-TR-PF.pdf}
  \end{overpic}
 }
\caption{Electronics time resolution.}
\label{Fig4:ECAL-electronics-TR-PF}
\end{figure*}
\fi
In summary, the conceptual design of the electronics can meet the requirements of the STCF calorimeter in terms of noise, dynamic range and time resolution.

%%%%%%%%%%%%%%%%%%%%%%%%%%%%%%%%%%%%%%%%%%%%%%%%%%%%%%%%%%%%%%%%%
\iffalse
\subsubsection{TIA-based Design}
The readout chain of TIA-based design splits into two parts, the on-detector part and the off-detector, as shown in Fig. \ref{Fig4:ECAL-electronics-2nd}. The current signal from APD is directly transformed to a voltage signal by TIA, and the signal is amplified (about 10 $kohm$) at the same time. Then the voltage signal is transmitted to the off-detector part via a cable. On the off-detector part, dual gain mode is adopted to cover the dynamic range and two flash ADCs (sample rate larger than 200 MSPS) are used to digitize the high gain and low gain signals respectively.
In the TIA-based technique route, the original waveform is amplified and sampled. The deposition energy is calculated by digital integration and the event time is obtained by fitting the sampling data. Although this methods may cost more money and provide worse noise performance compared with the CSA-based method, it can effectively solve the problem of signal pile-up because it does not contain the processes of analog integration and shaping, which always stretch the original waveform up to several microseconds. Due to the high luminosity of STCF, pile-up suppression must be taken into account in the electronics design. Therefore, besides the CSA-based method, the TIA-based method is taken as a backup, especially for the endcap.
\begin{figure*}[htb]
 \centering
 \mbox{
  %\vskip -1.5cm
  \begin{overpic}[width=0.7\textwidth, height=0.4\textwidth]{Figures/Figs_04_00_DetectorSubSystems/Figs_04_04_ElectromagneticCalorimeter/ECAL-Electronics-2nd.pdf}
  \end{overpic}
 }
\caption{The Readout Chain of TIA-based Design.}
\label{Fig4:ECAL-electronics-2nd}
\end{figure*}
\fi



%%%%%%%%%%%%%%%%%%%%%%%%%%%%%%%%%%%%%%%%%%%%%%%%%%%%%%%%%%%%%%%%%%%%%%%%%
%%%%%%%%%%%%%%%%%%%%%%%%%%%%%%%%%%%%%%%%%%%%%%%%%%%%%%%%%%%%%%%%%%%%%%%%%
%%%%%%%%%%%%%%%%%%%%%%%%%%%%%%%%%%%%%%%%%%%%%%%%%%%%%%%%%%%%%%%%%%%%%%%%%
\subsection{Summary}
The baseline design of the STCF EMC adopts a pCsI crystal scintillator coupled with large area APDs, and a charge-sensitive readout scheme is chosen for the readout electronics. The preliminary Monte Carlo simulation and experimental test results show that the conceptual designs can meet the requirements of the STCF. Extensive R\&D works are underway to verify the designs and the key technical aspects.
