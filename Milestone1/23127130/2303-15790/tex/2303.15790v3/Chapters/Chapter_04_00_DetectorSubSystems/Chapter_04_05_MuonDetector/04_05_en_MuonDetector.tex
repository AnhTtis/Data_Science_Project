\section{Muon Detector~(MUD)}
\label{sec:muc}

\subsection{Introduction}
The muon detector~(MUD), as the outermost part of the STCF detector system,
is used to provide muon identification in the presence of a significant pion background.
It can also be used for neutral hadron identification to complement the EMC identification, for example, of neutrons and $K_L$.
A MUD usually
has a sandwich-like structure that consists of a hadron absorber and a detector array. Multiple layers of steel plates are used as both the magnetic flux return yoke and the hadron absorber. The muon detection array is inserted into the gap of these hadron absorbers.


\subsubsection{Performance Requirements}
%{Physics requirements}
For the STCF MUD, a high detection efficiency of muons and a good suppression power for muons/pions are the main requirements~\cite{mud1}.
The momenta of the final state $\mu$ and $\pi$ produced at the STCF are mostly below 2.0~GeV/c, as shown in Sec.~\ref{sec:phys_requirements}.
Because of the EMC and solenoid
material preceding the MUD, muons with momenta less than 0.4 GeV/c cannot be detected by the MUD.
In contrast, muons with low momenta can be identified well by the PID system, as introduced in Sec.~\ref{sec:rich}.
According to the physics requirements, in the momentum range of $p>0.7$~GeV/c, the ideal muon detection efficiency should be higher than 95\%; in the range of $0.5<p<0.7$~GeV/c, the muon detection efficiency should be higher than 70\%, with the muon/hadron suppression power being better than 30.


Additionally, the identification of neutral hadrons with sufficiently high detection efficiency is important for
STCF physics, especially for particles with momenta in the range of [0.2, 1.2]~GeV/c.
The probability of obtaining a hadronic shower in the EMC or MUD for neutrons or \KLs\ is quite high, generating a high multiplicity of photons, neutrons, protons and other hadrons. Thus, a detector array with good photon and neutron sensitivity is required.

For the STCF, with a luminosity of $1\times10^{35}$~cm$^{-2}$s$^{-1}$, the MUD receives a nonnegligible background contribution dominated by neutrons with an energy of [10, 100]~keV \cite{mud2, belle2}. This background may cause more hits in the MUD, which may affect the track finding and identification efficiency of muons. Thus, the MUD must tolerate the very high background rate of the STCF~\cite{mud4, mud5}. In addition, the low energy photon and neutron background may cause significant contamination in the identification of neutral hadrons~\cite{mud6}. As mentioned in Section~\ref{sec:expcon}, the simulated highest background level is approximately 3.99 Hz/cm$^{2}$ and 265 Hz/cm$^{2}$ for barrel and endcap MUD, respectively. As a consequence, optimization of the detector layout to high-rate capabilities and excellent background suppression power is important in the MUD design.

\subsubsection{Technology Choices}
In particle physics experiments, the resistive plate chamber (RPC) and the plastic scintillator are among the most widely used detector technologies for the detection of muon particles.

%\subparagraph{Resistive Plate Chamber}
%\quad\\
A RPC is a traditional option for large-area muon detection with centimeter-level spatial resolution~\cite{mud7}, and has been used in many experiments as the MUD, such as ATLAS~\cite{mud8}, BaBar~\cite{mud9}, Belle~\cite{mud10}, STAR~\cite{star}, BESIII \cite{mud11} and Daya Bay~\cite{DayaBay:2015kir}. RPCs are robust and low cost for large detection areas and have simple manufacturing and maintenance processes. In addition, their centimeter-level spatial resolution can satisfy the demand of particle detection with an area of hundreds of square meters.
A Bakelite-RPC, which operates in the avalanche mode, has
count rate capabilities over 1~kHz/cm$^{2}$, and it also offers the advantage of low
background sensitivity.
Thus, Bakelite-RPC is chosen as a candidate for the STCF MUD.


A plastic scintillator with a silicon photo-multiplier (SiPM) is also a choice for the MUD.
In the Belle II experiment, a combination of polystyrene scintillator strips, wavelength shifting fibers and SiPM was selected as the upgrade of the $K_L$ and muon detector to replace the low-count rate RPCs \cite{mud13}. Compared with the Bakelite-RPC, the plastic scintillator detector has a much higher count rate and is more sensitive to photons and neutrons, resulting in powerful neutral/hadron separation. However, the plastic scintillator detector suffers from higher background counts, leading to worse track-finding and particle identification. As a result,
 the design of the MUD should be optimized to achieve a balance between the rate
capabilities and the identification of low-momentum muons.
Consequently, a hybrid muon detector design that combines a Bakelite-RPC with a plastic scintillator is proposed for the STCF.
Details regarding the design, optimization, and particle identification performance are presented in the following.


\subsection{MUD Conceptual Design}
\subsubsection{Detector Layout}
%\quad\\
The baseline design of the MUD is a combination of a Bakelite-RPC and plastic scintillator detector: 3 layers of Bakelite-RPC are placed in the innermost part, and 7 layers of the plastic scintillator detector form the outer layers.

Considering that the dominant background sources are photons and neutrons, the usage of the Bakelite-RPC in inner layers can decrease the background level in the MUD and help to separate the charged particle tracks and neutral particle shower signals.
The plastic scintillator detector is more sensitive to photon and neutron particles, which are the main background contributors to neutral hadron detection and identification. The {\sc Geant4}~\cite{Geant4_ref} simulation results indicate that with the expected STCF background level described in Sec.~\ref{sec:expcon}, the muon detection efficiency in the momentum range of [0.4, 0.6] GeV/c decreases by 10-20\% if the MUD detectors are all plastic scintillators compared with that of the STCF MUD baseline design.

Fig.~\ref{fig:4.5.02} shows a schematic of the baseline design of the MUD. The barrel MUD covers the solid angle of 79.2\%$\times$4$\pi$ (37.63$^{\circ}$$<$$\theta$$<$142.37$^{\circ}$), and the endcap MUD covers the solid angle of 14.8\%$\times$4$\pi$ (20$^{\circ}$$<$$\theta$$<$37.63$^{\circ}$ and 142.37$^{\circ}$$<$$\theta$$<$160$^{\circ}$). Both the barrel and endcap MUD contain 10 layers of detectors, and the iron yoke layout can be seen at Sec.~\ref{sec:yoke}.

In MUD, the width of the RPC $X/Y$ readout strips and the width of the plastic scintillator are both 4~cm. The maximum length of the Bakelite-RPC module is approximately 1.1 m, and the maximum length of the plastic scintillator strip is 2.4~m~\cite{mud13}. As shown in Fig.~\ref{fig:4.5.01}, each layer of barrel MUD consists of 8 rectangle detector module. For Bakelite-RPC, the module is divided into 5 sub-modules along Z direction, and 2 sub-modules along R$\phi$ direction to control the maximum length of readout strips around 1.1 m. In each sub-module, the 2-D readout strips are perpendicularly arranged. For plastic scintillator, the axis of the scintillator strip is perpendicular to the Z direction. Each layer of the endcap MUD consists of 8 trapezoidal modules. For both Bakelite-RPC and plastic scintillator, the module has 60 strips in R$\phi$ direction, and 49 to 43 strips in R direction, due to the increase of the inner radius.

%%%%%%%%%%%%%%%%%%% Fig %%%%%%%%%%%%%%%%%%%%%%%%%%
\begin{figure*}[htb]
    \centering
    {
        \includegraphics[width=0.7\textwidth]{Figures/Figs_04_00_DetectorSubSystems/Figs_04_05_MuonDetector/fig02_stcfMUDdesign3.jpg}
    }
    \hspace{5mm}   
    {
        \includegraphics[width=0.7\textwidth]{Figures/Figs_04_00_DetectorSubSystems/Figs_04_05_MuonDetector/fig02_stcfMUDdesign4.jpg}
    }
    \vspace{0cm}
\caption{Schematic of the MUD design. (a) Half-section view of the MUD, and partial enlarged view of the sandwich placement of the Bakelite-RPC, plastic scintillator, and iron yoke. (b) Cutaway view of the MUD and the setting of the main structural parameters.}
    \label{fig:4.5.02}
\end{figure*}
%%%%%%%%%%%%%%%%%%%%%%%%%%%%%%%%%%%%%%%%%%%%%%%%%%

%%%%%%%%%%%%%%%%%%% Fig %%%%%%%%%%%%%%%%%%%%%%%%%%
\begin{figure*}[htb]
    \centering
    {
        \includegraphics[width=0.85\textwidth]{Figures/Figs_04_00_DetectorSubSystems/Figs_04_05_MuonDetector/MUDmodules.png}
    }
    \vspace{0cm}
\caption{Module layout of the MUD design. (a) Bakelite-RPC in barrel MUD. (b) Scintillator in barrel MUD. (c) Bakelite-RPC and scintillator in endcap MUD}
    \label{fig:4.5.01}
\end{figure*}
%%%%%%%%%%%%%%%%%%%%%%%%%%%%%%%%%%%%%%%%%%%%%%%%%%

\subsubsection{Neutron Shield and Cylindrical Yoke Component}
%\quad\\
Previous studies by Belle II indicated that the detector system was affected by the neutron background generated by the GeV level electrons and positrons that escape from the beam background~\cite{mud5}.
To suppress the effect of the neutron background, the outer face of the endcap MUD is covered by a 15 cm neutron shielding layer, as shown in Fig.~\ref{fig:4.5.02}. The outer layer of the neutron shielding is 5 cm lead, and the inner layer is 10 cm boron-doped polyethylene (10\%wt of nat-boron). Lead can moderate fast neutrons to an energy of approximately 1 MeV, and boron-doped polyethylene can moderate neutrons to the thermal neutron level. Neutrons with energies less than 1~eV have a large probability of being absorbed by the boron atoms.
 A {\sc Geant4} simulation demonstrated that the 15 cm-thick composite neutron shielding could
decrease the MUD hits by approximately 90\% for neutrons with a kinetic energy less than 1~MeV.


A cylindrical yoke component is arranged between the beamline and endcap MUD detector array, which is designed to make the magnet field uniform in the STCF detector system.
The thickness of this cylindrical yoke is approximately $40\sim45$~cm. {\sc Geant4} simulations indicate that the yoke provides background suppression capabilities similar to those of composite shielding.



\FloatBarrier

\subsubsection{MUD Optimization}

The thickness of the iron yoke and the number of detector layers are optimized via a simulation study, and the detailed parameters are summarized in Table~\ref{tab:4.5.02}. Fig.~\ref{fig:4.5.03a} illustrates the muon detection efficiency curves for different detector layer settings and a yoke with a thickness of 51~cm. In the simulation, 9 to 11 layers are applied
and evaluated. The result indicates that 10 or 11 detector layers can produce a higher and smoother muon detection efficiency. Considering the detector complexity and manufacturing costs, the MUD baseline design incorporates 10 detector layers.
It should be noted that in the muon detection efficiency curves, $\mu/\pi$ suppression powers of both 33 and 100 are applied, while the former is sufficient for most physics processes containing muons at the STCF.



%%%%%%%%%%%%%%%%%  TABLE  %%%%%%%%%%%%%%%%%%%%%%%%
\begin{table*}[htb]
\small
    \caption{The structure parameters of the conceptual baseline design of the MUD. R$_{in}$ and R$_{out}$ are the inner and outer radius of the barrel MUD, respectively, including the 15 cm-thick iron plate shielding outside the detector system. R$_{e}$ is the inner radius of the endcap MUD. L$_{Barrel}$ and T$_{Endcap}$ are the length of the barrel and endcap MUD in the z-direction, respectively. The size of neutron shielding layer is not included.}
    \label{tab:4.5.02}
    \vspace{0pt}
    \centering
    \begin{tabular}{ll}
        \hline
        \thead[l]{Parameter} & \thead[l]{Baseline design}\\
        \hline
            R$_{in}$ [cm]	           &185 \\
            R$_{out}$ [cm]	          &291 \\
            R$_{e}$ [cm]	                &85 \\
            L$_{Barrel}$ [cm]	           &480 \\
            T$_{Endcap}$ [cm]	           &107 \\
            Segmentation in $\phi$	         &8 \\
            Number of detector layers	      &10 \\
            Iron yoke thickness [cm]    &4/4/4.5/4.5/6/6/6/8/8 cm \\
            ($\lambda$=16.77 cm)            &Total: 51 cm, 3.04$\lambda$ \\
            Solid angle	                &79.2\%$\times$4$\pi$ in barrel \\
                                &14.8\%$\times$4$\pi$ in endcap \\
                                &94\%$\times$4$\pi$ in total \\
            Total area [m$^2$]	         &Barrel $\sim$717 \\
                                &Endcap $\sim$520 \\
                                &Total $\sim$1237 \\
        \hline
    \end{tabular}
\end{table*}
%%%%%%%%%%%%%%%%%%%%%%%%%%%%%%%%%%%%%%%%%%%%%%%%%%

%%%%%%%%%%%%%%%%%%% Fig %%%%%%%%%%%%%%%%%%%%%%%%%%
\begin{figure*}[htb]
    \centering
    {
        \includegraphics[width=0.5\textwidth]{Figures/Figs_04_00_DetectorSubSystems/Figs_04_05_MuonDetector/fig03_muideff01.pdf}
    }
\caption{The muon detection efficiency curve from the Geant4 simulation. Efficiencies are compared with different detector layer settings.
    The results for two scenarios of muon/pion suppression power, 33 and 100, are shown.}
    \label{fig:4.5.03a}
\end{figure*}
%%%%%%%%%%%%%%%%%%%%%%%%%%%%%%%%%%%%%%%%%%%%%%%%%%

The arrangement of the Bakelite-RPC and plastic scintillator in the ten layers affects the performance
of the MUD. On one hand, the plastic scintillator has higher robustness and higher detection
efficiency for high-momentum muon. However, the {\sc Geant4} simulation indicates that the
Bakelite-RPC exhibits better performance in detecting low-momentum muons in the high-background region
of the MUD.
Fig.~\ref{fig:combination} shows the {\sc Geant4}-simulated muon detection efficiency as a function of momentum for
different MUD layouts at the full STCF luminosity. The
MUD designs with two-, three-, or four-layer Bakelite-RPCs exhibit similar $\mu/\pi$ separation
power under the current luminosity and background level.
For the design with the two-layer Bakelite-RPC, a high count rate and significant interference may occur.
in the 3rd (plastic scintillator) layer due to the potential fluctuations in the background level or the
future upgrades of the STCF. As a result, the hybrid MUD design with the three-layer Bakelite-RPC
and a seven-layer plastic scintillator is considered the optimal choice.

\begin{figure*}[htb]
    \centering
    {
        \includegraphics[width=0.5\textwidth]{Figures/Figs_04_00_DetectorSubSystems/Figs_04_05_MuonDetector/fig_combination.png}
    }
\caption{The muon detection efficiency curves with different combinations of Bakelite-RPC and plastic scintillator
in the MUD design along the direction of $\theta= 90^{\circ}$ and $\phi = 90^{\circ}$ ($\theta$: polar angle, $\phi$: azimuth angle) including the background.}
    \label{fig:combination}
\end{figure*}

The granularity is determined by the readout strip pitch of the Bakelite-RPC and the size of the
plastic scintillator strips. Physical simulations of the measurement precision for reconstructed
muons indicate that a spatial resolution of 1–2 cm is required in the MUD, equivalent to a detector
granularity of 3.5 - 7 cm. Fig.~\ref{fig:granularity} presents simulated muon detection efficiency curves with
different granularities, indicating similar particle detection performances. As a smaller granularity
implies additional electronic channels, both the readout strip pitch of the Bakelite-RPC and the width of the plastic scintillator strips are chosen to be 4~cm.


\begin{figure*}[htb]
    \centering
    {
        \includegraphics[width=0.5\textwidth]{Figures/Figs_04_00_DetectorSubSystems/Figs_04_05_MuonDetector/fig_granularity.png}
    }
\caption{{\sc Geant4}-simulated muon detection efficiency with different granularities along the zenith direction.}
    \label{fig:granularity}
\end{figure*}

\subsection{Expected Performance}
To evaluate the expected performance of the MUD, {\sc Geant4}-based full simulations are studied. The identification efficiency of muons and neutral hadrons is obtained using boosted decision tree~(BDT) algorithms. The details of BDT algorithms can be found in Ref.~\cite{Fang:2021fhm}. In the MUD, both the Bakelite-RPC and the plastic scintillator have very high sensitivity to charged muons and pions, and the required spatial resolution is within 2~cm, which can be ensured by the 4~cm detector granularity.

\subsubsection{Muon Identification Efficiency}
%\quad\\
Fig.~\ref{fig:4.5.prob} shows the probability of a muon arriving at the MUD as a function of the muon momentum. In the simulation, the bin width is 50~MeV/c. The result indicates that a muon with momentum between 400 and 450~MeV/c only has a probability of 0.68\% of arriving at the MUD. When the muon momentum reaches 550~MeV/c, almost all of muons can generate signals in the MUD.
%%%%%%%%%%%%%%%%%%% Fig %%%%%%%%%%%%%%%%%%%%%%%%%%
\begin{figure*}[htb]
    \centering
    \includegraphics[width=70mm]{Figures/Figs_04_00_DetectorSubSystems/Figs_04_05_MuonDetector/mu_arriveat_mud_prob.png}
    \vspace{0cm}
\caption{The probability that a muon arrives at the MUD as a function of the muon momentum in the zenith direction.}
    \label{fig:4.5.prob}
\end{figure*}
%%%%%%%%%%%%%%%%%%%%%%%%%%%%%%%%%%%%%%%%%%%%%%%%%%

Fig.~\ref{fig:4.5.03b} shows the simulated muon detection efficiency with a polar angle of 90 deg and with $\mu/\pi$ suppression power of 33 and 100.
A significant increase in the muon detection efficiency can be observed in the momentum range of [0.65, 1.5]~GeV/c
owing to a thicker yoke and an optimized detector setting.
This result indicates that the muon detection efficiency curve of the baseline design is smoother than that of the BESIII-like MUD geometry. In the low momentum range [0.4-0.6]~GeV/c, the STCF MUD design exhibits a performance similar to that of the BESIII geometry, ensuring an acceptable muon detection efficiency. Fig.~\ref{fig:4.5.04} shows the muon identification efficiency of the baseline design, with the particle momentum in [0, 2.5]~GeV/c and polar angle in [20$^{\circ}$, 160$^{\circ}$]
considering a pion rejection rate of 97\% and background influence.

%%%%%%%%%%%%%%%%%%% Fig %%%%%%%%%%%%%%%%%%%%%%%%%%
\begin{figure*}[htb]
    \centering
    \includegraphics[width=70mm]{Figures/Figs_04_00_DetectorSubSystems/Figs_04_05_MuonDetector/fig03_muideff02.pdf}
    \vspace{0cm}
\caption{The muon detection efficiency curve from the {\sc Geant4} simulation, with a polar angle of 90 degree.
    Results for two scenarios of muon/pion suppression power, 33 and 100, are also shown.}
    \label{fig:4.5.03b}
\end{figure*}
%%%%%%%%%%%%%%%%%%%%%%%%%%%%%%%%%%%%%%%%%%%%%%%%%%

%%%%%%%%%%%%%%%%%%% Fig %%%%%%%%%%%%%%%%%%%%%%%%%%
\begin{figure*}[htb]
    \centering
    {
        \includegraphics[height=50mm]{Figures/Figs_04_00_DetectorSubSystems/Figs_04_05_MuonDetector/fig04_upeff1.png}
    }
    \hspace{5mm}
    {
        \includegraphics[height=50mm]{Figures/Figs_04_00_DetectorSubSystems/Figs_04_05_MuonDetector/fig04_upeff2.png}
    }
    \vspace{0cm}
\caption{2D map of the muon identification efficiency as a function of the momentum and the $\theta$ angle from {\sc Geant4} simulation.
    The $\mu/\pi$ suppression power is assumed to be 33.}
    \label{fig:4.5.04}
\end{figure*}
%%%%%%%%%%%%%%%%%%%%%%%%%%%%%%%%%%%%%%%%%%%%%%%%%%


\subsubsection{Neutral Hadron Detection and Identification}
%\quad\\
At the STCF, the MUD is set as an auxiliary neutral hadron detector to complement the EMC. In this baseline MUD design, the 3-layer Bakelite-RPC acts as a filter, preventing too many secondary gamma and neutron hits from being generated by background interference. The {\sc Geant4} simulation illustrates that approximately 40\% of the neutrons and $K_L$ deposit a very small amount of energy, less than 40 MeV, in the EMC. Thus, the MUD is responsible for detecting them. Table~\ref{tab:4.5.03} shows the Geant4 simulated cluster size of neutral hadrons in the 7 layers of plastic scintillator detectors, and Fig.~\ref{fig:4.5.05} presents the neutral hadron detection efficiency curves with different fake rates. When a neutral hadron cannot be detected by EMC directly, the MUD has a quite high efficiency in terms of detection and identification of the hadron, demonstrating the excellent identification power of the MUD for neutral hadrons.

%%%%%%%%%%%%%%%%%  TABLE  %%%%%%%%%%%%%%%%%%%%%%%%
\begin{table*}[htb]
\small
    \caption{The {\sc Geant4} simulated neutron and $K_L$ cluster parameters in plastic scintillator detector layers.}
    \label{tab:4.5.03}
    \vspace{0pt}
    \centering
    \begin{tabular}{lllll}
        \hline
        \thead[l]{ } & \thead[l]{Neutron\\Average cluster size\\ in scintillators}&\thead[l]{\\Probability of cluster\\ size $\ge$2 } & \thead[l]{$K_L$\\Average cluster size\\ in scintillators}&\thead[l]{\\Probability of cluster\\ size $\ge$2 }\\
        \hline
            200 MeV/c	&2.42	&5\%	&4.42	&32\% \\
            400 MeV/c	&4.07	&31\%	&6.48	&50\% \\
            600 MeV/c	&5.57	&49\%	&7.88	&68\% \\
            800 MeV/c	&7.23	&66\%	&9.20	&74\% \\
            1000 MeV/c	&8.31	&74\%	&8.96	&76\% \\
            1200 MeV/c	&9.03	&79\%	&11.18	&84\% \\
        \hline
    \end{tabular}
\end{table*}
%%%%%%%%%%%%%%%%%%%%%%%%%%%%%%%%%%%%%%%%%%%%%%%%%%

%%%%%%%%%%%%%%%%%%% Fig %%%%%%%%%%%%%%%%%%%%%%%%%%
\begin{figure*}[htb]
    \centering
    {
        \includegraphics[height=50mm]{Figures/Figs_04_00_DetectorSubSystems/Figs_04_05_MuonDetector/fig05_neutralhadroneff.pdf}
    }
    \vspace{0cm}
\caption{The detection efficiency curves of various neutral hadrons in the MUD (notice: all of these neutral hadrons deposit less than 40 MeV energy in the EMC).}
    \label{fig:4.5.05}
\end{figure*}
%%%%%%%%%%%%%%%%%%%%%%%%%%%%%%%%%%%%%%%%%%%%%%%%%%

\subsubsection{Background Simulation}
%\quad\\
Table~\ref{tab:4.5.04} shows the background count rate in the barrel MUD, calculated based on the STCF full {\sc Geant4} simulations. The background estimation demonstrates that the rate capabilities of both the barrel and endcap MUD detectors can meet the requirements. For good neutral hadron detection performance, the usage of a plastic scintillator in the MUD is necessary. The simulation result indicates that with the compound detector design of the MUD, the first layer of the Bakelite-RPC and the first layer of the plastic scintillator have background count rates with the same order of magnitude, ensuring minimal background interference over the whole MUD detector volume. In this case, a higher accuracy of identification for both muons/pions and neutral hadrons can be obtained. The background level in the endcap MUD is almost 2 to 3 times that in the barrel MUD, indicating the necessity of a background shield on the outer surface of the endcap MUD.

%%%%%%%%%%%%%%%%%  TABLE  %%%%%%%%%%%%%%%%%%%%%%%%
\begin{table*}[htb]
\small
    \caption{The Geant4 simulated barrel MUD background count rate.}
    \label{tab:4.5.04}
    \vspace{0pt}
    \centering
    \begin{tabular}{lllllllllll}
        \hline
        \thead[l]{Detector type} &\multicolumn{3}{l}{Bakelite-RPC}&\multicolumn{7}{l}{Plastic scintillator}\\
        \hline
            Detector layer &1&2&3&4&5&6&7&8&9&10\\
            Simulated background & \multirow{3}*{9.2}&\multirow{3}*{3.54} &\multirow{3}*{1.42}&\multirow{3}*{4.25}&\multirow{3}*{6.50}&\multirow{3}*{2.80}&\multirow{3}*{1.77}&\multirow{3}*{0.76}&\multirow{3}*{0.39}&\multirow{3}*{0.36}\\
            count rate in the barrel &&&&&&&&&& \\
            (Hz/cm2) &&&&&&&&&& \\
        \hline
    \end{tabular}
\end{table*}
%%%%%%%%%%%%%%%%%%%%%%%%%%%%%%%%%%%%%%%%%%%%%%%%%%

As shown in Tables \ref{tab:TIDNIEL_max} and \ref{tab:TIDNIEL_eletronic}, the highest TID and NIEL damage values in the MUD are 0.37~Gy/y and $2.79\times10^{12}$ n/y, respectively. Good working conditions can be maintained in the Bakelite-RPC by flushing the working gas. Other studies indicate that the plastic scintillator can tolerate more than 1000~Gy~\cite{mud15,mud16}, indicating that the MUD detector system can work well under background radiation for decades.

\subsubsection{Readout Design}
%\quad\\
In the MUD conceptual design, a 3-layer Bakelite-RPC and 7-layer plastic scintillator are used. The plastic scintillator strips in the barrel are double-ended readouts, and those in the endcap are single-ended readouts. The numbers of readout channels in the MUD are estimated in Table~\ref{tab:4.5.05}, with a total number of readout channels of 16000 for the Bakelite-RPC and 25280 for the plastic scintillator, considering a 4 cm granularity in the detector.

In the MUD, the signal position information can be obtained by the number of channels, so it is only necessary to record the time information of signals whose amplitude exceeds the threshold for particle discrimination. For the Bakelite-RPC in avalanche mode, the front-end electronics are required to achieve an effective gain of 10-fold and a pulse time width of 50~ns. Considering that the count rate of the MUD is not high, an ordinary charge-sensitive amplifier and integral shaping circuit can satisfy the requirements. For the plastic scintillator, a SiPM is used as an electron multiplier. With 5~p.e. thresholds, the noise rate can be controlled within 1~kHz, and the measurement efficiency of MIP could exceed 99\%. This can meet our needs. Additionally, in the barrel-MUD, a double-ended readout is required to determine the hit position, which requires a time resolution of approximately 0.5~ns. Using the waveform sampling method, a 0.25~ns time resolution can be achieved, and we are still studying a lower-cost implementation for the barrel-MUD.

%%%%%%%%%%%%%%%%%  TABLE  %%%%%%%%%%%%%%%%%%%%%%%%
\begin{table*}[htb]
\small
    \caption{The estimated readout channel requirement in the MUD conceptual design.}
    \label{tab:4.5.05}
    \vspace{0pt}
    \centering
    \begin{tabular}{llllllllll}
        \hline
        \thead[l]{ } & \thead[l]{Detector } & \thead[l]{ }& \thead[l]{Barrel}& \thead[l]{MUD }& \thead[l]{ }& \thead[l]{ }& \thead[l]{Endcap}& \thead[l]{MUD }&\thead[l]{ } \\
         &layer &{Half-} &{Half-} &{Channel} &{Channel} &{Inner } &{Outer } &{Channel} &{Channel} \\
         & &{length} &{length} &{number} &{number} &{radius} &{radius} &{number} &{number} \\
         & &{in X} &{in Z} &{in X} &{in Z} &{(cm)} &{(cm)} &{in X} &{in Z} \\
         & &{(cm)} &{(cm)} &{} &{} &{} &{} &{} &{} \\
         \hline
            Bakelite-	&1	&76.6	&240	&1535	&1920	&94	&290	&960	&784 \\
            RPC	&2	&79.9	&240	&1600	&1920	&94	&290	&960	&784 \\
            	&3	&83.3	&240	&1670	&1920	&98	&290	&960	&768 \\
            Plastic 	&4	&86.8	&240	&0	&1920	&98&	290	&960	&768 \\
            scintillator	&5	&90.3	&240	&0	&1920	&102	&290	&960	&752 \\
            	&6	&94.4	&240	&0	&1920	&102	&290	&960	&752 \\
            	&7	&98.6	&240	&0	&1920	&106	&290	&960	&736 \\
            	&8	&102.7	&240	&0	&1920	&110	&290	&960	&720 \\
            	&9	&107.7	&240	&0	&1920	&114	&290	&960	&704 \\
            	&10	&112.7	&240	&0	&1920	&118	&290	&960	&688 \\
        \hline
    \end{tabular}
\end{table*}
%%%%%%%%%%%%%%%%%%%%%%%%%%%%%%%%%%%%%%%%%%%%%%%%%%


According to the background simulation results, the highest count rate per channel in the barrel MUD appears in the first layer of the Bakelite-RPC, with an average count rate of 4~Hz/cm$^{2}$. The corresponding Bakelite-RPC readout strip has a size of 1.1~m$~\times~$ 4~cm, leading to the highest count rate of 1.76~kHz. The highest count rate per channel in the endcap MUD is in the first layer of the plastic scintillator close to the MDI, which is approximately 102~Hz/cm$^{2}$. With the longest scintillator strip (2.4~m$~\times~$4~cm), the highest count rate per channel is calculated to be 98~kHz.

In the reconstruction of tracks and clusters in MUDs, the position information of hits can be reconstructed by the number of readout channels or strips; thus, only the hit-time information should be transported from the detector to the data storage system, corresponding to 4~bytes/signal (16 bits for time information, 12 bits for the channel number, 4 bits for quality). Considering the background level of the MUD and the 300 ns time window, the data stream size is approximately 105~MB/s for the MUD system.

\subsection{Conclusion and Outlook}
The conceptual baseline design of the MUD is a combination of Bakelite-based RPC and plastic scintillator detectors. This design can improve the neutral hadron identification efficiency with the proper $\mu/\pi$ separation capabilities. Detailed detector technologies and further optimization will be studied in the future, such as the optimization of the muon/pion identification algorithm, the detection and identification of neutral hadrons, and R\&D of MUD detectors.


