\section{Physics Requirements}
\label{sec:phys_requirements}

To access the physics potential and scientific merits of the STCF as discussed in Chapter~\ref{CDR_phys} and to understand the demands on the
machine and detector performance from a physics perspective,
the physics requirements of the detector design are proposed from several benchmark processes,
as listed in Table~\ref{phyreq} along with the corresponding parameters to be optimized.
The processes cover a wide range of physics programs
including $CPV$ tests in the $\tau$ lepton, baryon and charm meson sectors, searches for charged lepton flavor violations~(cLFVs), tests of the unitarity of the CKM matrix, nucleon structure researches via electromagnetic form factors and fragmentation functions {\it, etc. }
In Table~\ref{phyreq}, and in the text overall, $chrg1/chrg2$ denotes the sepatation of the signal particle
$chrg1$ from the background particle $chrg2$. Both the misidentification rate and the
suppression power are used to evaluate the power of the particle identification.


\begin{table}[htbp]
\caption{Summary of the physics processes and the corresponding responses to be optimized.}
\label{phyreq}
\begin{center}
\begin{tabular}{c|c}
\hline
\hline
  Physics Process  &  Optimized Parameter   \\
\hline
 $\tau\to K_{s}\pi\nu_{\tau}$; $J/\psi\to\Lambda\bar{\Lambda}$    & Vertex reconstruction; Tracking (efficiency, momentum resolution) \\
 $\tau\to\gamma\mu$; $\tau\to lll$; $D_{s}\to\mu\nu$; $D\to\pi\mu\nu$ 	 & PID (range, $\mu/\pi$ suppression power, efficiency) \\
 $e^{+}e^{-}\to\pipi+X$, $KK+X$; $D_{s}\to\tau \nu_{\tau}$        & PID (range, $\pi/K$ and $K/\pi$ misidentification, efficiency)  \\
 $\tau\to\gamma\mu$; $J/\psi\to\Lambda\bar{\Lambda}$              &  Photon  (position/energy resolution)\\
 $e^{+}e^{-}\to n\bar{n}$, $e^{+}e^{-}\to\gamma n\bar{n}$         & $n$  (position/time resolution)\\
 $D^{0}\to K_{L}\pi^{+}\pi^{-}$                                   & $K_{L}$  (position/time resolution) \\
\hline
\hline
\end{tabular}
\end{center}
\end{table}



\subsection{Charged Particles}

The main goal of the STCF detector is to precisely measure the production and decay properties of various particles, {\it i.e.}, charmonium states, $XYZ$ particles,
$\tau$-leptons, charm hadrons and all hyperons.
 It is crucial to be able to detect charged final-state particles with an excellent tracking
efficiency and momentum resolution.
Figure~\ref{pr:chgmom} shows the momentum distribution for charged particles ($e$, $\mu$, $\pi$, $K$ and $p$)
from several physics processes.
The momenta spectra cover a large range, spanning as high as 3.5~GeV/c, while most particles have momenta less than 2.0~GeV/c. In addition, there are a considerable number of particles with momenta values lower than 0.4 GeV/c.
This requires the detector to be able to cover a large momentum range with high reconstruction efficiency.

\begin{figure}[htbp]
\begin{center}
\begin{overpic}[width=9cm, height=7cm, angle=0]{Figures/Figs_01_PhysicalRequirement/plot_trk_p_v3.pdf}
\end{overpic}
\caption{ Momentum distributions of charged particles from various processes at the truth level, normalized to 10k entries.}
\label{pr:chgmom}
\end{center}
\end{figure}

\subsubsection{Vertex Resolution}
The STCF has unique features of rich resonances and the threshold production of hadron pairs,
which makes it a clean place to study their properties with high detection
precision and a low background.
Due to the threshold characteristics, the decay lengths of most particles with finite
lifetimes, {\it i.e.}, $D^{0}$, $D^{\pm}$, end immediately after the production points.
Therefore, the time-dependent analysis of $D^{0}$ decay is not applicable at STCF for
the study of $D^{0}-\bar{D}^{0}$ mixing.
Instead, with the charm meson pair produced near the threshold, {\it i.e.}, $e^{+}e^{-}\to D^{*0}\bar{D}^{0}+c.c$ at $\sqrt{s}=4.009$~GeV, the mixing parameters of $D^{0}-\bar{D}^{0}$ can be estimated with much better
sensitivity
by means of a quantum coherence approach.
As a consequence, the vertex measurement of the tracks with a precision of hundreds of $\mu$m is sufficient to meet the needs of
most physics programs at the STCF. For some special physics such as $\tau$ lifetime measurement, vertex resolution requirements should be re-evaluated.

\subsubsection {Tracking Efficiency}
In the benchmark processes of testing $CPV$s in hyperon decays, $J/\psi\to\Lambda\bar{\Lambda}$ and $J/\psi\to\Xi^{-}\bar{\Xi}^{+}$,
the charged pions in the final states have low momentum within the region of (0, 0.3)~GeV/c
while the momentum of protons is relatively high within the region of (0.6, 1.0)~GeV/c.
In the process of testing $CPV$s in $\tau$-lepton decay, $\tau^{-}\to K_{s}\pi^{-}\nu_{\tau}$, the charged pions mostly accumulate in the low-momentum region.
For most charm meson $D$ or $D_{s}$ decays, the momenta of the final states are low due to large multiplicities.

The optimization of tracking efficiency is studied for charged pions.
Six different efficiency curves of charged pions are applied, as shown in Fig.~\ref{trackeff1}(a),
where curve1 is the real detection efficiency curve obtained from BESIII detector, while
curve 2-6 are obtained by multiplying the detection efficiency of curve 1 by a factor of 1.1, 1.2, 1.3, 1.4 and 1.5, respectively.
These efficiency curves are applied to various benchmark processes, and the
improvement in the detection efficiency $\Delta(\epsilon)$ to that of the unoptimized state is depicted in Fig.~\ref{trackeff1}(b).
The value of $\Delta(\epsilon)$ is related to the number of final charged particles with low momentum; thus,
$\Delta(\epsilon)$ is larger in
$D_{s}^{+}\to K^{+}K^{-}\pi^{+}$ than in $J/\psi\to\Lambda\bar{\Lambda}\to p\pi^{-} \bar{n}\pi^{0}$.
The efficiency is most optimized for curve 2 in all the processes,
which corresponds to a pion tracking efficiency of 90\% at $p_{T}=0.1$~GeV/c
within the detector acceptance.
Under this optimized efficiency curve, the physics requirements can be achieved:
sensitivities for $CPV$ in $\Lambda$ hyperon decay
on the order of $\Delta A_{CP}^{\Lambda}< \mathcal{O}(10^{-4})$ and in
$\tau$ lepton decay on the order of $\mathcal{O}(10^{-3})$ and plenty of
reconstructed $D_{s}$ mesons for further (semi)leptonic decays.


The polar angles of the particles from various processes in Fig.~\ref{pr:chgmom} are usually uniformly distributed,
and therefore, a large acceptance, {\it i.e.} $|\cos\theta|<0.93$, is needed.

\begin{figure}[hbtp]
\begin{center}
\begin{overpic}[width=7.3 cm, height=5.0 cm, angle=0]{Figures/Figs_01_PhysicalRequirement/trkeffv3.pdf}
\put(23,55){\textbf{(a)}}
\end{overpic}
\begin{overpic}[width=7.3 cm, height=5.0 cm, angle=0]{Figures/Figs_01_PhysicalRequirement/effoptv3.pdf}
\put(23,55){\textbf{(b)}}
\end{overpic}
\end{center}
\caption{ (a) Six tracking efficiency curves of charged pions used for optimization of benchmark processes.
(b) The relative improvement in the detection efficiency $(\Delta(\epsilon)$ for the processes for the six efficiencies, where solid circles represent $D_{s}^{+}\to K^{+}K^{-}\pi^{+}$, triangles represent $\tau^{-}\to K_{S}\pi^{-}\nu_{\tau}$ and open circles represent $J/\psi\to\Lambda\bar{\Lambda}\to p\pi^{-} \bar{n}\pi^{0}$.}
\label{trackeff1}
\end{figure}

\subsubsection {Momentum Resolution}

A good momentum resolution for charged particles, especially for low-momentum tracks, is important in
distinguishing signals from background events. For example, in the rare semileptonic decays of hyperons,
$\Xi^{-}\to\Lambda e^{-}\nu_{e}$, the dominant background comes from pionic decay $\Xi^{-}\to \Lambda \pi^{-}$.
A good momentum resolution for low-momentum pions and electrons is essential to separate the signal from the background
in this process.
%It is essential to improve the momentum resolution of low-momentum $e$ and $\pi$ to have a good separation
%in the reconstruction of events.

The momenta of charged particles are usually measured by their flight trajectories in a magnetic field.
With more points measured on the trajectory, more accurate track position is obtained, and better momentum resolution
can be obtained. In addition, for tracks associated with low momentum, the main effect on the momentum resolution
comes from multiple Coulomb scattering on the material in the detector. Therefore, a material with a low atomic number Z is
required in the detector.

Next, we discuss the effect of position resolution on momentum measurement from $D$ meson decay processes
whose final states are in a moderate momentum region and $J/\psi\to\Lambda\bar{\Lambda}$ with low-momentum pions.
The reconstruction of $e^{+}e^{-}\to D^{0}\bar{D}^{0}$ produced near the threshold exploits two key variables that
discriminate the signal from the background, the energy difference $\Delta E=E-E_{\rm beam}$ and the beam constrained mass $M_{BC} = \sqrt{E_{\rm beam}^2/c^4-p^2/c^2}$,
where $E_{\rm beam}$ is the beam energy and $E$ and $p$ are the total measured energy and three-momentum of the charm meson, respectively.
Three sets of position resolutions are applied, {\it, i.e.}, $\sigma_{xy}=$100, 150 and 300~$\mu$m.
The results indicate that a better position resolution for the charged track
would improve the resolution of $\Delta E$ and $M_{BC}$, as shown in Fig.~\ref{D0}(a) and (b), but the improvement is not significant.
In addition, the variation in detection efficiency in $J/\psi\to\Lambda\bar{\Lambda}$ is examined, yielding similar
results, as shown in Fig.~\ref{D0}(c) and (d).

Considering that an acceptable position resolution that trivially affects the momentum measurement
compared with multiple scattering is sufficient for physics,
a charged particle position resolution
better than 130~$\mu$m and a corresponding charged
track momentum resolution of $\sigma_{p}/p=0.5\%$ at $p=1$~GeV/c are needed.
Possible contribution from multiple scattering in the momentum measurement will be discussed in Sec.~\ref{sec:mdc}.



\begin{figure}[htbp]
\begin{center}
\begin{overpic}[width=7.5cm, height=5cm, angle=0]{Figures/Figs_01_PhysicalRequirement/DE.pdf}
\put(20,50){\small{(a)}}
\end{overpic}
\begin{overpic}[width=7.5cm, height=5cm, angle=0]{Figures/Figs_01_PhysicalRequirement/Mbc.pdf}
\put(20,50){\small{(b)}}
\end{overpic}
\begin{overpic}[width=7.cm, height=5cm, angle=0]{Figures/Figs_01_PhysicalRequirement/Momentum.pdf}
\put(20,50){\small{(c)}}
\end{overpic}
\begin{overpic}[width=7.cm, height=5cm, angle=0]{Figures/Figs_01_PhysicalRequirement/mom_scale.jpg}
\put(20,50){\small{(d)}}
\end{overpic}
\end{center}
\caption{
The distribution of (a) $\Delta E$ and (b) $M_{BC}$ associated with the spatial resolution of the track system
in process $e^{+}e^{-}\to D^{0}\bar{D}^{0}$ at $\sqrt{s}=3.77$~GeV with $D^0\to K^-\pi^+$.
The different colored lines represent different spatial resolutions. (c) Detection efficiency with different
charged track position resolutions in the $J/\psi\to\Lambda\bar{\Lambda}$ process. (d) 2D scattering plot of position resolution versus momentum resolution in the $J/\psi\to\Lambda\bar{\Lambda}$ process. }
    \label{D0}
\end{figure}


\subsection{Photons}

Photons are one of the most important particles in the final states at the STCF, and they are involved in many physics programs,
as shown in Fig.~\ref{pr:phoene}.
The energy~($E$) of photons can be as high as
3.5~GeV, although multiplicity is rare when the energy is $E>2$~GeV.
For example, the $e^{+}e^{-}\to\gamma\gamma$ process with energy of photon equal to beam energy 
is essential for the luminosity measurement.
In addition, the photon energy can be as low as dozens of MeV, such as $\psi(3686)\to\gamma\eta_{c}(2S)$ and $D_{s}^{*}\to\gamma D_{s}$. The requires the detector
to be able to cover a large energy range with high efficiency.
The energy coverage of photon detection is therefore required to be from 25~MeV to 3.5~GeV.
The energy and position resolution
are two key parameters for photon detection.
Besides, to distinguish the neutral tracks and suppress the noise photons, a time resolution of hundred ps is required for the 
calorimeter system, which will be further discussed in Sec.~\ref{neutralhadrons}.

\begin{figure}[htbp]
\begin{center}
\begin{overpic}[width=9cm, height=7cm, angle=0]{Figures/Figs_01_PhysicalRequirement/plot_emc_photonv2.pdf}
\end{overpic}
\caption{ Energy distribution of photons at the truth level, normalized to 10k entries.}
\label{pr:phoene}
\end{center}
\end{figure}

The resolution of photon detection is crucial for the reconstructed mass spectra of various particles containing photons,
such as $\pi^{0}$, where the energy of most $\pi^{0}$ is located below 1.5~GeV at the STCF.
The evolution of the $\pi^{0}$ mass spectra is studied with various
photon energy/position resolutions.
The root mean square (RMS) value of $M_{\pi^0}$ shows that for low momentum $\pi^0$, the invariant mass resolution is significantly improved with
a finer photon energy resolution,
while for high momentum $\pi^0$, the mass resolution of $\pi^{0}$ is significantly improved with
finer photon position resolution, as shown in Fig.~\ref{mpi0}.
Similar conclusions can be drawn in the study of other mass spectra.



\begin{figure}[htbp]
\begin{center}
\begin{overpic}[width=7.5cm, height=5cm, angle=0]{Figures/Figs_01_PhysicalRequirement/Pi0E.pdf}
\put(26,58){\small{(a)}}
\end{overpic}
\begin{overpic}[width=7.5cm, height=5cm, angle=0]{Figures/Figs_01_PhysicalRequirement/Pi0P.pdf}
\put(26,58){\small{(b)}}
\end{overpic}
\end{center}
\vskip -0.6 cm
\caption{ The RMS of $M_{\pi^0}$ versus the momentum of $\pi^{0}$ under different (a) energy resolution $\Delta E/\sqrt{E}$ and
(b) spatial resolution $\sigma_z$ when $E=1$~GeV. }
    \label{mpi0}
\end{figure}


In the charged lepton-flavor-violation~(cLFV) process $\tau\to\gamma \mu$, the signals are distinguished from
the background by constructing the mass and energy distribution of the $\gamma\mu$ system, where the reconstruction of photons
is the main source of the resolution of these spectra.
%photon is the dominant source on the resolutions.
Moreover, one of the dominant background signals comes from the misidentification of $\gamma$ from $\pi^{0}$.
A good energy and position resolution can help to improve the resolution of signals and hence the signal-to-background ratio.
According to Fig.~\ref{pr:phoene}, the energy of photons in $\tau\to\gamma\mu$ ranges from 0.5 to 1.7 GeV, a range in which the position resolution of photons matters more.
Different sets of detector responses are applied in the estimation of experimental sensitivity in $\tau\to\gamma\mu$,
with the energy and position resolutions of photons varying from 2.0\% to 2.5\% and from 3~mm to 6~mm, respectively.
It is found that to obtain the ability to probe
the cLFV process $\tau\to\gamma\mu$ with an upper limit of better than $10^{-8}$ at the 90\% confidence level~(C.L. ),
the energy and position resolutions of photons with 1~GeV energy are required to be better than 2.5\% and
5~mm, respectively.
It should be noted that, though the detector is required to reconstruct photons with energy up to 3.5~GeV, the energy resolution is not that strict since the $e^{+}e^{-}\to\gamma\gamma$ process can be easily distinguished
from hadronic backgrounds
with a proper energy window and back-to-back angle requirements. 



A good position resolution is also important in the detection of other neutral particles.
In process $J/\psi\to\Lambda\bar{\Lambda}$ with $\Lambda\to p\pi^{-}$
and $\bar{\Lambda}\to\bar{n}\pi^{0}$, the dominant background comes from $J/\psi\to\Lambda\Sigma^{0}$ with an additional final state photon.
The various position resolutions can affect the mass window for the reconstructed
$\bar{\Lambda}$. The simulation result shows that with a better position resolution, the resolution
of $M_{\bar{\Lambda}}$ can be improved significantly.
The efficiency for reconstructing the process is increased by 9.3\% when the position resolution is
improved by a factor of 20\%, {\it i.e.}, from 6~mm to 5~mm in the photon case.

\subsection{Neutral Hadrons}
\label{neutralhadrons}
Apart from photons, other neutral particles, such as neutrons and $K_{L}$, are also involved in many interesting physics programs,
as shown in Fig.~\ref{pr:neuhad}:
the studies of neutrons are important to understanding their internal structures and improving
knowledge for hyperon decays containing neutrons; $D^{0}\to K_{L}\pi^{+}\pi^{-}$
is one of the benchmark processes for the study of $D^{0}-\bar{D}^{0}$ mixing
and $CPV$ by means of the quantum coherence of $D^{0}$ and $\bar{D}^{0}$ production.

\begin{figure}[htbp]
\begin{center}
\begin{overpic}[width=9cm, height=7cm, angle=0]{Figures/Figs_01_PhysicalRequirement/plot_emc_neutral.pdf}
\end{overpic}
\caption{ Momentum distribution of neutral particles
($K_{L}/n$) from various physics processes, which is depicted at the truth level and normalized to 10k entries.}
\label{pr:neuhad}
\end{center}
\end{figure}


It is essential to have a good ability to separate neutral hadrons and photons
since the latter are the dominant background sources for the former.
For example, in processes $e^{+}e^{-}\to n\bar{n}$ and $D_{0}\to K_{L}\pi^{+}\pi^{-}$, the dominant backgrounds come from the
photon contamination in
$e^{+}e^{-}\to\gamma\gamma$ and $D_{0}\to K_{S}\pi^{+}\pi^{-}$ with $K_{S}\to\pi^{0}\pi^{0}$, respectively.
Due to the electrically neutral nature of these hadrons and that they can deposit only part of their energy into the calorimeter, distinguishing neutral hadrons from photons is difficult.
According to the demand, more information from the detector is needed for the identification of neutral hadrons.
In fact, a time-of-flight difference can serve as an effective way to help distinguish neutrons and $K_{L}$ from
photons.
Figure~\ref{time} shows the expected time resolution for the separation of neutrons/$K_{L}$ from photons
with a power of $3\sigma$ with respect to their momentum for a flight length of 1.5~m.
We therefore propose a momentum-dependent time resolution for neutral hadrons,
{\it i.e.}, $\sigma_{T} = 300/\sqrt{p^{3}\rm (GeV^{3})}~\rm{ps}$, that fulfills $3\sigma$ $\gamma/n$ separation
and an approximately $2\sigma$ $\gamma/K_{L}$ separation.
Under these requirements, the physics goals of measuring mixing parameters of $D^{0}-\bar{D}^{0}$ with a precision better than 0.05\%,
and electric-magnetic neutron form factors with a precision better than 1\% can be achieved.


\begin{figure}[hbtp]
\begin{center}
\begin{overpic}[width=9cm, height=6cm, angle=0]{Figures/Figs_01_PhysicalRequirement/com_texp_v2.pdf}
\end{overpic}
\end{center}
\caption{Expected time resolution ($\Delta T$) for distinguishing neutral particles, neutrons/$K_{L}$ from photons with
a separation power of $3\sigma$ versus their incident momenta for a flight length of 1.5~m. }
\label{time}
\end{figure}



\subsection{Particle Identification}

\subsubsection{ $\pi/K$ Identification}

The identification of $\pi$ and $K$ at the STCF is essential for charm physics, $\tau$ physics 
and fragmentation function~(FF) studies etc.
The momentum region of $\pi/K$ produced from charm hadrons or $\tau$ leptons are within 1.5~GeV
and that from FF studies can be as high as 3.5~GeV. 
The quark-hadron FF is essential for understanding the formation of observed
hadrons from QCD partons, while an $e^{+}e^{-}$ collider
provides a clean place for the study of hadronization.
Both the unpolarized and polarized FFs can be measured at the STCF by inclusive production
of one or two hadrons. Since a wide momentum range of FFs with high precision
are necessary at the STCF, the STCF must have the ability to identify
all kinds of final state hadrons with excellent separation power.


Taken the study of FFs as an example,
in the study of the Collins FF via $e^{+}e^{-}\to KK+X$ at $\sqrt{s}=7$~GeV, the dominant background
comes from $\pi/K$ contamination.
The momentum spectra of $\pi$ and $K$ can be up to 3.5~GeV/c, with most events accumulating less than 2.0~GeV/c according to Fig.~\ref{pr:chgmom},
which requires good $\pi/K$ identification abilities with momentum up to at least 2.0~GeV.
In addition, the yields of $K$ are at least one magnitude lower than that of $\pi$ in the production, leading to a worse situation in the study of $e^{+}e^{-}\to KK+X$ due to $\pi$ contamination.
It is found that background contamination affects the asymmetry distributions of
final hadrons, which leads to an underestimation of the asymmetry.
The background effects are studied under two situations:
first with $\pi/K$ misidentification of 10\% at $p=2$~GeV/c, where the background level from
$\pi$ contamination in $e^{+}e^{-}\to KK+X$ is over 50\%, and second
with $\pi/K$ misidentification of $<2$\% at $p=2$~GeV/c, where
the background level from $\pi$ contamination is $4\%$.
Because the precision of the Collins effect must be better than
7\% for spin structure measurements in electron-ion colliders,
the $\pi/K$ misidentification must
be less than $2\%$ at $p=2.0$~GeV/c and the corresponding identification efficiencies for the hadrons must be over 97\% at the STCF.\\


\subsubsection{ $\mu/\pi$ Identification}
In electron-positron collider experiments, the identification of muons is of great importance
for a wide range of physics program involving XYZ physics, $\tau$ physics and (semi)leptonic decays of
charm mesons, rare decays {\it etc. }
According to Fig.~\ref{pr:chgmom}, muon identification with $p<2$~GeV is essential.
A good $\mu$ detection efficiency and suppression power of $\mu/\pi$ are required at the STCF.


Two benchmark processes are applied to study the requirement for $\mu/\pi$ separation at high energy, {\it e.g.}, $p>0.5$~GeV/c,
the $CPV$ of $\tau$ decay $\tau^{-}\to K_{s}\pi^{-}\nu_{\tau}$
with another $\tau^{+}\to \mu^{+}\nu_{\mu}\bar{\nu}_{\tau}$, and the pure leptonic decay of $D_{s}$ decay $D_{s}\to\mu\nu_{\mu}$.
Three values for the misidentification rate of $\mu/\pi$, 1\%, 1.6\%, and 3\% at $p=1$~GeV/c, are tested;
these values correspond to muon identification efficiencies of 85\%, 92\% and 97\%, respectively.
It is found that higher $\mu$-ID efficiency is favored under the low background level.
Therefore, a $\mu/\pi$ suppression power of 30 up to a momentum of 2~GeV/c with
a high identification efficiency for muon to be 95\% with momentum $p>1$~GeV/c are needed,
which can meet the physics goal of achieving a sensitivity of 0.15\% for the measurement of CKM matrix element $|V_{cs}|$ in $D_{s}\to\mu\nu_{\mu}$.
In addition, in the semileptonic decays of $D_{(s)}$, such as the $D_{(s)}\to\pi^{-}\mu^{+}\nu_{\mu}$ process,
low-momentum $\mu/\pi$ separation is essential in the background
estimation for the purity of the selected sample; 
thus, good $\mu/\pi$ separation at low momentum is needed.





\subsection{Summary of the Physics Requirements}
From the discussions illustrated above, the quantified requirements from the physics perspective are listed below and summarized in Table~\ref{phyreqv2}.
\begin{itemize}
\item Tracking: An excellent tracking efficiency better than 99\% is required for charged tracks
in the high $p_{T}$ region, {\it i.e.}, $p_{T}>0.3$~GeV/c, and good tracking efficiency at low momentum is needed, {\it i.e.}, larger than $90\%$
at $p_{T}=0.1$~GeV/c.
Good momentum resolution for charged tracks is needed, {\it, e.g.}, $\sigma_{p}/p=0.5\%$ at $p=1$~GeV/c,
where the position resolution provided by the tracking system should be better than 130~$\mu$m.
The magnetic field for the current machine running in the tau-charm region is set at 1~T.
However, to obtain a better detection efficiency for low-momentum tracks and
good momentum resolution for all charged tracks, the mean value
and uniformity of the magnetic field should be optimized further.


\item Particle identification:
The $\pi/K$ or $K/\pi$ misidentification rate must be less than 2\% at $p=2$~GeV/c with the corresponding PID efficiency for hadrons to be over 97\%.
A $\mu/\pi$ suppression power of 30 up to a momentum of 2~GeV/c is proposed, and a
high PID efficiency for muons is needed, {\it i.e.}, larger than 95\% at $p=1$~GeV/c.



\item Photons: The photons need to be detected in the EMC within a wide energy range, from $E=25$~MeV to 3.5~GeV.
A good energy resolution is needed, {\it i.e.}, $\sigma_{E}\approx2.5\%$ at $E=1$~GeV, and the position
resolution is required to be $\sigma_{\rm pos}\approx5$~mm at $E=1$~GeV.
Moreover, the granularity of the detector affects the identification of
high energy $\pi^{0}$, where the opening angle of two photons
from $\pi^{0}$ decay is small, which should be considered in the design of the EMC.


\item Other neutral particles: a momentum-dependent time resolution is required for $\gamma/n/K_{L}$ identification, {\it i.e.}, $\sigma_{T} = 300/\sqrt{p^{3}\rm (GeV^{3})}~\rm{ps}$.


\end{itemize}


\begin{table}[htbp]
\caption{Benchmark physics processes used to determine the physics requirements of the STCF detector.}
\label{phyreqv2}
\footnotesize
\begin{center}
\begin{spacing}{1.3}
\begin{tabular}{cccc}
\hline
\hline
\vspace{0.2cm}
\multirow{2}{*}{Process}  &   \multirow{2}{*}{Physics Interest} & Optimized  & \multirow{2}{*}{Requirements}  \\
                                  &                                      & Subdetector & \\
\hline
 $\tau\to K_{s}\pi\nu_{\tau}$,    &    CPV in the $\tau$ sector,  &   \multirow{3}*{ITK+MDC}  &  acceptance: 93\% of $4\pi$; trk. effi.:  \\
 $J/\psi\to\Lambda\bar{\Lambda}$, &    CPV in the hyperon sector,  &                          &   $>99\%$ at $p_{T}>0.3$~GeV/c; $>90$\% at $p_{T}=0.1$~GeV/c\\
 $D_{(s)}$ tag                          &    Charm physics     &                          &   $\sigma_{p}/p=0.5\%$, $\sigma_{\gamma\phi}=130~\mu$m at 1~GeV/c     \\
\hline
$e^{+}e^{-}\to KK+X$,            &    Fragmentation function,  & \multirow{2}*{PID}   &      $\pi/K$ and $K/\pi$ misidentification rate $<2\%$ \\
$D_{(s)}$ decays                 &    CKM matrix, LQCD etc. &                      &          ~~~~~~~~~~~~~~~~~PID efficiency of hadrons $>97\%$ at $p<2$~GeV/c   \\
%$D_{0}\to K_{1} e\mu$			  &    photon polarization   &                      &        \\
\hline
 $\tau\to \mu\mu\mu$, $\tau\to\gamma\mu$,             &   cLFV decay of $\tau$,  & \multirow{2}*{PID+MUD}  &   $\mu/\pi$ suppression power over 30 at $p<2$~GeV/c, \\
 $D_{s}\to\mu\nu$                  &   CKM matrix, LQCD etc. &                          &   $\mu$ efficiency over 95\% at $p=1$~GeV/c    \\
\hline
$\tau\to\gamma\mu$,                 &   cLFV decay of $\tau$,  & \multirow{2}*{EMC}     &    $\sigma_{E}/E\approx2.5\%$ at $E=1$~GeV    \\
$\psi(3686)\to\gamma\eta(2S)$     &   Charmonium transition    &                        &    $\sigma_{\rm pos}\approx 5$~mm at $E=1$~GeV\\
\hline
$e^{+}e^{-}\to n\bar{n}$,           &    Nucleon structure     & \multirow{2}*{EMC+MUD}  &   \multirow{2}*{ $\sigma_{T} = \frac{300}{\sqrt{p^{3}\rm (GeV^{3}})}~\rm{ps}$ }   \\
$D_{0}\to K_{L}\pi^{+}\pi^{-}$     &     Unity of CKM triangle      \\
\hline 
\hline
\end{tabular}
\end{spacing}
\end{center}
\end{table}


\newpage
