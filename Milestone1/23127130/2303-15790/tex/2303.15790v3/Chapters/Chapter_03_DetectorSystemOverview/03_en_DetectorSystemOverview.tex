\section{Detector Design Overview}


\subsection{General Considerations}
The STCF detector system has to detect and identify particles  in a large kinematic phase space and a high radiation and counting-rate environment
induced by the high luminosity,
as described in Sec.~\ref{sec:mdi_bkg}.
Adequate radiation resistance and fast response is required for the STCF detector, especially for the inner part of the detector. Data would be taken at a rate 2-3 orders of magnitude higher than that of the BEPCII. This places stringent demands on the performance of the trigger and data acquisition system.
 
For many benchmark physics studies at the STCF, as described in Chapter~\ref{chap_phyper}, systematic errors will be the dominant factor limiting the measurement precision and may come from
\begin{itemize}
\item uncertainty of detector acceptance and response, including uncertainties of the geometrical acceptance and detector efficiencies and non-linearity of detector response;
\item mismeasurements by detectors, such as misreconstructed tracks and photons, and particle misidentification;  
\item uncertainty in the luminosity measurement, energy calibration, trigger etc.
\end{itemize}

To achieve optimal detector performance for precise reconstruction of exclusive final states produced at the STCF, the general requirements for the STCF detector system include the following
\begin{itemize}
\item (nearly) $4\pi$ solid angle coverage for both charged and neutral particles and a uniform response for all particles;
\item excellent momentum and angular resolution for charged particles and high energy and position resolution for photons;
\item high reconstruction efficiency for low momentum/energy particles;
\item superior PID capability ($e/\mu/\pi/K/p/\gamma$ and other neutral particles);
\item precise luminosity measurement;
\item radiation hardness and high rate capability in the high radiation and counting-rate environment expected at the STCF.
\end{itemize}

Since the momenta of most final-state particles are below 1~GeV/c, a low-mass design is required for the tracking system, especially for the very low momentum region, where the multiple scattering effect dominates. A separate inner tracker is used in the tracking system to enhance tracking performance for low momentum particles. 
A low-mass tracking system would also greatly benefit the energy measurement of low-energy photons by the EMC.  Fast response is required for the crystal-based EMC to preserve its excellent intrinsic energy resolution. 
The extra high radiation level in the inner and forward regions of the STCF experiment demands detector and electronics technologies with significantly high radiation resistance and rate capability. 
Powerful trigger and DAQ systems are required to handle the very high physics event rate up to 400 kHz and the large data flow expected at the STCF. 



\subsection{Overall Detector Concept}

The conceptual layout of the STCF detector system is shown in Fig.~\ref{fig:fulldetector_overview}.
The 2D and 3D detector geometries are shown in Fig.~\ref{fig:fulldetector_geo}.
Along the radial direction from the interaction region, the major detector components are as follows:
\begin{itemize}
\item an ITK consisting of three layers of low-material budget silicon or gaseous detectors, closest to the beam pipe and covering a polar angle range of 20 degrees to 160 degrees, to achieve high tracking efficiency for very low-momentum charged particles;
\item an MDC tracking detector based on He-gas to provide efficient and precise trajectory measurements for charged particles;
\item a RICH detector for PID in the barrel to distinguish charged hadrons at high momentum;
\item a time-of-flight detector based on the detection of the internal total-reflected Cherenkov light, DTOF, for PID in the endcap;
\item a homogeneous EMC composed of trapezoid-shaped pure CsI crystal scintillators to precisely determine the photon energy;
\item a superconducting solenoid outside the EMC to produce a uniform and stable magnetic field of 1~T;
\item a multilayer flux return yoke instrumented with plastic scintillator strips and resistive plate chambers (RPCs) to serve as a MUD to provide sufficient $\mu/\pi$ suppression power.
\end{itemize}

The tracking system consists of two components, the ITK and MDC, to cope with the high radiation level of the tracking layer closest to the beam pipe and to reduce the material budget as much as possible. The PID system is also split into two parts, the RICH detector in the barrel and the DTOF detector in the endcap, to take into account the different times of flight of particles with different polar angles.

\begin{figure}[htbp]
\begin{center}
\includegraphics[width=0.8\textwidth]{Figures/Figs_03_DetectorSystemOverview/STCF_design_concept.pdf}
\caption{Schematic layout of the STCF detector concept}
\label{fig:fulldetector_overview}
\end{center}
\end{figure}

\begin{figure}[htbp]
\begin{center}
\subfloat[][]{\includegraphics[width=0.8\textwidth]{Figures/Figs_03_DetectorSystemOverview/geometry/STCF_3D_2.png}} \\
\subfloat[][]{\includegraphics[width=0.42\textwidth]{Figures/Figs_03_DetectorSystemOverview/geometry/STCF_XY.png}}
\hspace{5 mm}
\subfloat[][]{\includegraphics[width=0.42\textwidth]{Figures/Figs_03_DetectorSystemOverview/geometry/STCF_ZR.png}}
\caption{Geometry of the STCF detector: (a) 3D cutaway view, (b) cross-section view in the $x-y$ plane, and (c) cross-section view in the $z-r$ plane. It consists of ITK, PID system, EMC, SCS and MUD from inner to outmost. }
\label{fig:fulldetector_geo}
\end{center}
\end{figure}


The primary performance requirements for the STCF detector have been presented in Table~\ref{phyreqv2} and are listed below:
\begin{itemize}
\item Low-momentum tracking efficiency $>90\% @ 100$~MeV/c, low material budget ($\lt0.01X_{0}$) for ITK;
\item Momentum resolution $\lt0.5\% @1~$GeV/c, $dE/dx$ resolution $\lt6\%$, and low material budget ($\lt0.05X_{0}$) for MDC.
\item PID $\pi/K$ misidentification rate $<2\%$ and PID efficiency $>97\%$ up to 2~GeV/c with a modest material budget ($\lt0.3X_{0}$)
\item EMC energy resolution $\sim 2.5\% @ 1$~GeV, position resolution $\sim 5$~mm @ 1~GeV;
\item MUD $\mu/\pi$ suppression power $>30$, with $\mu$ detection efficiency $\gt$ 70\% @ $0.5<p<0.7$~GeV/c, and $\mu$ detection efficiency $\gt$ 95\% @ p $\gt$ 0.7 GeV/c.
\end{itemize}

\begin{comment}
\begin{table}[htbp]
\caption{Main physics requirements on the STCF detector.}
\label{phyreq_det}
\footnotesize
\begin{center}
\begin{spacing}{1.3}
\begin{tabular}{lc}
\hline
\hline
\vspace{0.2cm}

Subdetector  & Requirements \\
\hline
\multirow{3}*{ITK+MDC}        &  acceptance: 93\% of $4\pi$; trk. effi.:  \\
                              &   $>99\%$ at $p_{T}>0.3$~GeV/c; $>90$\% at $p_{T}=0.1$~GeV/c\\
                              &   $\sigma_{p}/p=0.5\%$, $\sigma_{\gamma\phi}=130~\mu$m at $p=1$~GeV/c     \\
\hline
\multirow{2}*{RICH+DTOF}      &  $\pi/K$ and $K/\pi$ misidentification rate $<2\%$ \\
                              &   PID efficiency for hadrons $>97\%$ at $p<2$~GeV/c   \\
\hline
\multirow{4}*{EMC}            &  $\sigma_{E}/E\approx2.5\%$ at $E=1$~GeV     \\
                              &   $\sigma_{\rm pos}\approx 5$~mm at $E=1$~GeV \\
                              &   \multirow{2}*{ $\sigma_{T} = \frac{300}{\sqrt{p^{3}\rm (GeV^{3}})}~\rm{ps}$ }   \\
                              & \\
\hline
\multirow{2}*{MUD}            &  $\mu$ efficiency over 95\% at $p=1$~GeV/c    \\
                              &  $\mu/\pi$ suppression power over 30 at $p<2$~GeV/c \\
                              & \\

\hline 
\hline
\end{tabular}
\end{spacing}
\end{center}
\end{table}
\end{comment}
\newpage
