\subsubsection{Inclusive Production of Single Hadron}

For large enough $\sqrt{s}$ the inclusive production of single hadron in $e^+e^- \to h +X$ can be predicted from QCD
with QCD factorization theorem\cite{Book}:
\begin{eqnarray}
\frac{d\sigma (e^+ e^- \rightarrow h +X)} {d z}  &=& \sum_{a=q,\bar q ,g} \int \frac{d\xi}{\xi} H_a (\frac{z}{\xi},Q^2,\mu^2) D_{a\rightarrow h} (\xi,\mu^2)
\nonumber\\
 &=& \sum_q \sigma (e^+ e^- \rightarrow q\bar q) \biggr ( D_{q\rightarrow h} (z) +   D_{\bar q\rightarrow h} (z) \biggr ) + {\mathcal O}(\alpha_s),
\label{FF}
\end{eqnarray}
where $z$ is the fraction of the energy carried by the observed hadron $h$. $H_a (a=q,\bar q,g)$ are functions which can be calculated
with perturbation theory. $D_{a\rightarrow h}$'s are parton fragmentation functions describing hadronization of a parton $a$ to $h$.
Eg.(\ref{FF}) is the statement from QCD collinear factorization.
The fragmentation functions are universal for any process where the QCD factorization is applicable. Extracting fragmentation functions
at rather low energy like the energy region of STCF around 4-5~GeV is specially important for the information about hadronization, because from the extracted fragmentation
functions one can test their energy evolution from a rather low energy scale to high energy scales.
