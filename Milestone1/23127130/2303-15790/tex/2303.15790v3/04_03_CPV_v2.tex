\subsection{$CP$ symmetry tests}


How $CP$ symmetry is broken may hold the key to why our universe contains more matter than antimatter. The violation of $CP$ symmetry is one of the required conditions to understand this. There is insufficient $CP$ violation in the SM to explain this fundamental question affecting our very existence in the Universe, and therefore, new sources of $CP$ violation are demanded. The search for new $CP$-violating effects is one of the most active areas in particle physics. Physical processes involving the $\tau$ lepton are potential sectors in which new $CP$-violating effects may appear.

\subsubsection{$CP$ violation in $\tau^- \to K^0_S \pi^- \nu_\tau$}

In the SM, because of the $CP$ violation in $K^0$--$\bar K^0$ mixing, a detectable $CP$-violating effect is predicted for this process~\cite{Bigi:2005ts1, Bigi:2005ts2}:
\begin{eqnarray}
A_Q = {B(\tau^+ \to K^0_S \pi^+ \bar \nu_\tau) - B(\tau^- \to K^0_S \pi^- \nu_\tau) \over
B(\tau^+ \to K^0_S \pi^+ \bar \nu_\tau) + B(\tau^- \to K^0_S \pi^- \nu_\tau)} = (+0.36\pm 0.01)\%\;.
\end{eqnarray}
While Belle observed no $CP$ violation in the angular distributions for the exclusive decays~\cite{Bischofberger:2011pw},
BaBar yielded a value of $A_Q=(-0.36\pm0.23\pm0.11)\%$ for the inclusive decays with $\ge 0\pi^0$ in the final states~\cite{BABAR:2011aa}, which is $2.8\sigma$ away from the SM prediction.

The above deviation represents a challenge to the SM. Theoretical efforts have been made to reconcile this deviation. However, even with beyond-the-SM effects included, it is not easy to obtain the central value of the BaBar data. The STCF can provide a crucial check with a large number of $\tau^+\tau^-$ pairs produced not far from the threshold, where the background can be well controlled. At the STCF, the expected luminosity of 1~ab$^{-1}$/year at an energy of 4.26~GeV can allow a statistical sensitivity of $9.7\times 10^{-4}$ to $CP$ violation to be reached. With 10 years of operation, the sensitivity can reach $3.1\times 10^{-4}$~\cite{Sang:2020ksa}, which will be comparable to the sensitivity of $10^{-4}$ projected for Belle II with a luminosity of 50~ab$^{-1}$~\cite{Chen:2020uxi}. The STCF can thus provide crucial information for resolving the $A_Q$ discrepancy.

\subsubsection{Measurement of the electric dipole moment of the $\tau$}

The initial state of an $e^+e^-$ pair in the center-of-mass system is a $CP$ eigenstate. Therefore, $CP$ tests can be conveniently performed at any $e^+e^-$ collider. By measuring the decay products from $\tau$ decays, a $CP$ test can be conducted based on the $e^+e^-\to  \tau^+\tau^-$ process, as suggested in~\cite{CPTau1, CPTau2}. By measuring $CP$-odd observables, one can determine the electric and weak dipole moments of the $\tau$. In the SM, these moments are predicted to be extremely small (for example, the electric dipole moment is expected to be on the order of $10^{-34}$~e~cm). If either of the two moments is nonzero at a level much larger than the SM predictions, it will be a clear signal of new physics beyond the SM. These two moments have been studied at LEP and $B$ factories. While the weak dipole moment is suppressed at low energy by the large masses of the weak gauge bosons, the electric dipole moment $d^\gamma_\tau$ can be probed at $B$ and $\tau$--charm factories. The newest result for the electric dipole moment obtained from the Belle experiment~\cite{TauBelle}, in units of $10^{-16}$~e~cm, is
\begin{equation}
-0.22 < {\rm Re} (d^\gamma_\tau ) < 0.45, \ \ \ \ -0.25 < {\rm Im} ( d^\gamma_\tau) < 0.08.
\end{equation}
These bounds can be tightened by 2 or 3 orders of magnitude through experiments at the STCF.

\subsubsection{CP violation with polarized beams}

With polarized $e^+$ and/or $e^-$ beams, highly polarized $\tau^\pm$s can be produced. $\tau$ polarizations normal ($N$) to their production plane can be measured by studying semileptonic decays of the form $\tau^\pm\to\pi^\pm/\rho^\pm\bar\nu_\tau(\nu_\tau)$. One can then construct asymmetry observables with respect to the left-hand ($L$) and right-hand ($R$) sides of the plane, which are directly related to the electric dipole moment of the $\tau^\pm$~\cite{bernabeu-cp}:
\begin{eqnarray}
A^\pm_N = {\sigma_L^\pm - \sigma_R^\pm\over \sigma} = \alpha_\pm {3\pi \beta\over 8 (3-\beta^2)}{2m_\tau \over e} \textrm{Re}(d^\gamma_\tau)\;,
\end{eqnarray}
where $\sigma$ is the cross section for $e^+e^- \to \tau^+\tau^- \to (\pi^+/\rho^+)\bar\nu_\tau (\pi^- /\rho^-)\nu_\tau$, $\beta = \sqrt{1-4m^2_\tau/s}$, and $\alpha_\pm$ is the polarization analyzer in the $\tau^\pm\to\pi^\pm/\rho^\pm\bar\nu_\tau(\nu_\tau)$ decays. Belle II can reach a sensitivity of $3\times 10^{-19}$~e~cm with an integrated luminosity of $50~\textrm{ab}^{-1}$. At the STCF, the sensitivity can be improved by a factor of approximately 30, reaching $10^{-20}$~e~cm.

With polarized $e^+$ and $e^-$ beams, one can also construct new $T$-odd observables to measure $CP$-violating effects. An interesting observable is the triple product $P^{\tau^{\pm}}_z\hat z\cdot(\vec p_{\pi^\pm}\times\vec p_{\pi^0})$ from measuring the two pion momenta in the $\tau^\pm\to\pi^\pm\pi^0\bar\nu_\tau(\nu_\tau)$ decays~\cite{paultsai}.
Here, $P^\tau_z = [(w_{e^-} + w_{e^+})/(1+w_{e^+}w_{e^-})][(1+2a)/(2+a^2)]$ is the component of the polarization vector of the $\tau$ obtained upon averaging over its momentum direction, with $w_{e^\pm}$ being the components of the polarization vectors of the $e^\pm$ in the $e^-$ beam direction $\hat z$ and $a = 2m_\tau/\sqrt{s}$. If the difference in the triple products for $\tau^+$ and $\tau^-$ is nonzero, this will be a signal of $CP$ violation. Since the SM predicts very small values for the triple products, the measurement of a nonzero difference would already signal new physics beyond the SM. This measurement can be done at the STCF to provide new information about sources of $CP$ violation. Similar measurements can be carried out by replacing $\pi^\pm$ with $K^\pm$.
