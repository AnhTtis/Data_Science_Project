\subsection{Limitations of current experiments}

\begin{table}[tb]
%\small
  \caption{\label{tab:XYZ} Some of the $XYZ$ states in the charmonium mass region as well as the observed production processes and decay modes. For the complete list and more detailed information, we refer to the latest version of the Review of Particle Physics (RPP)~\cite{ParticleDataGroup:2022pth}. }
\centering
% \begin{tabularx}{\textwidth}{l c c c }
%\begin{ruledtabular}
\begin{tabular}{l c c c}
% \hline
     $XYZ$ & $I^G(J^{PC})$ & Production processes
            & Decay modes % & Experiments
                 \\\hline
           \multirow{2}{*}{$X(3872)$} & \multirow{2}{*}{$0^+(1^{++})$} & $B \to K X/ K\pi X$, $e^+e^-\to\gamma X$,  & $\pi^+\pi^-J/\psi$, $\omega J/\psi$, $D^{*0}\bar{D}^0$, \\
           & &  $pp/p\bar p$ inclusive, PbPb, $\gamma\gamma^*$ &  $\gamma J/\psi$, $\gamma \psi(3686)$
                %\multirow{4}{*}{$X(3872)$}
                    \\ \hline
    $X(3915)$ & $0^+(0~\text{or}~2^{++})$ & $B \to K X$, $\gamma\gamma\to X$ & $\omega J/\psi$ \\ \hline
    $X(4140)$ & $0^+(1^{++})$ & $B \to K X$, $p\bar p$ inclusive & \multirow{4}{*}{$\phi J/\psi$} \\
    $X(4274)$ & $0^+(1^{++})$ & \multirow{3}{*}{$B \to K X$} & \\
    $X(4500)$ & $0^+(0^{++})$ & & \\
    $X(4700)$ & $0^+(0^{++})$ & & \\ \hline
   $X(3940)$ & $?^?(?^{??})$ &  \multirow{2}{*}{$e^+e^- \to J/\psi + X$} & $D {\bar D}^*$\\
  $X(4160)$ & $?^?(?^{??})$ &      & $D^* {\bar D}^*$\\ \hline
  $X(4350)$ & $0^+(?^{?+})$ & $\gamma\gamma \to X$ & $\phi J/\psi$\\ \hline\hline
    $Y(4008)$ & $0^-(1^{--})$  &  $e^+e^- \to Y$  & $\pi\pi J/\psi$ \\ %\hline
  $Y(4260)$ & $0^-(1^{--})$ & $e^+e^- \to Y$ & $\pi\pi J/\psi$, $D\bar D^*\pi$,$\chi_{c0}\omega$, $h_c\pi\pi$ \\ \hline
   $Y(4360)$ & $0^-(1^{--})$ &  \multirow{2}{*}{$e^+e^- \to Y$} & $\pi \pi \psi(3686)$ \\
  $Y(4660)$ & $0^-(1^{--})$ & & $\pi \pi \psi(3686)$, $\Lambda_c\bar \Lambda_c$ \\ \hline\hline
  $Z_c(3900)$ & $1^+(1^{+-})$ & $e^+e^-\to \pi Z_c$, inclusive $b$-hadron decays & $\pi J/\psi$, $D\bar D^*$ \\
  $Z_c(4020)$ & $1^+(?^{?-})$ & $e^+e^-\to \pi Z_c$  & $\pi h_c$, $D^*\bar D^*$ \\\hline
    $Z_1(4050)$ & $1^-(?^{?+})$ & \multirow{2}{*}{$B \to K  Z_c$} & \multirow{2}{*}{$\pi^\pm \chi_{c1}$}\\
   $Z_2(4250)$ & $1^-(?^{?+})$ & & \\ \hline
   $Z_c(4200)$ & $1^+(1^{+-})$ & \multirow{2}{*}{$B \to K Z_c$} & $\pi^\pm J/\psi$ \\
    $Z_c(4430)$ & $1^+(1^{+-})$ & & $\pi^\pm J/\psi$, $\pi^\pm \psi(3686)$ \\ \hline
    $Z_{cs}(3985)$ & $\frac12(?^?)$ & $e^+e^-\to K Z_{cs}$ & $\bar D_s D^*$, $\bar D_s^* D$\\
    $Z_{cs}(4000)$ & $\frac12(1^+)$ & $B^+\to \phi Z_{cs}$ & $J/\psi K$\\
    $Z_{cs}(4220)$ & $\frac12(1^+)$ & $B^+\to \phi Z_{cs}$ & $J/\psi K$
\end{tabular}
%\end{ruledtabular}
\end{table}
Most of the $XYZ$ states reported thus far, together with their observed production processes and decay modes, are listed in Table~\ref{tab:XYZ}. There are several basic types of production processes: $e^+e^-$ collisions, including the direct production and initial-state radiation (ISR) processes; $B$ decays with a kaon in the final state; $pp$ or $p\bar p$ collisions; photon--photon fusion and heavy-ion production. The first two are the main types of interest because they have cleaner backgrounds than hadron collisions and higher rates than photon--photon fusion processes. However, they need to be improved upon in the following aspects:
\begin{itemize}
\item $B\to K X$: The maximum mass of the $X$ or $Z_c$ states that can be found via this type of reaction is approximately 4.8~GeV, the mass difference between the $B$ meson and the kaon. To date, the heaviest charmonium-like state that has been observed is the $X(4700)$. Similarly, the $P_c$ states are difficult to be effectively study in $B$ decays. Additional complexity comes from the fact that these charmonium-like states were all observed as invariant mass distribution peaks in final states with two or more hadrons. Consequently, further complications arise in analyzing data coming from 1) resonances from cross channels and 2) possible triangle singularities (see Ref.~\cite{Guo:2019twa} for a review). Thus, the structures observed in the $B$ decays need to be further confirmed in other reactions, such as $e^+e^-$ collisions.
\item $e^+e^-$ collisions: Charmonia and charmonium-like states with vector quantum numbers can be easily produced either directly or via ISR processes. As a result, the $Y(4260)$ has been studied with unprecedented precision at BESIII. The heaviest vector $Y$ states is the $Y(4660)$, which is above the $\Lambda_c\bar \Lambda_c$ threshold. %which is beyond the current energy range of BES-III.
Charmonium-like states with other quantum numbers can only be produced from the decays of heavier vector states, along with the emission of pions or a photon. Thus, BESIII has observed only the $X(3872)$, $Z_c(3900)$, $Z_c(4020)$ and $Z_{cs}(3985)$ among the many nonvector states.
\end{itemize}

