%\documentclass[aps,eqsecnum,preprint,floats,epsf,epsfig,nofootinbib,letter]{revtex4}
%%\hoffset -.52in
%%\voffset 0.0in %% AS
%%\voffset 0.6in  States
%%\voffset -0.9in  %% pr, phys10
%%\voffset -0.5in    %% archive
%\textwidth 6.5in \textheight 9.0in
%\usepackage{epsfig,color}
%\usepackage{multirow}
%\usepackage{subfigure}
%
%\renewcommand{\baselinestretch}{1.00}
%
%\begin{document}
%\def\B{{\cal B}}
%\def\ov{\overline}
%\def\pr{{\sl Phys. Rev.}~}
%\def\prl{{\sl Phys. Rev. Lett.}~}
%\def\pl{{\sl Phys. Lett.}~}
%\def\np{{\sl Nucl. Phys.}~}
%\def\zp{{\sl Z. Phys.}~}
%\def\lsim{ {\ \lower-1.2pt\vbox{\hbox{\rlap{$<$}\lower5pt\vbox{\hbox{$\sim$}
%}}}\ } }
%\def\gsim{ {\ \lower-1.2pt\vbox{\hbox{\rlap{$>$}\lower5pt\vbox{\hbox{$\sim$}
%}}}\ } }
%
%\font\el=cmbx10 scaled \magstep2{\obeylines\hfill December, 2019}
%
%\vskip 1.5 cm
%
%\centerline{\large\bf Physics of Charmed Baryons}
%
%\vskip 1.5 cm
%
%\bigskip
%\bigskip
%\centerline{\bf Hai-Yang Cheng}
%\medskip
%\centerline{Institute of Physics, Academia Sinica}
%\centerline{Taipei, Taiwan 115, Republic of China}
%\medskip
%
%\bigskip
%\bigskip
%%\centerline{\bf Abstract}
%\bigskip
%\small
%
%
%
%
%\pagebreak
%
%\tableofcontents
%
%\newpage
%
%%\section{Introduction}

\subsection{Charmed baryons}

Theoretical interest in hadronic weak decays of charmed baryons peaked around the early 1990s and then faded away.
Nevertheless, there have been many progress in recent charmed baryon experiments in regard to hadronic weak decays of $\Lambda_c^+$. BESIII has played an essential role in these new developments. Motivated by the experimental progress, theoretical activity is growing in the study of hadronic weak decays of singly charmed baryons.

Charmed baryon spectroscopy provides an excellent basis for studying the dynamics of light quarks in the environment of a heavy quark. In the past decade,
many new excited charmed baryon states have been
discovered by BaBar, Belle, CLEO and LHCb. $B$ decays and the $e^+e^-\to c\bar c$ continuum are both very rich sources of charmed baryons. Many efforts have been made to identify the quantum numbers of these new states and to understand their properties.


\subsubsection{Hadronic weak decays}

Hadronic weak decays of singly charmed baryons, especially the two-body decay modes, provide essential inputs to understand the dynamics of strong interaction in the charm sector.

\begin{itemize}
\item{Nonleptonic decays of singly charmed baryons}

\noindent \underline{$\Lambda_c$ decays}
\vskip 0.2 cm
The branching fractions of the Cabibbo-allowed two-body decays of $\Lambda_c^+$ are listed in Table \ref{tab:BRs}. Many of these decays, such as $\Sigma^+\phi$, $\Xi^{(*)}K^{(*)+}$ and $\Delta ^{++}K^-$, can proceed only through $W$ exchange. Their experimental measurements imply the importance of $W$ exchange, which is not subject to color suppression in charmed baryon decays. Both Belle \cite{Zupanc} and BESIII \cite{BES:pKpi} have measured
the absolute branching fraction of the decay $\Lambda_c^+\to pK^-\pi^+$.
An average of $(6.28\pm0.32)\%$ for this benchmark mode is quoted by the PDG \cite{ParticleDataGroup:2022pth}. Furthermore,
the doubly Cabibbo-suppressed decay $\Lambda_c^+\to pK^+\pi^-$ has been observed by Belle \cite{Belle:DCS} and LHCb \cite{LHCb:DCS}. To complete the knowledge on the two-body decays, it is important to search for $pK^{0(*)}$ and $nK^{+(*)}$, which are doubly Cabibbo suppressed.

\begin{table}[hbtp]
\caption{The measured branching fractions of the Cabibbo-allowed two-body decays of $\Lambda_c^+$ (in units of \%) taken from the PDG~\cite{ParticleDataGroup:2022pth}. We have included the new BESIII measurements of $\Lambda_c^+\to \Lambda \rho^+$, $\Sigma^{*+}\pi^0$ and $\Sigma^{*0}\pi^+$~\cite{BESIII:2022udq}.} \label{tab:BRs}
\begin{center}
\begin{tabular}{lc | lc|lc}
\hline\hline ~~~Decay & $\B$ & ~~~Decay & $\B$ & ~~~Decay & $\B$  \\
\hline
~~$\Lambda^+_c\to \Lambda \pi^+$~~ & 1.30$\pm$0.07 & ~~$\Lambda^+_c\to \Lambda \rho^+$~~ & $ 4.06\pm 0.52$ & ~~$\Lambda^+_c\to \Delta^{++}K^-$  &  $1.08\pm0.25$\\
\hline
~~$\Lambda^+_c\to \Sigma^0 \pi^+$~~ & 1.29$\pm$0.07 & ~~$\Lambda^+_c\to \Sigma^0 \rho^+$  & & ~~$\Lambda^+_c\to \Sigma^{*0} \pi^+$ & $0.65\pm 0.10 $ \\
\hline
~~$\Lambda^+_c\to \Sigma^+ \pi^0$~~ & 1.25$\pm$0.10 & ~~$\Lambda^+_c\to \Sigma^+ \rho^0$~~  & $<1.7$ & ~~$\Lambda^+_c\to \Sigma^{*+} \pi^0$& $0.59\pm 0.08$\\
\hline
~~$\Lambda^+_c\to \Sigma^+ \eta$~~ & 0.44$\pm$0.20 & ~~$\Lambda^+_c\to \Sigma^+ \omega$~~  &  1.70$\pm$0.21 & ~~$\Lambda^+_c\to \Sigma^{*+}\eta$~~ & $1.05\pm0.23$ \\
\hline
~~$\Lambda^+_c\to \Sigma^+ \eta^\prime$~~ & 1.5$\pm$0.6 & ~~$\Lambda^+_c\to \Sigma^+ \phi$~~ & 0.38$\pm$0.06 & ~~$\Lambda^+_c\to \Sigma^{*+} \eta^\prime$ &\\
\hline
~~$\Lambda^+_c\to \Xi^0 K^+$~~ & 0.55$\pm$0.07 & ~~$\Lambda^+_c\to \Xi^0 K^{*+}$~~ &  & ~~$\Lambda^+_c\to \Xi^{*0}K^+$~~  & 0.43$\pm$0.09 \\
\hline
~~$\Lambda^+_c\to p K_S$~~ &  1.59$\pm$0.08 & ~~$\Lambda^+_c\to p \bar K^{*0}$~~  & 1.96$\pm$0.27 & ~~$\Lambda^+_c\to \Delta^+\bar K^0$~~ &  \\
\hline \hline
\end{tabular}
\end{center}
\end{table}

Various theoretical approaches to weak decays of heavy baryons have been investigated, including the current algebraic approach, the factorization scheme, the pole model, the relativistic quark model, the quark diagram scheme and the SU(3) flavor symmetry.
In general, the decay rates predicted by most models except the current algebraic scheme are below the experimental measurements.
Moreover, the decay asymmetries of the two-body hadronic weak decays of charmed baryons, defined as
$\alpha\equiv\frac{2{\rm Re}(s^* p)}{|s|^2+|p|^2}$, can be investigated. Here,
$s$ and $p$ represent the parity-violating $s$-wave and
parity-conserving $p$-wave amplitudes in the decay, respectively.
The pole model as well as the covariant quark model and its variants all predict a positive decay asymmetry $\alpha$ for both $\Lambda_c^+\to \Sigma^+\pi^0$ and $\Sigma^0\pi^+$; however, it was measured to be $-0.45\pm0.31\pm0.06$ for $\Sigma^+\pi^0$ by CLEO \cite{CLEO:alpha}. In contrast, the current algebraic approach always leads to a negative decay asymmetry for the two aforementioned modes: $-0.49$ in \cite{CT93}, $-0.31$ in \cite{Verma98}, $-0.76$ in \cite{Zen:1993} and $-0.47$ in \cite{Datta}. The issue with the sign of $\alpha_{\Sigma^+\pi^0}$ has finally been resolved by BESIII and Belle. The decay asymmetry parameters of $\Lambda_c^+\to \Lambda\pi^+$, $\Sigma^0\pi^+$, $\Sigma^+\pi^0$ and $pK_S$ were all recently measured by BESIII~\cite{BES:deasy}; for example, $\alpha_{\Sigma^+\pi^0}=-0.57\pm0.12$ was obtained. Hence, the negative sign of $\alpha_{\Sigma^+\pi^0}$ measured by CLEO has been confirmed by BESIII. Later, Belle confirmed the negative sign with the result of   $\alpha_{\Sigma^+\pi^0}=-0.463\pm0.018$~\cite{Belle:2022uod}.


\vskip 0.15 cm
\noindent \underline{$\Xi_c$ and $\Omega_c$ decays}
\vskip 0.2 cm
The absolute branching fractions of $\Xi_c^0\to \Xi^-\pi^+$ and $\Xi_c^+\to \Xi^-\pi^+\pi^+$
were measured by Belle \cite{Belle:Xic0,Belle:Xic+} to be
\begin{eqnarray}
\B(\Xi_c^0\to \Xi^-\pi^+)=(1.80\pm0.50\pm0.14)\%, \quad \B(\Xi_c^+\to \Xi^-\pi^+\pi^+)=(2.86\pm1.21\pm0.38)\%.
\end{eqnarray}
From these measurements, the branching fractions of other $\Xi_c^0$ and $\Xi_c^+$ decays can be inferred. No absolute branching fractions have been measured for the $\Omega_c^0$. The hadronic weak decays of the $\Omega_c^0$ have been theoretically studied in great detail in \cite{Dhir}, where most of the decay channels of $\Omega_c^0$ decays were found to proceed only through the
$W$-exchange diagram.

It is conceivable that nonleptonic decay modes of $\Lambda_c^+$ and $\Xi_c^{+,0}$ could be measured at the STCF with significantly improved precision. Priority will be given to the decay asymmetries $\alpha$ in various charmed baryon decays and the absolute branching fractions of $\Omega_c^0$ decays.


\item{Charm-flavor-conserving nonleptonic decays}

There is a special class of weak decays of charmed baryons that
can be studied reliably, namely, heavy-flavor-conserving
nonleptonic decays. Some examples are the singly
Cabibbo-suppressed decays $\Xi_c\to\Lambda_c\pi$ and
$\Omega_c\to\Xi_c\pi$. In these decays, not only
the light quarks inside the heavy baryon will participate in weak
interactions, but the charm quark also contributes to the $W$-exchange diagram through the transition $cs\to dc$.
The synthesis of the heavy quark and chiral
symmetries provides a natural setting for investigating these
reactions \cite{ChengHFC}. The predicted branching
fractions for the charm-flavor-conserving decays
$\Xi_c^0\to\Lambda_c^+\pi^-$ and $\Xi_c^+\to\Lambda_c^+\pi^0$ in early days were
of the order of $10^{-3}\sim 10^{-4}$ \cite{ChengHFC}. Recently, the first measurement of
the charm-flavor-conserving decay $\Xi_c^0\to\Lambda_c^+\pi^-$ was achieved by LHCb, with a branching fraction of $(0.55\pm0.02\pm0.18)\%$ \cite{Aaij:2020wtg}, which is confirmed later by Belle~\cite{Belle:2022kqi}. The theoretical estimates of $\B(\Xi_c\to\Lambda_c\pi)$ have been improved recently in \cite{Niu:2021qcc,Cheng:2022kea,Cheng:2022jbr}. The STCF should be able to cross-check the current measurements and search for another $c$-flavor-conserving weak decay, namely, $\Xi_c^+\to\Lambda_c^+\pi^0$.

\end{itemize}

\subsubsection{Semileptonic decays}

Exclusive semileptonic decays of charmed baryons, namely,
$\Lambda_c^+\to\Lambda e^+(\mu^+)\nu_{e(\mu)}$, $\Xi_c^+\to \Xi^0
e^+\nu_e$ and $\Xi_c^0\to \Xi^-e^+\nu_e$, have been observed
experimentally. Their rates depend on the ${\cal B}_c\to{\cal B}$
form factors $f_i(q^2)$ and $g_i(q^2)$ ($i=1,2,3$), defined as
\begin{eqnarray} \label{eq:FF}
 \langle {\cal B}_f(p_f)|V_\mu|{\cal B}_c(p_i)\rangle &=& \bar{u}_f(p_f)
[f_1(q^2)\gamma_\mu+if_2(q^2)\sigma_{\mu\nu}q^\nu+f_3(q^2)q_\mu] u_i(p_i),  \nonumber \\
 \langle {\cal B}_f(p_f)|A_\mu|{\cal B}_c(p_i)\rangle &=& \bar{u}_f(p_f)
[g_1(q^2)\gamma_\mu+ig_2(q^2)\sigma_{\mu\nu}q^\nu+g_3(q^2)q_\mu]\gamma_5
u_i(p_i).
\end{eqnarray}
These form factors have been evaluated using the
nonrelativistic quark model \cite{Marcial,Singleton,CT96,Pervin},
the MIT bag model \cite{Marcial}, the relativistic quark model \cite{Ivanov96,Gutsche,Faustov:semi}, the light-front quark model \cite{Luo}, QCD
sum rules \cite{Carvalho,Huang,Azizi} and LQCD \cite{Meinel:LamcLam,Meinel:Lamcn}. Many of the early predictions of
$\B(\Lambda_c^+\to\Lambda e^+\nu_e)$ are smaller than the first measurement of the absolute branching fraction of $(3.6\pm0.4)\%$ reported by BESIII \cite{BESIII:Lambdaenu}. However, the LQCD calculations in \cite{Meinel:LamcLam} show good agreement with the experimental results for both $\Lambda_c^+\to\Lambda e^+\nu_e$ and $\Lambda_c^+\to\Lambda \mu^+\nu_\mu$. Needless to say,
the semileptonic decays of $\Lambda_c^+$ (including the yet-to-be-observed $\Lambda_c^+\to ne^+\nu_e$), $\Xi_c^{+,0}$ and $\Omega_c^0$, which can be used to discriminate between different form-factor models, will be thoroughly studied at the STCF.


\subsubsection{Electromagnetic and weak radiative decays}

The electromagnetic decays of interest in the charmed baryon sector are the following:
(i) $\Sigma_c \rightarrow
\Lambda_c + \gamma$ and $\Xi^\prime_c \rightarrow \Xi_c + \gamma$; (ii)
$\Sigma^\ast_c \rightarrow \Lambda_c + \gamma$ and $\Xi^{
\ast}_c \rightarrow
\Xi_c + \gamma$; (iii) $\Sigma^\ast_c \rightarrow
\Sigma_c + \gamma$, $\Xi^{ \ast}_c \rightarrow
\Xi^\prime_c + \gamma$, and $\Omega^\ast_c \rightarrow \Omega_c
+ \gamma$; and (iv) $\Lambda_c(2595, 2625)\to\Lambda_c+\gamma$ and $\Xi_c(2790,2815)\to \Xi_c+\gamma$.
Among them, the decay modes $\Xi'^0_c\to\Xi_c^0\gamma$, $\Xi'^+_c\to
\Xi^+_c\gamma$ and $\Omega_c^{*0}\to\Omega_c^0\gamma$ have been experimentally observed.

{
%\squeezetable
\begin{table}[tp]
\centering
\footnotesize
\caption{Electromagnetic decay rates (in units of keV) of $s$-wave charmed
baryons in heavy hadron chiral perturbation theory to LO \cite{Cheng97,Cheng93}, NLO \cite{Jiang} and NNLO  \cite{Wang:2018}. } \label{tab:em}
\begin{center}
\begin{tabular}{c c c c c c c c c c}
\hline\hline  & $\Sigma^+_c\to \Lambda_c^+\gamma$ & $\Sigma_c^{*+}\to\Lambda_c^{+}\gamma$ & $\Sigma_c^{*++}\to\Lambda_c^{++}\gamma$ & $\Sigma_c^{*0}\to\Sigma_c^0\gamma$ &  $\Xi'^+_c\to\Xi_c^+\gamma$  &  $\Xi^{*+}_c\to\Xi_c^+\gamma$ & $\Xi^{*0}_c\to\Xi_c^0\gamma$ & $\Xi'^0_c\to\Xi_c^0\gamma$ & $\Omega_c^{*0}\to\Omega_c^0\gamma$ \\ \hline
 LO & 91.5 & 150.3 & 1.3 & 1.2 & 19.7 & 63.5 & 0.4 & 1.0 & 0.9 \\
 NLO & 164.2 & 893.0 & 11.6 & 2.9 & 54.3 & 502.1 & 0.02 & 3.8 & 4.8 \\
 NNLO & 65.6 & 161.8 & 1.2 & 0.49 & 5.4 & 21.6 & 0.46 & 0.42 & 0.32 \\
 \hline \hline
\end{tabular}
\end{center}
\end{table}
}


The calculated results of \cite{Cheng97,Cheng93}, \cite{Jiang} and \cite{Wang:2018}, denoted by (i), (ii) and (iii), respectively, in Table \ref{tab:em}, can be regarded as the predictions of heavy hadron chiral perturbation theory (HHChPT) to the leading order (LO), next-to-leading order (NLO) and next-to-next-to-leading order (NNLO), respectively.
It is not clear why the predictions of HHChPT to NLO are quite different from those to LO and NNLO for the following three modes: $\Sigma_c^{*+}\to\Lambda_c^+\gamma$, $\Sigma_c^{*++}\to \Sigma_c^{++}\gamma$ and $\Xi^{*+}_c\to\Xi_c^+\gamma$.
It is naively expected that all HHChPT approaches should agree with each other to the lowest order of chiral expansion provided that the coefficients are inferred from the nonrelativistic quark model. This issue can be clarified by the STCF through the measurement of these decay rates.


Very recently, Belle observed the electromagnetic decays of the orbitally excited charmed baryons $\Xi_c(2790)$ and $\Xi_c(2815)$ for the first time \cite{Belle:charme.m.}. The partial widths of $\Xi_c(2815)^0\to\Xi_c^0\gamma$ and $\Xi_c(2790)^0\to\Xi_c^0\gamma$ were measured to be $320\pm45^{+45}_{-80}$ keV and $\sim 800$ keV, respectively. However, no signal was found for the analogous decays of $\Xi_c(2815)^+$ and $\Xi_c(2790)^+$.


Weak radiative decays such as $\Lambda_c^+\to\Sigma^+\gamma$ and $\Lambda_c^+\to p\gamma$ can proceed through the bremsstrahlung processes $cd\to us\gamma$ (Cabibbo allowed) and $cd\to ud\gamma$ (Cabibbo suppressed), respectively. Upper limits on the branching fraction of the former have been set to be $2.6\times 10^{-4}$ and $4.4\times 10^{-4}$ by Belle \cite{Belle:2022raw} and BESIII \cite{BESIII:2022rox}, respectively, which are in agreement with standard model expectations. 


\subsubsection{$CP$ violation}

The CKM matrix contains a single phase that implies the
existence of $CP$ violation. This means that $CP$ violation can be studied in baryons as well. However, the predicted $CP$-violating asymmetries are small for charmed
baryons.
The search for $CP$ violation in charmed baryon decays has gained new momentum
with the large samples of $\Lambda_c$ obtained by BESIII and LHCb. For two-body
decays of the $\Lambda_c^+$, $CP$ violation can be explored through the measurement of the $CP$-violating asymmetry ${\cal A}=(\alpha+\bar\alpha)/(\alpha-\bar\alpha)$, which corresponds to the asymmetries $\alpha$ for the $\Lambda_c^+$ decays and $\bar\alpha$ for the $\bar{\Lambda}_c^-$ decays. For example, the most precise single measurement of 
${\cal A}$ in $\Lambda_c^+\to \Lambda K^+$ and $\bar\Lambda_c^-\to\bar\Lambda K^-$ is reported by BELLE to be $(-58.5\pm4.9\pm1.8)\%$ \cite{Belle:2022uod}. At the STCF, much more sensitive searches for $CP$ violation will be carried out by combining single-tag $\Lambda_c^+$ data~\cite{BES:deasy} with double-tag $\Lambda_c^+\bar{\Lambda}{}_c^-$ data, where the $\Lambda_c^+\bar{\Lambda}{}_c^-$ pairs are quantum correlated in regard to the alignment of their spins with the initial spins of the virtual photons.
In particular, with polarized beams~\cite{Bondar:2019zgm}, the unique advantage of enhanced sensitivities to the decay asymmetries and $CP$ violation can be achieved with prior knowledge of the spin direction of the produced $\Lambda_c^+$.
Regarding three-body decays,
LHCb has measured $\Delta A_{CP}$ as the difference between the $CP$ asymmetries in
the $\Lambda_c^+ \to p K^+ K^-$ and $\Lambda_c^+ \to p\pi^+ \pi^-$ decay channels.
The result is $\Delta A_{CP} = (0.30 \pm 0.91 \pm 0.61)\%$ \cite{Aaij:2017xva},
to be compared with the generic SM prediction of a fraction of 0.1\%~\cite{Bigi:2012ev}.
To probe the SM contribution to such asymmetries, it will be necessary to increase the available statistics by at least a factor of 100.

For $\Lambda_c^+$ decays with multiple hadrons in the final state, such as $\Lambda_c^+\to pK^-\pi^+\pi^0$, $\Lambda_c^+\to\Lambda\pi^+\pi^+\pi^-$ and $\Lambda_c^+\to pK_S\pi^+\pi^-$, $CP$ violation can be exploited through several $T$-odd observables. By virtue of its characteristics of high luminosity, broad center-of-mass energy acceptance, abundant production and a clean environment, the STCF will serve as an excellent platform for this kind of study.
A fast Monte Carlo simulation~\cite{Shi:2019vus} of 1 ab$^{-1}$ $e^+e^-$ annihilation data at $\sqrt{s}=4.64$ GeV, which is expected to be available at the future STCF, indicates that a sensitivity at the level of (0.25--0.5)\% is accessible for the three abovementioned decay modes. This will be sufficient to measure nonzero $CP$-violating asymmetries as large as 1\%.



\subsubsection{Spectroscopy}

\begin{table}[tp]
\caption{Antitriplet and sextet states of charmed baryons.
The mass differences $\Delta m_{\Xi_c\Lambda_c}\equiv m_{\Xi_c}-m_{\Lambda_c}$, $\Delta m_{\Xi'_c\Sigma_c}\equiv m_{\Xi'_c}-m_{\Sigma_c}$, and $\Delta m_{\Omega_c\Xi'_c}\equiv m_{\Omega_c}-m_{\Xi'_c}$ are all in units of MeV. } \label{tab:3and6}
\begin{center}
\begin{tabular}{c| ccc } \hline\hline
  & $J^P(nL)$ & States & Mass difference(s)  \\
 \hline
 ~~${\bf \bar 3}$~~ & ~~${1\over 2}^+(1S)$~~ &  $\Lambda_c(2287)^+$, $\Xi_c(2470)^+,\Xi_c(2470)^0$ & ~~$\Delta m_{\Xi_c\Lambda_c}=183$ ~~  \\
 & ~~${1\over 2}^-(1P)$~~ &  $\Lambda_c(2595)^+$, $\Xi_c(2790)^+,\Xi_c(2790)^0$ & $\Delta m_{\Xi_c\Lambda_c}=198$  \\
 & ~~${3\over 2}^-(1P)$~~ &  $\Lambda_c(2625)^+$, $\Xi_c(2815)^+,\Xi_c(2815)^0$ & $\Delta m_{\Xi_c\Lambda_c}=190$  \\
 & ~~${1\over 2}^+(2S)$~~ &  $\Lambda_c(2765)^+$, $\Xi_c(2970)^+,\Xi_c(2970)^0$ & $\Delta m_{\Xi_c\Lambda_c}=200$  \\
 & ~~${3\over 2}^+(1D)$~~ &  $\Lambda_c(2860)^+$, $\Xi_c(3055)^+,\Xi_c(3055)^0$ & $\Delta m_{\Xi_c\Lambda_c}=201$  \\
 & ~~${5\over 2}^+(1D)$~~ &  $\Lambda_c(2880)^+$, $\Xi_c(3080)^+,\Xi_c(3080)^0$ & $\Delta m_{\Xi_c\Lambda_c}=196$  \\
 \hline
 ~~${\bf 6}$~~ & ~~${1\over 2}^+(1S)$~~ &  $\Omega_c(2695)^0$, $\Xi'_c(2575)^{+,0},\Sigma_c(2455)^{++,+,0}$ & ~~~~$\Delta  m_{\Omega_c\Xi'_c}=119$, $\Delta m_{\Xi'_c\Sigma_c}=124$~~  \\
 & ~~~${3\over 2}^+(1S)$~~~ &  $\Omega_c(2770)^0$, $\Xi'_c(2645)^{+,0},\Sigma_c(2520)^{++,+,0}$ & ~~~~ $\Delta m_{\Omega_c\Xi'_c}=120$, $\Delta m_{\Xi'_c\Sigma_c}=128$~~ \\
 \hline\hline
\end{tabular}
\end{center}
\end{table}

The observed antitriplet and sextet states of charmed baryons are listed in Table \ref{tab:3and6}. At present, the $J^P={1\over 2}^+$, $\frac12^-$, $\frac32^+$, $\frac32^-$ and $\frac52^+$ antitriplet states of $\Lambda_c$ and $\Xi_c$ and the
$J^P={1\over 2}^+$ and ${3\over 2}^+$ sextet states of $\Omega_c$, $\Xi'_c$, and $\Sigma_c$
have been established. The highest state $\Lambda_c(2940)^+$ in the $\Lambda_c$ family was first discovered by BaBar in the $D^0p$ decay mode~\cite{BaBar:Lamc2940}, but its spin-parity assignment is quite diverse (see Refs.~\cite{Cheng:2015,HFLAV:2022pwe}for  review). The constraints on its spin and parity were recently found to be $J^P=\frac32^-$ by LHCb~\cite{LHCb:Lambdac2880}. It was suggested in Ref.~\cite{Cheng:Omegac} that the quantum numbers of $\Lambda_c(2940)^+$ are likely to be $\frac12^-(2P)$ based on the Regge analysis. However, it was argued in Ref.~\cite{Luo:2019qkm} that $\Lambda_c(2940)^+$ is a $\frac32^-(2P)$ state and that there exists a state $\frac12^-(2P)$ higher than the $\Lambda_c(2P, 3/2^-)$. 
This issue can be clarified by the STCF.

In 2017, LHCb explored the charmed baryon sector of the $\Omega_c$ and observed five narrow excited $\Omega_c$ states decaying into $\Xi_c^+K^-$: $\Omega_c(3000)$, $\Omega_c(3050)$, $\Omega_c(3066)$, $\Omega_c(3090)$ and $\Omega_c(3119)$ \cite{LHCb:Omegac}. With the exception of the $\Omega_c(3119)$ state, the first four states were also later confirmed by Belle \cite{Belle:Omegac}. This has triggered considerable interest in the possible identification of their spin-parity quantum numbers. In addition to the five previously observed excited $\Omega_c^0$ states, LHCb has recently reported two new excited states, $\Omega_c(3185)$ and $\Omega_c(3227)$, observed in the $\Xi_c^+ K^-$ spectrum \cite{LHCb:2023rtu}. 


Within the energy region of the STCF up to $7$~GeV, it will be feasible to study the spectra of the singly charmed baryon states $\Lambda_c$, $\Sigma_c$, $\Xi_c^{(\prime)}$, $\Omega_c$ and their excited states in the energy range of $5 \sim 7$ GeV. It will be important for the STCF to explore their possible structure and spin-parity quantum number assignments, especially for the five new narrow $\Omega_c$ resonances.
If the energy region were to be extended to above $7.4$~GeV, the production of the doubly charmed baryon $\Xi^{++}_{cc}$ would also be allowed. This would enable a more detailed study of the recently discovered doubly charmed baryons.
%It is a very tempting future project for upgrade.

