\subsection{Charmed mesons}

\subsubsection{$D^+_{(s)}$ leptonic decays}

Direct determination of the CKM matrix elements $|V_{cd}|$ and
$|V_{cs}|$ is one of the most important targets in charm physics.
These two quark-flavor mixing quantities not only govern the rates
of leptonic $D^+$ and $D^+_s$ decays but also play a crucial role in
testing the unitarity of the CKM matrix.
Hence, the precise measurement of $|V_{cd}|$ and $|V_{cs}|$
is a priority of the STCF experiment.
%A determination of $|V_{cd}|$ and $|V_{cs}|$ to
%a much better degree of accuracy is therefore desirable at STCF.



%%%%%%%%%%%%%%%%%%%%%%%%%% Table 1 %%%%%%%%%%%%%%%%%
\begin{table*}[htbp]
\centering
\caption{\label{tab:pure_LP}\small
For studies on $D^+_{(s)}\to \ell^+\nu_\ell$, the precisions achieved at BESIII and the projected precisions at the STCF
and Belle II.
Considering that the LQCD uncertainty of $f_{D_{(s)}^+}$
has been updated to be approximately 0.2\%~\cite{Bazavov:2017lyh},
the $|V_{cd}|$ value measured at BESIII has been recalculated; this recalculated value is marked with $^*$.
For Belle II, we assume that the systematic uncertainties can be reduced by
a factor of 2 compared to the Belle results.}
\begin{tabular}{lcccc} \hline\hline
\multicolumn{1}{c}{} & BESIII & STCF
& Belle II \\ \hline
\multicolumn{1}{c}{Luminosity} &2.93 fb$^{-1}$ at 3.773 GeV & 1 ab$^{-1}$ at 3.773 GeV
& 50 ab$^{-1}$ at $\Upsilon(nS)$ \\ \hline
${\mathcal B}(D^+\to \mu^+\nu_\mu)$ & $5.1\%_{\rm stat}\,1.6\%_{\rm syst}$~\cite{bes3_muv} &$0.28\%_{\rm stat}$ & $2.8\%_{\rm stat}$~\cite{Kou:2018nap}  \\
$f^{\mu}_{D^+}$ (MeV) &$2.6\%_{\rm stat}\,0.9\%_{\rm syst}$~\cite{bes3_muv} &$0.15\%_{\rm stat}$ &--   \\
$|V_{cd}|$ &$2.6\%_{\rm stat}\,1.0\%_{\rm syst}^*$~\cite{bes3_muv} &$0.15\%_{\rm stat}$  & --   \\
${\mathcal B}(D^+\to \tau^+\nu_\tau)$ &$20\%_{\rm stat}\,{10\%_{\rm syst}}$~\cite{Ablikim:2019rpl}  &  $0.41\%_{\rm stat}$ &-- \\
$\displaystyle \frac{{\mathcal B}(D^+\to \tau^+\nu_\tau)}{{\mathcal B}(D^+\to \mu^+\nu_\mu)}$&$21\%_{\rm stat}\,13\%_{\rm syst}$~\cite{Ablikim:2019rpl}  & $0.50\%_{\rm stat}$ &   --
\\ \hline \hline
\multicolumn{1}{c}{Luminosity} &6.3 fb$^{-1}$ at (4.178, 4.226) GeV &1 ab$^{-1}$ at 4.009 GeV
& 50 ab$^{-1}$ at $\Upsilon(nS)$ \\ \hline

${\mathcal B}(D^+_s\to \mu^+\nu_\mu)$ &$2.4\%_{\rm stat}\,3.0\%_{\rm syst}$~\cite{BESIII:2021anh}&$0.30\%_{\rm stat}$ & $0.8\%_{\rm stat}\,1.8\%_{\rm syst}$\\
$f^{\mu}_{D^+_s}$ (MeV) &$1.2\%_{\rm stat}\,1.5\%_{\rm syst}$~\cite{BESIII:2021anh}&$0.15\%_{\rm stat}$ &-- \\
$|V_{cs}|$ &$1.2\%_{\rm stat}\,1.5\%_{\rm syst}$~\cite{BESIII:2021anh}
&$0.15\%_{\rm stat}$ &-- \\\hline

${\mathcal B}(D^+_s\to \tau^+\nu_\tau)$ &$1.7\%_{\rm stat}\,2.1\%_{\rm syst}$~\cite{HFLAV:2022pwe} &$0.24\%_{\rm stat}$ &$0.6\%_{\rm stat}\,2.7\%_{\rm syst}$ \\
$f^{\tau}_{D^+_s}$ (MeV) &$0.8\%_{\rm stat}\,1.1\%_{\rm syst}$~\cite{HFLAV:2022pwe} &$0.11\%_{\rm stat}$ & -- \\
$|V_{cs}|$ &$0.8\%_{\rm stat}\,1.1\%_{\rm syst}$~\cite{HFLAV:2022pwe} &$0.11\%_{\rm stat}$ &-- \\ \hline

$\overline f^{\mu \&\tau}_{D^+_s}$ (MeV) &$0.7\%_{\rm stat}\,0.9\%_{\rm syst}$&$0.09\%_{\rm stat}$&$0.3\%_{\rm stat}\,1.0\%_{\rm syst}$\\
$|\overline V_{cs}^{\mu \&\tau}|$ &$0.7\%_{\rm stat}\,{0.9\%_{\rm syst}}$&$0.09\%_{\rm stat}$ &-- \\ \hline \hline
$f_{D^+_s}/f_{D^+}$ &$1.4\%_{\rm stat}\,1.7\%_{\rm syst}$~\cite{BESIII:2021bdp} &$0.21\%_{\rm stat}$ &-- \\ 
$\displaystyle \frac{{\mathcal B}(D^+_s\to \tau^+\nu_\tau)}{{\mathcal B}(D^+_s\to \mu^+\nu_\mu)}$&$2.9\%_{\rm stat}\,3.5\%_{\rm syst}$& $0.38\%_{\rm stat}$&$0.9\%_{\rm stat}\,3.2\%_{\rm syst}$
\\ \hline \hline
\end{tabular}
\end{table*}
%%%%%%%%%%%%%%%%%%%%%%%%%%

The most precise way to determine $|V_{cd}|$ and $|V_{cs}|$ at the STCF will be via pure leptonic decays of the form $D_{(s)}^+ \to \ell^+ \nu^{}_\ell$ (where $\ell = e, \mu, \tau$), as the semileptonic decays suffer from large uncertainties in the LQCD calculations of the form factors.
The product of the decay constant $f_{D_{(s)}^+}$ and $|V_{cd(s)}|$ can be directly accessed by measuring the widths
of $D_{(s)}^+ \to \ell^+ \nu^{}_\ell$. Then, with $f_{D_{(s)}^+}$ from LQCD as input, the values of $|V_{cd(s)}|$ or $f_{D_{(s)}^+}$ can be obtained.
Listed in Table~\ref{tab:pure_LP} are the most precise determinations to date of $|V_{cs(d)}|$ and $f_{D^+_{(s)}}$~\cite{bes3_muv,Ablikim:2019rpl,Ablikim:2018jun} at BESIII and the projected precisions at the STCF.
Note that for ${\mathcal B}(D^+\to \tau^+\nu_\tau)$,
several $\tau^+$ decay channels,
such as $\tau^+\to \pi^+\overline{\nu}_\tau$, $e^+\overline{\nu}_\tau\nu_e$,
$\mu^+\overline{\nu}_\tau\nu_\mu$, and $\rho^+\overline{\nu}_\tau$,
are combined to improve the statistical sensitivity.

The systematic uncertainties at the STCF are to be optimized to a subleading level, as the statistical uncertainties are expected to be less than 0.5\%. To reduce the systematic uncertainties due to background and fitting, it will be optimal for the STCF to study $D_s^+\to \ell^+\nu_\ell$ using $\ee\to D_s^+ D_s^{-}$ at 4.009 GeV.
Thus far, the $f_{D_{(s)}^+}$ values have been calculated via LQCD with precisions of approximately 0.2\%~\cite{Bazavov:2017lyh}; specifically,
$f_D^+=212.7\pm0.6$~\mev, $f_{D_s}^+=249.9\pm0.4~\mev$ and $f_{D_s}^+/f_D^+=1.1749\pm0.0016$. At the time when the STCF comes online,
%Editor: Please ensure that the intended meaning has been maintained in the above edit.
their precisions are expected to be below 0.1\%. This means that the sizes of the systematic uncertainties at the STCF will be crucial and must be improved to the similar level. The feasibility studies of $D^{+}\to\mu^{+}\nu_{\mu}$ and $D_{s}^{+}\to \tau^{+}\nu_{\tau}$ are presented in Ref.~\cite{Liu:2021qio,Li:2021ala}.  
In particular, the efficiencies of muon and electron identification will be critical and must be optimized to constrain the total uncertainty to reach the expected level.

On the other hand, precise measurements of the semileptonic branching fractions for $D_{(s)}\to h \ell^+\nu_\ell$, where $h$ is a charmless hadron, will be used to calibrate the LQCD calculations of the form factors involved by introducing the $|V_{cd(s)}|$ values from global CKM fits (such as those of CKMfitter~\cite{Charles:2004jd,CKMfitter_web} and UTfit~\cite{Bona:2005vz,utfit_web}).
For the case of $D_{(s)}\to V(h_1h_2) \ell^+\nu_\ell$ (where $V$ denotes a vector meson, decaying into hadrons $h_1$ and $h_2$), time reversal ($T$) invariance can be tested with high precision by constructing triple-product $T$-odd observables~\cite{Belanger:1991vx}. This will serve as a sensitive probe of $CP$ violation mechanisms beyond the Standard Model and of new physics~\cite{Grossman:2003qi}, such as models with multi-Higgs doublets or leptoquarks. Ref.~\cite{Wang:2019wee} proposes combined measurements of $D\to K_1 \ell^+\nu_\ell$ and $B\to K_1 \gamma$ to unambiguously determine the photon polarization in $b\to s\gamma$ in a clean way to probe right-handed couplings in new physics.


%%%%%%Test on the lepton universality %%%%%%%%%%

Lepton flavor universality (LFU) can also be tested in charmed meson leptonic decays. LFU violation may occur in $c\to s$ transitions due to an amplitude that includes a charged Higgs boson, which arises in a two-Higgs-doublet model, interfering with the SM amplitude involving a $W^{\pm}$ boson~\cite{prd91_094009}. In the SM, the ratio of the partial widths of
$D^+_{(s)}\to \tau^+\nu_\tau$ and $D^+_{(s)}\to \mu^+\nu_\mu$
is predicted to be
\begin{eqnarray}
R_{D^+_{(s)}}=\frac{\Gamma(D^+_{(s)}\to \tau^+\nu_\tau)}{\Gamma(D^+_{(s)}\to \mu^+\nu_\mu)}=
\frac{m^2_{\tau^+}\left(1-\frac{m^2_{\tau^+}}{m^2_{D^+_{(s)}}} \right )^2}{m^2_{\mu^+} \left(1-\frac{m^2_{\mu^+}}{m^2_{D^+_{(s)}}} \right )^2}.
\end{eqnarray}
Using the world average values of the masses of the leptons and $D^+_{(s)}$~\cite{ParticleDataGroup:2022pth}, one obtains $R_{D^+}=2.67\pm 0.01$ and $R_{D^+_s}=9.75\pm0.01$.
The measured values of $R_{D^+}$ and $R_{D^+_s}$ reported by BESIII are $3.21\pm 0.64_{\rm stat}\pm 0.43_{\rm syst}$~\cite{Ablikim:2019rpl} and $9.72\pm 0.37$~\cite{BESIII:2021bdp}, respectively, which agree with the SM predictions.
%Editor: Please ensure that the intended meaning has been maintained in the above edit.
However, these measurements are currently statistically limited.
At the STCF, as listed in Table~\ref{tab:pure_LP}, the statistical precision for $R_{D_{(s)}^{+}}$ will be comparable to the uncertainties of the predictions in the SM. Hence, it will provide a meaningful test of LFU via these channels.

Another LFU test could be performed via the semileptonic decay modes, of which the semitauonic decay is
kinematically forbidden or suppressed. Measurements of the ratios of the partial widths of $D^{0(+)}\to h \mu^+\nu_\mu$
over those of $D^{0(+)}\to h e^+\nu_e$ in different $q^2$ intervals
would constitute a test of LFU complementary to those using tauonic decays.
BESIII has reported precise measurements of the ratios ${\mathcal B}(D^0\to\pi^-\mu^+\nu_\mu)/{\mathcal B}(D^0\to\pi^-e^+\nu_e)=0.922\pm0.030\pm0.022$
and ${\mathcal B}(D^+\to\pi^0\mu^+\nu_\mu)/{\mathcal B}(D^+\to\pi^0e^+\nu_e)=0.964\pm0.037\pm0.026$~\cite{bes3_pimuv}.
These results are consistent with the SM predictions within $1.7\sigma$ and $0.5\sigma$~\cite{bes3_pimuv}, respectively.
These measurements are currently statistically limited~\cite{bes3_kmuv,bes3_pimuv}, and they could be significantly improved with 1 ab$^{-1}$ of data taken at the center-of-mass energy of 3.773 GeV at the STCF.

For the above LFU tests at STCF, control of systematic uncertainties will be an essential issue to enhance the sensitivity. Hence, a double ratio of two different types of leptonic decay modes could provide significant cancellation of detection systematics in the further measurement. 

\subsubsection{$\dzero$--$\dzerobar$ mixing and $CP$ violation}

The phenomenon of meson--antimeson mixing has been of great interest
throughout the long history of particle physics.
In contrast to the $B$-meson and kaon systems, $CP$ violation in the mixing of $D$ mesons has not been observed. The STCF will be an
ideal place for the study of $D^0$--$\bar{D}^0$ mixing and $CP$ violation. By convention,
the mass states of the two neutral $D$ mesons are written as
\begin{eqnarray}
|D^{}_1\rangle \hspace{-0.2cm} & = & \hspace{-0.2cm} p |D^0\rangle +
q |\bar{D}^0\rangle \; ,
\nonumber \\
|D^{}_2\rangle \hspace{-0.2cm} & = & \hspace{-0.2cm} p |D^0\rangle -
q |\bar{D}^0\rangle \; ,
%       (4)
\end{eqnarray}
where $|p|^2 + |q|^2 = 1$. The $D^0$--$\bar{D}^0$ mixing
parameters are defined as $x \equiv (M^{}_2 - M^{}_1)/\Gamma$ and $y
\equiv (\Gamma^{}_2 - \Gamma^{}_1)/(2\Gamma)$, where $M^{}_{1,2}$
and $\Gamma^{}_{1,2}$ are the masses and widths, respectively, of $D^{}_{1,2}$.
Additionally, $\Gamma \equiv (\Gamma^{}_1 + \Gamma^{}_2)/2$, and $M \equiv (M^{}_1
+ M^{}_2)/2$. This system is unique because it is the only
meson--antimeson system whose mixing (or oscillation) takes place via
the intermediate states with down-type quarks. It is also the only
meson--antimeson system whose mixing parameters $x$ and $y$ are
notoriously difficult to calculate in the SM, as they involve large long-distance uncertainties in this nonperturbative regime.
%
%The mixing between $D^0$ and $\bar{D}^0$ mesons arises from the fact
%that they couple to a subset of virtual or real intermediate states.
%While $x$ may be sensitive to new physics via the $\Delta C =2$
%contribution, $y$ is dominated by the SM (i.e., $\Delta C= 1$)
%contribution. They can be estimated at either the quark level or the
%hadron level, but both of them involve large long-distance
%uncertainties.
One expects $x\sim y\sim \sin^2\theta^{}_{\rm C}
\times [{\rm SU(3) ~ breaking}]^2$ as a second-order effect of the
flavor SU(3) symmetry breaking. A more careful analysis yields the
order-of-magnitude estimates $x\lesssim y$ and $10^{-3} < |x| <
10^{-2}$ \cite{Falk2004}.
A global fit to the world measurements of $x$ and $y$, carried out by the Heavy Flavor Averaging Group \cite{Amhis:2019ckw,hflav_web}, gives
intervals of $1.6 \times 10^{-3} \lesssim x \lesssim 6.1 \times 10^{-3}$
and $5.2 \times 10^{-3} \lesssim y \lesssim 7.9 \times 10^{-3}$
at the $95\%$ confidence level~\cite{Amhis:2019ckw,hflav_web}. We see that the allowed regions for $x$ and $y$ are
essentially consistent with the theoretical estimates (i.e.,
$x\lesssim y \sim 7 \times 10^{-3}$).
Much more precise measurements of
these two $D^0$--$\bar{D}^0$ mixing parameters can be achieved at the
STCF. Although their accurate values might not help much to
clarify the long-distance effects in $D^0$--$\bar{D}^0$ mixing, they
will be of great help in probing the presumably small effects of $CP$ violation
in neutral $D$-meson decays and mixing \cite{SuperB}.

%The following decay modes have been measured in different experiments
%to globally determine the values
%of $x$ and $y$ \cite{PDG}: $D^0 \to K^{(*)+}\ell^-
%\overline{\nu}^{}_\ell$, $K^+K^-$, $\pi^+\pi^-$, $K^+\pi^-$,
%$K^+\pi^-\pi^0$, $K^+\pi^-\pi^+\pi^-$, $K^0_{\rm S}\pi^+\pi^-$,
%$K^0_{\rm S}K^+K^-$ {\it etc} and (or) their $CP$-conjugate processes,
%together with the coherent $\psi(3770) \to D^0\bar{D}^0 \to f^{}_1 f^{}_2$
%decays.


The charm sector is a precision laboratory for exploring possible
$CP$-violating new physics because the SM-induced $CP$-violating asymmetries
in $D$-meson decays are typically in the range of $10^{-4}$ to $10^{-3}$~\cite{Xing:2007zz}
and are very challenging to detect in experiments. The $CP$-violating asymmetries
in the singly Cabibbo-suppressed $D$-meson decays are now expected to be much larger than
those in the Cabibbo-favored and doubly Cabibbo-suppressed decays~\cite{SuperB},
where such asymmetries vanish.
%One may easily understand this point by considering the
%{\it charmed} unitarity triangle of the CKM matrix as defined by the
%relation $V^*_{ud} V_{cd} + V^*_{us} V_{cs} + V^*_{ub}
%V_{cb} = 0$ in the complex plane, in which two sides are
%comparable in magnitude and much longer than the third one. In other
%words, the shape of this unitarity triangle is too sharp. Given
%${\rm Im}\left(V^*_{ud} V_{cd} + V^*_{us} V_{cs}\right) =
%-{\rm Im}\left(V^*_{ub} V_{cb}\right) \sim \lambda^6\sin\gamma$
%with $\lambda \simeq 0.22$ and $\gamma \simeq 74^\circ$ \cite{PDG},
%the ratio of the CP-violating part to the CP-conserving part in many
%$D$-meson decays is characterized by ${\rm Im}\left(V^*_{ud}
%V_{cd} + V^*_{us} V_{cs}\right)/ \left(|V^*_{ud} V_{cd}| +
%|V^*_{us} V_{cs}|\right) \sim \lambda^5 \sin\gamma \sim 5\times
%10^{-4}$.
There are, in general, three different types of $CP$-violating effects in
neutral and charged $D$-meson decays \cite{Xing97}: 1) $CP$ violation
in $D^0$--$\bar{D}^0$ mixing, 2) $CP$ violation in the direct decay, and 3)
$CP$ violation from the interplay of decay and mixing.
% 4) CP violation in the $CP$-forbidden decay of coherent $D^0$ and $\bar{D}^0$ mesons.
In addition to these three types of $CP$-violating effects in $D$-meson
decays, one may expect an effect of $CP$ violation induced by
$K^0$--$\bar{K}^0$ mixing in some decay modes with $K^{}_{\rm S}$ or
$K^{}_{\rm L}$ in their final states. The magnitude of this effect is typically $2
{\rm Re}(\epsilon^{}_K) \simeq 3.3 \times 10^{-3}$, which may be
comparable to or even larger than the {\it charmed} $CP$-violating
effects~\cite{Xing:1995jg,Yu:2017oky}.
To date, much effort has been put into searching for $CP$ violation
in $D$-meson decays. The LHCb collaboration has discovered
$CP$ violation in combined $D^0\to \pip\pim$ and $D^0\to K^+K^-$ decays with a
significance of 5.3$\sigma$. The time-integrated $CP$-violating asymmetry
is given as
\begin{eqnarray}
\Delta a_{CP}&=&\frac{\Gamma(D\to \kk)-\Gamma(\bar{D}\to\kk)}{\Gamma(D\to \kk)+\Gamma(\bar{D}\to\kk)}-\frac{\Gamma(D\to\pip\pim)-\Gamma(\bar D\to\pip\pim)}{\Gamma(D\to\pip\pim)+\Gamma(\bar D\to\pip\pim)}  \nonumber \\
&=& (-0.154\pm0.029)\%,
%     (5)
\end{eqnarray}
where $D$($\bar{D}$) is a $D^0$($\bar{D}{}^0$) at time $t$=0~\cite{Aaij:2019kcg},
and it mainly arises from direct $CP$ violation in the charm-quark decay~\cite{Saur:2020rgd}.
This result is consistent with some theoretical estimates within the SM
(see, e.g., Refs. \cite{Cheng,Li,Grinstein,Gronau,Italy,Italy2,Li:2019hho,Grossman:2019xcj});
however, the latter involve quite large uncertainties.
The STCF will have a sensitivity at the level of $10^{-4}$ in systematically searching
for $CP$ violation in different types of charmed-meson decays.
In particular, the advantage of kinematical constraints on the initial four-momenta of the $\ee$ collisions will make the STCF competitive in studies
of $CP$-violating asymmetries in multibody $D$ decays~\cite{Bigi:2011em},
such as 4-body hadronic decays and the $CP$ asymmetries therein in the local Dalitz region.
Considering that the CKM mechanism of $CP$ violation in the SM fails to explain the puzzle of
the observed matter--antimatter asymmetry in the Universe
by more than 10 orders of magnitude~\cite{Morrissey:2012db}, there is strong motivation to search for
new (heretofore undiscovered) sources of $CP$ violation associated with both quark and lepton
flavors. In this context, the charm-quark sector is certainly a promising playground.

Note that the STCF will be a unique place for the study of $D^0$--$\bar{D}^0$
mixing and $CP$ violation by means of the quantum coherence of $D^0$ and
$\bar{D}^0$ mesons produced at energy points near the threshold. In fact, a $D^0\bar{D}^0$ pair can be coherently produced
through the reactions $\ee \to (D^0\bar{D}^0)^{}_{\rm CP=-}$ at 3.773 GeV and
$\ee\to D^0\bar{D}^{*0} \to \pi^0 (D^0\bar{D}^0)^{}_{\rm
CP=-}$ or $\gamma (D^0\bar{D}^0)^{}_{\rm CP=+}$ at 4.009 GeV. One may
therefore obtain useful constraints on $D^0$--$\bar{D}^0$ mixing and $CP$-violating
parameters in the respective decays of correlated $D^0$ and
$\bar{D}^0$ events \cite{Xing97}.
For example, the $D^0$--$\bar{D}^0$ mixing rate $R_M=(x^2+y^2)/2$ can
be accessed via the same charged final states $(K^\pm\pi^\mp)(K^\pm\pi^\mp)$
or $(K^\pm\ell^\mp\nu)(K^\pm\ell^\mp\nu)$ with a sensitivity of $10^{-5}$
with 1 ab$^{-1}$ of data collected at 3.773 GeV.
Considering $\ee\to \gamma D^0\bar{D}^0$ at 4.009 GeV, the $D^0\bar{D}^0$ pairs
are in $C$-even states, and the charm mixing contribution is doubled compared with
the time-dependent (uncorrelated) case. With 1 ab$^{-1}$ of data at 4.009 GeV,
it is expected that the measurement sensitivities for the mixing parameters
($x$ and $y$) will reach a level of 0.05\%, and those for $|q/p|$ and $\arg(q/p)$ will be 1.5\% and $1.4^\circ$, respectively~\cite{Bondar:2010qs}. Another possible case is that the decay mode
$\left( D^0\bar{D}^0 \right)^{}_{\rm CP = \pm} \to \left( f^{}_1
f^{}_2 \right)^{}_{\rm CP = \mp}$, where $f^{}_1$ and $f^{}_2$ are
proper $CP$ eigenstates (e.g., $\pi^+\pi^-$, $K^+K^-$ and $K^{}_{\rm
S} \pi^0$), is a $CP$-forbidden process and can only occur due to $CP$ violation.
The rate for a pair of $CP$-even final states $f_+$ (such as $f_+=\pi^+\pi^-$)
can be expressed as
\begin{equation}
\Gamma^{++}_{D^0\bar{D}^0 } = \left[
\left(x^2+y^2\right)\left(\cosh^2 a_m - \cos^2 \phi\right) \right] \Gamma^2(D \to f_+),
%     (6)
\end{equation}
where $\phi = \arg(p/q)$, $R_M=|p/q|$, and $a_m=\log R_M$~\cite{Atwood:2002ak}.

$CPT$ is conserved in all locally Lorentz-invariant theories,
including the SM and all of its commonly discussed extensions.
When $CPT$ is conserved, $CP$ violation implies the violation of time reversal symmetry ($T$).
However, $CPT$ violation might also arise in string theory or some extradimensional
models with Lorentz-symmetry violation in four dimensions.
Hence, direct observation of $T$ violation without the presumption of $CPT$
conservation is very important~\cite{Shi:2016bvo}.
Experimental studies of the time evolution of $CP$-correlated $D^0$--$\bar{D}^0$
states at the STCF could be complementary to the $CPT$-violation studies at
super $B$ factories and the LHCb experiments~\cite{Kostelecky:2001ff}.
However, this becomes very challenging with symmetric $\ee$ collisions, as the produced $D$ mesons have very low momentum in the laboratory frame and hence have flight distances that are too short to be detected. Only an asymmetric $\ee$ collision mode can be feasible for such investigations.

The quantum correlation of a $D^0\bar D^0$ meson pair offers a
unique feature for probing the amplitudes of the $D^0$ decays and
determining the strong-phase difference between their Cabibbo-favored and
doubly Cabibbo-suppressed amplitudes. Measurements of the strong-phase difference
are well motivated from several perspectives: understanding the nonperturbative
QCD effects in the charm sector, serving as essential inputs for extracting
the angle $\gamma$ of the CKM unitarity triangle (UT), and relating the
measured mixing parameters $(x', y')$ in hadronic decay to the mass and width
difference parameters $(x, y)$~\cite{Amhis:2019ckw}.

Measurements of the CKM UT angles $\alpha$,
$\beta$, and $\gamma$ in $B$ decays are important tests of CKM unitarity
and provide another avenue to search for possible sources of $CP$ violation beyond the SM. Any discrepancy in
measurements of the UT involving tree- and loop-dominated processes would
indicate the existence of new heavy degrees of freedom contributing to
the loops. Among the three CKM angles, $\gamma$ is of particular importance
because it is the only $CP$-violating observable that can be determined
using tree-level decays. Currently, the world-best single measurement of $\gamma$
is from LHCb: $\gamma = (63.8^{+3.5}_{-3.7})^\circ$~\cite{LHCb:2022awq}.
The precision measurement of $\gamma$ will be one of the top priorities for the
LHCb upgrade(s) and the Belle II experiment.

The most precise method of measuring $\gamma$ is
based on the interference between the $B^{+}\to\bar{D}^{0}K^{+}$
and $B^{+}\to D^{0}K^{+}$ decays~\cite{GLW1, GLW2, ADS1, ADS2, GGSZ}.
In the future, the statistical uncertainties of these measurements
will be greatly reduced by using the large $B$ meson samples collected
by LHCb and Belle II. Hence, limited knowledge of the strong phases of
the $D$ decays will systematically restrict the overall sensitivity.
A 20 fb$^{-1}$ dataset collected at 3.773 GeV at BESIII would lead to a
systematic uncertainty of $\sim$0.4$^\circ$ for the $\gamma$ measurement~\cite{Ablikim:2019hff}.
Hence, to match the anticipated future statistical uncertainty of less than $0.4^\circ$
in the future LHCb upgrade II,
the STCF could provide important constraints to reduce the systematic
uncertainty from the $D$ strong phase to less than 0.1$^\circ$ and enable
detailed comparisons of the $\gamma$ results from different decay modes.

\subsubsection{Rare and forbidden decays}

With its high luminosity, clean collision environment and excellent
detector performance, the STCF has great potential to
perform searches for rare and forbidden $D$-meson decays,
which may serve as a useful tool for
probing new physics beyond the SM. Such decays can be classified into three
categories: (1) decays via a flavor-changing neutral current
(FCNC), such as the $D^{0(+)} \to \gamma V^{0(+)}$, $D^0 \to \gamma\gamma$,
$D^0 \to \ell^+\ell^-$, and
$D \to \ell^+\ell^- X$ channels (where $\ell = e, \mu$)
and $D \to \nu \overline{\nu} X$, which provide SM-allowed transitions between
$c$ and $u$ quarks; (2)
decays with lepton flavor violation (LFV), such as the $D^0 \to \ell^+
\ell^{\prime -}$ and $D \to \ell^+\ell^{\prime -} X$ channels (for
$\ell \neq \ell^\prime$), which are forbidden in the SM; and (3) decays
with lepton number violation (LNV), such as the $D^+ \to \ell^+\ell^{\prime +}
X^-$ and $D^+_s \to \ell^+\ell^{\prime +} X^-$ channels (for either
$\ell = \ell^\prime$ or $\ell \neq \ell^\prime$), which are also
forbidden in the SM. The discovery of neutrino oscillations has
confirmed the occurrence of LFV in the lepton sector, and LNV is also possible if the massive
neutrinos are Majorana particles. It is therefore needed to
search for the LFV and LNV phenomena in the charm-quark sector.

Although FCNC decays of $D$ mesons are allowed in the SM, they
can only occur via loop diagrams and hence are strongly
suppressed. The long-distance dynamics are expected to dominate the
SM contributions to such decays, but their branching fractions are
still tiny. For instance, ${\cal B}(D^0 \to \gamma\gamma) \sim 1
\times 10^{-8}$ and ${\cal B}(D^0 \to \mu^+\mu^-) \sim 3 \times
10^{-13}$ in the SM \cite{Burdman}, but they can be significantly
enhanced by new physics \cite{Golowich}. The current experimental bounds
on these two typical FCNC channels are ${\cal B}(D^0 \to
\gamma\gamma) < 8.5 \times 10^{-7}$ and ${\cal B}(D^0 \to \mu^+\mu^-)
< 6.2 \times 10^{-9}$ \cite{ParticleDataGroup:2022pth}. However, the semileptonic decays
$D^0\to\pp\MM$, $\kk\MM$ and $K^-\pi^+ \MM$
have been observed at LHCb with BFs at a level of $10^{-7}$~\cite{ParticleDataGroup:2022pth}.
In addition to the removal of helicity suppression dominating the highly suppressed BF for $D^0 \to \mu^+\mu^-$, the observed BFs for the semileptonic decays indicate nontrivial contributions from complicated long-distance effects.
At the STCF, it will be better to study di-electron modes of the form $D\to \ee X$~\cite{TheBESIIICollaboration2018a},
which will provide sensitivities of $10^{-8}\sim 10^{-9}$ for $m_{\ee}$
in the range less polluted by long-range resonance contributions.
Compared to Belle II and LHCb, the STCF has competitive sensitivities in channels that contain neutral final states, such as photons and $\pi^0$, benefit from the almost full acceptance and quasi background-free advantages.
Furthermore, BESIII carried out world-first search for charmed meson decays into di-nutrinos $D^0 \to \pi^0 \nu \overline{\nu}$ and set the upper limit of ${\cal B}(D^0 \to \pi^0 \nu \overline{\nu})$ to be $2.1\times 10^{-4}$~\cite{BESIII:2021slf}. The STCF has the advantage of best constraining the upper limits on the
BFs for $D$ rare decays with neutrinos, such as
$D^0 \to \pi^0 \nu \overline{\nu}$ and $D^0 \to \gamma \nu \overline{\nu}$.

No evidence has been found for forbidden $D_{(s)}$-meson decays with
either LFV or LNV or both of them. The present experimental bounds
on the LFV decays are generally at the level of $10^{-6}$ to $10^{-5}$
(with the exception of ${\cal B}(D^0 \to \mu^\pm e^\mp)<1.3\times 10^{-8}$)~\cite{ParticleDataGroup:2022pth}.
The STCF will provide more stringent limits on
such interesting LFV and LNV decay modes, with a sensitivity of
$10^{-8}$ to $10^{-9}$ or smaller, taking advantage of its clean environment
and accurate charge discrimination.

\subsubsection{Charmed-meson production and spectroscopy}

The STCF will also act as a good playground for studying the production of charmed mesons and exploring charmed-meson spectroscopy.
To date, all of the 1$S$ and 1$P$ $D_{(s)}$ states have been found in experiments~\cite{Chen:2016spr,HFLAV:2022pwe}. However, almost all of the other predicted excited states in QCD-derived effective models are missing.
Furthermore, many excited open-charm states have been reported in experiments, and attempts to formulate an understanding of their nature remain controversial. Some of them are candidates for exotic mesons. For instance, the narrow $D^*_{sJ}(2632)$ state was observed by SELEX, but CLEO, BaBar and FOCUS all reported negative search results.
The unexpectedly low masses of the $D_{s0}^{*}(2317)$ and $D_{s1}(2460)$ have given rise to various exotic explanations, such as the $D^{(*)}K$ molecule state~\cite{Guo:2017jvc}. It has been claimed that strong $S$-wave $D^{(*)}K$ scattering contributes to the mass drop.
Thus, further systematic research on the open-charm meson spectra is highly desired.

At the STCF, it will be possible to produce excited charmed-meson states $D^{**}$ via direct
$\ee$ production processes, such as $\ee\to D^{**}\bar{D}^{(*)}(\pi)$, in the energy range from 4.1 to 7.0 GeV.
This will allow higher excited open-charm states to be studied through their hadronic or radiative decays~\cite{Kato:2018ijx} to lower open-charm states.
Systematic studies at the STCF on the open-charm meson spectra will provide important data for exploring nonperturbative QCD in the charm regime and testing various theoretical models.


