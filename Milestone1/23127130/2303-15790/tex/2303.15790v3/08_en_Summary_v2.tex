\section{Summary}

The proposed STCF is a $4\pi$-solid-angle particle detector operating at an $e^{+}e^{-}$ collider with a high luminosity
($>0.5\times10^{35}$~cm$^{-2}$s$^{-1}$), center-of-mass energies spanning the range of $2\sim7$~GeV, and the future option
for a polarized $e^-$ beam. We have presented some of the interesting physics potential at the STCF, mainly on the basis of various particle systems ranging from higher masses, starting from the $XYZ$ states, to lower-mass systems, such as hyperon and glueball/hybrid states, as well as possible new light particles beyond the SM along with information can be extracted from their decays and interactions. Some studies that are important for extracting SM information not associated with energies on/near resonances have also been discussed. With unprecedentedly high luminosity, studies of the spectra in the relevant energy ranges can provide much more precise knowledge about known states and possibly other new and exotic states, and studies of decays and how they interact with other SM particles can yield new insights along with the most precise information about parameters in SM electroweak interactions, the perturbative and nonperturbative nature of strong QCD interactions, and possible new particles and interactions beyond the SM. The topics presented are not all inclusive; instead, we have focused on measurements that are unique to the STCF, with emphasis on reactions that challenge the SM, are sensitive to new physics and address poorly understood features in existing data. We summarize the highlights in the following.

\noindent
{\bf QCD dynamics and hadron physics:} The STCF will run in the energy range of 2 $\sim 7$ GeV, which is in the transition interval between nonperturbative and perturbative QCD based on the $SU(3)_C$ gauge interaction. Experimental data from the STCF will provide much more information to study the QCD dynamics of confinement
through the study of hadron spectra and interactions (see Sections~\ref{sec:charmonium}, \ref{sec:charmedhadron} and \ref{sec:qcd}). The energy region covers the pair production thresholds for the recently discovered doubly charmed baryon states, $XYZ$ states, charmed baryons, charmed mesons, $\tau$ leptons, and all of the strange hyperons. With the high luminosity of the STCF, firmer traces of the QCD-predicted glueballs/hybrids may finally be observed, and with operation above 7 GeV, higher-mass doubly charmed baryons or other possible new states may be studied in more detail.

More detailed discussions of hadron spectroscopy have been provided in the relevant sections. Two remarkable recent developments in hadron spectroscopy are worth emphasizing again:
1) the failure of hadron models to anticipate the rich charmonium spectrum of
hidden-charm states with masses above the open-charm pair threshold and
2) the emergence of clear experimental evidence for new, light-hadron spectra of QCD hybrids
and glueballs. These developments boost confidence that
more detailed spectroscopic studies are needed and will be fruitful. The STCF will be an ideal place to study related issues.
The mapping out of the {\it XYZ}, glueball and hybrid
spectra will require comprehensive measurements of as many decay modes as possible and
more sophisticated analysis techniques to extract and interpret exotic mesons that
overlap with conventional $q\bar{q}$ states.
Year-long STCF runs will produce data samples containing $\sim$3~T $J/\psi$ events and $\sim$500~B $\psi(3686)$ events
for in-depth explorations of light hadron physics. At CME~$\approx4230$~MeV, the STCF will function as an
``$XYZ$-meson factory'', producing $\sim$1B $Y(4260)$ events, $\sim$100M each of $Z_{c}(3900)$ and $Z_{c}(4020)$ events, and
$\sim$5M $X(3872)$ events
per year, thereby enabling precision Argand plots, studies of rare (including nonhidden-charm)
decays, precise mass and width measurements, etc.
With close cooperation
between high-precision experiments at the STCF and the LQCD community, a robust,
first-principles understanding of the confinement of quarks and gluons will be produced in the foreseeable future.

Many perturbative QCD properties can also be tested at the STCF, such as the determination of the $R$ values at various energies and therefore also the strong interaction coupling $\alpha_s$, especially at the $\tau$ threshold, as discussed in Section~\ref{subsec:tausm}. More data from the STCF can also be used to test NRQCD predictions for charmonium production with high precision.

The STCF can also measure strong interactions at the hadron level with great precision. Some of the interesting possible measurements are the time-like nucleon and hyperon form factors, as discussed. This will greatly help to improve the currently poor understanding of the structure of nucleons.
%After 100 years of experimental studies, the structure of nucleons is still poorly understood.\cite{Nayak,Yang}
In this context, time-like measurements are expected to play an increasingly important role.
Moreover, time-like pair production measurements are not restricted to nucleons;
the form factors of all of the weakly decaying hyperons can be measured and compared, thereby opening a new, previously
unexplored dimension of investigation. Currently available
(statistically limited) time-like experiments exhibit puzzling features in their threshold cross sections
and electric and magnetic form factors. At the STCF, the time-like nucleon and hyperon
form factors will be measured for $Q^{2}$ values as high as 40~GeV$^{2}$ with precisions that match
those of existing results for the proton and neutron in the space-like region. Moreover, hyperon polarizations will enable new determinations
of their parity-violating decay asymmetries and can be used to extract the complex phases between their electric and
magnetic form factors. At the STCF, the Collins effect will be able to be measured in the inclusive production
of two hadrons at the percent level, thereby providing valuable input for the interpretation of nucleon spin-structure
measurements at high-energy electron--ion colliders that are currently under construction in China
and the U.S.

\noindent
{\bf  Electroweak interaction, flavor physics and $CP$ violation:}
The electroweak interaction based on $SU(2)_L\times U(1)_Y$ is an integral part of the SM.
There are many free parameters in the theory. It is fair to say that the underlying structure of the SM is flavor physics;
most of the 19 fundamental parameters of the SM are the masses of the quarks and leptons and their flavor mixing angles.
As discussed in Section~\ref{sec:tau}, the large number of $\tau$ pairs produced at the STCF will provide much more accurate measurements of the electroweak couplings and test the universality of the weak interaction. Many other SM parameters can also be measured to further test the SM.

One of the most important tests of the SM is to see how well the CKM mechanism works. The most general tests of the SM that involve the CKM matrix are to confirm its unitarity and
the internal consistency of its elements. The SM coupling strengths for the $u\leftrightarrow s$
and $c\leftrightarrow d$ transitions are both equal to $G_{F}|\sin\theta_{c}|$, with a small, well-understood
$\mathcal{O}(10^{-4})$ correction. Here, $G_{F}$ is the Fermi constant, and $\theta_{C}$ is the Cabibbo angle.
Any significant difference in $|\sin\theta_{c}|$ extracted from different quark transitions
would be an unambiguous sign of new physics.
%Figure~\ref{fig}(b) summarizes the current status of
%$|\sin\theta_{c}|$ derived from different transitions,\cite{CKM1,CKM2,CKM3,CKM4,CKM5} where the mutual agreement
%is poor, the so-called $Cabbibo$ $angle$ $anomaly$.

The $\sim$0.2\% precision from nuclear $\beta$
($|V_{ud}|$) and kaon ($|V_{us}|$) decays is more than an order of magnitude better than the precision from
$D_s$ ($|V_{cs}|$) and $D$ ($|V_{cd}|$) decays, which is
$\sim$3\%, based on statistics-limited BESIII measurements of the $D_s^{+}\to\mu^+\nu$,
$D^+\to\mu^+\nu$ and $D^0\to K^-(\pi^-)\ell^+\nu$ decays. The clean environments for $D$ and $D_s$
mesons produced by $\psi(3770)\to D\bar{D}$ and $\psi(4160)\to D_s^*\bar{D}_s$, respectively, which are
unique to an STCF-like facility, are especially well suited for low-systematic-error $c$-quark transition
measurements. Year-long STCF runs at 3.773~GeV and 4.160~GeV would reduce the errors on $c$-quark-related
determinations of $|\sin\theta_c|$ to the 0.1$\sim$0.2\% level and match those from $\beta$ and kaon decays. The STCF will produce a large number of $\tau$ pairs, allowing more precise measurements of $\tau \to K^- \nu_\tau$ and $\tau \to \pi^- \nu_\tau$ to be carried out. This will also enable the determination of the value of $\theta_c$.

Searching for a non-SM source of $CPV$ is a promising strategy for uncovering signs of physics beyond
the SM. To date, intensive investigations of $CPV$ with beauty and charmed mesons and in the neutral kaon
system have not revealed any deviations from expectations based on the Kobayashi--Maskawa mechanism.
The good agreement between the SM calculation of $\epsilon '/\epsilon$ and its measured value
restricts the level of non-SM $CPV$ for non-SM parity-changing decays involving $s$ quarks to
$<6\times 10^{-5}$ but allows for asymmetries at $\mathcal{O}(10^{-3})$ in hyperon
parity-conserving decay processes such as $\Lambda\to p \pi^-$ and $\Xi^-\to \Lambda \pi^-$.
At the STCF, using quantum-entangled, coherent $\Lambda\bar{\Lambda}$ and $\Xi^-\bar{\Xi}^+$ pairs produced via
$J/\psi$ decays, a comprehensive search for non-SM CPV asymmetries could probe the sensitivity level between
$10^{-3}$ and the SM level of $\sim6\times10^{-5}$. Notably, the sensitivities for
CPV in hyperon decays depend linearly on the hyperon polarization, and thus, a future option for an $\sim$80\% polarized
$e^-$ beam at the STCF would boost the discovery potential for hyperon CPV by more than an order of magnitude.

Various CPV processes involving $\tau$ leptons have been discussed.
A particularly interesting one is CPV in $\tau\to K_{S}\pi\nu$. Until now, the sensitivity for CPV has been at only the
$\mathcal{O}(1\%)$ level when studying  $\tau\to K_{S}\pi\nu$ decays using unpolarized $\tau$ leptons.
The corresponding CPV sensitivity for one year of STCF data at $E_{\rm c.m.}=4.26$~GeV will be
$\mathcal{O}(10^{-4})$, which is the level expected for the well-understood influence of SM CPV effects in the
neutral kaon meson system. The future polarized $e^-$ beam option would enable unambiguous probes
for new-physics sources of CPV in $\tau$-lepton decays to final states that do not contain neutral kaons, such as
$\tau^-\to \pi^-\pi^0\nu$.
Searches for $CP$ violation in heavy hadron decays and $\eta/\eta' \to \pi \pi$ decays could also be carried out at the STCF.


%a different approach,
%i.e., one based on high-statistics studies of decays to final states
%that do not contain neutral kaons and with strict controls on systematic errors, is needed.
%Such a method for sensitive searches for non-SM CP violations in semileptonic $\tau\to\pi\pi^{0}\nu$
%decays of polarized $\tau$ leptons produced near threshold in an $e^{+}e^{-}$ collider was presented
%in detail by Tsai in ref.~\cite{tau4}. Here he established that a threshold facility like STCF with a
%polarized $e^{-}$ beam would be ideally well suited for high sensitivity searches fo CP-violating
%asymmetries.

\noindent
{\bf Other searches for new physics beyond the SM:}
With high luminosity, a clean collision environment and excellent detector performance, the STCF will have great potential
to search for rare and forbidden decays and will serve as a powerful instrument for other investigations of physics beyond the SM. Such searches can be classified into three categories:
(1) decays via a FCNC process,
(2) decays with LFV and
(3) decays with LNV.
The STCF will support searches for $\tau$-lepton LFV and LNV decays with sensitivities of $10^{-8}$ to
$10^{-9}$. In addition, as discussed in Section~\ref{sec:newphys}, it will serve as a platform to search for proposed new low-mass particles such as dark photons, light scalars and millicharged particles.

The physics program at the STCF is a multidimensional program. We emphasize that the unprecedentedly high luminosity in the energy region of $2\sim 7$ GeV offers great physics potential, enabling us to develop a much more in-depth understanding of the challenges facing the SM and hopefully providing some clues or solutions for overcoming them. It will play a crucial role in leading the high-intensity frontier of elementary particle physics worldwide.

%\input{07_01_Darksector}
%\input{07_02_MilliCharge}

%\input{07_ref_NewPhys}
