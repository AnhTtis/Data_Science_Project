\subsection{Spectroscopy}

The spectrum of the light hadrons serves as an excellent probe of nonperturbative QCD~\cite{Brambilla:2014jmp,Meyer:2010ku,Crede:2008vw,Klempt:2007cp,Amsler2004,Godfrey:1998pd}. The complexity of strong QCD manifests itself in hadrons, their properties and their internal structures. The quark model suggests that mesons are formed from a constituent quark and antiquark and that baryons consist of three such quarks. QCD, however, allows a richer spectrum of color singlets that takes into account not only the quark degrees of freedom but also the gluonic degrees of freedom. Additionally, excited and exotic hadronic states are sensitive to the details of quark confinement, which is only poorly understood within QCD.

%Lattice-QCD calculations of both the baryon and the meson spectra have made tremendous progress
%and have now reached a maturity so that they can provide some guidance in the experimental efforts.

%The mass spectrum of hadrons is clearly organized according to flavor content, spin, and parity.
In intermediate- and long-distance phenomena such as hadron properties, the full complexity of QCD emerges, which makes it difficult to understand hadronic phenomena at a fundamental level.
%However, many states are not well established and evidence remains vague,
%particularly in the baryon sector.
Based on quark model expectations, the experimental meson spectrum appears to be overpopulated, which has inspired speculation about states beyond the $q\bar{q}$ picture, whereas fewer states have been observed in the baryon spectrum, which has led to the problem of the so-called missing baryon resonances. Even for several well-established baryons, their spins and parities have never been measured and are based merely on quark model assignments, particularly for resonances involving strange quarks. Whether glueballs made of multiple gluons and hybrids made of gluons and quarks, as predicted by LQCD~\cite{Lee:1999kv,Bali:1993fb, Morningstar:1997ff, chen:2005mg, Lacock:1996ny, MILC:1997usn, Dudek:2011tt, Dudek:2013yja}, truly exist is still an open question. These are some of the important issues limiting the current understanding of hadronic physics.
Another critical and poorly studied sector is the light vector mesons, especially strangeonium states, which can provide critical information on the connection between the light quark and heavy quark sectors. Hadron production via $e^+e^-$ collisions with ISR~\cite{Druzhinin:2011qd} plays an important role. Current and future experiments present a real opportunity for a dramatic improvement in our knowledge of the spectrum.

At present, BESIII remains unique in its ability to study and search for QCD exotics and new excited baryons~\cite{bes3yellowbook}, as its high-statistics data sets of charmonia provide a gluon-rich environment with clearly defined initial- and final-state properties~\cite{beswhite}.
%Recent progress and future plan of light hadron physics at BESIII has been reviewed in ~\cite{bes3whitepaper}.
At the STCF, many more data sets of charmonia will be obtained. The expected high-statistics data samples for $J/\psi$ and $\psi(3686)$ decays, including both hadronic and radiative decay channels,
%Editor: Please ensure that the intended meaning has been maintained in the above edit.
will provide an unprecedented opportunity to obtain a better understanding of the spectrum of light hadrons, their properties and their couplings to all the channels in which they appear and, from these, to learn about the composition of these states, including glueballs and hybrid states.
An interesting example is the study of the glueball nature of some states using data from the STCF. The production properties suggest a prominent glueball nature of $f_0(1710)$ and a flavor octet structure of $f_0(1500)$~\cite{beswhite}. However, the scalar meson sector is the most complex one, and the interpretation of the
states' natures and their nonet assignments are still very controversial. There is no question that more states than can be
accommodated by a single meson nonet have been found. However, the nature of all of these states is still open
for discussion. At the STCF, a year of operation will provide $\sim$3~T $J/\psi$ and $\sim$500~B $\psi(3686)$ events at their peak cross sections for
exploring light hadron physics. Traces of glueballs and hybrid states may be found in some more confirmed ways. Measurements of electromagnetic couplings to glueball candidates would be extremely useful for the clarification of the nature of these states. The radiative transition rates of a relatively pure glueball would be anomalous relative to the expectations for a conventional $q\bar{q}$ state. The dilepton decay modes of the light unflavored mesons are expected to provide deeper insight into the meson structure, allowing the transition form factors to be measured in the time-like region. A glueball should have suppressed couplings to $\gamma\gamma$, which can be measured at the STCF. There has been a long history of experimental searches for the spin-exotic states $J^{PC}$ quantum numbers that can not be formed by a simple quark-antiquark pair. Recently, an isoscalar resonance with exotic $1^{-+}$ quantum numbers, $\eta_1(1855)$, has been observed by BESIII experiment~\cite{BESIII:2022riz}. At the STCF, further studies with more production mechanisms and decay modes will help clarify the nature of the
$\eta_1(1855)$. In addition, more precise study on the isoscalar $1^{-+}$ $\eta_1(1855)$, combined
with previous and also future measurements of the isovector $\pi$ states, will provide critical clue of searching for other partners of exotic supermulitplets.

Nevertheless, the extraction of resonance properties from experimental data is far from straightforward; the resonances tend to be broad and plentiful, leading to intricate interference patterns, or buried under a background in the same and other waves. The key to success lies in high statistical precision complemented by sophisticated analysis methods. Partial
wave or amplitude analysis (PWA) techniques~\cite{Battaglieri:2014gca} are the state-of-the-art way to disentangle the contributions from individual, even small, resonances and to determine
their quantum numbers. Nevertheless, the extremely high statistics at the STCF will present new challenges for
data handling and processing. High-performance computing harnessing heterogeneous acceleration (e.g., Ref.~\cite{Berger:2010zza}) will be a key requirement. The correct analytical properties of the amplitude will be essential for extrapolation from the experimental data to the complex plane to determine the pole positions. A key component of the necessary PWA will be close cooperation between experimentalists and theorists.


%}



%The subchapter on Spectroscopy is rather inconcrete, at least a few specic ex-
%amples should be given. It is also necessary to explain what analysis methods
%are planned to use to process very large samples of the J/psi  and psi(2S) decays.
%Here it would be useful to mention that in addition to charmonium decays a
%copious source of light mesons is provided by ISR resulting in dozens of nal
%states, which might be very simple like pi+pi-�� [37] or multihadronic with a num-
%ber of final pions, kaons and eta mesons



%Hadron Production via e+e- Collisions with Initial State Radiation ~\cite{Druzhinin:2011qd}

%Simplified quark models of the proton based on three quark degrees of freedom have historically been most useful
%in predicting the spectrum of excited states. Models in which the three quarks are independent of each other predict
%a richer spectrum of states than has been observed, which is known as the issue of ``missing baryons''. While models in %which two of the quarks are coupled together
%(quark-diquark models) explain the existing spectrum better, but are in disagreement with other observations on the
%structure of the proton. High statistics data samples of $J/\psi$ and $\psi(3686)$ decays provide
%an unprecedented opportunity to obtain a better understanding of the properties of excited baryons.

%In the meson sector,
%nearly all the observed states can be explained as simple $q\bar{q}$ systems. Within QCD, one of the perplexing issues %has been the existence of gluonic excitations. A long-standing goal of
%hadronic physics has been to understand what¡¯s the role of gluonic excitation and
%how does it connect to the confinement.
%How might the gluon-gluon
%interaction give rise to physical states with gluonic excitations (glueballs or hybrids)? The primary goal of the experimental %efforts is to conduct a definitive mapping of states in the light-meson
%sector, with an emphasis on searching for glueballs and hybrids. The radiative decays of the $J/\psi$ meson provide a
%gluon-rich environment and are therefore regarded as one of the most
%promising hunting grounds for glueballs. Isoscalar hybrids is also expected to be largely produced in the $J/\psi$ radiative %decays.

%As discussed in the physics program of BESIII~\cite{bes3yellowbook} ,
%At present BESIII remains unique for studying and searching for QCD exotics and
%new excited baryons~\cite{bes3yellowbook}, as its high-statistics data sets of charmonia provide a gluon rich
%environment with clearly defined initial and final state properties. Recent progress and future plan of light hadron
%physics at BESIII has been reviewed in ~\cite{bes3whitepaper}. With ultimately high statistics of charmonia at a
%super tau charm factory, there're great opportunities to further map out light mesons and baryons as complete,
%as precise as possible.



%The dilepton decay modes of the light unflavored mesons give a deeper insight into meson structure,
%allowing to measure transition form factors at the time-like region. In the baryon sector, the first step is still to establish the %spectrum of nucleons and hyperons. The fundamental symmetries could be addressed with the accumulation of more %data. New probes with high precison measurement will be enabled, such as radiative transitions, form factors, which will %provide critical information of the internal structure of  baryon excitations.
