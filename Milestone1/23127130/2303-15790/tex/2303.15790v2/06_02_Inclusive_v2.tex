\subsubsection{Inclusive production of a single hadron}

For a sufficiently large $\sqrt{s}$, the inclusive production of a single hadron in $e^+e^- \to h +X$ can be predicted from QCD
via the QCD factorization theorem \cite{Book}:
\begin{eqnarray}
\frac{d\sigma (e^+ e^- \rightarrow h +X)} {d z}  &=& \sum_{a=q,\bar q ,g} \int \frac{d\xi}{\xi} H_a (\frac{z}{\xi},Q^2,\mu^2) D_{a\rightarrow h} (\xi,\mu^2)
\nonumber\\
 &=& \sum_q \sigma (e^+ e^- \rightarrow q\bar q) \biggr ( D_{q\rightarrow h} (z) +   D_{\bar q\rightarrow h} (z) \biggr ) + {\mathcal O}(\alpha_s),
\label{FF}
\end{eqnarray}
where $z$ is the fraction of the energy carried by the observed hadron $h$, the functions $H_a$ ($a=q$, $\bar q$, and $g$) can be calculated
via perturbation theory, and $D_{a\rightarrow h}$ denotes parton fragmentation functions describing the hadronization of a parton $a$ to $h$.
Eq. (\ref{FF}) is the expression from QCD for collinear factorization.
The fragmentation functions are universal for any process in which QCD factorization is applicable. Extracting fragmentation functions
at rather low energy, such as the energy region of the STCF near 4--5~GeV, is especially important because with these extracted fragmentation
functions, it is possible to test their energy evolution from a rather low energy scale to high energy scales.
