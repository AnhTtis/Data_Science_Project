\subsubsection{Production of charmonia}

The inclusive production of a charmonium has been observed at Belle. The ratio has been measured as \cite{Belle09}
\begin{equation}
   R_{c\bar c} = \frac{\sigma (e^+ e^- \to J/\psi + c +\bar c +X)}
       {\sigma(e^+ e^- \to J/\psi  +X_{non. c\bar c} ) }   \approx 0.63.
\end{equation}
This is in conflict with theoretical expectations. Some progress in theoretically explaining
this result has been made by including various higher-order corrections. Although the experimental result can be explained by adding one-loop corrections \cite{MZC1,MZC2,BGJW1,BGJW2}, the outcome may be not consistent. If one includes the so-called color-octet contributions estimated
from the hadroproduction of $J/\psi$, there is still conflict between experiment and theory (see also \cite{REVW}).
Belle has also observed the exclusive production of double charmonia, $e^+ + e^-
\to J/\psi + \eta_c$\cite{Belle02}. Theoretically, the measured cross section is still not well explained,
even with the inclusion of two-loop predictions in the theory \cite{FJS}.
\par
With the STCF running at $\sqrt{s}$ larger than 6~GeV, it will be possible to experimentally study these production processes more precisely. This will be helpful for gaining a better understanding of production. This energy range offers a unique opportunity to study physics related to the production of two $c\bar c$ pairs. The production cross sections for $e^+e^-\to J/\psi\eta_c$ and $e^+e^-\to J/\psi c\bar c$ based on the NRQCD calculations in Refs.~\cite{ZGC1,ZGC2} are shown in Fig.~\ref{fig:xsec_NRQCD}. These cross sections can be tested at the STCF.

%%%%%%%%%%%%%%%%%%% Fig 2 %%%%%%%%%%%%%%%%%%%%%%%%
\begin{figure*}[t]
	\centering
	\includegraphics[width=0.45\textwidth]{Figs_02_CharmoniumXYZ/ee2psietac}~~
	\includegraphics[width=0.45\textwidth]{Figs_02_CharmoniumXYZ/ee2psicc}
\vspace{0cm}
\caption{Cross sections for $e^+e^-\to J/\psi\eta_c$ (left) and $e^+e^-\to J/\psi c\bar c$ (right) as calculated using NRQCD with the charm quark mass fixed at 1.5~GeV. The solid and dashed curves represent the results from the next-to-leading-order and leading-order calculations, respectively.
		\label{fig:xsec_NRQCD}}
\end{figure*}
%%%%%%%%%%%%%%%%%%%%%%%%%%%%%%%%%%%%%%%%%%%%%%%%%%

% *****{\color{red} Figure 5 already appear early in section 2.3. Need to make coherent display. May be combine the materials for this two small sections?}*****
