\subsection{Precision tests with light hadrons}
\subsubsection{Light meson decays}



At the STCF, it is expected that approximately $3.4\times10^{12}$ $J/\psi$ events will be collected per year; thus, the STCF will be a factory for light mesons due to their high production rates in $J/\psi$ decays. Taking $\eta/\eta'$ mesons as an example,
Table~\ref{tab:eta} indicates that approximately $10^9$ $\eta/\eta'$ events could be produced through $J/\psi$ radiative or hadronic decays. Accordingly, the STCF will offer an unprecedented opportunity to explore
light meson decays for a variety of physics at low energy scales, including precision tests
of effective field theories, investigations of the quark structure of the light mesons, tests of
fundamental symmetries, and searches for new particles.

At low energies,
nonperturbative QCD calculations are usually performed using an effective field theory called chiral
perturbation theory (ChPT). High-quality and precise measurements of low-energy hadronic processes
are necessary to verify the systematic expansion of ChPT. Thus, studies of light meson
decays can provide important guidance for our understanding of how QCD works in the nonperturbative
regime. In particular, the $\eta^\prime$ meson, which is much heavier than the
Goldstone bosons of chiral symmetry breaking, plays a special role
as the predominant singlet state arising from
the strong axial $U(1)$ anomaly. The decays of light mesons, such as the $\eta/\eta^\prime$ and $\omega$, as well as their excited states can provide useful information about chiral perturbation theory through hadronic decays~\cite{Gasser:1983yg, Kaiser:2000gs} and
anomalous Wess--Zumino--Witten (WZW) processes~\cite{Wess:1971yu,Witten:1983tw, Bijnens:1989jb}, such as ${\eta}^{\prime} \to \rho^0 \gamma$ and ${\eta}^{\prime} \to \pip\pim\pip\pim$. $\eta/\eta'$ decays can also provide model-independent information about low-energy meson interactions, such as vector meson dominance (VMD)~\cite{Sakurai:1960ju,Landsberg:1986fd}.
The $\eta/\eta^\prime\to\gamma\gamma\pi^0$ decays are of particular interest for tests of ChPT at the two-loop level. Since light vector mesons play a critical role in these models, the dynamical role of the vector mesons must be systematically included in the context of either VMD or the Nambu--Jona--Lasinio model to reach a deeper understanding of these decays.


%In this case, the expected $\eta/\eta^\prime$ decays could reach about $10^9$. Therefore STCF is also a of light mesons. In Table.~\ref{tab:eta}, $\eta/\eta'$ events per year through several $J/\psi$ decay channels are %listed.
%Both $\eta$ and $\eta^\prime$ mesons are very well suited for tests of the SM.





The $\eta/\eta^\prime\to\gamma l^+l^-$ ($l=e,\mu$) Dalitz decays, where the lepton pair is formed through the internal conversion of an intermediate virtual photon and
the decay rates are modified by the electromagnetic structure arising at the vertex of the transition, are of special interest. Deviations of measured quantities from their QED predictions are usually described in terms of a
time-like transition form factor, which, in addition to being an important probe into the meson's structure~\cite{Landsberg:1986fd}, has an important role in the evaluation of the hadronic light-by-light contribution to the muon anomalous magnetic moment (see the nice review in~\cite{Aoyama:2020ynm} for details).
In addition, using the expected large data sample to be collected at a center-of-mass energy above the $J/\psi$ peak at the STCF, measurements of the space-like transition form factors in the decay
$e^+e^-\rightarrow e^+e^- \pi^0(\eta,\eta^\prime)$ via $\gamma\gamma$ interactions in the transfer momentum ($Q^2$) range of $[0.3, 10]$ GeV/c$^2$ will be feasible. The measured space-like transition form factors will uniquely cover the $Q^2$ range that is relevant to the hadronic
light-by-light correction for the evaluation of the muon anomalous moment.



%Although $\eta/\eta^\prime$ cannot be produced directly from $e^+e^-$ collisions, their high production rate in $J/\psi$ decays provide an efficiency source of a great number of $\eta/\eta^\prime$ mesons. The STCF is %designed to have a luminosity of $~10^{35}$
%$ cm^{-2}s^{-1}$ and the goal is to have at least $10^{12}$ $J/\psi$ events produced per year.
%In this case, the expected $\eta/\eta^\prime$ decays could reach about $10^9$. Therefore STCF is also a of light mesons. In Table.~\ref{tab:eta}, $\eta/\eta'$ events per year through several $J/\psi$ decay channels are %listed.
%Both $\eta$ and $\eta^\prime$ mesons are very well suited for tests of the SM.



%The neutral members of the ground state pseudoscalar nonet, both $\eta$ and $\eta^\prime$
%play an important role in understanding low energy QCD. Decays of the $\eta/\eta^\prime$ probe a wide variety of
%physics issues {\it e.g.} $\pi^0-\eta$ mixing, light quark masses
%and pion-pion  scattering.
%In addition,
%being the eigenstates of the C, P and CP operators, the decays of $\eta/\eta^\prime$ offer a unique opportunity for testing these  fundamental
%discrete symmetries. Their rare and forbidden decays can also provide information about new physics beyond SM.
%\cite{UA1,TH01}.



%{\color{red}

The $\eta$ and $\eta^\prime$ mesons are eigenstates of $P$, $C$ and $CP$ whose strong and electromagnetic decays are either anomalous or forbidden at the lowest order by $P$, $C$, $CP$ and angular momentum conservation. Therefore, their decays provide a unique laboratory for testing the fundamental symmetries in flavor-conserving processes, as extensively reviewed in Ref.~\cite{Gan:2020aco}.

A straightforward way to test these symmetries is to search for $P$- and $CP$-violating
$\eta/\eta^\prime$ decays into two pions. In the SM, the branching fractions for these modes are very tiny~\cite{Jarlskog:2002zz}, but they may be enhanced by $CP$ violation in the extended Higgs sector of the electroweak theory~\cite{Jarlskog:1995gz}. Therefore, an observation of the $\eta\rightarrow 2\pi$ decay, with a rate considerably higher than that quoted above, would imply new sources of $CP$ violation beyond the SM.
Experimentally, $\eta/\eta^\prime\rightarrow l^+l^- \pi^0$ decays could be used to test charge-conjugation invariance. In the SM, this process can proceed via a two-virtual-photon exchange, whereas a one-photon exchange would violate $C$-parity. Within the framework of the VMD model, the most recent predictions~\cite{Escribano:2020rfs} for the branching fractions
are on the order of $10^{-9}$ for $\eta\rightarrow l^+l^- \pi^0$ and $10^{-10}$ for $\eta^\prime\rightarrow l^+l^- \pi^0 (\eta)$. Thus,
a significant enhancement of the branching fractions exceeding the predictions of the two-photon model may be indicative of $C$ violation.
With the expected $3.4\times 10^{12}$ $J/\psi$ events at the STCF, the branching fractions can reach a new high precision on the order of $10^{-9}$, making the investigation of these rare decays very promising.
Many other decays of the $\eta/\eta'$ mesons, as summarized in Table~\ref{tab:rareetap} for $\eta^\prime$ decays, are also useful for tests of the SM.
For example, the $\eta\rightarrow \mu^+\mu^-$ and $\eta\rightarrow e^+e^-$ decays are of interest when
searching for nonstandard physics.
Within the framework of the SM, the decays are dominated by a two-photon intermediate state, which suppresses the
branching ratios. However, beyond-the-SM interactions, such as leptoquark exchange, can enhance the branching ratios. Therefore,
larger-than-expected measurements will provide information about non-SM interactions, and the same can be concluded for their flavor-violating counterparts, $\eta\rightarrow e^\pm \mu^\mp$.







\begin{table}[htbp]
\begin{center}
%\centering
 \caption{\label{tab:eta} The expected numbers of $\eta/\eta^\prime$ events as calculated from the $3.4\times 10^{12}$ $J/\psi$ events anticipated to be produced at the STCF per year.}
 \begin{tabular}{l c c c }\hline\hline
        Decay mode    &       $\mathcal{B}$ ($\times 10^{-4}$) ~\cite{PDG}   & $\eta/\eta^\prime$ events \\ \hline
      $J/\psi\rightarrow\gamma\eta^\prime$ &$52.1\pm1.7$ &    $1.8\times 10^{10}$ \\ \hline
          $J/\psi\rightarrow\gamma\eta$ &$11.08\pm0.27$ &$3.7\times 10^9$\\  \hline
       $J/\psi\rightarrow\phi\eta^\prime$ & $7.4\pm0.8$ &    $2.5\times 10^9$ \\ \hline
          $J/\psi\rightarrow\phi\eta$ & $4.6\pm0.5$&   $1.6\times 10^9$ \\  \hline
                  \end{tabular}
                  \end{center}
\end{table}

\begin{table*}
\centering
  \caption{The sensitivities to rare and forbidden $\eta^\prime$ decays. The expected sensitivities are estimated by considering
the detector efficiencies for different decay modes at the STCF. We
assume that there is no background dilution and that the observed number of signal
events is zero. The STCF limits are given at the 90\% confidence
level. } {\begin{tabular}{llccc}
\hline
Decay mode &
% Best upper limits (measurements$^*$) & BES-III limit \\
Best upper limit  &STCF limit & Theoretical  & Physics \\
&90\% CL  &$(3.4\times 10^{12}$ $J/\psi$ events)  & prediction  & \\
\hline
$\eta^\prime \rightarrow e^+e^- $ &  $ 5.6\times 10^{-9} $  & 1.5 $\times 10^{-10}$  & $1.1\times 10^{-10}$ & leptoquark\\
$\eta^\prime \rightarrow \mu^+\mu^-$ &  $-$  & 1.5 $\times 10^{-10}$ & $1.1\times 10^{-7}$ & leptoquark\\
$\eta^\prime \rightarrow e^+ e^- e^+ e^- $ &  $-$ & 2.4$\times 10^{-10}$  & $1\times 10^{-4}$& $\gamma^*\gamma^*$\\
$\eta^\prime \rightarrow \mu^+ \mu^- \mu^+ \mu^- $ &  $-$  &2.4$\times 10^{-10}$ &  $4\times 10^{-7}$ &$\gamma^*\gamma^*$\\
%$\eta^\prime \rightarrow \pi^+\pi^- e^+ e^- $ &  $6.0\times 10^{-3}$  & 1.4$\times 10^{-7}$  & VMD, TFF\\
%$\eta^\prime \rightarrow \pi^+\pi^- e^+ e^-$ &  $(24^{+13}_{-10})\times 10^{-4}$  & 1.4$\times 10^{-7}$  \\
%$\eta^\prime \rightarrow \pi^+ \pi^- \mu^+ \mu^- $ &  $2.9\times 10^{-5}$  & 8$\times 10^{-10}$  & $2.2\times 10^{-5}$ & VMD, TFF\\
$\eta^\prime \rightarrow \pi^0 \mu^+ \mu^- $ &  $6.0 \times 10^{-5}$  & 2.4$\times 10^{-10}$  & &$C$ violation\\
$\eta^\prime \rightarrow \pi^0 e^+ e^- $ &  $1.4 \times 10^{-3}$  & 2.4$\times 10^{-10}$  & &$C$ violation\\
%$\eta^\prime \rightarrow \pi^0\gamma $ &  $-$  &7$\times 10^{-10}$  & &angular momentum \\
$\eta^\prime \rightarrow \pi^0\pi^0$ &  $9.0 \times 10^{-4}$  & 2.9$\times 10^{-9}$ & &$CP$ violation \\
$\eta^\prime \rightarrow \pi^+\pi^-  $ &  $2.9 \times 10^{-3}$  & 1.5$\times 10^{-10}$  & &$CP$ violation\\
$\eta^\prime \rightarrow \mu^+ e^- + \mu^- e^+ $ &  $4.7 \times 10^{-4}$  &1.5$\times 10^{-10}$ && LPV\\
$\eta^\prime \rightarrow $ invisible  &  $5.3\times10^{-4}$  & 3.3$\times 10^{-8}$  & &Dark matter\\
$\eta^\prime \rightarrow \eta e^+e^- $ &  $2.4 \times 10^{-3}$     & 5.9$\times 10^{-10}$ & &$C$ violation\\
$\eta^\prime \rightarrow \eta \mu^+\mu^- $ &  $1.5 \times 10^{-5} $     &  5.9$\times 10^{-10}$ & &$C$ violation\\
\hline
\end{tabular}\label{tab:rareetap}}
\end{table*}


In addition to the $\eta/\eta^\prime$ decays, the high production of other light mesons, $\omega$, $a_0(980)$, $f_0(980)$, and $\eta(1405)$, as well as other excited states, is also an important source for exploring many aspects of particle physics at low energy. The $\omega\to \pi^+\pi^-\pi^0$ decay could be employed to investigate the $\omega$ decay mechanism by comparing a high-statistics Dalitz plot density distribution with the predictions within the dispersive theoretical framework~\cite{Niecknig:2012sj,Danilkin:2014cra}; moreover, the $a_0(980)$--$f_0(980)$ mixing is sensitive to the quark structure of the light scalars, and the $\eta(1405)\to 3\pi$ process may help reveal the
the well-known triangle singularity mechanism~\cite{triangle}.

In general, despite the impressive progress that has been achieved in recent years, many light meson decays are still unobserved and need to be explored.
With the advantages of high production rates and excellent performance at the STCF, the highly abundant and clean samples of $e^+e^-$ annihilations will bring the study of light meson decays into a precision era, will certainly play an important role in the further development of chiral effective field theory and lattice QCD, and will make significant contributions to the understanding of hadron physics in the nonperturbative regime.

\subsubsection{Hyperon decays}
The ongoing experimental studies of $CP$-symmetry violation in particle decays aim to find effects
that are not expected in the Standard Model (SM) such that new
dynamics is revealed. The existence of $CP$ violation in kaon and beauty
meson decays is well established \cite{Christenson:1964fg,
Aubert:2001nu,Abe:2001xe}. The first observation of $CP$ violation in charm mesons was
reported in 2019 by the LHCb experiment \cite{Aaij:2019kcg}, but thus far, there is no evidence in the baryon sector. All the observations are consistent with the SM expectation. Baryons with strange quark (hyperon) decays offer promising possibilities for searches for new $CP$-violating effects since they involve $p$-wave decay amplitudes, which neutral kaon decays do not \cite{Donoghue:1985ww}.
%Editor: Please ensure that the intended meaning has been maintained in the above edit.
A possible signal of $CP$ violation would be a difference
in the decay distributions between charge-conjugated decay modes. The
main decay modes of the ground-state hyperons are weak hadronic transitions
into a baryon and a pseudoscalar meson, such as $\Lambda\to p\pi^-$
(${\cal B}\approx64\ \%$) and $\Xi^-\to\Lambda\pi^-$
(${\cal B}\approx100\ \%$) \cite{PDG}. They involve two amplitudes: one for a parity-conserving decay to the relative $p$ state and one for a parity-violating decay to the $s$
state. The angular distribution and polarization of the daughter
baryon are described by two decay parameters:
$\alpha=2{\rm Re}(s^*p)/(|p|^2+|s|^2)$ and $\phi={\rm
arg}((s-p)/(s+p))$.  Here, we denote the decay parameters $\alpha$ for $\Lambda\to p\pi^-$ and
$\Xi^-\to\Lambda\pi^-$ by $\alpha_\Lambda=0.750(10)$~\cite{Ablikim:2018zay} and $\alpha_\Xi=-0.392$,
respectively. In the $CP-$symmetry-conserving limit, the parameters
$\alpha_D/\alpha_{\bar D}$ and $\phi_D/\phi_{\overline{D}}$ for the charge-conjugated $D/\bar D$ decay modes have the same absolute values but opposite signs.
The $CP$ asymmetry $A_D$ is defined as follows:
\begin{align}
  A_{D}&\equiv\frac{\alpha_D+\alpha_{\bar D}}{\alpha_D-\alpha_{\bar D}}\approx -\tan(\delta_p-\delta_s)\sin(\zeta_p-\zeta_s),\label{eq:AD}
%  &\approx -\frac{\sqrt{1-\alpha_D^2}}{\alpha_D}\sin\phi_D\sin(\zeta_p-\zeta_s),
\end{align}
where $\delta_p-\delta_s$ is the strong $(p\!-\!s)$-wave phase difference in the decay due to final-state interaction and $\zeta_p-\zeta_s$ is the weak $CP$-violating phase difference.


The best limit for $CP$ violation in the strange baryon sector was obtained by comparing the complete
$\Xi^-\to\Lambda\pi^-\to p\pi^-\pi^-$ and c.c. decay chains of unpolarized $\Xi$ baryons at
the dedicated HyperCP (E871) experiment~\cite{Holmstrom:2004ar} by determining the
asymmetry
$A_{\Xi\Lambda}=(\alpha_\Lambda\alpha_\Xi-\alpha_{\bar\Lambda}\alpha_{\bar\Xi})/(\alpha_\Lambda\alpha_\Xi+\alpha_{\bar\Lambda}\alpha_{\bar\Xi})\approx A_{\Xi}+A_{\Lambda}$.
The result, $A_{\Xi\Lambda}=(0.0\pm5.1\pm4.7)\times10^{-4}$, is consistent with the SM prediction: $\left|A_{\Xi\Lambda}\right|\le
5\times10^{-5}$~\cite{Tandean:2002vy}. Moreover, an improved
preliminary HyperCP result presented at the BEACH
2008 Conference suggests a large asymmetry value of $A_{\Xi\Lambda}=(-6.0\pm2.1\pm2.0)\times10^{-4}$
\cite{Materniak:2009zz}. However, it is difficult to interpret this result in terms of the weak $CP$-violating phase difference. The $A_D$ asymmetries are, in general, not sensitive probes of the weak $CP$-violating phase
difference since the $\tan(\delta_p-\delta_s)$ term is very small and not well known. The values are
$-0.097(53)$ for  $\Lambda\to p\pi^-$ and $0.087(33)$ for $\Xi^-\to\Lambda\pi^-$,
as determined from the values of the $\phi_D$ decay parameters using the following relation:
\begin{equation}
   \tan(\delta_p-\delta_s)\approx -\frac{\sqrt{1-\alpha_D^2}}{\alpha_D}\sin\phi_D\ .
\end{equation}
A much more sensitive, independent determination of $\zeta_p-\zeta_s$ is obtained
by comparing the $\phi_D$ and  $\phi_{\bar D}$
parameters:
\begin{equation}
  \Delta\phi_D\equiv\frac{\phi_{D} + {\phi}_{\bar D}}{2}\approx \frac{\alpha_D}{\sqrt{1-\alpha_D^2}}\sin(\zeta_p-\zeta_s)\ . \label{eq:Dphi}
\end{equation}

\begin{table*}
\begin{tabular}{p{3cm}llrr}
\hline\hline
Decay mode&${\cal B}$ (units of $10^{-4}$)&Angular distribution&Detection&\multicolumn{1}{l}{No. of events }\\
          &&parameter $\alpha_\psi$ &efficiency& expected at the STCF\\
\hline
$J/\psi\to\Lambda\bar\Lambda$ \vphantom{$\int\limits^M$} &${19.43\pm0.03\pm0.33}$&$\phantom{-}0.469\pm0.026$&40\%&$1100\times10^6$\\
$\psi(2S)\to\Lambda\bar\Lambda$&$\phantom{0}{3.97\pm0.02\pm0.12} $&$\phantom{-}0.824\pm0.074$&40\%&$130\times10^6$\\
$J/\psi\to\Xi^{0}\bar\Xi^{0}$&$11.65\pm 0.04 $&$\phantom{-}0.66\pm 0.03$&14\%&$230\times10^6$\\
$\psi(2S)\to\Xi^{0}\bar\Xi^{0}$&$\phantom{0}2.73\pm 0.03 $&$\phantom{-}0.65\pm0.09$&14\%&$32\times10^6$\\
$J/\psi\to\Xi^{-}\bar\Xi^{+}$&$10.40\pm 0.06 $&$\phantom{-}0.58\pm0.04$&19\%&$270\times10^6$\\
$\psi(2S)\to\Xi^{-}\bar\Xi^{+}$&$\phantom{0}2.78\pm 0.05 $&$\phantom{-}0.91\pm0.13$&19\%&$42\times10^6$\\
\hline\hline
\end{tabular}
\caption[]{Branching fractions for some $J/\psi,\psi'\to B\bar B$
  decays and the estimated sizes of the data samples from the full
  data set of $3.4\times 10^{12}\ J/\psi$ and $3.2\times 10^{9}\ \psi'$
  to be collected
  by the
  STCF.  The approximate detection
  efficiencies for the final states reconstructed using the $\Lambda\to
  p\pi^-$ and $\Xi\to\Lambda\pi$ decay modes are based on the
  published BESIII analyses using partial data sets
  \cite{Ablikim:2017tys,Ablikim:2016sjb,Ablikim:2016iym}.
\label{tab:data}}
\end{table*}

With a well-defined initial state, the charmonium decay into a strange
baryon--antibaryon pair offers an ideal system for testing fundamental
symmetries. The vector charmonia $J/\psi$ and $\psi'$ can be directly
produced in an electron--positron collider with high yields and
have relatively large branching fractions into hyperon--antihyperon
pairs; see Table~\ref{tab:data}.
The potential power of such measurements was shown in a recent BESIII analysis using a
data set of $4.2\times10^{5}$  $e^+e^-\to
J/\psi\to\Lambda\bar\Lambda$ events reconstructed via the $\Lambda\to p\pi^-$ $+$
c.c. decay chain~\cite{BESIII:2018cnd}. The determination of the asymmetry parameters was possible
due to the
transverse polarization and spin correlations of the
$\Lambda$ and $\bar\Lambda$. In the analysis, the complete multidimensional
information of
the final-state particles was used in an unbinned maximum log likelihood fit
to the fully differential angular expressions from Ref.~\cite{Faldt:2017kgy}.
This
method allows direct comparison of the decay parameters of the
charge-conjugated decay modes and enables a test of the $CP$ symmetry.


In Ref.~\cite{Perotti:2018wxm}, the formalism was extended to
describe processes that include decay chains of multiple strange
hyperons, such as the $e^+e^-\to\Xi\bar\Xi$ reaction with the
$\Xi\to\Lambda\pi$ and $\Lambda\to p\pi^-$ $+$ c.c. decay sequences. The
expressions are much more complicated than the single-step weak decays
in $e^+e^-\to\Lambda\bar\Lambda$. The joint
distributions for $e^+e^-\to\Xi\bar\Xi$ allow all decay parameters to be determined
simultaneously, and the statistical uncertainties are independent of
the size of the transverse polarization in the production process.
The uncertainties of the $CP$-odd asymmetries that can be extracted from the exclusive
analysis were estimated in Ref.~\cite{Adlarson:2019jtw}.
\begin{table*}
\centering
  \begin{tabular}{lrrrrrrr}
\hline\hline
&$A_\Xi$&$A_\Lambda$&$A_{\Xi\Lambda}$&$(\zeta_p-\zeta_s)_\Xi$&$(\zeta_p-\zeta_s)_\Xi$\\
&&&&Eq.~\eqref{eq:AD}&Eq.~\eqref{eq:Dphi}\\\hline
$J/\psi\to\Lambda\bar\Lambda$ \vphantom{$\int\limits^M$}&$-$&$1.7\times10^{-4}$&$-$&$-$&$-$\\
    $J/\psi\to\Xi^-\bar\Xi^+$ ($\Delta\Phi=0$)&$2.2\times10^{-4}$&$2.1\times10^{-4}$&$2.5\times10^{-4}$&$2.4\times10^{-3}$&$6.5\times10^{-4}$\\
%$J/\psi\to\Xi^0\bar\Xi^0$ ($\Delta\Phi=\pi/2$)&2.0&3.0&5.2&2.9&15
%    &1.4&4.0&1.5&3.4&0.77&4.4&3.7&10\\
    \hline\hline
  \end{tabular}
  \caption[]{Standard errors for asymmetry parameters extracted using STCF data samples.
    The input values of the parameters
    are taken from Table~\ref{tab:data} and Ref.~\cite{Adlarson:2019jtw}.\label{tab:sigpar}}
  \end{table*}
To study the angular distribution for the
$e^+e^-\to \Xi^-\bar\Xi^+$ reaction, we fix the decay
parameters of the $\Lambda$ and $\Xi^-$ to the central values
from the PDG~\cite{PDG}. For the production process,
the
phase $\Delta\Phi$ is an unknown parameter, but the result shows almost no dependence
on this parameter, and we set $\Delta\Phi=0$. In
Table~\ref{tab:sigpar}, we report the statistical uncertainties in
the $J/\psi\to\Xi^-\bar\Xi^+$ decay.
By exploiting the spin entanglement between the $\Xi^{-}$ baryon and its antiparticle $\bar{Xi}^{+}$, 
BESIII has enabled a direct determination of the weak-phase difference, $(\zeta_p-\zeta_s)=(1.2\pm3.4\pm0.8)\times10^{-2}$rad~\cite{BESIII:2021ypr}.


The sensitivities for the $A_\Xi$, $A_\Lambda$ and $A_{\Xi\Lambda}$ asymmetries
with a data sample of $3.4\times 10^{12}$ $J/\psi$ events at the STCF (see Table~\ref{tab:data}) are given in Table~\ref{tab:sigpar}. The statistical uncertainty for the $A_{\Xi\Lambda}$ asymmetry from the dedicated HyperCP experiment will be surpassed at the STCF.
The SM predictions for the $A_{\Xi}$ and $A_{\Lambda}$ asymmetries
are $-3\times10^{-5}\le A_\Lambda\le
4\times10^{-5}$ and $-2\times10^{-5}\le A_\Xi\le 1\times10^{-5}$~\cite{Tandean:2002vy}.

Under the assumption of a value of $0.037$ for the
$\phi_\Xi$ parameter, five-sigma significance would require $3.1\times 10^5$
exclusive $\Xi^-\bar\Xi^+$ events. Reaching a statistical uncertainty
of 0.011, as in the HyperCP
experiment~\cite{Huang:2004jp}, would require $1.4\times10^{5}$
$J/\psi\to\Xi^-\bar\Xi^+$ events, while the single-cascade HyperCP result is based on
$114\times10^6$ events. In contrast, the present PDG precision of $\phi_{\Xi^{0}}$ could be achieved with only
$3\times 10^{2}$ $\Xi^{0}\bar\Xi^{0}$ events.

The sensitivities for the weak phase difference $(\zeta_p-\zeta_s)_\Xi$~\cite{Donoghue:1985ww}
using the two independent methods are also given in Table~\ref{tab:sigpar}. The SM estimate for $(\zeta_p-\zeta_s)_\Xi$ is $8\times10^{-5}$. However, it should be stressed that the SM predictions
for all asymmetries need to be updated in view of the recent and forthcoming
BESIII results on hyperon decay parameters using the collected $10^{10}$ $J/\psi$ events.
A wide range of precision $CP$ tests can be conducted
based on a single measurement. Thus, the spin-entangled cascade--anticascade system is a promising probe for testing fundamental symmetries in the strange baryon sector.

%\subsubsection{$\Lambda - \bar{\Lambda}$ oscillation in $J/\psi \to \Lambda \bar{\Lambda}$ decay}

A large data sample of $\Lambda\bar\Lambda$ from $J/\psi$ decays can also be used to study $\Lambda$--$\bar\Lambda$ oscillations.
The seesaw mechanism, as an explanation for the small neutrino masses \cite{Dutta:2005af}, predicts the existence of $\Delta(B-L)=2$
interactions and baryon--antibaryon oscillations.
To date, searches for processes that violate the baryon number by two units have been performed only in neutron--antineutron oscillation experiments~\cite{BaldoCeolin:1994jz}. Searches for $\Lambda$--$\bar\Lambda$ oscillations in the $J/\psi \to \Lambda \bar{\Lambda}$ decay at BESIII have been proposed \cite{Kang:2009xt}.
With 10 billion $J/\psi$ decay events at BESIII, the expected sensitivity of the measurement of
$\Lambda$--$\bar{\Lambda}$ oscillation is $\delta m_{\Lambda\bar\Lambda}<10^{-15}$ MeV
at the 90\% confidence level. This corresponds to a lower limit of $10^{-7}$ s on the oscillation time.
At the STCF, the expected constraint on $\delta m_{\Lambda\bar\Lambda}$ can be improved to the $10^{-17}$ MeV level or even better. This upper limit is already much larger than the lifetime of the $\Lambda$, and further significant improvements would require other approaches, such as the use of certain long-lived hypernuclei~\cite{Dalitz:1962eb1,Dalitz:1962eb2,Dalitz:1962eb3}.


\subsection{Tests of the $CPT$ invariance with $\jpsi$ decays}
While violations of $C$,~$P$,~$T$,~and~$CP$~symmetries have been well established and characterized, the validity
of $CPT$~symmetry remains intact at a high level of sensitivity. The $CPT$~theorem ~\cite{Schwinger:1951xk}
states that any quantum field theory  that is {\it Lorentz invariant}, has {\it local point-like interaction vertices},
and is {\it Hermitian} ({\it i.e.}, conserves probability) is invariant under the combined operations of
$C$~$P$~and~$T$. Since the three quantum field theories that make up the Standard Model---QED, QCD, and
Electroweak theory---all satisfy these criteria, $CPT$~symmetry has been elevated to  some kind mystical status
in particle physics.  However, there is good reason to believe that $CPT$, like all of the other discrete symmetries,
is violated at least a mass scale of ${\mathcal O}(10^{-35} m)$ {\it i.e.}, at the so-called Planck scale.

One of the requirements for a $CPT$-invariant theory is that it is {\it local}, which means that the
couplings at each vertex occurs at a single point in space-time. But theoretical physics has always had
troubles with point-like quantities. For example, the classical self-energy of the electron is
\begin{equation}
  W_{e}=\frac{e^2}{4\pi\eps_{0} r_{e}},
\end{equation}
which diverges for $r_e\rt 0$.  The {\it classical radius of the electron}, {\it i.e.}, the value of $r_e$
that makes $W_e=m_ec^2$, is $r^{\rm c.r.e.}_e= 2.8\times 10^{-13}$~cm (2.8 fermis),  which is three times the
radius of the proton, and $\sim$300 times larger than experimental upper limits on the electron radius, which
are $<10^{-16}$~cm~\cite{ZEUS:2003eqd}.  Infinities associated with point-like objects persist in quantum field
theories, where they are especially troublesome. In second-order and higher perturbation theory, all of
the diagrams that have virtual particle loops involve integrals over all possible configurations of the virtual
particles in the loops that conserve energy and momentum.  Whenever two of the point-like vertices coincide,
the integrands become infinite and cause the integrals to diverge.

In the QED, QCD and Electroweak quantum field theories that make up the Standard Model, these infinities are
removed by the well established methods of renormalization~\cite{Bethe:1947id,Dyson:1949bp,Wilson:1973jj}. In all
three of these theories, the perturbation expansions are in increasing powers of a dimensionless coupling
strength, $\alpha_{\rm QED}, \alpha_s$ and, $\alpha_{\rm EW}=\sqrt{2}M^2_WG_F/\pi$.\footnote{Specifically
  not just $G_F$, which has the dimension of mass$^{-2}$.}
As a result of this, in the renormalization procedure, relations that exist between different orders of the
perturbation expansion reduce the number of observed quantities that are needed to subtract off divergences.
In QED, for example, there are only two, the electron's mass, and charge (three, if the diagram includes
muons).   However, in quantum theories of gravity, where a massless spin=2 {\it graviton} plays the role of
the photon in QED, the  expansion constant is Newton's gravitational constant $G=\hbar c/M^2_{\rm P}$, where
$M_{\rm P}\equiv\sqrt{\hbar c/G}=1.2\times 10^{19}$~GeV is the {\it Planck mass}. Because of this, every
order in the perturbation expansion has different dimensions and, thus, needs a distinct observed quantity
is needed to carry out the subtraction at each step, which means that complete renormalization would
infinite number of observed quantities to complete the renormalization.  This means that a
$CPT$-conserving quantum theory of gravity would, in principle, be {\it nonrenormalizable}\cite{Weinberg:1980kq}.
Although difficulties associated with non-renormalizability ({\it i.e.}, higher-order perturbative effects)
will never show up at mass scales below the Planck mass, this problem demonstrates that there is nothing
especially sacred about $CPT$-invariance that prevents from being violated at a lower mass scale.  Because of
its close connection with the fundamental assumptions of the Standard Model, stringent experimental tests of
$CPT$~invariance should have high priority



\subsubsection{Kaon mixing and tests of the $CPT$~theorem}

Among the consequences of $CPT$~symmetry are that particle and antiparticle masses and lifetimes are equal.
Since lifetime differences can only come from on-mass-shell intermediate states and do not probe short distance,
high mass physics, these are unlikely to exhibit any $CPT$-violating asymmetry. Instead, the focus here is on the
possibility that particle and antiparticle masses may be different.

The particles with the best measured masses are the stable electron and proton, and, according the
PDG~2020 tables~\cite{Zyla:2020zbs}:
\begin{eqnarray}
  |m_{e^+}-m_{e^-}|&<&4\times 10^{-9}~{\rm MeV}\\
  \nonumber
  |m_{\bar{p}}-m_{p}|&<&7\times 10^{-7}~{\rm MeV}.
\end{eqnarray}
However, these limits do not provide the best tests of $CPT$; the most stringent experimental restriction on
$CPT$ violation comes from the difference between the $\Kzbar$ and $\Kz$ masses:
\begin{equation}
  |M_{\Kzbar}-M_{\Kz}|=<5\times 10^{-16}~{\rm MeV},
\end{equation}
which is $7$-$9$~orders of magnitude more strict than those from the electron and proton mass measurements
even though the value of $M_{\Kz}$ itself is only known to~$\pm 13$~keV. This is because the
Fig.~\ref{fig:k-mix_c-quark-KM} diagrams, taken together with the quantum mechanics of $\Kz$-$\Kzbar$ mixing,
maps the $M_{\Kzbar}-M_{\Kz}$ difference into the quantity $\Delta M=M_{\KL}-M_{\KS}\approx 3.5\times 10^{-12}$~MeV,
which is 14 orders of magnitude lower than $M_{\Kzbar}$ or $M_{\Kz}$ and the independent quantity 
$\Delta\Gamma=\Gamma_{\KS}-\Gamma_{\KL}\approx 7.4\times 10^{-12}$~MeV (which is, coincidently,
$\approx 2\times \Delta M$).

\begin{figure}[!bp]
\centering
\includegraphics[width=0.99\textwidth]{Figures/k-mix_c-quark-KM.pdf}
\caption{\footnotesize The box diagrams for the short-distance contributions to  $\Kz$-$\Kzbar$
mixing.}
 \label{fig:k-mix_c-quark-KM}
  \end{figure}


The mapping is done by computing the difference between the proper-time-dependence of the $K\rt\pipi$
decay rates for neutral $K$ mesons that are tagged as strangeness \Str$=+1$ and \Str$=-1$ at their time of
production ({\it i.e.}, $\tau=0$), and are denoted here as $\Kz(\tau)$ and $\Kzbar(\tau)$, respectively.
At STCF, this tagging is automatically done in $\jpsi\rt\Km\pip\Kz$ and $\jpsi\rt\Kp\pim\Kzbar$ decays,
by the sign of the charged-kaon's electric charge: a $\Km$ tags a $\Kz(\tau)$ and a $\Kp$ tags a $\Kzbar(\tau)$.
These events are quite distinct at a c.m. $\ee\rt\jpsi$ collider as shown in Fig.~\ref{fig:k0kpi-STCF}a, and occur
with a branching fraction $Bf[\jpsi\rt K^\mp\pi^\pm\Kz(\Kzbar)]=(0.56\pm 0.05)\%$, which, for $\jpsi$ decays,
is substantial. Moreover, at a c.m. collider these events are pretty much background free, the only significant
backgrounds are misidentified  $K\rt\pi^\pm\ell^\mp\nu$ decays that also depend on $\Delta M$.

\begin{figure}[!]
\centering
\includegraphics[width=0.99\textwidth]{Figures/k0kpi-STCF.pdf}
\caption{\footnotesize {\bf a)}  A simulated $\jpsi\rt K^-\pip\Kz(\tau)$; $\Kz(\tau)\rt\pipi$ event in
  the BESIII detector.
  {\bf b)} The solid circles show the proper time distribution for simulated strangeness-tagged
  $\Kz(\tau)\rt\pipi$ decays (the open circles are $\Kzbar(\tau)\rt\pipi$ decays). 
  {\bf c)} The reduced asymmetry,
  ${\mathcal A}^{\rm reduced}_{\pipi}={\mathcal A}_{\pipi}\times e^{-{\scriptstyle \frac{1}{2}}\Delta\Gamma\tau}$,
  which weights the asymmetries according to their relative statistical significance, is plotted
  for the events shown in panel~b.  Here $\phi=\phi_{\rm SW}+\delta\phi^{CPT}_{\eta}$. (Figures
  provided by Jian-Yu Zhang.) 
  }
 \label{fig:k0kpi-STCF}
 \end{figure}

Although the event shown in  Fig.~\ref{fig:k0kpi-STCF}a looks superficially like $K^\mp\pi^\pm\KS$, here
the neutral kaon is not a $\KS$ mass eigenstate with a simple exponential lifetime.  Instead, separate plots
of the proper-decay-times for simulated $K^-$-tagged $\Kz(\tau)\rt\pipi$ and $K^+$-tagged $\Kzbar(\tau)\rt\pipi$
events, shown in Fig.~\ref{fig:k0kpi-STCF}b, tell a different, and much more interesting story. Instead of
single exponential curves that decrease indefinitely with the $\KS$ lifetime, the decay curves contain
interfering $\KS$ and $\KL$ components, with interference patterns for $\Kz(\tau)$ and $\Kzbar(\tau)$ decays
that quite different. There is a pronounced asymmetry between the two modes that, according to numerous
available descriptions of the quantum mechanics of the neutral kaon system (see,
{\it e.g.}, ref.~~\cite{Schubert:2014ska}),  can be expressed as
\begin{eqnarray}
  \label{eqn:KKbar-asymm-CPT} 
   {\mathcal A}_{\pipi}&\equiv&\frac{\bar{N}(\Kzbar(\tau))-N(\Kz(\tau))}{\bar{N}(\Kzbar(\tau))+N(\Kz(\tau))}\\
   \nonumber
   &~&~~~~~~=2\Re(\eps-\delta)-2\frac{|\eta_{+-}|e^{{\scriptstyle \frac{1}{2}}\Delta\Gamma\tau}\cos(\Delta M\tau-\phi_{\rm SW}
                 +\delta\phi^{CPT}_{\eta})}{1+|\eta_{+-}|^2e^{\Delta\Gamma\tau}}.
\end{eqnarray}
Here the first term on the r.h.s. of the equation is a  small ($\sim$0.3\%) constant offset involving $\eps$, the
familiar mass-matrix parameter that characterizes the $CP$ impurities in the $\KS$ and $\KL$ mass eigenstates, and
$\delta$, which is an even smaller $CPT$-violating parameter (that is explained below). Inherent differences in the $\Kp$
and $\Km$ detection efficiencies make it impossible to measure this term with any significant precision. On the
other hand the phase of the  oscillation described by the second term, $\phi_{\rm SW} +\delta\phi^{CPT}_{\eta}$, is
not sensitive to experimental efficiencies and can, in principle, be accurately measured. Here $\phi_{\rm SW}$ is
the ``superweak phase'' and is equal to $\tan^{-1}(2\Delta M/\Delta \Gamma)$, where $\Delta M$ can be
measured from the wavelength of the oscillation, and $\Delta\Gamma$ is determined from the $\KS$ and $\KL$
decay curves.\footnote{There  is a small, ${\mathcal O}(0.03^\circ)$ theoretical correction to the
         $\phi_{\rm SW}=\tan^{-1}(2\Delta M/\Delta \Gamma)$ that is well understood~\cite{Bell:1965mn}.}
The difference between the measured phase of the eqn.~\ref{eqn:KKbar-asymm-CPT} oscillation term and 
$\phi_{\rm SW}$ is $\phi^{CPT}_{\eta}=\delta_{\perp}/\eps$, where
\begin{equation}
  \delta_{\perp}=\frac{M_{\Kzbar}-M_{\Kz}}{2\sqrt{2}\Delta M};
\end{equation}
a non-zero value would indicate the existence of a $\Kz$-$\Kzbar$ mass difference, and a violation of $CPT$..   

This translation of $M_{\Kzbar}-M_{\Kz}$ into a relation involving the $M_{\KL}$-$M_{\KS}$ is a trick that is
unique to the kaon system and is not applicable to other particles. Although the neutral $D^0$, $B^0$ and $B^0_s$
mesons mix with their antiparticles, they do not have a measurable equivalent of the kaon's $\eps$ mass-matrix
mixing parameter and, moreover, have a
large number of decay modes that are common to both of their mass eigenstates that make $CPT$-related analyses
that are easily used for neutral kaons impractical~\cite{Bell:1965mn}. As a result, tests of $CPT$~with neutral
$B$ mesons by BaBar~\cite{BaBar:2006zzh} and Belle~\cite{Higuchi:2012kx} did not place any limits on 
$|M_{\bar{B}^0}-M_{B^0}|$.

Although it is apparent from eqn.~\ref{eqn:KKbar-asymm-CPT} that the ${\mathcal A}_{\pipi}$ asymmetry increases
with the proper-time $\tau$, the statistical precision of measurements rapidly decreases with $\tau$. Because
of this, the CLEAR group, which measured this asymmetry in $\bar{p}_{\rm stop}p\rt K^\mp\pi^\pm\Kz(\Kzbar)$ events
produced in the annihilation of stopped antiprotons~\cite{Apostolakis:1999zw}, suggested that an alternative
{\it reduced asymmetry}, defined as
\begin{equation}
  {\mathcal A}^{\rm reduced}_{\pipi}={\mathcal A}_{\pipi}\times e^{-{\scriptstyle \frac{1}{2}}\Delta\Gamma\tau},
\end{equation} 
be used for display since it emphasizes the importance of the higher-statistics, low and intermediate
decay-time measurements that provide the bulk of the measurement sensitivity. The reduced asymmetry between
the decay curves for $\Kz(\tau)\rt\pipi$ and $\Kzbar(\tau)\pipi$ events shown in Fig.~\ref{fig:k0kpi-STCF}b,
is plotted in Fig.~\ref{fig:k0kpi-STCF}c.

The  simulated data shown in Figs.~~\ref{fig:k0kpi-STCF}b~\&~c correspond to 3.8B~tagged $K\rt\pipi$
decays that are almost equally split between $\jpsi\rt\Km\pip\Kz$ and $\jpsi\rt\Kp\pim\Kzbar$ that were generated 
with $\phi^{CPT}_{\eta}=0$ and $\phi_{\rm SW}=43.4^\circ$. This corresponds to what one would expect for a
total of $10^{12}$~$\jpsi$ decays in a detector that covered a $|\cos\theta|\le 0.85$ solid angle and was
otherwise almost perfect.  The red curve in the figure is the result of a fit to the data that determined
\begin{equation}
  \phi_{\rm SW}+\phi^{CPT}_{\eta} = 43.51^\circ\pm 0.05^\circ~~~~
          \Rightarrow~~~~\phi^{CPT}_{\eta} < 0.15^\circ~~(90\%~{\rm C.L.}),
\end{equation}
where the errors are statistical only. To date, the best existing measurements are a 1999 result from the
CPLEAR~$\bar{p}_{\rm stop}p$ experiment at CERN~\cite{Apostolakis:1999zw} that used a sample of $\sim$70M tagged
$\Kz(\tau)\rt\pip
i$ and $\Kzbar(\tau)\rt\pipi$ events and a 1995 result from Fermilab experiment
E773~that used $\sim$2M~$K\rt\pipi$ and $\sim$0.5M~$K\rt\piz\piz$ decays produced downstream of a regeneration
  located in a high-energy $\KL$ beam~\cite{Schwingenheuer:1995uf}:
\begin{eqnarray}
  {\rm CPLEAR:}~~~\phi_{\rm SW}+\phi^{CPT}_{\eta}&=&42.91^\circ\pm 0.53^\circ {\rm (stat)}\pm 0.28^\circ  {\rm (syst)}\\
  \nonumber
  {\rm E773:~~}~~~\phi_{\rm SW}+\phi^{CPT}_{\eta}&=&42.94^\circ\pm 0.58^\circ {\rm (stat)}\pm 0.49^\circ  {\rm (syst)},
\end{eqnarray}
where the systematic errors include $0.19^\circ$ (CPLEAR)~\&~$0.35^\circ$ (E733) contributions from uncertainties in
the regeneration phases.   The potential statistical error for a 1~trillion $\jpsi$~event data sample at STCF is
factor of ten better than previous experiments. The issue will be whether on not the systematic errors can be controlled
accordingly.  In the STCF detector the production and a fraction of the $K\rt\pipi$ decays occur in a high vacuum, and
the material traversed by rest of the decaying neutral kaons is very small, This will substantially reduce the effects of
regeneration, which was the largest source of systematic error in the previous experiments. 

As the results in the previous paragraph indicate, constraints on the validity of the $CPT$ theorem have not changed
in the twenty-five years.  This is not because these are not important or interesting, instead it is because that
no tests of $CPT$ invariance for particle systems other than the neutral kaons have anything like the same sensitivity,
and none of the world's active particle physics facilities have been capable of producing enough kaons to match, much less
improve on, the CPLEAR and E773 results. (The BESIII 10B $\jpsi$ event sample only has $\sim$20M tagged $\Kz(\tau)\rt\pipi$
and $\Kzbar(\tau)\rt\pipi$ decays, less than a third of the CPLEAR data sample.) The STCF project will provide a unique 
opportunity to make an order of magnitude sensitivity improvement over earlier measurements.
