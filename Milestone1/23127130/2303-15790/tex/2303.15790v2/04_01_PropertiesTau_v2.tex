\subsection{Precision measurement of the $\tau$ properties}

To test the SM and search for new physics in the $\tau$ sector, it is important for the properties of the $\tau$ to be known with great precision. Here, we list a few measurements at the STCF that can improve our understanding of the $\tau$ properties.

\subsubsection{$\tau$ mass and lifetime}
Many of the tests for the SM and beyond involve precise measurements of the $\tau$ mass ($m_\tau$) and lifetime. While at the threshold for $\tau^+\tau^-$ pair production, measurement of the $\tau$ lifetime is difficult, at the high-energy ($5\sim 7$~GeV) end of the STCF range, it could be possible to measure it by reconstructing the $\tau^+\tau^-$ vertex. With sufficiently high statistics, there is a chance to improve the measurement of the $\tau$ lifetime, for which a more dedicated study would be needed. On the other hand, the mass measurement can also be improved. The mass has been measured at the 70~ppm level, with a world average of~\cite{PDG} $m_\tau = 1776.86\pm0.12~\MeV$. In charged-current induced leptonic decays, $\tau \to \nu_\tau l \bar \nu_l$ $(l= e, \mu)$, the decay widths are proportional to the fifth power of $m_\tau$. Consequently, a small error in the mass can cause significant deviations in tests of the universality of the SM and in the search for new physics. At the STCF, the number of $\tau$s produced may be one to three orders of magnitude greater than at BESIII, which will greatly enhance the statistical significance achieved. With further improvements in particle ID and energy measurement capabilities, the improved sensitivity can increase the accuracy by a factor of 7 to reach a level better than 10~ppm. This improved $\tau$ mass measurement will consolidate the basis for any further $\tau$ physics studies.

\subsubsection{Measurement of $a_\tau = (g_\tau-2)/2$}
The anomalous magnetic dipole moment of the $\tau$ lepton, $a_\tau$, is another property of fundamental importance. The corresponding $a_l$ values for the electron and muon have been measured to high precision. For the electron, there is a $2\sigma$ deviation between the measurement and the SM prediction, $\Delta a_e = a^\textrm{exp}_{e} - a^\textrm{SM}_e =-78(36)\times 10^{-14}$~\cite{electron-mdm}. On the other hand, there is a longstanding and larger discrepancy for the muon moment $a_\mu$, which is currently being measured at Fermilab and J-PARC. Very recently, Fermilab reported their new result from the Run 1 measurement~\cite{Fermilab}. Upon combining it with previous data from BNL, the discrepancy is now $\Delta a_\mu = a^\textrm{exp}_\mu - a^\textrm{SM}_\mu = (251\pm 59)\times 10^{-11}$, and its significance level has been enhanced from $3.7\sigma$ to $4.2\sigma$. As this may be an indication of new physics, it has motivated extensive theoretical studies within the SM and beyond to understand possible causes. It is therefore important to test whether there is also a deviation in $a_\tau$. This is especially important for testing models of new physics that include states whose couplings are proportional to mass.

However, the measurement of $a_\tau$ is drastically different from that of $a_{e,\mu}$ due to the short lifetime of the $\tau$. The SM prediction for $a_\tau$ is $1177.21(5)\times 10^{-6}$~\cite{tau-mdm}. Currently, $a_\tau$ has been measured from the production cross section for $\tau$ pairs together with the spin or angular distributions of the $\tau$ decays; for instance, the current bounds of $-0.052 \leq a_\tau \leq 0.013$ (95\% C.L.) were obtained by the DELPHI collaboration~\cite{Abdallah:2003xd} from the cross section for the process $e^+e^-\to e^+e^-\tau^+\tau^-$ under the assumption that the SM tree-level result is modified only by the anomalous magnetic moment. These measurements are still far from constituting a precision test for the SM, and conventional measurements through similar processes may never reach the necessary level of precision. To overcome this bottleneck, a new method has recently been proposed in Ref.~\cite{Chen:2018cxt}, in which it was found to be feasible to reach a precision level of $1.75\times 10^{-5}$ at Belle II before considering systematics. In addition, it was shown some time ago that in $e^+e^- \to \tau^+\tau^-$ with a polarized electron beam, it would be plausible to achieve this precision goal at the STCF by measuring the transverse and longitudinal polarizations of the $\tau$ lepton~\cite{bernabeu-mdm}. It has been argued that if $\tau^+\tau^-$ pairs are produced on top of the narrow $\Upsilon(1S,2S,3S)$ resonances, with a very well-controlled background near the threshold, a precision even better than that of Belle II can be expected. Nevertheless, it has also been pointed out in Ref.~\cite{Eidelman:2016aih} that an energy spread with $e^+e^-$ beams on the order of a few MeV, which is likely to occur, would make such a measurement impractical because the resonant contributions would be contaminated by nonresonant ones of at least similar size, which would need to be subtracted to extract the dipole moment. In addition, the momentum transfer is too large to be directly related to dipole moments. The authors of Ref.~\cite{Eidelman:2016aih} proposed another method of measuring dipole moments, i.e., by means of radiative decays of the form $\tau^-\to l^-\nu_\tau\bar\nu_l\gamma$. However, they estimated that the sensitivity to $a_\tau$ would be approximately $0.085$ ($0.012$) using the full data of Belle (Belle II), which offers no meaningful improvement compared to the sensitivity of $0.017$ at DELPHI,
%Editor: Please ensure that the intended meaning has been maintained in the above edit.
and the sensitivity to $d_\tau^\gamma$ cannot be improved either. Therefore, more critical studies are needed.

