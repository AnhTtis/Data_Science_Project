\subsection{Flavor-violating $\tau$ decays}

Lepton flavor-changing neutral current (FCNC) interactions of the $\tau$ are suppressed in the SM when the neutrino masses and mixing are incorporated. In new physics models beyond the SM, larger FCNC effects may appear in some decays, such as $\tau$ decays into $3l$, $l \gamma$, and one or more hadrons plus charged leptons. With the increased statistics for $\tau$ events at the STCF, these decays can be searched for to test the SM and beyond.

\subsubsection{The $\tau^- \to 3l$ decay}

The $\tau^- \to 3l$ decay is one of the most sensitive probes of FCNC interactions. The current upper bound is on the order of $10^{-8}$. At Belle II, upon the accumulation of $50~\textrm{ab}^{-1}$ of integrated luminosity, the sensitivity can reach $4\times 10^{-10}$. When running the STCF at its peak energy ($\sqrt{s} = 4.26$ GeV), it will be possible to produce $3.5\times 10^{9}$ $\tau$ pairs each year, which could be used to push the branching ratio down to a level of $1.9\times 10^{-10}$ with 10~ab$^{-1}$ of luminosity~\cite{tau23l}.

\subsubsection{The $\tau^-\to l\gamma$ decays}

Equally interesting are the $\tau\to l\gamma$ decays, where $l=e$ and $\mu$. The current limits for these decays are also on the order of $10^{-8}$. Since initial-state radiation effects are strongly suppressed near the $\tau^+\tau^-$ production threshold, the STCF has an advantage over $B$ factories in a search for these decays~\cite{Bobrov:2012kg}. At the STCF, the sensitivity for this branching ratio will be able to reach $1.2\times 10^{-9}$ with 10~ab$^{-1}$ of luminosity.

\subsubsection{The $\tau^- \to lP_1P_2$ decays}

The $\tau^\pm \to l^\pm P_1P_2$ decays, where $P_i = \pi$ and $K$, have been previously searched for with a sensitivity on the order of $10^{-8}$. Similar to these decays are the lepton-number-violating $\tau^\pm\to l^\mp P_1^\pm P_2^\pm$ decays, for which the current bounds are also on the order of $10^{-8}$. At the STCF, the sensitivity for these decays can be increased by two orders of magnitude to a few times $10^{-10}$.

As mentioned earlier, FCNC interactions are highly suppressed in the SM. In some new physics models, however, FCNC interactions can be generated at the tree level and may therefore induce some of the above processes at a level close to their current bounds. In this circumstance, the STCF will be capable of providing very useful information on those models.

