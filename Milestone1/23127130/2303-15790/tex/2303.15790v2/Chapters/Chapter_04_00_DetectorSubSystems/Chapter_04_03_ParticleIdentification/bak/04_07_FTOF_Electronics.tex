\subsubsection{FTOF Electronics}

\paragraph{1) Design and Challenges}

\indent\par
In detectors which use the internally reflected Cherenkov light in quartz bars to provide the particle identification, it is important to get a high time resolution, because the arrival time of the Cherenkov photon not only contains information about the momentum of particles but also can be used to distinguish signals from the background. Listed in the Table \ref{DIRCelecsys} are features of the readout electronics from some internally reflected Cherenkov detectors which are now in use or under development, including DIRC for BaBar \cite{BabarTDR}, TOP for Belle II \cite{Belle2TDR}, barrel DIRC for PANDA \cite{PandaTDR}, TORCH for LHCb \cite{TorchElec} and FDIRC for GlueX \cite{GlueXDirc}. For comparison, the requirements of STCF FTOF are also listed in this table. The concept of Detection of Internally Reflected Cherenkov light (DIRC) was first developed in the BaBar experiment, where PMTs were chosen to be the photon detectors in consideration of the cost. In BaBar, the coordinate of the lighted PMT was used to calculate the Cherenkov angle and the timing information was just used to distinguish signals from the background, so the requirement of the timing resolution was only a few nanoseconds. With the development of detectors, microchannel plate photomultiplier tubes (MCP-PMTs) have been using in more and more high-energy-physics experiments. As a photon detector, the MCP-PMT can achieve the timing resolution of sub-100 picoseconds for single-photon detection, which has a great promotion for the time performance of the experiments.
In the MCP-PMT based FTOF detectors, the particle identification relies on the Cherenkov photon arrival time rather than the position information of MCP-PMT. Therefore, the high timing resolution requirement is a markedly feature of FTOF detectors. Furthermore, FTOF detectors usually have a large channel number and require a track timing resolution below 30 picoseconds, which is a great challenge for the front-end electronics.

\begin{table*}[tb]
\small
    \caption{Readout electronic systems of various DIRC-like detectors.}
    \label{DIRCelecsys}
    \vspace{0pt}
    \centering
    \begin{tabular}{lllllll}
        \hline
        \thead[l]{ } & \thead[l]{Babar DIRC} & \thead[l]{Belle II TOP} & \thead[l]{PANDA Barrel DIRC} & \thead[l]{LHCb TORCH} & \thead[l]{GlueX FDIRC} & \thead[l]{STCF FTOF} \\
        \hline
        Photodetector	&PMT        &MCP-PMT	&MCP-PMT	&MCP-PMT	&MAPMT	    &MCP-PMT \\
        Channels	    &10752	    &8192   	&\~11k      &\~200k     &11520     	&768     \\
        $\sigma_{T}$ (SPE)	&1.7ns	&100ps    	&100ps     	&70ps      	&600ps     	&70ps    \\
        Method	        &TDC     	&SCA	    &TOT      	&TOT       	&TDC       	&Multi-threshold TOT \\
        \hline
    \end{tabular}
\end{table*}

The preliminary structure of the FTOF readout electronics is shown in Fig. \ref{FTOF_electronics_architecture}. The readout electronics consists of the front-end board and the data-control board. The front-end board utilizes a time over multi-threshold scheme to extract timing information from analog signals. The signals from MCP-PMT are pre-amplified firstly. The gains of amplifiers are set independently to compensate for the gain variations of the individual MCP-PMT channels. The amplified signals are fed into high-performance comparators, each of them has a different threshold. The outputs of these comparators are fed into the FPGA. The time-to-digital converter (TDC) module will be implemented in the FPGA. And the TDC measures the arrival time of both edges of the comparator outputs with high accuracy. Then the front-end board passes the resulting binary data stream to the data-control board. The data-control board not only collect data stream from front-end boards but also fan out high-performance clock and control signal to front-end boards. According to the structure described above, we can briefly estimate the amount of data that the readout electronic system fed into the subsequent DAQ system. Assuming the event rate of the FTOF detector is 5k/s, the data rate output to the subsequent DAQ system can be simply calculated as 4.8MB/s without considering the effects of the detector background noise and crosstalk.
It is crucial for high time performance that the front-end electronics have high bandwidth and a high signal-to-noise ratio. The time resolution of the FPGA-based TDC was evaluated as 3.9ps in our previous work \cite{fastTDC}. And we will implement the TDC that can measure narrow pulse widths, of which the time resolution will be better than 5ps. And the TDC performance in the radiation environment will be evaluated. The results of the evaluation will affect our system structure design. A stable, low-jitter detector-wide clock distribution network also needs to be developed. Its long-term stability, short-term stability, and temperature stability will be carefully evaluated. The jitter of the clock distribution network which is better than 15ps would satisfy the design requirements.

\begin{figure}[!htb]
  \centering
  \includegraphics[width=0.8\textwidth]
  {Figures/Figs_04_00_DetectorSubSystems/Figs_04_03_ParticleIdentification/Figs_FTOF_Electronics/FTOF_Electronics_Struture.jpg}
  \caption{The preliminary structure of the FTOF readout electronic system.}
  \label{FTOF_electronics_architecture}
\end{figure}

\paragraph{2) Critical R and D}

\indent\par
The goal of the electronic R and D is to understand the influence of various electronic parts on the timing performance and study the possible problems that may appear in practical applications. Adequate R and D is the foundation to design an electronic system that satisfies all the requirements. In PANDA Barrel DIRC and TORCH, the TOT method is used to eliminate the time-amplitude walk while extracting the time information. To further improve the measurement resolution of the leading-edge timing method, the multi-threshold TOT method will be used in the front-end board. And the research of the multi-threshold timing method is also in progress, with selecting more measured points, the higher measurement resolution is supposed to achieve.
The influence of the front-end amplifier circuit on the timing performance of the leading edge needs to be carefully evaluated, including the impacts of the pre-amplifiers and comparators. With the understanding of key parameters that affect the timing performance, the circuit structure and design can be optimized to get higher integration and lower power consumption.
The readout electronics must sustain the radiation loads during the operational lifetime of STCF. Among them, the radiation influence on the FPGA-based TDC needs to be regarded first. The performance of TDCs under radiation will be evaluated in two situations. The first one is the damage of single event upsets to the FPGA-based TDC in the radiation environment. The second is the change in the FPGA-based TDC performance after long-term work in the radiation environment. Research on the performance of FPGA-based TDCs in the radiation environment will not only affect the structural design of the front-end electronics but also promote the application of FPGA-based TDC in high-energy physics experiments.
FTOF readout electronic system consists of many front-end boards that need to be synchronized, so the jitter of the distribution clock network is a key parameter of the electronic system. Currently, there is no working distribution clock network that can satisfy the requirements of FTOF. Therefore, it is necessary to develop the clock distribution network that meets the requirement of FTOF based on existing technology.


\paragraph{3) Test and Calibration}

\indent\par
The electronic readout system will be tested separately for each part and then the system will be tested totally. Single-photon timing performance is a key indicator of whether the electronics system meets the design requirements.
FTOF readout electronics need to calibrate the offset and drift in the system to reduce their impact on the timing performance of the system. The design of the electronic system and manufacturing of the components will introduce the offset. For the offset from the electronics system, some of them can be automatically calibrated by designing a self-calibration module, and the others need external signals to fulfill calibration. The external signals can be generated by signal generators or photon detectors illuminated by the laser. Short-term drift may be caused by environmental changes, such as temperature affecting the clock distribution network. And long-term drift may be caused by the long-term operation of electronic components. In the calibration progress, the short-term drift can be compensated by using sensors and corresponding control methods, while the Long-term drift requires long-time experiments on key components to find out the solution after determining the system design scheme.


\input{Chapters/Chapter_04_00_DetectorSubSystems/Chapter_04_03_ParticleIdentification/04_Ref_FTOFElectronics}
