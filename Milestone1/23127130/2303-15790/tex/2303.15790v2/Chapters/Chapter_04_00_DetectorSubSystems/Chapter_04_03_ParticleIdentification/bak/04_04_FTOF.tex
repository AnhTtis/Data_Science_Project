\subsubsection{DTOF}

\paragraph{Timing Resolution of DIRC-Type Detector}
\label{Timing_Resolution_DIRC_Detector}
The main sources contributing to the timing uncertainty of DIRC-like detector are as follows:
\begin{equation}
	\sigma^{2}_{tot} \sim \sigma^2_{trk} + \sigma^{2}_{T_{0}} + (\frac{\sigma_{DIRC}}{\sqrt{N_{p.e.}}})^{2} 
	= \sigma^2_{trk} + \sigma^{2}_{T_{0}} + (\frac{\sigma_{elec}}{\sqrt{N_{p.e.}}})^{2} + (\frac{TTS}{\sqrt{N_{p.e.}}})^{2} + (\frac{\sigma_{det}}{\sqrt{N_{p.e.}}})^{2}.
	\label{EQ:DIRC-Time-Reso}
\end{equation}
In the equation above $N_{p.e.}$ is the number of photoelectrons, $\sigma_{trk}$ is the error caused by track reconstruction, $\sigma_{T_{0}}$ is the event reference time ($T_{0}$, i.e. when physical collision happens) error mainly affected by the collider design of STCF, $\sigma_{elec}$ is the electronic timing accuracy, $TTS$ is the single-photon transit time spread of photon sensor, and $\sigma_{det}$ is the time reconstruction uncertainty of the DIRC detector. From this formula, we can see that the contribution from $\sigma_{elec}$, $TTS$ and $\sigma_{det}$, which combine to $\sigma_{DIRC}$, will decrease with increasing $N_{p.e.}$, while the timing errors from $\sigma_{trk}$ and $\sigma_{T_{0}}$ keep unchanged. Their relative importance should be studied and optimization can be achieved considering all their contribution. 

To estimate the timing uncertainty by using DIRC method, i.e. $\sigma_{DIRC}$, we study the main contributing factors by considering a simple case. If a relativistic charged particle incident into a thin Cherenkov radiator (e.g. fused quartz) plate at position (0,0,0) along the z direction, and the Cherenkov light is imaged by the photon sensor array at an effective distance z, then the timing resolution can be expressed as 
\begin{equation}
	\sigma_{DIRC}=\sigma_{T}\oplus\frac{(n_{p}\sigma_{n_{g}}+n_{g}\sigma_{n_{p}})z\beta}{c}\oplus\frac{n_{p}n_{g}z\sigma_{\beta}}{c}\oplus\frac{n_{p}n_{g}z\beta\sigma_{z}}{c},
	\label{EQ:DIRC-Time-Err}
\end{equation}

which includes $\sigma_{T}$ the single photon time resolution of the sensor along with the electronics (also known as $\sigma_{SPE}$), $\sigma_{n}$ the dispersion effect, and $\sigma_{z(x, y, T)}$ the position resolution which may depend on detailed algorithm, and $\sigma_{\beta}$ the measurement error of particle velocity. In the formula $n_{p}$, $n_{g}$ are the phase and group refractive index of quartz, $\beta$ and $c$ are the reduced velocity of light and the speed of light in vacuum.


To quantitatively estimate $\sigma_{DIRC}$, we assume a sensitive wavelength range of 300-600 nm for the photon detector and the dispersion effect would be $\sigma_{n}\sim0.0075$ and $<n>\sim n@390nm=1.471$. For simplicity the difference between $n_{p}$ and $n_{g}$ is neglected. The photon detector along with the readout electronics give an timing uncertainty of $\sigma_{T}=\sigma_{SPE}=50\sim70 ps$. The position error contains both the initial position uncertainty (quartz plate thickness ~2 cm) and the finite photodetector pixel size (~5 mm), which combine to $\sigma_{z}\sim6 mm$. The velocity error contribution is considered by a kaon meson at momentum $p=1 GeV/c$, which is $\sigma_{\beta} \sim 0.001$.

The calculation results are shown in Figure ~\ref{FIG:DIRCTIMINGERR1}, where the horizontal coordinate Z represents the extrapolated hit z position on the imaging plane. It can be seen that the timing jitter of the photon detector plays a major role when the photon propagation length is relative short ($Z \le 100cm$), while the dispersion effect gradually becomes the dominant factor when the photon transmission distance is large. The  $\sigma_{z}$ effect contributes $\sim30-40 ps$, in this calculation mainly comes from the influence of the radiator thickness. In the applicable momentum range of DIRC technology, the incident particle velocity (or momentum) is expected to contribute very little to the timing uncertainty. Furthermore, the angular resolution contribution is not included in Figure ~\ref{FIG:DIRCTIMINGERR1}, which mainly comes from the multiple Coulomb scattering (MCS) effect and is only obvious at low momentum.
\begin{figure}[!htb]
	\centering
	\includegraphics[width=0.65\textwidth]{Figures/Figs_04_00_DetectorSubSystems/Figs_04_03_ParticleIdentification/DIRC_Timing_Error1.jpg}
	\caption{The main DIRC timing error factors and their dependences on the photon propagation distance.}
	\label{FIG:DIRCTIMINGERR1}
\end{figure}

Monte Carlo (MC) simulation studies are also performed to understand the detector physics of the PID technological candidates, and estimate the PID performance achievable. The results are consistent with the calculations above and will be discussed in detail in the following parts.


\paragraph{DTOF Conceptual Design}
Based on the results from both simulation study and experimental test, we come up with the conceptual design of the DTOF at STCF, as shown in Fig. \ref{FIG:DTOF_Conceptual_Design}. The proposed DTOF detector consists of two identical endcap discs, positioned $~1400 mm$ along the beam direction away from the collision point. Each disc is made up with four quadrantal sectors, each with an inner radius of $\sim 560 mm$ and an outer radius of $\sim 1050 mm$, covering $\sim 21^\circ - 38^\circ$ in polar angle and $90^\circ$ in azimuthal angle. The detail structure inside a sector is also depicted in Fig. \ref{FIG:DTOF_Conceptual_Design}. The planar synthetic fused silica radiator is fan shaped that can be viewed as a composite structure of 3 trapezoidal units, each $\sim 295 mm$ (inner side) $/$ $\sim 533 mm$ (outer side) wide, $\sim 470 mm$ high and $15 mm$ thick. An array of $3 \times \sim 14-16$ multi-anode micro-channel plate photomultiplier tubes (MCP-PMT) are directly coupled to the radiator along the outer side. The whole sector is enclosed in a light-tight black box made of $5 mm$ thick carbon fibre, occupying $\sim 200 mm$ space along the beam (Z) direction.

\begin{figure}[!htb]
	\centering
	\includegraphics[width=0.75\textwidth]{Figures/Figs_04_00_DetectorSubSystems/Figs_04_03_ParticleIdentification/DTOF_Conceptual_Structure1.jpg}
	\includegraphics[width=0.20\textwidth]{Figures/Figs_04_00_DetectorSubSystems/Figs_04_03_ParticleIdentification/DTOF_Conceptual_Structure2.jpg}
	\caption{The conceptual design of DTOF.}
	\label{FIG:DTOF_Conceptual_Design}
\end{figure}


\paragraph{DTOF Performance in Simulation}
Geant4 simulations are performed to study the expected performance of DTOF with such design. A $20 mm$ thick Aluminum plate is added $100 mm$ before the DTOF detector, in order to simulate the material budget of the MDC endcap. When tracking the photon propagation, the inner and outer side of the DTOF radiator are set absorptive, while the two lateral sides are set reflective (reflection factor $\sim 92\%$). The surface roughness of the radiator is simulated by randomizing the normal direction of facet by $\sigma = 0.1^\circ$ (corresponding to an average reflection factor of $\sim 97\%$). Pion and kaon particles are emitted from the interaction point at different momenta and directions. Different angular, azimuth angles and particle momenta  are scanned. A typical Cherenkov photon hit pattern is displayed as in Fig. \ref{FIG:DTOF_TOP_Resolution} by pions at $p = 1 ~GeV/c$, $\theta = 23.66^\circ$ and $\phi = 15^\circ$. A clear correlation between the time of propagation (TOP) and the hit position is demonstrated. Two bands, one for direct photons and another for indirect photons with one reflection off the lateral side, are well separated except for a few sensors close to the side. The mean number of photons detected by the MCP-PMT arrays is $\sim 17$. By applying a TOP-position calibration, where the average track length collected by each sensor pixel and the average velocity of photons are used to calculate the TOP (no dispersion effect is accounted for), the timing resolution of $\sim 20 ps$ or better is obtained (as shown in Fig. \ref{FIG:DTOF_TOP_Resolution}).

\begin{figure}[!htb]
	\centering
	\includegraphics[width=0.45\textwidth]{Figures/Figs_04_00_DetectorSubSystems/Figs_04_03_ParticleIdentification/DTOF_TOP_1.jpg}
	\includegraphics[width=0.45\textwidth]{Figures/Figs_04_00_DetectorSubSystems/Figs_04_03_ParticleIdentification/DTOF_TOPReso_1.jpg}
	\caption{The simulated TOP vs. hit position pattern of DTOF, and the timing resolution after calibration.}
	\label{FIG:DTOF_TOP_Resolution}
\end{figure}

Besides the timing resolution mentioned above ($\sigma_{det}$), the total time uncertainty of the DTOF detector includes several other components: the TTS of the MCP-PMT ($\sigma_{TTS} = 70ps$ is used in our calculation), the electronics time jitter ($\sigma_{elec} = 10ps$), the time error due to the track reconstruction ($\sigma_{trk} = 10ps$), the event start time resolution ($\sigma_{T_{0}} = 40ps$). The expected timing resolution in different DTOF regions are deduced according to Equation (~\ref{EQ:DIRC-Time-Reso}), as shown in Fig. \ref{FIG:DTOF_TReso_NPE} for pions (kaons are very similar) at $2~GeV/c$. The black numbers are total time resolution while the red ones are the intrinsic detector timing resolution $\sigma_{det}$. It is clear $\sigma_{det}$ becomes worse at lower $\theta$, which corresponds to hits closer to the inner side of DTOF radiator, and slightly improved at larger $\phi$ (best at $\phi = 45^{\circ}$), which is further away from the lateral side of DTOF radiator. This $\theta$ and $\phi$ dependence is mainly reflected by the total number of detected Cherenkov photons, as also shown in Fig. \ref{FIG:DTOF_TReso_NPE}. It is also evident the event start time resolution $\sigma_{T_{0}} = 40 ps$ dominates the over all timing error, so optimized STCF bunch size is crucial. Furthermore we can find the constraint on $\sigma_{elec}$ ($10 ps$ in this case) can be relaxed, since the single photon response of MCP-PMT $\sigma_{TTS}$ dominates the terms with $\frac{1}{NPE}$ scaling.

\begin{figure}[!htb]
	\centering
	\includegraphics[width=0.45\textwidth]{Figures/Figs_04_00_DetectorSubSystems/Figs_04_03_ParticleIdentification/DTOF_TReso_All_1.jpg}
	\includegraphics[width=0.45\textwidth]{Figures/Figs_04_00_DetectorSubSystems/Figs_04_03_ParticleIdentification/DTOF_NPE_1.jpg}
	\caption{The expected overall DTOF timing resolution (left), and the number of detected direct photons by $\pi^{\pm}$ particle at $2~GeV/c$.}
	\label{FIG:DTOF_TReso_NPE}
\end{figure}

With the expected timing performance obtained, the PID power, i.e. $K/\pi$ separation power is defined by the following formula
\begin{equation}
\label{eq::DTOF-PID-Power}
\Delta T=\frac{L}{\beta c} \frac{\Delta m^2}{2p^2}\sim\frac{L}{c}\frac{\Delta m^2}{2p^2},
PID power = \frac{\Delta T}{\sigma_{tot}},
\end{equation}
where $\Delta T$ is the TOF difference between pion and kaon with the same momentum $p$, $L$ is the flight path length, and $\sigma$ is the total time resolution of DTOF. The results are shown in Fig. \ref{FIG:DTOF_PIDPower}, for $2~GeV/c$ $\pi/K$ at different regions of the DTOF. Due to longer $L$ and better timing resolution at larger $\phi$, the DTOF PID capability is better for tracks hitting the radiator closer to its outer side. The worst case is at the corner of inner and lateral edges, where the separation power is still acceptable, $3.1$.

\begin{figure}[!htb]
	\centering
	\includegraphics[width=0.6\textwidth]{Figures/Figs_04_00_DetectorSubSystems/Figs_04_03_ParticleIdentification/DTOF_PID_Power_1.jpg}
	\caption{The expected PID capability under such timing performance for $\pi/K$ separation at $2~GeV/c$.}
	\label{FIG:DTOF_PIDPower}
\end{figure}

%% The following paragraphs are based on the work by BinBin Qi
\paragraph{DTOF Reconstruction Algorithm}
The DTOF reconstruction is performed in the coordinate as in Fig. \ref{FIG:DTOF_Reco_Coor}, for one DTOF quadrant. According to the Cherenkov angle relation
\begin{equation}
cos(\bar{\theta_{c}}) = \frac{1}{n_{p}\beta} = \frac{{\vec{v_{t}}} \dot {\vec{v_{p}}}}{|v_{t}| \dot |v_{p}|},
\end{equation}
where $\vec{v_{t}}=(a,b,c)$ is the incident particle velocity vector when impinging the radiator, $\vec{v_{p}}$ is the velocity vector of emitted Cherenkov photon. The directional components of $\vec{v_{p}}$ can be expressed as $(\Delta{X},\Delta{Y},\Delta{Z})$, the 3D spatial difference between the photon sensor pixel and the incident position of the particle on the radiator surface, as depicted in Fig. \ref{FIG:DTOF_Reco_Coor} (right). Although the 2D (X and Y) difference can be readily obtained, $\Delta{Z}$ must be deduced with certain particle specie hypothesis. If $V = cos(\bar{\theta_{c}})$ is known, the equation about $\Delta{Z}$ can be found as
\begin{equation}
\label{eq::DTOF-DZ-Solver}
(c^{2}-V^{2})\Delta{Z}^{2} + 2c(a\Delta{X}+b\Delta{Y})\Delta{Z} + (a\Delta{X}+b\Delta{Y})^{2} - V^{2}(\Delta{X}^{2}+\Delta{Y}^{2}) = 0.
\end{equation}
By solving this equation we find $\Delta{Z}=\frac{-B\pm\sqrt{B^{2}-4AC}}{2A}$, with $A=c^{2}-V^{2}$, $B=2c(a\Delta{X}+b\Delta{Y})$ and $C=(a\Delta{X}+b\Delta{Y})^{2} - V^{2}(\Delta{X}^{2}+\Delta{Y}^{2})$. In order to get a real solution, $\Delta = B^{2}-4AC \ge 0$ is required. Furthermore, after some physical cuts $V > 0$ (Cherenkov photon forwardly emitted) and $\frac{\Delta{X}^{2}+\Delta{Y}^{2}}{\Delta{X}^{2}+\Delta{Y}^{2}+\Delta{Z}^{2}} \ge \frac{1}{n_{p}^{2}}$ (ensuring internal total reflection) are applied, the minimal solution of equation ~\ref{eq::DTOF-DZ-Solver} ($\Delta{Z} = min{|\Delta{Z}_{1}|,|\Delta{Z}_{2}|}$) is taken as the optimum.

\begin{figure}[!htb]
	\centering
	\includegraphics[width=0.55\textwidth]{Figures/Figs_04_00_DetectorSubSystems/Figs_04_03_ParticleIdentification/DTOF_Reco_Coor.jpg}
	\includegraphics[width=0.35\textwidth]{Figures/Figs_04_00_DetectorSubSystems/Figs_04_03_ParticleIdentification/DTOF_Reco_Dir.jpg}
	\caption{The coordinate system used in DTOF reconstruction (left) and the direction of Cherenkov photon (deep blue line).}
	\label{FIG:DTOF_Reco_Coor}
\end{figure}

The timing error of such approach is estimated by adding up possible factors, such as the dispersion effect, the finite photon sensor size and the propagation length of photons inside the radiator. Assuming a pion of $p \sim 1 - 2~GeV/c$ cross the radiator perpendicularly, with $H = 0.5 m$ (defined as in Fig. \ref{FIG:DTOF_Reco_Coor}), the expected timing uncertainty is shown in Fig. \ref{FIG:DTOF_Reco_TLerr} for sensors at different positions. The pitch of photon sensor is $5.5 mm$. No multiple Coulomb scattering (MCS) effect is accounted for. It's obvious the intrinsic detector timing uncertainty is no more than $40 ps$ with this DTOF structure. Furthermore we find the reconstructed length of propagation (LOP) of light inside the radiator agree with the MC truth to a precision of $\sim 3.3 mm$, as also shown in Fig. \ref{FIG:DTOF_Reco_TLerr}. The reconstruction algorithm works well for most photon sensors, no matter the incident position of particles, except for a few sensors near the lateral side.

\begin{figure}[!htb]
	\centering
	\includegraphics[width=0.45\textwidth]{Figures/Figs_04_00_DetectorSubSystems/Figs_04_03_ParticleIdentification/DTOF_Reco_Terr.jpg}
	\includegraphics[width=0.45\textwidth]{Figures/Figs_04_00_DetectorSubSystems/Figs_04_03_ParticleIdentification/DTOF_Reco_Lerr.jpg}
	\caption{The expected timing error and propagation length uncertainty of Cherenkov photons in the DTOF quadrant.}
	\label{FIG:DTOF_Reco_TLerr}
\end{figure}

By applying formula
\begin{equation}
TOF = T - TOP - T_{0} = T - \frac{LOP}{v_{g}} - T_{0} = = T - \frac{LOP\times\bar{n_{g}}}{c} - T_{0},
\end{equation}
where $v_{g}$ is the group velocity of Cherenkov light in the radiator, the TOF information is obtained. Shown in Fig. \ref{FIG:DTOF_Reco_TOFreso} are the time resolution of the DTOF detector for single photo-electron (SPE) and average of all photons, without taking the timing jitter of MCP-PMT and electronics into account. For a SPE, the intrinsic time resolution of the DTOF is $\sim 41 ps$, while by averaging timing information over $\sim 18$ detected photons the timing jitter shrinks to $\sim 10 ps$. It's also noted that in the TOF distribution plot, a low (but visible) long tail shows on both sides of the main peak. The tail is mainly caused by secondary particles along the primary pion, mostly $\delta$-electrons.

\begin{figure}[!htb]
	\centering
	\includegraphics[width=0.45\textwidth]{Figures/Figs_04_00_DetectorSubSystems/Figs_04_03_ParticleIdentification/DTOF_Reco_TOFspe.jpg}
	\includegraphics[width=0.45\textwidth]{Figures/Figs_04_00_DetectorSubSystems/Figs_04_03_ParticleIdentification/DTOF_Reco_TOFall.jpg}
	\caption{The TOF resolution of the DTOF detector for single photo-electron and average of all photons.}
	\label{FIG:DTOF_Reco_TOFreso}
\end{figure}

The TOF information is deduced and compared to the expectation of each hypothesis particle. Fig. \ref{FIG:DTOF_Reco_TOFPID} shows the reconstructed intrinsic TOF distributions of both pions and kaons at $2 GeV/c$. One can easily find if the particle hypothesis is correct, the reconstructed TOF peak is at its right position, with a resolution of $\sim 10ps$. However if the hypothesis is not true, the reconstructed TOF peak is shifted with respect to its expectation. The shift makes the separation between pion and kaon TOF peaks even larger, which may benefit the PID power. When convoluting all contributing factors, as in Equation (~\ref{EQ:DIRC-Time-Reso}), the overall reconstructed TOF time resolution is $\sim 45-50 ps$, which is shown in Fig. \ref{FIG:DTOF_Reco_TOFPID}. By directly comparing the TOF information, a $3.0 \sigma$ separation power for $\pi/K$ at $2 GeV/c$ is achieved, consistent with the GEANT4 simulation (Fig.~\ref{FIG:DTOF_PIDPower}). Furthermore the separation power gets stronger if we compare the reconstructed TOFs of various hypotheses for the same set of particles, mainly due to the beneficial time shift under wrong hypothesis (as in Fig. \ref{FIG:DTOF_Reco_TOFPID}). 

\begin{figure}[!htb]
	\centering
	\includegraphics[width=0.45\textwidth]{Figures/Figs_04_00_DetectorSubSystems/Figs_04_03_ParticleIdentification/DTOF_Reco_TOFPID_1.jpg}
	\includegraphics[width=0.45\textwidth]{Figures/Figs_04_00_DetectorSubSystems/Figs_04_03_ParticleIdentification/DTOF_Reco_TOFPID_2.jpg}
	\caption{The TOF PID capability of the DTOF detector for $\pi/K$ separation at $2 GeV/c$, without (left) and with (right) contributions from other timing uncertainties.}
	\label{FIG:DTOF_Reco_TOFPID}
\end{figure}

To evaluate the PID capability of DTOF, we apply the likelihood method. The likelihood function is constructed by
\begin{equation}
\label{eq::DTOF-Likelihood-Function}
\mathcal{L}_{h} = \Pi^{i}_{i=1}f_{h}(TOF^{h}_{i}), 
\Delta\mathcal{L} = \mathcal{L}_{\pi} - \mathcal{L}_{K},
\end{equation}
where $h$ denotes hadron species (in our case $\pi$ and $K$) and $i$ accounts for each detected photon. The probability density function $f_{h}$ is taken as a Gaussian fit to the expected TOF distribution (as in Fig. \ref{FIG:DTOF_Reco_TOFPID} (right)), plus a constant background of 0.05. Shown in Fig. \ref{FIG:DTOF_Reco_LHPID} are the reconstructed $\Delta\mathcal{L}$ for $2 GeV/c$ $\pi$ and $K$ emitted at different angles. Despite of the very different particle directions, the separation power of DTOF are similar, reaches $\sim 4\sigma$ or better over the full DTOF sensitive area.

\begin{figure}[!htb]
	\centering
	\includegraphics[width=0.45\textwidth]{Figures/Figs_04_00_DetectorSubSystems/Figs_04_03_ParticleIdentification/DTOF_Reco_LHPID_1.jpg}
	\includegraphics[width=0.45\textwidth]{Figures/Figs_04_00_DetectorSubSystems/Figs_04_03_ParticleIdentification/DTOF_Reco_LHPID_2.jpg}
	\caption{The likelihood PID capability of the DTOF detector for $\pi/K$ separation at $2 GeV/c$, emitted at different angles.}
	\label{FIG:DTOF_Reco_LHPID}
\end{figure}

\paragraph{DTOF Structure Optimization}
Different geometric parameters of DTOF are scanned to study the effects on PID. The $\pi$/K separation power of different geometry configurations are compared with the reconstruction algorithm and likelihood method described above. Some geometry configurations are listed in Table ~\ref{TAB:DTOF_GEO_CONFIG}. We study the effects coming from three main factors: radiator shape/size, radiator thickness and the setting of mirrors. 

%%%%%%%%%%%%%%%%%  TABLE  %%%%%%%%%%%%%%%%%%%%%%%%
\begin{table*}[hptb]
	\small
	\caption{Description of the different DTOF geometry configurations.}
	\label{TAB:DTOF_GEO_CONFIG}
	\vspace{0pt}
	\centering
	\begin{tabular}{llllllll}
	\hline
	\thead[l]{Configuration/Geometry ID} & \thead[l]{0} & \thead[l]{1} & \thead[l]{2} & \thead[l]{3} & \thead[l]{4} & \thead[l]{5} & \thead[l]{6} \\
	\hline
	Radiator shapes (sector number)	&4        &12	&24	&4	&4	    &4  &4 \\
	Radiator thickness (mm)	    &15	    &15   	&15      &10     &20    &15   	&15   \\
	Outer side surface	&A	&A    	&A    	&A      	&A     	&R   &$45^{\circ}$ R  \\
	Inner side surface	 &A	&A    	&A    	&A      	&A     	&A   &A  \\
	Lateral side surface	&R	&R    	&R    	&R      	&R    	&R   &R  \\
	\hline	\end{tabular}
\end{table*}
%%%%%%%%%%%%%%%%%%%%%%%%%%%%%%%%%%%%%%%%%%%%%%%%%%

In Table ~\ref{TAB:DTOF_GEO_CONFIG}, three different radiator shapes (and sizes) correspond to the Geometry 0, 1 and 2, where the DTOF disc of Geometry 0 is made up of 4 quadrant sectors as in sub-section "conceptual design", and for Geometry 1/2 the DTOF disc is made up of 12/24 trapezoidal sectors. Each sector is readout by 18/8 MCP-PMTs for Geometry 1/2. The effect of radiator thickness is studied by comparing Geometry 0, 3 and 4. The radiator thicknesses are 15 mm, 10 mm and 20 mm respectively. For all the geometry configurations listed in Table ~\ref{TAB:DTOF_GEO_CONFIG}, the inner/lateral side surfaces of radiator are covered by absorber/reflective mirror, labelled as A/R correspondingly. For the outer side surface, a mirror can extend the acceptance of Cherenkov light, which increases the detected number of photons. As shown in Figure \ref{FIG:DTOF_OPTIMIZATION}, in Geometry 5 a mirror is attached to the outer side surface of radiator, which is equivalent to putting a mirror MCP-PMT parallel to the real one, and in Geometry 6 a mirror is placed on the $45^{\circ}$ chamfer, which is equivalent to putting a mirror MCP-PMT perpendicular to the real one.

\begin{figure}[!htb]
	\centering
	\includegraphics[width=0.75\textwidth]{Figures/Figs_04_00_DetectorSubSystems/Figs_04_03_ParticleIdentification/DTOF_Geo_Optimization.jpg}
	\caption{Three different configuration on outer surface of radiator. An absorber (left) or mirror (right) on outer surface, and a mirror on the 45° chamfer of outer side surface (right).}
	\label{FIG:DTOF_OPTIMIZATION}
\end{figure}

Key results of DTOF performance for different geometry configurations are listed in Table ~\ref{TAB:DTOF_GEO_PERFORM}, for $\pi$/K mesons at $p = 2~GeV/c$, $\theta = 24^{\circ}$ and $\phi = 45^{\circ}$.

%%%%%%%%%%%%%%%%%  TABLE  %%%%%%%%%%%%%%%%%%%%%%%%
\begin{table*}[hptb]
	\small
	\caption{Performance of different geometries at $p = 2~GeV/c$, $\theta = 24^{\circ}$ and $\phi = 45^{\circ}$.}
	\label{TAB:DTOF_GEO_PERFORM}
	\vspace{0pt}
	\centering
	\begin{tabular}{llllllll}
		\hline
		\thead[l]{Configuration/Geometry ID} & \thead[l]{0} & \thead[l]{1} & \thead[l]{2} & \thead[l]{3} & \thead[l]{4} & \thead[l]{5} & \thead[l]{6} \\
		\hline
		$N_{pe}$ for pions	&21.8       &21.9	&17.0	&15.5	&25.7	    &33.2  &38.7 \\
		Accumulated charge density on MCP-PMT anode ($C/cm^{2}$)	    &10.8	    &10.5   	&9.6      &8.8     &11.8    &17.0   	&25.6   \\
		$\pi$/K separation power	&$4.17 \sigma$	&$4.08 \sigma$	&$3.66 \sigma$	&$3.99 \sigma$	&$4.27 \sigma$	&$4.26 \sigma$   &$4.19 \sigma$  \\
		\hline	\end{tabular}
\end{table*}
%%%%%%%%%%%%%%%%%%%%%%%%%%%%%%%%%%%%%%%%%%%%%%%%%%


The effect of radiator shape/size is studied by comparing the DTOF performance with Geometry 0, 1 and 2. As the radiator gets smaller, the reflection times of Cherenkov light off the lateral-side mirror increases, which causes more photon losses and, more importantly,  "confusion" in LOP reconstruction. So Geometry 0 has the best $\pi$/K separation power of $4.17 \sigma$, while Geometry 2 has the worst of $3.66 \sigma$, indicating that larger radiator is a better choice.

Geometry 3 and 4 have different radiator thicknesses compared to Geometry 0, which affects the p.e. yields. Although more detected photons mean better time resolution and PID performance, we take the 15 mm thick radiator as the best choice, which keeps low impact of DTOF material budget on EMC while providing a performance redundancy, e.g. to reduce the influence of detector ageing effect in the long-term operation.

In order to increase the p.e. yield, a mirror is attached to the outer side surface of radiator in different ways, as in Geometry 5 and 6. The number of p.e. received in these two geometries are ~33 and ~39, much higher than Geometry 0. However the mirror also increases the number of possible light path, which causes “confusion” similar to the effect of multiple reflections off the lateral-side mirror, and degrades the time resolution. Therefore, even with more detected photons, the $\pi$/K separation powers of Geometry 5 and 6 are similar to Geometry 0. In additions, the accumulated charge densities of these two geometry configurations are much higher, which will affect the lifetime of MCP-PMT. So we reject the options with mirror attached to the outer-side surface of the radiator.

By above comparison, the optimum geometry configuration of DTOF is Geometry 0, which is chosen as our baseline design. 

\paragraph{DTOF Background}
The distributions of background particles obtained by specific MDI and background study are used as input to the DTOF geant4 simulation, and then to simulate the effect of background on DTOF performance. The background particles on DTOF are mainly secondary electrons. The background hit rate on a single DTOF disc is about $7\times10^{7} Hz$, including two main parts: the beam-induced background ($25\%$) and the physical background ($75\%$). The beam-induced background is uniformly distributed in time, while the physical background is related to the collisions during bunch crossing (once per 8 ns), exhibiting characteristic time structure. The time distribution of the two background components are shown in Figure \ref{FIG:DTOF_BKG_T}. The beam background is uniformly distributed in time, representing by a uniform sampling in the simulation. The physical background is correlated with the collision frequency. By Monte Carlo sampling the physical background time distribution and combining with the uniform time distribution of the beam-induced background, the overall time distribution of the background hit on DTOF can be obtained, as shown in Figure \ref{FIG:DTOF_ALLBKG_T}.

\begin{figure}[!htb]
	\centering
	\includegraphics[width=0.45\textwidth]{Figures/Figs_04_00_DetectorSubSystems/Figs_04_03_ParticleIdentification/DTOF_BeamBkg_T.jpg}
	\includegraphics[width=0.45\textwidth]{Figures/Figs_04_00_DetectorSubSystems/Figs_04_03_ParticleIdentification/DTOF_PhyBkg_T.jpg}
	\caption{Time-of-flight distribution of background particles (starting from the collision time) on DTOF: beam-induced background (left) and physical background (right).}
	\label{FIG:DTOF_BKG_T}
\end{figure}

\begin{figure}[!htb]
	\centering
	\includegraphics[width=0.65\textwidth]{Figures/Figs_04_00_DetectorSubSystems/Figs_04_03_ParticleIdentification/DTOF_AllBkg_T.jpg}
	\caption{Overall time distribution of background particles on DTOF.}
	\label{FIG:DTOF_ALLBKG_T}
\end{figure}

In the Geant4 simulation, the time window of the signal acquisition is 100 ns, within the interval [-40 ns, 60 ns] so that the real signal is in the middle of the time window. In this time window, the number of background particles is given by Poisson distribution, and the time distribution is sampled according to Figure \ref{FIG:DTOF_ALLBKG_T}.

10000 $\pi$ and K particles are generated in the Geant4 simulation along with the background samples to study the effect of the background. The background may greatly increase the number of photoelectrons detected by DTOF in a single event, resulting increased possibility of multiple hits in a single channel. The correction of multiple hits is applied, which means that in the time window of [-40 ns, 60 ns] only the first arrival photoelectron signal is taken and all other hits are dismissed. The background level in the simulation is set to $7\times10^{7} Hz$, according to the background study. As shown in Figure \ref{FIG:DTOF_NPE_BKG}, the average number of photoelectrons for a pion at $p=2 GeV/c$ is 22, increased to about 124 after taking the background hits into account, and reduced to about 88 after the correction of multiple hits. It’s noted that by taking the background hits into account and assuming a MCP-PMT gain of $10^{6}$ , the average accumulated charge density on the MCP-PMT anode is $12 C/cm^{2}$, over 10-year STCF operation ($50\%$ run time).

\begin{figure}[!htb]
	\centering
	\includegraphics[width=0.65\textwidth]{Figures/Figs_04_00_DetectorSubSystems/Figs_04_03_ParticleIdentification/DTOF_NPE_BKG.jpg}
	\caption{Distribution of number of photoelectrons generated by a $2 GeV/c$ pion, without and with background.}
	\label{FIG:DTOF_NPE_BKG}
\end{figure}

Shown in Figure \ref{FIG:DTOF_XTHIT_BKG} is the 2-D time-position map of DTOF hit. It can be seen that hits by the background particles are uniformly distributed throughout the phase space, while the real signal hits are concentrated as a band. After time reconstruction, the TOF distribution of the single photoelectron signal can be obtained, also shown in Figure \ref{FIG:DTOF_XTHIT_BKG}. The reconstructed TOF of the real signal is a Gaussian distribution ($\sigma \sim100 ps$), while the TOF of background particles are uniformly distributed. Some single-photon electrons with zero TOF do not meet the reconstruction conditions and are taken as background. Due to the uniform distribution of the reconstructed background signal, the influence of the background can be eliminated by using the maximum likelihood method. The $\pi$/K resolution is found to be $4.1 \sigma$, as shown in Figure \ref{FIG:DTOF_PID_BKG}. 

\begin{figure}[!htb]
	\centering
	\includegraphics[width=0.45\textwidth]{Figures/Figs_04_00_DetectorSubSystems/Figs_04_03_ParticleIdentification/DTOF_XTHit_BKG_Cor.jpg}
	\includegraphics[width=0.45\textwidth]{Figures/Figs_04_00_DetectorSubSystems/Figs_04_03_ParticleIdentification/DTOF_SPETOF_BKG_Cor.jpg}
	\caption{2-D time-position map of DTOF hit (left) and reconstructed TOF distribution of single photoelectron signal (right), with multiple-hit correction.}
	\label{FIG:DTOF_XTHIT_BKG}
\end{figure}

\begin{figure}[!htb]
\centering
\includegraphics[width=0.65\textwidth]{Figures/Figs_04_00_DetectorSubSystems/Figs_04_03_ParticleIdentification/DTOF_PID_BKG_Cor.jpg}
\caption{$\pi$/K identification capability (at $2 GeV/c$) by DTOF, with multiple-hit correction.}
\label{FIG:DTOF_PID_BKG}
\end{figure}

Considering the Poisson fluctuation of background count with 3 standard deviation above the average level, i.e. the background hit rate is increased to $2.1\times10^{8} Hz$ per single DTOF disc, we find that the reconstruction results are similar to those in Figure \ref{FIG:DTOF_XTHIT_BKG} and \ref{FIG:DTOF_PID_BKG}. Although the number of photoelectrons from the background increases dramatically, the final effect on $\pi$/K separation is minor. 



