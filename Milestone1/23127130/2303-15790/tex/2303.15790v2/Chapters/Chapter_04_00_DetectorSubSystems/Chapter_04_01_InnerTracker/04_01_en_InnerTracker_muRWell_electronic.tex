
\paragraph{Readout Electronics}
\quad\\
The structure of the readout electronics of the $\mu$RWELL-based ITK detector is illustrated in Fig.~\ref{fig:4.1.01_electronic}. The front-end electronics are linked to detectors through high-density connectors, and a protection circuit is added at the input to protect the readout electronics from unexpected high-voltage discharge of the detector. Due to the small amplitude of the signal from the detector, the front-end readout electronics are placed close to the detector, and a high-density design is required for the front-end electronics. The front-end electronics are set near the endcaps of the inner tracker. The connector for $\mu$RWELL-based ITK is Hirose connector FH26W-71S-0.3SHW(60). It has a width of 23 mm and 0.3 mm channel pitch. The X and V strips of $\mu$RWELL are designed on two films, so that their readout electronics can be arranged in two complete circles. In this case, all the readout channels can arranged well by the FPCB connector. Additionally, high-precision signal measurements are required. For the reasons above, it is planned to design an application-specific integrated circuit (ASIC) chip that integrates front-end analog circuits, analog-to-digital conversion (ADC), and a charge $\&$ time calculation circuit within the chip. In addition, the calibration circuit is added to correct the mismatch among channels.
%%%%%%%%%%%%%%%%%%% Fig %%%%%%%%%%%%%%%%%%%%%%%%%%
\begin{figure*}[htb]
    \centering
{
        \includegraphics[width=160mm]{Figures/Figs_04_00_DetectorSubSystems/Figs_04_01_InnerTracker/readout_electronics_block_diagram.png}
}
\vspace{0cm}
\caption{Block diagram of the readout electronics for the $\mu$RWELL-based ITK detector.}
    \label{fig:4.1.01_electronic}
\end{figure*}
%%%%%%%%%%%%%%%%%%%%%%%%%%%%%%%%%%%%%%%%%%%%%%%%%%

The output data of the front-end ASICs are transferred to a digital ASIC or field-programmable gate array (FPGA) for data packaging and finally transferred to the DAQ through high-speed serial data interfaces. Since the ITK readout electronics need to be synchronized with the global clock, the FPGA is also responsible for fanning out the clock to each front-end ASIC.

ITK electronics do not take part in generating the global trigger signal and only receive the trigger signal for trigger matching. The data from the front-end ASICs are first stored in RAMs, and when the FPGA receives the global trigger signal, it picks out the valid data through trigger match logic and transfers the data to the DAQ.

The hardware system is composed of front-end electronics ~(FEE), readout units ~(RUs), and clock, trigger submodules. Multiple front-end ASICs, which complete analog signal processing, A/D conversion, and charge $\&$ time calculation, are integrated into one FEE module. The output data of these ASICs are transferred to the RUs through a high-speed serial data interface.

 Both the charge/amplitude and time information of the hits are necessary for track reconstruction. The recorded data of each hit signal in one channel represent a 96-bit word, including 8 bits for the header, 16 bits for the trigger number, 34 bits for the timing information, 6 bits for the amplitude, 16 bits for the FEE number, 8 bits for the check and 8 bits for the tail. As a consequence, the total data stream sizes are approximately 6.89~GB/s for all 3 layers.

\paragraph{Front-End ASIC}
\quad\\
The high channel density of the $\mu$RWELL-ITK requires customized ASICs for its front-end electronics that integrate functions of analog signal processing, ADC and charge \& time calculation. As shown in Fig.~\ref{fig:4.1.02_electronic}, each channel comprises a charge sensitive amplifier (CSA) for low noise amplification, a semi-Gaussian shaper network, a discriminator for generating a self-trigger signal, a switched capacitor array (SCA) for waveform sampling, a Wilkinson ADC for digitization, a digital circuit for charge \& time calculation, and a high-speed data transfer interface.

Due to the high event rate, full waveform data transfer would place high pressure on the data transfer interface and corresponding power consumption. Therefore, it is preferable to integrate the charge \& time calculation circuit into the chip.

%%%%%%%%%%%%%%%%%%% Fig %%%%%%%%%%%%%%%%%%%%%%%%%%
\begin{figure*}[htb]
    \centering
{
        \includegraphics[width=160mm]{Figures/Figs_04_00_DetectorSubSystems/Figs_04_01_InnerTracker/frontend_ASIC_block_diagram.png}
}
\vspace{0cm}
\caption{Block diagram of the front-end ASIC.}
    \label{fig:4.1.02_electronic}
\end{figure*}
%%%%%%%%%%%%%%%%%%%%%%%%%%%%%%%%%%%%%%%%%%%%%%%%%%


To further enhance the signal-to-noise ratio~(SNR) and to improve the accuracy of charge and time measurements, a digital deconvolution and filtering circuit is also integrated into the ASIC. The exponential signal is first unfolded into the unit impulse, and then the trapezoidal output pulse is processed by the moving average method (corresponding to the low-pass filter in Fig.~\ref{fig:4.1.03_electronic}) to filter out the high-frequency noise. The specific parameters of the deconvolution and moving average circuit need to be optimized according to the shaping time and knee frequency of the signal to achieve the best filtering result.

%%%%%%%%%%%%%%%%%%% Fig %%%%%%%%%%%%%%%%%%%%%%%%%%
\begin{figure*}[htb]
    \centering
{
        \includegraphics[width=160mm]{Figures/Figs_04_00_DetectorSubSystems/Figs_04_01_InnerTracker/circuit_block_diagram.png}
}
\vspace{0cm}
\caption{Block diagram of the digital deconvolution and low-pass filter circuit.}
    \label{fig:4.1.03_electronic}
\end{figure*}
%%%%%%%%%%%%%%%%%%%%%%%%%%%%%%%%%%%%%%%%%%%%%%%%%%
