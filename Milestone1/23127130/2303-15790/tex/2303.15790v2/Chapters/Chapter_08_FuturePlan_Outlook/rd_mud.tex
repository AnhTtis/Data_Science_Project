According to the intensive background at the STCF, the conceptual design of the muon detector (MUD) adopts an innovative approach combining an inner RPC detector and an outer plastic scintillator (PS) detector to avoid interference from background while retaining the required muon-ID abilities. To realize such a design, R\&D on various subjects is necessary, including PS+SiPM technology, RPC technology and readout electronics. A large prototype of the hybrid MUD is needed to verify the design and basic performance of the detector. The studies will involve the following:
\begin{itemize}
\item R\&D of a high-rate (up to approximately 100~kHz/channel) and large-area (size $>0.5$~m$^2$ and length $>1$~m for a single module) RPC detector;

\item R\&D of the large-size (size $>0.5$~m$^2$ and length $>1$~m for a single module) MUD module based on a plastic scintillator + wavelength shifting fiber (WLS) + SiPM technology;

\item R\&D of an electronic system suitable for both RPC and SiPM signal processing, which is required to exhibit a compact structure, low power consumption, high efficiency, radiation resistance and precision timing ($<100$~ps);

\item Construction and testing of an MUD prototype with 6 or more layers, equipped with a complete readout electronics system, to study the feasibility of the MUD.

\item Study on the MUD performance in a high rate and high radiation environment to explore the dependence of muon-ID abilities with a high background level.
\end{itemize}
