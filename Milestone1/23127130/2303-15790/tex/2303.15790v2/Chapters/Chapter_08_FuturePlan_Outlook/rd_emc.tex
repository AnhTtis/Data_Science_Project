The main function of the electromagnetic calorimeter (EMC) of the STCF is to realize accurate measurements of photon energy, position and arrival time under the conditions of high background count rates. For 1~GeV photons, the energy resolution should be better than 2.5\%, and the position resolution should be better than 5~mm. In addition, the time resolution should reach 300~ps @ 1~GeV to distinguish neutral particles (neutrons, photons). The pure cesium iodide crystal (pCsI) has the advantages of radiation hardness and fast time response. It is a very promising option for the EMC of the STCF spectrometer. In the following, the main R\&D items are listed.
\begin{itemize}
\item \textbf{Light Yield Study}\\
For the measurement of low-energy photons, the light yield of a crystal directly determines its energy measurement accuracy. It is necessary to study the crystal light yield according to the fluorescence characteristics of the pCsI crystal (wavelength band, decay time, etc.). On the basis of previous research, further study on wavelength shifting materials is likely to improve the light yield of the pCsI crystal.


  \item \textbf{Electronics Study}\\
The EMC readout electronics mainly include two modules: a preamplifier module and a digital processing module. The preamplifier module includes a photoelectric conversion device, an APD, and a front-end amplification circuit.
According to the requirements for high precision and large dynamic range detection of the EMC, a design yielding low noise and large dynamic range needs to be obtained for the preamplifier module. Additionally, the preamplifier module needs an antistacking design to solve the problem of preamplifier circuit saturation caused by a high background and high signal counting rate.

  \item \textbf{Pileup Study}\\
To cope with the high background, in addition to reducing the recovery time of the readout circuit as much as possible in the hardware design, the analysis method still needs to be studied.
It is planned to use a multiwaveform fitting method in the readout of the EMC, and it is also essential to verify the feasibility of this method with simulations and experiments.

  \item \textbf{Time Measurement Study}\\
It is planned to use waveform sampling readout and online waveform fitting to realize high-precision time measurement based on the premise of controllable data size. The detailed algorithm and technical implementation need to be studied.

\item \textbf{Prototype Study}\\
To verify the performance of the EMC, a small prototype needs to be developed, such as a 3 $\times$ 3 or 5 $\times$ 5 crystal array. This includes batch testing of crystals, photoelectronic devices and electronics, and finally, the test process and test standards must be established.
\end{itemize}
