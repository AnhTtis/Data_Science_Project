\subsubsection{FTOF}

\paragraph{Timing Resolution of DIRC-Type Detector}
\label{Timing_Resolution_DIRC_Detector}
The main sources contributing to the timing uncertainty of DIRC-like detector are as follows:
\begin{equation}
\sigma^{2}_{tot} \sim \sigma^2_{trk} + \sigma^{2}_{T_{0}} + (\frac{\sigma_{DIRC}}{\sqrt{N_{p.e.}}})^{2} 
= \sigma^2_{trk} + \sigma^{2}_{T_{0}} + (\frac{\sigma_{elec}}{\sqrt{N_{p.e.}}})^{2} + (\frac{TTS}{\sqrt{N_{p.e.}}})^{2} + (\frac{\sigma_{det}}{\sqrt{N_{p.e.}}})^{2}.
\label{EQ:DIRC-Time-Reso}
\end{equation}
In the equation above $N_{p.e.}$ is the number of photoelectrons, $\sigma_{trk}$ is the error caused by track reconstruction, $\sigma_{T_{0}}$ is the event reference time ($T_{0}$, i.e. when physical collision happens) error mainly affected by the collider design of STCF, $\sigma_{elec}$ is the electronic timing accuracy, $TTS$ is the single-photon transit time spread of MCP-PMT, and $\sigma_{det}$ is the time reconstruction uncertainty of the DIRC detector. From this formula, we can see that the contribution from $\sigma_{elec}$, $TTS$ and $\sigma_{det}$, which combine to $\sigma_{DIRC}$, will decrease with increasing $N_{p.e.}$, while the timing errors from $\sigma_{trk}$ and $\sigma_{T_{0}}$ keep unchanged. Their relative importance should be studied and optimization can be achieved considering all their contribution. It is noted that the uncertainty of $T_{0}$ is usually about $30 \sim 40 ps$, which is an important timing error source at STCF.

To estimate the "intrinsic" timing uncertainty by using DIRC method, i.e. $\sigma_{DIRC}$, we study the main contributing factors by considering a simple case. If a relativistic charged particle incident vertically into a thin Cherenkov radiator (e.g. fused quartz) plate, a Cherenkov light cone will be produced. The boundary of quartz plate is not considered here, and in the Cartesian coordinate system (x-y-z) the particle moves along the z direction and the incident point is at the origin (0, 0, 0). Photon detection plane is located at y=C. The Cherenkov light cone imaged at this plane exhibits a curve
\begin{equation}
x^{2}+C^{2}=z^{2} \times tan^{2}\alpha,  
\end{equation}
where $\alpha$ is the Cherenkov radiation angle. The time of propagation (TOP) of a Cherenkov photon hit the imaging plane at position (x, y, z) is given by
\begin{equation}
TOP=\frac{L}{v}=\frac{\sqrt{x^{2}+y^{2}+z^{2}}}{c/n_{g}}=\frac{n_{g}z\sqrt{1+tan^{2}\alpha}}{c}
=\frac{n_{g}z}{c cos(\alpha)}=\frac{n_{p}n_{g}z\beta}{c}, 
\label{EQ:DIRC-Time-TOP}
\end{equation}
where $n_{p}$, $n_{g}$ are the phase and group refractive index of quartz, v and ? are the velocity (group velocity) and reduced velocity of light in the medium, and L is the length of propagation (LOP) of Cherenkov photon.

The expression, 
\begin{equation}
T_{det}=T-TOP=T-\frac{n_{p}n_{g}z\beta}{c}, 
\label{EQ:DIRC-Time-DET}
\end{equation}
of the excitation time of Cherenkov radiation can be deduced, if the time T at the detection plane is measured. This $T_{det}$ is also the time when the charged particle hits the quartz plate, since Cherenkov photons emit instantaneously.

By Differentiating equation ~\ref{EQ:DIRC-Time-DET}, the sources of timing error are available, including $\sigma_{T}$ (i.e. the single photoelectron time resolution of the photon detector and the electronics, $\sigma_{SPE}$), $\sigma_{n}$ (dispersion effect), and $\sigma_{z(x, y, T)}$ the position resolution (which may depend on detailed algorithm), and $\sigma_{\beta}$ the measurement error of particle velocity, 
\begin{equation}
\sigma_{T_{DIRC}}=\sigma_{T}\oplus\frac{(n_{p}\sigma_{n_{g}}+n_{g}\sigma_{n_{p}})z\beta}{c}\oplus\frac{n_{p}n_{g}z\sigma_{\beta}}{c}\oplus\frac{n_{p}n_{g}z\beta\sigma_{z}}{c}. 
\label{EQ:DIRC-Time-Err}
\end{equation}

To quantitatively estimate $\sigma_{T_{DIRC}}$, we assume a sensitive wavelength range of 300-600 nm for the photon detector and the dispersion effect would be $\sigma_{n}\sim0.0075$, estimated from the wavelength dependence of quartz refractive index, and <n>~n@390nm=1.471. It's noted here the difference between $n_{p}$ and $n_{g}$ is neglected. The photon detector along with the readout electronics give an timing uncertainty of $\sigma_{T}=\sigma_{SPE}=50\sim70 ps$. The position error contains both the initial position uncertainty (quartz plate thickness ~2 cm) and the finite photodetector pixel size (~5 mm), which combine to $\sigma_{z}\sim6 mm$. The velocity error contribution is considered by a kaon meson at momentum $p=1 GeV/c$. It is found to be $\sigma_{\beta}\sim0.5\frac{\%\beta}{\gamma^{2}}\sim0.001$, which is rather small.

The calculation results are shown in Figure ~\ref{FIG:DIRCTIMINGERR1}, where the horizontal coordinate Z represents the extrapolated hit z position on the imaging plane. It can be seen that the timing jitter of the photon detector plays a major role when the photon propagation length is relative short (Z<=100cm), while the dispersion effect gradually becomes the dominant factor when the photon transmission distance is large, where the optical design may be optimized to correct the dispersion effect if precise timing is pursued. The  $\sigma_{z}$ effect contributes ~30-40 ps, in this calculation mainly comes from the influence of the radiator thickness, so the radiator cannot be too thick. In the applicable momentum range of DIRC technology, the incident particle velocity (or momentum) is expected to contribute very little to the timing uncertainty. Furthermore, the angular resolution contribution is not included in Figure ~\ref{FIG:DIRCTIMINGERR1}, which mainly comes from the multiple Coulomb scattering (MCS) effect and is only obvious at low momentum.
\begin{figure}[!htb]
	\centering
	\includegraphics[width=0.65\textwidth]{Figures/Figs_04_00_DetectorSubSystems/Figs_04_03_ParticleIdentification/DIRC_Timing_Error1.jpg}
	\caption{The main DIRC timing error factors and their dependences on the photon propagation distance.}
	\label{FIG:DIRCTIMINGERR1}
\end{figure}

Monte Carlo (MC) simulation studies are also performed to understand the detector physics of the PID technological candidates, and estimate the PID performance achievable. The results are consistent with the calculations and will be discussed in detail in the ~\ref{FTOF_Conceptual_Design} part.

\paragraph{Experimental Test on FTOF}

The Forward Time-Of-Flight (FTOF) detector is designed to provide charged hadron identification ($\pi/K$ separation up to $2~GeV/c$ and $p/K$ separation up to $2.5~GeV/c$) at STCF. It is positioned at $\sim 130 cm$ downstream from the collision point, in front of the ECAL, covering the polar angle of $20\deg \sim 34\deg$. Following the superB proposal, the FTOF is composed of 12 trapezoid silica radiators, each radiator attached with 14 MCP-PMTs (Hamamatsu R10754) and pico-second readout electronics at the outer end. It is supported by the carbon fabric structure. When a high energy charged particle pass through FTOF, it induces Cherenkov radiation within the radiator, emitting ultraviolet photons at specific Cherenkov angle. After multiply total reflections inside the radiator, the photons are collected by the MCP-PMT array and turn into photoelectron signal. Finally the readout electronics record the signals' arrival times and hit positions. 

Roughly, the flight time of charged particle from the collision point to FTOF is:
\begin{equation}
t = \frac{L}{c} \sqrt{1+(\frac{mc}{p})^{2}}.
\end{equation}
If we choose a possible minimum distance of FTOF from the collision point to be $1.2 m$, then $L_{min} ~ \frac{1.2}{cos(20\deg)} = 1.28 m$, m and p are the mass and momentum of charged particle.  Therefore to achieve $3\sigma$ $\pi/K$ separation at $p = 2~GeV/c$, an overall FTOF time resolution $\sim 40 ps$ is needed as shown in Fig. \ref{FIG:FTOF_PIDpower}.

\begin{figure}[!htb]
  \centering
  \includegraphics[width=0.8\textwidth]{Figures/Figs_04_00_DetectorSubSystems/Figs_04_03_ParticleIdentification/FTOF_PIDpower.jpg}
  \caption{The separation power for $\pi/K$ at different momenta.}
  \label{FIG:FTOF_PIDpower}
\end{figure}

To understand and validate the FTOF design, we performed some preliminary experimental test on the main components of FTOF, as well as a FTOF prototype. The FTOF prototype include a rectangle fused silica radiator plate and an array of PMTs. The fused silica radiator (what type? size?) has transmission rate of $95\%$ for ultraviolet light at the wavelength above $300 nm$. With an average refractive index of $1.46$, the maximum radiation angle of Cherenkov light is $46.8\deg$ with a light yield of $\sim 260 /cm$ in the wavelength range of $300 \sim 400 nm$. Its surface is polished with roughness below $0.8 nm$ (RMS) and flatness below $0.1 mm$. To be better coupled to the PMTs the coupling surface is coated with silicone oil (Rhodorsil Huile 47 V 1000) permeable to ultraviolet light. The radiator is installed in the carbon-fibre black box. There is black paper attached on the rear end of the radiator to reduce the time spread from multiple internal reflections of photons.

The type of PMT currently used for FTOF prototyping and test is R10754 from Hamamatsu Company. It is a $4 \times 4$ multi-channel micro-channel plate photomultiplier tube (MCP-PMT). Fig \ref{FIG:R10754} shows its main parameters. We also tested its performance by using a pico-second laser (Passat COMPILER, $FWHM = 4 ps$, $\lambda = 215 nm$). The laser is split into two beams, illuminating both the MCP-PMT under test (single-photon mode) and a reference MCP-PMT (multi-photon mode, used as the time reference for transit time measurement) respectively. A high bandwidth oscilloscope (type?, sampling rate: $40 GS/s$) is used to record the output signal waveforms. Fig. \ref{FIG:R10754_SPE_TTS} shows the performance of MCP-PMT at the high voltage (HV) of $3100 V$. Its gain is around $1 \times 10^{6}$, calculated by:
\begin{equation}
Gain = \frac{Q_{SPE}-Q_{ped}}{e}.
\end{equation}

The transit time spread (TTS) is $\sim 18 ps$, obtained from the time difference between the measured MCP-PMT and the reference MCP-PMT.

\begin{figure}[!htb]
  \centering
  \includegraphics[width=0.95\textwidth]{Figures/Figs_04_00_DetectorSubSystems/Figs_04_03_ParticleIdentification/R10754_parameters.jpg}
  \caption{Characteristics of R10754.}
  \label{FIG:R10754}
\end{figure}

\begin{figure}[!htb]
  \centering
  \includegraphics[width=0.45\textwidth]{Figures/Figs_04_00_DetectorSubSystems/Figs_04_03_ParticleIdentification/R10754_SPE.jpg}
  \includegraphics[width=0.45\textwidth]{Figures/Figs_04_00_DetectorSubSystems/Figs_04_03_ParticleIdentification/R10754_TTS.jpg}
  \caption{Measured characteristics of MCP-PMT: The single photoelectron (SPE) spectrum at HV of $3100 V$ and Gain of $1 \times 10^{6}$ is calculated (left). The transit time spread (TTS) of $18 ps$ is obtained from the time difference between the measured MCP-PMT and the reference MCP-PMT (right).}
  \label{FIG:R10754_SPE_TTS}
\end{figure}

As a crucial part of the pico-second level timing, the time uncertainty contributed from the electronics is expected to be below 10 picoseconds. To meet this requirement, an FPGA-based readout electronics with a multi-threshold leading-edge timing scheme is designed. It consists of a gain programmable differential amplifier (PDA), a multi-threshold differential discriminator (MDD), and a set of time-to-digital converters (TDC). Except for the PDA and the bias network, other components are implemented inside of a field-programmable gate array (FPGA) chip (Xilinx Kintex-7). The TDC module is implemented in the FPGA using an ones-counter encoding scheme, its intrinsic RMS time resolution is evaluated as $3.9 ps$. The intrinsic time performance of the entire electronics is evaluated as $5.6 ps$. More detail of the front-end electronics will be discussed in the ~\ref{FTOF_ELECTRONICS} part.

The performance of a FTOF prototype has been tested in the $5 GeV$ electron beam at DESY in 2019. The prototype installed in a black box measures the photons'arrival time as $T1$, with another MCP-PMT placed next to it as start time $T0$, as shown in Fig. \ref{FIG:FTOF_DESY_BeamTest}. Then we fit the distribution of $T1 - T0$ to estimate its time resolution. The test result after T-A correction with constant fraction thresholds is also shown in Fig. \ref{FIG:FTOF_DESY_BeamTest}. The time resolution achieves $\sim 100 ps$ for proper threshold setting. Since this result is without beam hit
position correction and any other optimization, it has large room for further improvement. According to a dedicated GEANT4 simulation with the same setup as the prototype, it's found the reconstruction of TOP of Cherenkov photon is not optimized and represents the dominant timing error source. An optimized version based on these experiences will be described in the ~\ref{FTOF_Conceptual_Design} part.

\begin{figure}[!htb]
  \centering
  \includegraphics[width=0.45\textwidth]{Figures/Figs_04_00_DetectorSubSystems/Figs_04_03_ParticleIdentification/FTOF_DESY_timing.jpg}
  \includegraphics[width=0.45\textwidth]{Figures/Figs_04_00_DetectorSubSystems/Figs_04_03_ParticleIdentification/FTOF_DESY_prototype.jpg}
  \caption{Time resolution of FTOF prototype in beam test (left) and test photo at DESY (right).}
  \label{FIG:FTOF_DESY_BeamTest}
\end{figure}

\paragraph{FTOF Electronics}
\label{FTOF_ELECTRONICS}

In detectors which use the internally reflected Cherenkov light in quartz bars to provide the particle identification, it is important to get a high time resolution, because the arrival time of the Cherenkov photon not only contains information about the momentum of particles but also can be used to distinguish signals from the background. Listed in the Table \ref{DIRCelecsys} are features of the readout electronics from some internally reflected Cherenkov detectors which are now in use or under development, including DIRC for BaBar \cite{BabarTDR}, TOP for Belle II \cite{Belle2TDR}, barrel DIRC for PANDA \cite{PandaTDR}, TORCH for LHCb \cite{TorchElec} and FDIRC for GlueX \cite{GlueXDirc}. For comparison, the requirements of STCF FTOF are also listed in this table. The concept of Detection of Internally Reflected Cherenkov light (DIRC) was first developed in the BaBar experiment, where PMTs were chosen to be the photon detectors in consideration of the cost. In BaBar, the coordinate of the lighted PMT was used to calculate the Cherenkov angle and the timing information was just used to distinguish signals from the background, so the requirement of the timing resolution was only a few nanoseconds. With the development of detectors, microchannel plate photomultiplier tubes (MCP-PMTs) have been using in more and more high-energy-physics experiments. As a photon detector, the MCP-PMT can achieve the timing resolution of sub-100 picoseconds for single-photon detection, which has a great promotion for the time performance of the experiments.

In the MCP-PMT based FTOF detectors, the particle identification relies on the Cherenkov photon arrival time rather than the position information of MCP-PMT. Therefore, the high timing resolution requirement is a markedly feature of FTOF detectors. Furthermore, FTOF detectors usually have a large channel number and require a track timing resolution below 30 picoseconds, which is a great challenge for the front-end electronics.

\begin{table*}[tb]
	\small
	\caption{Readout electronic systems of various DIRC-like detectors.}
	\label{DIRCelecsys}
	\vspace{0pt}
	\centering
	\begin{tabular}{lllllll}
		\hline
		\thead[l]{ } & \thead[l]{Babar DIRC} & \thead[l]{Belle II TOP} & \thead[l]{PANDA Barrel DIRC} & \thead[l]{LHCb TORCH} & \thead[l]{GlueX FDIRC} & \thead[l]{STCF FTOF} \\
		\hline
		Photodetector	&PMT        &MCP-PMT	&MCP-PMT	&MCP-PMT	&MAPMT	    &MCP-PMT \\
		Channels	    &10752	    &8192   	&\~11k      &\~200k     &11520     	&768     \\
		$\sigma_{T}$ (SPE)	&1.7ns	&100ps    	&100ps     	&70ps      	&600ps     	&70ps    \\
		Method	        &TDC     	&SCA	    &TOT      	&TOT       	&TDC       	&Multi-threshold TOT \\
		\hline
	\end{tabular}
\end{table*}

The preliminary structure of the FTOF readout electronics is shown in Fig. \ref{FTOF_electronics_architecture}. The readout electronics consists of the front-end board and the data-control board. The front-end board utilizes a time over multi-threshold scheme to extract timing information from analog signals. The signals from MCP-PMT are pre-amplified firstly. The gains of amplifiers are set independently to compensate for the gain variations of the individual MCP-PMT channels. The amplified signals are fed into high-performance comparators, each of them has a different threshold. The outputs of these comparators are fed into the FPGA. The time-to-digital converter (TDC) module will be implemented in the FPGA. And the TDC measures the arrival time of both edges of the comparator outputs with high accuracy. Then the front-end board passes the resulting binary data stream to the data-control board. The data-control board not only collect data stream from front-end boards but also fan out high-performance clock and control signal to front-end boards. According to the structure described above, we can briefly estimate the amount of data that the readout electronic system fed into the subsequent DAQ system. Assuming the event rate of the FTOF detector is 5k/s, the data rate output to the subsequent DAQ system can be simply calculated as 4.8MB/s without considering the effects of the detector background noise and crosstalk.

It is crucial for high time performance that the front-end electronics have high bandwidth and a high signal-to-noise ratio. The time resolution of the FPGA-based TDC was evaluated as 3.9ps in our previous work \cite{fastTDC}. And we will implement the TDC that can measure narrow pulse widths, of which the time resolution will be better than 5ps. And the TDC performance in the radiation environment will be evaluated. The results of the evaluation will affect our system structure design. A stable, low-jitter detector-wide clock distribution network also needs to be developed. Its long-term stability, short-term stability, and temperature stability will be carefully evaluated. The jitter of the clock distribution network which is better than 15ps would satisfy the design requirements.

\begin{figure}[!htb]
	\centering
	\includegraphics[width=0.8\textwidth]
	{Figures/Figs_04_00_DetectorSubSystems/Figs_04_03_ParticleIdentification/Figs_FTOF_Electronics/FTOF_Electronics_Struture.jpg}
	\caption{The preliminary structure of the FTOF readout electronic system.}
	\label{FTOF_electronics_architecture}
\end{figure}

The goal of the electronic $R\&D$ is to understand the influence of various electronic parts on the timing performance and study the possible problems that may appear in practical applications. Adequate R and D is the foundation to design an electronic system that satisfies all the requirements. In PANDA Barrel DIRC and TORCH, the TOT method is used to eliminate the time-amplitude walk while extracting the time information. To further improve the measurement resolution of the leading-edge timing method, the multi-threshold TOT method will be used in the front-end board. And the research of the multi-threshold timing method is also in progress, with selecting more measured points, the higher measurement resolution is supposed to achieve.
The influence of the front-end amplifier circuit on the timing performance of the leading edge needs to be carefully evaluated, including the impacts of the pre-amplifiers and comparators. With the understanding of key parameters that affect the timing performance, the circuit structure and design can be optimized to get higher integration and lower power consumption.

The readout electronics must sustain the radiation loads during the operational lifetime of STCF. Among them, the radiation influence on the FPGA-based TDC needs to be regarded first. The performance of TDCs under radiation will be evaluated in two situations. The first one is the damage of single event upsets to the FPGA-based TDC in the radiation environment. The second is the change in the FPGA-based TDC performance after long-term work in the radiation environment. Research on the performance of FPGA-based TDCs in the radiation environment will not only affect the structural design of the front-end electronics but also promote the application of FPGA-based TDC in high-energy physics experiments.

FTOF readout electronic system consists of many front-end boards that need to be synchronized, so the jitter of the distribution clock network is a key parameter of the electronic system. Currently, there is no working distribution clock network that can satisfy the requirements of FTOF. Therefore, it is necessary to develop the clock distribution network that meets the requirement of FTOF based on existing technology.

The electronic readout system will be tested separately for each part and then the system will be tested totally. Single-photon timing performance is a key indicator of whether the electronics system meets the design requirements. FTOF readout electronics need to calibrate the offset and drift in the system to reduce their impact on the timing performance of the system. The design of the electronic system and manufacturing of the components will introduce the offset. For the offset from the electronics system, some of them can be automatically calibrated by designing a self-calibration module, and the others need external signals to fulfill calibration. The external signals can be generated by signal generators or photon detectors illuminated by the laser. Short-term drift may be caused by environmental changes, such as temperature affecting the clock distribution network. And long-term drift may be caused by the long-term operation of electronic components. In the calibration progress, the short-term drift can be compensated by using sensors and corresponding control methods, while the Long-term drift requires long-time experiments on key components to find out the solution after determining the system design scheme.

\input{Chapters/Chapter_04_00_DetectorSubSystems/Chapter_04_03_ParticleIdentification/04_Ref_FTOFElectronics}
\input{Chapters/Chapter_04_00_DetectorSubSystems/Chapter_04_03_ParticleIdentification/04_Ref_FTOF}
