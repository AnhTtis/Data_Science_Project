\subsubsection{Conceptual Design}

\paragraph{1) RICH Conceptual Design}
The conceptual design of the RICH sub-system is shown in Fig. \ref{FIG:RICH_Conceptual_Design}. The RICH detector is placed between MDC and EMC, with 0.83 the solid coverage of the barrel part. The RICH consists of $12$ identical block modules. The inner radius is $~850$ mm, and the outter radius is $~950$\,mm. one of the RICH module will be $2800$\,mm long, $450$\,mm wide, and $130$\, mm height. The whole module is enclosed in a $~3$\,mm light-tight aluminum box with support from each side.

The RICH module consists radiators, light propagation zone, photo-cathode, multiplier, and anode. Except radiators, the other parts will be grouped as one unit by smaller modules. The radiator is segmented into 5 pieces, in which the middle one contains the largest quartz box with $~1800$ mm long. The other four segmented radiators are $253$\,mm and $266$\,mm, and rotated 10 and 20 degrees, respectively. The purpose of this design is to reduce the charged track incident angle, so to avoid the Cherenkov full reflection inside the radiator. These radiators are liquid C$_6$F$_14$ sealed inside quartz boxes, with pipes connected one to another. A purification system will be employed to continuously purify the liquid, together the Ar/CH$_4$ mixture gas inside the chamber.

\begin{figure}[!htb]
  \centering
  \includegraphics[width=0.75\textwidth]{Figures/Figs_04_00_DetectorSubSystems/Figs_04_03_ParticleIdentification/RICH_Conceptual_Structure.pdf}
  \caption{The conceptual design of RICH.}
  \label{FIG:RICH_Conceptual_Design}
\end{figure}

\paragraph{2) RICH Performance in Simulation}
Geant4 simulations are performed to study the expected performance of RICH. $\pi$/K/p are emitted from the interaction point inside a $1$ Tesla magnetic field. The optical properties for radiator(quartz and C$_6$F$_14$) are defined according to measurement results[], and the absorption of gas is calculated from H$_2$O and O$_2$ absorption cross section with the hypothesis contamination of $10$\,ppm each. The momentum and azimuth angle scan is then perform and a typical ring of Cherenkov hit pattern is shown in \ref{FIG:RICH_ring_result}). By applying the reconstruction algorithm, a $~1$ mrad angular resolution is obtained.

\begin{figure}[!htb]
  \centering
  \includegraphics[width=0.45\textwidth]{Figures/Figs_04_00_DetectorSubSystems/Figs_04_03_ParticleIdentification/pip_2GeV_Cerenkov.pdf}
  \caption{The simulated RICH hit pattern.}
  \label{FIG:RICH_ring_result}
\end{figure}


\paragraph{1) FTOF Conceptual Design}
Based on the results from both simulation study and experimental test, we come up with the conceptual design of the FTOF sub-system, as shown in Fig. \ref{FIG:FTOF_Conceptual_Design}. The proposed FTOF detector consists of two identical endcap discs, positioned $~1400 mm$ along the beam direction away from the collision point. Each disc is made up with four quadrantal sectors, each with an inner radius of $\sim 560 mm$ and an outer radius of $\sim 1050 mm$, covering $\sim 21 \deg - 38 \deg$ in polar angle and $90 \deg$ in azimuthal angle. The detail structure inside a sector is also depicted in Fig. \ref{FIG:FTOF_Conceptual_Design}. The planar synthetic fused silica radiator is fan shaped that can be viewed as a composite structure of 3 trapezoidal units, each $\sim 295 mm$ (inner side) $/$ $\sim 533 mm$ (outer side) wide, $\sim 470 mm$ high and $15 mm$ thick. An array of $3 \times \sim 14-16$ multi-anode MCP-PMTs are optical coupled to the radiator along the outer side. The whole sector is enclosed in a light-tight black box made of $5 mm$ thick carbon fibre, occupying $\sim 200 mm$ space along the beam (Z) direction.

\begin{figure}[!htb]
	\centering
	\includegraphics[width=0.75\textwidth]{Figures/Figs_04_00_DetectorSubSystems/Figs_04_03_ParticleIdentification/FTOF_Conceptual_Structure1.jpg}
	\includegraphics[width=0.20\textwidth]{Figures/Figs_04_00_DetectorSubSystems/Figs_04_03_ParticleIdentification/FTOF_Conceptual_Structure2.jpg}
	\caption{The conceptual design of FTOF.}
	\label{FIG:FTOF_Conceptual_Design}
\end{figure}

%% The following paragraphs are based on the work by Ziwei Li
\paragraph{2) FTOF Performance in Simulation}
Geant4 simulations are performed to study the expected performance of FTOF with such design. A $20 mm$ thick Aluminum plate is added $100 mm$ before the FTOF detector, in order to simulate the material budget of the main drift chamber (MDC) endcap. When tracking the photon propagation, the inner and outer side of the FTOF radiator are set absorptive, while the two lateral sides are set reflective (reflection factor $\sim 92\%$). The surface roughness of the radiator is simulated by randomizing the normal direction of facet by $\sigma = 0.1 \deg$ (corresponding to an average reflection factor of $\sim 97\%$). Pion and kaon particles are emitted from the interaction point at different momenta and directions. Angular angle $23.66 \deg$, $26.65 \deg$, $29.48 \deg$, $32.17 \deg$ and $34.71 \deg$, azimuth angle $5 \deg$, $15 \deg$, $25 \deg$, $35 \deg$ and $45 \deg$, and particle momenta $1.0 ~GeV/c$, $1.2 ~GeV/c$, $1.4 ~GeV/c$, $1.6 ~GeV/c$, $1.8 ~GeV/c$ and $2.0 ~GeV/c$ are scanned. A typical Cherenkov photon hit pattern is displayed as in Fig. \ref{FIG:FTOF_TOP_Resolution} by pions at $p = 1 ~GeV/c$, $\theta = 23.66 \deg$ and $\phi = 15 \deg$. A clear correlation between the time of propagation (TOP) and the hit position (sensor pixel) is demonstrated. Two bands, one for direct photons and another for indirect photons with one reflection off the lateral side, are well separated except for a few sensors close to the side. The mean number of photons detected by the MCP-PMT arrays is $\sim 17$. By applying a TOP-position calibration, where the average track length collected by each sensor pixel and the average velocity of photons are used to calculate the TOP (so no dispersion effect is accounted for), the timing resolution of $\sim 20 ps$ or better is obtained (as shown in Fig. \ref{FIG:FTOF_TOP_Resolution}), with no further correction (such as the dispersion effect).

\begin{figure}[!htb]
	\centering
	\includegraphics[width=0.45\textwidth]{Figures/Figs_04_00_DetectorSubSystems/Figs_04_03_ParticleIdentification/FTOF_TOP_1.jpg}
	\includegraphics[width=0.45\textwidth]{Figures/Figs_04_00_DetectorSubSystems/Figs_04_03_ParticleIdentification/FTOF_TOPReso_1.jpg}
	\caption{The simulated TOP vs. hit position pattern of FTOF, and the timing resolution after calibration.}
	\label{FIG:FTOF_TOP_Resolution}
\end{figure}

Besides the timing resolution mentioned above ($\sigma_{det}$), the total time uncertainty of the FTOF detector includes several other components: the TTS of the MCP-PMT ($\sigma_{TTS} = 70ps$ is used in our calculation), the electronics time jitter ($\sigma_{elec} = 10ps$), the time error due to the incident particles¡¯ track reconstruction ($\sigma_{trk} = 10ps$), the event start time resolution ($\sigma_{T_{0}} = 40ps$). It can be estimated by formula \ref{EQ:DIRC-Time-Reso}. With the simulation results, the expected timing resolution in different FTOF regions are deduced, as shown in Fig. \ref{FIG:FTOF_TReso_NPE} for pions (kaons are very similar) at $2 ~GeV/c$. The black numbers are total time resolution while the red ones are the intrinsic detector timing resolution $\sigma_{det}$. It is clear $\sigma_{det}$ becomes worse at lower $\theta$, which corresponds to hits closer to the inner side of FTOF radiator, and slightly improved at larger $\phi$ (best at $\phi = 45 \deg$), which is further away from the lateral side of FTOF radiator. This $\theta$ and $\phi$ dependence is mainly reflected by the total number of detected Cherenkov photons, as also shown in Fig. \ref{FIG:FTOF_TReso_NPE}. It is also evident the event start time resolution $\sigma_{T_{0}} = 40 ps$ dominates the over all timing error, so optimized design of STCF bunch size (which dictates the timing uncertainty of event start time) is crucial. Furthermore we can find the constraint on $\sigma_{elec}$ can be relaxed, since the single photon response of MCP-PMT $\sigma_{TTS}$ dominates the terms with $\frac{1}{NPE}$ scaling.

\begin{figure}[!htb]
	\centering
	\includegraphics[width=0.45\textwidth]{Figures/Figs_04_00_DetectorSubSystems/Figs_04_03_ParticleIdentification/FTOF_TReso_All_1.jpg}
	\includegraphics[width=0.45\textwidth]{Figures/Figs_04_00_DetectorSubSystems/Figs_04_03_ParticleIdentification/FTOF_NPE_1.jpg}
	\caption{The expected overall FTOF timing resolution (left), and the number of detected direct photons by $\pi^{\pm}$ particle at $2~GeV/c$.}
	\label{FIG:FTOF_TReso_NPE}
\end{figure}

With the expected timing performance obtained, the PID power, i.e. $K/\pi$ separation power is defined by the following formula
\begin{equation}
\label{eq::FTOF-PID-Power}
\Delta T=\frac{L}{\beta c} \frac{\Delta m^2}{2p^2}\sim\frac{L}{c}\frac{\Delta m^2}{2p^2},
PID power = \frac{\Delta T}{\sigma_{tot}},
\end{equation}
where $\Delta T$ is the TOF difference between pion and kaon with the same momentum $p$, $L$ is the flight length, and $\sigma$ is the total time resolution of FTOF. The results are shown in Fig. \ref{FIG:FTOF_PIDPower}, for $2~GeV/c$ $\pi/K$ at different regions of the FTOF. Due to longer $L$ and better timing resolution at larger $\phi$, the FTOF PID capability is better for tracks hitting the radiator closer to its outer side. The worst case is at the corner of inner and lateral edges, where the separation power is still acceptable, $3.1$.

\begin{figure}[!htb]
	\centering
	\includegraphics[width=0.7\textwidth]{Figures/Figs_04_00_DetectorSubSystems/Figs_04_03_ParticleIdentification/FTOF_PID_Power_1.jpg}
	\caption{The expected PID capability under such timing performance for $\pi/K$ separation at $2~GeV/c$.}
	\label{FIG:FTOF_PIDPower}
\end{figure}

%% The following paragraphs are based on the work by BinBin Qi
\paragraph{1) FTOF Reconstruction Algorithm}
The FTOF reconstruction is performed in the coordinate as in Fig. \ref{FIG:FTOF_Reco_Coor}, for one FTOF quadrant. According to the Cherenkov angle relation
\begin{equation}
cos(\bar{\theta_{c}}) = \frac{1}{n_{p}\beta} = \frac{\vec{v_{t}} \dot \vec{v_{p}}}{|v_{t}| \dot |v_{p}|},
\end{equation}
where $\vec{v_{t}}=(a,b,c)$ is the incident particle velocity vector when impinging the radiator, $\vec{v_{p}}$ is the velocity vector of emitted Cherenkov photon, $n_{p}$ is the refractive index of the radiator and $\beta$ is the reduced speed of the particle. The directional components of $\vec{v_{p}}$ can be expressed as $(\Delta{X},\Delta{Y},\Delta{Z})$, where the 3D spatial difference between the photon sensor pixel and the incident position of the particle on the radiator surface, as depicted in Fig. \ref{FIG:FTOF_Reco_Coor} (right). Although the 2D (X and Y) difference can be readily obtained, $\Delta{Z}$ must be deduced with certain particle specie hypothesis. If $V = cos(\bar{\theta_{c}})$ is known, the equation about $\Delta{Z}$ can be found as
\begin{equation}
\label{eq::FTOF-DZ-Solver}
(c^{2}-V^{2})\Delta{Z}^{2} + 2c(a\Delta{X}+b\Delta{Y})\Delta{Z} + (a\Delta{X}+b\Delta{Y})^{2} - V^{2}(\Delta{X}^{2}+\Delta{Y}^{2}) = 0.
\end{equation}
By solving this equation we find $\Delta{Z}=\frac{-B\pm\sqrt{B^{2}-4AC}}{2A}$, with $A=c^{2}-V^{2}$, $B=2c(a\Delta{X}+b\Delta{Y})$ and $C=(a\Delta{X}+b\Delta{Y})^{2} - V^{2}(\Delta{X}^{2}+\Delta{Y}^{2})$. In order to get a real solution, $\Delta = B^{2}-4AC \ge 0$ is required. Furthermore, after some physical cuts $V > 0$ (Cherenkov photon forwardly emitted) and $\frac{\Delta{X}^{2}+\Delta{Y}^{2}}{\Delta{X}^{2}+\Delta{Y}^{2}+\Delta{Z}^{2}} \ge \frac{1}{n_{p}^{2}}$ (ensuring internal total reflection) are applied, the minimal solution of equation ~\ref{eq::FTOF-DZ-Solver} ($\Delta{Z} = min{|\Delta{Z}_{1}|,|\Delta{Z}_{2}|}$) is taken as the optimum solution.

\begin{figure}[!htb]
	\centering
	\includegraphics[width=0.45\textwidth]{Figures/Figs_04_00_DetectorSubSystems/Figs_04_03_ParticleIdentification/FTOF_Reco_Coor.jpg}
	\includegraphics[width=0.45\textwidth]{Figures/Figs_04_00_DetectorSubSystems/Figs_04_03_ParticleIdentification/FTOF_Reco_Dir.jpg}
	\caption{The coordinate system used in FTOF reconstruction (left) and the direction of Cherenkov photon (deep blue line).}
	\label{FIG:FTOF_Reco_Coor}
\end{figure}

The timing error of such approach is estimated by adding up possible factors, such as the dispersion effect, the finite photon sensor size and the propagation length of photons inside the radiator. Assuming a pion of $p \sim 1 - 2~GeV/c$ cross the radiator perpendicularly, with $H = 0.5 m$ (defined as in Fig. \ref{FIG:FTOF_Reco_Coor}), the expected timing uncertainty is shown in Fig. \ref{FIG:FTOF_Reco_TLerr} for sensors at different positions. The pitch of photon sensor is $5.5 mm$. No multiple Coulomb scattering (MCS) effect is accounted for. It's obvious the detector originated timing uncertainty is no more than $40 ps$ with this FTOF structure. Furthermore we find the reconstructed length of photon propagation inside the radiator agree with the MC truth to a precision of $\sim 3.3 mm$, as also shown in Fig. \ref{FIG:FTOF_Reco_TLerr}. The reconstruction algorithm works well for most photon sensors, no matter the incident position of particles, except for a few sensors near the lateral side.

\begin{figure}[!htb]
	\centering
	\includegraphics[width=0.45\textwidth]{Figures/Figs_04_00_DetectorSubSystems/Figs_04_03_ParticleIdentification/FTOF_Reco_Terr.jpg}
	\includegraphics[width=0.45\textwidth]{Figures/Figs_04_00_DetectorSubSystems/Figs_04_03_ParticleIdentification/FTOF_Reco_Lerr.jpg}
	\caption{The expected timing error and propagation length uncertainty of Cherenkov photons in the FTOF quadrant.}
	\label{FIG:FTOF_Reco_TLerr}
\end{figure}

By applying formula
\begin{equation}
TOF = T - TOP - T_{0} = T - \frac{LOP}{v_{g}} - T_{0} = = T - \frac{LOP\times\bar{n_{g}}}{c} - T_{0},
\end{equation}
the TOF information is obtained. Shown in Fig. \ref{FIG:FTOF_Reco_TOFreso} are the time resolution of the FTOF detector for single photo-electron (SPE) and average of all photons, without taking the timing jitter of MCP-PMT and electronics into account. For a SPE, the intrinsic time resolution of the FTOF is $\sim 41 ps$, while by averaging timing information over $\sim 18$ detected photons the timing jitter shrinks to $\sim 10 ps$. It's also noted that in the TOF distribution plot, a low (but visible) long tail shows on both sides of the main peak. The tail is mainly caused by secondary particles along the primary pion, mainly delta-electrons.

\begin{figure}[!htb]
	\centering
	\includegraphics[width=0.45\textwidth]{Figures/Figs_04_00_DetectorSubSystems/Figs_04_03_ParticleIdentification/FTOF_Reco_TOFspe.jpg}
	\includegraphics[width=0.45\textwidth]{Figures/Figs_04_00_DetectorSubSystems/Figs_04_03_ParticleIdentification/FTOF_Reco_TOFall.jpg}
	\caption{The TOF resolution of the FTOF detector for single photo-electron and average of all photons.}
	\label{FIG:FTOF_Reco_TOFreso}
\end{figure}

The TOF information is deduced and compared to the expectation of each hypothesis particle. Fig. \ref{FIG:FTOF_Reco_TOFPID} shows the reconstructed intrinsic TOF distributions of both pions and kaons at $2 GeV/c$. One can easily find if the particle hypothesis is correct, the reconstructed TOF peak is at its right position, with a resolution of $\sim 10ps$. However if the hypothesis is not true, the reconstructed TOF peak is shifted with respect to its expectation. The shift makes the separation between pion and kaon TOF peaks even larger, which may benefit the PID power. When convoluting all contributing factors, as in formula \ref{EQ:DIRC-Time-Reso}, the overall reconstructed TOF time resolution is $\sim 45-50 ps$, which is shown in Fig. \ref{FIG:FTOF_Reco_TOFPID}. By directly comparing the TOF information, a $3.0 \sigma$ separation power for $\pi/K$ at $2 GeV/c$ is achieved, consistent with the GEANT4 simulation, and fulfils the required PID capability of FTOF. Furthermore the separation power gets stronger if we compare the reconstructed TOFs of various hypotheses for the same set of particles, mainly due to the beneficial time shift under wrong hypothesis (as in Fig. \ref{FIG:FTOF_Reco_TOFPID}). For either pion or kaon samples at $2 GeV/c$, the separation power reaches $\sim 4\sigma$ by comparing the TOF distributions under $\pi$ or $K$ hypotheses.

\begin{figure}[!htb]
	\centering
	\includegraphics[width=0.45\textwidth]{Figures/Figs_04_00_DetectorSubSystems/Figs_04_03_ParticleIdentification/FTOF_Reco_TOFPID_1.jpg}
	\includegraphics[width=0.45\textwidth]{Figures/Figs_04_00_DetectorSubSystems/Figs_04_03_ParticleIdentification/FTOF_Reco_TOFPID_2.jpg}
	\caption{The TOF PID capability of the FTOF detector for $\pi/K$ separation at $2 GeV/c$, without (left) and with (right) contributions from other timing uncertainties.}
	\label{FIG:FTOF_Reco_TOFPID}
\end{figure}

To further evaluate the PID capability of FTOF, we apply the likelihood method. The likelihood function is constructed by
\begin{equation}
\label{eq::FTOF-Likelihood-Function}
\mathcal{L}_{h} = \Pi^{i}_{i=1}f_{h}(TOF^{h}_{i}),
\mathcal{l} = \mathcal{L}_{\pi} - \mathcal{L}_{K},
\end{equation}
where $h$ denotes hadron species (in our case $\pi$ and $K$) and $i$ accounts for each detected photon. The probability density function $f_{h}$ is taken as a Gaussian fit to the expected TOF distribution (as in Fig. \ref{FIG:FTOF_Reco_TOFPID} (right)), plus a constant background of 0.05. Shown in Fig. \ref{FIG:FTOF_Reco_LHPID} are the reconstructed $\mathcal{l}$ for $2 GeV/c$ $\pi$ and $K$ emitted at different angles. Despite of the very different particle directions, the separation power of FTOF are similar, reaches $\sim 4\sigma$ or better all over the FTOF sensitive area. These results also agree with the results in Fig. \ref{FIG:FTOF_Reco_TOFPID}.

\begin{figure}[!htb]
	\centering
	\includegraphics[width=0.45\textwidth]{Figures/Figs_04_00_DetectorSubSystems/Figs_04_03_ParticleIdentification/FTOF_Reco_LHPID_1.jpg}
	\includegraphics[width=0.45\textwidth]{Figures/Figs_04_00_DetectorSubSystems/Figs_04_03_ParticleIdentification/FTOF_Reco_LHPID_2.jpg}
	\caption{The likelihood PID capability of the FTOF detector for $\pi/K$ separation at $2 GeV/c$, emitted at different angles.}
	\label{FIG:FTOF_Reco_LHPID}
\end{figure}

\paragraph{1) FTOF Background}
The distributions of background particles obtained by specific MDI and background study are used as input to the FTOF geant4 simulation, and then to simulate the effect of background on FTOF performance. The background particles on FTOF are mainly secondary electrons, for which the energy and angular distribution are shown in Figure \ref{FIG:FTOF_BKG_SR}. The background hit rate on a single FTOF disc is about $7\time10^{7} Hz$, including two main parts: the beam-induced background ($25\%$) and the physical background ($75\%$). The beam-induced background is uniformly distributed in time, which is not related to the beam bunch crossing, while the physical background is related to the collisions during bunch crossing (once per 8 ns), exhibiting characteristic time structure. The time distribution of the two background components are shown in Figure 1.2. The beam background is uniformly distributed in time, representing by a uniform sampling in the simulation. The physical background is correlated with the collision frequency, so the time $T=T_{01}+T_{02}+T_{flight}$, where $T_{01}$ (=0, 8, 16 ns …) is the beam bunch crossing time, $T_{02}$ is the beam collision time uncertainty due to the length of beam bunch, uniformly distributed in the [-50 ps, 50 ps] interval, $T_{flight}$ is the flight time of background particle from the beam collision point to the FTOF (illustrated in Figure \ref{FIG:FTOF_BKG_T} right). By Monte Carlo sampling, the physical background time distribution can be obtained. Combining with the uniform time distribution of the beam-induced background, the overall time distribution of the background hit on FTOF can be obtained, as shown in Figure \ref{FIG:FTOF_ALLBKG_T}.

\begin{figure}[!htb]
	\centering
	\includegraphics[width=0.45\textwidth]{Figures/Figs_04_00_DetectorSubSystems/Figs_04_03_ParticleIdentification/FTOF_BKG_Ek.jpg}
	\includegraphics[width=0.45\textwidth]{Figures/Figs_04_00_DetectorSubSystems/Figs_04_03_ParticleIdentification/FTOF_BKG_Ang.jpg}
	\caption{Energy distribution (left) and polar angle distribution (right) of background particles hitting FTOF.}
	\label{FIG:FTOF_BKG_SR}
\end{figure}

\begin{figure}[!htb]
	\centering
	\includegraphics[width=0.45\textwidth]{Figures/Figs_04_00_DetectorSubSystems/Figs_04_03_ParticleIdentification/FTOF_BeamBkg_T.jpg}
	\includegraphics[width=0.45\textwidth]{Figures/Figs_04_00_DetectorSubSystems/Figs_04_03_ParticleIdentification/FTOF_PhyBkg_T.jpg}
	\caption{Time-of-flight distribution of background particles (starting from the collision time) on FTOF: beam-induced background (left) and physical background (right).}
	\label{FIG:FTOF_BKG_T}
\end{figure}

\begin{figure}[!htb]
	\centering
	\includegraphics[width=0.65\textwidth]{Figures/Figs_04_00_DetectorSubSystems/Figs_04_03_ParticleIdentification/FTOF_AllBkg_T.jpg}
	\caption{Overall time distribution of background particles on FTOF.}
	\label{FIG:FTOF_ALLBKG_T}
\end{figure}

In the Geant4 simulation, the hit position of the background particles is uniformly distributed over the FTOF detector surface, and the energy and angle are sampled by the distribution in Figure \ref{FIG:FTOF_BKG_SR}. The time window of the signal acquisition in the simulation is 100 ns, within the interval [-40 ns, 60 ns] so that the real signal is in the middle of the time window. In this time window, the number of background particles is given by Poisson distribution, and the time distribution is sampled according to Figure \ref{FIG:FTOF_ALLBKG_T}.

10000 $\pi$ and K particles are generated in the Geant4 simulation along with the background samples to study the effect of the background. The background may greatly increase the number of photoelectrons detected by FTOF in a single event, resulting increased possibility of multiple hits in a single channel. The correction of multiple hits in data processing is applied, which means that in the time window of [-40 ns, 60 ns], when a single channel has multiple hits, only the first arrival photoelectron signal is taken and all other hits are dismissed. The background level in the simulation is set to $7\times10^{7} Hz$, according to the background study. As shown in Figure \ref{FIG:FTOF_NPE_BKG}, the average number of photoelectrons for a pion at $p=2 GeV/c$ is 22, increased to about 124 after taking the background hits into account, and reduced to about 88 after the correction of multiple hits in single channel. It’s noted that if the average number of photoelectrons is 124 for a time window of 100 ns, and assuming a MCP-PMT gain of $10^{6}$ , the average accumulated charge density on the MCP-PMT anode is $120 C/cm^{2}$, over 10-year STCF operation ($50\%$ run time).

\begin{figure}[!htb]
	\centering
	\includegraphics[width=0.65\textwidth]{Figures/Figs_04_00_DetectorSubSystems/Figs_04_03_ParticleIdentification/FTOF_NPE_BKG.jpg}
	\caption{Distribution of number of photoelectrons generated by a $2 GeV/c$ pion, without and with background.}
	\label{FIG:FTOF_NPE_BKG}
\end{figure}

Shown in Figure \ref{FIG:FTOF_XTHIT_BKG} is the 2-D time-position map of FTOF hit. It can be seen that hits by the background particles are uniformly distributed throughout the phase space, while the real signal hits are concentrated as a band. After time reconstruction, the TOF distribution of the single photoelectron signal can be obtained, as shown in Figure \ref{FIG:FTOF_SPETOF_BKG}. The reconstruction TOF of the real signal in the figure is a Gaussian distribution (mean ~5.2 ns, sigma ~100 ps), while the TOF of background particles are uniformly distributed. Some single-photon electrons with zero TOF in the figure do not meet the reconstruction conditions and are taken as background during the reconstruction process. Due to the uniform distribution of the reconstructed background signal, the influence of the background can be eliminated by using the maximum likelihood method. The $pi$/K resolution is found to be 4.2σ, as shown in Figure \ref{FIG:FTOF_PID_BKG}. Moreover, even with the correction of multiple hits, the pi/K resolution remains at 4.1σ, i.e. the background effect on $pi$/K identification is fairly small.

\begin{figure}[!htb]
	\centering
	\includegraphics[width=0.45\textwidth]{Figures/Figs_04_00_DetectorSubSystems/Figs_04_03_ParticleIdentification/FTOF_XTHit_BKG.jpg}
	\includegraphics[width=0.45\textwidth]{Figures/Figs_04_00_DetectorSubSystems/Figs_04_03_ParticleIdentification/FTOF_XTHit_BKG_Cor.jpg}
	\caption{2-D time-position map of FTOF hit, without (left) and with multiple-hit correction.}
	\label{FIG:FTOF_XTHIT_BKG}
\end{figure}

\begin{figure}[!htb]
\centering
\includegraphics[width=0.45\textwidth]{Figures/Figs_04_00_DetectorSubSystems/Figs_04_03_ParticleIdentification/FTOF_SPETOF_BKG.jpg}
\includegraphics[width=0.45\textwidth]{Figures/Figs_04_00_DetectorSubSystems/Figs_04_03_ParticleIdentification/FTOF_SPETOF_BKG_Cor.jpg}
\caption{Reconstructed TOF distribution of single photoelectron signal, without (left) and with multiple-hit correction.}
\label{FIG:FTOF_SPETOF_BKG}
\end{figure}

\begin{figure}[!htb]
\centering
\includegraphics[width=0.45\textwidth]{Figures/Figs_04_00_DetectorSubSystems/Figs_04_03_ParticleIdentification/FTOF_PID_BKG.jpg}
\includegraphics[width=0.45\textwidth]{Figures/Figs_04_00_DetectorSubSystems/Figs_04_03_ParticleIdentification/FTOF_PID_BKG_Cor.jpg}
\caption{$\pi$/K identification capability (at $2 GeV/c$) by FTOF, without (left) and with multiple-hit correction.}
\label{FIG:FTOF_PID_BKG}
\end{figure}

Considering the Poisson fluctuation of background count, the influence on the FTOF PID ability of an extreme condition with 3 standard deviation above the average background level is also studied by the simulation. The background hit rate is increased to $2.1\times10^{8} Hz$ per single FTOF disc. As shown in Figure \ref{FIG:FTOF_NPE_MAXBKG}, under such a high counting rate, the average number of photoelectrons in the 100 ns window is 300, and after the multiple-hit correction is about 200. Figure \ref{FIG:FTOF_XTHIT_MAXBKG} shows the 2D time-position distribution of photoelectrons within 100 ns window. Figure \ref{FIG:FTOF_SPETOF_MAXBKG} shows the TOF distribution of single photoelectron. It can be seen that the reconstruction results are similar to those in Figure \ref{FIG:FTOF_XTHIT_BKG} and \ref{FIG:FTOF_SPETOF_BKG}. The TOF distribution is slightly broadened than the previous result, with a Gaussian sigma ~ 104 ps. Although the number of photoelectrons from the background increases dramatically, the final effect on $\pi$/K resolution is minor, as shown in Figure \ref{FIG:FTOF_PID_MAXBKG}. The resolution $\pi$/K of 3.9 σ and 3.8 σ before and after multiple-hit correction still meet the requirement of 3σ $\pi$/K separation for the FTOF.

\begin{figure}[!htb]
\centering
\includegraphics[width=0.65\textwidth]{Figures/Figs_04_00_DetectorSubSystems/Figs_04_03_ParticleIdentification/FTOF_NPE_MAXBKG.jpg}
\caption{Distribution of number of photoelectrons generated by a $2 GeV/c$ pion, without and with enhanced background.}
\label{FIG:FTOF_NPE_MAXBKG}
\end{figure}

\begin{figure}[!htb]
\centering
\includegraphics[width=0.45\textwidth]{Figures/Figs_04_00_DetectorSubSystems/Figs_04_03_ParticleIdentification/FTOF_XTHit_MAXBKG.jpg}
\includegraphics[width=0.45\textwidth]{Figures/Figs_04_00_DetectorSubSystems/Figs_04_03_ParticleIdentification/FTOF_XTHit_MAXBKG_Cor.jpg}
\caption{2-D time-position map of FTOF hit in the condition of enhanced background, without (left) and with (right) multiple-hit correction.}
\label{FIG:FTOF_XTHIT_MAXBKG}
\end{figure}

\begin{figure}[!htb]
\centering
\includegraphics[width=0.45\textwidth]{Figures/Figs_04_00_DetectorSubSystems/Figs_04_03_ParticleIdentification/FTOF_SPETOF_MAXBKG.jpg}
\includegraphics[width=0.45\textwidth]{Figures/Figs_04_00_DetectorSubSystems/Figs_04_03_ParticleIdentification/FTOF_SPETOF_MAXBKG_Cor.jpg}
\caption{Reconstructed TOF distribution of single photoelectron signal in the condition of enhanced background, without (left) and with (right) multiple-hit correction.}
\label{FIG:FTOF_SPETOF_MAXBKG}
\end{figure}

\begin{figure}[!htb]
\centering
\includegraphics[width=0.45\textwidth]{Figures/Figs_04_00_DetectorSubSystems/Figs_04_03_ParticleIdentification/FTOF_PID_MAXBKG.jpg}
\includegraphics[width=0.45\textwidth]{Figures/Figs_04_00_DetectorSubSystems/Figs_04_03_ParticleIdentification/FTOF_PID_MAXBKG_Cor.jpg}
\caption{$\pi$/K identification capability (at $2 GeV/c$) by FTOF in the condition of enhanced background, without (left) and with (right) multiple-hit correction.}
\label{FIG:FTOF_PID_MAXBKG}
\end{figure}

\paragraph{FTOF Conceptual Design}
Based on the results from both simulation study and experimental test, we come up with the conceptual design of the FTOF sub-system, as shown in Fig. \ref{FIG:FTOF_Conceptual_Design}. The proposed FTOF detector consists of two identical endcap discs, positioned $~1400 mm$ along the beam direction away from the collision point. Each disc is made up with four quadrantal sectors, each with an inner radius of $\sim 560 mm$ and an outer radius of $\sim 1050 mm$, covering $\sim 21 \deg - 38 \deg$ in polar angle and $90 \deg$ in azimuthal angle. The detail structure inside a sector is also depicted in Fig. \ref{FIG:FTOF_Conceptual_Design}. The planar synthetic fused silica radiator is fan shaped that can be viewed as a composite structure of 3 trapezoidal units, each $\sim 295 mm$ (inner side) $/$ $\sim 533 mm$ (outer side) wide, $\sim 470 mm$ high and $15 mm$ thick. An array of $3 \times \sim 14-16$ multi-anode MCP-PMTs are optical coupled to the radiator along the outer side. The whole sector is enclosed in a light-tight black box made of $5 mm$ thick carbon fibre, occupying $\sim 200 mm$ space along the beam (Z) direction.

\begin{figure}[!htb]
	\centering
	\includegraphics[width=0.75\textwidth]{Figures/Figs_04_00_DetectorSubSystems/Figs_04_03_ParticleIdentification/FTOF_Conceptual_Structure1.jpg}
	\includegraphics[width=0.20\textwidth]{Figures/Figs_04_00_DetectorSubSystems/Figs_04_03_ParticleIdentification/FTOF_Conceptual_Structure2.jpg}
	\caption{The conceptual design of FTOF.}
	\label{FIG:FTOF_Conceptual_Design}
\end{figure}

%% The following paragraphs are based on the work by Ziwei Li
\emph{FTOF Performance in Simulation}
Geant4 simulations are performed to study the expected performance of FTOF with such design. A $20 mm$ thick Aluminum plate is added $100 mm$ before the FTOF detector, in order to simulate the material budget of the main drift chamber (MDC) endcap. When tracking the photon propagation, the inner and outer side of the FTOF radiator are set absorptive, while the two lateral sides are set reflective (reflection factor $\sim 92\%$). The surface roughness of the radiator is simulated by randomizing the normal direction of facet by $\sigma = 0.1 \deg$ (corresponding to an average reflection factor of $\sim 97\%$). Pion and kaon particles are emitted from the interaction point at different momenta and directions. Angular angle $23.66 \deg$, $26.65 \deg$, $29.48 \deg$, $32.17 \deg$ and $34.71 \deg$, azimuth angle $5 \deg$, $15 \deg$, $25 \deg$, $35 \deg$ and $45 \deg$, and particle momenta $1.0 ~GeV/c$, $1.2 ~GeV/c$, $1.4 ~GeV/c$, $1.6 ~GeV/c$, $1.8 ~GeV/c$ and $2.0 ~GeV/c$ are scanned. A typical Cherenkov photon hit pattern is displayed as in Fig. \ref{FIG:FTOF_TOP_Resolution} by pions at $p = 1 ~GeV/c$, $\theta = 23.66 \deg$ and $\phi = 15 \deg$. A clear correlation between the time of propagation (TOP) and the hit position (sensor pixel) is demonstrated. Two bands, one for direct photons and another for indirect photons with one reflection off the lateral side, are well separated except for a few sensors close to the side. The mean number of photons detected by the MCP-PMT arrays is $\sim 17$. By applying a TOP-position calibration, where the average track length collected by each sensor pixel and the average velocity of photons are used to calculate the TOP (so no dispersion effect is accounted for), the timing resolution of $\sim 20 ps$ or better is obtained (as shown in Fig. \ref{FIG:FTOF_TOP_Resolution}), with no further correction (such as the dispersion effect).

\begin{figure}[!htb]
	\centering
	\includegraphics[width=0.45\textwidth]{Figures/Figs_04_00_DetectorSubSystems/Figs_04_03_ParticleIdentification/FTOF_TOP_1.jpg}
	\includegraphics[width=0.45\textwidth]{Figures/Figs_04_00_DetectorSubSystems/Figs_04_03_ParticleIdentification/FTOF_TOPReso_1.jpg}
	\caption{The simulated TOP vs. hit position pattern of FTOF, and the timing resolution after calibration.}
	\label{FIG:FTOF_TOP_Resolution}
\end{figure}

Besides the timing resolution mentioned above ($\sigma_{det}$), the total time uncertainty of the FTOF detector includes several other components: the TTS of the MCP-PMT ($\sigma_{TTS} = 70ps$ is used in our calculation), the electronics time jitter ($\sigma_{elec} = 10ps$), the time error due to the incident particles¡¯ track reconstruction ($\sigma_{trk} = 10ps$), the event start time resolution ($\sigma_{T_{0}} = 40ps$). It can be estimated by formula \ref{EQ:DIRC-Time-Reso}. With the simulation results, the expected timing resolution in different FTOF regions are deduced, as shown in Fig. \ref{FIG:FTOF_TReso_NPE} for pions (kaons are very similar) at $2 ~GeV/c$. The black numbers are total time resolution while the red ones are the intrinsic detector timing resolution $\sigma_{det}$. It is clear $\sigma_{det}$ becomes worse at lower $\theta$, which corresponds to hits closer to the inner side of FTOF radiator, and slightly improved at larger $\phi$ (best at $\phi = 45 \deg$), which is further away from the lateral side of FTOF radiator. This $\theta$ and $\phi$ dependence is mainly reflected by the total number of detected Cherenkov photons, as also shown in Fig. \ref{FIG:FTOF_TReso_NPE}. It is also evident the event start time resolution $\sigma_{T_{0}} = 40 ps$ dominates the over all timing error, so optimized design of STCF bunch size (which dictates the timing uncertainty of event start time) is crucial. Furthermore we can find the constraint on $\sigma_{elec}$ can be relaxed, since the single photon response of MCP-PMT $\sigma_{TTS}$ dominates the terms with $\frac{1}{NPE}$ scaling.

\begin{figure}[!htb]
	\centering
	\includegraphics[width=0.45\textwidth]{Figures/Figs_04_00_DetectorSubSystems/Figs_04_03_ParticleIdentification/FTOF_TReso_All_1.jpg}
	\includegraphics[width=0.45\textwidth]{Figures/Figs_04_00_DetectorSubSystems/Figs_04_03_ParticleIdentification/FTOF_NPE_1.jpg}
	\caption{The expected overall FTOF timing resolution (left), and the number of detected direct photons by $\pi^{\pm}$ particle at $2~GeV/c$.}
	\label{FIG:FTOF_TReso_NPE}
\end{figure}

With the expected timing performance obtained, the PID power, i.e. $K/\pi$ separation power is defined by the following formula
\begin{equation}
\label{eq::FTOF-PID-Power}
\Delta T=\frac{L}{\beta c} \frac{\Delta m^2}{2p^2}\sim\frac{L}{c}\frac{\Delta m^2}{2p^2},
PID power = \frac{\Delta T}{\sigma_{tot}},
\end{equation}
where $\Delta T$ is the TOF difference between pion and kaon with the same momentum $p$, $L$ is the flight length, and $\sigma$ is the total time resolution of FTOF. The results are shown in Fig. \ref{FIG:FTOF_PIDPower}, for $2~GeV/c$ $\pi/K$ at different regions of the FTOF. Due to longer $L$ and better timing resolution at larger $\phi$, the FTOF PID capability is better for tracks hitting the radiator closer to its outer side. The worst case is at the corner of inner and lateral edges, where the separation power is still acceptable, $3.1$.

\begin{figure}[!htb]
	\centering
	\includegraphics[width=0.7\textwidth]{Figures/Figs_04_00_DetectorSubSystems/Figs_04_03_ParticleIdentification/FTOF_PID_Power_1.jpg}
	\caption{The expected PID capability under such timing performance for $\pi/K$ separation at $2~GeV/c$.}
	\label{FIG:FTOF_PIDPower}
\end{figure}

%% The following paragraphs are based on the work by BinBin Qi
\emph{FTOF Reconstruction Algorithm}
The FTOF reconstruction is performed in the coordinate as in Fig. \ref{FIG:FTOF_Reco_Coor}, for one FTOF quadrant. According to the Cherenkov angle relation
\begin{equation}
cos(\bar{\theta_{c}}) = \frac{1}{n_{p}\beta} = \frac{\vec{v_{t}} \dot \vec{v_{p}}}{|v_{t}| \dot |v_{p}|},
\end{equation}
where $\vec{v_{t}}=(a,b,c)$ is the incident particle velocity vector when impinging the radiator, $\vec{v_{p}}$ is the velocity vector of emitted Cherenkov photon, $n_{p}$ is the refractive index of the radiator and $\beta$ is the reduced speed of the particle. The directional components of $\vec{v_{p}}$ can be expressed as $(\Delta{X},\Delta{Y},\Delta{Z})$, where the 3D spatial difference between the photon sensor pixel and the incident position of the particle on the radiator surface, as depicted in Fig. \ref{FIG:FTOF_Reco_Coor} (right). Although the 2D (X and Y) difference can be readily obtained, $\Delta{Z}$ must be deduced with certain particle specie hypothesis. If $V = cos(\bar{\theta_{c}})$ is known, the equation about $\Delta{Z}$ can be found as
\begin{equation}
\label{eq::FTOF-DZ-Solver}
(c^{2}-V^{2})\Delta{Z}^{2} + 2c(a\Delta{X}+b\Delta{Y})\Delta{Z} + (a\Delta{X}+b\Delta{Y})^{2} - V^{2}(\Delta{X}^{2}+\Delta{Y}^{2}) = 0.
\end{equation}
By solving this equation we find $\Delta{Z}=\frac{-B\pm\sqrt{B^{2}-4AC}}{2A}$, with $A=c^{2}-V^{2}$, $B=2c(a\Delta{X}+b\Delta{Y})$ and $C=(a\Delta{X}+b\Delta{Y})^{2} - V^{2}(\Delta{X}^{2}+\Delta{Y}^{2})$. In order to get a real solution, $\Delta = B^{2}-4AC \ge 0$ is required. Furthermore, after some physical cuts $V > 0$ (Cherenkov photon forwardly emitted) and $\frac{\Delta{X}^{2}+\Delta{Y}^{2}}{\Delta{X}^{2}+\Delta{Y}^{2}+\Delta{Z}^{2}} \ge \frac{1}{n_{p}^{2}}$ (ensuring internal total reflection) are applied, the minimal solution of equation ~\ref{eq::FTOF-DZ-Solver} ($\Delta{Z} = min{|\Delta{Z}_{1}|,|\Delta{Z}_{2}|}$) is taken as the optimum solution.

\begin{figure}[!htb]
	\centering
	\includegraphics[width=0.45\textwidth]{Figures/Figs_04_00_DetectorSubSystems/Figs_04_03_ParticleIdentification/FTOF_Reco_Coor.jpg}
	\includegraphics[width=0.45\textwidth]{Figures/Figs_04_00_DetectorSubSystems/Figs_04_03_ParticleIdentification/FTOF_Reco_Dir.jpg}
	\caption{The coordinate system used in FTOF reconstruction (left) and the direction of Cherenkov photon (deep blue line).}
	\label{FIG:FTOF_Reco_Coor}
\end{figure}

The timing error of such approach is estimated by adding up possible factors, such as the dispersion effect, the finite photon sensor size and the propagation length of photons inside the radiator. Assuming a pion of $p \sim 1 - 2~GeV/c$ cross the radiator perpendicularly, with $H = 0.5 m$ (defined as in Fig. \ref{FIG:FTOF_Reco_Coor}), the expected timing uncertainty is shown in Fig. \ref{FIG:FTOF_Reco_TLerr} for sensors at different positions. The pitch of photon sensor is $5.5 mm$. No multiple Coulomb scattering (MCS) effect is accounted for. It's obvious the detector originated timing uncertainty is no more than $40 ps$ with this FTOF structure. Furthermore we find the reconstructed length of photon propagation inside the radiator agree with the MC truth to a precision of $\sim 3.3 mm$, as also shown in Fig. \ref{FIG:FTOF_Reco_TLerr}. The reconstruction algorithm works well for most photon sensors, no matter the incident position of particles, except for a few sensors near the lateral side.

\begin{figure}[!htb]
	\centering
	\includegraphics[width=0.45\textwidth]{Figures/Figs_04_00_DetectorSubSystems/Figs_04_03_ParticleIdentification/FTOF_Reco_Terr.jpg}
	\includegraphics[width=0.45\textwidth]{Figures/Figs_04_00_DetectorSubSystems/Figs_04_03_ParticleIdentification/FTOF_Reco_Lerr.jpg}
	\caption{The expected timing error and propagation length uncertainty of Cherenkov photons in the FTOF quadrant.}
	\label{FIG:FTOF_Reco_TLerr}
\end{figure}

By applying formula
\begin{equation}
TOF = T - TOP - T_{0} = T - \frac{LOP}{v_{g}} - T_{0} = = T - \frac{LOP\times\bar{n_{g}}}{c} - T_{0},
\end{equation}
the TOF information is obtained. Shown in Fig. \ref{FIG:FTOF_Reco_TOFreso} are the time resolution of the FTOF detector for single photo-electron (SPE) and average of all photons, without taking the timing jitter of MCP-PMT and electronics into account. For a SPE, the intrinsic time resolution of the FTOF is $\sim 41 ps$, while by averaging timing information over $\sim 18$ detected photons the timing jitter shrinks to $\sim 10 ps$. It's also noted that in the TOF distribution plot, a low (but visible) long tail shows on both sides of the main peak. The tail is mainly caused by secondary particles along the primary pion, mainly delta-electrons.

\begin{figure}[!htb]
	\centering
	\includegraphics[width=0.45\textwidth]{Figures/Figs_04_00_DetectorSubSystems/Figs_04_03_ParticleIdentification/FTOF_Reco_TOFspe.jpg}
	\includegraphics[width=0.45\textwidth]{Figures/Figs_04_00_DetectorSubSystems/Figs_04_03_ParticleIdentification/FTOF_Reco_TOFall.jpg}
	\caption{The TOF resolution of the FTOF detector for single photo-electron and average of all photons.}
	\label{FIG:FTOF_Reco_TOFreso}
\end{figure}

The TOF information is deduced and compared to the expectation of each hypothesis particle. Fig. \ref{FIG:FTOF_Reco_TOFPID} shows the reconstructed intrinsic TOF distributions of both pions and kaons at $2 GeV/c$. One can easily find if the particle hypothesis is correct, the reconstructed TOF peak is at its right position, with a resolution of $\sim 10ps$. However if the hypothesis is not true, the reconstructed TOF peak is shifted with respect to its expectation. The shift makes the separation between pion and kaon TOF peaks even larger, which may benefit the PID power. When convoluting all contributing factors, as in formula \ref{EQ:DIRC-Time-Reso}, the overall reconstructed TOF time resolution is $\sim 45-50 ps$, which is shown in Fig. \ref{FIG:FTOF_Reco_TOFPID}. By directly comparing the TOF information, a $3.0 \sigma$ separation power for $\pi/K$ at $2 GeV/c$ is achieved, consistent with the GEANT4 simulation, and fulfils the required PID capability of FTOF. Furthermore the separation power gets stronger if we compare the reconstructed TOFs of various hypotheses for the same set of particles, mainly due to the beneficial time shift under wrong hypothesis (as in Fig. \ref{FIG:FTOF_Reco_TOFPID}). For either pion or kaon samples at $2 GeV/c$, the separation power reaches $\sim 4\sigma$ by comparing the TOF distributions under $\pi$ or $K$ hypotheses.

\begin{figure}[!htb]
	\centering
	\includegraphics[width=0.45\textwidth]{Figures/Figs_04_00_DetectorSubSystems/Figs_04_03_ParticleIdentification/FTOF_Reco_TOFPID_1.jpg}
	\includegraphics[width=0.45\textwidth]{Figures/Figs_04_00_DetectorSubSystems/Figs_04_03_ParticleIdentification/FTOF_Reco_TOFPID_2.jpg}
	\caption{The TOF PID capability of the FTOF detector for $\pi/K$ separation at $2 GeV/c$, without (left) and with (right) contributions from other timing uncertainties.}
	\label{FIG:FTOF_Reco_TOFPID}
\end{figure}

To further evaluate the PID capability of FTOF, we apply the likelihood method. The likelihood function is constructed by
\begin{equation}
\label{eq::FTOF-Likelihood-Function}
\mathcal{L}_{h} = \Pi^{i}_{i=1}f_{h}(TOF^{h}_{i}),
\mathcal{l} = \mathcal{L}_{\pi} - \mathcal{L}_{K},
\end{equation}
where $h$ denotes hadron species (in our case $\pi$ and $K$) and $i$ accounts for each detected photon. The probability density function $f_{h}$ is taken as a Gaussian fit to the expected TOF distribution (as in Fig. \ref{FIG:FTOF_Reco_TOFPID} (right)), plus a constant background of 0.05. Shown in Fig. \ref{FIG:FTOF_Reco_LHPID} are the reconstructed $\mathcal{l}$ for $2 GeV/c$ $\pi$ and $K$ emitted at different angles. Despite of the very different particle directions, the separation power of FTOF are similar, reaches $\sim 4\sigma$ or better all over the FTOF sensitive area. These results also agree with the results in Fig. \ref{FIG:FTOF_Reco_TOFPID}.

\begin{figure}[!htb]
	\centering
	\includegraphics[width=0.45\textwidth]{Figures/Figs_04_00_DetectorSubSystems/Figs_04_03_ParticleIdentification/FTOF_Reco_LHPID_1.jpg}
	\includegraphics[width=0.45\textwidth]{Figures/Figs_04_00_DetectorSubSystems/Figs_04_03_ParticleIdentification/FTOF_Reco_LHPID_2.jpg}
	\caption{The likelihood PID capability of the FTOF detector for $\pi/K$ separation at $2 GeV/c$, emitted at different angles.}
	\label{FIG:FTOF_Reco_LHPID}
\end{figure}

\emph{FTOF Background}
The distributions of background particles obtained by specific MDI and background study are used as input to the FTOF geant4 simulation, and then to simulate the effect of background on FTOF performance. The background particles on FTOF are mainly secondary electrons, for which the energy and angular distribution are shown in Figure \ref{FIG:FTOF_BKG_SR}. The background hit rate on a single FTOF disc is about $7\time10^{7} Hz$, including two main parts: the beam-induced background ($25\%$) and the physical background ($75\%$). The beam-induced background is uniformly distributed in time, which is not related to the beam bunch crossing, while the physical background is related to the collisions during bunch crossing (once per 8 ns), exhibiting characteristic time structure. The time distribution of the two background components are shown in Figure 1.2. The beam background is uniformly distributed in time, representing by a uniform sampling in the simulation. The physical background is correlated with the collision frequency, so the time $T=T_{01}+T_{02}+T_{flight}$, where $T_{01}$ (=0, 8, 16 ns …) is the beam bunch crossing time, $T_{02}$ is the beam collision time uncertainty due to the length of beam bunch, uniformly distributed in the [-50 ps, 50 ps] interval, $T_{flight}$ is the flight time of background particle from the beam collision point to the FTOF (illustrated in Figure \ref{FIG:FTOF_BKG_T} right). By Monte Carlo sampling, the physical background time distribution can be obtained. Combining with the uniform time distribution of the beam-induced background, the overall time distribution of the background hit on FTOF can be obtained, as shown in Figure \ref{FIG:FTOF_ALLBKG_T}.

\begin{figure}[!htb]
	\centering
	\includegraphics[width=0.45\textwidth]{Figures/Figs_04_00_DetectorSubSystems/Figs_04_03_ParticleIdentification/FTOF_BKG_Ek.jpg}
	\includegraphics[width=0.45\textwidth]{Figures/Figs_04_00_DetectorSubSystems/Figs_04_03_ParticleIdentification/FTOF_BKG_Ang.jpg}
	\caption{Energy distribution (left) and polar angle distribution (right) of background particles hitting FTOF.}
	\label{FIG:FTOF_BKG_SR}
\end{figure}

\begin{figure}[!htb]
	\centering
	\includegraphics[width=0.45\textwidth]{Figures/Figs_04_00_DetectorSubSystems/Figs_04_03_ParticleIdentification/FTOF_BeamBkg_T.jpg}
	\includegraphics[width=0.45\textwidth]{Figures/Figs_04_00_DetectorSubSystems/Figs_04_03_ParticleIdentification/FTOF_PhyBkg_T.jpg}
	\caption{Time-of-flight distribution of background particles (starting from the collision time) on FTOF: beam-induced background (left) and physical background (right).}
	\label{FIG:FTOF_BKG_T}
\end{figure}

\begin{figure}[!htb]
	\centering
	\includegraphics[width=0.65\textwidth]{Figures/Figs_04_00_DetectorSubSystems/Figs_04_03_ParticleIdentification/FTOF_AllBkg_T.jpg}
	\caption{Overall time distribution of background particles on FTOF.}
	\label{FIG:FTOF_ALLBKG_T}
\end{figure}

In the Geant4 simulation, the hit position of the background particles is uniformly distributed over the FTOF detector surface, and the energy and angle are sampled by the distribution in Figure \ref{FIG:FTOF_BKG_SR}. The time window of the signal acquisition in the simulation is 100 ns, within the interval [-40 ns, 60 ns] so that the real signal is in the middle of the time window. In this time window, the number of background particles is given by Poisson distribution, and the time distribution is sampled according to Figure \ref{FIG:FTOF_ALLBKG_T}.

10000 $\pi$ and K particles are generated in the Geant4 simulation along with the background samples to study the effect of the background. The background may greatly increase the number of photoelectrons detected by FTOF in a single event, resulting increased possibility of multiple hits in a single channel. The correction of multiple hits in data processing is applied, which means that in the time window of [-40 ns, 60 ns], when a single channel has multiple hits, only the first arrival photoelectron signal is taken and all other hits are dismissed. The background level in the simulation is set to $7\times10^{7} Hz$, according to the background study. As shown in Figure \ref{FIG:FTOF_NPE_BKG}, the average number of photoelectrons for a pion at $p=2 GeV/c$ is 22, increased to about 124 after taking the background hits into account, and reduced to about 88 after the correction of multiple hits in single channel. It’s noted that if the average number of photoelectrons is 124 for a time window of 100 ns, and assuming a MCP-PMT gain of $10^{6}$ , the average accumulated charge density on the MCP-PMT anode is $120 C/cm^{2}$, over 10-year STCF operation ($50\%$ run time).

\begin{figure}[!htb]
	\centering
	\includegraphics[width=0.65\textwidth]{Figures/Figs_04_00_DetectorSubSystems/Figs_04_03_ParticleIdentification/FTOF_NPE_BKG.jpg}
	\caption{Distribution of number of photoelectrons generated by a $2 GeV/c$ pion, without and with background.}
	\label{FIG:FTOF_NPE_BKG}
\end{figure}

Shown in Figure \ref{FIG:FTOF_XTHIT_BKG} is the 2-D time-position map of FTOF hit. It can be seen that hits by the background particles are uniformly distributed throughout the phase space, while the real signal hits are concentrated as a band. After time reconstruction, the TOF distribution of the single photoelectron signal can be obtained, as shown in Figure \ref{FIG:FTOF_SPETOF_BKG}. The reconstruction TOF of the real signal in the figure is a Gaussian distribution (mean ~5.2 ns, sigma ~100 ps), while the TOF of background particles are uniformly distributed. Some single-photon electrons with zero TOF in the figure do not meet the reconstruction conditions and are taken as background during the reconstruction process. Due to the uniform distribution of the reconstructed background signal, the influence of the background can be eliminated by using the maximum likelihood method. The $pi$/K resolution is found to be 4.2σ, as shown in Figure \ref{FIG:FTOF_PID_BKG}. Moreover, even with the correction of multiple hits, the pi/K resolution remains at 4.1σ, i.e. the background effect on $pi$/K identification is fairly small.

\begin{figure}[!htb]
	\centering
	\includegraphics[width=0.45\textwidth]{Figures/Figs_04_00_DetectorSubSystems/Figs_04_03_ParticleIdentification/FTOF_XTHit_BKG.jpg}
	\includegraphics[width=0.45\textwidth]{Figures/Figs_04_00_DetectorSubSystems/Figs_04_03_ParticleIdentification/FTOF_XTHit_BKG_Cor.jpg}
	\caption{2-D time-position map of FTOF hit, without (left) and with multiple-hit correction.}
	\label{FIG:FTOF_XTHIT_BKG}
\end{figure}

\begin{figure}[!htb]
	\centering
	\includegraphics[width=0.45\textwidth]{Figures/Figs_04_00_DetectorSubSystems/Figs_04_03_ParticleIdentification/FTOF_SPETOF_BKG.jpg}
	\includegraphics[width=0.45\textwidth]{Figures/Figs_04_00_DetectorSubSystems/Figs_04_03_ParticleIdentification/FTOF_SPETOF_BKG_Cor.jpg}
	\caption{Reconstructed TOF distribution of single photoelectron signal, without (left) and with multiple-hit correction.}
	\label{FIG:FTOF_SPETOF_BKG}
\end{figure}

\begin{figure}[!htb]
	\centering
	\includegraphics[width=0.45\textwidth]{Figures/Figs_04_00_DetectorSubSystems/Figs_04_03_ParticleIdentification/FTOF_PID_BKG.jpg}
	\includegraphics[width=0.45\textwidth]{Figures/Figs_04_00_DetectorSubSystems/Figs_04_03_ParticleIdentification/FTOF_PID_BKG_Cor.jpg}
	\caption{$\pi$/K identification capability (at $2 GeV/c$) by FTOF, without (left) and with multiple-hit correction.}
	\label{FIG:FTOF_PID_BKG}
\end{figure}

Considering the Poisson fluctuation of background count, the influence on the FTOF PID ability of an extreme condition with 3 standard deviation above the average background level is also studied by the simulation. The background hit rate is increased to $2.1\times10^{8} Hz$ per single FTOF disc. As shown in Figure \ref{FIG:FTOF_NPE_MAXBKG}, under such a high counting rate, the average number of photoelectrons in the 100 ns window is 300, and after the multiple-hit correction is about 200. Figure \ref{FIG:FTOF_XTHIT_MAXBKG} shows the 2D time-position distribution of photoelectrons within 100 ns window. Figure \ref{FIG:FTOF_SPETOF_MAXBKG} shows the TOF distribution of single photoelectron. It can be seen that the reconstruction results are similar to those in Figure \ref{FIG:FTOF_XTHIT_BKG} and \ref{FIG:FTOF_SPETOF_BKG}. The TOF distribution is slightly broadened than the previous result, with a Gaussian sigma ~ 104 ps. Although the number of photoelectrons from the background increases dramatically, the final effect on $\pi$/K resolution is minor, as shown in Figure \ref{FIG:FTOF_PID_MAXBKG}. The resolution $\pi$/K of 3.9 σ and 3.8 σ before and after multiple-hit correction still meet the requirement of 3σ $\pi$/K separation for the FTOF.

\begin{figure}[!htb]
	\centering
	\includegraphics[width=0.65\textwidth]{Figures/Figs_04_00_DetectorSubSystems/Figs_04_03_ParticleIdentification/FTOF_NPE_MAXBKG.jpg}
	\caption{Distribution of number of photoelectrons generated by a $2 GeV/c$ pion, without and with enhanced background.}
	\label{FIG:FTOF_NPE_MAXBKG}
\end{figure}

\begin{figure}[!htb]
	\centering
	\includegraphics[width=0.45\textwidth]{Figures/Figs_04_00_DetectorSubSystems/Figs_04_03_ParticleIdentification/FTOF_XTHit_MAXBKG.jpg}
	\includegraphics[width=0.45\textwidth]{Figures/Figs_04_00_DetectorSubSystems/Figs_04_03_ParticleIdentification/FTOF_XTHit_MAXBKG_Cor.jpg}
	\caption{2-D time-position map of FTOF hit in the condition of enhanced background, without (left) and with (right) multiple-hit correction.}
	\label{FIG:FTOF_XTHIT_MAXBKG}
\end{figure}

\begin{figure}[!htb]
	\centering
	\includegraphics[width=0.45\textwidth]{Figures/Figs_04_00_DetectorSubSystems/Figs_04_03_ParticleIdentification/FTOF_SPETOF_MAXBKG.jpg}
	\includegraphics[width=0.45\textwidth]{Figures/Figs_04_00_DetectorSubSystems/Figs_04_03_ParticleIdentification/FTOF_SPETOF_MAXBKG_Cor.jpg}
	\caption{Reconstructed TOF distribution of single photoelectron signal in the condition of enhanced background, without (left) and with (right) multiple-hit correction.}
	\label{FIG:FTOF_SPETOF_MAXBKG}
\end{figure}

\begin{figure}[!htb]
	\centering
	\includegraphics[width=0.45\textwidth]{Figures/Figs_04_00_DetectorSubSystems/Figs_04_03_ParticleIdentification/FTOF_PID_MAXBKG.jpg}
	\includegraphics[width=0.45\textwidth]{Figures/Figs_04_00_DetectorSubSystems/Figs_04_03_ParticleIdentification/FTOF_PID_MAXBKG_Cor.jpg}
	\caption{$\pi$/K identification capability (at $2 GeV/c$) by FTOF in the condition of enhanced background, without (left) and with (right) multiple-hit correction.}
	\label{FIG:FTOF_PID_MAXBKG}
\end{figure}

\input{Chapters/Chapter_04_00_DetectorSubSystems/Chapter_04_03_ParticleIdentification/04_Ref_ConceptualDesign}

