\subsubsection{CandidateTechnology}
\label{section3}

In the detector system of STCF, the charged hadrons at low momentum can be identified by the characteristic ionization energy loss (dE/dx) in the tracking sub-system. With a dE/dx energy resolution of $6-7\%$, the $\pi$/K and K/p are separable when the momenta are lower than $0.8~GeV/c$ and $1.1  GeV/c$ respectively. In the higher momentum range (for example, up to $2~GeV/c$ at STCF), a specific PID sub-system is needed to accomplish the necessary particle identification capability. There are two common technical choices for the PID detector, namely the time-of-flight (TOF) detector and the Cherenkov detector.

{\bf (1) TOF Detector}
The TOF detector determines the particle mass by measuring the particle velocity under given momentum, so as to realize the identification of particle species. The most important factor of TOF is the accuracy of time measurement. In order to identify particles in the high momentum range, the TOF detector requires excellent time resolution. The basic formula of particle identification using TOF is

\begin{equation}
\label{eq::TOF-PID1}
\frac{\Delta T}{T}=\frac{\Delta\beta}{\beta}=\frac{(\Delta m^2)}{(2p^2)}
\end{equation}

where $T$, $b$, $m$ and $p$ are the flight time, reduced velocity (respect to the speed of light in vacuum), mass and momentum of the particle; $\Delta T$, $\Delta b$ and $\Delta m^{2}$ are the difference between flight time, velocity and mass square of different kinds of particles. Then one finds

\begin{equation}
\label{eq::TOF-PID2}
\Delta T=\frac{L}{\beta c} \frac{\Delta m^2}{2p^2}\sim\frac{L}{c}\frac{\Delta m^2}{2p^2}
\end{equation}

where $L$ is the flight path (trajectory) length of particle and $c$ is the speed of light in vacuum.

In the proposed experimental apparatus at STCF, the minimum flight distance $L$ is less than $\sim 1 m$. Considering the $\pi/ K $ and $K/ p $ separation at momentum $2 GeV/c $ and the flight distance $L$ of $1 m$ in the barrel region, the time differences of flight $\Delta T $ are about $0.1 ns$ and $0.27 ns$ respectively, so a total TOF time resolution of $\sim 30 ps$ is needed in order to effectively separate the $pi/K$. It is thus very challenging to apply TOF technique to barrel PID detector. However for the end-cap region, where the flight distance is $\ge 1.3 m$, a TOF resolution of $\sim 45 ps$ is adequate to identify  $pi/K$ up to $\sim 2GeV/c$.

{\bf(2) Cherenkov Detector}

Cherenkov radiation is widely used for PID in a wide momentum range in modern high-energy physics experiments. When the flight velocity of charged particle in transparent medium is greater than the speed of light, a specific electro-magnetic radiation, namely the Cherenkov radiation, will be emitted at a characteristic angle which depends on the refractive index of the medium and the particle velocity. Therefore, the particle momentum and Cherenkov light emission angle can be precisely measured to determine the type of the particle. According to the PID requirement of various particle species and momentum ranges, different kinds of medium (commonly called Cherenkov radiator) with different refractive indices can be chosen, so as to distinguish particles with momenta ranging from $\gevc$ to several hundred $\gevc$.

In order to effectively identify particles by the Cherenkov radiation, it is crucial to accurately determine the Cherenkov radiation angle, or alternatively the fine pattern of the Cherenkov light hits on the detection plane. Then the measurements are compared with the expected emission Cherenkov angle (or hit pattern) of various particle species to find the most probable one. The 3-vector momentum of the incident charged particle hitting the Cherenkov radiator must be known, usually measured by the tracking system, and the position and time resolution of the Cherenkov detector should also be good enough. The structure of such type of detector is usually more complex than the TOF detector, while the corresponding kinematic coverage is also broader and the PID ability is stronger. The way to realize the accurate measurement of the radiation angle or hit pattern of the Cherenkov light is copious. They can be roughly categorized to two main types, namely the Ringing Imaging Cherenkov detector (RICH), and the Detection of Internal total-Reflected Cherenkov light (DIRC). 


The principle of distinguishing different kinds of charged particles by the measurement of Cherenkov radiation angle and the particle momentum is evaluated by the following equations.

\begin{equation}
\sin^{2}{\theta_{c}}=\sin^{2}{\theta_{0}}-\cos^{2}{\theta_{0}}\frac{m^{2}}{p^{2}}
\end{equation}

where $\theta_0 = \cos^{2}{\theta_{0}}=\frac{1}{n}$ represents the Cherenkov radiation angle at the particle velocity limit ($v=c$); the $\sin^{2}{\theta_{c}}$ difference between two particles is

\begin{equation}
\Delta \sin^{2}\theta_{c}=\cos^2\theta_{0}\frac{(\Delta m^{2})}{p^2} =\frac{1}{n^2}\frac{\Delta m^2}{p^2}.
\end{equation}

The grey band in figure ~\ ref {Fig4:PID-CherenkPID2} simulates the experimental measurements, and the uncertainty is estimated to be

\begin{equation}
\sigma_{m}^{2}(p)=\delta^2(\sin^{2}\theta_{c})+4\cos^2\theta_{0}\frac{m^4}{p^4}(\frac{\delta p}{p})^2.
\end{equation}

The PID capability of the Cherenkov detector can be deduced through two parameters $\Delta \sin^{2}\theta_{c}$ and $\sigma_{m}^{2}(p)$. 

{\bf Ringing Imaging Cherenkov detector}

RICH detector is a suitable candidate for PID at STCF. In order to reduce the space occupied by the detector and avoid the relatively complex optical structure design, RICH detector with  proximity focusing can be adopted. the proximity gap can be selected according to the actual experimental requirements, e.g. the experimental apparatus at STCF. Due to the limited space allowed (e.g. $\sim 20 cm$ in radius direction in the barrel region), the proximity gap should not be larger than $\sim10 cm$.

In order to realize hadron PID in the full momentum range, the low momentum threshold of the Cherenkov detector needs to be less than $1~GeV/c$ to connect to the PID power of tracking sub-system (by dE/dx). This requires that the refractive index of the radiator is greater than $\sim1.12$. Neither the gas radiator nor the silica aerogel meet this criterion (in literature the refractive index of silica aerogel can be as high as 1.15, but in practice only those with $n\le1.1$ has been applied in HEP experiments), so it is necessary to consider the liquid or solid Cherenkov radiator option.

Fluorinated hexane (C6F14) is the only liquid Cherenkov radiator successfully applied in high-energy physics (HEP) experiments, e.g. DELPHI, SLD, NA35, STAR and ALICE. It has a refractive index of 1.3 at the wavelength of $175 nm$. As a liquid substance, it needs to be sealed in an ultraviolet-permeable container (e.g., fuse silica or quartz). As an example, figure \ref{Fig4:PID-HMPID31} shows the principle structure of the high-momentum particle identification detector (HMPID) at ALICE experiment. The Cherenkov photons generated in the liquid C6F14 radiator are detected, after a proximity gap of $8 cm$, by the sensor plane, which is composed of the multi-wire proportional chamber (MWPC) with its cathode surface plated with Cesium Iodide (CsI). The readout units are arranged as an array of $8\times8.4 mm^{2}$ pads. Such a position resolution capability ensures $3\sigma$ $\pi/K/p$ PID capability up to $p=3/5\gevc$.

\begin{figure*}[htbp]
 \centering
 \mbox{
  \begin{overpic}[width=0.4\textwidth, height=0.3\textwidth]{Figures/Figs_04_00_DetectorSubSystems/Figs_04_03_ParticleIdentification/PID-HMPID31.jpg}
  \end{overpic}
  \begin{overpic}[width=0.4\textwidth, height=0.3\textwidth]{Figures/Figs_04_00_DetectorSubSystems/Figs_04_03_ParticleIdentification/PID-HMPIDres32.jpg}
  \end{overpic}
  }
\caption{Schematic structure of the ALICE-HMPID detector and its experimental results of hadron PID performance.}
\label{Fig4:PID-HMPID31}
\end{figure*}

Fused silica is one kind of common solid Cherenkov radiator. Its refractive index is about 1.47 at wavelength $\sim390 nm$. Solid materials are easier to process than liquid radiators. However, at high momentum the Cherenkov light produced in fused silica will often undergo total reflection in the radiator, and the detection efficiency will be greatly reduced, which prevents the use of fused silica as Cherenkov radiator in a RICH detector. Lithium fluoride (LiF) is another kind of Cherenkov radiator material commonly used in HEP experiments. The refractive index of lithium fluoride 1.46 at photon energy of 7eV. At the same time, LiF has better UV light transparency than fused silica (or quartz), which is beneficial to increase the yield of Cherenkov light. Nevertheless, LiF has the problem of total reflection of Cherenkov radiation inside the radiator similar to fused silica. The CLEO-III experiment has adopted an unique surface treatment method to overcome this problem, that is, the light exit surface of the LiF radiator is machined into sawtooth shape to avoid the total reflection of Cherenkov light on the exit surface. Obviously, the complex processing of LiF surface increases the technical difficulty and cost to produce such a detector. According to the results of CLEO-III experiment, the treated RICH detector can realize $4\sigma$ separation of $\pi/K$ in the momentum range of $0.47-2.65 ~\gevc$.

In the endcap region, the track length is longer than that in the barrel region, so the dE/dx measurement can provide better resolution, and the separation of $\pi/K/p$ can be achieved at a higher momentum range. Thus the low momentum threshold of the Cherenkov detector can be higher in the endcap region, which means the refractive index of Cherenkov radiator can be lower (e.g. $\le1.1$). In such a case silica aerogel can also be used as a candidate material, leading to more choices and technical possibilities compared to the barrel region.

The BELLE-II experiment adopts the RICH technology based on the silica aerogel radiator in the endcap PID detector. As shown in figure ~\ref{Fig4:PID-ARICH61}, the radiator is composed of double silica aerogel layers, in which the second layer has a higher index of refraction than the first layer ($ n_{2}>n_{1}$). Such design is advantageous in improving the position resolution of the Cherenkov ring imaging on the sensor plane, thus better measuring the Cherenkov angle. In BELLE-II endcap silica aerogel with $n\sim1.06$ is chosen and the proximity gap is $20 cm$. The beam test of BELLE-II ARICH has shown a $\pi/K$ separation up to $4 \gevc$ at a $5.5\sigma$ significance. However its low momentum threshold is limited to $1.4 \gevc$. The radiation resistance and aging effect of this detector is still being studied.

\begin{figure*}[htbp]
 \centering
 \mbox{
  \begin{overpic}[width=0.4\textwidth, height=0.3\textwidth]{Figures/Figs_04_00_DetectorSubSystems/Figs_04_03_ParticleIdentification/PID-ARICH61.jpg}
  \end{overpic}
  \begin{overpic}[width=0.4\textwidth, height=0.3\textwidth]{Figures/Figs_04_00_DetectorSubSystems/Figs_04_03_ParticleIdentification/PID-ARICHEqu62.jpg}
  \end{overpic}
  }
  \caption{The dependence of Cherenkov radiation threshold momentum on the refractive index of radiator.}
  \label{Fig4:PID-ARICH61}
\end{figure*}

{\bf DIRC or DIRC-like detector}

The concept of DIRC was first introduced by the Babar experiment. Cherenkov light generated in the long fused silica bars are propagated to the end through total internal reflections, then projected to the sensor array via an expansion water volume. Fused silica is taken as both Cherenkov radiator and light guide. The angle of Cherenkov photons are kept through hundreds of reflections and the spatial pattern of Cherenkov ring can be recognized for PID purpose. The single photon angular resolution is determined mainly by the size of radiator bar, the sensor size and the length of the expansion volume. For the Babar DIRC, the end size of fused silica bar is $4 cm \times 5 cm$ (rectangle), the sensitive area of each PMT is $\pi \times 3^{2} cm^{2}$ (circle) and the distance between the radiator bar end and the PMT is $\sim 50 cm$. According to these parameters a single photon angular resolution of $\sim 10 mrad$ is obtained and the $\theta_{c}$ resolution for a charged particle is $\sim 3 mrad$ with an average Cherenkov light yield of 10. The $\pi/K$ separation at Babar can be achieved up to $p=4~\gevc$. It's worth noting that at Babar DIRC the time resolution of single photon is around $1 ns$, which is mainly applied to suppress uncorrelated background by setting a proper time window.

Time of Propagation (TOP) is a new technical development in the field of Cherenkov detector. Its main technical principle is shown in figure ~\ref{Fig4:PID-TOPTechPri52}. Cherenkov light is transmitted in the cuboid-shaped radiator plate through total reflection, just as DIRC, and then the two-dimensional position and time of photons hit are measured by the sensor at the end, with particular high precision timing resolution. In this development of DIRC technology, two-dimensional position plus one-dimensional time measurement is used to replace the three-dimensional position measurement in Babar-DIRC, and the excellent position and time measurement ability of the most advanced optical sensor is used to realize very compact Cherenkov radiation detection.

As the main PID detector at the barrel region of the BELLE-II experiment, TOP technology has developed significantly in recent years, and its technical principle has been verified in the beam test, as well as in the real physics run. The micro-channel plate photomultiplier (MCP-PMT) array is chosen as the Cherenkov photon sensor mainly due to its superior timing performance and ability of multi-anode readout, as shown in ~\ref{Fig4:PID-TOPTechPri52}. The active area of the MCP-PMT is $\sim5.3\times5.3 mm^{2}$ and its time resolution is $50 ps$. The two-dimensional position measurement, which is converted into 1-dimensional position coordinate, and the accurate time measurement as the vertical coordinate are showing unique texture pattern in the figure, which clearly reflects the Cherenkov light imaging feature. The optimal judgment of the particle type can be obtained by comparing the measurement result with the expected imaging structure for different particles. It has been shown that the simulation is well consistent with the experimental results, which provides a solid foundation for the maximum likelihood analysis of PID. The TOP technology has demonstrated good PID performance in the large scale high-energy experiment (BELLE II).

\begin{figure*}[htbp]
 \centering
 \mbox{
  %\vskip -1.5cm
  \begin{overpic}[width=0.4\textwidth, height=0.3\textwidth]{Figures/Figs_04_00_DetectorSubSystems/Figs_04_03_ParticleIdentification/PID-TOPTechPri52.jpg}
  \end{overpic}
  \begin{overpic}[width=0.4\textwidth, height=0.3\textwidth]{Figures/Figs_04_00_DetectorSubSystems/Figs_04_03_ParticleIdentification/PID-TOPExpPri51.jpg}
  \end{overpic}
  }
  \caption{Schematic diagram of TOP technology and the beam test results of TOP prototype.}
  \label{Fig4:PID-TOPTechPri52}
\end{figure*}

With improved timing resolution, better PID capability is expected for the new generation DIRC detector, such as those proposed in the future PANDA experiment at FAIR or EID at EIC. The Cherenkov radiator is usually high purity fuse silica with good transparency in the violet wavelength range. The 3-D measurements of (x, y, t) are achieved by multi-anode PMT with high precision timing performance, namely MCP-PMT. The volume of such detector is rather compact, with typical thickness $\le 5 cm$ (not including the optical focusing part usually implemented). The radiation resistance and mechanical robustness is good, as well as the rate capacity, all of which make it suitable detectors in high luminosity experiments. The typical requirements to achieve acceptable performance in the future experiments include: (1) time resolution $\le 100 ps$ for a single photon, (2) very smooth radiator surface with roughness $\le 0.5 nm$, (3) relatively complicated PID algorithm involving intensive calculation and simulation, and (4) dedicated optical system to improve the measurement resolution (spatial and/or time). 


In addition to the influence on the BELLE-II TOP and the PANDA-DIRC, the extension of time dimension measurement also leads to the direct application of DIRC principle to the TOF technology, such as the FTOF at superB and the TORCH proposal at LHCb. 

The FTOF principle is just like the DIRC-type Cherenkov counter. If the length of propagation (LOP) of Cherenkov photon inside the radiator can be deduced, then by precisely measuring the hit time T of the photon on the detector plane the incident time T0 of the charged particle is found straightforward by T0 = T – LOP/v, where v is the speed of light in the radiator. Figure \ref{FIG:SUPERB_FTOF} demonstrates the schematic structure of FTOF detector proposed by the superB. By averaging all photons hitting the detectors, the time resolution of T0 can be reduced by the factor $\frac{1}{\sqrt{N_{PE}}}$, where $N_{PE}$ is the number of practically detected Cherenkov photons. Consider a single photo-electron timing resolution of ~100ps, the for $N_{PE} = 10 $ the timing error of T0 will be $\sim 30 ps$. 

The speed of light in the radiator is wavelength dependent, i.e. the dispersion effect, which will worsen the single photon time resolution. By proper light focusing and pixelisation of the photon sensors this effect can be significantly reduced, as e.g. in the TORCH design of LHCb upgrade. TORCH foresees a single photon timing error of $\le 70 ps$ and a light yield of $N_{PE} \sim 25 $, so as to yield a very precision time resolution of $\le 15 ps$.

\begin{figure*}[htbp]
 \centering
\mbox{
%\vskip -1.5cm
  \begin{overpic}[width=0.6\textwidth, height=0.45\textwidth]{Figures/Figs_04_00_DetectorSubSystems/Figs_04_03_ParticleIdentification/superB_FTOF.jpg}
  \end{overpic}
 }
\caption{The superB FTOF design, taken from FTOF Ph.D thesis. Left: Front view of one DIRC-like TOF sector; the 14 MCP-PMTs are connected to the outer side of the quartz tile, the Cherenkov photons are shown in green. Right: Side view of the same sector; the charged track crossing the quartz is indicated by the red arrow; an example of Cherenkov photon path is shown as well in green.}
\label{FIG:SUPERB_FTOF}
\end{figure*}

\input{Chapters/Chapter_04_00_DetectorSubSystems/Chapter_04_03_ParticleIdentification/04_Ref_CandidateTechnology}
