
\subsection{MAPS-based Inner Tracker}
The MAPS-based ITK is composed of three layers of silicon pixel detectors~(PXDs) and is located inside the MDC (see Sec.~\ref{sec:mdc}) at radii of 36~mm, 98~mm and 160~mm.
The radii of the two inner layers are smaller than those of the design of the $\mu$RWELL-based ITK to take into account the higher rate-capabilities of PXDs and to achieve better ITK spatial resolution.
A pixel size of $100~\mu\mathrm{m} \times 250~\mu\mathrm{m}$ is sufficient to meet the spatial resolution requirement of the STCF ITK.
To reduce the multiscattering effect for charged particles with low momentum, especially for $p < 200$~MeV/c, it is crucial to reduce the material budget as much as possible.
For the baseline design of the STCF PXDs, a radiation length of 0.25\% $X_0$ per layer is assumed, including the material budget from the sensor, readout electronics and supporting material.

The monolithic active pixel sensor (MAPS) technology has the potential to satisfy the low-material and high-rate requirements for the STCF inner tracker. 
This technology has an attractive advantage of having both the sensor and readout electronics in the same pixel, thus reducing the material budget,
and it has been developing rapidly in the particle physics community.
The first-generation MAPS-based vertex detector for the STAR upgrade successfully completed a 3-year physics run~\cite{starmaps1,starmaps2},
The new generation complementary metal-oxide-semiconductor (CMOS) pixel sensor (CPS) for the ALICE-ITS upgrade~\cite{aliceits,alpide} is in mass production.
The CMOS MAPS sensor is chosen as the pixel sensor technology for the silicon-based ITK. The ITK is called MAPS-based ITK in this case. 

The high luminosity of the STCF places additional stringent requirements on the design of the ITK detector, and the challenges include the high hit rate and pileup effects.
From Table~\ref{tab:TIDNIEL_max}, the highest expected hit rate is approximately $1.04\times 10^6$~Hz/cm$^{2}$ at the innermost layer of the MAPS-based~($\mu$RWELL-based) ITK.
The state of the art MAPS technology can easily handle such a hit rate . For example, The STAR ULTIMATE MAPS can cope with a hit rate of
approximately 1 MHz/cm$^{2}$s$^{-1}$~\cite{starmaps1}, and the ALICE-ITS ALPIDE sensor can
operate with a hit rate of 3~MHz/cm$^{2}$~\cite{aliceits}




\subsubsection{Expected Tracking Performance}
%{Performance of the Silicon-based Inner Tracker}
Figure~\ref{fig:4.1.09.z} shows the expected performance of the momentum resolution and position resolution of the tracking system, comparing the two configurations MDC only and MDC + PXD with different settings for the radius of the PXD layers.
A single hit position resolution of 30~$\mu$~$\times$~75~$\mu$m is assumed for the PXD.
The results are obtained from simulation with combined track fitting of the MDC + PXD tracking system, and incident particles are assumed to have a polar angle of cos$\theta=0$.
With MDC + PXD, the momentum resolution is improved by a factor of approximately 1.5 at 1.8~GeV/c compared with MDC tracking only.

As expected, while the impact parameter resolution can be improved when the innermost layer is closer to the beam pipe, the resolution of the momentum and transverse momentum has little dependence on the radius of the innermost layer.


%%%%%%%%%%%%%%%%%%% Fig %%%%%%%%%%%%%%%%%%%%%%%%%%
\begin{figure*}[htb]
	\centering
\subfloat[][]{\includegraphics[height=50 mm]{Figures/Figs_04_00_DetectorSubSystems/Figs_04_02_MainDriftChamber/Si_pos_D0_0p0.png}}
\subfloat[][]{\includegraphics[height=50 mm]{Figures/Figs_04_00_DetectorSubSystems/Figs_04_02_MainDriftChamber/Si_pos_Z0_0p0.png}} \\
\subfloat[][]{\includegraphics[height=50 mm]{Figures/Figs_04_00_DetectorSubSystems/Figs_04_02_MainDriftChamber/Si_pos_p0p0.png}}
\subfloat[][]{\includegraphics[height=50 mm]{Figures/Figs_04_00_DetectorSubSystems/Figs_04_02_MainDriftChamber/Si_pos_pt0p0.png}}
\vspace{0cm}
\caption{The simulated resolution of the impact parameters (a) $d_0$ and (b) $z_0$ and (c) the momentum $p$ and (d) transverse momentum $p_T$ as a function of the $p_T$ of the incident particle. The results with different layout configurations, the default with radii of 36 mm, 98 mm and 160 mm and alternative radii of 60 mm, 110 mm and 160 mm, are compared. }
    \label{fig:4.1.09.z}
\end{figure*}
%%%%%%%%%%%%%%%%%%%%%%%%%%%%%%%%%%%%%%%%%%%%%%%%%%


To investigate the impact of the material budget on the performance of the ITK, the expected momentum resolution and position resolution with different material budgets, ranging from 0.25\%$X_0$ to 1.0\%$X_0$, are compared.
The results from the Geant4 simulation are shown in Fig.~\ref{fig:4.1.10.z}.
In the low momentum range, a degradation of the momentum resolution with a higher material budget is seen.
%Another important factor that needs to be considered is the track finding and reconstruction efficiency.
Further investigation is needed to understand the impact of the material budget on the track finding and reconstruction efficiency, which have a significant impact on the physics potential in the low momentum range at the STCF.


%%%%%%%%%%%%%%%%%%% Fig %%%%%%%%%%%%%%%%%%%%%%%%%%
\begin{figure*}[htb]
	\centering
\subfloat[][]{\includegraphics[height=50 mm]{Figures/Figs_04_00_DetectorSubSystems/Figs_04_02_MainDriftChamber/Si_mat_D0_0p0.png}}
\subfloat[][]{\includegraphics[height=50 mm]{Figures/Figs_04_00_DetectorSubSystems/Figs_04_02_MainDriftChamber/Si_mat_Z0_0p0.png}} \\
\subfloat[][]{\includegraphics[height=50 mm]{Figures/Figs_04_00_DetectorSubSystems/Figs_04_02_MainDriftChamber/Si_mat_p0p0.png}}
\subfloat[][]{\includegraphics[height=50 mm]{Figures/Figs_04_00_DetectorSubSystems/Figs_04_02_MainDriftChamber/Si_mat_pt0p0.png}}
\vspace{0cm}
\caption{The simulated resolution of the impact parameters (a) $d_0$ and (b) $z_0$ and (c) the momentum $p$ and (d) transverse momentum $p_T$ as a function of the $p_T$ of the incident particle. The results with different material budgets, expressed in terms of the radiation length, are compared.}
    \label{fig:4.1.10.z}
\end{figure*}
%%%%%%%%%%%%%%%%%%%%%%%%%%%%%%%%%%%%%%%%%%%%%%%%%%

\subsubsection{CMOS MAPS for the Inner Tracker}
%{Prospect of MAPS for PXD}
A good starting point for the STCF ITK is the ALPIDE design, which was developed for the aforementioned ALICE-ITS upgrade.
The ALICE-ITS has achieved a material budget of approximately 0.3\%$X_0$ with a sensor thickness of 50~$\mu$m.
CMOS pixel sensors, for instance, JadePix~\cite{jadepix}, are also being proposed as the vertex detector for the conceptual design of the Circular Electron Positron Collider (CEPC)~\cite{cepc}. JadePix-1 features a small pixel size ($16\times 16~\mu\mathrm{m}^2$), with the main goal being achieving low power consumption and material budget. For the STCF, the pixel size requirement can be relaxed, and the main challenges are the low power consumption and fast readout necessary to cope with the high event rate (see Sec.~\ref{sec:tdaq}).
In contrast to an ordinary MAPS, which collects ionization charge mainly by diffusion, high-voltage~(HV)/high-resistivity~(HR)-CMOS designs collect ionization charge mainly via drift, as the sensor can be fully depleted.
The fast collection of signal charges and the low noise HV/HR-CMOSs also allow them to be more radiation tolerant.
HV-MAPSs have been prototyped for several experiments, such Mu3e~\cite{mu3e}, ATLAS~\cite{atlaspix} and CLIC~\cite{clicpix}.
Another promising option for the STCF ITK is the MuPix sensor~\cite{mupix} to be used for the Mu3e experiment, which was designed to detect extremely low-momentum tracks ($<50$~MeV/c) with very high tracking efficiency and momentum resolution.

\subsubsection{Readout Circuitry}
The STCF CMOS pixel sensor is expected to be a $2\times2$~cm$^2$ chip that contains 16k pixels of $100\times 250~\mu\mathrm{m}^{2}$ size.
The readout circuitry provides time stamping and charge measurement based on the time-over-threshold (TOT).
The structure of the readout circuitry is shown in Fig.~\ref{fig:pix_readout}.
\begin{figure*}[htb]
	\centering
    \includegraphics[width=0.7\linewidth]{Figures/Figs_04_00_DetectorSubSystems/Figs_04_01_InnerTracker/pixel_readout.pdf}
\vspace{0cm}
\caption{Block diagram of the pixel sensor readout circuit.}
    \label{fig:pix_readout}
\end{figure*}

The readout circuitry mainly consists of an in-pixel part and a periphery part.
There is a low-power charge-sensitive amplifier (CSA) and a voltage comparator in each pixel.
The CSA integrates the charge collected from the sensor and outputs a voltage signal to the following comparator,
and the voltage threshold of the comparator can be tuned by a local DAC.
All the pulse signals from the comparators are driven to the periphery, and the arrival time of each pulse is recorded as a timestamp.
Meanwhile, the pulse width is measured based on the TOT method.
The time stamps and TOT messages are read out through the readout control block to an 8/10-bit encoder
and, finally, to a serializer with an output data rate of several Gbits/s.
Additionally, the configuration block, voltage-controlled oscillator (VCO), and phase-locked loop (PLL) are integrated in the periphery.

