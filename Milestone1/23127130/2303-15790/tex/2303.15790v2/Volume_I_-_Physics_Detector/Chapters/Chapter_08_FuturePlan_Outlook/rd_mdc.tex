The main drift chamber (MDC) is the central part of the tracking system of the STCF detector. The key factors affecting the tracking performance in the STCF experiment are multiple scattering and energy loss of charged particles traversing the MDC. Therefore, the driving force in the design of the MDC is reducing its material budget as much as possible. This also represents the core of the R\&D work required for the MDC detector. The MDC provides ionization measurements for charged particle identification as well as precise position measurements for charged particle tracking. Thus, precise time and charge measurements are both required for the MDC readout electronics. In addition, the high counting rates expected at the MDC inner layers and the extremely high rate of physics events of interest expected when running at the $J/\psi$ peak impose stringent requirements on the MDC readout electronics as well as the MDC detector. A vigorous R\&D program needs to be developed and carried out for the MDC to meet all the technical challenges. Such an R\&D program should cover the following aspects where some critical R\&D items have been identified.
\begin{itemize}
\item \textbf{Detector design optimization}\\
The material budget of the detector should be maximally reduced to enhance the tracking performance for low-momentum particles by optimizing the detector design in terms of various aspects, including the working gas, wire material and size, configuration of wire layers, chamber structure and material. Drift cells that are much smaller than regular small cells and yet maintain a very low material budget for the whole detector should be designed. The adoption of such drift cells would shorten the maximum drift time in a drift cell and thus make the response of the detector faster, allowing it to better cope with the high count rate and physics event rate. The structure  of the drift chamber should be designed, including a detailed deformation analysis that fully takes into account wire tensions and their possible creep effects.

\item \textbf{R\&D of low-mass wires} \\
Electrode wires are one of the primary contributors to the material budget of a drift chamber. It is therefore of great importance to develop low-mass wires for use in a drift chamber. Invention of such low-mass wires could represent a breakthrough in drift chamber technology. The use of low-mass wires would also require much less tension to be applied to the wires than necessary when using regular metal wires. This would effectively reduce the total load on the endplates of the drift chamber due to the wire tension and hence leave much more room for the design and engineering of a light chamber structure. Ideas regarding using light polymeric fibers or carbon monofilaments coated with low-mass metals as wires for a drift chamber have been proposed and are being explored by some Italian groups. This could serve as one direction to pursue in the R\&D of low-mass wires for the STCF drift chamber.

\item \textbf{R\&D of high-density wiring}\\
Very small drift cells imply a very large number of closely spaced wires in a drift chamber. This poses a great challenge to wiring the drift chamber, which includes threading wires through the chamber and fixing them at the endplates. In this case, manual wiring is unlikely to be adequate, and regular feedthroughs can no longer be used to hold wires. Thus, an automatic wiring method along with the corresponding key devices need to be developed to enable efficient and accurate wiring operations. A novel method to fix wires without feedthroughs also needs to be developed.

\item \textbf{R\&D of readout electronics}\\
The MDC readout electronics consist of a transimpedance amplifier (TIA) followed by a shaping circuit and an analog-to-digital converter (ADC). The digitized signals are further filtered and processed to reduce the pile-up and enhance the SNR by a data processing circuit. The charge and time information of the signals are extracted with the processed data. The hardware components of the MDC readout electronics include front-end electronics (FEE), readout units (RUs), and subclock and subtrigger modules. Dedicated R\&D work on the readout electronics is needed to meet the requirements for precise time and charge measurements under high rate conditions. Two technical approaches are planned for R\&D, one based on discrete devices and the other on ASIC chips. The primary weight would be placed on the design and development of an ASIC chip suitable for the STCF drift chamber. The proposed ASIC chip incorporates a TIA, a discriminator and a shaping circuit. The transistor parameters and feedback resistance of the TIA should be carefully optimized in terms of circuit input impedance, bandwidth, and dynamic range.

\item \textbf{Development and characterization of a full-size drift chamber prototype}\\
A full-length drift chamber prototype designed for the STCF needs to be developed using all the key techniques and components developed for the drift chambers. The design of the prototype should closely follow the optimized design of the STCF drift chamber. The prototype would also be fully instrumented with the readout electronics developed for the STCF drift chamber. The prototype along with its readout electronics will be fully tested and characterized to validate the design of the STCF MDC and the key techniques and components of the drift chamber.
\end{itemize}

