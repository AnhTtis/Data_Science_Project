The proposed 1~T solenoid with a 3~m diameter bore for the STCF detector solenoid magnet can be realized by adopting a self-supporting aluminum stabilized low temperature NbTi superconductor. However, a low-mass superconductor and thin-wall structure are important to make the solenoid more transparent so that particles can more easily cross the solenoid.
R\&D activities will focus on key technologies such as special superconductors, large superconducting coil manufacturing processes and liquid helium thermosiphon cooling.

\begin{itemize}
\item In the first stage, to develop a special low-temperature superconductor, an innovative coextruding technique will be used. A Rutherford cable that consists of strands of NbTi wires will be inserted into a pure aluminum stabilizer, forming an aluminum stabilized cable. This cable will then be inserted into high mechanical strength aluminum alloy reinforcement. The critical current $I_c$ must exceed 6~kA@4.2 K@4T.

\item In the second stage, automatic winding equipment with the ability to wind a superconducting coil with a 3 m aperture will be developed.

\item In the third stage, a method of liquid helium thermosiphon cooling for large superconducting solenoids will be developed. To study the phase transition process of helium in the circuit, the changes in the temperature distribution and the density distribution over time, a superconducting prototype of a suitable-scale thermosiphon circuit will be established for simulation and verification before formal solenoid construction.

\end{itemize}
