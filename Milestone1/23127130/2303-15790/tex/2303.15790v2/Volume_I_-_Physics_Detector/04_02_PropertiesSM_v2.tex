\subsection{Determination of the SM parameters}

The $\tau$ lepton has well-defined interactions with other particles in the SM. The experimental measurements are consistent with the SM predictions~\cite{TauR}. With a large sample of $\tau$s, many of the interaction parameters in the SM can be determined with great precision. Here, we discuss some of the most important of these tests: the universality properties, the Michel parameters, the strong coupling constant $\alpha_s$, and the element $V_{us}$ in the Cabibbo--Kobayashi--Maskawa (CKM) mixing matrix.

\subsubsection{The universality test}

The charged-current interaction of the left-handed leptons with the $W$ boson is described by
\begin{eqnarray}
{\cal L} = -{g_i\over \sqrt{2}} \bar l_i \gamma^\mu P_L \nu_i W^-_\mu + \textrm{H.C.},
\end{eqnarray}
where $P_L = (1-\gamma_5)/2$. The term `charged lepton universality' refers to the fact that $g_e=g_\mu = g_\tau$. This is indeed the case in the SM but is not necessarily so in models beyond the SM. Therefore, these quantities can be measured to test the SM. One can obtain the following~\cite{ExpTau} using the very good approximation $B(\mu\to e\bar\nu_e\nu_\mu(\gamma))\approx 1$:
\begin{eqnarray}
{g_\tau\over g_e} &=& \sqrt{ B(\tau^- \to \mu^- \bar \nu_\mu \nu_\tau(\gamma)) {\tau_\mu\over \tau_\tau}
{m^5_\mu\over m^5_\tau} {F_\textrm{corr}(m_\mu, m_e)\over F_\textrm{corr}(m_\tau, m_\mu)}} \;,
\nonumber\\
{g_\tau\over g_\mu} &=& \sqrt{ B(\tau^- \to e^- \bar \nu_e \nu_\tau(\gamma)) {\tau_\mu\over \tau_\tau}
{m^5_\mu\over m^5_\tau} {F_\textrm{corr}(m_\mu, m_e)\over F_\textrm{corr}(m_\tau, m_e)}} \;,
\end{eqnarray}
where $F_\textrm{corr}(m_i, m_j)$ includes radiative corrections and corrections due to the different charged lepton masses. The current data $g_\tau/g_e=1.0029\pm 0.0015$, $g_\mu /g_e = 1.0019\pm 0.0014$, and $g_\tau/g_\mu = 1.0010\pm 0.0015$~\cite{ExpTau} are consistent with the prediction of universality. As discussed earlier, by improving the measurement of the value of $m_\tau$ to a level better than 10~ppm, the universality prediction could be tested at a level more than 3 times better to constrain the allowed room for new physics.

Universality tests could also be carried out by combining the decays $\tau \to P\nu_\tau$ and $P\to l \bar \nu_l$ (with $P=\pi$ and $K$, $l = \mu$ and $e$, and $\nu_l = \nu_\mu$ and $\nu_e$), as the ratio of their decay widths is proportional to $g_\tau^2/g_l^2$:
\begin{eqnarray}
R_l = {\Gamma(\tau \to P\nu_\tau) \over \Gamma(P\to l \bar \nu_l)} {m_\tau/(m^2_\tau - m^2_P)^2\over m_P/(m^2_P - m^2_l)^2} = {g^2_\tau\over g^2_l} \;.
\end{eqnarray}
All these decays have been measured experimentally, with $B(\tau^- \to \pi^-\nu_\tau) = (10.82\pm 0.05)\% $,
$B(\tau^- \to K^-\nu_\tau) = (6.96\pm0.10)\%$,
$B(\pi^- \to \mu^- \bar \nu_\mu) = (99.98770\pm 0.00004)\%$,
$B(\pi^- \to e^- \bar \nu_e) = ( 1.230\pm0.0004)\%$,
$B(K^- \to \mu^- \bar \nu_\mu) = (63.56\pm 0.11)\%$,
and
$B(K^-\to e^-\bar\nu_e) = ( 1.582\pm0.007)10^{-5}$~\cite{PDG}.
The error bars for the $\tau \to \pi(K)\nu_\tau$ decays are presently not as good as those for the pure leptonic $\tau \to \nu_\tau l \bar \nu_l$ decays and yield a weaker constraint. However, with improved sensitivity for $\tau \to\pi(K)\nu_\tau$ (and especially with more monochromatic $\pi(K)$ data near the $\tau^+\tau^-$ production threshold) at the STCF together with improved higher-order theoretical corrections, these decays will provide complementary universality tests.


\subsubsection{The Michel parameters}

Decays of the form $\tau \to l \bar \nu_l \nu_\tau$ provide sensitive constraints on other forms of interactions due to new physics. Barring exotic interactions such as tensor couplings, the most general form of new physics can be parameterized in terms of the Michel parameters $\rho$, $\eta$, $\xi$, and $\delta$~\cite{PDG}:
\begin{eqnarray}
&&{d^2\Gamma(\tau \to l \bar \nu_{l} \nu_\tau) \over x^2 dx d\cos\theta}
{96\pi^3\over G^2_F m^5_\tau}
\nonumber\\
&=&3(1-x) + \rho_l \bigg({8\over 3}x -2\bigg) +
6\eta_l {m_l\over m_\tau}{(1-x)\over x}
-P_\tau\xi_l\cos\theta \bigg[(1-x) + \delta_l\bigg({8\over 3}x
-2\bigg)\bigg]\;,
\end{eqnarray}
where $P_\tau$ is the degree of $\tau$ polarization, $x= E_{l}/ E^\textrm{max}_{l}$, and $\theta$ is the angle between the $\tau$ spin and the $l$ momentum direction. In the SM, the Michel parameters are
\begin{eqnarray}
\rho_l = {3\over 4}\;,\;\;\eta_l = 0\;,\;\;\xi_l = 1\;,\;\;\xi_l\delta_l = {3\over 4}\;.
\end{eqnarray}
Experimentally, the values are~\cite{PDG}
\begin{eqnarray}
&&\rho_e = 0.747\pm 0.010,\;\rho_\mu = 0.763\pm 0.020,\;\;\xi_e = 0.994\pm 0.040,\;\;\xi_\mu = 1.030\pm 0.059,
\\
&&\eta_e = 0.013\pm 0.020,\;\;\eta_\mu = 0.094\pm 0.073,\;\;
(\xi\delta)_e = 0.734\pm 0.028,\;\;(\xi\delta)_\mu = 0.778\pm 0.037.
\nonumber
\end{eqnarray}
Again, the experimental measurements are consistent with the SM predictions.

With the production of a larger number of $\tau$s and improved sensitivities, the STCF will be capable of reducing the error bars by at least a factor of 2. In addition, rare decays such as radiative leptonic decays~\cite{Arbuzov:2016ywn,Lees:2015gea,Shimizu:2017dpq} and multi-charged-lepton decays~\cite{Eidelman:2016aih,Flores-Tlalpa:2015vga} can also be studied at the STCF. This will help to examine the SM electroweak interactions and place limits on new physics contributions.

\subsubsection{Extraction of the strong coupling $\alpha_s$}

It is well known that the strong coupling constant $\alpha_s$ can be extracted from the following ratio~\cite{Tau0}:
\begin{equation}
R_\tau = \frac{\Gamma (\tau^- \to \nu_\tau {\rm hadrons} )}{\Gamma (\tau^-\to\nu_\tau e^- \bar\nu_e)}.
\end{equation}
The theoretical predictions of this ratio have been carefully examined in \cite{Tau1,Tau2}. In accordance with the structure of the weak interactions and the classification of the final states, the ratio can be decomposed as follows:
\begin{equation}
R_\tau = R_{V,ud} + R_{A,ud} + R_{\tau,s}.
\end{equation}
Here, $R_{\tau,s}$ is the contribution from final states containing an $s$ quark, while $R_{V,ud}$ ($R_{A,ud}$) comes from nonstrange final states involving an even (odd) number of pions. Each term contains perturbative and nonperturbative contributions. The perturbative contributions are currently determined at the 5-loop level, while the nonperturbative contributions are estimated via QCD sum rules. Because of the large quark mass $m_s$, a large power correction exists in $R_{\tau,s}$, whose theoretical estimate therefore cannot reach the level of precision of $R_{V,ud}$ and $R_{A,ud}$. The analysis presented in~\cite{TauR} gives the value
\begin{equation}
  \alpha_s (m_\tau) = 0.331\pm 0.013,
\end{equation}
with one set of parameterizations of nonperturbative contributions. To improve the determination, an experimental study at the STCF will be important. Specifically, a precise measurement of $R_{\tau,s}$ and the spectral function containing the strange quark will help to understand the nonperturbative contributions and to precisely extract the CKM matrix element $V_{us}$.

\subsubsection{Extraction of the CKM matrix element $V_{us}$}

The experimental study of hadronic decays of $\tau$ has yielded one of the most precise measurements of $V_{us}$ to date. There are two main methods of determining this parameter. One is by measuring the ratio of the decay widths for $\tau^-\to\pi^- \nu_\tau$ and $\tau^- \to K^-\nu_\tau$, and the other is by measuring the ratio $R_\tau = R_{V,ud} + R_{A,ud} + R_{\tau,s}$ as discussed earlier. Theoretically,
\begin{eqnarray}
&&{B(\tau \to K^- \nu_\tau)\over B(\tau^- \to \pi^- \nu_\tau)} = {f^2_K\over f_\pi^2} {\vert V_{us}\vert^2\over \vert V_{ud}\vert^2}
{(m^2_\tau - m^2_K)^2\over (m^2_\tau - m^2_\pi)^2} {1+\delta R_{\tau/K}\over 1+\delta R_{\tau/\pi}} (1+\delta R_{K/\pi})\;,
\nonumber
\\
&&\vert V_{us}\vert ^2 =
{R_{\tau,s}\over [(R_{V,ud}+R_{A,us})/\vert V_{ud}\vert^2 - \delta R_\textrm{theory}]}\;.
\end{eqnarray}
With the known values from theoretical calculations and experimental measurements~\cite{ExpTau}, namely, $f_K/f_\pi=1.1930\pm 0.0030$,
$V_{ud} = 0.97417\pm 0.00021$, $1+\delta R_{\tau/K} = 1+(0.90\pm 0.22)\%$, $1+\delta R_{\tau/\pi} = 1+(0.16\pm 0.14)\%$, $1+\delta R_{K/\pi} = 1+(-1.13\pm 0.23)\%$, and $\delta R_\textrm{theory} = 0.242\pm 0.032$, one respectively obtains the following results from the above two methods:
\begin{eqnarray}
\vert V_{us} \vert_{\tau K/\pi} = 0.2236\pm 0.0018\;,\;\;\vert V_{us}\vert_{\tau s} = 0.2186\pm 0.0021\;.
\end{eqnarray}
The first value is $1.1~\sigma$ away from the value determined by the unitarity relation, $\vert V_{us}\vert_\textrm{uni} \approx \sqrt{1-\vert V_{ud}\vert^2}= 0.2258\pm0.0009 $, and the second is $3.1~\sigma$ away from $\vert V_{us}\vert_\textrm{uni}$. These deviations need to be further understood with better precision before evidence of new physics beyond the SM can be claimed.


The STCF can measure the values of $R_i$ and may therefore confirm or refute these deviations.

