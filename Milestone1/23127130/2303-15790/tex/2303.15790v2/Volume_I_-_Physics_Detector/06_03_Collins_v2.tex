\subsubsection{The Collins effect in the inclusive production of two hadrons}

If two hadrons in the final state are observed in the kinematic region such that the two hadrons are almost back to back, collinear factorization cannot be used.
However, there is another type of factorization, called transverse-momentum-dependent~(TMD) factorization, that holds in this region~\cite{TMDEP}. The angular distributions in this kinematic region are determined by TMD quark fragmentation functions. These functions describe the fragmentation of an initial parton into the observed hadron, where the hadron has a small transverse momentum with respect to the momentum of the initial parton.
The general form of these angular distributions can be found in~\cite{TMDFF}. Studies of the production
in this region are expected to yield many interesting results regarding TMD parton fragmentation functions. Among them, one, called the Collins function,
is of particular interest. This function describes how a transversely polarized quark fragments into a hadron~\cite{Collins}. Its value is zero
if there is no $T$-odd effect. Belle, operating at $\sqrt{s}=10.6$~GeV, has performed a study of the Collins function~\cite{ColBelle}. It will be interesting to see whether the
Collins function can be measured at the STCF. Theoretical predictions concerning the Collins effect in the energy region of $\sqrt{s}\sim 4$~GeV have been presented in~\cite{SY}.
In general, by studying the angular correlations of the two produced hadrons in the kinematic region, one can extract various TMD quark fragmentation functions. These functions contain information on how quarks are hadronized into a hadron. Studies of TMD parton fragmentation
functions will be important not only for understanding hadronization but also for exploring the inner structure of hadrons
in semi-inclusive DIS, for which one needs to know the TMD parton fragmentation functions in order to extract the TMD parton distribution functions.
