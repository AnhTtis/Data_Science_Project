\subsubsection{$R$ value}
The $R$ value is defined as
\begin{equation}
   R(s) = \frac{\sigma_{\rm tot} ( e^+ e^- \to \gamma^* \to  {\rm hadrons)}}{\sigma (e^+e^-\to \gamma^* \to \mu^+ \mu^-)},
\end{equation}
which is a function of $s$. An early measurement of $R$ was made at BES \cite{BES-R-2000, BES-R-2002}. Recently, it has also been measured by the KEDR and BESIII~\cite{KEDR, BESIII:2021wib}.

From experimental measurements of $R$, one can determine the running of the electroweak coupling and conduct precision
tests of the SM, as demonstrated in a recent study of the global SM fit \cite{Gfit}.
Precise measurements of $R$ enable the determination of the coupling constants in the SM.
Currently, the possible deviation of $(g-2)_\mu$ of the $\mu$ lepton has motivated many efforts to improve
the precision of the theoretical predictions and to explain this deviation as an effect of new physics beyond the SM. The newest result indicates that there are 4.2 standard deviations between the experimentally measured and theoretically predicted $(g-2)_\mu$ values \cite{NewM}.
An important contribution to the uncertainty of $(g-2)_\mu$ is the contribution from hadronic vacuum polarization. This contribution can be extracted from $R$ as measured in experiments.
Therefore, a precise measurement of $R$ can play an important role in precision tests of the SM.
It is clear that more precise results for the $R$ value will be obtained at the STCF.
