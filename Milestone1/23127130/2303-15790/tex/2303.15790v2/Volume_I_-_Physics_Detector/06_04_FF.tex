\subsubsection{Form Factors near Threshold}

{\color{red}
After 100 years of experimental studies, the structure of nucleons is still poorly understood.\cite{Nayak,Yang}
Now, time-like measurements are playing an increasingly important role.
Time-like pair-production measurements are not restricted to nucleons;
form-factors of all of the weakly decaying hyperons can be measured and compared, thereby opening a new, previously
unexplored dimension. Currently available
(statistically limited) time-like experiments demonstrate puzzling features in their threshold cross sections
and electric and magnetic form factors\cite{LarsOlsen:2020dpi}. At STCF, time-like nucleon and hyperon
form-factors  will be measured for $Q^{2}$ values as high as 40~GeV$^{2}$ with precisions that match
existing space-like region results for the proton \& neutron. Because the energy region
of STCF
the extracted form factors may not be at the energy scale at which perturbative QCD of exclusive processes works. However,
the behavior near the threshold is important for extracting information about the interaction between a baryon and antibaryon.
In the decay $J/\psi \to \gamma  p \bar p$
the enhancement has been observed by BES\cite{BES1} near the threshold of the $p\bar p$ system. Babar has reported the enhancement in the process
$e^+ e^- \to p\bar p, \Lambda \bar \Lambda, \Sigma^0 \bar \Sigma^0$\cite{BaB}, respectively. Likely, such an enhancement
is a common phenomenon near threshold. At STCF one can study this enhancement more precisely, and can extend the study to the system of $\Lambda_c \bar \Lambda_c$ to see if the enhancement happens in heavy baryon and antiheavy baryon system.}
