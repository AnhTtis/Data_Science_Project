\subsection{Tests of the $CPT$ invariance with $\jpsi$ decays}
While violations of $C$,~$P$,~$T$,~and~$CP$~symmetries have been well established and characterized, the validity
of $CPT$~symmetry remains intact at a high level of sensitivity. The $CPT$~theorem ~\cite{Schwinger:1951xk}
states that any quantum field theory  that is {\it Lorentz invariant}, has {\it local point-like interaction vertices},
and is {\it Hermitian} ({\it i.e.}, conserves probability) is invariant under the combined operations of
$C$~$P$~and~$T$. Since the three quantum field theories that make up the Standard Model---QED, QCD, and
Electroweak theory---all satisfy these criteria, $CPT$~symmetry has been elevated to  some kind mystical status
in particle physics.  However, there is good reason to believe that $CPT$, like all of the other discrete symmetries,
is violated at least a mass scale of ${\mathcal O}(10^{-35} m)$ {\it i.e.}, at the so-called Planck scale.

One of the requirements for a $CPT$-invariant theory is that it is {\it local}, which means that the
couplings at each vertex occurs at a single point in space-time. But theoretical physics has always had
troubles with point-like quantities. For example, the classical self-energy of the electron is
\begin{equation}
  W_{e}=\frac{e^2}{4\pi\eps_{0} r_{e}},
\end{equation}
which diverges for $r_e\rt 0$.  The {\it classical radius of the electron}, {\it i.e.}, the value of $r_e$
that makes $W_e=m_ec^2$, is $r^{\rm c.r.e.}_e= 2.8\times 10^{-13}$~cm (2.8 fermis),  which is three times the
radius of the proton, and $\sim$300 times larger than experimental upper limits on the electron radius, which
are $<10^{-16}$~cm~\cite{ZEUS:2003eqd}.  Infinities associated with point-like objects persist in quantum field
theories, where they are especially troublesome. In second-order and higher perturbation theory, all of
the diagrams that have virtual particle loops involve integrals over all possible configurations of the virtual
particles in the loops that conserve energy and momentum.  Whenever two of the point-like vertices coincide,
the integrands become infinite and cause the integrals to diverge.

In the QED, QCD and Electroweak quantum field theories that make up the Standard Model, these infinities are
removed by the well established methods of renormalization~\cite{Bethe:1947id,Dyson:1949bp,Wilson:1973jj}. In all
three of these theories, the perturbation expansions are in increasing powers of a dimensionless coupling
strength, $\alpha_{\rm QED}, \alpha_s$ and, $\alpha_{\rm EW}=\sqrt{2}M^2_WG_F/\pi$.\footnote{Specifically
  not just $G_F$, which has the dimension of mass$^{-2}$.}
As a result of this, in the renormalization procedure, relations that exist between different orders of the
perturbation expansion reduce the number of observed quantities that are needed to subtract off divergences.
In QED, for example, there are only two, the electron's mass, and charge (three, if the diagram includes
muons).   However, in quantum theories of gravity, where a massless spin=2 {\it graviton} plays the role of
the photon in QED, the  expansion constant is Newton's gravitational constant $G=\hbar c/M^2_{\rm P}$, where
$M_{\rm P}\equiv\sqrt{\hbar c/G}=1.2\times 10^{19}$~GeV is the {\it Planck mass}. Because of this, every
order in the perturbation expansion has different dimensions and, thus, needs a distinct observed quantity
is needed to carry out the subtraction at each step, which means that complete renormalization would
infinite number of observed quantities to complete the renormalization.  This means that a
$CPT$-conserving quantum theory of gravity would, in principle, be {\it nonrenormalizable}\cite{Weinberg:1980kq}.
Although difficulties associated with non-renormalizability ({\it i.e.}, higher-order perturbative effects)
will never show up at mass scales below the Planck mass, this problem demonstrates that there is nothing
especially sacred about $CPT$-invariance that prevents from being violated at a lower mass scale.  Because of
its close connection with the fundamental assumptions of the Standard Model, stringent experimental tests of
$CPT$~invariance should have high priority



\subsubsection{Kaon mixing and tests of the $CPT$~theorem}

Among the consequences of $CPT$~symmetry are that particle and antiparticle masses and lifetimes are equal.
Since lifetime differences can only come from on-mass-shell intermediate states and do not probe short distance,
high mass physics, these are unlikely to exhibit any $CPT$-violating asymmetry. Instead, the focus here is on the
possibility that particle and antiparticle masses may be different.

The particles with the best measured masses are the stable electron and proton, and, according the
PDG~2020 tables~\cite{Zyla:2020zbs}:
\begin{eqnarray}
  |m_{e^+}-m_{e^-}|&<&4\times 10^{-9}~{\rm MeV}\\
  \nonumber
  |m_{\bar{p}}-m_{p}|&<&7\times 10^{-7}~{\rm MeV}.
\end{eqnarray}
However, these limits do not provide the best tests of $CPT$; the most stringent experimental restriction on
$CPT$ violation comes from the difference between the $\Kzbar$ and $\Kz$ masses:
\begin{equation}
  |M_{\Kzbar}-M_{\Kz}|=<5\times 10^{-16}~{\rm MeV},
\end{equation}
which is $7$-$9$~orders of magnitude more strict than those from the electron and proton mass measurements
even though the value of $M_{\Kz}$ itself is only known to~$\pm 13$~keV. This is because the
Fig.~\ref{fig:k-mix_c-quark-KM} diagrams, taken together with the quantum mechanics of $\Kz$-$\Kzbar$ mixing,
maps the $M_{\Kzbar}-M_{\Kz}$ difference into the quantity $\Delta M=M_{\KL}-M_{\KS}\approx 3.5\times 10^{-12}$~MeV,
which is 14 orders of magnitude lower than $M_{\Kzbar}$ or $M_{\Kz}$ and the independent quantity 
$\Delta\Gamma=\Gamma_{\KS}-\Gamma_{\KL}\approx 7.4\times 10^{-12}$~MeV (which is, coincidently,
$\approx 2\times \Delta M$).

\begin{figure}[!bp]
\centering
\includegraphics[width=0.99\textwidth]{Figures/k-mix_c-quark-KM.pdf}
\caption{\footnotesize The box diagrams for the short-distance contributions to  $\Kz$-$\Kzbar$
mixing.}
 \label{fig:k-mix_c-quark-KM}
  \end{figure}


The mapping is done by computing the difference between the proper-time-dependence of the $K\rt\pipi$
decay rates for neutral $K$ mesons that are tagged as strangeness \Str$=+1$ and \Str$=-1$ at their time of
production ({\it i.e.}, $\tau=0$), and are denoted here as $\Kz(\tau)$ and $\Kzbar(\tau)$, respectively.
At STCF, this tagging is automatically done in $\jpsi\rt\Km\pip\Kz$ and $\jpsi\rt\Kp\pim\Kzbar$ decays,
by the sign of the charged-kaon's electric charge: a $\Km$ tags a $\Kz(\tau)$ and a $\Kp$ tags a $\Kzbar(\tau)$.
These events are quite distinct at a c.m. $\ee\rt\jpsi$ collider as shown in Fig.~\ref{fig:k0kpi-STCF}a, and occur
with a branching fraction $Bf[\jpsi\rt K^\mp\pi^\pm\Kz(\Kzbar)]=(0.56\pm 0.05)\%$, which, for $\jpsi$ decays,
is substantial. Moreover, at a c.m. collider these events are pretty much background free, the only significant
backgrounds are misidentified  $K\rt\pi^\pm\ell^\mp\nu$ decays that also depend on $\Delta M$.

\begin{figure}[!]
\centering
\includegraphics[width=0.99\textwidth]{Figures/k0kpi-STCF.pdf}
\caption{\footnotesize {\bf a)}  A simulated $\jpsi\rt K^-\pip\Kz(\tau)$; $\Kz(\tau)\rt\pipi$ event in
  the BESIII detector.
  {\bf b)} The solid circles show the proper time distribution for simulated strangeness-tagged
  $\Kz(\tau)\rt\pipi$ decays (the open circles are $\Kzbar(\tau)\rt\pipi$ decays). 
  {\bf c)} The reduced asymmetry,
  ${\mathcal A}^{\rm reduced}_{\pipi}={\mathcal A}_{\pipi}\times e^{-{\scriptstyle \frac{1}{2}}\Delta\Gamma\tau}$,
  which weights the asymmetries according to their relative statistical significance, is plotted
  for the events shown in panel~b.  Here $\phi=\phi_{\rm SW}+\delta\phi^{CPT}_{\eta}$. (Figures
  provided by Jian-Yu Zhang.) 
  }
 \label{fig:k0kpi-STCF}
 \end{figure}

Although the event shown in  Fig.~\ref{fig:k0kpi-STCF}a looks superficially like $K^\mp\pi^\pm\KS$, here
the neutral kaon is not a $\KS$ mass eigenstate with a simple exponential lifetime.  Instead, separate plots
of the proper-decay-times for simulated $K^-$-tagged $\Kz(\tau)\rt\pipi$ and $K^+$-tagged $\Kzbar(\tau)\rt\pipi$
events, shown in Fig.~\ref{fig:k0kpi-STCF}b, tell a different, and much more interesting story. Instead of
single exponential curves that decrease indefinitely with the $\KS$ lifetime, the decay curves contain
interfering $\KS$ and $\KL$ components, with interference patterns for $\Kz(\tau)$ and $\Kzbar(\tau)$ decays
that quite different. There is a pronounced asymmetry between the two modes that, according to numerous
available descriptions of the quantum mechanics of the neutral kaon system (see,
{\it e.g.}, ref.~~\cite{Schubert:2014ska}),  can be expressed as
\begin{eqnarray}
  \label{eqn:KKbar-asymm-CPT} 
   {\mathcal A}_{\pipi}&\equiv&\frac{\bar{N}(\Kzbar(\tau))-N(\Kz(\tau))}{\bar{N}(\Kzbar(\tau))+N(\Kz(\tau))}\\
   \nonumber
   &~&~~~~~~=2\Re(\eps-\delta)-2\frac{|\eta_{+-}|e^{{\scriptstyle \frac{1}{2}}\Delta\Gamma\tau}\cos(\Delta M\tau-\phi_{\rm SW}
                 +\delta\phi^{CPT}_{\eta})}{1+|\eta_{+-}|^2e^{\Delta\Gamma\tau}}.
\end{eqnarray}
Here the first term on the r.h.s. of the equation is a  small ($\sim$0.3\%) constant offset involving $\eps$, the
familiar mass-matrix parameter that characterizes the $CP$ impurities in the $\KS$ and $\KL$ mass eigenstates, and
$\delta$, which is an even smaller $CPT$-violating parameter (that is explained below). Inherent differences in the $\Kp$
and $\Km$ detection efficiencies make it impossible to measure this term with any significant precision. On the
other hand the phase of the  oscillation described by the second term, $\phi_{\rm SW} +\delta\phi^{CPT}_{\eta}$, is
not sensitive to experimental efficiencies and can, in principle, be accurately measured. Here $\phi_{\rm SW}$ is
the ``superweak phase'' and is equal to $\tan^{-1}(2\Delta M/\Delta \Gamma)$, where $\Delta M$ can be
measured from the wavelength of the oscillation, and $\Delta\Gamma$ is determined from the $\KS$ and $\KL$
decay curves.\footnote{There  is a small, ${\mathcal O}(0.03^\circ)$ theoretical correction to the
         $\phi_{\rm SW}=\tan^{-1}(2\Delta M/\Delta \Gamma)$ that is well understood~\cite{Bell:1965mn}.}
The difference between the measured phase of the eqn.~\ref{eqn:KKbar-asymm-CPT} oscillation term and 
$\phi_{\rm SW}$ is $\phi^{CPT}_{\eta}=\delta_{\perp}/\eps$, where
\begin{equation}
  \delta_{\perp}=\frac{M_{\Kzbar}-M_{\Kz}}{2\sqrt{2}\Delta M};
\end{equation}
a non-zero value would indicate the existence of a $\Kz$-$\Kzbar$ mass difference, and a violation of $CPT$..   

This translation of $M_{\Kzbar}-M_{\Kz}$ into a relation involving the $M_{\KL}$-$M_{\KS}$ is a trick that is
unique to the kaon system and is not applicable to other particles. Although the neutral $D^0$, $B^0$ and $B^0_s$
mesons mix with their antiparticles, they do not have a measurable equivalent of the kaon's $\eps$ mass-matrix
mixing parameter and, moreover, have a
large number of decay modes that are common to both of their mass eigenstates that make $CPT$-related analyses
that are easily used for neutral kaons impractical~\cite{Bell:1965mn}. As a result, tests of $CPT$~with neutral
$B$ mesons by BaBar~\cite{BaBar:2006zzh} and Belle~\cite{Higuchi:2012kx} did not place any limits on 
$|M_{\bar{B}^0}-M_{B^0}|$.

Although it is apparent from eqn.~\ref{eqn:KKbar-asymm-CPT} that the ${\mathcal A}_{\pipi}$ asymmetry increases
with the proper-time $\tau$, the statistical precision of measurements rapidly decreases with $\tau$. Because
of this, the CLEAR group, which measured this asymmetry in $\bar{p}_{\rm stop}p\rt K^\mp\pi^\pm\Kz(\Kzbar)$ events
produced in the annihilation of stopped antiprotons~\cite{Apostolakis:1999zw}, suggested that an alternative
{\it reduced asymmetry}, defined as
\begin{equation}
  {\mathcal A}^{\rm reduced}_{\pipi}={\mathcal A}_{\pipi}\times e^{-{\scriptstyle \frac{1}{2}}\Delta\Gamma\tau},
\end{equation} 
be used for display since it emphasizes the importance of the higher-statistics, low and intermediate
decay-time measurements that provide the bulk of the measurement sensitivity. The reduced asymmetry between
the decay curves for $\Kz(\tau)\rt\pipi$ and $\Kzbar(\tau)\pipi$ events shown in Fig.~\ref{fig:k0kpi-STCF}b,
is plotted in Fig.~\ref{fig:k0kpi-STCF}c.

The  simulated data shown in Figs.~~\ref{fig:k0kpi-STCF}b~\&~c correspond to 3.8B~tagged $K\rt\pipi$
decays that are almost equally split between $\jpsi\rt\Km\pip\Kz$ and $\jpsi\rt\Kp\pim\Kzbar$ that were generated 
with $\phi^{CPT}_{\eta}=0$ and $\phi_{\rm SW}=43.4^\circ$. This corresponds to what one would expect for a
total of $10^{12}$~$\jpsi$ decays in a detector that covered a $|\cos\theta|\le 0.85$ solid angle and was
otherwise almost perfect.  The red curve in the figure is the result of a fit to the data that determined
\begin{equation}
  \phi_{\rm SW}+\phi^{CPT}_{\eta} = 43.51^\circ\pm 0.05^\circ~~~~
          \Rightarrow~~~~\phi^{CPT}_{\eta} < 0.15^\circ~~(90\%~{\rm C.L.}),
\end{equation}
where the errors are statistical only. To date, the best existing measurements are a 1999 result from the
CPLEAR~$\bar{p}_{\rm stop}p$ experiment at CERN~\cite{Apostolakis:1999zw} that used a sample of $\sim$70M tagged
$\Kz(\tau)\rt\pip
i$ and $\Kzbar(\tau)\rt\pipi$ events and a 1995 result from Fermilab experiment
E773~that used $\sim$2M~$K\rt\pipi$ and $\sim$0.5M~$K\rt\piz\piz$ decays produced downstream of a regeneration
  located in a high-energy $\KL$ beam~\cite{Schwingenheuer:1995uf}:
\begin{eqnarray}
  {\rm CPLEAR:}~~~\phi_{\rm SW}+\phi^{CPT}_{\eta}&=&42.91^\circ\pm 0.53^\circ {\rm (stat)}\pm 0.28^\circ  {\rm (syst)}\\
  \nonumber
  {\rm E773:~~}~~~\phi_{\rm SW}+\phi^{CPT}_{\eta}&=&42.94^\circ\pm 0.58^\circ {\rm (stat)}\pm 0.49^\circ  {\rm (syst)},
\end{eqnarray}
where the systematic errors include $0.19^\circ$ (CPLEAR)~\&~$0.35^\circ$ (E733) contributions from uncertainties in
the regeneration phases.   The potential statistical error for a 1~trillion $\jpsi$~event data sample at STCF is
factor of ten better than previous experiments. The issue will be whether on not the systematic errors can be controlled
accordingly.  In the STCF detector the production and a fraction of the $K\rt\pipi$ decays occur in a high vacuum, and
the material traversed by rest of the decaying neutral kaons is very small, This will substantially reduce the effects of
regeneration, which was the largest source of systematic error in the previous experiments. 

As the results in the previous paragraph indicate, constraints on the validity of the $CPT$ theorem have not changed
in the twenty-five years.  This is not because these are not important or interesting, instead it is because that
no tests of $CPT$ invariance for particle systems other than the neutral kaons have anything like the same sensitivity,
and none of the world's active particle physics facilities have been capable of producing enough kaons to match, much less
improve on, the CPLEAR and E773 results. (The BESIII 10B $\jpsi$ event sample only has $\sim$20M tagged $\Kz(\tau)\rt\pipi$
and $\Kzbar(\tau)\rt\pipi$ decays, less than a third of the CPLEAR data sample.) The STCF project will provide a unique 
opportunity to make an order of magnitude sensitivity improvement over earlier measurements.
