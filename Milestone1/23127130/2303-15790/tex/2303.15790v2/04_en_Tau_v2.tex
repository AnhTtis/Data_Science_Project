\section{Tau Physics}

At the STCF, as many as $3.5\times10^{9}$ $\tau^+\tau^-$ pairs can be produced per year at $\sqrt{s} = 4.26~\GeV$, which is approximately 3 orders of magnitude higher than the currently accumulated number of $\tau^+\tau^-$ events at BESIII. At the production threshold, there could be as many as $10^{8}$ $\tau^+\tau^-$ events per year. Under near-threshold conditions, data from just below the threshold can be used to understand the background to achieve better control over systematic errors compared with BESIII~\cite{YB}. In this regard, the STCF has advantages over both LHCb and Belle II for $\tau$ physics studies. In the energy range covered by the STCF, good control can also be exerted over the polarizations of the $e^+$ and $e^-$ beams to extract new information about $\tau$ physics. The STCF will tremendously increase the statistical significance for $\tau$-related physics studies and will reach a level of precision that has never been achieved before.

The $\tau$ lepton occupies a unique place in the SM. Being the heaviest charged lepton, it has many more decay channels than the next lighter charged lepton, the muon ($\mu$). With an unprecedented number of $\tau$s produced not far from the threshold and possible polarization information at the STCF, one can gain more precise knowledge of not only the properties of the $\tau$ itself but also how it interacts with other particles; thus, one can more precisely determine the SM parameters, probe possible new interactions and possibly also shed light on some of the related anomalies in particle physics. In the following, we describe some of the interesting subjects in $\tau$ physics that can be addressed at the STCF.

\subsection{Precision measurement of the $\tau$ properties}

To test the SM and search for new physics in the $\tau$ sector, it is important for the properties of the $\tau$ to be known with great precision. Here, we list a few measurements at the STCF that can improve our understanding of the $\tau$ properties.

\subsubsection{$\tau$ mass and lifetime}
Many of the tests for the SM and beyond involve precise measurements of the $\tau$ mass ($m_\tau$) and lifetime. While at the threshold for $\tau^+\tau^-$ pair production, measurement of the $\tau$ lifetime is difficult, at the high-energy ($5\sim 7$~GeV) end of the STCF range, it could be possible to measure it by reconstructing the $\tau^+\tau^-$ vertex. With sufficiently high statistics, there is a chance to improve the measurement of the $\tau$ lifetime, for which a more dedicated study would be needed. On the other hand, the mass measurement can also be improved. The mass has been measured at the 70~ppm level, with a world average of~\cite{PDG} $m_\tau = 1776.86\pm0.12~\MeV$. In charged-current induced leptonic decays, $\tau \to \nu_\tau l \bar \nu_l$ $(l= e, \mu)$, the decay widths are proportional to the fifth power of $m_\tau$. Consequently, a small error in the mass can cause significant deviations in tests of the universality of the SM and in the search for new physics. At the STCF, the number of $\tau$s produced may be one to three orders of magnitude greater than at BESIII, which will greatly enhance the statistical significance achieved. With further improvements in particle ID and energy measurement capabilities, the improved sensitivity can increase the accuracy by a factor of 7 to reach a level better than 10~ppm. This improved $\tau$ mass measurement will consolidate the basis for any further $\tau$ physics studies.

\subsubsection{Measurement of $a_\tau = (g_\tau-2)/2$}
The anomalous magnetic dipole moment of the $\tau$ lepton, $a_\tau$, is another property of fundamental importance. The corresponding $a_l$ values for the electron and muon have been measured to high precision. For the electron, there is a $2\sigma$ deviation between the measurement and the SM prediction, $\Delta a_e = a^\textrm{exp}_{e} - a^\textrm{SM}_e =-78(36)\times 10^{-14}$~\cite{electron-mdm}. On the other hand, there is a longstanding and larger discrepancy for the muon moment $a_\mu$, which is currently being measured at Fermilab and J-PARC. Very recently, Fermilab reported their new result from the Run 1 measurement~\cite{Fermilab}. Upon combining it with previous data from BNL, the discrepancy is now $\Delta a_\mu = a^\textrm{exp}_\mu - a^\textrm{SM}_\mu = (251\pm 59)\times 10^{-11}$, and its significance level has been enhanced from $3.7\sigma$ to $4.2\sigma$. As this may be an indication of new physics, it has motivated extensive theoretical studies within the SM and beyond to understand possible causes. It is therefore important to test whether there is also a deviation in $a_\tau$. This is especially important for testing models of new physics that include states whose couplings are proportional to mass.

However, the measurement of $a_\tau$ is drastically different from that of $a_{e,\mu}$ due to the short lifetime of the $\tau$. The SM prediction for $a_\tau$ is $1177.21(5)\times 10^{-6}$~\cite{tau-mdm}. Currently, $a_\tau$ has been measured from the production cross section for $\tau$ pairs together with the spin or angular distributions of the $\tau$ decays; for instance, the current bounds of $-0.052 \leq a_\tau \leq 0.013$ (95\% C.L.) were obtained by the DELPHI collaboration~\cite{Abdallah:2003xd} from the cross section for the process $e^+e^-\to e^+e^-\tau^+\tau^-$ under the assumption that the SM tree-level result is modified only by the anomalous magnetic moment. These measurements are still far from constituting a precision test for the SM, and conventional measurements through similar processes may never reach the necessary level of precision. To overcome this bottleneck, a new method has recently been proposed in Ref.~\cite{Chen:2018cxt}, in which it was found to be feasible to reach a precision level of $1.75\times 10^{-5}$ at Belle II before considering systematics. In addition, it was shown some time ago that in $e^+e^- \to \tau^+\tau^-$ with a polarized electron beam, it would be plausible to achieve this precision goal at the STCF by measuring the transverse and longitudinal polarizations of the $\tau$ lepton~\cite{bernabeu-mdm}. It has been argued that if $\tau^+\tau^-$ pairs are produced on top of the narrow $\Upsilon(1S,2S,3S)$ resonances, with a very well-controlled background near the threshold, a precision even better than that of Belle II can be expected. Nevertheless, it has also been pointed out in Ref.~\cite{Eidelman:2016aih} that an energy spread with $e^+e^-$ beams on the order of a few MeV, which is likely to occur, would make such a measurement impractical because the resonant contributions would be contaminated by nonresonant ones of at least similar size, which would need to be subtracted to extract the dipole moment. In addition, the momentum transfer is too large to be directly related to dipole moments. The authors of Ref.~\cite{Eidelman:2016aih} proposed another method of measuring dipole moments, i.e., by means of radiative decays of the form $\tau^-\to l^-\nu_\tau\bar\nu_l\gamma$. However, they estimated that the sensitivity to $a_\tau$ would be approximately $0.085$ ($0.012$) using the full data of Belle (Belle II), which offers no meaningful improvement compared to the sensitivity of $0.017$ at DELPHI,
%Editor: Please ensure that the intended meaning has been maintained in the above edit.
and the sensitivity to $d_\tau^\gamma$ cannot be improved either. Therefore, more critical studies are needed.


\subsection{Determination of the SM parameters}

The $\tau$ lepton has well-defined interactions with other particles in the SM. The experimental measurements are consistent with the SM predictions~\cite{TauR}. With a large sample of $\tau$s, many of the interaction parameters in the SM can be determined with great precision. Here, we discuss some of the most important of these tests: the universality properties, the Michel parameters, the strong coupling constant $\alpha_s$, and the element $V_{us}$ in the Cabibbo--Kobayashi--Maskawa (CKM) mixing matrix.

\subsubsection{The universality test}

The charged-current interaction of the left-handed leptons with the $W$ boson is described by
\begin{eqnarray}
{\cal L} = -{g_i\over \sqrt{2}} \bar l_i \gamma^\mu P_L \nu_i W^-_\mu + \textrm{H.C.},
\end{eqnarray}
where $P_L = (1-\gamma_5)/2$. The term `charged lepton universality' refers to the fact that $g_e=g_\mu = g_\tau$. This is indeed the case in the SM but is not necessarily so in models beyond the SM. Therefore, these quantities can be measured to test the SM. One can obtain the following~\cite{ExpTau} using the very good approximation $B(\mu\to e\bar\nu_e\nu_\mu(\gamma))\approx 1$:
\begin{eqnarray}
{g_\tau\over g_e} &=& \sqrt{ B(\tau^- \to \mu^- \bar \nu_\mu \nu_\tau(\gamma)) {\tau_\mu\over \tau_\tau}
{m^5_\mu\over m^5_\tau} {F_\textrm{corr}(m_\mu, m_e)\over F_\textrm{corr}(m_\tau, m_\mu)}} \;,
\nonumber\\
{g_\tau\over g_\mu} &=& \sqrt{ B(\tau^- \to e^- \bar \nu_e \nu_\tau(\gamma)) {\tau_\mu\over \tau_\tau}
{m^5_\mu\over m^5_\tau} {F_\textrm{corr}(m_\mu, m_e)\over F_\textrm{corr}(m_\tau, m_e)}} \;,
\end{eqnarray}
where $F_\textrm{corr}(m_i, m_j)$ includes radiative corrections and corrections due to the different charged lepton masses. The current data $g_\tau/g_e=1.0029\pm 0.0015$, $g_\mu /g_e = 1.0019\pm 0.0014$, and $g_\tau/g_\mu = 1.0010\pm 0.0015$~\cite{ExpTau} are consistent with the prediction of universality. As discussed earlier, by improving the measurement of the value of $m_\tau$ to a level better than 10~ppm, the universality prediction could be tested at a level more than 3 times better to constrain the allowed room for new physics.

Universality tests could also be carried out by combining the decays $\tau \to P\nu_\tau$ and $P\to l \bar \nu_l$ (with $P=\pi$ and $K$, $l = \mu$ and $e$, and $\nu_l = \nu_\mu$ and $\nu_e$), as the ratio of their decay widths is proportional to $g_\tau^2/g_l^2$:
\begin{eqnarray}
R_l = {\Gamma(\tau \to P\nu_\tau) \over \Gamma(P\to l \bar \nu_l)} {m_\tau/(m^2_\tau - m^2_P)^2\over m_P/(m^2_P - m^2_l)^2} = {g^2_\tau\over g^2_l} \;.
\end{eqnarray}
All these decays have been measured experimentally, with $B(\tau^- \to \pi^-\nu_\tau) = (10.82\pm 0.05)\% $,
$B(\tau^- \to K^-\nu_\tau) = (6.96\pm0.10)\%$,
$B(\pi^- \to \mu^- \bar \nu_\mu) = (99.98770\pm 0.00004)\%$,
$B(\pi^- \to e^- \bar \nu_e) = ( 1.230\pm0.0004)\%$,
$B(K^- \to \mu^- \bar \nu_\mu) = (63.56\pm 0.11)\%$,
and
$B(K^-\to e^-\bar\nu_e) = ( 1.582\pm0.007)10^{-5}$~\cite{PDG}.
The error bars for the $\tau \to \pi(K)\nu_\tau$ decays are presently not as good as those for the pure leptonic $\tau \to \nu_\tau l \bar \nu_l$ decays and yield a weaker constraint. However, with improved sensitivity for $\tau \to\pi(K)\nu_\tau$ (and especially with more monochromatic $\pi(K)$ data near the $\tau^+\tau^-$ production threshold) at the STCF together with improved higher-order theoretical corrections, these decays will provide complementary universality tests.


\subsubsection{The Michel parameters}

Decays of the form $\tau \to l \bar \nu_l \nu_\tau$ provide sensitive constraints on other forms of interactions due to new physics. Barring exotic interactions such as tensor couplings, the most general form of new physics can be parameterized in terms of the Michel parameters $\rho$, $\eta$, $\xi$, and $\delta$~\cite{PDG}:
\begin{eqnarray}
&&{d^2\Gamma(\tau \to l \bar \nu_{l} \nu_\tau) \over x^2 dx d\cos\theta}
{96\pi^3\over G^2_F m^5_\tau}
\nonumber\\
&=&3(1-x) + \rho_l \bigg({8\over 3}x -2\bigg) +
6\eta_l {m_l\over m_\tau}{(1-x)\over x}
-P_\tau\xi_l\cos\theta \bigg[(1-x) + \delta_l\bigg({8\over 3}x
-2\bigg)\bigg]\;,
\end{eqnarray}
where $P_\tau$ is the degree of $\tau$ polarization, $x= E_{l}/ E^\textrm{max}_{l}$, and $\theta$ is the angle between the $\tau$ spin and the $l$ momentum direction. In the SM, the Michel parameters are
\begin{eqnarray}
\rho_l = {3\over 4}\;,\;\;\eta_l = 0\;,\;\;\xi_l = 1\;,\;\;\xi_l\delta_l = {3\over 4}\;.
\end{eqnarray}
Experimentally, the values are~\cite{PDG}
\begin{eqnarray}
&&\rho_e = 0.747\pm 0.010,\;\rho_\mu = 0.763\pm 0.020,\;\;\xi_e = 0.994\pm 0.040,\;\;\xi_\mu = 1.030\pm 0.059,
\\
&&\eta_e = 0.013\pm 0.020,\;\;\eta_\mu = 0.094\pm 0.073,\;\;
(\xi\delta)_e = 0.734\pm 0.028,\;\;(\xi\delta)_\mu = 0.778\pm 0.037.
\nonumber
\end{eqnarray}
Again, the experimental measurements are consistent with the SM predictions.

With the production of a larger number of $\tau$s and improved sensitivities, the STCF will be capable of reducing the error bars by at least a factor of 2. In addition, rare decays such as radiative leptonic decays~\cite{Arbuzov:2016ywn,Lees:2015gea,Shimizu:2017dpq} and multi-charged-lepton decays~\cite{Eidelman:2016aih,Flores-Tlalpa:2015vga} can also be studied at the STCF. This will help to examine the SM electroweak interactions and place limits on new physics contributions.

\subsubsection{Extraction of the strong coupling $\alpha_s$}

It is well known that the strong coupling constant $\alpha_s$ can be extracted from the following ratio~\cite{Tau0}:
\begin{equation}
R_\tau = \frac{\Gamma (\tau^- \to \nu_\tau {\rm hadrons} )}{\Gamma (\tau^-\to\nu_\tau e^- \bar\nu_e)}.
\end{equation}
The theoretical predictions of this ratio have been carefully examined in \cite{Tau1,Tau2}. In accordance with the structure of the weak interactions and the classification of the final states, the ratio can be decomposed as follows:
\begin{equation}
R_\tau = R_{V,ud} + R_{A,ud} + R_{\tau,s}.
\end{equation}
Here, $R_{\tau,s}$ is the contribution from final states containing an $s$ quark, while $R_{V,ud}$ ($R_{A,ud}$) comes from nonstrange final states involving an even (odd) number of pions. Each term contains perturbative and nonperturbative contributions. The perturbative contributions are currently determined at the 5-loop level, while the nonperturbative contributions are estimated via QCD sum rules. Because of the large quark mass $m_s$, a large power correction exists in $R_{\tau,s}$, whose theoretical estimate therefore cannot reach the level of precision of $R_{V,ud}$ and $R_{A,ud}$. The analysis presented in~\cite{TauR} gives the value
\begin{equation}
  \alpha_s (m_\tau) = 0.331\pm 0.013,
\end{equation}
with one set of parameterizations of nonperturbative contributions. To improve the determination, an experimental study at the STCF will be important. Specifically, a precise measurement of $R_{\tau,s}$ and the spectral function containing the strange quark will help to understand the nonperturbative contributions and to precisely extract the CKM matrix element $V_{us}$.

\subsubsection{Extraction of the CKM matrix element $V_{us}$}

The experimental study of hadronic decays of $\tau$ has yielded one of the most precise measurements of $V_{us}$ to date. There are two main methods of determining this parameter. One is by measuring the ratio of the decay widths for $\tau^-\to\pi^- \nu_\tau$ and $\tau^- \to K^-\nu_\tau$, and the other is by measuring the ratio $R_\tau = R_{V,ud} + R_{A,ud} + R_{\tau,s}$ as discussed earlier. Theoretically,
\begin{eqnarray}
&&{B(\tau \to K^- \nu_\tau)\over B(\tau^- \to \pi^- \nu_\tau)} = {f^2_K\over f_\pi^2} {\vert V_{us}\vert^2\over \vert V_{ud}\vert^2}
{(m^2_\tau - m^2_K)^2\over (m^2_\tau - m^2_\pi)^2} {1+\delta R_{\tau/K}\over 1+\delta R_{\tau/\pi}} (1+\delta R_{K/\pi})\;,
\nonumber
\\
&&\vert V_{us}\vert ^2 =
{R_{\tau,s}\over [(R_{V,ud}+R_{A,us})/\vert V_{ud}\vert^2 - \delta R_\textrm{theory}]}\;.
\end{eqnarray}
With the known values from theoretical calculations and experimental measurements~\cite{ExpTau}, namely, $f_K/f_\pi=1.1930\pm 0.0030$,
$V_{ud} = 0.97417\pm 0.00021$, $1+\delta R_{\tau/K} = 1+(0.90\pm 0.22)\%$, $1+\delta R_{\tau/\pi} = 1+(0.16\pm 0.14)\%$, $1+\delta R_{K/\pi} = 1+(-1.13\pm 0.23)\%$, and $\delta R_\textrm{theory} = 0.242\pm 0.032$, one respectively obtains the following results from the above two methods:
\begin{eqnarray}
\vert V_{us} \vert_{\tau K/\pi} = 0.2236\pm 0.0018\;,\;\;\vert V_{us}\vert_{\tau s} = 0.2186\pm 0.0021\;.
\end{eqnarray}
The first value is $1.1~\sigma$ away from the value determined by the unitarity relation, $\vert V_{us}\vert_\textrm{uni} \approx \sqrt{1-\vert V_{ud}\vert^2}= 0.2258\pm0.0009 $, and the second is $3.1~\sigma$ away from $\vert V_{us}\vert_\textrm{uni}$. These deviations need to be further understood with better precision before evidence of new physics beyond the SM can be claimed.


The STCF can measure the values of $R_i$ and may therefore confirm or refute these deviations.


\subsection{$CP$ symmetry tests}


How $CP$ symmetry is broken may hold the key to why our universe contains more matter than antimatter. The violation of $CP$ symmetry is one of the required conditions to understand this. There is insufficient $CP$ violation in the SM to explain this fundamental question affecting our very existence in the Universe, and therefore, new sources of $CP$ violation are demanded. The search for new $CP$-violating effects is one of the most active areas in particle physics. Physical processes involving the $\tau$ lepton are potential sectors in which new $CP$-violating effects may appear.

\subsubsection{$CP$ violation in $\tau^- \to K^0_S \pi^- \nu_\tau$}

In the SM, because of the $CP$ violation in $K^0$--$\bar K^0$ mixing, a detectable $CP$-violating effect is predicted for this process~\cite{Bigi:2005ts1, Bigi:2005ts2}:
\begin{eqnarray}
A_Q = {B(\tau^+ \to K^0_S \pi^+ \bar \nu_\tau) - B(\tau^- \to K^0_S \pi^- \nu_\tau) \over
B(\tau^+ \to K^0_S \pi^+ \bar \nu_\tau) + B(\tau^- \to K^0_S \pi^- \nu_\tau)} = (+0.36\pm 0.01)\%\;.
\end{eqnarray}
While Belle observed no $CP$ violation in the angular distributions for the exclusive decays~\cite{Bischofberger:2011pw},
BaBar yielded a value of $A_Q=(-0.36\pm0.23\pm0.11)\%$ for the inclusive decays with $\ge 0\pi^0$ in the final states~\cite{BABAR:2011aa}, which is $2.8\sigma$ away from the SM prediction.

The above deviation represents a challenge to the SM. Theoretical efforts have been made to reconcile this deviation. However, even with beyond-the-SM effects included, it is not easy to obtain the central value of the BaBar data. The STCF can provide a crucial check with a large number of $\tau^+\tau^-$ pairs produced not far from the threshold, where the background can be well controlled. At the STCF, the expected luminosity of 1~ab$^{-1}$/year at an energy of 4.26~GeV can allow a statistical sensitivity of $9.7\times 10^{-4}$ to $CP$ violation to be reached. With 10 years of operation, the sensitivity can reach $3.1\times 10^{-4}$~\cite{Sang:2020ksa}, which will be comparable to the sensitivity of $10^{-4}$ projected for Belle II with a luminosity of 50~ab$^{-1}$~\cite{Chen:2020uxi}. The STCF can thus provide crucial information for resolving the $A_Q$ discrepancy.

\subsubsection{Measurement of the electric dipole moment of the $\tau$}

The initial state of an $e^+e^-$ pair in the center-of-mass system is a $CP$ eigenstate. Therefore, $CP$ tests can be conveniently performed at any $e^+e^-$ collider. By measuring the decay products from $\tau$ decays, a $CP$ test can be conducted based on the $e^+e^-\to  \tau^+\tau^-$ process, as suggested in~\cite{CPTau1, CPTau2}. By measuring $CP$-odd observables, one can determine the electric and weak dipole moments of the $\tau$. In the SM, these moments are predicted to be extremely small (for example, the electric dipole moment is expected to be on the order of $10^{-34}$~e~cm). If either of the two moments is nonzero at a level much larger than the SM predictions, it will be a clear signal of new physics beyond the SM. These two moments have been studied at LEP and $B$ factories. While the weak dipole moment is suppressed at low energy by the large masses of the weak gauge bosons, the electric dipole moment $d^\gamma_\tau$ can be probed at $B$ and $\tau$--charm factories. The newest result for the electric dipole moment obtained from the Belle experiment~\cite{TauBelle}, in units of $10^{-16}$~e~cm, is
\begin{equation}
-0.22 < {\rm Re} (d^\gamma_\tau ) < 0.45, \ \ \ \ -0.25 < {\rm Im} ( d^\gamma_\tau) < 0.08.
\end{equation}
These bounds can be tightened by 2 or 3 orders of magnitude through experiments at the STCF.

\subsubsection{CP violation with polarized beams}

With polarized $e^+$ and/or $e^-$ beams, highly polarized $\tau^\pm$s can be produced. $\tau$ polarizations normal ($N$) to their production plane can be measured by studying semileptonic decays of the form $\tau^\pm\to\pi^\pm/\rho^\pm\bar\nu_\tau(\nu_\tau)$. One can then construct asymmetry observables with respect to the left-hand ($L$) and right-hand ($R$) sides of the plane, which are directly related to the electric dipole moment of the $\tau^\pm$~\cite{bernabeu-cp}:
\begin{eqnarray}
A^\pm_N = {\sigma_L^\pm - \sigma_R^\pm\over \sigma} = \alpha_\pm {3\pi \beta\over 8 (3-\beta^2)}{2m_\tau \over e} \textrm{Re}(d^\gamma_\tau)\;,
\end{eqnarray}
where $\sigma$ is the cross section for $e^+e^- \to \tau^+\tau^- \to (\pi^+/\rho^+)\bar\nu_\tau (\pi^- /\rho^-)\nu_\tau$, $\beta = \sqrt{1-4m^2_\tau/s}$, and $\alpha_\pm$ is the polarization analyzer in the $\tau^\pm\to\pi^\pm/\rho^\pm\bar\nu_\tau(\nu_\tau)$ decays. Belle II can reach a sensitivity of $3\times 10^{-19}$~e~cm with an integrated luminosity of $50~\textrm{ab}^{-1}$. At the STCF, the sensitivity can be improved by a factor of approximately 30, reaching $10^{-20}$~e~cm.

With polarized $e^+$ and $e^-$ beams, one can also construct new $T$-odd observables to measure $CP$-violating effects. An interesting observable is the triple product $P^{\tau^{\pm}}_z\hat z\cdot(\vec p_{\pi^\pm}\times\vec p_{\pi^0})$ from measuring the two pion momenta in the $\tau^\pm\to\pi^\pm\pi^0\bar\nu_\tau(\nu_\tau)$ decays~\cite{paultsai}.
Here, $P^\tau_z = [(w_{e^-} + w_{e^+})/(1+w_{e^+}w_{e^-})][(1+2a)/(2+a^2)]$ is the component of the polarization vector of the $\tau$ obtained upon averaging over its momentum direction, with $w_{e^\pm}$ being the components of the polarization vectors of the $e^\pm$ in the $e^-$ beam direction $\hat z$ and $a = 2m_\tau/\sqrt{s}$. If the difference in the triple products for $\tau^+$ and $\tau^-$ is nonzero, this will be a signal of $CP$ violation. Since the SM predicts very small values for the triple products, the measurement of a nonzero difference would already signal new physics beyond the SM. This measurement can be done at the STCF to provide new information about sources of $CP$ violation. Similar measurements can be carried out by replacing $\pi^\pm$ with $K^\pm$.

\subsection{Flavor-violating $\tau$ decays}

Lepton flavor-changing neutral current (FCNC) interactions of the $\tau$ are suppressed in the SM when the neutrino masses and mixing are incorporated. In new physics models beyond the SM, larger FCNC effects may appear in some decays, such as $\tau$ decays into $3l$, $l \gamma$, and one or more hadrons plus charged leptons. With the increased statistics for $\tau$ events at the STCF, these decays can be searched for to test the SM and beyond.

\subsubsection{The $\tau^- \to 3l$ decay}

The $\tau^- \to 3l$ decay is one of the most sensitive probes of FCNC interactions. The current upper bound is on the order of $10^{-8}$. At Belle II, upon the accumulation of $50~\textrm{ab}^{-1}$ of integrated luminosity, the sensitivity can reach $4\times 10^{-10}$. When running the STCF at its peak energy ($\sqrt{s} = 4.26$ GeV), it will be possible to produce $3.5\times 10^{9}$ $\tau$ pairs each year, which could be used to push the branching ratio down to a level of $1.9\times 10^{-10}$ with 10~ab$^{-1}$ of luminosity~\cite{tau23l}.

\subsubsection{The $\tau^-\to l\gamma$ decays}

Equally interesting are the $\tau\to l\gamma$ decays, where $l=e$ and $\mu$. The current limits for these decays are also on the order of $10^{-8}$. Since initial-state radiation effects are strongly suppressed near the $\tau^+\tau^-$ production threshold, the STCF has an advantage over $B$ factories in a search for these decays~\cite{Bobrov:2012kg}. At the STCF, the sensitivity for this branching ratio will be able to reach $1.2\times 10^{-9}$ with 10~ab$^{-1}$ of luminosity.

\subsubsection{The $\tau^- \to lP_1P_2$ decays}

The $\tau^\pm \to l^\pm P_1P_2$ decays, where $P_i = \pi$ and $K$, have been previously searched for with a sensitivity on the order of $10^{-8}$. Similar to these decays are the lepton-number-violating $\tau^\pm\to l^\mp P_1^\pm P_2^\pm$ decays, for which the current bounds are also on the order of $10^{-8}$. At the STCF, the sensitivity for these decays can be increased by two orders of magnitude to a few times $10^{-10}$.

As mentioned earlier, FCNC interactions are highly suppressed in the SM. In some new physics models, however, FCNC interactions can be generated at the tree level and may therefore induce some of the above processes at a level close to their current bounds. In this circumstance, the STCF will be capable of providing very useful information on those models.


%\input{04_05_Summary}
%\input{04_ref_Tau_v2}

