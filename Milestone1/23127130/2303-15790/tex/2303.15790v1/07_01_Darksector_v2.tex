
\subsection{Particles in the dark sector}

The existence of a dark sector that weakly couples to the SM sector is well motivated by many BSM theories. Some new physics particles may exist at the TeV scale or above and can only be probed at high-energy colliders. However, the messengers connecting the dark sector to the SM sector may exist at lower energies, such as the GeV scale. These messengers can be scalars, pseudoscalars, or gauge bosons that interact with SM particles through certain ``portals'' \cite{Essig:2013lka}. Because this new light sector would need to interact with SM particles very weakly to evade the constraints from current experiments, it is generally dubbed the ``dark sector''.

A particular motivation for such a scenario follows from the observations of anomalous cosmic-ray positrons. In 2008, the PAMELA collaboration reported excess positrons above $\sim$ 10~GeV \cite{Adriani:2008zr}, and this observation has since been confirmed by many other experiments, such as ATIC \cite{Chang:2008aa}, Fermi-LAT \cite{Abdo:2009zk} and AMS02 \cite{Aguilar:2013qda}. In one class of dark matter models, dark matter particles with masses of $\sim \mathcal{O}(\mathrm{TeV})$ annihilate into a pair of light bosons with masses of $\sim \mathcal{O}(\mathrm{GeV})$, which then decay into charged leptons \cite{ArkaniHamed:2008qn,Pospelov:2008jd}.


These light bosons may be massive dark photons in models with an extra U(1) gauge symmetry. These dark photons couple to photons through the kinetic mixing $\frac{\epsilon}{2} F^{\mu \nu} F'_{\mu \nu}$. Since QED is a well-tested model, the mixing strength $\epsilon$ should be small. In theory, $\epsilon$ can be generated by high-order effects \cite{ArkaniHamed:2008qp}. Therefore, $\epsilon$ is naturally $\sim 10^{-2}$--$10^{-3}$ or smaller. The dark photon could acquire a mass through the Higgs mechanism or the Stueckelberg mechanism. Some models could predict that the mass of the dark photon should be at the $\sim \mathcal{O}(\mathrm{MeV})$--$\mathcal{O}(\mathrm{GeV})$ scale \cite{ArkaniHamed:2008qp,Cheung:2009qd}. This suggests that the structure of the dark sector may be complicated. There could be a broad class of light particles, including scalars, pseudoscalars, gauge bosons and fermions, at the GeV scale.


Since the interaction between the dark sector and the SM sector must be very weak, a search for light dark photons (or other light particles) at the intensity frontier is well motivated. In the phenomenology of the dark photon, the most important parameters are the dark photon mass $m_{A'}$ and the mixing strength $\epsilon$. Fig.~\ref{fig:limits} shows the constraints on $\epsilon$ and $m_A'$ from the measurements of the electron and muon anomalous magnetic moments, low-energy $e^+e^-$ colliders, beam dump experiments and fixed-target experiments \cite{Essig:2013lka}. Due to their high luminosity and low center-of-mass energy, which should be close to the mass of the dark photon, electron--positron colliders are also suitable for probing dark photons through either direct production or rare decays of mesons.

\begin{figure}[!htbp]
\centering
\includegraphics[width=0.5\textwidth]{Figs_07_NewPhys/Darksect_limits.pdf}
\caption{Constraints on the mixing strength $\epsilon$ versus the dark photon mass $m_{A'}>1$ MeV from the measurements of the electron and muon anomalous magnetic moments, low-energy $e^+e^-$ colliders, beam dump xperiments and fixed-target experiments. For details, see Ref. \cite{Essig:2013lka}. Reproduced from Ref.~\cite{Essig:2013lka}.}
\label{fig:limits}
\end{figure}

Electron--positron collisions could directly produce dark photons, which would subsequently decay into charged leptons, via $e^+e^-\rightarrow \gamma+A' (\rightarrow l^+l^-)$ \cite{Zhu:2007zt,Fayet:2007ua,Reece:2009un,Essig:2009nc,Yin:2009mc}. In comparison with the irreducible QED background $e^+e^-\rightarrow \gamma l^+l^-$, dark photon production is suppressed by a factor of $\epsilon^2$. To reduce the influence of this background, precise reconstruction of the dark photon mass and a high luminosity are important. Such studies for $\Upsilon \rightarrow \gamma+A' (\rightarrow \mu^+\mu^-)$ have been performed by interpreting results from the BaBar experiment \cite{Aubert:2009cp,Bjorken:2009mm,Reece:2009un}. Since no new peak has been found in the data, the mixing strength $\epsilon$ is constrained to be smaller than $\sim 2 \times 10^{-3}$ for a dark photon with a mass of $\sim$ 1~GeV, and this limit could be reduced to $5\times 10^{-4}$ at SuperB \cite{O'Leary:2010af}.
With 20~fb$^{-1}$ of data collected at $\psi(3770)$, sensitivity to an $\epsilon$ as low as $2\times10^{-3}$ could be achieved for $e^+e^-\rightarrow \gamma+A' (\rightarrow l^+l^-)$
with $m_{A'}\sim 1$~GeV~\cite{Li:2009wz}. A similar estimate can be performed for the STCF, indicating that the sensitivity for $\epsilon$ will be
$\mathcal{O}(10^{-4})$ for $m_{A'} \sim 0.6-3.7$ GeV with 1~ab$^{-1}$ of data, as shown in Fig.~\ref{fig:BESIII}.



\begin{figure}[!htbp]
\centering
\includegraphics[width=0.6\textwidth]{Figs_07_NewPhys/Darksect_STCF.pdf}
\caption{The sensitivity to the mixing strength $\epsilon$ at the STCF for $e^+e^-\rightarrow \gamma+A' (\rightarrow l^+l^-)$ with 1~ab$^{-1}$ of data. Reproduced from Ref.~\cite{Li:2009wz}.}
\label{fig:BESIII}
\end{figure}

If there is also a light Higgs $h'$ that provides the mass of the dark photon, with a mass of $\sim \mathcal{O}(\mathrm{MeV})$--$\mathcal{O}(\mathrm{GeV})$, in the dark sector, then some new processes can be used to investigate the dark sector at electron--positron colliders \cite{Baumgart:2009tn,Batell:2009yf}. If $m_{h'}>2 m_{A'}$, then the signal process $e^+e^-\rightarrow A'+h' (\rightarrow 2 A')\rightarrow 3 l^+l^-$ will be very clean for dark research due to the presence of several resonances in the lepton pairs. If $m_{h'}< m_{A'}$, then $h'$ can only decay into lepton pairs via loop processes. In this case, the lifetime of the $h'$ will be long; possible signals are displaced vertices or even missing energy in the detector. Note that other light bosons may also exist, such as gauge bosons under an extra non-Abelian symmetry, in the dark sector \cite{Baumgart:2009tn}. The final states of direct production may contain more lepton pairs. In this case, it will be easier to extract the signals from large QED backgrounds via the reconstruction of resonances.

In general, if mesons have decay channels into photons, they could also decay into dark photons with branching ratios of approximately $\epsilon^2 \times \mathrm{BR}(\mathrm{meson}\rightarrow \gamma)$ \cite{Reece:2009un,Li:2009wz}. Since low-energy electron--positron colliders produce numerous mesons, such as $\pi$, $\rho$, $K$, $\phi$, and $J/\psi$, it is possible to search for dark photons in the rare decays of mesons. For instance, one can search for a resonance in the $\phi \rightarrow \eta+A'$ and $\pi/\eta\rightarrow \gamma+A'$ processes with $ A'\rightarrow l^+l^-$. At the STCF, where a large sample of charm mesons will be produced, charmonium decay channels, such as $J/\psi \rightarrow e^+e^-+A'$ \cite{Zhu:2007zt} and $\psi(2S)\rightarrow \chi_{c1,2}+A'$, can be used to probe dark photons.

Light $Z'$ bosons that decay predominantly into invisible states can be probed by looking for missing energy in the final states at electron colliders; see, e.g.\, ref.\ \cite{Adachi:2019otg} for the recent Belle II results. Other light particles, such as light axion-like particles, can also be searched for at electron colliders; see, e.g.\, ref.\ \cite{BelleII:2020fag} for the recent Belle II results.


