\section{Solenoid}
\label{sec:solenoid}
The STCF detector magnet is a superconducting solenoid coil which is surrounded by iron yoke to provide an axial magnetic field of 1.0 Tesla over the tracking volume. Particle detectors within this volume will measure the trajectories of charged tracks emerging from the collisions. Particle momentum is determined from the measured curvature of these tracks in the field.

The main parameters of the STCF detector superconducting magnet are listed in Table~\ref{tab:4.6.01}. Its room temperature bore diameter is approximately 2.98 m. The conceptual design of the magnet, including the design of the magnetic field, solenoid coil, cryogenics, quench protection, power supply and the iron Yoke are briefly described in this section.

%%%%%%%%%%%%%%%%%  TABLE  %%%%%%%%%%%%%%%%%%%%%%%%
\begin{table*}[hptb]
\small
    \caption{Main parameters of the STCF detector superconducting magnet.}
    \label{tab:4.6.01}
    \vspace{0pt}
    \centering
    \begin{tabular}{ll}
        \hline
        \thead[l]{Cryostat} & \thead[l]{ }\\
        \hline
        Inner radius&	1.450 m \\
        Outer radius&	1.850 m \\
        Length&	4.760 m \\
        \hline
        Coil & \\
        \hline
        Mean radius	&1.565 m \\
        Length	&4.000 m \\
        Conductor dimension	&4.67$\times$15.0 mm$^{2}$ \\
        \hline
        Electrical parameters & \\
        \hline
        Central field	&1.0 T \\
        Nominal current	&3820 A \\
        Inductance	&1.68 H \\
        Stored energy	&12.3 MJ \\
        Cold mass	&4.6 ton \\
        Radiation thickness	&1.9 X$_{0}$ \\
        Cool down time from room temperature	&$\leq$7 days \\
        Quench recovery time	&$\leq$7 hours \\
        \hline
    \end{tabular}
\end{table*}
%%%%%%%%%%%%%%%%%%%%%%%%%%%%%%%%%%%%%%%%%%%%%%%%%%

\subsection{Magnetic Field Design}
%\paragraph{Main parameters}
The STCF detector magnet will comprise the design concepts similar to TOPAZ magnet in Tristan and CMS magnet in LHC~\cite{1,2,3}. The magnet system consists of the superconducting coil and the iron yoke with a barrel yoke and two end-cap yokes. The superconducting coil is designed as a single layer solenoid wound on the inner surface of an aluminum cylinder. It will run in an operating current 3820 A which corresponding to 1 T magnetic field at the interaction point. The geometrical layout of magnet are shown in Fig.~\ref{fig:4.6.01}.

%%%%%%%%%%%%%%%%%%% Fig %%%%%%%%%%%%%%%%%%%%%%%%%%
\begin{figure*}[hp]
    \centering
{
        \includegraphics[width=0.7\linewidth]{Figures/Figs_04_00_DetectorSubSystems/Figs_04_06_Solenoid/fig01.png}
}
\vspace{0cm}
\caption{2D geometrical layout of the STCF detector magnet, which consists of a superconducting coil and an iron yoke with a barrel yoke and two end-cap yokes.}
    \label{fig:4.6.01}
\end{figure*}
%%%%%%%%%%%%%%%%%%%%%%%%%%%%%%%%%%%%%%%%%%%%%%%%%%

%\paragraph{Magnetic field design}
The magnetic field simulation has been calculated in 2D FEA model, with a fine structure of the barrel Yokes and end-cap Yokes with pole tips at each end side. The iron yoke act as the absorber plates of the MUD and provide the magnetic flux return. There are two main field parameters concerned, one is the uniformity in the tracking volume, and the other is the fringe field along the beam axis outside the detector.

Fig.~\ref{fig:4.6.02} shows the magnetic field flux of the magnet. The distribution of the magnetic field along the beam line is shown in Fig.~\ref{fig:4.6.03}(a), and the field uniformity is presented in Fig.~\ref{fig:4.6.03}(b). The uniformity in the tracking volume is approximately 2\%. The fringe field remains at 50 Gauss at a distance of 5.1~m from the IP. It decreases sharply to less than 50 Gauss with the addition of some iron material to the gap between the barrel yoke and end yoke.

%%%%%%%%%%%%%%%%%%% Fig %%%%%%%%%%%%%%%%%%%%%%%%%%
\begin{figure*}[hp]
    \centering
{
        \includegraphics[width=0.6\linewidth]{Figures/Figs_04_00_DetectorSubSystems/Figs_04_06_Solenoid/fig02.png}
}
\vspace{0cm}
\caption{Flux of the magnetic field.}
    \label{fig:4.6.02}
\end{figure*}
%%%%%%%%%%%%%%%%%%%%%%%%%%%%%%%%%%%%%%%%%%%%%%%%%%

%%%%%%%%%%%%%%%%%%% Fig %%%%%%%%%%%%%%%%%%%%%%%%%%
\begin{figure*}[hp]
    \centering
{
\subfloat[][]{\includegraphics[width=0.35\linewidth]{Figures/Figs_04_00_DetectorSubSystems/Figs_04_06_Solenoid/fig03.png}}
\vspace{0.5cm}
\subfloat[][]{\includegraphics[width=0.45\linewidth]{Figures/Figs_04_00_DetectorSubSystems/Figs_04_06_Solenoid/fig04.png}}
}
\vspace{0cm}
\caption{(a) Bz as a function of Z along the beam axis.
(b) The field uniformity in the tracking area.
}
    \label{fig:4.6.03}
\end{figure*}
%%%%%%%%%%%%%%%%%%%%%%%%%%%%%%%%%%%%%%%%%%%%%%%%%%


%\FloatBarrier

\subsection{Solenoid Coil Design}
By referring to similar magnets in other HEP laboratories, we decide to adopt a single layer of coil, indirect cooling by liquid helium, a pure aluminum-based stabilizer and a NbTi/Cu superconductor in the STCF superconducting magnet. The overall dimensions of the magnet should be a length of 4.76~m, with an inner diameter of 2.9~m and an outer diameter of 3.7~m; the coil effective length is 4.0~m, and the coil mean diameter is 3.13~m.

Assuming the nominal current as \emph{I}=4000~A, \emph{B$_{0}$=$\mu$$_{0}$nI}, and \emph{B$_{0}$}=1~T, the number of turns in a 1~m long coil should be n $\approx$ 208, and the width of the cable should be 4.67~mm. In total, 832 turns should be needed.
The energy stored by the solenoid is:
\begin{eqnarray}
E = (\frac{1}{2}H \cdot B)V = \frac{1}{2}\cdot\frac{B^2}{\mu_0}\cdot S\cdot l = \frac{1}{2} \cdot \frac{B^2}{\mu_0}\cdot\frac{\pi D^2}{4}\cdot l = 12.3 MJ
\end{eqnarray}

The inductance can be derived as follows:
\begin{eqnarray}
\Phi= B \cdot S \cdot n = B \cdot \frac{\pi D^2}{4} \cdot n = 6274.5 WB
\end{eqnarray}
\begin{eqnarray}
\frac{\mathrm{d}\Phi}{\mathrm{d}t} = L\frac{\mathrm{d}I}{\mathrm{d}t}
\end{eqnarray}
\begin{eqnarray}
L=\frac{\Phi}{I}
\end{eqnarray}
\quad\\
which gives \emph{L}=1.68~H.


From the limitation of the maximum temperature rise of 70~K after quenching, the enthalpy difference of the cable material can be obtained, and the height of the superconductor should be 15~mm.

To address the hoop stress, an aluminum cylinder is necessary; material A5083, which is good for welding and has high mechanical strength, is used; this cylinder is the primary structural component of the cold mass, and coolant is supplied through tubes attached to the outer surface of the restraining hoop cylinder. This eliminates the large cryogen inventory and thick cryostat necessary for cooled coils. The conductor is cooled by thermal conduction through the thickness of the hoop restraint. Two coaxial aluminum cylinders with close endplates cooled by liquid nitrogen act as the radiation heat shield.

From calculations, the thickness of this support cylinder should be 18 mm. After cooling-down and excite, the maximum equivalent stress on the support cylinder is 23.9 MPa. The equivalent stress distribution in the coil is shown in the Fig.~\ref{fig:4.6.04.1}, and the peak stress is 58.8 MPa. The total weight of the cold mass is 4.6 tons, including the cable, the support cylinder, the end flange and the cooling tube. Figure ~\ref{fig:4.6.04.2} shows the conceptual structure layout of the solenoid magnet.

%%%%%%%%%%%%%%%%%%% Fig %%%%%%%%%%%%%%%%%%%%%%%%%%
\begin{figure*}[hp]
    \centering
{
        \includegraphics[width=0.7\linewidth]{Figures/Figs_04_00_DetectorSubSystems/Figs_04_06_Solenoid/fig2_1.png}
}
\vspace{0cm}
\caption{The equivalent stress distribution in the coil.}
    \label{fig:4.6.04.1}
\end{figure*}
%%%%%%%%%%%%%%%%%%%%%%%%%%%%%%%%%%%%%%%%%%%%%%%%%%

%%%%%%%%%%%%%%%%%%% Fig %%%%%%%%%%%%%%%%%%%%%%%%%%
\begin{figure*}[hp]
    \centering
{
        \includegraphics[width=0.7\linewidth]{Figures/Figs_04_00_DetectorSubSystems/Figs_04_06_Solenoid/fig2_2.png}
}
\vspace{0cm}
\caption{Conceptual structure layout of the solenoid.}
    \label{fig:4.6.04.2}
\end{figure*}
%%%%%%%%%%%%%%%%%%%%%%%%%%%%%%%%%%%%%%%%%%%%%%%%%%

The superconducting conductor is composed of NbTi/Cu~(1:1) Rutherford cable and a high purity ($>$99.99\%, RRR$>$500) pure aluminum stabilizer. The Rutherford cable contains 16 NbTi strands. The Main parameters of the superconducting conductor are listed in Table~\ref{tab:4.6.02}. The cross section of the superconductor is shown in the Fig.~\ref{fig:4.6.06}, the superconducting wire is located in the center of the aluminum stabilizer. With this configuration, the coil temperature kept below 70 K after quenching at operating current.

%%%%%%%%%%%%%%%%%  TABLE  %%%%%%%%%%%%%%%%%%%%%%%%
\begin{table*}[hptb]
\small
    \caption{Main parameters of the superconducting conductor.}
    \label{tab:4.6.02}
    \vspace{0pt}
    \centering
    \begin{tabular}{ll}
        \hline
        \thead[l]{Rated current} & \thead[l]{3820 A}\\
        \hline
        Critical current at 4.2 k \& 2 T&	$\geq$15000 A \\
        Conductor length&	9.15 km \\
        Cable dimension&	4.67 mm $\times$ 15 mm \\
        \hline
        Rutherford cable parameters & \\
        \hline
        Number of strands	&16 \\
        Cable transposition pitch	&100$\pm$5 mm \\
        Cu:NbTi	&~1:1 \\
        NbTi filament diameter	&30$\pm$5 $\mu$m \\
        Number of filaments	&$\geq$600 \\
        N value@2T	&$\geq$35 \\
        \hline
        Aluminum stabilizer parameters & \\
        \hline
        RRR@0T,4.2K	&$\geq$500 \\
        Yield strength@4.2k	&$\geq$60MPa \\
        Impurity content	&$\textgreater$1000ppm \\
        Cross-section ratio of aluminum	&$\textgreater$80\% \\
        \hline
    \end{tabular}
\end{table*}
%%%%%%%%%%%%%%%%%%%%%%%%%%%%%%%%%%%%%%%%%%%%%%%%%%

%%%%%%%%%%%%%%%%%%% Fig %%%%%%%%%%%%%%%%%%%%%%%%%%
\begin{figure*}[hp]
    \centering
{
        \includegraphics[width=0.5\linewidth]{Figures/Figs_04_00_DetectorSubSystems/Figs_04_06_Solenoid/fig06.png}
}
\vspace{0cm}
\caption{Cross-sectional view of the superconductor for the STCF detector magnet.}
    \label{fig:4.6.06}
\end{figure*}
%%%%%%%%%%%%%%%%%%%%%%%%%%%%%%%%%%%%%%%%%%%%%%%%%%

The coil windings are wound by the inner winding technique with the aluminum-alloy cylinder support, which acts as an external supporting mandrel and removes part of the heat energy induced by quenching. To maintain the operating temperature of the detector magnet, the cooling tubes for the circular flow of liquid helium are welded on the outer surface of the aluminum-alloy cylinder.

The titanium tie rods will be used for the coil suspension system and provide axial and radial fixation, to ensure the precise and rigid suspension of the cold mass inside the vacuum vessel. The loads to be supported are the self weight of the cold mass and the magnetic forces due to the decentering and misalignment of the coil with respect to the return yoke. The design must also take into count the contraction of the coil during cooling and its deformation under magnetic forces, with enough strengths and smallest heat conducting. 8 vertical rods are used to counteract the electromagnetic and cold shrinkage forces, and the diameter of the rods is designed to be 25 mm.

\subsection{Iron Yoke Design}
The iron yoke has three functions. First, it provides a magnetic flux return path to shield the leaked field. Second, it is used as the absorber material for the muon detector, which is sandwiched between the multilayer iron plates. Finally, it is the mechanical support structure of the overall detector. Therefore, a high permeability material with high mechanical strength is required for the yoke material to account for the mechanical and magnetic field performance. Low-carbon steel is chosen for the yoke based on its magnetic properties. However, this material has relatively low strength. Thus, a balance between good magnetic properties and moderate strength must be achieved. The yoke is divided into two main components, one barrel yoke and two end-cap yokes. A detailed design would include data cables, cooling pipes, and gas pipes passing through the yoke from the inner detectors. The total weight of the iron yoke assembly is approximately 800 tons.
Fig.~\ref{fig:4.6.07} shows the iron yoke configuration with the main dimensions.

%%%%%%%%%%%%%%%%%%% Fig %%%%%%%%%%%%%%%%%%%%%%%%%%
\begin{figure*}[hp]
    \centering
{
        \includegraphics[width=0.7\linewidth]{Figures/Figs_04_00_DetectorSubSystems/Figs_04_06_Solenoid/fig07.png}
}
\vspace{0cm}
\caption{Configuration of the iron yoke with pole tips for the STCF detector.}
    \label{fig:4.6.07}
\end{figure*}
%%%%%%%%%%%%%%%%%%%%%%%%%%%%%%%%%%%%%%%%%%%%%%%%%%

The barrel yoke is an octagonal-shaped structure with a length of 4,760~mm. The outer and inner heights of the octagon are 5,820~mm and 3,760~mm, respectively. The barrel yoke supports the magnet cryostat. The inner vacuum vessels of the cryostat host inner detectors. The barrel yoke is subdivided into 10 layers, with 40~mm gaps between the layers for the muon detector.
The end-cap yoke also consists of 11 layers of steel plates. Used as the end door of the detector, it is designed to slide apart to provide an opening for access to the inner detectors. 
The end-cap muon chambers are mounted in the 4~cm gap between the vertical end-cap yoke disks and are normally inaccessible.
The details can be found in Sec.~\ref{sec:yoke}. 


\subsection{Quench Protection and Power Supply}
The protection of the solenoid is based on the classical concept of the extraction of stored energy by an external resistor.
Approximately 12~MJ of energy is stored in the magnet.
Selected voltage signals from the STCF detector magnet coil and power leads are monitored by a special quench detection device.
If a quench occurs, the power supply is switched off, and a dump resistor is switched into the electrical circuit; the large stored energy is extracted mainly by the dump resistor and partially by the coil itself. The value of the dump resistor limits the maximum voltage across the solenoid terminals.
A 0.1~Ohms dump resistor is mounted in parallel to the breakers, which are doubled for safety. As the operating current is approximately 3820 A, the maximum voltage is 382 V with respect to ground.
The power supply is designed to operate with 10~V and 4000~A, with a current stability of less than 10~ppm within 8 hours, a current ripple of less than 10~ppm and an accuracy of 20~ppm. The repeatability should reach 10~ppm.



\subsection{Magnet Cryogenics}
Solenoid cryogenics cooling is based on the thermosiphon cooling method, where the superconducting coil is indirectly cooled with saturated liquid helium. The thermosiphon circuit consists of a helium phase separator located in an elevated position and cooling tubes. A horizontal cryostat is designed, including a vacuum tank, an inner thermal shield, and an outer thermal shield. The stainless steel vacuum vessel is a cylinder with a length of 4.76~m. The thickness of the inner wall, outer wall, and endplates is 6 mm, 16 mm, and 32 mm, respectively. Figure 11.5.1 presents the vacuum vessel stress and deformation near the endplate, indicating a 41.80 MPa stress and 0.23 mm strain in the middle of the inner vessel.

%%%%%%%%%%%%%%%%%%% Fig %%%%%%%%%%%%%%%%%%%%%%%%%%
\begin{figure*}[htbp]
    \centering
{
        \includegraphics[width=0.6\linewidth]{Figures/Figs_04_00_DetectorSubSystems/Figs_04_06_Solenoid/fig09.png}
}
\vspace{0cm}
\caption{Stress and deformation of vacuum vessel(2D 1/2 Model).}
    \label{fig:4.6.09}
\end{figure*}
%%%%%%%%%%%%%%%%%%%%%%%%%%%%%%%%%%%%%%%%%%%%%%%%%%

The service tower is designed on the top section of the barrel iron yoke. Table~\ref{tab:11.5.1} shows the heat load estimation of the magnet.

%%%%%%%%%%%%%%%%%%%%%%%%%%%%%%%%%%%%%%%%%%%%%%%%%%
\begin{table*}[hptb]
\small
    \caption{Heat load estimation.}
    \label{tab:11.5.1}
    \vspace{0pt}
    \centering
    \begin{tabular}{lll}
        \hline
Heat Load Components & 77~K & 4.5~K \\
        \hline
Caused by the support rods in cryostat	& 27~W & 	1.0~W \\
Caused by the radiation in cryostat	& 74~W &	3.2~W \\
Caused by the current leads		& ——	& 7.9~W + 0.4~g/s \\
Caused by the radiation in chimney \& SP	&	10~W &	0.4~W \\
Caused by the support rods in chimney \& SP	&	4~W	& 0.1~W \\
Caused by the bayonet and valves in SP	& 46~W	&	13~W \\
Caused by the measuring wires		& 5~W	& 0.8~W \\
        \hline
Total	& 166~W	26.4~W + 0.4~g/s \\
        \hline
Adopted heat load~($\times 1.5$) & 249~W	& 39.6~W + 0.6~g/s \\
        \hline
    \end{tabular}
\end{table*}
%%%%%%%%%%%%%%%%%%%%%%%%%%%%%%%%%%%%%%%%%%%%%%%%%%

The STCF detector cryogenic system consists of a helium refrigerator, liquid and gas transfer lines, liquid and gas storage, and a nitrogen system. The major components include compressors, oil removal systems, cold boxes and control systems. All the helium supplied by this system, except for normal leakage and necessary venting, is circulated and reliquefied. Liquid nitrogen usage includes the cooling of the transfer line thermal shield, the cooling of the thermal shield and thermal intercepts in the superconducting solenoid magnet, and the cooling of the high-pressure helium feed stream in the refrigerator.




