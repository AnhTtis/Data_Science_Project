\section{Inner Tracker~(ITK)}
\label{sec:itk}

\subsection{Introduction}
The STCF physics programs demand high detection efficiency and good spatial resolution for charged particle tracks while imposing no explicit requirements on vertex reconstruction. As a consequence, the main tasks of the STCF inner tracker are to detect particle hits of charged particles, especially those with very low momentum, below 100~MeV/c, and to facilitate the reconstruction of charged particle tracks with the MDC. In this chapter, the performance requirements of the inner tracker and the conceptual baseline designs are described. The resulting expected performance is also discussed through a simulation study.

\subsection{Performance Requirements and Technology Choices}
The major performance requirements for the inner tracker for realizing the expected detection capabilities are listed below:
\begin{itemize}
\item Low material budget: about 0.25\% X/X$_{0}$ for each detector layer.
\item Spatial resolution: single-hit spatial resolution better than 100~$\mu$m in the $r-\phi$ direction. 
\item  Detector occupancy: not exceeding a few percent.
\item Radiation tolerance requirements as described in Sec.~\ref{sec:mdi_bkg}.
\end{itemize}

Several detector technologies can fulfill the requirements above, such as micropattern gaseous detectors (MPGDs) and silicon pixel detectors. 
A silicon pixel detector~ \cite{itk1} was used as the vertex detector at the STAR~\cite{itk2} and Belle II~\cite{itk3} experiments, showing good performance in spatial resolution and rate capabilities \cite{itk4}. MPGD-based detectors, 
a cylindrical gas electron multiplier (CGEM) detector~\cite{itk5}, was used as the inner tracker for the KLOE~\cite{itk6} detector and the BESIII upgrade~\cite{itk7}. 
%
In the CDR of the STCF, an MPGD-based design is the baseline choice for the inner track, while a silicon pixel detector is considered to be an alternative design. The sensitive lengths in each of the layers are determined by the required angular acceptance, {\it i.e.}, a polar angle range of 20 degrees to 160 degrees.

%% muRWell


\subsection{$\mu$RWELL-based Inner Tracker}

\subsubsection{Detector Design}

$\mu$RWELL~\cite{itk8,itk9,itk10} is a single-amplification stage resistive micropattern gaseous detector.
It has been introduced as a thin, simple and robust detector for very large area applications requiring operation in harsh radiation environments.
Figure~\ref{fig:4.1.02} shows the schematic structure of the inner tracker system based on the $\mu$RWELL detector, consisting of 3 cylindrical detector layers  with inner radii of 60~mm, 110~mm and 160~mm, and full-length of 330~mm, 610~mm, and 880~mm.
The radii are determined based on the radiation tolerance of the $\mu$RWELL detector and the MDC range according to the background simulation in Sec.~\ref{sec:mdi_bkg}. The innermost wire layer of the MDC (Sec.~\ref{sec:mdc}) has a radius of approximately 200~mm. The radii of the three layers of the $\mu$RWELL-based inner tracker ensure uniform gaps between the inner tracker hits and the first MDC hit.
The 3 layers are functionally and structurally independent of each other, thus simplifying the manufacturing and maintenance processes. As illustrated in Fig.~\ref{fig:4.1.02} (right), each detector layer consists of 2 cylinders and 4 pairs of sealing rings. The inner cylinder  has a sandwich-like structure made of polyimide film and supporting material, providing a detector frame with sufficient mechanical strength. The outer surface of the inner cylinder  is coated with a thin aluminum foil, acting as the drift cathode of the $\mu$RWELL detector. The outer cylinder  is the cylindrical $\mu$RWELL foil, which acts as both an electron multiplier and signal readout unit. Four pairs of sealing rings are located at both ends of the cylindrical detector, sealing the gap between the polyimide films and the $\mu$RWELL foil. The space between the inner cylinder  and the outer cylinder  is the gas volume. To ensure the gas tightness of the detector, each seam of the gas volume is sealed with epoxy resin.

A very low material budget is crucial for the measurement of low-momentum charged particles. It is essential to reduce the material contributions of the inner tracker.
In this baseline design, the mechanical strength of the detector is mainly provided by the sandwich-like inner cylinder , and the material budget in the detection area can be limited to a very low level, as shown in Table~\ref{tab:4.1.01}.

%%%%%%%%%%%%%%%%%%% Fig %%%%%%%%%%%%%%%%%%%%%%%%%%
\begin{figure*}[htb]
	\centering
\subfloat[][]{\includegraphics[width=0.7\textwidth]{Figures/Figs_04_00_DetectorSubSystems/Figs_04_01_InnerTracker/fig02_uRWELL_innertracker_structure.png}} \\
\subfloat[][]{\includegraphics[width=0.7\textwidth]{Figures/Figs_04_00_DetectorSubSystems/Figs_04_01_InnerTracker/fig03_mechanicaldesign_uRWELLdetector.png}}
\vspace{0cm}
\caption{(a) The schematic structure of the $\mu$RWELL-based inner tracker. (b) The structure of each layer of the $\mu$RWELL detector.}
    \label{fig:4.1.02}
\end{figure*}
%%%%%%%%%%%%%%%%%%%%%%%%%%%%%%%%%%%%%%%%%%%%%%%%%%

%%%%%%%%%%%%%%%%%  TABLE  %%%%%%%%%%%%%%%%%%%%%%%%
\begin{table*}[htb]
\small
    \caption{The material budget of the $\mu$RWELL-based inner tracker design.}
    \label{tab:4.1.01}
    \vspace{0pt}
    \centering
    \begin{tabular}{llll}
        \hline
        \thead[l]{Structure} & \thead[l]{Material}& \thead[l]{Thickness\\(cm)} & \thead[l]{Material budget\\(X/X$_0$)}\\
        \hline
        Inner cylinder  &Aluminum (X$_0$=8.897 cm)           &0.001	 &0.011\%  \\
        	       &Polyimide (X$_0$=28.57 cm)	       &0.01	 &0.035\%  \\
	               &Aramid honeycomb/Rohacell (X$_0$$\simeq$267 cm)	&0.2	&0.075\% \\
        Gas volume  &Argon-based gas mixture (X$_0$=11760 cm)	&0.5	&0.00425\% \\
        Outer cylinder  ($\mu$RWELL foil)	&Alumium (X$_0$=8.897 cm)	&0.0015	&0.017\% \\
	               &Polyimide (X$_0$=28.57 cm)	&0.03	&0.106\% \\
	               &DLC (X$_0$=12.13 cm)	&0.0001	&0.00082\% \\
        Total	   &	                     &	                  &0.249\% \\
        \hline
    \end{tabular}
\end{table*}

\subsubsection{Detector Simulation and Optimization}

%\paragraph{Working point and gas component optimization}
%\quad\\
To obtain optimal detection performance from the $\mu$RWELL-based inner tracker, it is crucial to determine the working point and gas component.
Based on a simulation study with {\sc Garfield-9}~\cite{garfield9} and {\sc Geant4}~\cite{geant4}, the impact of these parameters on the detector spatial resolution is investigated, and the detector design is optimized.
\quad\\
The spatial resolution is dependent on various detector parameters, such as the drift electric field strength, working gas component and width of the gas volume. The optimization of these parameters must take into account the effect of the 1~T magnetic field in the $z$-direction. To simplify the optimization, the full simulation starts from an ideal gas component. Figure~\ref{fig:4.1.ideal gas component} shows the Geant4 simulated spatial resolution as a function of various parameters being studied. The Lorentz angle and the electron drift velocity have optimal values of approximately 30 degrees and 2~cm/$\mu$s, respectively. Additionally, a larger gas volume width and a smaller transverse diffusion coefficient enhance the spatial resolution. The optimal Lorentz angle, electron drift velocity and transverse diffusion coefficient can be realized by choosing a suitable gas component and electric field. However, a larger gas volume width leads to a longer signal duration time, decreasing the counting rate capabilities and increasing the occupancy ratio of the detector. As a compromise, the gap width is set to 5~mm.

%%%%%%%%%%%%%%%%%%% Fig %%%%%%%%%%%%%%%%%%%%%%%%%%
\begin{figure*}[htb]
	\centering
    \includegraphics[width=140mm]{Figures/Figs_04_00_DetectorSubSystems/Figs_04_01_InnerTracker/ideal_gas_component.png}
\vspace{0cm}
\caption{The $r-\phi$ spatial resolution as a function of various parameters based on the Geant4 simulation.}
    \label{fig:4.1.ideal gas component}
\end{figure*}
%%%%%%%%%%%%%%%%%%%%%%%%%%%%%%%%%%%%%%%%%%%%%%%%%%

It is feasible to realize the expected performance of the gas mixture by adding an appropriate type and proportion of doping gas to the noble gas. With the {\sc Garfield-9} simulation and the analysis of tens of different argon-based gas mixtures, a suitable working gas component is found, as shown in Fig. \ref{fig:4.2.real gas component}. The {\sc Garfield-9} simulation shows that the gas mixture of Ar:CO$_{2}$=85:15 gives a Lorentz angle of 29.2 degrees, an electron drift velocity of 2.34~cm/$\mu$s, and a transverse diffusion coefficient of 191~$\mu$m/$\sqrt{\mathrm{cm}}$, with a drift electric field strength of 500~V/cm. All the parameters are within the preferred ranges, indicating that this gas mixture could be a suitable working gas for the $\mu$RWELL-based inner tracker.

%%%%%%%%%%%%%%%%%%% Fig %%%%%%%%%%%%%%%%%%%%%%%%%%
\begin{figure*}[htb]
	\centering
\subfloat[][]{\includegraphics[width=45 mm]{Figures/Figs_04_00_DetectorSubSystems/Figs_04_01_InnerTracker/real_gas_lorenzangle.png}}
\hspace{3 mm}
\subfloat[][]{\includegraphics[width=45 mm]{Figures/Figs_04_00_DetectorSubSystems/Figs_04_01_InnerTracker/real_gas_electrondriftvelocity.png}}
\hspace{3 mm}
\subfloat[][]{\includegraphics[width=45 mm]{Figures/Figs_04_00_DetectorSubSystems/Figs_04_01_InnerTracker/real_gas_transversediffusion.png}}
\vspace{0cm}
\caption{Dependence of the (a) Lorentz angle, (b) electron drift velocity and (c) transverse diffusion coefficient on the drift electric field strength with the 1 atm gas mixtures, simulated by {\sc Garfield-9}.}
    \label{fig:4.2.real gas component}
\end{figure*}
%%%%%%%%%%%%%%%%%%%%%%%%%%%%%%%%%%%%%%%%%%%%%%%%%%



\subsubsection{Detector Performance}


%\paragraph{Spatial resolution of single point}
%\quad\\
A previous study~\cite{itk12} on $\mu$RWELL indicates that, with the $\mu$-time projection chamber (TPC) %Editor: Please ensure that the intended meaning has been maintained in this edit.
mode, the spatial resolution of the detector is almost flat over a wide range of incidence angles, as shown in Fig.~\ref{fig:4.3.spatial resolution simulation result} (a). In the $\mu$-TPC mode, each ionization cluster is projected into a 2-D spatial and 1-D time distribution inside the conversion gap. With the drift time measurement of the primary ionized electrons, the track segment in the gas volume can be reconstructed, and a precise spatial resolution can be obtained. In this mode, the $\mu$RWELL detector can achieve a spatial resolution of approximately 100~$\mu$m.
The spatial resolution of different types of particles with the same transverse momentum of 100 MeV/c is determined by using {\sc Geant4} simulation, as shown in Fig.~\ref{fig:4.3.spatial resolution simulation result}. The spatial resolution in the $r-\phi$ and beamline directions can be limited below 100~$\mu$m and 450~$\mu$m, respectively. Additionally, positively charged particles have a better spatial resolution than negatively charged particles due to the influence of the magnetic field. Both charged particles and ionized electrons are deflected during migration in the $\mu$RWELL gap since the Lorenz angle is not 0 in the magnetic field. For positively charged particles, the deflection directions of primary particles and ionized electrons are opposite, leading to a larger track projection range. For negative particles, the deflection directions of primary particles and ionized electrons are the same, resulting in a smaller tracker projection range. In the $\mu$-TPC mode, a larger track range is beneficial to the tracking performance. Thus, positively charged particles have a better spatial resolution than negatively charged particles.

%%%%%%%%%%%%%%%%%%% Fig %%%%%%%%%%%%%%%%%%%%%%%%%%
\begin{figure*}[htb]
	\centering
    \subfloat[][]{\includegraphics[height=40mm]{Figures/Figs_04_00_DetectorSubSystems/Figs_04_01_InnerTracker/fig06_singlepointspatialresolution.png}} \\
    \subfloat[][]{\includegraphics[height=50mm]{Figures/Figs_04_00_DetectorSubSystems/Figs_04_01_InnerTracker/spatial_resolution_rphi.png}}
    \hspace {5 mm}
    \subfloat[][]{\includegraphics[height=50mm]{Figures/Figs_04_00_DetectorSubSystems/Figs_04_01_InnerTracker/spatial_resolution_z.png}}
	\vspace{0cm}
	\caption{
        (a) The spatial resolution $\sigma$$_r$$_\phi$ as a function of the incidence angle with the $\mu$RWELL detector operated in the $\mu$-TPC mode.
        The spatial resolution in the (b) $r-\phi$ and (c) beamline direction of various particles with the same transverse momentum of 100 MeV/c and the $\mu$RWELL detector operated in the $\mu$-TPC mode.
        }
    \label{fig:4.3.spatial resolution simulation result}
\end{figure*}
%%%%%%%%%%%%%%%%%%%%%%%%%%%%%%%%%%%%%%%%%%%%%%%%%%

%%%% material budget
To investigate the impact of the material budget on the performance of the ITK, the expected momentum resolution and position resolution with different material budgets, ranging from 0.15\%$X_0$ to 0.45\%$X_0$, are compared.
A single hit position resolution of $100\times400$~$\mu$m is assumed.
The results are shown in Fig.~\ref{fig:4.3.11} and are obtained from {\sc Geant4} simulation with a combined tracking fitting of the MDC + ITK tracking system. The incident particles ($\pi$) are assumed to have a polar angle of cos$\theta=0$.
In the low momentum range, the momentum resolution degrades with a higher material budget.
%Another important factor that needs to be considered is the track finding and reconstruction efficiency.


%%%%%%%%%%%%%%%%%%% Fig %%%%%%%%%%%%%%%%%%%%%%%%%%
\begin{figure*}[htb]
	\centering
\subfloat[][]{\includegraphics[height=50 mm]{Figures/Figs_04_00_DetectorSubSystems/Figs_04_02_MainDriftChamber/uRwell_mat_D0_0p0.png}}
\subfloat[][]{\includegraphics[height=50 mm]{Figures/Figs_04_00_DetectorSubSystems/Figs_04_02_MainDriftChamber/uRwell_mat_Z0_0p0.png}} \\
\subfloat[][]{\includegraphics[height=50 mm]{Figures/Figs_04_00_DetectorSubSystems/Figs_04_02_MainDriftChamber/uRwell_mat_p0p0.png}}
\subfloat[][]{\includegraphics[height=50 mm]{Figures/Figs_04_00_DetectorSubSystems/Figs_04_02_MainDriftChamber/uRwell_mat_pt0p0.png}}
\vspace{0cm}
\caption{The simulated resolution of the impact parameters (a) $d_0$ and (b) $z_0$ and (c) the momentum $p$ and (d) transverse momentum $p_T$ as a function of the $p_T$ of the incident particle. The results with different material budgets, expressed in terms of the radiation length, are compared.}
    \label{fig:4.3.11}
\end{figure*}
%%%%%%%%%%%%%%%%%%%%%%%%%%%%%%%%%%%%%%%%%%%%%%%%%%




\subsubsection{Counting rate and layout of readout strips}
%\paragraph{Readout channel, electronics, counting rate and data size estimation}
%\quad\\
From the background simulation study, the highest background count rate in the inner tracker appears in the first layer and is approximately 26.8~kHz/cm$^{2}$ (as discussed in Sec.~\ref{sec:mdi_bkg} at the peak luminosity. This is well within the rate capability of the uRWELL detector technology. For example, it has been demonstrated that a $\mu$RWELL detector with 400~$\mu$m pitch strips has a rate capability ranging from a few tens of
kHz/cm$^2$ up to a few MHz/cm$^2$~\cite{itk10}.

In the $\mu$RWELL-ITK conceptual design,, two-dimensional readout strips ($X$ and $V$) are used with a crossing angle of 15 degrees, as shown in Fig.~\ref{fig:muRWELL_XV}. The $X$ readout strip is along the beamline direction. Both the X and V strips have a pitch of 400~$\mu$m in all 3 layers of the $\mu$RWELL film and cover a polar angle acceptance of $20^{\circ}$ $<$ $\theta$ $<$ $160^{\circ}$. Considering that the radii of the 3 $\mu$RWELL layers are 60~mm, 110~mm and 160~mm, the numbers of readout channels are 1919, 3517, and 5116, respectively, with 10552 in total. 
With this readout strips layout, each hit generates 10 signals in the X strips and 10 signals in the V strips on average, resulting in a highest count rate per channel as 367~kHz. This background level corresponds to an occupancy of 11.2\%, 5.4\%, and 5.32\% with a time window of 400~ns for the three layers of inner tracker, respectively. Obviously, it is necessary to optimize the detector design, particularly the layout of readout strips  to reduce the occupancy to a acceptable level. Splitting the X and V strips at Z = 0 into two parts would be one of the effective ways to decrease the occupancy. The average number of hits produced by the passage of a charged particle on the detector can be reduced by optimizing the gas gap width and the working gas. This would be another way to reduced the occupancy. 
%%%%%%%%%%%%%%%%%%% Fig %%%%%%%%%%%%%%%%%%%%%%%%%%
\begin{figure*}[htb]
    \centering
{
        \includegraphics[width=0.4\textwidth]{Figures/Figs_04_00_DetectorSubSystems/Figs_04_01_InnerTracker/muRWELL_XVstrips.jpg}
}
\caption{The $X/V$ readout strips of the $\mu$RWELL detector.
}
    \label{fig:muRWELL_XV}
\end{figure*}
%%%%%%%%%%%%%%%%%%%%%%%%%%%%%%%%%%%%%%%%%%%%%%%%%%

 



%% muRWell electronics writen by Prof. Lei Zhao

\paragraph{Readout Electronics}
\quad\\
The structure of the readout electronics of the $\mu$RWELL-based ITK detector is illustrated in Fig.~\ref{fig:4.1.01_electronic}. The front-end electronics are linked to detectors through high-density connectors, and a protection circuit is added at the input to protect the readout electronics from unexpected high-voltage discharge of the detector. Due to the small amplitude of the signal from the detector, the front-end readout electronics are placed close to the detector, and a high-density design is required for the front-end electronics. The front-end electronics are set near the endcaps of the inner tracker. The connector for $\mu$RWELL-based ITK is Hirose connector FH26W-71S-0.3SHW(60). It has a width of 23 mm and 0.3 mm channel pitch. The X and V strips of $\mu$RWELL are designed on two films, so that their readout electronics can be arranged in two complete circles. In this case, all the readout channels can arranged well by the FPCB connector. Additionally, high-precision signal measurements are required. For the reasons above, it is planned to design an application-specific integrated circuit (ASIC) chip that integrates front-end analog circuits, analog-to-digital conversion (ADC), and a charge $\&$ time calculation circuit within the chip. In addition, the calibration circuit is added to correct the mismatch among channels.
%%%%%%%%%%%%%%%%%%% Fig %%%%%%%%%%%%%%%%%%%%%%%%%%
\begin{figure*}[htb]
    \centering
{
        \includegraphics[width=160mm]{Figures/Figs_04_00_DetectorSubSystems/Figs_04_01_InnerTracker/readout_electronics_block_diagram.png}
}
\vspace{0cm}
\caption{Block diagram of the readout electronics for the $\mu$RWELL-based ITK detector.}
    \label{fig:4.1.01_electronic}
\end{figure*}
%%%%%%%%%%%%%%%%%%%%%%%%%%%%%%%%%%%%%%%%%%%%%%%%%%

The output data of the front-end ASICs are transferred to a digital ASIC or field-programmable gate array (FPGA) for data packaging and finally transferred to the DAQ through high-speed serial data interfaces. Since the ITK readout electronics need to be synchronized with the global clock, the FPGA is also responsible for fanning out the clock to each front-end ASIC.

ITK electronics do not take part in generating the global trigger signal and only receive the trigger signal for trigger matching. The data from the front-end ASICs are first stored in RAMs, and when the FPGA receives the global trigger signal, it picks out the valid data through trigger match logic and transfers the data to the DAQ.

The hardware system is composed of front-end electronics ~(FEE), readout units ~(RUs), and clock, trigger submodules. Multiple front-end ASICs, which complete analog signal processing, A/D conversion, and charge $\&$ time calculation, are integrated into one FEE module. The output data of these ASICs are transferred to the RUs through a high-speed serial data interface.

 Both the charge/amplitude and time information of the hits are necessary for track reconstruction. The recorded data of each hit signal in one channel represent a 96-bit word, including 8 bits for the header, 16 bits for the trigger number, 34 bits for the timing information, 6 bits for the amplitude, 16 bits for the FEE number, 8 bits for the check and 8 bits for the tail. As a consequence, the total data stream sizes are approximately 6.89~GB/s for all 3 layers.

\paragraph{Front-End ASIC}
\quad\\
The high channel density of the $\mu$RWELL-ITK requires customized ASICs for its front-end electronics that integrate functions of analog signal processing, ADC and charge \& time calculation. As shown in Fig.~\ref{fig:4.1.02_electronic}, each channel comprises a charge sensitive amplifier (CSA) for low noise amplification, a semi-Gaussian shaper network, a discriminator for generating a self-trigger signal, a switched capacitor array (SCA) for waveform sampling, a Wilkinson ADC for digitization, a digital circuit for charge \& time calculation, and a high-speed data transfer interface.

Due to the high event rate, full waveform data transfer would place high pressure on the data transfer interface and corresponding power consumption. Therefore, it is preferable to integrate the charge \& time calculation circuit into the chip.

%%%%%%%%%%%%%%%%%%% Fig %%%%%%%%%%%%%%%%%%%%%%%%%%
\begin{figure*}[htb]
    \centering
{
        \includegraphics[width=160mm]{Figures/Figs_04_00_DetectorSubSystems/Figs_04_01_InnerTracker/frontend_ASIC_block_diagram.png}
}
\vspace{0cm}
\caption{Block diagram of the front-end ASIC.}
    \label{fig:4.1.02_electronic}
\end{figure*}
%%%%%%%%%%%%%%%%%%%%%%%%%%%%%%%%%%%%%%%%%%%%%%%%%%


To further enhance the signal-to-noise ratio~(SNR) and to improve the accuracy of charge and time measurements, a digital deconvolution and filtering circuit is also integrated into the ASIC. The exponential signal is first unfolded into the unit impulse, and then the trapezoidal output pulse is processed by the moving average method (corresponding to the low-pass filter in Fig.~\ref{fig:4.1.03_electronic}) to filter out the high-frequency noise. The specific parameters of the deconvolution and moving average circuit need to be optimized according to the shaping time and knee frequency of the signal to achieve the best filtering result.

%%%%%%%%%%%%%%%%%%% Fig %%%%%%%%%%%%%%%%%%%%%%%%%%
\begin{figure*}[htb]
    \centering
{
        \includegraphics[width=160mm]{Figures/Figs_04_00_DetectorSubSystems/Figs_04_01_InnerTracker/circuit_block_diagram.png}
}
\vspace{0cm}
\caption{Block diagram of the digital deconvolution and low-pass filter circuit.}
    \label{fig:4.1.03_electronic}
\end{figure*}
%%%%%%%%%%%%%%%%%%%%%%%%%%%%%%%%%%%%%%%%%%%%%%%%%%


\subsubsection{R\&D on Cylindrical $\mu$RWELL}
%\quad\\
Several kinds of structure supporting materials have been tested, of which aramid honeycomb~\cite{itk13} and Rohacell foam~\cite{itk14} showed good performance in terms of both material budget and mechanical strength. Additionally, two bonding methods have been developed for the aramid honeycomb and Rohacell foam. A solid sandwich structure with a very low material budget of adhesive is feasible. Fig.~\ref{fig:4.1.06} shows a sandwich-like inner cylinder  made of the two kinds of material, and Table~\ref{tab:4.1.03} presents the performance of these cylinder . The mechanical strength of 2~mm aramid honeycomb is sufficient, while that of 1~mm Rohacell foam is lower. Thus, in the next step, thinner aramid honeycomb and thicker Rohacell foam will be manufactured and tested. Additionally, the study of a new X/V strip readout is still ongoing.

%%%%%%%%%%%%%%%%%%% Fig %%%%%%%%%%%%%%%%%%%%%%%%%%
\begin{figure*}[htb]
    \centering
{
        \includegraphics[height=30mm]{Figures/Figs_04_00_DetectorSubSystems/Figs_04_01_InnerTracker/fig07_rohacellinnertube.png}
}
\hspace{5 mm}
{
        \includegraphics[height=30mm]{Figures/Figs_04_00_DetectorSubSystems/Figs_04_01_InnerTracker/fig07_aramidinnertube.png}
}
\vspace{0cm}
\caption{The Rohacell foam-based inner cylinder  (left) and aramid honeycomb-based detector model (right) produced in the preresearch stage.}
    \label{fig:4.1.06}
\end{figure*}
%%%%%%%%%%%%%%%%%%%%%%%%%%%%%%%%%%%%%%%%%%%%%%%%%%

%%%%%%%%%%%%%%%%%  TABLE  %%%%%%%%%%%%%%%%%%%%%%%%
\begin{table*}[htb]
\small
    \caption{The material budgets of the inner cylinders  manufactured in the preresearch stage.}
    \label{tab:4.1.03}
    \vspace{0pt}
    \centering
    \begin{tabular}{lllllll}
        \hline
        \thead[l]{ } & \thead[l]{Inner PI film}& \thead[l]{Inner adhesive} &\thead[l]{Structure support } & \thead[l]{Outer adhesive}& \thead[l]{Outer PI film} & \thead[l]{Total}\\
        & & &material & & & \\
        \hline
        Honeycomb-based	&0.028\%	&0.009\%	&0.033\%	&0.009\%	&0.030\%	&0.105\% \\
        Rohacell-based	&0.028\%	&0.009\%	&0.010\%	&0.008\%	&0.029\%	&0.084\% \\
        \hline
    \end{tabular}
\end{table*}
%%%%%%%%%%%%%%%%%%%%%%%%%%%%%%%%%%%%%%%%%%%%%%%%%%



%% Silicon


\subsection{MAPS-based Inner Tracker}
The MAPS-based ITK is composed of three layers of silicon pixel detectors~(PXDs) and is located inside the MDC (see Sec.~\ref{sec:mdc}) at radii of 36~mm, 98~mm and 160~mm.
The radii of the two inner layers are smaller than those of the design of the $\mu$RWELL-based ITK to take into account the higher rate-capabilities of PXDs and to achieve better ITK spatial resolution.
A pixel size of $100~\mu\mathrm{m} \times 250~\mu\mathrm{m}$ is sufficient to meet the spatial resolution requirement of the STCF ITK.
To reduce the multiscattering effect for charged particles with low momentum, especially for $p < 200$~MeV/c, it is crucial to reduce the material budget as much as possible.
For the baseline design of the STCF PXDs, a radiation length of 0.25\% $X_0$ per layer is assumed, including the material budget from the sensor, readout electronics and supporting material.

The monolithic active pixel sensor (MAPS) technology has the potential to satisfy the low-material and high-rate requirements for the STCF inner tracker. 
This technology has an attractive advantage of having both the sensor and readout electronics in the same pixel, thus reducing the material budget,
and it has been developing rapidly in the particle physics community.
The first-generation MAPS-based vertex detector for the STAR upgrade successfully completed a 3-year physics run~\cite{starmaps1,starmaps2},
The new generation complementary metal-oxide-semiconductor (CMOS) pixel sensor (CPS) for the ALICE-ITS upgrade~\cite{aliceits,alpide} is in mass production.
The CMOS MAPS sensor is chosen as the pixel sensor technology for the silicon-based ITK. The ITK is called MAPS-based ITK in this case. 

The high luminosity of the STCF places additional stringent requirements on the design of the ITK detector, and the challenges include the high hit rate and pileup effects.
From Table~\ref{tab:TIDNIEL_max}, the highest expected hit rate is approximately $1.04\times 10^6$~Hz/cm$^{2}$ at the innermost layer of the MAPS-based~($\mu$RWELL-based) ITK.
The state of the art MAPS technology can easily handle such a hit rate . For example, The STAR ULTIMATE MAPS can cope with a hit rate of
approximately 1 MHz/cm$^{2}$s$^{-1}$~\cite{starmaps1}, and the ALICE-ITS ALPIDE sensor can
operate with a hit rate of 3~MHz/cm$^{2}$~\cite{aliceits}




\subsubsection{Expected Tracking Performance}
%{Performance of the Silicon-based Inner Tracker}
Figure~\ref{fig:4.1.09.z} shows the expected performance of the momentum resolution and position resolution of the tracking system, comparing the two configurations MDC only and MDC + PXD with different settings for the radius of the PXD layers.
A single hit position resolution of 30~$\mu$~$\times$~75~$\mu$m is assumed for the PXD.
The results are obtained from simulation with combined track fitting of the MDC + PXD tracking system, and incident particles are assumed to have a polar angle of cos$\theta=0$.
With MDC + PXD, the momentum resolution is improved by a factor of approximately 1.5 at 1.8~GeV/c compared with MDC tracking only.

As expected, while the impact parameter resolution can be improved when the innermost layer is closer to the beam pipe, the resolution of the momentum and transverse momentum has little dependence on the radius of the innermost layer.


%%%%%%%%%%%%%%%%%%% Fig %%%%%%%%%%%%%%%%%%%%%%%%%%
\begin{figure*}[htb]
	\centering
\subfloat[][]{\includegraphics[height=50 mm]{Figures/Figs_04_00_DetectorSubSystems/Figs_04_02_MainDriftChamber/Si_pos_D0_0p0.png}}
\subfloat[][]{\includegraphics[height=50 mm]{Figures/Figs_04_00_DetectorSubSystems/Figs_04_02_MainDriftChamber/Si_pos_Z0_0p0.png}} \\
\subfloat[][]{\includegraphics[height=50 mm]{Figures/Figs_04_00_DetectorSubSystems/Figs_04_02_MainDriftChamber/Si_pos_p0p0.png}}
\subfloat[][]{\includegraphics[height=50 mm]{Figures/Figs_04_00_DetectorSubSystems/Figs_04_02_MainDriftChamber/Si_pos_pt0p0.png}}
\vspace{0cm}
\caption{The simulated resolution of the impact parameters (a) $d_0$ and (b) $z_0$ and (c) the momentum $p$ and (d) transverse momentum $p_T$ as a function of the $p_T$ of the incident particle. The results with different layout configurations, the default with radii of 36 mm, 98 mm and 160 mm and alternative radii of 60 mm, 110 mm and 160 mm, are compared. }
    \label{fig:4.1.09.z}
\end{figure*}
%%%%%%%%%%%%%%%%%%%%%%%%%%%%%%%%%%%%%%%%%%%%%%%%%%


To investigate the impact of the material budget on the performance of the ITK, the expected momentum resolution and position resolution with different material budgets, ranging from 0.25\%$X_0$ to 1.0\%$X_0$, are compared.
The results from the Geant4 simulation are shown in Fig.~\ref{fig:4.1.10.z}.
In the low momentum range, a degradation of the momentum resolution with a higher material budget is seen.
%Another important factor that needs to be considered is the track finding and reconstruction efficiency.
Further investigation is needed to understand the impact of the material budget on the track finding and reconstruction efficiency, which have a significant impact on the physics potential in the low momentum range at the STCF.


%%%%%%%%%%%%%%%%%%% Fig %%%%%%%%%%%%%%%%%%%%%%%%%%
\begin{figure*}[htb]
	\centering
\subfloat[][]{\includegraphics[height=50 mm]{Figures/Figs_04_00_DetectorSubSystems/Figs_04_02_MainDriftChamber/Si_mat_D0_0p0.png}}
\subfloat[][]{\includegraphics[height=50 mm]{Figures/Figs_04_00_DetectorSubSystems/Figs_04_02_MainDriftChamber/Si_mat_Z0_0p0.png}} \\
\subfloat[][]{\includegraphics[height=50 mm]{Figures/Figs_04_00_DetectorSubSystems/Figs_04_02_MainDriftChamber/Si_mat_p0p0.png}}
\subfloat[][]{\includegraphics[height=50 mm]{Figures/Figs_04_00_DetectorSubSystems/Figs_04_02_MainDriftChamber/Si_mat_pt0p0.png}}
\vspace{0cm}
\caption{The simulated resolution of the impact parameters (a) $d_0$ and (b) $z_0$ and (c) the momentum $p$ and (d) transverse momentum $p_T$ as a function of the $p_T$ of the incident particle. The results with different material budgets, expressed in terms of the radiation length, are compared.}
    \label{fig:4.1.10.z}
\end{figure*}
%%%%%%%%%%%%%%%%%%%%%%%%%%%%%%%%%%%%%%%%%%%%%%%%%%

\subsubsection{CMOS MAPS for the Inner Tracker}
%{Prospect of MAPS for PXD}
A good starting point for the STCF ITK is the ALPIDE design, which was developed for the aforementioned ALICE-ITS upgrade.
The ALICE-ITS has achieved a material budget of approximately 0.3\%$X_0$ with a sensor thickness of 50~$\mu$m.
CMOS pixel sensors, for instance, JadePix~\cite{jadepix}, are also being proposed as the vertex detector for the conceptual design of the Circular Electron Positron Collider (CEPC)~\cite{cepc}. JadePix-1 features a small pixel size ($16\times 16~\mu\mathrm{m}^2$), with the main goal being achieving low power consumption and material budget. For the STCF, the pixel size requirement can be relaxed, and the main challenges are the low power consumption and fast readout necessary to cope with the high event rate (see Sec.~\ref{sec:tdaq}).
In contrast to an ordinary MAPS, which collects ionization charge mainly by diffusion, high-voltage~(HV)/high-resistivity~(HR)-CMOS designs collect ionization charge mainly via drift, as the sensor can be fully depleted.
The fast collection of signal charges and the low noise HV/HR-CMOSs also allow them to be more radiation tolerant.
HV-MAPSs have been prototyped for several experiments, such Mu3e~\cite{mu3e}, ATLAS~\cite{atlaspix} and CLIC~\cite{clicpix}.
Another promising option for the STCF ITK is the MuPix sensor~\cite{mupix} to be used for the Mu3e experiment, which was designed to detect extremely low-momentum tracks ($<50$~MeV/c) with very high tracking efficiency and momentum resolution.

\subsubsection{Readout Circuitry}
The STCF CMOS pixel sensor is expected to be a $2\times2$~cm$^2$ chip that contains 16k pixels of $100\times 250~\mu\mathrm{m}^{2}$ size.
The readout circuitry provides time stamping and charge measurement based on the time-over-threshold (TOT).
The structure of the readout circuitry is shown in Fig.~\ref{fig:pix_readout}.
\begin{figure*}[htb]
	\centering
    \includegraphics[width=0.7\linewidth]{Figures/Figs_04_00_DetectorSubSystems/Figs_04_01_InnerTracker/pixel_readout.pdf}
\vspace{0cm}
\caption{Block diagram of the pixel sensor readout circuit.}
    \label{fig:pix_readout}
\end{figure*}

The readout circuitry mainly consists of an in-pixel part and a periphery part.
There is a low-power charge-sensitive amplifier (CSA) and a voltage comparator in each pixel.
The CSA integrates the charge collected from the sensor and outputs a voltage signal to the following comparator,
and the voltage threshold of the comparator can be tuned by a local DAC.
All the pulse signals from the comparators are driven to the periphery, and the arrival time of each pulse is recorded as a timestamp.
Meanwhile, the pulse width is measured based on the TOT method.
The time stamps and TOT messages are read out through the readout control block to an 8/10-bit encoder
and, finally, to a serializer with an output data rate of several Gbits/s.
Additionally, the configuration block, voltage-controlled oscillator (VCO), and phase-locked loop (PLL) are integrated in the periphery.



\subsection{Pileup and Radiation Effects}
% pileup
\subsubsection{Pileup Effects}

The STCF detector is expected to operate at an event rate as high as 400 kHz (Sec.~\ref{sec:tdaq}). With such a high event rate, the probability of events overlapping is approximately 8 (18)\% within a time window of 200 (500) ns. This indicates that fast response or good timing performance is required for detectors to cope with such a high event rate. Recent studies ~\cite{aplide-improved,malta2} have shown that depleted MAPS (DMPAS) with improved sensor designs and processes could provide a timing resolution of about 2~ns or better. Such excellent timing capability would be very helpful in solving the overlapping events by providing each charged track with precise time measurement. 

% radition and aging effects
\subsubsection{Radiation Effects}
As the innermost subdetector of the STCF detector, the ITK receives the highest level of background radiation.  Radiation hardness is a crucial factor in the design of the ITK.
A detailed background radiation simulation is described in Sec.~\ref{sec:mdi_bkg}, and the expected radiation levels in individual subdetectors are given in Table~\ref{tab:TIDNIEL_mean} and \ref{tab:TIDNIEL_max}.
However, note that the radiation simulation in Sec.~\ref{sec:mdi_bkg} 
 is subject to large uncertainties since many technical details for the designs of the accelerator and the machine detector interface are not available yet at the conceptual design stage. In addition, beam background usually can't be simulated precisely given too many factors in the simulation that can't be well determined. For example, significant discrepancies between simulation and actual measurement of beam background have been observed in the Belle II experiment~\cite{belle2bkg}. Therefore, sufficient safety margins have to be taken into account when assessing beam background levels. 

 From Table~\ref{tab:TIDNIEL_max}, the TID and NIEL of the uRWELL-based ITK are 120 Gy/y and $1.1\times 10^{10}$~n/cm$^{2}$/y~(1 MeV neutron equivalent), respectively. Those of the MAPS-based ITK are 3.5~kGy/y and a NIEL of $1.8\times 10^{11}$~n/cm$^{2}$/y, respectively. These levels of radiation are well below the limits of the state of the art MPGD and MAPS detectors. For example, The ALPIDE pixel sensor for the ALICE-ITS has been demonstrated to have a radiation hardness of up to $10^{13}$~n/cm$^{2}$ (NIEL) and 27~kGy~\cite{alpide2}.

For the $\mu$RWELL-based ITK, potential degradation of performance due to aging effects, such as a
specific gas gain reduction or high voltage instabilities during operation caused by irradiation, should be considered.
In general, aging is induced by plasma-chemical processes during gas amplification processes.
A complete overview and description of the aging phenomena in gaseous detectors can be found in Ref.~\cite{gaseousaging}.
A dedicated aging study is planned for the $\mu$RWELL-based ITK. During the long-term operation of the STCF, it can be replaced if it malfunctions due to radiation damage.

\subsection{Conclusion and Outlook}
The cylindrical $\mu$RWELL-based detector, with the advantages of low material budget, robustness, scalability, and simplified manufacturing and maintenance processes, is the baseline design of the STCF inner tracker and is expected to provide promising performance. Additional studies are needed in the future to realize the baseline design and to further optimize the detector performance, including material budget limitation, optimization of the cylindrical $\mu$RWELL detector structure, manufacturing of the detector prototype and performance testing. As an alternative design choice, CMOS silicon pixel detectors are being considered. They are expected to provide better vertex resolution and greater radiation hardness. It will be crucial to continue the pixel sensor R\&D program and to develop CMOS pixel sensors with radiation tolerance, lower power consumption and fast readout electronics.
The expected performance of the different ITK designs are discussed in Sec.~\ref{sec:mdc}, taking into account both the inner tracker and outer tracker.
\quad\\
%\input{Chapters/Chapter_04_00_DetectorSubSystems/Chapter_04_01_InnerTracker/04_01_Reference}
