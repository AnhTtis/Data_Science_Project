\subsubsection{Simulation Study}
\label{section4}

Monte Carlo (MC) simulation studies are performed to understand the detector physics of the PID technological candidates, and estimate the PID performance achievable. According to the three candidates of technique approaches, namely RICH, FTOF and DIRC, we apply different MC studies to each technique. It's noted that the simulation works of FTOF and DIRC are very similar so they are largely merged to save computation resources. Furthermore, DIRC or FTOF detectors are only simulated as PID detectors in the endcap region.

\paragraph{Timing Resolution of DIRC Detector}
The main sources contributing to the timing uncertainty of DIRC-like detector are as follows:
\begin{equation}
\sigma^{2}_{tot} \sim \sigma^2_{trk} + \sigma^{2}_{T_{0}} + (\frac{\sigma_{elec}}{\sqrt{N_{p.e.}}})^{2} + (\frac{TTS}{\sqrt{N_{p.e.}}})^{2} + (\frac{\sigma_{det}}{\sqrt{N_{p.e.}}})^{2}.
\label{EQ:DIRC-Time-Reso}
\end{equation}
In the equation above $N_{p.e.}$ is the number of photoelectrons, $\sigma_{trk}$ is the error caused by track reconstruction, $\sigma_{T_{0}}$ is the event reference time ($T_{0}$, i.e. when physical collision happens) error mainly affected by the collider design of STCF, $\sigma_{elec}$ is the electronic timing accuracy, $TTS$ is the single-photon transit time spread of MCP-PMT, and $\sigma_{det}$ is the time reconstruction uncertainty of the DIRC detector. From this formula, we can see that the contribution from $\sigma_{elec}$, $TTS$ and $\sigma_{det}$ will decrease with increasing $N_{p.e.}$, while the timing errors from $\sigma_{trk}$ and $\sigma_{T_{0}}$ keep unchanged. Their relative importance should be studied and optimization can be achieved considering all their contribution. It¡¯s noted that the uncertainty of $T_{0}$ is usually about $30 \sim 40 ps$, which is an important timing error source at STCF.

\paragraph{Reconstruction of $\theta_{c}$ and TOF}
As shown in figure~\ref{Fig4:DIRC-Cordinate1}, a three-dimensional coordinate system is established for the DIRC detector. The detector in the plot is a trapezoid-shaped plate, however the accurate shape is not very important since the calculation below is rather general. The photon propagation direction in the radiator (usually fused silica) plate can be decomposed into two angles relative to the Y axis in the X-Y plane and the Z-Y plane. In addition, H is the equivalent length of photon propagation distance along Y direction.
\begin{figure*}[htbp]
 \centering
 \mbox{
   \begin{overpic} [width=0.6\textwidth,height=0.45\textwidth] {Figures/Figs_04_00_DetectorSubSystems/Figs_04_03_ParticleIdentification/DIRC-Cordinate1.jpg}
   \end{overpic}
 }
\caption{The coordinate system of the DIRC detector, and the decomposition of the photon propagation direction into two angles.}
\label{Fig4:DIRC-Cordinate1}
\end{figure*}

As demonstrated in figure~\ref{Fig4:DIRC-Cordinate2}, the photon propagation length inside the radiator plate is
\begin{equation}\label{EQ:LOP}
L = \sqrt{ \frac{H^{2}}{\cos^{2}(\theta_{x})} + \frac{H^{2}}{\cos^{2}(\theta_{z})} - H^{2}}.
\end{equation}

The Cherenkov angle $\theta_{c}$ can be determined by
\begin{equation}\label{EQ:Chc1}
\cos(\theta_{c}) = \frac{1}{n_{p}\beta} = \vec{v_{t}} \dot \vec{v_{p}},
\end{equation}
where $\vec{v_{t}}$ is the direction of motion of incident charged particle, and $\vec{v_{p}} = (\tan\theta_{x}, 1, \tan\theta_{z})$ is the initial propagation direction of Cherenkov photon. Through the above formula, the phase refractive index in fused silica can be calculated, and the influence of chromatic dispersion on time measurement can be corrected.

\begin{figure*}[htbp]
 \centering
 \mbox{
   \begin{overpic} [width=0.6\textwidth,height=0.45\textwidth] {Figures/Figs_04_00_DetectorSubSystems/Figs_04_03_ParticleIdentification/DIRC-Cordinate2.jpg}
   \end{overpic}
 }
\caption{The dependence of photon propagation length on $\theta_{x}$, $\theta_{z}$ and $H$.}
\label{Fig4:DIRC-Cordinate2}
\end{figure*}

Then the flight time of charged particle from the collision point to the DIRC detector is
\begin{equation}\label{EQ:TOF}
TOF = T-TOP-T_{0} = T-\frac{Ln_{g}}{c}-T_{0},
\end{equation}
where $T$ is the detection time measured by MCP-PMT, $TOP = \frac{Ln_{g}}{c}$ is the time of propagation of photon in the radiator plate, $c$ is the speed of light in vacuum and $T_{0}$ is the generation time of charged particle. The group refractive index in fused silica $n_{g}$ at wavelength $\lambda$ is related to the phase refractive index $n_{p}$ through
\begin{equation}\label{EQ:RefIndex}
n_{g} = \frac{n_{p}}{1+\frac{\lambda}{n_{p}}\frac{dn_{p}}{d\lambda}}.
\end{equation}
Thus it can be seen that the Cherenkov angle and the time of flight of charged particle can be calculated by reconstructing $\theta_{x}$, $\theta_{z}$ and $H$.

\paragraph{TOP Errors}

In the following we focus on $\sigma_{det}$, i.e. the uncertainty of $TOP$. Since $TOP$ is determined by $\frac{Ln_{g}}{c}$, consequently $\sigma_{det}$ is affected by the error in reconstructing $L$ and $n_{g}$.

{\bf $L$ Contribution}
Following equation ~\ref{EQ:LOP}, the error of $L$ is expressed by
\begin{equation}\label{EQ:LOPerror}
\Delta L = \sqrt{(\frac{\partial L}{\partial H})^{2}\Delta H^{2} + \frac{\partial L}{\partial\theta_{x}})^{2}\Delta\theta_{x}^{2} + \frac{\partial L}{\partial\theta_{z}})^{2}\Delta\theta_{z}^{2}}.
\end{equation}
The partial derivative coefficients in the above formula are as follows:
\begin{equation}
\frac{\partial L}{\partial H}) = \sqrt{ \frac{1}{\cos^{2}(\theta_{x})} + \frac{1}{\cos^{2}(\theta_{z})} -1}
\end{equation}
\begin{equation}
\frac{\partial L}{\partial\Delta_{x}}) =\frac{H}{\sqrt{ \frac{1}{\cos^{2}(\theta_{x})} + \frac{1}{\cos^{2}(\theta_{z})} -1}} \frac{\sin\theta_{x}}{\cos^{3}\theta_{x}}
\end{equation}
\begin{equation}
\frac{\partial L}{\partial\Delta_{z}}) =\frac{H}{\sqrt{ \frac{1}{\cos^{2}(\theta_{x})} + \frac{1}{\cos^{2}(\theta_{z})} -1}} \frac{\sin\theta_{z}}{\cos^{3}\theta_{z}}
\end{equation}

Let us consider a DIRC structure with a cylindrical mirror that focuses Cherenkov light in the Z-Y plane. Consequently, in the direction of $\theta_{z}$, the beam of parallel light of same $\theta_{z}$ is focused to the detection plane with same z position. According to the principle of equal optical path, the maximum difference of optical path for parallel light emitted at different positions is $2W\sin\theta_{z}$, where $W$ is the thickness of radiator plate. And the equivalent height difference is $W\sin2\theta_{z}$, as illustrated in figure ~\ref{Fig4:HErr1}. Considering that the photon emission position is evenly distributed, we have
\begin{equation}
\Delta H = \frac{W\sin2\theta_{z}}{\sqrt{12}}.
\end{equation}
With a radiator plate thickness of $1.5 cm$, $\Delta H \sim 4 mm$, varying slightly within the $\theta_{z}$ range practically permitted.

\begin{figure*}[htbp]
 \centering
 \mbox{
   \begin{overpic} [width=0.6\textwidth,height=0.45\textwidth] {Figures/Figs_04_00_DetectorSubSystems/Figs_04_03_ParticleIdentification/HErr1.jpg}
   \end{overpic}
 }
\caption{The estimation of $\Delta H$ by geometric relation.}
\label{Fig4:HErr1}
\end{figure*}

In addition, $\theta_{x}$ is also reconstructed geometrically, as shown in figure ~\ref{Fig4:ThetaxErr1}, determined by the size of photon sensor long $X$ direction, $W_{0}$. Since $\theta_{x}$ can be considered distributed uniformly,
\begin{equation}
\Delta\theta_{x} = \frac{1}{\sqrt{12}}\frac{W_{0}\sin^{2}\theta_{x}}{H}.
\end{equation}

\begin{figure*}[htbp]
 \centering
 \mbox{
   \begin{overpic} [width=0.6\textwidth,height=0.45\textwidth] {Figures/Figs_04_00_DetectorSubSystems/Figs_04_03_ParticleIdentification/ThetaxErr1.jpg}
   \end{overpic}
 }
\caption{The estimation of $\Delta\theta_{x}$ by geometric relation.}
\label{Fig4:ThetaxErr1}
\end{figure*}

$\theta_{z}$ is obtained directly from the hit position along $Z$ direction on the detection plane, according to the design of cylindrical focusing optics. The error is mainly determined by the optical focusing performance and the size of sensor unit (along $Z$). With a sensor size $W_{1}$ of $0.7 mm$ and a focal length of $20 cm$, $\Delta\theta_{z} \sim 1 mrad$.


{\bf $n_{g}$ Contribution}

In order to suppress or correct the chromatic dispersion effect, the refractive index n needs to be accurately reconstructed. There are two different definitions of refractive index of light in fused silica plates: phase refractive index $n_{p}$ and group refractive index $n_{g}$, which are correlated by \ref{EQ:RefIndex}.

For high quality fused silica radiator, the phase refractive index $n_{p}$ can usually be calculated using a three-term Sellmeier equation,
\begin{equation}
n^{2}_{p}-1 = \frac{A_{1}\lambda^{2}}{\lambda^{2}-B_{1}} + \frac{A_{2}\lambda^{2}}{\lambda^{2}-B_{2}} + \frac{A_{3}\lambda^{2}}{\lambda^{2}-B_{3}} ,
\end{equation}
for $\lambda$ in $\mu m$. A set of parameters for Corning HPFS7980 are listed in table~\ref{TAB:Np1}.
\begin{table}[h]
  \centering
  \begin{tabular}{|l|c|}\hline
    Sellmeier constant&\\\hline
    $A_{1}$&0.68374049400\\
    $B_{1}$&0.00460352869\\
    $A_{2}$&0.42032361300\\
    $B_{2}$&0.01339688560\\
    $A_{3}$&0.58502748000\\
    $B_{3}$&64.4932732000\\\hline
  \end{tabular}
  \caption{Sellmeier constants for Corning HPFS7980, taken from [ ].}
  \label{TAB:Np1}
\end{table}

To calculate $n_{g}$ with equation \ref{EQ:RefIndex}, $n_{p}$ should be first reconstructed through the measured incident particle momentum and initial direction of Cherenkov photon via equation ~\ref{EQ:Chc1}. The error of reconstructed refractive index $n_{p}$ is mainly determined by the angular accuracy of reconstructed Cherenkov photon, which mainly depends on the optical focusing design and the sensor size, and the accuracy of the incident track direction, which is mainly affected by the multiple coulomb scattering (MCS) of charged particles inside the radiator plate. The scattering angle of charged particles in the medium due to MCS is \cite{PDG:MCS1}
\begin{equation}
\Delta\theta_{plane} = \frac{13.6MeV}{\beta cp}Z\sqrt{\frac{x}{X_{0}}[1+0.038ln(\frac{x}{X_{0}})]} ,
\end{equation}
where $p$ is the particle momentum, $z$ is its charge number, $x$ the path in the medium and $X_{0}$ the radiation length of the medium.

The angular uncertainty introduced by MCS is given by
\begin{equation}
\Delta\theta_MCS = \frac{1}{\sqrt{3}}\Delta\theta_plane.
\end{equation}
For $\pi$ meson at $2 GeV/c$, $\Delta\theta_MCS \sim 1.3 mrad$ in the $1.5 cm$ thick fused silica. As previously mention, $\theta_{z} \sim 1 mrad$ if the sensor size is $0.7 mm$ in $Z$ direction, whereas $\theta_{x}$ is less than $5 mrad$ in most cases.


{\bf Combined Errors}

Based on the two error sources discussed above, the time resolution of TOP is calculated. Figure ~\ref{Fig4:TOPErr1} shows the result of single Cherenkov photon in different $\theta_{x}$ directions for $2 GeV/c$ $pi$ mesons. $H$ is fixed at $0.8 m$. It can be seen from the figure that the time resolution gets worse for larger $\theta_{x}$ and/or larger Cherenkov angle $\theta_{c}$.  The average time resolution is about $50 ps$. Figure ~\ref{Fig4:TOPErr2} shows the time resolution of $\pi$ meson at different momentum, with $H = 0.8 m$ and $\theta_{x} = 0 rad$. The timing error varies around $30 - 70 ps$ except for low momentum particles, where the time resolution is dominated by the MCS effect.

\begin{figure*}[htbp]
 \centering
 \mbox{
   \begin{overpic} [width=0.45\textwidth,height=0.35\textwidth] {Figures/Figs_04_00_DetectorSubSystems/Figs_04_03_ParticleIdentification/TOPErr1.jpg}\label{Fig4:TOPErr1}
   \end{overpic}
   \begin{overpic} [width=0.45\textwidth,height=0.35\textwidth] {Figures/Figs_04_00_DetectorSubSystems/Figs_04_03_ParticleIdentification/TOPErr2.jpg}\label{Fig4:TOPErr2}
   \end{overpic}
 }
\caption{The expected TOP errors.}
\label{Fig4:TOPErr}
\end{figure*}

{\bf PID Capability by TOF}

From these calculations, we can see that for particle momentum of $2 GeV/c$, $\sigma_{det}$ for single photon is approximately $50 ps$. If we preliminarily estimate that $\sigma_{trk}$ and $\sigma_{T_{0}}$ are combined to approximately $40 ps$, $\sigma_{ele}$ and $\sigma_{TTS}$ are approximately $50 ps$, we have $\sigma_{tot} \sim 46 ps$ for 10 photo-electrons. Shown in figure ~\ref{Fig4:DTOFPiK} are the TOF differences between $\pi$ and $K$ mesons at different momenta and flight path lengths. For a flight path length of $1.3 m$, $\Delta TOF$ is $120 ps$, and the separation between these two particles is $2.6 \sigma$ at $2 GeV/c$.

\begin{figure*}[htbp]
 \centering
 \mbox{
   \begin{overpic} [width=0.6\textwidth,height=0.45\textwidth] {Figures/Figs_04_00_DetectorSubSystems/Figs_04_03_ParticleIdentification/DTOF1.jpg}
   \end{overpic}
 }
\caption{The expected TOP errors.}
\label{Fig4:DTOFPiK}
\end{figure*}

\paragraph{DIRC Based PID}

The proposed DIRC (or DIRC-like, e.g. FTOF) detector is located at the endcap region of STCF experiment apparatus. Its distance from the collision vertex is $\sim 1.3 m$. The inner and outer radius of the detector are $0.5 m$ and $\sim 1 m$ respectively, providing coverage within $21 \circ - 35 \circ$. Figure ~\ref{Fig4:DIRC_ENDCAP1} demonstrates the schematic structure and main geometric parameters of the DIRC detector.

\begin{figure*}[htbp]
 \centering
 \mbox{
   \begin{overpic} [width=0.45\textwidth,height=0.35\textwidth] {Figures/Figs_04_00_DetectorSubSystems/Figs_04_03_ParticleIdentification/DIRC_position1.jpg}\label{Fig4:DIRC_ENDCAP1}
   \end{overpic}
   \begin{overpic} [width=0.45\textwidth,height=0.35\textwidth] {Figures/Figs_04_00_DetectorSubSystems/Figs_04_03_ParticleIdentification/DIRC_position2.jpg}\label{Fig4:DIRC_ENDCAP2}
   \end{overpic}
 }
\caption{The proposed position and basic structure of DIRC detector at the endcap region.}
\label{Fig4:DIRC_ENDCAP}
\end{figure*}

As shown in figure \ref{Fig4:DIRC_ENDCAP2}, 12 trapezoidal DIRC detectors form an disc in each side of the endcap. Fused silica is used to construct the radiator plate, which is $1.5 cm$ thick, $47 cm$ high, and $46 cm$ $\/$ $24 cm$ wide for the outer $\/$ inner sides respectively. In order to keep the imaging pattern of different particles clear, the outer and inner side-face of the radiator plate is processed to be absorptive, as shown in figure ~\ref{Fig4:DIRC_OPT} (marked as black line). As also shown in figure ~\ref{Fig4:DIRC_OPT}, the outer end of each radiator plate is coupled with a optical focusing block, made by grinding part of a parallelogram quartz prism into a cylindrical shape (marked as red arc in the figure). Light from the radiator plate is focused through the cylindrical reflective surface and is eventually detected by photon sensors (e.g. MCP-PMTs) placed at the focal plane. An additional plane mirror is added to the focusing block (marked as blue line in the figure) so that the final detection plane is perpendicular to the magnetic field direction, in order to maintain the good performance of MCP-PMT.

\begin{figure*}[htbp]
 \centering
 \mbox{
   \begin{overpic} [width=0.6\textwidth,height=0.45\textwidth] {Figures/Figs_04_00_DetectorSubSystems/Figs_04_03_ParticleIdentification/DIRC_OPT1.jpg}
   \end{overpic}
 }
\caption{The basic structure of the focusing optics of DIRC detector.}
\label{Fig4:DIRC_OPT}
\end{figure*}

In the current design, as shown in figure ~\ref{Fig4:DIRC_MCP1}, the focusing unit of each module is coupled with $2\times10$ multi-anode MCP-PMTs, arranged in 2 rows. Every MCP-PMT provides a sensitive area of $45\times45 mm^{2}$, which is divided into $64\times8$ pixels with a size of $0.7\times5.6 mm^{2}$. The readout unit is finer segmented along the vertical direction than the horizontal direction in order to provide better angular resolution in the vertical direction combined with the cylindrical focusing prism.

\begin{figure*}[htbp]
 \centering
 \mbox{
   \begin{overpic} [width=0.6\textwidth,height=0.45\textwidth] {Figures/Figs_04_00_DetectorSubSystems/Figs_04_03_ParticleIdentification/DIRC_MCP1.jpg}
   \end{overpic}
 }
\caption{The arrangement of MCP-PMTs in one DIRC module.}
\label{Fig4:DIRC_MCP1}
\end{figure*}

{\bf GEANT4 Simulation}

Preliminary simulation of the DIRC detector is performed by using GEANT4 toolkit, based on the above design parameters. Since the simulation of Cherenkov light transmission strongly relies on the optical parameters, which are currently lacking without relevant experimental test, the optical parameters used in the simulation mainly refer to the Babar experiment. Some of the specific parameters are shown in figure \ref{Fig4:G4_PAR}.

\begin{figure*}[htbp]
 \centering
 \mbox{
   \begin{overpic} [width=0.45\textwidth,height=0.35\textwidth] {Figures/Figs_04_00_DetectorSubSystems/Figs_04_03_ParticleIdentification/G4_parameter1.jpg}\label{Fig4:G4_Par1}
   \end{overpic}
   \begin{overpic} [width=0.45\textwidth,height=0.35\textwidth] {Figures/Figs_04_00_DetectorSubSystems/Figs_04_03_ParticleIdentification/G4_parameter2.jpg}\label{Fig4:G4_Par2}
   \end{overpic}
   \begin{overpic} [width=0.45\textwidth,height=0.35\textwidth] {Figures/Figs_04_00_DetectorSubSystems/Figs_04_03_ParticleIdentification/G4_parameter3.jpg}\label{Fig4:G4_Par3}
   \end{overpic}
   \begin{overpic} [width=0.45\textwidth,height=0.35\textwidth] {Figures/Figs_04_00_DetectorSubSystems/Figs_04_03_ParticleIdentification/G4_parameter4.jpg}\label{Fig4:G4_Par4}
   \end{overpic}
 }
\caption{The optical parameters used in Geant4 simulation.}
\label{Fig4:G4_PAR}
\end{figure*}

Additionally, some other factors affecting the performance are considered, include the surface roughness, flatness and optical transmittance of the fused silica plate and the focusing prism. The glue used for PMT-quartz coupling, the quantum efficiency of MCP-PMT are also considered. In the simulation, the quartz is set with surface roughness of $\sigma_{\alpha} = 0.1 \deg$ and perfect flatness. The optical transmittance of the coupling glue is set to $100\%$. The quantum efficiency of MCP-PMT is taken from the datasheet of Hamamatsu 10754 model.

Figure ~\ref{Fig4:DIRC_IMAGE1} shows the simulation result of spatial hit pattern for $10000$ incident $\pi$ mesons at $2 GeV/c$. The imaging patterns generated by primary particles and secondary particles are displayed separately. Secondary particles contain mainly $\delta$ electrons produced by ionization, and $\pi$ decayed muons. The direction (and velocity) of muon is similar to that of the decayed $\pi$ meson, due to similar mass of muon and $\pi$, resulting in similar hit pattern on the detection plane, whereas the $\delta$ electrons produce hits uniformly distributed imaging due to the lack of regularity. The simulation results show that $\sim 27\%$ of the hits are produced by secondary particles.

\begin{figure*}[htbp]
 \centering
 \mbox{
   \begin{overpic} [width=0.6\textwidth,height=0.45\textwidth] {Figures/Figs_04_00_DetectorSubSystems/Figs_04_03_ParticleIdentification/DIRC_Imaging1.jpg}
   \end{overpic}
 }
\caption{The imaging hit pattern of one DIRC module.}
\label{Fig4:DIRC_IMAGE1}
\end{figure*}

Figure \ref{9} shows similar hit pattern as figure ~\ref{Fig4:DIRC_IMAGE2}, but for the time distribution rather than hit position. The timing signals in the lowest band in the figure are from photons detected by MCP-PMT without any reflections off the side surface of radiator plate. The signals in other higher bands are in turn from photons once and twice side-reflected. The width of these bands clearly illustrates the effect of chromatic dispersion on the timing, so it is important to correct the dispersion to improve the time resolution.

\begin{figure*}[htbp]
 \centering
 \mbox{
   \begin{overpic} [width=0.6\textwidth,height=0.45\textwidth] {Figures/Figs_04_00_DetectorSubSystems/Figs_04_03_ParticleIdentification/DIRC_Imaging2.jpg}
   \end{overpic}
 }
\caption{The imaging time pattern of one DIRC module.}
\label{Fig4:DIRC_IMAGE2}
\end{figure*}

{\bf Reconstruction Algorithm}

The transmission direction and distance of Cherenkov photons inside the radiator plate can be decomposed into three variables, $\theta_{x}$, $\theta_{z}$ and H, as already defined previously in the coordinate system shown in figure ~\ref{Fig4:DIRC-Cordinate1}. By reconstructing the above variables, the Cherenkov angle $\theta_{c}$ and the particle flight time TOF can be reconstructed by the aforementioned formula ~\ref{EQ:TOF}.

The value of H consists of two parts. The first part is the equivalent transmission distance in the quartz plate, determined by the incident position of the charged particle. The second part is the equivalent height in the focusing prism. As shown in figure ~\ref{Fig4:DIRCOPT_H_Y}, the equivalent height corresponding to different position on the focal plane (i.e., different angle $\theta_{z}$) is different, and there is a certain relation that can be used to reconstruct the equivalent H. At the same time, the equivalent height H of photons emitted from different positions (with the same angle $\theta_{z}$) is also different, and this ambiguity cannot be removed because the exact emission position cannot be determined due to finite thickness of the radiator. The simulation results show $\Delta_{H} \sim 4 mm$, which is in good agreement with the previous calculation results.

\begin{figure*}[htbp]
 \centering
 \mbox{
   \begin{overpic} [width=0.6\textwidth,height=0.45\textwidth] {Figures/Figs_04_00_DetectorSubSystems/Figs_04_03_ParticleIdentification/Opt_H_Y.jpg}
   \end{overpic}
 }
\caption{The relationship of equivalent H and sensor position (Y) in the DIRC focusing optics.}
\label{Fig4:DIRCOPT_H_Y}
\end{figure*}

According to the design of the optical focusing component, photons with different $\theta_{z}$ will be focused onto different $Y$ position of the focal plane. As shown in figure ~\ref{Fig4:DIRCOPT_THETAZ_Y}, $\theta_{z}$ can be obtained directly through the hit sensor position.

\begin{figure*}[htbp]
 \centering
 \mbox{
   \begin{overpic} [width=0.6\textwidth,height=0.45\textwidth] {Figures/Figs_04_00_DetectorSubSystems/Figs_04_03_ParticleIdentification/Opt_Thetaz_Y.jpg}
   \end{overpic}
 }
\caption{The relationship of $\theta_{z}$ and sensor position (Y) in the DIRC focusing optics.}
\label{Fig4:DIRCOPT_THETAZ_Y}
\end{figure*}

$\theta_{x}$ can be calculated from the incident position of the charged particle onto the radiator plate and the position of the photon sensor being hit. As shown in Figure ~\ref{Fig4: DIRCOPT_THETAXRECO}, x is the distance along $X$ direction between the particle incident point and the sensor, and H is given by the previous reconstruction method. The only difficulty lies in determining whether the photons are detected with or without lateral reflection. Several possible hypotheses (usually with 0, 1 or 2 lateral reflections, more reflections are neglected) will be tested, and the hypothesis fit the measured result best is selected as the final choice.

\begin{figure*}[htbp]
 \centering
 \mbox{
   \begin{overpic} [width=0.6\textwidth,height=0.45\textwidth] {Figures/Figs_04_00_DetectorSubSystems/Figs_04_03_ParticleIdentification/Opt_ThetaxReco.jpg}
   \end{overpic}
 }
\caption{The reconstruction of $\theta_{x}$ in the DIRC module.}
\label{Fig4:DIRCOPT_THETAXRECO}
\end{figure*}

For a simulation sample of $10000$ incident $\pi$ mesons at $2 GeV/c$, the reconstruction results are obtained by applying the above algorithm. Figure ~\ref{Fig4:DIRC_THETACRECO} shows the distribution of reconstructed Cherenkov angle from single photon. Compared to the truth $\theta_{c}$, the reconstructed results show a broadening to $6 mrad$. The plot on the right depicts the residual distribution $\theta_{c}^{reco} - \theta_{c}^{true}$, from which a $\theta_{c}$ reconstruction accuracy of $2.9 mrad$ is achieved.

Figure ~\ref{Fig4:DIRC_LOPRECO} shows the reconstructed photon transmission distance $L$. The results show clear difference between the reconstructed $L$ and the real one, especially for photons experiencing one or more lateral reflections. Further analysis shows that the error of $L$ mainly comes from the reconstruction error of equivalent height H, which correspondingly rises from the finite thickness of quartz plate. So choosing a thinner quartz plate helps to improve the reconstruction accuracy of the photon transmission distance $L$ (but at cost of less Cherenkov photon yield). The right plot of figure ~\ref{Fig4:DIRC_LOPRECO} shows the residual distribution of $L$, which gives a reconstruction accuracy of $2.7 mm$.

Figure ~\ref{Fig4:DIRC_TOFRECO} shows the reconstructed particle TOF distribution. For a single photon, the TOF error is 33 ps. It should be noted that the TOF uncertainty here has only considered $\sigma_{det}$.

\begin{figure*}[htbp]
 \centering
 \mbox{
   \begin{overpic} [width=0.45\textwidth,height=0.35\textwidth] {Figures/Figs_04_00_DetectorSubSystems/Figs_04_03_ParticleIdentification/DIRC_ThetacReco1.jpg}
   \end{overpic}
   \begin{overpic} [width=0.45\textwidth,height=0.35\textwidth] {Figures/Figs_04_00_DetectorSubSystems/Figs_04_03_ParticleIdentification/DIRC_ThetacReco2.jpg}
   \end{overpic}
 }
\caption{The reconstructed Cherenkov angle $\theta_{c}$.}
\label{Fig4:DIRC_THETACRECO}
\end{figure*}

\begin{figure*}[htbp]
 \centering
 \mbox{
   \begin{overpic} [width=0.45\textwidth,height=0.35\textwidth] {Figures/Figs_04_00_DetectorSubSystems/Figs_04_03_ParticleIdentification/DIRC_LOPReco1.jpg}
   \end{overpic}
   \begin{overpic} [width=0.45\textwidth,height=0.35\textwidth] {Figures/Figs_04_00_DetectorSubSystems/Figs_04_03_ParticleIdentification/DIRC_LOPReco2.jpg}
   \end{overpic}
 }
\caption{The reconstructed length of propagation of Cherenkov photons in the DIRC detector.}
\label{Fig4:DIRC_LOPRECO}
\end{figure*}

\begin{figure*}[htbp]
 \centering
 \mbox{
   \begin{overpic} [width=0.6\textwidth,height=0.45\textwidth] {Figures/Figs_04_00_DetectorSubSystems/Figs_04_03_ParticleIdentification/DIRC_TOFReco.jpg}
   \end{overpic}
 }
\caption{The reconstructed uncertainty of TOF in the DIRC detector.}
\label{Fig4:DIRC_TOFRECO}
\end{figure*}





\input{Chapters/Chapter_04_00_DetectorSubSystems/Chapter_04_03_ParticleIdentification/04_Ref_SimulationStudy}
