\subsubsection{Motivation}
\label{section1}

Excellent particle identification (PID) ability is one of the most important parts for the high-energy particle/nulcear experiment in a modern high luminosity accelerator, especially when the energy scale involved is below Z/W/H/t mass. In the physics research of the super-c or super-b factories,  the demanding for PID is crucial and the requirements for corresponding detector technologies are challenging, primarily due to much higher beam luminosity and event rates, very short response time and intensive radiation level.

In the super tau-charm facility (STCF), the effective PID over the full kinetic space is required within the detector acceptance for charged hadrons ($\pi^{\pm}$, $K^{\pm}$, and $p/\bar{p}$), with a statistical separation power better than $3\sigma$.  In this chapter the PID of charged hadrons is described, while the identification of leptons and neutral particles will be discussed in the other chapters, such as Electromagnetic calorimeter (EMC) and Muon Detector (MUD), according to the very different particle detection technologies.

The experimental system of STCF is shown in Figure \ref{FIG:STCF_DET}. The PID detector is located between the tracking detector, main drift chamber (MDC), and the EMC. Because the luminosity is expected to reach $\mathcal{L} =10^{35} cm^{-2}s^{-1}$, the PID detector is dictated to have fast response and good radiation resistance (which will be addressed in section ~\ref{section2}). At the same time, considering the required energy and position resolution of the outer EMC, the material budget of PID detector should not exceed at most 0.5 radiation length ($X_{0}$).  In addition, the PID detector should choose compact design as far as possible to reduce the total cost of the whole experimental device. These demanding also fit the latest research and development ($R\&D$) trend of PID technology in the field.

\begin{figure}[htbp]
	\begin{center}
		\begin{overpic}[width=10.cm,height=7cm,angle=0]{Figures/Figs_04_00_DetectorSubSystems/Figs_04_03_ParticleIdentification/detector.jpg}
		\end{overpic}
		\caption{The schematic experimental structure at STCF .}
		\label{FIG:STCF_DET}
	\end{center}
\end{figure}

\input{Chapters/Chapter_04_00_DetectorSubSystems/Chapter_04_03_ParticleIdentification/04_Ref_Motivation}

