\subsubsection{FTOF}

\paragraph{1) FTOF Design}

\indent\par
In order to meet the requirement of PID in the high density, large luminosity collisions for the next generation particle accelerator, TORCH (Time Of internally Reflected Cherenkov detector) is developed as an innovative TOF detector for the LHCb experiment. It is able to identify high energy charged particles with excellent time resolution at the extreme data taking conditions under high luminosity and high backgrounds.

The Forward Time-Of-Flight (FTOF) detector is designed to provide charged hadron identification ($\pi/K$ separation up to $2~GeV/c$ and $p/K$ separation up to $2.5~GeV/c$). It is positioned at $140 \sim 160 cm$ downstream from the collision point, in front of the ECAL, covering the polar angle of $20\deg \sim 34\deg$. The proposed FTOF is applied TORCH technology, composed of 12 trapezoid silica radiators, each radiator attached with 14 MCP-PMTs (Hamamatsu R10754) and picosecond readout electronics at the wider end. It is supported by the carbon fabric structure. When a high energy charged particle pass through FTOF, it induces Cherenkov radiation within the radiator, emitting ultraviolet photons at the Cherenkov angle. After multiply total reflections at the boundary of the radiator, the photon is collected by the PMT array and turn into photoelectron signal. Finally the readout electronics records the signal¡¯s arrival time and hit position for PID. The detailed size of FTOF and its installed position are shown in Fig. \ref{FIG:FTOF_Design}.

\begin{figure}[!htb]
  \centering
  \includegraphics[width=0.35\textwidth]{Figures/Figs_04_00_DetectorSubSystems/Figs_04_03_ParticleIdentification/FTOF_module.jpg}
  \includegraphics[width=0.3\textwidth]{Figures/Figs_04_00_DetectorSubSystems/Figs_04_03_ParticleIdentification/R10754_hamamatsu.jpg}
  \includegraphics[width=0.35\textwidth]{Figures/Figs_04_00_DetectorSubSystems/Figs_04_03_ParticleIdentification/FTOF_disc.jpg}
  \includegraphics[width=0.8\textwidth]{Figures/Figs_04_00_DetectorSubSystems/Figs_04_03_ParticleIdentification/FTOF_in_DetSys.jpg}
  \caption{The design of FTOF: single trapezoid FTOF sector (left), single MCP-PMT (middle), the layout of FTOF (right), the FTOF position in the spectrometer (bottom).}
  \label{FIG:FTOF_Design}
\end{figure}


\paragraph{2) FTOF PID requirements}

\indent\par
The Cherenkov detector is commonly used in high-energy experiments to identify particles at high momentum. This specific radiation happens if a charged particle travels faster than the speed of light in the medium. Fig. \ref{FIG:CherenkovWaveFront} shows the principle of Cherenkov radiation. With a refractive index n of the radiator and particle speed $v = \beta c$, the Cherenkov radiation emits ultraviolet light at an angle relative to the particle moving direction with a high velocity $v = \frac{c}{n}$.

\begin{figure}[!htb]
  \centering
  \includegraphics[width=0.5\textwidth]{Figures/Figs_04_00_DetectorSubSystems/Figs_04_03_ParticleIdentification/CherenkovWavefront.jpg}
  \caption{The principle of Cherenkov radiation and main parameters.}
  \label{FIG:CherenkovWaveFront}
\end{figure}

With the utilization of Cherenkov radiation, the FTOF is aiming at measuring the flight time of various charged particles with high timing resolution of picosecond level to achieve the charged hadron identification up to $2~GeV/c$. Roughly, the flight time of charged particle from the collision point to FTOF is:
\begin{equation}
t = \frac{L}{c} \sqrt{1+(\frac{mc}{p})^{2}}.
\end{equation}
Here $L = \frac{1.2}{cos(20\deg)} = 1.28 m$, m and p are the mass and momentum of charged particle. Typically for $\pi/K$ separation at $p = 2~GeV/c$, the flight time difference $\Delta t$ and separation power can be estimated as:
\begin{equation}
\Delta t = \frac{Lc}{2p^{2}}(m_{1}^{2}-m_{2}^{2}),
Separation power = \frac{\Delta t}{\sigma}.
\end{equation}
Here $m_{1}$ and $m_{2}$ represent the mass of $K$ and $\pi$. Therefore to achieve $3\sigma$ $\pi/K$ separation at $p = 2~GeV/c$, an overall FTOF time resolution $\sim 40 ps$ is needed as shown in Fig. \ref{FIG:FTOF_PIDpower}.

\begin{figure}[!htb]
  \centering
  \includegraphics[width=0.8\textwidth]{Figures/Figs_04_00_DetectorSubSystems/Figs_04_03_ParticleIdentification/FTOF_PIDpower.jpg}
  \caption{The separation power for $\pi/K$ at different momenta.}
  \label{FIG:FTOF_PIDpower}
\end{figure}

To further study the separation power requirement of FTOF by Geant4 simulation, the overall time resolution can be described as:
\begin{equation}
\sigma^{2}_{tot} \sim (\frac{\sigma_{elec}}{\sqrt{N_{p.e.}}})^{2} + (\frac{TTS}{\sqrt{N_{p.e.}}})^{2} + (\frac{\sigma_{det}}{\sqrt{N_{p.e.}}})^{2} + \sigma^2_{trk} + \sigma^{2}_{T_{0}}.
\end{equation}

Here $N_{p.e.}$ is the number of photoelectron collected by the PMT. $\sigma_{elec}$ describes the readout electronics fluctuation
($<10ps$ in preliminary test). $\sigma_{TTS}$ is the transit time spread (TTS) of the MCP-PMT, for R10754 it is $<70ps$.
$\sigma_{T_{0}}$ is the start time resolution designed as $40ps$. $\sigma_{det}$ is the time spread coming from various detector effects, including:
$\sigma_{chromatic}$ driven by the refraction index which is depends on wavelength.
$\sigma_{sensorsize}$ is due to the finite size of the PMT channels, that is, photons produced by a charged track can follow slightly different paths to a given channel, which cause additional time smearing.
$\sigma_{transit}$ appears due to the fact that charged particle emits photons all along its path in the Cherenkov radiator and cause signal fluctuation. It depends on the thickness of radiator.
$\sigma_{impactpos}$, $\sigma_{p}$: fluctuation caused by the impact position where the track pass through the radiator and its track momentum.
$\sigma_{trk}$ is due to the track parameter reconstruction, including the time uncertainties coming from the precision of the track momentum and track path length. Simulation shows it is $\sim 10 ps$.

In summary, assuming $N_{p.e.} = 10$, with $\sigma_{TTS} = 70 ps$ and $\sigma_{T_{0}} = 40 ps$, simulation shows that $\sigma_{det}$ for single
photoelectron should be $\sim 80 ps$, corresponding to $\sigma_{tot} \sim 40 ps$.


%\newpage
\paragraph{3) FTOF Components and Tests}

\indent\par
Each FTOF sector is composed of a $(24cm/45cm) \times 40cm \times 1.5cm$ trapezoid fused silica radiator attached with 14 MCP-PMT (Hamamatsu R10754) array. The fused silica radiator has transmission rate of $95\%$ for ultraviolet light at the wavelength above $300 nm$. With an average refractive index of $1.46$, the maximum radiation angle of Cherenkov light is $46.8\deg$ with a light yield of $\sim 260 /cm$ in the wavelength range of $300 \sim 400 nm$. Its surface is polished with roughness below $0.8 nm$ (RMS) and flatness below $0.1 mm$. To be better coupled to PMT the coupling surface is coated with silicone oil (Rhodorsil Huile 47 V 1000) permeable to ultraviolet light. The radiator is installed in carbon-fibre black box. There is black paper attached on the rear end to reduce the time spread from multiple internal reflections of photons.

The type of PMT for FTOF is R10754 from Hamamatsu Company. It is a $4 \times 4$ multi-channel micro-channel plate photomultiplier tube (MCP-PMT). Fig \ref{FIG:R10754} shows its main parameters. We also test its performance by using a picosecond laser (Passat COMPILER, $FWHM = 4 ps$, $\lambda = 215 nm$). The laser is split into two beams, illuminating both the MCP-PMT under test (single-photon mode) and a reference MCP-PMT (multi-photon
mode, used as the time reference for transit time measurement) respectively. A high bandwidth oscilloscope (sampling rate: $40 GS/s$) is used to record the output signal waveforms. Fig. \ref{FIG:R10754_SPE_TTS} shows the performance of MCP-PMT at the high voltage
(HV) of $3100 V$. Its gain is around $1 \times 10^{6}$, calculated by:
\begin{equation}
Gain = \frac{Q_{SPE}-Q_{ped}}{e}.
\end{equation}

The transit time spread (TTS) is $\sim 18 ps$, obtained from the time difference between the measured MCP-PMT and the reference MCP-PMT.

\begin{figure}[!htb]
  \centering
  \includegraphics[width=0.95\textwidth]{Figures/Figs_04_00_DetectorSubSystems/Figs_04_03_ParticleIdentification/R10754_parameters.jpg}
  \caption{Characteristics of R10754.}
  \label{FIG:R10754}
\end{figure}

\begin{figure}[!htb]
  \centering
  \includegraphics[width=0.45\textwidth]{Figures/Figs_04_00_DetectorSubSystems/Figs_04_03_ParticleIdentification/R10754_SPE.jpg}
  \includegraphics[width=0.45\textwidth]{Figures/Figs_04_00_DetectorSubSystems/Figs_04_03_ParticleIdentification/R10754_TTS.jpg}
  \caption{Measured characteristics of MCP-PMT: The single photoelectron (SPE) spectrum at HV of $3100 V$ and Gain of $1 \times 10^{6}$ is calculated (left). The transit time spread (TTS) of $18 ps$ is obtained from the time difference between the measured MCP-PMT and the reference MCP-PMT (right).}
  \label{FIG:R10754_SPE_TTS}
\end{figure}

%\newpage

\paragraph{4) FTOF readout electronics}

\indent\par
As a crucial part of the picosecond level timing, the time uncertainty contributed from the electronics is expected to be below 10 picoseconds. To meet this requirement, an FPGA-based readout electronics with a multi-threshold leading-edge timing scheme is designed. It consists of a gain programmable differential amplifier (PDA), a multi-threshold differential discriminator (MDD), and a set of time-to-digital converters (TDC). Except for the PDA and the bias network, other components are implemented inside of a field-programmable gate array (FPGA) chip (Xilinx Kintex-7). The TDC module is implemented in the FPGA using an ones-counter encoding scheme, its intrinsic RMS time resolution is evaluated as $3.9 ps$. The intrinsic time performance of the entire electronics is evaluated as $5.6 ps$. More detail of the front-end electronics will be discussed in the FTOF readout electronics section.

\paragraph{5) FTOF Prototype and Further Design Consideration}

\indent\par

The performance of a FTOF prototype has been tested in the $5 GeV$ electron beam at DESY in 2019. The prototype installed in a black box measures the photons'arrival time as $T1$, with another MCP-PMT placed next to it as start time $T0$, as shown in Fig. \ref{FIG:FTOF_DESY_BeamTest}. Then we fit the distribution of $T1 - T0$ to estimate its time resolution. The test result after T-A correction with constant fraction thresholds is also shown in Fig. \ref{FIG:FTOF_DESY_BeamTest}. The time resolution achieves $\sim 100 ps$ for proper threshold setting. Since this result is without beam hit
position correction and any other optimization, it still has the potential for further improvement. An obvious future improvement may be focusing photons to the MCP-PMT to reduce the photons' hit position difference and increase the PMT acceptance, as shown in Fig. \ref{FIG:DIRC_FTOF_Optics}. We also need to find a way to reduce the crosstalk between adjacent PMT channels.

\begin{figure}[!htb]
  \centering
  \includegraphics[width=0.45\textwidth]{Figures/Figs_04_00_DetectorSubSystems/Figs_04_03_ParticleIdentification/FTOF_DESY_timing.jpg}
  \includegraphics[width=0.45\textwidth]{Figures/Figs_04_00_DetectorSubSystems/Figs_04_03_ParticleIdentification/FTOF_DESY_prototype.jpg}
  \caption{Time resolution of FTOF prototype in beam test (left) and test photo at DESY (right).}
  \label{FIG:FTOF_DESY_BeamTest}
\end{figure}

\begin{figure}[!htb]
  \centering
  \includegraphics[width=0.18\textwidth]{Figures/Figs_04_00_DetectorSubSystems/Figs_04_03_ParticleIdentification/DIRC_FTOF_structure.jpg}
  \includegraphics[width=0.5\textwidth]{Figures/Figs_04_00_DetectorSubSystems/Figs_04_03_ParticleIdentification/DIRC_FTOF_optics.jpg}
  \caption{Time resolution of FTOF prototype in beam test (left) and test photo at DESY (right).}
  \label{FIG:DIRC_FTOF_Optics}
\end{figure}



\input{Chapters/Chapter_04_00_DetectorSubSystems/Chapter_04_03_ParticleIdentification/04_Ref_FTOF}
