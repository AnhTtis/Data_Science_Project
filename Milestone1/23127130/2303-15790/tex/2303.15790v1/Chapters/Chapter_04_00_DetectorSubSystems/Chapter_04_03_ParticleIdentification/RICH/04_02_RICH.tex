\subsection{RICH Detector Concept}

\subsubsection{Conceptual Design}
The barrel PID system is placed between the MDC and EMC, with a solid coverage of $0.83$ and a distance of approximately $85$~cm to the collision point. This requires the system to be thin enough to leave space for the calorimeter, and the material budget needs to be low to reduce the energy distortion. Hence, the thickness must be less than $20$~cm, and the material budget must be less than $30\%X{_0}$. The detector also needs to have a fast time response to operate under a high luminosity environment, with $\mathcal{L} =10^{35}$~cm$^{-2}$s$^{-1}$.	

To satisfy these requirements, a micropattern gas detector (MPGD)-based RICH detector is chosen as the baseline design for the barrel part of the PID detector. The RICH particle separation power can be estimated from Eq.\,\ref{eq:nsigma},
\begin{equation}
N_\sigma\approx\frac{|m_1^2 - m_2^2|}{2p^2\sigma_\theta\sqrt{n^2-1}} \label{eq:nsigma},
\end{equation}
where $m_1$ and $m_2$ are the masses from different particle hypotheses, $\sigma_\theta$ is the angular resolution, $p$ is the momentum and $n$ is the refractive index. Fig.~\ref{fig:pidvsrad} shows the $\pi/K$ and $K/p$ separation capability in terms of standard deviation vs. momentum for different types of radiators, where the angular resolution of the RICH detector is assumed to be $2.5\,$mrad. For the momentum range of interest of the STCF ($<2$\,$\gevc$), liquid and quartz have adequate refractive indexes.



\begin{figure}
\centering
\includegraphics[width=.8\textwidth]{Figures/Figs_04_00_DetectorSubSystems/Figs_04_03_ParticleIdentification/RICH/1pid-vs-rad.pdf}
\caption{The $\pi/K/p$ PID separation abilities of different radiators assuming $2.5\,$mrad angular resolution. The results for quartz and C$_6$F$_{14}$ with a refractive index of $180\,$nm are depicted.}
\label{fig:pidvsrad}
\end{figure}


A sketch of the approximately focused RICH detector is shown in Fig.~\ref{fig:RICHstructure}. It includes a liquid radiator layer, optical transparent quartz, a working gas, and photon detectors. A charged particle moving outward can emit photons in the radiator with a Cherenkov angle of $\theta_c=\cos^{-1}(1/\beta n)$ with respect to the particle direction, where $\beta$ is the velocity of the particle and $n$ is the refractive index of the radiator medium. These photons are approximately focused on the anode layer of the photon detectors to produce a circular ring image. For each detected photon, a value of $\theta_c$ can be calculated for comparison with different particle hypotheses.
\begin{figure}
\centering
\includegraphics[width=.8\textwidth]{Figures/Figs_04_00_DetectorSubSystems/Figs_04_03_ParticleIdentification/RICH/2RICHstructure.pdf}
\caption{The RICH detector structure.}
\label{fig:RICHstructure}
\end{figure}

The present baseline design of the RICH detector considering the above discussion is described as follows. Each module contains a 10.0\,mm-thick liquid radiator with a 3.0\,mm quartz window. A light propagation region of 100.0\,mm separates the quartz window and the cathode wires (mesh) from the CsI-coated thick gaseous electron multiplier (THGEM) placed under the cathode wires.

The radiator container, the chamber gas and the remaining components must be UV transparent or at least exhibit UV transparency comparable to that of the radiator. However, the gas may be contaminated by water vapor and oxygen, which reduces the transparency range. Hence, this contamination must be controlled to less than $10$\,ppm.

The RICH detector consists of a CsI-coated THGEM as the photocathode and a Micromegas (MM) layer to amplify the photoelectrons. The distance between the THGEM layer and the MM layer is 2~mm. A small reverse electric field is applied between the cathode and the top layer of the THGEM to increase the quantum efficiency. The converted photoelectrons drift to the bottom layer of the Micromegas plane with a gain of approximately 10$^5$. The induced signal is then picked up by the anode. The ions generated during the amplification drift along the electrical lines and bombard the photocathode. This induces a second signal and results in photocathode aging. With the hybrid combination of the THGEM and MM, the ion backflow is demonstrated to be less than $4$\%~\cite{compass}.
Considering the goal of ten years of operation, the total accumulated charge for the STCF RICH detector would be approximately $\approx 2\,\mu$C/cm$^2$.
The usage of large-area CsI photocathodes has been demonstrated in the ALICE~\cite{alice} and COMPASS~\cite{compass} experiments, and no severe aging effects have been seen under $10\,\mu$C/cm$^2$.

The overall material budget with aluminum plating as the container is listed in Table \,\ref{TAB:richbudget}. The total material budget is approximately 24\% $X_0$.

\begin{table}[h]
  \centering
  \begin{tabular}{|l|c|c|}\hline
    &thickness [mm]&$X/X_0$\\\hline
    Top ceramic plate &3&0.03\\
    Quartz window&3&0.03\\
    Radiator C$_6$F$_{14}$&10&0.05\\
    THGEM+Micromegas&0.4&0.01\\
    Anode+FEE&8&0.02\\
    Aluminum plate&5&0.05\\
    FEE cooling&5&0.05\\
    Total&&0.24\\\hline
  \end{tabular}
   \caption{Material budget of the RICH detector.}
  \label{TAB:richbudget}
\end{table}


\subsubsection{Single Photon Angular Resolution}

RICH signals can be simulated analytically based on the propagation of Cherenkov light, as shown in Fig.\,\ref{fig:richrec}, where $L$ is the thickness and $n$ is the refractive index. The subscript indexes $r$, $q$, and $g$ represent the radiator, the quartz box, and the transport gas medium, respectively. $L_0$ is the distance between the emission point where Cherenkov light is generated along a charged particle and the bottom of the radiator. $\theta_0$ can be calculated from the charged particle incident angle $\theta$, Cherenkov emission angle $\theta_c$ and azimuth angle $\phi$. Thus, the expected hit position on the anode can be expressed as $r = L_0\cdot\tan\theta_0+L_q\cdot\tan\theta_1+L_g\cdot\tan\theta_2$.

To reconstruct the Cherenkov angle $\theta_c$ from each hit position $r$, two assumptions are made: 1) the Cherenkov radiation is emitted from the center of the charged track trajectory inside the radiator because the absorption effects are neglected; 2) although Cherenkov radiation is a spectrum, the refractive indexes of each component at 180\,nm are taken because this is the most likely value considering the absorption of each medium and the quantum effect of the cathode.

For illustration purposes, it is useful to consider the case of a vertical incident particle ($\theta=0$). By defining $R_0 = L_g\frac{n\sin\theta_c}{\sqrt{1-n^2\sin^2\theta_c}}$, the Cherenkov angle can be expressed as:
\begin{equation}
\theta_c=acos(\frac{1}{\sqrt{\frac{\beta^2}{1+\frac{L_g^2}{R_0^2}}+1}})
\end{equation}

\begin{figure}
\centering
\includegraphics[width=.8\textwidth]{Figures/Figs_04_00_DetectorSubSystems/Figs_04_03_ParticleIdentification/RICH/3rich-rec.pdf}
\caption{The Cherenkov light propagation in the RICH detector.}
\label{fig:richrec}
\end{figure}

The angular resolution of this type of RICH detector includes contributions from the following components:
\begin{equation}
\sigma^2_\theta=\frac{\sigma^2_E + \sigma^2_{L_{rad}} + \sigma^2_{xy} + \sigma^2_{ms} + \sigma^2_{\theta_0}}{N_{pe}} \label{eq:rich_sigma_theta}
\end{equation}

The term $\sigma^2_E$ is the contribution from chromatic dispersion. Cherenkov radiation consists of a continuous spectrum of wavelengths extending from the ultraviolet region into the visible spectrum. Liquid perfluorohexane C$_6$F$_{14}$ and high purity quartz are both candidates for Cherenkov radiators for the STCF. The UV threshold is mainly set by quartz, at approximately 170~nm. Cesium iodide (CsI) provides a relatively high quantum efficiency ($\sim22\%@180$\,nm) in this region and decreases to zero at approximately 210\,nm. However, the refractive indexes of quartz and C$_6$F$_{14}$ vary by approximately $\sim 8$\% and $\sim 3$\% in this region, respectively. The chromatic uncertainty is intrinsic and the main source of the systematic error.

The term $\sigma^2_{L_{rad}}$ comes from the uncertainty of the emission point of the Cherenkov radiation. This is estimated by taking $L/\sqrt{12}$, where $L$ is the charged track length inside the radiator, and the expected emission point is near the center of the radiator. This contribution is proportional to the fraction of the radiator length $L_{rad}$ and the light propagation length.

The term $\sigma^2_{xy}$ comes from the anode spatial resolution. The granularity of the detector, i.e., the anode pad size for the readout, is driven by the required angular resolution, the light propagation distance and the number of photoelectrons. The gaseous detector provides a negligible material budget as well as possibly a long distance for light propagation. From our previous study, $5\,$mm anode pads with a $10\,$cm light propagation length would be sufficient.

The term $\sigma^2_{ms}$ represents the multiple scattering of charged particles inside the radiator medium. This dispersion can be estimated by taking $\sigma_{ms}\sim\Delta\theta_{ms}$, where $\Delta\theta_{ms}\propto (1/p)\sqrt{L/X_0}$. This term contributes mainly to the low momentum range and decreases rapidly when momentum increases.

The term $\sigma^2_{\theta_0}$ comes from the incident angle uncertainty. This can be estimated by extrapolating the reconstructed charged track from the MDC to the PID detector and is not taken into account for current optimization.

To estimate the RICH reconstruction accuracy, we can deduce the chromatic error related to the refractive index $n$ of the radiator, the geometric error related to the emission point of Cherenkov radiation $L_0$, the localization error related to the spatial resolution of the detector, and the multiple scattering error of the charged incident particle. By taking the thicknesses for each part as $L_r=10$\,mm, $L_q=3$\,mm and $L_g=100$\,mm, which are the default designs for the RICH detector, we can obtain the contribution of each systematic error on the angular resolution $\theta_c$. The results are summarized in Table~\ref{TAB:richsys}. Fig.\,\ref{fig:richsyserr} shows these four contributions to $\theta_c$ resolution as a function of $\gamma$ in this configuration, and it can be seen that the dominant contribution is the chromatic error and the geometric error. A comparison to the Geant4 simulation for $2\,$GeV/c $\pi$ is made, showing consistency.


\begin{table}[h]
  \centering
  \begin{tabular}{|l|c|c|}\hline
    Source&Error (mrad)&Simulation (mrad)\\\hline
    Chromatic&6.0&5.0\\
    Geometric&2.6&3.1\\
    Localization&1.6&1.8\\
    Multiple scattering&1.1&1.1\\
    Total&6.8&6.2\\\hline
  \end{tabular}
   \caption{Systematic error for RICH reconstruction.}
  \label{TAB:richsys}
\end{table}

\begin{figure}
\centering
\includegraphics[width=.8\textwidth]{Figures/Figs_04_00_DetectorSubSystems/Figs_04_03_ParticleIdentification/RICH/richsyserr.pdf}
\caption{RICH reconstruction system error versus the Lorentz factor $\gamma$. }
\label{fig:richsyserr}
\end{figure}


\subsubsection{Number of Photoelectrons}
The number of photon electrons $N_{pe}$ depends on the thickness of the radiator, the attenuation length of the radiator and the light propagation distance, and the quantum efficiency of the CsI photocathode. This number is given by:
\begin{equation}
N_{pe}=\int{N_0 \cdot L_t \cdot \sin^2\theta_c \prod \limits_{i=0}^n e^{-l_i/L_{ai}}\epsilon(\lambda)d\lambda}
\end{equation}
where $N_0 \cdot L_t \cdot \sin^2\theta_c$ is the Cherenkov light output per unit thickness, which is related to the particle velocity $\beta$ and the refractive index $n$ of the radiator; $\theta_c$ is the mean Cherenkov angle over the detected photon energy spectrum; $L_t$ is the thickness of the particle passing through the radiator; $l_i$ is the distance traveled by the Cherenkov light passing through the $i$-th optical component; $L_{a}$ is the attenuation length of the $i$-th optical component; and $\epsilon(\lambda)$ is the CsI quantum efficiency, which is taken from Ref.~\cite{alice}.
Fig.~\ref{fig:RICHCompDist}(a) shows the refractive indexes of the liquid C$_6$F$_{14}$ and quartz used as the RICH radiators, (b) shows the transmission rate of each optical component, including the working gas with different humidity and oxygen contamination values, (c) shows the number of photoelectrons for 2~$\gevc$ $\pi$ and $K$, and (d) shows the reconstructed Cherenkov angle distribution for 2~$\gevc$ $\pi$ and $K$.
An average of $\sim10$\,p.e. is expected. In total, the RICH detector is expected to have $6.9$\,mrad SPE angular resolution, and better than $2.5$\,mrad angular resolution can be achieved.

\begin{figure*}[htbp]
 \centering
  \subfloat[][]{\includegraphics[width=0.45\textwidth]{Figures/Figs_04_00_DetectorSubSystems/Figs_04_03_ParticleIdentification/RICH/4drawRefInd.pdf}}\quad
\subfloat[][]{\includegraphics[width=0.45\textwidth]{Figures/Figs_04_00_DetectorSubSystems/Figs_04_03_ParticleIdentification/RICH/4drawTrans.pdf}} \quad
  \subfloat[][]{\includegraphics[width=0.45\textwidth]{Figures/Figs_04_00_DetectorSubSystems/Figs_04_03_ParticleIdentification/RICH/4drawe_nphoton.pdf}}\quad
  \subfloat[][]{\includegraphics[width=0.45\textwidth]{Figures/Figs_04_00_DetectorSubSystems/Figs_04_03_ParticleIdentification/RICH/4drawevent.pdf}}
\caption{(a) Refractive indexes for liquid C$_6$F$_{14}$ and quartz, (b) transmission rate of each optical component, (c) photoelectron distribution, and (d) reconstructed Cherenkov angle distribution.}
\label{fig:RICHCompDist}
\end{figure*}



\subsection{RICH Detector Performance Simulation}
\subsubsection{Expected PID Capabilities}

{\sc Geant4} simulations are performed to study the expected performance of the RICH detector. Incident particles $\pi/K/p$ are emitted from the IP inside a 1~T magnetic field. The optical properties of the radiator (quartz and C$_6$F$_{14}$) are defined according to the measurement results~\cite{alice}, and the absorption of gas is calculated from the H$_2$O and O$_2$ absorption cross-section, with an assumed contamination of $10$\,ppm for each. During the simulation, the detector response is considered using the CsI converter quantum efficiency. A momentum and azimuth angle scan is then performed, and a typical ring of the Cherenkov hit pattern is shown in Fig.\,\ref{FIG:RICH_Cherenkov_Example}. The blue and red curves represent $2\,\rm{GeV}/c$ pions with polar angles $\theta = 0^\circ$ and $40^\circ$, respectively.

\begin{figure}[!htb]
  \centering
  \includegraphics[width=1.0\textwidth]{Figures/Figs_04_00_DetectorSubSystems/Figs_04_03_ParticleIdentification/RICH/5CherenkovExample.pdf}
  \caption{Examples of Cherenkov images in a RICH module. The blue image depicts the distribution of hits for $2\,\rm{GeV}/c$ pion with incident angle $\theta = 0^\circ$, perpendicular to RICH, while the red image depicts $\theta=40^\circ$.}
  \label{FIG:RICH_Cherenkov_Example}
\end{figure}

To further evaluate the PID capabilities of the RICH, a likelihood-based PID method is studied. The number of Cherenkov photons collected by the $5\times 5\rm\,cm^2$ anode pads follows a Poisson distribution. Therefore, the probability density function for the signal of an anode pad can be constructed as:

\begin{equation}
  pdf_{i, h} = Poisson(N_i + 10^{-3}, mean_{i, h} + 10^{-3}),
  \label{PIDlikelihood}
\end{equation}
\noindent
where $i$ represents the pad index, $h$ denotes hadron species (in our case, $\pi,~K, ~p$), $N$ is the photon number collected by this pad, $mean_{i, h}$ represents the expected average number of photons of each anode pad, which is simulated in Geant4, and the constant $10^{-3}$ is a conservative estimation of the background level for each anode pad. Under each particle type hypothesis, the log-likelihood of the RICH detector is calculated by summarizing the log-likelihood of all pads:

\begin{equation}
  \ln\mathcal{L}_{h} = \sum_{i}^{\rm {npads}}\ln pdf_{i, h}
\end{equation}

However, instead of calculating the absolute log-likelihood, in separating particle types, the difference in log-likelihood (DLL) between two hypotheses is calculated. For instance, in $\pi,~ K$ separation, DLL is defined as:

\begin{equation}
  {\rm {DLL}} = \sum_{i}^{\rm {npads}} \ln\frac{pdf_{i,\pi}}{pdf_{i,K}}
\end{equation}
\noindent
If $\rm DLL>0$, the $\pi$ hypothesis is accepted; otherwise, the $K$ hypothesis is accepted. The PID efficiency vs. momentum obtained by applying the reconstruction algorithm is shown in Fig.\,\ref{fig:richpideffsim}.

\begin{figure*}[htbp]
 \centering
   \subfloat[][]{\includegraphics[width=0.45\textwidth]{Figures/Figs_04_00_DetectorSubSystems/Figs_04_03_ParticleIdentification/RICH/6pi_k_eff.pdf}}
  \subfloat[][]{\includegraphics[width=0.45\textwidth]{Figures/Figs_04_00_DetectorSubSystems/Figs_04_03_ParticleIdentification/RICH/6pi_k_misid.pdf}}
\caption{RICH PID capabilities in terms of (a) $\pi$/K efficiency and (b) misidentification efficiency.}

\label{fig:richpideffsim}
\end{figure*}

A detailed scan for the $\pi/K/p$ hadron hypothesis is also performed. The momentum of the tracks ranges from $0.8\,\rm {GeV}/c$ to $2.4\,\rm{GeV}/c$, and the incident polar angle ranges from $0^\circ $ to $50^\circ$. The momentum range is divided in $30\,{\rm MeV}/c$ step, and the polar angle step is $1^\circ$. The $\pi$ PID efficiency and $K/p$ mis-ID rate are shown in Fig.~\ref{fig_RICH_pidetail}. The PID efficiency increases with momentum and varies along the incident angle. Especially for a very large incident angle, such as an angle larger than 50 degrees, the Cherenkov light mainly comes from the quartz box instead of the radiator; thus, the PID efficiency decreases dramatically. Additionally, there are some small zones where the PID efficiency drops due to some other misidentifications. This is because of the difficulties in distinguishing the light from the quartz box and that from the radiator. In general, the RICH system can fulfill the hadron PID requirements of the STCF.

\begin{figure}[h!]
  \centering
   \subfloat[][]{\includegraphics[width=0.45\textwidth]{Figures/Figs_04_00_DetectorSubSystems/Figs_04_03_ParticleIdentification/RICH/7Pi2Pi-2d.pdf}}
   \subfloat[][]{\includegraphics[width=0.45\textwidth]{Figures/Figs_04_00_DetectorSubSystems/Figs_04_03_ParticleIdentification/RICH/7K2Pi-2d.pdf}}
  \caption{PID capability scans for (a) $\pi$ efficiency, (b) $\pi/K$ mis-ID rate and (c) $\pi/p$ mis-ID rate.}\label{fig_RICH_pidetail}
\end{figure}



\subsubsection{Background simulation and occupancy}

As shown in Table~\ref{tab:5.2.01}, the major component of the background is expected to be luminosity-related background. To estimate the occupancy from the background, each charged track is assumed to produce a signal in the detector. Each Cherenkov photon that hits the anode plane and is sampled according to the quantum efficiency of the CsI photocathode is assumed to produce a signal. The charged tracks, including Cherenkov photon-electron signals, are summed as the RICH rate and listed in Table~\ref{tab:pidbackground}. Radiative Bhabha scattering is the dominant source of background. On average, the occupancy is approximately $7.3\times10^{-5}$~Hz for a $5\times5$~mm$^2$ anode pad, which is smaller than the previous $10^{-3}$ background estimation from Eq.~\ref{PIDlikelihood}.


%%%%%%%%%%%%%%%%%  TABLE  %%%%%%%%%%%%%%%%%%%%%%%%
\begin{table*}[hptb]
\small
    \caption{The background simulation for the RICH detector.}
    \label{tab:pidbackground}
    \vspace{0pt}
    \centering
        \begin{tabular}{lrrr}
        \hline
        \hline
          &  \multicolumn{1}{l}{Expected rate (Hz)} & \multicolumn{1}{l}{RICH rate (Hz)} & \multicolumn{1}{l}{Occupancy (Hz/mm$^2$)} \\
        \hline
    RBB e$^{\pm}$   &  5.98$\times 10^8$   & 1.25$\times 10^8$ & 8.36$\times 10^{-7}$ \\
    RBB $\gamma$  &  1.07$\times 10^8$    & 3.71$\times 10^6$ & negligible \\
    Two photon &  1.03$\times 10^9$          & 2.44$\times 10^7$ & 1.63$\times 10^{-7}$ \\
    Touschek &  1.12$\times 10^9$               & 5.04$\times 10^6$      & negligible \\
    Coulomb  &  2.09$\times 10^8$            & 2.90$\times 10^8$ & 1.94$\times 10^{-6}$ \\
    Brems & 2.10$\times 10^6$                  & 2.10$\times 10^2$      & negligible \\
    \hline
    \end{tabular}
\end{table*}
%%%%%%%%%%%%%%%%%%%%%%%%%%%%%%%%%%%%%%%%%%%%%%%%%%


\subsection{Detector Layout}

The RICH detector is placed between the MDC and EMC, with 0.83 solid coverage of the barrel part. The inner radius is $850$\,mm, and the outer radius is $~950$\,mm. As shown in Fig.\,\ref{FIG:RICH_Conceptual_Design}(a), the RICH detector consists of $12$ identical block modules. Each RICH module is $2400$\,mm long, $450$\,mm wide, and $130$\, mm in height. The whole module is enclosed in a light-tight aluminum box with support from each side.

Each RICH module consists of radiators, a light propagation zone, a CsI-coated THGEM layer, a MicroMegas layer, and anode pads.
A schematic view of the RICH module is shown in Fig.~\ref{FIG:RICH_Conceptual_Design}~(b).
The radiator contains $4$ quartz boxes that are $\sim600$\,mm long and $\sim450$\,mm wide. The quartz boxes are glued and sealed, the bottom layer of which is a UV transparent quartz plate. These liquid C$_6$F$_{14}$ radiators are sealed inside quartz boxes, with pipes connected one to another. A purification system is employed to continuously purify the liquid and pump it to the highest module. Modules are connected by pipes as well. The liquid is driven by gravity and flows back to the tank of the purification system. Note that due to the aluminum box and radiator quartz container, the insensitive area of barrel PID is less than $5\%$.
The light propagation zone is filled with an argon-based gas, which acts as the working gas for the photon detector. Since humidity and oxygen contamination result in the absorption of UV light, a purification system for the gas system is required.
The THGEM and Micromegas are the same size as the radiator boxes for convenience.
The readout pads are $5\times 5$\,mm$^2$. In total, there are $43200$ channels for each RICH module. The electronics are described in the next section.

\begin{figure}[!htb]
  \centering
  \subfloat[][]{\includegraphics[width=0.45\textwidth]{Figures/Figs_04_00_DetectorSubSystems/Figs_04_03_ParticleIdentification/RICH/8RICH_Conceptual_Structure.pdf}}
  \subfloat[][]{\includegraphics[width=0.45\textwidth]{Figures/Figs_04_00_DetectorSubSystems/Figs_04_03_ParticleIdentification/RICH/8radBox.pdf}}
  \caption{The conceptual design of the RICH detector: (a) the overall layout and (b) a schematic view of the module box.}
  \label{FIG:RICH_Conceptual_Design}
\end{figure}


\subsection{Readout Electronics}
\subsubsection{Design Overview}
High precision time ($~1$\,ns RMS @ $48$\,fC) and charge ($~1$\,fC RMS) measurements are required for the RICH readout electronics, with an input capacitance of approximately $20$\,pF. The total number of channels is estimated to be approximately $518,400$.

The structure of the readout electronics of the PID RICH detector is illustrated in Fig.\,\ref{fig:rich-elec} and is composed of FEE and RUs. Multiple front-end ASICs, which perform analog signal manipulation and A/D conversion, are integrated into one FEE module, and the output data of these ASICs are transferred to a digital ASIC or FPGA, which is responsible for data packaging and transferring the data to the RU through a high-speed serial data interface. The RU receives the data from multiple FEE modules and finally transfers them to the DAQ.

RICH electronics do not participate in the generation of the global trigger signal. The RICH readout electronics must receive the global trigger signal and implement trigger matching in the FEE. The RICH detector readout electronics must also be synchronized with a global clock signal, and this clock is fanned out to the ASICs in the FEE through RUs.

\begin{figure}
\centering
\includegraphics[width=1.0\textwidth]{Figures/Figs_04_00_DetectorSubSystems/Figs_04_03_ParticleIdentification/Figs_RICH_Electronics/Block-diagram-of-RICH-electronics.pdf}
\caption{Block diagram of the RICH electronics.}
\label{fig:rich-elec}
\end{figure}

\subsubsection{Front-End ASIC}
Given the large number of readout channels and the requirements for high precision (time and charge) measurements,
the RICH electronics must feature high density, low noise, and low power consumption, and thus, it is necessary to employ a suitable ASIC that satisfies these requirements.

The block diagram of the front-end ASIC is shown in Fig.\,\ref{fig:rich-asic}. The CSA integrates the input signal and generates a signal at its output with an amplitude proportional to the input charge. This signal is then fed to a shaping circuit that outputs a semi-Gaussian pulse while enhancing the SNR. To digitize the waveform, SCAs followed by Wilkinson ADCs are integrated into the ASIC. The digitized signal is sent to the FPGA, and then we can obtain the charge information through peak detection or area calculation of the digitized waveform. Additionally, time information can be obtained using a leading-edge discrimination method. In addition, a coarse counter is designed to expand the time measurement range.

\begin{figure}[!htb]
\centering
\includegraphics[width=1.0\textwidth]{Figures/Figs_04_00_DetectorSubSystems/Figs_04_03_ParticleIdentification/Figs_RICH_Electronics/Block-diagram-of-the-Front-End-ASIC.pdf}
\caption{Block diagram of the front-end ASIC.}
\label{fig:rich-asic}
\end{figure}

\newpage