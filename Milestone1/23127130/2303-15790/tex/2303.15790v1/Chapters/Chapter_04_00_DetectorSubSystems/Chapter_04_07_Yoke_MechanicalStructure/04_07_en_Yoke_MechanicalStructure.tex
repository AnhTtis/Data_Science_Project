\section{Iron Yoke and Mechanical Structure}
\label{sec:yoke}
\subsection{Mechanical Structure of the Iron Yoke}
The iron yoke is the base structure of the MUD and the base support of the subdetector components attached to it. It has two important functions: one is absorbing all the high-energy particles except muons in the MUD, and the other is conducting a magnetic field from the superconducting solenoid that serves as leak shielding. The iron yoke material and structure must be magnetically permeable and have high mechanical strength. Low carbon steel is selected as the baseline iron yoke material and is measured piece by piece with strict requirements for permeability before machining.

%\quad\\
The iron yoke consists of a barrel part and an endcap part, both of which have a multilayer structure. The barrel yoke, as shown in Fig.~\ref{fig:4.7.04}(a), has ten layers with thicknesses of 40, 40, 45, 45, 60, 60, 60, 80, 80, and 150~mm from the innermost layer to the outermost layer. Every layer has an octagonal column configuration and is formed by eight pieces of separate iron plates. The net inner height of the 1st layer is 3.78~m and that of the 10th layer is 5.52~m.

%\quad\\
The endcap yoke, as shown in Fig.~\ref{fig:4.7.04}(b), consists of eleven layers of iron plates in an octagonal shape. Every layer is formed by two half-octagonal iron plates. The thicknesses of the endcap yoke layers are 40, 40, 40, 45, 45, 60, 60, 60, 80, 80, and 150~mm along the beampipe direction. Polar iron is situated in the center area of the endcap yoke and is separated into two halves: one half is fixed to the endcap iron, and the other half can be pushed and pulled during installation and maintenance. An adequate gap between iron plates is maintained for the installation of cables and pipes.

%%%%%%%%%%%%%%%%%%% Fig %%%%%%%%%%%%%%%%%%%%%%%%%%
\begin{figure*}[htb]
	\centering
\subfloat[][]{\includegraphics[height=0.45\linewidth]{Figures/Figs_04_00_DetectorSubSystems/Figs_04_07_Yoke&MechanicalStructure/york.png}}
\subfloat[][]{\includegraphics[height=0.45\linewidth]{Figures/Figs_04_00_DetectorSubSystems/Figs_04_07_Yoke&MechanicalStructure/endcap.png}}
\vspace{0cm}
\caption{A schematic view of the iron yoke in the (a) barrel and (b) endcap regions.}
    \label{fig:4.7.04}
\end{figure*}
%%%%%%%%%%%%%%%%%%%%%%%%%%%%%%%%%%%%%%%%%%%%%%%%%%



%\quad\\
The total weight of the iron yoke is up to approximately 800~tons. The iron yoke is subject to a large electrodynamic force when the superconducting solenoid is in operation. The structure of the iron yoke should be strong enough to resist this strong force. Welding is avoided in the iron plate connections to mitigate deformation effects. Bolting is convenient for position adjustment. The dead zone of the MUD should be minimized when considering connections.

%\quad\\
The general structure specifications of the iron yoke are as follows:
\begin{itemize}
\item The center of the iron yoke geometry should be below 3.9~m.
\item The final assembly of the iron yoke will be completed in the detector hall, and the yoke will be transported to and positioned at the positron-electron collision point.
\item The interior areas of the iron yoke should be easily accessible, and the vacuum conditions should be maintained during maintenance.
\item The support structure of the solenoid should guarantee that the magnetic field is reproducible after detector maintenance. Electric cables should be routed along magnetic lines to avoid additional stray magnetic fields.
\item Enough space should be left for electric cables and media coming in and out of the iron yoke.
\end{itemize}

\subsection{Mechanical Movement of the Iron Yoke}
The iron yoke assembly stands on a steel base support that includes the main frame, a guiding frame, an enforcement frame and a lifting jack. The mechanical support allows the detector to move in both horizontal and vertical direction. In the final positioning step, the detector is transported horizontally to the collision point on a guiding rail and raised in the vertical direction on the hydraulically driven jack system. The position precision is guaranteed by the straightness of the guiding rail and the accuracy of the motor controller system. The base support structure should be strong enough to withstand the load of the whole detector, which may be as heavy as 800~tons. At the end of the guiding rail, a disc spring is mounted as a buffer to prevent the detector from going out of range.

The endcap iron yoke can be opened and closed to allow people to enter the detector to perform maintenance. To make this easy, half of the polar ring iron in the center is pulled out before the endcap iron opens, and the four endcap iron yokes move independently and are controlled by servo motors. On the top of the detector, extra guiding rails are mounted to increase the position precision and stability of the movement.


\subsection{Detector Hall and Infrastructure}
The detector hall is the main place where the whole setup will be assembled and the detector will be positioned in its final working location. The requirements of the detector hall are as follows:
\begin{itemize}
\item The detector hall dimensions should be be no less than 20~m along the beam direction, 7~m in height, and 30~m in the transverse beam direction.
\item The collision point should be approximately 3.9~m above the ground floor.
\item The load capacity of the ground floor should be 400~kN/m${^2}$.
\item The sum of the floor vibration amplitude should be less than a $\mu$m in the range of $1\sim60$~Hz.
\item The ceiling crane should have a 50~ton lifting load capacity.
\item The main gate should be no less than 5.5~m in width and 6.0~m in height. The dimensions of the detector component and subsystem that need to be brought into the hall through the main gate are restricted by the gate size.
\end{itemize}


%%%%%%%%%%%%%%%%%%% Fig %%%%%%%%%%%%%%%%%%%%%%%%%%
\begin{figure*}[htb]
	\centering
    \includegraphics[height=0.6\linewidth]{Figures/Figs_04_00_DetectorSubSystems/Figs_04_07_Yoke&MechanicalStructure/CrossSection.jpg} \\
    \includegraphics[height=0.6\linewidth]{Figures/Figs_04_00_DetectorSubSystems/Figs_04_07_Yoke&MechanicalStructure/GeneralView.jpg}
\vspace{0cm}
\caption{A cross-sectional view of the detector (upper) and an overall view of the detector (lower).}
    \label{fig:4.7.03}
\end{figure*}
%%%%%%%%%%%%%%%%%%%%%%%%%%%%%%%%%%%%%%%%%%%%%%%%%%






