According to the operation environment and physics requirements of the STCF imposed on the endcap PID subsystem, a DTOF detector, we will carry out extensive $R\&D$ on the key technologies of the DTOF detector and readout electronics. In the following, the main research items are listed.
\begin{itemize}
  \item \textbf{The DTOF detector}\\
In terms of achieving high-precision time measurement, key technologies of the DTOF detector include the optical design of the radiator and photosensor, fine processing of a large area radiator (fused silica) and fast-response photoelectric detection technology.
The design of the radiator structure, especially its optical performance, is very important for a DIRC detector. Another technical challenge is the high-precision processing and surface control of large-area radiators, which will be studied in cooperation with the Beijing Special Glass Research Institute.
Research on fast-response photoelectric detection technology includes the design and development of MCP-PMT readout circuits and the suppression of signal oscillation and crosstalk.

  \item \textbf{Readout electronics}\\
The output signals of the MCP-PMTs in the DTOF detector demonstrate a very short rise time and a very narrow width, which poses a significant challenge for high-precision readout electronics. To fully exploit the timing potential of the MCP-PMTs, the following $R\&D$ topics are important: multithreshold high-precision timing technology based on an FPGA TDC, including studies on the ultrafast signal processing and timing circuits; high-precision and highly integrated TDC technology development based on FPGAs; application specific integrated circuit (ASIC) development for the DTOF detector; multichannel high-precision clock synchronization and distribution technology; and the design and realization of high-bandwidth data acquisition system.

  \item \textbf{DTOF prototyping} \\
Verifications of the key technologies and the system-level detector performance are important to demonstrate the feasibility of the DTOF detector. Through the operation of a large-size DTOF prototype and comparison of experimental test data to simulation results, the performance of the technology and the important features in the design, processing, installation and testing of the DTOF detector will be explored and understood. Further optimization of the DTOF engineering design can be achieved to fulfill the required PID capabilities of the STCF experiment.

  \item \textbf{Radiation resistance and aging}\\
In STCF operation, the spectrometer system will encounter an unprecedented radiation dose, which poses a great challenge in terms of the radiation resistance of the detector and the front-end electronics (FEE). Particularly important are the radiation hardness of the ASIC chip and the aging properties of the photosensors, ({\it i.e.}, the MCP-PMTs). The R\&D of the ASIC chip needs to include radiation reinforcement technology.
\end{itemize}
