The STCF PID barrel detector, a RICH detector, is designed to provide a $\pi/K$ misidentification rate lower than $2\%$ in the momentum range up to $p=2.0\gevc$, with the corresponding identification efficiency being larger than 97\% under a high luminosity environment. Comprehensive research is needed on the following items.

\begin{itemize}
\item \textbf{Radiator study}

The radiator is essential for the RICH detector. For the STCF momentum range, liquid C$_6$F$_{14}$ is chosen as the radiator.
Although this type of radiator has been successfully used in the ALICE experiment, several aspects need further study to fulfill the STCF requirements, for example, the refractive index varies with pressure and temperature and needs to be calibrated, and the UV transmission rate and the radiation hardness need to be investigated.
Additionally, purification systems and online monitoring systems are essential to guarantee long-term stable operation and need to be further researched.
In addition, research on new radiators, such as photonic crystals and silicon aerogels, is being carried out.

\item \textbf{Large area gaseous UV-photon detector} \\
The hybrid micropattern gaseous detector THGEM with a CsI-coated photocathode and Micromegas is the baseline design for the large-area UV-photon detector.
To achieve a high detection efficiency for Cherenkov radiation, the quantum efficiency of the CsI cathode should be relatively high and uniform.
Thus, it is necessary to study the decrease in quantum efficiency for different working gases and electric extraction fields and different pre- and post-treatments for the CsI coating.
The amplification of the gaseous detector must be large enough for single photon-electron detection, and the ion back flow must be low.
The detector also must be expandable with minimum dead space.

\item \textbf{Readout electronics} \\
Considering the large channel number, on the order of 10$^4$, and the requirement of high-precision signal measurements, it is preferable to implement an application-specific integrated circuit (ASIC) that can integrate front-end analog circuits and analog-to-digital conversion circuits within the chips to reduce the system complexity.

\item \textbf{RICH prototype} \\
To verify the performance and PID capabilities of the RICH detector, a large RICH prototype will be developed to demonstrate the feasibility of the detector.
The prototype will be expandable toward future STCF experiments with a connected liquid radiator purification system.
A beam test for a small prototype with $160\times160\,$mm$^2$ sensitive area has been performed at DESY with $5$ GeV/$c$ electron beam. Full-size module will be built and the beam test will be performed. The result will be compared with Monte Carlo simulations.
The FEE and DAQ electronics will also be commissioned with the prototype.

\end{itemize}
