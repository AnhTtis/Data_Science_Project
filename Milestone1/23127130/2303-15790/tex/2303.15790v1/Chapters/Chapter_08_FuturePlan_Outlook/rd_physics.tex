The physical potentials at the STCF are presented in CDR Volume I.
During R\&D, it is necessary to provide a feasibility study of these physics
programs under the software framework developed at the STCF, which can be achieved
by establishing and optimizing the reconstruction algorithm, developing the physics
tools for partial-wave analysis or Argand diagram analysis, analyzing the systematic
uncertainty sources and finally extracting the key physical parameters.
The physics simulation will be expanded in terms the following three aspects.
\begin{itemize}
\item The hadron spectrum and hadronic structure, including studies on properties
of $XYZ$ particles; spectrum analysis of light hadrons; charmonium states and
charm hadrons; precise tests of SM parameters such as
muon anomaly magnetic moments, R values and $\tau$-mass; and
hadronic structures from electromagnetic form factors and fragmentation functions. At higher c.m.~energy $\sqrt{s}>5$~GeV, a feasibility study of searching for penta-quark, doubly charm baryon, and di-charmonium production will be carried out under
the guidance of theory. Useful physical tools will be developed during the R\&D for
a highly efficient physics analysis.

\item Flavor physics and $CP$ violations.
The STCF will be a flavor factory with very large amounts of charm hadrons produced.
In R\&D, a sensitivity study of the fundamental parameters will be performed
including the CKM matrix elements $|V_{cs}|$, $|V_{cd}|$,
$\gamma/\phi_{3}$ of the
CKM triangle, $D^{0}-\bar{D}^{0}$ mixing, decay of charm hadrons, etc.
In addition, the sensitivity of $CP$ violations at the STCF from
various aspects in the hyperon, $\tau$-lepton and charm hadron sectors
will be studied under unpolarized and polarized electron beams.
In these studies, the sources of systematic uncertainty should be
carefully examined to match the unprecedented statistical
uncertainty accuracy.

\item Probing of new physics beyond the SM. During R\&D, the mixing strengths of new particles, such as dark photons and millicharged particles, will be studied at the STCF
to test various models beyond the SM.
Sensitivity studies of the processes that violate quantum number conservation
such as lepton flavor violations, lepton number violations or baryon number violations,
and flavor-changing-neutral-current processes
will be carried out. Moreover, the rare decays from $J/\psi$, charm meson decays will be
studied with large data samples. The background needs to be carefully analyzed
in these studies to achieve a high level of sensitivity.
This is expected to extend the reach of the current experimental efforts in both the energy
and intensity frontiers and
to make several quantitative estimations
of the sensitivities in the probing of these new physics to test several scenarios beyond the SM.


\end{itemize}



