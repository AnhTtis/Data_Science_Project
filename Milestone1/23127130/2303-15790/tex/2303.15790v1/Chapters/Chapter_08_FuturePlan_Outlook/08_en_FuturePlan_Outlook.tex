\chapter{Future Plans and R\&D Prospects}
\label{chap_plan}
\label{sec:future_RD}
\label{sec:cost_timeline}

As presented in this CDR, substantial efforts on detector R\&D and physics performance are underway.
It has been demonstrated that the proposed conceptual detectors can meet the physics requirements and can feasibly be built on the timescale of the STCF project.
Moving forward, more in-depth R\&D and studies are required for the preparation of the Technical Design Report~(TDR).
These include sharpening the physics case, finalizing detector technological choices, further optimizing to improve performance and
reduce cost, designing mechanical, electrical and thermal systems, and developing installation and integration schemes.


\section{Costs and Project Timeline}
\label{sec:costs}
\label{sec:timeline}

The current estimates of the cost of the STCF project are given in Table~\ref{tab:stcf_cost}.

\begin{table*}[htb]
    \caption{Estimated cost of the STCF project.}
    \label{tab:stcf_cost}
    \centering
    \begin{tabular}{lr}
        \hline
        Component  & Estimated cost (million RMB) \\
        \hline
        ITK (Silicon)   & 30 \\ 
        ITK ($\mu$RWELL) & 6  \\ 
        MDC & 13 \\ 
        PID (RICH) & 51 \\ 
        PID (DTOF) & 57 \\ 
        EMC & 260 \\ 
        MUD & 10 \\ 
        Yoke & 28 \\ 
        Magnet & 62 \\ 
        \hline
        Trigger \& DAQ & 50 \\ 
        \hline
        Accessories & 10 \\ 
        \hline
        Total & 550 \\ 
        \hline
    \end{tabular}
\end{table*}


The current timeline for the STCF project is shown in Fig.~\ref{fig:stcf_timeline}.
The first stage of the project is the completion of this CDR.
The next stage is the 5-year period for detector R\&D toward the completion of the Technical Design Report (TDR).
The construction of both the accelerator and the detectors will commence immediately after the approval of the TDR and is expected to take approximately 7 years.
An approximately 15 year data-taking period is envisaged to achieve the physics targets of the STCF, and there is also a possibility to further upgrade the STCF project with polarized electron beam during the data-taking period.

\begin{figure}[htbp]
\begin{center}
\includegraphics[width=0.8\textwidth]{Figures/Figs_08_FuterePlan/STCF_timeline.pdf}
\caption{Expected timeline for the STCF project.}
\label{fig:stcf_timeline}
\end{center}
\end{figure}

\section{R\&D Prospects}


\subsection{ITK}
\label{sec:rd_itk}
The inner tracker is designed to have a low material budget and fast readout and to provide high detection efficiencies for charged particles,
especially those with very low momentum, below 100 MeV/c.
It must also satisfy the stringent requirements imposed by the high luminosity and trigger rate conditions.
Two conceptual designs of the ITK have been proposed: a cylindrical $\mu$RWELL-based detector and a monolithic active pixel sensor-based detector.
Several critical R\&D items have been identified for the TDR:
\begin{itemize}
  \item $\mu$RWELL-based ITK:
    \begin{itemize}
      \item Manufacturing methods for large area diamond-like carbon (DLC) resistive electrodes;
      \item Electrode structure design and manufacturing methods for a large-area WELL detector and a large-area canal type (GROOVE) detector;
      \item Mechanical structure and installation method for a cylindrical detector;
      \item Design of front-end ASICs to integrate analog signal manipulation, ADC, charge and time calculations, and a high-speed data transfer interface;
      \item Design of the readout electronics, including the front-end electronics with customized ASICs, readout units, and subclock and subtrigger modules;
      \item Design, construction and characterization of a small cylindrical $\mu$RWELL detector prototype.
    \end{itemize}
  \item MAPS-based ITK:
    \begin{itemize}
      \item Track reconstruction performance study and further optimization of the layout;
      \item Design of the MAPS pixel layout with optimal sensor parameters, such as low capacitance, high charge collection efficiency and short charge collection time;
      \item Design of a low-power low-noise in-pixel circuit;
      \item Design of the readout strategy for the hit signals and the architecture of on-chip readout circuitry;
      \item Design of the readout system, including readout units, power units and common readout units, for data receiving, collecting, transferring, and configuring of the MAPS;
      \item Design of the stave module, including support mechanics and a cooling system;
      \item Design, construction and characterization of a MAPS stave prototype module.
    \end{itemize}
\end{itemize}


\subsection{MDC}
\label{sec:rd_mdc}
The main drift chamber (MDC) is the central part of the tracking system of the STCF detector. The key factors affecting the tracking performance in the STCF experiment are multiple scattering and energy loss of charged particles traversing the MDC. Therefore, the driving force in the design of the MDC is reducing its material budget as much as possible. This also represents the core of the R\&D work required for the MDC detector. The MDC provides ionization measurements for charged particle identification as well as precise position measurements for charged particle tracking. Thus, precise time and charge measurements are both required for the MDC readout electronics. In addition, the high counting rates expected at the MDC inner layers and the extremely high rate of physics events of interest expected when running at the $J/\psi$ peak impose stringent requirements on the MDC readout electronics as well as the MDC detector. A vigorous R\&D program needs to be developed and carried out for the MDC to meet all the technical challenges. Such an R\&D program should cover the following aspects where some critical R\&D items have been identified.
\begin{itemize}
\item \textbf{Detector design optimization}\\
The material budget of the detector should be maximally reduced to enhance the tracking performance for low-momentum particles by optimizing the detector design in terms of various aspects, including the working gas, wire material and size, configuration of wire layers, chamber structure and material. Drift cells that are much smaller than regular small cells and yet maintain a very low material budget for the whole detector should be designed. The adoption of such drift cells would shorten the maximum drift time in a drift cell and thus make the response of the detector faster, allowing it to better cope with the high count rate and physics event rate. The structure  of the drift chamber should be designed, including a detailed deformation analysis that fully takes into account wire tensions and their possible creep effects.

\item \textbf{R\&D of low-mass wires} \\
Electrode wires are one of the primary contributors to the material budget of a drift chamber. It is therefore of great importance to develop low-mass wires for use in a drift chamber. Invention of such low-mass wires could represent a breakthrough in drift chamber technology. The use of low-mass wires would also require much less tension to be applied to the wires than necessary when using regular metal wires. This would effectively reduce the total load on the endplates of the drift chamber due to the wire tension and hence leave much more room for the design and engineering of a light chamber structure. Ideas regarding using light polymeric fibers or carbon monofilaments coated with low-mass metals as wires for a drift chamber have been proposed and are being explored by some Italian groups. This could serve as one direction to pursue in the R\&D of low-mass wires for the STCF drift chamber.

\item \textbf{R\&D of high-density wiring}\\
Very small drift cells imply a very large number of closely spaced wires in a drift chamber. This poses a great challenge to wiring the drift chamber, which includes threading wires through the chamber and fixing them at the endplates. In this case, manual wiring is unlikely to be adequate, and regular feedthroughs can no longer be used to hold wires. Thus, an automatic wiring method along with the corresponding key devices need to be developed to enable efficient and accurate wiring operations. A novel method to fix wires without feedthroughs also needs to be developed.

\item \textbf{R\&D of readout electronics}\\
The MDC readout electronics consist of a transimpedance amplifier (TIA) followed by a shaping circuit and an analog-to-digital converter (ADC). The digitized signals are further filtered and processed to reduce the pile-up and enhance the SNR by a data processing circuit. The charge and time information of the signals are extracted with the processed data. The hardware components of the MDC readout electronics include front-end electronics (FEE), readout units (RUs), and subclock and subtrigger modules. Dedicated R\&D work on the readout electronics is needed to meet the requirements for precise time and charge measurements under high rate conditions. Two technical approaches are planned for R\&D, one based on discrete devices and the other on ASIC chips. The primary weight would be placed on the design and development of an ASIC chip suitable for the STCF drift chamber. The proposed ASIC chip incorporates a TIA, a discriminator and a shaping circuit. The transistor parameters and feedback resistance of the TIA should be carefully optimized in terms of circuit input impedance, bandwidth, and dynamic range.

\item \textbf{Development and characterization of a full-size drift chamber prototype}\\
A full-length drift chamber prototype designed for the STCF needs to be developed using all the key techniques and components developed for the drift chambers. The design of the prototype should closely follow the optimized design of the STCF drift chamber. The prototype would also be fully instrumented with the readout electronics developed for the STCF drift chamber. The prototype along with its readout electronics will be fully tested and characterized to validate the design of the STCF MDC and the key techniques and components of the drift chamber.
\end{itemize}



\subsection{RICH}
\label{sec:rd_rich}
The STCF PID barrel detector, a RICH detector, is designed to provide a $\pi/K$ misidentification rate lower than $2\%$ in the momentum range up to $p=2.0\gevc$, with the corresponding identification efficiency being larger than 97\% under a high luminosity environment. Comprehensive research is needed on the following items.

\begin{itemize}
\item \textbf{Radiator study}

The radiator is essential for the RICH detector. For the STCF momentum range, liquid C$_6$F$_{14}$ is chosen as the radiator.
Although this type of radiator has been successfully used in the ALICE experiment, several aspects need further study to fulfill the STCF requirements, for example, the refractive index varies with pressure and temperature and needs to be calibrated, and the UV transmission rate and the radiation hardness need to be investigated.
Additionally, purification systems and online monitoring systems are essential to guarantee long-term stable operation and need to be further researched.
In addition, research on new radiators, such as photonic crystals and silicon aerogels, is being carried out.

\item \textbf{Large area gaseous UV-photon detector} \\
The hybrid micropattern gaseous detector THGEM with a CsI-coated photocathode and Micromegas is the baseline design for the large-area UV-photon detector.
To achieve a high detection efficiency for Cherenkov radiation, the quantum efficiency of the CsI cathode should be relatively high and uniform.
Thus, it is necessary to study the decrease in quantum efficiency for different working gases and electric extraction fields and different pre- and post-treatments for the CsI coating.
The amplification of the gaseous detector must be large enough for single photon-electron detection, and the ion back flow must be low.
The detector also must be expandable with minimum dead space.

\item \textbf{Readout electronics} \\
Considering the large channel number, on the order of 10$^4$, and the requirement of high-precision signal measurements, it is preferable to implement an application-specific integrated circuit (ASIC) that can integrate front-end analog circuits and analog-to-digital conversion circuits within the chips to reduce the system complexity.

\item \textbf{RICH prototype} \\
To verify the performance and PID capabilities of the RICH detector, a large RICH prototype will be developed to demonstrate the feasibility of the detector.
The prototype will be expandable toward future STCF experiments with a connected liquid radiator purification system.
A beam test for a small prototype with $160\times160\,$mm$^2$ sensitive area has been performed at DESY with $5$ GeV/$c$ electron beam. Full-size module will be built and the beam test will be performed. The result will be compared with Monte Carlo simulations.
The FEE and DAQ electronics will also be commissioned with the prototype.

\end{itemize}


\subsection{DTOF}
\label{sec:rd_dtof}
According to the operation environment and physics requirements of the STCF imposed on the endcap PID subsystem, a DTOF detector, we will carry out extensive $R\&D$ on the key technologies of the DTOF detector and readout electronics. In the following, the main research items are listed.
\begin{itemize}
  \item \textbf{The DTOF detector}\\
In terms of achieving high-precision time measurement, key technologies of the DTOF detector include the optical design of the radiator and photosensor, fine processing of a large area radiator (fused silica) and fast-response photoelectric detection technology.
The design of the radiator structure, especially its optical performance, is very important for a DIRC detector. Another technical challenge is the high-precision processing and surface control of large-area radiators, which will be studied in cooperation with the Beijing Special Glass Research Institute.
Research on fast-response photoelectric detection technology includes the design and development of MCP-PMT readout circuits and the suppression of signal oscillation and crosstalk.

  \item \textbf{Readout electronics}\\
The output signals of the MCP-PMTs in the DTOF detector demonstrate a very short rise time and a very narrow width, which poses a significant challenge for high-precision readout electronics. To fully exploit the timing potential of the MCP-PMTs, the following $R\&D$ topics are important: multithreshold high-precision timing technology based on an FPGA TDC, including studies on the ultrafast signal processing and timing circuits; high-precision and highly integrated TDC technology development based on FPGAs; application specific integrated circuit (ASIC) development for the DTOF detector; multichannel high-precision clock synchronization and distribution technology; and the design and realization of high-bandwidth data acquisition system.

  \item \textbf{DTOF prototyping} \\
Verifications of the key technologies and the system-level detector performance are important to demonstrate the feasibility of the DTOF detector. Through the operation of a large-size DTOF prototype and comparison of experimental test data to simulation results, the performance of the technology and the important features in the design, processing, installation and testing of the DTOF detector will be explored and understood. Further optimization of the DTOF engineering design can be achieved to fulfill the required PID capabilities of the STCF experiment.

  \item \textbf{Radiation resistance and aging}\\
In STCF operation, the spectrometer system will encounter an unprecedented radiation dose, which poses a great challenge in terms of the radiation resistance of the detector and the front-end electronics (FEE). Particularly important are the radiation hardness of the ASIC chip and the aging properties of the photosensors, ({\it i.e.}, the MCP-PMTs). The R\&D of the ASIC chip needs to include radiation reinforcement technology.
\end{itemize}


\subsection{EMC}
\label{sec:rd_emc}
The main function of the electromagnetic calorimeter (EMC) of the STCF is to realize accurate measurements of photon energy, position and arrival time under the conditions of high background count rates. For 1~GeV photons, the energy resolution should be better than 2.5\%, and the position resolution should be better than 5~mm. In addition, the time resolution should reach 300~ps @ 1~GeV to distinguish neutral particles (neutrons, photons). The pure cesium iodide crystal (pCsI) has the advantages of radiation hardness and fast time response. It is a very promising option for the EMC of the STCF spectrometer. In the following, the main R\&D items are listed.
\begin{itemize}
\item \textbf{Light Yield Study}\\
For the measurement of low-energy photons, the light yield of a crystal directly determines its energy measurement accuracy. It is necessary to study the crystal light yield according to the fluorescence characteristics of the pCsI crystal (wavelength band, decay time, etc.). On the basis of previous research, further study on wavelength shifting materials is likely to improve the light yield of the pCsI crystal.


  \item \textbf{Electronics Study}\\
The EMC readout electronics mainly include two modules: a preamplifier module and a digital processing module. The preamplifier module includes a photoelectric conversion device, an APD, and a front-end amplification circuit.
According to the requirements for high precision and large dynamic range detection of the EMC, a design yielding low noise and large dynamic range needs to be obtained for the preamplifier module. Additionally, the preamplifier module needs an antistacking design to solve the problem of preamplifier circuit saturation caused by a high background and high signal counting rate.

  \item \textbf{Pileup Study}\\
To cope with the high background, in addition to reducing the recovery time of the readout circuit as much as possible in the hardware design, the analysis method still needs to be studied.
It is planned to use a multiwaveform fitting method in the readout of the EMC, and it is also essential to verify the feasibility of this method with simulations and experiments.

  \item \textbf{Time Measurement Study}\\
It is planned to use waveform sampling readout and online waveform fitting to realize high-precision time measurement based on the premise of controllable data size. The detailed algorithm and technical implementation need to be studied.

\item \textbf{Prototype Study}\\
To verify the performance of the EMC, a small prototype needs to be developed, such as a 3 $\times$ 3 or 5 $\times$ 5 crystal array. This includes batch testing of crystals, photoelectronic devices and electronics, and finally, the test process and test standards must be established.
\end{itemize}


\subsection{MUD}
\label{sec:rd_mud}
According to the intensive background at the STCF, the conceptual design of the muon detector (MUD) adopts an innovative approach combining an inner RPC detector and an outer plastic scintillator (PS) detector to avoid interference from background while retaining the required muon-ID abilities. To realize such a design, R\&D on various subjects is necessary, including PS+SiPM technology, RPC technology and readout electronics. A large prototype of the hybrid MUD is needed to verify the design and basic performance of the detector. The studies will involve the following:
\begin{itemize}
\item R\&D of a high-rate (up to approximately 100~kHz/channel) and large-area (size $>0.5$~m$^2$ and length $>1$~m for a single module) RPC detector;

\item R\&D of the large-size (size $>0.5$~m$^2$ and length $>1$~m for a single module) MUD module based on a plastic scintillator + wavelength shifting fiber (WLS) + SiPM technology;

\item R\&D of an electronic system suitable for both RPC and SiPM signal processing, which is required to exhibit a compact structure, low power consumption, high efficiency, radiation resistance and precision timing ($<100$~ps);

\item Construction and testing of an MUD prototype with 6 or more layers, equipped with a complete readout electronics system, to study the feasibility of the MUD.

\item Study on the MUD performance in a high rate and high radiation environment to explore the dependence of muon-ID abilities with a high background level.
\end{itemize}


\subsection{Magnet}
\label{sec:rd_magnet}
The proposed 1~T solenoid with a 3~m diameter bore for the STCF detector solenoid magnet can be realized by adopting a self-supporting aluminum stabilized low temperature NbTi superconductor. However, a low-mass superconductor and thin-wall structure are important to make the solenoid more transparent so that particles can more easily cross the solenoid.
R\&D activities will focus on key technologies such as special superconductors, large superconducting coil manufacturing processes and liquid helium thermosiphon cooling.

\begin{itemize}
\item In the first stage, to develop a special low-temperature superconductor, an innovative coextruding technique will be used. A Rutherford cable that consists of strands of NbTi wires will be inserted into a pure aluminum stabilizer, forming an aluminum stabilized cable. This cable will then be inserted into high mechanical strength aluminum alloy reinforcement. The critical current $I_c$ must exceed 6~kA@4.2 K@4T.

\item In the second stage, automatic winding equipment with the ability to wind a superconducting coil with a 3 m aperture will be developed.

\item In the third stage, a method of liquid helium thermosiphon cooling for large superconducting solenoids will be developed. To study the phase transition process of helium in the circuit, the changes in the temperature distribution and the density distribution over time, a superconducting prototype of a suitable-scale thermosiphon circuit will be established for simulation and verification before formal solenoid construction.

\end{itemize}


\subsection{TDAQ}
\label{sec:rd_tdaq}
The data acquisition (DAQ) system performs the data processing and the system control of the STCF experiment, such as the L1 trigger data readout, data compression, information extraction, event building, high-level trigger (HLT) computing, parameter configuration, status monitoring, and online data decimating. In the STCF, the DAQ system is expected to process the data from $\sim$50.6 M FEE channels at an L1 trigger rate of approximately 400~kHz, with a total data rate after the L1 trigger of $\sim30$~GB/s.
Aiming at such design goals, the following studies on the STCF DAQ system design are being planned:
\begin{itemize}
\item Design of high-performance electronic modules used in the DAQ system to provide more interfaces, higher transmission speed, more resources, and lower average power consumption and cost for each channel;
\item High-speed data transmission and processing techniques supported by high-speed interfaces, high-speed networks, high-performance computing systems, high-performance firmware and software and high-speed storage techniques;
\item Design of the DAQ architecture to perform real-time trigger generation, trigger matching, and event building at a high trigger rate level of $10^5\sim 10^6$~Hz, supporting both the trigger mode and trigger-less mode;
\item The trigger algorithm and its real-time hardware or software implementation techniques in trigger-less mode;
\item Techniques that can improve the robustness, reliability, maintainability and scalability of the system;
\item Design of system operation and management of both the DAQ system and other related systems, such as the trigger system (under trigger mode), slow control system, fast control system, and online system.
\end{itemize}


\subsection{Software}
\label{sec:rd_software}
Offline software is designed and developed for Monte Carlo simulation and offline data processing. It mainly consists of a software framework, detector simulation, calibration and reconstruction as well as physics analysis tools. In the prestudy phase of the STCF experiment, offline software is deployed to optimize the detector options and study the detector performance as well as the physics potentials. After the experiment is running, it will be used to conduct complicated offline data processing on the data collected with the detector and to convert them into physics results.

The STCF, with a peak luminosity of $\stcflum$ or higher, producing a data sample approximately 100 times larger than current $\tau$-charm factories, presents a very large challenge for offline software and computing in terms of both rate and complexity; therefore, a specific offline software needs to be redesigned and developed with the state-of-art technologies to meet the STCF requirements. The main tasks are listed below.

\begin{itemize}

\item \textbf{Development of a high-performance software framework} \\
One of the great challenges of the STCF offline software is the management and processing of a higher volume of data (approximately several petabytes per day) than the present $\tau$-charm factories; the software framework, providing common functions for offline data processing and integrating all the applications into the unified software platform, plays a very important role. Therefore, one of the most crucial tasks is to design and develop a new framework that supports heterogeneous computing, including algorithms, data models and workflows running in heterogeneous environments, and provides interfaces to new toolkits, such as machine learning toolkits, IO systems, and simulation engines.

\item \textbf{Development of fast and accurate detector simulation suites} \\
Detector simulations serve many purposes at each point in the lifecycle of the STCF facility. The new toolkit DD4hep is adopted for the detector description, including the ITK, MDC, RICH, DTOF, EMC and MUD, as well as the MDI and support systems. A detector geometry management system is needed to manage different versions of subdetector options, to support fast iterations of detector performance studies and to provide consistent geometry information for different applications, such as detector simulation, reconstruction and visualization. Further study on the accuracy of detector description, the physics interaction of the different types of particles with the detector medium, and a realistic electronics response is the part of the detector simulation most crucial to achieving a high degree of compliance between the simulation and the data. Additionally, it is necessary to explore emerging technologies, such as parallel computing, heterogeneous computing and machine learning toolkits, to speed up the detector simulation and improve its performance.


\item \textbf{Development of the calibration methods and algorithms}\\
The main task is to study the calibration methods for the key measurements from each subdetector, such as the relationship between drift distance and drift time, event start time, energy loss of the MDC, refractive index of the PID radiator, energy and position of the shower from the EMC, and noise level of the MUD, to develop the corresponding calibration algorithms and establish the complete calibration system to perform accurate conversions between electronic readouts and physical quantities, minimizing the influence of external factors of the experiment and the operating status of the detector itself on the physical measurements.


\item \textbf{Development of event reconstruction methods and algorithms}\\
Event reconstruction is a very complicated and challenging task in offline data processing, including reconstruction of the charged tracks, electromagnetic showers and particle identifications to produce the momentum, energy and type of the particles for further physics analysis. For the charge tracks, we develop a track finding method with conformal transformation and Hough transformation, a track fitting method based on the deterministic annealing filter (DAF) and track extrapolation based on Geant4. The likelihood-based PID methods are also studied for the RICH and DTOF detectors. The procedure for the EMC shower is also built up from clustering, seed finding, cluster splitting and the correction of the energy. For the MUD, one algorithm is developed based on the BDT method. Further study of these methods is crucial to achieving the design specifications of the detector hardware.

\item \textbf{Development of physics simulation software}\\
In the prestudy stage of the STCF, a parameterized (fast) simulation toolkit is necessary for detector optimization and determining the physics potential capabilities of the STCF. The fast simulation takes as inputs the response of physical objects in each subsystem of the detector, including the resolution, efficiency and related variables for the kinematic fit and the secondary vertex reconstruction algorithm. Therefore, the physics signal significance can be used as a metric to evaluate the detector options and the physics reach studies. The further optimization of the current fast simulation tool according to the new requirements of the physics study and detector design is one of the key tasks of offline software.

\end{itemize}


\subsection{Physics}
\label{sec:rd_physics}
The physical potentials at the STCF are presented in CDR Volume I.
During R\&D, it is necessary to provide a feasibility study of these physics
programs under the software framework developed at the STCF, which can be achieved
by establishing and optimizing the reconstruction algorithm, developing the physics
tools for partial-wave analysis or Argand diagram analysis, analyzing the systematic
uncertainty sources and finally extracting the key physical parameters.
The physics simulation will be expanded in terms the following three aspects.
\begin{itemize}
\item The hadron spectrum and hadronic structure, including studies on properties
of $XYZ$ particles; spectrum analysis of light hadrons; charmonium states and
charm hadrons; precise tests of SM parameters such as
muon anomaly magnetic moments, R values and $\tau$-mass; and
hadronic structures from electromagnetic form factors and fragmentation functions. At higher c.m.~energy $\sqrt{s}>5$~GeV, a feasibility study of searching for penta-quark, doubly charm baryon, and di-charmonium production will be carried out under
the guidance of theory. Useful physical tools will be developed during the R\&D for
a highly efficient physics analysis.

\item Flavor physics and $CP$ violations.
The STCF will be a flavor factory with very large amounts of charm hadrons produced.
In R\&D, a sensitivity study of the fundamental parameters will be performed
including the CKM matrix elements $|V_{cs}|$, $|V_{cd}|$,
$\gamma/\phi_{3}$ of the
CKM triangle, $D^{0}-\bar{D}^{0}$ mixing, decay of charm hadrons, etc.
In addition, the sensitivity of $CP$ violations at the STCF from
various aspects in the hyperon, $\tau$-lepton and charm hadron sectors
will be studied under unpolarized and polarized electron beams.
In these studies, the sources of systematic uncertainty should be
carefully examined to match the unprecedented statistical
uncertainty accuracy.

\item Probing of new physics beyond the SM. During R\&D, the mixing strengths of new particles, such as dark photons and millicharged particles, will be studied at the STCF
to test various models beyond the SM.
Sensitivity studies of the processes that violate quantum number conservation
such as lepton flavor violations, lepton number violations or baryon number violations,
and flavor-changing-neutral-current processes
will be carried out. Moreover, the rare decays from $J/\psi$, charm meson decays will be
studied with large data samples. The background needs to be carefully analyzed
in these studies to achieve a high level of sensitivity.
This is expected to extend the reach of the current experimental efforts in both the energy
and intensity frontiers and
to make several quantitative estimations
of the sensitivities in the probing of these new physics to test several scenarios beyond the SM.


\end{itemize}





