The data acquisition (DAQ) system performs the data processing and the system control of the STCF experiment, such as the L1 trigger data readout, data compression, information extraction, event building, high-level trigger (HLT) computing, parameter configuration, status monitoring, and online data decimating. In the STCF, the DAQ system is expected to process the data from $\sim$50.6 M FEE channels at an L1 trigger rate of approximately 400~kHz, with a total data rate after the L1 trigger of $\sim30$~GB/s.
Aiming at such design goals, the following studies on the STCF DAQ system design are being planned:
\begin{itemize}
\item Design of high-performance electronic modules used in the DAQ system to provide more interfaces, higher transmission speed, more resources, and lower average power consumption and cost for each channel;
\item High-speed data transmission and processing techniques supported by high-speed interfaces, high-speed networks, high-performance computing systems, high-performance firmware and software and high-speed storage techniques;
\item Design of the DAQ architecture to perform real-time trigger generation, trigger matching, and event building at a high trigger rate level of $10^5\sim 10^6$~Hz, supporting both the trigger mode and trigger-less mode;
\item The trigger algorithm and its real-time hardware or software implementation techniques in trigger-less mode;
\item Techniques that can improve the robustness, reliability, maintainability and scalability of the system;
\item Design of system operation and management of both the DAQ system and other related systems, such as the trigger system (under trigger mode), slow control system, fast control system, and online system.
\end{itemize}
