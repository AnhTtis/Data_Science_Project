\subsubsection{Form factors of hadrons}


Measuring the electromagnetic~(EM) form factors of nucleons has played an important role in exploring the inner structure
of nucleons. At present, these form factors are still the simplest structure observables for testing QCD predictions.
In the past, these form factors have been studied mostly in the space-like region. Recently, however,
such studies have been extended to the time-like region. In the time-like region, it is also possible to measure
EM form factors for baryons other than nucleons.
Currently available
time-like experiments demonstrate several puzzling features of EM form factors. Enhancement has been observed by BES \cite{BES1} near the threshold of the $p\bar p$ system. BaBar has also reported enhancement in the
$e^+ e^- \to p\bar p$, $\Lambda \bar \Lambda$, and $\Sigma^0 \bar \Sigma^0$ processes \cite{BaB}.
In the space-like region, the EM form factors of the proton and neutron have been measured with precision at the $1\sim 2\%$ level. In the time-like region, the EM form factors of the proton have a precision o $3.4\%$ \cite{Ablikim:2019eau}, while those of the neutron have an error at the $20\%$ level\cite{BESIII:2021tbq}.
At the STCF with the suggested luminosity, it will be possible to measure the EM form factors in the time-like region with a precision
%Editor: Please ensure that the intended meaning has been maintained in the above edit.
of $0.4\%$ for the
proton and $2\%$ for the neutron, comparable to that in the space-like region. Moreover,
it will be possible to study the enhancement in the production of baryon--antibaryon pairs near the threshold more precisely
to extract information about the interaction between a baryon and an antibaryon,
and it will also be possible to extend this study to the $\Lambda_c \bar \Lambda_c$ system to see whether enhancement occurs in the heavy baryon--heavy antibaryon system.
%Editor: Please ensure that the intended meaning has been maintained in the above edit.


Processes such as $\gamma\gamma\to {\rm hadrons}$ can be studied at the STCF. In addition to the general interest in photon--photon physics, particular quantities of interest are the transition form factors of mesons.
These form factors determine the leading contributions to hadronic light--light scattering, which is closely related
to the precise prediction of the interesting quantity $(g-2)_\mu$. Precise results for these form factors will greatly help to reduce the uncertainties in the contribution to $(g-2)_\mu$ from hadronic light--light scattering.



