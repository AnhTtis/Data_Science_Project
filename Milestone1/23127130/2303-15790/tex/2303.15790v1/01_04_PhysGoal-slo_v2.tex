\subsection{Physics potential at the STCF}

An STCF operating at CMEs ranging from 2 to $\sim$7~GeV would be of great importance to the entire field of elementary particle physics. It would address a very broad range of physics topics,
including QCD tests, hadron spectroscopy, precise tests of the electroweak sector of the
SM, and searches for new physics beyond the SM.
The proposed luminosity of the STCF is above $\stcflum$; at this level, it is expected to deliver more than 1~$\invab$ of data
samples each year. A possible data-taking plan for the STCF, along with the expected numbers of conventional
events and/or particles, is shown in Table~\ref{tablelumi}.

\par
\begin{table}[htbp]
\begin{center}
\caption{The expected numbers of events per year at different STCF energy points.}
\label{tablelumi}
\begin{tabular}{c|c|c|c|c| c}
     \hline \hline
     CME ({\rm GeV})   & Lumi ($\invab$)   &  ~~~~~~~Samples~~~~~~~ & $ ~~~\sigma ({\rm nb})~~~$  & ~No. of Events~ & Remarks \\ \hline
       3.097                  & 1                  &  $\jpsi$                &  3400 & $3.4\times 10^{12}$   \\ \hline
       3.670                  & 1                  &  $\tau^+\tau^-$         &  2.4  & $2.4\times 10^{9} $   \\ \hline
       \multirow{3}{*}{3.686} & \multirow{3}{*}{1} &  $\psip $               &  640  & $6.4\times 10^{11}$  \\
                              &                    &  $\tau^+\tau^-$         &  2.5  & $2.5\times 10^{9} $   \\
                              &                    &  $\psip\to \tau^+\tau^-$&       & $2.0\times 10^{9} $   \\ \hline
       \multirow{5}{*}{3.770} & \multirow{5}{*}{1} &  $\DDzbar $             &  3.6  & $3.6\times 10^{9} $   \\
                              &                    &  $\DpDm   $             &  2.8  & $2.8\times 10^{9} $   \\
                              &                    &  $\DDzbar $             &       & $7.9\times 10^{8} $ & Single tag \\
                              &                    &  $\DpDm   $             &       & $5.5\times 10^{8} $ & Single tag \\
                              &                    &  $\tau^+\tau^-$         & 2.9   & $2.9\times 10^{9} $\\  \hline
       \multirow{4}{*}{4.009} &  \multirow{4}{*}{1}&  $D^{*0}\bar{D}^{0}+c.c$  & 4.0  & $1.4\times 10^{9} $ & $\rm CP_{\DDzbar}=+$ \\
                              &                    & $D^{*0}\bar{D}^{0}+c.c$   & 4.0  & $2.6\times 10^{9} $ & $\rm CP_{\DDzbar}=-$ \\
                              &                    &  $\Dsp\Dsm$             & 0.20  & $2.0\times 10^{8} $ &       \\
                              &                    &  $\tau^+\tau^-$         & 3.5   & $3.5\times 10^{9} $ \\  \hline
       \multirow{3}{*}{4.180} &  \multirow{3}{*}{1}&  $\Dsps\Dsm$+c.c.       & 0.90  & $9.0\times 10^{8} $      \\
                              &                    &  $\Dsps\Dsm$+c.c.       &       & $1.3\times 10^{8} $ & Single tag          \\
                              &                    &  $\tau^+\tau^-$         & 3.6   & $3.6\times 10^{9} $  \\  \hline
       \multirow{3}{*}{4.230} &  \multirow{3}{*}{1}&  $\jpsi\pppm$           & 0.085 & $8.5\times 10^{7} $  \\
                              &                    &  $\tau^+\tau^-$         & 3.6   & $3.6\times 10^{9} $  \\
                              &                    &  $\gamma X(3872)$       &       &                      \\ \hline
       \multirow{2}{*}{4.360} &  \multirow{2}{*}{1}&  $\psip\pppm$           & 0.058 & $5.8\times 10^{7} $\\
                              &                    &  $\tau^+\tau^-$         & 3.5   & $3.5\times 10^{9} $  \\  \hline
       \multirow{2}{*}{4.420} &  \multirow{2}{*}{1}&  $\psip\pppm$           & 0.040 & $4.0\times 10^{7} $ \\
                              &                    &  $\tau^+\tau^-$         & 3.5   & $3.5\times 10^{9} $   \\  \hline
       \multirow{2}{*}{4.630} & \multirow{4}{*}{1} &  $\psip\pppm$            & 0.033& $3.3\times 10^{7} $     \\
                              &                    &  $\Lambda_c\bar\Lambda_c$& 0.56 & $5.6\times 10^{8} $  \\
                              &                    &  $\Lambda_c\bar\Lambda_c$&      & $6.4\times 10^{7} $ & Single tag \\
                              &                    &  $\tau^+\tau^-$          & 3.4  & $3.4\times 10^{9} $   \\  \hline
       4.0--7.0                & 3                  &  \multicolumn{4}{c}{300-point scan with 10~MeV steps, 1 $\invfb$/point}   \\
        $>5$                  & 2--7                &  \multicolumn{4}{c}{Several $\invab$ of high-energy data, details dependent on scan results} \\

   \hline\end{tabular}
\end{center}
\end{table}


\par



B-factory experiments and BESIII have found a striking failure of the charmonium model to provide an explanation
of the spectrum of hidden charm states with masses above $2m_D = 3.63$~GeV, which is a threshold for open charm meson
production. In addition to some conventional $c\bar{c}$ charmonium states, a larger number of unexplained
charmonium-like meson states, the so-called $XYZ$ mesons, with masses in the 3.8$\sim$5 GeV mass region, have been discovered.
These discoveries underline a glaring weakness of the SM: the lack of understanding of how QCD, the strong interaction sector of the theory that deals only with quarks and gluons, explains experimental data that involve only hadrons. In addition,
after a decade of searches, strong candidates for light non-$q\bar q$ hadrons such as glueballs and $q\bar q$--gluon QCD
hybrids with exotic spin-parity quantum numbers $J^{PC} = 1^{-+}$ have been found in large samples of radiative
$J/\psi$ decays. Both the $XYZ$ and exotic light hadrons point to entirely new hadron spectra
%Editor: Please ensure that the intended meaning has been maintained in the above edit.
that must be explored
and understood. At the moment, many of the properties of the $XYZ$ particles are unknown, and there is no clearly
identifiable pattern to the $XYZ$ particle spectrum. In certain circumstances, it is even unclear whether the
$XYZ$ resonance signals are partially or totally produced by kinematic singularities. These uncertainties prevent
us from obtaining an unambiguous mass spectrum and obscure insight into the inner structure of the $XYZ$ particles.
At the STCF, not only can large data samples of conventional particles be collected, as summarized in Table~\ref{tablelumi}, but
copious $XYZ$-particle event samples will also be produced; the expected event numbers for some of the $XYZ$ mesons are given
in Table~\ref{tableXYZ}. These large data samples will enable detailed studies of the properties of the $XYZ$ mesons through precisely studying Argand plots, searching for rare decays, and precisely measuring masses and widths, which will lead
to more conclusive results.


\begin{table}
\begin{center}
\caption{The expected numbers of $XYZ$-particle events per year  at the STCF}
\label{tableXYZ}
\begin{tabular}{c c|c|c|c|cl}
     \hline \hline
     & XYZ  &  $Y(4260)$  &  $Z_c(3900)$ & $ Z_c(4020) $   & $X(3872)$  \\ \hline
     & No. of events  & $10^{10}$                  &  $ 10^9$                &  $10^9$ & $5\times10^6$   \\ \hline
   \hline\end{tabular}
\end{center}
\end{table}



In addition to mesons containing a charmed--anticharmed quark pair, new heavy baryons containing a charmed quark and
doubly charmed baryons have been discovered, opening the way to new territories for QCD spectroscopic studies. A comprehensive
portfolio of high-precision and comprehensive measurements of these spectra could challenge and calibrate predictions from LQCD,
which is rapidly emerging as a powerful theoretical tool for performing precision first-principles QCD calculations for
long-distance phenomena. The STCF's high luminosity will help us complete the task of constructing a comprehensive and
precise spectrum of these hadrons. The extension of the STCF's high-energy coverage to approximately 7 GeV is motivated by the need
to understand the dynamics of these doubly charmed heavy baryons.

%%%

\par
With the ability to produce the large data samples indicated in Table~\ref{tablelumi}, the STCF will serve as an ideal facility
for studies of the physics of charmed hadron decays. A large $D$-meson production rate will support rigorous tests
of the SM. For example, purely leptonic decays of tagged $D^\pm$ and $D_s^{\pm}$ mesons produced in large numbers at the $\psi(3770)$
and $\psi(4040)$ (or $\psi(4160)$) resonances would enable precise measurements of the $|V_{cd}|$ and $|V_{cs}|$ matrix elements to test
the second-row unitarity of the CKM matrix and uniquely address the Cabibbo angle anomaly, i.e., the $\sim 4\sigma$ discrepancy
in the $\theta_c$ values measured in different processes~\cite{Grossman:2019bzp}. In addition, the $D^0-\bar D^0$ mixing parameters could be measured with
significantly improved precision. Measurements of and searches for rare and forbidden decays with improvements of up to two orders of magnitude
in sensitivity could be realized as part of a search for new physics.


The $\tau$, as the heaviest charged lepton, occupies a unique place in the SM. It has more decay channels than the muon and thus can provide unique access to new physics beyond the SM. At the STCF, the number of accumulated $\tau^+\tau^-$ pair events
will be approximately three orders of magnitude higher than the currently accumulated number of such events at BESIII. As many as a few billion $\tau$ pairs could be obtained in a one-year run at the CME$=2m_{\tau}$ threshold. Operation near the threshold would provide
the STCF with unique advantages over Belle II~\cite{Kou:2018napp} and LHCb~\cite{Bediaga:2018lhg}, even though the latter would have larger $\tau$-pair event samples.
For example, these events, together with well-controlled background studies using data accumulated just below the threshold, would be uniquely well suited for a high-sensitivity
study of the anomalous ($\sim$3$\sigma$) sign of $CP$ violation in $\tau \to K_S \pi \nu_\tau$ decays that was reported by
BaBar~\cite{BABAR:2011aa}.
%@article{BABAR:2011aa,
%    author = "Lees, J.P. and others",
%    collaboration = "BaBar",
%    title = "{Search for CP Violation in the Decay $\tau^- -> \pi^- K^0_S (>= 0 \pi^0) \nu_tau$}",
%    eprint = "1109.1527",
%    archivePrefix = "arXiv",
%    primaryClass = "hep-ex",
%    reportNumber = "SLAC-PUB-14556, BABAR-PUB-11-009",
%    doi = "10.1103/PhysRevD.85.031102",
%    journal = "Phys. Rev. D",
%    volume = "85",
%    pages = "031102",
%    year = "2012",
%    note = "[Erratum: Phys.Rev.D 85, 099904 (2012)]"
%}
Another unique advantage of $\tau$ pairs that are produced near the threshold is that they are primarily
produced in an $S$-wave, and thus, if the electron beam is polarized, this polarization translates nearly 100\% into
a well-understood polarization of the two final-state $\tau$ leptons~\cite{Tsai:1994rc}.
Therefore, operation of the STCF with a polarized electron beam just above the $\tau$-pair threshold would enable a high-sensitivity
search for $CP$-violating asymmetries in $\tau^{\mp}\to\pi^{\mp}\pi^0\nu$ decays~\cite{Tsai:1994rc}.
%@article{Tsai:1994rc,
%    author = "Tsai, Yung Su",
%    title = "{Production of polarized tau pairs and tests of CP violation using polarized e+- colliders near threshold}",
%    eprint = "hep-ph/9410265",
%    archivePrefix = "arXiv",
%    reportNumber = "SLAC-PUB-6685",
%    doi = "10.1103/PhysRevD.51.3172",
%    journal = "Phys. Rev. D",
%    volume = "51",
%    pages = "3172--3181",
%    year = "1995"
%}
The same data sample would also enable better determinations of the SM $\tau$-lepton parameters and stringent tests of
the lepton-flavor universality of weak interactions and might reveal possible clues toward the understanding and
study of $g-2$ for the $\tau$, which may shed light on the anomaly in $g-2$ for the muon.

The large matter--antimatter asymmetries in the $b$-quark sector observed by $B$-factory experiments confirmed
the CKM ansatz as the SM mechanism for $CP$ violation. This model can also explain the $CP$ violations that were first observed
in neutral kaon mixing and kaon decays into two and three pions. However, this mechanism fails to explain the baryon asymmetry
of the universe by approximately ten orders of magnitude, which strongly suggests the presence of additional, non-SM $CP$-violating
interactions. Promising channels for searching for new sources of $CP$ violation include the weak decays of the $\Lambda$ and $\Xi$
hyperons, where SM-CPV effects are small but effects of new, beyond-the-SM interactions could be large~\cite{Adlarson:2019jtw}. These measurements can be elegantly
done with high-statistics samples of quantum-entangled hyperon--antihyperon pair events produced via
$J/\psi \to \Lambda \bar \Lambda$ and $\Xi \bar \Xi$ decays. A one-year STCF run at the $J/\psi$ resonance
would produce data samples of 160M (60M) fully reconstructed $J/\psi \to \Lambda \bar \Lambda$ ($\Xi \bar \Xi $) events
and more than an order-of-magnitude improvement over the BESIII $CP$ sensitivity. With $\sim$80$\%$ electron beam polarization,
this sensitivity would be improved by an additional factor of four.

BESIII measurements demonstrate that the hadronic final states produced in radiative $J/\psi$ decays are replete with
QCD hybrids and glueballs and are ideally well suited for studying the spectra of these mostly unexplored systems.
Searches for anomalous weak decays of the $J/\psi$ at the STCF would have sensitivities extending all the way down to the
level of SM expectations. With STCF data runs at a variety of energies, interesting $Q^2$-dependent quantities could
be studied with high precision. These include time-like nucleon form factors, which
could be measured for $Q^2$ values as high as 50~GeV$^2$ with precisions matching those of the existing measurements in the space-like
region. The puzzling threshold behavior and peculiar oscillation patterns observed in recent low-statistics experiments could
be studied in precise detail. Moreover, unlike space-like form-factor measurements, which are possible only for the proton
and neutron, such time-like form-factor studies could be repeated for the $\Lambda$, $\Sigma$, $\Xi$ and $\Omega^-$ strange
hyperons, offering a unique new window on the baryon structure.
With this very high luminosity, high-sensitivity searches for new light particles and new interactions that are predicted by
a number of beyond-the-SM theories could be performed using decays of all of the weakly and electromagnetically decaying
particle systems that are accessible in the STCF energy range.

In short, the STCF will undoubtedly cover a very broad physics program %covering QCD tests, hadron spectroscopy, precisely tests of electroweak interactions of the SM and hunt for new physics beyond.
and could support a multidimensional program of experimental
measurements with state-of-the-art sensitivities, allowing many of the challenges of the SM to be addressed.
In the following chapters, more details on some of the highlighted physics topics that could be addressed at the STCF
are provided. These materials include discussions of research opportunities regarding particles ranging from the high-mass $XYZ$
states to low-mass systems such as hyperons, glueball/hybrid states, and possible new, beyond-the-SM light particles. Some
potential studies that could extract important SM information for nonresonance energies will also be presented. Studies of decays and
interactions can provide essential information to both flesh out the SM and search for clues toward new physics beyond the SM.
In addition to spectroscopic issues, the precision of the determination of strong- and weak-interaction parameters, the sensitivities
of measurements of and searches for rare and forbidden decays and $CP$-violating asymmetries, and how new particles
and new, beyond-the-SM interactions might manifest are discussed.
%It is hoped that such a coordinated multi-dimensional program at STCF will enable us to have a much
%more in-depth understanding challenges facing the SM and hopefully to provide some solutions to them.

%In Section II, the charmonium and $XYZ$-meson systems are discussed, with an emphasis on opportunities for solving the
%$XYZ$ puzzle and the discovery of other higher charmonium states. In Sections III and IV, charmed meson and baryon physics
%and tau physics are discussed, including the determination of SM parameters as well as searches for rare and forbidden decays and
%$CP$ violations.
%In Section V, several topics related to QCD, such as $R$-value and Collins effect measurements, the $Q^2$ behavior of
%baryon form factors, precision tests of rare/forbidden decays and CP violation in $\eta/\eta'$ and hyperon decays, and
%studies of glueballs and hybrids are discussed. In Section VI, the discovery potentials for new, beyond-the-SM light particles
%are presented. Section VII is a summary.



%{\color{red} References need more work!!!!}
