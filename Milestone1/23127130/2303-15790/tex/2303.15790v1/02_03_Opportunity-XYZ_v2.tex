\subsection{Opportunities for solving the $XYZ$ puzzles}

To date, no clear pattern has emerged for the complicated spectrum of the $XYZ$ states. To establish a pattern such that the $XYZ$ states can be classified, more measurements will be necessary, including searches for new charmonium-like structures. There are a few guidelines for possible measurements: 1) No matter what kind of internal structure the states have, there should be partners in the same heavy quark spin multiplet~\cite{Cleven:2015era}, which need to be searched for.
Complications arise from the mixing of them and their partners with spin multiplets of other structures (such as $c\bar c$) with the same quantum numbers. These can only be sorted out through observations, and this can only be done with sufficient measurements.
For instance, the $(0^{++}, 1^{++}, 2^{++}, 1^{+-})$ states have the $J^{PC}$ quantum numbers of $P$-wave $c\bar c$. Thus, the states with these quantum numbers having masses of approximately 3.9~GeV need to be systematically studied in decays to as many final states as possible. 2) It will be important to disentangle the contributions of kinematic singularities from those of resonances in order to establish the correct mass spectrum, and thus, the energy dependence of structures such as the $Z_c$ and $Z_{cs}$ will need to be measured. 3) Some of the structures that have been reported have similar masses and might have the same origin. To check this, it will be important to search for them in other channels and to measure their properties more precisely. 4)
It is expected to be worthwhile to pay special attention to energies around the $S$-wave open-charm thresholds.

Below, we list the opportunities at the STCF regarding the physics of hidden-charm $XYZ$ states:
\begin{itemize}
\item At the STCF luminosity of $0.5\times10^{35}$\,cm$^{-2}$s$^{-1}$ optimized at $\sqrt{s}=4$~GeV, two orders of magnitude higher than that of BEPCII, the vector charmonium-like states that are being investigated at BESIII can be studied in much more detail. Greatly improved knowledge of the intriguing $Z_c(3900)$ and $Z_c(4020)$ through $e^+e^-\to \pi^\pm Z_c^\mp$, as well as that of the $Z_{cs}(3985)$ and its possible spin partner in $e^+e^-\to K^\pm Z_{cs}^\mp$, will be obtained at various CMEs. The dependence of the $Z_c$ and $Z_{cs}$ line shapes and production rates on the CME will be crucial to keep kinematic effects from triangle singularities~\cite{Wang:2013cya,Pilloni:2016obd,Yang:2020nrt} under control.
\item Among all of the $PC=++$ $XYZ$ states, only the $X(3872)$ has been observed in $e^+e^-$ collisions, associated with a photon, and all others have only been seen in $B$ decays. This is because of the low production rates of the radiative processes and the fact that $X(3872)$ production receives an enhancement due to its large coupling to the $D\bar D^*$ pair. At the STCF with $E_\text{cm}\gtrsim4.7$~GeV, it will be possible to produce the $J^{++}$ states, $X(3915)$, $\chi_{c0}(3860)$ and $\chi_{c2}(3930)$, via $e^+e^-\to \omega X$ transitions, which should have much higher rates than the radiative processes.
\item  At the STCF with $E_\text{cm}\gtrsim5$~GeV, the $J^{++}$ states observed in the $\phi J/\psi$ invariant mass distributions can be investigated via $e^+e^-\to \phi X$. Searching for these states and others mentioned in the above item will be crucial for establishing the spectrum in the highly excited charmonium mass region and thus important in understanding the effects of the hadron thresholds on the spectrum and confinement. In addition to the abovementioned transitions, processes such as $e^+e^-\to \eta X$ should also be studied.

\item Energies higher than 5~GeV at the STCF will be useful for searches for the hadronic transitions to the spin partners of the $Z_c(3900)$ and $Z_c(4020)$ exotic states, named the $W_{c}$, as well as for conventional but not-yet-observed charmonium states.
The spin partners of the $Z_c$ are similar to those of the $Z_b$ proposed in Ref.~\cite{Bondar:2011ev}. They are isospin vector states with $J^{PC}=0^{++}$, $1^{++}$ and $2^{++}$, where the $C$ parity is for the charge-neutral state. The neutral ones can decay into $J/\psi \pi^+\pi^-$.
The $W_c$ can be studied in $e^+e^-\to \rho X$ transitions.

%     %%%%%%%%%%%%%%%%%%% Fig 4 %%%%%%%%%%%%%%%%%%%%%%%%
% \begin{figure*}[tp]
% 	\centering
% 	\includegraphics[width=0.92\textwidth]{Figs_02_CharmoniumXYZ/BB00}
% 	\caption{The spectrum of hadronic molecules consisting of a pair of charmed-anticharmed hadrons with negative parity and $(\text{isospin}, \text{strangeness})=(0,0)$ predicted in Ref.~\cite{Dong:2021juy}.
%     The colored rectangle, green for a bound state and orange for a virtual state, covers the uncertainty of the predicted mass. Thresholds are marked by dotted horizontal lines.
%     The rectangle closest to, but below, the threshold corresponds to the hadronic molecule in that system. When the masses of two hadronic molecules overlap, small rectangles are used with the left (right) one for the system with the higher (lower) threshold. The blue line (band) represents the center value (error) of the mass of the $\psi(4230)$~\cite{Zyla:2020zbs}.}\label{fig:specBB00}
% \end{figure*}
% %%%%%%%%%%%%%%%%%%%%%%%%%%%%%%%%%%%%%%%%%%%%%%%%%%
\item The lowest charmed baryon--antibaryon threshold, $\Lambda_c\bar \Lambda_c$, is at 4.57~GeV.
The BESIII measurement of the $e^+e^-\to \Lambda_c\bar \Lambda_c$ near-threshold production cross section indicates a state below the $\Lambda_c\bar \Lambda_c$ threshold~\cite{Cao:2019wwt,Dong:2021juy},
which is the lowest among a wealth of charmed baryon--antibaryon molecules recently predicted~\cite{Dong:2021juy}.
With $E_\text{cm}\gtrsim5$~GeV, the STCF will be able to reveal the expected rich phenomena due to the charmed baryon--antibaryon channels as well as those of excited charmed mesons.

\item With $E_\text{cm}\gtrsim 5$~GeV, it will also be possible to study hidden-charm pentaquark states in processes such as $e^+e^-\to J/\psi p\bar p$ and $e^+e^-\to \Lambda_c \bar D\bar p$. Similar to the $XYZ$ states above the $D\bar D$ threshold, there should be rich phenomena above the $\Lambda_c\bar D$ threshold. The cross section for $e^+e^-\to J/\psi p\bar p$ between 5 and 7~GeV may be estimated as
% \begin{equation}
$\sigma(e^+e^-\to J/\psi p\bar p) =\mathcal{O}(4~\text{fb})$~\cite{Chao:2023} based on the result for the $e^+e^-\to J/\psi gg$ cross section estimated using nonrelativistic QCD (NRQCD)~\cite{Ma:2008gq}. With an integrated luminosity of 2~ab$^{-1}$/year, $\mathcal{O}(8\times10^3)$ $J/\psi p\bar p$ events can be produced per year. A similar amount is expected for $J/\psi n\bar n$, and it will be possible to study this process at the STCF, whereas this is impossible for LHCb. The open-charm final states are expected to have larger cross sections. Furthermore, the hidden-charm pentaquarks are expected to decay much more easily into $\Lambda_c\bar D^{(*)}$ than into $J/\psi N$~\cite{Shen:2016tzq}, and the $\Sigma_c^{(*)}\bar D^{(*)}$ hadronic molecules, proposed by many authors to explain the LHCb $P_c$ states, couple strongly to $\Sigma_c^{(*)}\bar D^{(*)}$. Therefore, promising channels for the search for hidden-charm pentaquarks at the STCF include $e^+e^-\to  \Lambda_c \bar D^{(*)} \bar p$ and $\Sigma_c^{(*)} \bar D^{(*)} \bar p$. The STCF will provide good opportunities to search for hidden-charm $P_c$ and anomalous $P_{cs}$ pentaquarks.


\item For the interpretation of the nature of well-established highly excited charmonium states, detailed measurements of the production rates of open-charm final states such as $D^{(*)}\bar D^{(*)}$, $D_s^{(*)} \bar D_s^{(*)}$, and $D^{(*)}\bar D^{(*)} \pi(\pi)$ throughout the whole energy range of the STCF will be necessary.
To measure the cross sections of the three independent $D^*\bar D^*$ processes, namely, the $P$-wave with total spin $S=0$, the $P$-wave with $S=2$, and the $F$-wave with $S=2$, studies of the angular correlations of the $D^*$ decay products will need to be performed.

%%%%%%%%%%%%%%%%%%% Fig 2 %%%%%%%%%%%%%%%%%%%%%%%%
% \begin{figure*}[t]
% 	\centering
% 	\includegraphics[width=0.45\textwidth]{Figs_02_CharmoniumXYZ/ee2psietac}~~
% 	\includegraphics[width=0.45\textwidth]{Figs_02_CharmoniumXYZ/ee2psicc}
% 	\vspace{0cm}
% 	\caption{The cross sections for $e^+e^-\to J/\psi\eta_c$ (left) and $e^+e^-\to J/\psi c\bar c$ (right) calculated using NRQCD with the charm quark mass fixed at 1.5~GeV. The solid and dashed curves represent the results from the next-to-leading order and leading order calculations, respectively.
% 		\label{fig:xsec_NRQCD}}
% \end{figure*}
%%%%%%%%%%%%%%%%%%%%%%%%%%%%%%%%%%%%%%%%%%%%%%%%%%
\item There is a unique physics opportunity with $E_\text{cm}\in [6,7]$~GeV: this energy range offers an opportunity to study physics related to the production of two $c\bar c$ pairs. The production cross sections for $e^+e^-\to J/\psi\eta_c$ and $e^+e^-\to J/\psi c\bar c$ based on the NRQCD calculations in Refs.~\cite{Zhang:2005cha,Zhang:2006ay} are on the order of tens of fb; see also Section 5.1.5.
% are shown in Fig.~\ref{fig:xsec_NRQCD}.
In addition to double-charmonium production, which is also of interest, this energy range is ideal for the search for fully charmed tetraquark states, which are expected to have a mass of above 6~GeV~(see Refs.~\cite{Chao:1980dv,Karliner:2016zzc,Debastiani:2016xgg,Wang:2019rdo}). While it is uncertain whether the ground state $cc\bar c\bar c$ is below the double-$J/\psi$ or double-$\eta_c$ threshold, the low-lying $cc\bar c\bar c$ states are expected to decay predominantly into final states containing a pair of charm and anticharm hadrons via the annihilation of a $c\bar c$ pair into a gluon, with widths on the order of 100~MeV~\cite{Chao:1980dv,Anwar:2017toa}. Excited states with a mass well above the 6.2~GeV threshold can also easily decay into $J/\psi J/\psi$.
The LHCb measurement of the double-$J/\psi$ invariant mass spectrum in semi-inclusive processes of $pp$ collisions shows clear evidence for the existence of such states~\cite{Aaij:2020fnh}.
Searching for fully charmed tetraquarks in final states other than charged leptons is difficult at hadron colliders due to the high background; hence, the STCF is rather unique in its ability to support such a search.

% ******{\color{red} figure 5 should be figure 4 if appear here. It appear again in 5.1.5. May be should combined materials for these two small sections!}******

\item Charmonium-like hybrid candidates are also important targets to be searched for at the STCF, among which the most intriguing one is the lowest $1^{-+}$ state since the quantum numbers are prohibited for quark--antiquark states and from extensive lattice studies, it is expected to be the lowest charmonium-like hybrid. The mass is approximately 4.1--4.3 GeV. In addition, one expects a hybrid supermultiplet including $(0,1,2)^{-+}$ and $1^{--}$ states with nearly degenerate masses of approximately 4.4 GeV~\cite{Liu:2012ze}. At the STCF with $E_\text{cm}\gtrsim4.5$~GeV, the
$(0,1,2)^{-+}$ states can be produced either from the hadronic and radiative transitions from highly excited charmonia, such as $\psi(4S)$ and higher excitations, or from the final-state radiations in $e^+e^-$ annihilations. The $Y(4260)$ has a possible assignment of a $1^{--}$ charmonium-like hybrid~\cite{Zhu:2005hp}, but further experimental and theoretical efforts should be made to unravel its nature. At the STCF, with its much higher luminosity than BEPCII/BESIII, it will be possible to measure the decay properties of the $Y(4260)$ more precisely and to search for other open-charm decay modes, along with the possible connections between $Y(4260)$, $X(3872)$ and $Z_c(3900)$. It is expected that the STCF will enable the final determination of the status of the $Y(4260)$.

\end{itemize}
