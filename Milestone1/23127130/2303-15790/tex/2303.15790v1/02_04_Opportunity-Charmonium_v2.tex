\subsection{Opportunities in higher charmonium states}

Closely related to the $XYZ$ puzzles, there are also predictions from the quark model~\cite{Godfrey:1985xjl} and from lattice QCD~\cite{Liu:2012ze} of states that have not been identified. Some of them, such as the $2P$ states, are certainly intertwined with the $X$ states with the same quantum numbers. However, there are also still missing states that are believed to be relatively clean, such as the $1D$ state $\eta_{c2}$ and other higher-$L$ excitations.

% \subsubsection*{2.1.1 $2P$ charmonia}

% $2P$ charmonium states are called $\chi_{cJ}'$ and $h_c'$ according to the naming rules of hadrons. Quark model
% studies predict their masses to be around 4.0 GeV. The $\chi_{c2}'$ state (named as $\chi_{c2}(3930)$ by PDG2018~\cite{Tanabashi:2018oca}) is almost established with $M=3927.2\pm 2.6$ MeV and $\Gamma=24\pm 6$. Belle observes a wide resonance structure of $D\bar{D}$ with $M=3862^{+26+40}_{-32-13}$ MeV and $\Gamma=201^{+154+88}_{-67-82}$ MeV in the process $e^+e^-\rightarrow J/\psi D\bar{D}$ using the full amplitude analysis~\cite{Chilikin:2017evr}, which is now tentatively assigned to be $\chi_{c0}(3860)$ by PDG2018. PDG2018 now names $X(3872)$ to be $\chi_{c1}(3872)$, but admit that its properties are different from a conventional $q\bar{q}$ state
% and can be a candidate for an exotic structure. There is no clear evidence of $h_c'$ yet. Obviously, the
% masses of these $2P$ states or candidates are a little lower than the quark model prediction. This raises natural questions what on earth their inner dynamics are. On future STCF, $2P$ states can be produced by the radiative transitions from higher vector charmonia. This requires a considerable statistics accumulated at the $\psi(4040)$, $\psi(4160)$ and $\psi(4415)$ energy scales.


% \subsubsection*{2.1.2  $1D$ charmonia $\eta_{c2}$ and $\psi_3$}

The supermultiplet of $1D$ states includes $1^3 D_{1,2,3}$ and $1^1 D_2$ (named $\eta_{c_2}$), with the quantum numbers $(1,2,3)^{--}$ and $2^{-+}$, respectively. Apart from the well-known $\psi(3770)$, the $2^{--}$ state has likely been observed by Belle~\cite{Bhardwaj:2013rmw} and BESIII~\cite{Ablikim:2015dlj} and is labeled $\psi_2(3823)$ in the 2018 PDG. Very recently, LHCb reported a candidate for the $1^3 D_3$ state (named $\psi_3$). However, the spin singlet $1D$ state $\eta_{c_2}$ continues to evade experimental searches. Lattice QCD studies predict that the mass of $\eta_{c2}$ is approximately $3.8$ GeV~\cite{Liu:2012ze,Yang:2012mya}, nearly degenerate with other $1D$ states. Experimentally, $\eta_{c2}$ can be produced directly from $\psi(4040)$ through the M1 transition. If the partial width of $\psi(4040)\to \gamma \eta_{c2}$ is a few keV, then the
corresponding branching fraction is $O(10^{-5})$. Therefore, it is difficult for BESIII to observe $\eta_{c2}$ in this process (the number of $\psi(4040)$ events at BESIII is $O(10^{6})$~\cite{Asner:2008nq}). However, the STCF, with a luminosity 100 times higher, will have the possibility to search for $\eta_{c2}$. Since it has no open-charm decay modes, the hadronic transitions, such as the decay modes $\chi_{c1}\pi\pi$ and $J/\psi \pi^0 \pi^- \pi^+$, and the E1 radiative transition $\eta_{c2}\to \gamma h_c$ can be important.

% $\psi_3$ can be search in the processes $e^+e^-\to \pi\pi \psi_3$ with $\psi_3\to \gamma \chi_{c2}$, similar to the case of $\psi_2(3823)\to \gamma \chi_{c1}$, and  also $\psi_3 \to D\bar{D}$. BESIII does not observe $\psi_3$ in the $e^+e^-\to \pi\pi D\bar{D}$ process~\cite{Ablikim:2019faj}. This is understandable since the partial decay width of $\psi_3\to D\bar{D}(L=3)$ is highly suppressed by the centrifugal potential barrier. Non-relativistic models predict that that both the decay widths of $\psi_3\to \gamma \chi_{c2}$ and $\psi_2\to \gamma \chi_{c1}$ are around 280 keV~\cite{Barnes:2005pb}



% \subsubsection*{2.1.4  Radiative tansitions and decays of charmonia}

Apart from spectroscopy, the understanding of the known charmonium states can be greatly improved through more precise measurements of their radiative and hadronic states~\cite{Asner:2008nq}. In the following, the two types of decays will be discussed.

\par
For the radiative transitions, at the STCF, it will be possible to measure the rare electric-dipole transitions $\eta_c(2S)\rightarrow h_c\gamma$ and $\psi(3770)\rightarrow \chi_{c0}\gamma$ and the magnetic-dipole transitions $\psi(2S)\rightarrow \eta_c(2S)\gamma$, $\eta_c(2S)\rightarrow J/\psi\gamma$, and $h_c\rightarrow \chi_{c0}\gamma $. It will also be possible to measure the total and leptonic or two-photon widths with high precision.
These transitions and decay widths can be calculated in both the quark model and lattice QCD. Comparisons between experimental data and these theoretical predictions will help us more clearly understand the inner structure of charmonia.

A more systematic and comprehensive study of the decays of low-lying charmonia can also be performed at the STCF. These states are below the threshold for $D$-meson production and decay predominantly into hadrons consisting of light $u$, $d$ and $s$ quarks. However, information about their decays is incomplete at present. For the best-studied $J/\psi$ meson, only approximately 40\% of its hadronic decays have been measured. For other states, the situation is even worse. The high luminosity of the STCF will facilitate more precise measurements of the properties of light hadrons from low-lying charmonium decays and the subsequent acquisition of a more complete understanding of the scenario of low-energy strong interactions.
