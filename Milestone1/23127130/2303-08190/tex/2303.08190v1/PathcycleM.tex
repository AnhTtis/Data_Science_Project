\chapter{The $i$-Graphs of Paths and Cycles}
\label{ch:pathcycles}



%%%% COMMENTED Note:  WTF is the following for? Why is it here?

%$S_1=\{v_1,v_4, v_7, v_{10},\dots, v_{3k+1}\}$.

%$\edge{S_1, v_1,v_2,S_2} \adedge{v_4,v_5,S_3}  \adedge{v_7,v_8,S_4} \dots \edge{S_{k+1},v_{3k+1},v_{3k+2},S_{k+2}}$



% +++++++++++++++++++++++++++  PRE INFO / REFERENCES  BLAH BLAH +++++++++++++++++



In our previous chapters, we have explored whether a given graph $H$ is $i$-graph realizable.  That is, does there exist a graph $G$ such that $\ig{G} = H$?  We now move to the opposing question: given a graph $G$, what is the structure of $\ig{G}$?  As we have already observed, although not every graph is $i$-graph realizable, every graph does have an $i$-graph.  The exact structure of the resulting $i$-graph can vary among families of graphs from the simplest isolated vertex to surprisingly complex structures.  


In this chapter, we examine the $i$-graphs of two of the most famous classes of graphs: paths and cycles.  To  begin, we use generating functions to count the distinct $i$-sets of the path $P_n$ and the cycle $C_n$.  That is, we determine $ \left|  V \left(\ig{P_n} \right)  \right|$ for $n\geq 1$  and $\left|  V \left(\ig{C_n} \right)  \right|$ for $n \geq 3$.  




% +++++++++++++++++++++++++++  PRE INFO / REFERENCES  BLAH BLAH +++++++++++++++++


\section{The Number of $i$-Sets of Paths} \label{sec:i:numbPaths}

For the remainder of this chapter, we assume that the vertices of the path $P_{n}$  are labelled as $P_{n} = (v_1,v_2,\dots, v_{n})$. 
Given that we are discussing $i$-sets, which are both independent and dominating, if $X$ is an $i$-set of $P_n$, then two consecutive vertices of $X$ are separated by one or two vertices of $P_n-X$; the different interval lengths between these consecutive vertices of $X$ therefore correspond to the different $i$-sets of $P_n$.  This provides our method for counting the distinct $i$-sets of $P_n$.



To begin, recall the following well-known result regarding the  independent domination number for both paths and cycles.



%>>> S: Lemma {lem:i:iPC}
\begin{lemma} \label{lem:i:iPC} \cite{GH13}
	For the path and cycle, $i(P_n) = i(C_n) = \left \lceil n/3 \right \rceil $.
\end{lemma}
% <<< E: Lemma {lem:i:iPC}

\noindent 
Thus, deleting the vertices of an arbitrary $i$-set $X$ of $P_n$ divides $V(P_n)-X$ into $t=i(P_n)+1$ subsets $X_1, X_2,\dots,X_t$, where $X_1$ and $X_t$ could be empty or consist of a single vertex, and where $1 \leq |X_j| \leq 2$ for $ j \neq \{1,t\}$.  An example for $P_{10}$ with sets $X_1, X_2, \dots, X_5$ is given below in Figure \ref{fig:i:GF}.  In particular, notice that $X_1 = \varnothing$.  




%>>> S: Figure {fig:i:GF}	
\begin{figure}[H] \centering	
	\begin{tikzpicture}			 		
		
		%%%% REQUIRED TO FIX THE OVER BRACKET TEXT LABEL
		%% https://tex.stackexchange.com/questions/34446/how-to-join-underbrace-overbrace-between-nodes
		\tikzset{
			position label/.style={
				above = 1pt,
				text height = 1ex,
				text depth = 1ex
			},
			brace/.style={
				decoration={brace, mirror},
				decorate
			}
		}
		
		
		%---------- REF-------------------
		\coordinate (cent) at (0,0) {};
		\coordinate (v0) at (0,0) {};
		
		\foreach \i / \j in {1,2,3,4,5,6,7,8,9,10}
		{
			\path(cent) ++(0: \i*10 mm) node[std,label={ 270:$v_{\i}$}]  (v\i) {};
		}
		\draw[thick] (v1)--(v10);		
		
		
		\foreach \i / \j in {1,3,6,9}
		\node[bred] (w\i) at (v\i) {};
		
		
		
		\foreach \i / \j in {0,1,2,3,4,5,6,7,8,9,10}
		{
			\path(v\i) ++(-3mm,2mm) coordinate (l\i) {};
			\path(v\i) ++(3mm,2mm) coordinate (r\i) {};
		}
		
		\draw [decoration={brace}, decorate, color=blue] (l0.north) -- node [position label, pos=0.5] {$X_1$} (l1.north);
		\draw [decoration={brace}, decorate, color=blue] (r1.north) -- node [position label, pos=0.5] {$X_2$} (l3.north);
		\draw [decoration={brace}, decorate, color=blue] (r3.north) -- node [position label, pos=0.5] {$X_3$} (l6.north);
		\draw [decoration={brace}, decorate, color=blue] (r6.north) -- node [position label, pos=0.5] {$X_4$} (l9.north);
		\draw [decoration={brace}, decorate, color=blue] (r9.north) -- node [position label, pos=0.5] {$X_5$} (r10.north);		
	\end{tikzpicture}   
	
	\caption{The sets $X_1,X_2,\dots,X_5$ of $P_{10}$.}
	\label{fig:i:GF}	
\end{figure}	
% <<< E: Figure {fig:i:GF}	

The number of $i$-sets, therefore, equals the number of integer solutions to the following equation, where $r=|V(P) - X| = n - i(P_n) = n -  \left \lceil n/3 \right \rceil $, and where $x_j = |X_j|$:

\vspace{-1cm}
\begin{align}\label{eq:i:pathEq}
	x_1 + \dots + x_t = r, \text{ where } 0 \leq x_1, x_t \leq 1, \text{ and } 1 \leq x_j \leq 2 \text{ for } j \in \{2,\dots,t-1\}.
\end{align}

\noindent The corresponding generating function is
\begin{align} \label{eq:i:pathGF}
	(1+x)^2(x+x^2)^{t-2} = x^{t-2}(1+x)^t.
\end{align}
\begin{enumerate}[label=$\bullet$]
	\item If $n=3k+1$ then $t=i(P_n)+1=k+2$ and $r=2k$.  Thus we want to the coefficient of $2k$ in  (\ref{eq:i:pathGF}), that is, the coefficient of $x^k$ in $(1+x)^{k+2}$, which is $\binom{k+2}{k}$.
	
	\item If $n=3k+2$, then $t=k+2$ and $r=2k+1$.  Here we want the coefficient of $x^{k+1}$ in $(1+x)^{k+2}$, which is $\binom{k+2}{k+1} = k+2$.
	
	\item If $n=3k+3$, then $t=k+2$ and $r=2k+2$.  Here we want the coefficient of $x^{k+2}$ in $(1+x)^{k+2}$, which is $1$.
\end{enumerate}
\noindent We summarize  the above results in the following lemma. 


%>>> S: Lemma {lem:i:pathSize}
\begin{lemma} \label{lem:i:pathSize}
	For $n \geq 1$, the order of $\ig{P_n}$ is 	
	\begin{align*}
		\left|  V \left(\ig{P_n} \right)  \right|  = 
		\begin{cases}
			1 & \text{ if } \ n = 3k\\ 
			\binom{k+2}{k} & \text{ if }  \ n = 3k+1 \\
			k+2 & \text{ if }  \ n=3k+2.
		\end{cases}
	\end{align*}
\end{lemma}
% <<< E: Lemma {lem:i:pathSize}





% +++++++++++++++++++++++++++  LATTICE GRAPH  +++++++++++++++++

\section{The $i$-Graph of $P_n$} \label{sec:i:Pn}


We remind the reader that the vertices of the path $P_{n}$  are labelled as $P_{n} = (v_1,v_2,\dots, v_{n})$.   As we return to our token-sliding model for $i$-graphs, notice that the $i$-set tokens on $P_n$ that are free to slide are very limited.  

For example, in Figure \ref{fig:i:P10}	below, we have two different $i$-sets on $P_{10}$.  In the first case, the token at $v_6$ which is surrounded on each side by two consecutive non-$i$-set vertices is frozen; a slide in either direction would leave either $v_5$ or $v_7$ undominated.   That is, in this first configuration $|X_3|=|X_4|=2$.   Similarly, near the end of the path, the token at $v_9$ is frozen.  Indeed, the only  token that can slide is $v_3$.  Thus the $i$-graph vertex  associated with the $i$-set in the first figure would have only degree 1.  On the other hand, in the second $i$-set of Figure \ref{fig:i:P10}, we see that there are 4 possible token slides, corresponding to an $i$-graph vertex of degree four.  Indeed, we see  that due to the limited availability of slack between $i$-set vertices, the maximum degree of an $i$-graph vertex of $P_n$ is 4, as we now explain.  Moving the token between $X_i$ and $X_{i+1}$ to the right, we increase $|X_i|$ by one, and decrease $|X_{i+1}|$ by one.




%>>> S: Figure {fig:i:P10}	
\begin{figure}[H] \centering	
	\begin{tikzpicture}			
		%---------- REF-------------------
		\coordinate (cent) at (0,0) {};
		
		\foreach \i / \j in {1,2,3,4,5,6,7,8,9,10}
		{
			\path(cent) ++(0: \i*8 mm) node[std,,label={ 270:$v_{\i}$}]  (v\i) {};
		}
		\draw[thick] (v1)--(v10);		
		
		
		\foreach \i / \j in {1,3,6,9}
		\node[bred] (w\i) at (v\i) {};
		
		\draw[ltteal,thick, ->]  (v3) to [out=45,in=135]  (v4);
		
		
	\end{tikzpicture} 
	
	\vspace{1cm}
	
	\begin{tikzpicture}			
		%---------- REF-------------------
		\coordinate (cent) at (0,0) {};
		
		\foreach \i / \j in {1,2,3,4,5,6,7,8,9,10}
		{
			\path(cent) ++(0: \i*8 mm) node[std,,label={ 270:$v_{\i}$}]  (v\i) {};
		}
		\draw[thick] (v1)--(v10);		
		
		
		\foreach \i / \j in {2,4,7,9}
		\node[bred] (w\i) at (v\i) {};
		
		\draw[ltteal,thick, <-]  (v1) to [out=45,in=135]  (v2);
		\draw[ltteal,thick, ->]  (v4) to [out=45,in=135]  (v5);
		\draw[ltteal,thick, <-]  (v6) to [out=45,in=135]  (v7);
		\draw[ltteal,thick, ->]  (v9) to [out=45,in=135]  (v10);
		
	\end{tikzpicture} 
	
	\caption{Two $i$-sets of $P_{10}$.}
	\label{fig:i:P10}
\end{figure}	
% <<< E: Figure {fig:i:P10}

Thus, a token between $X_i$ and $X_{i+1}$ can slide to the right if and only if 
\begin{align*}
	|X_i| = \begin{cases} 0 & \text{ if } i=0 \\ 1 &  \text{ if }  i \neq 0 \end{cases}  \ \hspace{3mm} \text { and } \  \hspace{3mm}
	|X_{i+1}| = \begin{cases} 1 & \text{ if }    i=t \\ 2 & \text{ if } i \neq t. \end{cases}
\end{align*}

\noindent Similarly, a token between $X_i$ and $X_{i+1}$ can slide to the left if and only if 
\begin{align*}
	|X_i| = \begin{cases} 1 & \text{ if } i=0 \\ 2 &  \text{ if }  i \neq 0 \end{cases}  \ \hspace{3mm}  \text { and } \  \hspace{3mm}
	|X_{i+1}| = \begin{cases} 0 & \text{ if }    i=t \\ 1 & \text{ if } i \neq t. \end{cases}
\end{align*}

\noindent With these results in mind, we can now consider the actual structure of $\ig{P_n}$. 


From Lemma \ref{lem:i:pathSize}, the $i$-graphs of $P_n$ for $n=3k$ and $n=3k+2$ $(k\geq 0$) are immediate.  In particular, a token can move only if it is adjacent to some $X_i$ when $x_i=|X_i|$ is at its lower bound in (\ref{eq:i:pathEq}).  For $n=3k$, there are no such $X_i$, so all tokens are frozen, and $\ig{P_n}=K_1$.  If $n=3k+2$, then exactly one  $X_i$ is at its lower bound.  If $i=0$ or $t$, then one token can move.  Otherwise, $1 \leq i \leq t-1$ and $X_i$ has both a left and right token that can move. Hence $\Delta(\ig{P_{3k+2}}) \leq  2$.  It is easy to confirm that $\ig{P_{3k+2}} = P_{k+2}$.



%>>> S: Lemma {lem:i:2Pn}
\begin{lemma} \label{lem:i:2Pn}
	For $n \geq 1$, $k \geq 0$,		
	\begin{align*}
		\ig{P_n} = 	
		\begin{cases}
			K_1 & \text{ if } n=3k \\
			P_{k+2} & \text{ if } n=3k+2.
		\end{cases}
	\end{align*}
\end{lemma}
% <<< E: Lemma {lem:i:2Pn}

\noindent It is only the third case where $n=3k+1$ that remains to be described.   Using the fact that two $X_i$ are at their lower bound, we get $\Delta(\ig{P_n}) \leq 4$. In order to eventually determine the structure of $\ig{P_{3k+1}}$, we first detour slightly to define a new class of graphs.

%\noindent \textbf{Definition} \\
We define the \emph{worn $k$-lattice graph} $\wlat_k$ to be the graph with $1+2+\dots+k +(k+1)=  \binom{k+2}{2} $ vertices, such that  $V(\wlat_k) = \{w_{i,j}: 0 \leq j \leq i \leq k\}$, and where the vertex $w_{a,b}$ is adjacent to vertex $w_{c,d}$  if and only if one of the following holds:

\begin{enumerate}[label=(\roman*)]
	\item  $a = c-1$ and $b=d \text{ or } d-1$, or
	\item  $a=c+1$ and $b=d \text{ or } d+1$.
\end{enumerate}	

\noindent The graph $\wlat_3$ is given as an example below in Figure \ref{fig:i:wlat3}.	






%>>> S: Figure {fig:i:wlat3}
\begin{figure}[H] \centering	
	\begin{tikzpicture}[scale=1.2]
		
		
		%---------- REF-------------------
		\coordinate (cent) at (0,0) {};
		
		\path (cent) ++(-90: 15 mm) coordinate (centL1);
		\path (centL1) ++(270: 15 mm) coordinate (centL2);
		
		%--------- Level 0
		\node[std,label={ 0: $w_{1,1}$}] (z) at (cent) {}; 
		
		%--------- Level 1
		\path (centL1) ++(180: 15 mm) node[std,label={ 0:  $w_{2,1}$}] (a1) {};
		\path (a1) ++(0: 30 mm) node[std,label={ 0:  $w_{2,2}$}](a2) {};
		
		
		%--------- Level 2
		\path (centL2) ++(180: 30 mm) node[std,label={ 270: $w_{3,1}$}] (b1) {};
		
		
		\path (b1) ++(0: 30 mm) node[std,label={ 270: $w_{3,2}$}] (b2) {};	
		\path (b1) ++(0: 60 mm) node[std,label={ 270: $w_{3,3}$}] (b3) {};	
		
		
		%============================ D LINES
		
		\path (a1) ++(90: 30 mm) node[std,label={ 0: $w_{0,0}$}] (d1) {};
		\path (z) ++(180: 30 mm) node[std,label={ 0: $w_{1,0}$}] (d2) {};
		\path (a1) ++(180: 30 mm) node[std,label={ 0: $w_{2,0}$}] (d3) {};
		\path (b1) ++(180: 30 mm) node[std,label={ 270: $w_{3,0}$}](d4) {};
		
		%--------- Paths
		\draw[thick](z)--(a1)--(b1);
		\draw[thick](z)--(a2)--(b3);
		\draw[thick](a1)--(b2)--(a2);
		\draw[thick](z)--(d1)--(d2)--(d3)--(d4);
		\draw[thick](d2)--(a1);
		\draw[thick](d3)--(b1);
		
	\end{tikzpicture} 
	
	\caption{The worn lattice graph $\wlat_3$.}
	\label{fig:i:wlat3}	
\end{figure}	
% <<< E: Figure {fig:i:wlat3}	





The observant reader will have already noted several similarities to $\wlat_k$ and the structure of $\ig{P_{3k+1}}$.  Notably, recall from Lemma \ref{lem:i:pathSize} that $\ig{P_{3k+1}}$ also has $\binom{k+2}{2}$ vertices.  We have also noted that for each $v\in V(\ig{P_k})$, $\deg(v) \leq 4$.  In Lemma  \ref{lem:i:p3k1} below, we show exactly this connection, with a key observation being that just as one can imagine growing $P_{3k+1}$ into $P_{3(k+1)+1}$ by adding a \emph{tail} of three additional vertices to the end of $P_{3k+1}$, so can we grow $\wlat_k$ into $\wlat_{k+1}$ by adding an additional path of $(k+1)+1$ vertices down one outer-diagonal. We call these vertices the \emph{path tail}  vertices of $\wlat_{k+1}$.

%\begin{m} \label{thm:i:p3k1} cc
% For $k \geq 1$, $\ig{P_{3k+1}} = \wlat_{k-1}$.
%\end{theorem}









The cases for $k=1$ and $k=2$ (i.e. with $P_4$ and $P_7$) are illustrated in Figures \ref{fig:i:P4ig} and  \ref{fig:i:P7ig} below.  Here the vertices  of $\wlat_k$ are labelled with the subscripts of the vertices of an $i$-set of $P_{3k+1}$.  So, for example, the vertex labelled ``$1,4$" at the top of $\wlat_{1}$ corresponds to the $i$-set $\{v_1,v_4\}$ of $P_4$, while the vertex labelled ``$2,4$" corresponds to the $i$-set $\{v_2,v_4\}$.  Cases $k=3$ ($P_{10}$) and $k=4$ ($P_{13}$) are given in Figures \ref {fig:i:P10ig} and \ref {fig:i:P13ig}, respectively, and the added path tail vertices highlighted in green and blue.




%>>> S: Figure {fig:i:P4ig}	
\begin{figure}[H] \centering	
	
	\begin{tikzpicture}			
		%---------- REF-------------------
		\coordinate (cent) at (0,0) {};
		
		\foreach \i / \j in {1,2,3,4}
		{
			\path(cent) ++(0: \i*8 mm) node[std,,label={ 270:$v_{\i}$}]  (v\i) {};
		}
		\draw[thick] (v1)--(v4);		
	\end{tikzpicture} 	\hspace{1cm}
	\begin{tikzpicture}			
		\node[broadArrow,fill=white, draw=black, ultra thick, text width=6mm] at (0,0){};	
	\end{tikzpicture} \hspace{1cm}
	\begin{tikzpicture}			
		
		%---------- REF-------------------
		\coordinate (cent) at (0,0) {};
		
		\path (cent) ++(-90: 15 mm) coordinate (centL1);
		%	\path (centL1) ++(270: 15 mm) coordinate (centL2);
		
		%--------- Level 0
		\node[std,label={ 0:\scriptsize$1,4$}] (z) at (cent) {}; 
		
		%--------- Level 1
		\path (centL1) ++(180: 15 mm) node[std,label={ 0:\scriptsize$2,4$}] (a1) {};
		\path (a1) ++(0: 30 mm) node[std,label={ 0:\scriptsize$1,3$}] (a2) {};	 
		
		%--------- Paths
		\draw[thick](a1)--(z)--(a2);		
		
	\end{tikzpicture} 
	
	\caption{$\ig{P_4} \cong \wlat_{1}$.}
	\label{fig:i:P4ig}	
\end{figure}	
% <<< E: Figure {fig:i:P4ig}	


%>>> S: Figure {fig:i:P7ig}	
\begin{figure}[H] \centering	 
	\begin{tikzpicture}	
		
		
		%---------- REF-------------------
		\coordinate (cent) at (0,0) {};
		
		\path (cent) ++(-90: 15 mm) coordinate (centL1);
		\path (centL1) ++(270: 15 mm) coordinate (centL2);
		
		%--------- Level 0
		\node[std,label={ 0:\scriptsize$1,4,7$}] (z) at (cent) {}; 
		
		%--------- Level 1
		\path (centL1) ++(180: 15 mm) node[std,label={ 0:\scriptsize$2,4,7$}] (a1) {};
		\path (a1) ++(0: 30 mm) node[std,label={ 0:\scriptsize$1,4,6$}] (a2) {};
		
		
		%--------- Level 2
		\path (centL2) ++(180: 30 mm) node[std,label={ 270:\scriptsize$2,5,7$}] (b1) {};
		
		
		\path (b1) ++(0: 30 mm) node[std,label={ 270:\scriptsize$2,4,6$}] (b2) {};	
		\path (b1) ++(0: 60 mm) node[std,label={ 270:\scriptsize$1,3,6$}] (b3) {};	
		
		%--------- Paths
		\draw[thick](z)--(a1)--(b1);
		\draw[thick](z)--(a2)--(b3);
		\draw[thick](a1)--(b2)--(a2);
		
		
		
		
		
	\end{tikzpicture} 
	
	\caption{$\ig{P_7} \cong \wlat_{2}$.}
	\label{fig:i:P7ig}	
\end{figure}	
% <<< E: Figure {fig:i:P7ig}	






%>>> S: Figure {fig:i:P10ig}
\begin{figure}[H] \centering	
	\begin{tikzpicture}	
		
		
		%---------- REF-------------------
		\coordinate (cent) at (0,0) {};
		
		\path (cent) ++(-90: 15 mm) coordinate (centL1);
		\path (centL1) ++(270: 15 mm) coordinate (centL2);
		
		%--------- Level 0
		\node[std,label={ 0:\scriptsize $1,4,7, {\color{forGreen}9}$}] (z) at (cent) {}; 
		
		%--------- Level 1
		\path (centL1) ++(180: 15 mm) node[std,label={ 0:\scriptsize$2,4,7,{\color{forGreen}9}$}] (a1) {};
		\path (a1) ++(0: 30 mm) node[std,label={ 0:\scriptsize$1,4,6,{\color{forGreen}9}$}] (a2) {};
		
		
		%--------- Level 2
		\path (centL2) ++(180: 30 mm) node[std,label={ 270:\scriptsize$2,5,7,{\color{forGreen}9}$}] (b1) {};
		
		
		\path (b1) ++(0: 30 mm) node[std,label={ 270:\scriptsize$2,4,6,{\color{forGreen}9}$}] (b2) {};	
		\path (b1) ++(0: 60 mm) node[std,label={ 270:\scriptsize$1,3,6,{\color{forGreen}9}$}] (b3) {};	
		
		
		%============================ D LINES
		
		\path (a1) ++(90: 30 mm) node[bgreen,label={ 0:\scriptsize $1,4,7, {\color{forGreen}10}$}] (d1) {};
		\path (z) ++(180: 30 mm) node[bgreen,label={ 0:\scriptsize$2,4,7,{\color{forGreen}10}$}] (d2) {};
		\path (a1) ++(180: 30 mm) node[bgreen,label={ 0:\scriptsize$2,5,7,{\color{forGreen}10}$}] (d3) {};
		\path (b1) ++(180: 30 mm) node[bgreen,label={ 270:\scriptsize$2,5,8,{\color{forGreen}10}$}] (d4) {};
		
		%--------- Paths
		\draw[thick](z)--(a1)--(b1);
		\draw[thick](z)--(a2)--(b3);
		\draw[thick](a1)--(b2)--(a2);
		\draw[forGreen, thick](z)--(d1)--(d2)--(d3)--(d4);
		\draw[forGreen, thick](d2)--(a1);
		\draw[forGreen, thick](d3)--(b1);
		
	\end{tikzpicture} 
	
	\caption{$\ig{P_{10}} \cong \wlat_{3}$.}
	\label{fig:i:P10ig}	
\end{figure}	
% <<< E: Figure {fig:i:P10ig}	




%>>> S: Figure {fig:i:P13ig}
\begin{figure}[H] \centering	
	\begin{tikzpicture}			
		
		%---------- REF-------------------
		\coordinate (cent) at (0,0) {};
		
		\path (cent) ++(-90: 15 mm) coordinate (centL1);
		\path (centL1) ++(270: 15 mm) coordinate (centL2);
		
		%--------- Level 0
		\node[std,label={ 0:\scriptsize $1,4,7,  {\color{forGreen}9},  {\color{blue}12}$}] (z) at (cent) {}; 
		
		%--------- Level 1
		\path (centL1) ++(180: 15 mm) node[std,label={ 0:\scriptsize$2,4,7, {\color{forGreen}9},  {\color{blue}12}$}] (a1) {};
		\path (a1) ++(0: 30 mm) node[std,label={ 0:\scriptsize$1,4,6, {\color{forGreen}9},  {\color{blue}12}$}] (a2) {};		
		
		%--------- Level 2
		\path (centL2) ++(180: 30 mm) node[std,label={ 270:\scriptsize$2,5,7, {\color{forGreen}9},  {\color{blue}12}$}] (b1) {};		
		
		\path (b1) ++(0: 30 mm) node[std,label={ 270:\scriptsize$2,4,6,  {\color{forGreen}9},  {\color{blue}12}$}] (b2) {};	
		\path (b1) ++(0: 60 mm) node[std,label={ 270:\scriptsize$1,3,6, {\color{forGreen}9},  {\color{blue}12}$}] (b3) {};			
		
		%============================ D LINES
		\path (a1) ++(90: 30 mm) node[bgreen,label={ 0:\scriptsize $1,4,7, {\color{forGreen}10},  {\color{blue}12}$}] (d1) {};
		\path (z) ++(180: 30 mm) node[bgreen,label={ 0:\scriptsize$2,4,7, {\color{forGreen}10},  {\color{blue}12}$}] (d2) {};
		\path (a1) ++(180: 30 mm) node[bgreen,label={ 0:\scriptsize$2,5,7, {\color{forGreen}10},  {\color{blue}12}$}] (d3) {};
		\path (b1) ++(180: 30 mm) node[bgreen,label={ 270:\scriptsize$2,5,8, {\color{forGreen}10},  {\color{blue}12}$}] (d4) {};
		
		%============================ F LINES
		\path (d2) ++(90: 30 mm) node[bblue,label={ 0:\scriptsize $1,4,7, {\color{forGreen}10},  {\color{blue}13}$}] (f1) {};
		\path (d1) ++(180: 30 mm) node[bblue,label={ 0:\scriptsize$2,4,7, {\color{forGreen}10},  {\color{blue}13}$}] (f2) {};
		\path (d2) ++(180: 30 mm) node[bblue,label={ 0:\scriptsize$2,5,7, {\color{forGreen}10},  {\color{blue}13}$}] (f3) {};
		\path (d3) ++(180: 30 mm) node[bblue,label={ 0:\scriptsize$2,5,8, {\color{forGreen}10},  {\color{blue}13}$}] (f4) {};
		\path (d4) ++(180: 30 mm) node[bblue,label={ 270:\scriptsize$2,5,8, {\color{forGreen}11},  {\color{blue}13}$}] (f5) {};
		
		%--------- Paths
		\draw[thick](z)--(a1)--(b1);
		\draw[thick](z)--(a2)--(b3);
		\draw[thick](a1)--(b2)--(a2);
		\draw[forGreen, thick](z)--(d1)--(d2)--(d3)--(d4);
		\draw[forGreen, thick](d2)--(a1);
		\draw[forGreen, thick](d3)--(b1);
		\draw[blue,thick](f1)--(f2)--(f3)--(f4)--(f5);
		\draw[blue,thick](f1)--(d1);
		\draw[blue,thick](f2)--(d2);
		\draw[blue,thick](f3)--(d3);
		\draw[blue,thick](f4)--(d4);
		
	\end{tikzpicture} 
	
	\caption{$\ig{P_{13}} \cong \wlat_{4}$.}
	\label{fig:i:P13ig}
\end{figure}	
% <<< E: Figure {fig:i:P13ig}







%>>> S: Lemma {lem:i:p3k1} 
\begin{lemma} \label{lem:i:p3k1} 
	For $k\geq 1$, $\ig{P_{3k+1}} \cong \wlat_{k}$.  
\end{lemma}	



%============================ RICK ATTEMPT
\begin{proof}
	We proceed by induction on $k$.  The base cases with $k \in \{1,2,3,4\}$ are given explicitly above in Figures \ref{fig:i:P4ig} - \ref{fig:i:P13ig}.  Assume that $\ig{P_{3k+1}} \cong \wlat_k$.  We now consider $\ig{P_{3k+4}}$.  Observe that every $i$-set of $P_{3k+4}$ contains $v_{3k+3}$ or $v_{3k+4}$.  
	
	%	In our induction, we include the following claims for each value of $k \geq 1$:
	
	%	\begin{enumerate}[label=(\roman*)]
		%	\item 
		%		The $i$-sets down the (pictured as left) side of the $\wlat_k$ correspond to the $i$-sets:
		
		%		\begin{align*} w_{\ell,0} =  \bigcup_{j=0}^{\ell-1} \{v_{3j+2}\} \  \cup  \ \ \bigcup_{j=\ell}^{k} \{v_{3j+1}\}. \end{align*}
		
		%	\item 
		
		
		
		%	\end{enumerate}
	
	
	
	Let $S$ be some $i$-set of ${P_{3k+1}}$, and define  in $P_{3k+4}$ the set $S' = S \cup \{v_{3k+3}\}$.  Then $S'$ is both independent and dominating in $P_{3k+4}$, and since $i(P_{3k+4}) = i(P_{3k+1}) +1 = k+2$,  $S'$ is an $i$-set of $P_{3k+4}$ (and corresponds to a vertex of $\ig{P_{3k+4}}$).  Conversely, given an $i$-set of $P_{3k+4}$ containing $v_{3k+3}$, say $S'$, the set $S = S' - \{v_{3k+3}\}$ is an $i$-set of $P_{3k+1}$.  This follows since $N[v_{3k+3}] = \{v_{3k+2}, v_{3k+3}, v_{3k+4}\}$.
	
	
	
	
	
	
	
	%>>> S: Figure {fig:i:P3k4}
	\begin{figure}[H] \centering	
		\begin{tikzpicture}			
			%---------- REF-------------------
			\coordinate (cent) at (0,0) {};
			
			
			\coordinate (v1) at (cent) {};
			\path(cent) ++(0: 1.3 cm) coordinate  (v2) {};
			\path(v2) ++(0:  2.3  cm) coordinate  (v3k) {};
			
			\foreach \i in {1,2,3,4}
			{
				\path(v3k) ++(0: \i*1.3 cm) node[std,label={ 270:$v_{3k+\i}$}]  (v3k\i) {};
			}
			
			\draw[draw=gray, fill=lgray] \convexpath{v1,v3k1}{7mm};		
			
			\node[std, label={ 270:$v_1$}] (v1x) at (v1) {};
			\node[std, label={ 270:$v_2$}] (v2x) at (v2) {};
			%\node[std, label={ 270:$v_3$}] (v3x) at (v3) {};
			\node[std, label={ 270:$v_{3k}$}] (v3kx) at (v3k) {};
			\node[std, label={ 270:$v_{3k+1}$}] (v3k1x) at (v3k1) {};
			
			
			
			\path(v2) ++(0:  0.6 cm) coordinate (v2a) {};
			\path(v3k) ++(0:  -0.6  cm) coordinate (v3ka) {};
			
			
			\draw[thick] (v1)--(v2a);
			\draw[thick] (v3ka)--(v3k4);
			\draw[thick, dashed] (v3ka)--(v2a);
			
			\node[bgreen] (v3k3G) at (v3k3) {};
			
			
			%	\coordinate (P3k) at ($(v1)!0.5!(v3k1)$) {};	
			%	\path(P3k) ++(90: 1 cm) node  (P3kL) {\color{mgray}$P_{3k+1}$};
			
			%	\foreach \i / \j in {1,3,6,9}
			%	\node[bred] (w\i) at (v\i) {};
			
			%	\draw[ltteal,thick, ->]  (v3) to [out=45,in=135]  (v4);
			
		\end{tikzpicture} 
		\caption{$P_{3k+4}$.}
		\label{fig:i:P3k4}
	\end{figure}	
	% <<< E: Figure {fig:i:P3k4}
	
	
	
	
	
	
	Moreover, if $S_1$ and $S_2$ are $i$-sets of $P_{3k+1}$,  
	we have that $\onlyedge{S_1,S_2}$ if and only if $\onlyedge{S_1',S_2'}$.  It follows that $\ig{P_{3k+4}}$ contains an induced copy of $\ig{P_{3k+1}} = \wlat_k$.  This accounts for $\binom{k+2}{2}$ of the $\binom{k+3}{2}$ vertices in $\ig{P_{3k+4}}$.  
	These are precisely the $i$-sets of $P_{3k+4}$ containing $v_{3k+3}$ (these correspond to the black and green vertices in Figure \ref{fig:i:P13ig}).
	
	Now, consider some $i$-set of $P_{3k+4}$ containing $v_{3k+4}$, say $S'$.  We observe $S'$ must contain either $v_{3k+1}$ or $v_{3k+2}$.  We consider the former first.   Let $S= S'-\{v_{3k+4}\}$.  Then $S$ dominates $\{v_1,v_2,\dots,v_{3k+1}\}$, is independent, and has size $k$.  So, $S$ is an $i$-set of $P_{3k+1}$.  
	
	Conversely, given an $i$-set of $P_{3k+1}$ containing the vertex $v_{3k+1}$, say $S$, the set $S' = S \cup \{v_{3k+4}\}$ is an $i$-set of $P_{3k+4}$.  Finally, given two such sets $S_1$ and $S_2$ and the corresponding $S_1'$ and $S_2'$, we have that $\onlyedge{S_1,S_2}$ if and only if $\onlyedge{S_1',S_2'}$.  This accounts for $k+1$ $i$-sets of $P_{3k+4}$.  We emphasize that these sets are all different from the $\binom{k+2}{2}$ $i$-sets mentioned above.
	
	%	\nts{RE Kieka Note: I don't know.  This entire proof is terrible.}
	
	Let $S_1'$ be an $i$-set of $P_{3k+4}$ containing $v_{3k+4}$ and not $v_{3k+2}$.  Then, let $S_2' = \left\lbrace S_1' \cup \{v_{3k+3}\}\right\rbrace - \{v_{3k+4}\}$.  Now $S_2'$ is  an independent dominating set of size $k+1$ and clearly $\onlyedge{S_1', S_2'}$.  Further, $S_2=S_2' - \{v_{3k+3}\}$ is a dominating set of $P_{3k+1}$ containing $v_{3k+1}$, and so $S_2$ is a path-tail vertex of $\wlat_k$.  Thus, $S_2$ is a path-tail vertex the left-hand side of the copy of $\wlat_k$ inside of $\wlat_{k+1}$ constructed above.    These correspond to the blue vertices matched to the green vertices in Figure \ref{fig:i:P13ig}.  
	
	
	The only remaining $i$-set of $P_{3k+4}$ has both $v_{3k+2}$ and $v_{3k+4}$.  This set is precisely $\{v_2, v_5, \dots, v_{3k+2}, v_{3k+4}\}$.  
	This is the lower left blue vertex in Figure \ref{fig:i:P13ig}, and is adjacent only to  $\{v_2, v_5, \dots, v_{3k+1}, v_{3k+4}\}$, thus forming the (unique) path-tail vertex of $\wlat_{k+1}$ of degree 1.  
	
	Note there are $\binom{k+2}{2}+(k+1)+1 = \binom{k+3}{2}$ $i$-sets, which accounts for all the $i$-sets of $P_{3k+4}$.  
	We emphasize that the $i$-sets of $P_{3k+4}$ containing $v_{3k+4}$ are precisely the path-tail vertices of $\wlat_{k+1}$ (down the left-hand side), verifying the invariant for $k+1$.  The result follows by induction.  
\end{proof}



Bringing Lemmas \ref{lem:i:2Pn} and \ref{lem:i:p3k1}  together yields the following theorem.

%>>> S: Theorem {thm:i:Pn}
\begin{theorem} \label{thm:i:Pn}
	For $n \geq 1$, $k \geq 0$,	and $\wlat_k$ as  defined	in Section \ref{sec:i:Pn},
	\begin{align*}
		\ig{P_n} = 	
		\begin{cases}
			K_1 & \text{ if } n=3k, \\
			\wlat_k & \text{ if } n=3k+1, \\
			P_{k+2} & \text{ if } n=3k+2.
		\end{cases}
	\end{align*}
\end{theorem}
% <<< E: Theorem {thm:i:Pn}


\section{The Number of $i$-Sets of Cycles}	\label{sec:i:numCycles}

Our methods for counting the $i$-sets of cycles are similar to paths, but with an added complication.  Although one might be 
tempted to use the equation in Equation (\ref{eq:i:pathEq})  (restated here for reference)  with the constraints $1 \leq x_j \leq 2$ for $j \in \{1, \dots, t\}$,  this method does not take the rotational symmetries of the $i$-sets of $C_n$ into account: 

\vspace{-1cm}
\begin{align} \label{eq:i:pathEq2}
	x_1 + \dots + x_t = r, \text{ where } 0 \leq x_1, x_t \leq 1, \text{ and } 1 \leq x_j \leq 2 \text{ for } j \in \{2,\dots,t-1\}.
\end{align}




For example, it will count only one $i$-set of $C_n$, $n \equiv 0 \Mod{3}$, whereas $C_n$ has three $i$-sets in this case. Moreover, different $i$-sets have different types of rotational symmetries, and so we cannot simply proceed as above and then multiply the answer by a constant.  

Let $e=v_0v_{n-1}$ be an edge of $C_n$ and consider $P_n = C_n - e = (v_0, \dots, v_{n-1})$.  Any $i$-set $X$ of $C_n$ corresponds to either an $i$-set of $P_n$ or to a ``near $i$-set of $P_n$'', in which 

\begin{enumerate}[label=(\roman*)]
	\item $v_2$ is the first vertex and $v_{n-1}$ is the last vertex of $X$ on $P_n$, or
	\item $v_0$ is the first vertex and $v_{n-3}$ is the last vertex of $X$ on $P_n$.
\end{enumerate}

These three varieties of $i$-sets are pairwise disjoint; therefore, we count the number of $i$-sets by using Equation (\ref{eq:i:pathGF})
with different conditions on the integer variables.  The number of $i$-sets of $C_n$ equals the sum of the number of integer solutions to the following equations, where $r$, $t$ and $x_j$ are defined as in Section \ref{sec:i:numbPaths}:
\begin{align} \label{eq:i:cycleEQ}
	x_1 + \dots + x_t = r, & \text{ where } x_1=0 \text{ and } 1 \leq x_j \leq 2 \text{ for } j \in \{ 2,\dots,t \},\\
	x_1 + \dots + x_t = r, & \text{ where } x_t=0 \text{ and } 1 \leq x_j \leq 2 \text{ for } j \in \{1,\dots,t-1\},\\
	x_1 + \dots + x_t = r, & \text{ where } x_1=x_t=1 \text{ and } 1 \leq x_j \leq 2 \text{ for } j \in \{2,\dots,t-1\}.
\end{align}

The corresponding generating function is 
\begin{align} \label{eq:i:cycleGF}
	(x + x^2)^{t-1} + (x+x^2)^{t-1} + x^2(x+x^2)^{t-2} = 2x^{t-1}(1+x)^{t-1} + x^t(1+x)^{t-2}.
\end{align}

\begin{enumerate}[label=$\bullet$]
	\item If $n=3k+1$, then $t = i(P_n) + 1 = k+2$ and $r=2k$.  Thus, we require the coefficient of $2k$ in (\ref{eq:i:cycleGF}),  that is, in $2x^{k+1}(1+x)^{k+1} + x^{k+2}(1+x)^k$, which is $2 \binom{k+1}{k-1} + \binom{k}{k-2} = k(3k+1)/2$.
	
	\item If $n = 3k+2$, then $t=k+2$ and $r=2k+1$.  Thus, we require the coefficient of $2k+1$ in $2x^{k+1}(1+x)^{k+1}+x^{k+2}(1+x)^k$, which is $2\binom{k+1}{k} + \binom{k}{k-1} = 3k+2 = n$.
	
	\item If $n=3k+3$, then $t=k+2$ and $r=2k+2$.  Thus, we require the coefficient of $2k+2$ in $2x^{k+1}(1+x)^{k+1} + x^{k+2}(1+x)^k$, which is 3.
	
\end{enumerate}

We summarize this result in the lemma below.

%>>> S: Lemma {lem:i:cycleSize}
\begin{lemma} \label{lem:i:cycleSize}
	For $n \geq 3$, the order of $\ig{C_n}$ is 
	
	\begin{align*}
		\left|  V \left(\ig{C_n} \right)  \right|  = 
		\begin{cases}
			3 & \text{ if } \ n = 3k\\ 
			k(3k+1)/2 & \text{ if }  \ n = 3k+1 \\
			n & \text{ if }  \ 3k+2.
		\end{cases}
	\end{align*}
\end{lemma}
% <<< E: Lemma {lem:i:cycleSize}





%---------------------------------------SECTION: $i$-graphs of cycles ----------------
\section{The $i$-Graphs of Cycles}	\label{sec:i:cycles}


Contrary to our conventions in the previous sections on paths, we assume that all cycles have labelled vertices $V(C_n) = (v_0,v_1,\dots,v_{n-1})$.  

Immediately, we discover that some of the $i$-graphs for $C_n$ are fairly straight-forward.  For a $C_{3}$, this is a complete graph, and hence $\ig{C_3} = \ig{K_3} = C_3$.  For $C_{3k}$ with  $3k \geq 6$,  $|V(C_{3k})|$ is divisible by 3, and hence there are three distinct $i$-sets of $C_{3k}$, in which each $i$-set vertex has two non-$i$-set vertices between it and the next $i$-set vertex; thus each $i$-set vertex is frozen, and hence $\ig{C_n}$ consists of three singletons.   

For $C_{n}$ with $n \equiv 2 \Mod{3}$, say $n=3k+2$, each $i$-set of $C_n$ contains exactly one pair of vertices $v_{j-1}$ and $v_{j+1}$ that are separated by exactly one vertex, $v_j$, not in the $i$-set (the common neighbour of these two vertices), while all other pairs  of consecutive $i$-set vertices are separated by exactly two vertices not in the $i$-set.  Hence, the $i$-set has exactly two vertices, namely $v_{j-1}$ and $v_{j+1}$, that are not frozen, and each of them can slide in only one direction.  The vertex $v_{j-1}$ can move to $v_{j-2}$, and $v_{j+1}$ can move to $v_{j+2}$.  In the former case,  the vertex that has two neighbours  in the $i$-set is now $v_{j-3}$, while in the latter, it is $v_{j+3}$.  As a result, $\ig{C_n}$ is 2-regular and since 3 is coprime to $n$, the sequence $v_j$, $v_{j+3}$, $v_{j+6}$, $\dots$, of the vertex dominated by two vertices in the $i$-set will visit each vertex in $C_n$  after $n$-slides.  Thus, each $i$-set is generated; that is, $\ig{C_n}$ is connected.  As it is 2-regular and connected, we conclude it is a cycle.




We provide examples of $C_5$ and $C_8$ in Figures \ref{fig:i:C5} and \ref{fig:i:C8} below.  




%>>> S: Figure {fig:i:C5}	
\begin{figure}[H] \centering
	\begin{subfigure}{.4\textwidth}
		\begin{tikzpicture}			
			%---------- REF-------------------
			\coordinate (cent) at (0,0) {};
			
			
			%=========  MAIN CYCLE W/O BOXED NODES
			\foreach \i  in {0,1,2,3,4} 
			{
				\node[std,label={90-\i*72: ${v_{\i}}$}] (v\i) at (90-\i*72:2cm) {}; 
			}
			
			
			\draw[thick] (v0)--(v1)--(v2)--(v3)--(v4)--(v0);
			
			%======== RED NODES 
			\foreach \i  in {1,3} 
			{
				\node[regRed] (r\i) at (v\i) {};
			}
		\end{tikzpicture} 
		\caption{$C_5$ with the $i$-set $\{v_1,v_3\}$ in red.}
		\label{subfig:i:C5}
	\end{subfigure}
	\hspace{10mm}
	\begin{subfigure}{.4\textwidth}
		\begin{tikzpicture}			
			%---------- REF-------------------
			\coordinate (cent) at (0,0) {};
			
			
			%=========  MAIN CYCLE W/O BOXED NODES
			\foreach \i / \x / \y  in {0/0/2,1/0/3,2/1/3,3/1/4,4/2/4} 
			{
				\node[std,label={90-\i*72: $\{v_{\x},v_{\y}\}$}] (v\i) at (90-\i*72:2cm) {}; 
			}	
			
			\draw[thick] (v0)--(v1)--(v2)--(v3)--(v4)--(v0);
			
			%======== RED NODES 
			\foreach \i  in {2} 
			{
				\node[regRed] (r\i) at (v\i) {};
			}
		\end{tikzpicture} 
		\caption{$\ig{C_5}$ with the vertex corresponding to the $i$-set $\{v_1,v_3\}$ in red. }
		\label{subfig:i:C5i}
	\end{subfigure}
	\caption{$C_5$ and its $i$-graph.}
	\label{fig:i:C5}
\end{figure}
% <<< E: Figure {fig:i:C5}


%>>> S: Figure {fig:i:C8}	
\begin{figure}[H] \centering
	\begin{subfigure}{.4\textwidth}
		\begin{tikzpicture}			
			%---------- REF-------------------
			\coordinate (cent) at (0,0) {};
			
			
			%=========  MAIN CYCLE W/O BOXED NODES
			\foreach \i  in {0,1,2,3,4,5,6,7} 
			{
				\node[std,label={90-\i*45: ${v_{\i}}$}] (v\i) at (90-\i*45:2cm) {}; 
			}
			
			
			\draw[thick] (v0)--(v1)--(v2)--(v3)--(v4)--(v5)--(v6)--(v7)--(v0);
			
			%======== RED NODES 
			\foreach \i  in {1,3,6} 
			{
				\node[regRed] (r\i) at (v\i) {};
			}
		\end{tikzpicture} 
		\caption{$C_8$ with the $i$-set $\{v_1,v_3,v_6\}$ in red.}
		\label{subfig:i:C8}
	\end{subfigure}
	\hspace{6mm}
	\begin{subfigure}{.5\textwidth}
		\begin{tikzpicture}			
			%---------- REF-------------------
			\coordinate (cent) at (0,0) {};
			
			
			%=========  MAIN CYCLE W/O BOXED NODES
			\foreach \i / \x / \y / \z in {0/0/2/5,1/0/3/5,2/0/3/6,3/1/3/6,4/1/4/6,5/1/4/7,6/2/4/7,7/2/5/7} 
			{
				\node[std,label={90-\i*45: $\{v_{\x},v_{\y},v_{\z}\}$}] (v\i) at (90-\i*45:2cm) {}; 
			}	
			
			\draw[thick] (v0)--(v1)--(v2)--(v3)--(v4)--(v5)--(v6)--(v7)--(v0);
			
			%======== RED NODES 
			\foreach \i  in {3} 
			{
				\node[regRed] (r\i) at (v\i) {};
			}
		\end{tikzpicture} 
		\caption{$\ig{C_8}$ with the vertex corresponding to the $i$-set $\{v_1,v_3,v_6\}$ in red. }
		\label{subfig:i:C8i}
	\end{subfigure}
	\caption{$C_8$ and its $i$-graph.}
	\label{fig:i:C8}
\end{figure}
% <<< E: Figure {fig:i:C8}





We summarize these results in the lemma below.  
% <<< E: Lemma {lem:i:2Cn}
%>>> S: Lemma {lem:i:2Cn}
\begin{lemma} \label{lem:i:2Cn}
	For $n \geq 3$, $k \geq 0$,	
	\begin{align*}
		\ig{C_n} \cong 	
		\begin{cases}
			K_3 & \text{ if } n=3 \\
			3K_1 & \text{ if } n=3k \geq 6 \\
			C_n & \text{ if } n \equiv 2 \Mod{3}.
		\end{cases}
	\end{align*}
\end{lemma}
% <<< E: Lemma {lem:i:2Cn}




%\nts{Does third case of above $\ig{C_n} \cong C_n$ need more of a formal proof than the "frozen token" explanation on the previous page?}


Once again, the case that requires deeper analysis is $n = 3k+1$ for $k \geq 1$.  To help us tackle this final case, we first introduce a new notation for referencing the $i$-sets of cycles.  

For cycles $C_{3k+1}$, given any $i$-set $X$, there are exactly two vertices in $C_{3k+1} - X$ that are doubly dominated (that is, are adjacent to two different vertices of $X$).  Rather than referring to the $i$-set by its elements, which given a large $k$, could be numerous, we instead refer to the $i$-sets by these two unique vertices.  For clarity, we use a wide-angled bracketed notation when using this convention.  Figure \ref{fig:i:C13}	below illustrates this system:  for (a), rather than calling the $i$-set $X=\{v_1,v_3,v_6,v_9,v_{12}\}$, we refer to it as $\left\langle 0,2\right\rangle = \left\langle 2,0\right\rangle $.  Similarly in (b), instead of $Y=\{v_1,v_4,v_6,v_9,v_{12}\}$, we denote the $i$-set as $\left\langle 0,5\right\rangle $.





%>>> S: Figure {fig:i:C13}	
\begin{figure}[H] \centering	
	
	\begin{subfigure}{.35\textwidth}
		\begin{tikzpicture}			
			%---------- REF-------------------
			\coordinate (cent) at (0,0) {};
			
			
			%=========  MAIN CYCLE W/O BOXED NODES
			\foreach \i  in {1,3,4,5,6,7,8,9,10,11,12} 
			{
				\node[std,label={90-\i*27.6923: ${\i}$}] (v\i) at (90-\i*27.6923:2cm) {}; 
			}
			
			
			%========= BOXED NODES
			\foreach \i  in {0,2} 
			{
				
				\node[std,label={90-\i*27.6923: $\color{blue} \fbox{{\i}}$}] (v\i) at (90-\i*27.6923:2cm) {}; 
				
			}	
			
			\draw[thick] (v0)--(v1)--(v2)--(v3)--(v4)--(v5)--(v6)--(v7)--(v8)--(v9)--(v10)--(v11)--(v12)--(v0);
			
			%======== RED NODES 
			\foreach \i  in {1,3,6,9,12} 
			{
				\node[regRed] (r\i) at (v\i) {};
			}
		\end{tikzpicture} 
		\caption{$X=\{v_1,v_3,v_6,v_9,v_{12}\}$, \\ denoted as $\left\langle 0,2\right\rangle$.}
		\label{subfig:i:C13b}
	\end{subfigure}
	\hspace{3cm}
	\begin{subfigure}{.35\textwidth}
		\begin{tikzpicture}			
			%---------- REF-------------------
			\coordinate (cent) at (0,0) {};
			
			
			%=========  MAIN CYCLE W/O BOXED NODES
			\foreach \i  in {1,2,3,4,6,7,8,9,10,11,12} 
			{
				\node[std,label={90-\i*27.6923: ${\i}$}] (v\i) at (90-\i*27.6923:2cm) {}; 
			}
			
			
			%========= BOXED NODES
			\foreach \i  in {0,5} 
			{
				
				\node[std,label={90-\i*27.6923: $\color{blue} \fbox{{\i}}$}] (v\i) at (90-\i*27.6923:2cm) {}; 
				
			}	
			
			\draw[thick] (v0)--(v1)--(v2)--(v3)--(v4)--(v5)--(v6)--(v7)--(v8)--(v9)--(v10)--(v11)--(v12)--(v0);
			
			%======== RED NODES 
			\foreach \i  in {1,4,6,9,12} 
			{
				\node[regRed] (r\i) at (v\i) {};
			}
		\end{tikzpicture} 
		\caption{$Y=\{v_1,v_4,v_6,v_9,v_{12}\}$, \\ denoted as $\left\langle 0,5\right\rangle$.}
		\label{subfig:i:C13a}
	\end{subfigure}
	\caption{Two $i$-sets of $C_{13}$.}
	\label{fig:i:C13}
\end{figure}	
% <<< E: Figure {fig:i:C13}


%----------------

With this new labelling in place, we move now to a proposed family of graphs that we have dubbed \emph{bracelet graphs}, $\bcl_{k}$.
The vertex set of $\bcl_{k}$ consists of all distinct $2$-subsets
$\{j,\ell\}$ of $\{0,1,\dots,3k\}$ such that $0\leq j\leq3k$ and $\ell\equiv
j+3s+2(\operatorname{mod}\ 3k+1)$, $s\in\{0,1,\dots,k-1\}$. For example, the
subsets containing $0$ are $\{0,2\},\{0,5\},\dots,\{0,3k-4\},\{0,3k-1\}$.

To simplify notation, assume the vertices of $C_{3k+1}$ are labelled $0,1,\dots,3k$ in clockwise
order, as illustrated in Figure \ref{fig:i:C13}. In $\mathfrak{B}_{k}$, the neighbours of
the vertex $\{j,\ell\}$ are described below.
\begin{enumerate}
	\item Suppose $j-\ell\equiv2$ or $-2\ (\operatorname{mod}\ 3k+1)$. Assume
	without loss of generality that $j$ precedes $\ell$ in clockwise order around
	$C_{3k+1}$. (Thus, for the subsets $\{0,2\}$ and $\{11,0\}$ of $C_{13}$, for
	example, $0$ precedes $2$, while $11$ precedes $0$.) With arithmetic performed
	modulo $3k+1$, the neighbours of $\{j,\ell\}$ are $\{j,\ell+3\}$ and
	$\{j-3,\ell\}$, and $\{j,\ell\}$ has degree $2$ in $\mathfrak{B}_{k}$. (Thus,
	$\{0,2\}$ is adjacent to $\{0,5\}$ and $\{10,2\}$ in $\mathfrak{B}_{4}$, while
	$\{11,0\}$ is adjacent to $\{11,3\}$ and $\{8,0\}$.)
	
	\item Suppose $j-\ell\not \equiv 2$ or $-2\ (\operatorname{mod}\ 3k+1)$. Then
	the neighbours of $\{j,\ell\}$ in $\mathfrak{B}_{k}$ are $\{j-3,\ell
	\},\{j+3,\ell\},\{j,\ell-3\}$ and $\{j,\ell+3\}$ (arithmetic modulo $3k+1$),
	and $\{j,\ell\}$ has degree $4$ in $\mathfrak{B}_{k}$. (For example, the
	neighbours, in $\mathfrak{B}_{4}$, of $\{0,5\}$ are $\{10,5\},\{3,5\},\{0,2\}$
	and $\{0,8\}$).
\end{enumerate}



\begin{comment}
	
	For $k \geq 1$, define the graphs $\bcl_{k}$ as follows. The vertex set of $\bcl_{k}$ consists of all distinct 2-subsets
	$\cid{j,\ell}$ of $\{0,1,\dots,3k\}$ such that $0 \leq j \leq 3k$ and $\ell \equiv j+3s+2 \Mod{3k+1}$, $s \in \{0,1,\dots,k-1\}$.  
	Note that $\bcl_{k}$ has order $\frac{1}{2} k (3k+1)$.  For $r\in \{1,2\}$, define $j_r$ and $\ell_r$ to be the least residues
	
	
	
	\begin{align*}
		\ell_1 \equiv \ell+3 \Mod{3k+1}, \ell_2 \equiv \ell-3 \Mod {3k+1} \\
		j_1 \equiv j+3 \Mod{3k+1}, j_2 \equiv j-3 \Mod{3k+1}.
	\end{align*}
	
	The neighbours in $\bcl_k$ of the vertex $\{j,\ell\}$ are all vertices in the set
	
	\begin{align*}
		\{\{j,\ell_1\}, \{j, \ell_2\} \in V(F_k): |\ell_r -j| \notin \{1,3k\}, \  r \in \{1,2\}  \} \cup  \\
		\{\{j_1,\ell\}, \{j_2, \ell\} \in V(F_k): |\ell -j_r| \notin \{1,3k\}, \  r \in \{1,2\}  \} .
	\end{align*}
	
\end{comment}

%\nts{RB: Include a small example $\{0,2\},\{0,5\},...$}


%\nts{Stuck on notation here. Bracelets should exist independently outside of the context of i-graphs  and hence use a $\{a,b\}$ notation.  But when specify the i-graph of a cycle (which so happens to be bracelet) we want $\bk{a,b}$ to make it clear that $\{a,b\}$ is not the $i$-set itself.}

We emphasize that the sets $\bk{j,\ell}$ are unordered.  For example, the neighbours of $\bk{2,7}$ in $\bcl_3$ are $\bk{2,0} = \bk{0,2}, \bk{2,4},\bk{9,7} = \bk{7,9}$ and $\bk{5,7}$.  Examples of these graphs are given in Figures \ref{fig:i:iC7},	\ref{fig:i:iC10},  \ref{fig:i:iC13},  and \ref{fig:i:iC16} below, but where the set braces are removed to reduce visual clutter.  			

The two-number identifiers on the vertices of a bracelet graph and the $i$-sets of a cycle $C_{3k+1}$ are no coincidence; in the following series of lemmas and observations, we show that they are one and the same.  That is, we show that the vertex $\{i,j\}$ in the bracelet graph $\bcl_k$ corresponds to the $i$-set of $C_{3k+1}$ represented by $\cid{i,j}$, and hence that $\ig{C_{3k+1}} \cong \bcl_k$.

%\nts{Still need to fix this}.



%......................  SUB SECTION:  theta (1,k,l) .........................
%\subsection{$\thet{1,k,\ell}$}  \label{subsec:c:1kl}

% +++++++++++++++++++++++++++  LABELLING +++++++++++++++++

%>>> S: Figure {fig:i:iC7}	
\begin{figure}[H] \centering	 	 
	\begin{tikzpicture}			
		%---------- REF-------------------
		%	\coordinate (cent) at (0,0) {};			
		
		%----- TOP LINE
		\foreach \c / \i / \j in {
			0/	0	/	2	,
			1/	3	/	5	,
			2/	6	/	1	,
			3/	2	/	4		
		}
		{
			\node[std,label={0: \scriptsize ${\i,\j}$}] (v\c) at (2*\c, 2) {}; 					
		}				
		
		%----- BOTTOM LINE
		\foreach \c / \i / \j in {
			0/	0	/	5	,
			1/	3	/	1	,
			2/	6	/	4								
		}
		{
			\node[std,label={0: \scriptsize ${\i,\j}$}] (z\c) at (1+2*\c, 0) {}; 					
		}			
		
		%----- EDGES		
		\foreach \i / \j in {0/1,1/2,2/3}
		{
			\draw (z\i)--(v\j);
			\draw (z\i)--(v\i);
		}
		
		\draw (v0) to [out=270,in=160]  (1,-1) to [out=0, in=180]  (5,-1)  to [out=10,in=270]  (v3); 
		%	\draw (z0) to [out=-55,in=-45]  (w4); 
		
	\end{tikzpicture} 
	\caption{$\bcl_2 \cong \ig{C_{7}}$.}
	\label{fig:i:iC7}		 
\end{figure}
% <<< E: Figure {fig:i:iC7}



%>>> S: Figure {fig:i:iC10}	
\begin{figure}[H] \centering	 	 
	\begin{tikzpicture}			\hspace{3mm}
		%---------- REF-------------------
		%	\coordinate (cent) at (0,0) {};			
		
		%----- TOP LINE
		\foreach \c / \i / \j in {
			0/	7	/	9	,
			1/	0	/	2	,
			2/	3	/	5	,
			3/	6	/	8	,
			4/	1	/	9	}
		{
			\node[std,label={0: \scriptsize ${\i,\j}$}] (v\c) at (2*\c, 2) {}; 					
		}				
		
		%----- MID LINE
		\foreach \c / \i / \j in {
			0/	2	/	7	,
			1/	0	/	5	,
			2/	3	/	8	,
			3/	1	/	6	,
			4/	4	/	9									
		}
		{
			\node[std,label={0: \scriptsize ${\i,\j}$}] (w\c) at (1+2*\c, 1) {}; 					
		}	
		
		%----- BOTTOM LINE
		\foreach \c / \i / \j in {
			0/	2	/	4	,
			1/	5	/	7	,
			2/	0	/	8	,
			3/	1	/	3	,
			4/	4	/	6							
		}
		{
			\node[std,label={0: \scriptsize ${\i,\j}$}] (z\c) at (2*\c, 0) {}; 					
		}			
		
		%----- EDGES
		\draw	(v0)--(w0)--(z0);
		\draw	(v1)--(w1)--(z1);
		\draw	(v2)--(w2)--(z2);
		\draw	(v3)--(w3)--(z3);
		\draw	(v4)--(w4)--(z4);
		
		\draw	(v1)--(w0)--(z1);
		\draw	(v2)--(w1)--(z2);
		\draw	(v3)--(w2)--(z3);
		\draw	(v4)--(w3)--(z4);
		
		\draw (v0) to [out=55,in=45]  (w4); 
		\draw (z0) to [out=-55,in=-45]  (w4); 
		
	\end{tikzpicture} 
	\caption{$\bcl_3 \cong \ig{C_{10}}$.}
	\label{fig:i:iC10}		 
\end{figure}
% <<< E: Figure {fig:i:iC10}


%>>> S: Figure {fig:i:C13}	
\begin{figure}[H]  \begin{minipage}{0.1\textwidth} 
	\begin{tikzpicture}		\hspace{-13mm}	
		%---------- REF-------------------
		%	\coordinate (cent) at (0,0) {};				
		
		%----- TOP LINE - T
		\foreach \c / \i / \j in {
			0/	0	/	2	,
			1/	3	/	5	,
			2/	6	/	8	,
			3/	9	/	11	,
			4/	1	/	12	,
			5/	2	/	4	
		}
		{
			\node[std,label={0: \scriptsize ${\i,\j}$}] (t\c) at (1+2*\c, 3) {}; 					
		}				
		
		%----- MID 1 LINE - V
		\foreach \c / \i / \j in {
			0/	2	/	10	,
			1/	0	/	5	,
			2/	3	/	8	,
			3/	6	/	11	,
			4/	1	/	9	,
			5/	4	/	12	,
			6/	2	/	7										
		}
		{
			\node[std,label={0: \scriptsize ${\i,\j}$}] (v\c) at (2*\c, 2) {}; 					
		}	
		
		%----- MID 2 LINE - W
		\foreach \c / \i / \j in {
			0/	5	/	10	,
			1/	0	/	8	,
			2/	3	/	11	,
			3/	1	/	6	,
			4/	4	/	9	,
			5/	7	/	12									
		}
		{
			\node[std,label={0: \scriptsize ${\i,\j}$}] (w\c) at (1+2*\c, 1) {}; 					
		}					
		
		%----- MID 3 LINE - Z
		\foreach \c / \i / \j in {
			0/	5	/	7	,
			1/	8	/	10	,
			2/	0	/	11	,
			3/	1	/	3	,
			4/	4	/	6	,
			5/	7	/	9	,
			6/	10	/	12										
		}
		{
			\node[std,label={0: \scriptsize ${\i,\j}$}] (z\c) at (2*\c, 0) {}; 					
		}			
		
		%----- EDGES
		\draw	(t0)--(v0)--(w0)--(z0);
		\draw	(t1)--(v1)--(w1)--(z1);
		\draw	(t2)--(v2)--(w2)--(z2);
		\draw	(t3)--(v3)--(w3)--(z3);
		\draw	(t4)--(v4)--(w4)--(z4);
		\draw	(t5)--(v5)--(w5)--(z5);
		
		\draw	(t0)--(v1)--(w0)--(z1);
		\draw	(t1)--(v2)--(w1)--(z2);
		\draw	(t2)--(v3)--(w2)--(z3);
		\draw	(t3)--(v4)--(w3)--(z4);
		\draw	(t4)--(v5)--(w4)--(z5);
		\draw	(t5)--(v6)--(w5)--(z6);			
		
		\draw (v0) to [out=135,in=45]  (v6); 
		\draw[very thick,red] (v0) to [out=-135,in=160]  (0.2,-1) to [out=-20,in=210] (z6); 
		\draw[very thick,red] (v6) to [out=-25,in=30]  (13,-0.5) to [out=210,in=-30] (z0); 
		
		
		%---- RED HAMIL EDGES
		\foreach \i / \j in {0/1,1/2,2/3,3/4,4/5,5/6}
		{
			\draw[very thick,red] (t\i)--(v\j);
			\draw[very thick,red] (t\i)--(v\i);
		}		
		
		\foreach \i / \j in {0/1,1/2,2/3,3/4,4/5,5/6}
		{
			\draw[very thick,red] (w\i)--(z\j);
			\draw[very thick,red] (z\i)--(w\i);
		}		
		
		
	\end{tikzpicture} 
\end{minipage}
	\caption{$\bcl_4 \cong \ig{C_{13}}$ with Hamiltonian cycle in red.}
	\label{fig:i:iC13}
\end{figure}
% <<< E: Figure {fig:i:iC13}


%>>> S: Figure {fig:i:C16}	
\begin{figure}[H] 	 \centering
	\begin{tikzpicture}	[scale=0.9]	 %	\hspace{-3cm}	
		%---------- REF-------------------
		%	\coordinate (cent) at (0,0) {};				
		
		%----- TOP LINE
		\foreach \c / \i / \j in {
			0/	0	/	2	,
			1/	3	/	5	,
			2/	6	/	8	,
			3/	9	/	11	,
			4/	12	/	14	,
			5/	1	/	15	,
			6/	2	/	4	,
			7/	5	/	7			
		}
		{
			\node[std,label={0: \scriptsize ${\i,\j}$}] (r\c) at (2*\c, 4) {}; 					
		}			
		
		%----- Mid1 LINE
		\foreach \c / \i / \j in {
			0/	0	/	5	,
			1/	3	/	8	,
			2/	6	/	11	,
			3/	9	/	14	,
			4/	1	/	12	,
			5/	4	/	15	,
			6/	2	/	7	,
			7/	5	/	10				
		}
		{
			\node[std,label={0: \scriptsize ${\i,\j}$}] (t\c) at (1+2*\c, 3) {}; 					
		}		
		
		%----- MID 2 LINE
		\foreach \c / \i / \j in {
			0/	5	/	13	,
			1/	0	/	8	,
			2/	3	/	11	,
			3/	6	/	14	,
			4/	1	/	9	,
			5/	4	/	12	,
			6/	7	/	15	,
			7/	2	/	10										
		}
		{
			\node[std,label={0: \scriptsize ${\i,\j}$}] (v\c) at (2*\c, 2) {}; 					
		}	
		
		%----- MID 3 LINE
		\foreach \c / \i / \j in {
			0/	8	/	13	,
			1/	0	/	11	,
			2/	3	/	14	,
			3/	1	/	6	,
			4/	4	/	9	,
			5/	7	/	12	,
			6/	10	/	15	,
			7/	2	/	13												
		}
		{
			\node[std,label={0: \scriptsize ${\i,\j}$}] (w\c) at (1+2*\c, 1) {}; 					
		}					
		
		%----- MID 3 LINE
		\foreach \c / \i / \j in {
			0/	8	/	10	,
			1/	11	/	13	,
			2/	0	/	14	,
			3/	1	/	3	,
			4/	4	/	6	,
			5/	7	/	9	,
			6/	10	/	12	,
			7/	13	/	15										
		}
		{
			\node[std,label={0: \scriptsize ${\i,\j}$}] (z\c) at (2*\c, 0) {}; 					
		}			
		
		%----- EDGES
		\draw	(r0)--(t0)--(v0)--(w0)--(z0);
		\draw	(r1)--(t1)--(v1)--(w1)--(z1);
		\draw	(r2)--(t2)--(v2)--(w2)--(z2);
		\draw	(r3)--(t3)--(v3)--(w3)--(z3);
		\draw	(r4)--(t4)--(v4)--(w4)--(z4);
		\draw	(r5)--(t5)--(v5)--(w5)--(z5);
		\draw	(r6)--(t6)--(v6)--(w6)--(z6);
		\draw	(r7)--(t7)--(v7)--(w7)--(z7);
		
		\draw	(r1)--(t0)--(v1)--(w0)--(z1);
		\draw	(r2)--(t1)--(v2)--(w1)--(z2);
		\draw	(r3)--(t2)--(v3)--(w2)--(z3);
		\draw	(r4)--(t3)--(v4)--(w3)--(z4);
		\draw	(r5)--(t4)--(v5)--(w4)--(z5);
		\draw	(r6)--(t5)--(v6)--(w5)--(z6);
		\draw	(r7)--(t6)--(v7)--(w6)--(z7);		
		
		
		\draw (v0) to [out=155,in=190]  (0.8,5) to [out=10, in=180]  (8,5.5)  to [out=0,in=35]  (t7);    
		\draw[red] (v0) to [out=-155,in=-190]  (0.8,-1) to [out=-10, in=180]  (8,-1.5)  to [out=0,in=-35]  (w7);    		
		
		
		\draw[color=blue] (r0) to [out=70,in=190]  (3,6) to [out=10, in=160]  (15,5.5)  to [out=-20,in=25]  (w7);    
		\draw[color=ForestGreen] (z0) to [out=-70,in=170]  (3,-2) to [out=-10, in=200]  (15,-1.5)  to [out=20,in=-25]  (t7);    		
		
	\end{tikzpicture} 
	\caption{$\bcl_5 \cong \ig{C_{16}}$.}
	\label{fig:i:iC16}	
\end{figure}
% <<< E: Figure {fig:i:iC16}


% +++++++++++++++++++++++++++  OBSERVATIONS  +++++++++++++++++





%>>> S: Lemma {lem:i:bcl}
\begin{lemma} \label{lem:i:bcl}
	For $k \geq 1$,		
	%	\begin{align*}
		$	\ig{C_{3k+1}} \cong 	\bcl_{k}$.
		%	\end{align*}
\end{lemma}
% <<< E: Lemma {lem:i:bcl}


To prove Lemma \ref{lem:i:bcl}, we first establish several lemmas, beginning with a formulation of the observation from page \pageref{sec:i:cycles}, for the case $n=3k+1$.  Going forward, we assume that all arithmetic in the $i$-set notation is performed modulo $3k+1$.


%>>> S: OBS {obs:i:2ver}
\begin{obs} \label{obs:i:2ver}
	Each $i$-set $X$ of $C_{3k+1}$ contains exactly two pairs of vertices, say $v_{j-1},v_{j+1}$ and $v_{\ell-1},v_{\ell+1}$,
	such that each pair has a common neighbour in $C_{3k+1}-X$; all other pairs of consecutive
	vertices of $X$ on $C_{3k+1}$ are separated by two vertices of $C_{3k+1}-X$. 
\end{obs}
% <<< E: OBS {obs:i:2ver}


%\nts{notation splits here - vertex in $C_n$ is $v_1,v_2,...$, with represented by ``short gap" $i$-set at, say $v_4$ and $v_6$ to be $\cid{4,6}$. - vertex is $v_k$ not $k$.}


%>>> S: Lemma {lem:i:Fdegree}
\begin{lemma} \label{lem:i:Fdegree}
	Each vertex of $\ig{C_{3k+1}}$ has degree 2 or 4.  
\end{lemma}


\begin{proof}
	For each $i$-set $X$ of $C_{3k+1}$, we  deduce from Observation \ref{obs:i:2ver} that $X$ is one of two types (with notation as in Observation \ref{obs:i:2ver}):
	
	\begin{enumerate}[label=]
		\item \textbf{Type 1:}\ $\{v_{j-1},v_{j+1}\}\cap\{v_{\ell-1},v_{\ell+1}\}\neq\varnothing$; in this case
		$|\{v_{j-1},v_{j+1}\}\cap\{v_{\ell-1},v_{\ell+1}\}|=1$. Say ${j+1}={\ell-1}$. In our wide-angled notation, Type 1
		$i$-sets are of the form $X=\left\langle j,j+2 \right\rangle$. %\ (\operatorname{mod}%
		%\ 3k+1)\right\rangle $. 
		%If $v_{j-1},v_{j+1},v_{\ell+1}$ occur in this order in a clockwise
		%direction on $C_{3k+1}$, 
		Then the subpath $(v_{j-1},v_j,v_{j+1},v_{j+2},v_{j+3})$  of $C_{3k+1}$ has tokens on $v_{j-1}, v_{j+1},$ and $v_{j+3}$. A
		token on $v_{j-1}$ can only slide counterclockwise to $v_{j-2}$, a token on $v_{j+3}$ can only slide clockwise to $v_{j+4}$, and a token on $v_{j+1}$ is frozen. Hence $X$ has degree $2$ in $\ig{C_{3k+1}}$ (all other tokens are frozen).
		
		\item \textbf{Type 2:}\ $\{v_{j-1},v_{j+1}\}\cap\{v_{\ell-1},v_{\ell+1}\}=\varnothing$. When $X$ is a Type 2
		$i$-set, and $v_{j-1},v_{j+1},v_{\ell-1},$ and $v_{\ell+1}$ occur in this order in a clockwise direction on
		$C_{3k+1}$, then each of $v_{j-1}$ and $v_{\ell-1}$ is immediately preceded
		(counterclockwise) by two vertices of $C_{3k+1}-X$, and each of $v_{j+1}$ and $v_{\ell+1}$ is
		immediately followed (clockwise) by two vertices of $C_{3k+1}-X$. Hence tokens
		on $v_{j-1}$ and $v_{\ell-1}$ can slide counterclockwise to $v_{j-2}$
		and $v_{\ell-1}$, respectively, while tokens on $v_{j+1}$ and
		$v_{\ell+1}$ can slide clockwise to $v_{j+2}$ and
		$v_{\ell+2}$, respectively. Hence $X$ has degree $4$ in
		$\ig{C_{3k+1}}$.
	\end{enumerate}
\end{proof}

% <<< E: Lemma {lem:i:Fdegree}



The following is straightforward from the above proof, but we include a proof here for completeness.

\newpage

%>>> S: Lemma {lem:i:bkNeigh}
\begin{lemma} \label{lem:i:bkNeigh}   
	
	Let $X$ be an $i$-set of $C_{3k+1}$.	
	\begin{enumerate}[label=(\roman*)]
		\item  \label{lem:i:bkNeigh:i}    If $X=\left\langle j,j+2\right\rangle $ for some $j\in
		\{0,...,3k\}$, then the neighbours of $X$ in $\ig{C_{3k+1}}$ are
		$\left\langle j,j+5\right\rangle $ and $\left\langle j-3,j+2\right\rangle $. 
		
		\item \label{lem:i:bkNeigh:ii}  If $X=\left\langle j,\ell\right\rangle $ for some $j\in
		\{0,...,3k\}$ and some $\ell$, where $j-\ell\not \equiv 2$ or
		$-2\ (\operatorname{mod}\ 3k+1)$, then $\ell=j+2+3s$, where $s\in
		\{1,...,k-2\}$. The neighbours of $X$ in $\ig{C_{3k+1}}$ are
		$\left\langle j,\ell+3\right\rangle ,\left\langle j,\ell-3\right\rangle
		,\left\langle j+3,\ell\right\rangle ,$ and $\left\langle j-3,\ell\right\rangle
		$.
	\end{enumerate}
\end{lemma}


\begin{proof} We prove the statement for $j=0$; the general
	cases follow by symmetry.	
	\begin{enumerate} [label=(\roman*)]
		\item Consider the $i$-set $X=\left\langle
		0,2\right\rangle $. Then $X$ is a Type 1  $i$-set (see Lemma \ref{lem:i:Fdegree}) of $C_{3k+1}$. Then
		$\{v_{-1},v_1,v_3\}\subseteq X$. By Lemma \ref{lem:i:Fdegree}, the token on $v_1$ is frozen, the token on $v_{-1}$ can slide to $v_{-2}$,
		creating the $i$-set $\left\langle -3,2\right\rangle $, and the token on $v_3$
		can slide to $v_4$, creating the $i$-set $\left\langle 0,5\right\rangle$.
		
		\item Consider the $i$-set $X=\left\langle
		0,\ell\right\rangle $, where $\ell\not \equiv 2$ or $-2\ (\operatorname{mod}%
		\ 3k+1)$. Then $X$ is a Type 2 $i$-set of $C_{3k+1}$, and there are tokens on
		$v_{-1},v_1,v_{\ell-1}$, and $v_{\ell+1}$. Moreover, since $v_{0}$ and $v_{\ell}$ are the only
		vertices that have two neighbours in $X$, the other vertices, in a clockwise
		direction from $v_{1}$, contained in $X$ are $v_{4},v_{7},...,v_{\ell-1}$, which shows that
		$\ell=3s+2$ for some $s\geq1$. Since $X$ is a Type 2 $i$-set, $\ell\neq3k-1$,
		that is, $s\neq k-1$. Hence $\ell=3s+2$, where $s\in\{1,...,k-2\}$.
		
		Similarly to the above, the token on $v_{-1}$ can slide to $v_{-2}$,
		forming the $i$-set $\left\langle -3,\ell\right\rangle $, the token on $v_{1}$ can
		slide to $v_{2}$, forming the $i$-set $\left\langle 3,\ell\right\rangle $, the
		token on $v_{\ell-1}$ can slide to $v_{\ell-2}$, forming the $i$-set $\left\langle
		0,\ell-3\right\rangle $, and the token on $v_{\ell+1}$ can slide to $v_{\ell+2}$,
		forming the $i$-set $\left\langle 0,\ell+3\right\rangle $. 		
	\end{enumerate}	
\end{proof}


% <<< E: Lemma {lem:i:bkNeigh}


This completes the proof of Lemma \ref{lem:i:bcl}.  Finally, combining Lemma \ref{lem:i:2Cn} and Lemma \ref{lem:i:bcl} reveals the full result for $i$-graphs of cycles.




%\emph{Now you can say something about a 1-1 correspondence between the
	%	vertices of }$\mathfrak{B}_{k}$ \emph{and the }$i$\emph{-sets in }$C_{3k+1}%
%$,\emph{ and that the neighbours of each i-set correspond to the neighbours of
	%	the corresponding vertex of }$\mathfrak{B}_{k}$\emph{. This proves that
	%}$\mathcal{I}(C_{3k+1})\cong\mathfrak{B}_{k}$.

% <<< E: Lemma {lem:i:bkNeigh}


\begin{comment}
	
	%>>> S: Lemma {lem:i:Fcon}
	\begin{lemma} \label{lem:i:Fcon}
		$\ig{C_{3k+1}}$  is connected.
	\end{lemma}
	% <<< E:Lemma {lem:i:Fcon}
	
	
	\nts{This is still all wrong.}
	
	%\textbf{Claim:}  
	If $\cid{a,b}$ and $\cid{a,c}$ are $i$-sets of $C_{3k+1}$, then $\cid{a,b}$ and $\cid{a,c}$ are in the same component of $\ig{C_{3k+1}}$.  That is, the $i$-set corresponding to the labelling $\cid{a,b}$ can be transformed into the $i$-set corresponding to $\cid{a,c}$ through a series of token slides.
	
	
	
	If $\cid{a,b}$ is an $i$-set, then $a-b \equiv 2 \Mod{3}$ and $b = a + 2 + 3n \Mod{3k +1}$.
	
	
	We  claim that if there is $a'$ and $b'$ such that $\cid{a,a'}$ and $\cid{b,b'}$ are $i$-sets, then $a'\neq b'$.  
	Notice $a'=b+3j \Mod{3k+1}$ and $b'=a+3\ell \Mod{3k+1}$
	
	Suppose not.  Then, 
	
	\begin{align*}
		a'&=b'\\
		b+3j &= a+3\ell \\
		b-a &= 3\ell -3j
	\end{align*}
	
	Then, $b-a \equiv 2 \Mod{3}$ but $3\ell-3j \equiv 0 \Mod{3}$.  
	
\end{comment}




%Bringing Lemmas \ref{lem:i:2Cn} and \ref{lem:i:bcl} together yields the following.

%>>> S: Theorem {thm:i:Cn}
\begin{theorem} \label{thm:i:Cn}
	For $n \geq 3$, $k \geq 0$,		
	\begin{align*}
		\ig{C_n} = 	
		\begin{cases}
			K_3 & \text{ if } n=3 \\
			3K_1 & \text{ if } n=3k \geq 6 \\
			\bcl_{k} & \text{ if } n = 3k+1\\
			C_n & \text{ if } n \equiv 2 \Mod{3}.
		\end{cases}
	\end{align*}
\end{theorem}
% <<< E:Theorem {thm:i:Cn}



%---------------------------------------SECTION: $i$-graphs of cycles ----------------
\section{Hamiltonicity of $\ig{C_n}$}	\label{sec:i:hamCycle}

In some of the figures presented in Section \ref{sec:i:cycles}, a Hamiltonian cycle or path is easily found; in others, it is not.  This leads to the problem of determining for which values of  $n$ $\ig{C_n}$ is Hamiltonian or Hamilton traceable (i.e. has a Hamiltonian path).  In most cases, this is not too difficult to determine, as we show next.  

%>>> S: Theorem {thm:i:Ham}
\begin{theorem} \label{thm:i:Ham}  For $n\geq 3$,
	\begin{enumerate}[label=$\bullet$]
		\item If $n \equiv 0 \Mod{3}$, $\ig{C_n}$ is disconnected.
		\item If $n \equiv 2 \Mod{3}$, $\ig{C_n}$ is trivially Hamiltonian.
		\item If $n  \equiv 4 \Mod{6}$, $\ig{C_n}$ is neither Hamiltonian nor Hamilton traceable.
	\end{enumerate}
\end{theorem}
% <<< E: Theorem {thm:i:Ham}


\begin{proof}	
	The first two cases are trivial, and so assume that $n \equiv 4 \Mod{6}$. Since $\ig{C_4} = \overline{K_2}$, it is non-Hamiltonian;
	hence, assume $n >4$, say $n = 6k+4$, $k \geq 1$.   With notation as above, we first count the
	number of vertices $\cid{j, \ell}$ of $\ig{C_n}$ such that $j \equiv \ell \Mod{2}$.  For $j=0$, the set of these
	vertices is
	\begin{align*}
		\mathscr{X}_0 = \{\cid{0,2}, \cid{0,8}, \dots, \cid{0,6k+2}\}
	\end{align*}	
	and $| \mathscr{X}_0| = k+1$.   Similarly, 	
	\begin{align*}
		\mathscr{X}_j = \{\cid{j,j+2}, \cid{j,j+8}, \dots, \cid{j,j+6k+2}\}
	\end{align*}	
	and
	\begin{align*}
		\sum^{6k+3}_{j=0} |\mathscr{X}_j| = (k+1)(6k+4).
	\end{align*}	
	But each vertex $\cid{j,\ell}, j \equiv \ell \Mod{2}$  occurs in exactly two sets, namely $\mathscr{X}_j$ and $\mathscr{X_{\ell}}$.	
	Hence,
	\begin{align*}
		\left| \bigcup_{j=0}^{6k+3} \mathscr{X}_j \right| = (k+1)(3k+2).
	\end{align*}
	
	\noindent Since 
	\begin{align*}
		V(\ig{C_n}) = \frac{n(n-1)}{6} = (3k+2)(2k+1),
	\end{align*}
	$\ig{C_n}$ has $k(3k+2)$ vertices $\cid{j,\ell}$ such that $j \not\equiv \ell \Mod{2}$.  
	%To show that $\ig{C_n}$ is non-Hamiltonian, we need the following result.
	
	
	
	%>>> S: Theorem {thm:i:HamComp}
	%\begin{theorem} \label{thm:i:HamComp} \cite{CLZ}   If $G$ is a Hamiltonian graph, then for every nonempty proper
	%	subset $S$ of $V(G)$, the number of components of $G-S$ does not exceed $|S|$
	%\end{theorem}
	% <<< E: Theorem {thm:i:HamComp}
	
	
	
	%>>> S: Theorem {thm:i:C6k4Ham}
	%\begin{theorem} \label{thm:i:C6k4Ham} For $n=6k+4$, $\ig{C_n}$ is neither Hamiltonian nor Hamilton traceable.
	%\end{theorem}
	% <<< E: Theorem {thm:i:C6k4Ham}
	
	
	
	%>>> S: Proof of Theorem {thm:i:Ham}
	%\begin{proof}
	We show next that each vertex $\cid{j,\ell}$ such that $j \equiv \ell \Mod{2}$ is adjacent only to vertices $\cid{j',\ell'}$ such that $j' \not\equiv \ell' \Mod{2}$, and vice versa.  
	
	
	Let $\cid{j,\ell}$ be any vertex of $\ig{C_n}$ such that $j \equiv \ell \Mod{2}$.  Then, with arithmetic in
	the subscripts performed modulo $n$, 
	\begin{align*}
		N(\cid{j,\ell}) = \begin{cases}
			\{  \cid{j, \ell+3},  \cid{j, \ell-3},  \cid{j+3, \ell},  \cid{j-3, \ell}  \} & \text{ if } \ell -j \not\equiv \pm 2 \Mod{n} \\
			\{  \cid{j, \ell+3},   \cid{j-3, \ell}  \} & \text{ if } \ell -j \equiv  2 \Mod{n} \\
			\{    \cid{j, \ell-3},  \cid{j+3, \ell} \} & \text{ if } \ell -j \equiv -2 \Mod{n}, \\			
		\end{cases}
	\end{align*}
	
	
	
	\noindent	where, since $n$  is even, $j \not\equiv \ell \pm 3  \Mod{2}$ and  $\ell \not\equiv j \pm 3 \Mod{2}$. Hence each vertex $\cid{j,\ell}$ of
	$\ig{C_n}$ such that $j \equiv \ell \Mod{2}$  is adjacent only to vertices  $\cid{j',\ell'}$ such that $j' \not\equiv \ell' \Mod{2}$.  
	
	
	Therefore, $\ig{C_n}$ is bipartite with $(k+1)(3k+2)$ vertices in one partite set, and $k(3k+2)$ in the other.   Since the cardinalities of the partite sets differ by more than one, the result follows.  
\end{proof}

%	Let $S= \{\cid{j,\ell} : j \not\equiv \ell \Mod{2} \}$.  Then $|S| = k(3k+2)$ and $G-S$ consists of the $(k+1)(3k+2)$ isolated vertices $\cid{j,\ell}$, $j \equiv \ell \Mod{2}$.  By (the contrapositive of) Theorem \ref{thm:i:HamComp}, $\ig{C_n}$ is non-Hamiltonian.  
%\end{proof}


The case for $n \equiv 1 \Mod{6}$ is more complicated.   From Figures \ref{fig:i:iC7} and \ref{fig:i:iC13} given above, $\ig{C_7}$ and $\ig{C_{13}}$ are Hamiltonian:  $\ig{C_7}$ trivially so,  and for  $\ig{C_{13}}$, illustrated in red.    For $n \equiv 1 \Mod{6}$ with $n \geq 19$, we claim that $\ig{C_n}$ is not Hamiltonian.  Consider $\ig{C_{19}}$ given in Figure \ref{fig:i:iC19} below.   Any Hamiltonian cycle on $\ig{C_{19}}$ would include all of the vertices of degree 2 and their  degree 4 neighbours, as highlighted in red in Figure \ref{fig:i:iC19}.  However, this (proper) subset of vertices induces a cycle in $\ig{C_{19}}$, and so $\ig{C_{19}}$ is not Hamiltonian.  A similar argument follows for larger $n$  with $n \equiv 1 \Mod{6}$, as we show next.

% <<< E:  Proof of Theorem {thm:i:Ham}

%>>> S: Figure {fig:i:iC19}	
\begin{figure}[H] \centering	 	 
	\begin{tikzpicture}[scale=0.88]			
		%---------- REF-------------------
		%	\coordinate (cent) at (0,0) {};			
		
		%----- TOP LINE - V
		\foreach \c / \i / \j in {
			0/	0	/	2	,
			1/	3	/	5	,
			2/	6	/	8	,
			3/	9	/	11	,
			4/	12	/	14	,
			5/	15	/	17	,
			6/	18	/	1	,
			7/	2	/	4	,
			8/	5	/	7	,
			9/	8	/	10	,
			10/	11	/	13	,
			11/	14	/	16	
		}
		{
			\node[std,label={0: \scriptsize ${\i,\j}$}] (v\c) at (1.5*\c, 5) {}; 					
		}				
		
		%----- MID 1  - W
		\foreach \c / \i / \j in {
			0/	0	/	5	,
			1/	3	/	8	,
			2/	6	/	11	,
			3/	9	/	14	,
			4/	12	/	17	,
			5/	15	/	1	,
			6/	18	/	4	,
			7/	2	/	7	,
			8/	5	/	10	,
			9/	8	/	13	,
			10/	11	/	16										
		}
		{
			\node[std,label={0: \scriptsize ${\i,\j}$}] (w\c) at (0.75+1.5*\c, 4) {}; 					
		}			
		
		%----- MID 2 - T
		\foreach \c / \i / \j in {
			0/	0	/	8	,
			1/	3	/	11	,
			2/	6	/	14	,
			3/	9	/	17	,
			4/	12	/	1	,
			5/	15	/	4	,
			6/	18	/	7	,
			7/	2	/	10	,
			8/	5	/	13	,
			9/	8	/	16							
		}
		{
			\node[std,label={0: \scriptsize ${\i,\j}$}] (t\c) at (1.5+1.5*\c, 3) {}; 					
		}	
		
		
		%----- MID 3 - Y
		\foreach \c / \i / \j in {
			0/	0	/	11	,
			1/	3	/	14	,
			2/	6	/	17	,
			3/	9	/	1	,
			4/	12	/	4	,
			5/	15	/	7	,
			6/	18	/	10	,
			7/	2	/	13	,
			8/	5	/	16			
		}
		{
			\node[std,label={0: \scriptsize ${\i,\j}$}] (y\c) at (2.25+1.5*\c, 2) {}; 					
		}
		
		
		%----- MID 4 - X
		\foreach \c / \i / \j in {
			0/	0	/	14	,
			1/	3	/	17	,
			2/	6	/	1	,
			3/	9	/	4	,
			4/	12	/	7	,
			5/	15	/	10	,
			6/	18	/	13	,
			7/	2	/	16			
		}
		{
			\node[std,label={0: \scriptsize ${\i,\j}$}] (x\c) at (3+1.5*\c, 1) {}; 					
		}		
		
		
		%----- BOTTOM LINE - Z
		\foreach \c / \i / \j in {
			0/	0	/	17	,
			1/	3	/	1	,
			2/	6	/	4	,
			3/	9	/	7	,
			4/	12	/	10	,
			5/	15	/	13	,
			6/	18	/	16							
		}
		{
			\node[std,label={0: \scriptsize ${\i,\j}$}] (z\c) at (3.75+1.5*\c, 0) {}; 		
		}			
		
		%----- EDGES
		
		\foreach \c in {0,1,2,3,4,5,6}
		{\draw	(v\c)--(w\c)--(t\c)--(y\c)--(x\c)--(z\c);
		}
		
		\draw (v7)--(w7)--(t7)--(y7)--(x7);
		\draw (v8)--(w8)--(t8)--(y8);
		\draw (v9)--(w9)--(t9);
		\draw (v10)--(w10);
		
		\draw (v11)--(w10)--(t9)--(y8)--(x7)--(z6);
		\draw (v10)--(w9)--(t8)--(y7)--(x6)--(z5);
		\draw (v9)--(w8)--(t7)--(y6)--(x5)--(z4);
		\draw (v8)--(w7)--(t6)--(y5)--(x4)--(z3);
		\draw (v7)--(w6)--(t5)--(y4)--(x3)--(z2);	
		\draw (v6)--(w5)--(t4)--(y3)--(x2)--(z1);
		\draw (v5)--(w4)--(t3)--(y2)--(x1)--(z0);
		\draw (v4)--(w3)--(t2)--(y1)--(x0);
		\draw (v3)--(w2)--(t1)--(y0);
		\draw (v2)--(w1)--(t0);
		\draw (v1)--(w0);
		
		%	\draw	(v2)--(w2)--(z2);
		%	\draw	(v3)--(w3)--(z3);
		%	\draw	(v4)--(w4)--(z4);
		
		%	\draw	(v1)--(w0)--(z1);
		%	\draw	(v2)--(w1)--(z2);
		%	\draw	(v3)--(w2)--(z3);
		%	\draw	(v4)--(w3)--(z4);
		
		
		
		% --- RED CYCLE EDGES 
		\draw[very thick,red] (v0) to [out=250,in=160]  (3,-1) to [out=0, in=180]  (13.5,-1)  to [out=10,in=-35]  (x7);    		
		
		\draw[very thick,red]  (v11) to [out=300,in=20]  (13.5,-2.5) to [out=180, in=0]  (4.5,-2.5)  to [out=170,in=215]  (x0);    
		
		\foreach \i / \j in {0/1,1/2,2/3,3/4,4/5,5/6,6/7,7/8,8/9,9/10,10/11}
		{
			\draw[very thick,red] (w\i)--(v\j);
			\draw[very thick,red] (v\i)--(w\i);
		}
		
		
		\foreach \i / \j in {0/1,1/2,2/3,3/4,4/5,5/6,6/7}
		{
			\draw[very thick,red] (z\i)--(x\j);
			\draw[very thick,red] (z\i)--(x\i);
		}
		
		
		%--- secondary curved edges
		\draw[thick,blue]  (t0) to [out=250,in=160]  (3,-1.5) to [out=0, in=180]  (13.5,-1.5)  to [out=10,in=-35]  (t9); 
		\draw (w0) to [out=250,in=160]  (3,-1.7) to [out=0, in=180]  (13.5,-1.7)  to [out=10,in=-35]  (y8); 
		\draw (y0) to [out=250,in=160]  (3,-1.9) to [out=0, in=180]  (13.5,-1.9)  to [out=10,in=-35]  (w10);   
		
		%---- PURPLE FOR HPATH
		\foreach \i / \j in {0/1,1/2,2/3,3/4,4/5,5/6,6/7,7/8,8/9}
		{
			\draw[thick,blue] (y\i)--(t\j);
			\draw[thick,blue] (y\i)--(t\i);
		}			
		%	\draw[color=red] (r0) to [out=70,in=190]  (3,6) to [out=10, in=160]  (15,5.5)  to [out=-20,in=25]  (w7);    
		%	\draw[color=red] (z0) to [out=-70,in=170]  (3,-2) to [out=-10, in=200]  (15,-1.5)  to [out=20,in=-25]  (t7);    		
		
		%	\draw[ultra thick,green, dashed] (x7)--(y8);
		
	\end{tikzpicture} 
	\caption{$\ig{C_{19}}$.}
	\label{fig:i:iC19}		 
\end{figure}
% <<< E: Figure {fig:i:iC19}


%As a consolation, although not Hamiltonian, $\ig{C_{19}}$ does contain a Hamilton path.  In Figure \ref{fig:i:iC19}, start at the vertex for $\cid{0,2}$ and move along the red path at the top of the graph through $\cid{3,5}$, $\cid{3,8} \dots  \cid{14,16}$.  Then, take the red edge down to the bottom left of the graph to 
%$\cid{0,14}$ and follow the red path along the bottom of the graph through $\cid{0,17}$, $\cid{3,17}$, \dots, $\cid{2,16}$.  From there,  take the green edge up to $\cid{5,16}$, and follow the blue path through $\cid{ 8,16}$, and then around the long blue edge to the left side of the graph to $\cid{0,8}$, through the blue path and finally ending at $\cid{5,13}$.
%We claim that for all  $n \equiv 1 \Mod{6}$, $\ig{C_{n}}$ contains a Hamilton path, constructed similarly to the path in Figure \ref{fig:i:iC19}: traversing first the outer alternating degree 2 (red) pathway and then moving through the inner (blue) cycles one level at a time.  

%>>> S: Figure {thm:i:1m6nonHam}
\begin{theorem} \label{thm:i:1m6nonHam}
	If $n\geq 19$ and $n \equiv 1 \Mod{6}$, then $\ig{C_n}$ is not Hamiltonian.
\end{theorem}
% <<< E: Figure {thm:i:1m6nonHam}

%We know what happens if $k\leq3$, so assume $k\geq4$.\bigskip

\begin{proof}
	As shown in Lemma \ref{lem:i:bkNeigh} \ref{lem:i:bkNeigh:i}, the neighbours of the $i$-set
	$X=\cid{0,2}$ in $\ig{C_{6k+1}}$ are
	$\cid{-3,2}$ and $\cid{ 0,5}$. On the
	other hand, the neighbours of $\cid{0,5}$ are
	$\cid{0,2}$, $\cid{0,8}$,
	$\cid{ 3,5}$ and $\cid{-3,5}$. Note
	that $\cid{0,2}$ and $\cid{3,5}$ are
	Type 1 vertices while $\cid{0,8}$ and $\cid{
		-3,5}$ are Type 2. Similarly, $\cid{-3,2}$
	has Type 1 neighbours $\cid{0,2}$ and $\cid{
		-3,-1}$, and Type 2 neighbours $\cid{ -3,5}$
	and $\cid{-6,2}$. A similar remark holds for any $i$-set
	$\cid{j,j+5},\ j\in\{0,...,3k\}$. We call these $i$-sets
	\emph{Type 2a} $i$-sets.
	
	Consider the $i$-set $X=\left\langle 0,2+3s\right\rangle $, where
	$s\in\{2,...,2k-3\}$ (hence $k\geq 3$). As proved in Lemma  \ref{lem:i:bkNeigh} \ref{lem:i:bkNeigh:ii}, the neighbours of $X$ in $\ig{C_{6k+1}}$ are the Type 2
	$i$-sets
	\[
	\cid{ 0,2+3(s+1)},\cid{ 0,2+3(s-1)}
	,\cid{ 3,2+3s},\cid{ -3,2+3s}.
	\]
	Observe that for the given range of $s$, these are all Type 2 $i$-sets. A
	similar remark holds for any $i$-set $\cid{ j,j+2+3s}$,
	where $s\in\{2,...,k-3\}$. We call these $i$-sets \emph{Type 2b} $i$-sets.
	
	Therefore there are exactly $6k+1$ Type 1 $i$-sets of $C_{6k+1}$ and the same
	number of Type 2a $i$-sets, and each Type 1 $i$-set has exactly two neighbours
	in $\ig{C_{6k+1}}$, both of which are Type 2a $i$-sets, and,
	conversely, each Type 2a $i$-set has two Type 1 neighbours in $\ig{C_{6k+1}}$. We deduce that the subgraph of $\ig{C_{6k+1}}$ induced by
	its Type 1 vertices and their neighbours consists of only Type 1 and Type 2a
	vertices, and is $2$-regular. Hence if $\ig{C_{6k+1}}$ has Type 2b
	vertices, that is, if $k\ge3$, then $\ig{C_{6k+1}}$ is
	non-Hamiltonian. 
	%This leaves only the case where $k=4$, and a Hamiltonian
	%cycle of $\ig{C_{13}}$ is depicted in Figure \ref{fig:i:C13}.
\end{proof}



%--------------------


%--------------------
\subsection*{Traceability of $\ig{C_{6k+1}}$, where $k\geq3$.}  \label{subsec:i:C3k1Trac}

When $k\geq 3$ and $n=6k+1$, we have previously established that $\ig{C_{n}}$ has no Hamiltonian cycle.  We now instead prove that it has a Hamilton path.


%>>> S: E: Theorem {thm:i:C6r1}
\begin{theorem} \label{thm:i:C6r1}
	For $n=6k+1$, $k \geq 3$, $\ig{C_n}$ is Hamilton traceable.
\end{theorem}


\begin{proof}	
	Say $k\geq3$, and consider $C_{6k+1}$ and an $i$-set $\cid{j,\ell} $. The distance from $j$ to $\ell$ on $C_{6k+1}$ is the length of the shorter path, thus $d(j,\ell)\in
	\{2,5,8, \dots ,3k-1\}$.
	
	\begin{enumerate}[label=$\bullet$]
		\item If $k$ is even, say $k=2k^{\prime}$, then, in $C_{12k^{\prime}+1}$, we
		see that $d(j,\ell)\in\{2,5,8, \dots ,6k^{\prime}-1\}$. This set contains an equal
		number of even and odd integers.
		
		\item If $k$ is odd, say $k=2k^{\prime}+1$, then, in $C_{12k^{\prime}+7}$, we
		see that $d(j,\ell)\in\{2,5,8, \dots ,6k^{\prime}+2\}$. This set contains more even
		than odd integers.
	\end{enumerate}
	
	Consider, again, the subgraph of $\ig{C_{6k+1}}$ induced by its Type 1
	vertices and their neighbours, which is $2$-regular (as above) and has order
	$2(6k+1)$. Denote this graph by $\mathcal{H}_{2,5}$. 
	In Figure \ref{fig:i:iC19}, $\mathcal{H}_{2,5}$ is the subgraph induced by the vertices: 
	\begin{align*} \{\cid{0,2}, \cid{0,5}, \cid{3,5}, \cid{3,8}, \cid{6,8} \dots, \cid{2,16} \}. 
	\end{align*}
	
	\noindent We argue below that
	$\mathcal{H}_{2,5}$ is in fact connected; that is, $\mathcal{H}_{2,5}$ is a cycle.
	
	%\bigskip
	
	Note that, with arithmetic modulo $6k+1$,
	\begin{align*}
		\mathcal{W}_{2,5}  = 
		&\cid{ 0,2}, \cid{0,2+3}, \cid{3,2+3}, \cid{3,2+2\cdot3}, \cid{ 2\cdot3,2+2\cdot3},  \\
		& \dots, \cid{3x,2+3x}, \cid{3x, 2+3(x+1)} \dots
	\end{align*}	
	\noindent is a walk in $\mathcal{H}_{2,5}$. When does $\cid{ 0,2} $
	recur? There are two cases to consider, each having two subcases.
	
	%\bigskip
	
	\noindent\textbf{Case 1}:\textbf{\hspace{0.1in}}When $\cid{
		0,2} $ recurs for the first time, an even cycle is formed. Then
	$\cid{ 0,2} =\cid{ 3x,2+3x} $ for some
	integer $x$. Since $\cid{ 0,2} =\cid{
		2,0} $, there are two subcases.\smallskip
	
	\noindent\textbf{Case 1.1}:\hspace{0.1in}$3x\equiv2\ (\operatorname{mod}%
	\ 6k+1)$ and $2+3x\equiv0\ (\operatorname{mod}\ 6k+1)$, that is,
	$3x\equiv-2\ (\operatorname{mod}\ 6k+1)$. But then $2\equiv
	-2\ (\operatorname{mod}\ 6k+1)$, which is impossible because $k>0$.\smallskip
	
	\noindent\textbf{Case 1.2}:\hspace{0.1in}$3x\equiv0\ (\operatorname{mod}%
	\ 6k+1)$ and $2+3x\equiv2\ (\operatorname{mod}\ 6k+1)$. Since $\gcd
	(3,6k+1)=1$, $x\equiv0\ (\operatorname{mod}\ 6k+1)$. Then the first time
	$\cid{ 0,2} $ recurs on $\mathcal{W}_{2,5}$ is therefore
	when $x=6k+1$. It follows that $\mathcal{W}_{2,5}$ contains the cycle
	\begin{align*}
		\cid{0,2}, \cid{0,2+3}, \cid{3,2+3}, \cid{3, 2+2 \cdot 3}, \cid{2\cdot3,2+2\cdot3}, \dots, \cid{0,2}
	\end{align*}	
	of length $2(6k+1)=|V(\mathcal{H}_{2,5})|$. Therefore $\mathcal{H}_{2,5}\cong
	C_{12k+2}$.
	
	%	\bigskip
	
	\noindent\textbf{Case 2}:\textbf{\hspace{0.1in}}When $\cid{
		0,2} $ recurs for the first time, an odd cycle is formed. Then
	$\cid{ 0,2} =\cid{ 3x,2+3(x+1)} $ for
	some integer $x$. Again there are two subcases.\smallskip
	
	\noindent\textbf{Case 2.1}:\textbf{\hspace{0.1in}}$3x\equiv
	2\ (\operatorname{mod}\ 6k+1)$ and $2+3(x+1)\equiv0\ (\operatorname{mod}%
	\ 6k+1)$. Then $7\equiv0\ (\operatorname{mod}\ 6k+1)$, which is impossible
	because $k>1$.\smallskip
	
	\noindent\textbf{Case 2.2}:\textbf{\hspace{0.1in}}$3x\equiv
	0\ (\operatorname{mod}\ 6k+1)$ and $2+3(x+1)\equiv2\ (\operatorname{mod}%
	\ 6k+1)$. This is likewise impossible.
	
	Therefore, we conclude in all cases that $\mathcal{H}_{2,5} \cong C_{12k+2}$.
	
	
	%	\bigskip
	
	In general, for fixed $\ell\equiv2\ (\operatorname{mod}\ 6)$ and $2\leq
	\ell\leq3k-1$, denote the subgraph of $\ig{C_{6k+1}}$ induced by the
	$i$-sets of the form $\cid{ j,j+\ell} $ and $\cid{
		j,j+\ell+3} $, where $j\in\{0,...,6k\}$, by $\mathcal{H}_{\ell,\ell+3}$.
	
	Then	
	\begin{align*} 
		\mathcal{W}_{\ell, \ell+3} & = \cid{0,\ell}, \cid{0,\ell+3}, \cid{3, \ell+3}, \cid{3,\ell+2\cdot3}, \cid{2\cdot3, \ell+2\cdot3}, \\ 
		& \dots, \cid{3x, \ell+3x}, 
		\cid{3x, \ell+3(x+1)} \dots
	\end{align*}	
	is a walk in $\mathcal{H}_{\ell,\ell+3}$. When does $\cid{ 0,\ell} $
	recur? Again there are two cases to consider.
	
	\bigskip
	
	\noindent\textbf{Case 3}:\textbf{\hspace{0.1in}} When $\cid{ 0,\ell} $ recurs for the first time, an
	even cycle is formed. Then $\cid{ 0,\ell} =\cid{
		3x,\ell+3x} $ for some integer $x$. Since $\cid{
		0,\ell} =\cid{ \ell,0} $, there are two
	subcases.\smallskip
	
	\noindent\textbf{Case 3.1}:\textbf{\hspace{0.1in}}$3x\equiv
	\ell\ (\operatorname{mod}\ 6k+1)$ and $\ell+3x\equiv0\ (\operatorname{mod}\ 6k+1)$,
	that is, $3x\equiv-\ell\ (\operatorname{mod}\ 6k+1)$. Then $2\ell\equiv
	0\ (\operatorname{mod}\ 6k+1)$ and, since $\gcd(2,6k+1)=1$, $\ell\equiv
	0\ (\operatorname{mod}\ 6k+1)$. Since $2\leq \ell\leq3k-1$, this is
	impossible.\smallskip
	
	\noindent\textbf{Case 3.2}:\textbf{\hspace{0.1in}}$3x\equiv
	0\ (\operatorname{mod}\ 6k+1)$ and $\ell+3x\equiv \ell\ (\operatorname{mod}\ 6k+1)$,
	i.e., $x\equiv0\ (\operatorname{mod}\ 6k+1)$. Therefore the first time
	$\cid{ 0,\ell} $ recurs on $\mathcal{W}_{\ell,\ell+3}$ is when
	$x=6k+1$. It follows that $\mathcal{W}_{\ell,\ell+3}$ contains the cycle
	\begin{align*}
	\cid{ 0,\ell} ,\cid{ 0,\ell+3} ,\cid{
		3,\ell+3} ,...,\cid{ 3(x-1),\ell+3x} ,\cid{
		0,\ell}
	\end{align*}
	of length $2(6k+1)=|V(\mathcal{H}_{\ell,\ell+3})|$. Therefore $\mathcal{H}%
	_{\ell,\ell+3}\cong C_{12k+2}$.
	
	\bigskip
	
	\noindent\textbf{Case 4}:\textbf{\hspace{0.1in}}When $\cid{
		0,\ell} $ recurs for the first time, an odd cycle is formed. Then
	$\cid{ 0,\ell} =\cid{ 3x,\ell+3(x+1)} $ for
	some integer $x$.\smallskip
	
	\noindent\textbf{Case 4.1}:\textbf{\hspace{0.1in}}$3x\equiv
	\ell\ (\operatorname{mod}\ 6k+1)$ and $\ell+3(x+1)\equiv0\ (\operatorname{mod}%
	\ 6k+1)$. Then $2\ell+3\equiv0\ (\operatorname{mod}\ 6k+1)$, or $2\ell\equiv
	-3\equiv6k-2\ (\operatorname{mod}\ 6k+1)$. This implies that $\ell\equiv
	3k-1\ (\operatorname{mod}\ 6k+1)$ and the restrictions on $\ell$ show that
	$\ell=3k-1$. That is, there is exactly one value of $\ell$ for which these
	congruences hold. Moreover, $6x+3\equiv0\ (\operatorname{mod}\ 6k+1)$, i.e.,
	$2x\equiv-1\equiv6k\ (\operatorname{mod}\ 6k+1)$. Hence $x=3k$. 
	
	Therefore $\mathcal{W}_{\ell,\ell+3}$ contains the cycle%
	\begin{align*}
	\cid{ 0,\ell} ,\cid{ 0,\ell+3} ,\cid{
		3,\ell+3} ,...,\cid{ 3x,\ell+3x} ,\cid{
		\ell,0}
	\end{align*}
	of length $2x+1=6k+1$. Consider the distances $d(0,3k-1)$ and $d(0,3k+2)$ on
	$C_{6k+1}$. Observe that $d(0,\ell)=d(0,3k-1)=3k-1$ and
	$d(0,\ell+3)=d(0,3k+2)=6k+1-(3k+2)=3k-1$. It follows that $\mathcal{H}_{\ell,\ell+3}$
	consists of all $i$-sets $\cid{ p,q} $ such that, on
	$C_{6k+1}$, $d(p,q)=3k-1$, and there are exactly $6k+1$ such $i$-sets. Hence
	$|V(\mathcal{H}_{\ell,\ell+3})|=6k+1$, that is, $\mathcal{H}_{\ell,\ell+3}$ is exactly the
	cycle $\cid{ 0,\ell} ,\cid{ 0,\ell+3}
	,\cid{ 3,\ell+3} ,...,\cid{ 3x,\ell+3x}
	,\cid{ \ell,0} $. 
	
	Since $\ell=3k-1$ and $\ell\equiv2\ (\operatorname{mod}\ 6)$, we deduce that $k$ is
	odd; say $k=2k^{\prime}+1$, where $k^{\prime}\geq1$. Then $\ell=6k^{\prime}+2$.
	The smallest cycle $C_{6k+1}$ where $k\geq3$ for which this case occurs is
	$C_{19}$, in which case $\ell=8$ (see Figure \ref{fig:i:iC19}).
	
	\noindent\textbf{Case 4.2}:\textbf{\hspace{0.1in}}$3x\equiv
	0\ (\operatorname{mod}\ 6k+1)$ and $\ell+3(x+1)\equiv \ell\ (\operatorname{mod}%
	\ 6k+1)$. From the first congruence, $x\equiv0\ (\operatorname{mod}\ 6k+1)$,
	and so, from the second congruence, $3\equiv0\ (\operatorname{mod}\ 6k+1)$.
	This is impossible.
	
	\paragraph*{To summarize:}
	
	Fix $\ell\in\{2,5,8,...,3k-1\}$.  	
	\begin{enumerate}[label=$\bullet$]
		\item If $k=2k^{\prime}$, then by (1) and Cases 3 and 4, the subgraphs
		$\mathcal{H}_{2,5},...,\mathcal{H}_{6k^{\prime}-4,6k^{\prime}-1}$ of
		$\ig{C_{12k^{\prime}+1}}$ all have order $24k^{\prime}+2$, and
		$\mathcal{H}_{2,5}\cong\cdots\cong\mathcal{H}_{6k^{\prime}-4,6k^{\prime}%
			-1}\cong C_{24k^{\prime}+2}$.
		
		\item If $k=2k^{\prime}+1$, then by (2) and Cases 3 and 4, the subgraphs
		$\mathcal{H}_{2,5},...,\mathcal{H}_{6k^{\prime}-4,6k^{\prime}-1}$ of
		$\ig{C_{12k^{\prime}+7}}$ all have order $24k^{\prime}+14$, and
		$\mathcal{H}_{2,5}\cong\cdots\cong\mathcal{H}_{6k^{\prime}-4,6k^{\prime}%
			-1}\cong C_{24k^{\prime}+14}$. However, the subgraph $\mathcal{H}_{6k^{\prime
			}+2,6k^{\prime}+5}$ of $\ig{C_{12k^{\prime}+7}}$ has order
		$12k^{\prime}+7$ and, as for the subgraph $\mathcal{H}_{8,11}$ of
		$\ig{C_{19}}$, is a cycle, that is, $\mathcal{H}_{6k^{\prime
			}+2,6k^{\prime}+5}\cong C_{12k^{\prime}+7}$.
	\end{enumerate}	
	\noindent 	In either case, each vertex of $\ig{C_{6k+1}}$ belongs to $H_{\ell,\ell+3}$
	for some $\ell\equiv2\ (\operatorname{mod}\ 6)$. 
	
	\subsubsection*{Connecting the Subgraphs $\mathcal{H}_{\ell,\ell+3}$ to Form a
		Hamilton Path of $\ig{C_{6k+1}}$}
	
	Denote the subgraph of $\ig{C_{6k+1}}$ that consists of the union of
	the cycles $\mathcal{H}_{\ell,\ell+3}$ by $\mathcal{H}$, and the set of edges of
	$\ig{C_{6k+1}}$ that do not belong to $\mathcal{H}$ by $\mathcal{E}$.
	Since each vertex of $\ig{C_{6k+1}}$ belongs to $H_{\ell,\ell+3}$ for some
	$\ell\equiv2\ (\operatorname{mod}\ 6)$, $\mathcal{H}$ is a spanning subgraph of
	$\ig{C_{6k+1}}$. We consider two cases, depending on whether $k$ is
	even or odd.  
	
		
	\noindent\textbf{Case 1:\hspace{0.1in}}$k=2k^{\prime}$. Then
	\begin{align*}
	\mathcal{P}:\cid{ 0,2} ,\cid{ 0,5}
	,\cid{ 0,8} ,...,\cid{ 0,6k^{\prime}-4}
	,\cid{ 0,6k^{\prime}-1}
	\end{align*}
	is a path in $\ig{C_{12k^{\prime}+1}}$ whose edges belong alternately
	to $\mathcal{H}$ and to $\mathcal{E}$, beginning with the edge $(\cid{
		0,2} ,\cid{ 0,5} )$ in $\mathcal{H}_{2,5}$ and
	ending with the edge $(\cid{ 0,6k^{\prime}-4} ,\cid{
		0,6k^{\prime}-1} )$ in $\mathcal{H}_{6k^{\prime}-4,6k^{\prime}-1}%
	$. Moreover, $\mathcal{P}$ contains at least one vertex of each $\mathcal{H}%
	_{\ell,\ell+3}$. Let $\mathcal{T}$ be the subgraph of $\ig{C_{12k^{\prime
			}+1}}$ obtained by deleting all edges of $\mathcal{P}$ from $\mathcal{H}$,
	then adding the edges of $E(\mathcal{P})\cap\mathcal{E}$. Observe that
	$\mathcal{T}$ is a spanning subgraph of $\ig{C_{12k^{\prime}+1}}$.
	Since the edges of $\mathcal{P}$ were alternately deleted and added, all
	vertices of $\mathcal{T}$ have degree $2$, except for $\cid{
		0,2} $ and $\cid{ 0,6k^{\prime}-1} $, which
	have degree $1$. Also, by construction, $\mathcal{T}$ is connected. Therefore,
	$\mathcal{T}$ is a Hamiltonian path of $\ig{C_{12k^{\prime}+1}}$.
	
	
	
	\noindent\textbf{Case 2:\hspace{0.1in}}$k=2k^{\prime}+1$. The argument is similar.
	
	
	Therefore, in all cases $\ig{C_{6k+1}}, k\geq 3$ is Hamilton traceable.
\end{proof}

This completes the characterization of which cycles have Hamiltonian or Hamiltonian
traceable $i$-graphs.

