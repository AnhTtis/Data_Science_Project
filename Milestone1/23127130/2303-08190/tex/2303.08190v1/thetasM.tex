\chapter{Constructions of $i$-Graphs Using Complements}

\label{ch:comp}




	
When visualizing the connections between the $i$-sets of a graph $G$, it is sometimes advantageous to consider its complement, $\cp{G}$, instead. 
From a human perspective, it is curiously easier see to which vertices a vertex $v$ is adjacent, rather than to which vertices $v$ is nonadjacent.
This is especially true when $i(G) = 2$ or $3$, when we may interpret the adjacency of $i$-sets of $G$ as the adjacencies of edges and triangles (i.e. $K_3$), respectively, in $\cp{G}$.  

In the following sections, we examine how the use of graph complements can be exploited to construct the $i$-graph seeds for certain classes of line graphs, theta graphs, and maximal planar graphs.   In Sections \ref{sec:c:thetaComp} and \ref{sec:c:nonthetas} we further confirm that these theta graph constructions complete the full characterization of theta graphs that are $i$-graphs, by proving that each of the remaining theta graphs are not   realizable as $i$-graphs.



%---------------------------------------SECTION: Line Graphs----------------
\section{Line Graphs and Claw-free Graphs}	\label{sec:c:line}


To begin, consider a graph $G$ with $i(G)=2$ and where $X = \{u,v\}$ is an $i$-set of $G$.  In $\cp{G}$, $u$ and $v$ are adjacent, so $X$ is represented as the edge $uv$.  Moreover, no other vertex $w$ is adjacent to both vertices of $X$ in $\cp{G}$; otherwise, $\{u,v,w\}$ is independent in $G$, contrary to $X$ being an $i$-set.   %That is, if $uv\not\in E(G)$, then $uv$ does not belong to any triangles in $\cp{G}$.





Consider now the line graph of $\cp{G}$, $L(\cp{G})$.  If $X=\{u,v\}$ is an $i$-set of $G$, then $e=uv$ is an edge of $\cp{G}$ and hence $e$ is a vertex of $L(\cp{G})$.  Thus, the $i$-sets of $G$ correspond to a subset of the vertices of $L(\cp{G})$. In the case where $\cp{G}$ is triangle-free (that is, $G$ has no independent sets of cardinality three or more,  i.e. $G$  is well-covered), these  $i$-sets of $G$ are exactly the vertices of $L(\cp{G})$.
Now suppose $Y$ is an $i$-set of $G$ adjacent to $X$; say, $\edge{X,u,w,Y}$, so that $Y=\{v,w\}$.  Then, in $\cp{G}$, $f=vw$ is an edge, and so in $L(\cp{G})$, $f \in V(L(\cp{G}))$.  Since $e$ and $f$ are both incident with $v$ in $\cp{G}$, $ef \in E(L(\cp{G}))$.
That is, for $i$-sets $X$ and $Y$ of a well-covered graph $G$, $X \sim Y$ if and only if $X$ and $Y$ correspond to adjacent vertices in $L(\cp{G})$.   Thus $\ig{G} \cong L(\cp{G})$ for well-covered  graphs $G$ with $i(G) = \alpha(G)=2$.

In the example illustrated in Figure \ref{fig:c:compEx} below, $\mathcal{H}$ is the house graph, where $X=\{a,c\}$ and $Y=\{c,e\}$ are $i$-sets with $\edge{X,a,e,Y}$.  In $L(\cp{\mathcal{H}})$, the two vertices in each of these $i$-sets are likewise adjacent.   


%~~~~~~~~~~~~FIG START~~~~~~~~~~~
\begin{figure}[H]
	\centering
	\begin{tikzpicture}%[line width=0.3mm,scale=1]			
		
		%---------- REF-------------------
		\node(cent) at (0,0) {};			
		\path (cent) ++(0:50 mm) node(mcent)  {};	
		\path (cent) ++(0:100 mm) node(rcent)  {};
		
		%----- box ----------	
		\foreach \i/\c in {a/-135,b/135,d/45, e/-45} {				
			\path (cent) ++(\c:10 mm) node [std] (v\i) {};
			\path (v\i) ++(\c:5 mm) node (Lv\i) {${\i}$};
		}
		
		\draw[thick] (va)--(vb)--(vd)--(ve)--(va)--cycle;
		
		\path (cent) ++(90:15 mm) node [std] (vc) {};
		\path (vc) ++(90:5 mm) node (Lvc) {${c}$};
		
		\draw[thick] (vb)--(vc)--(vd);	
		
		\path (cent) ++(-90: 20mm) node (HL)  {$\mathcal{H}$};
		
		
		%----- mid obj ----------	
		\foreach \i/\c in {a/-135,b/135,d/45, e/-45} {				
			\path (mcent) ++(\c:10 mm) node [std] (m\i) {};
			\path (m\i) ++(\c:5 mm) node (Lm\i) {${\i}$};
		}
		
		\path (mcent) ++(90:15 mm) node [std] (mc) {};
		\path (mc) ++(90:5 mm) node (Lmc) {${c}$};
		
		\draw[thick] (mb)--(me)--(mc)--(ma)--(md);
		
		\path (mcent) ++(-90: 20mm) node (HcL)  {$\cp{\mathcal{H}}$};
		
		%----- far obj ---------
		\foreach \i/\j/\c in {c/e/-135,b/e/135,a/d/45, a/c/-45} {				
			\path (rcent) ++(\c:10 mm) node [std] (r\i\j) {};
			\path (r\i\j) ++(\c:8 mm) node (Lr\i\j) {$\{\i,\j\}$};
		}
		
		\draw[thick] (rbe)--(rce)--(rac)--(rad);
		\path (rcent) ++(-90: 20mm) node (HiL)  {$L(\cp{\mathcal{H}}) \cong \ig{\mathcal{H}}$};
		
	\end{tikzpicture}				
	\caption{The complement and line graphs complement of the well-covered house graph $\mathcal{H}$.}
	\label{fig:c:compEx}
\end{figure}
%~~~~~~~~~~~~~FIG END~~~~~~~~~~~~~


Before continuing with the theme of $i$-graph realizability, we make the following observation on the order of $\ig{G}$ when $i(G)=2$.	


%>>> START Prop {prop:c:numb2isets}
\begin{prop} 
	\label{prop:c:numb2isets}
	Let $G$ be a graph of order $n$ with $i(G)=2$.  Then $G$ has at most $ \frac{1}{2}n(n-1) - \frac{1}{2}\sum_{v\in V(G)} deg(v)$ distinct $i$-sets.  That is, $|V(\ig{G})| \leq  \frac{1}{2}n(n-1) - \frac{1}{2}\sum_{v\in V(G)} \deg(v)$.
\end{prop}

\begin{proof}
	Since $i(G)=2$, each $i$-set of $G$ corresponds to an edge of $\cp{G}$.  So, $|V(\ig{G})| \leq  |E(\cp{G})|=\binom{n}{2} - |E(G)| =  \frac{1}{2}n(n-1) - \frac{1}{2}\sum_{v\in V(G)} \deg(v)$.
\end{proof}
% <<< END: Prop {prop:c:numb2ise`ts}

This connection between graphs with $i(G)=2$ and line graphs helps us not only understand the structure of $\ig{G}$, but also lends itself towards some interesting realizability results.  
We follow this thread for the remainder of this section, and build towards determining the $i$-graph realizability of line graphs and claw-free graphs.

%>>> START Lemma {lem:c:lineK3}
\begin{lemma}
	\label{lem:c:lineK3}
	%The line graphs of a connected graph $G$ contains $\Dia$ as an induced subgraph if and only if $G$ is not $K_3$ and $G$ contains a triangle.
	
	%\ft{OR:} 
	The line graph of a connected graph $G$ of order at least four contains $\Dia$ as an induced subgraph if and only if $G$ contains a triangle. 
\end{lemma}

\begin{proof}
	Suppose $G$ contains a triangle with vertices $a,b,$ and $c$, and let $d\notin \{a,b,c\}$ be a vertex adjacent to (say) $a$.  Then $\{ab, ac, ad\}$ and $\{ab,ac,bc\}$ are triangles in $L(G)$.  Since $\{b,c\} \cap \{a,d\} = \varnothing$ in $G$, $bc$ is nonadjacent to $ad$ in $L(G)$.  That is,  $\Dia$ is an induced subgraph of $L(G)$.
	
	Conversely, suppose $L(G)$ contains $\Dia$ as an induced subgraph.  Then, $G$ has at least four edges, and hence $G \not\cong K_3$.  
	Since the graph $H$ in Figure \ref{fig:c:paw} (i.e. ``the paw" or the ``3-pan") has $L(H) \cong \Dia$, it follows from the uniqueness of line graphs  (see \cite{W32})
	% in \cite{Hara}) 
	that $G$ contains $H$ as a subgraph.
\end{proof}
% <<< END: Lemma {lem:c:lineK3}



\begin{figure}[H] \centering	
	\begin{tikzpicture}			
		%---------- REF-------------------
		\node(cent) at (0,0) {};
		\path (cent) ++(0,0 mm) node(lcent)  {};			
		\path (cent) ++(0:50 mm) node(mcent)  {};	
		
		%----- paw ----------	
		\node[std,label={0:$a$}]  at (lcent) (va) {}; 
		
		%-d
		\path (lcent) ++(90:10 mm) node [std] (vd) {};
		\path (vd) ++(90:5 mm) node (Lvd) {${d}$};
		%-c
		\path (lcent) ++(180:10 mm) coordinate (hc1) {};	
		\path (hc1) ++(-90:10 mm) node [std] (vc) {};
		\path (vc) ++(-90:5 mm) node (Lvc) {${c}$};	
		
		%-b
		\path (vc) ++(0:20 mm) node [std] (vb) {};
		\path (vb) ++(-90:5 mm) node (Lvb) {${b}$};	
		
		\draw[thick](va)--(vb)--(vc)--(va)--cycle;
		\draw[thick](va)--(vd);	
		
		\path (cent) ++(-90: 22mm) node (HL)  {$H$};		
		
		%----- mid obj ----------	
		\foreach \i/\c in {ad/0,ac/90,bc/180, ab/-90} {				
			\path (mcent) ++(\c:10 mm) node [std] (m\i) {};
			\path (m\i) ++(\c:5 mm) node (Lm\i) {${\i}$};
		}
		
		\draw[thick] (mad)--(mac)--(mbc)--(mab)--(mad) -- cycle;
		\draw[thick] (mac)--(mab);
		
		\path (mcent) ++(-90: 22mm) node (HcL)  {$L(H) \cong \Dia$};		
	\end{tikzpicture}
	\caption{The ``paw'' $H$ with $L(H) \cong \Dia$.}
	\label{fig:c:paw}		
\end{figure}


%*REFREF* Harary Textbook Theorem 8.3 - H. Whitney, Congruent Graph and Connectivity of Graphs, Amer J Math. 54 1932 150-169


%>>> START Theorem {thm:c:line}
\begin{theorem}
	\label{thm:c:line}  Let $H$ be a connected line graph.  Then $H$ is an $i$-graph if and only if $H$ is $\Dia$-free.	
\end{theorem}

\begin{proof}  Suppose $H$ is an $i$-graph.  By Proposition \ref{prop:i:diamond} 
	%($\Dia$ result)
	and Corollary \ref{coro:i:notInduced},  
	%(induced subgraphs result), 
	$H$ is $\Dia$-free.
	
	Conversely, suppose that $H$ is $\Dia$-free.  If $H$ is complete, then $H$ is the $i$-graph of itself.  So, assume that $H$ is not complete.  Say $H$ is the line graph of some graph $F$, where we may assume $F$ has no isolated vertices (as isolated vertices do not affect line graphs).  Since $H$ is $\Dia$-free and connected, $F$ has no triangles by Lemma  \ref{lem:c:lineK3}.  Since $F$ has edges (which it does since $H$ exists), $
	\alpha(\cp{F}) \leq 2$.  Moreover, as $F$ is connected, $\cp{F}$ has no universal vertices, and so $i(\cp{F})\geq 2$.   Thus, $i(\cp{F}) = \alpha(\cp{F})$ and $\cp{F}$ is well-covered. It follows that every edge of $F$ corresponds to an $i$-set of $\cp{F}$.  Since $H$ is the line graph of $F$, it is the $i$-graph of $\cp{F}$.  
\end{proof}
% <<< END: Theorem {thm:c:line}

Finally, if we examine Beineke's forbidden subgraph characterization for line graphs (see \cite{B70}), it is interesting to note that eight of the nine minimal non-line graphs contain an induced $\Dia$, and are therefore not $i$-graphs.  The ninth minimal non-line graph is the claw, $K_{1,3}$.  Thus, claw-free graphs without an induced $\Dia$ are $\Dia$-free line graphs, and therefore are $i$-graphs. 

%>>> START  Coro {coro:c:clawfree} 
\begin{coro}	\label{coro:c:clawfree}  
	Let $H$ be a connected claw-free graph.  Then $H$ is an $i$-graph if and only if $H$ is $\Dia$-free.	
\end{coro}

\begin{proof}
	If $H$ is $i$-graph realizable, it is $\Dia$-free.
	Conversely, suppose that $H$ is $\Dia$-free.  Then by the forbidden subgraph characterization of line graphs \cite{B70}, $H$ is a line graph.  From Theorem \ref{thm:c:line}, $H$ is a $\Dia$-free line graphs and thus an $i$-graph.
\end{proof}
% <<< END: Coro {coro:c:clawfree} 

While Theorem \ref{thm:c:line} and Corollary \ref{coro:c:clawfree} reveal the $i$-graph realizability of many famous graph families (including another construction for cycles, which are connected, claw-free, and $\Dia$-free), the realizability problem for graphs containing claws remains unresolved.  Moreover, among clawed graphs are the theta graphs which we first alluded to in Section \ref{sec:i:Real} as containing three of the small known non-$i$-graphs.  In the next section we apply similar techniques with graph complements to construct all theta graphs that are realizable as $i$-graphs.    


%---------------------------------------SECTION: Graph Complements ----------------
\section{Theta Graphs From Graph Complements} \label{sec:c:thetaComp}

Consider now a graph $G$ with $i(G)=3$.  Each $i$-set of $G$ is represented as a triangle (i.e., an induced $K_3$) in $\cp{G}$.  
%We adhere to the convention that all triangles are \emph{maximal} $K_3$ cliques; that is, if the vertices $\{u,v,w,x\}$ form a clique, then $\{u,v,w\}$ is not considered to be a triangle.  
If $X$ and $Y$ are two $i$-sets of $G$ with $\edge{X,u,v,Y}$, then $\cp{G}[X]$ and $\cp{G}[Y]$ are triangles in $\cp{G}$, and have $|X \cap Y| =2$.  Although it is technically the induced subgraphs $\cp{G}[X]$ and $\cp{G}[X]$ that are the triangles of $\cp{G}$,  for notational simplicity going forward, we refer to $X$ and $Y$ as triangles.  In $\cp{G}$, the triangle $X$ can be transformed into the triangle $Y$ by removing the vertex $u$ and adding in the vertex $v$  (where $u\not\sim_{\cp{G}} v$).  Thus, we say that two triangles are \emph{adjacent} if they share exactly one edge (two vertices).  Moreover, since two $i$-sets of a graph $G$ with $i(G)=3$ are adjacent if and only if their associated triangles in $\cp{G}$ are adjacent, we use the same notation for $i$-set adjacency in $G$  as triangle adjacency in $\cp{G}$; that is, the notation $X \sim Y$ represents both $i$-sets $X$ and $Y$ of $G$ being adjacent, and triangles $X$ and $Y$ of $\cp{G}$ being adjacent.

In the following section,  we use triangle adjacency to construct \emph{complement seed graphs} for the $i$-graphs of theta graphs; that is, a graph $\cp{G}$ such that $\ig{G}$ is isomorphic to some desired theta graph. Before proceeding with these constructions, we note some observations which will help us with this process.


\begin{obs}\label{obs:c:cutEdge}
	If a graph $\cp{G}$ has a bridge, then $i(G) \leq 2$.   
\end{obs}


\begin{obs}\label{obs:c:i3}
	A graph  $G$ has $i(G)=2$ if and only if  $\cp{G}$ is nonempty and has an edge that does not lie on a triangle.
\end{obs}

\noindent If $\cp{G}$ has an edge $uv$ that does not lie on a triangle, then every other vertex is  adjacent to at least one of $u$ or $v$ in $G$.  That is, $\{u,v\}$ is independent and dominating in $G$, and so $i(G) \leq 2$.  When building our various seed graphs with $i(G)=3$, it is therefore necessary to ensure that every edge of the complement $\cp{G}$ belongs to a triangle.

%Similarly, if $G$ is a graph with $i(G)=3$, then $\cp{G}$ contains at least one maximal $K_3$ clique (and no cut-edges).  If the vertices $\{u,v,w,x\}$ form a (possibly non-maximal) clique in $\cp{G}$, then we can disregard the triangle formed by any three vertices of the set as it does not represent an $i$-set of $G$.  For example, although $\{u,v,w\}$ forms a triangle in $\cp{G}$, $x$ is not dominated by $\{u,v,w\}$.  

\begin{obs}
	Let $G$ be a graph with $i(G)=3$.  If  $S$ with $|S| \geq 4$ is a (possibly non-maximal) clique in $\cp{G}$, then no three-vertex subset of $S$ is  an $i$-set of $G$.
\end{obs}


\noindent Suppose that $S=\{u,v,w,x\}$ is such a clique of $\cp{G}$.  Then, for example, $x$ is undominated by $\{u,v,w\}$ in $G$, and so $\{u,v,w\}$ is not an $i$-set of $G$.  
Conversely, suppose that $\{u,v,w\}$ is a triangle in a graph $\cp{G}$ with $i(G)=3$.  By attaching a new vertex $x$ to all of $\{u,v,w\}$ in $\cp{G}$, we remove $\{u,v,w\}$ as an $i$-set of $G$, while keeping all other $i$-sets of $G$.  This observation proves to be very useful in the constructions  in the following section: we now have a technique to eliminate any unwanted triangles in $\cp{G}$ (and hence $i$-sets of $G$) that may arise.  Notice that this technique is an application of the Deletion Lemma (Lemma \ref{lem:i:inducedI}) in $\cp{G}$, instead of the usual $G$.

%\begin{obs}
%	Let $G$ be a graph on at least two vertices.   If for each $uv\in E(\cp{G})$, there exists some vertex $w$ such that $uw$ and $vw \in E(\cp{G})$, then $i(G) =3$. 
%\end{obs}


The use of triangle adjacency in a graph $\cp{G}$ to determine $i$-set adjacency in $G$ provides a key technique to finally resolve a question posed in Chapter \ref{ch:igraphs}: which theta graphs are $i$-graphs?  In the following Theorem \ref{thm:c:thetas}, we show that all theta graphs except the seven listed exceptions are $i$-graphs.  The proofs of the lemmas for the affirmative cases make up most of the remainder of Section \ref{sec:c:thetaComp}, and  almost all use this method of complement triangles.  The proofs of  the lemmas for the seven negative cases are given in Section \ref{sec:c:nonthetas}.  

\begin{theorem}\label{thm:c:thetas}
	%All theta graphs except the seven listed in Table \ref{tab:c:thetaLong} are $i$-graphs.
	
	A theta graph is an $i$-graph if and only if it is not one of the seven exceptions listed below: 
	
	\begin{tabular}{lllll}  \centering
		$\thet{1,2,2}$, 	&&&&	\\
		$\thet{2,2,2}$,	\	&		$\thet{2,2,3}$,	\	&		$\thet{2,2,4}$,  \  &	$\thet{2,3,3}$, \ 	&		$\thet{2,3,4}$, \ 	\\
		$\thet{3,3,3}.$ &&&& \\
	\end{tabular}
	
	
\end{theorem}

%\nts{Is the list of 7 above good as is?  Or would it be better with all 7 on a single line?}


\noindent Table \ref{tab:c:thetaLong} summarizes the cases used to establish Theorem \ref{thm:c:thetas} and their associated results.   





\begin{table}[H] \centering
	\begin{tabular}{|l|l|l|}  \hline
		$\thet{j,k,\ell}$	&	Realizability	&	Result	\\	\hline
	%	&		&		\\	
		$\thet{1,2,2}$	&	non-$i$-graph	&	$\Dia$.  Proposition \ref {prop:i:diamond}	\\	
		$\thet{1,2,\ell}$, $\ell \geq 3$ 	&	$i$-graph	&	Lemma \ref{lem:i:cliqueRep}	\\	
		$\thet{1,k,\ell}$, $3 \leq k \leq \ell$ 	&	$i$-graph	&	Lemma \ref{lem:c:thet1kl}	\\	
		&		&		\\	
		$\thet{2,2,2}$	&	non-$i$-graph	&	$K_{2,3}$.  Proposition \ref{prop:i:K23}	\\	
		$\thet{2,2,3}$	&	non-$i$-graph	&	$\kappa$.  Proposition \ref{prop:i:kappa}	\\	
		$\thet{2,2,4}$	&	non-$i$-graph	&	Proposition \ref{prop:c:thet224}	\\	
		$\thet{2,2,\ell}$, $\ell\geq 5$	&	$i$-graph	&	Lemma \ref{lem:c:thet22l}	\\	
		$\thet{2,3,3}$	&	non-$i$-graph	&	Proposition \ref{prop:c:thet233}	\\	
		$\thet{2,3,4}$	&	non-$i$-graph	&	Proposition \ref{prop:c:thet234}	\\	
		$\thet{2,3,\ell}$, $\ell\geq 5 $	&	$i$-graph	&	Lemma \ref{lem:c:thet23k}	\\	
		$\thet{2,4,4}$	&	$i$-graph	&	Lemma \ref {lem:c:thet244}	\\	
		$\thet{2,k,5}$, $4 \leq k \leq 5$	&	$i$-graph	&	Lemma \ref{lem:c:thet2k5}	\\	
		$\thet{2,k,\ell}$, $k \geq 4$, $\ell \geq 6$, $\ell \geq k$	&	$i$-graph	&	 Lemma \ref{lem:c:thet2kl}	\\	
		&		&		\\	
		$\thet{3,3,3}$	&	non-$i$-graph	&	 Proposition \ref{prop:c:thet333}	\\	
		$\thet{3,3,4}$	&	$i$-graph	&	Lemma \ref{lem:c:thet334}	\\	
		$\thet{3,3,5}$	&	$i$-graph	&	Lemma \ref {lem:c:thet335}	\\	
		$\thet{3,3,\ell}$, $\ell\geq 6$	&	$i$-graph	&	Lemma \ref {lem:c:thet33k}	\\	
		$\thet{3,4,4}$	&	$i$-graph	&	Lemma \ref{lem:c:thet344}	\\	
		$\thet{3,4,\ell}$, $\ell\geq 5$	&	$i$-graph	&	Lemma \ref{lem:c:thet34k}	\\	
		$\thet{3,5,5}$	&	$i$-graph	&	Lemma \ref{lem:c:thet355}	\\	
		&		&		\\	
		$\thet{4,4,4}$	&	$i$-graph	&	Lemma \ref{lem:c:thet444}	\\	
		$\thet{j,k,5}$, $4 \leq j \leq k \leq 5 $.	&	$i$-graph	&	Lemma \ref{lem:c:thetjk5}	\\	
		$\thet{j,k,\ell}$, $3 \leq j \leq k \leq \ell$, and $\ell \geq 6$.	&	$i$-graph	&	Lemma \ref{lem:c:thetjkl}	\\	\hline
	\end{tabular} \caption{$i$-graph realizability of theta graphs.} \label{tab:c:thetaLong}
\end{table}



\begin{comment}
	
	\begin{table}[H] \centering
		\begin{tabular}{|l|l|l|}  \hline
			$\thet{j,k,\ell}$	&	Realizability	&	Note	\\ \hline
			&		&		\\
			$\thet{1,2,2}$	&	non-$i$-graph	&	$\Dia$.  Proposition \ref {prop:i:diamond}	\\
			$\thet{1,2,\ell}$, $\ell \geq 3$ 	&	$i$-graph	&	Lemma \ref{lem:i:cliqueRep}	\\
			$\thet{1,k,\ell}$, $3 \leq k \leq \ell$ 	&	$i$-graph	&	Lemma \ref{lem:i:t(1kl)}	\\
			&		&		\\
			$\thet{2,2,2}$	&	non-$i$-graph	&	$K_{2,3}$.  Proposition \ref{prop:i:K23}	\\
			$\thet{2,2,3}$	&	non-$i$-graph	&	$\kappa$.  Proposition \ref{prop:i:kappa}	\\
			$\thet{2,2,4}$	&	non-$i$-graph	&	Proposition \ref{prop:c:thet224}	\\
			$\thet{2,2,\ell}$, $\ell\geq 5$	&	$i$-graph	&	K1: Lemma 3	\\
			$\thet{2,3,3}$	&	non-$i$-graph	&	Proposition \ref{prop:c:thet233}	\\
			$\thet{2,3,4}$	&	non-$i$-graph	&	Proposition \ref{prop:c:thet234}	\\
			$\thet{2,3,\ell}$, $\ell\geq 5$	&	$i$-graph	&	K1: Lemma 4	\\
			$\thet{2,4,4}$	&	$i$-graph	&	K1: Lemma 5	\\
			$\thet{2,k,\ell}$, $k\geq 4$, $\ell \geq 5$	&	$i$-graph	&	K1: Lemma 4	\\
			&		&		\\
			$\thet{3,3,3}$	&	non-$i$-graph	& Proposition \ref{prop:c:thet333}	\\
			$\thet{3,3,4}$	&	$i$-graph	&	K1: Lemma 6	\\
			$\thet{3,3,\ell}$, $\ell\geq 5$	&	$i$-graph	&	K1: Lemma 7	\\
			$\thet{3,4,4}$	&	$i$-graph	&	K1: Lemma 6	\\
			$\thet{3,4,\ell}$, $\ell\geq 5$	&	$i$-graph	&	K1: Lemma 7	\\
			$\thet{3,k,\ell}$, $5\leq k \leq \ell$	&	$i$-graph	&	K1: Lemma 7	\\
			&		&		\\
			$\thet{j,k,\ell}$, $4 \leq j \leq k \leq \ell$.	& $i$-graph	&	No reference!	\\	
			\hline
		\end{tabular} \caption{$i$-graph realizability of theta graphs.} \label{tab:c:thetaLong}
	\end{table}
	
\end{comment}

%......................  SUB SECTION:  theta (1,k,l) .........................
\subsection{$\thet{1,k,\ell}$}  \label{subsec:c:1kl}



We have already seen that the diamond graph $\Dia = \thet{1,2,2}$ is not an $i$-graph  (Proposition \ref{prop:i:diamond}) and that the House graph $\mathcal{H} = \thet{1,2,3}$ is an $i$-graph (Proposition \ref{prop:i:house}).  Indeed, we can further exploit previous results to see that all graphs $\thet{1,2,\ell}$ for $\ell \geq 3$ are $i$-graphs by taking a cycle $C_n$ with $n \geq 4$, and replacing one of its maximal cliques (i.e. an edge) with a $K_3$.  By the Max Clique Replacement Lemma (Lemma \ref{lem:i:cliqueRep}), the resultant $\thet{1,2,n-1}$ is also $i$-graph.
Notice that when we likewise attempt to replace an edge of a $C_3$ with a $K_3$, the result no longer holds; a single edge in a $C_3$ is not  a maximal clique.   For reference, we explicitly state this result as a lemma below.

\begin{lemma} \label{lem:c:thet12k}  
	For $\ell \geq 3$, the theta graphs $\thet{1,2,\ell}$ are $i$-graphs.
\end{lemma}



%In the following sections, we determine the $i$-graph realizability of all theta graphs, dividing our results into subsections by the length of the theta graph's shortest path.  




%______________________  SUBSUB SECTION:  theta (1,k,l) ______________________
\subsubsection{Construction of $\cp{G}$ for $\thet{1,k,\ell}$, $3 \leq k \leq \ell$}  \label{subsubsec:c:1kl+}




In Figure \ref{fig:c:thet145} below, we provide a first example of the technique we employ repeatedly throughout this section to construct our theta graphs.  To the left is a graph $\cp{G}$, where each of its nine triangles corresponds to an $i$-set of its complement $G$.  The resultant $i$-graph of $G$, $\ig{G} = \thet{1,4,5}$, is presented on the right.  For consistency, we use $X$ and $Y$ to denote the triangles corresponding to the degree 3 vertices in the theta graphs in this example, as well as all constructions to follow.



%>>>S: Figure {fig:c:thet145}	
\begin{figure}[H] \centering	
	\begin{tikzpicture}	
		
		%---------- REF-------------------
		\node (cent) at (0,0) {};
		\path (cent) ++(0:40 mm) node (rcent) {};	
		
		%----- W0 - Left Wheel ----------
		\node [std] at (cent) (w0) {};
		\path (w0) ++(40:4 mm) node (w0L) {$w_0$};
		
		\foreach \i/\c in {1/90,3/270,4/210, 5/150} {				
			\path (w0) ++(\c:20 mm) node [std] (w\i) {};
			\path (w\i) ++(\c:5 mm) node (Lw\i) {$w_{\i}$};
			\draw [thick] (w0)--(w\i);
		}
		
		%Node W2
		\path (w0) ++(0:20 mm) node [std] (w2) {};
		\path (w2) ++(90:5 mm) node (Lw2) {$w_{2}$};
		\draw [thick] (w0)--(w2);	
		
		\draw  [thick] (w1)--(w2)--(w3)--(w4)--(w5)--(w1);		
		
		%----- Right Path ----------
		\foreach \i/\c in {1/15,2/0,3/-15} {				
			\path (rcent) ++(90:\c mm) node [bblue] (v\i) {};
			\path (v\i) ++(0:5 mm) node [color=blue] (Lv\i) {$v_{\i}$};
			\draw[bline] (w2)--(v\i);
		}
		\draw[bline] (w1)--(v1)--(v2)--(v3)--(w3);	
		
		%-- X and Y
		\path (w0) ++(45:10 mm) node  [color=red] (LX) {$X$};
		\path (w0) ++(-45:10 mm) node  [color=red] (LY) {$Y$};
		
		%-- D's	
		\foreach \i/\c in {1/120,2/180,3/240} {				
			\path (w0) ++(\c:12 mm) node [color=red] (A\i) {$A_{\i}$};	
		}		
		
		%-- B's	
		\foreach \i/\c in {1/90, 2/18, 3/-18, 4/-90} {				
			\path (w2) ++(\c:12 mm) node [color=red] (B\i) {$B_{\i}$};	
		}	
		\node[broadArrow,fill=white, draw=black, ultra thick, text width=6mm] at (5.8cm,0){};
	\end{tikzpicture}
	\begin{tikzpicture}		
		%---------- REF-------------------	
		\node (cent) at (0,0) {};
		\node  at (cent) (w0) {};
		\path (w0) ++(0:20 mm) node (w2) {};	
		
		%-- X and Y
		\path (w0) ++(45:10 mm) node [bred] (X) {};
		\path (X) ++(90:5 mm) node [color=red] (LX) {$X$};
		
		\path (w0) ++(-45:10 mm) node [bred] (Y) {};
		\path (Y) ++(-90:5 mm) node [color=red] (LY) {$Y$};
		
		\draw [rline] (X)--(Y);	
		
		%-- A's	
		\foreach \i/\c in {1/120,2/180,3/240} {			
			\path (w0) ++(\c:12 mm) node [bred] (A\i) {};		
			\path (A\i) ++(\c:5 mm) node [color=red] (LA\i) {$A_{\i}$};	
		}		
		\draw [rline] (X)--(A1)--(A2)--(A3)--(Y);
		
		%-- B's	
		\foreach \i/\c in {1/90, 2/22, 3/-22, 4/-90} {				
			\path (w2) ++(\c:15 mm) node [bred] (B\i) {};
			\path (B\i) ++(\c:5 mm) node [color=red] (LB\i) {$B_{\i}$};	
		}	
		\draw [rline] (X)--(B1)--(B2)--(B3)--(B4)--(Y);	
		
		%--- Force vertical spacing
		\path (Y) ++(-90: 20mm) node (S) {};		
		
	\end{tikzpicture}
	\caption{A graph $\cp{G}$ (left) such that $\ig{G} = \thet{1,4,5}$ (right).}
	\label{fig:c:thet145}	
\end{figure}						
% <<< E: Figure {fig:c:thet145}

We proceed now to the general construction of a graph $\cp{G}$ with $\ig{G}=\thet{1,k,\ell}$ for $3 \leq k \leq \ell$.   As it is our first construction using this triangle technique, we provide the construction and proof for Lemma \ref{lem:c:thet1kl} with an abundance of detail.   




\begin{cons} \label{cons:c:thet1kl} (See Figure \ref{fig:c:thet1kl})  
	Let $\overline{H}\cong W_{k+2}=C_{k+1}\vee K_{1}$, where
	$C_{k+1}=(w_{1},...,w_{k+1},w_{1})$ and $w_{0}$ 
	is the central hub, that is, the vertex with degree $k+1$.  Add a path $P_{\ell-2}%
	:(v_{1},...,v_{\ell-2})$, joining each $v_{i},\ i=1,...,\ell-2$, to $w_{2}$.
	Also join $v_{1}$ to $w_{1}$ and $v_{\ell-2}$ to $w_{3}$. (If $\ell=3$, then
	$v_{1}=v_{\ell-2}$, hence $v_{1}$ is adjacent to $w_{1},w_{2}$ and $w_{3}$.)
	This is the (planar) graph $\overline{G}$.
\end{cons}




%>>>S: Lemma {lem:c:thet1kl}	
\begin{lemma} \label {lem:c:thet1kl}	
	If $\cp{G}$ is the graph constructed by Construction \ref{cons:c:thet1kl}, then $\ig{G} = \ag{G} = \thet{1,k,\ell}$, $ 3 \leq k \leq \ell$.
\end{lemma}





%>>>S: Figure {fig:c:thet1kl}	
\begin{figure}[H] \centering	
	\begin{tikzpicture}	
		
		%---------- REF-------------------
		\node (cent) at (0,0) {};
		\path (cent) ++(0:45 mm) node (rcent) {};	
		
		%----- W0 - Left Wheel ----------
		\node [std] at (cent) (w0) {};
		\path (w0) ++(15:6 mm) node (w0L) {$w_0$};
		
		%	\foreach \i/\c /\l in {1/90/1,3/270/3,4/210/4, 5/150/{k}} {		
			\foreach \i/\c /\l in {1/45/1,3/-45/3,4/-90/4, 5/-135/{5}, 6/135/{k}, 7/90/{k+1}} {						
				\path (w0) ++(\c:20mm) node [std] (w\i) {};
				\path (w\i) ++(\c:5 mm) node (Lw\i) {$w_{\l}$};
				\draw[thick] (w0)--(w\i);
			}
			
			%Node W2
			\path (w0) ++(0:20 mm) node [std] (w2) {};
			\path (w2) ++(80:5 mm) node (Lw2) {$w_{2}$};
			\draw[thick]  (w0)--(w2);	
			
			\draw[thick]  (w1)--(w2)--(w3)--(w4)--(w5);
			\draw[thick]  (w6)--(w7)--(w1);		
			
			\path (w0) ++(165:20mm) coordinate (z1) {};
			\path (w0) ++(195:20mm)coordinate (z2) {};
			
			\draw[thick] (w5)--(z2)--(z1)--(w6);
			
			\draw [dashed,thick] (z2)--(w0);
			\draw [dashed,thick] (z1)--(w0);
			
			%---- V-Path	
			\foreach \i / \j  / \k in {1/2.2/1, 2/1/2, 3/-1/{\ell-3}, 4/-2.2/{\ell-2}}
			{	
				\path (rcent) ++(90:\j cm) node [bblue] (v\i) {};
				\path (v\i) ++(0:5 mm) node (Lv\i) {\color{blue} $v_{\k}$};
				
				\draw[bline] (w2)--(v\i);
			}
			
			\draw[bline] (v1)--(v2);
			\draw[bline] (v3)--(v4);
			
			\draw[bline] (v2)--(v3);
			
			\draw[bline] (v1)--(w1);
			\draw[bline] (v4)--(w3);	
			
			\coordinate  (q1) at ($(v2)!0.33!(v3)$) {};
			\coordinate  (q2) at ($(v2)!0.66!(v3)$) {};
			
			\draw [dashed,bline] (q2)--(w2);
			\draw [dashed,bline] (q1)--(w2);		
			
			%-- X and Y
			\path (w0) ++(22:14 mm) node  [color=red] (LX) {$X$};
			\path (w0) ++(-22:14 mm) node  [color=red] (LY) {$Y$};		
			
			%-- A's	
			\foreach \i/\c / \l in {1/67/1,2/112/2,3/-112/{k-2}, 4/-67/{k-1}} 			
			\path (w0) ++(\c:14 mm) node [color=red] (A\i) {$A_{\l}$};				
			
			%-- B's	
			\foreach \i/\c /\l in {1/80/1,  4/-80/{\ell-1}} 				
			\path (w2) ++(\c:10 mm) node [color=red] (B\i) {$B_{\l}$};	
			\foreach \i/\c /\l in { 2/32/2, 3/-30/{\ell-2}} 			
			\path (w2) ++(\c:20 mm) node [color=red] (B\i) {$B_{\l}$};	
			
		\end{tikzpicture} \hspace{5mm}	
		\caption{The graph $\cp{G}$ from Construction \ref{cons:c:thet1kl} such that $\ig{G} = \thet{1,k,\ell}$ for $3 \leq k \leq \ell$.}
		\label{fig:c:thet1kl}	
	\end{figure}						
	% <<< E: Figure {fig:c:thet1kl}
	
	
	\begin{proof}
		To begin, notice that since in $\cp{G}$, the vertices $\mathcal{W}=\{w_0,w_1,\dots,w_{k},w_{k+1}\}$ form a wheel on at least five vertices, $\cp{H} \ncong K_4$.  Likewise, the graph induced by $\{w_0, w_1, w_2, w_3, v_1, v_2,$ $\dots, v_{\ell-2}\}$ in $\cp{G}$ is also a wheel on $\ell +2$ vertices, where $w_2$ is the central hub, and so it too contains no $K_4$.  Therefore, $\cp{G}$ is $K_4$-free. Moreover, since the triangles of $\cp{G}$ are its smallest maximal cliques, these triangles are precisely the maximal cliques of $\cp{G}$, and so $\omega(\cp{G}) = i(G) = \alpha(G) = 3$.  Since the $i$-sets of $G$ are identical to its $\alpha$-sets, $\ig{G}=\ag{G}$, and so for ease of notation, we will refer only to  $\ig{G}$ throughout the remainder of this proof.
		
		We label the triangles as in Figure \ref{fig:c:thet1kl} by dividing them into two collections.  The first are the triangles composed only of the vertices from $\mathcal{W}$ and each containing $w_0$: let $X = \{w_0,w_1,w_2\}$, $Y = \{w_0, w_2,w_3\}$, $A_1 =\{w_0,w_{k+1},w_1\} $, $A_2=\{w_0, w_{k}, w_{k+1}\}$, $\dots$ $A_{k-2} = \{w_0, w_{4}, w_{5}\}$,  $A_{k-1} = \{w_0, w_3, w_4\}$.
		%$A_3 = \{w_0, w_3,w_4\}$, $\dots$, and $A_{k+1}  = \{w_0, w_{k+1}, w_1\}$.  
		The remainder are the triangles with vertex sets not fully contained in $\cp{H}$: $B_1 = \{w_2,w_1,v_1\}$, $B_2 = \{w_2, v_1, v_2\}$, $B_3 = \{w_2, v_2, v_3\}$, $\dots$, $B_{\ell-2} = \{w_2, v_{\ell-3}, v_{\ell-2}\}$,  and $B_{\ell-1} = \{w_2, v_{\ell-2},w_3\}$.  We refer to these collections as $\mathcal{S} = \{X, Y\}$,  $\mathcal{A}=\{A_1,A_2,\dots,A_{k-1}\}$, and $\mathcal{B} = \{B_1, B_2, \dots, B_{\ell-1}\}$, and claim that these are the only triangles of $\cp{G}$.  
		
		To demonstrate this, we show that each vertex $v_i$ on the path $P_{\ell-2}$ is incident with exactly two of the triangles defined above.  
		
		Consider first a vertex $v_i$ for $2 \leq i \leq \ell-3$ on $P_{\ell-1}$ (in $\cp{G}$).  From the above definitions, $v_i$ is already on the triangles $B_i$ and $B_{i+1}$.  Since $\cp{G}$ is a planar graph, and $\deg(v_i)=3$, there is at most one more triangle incident with $v_i$.   The triangle $B_{i}$ is incident with edges $v_{i-1}v_i$ and $v_iw_2$, and $B_{i+1}$  is incident with edges $v_iv_{i+1}$ and $v_iw_2$.  Thus, a third triangle incident with $v_i$ would make use of edges $v_iv_{i-1}$ and $v_iv_{i+1}$.  However, since $v_{i-1}v_{i+1} \notin  E(\cp{G})$, no such third triangle exists; $v_i$ is incident only with $B_{i-1}$ and  $B_i$.  
		In the special case where $i=1$, the vertex $v_1$ is incident with triangles $B_1$ and $B_2$.  Since $w_1v_2 \notin E(\cp{G})$, these are the only two such triangles.  Similarly, for $i=\ell-3$, $v_{\ell-3}$ is incident only to $B_{\ell-2}$ and $B_{\ell-1}$, since $w_3v_{\ell-3} \notin E(\cp{G})$.  Thus, we conclude that for $1 \leq i \leq \ell-2$, the vertex $v_i$ is incident to the triangles $B_i$ and  $B_{i+1}$, and no others.
		
		Now, since all triangles incident with $v_i$ have already been considered, any unaccounted for triangles would be composed only of vertices from $\mathcal{W}$.  However, clearly the vertices of $\mathcal{W}$ form a wheel, for which we have  already accounted for all its $2+(k-1)=k+1$ triangles.
		We conclude that the only triangles of $\cp{G}$ are the $2+(k-1)+(\ell-1)=k+\ell$ such ones defined above.  It follows that these $k+\ell$ triangles represent all the $i$-sets of $G$, and so  $V(\ig{G})=\{X,Y, A_1,A_2,\dots,A_{k-1}, B_1,B_2,\dots,B_{\ell-1}\}$.
		
		Having determined all triangles of $\cp{G}$ (and thus $V(\ig{G})$), we now show that the required adjacencies hold.  		
		From the original construction of $\cp{G}$, the following are immediate for $\ig{G}$:
		
		\newpage
		
		\begin{enumerate}[itemsep=0pt, label=(\roman*)]
			\item \label{it:c:thet1kl:1}  $\edge{X, w_1, w_3,Y}$, 
			
			\item \label{it:c:thet1kl:j} $\edge{X, w_0,  v_1,B_1}  \adedge{w_1,v_2,B_2} \adedge{v_1,v_3,B_3} \dots \edge{B_{\ell-2}, v_{\ell-3},  w_3,B_{\ell-1}} \adedge{v_{\ell-2},w_0,Y}$, 
			
			\item \label{it:c:thet1kl:k}  $\edge{X, w_2,  w_{k+1}, A_1}  \adedge{w_1,w_k,A_2} \dots \edge{A_{k-2}, w_{5},  w_3, A_{k-1}} \adedge{w_{4},w_2,Y}$. 
		\end{enumerate}
		
		\noindent Hence, we need only show that there are no additional unwanted edges generated in the construction of $\ig{G}$.  	
		
		Since $\cp{G}$ is a planar graph and all of its triangles are facial (that is, the edges of the $K_3$ form a face in the planar embedding of the graph shown in Figure \ref{fig:c:thet1kl}), each triangle is adjacent to at most three others.  From \ref{it:c:thet1kl:1}-\ref{it:c:thet1kl:k}  above, triangles $X$ and $Y$ are both adjacent to the maximum three (and hence $\deg_{\ig{G}}(X)=\deg_{\ig{G}}(Y)=3$).  
		
		Recall that to be adjacent, two triangles share exactly two vertices.  Notice that the triangles of $\mathcal{A}$ are composed entirely of vertices from $\mathcal{W}-\{w_2\}$, and that for $2\leq i \leq \ell-2$, $B_i\cap\mathcal{W}=\{w_2\}$; furthermore, $B_1\cap\mathcal{W}=\{w_1,w_2\}$ and $B_{\ell-2}\cap\mathcal{W}=\{w_2,w_3\}$.  We conclude that no triangle of $\mathcal{B}$ is adjacent to any triangle of $\mathcal{A}$.
		
		It remains only to show that there are no additional unwanted adjacencies between two triangles of $\mathcal{A}$ or two triangles of $\mathcal{B}$.
		Clearly from the definitions, there is no triangle of $\mathcal{A}$ aside from $A_1$ that contains both the vertices $w_1$ and $w_{k+1}$.  For $2 \leq i \leq k-2$, consider a triangle $A_i$.  From \ref{it:c:thet1kl:k}, we have already shown that $A_i$ is adjacent to two other triangles: one sharing the vertex pair $\{w_0,w_{(k+2)-i}\}$ and the other sharing the vertices $\{w_0,w_{(k+3)-i}\}$.  From their definitions, we see that there is no triangle $A_j$ with $i\neq j$ sharing the third vertex pair with $A_i$.  That is, there is no $A_j$ such that $\{w_{(k+2)-i},w_{(k+3)-i} \} \subseteq A_j$.  Hence, for all $A_i\in \mathcal{A}$, $A_i$ is adjacent to exactly two other triangles, and so $\deg_{\ig{G}}(A_i)=2$.
		
		Applying arguments very similar to those from $\mathcal{A}$ to the triangles of $\mathcal{B}$ shows that no  additional unwanted edges are generated in the construction of the $i$-graph.  		
		This completes the exhaustive examination of the triangles of $\cp{G}$.  We conclude that the graph $\cp{G}$ generated by Construction \ref{cons:c:thet1kl} yields $\ig{G} =\ag{G}= \thet{1,k,\ell}$.
	\end{proof}
	% <<< E: Lemma {lem:c:thet1kl}
	
	
	
	%--------------------------------------------
	%\subsubsection{The $i$-graphs of the Complements of Wheels Are Cycles}  \label{subsubsec:c:wheel}
	\subsection{The $i$-graphs of the Complements of Wheels Are Cycles}  \label{subsec:c:wheel}
	
	
	Before we proceed with the remainder of the theta graph constructions, let us return to Figure \ref{fig:c:thet1kl} to notice the prominence of the wheel subgraph in the complement seed graph $\cp{G}$.  In the constructions throughout this chapter, this wheel subgraph will appear repeatedly; indeed, all of the complement seed graphs for the $i$-graphs of theta graphs have a similar basic form: begin with a wheel, add a path of some length, and then add some collection of edges between them.   Lemma \ref{lem:c:wheel} below demonstrates that a wheel in a triangle-based complement seed graph $\cp{G}$ corresponds to a cycle in the $i$-graph of $G$.  Using this result, in our later constructions with a wheel subgraph, we already have two of the three paths of a theta graph formed.  Hence, we need only confirm that whatever unique additions are present in a given construction form the third path in the $i$-graph.
	
	
	Notice that when $k=3$, the wheel $W_4$ is isomorphic to the complete graph $K_4$, the complement of which has only a single $i$-set, namely, the set including all vertices.  Hence, $\ig{\cp{W_4}} = \ig{\cp{K_4}} =  K_1$.
	
	
	
	%>>>S: Lemma {lem:c:wheel}	
	\begin{lemma}\label{lem:c:wheel}
		For $k\geq 4$, let $H_{k}$ be the wheel $W_{k+1} =C_{k} \vee K_1$.  Then $\ig{\cp{H_{k}}} \cong \ag{\cp{H_{k}}} \cong C_k$.
	\end{lemma}
	
	

	
	%>>>S: Figure {fig:c:wheel}
	\begin{figure}[H] \centering	
		\begin{tikzpicture}	
			%---------- REF-------------------
			\node (cent) at (0,0) {};
			
			%----Left Wheel
			\node[std,label={0:$w_0$}] (w0) at (cent) {};
			
			\foreach \i/\j in {1/1,2/2,3/3,4/4,5/5,6/k}
			{
				\node[std,label={150-\i*60: $w_{\j}$}] (w\i) at (150-\i*60:2.5cm) {}; 
				\draw[thick] (w0)--(w\i);        
			}
			
			\draw[thick] (w1)--(w2)--(w3)--(w4)--(w5);
			\draw[thick] (w6)--(w1);
			\draw[thick] (w5)--(w6);
			
			\coordinate  (v1) at ($(w5)!0.33!(w6)$) {};
			\coordinate  (v2) at ($(w5)!0.66!(w6)$) {};
			
			\draw [dashed] (v2)--(w0);
			\draw [dashed] (v1)--(w0);	
			
			%\path (w0) ++(170:25mm) coordinate (t1) {};
			%\path (w0) ++(190:25mm)coordinate (t2) {};
			%\draw[thick] (w5)--(t2)--(t1)--(w6);
			
			%\draw [dashed,thick] (t2)--(w0);
			%\draw [dashed,thick] (t1)--(w0);
			
			%----- Internal Red Nodes	
			\foreach \i/\j in {1/1,2/2,3/3,4/4,6/k}
			\node[dred,label={120-\i*60: \color{red}$A_{\j}$}] (a\i) at (120-\i*60:12mm) {}; 
			
			\draw[thick,red, densely dashdotted](a6)--(a1)--(a2)--(a3)--(a4);
			
			\foreach \i/\j in {5/195,6/180,7/165}
			\coordinate (b\i) at (\j:12mm) {}; 		
			
			\draw[dashed,red] (a4)--(b5)--(b6)--(b7)--(a6);
			
		\end{tikzpicture} 
		\caption{The wheel  $H_k=W_{k+1}$, with the $i$-graph of its complement, $\ig{\cp{H_k}} \cong \ag{\cp{H_k}} \cong  C_k$, embedded in red.}
		\label{fig:c:wheel}
	\end{figure}		
	% <<< E: Figure {fig:c:wheel}
	
	\begin{proof}
		Let $H_k$ be the wheel graph $W_{k+1}$ for $k \geq 4$.  We apply the standard vertex labelling used throughout this section, where the degree 3 vertices are labelled $w_1,w_2,\dots, w_k$, and the central degree $k$ vertex is  $w_0$. 
		
		Since $H_k$ is $K_4$-free and each vertex of $H_k$ lies on a triangle, $i(\cp{H_k}) = 3$; moreover, each triangle of $H_k$ represents an $i$-set of $\cp{H_k}$.   %Likewise, since there are no induced $K_4$'s in $H_k$, 
		For the same reasons, we have that $\alpha(\cp{H_k})=3$ and the triangles represent also the $\alpha$-sets of $\cp{H_k}$.
		
		We label these triangles as  $A_i = \{w_0, w_i, w_{i+1}\}$ for $1 \leq i \leq k-1$ and $A_k=\{w_0,w_k,w_1\}$.  
		
		To check that there are no other unaccounted for triangles in $H_k$, consider one of the exterior wheel vertices of $H_k$, say $w_i$ with, for ease of notation, $2 \leq i \leq k-1$ (similar arguments with worse notation hold for $i=1$ and $i=k$).  Since $deg(w_i) = 3$, $w_i$ is on at most $\binom{3}{2}=3$  triangles, of which we have already accounted for two: $A_{i-1}=\{w_0,w_{i-1},w_i\}$ (using edges  $w_0w_i$ and $w_0w_{i-1}$) and $A_i =\{w_0, w_i, w_{i+1}\}$  (using edges  $w_0w_i$ and $w_0w_{i+1}$).  The final potential triangle would make use  of the edges $w_iw_{i+1}$ and $w_{i-1}w_i$; however, since $w_{i-1}w_{i+1} \notin E(H_k)$, this does not form a triangle.  Thus, $w_i$ (and hence all of the exterior wheel vertices) is (are) on exactly two triangles. We have accounted for all triangles of $H_k$ and therefore all $i$-sets of $\cp{H_k}$, and so $V(\ig{\cp{H_k}}) = \{A_1,A_2,\dots,A_k\}$.
		
		Now, since $H_k$ is planar, and each triangle $A_1,A_2,\dots,A_k$ is a facial triangle, each triangle is  adjacent to at most three other triangles.  Clearly, for $2 \leq i \leq k-1$, $A_{i-1} \sim A_i \sim A_{i+1}$, which makes use of the shared edges $w_0w_i$ and $w_0w_{i+1}$ of $A_i$.  As observed before, the third edge, $w_iw_{i+1}$, is only on $A_i$, and so $A_i$ is adjacent only to $A_{i-1}$ and $A_{i+1}$.   Similar arguments follow for $A_1$ and $A_k$.  Thus, $\deg_{\ig{\cp{H_k}}}(A_i)=2$ for all $1 \leq i \leq k$ and since $A_1 \sim  A_2  \sim \dots \sim A_k \sim A_1$, it follows that $\ig{\cp{H_{k}}} \cong \ag{\cp{H_{k}}} \cong C_k$.	
	\end{proof}
	% <<< E: Lemma {lem:c:wheel}
	
	We note the following analogous  - although less frequently applied - result for fans of the form $K_1+P_k$.
	The proof proceeds similarly to the previous result for wheels, and is omitted.
	
	
	%>>>S: Lemma {lem:c:wheel}	
	\begin{lemma}\label{lem:c:fan}
		For $k\geq 2$, let $H_{k}$ be the $k$-fan $K_1 \vee P_{k}$.  Then $\ig{\cp{H_{k}}} \cong \ag{\cp{H_k}} \cong P_{k-1}$.
	\end{lemma}
	
	
	
	%>>>S: Figure {fig:c:fan}
	
	\begin{figure}[H] \centering	
		\begin{tikzpicture}	
			%---------- REF-------------------
			\node (cent) at (0,0) {};
			\path (cent) ++(0:40 mm) node (rcent) {};	
			
			%----Right Path
			\node[std,label={180:$v_0$}] (w0) at (cent) {};
			
			\foreach \i/\d/\j in {1/1/1,2/2/2,3/3/3,4/4/4,5/6/{k-1},6/7/{k}}
			{
				\path (rcent) ++(90:40mm-\d*10mm) node[std,label={0: $v_{\j}$}] (w\i) {};
				\draw[thick] (w0)--(w\i);        
			}
			
			\draw[thick] (w1)--(w2)--(w3)--(w4);
			\draw[thick] (w5)--(w6);
			\draw[dashed] (w4)--(w5);
			
			\coordinate  (v1) at ($(w4)!0.33!(w5)$) {};
			\coordinate  (v2) at ($(w4)!0.66!(w5)$) {};
			
			\draw [dashed] (v2)--(w0);
			\draw [dashed] (v1)--(w0);	
			
			%----- Internal Red Nodes		
			\foreach \i/\d /\r /\j in {1/1/25/1,2/2/10/2,3/3/0/3,5/6/-25/{k-1}}
			{
				\path (rcent) ++(90:35mm-\d*10mm) coordinate  (z\i) {};	
				\coordinate  (y\i) at ($(z\i)!0.25!(w0)$) {};
				\node[dred] (a\i) at (y\i) {};	
				\path (a\i) ++(\r:5mm) node  (AL\i) { \color{red}$A_{\j}$};	
			}
			
			%----- Lines	
			\path (a3) ++(270:5mm) coordinate (d3) {};	
			\path (a5) ++(90:5mm) coordinate (d5) {};	
			
			\draw[thick,red](a1)--(a2)--(a3);
			\draw[thick,red](a3)--(d3);
			\draw[thick,red](a5)--(d5);
			
			\draw[dashed,thick, red, densely dashdotted] (d3)--(d5);
		\end{tikzpicture} 	
		\caption{The  fan  $H_k=K_1 \vee P_k$, with the $i$-graph of its complement, $\ig{\cp{H_k}} \cong \ag{\cp{H_k}} \cong P_{k-1}$, embedded in red.}
		\label{fig:c:fan}	
	\end{figure}
	% <<< E: Figure {fig:c:fan}
	
	
	
	
	
	
	
	%......................  SUB SECTION:  theta (2,k,l) .........................
	\subsection{$\thet{2,k,\ell}$  for $2  \leq k  \leq \ell$ }  \label{subsec:c:2kl}
	
	
	%______________________  SUBSUB SECTION:  theta (2,2,\ell) ______________________
	\subsubsection{Construction of $\cp{G}$ for $\thet{2,2,\ell}$, $\ell \geq 5$}  \label{subsubsec:c:22l}
	
	
	In Chapter \ref{ch:igraphs}, we saw that $K_{2,3} \cong \thet{2,2,2}$  (Proposition \ref{prop:i:K23})  and $\kappa \cong \thet{2,2,3}$ (Proposition \ref{prop:i:kappa}) are not $i$-graphs.   Extending these results, we find that the length of the third path in $\thet{2,2,\ell}$ has a transition point between $\ell=4$ and $\ell=5$; while $\ell=4$ is still too short to form an $i$-graph (see Lemma \ref{prop:c:thet224}), for  $\ell \geq 5$, $\thet{2,2,\ell}$ is $i$-graph realizable.
	
	
	%In the following constructions, we build our graphs around the graph $\cp{H} \cong W_5 = C_4 \vee K_1$, labelling the degree 3 vertices as $w_1,w_2,w_3,w_4$ and the central degree 4 vertex as $w_0$.
	
	
	%>>> S:CONS {con:c:thet22k}
	\begin{cons} \label{con:c:thet22k}
		Refer to Figures \ref{fig:c:thet22k} and \ref{fig:c:thet225}.	Begin with a copy of the graph $\cp{H} \cong W_5 = C_4 \vee K_1$, labelling the degree 3 vertices as $w_1,w_2,w_3,w_4$ and the central degree 4 vertex as $w_0$.
		
		
		\begin{enumerate}[itemsep=0pt, label=(\alph*)]
			\item \label{con:c:thet22k:ell}
			If $\ell \geq 6$ (as in Figure \ref{fig:c:thet22k}), attach to $\cp{H}$ a path  $P_{\ell -3}: (v_1, v_2,\dots,v_{\ell-3})$ by joining $w_1$ to $v_1, v_2, \dots, v_{\ell-4}$.  Join $v_{\ell-5}$ to $v_{\ell-3}$.  Next, join $w_2$ to $v_1$, and $w_3$ to $v_{\ell-3}$.  Then, join $w_4$ to $v_{\ell-4}$ and $v_{\ell-3}$.  Add a new vertex $z$, joined to $w_1, w_4,$ and $v_{\ell-4}$.
			
			
			\item \label{con:c:thet22k:5} If $\ell = 5$ (as in Figure \ref{fig:c:thet225}), attach to $\cp{H}$ a path of $P_{2}: (v_1, v_2)$, by joining $w_1$ to $v_1$, and $w_2$ to $v_1$ and $v_2$.  Then join $w_3$ to $v_2$, and $w_4$ to both $v_1$ and $v_2$.  Add two new vertices, $z_1$ and $z_2$, joining $z_1$ to $v_1, w_1$, and $w_4$, and $z_2$ to $v_2, w_2$ and $w_3$.
			
		\end{enumerate}
		
		We label the resultant (planar) graph $\cp{G}$.	
	\end{cons}
	% <<< E: Cons {con:c:thet22k}
	
	
	
	%>>> S: Figure {fig:c:thet22k}		
	\begin{figure}[H] \centering	
		\begin{tikzpicture}	
			%---------- REF-------------------
			\node (cent) at (0,0) {};
			\path (cent) ++(0:45 mm) node (rcent) {};	
			
			%----Left Wheel
			\node[std,label={10:$w_0$}] (w0) at (cent) {};
			
			\foreach \i in {1,2,3,4}
			{
				\node[std,label={180-\i*90: $w_{\i}$}] (w\i) at (180-\i*90:2cm) {}; 
				\draw[thick] (w0)--(w\i);        
			}
			\draw[thick] (w1)--(w2)--(w3)--(w4)--(w1);
			
			%---- V-Path	
			\foreach \i / \j  / \k in {1/3/1, 2/2/2, 3/-1/{\ell-5}, 4/-2.2/{\ell-4}, 5/-3.4/{\ell-3}}
			{	
				\path (rcent) ++(90:\j cm) node [bblue] (v\i) {};
			}
			
			\path (v1) ++(0:4 mm) node (Lv1) {\color{blue} $v_{1}$};
			\path (v2) ++(0:4 mm) node (Lv2) {\color{blue} $v_{2}$};
			\path (v3) ++(-10:5 mm) node (Lv3) {\color{blue} $v_{\ell-5}$};
			\path (v4) ++(15:8 mm) node (Lv4) {\color{blue} $v_{\ell-4}$};
			\path (v5) ++(260:4 mm) node (Lv5) {\color{blue} $v_{\ell-3}$};		
			
			\draw[bline] (v1)--(v2);
			\draw[bline] (v3)--(v4)--(v5);
			
			\draw[bline,dashed] (v2)--(v3);
			
			%---- Zs
			\path (v4) ++(0:15 mm) node [bzed, label={[zelim2]0:$\boldsymbol{z}$}] (z) {};
			
			\path (v4) ++(270:3cm) coordinate  (c1) {};
			\draw[zline] (z) to[out=-40, in=20]  (c1) to [out=200,in=250] (w4); 
			
			\path (v1) ++(90:2cm) coordinate  (c2) {};
			%\draw[rline] (z) to[out=0, in=20]  (c2) to [out=200,in=45] (w1); 
			\draw[zline] (w1) to[out=50, in=170]  (35:6.9cm) to [out=350,in=40]  (z); 
			
			\draw[zline] (v4)--(z);		
			
			%---- V3 to V5 
			\path (v4) ++(180:1.3 cm) coordinate  (c1) {};
			\draw[bline] (v3) to[out=190, in=90]  (c1) to [out=270,in=170] (v5);  
			
			%---- V1 to W1 W2
			\draw[bline] (w1)--(v1)--(w2);	
			
			%-- V5 to W4
			\draw[bline] (v5) to[out=180,in=-90] (w4);  
			
			%-- V5 to W3
			\draw[bline] (v5) to[out=170,in=-30] (w3);  
			
			%-- V4 to W4
			\draw[bline] (v4) to[out=-35, in=25]  (4.6cm,-4.2cm) to [out=205,in=260] (w4);  	
			%	\node[std] at (4.6cm,-4.2cm) {};
			
			%-- W1 to V2  
			\draw[bline] (w1) to[out=20, in=170]  (35:6.1cm) to [out=350,in=45]  (v2); 
			%	\node[std] at (35:6.2cm)  {};
			
			%-- W1 to  V3 
			\draw[bline] (w1) to[out=38, in=170]  (35:6.4cm) to [out=350,in=45]  (v3); 
			%\node[std] at (30:5.8cm) {};
			
			%-- W1 to V4
			\draw[bline] (w1) to[out=46, in=170]  (35:6.6cm) to [out=350,in=40]  (v4); 
			%\node[std] at (30:5.8cm) {};
			
			%-- W1 to Dashed T1 T2
			\path (rcent) ++(90:1 cm) coordinate  (t1) {};
			\draw[bline, dashed] (w1) to[out=28, in=170]   (35:6.2cm)  to [out=350,in=45]  (t1); 
			
			\path (rcent) ++(90:0 cm) coordinate  (t2) {};
			\draw[bline, dashed] (w1) to[out=34, in=170]  (35:6.3cm) to [out=350,in=45]  (t2); 
			
			%----- Internal Red Labellings	
			\path (w0) ++(45:9 mm) node [color=red] (X) {$X$};			
			\path (w0) ++(45:-9 mm) node [color=red] (Y) {$Y$};
			
			%-- A's
			\path (w0) ++(-45:8 mm) node [color=red] (A1) {$A$};	
			
			%-- B's	
			\path (w0) ++(135:8 mm) node [color=red] (B1) {$B$};			
			
			%-- D's
			\path (w2) ++(90:15 mm) node [color=red] (D1) {$D_1$};	
			\path (v1) ++(170:9 mm) node [color=red] (D2) {$D_2$};	
			\path (v4) ++(160:7 mm) node [color=red] (D3) {$D_{\ell-3}$};
			\path (v5) ++(195:20mm) node [color=red] (D4) {$D_{\ell-2}$};
			\path (w4) ++(-55:20 mm) node [color=red] (D5) {$D_{\ell-1}$};			
			
		\end{tikzpicture} 
		\caption{The graph $\cp{G}$ from Construction \ref{con:c:thet22k} such that $\ig{G} = \thet{2,2,\ell}$ for $\ell \geq 6$.}
		\label{fig:c:thet22k}
	\end{figure}	
	% <<< E: Figure {fig:c:thet22k}
	
	
	
	
	%>>> S: Figure {fig:c:thet225}	
	\begin{figure}[H] \centering	
		\begin{tikzpicture}	
			
			%---------- REF-------------------
			\node (cent) at (0,0) {};
			\path (cent) ++(0:45 mm) node (rcent) {};	
			
			%----Left Wheel
			\node[std] (w0) at (cent) {};
			\path (w0) ++(30: 4 mm) node (Lw0) {${w_0}$};	
			
			\foreach \i in {1,2,3,4}
			{
				\node[std,label={180-\i*90: $w_{\i}$}] (w\i) at (180-\i*90:2.5cm) {}; 
				\draw[thick] (w0)--(w\i);        
			}
			\draw[thick] (w1)--(w2)--(w3)--(w4)--(w1);
			
			%---- V-Path	
			\foreach \i / \j in {1/1.5,2/-1.5}
			{	
				\path (rcent) ++(90:\j cm) node [std] (v\i) {};
				\path (v\i) ++(0:4 mm) node (Lv\i) {$v_{\i}$};
				%\node[bblue,label={0: $v_\i$}] (v\i) at (4,2-\i) {};
			}
			\draw[thick] (w1)--(v1)--(v2)--(w3);
			\draw[thick] (v2)--(w2)--(v1);	
			
			%---- Zs
			\path (w2) ++(270:1 cm) node [bzed, label={[zelim]15:$ \boldsymbol{z_2}$}] (z2) {};
			\draw[zline] (w2)--(z2)--(w3);
			\draw[zline] (z2)--(v2);
			
			\path (w1) ++(125:1.2 cm) node [bzed, label={[zelim]14:$\boldsymbol{z_1}$}] (z1) {};
			\draw[zline] (w4)--(z1)--(w1);
			\draw[zline] (z1)--(v1);
			
			%--- V2 to W4 
			\draw[thick] (v2) to[out=230, in=-5]  (270:3.3cm) to [out=170,in=280] (w4);
			
			%--- V1 to W4
			\draw[thick] (v1) to[out=130, in=0]  (90:4.3cm) to [out=180,in=90] (w4);	
			
			%------------ THETA
			
			%---- INNER Cycle	
			\foreach \i / \j  in {1/X,2/B,3/Y,4/A}
			{	
				\node[dred] (t\i) at (-45+\i*90:1.0cm) {}; 
				\path (t\i) ++(-45+\i*90: 4 mm) node (Lt\i) {\color{red}${\j}$};				
			}
			
			\path (w2) ++(90: 7mm) node [dred] (d1) {};
			\path (d1) ++(90: 4 mm) node (Ld1) {\color{red}${D_1}$};
			
			\path (w2) ++(0: 15mm) node [dred] (d2) {};
			\path (d2) ++(90: 4 mm) node (Ld2) {\color{red}${D_2}$};	
			
			\path (v2) ++(-45: 10mm) node [dred] (d3) {};
			\path (d3) ++(0: 4 mm) node (Ld3) {\color{red}${D_3}$};	
			
			\path (t3) ++(260: 15mm) node [dred] (d4) {};
			\path (d4) ++(170: 4 mm) node (Ld4) {\color{red}${D_4}$};					
			
			\draw[rlinemb](t1)--(t2)--(t3)--(t4)--(t1);
			\draw[rlinemb](t1)--(d1)--(d2);	
			\draw[rlinemb](t3)--(d4);
			
			%--- D2 to D3 
			\draw[rlinemb] (d2) to [out=-45,in=90] (d3);
			
			%--- D3 to D4
			\draw[rlinemb] (d3) to [out=230,in=300] (d4);
			
		\end{tikzpicture} 
		
		\caption{The graph $\cp{G}$ from Construction \ref{con:c:thet22k} such that $\ig{G} = \thet{2,2,5}$, with $\thet{2,2,5}$ overlaid in red.}
		\label{fig:c:thet225}
	\end{figure}	
	% <<< E: Figure {fig:c:thet225}
	
	
	
	
	
	
	
	% <<< E: Figure {fig:c:thet22l}
	
	
	As with our other constructions, the triangles of $\cp{G}$ are its smallest maximal cliques, and so $i(G)=3$. As before, we use these triangle faces to visualize the movements of tokens between the $i$-set reconfigurations; however, we now employ a technique of adding vertices to create $K_4$'s through $\cp{G}$ and eliminate any ``unwanted" triangles that might arise in our construction.  In  \ref{con:c:thet22k:ell}, the addition of $z$  prevents $\{w_1,w_4,v_{\ell-3}\}$ from being a maximal clique of $\cp{G}$ and hence an $i$-set of $G$. Similarly, in \ref{con:c:thet22k:5}, $z_1$  and $z_2$  eliminate triangles $\{w_1,w_4,v_1\}$ and $\{w_2,w_3,v_2\}$, respectively.  
	The unfortunate trade-off in this triangle-elimination technique is that the remaining triangles are no longer $\alpha$-sets; the constructions work only for $i$-graphs, not $\alpha$-graphs. 
	
	
	%>>>S: Lemma {lem:c:thet22l}	
	\begin{lemma} \label{lem:c:thet22l}
		If $\cp{G}$ is the graph constructed by Construction \ref{con:c:thet22k}, then $\ig{G} = \thet{2,2,\ell}$, for $\ell \geq 5.$
	\end{lemma}
	
	
	\begin{proof}		
		We prove the two constructions separately, beginning with the more general case.
		
		\begin{enumerate}[itemsep=0pt,label=(\alph*)]
			
			% ``````````````````````````````````  THET (2,2,l) ``````````````````````````````

\newpage
			
			\item $\mathbf{\thet{2,2,\ell}, \ell \geq 6}$.	\label{lem:c:thet22lpf} 
			
			As in the previous constructions, since each edge of $\cp{G}$ belongs to a triangle and some triangles are not contained in $K_4$'s, these maximal triangles of $\cp{G}$ form the smallest maximal cliques of $\cp{G}$ and, hence, the $i$-sets of $G$.  We label these triangles as in Figure \ref{fig:c:thet22k}; in particular,	
			
			\begin{center}	
				\begin{tabular}{rlcrl}
					$X$ & $=\{w_0,w_1,w_2\},$  & \null \hspace{20mm} \null &  	$D_1$ & $= \{w_1,w_2,v_1\},$ \\
					$Y$ & $= \{w_0,w_3,w_4\},$ &&   	$D_i$ & $= \{w_1,v_{i-1},v_i\}$  for $  2 \leq i \leq \ell-4,$\\
					$A$ & $=\{w_0,w_2,w_3\},$  &&		$D_{\ell-3}$ & $= \{v_{\ell-5}, v_{\ell-4}, v_{\ell-3} \},$ \\
					$B$ & $=\{w_0,w_1,w_4\},$  &&   	$D_{\ell-2}$ & $= \{w_{4}, v_{\ell-4}, v_{\ell-3} \},$ \\
					& && 							$D_{\ell-1}$ & $= \{w_3,w_4,v_{\ell-3}\}.$
				\end{tabular}
			\end{center}
			
			We further partition the vertices of $\cp{G}$ into two sets: the wheel vertices \linebreak  $W = \{w_0,w_1,\dots,w_4\}$ and the path vertices $Q=\{v_1,v_2,\dots, v_{\ell-3}\}$.  
			
			We first show that the only induced $K_4$ in $\cp{G}$ is $Z=\{z, w_1,w_4,v_{\ell-4}\}$, and hence each of the above listed $\ell+3$ sets is an $i$-set of $G$.  
			The vertices of $Q$ induce a path of length $\ell-2$, with an extra edge between $v_{\ell-3}$ and $v_{\ell-5}$, so no $K_4$ is contained entirely in $Q$.  There is likewise no $K_4$ with three vertices in $Q$, since the only set of three mutually adjacent vertices from $Q$ is $\{v_{\ell-5}, v_{\ell-4}, v_{\ell-3}\}$,  but $v_{\ell-5}$ and  $v_{\ell-3}$ have no common neighbours outside of $Q$ in $\cp{G}$.   Moreover, there is also no $K_4$ using exactly two vertices of $Q$; notice that $w_2$ and $w_3$ both have only a single edge  to $Q$, and so any $K_4$ with two vertices on $Q$ would necessarily contain $w_1$ and $w_4$. Since $N_{\cp{G}}(\{w_1,w_4\})\cap Q=\{v_{\ell-4}\}$, the only $K_4$ containing two vertices from $\mathcal{W}$ is $Z$, as required.
			
				
			%=========== NO EXTRA TRIANGLES ==========
			Next, we show that there are no additional $i$-sets in $G$ beyond the $\ell+3$ labelled maximal clique triangle faces of $\cp{G}$ listed above. As all of the neighbours of $z$ lie on a $K_4$, we can divide any potential unaccounted for $i$-sets into two categories: those with at least two vertices on the wheel $W$, and those with at least two vertices on the path $Q$.
			
			In the first case, $N_{\cp{G}}(w_0) = \{w_1,w_2,w_3,w_4\}$, so every triangle containing $w_0$ has already been accounted for (on the faces of the wheel), including those with three vertices on $W$.  As  noted above, the open neighbourhoods of $w_3$ and $w_4$ share only $v_{\ell-3}$ on $Q$, and we have already labelled $D_{\ell-1}  = \{w_3, w_4, v_{\ell-3}\}$.  The only neighbour of $w_2$ on $Q$ is $v_1$, which is not adjacent to $w_3$, so there are no triangles containing both $w_2$ and $w_3$, aside from $A$.  The only other wheel vertex adjacent to $v_1$ is $w_1$, but we have again already accounted for $D_1 = \{w_1,w_2,v_1\}$.  Finally, $\left( N_{\cp{G}}(w_1) \cap N_{\cp{G}}(w_4) \right) \cap Q = \{v_{\ell-4}\}$, and since $\{w_1,w_4,v_{\ell-4}\}$ is a part of the clique  $Z$, it is not an $i$-set of $G$.
			
			We now show that there are no additional triangles of $\cp{G}$ with at least two vertices on $Q$ that are $i$-sets of $G$.  For a vertex $v_i \in Q$ with $ 2 \leq i \leq {\ell-6}$, we have that $N(v_i) \cap Q = \{v_{i-1}, v_{i+1}\}$ and, furthermore, $\left(N_{\cp{G}}(v_i) \cap N_{\cp{G}}(v_{i+1}) \right) \cap W = \{w_1\}$.  Each such triangle $D_{i+1} = \{w_i,v_{i},v_{i+1}\}$ has already been accounted for.  With the vertices of  $Q$ along  the middle of the path now checked, we consider special cases present in the end vertices.  The only triangle entirely contained in $Q$ is $D_{\ell-3} = \{v_{\ell-5}, v_{\ell-4}, v_{\ell-3} \}$, which has already been considered.  
			For $v_1$, $N_{\cp{G}}(v_{1}) \cap Q = \{v_2\}$, and as before, since $N_{\cp{G}}(v_1) \cap N_{\cp{G}}(v_{2}) = \{w_1\}$, we have already listed the triangle $D_2 = \{v_1,v_2,w_2\}$ as an $i$-set of ${G}$.  
			Once again,  $N_{\cp{G}}(v_{\ell-4}) \cap N_{\cp{G}}(v_{\ell-5}) = \{w_1, v_{\ell-3}\}$, so $v_{\ell-4}$ and $v_{\ell-5}$ are on only the previously considered triangles $D_{\ell-3}$ and $D_{\ell-4}$.  Finally, 
			$N_{\cp{G}}(v_{\ell-3}) \cap N_{\cp{G}}(v_{\ell-4}) = \{w_4\}$, which is once more associated only with the previously labelled triangle $	D_{\ell-1}  = \{w_3,w_4,v_{\ell-3}\}$.  This concludes all possible cases where a triangle of $\cp{G}$ could be an $i$-set of $G$;  the $i$-sets of $G$, and hence the vertices of $\ig{G}$, are exactly the $\ell+3$ sets listed above.
			
	
			Moving to the edges of $\ig{G}$, the following adjacencies are clear from Figure \ref{fig:c:thet22k}.
			%=========== EDGES =====================
			\begin{enumerate}[itemsep=0pt, label=(\roman*)]
				\item \label{lem:c:thet22l:i}  $\edge{X, w_1, w_3,A} \adedge{w_2,w_4,Y}$, 
				
				\item \label{lem:c:thet22l:j} $\edge{X, w_2, w_4,B} \adedge{w_1,w_3,Y}$, 
				
				\item \label{lem:c:thet22l:k} $\edge{X, w_0,  v_1,D_1}  \adedge{w_2,v_2,D_2} \adedge{v_1,v_3,D_3} \dots \edge{D_{\ell-3}, w_1,  v_{\ell-2},D_{\ell-2}} \adedge{v_{\ell-4},w_4,D_{\ell-1}}  \adedge{v_{\ell-2},w_0,Y}$. 
			\end{enumerate}
			% =========== NO EXTRA EDGES =============
			
			We need only show that there are no additional adjacencies in $\ig{G}$ not listed above.
			Since $\cp{G}$ is planar, and an $i$-set triangle is adjacent to another $i$-set triangle if and only if they share exactly one edge if $\cp{G}$, each $i$-set of $G$ is adjacent to at most three others.  
			In our lists above, $X$ and $Y$ are each adjacent to three other $i$-sets, and thus no others.  The $i$-set $A$ is adjacent to both $X$ and $Y$ in the list, and from the above, no other triangles of $\cp{G}$ contain both $w_2$ and $w_3$, and so there is no triangle adjacent to $A$ sharing the edge $w_2w_3$.  
			Similarly, the $i$-set $B$ is listed above as being adjacent to both $X$ and $Y$.  While $\{w_1,w_4,z\}$ is a triangle adjacent to $B$, it is part of the $K_4$, $Z$, and so not an $i$-set.
			
			
			All that remains is to show that each $D_i \in \ig{V(G)}$ has $\deg_{\ig{G}}(D_i)=2$ for all $i=1,2, \dots \ell-1$.  Suppose that some $D_i$ had  $\deg_{\ig{G}}(D_i)\geq 3$.  We have already established that $D_i$ is not adjacent to $A$ nor to $B$, and   unless $i=1$ or $ \ell-1$, $D_i$ is not adjacent to any of $X$ or $Y$ either.  
			%So say that $D_i$ is adjacent to some  $D_j$ (without loss of generality, $i  < j$), where $j-i \geq 2$. 
			If $ 1 \leq i \leq \ell-5$, then because $\deg_{\cp{G}} (v_i) =3$, the triangle $D_i$  is adjacent only to $D_{i-1}$ and $D_{i+1}$ as in the lists above.	
			The triangle $D_{\ell-4}$ is adjacent to both $D_{\ell-3}$ and $D_{\ell-5}$, and while it does share the edge $w_1v_4$ with the triangle $\{w_1,v_4,z\} \subseteq Z$, the triangle is not an $i$-set of $G$.	
			The triangle $D_{\ell-3}$ is adjacent to both $D_{\ell-2}$ and $D_{\ell-4}$, and no other triangles contain the edge $v_{\ell-5}v_{\ell-3}$.
			Similarly to $D_{\ell-4}$, the triangle $D_{\ell-2}$ does share the edge $w_4v_4$ with the triangle $\{w_4,v_4,z\} \subseteq Z$, which is not an $i$-set of $G$.	
			The triangle $D_{\ell-1}$ is adjacent to both $Y$ and $D_{\ell-2}$, and no other triangles contain the edge $w_3v_{\ell-3}$.  This completes the exhaustive verification that $\deg_{\ig{G}}(D_i) =2$ for all $1 \leq i \leq  \ell-1$. Moreover, we have confirmed that no additional edges are  present in the $i$-graph of $G$ beyond those given in $(i)-(iii)$.  We conclude that $\ig{G} = \thet{2,2,\ell}$.
			
			
			% ``````````````````````````````````  THET (2,2,l) ``````````````````````````````	
			
			\item $\mathbf{\thet{2,2,5}}$	\label{lem:c:thet225pf}
			
			As in part (a), we begin by listing the triangle sets:
			
			%=========== TRIANGLES =====================
			\begin{center}	
				\begin{tabular}{rlcrl}
					$X$ & $=\{w_0,w_1,w_2\},$  & \null \hspace{20mm} \null &  	$D_1$ & $= \{w_1,w_2,v_1\},$ \\
					$Y$ & $= \{w_0,w_3,w_4\},$ &&   	$D_2$ & $= \{w_2,v_{1},v_2\},$  \\
					$A$ & $=\{w_0,w_2,w_3\},$  &&		$D_{3}$ & $= \{ w_4,v_1,v_2 \},$ \\
					$B$ & $=\{w_0,w_1,w_4\},$  &&   	$D_4$ & $= \{ w_3,w_4,v_2 \}.$ \\
				\end{tabular}
			\end{center}
			
			\noindent Using this labelling of the vertices and triangles of $\cp{G}$ (and as in Figure \ref{fig:c:thet225}), we see that many of the arguments for this case proceed similarly to those in (a).  
			In particular, the triangles of $\cp{G}$ correspond to the $i$-sets of $G$, and so the triangles $A, B, X$ and $Y$, which form a wheel in $\cp{G}$, are an induced $C_4$ in $\ig{G}$.  We need only confirm that a third disjoint path in $\ig{G}$ is formed between $X$ and $Y$ (and nothing else).   Explicitly, the three pathways are:
			
			%=========== EDGES =====================
			\begin{enumerate}[itemsep=0pt, label=(\roman*)]
				\item \label{lem:c:thet225:i}  $\edge{X, w_1, w_3,A} \adedge{w_2,w_4,Y}$, 
				
				\item \label{lem:c:thet225:j} $\edge{X, w_2, w_4,B} \adedge{w_1,w_3,Y}$, 
				
				\item \label{lem:c:thet225:k} $\edge{X, w_0,  v_1,D_1}  \adedge{w_1,v_2,D_2} 
				\adedge{w_2,w_4,D_3} 
				\adedge{v_1,w_3,D_4} 
				\adedge{v_2,w_0,Y}$.		
			\end{enumerate}
			
			
			To see that there are no additional  $i$-sets of $G$ beyond those corresponding to the eight labelled triangles of Figure \ref{fig:c:thet225}, notice first that neither $z_1$ nor $z_2$ are in any $i$-sets as each of their closed neighbourhoods forms a $K_4$.  The vertices of the wheel $\{w_0,w_1,\dots,w_4\}$ form a $C_4$ in $\ig{G}$ by Lemma \ref{lem:c:wheel}, so any potential additional triangles contain at least one of $v_1$ or $v_2$.  Since $\cp{G}$ is planar, each vertex $v_i$ is in at most $\deg_{\cp{G}}(v_i)$ triangles of $\cp{G}$ and hence at $\deg_{\cp{G}}(v)$ $i$-sets of $G$.  For $v_1$, since $z_1\in N_{\cp{G}}(v_1)$ and $z_1$ is not any $i$-sets, $v_1$ is in at most $\deg_{\cp{G}}(v_1)-2 = 5-2=3$ $i$-sets, of which we have already accounted for all three ($D_1, D_2$, and $D_3$).   Similar arguments for $v_2$ follow.  Thus, the eight $i$-sets listed above are the only $i$-sets of $G$.
			
			Again, since $\cp{G}$ is planar and all triangles are facial, each $i$-set of $G$ is adjacent to at most three others in $\ig{G}$. 
			For $D_1$, notice that the edge $w_1v_1$ in $\cp{G}$ is part of a $K_4$ with $z_1$, and hence not adjacent to another triangle over that shared edge, and so $\deg_{\ig{G}}(D_1)=2$.    Similarly, each $D_i$  for $i\in \{2,3,4\}$ also has at least one shared edge with a $K_4$ in $\cp{G}$, containing either $z_1$ or $z_2$, so that $\deg_{\ig{G}}(D_i)=2$ for all $1 \leq i \leq 4$; there are no additional edges in $\ig{G}$ than those listed above.  We can therefore conclude that $X,D_1, D_2, D_3, D_4, Y$  forms the third disjoint path of length five  of a theta graph in $\ig{G}$, as required.
		\end{enumerate}	
	\end{proof}
	
	
	% <<< E: Lemma {lem:c:thet22l}
	
	
	
	
	
	
	
	%______________________  SUBSUB SECTION:  theta (2,3,ell) ______________________
	
	\subsubsection{Construction of $\cp{G}$ for $\thet{2,3,\ell}$ for $\ell \geq 5$}  \label{subsubsec:c:23k}
	
	
	For many of the results going forward, we apply small modifications to previous constructions.  In the first of these, we begin with the graphs from Construction \ref{con:c:thet22k}, which were used to find $i$-graphs for $\thet{2,2,5}$ and $\thet{2,2,\ell}$ for $\ell \geq 6$,  and expand the central wheel used in there  to build $i$-graphs for $\thet{2,3,5}$ and $\thet{2,3,\ell}$ for $\ell \geq 6$. 
	
	
	%>>> S:CONS {con:c:thet23k}
	\begin{cons} \label{con:c:thet23k}  Refer to Figures \ref{fig:c:thet23k} and \ref{fig:c:thet235}.	
		
		\begin{enumerate}[itemsep=0pt, label=(\alph*)]
			\item \label{con:c:thet23k:ell}
			If $\ell \geq 6$, begin with a copy of the graph $\cp{G}$ from Construction \ref{con:c:thet22k}.  Subdivide the edge $w_1w_4$, adding the new vertex $w_5$.  Join $w_5$ to $w_0$, so that $w_0,w_1,\dots,w_5$ forms a wheel.  Delete the vertex $z$.
			
			
			\item \label{con:c:thet23k:5} If $\ell=5$,	begin with a copy of the graph $\cp{G}$ from Construction \ref{con:c:thet22k}.  Subdivide the edge $w_1w_4$, adding the new vertex $w_5$.  Join $w_5$ to $w_0$, so that $w_0,w_1,\dots,w_5$ forms a wheel.  Delete the vertex $z_1$.
		\end{enumerate}
		
		We rename the resultant (planar) graph $\cp{G_{2,3,\ell}}$ for $\ell \geq 5$.	
	\end{cons}
	% <<< E: Cons {con:c:thet23k}
	
	
	
	%\nts{For any of the modified constructions, be careful that $\cp{G^*}$ is being used, not just $\cp{G}$}.
	
	
	%>>> S: Figure {fig:c:thet23k}		
	\begin{figure}[H] \centering	
		\begin{tikzpicture}	
			%---------- REF-------------------
			\node (cent) at (0,0) {};
			\path (cent) ++(0:45 mm) node (rcent) {};	
			
			%----Left Wheel
			\node[std,label={10:$w_0$}] (w0) at (cent) {};
			
			\foreach \i in {1,2,3,4}
			{
				\node[std,label={180-\i*90: $w_{\i}$}] (w\i) at (180-\i*90:2cm) {}; 
				\draw[thick] (w0)--(w\i);        
			}
			\draw[thick] (w1)--(w2)--(w3)--(w4);
			
			
			%---- Wheel Division
			
			\node[std,label={135: $w_5$}] (w5) at (135:2cm) {}; 
			\draw[thick] (w0)--(w5); 
			\draw[thick] (w4)--(w5)--(w1);
			
			
			%---- V-Path	
			\foreach \i / \j  / \k in {1/3/1, 2/2/2, 3/-1/{\ell-5}, 4/-2.2/{\ell-4}, 5/-3.4/{\ell-3}}
			{	
				\path (rcent) ++(90:\j cm) node [bblue] (v\i) {};
			}
			
			\path (v1) ++(0:4 mm) node (Lv1) {\color{blue} $v_{1}$};
			\path (v2) ++(0:4 mm) node (Lv2) {\color{blue} $v_{2}$};
			\path (v3) ++(-10:5 mm) node (Lv3) {\color{blue} $v_{\ell-5}$};
			\path (v4) ++(15:8 mm) node (Lv4) {\color{blue} $v_{\ell-4}$};
			\path (v5) ++(260:4 mm) node (Lv5) {\color{blue} $v_{\ell-3}$};		
			
			\draw[bline] (v1)--(v2);
			\draw[bline] (v3)--(v4)--(v5);
			
			\draw[bline,dashed] (v2)--(v3);
							
			%---- V3 to V5 
			\path (v4) ++(180:1.3 cm) coordinate  (c1) {};
			\draw[bline] (v3) to[out=190, in=90]  (c1) to [out=270,in=170] (v5);  
			
			%---- V1 to W1 W2
			\draw[bline] (w1)--(v1)--(w2);	
			
			%-- V5 to W4
			\draw[bline] (v5) to[out=180,in=-90] (w4);  
			
			%-- V5 to W3
			\draw[bline] (v5) to[out=170,in=-30] (w3);  
			
			%-- V4 to W4
			\draw[bline] (v4) to[out=-35, in=25]  (4.6cm,-4.2cm) to [out=205,in=260] (w4);  	
			%	\node[std] at (4.6cm,-4.2cm) {};
			
			%-- W1 to V2  
			\draw[bline] (w1) to[out=20, in=170]  (35:6.1cm) to [out=350,in=45]  (v2); 
			%	\node[std] at (35:6.2cm)  {};
			
			%-- W1 to  V3 
			\draw[bline] (w1) to[out=38, in=170]  (35:6.4cm) to [out=350,in=45]  (v3); 
			%\node[std] at (30:5.8cm) {};
			
			%-- W1 to V4
			\draw[bline] (w1) to[out=46, in=170]  (35:6.6cm) to [out=350,in=40]  (v4); 
			%\node[std] at (30:5.8cm) {};
			
			%-- W1 to Dashed T1 T2
			\path (rcent) ++(90:1 cm) coordinate  (t1) {};
			\draw[bline, dashed] (w1) to[out=28, in=170]   (35:6.2cm)  to [out=350,in=45]  (t1); 
			
			\path (rcent) ++(90:0 cm) coordinate  (t2) {};
			\draw[bline, dashed] (w1) to[out=34, in=170]  (35:6.3cm) to [out=350,in=45]  (t2); 
			
			%----- Internal Red Labellings	
			\path (w0) ++(45:9 mm) node [color=red] (X) {$X$};			
			\path (w0) ++(45:-9 mm) node [color=red] (Y) {$Y$};
			
			%-- A's
			\path (w0) ++(-45:8 mm) node [color=red] (A1) {$A$};	
			
			%-- B's	
			%\path (w0) ++(135:8 mm) node [color=red] (B1) {$B$};		
			\path (w0) ++(113:12 mm) node [color=red] (B1) {$B_1$};	
			\path (w0) ++(158:12 mm) node [color=red] (B2) {$B_2$};	
			
			%-- D's
			\path (w2) ++(90:15 mm) node [color=red] (D1) {$D_1$};	
			\path (v1) ++(170:9 mm) node [color=red] (D2) {$D_2$};	
			\path (v4) ++(160:7 mm) node [color=red] (D3) {$D_{\ell-3}$};
			\path (v5) ++(195:20mm) node [color=red] (D4) {$D_{\ell-2}$};
			\path (w4) ++(-55:20 mm) node [color=red] (D5) {$D_{\ell-1}$};			
			
		\end{tikzpicture} 
		\caption{The graph $\cp{G_{2,3,\ell}}$ from Construction \ref{con:c:thet23k} \ref{con:c:thet23k:ell} such that $\ig{G_{2,3,\ell}} = \ag{G_{2,3,\ell}} =  \thet{2,3,\ell}$ for $\ell \geq 6$.}
		\label{fig:c:thet23k}
	\end{figure}	
	% <<< E: Figure {fig:c:thet23k}
	
	
	
	
	
	%>>> S: Figure {fig:c:thet235}	
	\begin{figure}[H] \centering	
		\begin{tikzpicture}				
			%---------- REF-------------------
			\node (cent) at (0,0) {};
			\path (cent) ++(0:45 mm) node (rcent) {};	
			
			%----Left Wheel
			\node[std] (w0) at (cent) {};
			\path (w0) ++(30: 4 mm) node (Lw0) {${w_0}$};	
			
			\foreach \i in {1,2,3,4}
			{
				\node[std,label={180-\i*90: $w_{\i}$}] (w\i) at (180-\i*90:2.5cm) {}; 
				\draw[thick] (w0)--(w\i);        
			}
			\draw[thick] (w1)--(w2)--(w3)--(w4);
			
			%---- Wheel Division
			\node[std,label={90: $w_5$}] (w5) at (135:2.2cm) {}; 
			\draw[thick] (w0)--(w5); 
			\draw[thick] (w4)--(w5)--(w1);				
			
			%---- V-Path	
			\foreach \i / \j in {1/1.5,2/-1.5}
			{	
				\path (rcent) ++(90:\j cm) node [std] (v\i) {};
				\path (v\i) ++(0:4 mm) node (Lv\i) {$v_{\i}$};
				%\node[bblue,label={0: $v_\i$}] (v\i) at (4,2-\i) {};
			}
			\draw[thick] (w1)--(v1)--(v2)--(w3);
			\draw[thick] (v2)--(w2)--(v1);	
			
			%---- Zs
			\path (w2) ++(270:1 cm) node [bzed, label={[zelim]15:$\boldsymbol{z_2}$}] (z2) {};
			\draw[zline] (w2)--(z2)--(w3);
			\draw[zline] (z2)--(v2);			
					
			%--- V2 to W4 
			\draw[thick] (v2) to[out=230, in=-5]  (270:3.3cm) to [out=170,in=280] (w4);
			
			%--- V1 to W4
			\draw[thick] (v1) to[out=130, in=5]  (90:3.3cm) to [out=190,in=80] (w4);	
			
			%------------ THETA--------------------			
			%---- INNER Cycle	
			\foreach \i / \j  in {1/X,3/Y,4/A}
			{	
				\node[dred] (t\i) at (-45+\i*90:1.0cm) {}; 
				\path (t\i) ++(-45+\i*90: 4 mm) node (Lt\i) {\color{red}${\j}$};
				
			}
			
			%-- Modified B
			\path (w0) ++(113:10 mm) node [dred, label={[red]113: $B_1$}] (t2a) {};	
			\path (w0) ++(158:10 mm) node [dred, label={[red]158: $B_2$}] (t2b) {};			
			
			%---- D- Path
			\path (w2) ++(90: 7mm) node [dred] (d1) {};
			\path (d1) ++(90: 4 mm) node (Ld1) {\color{red}${D_1}$};
			
			\path (w2) ++(0: 15mm) node [dred] (d2) {};
			\path (d2) ++(90: 4 mm) node (Ld2) {\color{red}${D_2}$};	
			
			\path (v2) ++(-45: 10mm) node [dred] (d3) {};
			\path (d3) ++(0: 4 mm) node (Ld3) {\color{red}${D_3}$};	
			
			\path (t3) ++(260: 15mm) node [dred] (d4) {};
			\path (d4) ++(170: 4 mm) node (Ld4) {\color{red}${D_4}$};					
			
			\draw[rlinemb](t1)--(t2a)--(t2b)--(t3)--(t4)--(t1);
			\draw[rlinemb](t1)--(d1)--(d2);	
			\draw[rlinemb](t3)--(d4);
			
			%--- D2 to D3 
			\draw[rlinemb] (d2) to [out=-45,in=90] (d3);
			
			%--- D3 to D4
			\draw[rlinemb] (d3) to [out=230,in=300] (d4);
			
		\end{tikzpicture} 
		
		\caption{The graph $\cp{G_{2,3,5}}$ from Construction \ref{con:c:thet23k} \ref{con:c:thet23k:5} such that $\ig{G_{2,3,5}} = \thet{2,3,5}$, with $\ig{G_{2,3,5}}$ overlaid in red.}
		\label{fig:c:thet235}
	\end{figure}	
	% <<< E: Figure {fig:c:thet235}
	
	
	
	In Construction \ref{con:c:thet23k} \ref{con:c:thet23k:ell}, notice that the vertex $z$ is deleted from $\cp{G}$.    In the original Construction \ref{con:c:thet22k} for a graph $\cp{G}$ with $\ig{G} = \thet{2,2,\ell}$ for $\ell \geq 6$, $z$ served to eliminate the unwanted triangle formed by $\{w_1,w_4, v_{\ell-4}\}$.   Now with the expanded wheel including $w_5$, $\{w_1,w_4, v_{\ell-4}\}$ is not a triangle in $\cp{G_{2,3,\ell}}$ ($\ell \geq 6$), and $z$ is not needed.  Indeed, as $\cp{G_{2,3,\ell}}$ now has $\alpha({G_{2,3,\ell}}) = 3$, its triangles are also $\alpha$-sets in $G$, and so we can immediately extend the construction from $i$-graphs to $\alpha$-graphs.
	
	The extension, however, does not apply to  Construction \ref{con:c:thet23k} \ref{con:c:thet23k:5} for the graph $\cp{G_{2,3,5}}$.  Here, we no longer require  $z_1$ (which served to eliminate the unwanted triangle formed by $\{w_1,w_4, v_1\}$), but $z_2$ remains and forms the clique $\{w_2,w_3,z_2,v_2\}$; thus, $\alpha({G_{2,3,5}}) = 4$.  
	
	%In Construction \ref{con:c:thet23k} \ref{con:c:thet23k:5}, notice the vertex $z_1$ is deleted from $\cp{G}$.    In the original Construction \ref{con:c:thet22k} for a graph $\cp{G}$ with $\ig{G} = \thet{2,2,5}$, $z_1$ served to eliminate the unwanted triangle formed by $\{w_1,w_4, v_1\}$.   Now with the expanded wheel including $w_5$, $\{w_1,w_4, v_1\}$ is not a triangle in $\cp{G^*}$, and $z_1$ is not needed.  The proof for the  following Lemma \ref{lem:c:thet23k} is otherwise very similar to the proof of Lemma \ref{lem:c:thet22l}, and so is omitted.
	
	The proof for the  following Lemma \ref{lem:c:thet23k} is otherwise very similar to the proof of Lemma \ref{lem:c:thet22l}, and so is omitted.
	
	
	%>>>S: Lemma {lem:c:thet23k}
	
	\begin{lemma}\label{lem:c:thet23k}
		If $\cp{G_{2,3,5}}$ is the graph constructed in Construction \ref{con:c:thet23k} \ref{con:c:thet23k:5}, then $\ig{G_{2,3,5}} = \thet{2,3,5}$.  
		For $\ell \geq 6$, if $\cp{G_{2,3,\ell}}$  is the graph constructed in Construction \ref{con:c:thet23k} \ref{con:c:thet23k:ell}, then $\ig{G_{2,3,\ell}} = \ag{G_{2,3,\ell}} = \thet{2,3,\ell}$ for $\ell \geq 6$.  	
	\end{lemma}
	
	% <<< E: Lemma {lem:c:thet23k}
	
	
	
	%Given the similarly between Constructions \ref{con:c:thet22k} and \ref{con:c:thet23k}, the proof for Lemma \ref{lem:c:thet23k} 
	
	
	%______________________  SUBSUB SECTION:  theta (2,4,4) ______________________
	
	\subsubsection{Construction of $\cp{G}$ for $\thet{2,4,4}$}  \label{subsubsec:c:244}
	
	
	In the following construction for a graph $\cp{G}$ with $\ig{G} = \thet{2,4,4}$, we again apply the technique of  adding vertices to eliminate unwanted triangles.
	
	%>>>S: Figure {fig:c:thet244}
	\begin{figure}[H] \centering	
		\begin{tikzpicture}		
			%---------- REF-------------------
			\node (cent) at (0,0) {};
			\path (cent) ++(0:50mm) node (rcent) {};	
			
			%----Left Wheel
			\node[std,label={0:$w_0$}] (w0) at (cent) {};
			
			\foreach \i in {1,2,3,4,5,6}
			{
				\node[std,label={150-\i*60: $w_{\i}$}] (w\i) at (150-\i*60:2cm) {}; 
				\draw[thick] (w0)--(w\i);        
			}
			\draw[thick] (w1)--(w2)--(w3)--(w4)--(w5)--(w6)--(w1);
			
			%---- V-Path
			\path (rcent) ++(90:0 cm) node [bblue] (v) {};
			\path (v) ++(0:4 mm) node (Lv) {\color{blue} $v$};	
			
			\foreach \i / \j in {2,3}
			\draw[bline] (v)--(w\i);	
			
			%--- W1 to V
			\draw[bline] (w1)   to[out=0,in=100] (v); 	
			%--- W4 to V
			\draw[bline] (w4)   to[out=0,in=-100] (v); 
			%--- W4 to W1
			\draw[bline] (w4) to[out=-15, in=270]  (0:6.5cm)  to[out=90,in=15] (w1); 		
			
			%---- Zs
			\path (w0) ++(0:3.5 cm) node [bzed, label={[zelim]180:$\boldsymbol{z}$}] (z) {};
			\draw[zline] (w2)--(z)--(w3);
			\draw[zline] (z)--(v);
			
			%\path (w0) ++(60:-3 cm) node [bred, label={[red]180:$z'$}] (z1) {};
			\path (w0) ++(180:4 cm) node [bzed, label={[zelim]180:$\boldsymbol{z'}$}] (z1) {};
			%	\draw[rline] (z1)--(w0);
			\draw[zline] (w0)   to[out=165,in=15] (z1); 
			\draw[zline] (w1)   to[out=180,in=80] (z1); 		
			\draw[zline] (w4)   to[out=180,in=-80] (z1); 	
			
			%----- Internal Red Labellings	
			\path (w0) ++(60:12 mm) node [color=red] (X) {$X$};	
			\path (w0) ++(0:12 mm) node [color=red] (A) {$A$};	
			\path (w0) ++(-60:12 mm) node [color=red] (Y) {$Y$};	
			
			%-- B's	
			\foreach \i/\c in {1/120,2/180,3/240} {				
				\path (w0) ++(\c:12 mm) node [color=red] (B\i) {$B_{\i}$};	
			}
			
			%-- D's
			\path (w2) ++(05:15 mm) node [color=red] (D1) {$D_1$};	
			\path (v) ++(60:8 mm) node [color=red] (D2) {$D_2$};	
			\path (w3) ++(-05:15 mm) node [color=red] (D3) {$D_3$};
			
		\end{tikzpicture} 
		\caption{A graph $\cp{G}$ such that $\ig{G} = \thet{2,4,4}$.}
		\label{fig:c:thet244}
	\end{figure}	
	% <<< E: Figure {fig:c:thet244}
	
	
	\begin{cons} \label{con:c:thet244}
		Refer to Figure \ref{fig:c:thet244}.	Begin with a copy of the graph $\cp{H} \cong W_7 = C_6 \vee K_1$, labelling the degree 3 vertices as $w_1,w_2,\dots, w_6$ and the central degree 6 vertex as $w_0$.  Join $w_1$ to $w_4$.    Add a new vertex $v$ to $\cp{H}$, joining $v$ to $w_1, \dots, w_4$.    Then, add the new vertex $z$, joined to $v$, $w_2$ and $w_3$, and the new vertex $z'$, joined to $w_0, w_1$, and $w_4$.  We label the resultant (non-planar) graph $\cp{G}$.	
	\end{cons}
	
	
	%>>>S: Lemma {lem:c:thet244}	
	\begin{lemma}\label{lem:c:thet244}
		If $\cp{G}$ is the graph constructed by Construction \ref{con:c:thet244}, then $\ig{G} = \thet{2,4,4}$.
	\end{lemma}
	
	
	\begin{proof}	
		Once again, the triangles of $\cp{G}$ form the smallest maximal cliques of $\cp{G}$, and are therefore the $i$-sets of $G$.  As in Figure \ref{fig:c:thet244}, we label them as follows:	
		
		%=========== VERTICES  ==========		
		\begin{center}	
			\begin{tabular}{rlcrlcrl}
				$X$ & $=\{w_0,w_1,w_2\},$  & \null \hspace{1mm} \null &  	$B_1$ & $= \{w_0,w_1,w_6\},$ & \null \hspace{1mm} \null &  	$D_1$ & $= \{w_1,w_2,v\},$ \\
				$Y$ & $= \{w_0,w_3,w_4\},$ &&   	$B_2$ & $= \{w_0,w_5,w_6\}$,  & \null \hspace{1mm} \null &  	$D_2$ & $= \{w_1,w_4,v\},$ \\
				$A$ & $=\{w_0,w_2,w_3\},$  &&		$B_3$ & $= \{w_0,w_4,w_5 \},$ & \null \hspace{1mm} \null &  	$D_3$ & $= \{w_3,w_4,v\}.$ \		
			\end{tabular}
		\end{center}
		
		%=========== NO EXTRA TRIANGLES ==========	
		Similarly to the construction for $\thet{2,2,5}$  in the proof of Lemma \ref{lem:c:thet22l}, the vertices $z$ and $z'$ are added to ensure that $\{v,w_2,w_3\}$ and $\{w_0,w_1,w_4\}$, respectively, are not maximal cliques in  $\cp{G}$, and hence are not $i$-sets of $G$.
		
		To see that there are no other triangles in $\cp{G}$ beyond the nine listed above (and thus, no unaccounted for $i$-sets in $G$), consider how one such unaccounted for triangle might be constructed.  Since $N[z]$ forms a maximal clique, $z$ (and likewise $z'$) is not on any such triangle.  Moreover, all triangles comprised entirely of vertices from the wheel $W_7 = \{w_0,w_1,\dots,w_6\}$ have already been considered.  This leaves only triangles potentially formed by two adjacent wheel vertices and $v$.  However, $\deg(v)=4$, so it is incident with at most $\binom{4}{2}=6$ triangles, three of which have already been accounted for in $D_1, D_2, D_3$, and one that has been eliminated by the clique $\{v,w_2,w_3,z\}$.  For the remaining two edge combinations, $vw_3$ and $vw_1$ are not on a triangle as $w_1w_2 \notin E(\cp{G})$, and likewise, $vw_2$ and $vw_4$ are not on a triangle as $w_2w_4 \notin E(\cp{G})$. We therefore conclude that the collection of triangles of $G$ are exactly the nine sets listed above; moreover, these nine sets are the $i$-sets of $G$ and hence form $V(\ig{G})$.
		
		
		From Figure \ref{fig:c:thet244}, the following triangle adjacencies are immediate:
		%=========== EDGES =====================		
		\begin{enumerate}[itemsep=0pt, label=(\roman*)]
			\item \label{lem:c:thet244:i}  $\edge{X, w_1, w_3,A} \adedge{w_2,w_4,Y}$, 
			
			\item \label{lem:c:thet244:j} $\edge{X, w_2, w_6,B_1} \adedge{w_1,w_5,B_2} \adedge{w_6,w_4,B_3} \adedge{w_5,w_3,Y} $, 
			
			\item \label{lem:c:thet244:k} $\edge{X, w_0,  v,D_1}  \adedge{w_2,w_4,D_2} \adedge{w_1,w_3,D_3}  \adedge{v,w_0,Y}$. 
		\end{enumerate}
		
		
		% =========== NO EXTRA EDGES =============	
		
		Lemma \ref{lem:c:wheel} ensures that there are no additional adjacencies between two triangles on the wheel $\cp{H}$. Thus, we need only check that there are no unaccounted for edges between the triangles of the collection $\mathcal{D} =\{D_1,D_2,D_3\}$, either among themselves, or with triangles from the wheel.  The former is easily removed by observing from Figure \ref{fig:c:thet244} that $D_1$ and $D_3$ share no edges and are therefore not adjacent.   For the latter, notice that each wheel triangle contains the vertex $w_0$, which  is not a part of any of $\mathcal{D}$.  Thus only triangles of $\mathcal{D}$ containing two consecutively numbered wheel vertices (i.e. $w_iw_{i+1}$ or $w_6w_1$) are adjacent to wheel triangles.  From the figure, this occurs exactly twice: $w_1w_2$ in $D_1$ and $w_3w_4$ in $D_3$, and has already been accounted for in the adjacencies $\edge{X, w_0,  v,D_1} $ and $\edge{Y,w_0,v,D_3}$.  This ensures that there are no additional edges in $\ig{G}$ than the ten outlined in the list above, and so  we conclude that $\ig{G} = \thet{2,4,4}$ as required.
	\end{proof}
	% <<< E:  Lemma {lem:c:thet244}	
	
	
	
	
	Notice that Construction \ref{con:c:thet244}  is our first theta graph construction that is not planar.  In particular, the constructed graph $\cp{G}$ has a $K_{3,3}$ minor, as illustrated in Figure \ref{fig:c:thet244nonPlanar}.  Indeed, with the exception of the constructions that are based upon Construction \ref{con:c:thet244}, all of our $i$-graph constructions use planar complement seed graphs.   
	
	
	%>>> S: Figure {fig:c:thet244nonPlanar}	
	\begin{figure}[H] \centering	
		\begin{tikzpicture}	
			
			%---------- REF-------------------
			\node (cent) at (0,0) {};
			\coordinate(arrowLeft) at (2.7cm,0){};
			\coordinate(arrowRight) at (4.5cm,0){};
			
			
			%----Left Wheel
			\node[std,label={0:$w_0$}] (w0) at (cent) {};
			
			\foreach \i in {1,2,3,4,5,6}
			{
				\node[std,label={150-\i*60: $w_{\i}$}] (w\i) at (150-\i*60:2cm) {}; 
				\draw[thick] (w0)--(w\i);        
			}
			\draw[thick] (w1)--(w2)--(w3)--(w4)--(w5)--(w6)--(w1);
			
			%---- Zs
			\path (w0) ++(180:3.6 cm) node [std, label={[red]180:$z'$}] (z1) {};
			\draw[thick] (w0)   to[out=165,in=15] (z1); 
			\draw[thick] (w1)   to[out=180,in=80] (z1); 		
			\draw[thick] (w4)   to[out=180,in=-80] (z1); 
			
			
			%---- Arrow
			\node[broadArrow,fill=white, draw=black, ultra thick, text width=1cm] at (3.5cm,0){};
			
		\end{tikzpicture} 	
		\begin{tikzpicture}	
			
			%---------- REF-------------------
			\node (cent) at (0,0) {};
			
			%----Left Wheel
			\node[bblue,label={0:$w_0$}] (w0) at (cent) {};
			
			\foreach \i in {1,2,3,4,5,6}
			\coordinate (w\i) at (150-\i*60:2cm) {}; 
			
			\draw[thick] (w1)--(w2)--(w3)--(w4)--(w5)--(w6)--(w1);
			
			\foreach \i in {2,6}
			\node[regRed,label={150-\i*60: $w_{\i}$}] (nw\i) at (w\i) {}; 
			
			\foreach \i in {1,4}
			\node[bblue,label={150-\i*60: $w_{\i}$}] (nw\i) at (w\i) {}; 
			
			\draw[thick] (nw6)--(w0)--(nw2);
			
			%---- Zs
			\path (w0) ++(180:3.6 cm) node [regRed, label={[red]180:$z'$}] (z1) {};
			\draw[thick] (w0)   to[out=165,in=15] (z1); 
			\draw[thick] (nw1)   to[out=180,in=80] (z1); 		
			\draw[thick] (nw4)   to[out=180,in=-80] (z1); 
			
		\end{tikzpicture} 
		\caption{A $K_{3,3}$ minor in a subgraph of $\cp{G}$ from Figure \ref{fig:c:thet244}.}
		\label{fig:c:thet244nonPlanar}
	\end{figure}	
	% <<< E: Figure {fig:c:thet244nonPlanar}
	
	
	\begin{problem}	\null 	
		\begin{enumerate}[label=(\roman*)]
			\item Find a planar graph-complement construction for $\thet{2,4,4}$.  
			
			\item 	 Do all $i$-graphs with largest induced stars of $K_{1,3}$, always have a planar graph-complement construction?
			
			\emph{A large target graph requires a large seed graph in order to generate a sufficient number of unique $i$-sets.   Can a target graph become too dense to allow for a planar graph-complement construction?}	
		\end{enumerate}
		
	\end{problem}
	
	
	
	
	%~~~~~~~~~~~~~~~~~~~~
	Moving forward, we will no longer explicitly check that there are no additional unaccounted for triangles in our constructions.  Should the construction indeed result in triangles of $\cp{G}$ that produce extraneous vertices in $\ig{G}$, we can easily remove them using the Deletion Lemma (Lemma \ref{lem:i:inducedI}).   In fact, in the previous results like in Construction \ref{con:c:thet244}, the addition of the vertices $z$ and $z'$ were exactly such applications.  Recall that in the Deletion Lemma, when removing a vertex $X$ from the $i$-graph, a new vertex $z$ is added to the seed graph and joined to every vertex \emph{except} those of $X$.  In the complement seed graph $\cp{G}$, this amounts to adding $z$ and joining to everything in $X$ (and nothing else), exactly as we did in Construction \ref{con:c:thet244}.
	
	Although we are quite certain that the graphs presented in the following constructions are the exact ones that build the given theta graphs, it is of no actual consequence to the existence of the theta graph as an $i$-graph if they are not; by the Deletion Lemma, it is enough to present the graph $\cp{G}$ with the knowledge that its complement produces  an $i$-graph that contains the desired theta graph as an induced subgraph.
	
	
	
	
	
	
	
	%______________________  SUBSUB SECTION:  theta (2,4,5) ______________________
	
	\subsubsection{Constructions of $\cp{G}$ for $\thet{2,4,5}$ and $\thet{2,5,5}$}  \label{subsubsec:c:245}
	
	%	\nts{Subdivision of 235 on B-Line once and twice, resp}
	
	%	\nts{Check construction environment}
	
	
	\begin{cons} \label{cons:c:thet2k5}	  Refer to Figure  \ref{fig:c:thet255}.
		\begin{enumerate}[itemsep=0pt, label=(\alph*)]
			\item \label{con:c:thet245:a}
			Begin with a copy of the graph $\cp{G_{2,3,5}}$ from Construction \ref{con:c:thet23k} \ref{con:c:thet23k:5} for $\thet{2,3,5}$.  Subdivide the edge $w_1w_5$, adding the new vertex $w_6$.  Join $w_6$ to $w_0$, so that $w_0,w_1,\dots,w_6$ forms a wheel.  Call this graph $\cp{G_{2,4,5}}$. 
			
			
			\item \label{con:c:thet255:b} Begin with a copy of the graph $\cp{G_{2,4,5}}$  from  Construction \ref{cons:c:thet2k5} \ref{con:c:thet245:a}.  Subdivide the edge $w_1w_6$, adding the new vertex $w_7$.  Join $w_7$ to $w_0$, so that $w_0,w_1,\dots,w_7$ forms a wheel.  Call this graph $\cp{G_{2,5,5}}$.  
		\end{enumerate}
	\end{cons}
	
	
	%>>> S: Figure {fig:c:thet255}	
	\begin{figure}[H] \centering	
		\begin{tikzpicture}	
			
			%---------- REF-------------------
			\node (cent) at (0,0) {};
			\path (cent) ++(0:45 mm) node (rcent) {};	
			
			%----Left Wheel
			\node[std] (w0) at (cent) {};
			\path (w0) ++(30: 4 mm) node (Lw0) {${w_0}$};	
			
			\foreach \i in {1,2,3,4}
			{
				\node[std,label={180-\i*90: $w_{\i}$}] (w\i) at (180-\i*90:2.5cm) {}; 
				\draw[thick] (w0)--(w\i);        
			}
			\draw[thick] (w1)--(w2)--(w3)--(w4);
			
			%---- Wheel Division
			%	\node[std,label={92: $w_5$}] (w5) at (157.5:2.5cm) {}; 
			\node[std] (w5) at (157.5:2.5cm) {}; 
			\path (w5) ++(94: 5 mm) node (Lw5) {${w_5}$};	
			\draw[thick] (w0)--(w5); 
			
			\node[std,label={90: $w_6$}] (w6) at (135:2.5cm) {}; 
			\draw[thick] (w0)--(w6); 
			
			\node[std,label={90: $w_7$}] (w7) at (112.5:2.5cm) {}; 
			\draw[thick] (w0)--(w7); 
			\draw[thick] (w4)--(w5)--(w6)--(w7)--(w1);		
			
			
			%---- V-Path	
			\foreach \i / \j in {1/1.5,2/-1.5}
			{	
				\path (rcent) ++(90:\j cm) node [std] (v\i) {};
				\path (v\i) ++(0:4 mm) node (Lv\i) {$v_{\i}$};
				%\node[bblue,label={0: $v_\i$}] (v\i) at (4,2-\i) {};
			}
			\draw[thick] (w1)--(v1)--(v2)--(w3);
			\draw[thick] (v2)--(w2)--(v1);	
			
			%---- Zs
			\path (w2) ++(270:1 cm) node [bzed, label={[zelim]15:$\boldsymbol{z_2}$}] (z2) {};
			\draw[zline] (w2)--(z2)--(w3);
			\draw[zline] (z2)--(v2);	
			
			%--- V2 to W4 
			\draw[thick] (v2) to[out=230, in=-5]  (270:3.3cm) to [out=170,in=280] (w4);
			
			%--- V1 to W4
			\draw[thick] (v1) to[out=130, in=5]  (110:3.5cm) to [out=195,in=110] (w4);	
			
			%	\node[std] (t1) at (110:3.3cm) {};
			
			
			%------------ THETA--------------------
			
			%---- INNER Cycle	
			\foreach \i / \j  in {1/X,3/Y,4/A}
			{	
				\node[dred] (t\i) at (-45+\i*90:1.0cm) {}; 
				\path (t\i) ++(-45+\i*90: 4 mm) node (Lt\i) {\color{red}${\j}$};
				
			}
			
			%-- Modified B
			%\path (w0) ++(101.25:10 mm) node [bred, label={[red]101.25: $B_1$}] (t2a) {};
			\path (w0) ++(101.25:10 mm) node [dred] (t2a) {};	
			\path (t2a) ++(101.25: 4 mm) node (Lt2a) {\color{red}${B_1}$};		
			\path (w0) ++(123.75:10 mm) node [dred] (t2b) {};
			\path (t2b) ++(123.75: 4 mm) node (Lt2b) {\color{red}${B_2}$};	
			\path (w0) ++(146.25:10 mm) node [dred] (t2c) {};	
			\path (t2c) ++(146.25: 4 mm) node (Lt2c) {\color{red}${B_3}$};	
			\path (w0) ++(168.75:10 mm) node [dred] (t2d) {};	
			\path (t2d) ++(168.75: 4 mm) node (Lt2d) {\color{red}${B_4}$};	
			
			
			%---- D- Path
			\path (w2) ++(90: 7mm) node [dred] (d1) {};
			\path (d1) ++(90: 4 mm) node (Ld1) {\color{red}${D_1}$};
			
			\path (w2) ++(0: 15mm) node [dred] (d2) {};
			\path (d2) ++(90: 4 mm) node (Ld2) {\color{red}${D_2}$};	
			
			\path (v2) ++(-45: 10mm) node [dred] (d3) {};
			\path (d3) ++(0: 4 mm) node (Ld3) {\color{red}${D_3}$};	
			
			\path (t3) ++(260: 15mm) node [dred] (d4) {};
			\path (d4) ++(170: 4 mm) node (Ld4) {\color{red}${D_4}$};			
			
			
			\draw[rlinemb](t1)--(t2a)--(t2b)--(t2c)--(t2d)--(t3)--(t4)--(t1);
			\draw[rlinemb](t1)--(d1)--(d2);	
			\draw[rlinemb](t3)--(d4);
			
			%--- D2 to D3 
			\draw[rlinemb] (d2) to [out=-45,in=90] (d3);
			
			%--- D3 to D4
			\draw[rlinemb] (d3) to [out=230,in=300] (d4);
			
		\end{tikzpicture} 
		
		\caption{The graph $\cp{G_{2,5,5}}$ from Construction \ref{cons:c:thet2k5}	 such that $\ig{G} = \thet{2,5,5}$, with $\ig{G}$ overlaid in red.}
		\label{fig:c:thet255}
	\end{figure}	
	% <<< E: Figure {fig:c:thet255}
	
	
	%>>>S: Lemma {lem:c:thet2k5}	
	\begin{lemma} \label{lem:c:thet2k5}	
		If $\cp{G_{2,k,5}}$ is the graph constructed by Construction \ref{cons:c:thet2k5}, then $\ig{G_{2,k,5}} = \thet{2,k,5}$, for $4 \leq k \leq 5$.
	\end{lemma}
	% <<< E: Lemma {lem:c:thet2k5}	
	
	
	\noindent The proof for Lemma \ref{lem:c:thet2k5} is similar to that of Lemma \ref{lem:c:thet22l}\ref{lem:c:thet225pf} and is omitted.
	
	%______________________  SUBSUB SECTION:  theta (2,4,4) ______________________
	
	\subsubsection{Construction of $\cp{G}$ for $\thet{2,k,\ell}$ for $\ell \geq k \geq 4$ and $\ell \geq 6$}  \label{subsubsec:c:2kl}
	
	
	
	%	\nts{Subdivision of 23L on B-Line (and D line)}
	
	\begin{cons}\label{cons:c:thet2kl} 
		See Figure \ref{fig:c:thet2kl}.   Begin with a copy of the graph $\cp{G_{2,3,\ell}}$ from Construction \ref{con:c:thet23k} \ref{con:c:thet23k:ell} for $\ell\geq 5$.  Subdivide the edge $w_1w_5$ $k-3$ times (for $k \leq \ell$), adding the new vertices $w_6$, $w_7, \dots, w_{k+2}$.  Join $w_6, w_7, \dots, w_{k+2}$ to $w_0$, so that $w_0,w_1,\dots,w_{k+2}$ forms a wheel.  Call this graph $\cp{G_{2,k,\ell}}$.		
	\end{cons}
	
	
	
	
	%>>> S: Figure {fig:c:thet2kl}	
	\begin{figure}[H] \centering	
		\begin{tikzpicture}	
			%---------- REF-------------------
			\node (cent) at (0,0) {};
			\path (cent) ++(0:45 mm) node (rcent) {};	
			
			%----Left Wheel
			\node[std,label={10:$w_0$}] (w0) at (cent) {};
			
			\foreach \i in {1,2,3,4}
			{
				\node[std,label={180-\i*90: $w_{\i}$}] (w\i) at (180-\i*90:2cm) {}; 
				\draw[thick] (w0)--(w\i);        
			}
			\draw[thick] (w1)--(w2)--(w3)--(w4);
			
			
			
			
			\node[std] (w5) at (157.5:2cm) {}; 
			\path (w5) ++(157: 5 mm) node (Lw5) {${w_5}$};	
			\draw[thick] (w0)--(w5); 
			
			
			\node[std,label={112: $w_{k-3}$}] (w7) at (112.5:2cm) {}; 
			\draw[thick] (w0)--(w7); 
			\draw[thick] (w4)--(w5);
			\draw[thick] (w7)--(w1);	
			\draw[dashed](w5)--(w7);	
			
			
			\coordinate  (c1) at ($(w5)!0.33!(w7)$) {};
			\coordinate  (c2) at ($(w5)!0.66!(w7)$) {};
			
			\draw [dashed] (c2)--(w0);
			\draw [dashed] (c1)--(w0);		
			
			
			
			%---- V-Path	
			\foreach \i / \j  / \k in {1/3/1, 2/2/2, 3/-1/{\ell-5}, 4/-2.2/{\ell-4}, 5/-3.4/{\ell-3}}
			{	
				\path (rcent) ++(90:\j cm) node [bblue] (v\i) {};
			}
			
			\path (v1) ++(0:4 mm) node (Lv1) {\color{blue} $v_{1}$};
			\path (v2) ++(0:4 mm) node (Lv2) {\color{blue} $v_{2}$};
			\path (v3) ++(-10:5 mm) node (Lv3) {\color{blue} $v_{\ell-5}$};
			\path (v4) ++(15:8 mm) node (Lv4) {\color{blue} $v_{\ell-4}$};
			\path (v5) ++(260:4 mm) node (Lv5) {\color{blue} $v_{\ell-3}$};		
			
			\draw[bline] (v1)--(v2);
			\draw[bline] (v3)--(v4)--(v5);
			
			\draw[bline,dashed] (v2)--(v3);
			
			
			%---- V3 to V5 
			\path (v4) ++(180:1.3 cm) coordinate  (c1) {};
			\draw[bline] (v3) to[out=190, in=90]  (c1) to [out=270,in=170] (v5);  
			
			%---- V1 to W1 W2
			\draw[bline] (w1)--(v1)--(w2);	
			
			%-- V5 to W4
			\draw[bline] (v5) to[out=180,in=-90] (w4);  
			
			%-- V5 to W3
			\draw[bline] (v5) to[out=170,in=-30] (w3);  
			
			%-- V4 to W4
			\draw[bline] (v4) to[out=-35, in=25]  (4.6cm,-4.2cm) to [out=205,in=260] (w4);  	
			%	\node[std] at (4.6cm,-4.2cm) {};
			
			%-- W1 to V2  
			\draw[bline] (w1) to[out=20, in=170]  (35:6.1cm) to [out=350,in=45]  (v2); 
			%	\node[std] at (35:6.2cm)  {};
			
			%-- W1 to  V3 
			\draw[bline] (w1) to[out=38, in=170]  (35:6.4cm) to [out=350,in=45]  (v3); 
			%\node[std] at (30:5.8cm) {};
			
			%-- W1 to V4
			\draw[bline] (w1) to[out=46, in=170]  (35:6.6cm) to [out=350,in=40]  (v4); 
			%\node[std] at (30:5.8cm) {};
			
			%-- W1 to Dashed T1 T2
			\path (rcent) ++(90:1 cm) coordinate  (t1) {};
			\draw[bline, dashed] (w1) to[out=28, in=170]   (35:6.2cm)  to [out=350,in=45]  (t1); 
			
			\path (rcent) ++(90:0 cm) coordinate  (t2) {};
			\draw[bline, dashed] (w1) to[out=34, in=170]  (35:6.3cm) to [out=350,in=45]  (t2); 
			
			%----- Internal Red Labellings	
			\path (w0) ++(45:9 mm) node [color=red] (X) {$X$};			
			\path (w0) ++(45:-9 mm) node [color=red] (Y) {$Y$};
			
			%-- A's
			\path (w0) ++(-45:8 mm) node [color=red] (A1) {$A$};	
			
			
			%-- B's	
			\path (w0) ++(100:14 mm) node [color=red] (B1) {$B_1$};	
			\path (w0) ++(170:14 mm) node [color=red] (B2) {$B_{k-1}$};
			
			
			%-- D's
			\path (w2) ++(90:15 mm) node [color=red] (D1) {$D_1$};	
			\path (v1) ++(170:9 mm) node [color=red] (D2) {$D_2$};	
			\path (v4) ++(160:7 mm) node [color=red] (D3) {$D_{\ell-3}$};
			\path (v5) ++(195:20mm) node [color=red] (D4) {$D_{\ell-2}$};
			\path (w4) ++(-55:20 mm) node [color=red] (D5) {$D_{\ell-1}$};			
			
		\end{tikzpicture} 
		\caption{The graph $\cp{G_{2,k,\ell}}$ from Construction \ref{cons:c:thet2kl} such that $\ig{\cp{G_{2,k,\ell}}} = \thet{2,k,\ell}$ for $\ell \geq k \geq 4$ and $\ell \geq 6$.}
		\label {fig:c:thet2kl}
	\end{figure}	
	% <<< E: Figure {fig:c:thet2kl}
	
	
	
	
	
	\noindent The proof for Lemma \ref{lem:c:thet2kl} again proceed similarly to the proof of Lemma \ref{lem:c:thet22l}\ref{lem:c:thet22lpf} and is also omitted.
	
	%>>>S: Lemma {lem:c:thet2kl}	
	\begin{lemma} \label{lem:c:thet2kl}
		If $\cp{G_{2,k,\ell}}$ is the graph constructed by Construction \ref{cons:c:thet2kl}, then $\ig{G_{2,k,\ell}} = \ag{G_{2,k,\ell}} = \thet{2,k,\ell}$, for $\ell \geq k \geq 4$ and $\ell \geq 6$.
	\end{lemma}
	% <<< E: Lemma {lem:c:thet2kl}
	
	
	
	%......................  SUB SECTION:  theta (3,k,l) .........................
	\subsection{$\thet{3,k,\ell}$}  \label{subsec:c:3kl}
	
	
	
	%.______________________ SUBSUB SECTION:  theta (3,3,4) ______________________
	
	\subsubsection{Construction of $\cp{G}$ for $\thet{3,3,4}$}  \label{subsubsec:c:334}
	
	When initially exploring the $i$-graph realizability of $\thet{3,3,4}$, we encountered difficulties in trying to use a similar method as above utilizing graph complements to create a graph $\cp{G}$ such that $\ig{G} = \thet{3,3,4}$.  Through trial and error, we instead found the graph $G$ in Figure \ref{fig:c:thet334}  which  has $\ig{G} = \thet{3,3,4}$.  Although the following Lemma \ref{lem:c:thet334} does not use graph complements, we include it here for the completeness of determining the $i$-graph realizability of theta graphs.
	
	
	%>>> START Prop {lem:c:thet334}
	\begin{lemma} \label{lem:c:thet334}
		$\thet{3,3,4}$ is an $i$-graph.
	\end{lemma}
	
	
	\begin{proof}		
		Consider the graph $G$ in Figure \ref{fig:c:thet334}.	We claim that $\ig{G}= \thet{3,3,4}$, with corresponding $i$-sets as illustrated in Figure \ref{fig:c:thet334Sets}.
		
		%>>>S: Figure {fig:c:thet334}
		\begin{figure}[H]\centering
			\begin{tikzpicture}	
				
				%---------- REF-------------------
				\node (cent) at (0,0) {};
				
				%----- W0 - Left Wheel ----------
				
				\node [std] at (cent) (v0) {};
				\path (v0) ++(90:4 mm) node (v0L) {$v_0$};		
				
				\foreach \i/\c in {1/25,2/155,3/-155, 4/-25} {				
					\path (v0) ++(\c:20 mm) node [std] (v\i) {};
					\path (v\i) ++(\c:5 mm) node (Lv\i) {$v_{\i}$};
					\draw[thick] (v0)--(v\i);
				}
				
				\draw[thick] (v1)--(v2)--(v3)--(v4)--(v1);
				
				\foreach \i/\c/\j /\d in {5/90/90/15,6/-90/-90/15,7/0/45/30} {				
					\path (v0) ++(\c:\d mm) node [std] (v\i) {};
					\path (v\i) ++(\j:5 mm) node (Lv\i) {$v_{\i}$};
				}	
				
				\draw[thick] (v1)--(v5)--(v2);
				\draw[thick] (v3)--(v6)--(v4);
				
				\draw[thick] (v1)--(v7)--(v4);
				\draw[thick] (v0)--(v7);
				
				\path (v0) ++(0:45 mm) node [std] (v8) {};
				\path (v8) ++(0:5 mm) node (Lv8) {$v_{8}$};
				\draw[thick] (v7)--(v8);
				
			\end{tikzpicture}
			\caption{A graph $G$ such that $\ig{G} = \thet{3,3,4}$.}
			\label{fig:c:thet334}	
		\end{figure}						
		% <<< E: Figure {fig:c:thet334}
		
		
		
		%>>>S: Figure {fig:c:thet334Sets}
		\begin{figure}[H]\centering
			\begin{tikzpicture}	
				
				%---------- REF-------------------
				\node (cent) at (0,0) {};
				\path (cent) ++(0:30 mm) node (rcent) {};	
				\path (cent) ++(180:10 mm) node (lcent) {};	
				
				%----- cent path ----------
				
				\foreach \i/\c in {1/20,2/5,3/-5, 4/-20} {				
					\path (cent) ++(90:\c mm) node [std] (v\i) {};
				}
				
				\draw[thick] (v1)--(v2)--(v3)--(v4);
				
				% --- left path ---
				\foreach \i/\c in {5/5,6/-5} {				
					\path (lcent) ++(90:\c mm) node [std] (v\i) {};		
				}	
				
				\draw[thick] (v1)--(v5)--(v6)--(v4);
				
				% --- right path ---
				\foreach \i/\c in {7/10,8/0,9/-10} {				
					\path (rcent) ++(90:\c mm) node [std] (v\i) {};
				}	
				
				\draw[thick] (v1)--(v7)--(v8)--(v9)--(v4);		
				
				% ---Set Coding ---	
				\foreach \i/\j/\k /\l / \a in {1/2/6/8/15,2/2/4/8/0,3/4/5/8/0,4/3/5/8/-15}
				\path (v\i) ++(\a:12 mm) node (Ls\i) {$\{v_{\j},v_{\k},v_{\l}\}$};
				
				\foreach \i/\j/\k /\l in {5/1/6/8,6/1/3/8}
				\path (v\i) ++(180:12 mm) node (Ls\i) {$\{v_{\j},v_{\k},v_{\l}\}$};
				
				\foreach \i/\j/\k /\l in {7/2/6/7,8/5/6/7,9/3/5/7}
				\path (v\i) ++(0:12 mm) node (Ls\i) {$\{v_{\j},v_{\k},v_{\l}\}$};		
				
			\end{tikzpicture}
			\caption{The $i$-graph of $G$ from Figure \ref{fig:c:thet334}.}
			\label{fig:c:thet334Sets}	
		\end{figure}						
		% <<< E: Figure {fig:c:thet334Sets}
		
		Notice that  at least one vertex of $T_1=\{v_1,v_2,v_5\}$, $T_2=\{v_3,v_4,v_6\}$ and $\{v_7,v_8\}$ is in each $i$-set of $G$, and hence $i(G)=3$.  It is straight-forward to confirm that each set illustrated in Figure \ref{fig:c:thet334} is an $i$-set of $G$, and that the $i$-set adjacencies (and edges of $\ig{G}$) are as drawn.
		
		Thus, we need only confirm that $G$ has only the nine $i$-sets listed there, and that none are unaccounted for in Figure \ref{fig:c:thet334}.  To do so, we proceed with a counting argument.
		If $v_8 \in S$, then there are three options for each of $T_1$ and $T_2$, except for the cases where the selected vertices of each triangle are adjacent to each other: $\{v_2,v_3\}$, $\{v_1,v_4\}$, and the case where $v_0$ is left undominated, $\{v_5,v_6\}$.  Thus there are $(3)(3) -3 = 6$ $i$-sets containing $v_8$.  If an $i$-set contains $v_7$, then $v_1$ and $v_4$ are not in the $i$-set, so exactly one each of $T_1-\{v_1\} = \{v_2,v_5\}$ and $T_2 - \{v_4\} = \{v_3,v_6\}$ are in the $i$-set.  We again must exclude the case where both adjacent $v_2$ and $v_3$ are present; since $v_7$ dominates $v_0$, we no longer exclude the case where both $v_5$ and $v_6$ are present.    Thus, there are $(2)(2)-1=3$ $i$-sets containing $v_7$.  This accounts for all nine of the $i$-sets required to construct $\thet{3,3,4}$.
	\end{proof}
	
	
	
	For completeness, in Figure \ref{fig:c:thet334comp} we include a sketch of the complement of the graph $G$ given in Figure \ref{fig:c:thet334}, using a layout more familiar to the format we saw in the previous constructions, but with the same vertex labellings as in Figure \ref{fig:c:thet334}.  Although very similar, it is one of only two the constructions (along with Construction \ref{con:c:thet244}) that require an additional edge between two of the wheel vertices (i.e. $v_5$ and $v_6$).   Moreover, this representation of $\cp{G}$ -  and the set $D_2$ in particular - highlights how easily triangles of $\cp{G}$ can be overlooked when they are non-facial triangles.  We also note that this construction joins $\thet{2,4,4}$ among those currently lacking a planar graph complement construction.
	
	
	\begin{figure}[H] \centering	
		\begin{tikzpicture}	
			%---------- REF-------------------
			\node (cent) at (0,0) {};
			\path (cent) ++(0:40 mm) node (rcent) {};	
			
			%----Left Wheel
			\node[std,label={0:$w_0$}] (w0) at (cent) {};
			
			\foreach \i / \j in {1/2,2/6,3/1,4/3,5/5,6/4}
			{
				\node[std,label={150-\i*60: $v_{\j}$}] (w\i) at (150-\i*60:2cm) {}; 
				\draw[thick] (w0)--(w\i);        
			}
			\draw[thick] (w1)--(w2)--(w3)--(w4)--(w5)--(w6)--(w1);
			
			%---- V-Path	
			\path (rcent) ++(90:-0.5 cm) node [bblue,label={[blue]0: $v_7$}] (v) {};	
			
			%---- W2 to V
			\draw[bline] (v)--(w2);
			
			%---- V to W1-Path
			\draw[bline] (v) to[out=90,in=0] (w1);  	
			
			
			%---- V2 to W4-Path
			\draw[bline] (v) to[out=230,in=0] (w4);  	
			
			%-- V to W5
			\draw[bline] (v) to [out=260,in=15]  (-60:3.5cm) to [out=195,in=270] (w5);  
			
			%--- W2 to W5	
			\draw[bline] (w2) to [out=90, in=20]  (120:3cm) to[out=200,in=160] (w5); 
			
			%------ Z 
			\path (w0) ++(120:5 mm) node [bzed, label={[zelim]90:$\boldsymbol{v_0}$}] (z) {};
			\draw[zline] (z)--(w0);
			\draw[zline] (w5)--(z)--(w2);
			
			
			%----- Internal Red Labellings	
			\path (w0) ++(60:12 mm) node [color=red] (X) {$X$};			
			\path (w0) ++(60:-12 mm) node [color=red] (Y) {$Y$};
			
			%-- A's
			\path (w0) ++(0:12 mm) node [color=red] (A1) {$A_1$};	
			\path (w0) ++(-60:12 mm) node [color=red] (A2) {$A_2$};
			
			%-- B's	
			\path (w0) ++(-60:-14 mm) node [color=red] (B1) {$B_1$};	
			\path (w0) ++(180:13 mm) node [color=red] (B2) {$B_2$};
			
			%-- D's
			\path (w2) ++(-10:15mm) node [color=red] (D1) {$D_1$};	
			\path (v) ++(280:10 mm) node [color=red] (D2) {$D_2$};
			\path (w4) ++(-25:15 mm) node [color=red] (D3) {$D_3$};
			
		\end{tikzpicture} 
		\caption{The complement of the graph $G$ from Figure  \ref{fig:c:thet334}: a graph $\cp{G}$ such that $\ig{G} = \thet{3,3,4}$.}
		\label{fig:c:thet334comp}
	\end{figure}		
	
	
	%______________________  SUBSUB SECTION:  theta (3,3,5) ______________________
	\subsubsection{Construction of $\cp{G}$ for $\thet{3,3,5}$}  \label{subsubsec:c:335}
	
	
	%>>> S: Cons {con:c:thet335}
	\begin{cons} \label{con:c:thet335}
		Refer to  Figure \ref{fig:c:thet335}.	Begin with a copy of the graph $\cp{H} \cong W_7 = C_6 \vee K_1$, labelling the degree 3 vertices as $w_1,w_2,\dots, w_6$ and the central degree 6 vertex as $w_0$.     Add new vertices $v_1$ and $v_2$ to $\cp{H}$, joined to each other.  Then, join $v_1$ to each of $\{w_1,w_2,w_5\}$, and $v_2$ to each of $\{w_2, w_4,w_5\}$.  Call this graph $\cp{G_{3,3,5}}$.
	\end{cons}
	% <<< E: Cons {con:c:thet335}
	
	
	%>>> S: Figure {fig:c:thet335}	
	\begin{figure}[H] \centering	
		\begin{tikzpicture}	
			%---------- REF-------------------
			\node (cent) at (0,0) {};
			\path (cent) ++(0:40 mm) node (rcent) {};	
			
			%----Left Wheel
			\node[std,label={0:$w_0$}] (w0) at (cent) {};
			
			\foreach \i in {1,2,3,4,5,6}
			{
				\node[std,label={150-\i*60: $w_{\i}$}] (w\i) at (150-\i*60:2cm) {}; 
				\draw[thick] (w0)--(w\i);        
			}
			\draw[thick] (w1)--(w2)--(w3)--(w4)--(w5)--(w6)--(w1);
			
			%---- V-Path	
			%\foreach \i in {1,2}
			\foreach \i/\j in {1/1.7, 2/-0.5}
			{
				%\node[bblue,label={[blue]0: $v_\i$}] (v\i) at (4,3.5-2*\i) {};
				\path (rcent) ++(90:\j cm) node [bblue,label={[blue]0: $v_\i$}] (v\i) {};
				%\path (v\i) ++(0:4 mm) node (Lv\i) {\color{blue} $v_{\i}$};		
			}	
			
			\draw[bline] (w1)--(v1)--(v2)--(w2)--(v1);
			
			%---- V2 to W4-Path
			\draw[bline] (v2) to[out=230,in=0] (w4);  	
			
			%-- V2 to W5
			%\node[std] (v2w5A) at (-60:3.5cm) {};
			%\node[std] (v2w5B) at  (-60:4cm){};
			\draw[bline] (v2) to [out=260,in=15]  (-60:3.5cm) to [out=195,in=270] (w5);  
			
			%--- V1 to W5		
			%\node[std] (v1w5A) at (-15:5cm){};
			%\node[std] (v1w5B) at  (-60:4cm){};
			\draw[bline] (v1) to [out=-45, in=70]  (-15:5cm) to [out=250,in=15]  (-60:4cm) to[out=195,in=260] (w5); 
			
			%----- Internal Red Labellings	
			\path (w0) ++(60:12 mm) node [color=red] (X) {$X$};			
			\path (w0) ++(60:-12 mm) node [color=red] (Y) {$Y$};
			
			%-- A's
			\path (w0) ++(0:12 mm) node [color=red] (A1) {$A_1$};	
			\path (w0) ++(-60:12 mm) node [color=red] (A2) {$A_2$};
			
			%-- B's	
			\path (w0) ++(-60:-12 mm) node [color=red] (B1) {$B_1$};	
			\path (w0) ++(180:12 mm) node [color=red] (B2) {$B_2$};
			
			%-- D's
			\path (w2) ++(100:5 mm) node [color=red] (D1) {$D_1$};	
			\path (w2) ++(-10:15mm) node [color=red] (D2) {$D_2$};	
			\path (v2) ++(280:10 mm) node [color=red] (D3) {$D_3$};
			\path (w4) ++(-25:15 mm) node [color=red] (D4) {$D_4$};	
			
		\end{tikzpicture} 
		\caption{A graph $\cp{G_{3,3,5}}$ from Construction \ref{con:c:thet335} such that $\ig{G_{3,3,5}} =$ $\thet{3,3,5}$.}
		\label{fig:c:thet335}
	\end{figure}		
	% <<< E: Figure {fig:c:thet335}
	
	
	\newpage
	
	%>>>S: Lemma {lem:c:thet335}
	\begin{lemma}\label{lem:c:thet335}
		If $\cp{G_{3,3,5}}$ is the graph constructed by Construction \ref{con:c:thet335}, then $\ig{G_{3,3,5}} = \ag{G_{3,3,5}} = \thet{3,3,5}$.
	\end{lemma}
	
	
	\begin{proof}	
		From Lemma \ref{lem:c:wheel}, the wheel $\mathcal{W} = \{w_0,w_1,\dots,w_6\}$ in $\cp{G}$ forms a cycle $C_6$ with vertices $\{X,A_1,A_2,Y,B_1,B_2\}$  in $\ig{G}$.	We need only check that in $\ig{G}$, $\mathcal{D}=\{D_1,D_2,D_3,D_4\}$ (where each $D_i$ is defined as in Figure \ref{fig:c:thet335}) forms the necessary third disjoint path between $X$ and $Y$.  As discussed above, we need not concern ourselves over whether there are additional triangles of $\cp{G}$ (and hence vertices of $\ig{G}$)  present beyond the ten labelled in red in Figure \ref{fig:c:thet335}: they can easily be removed via the Deletion Lemma if needed. 
		
		From Figure \ref{fig:c:thet335}, we have that
		\begin{align*}
			\edge{X, w_0,  v_1,D_1}  \adedge{w_1,v_2,D_2} \adedge{w_2,w_5,D_3}   \adedge{v_1,w_4,D_4}  \adedge{v_2,w_0,Y},
		\end{align*}
		which forms the additional path between $X$ and $Y$ in $\ig{G}$.  Since $D_2$ and $D_3$ both contain $v_1$ and $v_2$, of the four triangles in $\mathcal{D}$, only $D_1$ and $D_4$ contain at least two of the wheel vertices from $\mathcal{W}$.  Thus, only $D_1$ and $D_2$ are adjacent to wheel triangles, namely $X$ and $Y$, respectively.  It follows that there are no additional edges between the vertices of $\mathcal{D}$ in $\ig{G}$.  
		We conclude that $\ig{\cp{G}}= \thet{3,3,5}$. 		
	\end{proof}
	% <<< E: Lemma {lem:c:thet335}
	
	
	
	
	
	
	
	
	%______________________  SUBSUB SECTION:  theta (3,3,k) ______________________
	
	\subsubsection{Construction of $\cp{G}$ for $\thet{3,3,\ell}$ for $\ell \geq 6$}  \label{subsubsec:c:33k}
	
	
	
	%>>> S: Cons {con:c:thet33k}
	\begin{cons} \label{con:c:thet33k}
		Refer to Figure \ref{fig:c:thet33k}.	Begin with a copy of the graph $\cp{H} \cong W_7 = C_6 \vee K_1$, labelling the degree 3 vertices as $w_1,w_2,\dots, w_6$ and the central degree 4 vertex as $w_0$.     For $\ell \geq 6$, add to $\cp{H}$  a new path of $\ell-3$ vertices labelled as $v_1$, $v_2, \dots, v_{\ell-3}$.  Then, join $w_1$ to each of $\{v_1,v_2,\dots, v_{\ell-4}\}$, $w_4$ to $v_{\ell-3}$, and $w_5$ to $v_{\ell-4}$ and $v_{\ell-3}$.  Finally, join $v_{\ell-5}$ to $v_{\ell-3}$, so that $\{v_{\ell-5}, v_{\ell-4}, v_{\ell-3}\}$ form a  $K_3$.  
		Call this graph $\cp{G_{3,3,\ell}}$ for $\ell \geq 6$.
	\end{cons}
	% <<< E: Cons {con:c:thet33k}
	
	
	Although the general construction still applies for the case when $\ell=6$, we include a separate figure for the construction of $\thet{3,3,6}$ for reference below, because of the additional complication that now $v_1 = v_{\ell-5}$, and so the single vertex now has dual roles in the construction.
	
	
	%>>> S: Figure {fig:c:thet336}	
	\begin{figure}[H] \centering	
		\begin{tikzpicture}	
			
			%---------- REF-------------------
			\node (cent) at (0,0) {};
			\path (cent) ++(0:40 mm) node (rcent) {};	
			
			%----Left Wheel
			\node[std,label={0:$w_0$}] (w0) at (cent) {};
			
			\foreach \i in {1,2,3,4,5,6}
			{
				\node[std,label={150-\i*60: $w_{\i}$}] (w\i) at (150-\i*60:2cm) {}; 
				\draw[thick] (w0)--(w\i);        
			}
			\draw[thick] (w1)--(w2)--(w3)--(w4)--(w5)--(w6)--(w1);
			
			%---- V-Path	
			\foreach \i / \j in {1/1.5,2/0,3/-1.5}
			{	
				\path (rcent) ++(90:\j cm) node [bblue] (v\i) {};
				\path (v\i) ++(0:4 mm) node (Lv\i) {\color{blue} $v_{\i}$};
				%\node[bblue,label={0: $v_\i$}] (v\i) at (4,2-\i) {};
			}
			\draw[bline] (w1)--(v1)--(v2)--(v3)--(w4);
			\draw[bline] (w2)--(v1);
			
			%---- V1 to V3 
			\draw[bline] (v1) to[out=210,in=150] (v3);  	
			
			%-- V3 to W5
			\draw[bline] (v3) to[out=230,in=290] (w5);  
			
			%--- V2 to W5 and W1
			\draw[bline] (v2) to[out=-45, in=90]  (-15:4.9cm) to [out=270,in=0]  (-60:3.5cm) to[out=180,in=260] (w5); 
			\draw[bline] (v2) to[out=45, in=-90]  (15:4.9cm) to [out=-270,in=0]  (45:3.5cm) to[out=-180,in=20] (w1);	
			
			%----- Internal Red Labellings	
			\path (w0) ++(60:12 mm) node [color=red] (X) {$X$};			
			\path (w0) ++(60:-12 mm) node [color=red] (Y) {$Y$};
			
			%-- A's
			\path (w0) ++(0:12 mm) node [color=red] (A1) {$A_1$};	
			\path (w0) ++(-60:12 mm) node [color=red] (A2) {$A_2$};
			
			%-- B's	
			\path (w0) ++(-60:-12 mm) node [color=red] (B1) {$B_1$};	
			\path (w0) ++(180:12 mm) node [color=red] (B2) {$B_2$};
			
			%-- D's
			\path (w2) ++(100:5 mm) node [color=red] (D1) {$D_1$};	
			\path (v1) ++(140:9 mm) node [color=red] (D2) {$D_2$};	
			\path (v2) ++(180:5 mm) node [color=red] (D3) {$D_3$};
			\path (v3) ++(270:6 mm) node [color=red] (D4) {$D_4$};
			\path (w4) ++(-10:15 mm) node [color=red] (D5) {$D_5$};		
			
		\end{tikzpicture} 
		\caption{The graph $\cp{G_{3,3,6}}$ from Construction \ref{con:c:thet33k} such that $\ig{G_{3,3,6}} =  \ag{G_{3,3,6}}= \thet{3,3,6}$.}
		\label{fig:c:thet336}
	\end{figure}	
	% <<< E: Figure {fig:c:thet336}
	
	
	
	
	%>>> S: Figure {fig:c:thet33k}	
	%$\thet{3,3,k}$
	\begin{figure}[H] \centering	
		\begin{tikzpicture}	
			%---------- REF-------------------
			\node (cent) at (0,0) {};
			\path (cent) ++(0:45 mm) node (rcent) {};	
			
			%----Left Wheel
			\node[std,label={0:$w_0$}] (w0) at (cent) {};
			
			\foreach \i in {1,3,4,5,6}
			{
				\node[std,label={150-\i*60: $w_{\i}$}] (w\i) at (150-\i*60:2cm) {}; 
				\draw[thick] (w0)--(w\i);        
			}
			\node[std,label={0: $w_{2}$}] (w2) at (30:2cm) {}; 
			\draw[thick] (w0)--(w2);        
			
			\draw[thick] (w1)--(w2)--(w3)--(w4)--(w5)--(w6)--(w1);
			
			%---- V-Path	
			\foreach \i / \j  / \k in {1/3/1, 2/2/2, 3/-1/{\ell-5}, 4/-2.2/{\ell-4}, 5/-3.4/{\ell-3}}
			{	
				\path (rcent) ++(90:\j cm) node [bblue] (v\i) {};
				\path (v\i) ++(0:5 mm) node (Lv\i) {\color{blue} $v_{\k}$};
				%\node[bblue,label={0: $v_\i$}] (v\i) at (4,2-\i) {};
			}
			
			\draw[bline] (v1)--(v2);
			\draw[bline] (v3)--(v4)--(v5);
			
			\draw[bline,dashed] (v2)--(v3);
			
			%---- V3 to V5 
			\path (v4) ++(180:1.3 cm) coordinate  (c1) {};
			\draw[bline] (v3) to[out=190, in=90]  (c1) to [out=270,in=170] (v5);  
			
			%---- V1 to W1 W2
			\draw[bline] (w1)--(v1)--(w2);	
			
			%-- V5 to W4
			\draw[bline] (v5) to[out=180,in=-45] (w4);  
			
			%-- V5 to W5
			\draw[bline] (v5) to[out=200,in=290] (w5);  
			
			%-- V4 to W5
			\draw[bline] (v4) to[out=-35, in=25]  (4.6cm,-4.2cm) to [out=205,in=270] (w5);  	
			%	\node[std] at (4.6cm,-4.2cm) {};
			
			%-- W1 to V2  
			\draw[bline] (w1) to[out=30, in=170]  (35:6.1cm) to [out=350,in=45]  (v2); 
			%	\node[std] at (35:6.2cm)  {};
			
			%-- W1 to  V3 
			\draw[bline] (w1) to[out=38, in=170]  (35:6.4cm) to [out=350,in=45]  (v3); 
			%\node[std] at (30:5.8cm) {};
			
			%-- W1 to V4
			\draw[bline] (w1) to[out=46, in=170]  (35:6.6cm) to [out=350,in=40]  (v4); 
			%\node[std] at (30:5.8cm) {};
			
			%-- W1 to Dashed T1 T2
			\path (rcent) ++(90:1 cm) coordinate  (t1) {};
			\draw[bline, dashed] (w1) to[out=32, in=170]   (35:6.2cm)  to [out=350,in=45]  (t1); 
			
			\path (rcent) ++(90:0 cm) coordinate  (t2) {};
			\draw[bline, dashed] (w1) to[out=34, in=170]  (35:6.3cm) to [out=350,in=45]  (t2); 
			
			%----- Internal Red Labellings	
			\path (w0) ++(60:12 mm) node [color=red] (X) {$X$};			
			\path (w0) ++(60:-12 mm) node [color=red] (Y) {$Y$};
			
			%-- A's
			\path (w0) ++(0:12 mm) node [color=red] (A1) {$A_1$};	
			\path (w0) ++(-60:12 mm) node [color=red] (A2) {$A_2$};
			
			%-- B's	
			\path (w0) ++(-60:-12 mm) node [color=red] (B1) {$B_1$};	
			\path (w0) ++(180:12 mm) node [color=red] (B2) {$B_2$};
			
			%-- D's
			\path (w2) ++(105:8 mm) node [color=red] (D1) {$D_1$};	
			\path (v1) ++(170:9 mm) node [color=red] (D2) {$D_2$};	
			\path (v4) ++(160:7 mm) node [color=red] (D3) {$D_{\ell-3}$};
			\path (w4) ++(-55:25mm) node [color=red] (D4) {$D_{\ell-2}$};
			\path (w4) ++(-50:15 mm) node [color=red] (D5) {$D_{\ell-1}$};		
			
		\end{tikzpicture} 
		\caption{The graph $\cp{G_{3,3,\ell}}$ from Construction \ref{con:c:thet33k} such that $\ig{G} = \ag{G} = \thet{3,3,\ell}$ for $\ell \geq 6$.}
		\label{fig:c:thet33k}
	\end{figure}	
	% <<< E: Figure {fig:c:thet33k}
	
	
	
	
	%>>>S: Lemma {lem:c:thet33k}
	\begin{lemma}\label{lem:c:thet33k}
		If $\cp{G_{3,3,\ell}}$ is the graph constructed by Construction \ref{con:c:thet33k}, then $\ig{G_{3,3,\ell}} = \ag{G_{3,3,\ell}} = \thet{3,3,\ell}$ for $\ell \geq 6$.
	\end{lemma}
	
	
	\begin{proof}
		We proceed as in the proof of Lemma \ref{lem:c:thet335} for $\thet{3,3,5}$, labelling the triangles as in Figure \ref{fig:c:thet33k}, with the amendment that we now ensure that $\mathcal{D}=\{D_1,D_2,\dots,D_{\ell-1}\}$ forms the necessary third disjoint path between $X$ and $Y$ in $\ig{\cp{G}}$.
		
		Clearly from Figure \ref{fig:c:thet33k},
		\begin{align*}
			\edge{X, w_0,  v_1,D_1}  \adedge{w_2,v_2,D_2} \adedge{v_1,v_3,D_3}\onlyTilde{v_2,v_4} \dots
			\adedge{v_{\ell-6},v_{\ell-4},D_{\ell-4}} 
			\adedge{w_1,v_{\ell-3},D_{\ell-3}} 
			\adedge{v_{\ell-5},w_5,D_{\ell-2}}  
			\adedge{v_{\ell-4},w_4,D_{\ell-1}}
			\adedge{v_{\ell-3},w_0,Y},
		\end{align*}
		where $\edge{D_i, v_{i-1}, v_{i+1}, D_{i+1}}$ 	for $ 3 \leq i \leq \ell-5$.
		
		Similarly to  before, of the triangles in $\mathcal{D}$, only $D_1$ and $D_{\ell-1}$ contain at least two of the wheel vertices from $\mathcal{W}$, and so only $D_1$ and $D_{\ell-1}$  are adjacent to wheel triangles. 
		We may assume that each $i$-set of $\cp{G}$ corresponds to  a facial triangle of $G$; if this is not the case, then as previously observed, we can use the Deletion Lemma (Lemma \ref{lem:i:inducedI}) to remove any non-facial triangle $i$-sets from the construction.  Thus, a vertex of $\ig{G}$ corresponding to a triangle $D_i$ of $\cp{G}$  is adjacent to at most three others.
		
		For $i=2,3,\dots,\ell-5$, we have already seen that $\edge{D_{i-1},v_{i-2},v_i, D_i}\adedge{ v_{i-1}, v_{i+1}, D_{i+1}}$.  If $D_i$ were adjacent to a third facial triangle, it would be to a triangle sharing the edge  $v_{i-1}v_i \in E(\cp{G})$; however, $N_{\cp{G}}(\{v_{i-1},v_i\})	= \{w_1\}$ and since $D_i = \{w_1, v_{i-1},v_i\}$, no such third triangle exists.  Thus, we need only check that for $i\in\{1,\ell-4,\ell-3,\ell-2,\ell-1\}$,  $\deg_{\cp{G}}(D_i)=2$.  For these cases, it is not overly arduous to exhaustively check that the edges $v_1w_2$, $w_1v_{\ell-4}$, $v_{\ell-5}v_{\ell-3}$, $v_{\ell-4}w_5$, and $v_{\ell-3}w_4$, respectively, are incident with exactly one triangle in $\cp{G}$.
		It follows that in $\ig{G}$, $X,D_1, D_2, \dots, D_{\ell-2}, D_{\ell-1},Y$ forms a path of length $\ell$ from $X$ to $Y$, and hence that $\ig{G}=\thet{3,3,\ell}$.		
	\end{proof}
	
	
	% <<< E: Lemma {lem:c:thet33k}
	
	
	
	
	
	
	The proofs of the remaining lemmas in Section \ref{sec:c:thetaComp} are all similar to proofs of previous lemmas and are therefore omitted without further comments.
	
	
	
	
	
	
	%______________________  SUBSUB SECTION:  theta (3,4,4) ______________________
	\subsubsection{Construction of $\cp{G}$ for $\thet{3,4,4}$ }  \label{subsubsec:c:344}
	
	
	
	%\nts{PLACEHOLDER}
	%>>> S: Cons {con:c:thet344}
	\begin{cons}\label{con:c:thet344} See Figure \ref{fig:c:thet344}.	
		Begin with a copy of the graph $\cp{G}$ from Construction \ref{con:c:thet244}   for $\thet{2,4,4}$, which we rename here as $\cp{G_{2,4,4}}$.  Subdivide the edge $w_2w_3$, adding a new vertex $u_1$, and joining $u_1$ to $w_0$.  
		Delete the vertex $z$.   Call this graph $\cp{G_{3,4,4}}$.
		
		%Then, for each $3 \leq i \leq 6$,  relabel the vertices of the outter wheel from $w_i$ to $w_{i+1}$, and relabel $w^*$ as $w_3$.
		
		
		
	\end{cons}
	% <<< E: Cons {con:c:thet344}
	
	
	
	
	
	%>>> S: Figure {fig:c:thet344}	
	\begin{figure}[H] \centering	
		\begin{tikzpicture}	
			
			%---------- REF-------------------
			\node (cent) at (0,0) {};
			\path (cent) ++(0:50mm) node (rcent) {};	
			
			%----Left Wheel
			\node[std] (w0) at (cent) {};
			\path (w0) ++(55:6 mm) node (Lw0) {$w_0$};	
			
			\foreach \i in {1,2,3,4,5,6}
			{
				\node[std,label={150-\i*60: $w_{\i}$}] (w\i) at (150-\i*60:2cm) {}; 
				\draw[thick] (w0)--(w\i);        
			}
			
			\node[std,label={0: $u_1$}] (w) at (0:2cm) {}; 
			\draw[thick] (w0)--(w);
			
			\draw[thick] (w1)--(w2)--(w)--(w3)--(w4)--(w5)--(w6)--(w1);
			
			%---- V-Path
			
			\path (rcent) ++(90:0 cm) node [bblue] (v) {};
			\path (v) ++(0:4 mm) node (Lv) {\color{blue} $v$};	
			
			\foreach \i / \j in {2,3}
			\draw[bline] (v)--(w\i);	
			
			%--- W1 to V
			\draw[bline] (w1)   to[out=0,in=100] (v); 	
			%--- W4 to V
			\draw[bline] (w4)   to[out=0,in=-100] (v); 
			%--- W4 to W1
			\draw[bline] (w4) to[out=-15, in=270]  (0:6.5cm)  to[out=90,in=15] (w1); 	
			
			
			%\path (w0) ++(60:-3 cm) node [bred, label={[red]180:$z'$}] (z1) {};
			\path (w0) ++(180:4 cm) node [bzed, label={[zelim]180:$\boldsymbol{z'}$}] (z1) {};
			%	\draw[rline] (z1)--(w0);
			\draw[zline] (w0)   to[out=165,in=15] (z1); 
			\draw[zline] (w1)   to[out=180,in=80] (z1); 		
			\draw[zline] (w4)   to[out=180,in=-80] (z1); 
			
			
			%----- Internal Red Labellings	
			\path (w0) ++(60:12 mm) node [color=red] (X) {$X$};	
			\path (w0) ++(15:12 mm) node [color=red] (A1) {$A_1$};	
			\path (w0) ++(-15:12 mm) node [color=red] (A2) {$A_2$};	
			\path (w0) ++(-60:12 mm) node [color=red] (Y) {$Y$};	
			
			%-- B's	
			\foreach \i/\c in {1/120,2/180,3/240} {				
				\path (w0) ++(\c:12 mm) node [color=red] (B\i) {$B_{\i}$};	
			}
			
			%-- D's
			\path (w2) ++(05:15 mm) node [color=red] (D1) {$D_1$};	
			\path (v) ++(60:8 mm) node [color=red] (D2) {$D_2$};	
			\path (w3) ++(-05:15 mm) node [color=red] (D3) {$D_3$};
			
		\end{tikzpicture} 
		\caption{The   graph  \; $\cp{G_{3,4,4}}$ \;  from    Construction   \ref{con:c:thet344}    such    that    $\ig{G_{3,4,4}}$    $=$  $\thet{3,4,4}$.}
		\label{fig:c:thet344}
	\end{figure}		
	% <<< E: Figure {fig:c:thet344}
	
	
	
	%>>>S: Lemma {lem:c:thet344}
	\begin{lemma}\label{lem:c:thet344}
		If $\cp{G_{3,4,4}}$ is the graph constructed by Construction \ref{con:c:thet344}, then $\ig{G_{3,4,4}} = \thet{3,4,4}$.
	\end{lemma}
	%>>>E: Lemma {lem:c:thet344}
	
	
	%______________________  SUBSUB SECTION:  theta (3,4,l) ______________________
	\subsubsection{Construction of $\cp{G}$ for $\thet{3,4,\ell}$, $\ell \geq 5$}  \label{subsubsec:c:34l}
	
	
	%\nts{SUBDIVISION of 344 on B-line}
	
	
	
	%>>> S: Cons {con:c:thet34k}
	\begin{cons}\label{cons:c:thet34k} 
		Begin with a copy of the graph $\cp{G_{3,4,4}}$ from Construction \ref{con:c:thet344} for $\thet{3,4,4}$.  Subdivide the edge $w_1w_6$ $\ell-4$ times (for $\ell \geq 5$), adding the new vertices $w_7$, $w_8, \dots, w_{\ell+2}$.  Join $w_7, w_8, \dots, w_{\ell+2}$ to $w_0$, so that $w_0,w_1,\dots,w_{\ell+2}$ forms a wheel.  Call this graph $\cp{G_{3,4,\ell}}$. 
	\end{cons}
	% <<< E: Cons {con:c:thet34k}
	
	%>>>S: Lemma {lem:c:thet34k}
	\begin{lemma}\label{lem:c:thet34k}
		If $\cp{G_{3,4,\ell}}$ is the graph constructed by Construction \ref{cons:c:thet34k}, then $\ig{G_{3,4,\ell}} = \thet{3,4,\ell}$ for $\ell \geq 5$.
	\end{lemma}
	%>>>E: Lemma {lem:c:thet34k}
	
	%______________________  SUBSUB SECTION:  theta (3,5,5) ______________________
	\subsubsection{Construction of $\cp{G}$ for $\thet{3,5,5}$}  \label{subsubsec:c:355}
	
	
	%\nts{SUBDIVISION of 335 twice on B-line}
	
	
	
	%>>> S: Cons {con:c:thet355}
	\begin{cons} \label{con:c:thet355}
		Refer to  Figure \ref{fig:c:thet355}.	
		Begin with a copy of the graph $\cp{G_{3,3,5}}$ from  Construction \ref{con:c:thet335}.
		Subdivide the edge $w_1w_6$ twice, adding the new vertices $w_7$ and  $w_8$.  Join $w_7$ and $w_8$ to $w_0$, so that $w_0,w_1,\dots,w_{8}$ forms a wheel.  Call this graph $\cp{G_{3,5,5}}$.
	\end{cons}
	% <<< E: Cons {con:c:thet355}
	
	
	%>>> S: Figure {fig:c:thet355}	
	\begin{figure}[H] \centering	
		\begin{tikzpicture}	
			%---------- REF-------------------
			\node (cent) at (0,0) {};
			\path (cent) ++(0:40 mm) node (rcent) {};	
			
			%----Left Wheel
			\node[std,label={0:$w_0$}] (w0) at (cent) {};
			
			\foreach \i in {1,2,3,4,5}
			{
				\node[std,label={150-\i*60: $w_{\i}$}] (w\i) at (150-\i*60:2cm) {}; 
				\draw[thick] (w0)--(w\i);        
			}
			
			\foreach \i in {6,7,8}
			{
				\node[std,label={360-\i*30: $w_{\i}$}] (w\i) at (360-\i*30:2cm) {}; 
				\draw[thick] (w0)--(w\i);   
			}
			
			\draw[thick] (w1)--(w2)--(w3)--(w4)--(w5)--(w6)--(w7)--(w8)--(w1);
			
			%---- V-Path	
			%\foreach \i in {1,2}
			\foreach \i/\j in {1/1.7, 2/-0.5}
			{
				%\node[bblue,label={[blue]0: $v_\i$}] (v\i) at (4,3.5-2*\i) {};
				\path (rcent) ++(90:\j cm) node [bblue,label={[blue]0: $v_\i$}] (v\i) {};
				%\path (v\i) ++(0:4 mm) node (Lv\i) {\color{blue} $v_{\i}$};		
			}	
			
			\draw[bline] (w1)--(v1)--(v2)--(w2)--(v1);
			
			%---- V2 to W4-Path
			\draw[bline] (v2) to[out=230,in=0] (w4);  	
			
			%-- V2 to W5
			%\node[std] (v2w5A) at (-60:3.5cm) {};
			%\node[std] (v2w5B) at  (-60:4cm){};
			\draw[bline] (v2) to [out=260,in=15]  (-60:3.5cm) to [out=195,in=270] (w5);  
			
			%--- V1 to W5		
			%\node[std] (v1w5A) at (-15:5cm){};
			%\node[std] (v1w5B) at  (-60:4cm){};
			\draw[bline] (v1) to [out=-45, in=70]  (-15:5cm) to [out=250,in=15]  (-60:4cm) to[out=195,in=260] (w5); 
			
			%----- Internal Red Labellings	
			\path (w0) ++(60:12 mm) node [color=red] (X) {$X$};			
			\path (w0) ++(60:-12 mm) node [color=red] (Y) {$Y$};
			
			%-- A's
			\path (w0) ++(0:12 mm) node [color=red] (A1) {$A_1$};	
			\path (w0) ++(-60:12 mm) node [color=red] (A2) {$A_2$};
			
			%-- B's	
			\path (w0) ++(105:12 mm) node [color=red] (B1) {$B_1$};	
			\path (w0) ++(135:12 mm) node [color=red] (B2) {$B_2$};
			\path (w0) ++(165:12 mm) node [color=red] (B2) {$B_3$};
			\path (w0) ++(195:12 mm) node [color=red] (B2) {$B_4$};
			
			%-- D's
			\path (w2) ++(100:5 mm) node [color=red] (D1) {$D_1$};	
			\path (w2) ++(-10:15mm) node [color=red] (D2) {$D_2$};	
			\path (v2) ++(280:10 mm) node [color=red] (D3) {$D_3$};
			\path (w4) ++(-25:15 mm) node [color=red] (D4) {$D_4$};	
			
		\end{tikzpicture} 
		\caption{A graph $\cp{G_{3,5,5}}$ from Construction \ref{con:c:thet355} such that $\ig{G_{3,5,5}} = \thet{3,5,5}$.}
		\label{fig:c:thet355}
	\end{figure}		
	% <<< E: Figure {fig:c:thet355}
	
	
	%>>>S: Lemma {lem:c:thet355}
	\begin{lemma}\label{lem:c:thet355}
		If $\cp{G_{3,5,5}}$ is the graph constructed by Construction \ref{con:c:thet355}, then $\ig{G_{3,5,5}} = \ag{G_{3,5,5}} =\thet{3,5,5}$.
	\end{lemma}
	%>>>E: Lemma {lem:c:thet355}
	
	
	
	Notice that subdividing only once in Construction \ref{con:c:thet355} (adding only $w_7$ and not $w_8$) gives an alternative  (planar) construction for $\thet{3,4,5}$.
	
	
	
	
	
	
	
	
	
	
	
	
	
	
	
	
	
	%......................  SUB SECTION:  theta (j,k,l) .........................
	\subsection{$\thet{j,k,\ell}$ for $4 \leq j \leq k \leq \ell$ and $3 \leq j \leq k \leq \ell$, $\ell \geq 6$}  \label{subsec:c:jkl}
	
	
	
	
	%______________________  SUBSUB SECTION:  theta (4,k,l) ______________________
	\subsubsection{Construction of $\cp{G}$ for $\thet{j,k,\ell}$ for   $4 \leq j \leq k \leq \ell \leq 5$}  \label{subsubsec:c:4kl}
	
	
	
	
	%>>> S: Cons {con:c:thet444}
	\begin{cons} \label{con:c:thet444}
		Refer to  Figure \ref{fig:c:thet444}.	
		Begin with a copy of the graph $\cp{G_{3,4,4}}$ from  Construction \ref{con:c:thet344}.
		Subdivide the edge $u_1w_3$, adding the new vertex $u_2$.  Join $u_2$ to $w_0$, so that
		\begin{align*}
			w_0,w_1,w_2,u_1,u_2,w_3,\dots,w_{6}
		\end{align*}
		forms a wheel.  Call this graph $\cp{G_{4,4,4}}$.
	\end{cons}
	% <<< E: Cons {con:c:thet444}
	
	
	
	
	%>>> S: Figure {fig:c:thet444}
	\begin{figure}[H] \centering	
		\begin{tikzpicture}	
			
			%---------- REF-------------------
			\node (cent) at (0,0) {};
			\path (cent) ++(0:50mm) node (rcent) {};	
			
			%----Left Wheel
			\node[std] (w0) at (cent) {};
			\path (w0) ++(55:6 mm) node (Lw0) {$w_0$};	
			
			\foreach \i in {1,2,3,4,5,6}
			{
				\node[std,label={150-\i*60: $w_{\i}$}] (w\i) at (150-\i*60:2cm) {}; 
				\draw[thick] (w0)--(w\i);        
			}
			
			\node[std,label={0: $u_1$}] (u1) at (10:2cm) {}; 
			\draw[thick] (w0)--(u1);
			
			
			\node[std,label={0: $u_2$}] (u2) at (-10:2cm) {}; 
			\draw[thick] (w0)--(u2);
			
			
			\draw[thick] (w1)--(w2)--(u1)--(u2)--(w3)--(w4)--(w5)--(w6)--(w1);
			
			%---- V-Path
			
			\path (rcent) ++(90:0 cm) node [bblue] (v) {};
			\path (v) ++(0:4 mm) node (Lv) {\color{blue} $v$};	
			
			\foreach \i / \j in {2,3}
			\draw[bline] (v)--(w\i);	
			
			%--- W1 to V
			\draw[bline] (w1)   to[out=0,in=100] (v); 	
			%--- W4 to V
			\draw[bline] (w4)   to[out=0,in=-100] (v); 
			%--- W4 to W1
			\draw[bline] (w4) to[out=-15, in=270]  (0:6.5cm)  to[out=90,in=15] (w1); 	
			
			
			%\path (w0) ++(60:-3 cm) node [bred, label={[red]180:$z'$}] (z1) {};
			\path (w0) ++(180:4 cm) node [bzed, label={[zelim]180:$\boldsymbol{z'}$}] (z1) {};
			%	\draw[rline] (z1)--(w0);
			\draw[zline] (w0)   to[out=165,in=15] (z1); 
			\draw[zline] (w1)   to[out=180,in=80] (z1); 		
			\draw[zline] (w4)   to[out=180,in=-80] (z1); 
			
			
			%----- Internal Red Labellings	
			\path (w0) ++(60:12 mm) node [color=red] (X) {$X$};	
			\path (w0) ++(20:15 mm) node [color=red] (A1) {$A_1$};	
			\path (w0) ++(0:15 mm) node [color=red] (A2) {$A_2$};	
			\path (w0) ++(-20:15 mm) node [color=red] (A3) {$A_3$};	
			\path (w0) ++(-60:12 mm) node [color=red] (Y) {$Y$};	
			
			%-- B's	
			\foreach \i/\c in {1/120,2/180,3/240} {				
				\path (w0) ++(\c:12 mm) node [color=red] (B\i) {$B_{\i}$};	
			}
			
			%-- D's
			\path (w2) ++(05:15 mm) node [color=red] (D1) {$D_1$};	
			\path (v) ++(60:8 mm) node [color=red] (D2) {$D_2$};	
			\path (w3) ++(-05:15 mm) node [color=red] (D3) {$D_3$};
			
		\end{tikzpicture} 
		\caption{The graph \; $\cp{G_{4,4,4}}$ \; from Construction \ref{con:c:thet444} such that $\ig{G_{4,4,4}}$ $=$ $\thet{4,4,4}$.}
		\label{fig:c:thet444}
	\end{figure}		
	% <<< E: Figure {fig:c:thet444}
	
	
	%>>>S: Lemma {lem:c:thet444}
	\begin{lemma}\label{lem:c:thet444}
		If $\cp{G_{4,4,4}}$ is the graph constructed by Construction \ref{con:c:thet444}, then $\ig{G_{4,4,4}} = \thet{4,4,4}$.
	\end{lemma}
	%>>>E: Lemma {lem:c:thet444}
	
	
	
	
	
	
	%>>> S: Cons {con:c:thetjk5}
	\begin{cons} \label{con:c:thetjk5}
		%Refer to  Figure \ref{fig:c:thetjk5}
		Begin with a copy of the graph $\cp{G_{3,3,5}}$ from  Construction \ref{con:c:thet335}.
		For $k=4$ subdivide the edge $w_1w_6$ once, adding the vertex $w_7$; for $k=5$, subdivide a second time, adding the vertex $w_8$. 
		For $j=4$, subdivide the edge $w_2w_3$, adding the vertex $u_1$; for $j=5$ ($j \leq k$), subdivide a second time, adding the vertex $u_2$.  Connect all new vertices to $w_0$ to form a wheel.
		Call this graph $\cp{G_{j,k,5}}$ for $4 \leq j \leq k \leq 5$.
	\end{cons}
	% <<< E: Cons {con:c:thetjk5}
	
	An example of the construction of $\cp{G_{5,5,5}}$ is given in Figure \ref{fig:c:thet555} below.
	
	
	
	%>>> S: Figure {fig:c:thet555}	
	\begin{figure}[H] \centering	
		\begin{tikzpicture}	
			%---------- REF-------------------
			\node (cent) at (0,0) {};
			\path (cent) ++(0:40 mm) node (rcent) {};	
			
			%----Left Wheel
			\node[std,label={87:$w_0$}] (w0) at (cent) {};
			
			\foreach \i in {1,2,4,5}
			{
				\node[std,label={150-\i*60: $w_{\i}$}] (w\i) at (150-\i*60:2cm) {}; 
				\draw[thick] (w0)--(w\i);        
			}
			
			\foreach \i in {6,7,8}
			{
				\node[std,label={360-\i*30: $w_{\i}$}] (w\i) at (360-\i*30:2cm) {}; 
				\draw[thick] (w0)--(w\i);   
			}
			
			\foreach \i in {1,2}
			{
				\node[std,label={0: $u_{\i}$}] (u\i) at (30-\i*30:2cm) {}; 
				\draw[thick] (w0)--(u\i);   
			}		
			
			\node[std,label={0: $w_{3}$}] (w3) at (300:2cm) {}; 
			\draw[thick] (w0)--(w3);   
			
			
			\draw[thick] (w1)--(w2)--(u1)--(u2)--(w3)--(w4)--(w5)--(w6)--(w7)--(w8)--(w1);
			
			%---- V-Path	
			\foreach \i/\j in {1/1.7, 2/-0.5}
			{
				\path (rcent) ++(90:\j cm) node [bblue,label={[blue]0: $v_\i$}] (v\i) {};
			}	
			
			\draw[bline] (w1)--(v1)--(v2)--(w2)--(v1);
			
			%---- V2 to W4-Path
			\draw[bline] (v2) to[out=230,in=0] (w4);  	
			
			%-- V2 to W5	
			\draw[bline] (v2) to [out=260,in=15]  (-60:3.5cm) to [out=195,in=270] (w5);  
			
			%--- V1 to W5		
			\draw[bline] (v1) to [out=-45, in=70]  (-15:5cm) to [out=250,in=15]  (-60:4cm) to[out=195,in=260] (w5); 
			
			%----- Internal Red Labellings	
			\path (w0) ++(60:12 mm) node [color=red] (X) {$X$};			
			\path (w0) ++(60:-12 mm) node [color=red] (Y) {$Y$};
			
			%-- A's
			\path (w0) ++(15:15 mm) node [color=red] (A1) {$A_1$};	
			\path (w0) ++(-15:15 mm) node [color=red] (A2) {$A_2$};
			\path (w0) ++(-45:15 mm) node [color=red] (A1) {$A_3$};	
			\path (w0) ++(-75:15 mm) node [color=red] (A2) {$A_4$};
			
			%-- B's	
			\path (w0) ++(105:15 mm) node [color=red] (B1) {$B_1$};	
			\path (w0) ++(135:15 mm) node [color=red] (B2) {$B_2$};
			\path (w0) ++(165:15 mm) node [color=red] (B2) {$B_3$};
			\path (w0) ++(195:15 mm) node [color=red] (B2) {$B_4$};
			
			%-- D's
			\path (w2) ++(100:5 mm) node [color=red] (D1) {$D_1$};	
			\path (w2) ++(-10:15mm) node [color=red] (D2) {$D_2$};	
			\path (v2) ++(280:10 mm) node [color=red] (D3) {$D_3$};
			\path (w4) ++(-25:15 mm) node [color=red] (D4) {$D_4$};	
			
		\end{tikzpicture} 
		\caption{A graph $\cp{G_{5,5,5}}$ from Construction \ref{con:c:thet355} such that $\ig{G_{5,5,5}} = \thet{5,5,5}$.}
		\label{fig:c:thet555}
	\end{figure}		
	% <<< E: Figure {fig:c:thet555}
	
	
	
	
	%>>>S: Lemma {lem:c:thetjk5}
	\begin{lemma}\label{lem:c:thetjk5}
		If $\cp{G_{j,k,5}}$ is the graph constructed by Construction \ref{con:c:thetjk5}, then $\ig{G_{j,k,5}} = \ag{G_{j,k,5}} =\thet{j,k,5}$ for $4 \leq j \leq k \leq 5$.
	\end{lemma}
	%>>>E: Lemma {lem:c:thetjk5}
	
	
	
	
	
	
	
	
	
	
	
	
	%______________________  SUBSUB SECTION:  theta (j,k,l) ______________________
	\subsubsection{Construction of $\cp{G}$ for $\thet{j,k,\ell}$ for   $3 \leq j \leq k \leq \ell$, $\ell \geq 6$.}  \label{subsubsec:c:jkl}
	
	
	
	
	%>>> S: Cons {con:c:thetijk}
	\begin{cons} \label{con:c:thetijk}
		%Refer to  Figure \ref{fig:c:thetijk}.	
		Begin with a copy of the graph $\cp{G_{3,3,\ell}}$ from  Construction \ref{con:c:thet33k} for $\thet{3,5,\ell}$ for $\ell \geq 6$.
		For $3 \leq k \leq \ell$, subdivide the edge $w_1w_6$ $k-3$ times, adding the new vertices $w_7, w_8, \dots, w_{k+3}$.  Join each of  $w_7, w_8, \dots, w_{k+3}$  to $w_0$, so that $w_0,w_1,\dots,w_{k+3}$ forms a wheel. 
		Then, for  $3 \leq j \leq k$,  subdivide the edge $w_2w_3$ $j-3$ times, adding the new vertices $u_1, u_2, \dots, u_{j-3}$.  Again, join each  of $u_1, u_2, \dots, u_{j-3}$ to $w_0$ to form a wheel.
		Call this graph $\cp{G_{j,k,\ell}}$ for $3 \leq j \leq k \leq \ell$ and $\ell \geq 6$.
	\end{cons}
	% <<< E: Cons {con:c:thetijk}
	
	
	
	
	
	%>>> S: Figure {fig:c:thet33k}	
	%$\thet{i,k,k}$
	\begin{figure}[H] \centering	
		\begin{tikzpicture}	
			%---------- REF-------------------
			\node (cent) at (0,0) {};
			\path (cent) ++(0:45 mm) node (rcent) {};	
			
			%----Left Wheel
			\node[std,label={85:$w_0$}] (w0) at (cent) {};
			
			%	\foreach \i / \j in {1/1,  4/{k+1},5/{k+2}}
			\foreach \i / \j / \r in {1/1/2,  {4b}/{3}/7, 4/{4}/8, 5/{5}/10, {5b}/{6}/11, {6b}/{k+3}/13}
			{
				\node[std,label={150-\r*30: $w_{\j}$}] (w\i) at (150-\r*30:2cm) {}; 
				\draw[thick] (w0)--(w\i);        
			}
			\node[std,label={0: $w_{2}$}] (w2) at (30:2cm) {}; 
			\draw[thick] (w0)--(w2);       			
			
			%-- cycle insert nodes
			\node[std,label={0: $u_1$}] (w3) at (0:2cm) {}; 
			\draw[thick] (w0)--(w3);     
					
			\node[std,label={-15: $u_{2}$}] (u2) at (-15:2cm) {}; 
			\draw[dashed] (w0)--(u2);        
			
			\node[std,label={-45: $u_{j-3}$}] (uj) at (-45:2cm) {};
			\draw[dashed] (w0)--(uj); 
			
			
			\node[std,label={165: $w_{7}$}] (q1) at (165:2cm) {}; 
			\draw[dashed] (w0)--(q1);        
			
			\node[std,label={135: $w_{k+2}$}] (qk) at (135:2cm) {};
			\draw[dashed] (w0)--(qk); 
						
			
			\draw[thick] (w5)--(w5b)--(q1)--(qk)--(w6b)--(w1);
			
			\draw[thick] (w2)--(w3)--(u2)--(uj)--(w4b)--(w4);
			
			
			\draw[thick] (w1)--(w2);
			\draw[thick] (w4)--(w5);
			
			
			%\draw[thick] (w1)--(w2)--(w3)--(w4)--(w5)--(w6)--(w1);
			
			%---- V-Path	
			\foreach \i / \j  / \k in {1/3/1, 2/2/2, 3/-1/{\ell-5}, 4/-2.2/{\ell-4}, 5/-3.4/{\ell-3}}
			{	
				\path (rcent) ++(90:\j cm) node [bblue] (v\i) {};
				\path (v\i) ++(0:5 mm) node (Lv\i) {\color{blue} $v_{\k}$};
				%\node[bblue,label={0: $v_\i$}] (v\i) at (4,2-\i) {};
			}
			
			\draw[bline] (v1)--(v2);
			\draw[bline] (v3)--(v4)--(v5);
			
			\draw[bline,dashed] (v2)--(v3);
			
			%---- V3 to V5 
			\path (v4) ++(180:1.3 cm) coordinate  (c1) {};
			\draw[bline] (v3) to[out=190, in=90]  (c1) to [out=270,in=170] (v5);  
			
			%---- V1 to W1 W2
			\draw[bline] (w1)--(v1)--(w2);	
			
			%-- V5 to W4
			\draw[bline] (v5) to[out=180,in=-45] (w4);  
			
			%-- V5 to W5
			\draw[bline] (v5) to[out=200,in=290] (w5);  
			
			%-- V4 to W5
			\draw[bline] (v4) to[out=-35, in=25]  (4.6cm,-4.2cm) to [out=205,in=270] (w5);  	
			%	\node[std] at (4.6cm,-4.2cm) {};
			
			%-- W1 to V2  
			\draw[bline] (w1) to[out=30, in=170]  (35:6.1cm) to [out=350,in=45]  (v2); 
			%	\node[std] at (35:6.2cm)  {};
			
			%-- W1 to  V3 
			\draw[bline] (w1) to[out=38, in=170]  (35:6.4cm) to [out=350,in=45]  (v3); 
			%\node[std] at (30:5.8cm) {};
			
			%-- W1 to V4
			\draw[bline] (w1) to[out=46, in=170]  (35:6.6cm) to [out=350,in=40]  (v4); 
			%\node[std] at (30:5.8cm) {};
			
			%-- W1 to Dashed T1 T2
			\path (rcent) ++(90:1 cm) coordinate  (t1) {};
			\draw[bline, dashed] (w1) to[out=32, in=170]   (35:6.2cm)  to [out=350,in=45]  (t1); 
			
			\path (rcent) ++(90:0 cm) coordinate  (t2) {};
			\draw[bline, dashed] (w1) to[out=34, in=170]  (35:6.3cm) to [out=350,in=45]  (t2); 
			
			%----- Internal Red Labellings	
			\path (w0) ++(60:12 mm) node [color=red] (X) {$X$};			
			\path (w0) ++(60:-12 mm) node [color=red] (Y) {$Y$};
			
			%-- A's
			\path (w0) ++(15:12 mm) node [color=red] (A1) {$A_1$};	
			%\path (w0) ++(60:-12 mm) node [color=red] (Y) {$Y$};
			\path (w0) ++(-75:15 mm) node [color=red] (A2) {$A_{j-1}$};
			
			%-- B's	
			\path (w0) ++(-75:-12 mm) node [color=red] (B1) {$B_1$};	
			\path (w0) ++(195:14 mm) node [color=red] (B2) {$B_{k-1}$};
			
			
			%-- D's
			\path (w2) ++(105:8 mm) node [color=red] (D1) {$D_1$};	
			\path (v1) ++(170:9 mm) node [color=red] (D2) {$D_2$};	
			\path (v4) ++(160:7 mm) node [color=red] (D3) {$D_{\ell-3}$};
			\path (w4) ++(-55:25mm) node [color=red] (D4) {$D_{\ell-2}$};
			\path (w4) ++(-50:15 mm) node [color=red] (D5) {$D_{\ell-1}$};		
			
			
			%----- DASH SPLITS
			\coordinate  (c1) at ($(u2)!0.33!(uj)$) {};
			\coordinate  (c2) at ($(u2)!0.66!(uj)$) {};
			
			\draw [dashed] (c2)--(w0);
			\draw [dashed] (c1)--(w0);		
			
			%---	
			\coordinate  (d1) at ($(q1)!0.33!(qk)$) {};
			\coordinate  (d2) at ($(q1)!0.66!(qk)$) {};
			
			\draw [dashed] (d2)--(w0);
			\draw [dashed] (d1)--(w0);				
			
			
		\end{tikzpicture} 
		\caption{The graph $\cp{G_{j,k,\ell}}$ from Construction \ref{con:c:thetijk} such that $\ig{G_{j,k,\ell}} = \thet{j,k,\ell}$ for $3\leq j \leq k \leq \ell$ and $\ell \geq 6$.}
		\label{fig:c:thetijk}
	\end{figure}	
	% <<< E: Figure {fig:c:thetijk}
	
	
	
	%>>>S: Lemma {lem:c:thetjkl}
	\begin{lemma}\label{lem:c:thetjkl}
		If  $\cp{G_{j,k,\ell}}$ for $3 \leq j \leq k \leq \ell$ and $\ell \geq 6$ is the graph constructed by Construction \ref{con:c:thetijk}, then $\ig{G_{j,k,\ell}} = \ag{G_{j,k,\ell}} = \thet{j,k,\ell}$.
	\end{lemma}
	%>>>E: Lemma {lem:c:thetjkl}
	
	
	%\nts{Note here that having shown all of the above lemmas, then we have proven one direction of THeorem 8 (if it is one of these graphs, then it is an i-graph)}
	
	
	
	The lemmas above imply the sufficiency of Theorem \ref{thm:c:thetas}: if a theta graph is not one of seven exceptions listed, then it is an $i$-graph.  In the next section, we complete the proof by examining the exception cases.
	
	
	%From the Lemmas of Section \ref{sec:c:thetaComp}, we have shown one direction of the proof of Theorem \ref{thm:c:thetas}; that is, if a theta graph is one of those constructed in Section \ref{sec:c:thetaComp}, then it is and $i$-graph.  In the next section, we complete the proof by examining the exception case: the seven theta graphs that are not $i$-graphs.
	
	%@@@@@@@@@@@@@@@@@@@@@@@@@@@@@@@@@@@@@@ SPLIT TO NON IGRAPHS @@@@@@@@@@@@@@@@@@@@@@@
	
	
	
	
	
	
	
	
	
	
	
	
	
	%---------------------------------------SECTION: Graph Complements ----------------
	\section{Theta Graphs That Are Not $i$-Graphs} \label{sec:c:nonthetas}
	
	
	In this section we show that $\thet{2,2,4}$, $\thet{2,3,3}$, $\thet{2,3,4}$, and $\thet{3,3,3}$ are not $i$-graphs; together with Propositions \ref{prop:i:diamond},  \ref{prop:i:K23}, and \ref{prop:i:kappa}, this completes the proof of Theorem \ref{thm:c:thetas}.
	
	
	
	%______________________  SUBSUB SECTION:  theta (2,2,4) ______________________
	\subsubsection{$\thet{2,2,4}$ Is Not an $i$-Graph}  \label{subsubsec:c:224}
	
	%>>>S: Prop {prop:c:thet224}
	\begin{prop}	\label{prop:c:thet224}
		The graph $\thet{2,2,4}$ is not $i$-graph realizable.
	\end{prop}
	
	%~~~~~~~~~~~~FIG START~~~~~~~~~~~
	\begin{figure}[H]
		\centering
		\begin{tikzpicture}%[line width=0.3mm,le=1]			
			
			%---------- REF-------------------
			\node(cent) at (0,0) {};			
			\path (cent) ++(0:60 mm) node(rcent)  {};		
			
			%----- left C4 ----------	
			\foreach \i/\j/\c in {x/X/90,y/Y/-90} {				
				\path (cent) ++(\c:20 mm) node [std] (v\i) {};
				%\path (v\i) ++(\c:5 mm) node (Lv\i) {${\j}$};
			}
			
			\path (vx) ++(90:5 mm) node (Lvx) {$\color{blue}X=\{x_1,x_2,x_3,\dots, x_k\}$};
			\path (vy) ++(-90:5 mm) node (Lvy) {$\color{blue}Y=\{y_1,y_2,x_3,\dots, x_k\}$};
			
			\foreach \i/\j/\c in {a/A/180,b/B/0} {				
				\path (cent) ++(\c:8 mm) node [std] (v\i) {};
				%\path (v\i) ++(\c:5 mm) node (Lv\i) {${\j}$};
			}
			
			
			%\path (va) ++(180:25 mm) node (Lva) {$A=\{y_1,x_2,x_3,\dots, x_k\}$};
			\path (va) ++(175:20 mm) node (Lvau) {$\color{blue}A=$};
			\path (va) ++(185:20 mm) node (Lvad) {$\color{blue}\{y_1,x_2,x_3,\dots, x_k\}$};
			\path (vb) ++(0:25 mm) node (Lvb) {\color{blue}$B=\{x_1,y_2,x_3,\dots, x_k\}$};
			
			
			\draw[thick] (vx)--(va)--(vy)--(vb)--(vx)--cycle;				
			
			%----- right P4 ----------
			\foreach \i/\c in {1/15,2/0,3/-15} {				
				\path (rcent) ++(90:\c mm) node [std] (c\i) {};
				%\path (c\i) ++(0:5 mm) node (Lc\i) {$C_{\i}$};
			}
			
			\draw[thick] (vx)--(c1)--(c2)--(c3)--(vy);
			
			%\path (cent) ++(-90: 20mm) node (HL)  {$\mathcal{H}$};
			
			\path (c1) ++(0:25 mm) node (Lvc1) {\color{blue}$C_1=\{x_1,x_2,y_3,\dots, x_k\}$};
			%\path (c2) ++(0:25 mm) node (Lvc2) {\color{red}$C_2=\{x_1,x_2,y_3,\dots, x_k\}$};
			%\path (c3) ++(0:25 mm) node (Lvc3) {\color{red}$C_3=\{x_1,x_2,y_3,\dots, x_k\}$};
			\path (c2) ++(0:8 mm) node (Lvc2) {\color{red}$C_2=$};
			\path (Lvc2) ++(0:20 mm) node (Lvc2C) {\color{red} OR };
			\path (Lvc2C) ++(90:5 mm) node (Lvc2A) {\color{red} $\{y_1,x_2,y_3,\dots, x_k\}$};
			\path (Lvc2C) ++(-90:5 mm) node (Lvc2B) {\color{red} $\{x_1,y_2,y_3,\dots, x_k\}$};
			
			\path (c3) ++(0:5 mm) node (Lvc3) {\color{red}$C_3$};
			
			
		\end{tikzpicture}				
		\caption{$H=\thet{2,2,4}$ non-construction.}
		\label{fig:c:thet224}%
	\end{figure}
	
	
	\begin{proof}
		Suppose to the contrary that $\thet{2,2,4}$ is realizable as an $i$-graph, and  that $H = \thet{2,2,4} \cong \ig{G}$ for some graph $G$.  Label the vertices of $H$ as in Figure \ref{fig:c:thet224}.
		
		
		From Proposition \ref{fig:i:C4struct}, and similarly to the proofs of Propositions \ref{prop:i:K23} and \ref{prop:i:kappa}, the composition of the following $i$-sets of $G$ are immediate: 
		
		\begin{align*}
		X&=\{x_1,x_2,x_3,\dots,x_k\},& Y=\{y_1,y_2,x_3,\dots,x_k\}, \\
		A&=\{y_1,x_2,x_3,\dots,x_k\},& B=\{x_1,y_2,x_3,\dots,x_k\}, \\
		C_1&=\{x_1,x_2,y_3,\dots,x_k\} 
		\end{align*}
		
		\noindent
		where $k \geq 3$ and $y_1, y_2, y_3$ are three distinct vertices in $G-X$.  These sets are illustrated in blue in Figure \ref{fig:c:thet224}.  This leaves only the composition of $C_2$ and $C_3$ (in red) to be determined.  
		As we construct $C_2$, notice first that $y_3\in C_2$; otherwise, if say some other $z \in C_2$ so that $\edge{C_1,{y_3},z,C_2}$, then $\edge{X,x_3,z,C_2}$, and $XC_2 \in E(H)$.  Thus, a token on one of $\{x_1,x_2,x_4,\dots,x_k\}$ moves in the transition from $C_1$ to $C_2$.  We consider three cases.
		
		\textbf{Case 1:}  The token on $x_1$ moves.  If $\edge{C_1,{x_1},z,C_2}$ for some $z \notin \{y_1,y_2\}$, then $|C_2 \cap Y |=3$, contradicting the distance requirement between $i$-sets from Observation \ref{obs:i:dist}.    Moreover, from the composition of  $B$, $x_1 \not\sim y_2$, and so  $\edge{C_1,{x_1},y_1,C_2}$, so that $C_2 = \{y_1,x_2,y_3,\dots, x_k\}$.  However, since $x_3 \sim y_3$, we have that $\edge{A,x_3,y_3,C_2}$, so that $AC_2 \in E(G)$, a contradiction.
		
		%Again from the distance requirement, in the two-step transition from $C_2$ to $Y$, tokens are removed from $x_2$ and $y_3$, and tokens are moved to $y_2$ and $x_3$.  If $\edge{C_2,x_2,y_2,C_3}$ so that $C_3=\{y_1,y_2,y_3,\dots, x_k\}$,
		
		\textbf{Case 2:}  The token on $x_2$ moves.  An argument similar to Case 1 constructs $C_2 = \{x_1,y_2,y_3,\dots,x_k\}$, with $\edge{B,x_3,y_3,C_2}$, resulting in the contradiction $BC_2\in E(G)$.
		
		
		\textbf{Case 3:}  The token on $x_i$ for some $i\in \{4,5,\dots,k\}$ moves.	From the compositions of $X$ and $Y$, $x_i$ is not adjacent to any of $\{x_3,y_1,y_2\}$, so the token at $x_i$ moves to some other vertex, say $z$, so that $\edge{C_1,x_i,z,C_2}$ and $\{x_1,x_2,y_3,z\} \subseteq C_2$.  This again contradicts the distance requirement of Observation \ref{obs:i:dist} as $|C_2 \cap Y| = 4$.
		
		
		\noindent In all cases, we fail to construct a graph $G$ with $\ig{G} \cong \thet{2,2,4}$ and so conclude that no such graph exists.		
	\end{proof}
	
	% <<< E: Prop {prop:c:thet224}
	
	
	\newpage
	
	%______________________  SUBSUB SECTION:  theta (2,3,3) ______________________
	\subsubsection{ $\thet{2,3,3}$ Is Not an $i$-Graph} \label{subsubsec:c:233}
	
	
	
	
	%>>>S: Prop {prop:c:thet233}
	\begin{prop}	\label{prop:c:thet233}
		The graph $\thet{2,3,3}$ is not $i$-graph realizable.
	\end{prop}
	
	%~~~~~~~~~~~~FIG START~~~~~~~~~~~
	\begin{figure}[H]
		\centering
		\begin{tikzpicture}%[line width=0.3mm,scale=1]			
			
			%---------- REF-------------------
			\node(cent) at (0,0) {};			
			\path (cent) ++(0:60 mm) node(rcent)  {};		
			
			%----- left C4 ----------	
			\foreach \i/\j/\c in {x/X/90,y/Y/-90} {				
				\path (cent) ++(\c:20 mm) node [std] (v\i) {};
				%\path (v\i) ++(\c:5 mm) node (Lv\i) {${\j}$};
			}
			
			\path (vx) ++(90:5 mm) node (Lvx) {$\color{blue}X=\{x_1,x_2,x_3,\dots, x_k\}$};
			\path (vy) ++(-90:5 mm) node (Lvy) {$\color{red}Y$};
			%\path (vy) ++(-90:5 mm) node (Lvy) {$\color{blue}Y=\{y_1,y_2,x_3,\dots, x_k\}$};
			
			\path (cent) ++(180:8 mm) node [std] (va) {};	
			
			%\path (va) ++(180:25 mm) node (Lva) {$A=\{y_1,x_2,x_3,\dots, x_k\}$};
			\path (va) ++(175:20 mm) node (Lvau) {$\color{blue}A=$};
			\path (va) ++(185:20 mm) node (Lvad) {$\color{blue}\{y_1,x_2,x_3,\dots, x_k\}$};
			
			\draw[thick] (vx)--(va)--(vy);		
			
			
			%------------ B Pathway
			
			\path (cent) ++(0:8 mm) node (vb) {};
			
			\path (vb) ++(90:8 mm) node [std] (vb1) {};
			\path (vb1) ++(0:25 mm) node (Lvb) {\color{blue}$B_1=\{x_1,y_2,x_3,\dots, x_k\}$};
			
			\path (vb) ++(90:-8 mm) node [std] (vb2) {};
			\path (vb2) ++(0:5 mm) node (Lvb) {\color{red}$B_2$};	
			%  \path (vb2) ++(0:25 mm) node (Lvb) {\color{blue}$B_2=\{x_1,y_2,x_3,\dots, x_k\}$};	
			
			\draw[thick](vx)--(vb1)--(vb2)--(vy); 
			
			
			%----- right P4 ----------
			\foreach \i/\c in {1/14,2/-14} {				
				\path (rcent) ++(90:\c mm) node [std] (c\i) {};
				%\path (c\i) ++(0:5 mm) node (Lc\i) {$C_{\i}$};
			}
			
			\draw[thick] (vx)--(c1)--(c2)--(vy);
			
			%\path (cent) ++(-90: 20mm) node (HL)  {$\mathcal{H}$};
			
			\path (c1) ++(0:25 mm) node (Lvc1) {\color{blue}$C_1=\{x_1,x_2,y_3,\dots, x_k\}$};
			%\path (c2) ++(0:25 mm) node (Lvc2) {\color{red}$C_2=\{x_1,x_2,y_3,\dots, x_k\}$};
			%\path (c3) ++(0:25 mm) node (Lvc3) {\color{red}$C_3=\{x_1,x_2,y_3,\dots, x_k\}$};
			\path (c2) ++(0:5 mm) node (Lvc2) {\color{red}$C_2$};
			%	\path (Lvc2) ++(0:20 mm) node (Lvc2C) {\color{red} OR };
			%	\path (Lvc2C) ++(90:5 mm) node (Lvc2A) {\color{red} $\{y_1,x_2,y_3,\dots, x_k\}$};
			%	\path (Lvc2C) ++(-90:5 mm) node (Lvc2B) {\color{red} $\{x_1,y_2,y_3,\dots, x_k\}$};
			
			%	\path (c3) ++(0:5 mm) node (Lvc3) {\color{red}$C_3$};
			
			
		\end{tikzpicture}				
		\caption{$H=\thet{2,3,3}$ non-construction.}
		\label{fig:c:thet233}%
	\end{figure}
	
	
	\begin{proof}
		To begin, we proceed similarly to the proof of Proposition \ref{prop:c:thet224}: suppose to the contrary that $\thet{2,3,3}$ is realizable as an $i$-graph, and  that $H = \thet{2,3,3} \cong \ig{G}$ for some graph $G$.  Label the vertices of $H$ as in Figure \ref{fig:c:thet233}.  As before, the corresponding $i$-sets in blue are established from previous results, and those in red are yet to be determined.
		Moreover, from the composition of these four blue $i$-sets, we observe that for each $i\in \{1,2,3\}$, $x_i \sim y_j$ if and only if $i=j$. 
		
		
		
		Unlike the construction for $\thet{2,2,4}$, we no longer start with knowledge of the exact composition of $Y$.  	
		We proceed with a series of observations on the contents of the various $i$-sets:
		
		\begin{enumerate}[itemsep=1pt, label=(\roman*)]
			\item $y_1 \in Y$, $y_2\in B_2$, and $y_3 \in C_2$  by three applications of  Proposition \ref{prop:i:C4struct}.  \label{prop:c:thet233:a}
			
			\item $y_1 \not\in B_2$ and $y_1 \not\in C_2$.  If $y_1 \in B_2$, then $\edge{B_1,x_1,y_1,B_2}$ (because $A$ shows that $y_1$ is not adjacent to $x_3,\dots,x_k$) so that $B_2  = \{y_1,y_2,x_3,\dots,x_k\}$, and therefore $\edge{A,x_2,y_2,B_2}$, which is impossible.   Similarly, if $y_1 \in C_2$, then $\edge{A, x_3,y_3,C_2}$, which is also impossible.  \label{prop:c:thet233:b}
			
			\item $y_3 \notin B_2$.  Otherwise, $B_2 = \{x_1,y_2,y_3,x_4,\dots,x_k\}$ and so $\edge{C_1,x_2,y_2,B_2}$.  \label{prop:c:thet233:c}
			
			\item $y_2 \not\in Y$ and $y_3\notin Y$.  If $y_2 \in Y$, then  $Y=\{y_1,y_2,x_3,\dots,x_k\}$ and so $\edge{B_1,x_1,y_1,Y}$.	Likewise, if $y_3 \in Y$ then $\edge{C_1,x_1,y_1,Y}$.  \label{prop:c:thet233:d}
			
		\end{enumerate}
		
		
		
		From \ref{prop:c:thet233:a} and \ref{prop:c:thet233:b}, $y_1 \in Y$ but $y_1 \not\in B_2$, and similarly from \ref{prop:c:thet233:d} $y_2 \in B_2$ but $y_2 \not\in Y$; therefore,
		$\edge{B_2,y_2,y_1,Y}$.   Now, since $x_1 \sim y_1$, and $y_1 \in Y$, we have that $x_1 \not\in Y$. Thus, if $x_1$ were in $B_2$, its token would move in the transition from $B_2$ to $Y$.  However,  have already established that it is the token at $y_2$ that moves, and so $x_1 \not\in B_2$.  We conclude that $\edge{B_1,x_1,z,B_2}$ for some $z \not\in \{y_1,y_3\}$, so that $B_2 = \{z,y_2,x_3,\dots,x_k\}$.  Notice that since $B_2$ is independent, and $x_2 \sim y_2$, it follows that $z \neq x_2$.
		
		Using similar arguments, we determine that $\edge{C_2,y_3,y_1,Y}$, and that $\edge{C_1,x_1, w,C_2}$ for some $w\notin \{y_1,y_2,x_1\}$.  Moreover, $C_2=\{w,x_2,y_3,x_4,\dots,x_k\}$.  Again, note that since $x_3 \sim y_3$, $w \neq x_3$.
		
		From $\edge{B_2,y_2,y_1,Y}$, we have that $Y=\{y_1,z,x_3,x_4\dots,x_k\}$.  However, from $\edge{C_2,y_3,y_1,Y}$, we also have that $Y=\{y_1,x_2,w,x_4,\dots,x_k\}$.  As we have already established that $z \neq x_2$ and $w \neq x_3$, we arrive at two contradicting compositions of $Y$.
		Thus, no such graph $G$ exists, and we conclude that $\thet{2,3,3}$ is not an $i$-graph.
	\end{proof}
	
	% <<< E: Prop {prop:c:thet233}
	
	
	
	%______________________  SUBSUB SECTION:  theta (2,3,4) ______________________
	\subsubsection{ $\thet{2,3,4}$ Is Not an $i$-Graph} \label{subsubsec:c:234}
	
	
	
	
	%>>>S: Prop {prop:c:thet234}
	\begin{prop}	\label{prop:c:thet234}
		The graph $\thet{2,3,4}$ is not $i$-graph realizable.
	\end{prop}
	
	%~~~~~~~~~~~~FIG START~~~~~~~~~~~
	\begin{figure}[H]
		\centering
		\begin{tikzpicture}%[line width=0.3mm,scale=1]			
			
			%---------- REF-------------------
			\node(cent) at (0,0) {};			
			\path (cent) ++(0:60 mm) node(rcent)  {};		
			
			%----- left C4 ----------	
			\foreach \i/\j/\c in {x/X/90,y/Y/-90} {				
				\path (cent) ++(\c:20 mm) node [std] (v\i) {};
				%\path (v\i) ++(\c:5 mm) node (Lv\i) {${\j}$};
			}
			
			\path (vx) ++(90:5 mm) node (Lvx) {$\color{blue}X=\{x_1,x_2,x_3,\dots, x_k\}$};
			\path (vy) ++(-90:5 mm) node (Lvy) {$\color{red}Y = \{y_1,z,x_3,\dots, x_k\}$};
			%\path (vy) ++(-90:5 mm) node (Lvy) {$\color{blue}Y=\{y_1,y_2,x_3,\dots, x_k\}$};
			
			\path (cent) ++(180:8 mm) node [std] (va) {};	
			
			
			
			
			%\path (va) ++(180:25 mm) node (Lva) {$A=\{y_1,x_2,x_3,\dots, x_k\}$};
			\path (va) ++(175:20 mm) node (Lvau) {$\color{blue}A=$};
			\path (va) ++(185:20 mm) node (Lvad) {$\color{blue}\{y_1,x_2,x_3,\dots, x_k\}$};
			
			
			
			\draw[thick] (vx)--(va)--(vy);		
			
			
			%------------ B Pathway
			
			\path (cent) ++(0:8 mm) node (vb) {};
			
			\path (vb) ++(90:8 mm) node [std] (vb1) {};
			\path (vb1) ++(0:25 mm) node (Lvb) {\color{blue}$B_1=\{x_1,y_2,x_3,\dots, x_k\}$};
			
			\path (vb) ++(90:-8 mm) node [std] (vb2) {};
			\path (vb2) ++(0:25 mm) node (Lvb) {\color{red}$B_2=\{z, y_2,x_3,\dots, x_k\}$};	
			%  \path (vb2) ++(0:25 mm) node (Lvb) {\color{blue}$B_2=\{x_1,y_2,x_3,\dots, x_k\}$};	
			
			\draw[thick](vx)--(vb1)--(vb2)--(vy); 
			
			
			%----- right P4 ----------
			\foreach \i/\c in {1/15,2/0,3/-15} {				
				\path (rcent) ++(90:\c mm) node [std] (c\i) {};
				%\path (c\i) ++(0:5 mm) node (Lc\i) {$C_{\i}$};
			}
			
			\draw[thick] (vx)--(c1)--(c2)--(c3)--(vy);
			
			%\path (cent) ++(-90: 20mm) node (HL)  {$\mathcal{H}$};
			
			\path (c1) ++(0:25 mm) node (Lvc1) {\color{blue}$C_1=\{x_1,x_2,y_3,\dots, x_k\}$};
			%\path (c2) ++(0:25 mm) node (Lvc2) {\color{red}$C_2=\{x_1,x_2,y_3,\dots, x_k\}$};
			%\path (c3) ++(0:25 mm) node (Lvc3) {\color{red}$C_3=\{x_1,x_2,y_3,\dots, x_k\}$};
			
			%\path (c2) ++(0:28 mm) node (Lvc2) {\color{red}$C_2 = \{w,x_2,y_3,x_4\dots, x_k\}$};
			\path (c2) ++(0:5 mm) node (Lvc2) {\color{red}$C_2$};
			
			%	\path (Lvc2) ++(0:20 mm) node (Lvc2C) {\color{red} OR };
			%	\path (Lvc2C) ++(90:5 mm) node (Lvc2A) {\color{red} $\{y_1,x_2,y_3,\dots, x_k\}$};
			%	\path (Lvc2C) ++(-90:5 mm) node (Lvc2B) {\color{red} $\{x_1,y_2,y_3,\dots, x_k\}$};
			
			\path (c3) ++(0:5 mm) node (Lvc3) {\color{red}$C_3$};
			
			
		\end{tikzpicture}				
		\caption{$H=\thet{2,3,4}$ non-construction.}
		\label{fig:c:thet234}%
	\end{figure}
	
	
	\begin{proof}
		The construction for our contradiction begins similarly to that of $\thet{2,3,3}$ in Proposition \ref{prop:c:thet233}.  As before, we illustrate the graph  in Figure \ref{fig:c:thet234}, labelling the known sets in blue, and those yet to be determined in red.  Given the similarity of $\thet{2,3,3}$ and $\thet{2,3,4}$, many of the observations from Proposition \ref{prop:c:thet233} carry through to our current proof.  In particular, all of \ref{prop:c:thet233:a} - \ref{prop:c:thet233:d} hold here, including that $y_1\notin C_2$ from  \ref{prop:c:thet233:b}.  Moreover, the compositions of $Y$ and $B_2$ also hold, where $z$ is some vertex with $z \not\in \{y_1,x_2,y_2\}$.
		
		We now attempt to build $C_2$.  From Proposition \ref{prop:i:C4struct}, since $X \not\sim C_2$, $y_3 \in C_2$ (and $x_3 \not\in C_2$).  From the distance requirement of Observation \ref{obs:i:dist}, $|X - C_2| \leq 2$, and so at least one of $x_1$ or $x_2$ is in $C_2$.  Recall from the construction for Proposition \ref{prop:c:thet233} that $\edge{A,x_2,z,Y}$ and $\edge{B_1,x_1,z,B_2}$, and so $z$ is adjacent to both $x_1$ and $x_2$.  Hence, $z \not\in C_2$.  
		
		Gathering these results shows that none of $\{x_3,z,y_1\}$ are in $C_2$, and thus, $d(C_2, Y) \geq 3$, contradicting the distance requirement of Observation \ref{obs:i:dist}.  We conclude that no graph $G$ exists such that $\ig{G} = \thet{2,3,4}$.
	\end{proof}
	
	% <<< E: Prop {prop:c:thet234}
	
	
	
	
	
	
	
	
	
	%______________________SUBSUB SECTION:  theta (3,3,3) ______________________
	
	\subsubsection{$\thet{3,3,3}$ Is Not an $i$-Graph}  \label{subsubsec:c:333}
	
	
	%>>>S: Prop {prop:c:thet333}
	\begin{prop}	\label{prop:c:thet333}
		The graph $\thet{3,3,3}$ is not $i$-graph realizable.
	\end{prop}
	
	%~~~~~~~~~~~~FIG START~~~~~~~~~~~
	\begin{figure}[H]
		\centering
		\begin{tikzpicture}%[line width=0.3mm,scale=1]			
			
			%---------- REF-------------------
			\node(cent) at (0,0) {};			
			\path (cent) ++(0:60 mm) node(rcent)  {};		
			
			%----- left C4 ----------	
			\foreach \i/\j/\c in {x/X/90,y/Y/-90} {				
				\path (cent) ++(\c:20 mm) node [std] (v\i) {};
				%\path (v\i) ++(\c:5 mm) node (Lv\i) {${\j}$};
			}
			
			\path (vx) ++(90:5 mm) node (Lvx) {$\color{blue}X=\{x_1,x_2,x_3,\dots, x_k\}$};
			\path (vy) ++(-90:5 mm) node (Lvy) {$\color{red}Y$};
			%\path (vy) ++(-90:5 mm) node (Lvy) {$\color{blue}Y=\{y_1,y_2,x_3,\dots, x_k\}$};
			
			\path (cent) ++(180:8 mm) node (va) {};	
			
			%\path (va) ++(180:25 mm) node (Lva) {$A=\{y_1,x_2,x_3,\dots, x_k\}$};
			
			\path (va) ++(90:8 mm) node [std] (va1) {};
			\path (va1) ++(180:25 mm) node (Lva1) {\color{blue}$A_1=\{y_1,x_2,x_3,\dots, x_k\}$};
			
			\path (va) ++(90:-8 mm) node [std] (va2) {};
			\path (va2) ++(180:5 mm) node (Lva2) {\color{red}$A_2$};	
			
			
			
			\draw[thick] (vx)--(va1)--(va2)--(vy);		
			
			
			%------------ B Pathway
			
			\path (cent) ++(0:8 mm) node (vb) {};
			
			\path (vb) ++(90:8 mm) node [std] (vb1) {};
			\path (vb1) ++(0:25 mm) node (Lvb) {\color{blue}$B_1=\{x_1,y_2,x_3,\dots, x_k\}$};
			
			\path (vb) ++(90:-8 mm) node [std] (vb2) {};
			\path (vb2) ++(0:5 mm) node (Lvb) {\color{red}$B_2$};	
			%  \path (vb2) ++(0:25 mm) node (Lvb) {\color{blue}$B_2=\{x_1,y_2,x_3,\dots, x_k\}$};	
			
			\draw[thick](vx)--(vb1)--(vb2)--(vy); 
			
			
			%----- right P4 ----------
			\foreach \i/\c in {1/14,2/-14} {				
				\path (rcent) ++(90:\c mm) node [std] (c\i) {};
				%\path (c\i) ++(0:5 mm) node (Lc\i) {$C_{\i}$};
			}
			
			\draw[thick] (vx)--(c1)--(c2)--(vy);
			
			%\path (cent) ++(-90: 20mm) node (HL)  {$\mathcal{H}$};
			
			\path (c1) ++(0:25 mm) node (Lvc1) {\color{blue}$C_1=\{x_1,x_2,y_3,\dots, x_k\}$};
			%\path (c2) ++(0:25 mm) node (Lvc2) {\color{red}$C_2=\{x_1,x_2,y_3,\dots, x_k\}$};
			%\path (c3) ++(0:25 mm) node (Lvc3) {\color{red}$C_3=\{x_1,x_2,y_3,\dots, x_k\}$};
			\path (c2) ++(0:5 mm) node (Lvc2) {\color{red}$C_2$};
			
			
			
		\end{tikzpicture}				
		\caption{$H=\thet{3,3,3}$ non-construction.}
		\label{fig:c:thet333}%
	\end{figure}
	
	
	\begin{proof}
		Let $H$ be the theta graph $\thet{3,3,3}$, with vertices labelled as in Figure \ref{fig:c:thet333}.  Suppose to the contrary that there exists some graph $G$ such that $H$ is the $i$-graph of $G$;  that is, $\ig{G} = \thet{3,3,3}$.
		
		%From Observation \ref{obs:i:dist}, since $d_H(X,Y) = 3$, $ 2\leq |X-Y | \leq 3$. We consider the two case for $|X-Y|$ separately.
		
		%\noindent \textbf{Case 1:} $|X-Y|=3$. %Without loss of generality, say $Y=\{y_1,y_2,y_3,x_4,\dots,x_k\}$, where $y_1, y_2, y_3 \not\in V(G)-X$.  
		
		Since $d_H(A_1,Y)=2$,  by Observation  \ref{obs:i:d2}, $|A_1 - Y|=2$.  Similarly, $|B_1-Y|=2$ and $|C_1 - Y|=2$.  
		Suppose that, say, $x_4 \not\in Y$. Hence, by Observation \ref{obs:i:dist} $|\{x_1,x_2,x_3\} \cap Y | \geq 1$.  Without loss of generality, say $x_1 \in Y$.    Then since $y_1 \not\in Y$ and $x_4 \notin Y$, both $x_2$ and $x_3 \in Y$ to satisfy $|A_1 - Y|=2$.  However, then $Y=\{x_1,x_2,x_3,z,x_5,\dots,x_k\}$ for some vertex $z \sim x_4$, and so $\edge{X,x_4,z,Y}$, which is not so.  	
		We therefore conclude that $x_4 \in Y$, and likewise $x_i \in Y$ for $i\geq 4$.  Thus, $\{x_1,x_2,x_3\} \cap Y = \varnothing$.
		
		Returning to $A_1$, since $d(A_1,Y)= 2$ and  $x_2,x_3 \notin Y$,  we have that $y_1 \in Y$.  Similarly, $y_2, y_3 \in Y$.  Thus, $Y=\{y_1,y_2,y_3,x_3,\dots,x_k\}$.  Moreover, $A_2$ is obtained from $A_1$ by replacing one of $x_2$ or $x_3$, by $y_2$ or $y_3$,  respectively.  Say, $\edge{A_1,x_2,y_2,A_2}$ so that $A_2=\{y_1,y_2,x_3,\dots,x_k\}$.  Now, however, we have that $\edge{B_1,x_1,y_1,A_2}$, but clearly $B_1 \not\sim A_2$.  It follows that $\thet{3,3,3}$ is not an $i$-graph.
	\end{proof}
	
	% <<< E: Prop {prop:c:thet333}
	
	
	
	
	
	
	
	
	
	
	
	This completes the proof of Theorem \ref{thm:c:thetas}.
	
	
	
	
	%---------------------------------------SECTION: Other Results ----------------
	
	\section{Other Results}  \label{sec:c:other}
	In the final section of this chapter we first display a graph that is neither a theta graph nor an $i$-graph, and then use the method of graph complements to show that every cubic 3-connected bipartite planar graph is an $i$-graph.
	
	
	\subsection{A Non-Theta Non-$i$-Graph}  \label{subsubsec:c:nonThetnonI}
	
	
	So far, every non-$i$-graph we have observed is either one of the seven theta graphs from Theorem \ref{thm:c:thetas}, or contains one of those seven  as an induced subgraph (as per Corollary \ref{coro:i:notInduced}).  This leads naturally to the question of whether theta graphs provide a forbidden subgraph characterization for $i$-graphs.  Unfortunately, this is not the case.  
	
	Consider the graph $\ntni$ in Figure \ref{fig:c:nonThetnonI}: it is not a theta graph, and although it contains several theta graphs as induced subgraphs, none of those induced subgraphs are among the seven non-$i$-graph theta graphs.  In Proposition \ref{prop:c:nonThetnonI} we confirm  that $\ntni$ is not an $i$-graph.
	
	
	
	
	%~~~~~~~~~~~~FIG START~~~~~~~~~~~
	\begin{figure}[H]
		\centering
		\begin{tikzpicture}%[line width=0.3mm,scale=1]			
			
			%---------- REF-------------------
			\node(cent) at (0,0) {};			
			\path (cent) ++(0:65 mm) node(rcent)  {};		
			
			%----- left C4 ----------	
			\foreach \i/\j/\c in {x/X/90,y/Y/-90} {				
				\path (cent) ++(\c:20 mm) node [std] (v\i) {};
				%\path (v\i) ++(\c:5 mm) node (Lv\i) {${\j}$};
			}
			
			\path (vx) ++(90:5 mm) node (Lvx) {$\color{blue}X=\{x_1,x_2,x_3,x_4\dots, x_k\}$};
			\path (vy) ++(-90:5 mm) node (Lvy) {$\color{red}Y=\{y_1,y_2,y_3,x_4,\dots,x_k\}$};
			%\path (vy) ++(-90:5 mm) node (Lvy) {$\color{blue}Y=\{y_1,y_2,x_3,\dots, x_k\}$};
			
			\path (cent) ++(180:8 mm) node (va) {};	
			
			%\path (va) ++(180:25 mm) node (Lva) {$A=\{y_1,x_2,x_3,\dots, x_k\}$};
			
			\path (va) ++(90:8 mm) node [std] (va1) {};
			\path (va1) ++(180:25 mm) node (Lva1) {\color{blue}$A_1=\{y_1,x_2,x_3,\dots, x_k\}$};
			
			\path (va) ++(90:-8 mm) node [std] (va2) {};
			\path (va2) ++(180:25 mm) node (Lva2) {\color{red}$A_2=\{y_1,x_2,y_3,\dots,x_k\}$};	
			
			
			
			\draw[thick] (vx)--(va1)--(va2)--(vy);		
			
			
			%------------ B Pathway
			
			\path (cent) ++(0:8 mm) node (vb) {};
			
			\path (vb) ++(90:8 mm) node [std] (vb1) {};
			\path (vb1) ++(0:25 mm) node (Lvb) {\color{blue}$B_1=\{x_1,y_2,x_3,\dots, x_k\}$};
			
			\path (vb) ++(90:-8 mm) node [std] (vb2) {};
			\path (vb2) ++(0:25 mm) node (Lvb) {\color{blue}$B_2=\{y_1,y_2,x_3,\dots, x_k\}$};	
			%  \path (vb2) ++(0:25 mm) node (Lvb) {\color{blue}$B_2=\{x_1,y_2,x_3,\dots, x_k\}$};	
			
			\draw[thick](vx)--(vb1)--(vb2)--(vy); 
			
			
			%----- right P4 ----------
			\foreach \i/\c in {1/14,2/0,3/-13} {				
				\path (rcent) ++(90:\c mm) node [std] (c\i) {};
				%\path (c\i) ++(0:5 mm) node (Lc\i) {$C_{\i}$};
			}
			
			\draw[thick] (vx)--(c1)--(c2)--(c3)--(vy);
			
			%\path (cent) ++(-90: 20mm) node (HL)  {$\mathcal{H}$};
			
			\path (c1) ++(0:5 mm) node (Lvc1) {\color{red}$D_1$};
			%\path (c2) ++(0:25 mm) node (Lvc2) {\color{red}$C_2=\{x_1,x_2,y_3,\dots, x_k\}$};
			%\path (c3) ++(0:25 mm) node (Lvc3) {\color{red}$C_3=\{x_1,x_2,y_3,\dots, x_k\}$};
			\path (c2) ++(0:5 mm) node (Lvc2) {\color{red}$D_2$};
			\path (c3) ++(0:5 mm) node (Lvc3) {\color{red}$D_3$};
			
			
			\draw[thick](va1)--(vb2);
			
		\end{tikzpicture}				
		\caption{A non-theta non-$i$-graph $\ntni$.}
		\label{fig:c:nonThetnonI}%
	\end{figure}
	
	%>>> S: Prop {prop:c:nonThetnonI}
	\begin{prop} \label{prop:c:nonThetnonI}
		The graph $\ntni$ in Figure \ref{fig:c:nonThetnonI} is not an $i$-graph.
	\end{prop}
	
	
	\begin{proof}
		We proceed similarly to the proofs for the theta non-$i$-graphs in Section \ref{sec:c:nonthetas}.   Let $\ntni$ be the graph in Figure \ref{fig:c:nonThetnonI}, with vertices as labelled, and suppose to the contrary that there is some graph $G$ such that $\ig{G}\cong \ntni$.  
		
		To begin, we determine the vertices of the two induced $C_4$'s of $\ntni$.  Immediately from Proposition \ref{prop:i:C4struct}, the vertices of $X$, $A_1$, $B_1$, and $B_2$ are as labelled in Figure \ref{fig:c:nonThetnonI}.  Using a second application of Proposition \ref{prop:i:C4struct}, $A_2$ differs from $A_1$ in exactly one position, different from $B_2$.  Without loss of generality, say, $\edge{A_1, x_3, y_3, A_2}$.  Then again by the proposition, $Y=\{y_1, y_2, y_3, x_4 \dots, x_k\}$ as in Figure \ref{fig:c:nonThetnonI}.
		
		We now construct the vertices of $D_1$ through a series of claims:
		
		% From Lemma \ref{lem:i:claw}, $D_1$ differs from $X$ in one vertex, different from both $A_1$ and $B_1$.  
		
		%	By Observation \ref{obs:i:d2}, since $|D_1 - Y| =3$,
		
		\begin{enumerate}[label=(\roman*)]
			\item \label{prop:c:nonThetnonI:i} $x_3$ is not in $D_1$. 
			
			From Lemma \ref{lem:i:claw}, $D_1$ differs from $X$ in one vertex, different from both $A_1$ and $B_1$; moreover,  $x_1$ and $x_2$ are in $D_1$.    		
			Notice that $\{x_4,x_5,\dots,x_k\} \subseteq D_1$: suppose to the contrary that, say,  $x_4 \notin D_1$, so that $\edge{X, x_4,z,D_1}$.  Then $z \neq y_1$, since $y_1 \sim_G x_1$ and $D_1$ is independent.  Likewise $z \neq y_2$ and $z \neq y_3$. Thus $D_1 = \{x_1, x_2, x_3, z, x_5,\dots, x_k\}$ has $|Y \cap D_1| =4$, but $d(D_1,Y)=3$, contradicting Observation \ref{obs:i:d2}. 
			
			
			
			\item \label{prop:c:nonThetnonI:ii} $y_1$, $y_2,$ and $y_3$ are not in $D_1$.  
			
			
			As above, $y_1$ and $y_2$ are not in $D_1$, as $D_1$ is an independent set containing $x_1$ and $x_2$.  
			Suppose to the contrary that $\edge{X,x_3,y_3,D_1}$, so that $D_1 = \{x_1, x_2, y_3,  \dots, x_k\}$. Then, since $x_1 \sim_G y_1$, we have that $\edge{D_1,x_1,y_1,A_2}$, a contradiction.
			
			\item \label{prop:c:nonThetnonI:iii} $D_1 = \{x_1, x_2, z, x_4, \dots, x_k\}$, where $z \notin \{y_1,y_2,y_3\}$. 
			
			Immediate from \ref{prop:c:nonThetnonI:i} and \ref{prop:c:nonThetnonI:ii}.
			
			
		\end{enumerate}
		
		A contradiction for the existence of $G$ arises as we construct $D_2$.  Since $d_{\ntni}(D_1,Y)=3$ and $|D_1 \cap Y| = 3$, at each step along the path through $D_2, D_3,$ and $Y$, exactly one token departs  from a vertex of $D_1$  and moves to one of $Y-D_1=\{y_1,y_2,y_3\}$.
		However, $D_2 \neq \{y_1,x_2,z,x_4,\dots,x_k\}$, since otherwise, $\edge{A_1,x_3,z,D_2}$.  Likewise, $D_2 \neq \{x_1,y_2,z,x_4,\dots,x_k\}$.  Finally $D_2 \neq \{x_1,x_2,y_3,x_4,\dots,x_k\}$ as again we have $\edge{A_2,y_1,x_1,D_2}$.  Thus, we cannot build a set $D_2$ such that $|D_2 \cap Y| \leq 2$ as required.  We so conclude that no such graph $G$ exists, and that $\ntni$ is not an $i$-graph.
	\end{proof}
	
	Like the seven non-$i$-graph realizable theta graphs of Theorem \ref{thm:c:thetas},  $\ntni$ is a minimal obstruction to a graph being an $i$-graph; every induced subgraph of  $\ntni$ is a theta graph.

	
	
	
	%---------
	\subsection{Maximal Planar Graphs} \label{subsubsec:c:maxPlanar}
	
	
	We conclude this chapter with a result to demonstrate that certain planar graphs are $i$-graphs and $\agraph$-graphs.  Our proof uses the following three known results.
	
	
	
	
	%\nts{Fix reference}
	
	%>>> S: Theorem {thm:c:ref:planarDual}  
	\begin{theorem} \cite[Theorem 4.6]{CLZ} \label{thm:c:ref:cub3con}  A cubic graph is 3-connected if and only if it is 3-edge connected.
	\end{theorem}
	% <<< E: Theorem {thm:c:ref:planarDual}  
	
	
	
	
	
	%>>> S: Theorem {thm:c:ref:planarDual}  
	\begin{theorem} \cite{W69} \label{thm:c:ref:3col}  
		A connected planar graph $G$ is bipartite if and only if its dual $\dual{G}$ is Eulerian.
	\end{theorem}
	% <<< E: Theorem {thm:c:ref:planarDual}  
	
	
	
	
	%>>> S: Theorem {thm:c:ref:maxPlan}   
	\begin{theorem} \cite{H1898, TW11} \label{thm:c:ref:maxPlan}  
		A maximal planar graph $G$ of order at least 3 has $\chi(G) = 3$ if and only if ${G}$ is Eulerian.
	\end{theorem}
	% <<< E: Theorem {thm:c:ref:maxPlan}  
	
	
	
	In our proof of the following theorem, we consider a graph $G$ that is cubic, 3-connected, bipartite, and planar. We then examine its dual $\dual{G}$, and how those specific properties of $G$ translate to $\dual{G}$.  Then, we construct the complement of $\dual{G}$, which we refer to as $H$.  We claim that $H$ is a seed graph of $G$; that is, $\ig{H}$ contains an induced copy of $G$. 
	
	
	
	\begin{comment}
		%>>> S: Theorem {thm:c:ref:planarDual}  
		\begin{theorem} \cite{T1963} \label{thm:c:3conn}  
			Let $G$ be a 3-connected planar graph.  A cycle $C$ of $G$ is a facial cycle of some plane embedding of $G$ if and only if it is an induced cycle and non-separating.
		\end{theorem}
		% <<< E: Theorem {thm:c:ref:maxPlan}  
	\end{comment}
	
	%>>> S: Theorem {thm:c:3con}   
	\begin{theorem} \label{thm:c:3con}  
		Every cubic 3-connected bipartite planar graph  is an $i$-graph and an $\agraph$-graph.	
	\end{theorem}
	
	\begin{proof}
		Let $G$ be a cubic 3-connected bipartite planar graph and consider the dual $\dual{G}$ of $G$.  Since $G$ is bridgeless, $\dual{G}$ has no loops.  Moreover, since $G$ is 3-connected, Theorem \ref{thm:c:ref:cub3con} implies that no two edges separate $G$; hence, $\dual{G}$ has no multiple edges.  Therefore, $\dual{G}$  is a (simple) graph.  Further, since $G$ is cubic, each face of $\dual{G}$ is a triangle, and so $\dual{G}$ is a maximal planar graph.  
		
		We note the following two key observations.  First, since each face of $\dual{G}$ is a triangle,
		
		
		\begin{enumerate}[label=(\arabic*)]
			\item  \label{thm:c:3con:i}   each edge of $\dual{G}$ belongs to a triangle.
			
		\end{enumerate}
		
		\noindent For the second, note that since $G$ is bipartite, $\dual{G}$ is Eulerian (by Theorem \ref{thm:c:ref:3col}).  Then, by Theorem \ref{thm:c:ref:maxPlan}, $\chi(\dual{G})=3$, so $\dual{G}$ does not contain a copy of $K_4$.  Thus,
		
		\begin{enumerate}[label=(\arabic*),resume]
			\item \label{thm:c:3con:ii}   every triangle of $\dual{G}$ is a maximal clique.
		\end{enumerate}
		
		
		Now, by the duality of $\dual{G}$ and $G$, there is a one-to-one correspondence between the facial triangles of $\dual{G}$ and the vertices of $G$.   Let $H$ be the complement of $\dual{G}$.  By \ref{thm:c:3con:i} and \ref{thm:c:3con:ii}, $i(H) = \alpha(H)=3$, and every maximal independent set of $G$ corresponds to a triangle of $\dual{G}$.  But again, by duality, the facial triangles of $\dual{G}$ and their adjacencies correspond to the vertices of $G$ and their adjacencies.  Therefore, $\ig{H}$ contains $G$ as an induced subgraph.   Any additional unwanted vertices of $\ig{H}$ can be removed by applying the Deletion Lemma (Lemma \ref {lem:i:inducedI}).
	\end{proof}



