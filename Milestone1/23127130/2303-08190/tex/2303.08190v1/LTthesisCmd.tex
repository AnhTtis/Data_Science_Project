%%%%%%%%%%######################################### MASTER FILE ###############################################


%\documentclass[12pt]{article}%
\usepackage{amssymb, amsfonts, amsmath}
\usepackage{mathrsfs}
\usepackage{graphicx}
\usepackage[usenames,dvipsnames]{xcolor}
\usepackage{tikz}
\usepackage{float}
\usepackage{enumitem}
\usepackage{comment}
\usepackage{setspace}
\usepackage{caption}
\usepackage{subcaption}
\usepackage[hidelinks]{hyperref}
\usepackage{array}%
%\usepackage{eufrak}
\usepackage{multicol}

\setcounter{MaxMatrixCols}{30}%

\newcolumntype{C}[1]{>{\centering\let\newline\\\arraybackslash\hspace{0pt}}m{#1}}
\providecommand{\U}[1]{\protect\rule{.1in}{.1in}}
\providecommand{\boksie}{\ensuremath{\mathbin{\raisebox{0.3mm}{$\scriptstyle\square$}}}}

% -- DIMENSIONS --
\textwidth=6.4in
\textheight=9in
\evensidemargin=0in
\oddsidemargin=0in
\topmargin=-0.6in
\topskip=0pt
\baselineskip=12pt
\parskip=2mm

\begin{comment}
\newtheorem{theorem}{Theorem}
\newtheorem{acknowledgment}[theorem]{Acknowledgment}  
\newtheorem{algo}{Algorithm}
\newtheorem{axiom}[theorem]{Axiom}
\newtheorem{case}[theorem]{Case}
\newtheorem{claim}[theorem]{Claim}
\newtheorem{conclusion}[theorem]{Conclusion}
\newtheorem{condition}[theorem]{Condition}
\newtheorem{conj}{Conjecture}
\newtheorem{cons}[theorem]{Construction}
\newtheorem{coro}[theorem]{Corollary}
\newtheorem{criterion}[theorem]{Criterion}
\newtheorem{defn}[theorem]{Definition}
\newtheorem{exercise}[theorem]{Exercise}
\newtheorem{lemma}[theorem]{Lemma}
\newtheorem{notation}[theorem]{Notation}
\newtheorem{obs}[theorem]{Observation}
\newtheorem{problem}{Problem}
\newtheorem{prop}[theorem]{Proposition}
\newtheorem{remark}[theorem]{Remark}
\newtheorem{solution}[theorem]{Solution}
\newtheorem{summary}[theorem]{Summary}
\end{comment}

%\newenvironment{proof}[1][Proof]{\noindent\textbf{#1.} }{\ \hfill \rule{0.5em}{0.5em}}
%\renewenvironment{proof}[1][Proof]{\noindent\textbf{#1.} }{\ \hfill \rule{0.5em}{0.5em}}

%--------------------------COMMANDS -------------------------------------------------


%%% Notes to self: %%%
%\newcommand{\nts}[1]{{ \noindent \color{blue} \vspace{5mm} {\bf **NOTE** }  #1}}
\newcommand{\nts}[1]{{ \noindent \color{blue} {\bf **NOTE** }  #1}}
\newcommand{\pr}[1]{{ \color{ForestGreen} [{\bf PR: }  #1]}}
\newcommand{\cmty}[1]{{\color{Fuchsia}(#1)}}
\newcommand{\ft}[1]{{\color{red}#1}}

%%% Math Shortcuts  %%%
\newcommand{\Mod}[1]{\ (\text{mod}\ #1)}
%\newcommand{\ig}[1]{#1(i)}
%\newcommand{\ig}[1]{\mathswab{I}(#1)}
\newcommand{\ig}[1]{\mathscr{I}(#1)}
\newcommand{\agraph}{\alpha}
\newcommand{\ag}[1]{\mathscr{A}(#1)}
\newcommand{\cp}[1]{\overline{#1}}
\newcommand{\thet}[1]{\Theta\left\langle #1\right\rangle }


\providecommand{\diam}{\operatorname{diam}}
\providecommand{\rad}{\operatorname{rad}}
\providecommand{\dist}{\operatorname{dist}}
\providecommand{\bm}{\boldsymbol}
%\renewcommand{\mod}{\operatorname{mod}\ }
%\renewcommand{\vec}[1]{\mathbf{#1}}
\providecommand{\pn}{\operatorname{pn}}
\providecommand{\gid}{\gamma^{ID}}
\providecommand{\ggid}{G(\gamma^{ID})}
\providecommand{\leqn}{1 \leq i \leq n}
\providecommand{\gpr}{\gamma_{pr}}
\providecommand{\gt}{\gamma_{t}}
\providecommand{\ir}{\operatorname{ir}}

%--- manual proof slug
\providecommand{\slug}{\rule{0.5em}{0.5em}}

%--- special graphs
\providecommand{\bull}{\mathcal{B}}
\newcommand{\Dia}{\mathfrak{D}}
\newcommand{\house}{\mathcal{H}}
\newcommand{\wlat}{\mathfrak{L}}
\newcommand{\bcl}{\mathfrak{B}}
\newcommand{\ntni}{\mathfrak{T}}

%--- Notation for cycle i-graphs
\newcommand{\cid}[1]{\left\langle #1 \right\rangle }
\newcommand{\bk}[1]{\left\lbrace  #1 \right\rbrace  }


%%% Common Graph Things  
\newcommand{\dual}[1]{\widetilde{#1}}


%%% 2-part work around for \edge definition.  No idea how it works.  Why do you do this to yourself?
%- 4 parameter - X,x,y,Y - version
\def\ieInner(#1,#2,#3,#4){#1 \overset{#2 #3}{\sim} #4}
\def\edge#1{\ieInner(#1)}
%
%- 3 parameter - X,x,y - mid list - version
\def\ieInnerA(#1,#2,#3){\overset{#1 #2}{\sim} #3}
\def\adedge#1{\ieInnerA(#1)}
\def\sedge(#1,#2){#1\sim #2}
%
%- 5 parameter - X,x,y,Y,G - version
%\def\ieInnerB(#1,#2,#3,#4,#5){#1 \underset{#5}{\overset{#2 #3}{\sim}} #4} %True Under Version
\def\ieInnerB(#1,#2,#3,#4,#5){#1 \overset{#2 #3}{\sim_{#5}} #4} %Under to side version
\def\edgeG#1{\ieInnerB(#1)}
%
%- 3 parameter - X,Y,G - version
\def\ieInnerC(#1,#2,#3){#1 {\sim_{#3}} #2 }
\def\adjG#1{\ieInnerC(#1)}
%
%-  parameter - X1,X2,...Xk - version
\def\ieInnerD(#1,#2,#3,#4){#1 \sim #2 \sim \dots \sim #3 \sim #4}
\def\adjL#1{\ieInnerD(#1)}
%
%- no inner parameter X~Y version
\def\ieInnerE(#1,#2){#1 \sim #2}
\def\onlyedge#1{\ieInnerE(#1)}

%- tilde only version
\def\ieInnerF(#1,#2){\overset{#1 #2}{\sim}}
\def\onlyTilde#1{\ieInnerF(#1)}


%%% HELP %%% \providecommand defines a new command if it isn't already defined.
%%% HELP %%% \newcommand defines a new command, and makes an error if it is already defined.
%%% HELP %%% \renewcommand redefines a predefined command, and makes an error if it is not yet defined.

%+++++ Custom Colours ++++
\definecolor{ltsky}{RGB}{0,191,255}
\definecolor{ltteal}{RGB}{0, 128, 128 }
\definecolor{medOrch}{RGB}{122,55,139}
\definecolor{royalBlue}{RGB}{65,105,225}
\definecolor{forGreen}{RGB}{34,139,34}
\definecolor{dand}{RGB}{255,193,37}
\definecolor{lightBlue}{RGB}{176,226,255}
\definecolor{lgrey}{RGB}{209,209,209}
\definecolor{lgray}{gray}{0.95}
\definecolor{mgray}{gray}{0.40}
\definecolor{mmgray}{gray}{0.60}
\definecolor{zelim}{RGB}{7,163,82}
\definecolor{zelim2}{RGB}{5,114,57}


%++++ TIKZ  GARBAGE ++++
\usetikzlibrary{fit,positioning,calc, arrows, shapes}
\usetikzlibrary{decorations.pathreplacing}
\usetikzlibrary{backgrounds}
%\usetikzlibrary{patterns}
%\usetikzlibrary{intersections}
%\usepackage{pgfplots}
\tikzstyle{std}=[ circle, draw=black,fill=black,thick, inner sep=2pt, minimum size=2.5mm]
\tikzstyle{wstd}=[ circle, draw=black,fill=white,thick, inner sep=2pt, minimum size=2.5mm]
\tikzstyle{ir}=[ circle, draw=black,fill=green,thick,  inner sep=2pt, minimum size=2mm]
\tikzstyle{mp}=[circle, draw=black,fill=Dandelion,thick,  inner sep=2pt, minimum size=2mm]
\tikzstyle{bred}=[circle, draw=black,fill=red,thick,  inner sep=2pt, minimum size=2mm]
\tikzstyle{byellow}=[circle, draw=black,fill=yellow,very thick,  inner sep=2pt, minimum size=3mm]
\tikzstyle{bgray}=[circle, draw=black,fill=mmgray,thick,  inner sep=2pt, minimum size=2mm]
\tikzstyle{bzed}=[circle, draw=black,fill=zelim,thick,  inner sep=2pt, minimum size=2mm]
\tikzstyle{smred}=[ circle, draw=black,fill=red,thick,  inner sep=1pt, minimum size=1.5mm]
\tikzstyle{bblue}=[ circle, draw=black,fill=blue,thick,  inner sep=2pt, minimum size=2.5mm]
\tikzstyle{bgreen}=[ circle, draw=black,fill=green,thick,  inner sep=2pt, minimum size=2.5mm]
\tikzstyle{regRed}=[ circle, draw=black,fill=red,thick,  inner sep=2pt, minimum size=2.5mm]
\tikzstyle{sqRed}=[rectangle, draw=black,fill=red,thick,  inner sep=2pt, minimum size=2.5mm]
\tikzstyle{regG}=[ circle, draw=black,fill=green,thick,  inner sep=2pt, minimum size=2mm]
\tikzstyle{dred}=[diamond, draw=black,fill=red,thick,  inner sep=2pt, minimum size=2mm]
\tikzstyle{tr}=[color=black, style=dotted]
\tikzstyle{sp}=[color=ProcessBlue, line width = 2pt]

\tikzstyle{bline}=[color=blue, line width = 1pt]
\tikzstyle{rline}=[color=red, line width = 1pt]
\tikzstyle{rlinelt}=[color=red, line width = 1pt,opacity=0.2]
\tikzstyle{rlinemb}=[color=red, line width = 1pt, densely dashdotted]
\tikzstyle{gline}=[color=mmgray, line width = 1pt]
\tikzstyle{zline}=[color=zelim, line width = 1pt]
\tikzstyle{ll}=[color=gray]

\tikzstyle{vertex}=[circle, fill=black, inner sep= 0, minimum size = 4]

%%% Custom Broad Arrow Shape with optional fill, border and internal text
\tikzset{broadArrow/.style={single arrow, fill=red!50, anchor=base, align=center,text width=2.8cm}}
% Example Call: \node[broadArrow,fill=white, draw=black, ultra thick, text width=1cm] at (3.5cm,0){text goes here};
% Edit from: https://tex.stackexchange.com/questions/84143/fancy-arrows-with-tikz

\pgfdeclarelayer{blob}    % declare background layer
\pgfdeclarelayer{bedge}    % declare background layer
\pgfsetlayers{bedge,blob,main}


%-- CONVEX PATH
\newcommand{\convexpath}[2]{
	[   
	create hullcoords/.code={
		\global\edef\namelist{#1}
		\foreach [count=\counter] \nodename in \namelist {
			\global\edef\numberofnodes{\counter}
			\coordinate (hullcoord\counter) at (\nodename);
		}
		\coordinate (hullcoord0) at (hullcoord\numberofnodes);
		\pgfmathtruncatemacro\lastnumber{\numberofnodes+1}
		\coordinate (hullcoord\lastnumber) at (hullcoord1);
	},
	create hullcoords
	]
	($(hullcoord1)!#2!-90:(hullcoord0)$)
	\foreach [
	evaluate=\currentnode as \previousnode using \currentnode-1,
	evaluate=\currentnode as \nextnode using \currentnode+1
	] \currentnode in {1,...,\numberofnodes} {
		let \p1 = ($(hullcoord\currentnode) - (hullcoord\previousnode)$),
		%\n1 = {atan2(\x1,\y1) + 90},
		\n1 = {atan2(\y1,\x1) + 90},
		\p2 = ($(hullcoord\nextnode) - (hullcoord\currentnode)$),
		%\n2 = {atan2(\x2,\y2) + 90},
		\n2 = {atan2(\y2,\x2) + 90},
		\n{delta} = {Mod(\n2-\n1,360) - 360}
		in 
		{arc [start angle=\n1, delta angle=\n{delta}, radius=#2]}
		-- ($(hullcoord\nextnode)!#2!-90:(hullcoord\currentnode)$) 
	}
}





%%%%%%%%%% STYLE NOTES %%%%%%%%%%%

%+++++ Spellings and Style ++++
% "nonplanar" - one word,  no hyphen,.

