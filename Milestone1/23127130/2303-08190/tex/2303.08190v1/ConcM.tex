
\chapter{Conclusion}

\label{ch:conc}





In this dissertation, we provided new results on the slide graph model of reconfiguration graphs for domination parameters. 


In Chapter \ref{ch:slater}, we defined several variations on the $\gamma$-graph using other  parameters such as  total domination number, the paired domination number,  the irredundance number, and the identification number, and showed that in each case all graphs are $\pi$-graphs for their respective parameters $\pi$.  The singular exception was the Roman dominating graph, where we demonstrated that $K_2$ is not a $\gamma_R$-graph.



In Chapter \ref{ch:igraphs}, we showed that unlike the graphs of  Chapter \ref{ch:slater}, not all graphs are $i$-graph realizable.  We built a series of tools to show that known $i$-graphs can be used to construct new $i$-graphs and applied these results to build other classes of $i$-graphs, such as block graphs, hypercubes, forests, and unicyclic graphs.  We also showed that a chordal graph is $i$-graph realizable if and only if it is diamond-free.


In Chapter \ref{ch:pathcycles}, we determined the structure of the $i$-graphs of paths and cycles, and in the case of cycles, determined exactly which cycles have Hamiltonian or Hamiltonian traceable $i$-graphs. 

In Chapter \ref{ch:comp} we used graph complements to construct the $i$-graph seeds for certain classes of line graphs, theta graphs, and maximal planar graphs.  In doings so, we characterized the line graphs and theta graphs that are $i$-graphs.   
The results concerning the $i$-graph realizability of theta graph are listed in Table \ref{tab:c:thetaLong}.

We conclude this dissertation with a list of open problems, some of which were previously mentioned in the text.




\section*{Open Problems}


%---- S: {op:forbidSubG}

Of the open problems presented in this dissertation, we believe the following to be the most important:

\begin{problem}\label{op:forbidSubG}
	Does there exist a finite forbidden subgraph characterization of $i$-graph realizable graphs?
\end{problem}


\noindent We now list the remaining open problems in the order in which their associated topic was discussed in the dissertation.




%================= SLATER PROBLEMS

\begin{problem}
	Determine conditions on the graph $G$ under which each $\gamma$-graph
	variation is connected/disconnected. 
\end{problem}	

\begin{problem}
	Let $\pi$ be any of the above-mentioned domination-related parameters for which every graph is a $\pi$-graph.
	Is it true that every \textbf{bipartite} graph is the $\pi$-graph of a
	\textbf{bipartite }graph?
\end{problem}

\begin{problem}
	If $H$ is $i$-graph realizable, does there exist a well-covered seed graph $G$ such that $\ig{G}\cong H$?  
\end{problem}

\begin{problem} 
	We have already seen that classes of graphs trees, cycles, and, more generally, block graphs, are $i$-graphs.  As we build new graphs from these families of $i$-graphs, using tree structures, which of those are also $i$-graphs?  
	For example, are cycle-trees $i$-graphs?  Path-trees?  For what families of graphs $H$ are $H$-trees $i$-graphs?
\end{problem}


\begin{problem} \label{con:op:tree}
	Determine the structure of  $i$-graphs of various families of trees.  For example, consider caterpillars in which every vertex has degree 1 or 3.  
	
\end{problem}



\begin{problem} \label{con:op:ham}
	Find more classes of $i$-graphs that are Hamiltonian, or Hamiltonian traceable.
\end{problem}


\begin{problem} \label{con:op:domChain}
	Let $\pi$ be any of the previously mentioned domination-related parameters.  Suppose $G_1, G_2,\dots$ are graphs such that $\pi(G_1) \cong G_2$, $\pi(G_2)\cong G_3$, $\pi(G_3) \cong G_4, \dots$  Under which conditions does there exist an integer $k$ such that $\pi(G_k) \cong G_1$?
	
	%Hedetniemi Domination - Chain type questions
\end{problem}

As a special case of Problem \ref{con:op:domChain},  we saw in Chapter \ref{ch:igraphs}  that for any $n\geq 1$, $\ig{K_n} \cong K_n$, and that for $k \equiv 2 \Mod{3}$, $\ig{C_k} \cong C_k$.   

\begin{problem} 
	Determine for which graphs $G$, $\ig{G} \cong G$.
\end{problem}