\section{Conclusion}
\label{sec:conclusion}

In this work, we investigated the evolution of smart contract weaknesses as reported by analysis tools.
We managed to cover all 48 million smart contracts deployed on the Ethereum main chain up to block 14\,000\,000, by selecting tools that are able to process runtime bytecode, and by choosing only one representative for each group of contracts with the same skeleton.
In total, we ran 12 tools on \np{248328} contracts with a cumulative execution time of 30 years.
We summarize our contributions and observations.

\ITEM{Skeletons are an effective technique to identify similar contracts.}
Clustering contracts by their skeleton reduces the computing effort as well as the bias that is introduced when sampling contracts from a population that contains some contracts once and others thousandfold.
We show that the validity of studies like ours is not affected by picking only one contract per cluster.

\ITEM{The rate of reported weaknesses decreases over time.}
The tools report a total of \np{1307484} weaknesses, the most common ones being \emph{\SWC{107}} (\percent{184610}{1307484}), \emph{\SWC{104}} (\percent{183277}{1307484}) and \emph{\SWC{101}} (\percent{123255}{1307484}).
The weaknesses are not equally distributed over the study period, though.
By and large, we observe for all tools and all weaknesses a decrease in flagged contracts over time. We offer three explanations.

Some tools are no longer maintained and cannot handle operations added to the EVM later on.
As such operations get more widely used, the tools increasingly fail in their analyses.

Even tools interpreting all operations correctly, may detect weaknesses by code patterns that are tied to specific compiler versions.
As the code generator changes with newer compilers, the code patterns become less effective in indicating the weakness.

But the decrease in flagged contracts can also be observed for maintained and recent tools, which indicates that the weaknesses become indeed less prevalent over time.
This may be attributed to factors like the adoption of good programming practices, public repositories with tested code, enhancements to the programming language Solidity, and checks added by the Solidity compiler.

\ITEM{The analysis tools differ considerably regarding resource consumption and engineering aspects.}
We see large differences in average runtimes, in the number of analyses timing out or running out of memory, and in the number of errors and failures.
These aspects are of relevance in practice, e.g.\ when integrating analysis tools in CI/CD workflows.

\ITEM{The tools agree only partially in their judgment of contracts, with the disagreement increasing over time.}
Our overlap analysis shows that tools targeting the same weakness flag rather different sets of contracts.
The intersection of these sets decreases over time.
We attribute this phenomenon to diverging interpretations of the weaknesses, as precise and commonly accepted definitions are lacking.
Regarding the change over time, our data provides no explanation.

\ITEM{Service to the community.}
In the course of our study, we found several bugs in tools, which we reported either by filing issues or by exchanging emails and engaging in discussions.
The extension of SmartBugs to process bytecode has already been taken up by the framework Centaur\footnote{\url{https://github.com/mchara01/centaur}}.

\ITEM{Recommendations to smart contract developers, tool authors, and the community at large.}
From the experience gathered in this study, we derive the following recommendations and wishes.
\begin{itemize}
\item
  When hardening or auditing smart contracts, use a range of analysis tools, as their approaches and abilities are complementary.
  Grant the tools sufficient resources, memory- and timewise.
\item
  Maintain academic tools for some years after publishing the accompanying article, and keep them public, as a service to the community.
  This allows researchers to evaluate new methods against the state of the art, on recent data.
\item
  Strive for an abstract definition of the weakness addressed, using e.g.\ some formal semantics, execution traces, and path conditions.
  Give a precise definition of the code patterns used to detect the weakness.
  This makes it easier to analyze the scope of tools and to interpret their results.
\item
  Work towards a comprehensive, balanced ground truth.
  Ultimately, many interesting questions regarding the quality of tools and their methods can only be answered by having access to an `oracle' saying true or false.
  This goal is interlinked with the previous one, as the latter determines the meaning of the former.
\end{itemize}


\section*{Data and Code Availability}

The data and the scripts of our study are available from \url{https://figshare.com/s/5efef6335fa98ddc3ae2}.
The dataset of \np{248328} contracts with distinct skeletons has additionally been published at \url{https://github.com/gsalzer/skelcodes}.
SmartBugs is developed as a GitHub project at \url{https://github.com/smartbugs/smartbugs}.
Some utilities for the manipulation of bytecode, like the computation of skeletons, are maintained at \url{https://github.com/gsalzer/ethutils}.

\section*{Acknowledgements}
This project was partially supported by national funds through Funda\c{c}\~ao para a Ci\^encia e a Tecnologia (FCT) under project UIDB/50021/2020.
The project was also partially supported by the CASTOR Software Research Centre.

We are particularly grateful to three anonymous reviewers, whose detailed reports and constructive advice lead to substantial improvements.