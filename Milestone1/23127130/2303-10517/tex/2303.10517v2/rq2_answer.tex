\section{RQ2 Weakness Detection over Time}\label{sec:rq2}
In this section, we portray the evolution of weaknesses on a timeline of blocks.
We look at the percentage of contracts flagged by a particular tool as possessing any weakness (\autoref{fig:findings}) as well as at the percentage of contracts flagged by any tool as possessing a particular weakness (\autoref{fig:swc_classes}).

\begin{figure}
\centering
\includegraphics[width=.9\columnwidth]{hits}
\caption{Accumulated findings per tool over time. Each data point shows the percentage of bytecodes for which the tool reports a weakness, in bins of 100\,k blocks.}
\label{fig:findings}
\includegraphics[width=.9\columnwidth]{hits2}
\caption{Accumulated findings for Maian, Osiris, Oyente and Vandal. Compared to the third plot in \autoref{fig:findings}, some spurious findings have been omitted (see the text for details).}
\label{fig:findings2}
\end{figure}


\subsection{Tool reports}\label{sec:tool_reports}
\autoref{fig:findings} depicts the reporting rate of each tool over the range of 14\,M blocks.
Each data point represents the percentage of bytecodes in a bin of 100\,k blocks that were marked with at least one non-omitted finding by the respective tool.
The vertical lines in gray indicate forks that added EVM opcodes and thus may affect weakness detection.
To improve readability, the diagram is split into three plots with four tools each.

\ITEM{Upper plot.}
Pakala (green) and teEther (red) both flag a few bytecodes only.
This can be attributed to the fact that Pakala scans for two rather infrequent weaknesses (SWC\,105, 112), and teEther just for one (SWC\,105).
eThor (orange) also scans for a single weakness (SWC\,107), albeit for a far more prevalent one.
Conkas (blue) scans for five weaknesses (SWC\,101, 104, 107, 114, 116), among them the most frequent ones (SWC\,101 and 107).

\ITEM{Middle plot.} 
MadMax (blue) and Ethainter (orange) specialize in rather specific weaknesses that they detect for a small number of contracts only.
MadMax is geared towards three gas issues, loosely related to SWC\,101, 113, and 128, while Ethainter scans for five weaknesses (SWC\,105, 106, 112, 124, and unchecked tainted static call).
Securify (red) also scans for five weaknesses (SWC\,104, 105, 107, 114, and missing input validation), including the popular reentrancy bug.

Mythril (green) tests for the largest number of weaknesses (SWC\,101, 104--107, 110, 112, 113, 115, 116, 120, 124, 127), including the most prevalent ones.
While we see a peak with more than 90\,\% of contracts flagged in the early days, the rate of contracts with reported weaknesses continuously drops to below 40\,\% towards the end of the timeline.

\ITEM{Lower plot.}
The tools in this plot differ from the others, as the rate of flagged contracts stays high or even increases towards the end of the timeline.
Maian (blue) scans for three weaknesses (SWC\,105, 106, and locked Ether), Osiris (orange) for nine (SWC\,101, 107, 114, 116, integer issues beyond SWC\,101, and the callstack depth bug), Oyente (green) for four (SWC\,107, 114, 166, and the callstack depth bug), and Vandal (red) for five (SWC\,104--107 and 115).

In terms of EVM operations supported (see \autoref{sec:supported_ops}), Maian, Oyente and Osiris are the oldest tools in our collection.
They do not handle the operation \OP{SHR}, which is central to newer contracts (\autoref{fig:ops_evol}).
Hence, we expect the rate of findings to drop over time rather than to rise.
It turns out that Oyente checks for \emph{Callstack Depth Bugs} by searching for a specific code pattern (instead of using symbolic execution as for the other weaknesses), and Osiris inherits this functionality from Oyente.
Even though the bug has become obsolete with the fork at block 2.463\,M (see \autoref{sec:synopsis_weaknesses}), the pattern is detected at an increasing rate and causes spurious findings.
Regarding Maian, it checks, among other weaknesses, for \emph{Ether lock}.
This property requires to check all execution paths for the absence of operations that are able to transfer Ether.
As the inability to handle \OP{SHR} cuts short more and more of the paths, the number of falsely reported Ether locks increases.

Vandal reports \np[\%]{96.6} of the contracts with a \OP{CALL} instruction as containing an \textit{Unchecked Call} and \np[\%]{88.4} as containing a \textit{Reentrant Call}.
Given that the majority of calls are method calls, for which the Solidity compiler adds checks automatically, and given that reentrancy is known to be a common but not a universal problem, we suspect that Vandal applies weak criteria and thus reports numerous false positives.

In \autoref{fig:findings2}, we omit the problematic findings \emph{Callstack Depth Bug} for Oyente and Osiris, the finding \emph{Ether lock} for Maian, and the weaknesses \textit{unchecked call} (SWC\,104) and \textit{reentrant call} (SWC\,107) for Vandal.
With these omissions, the number of flagged contracts either is constantly low or drops low.

\ITEM{General Observation.}
Overall, the share of flagged contracts diminishes over time.
For unmaintained tools, this may be related to the fact that they are no longer able to analyze recent contracts containing e.g.\ new instructions.
Moreover, code patterns tailored to specific compiler versions may fail to detect a weakness in bytecode obtained by later versions.
For actively maintained or new tools, the decreasing number of flagged contracts may indeed indicate that newer contracts are less vulnerable than older ones.

\begin{figure}
  \centering
  \includegraphics[width=.9\columnwidth]{evolution_vuls}
  \caption{SWC classes over time. Each data point shows the percentage of bytecodes flagged with a specific weakness, in bins of 100\,k blocks.}
  \label{fig:swc_classes}
\end{figure}

\subsection{SWC classes detected}

For weaknesses mapped to a suitable SWC class, \autoref{tab:vuls} gives an overview of their prevalence.
The column \emph{frequency} counts the number of unique skeleton bytecodes, where at least one tool reports the respective weakness%
\footnote{%
  Most tools do not verify their assessment by providing an exploit (like teEther does) or by proving the absence of the vulnerability (like eThor does).
  Hence, the table counts warnings rather than vulnerabilities.}.
As the tools tackle differing subsets of the SWC classes, the number of tools addressing a specific weakness varies from one to seven.
Due to our cumulative counting, the frequency of a weakness increases with the number of tools claiming to detect it, especially with overreporting tools.

\autoref{fig:swc_classes} depicts the 15 SWC classes on the timeline of 14\,M blocks.
For every SWC class, a data point represents the percentage of skeleton bytecodes in a bin of 100\,k blocks that were marked with the respective weakness by at least one tool.
The top plot shows the classes detected by four or more tools (SWC\,101, 105, 107, 114, 116), the middle one those handled by two or three tools (SWC\,104, 106, 112, 113, 124), and the third one those addressed by just one (SWC\,110, 115, 120, 127, 128).
In accordance with our discussion of Vandal above, we omit its findings from the plots, as its excessive reporting for SWC\,104 and 107 would distort the picture.

We see five weaknesses decrease over time from a high ($\geq$~\np[\%]{50}) or medium (\np[\%]{20}) level to a medium or low ($\leq$~\np[\%]{10}) level:
The findings of classes 101, 104, 107, 110, and 114 start to fall from about block~4\,M onwards.
The other 10 weaknesses stay on a steady, but low level after block~4\,M, except for 113 (middle plot), which fluctuates around \np[\%]{10} and 116 (top plot), which fluctuates around \np[\%]{20}.

The decline of potential integer overflows (101) seems plausible: Since version 0.8.0, the Solidity compiler adds appropriate checks automatically, and already some time before, the use of math libraries with the same effect had become quasi-standard.
Reentrancy (107) is probably the most (in)famous vulnerability.
The decrease in detection can be attributed at least partially to developers taking adequate precautions.

\newpage
\begin{mdframed}[style=mpdframe]
  \textbf{Observation 2.}
  Of the 37 SWC classes, 15 are covered by at least one tool, and 7 by at least three tools.
  For all weaknesses, the number of flagged contracts decreases over time or stagnates on a low level.
  The decreasing detection rates can be attributed to unmaintained tools that do not adequately cope with newer EVM instructions as well as to compilers and programmers taking counter-measures.
  At the end of the timeline, \emph{integer bugs} (SWC\,101), \emph{reentrancy} (SWC\,107) and \emph{block values as a proxy for time} (SWC\,116) are the most frequently detected weaknesses with a share of about 20\,\% each.
\end{mdframed}
