\section{Study Design}
\label{sec:design}

Our study aims to provide a comprehensive perspective on the evolution of weaknesses in smart contracts, as detected by automated analysis tools.
We focus on the Ethereum blockchain, which is the major platform for smart contracts in terms of the number of applications, market cap, attacks, and countermeasures. 
To address the research questions outlined below, we proceed as follows.


\ITEM{Collecting the Contract Data.}
The study period covers the 14 million blocks from Ethereum's start on 30~July 2015 up to 13~Jan 2022.
During this period, there were 48 million contract deployments. 
Analyzing the contracts in their entirety is not only infeasible, but wastes resources and introduces biases, as the deployments range from one-of-a-kind contracts to code deployed hundreds of thousands of times.
In \autoref{sec:data}, we introduce the \emph{skeleton} of contracts.
By grouping contracts with identical skeletons and selecting only one contract per group, we capture the diversity of the Ethereum ecosystem by analyzing just \np{248328} contracts.
We assess various properties of the data needed later on. 

\ITEM{Selecting the Analysis Tools.}
Performing a large-scale analysis on smart contracts that, in general, are only available as bytecode, restricts the number of applicable tools.
In \autoref{sec:tools}, we specify the selection criteria and apply them to the \np{140} tools identified by \cite{Rameder2022}.
We describe the 12 tools that remain regarding engineering aspects and their approach to contract analysis.

\ITEM{Analyzing the contracts.}
To execute 12 tools with diverse requirements and I/O formats on \np{248328} contracts, we select the execution framework~\emph{SmartBugs}.
Initially, it contained only half of the needed tools and required Solidity source code as input.
We extended SmartBugs to accept bytecode as well and added the other six tools.
After a cumulative execution time of 30 years, we obtain three million records, each specifying the result of running a single tool on a specific bytecode.
We refer to \autoref{sec:execution} for the details regarding the execution framework.

\ITEM{Mapping the Weaknesses to a Common Taxonomy.}
The execution data allows us to analyze the detection results per tool.
To facilitate the comparison and aggregation of results from multiple tools, we map the tool findings to a common taxonomy that is described in \autoref{sec:weaknesses}.

\ITEM{}%
Based on the results of running the analyzers on the bytecodes, we address the following research questions.

\ITEM{\bfseries RQ1 Abstraction.} \emph{How well are skeletons suited as an abstraction of functionally similar bytecode in the context of weakness analysis?}
In \autoref{sec:rq1}, we investigate whether and how the weakness analysis of bytecodes with the same skeleton differs.

\ITEM{\bfseries RQ2 Weakness Detection.} \emph{Which trends in the weakness reports of analysis tools can be identified for the contracts on Ethereum's main chain?}
In \autoref{sec:rq2}, we are interested in the evolution of types and numbers of weaknesses reported for the deployments up to early 2022.

\ITEM{\bfseries RQ3 Tool Quality.} \emph{How do analysis tools change their behavior in a weakness analysis with bytecode input?}
In \autoref{sec:rq3}, we investigate the evolution of tool quality with respect to maintenance aspects, execution time, errors, and failures. 

We do not assess the individual performance of the tools, like the rates of true/false positives/negatives, as there is no ground truth that is sufficiently large, consistent and balanced for a conclusive evaluation~\citep{MdAGS2023GT}.

\ITEM{\bfseries RQ4 Overlap Analysis.} \emph{To which extent do the tools agree when addressing similar weaknesses?}
In \autoref{sec:rq4}, we determine the overlap of tools for weaknesses that can be mapped to the same class of the SWC registry.

\ITEM{Discussion.}
In \autoref{sec:discussion}, we combine the results of our research questions.
For two specific SWC classes, we investigate how the agreement of the tools evolve over time, and provide an explanation.
Moreover, we consider the limitations of our study.

\ITEM{Related work.}
\autoref{sec:related} gives an overview of studies similar to ours and highlights the differences.
Moreover, it compares two execution frameworks and justifies our decision for using SmartBugs.