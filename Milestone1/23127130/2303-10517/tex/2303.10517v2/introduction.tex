\section{Introduction}
\label{sec:introduction}

\emph{Smart contracts} are event-driven programs running on the nodes of decentralized networks known as \emph{blockchains}.
Specific transactions, once included in the blockchain, trigger the execution of these blockchain programs.
Every node executes the code locally within a virtual machine and updates its state of the blockchain.
The computations are deterministic, ensuring that all nodes arrive at the same state.
The flexibility of smart contracts and the unique properties of blockchains, most notably decentralization and immutability, gave rise to innovative applications in areas like decentralized finance and supply chain management.
Their potential has led to ecosystems with large numbers of start-ups and market caps of hundreds of billions of USD.

Against this background, errors in smart contracts can lead, and have led, to costly disruptions and losses.
Early on, academia and industry focused on methods and tools for developing \emph{secure} smart contracts.
In a survey on automated vulnerability detection conducted in mid-2021, \cite{Rameder2022} identified 140 tools for Ethereum, the major smart contract platform.
The sheer number makes it hard to decide which tools may be suited for the task at hand, and calls for regular tool evaluations and comparisons.

In this paper, we present a comprehensive evaluation of 12 tools for vulnerability detection on the Ethereum main chain. The goal of this study is to analyze how typical tools behave within the Ethereum ecosystem. We focus on the evolution of tools and their findings to identify common patterns and trends. The results of our study can be utilized to inform developers about the state of the art in automated vulnerability detection and to guide researchers in the development of new tools. Additionally, it provides an overview of the reliability of the selected tools and whether they should be included in future studies.

Our work differs from previous studies in several aspects. First, we analyze the \emph{temporal evolution of weakness detection}, focusing on the robustness of tools over time (rather than assessing their detection capabilities against a set of contracts).

Second, we aim at a \emph{complete coverage of the Ethereum main chain}, which is a formidable endeavor in light of 48 million deployments of smart contracts (up to Jan 2022).
This enables us to investigate the evolution of weakness detection over a period of more than six years.
We select one contract per skeleton of bytecode (cf.\ \autoref{sec:data}), which reduces the number of objects to analyze to \np{248328}.

Third, we concentrate on the \emph{runtime bytecode} as input to the tools.
Surveys usually evaluate tools on benchmarks of Solidity source code (cf.\ \autoref{sec:related}), and hence omit tools analyzing bytecode only.
Moreover, for many contracts on the blockchain, the source code is not available.
By choosing runtime bytecode as the least denominator, we can include tools rarely considered, and we are able to analyze all smart contracts deployed so far.

Finally, to perform our study, we extend SmartBugs~\citep{ferreira2020smartbugs}, a framework for executing analysis tools in a unified manner.
Integrating new tools into the framework makes them available for future evaluations by others.

With 12 tools, 15 weakness classes, \np{248328} runtime bytecodes of smart contracts and an execution time of 30 years, our evaluation is more comprehensive than previous studies.
The large number of samples as well as the method of selection allows us to add a unique temporal perspective.
In summary, the contributions of this paper are:
\begin{itemize}
\item A method for selecting a feasible number of smart contracts that are representative of \np[M]{48} blockchain programs deployed on Ethereum in the course of six years.
\item A public dataset of \np{248328} smart contracts that may serve as the basis of further evaluations.%
  \footnote{Available at \url{https://github.com/gsalzer/skelcodes}}
\item An extension of the framework SmartBugs to include 12 tools for vulnerability detection with bytecode-only input.%
  \footnote{Available at \url{https://github.com/smartbugs/smartbugs}}
\item Methods for analyzing and visualizing the temporal evolution of tool results and the overlap between tools.
\item A portrait of the evolution of tool behavior and weakness detection on six years of blockchain activity.
\end{itemize} 
