Blockchain programs manage valuable assets like crypto-currencies and tokens, and implement protocols for decentralized finance (DeFi), logistics and logging, where security is important. To find potential issues, numerous tools support developers and analysts. Being a recent technology, blockchain technology and programs still evolve fast, making it challenging for tools and developers to keep up with the changes.

In this work, we study the evolution of tools and patterns detected. We focus on Ethereum, the crypto ecosystem with most developers and most contracts, by far. We investigate the changes in the tools’ behavior in terms of detected weaknesses, quality and behavior, and agreements between the tools. 

We are the first to fully cover the entire body of deployed bytecode on the Ethereum mainchain. We achieve full coverage by considering bytecodes as equivalent if they share the same skeleton. The skeleton of a bytecode is obtained by omitting functionally irrelevant parts. This reduces the 48 million contracts deployed on Ethereum to 248\,328 contracts with distinct skeletons.
For bulk execution, we utilize the open-source framework SmartBugs that facilitates the analysis of Solidity smart contracts, and enhance it to also accept bytecode as the only input. Moreover, we integrate six further tools that accept bytecode.
The execution of the 13 included tools took 31 years in total.
While the tools are reporting a total of 1\,307\,486 potential weaknesses, over time we observe a decreasing number of reported vulnerabilities and tools degrading to varying degrees.