\section{Conclusion}
\label{sec:conclusion}
In this work, we study the evolution of both, analysis tools and the findings they report.
Regarding tools, we select open-source tools that take runtime bytecode as input to detect weaknesses.
As for the tested blockchain programs, we aim at a full coverage of the smart contracts deployed on the Ethereum main chain up to block 14\,M (48\,M contracts). 
By considering only smart contracts with different skeletons, we manage to scale to the entire blockchain. 
In total, we run 13 tools on \np{248328} contracts with an execution time of 31 years.

We show that skeletons are a suitable abstraction for performing bytecode analysis, particularly for tools that do not require any dead code or meta-data. 
Moreover, we analyze the evolution of detected weaknesses as well as tool failures and errors and investigate agreements between the tool findings.
We detect a total of \np{1307484}  weaknesses with most of them related to \emph{\SWC{107}} (\percent{184610}{1307484}), \emph{\SWC{104}} (\percent{183277}{1307484}) and \emph{\SWC{101}} (\percent{123255}{1307484}).
Interestingly, the frequency of these weaknesses is declining over time, partly due to better developer awareness and compiler improvements, but also due to increasing tool failures.

We observe that type errors are a frequent cause of tool failure. 
Additionally, we noticed that execution times for failed executions increased significantly, suggesting that reasonable timeouts can be set while still obtaining useful findings (this is particularly important for CI/CD).
Our study indicates that there is still room for improvement regarding automated weakness detection.

The overlap analysis revealed a low agreement between the tools.
This is mainly due to their differing definitions of the weaknesses.
Therefore, it is beneficial to use a range of tools even for the same weakness class, since the tools will deliver complementary results in many cases.

Our work has already contributed to the community.
We contacted several tool developers by filing issues (Conkas, Maian, Mythril) as well as exchanging emails and engaging in discussions (eThor, Ethainter, MadMax, Mythril, Osiris, Vandal).
The extension of SmartBugs to also accept bytecode as input is already taken up by the framework Centaur\footnote{\url{https://github.com/mchara01/centaur}}.

\section*{Data Availability}
Our code and data are available at \url{https://figshare.com/s/5efef6335fa98ddc3ae2}.
We believe that these are valuable assets for driving reproducible research in automated analysis of smart contracts. 
Moreover, developers of tools that analyze EVM bytecode can use our dataset of skeleton bytecodes for high-coverage testing.

\section*{Acknowledgements}
This project was partially supported by national funds through Funda\c{c}\~ao para a Ci\^encia e a Tecnologia (FCT) under project UIDB/50021/2020.
The project was also partially supported by the CASTOR Software Research Centre.