%%%%%%%%%%%% abs %%%%%%%%%%%%%%%%%%%%%
\begin{abstract}
   Spiking neural networks (SNNs) present a promising energy efficient alternative to traditional Artificial Neural Networks (ANNs) due to their multiplication-free operations enabled by binarized intermediate activations. However, this binarization leads to precision loss, hindering the SNN performance.
   In this paper, we introduce Multiple Threshold (MT) approaches to significantly enhance SNN accuracy by mitigating precision loss. We propose two distinct modes for MT implementation, depending on the membrane update rule: parallel mode and cascade mode. MT-SNN models can be efficiently trained on standard hardwares like GPUs and TPUs, while retaining the multiplication-free advantage crucial for deployment on neuromorphic devices.  Our extensive experiments on CIFAR10, CIFAR100, ImageNet, and DVS-CIFAR10 datasets demonstrate that both MT modes substantially improve the performance of single-threshold SNNs, achieving higher accuracy with fewer time steps and comparable energy consumption. Moreover, MT-SNNs outperform state-of-the-art (SOTA) results. Notably, with MT, a Parametric-Leaky-Integrate-Fire (PLIF) based ResNet-34 architecture reaches 72.17\% accuracy on ImageNet with a single time step, surpassing the previous SOTA by 2.75\% despite using 4 steps.

   % Spiking neural networks (SNNs) present a promising energy efficient alternative to traditional Artificial Neural Networks (ANNs). The energy saving of SNNs is due to the multiplication free property brought by binarized intermediate activations with reduced precision, which caused quality degradation. In this paper, we propose the Multiple Threshold (MT) approaches to improve the accuracy of SNNs by recovering the precision, depending on the membrane update rule, two modes are proposed: the parallel mode and cascade mode. The MT-SNN models can be trained on standard devices like GPU, TPU efficiently and deployed on neuromorphic devices while maintaining the multiplication-free property. Experimental results on CIFAR10, CIFAR100, ImageNet and DVS-CIFAR10 show that both MT modes can improve the performance of single threshold SNNs largely with fewer steps and similar energy consumption, and also outperform the state-of-the-art (SOTA) results considerably. Specifically, with MT, Parametric-Leaky-Integrate-Fire(PLIF) based ResNet-34 can achieve 72.17\% accuracy on ImageNet with one single step, 2.75\% higher than the previous SOTA with 4 steps.

\keywords{Spiking Neural Network \and Image Classification \and Computer Vision }
\end{abstract}
%%%%%%%%%%%%%%%%%%%%%%%%%%%%%%%%%%%%%%