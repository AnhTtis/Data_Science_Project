\vspace{-2mm}
\section{Formulation and Method}
\begin{algorithm}[t]
\vspace{-1mm}
\caption{Inception Depthwise Convolution (PyTorch-like Code)}
\label{alg:code}
\definecolor{codeblue}{rgb}{0.25,0.5,0.5}
\definecolor{codekw}{rgb}{0.85, 0.18, 0.50}
\lstset{
  backgroundcolor=\color{white},
  basicstyle=\fontsize{7.5pt}{7.5pt}\ttfamily\selectfont,
  columns=fullflexible,
  breaklines=true,
  captionpos=b,
  commentstyle=\fontsize{7.5pt}{7.5pt}\color{codeblue},
  keywordstyle=\fontsize{7.5pt}{7.5pt}\color{codekw},
}
\begin{lstlisting}[language=python]
import torch.nn as nn

class InceptionDWConv2d(nn.Module):
    def __init__(self, in_channels, square_kernel_size=3, band_kernel_size=11, branch_ratio=1/8):
        super().__init__()
        
        gc = int(in_channels * branch_ratio) # channel number of a convolution branch
        
        self.dwconv_hw = nn.Conv2d(gc, gc, square_kernel_size, padding=square_kernel_size//2, groups=gc)
        
        self.dwconv_w = nn.Conv2d(gc, gc, kernel_size=(1, band_kernel_size), padding=(0, band_kernel_size//2), groups=gc)
        
        self.dwconv_h = nn.Conv2d(gc, gc, kernel_size=(band_kernel_size, 1), padding=(band_kernel_size//2, 0), groups=gc)
        
        self.split_indexes = (gc, gc, gc, in_channels - 3 * gc)
        
    def forward(self, x):
        # B, C, H, W = x.shape
        x_hw, x_w, x_h, x_id = torch.split(x, self.split_indexes, dim=1)
        
        return torch.cat(
            (self.dwconv_hw(x_hw), 
            self.dwconv_w(x_w), 
            self.dwconv_h(x_h), 
            x_id), 
            dim=1)
\end{lstlisting}
\vspace{-2mm}
\end{algorithm}

\vspace{-3mm}
\subsection{MetaNeXt}
\vspace{-1mm}
\myPara{Formulation of MetaNeXt Block}
In ConvNeXt \cite{convnext}, for its each ConvNeXt block, the input $X$ is first processed by a depthwise convolutioin to propagate information along spatial dimensions. 
We follow MetaFormer \cite{metaformer} to abstract the depthwise convolution as a \textit{token mixer} which is responsible for spatial information interaction. Accordingly, as shown in the second subfigure in Figure \ref{fig:block}, the ConvNeXt  is abstracted as \textit{MetaNeXt} block. Formally, in a MetaNeXt block, its  input $X$ is firstly processed as
\begin{equation}
    X' = \mathrm{TokenMixer}(X),
\end{equation}
where $X, X' \in \mathbb{R}^{B \times C \times H \times W}$ with $B$, $C$, $H$ and $W$ respectively denoting batch size, channel number, height and width. 
Then the output from the token mixer is normalized
\begin{equation}
    Y = \mathrm{Norm}(X').
\end{equation}
After normalization \cite{batch_norm, layer_norm}, the features are then fed into an MLP module which consists of two fully-connected layers with an activation function between them, the same as feed-forward network in Transformer \cite{transformer}. The two fully-connected layers can also be implemented by $1 \times 1$ convolutions. Also, shortcut connection \cite{resnet, highway} is adopted. This process can be expressed by
\begin{equation}
     Y = \mathrm{Conv}_{1 \times 1}^{rC\rightarrow C}\{\sigma[\mathrm{Conv}_{1 \times 1}^{C \rightarrow rC}(Y)]\} + X,
\end{equation}
where $\mathrm{Conv}_{k \times k}^{C_i \rightarrow C_o}$ means convolution with kernel size of $k \times k$, input channels of $C_i$ and output channels of $C_o$; $r$ is the expansion ratio and $\sigma$ denotes activation function.

\myPara{Comparison to MetaFormer block} As shown in Figure \ref{fig:block}, it can be found that MetaNeXt block shares similar modules with MetaFormer block \cite{metaformer}, \eg,~token mixer and MLP. Nevertheless, a critical difference between the two models lies in the number of shortcut connections \cite{resnet, highway}. MetaNeXt block implements a single shortcut connection, whereas the MetaFormer block incorporates two, one for the token mixer and the other for the MLP.
From this aspect, MetaNeXt block can be regarded as a result of merging two residual sub-blocks from MetaFormer, thereby simplifying the overall architecture.
As a result, the MetaNeXt architecture exhibits a higher speed compared to MetaFormer. 
However, this simpler design comes with a limitation: the token mixer component in MetaNeXt cannot be complicated (\eg, Attention) as shown in our experiments (Table \ref{tab:iso}).

\myPara{Instantiation to ConvNeXt} As shown in Figure \ref{fig:block}, in ConvNeXt, the token mixer is simply implemented by a depthwise convolution
\begin{equation}
    X' = \mathrm{TokenMixer}(X) = \mathrm{DWConv}_{k \times k}^{C\rightarrow C}(X),
\end{equation}
where $\mathrm{DWConv}_{k \times k}^{C \rightarrow C}$ denotes depthwise convolution with kernel size of $k \times k$. In ConvNeXt, $k$ is set as 7 by default.


\subsection{Inception depthwise convolution}
\begin{table}[h]
\vspace{-2mm}
\setlength{\tabcolsep}{3pt}
\footnotesize
\centering
\begin{tabular}{c c c c c c c }
\whline
	\multirow{2}{*}{\makecell[c]{Kernel size \\ of DWConv}} & \multirow{2}{*}{\makecell[c]{Convolution \\ ratio}} & \multirow{2}{*}{\makecell[c]{Params \\ (M)}} & \multirow{2}{*}{\makecell[c]{MACs \\ (G)}} & \multicolumn{2}{c}{Throughput} & \multirow{2}{*}{\makecell[c]{Top-1 \\ (\%)}} \\
 ~ & ~ & ~ & ~ & Train & Inference & ~ \\
\whline
 $7\times 7$ & 1.0 & 28.6 & 4.5 & 575 & 2413 & 82.1* \\
 $5 \times 5$ & 1.0 & 28.4 & 4.4 & 675 & 2704 & 82.0 \\
 $3 \times 3$ & 1.0 & 28.3 & 4.4 & 798 & 2802 & 81.5 \\
 $3 \times 3$ & $1/2$ & 28.3 & 4.4 & 818 & 2740 & 81.4 \\
$3 \times 3$ & $3/8$ & 28.3 & 4.4 & 847 & 2762 & 81.4 \\
$3 \times 3$ & $1/4$ & 28.3 & 4.4 & 871 & 2808 & 81.3 \\
$3 \times 3$ & $1/8$ & 28.3 & 4.4 & 901 & 2833 & 80.8 \\
$3 \times 3$ & $1/16$ & 28.3 & 4.4 & 916 & 2846 & 80.1 \\
 
\whline
\end{tabular}

\caption{\textbf{Preliminary experiments based on ConvNeXt-T. } Convolution ratio means the ratio of channels to be processed by depthwise convolution while the other channels keep unchanged. Throughputs are measured on an A100 GPU with batch size of 128 and TF32. * The result is reported in ConvNeXt paper \cite{convnext}.
\label{tab:pre_exp}
}
\vspace{-5mm}
\end{table}
\myPara{Preliminary experiments on ConvNeXt-T}
We first conducted preliminary experiments based on ConvNeXt-T and report the results in Table \ref{tab:pre_exp}. Firstly, the kernel size of depthwise convolution is reduced from $7 \times 7$ to $3 \times 3$. Compared to the model with kernel size of $7 \times 7$, the one with kernel size of $3 \times 3$ enjoys $1.4 \times$ higher training throughput, but suffers a significant performance drop from 82.1\% to 81.5\%. Next, inspired by ShuffleNet V2 \cite{shufflenet_v2}, we only feed partial input channels into depthwise convolution while the remaining ones keep unchanged. The number of processed input channels is controlled by a ratio. It is found that when the ratio is reduced from 1 to $1/4$, the training throughput can be further improved while the performance almost maintains. In summary, these preliminary experiments convey two findings on ConvNeXt. \underline{Finding 1}: Large-kernel depthwise convolution is the speed bottleneck. \underline{Finding 2}: Processing partial channels is good enough in single depthwise convolution layer \cite{shufflenet_v2}. 

\begin{table}
\vspace{-3mm}
\begin{center}
\footnotesize
\setlength{\tabcolsep}{10pt}
\begin{tabular}{l|c|c}
\whline
Conv. type &  Params & FLOPs \\
\whline
Conventional conv. & $k^2C^2$ & $2k^2C^2HW$\\
Depthwise conv. & $k^2C$ & $2k^2CHW$\\
Inception dep. conv. & $(2k+9)C/8$ & $(2k+9)CHW/4$ \\
\whline
\end{tabular}
\end{center}
\vspace{-2mm}
\caption{\textbf{Complexity of different types of convolution.} For simplicity, assume input and output channels are the same, and the bias term is omitted. $k$, $C$, $H$ and $W$ denote kernel size, channel number, height and width, respectively. The parameters and FLOPs of vanilla convolution and depthwise convolution are quadratic to kernel size $k$. In contrast, Inception depthwise convolution is linear to $k$.}
\label{tab:complexity}
\vspace{-3mm}
\end{table}


\begin{figure}[t]
\vspace{-2mm}
\begin{center}
   \includegraphics[width=0.8\linewidth]{figures/two_types_of_conv.pdf}
\end{center}
\vspace{-8mm}
\caption{\textbf{Comparison of FLOPs between depthwise convolution and Inception depthwise convolution.} Inception depthwise convolution is much more efficient than depthwise convolution as kernel size increases.}
\label{fig:two_types}
\vspace{-2mm}
\end{figure}

\begin{table}[t]
\footnotesize
\centering
\setlength{\tabcolsep}{1pt}

\begin{figure*}
    \centering
    \subfloat[]{\includegraphics[width=.33\textwidth]{images/melisa_pretrain.png}\label{fig:melisa_pretrain}}
    \hfil
    \subfloat[]{\includegraphics[width=.33\textwidth]{images/finetune_cine_vs_samples.png}\label{fig:finetune_cine}}
    \hfil
    \subfloat[]{\includegraphics[width=.33\textwidth]{images/finetune_tass_vs_samples.png}\label{fig:finetune_tass}}
    
    \subfloat[]{\includegraphics[width=.33\textwidth]{images/melisa_pretrain_bert.png}\label{fig:melisa_pretrain_bert}}
    \hfil
    \subfloat[]{\includegraphics[width=.33\textwidth]{images/finetune_cine_vs_samples_bert.png}\label{fig:finetune_cine_bert}}
    \hfil
    \subfloat[]{\includegraphics[width=.33\textwidth]{images/finetune_tass_vs_samples_bert.png}\label{fig:finetune_tass_bert}}
    \caption{Training curves ($F_1$-score) for different amounts of training samples. Figures (a), (b) and (c) shows the fine-tuning results on the validation split for the biLSTM model and Figures (d), (e) and (f) for BERT. (a) and (d) are the validation scores on MeLiSA, (b) and (e) are the scores on the MuchoCine validation split and (c) and (f) the scores on the TASS validation.}
    \label{fig:finetunning}
\end{figure*}

\subsection{Classification Models}

As mentioned in Section \ref{sec:intro}, we explored the cross-domain analysis using two different neural-based classification models which are shown in Fig. \ref{fig:classification_models}.

\begin{itemize}
    \item \textbf{BiLSTM classifier}. This architecture is based on a bidirectional recurrent neural network with a LSTM unit activation and it is illustrated in Figure \ref{fig:blstm_classifier}. In this model, each word $w_t$ of the input sequence $w_1,\ldots,w_T$ is represented as a continuous vector $\mathbf{x}_t$ trough an embedding layer at the beginning of the network. This vector sequence is then forwarded to two different LSTM networks~\cite{lstm} (one forward and one backward). The outputs at the last step of these layers are then concatenated and forwarded to the linear output layer, which gives the probabilities of each class through a Softmax activation function. 
    \item \textbf{BERT classifier}. Since the emergence of the Transformer architecture~\cite{transformer}, a series of models based on self-attention mechanisms have been proposed to pre-train a language model. One of this models is the Bidirectional Encoder Representations from Transformers (BERT), illustrated in Figure \ref{fig:bert_classifier}, which consists of 12 identical transformer encoder layers. These layers contains a multi-head self-attention layer~\cite{transformer} at the input followed by a linear layer (feed forward) with some residual connections and layer normalization~\cite{layernorm} in between. As in the LSTM classifier, every word at the input of the network is represented as a continuous vector, but additionally some special tokens are added to the sentence. In particular, the \texttt{[CLS]} token is included at the start of every sentence and it is used to extract features of the entire sequence. That is, at the output of the encoder a sequence of the same length as the input is obtained but only the first vector is used as an input of the output layer, which returns the class probability.
\end{itemize}

The key difference in our analysis of these two models is that the biLSTM network was trained from scratch, whereas the BERT classifier used was pre-trained on a Language Modeling task. Specifically, the pre-trained model called BETO~\cite{beto}, which is trained on a 3 billion words corpus called Spanish Unnanotated Corpora\footnote{https://github.com/josecannete/spanish-corpora} (SUC) was used. %The SUC database is a collection of unnanotated documents, most of them extracted from the Spanish portions of the Open Parallel Corpus (OPUS) subcorpora. The OPUS project is intended to provide the community with a publicly available parallel corpus of free online data. 
Documents included in the SUC contains all the data from Spanish Wikipedia available at the time the corpus was released and all of the sources of the OPUS Project~\cite{opus} that had text in Spanish. This sources
includes United Nations and Government journals, TED Talks, Subtitles, News Stories and more. However, none of these sources are review-like documents. The total size of the corpora gathered was comparable with the corpora used in the original BERT.
% @misc{cardelino,
%     title = "Spanish Billion Words Corpus and Embeddings",
%     author = "Cristian Cardellino",
%     year = "2016",
%     month = "March",
%     URL = "https: //crscardellino.github.io/SBWCE/"
% }

\begin{table*}
    \centering
    \caption{Test results ($F_1$-score) in the fine-tuning configuration.}
    \label{tab:melisa_finetunning}
    \begin{tabular}{r|cc|cc|cc}
        & \multicolumn{2}{c|}{Amazon} & \multicolumn{2}{c|}{TASS} & \multicolumn{2}{c}{MuchoCine} \\
        (\%) & biLSTM & BERT & biLSTM & BERT & biLSTM & BERT \\\hline
        0.0 & 0.5594 & 0.5860 & 0.3470 & 0.3850 & 0.2441 & 0.2758 \\
        0.1 & 0.5643(+0.88 \%) & 0.5860(+0.00 \%) & 0.3557(+2.51 \%) & 0.4104(+6.60 \%) & 0.1868(-23.47 \%) & 0.2834(+2.76 \%) \\
        10.0 & 0.5654(+1.07 \%) & 0.5865(+0.09 \%) & 0.3761(+8.39 \%) & 0.3974(+3.22 \%) & 0.2255(-7.62 \%) & 0.3210(+16.39 \%) \\
        100.0 & 0.5662(+1.22 \%) & 0.5845(-0.26 \%) & 0.3503(+0.96 \%) & 0.4045(+5.06 \%) & 0.2616(+7.17 \%) & 0.3125(+13.31 \%) \\
    \end{tabular}
\end{table*}

\subsection{Experimental Set-Up}

The above mentioned models were used to perform CDSC using the fine-tuning and the zero-shot configurations. For the fine-tuning case, the following steps were applied:
\begin{enumerate}
    \item The model was trained using the train split of the source dataset (MeLiSA, for instance) and hyperparameter search was done with its corresponding validation portion.
    \item Once trained, the model was trained again using the training portion of the target dataset (MuchoCine, for instance), and a new hyperparameter search was done with the target validation split.
    \item Once retrained, the model was evaluated on the test split of the target dataset.
\end{enumerate}

The only difference between both configurations is that step 2 is omitted for the zero-shot learning. This means that evaluation on the target dataset was done without using any training sample of that dataset. As a consequence, zero-shot learning is usually more challenging than fine-tuning and tends to show lower performance. However, it can provide a better idea of the model's generalization capability.

In order to test if CDSC can be achieved from product domains to more general domains like movie reviews or tweets, we used our MeLiSA dataset as source domain and MuchoCine and TASS as target domains. We also used the Amazon dataset as target domain to keep track of the model's learning capability, although this domain is, in principle, be very similar to the source domain. 

Experiments were carried on in \texttt{Python}, and the \texttt{Pytorch} module was used to implement the model training algorithm. We also used the \texttt{Huggingface Transformers} library to load the Spanish BERT pre-trained parameters and a NVIDA GTX 1080 GPU to reduce time computation. We followed \cite{random_grid_hyperparms} to perform random grid sample hyperparameter search on the biLSTM model. The best biLSTM model found consisted in a two-layer LSTM cell with hidden dimension of 50 and an embedding matrix of dimension $60,\!000\times 300$. Dropout was used as a regularization technique with a probability of $0.1$ and Adam Optimization with a batch size of 16 and a learning rate of 1e-3 was found to give the best validation results. For the pre-trained BERT model, layer dimensions are fixed in advance (12 layers with inner dimension of 768). We also used Adam Optimization to train this model and fixed the learning rate and the batch size to 5e-5 and 16 respectively, as suggested in \cite{beto}.




\caption{\textbf{ Configurations of \modelname{} models} which have similar model configurations to ConvNeXt \cite{convnext}. ``A'', ``T'', ``S'' and ``B'' represent ``Atto'', ``Tiny'', ``Small'' and ``Base'', respectively.
}
\label{tab:model}
\vspace{-9mm}
\end{table}





\myPara{Formulation}
Based on the above findings, we propose a new type of convolution to keep both accuracy and efficiency. According to \underline{Fingding 2}, we leave partial channels unchanged and denote them as a branch of identity mapping. Motivated by \underline{Fingding 1}, for the processing channels, we propose to decompose the depthwise operations in Inception style \cite{inception_v1, inception_v3, inception_v4}. 
Inception \cite{inception_v1} utilizes several branches of small kernels (\eg,~$3 \times 3$) and large kernels (\eg,~$5 \times 5$). Similarly, we adopt $3 \times 3$ as one of our branches but get rid of the usage of the large square kernels because of their slow practical speed.  Instead, large kernel $k_h \times k_w$ is decomposed as $1 \times k_w$ and $k_h \times 1$ inspired by Inception v3 \cite{inception_v3}. 

Specifically, for input $X$, we split it into four groups along the channel dimension,
\begin{equation}
\begin{split}
X_\mathrm{hw}, X_\mathrm{w}, X_\mathrm{h}, X_\mathrm{id} &= \mathrm{Split}(X) \\
&= X_{:, :g}, X_{:, g:2g}, X_{:, 2g:3g}, X_{:, 3g:} ,
\end{split}
\end{equation}
where $g$ is the channel numbers of convolution branches. We can set a ratio $r_g$ to determine the branch channel numbers by $g = r_g C$. Next, the splitting inputs are fed into different parallel branches,
\begin{equation}
\begin{split}
X'_\mathrm{hw} &= \mathrm{DWConv}_{k_s \times k_s}^{g\rightarrow g}(X_\mathrm{hw}), \\
X'_\mathrm{w} &= \mathrm{DWConv}_{1\times k_b}^{g\rightarrow g}(X_\mathrm{w}), \\
X'_\mathrm{h} &= \mathrm{DWConv}_{k_b\times 1}^{g\rightarrow g}(X_\mathrm{h}), \\
X'_\mathrm{id} &= X_\mathrm{id}, \\
\end{split}
\end{equation}
where $k_s$ denotes  the small square kernel size set as 3 by default;   $k_b$ represents the band kernel size set as 11 by default. Finally, the outputs from each branch are concatenated,
\begin{equation} \label{eq1}
X' = \mathrm{Concat}(X'_\mathrm{hw}, X'_\mathrm{w}, X'_\mathrm{h}, X'_\mathrm{id}).
\end{equation}
The illustration of \modelname{} block is shown in Figure \ref{fig:block}. Moreover,  its  PyTorch \cite{pytorch} code is summarized  in Algorithm \ref{alg:code}.




\myPara{Complexity} The complexity of three types of convolution, \ie, conventional, depthwise, and Inception depthwise convolution is shown in Table \ref{tab:complexity}. As can be seen, Inception depthwise convolution is much more efficient than the other two types of convolution in terms of parameter numbers of FLOPs. Inception depthwise convolution consumes parameters and FLOPs linear to both channel and kernel size. The comparison of depthwise and Inception depthwise convolutions regarding FLOPs is also clearly shown in Figure \ref{fig:two_types}. 




\vspace{-1mm}
\subsection{\modelname{}}
\vspace{-1mm}
Based on InceptionNeXt block, we can build a series of models named InceptionNeXt. Since ConvNeXt \cite{convnext} is the our main comparing baseline, we mainly follow it to build models with several sizes \cite{rw2019timm}. Specifically, similar to ResNet \cite{resnet} and ConvNeXt, \modelname{} also adopts 4-stage framework.  The same as ConvNeXt, the numbers of 4 stages are [2, 2, 6, 2] for atto size,  [3, 3, 9, 3] for small size and [3, 3, 27, 3] for base size. 
We adopt Batch Normalization since this paper emphasizes speed.  Another difference with ConvNeXt is that \modelname{} uses an MLP ratio of 3 in stage 4 and moves the saved parameters to the classifier, which can help reduce a few FLOPs (\eg,~3\% for base size). The detailed model configurations are reported in Table \ref{tab:model}.


