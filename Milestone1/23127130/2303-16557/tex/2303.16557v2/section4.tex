\section{Conclusion}\label{sec:conclusion}
This work studied the Sauvegrain-based BAA and identified issues with DNNs trained for the MV-MT ordinal classification problem. Ensembling CNNs increases computational costs and vanilla ViT leads to anisotropic attention and prediction discrepancies. To address these issues, this work introduced SAT, consisting of token replay and regional attention bias techniques, which were effective in mitigating these problems. This approach has broader implications for training ViT for MV-MT ordinal classification. Applied to Sauvegrain-based BAA, SAT is clinically meaningful in assisting diagnosis of precocious and delayed maturity in adolescents.