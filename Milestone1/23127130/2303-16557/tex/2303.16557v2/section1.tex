\section{Introduction}\label{sec:introduction}
 \begin{figure*}[t]
	\centering 
 \subfigure[\label{fig:landmark} ]
    {\hspace{1mm}\includegraphics[width=0.58\columnwidth]{./new_figures/fig_sauvegrain_overview.pdf}\hspace{1mm}}
     \subfigure[\label{fig:corr} ]{\hspace{1mm}\includegraphics[width=0.37\columnwidth]{./new_figures/fig_correlation.pdf}\hspace{1mm}}
     \caption{(a) An example of labeling procedure for elbow radiographs in the anterior posterior (AP) and lateral view points. The total score is calculated by summing each maturity score, with the expectation of two proximal landmarks. In this case, the total score is calculated as 9 + 5 + (6 + 6) / 2 + 7 = 27. (b) Pearson correlation coefficient of inter-landmark maturity scores.
    }
\end{figure*}

Bone age assessment (BAA) is widely used to diagnose precocious or delayed puberty.
Deep neural network (DNN) has demonstrated remarkable success in a wide range of safety-critical applications, and it also has been actively adopted in BAA with successful applications to computer-aided diagnosis systems~\cite{lee2017fully,spampinato2017deep}.
To develop an application that predicts bone age, the DNN can be trained with several clinical criteria, including Greulich-Pyle \cite{greulich1959radiographic}, Tanner-Whitehouse \cite{tanner2001assessment}, and Sauvegrain method \cite{sauvegrain1962study}.
Among these, the Sauvegrain method, which evaluates the skeletal age based on four elbow landmarks (i.e., lateral condyle, trochlea, proximal, olecranon) from two different views (see Fig \ref{fig:landmark}), is more adequate for the age of puberty \cite{ahn2021assessment,dimeglio2005accuracy}.
Specifically, the Sauvegrain method has an important trait in that the relationship among landmarks has a correlation on the label space (see Fig. \ref{fig:corr}).

A recent study has used deep learning algorithms to apply the Sauvegrain method by training an individual convolutional neural network (CNN) to assess the maturity point for each landmark \cite{ahn2021assessment}.
This approach involves each classification model inferring a score for a region of interest (RoI), and the aggregate of scores from multiple models is used to determine skeletal age.

Although the previous study shows that the classification performance of the CNNs is comparable to experts, their approach has two limitations;
First, while ground-truth labels between inter-landmarks have a strong correlation, the incorporated model that each classifier trained with single landmark images can produce misclassified predictions with high variance. These inaccurate predictions may confuse radiologists when interpreting the model’s decisions.
Second, the previous method requires excessive computational costs both training and inference because multiple landmark networks should be trained independently.

To address the above issues, this study poses a novel approach to solving the multi-view and multi-task problem for Sauvegrain-based BAA using a vision transformer (ViT) \cite{vit} instead of an ensemble of single-view CNNs.
By leveraging an attention mechanism in ViT, the model learns effective relations within input sequences which consist of multi-view inputs.
Moreover, we can reduce the number of parameters and computational costs by adopting shallow RoI-specific classifiers at the top of the shared encoder.
Although ViT has been applied to multi-view \cite{mvt,neimark2021video,sun2022transformer} and multi-task \cite{multitask_vit} problems, but not when they coexist.

However, we find that the vanilla ViT trained with a multi-view and multi-task (MV-MT) manner suffers from poor optimization and generalization. 
One of the reasons is that inter-landmarks are often excessively accentuated.
As a result, anisotropic behavior in the attention layer leads to sub-par classification performance.
To overcome the above challenge, we propose the \textit{self-accumulative vision transformer} (SAT) that accumulates their intra-region information by two components: (1) \textit{token replay} that prevents semantic representations of tokens with the same landmark from being overwhelmed by other regional tokens by using residual connections between class tokens and their corresponding regional tokens, and (2) \textit{regional attention bias (RAB)}, modified self-attention mechanism, to impose an intra-region attention.
Despite having significantly fewer parameters, the proposed SAT predicts maturity scores across landmarks much more accurately and outperforms other state-of-the-art models on most landmarks.