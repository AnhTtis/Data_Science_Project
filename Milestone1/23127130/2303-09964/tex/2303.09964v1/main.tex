\documentclass{article}
\usepackage[utf8]{inputenc}
\usepackage{latexsym}
\usepackage{amsmath}
\usepackage{amssymb}
\usepackage{amsthm}
\newtheorem{theorem}{Theorem}
\newtheorem{axiom}[theorem]{Axiom}
\newtheorem{law}[theorem]{Law}
\newtheorem{lemma}[theorem]{Lemma}
\newtheorem{conjecture}[theorem]{Conjecture}
\newtheorem{corollary}[theorem]{Corollary}
\newtheorem{question}[theorem]{Question}
\newtheorem{exercise}[theorem]{Exercise}
\newtheorem{system}[theorem]{System}
\newtheorem{definition}[theorem]{Definition}
\newtheorem{assumption}[theorem]{Assumption}
\newtheorem{principle}[theorem]{Principle}
\newtheorem{fact}[theorem]{Fact}
\newtheorem{rmk}[theorem]{Remark}
\newtheorem{factrule}[theorem]{Rule}
\newtheorem{example}[theorem]{Example}
\newtheorem{test}[theorem]{Test}
\newtheorem{method}[theorem]{Method}
\newtheorem{prop}[theorem]{Proposition}
\usepackage{todonotes}
\usepackage{hyperref}
\usepackage{enumitem}
\makeatletter
\newcommand*{\rom}[1]{\expandafter\@slowromancap\romannumeral #1@}
\newcommand{\upperRomannumeral}[1]{\uppercase\expandafter{\romannumeral#1}}
\newcommand{\lowerromannumeral}[1]{\romannumeral#1\relax}
\hypersetup{
	backref,colorlinks=true,linkcolor=blue,citecolor=blue,urlcolor=blue,citebordercolor={0 0 1},urlbordercolor={0 0 1},linkbordercolor={0 0 1}}
 
\usepackage[utf8]{inputenc}
\newcommand{\ZZ}{\mathbb{Z}}
\newcommand{\QQ}{\mathbb{Q}}
\newcommand{\CC}{\mathbb{C}}
\newcommand{\RR}{\mathbb{R}}
\newcommand{\PP}{\mathbb{P}}
\newcommand{\XX}{\mathbb{X}}
\newcommand{\dd}{\delta}

\title{Almost toric presentations of symplectic log Calabi-Yau pairs }

\author{Tian-Jun Li, Jie Min, Shengzhen Ning}

\AtEndDocument{\bigskip{\footnotesize
		\textsc{School of Mathematics, University of Minnesota, Minneapolis, MN, US} \par
		\textit{E-mail address}: \texttt{tjli@math.umn.edu} \par
		\addvspace{\medskipamount}
		\textsc{Department of Mathematics and Statistics, University of Massachusetts, Amherst, US} \par
		\textit{E-mail address}: \texttt{minxx127@umn.edu} \par
		\addvspace{\medskipamount}
		\textsc{School of Mathematics, University of Minnesota, Minneapolis, MN, US} \par
		\textit{E-mail address}: \texttt{ning0040@umn.edu} \par
		
}}
\begin{document}
\maketitle
\begin{abstract}
    It's known (\cite{symington}) that the union of fibers over elliptic singularities of an almost toric fibered (ATF) symplectic four-manifold gives a symplectic log Calabi-Yau (LCY) divisor. In this note, we show the converse also holds: any symplectic log Calabi-Yau divisor can be realized as the boundary divisor of an almost toric fibration. This realization is canonical once we choose an extra data called the framing on the space of LCY. This is achieved by considering the symplectic analogue of the toric model used by \cite{GHK} in the algebraic geometrical settings.
\end{abstract}

\tableofcontents
\section{Introduction}
Almost toric fibration (ATF) on symplectic four-manifold, which is a Lagrangian fibration with only focus-focus and elliptic singularities, was introduced by Symington in \cite{symington}. Leung and Symington also gave a classification of almost toric fibered closed manifolds up to diffeomorphisms in \cite{LS}. The possible manifolds are rational surfaces, ruled surfaces of genus $1$, $K3$ surface, Enriques surface and some torus bundles over torus with specific monodromy. Among the list of their classifications, rational manifolds should be viewed as the most visible ones since their bases are disks (possibly with nodes) which can be drawn in the plane (possibly after some cutting). There are lots of applications using almost toric fibrations on rational manifolds, see \cite{brendlschlenk}, \cite{CasalVianna}, \cite{CGHMP-toric-staircase} for symplectic embedding problems, \cite{Vianna1}, \cite{viannadelpezzo}, \cite{LeeOhVianna}, \cite{flux} for the study of Lagrangian torus and \cite{evans} for a comprehensive introduction.

 There are some other related notions that have been studied thoroughly: symplectic toric manifold (\cite{Delzant}), Hamiltonian $S^1$-space (\cite{KarshonS1}) and semitoric integrable system (\cite{PV2},\cite{PV1}). They are classified up to isomorphisms by Delzant polygons, Karshon's decorated graphs and Pelayo and Ng\d oc's five invariants. These results tell us more than just the diffeomorphism types of the underlying manifolds. For example, it's well known that not every symplectic form admits a toric action, or Hamiltonian $S^1$-action, in particular, semi-toric structure. Actually the symplectic forms admitting such structures can be computed explicitly in small blowups (section 3.4 in \cite{Enumerate}). However, almost toric fibrations seem to be a more flexible notion since they have a large diversity of base diagrams. Even for $\CC\PP^2$, by mutations there are infinitely many base diagrams each of which corresponds to a Markov triple. In this note we are curious about the symplectic forms on rational manifolds admitting an almost toric fibration. It turns out the only requirement is very simple:
 \begin{theorem}[Corollary \ref{maincor}]\label{thm1}
     Every symplectic rational manifold $(X,\omega)$ with $\omega\cdot c_1(X,\omega)>0$ admits an almost toric fibration.
 \end{theorem}

 Such a condition is natural since by \cite{symington} Proposition 8.2, the union of fibers over $0$ and $1$-strata will form some symplectic submanifolds whose intersections are transvers and the sum of homology classes is Poincar\'e dual to $c_1(X,\omega)$. They are exactly symplectic log Calabi-Yau divisors introduced in \cite{limak}, which is the symplectic analogue of the ones in algebraic geometry studied by \cite{Looijenga},\cite{GHK} and \cite{Friedman}. In fact, the above result is the corollary of the following:

  \begin{theorem}[Theorem \ref{thm:main}]
     Every symplectic log Calabi-Yau divisors can be realized as the boundary divisor of some almost toric fibration.
 \end{theorem}

Engel and Friedman (\cite{Engel2021} Section 5) gave a proof of the above result in settings of algebraic toric geometry. However they only assumed $\omega\in H^2(X;\QQ)$ to be a nef divisor, which might be a degenerate form. Also their choice of almost toric realization is not canonical. Our proof works for all symplectic forms and is written in purely symplectic language. Besides, Theorem \ref{thm:main} actually says more than existence: once we fix a framing we will have a canonical choice of realizations.

 \begin{figure*}[h]
		\includegraphics*[width=\linewidth]{game}
  \caption{Packings of possibly folded triangles in Markov's $(a,b,c)$-triangles. }\label{fig:game}
	\end{figure*}

\begin{rmk}
One might try to prove Theorem \ref{thm1} directly starting from the ATF on $\CC\PP^2$, where the diagram is given by perfoming three nodal trades on the standard toric moment triangle. Assume the perimeter of the triangle is $3$, given positive numbers $\dd_1,\dd_2,\cdots,\dd_n$ with $\sum_i\dd_i<3$ (so that $c_1\cdot \omega>0$), one can play the game of packing possibly folded triangles with sizes $\dd_i$'s into the diagram for fun. More generally, by using Markov's $(a,b,c)$-triangles coming from mutations considered in \cite{Vianna1}, This works when $\dd_i\leq \frac{1}{3}$ for all $i$'s because the longest edge has length $\frac{a}{bc}$ and the height is $\frac{bc}{a}$ (see \cite{brendlschlenk} and their intepretation of this number as relative Gromov width). One can then take a sequence of Markov triples $(a_n,b_n,c_n)$ such that $\lim_{n\rightarrow\infty}\frac{a_n}{b_nc_n}=3$. However it still seems to be difficult to prove in this way for all general cases. See figure \ref{fig:game}.
\end{rmk}

The structure of this paper is as follows. In section \ref{section:lcy}, we review symplectic log Calabi-Yau divisors studied in \cite{limak} and \cite{Enumerate} in a more succinct way. The new inputs are the concepts of $\varepsilon$-replacement and symplectic toric model which will be used later. Then we introduce the base diagrams realizing all LCY's in section \ref{section:bd}. A careful argument is given to show the relation between the base diagrams and divisors. After some preparations dealing with some technical issues in section \ref{section:elementarygeometry}, we finally give the proof of our main theorem in section \ref{section:proof}. Throughout this paper, we use the convention $H_{i_1i_2\cdots}$ to denote the homology class $H-E_{i_1}-E_{i_2}-\cdots$.

\hfill \break
{\bf Acknowledgement:} The authors would like to thank Philip Engel, Jonny Evans, Margaret Symington and Weiwei Wu for helpful communications.

\section{Preliminaries on $\mathcal{LCY}$}\label{section:lcy}

We rapidly review some necessary backgrounds of LCY. More details can be found in \cite{limak} and \cite{Enumerate}. A symplectic log Calabi-Yau divisor in a symplectic rational manifold is a cycle of symplectically embedded spheres whose intersections are positively transverse and the sum of homology classes are Poincar\'e to the first Chern class. After some deformation the intersections can also be made orthogonal. By a symplectic log Calabi-Yau pair (or simply pair), we mean a triple $(X,\omega,D)$ where $D$ is a symplectic log Calabi-Yau divisor in $(X,\omega)$. In this paper we only consider Looijenga pairs which requires $D$ has at least two components and always assume $D$ has orthogonal intersections. A pair $(X,\omega,D)$ is called toric if its charge $q(D):=12-D^2-$length of $D$ is $0$. Two pairs are said to be isomorphic if there is a symplectomorphism which also preserves divisors. We use $[X,\omega,D]$ to denote the isomorphism class of $(X,\omega,D)$.

Let $\mathbb{LCY}$ denote the set of the isomorphism classes of all symplectic log Calabi-Yau pairs. This set can be naturally equipped with a quiver structure since many different pairs are related by blowup operations: we put an arrow going from $A$ to $B$ whenever there exists $(X,\omega,D)$ in class $A$ and its toric or non-toric blowup $(X',\omega',D')$ in class $B$.\footnote{Two vertices can only have at most one arrow between them.} Furthermore, we can take the free category $\mathcal{LCY}$ generated by this quiver $\mathbb{LCY}$ (\rom{2}.7 in \cite{cat}) : the objects of the category are the vertices of the quiver, the morphisms are paths between objects and the composition operation is given by concatenation of paths. For example, given two paths of blowup procedures 
%Let $\mathcal{LCY}$ denote the category with objects being orthogonal LCY (Looijenga) pairs and morphisms being a path $\Gamma$ of LCY surgeries (symplectomorphisms, toric/non-toric blowups) modulo compositions of symplectomorphisms. More precisely, a morphism $\Gamma \in Hom((X,\omega,D), (X',\omega',D'))$ are sequences of LCYs $\{(X_i,\omega_i,D_i)\}_{i=0}^n$ with $(X_0,\omega_0,D_0)=(X,\omega,D), (X_n,\omega_n,D_n)=(X',\omega',D')$ and each $(X_{i+1},\omega_{i+1},D_{i+1})$ is obtained from $(X_i,\omega_i,D_i)$ by performing a toric/non-toric blowup/blowdown or a symplectomorphism. And we identify two consecutive symplectomorphisms:
 %\begin{align*}
%\Gamma
%&=\biggl \langle	\cdots\rightarrow (X_{i},\omega_i,D_i)\xrightarrow[]{f}(X_{i+1},\omega_{i+1},D_{i+1})\xrightarrow[]{g}(X_{i+2},\omega_{i+2},D_{i+2})\rightarrow\cdots\biggl \rangle\\
%&=\biggl \langle\cdots\rightarrow (X_i,\omega_i,D_i)\xrightarrow[]{g\circ f}(X_{i+2},\omega_{i+2},D_{i+2})\rightarrow \cdots\biggl \rangle.
%\end{align*}
\[[X_0,\omega_0,D_0]\xrightarrow{f_1}\cdots\xrightarrow{f_n}[X_n,\omega_n,D_n],\,\,\, [X_n,\omega_n,D_n]\xrightarrow{f_{n+1}}\cdots\xrightarrow{f_m}[X_m,\omega_m,D_m]\]
where each $f_i$ could be either toric or non-toric blowup, we then get morphisms in $Hom([X_0,\omega_0,D_0],[X_n,\omega_n,D_n])$ and $Hom([X_n,\omega_n,D_n],[X_m,\omega_m,D_m])$ and they can be composed to give a morphism in $Hom(([X_0,\omega_0,D_0],[X_m,\omega_m,D_m]))$
\[[X_0,\omega_0,D_0]\xrightarrow{f_1}\cdots\xrightarrow{f_n}[X_n,\omega_n,D_n]\xrightarrow{f_{n+1}}\cdots\xrightarrow{f_m}[X_m,\omega_m,D_m].\]


By a {\bf framing} on a symplectic rational manifold $(X,\omega)$ where $X=\CC\PP\#n\overline{\CC\PP}^2$ with $n\geq 2$, we mean an ordered choice of the $\ZZ$-basis of its second homology group $\{H,E_1,\cdots,E_n\}$ such that they have disjoint embedded symplectic spherical representations and the first Chern class is Poincar\'e dual to $3H-E_1-\cdots-E_n$. The framing is said to be {\bf reduced} if $\omega(E_1)\geq\cdots\omega(E_n)$ and $\omega(H)\geq \omega(E_1)+\omega(E_2)+\omega(E_3)$ if $n\geq 3$; $\omega(H)\geq \omega(E_1)+\omega(E_2)$ if $n=2$.  A {\bf framing} $\mathfrak{f}$ on $\mathbb{LCY}$ is a choice of a reduced framed representative for each isomorphism class $[X,\omega,D]$ where $X=\CC\PP\#n\overline{\CC\PP}^2$ with $n\geq 2$. Such a choice always exists by \cite{KK17}.


For $n\geq 3$, let $$\widetilde{\Delta}_n:=\{(a,b_1,\cdots,b_n)\,|\,a\geq b_1+b_2+b_3, b_1\geq b_2\geq \cdots \geq b_n>0\}$$ be the reduced cone in $\RR^n$ and $$ \Delta_n:=\{(1,b_1,\cdots,b_n)\in\widetilde{\Delta}_n\,|\,b_1+\cdots+b_n<3\}$$ be a slice region which is a polytope of dimension $n-1$. We call $\Delta_n$ the $c_1${\bf -nef polytope}. There is a well-defined map $$p:\mathbb{LCY}\rightarrow \bigsqcup_{n}\Delta_n$$ by choosing a framing, performing the algorithm in \cite{KK17} and then looking at the symplectic areas. This map has finite fibers and \cite{Enumerate} gives a counting of the fiber. There is a general counting formula when the fiber is over a restrictive region of $\Delta_n$. The advantage of considering those symplectic classes is we can actually count certain morphisms (which was called `blowup pattern' in \cite{Enumerate}) from some specific objects (which was called `germ' in \cite{Enumerate}). 

Finally we recall two results from \cite{Enumerate} which will be used later.
\begin{prop}[Symplectic Torelli theorem]\label{prop:Torelli}
    If there is an integral isometry $\gamma:H_2(X';\ZZ)\rightarrow H_2(X;\ZZ)$ such that it maps $D'$ to $D$ componentwisely, and its real extension $\gamma_\RR:H_2(X';\RR)\rightarrow H_2(X;\RR)$ maps $PD([\omega'])$ to $PD([\omega])$, then $(X',\omega',D')$ is isomorphic to $(X,\omega,D)$.
\end{prop}

We will apply the above result in the following particular way. If one already knows two pairs are isomorphic, then after the toric blowup (blowdown) at the corresponding node (component) or non-toric blowup (blowdown) at the corresponding component of the same the size, the outcomes will also be isomorphic. 


\begin{lemma}[Tautness of toric pairs]\label{lem:torictaut}
    Toric elements in $\mathbb{LCY}$ are taut. That is, they are uniquely determined by its self-intersection sequence and area sequence, up to cyclic and anti-cyclic permutation.
\end{lemma}
    

\subsection{Symplectic reduced model}\label{section:reducedmodel}
By repeatedly blowing down exceptional spheres with minimal areas, \cite{limak} produces the so-called minimal model for symplectic log Calabi-Yau pairs. However, the choice of exceptional sphere with minimal area is not unique at each step. For the purpose of counting, the improvement in \cite{Enumerate} is to eliminate this non-uniqueness by adding a framing and blowing down the specific exceptional sphere at each step. So let's now fix a framing $\mathfrak{f}$ on $\mathbb{LCY}$. Recall that by Lemma 2.31 in \cite{Enumerate}, for any object $\XX=[X,\omega,D]$ in $\mathcal{LCY}$ with framing $\{H,E_1,\cdots,E_n\}\subset H_2(X;\ZZ)$, we can associate a unique morphism from a specific $[X',\omega',D']$ to it by blowing down $E_n,E_{n-1},\cdots$ until the manifold becomes $\CC\PP^2\#\overline{\CC\PP}^2$ or the length of the divisor becomes $2$ with one component having the next exceptional class $E_l$. To distinguish it from minimal model used in \cite{limak}, such a unique $[X',\omega',D']$ is called the {\bf reduced model} of $\XX$, which we denote by $\XX_{red}$. More precisely, $\XX_{red}$ is one of the followings ($X'=\CC\PP^2\#\overline{\CC\PP}^2$ in the first four cases and $X'=\CC\PP^2\#l\overline{\CC\PP}^2$ with some $l\leq n$ in the last case).
\begin{enumerate}[label=(\roman*)]
	\item $(2H,H-E_1)$
    \item $((a+1)H-aE_1,(-a+2)H+(a-1)E_1)_{a\in\ZZ_+}$
	\item $(aH+(-a+1)E_1,H-E_1,(-a+2)H+(a-1)E_1)_{a\in\ZZ_+}$
	\item $(aH+(-a+1)E_1,H-E_1,(-a+1)H+aE_1,H-E_1)_{a\in\ZZ_+}$
    \item $(3H-E_1-\cdots-E_{l-1}-2E_l,E_l).$
\end{enumerate}

The reduced model depends on the framing $\mathfrak{f}$. For symplectic classes lying on the boundary of the $c_1$-nef polytope, there could be two morphisms starting from different objects but ending with the same object. 
\begin{example}
    Consider $\CC\PP^2\#3\overline{\CC\PP}^2$ with monotone symplectic form, framing $\{H,E_1,E_2,E_3\}$ and $D=(H,H_3,H_{12})$. By a symplectomorphism generated by Dehn twist along the Lagrangian sphere in class $H_{123}$, there will be another framing $\{h,e_1,e_2,e_3\}:=\{2H-E_1-E_2-E_3,H_{23},H_{13},H_{12}\}$. Thus we will have two different reduced models depending on the choice of framings:
    \[\Gamma_1=\biggl \langle \XX^{(1)}_{red}=(H,H,H_1)\xrightarrow[]{\text{nontoric}}(H,H,H_{12})\xrightarrow[]{\text{nontoric}}(H,H_3,H_{12})\biggl \rangle\]
    \[\Gamma_2=\biggl \langle \XX^{(2)}_{red}=(2h-e_1,h)\xrightarrow[]{\text{nontoric}}(2h-e_1-e_2,h)\xrightarrow[]{\text{toric}}(2h-e_1-e_2-e_3,h_3,e_3)\biggl \rangle.\]
\end{example}

\subsection{$\varepsilon$-replacement $\XX_{\varepsilon}$}\label{subsection:replace}

%So let's first suppose the following:

%\begin{assumption}\label{ass0}
 %  $E_2$ doesn't appear as the non-toric blowup on the component $2H$ in case $(\lowerromannumeral{1})$; $(a+1)H-aE_1$ in case $(\lowerromannumeral{2})$ or $aH+(-a+1)E_1$ in case $(\lowerromannumeral{3})$. 
%\end{assumption}

%Under the assumption \ref{ass0}, 
 Throughout this section, $\varepsilon\ll$ min$\{\omega(D_i)\,|\,D_i\text{ is a component of } D\}$ is a very small positive number. Given the reduced model $\Gamma=\bigl\langle \XX_{red}\xrightarrow{f}\cdots\rightarrow \XX \bigl\rangle\in Hom(\XX_{red},\XX)$, the following diagrams \ref{fig:curve1},  \ref{fig:curve2}, \ref{fig:curve3}, \ref{fig:curve4} and \ref{fig:curve5} of curve configurations show the possible blowup patterns. 

 \begin{figure*}[]\label{fig:curve1}
		\includegraphics*[width=\linewidth]{curve1}
  \caption{When $\XX_{red}$ is (\lowerromannumeral{1}), the first sequence describes the pattern when $f$ is the non-toric blowup on the component $2H$; the second sequence describes the other cases. }\label{fig:curve1}
	\end{figure*}

  \begin{figure*}[]\label{fig:curve2}
		\includegraphics*[width=\linewidth]{curve2}
  \caption{The blowup patterns when $\XX_{red}$ is (\lowerromannumeral{2}). }\label{fig:curve2}
	\end{figure*}
 
   \begin{figure*}[]\label{fig:curve3}
		\includegraphics*[width=\linewidth]{curve3}
  \caption{The blowup patterns when $\XX_{red}$ is (\lowerromannumeral{3}). }\label{fig:curve3}
	\end{figure*}

  \begin{figure*}[]\label{fig:curve4}
		\includegraphics*[width=\linewidth]{curve4}
  \caption{The blowup patterns when $\XX_{red}$ is (\lowerromannumeral{4}). }\label{fig:curve4}
	\end{figure*}
 
  \begin{figure*}[]\label{fig:curve5}
		\includegraphics*[width=\linewidth]{curve5}
  \caption{The blowup patterns when $\XX_{red}$ is (\lowerromannumeral{5}). }\label{fig:curve5}
	\end{figure*}
 
 Note that in each case we have marked some nodes in the picture. The only subtlety arises in case (\lowerromannumeral{1}) and $f$ is the non-toric blowup on the component $2H$ since two nodes of $\XX_{red}$ have equal status and both can be the marked nodes shown in the picture. To eliminate this ambiguity, we always let $E_i$ be the exceptional class of the first toric blowup in $\Gamma$. And the second marked node is always set to be the intersection between this $E_i$ and $2H-E_i-\cdots$. 
 
 The above diagrams are all in the situation that they all contain toric blowups at the marked nodes which produce the red and blue curves. In general, $\Gamma$ may or may not contain those toric blowups. If it contains we then record their sizes $\dd_i,\dd_j$ which are the symplectic areas of some $E_i,E_j$ where $i<j$. They are called {\bf distinguished blowup sizes}. In this case we don't modify $\XX$ and just let $\XX_{\varepsilon}:=\XX$. Otherwise, there will be one or two marked nodes in the configuration diagrams which don't admit toric blowups (so that the red or blue curve in the pictures doesn't appear). We then just let $\XX_{\varepsilon}$ be the toric blowups of $\XX$ with small sizes at those nodes. In all cases other than case (\lowerromannumeral{1}), we can just take the sizes both to be $\varepsilon$. When we need two blowups in case \lowerromannumeral{1}$(a)$, the blowup sizes are chosen to be $\varepsilon$ and $\frac{\varepsilon}{2}$ for the first and second respectively. In this case we also call those $\varepsilon$ (or $\frac{\varepsilon}{2}$ in (\lowerromannumeral{1})) the {\bf distinguished blowup sizes} of $\XX_{\varepsilon}$ and let $\dd_i,\dd_j$ denote them\footnote{If the framing on $\XX$ has $\{H,E_1,\cdots,E_n\}$, $i,j$ can be understood as $n+1,n+2$.}.  
 \begin{example}
     Let's do some exercises to make our assignment $\XX_{\varepsilon}$ clearer.
     \begin{itemize}
         \item If $\XX$ is $(H_{134},E_3,2H-E_2-E_3)$, then $\XX_{\varepsilon}$ will be $(H_{1345},E_3,2H-E_2-E_3-E_5,E_5)$ where $E_5$ has area $\varepsilon$.
         \item If $\XX$ is $(H_{124},E_3,2H-E_3-E_4,E_4)$, then $\XX_{\varepsilon}$ will be $(H_{124},E_3-E_5,E_5,2H-E_3-E_4-E_5,E_4)$ where $E_5$ has area $\varepsilon$.
         \item If $\XX$ is $(H_{123},2H-E_4)$, then $\XX_{\varepsilon}$ will be $(H_{1235},E_5-E_6,E_6,2H-E_4-E_5-E_6)$ where $E_5,E_6$ has areas $\varepsilon,\frac{\varepsilon}{2}$ respectively.
           \item If $\XX$ is $(100H-99E_1-E_2,-97H+98E_1-E_3)$, then $\XX_{\varepsilon}$ will be $(100H-99E_1-E_2-E_4-E_5,E_4,-97H+98E_1-E_3-E_4-E_5,E_5)$ where $E_4,E_5$ both have areas $\varepsilon$.
     \end{itemize}
 \end{example}



\subsection{Symplectic toric model}\label{section:toricmodel}
Given $\XX=[X,\omega,D]$, we now introduce another path of blowdowns different from the ones leading to the reduced model. An exceptional class in $H_2(X;\ZZ)$ is a class which can be represented by symplectic embedded $(-1)$-sphere. And it's called {\bf non-toric} if it has trivial intersection
pairing with all but one of the homology classes of the irreducible components of $D$ and
the only non-trivial pairing is $1$.


\begin{lemma}\label{lemma:exceptional}
    Given a set $\mathcal{E}=\{S_1,\cdots,S_k\}$ of pairwise orthogonal non-toric exceptional classes, one can choose their embedded representatives $C^{S_1},\cdots,C^{S_k}$ which are pairwise disjoint and each one intersects (exactly) one component of $D$ transversally.
\end{lemma}
\begin{proof}
    One can use Theorem 1.2.7 in \cite{MO} (see also Lemma 2.4 in \cite{limak}) to obtain an $\omega$-tamed almost complex structure $J$ such that each component of $D$ and every $S_i$ are represented by $J$-holomorphic spheres. The result then follows from positivity of intersection.
\end{proof}

Assume the blowup pattern from the reduced model is given by $\Gamma=\bigl\langle \XX_{red}\xrightarrow{f_2}\XX_2\xrightarrow{f_3}\cdots\xrightarrow{f_n} \XX \bigl\rangle\in Hom(\XX_{red},\XX)$ (for case (\lowerromannumeral{5}) the index of blowup is starting from $f_{l+1}$). Let $\mathcal{N}\subset\{2,\cdots,n\}$ (or $\{l+1,\cdots,n\}$ in case (\lowerromannumeral{5})) be the index set for $f_*$ being non-toric blowup. Now for the $\varepsilon$-replacement $\XX_{\varepsilon}$, we choose such the exceptional set $\mathcal{E}:=$
\begin{enumerate}[label=(\roman*)]
    \item  
    $\begin{cases}
      \{E_1,H_{2i},H_{2j}\}\cup\{E_k\,|\, k\in\mathcal{N},k\geq 3\} , & f_2 \text{ is non-toric blowup on }2H \\
      \{E_1,H_{ij}\}\cup\{E_k\,|\, k\in\mathcal{N}\}, & \text{otherwise}
    \end{cases}$

    \item  
    $\begin{cases}
      \{H_{12},H_{1i},H_{1j}\}\cup\{E_k\,|\, k\in\mathcal{N},k\geq 3\} , & f_2 \text{ is non-toric blowup on }(a+1)H-aE_1 \\
      \{H_{1i},H_{1j}\}\cup\{E_k\,|\, k\in\mathcal{N}\}, & \text{otherwise}
    \end{cases}$
    
  \item  
    $\begin{cases}
      \{H_{12},H_{1i}\}\cup\{E_k\,|\, k\in\mathcal{N},k\geq 3\} , & f_2 \text{ is non-toric blowup on }aH-(a-1)E_1 \\
      \{H_{1i}\}\cup\{E_k\,|\, k\in\mathcal{N}\}, & \text{otherwise}
    \end{cases}$
    
      \item  
        $\begin{cases}
      \{H_{12}\}\cup\{E_k\,|\, k\in\mathcal{N},k\geq 3\} , & f_2 \text{ is non-toric blowup on }aH-(a-1)E_1 \\
      \{E_k\,|\, k\in\mathcal{N}\}, & \text{otherwise}
    \end{cases}$
    
      \item  
    $\{E_3,E_4,\cdots,E_{l-1},H_{1l},H_{2l},H_{li},H_{lj}\}\cup\{E_k\,|\, k\in\mathcal{N}\}$
\end{enumerate}

One can easily check the above lists all satisfy the condition of Lemma \ref{lemma:exceptional}. Therefore, we can perform non-toric blowdowns for all such exceptional spheres. The outcome will be toric by a simple computation of its charge. It is called the {\bf toric model}, 
 which will be denoted by $\XX_{tor}$. Note that we can get some path of non-toric blowups $\Gamma'=\bigl\langle \mathbb{X}_{tor}\xrightarrow{}\cdots\rightarrow \XX_{\varepsilon} \bigl\rangle\in Hom(\XX_{tor},\XX_{\varepsilon})$ but it's not unique since there is no canonical choice of the blowup order.

%Given the reduced model $\Gamma=\bigl\langle \XX_{red}\xrightarrow{f}\cdots\rightarrow \XX \bigl\rangle\in Hom(\XX_{red},\XX)$, there is also a {\bf toric model} $\Gamma'=\bigl\langle\mathbb{X}_{tor}^1\rightarrow\cdots\rightarrow \XX_{tor}^m\bigl\rangle\in Hom(\XX_{tor}^1,\XX_{tor}^m)$ where $\XX_{tor}^1$ is the boundary divisor of a symplectic toric Hirzebruch surface and each arrow denotes a toric blowup. It will depend on the pattern $f$ in all cases except for the last case, which gives the position of the exceptional curve of class $E_2$. 

 We now carefully choose $\XX_{tor}'$, a boundary divisor of some symplectic toric Hirzebruch surface, as the toric minimal reduction for $\XX_{tor}$. This process will also be canonical. Note that all symplectic toric Hirzebruch surfaces modulo equivariant symplectomorphism are classified by their moment polygons modulo $AGL(2,\ZZ)$-action. Those polygons are all trapezoids whose edges have affine lengths\footnote{See Definition \ref{def:affine}.} $x,y,x,z$ (in a cyclic order) with $y-z=kx$ for some $k\in\ZZ$. In this note we call this $k$-trapezoid. Therefore the moduli can be described by $\{(k,x,y,z)\,|\,y-z=kx\}\subset \ZZ\times \RR_+^3$ modulo the equivalent relations $(k,x,y,z)\sim (-k,x,z,y)$ and $(0,x,y,y)\sim(0,y,x,x)$. $\XX_{tor}'$ is then defined by the $k$-trapezoid given by the quadruple $(k,x,y,z):=$ 
\begin{enumerate}[label=(\roman*)]
\item $\begin{cases}
      (1,1-\dd_2,2-\dd_2-\dd_i-\dd_j,1-\dd_i-\dd_j) , & f_2 \text{ is non-toric blowup on }2H \\
     (2,1-\dd_i,2-\dd_i-\dd_j,\dd_i-\dd_j), & \text{otherwise}
    \end{cases}$
    
	%\item (a) if $f$ is not non-toric blowup on the component $2H$, then $k=2,x=1-\alpha,y=2-\alpha-\beta,z=\alpha-\beta$;

 %(b) otherwise, $k=1,x=1-\dd_2,y=2-\dd_2-\alpha-\beta,z=1-\alpha-\beta$;

 \item $\begin{cases}
      (2a-2,1-\dd_1,1+a-a\dd_1-\dd_2-\dd_i-\dd_j,3-a+(a-2)\dd_1-\dd_2-\dd_i-\dd_j) , \\ f_2 \text{ is non-toric blowup on }(a+1)H-aE_1; \\
     (2a-1,1-\dd_1,a+1-a\dd_1-\dd_i-\dd_j,-a+2+(a-1)\dd_1-\dd_i-\dd_j),\text{otherwise}
    \end{cases}$
    
   % \item (a) if $f$ is not non-toric blowup on the component $(a+1)H-aE_1$, then $k=2a-1,x=1-\dd_1,y=a+1-a\dd_1-\alpha-\beta,z=-a+2+(a-1)\dd_1-\alpha-\beta$;

%(b) otherwise, $k=2a-2,x=1-\dd_1,y=a+1-a\dd_1-\dd_2-\alpha-\beta,z=-a+3+(a-2)\dd_1-\dd_2-\alpha-\beta$;

     \item $\begin{cases}
      (2a-3,1-\dd_1,a-(a-1)\dd_1-\dd_2-\dd_i,3-a+(a-2)\dd_1-\dd_2-\dd_i) , \\ f_2 \text{ is either non-toric blowup on }aH+(-a+1)E_1 \text{ or}\\
      \text{ toric blowup at the node between } aH+(-a+1)E_1 \text{ and } H-E_1; \\
     (2a-2,1-\dd_1,a-(a-1)\dd_1-\dd_i,-a+2+(a-1)\dd_1-\dd_i),\text{ otherwise}
    \end{cases}$
    
	%\item (a) if either $a=1$, or $f$ is neither the toric blowup at the node between $aH+(-a+1)E_1$ and $H-E_1$ nor non-toric blowup on the component $aH+(-a+1)E_1$, then $k=2a-2,x=1-\dd_1,y=a-(a-1)\dd_1-\alpha,z=-a+2+(a-1)\dd_1-\alpha$;

%(b) otherwise, $k=2a-3,x=1-\dd_1,y=a-(a-1)\dd_1-\dd_2-\alpha,z=-a+3+(a-2)\dd_1-\dd_2-\alpha$;

 
 \item $\begin{cases}
      (2a-2,1-\dd_1,a-(a-1)\dd_1-\dd_2,2-a+(a-1)\dd_1-\dd_2) ,\\ f_2 \text{ is either non-toric blowup on }aH+(-a+1)E_1 \text{ or}\\
      \text{ toric blowup at the node between } aH+(-a+1)E_1 \text{ and } H-E_1; \\
     (2a-1,1-\dd_1,a-(a-1)\dd_1,1-a+a\dd_1),\text{ otherwise}
    \end{cases}$

%	\item (a) if $E_2$ doesn't appear as the toric blowup at the node between $aH+(-a+1)E_1$ and $H-E_1$, then $k=2a-1,x=1-\dd_1,y=a-(a-1)\dd_1,z=-a+1+a\dd_1$;

% (b) otherwise, $k=2a-2,x=1-\dd_1, y=a-(a-1)\dd_1-\dd_2,z=-a+2+(a-1)\dd_1-\dd_2$;


    \item $(1,1-\dd_l,3-\dd_1-\dd_2-2\dd_l-\dd_i-\dd_j,2-\dd_1-\dd_2-\dd_l-\dd_i-\dd_j)$
    
    %\item $k=1,x=1-\dd_l,y=3-\dd_1-\dd_2-2\dd_l-\alpha-\beta,z=2-\dd_1-\dd_2-\dd_l-\alpha-\beta$.
\end{enumerate}

The sizes $x,y,z$ we choose correspond to the `initial' areas of the four (in black, red and blue) curves plus non-toric exceptional curves in the set $\mathcal{E}$. Here `initial' means the one before any toric blowup on it. With this in mind, we can take a copy of all the toric blowups in $\Gamma$ other than the ones producing $E_i$ or $E_j$, into another sequence of toric blowups $\Gamma''=\bigl\langle\mathbb{X}_{tor}'\rightarrow\cdots\rightarrow \XX_{tor}\bigl\rangle$ since $x,y,z$ are large enough to allow those toric blowups.  The first cases in (\lowerromannumeral{3}) and (\lowerromannumeral{4}) need special care. When $f_2$ is the toric blowup at the node between $aH+(-a+1)E_1$ and $H-E_1$, the first toric blowup chosen in the sequence $\Gamma'$ is of size $1-\dd_1-\dd_2$ between the edges of affine length $x$ and $z$.\footnote{The reason why we make such a choice will be clear in the proof of main Theorem \ref{thm:main}, where we need to check the triangle packing problem is solvable.} Note that the ending object must be $\XX_{tor}$ by our choice of $x,y,z$ and tautness of toric pairs.

%Now let's consider the opposite of Assumption \ref{ass0}:
%\begin{assumption}\label{ass00}
 %      $E_2$ exactly appears as the non-toric blowup on the component $2H$ in case $(\lowerromannumeral{1})$; $(a+1)H-aE_1$ in case $(\lowerromannumeral{2})$ or $aH+(-a+1)E_1$ in case $(\lowerromannumeral{3})$.
%\end{assumption}

%The remaining cases in the above assumption can be summarized in the following curve configuration diagrams:

%As before, we use the notation $\alpha,\beta$ to denote the symplectic area (some $\dd_i,\dd_j$) of the exceptional classes produced by the toric blowups on the specific nodes ($\alpha,\beta$ maybe $0$ if there is no such toric blowup). Now let $\XX_{tor}$ be the divisor corresponding to the trapezoid $P$ with
%\begin{enumerate}[label=(\roman*)]\addtocounter{enumi}{5}
 %   \item $k=1,x=1-\dd_2,y=2-\dd_2-\alpha-\beta,z=1-\alpha-\beta$;
  %  \item $k=2a-2,x=1-\dd_1,y=a+1-a\dd_1-\dd_2-\alpha-\beta,z=-a+3+(a-2)\dd_1-\dd_2-\alpha-\beta$;
   % \item $k=2a-3,x=1-\dd_1,y=a-(a-1)\dd_1-\dd_2-\alpha,z=-a+3+(a-2)\dd_1-\dd_2-\alpha$.
%\end{enumerate}


Finally, let's summarize what we have introduced in this section for the future reference. Given a framing $\mathfrak{f}$, any $\XX\in\mathcal{LCY}$ has 
\begin{itemize}
    \item a canonical reduced model $\Gamma=\bigl\langle \XX_{red}\xrightarrow{}\cdots\rightarrow \XX \bigl\rangle\in Hom(\XX_{red},\XX)$;
    \item  a canonical $\varepsilon$-replacement $\XX_\varepsilon$; 
    \item a canonical toric model $\XX_{tor}$ with non-canonical $\Gamma'=\bigl\langle \mathbb{X}_{tor}\xrightarrow{}\cdots\rightarrow \XX_{\varepsilon} \bigl\rangle\in Hom(\XX_{tor},\XX_{\varepsilon})$;
    \item a canoical toric Hirzebruch surface boundary divisor $\XX_{tor}'$ with canonical $\Gamma''=\bigl\langle\mathbb{X}_{tor}'\rightarrow\cdots\rightarrow \XX_{tor}\bigl\rangle$.
\end{itemize}




\section{Base diagram set $\mathbb{BD}$}\label{section:bd}

Let's describe the base diagram set $\mathbb{BD}$. Throughout this note, we use the word `($k$-th) trapezoid' to denote the moment map image of ($k$-th) Hirzebruch surface. 
\subsection{Bitten Delzant polygons}
\begin{definition}\label{def:affine}
We say a nonzero $\overrightarrow{p}=(a,b)\in \ZZ^2$ is a {\bf primitive vector} if $gcd(a,b)=1$. Given a segment $AB$ in $\RR^2$ of rational slope, there is a unique primitive vector $\overrightarrow{p}$ such that $\overrightarrow{AB}=c\overrightarrow{p}$ with $c\in \RR_+$. We call such $c$ the {\bf affine length} of $AB$, denoted by $l(AB)$. If $O$ is a point in $\RR^2$, the {\bf affine distance} is defined by pick a smooth curve $\gamma:[0,1]\rightarrow \RR^2$ with $\gamma(0)=O,\gamma(1)\in AB$ and $$d(O,AB)=\Big|\int_{0}^1\overrightarrow{p}\times \gamma'(t)\,dt\Big|$$
\end{definition}

It's easy to see the affine distance doesn't depend on the choice of $\gamma$. Geometrically, it will be the symplectic area of a visible surface lying over $\gamma$ (Proposition 7.8 in \cite{symington}).

For a Delzant polygon $P$, a {\bf surgery triangle} is an embedded triangle such that one edge has affine length $d$ lying on an edge $AB$ of $P$ and the vertex opposite to that edge has affine distance $d$ to $AB$, where $d$ is required to be strictly less than the $l(AB)$ and is called the size of the surgery triangle.

We now introduce the {\bf bitten Delzant polygons}, which are the datum consisting of a corner chopping procedure from a trapezoid $P_0$ to $P_1$, and disjoint surgery triangles lying on the edges of $P_1$. The following figure \ref{fig:delzant} shows an example.

 \begin{figure}[h]
		\includegraphics*[width=\linewidth]{delzant}
 
		\caption{An example of a (labelled) bitten Delzant polygon. $P_0$ is a $2$-trapezoid and $P_1$ is obtained by performing corner chopping four times. It is the almost toric base diagram of a symplectic rational manifold $(S^2\times S^2\# 9\overline{\CC\PP}^2,\omega)$. From the labelling we can give a framing $\{b,f,e_1,\cdots,e_9\}$ satisfying $\omega(f)=7,\omega(b)=10,\omega(e_1)=3,\omega(e_2)=\omega(e_3)=\omega(e_4)=\omega(e_6)=2,\omega(e_5)=\omega(e_7)=\omega(e_8)=\omega(e_9)=1$.\label{fig:delzant}}
	\end{figure}

We say two bitten Delzant polygons are only differed by a {\bf rearrangement of surgery triangles}, if one can smoothly deform one into another through a family of bitten Delzant polygons, up to $AGL(2,\ZZ)$-transformation and permutation of the position of surgery triangles lying on the same edge. Intuitively, just as each Delzant polygon gives a symplectic toric manifold, if we label all the chopped corners and surgery triangles, there will be a way assigning each bitten Delzant polygon an almost toric fibered symplectic rational manifold with a framing. And the rearrangement of surgery triangles will not affect the symplectomorphism type. This will be explained Proposition \ref{prop:bdtolcy}. 

\subsection{Two extra types of diagrams}\label{extra}
However, to realize all log Calabi-Yau divisors as almost toric boundary divisors, we need to enlarge the set of base digrams by adding two {\bf extra types} which might look strange at a first glance:
\begin{enumerate}[label=(\Roman*)]

    \item Note that a $k$-trapezoid has two opposite edges $X,X'$\footnote{Unfortunately, this notation coincides with the one we have used for the symplectic manifold. However the meaning should be clear in context.} with the same affine lengths. We now allow `{\bf full bites}' on $X,X'$, that is, the sizes of the surgery triangles lying on those two edges are the same as the affine length of $X,X'$. The diagram can be viewed as the `filling in a corner' of a bitten Delzant polygon coming from an $\varepsilon${\bf -shift} of the edges $X,X'$. Geometrically this corresponds to the toric blowdown operation and thus is indeed an almost toric base diagram. See figure \ref{fig:bd1}.
    
%    Furthermore when its $\varepsilon$-small shift comes with a labelling, we can also give it a canonical framing. Assume the labels on the full bites on $X,X'$ are $i,j$. If $k$ is odd, in the case of only one full bite, we contract $H_{1i}$ so that the basis after blowdown can be chosen as $f=H_1,b=H_i,\{e_k\}=\{E_k\}_{k\neq i}$; in the case of two bites, $H_{1i},H_{1j}$ are contracted so that we can choose $f=H_1,b=2H-E_1-E_i-E_j,\{e_k\}=\{E_k\}_{k\neq i,j}$. If $k$ is even, 
     \begin{figure}[h]
		\includegraphics*[width=\linewidth]{bd1}
 
		\caption{The right almost toric base diagram is from a bitten Delzant polygon, which is the $\varepsilon$-shift of the edge $X$. Therefore the left diagram can also be viewed as an almost toric base diagram of the toric blowdown of the symplectic manifold represented by the right diagram. That will be $(S^2\times S^2,\omega)$ with basis $\{b,f\}$ and $\omega(b)=3,\omega(f)=4$. Note that the two marked vertices are actually the same point (rank $0$ elliptic singularity) in the ATF base disk.\label{fig:bd1}}
	\end{figure}

 
    \item We also allow the $2$-trapezoid to be degenerate in the sense that the shortest edge has length $0$ (so that it becomes a triangle) and there is one surgery triangle lying on either $X$ or $X'$ with size same as the length of the edge. We can use the trick of the branch move to rotate around the node. Then it can be dealt with similarly as (\upperRomannumeral{1}) by `filling up two small corners' of a bitten Delzant polygon. This will correspond to performing toric blowdown twice. We also call that bitten Delzant polygon an $(\varepsilon,\frac{\varepsilon}{2})${\bf -shift} of the original degenerate ones.   See figure \ref{fig:bd2}.

     \begin{figure}[h]
		\includegraphics*[width=\linewidth]{bd2}
 
		\caption{The leftmost one is a degenerate $2$-trapezoid with a full bite. By choosing a different branch cut (in blue), it can be viewed as performing toric blowdown twice from the diagram of its $(\varepsilon,\frac{\varepsilon}{2})$-shift. In this specific example what we obtain is just $(\CC\PP^2,\omega)$ with $\omega(H)=3$.  As before, the marked vertices should be the same point (rank $0$ elliptic singularity) in the ATF base disk.\label{fig:bd2}}
	\end{figure}
\end{enumerate}
%The morphisms are sequences of $AGL(2,\ZZ)$-transformations, rearrangements of surgery triangles lying on the same edge, adding one corner chopping or surgery triangle modulo compositions of the first two operations. 
%\todo{need a figure here}

\subsection{$\mathbb{BD}$ as almost toric base diagrams}
  Two base diagrams are equivalent if they are only differed by rearrangement of surgery triangles. The base diagram set $\mathbb{BD}$ is then defined to be the equivalent classes of all bitten Delzant polygons and the above two extra types of diagrams. As what we did in section \ref{section:lcy}, $\mathbb{BD}$ can also be equipped with a quiver structure where each arrow corresponds to chopping a corner or adding a new surgery triangle.
  
  %Also adding one corner chopping or surgery triangle will naturally give a morphism in $\mathcal{ATF}$. Therefore we get a realization functor $\mathbf{R}:\mathcal{BD}\rightarrow\mathcal{ATF}$, which is unique up to natural isomorphisms.

\hfill

Now let's carefully explain the relation between $\mathbb{BD}$ and $\mathbb{LCY}$ through almost toric fibration. 

\begin{prop}\label{prop:bdtolcy}
There is a canonical morphism between quivers $\mathbf{B}:\mathbb{BD}\rightarrow \mathbb{LCY}$ by taking the boundary divisor.
 \end{prop}
 
\begin{proof}
    On the level before modulo equivalence relations in both 
 $\mathbb{BD}$ and $\mathbb{LCY}$, each base diagram gives a stratified integral affine disk with nodes $(B,\mathcal{A},\mathcal{S})$. According to \cite{symington} Theorem 5.2 and Corollary 5.4, one can produce an almost toric fibered symplectic four-manifold over the base $(B,\mathcal{A},\mathcal{S})$. Then by \cite{symington} Proposition 8.2, the fibers over $0$ and $1$-strata will form a divisor whose homology class is Poincar\'e dual to the first Chern class. Thus we will obtain a log Calabi-Yau pair. 

 To make this assignment into a map $\mathbf{B}:\mathbb{BD}\rightarrow \mathbb{LCY}$, we use the following observations. Let's first assume the base diagram is a bitten Delzant polygon (that is, not of the extra types). For any surgery triangle lying on some edge, by \cite{symington} Theorem 7.4, one always has an embedded symplectic sphere (unique up to isotopy) lying over the curve going from the node to the edge. This sphere intersects the boundary divisor transversally and has self intersection $-1$ due to adjunction formula. After blowing down all of those exceptional spheres, we will get a toric divisor. Note that the area sequence and self-intersection sequence of this toric divisor will not change under the rearrangement of surgery triangles on the bitten Delzant polygon. Thus, by tautness of toric pairs (Lemma \ref{lem:torictaut}), we will get an element in $\mathbb{LCY}$. Then by Proposition \ref{prop:Torelli} and the followed illustration therein, the original log Calabi-Yau pair assigned to the bitten Delzant polygon will also represent a well-defined element in $\mathbb{LCY}$. 
 
 Now if the base diagram is of extra types, its $\varepsilon$ or $(\varepsilon,\frac{\varepsilon}{2})$-shift will be a bitten Delzant polygon. Thus we can firstly obtain an element in $\mathbb{LCY}$ corresponding to the shift. The original space,  associated to the extra types base diagram, is then the toric blowdowns with sizes $\varepsilon$ or $\frac{\varepsilon}{2}$ of that element. When $\varepsilon$ is small enough, the exceptional spheres blown down are distinguished among the components of the log Calabi-Yau divisor. Therefore we will also get a well-defined element in $\mathbb{LCY}$ again by Proposition \ref{prop:Torelli}.
    
    To further upgrade this map into a morphism between quivers, firstly notice that the arrow of chopping a corner in $\mathbb{BD}$ naturally correspond to the arrow of toric blowup in $\mathbb{LCY}$. To establish the correspondence between adding a surgery triangle and a non-toric blowup, we blow down the exceptional sphere lying over the newly added surgery triangle. It remains to show the divisor after blowdown represents the element in $\mathbb{LCY}$ associated to the base diagram before adding the surgery triangle. One then can run a similar argument as above by blowing down all the other exceptional spheres lying over other surgery triangles into some toric pairs and then use tautness of toric pairs and Torelli theorem. 
\end{proof}


Now we can state the main result of this note:

\begin{theorem}\label{thm:main}
   Given a framing $\mathfrak{f}$ on $\mathbb{LCY}$, there is a canonical section $\mathbf{R}_{\mathfrak{f}}:\mathbb{LCY}\rightarrow \mathbb{BD}$ of the map $\mathbf{B}:\mathbb{BD}\rightarrow \mathbb{LCY}$.
\end{theorem}

\begin{corollary}\label{maincor}
    Each symplectic rational manifold $(X,\omega)$ with $\omega\cdot c_1>0$ admits an almost toric fibration.
\end{corollary}

\begin{proof}
After picking a framing $\{H,E_1,\cdots,E_l\}$ with $\omega(E_1)+\cdots+\omega(E_l)<3\omega(H)$, one can assume in its small blowup there exists a Looijenga pair $(3H-E_1-\cdots-E_l-2E_{l+1},E_{l+1})$. Theorem \ref{thm:main} then gives an almost toric realization of this Looijenga pair. By the trick of branch moves, we can make the codimension $1$-stratum representing the component $E_l$ fit together into a segment and then fill in a corner. The resulting diagram will denote the toric blowdown of the exceptional class $E_l$, which gives an almost toric base diagram for $(X,\omega)$. 
\end{proof}
%\begin{rmk}
    %It's well known the above statement is not true if we replace almost toric fibration by toric action, or Hamiltonian $S^1$-action, in particular, semi-toric fibration. The symplectic classes admitting toric or Hamiltonian $S^1$-action structures can be computed in small blowups.
%\end{rmk}

%\section{The category $\mathcal{ATF}$}
%Let $\mathcal{ATF}$ denote the category with objects being a pair of symplectic rational manifold and an almost toric fibration on it, morphisms being sequence of almost toric surgeries (symplectomorphisms preserving boundary divisors, toric/almost toric blowups). 
%\todo{need to figure out almost toric blowup!}

%There is a natural functor taking boundary divisor $\mathbf{B}:\mathcal{ATF}\rightarrow \mathcal{LCY}^{\perp}$.\todo{need to describe the categories and functor more precisely}
%\begin{theorem}
 %   $\mathbf{B}$ is fully faithful and essentially surjective (thus an equivalence of categories).
%\end{theorem}



\section{Elementary geometry of Delzant polygons}\label{section:elementarygeometry}

Let's first study some elementary geometry of Delzant polygons to prepare for the proof of theorem \ref{thm:main}. Given a Delzant polygon $P$ with edges $Q_1,\cdots,Q_n$, by a {\bf triangle packing problem} with weights $a_1,\cdots,a_n\in\RR_+$ where we require each $a_i<l(Q_i)$, we mean the problem of finding an interior point $O$ in $P$ such that $d(O,Q_i)\geq a_i$. If such a point exists we say the problem is {\bf solvable} and call it a {\bf solution point}. Intuitively, this leads to the realization of almost toric blowups with sizes being those weights. 

 We consider the following situation which will be used later for our constructions. Let $P_0$ be a $k$-trapezoid having $x,y,x,z$ with $y\geq z$ as affine lengths of its edges $X,Y,X',Z$. Suppose $P_1$ is a Delzant polygon coming from corner chopping of $P_0$ and the largest chopped corners lying over the initial four edges have size $b_X,b_Y,b_{X'},b_Z$ respectively. Denote by $X_1,Y_1,X'_1,Z_1$ the edges in $P_1$ corresponding to the original four edges in $P_0$ after corner choppings. See figure \ref{fig:trianglepacking} for an example. Now assume there is a triangle packing problem on $P_1$ whose sizes on $X_1,Y_1,X'_1,Z_1$ are $a_{X_1},a_{Y_1},a_{X'_1},a_{Z_1}$. We introduce the following quantities associated to it:
 \[\Theta_1:=\max\{a_{Y_1},b_{Y}\}+\max\{a_{Z_1},b_{Z}\},\]
 \[\Theta_2:=\max\{a_{X_1},b_{X}\}+\max\{a_{X'_1},b_{X'}\}+k\max\{a_{Y_1},b_{Y}\}.\]

 \begin{figure}[h]
		\includegraphics*[width=\linewidth]{trianglepacking}
 
		\caption{The left polygon is a $2$-trapezoid $P_0$ with $x=6,y=16,z=4$. After truncating the corners indicated by the red segments we obtain the right polygon $P_1$. In this case, $b_X=\max\{1,3\}=3,b_Y=\max\{3,1,2\}=3,b_{X'}=\max\{2,0\}=2,b_{Z}=\max\{0,1\}=1$.\label{fig:trianglepacking}}
	\end{figure}
 
\begin{lemma}\label{lemma:packing}
    This triangle packing problem is solvable if $\Theta_1\leq x$ and $\Theta_2\leq y$.
\end{lemma}

\begin{proof}
The above inequalities immediately guarantee we can find a solution point $O$ for the four edges $X_1,Y_1,X_1',Z_1$. To check $O$ is also the solution point for the other edges coming from corner chopping, we have the following observation. Assume a point in the Delzant polygon has affine distance at least $d$ to two adjacent edges. Then if we truncate the corner of size $\leq d$ between those two edges, the affine distance between the point and the new edge must be $\leq d$ as well. Since we have required each weight $a_i<$ affine length of the corresponding edge and in our definition of $\Theta_1,\Theta_2$ we took the maximum with those corner chopping sizes $b_*$'s, the point $O$ is indeed the solution point for the triangle packing problem.
\end{proof}

Unfortunately, the existence of such a point doesn't immediately give the disjoint surgery triangles we expect since they might overlap at the point $O$. See the following figure \ref{fig:perturbation} for an example about such issue. 

 \begin{figure}[h]
		\includegraphics*[width=\linewidth]{perturbation}
 
		\caption{Suppose the Delzant polygon in the triangle packing problem is a square whose edges have affine length $4$ and the weights $a_1,a_2,a_3,a_4$ are $2,2,2,2$. There is a unique point solving this problem. However it doesn't give the diagram of almost toric blowup we want at once. One needs some perturbations.\label{fig:perturbation}}
	\end{figure}


We need the following triangle perturbation lemma to solve this issue.
\begin{lemma}[Triangle perturbation]\label{lemma:embed}
Suppose there is a solvable triangle packing problem with weights $a_i$ on the edge $Q_i$. Assume for each $i$ there is a finite sequence $\{b_{ij}\}_j$ such that $b_{ij}\leq a_i$ and $\sum_jb_{ij}<l(Q_i)$. Then one can always have a disjoint embedding of surgery triangles of sizes $b_{ij}$ on the edge $Q_i$.
\end{lemma}
\begin{proof}
    We only prove the case when each $\{b_{ij}\}_j$ is just $\{a_i\}$. The general case is only a matter of more complicated notations. Assume each edge has primitive direction $\overrightarrow{p_i}$ (in a counterclockwise way) and the point $O$ is the solution for the triangle packing problem. We wish to shift each surgery triangle lying on $Q_i$ along the vector $\varepsilon_i\overrightarrow{p_i}$. Denote by $O_i:=O+\varepsilon_i\overrightarrow{p_i}$ the vertex of the shifted surgery triangle. Let's take a closer look (figure \ref{fig:triangleshift}) at the corner $R$ between the edge $Q_{i-1}$ and $Q_i$. Note that we have slopes $s_{OB}>s_{OR}>s_{OA}$. For the shifted surgery triangles not to overlap, it suffices to make $O_{i-1}$ not intersect $O_iB'$, which will be guaranteed by $\frac{\varepsilon_{i-1}}{\varepsilon_{i}}<s_{OB}$. So it also suffices to have $\frac{\varepsilon_{i-1}}{\varepsilon_{i}}\leq s_{OR}=\frac{a_i}{a_{i-1}}$. Now we can take small enough $\varepsilon\ll l(Q_i)$ for all $i$'s and let each $\varepsilon_i:=\frac{a_1}{a_i}\varepsilon$. This will satisfy our purpose.
\end{proof}

\begin{figure}[h]
		\includegraphics*[width=\linewidth]{triangleshift}
 \caption{The way of shifting surgery triangles.}\label{fig:triangleshift}
	\end{figure}




\section{Proof of the main theorem}\label{section:proof}
Now we canonically assign an almost toric realization $\mathbf{R}_{\mathfrak{f}}(\XX)\in\mathbb{BD}$ where $\XX=[X,\omega,D]\in\mathbb{LCY}$. Suppose $X=\CC\PP^2\#n\overline{\CC\PP}^2$ with $n\geq 2$ since the remaining cases are quite simple (have already been dealt with in \cite{Enumerate}). The framing $\mathfrak{f}$ gives a basis $H,E_1,\cdots,E_n$ of $H_2(X;\ZZ)$ with $\omega(H)=c,\omega(E_i)=\dd_i$ being reduced. Without loss of generality, let's also assume the normalized condition $c=1$. Recall that in Section \ref{section:lcy}, there are canonical $\varepsilon$-replacement $\XX_{\varepsilon}$ and toric model $\XX_{tor}$ with some (non-canonical) $\Gamma'=\bigl\langle\mathbb{X}_{tor}\rightarrow\cdots\rightarrow \XX_{\varepsilon}\bigl\rangle$. And we also have $\Gamma''=\bigl\langle\mathbb{X}_{tor}'\rightarrow\cdots\rightarrow \XX_{tor}\bigl\rangle$ where $\XX_{tor}'$ is given by the boundary divisor of a symplectic toric Hirzebruch surface with moment polygon $P_0$

%From now on we always assume $\omega$ is normalized reduced. That is, $\omega(H)=1$, $\omega(E_i)=\dd_i$ and $(\delta_1,\cdots,\delta_n)$ is a reduced vector. We showed in \cite{Enumerate} that any object $(X,\omega,D)$ has a morphism $\Gamma$ coming from $(X',\omega',D')$ with $D'$ being one of the followings ($X'=\CC\PP^2\#\overline{\CC\PP}^2$ in the first four cases and $X'=\CC\PP^2\#l\overline{\CC\PP}^2$ with some $l\leq n$ in the last case).
%\begin{enumerate}[label=(\roman*)]
%	\item $(2H,H-E_1)$
 %   \item $((a+1)H-aE_1,(-a+2)H+(a-1)E_1)_{a\in\ZZ_+}$
%	\item $(aH+(-a+1)E_1,H-E_1,(-a+2)H+(a-1)E_1)_{a\in\ZZ_+}$
%	\item $(aH+(-a+1)E_1,H-E_1,(-a+1)H+aE_1,H-E_1)_{a\in\ZZ_+}$
 %   \item $(3H-E_1-\cdots-E_{l-1}-2E_l,E_l).$
%\end{enumerate}
%{\bf Step 1 (find the symplectic toric model):}

%Now we want to associate some trapezoid $P_0$  to $(X,\omega,D)$ according to the pattern of $\Gamma$, which is the {\bf symplectic toric model} analogous to \cite{GrHaKe11} and \cite{Engel2021} in the holomorphic setting. This association will depend on the position of the exceptional class $E_2$. So let's first suppose the following:

%\begin{assumption}\label{ass1}
%   $E_2$ doesn't appear as the non-toric blowup on the component $2H$ in case $(\lowerromannumeral{1})$; $(a+1)H-aE_1$ in case $(\lowerromannumeral{2})$ or $aH+(-a+1)E_1$ in case $(\lowerromannumeral{3})$. 
%\end{assumption}

%Under the assumption \ref{ass1}, the following diagrams of configurations show the possible patterns of $\Gamma$. Note that in each case we have marked some nodes. $\Gamma$ may or may not contain toric blowups at those nodes. If it contains we use $\alpha$ or $\beta$ to denote the sizes which are some $\dd_i,\dd_j$. Otherwise we just let $\alpha$ and $\beta$ be $0$. 


% \begin{figure*}[h]\label{fig:pattern}
%		\includegraphics*[width=\linewidth]{pattern}
%	\end{figure*}
% The trapezoids $P_0$ whose edges have affine lengths $x,y,x,z$ with $y\geq z$ are given by the followings:
%\begin{enumerate}[label=(\roman*)]
%	\item $k=2,x=1-\alpha,y=2-\alpha-\beta,z=\alpha-\beta$;
 %   \item $k=2a-1,x=1-\dd_1,y=a+1-a\dd_1-\alpha-\beta,z=-a+2+(a-1)\dd_1-\alpha-\beta$;
%	\item (a) if $a=1$ or $E_2$ doesn't appear as the toric blowup at the node between $aH+(-a+1)E_1$ and $H-E_1$, then $k=2a-2,x=1-\dd_1,y=a-(a-1)\dd_1-\alpha,z=-a+2+(a-1)\dd_1-\alpha$;

%(b) otherwise, $k=2a-3,x=1-\dd_1,y=a-(a-1)\dd_1-\dd_2-\alpha,z=-a+3+\dd_1-\dd_2-\alpha$;
%	\item (a) if $E_2$ doesn't appear as the toric blowup at the node between $aH+(-a+1)E_1$ and $H-E_1$, then $k=2a-1,x=1-\dd_1,y=a-(a-1)\dd_1,z=-a+1+a\dd_1$;

% (b) otherwise, $k=2a-2,x=1-\dd_1, y=a-(a-1)\dd_1-\dd_2,z=-a+2+(a-1)\dd_1-\dd-2$;
 %   \item $k=1,x=1-\dd_l,y=3-\dd_1-\dd_2-2\dd_l-\alpha-\beta,z=2-\dd_1-\dd_2-\dd_l-\alpha-\beta$.
%\end{enumerate}
%The above trapezoids are actually obtained by blowing down certain exceptional curves shown as follows:

% \begin{figure*}[h]\label{fig:pattern1}
%		\includegraphics*[width=\linewidth]{pattern1}
%	\end{figure*}

%Note that the pattern of `$\cdots$' in the above curve configuration diagrams can be carried to produce a corner chopping procedure on the trapezoid. The only special cases are $(\lowerromannumeral{3})(b)$ and $(\lowerromannumeral{4})(b)$, where we need to truncate a corner of size $1-\dd_1-\dd_2$ instead of $\dd_2$. Denote the Delzant polygon after corner chopping by $P_1$. $P_1$ has four distinguished edges $X_1,Y_1,X_1',Z_1$ just as the setting in Lemma \ref{lemma:packing}. The non-toric blowups information contained in $\Gamma$ will then given a triangle packing problem on $P_1$. The size is taken to be the maximum among all non-toric blowup sizes on that edge contained in $\Gamma$ and some extra ones shown as the green curves in the above configuration diagrams.

%{\bf Step 1 (check the triangle packing problem is solvable):}

\begin{proof}[Proof of Theorem \ref{thm:main}]

The pattern $\Gamma'$ and $\Gamma''$ give us a triangle packing problem in the following manner. Starting from the trapezoid $P_0$, one can extract a corner chopping pattern from the toric blowups in $\Gamma'$, which will give the polygon $P_1$ after corner chopping. The weights of triangle packing problem we put on the edges of $P_1$ is then determined by the non-toric blowups in $\Gamma''$: the maximum among the sizes of all non-toric blowups on the edge.\footnote{To be more rigorous, we require all the weights to be positive since otherwise the solution point may lie on the boundary of polygon. There might be some edges which don't admit non-toric blowup, in which case we take the weight on that edge to be a very small number $\ll \dd_i$ for all $i$'s. The following inequality estimates will still hold. } Note that another choice of the non-canonical $\Gamma'$ will give the same triangle packing problem since they are only differed by the blowup order. 

Let's show these are solvable by checking the conditions of Lemma \ref{lemma:packing} (details are fully spelled out for the first case). Let $\dd_i,\dd_j$ be two distinguished blowup sizes introduced in section \ref{subsection:replace}. Remember we have the normalized reduced conditions on $(\dd_1,\dd_2,\cdots)$.

\begin{enumerate}[label=(\roman*)]
	\item If $f_2$ is the non-toric blowup on $2H$, $\XX_{tor}'$ is given by the trapezoid with $(k,x,y,z)=(1,1-\dd_2,2-\dd_2-\dd_i-\dd_j,1-\dd_i-\dd_j)$. By the assumption on $f_2$ we have $b_X,b_{X'},b_{Y},b_Z\leq \dd_3$ (actually must be $\leq\dd_5$ since the exceptional class $E_2$ is non-toric, $E_3,E_4$ are either non-toric or distinguished). According to our choice of $\Gamma'$, we also have $a_{X_1}\leq 1-\dd_2-\dd_i,a_{X_1'}\leq 1-\dd_2-\dd_j,a_{Y_1}\leq \dd_3$ and $a_{Z_1}=\dd_1$. Therefore 
 $$\Theta_1\leq \dd_1+\dd_3\leq 1-\dd_2=x,$$
 $$\Theta_2\leq (1-\dd_2-\dd_i)+(1-\dd_2-\dd_j)+\dd_3=y+(\dd_3-\dd_2)\leq y;$$

 Otherwise, $\XX_{tor}'$ is given by the trapezoid with $(k,x,y,z)=(2,1-\dd_i,2-\dd_i-\dd_j,\dd_i-\dd_j)$. Note that we still have $b_X,b_{X'},b_{Y},b_Z\leq \dd_3$ (actually must be $\leq\dd_4$ since the exceptional classes $E_2,E_3$ are either non-toric or distinguished). Also $a_{X_1}\leq \dd_1,a_{X_1'}\leq 1-\dd_i-\dd_j,a_{Z_1}\leq \dd_2$ and $a_{Y_1}\leq\dd_3$ by assumption on $f_2$. Therefore
 $$\Theta_1\leq \dd_2+\dd_3\leq 1-\dd_1\leq 1-\dd_i =x,$$
 $$\Theta_2\leq \dd_1+(1-\dd_i-\dd_j)+2\dd_3\leq 2-\dd_1-\dd_2=y.$$

  
 \item If $f_2$ is the non-toric blowup on $(a+1)H-aE_1$,
\[\Theta_1\leq \dd_3+(1-\dd_1-\dd_2)\leq 1-\dd_1=x,\]
 $$\Theta_2\leq (1-\dd_1-\dd_i)+(1-\dd_1-\dd_j)+(2a-2)\dd_3=2-\dd_i-\dd_j+(a-1)(\dd_1+2\dd_3)-a\dd_1-\dd_1 $$
     $$\leq 2-\dd_i-\dd_j+a-1-a\dd_1-\dd_2=y;$$


 otherwise, $a_{Y_1},b_{Y_1}\leq \dd_3$ holds and we have 
 $$\Theta_1\leq \dd_2+\dd_3\leq 1-\dd_1=x,$$
 $$\Theta_2\leq (1-\dd_1-\dd_i)+(1-\dd_1-\dd_j)+(2a-1)\dd_3=(2-\dd_i-\dd_j)+(a-2)(\dd_1+2\dd_3)+3\dd_3-a\dd_1$$
 $$\leq (2-\dd_i-\dd_j)+a-2+1-a\dd_1=y.$$

 
 %\item (a) $\Theta_1\leq \dd_2+\dd_3\leq 1-\dd_1=x,$
 
 %$\Theta_2\leq (1-\dd_1-\alpha)+(1-\dd_1-\beta)+(2a-1)\dd_3=(2-\alpha-\beta)+(a-2)(\dd_1+2\dd_3)+3\dd_3-a\dd_1\leq (2-\alpha-\beta)+a-2+1-a\dd_1=y$;

 %(b) $\Theta_1\leq \dd_3+(1-\dd_1-\dd_2)\leq 1-\dd_1=x,$
 
 %$\Theta_2\leq (1-\dd_1-\alpha)+(1-\dd_1-\beta)+(2a-2)\dd_3=(2-\alpha-\beta)+(\dd_1+\dd_2+\dd_3)+(a-3)(\dd_1+2\dd_3)+3\dd_3-a\dd_1-\dd_2\leq (2-\alpha-\beta)+1+(a-3)+1-a\dd_1-\dd_2=y$;

 \item If $f_2$ is the non-toric blowup on $aH-(a-1)E_1$,
\[\Theta_1\leq \dd_3+(1-\dd_1-\dd_2)\leq 1-\dd_1=x,\]
 $$\Theta_2\leq (1-\dd_1-\dd_i)+\dd_3+(2a-3)\dd_3=1-\dd_i+(a-1)(\dd_1+2\dd_3)-(a-1)\dd_1-\dd_1$$
     $$\leq 1-\dd_i+a-1-(a-1)\dd_1-\dd_2=y;$$

if $f_2$ is the toric blowup on the node between $aH-(a-1)E_1$ and $H-E_1$, then $a_{Y_1},b_{Y_1}\leq \dd_3$, $a_{X_1},b_{X_1}\leq 1-\dd_1-\dd_2$, $a_{X_1'},b_{X_1'}\leq 1-\dd_1-\dd_i$ hold and we have
\[\Theta_1\leq \dd_3+(1-\dd_1-\dd_2)\leq 1-\dd_1=x,\]
 $$\Theta_2\leq (1-\dd_1-\dd_i)+(1-\dd_1-\dd_2)+(2a-3)\dd_3=2-\dd_2-\dd_i+(a-2)(\dd_1+2\dd_3)-(a-1)\dd_1+(\dd_3-\dd_1)$$
     $$\leq 2-\dd_2-\dd_i+a-2-(a-1)\dd_1=y;$$
     
 otherwise, $a_{Y_1},b_{Y_1}\leq \dd_3$ still holds and we have 
 $$\Theta_1\leq \dd_2+\dd_3\leq 1-\dd_1=x,$$
 $$\Theta_2\leq (1-\dd_1-\dd_i)+\dd_2+(2a-2)\dd_3=1-\dd_i+(a-1)(\dd_1+2\dd_3)-(a-1)\dd_1+(\dd_2-\dd_1)$$
 $$\leq 1-\dd_i+a-1-(a-1)\dd_1=y.$$
 
 %\item (a) $\Theta_1\leq \dd_2+\dd_3\leq 1-\dd_1=x,$
 
 %$\Theta_2\leq \dd_2+(1-\dd_1-\alpha)+(2a-2)\dd_3=(1-\alpha)+(\dd_1+\dd_2+\dd_3)+(a-3)(\dd_1+2\dd_3)+3\dd_3-(a-1)\dd_1\leq (1-\alpha)+1+(a-3)+1-(a-1)\dd_1=y$;
 
% (b) $\Theta_1\leq \dd_3+(1-\dd_1-\dd_2)\leq 1-\dd_1=x,$
 
 %$\Theta_2\leq (1-\dd_1-\dd_2)+(1-\dd_1-\alpha)+(2a-3)\dd_3=(2-\dd_2-\alpha)+(a-3)(\dd_1+2\dd_3)+3\dd_3-(a-1)\dd_1\leq (2-\dd_2-\alpha)+(a-3)+1-(a-1)\dd_1=y$;

\item If $f_2$ is the non-toric blowup on $aH-(a-1)E_1$,
\[\Theta_1\leq \dd_3+(1-\dd_1-\dd_2)\leq 1-\dd_1=x,\]
 $$\Theta_2\leq \dd_3+\dd_4+(2a-2)\dd_3=a(\dd_1+2\dd_3)-(a-1)\dd_1-\dd_2+(\dd_2-\dd_1)+(\dd_4-\dd_3)$$
     $$\leq a-(a-1)\dd_1-\dd_2=y;$$

if $f_2$ is the toric blowup on the node between $aH-(a-1)E_1$ and $H-E_1$, then $a_{Y_1},b_{Y_1}\leq \dd_3$, $a_{X_1},b_{X_1}\leq 1-\dd_1-\dd_2$, $a_{X_1'},b_{X_1'}\leq \dd_3$ hold and we have
\[\Theta_1\leq \dd_3+(1-\dd_1-\dd_2)\leq 1-\dd_1=x,\]
 $$\Theta_2\leq (1-\dd_1-\dd_2)+\dd_3+(2a-2)\dd_3=1-\dd_2+(a-1)(\dd_1+2\dd_3)-(a-1)\dd_1+(\dd_3-\dd_1)$$
     $$\leq 1-\dd_2+a-1-(a-1)\dd_1=y;$$

     otherwise, $a_{Y_1},b_{Y_1}\leq \dd_3$ still holds and we have
     $$\Theta_1\leq \dd_2+\dd_3\leq 1-\dd_1=x,$$
 $$\Theta_2\leq \dd_2+\dd_3+(2a-1)\dd_3=a(\dd_1+2\dd_3)-(a-1)\dd_1+(\dd_2-\dd_1)$$
 $$\leq a-(a-1)\dd_1=y.$$
 
% \item (a) $\Theta_1\leq \dd_2+\dd_3\leq 1-\dd_1=x,$
 
% $\Theta_2\leq \dd_2+\dd_3+(2a-1)\dd_3\leq (\dd_2+2\dd_3)+(a-1)(\dd_1+2\dd_3)-(a-1)\dd_1\leq a-(a-1)\dd_1=y$; 

 %(b) $\Theta_1\leq \dd_2+\dd_3\leq 1-\dd_1=x,$
 
 %$\Theta_2\leq (1-\dd_1-\dd_2)+\dd_3+(2a-2)\dd_2=1+(\dd_1+\dd_2+\dd_3)+(a-3)(\dd_1+2\dd_2)+3\dd_2-(a-1)\dd_1-\dd_2$; 
 
 %Note that in general the reduced condition doesn't give us $\dd_1+2\dd_2\leq 1$ or $3\dd_2\leq 1$. However in this particular situation, the existence of $(-a+1)H+aE_1$ guarantees $\dd_1>\frac{a-1}{a}$ in which case $\dd_1+2\dd_2<\frac{a-1}{a}+2\frac{1}{a}=\frac{a+1}{a}$ and $3\dd_2<\frac{3}{a}$. So the above can still be bounded by $1+1+\frac{(a-3)(a+1)}{a}+\frac{3}{a}-(a-1)\dd_1-\dd_2=a-(a-1)\dd_1-\dd_2=y$;

 
 \item $$\Theta_1\leq \dd_3+(1-\dd_2-\dd_l)\leq 1-\dd_l=x,$$
 $$\Theta_2\leq (1-\dd_l-\dd_i)+(1-\dd_l-\dd_j)+\dd_3=y-(1-\dd_1-\dd_2-\dd_3)\leq y.$$
 
  % \item $\Theta_1\leq \dd_3+\dd_1\leq 1-\dd_2=x,$
 
% $\Theta_2\leq (1-\dd_2-\alpha)+(1-\dd_2-\beta)+\dd_3=y+(\dd_3-\dd_2)\leq y$;
    
    %\item $\Theta_1\leq \dd_3+(1-\dd_1-\dd_2)\leq 1-\dd_1=x,$
 
 %$\Theta_2\leq (1-\dd_1-\alpha)+(1-\dd_1-\beta)+(2a-2)\dd_3=(2-\alpha-\beta)+(\dd_1+\dd_2+\dd_3)+(a-3)(\dd_1+2\dd_3)+3\dd_3-a\dd_1-\dd_2\leq (2-\alpha-\beta)+1+(a-3)+1-a\dd_1-\dd_2=y$;
 
%    \item $\Theta_1\leq \dd_3+(1-\dd_1-\dd_2)\leq 1-\dd_1=x,$
 
 %$\Theta_2\leq \dd_3+(1-\dd_1-\alpha)+(2a-3)\dd_3 =y-((a-1)-(a-3)(\dd_1+2\dd_3)-(\dd_1+\dd_2+\dd_3)-3\dd_3)\leq y$.
\end{enumerate}


%{\bf Step 2 (reduce to triangle packing problems):}

 Therefore Lemma \ref{lemma:packing} gives us an disjoint embedding of all the surgery triangles and we get an element $\mathbb{P}$ in $\mathbb{BD}$. When $\XX_{\varepsilon}$ is equal to $\XX$, $\mathbf{R}_{\mathfrak{f}}(\XX)$ is then defined to be this $\mathbb{P}$. Otherwise, according to our construction of $\XX_{\varepsilon}$, there will be one or two (depending on small blowup times) distinguished edges of the bitten Delzant polygon whose affine length are much less than others. As explained in section \ref{extra}, there will be an extra type diagram $\mathbb{P}'\in\mathbb{BD}$ whose $\varepsilon$-shift (or $(\varepsilon,\frac{\varepsilon}{2})$-shift) is $\mathbb{P}$, in which case $\mathbf{R}_{\mathfrak{f}}(\XX)$ is defined to be this $\mathbb{P}'$. 
 
 To see $\mathbf{B}(\mathbf{R}_{\mathfrak{f}}(\XX))=\XX$, first notice that $\mathbf{B}(\mathbb{P})$ always gives $\XX_{\varepsilon}$ due to our construction of $\XX_{tor}$ in section \ref{section:toricmodel} and $\mathbf{B}$ in Proposition \ref{prop:bdtolcy}. So if $\XX=\XX_{\varepsilon}$ the result immediately follows. Otherwise, since the inverse of $\varepsilon$-shift (or $(\varepsilon,\frac{\varepsilon}{2})$-shift) corresponds to $\varepsilon$-replacement $\XX\rightarrow\cdots\rightarrow \XX_{\varepsilon}$, $\mathbf{B}(\mathbb{P}')$ will give us $\XX$. The proof of Theorem \ref{thm:main} is now completed.
 
 \end{proof}
 
 
 %First assume $\alpha,\beta\neq 0$ so that $\alpha=\dd_i$ and $\beta=\dd_j$ for some $i,j$. Note that from our construction of $\mathbf{B}$, $X'$ can be given the framing
%\begin{enumerate}[label=(\roman*)]
%	\item  (a) $\{b,f,e_1,e_2,\cdots\}$ such that $\omega'(b+f)=y,\omega'(b-f)=z,\omega'(e_1)=\dd_1,\omega'(e_2)=1-\dd_i-\dd_j$ and when $p\geq 3$ $\omega'(e_p)$ is some $\dd_q$ with $q\neq 1,i,j$. 

%(b) $\{h,e_1,e_2,\cdots\}$ such that $\omega'(h)=y,\omega'(e_1)=z,\omega'(e_2)=1-\dd_2-\dd_i,\omega'(e_3)=1-\dd_2-\dd_j$ and when $p\geq 4$ $\omega'(e_p)$ is some $\dd_q$ with $q\neq 2,i,j$.
 
 %$\mathbf{R}(\Delta)$ consists of a framed symplectic rational manifold $(S^2\times S^2\#(n-1)\overline{\CC\PP}^2,\Omega)$ with basis $(b,f,e_1,\cdots,e_{n-1})$ such that $\Omega(f)=1-\alpha,\Omega(b)=1-\beta,\Omega(e_1)=\dd_1,\Omega(e_2)=1-\alpha-\beta$ and other $\Omega(e_p)$'s are some $\dd_q$ with $q\neq i,j$. Then apply a diffeomorphism $F:S^2\times S^2\#(n-1)\overline{\CC\PP}^2\rightarrow \CC\PP^2\#(n-1)\overline{\CC\PP}^2$ such that $F_*(f)=H-E_i,F_*(b)=H-E_j,F_*(e_2)=H-E_i-E_j$ and 
 
% \item (a) $\{h,e_1,e_2,\cdots\}$ such that $\omega'(ah-(a-1)e_1)=y,\omega'((-a+1)h+ae_1)=z,\omega'(e_2)=1-\dd_1-\dd_i,\omega'(e_3)=1-\dd_1-\dd_j$ and when $p\geq 4$ $\omega'(e_p)$ is some $\dd_q$ with $q\neq 1,i,j$.

% (b) $\{b,f,e_1,e_2,\cdots\}$ such that $\omega'(b+(a-1)f)=y,\omega'(b-(a-1)f)=z,\omega'(e_1)=1-\dd_1-\dd_2,\omega'(e_2)=1-\dd_1-\dd_i,\omega'(e_3)=1-\dd_1-\dd_j$ and when $p\geq 4$ $\omega'(e_p)$ is some $\dd_q$ with $q\neq 1,2,i,j$.

 
% \item (a) $\{b,f,e_1,e_2,\cdots\}$ such that $\omega'(b+(a-1)f)=y,\omega'(b-(a-1)f)=z,\omega'(e_1)=1-\dd_1-\dd_i$ and when $p\geq 2$ $\omega'(e_p)$ is some $\dd_q$ with $q\neq 1,i$.

 %(b) $\{h,e_1,e_2,\cdots\}$ such that $\omega'((a-1)h-(a-2)e_1)=y,\omega'((-a+2)h+(a-1)e_1)=z,\omega'(e_2)=1-\dd_1-\dd_2,\omega'(e_3)=1-\dd_1-\dd_i$ and when $p\geq 4$ $\omega'(e_p)$ is some $\dd_q$ with $q\neq 1,2,i$.
 

 %\item (a) $\{h,e_1,e_2,\cdots\}$ such that $\omega'(ah-(a-1)e_1)=y,\omega'((-a+1)h+ae_1)=z$ and when $p\geq 2$ $\omega'(e_p)=\dd_p$.

%(b) $\{b,f,e_1,e_2,\cdots\}$ such that $\omega'(b+(a-1)f)=y,\omega'(b-(a-1)f)=z,\omega'(e_1)=1-\dd_1-\dd_2$ and when $p\geq 2$ $\omega'(e_p)$ is some $\dd_q$ with $q\neq 1,2$.

 
 %\item $\{h,e_1,e_2,\cdots\}$ such that $\omega'(h)=y,\omega'(e_1)=z,\omega'(e_2)=1-\dd_1-\dd_l,\omega'(e_3)=1-\dd_2-\dd_l,\omega'(e_4)=1-\dd_l-\dd_i,\omega'(e_5)=1-\dd_l-\dd_j$ and when $p\geq 6$ $\omega'(e_p)$ is some $\dd_q$ with $q\neq 1,2,i,j,l$.

 %\item $\{h,e_1,e_2,\cdots\}$ such that $\omega'(h)=y,\omega'(e_1)=z,\omega'(e_2)=1-\dd_2-\dd_i,\omega'(e_3)=1-\dd_2-\dd_j$ and when $p\geq 4$ $\omega'(e_p)$ is some $\dd_q$ with $q\neq 2,i,j$.

 %\item $\{b,f,e_1,e_2,\cdots\}$ such that $\omega'(b+(a-1)f)=y,\omega'(b-(a-1)f)=z,\omega'(e_1)=1-\dd_1-\dd_2,\omega'(e_2)=1-\dd_1-\dd_i,\omega'(e_3)=1-\dd_1-\dd_j$ and when $p\geq 4$ $\omega'(e_p)$ is some $\dd_q$ with $q\neq 1,2,i,j$.

 %\item  $\{h,e_1,e_2,\cdots\}$ such that $\omega'((a-1)h-(a-2)e_1)=y,\omega'((-a+2)h+(a-1)e_1)=z,\omega'(e_2)=1-\dd_1-\dd_2,\omega'(e_3)=1-\dd_1-\dd_i$ and when $p\geq 4$ $\omega'(e_p)$ is some $\dd_q$ with $q\neq 1,2,i$.
 
%\end{enumerate}

%To show $\mathbf{B}(\mathbf{R}_{\mathfrak{f}}(\XX))=\XX$, we then apply Proposition \ref{prop:Torelli} by defining the integral isometry accordingly:

%\begin{enumerate}[label=(\roman*)]
 %   \item (a) $\gamma(b)=H_j,\gamma(f)=H_i,\gamma(e_1)=E_1,\gamma(e_2)=H_{ij}$ and $\gamma(e_p)=E_q$;

%(b) $\gamma(h)=2H-E_2-E_i-E_j,\gamma(e_1)=H_{ij},\gamma(e_2)=H_{2i},\gamma(e_3)=H_{2j}$ and when $p\geq 4$ $\gamma(e_p)=E_q$ for some $q\neq 2,i,j$;
    
%    \item (a) $\gamma(h)=2H-E_1-E_i-E_j,\gamma(e_1)=H_{ij},\gamma(e_2)=H_{1i},\gamma(e_3)=H_{2j}$ and $\gamma(e_p)=E_q$;

%    (b) $\gamma(b)=2H-E_1-E_2-E_i-E_j,\gamma(f)=H_1,\gamma(e_1)=H_{12},\gamma(e_2)=H_{1i},\gamma(e_3)=H_{1j}$ and when $p\geq 4$ $\gamma(e_p)=E_q$ for some $q\neq 1,2,i,j$;
    
%    \item (a) $\gamma (b)=H_i,\gamma(f)=H_1,\gamma(e_1)=H_{1i}$ and $\gamma(e_p)=E_q$;
    
%    (b) $\gamma(h)=2H-E_1-E_2-E_i,\gamma(e_1)=H_{2i},\gamma(e_2)=H_{12},\gamma(e_3)=H_{1i}$ and $\gamma(e_p)=E_q$;
    
 %   \item (a) $\gamma(h)=H$ and $\gamma(e_p)=E_p$;

 %   (b) $\gamma(b)=H_2,\gamma(f)=H_1,\gamma(e_1)=H_{12}$ and and $\gamma(e_p)=E_q$;
    
 %   \item $\gamma(h)=3H-E_1-E_2-2E_l-E_i-E_j,\gamma(e_1)=2H-E_1-E_2-E_l-E_i-E_j,\gamma(e_2)=H_{1l},\gamma(e_3)=H_{2l},\gamma(e_4)=H_{li},\gamma(e_5)=H_{lj}$ and $\gamma(e_p)=E_q$.
    %\item $\gamma(h)=2H-E_2-E_i-E_j,\gamma(e_1)=H_{ij},\gamma(e_2)=H_{2i},\gamma(e_3)=H_{2j}$ and when $p\geq 4$ $\gamma(e_p)=E_q$ for some $q\neq 2,i,j$;
    %\item $\gamma(b)=2H-E_1-E_2-E_i-E_j,\gamma(f)=H_1,\gamma(e_1)=H_{12},\gamma(e_2)=H_{1i},\gamma(e_3)=H_{1j}$ and when $p\geq 4$ $\gamma(e_p)=E_q$ for some $q\neq 1,2,i,j$;
    %\item $\gamma(h)=2H-E_1-E_2-E_i,\gamma(e_1)=H_{2i},\gamma(e_2)=H_{12},\gamma(e_3)=H_{1i}$ and when $p\geq 4$ $\gamma(e_p)=E_q$ for some $q\neq 1,2,i$.
%\end{enumerate}

%One can firstly check $\gamma$ maps $D'$ to $D$ componentwisely and then show $\gamma_{\RR}([\omega'])=[\omega]$ by checking $\omega'(\cdot)=\omega(\gamma(\cdot))$. Thus all the integral isometries above will satisfy the requirement of Proposition \ref{prop:Torelli}.



%Now let's consider the opposite of Assumption \ref{ass1}, in which case we have to choose other suitable symplectic toric models to make the triangle packing problems solvable. Other than this choice, the proof is almost the same.
%\begin{assumption}\label{ass2}
 %      $E_2$ exactly appears as the non-toric blowup on the component $2H$ in case $(\lowerromannumeral{1})$; $(a+1)H-aE_1$ in case $(\lowerromannumeral{2})$ or $aH+(-a+1)E_1$ in case $(\lowerromannumeral{3})$.
%\end{assumption}


%{\bf Step 3 (remove assumption $\alpha$,$\beta\neq 0$):}
%The remaining cases in the above assumption can be summarized in the following curve configuration diagrams:

%As before, we use the notation $\alpha,\beta$ to denote the symplectic area (some $\dd_i,\dd_j$) of the exceptional classes produced by the toric blowups on the specific nodes ($\alpha,\beta$ maybe $0$ if there is no such toric blowup). Let the initial trapezoid $P_0$ have
%\begin{enumerate}[label=(\roman*)]
 %   \item $k=1,x=1-\dd_2,y=2-\dd_2-\alpha-\beta,z=1-\alpha-\beta$;
  %  \item $k=2a-2,x=1-\dd_1,y=a+1-a\dd_1-\dd_2-\alpha-\beta,z=-a+3+(a-2)\dd_1-\dd_2-\alpha-\beta$;
 %   \item $k=2a-3,x=1-\dd_1,y=a-(a-1)\dd_1-\dd_2-\alpha,z=-a+3+(a-2)\dd_1-\dd_2-\alpha$.
%\end{enumerate}

%Then we can read the corner chopping pattern from the toric blowup pattern on the divisors (shown as `$\cdots$' in the curve configuration diagram) and we will get $P_1$ after the corner chopping. The green exceptional curves in the curve configuration diagram and the non-toric blowup pattern will determine a triangle packing problem on $P_1$. They are solvable since:
%\begin{enumerate}[label=(\roman*)]
 %   \item $\max\{a_{Y_1},b_{Y}\}+\max\{a_{Z_1},b_{Z}\}\leq \dd_3+\dd_1\leq 1-\dd_2=x,$
 
% $\max\{a_{X_1},b_{X}\}+\max\{a_{X'_1},b_{X'}\}+\max\{a_{Y_1},b_{Y}\}\leq (1-\dd_2-\alpha)+(1-\dd_2-\beta)+\dd_3=y+(\dd_3-\dd_2)\leq y$;
 %   \item $\max\{a_{Y_1},b_{Y}\}+\max\{a_{Z_1},b_{Z}\}\leq \dd_3+(1-\dd_1-\dd_2)\leq 1-\dd_1=x,$
 
 %$\max\{a_{X_1},b_{X}\}+\max\{a_{X'_1},b_{X'}\}+(2a-2)\max\{a_{Y_1},b_{Y}\}\leq (1-\dd_1-\alpha)+(1-\dd_1-\beta)+(2a-2)\dd_3=(2-\alpha-\beta)+(\dd_1+\dd_2+\dd_3)+(a-3)(\dd_1+2\dd_3)+3\dd_3-a\dd_1-\dd_2\leq (2-\alpha-\beta)+1+(a-3)+1-a\dd_1-\dd_2=y$;
 %   \item $\max\{a_{Y_1},b_{Y}\}+\max\{a_{Z_1},b_{Z}\}\leq \dd_3+(1-\dd_1-\dd_2)\leq 1-\dd_1=x,$
 
 %$\max\{a_{X_1},b_{X}\}+\max\{a_{X'_1},b_{X'}\}+(2a-3)\max\{a_{Y_1},b_{Y}\}\leq \dd_3+(1-\dd_1-\alpha)+(2a-3)\dd_3 =y-((a-1)-(a-3)(\dd_1+2\dd_3)-(\dd_1+\dd_2+\dd_3)-3\dd_3)\leq y$.
%\end{enumerate}

%Finally we define the integral isometry when $\alpha,\beta\neq 0$:
%\begin{enumerate}[label=(\roman*)]
 %   \item $\gamma(h)=2H-E_2-E_i-E_j,\gamma(e_1)=E_{ij},\gamma(e_2)=H_{2i},\gamma(e_3)=H_{2j}$ and when $p\geq 4$ $\gamma(e_p)=E_q$ for some $q\neq 2,i,j$;
 %   \item $\gamma(b)=2H-E_1-E_2-E_i-E_j,\gamma(f)=H_1,\gamma(e_1)=H_{12},\gamma(e_2)=H_{1i},\gamma(e_3)=H_{1j}$ and when $p\geq 4$ $\gamma(e_p)=E_q$ for some $q\neq 1,2,i,j$;
  %  \item $\gamma(h)=2H-E_1-E_2-E_i,\gamma(e_1)=H_{2i},\gamma(e_2)=H_{12},\gamma(e_3)=H_{1i}$ and when $p\geq 4$ $\gamma(e_p)=E_q$ for some $q\neq 1,2,i$.
%\end{enumerate}
%And if $\alpha$ or $\beta$ is $0$ we still can use the previous orthogonal complement argument since one can check
%\begin{enumerate}[label=(\roman*)]
%    \item $\gamma(h-e_1-e_2)=E_i,\gamma(h-e_1-e_3)=E_j$;
 %   \item $\gamma(f-e_2)=E_i,\gamma(f-e_3)=E_j$;
  %  \item $\gamma(h-e_1-e_3)=E_i$.
   % \end{enumerate}
    %If $\alpha$ or $\beta$ is $0$, there might be a full bite or a degeneration of the trapezoid explained in type $(\upperRomannumeral{1})$ or $(\upperRomannumeral{2})$ extra base diagrams. Then $H_{2}(X';\ZZ)$ and $H_2(X;\ZZ)$ can be realized in the second homology group of their one point or two points blowups\footnote{By abuse of notations, we still use $i,j$ to index the newly created exceptional classes.} as the orthogonal complements of 
%\begin{enumerate}[label=(\roman*)]
 %   \item $f-e_2,b-e_2$ and $E_j=\gamma(f-e_2),E_i=\gamma(b-e_2)$ respectively;
  %  \item $h-e_1-e_2,h-e_1-e_3$ and $E_i=\gamma(h-e_1-e_2),E_j=\gamma(h-e_1-e_3)$ respectively;
   % \item (a) $f-e_1$ and $E_i=\gamma(f-e_1)$;

    %(b) $h-e_1-e_3$ and $E_i=\gamma(h-e_1-e_3)$;
    %\item no need to discuss;
    %\item $h-e_1-e_4,h-e_1-e_5$ and $E_i=\gamma(h-e_1-e_4),E_j=\gamma(h-e_1-e_5)$ respectively.
%\end{enumerate}
 
 % As a consequence, we can use the restriction of $\gamma$ on the orthogonal complement as our desired integral isometry. The proof under assumption \ref{ass1} is now completed.
    
%\begin{definition}
%    An object $(X,\omega,D)$ is called {\bf simple} if there exists a morphism $\Gamma\in Hom((X',\omega',D'),(X,\omega,D))$ with either $X'=\CC\PP^2\#\overline{\CC\PP^2}$ or $D'=(3H-E_1-\cdots-E_{l-1}-2E_l,E_l)$ such that
%    \begin{itemize}
 %       \item If $X'=\CC\PP^2\#\overline{\CC\PP^2}$, when $l(D')=2$, $\Gamma$ contains toric blowups at the two nodes; when $l(D')=3$, $\Gamma$ contains toric blowup at the node between $aH-(a-1)E_1,(-a+2)H+(a-1)E_1$.
  %      \item If $D'=(3H-E_1-\cdots-E_{l-1}-2E_l,E_l)$, then $\Gamma$ contains toric blowups at the two nodes. 
   % \end{itemize}
%\end{definition}

%Next we will use a specific case to illustrate the strategy of finding a base diagram of some $(X,\omega,D)$. Assume $X$ is diffeomorphic to $(l+1)$-th blowup of $\CC\PP^2$ with $l\geq 1$ and $D$ has at least two components. Firstly we have the unique reduced ($c_1$-nef) sequence $(\delta_1,\cdots,\delta_n)$ indicating the symplectic form. Secondly we have a unique model in $\mathcal{LCY}(M_2,\dd_1,\dd_2)$. Let's first discuss the case when this model is 
%\[((k+2)H-(k+1)E_1,-(k-1)H+kE_1-E_2),\,k\geq 0.\]
%There will be a pattern of blowups associated to $D$. This pattern is used in the general counting in [Enumerative section 3.5] but may not be unique due to symplectomorphisms. But we only need to choose one. Let's also firstly assume the pattern contains toric blowups at two nodes of the divisor in $M_2$. Suppose the $i$ and $j$-th blowups are the toric blowups at those two nodes.\todo{definition of quasi-toric} The other toric blowups will give a pattern of truncating the corners of the moment polygon of $\mathbb{F}_{2k+2}$, whose edges have affine lengths
%\[(k+3-(k+2)\dd_1-\dd_2-\dd_i-\dd_j, 1-\dd_1, -(k-1)+k\dd_1-\dd_2-\dd_i-\dd_j, 1-\dd_1).\]

%Note that we should think of this trapezoid coming from contracting $(-1)$-curves $H-E_1-E_2, H-E_1-E_i, H-E_1-E_j, \{E_a\}_{3\leq a\leq l+1, l\neq i,j}$ from the configuration of $(X,\omega,D)$. Together with the information from non-toric blowups pattern, one gets a corner chopping procedure from $P_0$ to $P_1$ and some almost toric blowup sizes associated to each edge of $P_1$. This $P_1$ is actually the symplectic version of toric model in [GHK,F-Engel]. Then we need lemma \ref{lemma:packing} to show there exists a point in the interior of $P$ whose affine distance to each edge of $P$ no less than those sizes, in which case one can have an embedding of disjoint surgery triangles by lemma \ref{lemma:embed}. Combined with the order in the blowup pattern producing $(X,\omega,D)$, a labelling can be assigned as follows. The surgery triangles of sizes $1-\dd_1-\dd_2, 1-\dd_1-\dd_i, 1-\dd_1-\dd_j$ are labelled by $1,2,3$. And for $4\leq a\leq l$, the chopped corner or surgery triangle representing class $E_{\psi(a)}$ is labelled by $a$, where $\psi:\{4,\cdots,l\}\rightarrow \{3,\cdots,l+1\}\setminus\{i,j\}$ is the unique non decreasing bijection.  Therefore we will have a bitten Delzant polygon $\mathcal{P}$. According to our construction of the realization functor $\mathbf{R}$, $\mathbf{B}\circ\mathbf{R}(\mathcal{P})$ will consist of a framed rational symplectic manifold $(\tilde{X},\tilde{\omega})$ with basis
%\[\{B,F,e_1,\cdots,e_l\},\] 
%symplectic form satisfying
%\[\tilde{\omega}(F)=1-\dd_1, \tilde{\omega}(B)=2-\dd_1-\dd_2-\dd_i-\dd_j, \tilde{\omega}(e_a)=\delta_{\psi(a)}\text{ for }a>3,\] 
%\[\tilde{\omega}(e_1)=1-\dd_1-\dd_2, \tilde{\omega}(e_{2})=1-\dd_1-\dd_i, \tilde{\omega}(e_{3})=1-\dd_1-\dd_j,\]
%and a divisor $\tilde{D}$ whose components have homology classes
%\[(B+(k+1)F-e_1-\cdots,\cdots,F-e_2-\cdots,\cdots,B-(k+1)F-\cdots,\cdots,F-e_3-\cdots,\cdots),\]
%where the dots contents are some combinations of $e_i$, determined by the datum of $\mathcal{P}$. Now we use a diffeomorphism $f:\tilde{X}\rightarrow{X}$ such that 
%\[f_*(F)=H-E_1,f_*(B)=2H-E_1-E_2-E_i-E_j,f_*(e_a)=E_{\psi(a)}\text{ for }a>3,\]
%\[f_*(e_1)=H-E_1-E_2, f_*(e_2)=H-E_1-E_i, f_*(e_3)=H-E_1-E_j.\]
%The existence of such a diffeomorphism is well-known. Applying $f$ to transfer the symplectic form and divisor from $\tilde{X}$ to $X$, one can observe that $[(f^{-1})^*\tilde{\omega}]=[\omega]$ and homologically $f_*$ maps components of $\tilde{D}$ to components of $D$. Now by the rigidity of symplectic log Calabi-Yau divisors, $(\tilde{X},\tilde{\omega},\tilde{D})$ is isomorphic to $(X,\omega,D)$ in the category $\mathcal{LCY}$. 

%In summary, what we just did is to associate some special object in $\mathcal{LCY}$ a bitten Delzant polygon $\mathcal{P}$ such that $\mathbf{B}\circ\mathbf{R}(\mathcal{P})$ is isomorphic to that object. Therefore, to check for verification of the essential surjectivity of $\mathbf{B}\circ\mathbf{R}$, we only need to extend such an association for all objects in $\mathcal{LCY}$.


%Note that for the original four edges coming from the trapezoid, the sizes are 
%\[(1-\dd_1-\dd_2,1-\dd_1-\dd_j,\dd_a,1-\dd_1-\dd_i).\]


%Finally we have to discuss all the other models in $M_2$.
%\\



%Given a Delzant polygon $P$ with at least $5$ edges, we can associate a reduced sequence $\delta_1\geq \delta_2\geq\cdots\geq\delta_n$ indicating the reduced symplectic form. We define $\delta_P$ to be the supremum among numbers such that $(\delta_1,\cdots,\delta_n,\delta_P)$ is in the $c_1$-nef cone.

%If a corner of $P$ has $a,b$ as the affine lengths of its two edges and there exists a point in $P$ whose affine distances to both two edges are less than min$\{a,b,\delta_P\}$, then we call it a good corner. It's not hard to see for good corner, we can choose the point with affine distances exactly equal to min$\{a,b,\delta_P\}$. 

%If all the edges have affine lengths $a_1,\cdots,a_{n+2}$, we say an ordered sequence $(c_1,\cdots,c_{n+2})\in [0,a_1)\times\cdots\times[0,a_{n+2})$ is allowable if after some permutations and dropping zeroes $(\delta_1,\cdots,\delta_n,c_1,\cdots,c_{n+2})$ is in the $c_1$-nef cone.

%$P$ is called a good Delzant polygon if for all allowable sequence, we can find a point in $P$ such that the affine distance to each edge is at least $c_j$, $\forall 1\leq j\leq n+2$.

%\begin{theorem}
 %   All Delzant polygons are good.
%\end{theorem}

%We first prove the following:

%\begin{prop}
%    All the corners of a Delzant polygon $P$ with at least $5$ edges are good.
%\end{prop}

%\begin{corollary}
%    If a LCY comes from non-toric blowups of a tLCY, then it has an ATF realization.
%\end{corollary}

\bibliographystyle{amsalpha}
	\bibliography{main}{}
\end{document}
