\documentclass{article}

\usepackage[english]{babel}
\usepackage{bm}

\usepackage[a4paper,top=2cm,bottom=2cm,left=3cm,right=3cm,marginparwidth=1.75cm]{geometry}

% Useful packages
\usepackage{textgreek}
\usepackage{amsmath}
\usepackage{graphicx}
\usepackage[colorlinks=true, allcolors=blue]{hyperref}

\newcommand{\I}{\mathrm{i}}
\newcommand{\be}{\begin{equation}}
\newcommand{\ee}{\end{equation}}
\renewcommand{\Re}{\mathrm{Re}\,}
\renewcommand{\Im}{\mathrm{Im}\,}
\newcommand{\dt}{\mathrm{d}t}
\newcommand{\dom}{\mathrm{d\omega}}

\usepackage{xcolor}

%\title{Atomic intelligence: Machine learning on the fundamental electronic quantum scale}
\title{Ultrafast artificial intelligence: Machine learning with atomic-scale quantum systems}
\author{Thomas Pfeifer$^1$, Matthias Wollenhaupt$^2$, Manfred Lein$^3$}
%\address{}

\date{\small\today}

\begin{document}

\maketitle

\noindent
$^1$Max-Planck-Institut für Kernphysik, 69117 Heidelberg, Germany\\
\noindent
$^2$Carl-von-Ossietzky Universit\"at Oldenburg, Institut f\"ur Physik, Carl-von-Ossietzky-Stra{\ss}e 9-11, 26129 Oldenburg, Germany\\
\noindent
$^3$Leibniz Universit\"at Hannover, Institut f\"ur Theoretische Physik, Appelstra{\ss}e 2, 30167 Hannover, Germany\\
\\
\noindent
corresponding authors: thomas.pfeifer@mpi-hd.mpg.de, matthias.wollenhaupt@uni-oldenburg.de, lein@itp.uni-hannover.de

%\begin{center}
%\includegraphics[height=4cm]{matthias.jpg}   
%\includegraphics[height=4cm]{thomas.jpg}
%\includegraphics[height=4cm]{manfred.jpg}
%\end{center}
%\vspace{.4cm}

\begin{abstract}

We train a model atom to recognize hand-written digits between 0 and 9, employing intense light--matter interaction as a computational resource. For training, individual images of hand-written digits in the range 0-9 are converted into shaped laser pulses (data input pulses). Simultaneously with an input pulse, another shaped pulse (program pulse), polarized in the orthogonal direction, is applied to the atom and the system evolves quantum mechanically according to the time-dependent Schr\"odinger equation. The purpose of the optimal program pulse is to direct the system into specific atomic final states that correspond to the input digits.  A success rate of about 40\% is demonstrated here for a basic optimization scheme, so far limited by the computational power to find the optimal program pulse in a high-dimensional search space. This atomic-intelligence image-recognition scheme is scalable towards larger (e.g. molecular) systems, is readily reprogrammable towards other learning/classification tasks and operates on time scales down to tens of femtoseconds. It has the potential to outpace other currently implemented machine-learning approaches, including the fastest optical on-chip neuromorphic systems and optical accelerators, by orders of magnitude.

\end{abstract}
Artificial intelligence is a research and application field of growing interest, due to recent successes and breakthroughs in deep (multi-layer) learning enabled by constant increases in (classical) computational power.  Within this field, image recognition, i.e., the classification of different but conceptually equivalent (input) images into unique (output) categories, has been one of the prime applications of artificial intelligence and machine learning for many years~\cite{Plamondon2000}.  With recent advances in optical science and technology, in particular optical neuromorphic hardware~\cite{Feldmann2019}, it has recently been possible to accelerate image recognition to sub-nanosecond timescales~\cite{Ashtiani2022}. Here, the operation timescale is determined by the speed of light at which information propagates in microscopic waveguides on the chip, and is thus directly proportional to its size:  The smaller the device and computational units, the faster the clock rates for the recognition operations will become.  The fundamental limit of further minimization is the atomic scale.

In atomic and molecular physics, extreme timescales down to femtosecond and even attosecond duration are currently explored and controlled by measuring and steering the motion of one, two, or several electrons~\cite{Pengel2017,Jiang2022,Kretschmar2022,Yu2022} with lasers.  Specific atomic states can be excited and laser coupled on ultrafast (femtosecond) time scales~\cite{Meister2021}, providing a quantum analog of neurons (quantum states) and axons (laser coupling between states).  Ground-state atoms and molecules are fully quantum-correlated systems and therefore entanglement naturally arises when these systems are fragmenting into two or more particles~\cite{Akoury2007,Schoffler2008}, e.g. due to photoionization or dissociation.  Control of entangled states using attosecond and femtosecond laser pulses has also recently been addressed~\cite{Vrakking2021,Koll2022,Busto2022,Shobeiry2022}. The naturally arising question is whether atoms and their interaction with intense laser light can be used as a high-speed computational resource in applications such as machine learning for image recognition.  This question has recently been addressed for the case of two-class recognition of hand-written digits and three-class recognition of iris-flower types, using the process of high-order harmonic generation and thus an optical output channel~\cite{McCaul2023}. In the present work, we investigate whether an atom could at the same time act as a quantum processor \emph{and} readout register for machine learning, mapping two-dimensional images of digits directly to atomic quantum states.  The latter could then be either read out or serve as input to subsequent (quantum) processing tasks.

%Figure~\ref{ConceptualIdea}
\begin{figure}
    \includegraphics[width=\textwidth] {ConceptualIdea.png}
    \caption{\label{ConceptualIdea} Conceptual representation of atomic artificial intelligence. A quantum system (here: an atom) interacts with an input/data (id) pulse and a program/code (pc) pulse (here: along two orthogonal polarization directions) to deliver the data and the code, respectively.  The quantum state populations after the interaction are read out (e.g. by projecting them into the continuum and employing an electron spectrometer) and the maximum population in a certain state (after renormalization) is used to classify the result.  The pc pulse is found by training on a large number of hand-written digits.}
\end{figure}

In the following, we introduce our scheme and present the results for a model-atom simulation towards a proof-of-principle demonstration. Both the input data and the code that is necessary to process the data are fed into the atom in the form of femtosecond light pulses (id~pulse and pc~pulse), see Fig.~\ref{ConceptualIdea}. Due to the ultrafast time scale, incoherent coupling to the environment can be neglected, i.e., the dynamics of the atom can be described as the time evolution of a pure state evolving according to the time-dependent Schr\"odinger equation (TDSE) 
$\I \partial_t\Psi=\mathcal{H}\Psi$ (atomic units are used unless stated otherwise). We show that 
%by a very simple approach 
the atom can 
%already 
be successfully trained, despite a huge parameter space for the optical program pulse that is just obtained by evolutionary optimization.  In  future work, the training approach can be vastly improved by using more suitable parametrizations for the program pulse and advanced statistical methods based on Bayesian inference.

\begin{figure}
    \includegraphics[width=\textwidth] {Figure--PulseShapes2.png}
    \caption{\label{fig:PulseShapes} A sample of hand-written digits and their encoding into electric fields (in arbitrary units) along one of the polarization axes.}
\end{figure}

\begin{figure}
    \includegraphics[width=\textwidth] {Pulses-Levels-Iteration.png}
    \caption{\label{fig:Pulses-Levels-Iterations} (A) Experimental implementation of the code and the number by shaped pulses in the $x-$ and $y-$direction. (B) Atomic system with target states (0-9). (C) Scheme for the iterative optimization of the code.  Each of the 20 individuals (represented by their 8x8 numerical codes) of one generation are put together with all 100 hand-written digits of the training set and each individual's fitness $F$ is obtained by the sum over all the single fitness values, $F=\sum_{n=1}^{100} f_n$.}
\end{figure}

%To model an atom, 
We employ a multi-level Hamiltonian $\mathcal H_0$ to describe the atom, which is dipole coupled to an external arbitrarily polarized time-dependent laser field vector $\bm{E}(t)=E_x(t)\,\bm{e}_x+E_y(t)\,\bm{e}_y$
with Cartesian unit vectors $\bm{e}_x, \bm{e}_y$. 
The interacting Hamiltonian thus reads
\begin{equation}
    \mathcal H= \mathcal H_0 +  \mathcal V_x\,E_x(t) + \mathcal V_y\,E_y(t).
\end{equation}
Using a 20-level model with basis states 
1s, 2s, 2p$_{-1,1}$, 3s, 3p$_{-1,1}$, 3d$_{-2,0,2}$, 4s, 4p$_{-1,1}$, 4d$_{-2,0,2}$, 4f$_{-3,-1,1,3}$, we have
\begin{align}
&
    \mathcal H_0=
    \begin{pmatrix}
E_{\mathrm{1s}} & 0 & 0 & \\
0 & \!\!E_{\mathrm{2s}} & 0 & \\
0 & 0 & \!\!E_{\mathrm{2p}} & \\[-.23cm]
%0 & 0 & 0 & \!\!E_{\mathrm{2p}}& \\
&&&\hspace*{-.25cm}{\mbox{\footnotesize{$\ddots$}}}
\end{pmatrix}
\end{align}
and
\begin{align}
    \label{coupling}    
    \mathcal V_x=
    \begin{pmatrix}
0 & 0 & a & -a &\\
0 & 0 & a & -a &\\
a & a & 0 & 0 &\\
-a & -a & 0 & 0 & \\[-.25cm]
&&&&\hspace*{-.22cm}{\mbox{\footnotesize{$\ddots$}}}
\end{pmatrix}, 
\hspace{.2cm}
    \mathcal V_y=
    \begin{pmatrix}
0 & 0 & -\I a & -\I a \\
0 & 0 & -\I a & -\I a \\
\I a & \I a & 0 & 0 \\
\I a & \I a & 0 & 0 \\[-.25cm]
&&&&\hspace*{-.22cm}{\mbox{\footnotesize{$\ddots$}}}
\end{pmatrix}.
%\quad
%\begin{matrix}
%\leftarrow \text{transitions to 1s}\phantom{0{-}}\\
%\leftarrow \text{transitions to 2p}_{-1}\\
%\leftarrow \text{transitions to 2p}_{0\phantom{-}}\\
%\leftarrow \text{transitions to 2p}_{+1}
%\end{matrix},
\end{align}
For simplicity, the energies are chosen as $E_{nl}=-1/(n+1)^2$. The coupling
matrix elements read
$\mathcal V_x^{jk} = \, \langle l,m | \sin\theta\cos\phi | l',m' \rangle$ and $\mathcal V_y^{jk} = \, \langle l,m | \sin\theta\sin\phi | l',m' \rangle$,
where the index $j$ corresponds to the state $|n,l,m\rangle$, the index $k$ corresponds to the state $|n',l',m'\rangle$, and $|l,m\rangle$
are angular momentum states. Here, the radial integrals have been set to unity to emphasize the universality of the model. This means that, for example, the number $a$ appearing in Eq.~(\ref{coupling}) is $a = \langle 0,0|\sin\theta\cos\phi|1,-1\rangle$. The 1s state $|1,0,0\rangle$ is taken as the initial state for the time evolution.  
%Considering a set of 20 states with $| i \rangle = | n,l,m \rangle$ and $| j \rangle = | n',l',m' \rangle$, the  matrix elements (abbreviated $a$ in above short-hand representation) of the Hamiltonian read $a=H_0^{ij} = \delta_{ij} \, E_i$, $D_x^{ij} = d_{ij} \, \langle l,m | \hat{x} | l',m' \rangle$ and $D_y^{ij} = d_{ij} \, \langle l,m | \hat{y} | l',m' \rangle$, where $\hat{x}$ and $\hat{y}$ are the cartesian normal vectors and $d_{ij}$ the radial part of the dipole moment. To emphasize the model-independence of our ansatz, we set $d_{ij}=1$.  

%{
%\color{gray}
%{\bfseries{Previously:}}
%In linear representation (polarization axes along unit vectors $\bm{e}_x, %\bm{e}_y$), and written here only for the example of a four-state model with %the set of states (1s, 2p$_{-1}$, 2p$_0$, 2p$_{+1}$) (note: we will employ %20 states for the results shown below, including also 3/4s, 3/4p, 3/4d and %4f), we have the interacting Hamiltonian:
%\begin{equation}
%    \mathcal H= \mathcal H_0 +  \mathcal V_x\,E_x(t) + \mathcal V_y\,E_y(t)
%\end{equation}
%
%\begin{align*}
%&
%    \mathcal H_0=
%    \begin{pmatrix}
%E_{\mathrm{1s}} & 0 & 0 & 0 \\
%0 & E_{\mathrm{2p}} & 0 & 0 \\
%0 & 0 & E_{\mathrm{2p}} & 0 \\
%0 & 0 & 0 & E_{\mathrm{2p}} 
%\end{pmatrix}\\
%&
%    \mathcal V_x=
%    \begin{pmatrix}
%0 & a & 0 & a \\
%a & 0 & 0 & 0 \\
%0 & 0 & 0 & 0 \\
%a & 0 & 0 & 0 
%\end{pmatrix}, 
%    \mathcal V_y=
%    \begin{pmatrix}
%0 & -\I a & 0 & \I a \\
%\I a & 0 & 0 & 0 \\
%0 & 0 & 0 & 0 \\
%-\I a & 0 & 0 & 0 
%\end{pmatrix}
%\quad
%\begin{matrix}
%\leftarrow \text{transitions to 1s}\phantom{0{-}}\\
%\leftarrow \text{transitions to 2p}_{-1}\\
%\leftarrow \text{transitions to 2p}_{0\phantom{-}}\\
%\leftarrow \text{transitions to 2p}_{+1}
%\end{matrix},
%\end{align*}
%with $E_nl=1/(n+1)**2$ as energies (all quantities in atomic units, a.u., %unless noted otherwise) and $a$ dipole matrix elements, taking into account %the geometric (Wigner-Eckart) properties of the angular-momentum eigenstates %to correctly model atomic interaction with polarized light.
%}

In our first approach, the fields $E_x$ and $E_y$ are constructed as follows. From a $\cos^2$-shaped spectral amplitude
\begin{equation}
   \tilde{E}(\omega) = \tilde{E}_0
   \, \cos^2{
   \left( \frac{\omega-\omega_0}{\Omega}{\pi}  \right)},\quad
   \omega_0-\Omega/2 \leq\omega\leq\omega_0+\Omega/2,
\end{equation}
where the total length of the spectrum is $\Omega=\pi/32\, \rm a.u.$ and the central frequency $\omega_0$ corresponds to a laser wavelength of 800$\,$nm,
the field is obtained as a Fourier-synthesis multiplied by an additional $\cos^2$ temporal envelope to restrict the pulse to a finite total length $T$, i.e.,
\begin{equation}
E(t) = \sum_{j=0}^{63}\,\tilde{E}(\omega_j)\,
\cos(-\omega_j t + \varphi_j)\, \cos^2(t\pi/T),\quad
-T/2\leq t \leq T/2
\end{equation}
with
\begin{equation}    
    \omega_j = \omega_0 + (j-31)\,\Delta\omega,\quad   
    \Delta\omega = 2\pi/T,\quad
    T=4096\,\textrm{a.u.},\quad 
    \varphi_j = - 3v_j \textrm{rad}.
\end{equation}
Here, $v_j$ are the 64 values describing the input (hand-written digits from the scikit-learn python package, 8x8 pixel representation, processed column-wise from the lower right to the upper left corner, $0\leq v_j \leq1$) or program (numerical phase function of 64 different values), see Fig.~\ref{fig:PulseShapes} for illustration.

%\begin{figure}
%    \includegraphics[width=\textwidth] {Pulse.png}
%    \caption{\label{fig:Pulse} (A) Experimental implementation of the code and the number by shaped pulses in the %$x-$ and $y-$direction. (B) Atomic system with target states (0-9).}
%\end{figure}

The TDSE is then solved numerically by a split-step operator approach with a time step of 1 a.u., and the state populations $p_i$ (``output'') are read out for each pair of 100 input and 20 program fields.  As not all states will be equally populatable for an arbitrary set of pulses, we define normalized populations $P_i=p_i/p_{0i}$, where $p_{0i}$ is obtained by applying random pc pulses and reading out their final state populations.  Among the final states, we select 10 states (all dipole-accessible 4s, 4p, 4d, 4f states) to encode the digits $i$ from ``0'' to ``9'', and assign to each program field a fitness $F=3N+P$ with $N$ the number of matches between the input (hand-written digit) and classification output (state with highest normalized population).  To reward the pulses with particularly high population in the correct classification state, $P$ represents the sum over all corresponding normalized populations $P_{i}$ for each of the 100 input fields from the training set.

The program fields (here: their 64-step phase functions) are then iteratively optimized by an evolutionary algorithm to maximize their fitness.  The aim is to find an optimal program field that realizes correct classification of all training digits.  The basic evolutionary algorithm employed here for training the model atom uses a population size of 20 individuals, each represented by an array of 64 numbers (its ``genes''), determining the spectral phase of the pc pulse in the same way as for the input pulses. For each individual pc pulse its fitness is calculated by applying the pulse along with 100 hand-written digits from the training set.  The best pc pulse is always kept for the next generation (``cloning'') while the remaining 19 individuals are obtained by a combination of cross-over and mutation (using random numbers) for its 64-number genetic array.

The results of three example optimization runs are shown in Fig.~\ref{optimization}.  While  the percentage of correctly classified digits generally rises throughout the optimization (black line), it is particularly interesting to observe the correlated increase of the success rate on the test set (grey line).  As the 100-sample test set of digits is not shown to the algorithm during training with the 100-sample training set, the increase of the test-set success rate signifies that generalization is achieved in this approach, i.e., small deviations between different hand-written versions of the same digit do not prevent a correct classification.

\begin{figure}
    \begin{center}
    \includegraphics[width=.5\textwidth]{Optimization3.png}
    \end{center}
    \caption{\label{optimization} Results of three example runs of the evolutionary algorithm to optimize the program fields for digit recognition.  Black: Fittest individual when applied to the training set. Grey: Fittest individual when applied to an (unseen) test set for validation of the generalization of learning.  Levels of $>\,$40\% and $\sim\,$30\% recognition success rates ($N$/100 in \%) for training and test sets are routinely achieved, respectively.}
\end{figure}

In summary, we have introduced the concept of optically programmable learning employing atomic quantum states for classification.  For a proof of principle, the concept was applied here to recognition of hand-written digits, implemented with a few-level model of an atom and a straightforward encoding of input/data and program/code as spectral phase functions of two orthogonally polarized femtosecond optical laser fields.

Once the optimal program pulse is known, the digit-recognition code runs on a femtosecond time scale, which is much faster than the processing time in any classical or quantum computer. The ultrafast time evolution excludes environment-induced decoherence and makes the proposed scheme robust. We note that replacing the atom by a larger quantum system, such as a complex molecule, will increase the size of the Hilbert space enormously, while the ultrafast time scale is retained.  

A key benefit of our approach is the following:  The program field could be modified towards other computational tasks, such as identifying letters, other images, or prime numbers in future applications, always using the same quantum system as computational core.  The quantum system can thus be considered an optically reprogrammable multi-purpose quantum processor \mbox{(OpReMuQ)}.  Moreover, due to the intrinsic memory capability of atoms, they can be used as nodes in larger networks of many atoms (or molecules), thus involving many entangled electrons and vastly increasing the number of layers that can be used for achieving more complex tasks.  While in this first demonstration we essentially only explored a one-particle few-level model, future experimental implementations will involve few- or many-body dynamics to employ a much larger state space and thus higher effective number of coupled layers of states (neurons).  A crucial point that sets this scheme apart from existing quantum-information approaches and platforms is the fact that (entangling) operations are not performed on spatially \emph{separated} entities but on \emph{compact} systems of interacting particles.  The implementation of traditional quantum gates is therefore not possible; instead, suitably-shaped structured light has to be found to perform the operations necessary for the envisaged task.

Our scheme can be viewed as machine learning with a microscopic quantum system. Similar to classical machine learning, there is not necessarily a simple explanation as to how the trained machine takes its decisions. This ``lack of insight'' has not prevented conventional artificial intelligence from revolutionizing technology in recent years. We thus expect that ultrafast artificial intelligence on the atomic scale has the potential to exploit quantum mechanics for high-speed computational tasks in the future.


%\section{Supplementary Material}\label{SM}

%\subsection{Optimization Algorithm}
%\subsubsection{Evolutionary Algorithm}
%For finding the program code (pc) pulses, the basic evolutionary algorithm employed here for training the model atom uses a %population size of 20 individuals, represented by an array of 64 numbers (its "genes") between 0 and 2$\pi$, determining the %spectral phase of the pc pulse while its spectrum is always chosen identical.  The latter is done to allow for most straightforward %experimental implementation. For each individual pc pulse its fitness is calculated by applying the pulse along with 100 %hand-written digits from the training set.  The best pc pulse is always kept for the next generation ("cloning") while the remaining %19 individuals are obtained by a combination of cross-over and mutation (random numbers) for its 64-number genetic array.

%\subsection{Spectral modulation}
%The super-Gaussian spectrum
%\begin{equation}
%   \tilde{E}(\omega) = \frac{E_0 \Delta t}{4 \sqrt{\ln 2 }} \, e^{-\ln(4) \left( %\frac{\omega-\omega_0}{\Delta \omega}  \right)^6}
%\end{equation}
%with a spectral width of $\Delta \omega = 1.10904 \, \rm fs^{-1}$ is modulated %by a wavelet-type spectral phase function 
%\begin{equation} \label{eq:SpectralPhaseModulation}
%    \varphi(\omega) = \sum_{j,k=1}^8 A_{jk} \, e^{-\ln(4) \left( %\frac{\omega-\omega_j}{\varpi} \right)^2} \, \cos[ k (\omega-\omega_j) \, %\tau ].
%\end{equation}
%where $A_{jk}$ are the entries of the number matrix, $\omega_j = \omega_0 + %(j-4.5) \, \Omega $ is the shifted angular frequency and $\tau=3.0\,\rm fs$, %$\Omega=0.3 \,\rm fs^{-1}$ and $\varpi=0.833 \,\rm fs^{-1}$ are the parameters %of the modulation function.
%
%\subsection{Pulses}
%
%\begin{figure}
%    \includegraphics[width=\textwidth] {Fields.png}
%    \caption{\label{fig:Optimization} Some shaped pulses derived from the python %numbers using Eq.~\eqref{eq:SpectralPhaseModulation} .}
%\end{figure}

%\subsection{Hamiltonian}
%The Hamiltonian in dipole approximation, using the dipole operator $\mathbf{D}$ with components $D_x$,$D_y$ (in a one-electron atom, %we have $D_x=-ex$, $D_y=-ey$) is
%\be
%    \label{hamiltonianlin}
%    H = H_0 - \mathbf{D}\cdot\mathbf{E}(t) = H_0 - D_x E_x(t) - D_y E_y(t).
%\ee
%Considering a set of 20 states with $| i \rangle = | n,l,m \rangle$ and $| j \rangle = | n',l',m' \rangle$, the  matrix elements of %the Hamiltonian read $H_0^{ij} = \delta_{ij} \, E_i$, $D_x^{ij} = d_{ij} \, \langle l,m | \hat{x} | l',m' \rangle$ and $D_y^{ij} = %d_{ij} \, \langle l,m | \hat{y} | l',m' \rangle$, where $\hat{x}$ and $\hat{y}$ are the cartesian normal vectors and $d_{ij}$ the %radial part of the dipole moment. To emphasize the model-independence of our ansatz, we set $d_{ij}=1$.  
%
%\tiny
%\arraycolsep=1.4pt\def\arraystretch{1.8}

%\begin{equation*}
%    \mathcal{H}_0 = 
%   \left(
%    \begin{array}{cccccccccccccccccccc}
%     0. & 0. & 0. & 0. & 0. & 0. & 0. & 0. & 0. & 0. & 0. & 0. & 0. & 0. & 0. & %0. & 0. & 0. & 0. & 0. \\
%     0. & 1.941 & 0. & 0. & 0. & 0. & 0. & 0. & 0. & 0. & 0. & 0. & 0. & 0. & 0. %& 0. & 0. & 0. & 0. & 0. \\
%     0. & 0. & 1.941 & 0. & 0. & 0. & 0. & 0. & 0. & 0. & 0. & 0. & 0. & 0. & 0. %& 0. & 0. & 0. & 0. & 0. \\
%     0. & 0. & 0. & 1.941 & 0. & 0. & 0. & 0. & 0. & 0. & 0. & 0. & 0. & 0. & 0. %& 0. & 0. & 0. & 0. & 0. \\
%     0. & 0. & 0. & 0. & 3.451 & 0. & 0. & 0. & 0. & 0. & 0. & 0. & 0. & 0. & 0. %& 0. & 0. & 0. & 0. & 0. \\
%     0. & 0. & 0. & 0. & 0. & 3.451 & 0. & 0. & 0. & 0. & 0. & 0. & 0. & 0. & 0. %& 0. & 0. & 0. & 0. & 0. \\
%     0. & 0. & 0. & 0. & 0. & 0. & 3.451 & 0. & 0. & 0. & 0. & 0. & 0. & 0. & 0. %& 0. & 0. & 0. & 0. & 0. \\
%     0. & 0. & 0. & 0. & 0. & 0. & 0. & 3.451 & 0. & 0. & 0. & 0. & 0. & 0. & 0. %& 0. & 0. & 0. & 0. & 0. \\
%     0. & 0. & 0. & 0. & 0. & 0. & 0. & 0. & 3.451 & 0. & 0. & 0. & 0. & 0. & 0. %& 0. & 0. & 0. & 0. & 0. \\
%     0. & 0. & 0. & 0. & 0. & 0. & 0. & 0. & 0. & 3.451 & 0. & 0. & 0. & 0. & 0. %& 0. & 0. & 0. & 0. & 0. \\
%     0. & 0. & 0. & 0. & 0. & 0. & 0. & 0. & 0. & 0. & 4.649 & 0. & 0. & 0. & 0. %& 0. & 0. & 0. & 0. & 0. \\
%     0. & 0. & 0. & 0. & 0. & 0. & 0. & 0. & 0. & 0. & 0. & 4.649 & 0. & 0. & 0. %& 0. & 0. & 0. & 0. & 0. \\
%     0. & 0. & 0. & 0. & 0. & 0. & 0. & 0. & 0. & 0. & 0. & 0. & 4.649 & 0. & 0. %& 0. & 0. & 0. & 0. & 0. \\
%     0. & 0. & 0. & 0. & 0. & 0. & 0. & 0. & 0. & 0. & 0. & 0. & 0. & 4.649 & 0. %& 0. & 0. & 0. & 0. & 0. \\
%     0. & 0. & 0. & 0. & 0. & 0. & 0. & 0. & 0. & 0. & 0. & 0. & 0. & 0. & 4.649 %& 0. & 0. & 0. & 0. & 0. \\
%     0. & 0. & 0. & 0. & 0. & 0. & 0. & 0. & 0. & 0. & 0. & 0. & 0. & 0. & 0. & %4.649 & 0. & 0. & 0. & 0. \\
%     0. & 0. & 0. & 0. & 0. & 0. & 0. & 0. & 0. & 0. & 0. & 0. & 0. & 0. & 0. & %0. & 4.649 & 0. & 0. & 0. \\
%     0. & 0. & 0. & 0. & 0. & 0. & 0. & 0. & 0. & 0. & 0. & 0. & 0. & 0. & 0. & %0. & 0. & 4.649 & 0. & 0. \\
%     0. & 0. & 0. & 0. & 0. & 0. & 0. & 0. & 0. & 0. & 0. & 0. & 0. & 0. & 0. & %0. & 0. & 0. & 4.649 & 0. \\
%     0. & 0. & 0. & 0. & 0. & 0. & 0. & 0. & 0. & 0. & 0. & 0. & 0. & 0. & 0. & %0. & 0. & 0. & 0. & 4.649 \\
%    \end{array}
%    \right)
%\end{equation*}


%\begin{equation*}
%    \mathcal{D}_x = 
%    \left(
%    \begin{array}{cccccccccccccccccccc}
%     0. & 0. & 0.41 & -0.41 & 0. & 0.41 & -0.41 & 0. & 0. & 0. & 0. & 0.41 & -0.41 & 0. & 0. & 0. & 0. & 0. & 0. & 0. \\
%     0. & 0. & 0.41 & -0.41 & 0. & 0.41 & -0.41 & 0. & 0. & 0. & 0. & 0.41 & -0.41 & 0. & 0. & 0. & 0. & 0. & 0. & 0. \\
%     0.41 & 0.41 & 0. & 0. & 0.41 & 0. & 0. & 0.45 & -0.18 & 0. & 0.41 & 0. & 0. & 0.45 & -0.18 & 0. & 0. & 0. & 0. & 0. \\
%     -0.41 & -0.41 & 0. & 0. & -0.41 & 0. & 0. & 0. & 0.18 & -0.45 & -0.41 & 0. & 0. & 0. & 0.18 & -0.45 & 0. & 0. & 0. & 0. \\
%     0. & 0. & 0.41 & -0.41 & 0. & 0.41 & -0.41 & 0. & 0. & 0. & 0. & 0.41 & -0.41 & 0. & 0. & 0. & 0. & 0. & 0. & 0. \\
%     0.41 & 0.41 & 0. & 0. & 0.41 & 0. & 0. & 0.45 & -0.18 & 0. & 0.41 & 0. & 0. & 0.45 & -0.18 & 0. & 0. & 0. & 0. & 0. \\
%     -0.41 & -0.41 & 0. & 0. & -0.41 & 0. & 0. & 0. & 0.18 & -0.45 & -0.41 & 0. & 0. & 0. & 0.18 & -0.45 & 0. & 0. & 0. & 0. \\
%     0. & 0. & 0.45 & 0. & 0. & 0.45 & 0. & 0. & 0. & 0. & 0. & 0.45 & 0. & 0. & 0. & 0. & 0.46 & -0.12 & 0. & 0. \\
%     0. & 0. & -0.18 & 0.18 & 0. & -0.18 & 0.18 & 0. & 0. & 0. & 0. & -0.18 & 0.18 & 0. & 0. & 0. & 0. & 0.29 & -0.29 & 0. \\
%     0. & 0. & 0. & -0.45 & 0. & 0. & -0.45 & 0. & 0. & 0. & 0. & 0. & -0.45 & 0. & 0. & 0. & 0. & 0. & 0.12 & -0.46 \\
%     0. & 0. & 0.41 & -0.41 & 0. & 0.41 & -0.41 & 0. & 0. & 0. & 0. & 0.41 & -0.41 & 0. & 0. & 0. & 0. & 0. & 0. & 0. \\
%     0.41 & 0.41 & 0. & 0. & 0.41 & 0. & 0. & 0.45 & -0.18 & 0. & 0.41 & 0. & 0. & 0.45 & -0.18 & 0. & 0. & 0. & 0. & 0. \\
%     -0.41 & -0.41 & 0. & 0. & -0.41 & 0. & 0. & 0. & 0.18 & -0.45 & -0.41 & 0. & 0. & 0. & 0.18 & -0.45 & 0. & 0. & 0. & 0. \\
%     0. & 0. & 0.45 & 0. & 0. & 0.45 & 0. & 0. & 0. & 0. & 0. & 0.45 & 0. & 0. & 0. & 0. & 0.46 & -0.12 & 0. & 0. \\
%     0. & 0. & -0.18 & 0.18 & 0. & -0.18 & 0.18 & 0. & 0. & 0. & 0. & -0.18 & 0.18 & 0. & 0. & 0. & 0. & 0.29 & -0.29 & 0. \\
%     0. & 0. & 0. & -0.45 & 0. & 0. & -0.45 & 0. & 0. & 0. & 0. & 0. & -0.45 & 0. & 0. & 0. & 0. & 0. & 0.12 & -0.46 \\
%     0. & 0. & 0. & 0. & 0. & 0. & 0. & 0.46 & 0. & 0. & 0. & 0. & 0. & 0.46 & 0. & 0. & 0. & 0. & 0. & 0. \\
%     0. & 0. & 0. & 0. & 0. & 0. & 0. & -0.12 & 0.29 & 0. & 0. & 0. & 0. & -0.12 & 0.29 & 0. & 0. & 0. & 0. & 0. \\
%     0. & 0. & 0. & 0. & 0. & 0. & 0. & 0. & -0.29 & 0.12 & 0. & 0. & 0. & 0. & -0.29 & 0.12 & 0. & 0. & 0. & 0. \\
%     0. & 0. & 0. & 0. & 0. & 0. & 0. & 0. & 0. & -0.46 & 0. & 0. & 0. & 0. & 0. & -0.46 & 0. & 0. & 0. & 0. \\
%    \end{array}
%    \right)
%\end{equation*}
%
%\begin{equation*}
%    \mathcal{D}_y = i \cdot
%    \left(
%    \begin{array}{cccccccccccccccccccc}
%     0. & 0. & -0.41 & -0.41 & 0. & -0.41 & -0.41 & 0. & 0. & 0. & 0. & -0.41 & -0.41 & 0. & 0. & 0. & 0. & 0. & 0. & 0. \\
%     0. & 0. & -0.41 & -0.41 & 0. & -0.41 & -0.41 & 0. & 0. & 0. & 0. & -0.41 & -0.41 & 0. & 0. & 0. & 0. & 0. & 0. & 0. \\
%     0.41 & 0.41 & 0. & 0. & 0.41 & 0. & 0. & -0.45 & -0.18 & 0. & 0.41 & 0. & 0. & -0.45 & -0.18 & 0. & 0. & 0. & 0. & 0. \\
%     0.41 & 0.41 & 0. & 0. & 0.41 & 0. & 0. & 0. & -0.18 & -0.45 & 0.41 & 0. & 0. & 0. & -0.18 & -0.45 & 0. & 0. & 0. & 0. \\
%     0. & 0. & -0.41 & -0.41 & 0. & -0.41 & -0.41 & 0. & 0. & 0. & 0. & -0.41 & -0.41 & 0. & 0. & 0. & 0. & 0. & 0. & 0. \\
%     0.41 & 0.41 & 0. & 0. & 0.41 & 0. & 0. & -0.45 & -0.18 & 0. & 0.41 & 0. & 0. & -0.45 & -0.18 & 0. & 0. & 0. & 0. & 0. \\
%     0.41 & 0.41 & 0. & 0. & 0.41 & 0. & 0. & 0. & -0.18 & -0.45 & 0.41 & 0. & 0. & 0. & -0.18 & -0.45 & 0. & 0. & 0. & 0. \\
%     0. & 0. & 0.45 & 0. & 0. & 0.45 & 0. & 0. & 0. & 0. & 0. & 0.45 & 0. & 0. & 0. & 0. & -0.46 & -0.12 & 0. & 0. \\
%     0. & 0. & 0.18 & 0.18 & 0. & 0.18 & 0.18 & 0. & 0. & 0. & 0. & 0.18 & 0.18 & 0. & 0. & 0. & 0. & -0.29 & -0.29 & 0. \\
%     0. & 0. & 0. & 0.45 & 0. & 0. & 0.45 & 0. & 0. & 0. & 0. & 0. & 0.45 & 0. & 0. & 0. & 0. & 0. & -0.12 & -0.46 \\
%     0. & 0. & -0.41 & -0.41 & 0. & -0.41 & -0.41 & 0. & 0. & 0. & 0. & -0.41 & -0.41 & 0. & 0. & 0. & 0. & 0. & 0. & 0. \\
%     0.41 & 0.41 & 0. & 0. & 0.41 & 0. & 0. & -0.45 & -0.18 & 0. & 0.41 & 0. & 0. & -0.45 & -0.18 & 0. & 0. & 0. & 0. & 0. \\
%     0.41 & 0.41 & 0. & 0. & 0.41 & 0. & 0. & 0. & -0.18 & -0.45 & 0.41 & 0. & 0. & 0. & -0.18 & -0.45 & 0. & 0. & 0. & 0. \\
%     0. & 0. & 0.45 & 0. & 0. & 0.45 & 0. & 0. & 0. & 0. & 0. & 0.45 & 0. & 0. & 0. & 0. & -0.46 & -0.12 & 0. & 0. \\
%     0. & 0. & 0.18 & 0.18 & 0. & 0.18 & 0.18 & 0. & 0. & 0. & 0. & 0.18 & 0.18 & 0. & 0. & 0. & 0. & -0.29 & -0.29 & 0. \\
%     0. & 0. & 0. & 0.45 & 0. & 0. & 0.45 & 0. & 0. & 0. & 0. & 0. & 0.45 & 0. & 0. & 0. & 0. & 0. & -0.12 & -0.46 \\
%     0. & 0. & 0. & 0. & 0. & 0. & 0. & 0.46 & 0. & 0. & 0. & 0. & 0. & 0.46 & 0. & 0. & 0. & 0. & 0. & 0. \\
%     0. & 0. & 0. & 0. & 0. & 0. & 0. & 0.12 & 0.29 & 0. & 0. & 0. & 0. & 0.12 & 0.29 & 0. & 0. & 0. & 0. & 0. \\
%     0. & 0. & 0. & 0. & 0. & 0. & 0. & 0. & 0.29 & 0.12 & 0. & 0. & 0. & 0. & 0.29 & 0.12 & 0. & 0. & 0. & 0. \\
%     0. & 0. & 0. & 0. & 0. & 0. & 0. & 0. & 0. & 0.46 & 0. & 0. & 0. & 0. & 0. & 0.46 & 0. & 0. & 0. & 0. \\
%    \end{array}
%    \right)
%\end{equation*}
%
%\normalsize
%
%
%%\section*{References}
%%Feldmann et al.
%%Nature 569, 208–214 (2019)
%%Neuromorphic optical network.
%%\\ \\
%%Ashtiani et al.
%%Nature 606,501–506 (2022):
%%Image recognition, optical chip, sub-nanosecond
%%\\ \\
%%R. Plamondon, S.N. Srihari,
%%On-line and off-line handwriting recognition: A comprehensive survey
%%IEEE Transactions on Pattern Analysis and Machine Intelligence 22, 63-84 (2000)
%%I10.1109/34.824821

\bibliographystyle{unsrt}

\bibliography{bibatomicai}

\end{document}