%% ****** Start of file apstemplate.tex ****** %
%%
%%
%%   This file is part of the APS files in the REVTeX 4.2 distribution.
%%   Version 4.2a of REVTeX, January, 2015
%%
%%
%%   Copyright (c) 2015 The American Physical Society.
%%
%%   See the REVTeX 4 README file for restrictions and more information.
%%
%
% This is a template for producing manuscripts for use with REVTEX 4.2
% Copy this file to another name and then work on that file.
% That way, you always have this original template file to use.
%
% Group addresses by affiliation; use superscriptaddress for long
% author lists, or if there are many overlapping affiliations.
% For Phys. Rev. appearance, change preprint to twocolumn.
% Choose pra, prb, prc, prd, pre, prl, prstab, prstper, or rmp for journal
%  Add 'draft' option to mark overfull boxes with black boxes
%  Add 'showkeys' option to make keywords appear
\documentclass[aps,prb,twocolumn,superscriptaddress]{revtex4-2}
%\documentclass[aps,prl,preprint,superscriptaddress]{revtex4-2}
%\documentclass[aps,prl,reprint,groupedaddress]{revtex4-2}

% You should use BibTeX and apsrev.bst for references
% Choosing a journal automatically selects the correct APS
% BibTeX style file (bst file), so only uncomment the line
% below if necessary.
\bibliographystyle{apsrev4-2}

%Wanted packages:
\usepackage{graphicx}% Include figure files
\usepackage[dvipsnames]{xcolor} %ADD COLOR TO TEXT
\usepackage{amssymb} %So that \square is visible

\begin{document}

% Use the \preprint command to place your local institutional report
% number in the upper righthand corner of the title page in preprint mode.
% Multiple \preprint commands are allowed.
% Use the 'preprintnumbers' class option to override journal defaults
% to display numbers if necessary
%\preprint{}

%Title of paper
\title{Wafer-scale method for amorphizing superconducting thin films}

% repeat the \author .. \affiliation  etc. as needed
% \email, \thanks, \homepage, \altaffiliation all apply to the current
% author. Explanatory text should go in the []'s, actual e-mail
% address or url should go in the {}'s for \email and \homepage.
% Please use the appropriate macro foreach each type of information

% \affiliation command applies to all authors since the last
% \affiliation command. The \affiliation command should follow the
% other information
% \affiliation can be followed by \email, \homepage, \thanks as well.
\author{Katja Kohop\"a\"a}
\email[]{katja.kohopaa@vtt.fi}
%\homepage[]{Your web page}
%\thanks{}
%\altaffiliation{}
\affiliation{QTF Centre of Excellence, VTT Technical Research Centre of Finland Ltd, P.O. Box 1000, FI-02044 VTT, Finland}
\author{Alberto Ronzani}
\affiliation{QTF Centre of Excellence, VTT Technical Research Centre of Finland Ltd, P.O. Box 1000, FI-02044 VTT, Finland}
\author{Robab Najafi Jabdaraghi}
\affiliation{VTT Technical Research Centre of Finland Ltd, P.O. Box 1000, FI-02044 VTT, Finland}
\author{Arijit Bera}
\affiliation{VTT Technical Research Centre of Finland Ltd, P.O. Box 1000, FI-02044 VTT, Finland}
\author{M\'ario Ribeiro}
\affiliation{QTF Centre of Excellence, VTT Technical Research Centre of Finland Ltd, P.O. Box 1000, FI-02044 VTT, Finland}
\author{Dibyendu Hazra}
\affiliation{QTF Centre of Excellence, VTT Technical Research Centre of Finland Ltd, P.O. Box 1000, FI-02044 VTT, Finland}
\author{Emma Mykk\"anen}
\affiliation{QTF Centre of Excellence, VTT Technical Research Centre of Finland Ltd, P.O. Box 1000, FI-02044 VTT, Finland}
\author{Jorden Senior}
\affiliation{QTF Centre of Excellence, VTT Technical Research Centre of Finland Ltd, P.O. Box 1000, FI-02044 VTT, Finland}
\affiliation{IST Austria, Am Campus 1, 3400 Klosterneuburg, Austria}
\author{Mika Prunnila}
\affiliation{QTF Centre of Excellence, VTT Technical Research Centre of Finland Ltd, P.O. Box 1000, FI-02044 VTT, Finland}
\author{Joonas Govenius}
\affiliation{QTF Centre of Excellence, VTT Technical Research Centre of Finland Ltd, P.O. Box 1000, FI-02044 VTT, Finland}
\author{Janne S. Lehtinen}
%\altaffiliation{Present address: SemiQon}
\affiliation{VTT Technical Research Centre of Finland Ltd, P.O. Box 1000, FI-02044 VTT, Finland}
\author{Antti Kemppinen}
\affiliation{QTF Centre of Excellence, VTT Technical Research Centre of Finland Ltd, P.O. Box 1000, FI-02044 VTT, Finland}

%Collaboration name if desired (requires use of superscriptaddress
%option in \documentclass). \noaffiliation is required (may also be
%used with the \author command).
%\collaboration can be followed by \email, \homepage, \thanks as well.
%\collaboration{}
%\noaffiliation

\date{\today}

\begin{abstract}

We demonstrate ion irradiation as a wafer-scale method for the fabrication of amorphous superconducting thin films. We explore irradiation using argon and gallium ions on single-element and compound materials. Our results indicate that both ions increase disorder in the atomic structure in a qualitatively similar manner, i.e., they destroy the grain structure, increase resistivity and alter the superconducting transition temperature. However, argon irradiation tends to yield gas pockets that can be detrimental for applications. We show that gallium irradiation allows to produce a thin, uniform, and amorphous molybdenum silicide film that is promising, e.g., for superconducting nanowire single-photon detectors.


\end{abstract}

% insert suggested keywords - APS authors don't need to do this
%\keywords{}

%\maketitle must follow title, authors, abstract, and keywords
\maketitle

% body of paper here - Use proper section commands
% References should be done using the \cite, \ref, and \label commands

\section{Introduction} \label{introduction}

Nanoscale superconducting structures allow both new functionalities~\cite{goltsman_picosecond_2001, esmaeil_zadeh_superconducting_2021,astafiev_coherent_2012, lehtinen_coulomb_2012, shaikhaidarov_quantized_2022} and opportunities for improving the packaging density and scaling up quantum technologies~\cite{zhao_merged-element_2020}. Typical superconducting devices are fabricated from polycrystalline thin films, such as sputter deposited Nb or evaporated Al. However, the geometries of nanostructures become irreproducible when their critical dimensions approach the grain size of the material. One promising  way to solve  this  problem is  to  use  novel, single-crystalline two-dimensional superconductors~\cite{zou_superconductivity_2017,cao_unconventional_2018} whose processing, however, is not sufficiently mature to allow large-scale fabrication. Here we study an opposite approach to improve reproducibility: decreasing the grain size, ultimately down to that of an amorphous material.

Increase of atomic disorder typically enhances the normal-state resistivity which, at nanoscale thickness, enables films with high sheet resistance $R_{\square}$ and kinetic inductance $L_{\square}=h R_{\square}/(2\pi^2\Delta)$ in the normal and superconducting states, respectively. Here, $h$ is the Planck constant and $\Delta$ is the superconducting energy gap. These properties allow, e.g., compact superinductors based on high $L_{\square}$ for quantum information processing~\cite{maleeva_circuit_2018,grunhaupt_granular_2019}, quantum phase slip (QPS) devices for quantum metrology~\cite{astafiev_coherent_2012, lehtinen_coulomb_2012, shaikhaidarov_quantized_2022}, and superconducting nanowire single-photon detectors (SNSPD) for quantum communication~\cite{goltsman_picosecond_2001, esmaeil_zadeh_superconducting_2021}. Typical $R_{\square}$ may span roughly 1--4~$\mathrm{k\Omega}$ for superinductors or QPS and 100--200~$\mathrm{\Omega}$ for SNSPDs. These ranges should be compared to the limit in which superconductivity is destroyed by fluctuations: $R_\square\sim R_Q=h/(4e^2)\approx 6.5$~k$\Omega$~\cite{jaeger_threshold_1986,sacepe_localization_2011} where $R_Q$ is the quantum resistance of Cooper pairs and $e$ is the elementary charge.

Because conventional superconductivity arises from the interplay between electrons and lattice, it is expected that the increase of disorder may affect the critical temperature $T_c$, but the direction of the change is less obvious. Already since the 1950s, several studies on single-element materials, e.g., Al, W, Ga, and Mo, have demonstrated an increase of $T_c$ when the growth of large grains has been prevented, e.g., by evaporating on a cold substrate or in the presence of oxygen, or by creating a layered structure with other materials~\cite{buckel_einflus_1954,kammerer_superconductivity_1965,abeles_enhancement_1966,strongin_effect_1967,pettit_film_1976,jaeger_threshold_1986}. Evaporating with oxygen is a key fabrication step for the modern superinductors based on granular aluminium~\cite{maleeva_circuit_2018,grunhaupt_granular_2019}. In contrast, applying neutron irradiation to increase disorder in several high-$T_c$ compounds with an $A$-15 lattice yields a dramatic decrease of $T_c$~\cite{sweedler_atomic_1974}. Other examples of the subtle interplay between disorder and superconductivity include, e.g., the dependence of $T_c$ on the crystalline phase of tantalum~\cite{lita_tuning_2005}, and that proton irradiation decreases $T_c$, but increases the upper critical field of Nb~\cite{tanatar_anisotropic_2022}.

The disordered structure is unstable in many single-element materials, and already room temperature conditions are sufficient to crystallize the material into relatively large grains. Disorder can be stabilized, e.g., with oxidized grain surfaces or with alloyed compounds, which may already be of interest in terms of increased $T_c$~\cite{osofsky_new_2001}.

One such compound with a metastable disordered phase is MoSi, which is a well-known material for SNSPDs and can be fabricated by, e.g, sputtering from an alloy-target to a cooled substrate~\cite{bosworth_amorphous_2015} or by co-sputtering at room temperature~\cite{banerjee_characterisation_2017,zhang_physical_2021}. In the referred experiments, the optimal stoichiometry was found to be about 80\% Mo and 20\% Si, as a result of a trade-off between maintaining the stability of an amorphous phase, which requires Si, and maximizing $T_c$, which would favour pure electron-dense amorphous Mo. Other methods for fabricating disordered MoSi and other silicides include mixing a deposited metal film into a silicon substrate by ion irradiation~\cite{tsaur_ionbeaminduced_1979} or by annealing, a technique that was used in the observation of QPS~\cite{lehtinen_superconducting_2017}. Recently, it was shown that focused ion beam (FIB) irradiation by gallium or helium ions increases the disorder, $T_c$, and $R_\square$ of MoSi formed by thermal annealing~\cite{mykkanen_enhancement_2020}. The FIB method is a relatively slow direct-writing process, with time consumption that scales linearly with the processed area.

In this article, we explore ion irradiation as a wafer-scale process for increasing disorder in superconducting thin films with high throughput and reproducibility. We study both single-element materials and compounds in two common groups of superconductors: nitrides and silicides. The films studied in this paper have been fabricated by sputtering, which typically yields polycrystalline films. In the present work, our focus is two-fold: \emph{(i)} tentative testing for a wide range of materials and \emph{(ii)} systematically investigating MoSi, which has shown promising results under local irradiation treatment~\cite{mykkanen_enhancement_2020}.



\section{Materials and methods}

\begin{figure}[h]
    \includegraphics[width=\linewidth]{wafer_illustration_Q3_tuned_colors_NEW_NAMES_SMALLERSIZE_dpi40.png}
    \caption{(a) Illustration of the wafer structure and the ion irradiation process. (b) Transmission electron microscope image of MoSi film S3. The layer of platinum on top of the protective layer was deposited for the TEM imaging and is not a part of the actual amorphization process. The title describes the nominal thicknesses of the sputtered Mo and Si layers as well as the ion fluence.}
    \label{wafer_illustration}
\end{figure}

Figure~\ref{wafer_illustration} illustrates the wafer structure and ion implantantion treatment and shows the corresponding transmission electron microscope (TEM) image of a MoSi wafer which was taken after the ion irradiation treatment. The thin films were sputtered on top of a silicon wafer terminated by a silicon dioxide layer. A dielectric layer was added on top of some of the wafers to protect the superconducting film from, e.g., oxidation. The wafers were irradiated with argon or gallium ions. As a noble gas, argon should irradiate without any doping of the superconducting film. On the other hand, gallium had provided promising results in Ref.~\cite{mykkanen_enhancement_2020} without any evidence of doping problems.

We used Monte Carlo simulations (not shown) to choose an acceleration voltage for each wafer that is expected to maximize the fraction of kinetic energy deposited into the superconduting film. Therefore, also the majority of the irradiation ions are deposited into the film. We report the irradiation fluences in units of \AA$^{-2} = 10^{16}$~cm$^{-2}$, which roughly corresponds to the number of ions deposited on the area of a single atom. It also yields a rough estimate for the percentage of impurity atoms deposited into the superconducting film, since our films of about $10~\mathrm{nm}$ have the thickness of the order of 100 atoms.


\begin{table}%[H] add [H] placement to break table across pages
\caption{Materials studied with the ion irradiation treatment, including information about the nominal thicknesses of the sputtered films (for MoSi, the first thickness is for the Mo layer and the second thickness for the Si layer), protective layers, and ions used in the treatment.}\label{table_materials} 
\begin{ruledtabular}
\begin{tabular}{c c c c}
Material        & Thickness (nm)    & Protective layer  & Ion \\ \hline
\multicolumn{4}{l}{Single-elements} \\ \hline
Al              & $14$      & none                              & $\mathrm{Ar^+}$  \\ 
V               & $11$--$15$      & none                              & $\mathrm{Ar^+}$  \\
Nb              & $20$      & Al$_\mathrm{2}$O$_\mathrm{3}$/none             & $\mathrm{Ar^+}$  \\ \hline
\multicolumn{4}{l}{Nitrides} \\ \hline
TiN             & $10$      & Al$_\mathrm{2}$O$_\mathrm{3}$/none             & $\mathrm{Ar^+}$  \\
NbN             & $10$      & Al$_\mathrm{2}$O$_\mathrm{3}$/none             & $\mathrm{Ar^+}$  \\ \hline
\multicolumn{4}{l}{Silicides} \\ \hline
MoSi (M-series) & $10+7$     & SiO$_\mathrm{2}$/Al$_\mathrm{2}$O$_\mathrm{3}$& $\mathrm{Ar^+}$/$\mathrm{Ga^+}$ \\
MoSi (S-series) & $5+7$     & Al$_\mathrm{2}$O$_\mathrm{3}$                  & $\mathrm{Ar^+}$/$\mathrm{Ga^+}$ \\
MoSi (S-series) & $3+7$     & Al$_\mathrm{2}$O$_\mathrm{3}$                  & $\mathrm{Ar^+}$/$\mathrm{Ga^+}$  
\end{tabular}
\end{ruledtabular}
\end{table}

Table~\ref{table_materials} provides an overview of the thin film materials, their targeted sputtering thicknesses, protective layers, and the ions of the irradiation treatment. We studied 3 single-element materials (Al, V, and Nb), 2 nitrides (TiN and NbN), and 1 silicide (MoSi). For MoSi, we performed a study of the effect of ion fluence for Mo-rich stoichiometry (M-series wafers), and a smaller number of process variants for Si-rich stoichiometry (S-series).

Nitrides were deposited by reactive sputtering of Nb or Ti in a flow of argon and nitrogen. For TiN, we used a recipe that typically produces $T_c\approx 3.5~\mathrm{K}$. For NbN, the maximum $T_c$ of $14~\mathrm{K}$ is obtained with 1:1 stoichiometry~\cite{linzen_structural_2017}, but in this work, we intentionally used two higher nitrogen flow values to increase $R_\square$ and decrease $T_c$ to about 8~K or 6~K, respectively. For MoSi films, first a layer of molybdenum and then a layer of silicon was deposited, which allowed tuning the stoichiometry by changing the ratio of the film thicknesses. The thickness estimates are based on longer depositions using the same sputtering parameters, and measuring the resulting thicker films. However, the short deposition times of thin films yield a significant uncertainty of the film thickness. We estimate the stoichiometries of about 65\% Mo + 35\% Si, 48\% Mo + 52\% Si, and 35\% Mo + 65\% Si for our film variants with $10~\mathrm{nm}$, $5~\mathrm{nm}$, and $3~\mathrm{nm}$ of Mo (and $7~\mathrm{nm}$ of Si), respectively.

We experimented with two types of protective dielectric films: $20~\mathrm{nm}$ SiO$_\mathrm{2}$ grown by plasma enhanced chemical vapor deposition (PECVD) or $15~\mathrm{nm}$ of Al$_\mathrm{2}$O$_\mathrm{3}$ grown by atomic layer deposition (ALD). The MoSi compounds were formed by annealing in nitrogen environment at $600~^\circ\mathrm{C}$ for $15$ minutes after the deposition of the protective layer. Other materials were not annealed. After thin film fabrication, the wafers went through a wafer-scale broad beam ion irradiation treatment. We varied the ion fluence to observe potential progressive effects of irradiation and used higher fluences when using argon ions because their impact is smaller due to the smaller atomic weight. The protective layers were deposited on many wafers at a time significantly after sputtering the superconducting films, which exposed the superconducting films for native oxide growth before the protection, but not during the irradiation process.

In total, we produced 46 irradiated wafers and measured their $R_\square$ at room temperature. Most of the films were also imaged with a scanning electron microscope (SEM). We used data from this fast characterization to select an illustrative set of samples for the more elaborate cryogenic characterization (presented in Sec.~\ref{Rs_Tc_section}). Based on all these data, a set of 9 films were selected for TEM imaging (Sec.~\ref{Sec_imaging}): NbN with and without protective layer, identical MoSi films with varied ion fluence, films that seemed visually defective in SEM, MoSi with highest Tc, and MoSi films that were promising in all of the initial measurements.

\section{Imaging\label{Sec_imaging}}

Figure~\ref{TEM_MoSi}(a) shows the TEM image of a Mo-poor MoSi film irradiated with argon, which yielded the highest $T_c=5.7$~K of all our MoSi films (see Sec.~\ref{Rs_Tc_section} below). There are gas pockets both in the substrate and protective layer, which deform and alter the thickness of the MoSi film. The energy-dispersive X-ray spectroscopy (EDS) data (not shown) relate the gas pockets to argon. The film is continuous, with grains not resolvable with TEM. Fast Fourier transform (FFT) analysis shows a broad ring at 0.23 nm/cycle. This suggests that the film is amorphous with short-range order.

Figures~\ref{TEM_MoSi}(b--d) show TEM images of Mo-rich MoSi films treated with an increasing fluence of $\mathrm{Ga^+}$ ions. The films are heterogeneous and have a bilayer structure, with a polycrystalline bottom and an amorphous top layer. Inset of FFT analysis of the bottom layer confirms the crystallinity of the grains. Inset of EDS data indicates that the bottom layer consists mostly of molybdenum whereas the top layer is a MoSi alloy. Higher fluences of gallium result in less heterogeneous films.

Figure~\ref{wafer_illustration}(b) shows the TEM image of Mo-poor film (S3) with $5~\mathrm{nm}$ of Mo. The image shows a bilayer structure, similarly as for the Mo-rich films (Fig.~\ref{TEM_MoSi}(b--c)).

Figures~\ref{TEM_MoSi}(e--f) show TEM images of Mo-poor MoSi films (S1--S2) with $3~\mathrm{nm}$ of Mo. The thinner layer of molybdenum yields a better intermixing with the silicon top layer (compare Fig.~\ref{wafer_illustration}(b) and Fig.~\ref{TEM_MoSi}(e)). Further increasing the gallium fluence to 1.0~\AA$^{-2}$ leads to a continuous, homogeneous film of molybdenum and silicon, without resolvable grains by TEM. The FFT analysis shows a broad ring at 0.23 nm/cycle, suggesting that the film is amorphous with short-range order.

\begin{figure*}
    \includegraphics[width=\linewidth]{TEM_MoSi_NoQ3_with_FFT_tuned_colors_EDS_NAMES_SMALLERSIZE_15dpi.png}
    \caption{Transmission electron microscope images of MoSi films S5, M7, M12, M14, S1, and S2 (a--f, respectively). The title of each image describes the nominal thicknesses of the sputtered Mo and Si layers, and the fluence and ion of the irradiation treatment. The layer stack is the same in each figure: the bright bottom layer is SiO$_\mathrm{2}$ (oxide capping of the silicon wafer) and the bright top layer is of Al$_\mathrm{2}$O$_\mathrm{3}$ deposited to protect the darker layer consisting of Mo and Si. The insets in (a--b) and (f) show the FFT analysis from the region shown with the orange square. The inset of (c) shows the EDS data of Mo and Si (blue for Si, red for Mo) combined. The grey dashed lines illustrate where the MoSi layer is shown in the EDS inset. The inset in (e) shows the zoom of the region shown with the blue circle. \label{TEM_MoSi}}
\end{figure*}

Figure~\ref{TEM_MoSi} indicates that our annealing process did not result in complete mixing of Mo and Si especially for the Mo-rich stoichiometries. This problem could be overcome, for example, by sputtering from a MoSi compound target or by cosputtering~\cite{bosworth_amorphous_2015,banerjee_characterisation_2017,zhang_physical_2021}. Ion irradiation improves mixing, but its efficiency is expected to depend on the fact that the irradiation atoms yield the largest impact on atoms with a roughly similar atomic mass. For Ga$^+$ (70 u), the order of our Mo and Si layers is non-optimal, since the ions should be more effective at driving Mo atoms (96 u) into the substrate than Si atoms (28 u) into the Mo layer. On the other hand, Ar$^+$ (40 u) is a closer match to Si.

Figures~\ref{TEM_NbN}(a--b) show the TEM images of NbN films without and with an Al$_\mathrm{2}$O$_\mathrm{3}$ protective layer (films NbN5--NbN6), respectively. Both films were irradiated with the same fluence of argon. The film with a protective layer (NbN6) shows rotated domains and polycrystalline grains. There are well-defined peaks over a diffuse background in FFT analysis performed on the area highlighted with orange square. The unprotected film (NbN5) shows a thin amorphized layer over polycrystalline NbN (see FFT and zoom insets). However, the amorphization of the top-most layer may also result from the deposition of platinum instead of the original irradiation process.

\begin{figure*}
    \includegraphics[width=0.9\linewidth]{TEM_NbN_colors_tuned_with_FFTs_SMALLERSIZE_20dpi.png}
    \caption{Transmission electron microscope images of NbN films NbN5 and NbN6 (a--b, respectively). The title of each image describes the nominal thickness of the sputtered NbN film, and the fluence and ion of the irradiation treatment. The light-colored bottom layer in both images is SiO$_\mathrm{2}$ and the darker layer is the NbN film. (a) NbN5 does not have a protective layer: the NbN layer is in direct contact with the platinum deposited in the TEM imaging process. (b) NbN6 has a protective layer (Al$_\mathrm{2}$O$_\mathrm{3}$) on top of the NbN film. The insets highlighted with orange show the FFT analysis performed to the area shown by the square. The blue inset in (a) shows a zoom of the region shown with the circle.} \label{TEM_NbN}
\end{figure*}

While $\mathrm{Ar^+}$ irradiation seemed to yield detrimental gas pockets into the substrate of MoSi films, it did not into that of NbN films. One potential explanation is in the higher fluence of film S5 (3 \AA$^{-2}$) than our NbN films (1.5 \AA$^{-2}$), but the acceleration voltage has an effect on both the required fluence and the stopping range of the implanted ions. The present amount of data does not rule out the use of $\mathrm{Ar^+}$ ions for irradiation, but it would also be interesting to explore noble gases with higher atomic masses.

\section{Sheet resistance $R_{\square}$ and critical temperature $T_c$\label{Rs_Tc_section}}

We measured the room temperature sheet resistance $R_{\square,\mathrm{RT}}$ with a wafer-scale prober. For cryogenic measurements, we cleaved the wafers into small chips that were fully covered with the film on the top surface. The relative temperature dependence of $R_\square$
down to the base temperature of 0.3~K of our cryostat is recorded through 4-probe measurements. Combining these data sets yields sheet resistance as a function of temperature $R_{\square}(T)$. Figure~\ref{Tc_sweep_figure} shows cryogenic measurements of MoSi, Nb and NbN films that provide data for $R_{\square}$ and $T_c$. The most relevant data is collated into Table~\ref{TEM_MoSi_NbN_table}, which excludes V and TiN that were not superconducting in our experiments. 

\begin{figure}
    \includegraphics[width=\linewidth]{data_from_cold_abadded.png}
    \caption{Normalized sheet resistance $R_{\square}/R_{\square,\mathrm{RT}}$ as a function of temperature of (a) S- and M-series MoSi films and (b) Nb and NbN films.}
    \label{Tc_sweep_figure}
\end{figure}

\begin{table*}
\caption{Electrical measurement data on films that are most relevant for our conclusions. The columns describe  wafer labels, deposition thicknesses, the protective layers, ions and fluences of the ion irradiation treatments, room temperature sheet resistances before and after ion irradiation treatment, corresponding increases in room temperature sheet resistances, sheet resistances at $10~\mathrm{K}$, and critical temperatures. Hyphen means that the corresponding measurement was not performed. The MoSi and NbN films that are shown in Figs.~\ref{wafer_illustration}, ~\ref{TEM_MoSi}, and~\ref{TEM_NbN} are indicated in bold. The table also contains data about the Nb wafers shown in Fig.~\ref{Tc_sweep_figure}. Films NbN7--8 were sputtered with the largest nitrogen flow that is expected to yield $T_c\approx 6$~K before irradiation, in contrast to the value of 8~K for the other NbN films.
\label{TEM_MoSi_NbN_table}}
\begin{ruledtabular}
\begin{tabular}{c c c c c c c c c c}
Label & Material & Prot. & Ion & Fluence (\AA$^{-2}$) & $R_{\square, \mathrm{RT, bef}}$ $(\Omega)$ & $R_{\square, \mathrm{RT}}$ $(\Omega)$ & $R_{\square, \mathrm{RT}}$/$R_{\square, \mathrm{RT, bef}}$ & $R_{\square, \mathrm{10~K}}$ & $T_c$ (K) \\ \hline
M1              & $10~\mathrm{nm}$ Mo $+ 7~\mathrm{nm}$ Si  & SiO$_\mathrm{2}$  & $\mathrm{Ar^+}$ & 1       & --    & 115   & --    & 120   & 4.8   \\
\textbf{M7}     & $10~\mathrm{nm}$ Mo $+ 7~\mathrm{nm}$ Si  & Al$_\mathrm{2}$O$_\mathrm{3}$  & $\mathrm{Ga^+}$ & 0.03    & --    & 48    & --    & 43    & 4.0   \\
\textbf{M12}    & $10~\mathrm{nm}$ Mo $+ 7~\mathrm{nm}$ Si  & Al$_\mathrm{2}$O$_\mathrm{3}$  & $\mathrm{Ga^+}$ & 0.1     & --    & 71    & --    & 69    & 4.3   \\
\textbf{M14}    & $10~\mathrm{nm}$ Mo $+ 7~\mathrm{nm}$ Si  & Al$_\mathrm{2}$O$_\mathrm{3}$  & $\mathrm{Ga^+}$ & 0.6     & --    & 105   & --    & 109   & 4.8   \\
\textbf{S1}     & $3~\mathrm{nm}$ Mo $+ 7~\mathrm{nm}$ Si   & Al$_\mathrm{2}$O$_\mathrm{3}$  & $\mathrm{Ga^+}$ & 0.5     & --    & 166   & --    & 176   & 5.0   \\ 
\textbf{S2}     & $3~\mathrm{nm}$ Mo $+ 7~\mathrm{nm}$ Si   & Al$_\mathrm{2}$O$_\mathrm{3}$  & $\mathrm{Ga^+}$ & 1       & --    & 177   & --    & 188   & 4.9   \\ 
\textbf{S3}     & $5~\mathrm{nm}$ Mo $+ 7~\mathrm{nm}$ Si   & Al$_\mathrm{2}$O$_\mathrm{3}$  & $\mathrm{Ga^+}$ & 0.5     & --    & 169   & --    & 177   & 4.1   \\
\textbf{S5}     & $3~\mathrm{nm}$ Mo $+ 7~\mathrm{nm}$ Si   & Al$_\mathrm{2}$O$_\mathrm{3}$  & $\mathrm{Ar^+}$ & 3       & --    & 177   & --    & 187   & 5.7   \\ \hline
NbN3              & $10~\mathrm{nm}$ NbN                      & none              & $\mathrm{Ar^+}$ & 1.5     & 474   & 2080  & 4.4   & 3340  & 2.5   \\ 
\textbf{NbN5}     & $10~\mathrm{nm}$ NbN                      & none              & $\mathrm{Ar^+}$ & 1.5     & 535   & 2960  & 5.5   & 4760  & 2.4   \\
\textbf{NbN6}     & $10~\mathrm{nm}$ NbN                      & Al$_\mathrm{2}$O$_\mathrm{3}$  & $\mathrm{Ar^+}$ & 1.5     & 530   & 1220  & 2.3   & 2200  & 1.9   \\
NbN7              & $10~\mathrm{nm}$ NbN                      & none              & $\mathrm{Ar^+}$ & 1.5     & 694   & 5090  & 7.3   & --    & --    \\ 
NbN8             & $10~\mathrm{nm}$ NbN                      & Al$_\mathrm{2}$O$_\mathrm{3}$  & $\mathrm{Ar^+}$ & 1.5     & 694   & 1550  & 2.2   & --    & --    \\ \hline
Nb1             & $20~\mathrm{nm}$ Nb                       & none              & $\mathrm{Ar^+}$ & 0.5     & 13    & 30    & 2.3   & 21    & 4.7   \\
Nb2             & $20~\mathrm{nm}$ Nb                       & Al$_\mathrm{2}$O$_\mathrm{3}$  & $\mathrm{Ar^+}$ & 0.5     & 13    & 40    & 3.0   & 33    & 2.9   \\
Nb3             & $20~\mathrm{nm}$ Nb                       & none              & $\mathrm{Ar^+}$ & 1.5     & 13    & 71    & 5.7   & 68    & 1.7   \\
Nb4             & $20~\mathrm{nm}$ Nb                       & Al$_\mathrm{2}$O$_\mathrm{3}$  & $\mathrm{Ar^+}$ & 1.5     & 13    & 64    & 5.0   & 63    & 1.6   \\ \hline
Al1              & $14~\mathrm{nm}$ Al                       & none              & $\mathrm{Ar^+}$ & 1.5     & 3     & 25    & 9.5   & 19    & 1.7   
\end{tabular}
\end{ruledtabular}
\end{table*}

In the MoSi films of Fig.~\ref{Tc_sweep_figure}, the resistance of Mo-rich films containing a layer of polycrystalline Mo decreases slightly when cooled down from room temperature toward $T_c$. Film M14 had only separate spots of Mo and the resistance increases slightly when cooled down. In addition, there is a slight increase of $R_\square$ for the most disordered films (S2 and S5, Mo-poor MoSi). The most disordered films also have the highest $R_\square$ at 300~K. A similar trend can be observed for Nb: The $R_{\square}(T)$ of the most disordered film Nb4 (highest $R_{\square,\mathrm{RT}}$) is constant above $T_c$ whereas $R_{\square}(T)$ of the less disordered Nb films decreases at low temperatures. The NbN films, which are the most resistive of our wafers, become even more resistive at low temperatures before a broad transition into the superconducting state, but the low-temperature increase of $R_{\square}(T)$ is larger for film NbN6 that is less disordered than NbN5.

The main trend in Table~\ref{TEM_MoSi_NbN_table} is that the ion irradiation process and a higher ion fluence generally increase $R_\square$. For NbN, there is a significant difference in $R_\square$ also between films with and without a protective layer, which is in agreement with Fig.~\ref{TEM_NbN}. 

Increasing disorder correlates with the increased $T_c$ of Al and MoSi. For MoSi, however, the interpretation of the results is less straightforward, since most of the films are not homogeneous. The increase of ion fluence also reduces the amount of polycrystalline Mo, which may affect the $T_c$ of the MoSi layer through the promiximity effect. The highest $T_c$ in MoSi, 5.7~K, was obtained for film S5, which is a Mo-poor film irradiated by $\mathrm{Ar^+}$ ions.

In the case of Nb and NbN, the increase of disorder correlates with decreased $T_c$. In TiN and V, we were not able to observe superconductivity at and above $0.3~\mathrm{K}$. We studied Nb, NbN, and TiN samples both with and without protective layers, and therefore the effect is not likely due to oxidation during the irradiation process. Vanadium did not have any protective layer, and it is not clear whether the lack of superconductivity was due to oxidation, impurities or disorder.

One of the limitations of our technique is that it requires a stable disordered phase, whereas some materials, including Al, tend to grow grains already at room temperature~\cite{abeles_enhancement_1966}. Our experiments on aluminium increased $T_c$ up to 1.7~K compared to the bulk value of about 1.2~K and to a reference value of about 1.4~K for 14~nm film deposited on a room temperature substrate~\cite{meservey_properties_1971}. There was no intended oxidation in our process, but as there was no protective layer, some oxidation of grain boundaries is possible.

\begin{figure*}[h!t]
    \includegraphics[width=\linewidth]{Tc_Rscold_dose_abadded.png}
    \caption{The relationships between the ion fluence, $R_{\square, \mathrm{RT}}$, and $T_c$. The material label are the nominal thicknesses of the sputtered films and different colors are used for different materials. The size of the markers illustrates the magnitude of fluence (for argon and gallium independently). The films treated with gallium are shown with triangle markers and the ones treated with argon are shown with circles. Films that were TEM imaged are labeled. Film that had a SiO$_\mathrm{2}$ protective layer is labeled (other films with a protective layer had Al$_\mathrm{2}$O$_\mathrm{3}$). The empty markers (also labeled "NP") mean that those wafers did not have a protective layer. (a) $R_{\square, \mathrm{RT}}$ as a function of fluence for $\mathrm{Ga^+}$-treated M-series of MoSi. (b) $T_c$ as a function of argon or gallium ion fluence for M- and S-series of MoSi. (c) $T_c$ as a function of $R_{\square, 10~\mathrm{K}}$. Examples of parameter ranges with alternative techniques, granular aluminium (grAl)~\cite{grunhaupt_granular_2019-1} and ALD-grown NbN~\cite{linzen_structural_2017}, are shown with dotted green and dashed blue lines, respectively.}
    \label{Rs_Tc_dose_figure}
\end{figure*}

Even small irradiation fluence in niobium seems to decrease $T_c$, which makes the material less promising as a disordered superconductor. On the other hand, lithography-defined irradiation may allow Nb structures of different $T_c$ in the same layer. A significantly smaller decrease of $T_c$, 0.16~K, has been observed in Nb films irradiated with protons~\cite{tanatar_anisotropic_2022}, which was in agreement with theoretical calculations~\cite{zarea_effects_2022}. However, the proton-irradiated Nb films had an order of magnitude smaller resistivity than films Nb3--4.

One should also note that the protective layer of films Nb2 and Nb4 were fabricated after the deposition of the films as a separate step. Hence we cannot outrule in any of our Nb films the role of growth of native oxides, such as Nb$_2$O$_5$ that is known to have paramagnetic vacancies~\cite{altoe_localization_2022,proslier_evidence_2011}. Experiments on Nb have also reported a significant amount of quasiparticle states in the superconducting gap, which makes the material unsuitable for electronic cooling~\cite{nevala_sub-micron_2012} and is associated with an anomalously high heat conductivity of the material~\cite{feshchenko_thermal_2017}. Our results may contribute to these observations: If superconductivity in niobium is very sensitive to disorder, any defects in the material might locally suppress the superconductivity and introduce quasiparticle states in the material through the proximity effect. 

Figure~\ref{Rs_Tc_dose_figure} illustrates our results on $R_{\square}$ and $T_c$. Figure~\ref{Rs_Tc_dose_figure}(a) shows $R_{\square,\mathrm{RT}}$ of M-series MoSi wafers as a function of $\mathrm{Ga^+}$ ion fluence. As expected, $R_{\square, \mathrm{RT}}$ increases as a function of fluence, but there is significant spread between several films with the same fluence. It is probable that the lowest resistance path in these films is through the Mo grains seen in Fig.~\ref{TEM_MoSi}.

Figure~\ref{Rs_Tc_dose_figure}(b) shows $T_c$ of M- and S-series MoSi as a function of ion fluence. The values of $T_c$ are between $4~\mathrm{K}$ and $6~\mathrm{K}$, and the highest values are obtained for the most disordered Mo-poor films. Films irradiated with argon have higher $T_c$ than those irradiated with gallium, but the present amount of data does not allow a definite conclusion. Our results indicate, though, that the potential doping of MoSi with $\mathrm{Ga^+}$ does not explain the increase of $T_c$, which is supported also by the small fluences, e.g., 1 \AA$^{-2}$ for our most promising MoSi film, labeled S2.

Figure~\ref{Rs_Tc_dose_figure}(b) shows one film that had SiO$_\mathrm{2}$ as a protective dielectric (see Table~\ref{TEM_MoSi_NbN_table}, wafer M1). All other films in Fig.~\ref{Rs_Tc_dose_figure} have Al$_\mathrm{2}$O$_\mathrm{3}$ or no protective layer. However, several films similar to M1, protected with SiO$_\mathrm{2}$ and irradiated with the same fluence had reproducible values of
$R_{\square, \mathrm{RT}}$ between 98--120$~\mathrm{\Omega}$. The $T_c$ of wafer M1 is also the same as the highest $T_c$ measured from gallium-treated M-series films with Al$_\mathrm{2}$O$_\mathrm{3}$ layer. These results indicate that both materials, Al$_\mathrm{2}$O$_\mathrm{3}$ and SiO$_\mathrm{2}$, are suitable for the protective layer. Both layers can also be removed by selective etching.

Finally, Fig.~\ref{Rs_Tc_dose_figure}(c) shows the relationship between $T_c$ and $R_{\square, 10~\mathrm{K}}$ of MoSi, NbN, Nb, and Al wafers. We consider it as an application-oriented map into disordered superconductors, from which one could pick suitable materials for each purpose. Since ion irradiation is in principle suitable for increasing disorder of any superconducting thin film, future research on other materials may help to fill the empty parts of the map, e.g., in the regime $R_\square> 200$~$\Omega$. As examples of alternative techniques, tuning the oxidation of granular aluminium provides a wide range of $R_\square$ at $T_c\approx 2$~K~\cite{grunhaupt_granular_2019-1}, while reducing the thickness of ALD-grown NbN can provide values from about $T_c\approx 14$~K and $R_\square \approx 60$~$\Omega$ to $T_c\approx 5$~K and $R_\square \approx 5$~k$\Omega$~\cite{linzen_structural_2017}. These value ranges of the alternative techniques are illustrated in Fig.~\ref{Rs_Tc_dose_figure}(c) by the green and blue lines, respectively. Other interesting materials include, e.g., ALD-grown NbTiN~\cite{burdastyh_superconducting_2020}.

\section{Conclusions and outlook}

We present a wafer-scale method for the amorphization of superconducting thin films by using broad beam ion irradiation. Our treatment can be used independently of the material and its fabrication method. It would also be possible to obtain different levels of disorder in different areas of the same thin film by using lithography and, e.g., a thicker protective layer to block ion irradiation.

We use both gallium and argon ions and have studied various single-element (Al, V, Nb) and compound materials (NbN, TiN, MoSi), with the largest effort on MoSi films. Both ions increased the disorder and sheet resistance of the films, but in some cases, argon yielded gas pockets that may be detrimental for applications. As a result of disorder, the critical temperature increased in MoSi and Al, but decreased in TiN, NbN, and Nb. The critical temperature did not have significant dependence on the choice of ions, which indicates that the effect is not dominated by doping.

Both ions were used to produce a MoSi film that is amorphous with some short-range order. Our results show that relatively high $T_c$ can be achieved also with Mo-poor MoSi films (35\% Mo, 65\% Si), which was not expected based on literature~\cite{bosworth_amorphous_2015,banerjee_characterisation_2017}. Our Mo-rich MoSi films had the tendency of forming a two-layer structure of polycrystalline Mo and amorphous MoSi. Further research on the effect of stoichiometry would benefit from better mixing of the material before ion irradiation, which could be achieved, e.g., through co-sputtering from Mo and Si targets or sputtering from a MoSi compound target. Such methods may also improve the uncertainty and reproducibility of the stoichiometry for nanoscale films.

The ion irradiation method enables tuning of $R_{\square}$ and $T_c$ of superconducting thin films, potentially expanding the range of promising materials for different devices in quantum technology. In this work, we have demonstrated amorphous and uniform MoSi that is promising for superconducting nanowire single-photon detectors.


% Put \label in argument of \section for cross-referencing
%\section{\label{}}
%\subsection{}
%\subsubsection{}

% If in two-column mode, this environment will change to single-column
% format so that long equations can be displayed. Use
% sparingly.
%\begin{widetext}
% put long equation here
%\end{widetext}

% figures should be put into the text as floats.
% Use the graphics or graphicx packages (distributed with LaTeX2e)
% and the \includegraphics macro defined in those packages.
% See the LaTeX Graphics Companion by Michel Goosens, Sebastian Rahtz,
% and Frank Mittelbach for instance.
%
% Here is an example of the general form of a figure:
% Fill in the caption in the braces of the \caption{} command. Put the label
% that you will use with \ref{} command in the braces of the \label{} command.
% Use the figure* environment if the figure should span across the
% entire page. There is no need to do explicit centering.

% \begin{figure}
% \includegraphics{}%
% \caption{\label{}}
% \end{figure}

% Surround figure environment with turnpage environment for landscape
% figure
% \begin{turnpage}
% \begin{figure}
% \includegraphics{}%
% \caption{\label{}}
% \end{figure}
% \end{turnpage}

% tables should appear as floats within the text
%
% Here is an example of the general form of a table:
% Fill in the caption in the braces of the \caption{} command. Put the label
% that you will use with \ref{} command in the braces of the \label{} command.
% Insert the column specifiers (l, r, c, d, etc.) in the empty braces of the
% \begin{tabular}{} command.
% The ruledtabular enviroment adds doubled rules to table and sets a
% reasonable default table settings.
% Use the table* environment to get a full-width table in two-column
% Add \usepackage{longtable} and the longtable (or longtable*}
% environment for nicely formatted long tables. Or use the the [H]
% placement option to break a long table (with less control than 
% in longtable).
% \begin{table}%[H] add [H] placement to break table across pages
% \caption{\label{}}
% \begin{ruledtabular}
% \begin{tabular}{}
% Lines of table here ending with \\
% \end{tabular}
% \end{ruledtabular}
% \end{table}

% Surround table environment with turnpage environment for landscape
% table
% \begin{turnpage}
% \begin{table}
% \caption{\label{}}
% \begin{ruledtabular}
% \begin{tabular}{}
% \end{tabular}
% \end{ruledtabular}
% \end{table}
% \end{turnpage}

% Specify following sections are appendices. Use \appendix* if there
% only one appendix.
%\appendix
%\section{}

% If you have acknowledgments, this puts in the proper section head.
%\begin{acknowledgments}
% put your acknowledgments here.
%\end{acknowledgments}

%\appendix


\begin{acknowledgments}
We thank J. A. Sauls for useful discussions. For funding of our research project, we acknowledge the European Union’s Horizon 2020 Research and Innovation Programme under the Grant Agreement Nos.~862660/Quantum e-leaps, 899558/aCryComm, and 766853/EFINED. This project has also received funding from Business Finland through Quantum Technologies Industrial (QuTI) project No.~128291 and from Academy of Finland through Grant Nos.~310909 and 350220. This work was performed as part of the Academy of Finland Centre of Excellence program (project 336817 and 336819).  We also acknowledge funding from an internal strategic innovation project of VTT related to the development of quantum computing technologies. This research was supported by the Scientific Service Units of IST Austria through resources provided by Electron Microscopy Facility. J. Senior acknowledges funding from the European Union’s Horizon 2020 Research and Innovation Programme under the Marie Sk\l odowska-Curie Grant Agreement No. 754411.
\end{acknowledgments}



% Create the reference section using BibTeX:
%\bibliography{eleaps.bib}
%apsrev4-2.bst 2019-01-14 (MD) hand-edited version of apsrev4-1.bst
%Control: key (0)
%Control: author (72) initials jnrlst
%Control: editor formatted (1) identically to author
%Control: production of article title (-1) disabled
%Control: page (0) single
%Control: year (1) truncated
%Control: production of eprint (0) enabled
\begin{thebibliography}{36}%
\makeatletter
\providecommand \@ifxundefined [1]{%
 \@ifx{#1\undefined}
}%
\providecommand \@ifnum [1]{%
 \ifnum #1\expandafter \@firstoftwo
 \else \expandafter \@secondoftwo
 \fi
}%
\providecommand \@ifx [1]{%
 \ifx #1\expandafter \@firstoftwo
 \else \expandafter \@secondoftwo
 \fi
}%
\providecommand \natexlab [1]{#1}%
\providecommand \enquote  [1]{``#1''}%
\providecommand \bibnamefont  [1]{#1}%
\providecommand \bibfnamefont [1]{#1}%
\providecommand \citenamefont [1]{#1}%
\providecommand \href@noop [0]{\@secondoftwo}%
\providecommand \href [0]{\begingroup \@sanitize@url \@href}%
\providecommand \@href[1]{\@@startlink{#1}\@@href}%
\providecommand \@@href[1]{\endgroup#1\@@endlink}%
\providecommand \@sanitize@url [0]{\catcode `\\12\catcode `\$12\catcode
  `\&12\catcode `\#12\catcode `\^12\catcode `\_12\catcode `\%12\relax}%
\providecommand \@@startlink[1]{}%
\providecommand \@@endlink[0]{}%
\providecommand \url  [0]{\begingroup\@sanitize@url \@url }%
\providecommand \@url [1]{\endgroup\@href {#1}{\urlprefix }}%
\providecommand \urlprefix  [0]{URL }%
\providecommand \Eprint [0]{\href }%
\providecommand \doibase [0]{https://doi.org/}%
\providecommand \selectlanguage [0]{\@gobble}%
\providecommand \bibinfo  [0]{\@secondoftwo}%
\providecommand \bibfield  [0]{\@secondoftwo}%
\providecommand \translation [1]{[#1]}%
\providecommand \BibitemOpen [0]{}%
\providecommand \bibitemStop [0]{}%
\providecommand \bibitemNoStop [0]{.\EOS\space}%
\providecommand \EOS [0]{\spacefactor3000\relax}%
\providecommand \BibitemShut  [1]{\csname bibitem#1\endcsname}%
\let\auto@bib@innerbib\@empty
%</preamble>
\bibitem [{\citenamefont {Gol{'}tsman}\ \emph {et~al.}(2001)\citenamefont
  {Gol{'}tsman}, \citenamefont {Okunev}, \citenamefont {Chulkova},
  \citenamefont {Lipatov}, \citenamefont {Semenov}, \citenamefont {Smirnov},
  \citenamefont {Voronov}, \citenamefont {Dzardanov}, \citenamefont
  {Williams},\ and\ \citenamefont {Sobolewski}}]{goltsman_picosecond_2001}%
  \BibitemOpen
  \bibfield  {author} {\bibinfo {author} {\bibfnamefont {G.~N.}\ \bibnamefont
  {Gol{'}tsman}}, \bibinfo {author} {\bibfnamefont {O.}~\bibnamefont {Okunev}},
  \bibinfo {author} {\bibfnamefont {G.}~\bibnamefont {Chulkova}}, \bibinfo
  {author} {\bibfnamefont {A.}~\bibnamefont {Lipatov}}, \bibinfo {author}
  {\bibfnamefont {A.}~\bibnamefont {Semenov}}, \bibinfo {author} {\bibfnamefont
  {K.}~\bibnamefont {Smirnov}}, \bibinfo {author} {\bibfnamefont
  {B.}~\bibnamefont {Voronov}}, \bibinfo {author} {\bibfnamefont
  {A.}~\bibnamefont {Dzardanov}}, \bibinfo {author} {\bibfnamefont
  {C.}~\bibnamefont {Williams}},\ and\ \bibinfo {author} {\bibfnamefont
  {R.}~\bibnamefont {Sobolewski}},\ }\href {https://doi.org/10.1063/1.1388868}
  {\bibinfo {title} {Picosecond superconducting single-photon optical
  detector}},\ \href {https://doi.org/10.1063/1.1388868} {\bibfield  {journal}
  {\bibinfo  {journal} {Applied Physics Letters}\ }\textbf {\bibinfo {volume}
  {79}},\ \bibinfo {pages} {705} (\bibinfo {year} {2001})},\ \href
  {https://doi.org/10.1063/1.1388868} {10.1063/1.1388868}\BibitemShut {NoStop}%
\bibitem [{\citenamefont {Esmaeil~Zadeh}\ \emph {et~al.}(2021)\citenamefont
  {Esmaeil~Zadeh}, \citenamefont {Chang}, \citenamefont {Los}, \citenamefont
  {Gyger}, \citenamefont {Elshaari}, \citenamefont {Steinhauer}, \citenamefont
  {Dorenbos},\ and\ \citenamefont
  {Zwiller}}]{esmaeil_zadeh_superconducting_2021}%
  \BibitemOpen
  \bibfield  {author} {\bibinfo {author} {\bibfnamefont {I.}~\bibnamefont
  {Esmaeil~Zadeh}}, \bibinfo {author} {\bibfnamefont {J.}~\bibnamefont
  {Chang}}, \bibinfo {author} {\bibfnamefont {J.~W.~N.}\ \bibnamefont {Los}},
  \bibinfo {author} {\bibfnamefont {S.}~\bibnamefont {Gyger}}, \bibinfo
  {author} {\bibfnamefont {A.~W.}\ \bibnamefont {Elshaari}}, \bibinfo {author}
  {\bibfnamefont {S.}~\bibnamefont {Steinhauer}}, \bibinfo {author}
  {\bibfnamefont {S.~N.}\ \bibnamefont {Dorenbos}},\ and\ \bibinfo {author}
  {\bibfnamefont {V.}~\bibnamefont {Zwiller}},\ }\href
  {https://doi.org/10.1063/5.0045990} {\bibinfo {title} {Superconducting
  nanowire single-photon detectors: {A} perspective on evolution,
  state-of-the-art, future developments, and applications}},\ \href
  {https://doi.org/10.1063/5.0045990} {\bibfield  {journal} {\bibinfo
  {journal} {Applied Physics Letters}\ }\textbf {\bibinfo {volume} {118}},\
  \bibinfo {pages} {190502} (\bibinfo {year} {2021})},\ \href
  {https://doi.org/10.1063/5.0045990} {10.1063/5.0045990}\BibitemShut {NoStop}%
\bibitem [{\citenamefont {Astafiev}\ \emph {et~al.}(2012)\citenamefont
  {Astafiev}, \citenamefont {Ioffe}, \citenamefont {Kafanov}, \citenamefont
  {Pashkin}, \citenamefont {Arutyunov}, \citenamefont {Shahar}, \citenamefont
  {Cohen},\ and\ \citenamefont {Tsai}}]{astafiev_coherent_2012}%
  \BibitemOpen
  \bibfield  {author} {\bibinfo {author} {\bibfnamefont {O.~V.}\ \bibnamefont
  {Astafiev}}, \bibinfo {author} {\bibfnamefont {L.~B.}\ \bibnamefont {Ioffe}},
  \bibinfo {author} {\bibfnamefont {S.}~\bibnamefont {Kafanov}}, \bibinfo
  {author} {\bibfnamefont {Y.~A.}\ \bibnamefont {Pashkin}}, \bibinfo {author}
  {\bibfnamefont {K.~Y.}\ \bibnamefont {Arutyunov}}, \bibinfo {author}
  {\bibfnamefont {D.}~\bibnamefont {Shahar}}, \bibinfo {author} {\bibfnamefont
  {O.}~\bibnamefont {Cohen}},\ and\ \bibinfo {author} {\bibfnamefont {J.~S.}\
  \bibnamefont {Tsai}},\ }\href {https://doi.org/10.1038/nature10930} {\bibinfo
  {title} {Coherent quantum phase slip}},\ \href
  {https://doi.org/10.1038/nature10930} {\bibfield  {journal} {\bibinfo
  {journal} {Nature}\ }\textbf {\bibinfo {volume} {484}},\ \bibinfo {pages}
  {355} (\bibinfo {year} {2012})},\ \href {https://doi.org/10.1038/nature10930}
  {10.1038/nature10930}\BibitemShut {NoStop}%
\bibitem [{\citenamefont {Lehtinen}\ \emph {et~al.}(2012)\citenamefont
  {Lehtinen}, \citenamefont {Zakharov},\ and\ \citenamefont
  {Arutyunov}}]{lehtinen_coulomb_2012}%
  \BibitemOpen
  \bibfield  {author} {\bibinfo {author} {\bibfnamefont {J.~S.}\ \bibnamefont
  {Lehtinen}}, \bibinfo {author} {\bibfnamefont {K.}~\bibnamefont {Zakharov}},\
  and\ \bibinfo {author} {\bibfnamefont {K.~Y.}\ \bibnamefont {Arutyunov}},\
  }\href {https://doi.org/10.1103/PhysRevLett.109.187001} {\bibinfo {title}
  {Coulomb {Blockade} and {Bloch} {Oscillations} in {Superconducting} {Ti}
  {Nanowires}}},\ \href {https://doi.org/10.1103/PhysRevLett.109.187001}
  {\bibfield  {journal} {\bibinfo  {journal} {Physical Review Letters}\
  }\textbf {\bibinfo {volume} {109}},\ \bibinfo {pages} {187001} (\bibinfo
  {year} {2012})},\ \href {https://doi.org/10.1103/PhysRevLett.109.187001}
  {10.1103/PhysRevLett.109.187001}\BibitemShut {NoStop}%
\bibitem [{\citenamefont {Shaikhaidarov}\ \emph {et~al.}(2022)\citenamefont
  {Shaikhaidarov}, \citenamefont {Kim}, \citenamefont {Dunstan}, \citenamefont
  {Antonov}, \citenamefont {Linzen}, \citenamefont {Ziegler}, \citenamefont
  {Golubev}, \citenamefont {Antonov}, \citenamefont {Il{'}ichev},\ and\
  \citenamefont {Astafiev}}]{shaikhaidarov_quantized_2022}%
  \BibitemOpen
  \bibfield  {author} {\bibinfo {author} {\bibfnamefont {R.~S.}\ \bibnamefont
  {Shaikhaidarov}}, \bibinfo {author} {\bibfnamefont {K.~H.}\ \bibnamefont
  {Kim}}, \bibinfo {author} {\bibfnamefont {J.~W.}\ \bibnamefont {Dunstan}},
  \bibinfo {author} {\bibfnamefont {I.~V.}\ \bibnamefont {Antonov}}, \bibinfo
  {author} {\bibfnamefont {S.}~\bibnamefont {Linzen}}, \bibinfo {author}
  {\bibfnamefont {M.}~\bibnamefont {Ziegler}}, \bibinfo {author} {\bibfnamefont
  {D.~S.}\ \bibnamefont {Golubev}}, \bibinfo {author} {\bibfnamefont {V.~N.}\
  \bibnamefont {Antonov}}, \bibinfo {author} {\bibfnamefont {E.~V.}\
  \bibnamefont {Il{'}ichev}},\ and\ \bibinfo {author} {\bibfnamefont {O.~V.}\
  \bibnamefont {Astafiev}},\ }\href
  {https://doi.org/10.1038/s41586-022-04947-z} {\bibinfo {title} {Quantized
  current steps due to the a.c. coherent quantum phase-slip effect}},\ \href
  {https://doi.org/10.1038/s41586-022-04947-z} {\bibfield  {journal} {\bibinfo
  {journal} {Nature}\ }\textbf {\bibinfo {volume} {608}},\ \bibinfo {pages}
  {45} (\bibinfo {year} {2022})},\ \href
  {https://doi.org/10.1038/s41586-022-04947-z}
  {10.1038/s41586-022-04947-z}\BibitemShut {NoStop}%
\bibitem [{\citenamefont {Zhao}\ \emph {et~al.}(2020)\citenamefont {Zhao},
  \citenamefont {Park}, \citenamefont {Zhao}, \citenamefont {Bal},
  \citenamefont {McRae}, \citenamefont {Long},\ and\ \citenamefont
  {Pappas}}]{zhao_merged-element_2020}%
  \BibitemOpen
  \bibfield  {author} {\bibinfo {author} {\bibfnamefont {R.}~\bibnamefont
  {Zhao}}, \bibinfo {author} {\bibfnamefont {S.}~\bibnamefont {Park}}, \bibinfo
  {author} {\bibfnamefont {T.}~\bibnamefont {Zhao}}, \bibinfo {author}
  {\bibfnamefont {M.}~\bibnamefont {Bal}}, \bibinfo {author} {\bibfnamefont
  {C.~R.~H.}\ \bibnamefont {McRae}}, \bibinfo {author} {\bibfnamefont
  {J.}~\bibnamefont {Long}},\ and\ \bibinfo {author} {\bibfnamefont {D.~P.}\
  \bibnamefont {Pappas}},\ }\href
  {https://doi.org/10.1103/PhysRevApplied.14.064006} {\bibinfo {title}
  {Merged-{Element} {Transmon}}},\ \href
  {https://doi.org/10.1103/PhysRevApplied.14.064006} {\bibfield  {journal}
  {\bibinfo  {journal} {Physical Review Applied}\ }\textbf {\bibinfo {volume}
  {14}},\ \bibinfo {pages} {064006} (\bibinfo {year} {2020})},\ \href
  {https://doi.org/10.1103/PhysRevApplied.14.064006}
  {10.1103/PhysRevApplied.14.064006}\BibitemShut {NoStop}%
\bibitem [{\citenamefont {Zou}\ \emph {et~al.}(2017)\citenamefont {Zou},
  \citenamefont {Chen}, \citenamefont {Zhang}, \citenamefont {Xiu},
  \citenamefont {Matsumura}, \citenamefont {Yang}, \citenamefont {Hong},\ and\
  \citenamefont {Zou}}]{zou_superconductivity_2017}%
  \BibitemOpen
  \bibfield  {author} {\bibinfo {author} {\bibfnamefont {Y.-C.}\ \bibnamefont
  {Zou}}, \bibinfo {author} {\bibfnamefont {Z.-G.}\ \bibnamefont {Chen}},
  \bibinfo {author} {\bibfnamefont {E.}~\bibnamefont {Zhang}}, \bibinfo
  {author} {\bibfnamefont {F.}~\bibnamefont {Xiu}}, \bibinfo {author}
  {\bibfnamefont {S.}~\bibnamefont {Matsumura}}, \bibinfo {author}
  {\bibfnamefont {L.}~\bibnamefont {Yang}}, \bibinfo {author} {\bibfnamefont
  {M.}~\bibnamefont {Hong}},\ and\ \bibinfo {author} {\bibfnamefont
  {J.}~\bibnamefont {Zou}},\ }\href {https://doi.org/10.1039/C7NR06617A}
  {\bibinfo {title} {Superconductivity and magnetotransport of
  single-crystalline {NbSe2} nanoplates grown by chemical vapour deposition}},\
  \href {https://doi.org/10.1039/C7NR06617A} {\bibfield  {journal} {\bibinfo
  {journal} {Nanoscale}\ }\textbf {\bibinfo {volume} {9}},\ \bibinfo {pages}
  {16591} (\bibinfo {year} {2017})},\ \href
  {https://doi.org/10.1039/C7NR06617A} {10.1039/C7NR06617A}\BibitemShut
  {NoStop}%
\bibitem [{\citenamefont {Cao}\ \emph {et~al.}(2018)\citenamefont {Cao},
  \citenamefont {Fatemi}, \citenamefont {Fang}, \citenamefont {Watanabe},
  \citenamefont {Taniguchi}, \citenamefont {Kaxiras},\ and\ \citenamefont
  {Jarillo-Herrero}}]{cao_unconventional_2018}%
  \BibitemOpen
  \bibfield  {author} {\bibinfo {author} {\bibfnamefont {Y.}~\bibnamefont
  {Cao}}, \bibinfo {author} {\bibfnamefont {V.}~\bibnamefont {Fatemi}},
  \bibinfo {author} {\bibfnamefont {S.}~\bibnamefont {Fang}}, \bibinfo {author}
  {\bibfnamefont {K.}~\bibnamefont {Watanabe}}, \bibinfo {author}
  {\bibfnamefont {T.}~\bibnamefont {Taniguchi}}, \bibinfo {author}
  {\bibfnamefont {E.}~\bibnamefont {Kaxiras}},\ and\ \bibinfo {author}
  {\bibfnamefont {P.}~\bibnamefont {Jarillo-Herrero}},\ }\href
  {https://doi.org/10.1038/nature26160} {\bibinfo {title} {Unconventional
  superconductivity in magic-angle graphene superlattices}},\ \href
  {https://doi.org/10.1038/nature26160} {\bibfield  {journal} {\bibinfo
  {journal} {Nature}\ }\textbf {\bibinfo {volume} {556}},\ \bibinfo {pages}
  {43} (\bibinfo {year} {2018})},\ \href {https://doi.org/10.1038/nature26160}
  {10.1038/nature26160}\BibitemShut {NoStop}%
\bibitem [{\citenamefont {Maleeva}\ \emph {et~al.}(2018)\citenamefont
  {Maleeva}, \citenamefont {Gr\"unhaupt}, \citenamefont {Klein}, \citenamefont
  {Levy-Bertrand}, \citenamefont {Dupre}, \citenamefont {Calvo}, \citenamefont
  {Valenti}, \citenamefont {Winkel}, \citenamefont {Friedrich}, \citenamefont
  {Wernsdorfer}, \citenamefont {Ustinov}, \citenamefont {Rotzinger},
  \citenamefont {Monfardini}, \citenamefont {Fistul},\ and\ \citenamefont
  {Pop}}]{maleeva_circuit_2018}%
  \BibitemOpen
  \bibfield  {author} {\bibinfo {author} {\bibfnamefont {N.}~\bibnamefont
  {Maleeva}}, \bibinfo {author} {\bibfnamefont {L.}~\bibnamefont {Gr\"unhaupt}},
  \bibinfo {author} {\bibfnamefont {T.}~\bibnamefont {Klein}}, \bibinfo
  {author} {\bibfnamefont {F.}~\bibnamefont {Levy-Bertrand}}, \bibinfo {author}
  {\bibfnamefont {O.}~\bibnamefont {Dupre}}, \bibinfo {author} {\bibfnamefont
  {M.}~\bibnamefont {Calvo}}, \bibinfo {author} {\bibfnamefont
  {F.}~\bibnamefont {Valenti}}, \bibinfo {author} {\bibfnamefont
  {P.}~\bibnamefont {Winkel}}, \bibinfo {author} {\bibfnamefont
  {F.}~\bibnamefont {Friedrich}}, \bibinfo {author} {\bibfnamefont
  {W.}~\bibnamefont {Wernsdorfer}}, \bibinfo {author} {\bibfnamefont {A.~V.}\
  \bibnamefont {Ustinov}}, \bibinfo {author} {\bibfnamefont {H.}~\bibnamefont
  {Rotzinger}}, \bibinfo {author} {\bibfnamefont {A.}~\bibnamefont
  {Monfardini}}, \bibinfo {author} {\bibfnamefont {M.~V.}\ \bibnamefont
  {Fistul}},\ and\ \bibinfo {author} {\bibfnamefont {I.~M.}\ \bibnamefont
  {Pop}},\ }\href {https://doi.org/10.1038/s41467-018-06386-9} {\bibinfo
  {title} {Circuit quantum electrodynamics of granular aluminum resonators}},\
  \href {https://doi.org/10.1038/s41467-018-06386-9} {\bibfield  {journal}
  {\bibinfo  {journal} {Nature Communications}\ }\textbf {\bibinfo {volume}
  {9}},\ \bibinfo {pages} {3889} (\bibinfo {year} {2018})},\ \href
  {https://doi.org/10.1038/s41467-018-06386-9}
  {10.1038/s41467-018-06386-9}\BibitemShut {NoStop}%
\bibitem [{\citenamefont {Gr\"unhaupt}\ \emph {et~al.}(2019)\citenamefont
  {Gr\"unhaupt}, \citenamefont {Spiecker}, \citenamefont {Gusenkova},
  \citenamefont {Maleeva}, \citenamefont {Skacel}, \citenamefont {Takmakov},
  \citenamefont {Valenti}, \citenamefont {Winkel}, \citenamefont {Rotzinger},
  \citenamefont {Wernsdorfer}, \citenamefont {Ustinov},\ and\ \citenamefont
  {Pop}}]{grunhaupt_granular_2019}%
  \BibitemOpen
  \bibfield  {author} {\bibinfo {author} {\bibfnamefont {L.}~\bibnamefont
  {Gr\"unhaupt}}, \bibinfo {author} {\bibfnamefont {M.}~\bibnamefont
  {Spiecker}}, \bibinfo {author} {\bibfnamefont {D.}~\bibnamefont {Gusenkova}},
  \bibinfo {author} {\bibfnamefont {N.}~\bibnamefont {Maleeva}}, \bibinfo
  {author} {\bibfnamefont {S.~T.}\ \bibnamefont {Skacel}}, \bibinfo {author}
  {\bibfnamefont {I.}~\bibnamefont {Takmakov}}, \bibinfo {author}
  {\bibfnamefont {F.}~\bibnamefont {Valenti}}, \bibinfo {author} {\bibfnamefont
  {P.}~\bibnamefont {Winkel}}, \bibinfo {author} {\bibfnamefont
  {H.}~\bibnamefont {Rotzinger}}, \bibinfo {author} {\bibfnamefont
  {W.}~\bibnamefont {Wernsdorfer}}, \bibinfo {author} {\bibfnamefont {A.~V.}\
  \bibnamefont {Ustinov}},\ and\ \bibinfo {author} {\bibfnamefont {I.~M.}\
  \bibnamefont {Pop}},\ }\href {https://doi.org/10.1038/s41563-019-0350-3}
  {\bibinfo {title} {Granular aluminium as a superconducting material for
  high-impedance quantum circuits}},\ \href
  {https://doi.org/10.1038/s41563-019-0350-3} {\bibfield  {journal} {\bibinfo
  {journal} {Nature Materials}\ }\textbf {\bibinfo {volume} {18}},\ \bibinfo
  {pages} {816} (\bibinfo {year} {2019})},\ \href
  {https://doi.org/10.1038/s41563-019-0350-3}
  {10.1038/s41563-019-0350-3}\BibitemShut {NoStop}%
\bibitem [{\citenamefont {Jaeger}\ \emph {et~al.}(1986)\citenamefont {Jaeger},
  \citenamefont {Haviland}, \citenamefont {Goldman},\ and\ \citenamefont
  {Orr}}]{jaeger_threshold_1986}%
  \BibitemOpen
  \bibfield  {author} {\bibinfo {author} {\bibfnamefont {H.~M.}\ \bibnamefont
  {Jaeger}}, \bibinfo {author} {\bibfnamefont {D.~B.}\ \bibnamefont
  {Haviland}}, \bibinfo {author} {\bibfnamefont {A.~M.}\ \bibnamefont
  {Goldman}},\ and\ \bibinfo {author} {\bibfnamefont {B.~G.}\ \bibnamefont
  {Orr}},\ }\href {https://doi.org/10.1103/PhysRevB.34.4920} {\bibinfo {title}
  {Threshold for superconductivity in ultrathin amorphous gallium films}},\
  \href {https://doi.org/10.1103/PhysRevB.34.4920} {\bibfield  {journal}
  {\bibinfo  {journal} {Physical Review B}\ }\textbf {\bibinfo {volume} {34}},\
  \bibinfo {pages} {4920} (\bibinfo {year} {1986})},\ \href
  {https://doi.org/10.1103/PhysRevB.34.4920}
  {10.1103/PhysRevB.34.4920}\BibitemShut {NoStop}%
\bibitem [{\citenamefont {Sac\'ep\'e}\ \emph {et~al.}(2011)\citenamefont
  {Sac\'ep\'e}, \citenamefont {Dubouchet}, \citenamefont {Chapelier},
  \citenamefont {Sanquer}, \citenamefont {Ovadia}, \citenamefont {Shahar},
  \citenamefont {Feigel{'}man},\ and\ \citenamefont
  {Ioffe}}]{sacepe_localization_2011}%
  \BibitemOpen
  \bibfield  {author} {\bibinfo {author} {\bibfnamefont {B.}~\bibnamefont
  {Sac\'ep\'e}}, \bibinfo {author} {\bibfnamefont {T.}~\bibnamefont {Dubouchet}},
  \bibinfo {author} {\bibfnamefont {C.}~\bibnamefont {Chapelier}}, \bibinfo
  {author} {\bibfnamefont {M.}~\bibnamefont {Sanquer}}, \bibinfo {author}
  {\bibfnamefont {M.}~\bibnamefont {Ovadia}}, \bibinfo {author} {\bibfnamefont
  {D.}~\bibnamefont {Shahar}}, \bibinfo {author} {\bibfnamefont
  {M.}~\bibnamefont {Feigel{'}man}},\ and\ \bibinfo {author} {\bibfnamefont
  {L.}~\bibnamefont {Ioffe}},\ }\href {https://doi.org/10.1038/nphys1892}
  {\bibinfo {title} {Localization of preformed {Cooper} pairs in disordered
  superconductors}},\ \href {https://doi.org/10.1038/nphys1892} {\bibfield
  {journal} {\bibinfo  {journal} {Nature Physics}\ }\textbf {\bibinfo {volume}
  {7}},\ \bibinfo {pages} {239} (\bibinfo {year} {2011})},\ \href
  {https://doi.org/10.1038/nphys1892} {10.1038/nphys1892}\BibitemShut {NoStop}%
\bibitem [{\citenamefont {Buckel}\ and\ \citenamefont
  {Hilsch}(1954)}]{buckel_einflus_1954}%
  \BibitemOpen
  \bibfield  {author} {\bibinfo {author} {\bibfnamefont {W.}~\bibnamefont
  {Buckel}}\ and\ \bibinfo {author} {\bibfnamefont {R.}~\bibnamefont
  {Hilsch}},\ }\href {https://doi.org/10.1007/BF01337903} {\bibinfo {title}
  {Einflu{\ss} der {Kondensation} bei tiefen {Temperaturen} auf den elektrischen
  {Widerstand} und die {Supraleitung} f\"ur verschiedene {Metalle}}},\ \href
  {https://doi.org/10.1007/BF01337903} {\bibfield  {journal} {\bibinfo
  {journal} {Zeitschrift f\"ur Physik}\ }\textbf {\bibinfo {volume} {138}},\
  \bibinfo {pages} {109} (\bibinfo {year} {1954})},\ \href
  {https://doi.org/10.1007/BF01337903} {10.1007/BF01337903}\BibitemShut
  {NoStop}%
\bibitem [{\citenamefont {Kammerer}\ and\ \citenamefont
  {Strongin}(1965)}]{kammerer_superconductivity_1965}%
  \BibitemOpen
  \bibfield  {author} {\bibinfo {author} {\bibfnamefont {O.~F.}\ \bibnamefont
  {Kammerer}}\ and\ \bibinfo {author} {\bibfnamefont {M.}~\bibnamefont
  {Strongin}},\ }\href {https://doi.org/10.1016/0031-9163(65)90496-8} {\bibinfo
  {title} {Superconductivity in tungsten films}},\ \href
  {https://doi.org/10.1016/0031-9163(65)90496-8} {\bibfield  {journal}
  {\bibinfo  {journal} {Physics Letters}\ }\textbf {\bibinfo {volume} {17}},\
  \bibinfo {pages} {224} (\bibinfo {year} {1965})},\ \href
  {https://doi.org/10.1016/0031-9163(65)90496-8}
  {10.1016/0031-9163(65)90496-8}\BibitemShut {NoStop}%
\bibitem [{\citenamefont {Abeles}\ \emph {et~al.}(1966)\citenamefont {Abeles},
  \citenamefont {Cohen},\ and\ \citenamefont
  {Cullen}}]{abeles_enhancement_1966}%
  \BibitemOpen
  \bibfield  {author} {\bibinfo {author} {\bibfnamefont {B.}~\bibnamefont
  {Abeles}}, \bibinfo {author} {\bibfnamefont {R.~W.}\ \bibnamefont {Cohen}},\
  and\ \bibinfo {author} {\bibfnamefont {G.~W.}\ \bibnamefont {Cullen}},\
  }\href {https://doi.org/10.1103/PhysRevLett.17.632} {\bibinfo {title}
  {Enhancement of {Superconductivity} in {Metal} {Films}}},\ \href
  {https://doi.org/10.1103/PhysRevLett.17.632} {\bibfield  {journal} {\bibinfo
  {journal} {Physical Review Letters}\ }\textbf {\bibinfo {volume} {17}},\
  \bibinfo {pages} {632} (\bibinfo {year} {1966})},\ \href
  {https://doi.org/10.1103/PhysRevLett.17.632}
  {10.1103/PhysRevLett.17.632}\BibitemShut {NoStop}%
\bibitem [{\citenamefont {Strongin}\ \emph {et~al.}(1967)\citenamefont
  {Strongin}, \citenamefont {Kammerer}, \citenamefont {Douglass},\ and\
  \citenamefont {Cohen}}]{strongin_effect_1967}%
  \BibitemOpen
  \bibfield  {author} {\bibinfo {author} {\bibfnamefont {M.}~\bibnamefont
  {Strongin}}, \bibinfo {author} {\bibfnamefont {O.~F.}\ \bibnamefont
  {Kammerer}}, \bibinfo {author} {\bibfnamefont {D.~H.}\ \bibnamefont
  {Douglass}},\ and\ \bibinfo {author} {\bibfnamefont {M.~H.}\ \bibnamefont
  {Cohen}},\ }\href {https://doi.org/10.1103/PhysRevLett.19.121} {\bibinfo
  {title} {Effect of {Dielectric} and {High}-{Resistivity} {Barriers} on the
  {Superconducting} {Transition} {Temperature} of {Thin} {Films}}},\ \href
  {https://doi.org/10.1103/PhysRevLett.19.121} {\bibfield  {journal} {\bibinfo
  {journal} {Physical Review Letters}\ }\textbf {\bibinfo {volume} {19}},\
  \bibinfo {pages} {121} (\bibinfo {year} {1967})},\ \href
  {https://doi.org/10.1103/PhysRevLett.19.121}
  {10.1103/PhysRevLett.19.121}\BibitemShut {NoStop}%
\bibitem [{\citenamefont {Pettit}\ and\ \citenamefont
  {Silcox}(1976)}]{pettit_film_1976}%
  \BibitemOpen
  \bibfield  {author} {\bibinfo {author} {\bibfnamefont {R.~B.}\ \bibnamefont
  {Pettit}}\ and\ \bibinfo {author} {\bibfnamefont {J.}~\bibnamefont
  {Silcox}},\ }\href {https://doi.org/10.1103/PhysRevB.13.2865} {\bibinfo
  {title} {Film structure and enhanced superconductivity in evaporated aluminum
  films}},\ \href {https://doi.org/10.1103/PhysRevB.13.2865} {\bibfield
  {journal} {\bibinfo  {journal} {Physical Review B}\ }\textbf {\bibinfo
  {volume} {13}},\ \bibinfo {pages} {2865} (\bibinfo {year} {1976})},\ \href
  {https://doi.org/10.1103/PhysRevB.13.2865}
  {10.1103/PhysRevB.13.2865}\BibitemShut {NoStop}%
\bibitem [{\citenamefont {Sweedler}\ \emph {et~al.}(1974)\citenamefont
  {Sweedler}, \citenamefont {Schweitzer},\ and\ \citenamefont
  {Webb}}]{sweedler_atomic_1974}%
  \BibitemOpen
  \bibfield  {author} {\bibinfo {author} {\bibfnamefont {A.~R.}\ \bibnamefont
  {Sweedler}}, \bibinfo {author} {\bibfnamefont {D.~G.}\ \bibnamefont
  {Schweitzer}},\ and\ \bibinfo {author} {\bibfnamefont {G.~W.}\ \bibnamefont
  {Webb}},\ }\href {https://doi.org/10.1103/PhysRevLett.33.168} {\bibinfo
  {title} {Atomic {Ordering} and {Superconductivity} in {High}-{Tc} {A}-15
  {Compounds}}},\ \href {https://doi.org/10.1103/PhysRevLett.33.168} {\bibfield
   {journal} {\bibinfo  {journal} {Physical Review Letters}\ }\textbf {\bibinfo
  {volume} {33}},\ \bibinfo {pages} {168} (\bibinfo {year} {1974})},\ \href
  {https://doi.org/10.1103/PhysRevLett.33.168}
  {10.1103/PhysRevLett.33.168}\BibitemShut {NoStop}%
\bibitem [{\citenamefont {Lita}\ \emph {et~al.}(2005)\citenamefont {Lita},
  \citenamefont {Rosenberg}, \citenamefont {Nam}, \citenamefont {Miller},
  \citenamefont {Balzar}, \citenamefont {Kaatz},\ and\ \citenamefont
  {Schwall}}]{lita_tuning_2005}%
  \BibitemOpen
  \bibfield  {author} {\bibinfo {author} {\bibfnamefont {A.~E.}\ \bibnamefont
  {Lita}}, \bibinfo {author} {\bibfnamefont {D.}~\bibnamefont {Rosenberg}},
  \bibinfo {author} {\bibfnamefont {S.}~\bibnamefont {Nam}}, \bibinfo {author}
  {\bibfnamefont {A.~J.}\ \bibnamefont {Miller}}, \bibinfo {author}
  {\bibfnamefont {D.}~\bibnamefont {Balzar}}, \bibinfo {author} {\bibfnamefont
  {L.~M.}\ \bibnamefont {Kaatz}},\ and\ \bibinfo {author} {\bibfnamefont
  {R.~E.}\ \bibnamefont {Schwall}},\ }\href
  {https://doi.org/10.1109/TASC.2005.849033} {\bibinfo {title} {Tuning of
  tungsten thin film superconducting transition temperature for fabrication of
  photon number resolving detectors}},\ \href
  {https://doi.org/10.1109/TASC.2005.849033} {\bibfield  {journal} {\bibinfo
  {journal} {IEEE Transactions on Applied Superconductivity}\ }\textbf
  {\bibinfo {volume} {15}},\ \bibinfo {pages} {3528} (\bibinfo {year}
  {2005})},\ \href {https://doi.org/10.1109/TASC.2005.849033}
  {10.1109/TASC.2005.849033}\BibitemShut {NoStop}%
\bibitem [{\citenamefont {Tanatar}\ \emph {et~al.}(2022)\citenamefont
  {Tanatar}, \citenamefont {Torsello}, \citenamefont {Joshi}, \citenamefont
  {Ghimire}, \citenamefont {Kopas}, \citenamefont {Marshall}, \citenamefont
  {Mutus}, \citenamefont {Ghigo}, \citenamefont {Zarea}, \citenamefont
  {Sauls},\ and\ \citenamefont {Prozorov}}]{tanatar_anisotropic_2022}%
  \BibitemOpen
  \bibfield  {author} {\bibinfo {author} {\bibfnamefont {M.~A.}\ \bibnamefont
  {Tanatar}}, \bibinfo {author} {\bibfnamefont {D.}~\bibnamefont {Torsello}},
  \bibinfo {author} {\bibfnamefont {K.~R.}\ \bibnamefont {Joshi}}, \bibinfo
  {author} {\bibfnamefont {S.}~\bibnamefont {Ghimire}}, \bibinfo {author}
  {\bibfnamefont {C.~J.}\ \bibnamefont {Kopas}}, \bibinfo {author}
  {\bibfnamefont {J.}~\bibnamefont {Marshall}}, \bibinfo {author}
  {\bibfnamefont {J.~Y.}\ \bibnamefont {Mutus}}, \bibinfo {author}
  {\bibfnamefont {G.}~\bibnamefont {Ghigo}}, \bibinfo {author} {\bibfnamefont
  {M.}~\bibnamefont {Zarea}}, \bibinfo {author} {\bibfnamefont {J.~A.}\
  \bibnamefont {Sauls}},\ and\ \bibinfo {author} {\bibfnamefont
  {R.}~\bibnamefont {Prozorov}},\ }\href
  {https://doi.org/10.1103/PhysRevB.106.224511} {\bibinfo {title} {Anisotropic
  superconductivity of niobium based on its response to nonmagnetic
  disorder}},\ \href {https://doi.org/10.1103/PhysRevB.106.224511} {\bibfield
  {journal} {\bibinfo  {journal} {Physical Review B}\ }\textbf {\bibinfo
  {volume} {106}},\ \bibinfo {pages} {224511} (\bibinfo {year} {2022})},\ \href
  {https://doi.org/10.1103/PhysRevB.106.224511}
  {10.1103/PhysRevB.106.224511}\BibitemShut {NoStop}%
\bibitem [{\citenamefont {Osofsky}\ \emph {et~al.}(2001)\citenamefont
  {Osofsky}, \citenamefont {Soulen}, \citenamefont {Claassen}, \citenamefont
  {Trotter}, \citenamefont {Kim},\ and\ \citenamefont
  {Horwitz}}]{osofsky_new_2001}%
  \BibitemOpen
  \bibfield  {author} {\bibinfo {author} {\bibfnamefont {M.~S.}\ \bibnamefont
  {Osofsky}}, \bibinfo {author} {\bibfnamefont {R.~J.}\ \bibnamefont {Soulen}},
  \bibinfo {author} {\bibfnamefont {J.~H.}\ \bibnamefont {Claassen}}, \bibinfo
  {author} {\bibfnamefont {G.}~\bibnamefont {Trotter}}, \bibinfo {author}
  {\bibfnamefont {H.}~\bibnamefont {Kim}},\ and\ \bibinfo {author}
  {\bibfnamefont {J.~S.}\ \bibnamefont {Horwitz}},\ }\href
  {https://doi.org/10.1103/PhysRevLett.87.197004} {\bibinfo {title} {New
  {Insight} into {Enhanced} {Superconductivity} in {Metals} near the
  {Metal}-{Insulator} {Transition}}},\ \href
  {https://doi.org/10.1103/PhysRevLett.87.197004} {\bibfield  {journal}
  {\bibinfo  {journal} {Physical Review Letters}\ }\textbf {\bibinfo {volume}
  {87}},\ \bibinfo {pages} {197004} (\bibinfo {year} {2001})},\ \href
  {https://doi.org/10.1103/PhysRevLett.87.197004}
  {10.1103/PhysRevLett.87.197004}\BibitemShut {NoStop}%
\bibitem [{\citenamefont {Bosworth}\ \emph {et~al.}(2015)\citenamefont
  {Bosworth}, \citenamefont {Sahonta}, \citenamefont {Hadfield},\ and\
  \citenamefont {Barber}}]{bosworth_amorphous_2015}%
  \BibitemOpen
  \bibfield  {author} {\bibinfo {author} {\bibfnamefont {D.}~\bibnamefont
  {Bosworth}}, \bibinfo {author} {\bibfnamefont {S.-L.}\ \bibnamefont
  {Sahonta}}, \bibinfo {author} {\bibfnamefont {R.~H.}\ \bibnamefont
  {Hadfield}},\ and\ \bibinfo {author} {\bibfnamefont {Z.~H.}\ \bibnamefont
  {Barber}},\ }\href {https://doi.org/10.1063/1.4928285} {\bibinfo {title}
  {Amorphous molybdenum silicon superconducting thin films}},\ \href
  {https://doi.org/10.1063/1.4928285} {\bibfield  {journal} {\bibinfo
  {journal} {AIP Advances}\ }\textbf {\bibinfo {volume} {5}},\ \bibinfo {pages}
  {087106} (\bibinfo {year} {2015})},\ \href
  {https://doi.org/10.1063/1.4928285} {10.1063/1.4928285}\BibitemShut {NoStop}%
\bibitem [{\citenamefont {Banerjee}\ \emph {et~al.}(2017)\citenamefont
  {Banerjee}, \citenamefont {Baker}, \citenamefont {Doye}, \citenamefont
  {Nord}, \citenamefont {Heath}, \citenamefont {Erotokritou}, \citenamefont
  {Bosworth}, \citenamefont {Barber}, \citenamefont {MacLaren},\ and\
  \citenamefont {Hadfield}}]{banerjee_characterisation_2017}%
  \BibitemOpen
  \bibfield  {author} {\bibinfo {author} {\bibfnamefont {A.}~\bibnamefont
  {Banerjee}}, \bibinfo {author} {\bibfnamefont {L.~J.}\ \bibnamefont {Baker}},
  \bibinfo {author} {\bibfnamefont {A.}~\bibnamefont {Doye}}, \bibinfo {author}
  {\bibfnamefont {M.}~\bibnamefont {Nord}}, \bibinfo {author} {\bibfnamefont
  {R.~M.}\ \bibnamefont {Heath}}, \bibinfo {author} {\bibfnamefont
  {K.}~\bibnamefont {Erotokritou}}, \bibinfo {author} {\bibfnamefont
  {D.}~\bibnamefont {Bosworth}}, \bibinfo {author} {\bibfnamefont {Z.~H.}\
  \bibnamefont {Barber}}, \bibinfo {author} {\bibfnamefont {I.}~\bibnamefont
  {MacLaren}},\ and\ \bibinfo {author} {\bibfnamefont {R.~H.}\ \bibnamefont
  {Hadfield}},\ }\href {https://doi.org/10.1088/1361-6668/aa76d8} {\bibinfo
  {title} {Characterisation of amorphous molybdenum silicide ({MoSi})
  superconducting thin films and nanowires}},\ \href
  {https://doi.org/10.1088/1361-6668/aa76d8} {\bibfield  {journal} {\bibinfo
  {journal} {Superconductor Science and Technology}\ }\textbf {\bibinfo
  {volume} {30}},\ \bibinfo {pages} {084010} (\bibinfo {year} {2017})},\ \href
  {https://doi.org/10.1088/1361-6668/aa76d8}
  {10.1088/1361-6668/aa76d8}\BibitemShut {NoStop}%
\bibitem [{\citenamefont {Zhang}\ \emph {et~al.}(2021)\citenamefont {Zhang},
  \citenamefont {Charaev}, \citenamefont {Liu}, \citenamefont {Zhou},
  \citenamefont {Zhu}, \citenamefont {Berggren},\ and\ \citenamefont
  {Schilling}}]{zhang_physical_2021}%
  \BibitemOpen
  \bibfield  {author} {\bibinfo {author} {\bibfnamefont {X.}~\bibnamefont
  {Zhang}}, \bibinfo {author} {\bibfnamefont {I.}~\bibnamefont {Charaev}},
  \bibinfo {author} {\bibfnamefont {H.}~\bibnamefont {Liu}}, \bibinfo {author}
  {\bibfnamefont {T.~X.}\ \bibnamefont {Zhou}}, \bibinfo {author}
  {\bibfnamefont {D.}~\bibnamefont {Zhu}}, \bibinfo {author} {\bibfnamefont
  {K.~K.}\ \bibnamefont {Berggren}},\ and\ \bibinfo {author} {\bibfnamefont
  {A.}~\bibnamefont {Schilling}},\ }\href
  {https://doi.org/10.1088/1361-6668/ac1524} {\bibinfo {title} {Physical
  properties of amorphous molybdenum silicide films for single-photon
  detectors}},\ \href {https://doi.org/10.1088/1361-6668/ac1524} {\bibfield
  {journal} {\bibinfo  {journal} {Superconductor Science and Technology}\
  }\textbf {\bibinfo {volume} {34}},\ \bibinfo {pages} {095003} (\bibinfo
  {year} {2021})},\ \href {https://doi.org/10.1088/1361-6668/ac1524}
  {10.1088/1361-6668/ac1524}\BibitemShut {NoStop}%
\bibitem [{\citenamefont {Tsaur}\ \emph {et~al.}(1979)\citenamefont {Tsaur},
  \citenamefont {Liau},\ and\ \citenamefont
  {Mayer}}]{tsaur_ionbeaminduced_1979}%
  \BibitemOpen
  \bibfield  {author} {\bibinfo {author} {\bibfnamefont {B.~Y.}\ \bibnamefont
  {Tsaur}}, \bibinfo {author} {\bibfnamefont {Z.~L.}\ \bibnamefont {Liau}},\
  and\ \bibinfo {author} {\bibfnamefont {J.~W.}\ \bibnamefont {Mayer}},\ }\href
  {https://doi.org/10.1063/1.90716} {\bibinfo {title} {Ion‐beam‐induced
  silicide formation}},\ \href {https://doi.org/10.1063/1.90716} {\bibfield
  {journal} {\bibinfo  {journal} {Applied Physics Letters}\ }\textbf {\bibinfo
  {volume} {34}},\ \bibinfo {pages} {168} (\bibinfo {year} {1979})},\ \href
  {https://doi.org/10.1063/1.90716} {10.1063/1.90716}\BibitemShut {NoStop}%
\bibitem [{\citenamefont {Lehtinen}\ \emph {et~al.}(2017)\citenamefont
  {Lehtinen}, \citenamefont {Kemppinen}, \citenamefont {Mykk\"anen},
  \citenamefont {Prunnila},\ and\ \citenamefont
  {Manninen}}]{lehtinen_superconducting_2017}%
  \BibitemOpen
  \bibfield  {author} {\bibinfo {author} {\bibfnamefont {J.~S.}\ \bibnamefont
  {Lehtinen}}, \bibinfo {author} {\bibfnamefont {A.}~\bibnamefont {Kemppinen}},
  \bibinfo {author} {\bibfnamefont {E.}~\bibnamefont {Mykk\"anen}}, \bibinfo
  {author} {\bibfnamefont {M.}~\bibnamefont {Prunnila}},\ and\ \bibinfo
  {author} {\bibfnamefont {A.~J.}\ \bibnamefont {Manninen}},\ }\href
  {https://doi.org/10.1088/1361-6668/aa954b} {\bibinfo {title} {Superconducting
  {MoSi} nanowires}},\ \href {https://doi.org/10.1088/1361-6668/aa954b}
  {\bibfield  {journal} {\bibinfo  {journal} {Superconductor Science and
  Technology}\ }\textbf {\bibinfo {volume} {31}},\ \bibinfo {pages} {015002}
  (\bibinfo {year} {2017})},\ \href {https://doi.org/10.1088/1361-6668/aa954b}
  {10.1088/1361-6668/aa954b}\BibitemShut {NoStop}%
\bibitem [{\citenamefont {Mykk\"anen}\ \emph {et~al.}(2020)\citenamefont
  {Mykk\"anen}, \citenamefont {Bera}, \citenamefont {Lehtinen}, \citenamefont
  {Ronzani}, \citenamefont {Kohop\"a\"a}, \citenamefont {H\"onigl-Decrinis},
  \citenamefont {Shaikhaidarov}, \citenamefont {de~Graaf}, \citenamefont
  {Govenius},\ and\ \citenamefont {Prunnila}}]{mykkanen_enhancement_2020}%
  \BibitemOpen
  \bibfield  {author} {\bibinfo {author} {\bibfnamefont {E.}~\bibnamefont
  {Mykk\"anen}}, \bibinfo {author} {\bibfnamefont {A.}~\bibnamefont {Bera}},
  \bibinfo {author} {\bibfnamefont {J.~S.}\ \bibnamefont {Lehtinen}}, \bibinfo
  {author} {\bibfnamefont {A.}~\bibnamefont {Ronzani}}, \bibinfo {author}
  {\bibfnamefont {K.}~\bibnamefont {Kohop\"a\"a}}, \bibinfo {author}
  {\bibfnamefont {T.}~\bibnamefont {H\"onigl-Decrinis}}, \bibinfo {author}
  {\bibfnamefont {R.}~\bibnamefont {Shaikhaidarov}}, \bibinfo {author}
  {\bibfnamefont {S.~E.}\ \bibnamefont {de~Graaf}}, \bibinfo {author}
  {\bibfnamefont {J.}~\bibnamefont {Govenius}},\ and\ \bibinfo {author}
  {\bibfnamefont {M.}~\bibnamefont {Prunnila}},\ }\href
  {https://doi.org/10.3390/nano10050950} {\bibinfo {title} {Enhancement of
  {Superconductivity} by {Amorphizing} {Molybdenum} {Silicide} {Films} {Using}
  a {Focused} {Ion} {Beam}}},\ \href {https://doi.org/10.3390/nano10050950}
  {\bibfield  {journal} {\bibinfo  {journal} {Nanomaterials}\ }\textbf
  {\bibinfo {volume} {10}},\ \bibinfo {pages} {950} (\bibinfo {year} {2020})},\
  \href {https://doi.org/10.3390/nano10050950}
  {10.3390/nano10050950}\BibitemShut {NoStop}%
\bibitem [{\citenamefont {Linzen}\ \emph {et~al.}(2017)\citenamefont {Linzen},
  \citenamefont {Ziegler}, \citenamefont {Astafiev}, \citenamefont {Schmelz},
  \citenamefont {H\"ubner}, \citenamefont {Diegel}, \citenamefont {Il{'}ichev},\
  and\ \citenamefont {Meyer}}]{linzen_structural_2017}%
  \BibitemOpen
  \bibfield  {author} {\bibinfo {author} {\bibfnamefont {S.}~\bibnamefont
  {Linzen}}, \bibinfo {author} {\bibfnamefont {M.}~\bibnamefont {Ziegler}},
  \bibinfo {author} {\bibfnamefont {O.~V.}\ \bibnamefont {Astafiev}}, \bibinfo
  {author} {\bibfnamefont {M.}~\bibnamefont {Schmelz}}, \bibinfo {author}
  {\bibfnamefont {U.}~\bibnamefont {H\"ubner}}, \bibinfo {author} {\bibfnamefont
  {M.}~\bibnamefont {Diegel}}, \bibinfo {author} {\bibfnamefont
  {E.}~\bibnamefont {Il{'}ichev}},\ and\ \bibinfo {author} {\bibfnamefont
  {H.-G.}\ \bibnamefont {Meyer}},\ }\href
  {https://doi.org/10.1088/1361-6668/aa572a} {\bibinfo {title} {Structural and
  electrical properties of ultrathin niobium nitride films grown by atomic
  layer deposition}},\ \href {https://doi.org/10.1088/1361-6668/aa572a}
  {\bibfield  {journal} {\bibinfo  {journal} {Superconductor Science and
  Technology}\ }\textbf {\bibinfo {volume} {30}},\ \bibinfo {pages} {035010}
  (\bibinfo {year} {2017})},\ \href {https://doi.org/10.1088/1361-6668/aa572a}
  {10.1088/1361-6668/aa572a}\BibitemShut {NoStop}%
\bibitem [{\citenamefont {Meservey}\ and\ \citenamefont
  {Tedrow}(1971)}]{meservey_properties_1971}%
  \BibitemOpen
  \bibfield  {author} {\bibinfo {author} {\bibfnamefont {R.}~\bibnamefont
  {Meservey}}\ and\ \bibinfo {author} {\bibfnamefont {P.~M.}\ \bibnamefont
  {Tedrow}},\ }\href {https://doi.org/10.1063/1.1659648} {\bibinfo {title}
  {Properties of {Very} {Thin} {Aluminum} {Films}}},\ \href
  {https://doi.org/10.1063/1.1659648} {\bibfield  {journal} {\bibinfo
  {journal} {Journal of Applied Physics}\ }\textbf {\bibinfo {volume} {42}},\
  \bibinfo {pages} {51} (\bibinfo {year} {1971})},\ \href
  {https://doi.org/10.1063/1.1659648} {10.1063/1.1659648}\BibitemShut {NoStop}%
\bibitem [{\citenamefont {Gr\"unhaupt}(2019)}]{grunhaupt_granular_2019-1}%
  \BibitemOpen
  \bibfield  {author} {\bibinfo {author} {\bibfnamefont {L.}~\bibnamefont
  {Gr\"unhaupt}},\ }\href {https://doi.org/10.5445/KSP/1000097320} {\bibinfo
  {title} {Granular aluminium superinductors}}\ (\bibinfo  {publisher} {KIT
  Scientific Publishing},\ \bibinfo {year} {2019})\ \href
  {https://doi.org/10.5445/KSP/1000097320} {10.5445/KSP/1000097320}\BibitemShut
  {NoStop}%
\bibitem [{\citenamefont {Zarea}\ \emph {et~al.}(2022)\citenamefont {Zarea},
  \citenamefont {Ueki},\ and\ \citenamefont {Sauls}}]{zarea_effects_2022}%
  \BibitemOpen
  \bibfield  {author} {\bibinfo {author} {\bibfnamefont {M.}~\bibnamefont
  {Zarea}}, \bibinfo {author} {\bibfnamefont {H.}~\bibnamefont {Ueki}},\ and\
  \bibinfo {author} {\bibfnamefont {J.~A.}\ \bibnamefont {Sauls}},\ }\href
  {https://doi.org/10.48550/arXiv.2201.07403} {\bibinfo {title} {Effects of
  anisotropy and disorder on the superconducting properties of {Niobium}}},\ \
  \bibinfo {number} {arXiv:2201.07403}\ (\bibinfo  {institution} {arXiv},\
  \bibinfo {year} {2022})\BibitemShut {NoStop}%
\bibitem [{\citenamefont {Alto\'e}\ \emph {et~al.}(2022)\citenamefont {Alto\'e},
  \citenamefont {Banerjee}, \citenamefont {Berk}, \citenamefont {Hajr},
  \citenamefont {Schwartzberg}, \citenamefont {Song}, \citenamefont
  {Alghadeer}, \citenamefont {Aloni}, \citenamefont {Elowson}, \citenamefont
  {Kreikebaum}, \citenamefont {Wong}, \citenamefont {Griffin}, \citenamefont
  {Rao}, \citenamefont {Weber-Bargioni}, \citenamefont {Minor}, \citenamefont
  {Santiago}, \citenamefont {Cabrini}, \citenamefont {Siddiqi},\ and\
  \citenamefont {Ogletree}}]{altoe_localization_2022}%
  \BibitemOpen
  \bibfield  {author} {\bibinfo {author} {\bibfnamefont {M.~V.~P.}\
  \bibnamefont {Alto\'e}}, \bibinfo {author} {\bibfnamefont {A.}~\bibnamefont
  {Banerjee}}, \bibinfo {author} {\bibfnamefont {C.}~\bibnamefont {Berk}},
  \bibinfo {author} {\bibfnamefont {A.}~\bibnamefont {Hajr}}, \bibinfo {author}
  {\bibfnamefont {A.}~\bibnamefont {Schwartzberg}}, \bibinfo {author}
  {\bibfnamefont {C.}~\bibnamefont {Song}}, \bibinfo {author} {\bibfnamefont
  {M.}~\bibnamefont {Alghadeer}}, \bibinfo {author} {\bibfnamefont
  {S.}~\bibnamefont {Aloni}}, \bibinfo {author} {\bibfnamefont {M.~J.}\
  \bibnamefont {Elowson}}, \bibinfo {author} {\bibfnamefont {J.~M.}\
  \bibnamefont {Kreikebaum}}, \bibinfo {author} {\bibfnamefont {E.~K.}\
  \bibnamefont {Wong}}, \bibinfo {author} {\bibfnamefont {S.~M.}\ \bibnamefont
  {Griffin}}, \bibinfo {author} {\bibfnamefont {S.}~\bibnamefont {Rao}},
  \bibinfo {author} {\bibfnamefont {A.}~\bibnamefont {Weber-Bargioni}},
  \bibinfo {author} {\bibfnamefont {A.~M.}\ \bibnamefont {Minor}}, \bibinfo
  {author} {\bibfnamefont {D.~I.}\ \bibnamefont {Santiago}}, \bibinfo {author}
  {\bibfnamefont {S.}~\bibnamefont {Cabrini}}, \bibinfo {author} {\bibfnamefont
  {I.}~\bibnamefont {Siddiqi}},\ and\ \bibinfo {author} {\bibfnamefont {D.~F.}\
  \bibnamefont {Ogletree}},\ }\href
  {https://doi.org/10.1103/PRXQuantum.3.020312} {\bibinfo {title} {Localization
  and {Mitigation} of {Loss} in {Niobium} {Superconducting} {Circuits}}},\
  \href {https://doi.org/10.1103/PRXQuantum.3.020312} {\bibfield  {journal}
  {\bibinfo  {journal} {PRX Quantum}\ }\textbf {\bibinfo {volume} {3}},\
  \bibinfo {pages} {020312} (\bibinfo {year} {2022})},\ \href
  {https://doi.org/10.1103/PRXQuantum.3.020312}
  {10.1103/PRXQuantum.3.020312}\BibitemShut {NoStop}%
\bibitem [{\citenamefont {Proslier}\ \emph {et~al.}(2011)\citenamefont
  {Proslier}, \citenamefont {Kharitonov}, \citenamefont {Pellin}, \citenamefont
  {Zasadzinski},\ and\ \citenamefont {{Ciovati}}}]{proslier_evidence_2011}%
  \BibitemOpen
  \bibfield  {author} {\bibinfo {author} {\bibfnamefont {T.}~\bibnamefont
  {Proslier}}, \bibinfo {author} {\bibfnamefont {M.}~\bibnamefont
  {Kharitonov}}, \bibinfo {author} {\bibfnamefont {M.}~\bibnamefont {Pellin}},
  \bibinfo {author} {\bibfnamefont {J.}~\bibnamefont {Zasadzinski}},\ and\
  \bibinfo {author} {\bibnamefont {{Ciovati}}},\ }\href
  {https://doi.org/10.1109/TASC.2011.2107491} {\bibinfo {title} {Evidence of
  {Surface} {Paramagnetism} in {Niobium} and {Consequences} for the
  {Superconducting} {Cavity} {Surface} {Impedance}}},\ \href
  {https://doi.org/10.1109/TASC.2011.2107491} {\bibfield  {journal} {\bibinfo
  {journal} {IEEE Transactions on Applied Superconductivity}\ }\textbf
  {\bibinfo {volume} {21}},\ \bibinfo {pages} {2619} (\bibinfo {year}
  {2011})},\ \href {https://doi.org/10.1109/TASC.2011.2107491}
  {10.1109/TASC.2011.2107491}\BibitemShut {NoStop}%
\bibitem [{\citenamefont {Nevala}\ \emph {et~al.}(2012)\citenamefont {Nevala},
  \citenamefont {Chaudhuri}, \citenamefont {Halkosaari}, \citenamefont
  {Karvonen},\ and\ \citenamefont {Maasilta}}]{nevala_sub-micron_2012}%
  \BibitemOpen
  \bibfield  {author} {\bibinfo {author} {\bibfnamefont {M.~R.}\ \bibnamefont
  {Nevala}}, \bibinfo {author} {\bibfnamefont {S.}~\bibnamefont {Chaudhuri}},
  \bibinfo {author} {\bibfnamefont {J.}~\bibnamefont {Halkosaari}}, \bibinfo
  {author} {\bibfnamefont {J.~T.}\ \bibnamefont {Karvonen}},\ and\ \bibinfo
  {author} {\bibfnamefont {I.~J.}\ \bibnamefont {Maasilta}},\ }\href
  {https://doi.org/10.1063/1.4751355} {\bibinfo {title} {Sub-micron
  normal-metal/insulator/superconductor tunnel junction thermometer and cooler
  using {Nb}}},\ \href {https://doi.org/10.1063/1.4751355} {\bibfield
  {journal} {\bibinfo  {journal} {Applied Physics Letters}\ }\textbf {\bibinfo
  {volume} {101}},\ \bibinfo {pages} {112601} (\bibinfo {year} {2012})},\ \href
  {https://doi.org/10.1063/1.4751355} {10.1063/1.4751355}\BibitemShut {NoStop}%
\bibitem [{\citenamefont {Feshchenko}\ \emph {et~al.}(2017)\citenamefont
  {Feshchenko}, \citenamefont {Saira}, \citenamefont {Peltonen},\ and\
  \citenamefont {Pekola}}]{feshchenko_thermal_2017}%
  \BibitemOpen
  \bibfield  {author} {\bibinfo {author} {\bibfnamefont {A.~V.}\ \bibnamefont
  {Feshchenko}}, \bibinfo {author} {\bibfnamefont {O.-P.}\ \bibnamefont
  {Saira}}, \bibinfo {author} {\bibfnamefont {J.~T.}\ \bibnamefont
  {Peltonen}},\ and\ \bibinfo {author} {\bibfnamefont {J.~P.}\ \bibnamefont
  {Pekola}},\ }\href {https://doi.org/10.1038/srep41728} {\bibinfo {title}
  {Thermal conductance of {Nb} thin films at sub-kelvin temperatures}},\ \href
  {https://doi.org/10.1038/srep41728} {\bibfield  {journal} {\bibinfo
  {journal} {Scientific Reports}\ }\textbf {\bibinfo {volume} {7}},\ \bibinfo
  {pages} {41728} (\bibinfo {year} {2017})},\ \href
  {https://doi.org/10.1038/srep41728} {10.1038/srep41728}\BibitemShut {NoStop}%
\bibitem [{\citenamefont {Burdastyh}\ \emph {et~al.}(2020)\citenamefont
  {Burdastyh}, \citenamefont {Postolova}, \citenamefont {Proslier},
  \citenamefont {Ustavshikov}, \citenamefont {Antonov}, \citenamefont
  {Vinokur},\ and\ \citenamefont {Mironov}}]{burdastyh_superconducting_2020}%
  \BibitemOpen
  \bibfield  {author} {\bibinfo {author} {\bibfnamefont {M.~V.}\ \bibnamefont
  {Burdastyh}}, \bibinfo {author} {\bibfnamefont {S.~V.}\ \bibnamefont
  {Postolova}}, \bibinfo {author} {\bibfnamefont {T.}~\bibnamefont {Proslier}},
  \bibinfo {author} {\bibfnamefont {S.~S.}\ \bibnamefont {Ustavshikov}},
  \bibinfo {author} {\bibfnamefont {A.~V.}\ \bibnamefont {Antonov}}, \bibinfo
  {author} {\bibfnamefont {V.~M.}\ \bibnamefont {Vinokur}},\ and\ \bibinfo
  {author} {\bibfnamefont {A.~Y.}\ \bibnamefont {Mironov}},\ }\href
  {https://doi.org/10.1038/s41598-020-58192-3} {\bibinfo {title}
  {Superconducting phase transitions in disordered {NbTiN} films}},\ \href
  {https://doi.org/10.1038/s41598-020-58192-3} {\bibfield  {journal} {\bibinfo
  {journal} {Scientific Reports}\ }\textbf {\bibinfo {volume} {10}},\ \bibinfo
  {pages} {1471} (\bibinfo {year} {2020})},\ \href
  {https://doi.org/10.1038/s41598-020-58192-3}
  {10.1038/s41598-020-58192-3}\BibitemShut {NoStop}%
\end{thebibliography}%


\end{document}
%
% ****** End of file apstemplate.tex ******




