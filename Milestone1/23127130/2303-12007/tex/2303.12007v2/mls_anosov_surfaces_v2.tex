%%  La classe cedram-aif est seulement un emballage par dessus
%% amsart.cls (version 2) qui implémente la mise en page de la revue,
%% ainsi que quelques commandes à des fins administratives.  
%% Les options disponibles : 
%% francais pour les articles écrits en français.
%% Une autre option, pour les auteurs qui ont préparé leur manuscrit
%%  avec amsart.cls et souhaitent que les \subsection soient présentés
%%  comme des \paragraph : SubsectAsParagraph
%% Pour des compteurs de flottants (table, figure) courant
%%  continûment sur tout l'article : NoFloatCountersInSection
%% Idem equations: NoEqCountersInSection
%% Pour utiliser le même compteur pour tout (à part les flottants,
%%  soit : theorems, subsections, equations) sous la forme
%%  <section>.<sub> : OneCounterForEverything
%% Il existe une option qui résume les trois dernières :
%%  AmsartStyle=NoFloatCountersInSection+OneCounterForEverything+SubsectAsParagraph
%% \documentclass[AmsartStyle]{cedram-aif}



\documentclass[a4paper, 12pt, oneside, reqno, notitlepage]{amsart}
\usepackage{amsmath,amssymb,amsthm,graphicx,mathrsfs,bbm,url}
\usepackage[margin=3cm]{geometry}
\usepackage{amsthm}
\usepackage{wrapfig}
\usepackage{enumitem}
\usepackage[all]{xy}
\usepackage{mathtools}
\usepackage[utf8]{inputenc}
%% Permet la coupure des mots accentués.
 \usepackage[T1]{fontenc}
\usepackage[usenames,dvipsnames]{color}
\usepackage[colorlinks=true,linkcolor=Red,citecolor=Green]{hyperref}
\usepackage[super]{nth}
\usepackage[open, openlevel=2, depth=3, atend]{bookmark}
\hypersetup{pdfstartview=XYZ}
\usepackage[font=footnotesize]{caption}
\usepackage{mathabx}
%\usepackage{a4wide}
\usepackage[colorinlistoftodos]{todonotes}
%\usepackage{ulem}

\renewcommand{\baselinestretch}{1}

%% Pour un article écrit en français, on appréciera souvent les
%% améliorations apportées par les extensions latex suivantes :
%% Babel pour une francisation des termes et une meilleure gestion
%% typographique du français


\usepackage{epstopdf}
 
\usepackage{hyperref}


\theoremstyle{plain}
\newtheorem{theorem}{Theorem}[section]
\newtheorem*{theorem*}{Theorem}
\newtheorem{theoremintro}[theorem]{Théorème}
\renewcommand{\thetheoremintro}{\Roman{theoremintro}}
\newtheorem{lemma}[theorem]{Lemma}
\newtheorem{claim}[theorem]{Claim}
\newtheorem{proposition}[theorem]{Proposition}
\newtheorem{corollary}[theorem]{Corollary}
\newtheorem{corollaryintro}[theorem]{Corollaire}
\renewcommand{\thecorollaryintro}{\Roman{corollaryintro}}
\newtheorem{conjecture}[theorem]{Conjecture}


\theoremstyle{definition}
\newtheorem{definition}[theorem]{Definition}
\newtheorem{example}[theorem]{Example}
\newtheorem{exercise}[theorem]{Exercise}

\theoremstyle{remark}
\newtheorem{remark}[theorem]{Remark}

%\addto\captionsfrench{\renewcommand\proofname{Proof}}

\numberwithin{equation}{section}

\newcommand{\LL}{\mathcal{L}(M)}
\newcommand{\Ho}{\mathcal{H}(M,\mu)}
\newcommand{\Fl}{\mathcal{F}(X,\mu)}
\newcommand{\Di}{\mathcal{D}^1(X)}
\newcommand{\Mnk}{\mathcal{M}_{n}(\mathbb{K})}
\newcommand{\K}{\mathbb{K}}
\newcommand{\C}{\mathbb{C}}
\newcommand{\R}{\mathbb{R}}
\newcommand{\Q}{\mathbb{Q}}
\newcommand{\Z}{\mathbb{Z}}
\newcommand{\D}{\mathbb{D}}
\newcommand{\E}{\mathcal{E}}
\newcommand{\M}{\mathcal{M}}
\newcommand{\N}{\mathbb{N}}
\newcommand{\T}{\mathbb{T}}
\newcommand{\V}{\mathbb{V}}
\newcommand{\X}{\mathbf{X}}
\newcommand{\OO}{\mathcal{O}}
\newcommand{\HH}{\mathbb{H}}
\newcommand{\Ss}{\mathbb{S}}
\newcommand{\eps}{\varepsilon}
\newcommand{\ke}{\text{ker }}
\newcommand{\wt}{\widetilde}
\newcommand{\til}{\widetilde}
\newcommand{\mc}{\mathcal}
\newcommand{\Cm}{\mc{C}^\infty(M,S^m(T^*M))}
\newcommand{\la}{\lambda}
\newcommand{\demi}{\tfrac{1}{2}}
\newcommand{\dd}{\mathrm{d}}
\newcommand{\pol}{\mathrm{pol}}
\newcommand{\e}{\mathbf{e}}


\DeclareMathOperator{\vol}{vol}
\DeclareMathOperator{\Ell}{ell}
\DeclareMathOperator{\Tr}{Tr}
\DeclareMathOperator{\Op}{Op}
\DeclareMathOperator{\WF}{WF}
\DeclareMathOperator{\Var}{Var}
\DeclareMathOperator{\ran}{ran}
\DeclareMathOperator{\id}{id}
\DeclareMathOperator{\Diff}{Diff}
\DeclareMathOperator{\Div}{div}
\DeclareMathOperator{\Id}{id}
\DeclareMathOperator{\supp}{supp}
\DeclareMathOperator{\suppsing}{suppsing}
\DeclareMathOperator{\Graph}{Graph}
\DeclareMathOperator{\sol}{sol}

\DeclarePairedDelimiter{\abs}{\lvert}{\rvert}
\DeclarePairedDelimiter{\norm}{\lVert}{\rVert}


\newcommand{\be}{\begin{equation}}
\newcommand{\ee}{\end{equation}}


\DeclarePairedDelimiter\floor{\lfloor}{\rfloor}




%% On peut charger ici des extensions standard si les fonctions
%% fournies sont nécessaires à la compilation de l'article. 
%\usepackage{hyperref}
%\usepackage{graphicx}
%\usepackage[matrix,arrow,tips,curve]{xy}
% ...

%% Définitions utilisateur et macros pratiques...  De telles
%% définitions sont interdites dans les titres, les résumés ou la
%% bibliographie.
%\newcommand{\la}{\longrightarrow}
% ...

%% Un ensemble de théorèmes sont prédéfinis. La règle mnémotechnique
%% est que le nom de l'environnement est formé des quatre premières
%% lettres (sans accents) de l'étiquette utilisée (theo, exam, rema,
%% coro, conj, etc.) ; les versions astérisquées (non numérotées)
%% existent de même (theo*, etc.). Pour ne pas modifier vos habitudes
%% de saisie, il est possible de déclarer, par exemple :
%\equalenv{remark}{remark}
%% qui définit un environnement « remark » identique à « rema ».

%% Le titre de l'article: syntaxe d'amsart.
\title
%% L'argument optionnel donne la version courte pour les entêtes.
[Marked length spectrum rigidity for Anosov surfaces]
%% L'argument obligatoire est imprimé sur la première page, dans les
%% sommaires, entêtes si la version courte n'est pas spécifiée.
{Marked length spectrum rigidity for Anosov surfaces}

%% Le titre anglais de l'article.
%\alttitle{Sur quelques problèmes de tomographie des tenseurs sur les surfaces}

%% Les noms des auteurs, selon la syntaxe d'amsart, avec en outre la
%% distinction prénom/nom

\author[Guillarmou]{Colin Guillarmou}
\address{Universit\'e Paris-Saclay, CNRS,  Laboratoire de math\'ematiques d'Orsay, 91405, Orsay, France.}
\email{colin.guillarmou@universite-paris-saclay.fr}

\author[Lefeuvre]{Thibault Lefeuvre}
\address{Université de Paris and Sorbonne Université, CNRS, IMJ-PRG, F-75006 Paris, France.}
\email{tlefeuvre@imj-prg.fr}

\author[Paternain]{Gabriel P. Paternain}
\address{Department of Pure Mathematics and Mathematical Statistics, University of
Cambridge, Cambridge CB3 0WB, UK}
\email{g.p.paternain@dpmms.cam.ac.uk}

\begin{document}
%% Résumé


%% Résumé anglais
\begin{abstract}
Let $\Sigma$ be a smooth closed oriented surface of genus $\geq 2$. We prove that two metrics on $\Sigma$ with same marked length spectrum and Anosov geodesic flow are isometric (via an isometry isotopic to the identity). This generalizes to the Anosov setting the marked length spectrum rigidity result of Croke \cite{Croke-90} and Otal \cite{Otal-90} for negatively curved surfaces. The proof combines microlocal tools with the geometry of complex curves. 
\end{abstract}


\maketitle



\section{Introduction}

\subsection{Main results}

On a smooth closed connected oriented manifold $\Sigma$, a metric $g$ is \emph{Anosov} if its geodesic flow on the unit tangent bundle $S\Sigma$ satisfies the Anosov property defined in \eqref{equation:anosov}. While negatively curved metrics are typical examples of Anosov metrics, the set of Anosov metrics is a considerably larger open set compared to that of negatively curved metrics (cf. \cite{Eberlein-73}). However, this set is less well understood: for instance it is still unknown whether the set of Anosov surfaces is path-connected or whether any manifold (of dimension $\geq 4$) that admits an Anosov metric also admits a negatively curved metric.  Anosov metrics coincide with the $C^2$-interior of metrics without conjugate points \cite{Ruggiero-91}, and they can also exist isometrically embedded in $\mathbb{R}^3$ \cite{Donnay-Pugh-03}, unlike negatively curved metrics.


%Let $\Sigma$ be a smooth closed connected oriented manifold. We say that a metric $g$ on $\Sigma$ is \emph{Anosov} if the geodesic flow on the unit tangent bundle $S\Sigma$ is Anosov, see \eqref{equation:anosov} for a definition. Typical examples are provided by metrics of negative sectional curvature but it should be observed that the set of Anosov metrics is a considerably larger open set than that of negatively-curved metrics (cf. \cite{Eberlein-73}) and it is also less well understood, for instance, it is still unknown if it is path-connected.
%Anosov metrics coincide with the $C^2$-interior of metrics without conjugate points \cite{Ruggiero-91}, and may also exist isometrically embedded in $\mathbb{R}^3$ \cite{Donnay-Pugh-03} (unlike negatively curved metrics).
On $\Sigma$, we shall denote by $\mc{M}_{\mathrm{Anosov}}(\Sigma)$ the set of all Anosov metrics (assuming it is non-empty) and by $\mathbf{M}_{\mathrm{Anosov}}(\Sigma) := \mc{M}_{\mathrm{Anosov}}(\Sigma)/\mathrm{Diffeo}^0(\Sigma)$ the \emph{moduli space} of Anosov metrics, where we denote by $\mathrm{Diffeo}^0(\Sigma)$ the group of diffeomorphisms isotopic to the identity.
Let $\mc{C}$ be the set of free homotopy classes on $\Sigma$. This set is countable and in natural correspondence with conjugacy classes of $\pi_1(\Sigma,\star)$. It is well-known that for an Anosov metric $g$ there exists a unique closed geodesic $\gamma_g(c)$ in each free homotopy classes $c \in \mc{C}$. The marked length spectrum is then defined as the map
\begin{equation}
\label{equation:mls}
\mc{L} : \mathbf{M}_{\mathrm{Anosov}}(\Sigma) \to (0,\infty)^{\mc{C}}, \qquad \mc{L}_g(c) := \ell_g(\gamma_g(c)),
\end{equation}
where $\ell_g(\gamma)$ denotes the length of the curve $\gamma$ computed with respect to $g$.

It is conjectured that the marked length spectrum map \eqref{equation:mls} is injective. For negatively curved metrics, this is known as the Burns-Katok conjecture \cite{Burns-Katok-85}. The purpose of the present paper is to establish this conjecture when $\dim \Sigma =2$, that is, when $\Sigma$ is a surface.

\begin{theorem}
\label{theorem:main}
Let $\Sigma$ be a smooth closed connected oriented surface. Then the marked length spectrum
\[
\mc{L} : \mathbf{M}_{\mathrm{Anosov}}(\Sigma) \to (0,\infty)^{\mc{C}}
\]
is injective. In other words, if $g_1$ and $g_2$ are two Anosov metrics on $\Sigma$ with same marked length spectrum (that is, $\mc{L}_{g_1} = \mc{L}_{g_2}$), then there exists a smooth diffeomorphism $\phi : \Sigma \to \Sigma$, isotopic to the identity, such that $\phi^* g_1 = g_2$.
\end{theorem}

Theorem \ref{theorem:main} should be regarded as the analogue for closed surfaces of the celebrated boundary rigidity result of Pestov and Uhlmann for simple surfaces \cite{Pestov-Uhlmann-05}. Recall that a surface is said to be simple if it has strictly convex boundary, it is non-trapping and it has
no conjugate points. 
These surfaces, similar to Anosov surfaces, belong to a $C^2$-open set of metrics where the boundary is the focal point of all the activity. Our proof draws inspiration from Pestov-Uhlmann's approach that depends on utilizing ``fiberwise holomorphic'' smooth invariant functions for the geodesic flow, see \cite[Chapter 8]{Paternain-Salo-Uhlmann-book}. However, in the case of closed surfaces, there is no natural Calderón problem connected to the marked length spectrum as in \cite{Pestov-Uhlmann-05}, and the smooth invariant functions need to be replaced by suitable singular invariant distributions, which has made it difficult to apply this strategy in our context until now. We expect that boundary rigidity for simple surfaces in fact follows from Theorem \ref{theorem:main} in conjunction with the embedding theorem \cite{Chen-Erchenko-Gogolev-20}; this will be discussed elsewhere.







%As it will become clear below, the proof presented here is inspired by the Pestov-Uhlmann approach based on the use of ``fiberwise holomorphic'' invariant distributions for the flow. However, in the closed surface case, there is no natural Calder\'on problem related to the marked length spectrum as in \cite{Pestov-Uhlmann-05}, 
%and the invariant distributions are singular. These two facts made unclear until now how to apply this strategy in our setting. 
%\todo[inline]{C: I added a comment to emphasize that this is not like straightforward application of the idea of Pestov-Uhlmann}

%In fact the analogy is more than such: using the embedding theorem of Chen, Erchenko and Gogolev \cite{Chen-Erchenko-Gogolev-20}, Theorem \ref{theorem:main} should imply the rigidity of the (marked) boundary distance for surfaces with strictly convex boundary and hyperbolic trapped set (this is ongoing work in collaboration with A. Erchenko). As a corollary, Theorem \ref{theorem:main} should therefore imply boundary rigidity for simple surfaces.




%\todo[inline]{G: I somehow did not agree with some of the written points. Given the results on surfaces with no conjugate points we were getting into debatable territory. I also checked \cite{Bonahon-93} and I could not see Finsler metrics, although I do not doubt it fails there. But this is just saying this class is too wild (and Finsler are alway too wild for this kind of things). In fact I think there are great classes out there for a which some version of this is true, for example projective structures and Hilbert geodesic flows. That would in fact be a great project, but you both know I am too keen in coming out of the contact world. Most Anosov flows dissipate!}
%


%\begin{enumerate}[label=(\roman*)]
%\item First, on closed surfaces, if the metric is not Anosov, uniqueness of the closed geodesic in a fixed homotopy class may fail (even for metrics without conjugate points such as nonpositively curved metrics with an embedded flat cylinder) and thus prevents to even define the marked length spectrum map \eqref{equation:mls}.
%\item Second, by using the embedding theorem of Chen, Erchenko and Gogolev \cite{Chen-Erchenko-Gogolev-20}, Theorem \ref{theorem:main} should imply the rigidity of the (marked) boundary distance for surfaces with strictly convex boundary and hyperbolic trapped set (this is ongoing work in collaboration with A. Erchenko). As a corollary, Theorem \ref{theorem:main} should therefore imply the Pestov-Uhlmann rigidity Theorem \cite{Pestov-Uhlmann-05} of simple disks.
%\item Eventually, beyond the realm of Riemannian metrics, it was shown by Bonahon \cite{Bonahon-93} that the marked length spectrum rigidity conjecture does not hold for Finsler metrics.
%\end{enumerate}

%Prior to Theorem \ref{theorem:main}, the injectivity of the marked length spectrum on closed surfaces was only known for non-positively curved metrics, see the work by Croke \cite{Croke-90} and Otal \cite{Otal-90}, and the generalization by Croke, Fathi and Feldman \cite{Croke-Fathi-Feldman-92}. Beyond surfaces, fewer results are known and they all assume that the metrics have non-positive curvature: Katok \cite{Katok-88} proved injectivity within the same conformal class, Hamenstädt \cite{Hamenstadt-99}, based on the entropy paper of Besson-Courtois-Gallot \cite{Besson-Courtois-Gallot-95}, proved the conjecture when one of the two metrics is locally symmetric. The first two authors established a local version of the conjecture \cite{Guillarmou-Lefeuvre-19} in any dimension when the curvature is non-positive, see also \cite{Guillarmou-Knieper-Lefeuvre-22} for an alternative proof based on the notion of geodesic stretch.


 Prior to Theorem \ref{theorem:main}, the only known cases of injectivity of the marked length spectrum for closed surfaces were essentially limited to metrics with non-positive curvature, as shown in the works of Croke \cite{Croke-90} and Otal \cite{Otal-90}, and the subsequent generalization by Croke, Fathi, and Feldman \cite{Croke-Fathi-Feldman-92}. These proofs {\it do not} extend to the Anosov setting as they rely crucially on the assumption that the Gauss curvature is non-positive (or that the Morse correspondence preserves angles); thus a new approach is needed to establish Theorem \ref{theorem:main}.
 For dimensions greater than or equal to three, fewer results exist. Given two Anosov metrics in the same conformal class, Katok's argument in \cite{Katok-88} provides injectivity of the marked length spectrum. All other results require non-positive sectional curvature: Hamenstädt \cite{Hamenstadt-99} proved the conjecture when one of the two metrics is locally symmetric, relying on the entropy paper of Besson-Courtois-Gallot \cite{Besson-Courtois-Gallot-95}. Recently, the first two authors of this paper proved a local version of the conjecture in any dimension when the curvature is non-positive \cite{Guillarmou-Lefeuvre-19}. An alternative proof based on the concept of geodesic stretch can be found in \cite{Guillarmou-Knieper-Lefeuvre-22}.
 





%Note that the techniques of \cite{Guillarmou-Lefeuvre-19,Guillarmou-Knieper-Lefeuvre-22} provide \emph{stability estimates} for the marked length spectrum (that is, the distance between the isometry classes is controlled by the ratio of the marked length spectra) and combining both local stability estimates with global injectivity in Theorem \ref{theorem:main}, global estimates could be derived.
%\todo[inline]{G: The way this is written it could be interpreted as saying that you guys proved local rigidity at any Anosov metric in any dimension, although you do say that all results assume negative curvature. 
%You proved this in 2D though, thanks to tensor tomography. Btw our main thm does NOT need tensor tomography for 2-tensors, (Pestov-Uhlmann also does not need tensor tomography for 2-tensors). But don't you guys prove it assuming Anosov and $K\leq 0$ which is a bit more general?}


%\todo{G: I'm now determined to hunt down all those "Eventually"!}
As a direct consequence of Theorem \ref{theorem:main}, the proof of \cite[Theorem 1.1]{Croke-Dairbekov-04} provides a more general version of Theorem \ref{theorem:main} related to the minimal filling problem of Gromov, see \cite[Question 6.8]{Croke-04} where this is discussed.

%\todo[inline]{C: I did put the result below as a Corollary since we don't prove anything here, it's all proved in Croke-Dairbekov once Theorem 1 is true.}
\begin{corollary}
\label{theorem:main2}
Let $\Sigma$ be a smooth closed connected oriented surface and $g_1, g_2 \in \mathbf{M}_{\mathrm{Anosov}}(\Sigma)$. If $\mc{L}_{g_1} \geq \mc{L}_{g_2}$, then $\mathrm{vol}_{g_1} \geq \mathrm{vol}_{g_2}$ with equality of volumes if and only if there exists a smooth diffeomorphism $\phi : \Sigma \to \Sigma$, isotopic to the identity, such that $\phi^* g_1 = g_2$.
\end{corollary}

In Corollary \ref{theorem:main2}, the volumes of the metrics are computed with respect to the Riemannian measure. Corollary \ref{theorem:main2} follows from the positive Liv\v sic Theorem of Lopes and Thieullen \cite{Lopes-Thieullen-05} and Theorem \ref{theorem:main}, see \cite{Croke-Dairbekov-04} for a proof.


\subsection{Strategy}
The proof strategy involves demonstrating that the marked length spectrum $\mc{L}_g$ captures the complex structure of the metric $[g]$ up to biholomorphisms isotopic to the identity, and hence determines the class of the underlying complex structure in Teichmüller space (Proposition \ref{proposition:key2}). Once this is established, the injectivity of the marked length spectrum in the same conformal class, as shown by Katok \cite{Katok-88}, leads to a straightforward proof of Theorem \ref{theorem:main}.

To establish that $\mc{L}_g$ determines the complex structure, we show that it encodes the period matrix of the underlying Riemann surface (Proposition \ref{proposition:key}) and then observe that the argument may be repeated on any finite cover
to recover the structure in Teichmüller space. This relies on the fact that given two different points
in Teichm\"uller space, there is a finite cover of the surface where the lifted complex structures are in different orbits of the mapping class group
(Lemma \ref{lemma:cover}).


To recover the period matrix, we introduce an algebra $\mc{A}_+(S\Sigma)$ of \emph{fiberwise holomorphic flow-invariant} distributions on $S\Sigma$ that are distributions invariant under the geodesic flow and have a specific Fourier decomposition in the circle fibers of $S\Sigma$ (see \S\ref{sssection:degree}). Their first Fourier mode corresponds to genuine holomorphic differentials on the Riemann surface. It is worth noting that the existence of such a well-behaved algebra of distributions is a non-trivial fact and relies on advances in tensor tomography on surfaces \cite{Paternain-Salo-Uhlmann-14-2,Guillarmou-17-1}. 

Finally, we introduce a \emph{generalized intersection number} of elements in $\mc{A}_+(S\Sigma)$ with closed geodesics, extending the usual definition of the intersection number for currents \emph{à la} Bonahon \cite{Bonahon-88}, Otal \cite{Otal-90}, and others. This allows for the recovery of integrals 
of holomorphic $1$-forms on closed geodesics, and thus the period matrix, from the conjugacy class of the flow (Lemma \ref{lemma:magical} and Remark \ref{remark:number}).



\subsection{Organization of the paper}

The proof of Theorem \ref{theorem:main} relies on several tools, which we introduce in Section \ref{section:tools}. Specifically, \S\ref{ssection:complex} provides a brief review of the geometry of complex curves, while \S\ref{ssection:unit} is devoted to the geometry and harmonic analysis of the unit tangent bundle of a surface.
 In  \S\ref{ssection:hyperbolic}, we delve into hyperbolic dynamics and tensor tomography. Finally, we present the proof of Theorem \ref{theorem:main} in Section \ref{section:proof}. \\
 
 
\noindent \textbf{Acknowledgements.} We warmly thank J. Marché, B. Petri, I. Smith and M. Wolff for very helpful discussions related to the proof of Lemma \ref{lemma:cover}. We also thank C. Matheus for pointing out an error in an earlier draft.

\section{Tools}

\label{section:tools}

\subsection{Complex geometry}

\label{ssection:complex}

In what follows, $\Sigma$ is a smooth closed oriented surface of genus $\geq 2$.

\subsubsection{General facts} Let $g$ be a smooth Riemannian metric on $\Sigma$. The conformal class $[g]$ of $g$ (and the orientation of $\Sigma$) induces a complex structure $J \in C^\infty(\Sigma,\mathrm{End}(T\Sigma))$ on $\Sigma$, turning it into a Riemann surface which we shall denote by $(\Sigma,J)$. 


We denote by $\mc{T}(\Sigma)$ the Teichmüller space of $\Sigma$, that is, the space of complex structures $J$ on $\Sigma$ modulo the equivalence relation that $J \sim J'$ iff there exists a diffeomorphism $\psi : \Sigma \to \Sigma$, isotopic to the identity, such that $\psi^*J = J'$. Such an equivalence class of complex structures will be denoted by $[J]$. The mapping class group $\mathrm{MCG}(\Sigma)$ is defined as the quotient of orientation preserving diffeomorphisms $\mathrm{Diff}^+(\Sigma)$ modulo isotopy.
%, while the \emph{extended} mapping class group is defined as the quotient of all diffeomorphisms $\mathrm{Diff}(\Sigma)$ modulo isotopy. Note that $\mathrm{MCG}^{\pm}(\Sigma)$ is a degree $2$ extension of $\mathrm{MCG}(\Sigma)$ as $\Sigma$ admits an orientation reversing involution.

There is a well-defined action of $\mathrm{MCG}(\Sigma)$ on $\mc{T}(\Sigma)$ (by pullback). The quotient space $\mc{M}(\Sigma) := \mc{T}(\Sigma)/\mathrm{MCG}(\Sigma)$ is called the \emph{moduli space} of (complex structures on) $\Sigma$. We refer to \cite{Farb-Margalit-11} for a general introduction to the mapping class group of closed surfaces.

%Any element $[\psi] \in \mathrm{MCG}(\Sigma)$ has a well-defined action on $H_1(\Sigma,\Z)$ (by choosing any representative $\psi$ in the class of $[\psi]$ and looking at its induced action on $H_1(\Sigma,\Z)$). This defines a natural (symplectic) representation 
%\begin{equation}
%\label{equation:rho}
%\rho : \mathrm{MCG}(\Sigma) \longrightarrow {\rm Aut}(H_1(\Sigma,\Z))
%\end{equation}
%whose kernel $\mathcal{I}(\Sigma) := \ker \rho$ is called the \emph{Torelli group} of $\Sigma$, see \cite[Section 6.2]{Farb-Margalit-11} for instance. We shall also need $\mc{I}^{\pm}(\Sigma)$ the degree $2$ extension of the Torelli group obtained by considering elements of $\mathrm{MCG}(\Sigma)$ acting as $\pm \mathbf{1}$ on $H_1(\Sigma,\Z)$. {\color{red}I would remove the discussion on the Torelli group as this is not needed anymore.}



\subsubsection{Period matrix. Jacobian.} Let $\left\{a_i,b_j\right\}$ be a canonical basis of the homology $H_1(\Sigma,\Z)$ on the surface $\Sigma$. Let $(\Sigma,J)$ be a Riemann surface structure on $\Sigma$ and denote by $H_J^0(\Sigma,K^m)$ the space of holomorphic sections of the $m$-th power of the canonical bundle $K := T^*_{\C}\Sigma^{1,0}$ of $(\Sigma,J)$, for $m\geq 1$.
It is well-known that there exists a unique basis $\left\{\zeta_i\right\}$ of holomorphic Abelian differentials in $H_J^0(\Sigma,K)$ such that
\[
\int_{a_j} \zeta_k = \delta_{jk},
\]
see \cite[Proposition, page 63]{Farkas-Kra-92} for instance. The \emph{period matrix} of $(\Sigma,J)$ is then defined as the matrix $\Pi(J)$ whose $jk$-entry is
\[
\Pi(J)_{jk} = \int_{b_j} \zeta_k.
\]
It is a symmetric matrix with positive definite imaginary part. The space of symmetric matrices with positive definite imaginary part and size given by the genus of $\Sigma$ is called the \emph{Siegel upper half-space} $\mc{H}(\Sigma)$. Hence, we get a well-defined period matrix map
\[
\Pi : \mc{T}(\Sigma) \longrightarrow \mc{H}(\Sigma).
\]
We will need the Torelli theorem in the following form: 
\begin{theorem}
\label{theorem:torelli}
Assume that $\Sigma$ has genus $\geq 2$. If $\Pi(J_1) = \Pi(J_2)$, then there exists an orientation-preserving diffeomorphism $\psi : \Sigma \to \Sigma$ such that $\psi^*J_2 =  J_1$.
\end{theorem}

We refer to \cite[Theorem III.12.3]{Farkas-Kra-92} for a proof. Actually, it can be proved that $[\psi] \in \mathrm{MCG}(\Sigma)$ lives in a degree $2$ extension of the \emph{Torelli group} that is, it acts as $\pm \mathbf{1}$ on homology $H_1(\Sigma,\Z)$ but this will not be needed in what follows.


\subsection{Unit tangent bundle of the surface}

\label{ssection:unit}

Let
\[
S\Sigma := \left\{ (x,v) \in T\Sigma ~|~ |v|_g=1\right\}
\]
be the unit tangent bundle of $(\Sigma,g)$ and $\pi:S\Sigma\to \Sigma$ the projection. 

\subsubsection{Geometry of $S\Sigma$}

Let $(\varphi_t)_{t \in \R}$ be the geodesic flow on $S\Sigma$ and $X$ its infinitesimal generator. Let $V$ be the vertical vector field generating the $\mathrm{SO}(2)$-rotation group $(R_{\theta})_{\theta \in [0,2\pi]}$ in the fibers and let $\V := \R V$. Define $H := -[X,V]$ and $\HH := \R H$. The vector fields $\left\{X, H, V\right\}$ form an orthonormal basis on $SM$ for the \emph{Sasaki metric} (the natural lift of $g$ to $S\Sigma$). We also define
\begin{equation}
\label{equation:etapm}
\eta_\pm := \frac{1}{2}(X \mp iH).
\end{equation}
These operators are called the \emph{raising} ($+$) and \emph{lowering} ($-$) operators.

The \emph{Liouville} $1$-form $\lambda \in C^\infty(S\Sigma, T^*(S\Sigma))$ is defined by $\lambda(X) = 1$ and $\lambda(H)=\lambda(V)=0$. It is invariant by the geodesic flow, that is, $\mc{L}_X \lambda = 0$. Moreover, $d\lambda$ is a $2$-form that is non-degenerate on the \emph{contact plane} $\R H \oplus \R V$ and such that $\iota_X \dd\lambda = 0$. Hence
\[
\mu :=-\lambda \wedge \dd\lambda
\]
is a volume-form, invariant by the geodesic flow, called the \emph{Liouville} volume form. Equivalently, $\mu$ is the Riemannian volume form induced by the Sasaki metric on $S\Sigma$. From now on, the $L^2$ space on $S\Sigma$ is defined as $L^2(S\Sigma) := L^2(S\Sigma, \mu)$.


We define the $1$-forms $\beta, \psi$ on $S\Sigma$ by $\beta(H)=1=\psi(V)$ and $\beta(X)=\beta(V)=0=\psi(X)=\psi(H)$. It can then be checked that
\begin{equation}
\label{equation:dlambda}
\dd\lambda =  \psi \wedge \beta , \qquad \mu = \lambda \wedge \beta \wedge \psi.
\end{equation}
We set $(E^0)^* := \R\lambda$, $\HH^* := \R \beta$ and $\V^* := \R \psi$. We refer to \cite{Paternain-99} and \cite[Chapter 3]{Paternain-Salo-Uhlmann-book} for further details on the geometric structure on $S\Sigma$, see also Figure \ref{figure} below for a representation of the bundles introduced above.

\subsubsection{Powers of the canonical line bundle}

\label{sssection:powers}

We introduce the complex line bundle $\Omega_1 \to \Sigma$ whose fiber over $x \in \Sigma$ is given by:
\[
(\Omega_1)_x := \left\{ u(x, \cdot) ~|~ u \in C^\infty(S\Sigma), Vu = iu\right\}.
\]
The line bundle $\Omega_1$ is isomorphic to the canonical line bundle $K$ of the underlying Riemann surface $(\Sigma,J)$, that is, there exists a fiberwise linear map $\pi_1^* : K \to \Omega_1$ given for all $x \in \Sigma$ by
\[
\pi_1^* : K_x \ni f \mapsto \left(S_x\Sigma \ni v \mapsto f(v)\right) \in (\Omega_1)_x.
\]
The powers $\Omega_n := \Omega_1^{\otimes n}$ for $n \in \Z$, correspond to
\[
(\Omega_n)_x = \left\{ u(x, \cdot) ~|~ u \in C^\infty(S\Sigma), Vu = inu\right\}
\]
and
\[
\pi_n^* : K_x^{\otimes n} \ni f \mapsto  \left(S_x\Sigma \ni v \mapsto f(v,...,v)\right) \in (\Omega_n)_x
\]
is an isomorphism. For $n \geq 0$ (resp. $n \leq 0$), it can be checked that $\pi_n^* : K^{\otimes n} \to \Omega_n$ intertwines the operators $\bar{\partial}$ and $\eta_-$ (resp. $\partial$ and $\eta_+$), see \cite[Lemma 2.1]{Paternain-Salo-Uhlmann-14-2} and the discussion below. Hence, from now on, we will freely identify $K^{\otimes n}$ with $\Omega_n$ via $\pi_n^*$. We denote its adjoint by ${\pi_n}_*$ (with respect to the natural inner products on $K^{\otimes n}$ and $\Omega_n$). Note that $(2\pi)^{-1} {\pi_n}_* \pi_n^*$ is the identity on $C^\infty(S\Sigma,K^{\otimes n})$ and
\begin{equation}
\label{equation:proj}
\dfrac{1}{2\pi} {\pi_n}^* {\pi_n}_* : L^2(S\Sigma) \to L^2(S\Sigma)
\end{equation}
is the $L^2$-orthogonal projection onto the $n$-th Fourier mode. Any function $f \in L^2(S\Sigma)$ can thus be decomposed as $f = \sum_{n \in \Z} f_n$, where $f_n := (2\pi)^{-1} {\pi_n}^* {\pi_n}_* f$.

Define
\begin{equation}
\label{equation:h0}
H^0(\Sigma,\Omega_n) := \left\{ u \in C^\infty(S\Sigma), Vu = inu, \eta_- u = 0\right\}.
\end{equation}
We shall denote by $H^0_J(\Sigma,K^{\otimes n})$ the complex vector space of holomorphic differentials of degree $n$. The subscript $J$ indicates that this is computed with respect to the complex structure $J$. Observe that by the previous discussion, $\pi_n^*$ identifies $H^0_J(\Sigma,K^{\otimes n})$ with $H^0(\Sigma,\Omega_n)$ for $n \geq 0$.

By the Riemann-Roch theorem, the following holds true:
\begin{lemma}
\label{lemma:rr}
As complex vector spaces, the dimensions of $\ker \eta_\pm|_{C^\infty(\Sigma,\Omega_n)}$ are given by the following:
	\begin{itemize}
		\item Kernels of $\eta_+$: For $n \leq-2$, $\dim \ker \eta_+ = -(2n+1)(g-1)$; for $n=-1$, $\dim \ker \eta_+ = g$; for $n=0$, $\dim \ker \eta_+=1$; for $n \geq 1$, $\eta_+$ is injective.
\item Kernels of $\eta_-$: For $n \geq 2$, $\dim \ker \eta_- = (2n-1)(g-1)$; for $n=1$, $\dim \ker \eta_- = g$; for $n =0$, $\dim \ker \eta_- = 1$; for $n \leq -1$, $\eta_-$ is injective.
	\end{itemize}
\end{lemma}

We refer to \cite[Lemma 2.1]{Paternain-Salo-Uhlmann-14-2} for a proof. We let $\K := \oplus_{n \geq 0} K^{\otimes n}$. Observe that $H^0_J(\Sigma,\K):=\oplus_{n\geq 0}H^{0}_J(\Sigma,K^{\otimes n})$ is naturally a graded Abelian unital algebra (the constant function $\mathbf{1}_{\Sigma}$ is the unit) and that there is a natural algebra homomorphism
\[
H^0_J(\Sigma,\K) \to C^\infty(S\Sigma).
\]
Here, sections of $H^0_J(\Sigma,\K)$ are understood in the \emph{algebraic sense}, that is, they correspond to \emph{finite sums} of holomorphic sections. 


\subsubsection{Fiberwise holomorphic distributions. Notion of degree.}

\label{sssection:degree}

Denote by $\mc{D}'(S\Sigma)$ the space of distributions on $S\Sigma$ seen as the topological dual of volume forms, that is:
\[
\mc{D}'(S\Sigma) := (C^\infty(S\Sigma, \Lambda^3 T^* S\Sigma))'.
\]
Elements in $\mc{D}'(S\Sigma)$ are ``generalized'' functions which can be paired intrinsically against smooth volume forms. The \emph{wavefront set} $\WF(f)$ of a distribution $f \in \mc{D}'(S\Sigma)$ describes the (co)directions in $T^*(S\Sigma)$ in which the distribution is irregular, see \cite[Chapter 8]{Hormander-90} for a detailed account. If $\Gamma \subset T^*(S\Sigma) \setminus \left\{0\right\}$ is a closed conic subset, we denote by $\mc{D}'_{\Gamma}(S\Sigma)$ the set of distributions with wavefront set contained in $\Gamma$.

The vector field $X$ acts on $\mc{D}'(S\Sigma)$ by duality, namely, given $u \in \mc{D}'(S\Sigma)$ and $\omega \in C^\infty(S\Sigma, \Lambda^3 T^* S\Sigma)$, $(Xu,\omega) := -(u,\mc{L}_X \omega)$. As for $L^2$-functions, any $f \in \mc{D}'(S\Sigma)$ can be decomposed as a sum of Fourier modes
\begin{equation}
\label{equation:decomp}
f = \sum_{n \in \Z} f_n,
\end{equation}
where $f_n \in \mc{D}'(S\Sigma)$ satisfies $V f_n = in f_n$ and $f_n := \tfrac{1}{2\pi}{\pi_n}^* {\pi_n}_* f$. 


A distribution (or a function) is said to have \emph{finite Fourier degree} if the sum \eqref{equation:decomp} only contains a finite number of terms. It is said to be \emph{fiberwise holomorphic} (resp. \emph{antiholomorphic}) if \eqref{equation:decomp} only contains $f_n$ terms for $n\geq 0$ (resp. non-positive). Equivalently, defining the Szegö projections $\mathbf{S}_\pm : \mc{D}'(S\Sigma) \to \mc{D}'(S\Sigma)$ by
\begin{equation}
\label{equation:szego}
\mathbf{S}_+\left(\sum_{n \in \Z} f_n\right) :=\sum_{n \geq 0} f_n, \qquad \mathbf{S}_-\left(\sum_{n \in \Z} f_n\right) :=\sum_{n \leq 0} f_n,
\end{equation}
a distribution $f$ is fiberwise holomorphic iff $\mathbf{S}_+f =f$ and antiholomorphic iff $\mathbf{S}_-f=f$. A distribution is \emph{even} (resp. \emph{odd}) if its decomposition \eqref{equation:decomp} only contains even (resp. odd) terms.



The operators $\eta_\pm$ act as raising/lowering operators on the Fourier decomposition \eqref{equation:decomp}, that is:
\[
\eta_\pm : C^\infty(\Sigma, \Omega_n) \to C^\infty(\Sigma,\Omega_{n\pm 1})
\]
is continuous. Hence, the operator $X = \eta_- + \eta_+$ acts on the decomposition \eqref{equation:decomp} as
\[
X\left(\sum_{n \in \Z} f_n\right) = \sum_{n \in \Z} \eta_+ f_{n-1} + \eta_- f_{n+1}.
\] 

Let
\begin{equation}\label{DefA_+}
\mc{A}_\pm(S\Sigma) := \left\{ f \in \mc{D}'(S\Sigma) ~|~ X f = 0, \; \mathbf{S}_\pm f = f\right\},
\end{equation}
be the algebra of flow-invariant fiberwise holomorphic (resp. antiholomorphic) distributions. It can be checked that $\mc{A}_\pm(S\Sigma)$ is indeed an algebra, regardless of the underlying Riemannian geometry of the surface, that is, multiplication is well-defined and continuous with respect to the $\mc{D}'(S\Sigma)$ topology, see \cite[Theorem 1.1]{Bohr-Lefeuvre-Paternain-23}. Nevertheless, the algebra property will not be needed in the proof of Theorem \ref{theorem:main}.
%In order to simplify the discussion, we will call \emph{holomorphic} the functions\footnote{They actually correspond exactly to traces of genuine holomorphic functions on a complex surface diffeomorphic to the unit disk bundle over $M$ but this will not be needed (see \cite{Bohr-Paternain-21, Bohr-Lefeuvre-Paternain-23} for further details).} in $\mc{A}(SM)$. 


\begin{definition}
\label{definition:degree}
We define $\deg : \mc{A}_+(S\Sigma) \to \Z_{\geq 0}$ and $\sigma: \mc{A}_+(S\Sigma) \to H_J^0(\Sigma,\K)$ as follows: given $f = \sum_{n \geq 0} f_n \in \mc{A}_+(S\Sigma)$, $\sigma(f) \in H_J^0(\Sigma,\K)$ is the first non-zero Fourier mode $f_{n_0}$ and $\deg(f) := n_0$. 
\end{definition}

Similarly, a nonpositive degree map $\deg : \mc{A}_-(S\Sigma) \to \Z_{\leq 0}$ can be defined for fiberwise antiholomorphic invariant distributions. The function $\deg$ on $\mc{A}_+(S\Sigma)$ is a valuation map (although, we do not have a field but rather a ring, which is why the valuation takes values in $\Z_{\geq 0}$, not $\Z$). We have the obvious properties:
\begin{equation}
\label{equation:properties-degree}
\deg(f + g) \geq \min(\deg(f),\deg(g)), \qquad \deg(fg) = \deg(f) + \deg(g).
\end{equation}
In the following, we will denote by $\mc{A}_{\geq n}(S\Sigma)$ the fiberwise holomorphic invariant distributions of degree $\geq n$.


%
%\subsection{Wavefront set of distributions. Pairings}
%
%XX define XX
%
%In what follows, if $\iota : Y \hookrightarrow S\Sigma$ is a closed $k$-dimensional submanifold of $S\sigma$ and $\omega \in \mc{D}'(S\Sigma, \Lambda^k T^*(S\Sigma))$ is a distributional $k$-form, we denote by
%\[
%(Y,\omega) := \int_Y \iota^*\omega,
%\]
%the pairing of $\omega$ with $Y$ when defined under the wavefront set condition
%\begin{equation}
%\label{equation:wf-condition}
%\WF(\omega) \cap N^* Y = \emptyset,
%\end{equation}
%where $N^*Y$ denotes the conormal to $Y$.


\subsection{Hyperbolic dynamics}

\label{ssection:hyperbolic}

We now further assume that $(\Sigma,g)$ is an \emph{Anosov} metric, that is, the geodesic flow on $S\Sigma$ is Anosov (or \emph{uniformly hyperbolic}).

\subsubsection{Definition. First properties}

Recall that the geodesic flow $(\varphi_t)_{t \in \R}$ is \emph{Anosov} if there exists a flow-invariant continuous splitting 
\[
T(S\Sigma) = \R X \oplus E^s \oplus E^u
\]
and uniform constants $C,\lambda > 0$ such that
\begin{equation}
\label{equation:anosov}
\begin{array}{ll}
 |d\varphi_t(w)| \leq C e^{-\lambda t}|w|, & \qquad  \forall t \geq 0, \forall w \in E^s, \\
  |d\varphi_{-t}(w)| \leq C e^{-\lambda t}|w|, & \qquad  \forall t \geq 0, \forall w \in E^u.
\end{array}
\end{equation}
By \cite{Klingenberg-74} (see also \cite{Mane-87} for an alternative proof), the bundles $E^s$ and $E^u$ are known to intersect the vertical bundle $\V$ trivially, that is
\begin{equation}
\label{equation:disjoint0}
E^s \cap \V = E^u \cap \V = \{0\}.
\end{equation}
Moreover, there exists $C^{2-}(S\Sigma)$ functions $r_\pm$ solutions to the Ricatti equations
\[
X r_\pm + r_\pm^2 + \pi^*\kappa = 0,
\]
where $\kappa$ denotes the Gauss curvature on $(\Sigma,g)$, such that
\[
E^s = \R(H + r_- V), \qquad E^u = \R(H+r_+ V).
\]
If $\kappa < 0$, then $r_- < 0$ while $r_+ > 0$. We set $Y^s := H + r_- V, Y^u := H+r_+ V$. Note that the basis $\left\{X,Y^s,Y^u\right\}$ is positively-oriented.

\begin{center}
\begin{figure}[htbp!]
\includegraphics[scale=0.9]{figure.eps}
\caption{The various subbundles in tangent and cotangent space.}
\label{figure}
\end{figure}
\end{center} 



The dual bundles are defined by
\[
(E^0)^*(E^s \oplus E^u) = 0 = (E^s)^*(E^0\oplus E^s) = (E^u)^*(E^0 \oplus E^u),
\]
where $E^{0}:= \R X$. It can be checked that $(E^s)^*$ (resp. $(E^u)^*$) satisfies similar estimates \eqref{equation:anosov} to $E^s$ (resp. $E^u$) with $d\varphi_t$ being replaced by $d\varphi_t^{-\top}$ (inverse transpose). Equation \eqref{equation:disjoint0} becomes in the cotangent bundle
\begin{equation}
\label{equation:disjoint}
(E^s)^* \cap \HH^* = (E^u)^* \cap \HH^* = \{0\}.
\end{equation}



\subsubsection{Tensor tomography}

The tensor tomography problem consists in studying the transport equation $X u = f$, where $u, f \in C^\infty(S\Sigma)$ are smooth, $f$ has finite Fourier degree equal to $n$, and showing that $u$ has finite Fourier degree bounded by $\max(n-1,0)$.



\begin{theorem}[Tensor tomography]
\label{theorem:tensor}
Assume that $(\Sigma, g)$ is Anosov. If $X u = f$ with $f, u \in C^\infty(S\Sigma)$ and $f$ has degree $n$, then $u$ has degree $\max(n-1,0)$, and $u \equiv 0$ if $f$ has degree $0$.
\end{theorem}

Theorem \ref{theorem:tensor} was first proved in negative curvature in \cite{Guillemin-Kazhdan-80} and generalized to the Anosov setting in \cite{Dairbekov-Sharafutdinov-03} for $n=0,1$, \cite{Paternain-Salo-Uhlmann-14-2} for $n=2$, and \cite{Guillarmou-17-1} for $n \geq 3$. Note that Theorem \ref{theorem:tensor} is equivalent to the fact that the X-ray transform operator $I_n^g : C^\infty(\Sigma, S^n T^*\Sigma) \to \ell^\infty(\mc{C})$ defined by
\[
I^g_nh(c) := \dfrac{1}{\mc{L}_g(c)} \int_0^{\mc{L}_g(c)} h_{\gamma(t)}(\dot{\gamma}(t),...,\dot{\gamma}(t))~ \dd t
\]
is injective when restricted to divergence-free symmetric $n$-tensors, see \cite[Chapter 2]{Lefeuvre-thesis} for instance. For $n=2$, this operator is the linearization (at $g$) of the marked length spectrum operator $g \mapsto \mc{L}_g$.

\section{Proofs}

\label{section:proof}

In this section, we first prove that $\mc{L}_g$ determines the class $[J_g] \in \mc{M}(\Sigma)$ in the moduli space. Note that all the objects defined above (complex structures, stable and unstable bundles, etc.) now depend on the metric $g_1$ or $g_2$ and we shall add a subscript to distinguish them. The following is key to Theorem \ref{theorem:main}:

\begin{proposition}
\label{proposition:key}
Let $g_1,g_2$ be two smooth Anosov metrics on $\Sigma$. If $\mc{L}_{g_1} = \mc{L}_{g_2}$ then $[J_1]=  [J_2]$ in $\mc{M}(\Sigma)$. Equivalently, there exists an orientation-preserving diffeomorphism $\psi : \Sigma \to \Sigma$ such that $\psi^*J_2 = J_1$ and $\psi^* g_2 = e^{2 f} g_1$ for some $f \in C^\infty(\Sigma)$.
\end{proposition}

%Note that in the case where $\psi^*J_1 = -J_2$, $\psi$ is an orientation-reversing diffeomorphism of the oriented surface $\Sigma$.  

This should be compared with the case of a surface with boundary, see \cite{Pestov-Uhlmann-05,Guillarmou-17-2}. The proof of Proposition \ref{proposition:key} is the content of \S\ref{ssection:algebra} and \S\ref{ssection:period}. We then prove in Proposition \ref{proposition:key2} that we can recover from $\mc{L}_g$ the class $[J_g] \in \mc{T}(\Sigma)$ in Teichmüller space, and then the conformal factor of the metric (the latter is a standard argument due to Katok \cite{Katok-88}).


\subsection{The algebra isomorphism}


\label{ssection:algebra}





Consider an algebra homomorphism \[
\Phi : \mathcal A_+(S\Sigma_1) \longrightarrow \mathcal A_+(S\Sigma_2)
\]
with $\mathcal A_+(S\Sigma_i)$ defined by \eqref{DefA_+}. We say that it is \emph{proper} if $\Phi(\mathbf{1}_{S\Sigma_1}) = \mathbf{1}_{S\Sigma_2}$ (the constants are mapped to the constants) and if, when $f \in \mathcal A_+(S\Sigma_1)$ and $\deg(f) \geq 1$, then $\deg(\Phi(f)) \geq 1$. Equivalently,
\[
\Phi : \mathcal{A}_{\geq 1}(S\Sigma_1) \longrightarrow \mc{A}_{\geq 1}(S\Sigma_2)
\]
is a homomorphism. This definition also naturally extends for algebra homomorphisms $\Phi : \mathcal A_+(S\Sigma_1) \longrightarrow \mathcal A_-(S\Sigma_2)$ (in which case $\deg \geq 1$ is mapped isomorphically to $\deg \leq -1$).

There is a natural identification of $S\Sigma_1$ with $S\Sigma_2$ by simply scaling the fibers via the map
\[
s : S\Sigma_1 \longrightarrow S\Sigma_2, \qquad v \mapsto v/|v|_{g_2}.
\]
The following holds:

\begin{proposition}
\label{proposition:algebra-iso}
Assume that $\mc{L}_{g_1}=\mc{L}_{g_2}$. Then there exists a smooth diffeomorphism $\phi : S\Sigma_2 \to S\Sigma_1$ such that:
\begin{enumerate}[label=\emph{(\roman*)}]
\item $\phi \circ \varphi_t^{g_2} = \varphi_t^{g_1} \circ \phi$, for all $t \in \R$,
\item $s \circ \phi : S\Sigma_2 \to S\Sigma_2$ is isotopic to the identity,
\item $\phi$ induces a proper algebra isomorphism:
\[
\phi^* : \mathcal A_+(S\Sigma_1) \longrightarrow \mathcal A_\pm(S\Sigma_2),
\]
by pullback of distributions.
\end{enumerate}
\end{proposition}

At this stage, it is not yet clear that $\phi^*$ maps fiberwise holomorphic invariant distributions of the first metric to those of the second, and it could happen that $\phi^*$ interchanges fibrewise holomorphic with antiholomorphic. We will refer to these cases as the $(\pm)$ cases.



\begin{remark}
Actually, one can show that if $g_1$ and $g_2$ are connected by a path of metrics that are all Anosov, then $\phi^* : \mathcal A_+(S\Sigma_1) \longrightarrow \mathcal A_+(S\Sigma_2)$ preserves fiberwise holomorphic invariant distributions. This is based on the fact that weak unstable leaves admit a natural orientation and that such a $\phi$ would have to preserve this orientation. However, path-connectedness on the space of Anosov metrics on surfaces is an open question. Note that for negatively curved metrics, path-connectedness follows from the uniformization theorem and a straightforward computation (together with the fact that Teichm\"uller space is path-connected).
\end{remark}

We introduce $\mc{C} \subset \{(v,\xi)\in T^*(S\Sigma) \,|\, \xi(X(v))=0\}$, the closed cone enclosed by $(E^s)^*$ and $(E^u)^*$ in the half-space $\left\{ \xi(V(v)) \geq 0\right\}$, see Figure \ref{figure} for a description.

\begin{lemma}
\label{lemma:wf-c}
Let $f \in \mc{A}_\pm(S\Sigma)$. Then $\WF(f) \subset \pm \mc{C}$ and ${\pi_n}_* f \in C^\infty(\Sigma,K^{\otimes n})$ for all $n \in \Z$.
\end{lemma}

\begin{proof}
We only treat the $(+)$ case as the $(-)$ one is similar. Observe that $f \in \mc{A}_+(S\Sigma)$ is fiberwise holomorphic, i.e. $\mathbf{S}_+f=f$ so
\begin{equation}
\label{equation:wf}
\WF(f) = \WF(\mathbf{S}_+f) \subset \left\{(v,\xi) \in T^*(S\Sigma) ~|~ \xi(V(v)) \geq 0 \right\},
\end{equation}
by using the wavefront set description of the Schwartz kernel of $\mathbf{S}_+$, see \cite[Lemma 3.10]{Guillarmou-17-1}. Moreover, $X f = 0$ so by elliptic regularity one has
\[
\WF(f)\subset \left\{(v,\xi) \in T^*(S\Sigma) ~|~ \xi(X(v)) =0 \right\},
\]
while by standard propagation of singularities (see \cite[Theorem 26.1.1]{Hormander-4}) $\WF(f)$ is invariant by the symplectic lift of $(\varphi_t)_{t \in \R}$. But the maximal flow-invariant subset of $T^*(S\Sigma)$ contained in $\left\{ \xi(V(v)) \geq 0, \xi(X(v))=0 \right\}$ is $\mc{C}$, so this proves the claim. Finally, ${\pi_n}_* f \in C^\infty(\Sigma,K^{\otimes n})$ as the pushforward operator ${\pi_n}_*$ only selects the wavefront set of $f$ in $(E^0)^* \oplus \HH^*$ (see \cite[Proposition 11.3.3]{FriedlanderJoshi} for instance)  which is empty as $\mc{C} \cap (E^0)^* \oplus \HH^* = \left\{0\right\}$.
\end{proof}

We now prove Proposition \ref{proposition:algebra-iso}.


\begin{proof}[Proof of Proposition \ref{proposition:algebra-iso}] Two Anosov geodesic flows are orbit equivalent via a H\"older homeomorphism isotopic to the identity (see \cite{Ghys-84,Gromov-00} and \cite[Lemma B.1]{Guillarmou-Knieper-Lefeuvre-22}).  The equality of the marked length spectra $\mc{L}_{g_1}=\mc{L}_{g_2}$ 
and the Liv\v sic theorem imply that such orbit equivalence can be upgraded to a conjugacy $\phi$.
%By Anosov structural stability and the Liv\v sic theorem, the equality of the marked length spectra $\mc{L}_{g_1}=\mc{L}_{g_2}$ implies that the geodesic flows are H\"older conjugate by a H\"older homeomorphism $\phi : S\Sigma_2 \to S\Sigma_1$. In addition, the conjugacy can be constructed on the universal cover and is homotopic to the identity  (see \cite{Ghys-84,Gromov-00,Hamenstadt-93,Guillarmou-Knieper-Lefeuvre-22}). 
Moreover, such a conjugacy is necessarily smooth for three-dimensional contact Anosov flows: the $C^1$-regularity follows from \cite{Feldman-Ornstein-87} and the $C^\infty$ regularity from \cite{Delallave-Moriyon88} (see also \cite{Gogolev-RodriguezHertz-22} for a more general smoothness result on conjugacies of three-dimensional volume preserving Anosov flows).

%\todo[inline]{G: 1) the construction I know of the orbit equivalence is sometimes called Morse correspondence and I think Gromov wrote it down in general in his famous short note on "Three remarks..." This proof does not really need Busemann functions but just the notion of quasigeodesic as you essentially project. Ghys also wrote a similar proof in 80s to show that any Anosov flow on a Seifert 3-manifold is orbit equivalent to a geodesic flow.
%You guys wrote a detailed proof in Appendix B of your paper -which is great- but not quoting the earlier papers seems a bit strange... Gerhard also has a very clear proof in his 2002 survey.
%2) Should we mention here the recent preprint by Gogolyev and Rodriguez-Hertz to be on the politically correct side of things?}


% {\color{red}As $g_1$ and $g_2$ are connected by a path of Anosov metrics $(g_s)_{s \in [0,1]}$, we get by structural stability that there is an isotopy $(\phi_s)_{s \in [0,1]}$ such that $\phi_0 = \mathrm{id}$ and $\phi_1 = \phi$, and $\phi_s$ is an orbit-conjugacy between the geodesic flow of $g_2$ and that of $g_s$. This implies that $s \circ \phi$ is isotopic to the identity. XX This is probably not even needed to get that $\phi$ is iso to the identity ? XX}


As $\phi$ maps $E^s_{2}$ (resp. $E^u_2$) to $E^s_1$ (resp. $E^u_1$), we get that $\ker d\lambda_2 = E^s_2 \oplus E^u_2$ is mapped by $\phi$ to $\ker d\lambda_1$. Hence, $\phi^* \lambda_1 = \lambda_2$ and $\phi^*(-\lambda_1 \wedge \dd \lambda_1) = -\lambda_2 \wedge \dd \lambda_2$, that is, $\phi$ preserves the Liouville volume form hence the orientation.

As $\phi$ preserves the orientation, $\Sigma$ is connected, and $d \phi^{\top}(\lambda_1) = \lambda_2$, we deduce that $\mc{C}' := d\phi^{\top}(\mc{C}_{1}) = \pm \mc{C}_{2}$. Indeed, since $d\phi^\top$ maps connected sets to connected sets, $(E_1^u)^*$ to $(E^u_2)^*$ and $(E^s_1)^*$ to $(E^s_2)^*$, $d\phi^\top(\mc{C}_1)$ must be one of the 4 cones in the right Figure \ref{figure} (inside the subspace $\{\xi(X_2(v))=0\}$). Now, if $\mc{C}_{1}$ were mapped by $\phi$ to one of the other two cones on Figure \ref{figure}, $\phi$ would reverse the orientation, which is absurd.

We now claim that if $\mc{C}' = + \mc{C}_2$, then $\phi^* : \mc{A}_+(S\Sigma_1) \to \mc{A}_+(S\Sigma_2)$ is a proper algebra homomorphism, while if $\mc{C}' = - \mc{C}_2$, then $\phi^* : \mc{A}_+(S\Sigma_1) \to \mc{A}_-(S\Sigma_2)$ is a proper algebra homomorphism. Let $f \in \mathcal{A}_+(S\Sigma_1)$ and define $f' := \phi^*f \in \mc{D}'(S\Sigma_2)$. By Lemma \ref{lemma:wf-c}, the wavefront set of $f$ is contained in $\mc{C}_{1}$ so the wavefront set of $f'$ is contained in the cone $\mc{C}' = d\phi^{\top}(\mc{C}_{1}) = \pm \mc{C}_2$. 

%We now show that $\mc{C}' = - \mc{C}_{g_2}$ is impossible. {\color{red}Recall that the isotopy $(\phi_s)_{s \in [0,1]}$ is an orbit conjugacy between the geodesic flow of $g_2$ and that of $g_s$. In particular, it is a homeomorphism $S\Sigma_{g_2} \longrightarrow S\Sigma_{g_s}$ preserving the weak unstable foliation. Now, for a fixed metric $g$, the weak unstable leaves  can be oriented by choosing $\left\{X,Y_u\right\}$ to be an oriented basis. But then $\phi_s$ preserves this orientation (as it is isotopic to the identity), and thus so does $\phi = \phi_1$. Since $d\phi(X_{g_2}) = X_{g_1} \circ \phi$, this implies that $d\phi$ maps $Y_u^{g_2}$ to a positive multiple of $Y_u^{g_1}$ and so $\mc{C}' = - \mc{C}_{g_2}$ is impossible.}

The proof being similar in the $(-)$ case, we will now assume that $\mc{C}'=+\mc{C}_2$. Then $\mathbf{S}_- f' \in C^\infty(S\Sigma_2)$ and, since $X_2f'=\phi^*(X_1f)=0$, we also have that $X_{2}(\mathbf{S}_- f') = \eta_+ f'_0 + \eta_+ f'_{-1}$. By the tensor tomography Theorem \ref{theorem:tensor}, we deduce that $\mathbf{S}_- f'$ is of degree $0$. Hence $f'$ is fiberwise holomorphic. It then remains to show that $\phi^* : \mc{A}_+(S\Sigma_1) \longrightarrow \mc{A}_+(S\Sigma_2)$ is an algebra isomorphism but this is straightforward as it admits an inverse $(\phi^{-1})^*$ (and the product property is obviously satisfied). We now show that $\phi^*$ is proper. Obviously, $\phi^*(\mathbf{1}_{S\Sigma_1}) = \mathbf{1}_{S\Sigma_2}$. Now, assume that $f \in \mc{A}_+(S\Sigma_1)$ has $\deg(f) \geq 1$ and consider $f' := \phi^*f$. The term of degree $0$ of $f' = \sum_{k \geq 0} f'_k$ is a constant which can be computed as follows: 
\[
\begin{split}
f_0' \times \vol(S\Sigma_2)& = \int_{S\Sigma_2} f' \mu_2 = \int_{S\Sigma_2} \phi^*(f \mu_1) = \int_{S\Sigma_1} f \mu_1 = f_0 \times \vol(S\Sigma_1) = 0.
\end{split}
\]
This proves the claim. The $(-)$ case is treated similarly.
\end{proof}

\subsection{Extension operator. Period preservation}


\label{ssection:period}

The aim of this paragraph is to show that, in some sense, the algebra isomorphism $\phi^*$ maps holomorphic differentials of the first surface to the second, and that it preserves periods.

We first claim that the map
\begin{equation}
\label{equation:surj-pi1}
{\pi_1}_* : \mc{A}_{\geq 1}(S\Sigma)  \longrightarrow H_J(\Sigma, K)
\end{equation}
is well-defined. Indeed, a fiberwise holomorphic invariant distribution $f \in \mc{A}_{\geq 1}(S\Sigma)$ satisfies $Xf = 0$ so its first Fourier mode $f_1$ satisfies $\eta_- f_1 = 0$ which, in turn, is equivalent to $\bar{\partial} {\pi_1}_* f = 0$ by \S\ref{sssection:powers}.

A key input that we now use is that the map \eqref{equation:surj-pi1} is known to be surjective by \cite[Theorem 1.5]{Paternain-Salo-Uhlmann-14-2}. This allows to define a right-inverse $\mathbf{e}_1$ such that
\[\e_1 : H^0_J(\Sigma,K) \longrightarrow \mc{A}_{\geq 1}(S\Sigma) , \quad {\pi_1}_* \circ \e_1 = \mathbf{1}_{H^0_J(\Sigma,K)},\]
which we call the \emph{extension operator}. 

\begin{remark}
A similar right-inverse $\mathbf{e}_{-1}$ can also be defined for ${\pi_{-1}}_* :  \mc{A}_{\leq -1}(S\Sigma)  \longrightarrow H_J(\Sigma, K^{-1})$. More generally, using the spectral theory of Anosov flows, it was proved in \cite[Corollary 3.8]{Guillarmou-17-1} that the map ${\pi_n}_* : \mc{A}_{\geq n}(S\Sigma)  \longrightarrow H_J(\Sigma, K^n)$ is surjective for all $n \geq 0$ (along with other more general results on the existence of flow-invariant distributions with prescribed $n$-th Fourier mode). However, this will not be needed in the rest of the paper.
\end{remark}
%
%We now define for $f \in H^0_J(\Sigma,K)$, the \emph{extension} operator
%\begin{equation}
%\label{equation:en}
%\e_1 f := \mathbf{S}_+ \mathbf{S}_{\mathrm{odd}} \Pi \pi_1^* \Pi_{1}^{-1} f,
%\end{equation}
%wher $\mathbf{S}_{\text{odd}}$ denotes the $L^2$-orthogonal projection onto odd Fourier modes. We claim that \eqref{equation:en} produces a flow-invariant odd distribution $u := \e_1 f \in \mc{D}'(S\Sigma)$ of prescribed first Fourier mode given by $u_1 = \tfrac{1}{2\pi} \pi_1^* f$.



%Note that $\Pi_n$ is an isomorphism on the $L^2$-orthogonal to its kernel, that is,
%\[
%\Pi_n : C^\infty(M,K^{\otimes n}) \cap \left(\eta_+ H^0(M, K^{\otimes (n-1)})\right)^{\perp} \righttoleftarrow
%\]
%is an isomorphism since $\Pi_n$ is formally self-adjoint. Hence, as $f \in H^0(M,K^{\otimes n})$ belongs to $\left(\eta_+ H^0(M, K^{\otimes (n-1)})\right)^{\perp}$, $\Pi_n^{-1}f$ is well-defined.


%\begin{lemma}
%\label{lemma:extension} 
%The following holds true: 
%\[\e_1 : H^0_J(\Sigma,K) \longrightarrow \mc{A}_{\geq 1}(S\Sigma) , \quad {\pi_1}_* \circ \e_1 = \mathbf{1}_{H^0_J(\Sigma,K)}.\]
%\end{lemma}
%
%\begin{proof}
%By construction, the distribution $u := \Pi  \pi_1^* \Pi_{1}^{-1} f \in \mc{D}'(S\Sigma)$ satisfies $X u = 0$ and ${\pi_1}_* u = f$. The same holds for $u_{\mathrm{odd}} := \mathbf{S}_{\mathrm{odd}} u$ as $X$ preserves the decomposition into odd/even Fourier modes. Moreover, $\eta_-  (u_{\mathrm{odd}})_1 = 0$ because $f \in H^0_J(\Sigma,K)$ and thus $X \mathbf{S}_+ u_{\mathrm{odd}} = 0$. This also implies that for $k=0,1$, $\eta_+ (u_{\mathrm{odd}})_k = 0$. But by Lemma \ref{lemma:rr}, $\eta_+$ is injective for $k > 0$ so $(u_{\mathrm{odd}})_1 = 0$. Moreover, $(u_{\mathrm{odd}})_0 = 0$ because the range of $\Pi$ is orthogonal to the constants.
%\end{proof}
%
%\todo[inline]{G: this lemma should go, but for future reference, I can't make sense as to why the last two sentences are needed...}


Let 
\[
F : H^0_{J_1}(\Sigma, K) \longrightarrow H^0_{\pm J_2}(\Sigma, K)
\]
be the map defined by one of the two following commutative diagrams:
\begin{enumerate}[label=(\roman*)]
\item In the $(+)$ case,
\[
\xymatrix{
    \mc{A}_{\geq 1}(S\Sigma_{1})  \ar[r]^{\phi^*}  & \mc{A}_{\geq 1}(S\Sigma_{2}) \ar[d]_{{\pi_1}_*} \\
    H^0_{J_1}(\Sigma, K) \ar[u]_{\e_1} \ar[r]^F & H^0_{J_2}(\Sigma, K)
  }
\]
\item In the $(-)$ case,
\[
\xymatrix{
    \mc{A}_{\geq 1}(S\Sigma_{1})  \ar[r]^{\phi^*}  & \mc{A}_{\leq -1}(S\Sigma_{2}) \ar[d]_{{\pi_{-1}}_*} \\
    H^0_{J_1}(\Sigma, K) \ar[u]_{\e_1} \ar[r]^F & H^0_{-J_2}(\Sigma, K)
  }
\]
\end{enumerate}
Note that, in the $(-)$ case, if $f \in \mc{A}_-(S\Sigma_{2})$, then ${\pi_{-1}}_* f \in C^\infty(\Sigma,K_{J_2}^{-1})$ is a $(0,1)$-form for the complex structure $J_2$ satisfying $\partial_{J_2} f = 0$, and it can be naturally identified with a $(1,0)$-form $f \in C^\infty(\Sigma, K_{-J_2})$ for the complex structure $-J_2$ satisfying $\bar{\partial}_{-J_2} f = 0$.

The following result unlocks Theorem \ref{theorem:main}:

\begin{proposition}
\label{proposition:period}
The map
\[
F : H^0_{J_1}(\Sigma, K) \longrightarrow H^0_{\pm J_2}(\Sigma, K)
\]
 is a period-preserving $\C$-linear isomorphism in the following sense: for all $[\gamma] \in H_1(\Sigma,\Z)$, $\omega \in  H^0_{J_1}(\Sigma, K) $,
 \begin{equation}
 \label{equation:period}
 \int_{[\gamma]} \omega = \int_{[\gamma]} F \omega.
 \end{equation}
\end{proposition}


That $F$ is an isomorphism follows immediately from the period preservation \eqref{equation:period}. Hence, the only non-trivial part of Proposition \ref{proposition:period} is \eqref{equation:period}. Here and below, given a homology class $[\gamma] \in H_1(\Sigma,\Z)$, we shall denote by $\gamma_{g_i}$ a closed oriented geodesic (for the metric $g_i, i=1,2$) representing $[\gamma]$, and $\gamma$, when no preferred metric is chosen. We then set $S^1\gamma \subset S\Sigma$ to be the circle bundle over $\gamma$. If $[0,\ell_g(\gamma)] \ni \tau \mapsto \gamma(\tau)$ is an arc-length parametrization of $\gamma$, we can parametrize $S^1\gamma$ by coordinates $(\tau,\theta) \in \R/(\ell_g(\gamma)\Z) \times \R/(2\pi\Z)$ such that:
\[
S^1\gamma = \left\{ (\gamma(\tau), R_\theta \dot{\gamma}(\tau)) ~|~ \tau \in [0,\ell_g(\gamma)], \theta \in [0,2\pi]\right\}
\]
where $R_\theta$ denotes rotation by angle $\theta$ in $S\Sigma$.
Note that the tangent space of $S^1\gamma$ at the point $(\tau,\theta)$ is spanned by $((\cos \theta) X + (\sin \theta) H, V)$. We then endow $S^1\gamma$ with the orientation given by $(\cos \theta~ \lambda + \sin \theta~ \beta) \wedge \psi$. From now on, $S^1\gamma$ will always denote the \emph{oriented} submanifold of $S\Sigma$ with orientation given by the $2$-form above.
The following holds:

\begin{lemma}
In the coordinate system $(\tau,\theta)$, $d\lambda = \sin \theta~ \dd\theta \wedge \dd\tau$.
\end{lemma}

Note that for $\theta \in (0,\pi)$, this is a \emph{negative} multiple of the $2$-form defining the orientation on $S^1\gamma$.

\begin{proof}
This follows from \eqref{equation:dlambda}, see also \cite[p. 156]{Otal-90}.
\end{proof}


%Note that, any surface admits a finite cover which is not hyperelliptic. Hence, in the argument for the proof of the MLS result, up to passing to a finite cover, we will always be able to apply this lemma.
%
%\begin{proof}
%It suffices to show that $\deg(\Phi(f)) \geq \deg(f)$ since $\Phi^{-1}$ will satisfy the same property (it is also proper) and thus
%\[
%\deg(f) = \deg(\Phi^{-1} \Phi f) \geq \deg(\Phi f) \geq \deg(f).
%\]
%Assume that there is $f \in  \mc{A}_\pol(Z_1)$ such that $1 \leq k_0 := \deg(\Phi f) < \deg f =: k_1$. We write $T(\Phi f) = \sum_{k \geq k_0} u'_k$. Note that $\eta_-(u'_{k_0}) = 0$ is holomorphic. By hyperellipticity, we can find $a_1, ..., a_{k_0} \in H^0(M_2,K)$ such that $u'_{k_0} = a_1 ... a_{k_0}$. Let $A_1^{(1)}, ..., A_{k_0}^{(1)} \in \mc{D}'(SM_2)$ such that $X A_i = 0$, $A_i$ is fiberwise holomorphic with modes of degree $\geq 1$ and $(A_i^{(1)})_1 = a_1$ (possible by tensor tomography for Anosov surfaces). Then $\omega_1 := u'-A_1^{(1)}...A_{k_0}^{(1)}$ is fiberwise holomorphic, $X \omega_1 = 0$ and $\omega_1$ has degree $\geq k_0 + 1$. We can iterate this construction further until we write
%\[
%u' = \sum_{j=1}^{k_1-k_0} A_1^{(j)}... A_{k_0} ... A_{k_0+j-1}^{(j)} + \omega_j,
%\]
%where $\omega_j$ has degree $\geq k_1$. But then
%\[
%\psi^* u = u = \sum_{j=1}^{k_1-k_0} \psi^*A_1^{(j)}... \psi^* A_{k_0} ... \psi^* A_{k_0+j-1}^{(j)} + \psi^* \omega_j
%\]
%\end{proof}

We next derive the following remarkable formula:

\begin{lemma}
\label{lemma:magical}
Let $f \in \mc{A}_{\pm}(S\Sigma)$ and $\gamma$ be a closed oriented geodesic. Then the pairing $(S^1\gamma, f d\lambda)$ is well-defined and
\begin{equation}
\label{equation:magical}
\pm 2i\int_{S^1\gamma} f~ \dd\lambda =  \int_{[\gamma]} {\pi_{\pm 1}}_* f,
\end{equation}
where $[\gamma] \in H_1(\Sigma,\Z)$ denotes the homology class of the geodesic $\gamma$.
\end{lemma}

Actually, if $f \in C^\infty(S\Sigma)$ (or if $f$ is merely a distribution but the pairing with $S^1\gamma$ makes sense), we will prove the formula
\begin{equation}
\label{equation:smooth}
\frac{i}{\pi} \int_{S^1\gamma} f~ \dd\lambda =  \int_0^{\ell_g(\gamma)} f_1(\gamma(\tau),\dot{\gamma}(\tau)) \dd \tau -\int_0^{\ell_g(\gamma)} f_{-1}(\gamma(\tau),\dot{\gamma}(\tau)) \dd \tau,
\end{equation}
where $\tau \mapsto \gamma(\tau)$ is an arc-length parametrization of the geodesic. It then suffices to observe that \eqref{equation:magical} is a mere rewriting of \eqref{equation:smooth} since for $f \in \mc{A}_{\pm}(S\Sigma)$, ${\pi_{\pm 1}}_*f$ is an (anti-)holomorphic differential (and the factor $2\pi$ comes from \eqref{equation:proj}).

\begin{remark}[Generalized intersection number]
\label{remark:number}
The pairing $(S^1\gamma, f\dd\lambda)$ should be thought of as a generalized \emph{intersection number}, following the terminology used by Bonahon \cite{Bonahon-88}, Otal \cite{Otal-90} and others. The difference is that we allow $f$ to be \emph{any} fiberwise holomorphic invariant distribution in $\mc{A}_{+}(S\Sigma)$ whereas, in some sense, the aforementioned authors only allowed $f$ to be a constant, which is a (very) particular case of a fiberwise holomorphic distribution. Moreover, we integrate over the full circle bundle $S^1\gamma$ whereas these authors integrate over $S^1\gamma_+$, that is, for angles $\theta \in (0,\pi)$. The usual intersection number of the \emph{Liouville current} $\eta$ (that is, the Liouville form modded out by the flow direction in the universal cover) with the geodesic $\gamma$ is commonly denoted by $\iota(\eta,\gamma)$ and is equal to $2\ell_g(\gamma)$ (or $\ell_g(\gamma)$, depending on the convention). It simply corresponds for us to $(S^1\gamma_+,d\lambda)$.\end{remark}

\begin{proof}[Proof of Lemma \ref{lemma:magical}]
By the wavefront set calculus (see \cite[Corollary 8.2.7]{Hormander-90} for instance), the pairing $(S^1\gamma,f \dd\lambda)$ is well-defined as long as
\begin{equation}
\label{equation:inter}
N^*(S^1\gamma) \cap \WF(f \dd\lambda) = \emptyset.
\end{equation}
Now, $N^*(S^1\gamma)$ is a line contained in $(E^0)^* \oplus \HH^*$ and by Lemma \ref{lemma:wf-c} and \eqref{equation:disjoint}, the intersection \eqref{equation:inter} is indeed empty. Hence, the pairing $(S^1\gamma,f \dd\lambda)$ is well-defined and extends the pairing computed for $f \in C^\infty(S\Sigma)$. As a consequence, it suffices to establish \eqref{equation:smooth} for smooth functions and \eqref{equation:magical} then follows immediately.

Now, if $f \in C^\infty(S\Sigma)$, we can decompose $f$ in Fourier modes along $S^1\gamma$ and write
\begin{equation}
\label{equation:decomp2}
f(\theta,\tau) = \sum_{k \in \Z} a_k(\tau) e^{ik \theta}.
\end{equation}
Then, using \eqref{equation:decomp2} and $\sin \theta = (2i)^{-1}(e^{i \theta} - e^{-i \theta})$, we get:
\[
\begin{split}
\frac{i}{\pi}  \int_{S^1\gamma} f~ d\lambda &= - \frac{i}{\pi}  \int_0^{\ell_g(\gamma)} \int_0^{2\pi} f(\theta,\tau) \sin \theta~ \dd \theta \dd \tau   = \int_0^{\ell_g(\gamma)} a_1(\tau) \dd \tau -\int_0^{\ell_g(\gamma)} a_{-1}(\tau) \dd \tau.
\end{split}
\]
Note that the minus on the first line comes from the fact that $f\dd\lambda$ is a negative multiple of the $2$-form defining the orientation on $S^1\gamma$. The previous equality corresponds exactly to \eqref{equation:smooth}.
\end{proof}

If $\omega \in H^0_J(\Sigma,K)$, $[\gamma] \in H_1(\Sigma,\Z)$ and $\gamma$ is any closed geodesic representative for the metric $g$ whose homology class is $[\gamma]$, Lemma \ref{lemma:magical} yields
\begin{equation}
\label{equation:integration}
2 i  \int_{S^1\gamma} \mathbf{e}_1 \omega~\dd\lambda = \int_{[\gamma]} \omega.
\end{equation}
%
%\begin{proof}
%Indeed:
%\[
%\begin{split}
%\dfrac{i}{\pi} \int_{S^1\gamma} \mathbf{e}_1 \omega~\dd\lambda & = - \int_0^{\ell_g(\gamma)} (\mathbf{e}_1 \omega)_1(\gamma(\tau),\dot{\gamma}(\tau)) \dd\tau \\
%&  =- \dfrac{1}{2\pi} \int_{0}^{\ell_g(\gamma)} \pi_1^* \omega(\gamma(\tau),\dot{\gamma}(\tau)) \dd \tau = - \dfrac{1}{2\pi} \int_{[\gamma]} \omega,
%\end{split}
%\]
%where $\tau \mapsto \gamma(\tau)$ is an arc-length parametrization of the geodesic.
%\end{proof}
We can now complete the proof of Proposition \ref{proposition:period}.

\begin{proof}[Proof of Proposition \ref{proposition:period}]
We first treat the $(+)$ case. Let $[\gamma] \in H_1(\Sigma,\Z)$ and let $\gamma_{g_1}(c), \gamma_{g_2}(c)$ be two geodesic representatives for $[\gamma]$ (with respect to $g_1$ and $g_2$) in the same free homotopy class $c \in \mc{C}$. Since $\phi$ is isotopic to the identity by Proposition \ref{proposition:algebra-iso}, we get that $[\phi(S^1\gamma_{g_2}(c))] = [S^1\gamma_{g_1}(c)]$ in $H_2(S\Sigma_1,\Z)$. 

We also claim that the pairing $(\phi(S^1\gamma_{g_2}(c)), \mathbf{e}_1 \omega~\dd\lambda_1)$ is well-defined, similarly to Lemma \ref{lemma:magical}. Indeed, the tangent space to $S^1\gamma_{g_2}(c)$  does not intersect the closed cone $\mc{B}_2 \subset \left\{(v,w) \in T(S\Sigma_2) ~|~ \la_2(w)=0\right\}$ enclosed by $E_2^s$ and $E_2^u$, and containing $\HH_2$, see Figure \ref{figure}. Since $\phi$ preserves the orientation, following the same arguments as in Proposition \ref{proposition:algebra-iso}, the tangent space to $\phi(S^1\gamma_{g_2}(c))$ is also a $2$-plane avoiding the closed cone $\mc{B}_1$ and thus its conormal avoids the closed cone $\mc{C}_1$ which contains $\WF(\mathbf{e}_1 \omega~\dd\lambda_1)$ by Lemma \ref{lemma:wf-c}, so the pairing is well-defined.

Moreover, given a fiberwise holomorphic invariant distribution $f \in \mc{A}_+(S\Sigma_1)$, observe that $f \dd\lambda$ is a closed $2$-form. We claim that
\[
\int_{S^1\gamma_{g_1}(c)} \mathbf{e}_1 \omega~\dd\lambda_1= \int_{\phi(S^1\gamma_{g_2}(c))} \mathbf{e}_1 \omega~\dd\lambda_1.
\]
Indeed, choose the harmonic representative $\alpha \in \mathscr{H}^2(S\Sigma_1)$ of $f \dd\lambda$ that is, such that $[\alpha] = [f\dd\lambda]$ in $H^2(S\Sigma_1,\C)$. By the Hodge decomposition Theorem, we have $\alpha = f \dd \lambda + d u$, where $u \in \mc{D}'(S\Sigma_1)$ and $\WF(u) = \WF(f)$. Hence, both pairings $(S^1\gamma_{g_1}(c), du)$ and $(\phi(S^1\gamma_{g_2}(c)), du)$ are well-defined by the wavefront set calculus (same arguments as for $f\dd\lambda$) and equal to $0$ since $du$ is exact.

Then, we get:
\[
\begin{split}
\int_{S^1\gamma_{g_1}(c)} \mathbf{e}_1 \omega~\dd\lambda_1 & = \int_{S^1\gamma_{g_1}(c)} \alpha =  \int_{\phi(S^1\gamma_{g_2}(c))} \alpha = \int_{\phi(S^1\gamma_{g_2}(c))} \mathbf{e}_1 \omega~\dd\lambda_1,
\end{split}
\]
where the second equality simply follows from the fact that $\alpha$ is harmonic and $[S^1\gamma_{g_1}(c)] = [\phi(S^1\gamma_{g_2}(c))]$ in $H_2(S\Sigma_1,\Z)$.

As a consequence, given $\omega \in H^0_{J_1}(\Sigma_1,K)$ we get by \eqref{equation:integration} and the discussion above that
\[
\begin{split}
\int_{[\gamma]} \omega &= 2i \int_{S^1\gamma_{g_1}(c)} \mathbf{e}_1 \omega~\dd\lambda_1 \\
&  = 2i \int_{\phi(S^1\gamma_{g_2}(c))} \mathbf{e}_1 \omega~\dd\lambda_1 = 2i\int_{S^1\gamma_{g_2}(c)} \phi^*(\mathbf{e}_1 \omega~\dd\lambda_1) \\
& = 2i \int_{S^1\gamma_{g_2}(c)} \phi^*(\mathbf{e}_1\omega) ~\dd \lambda_2 = \int_{[\gamma]} {\pi_1}_*  \phi^*(\mathbf{e}_1\omega) = \int_{[\gamma]} F\omega.
\end{split}
\]
This concludes the proof in the $(+)$ case.

In the $(-)$ case, it should be observed that $\phi$ actually flips the orientation of $[S^1_{\gamma_{g_2}}(c)]$, that is $[\phi(S^1_{\gamma_{g_2}}(c))] = - [S^1\gamma_{g_1}(c)]$ in $H_2(S\Sigma_1,\Z)$, which gives:
\[
\begin{split}
\int_{[\gamma]} \omega &= 2i \int_{S^1\gamma_{g_1}(c)} \mathbf{e}_1 \omega~\dd\lambda_1 \\
&  = -2i \int_{\phi(S^1\gamma_{g_2}(c))} \mathbf{e}_1 \omega~\dd\lambda_1 = -2i\int_{S^1\gamma_{g_2}(c)} \phi^*(\mathbf{e}_1 \omega~\dd\lambda_1) \\
& = -2i \int_{S^1\gamma_{g_2}(c)} \phi^*(\mathbf{e}_1\omega) ~\dd \lambda_2 = \int_{[\gamma]} {\pi_{-1}}_*  \phi^*(\mathbf{e}_1\omega) = \int_{[\gamma]} F\omega,
\end{split}
\]
by \eqref{equation:magical} again.
\end{proof}
 

We can now conclude the proof of Proposition \ref{proposition:key}.

\begin{proof}[Proof of Proposition \ref{proposition:key}]
If $\mc{L}_{g_1} = \mc{L}_{g_2}$, we get by the above results that
\[
F : H^0_{J_1}(\Sigma,K) \longrightarrow H^0_{\pm J_2}(\Sigma,K)
\]
is a period-preserving $\C$-linear complex isomorphism. Hence, by the Torelli Theorem \ref{theorem:torelli}, we get that $(\Sigma,J_1)$ and $(\Sigma,\pm J_2)$ have same period matrix, so they are biholomorphic and differ in the Teichm\"uller space by an element $[\psi] \in \mathrm{MCG}(\Sigma)$. Note that in the $(-)$ case, $[\psi]$ would reverse the orientation which is absurd as $[\psi] \in \mathrm{MCG}(\Sigma)$, hence only the $(+)$ case survives which is the content of Proposition \ref{proposition:key}.
\end{proof}



\subsection{End of the proof}

We now prove that $\mc{L}_g$ determines the class of $[J_g] \in \mc{T}(\Sigma)$ in the Teichmüller space of $\Sigma$.


\begin{proposition}
\label{proposition:key2}
Let $g_1,g_2$ be two smooth Anosov metrics. If $\mc{L}_{g_1} = \mc{L}_{g_2}$ then $[J_{g_1}]=[J_{g_2}]$ in $\mc{T}(\Sigma)$. Equivalently, there exists a diffeomorphism $\psi : \Sigma \to \Sigma$ isotopic to the identity such that $\psi^* g_2 = e^{2 f} g_1$ for some $f \in C^\infty(\Sigma)$.
\end{proposition}

Proposition \ref{proposition:key2} will follow from Proposition \ref{proposition:key} and the following:

\begin{lemma}
\label{lemma:cover}
Let $J_1$ and $J_2$ be two complex structures on $\Sigma$ compatible with orientation such that $[J_1] \neq [J_2]$ in $\mc{T}(\Sigma)$. Then, there exists a finite cover $\Sigma' \to \Sigma$ such that the lifts $[J_1'], [J_2'] \in \mc{T}(\Sigma')$ are not in the same $\mathrm{MCG}(\Sigma')$-orbit.
\end{lemma}

Let us first derive Proposition \ref{proposition:key2} from that.

\begin{proof}[Proof of Proposition \ref{proposition:key2}]
If $\mc{L}_{g_1} = \mc{L}_{g_2}$ on $\Sigma$, then the same equality $\mc{L}_{g'_1} = \mc{L}_{g'_2}$ holds on any finite cover $\Sigma' \to \Sigma$. Thus, the desired conclusion follows by combining directly Proposition \ref{proposition:key} with Lemma \ref{lemma:cover}.
\end{proof}

The proof of Lemma \ref{lemma:cover} was kindly explained to us by M. Wolff.

\begin{proof}[Proof of Lemma \ref{lemma:cover}]
In order to simplify notation, given a free homotopy class $c\in\mc{C}$ we write $\ell_g(c)$ for the length of the unique $g$-geodesic in the free homotopy class $c$. We use the identification of Teichmüller space $\mc{T}(\Sigma)$ with marked hyperbolic structures on $\Sigma$. Let $[h_1], [h_2] \in \mc{T}(\Sigma)$ be the (distinct) hyperbolic structures given by $J_1$ and $J_2$. Let $\alpha \in \mc{C}(\Sigma)$ be a free homotopy class of simple closed curves on $\Sigma$ such that $\ell_{h_1}(\alpha) \neq \ell_{h_2}(\alpha)$. Note that such a class always exists otherwise we would have $[h_1]=[h_2]$ by the $9g-9$ Theorem (see \cite[Theorem 10.7]{Farb-Margalit-11}).

We then consider a finite cover $\Sigma' \to \Sigma$ which unfolds every simple closed curve, except $\alpha$. More precisely, for $K > 0$ fixed, we consider a finite cover $\pi : \Sigma' \to \Sigma$ such that the following holds: denoting by $h_1', h_2'$ the lifts of $h_1,h_2$, there exists a free homotopy class of simple closed curves $\alpha' \in \mc{C}(\Sigma')$ on $\Sigma'$ such that $\pi : \gamma_{h'_1}(\alpha') \to \gamma_{h_1}(\alpha)$ is $1$-to-$1$, and for every free homotopy class of simple closed curves $\beta \in \mc{C}(\Sigma')$ such that $\beta \neq \alpha'$ , $\ell_{h'_1}(\beta) > K$ (see \cite[Appendix A, Theorem C]{Alvarez-Brum-Martinez-Potrie-19} for the existence of such a covering). In particular, choosing $K > 0$ such that $K > \ell_{h_1}(\alpha)$, we can guarantee that $\gamma_{h'_1}(\alpha')$ is the systole of $(\Sigma',h_1')$ with length $\ell_{h_1}(\alpha)$.

Now, as $\Sigma$ is compact, there exists a constant $C > 1$ such that for all $c \in \mc{C}$ (not necessarily simple),
\[
1/C \leq \dfrac{\ell_{h_1}(c)}{\ell_{h_2}(c)} \leq C.
\]
We fix $K > 0$ in the construction above such that $K/C > \max(\ell_{h_1}(\alpha),\ell_{h_2}(\alpha))$. We claim that $\gamma_{h'_2}(\alpha')$ is also the systole of $(\Sigma',h_2')$ with length $\ell_{h_2}(\alpha)$. Indeed, observe first that $\pi : \gamma_{h'_2}(\alpha') \to \gamma_{h_2}(\alpha)$ is also $1$-to-$1$ (so $\ell_{h_2'}(\alpha') = \ell_{h_2}(\alpha)$), and for every free homotopy class of simple closed curves $\beta \in \mc{C}(\Sigma'), \beta \neq \alpha'$, one has:
\[
\begin{split}
\ell_{h_2'}(\beta) & =\ell_{h_1'}(\beta) \cdot \dfrac{\ell_{h_2'}(\beta)}{\ell_{h_1'}(\beta)} \\
& = \ell_{h_1'}(\beta) \cdot \dfrac{\ell_{h_2}(\pi(\beta))}{\ell_{h_1}(\pi(\beta))} > K/C > \max(\ell_{h_1}(\alpha),\ell_{h_2}(\alpha)) \geq \ell_{h_2'}(\alpha').
\end{split}
\]
As a consequence, the systoles of $(\Sigma',h_1')$ and $(\Sigma',h_2')$ have different lengths, so $[h_1']$ and $[h_2']$ are not in the same $\mathrm{MCG}(\Sigma')$-orbit.
\end{proof}


Finally, we complete the proof of Theorem \ref{theorem:main}.

\begin{proof}[Proof of Theorem \ref{theorem:main}]
If $\mc{L}_{g_1}=\mc{L}_{g_2}$, Proposition \ref{proposition:key2} yields that there exists $\psi : \Sigma \to \Sigma$ isotopic to the identity such that $\psi^*g_1 = e^{2f} g_2$. But then, applying Katok's argument \cite{Katok-88} for conformal metrics with same marked length spectrum, we get that $f=0$. This concludes the proof.
\end{proof}




%\section{A conjecture}
%
%{\color{red}explain the Max Noether transport Conjecture?}
%
%
%\begin{conjecture}
%$\mc{A}_{\geq 1}(S\Sigma)$ is generated by elements of degree $1$.
%\end{conjecture}

%
%\section{A conjecture}
%
%
%\subsection{A transport Max Noether Theorem}
%
%
%
%Let $a, b \in H^0(M,K)$ with no common zeroes. We start with the following lemma:
%
%\begin{lemma}
%\label{lemma:surj}
%For all $n \geq 3$, the map
%\[
%\Theta_n : H^0(M,K^{\otimes n}) \oplus H^0(M,K^{\otimes n}) \to H^0(M,K^{\otimes (n+1)}), \qquad (u,v) \mapsto au + bv,
%\]
%is surjective.
%\end{lemma}
%
%\begin{proof}
%Actually, we have the short exact sequence
%\[
%0 \to H^0(M, K^{\otimes (n-1)}) \to  H^0(M,K^{\otimes n}) \oplus H^0(M,K^{\otimes n}) \to H^0(M,K^{\otimes (n+1)}) \to 0,
%\]
%with first arrow given by 
%\[
%H^0(M, K^{\otimes (n-1)}) \ni \omega \mapsto (b \omega, -a \omega) \in H^0(M,K^{\otimes n}) \oplus H^0(M,K^{\otimes n}).
%\]
%Indeed, assume that $a u + b  v = 0$. Then, if $z_0 \in M$ is a zero of degree $k$ of $a$, it is a zero of degree $\geq k$ of $v$ so $\omega := v/a \in H^0(M,K^{\otimes (n-1)})$ is well-defined and holomorphic. Similarly, $\omega' := u/b$ is well-defined and it is immediate that $\omega' = -\omega$. Hence $\ker \Theta_n \cong H^0(M,K^{\otimes (n-1)})$.
%
%The rest of the proof is just a mere dimension counting argument since
%\[
%\begin{split}
%2 \dim H^0(M,K^{\otimes n}) &- \dim H^0(M, K^{\otimes (n-1)}) \\
%& = 2(2n-1)(g-1)-(2n-3)(g-1) \\
%& =(2n+1)(g-1) = \dim H^0(M, K^{\otimes (n+1)}).
%\end{split}
%\]
%\end{proof}
%
%
%Define
%\[
%A := [\Pi \pi_1^* \Pi_1^{-1} a]_{\geq 1}, B :=  [\Pi \pi_1^* \Pi_1^{-1} b]_{\geq 1}.
%\]
%By construction, $A, B \in \mc{A}(SM)$, $\deg(A)=\deg(B)=1$ and $A_1 = a, B_1=b$. We would like to show the following:
%
%\begin{lemma}
%\label{lemma:noether}
%Let $W \in \mc{A}(SM)$ of degree $\geq 2$. There exists (?) $U, V \in \mc{A}(SM)$ of degree $\geq 1$ such that
%\[
%W = AU + BV.
%\]
%\end{lemma}
%
%Using Lemma \ref{lemma:surj}, it is not hard to prove Lemma \ref{lemma:noether} with $W = A U+BV$ with $U, V \in \mc{C}'(SM) \cap \ker X$ (dual of frequentially compactly supported functions on $SM$) but it is not clear that $U, V \in \mc{D}'(SM)$ (for that, we would need to have that the Fourier modes of $U,V$ grow at least polynomially).
%
%\subsection{Degree preservation} In the following, \textbf{we assume that Lemma \ref{lemma:noether} holds.} The following holds:
%
%\begin{lemma}
%\label{lemma:degree-increasing}
%Any proper algebra isomorphism $\Phi :  \mathcal A(SM_1) \xrightarrow{\sim} \mathcal A(SM_2)$ preserves the degree, that is $\deg(\Phi(f)) = \deg(f)$.
%\end{lemma}
%
%\begin{proof}
%It suffices to show that $\deg(\Phi(f)) \geq \deg(f)$ since $\Phi^{-1}$ will satisfy the same property (it is also proper) and thus
%\[
%\deg(f) = \deg(\Phi^{-1} \Phi f) \geq \deg(\Phi f) \geq \deg(f).
%\]
%But this can be proved easily by iteration on the degree of $f$. Indeed, this is true if $\deg(f)=1$. Assume it is true for $\deg(f)=k$ and consider $f \in \mc{A}(SM)$ of degree $k+1$. By Lemma \ref{lemma:noether}, write $f = AU + BV$ with $\deg(U) = \deg(V)=k$, and apply the induction assumption
%\[
%\begin{split}
%\deg(\Phi(AU+BV)) &\geq \min\left( \deg(\Phi(A)) + \deg(\Phi(U)), \deg(\Phi(B))+\deg(\Phi(V))\right) \\
%& \geq k+1=\deg(f).
%\end{split}
%\]
%\end{proof}
%
%There is a (non-canonical) embedding 
%\[
%\iota_n : H^0(M,K^{\otimes n}) \hookrightarrow \mc{A}_{\geq n}(SM)
%\]
%given by
%\[
%\iota_n(\omega) := [\Pi \pi_n^* \Pi_n^{-1} \omega]_{\geq n},
%\]
%where $\Pi_n := {\pi_n}_* \Pi \pi_n^*$. Note that $\deg(\iota_n(\omega))=n$ since its first non-zero Fourier mode is precisely $\omega$. The operator $\Pi_n$ is elliptic pseudodifferential but not invertible: its kernel is given by $\eta_+ H^0(M,K^{\otimes (n-1)})$. Nevertheless, $H^0(M,K^{\otimes n})$ is $L^2$-orthogonal to $\eta_+ H^0(M,K^{\otimes (n-1)})$. We then set:
%\[
%\Phi_n := {\pi_n}_* \circ \Phi \circ \iota_n : H^0(M_1, K^{\otimes n}) \to H^0(M_2, K^{\otimes n}).
%\]
%The following holds:
%
%\begin{lemma}
%\label{lemma:iso}
%$\Phi_n : H^0(M_1, K^{\otimes n}) \to H^0(M_2, K^{\otimes n})$ is an isomorphism.
%\end{lemma}
%
%\begin{proof}
%Let $\omega \in H^0(M_1, K^{\otimes n})$, then $\iota_n(\omega)$ has degree $n$ and by Lemma \ref{lemma:degree-increasing}, $\Phi_n(\iota_n \omega)$ has degree $n$. Hence if $\Phi_n \omega = 0$, we get $\omega = 0$.
%\end{proof}
%
%We then define 
%\[
%F : H^0(M_1,\K) \to H^0(M_2,\K), \qquad F := \oplus_{n \geq 0} \Phi_n.
%\]
%The following holds:
%
%\begin{lemma}
%$F : H^0(M_1,\K) \to H^0(M_2,\K)$ is an algebra isomorphism.
%\end{lemma}
%
%\begin{proof}
%That $F$ is an isomorphism follows from \ref{lemma:iso}. Let us show the algebra property. Take $u \in H^0(M_1, K^i), v \in H^0(M_1,K^j)$, we want to show that $F(uv) = F(u) F(v)$. Let $U := \iota_i(u), V = \iota_j(v)$ and $W := \iota_{i+j}(uv)$. Note that $UV - W$ has degree $\geq i+j+1$. Hence $\Phi(UV-W) = \Phi(U)\Phi(V)-\Phi(W)$ has degree $\geq i+j+1$ so we get that
%\[
%{\pi_{i+j}}_* \Phi (W) = F(uv) = {\pi_{i+j}}_*(\Phi(U)\Phi(V)) = {\pi_i}_* \Phi(U) {\pi_j}_* \Phi(V) = F(u) F(v).
%\]
%\end{proof}
%
%\subsection{An abstract result} Let $M$ be a Riemann surface and denote by $g$ a metric inducing the same complex structure. For $z_0 \in M$, define
%\[
%D_{z_0} : H^0(M,\K) \to C^\infty(S^1) 
%\]
%as follows: take an isomorphism $S_{z_0}M \cong S^1$, then define the map $H^0(M,\K) \to C^\infty(S_{z_0}M) \to C^\infty(S^1)$. Note that the isomorphism $S_{z_0}M \cong S^1$ is only well-defined up to multiplication by $e^{i\theta_0}$ which amounts to composing $\tau_{\theta_0} \circ D_{z_0}$ by the translation by $\theta_0$ and there is also probably a multiplication by a scaling factor if we change $g$ by $e^{2f} g$. Hence, $D_{z_0}$ is only well-defined up to composition with the algebra isomorphism $\tau_c : C^\infty(S^1) \to C^\infty(S^1)$ such that $\tau_c(\sum_{k \geq 0} a_k e^{i k \theta}) = \sum_{k \geq 0} a_k c^k e^{i k \theta}$.
%
%We now prove the following:
%
%\begin{lemma}
%\label{lemma:abstract}
%Let $\varphi : H^0(M,\K) \to C^\infty(S^1)$ be a non-zero degree-preserving algebra homomorphism. Then $\varphi = \tau_c \circ D_{z_0}$ for some unique point $z_0 \in M$ and $c \in \C$.
%\end{lemma}
%
%\begin{proof}
%Take $\varphi : H^0(M,\K) \to C^\infty(S^1)$. Since $H^0(M,K)$ is mapped to $\C  e^{i\theta}$ and $\varphi \neq 0$, we can find a basis $\omega_1,...,\omega_g \in H^0(M,K)$ such that $\varphi(\omega_1) = e^{i \theta}, \varphi(\omega_2)=...=\varphi(\omega_g) = 0$. We claim that the $(\omega_i)_{2 \leq i \leq g}$ have a unique common zero $z_0 \in M$. First of all, observe that if such a zero exists, it is unique. Indeed, the maps $\mathrm{ev}_z : H^0(M,K) \to K_z \simeq \C$ for $z \in M$ are non-zero linear forms so their kernel has dimension $g-1$. Hence $\ker \mathrm{ev}_{z_0} \cap \ker \mathrm{ev}_{z_1}$ has dimension $g-2$ for $z_0 \neq z_1$ and thus the forms $\omega_2,...,\omega_g$ cannot all vanish at both $z_0$ and $z_1$.
%
%We start by proving that any $a,b \in \ker \varphi$ have a common zero. Assume that they do not have a common zero $z(a,b) \in M$. Then by Lemma \ref{lemma:surj}, if $a, b \neq 0$, we can write any $\omega \in H^0(M,K^{\otimes n})$ for $n \geq 2$ as $\omega = a   u + b   v$. But then, using the algebra property: $\varphi(\omega) = \varphi(a) \varphi(u) + \varphi(b)\varphi(v) = 0$. However, $\varphi(\omega_1^n) = \varphi(\omega_1)^n = e^{i n\theta} \neq 0$ which is a contradiction. Hence, there is a common zero $a$ and $b$. 
%
%
%
%Now, assume that we have proven that all $\eta_1,...,\eta_k$ have a common zero. We want to show that the same holds true for $\eta_1,...,\eta_k,\eta_{k+1}$. Take $z_{1...k}$ the common zero of $\eta_1, ..., \eta_k$. If $\eta_{k+1}(z_{1...k}) = 0$, then we are done. If not $\eta_{k+1}(z_{1...k}) \neq 0$, consider $\eta_1, ..., \eta_{k-1}, \eta_k + \lambda \eta_{k+1}$. For $\lambda$ small enough, they have a common zero $z(\lambda)$ near $z_{1...k}$ such that $z(0)=z_{1...k}$ and $z(\lambda) \neq z_{1...k}$ for $\lambda \neq 0$. But this is impossible because the only zero of $\eta_1$ near $z_{1...k}$ is $z_{1...k}$. This proves the claim.
%
%
%
%Observe now that $D_{z_0} : H^0(M,K) \to C^\infty(S^1)$ satisfies $D_{z_0}(\omega_2)=...=D_{z_0}(\omega_g)=0$ and $D_{z_0}(\omega_1) = 1/c \times e^{i \theta}$ for some $c \neq 0$. Hence, $\varphi - \tau_c \circ D_{z_0} = 0$ in restriction to $H^0(M,K)$ and since it generates the algebra $H^0(M,\K)$, we get that $\varphi = \tau_c \circ D_{z_0}$.
%\end{proof}
%
%\subsection{Constructing the biholomorphism} We now use the previous result to construct the biholomorphism $\kappa : M_1 \to M_2$. We now denote by $D_{z_0} : H^0(M,\K) \to C^\infty(S_{z_0}M)$ the mapping of the previous paragraph. This one is well-defined and intrinsic.
%
%Take $z_0 \in M_1$ and consider $D_{z_0} \circ F^{-1} : H^0(M_2,\K) \to C^\infty(S_{z_0}M_1)$. By Lemma \ref{lemma:abstract}, we have $D_{z_0} \circ F^{-1} = \tau_{c(z_0)} \circ D_{\kappa(z_0)}$ for some unique point $\kappa(z_0) \in M_2$, where $c(z_0) \in \C$ and $\tau_c : C^\infty(S_{z_0}M_1) \to C^\infty(S_{z_0}M_2)$ is the unique algebra isomorphism acting as the identity on the constants and by multiplication by $c$ on the Fourier degree $1$. In particular, for any $n \geq 1$, this implies that the hyperplane $E_{z} \subset H^0(M_1,K^{\otimes n})$ (holomorphic $n$-differentials vanishing at $z \in M_1$) is mapped by $F$ to $F(E_z) = E_{\kappa(z)} \subset H^0(M_2,K^{\otimes n})$, the hyperplane of holomorphic differentials vanishing at $\kappa(z)$.
%
%We then use Kodaira's embedding to conclude. For $n \gg 1$ large enough, denote by $E_z^* \subset H^0(M_2,K^{\otimes n})^*$, the $1$-dimensional space of forms vanishing on $E_z$ and denote by $p_1(z) := [E_z^*] \in P (H^0(M_1,K^{\otimes n}))^*$ the element in the projective space. Then:
%\[
%\xymatrix{
%    M_1 \ar[r] \ar[d]_{\kappa}  &   P (H^0(M_1,K^{\otimes n}))^* \ar[d]^{F_*} \\
%    M_2 \ar[r] & P (H^0(M_2,K^{\otimes n}))^*
%  }
%\]
%is a commutative diagram. This shows that $\kappa = \iota_{M_2}^{-1} \circ F \circ \iota_{M_1}$ is a biholomorphism. 



\bibliographystyle{alpha}
%\nocite{*}
\bibliography{Biblio}

\end{document}
