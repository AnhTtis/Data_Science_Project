% mnras_template.tex 
%
% LaTeX template for creating an MNRAS paper
%
% v3.0 released 14 May 2015
% (version numbers match those of mnras.cls)
%
% Copyright (C) Royal Astronomical Society 2015
% Authors:
% Keith T. Smith (Royal Astronomical Society)

% Change log
%
% v3.0 May 2015
%    Renamed to match the new package name
%    Version number matches mnras.cls
%    A few minor tweaks to wording
% v1.0 September 2013
%    Beta testing only - never publicly released
%    First version: a simple (ish) template for creating an MNRAS paper

%%%%%%%%%%%%%%%%%%%%%%%%%%%%%%%%%%%%%%%%%%%%%%%%%%
% Basic setup. Most papers should leave these options alone.
\documentclass[fleqn,usenatbib]{mnras}

% MNRAS is set in Times font. If you don't have this installed (most LaTeX
% installations will be fine) or prefer the old Computer Modern fonts, comment
% out the following line
\usepackage{newtxtext,newtxmath}
% Depending on your LaTeX fonts installation, you might get better results with one of these:
%\usepackage{mathptmx}
%\usepackage{txfonts}

% Use vector fonts, so it zooms properly in on-screen viewing software
% Don't change these lines unless you know what you are doing
\usepackage[T1]{fontenc}

% Allow "Thomas van Noord" and "Simon de Laguarde" and alike to be sorted by "N" and "L" etc. in the bibliography.
% Write the name in the bibliography as "\VAN{Noord}{Van}{van} Noord, Thomas"
\DeclareRobustCommand{\VAN}[3]{#2}
\let\VANthebibliography\thebibliography
\def\thebibliography{\DeclareRobustCommand{\VAN}[3]{##3}\VANthebibliography}

%%%%% AUTHORS - PLACE YOUR OWN PACKAGES HERE %%%%%

% Only include extra packages if you really need them. Common packages are:
\usepackage{graphicx}	% Including Figure files
\usepackage{amsmath}	% Advanced maths commands
\usepackage{caption}
\usepackage{xcolor}
\usepackage{subcaption}
\usepackage[normalem]{ulem} % For \sout{}
% \usepackage{amssymb}	% Extra maths symbols

%%%%%%%%%%%%%%%%%%%%%%%%%%%%%%%%%%%%%%%%%%%%%%%%%%

%%%%% AUTHORS - PLACE YOUR OWN COMMANDS HERE %%%%%

% Please keep new commands to a minimum, and use \newcommand not \def to avoid
% overwriting existing commands. Example:
%\newcommand{\pcm}{\,cm$^{-2}$}	% per cm-squared
\defcitealias{Donkelaar:2022ab}{Paper~I}
\newcommand{\PRC}[1]{\textcolor{cyan}{[{\bf PRC:} #1]}}
\newcommand{\PC}[1]{\textcolor{cyan}{#1}}
\newcommand{\soutPC}{\bgroup\markoverwith{\textcolor{cyan}{\rule[0.5ex]{2pt}{1pt}}}\ULon}
\newcommand{\todo}[1]{\textcolor{red}{[{\bf TO DO:} #1]}}

%%%%%%%%%%%%%%%%%%%%%%%%%%%%%%%%%%%%%%%%%%%%%%%%%%

%%%%%%%%%%%%%%%%%%% TITLE PAGE %%%%%%%%%%%%%%%%%%%

% Title of the paper, and the short title which is used in the headers.
% Keep the title short and informative.
\title[Hybrid nuclear star cluster formation]{Stellar cluster formation in a Milky Way-sized galaxy at $z>4$ -- II. A hybrid formation scenario for
the nuclear star cluster and its connection to the nuclear stellar ring}

% The list of authors, and the short list which is used in the headers.
% If you need two or more lines of authors, add an extra line using \newauthor
\author[F. van Donkelaar et al.] 
{Floor van Donkelaar,$^{1}$\thanks{floor.vandonkelaar@uzh.ch} Lucio Mayer,$^{1}$ Pedro R. Capelo,$^{1}$ Tomas Tamfal,$^{1}$ Thomas R. Quinn$^{2}$ \newauthor and Piero Madau$^{3}$\\
$^1$Center for Theoretical Astrophysics and Cosmology, Institute for Computational Science, University of Zurich,\\ Winterthurerstrasse 190, CH-8057 Z\"urich, Switzerland\\
$^2$Astronomy Department, University of Washington, Seattle, WA 98195, USA\\
$^3$Department of Astronomy and Astrophysics, University of California, 1156 High Street, Santa Cruz, CA 95064, USA}

% These dates will be filled out by the publisher
\date{Accepted XXX. Received YYY; in original form ZZZ}

% Enter the current year, for the copyright statements etc.
\pubyear{2023}

% Don't change these lines
\begin{document}
\label{firstpage}
\pagerange{\pageref{firstpage}--\pageref{lastpage}}
\maketitle

% Abstract of the paper
\begin{abstract}
Nuclear star clusters (NSCs) are massive star clusters found in the innermost region of the majority of galaxies. While recent studies suggest that low-mass NSCs in dwarf galaxies form largely out of the merger of globular clusters and NSCs in massive galaxies have assembled most of their mass through central star formation, the formation channel  of the Milky Way's NSC is still uncertain.  In this work, we use GigaEris, a very high resolution $N$-body hydrodynamical cosmological ``zoom-in'' simulation, to investigate NSC formation in the progenitor of a Milky Way-sized galaxy, as well as its relation to the assembly and evolution of the galactic nuclear region. We study the possibility that bound, young, gas-rich, stellar clusters within a radius of 1.5~kpc of the main galaxy's centre at $z>4$ are the  NSC predecessors (NSCPs). We identify 53 systems which satisfy our criteria, with a total baryonic mass of $10^{7.7}$~M$_{\sun}$. They have a relatively low mean stellar metallicity ($-0.47 \lesssim {\rm [Fe/H]} \lesssim -0.11$) in comparison to the present-day stars in the Milky Way's NSC. The NSCPs with a `born thin-disc' star fraction, $F_{\rm thin}$,  higher than 0.5 are older and display slightly different properties than the clusters with $F_{\rm thin} \leq 0.5$.  We demonstrate that both stellar cluster accretion and in-situ star formation will contribute to the formation of the NSC, providing evidence for an hybrid formation scenario for the first time in an $N$-body, hydrodynamical, cosmological ``zoom-in'' simulation. Furthermore, we also identify a nuclear stellar ring in the simulation, with  properties similar to those of the Milky Way's nuclear stellar disc. 
\end{abstract}

% Select between one and six entries from the list of approved keywords.
% Don't make up new ones.
\begin{keywords}
galaxies: formation -- galaxies: high-redshift -- galaxies: nuclei  -- methods: numerical
\end{keywords}

%%%%%%%%%%%%%%%%% BODY OF PAPER %%%%%%%%%%%%%%%%%%

\section{Introduction} \label{sec:intro}

The Galactic Centre of the Milky Way (MW) is an excellent laboratory for studying phenomena and physical processes that may be occurring in many other galactic nuclei.  The MW's nuclear star cluster (NSC) and nuclear stellar disc (NSD) are the main features of the Galactic Centre. Nevertheless, their observation is hampered by the extreme source crowding and high extinction. Hence, their relation and formation scenario are not fully clear yet \citep[][]{Schodel:2021aa, Nogueras:2021aa}.

NSCs are extremely dense and massive star clusters occupying the innermost region of a majority of galaxies of all types \citep[e.g.][]{Carollo:1997aa, Matthews:1999aa, Boker:2002aa, Cote:2006aa}. They are more luminous than globular clusters \citep[GCs; e.g.][]{Boker:2010aa} and have masses of the order of $\sim$$10^4$--$10^9$~M$_{\sun}$ \citep[][]{Walcher:2005aa, Fahrion:2020aa, Fahrion:2021aa} and effective radii of the order of 1--20~pc \citep[see][and references therein]{Neumayer:2020aa}. Many NSCs appear to be non-spherical. This is supported by observations of edge-on spirals which identified elongated, i.e. disc-like, structures in NSCs that are well-aligned with the disc of their host galaxies \citep{Seth:2006aa}. The NSCs with a higher stellar mass tend to be more flattened than lower-mass NSCs \citep[e.g.][]{Spengler:2017aa, Georgiev:2014aa}.

NSCs exist in very different host environments \citep[][]{Neumayer:2020aa}, which raises the question of whether NSC formation is controlled by similar processes in all galaxy types, or if NSCs follow evolutionary paths that depend on the properties of their host galaxy. As the NSC is one the main features of the MW's Galactic Centre, its formation path is of exceptional interest for the understanding of physical processes that occur in the central region of the MW. The MW's NSC extends up to hundreds of arcseconds across from the central supermassive black hole  and is believed to have a mass of $\sim$$10^{7.4}$~M$_{\sun}$ \citep[e.g.][]{Launhardt:2002aa, Schodel:2014aa, Fledmeier:2017aa}.

The formation and evolution path of the MW's NSC  are still unknown. There are two main hypotheses that have been suggested for the formation of the cluster: (a) through GC accretion; (b) through in-situ star formation \citep[SF;][]{Neumayer:2020aa}.  In the GC-accretion scenario, the NSC forms out of the gas-free merger of GCs that spiral into a galaxy’s centre due to dynamical friction \citep[e.g.][see also \citealt{Clarke:2019aa} for clump accretion]{Tremaine:1975aa, Capuzzo:1993aa, Capuzzo:2008aa, Agarwal:2011aa, Arca:2014aa, Gnedin:2014aa}. An NSC formed through GC accretion is expected to reflect properties typical of GCs, which are in general characterised by simple SF histories (SFHs), low metallicity, and a high fraction of old stars. The alternative formation path, in-situ SF, considers an NSC to form directly at the galactic centre out of star-forming gas \citep[e.g.][]{Milsavljevi:2004aa, McLaugh:2006aa, Bekki:2007aa}. In this nuclear SF scenario, gas falls into the nucleus and then transforms into stars \citep{Loose:1982}. As SF can proceed in several episodes during a galaxy’s evolution, an NSC formed in this way will exhibit a more complex SFH and is expected to have a low mass fraction of old and metal-poor stellar populations in comparison to NSCs that formed through GC accretion \citep[e.g.][]{Antonini:2012aa, Feldmeier:2015aa, Arca:2020aa, Fahrion:2020aa, Fahrion:2021aa, Fahrion:2022aa}.

Both processes may contribute with different weights in different galaxies, meaning that both mechanisms could contribute to the build-up of NSCs over a Hubble time \citep[see, e.g.][]{Guillard:2016aa}, through a hybrid formation scenario. The infall of GCs alone is not a viable formation scenario for the NSCs in massive early-type galaxies with stellar mass $M_{\star} > 10^9$~M$_{\sun}$, as the observed metallicity is too high and the SFH shows an ongoing process \citep[][]{Walcher:2005aa, Fahrior:2019aa, Pinna:2021aa}. However, mergers and accretion of gas-rich young stellar clusters \citep[][]{Figer:2002aa, Paudel:2020aa} could still provide a viable formation mechanism. Such a formation pathway can lead to similar stellar population properties as those observed in the MW's NSC.

In spite of its proximity, the observation of the MW's NSC is restrained by the extreme source crowding and the high interstellar extinction that limits its analysis to the infrared regime \citep[e.g.][]{Nishiyama:2008aa, schodel:2010aa, Nogueras:2018aa, Nogueras:2020ab}. Consequently, the relation between the NSC and the NSD is also not well understood yet \citep[see, e.g.][]{Launhardt:2002aa, Schodel:2020aa, Nogueras:2020aa}. There is some evidence that the NSC and the NSD may host different stellar populations with different SFHs \citep[e.g.][]{Nogueras:2020aa, Schodel:2020aa, Schultheis:2021aa}. Both components seem to have a predominantly old stellar population, with the initial starburst followed by several billion years of quiescence. Nevertheless, there is some evidence for a $\sim$3~Gyr old intermediate-age population in the NSC, which cannot be found in the NSD. On the other hand, there is evidence for a $\sim$1~Gyr old SF event associated to the NSD, that is not found when analysing the stellar population of the NSC \citep[see][]{Schodel:2020aa, Nogueras:2020aa, Nogueras:2021aa}. Because of the differences, understanding the formation paths of both systems can help us interpret the physical processes that occur in galactic nuclei.

This was initiated by \citet{Becklin:1982aa}, who showed that the region within 2~pc of the MW's Galactic Centre is largely devoid of interstellar matter and is surrounded by a dust ring or disc. More recent observations and simulations have determined this to be an NSD which includes stars and is star-forming \citep[e.g.][]{Launhardt:2002aa, Sungsoo:2012aa, Schultheis:2021aa}. The NSD of the MW extends up to a radius of 220~pc with a scale height of $\sim$50~pc \citep[][]{Piere:2000aa, Launhardt:2002aa,Nogueras:2020aa, Gallego:2020aa}. NSDs are also detected in extragalactic systems and are quite common in early-type galaxies \citep[][]{Pizella:2002aa, Gadotti:2019aa}. However, given that no NSDs have been clearly identified in late-type galaxies and provided that the MW is a barred spiral galaxy, it is plausible that the NSD of the MW is rather a mixture of a nuclear star-forming ring (the nuclear stellar ring; NSR) and a nuclear spiral \citep[][]{Schodel:2021aa}. Nevertheless, the fact that the MW has not had any major merger in the past $\sim$10 Gyr \citep[e.g][]{Wyse:2001aa,Helmi:2018aa, Renaud:2021aa, Sotillo:2022aa}  makes the existence of a nuclear disc, such as those observed in S0 galaxies, still a possibility in the MW. The bulk of stars in the MW's NSD is old and formed at least 8~Gyr ago, followed first by a phase of quiescence and then by recent SF activity \citep[about 1~Gyr ago, when 5 per cent of the mass of the NSD was formed very quickly;][]{Nogueras:2019aa, Nogueras:2020aa, Nogueras:2021aa, Nogueras:2023aa, Schodel:2020aa}.

Galactic bars can lead to the creation of substructures in the nuclear region of disc galaxies such as NSDs and NSRs by redistributing angular momentum \citep[][]{Combes:1985aa}. There is now growing evidence that these substructures in the galactic centre are built from gas that was funnelled to the centre by the bar \citep[e.g.][]{Contopoulos:1989aa, Binney:1991aa,Knapen:1999aa, kim:2011aa, Fragkoudi:2016aa}. Another formation scenario of NSDs could be galaxy mergers \citep[e.g.][]{Mayer:2008aa}; these, however, fail to reproduce the sizes of typically observed NSDs in nearby galaxies \citep[][]{Schultheis:2021aa}.

Moreover, since we do not detect an age-metallicity relation in the solar neighbourhood, clusters of stars are expected to undergo radial migration \citep[][]{Sellwood:2002aa, Haywood:2008aa, Roskar:2008aa, Schonrich:2009aa}. Therefore, one could expect that gas-rich stellar clusters migrated from the outer parts of the Galaxy towards the Galactic Centre. For example, in \citet{Schonrich:2009aa}, the stars are trapped on to a resonant co-rotation with spiral arms and may migrate inwards and outwards along the spiral waves. The thin disc is expected to start forming the earliest assembly stage of a galaxy \citep[e.g.][see also \citealt{Agertz2021, Silva:2021aa, Michael:2022aa, vanDonkelaar:2022aa} for the early formation of a thin-disc component through co-formation of the discs]{Tamfal:2022aa}, therefore a significant population of old thin-disc stars could have influenced the formation of the NSC and NSD.

From a theoretical point of view, hydrodynamic simulations of MW analogs have shown that the formation of the bar can trigger gas funnelling to the centre of the MW, forming a kinematically cold, rotating NSD \citep[e.g.][]{Fux:1999aa, Li:2015aa, Ridley:2017aa, Sormani:2019aa, Tress:2020aa, Moon:2021aa, Sormani:2022aa}. Additionally, a large number of simulations and semi-analytic models have calculated the efficiency of dynamical friction for a range of starting conditions for GC systems and their host galaxies, confirming that dynamical friction provides a plausible mechanism to form an NSC \citep[e.g.][]{Capuzzo:1993aa, Lotz:2001aa}. Detailed $N$-body simulations of the infall of GCs through dynamical friction have shown consistency with the flattening and the kinematic properties of observed NSCs, but that matching the kinematics and luminosity function of the stellar clusters likely requires roughly half of the NSC mass to come from in-situ SF and accreted gas \citep[e.g.][]{Hartmann:2011aa, Antonini:2012aa, Tsatsi:2017aa}. Furthermore, semi-analytic models that follow the evolution of GC systems with dynamical friction to track the growth of the NSCs have been able to reproduce some of the properties, like mass and radius, of both the present-day NSCs and the GC systems \citep[e.g.][]{Antonini:2013aa, Gnedin:2014aa}. \citet{bekki:2010aa} carried out $N$-body simulations of the orbital decay of stellar clusters in the background of field stars in a disc galaxy embedded in a dark matter (DM) halo, finding that NSCs could have formed from stars delivered by inspiralling stellar clusters.

In this work, we aim to pinpoint the formation channels of the MW's NSC  with an $N$-body, hydrodynamical, cosmological ``zoom-in''  simulation of unprecedented resolution, GigaEris \citep[][]{Tamfal:2022aa}. Additionally, we explore the relation between the MW's NSC and its NSD, and how their formation may have been connected. For that reason, we delve into the properties of young, gas-rich, stellar clusters at $z>4$ in the nuclear region of an MW-sized galaxy. The layout of this paper is the following: Section~\ref{sec:method} briefly summarises the simulation setup and describes how we identified the clusters. In Section~\ref{sec:results}, we present the simulation results, first with a focus on the properties of the possible NSC predecessors (NSCPs) at $z= 4.4$ and the possibility that these stellar clusters will form an NSC with the properties of the MW's NSC, and then presenting the NSR within the simulation and its link to the NSC. Finally, we discuss our results in Section~\ref{sec:disc} and conclude in Section~\ref{sec:conc}.

\section{Methods}\label{sec:method}

\subsection{Simulation code and initial conditions}

We base our analysis on an $N$-body, hydrodynamical, cosmological ``zoom-in'' simulation of a MW-like galaxy, GigaEris \citep[][]{Tamfal:2022aa}, carried out with the $N$-body smoothed-particle hydrodynamics (SPH) code \textsc{ChaNGa} \citep[][]{Jetley:2008aa,Jetley:2010aa, Menon:2015aa}. A brief summary of the numerical recipes is provided below; one can find a more detailed discussion on the set-up in \citet{Tamfal:2022aa}.

The GigaEris simulation follows a Galactic-scale halo identified in a low-resolution, DM-only simulation at $z = 0$ in a periodic cube of side 90 cMpc. It was chosen to have a similar mass as that of the MW and a rather quiet late merging history. This method is similar to the way the galaxy halo was selected in the original Eris suite in \citet{Guedes:2011aa}. Next, the selected halo was re-simulated at several orders of magnitude higher resolution than the DM-only simulation, adding gas particles as well as the necessary short-wavelength modes. The initial conditions were generated with the \textsc{MUSIC} code \citep[][]{Hahn:2011aa}, with 14 levels of refinement and the cosmological parameters $\Omega_{\rm m}$ = 0.3089, $\Omega_{\rm b}$ = 0.0486, $\Omega_{\Lambda}$ = 0.6911, $\sigma_8$ = 0.8159, $n_{\rm s}$ = 0.9667, and $H_0$ = 67.74~km~s$^{-1}$ Mpc$^{-1}$ \citep[see][]{Planck:2016aa}. The gravitational softening of all particles is set to a constant in physical coordinates ($\epsilon_{\rm c} = 0.043$~kpc) for redshifts smaller than $z = 10$ and otherwise evolves as $\epsilon = 11\epsilon_{\rm c}/(1+z)$.  For the final snapshot at $z=4.4$, the DM, gas, and stellar particle numbers in the entire simulation box are $n_{\rm DM} = 5.7 \times 10^8$, $n_{\rm gas} = 5.2 \times 10^8$, and $n_{\star} = 4.4 \times 10^7$, respectively.

Each star particle is created stochastically with an initial mass of $m_{\star} = 1026$~M$_{\sun}$ using a simple gas density and temperature threshold criterion \citep[][]{Stinson:2006aa}, with $n_{\rm SF} > 100$~$m_{\rm H}$~cm$^{\text{-} 3}$ and $T_{\rm SF} < 3 \times 10^4$~K,  and the gas particle that spawns the new star has its own mass reduced accordingly. A star particle represents an entire stellar population with its own \citet{Kroupa:2001aa} initial stellar mass function (IMF). The SF  proceeds at a rate which is given by

\begin{equation}
    \frac{\rm{d} \rho_{\star}}{\rm{d}t} = \epsilon_{\rm SF} \frac{\rho_{\rm gas}}{t_{\rm dyn}},
\end{equation}

\noindent with $\rho_{\star}$ indicating the stellar density, $\rho_{\rm gas}$ the gas density, $t_{\rm dyn}$ the local dynamical time, and $\epsilon_{\rm SF}$ the SF efficiency, which is set to $0.1$. The code solves for the non-equilibrium abundances and cooling of H and He species \citep[assuming self-shielding and a redshift-dependent radiation background;][]{Pontzen:2008aa,Haardt:2012aa}, whereas the cooling from the fine structure lines of metals is calculated in photoionization equilibrium from the same radiation background (assuming no self-shielding; see \citealt{Capelo:2018aa} for a discussion), using tabulated rates from Cloudy \citep{Ferland:2010aa, Ferland:2013aa} and following the method described in \citet{Shen:2010aa, Shen:2013aa}.

Feedback from supernovae SNae Type Ia  is implemented by injecting energy and a fixed amount of mass and metals, independent of the progenitor mass, into the surroundings \citep[see][]{Thielemann:1986, Stinson:2006aa}, whereas SNae Type II (SNII) feedback is implemented following the delayed-cooling recipe of \citet{Stinson:2006aa}, with metals and energy, $\epsilon_{\rm SF} =10^{51}$~erg, being injected per event into the interstellar medium  as thermal energy, according to the ‘blastwave model’ of \citet{Stinson:2006aa}. For each SNII event, a given amount of oxygen and iron mass, dependent on the mass of the star, is injected into the surrounding gas \citep{Woosley:1995aa, raiteri:1996aa}. The stars with masses between 8 and 40~M$_{\sun}$ will explode as SNII, whereas stars with masses between 1 and 8~M$_{\sun}$ do not explode as SNae but release part of their mass as stellar winds, with the returned gas having the same metallicity of the low-mass stars.

\subsection{Cluster finding}

We use the adaptive mesh grid \textsc{AMIGA Halo Finder} \citep[\textsc{AHF};][]{Gill:2004aa, Knollmann:2009aa} to identify the gas-rich stellar substructures that possibly could be connected to the formation of the NSCs in the final simulation step  at $z=4.4$. The clusters were selected in an identical way as in  \citeauthor{Donkelaar:2022ab} (\citeyear{Donkelaar:2022ab}; hereafter \citetalias{Donkelaar:2022ab}), namely with a minimum threshold of 64 baryonic particles in the virial radius per cluster and 0 subclusters inside the identified cluster. Furthermore, to validate that the clusters are bound, we have calculated the binding energies of the identified clusters and unbound clusters were removed from the set. Hereafter the word cluster in this paper refers to a bound object found by \textsc{AHF}.

\section{Results}\label{sec:results}

\subsection{Classifying stellar systems}\label{sec:class}

\begin{figure}
\centering
\setlength\tabcolsep{2pt}
\includegraphics[ trim={0cm 0cm 0cm 0cm}, clip, width=0.46\textwidth, keepaspectratio]{Images/Figure1.pdf}
\caption{Top panel: the object's distance away from the galactic centre of the main galaxy halo plotted against its baryonic fraction, $F_{\rm b}$, for all halos identified with a baryonic mass range $10^4$--$10^8$~M$_{\sun}$ at $z=4.4$. The sizes of the markers indicate the mass range of the clusters. Bottom panel: the stellar fraction, $F_{\star}$, of the 56 clusters within the red circle of the top panel is plotted against the baryonic fraction. The gray-shaded area indicates the region where $F_{\rm b} \geq 0.75$ (one of the criteria used to categorize a cluster as a possible proto-GC; see \citetalias{Donkelaar:2022ab}). The colour bar represents the stellar metallicity of the clusters.}
\label{fig:F_z}
\end{figure}

To investigate the possibility that gas-rich stellar clusters at $z>4$ are NSCP candidates within our simulation, we extract all \textsc{AHF} identified substructures in the simulated box with a baryonic mass range $10^4$--$10^8$~M$_{\sun}$. For each of the substructures, the total baryonic mass fraction,

\begin{equation}\label{eq:fb}
    F_{\rm b} = \frac{M_{\star}+M_{\rm gas}}{M_{\rm total}},
\end{equation}

\noindent and stellar mass fraction,

\begin{equation}\label{eq:fs}
    F_{\star} = \frac{M_{\star}}{M_{\star}+M_{\rm gas}},
\end{equation}

\noindent are calculated, where $M_{\rm total}$ is the sum of the baryonic and DM masses, $M_{\star}$ the stellar mass, and $M_{\rm gas}$ the gas mass, all computed at half the virial radius. This approach is similar to that of \citetalias{Donkelaar:2022ab}. The distance away from the galactic centre of the main galaxy of the identified clusters is plotted against their baryonic mass fraction in the top panel of Figure~\ref{fig:F_z}. In \citetalias{Donkelaar:2022ab}, we selected all clusters with an $F_{\rm b} \geq 0.75$ as proto-GC systems (excluding systems with a $\sigma_{\star} < 20$~km~s$^{-1}$): this region has been indicated by the gray-shaded area in the Figure. In the top panel, we see a new group of interesting clusters, all being very close to the galactic centre of the main galaxy ($r <1.5$~kpc), with a relatively high baryonic fraction ($F_{\rm b}>0.35$), and with a similar stellar metallicity (around solar values).\footnote{In this work, we compute the abundance ratios (e.g. [Fe/H] and [O/Fe]) normalising them to the solar values provided by \citet{Asplund:2009aa}.} We select all bound clusters within this region, indicated by the red circle in the top panel of the Figure, as possible NSCPs.

In the bottom panel of Figure~\ref{fig:F_z}, the stellar mass fraction is plotted against the baryonic mass fraction for these 56 possible NSCPs. From this Figure, we can conclude that, even though this group of stellar clusters are in a similar region of the main galaxy and have a similar metallicity, they have a wide variety of stellar and baryonic mass fractions.

We have not used a minimum baryonic mass fraction as part of our selection criteria. This is because the NSCPs are not the final product. Their baryonic mass fraction can change during their evolution, especially when they fall into the high-density central region of the galaxy. This is for example similar to what happened to ``The Imposter'' in \citetalias{Donkelaar:2022ab}, which lost all of its  DM while spiralling into the MW analog.

\subsubsection{Dynamical friction time-scale}\label{sec:dynamicaltime}

\begin{figure}
\centering
\setlength\tabcolsep{2pt}%
\includegraphics[ trim={0cm 0cm 0cm 0cm}, clip, width=0.48\textwidth, keepaspectratio]{Images/Figure2.pdf}
\caption{ Stellar density profile of the central region of our simulated galaxy at $z = 4.4$ (blue, solid line). The black, dashed line shows the \citeauthor{Hernquist:1990aa} profile with a scale radius $a = 78$~pc and a total stellar mass $M_{\rm b,\star} = 10^{10.49}$~M$_{\sun}$. }
\label{fig:hern}
\end{figure}

In an attempt to estimate the time taken by a given possible NSCP to decay to the centre of the galaxy, we first approximate the inner parts of the stellar density profile with the \citet{Hernquist:1990aa} analytical model. This can be done, as the central stellar region of the main galaxy halo follows quite well the \citeauthor{Hernquist:1990aa} profile, as shown in Figure~\ref{fig:hern}. The solid, blue line shows the central stellar density profile when assuming a spherically averaged distribution of the galaxy between 0 and 2 kpc, whereas the black, dashed line shows the \citeauthor{Hernquist:1990aa} profile with a scale radius of $78$~pc and a total enclosed stellar mass of $M_{\rm b,\star} = 10^{10.49}$~M$_{\sun}$. 

By taking into account the effects of dynamical friction in a fictitious stellar \citeauthor{Hernquist:1990aa} bulge with a scale radius $a$ and total mass $M_{\rm b, \star}$, the corresponding time-scale for a stellar cluster of total stellar mass $M_{\star}$ computed at half the virial radius in circular motion inside such a bulge to decay from an initial radial distance $r_{\rm i}$ to a final one $r_{\rm f}$ is given by \citep[][]{Lima:2017aa,Tamburello:2017aa}

\begin{equation}\label{eq:tdf}
   t_{\rm DF} = \frac{1}{4 \xi \ln{\Lambda}} \sqrt{\frac{a^3}{GM_{\rm b, \star}}} \frac{M_{\rm b, \star}}{M_{\star}} \int_{\chi_{\rm f}}^{\chi_{\rm i}} \frac{\chi^{3/2}(\chi +3)}{\chi + 1} d\chi ,
\end{equation}

\noindent with $\xi$ being a correction factor of order unity, $\ln{\Lambda}$ the Coulomb logarithm, $G$ the gravitational constant, and $\chi = r/a$.  From Figure~\ref{fig:F_z}, we can see that the selected clusters are located between $\sim$0.35 and 1.5~kpc from the galactic centre. Therefore, we can make the assumption that $\chi_i \gg 1$ (i.e. $r_i \gg a$), so that, additionally imposing $\chi_{\rm f} = 0$, the integral of Equation~\eqref{eq:tdf} can be approximated as $2\chi^{5/2}/5$ and we can write \citep[][]{Tamburello:2017aa}

\begin{equation} 
   t_{\rm DF} = \frac{ M_{\rm b, \star} \chi^{5/2}}{10 \xi \ln{\Lambda} M_{\star} } \sqrt{\frac{a^3}{GM_{\rm b, \star}}}.
\end{equation} 

The Coulomb logarithm,  $\ln{\Lambda}$, can be approximated as \citep[][]{Binney:2008aa}

\begin{equation} 
   \Lambda \approx \frac{b_{\rm max}}{b_{\rm 90}} \approx \frac{b_{\rm max} v^2_{\rm typ}}{GM_{\star}},
\end{equation} 

\noindent where  $v_{\rm typ}$ is the typical relative velocity, $b_{\rm max}$ is the maximum impact parameter, and $b_{\rm 90}$ is the 90$^{\circ}$ deflection radius. Taking the mean stellar mass, distance from the galactic centre, and typical velocity of the clusters for $M_{\star}$, $b_{\rm max}$, and $v_{\rm typ}$, respectively, we a value for the Coulomb logarithm of approximately 8.8. Lastly, we set $\xi = 1$  (as done in \citealt{Lima:2017aa} and \citealt{Tamburello:2017aa}).

The resulting $t_{\rm DF}$ for the stellar clusters is shown in Figure~\ref{fig:tdf}. It shows that 53 of the 56 selected stellar clusters in Figure~\ref{fig:F_z} will decay to the centre within 12~Gyr (indicated by the dashed, vertical line), which is the time between the last snapshot in the simulation, at $z=4.4$, and $z \sim 0$. Moreover, the influence of the disc, bar, or other galactic structures have not been included in this model \citep[see, e.g.][]{Bar:2022aa}. Owing to the galactic bar, for example, a massive object with a similar inclination and distance away from the centre as our NSCPs can have its inspiral time-scale decreased, as shown by  \citet{Bortolas:2020aa,Bortolas:2022aa}. They concluded that for in-plane perturbers there is a clear tendency to decrease the dynamical friction time-scale, whereas for perturbers on arbitrary inclinations the effect is stochastic, with both increases and decreases of the dynamical friction time-scale similarly possible. Note that our NSCPs are found in a symmetric ring with a  small inclination of $\sim$14$^{\circ}$ with respect to the galactic plane (see Section~\ref{sec:ring}). Therefore, one would expect that the decay time-scale would decrease when adding these structures to the model. We can thus conclude that most of the selected clusters from Figure~\ref{fig:F_z} are consistent with the chosen definition of NSCPs. 

\begin{figure}
\centering
\setlength\tabcolsep{2pt}%
\includegraphics[ trim={0cm 0cm 0cm 0cm}, clip, width=0.48\textwidth, keepaspectratio]{Images/Figure3.pdf}
\caption{The full dynamical friction time-scale distribution for all the identified possible NSCPs. The dashed vertical line indicates 12~Gyr, which is the time between the last snapshot in the simulation, at $z = 4.4$,  and $z \sim 0$. Just three systems out of 56 have a $t_{\rm DF} > 12$~Gyr.}
\label{fig:tdf}
\end{figure}

\subsection{Properties of the NSCPs}\label{sec:prop}

\begin{figure}
\centering
\setlength\tabcolsep{2pt}%
\includegraphics[ trim={0cm 0cm 0cm 0cm}, clip, width=0.48\textwidth, keepaspectratio]{Images/Figure4.pdf}
\caption{Abundance ratios (top panel: [Fe/H]; bottom panel: [O/Fe]) as a function of stellar mass. The blue dots indicate our new stellar cluster sample of 53 systems, defined by applying the selection criterion described in Section~\ref{sec:class} ($r < 1.5$~kpc and $F_{\rm b} > 0.35$) and additionally imposing $t_{\rm DF} < 12$~Gyr (Section~\ref{sec:dynamicaltime}). The dots with a black outline represent the stellar clusters with $F_{\rm thin} > 0.5$ (see Section~\ref{sec:thin}). The gray horizontal lines indicate the mean values.}
\label{fig:prop}
\end{figure}

We carry on by studying the properties of the selected stellar clusters within 1.5~kpc of the centre of the main galaxy. Together with the dynamical friction time-scale estimation, these properties will allow us to investigate the hypothesis that the clusters are indeed predecessors of the NSC. Figure~\ref{fig:prop} shows how some properties of these selected clusters relate to their stellar mass ($M_{\star}$). The total sum of the stellar masses of all 53 NSCPs (hereafter, we exclude the three systems with $t_{\rm DF} > 12$~Gyr), $\sim$$10^{7.5}$~M$_{\sun}$, is of the same order of magnitude as the observed mass of the MW's NSC \citep[$\sim$$10^{7.4}$~M$_{\sun}$, e.g.][]{Schodel:2014aa,Fledmeier:2017aa}, which reinforces the possibility of these clusters being NSCPs. Nevertheless, one should note that 28 per cent of this total stellar mass comes from one massive cluster. This cluster has, discarding the mass, average properties in comparison to the other clusters and is therefore still included as a possible NSCP. Furthermore, all the NSCPs are spherical clumps (as shown by the examples in Figure~\ref{fig:clumps}) and follow the radial density profiles of the \citet{King:1972aa} model.

\begin{figure*}
\centering
\setlength\tabcolsep{2pt}
\includegraphics[ trim={0cm 0cm 0cm 0cm}, clip, width=0.98\textwidth, keepaspectratio]{Images/Figure5.pdf}
\caption{Stellar surface density maps of three selected NSCPs at $z=4.4$ centred at their centre of mass. Combining the surface density maps with the knowledge that they follow the radial density profiles of the \citeauthor{King:1972aa} model, we can assume they resemble spherical clumps. The total stellar mass of the clusters at this redshift is given by the title of the plot. The gas mass of the shown clusters is between $10^{5.1} \lesssim {M_{\rm gas} [{\rm M}_{\sun}]} \lesssim 10^{5.8}$.}
\label{fig:clumps}
\end{figure*}

In the top panel of Figure~\ref{fig:prop}, the stellar metallicity, [Fe/H], is plotted against the stellar mass for the selected sample of NSCPs. The metallicity of the possible NSCPs is spread between $-0.47$ and $-0.11$, with a mean of $-0.32$. The clusters have a lower mean metallicity than what one would expect from today's MW's NSC, which is solar to super solar, and at the lower end of what is expected from galaxies with a stellar mass above $10^9$~M$_{\sun}$ \citep[$-0.5 \lesssim {\rm [Fe/H]} \lesssim 0.5$; e.g.][]{Kacharov:2018aa, Schodel:2020aa}. However, this is reasonable at $z=4.4$, as our clusters are still star-forming at this redshift and thus the mean stellar metallicity of the stars in the NSCPs can still be raised. Furthermore, assuming a hybrid formation scenario for the MW's NSC, stars formed through central SF would also raise the mean metallicity over time. Comparing this to the metallicity of proto-GCs in \citetalias{Donkelaar:2022ab} ($-1.8 \lesssim {\rm [Fe/H]} \lesssim -0.8$), we can assume that the metallicity of these selected stellar clusters is too high to be a proto-GC and closer to what one would expect from the MW's NSC.

\begin{figure}
\centering
\setlength\tabcolsep{2pt}%
\includegraphics[ trim={0cm 0cm 0cm 0cm}, clip, width=0.49\textwidth, keepaspectratio]{Images/Figure6.pdf}
\caption{The location of the 53 NSCPs in the $x$-$y$-$z$-space, with the stellar galactic angular momentum pointing in the $+z$ direction. The black `$\times$' indicates the centre of the main galaxy halo. The colour bar represents the mean age of the stars within the clusters.}
\label{fig:thindisc}
\end{figure}


The bottom panel of Figure~\ref{fig:prop} shows the [O/Fe] ratio of the possible NSCPs, all within $0.02 \lesssim {\rm [O/Fe]} \lesssim 0.26$. In the MW, thick-disc stars have a larger oxygen abundance than thin-disc stars with the same [Fe/H], as shown by, e.g. \citet{Franchini:2021aa}. For [Fe/H] $= -0.5$, the mean abundance ratios are [O/Fe] = $0.36 \pm 0.19$ and $0.24 \pm 0.07$ for the thick- and thin-disc stars, respectively \citep[see][]{Reddy:2006aa, Bertan:2016aa}. The mean [O/Fe] ratio for the possible NSCPs is $\sim$0.15, whereas the mean [O/Fe] for the thin-disc stars in the simulation is $\sim$0.13. Hence, from the [O/Fe] ratio in our sample, we can  deduce that most of the stars within the NSCPs, especially the ones with a low [O/Fe] ratio, could have originated out of the thin disc (see Section~\ref{sec:thin} for further reasoning).

All the NSCPs are still star-forming at $z = 4.4$, which could indicate that we are looking at a hybrid formation scenario of the MW's NSC if still star-forming clusters will inspiral in and bring the star-forming gas to the galactic centre. For all clusters at $z = 4.4$, the SF episodes are between 0.7 and 1.2~Gyr long, with a mean specific SF rate (SFR) between 0.8 and 1.2~Gyr$^{-1}$. The fact that the NSCPs are still star-forming again shows that the NSCPs are different from the proto-GCs discussed in \citetalias{Donkelaar:2022ab}, as we expect a burst of SF for GCs. At $z=4.4$, all clusters experience their highest SFR since birth, with SFRs between $10^{-3.9}$  and $10^{-1.6}$~M$_{\sun}$~yr~$^{-1}$.

\subsubsection{Thin-disc stars}\label{sec:thin}


As shown in Figure~\ref{fig:prop}, approximately half of the NSCPs have the fraction of ``born thin-disc'' stars higher than 0.5. The ``born thin-disc'' stars are defined by applying the \textsc{DBSCAN} \citep[see][]{Ester:1996aa} clustering algorithm to our simulation at different time steps.\footnote{See \citet{Tamfal:2022aa} for an explicit approach of identifying stars as thin-disc stars.} This way, sequences of mutually spatially connected particles are identified with a process comparable to that of a scatter kernel interpolation in SPH. The ID of the stars born in a thin disc is saved and from this we can calculate the fraction of ``born thin-disc'' stars, $F_{\rm thin}$. The NSCPs with $F_{\rm thin} > 0.5$, indicated with the black outline in Figure~\ref{fig:prop}, have a higher mean stellar metallicity ([Fe/H] $\sim -0.22$ ) and a lower [O/Fe] ratio than the mean of the whole sample, which is expected from thin-disc stars \citep[see, e.g.][]{Franchini:2021aa, Reddy:2006aa, Bertan:2016aa}. Combining the birth environment determined by \textsc{DBSCAN} and the discrepancy between the metallicity properties of the two populations of NSCPs in Figure~\ref{fig:prop}, we can confidently say that the  stars identified using \textsc{DBSCAN} are indeed ``born thin-disc'' stars. We find that all NSCPs consist a minimum of 3 per cent out of ``born thin-disc'' stars and 48 per cent of the clusters the ``born thin-disc'' stars make up more than half of their total stars at $z=4.4$.

The different properties of these two types of NSCPs could suggest that the NSCPs with $F_{\rm thin} >0.5$ formed under different conditions. This is further investigated in Figure~\ref{fig:thindisc}, where the NSCPs are plotted in the $x$-$y$-$z$-space with the stellar galactic angular momentum pointing in the $+z$ direction at $z = 4.4$. The NSCPs form an elliptical ring around the galactic centre of the main galaxy halo. From the Figure, it is clear that the two populations of NSCPs, high- and low-$F_{\rm thin}$, can be found in different regions in this elliptical ring and there is barely any mixing between the two groups. The colour bar in Figure~\ref{fig:thindisc} shows the mean age of the stars within the NSCPs, from which we can deduce that the high-$F_{\rm thin}$ NSCPs have an older mean stellar age. We find a mean stellar age for high-$F_{\rm thin}$ NSCPs of $250.2 \pm 45.2$~Myr. For low-$F_{\rm thin}$ NSCPs, we find a mean stellar age of $102.1 \pm 45.1$~Myr. Notably, the ``born thin-disc'' stars are the oldest stars within the stellar clusters. The mean age of the NSCP stars is therefore correlated with where the cluster can be found within the elliptical ring at $z = 4.4$. Tantalizingly, there is no clear relation between the mean age of the ``born thin-disc'' stars within the NSCPs and the birth radius of the oldest star within the cluster, even though some clusters include stars born far outside the disc.

As expected, the SFHs of the two NSCP groups, high- and low-$F_{\rm thin}$,  are also different. The $F_{\rm thin} >0.5$ NSCPs start forming stars  earlier, consistent with that fact that these clusters are older. However, they also show an approximately constant SFR of $\sim$$0.4 \times 10^{-2}$~M$_{\sun}$~yr~$^{-1}$ from $z \lesssim 6$, with an increase in SFR around $z \sim 4.5$. On the contrary, the $F_{\rm thin} \leq 0.5$ NSCPs have on average a lower SFR and show this increase in SF from $z=5$.

\subsection{The hybrid formation scenario}\label{sec:hybrid}

\begin{figure}
\centering
\setlength\tabcolsep{2pt}%
\includegraphics[ trim={0cm 0cm 0cm 0cm}, clip, width=0.48\textwidth, keepaspectratio]{Images/Figure7.pdf}
\caption{Gas mass as a function of stellar mass. The 53 dots indicate the stellar cluster sample defined by applying the selection criterion described in Section~\ref{sec:class} ($r < 1.5$~kpc and $F_{\rm b} > 0.35$) and additionally imposing $t_{\rm DF} < 12$~Gyr (Section~\ref{sec:dynamicaltime}). The colour bar represents the total baryonic mass, $M_{\rm b}$,  of the stellar clusters. The gray, dashed line displays the 1:1 ratio between the stellar and gas mass.}
\label{fig:mgas}
\end{figure}

As discussed in Section~\ref{sec:intro}, both GC accretion and in-situ SF could contribute to the build-up of NSCs over a Hubble time \citep[see, e.g.][]{Guillard:2016aa}. The NSC of the MW is a prominent example, as it shows both young stars likely formed in-situ and old, metal-poor components that possibly were accreted from GCs \citep[e.g.][]{Antonini:2012aa, Feldmeier:2015aa, Feldmeier:2020aa, Arca:2020aa}. We have shown so far that the process of infalling stellar clusters will contribute to the formation of the NSC in our simulated galaxy. Nevertheless, we have also shown that in-situ SF will be needed to acquire a stellar metallicity that is consistent with that of the MW's NSC.



For the in-situ SF formation scenario, gas will need to fall into the nucleus and then transform into stars \citep[e.g][]{Loose:1982, Milsavljevi:2004aa}. From the stellar-gas mass relation shown in Figure~\ref{fig:mgas}, one can conclude that the clusters with a high baryonic mass still include gas. The NSCPs have a $F_{\rm \star}$ between $0.19$ and $0.94$, with a mean stellar mass fraction of $0.56$. Because the NSCs of massive galaxies show significant contributions from young and enriched populations \citep[e.g.][]{Kacharov:2018aa, Pinna:2021aa, Fahrion:2021aa, Fahrion:2022aa}, we would assume that the gas from these clusters that is not yet converted into stars when the cluster reaches the galactic centre would become part of the gas reservoir used for the central SF. Combining this with the dynamical friction time-scale from Figure~\ref{fig:tdf}, we can infer that high-mass, gas-rich stellar clusters will contribute to the gas reservoir in the centre of the galaxy, which will contribute to in-situ SF of the NSC.

Using the previous assumption, we could obtain an estimate on the maximum amount of stellar mass that can assemble from gas coming from infalling stellar clusters. When only considering the stellar clusters that have a dynamical friction time of $t_{\rm DF} \leq 1$~Gyr to still be gas-rich when reaching the galactic centre of the galaxy, we find a total of gas mass that can be used for SF in the central region to be $10^{7.3}$~M$_{\sun}$. Combining this to the SFR of the clusters discussed in Section~\ref{sec:prop}, we find that a total gas mass between $\sim$$10^{6.1}$ and $10^{7.3}$~M$_{\sun}$ will be transformed into stars through in-situ SF. Assuming that NSCPs are the main source of baryonic mass for the NSC, we can obtain an estimate on the contribution of in-situ SF within an MW-like galaxy for the formation of an NSC. We find that in-situ SF from gas of accreted clusters contributes to a maximum 30 per cent  of the final NSC cluster mass.

\begin{figure}
\centering
\setlength\tabcolsep{2pt}%
\includegraphics[ trim={0cm 0cm 0cm 0cm}, clip, width=0.49\textwidth, keepaspectratio]{Images/Figure8.pdf}
\caption{Gas surface density map of the main galaxy at $z=5.0$, with the velocity vectors of the gas overlaid. The galaxy is centred face-on on the disc stars from the main galaxy halo.}
\label{fig:gasstreams}
\end{figure}

Furthermore, in a barred galaxy system, it is expected that the bar will efficiently funnel the gas toward to centre, where it could settle onto an elliptical ring \citep[see Section~\ref{sec:ring}; see, e.g.][]{Contopoulos:1989aa, Binney:1991aa,Knapen:1999aa, Regan:2003, Li:2017aa, Sormani:2022aa,  Rebecca:2022aa}. The simulation displays bar formation \citep[as discussed in][]{Tamfal:2022aa}: this could therefore be another mechanism that will contribute to the gas reservoir in the centre needed for in-situ SF. Figure~\ref{fig:gasstreams} displays the gas surface density\footnote{The depth of the surface density plots is the depth of the simulated box.} at $z=5.0$ with the velocity vectors of the gas overlaid.  We can identify a couple of gas streams coming from the outer part of the galaxy towards the centre. However, further research is needed to determine the importance of this effect. 

The argument for a hybrid formation scenario for the NSC in an MW-like galaxy is further supported by the evolution of the SFR within the galactic centre of the main galaxy, shown in the top panel of Figure~\ref{fig:SFR}, as the SFR is already happening within the central region of the galaxy. Furthermore, because high-mass NSCs have a high ellipticity \citep[e.g.][]{Seth:2006aa, Spengler:2017aa, Georgiev:2014aa}, one could infer that such ellipticity is more easily explained by in-situ SF from inflowing gas \citep[see, e.g.][]{Spengler:2017aa}. This is because elongation could originate from rotational flattening, which requires a dynamically important coherent angular momentum on the star-forming baryons, which is easier to maintain through dissipative accretion in the galactic nucleus than from  non-dissipative merger dynamics of pre-existing stellar clusters.

As a result, we have three channels that contribute to the total stellar mass of the NSC: (i) gas-rich stellar cluster accretion brings in stars formed outside of the galactic nucleus and (ii) contributes  gas  fuel to the central reservoir for subsequent in-situ SF; concurrently, (iii) galactic-scale non-axisymmetric structures such as  bar and spiral structures funnel towards the centre diffuse gas (not previously in clusters), which can then provide fuel for in-situ SF. 

\begin{figure}
\centering
\setlength\tabcolsep{2pt}%
\includegraphics[ trim={0cm 0cm 0cm 0cm}, clip, width=0.47\textwidth, keepaspectratio]{Images/Figure9.pdf}
\caption{The evolution of the SFR within the galactic centre and within the NSR. We defined the galactic centre as a sphere of 100~pc around the centre of mass of the main galaxy and the NSR to be the region between 200~pc and 1~kpc around the galactic centre of the simulated galaxy in the $y$-$z$-plane and with a maximum height of 500~pc.}
\label{fig:SFR}
\end{figure}

\subsection{The nuclear stellar ring}\label{sec:ring}

In addition to the  NSC, it is known that the MW has also an NSD-like structure \citep[e.g.][]{Schodel:2021aa, Schultheis:2021aa}. Figure~\ref{fig:Nring} projects the stellar clusters found within 1.5~kpc from the galactic centre at $z = 4.4$ on the gas surface density of the galaxy. This shows that there is a striking resemblance between the location of the stellar clusters and what one would identify as an NSR. From Section \ref{sec:dynamicaltime}, we determined that most of the NSCPs will have fallen to the galactic centre  by $z = 0$. However, this estimation did not account for mutual interactions between the stellar clusters. Consequently, at $z=0$ there could still be a few of the stellar clusters in the spherical ring surrounding the NSC.

The ring visible in Figure~\ref{fig:Nring} has a larger radius and scale-height than those of the present-day NSD of the MW \citep[R $\approx 220$~pc and h $\approx 50$~pc;][]{Launhardt:2002aa, Nishiyama:2013aa, Nogueras:2020aa, Gallego:2020aa}. Nevertheless, the radius of the NSR could decrease over time due to dynamical friction  strengthened by the bar-induced resonances \citep[see][]{Bortolas:2022aa}. Additionally, a ring forming at the inner ends of bar-torque-produced dust lanes, as shown in \citet{Kim:2012aa}, will shrink in size by 10 to 20 per cent as collisions of dense clumps inside the ring take away angular momentum from the ring. Therefore, we can assume that, at $z=0$, the ring shown in Figure~\ref{fig:Nring} will most likely have a radius and scale-height closer to the MW's NSD's values.

The MW's NSD local orbital inclination relative to the disc’s midplane is between $7^{\circ}$ and $14{^\circ}$, with a mean of $\sim$$10^{\circ}$ \citep[][]{Gillessen:2009aa}. The ring displayed in Figure~\ref{fig:Nring} is on a $\sim$$14^{\circ}$ inclination relative to the disc stars, which is thus in correspondence with the observations of the MW. Furthermore, both in the MW and the simulation, the stellar orbits are on average not circular in the nuclear ring and disc. The mean ellipticity of the MW's nuclear disc inferred by \citet{Bartko:2009aa} is 0.37 $\pm 0.07$, whereas the ellipticity of the ring formed by the NSCPs in the simulation is 0.23.

\begin{figure}
\centering
\setlength\tabcolsep{2pt}%
\includegraphics[ trim={0cm 0cm 0cm 0cm}, clip, width=0.48\textwidth, keepaspectratio]{Images/Figure10.pdf}
\caption{Gas surface density map of the main galaxy at $z=4.4$. The dots represent all identified clusters within $R<1.5$~kpc. The stellar galactic angular momentum is pointing in the $+y$ direction.}
\label{fig:Nring}
\end{figure}

The gas surface density plot in Figure~\ref{fig:Nring} also depicts a gaseous ring surrounding the galactic centre of the main galaxy. This is expected from NSRs, since these rings are most likely produced by the radial infall of gas caused by angular momentum loss as a consequence of  their nonlinear interactions with an underlying stellar bar potential \citep[e.g.][]{Combes:1985aa, Athanassoula:1992aa, Buta:1996aa, Patsis:2010aa, Kim:2012ab}. Tantalizingly, the orbits of the stellar clusters are not aligned with this gaseous ring, which has an inclination relative to the disc stars of  $\sim$$14^{\circ}$.  Numerical simulations of SF in the nuclear ring find that the stellar clusters and gaseous ring overlap, and have a similar inclination relative to the disc's midplane \citep[e.g.][]{Seo:2013aa, Seo:2014aa, Moon:2021aa}. This difference can be explained by the different formation environments of the stellar clusters.  The stellar clusters within the simulations of \citet{Seo:2013aa, Seo:2014aa} were entirely formed within the ring, whereas we show that most of the NSCPs in this simulation include stars that have migrated from the stellar disc towards the nuclear region. Therefore, the inclination of the clusters relative to the gaseous ring could be due to the different environmental effects both systems have experienced and differing initial angular momentum.

The total gas mass of the NSR is $3.4 \times 10^8$~M$_{\sun}$, which is in line with what is observed in NSRs \citep[$\sim$$1$ to $6 \times 10^8$~M$_{\sun}$;][]{Buta:2000aa, Benedict:2002aa, Sheth:2005aa, Schinnerer:2006aa}. The bottom panel of Figure~\ref{fig:SFR} plots the temporal evolution of the SFR in the ring. As expected from the literature, the SFR of the NSR is within a range of $\sim$1 to $20$~M$_{\sun}$~yr$^{-1}$ \citep[][]{Mazz:2008aa, Comeron:2010aa}. We find an SFR in the nuclear ring that is slowly increasing with time, with short bursts of SF with time intervals between the bursts of roughly $\sim$50~Myr.

\section{Discussion}\label{sec:disc}

We explored the possibility of stellar clusters within 1.5~kpc of the main galaxy halo at $z>4$ to be NSCPs, and their link to the thin disc and the NSR, through the use of the $N$-body, hydrodynamical, cosmological ``zoom-in'' simulation GigaEris \citep[][]{Tamfal:2022aa}. There is no guarantee that all the stellar clusters identified as NSCPs in this paper will become part of the NSC at present day, as the calculation in Section~\ref{sec:dynamicaltime} is very rough. In the calculation, we assume a spherically averaged distribution of the galaxy following a \citeauthor{Hernquist:1990aa} profile. This, for example, does not take the interaction between the stellar clusters into account or the influence of galactic structures like the bar and spirals, meaning that in reality the time frame for a cluster to reach the galactic centre could be very different. Nevertheless, as the goal of this paper is to show the possibility of the MW's NSC being formed using the hybrid scenario and the link with the thin-disc stars, making these assumptions for the dynamical friction time-scale is reasonable.

Although we are unable to follow the evolution of the selected stellar clusters to $z=0$ and, therefore, cannot make definitive statements on whether these clusters are truly NSCPs, many of the discussed properties are grossly consistent with the present-day NSC of the MW. One clear difference from observations is the metallicity of the stars in the NSCPs in our simulation, which on average is slightly lower than the average metallicity of the stars in the MW's NSC \citep[e.g.][]{Schodel:2020aa}.  Nevertheless, as we argue for a hybrid formation scenario, the in-situ SF from the gas-rich infalling clusters will increase the average metallicity of the stars in this region.

In the case of late-type host galaxies such as our MW, both the GC and in-situ SF scenario are expected to work in parallel, as supported by their observed metallicity spreads and complex SFHs \citep[e.g.][]{Do:2015aa}. However, the contribution of each mechanism to the predominant build-up process of the MW's NSC is not clear. We find that 60 per cent of the baryonic mass of the NSCPs is stellar. Using the 1.5 and 2 times solar metallicity models for the SFH of the MW's NSC described by \citet{Schodel:2020aa}, we can estimate that between $10$ and $35$ per cent of the total expected stellar mass will form after $z = 4.4$. The total stellar mass at $z=4.4$ of the NSCPs in our simulation is $10^{7.5}$~M$_{\sun}$.  Thus, using the model described by \citet{Schodel:2020aa}, we find that between $10^{6.5}$ and $10^{7.9}$~M$_{\sun}$ stellar mass will form after $z = 4.4$, resulting in a total mass between $10^{7.55}$ and $10^{7.70}$~M$_{\sun}$ at $z=0$. This is in line with \citet{Blum:2003aa},  who found that approximately 80 per cent of the stars in the MW’s Galactic Centre were formed more than 5~Gyr ago. In Section~\ref{sec:hybrid}, we estimated the contribution of stellar mass that will form after $z = 4.4$ from the gas-rich infalling clusters to be between $\sim$$10^{6.1}$ and $10^{7.2}$~M$_{\sun}$, thus between 5 and 50 per cent of the stellar mass contribution of the NSCPs will be in the form of in-situ SF. Nevertheless, one could conclude from Figure~\ref{fig:SFR} that the final mass of the NSC will surpass the estimated mass of maximum $10^{7.70}$~M$_{\sun}$, when the in-situ SF at the centre from other gas is included. However, we assume that only part of the SF at the centre will contribute to the NSC and that most of the stars will be part of the pseudoubulge \citep[as argued in][]{Tamfal:2022aa}. Therefore, the estimated final mass is most likely still in the correct order of magnitude.

From the estimated present-day stellar mass of $10^{7.70}$~M$_{\sun}$, we can also calculate roughly the final effective radius of the NSC at $z=0$ to be between 6.1 and 8.8~pc, using the size-mass relation for late-type galaxies described by \citet{Iskren:2016aa}. Published values for the effective radius of the MW's NSC fall within the range of $4.2 \pm 0.4$ to $7.2 \pm 2.0$~pc \citep[e.g.][]{Schodel:2014aa, Fritz:2016aa}, hence the estimated present-day radius of the simulated NSC in this paper is in accordance with observations.

\subsection{Radial migration of stars}

Most of the stars within the NSCPs were born at around $\sim$1~kpc from the galactic centre. Nevertheless, there is a small group of stars that were born further away. Before all the bound stars of the NSCPs become part of the elliptical ring around the galactic centre, as shown in Figure~\ref{fig:thindisc} at $z =4.4$, the stars born far outside the galactic centre within these clusters must have migrated towards the centre of the galaxy.

We followed the stellar particles for a couple of NSCPs from $z = 6.91$ to $z=4.4$. We find that, on average, roughly $\sim$20~per cent of the stars were born outside the cluster and migrated ``alone''\footnote{In the code, a stellar particle represents an entire stellar population, meaning that when one stellar particle migrates, it is in reality an ensemble of stars. The stars are thus never really alone.} towards the cluster. These stellar particles often formed in the thin disc. For most stars, the migration towards the centre occurs around $zv\sim 6.8$. In the galactic centre they will become part of the main NSCPs and  correspond on average to $\sim$0.05~per cent of the total final mass of the NSCPs. The exact dynamics and kinematics of the radial migration at $z>4$ will be studied in a subsequent paper.

\subsection{Comparison to literature}

While our work is one of the first to discuss the formation of an NSC using a high-resolution, cosmological ``zoom-in'' simulation, our results are mostly consistent with those from previous numerical simulations \citep[e.g.][]{Antonini:2013aa, Seo:2013aa, Seo:2014aa, Gnedin:2014aa, Li:2015aa, Guillard:2016aa, Tsatsi:2017aa,  Seo:2019aa, Sormani:2022aa}. For example, \citet{Gnedin:2014aa} found that for the MW the most probable value for the mass of the central cluster formed by the infall of GCs is 2--$6 \times 10^7$~M$_{\sun}$, which is similar to the total stellar mass of our NSCPs, $3.2 \times 10^7$~M$_{\sun}$.

Furthermore, in our simulation we find an SFR in the nuclear ring that is slowly increasing with time, with short bursts of SF between $\sim$1 and $20$~M$_{\sun}$~yr$^{-1}$. The time intervals between the bursts are roughly $\sim$50~Myr. This is consistent with \citet{Seo:2013aa}, who found that the SFR in nuclear rings displays a single primary burst followed by a few secondary bursts with a time interval between the bursts to be roughly $\sim$50--80~Myr. Interestingly, both \citet{Seo:2013aa} and \citet{Seo:2019aa} found that the SFR will decrease during the simulation, whereas we see an increase in SFR over time. In \citet{Seo:2013aa}, the SFR decreases to small values after the second burst. This difference is probably due to the fact that they start with an infinitesimally thin, rotating disc, which is unmagnetized and isothermal. We, on the other hand, used a high-resolution, cosmological ``zoom-in'' simulation of an MW-sized galaxy cluster to investigate the NSR. It could be that \citet{Seo:2013aa} showed what would happen after $z=4.4$, as our simulation ends with a thin-disc-like structure.

The nuclear ring detected in the simulation has a larger radius than that of the present-day NSD of the MW, but also larger than the simulations performed by for example \citeauthor{Seo:2019aa} (\citeyear{Seo:2019aa}; less than $\sim$0.6~kpc). Nevertheless, \citet{Li:2017aa} used hydrodynamic simulations with static stellar potentials to show that a nuclear ring forms only in models with a central object exceeding $\sim$1 per cent of the total disc mass, and that the ring size increases almost linearly with the mass of the central object. As the simulation already has a dense central region, from which most of the stars will become part of the pseudoubulge, it opens the possibility that the presence of a massive compact star-forming region or pseudobulge would make a ring large when it first forms. The formed NSR can even become larger as it grows due to an addition of gas with larger angular momentum from outer regions, as discussed in \citet{Seo:2019aa}.

\section{Conclusions}\label{sec:conc}

Using a high-resolution, cosmological ``zoom-in'' simulation of an MW-sized galaxy halo \citep[GigaEris;][]{Tamfal:2022aa}, we have shown the possibility of stellar clusters within 1.5~kpc of the main galaxy halo at $z>4$ to be NSCPs, and their connection to the thin disc. Moreover, we explored the relationship between the NSC and the NSD. Our main conclusions are as follows:

\begin{itemize}

    \item We define NSCPs as clusters within 1.5 kpc from the centre of the main galaxy and a $t_{\rm DF} <12$~Gyr. The total stellar mass of the clusters together, $\sim$$10^{7.5}$~M$_{\sun}$, is in the same order of magnitude as the observed mass of the MW's NSC, $\sim$$10^{7.4}$~M$_{\sun}$.
    
    \item NSCPs at $z=4.4$ have a relatively low stellar metallicity, $-0.47 \lesssim {\rm [Fe/H]} \lesssim -0.11$, in comparison to the stars in the MW's NSC. Nevertheless, this can be increased by the in-situ SF later during the formation of the NSC. The detected stellar metallicity  makes the selected star clusters in this paper very different from the proto-GCs discussed in \citetalias{Donkelaar:2022ab}.
    
    \item The NSCPs have a low oxygen-to-iron ratio,  $0.02 \lesssim {\rm [O/Fe]} \lesssim 0.26$, which points to the fact that a fraction of the stars within the clusters are born within the thin disc \citep[e.g][]{Reddy:2006aa, Bertan:2016aa,Franchini:2021aa}. Combining the birth environment of the stars with the oxygen-to-iron ratio, we have determined that the NSCPs consist of a minimum of 3 per cent out of ``born thin-disc'' stars, and for 48 per cent of the clusters the ``born thin-disc'' stars make up more than half of the total stars at $z = 4.4$.
    
    \item The NSCPs form an elliptical ring around the galactic centre of the main galaxy halo at $z = 4.4$. The NSCPs with $F_{\rm thin} > 0.5$ can be found in different regions than the low-$F_{\rm thin}$ clusters in this ring and have a higher iron-to-hydrogen ratio. Furthermore, the NSCPs with $F_{\rm thin} > 0.5$ have a higher mean age of stars in the cluster than the other clusters.
    
    \item The hybrid formation scenario for the NSC of an MW-like galaxy is the most likely. This is the result of three channels that contribute to the total stellar mass of the NSC. Gas-rich stellar cluster accretion brings in stars formed outside of the NSC and adds to the gas reservoir in the centre needed for in-situ SF. Conjointly, the galactic structures like the disc and the bar will funnel the gas towards to centre, which can also be used for in-situ SF.
    
    \item  Assuming that all the gas from the high-mass infalling stellar clusters with a $t_{\rm DF} \leq 1$~Gyr will turn into stars within the NSCPs, we find that in-situ SF contributes to a maximum of 30 per cent  of the stellar mass of the NSC cluster, only including the baryonic mass added from the NSCPs, of an MW-like galaxy.
    
    \item We detect an NSR within the GigaEris simulation with a radius of $\sim$500~pc at $z = 4.4$. This ring hosts the NSCPs at this redshift and has periodic SF short bursts of $\sim$$10$~M$_{\sun}$~yr$^{-1}$, with time intervals of $\sim$50~Myr.
    
\end{itemize}

\section*{Acknowledgements}
PRC, LM, and FvD acknowledge support from the Swiss National Science Foundation under the Grant 200020\_207406. FvD would like to thank Katja Fahrion for initiating the idea that the clusters in the centre could be part of the NSC and useful discussion along the way. 

%%%%%%%%%%%%%%%%%%%%%%%%%%%%%%%%%%%%%%%%%%%%%%%%%%
\section*{Data Availability}
The data underlying this article will be shared on reasonable request to the corresponding author.

%%%%%%%%%%%%%%%%%%%% REFERENCES %%%%%%%%%%%%%%%%%%

% The best way to enter references is to use BibTeX:

\bibliographystyle{mnras}
\bibliography{paper} % if your bibtex file is called example.bib

% Alternatively you could enter them by hand, like this:
% This method is tedious and prone to error if you have lots of references
%\begin{thebibliography}{99}
%\bibitem[\protect\citeauthoryear{Author}{2012}]{Author2012}
%Author A.~N., 2013, Journal of Improbable Astronomy, 1, 1
%\bibitem[\protect\citeauthoryear{Others}{2013}]{Others2013}
%Others S., 2012, Journal of Interesting Stuff, 17, 198
%\end{thebibliography}

%%%%%%%%%%%%%%%%%%%%%%%%%%%%%%%%%%%%%%%%%%%%%%%%%%

%%%%%%%%%%%%%%%%% APPENDICES %%%%%%%%%%%%%%%%%%%%%

%%%%%%%%%%%%%%%%%%%%%%%%%%%%%%%%%%%%%%%%%%%%%%%%%%

% Don't change these lines
\bsp	% typesetting comment
\label{lastpage}
\end{document}

% End of mnras_template.tex
