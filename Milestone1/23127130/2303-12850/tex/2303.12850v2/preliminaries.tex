\section{Preliminaries}
%\knote{Define integrality gap. Note: integrality gap is the worst-case gap among all ``non-negative'' cost functions.} 
For a minimization IP $\min\{c^Tx: Ax\le b, x\in \Z^n_{\geq 0}\}$, the \emph{integrality gap} of its LP-relaxation $\min\{c^Tx: Ax\le b, x\geq 0\}$ is the maximum ratio, over all non-negative cost functions, of the minimum cost of a feasible solution to the IP and that of the LP, i.e., $\max_{c\in \R^n} \min\{c^Tx: Ax\le b, x\in \Z^n_{\geq 0}\}/\min\{c^Tx: Ax\le b, x \geq 0\}$.
%\knote{Define $\alpha$-approximation with respect to a LP.}
%\knote{Glossary of all polyhedra and all IPs and LPs?}
%\knote{State only those preliminaries which are necessary for all sections. } 
For a graph $G = (V, E)$, we will use $\delta(A, B)$ to denote the set of edges with exactly one end-vertex in $A$ and exactly one end-vertex in $B$.
We summarize the results of Lin, Feng, Fu, and Wang \cite{LFFW19} in a manner that will be useful for our polyhedral study.  
\begin{theorem}\label{thm:PFDS-WD-and-2PT-cover}
(\ref{PFDS-IP: WD}) and (\ref{PFDS-IP: WD-and-2PT-cover}) are ILP formulations for PFDS. Furthemore, the LP-relaxation (\ref{PFDS-LP: WD-and-2PT-cover}) has integrality gap at most $2$. 
\end{theorem}
We emphasize that Lin, Feng, Fu, and Wang did not prove \Cref{thm:PFDS-WD-and-2PT-cover} directly, but their local ratio based algorithm and standard techniques to convert local ratio based algorithms to LP-based primal-dual algorithms (following the ideas in \cite{CHUDAK1998111}) lead to the theorem.  