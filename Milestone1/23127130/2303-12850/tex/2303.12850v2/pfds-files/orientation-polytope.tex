% \section{Polynomial-time solvable LP-relaxation for PFDS with integrality gap at most $2$}\label{sec:orientatin-polytope}
In this section, we restate and prove \Cref{lemma:orientation-formulation}. 

\lemmaOrientationFormulation*
\begin{proof}
In order to show that (\ref{PFDS-IP:orient}) and (\ref{PFDS-IP:Orient-and-2PT-cover}) are ILP formulations for PFDS, it suffices to show that the ILP (\ref{PFDS-IP:orient}) is a formulation for PFDS. This is because $2$-pseudotree cover constraints are valid for PFDS. 


For a subgraph $\Tilde{G} = (\Tilde{V}, \Tilde{E})$ of the graph $G$, we let $d_{\Tilde{G}}(u):=|\delta(u)\cap \Tilde{E}|$ denote the degree of the vertex $u$ in the subgraph $\Tilde{G}$. We now define an intermediate polyhedron $\weakdensitypolyhedronforsubgraphs(G)$.
\begin{align}
\weakdensitypolyhedronforsubgraphs(G) 
    &:= 
    \left\{ x\in \R^{V}_{\ge 0}:\ \sum_{u\in \Tilde{V}}(d_{\Tilde{G}}(u)-1)x_u\geq |\Tilde{E}| - |\Tilde{V}| \ \forall \text{ subgraphs $\Tilde{G} = (\Tilde{V}, \Tilde{E})$ of $G$}.
\right\}. \label{eqn:weak-density-for-subgraphs}
\end{align}
We note that $\weakdensitypolyhedronforsubgraphs(G) \subseteq \weakdensitypolyhedron(G)$. We will use the following claim to show that the ILP  (\ref{PFDS-IP:orient}) formulates PFDS and also to conclude that $\projectedorientationpolyhedron(G)\subseteq \weakdensitypolyhedron(G)$. 

\begin{claim}\label{claim:orientation-inside-weak-density-subgraphs}
$\projectedorientationpolyhedron(G)\subseteq \weakdensitypolyhedronforsubgraphs(G)$.
\end{claim}
\begin{proof}
Let $x\in \projectedorientationpolyhedron(G)$. Then, there exists a vector $y$ such that $(x, y)\in \orientationpolyhedron(G)$. Let $\Tilde{G} = (\Tilde{V}, \Tilde{E})$ be an arbitrary subgraph of $G$. We have the following:
\begin{align*}
|\Tilde{E}| &\leq \sum_{e = vw \in \Tilde{E}} (x_v + x_w + y_{e,v} + y_{e,w})\\
       &= \sum_{v \in \Tilde{V}} (d_{\Tilde{G}}(v)-1) x_v + \sum_{v \in \Tilde{V}} \left( x_v + \sum_{e \in \delta_{\Tilde{G}}(v)} y_{e,v}\right)\\ 
       &\leq \sum_{v \in \Tilde{V}} (d_{\Tilde{G}}(v)-1) x_v + |\Tilde{V}|,
\end{align*}
where the inequalities are because the vector $(x,y) \in \orientationpolyhedron$. Rearranging the above inequality gives the constraint for $\tilde{G}$ given in $\weakdensitypolyhedronforsubgraphs(G)$. Hence, $x\in \weakdensitypolyhedronforsubgraphs(G)$.      
\end{proof}

We remark that \Cref{claim:orientation-inside-weak-density-subgraphs} can be strengthened to show that $\projectedorientationpolyhedron(G)= \weakdensitypolyhedronforsubgraphs(G)$, but we will not need this stronger version for the purposes of this theorem. We now show that the ILP (\ref{PFDS-IP:orient}) formulates PFDS. 
Let $S \subseteq V$ be a pseudoforest deletion set, i.e., 
 the subgraph $F := G - S$ is a pseudoforest. Let $x := \chi^S \in \{0,1\}^{V}$ denote the indicator vector of the set $S$.  
 Select an arbitrary orientation of the pseudoforest $F$ such that the maximum indegree of every vertex is at most $1$, and let $y_{e,v} := 1$ for the edge $e = vw$ if $e$ is an edge of $F$ that is oriented towards $v$, and $y_{e,v} := 0$ otherwise. 
 Then, $(x,y) \in \orientationpolyhedron(G)$. 

Next, suppose $x\in \projectedorientationpolyhedron(G)\cap \Z^V$. Then, we have that $x\in \{0,1\}^V$. By \Cref{claim:orientation-inside-weak-density-subgraphs}, $x\in \weakdensitypolyhedronforsubgraphs(G)\subseteq \weakdensitypolyhedron(G)$. 
Since (\ref{PFDS-IP: WD}) is an ILP formulation for PFDS by \Cref{thm:PFDS-WD-and-2PT-cover}, it follows that the set $S:=\{u\in V: x_u =1\}$ is a pseudoforest deletion set for the graph $G$. This concludes the proof that the ILP (\ref{PFDS-IP:orient}) formulates PFDS. 

We now prove the two additional conclusions of the theorem statement. 

\begin{enumerate}[label=(\arabic*)]
\item By Claim \ref{claim:orientation-inside-weak-density-subgraphs}, we have that $\projectedorientationpolyhedron(G)\subseteq \weakdensitypolyhedronforsubgraphs(G)\subseteq \weakdensitypolyhedron(G)$. 
We now show that there is a graph $G$ such that $\weakdensitypolyhedronforsubgraphs(G)\subsetneq \weakdensitypolyhedron(G)$. In particular, we consider the graph $K_5 = (V, E)$ where $V = \{v_1, v_2, \ldots, v_5\}$ and $E = {V \choose 2}$. Let $x = (2/3, 2/3, 1/3, 0, 0)$. We note that $x \in \weakdensitypolyhedron(K_5)$.
We now show that $x \not \in \weakdensitypolyhedronforsubgraphs(K_5)$.
    Consider the subgraph $\Tilde{G} = (\Tilde{V}, \Tilde{E})$ obtained by removing the edge $\{v_1, v_2\}$ from the graph $G$, i.e. $\Tilde{V} = V$ and $\Tilde{E} = {V\choose 2} - \{v_1, v_2\}$.
    Then, we have the following:
    %$$\sum_{u \in \Tilde{V}}(d_{\Tilde{G}}(u) - 1)x_u = 3x_1 + 3x_2 + 4x_3 + 4x_4 + 4x_5 = \frac{46}{12} < 4 = |\Tilde{E}| - |\Tilde{V}|.$$
    $$\sum_{u \in \Tilde{V}}(d_{\Tilde{G}}(u) - 1)x_u = 2x_1 + 2x_2 + 3x_3 + 3x_4 + 3x_5 = \frac{11}{3} < 4 = |\Tilde{E}| - |\Tilde{V}|.$$
    
    In particular, the vector $x$ does not satisfy the constraint of $\weakdensitypolyhedronforsubgraphs(K_5)$ defined by the subgraph $\Tilde{G}$. Thus, we have that $\projectedorientationpolyhedron(K_5) \subseteq \weakdensitypolyhedronforsubgraphs(K_5)\subset \weakdensitypolyhedron(K_5)$. 

    \item By \Cref{thm:PFDS-WD-and-2PT-cover}, the ILP (\ref{PFDS-IP: WD-and-2PT-cover}) is a valid formulation of PFDS and its LP-relaxation (\ref{PFDS-LP: WD-and-2PT-cover}) has integrality gap at most $2$. We have already shown that the ILP (\ref{PFDS-IP:Orient-and-2PT-cover}) is a valid formulation of PFDS and that $\projectedorientationpolyhedron(G)\subseteq \weakdensitypolyhedron(G)$. Consequently, the integrality gap of (\ref{PFDS-LP:Orient-and-2PT-cover}) is at most $2$. Finally, we note that (\ref{PFDS-LP: WD-and-2PT-cover}) can be solved in polynomial time because $\projectedorientationpolyhedron(G)$ is the projection of $\orientationpolyhedron(G)$ which admits a polynomial sized description, and because 2-pseudotree-cover constraints admit a polynomial time separation oracle by \Cref{thm:MC2PT-polytime:main} (in \Cref{sec:2pt-cover-constraints-separation-oracle}). The polynomial time separation oracle for 2-pseudotree-cover constraints is based on the fact that node-weighted Steiner tree for constant number of terminals is solvable in polynomial time. We present its proof in a separate subsection for clarity. 

%\item The third conclusion follows from \Cref{thm:MC2PT-polytime:main} in \shubhang{the appendix}. It is based on the fact that node-weighted Steiner tree for constant number of terminals is solvable in polynomial time. We present its proof in a separate subsection for clarity. 
\end{enumerate}
\end{proof}