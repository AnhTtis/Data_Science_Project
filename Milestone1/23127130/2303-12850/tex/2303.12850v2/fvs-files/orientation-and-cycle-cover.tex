% \subsection{Orientation and Cycle Cover based Formulation}\label{sec:orientation-and-cycle-cover-formulation}
In this section, we give a polynomial-sized ILP for FVS based on orientation and cycle cover constraints. Moreover, we show that the integrality gap of the LP-relaxation of this formulation is at most $2$. 
We consider the following formulation: 

\begin{align}
\min\left\{\sum_{u\in V}c_u x_u: x\in \projectedorientationpolyhedron(G)\cap \cyclecoverpolyhedron(G)\cap \Z^V \right\}. \tag{FVS-IP: orient-and-cycle-cover} \label{FVS-IP: orient-and-cycle-cover}
\end{align}

%Our ILP is based on the following lemma.
\begin{lemma}\label{lemma:integrality-gap-orient-intersect-cycle-cover}
For an input graph $G=(V, E)$ with non-negative vertex costs $c: V\rightarrow \R_{\ge 0}$, \eqref{FVS-IP: orient-and-cycle-cover} is an integer linear programming formulation for FVS. Moreover, its LP-relaxation has integrality gap at most $2$. 
\end{lemma}
\begin{proof}
The indicator vector $x$ of a feedback vertex set of $G$ is contained in $\projectedorientationpolyhedron(G)\cap\cyclecoverpolyhedron(G)$. Moreover, every integral solution $x\in \projectedorientationpolyhedron(G)\cap\cyclecoverpolyhedron(G)$ is the indicator vector of a feedback vertex set: since $x\in \projectedorientationpolyhedron(G)\cap \Z^V$, we have that $x\in \{0,1\}^V$ and since $x\in \cyclecoverpolyhedron(G)$, we have that $x$ is the indicator vector of a feedback vertex. 

We now show that the integrality gap of the LP-relaxation is at most $2$. 
By \Cref{lemma:orientation-formulation}(1), we have that $\projectedorientationpolyhedron(G)\subseteq \weakdensitypolyhedron(G)$ and hence, the integrality gap of the LP-relaxation is at most the integrality gap of (\ref{FVS-LP:weak-density-cycle-cover}). By Theorem \ref{thm:integrality-gap-wd+cycle-cover}, the integrality gap of (\ref{FVS-LP:weak-density-cycle-cover}) is at most $2$. 
\end{proof}

\Cref{thm:poly-sized-LP-with-integrality-gap-atmost-2-for-FVS}  follows from \Cref{lemma:integrality-gap-orient-intersect-cycle-cover}, the observation that  $\projectedorientationpolyhedron(G)$ can be expressed using polynomial number of variables and constraints (see the description of $\orientationpolyhedron(G)$), and the fact that the cycle cover polyhedron can equivalently be described using polynomial number of variables and constraints. We note that this fact is folklore, but we include a proof of it in \Cref{lemma:cycle-cover-poly-sized-lp} in the appendix.



