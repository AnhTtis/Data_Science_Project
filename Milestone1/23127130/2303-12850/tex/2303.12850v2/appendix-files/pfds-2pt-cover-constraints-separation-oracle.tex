%\subsection{Separation oracle for the family of $2$-pseudotree cover constraints}
%\chandra{Change title to Polynomial-time? Use polynomial-time with hyphen since we use it for space later?}
In this section, we show that the family of $2$-pseudotree cover constraints admits a polynomial time separation oracle.

\begin{restatable*}{lemma}{NWtwoPTPolytime}
\label{thm:MC2PT-polytime:main}
The family of $2$-pseudotree cover constraints admits a polynomial-time separation oracle. 
%The MC2PT problem can be solved in polynomial time.
\end{restatable*}

In order to solve the separation problem, it suffices to design a polynomial time algorithm for the \emph{Minimum Cost $2$-Pseudotree} problem:  The input to MC2PT is a vertex-weighted graph $\left(G = \left(V,E\right), w:V\rightarrow\R_{\geq 0}\right)$, and the goal is to compute a minimum weight subset $U\subseteq V$ of vertices such that $G[U]$ is a $2$-pseudotree, i.e., $$\min\left\{\sum \nolimits_{u\in U}w(u): U\subseteq V \text{ and } G[U] \text{ is a }2\text{-pseudotree}\right\}.$$ 
Our strategy to solve MC2PT is to reduce it to solving a polynomial number of special instances of the \emph{Minimum Node-Weighted Steiner Tree}\footnote{We refer to \emph{nodes} as \emph{vertices} for consistency with the rest of our technical sections.} (NWST), and use the result of \cite{NWST-Buchanan-et.al} which says that these special instances of NWST can be solved in polynomial time. Formally, the input to NWST is a vertex-weighted graph $\left(G = \left(V,E\right), w:V\rightarrow\R_{\geq 0}\right)$ and a terminal set $S \subseteq V$. We will say that a graph $H$ is a \emph{Steiner tree} in $G$ for terminal set $S$ if $H$ is connected, acyclic, and is a subgraph of $G$ with $S\subseteq V_H$. Moreover, the weight of a subgraph $H$ of $G$ is the sum of weights of vertices in $H$. The NWST problem is to find a  minimum weight Steiner tree in $G$ for terminal set $S$, i.e., $\min\{\sum_{u\in V_H} w(u): H=(V_H, E_H) \text{ is a Steiner tree in } G \text{ for terminal set } S\}$.
NWST can be solved in polynomial time if the number of terminals is a constant.

\begin{proposition}[Theorem 1 of \cite{NWST-Buchanan-et.al}]\label{prop:NWST-polytime} There exists a $O(3^kn + 2^kn^2+n^3)$ time algorithm for NWST, where $k$ denotes the number of terminals and $n$ denotes the number of vertices in the input instance.
\end{proposition}

The following proposition shows a correspondence between minimum weight $2$-pseudotrees in a vertex-weighted graph and Steiner trees.

\begin{proposition}\label{prop:MC2PT-to-NWST}
    Let $G=(V, E)$ be a graph with non-negative vertex weights $w: V\rightarrow \R_{\ge 0}$ and let $W\ge 0$. Then, there exists a subset $U\subseteq V$ such that $G[U]$ is a $2$-pseudotree with $\sum_{u\in U} w(u)\le W$ if and only if there exists a pair of edges $e_1=u_1v_1, e_2=u_2v_2$ in $G$ such that there exists a Steiner tree $H=(V_H, E_H)$ in the graph $G':=G-\{e_1, e_2\}$ for  terminal set $S:=\{u_1, v_1, u_2, v_2\}$ with $\sum_{u\in V_H}w(u)\le W$.
\end{proposition}
\begin{proof}
We will say that a graph $T=(V_T, E_T)$ is a \emph{minimal} $2$-pseudotree if it is connected and has exactly $|V_T|+1$ edges. Equivalently, $T$ has exactly $2$ edges in addition to a spanning tree. We observe that for a subset $U\subseteq V$, the subgraph $G[U]$ is a $2$-pseudotree if and only if there exists a subgraph $T=(U, E_T)$ of $G[U]$ such that $T$ is a minimal $2$-pseudotree. Hence, it suffices to show that there exists a subset $U\subseteq V$ such that $G[U]$ has a subgraph that is a minimal $2$-pseudotree with $\sum_{u\in U} w(u)\le W$ if and only if there exists a pair of edges $e_1=u_1v_1, e_2=u_2v_2$ in $G$ such that there exists a Steiner tree $H=(V_H, E_H)$ in the graph $G':=G-\{e_1, e_2\}$ for  terminal set $S:=\{u_1, v_1, u_2, v_2\}$ with $\sum_{u\in V_H}w(u)\le W$. We prove this statement now. 

Let $U\subseteq V$ such that $G[U]$ contains a subgraph $T=(U, E_T)$ that is a minimal $2$-pseudotree with $\sum_{u\in U} w(u)\le W$.  
Then, there exists a pair of edges $e_1=u_1v_2$ and $e_2=u_2v_2$ in $T$ such that the subgraph $H:=T-\{e_1, e_2\}$ is acyclic, connected, and is a subgraph of $T$. In particular, we have that $H$ is acyclic, connected, and is a subgraph of $G':=G-\{e_1, e_2\}$ with $S=\{u_1, v_1, u_2, v_2\}\subseteq U=V(H)$. Hence, for the pair of edges $e_1, e_2$ in $G$, we have that $H$ is a Steiner tree in $G'$ for terminal set $S$ with $\sum_{u\in V(H)}w(u)=\sum_{u\in U} w(u) \le W$. 

Next, let $e_1=u_1v_2, e_2 = u_2v_2$ be a pair of edges in $G$ such that there exists a Steiner tree $H=(V_H, E_H)$ in the graph $G':=G-\{e_1, e_2\}$ for terminal set $S=\{u_1, v_1, u_2, v_2\}$ with $\sum_{u\in V_H}w(u)\le W$. Consider the graph $T:=H+\{e_1, e_2\}$. Then, $T$ is a minimal $2$-pseudotree. The vertex set of $T$ is $U:=V_H$ and $T$ is a subgraph of $G[U]$. Hence, we have a subset $U\subseteq V$ such that $G[U]$ has a subgraph that is a minimal $2$-pseudotree with $\sum_{u\in U}w(u)\le W$. 
\end{proof}

We now restate and prove \Cref{thm:MC2PT-polytime:main}.

\NWtwoPTPolytime
\begin{proof}
    It suffices to show that MC2PT can be solved in polynomial time. Let $G=(V,E)$ with vertex weights $w:V\rightarrow \R_{\ge 0}$ be the input instance of MC2PT.
    Consider the following algorithm: for all pairs of edges $e_1 = u_1v_1$ and $e_2 = u_2v_2$ in $E$, use the algorithm guaranteed by \Cref{prop:NWST-polytime} to solve NWST on the graph $G - \{e_1, e_2\}$ with vertex weights $w$ for terminal set $\{u_1, v_1, u_2, v_2\}$, and return the minimum weight solution over all instances. The correctness of the algorithm follows from \Cref{prop:MC2PT-to-NWST}. We now analyze the runtime of the algorithm. There are $O(|V|^2)$ pairs of edges to enumerate. Furthermore, for each pair of edges, the associated NWST instance can be solved in polynomial time using the algorithm from \Cref{prop:NWST-polytime} since each such instance has only four terminals. Thus, the algorithm runs in polynomial time.
\end{proof}