%\subsection{Polynomial-sized formulation of $\cyclecoverpolyhedron(G)$}\label{sec:cycle-cover-poly-sized-formulation}
In this section, we show that $\cyclecoverpolyhedron(G)$ can be expressed using polynomial number of variables and constraints. 
This is folklore, but we include its proof for the sake of completeness. For a graph $G=(V, E)$, we define $E':=\{(e, s), (e, t):\ \exists \text{ cycle in $G$ containing } e = st \in E\}$ and consider
\begin{equation}\label{eqn:distance-cycle-cover}
    \distancebasedcyclecoverpolyhedron(G) := \left\{ (x,d)\in \R^V_{\ge 0} \times \R^{V\times E'}_{\ge 0}:\ \begin{array}{l}
{d_s^{(e, s)} = 0} \hfill {\qquad \forall (e=st, s)\in E'} \\
 {d_t^{(e, s)} + x_s \ge 1} \hfill  \forall (e=st, s)\in E' \\
 {d_a^{(e,s)} + x_b \ge d_b^{(e, s)} }\hfill \qquad \forall (e, s)\in E', \ ab\in E - \{e\}
  \end{array}\right\}.  
\end{equation}

\begin{lemma}\label{lemma:cycle-cover-poly-sized-lp}
For every graph $G=(V, E)$, the polyhedron $\cyclecoverpolyhedron(G)$ is the projection of the polyhedron $\distancebasedcyclecoverpolyhedron(G)$ to $x$ variables. Consequently, $\cyclecoverpolyhedron(G)$ admits a polynomial-sized description. 
\end{lemma}
\begin{proof}
Let $\projecteddistancebasedcyclecoverpolyhedron(G)$ be the projection of $\distancebasedcyclecoverpolyhedron(G)$ to the $x$ variables. We prove that $\projecteddistancebasedcyclecoverpolyhedron(G)=\cyclecoverpolyhedron(G)$ by showing inclusion in both directions. 

First, we show that $\projecteddistancebasedcyclecoverpolyhedron(G)\subseteq \cyclecoverpolyhedron(G)$. Let $x\in \projecteddistancebasedcyclecoverpolyhedron(G)$. Let $d\in \R^{V\times E'}_{\ge 0}$ be a vector such that $(x,d)\in \distancebasedcyclecoverpolyhedron(G)$. It suffices to show that for every cycle $C$ in $G$, we have that $\sum_{u\in V(C)}x_u\ge 1$. Let $C$ be a cycle in $G$. Fix an edge $e=st\in E(C)$. By definition of the set $E'$, we have that $(e, s)\in E'$. Consequently, $d^{(e,s)}_s=0$ and $d^{(e,s)}_t + x_s\ge 1$ by the first two constraints of $\distancebasedcyclecoverpolyhedron(G)$. Let $a_1=s, a_2, a_3, \ldots, a_{r-1}, a_r=t$ be the ordered vertices along the $s-t$ path induced by $E(C) - e$. Then, for every $i\in [r-1]$, we have that $d^{(e,s)}_{a_i}+x_{a_{i+1}}\ge d^{(e,s)}_{a_{i+1}}$ by the third constraint of $\distancebasedcyclecoverpolyhedron(G)$. Adding all of these constraints gives us that $\sum_{i=2}^r x_{a_i}\ge d^{(e,s)}_{a_r}=d^{(e,s)}_t\ge 1-x_{s}=1-x_{a_1}$. Thus, the constraint $\sum_{i=1}^r x_{a_i}\ge 1$ holds. 

Next, we show that $\projecteddistancebasedcyclecoverpolyhedron(G)\supseteq \cyclecoverpolyhedron(G)$. Let $x\in \cyclecoverpolyhedron(G)$. Let $d\in \R^{V\times E'}_{\ge 0}$ be defined as follows: for each $(e=st, s)\in E'$, let 
\[
d^{(e,s)}_u:=\min\left\{\sum_{v\in P-\{s\}}x_v:\ P \text{ is a path from $s$ to $u$ 
and $P\neq \{s, t\}$
}\right\}.
\]
We show that $(x, d)\in \distancebasedcyclecoverpolyhedron(G)$. We note that the vector $(x,d)$ is non-negative by definition. Let $(e=st, s)\in E'$. Then,  $d^{(e, s)}_s=0$ by definition. 
Moreover, $d^{(e,s)}_t+x_s\ge 1$ holds because of the following reasoning: let $P$ be a path from $s$ to $t$ such that $d^{(e,s)}_t=\sum_{v\in P-\{s\}}x_v$ and $P\neq \{s, t\}$. Then, $P$ concatenated with the edge $st$ forms a cycle $C$ and $d^{(e,s)}_t + x_s = x_s+\sum_{v\in P-\{s\}}x_v = \sum_{u\in V(C)}x_u \ge 1$. 
Finally, the inequality $d^{(e,s)}_a + x_b \ge d^{(e,s)}_b$ for every edge $ab\in E$ holds because of the following reasoning: for an edge $ab\in E$, let $P$ be a path from $s$ to $a$ such that $d^{(e,s)}_a=\sum_{v\in P-\{s\}}x_v$. Then, $P$ concatenated with $b$ is a path from $s$ to $b$ and consequently, $d^{(e,s)}_b\le \sum_{v\in P+\{b\}-\{s\}}x_v$ by definition.  
\end{proof}

%% old proof
\iffalse
\begin{lemma}\label{lemma:cycle-cover-poly-sized-lp}
Let $G=(V, E)$ and $E':=\{e\in E:\ \exists \text{ cycle in $G$ containing } e\}$. The cycle cover polyhedron $\cyclecoverpolyhedron(G)$ is the projection of the following polyhedron to $x$ variables:   
\begin{equation}\label{eqn:distance-cycle-cover}
    \distancebasedcyclecoverpolyhedron(G) := \left\{ (x,d)\in \R^V_{\ge 0} \times \R^{V\times E'}_{\ge 0}:\ \begin{array}{l}
{d_s^e = 0} \hfill {\qquad \forall e=st\in E'} \\
 {d_t^e + x_s \ge 1} \hfill \qquad \forall e=st\in E' \\
 {d_a^e + x_b \ge d_b^e }\hfill \qquad \forall ab\in E,\ e\in E'
  \end{array}\right\}, 
\end{equation}
Consequently, $\cyclecoverpolyhedron(G)$ admits a polynomial-sized description. 
\end{lemma}
\begin{proof}
Let $\projecteddistancebasedcyclecoverpolyhedron(G)$ be the projection of $\distancebasedcyclecoverpolyhedron(G)$ to the $x$ variables. We prove that $\projecteddistancebasedcyclecoverpolyhedron(G)=\cyclecoverpolyhedron(G)$. 

We first show that $\projecteddistancebasedcyclecoverpolyhedron(G)\subseteq \cyclecoverpolyhedron(G)$. Let $x\in \projecteddistancebasedcyclecoverpolyhedron(G)$. Let $d\in \R^{V\times E'}_{\ge 0}$ be a vector such that $(x,d)\in \distancebasedcyclecoverpolyhedron(G)$. It suffices to show that for every cycle $C$ in $G$, we have that $\sum_{u\in V(C)}x_u\ge 1$. Let $C$ be a cycle in $G$. Fix $e=st\in E(C)$. Then, $e\in E'$ and hence, $d^e_s=0$, and $d^e_t + x_s\ge 1$. Let $a_1=s, a_2, a_3, \ldots, a_{r-1}, a_r=t$ be the vertices of the cycle in that order. Then, we have the constraint $d^e_{a_i}+x_{a_{i+1}}\ge d^e_{a_{i+1}}$ for every $i\in [r-1]$. Adding these constraints gives us that $\sum_{i=2}^r x_{a_i}\ge d^e_{a_r}=d^e_t\ge 1-x_{s}=1-x_{a_1}$. Thus, the constraint $\sum_{i=1}^r x_{a_i}\ge 1$ holds. 

Next, we show that $\projecteddistancebasedcyclecoverpolyhedron(G)\supseteq \cyclecoverpolyhedron(G)$. Let $x\in \cyclecoverpolyhedron(G)$. Let $d\in \R^{V\times E'}_{\ge 0}$ be defined as follows: for each $e=st\in E'$, let 
\[
d^e_u:=\min\left\{\sum_{v\in P-\{s\}}x_v:\ P \text{ is a path from $s$ to $u$ and $P\neq \{s, t\}$}\right\}.
\]
We show that $(x, d)\in \distancebasedcyclecoverpolyhedron(G)$. The vector $(x,d)$ is non-negative by definition. Let $e=st\in E'$. Then,  $d^e_s=0$ by definition. 
Moreover, $d^e_t+x_s\ge 1$ due to the following reasoning: let $P$ be a path from $s$ to $t$ such that $d^e_t=\sum_{v\in P-\{s\}}x_v$ and $P\neq \{s, t\}$. Then, $P$ concatenated with the edge $st$ forms a cycle $C$ and $d^e_t + x_s = x_s+\sum_{v\in P-\{s\}}x_v = \sum_{u\in V(C)}x_u \ge 1$. 
Finally, $d^e_a + x_b \ge d^e_b$ for every edge $ab\in E$ due to the following reasoning: for an edge $ab\in E$, let $P$ be a path from $s$ to $a$ such that $d^e_a=\sum_{v\in P-\{s\}}x_v$. Then, $P$ concatenated with $b$ is a path from $s$ to $b$ and consequently, $d^e_b\le \sum_{v\in P+\{b\}-\{s\}}x_v$ by definition.  
\end{proof}
\fi