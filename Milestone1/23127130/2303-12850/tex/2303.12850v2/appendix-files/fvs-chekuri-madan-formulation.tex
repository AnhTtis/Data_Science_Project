% \subsection{Chekuri-Madan's Formulation}
%\chandra{Use the abbreviation CM rather than repeat the names.}
In this section, we show that the polynomial-sized ILP for FVS given by Chekuri and Madan 
%(henceforth referred to as the CM-FVS ILP) 
is such that its LP-relaxation has integrality gap at most $2$. Chekuri and Madan formulated a polynomial-sized ILP for the more general problem of Subset Feedback Vertex Set (SFVS). SFVS is defined as follows: The input is a graph $G=(V, E)$ with non-negative vertex costs $c: V\rightarrow \R_{\ge 0}$ along with a  set of terminals $T\subseteq V$. A subset $U$ of non-terminals is said to be a subset feedback vertex set if $G-U$ has no cycle containing a vertex of $T$. 
The goal is to find a least cost subset feedback vertex set. In Section \ref{sec:CM-LP}, we discuss Chekuri-Madan's ILP for SFVS (henceforth referred to as (\textsc{SFVS-IP: CM})) with some background and notation. In Section \ref{sec:reduction-from-FVS-to-SFVS}, we discuss a reduction from FVS to SFVS satisfying certain technical properties. In Section \ref{sec:cm-lp-integrality-gap}, we  show that the integrality gap of the LP-relaxation of (\textsc{SFVS-IP: CM}) specialized for FVS (henceforth referred to as the (\textsc{FVS-LP: CM})) is at most $2$. 


\subsubsection{Chekuri-Madan's ILP for SFVS}
\label{sec:CM-LP}

In this section, we discuss the (\textsc{SFVS-IP: CM}).
%Chekuri-Madan's ILP for SFVS. 
Let $H = (V_H, E_H)$ denote the input graph, $S_H\subseteq V_H$ denote the terminal set, and $c_H:V_H\rightarrow \mathbb{R}_{\ge 0}$ denote the vertex cost function. We will say that a cycle is \emph{interesting} if it contains a terminal. Let $\calC_H$ denote the set of interesting cycles in the graph $H$.

Henceforth, we will assume that a SFVS of finite cost exists in the graph $H$. Furthermore, we will assume without loss of generality that the instance $H$ satisfies the following properties (see \cite{chekuri-madan16} for a justification of these assumptions): 
\begin{enumerate}[label=(\roman*)]
\item the graph $H$ is connected, 
\item each terminal $s_i \in S_H$
has infinite cost, is a degree two vertex with
neighbors $a_i, b_i$ having infinite cost, 
\item no two terminals
are connected by an edge or share a neighbor, 
\item there
exists a special non-terminal degree one vertex $r\in V$ with
infinite cost, and 
\item each interesting cycle contains at
least two terminals.
\end{enumerate} 
We refer to properties (i)-(v) as CM-properties.

\paragraph{Notation.} Throughout this section we will use the following notation: We let $k = |S_H|$ denote the number of terminals. We refer to the set $P_i := \{a_i, b_i\}$ as the \emph{pivot set} for the terminal $s_i$, and the set $P_H = \cup_{i \in [k]}P_i$ as the set of all pivots. We define the set 
\[
\ell_H(u) = \{i\in [k]: \text{$s_i$ is reachable from $u$  via a path not containing any other terminals}\} \cup \{k+1\} 
\]
as the set of \emph{labels} for each vertex $u \in V_H \backslash S$. We denote $\Tilde{V}_H$ as the set of vertices in $V_H$ that are not in $S_H \uplus P_H \uplus \{r\}$. We denote $\Tilde{E}_H$ as the set of edges in $E_H$ that are not incident to any terminals and the set $E'_H$ as the set of edges incident to terminals. We will refer to $E'_H$ as special edges. 
%Finally, for each $u \in V_H \backslash S$, we define the set $\ell_H(u) = \{i\in [k]: \text{$s_i$ is reachable from $u$  via a path not containing any other terminals}\} \cup \{k+1\}$ as the set of \emph{labels} for vertex $u$. 

%\paragraph{Chekuri-Madan's ILP.} 
\paragraph{(\textsc{SFVS-IP: CM}).} 
See \Cref{fig:CM-lp-SFVS} for %Chekuri-Madan's 
the labelling-based (\textsc{SFVS-IP: CM}).
%for SFVS. 
In this ILP, we have vertex variables $x_u$ for each $u \in \Tilde{V}$ indicating whether the vertex $u$ is in the SFVS $F$, and vertex labelling variables $z_{u, i}$ for each $u \in \Tilde{V}\cup P_H, i\in [k]$ indicating whether vertex $u$ receives label $i$. We note that \Cref{fig:CM-lp-SFVS} simplifies the exposition of the ILP from \cite{chekuri-madan16} by explicitly substituting the values for variables whose values are directly set by constraints, i.e. $x_u = 0$ for $u \in S \cup\{r\}$, $z_{s_i, i} = 1$ for $i \in [k]$, and $z_{r, k+1} = 1$. Moreover, it slightly strengthens the first constraint (Chekuri-Madan used a slightly weaker constraint: $x_u + \sum_{i\in [k+1]}z_{u, i}=1$).

%Imposing integrality constraints on the LP-relaxation formulates SFVS. 
The constraints can be interpreted as follows: Let $F$ be a minimal finite cost subset feedback vertex set of $H$. 
Constraint (1) says that if a vertex $u \in \Tilde{V}_H$ is not in $F$, then it must receive exactly one of the labels in $\ell_H(u)$. Constraint (2) says that exactly one of $a_i$ and $b_i$ can receive the label $i$ for each $i \in [k]$. Constraint (3) is a spreading constraint that says that both end vertices of every non-special edge in $H - F$ must receive the same label. Constraint (4) is a cycle covering constraint and says that the solution $F$ must intersect every interesting cycle in at least one vertex. Constraint (5) ensures that the set $F$ contains no pivot vertices.

\iffalse
\begin{figure}[H]
    \centering
    \begin{mdframed}
    \textbf{Subset Feedback Vertex Set:}
    \begin{align*}
        \text{min}\qquad&\displaystyle\sum\limits_{u \in \Tilde{V}_H} w_{u}x_{u}&\\
        \text{s.t.}\qquad&&\\
        (1)\ \ &x_u + \sum_{i \in [k+1]}z_{u,i} = 1& \forall\ u\in \Tilde{V}_H\cup P_H\\
        (2)\ \ &z_{a_i, i} + z_{b_i, i} = 1& \forall\ i \in [k]\\
        (3) \ \ &x_u + z_{u,i} - z_{v, i} \geq 0&     \forall\ uv\in \Tilde{E}_H,  \ i\in [k+1] \\
        %(4)\ \ &z_{u,i} - z_{v, i} \geq 0&     uv\in \Tilde{E}_H : u \in  P_H\cup\{r\}, \ i\in [k+1] \\
        %&z_{s_i, i} = 1& \  \text{ $i \in [k]$}\\
        %&z_{r, k+1} = 1&\\
        %&x_{u} = 0& u\in S_H \cup P_H \\
        (4)\ \ &\sum_{u\in C}x_u \geq 1& \forall\ C\in\calC_H\\
        (5)\ \ &x_u = 0&\forall\ u \in P_H\\
        (6)\ \ &z_{u, i} \geq 0&\forall\  \text{  $u \in \Tilde{V}_H \cup P_H, i\in [k+1]$}\\
        (7)\ \ &x_u \geq 0&\forall\  u \in \Tilde{V}_H
    \end{align*}
    \end{mdframed}
    \caption{Chekuri-Madan LP relaxation for SFVS}
  \label{fig:CM-lp-SFVS-as-is}
\end{figure}
\fi

We now briefly address why a labeling satisfying the LP constraints must exist for every minimal finite weight subset feedback vertex set $F$ (see \cite{chekuri-madan16} for further details). We note that none of the vertices of $S\cup P\cup \{r\}$ are in $F$ as these have infinite cost. Let the graph $H' = (V_{H'}, E_{H'})$ be defined as $H' = H - F$. For simplicity, we first assume that the graph $H'$ is connected and then address the more general case when $H'$ is not connected. Since there is no cycle containing any terminal in $H'$, each terminal is a cut vertex in $H'$. Consider the \emph{block-cut-vertex tree} decomposition $T'$ of $H'$ --- (refer to \cite{west_introduction_2000} for details on block-cut-vertex tree decompositions). We label a vertex $u$ of $H'$ as $i \in [k]$ if terminal $s_i$ is the first terminal encountered on any path from vertex $u$ to the root $r$ in the graph $H'$. If no terminals are encountered, then we label the vertex as $k+1$. We note that such a labelling is well-defined as $T'$ is the block-cut-vertex tree and the set $S$ are cut vertices. In this labelling, we observe that every non-special edge $uv \in \Tilde{E}_{H'}$ has the property that both end points receive the same label. Finally, in the case where $H'$ is not connected, we
can justify the existence of such a labeling by picking an arbitrary
non-terminal vertex from each component, imagining a dummy edge connecting it to the root $r$, and considering the labeling corresponding to block-cut-vertex tree of this (implicit) graph.

\iffalse
\paragraph{Strengthening the Chekuri-Madan LP.} We make a simple observation that strengthens the original Chekuri-Madan LP. We will subsequently show that the integrality gap of this strengthened LP is at most $2$. We recall the block-cut-vertex tree based labelling of $H' = H - F$ where each vertex is labelled by an index $i \in [k]$ if $s_i$ is the first terminal encountered on every path from vertex $u$ to the root $r$ in the tree $T'$. We define the set $\ell_H(u) = \{i\in [k]: \text{$s_i$ is reachable from $u$  via a path not containing any other terminals}\} \cup \{k+1\}$ as the set of \emph{labels} for each vertex $u \in V_H \backslash S$. Then, we observe that in the block-cut-vertex tree based labelling, a vertex $u$ can receive a label $i \in [k]$ only if $i \in \ell_H(u)$. Thus, we strengthen constraint (1) of the LP of \Cref{fig:CM-lp-SFVS-as-is} to be $$x_u + \sum_{i \in \ell_H(u)}z_{u, i} = 1 \qquad \text{ for each } u \in \Tilde{V}_H.$$ We write this LP in \Cref{fig:CM-lp-SFVS}.
\fi

\begin{figure}
    \centering
    \begin{mdframed}
    \textbf{Subset Feedback Vertex Set:}
    \begin{align*}
        \text{min}\qquad&\displaystyle\sum\limits_{u \in \Tilde{V}_H} c_{u}x_{u}&\\
        \text{s.t.}\qquad&&\\
        (1)\ \ &x_u + \sum_{i \in \ell_H(u)}z_{u,i} = 1& \forall\ u\in \Tilde{V}_H\cup P_H\\
        (2)\ \ &z_{a_i, i} + z_{b_i, i} = 1& \forall\ i \in [k]\\
        (3) \ \ &x_u + z_{u,i} - z_{v, i} \geq 0&     \forall\ uv\in \Tilde{E}_H,  \ i\in [k+1] \\
        %(4)\ \ &z_{u,i} - z_{v, i} \geq 0&     uv\in \Tilde{E}_H : u \in  P_H\cup\{r\}, \ i\in [k+1] \\
        %&z_{s_i, i} = 1& \  \text{ $i \in [k]$}\\
        %&z_{r, k+1} = 1&\\
        %&x_{u} = 0& u\in S_H \cup P_H \\
        (4)\ \ &\sum_{u\in C}x_u \geq 1& \forall\ C\in\calC_H\\
        (5)\ \ &x_u = 0&\forall\ u \in P_H\\
        (6)\ \ &z_{u, i} \geq 0&\forall\  \text{  $u \in \Tilde{V}_H \cup P_H, i\in [k+1]$}\\
        (7)\ \ &x_u \geq 0&\forall\  u \in \Tilde{V}_H\\
        (8)\ \ &z_{u, i} \in \Z&\forall\  \text{  $u \in \Tilde{V}_H \cup P_H, i\in [k+1]$}\\
        (9)\ \ &x_u \in \Z &\forall\  u \in \Tilde{V}_H
    \end{align*}
    \end{mdframed}
    \caption{
    (\textsc{SFVS-IP: CM})
    %Chekuri-Madan's ILP for SFVS
    }
  \label{fig:CM-lp-SFVS}
\end{figure}

\subsubsection{Reduction from FVS to SFVS satisfying the CM-properties}\label{sec:reduction-from-FVS-to-SFVS}
FVS can be seen as a special case of SFVS where every vertex in the graph is a terminal. However, this simple reduction does not result in an SFVS instance satisfying the CM-properties (see properties (i)--(v) mentioned in the first paragraph of Section \ref{sec:CM-LP}). In this section, we show how to construct 
an SFVS instance from an FVS instance such that the SFVS instance satisfies the required properties. Let $G= (V_G, E_G)$ with vertex costs $c_G:V_G\rightarrow \R_{\ge 0}$ be the FVS instance. We will construct 
an SFVS instance $H = (V_H, E_H)$ with terminals $S_H$ and vertex costs  $c_H:V_{H}\rightarrow\mathbb{R}_{\ge 0}$ satisfying the required properties. We may assume that the FVS instance $G$ has an infinite-weighted degree-1 root vertex $r$, satisfying property (iv) since adding such a vertex does not change the set of minimal feasible solutions. We construct $H$ as follows:

\begin{enumerate}
    \item For each edge $e = uv \in E_G$, we subdivide the edge $e$ so that the edge becomes a path $u, a_e, s_e, b_e, v$ of length $4$. %Let $a_e, s_e, b_e$ be the edges added by this step. 
    %\item For an edge $e \in E_G$, we denote the vertices added in step (1) as $a_e, s_e, b_e$. Thus, each edge $e=uv$ becomes the path $u,a_e,s_e,b_e,v$ in the graph $H$.
    \item We let $S_H = \bigcup_{e\in E(G)} \{s_e\}$ denote the terminal set. Thus, the set $P_e := \{a_e, b_e\}$ is the set of pivot vertices for each terminal $s_e \in S_H$, and the set $P_H := \bigcup_{e\in E(G)}P_e$ is the set of all pivots of the graph $H$.
    \item We define the cost function as $c_H(u) = c_G(u)$ if $u \in V_G$, and $c_H(u) = \infty$ otherwise (i.e., if $u \in S_H\cup P_H$).
\end{enumerate}

We observe that the graph $H$ satisfies properties (i)-(v) assumed by the Chekuri-Madan LP. The next proposition says that solving FVS in the graph $G$ with cost function $c_G$ is equivalent to solving SFVS in the graph $H$ with terminal set $S_H$ with cost function $c_H$. The proposition follows by construction.

\begin{proposition}\label{prop:reduction-FVS-to-SFVS}
Let $F\subseteq V_G$. The set $F$ is a feedback vertex set for $G$ if and only if the set $F$ is a subset feedback vertex set for the graph $H$ with terminal set $S_H$. Furthermore, the cost of $F$ in both graphs is the same, i.e. $c_G(F) = c_{H}(F)$.
\end{proposition}

\cite{chekuri-madan16} showed that the LP-relaxation of (\textsc{SFVS-IP: CM}) given in \Cref{fig:CM-lp-SFVS} has integrality gap at most $13$. It immediately follows that the integrality gap of that LP-relaxation on the instance $H$ constructed as above is also at most $13$. 
In the subsequent section, we tighten this integrality gap to $2$ for instances $H$ constructed as above.

\subsubsection{Bounding the integrality gap}\label{sec:cm-lp-integrality-gap}
In this section, we show that the LP-relaxation of the (\textsc{SFVS-IP: CM}) given in \Cref{fig:CM-lp-SFVS} for the instance $H$ constructed as given in the reduction in Section \ref{sec:reduction-from-FVS-to-SFVS} has integrality gap at 
 most $2$. We prove this by showing that the LP constraints imply orientation and cycle cover constraints. 
Consequently, the integrality gap of the LP-relaxation is at most that of $\min\{\sum_{u\in V} c_ux_u: x\in \projectedorientationpolyhedron(G)\cap \cyclecoverpolyhedron(G)\}$. By \Cref{lemma:integrality-gap-orient-intersect-cycle-cover}, it follows that the integrality gap of the LP-relaxation is at most $2$. 

%This, coupled with \Cref{thm:integrality-gap-wd+cycle-cover}, will help us conclude that the integrality gap of the LP in \Cref{fig:CM-lp-FVS} is at most $2$ for SFVS instances that are constructed from FVS instances as given in the reduction in Section \ref{sec:reduction-from-FVS-to-SFVS}.

%will complete the proof of \Cref{thm:poly-sized-LP-with-integrality-gap-atmost-2-for-FVS}. 

We begin by simplifying the (\textsc{SFVS-IP: CM}) given in \Cref{fig:CM-lp-SFVS} for SFVS instances $H$ that are constructed from FVS instances as obtained from the reduction in Section \ref{sec:reduction-from-FVS-to-SFVS}. 
We make two observations that will help us in the simplification.
The first observation is that each edge $e \in E_G$ corresponds to a unique terminal $s_e \in V_H$. Thus, we may replace the set $[k]$ which indexed the set of terminals in the ILP of \Cref{fig:CM-lp-SFVS} with the set $E_G$. For ease of exposition, we denote $e_r$ by the label $k+1$.
We now give the second observation. Let $u \in V_G$ be arbitrary. Since each edge $e \in \delta_G(u)$ has been subdivided in the graph $H$ and contains a terminal $s_e$, the only terminals that $u$ can reach via a path that does not contain any other terminal is exactly $\{s_e: e \in \delta_G(u)\}$. For an arbitrary pivot vertex $u \in P_H$, let $g(u)$ denote the unique non-terminal neighbor of $u$ in $H$. Then, the only terminals that $u$ can reach via a path that does not contain any other terminal is exactly $\{s_e: e \in \delta_G(g(u))\}$. We summarize these observations in the next proposition.

% \begin{figure}[H]
%     \centering
%     \begin{mdframed}
%     \begin{align*}
%         \text{min}&\displaystyle\sum\limits_{u \in V(G)} w_{u}x_{u}&\\
%         \text{s.t.}&&\\
%         &x_u + \sum_{e \in \delta(u)}z_{u,e} = 1& \  \text{  $u\in V(H)$}\\
%         &z_{s_e, e} = 1& \  \text{ $e \in E(G)$}\\
%         &z_{r, |E(G)|+1} = 1&\\
%         &z_{a_e, e} + z_{b_e, e} = 1& \text{ $e \in E(G)$}\\
%         &x_u + z_{v,e} - z_{u, e} \geq 0& \  \text{ $e = uv\in E - E_S$}\\
%         &x_{u} = 0& \  \text{  $u \in \cup_{e \in E(G)}\{a_e, s_e, b_e\}$} \\
%         &\sum_{u\in C}x_u \geq 1& \  C\in\calC\\
%         &z_{u, e} \geq 0&\  \text{  $u \in V(H), e \in E(G)$}\\
%         &x_u \geq 0&\  \text{ $u \in V(H)$}
%     \end{align*}
%     \end{mdframed}
%     \caption{LP relaxation for FVS}
%   \label{fig:CM-lp-FVS}
% \end{figure}

\begin{proposition}\label{prop:cm-lp-fvs-stronger-label-set}
We have that
\begin{enumerate}
    \item For each vertex $u \in V_G$, we have that $\ell_H(u) = \delta_G(u)\cup \{e_r\}$;
    \item For each pivot vertex $u \in P$, we have that $\ell_H(u) = \delta_G(g(u))\cup \{e_r\}$.
\end{enumerate}
\end{proposition}

Using \Cref{prop:cm-lp-fvs-stronger-label-set} and the first observation mentioned above, we simplify the (\textsc{SFVS-IP: CM}) in \Cref{fig:CM-lp-SFVS} for the SFVS instance $H$ and write its LP-relaxation (\textsc{FVS-LP: CM}) in \Cref{fig:CM-lp-FVS}. 
\begin{figure}[H]
    \centering
    \begin{mdframed}
    \textbf{LP-relaxation of (\textsc{SFVS-IP: CM}) for graph $H$:}
    \begin{align*}
        \text{min}\qquad&\displaystyle\sum\limits_{u \in \Tilde{V}_H} c_{u}x_{u}&\\
        \text{s.t.}\qquad&&\\
        (1)\ \ &x_u + \sum_{e \in \ell_H(u)}z_{u,e} = 1& \forall u\in \Tilde{V}_H\cup P_H\\
        (2)\ \ &z_{a_e, e} + z_{b_e, e} = 1& \forall e \in E_G\\
        (3) \ \ &x_u + z_{u,e} - z_{v, e} \geq 0&  \forall   uv\in \Tilde{E}_H, \  e\in E_G\cup\{e_r\} \\
        %(4)\ \ &z_{u,e} - z_{v, e} \geq 0&     uv\in \Tilde{E}_H : u \in  P_H\cup\{r\}, \ e\in E_G \\
        %&z_{s_i, i} = 1& \  \text{ $i \in [k]$}\\
        %&z_{r, k+1} = 1&\\
        %&x_{u} = 0& u\in S_H \cup P_H \\
        (4)\ \ &\sum_{u\in C}x_u \geq 1& \ \forall C\in\calC_H\\
        (5)\ \ &x_u = 0&\  \text{ $\forall u \in P_H$}\\
        (6)\ \ &z_{u, e} \geq 0&\  \text{  $\forall u \in \Tilde{V}_H\cup P_H, e\in E_{G}\cup \{e_r\}$}\\
        (7)\ \ &x_u \geq 0&\  \text{ $\forall u \in \Tilde{V}_H$}
    \end{align*}
    \end{mdframed}
    \caption{
    (\textsc{FVS-LP: CM})
    %Chekuri-Madan LP relaxation for FVS
    }
  \label{fig:CM-lp-FVS}
\end{figure}


We now show that the constraints of the LP in \Cref{fig:CM-lp-FVS} imply the orientation and cycle cover constraints for $G$. 

\begin{lemma}\label{lem:cm-implies-orientation}
Let $(\overline{x}, \overline{z})$ be a feasible solution to (\textsc{FVS-LP: CM}) from \Cref{fig:CM-lp-FVS}. Then,
\begin{enumerate}
\item $\overline{x}\in \cyclecoverpolyhedron(G)$ and 
\item  
$(\overline{x}, \overline{y}) \in \orientationpolyhedron(G)$, where $\overline{y}_{e, u}=\overline{z}_{u, e}$ for all $e\in \delta(u), u\in V_G$. 
\end{enumerate}
\end{lemma}
\begin{proof}
We have that $\overline{x}\in \cyclecoverpolyhedron(G)$ owing to inequality (4). We now show that $(\overline{x}, \overline{y}) \in \orientationpolyhedron(G)$.
We have that $\overline{x}_u\ge 0$ for all $u\in V_G$ and $\overline{y}_{e, u}\ge 0$ for all $e\in \delta_G(u), u\in V_G$. We now show that $\overline{x}_u + \sum_{e\in \delta_G(u)}\overline{y}_{e,u}\le 1$ for every $u\in V_G$. Let $u\in V$. We have that 
\[
\overline{x}_u + \sum_{e\in \delta_G(u)}\overline{y}_{e,u}=\overline{x}_u + \sum_{e\in \ell_H(u)}\overline{z}_{u, e}-\overline{z}_{u, e_r}= 1-\overline{z}_{u, e_r}\le 1,
\]
where the last inequality holds since $\overline{z}_{u, e_r}\ge 0$. 

Finally, we show that $\overline{x}_u + \overline{x}_v + \overline{y}_{e, u} +\overline{y}_{e,v}\ge 1$ for every $e=uv\in E_G$. Let $e=uv\in E_G$. Let $u, a_e, s_e, b_e,v$ be the path in the graph $H$ corresponding to the subdivided edge $e$. Then, we have that 
\begin{align*}
\overline{x}_u + \overline{x}_v + \overline{y}_{e, u} + \overline{y}_{e, v}& =
    \overline{x}_u + \overline{x}_v + \overline{z}_{u, e} + \overline{z}_{v, e}&\\
    &= (\overline{x}_u + \overline{x}_v + \overline{z}_{u, e} + \overline{z}_{v, e}) + (\overline{z}_{a_e, e} + \overline{z}_{b_e, e}) - (\overline{z}_{a_e, e} + \overline{z}_{b_e, e}) &\\
    &= 1 + ( \overline{x}_u + \overline{z}_{u,e} - \overline{z}_{a_e, e})+ ( \overline{x}_v + \overline{z}_{v,e} - \overline{z}_{b_e, e})&\\
    & \geq 1.&
\end{align*}
Here, the third equality follows from constraint (2) and the final inequality follows from constraint (3) of (\textsc{FVS-LP: CM}) in \Cref{fig:CM-lp-FVS}.
\end{proof}

\Cref{lem:cm-implies-orientation}  implies that $\min\{\sum_{u\in V}c_u x_u: x\in \projectedorientationpolyhedron(G)\cap \cyclecoverpolyhedron(G)\}$ is a relaxation of (\textsc{FVS-LP: CM}) from \Cref{fig:CM-lp-FVS}.  \Cref{lemma:integrality-gap-orient-intersect-cycle-cover} tells us that the integrality gap of $\min\{\sum_{u\in V}c_u x_u: x\in \projectedorientationpolyhedron(G)\cap \cyclecoverpolyhedron(G)\}$ is at most $2$. Consequently, the integrality gap of (\textsc{FVS-LP: CM}) from \Cref{fig:CM-lp-FVS} is also at most $2$. We note that (\textsc{FVS-LP: CM}) from \Cref{fig:CM-lp-FVS} can be converted to a polynomial-sized LP by replacing inequality (4) using a polynomial-sized description of cycle cover constraints as given in \Cref{lemma:cycle-cover-poly-sized-lp}. 

