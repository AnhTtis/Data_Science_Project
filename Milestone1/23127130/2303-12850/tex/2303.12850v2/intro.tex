\section{Introduction}
\label{sec:intro}
We consider the feedback vertex set problem in undirected graphs. For
a graph $G=(V, E)$, a subset $U\subseteq V$ is a \emph{feedback vertex
  set} if $G-U$ is acyclic; in other words $U$ is a hitting set for
the cycles of $G$. In the feedback vertex set problem (FVS), the input
is an undirected graph $G=(V, E)$ with non-negative vertex costs
$c: V\rightarrow \R_{\ge 0}$ and the goal is to find a feedback vertex
set of minimum cost, i.e.,
$\min\left\{\sum_{u\in U} c(u): U\subseteq V\ \text{and $U$ is a feedback
  vertex set for $G$}\right\}$.  FVS is a fundamental vertex deletion
problem in the field of combinatorial optimization and appears in
Karp's list of 21 NP-Complete problems (as a decision version).  We
consider the approximability of FVS, in particular via polynomial-time
solvable polyhedral formulations. First, we mention some known lower
bounds. Via a simple reduction, the well-known weighted vertex cover
problem can be reduced to FVS in an approximation preserving
fashion. %\chandra{I wondered if we should capitalize problem names such a Vertex Cover, Feeback Vertex Set etc.} 
Hence, known results for the  vertex cover problem imply that there is no
polynomial-time $(2-\epsilon)$-approximation for every fixed constant
$\epsilon>0$ under the Unique Games Conjecture (UGC) \cite{KhotR08} and that there is no polynomial-time $(1.36-\epsilon)$-approximation for every fixed constant $\epsilon>0$ under the $P \neq NP$ assumption \cite{DinurS05}.

Approximation algorithms for the unweighted version of FVS (UFVS), i.e., when all
vertex costs are one, have been studied since 1980s.  An
$O(\log n)$-approximation for UFVS is implicit in the work of Erd\"os
and P\'osa \cite{ErdosP62}; Monien and Schulz \cite{MonienS81} seem to be the first ones
to explicitly study the problem, and obtained an
$O(\sqrt{\log n})$-approximation for UFVS. Bar-Yehuda, Geiger, Naor, and Roth \cite{BarYehudaGNR98} improved
the ratio for UFVS to $4$, and also noted that an $O(\log
n)$-approximation holds for FVS. Soon after that,
Bafna, Berman, and Fujito \cite{Bafna-Berman-Fujito95}, and independently Becker and Geiger \cite{BG96},
obtained $2$-approximation algorithms for FVS. While \cite{Bafna-Berman-Fujito95}
explicitly uses the local-ratio terminology, \cite{BG96} describes the algorithm
in a purely combinatorial fashion.
Although the algorithm in \cite{Bafna-Berman-Fujito95} is described via the
local-ratio method, the underlying LP relaxation is not obvious.
As observed in \cite{BarYehudaGNR98}, the natural hitting set LP relaxation for FVS
has an integrality gap of $\Theta(\log n)$.
Chudak, Goemans, Hochbaum, and Williamson \cite{CHUDAK1998111} described
exponential sized integer linear programming (ILP) formulations for FVS, and showed that the algorithms in \cite{Bafna-Berman-Fujito95,BG96} can be viewed as primal-dual algorithms
with respect to the LP relaxations of these formulations. This also established that
these LP relaxations have an integrality gap of at most $2$. 
Fujito \cite{Fuj-matroid-FVS} considered a unified generalization of FVS and vertex cover, namely matroidal FVS, and gave a $O(\log n)$-approximation for matroidal FVS via connections to submodular set cover. In the same work, Fujito formulated an exponential 
sized ILP for matroidal FVS, and designed a primal-dual algorithm with respect to its LP relaxation. He proved that the algorithm has an approximation
ratio of $2$ for a certain family of matroids. When specialized to FVS, we note that the ILP and the resulting primal-dual algorithm
in \cite{Fuj-matroid-FVS} is slightly different 
from the one in \cite{CHUDAK1998111} although they both yield $2$-approximations. %\chandra{Elaborated here.}

We recall the integer linear program (ILP) formulation and the
associated LP-relaxation from \cite{CHUDAK1998111}. For a graph $G=(V,
E)$ and a subset $S\subseteq V$, let $E[S]:=\left\{\left\{u,
v\right\}\in E: u, v\in S\right\}$ and $G[S]:=(S, E[S])$, and for a
vertex $v\in V$, let $d_S(v)$ denote the degree of $v$ in $G[S]$. We
will denote the following polyhedron as the strong density polyhedron
and the constraints describing the polyhedron as strong density
constraints\footnote{The use of the ``density'' terminology will be clear later when we discuss
the connection to the densest subgraph problem and its LP relaxation.}:
\begin{align}
    \strongdensitypolyhedron(G)
    &:=
    \left\{ x\in \R^{V}_{\ge 0}:\ \sum \nolimits_{u\in S}(d_S(u)-1)x_u\geq |E[S]| - |S| + 1 \ \forall S\text{ such that } E[S]\not = \emptyset
\right\}.
\end{align}
Chudak, Goemans, Hochbaum, and Williamson proved that strong density constraints are valid for FVS and that
the following is an integer linear programming formulation for FVS:
\begin{align}
\min\left\{\sum \nolimits_{u\in V}c_u x_u:\ x\in \strongdensitypolyhedron(G)\cap \Z^V\right\}. \tag{FVS-IP: SD} \label{FVS-IP: SD}
\end{align}
%$\min\{\sum_{u\in V}c_u x_u:\ x\in \strongdensitypolyhedron(G)\cap \Z^V\}$
They interpreted the local-ratio algorithm from \cite{Bafna-Berman-Fujito95} as a
primal-dual algorithm with respect to\footnote{An $\alpha$-approximation with respect to an LP $\min\{c^Tx: Ax\le b\}$ is an algorithm that returns an integral solution $z^*$ satisfying $Ax\le b$ such that $c^Tz^*\le \alpha \text{OPT}_{\text{LP}}$, where $\text{OPT}_{\text{LP}}$ is the optimum objective value of the LP.} the LP-relaxation of (\ref{FVS-IP: SD}) that we explicitly write below:
\begin{align}
    \min\left\{\sum \nolimits_{u\in V}c_u x_u:\ x\in \strongdensitypolyhedron(G)\right\}. \tag{FVS-LP: SD} \label{FVS-LP: SD}
\end{align}
%$\min\{\sum_{u\in V}c_u x_u:\ x\in \strongdensitypolyhedron(G)\}$

It is, however, not known whether (\ref{FVS-LP: SD}) can be solved in
polynomial time. Equivalently, it is open to design a polynomial-time
separation oracle for the family of strong density constraints.
Moreover, there has been no other polynomial-time solvable LP
relaxation through which one could obtain a $2$-approximation for FVS.
This status leads to the following natural question:

\begin{question}
  \label{q:intro}
  Does there exist an ILP formulation for FVS whose LP-relaxation can be solved in polynomial time and has integrality gap at most $2$?
\end{question}

\iffalse
\begin{center}
\noindent\fbox{%
    \parbox{5.5in}{%
        \textbf{Question.} Does there exist an ILP formulation for FVS whose LP-relaxation can be solved in polynomial time and has integrality gap at most $2$?
    }%
}
\end{center}
%\vspace{1mm}
\fi

\iffalse
\begin{center}
\text{\textbf{Question.} Are there integer linear programs  for FVS}\\
\text{whose LP-relaxation can be solved in polynomial time and have integrality gap at most $2$?}
\end{center}
\fi

%Given this status, it would be interesting to formulate integer linear programs for FVS whose LP-relaxations are solvable in polynomial time and also have integrality gap at most $2$.
%polynomial-time solvable LP-relaxations with integrality gap at most $2$ for both FVS and PFDS.
%Our work is motivated by a question raised by Fiorini.

The lack of solvable LP relaxations for FVS with small integrality gap has also been a stumbling
block for the design of approximation algorithms for a generalization
of FVS called the subset feedback vertex set problem (Subset-FVS) \cite{EvenNSZ00}: the input is a
vertex-weighted graph $G$ and a terminal set $T \subseteq V$, and the goal is
to remove a minimum cost subset of vertices $S$ to ensure that $G-S$ has
no cycle containing a terminal $t \in T$. There is an $8$-approximation for
Subset-FVS \cite{EvenNZ00} and this is based on a complex algorithm that combines 
combinatorial and LP-based techniques. Chekuri and Madan \cite{chekuri-madan16} formulated a polynomial-sized integer linear
program for Subset-FVS and showed that the integrality gap of its
LP-relaxation is at most $13$. They explicitly raised the question of
whether the integrality gap of their formulation is better for FVS; in fact
it is open whether their formulation's integrality gap is at most $2$ for Subset-FVS.

In this work, we provide an affirmative answer to
Question~\ref{q:intro} by undertaking a polyhedral study of FVS and a
closely related problem, namely the pseudoforest deletion set problem
that will be described shortly. In addition to formulating
polynomial-time solvable LP relaxations with small integrality gap, it
is also of interest to find algorithms that can round fractional
solutions to the LP relaxations, and to understand properties of extreme points of
the corresponding polyhedra. Based on several interrelated technical
considerations, we conjecture the following property for
the strong density polyhedron:

\begin{conjecture}\label{conj:strong-density-extreme-point}
Let $G=(V, E)$ be a graph that contains a cycle. For every extreme
point $x$ of the polyhedron $\strongdensitypolyhedron(G)$, there
exists a vertex $u\in V$ such that $x_u\ge 1/2$.
\end{conjecture}
%Although the conjecture is not directly helpful from an algorithmic perspective, it is appealing from a structural perspective.
This conjecture would lead to an alternative proof that the
integrality gap of the LP-relaxation (\ref{FVS-LP: SD}) is at most $2$
via iterative rounding (instead of the primal-dual technique).
Although we were unable to resolve this conjecture for the strong
density polyhedron, we were able to show that a variant of the
conjecture holds for a \emph{weak density polyhedron} (to be defined
later). The weak density polyhedron closely resembles the strong
density polyhedron and is associated with an ILP formulation of the
pseudoforest deletion set problem.
%Although we were unable to resolve this conjecture, we discovered strong connections between the strong density polyhedron and a closely related \emph{weak density polyhedron} (to be defined later) that is associated with an ILP formulation for the pseudoforest deletion problem and were also able to show extreme point properties for the weak density polyhedron.
We also discuss extreme point properties for other formulations later in the paper.

\paragraph{Pseudoforest Deletion Set Problem (PFDS).} A connected graph is a \emph{pseudotree} if
it has exactly one cycle; in other words there is an edge whose removal
results in a spanning tree. A graph is a \emph{pseudoforest} if every connected component is either acyclic or a pseudotree.  For a graph $G=(V, E)$, a
subset $U\subseteq V$ is a \emph{pseudoforest deletion set} if $G-U$
is a pseudoforest. In the pseudoforest deletion set problem (PFDS),
the input is an undirected graph $G=(V, E)$ with non-negative vertex
costs $c: V\rightarrow \R_{\ge 0}$ and the goal is to find a
pseudoforest deletion set of minimum cost, i.e., $\min \left\{\sum_{u\in U}
c(u): U\subseteq V\ \text{and $U$ is a pseudoforest deletion set for
  $G$}\right\}$. Intuitively, one can see that PFDS is closely related to FVS: 
we note that a feasible solution for FVS in a given graph $G$ is a feasible
solution to the PFDS instance on $G$. Also, finding an FVS in a graph that is
a pseudoforest is easy: for each connected component that is a pseudotree, we remove
the cheapest vertex in its unique cycle. 
PFDS and FVS are special cases of the more general $\ell$-pseudoforest deletion problem that was introduced in \cite{PhilipRS18} from the perspective of parameterized algorithms (FVS corresponds to $\ell=0$ and PFDS to $\ell=1$). Lin, Feng, Fu, and Wang \cite{LFFW19} studied approximation algorithms for $\ell$-pseudoforest deletion problem. In this paper, we restrict attention to PFDS and FVS and do not discuss the more general $\ell$-pseudoforest deletion problem.
The status of PFDS is very similar to that of FVS. It has an approximation preserving reduction from the vertex cover problem and consequently, it is 
NP-hard, and does not have a polynomial time $(2-\epsilon)$-approximation
for every constant $\epsilon >0$ assuming the UGC. It admits a polynomial-time $2$-approximation based on the local-ratio technique \cite{LFFW19}. 
The authors in \cite{LFFW19} do not discuss LP relaxations for PFDS. However, the local-ratio technique of \cite{LFFW19} can be converted to an LP-based $2$-approximation for PFDS following the ideas in \cite{CHUDAK1998111}. 
%\chandra{I added some additional discussion about ell-pseudoforest, LP etc. Otherwise the context for PFDS may not be clear.} 
We describe the associated LP relaxation. A graph $G=(V, E)$ is a \emph{$2$-pseudotree} if it is connected
and has $|E| \geq |V|+1$ (i.e., the graph has at least $2$ edges in
addition to a spanning tree). We will denote the following polyhedra
as weak density polyhedron and $2$-pseudotree cover polyhedron
respectively, and the constraints describing them as weak density
constraints and $2$-pseudotree cover constraints respectively:
\begin{align}
    \weakdensitypolyhedron(G)
    &:=
    \left\{ x\in \R^{V}_{\ge 0}:\ \sum \nolimits_{u\in S}(d_S(u)-1)x_u\geq |E[S]| - |S| \ \forall S \subseteq V
\right\}, \text{ and}  \label{eqn:weak-density}\\
\twopseudotreecoverpolyhedron(G)&:=\left\{x\in \R^V_{\ge 0}: \sum \nolimits_{u\in U}x_u \ge 1 \ \forall U\subseteq V \text{ such that $G[U]$ contains a $2$-pseudotree}\right\}. \label{eqn:2PT-cover}
\end{align}
%We note that strong density constraints differ from weak density constraints by an additive $1$ and moreover, strong density constraints are imposed only for vertex subsets that contain at least one edge while weak density constraints are imposed for all subsets of vertices.
We encourage the reader to compare and contrast the weak density constraints with the strong density constraints.
Weak density constraints and $2$-pseudotree covering constraints are valid for PFDS. In particular, the following are ILP formulations for PFDS:
\begin{align}
    \min &\left\{\sum \nolimits_{u\in V}c_u x_u:\ x\in \weakdensitypolyhedron(G)\cap \Z^V\right\} \text{ and}\tag{PFDS-IP: WD} \label{PFDS-IP: WD}\\
    \min &\left\{\sum \nolimits_{u\in V}c_u x_u:\ x\in \weakdensitypolyhedron(G)\cap \twopseudotreecoverpolyhedron(G) \cap \Z^V\right\}. \tag{PFDS-IP: WD-and-2PT-cover} \label{PFDS-IP: WD-and-2PT-cover}
\end{align}
%$\min\{\sum_{u\in V}c_u x_u:\ x\in \weakdensitypolyhedron(G)\cap \twopseudotreecoverpolyhedron(G) \cap \Z^V\}$
The local-ratio technique for PFDS due to Lin, Feng, Fu, and Wang \cite{LFFW19} can be converted to a $2$-approximation for PFDS with respect to the following LP-relaxation via the primal-dual technique:
\begin{align}
    \min\left\{\sum \nolimits_{u\in V}c_u x_u:\ x\in \weakdensitypolyhedron(G)\cap \twopseudotreecoverpolyhedron(G)\right\} \tag{PFDS-LP: WD-and-2PT-cover}. \label{PFDS-LP: WD-and-2PT-cover}
\end{align}
%$\min\{\sum_{u\in V}c_u x_u:\ x\in \weakdensitypolyhedron(G)\cap \twopseudotreecoverpolyhedron(G)\}$
%It remains open to solve (\ref{PFDS-IP: WD-and-2PT-cover}) in
%polynomial time. 
The family of $2$-pseudotree cover constraints admits a polynomial-time separation oracle (see
\Cref{thm:MC2PT-polytime:main}). However, we do not know a
polynomial-time separation oracle for the family of weak density
constraints (similar to the status of strong density constraints), and for this reason we do not know how to solve (\ref{PFDS-LP: WD-and-2PT-cover}) in polynomial time\footnote{We note that Fujito's framework \cite{Fuj-matroid-FVS} encompasses PFDS and hence a $2$-approximation follows from his work.
It appears that \cite{LFFW19} were unaware of \cite{Fuj-matroid-FVS}. The authors of this paper also became aware of \cite{Fuj-matroid-FVS} after completing
an earlier version. The LP relaxation in \cite{Fuj-matroid-FVS} does not appear to have a polynomial-time separation oracle.}.
%\chandra{Added a footnote. We should check to see whether the LP in \cite{Fuj-matroid-FVS} is solvable or not.}
%\chandra{Did some editing here. Instead of open which seems like a long-standing thing I said we do not know.}
This leads to the following question (which is the counterpart of Question \ref{q:intro} for PFDS):
\begin{question}
  \label{q:intro-pfds}
  Does there exist an ILP formulation for PFDS whose LP-relaxation can be solved in polynomial time and has integrality gap at most $2$?
\end{question}

We emphasize that weak density constraints and our answer to Question \ref{q:intro-pfds} will play a crucial role in our answer to Question \ref{q:intro}. 
%our polynomial-sized integer linear program for FVS with integrality gap at most $2$. 
Furthermore, we prove an extreme point property of
the weak density polyhedron that closely resembles the extreme point
property of the strong density polyhedron mentioned in
\Cref{conj:strong-density-extreme-point}.

\iffalse
In this work, we show several results on LP-relaxations for FVS and PFDS.
\begin{enumerate}
\item We show an extreme point result for the weak density polyhedron---every extreme point optimum solution $x$ has a vertex $u\in V$ such that $x_u\ge 1/3$. Via the  iterative rounding technique, this result immediately implies that the integrality gap of the LP-relaxation for PFDS based on weak density constraints, namely $\min \{\sum_{u\in V}c_u x_u: x\in \weakdensitypolyhedron(G)\}$, is at most $3$.
%\knote{This integrality gap is tight.}
However, we note that this LP is still not known to be polynomial-time solvable and hence, it is unclear if the iterative rounding technique can be used to design a \emph{polynomial-time} $3$-approximation with respect to this LP.

\item We give a new \emph{polynomial-sized} integer linear programming formulation for PFDS via an orientation viewpoint. We will call the polyhedron associated with our formulation as the orientation polyhedron of $G$ and denote it by  $\orientationpolyhedron(G)$. The orientation viewpoint arises by interpreting pseudoforests as graphs of density at most one and using the dual of Charikar's LP for the densest subgraph problem.

\item We show that every \emph{minimal} extreme point $(x,y)$ of $\orientationpolyhedron(G)$ along non-negative objective directions has the previously mentioned property: there exists a vertex $u\in V$ such that $x_u\ge 1/3$. Once again, via the iterative rounding technique, this extreme point result immediately
implies that the integrality gap of the LP-relaxation for PFDS based on the orientation polyhedron, namely $\min\{\sum_{u \in V}c_u x_u: (x,y)\in \orientationpolyhedron(G)\}$, is at most $3$.
%\knote{This integrality gap is tight.}
Thus, we obtain a polynomial-sized integer linear program for PFDS whose LP-relaxation has integrality gap at most $3$. We note that this is weaker than the factor of $2$ that we set out to achieve.
%\knote{It would be interesting to formulate poly-time solvable LP-relaxations for PFDS with integrality gap at most $2$.}

\item Armed with the above results for PFDS, we return to the question
  of whether there exists an integer linear program for FVS whose
  LP-relaxation is polynomial-time solvable and has integrality gap at
  most $2$. We answer this question affirmatively by showing three
  different \emph{polynomial-sized} integer linear programs for FVS
  all of whose LP-relaxations have integrality gap at most $2$.  One
  of them is simply the formulation of Chekuri and Madan mentioned
  above specialized to FVS, the second one is based on
  $\min\{\sum_{u\in V}c_u x_u: x\in \orientationpolyhedron(G)\cap
  \cyclecoverpolyhedron(G)\cap \Z^V\}$, and the third one is based on
  an orientation viewpoint but without cycle cover constraints. Our
  proof of integrality gap of the first two formulations proceeds as
  follows: we show that the LP-relaxation of the first formulation is
  at least as strong as the LP-relaxation of the second formulation
  and that the LP-relaxation of the second formulation is at least as
  strong as the LP-relaxation of another ILP formulation for FVS,
  namely $\min \{\sum_{u\in V} c_u x_u: x\in
  \weakdensitypolyhedron(G)\cap \cyclecoverpolyhedron(G)\cap
  \Z^V\}$. Finally, we show that the integrality gap of the LP $\min
  \{\sum_{u\in V} c_u x_u: x\in \weakdensitypolyhedron(G)\cap
  \cyclecoverpolyhedron(G)\}$ is at most $2$ which in turn implies
  that the integrality gaps of the LP-relaxation of the first two
  formulations are also at most $2$.  To bound the integrality gap of
  the LP-relaxation of the third formulation, we show that its
  associated polyhedron is contained in $\strongdensitypolyhedron(G)$
  and since the integrality gap of $\min\{\sum_{u\in V}c_u x_u: x\in
  \strongdensitypolyhedron(G)\}$ is at most $2$ \cite{CHUDAK1998111},
  the integrality gap of our third formulation is also at most $2$.

%\knote{It would be nice to bound the integrality gap of each of these three formulations by direct rounding. That might be helpful in getting better approximations for Subset-FVS.}
%\knote{Another open problem is to show extreme point properties for the strong density polyhedron: does every extreme point have a }
\end{enumerate}
\fi
