\section{Polynomial-time solvable LP-relaxation for PFDS with integrality gap at most $2$}
%{Orientation Polyhedron}
\label{sec:orientatin-polytope}
In this section, we restate and prove \Cref{lemma:orientation-formulation}. 

\lemmaOrientationFormulation*
%\begin{proof}[Proof of \Cref{lemma:orientation-formulation}]
\begin{proof}
In order to show that (\ref{PFDS-IP:orient}) and (\ref{PFDS-IP:Orient-and-2PT-cover}) are ILP formulations for PFDS, it suffices to show that the ILP (\ref{PFDS-IP:orient}) is a formulation for PFDS. This is because $2$-pseudotree cover constraints are valid for PFDS. 


For a subgraph $\Tilde{G} = (\Tilde{V}, \Tilde{E})$ of the graph $G$, we let $d_{\Tilde{G}}(u):=|\delta(u)\cap \Tilde{E}|$ denote the degree of the vertex $u$ in the subgraph $\Tilde{G}$. We now define an intermediate polyhedron $\weakdensitypolyhedronforsubgraphs(G)$.
\begin{align}
\weakdensitypolyhedronforsubgraphs(G) 
    &:= 
    \left\{ x\in \R^{V}_{\ge 0}:\ \sum_{u\in \Tilde{V}}(d_{\Tilde{G}}(u)-1)x_u\geq |\Tilde{E}| - |\Tilde{V}| \ \forall \text{ subgraphs $\Tilde{G} = (\Tilde{V}, \Tilde{E})$ of $G$}.
\right\}. \label{eqn:weak-density-for-subgraphs}
\end{align}
We note that $\weakdensitypolyhedronforsubgraphs(G) \subseteq \weakdensitypolyhedron(G)$. We will use the following claim to show that the ILP  (\ref{PFDS-IP:orient}) formulates PFDS and also to conclude that $\projectedorientationpolyhedron(G)\subseteq \weakdensitypolyhedronforsubgraphs(G)$. 

\begin{claim}\label{claim:orientation-inside-weak-density-subgraphs}
$\projectedorientationpolyhedron(G)\subseteq \weakdensitypolyhedronforsubgraphs(G)$.
\end{claim}
\begin{proof}
Let $x\in \projectedorientationpolyhedron(G)$. Then, there exists a vector $y$ such that $(x, y)\in \orientationpolyhedron(G)$. Let $\Tilde{G} = (\Tilde{V}, \Tilde{E})$ be an arbitrary subgraph of $G$. We have the following:
\begin{align*}
|\Tilde{E}| &\leq \sum_{e = vw \in \Tilde{E}} (x_v + x_w + y_{e,v} + y_{e,w})\\
       &= \sum_{v \in \Tilde{V}} (d_{\Tilde{G}}(v)-1) x_v + \sum_{v \in \Tilde{V}} \left( x_v + \sum_{e \in \delta_{\Tilde{G}}(v)} y_{e,v}\right)\\ 
       &\leq \sum_{v \in \Tilde{V}} (d_{\Tilde{G}}(v)-1) x_v + |\Tilde{V}|,
\end{align*}
where the inequalities are because the vector $(x,y) \in \orientationpolyhedron$. Rearranging the above inequality gives the constraint for $\tilde{G}$ given in $\weakdensitypolyhedronforsubgraphs(G)$. Hence, $x\in \weakdensitypolyhedronforsubgraphs(G)$.      
\end{proof}

\Cref{claim:orientation-inside-weak-density-subgraphs} can be strengthened to show that $\projectedorientationpolyhedron(G)= \weakdensitypolyhedronforsubgraphs(G)$, but we will not need this stronger version for the purposes of this theorem.

We now show that the ILP (\ref{PFDS-IP:orient}) formulates PFDS. 
Let $S \subseteq V$ be a pseudoforest deletion set, i.e., 
 the subgraph $F := G - S$ is a pseudoforest. Let $x := \chi^S \in \{0,1\}^{V}$ denote the indicator vector of the set $S$.  
 Select an arbitrary orientation of the pseudoforest $F$ such that the maximum indegree of every vertex is at most $1$, and let $y_{e,v} := 1$ for the edge $e = vw$ if $e$ is an edge of $F$ that is oriented towards $v$, and $y_{e,v} := 0$ otherwise. 
 Then, $(x,y) \in \orientationpolyhedron(G)$. 

Next, suppose $x\in \projectedorientationpolyhedron(G)\cap \Z^V$. Then, we have that $x\in \{0,1\}^V$. By \Cref{claim:orientation-inside-weak-density-subgraphs}, $x\in \weakdensitypolyhedronforsubgraphs(G)\subseteq \weakdensitypolyhedron(G)$. 
%Since the ILP  will show that the set $S$ is a pseudoforest deletion set of the graph $G$, i.e. the subgraph $G - S$ is a pseudoforest. 
Since (\ref{PFDS-IP: WD}) is an ILP formulation for PFDS by \Cref{thm:PFDS-WD-and-2PT-cover}, it follows that the set $S:=\{u\in V: x_u =1\}$ is a pseudoforest deletion set for the graph $G$. This concludes the proof that the ILP (\ref{PFDS-IP:orient}) formulates PFDS. 

We now prove the additional three conclusions of the theorem statement. 

\begin{enumerate}
\item By Claim \ref{claim:orientation-inside-weak-density-subgraphs}, we have that $\projectedorientationpolyhedron(G)\subseteq \weakdensitypolyhedronforsubgraphs(G)\subseteq \weakdensitypolyhedron(G)$. 
We now show that there is a graph $G$ such that $\weakdensitypolyhedronforsubgraphs(G)\subsetneq \weakdensitypolyhedron(G)$. In particular, we consider the graph $K_5 = (V, E)$ where $V = \{v_1, v_2, \ldots, v_5\}$ and $E = {V \choose 2}$. Let $x = (7/12, 7/12, 1/12, 0, 0)$. We note that $x \in \weakdensitypolyhedron(K_5)$.
We now show that $x \not \in \weakdensitypolyhedronforsubgraphs(K_5)$.
    Consider the subgraph $\Tilde{G} = (\Tilde{V}, \Tilde{E})$ obtained by removing the edge $\{v_1, v_2\}$ from the graph $G$, i.e. $\Tilde{V} = V$ and $\Tilde{E} = {V\choose 2} - \{v_1, v_2\}$.
    Then, we have the following:
    $$\sum_{u \in \Tilde{V}}(d_{\Tilde{G}}(u) - 1)x_u = 3x_1 + 3x_2 + 4x_3 + 4x_4 + 4x_5 = \frac{46}{12} < 4 = |\Tilde{E}| - |\Tilde{V}|.$$
    In particular, the vector $x$ does not satisfy the constraint of $\weakdensitypolyhedronforsubgraphs(K_5)$ defined by the subgraph $\Tilde{G}$. Thus, we have that $\projectedorientationpolyhedron(K_5) \subseteq \weakdensitypolyhedronforsubgraphs(K_5)\subset \weakdensitypolyhedron(K_5)$. 
% For a point $(x,y) \in \orientationpolyhedron(G)$, and for every $S \subseteq V(G)$, we have
% \begin{align*}
% |E(S)| &\leq \sum_{e = vw \in E(S)} (x_v + x_w + y_{e,v} + y_{e,w})\\
%        &\leq \sum_{v \in S} (d_S(v)-1) x_v + \sum_{v \in S} \left( x_v + \sum_{e \in \delta(v)} y_{e,v}\right)\\ 
%        &\leq \sum_{v \in S} (d_S(v)-1) x_v + |S|
% \end{align*}
% which implies weak density constraints. 
% \knote{Add Stefan's example to show that the containment is strict.}

    \item By \Cref{thm:PFDS-WD-and-2PT-cover}, the ILP (\ref{PFDS-IP: WD-and-2PT-cover}) is a valid formulation of PFDS and its LP-relaxation (\ref{PFDS-LP: WD-and-2PT-cover}) has integrality gap at most $2$. We have already shown that the ILP (\ref{PFDS-IP:Orient-and-2PT-cover}) is a valid formulation of PFDS and that $\projectedorientationpolyhedron(G)\subseteq \weakdensitypolyhedron(G)$. Consequently, the integrality gap of (\ref{PFDS-LP:Orient-and-2PT-cover}) is at most $2$. 

\item The third conclusion follows from \Cref{thm:MC2PT-polytime:main} in the next subsection. It is based on the fact that node-weighted Steiner tree for constant number of terminals is solvable in polynomial time. We present its proof in a separate subsection for clarity. 


\end{enumerate}




% We now show that the ILP (\ref{PFDS-IP:orient}) formulates PFDS. 
% Let $S \subseteq V$ be a pseudoforest deletion set, i.e., 
%  the subgraph $F := G - S$ is a pseudoforest. Let $x := \chi^S \in \{0,1\}^{V}$ denote the indicator vector of the set $S$.  
%  Select an arbitrary orientation of the pseudoforest $F$ such that the maximum indegree of every vertex is at most $1$, and let $y_{e,v} := 1$ for the edge $e = vw$ if $e$ is an edge of $F$ that is oriented towards $v$, and $y_{e,v} := 0$ otherwise. 
%  Then, $(x,y) \in \orientationpolyhedron(G)$. 

\iffalse
We recall that the density of the graph $H$ is defined to be $\max\{|E[S]|/|S|: \emptyset\neq S\subseteq V(H)\}$. It suffices to show that the density of the graph $H$ is at most $1$. Let $A := \left\{\left(e, u\right), \left(e, v\right) : e = \left\{u, v\right\} \in E\right\}$. Since $x\in \projectedorientationpolyhedron(G)$, there exists a vector $y \in \R_{\ge 0}^{A}$ such that $(x, y) \in \orientationpolyhedron(G)$. For notational convenience, we let $V' := V - S$ and $E' := E[V - S]$ so that the graph $H= (V', E')$. Let $A' :=  \left\{\left(e, u\right), \left(e, v\right) : e = \left\{u, v\right\} \in E'\right\}$ and let $y' \in \R_{\ge 0}^{A'}$ be the projection of the vector $y$ onto the co-ordinates of the set $A'$. The following proposition establishes useful properties of this vector $y'$.

\begin{proposition}\label{claim:y'-feasible-for-charikarLPDual}
    We have the following:
    \begin{enumerate}
    \item $y'_{e, u} + y'_{e, v} \geq 1\ \forall e=uv \in E'$,
    \item $\sum_{e\in \delta_H(u)}y'_{e, u} \leq 1\ \forall u \in V'$, and
    \item $y'_{e, u} \geq 0\ \forall e \in \delta_H(u), u \in V'$.
\end{enumerate}
\end{proposition}
\begin{proof}
    The proposition holds since $x_u = 0$ for each $u \in V'$ and the vector $(x, y)$ satisfies the constraints describing the polyhedron $\orientationpolyhedron(G)$.
\end{proof}


We recall that the density of a graph is at most $1$ if and only if there exists a fractional orientation of its edges such that the indegree of every vertex in the oriented graph is at most 1. This follows from Charikar's results on the densest subgraph problem and we state it formally below\footnote{We note that \Cref{prop:charikar} is the feasibility version of the (dual) LP given in Section 3.1 of \cite{charikar_greedy_2000}.}: 


\begin{proposition}[\cite{charikar_greedy_2000}]\label{prop:charikar}
    Let $G = (V, E)$ be a graph and $A = \{(e, u), (e, v) : e =uv \in E\}$. Then, the density of the graph $G$ is at most 1 if and only if $P_{density}(G) \not = \emptyset$, where 
    \begin{align*}
P_{\text{density}}(G):=
\left\{ y \in \R^A: \begin{array}{l}
y_{e,u} + y_{e,v} \ge 1\ \forall e=uv\in E\\
\sum_{e\in \delta(u)} y_{e,u} \le 1\ \forall u \in V\\
y_{e,u} \ge 0\ \forall e\in \delta(u), u\in V.
  \end{array}\right\}.
\end{align*}

\end{proposition}

By \Cref{claim:y'-feasible-for-charikarLPDual}, we have that $y' \in P_{\text{density}}(H)$. Thus, the density of the graph $H$ is at most $1$ by \cref{prop:charikar}, and hence the graph $H=G-S$ is a pseudoforest.
%\knote{Need to show that support of $x$ is a feedback vertex set. The graph $G-S$ has a fractional orientation $y$ satisfying the ... constraints. By Charikar's results, this implies that the density of $G-S$ is at most $1$ and consequently, $G-S$ is a pseudoforest.} 
This completes the proof that the ILP (\ref{PFDS-IP:orient}) is indeed a valid formulation for PFDS. 
\fi

% Next, suppose $x\in \projectedorientationpolyhedron(G)\cap \Z^V$. Then, we have that $x\in \{0,1\}^V$. Let $S:=\{u\in V: x_u =1\}$. We will show that the set $S$ is a pseudoforest deletion set of the graph $G$, i.e. the subgraph $G - S$ is a pseudoforest. For a subgraph $\Tilde{G} = (\Tilde{V}, \Tilde{E})$ of the graph $G$, we let $d_{\Tilde{G}}(u):=|\delta(u)\cap \Tilde{E}|$ denote the degree of the vertex $u$ in the subgraph $\Tilde{G}$. We now define an intermediate polyhedron $\weakdensitypolyhedronforsubgraphs(G)$.
% \begin{align}
% \weakdensitypolyhedronforsubgraphs(G) 
%     &:= 
%     \left\{ x\in \R^{V}_{\ge 0}:\ \sum_{u\in \Tilde{V}}(d_{\Tilde{G}}(u)-1)x_u\geq |\Tilde{E}| - |\Tilde{V}| \ \forall \text{ subgraphs $\Tilde{G} = (\Tilde{V}, \Tilde{E})$ of $G$}.
% \right\}. \label{eqn:weak-density-for-subgraphs}
% \end{align}
% We note that $\weakdensitypolyhedronforsubgraphs(G) \subseteq \weakdensitypolyhedron(G)$. We now show that $\projectedorientationpolyhedron(G)\subseteq \weakdensitypolyhedronforsubgraphs(G)$. 
% Consequently, we will have that $x \in \weakdensitypolyhedron$. In particular, this will imply that the set $S$ is a pseudoforest deletion set for the graph $G$ since (\ref{PFDS-IP: WD}) is an ILP formulation for PFDS by \Cref{thm:PFDS-WD-and-2PT-cover}. 
% %Let $x \in\weakdensitypolyhedron(G)$ be an arbitrary vector. 
% Since $x \in \projectedorientationpolyhedron$, there exists a vector $y$ such that $(x, y) \in \orientationpolyhedron(G)$. 
% Let $\Tilde{G} = (\Tilde{V}, \Tilde{E})$ be an arbitrary subgraph of $G$. We have the following:
% \begin{align*}
% |\Tilde{E}| &\leq \sum_{e = vw \in \Tilde{E}} (x_v + x_w + y_{e,v} + y_{e,w})\\
%        &= \sum_{v \in \Tilde{V}} (d_{\Tilde{G}}(v)-1) x_v + \sum_{v \in \Tilde{V}} \left( x_v + \sum_{e \in \delta_{\Tilde{G}}(v)} y_{e,v}\right)\\ 
%        &\leq \sum_{v \in \Tilde{V}} (d_{\Tilde{G}}(v)-1) x_v + |\Tilde{V}|,
% \end{align*}
% where the inequalities are because the vector $(x,y) \in \orientationpolyhedron$. Rearranging the final terms gives the constraint of $\weakdensitypolyhedronforsubgraphs(G)$ as required. We now prove the three additional conclusions of the theorem statement. 

% \begin{enumerate}
% \item We begin by showing that there is a graph $G$ such that $\weakdensitypolyhedronforsubgraphs(G)\subset \weakdensitypolyhedron(G)$. In particular, we consider the graph $K_5 = (V, E)$ where $V = \{v_1, v_2, \ldots, v_5\}$ and $E = {V \choose 2}$. Let $x = (7/12, 7/12, 1/12, 0, 0)$. We note that $x \in \weakdensitypolyhedron(K_5)$.
% We now show that $x \not \in \weakdensitypolyhedronforsubgraphs(K_5)$.
%     Consider the subgraph $\Tilde{G} = (\Tilde{V}, \Tilde{E})$ obtained by removing the edge $\{v_1, v_2\}$ from the graph $G$, i.e. $\Tilde{V} = V$ and $\Tilde{E} = {V\choose 2} - \{v_1, v_2\}$.
%     Then, we have the following:
%     $$\sum_{u \in \Tilde{V}}(d_{\Tilde{G}}(u) - 1)x_u = 3x_1 + 3x_2 + 4x_3 + 4x_4 + 4x_5 = \frac{46}{12} < 4 = |\Tilde{E}| - |\Tilde{V}|.$$
%     In particular, the vector $x$ does not satisfy the constraint of $\weakdensitypolyhedronforsubgraphs(K_5)$ defined by the subgraph $\Tilde{G}$. Thus, we have that $\projectedorientationpolyhedron(K_5) \subseteq \weakdensitypolyhedronforsubgraphs(K_5)\subset \weakdensitypolyhedron(K_5)$. 
% % For a point $(x,y) \in \orientationpolyhedron(G)$, and for every $S \subseteq V(G)$, we have
% % \begin{align*}
% % |E(S)| &\leq \sum_{e = vw \in E(S)} (x_v + x_w + y_{e,v} + y_{e,w})\\
% %        &\leq \sum_{v \in S} (d_S(v)-1) x_v + \sum_{v \in S} \left( x_v + \sum_{e \in \delta(v)} y_{e,v}\right)\\ 
% %        &\leq \sum_{v \in S} (d_S(v)-1) x_v + |S|
% % \end{align*}
% % which implies weak density constraints. 
% % \knote{Add Stefan's example to show that the containment is strict.}

%     \item By \Cref{thm:PFDS-WD-and-2PT-cover}, the ILP (\ref{PFDS-IP: WD-and-2PT-cover}) is a valid formulation of PFDS and its LP-relaxation (\ref{PFDS-LP: WD-and-2PT-cover}) has integrality gap at most $2$. We have already shown that the ILP (\ref{PFDS-IP:Orient-and-2PT-cover}) is a valid formulation of PFDS and that $\projectedorientationpolyhedron(G)\subseteq \weakdensitypolyhedron(G)$. Consequently, the integrality gap of (\ref{PFDS-LP:Orient-and-2PT-cover}) is at most $2$. 

% \item The third conclusion follows from \Cref{thm:MC2PT-polytime:main} in the next subsection.


% \end{enumerate}
\end{proof}

\subsection{Separation oracle for the family of $2$-pseudotree cover constraints}
In this section, we show that the family of $2$-pseudotree cover constraints admits a polynomial time separation oracle.

%the polynomial-time tractability of the \emph{Minimum Cost $2$-Pseudotree} problem (MC2PT), which is the separation oracle for $2$-pseudotree cover constraints. The input to MC2PT is a vertex-weighted graph $\left(G = \left(V,E\right), w:V\rightarrow\R_{\geq 0}\right)$, and the goal is to find the minimum weight $2$-PT of the graph $G$ i.e., $\min\left\{\sum_{u\in U}w(u): U \text{ is a }2\text{-pseudotree of }G\right\}$. The following is the main result of this section. 

\begin{restatable*}{lemma}{NWtwoPTPolytime}
\label{thm:MC2PT-polytime:main}
The family of $2$-pseudotree cover constraints admits a polynomial time separation oracle. 
%The MC2PT problem can be solved in polynomial time.
\end{restatable*}

In order to solve the separation problem, it suffices to design a polynomial time algorithm for the \emph{Minimum Cost $2$-Pseudotree} problem:  The input to MC2PT is a vertex-weighted graph $\left(G = \left(V,E\right), w:V\rightarrow\R_{\geq 0}\right)$, and the goal is to compute a minimum weight subset $U\subseteq V$ of vertices such that $G[U]$ is a $2$-pseudotree, i.e., $$\min\left\{\sum \nolimits_{u\in U}w(u): U\subseteq V \text{ and } G[U] \text{ is a }2\text{-pseudotree}\right\}.$$ 
Our strategy to solve MC2PT is to reduce it to solving a polynomial number of special instances of the \emph{Minimum Node-Weighted Steiner Tree}\footnote{We refer to \emph{nodes} as \emph{vertices} for consistency with the rest of our technical sections.} (NWST), and use the result of \cite{NWST-Buchanan-et.al} which says that these special instances of NWST can be solved in polynomial time. Formally, the input to NWST is a vertex-weighted graph $\left(G = \left(V,E\right), w:V\rightarrow\R_{\geq 0}\right)$ and a terminal set $S \subseteq V$. We will say that a graph $H$ is a \emph{Steiner tree} in $G$ for terminal set $S$ if $H$ is connected, acyclic, and is a subgraph of $G$ with $S\subseteq V_H$. Moreover, the weight of a subgraph $H$ of $G$ is the sum of weights of vertices in $H$. The NWST problem is to find a  minimum weight Steiner tree in $G$ for terminal set $S$, i.e., $\min\{\sum_{u\in V_H} w(u): H=(V_H, E_H) \text{ is a Steiner tree in } G \text{ for terminal set } S\}$.
NWST can be solved in polynomial time if the number of terminals is a constant.

\begin{proposition}[Theorem 1 of \cite{NWST-Buchanan-et.al}]\label{prop:NWST-polytime} There exists a $O(3^kn + 2^kn^2+n^3)$ time algorithm for NWST, where $k$ denotes the number of terminals and $n$ denotes the number of vertices in the input instance.
\end{proposition}

The following proposition shows a correspondence between minimum weight $2$-pseudotrees in a vertex-weighted graph and Steiner trees.

\begin{proposition}\label{prop:MC2PT-to-NWST}
    Let $G=(V, E)$ be a graph with non-negative vertex weights $w: V\rightarrow \R_{\ge 0}$ and let $W\ge 0$. Then, there exists a subset $U\subseteq V$ such that $G[U]$ is a $2$-pseudotree with $\sum_{u\in U} w(u)\le W$ if and only if there exists a pair of edges $e_1=u_1v_1, e_2=u_2v_2$ in $G$ such that there exists a Steiner tree $H=(V_H, E_H)$ in the graph $G':=G-\{e_1, e_2\}$ for  terminal set $S:=\{u_1, v_1, u_2, v_2\}$ with $\sum_{u\in V_H}w(u)\le W$.
\end{proposition}
\begin{proof}
We will say that a graph $T=(V_T, E_T)$ is a \emph{minimal} $2$-pseudotree if it is connected and has exactly $|V_T|+1$ edges. Equivalently, $T$ has exactly $2$ edges in addition to a spanning tree. We observe that for a subset $U\subseteq V$, the subgraph $G[U]$ is a $2$-pseudotree if and only if there exists a subgraph $T=(U, E_T)$ of $G[U]$ such that $T$ is a minimal $2$-pseudotree. Hence, it suffices to show that there exists a subset $U\subseteq V$ such that $G[U]$ has a subgraph that is a minimal $2$-pseudotree with $\sum_{u\in U} w(u)\le W$ if and only if there exists a pair of edges $e_1=u_1v_1, e_2=u_2v_2$ in $G$ such that there exists a Steiner tree $H=(V_H, E_H)$ in the graph $G':=G-\{e_1, e_2\}$ for  terminal set $S:=\{u_1, v_1, u_2, v_2\}$ with $\sum_{u\in V_H}w(u)\le W$. We prove this statement now. 

Let $U\subseteq V$ such that $G[U]$ contains a subgraph $T=(U, E_T)$ that is a minimal $2$-pseudotree with $\sum_{u\in U} w(u)\le W$.  
Then, there exists a pair of edges $e_1=u_1v_2$ and $e_2=u_2v_2$ in $T$ such that the subgraph $H:=T-\{e_1, e_2\}$ is acyclic, connected, and is a subgraph of $T$. In particular, we have that $H$ is acyclic, connected, and is a subgraph of $G':=G-\{e_1, e_2\}$ with $S=\{u_1, v_1, u_2, v_2\}\subseteq U=V(H)$. Hence, for the pair of edges $e_1, e_2$ in $G$, we have that $H$ is a Steiner tree in $G'$ for terminal set $S$ with $\sum_{u\in V(H)}w(u)=\sum_{u\in U} w(u) \le W$. 

Next, let $e_1=u_1v_2, e_2 = u_2v_2$ be a pair of edges in $G$ such that there exists a Steiner tree $H=(V_H, E_H)$ in the graph $G':=G-\{e_1, e_2\}$ for terminal set $S=\{u_1, v_1, u_2, v_2\}$ with $\sum_{u\in V_H}w(u)\le W$. Consider the graph $T:=H+\{e_1, e_2\}$. Then, $T$ is a minimal $2$-pseudotree. The vertex set of $T$ is $U:=V_H$ and $T$ is a subgraph of $G[U]$. Hence, we have a subset $U\subseteq V$ such that $G[U]$ has a subgraph that is a minimal $2$-pseudotree with $\sum_{u\in U}w(u)\le W$. 
\end{proof}


% \knote{Cite the above proposition in the proof of Lemma 1 below and complete its proof. Remove the rest.} 

% A tree $H = (V_H, E_H)$ of a graph $G = (V, E)$ is a \emph{Steiner tree} for the terminal set $S\subseteq V$ if $S \subseteq V_{H}$. Furthermore, the subgraph $T = (V_T, E_T)$ is said to be a \emph{minimal} $2$-pseudotree of the graph $G$ if $T$ is connected and the exclusion of \emph{exactly} 2 edges of $T$ results in a tree of $G$ \knote{\st{of $G$} spanning $V_T$}. We note that both Steiner trees and minimal $2$-pseudotrees are subgraphs, and need not be induced subgraphs of the graph $G$ \knote{Minimal $2$-pseudotrees have nothing to do with $G$. They are simply inclusionwise minimal $2$-pseudotrees, where inclusionwise minimal is with respect to the edge set. }.
% The following proposition shows a correspondence between minimal $2$-pseudotrees and Steiner trees.

% \begin{proposition}\label{lem:MC2PT-to-NWST}
% \knote{Let $G = (V, E)$ be a graph and $T=(V_T, E_T)$ be a subgraph of $G$. Then, $T = (V_T, E_T)$ is a minimal 2-pseudotree if and only if there exist a pair of edges $e_1 = u_1v_1, e_2 = u_2v_2\in E_T$ such that the subgraph $H = (V_T, E_T - \{e_1, e_2\})$ is a Steiner tree in the graph $G' = G - \{e_1, e_2\}$ for the terminal set $S = \{u_1, v_1, u_2, v_2\}$.}

% Let $G = (V, E)$ be a graph \knote{and $T=(V_T, E_T)$ be a subgraph of $G$}. Then, the subgraph $T = (V_T, E_T)$ is a minimal 2-pseudotree of the graph $G$ \knote{\st{of the graph $G$}} if and only if there exist edges $e_1 = u_1v_1$ and $e_2 = u_2v_2$ \knote{in $T$} such that the subgraph $H = (V_T, E_T - \{e_1, e_2\})$ is a Steiner tree in the graph $G' = G - \{e_1, e_2\}$ for the terminal set $S = \{u_1, v_1, u_2, v_2\}$.
% \end{proposition}
% \begin{proof}
% \knote{Edit the proof to reflect the statement.}

% Let $T = (V_T, E_T)$ be a minimal $2$-pseudotree of the graph $G$. 
% Then, there exist two edges $e_1 = u_1v_1$ and $e_2 = u_2v_2$ such that the subgraph $H := T - \{e_1, e_2\}$ is a tree in $G$ that spans $V_T$. We note that the subgraph $T$ is in fact a tree in the graph $G' := G - \{e_1, e_2\}$ and contains the vertex set $S := \{u_1, v_1, u_2, v_2\}$. Thus, the subgraph $H$ is a Steiner tree in the graph $G'$ for the terminal set $S$.

% Let $T = (V_T, E_T)$ be a subgraph of $G$ and edges $e_1 = u_1v_1$, $e_2 = u_2v_2$ be such that the subgraph $H := T - \{e_1, e_2\} $ is a Steiner tree in the graph $G' := G - \{e_1, e_2\}$ for the terminal set $S := \{u_1, v_1, u_2, v_2\}$. We note that the subgraph $H$ is also a tree in the graph $G$, i.e. the deletion of the two edges $e_1, e_2$ results in a tree in $G$. Thus, the subgraph $T$ is a minimal $2$-pseudotree of the graph $G$ by definition.
% \end{proof}

% The following corollary to \Cref{lem:MC2PT-to-NWST} establishes the exact reduction between the two optimization problems MC2PT and NWST.

% \begin{corollary}\label{cor:MC2PT-to-NWST}
% Let $G = \left(V, E\right)$ be a graph with vertex weights $w:V\rightarrow\R_{\geq 0}$ \knote{and $W\in \R_{\ge 0}$}. Then, the graph $G$ has a $2$-pseudotree $T = (V_T, E_T)$ with weight at most $ w(V_T)$ \knote{$W$} if and only if there exist edges $e_1 = u_1v_1$ and $e_2 = u_2v_2$ such that the graph $G' = G - \{e_1, e_2\}$ with vertex weights $w$ has a Steiner tree for the terminal set $S = \{u_1, v_1, u_2, v_2\}$ with weight at most $w(V_T)$ \knote{$W$}.
% \end{corollary}
% \begin{proof}
%     We note that every $2$-pseudotree contains a minimal $2$-pseudotree.
%     The claim then follows from \Cref{lem:MC2PT-to-NWST} and the observation that both the $2$-pseudotree $T$ and the steiner tree $H$ have the same vertex set.
% \end{proof}

We now restate and prove \Cref{thm:MC2PT-polytime:main}.

\NWtwoPTPolytime
\begin{proof}
    It suffices to show that MC2PT can be solved in polynomial time. Let $G=(V,E)$ with vertex weights $w:V\rightarrow \R_{\ge 0}$ be the input instance of MC2PT.
    Consider the following algorithm: for all pairs of edges $e_1 = u_1v_1$ and $e_2 = u_2v_2$ in $E$, use the algorithm guaranteed by \Cref{prop:NWST-polytime} to solve NWST on the graph $G - \{e_1, e_2\}$ with vertex weights $w$ for terminal set $\{u_1, v_1, u_2, v_2\}$, and return the minimum weight solution over all instances. The correctness of the algorithm follows from \Cref{prop:MC2PT-to-NWST}. We now analyze the runtime of the algorithm. There are $O(|V|^2)$ pairs of edges to enumerate. Furthermore, for each pair of edges, the associated NWST instance can be solved in polynomial time using the algorithm from \Cref{prop:NWST-polytime} since each such instance has only four terminals. Thus, the algorithm runs in polynomial time.
\end{proof}