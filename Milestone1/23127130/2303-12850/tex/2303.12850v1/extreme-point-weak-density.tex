
\section{Extreme Point Property of the Weak Density Polyhedron}\label{sec:extreme-point-weak-density}
We prove  \Cref{thm:weak-density-extreme-point} in this section. 
For a fixed $x\in \R^V$, we define the function $f_x: 2^V\rightarrow \R$ as 
\[
f_x(S) := |E[S]|-|S| - \sum_{u\in S}(d_S(u)-1)x_u\ \forall S\subseteq V. 
\]
Using this function, we can express the weak density polyhedron as 
\[
    \weakdensitypolyhedron(G)
     = \left\{x\in\nnreal^{V} : f_x(S) \leq 0\  \forall S \subseteq V\right\}.
\]

In \Cref{sec:supermodular-background}, we recall properties of supermodular functions and chain set families. 
In  \Cref{sec:conditional-supermodularity-of-weak-density-constraints}, we show that if $x_u\le 1/2$ for all $u\in V$, then the function $f_x:2^V\rightarrow \R$ is supermodular. This conditional supermodularity property allows us to uncross the tight constraints that form a basis of an extreme point $x$ for which $x_u\le 1/2$ for all $u\in V$. We prove properties about the tight constraints in  \Cref{sec:conditional-properties-tight-sets} and show the existence of a \emph{chain basis} in  \Cref{sec:conditional-basis-structure-for-extreme-point}. Using the chain basis structure and its properties, we complete the proof of  \Cref{thm:weak-density-extreme-point} in \Cref{sec:counting-argument}. 

\paragraph{Notation.} Let $b(S):=|E[S]|-|S|$ for all $S\subseteq V$. For vertex $u\in V$, let $\chi_u \in \{0,1\}^V$ denote the indicator vector of $u$. For a subset $S\subseteq V$, let $\row(S)$ denote the row of the constraint matrix of $\weakdensitypolyhedron(G)$ corresponding to the set $S$, i.e, $$\row(S)_u = \begin{cases}
    d_S(u) - 1 & \text{ if $u \in S$} \\
    0 & \text{ o.w.}\\
\end{cases} \qquad \forall u \in V.$$
For $J \subseteq 2^V$, let $\Rows(J) = \bigcup_{S\in J}\row(S)$. When the context is clear, we also use $\Rows(J)$ to denote the submatrix of the weak density polyhedron constraint matrix given by the set $\Rows(J)$.
We let $\spanfunc(\Rows(J))$ denote the \emph{span} of the set of vectors $\Rows(J)$, i.e. the smallest linear subspace that contains the set  $\Rows(J)$. We say that the set $\Rows(J)$ is a \emph{basis} for an extreme point $x \in P(G)$ of the weak density polyhedron if $\spanfunc(\Rows(J)) = \spanfunc(\Rows(2^V))$. We let $\mathsf{columns}(J)$ denote the set of columns of the submatrix $\Rows(J)$. For matrix $M$, we denote $\mathsf{rank}(M)$ as the \emph{rank} of the matrix. We note that $\mathsf{rank}(\Rows(J)) = \mathsf{rank}(\mathsf{columns}(J))$. For $x\in \R^V$, we will use $g_x(S):=\sum_{u\in S}(d_S(u)-1)x_u$ for all $S\subseteq V$, $\calT:=\{S\subseteq V: f_x(S)=0\}$ be the tight sets for $x$, and $\calZ:=\{u\in V: x_u=0\}$.

\subsection{Background on supermodular functions and set families}\label{sec:supermodular-background}
In this section, we recall supermodular functions, chain set families, and some of their properties that will be useful while proving the extreme point property of the weak density polyhedron. 
A set function $f:2^V\rightarrow \mathbb{R}$ is said to be \emph{supermodular} if $f(A) + f(B) \leq f(A\cap B) + f(A\cup B)$ for all subsets $A, B \subseteq V$. 
We refer the reader to \cite{Fujishige05} for background on supermodular set functions and their properties. 
We will rely on the following property:
%that the function $f: 2^V\rightarrow \R_{\ge 0}$ defined by $f(S) := \sum_{e\in E[S]}w_e$ for all $S\subseteq V$ is supermodular. 
\begin{proposition}\label{prop:supermodularity-of-graph-edge-functions}
Let $G = (V, E)$ be an undirected graph with non-negative edge weights $w: E\rightarrow \R_{\ge 0}$. Then, the function $f: 2^V\rightarrow \R_{\ge 0}$ defined by $f(S) := \sum_{e\in E[S]}w_e$ for all $S\subseteq V$ is supermodular. 
\end{proposition}
Two sets $A$ and $B$ are said to \emph{cross} if they have a non-empty intersection and neither set is contained in the other, i.e. $A\cap B, A - B, B - A \not = \emptyset$.
For a ground set $V$, the set $\calC = \{A_1, A_2, \ldots A_\ell \} \subseteq 2^{V}$ is said to be a \emph{chain family} if its elements can be ordered such that $A_1 \subseteq A_2 \subseteq \ldots \subseteq A_\ell$. We will need the following proposition on chain families. 

%\knote{Define chain family before proposition.}
\begin{proposition}\label{prop:chain-uncrossing}
Let $\calC$ be a chain family and $A$ be a subset that crosses some set $B\in \calC$. Then, the number of sets in $\calC$ crossed by $A\cup B$ and $A\cap B$ is atmost the number of sets in $\calC$ crossed by $A$.
\end{proposition}
\begin{proof}
Our strategy will be to pick an arbitrary set in $\calC$ that crosses the set $A\cup B$ (resp. $A\cap B$) and show that this picked set also crosses the set $AS$. The claim then follows since the set $B$ crosses the set $A$ but not the set $A\cup B$ (resp $A\cap B$).

Let $P\in\calC$ be an arbitrary set that crosses the set $A\cup B$. Since both $B, P\in \calC$, it must be that either $P \subset B$ or $B\subset P$. First, we consider the case where $P \subset B$. Then $P \subset A\cup B$ and thus does not cross the set $A\cup B$ as it is contained in it. This contradicts the choice of $P$. Next, we consider the case where $B\subset P$. If $A \subseteq P$ then $A\cup B \subseteq P$, contradicting choice of $P$ crossing $A\cup B$. Furthermore, if $P \subseteq A$, then $B \subset P \subseteq A$, contradicting the hypothesis that the set $B$ crosses the set $A$. Finally, $A\cap P \not = \emptyset$ as $A\cap B\not = 0$ and $B\subset P$. Thus the set $P$ crosses the set $A$.

Let $P\in\calC$ be an arbitrary set that crosses the set $A\cap B$. Since the sets $P, B \in \calC$ are part of a chain family, it must be that either $P\subseteq B$ or $B\subseteq P$. If $B \subseteq P$, then $A\cap B \subseteq P$, contradicting the choice of set $P$. Thus $P \subseteq B$. If $P \subseteq A$, then $P \subseteq A\cap B$, contradicting the choice of set $P$. Thus the set $P$ crosses the set $A$.
\end{proof}



\subsection{Conditional Supermodularity of Weak Density Constraints}\label{sec:conditional-supermodularity-of-weak-density-constraints}
In this section, we show that the function $f_x$ is supermodular if  $x_u < \frac{1}{2}$ for each vertex $u \in V$. 

\begin{lemma}\label{lem:supermodularity}
%[Conditional supermodularity of $f_x$]
Let $x\in \R^V$. If $x_u < \frac{1}{2}\ \forall u\in V$, then the function $f_x$ is  supermodular.
\end{lemma}
\begin{proof}
For each $S\subseteq V$, we have that 
\begin{align*}
    f_x(S)
    & = |E[S]| - |S| - \sum_{u\in S}(d_S(u) - 1)x_u&\\
    & = |E[S]| - \sum_{u\in S}d_S(u) - |S| + \sum_{u\in S}x_u&\\
    &= |E[S]| - \sum_{u \in S}\sum_{v \in \delta(u)}x_u - \sum_{u \in S}(1 - x_u)&\\
    & = |E[S]| - \sum_{uv\in E[S]}(x_u + x_v) - \sum_{u \in S}(1 - x_u)&\\
    &= \sum_{uv\in E[S]}\left(1 - (x_u+x_v)\right) - \sum_{u\in S}\left(1 - x_u\right).&
\end{align*}
Thus, the funcion $f_x$ can be expressed as $f_x=p_x-q_x$ for functions $p_x, q_x: 2^V\rightarrow \R$ defined by $p_x(S) :=\sum_{uv\in E[S]}(1 - (x_u + x_v))$ and $q_x(S) := \sum_{u\in S}(1 - x_u)$ for all $S\subseteq V$. Since $x_u<1/2$ for all $u\in V$, we have that $p_x(S)\ge 0$ and $q_x(S)\ge 0$ for all $S\subseteq V$. 

Now, let $H=(V, E)$ be a graph with edge weights $w:E\rightarrow \R$ given by $w(uv):=1-(x_u+x_v)$ for each $uv\in E$. We note that all edge weights are non-negative since $x_a<1/2$ for all $a\in V$ and $p_x(S)=\sum_{uv\in E[S]}w(uv)$ for all $S\subseteq V$. By \Cref{prop:supermodularity-of-graph-edge-functions}, the function $p_x$ is supermodular. Moreover, the function $q_x$ is modular. Thus, the function $f_x$ is the sum of a supermodular function and a modular function. Consequently, $f_x$ is a supermodular function. 
\end{proof}


\subsection{Conditional Properties of Tight Sets}\label{sec:conditional-properties-tight-sets}
In this section, we prove certain properties of tight sets which will help us obtain a well-structured basis for extreme point $x$ under the condition that $x_u<1/2$ for every $u\in V$. 
%For an extreme point $x$ of $\weakdensitypolyhedron(G)$, let $\calT := \{S\subseteq V : f_x(S) = 0\}$ denote the collection of \emph{tight} sets for $x$.  
%Let $\calZ = \{u : x_u = 0\}$ be the set of $0$ assigned vertices at extreme point $x$. 

%\knote{In this section, we prove certain properties of tight sets which will help us obtain a well-structured basis for extreme points $x$ for which $x_u<1/2$ for every $u\in V$. }
%\knote{Unify lemmas into a single lemma with several items.}

\begin{lemma}[Conditional Uncrossing Properties]\label{lem:tight-sets-conditional-uncrossing-properties}
Let $x$ be an extreme point of $\weakdensitypolyhedron(G)$ such that $x_u < \frac{1}{2}$ for all $u\in V$ and the family of tight sets for $x$ be $\calT := \{S\subseteq V : f_x(S) = 0\}$. 
Let $A,B\in\calT$. Then,
\begin{enumerate}
    \item $A\cap B \neq \emptyset$, i.e., tight sets overlap,
    \item $A\cap B, A\cup B\in \calT$, i.e. tight sets form a lattice family, 
    \item $\delta(A-B, B-A)=\emptyset$, i.e. tight sets admit no crossing edges, and 
    \item $\row(A) + \row(B) = \row(A\cap B) +\row(A\cup B)$.%, i.e. row vectors of tight sets span the row vectors of their meet and join.
    %\item $G[A]$ is connected, and 
\end{enumerate}
\end{lemma}
\begin{proof}
\begin{enumerate}
    \item By way of contradiction, assume that $A\cap B = \emptyset$. We have that
    \begin{align*}
        b(A\cup B)&\leq g_x(A\cup B)&\\
        & = g_x(A) + g_x(B) + \sum_{ab \in \delta(A,B)}(x_a + x_b)&\\
        & = |E[A] - |A| + |E[B] - |B| + \sum_{ab \in \delta(A,B)}(x_a + x_b)&\\
        & = |E[A\cup B]| - |A\cup B|- |\delta(A, B)| + \sum_{ab \in \delta(A,B)}(x_a + x_b)&\\
        & = b(A \cup B) - |\delta(A, B)| + \sum_{ab \in \delta(A,B)}(x_a + x_b)&\\
        &< b(A \cup B) - |\delta(A, B)| + |\delta(A, B)|,&
    \end{align*}
    a contradiction. Here, the first inequality is by the weak density constraint for the set $A\cup B$, and the next equality is by the hypothesis that $A\cap B = \emptyset$. The final inequality is due to $x_u < \frac{1}{2}$ for each vertex $u \in V$.
    
    \item We have the following: 
    $$0 = f_x(A) + f_x(B) \leq f_x(A\cup B) + f_x(A\cap B) \leq 0.$$
    Here, the first equality is due to the sets $A, B \in \calT$, and thus $f_x(A) = f_x(B) = 0$. The first inequality follows from supermodularity of the function $f_x$ shown in \Cref{lem:supermodularity}. The final inequality follows from weak density constraints for the sets $A\cup B$ and $A\cap B$.
    Thus, all inequalities are  equalities, and we have that $f_x(A\cap B) = f_x(A\cup B) = 0$ using weak density constraints on the respective sets.
    
    \item By way of contradiction, assume that $\delta(A-B, B-A)\not = \emptyset$.
    Since $A,B\in\calT$, we have that $A\cap B$, $A\cup B \in\calT$ by the previously shown property that tight sets uncross. Thus, $f_x(A) = f_x(B) = f_x(A\cap B) = f_x(A\cup B) = 0$. 
    %We recall from \Cref{prop:function-conditional-reformulation} that $f_x = p_x - q_x$ where $p_x$ is a supermodular set function and $q_x$ is a modular set function. This gives us the following:
    We have that 
    \begin{align*}
        0  = f_x(A) + f_x(B) - f_x(A\cap B) - f_x(A\cup B)
        %& = p_x(A) + p_x(B) - p_x(A\cap B) - p_x(A\cup B)& \\
         = \sum_{uv\in \delta(A-B, B-A)}\left(1 - (x_u+x_v)\right)
        > 0,
    \end{align*}
    a contradiction. Here, the second equality is by definition of $f_x$, and the final inequality is because $x_u <\frac{1}{2}$ for each $u\in V$.
%\end{enumerate}
%\end{proof}

%\begin{lemma}[$\row$ Vectors Uncross]\label{lem:row-vector-uncrossing}
%Suppose $x_u < \frac{1}{2}$ for all $u\in V$. If $A, B \in \calT$, then  $$\row(A) + \row(B) = \row(A\cap B) +\row(A\cup B).$$
%\end{lemma}
%\begin{proof}
\item Let $u \in V$ be an arbitrary vertex. It suffices to show that $(d_A(u) - 1)+ (d_B(u) - 1) = (d_{A\cup B}(u) - 1) + (d_{A\cap B}(u) - 1)$. We consider four cases:

First, we consider the case $u \in A-B$. Then, we have that (1) $d_A(u) = d_{A-B}(u)+\delta(u, A\cap B)$, (2) $d_B(u) = 0$ (3) $d_{A\cap B}(u) = 0$ and (4) $d_{A\cup B}(u) = d_{A-B}(u)+\delta(u, A\cap B)$. Here, (4) follows from $\delta(u, B - A) = \emptyset$ by the previous part. Thus, the claimed equality follows. We note that the argument for the case where $u \in B-A$ is similar to this case.

Next, we consider the case $u \in A\cap B$. Then, we have that (1) $d_A(u) = d_{A\cap B}(u)+\delta(u, A - B)$, (2) $d_B(u) = d_{A\cap B}(u)+\delta(u, B - A)$,  and (3) $d_{A\cap B}(u) = d_{A\cap B}(u)+\delta(u, A - B)+ \delta(u, B - A)$. Thus, the claimed equality  follows.

Finally, we consider the case $u \in V - (A\cup B)$. Then, we have that $d_A(u) =d_B(u) =d_{A\cap B}(u) =d_{A\cup B}(u) = 0$ and the claimed equality follows.
%\end{proof}
\end{enumerate}
\end{proof}

Next, we show that every tight set is of size at least $2$ and the graph induced over the tight set is connected. 
\begin{lemma}\label{lem:tight-set-connected}
Let $x$ be an extreme point of $\weakdensitypolyhedron(G)$ such that $x_u < \frac{1}{2}$ for all $u\in V$ and the family of tight sets for $x$ be $\calT := \{S\subseteq V : f_x(S) = 0\}$. For every $A\in \calT$, we have that $|A|\ge 2$ and the graph $G[A]$ is connected. 
\end{lemma}
\begin{proof}
Let $A\in\calT$. 
Suppose that $|A|=1$. Let $A=\{u\}$. Then, $A\in \calT$ implies that $f_x(\{u\})=0$. Equivalently, $-x_u=-1$ and hence, $x_u=1$, a contradiction. Hence, $|A|\ge 2$. Next, we show that $G[A]$ is connected. By way of contradiction, let $A = A_1\uplus A_2$ such that $G[A_1], G[A_2]$ are disconnected components of $G[A]$. Then, we have the following:
\begin{align*}
    g(A_1) + g(A_2)&= g(A) &\\
    &= b(A)&\\
    &= |E[A]| - |A|&\\
    &= |E[A_1]| - |A_1| + |E[A_2]| - |A_2|&\\
    &= b(A_1) + b(A_2)&\\
    &\leq g(A_1) + g(A_2).&
\end{align*}
Here, the first and fourth equalities are because $A_1\cap A_2 = \emptyset$. The second equality holds because $A\in\calT$. The final inequality holds by the weak density constraints on $A_1$ and $A_2$.

The chain of inequalities implies that the final inequality is an equality. By weak density constraints for $A_1$ and $A_2$, we also have that $b(A_1) \le g(A_1)$ and $b(A_2)\le g(A_2)$ and consequently, $A_1$ and $A_2$ are tight sets, i.e., $A_1, A_2\in\calT$. However, $A_1\cap A_2 = \emptyset$ by assumption, contradicting \Cref{lem:tight-sets-conditional-uncrossing-properties} that tight sets must overlap.
\end{proof}

% \knote{
% % \begin{lemma}
% % Suppose $x_u < \frac{1}{2}$ for all $u\in V$. Let $A,B\in\calT$. Then,
% % \begin{enumerate}
% %     \item $A\cap B \neq \emptyset$, i.e., tight sets overlap,
% %     \item $A\cap B, A\cup B\in \calT$, 
% %     \item $\delta(A-B, B-A)=\emptyset$. 
% % \end{enumerate}
% % \end{lemma}
% % \begin{proof}
% % TODO
% % \end{proof}

% \begin{lemma}
% Suppose $x_u < \frac{1}{2}$ for all $u\in V$. For every $A\in \calT$, the following hold:
% \begin{enumerate}
%     \item The graph $G[A]$ has no isolated vertices and
%     \item The graph $G[A]$ is connected.
% \end{enumerate}
% \end{lemma}
% \begin{proof}
% TODO
% \end{proof}

% }


\subsection{Conditional Basis Structure for Extreme Points}
\label{sec:conditional-basis-structure-for-extreme-point}

In this section, we use the conditional structural properties of tight sets proved in \Cref{sec:conditional-properties-tight-sets} to show that every extreme point $x$ for the weak density polyhedron for which $x_u<1/2$ for all $u\in V$ has a well-structured basis. 
%In particular, we show in \Cref{lem:chain-basis-existence} that there is a basis in which the corresponding tight sets form a chain family---we refer to this as a \emph{chain basis}. Next, in \Cref{claim:2-pseudotree-containment}, we show that each tight set in a basis must contain a $2$-pseudotree. Finally, in \Cref{lem:basis-with-tight-set-degree-atleast-2} we show that there is a chain basis in which every subgraph induced by every tight set in the chain family must have minimum degree at least $2$.
 %Let $\calZ = \left\{u \in V : x_u = 0\right\}$ i.e. $\calZ$ denotes the set of vertices which do not belong to the support of the extreme point $x$. We recall that $\calT := \{S\subseteq V : f_x(S) = 0\}$ denotes the family of tight sets.
We recall that a $2$-pseudotree is a connected graph that has at least one more vertex than the number of edges. The following lemma is the main result of this section.

\begin{lemma}\label{lem:chain-basis-existence}
Let $x$ be an extreme point of $\weakdensitypolyhedron(G)$ such that $x_u < \frac{1}{2}$ for all $u\in V$, the family of tight sets for $x$ be $\calT := \{S\subseteq V : f_x(S) = 0\}$, and let $\calZ := \left\{u \in V : x_u = 0\right\}$. 
Then, there exists a family $\calC\subseteq\calT$ such that
\begin{enumerate}[label=(\arabic*)]
    \item the family $\calC$ is a chain family,
    \item the set of vectors $\Rows(\calC\cup\calZ)$ is linearly independent,
    \item $\spanfunc(\Rows(\calC\cup\calZ)) = \spanfunc(\Rows(\calT\cup\calZ))$,
    \item For each $S \in \calC$, the subgraph $G[S]$ contains a $2$-psuedotree, 
    \item For each $S \in \calC$ and each vertex $u \in S$, we have that $d_S(u) \geq 2$, and 
    \item For every $A, B\in \calC$ such that $A\subset B$, there exists a vertex $v\in B-A$ such that $x_v>0$. 
\end{enumerate}
%$$\spanfunc(\Rows(\calC)) = \spanfunc(\Rows(\calT)).$$
\end{lemma}
\begin{proof}
    We first show that there exists a family satisfying properties (1)-(4). Let $\calC^{(1)} \subseteq \calT$ be an inclusion-wise maximal chain family. \Cref{claim:Cspan=Tspan} below shows that $\spanfunc(\Rows(\calC^{(1)})) = \spanfunc(\Rows(\calT))$.

\begin{claim}\label{claim:Cspan=Tspan}
$\spanfunc(\Rows(\calC^{(1)})) = \spanfunc(\Rows(\calT)).$
\end{claim}
\begin{proof}
%Let $\calC$ be an inclusion-wise maximal chain family in $\calT$. We show that this chain family $\calC$ satisfies the claim.
 By way of contradiction assume false. Let $A \in \calT$ such that $\row(A) \not \in \spanfunc(\Rows(\calC^{(1)}))$ and $A$ crosses the fewest number of sets in $\calC^{(1)}$. Consider a set $B\in \calC^{(1)}$. Recall that by \Cref{lem:tight-sets-conditional-uncrossing-properties}, the sets $A\cap B, A\cup B \in \calT$ are also tight. We note that by \Cref{prop:chain-uncrossing}, the sets $A\cap B$ and $A\cup B$ cross fewer sets in $\calC^{(1)}$ than the number of sets in $\calC^{(1)}$ crossed by $A$. We consider two cases based on whether $\row(A\cup B), \row(A\cap B) \in \spanfunc(\Rows(\calC^{(1)}))$. First, consider the case where $\row(A\cup B) \not \in \spanfunc(\Rows(\calC^{(1)}))$ without loss of generality. Since $A\cup B$ crosses fewer sets in $\calC$, the set $A\cup B$ contradicts the choice of $A$. 
Next, consider the case where $\row(A\cup B), \row(A\cap B)  \in \spanfunc(\Rows(\calC^{(1)}))$.
By \Cref{lem:tight-sets-conditional-uncrossing-properties}, 
%\Cref{lem:row-vector-uncrossing}, 
we have that $\row(A) + \row(B) = \row(A\cup B) + \row(A\cap B)$. Thus $\row(A) \in \spanfunc(\Rows(B, A\cap B, A\cup B))$, contradicting choice of $A$.
\end{proof}

    
    Let $\calC^{(2)} \subseteq \calC^{(1)}$ be an inclusion-wise maximal family such that the set $\Rows(\calC^{(2)} \cup \calZ)$ is linearly independent. We note that 
    $\spanfunc(\Rows(\calC^{(2)}\cup\calZ)) 
 = \spanfunc(\Rows(\calC^{(1)}\cup\calZ)) = \spanfunc(\Rows(\calT\cup\calZ))$, where the first equality is because the family $\calC^{(2)}$ is inclusion-wise maximal. In particular, we have that the family $\calC^{(2)}$ satisfies properties (1)-(3).
%our choice of the family $\calC$ satisfies properties (1) and (2). \Cref{claim:Cspan=Tspan} below shows that $\spanfunc(\Rows(\calC)) = \spanfunc(\Rows(\calT))$. By our choice of the family $\calC$, this gives us that the set $\Rows(\calC \cup \calZ)$ is a basis for the vector space $\Rows(\calT \cup \calZ)$ and thus satisfies propety (3). 
\Cref{claim:2-pseudotree-containment} below shows that the family $\calC^{(2)}$ also satisfies property (4). 

\begin{claim}\label{claim:2-pseudotree-containment}
%Let $x$ be an extreme point of $\weakdensitypolyhedron(G)$ such that $x_u < \frac{1}{2}$ for all $u\in V$ and $\calT := \{S\subseteq V : f_x(S) = 0\}$. Then, 
For every $S\in \calC^{(2)}$, the subgraph $G[S]$ contains a $2$-pseudotree.
\end{claim}
\begin{proof}
By way of contradiction, let $S\in\calC^{(2)}$ such that $G[S]$ does not contain a $2$-pseudotree. Let $S^{i}:=\{u\in S: d_S(u)=i\}$ and  $S^{\geq i}:=\{u\in S: d_S(u)\geq i\}$. 
We note that $b(S) \leq 0$ since $G[S]$ does not contain a $2$-pseudotree. Hence, we have that
\begin{align*}
    0 \ \geq \ b(S) \ &= g(S)&\\
    &= \sum_{u\in S}(d_S(u)-1)x_u&\\
    &= \sum_{u\in S^{0}}(d_S(u)-1)x_u + \sum_{u\in S^{1}}(d_S(u)-1)x_u + \sum_{u\in S^{\geq 2}}(d_S(u)-1)x_u&\\
    &\geq \sum_{u\in S^{\geq 2}}x_u&\\
    &\geq 0&
\end{align*}
Here, the first equality holds because $S\in\calT$. The final inequality holds since $|S|\ge 2$ and $G[S]$ has no isolated vertices by  \Cref{lem:tight-set-connected}, and by the non-negativity constraints on vertex variables of $\weakdensitypolyhedron(G)$.

Thus, all inequalities above are equalities, and consequently, we have that $\sum_{u\in S^{\geq 2}}x_u = 0$ 
%\knote{DEfine $S^{\ge 2}$ and $S^{\le 1}$}.
This, coupled with the non-negativity constraints on vertex variables, implies that $x_u = 0$ for each $u\in S^{\geq 2}$. We also note that $\row(S)_u = 0$ for each $u \in S^{ 1}\cup S^{0}$ as $S^{0} = \emptyset$ and $d_S(u) - 1 = 0$ for $u\in S^1$.  Thus, we have that $\row(S) \in \spanfunc(\Rows(\calZ))$, contradicting linear independence of $\Rows(\calC^{(2)}\cup\calZ)$. %\knote{Why is $\calC\cup \calZ$ linearly independent? If it is linearly independent, then it should be stated as an additional conclusion in the previous lemma statement.}
%\shubhang{Fixed. The family is now linearly independent by choice.}.
\end{proof}


We now show the existence of a family satisfying properties (1)-(5). Let $\calC\subseteq\calT$ be a family satisfying properties (1)-(4) which minimizes $\sum_{S \in\calC}|S|$. We will show that this $\calC$ also satisfies property (5).
%We note if the family $\calC^{(2)}$ satisfies property (5), then we are done. Thus, we assume that property (5) fails and show how to construct a new family $\calC$ from $\calC^{(2)}$ that satisfies property (5) while maintaining properties (1)-(4). 
    By way of contradiction, suppose there exists $S\in\calC$ with $u \in S$ such that $d_S(u) < 2$. Since $S\in\calT$, we have that $|S|\ge 2$ and $d_S(u)\geq 1$ since tight sets cannot have isolated vertices by \Cref{lem:tight-set-connected}. Thus, $d_S(u) = 1$. Let $uv \in E[S]$ be the unique edge incident to $u$ in $G[S]$. 
We note that $E[S]$ should contain at least one more edge apart from the edge $uv$ as otherwise $\row(S) = 0$. Furthermore, by \Cref{lem:tight-set-connected}, the subgraph $G[S]$ must be connected. Thus, $d_S(v) \geq 2$ and we have the following: 
\begin{align*}
    g_x(S - u)& = g_x(S) - x_v&\\
    & = b(S) - x_v&\\
    & = |E[S]| - |S| - x_v&\\
    & = (|E[S-u]| + 1) - (|S - v| + 1) - x_v&\\
    & = b(S - u) - x_v&\\
    &\leq g_x(S - u) - x_v&
\end{align*}
The above chain of inequalities implies that $x_v \leq 0$. By the non-negativity constraints on $x_v$, we have that $x_v = 0$. Thus, the final inequality in the above chain is in fact an equality, and hence, the set $S-u$ is in $\calT$. 

We now make three observations: 
\begin{enumerate}
    \item $\row(S) - \chi_v = \row(S - u)$ (we note that the LHS subtracts the indicator vector of the vertex $v$ while the RHS considers the row vector of the set $S-u$, where $u$ is the vertex with $d_S(u)=1$),
    \item Every $X \in\calC$ such that $X\subset S$ either has $d_X(u) = 1$ or $X$ does not contain $u$, and 
    \item For every $X\subset S$ such that $X\in\calC$, $\row(X) - \chi_v = \row(X - u)$. 
\end{enumerate} 
Here, the first observation is due to $d_S(u) = 1$ and thus $\row(S)_u = 0$. The second observation follows because of monotonicity of the induced-degree function and the fact that tight sets cannot have isolated vertices by \Cref{lem:tight-set-connected}. The third observation follows by writing down the corresponding chain of inequalities as above for each such set $X$. 
Then, by the three observations, we can remove the vertex $u$ from every set in $\calC$ that contains $u$ resulting in a set family $\calC' \not = \calC$ that still satisfies properties (1)-(4). However, we have that $\sum_{S\in\calC'}|S| < \sum_{S\in\calC}|S|$, contradicting our choice of family $\calC$.
%\end{proof}



% \begin{claim}\label{lem:basis-with-tight-set-degree-atleast-2}
% %Let $x$ be an extreme point of $\weakdensitypolyhedron(G)$ such that $x_u < \frac{1}{2}$ for all $u\in V$ and $\calT := \{S\subseteq V : f_x(S) = 0\}$. Then, 
% There exists a family $\calC'\subseteq \calC$ satisfying properties (1)-(4) of \Cref{lem:chain-basis-existence} such that for every $A\in \calC'$, we have that 
% $d_A(u) \geq 2$ for every $u \in A$.
% \end{claim}
% \begin{proof}

% By way of contradiction, let $A\in\calC$ with $u \in A$ such that $d_A(u) < 2$. We note that since $A\in\calT$, $d_A(u)\geq 1$ as tight sets cannot have isolated vertices by \Cref{lem:tight-set-connected}. Thus, $d_A(u) = 1$. Let $uv \in E[A]$ be the unique edge incident to $u$ in $G[A]$. 
% %Then, the edge $uv$ cannot be the unique edge in $E[A]$ 
% We note that $E[A]$ should contain at least one more edge apart from the edge $uv$ as otherwise $\row(A) = 0$. Furthermore, by \Cref{lem:tight-set-connected}, the subgraph $G[A]$ must be connected. Thus, $d_A(v) \geq 2$ and we have the following: 
% \begin{align*}
%     g_x(A - u)& = g_x(A) - x_v&\\
%     & = b(A) - x_v&\\
%     & = |E[A]| - |A| - x_v&\\
%     & = (|E[A-u]| + 1) - (|A - v| + 1) - x_v&\\
%     & = b(A - u) - x_v&\\
%     &\leq g_x(A - u) - x_v&
% \end{align*}
% The above chain of inequalities implies that $x_v \leq 0$. By the non-negativity constraints on $x_v$, we have that $x_v = 0$. Thus, the final inequality in the above chain is in fact an equality, and the set $A-u \in \calT$. 

% We now make three crucial observations: 
% \begin{enumerate}
%     \item $\row(A) - \chi_v = \row(A - u)$ (we note that the LHS subtracts the indicator vector of the vertex $v$ while the RHS considers the row vector of the set $A-u$, where $u$ is the vertex with $d_A(u)=1$); %\knote{$\row(A) - \chi_v = \row(A - u)$?}
%     \item Every $B \in\calC$ such that $B\subset A$ either has $d_B(u) = 1$, or does not contain $u$; and 
%     \item For every $B\subset A$ such that $B\in\calC$, $\row(B) - \chi_v = \row(B - u)$. %\knote{$\row(B) - \chi_v = \row(B - u)$?}
% \end{enumerate} 
% Here, the first observation is due to $d_A(u) = 1$ and thus $\row(A)_u = 0$. The second observation follows because of monotonicity of the induced-degree function and the fact that tight sets cannot have isolated vertices by \Cref{lem:tight-set-connected}. The third observation follows by writing down the corresponding chain of inequalities as above for each such set $B$. 
% Then, by the three observations, we can remove the vertex $u$ from every set in $\calC$ that contains $u$ resulting in a set family $\calC'$ that is still a chain basis for $\calT$. Iteratively applying this procedure leads to a chain basis $\calC$ for $\calT$ such that every $A\in \calC$ does not contain a vertex $u\in A$ with $d_A(u)\le 1$. 
% \end{proof}

\iffalse
We will need an additional property of tight sets in the chain basis that we now show. 
%Throughout this section, let $x \in \R^{V}$ be an arbitrary extreme point of $\weakdensitypolyhedron(G)$, $\calT := \{S\subseteq V : f_x(S) = 0\}$ denote the collection of \emph{tight} sets for $x$, and $\calZ = \{u : x_u = 0\}$. 

\begin{lemma}\label{lem:non-zero-difference-of-tight-sets}
Let $x$ be an extreme point of $\weakdensitypolyhedron(G)$ such that $x_u < \frac{1}{2}$ for all $u\in V$, the family of tight sets for $x$ be $\calT := \{S\subseteq V : f_x(S) = 0\}$, and let $\calZ := \left\{u \in V : x_u = 0\right\}$. 
Let $\calC\subseteq\calT$ be a chain family satisfying the properties in \Cref{lem:chain-basis-existence}. 
%such that $\Rows(\calC))$ form a basis for $\Rows(\calT)$ and $d_A(u)\ge 2$ for every $u\in A\in \calC$. 
Let $A, B \in \calC$ such that $A\subset B$. Then, there exists a vertex $v \in B-A$ such that $x_v > 0$.
\end{lemma}
\begin{proof}
\fi

Finally, we show that $\calC$ also satisfies property (6). 
By way of contradiction, assume that there exists $A, B\in \calC$ such that $A\subset B$ and $x_u = 0$ for each $u \in B - A$. We first show a lower bound on $|\delta(A, B-A)|$.
%\knote{We first show a lower bound on $|\delta(A, B-A)|$}. 
\begin{claim}\label{claim:non-zero-difference-of-tight-sets}
$\frac{|\delta(A, B-A)|}{2} \geq -b(B - A)$.
\end{claim}
\begin{proof}
First, we argue that the subgraph $G[B - A]$ is a forest. By way of contradiction, suppose that the subgraph $G[B-A]$ contains a cycle. Let $C$ be a minimal cycle in the subgraph $G[B-A]$. Then, $g_x(C) = 0$ as $x_u = 0$ for each $u \in B - A$ by our hypothesis. Furthermore, $b(C) = 0$ as $C$ is a cycle. Thus, $C \in \calT$ is a tight set. However, since $C\subseteq B - A$, we have that $C\cap A = \emptyset$. This contradicts the fact that all tight sets must overlap (as shown by \Cref{lem:tight-sets-conditional-uncrossing-properties}).

Let $k$ be the number of disconnected acyclic components of the forest $G[B - A]$. Then, we have the following two observations: %(1) $b(B - A) = (-k)$; and (2)$|\delta(A, B - A)| \geq 2k$.
\begin{enumerate}
    \item $b(B - A) = -k$ and 
    \item $|\delta(A, B - A)| \geq 2k$.
\end{enumerate}
We note that the claim  %\knote{remove `directly'}
follows from above observations. We justify these observations next. 

The first observation follows from the fact that the subgraph $G[B-A]$ is a forest. In particular, we have that $$b(B-A) = \left|E[B-A]\right| - \left|B - A\right| = \left(\left(\left|B-A\right| - 1\right) - (k-1)\right) - |B - A| = -k.$$
We now prove the second observation. Each of the $k$ acyclic components of the subgraph $G[B - A]$ must have at least $2$ leaves. For every such leaf vertex $v$, we have that the induced degree $d_{B-A}(v) = 1$. However, since the set $B\in\calT$ is tight, it must be that $d_{B}(v) \geq 2$. Thus, $|\delta(v, A)| \geq 1$ for each such leaf vertex $v$.
\end{proof}
With the above lower bound on the cut size $|\delta(A, B-A)|$, we get the required contradiction as follows:
\begin{align*}
    g_x(A) + \frac{|\delta(A, B-A)|}{2} \ & > g_x(A) + \sum_{uv\in\delta(A, B-A)}x_u&\\
    & = g_x(A) + g_x(B-A) + \sum_{uv\in\delta(A, B-A)}(x_u + x_v)&\\
    & = g_x(B)&\\
    &= b(B)&\\
    & = |E[B]| - |B|&\\
    &= \left( |E[A]| + |E[B - A]| + |\delta(A, B-A)|\right) - \left(|A| + |B - A|\right)&\\
    &=b(A) + b(B - A) + |\delta(A, B-A)|&\\
    &\geq b(A) + \frac{|\delta(A, B-A)|}{2}&\\
    & = g_x(A) + \frac{|\delta(A, B-A)|}{2}.&
\end{align*}
The first inequality follows from $x_u < 1/2$ for each vertex $u\in V$. The first equality follows from the assumption that $x_u = 0$ for each $u \in B - A$. The penultimate inequality follows from \Cref{claim:non-zero-difference-of-tight-sets}, and the final equality holds since the set $A\in \calT$ is a tight set.
\end{proof}

\subsection{Proof of Theorem \ref{thm:weak-density-extreme-point}}\label{sec:counting-argument}
In this section, we complete the proof of \Cref{thm:weak-density-extreme-point} using the properties proved in previous subsections. 
%every tight set contains at least one new vertex $u$ for which $x_u>0$. 
In the next lemma, we use a stronger hypothesis that $x_u<1/3$ for all $u\in V$ to 
show that there exist at least two vertices $u, v$ which take non-zero values. We emphasize that the lemma does not rely on extreme point properties and holds for every feasible point $x$ satisfying the hypothesis.  
%show that all tights with a few structural properties contain two vertices with non-zero vertex variables. We will subsequently show that there exists a chain basis in which these structural conditions hold for all tight sets in the family. 

\begin{lemma}\label{lem:two-non-zero-vertices-general-statement}
%\knote{Let $G=(V, E)$ be a connected graph with minimum degree at least $2$ such that $G$ contains a $2$-pseudotree. Let $x\in P(G)$ such that $x_u<1/3$ for all $u\in V$. Then, there exist two distinct vertices $u, v\in V$ such that $x_u, x_v>0$.}
%Suppose $x_u < \frac{1}{3}$ for all $u\in V$. Let $S\subseteq V$ be a set such that the subgraph $G[S]$ is connected, has minimum degree at least $2$, and contains a $2$-pseudotree. Then, there exist two vertices $u,v \in S$ such that $x_u, x_v > 0$ and $u\not = v$.
Let $G=(V, E)$ be a connected graph with minimum degree at least $2$ such that $G$ contains a $2$-pseudotree. Let $x\in \weakdensitypolyhedron(G)$ such that $x_u<1/3$ for all $u\in V$. Then, there exist distinct vertices $u, v\in V$ such that $x_u, x_v>0$.
\end{lemma}
\begin{proof}
By way of contradiction, let $S\subseteq V$ be a set such that the subgraph $G[S]$ is connected, has minimum-degree at least $2$, contains a $2$-pseudotree, but 
$S$ contains at most one vertex with a corresponding non-zero vertex variable.
We first consider the case where $S$ contains no vertices with positive vertex variables. Observe that in this case, $g_x(S) = 0$. However, since the subgraph $G[S]$ is connected and contains a $2$-pseudotree, $g_x(S) > 0$, a contradiction.
%Note that if $S$ contains no such vertices, then $g_x(S) = 0$.
%However, $g_x(S)$ must be positive as $S$ contains a $2$-pseudotree by \Cref{claim:2-pseudotree-containment}, a contradiction.
%\shubhang{todo -- perhaps make a proposition out of this.}
%$\row(S) \in \spanfunc(\Rows(\calZ))$, contradicting linear independence of $\calC\cup\calZ$ 
%\knote{if $S$ contains no such vertices, then $S$ cannot contain a $2$-pseudotree; but you have already shown that $S$ should contain a $2$-pseudotree. Avoid relying on linear independence here.}. 
We next consider the alternative case where the set $S$ contains exactly one vertex $v\in S$ with $x_v > 0$. It follows that 
$$(d_S(v)-1)x_v = g_x(S) \geq b(S) = |E[S]| - |S|.$$
Here the first equality is because $x_v$ is the only non-zero variable, while the first inequality is by the weak-density constraint on $S$. We now consider two cases based on the degree of $v$ in $G[S]$.

First, consider the case where $d_S(v) \geq 4$. Then we have the following:
$$x_v = \frac{|E[S]| - |S|}{d_S(v) - 1} \geq \frac{|S| - 1 + \frac{d_S(v)}{2} - |S|}{d_S(v) - 1}= \frac{1}{2}\left(1 - \frac{1}{d_S(v) - 1}\right)\geq \frac{1}{2}\cdot\frac{2}{3} = \frac{1}{3}.$$
Here, the first inequality is due to the hypothesis that $d_S(u) \geq 2$ for each $u \in S$.
%from \Cref{lem:basis-with-tight-set-degree-atleast-2}.

Next, consider the case where $d_S(v) \leq 3$. 
%By \Cref{claim:2-pseudotree-containment}, the subgraph $G[S]$ contains a $2$-psuedotree. Furthermore, by \Cref{lem:tight-set-connected}, the subgraph $G[S]$ is connected. 
Since $G[S]$ is connected and contains a $2$-pseudotree, we have that $b(S) = |E[S]| - |S| \geq 1$ --- this can be observed by starting with the $2$-pseudotree, and then charging the remaining vertices to the edges which connect them to the $2$-pseudotree. This gives us the required contradiction as follows:
$$1 \leq b(S) \leq g_x(S) = (d_S(v)-1)x_v \leq 2x_v < 1.$$
Here, the first inequality holds by the weak-density constraints on $S$, while the the final inequality holds due to our hypothesis that $x_v < \frac{1}{2}$.

%\knote{I think the proof of this lemma does not really need basis or chain family: it only needs that $S$ is connected and contains a $2$-pseudotree and $x$ is feasible on $G[S]$. Instead of $g_x(S) = b(S)$, you need to use $g_x(S) \ge b(S)$. This holds by the LP constraint. I suggest phrasing and proving the more general form and deriving the lemma as a corollary by observing that every $S\in \calC$ is connected and contains a $2$-pseudotree which you have shown before.} \shubhang{(done) It also requires minimum degree to be at least 2 which I also included in the hypothesis.}
\end{proof}

%The next lemma is a corollary to \Cref{lem:two-non-zero-vertices-general-statement}
%We now give a corollary to \Cref{lem:two-non-zero-vertices-general-statement}. 
We use \Cref{lem:two-non-zero-vertices-general-statement} to conclude the following: 
%show that every tight set in a chain basis for an extreme point $x$ with $x_u<1/3$ for all $u\in V$ satisfies the conditions of \Cref{lem:two-non-zero-vertices-general-statement} and thus contains two non-zero elements. In particular, this implies that the innermost set in the chain basis has two non-zero variables.

\begin{corollary}\label{cor:two-non-zero-vertices-in-tight-sets}
Let $G=(V, E)$ be a connected graph with minimum degree at least $2$ such that $G$ contains a $2$-pseudotree. 
%Suppose $x_u < \frac{1}{3}$ for all $u\in V$. 
Let $x$ be an extreme point of $\weakdensitypolyhedron(G)$ such that $x_u < \frac{1}{3}$ for all $u\in V$, the family of tight sets for $x$ be $\calT := \{S\subseteq V : f_x(S) = 0\}$, and let $\calZ:= \left\{u \in V : x_u = 0\right\}$. 
Let $\calC\subseteq\calT$ be a chain family satisfying the properties in \Cref{lem:chain-basis-existence}. 
Then, 
%there exists a chain family $\calC\subseteq\calT$ such that $\Rows(\calC))$ form a basis for $\Rows(\calT)$ and 
for every $S\in \calC$, there exist distinct vertices $u,v \in S$ such that $x_u, x_v > 0$.
\end{corollary}
\begin{proof}
%By \Cref{lem:chain-basis-existence}, there exists a chain basis $\calC$ such that for every tight set $A\in \calC$, we have that $d_A(u)\geq 2$ for each $u\in A$ and the subgraph $G[A]$ contains a $2$-pseudotree. Furthermore, by 
By \Cref{lem:chain-basis-existence}, we have that for every $A\in\calC$, the subgraph $G[A]$ contains a $2$-pseudotree. The claim follows by \Cref{lem:two-non-zero-vertices-general-statement}.
\end{proof}

We now restate and prove Theorem \ref{thm:weak-density-extreme-point}.

\thmWeakDensityExtremePoint*

\iffalse
The next lemma completes the proof of Theorem \ref{thm:weak-density-extreme-point}. \knote{Theorem 1 does not have anything about min-degree being at least $2$?}

\begin{lemma}\label{lem:main}
Let $G=(V, E)$ be a connected graph with minimum degree at least $2$ such that $G$ contains a $2$-pseudotree  
%Suppose $x_u < \frac{1}{3}$ for all $u\in V$. 
and $x$ be an extreme point of $\weakdensitypolyhedron(G)$. 
Then, there exists a vertex $u \in V$ such that $x_u \geq \frac{1}{3}$.
\end{lemma}
\fi
\begin{proof}
By way of contradiction, let $x_u < \frac{1}{3}$ for all $u \in V$. 
%$\calN\calZ:=\{u\in V: x_u>0\}$ %$\calN\calZ = V - \calZ$ 
%denote the set of vertices with non-zero vertex variables. 
Let $\calC =\{C_1, C_2, C_3, \ldots, C_t\}$ be the chain basis guaranteed by \Cref{lem:chain-basis-existence}. 
%\Cref{cor:two-non-zero-vertices-in-tight-sets}. 
We order the sets in the basis so that
 $C_0 \subseteq C_1\subseteq C_2 \subseteq C_3 \subseteq \cdots\subseteq C_t \subseteq C_{t+1}$, where we denote $C_0 := \emptyset$ and $C_{t+1} := V$ for notational convenience. Let $\calN\calZ(C_i) = \{v \in C_i : x_v > 0\}$.
 Let $\calN\calZ := \calN\calZ(V)$ be the set of vertices $v$ with $x_v>0$. Then, $\calN\calZ$ can be partitioned as follows: $\calN\calZ(V) = \uplus_{i \in [t+1]} \calN\calC(C_i) \backslash \calN\calC(C_{i-1})$.
Hence, we have that 
\[
|\calC| < \sum_{i\in [t]} \left|\calN\calZ(C_i) \backslash \calN\calZ(C_{i-1})\right|\leq |\calN\calZ|.
\]
The first inequality holds due to the following two reasons: (i)$|\calN\calZ(C_1) \backslash \calN\calZ(C_0)| = |\calN\calZ(C_1)| \geq 2$ by \Cref{cor:two-non-zero-vertices-in-tight-sets} and (ii) $|\calN\calZ(C_i) \backslash \calN\calZ(C_{i-1})| \geq 1$ for $i \in [2,t]$ by property (6) of \Cref{lem:chain-basis-existence}. The second inequality is because $|\calN\calZ(C_{t+1}) \backslash \calN\calZ(C_{t})| \geq 0$ and our observation that the set of differences of subsequent basis sets in the chain ordering of $\calC$ partitions the set of non-zero variables.
Thus, we have that 
\[
|V| 
%= |\calZ \cup \calC|
= |\calZ|+|\calC|
< |\calZ|+|\calN\calZ|
= |V|,
\]
a contradiction. The first equality is because $x$ is an extreme point and  $\calC$ is such that $\Rows(\calC\cup \calZ)$ is linearly independent and $\spanfunc(\Rows(\calC\cup \calZ))=\spanfunc(\Rows(\calT \cup \calZ))$ by \Cref{lem:chain-basis-existence}. The last equality is because the number of variables is equal to the sum of the number of zero variables and the number of non-zero variables. 

\iffalse
Then, we have that 
$$|\calN\calZ(V)|= \mathtt{rank}(\mathtt{Columns}(\calT))= \mathtt{rank}(\Rows(\calT))= |\calC| < \sum_{i\in [t]} \left|\calN\calC(C_i) \backslash \calN\calC(C_{i-1})\right|\leq |\calN\calZ|,$$
a contradiction since we have strict inequality. 
Here, the first equality is because $x$ is an extreme point, 
the second equality is because the row rank of a matrix is equal to its column rank, and the third equality is because $\calC$ is a basis for $\calT$. The first inequality holds due to the following two reasons: (i)$|\calN\calZ(C_1) \backslash \calN\calZ(C_0)| = |\calN\calZ(C_1)| \geq 2$ by \Cref{cor:two-non-zero-vertices-in-tight-sets}; and (ii) $|\calN\calZ(C_i) \backslash \calN\calZ(C_{i-1})| \geq 1$ for $i \in [2,t]$ by \Cref{lem:non-zero-difference-of-tight-sets}. The final inequality is because $|\calN\calZ(C_{t+1}) \backslash \calN\calZ(C_{t})| \geq 0$ and our observation that the set of differences of subsequent basis sets in the chain ordering of $\calC$ partitions the set of non-zero variables.
\fi
%\knote{Expand out and write the last inequality - use the paragraph above \Cref{lem:two-non-zero-vertices-in-tight-sets}.}\shubhang{(done)}. 
\end{proof}

%\Cref{thm:main:P_ell_extereme_point} follows from \Cref{lem:main}.