\section{Weak Density and Cycle Cover Constraints for FVS}\label{sec:weak-density-cycle-cover}
%\knote{Might be better to phrase this as a section titled `Integrality gap of weak-density and cycle cover constraints for FVS'. The proof that the integrality gap is at most $2$ is via a primal-dual algorithm. }

In this section, we prove \Cref{thm:integrality-gap-wd+cycle-cover}. Let $G=(V, E)$ be the input graph with non-negative vertex costs $c: V\rightarrow \R_{\ge 0}$. By definition, the ILP $\min\{\sum_{u\in V}c_ux_u: x\in \cyclecoverpolyhedron(G)\cap \Z^V\}$ formulates FVS. We note that weak density constraints are valid for FVS. Consequently, (\ref{FVS-IP: WD-and-cycle-cover}) is an ILP formulation for FVS. 
%\Cref{lemma:cycle-cover-poly-sized-lp} in Section \ref{sec:cycle-cover-poly-sized-formulation} shows that there exists a polynomial-sized formulation for $\cyclecoverpolyhedron(G)$. The rest of this section is devoted to showing that the integrality gap of (\ref{FVS-LP:weak-density-cycle-cover}) is at most $2$. 
The additional conclusions of \Cref{thm:integrality-gap-wd+cycle-cover} follow from \Cref{lemma:weak-density-cycle-cover-integrality-gap} shown in Section \ref{sec:integrality-gap-of-weak-density-and-cycle-cover} and \Cref{lemma:cycle-cover-poly-sized-lp} shown in Section \ref{sec:cycle-cover-poly-sized-formulation}. 

\subsection{Integrality gap of weak density and cycle cover constraints}\label{sec:integrality-gap-of-weak-density-and-cycle-cover}
In this section, we show that the integrality gap of (\ref{FVS-LP:weak-density-cycle-cover}) is at most $2$. 
\begin{lemma}\label{lemma:weak-density-cycle-cover-integrality-gap}
    The integrality gap of the LP (\ref{FVS-LP:weak-density-cycle-cover}) is at most $2$. 
\end{lemma}

Let $\calC$ be the set of cycles in the input graph $G$.  Let $b(S):=|E[S]|-|S|$ for all $S\subseteq V$. 
We rewrite (\ref{FVS-LP:weak-density-cycle-cover}) and its dual in \Cref{fig:primal-dual-LPs}. 
%We note that the LP relaxation has integrality gap at least $2$ by observing that that for the graph $G$ being the complete graph with vertex weights $w_u = 1$ for all $u\in V$, the solution $x_u = \frac{1}{2}$ for each $u \in V$ is feasible for the LP with objective value $\frac{n}{2}$, but the optimum integral solution has objective value $n-2$ since a feedback vertex set  must contain at least $n-2$ vertices. %\knote{Discuss a quick instance showing that the gap is at least $2$.} 

%% primal and dual LPs
\begin{figure}[H]
    \centering
    \begin{mdframed}
    \footnotesize
    \begin{tabular}{cc|cc}
    % First column
    $\begin{array}{ll@{}ll}
        & \text{\textbf{Primal}:}  & &\\
        &&&\\ % skip a line
        \text{min}  &\displaystyle\sum\limits_{u \in V} c_{u}x_{u}&   &\\
        &&&\\ % skip a line
        \text{s.t.}& \displaystyle\sum\limits_{u\in S}(d_S(u)-1)x_u \geq b(S), & & \forall S \subseteq V\\
        &\displaystyle\sum\limits_{u\in C}x_{u} \geq 1 &    & \forall C \in \calC\\
        &x_u \geq 0&&\forall u \in V
    \end{array}$ 
    & & & % Second column
    $\begin{array}{ll@{}ll}
        & \text{\textbf{Dual}:} & & \\
        &&&\\ % skip a line
        \text{max}  &\displaystyle\sum\limits_{S\subseteq V} b(S)y_S + \displaystyle\sum\limits_{C\in\calC}z_C &   &\\
        &&&\\ % skip a line
        \text{s.t.}& \displaystyle\sum\limits_{S\subseteq V:u\in S}(d_S(u)-1)y_S + \displaystyle\sum\limits_{C\in \calC:u\in C}z_C \leq w_u & & \forall u \in V\\
        &y_S \geq 0 &    & \forall S \subseteq V\\
        &z_C \geq 0&&\forall C \in \calC
    \end{array}$ 
    \end{tabular}
\end{mdframed}
    \caption{\textbf{WD-CC} Primal and Dual LPs for FVS}
  \label{fig:primal-dual-LPs}
\end{figure}

We will need the definition of \emph{semi-disjoint cycles} from \cite{CHUDAK1998111}:  
%which will be used throughout this section. 
A cycle $C$ in graph G is \emph{semi-disjoint} (with respect to the graph $G$) if there is at most one vertex in $C$ of degree strictly larger than $2$. We call such a vertex as the \emph{pivot} vertex of the semi-disjoint cycle. We note that such cycles can be viewed as ``hanging off'' the graph via their pivots which are cut vertices. 

\subsubsection{Minimal FVS and Weak Density Constraints}
In this section, we show an important property of inclusion-wise \emph{minimal} feedback vertex sets. Let $F$ be a feedback vertex set for a graph $G$. A \emph{witness cycle} for a vertex $f \in F$ is a cycle $C$ of the graph $G$ such that $F\cap C = \{f\}$ i.e.\ $f$ is the only vertex that \emph{witnesses} the cycle $C$. 
%The following proposition states that
We note that 
every vertex of a minimal FVS must have a witness cycle.
In \Cref{lem:minimal-fvs-is-2apx}, we show that for a graph with non-trivial degree and no semi-disjoint cycles, every \emph{minimal} feedback vertex set is essentially a $2$-approximation to the optimal feedback vertex set with respect to an appropriate cost function. We note that our proof of the lemma appears implicitly in \cite{CHUDAK1998111}. However, we include it here for completeness.

\iffalse
\begin{proposition}\label{prop:minimal-fvs-witness-cycle}
Let $F\subseteq V$ be a minimal FVS for graph $G = (V, E)$. Then every vertex of $F$ has a witness cycle.
\end{proposition}
\begin{proof}
By way of contradiction assume false i.e. there exists $f \in F$ such that $f$ has no witness cycle. Then, every cycle in $G$ that contains $f$, also contains a vertex from $F - \{f\}$. Thus, $F - \{f\}$ is also a feasible FVS for $G$, contradicting minimality of $F$.
\end{proof}
\fi

%\shubhang{I got confused by the notation in my notes and that by CGHW. I actually need the claim below and not the one which I previously had. This claim is also not implied by the lemma statements in CGHW, but directly falls out of their proof. I am including a proof here for completeness/our sake.}

\begin{lemma}\label{lem:minimal-fvs-is-2apx}
Let $G$ be a graph with minimum-degree at least $2$ and containing no semi-disjoint cycles. Then, for every minimal feedback vertex set $F$, we have that $$\sum_{v\in F}(d(v) - 1) \leq 2b(V).$$
\end{lemma}
\begin{proof}
We first show a convenient sufficient condition that proves the claimed upper bound. We subsequently prove this sufficient condition by an edge-counting argument. Let $t$ denote the number of connected components in $G - F$. 

%Let $t$ denote the number of connected components in $G - F$. We first show a convenient sufficient condition that proves the claimed upper bound. This condition is a lower bound on the size of the cut $|\delta(F, \overline{F})|$. We then subsequently argue this sufficient condition by a simple \knote{remove `simple' - avoid adjectives} edge-counting argument.
\begin{claim}\label{claim:lower-bound-cut-fvs}
If $|\delta(F, V-F)| \geq |F| + 2t$, then $\sum_{v \in F} (d(v) - 1)\leq 2b(V)$.
\end{claim}
\begin{proof}
We have the following:
\begin{align*}
    2|V| - |F|&\leq |\delta(F, V - F)| + 2(|V| - |F| - t)&\\
    & = |\delta(F, V - F)| + 2|E[V - F]|&\\
    & = \sum_{v \in V - F} d(v)&\\
    & = 2|E| - \sum_{v\in F}d(v).&
\end{align*}
The above chain of inequalities gives us the claim (by rearranging terms). Here, the first inequality holds due to the hypothesis that $|\delta(F, V-F)| \geq |F| + 2t$. The first equality holds because $G-F$ is acyclic. The final equality holds because the sum of all vertex degrees is $2|E|$.
\end{proof}

We now show that a minimal feedback vertex set $F$ has at least $|F|+2t$ edges crossing it.  
%\knote{has at least $|F|+2t$ edges crossing it} achieves the lower bound given by the previous claim. 
For this, we consider the auxiliary bipartite graph $H = (K\cup F, E_H)$ as follows: Each vertex $k \in K$ corresponds to a connected component $C_k$ in $G - F$. For vertices $k \in K, f \in F$, the graph $H$ contains the edge $(k,f)$ if 
%\knote{use of `only if' is confusing - just say, for vertices ..., the graph $H$ contains an edge ... if there exists a ...}
there exists a vertex $v \in C_k$ such that the edge $(v, f) \in E$ exists in the original graph $G$. 
Furthermore, we define the weight of an edge $(k, f) \in E_H$ as $w_H(k,f):=|\delta_G(C_k, f)|$. We note that $|K|=t$.
%Furthermore, we define the weight of an edge $(k, f) \in E_H$ as the number of edges in the original graph cut $|\delta_G(C_k, f)|$. We denote the weight of an edge $(k,f)$ as $w_H(k, f)$. \knote{Furthermore, we define the weight of an edge $(k, f) \in E_H$ as $w_H(k,f):=|\delta_G(C_k, f)|$.} \knote{We note that $|K|=t$.}

%By \Cref{prop:minimal-fvs-witness-cycle}, 
Since $F$ is an inclusionwise minimal feedback vertex set, every vertex $f \in F$ has a witness cycle. We note that there are no edges between the components of $G - F$. Thus, every witness cycle is completely contained in some component $C_k$ for some $k \in K$. In particular, this implies that in the graph $H$, every vertex $f\in F$ has an edge of weight at least $2$ incident on it. For each $f\in F$ pick an arbitrary such edge $e_f \in \delta_H(f)$ and call it $f$'s \emph{primary} edge. Then, reducing the weight of all primary edges by $1$ still leaves all edges in $E_H$ with positive weight, but counts $|F|$ edges in the cut $|\delta(F, V-F)|$ in the original graph $G$. Let $H'$ denote the residual graph after this weight reduction. By \Cref{claim:lower-bound-cut-fvs}, it now suffices to show that the weight of all edges in $H'$ is at least $2|K|$.

For this, we claim that each vertex $k\in K$ must have a cumulative edge-weight of at least $2$ incident on it in the residual graph $H'$. By way of contradiction, suppose that there exists a vertex $k\in K$ with total incident edge-weight at most $1$. We consider two cases. First, suppose that $k$ has no incident edges. We recall that we reduced the weight of only primary edges in our weight-reduction step. Thus, if there was a primary edge incident to $k$, then $k$ would have a weight $1$ edge incident on it in the residual graph, contradicting the case assumption. Thus, there are no edges incident to $C_k$ in $G$, i.e. $C_k$ is a disconnected acyclic component of $G$. In particular, $C_k$ contains a leaf vertex, contradicting the hypothesis that the minimum degree of $G$ is at least $2$.

Next, consider the case where $k$ has a total weight of $1$ incident on it in the residual graph $H'$. Let $f \in F$ be its unique neighbor in the residual graph. Then, $f$ must also be its unique neighbor in $H$. We note that if $C_k$ has only one edge to $f$ in $G$, then $C_k$ must contain a leaf node in $G$ as it is acyclic, a contradiction. Thus, $C_k$ does not contain any leaf nodes, is acyclic, and has exactly $2$ edges to $f$ in $G$. In particular, $C_k \cup \{f\}$ is a semi-disjoint cycle in $G$, contradicting the hypothesis that $G$ has no semi-disjoint cycles. 
%\shubhang{Todo.}
%\knote{If you are restating a lemma from CGHW, then provide a citation. No need for a proof.} 
\end{proof}

\subsubsection{Proof of Integrality Gap}
In this section, we prove \Cref{lemma:weak-density-cycle-cover-integrality-gap}, 
%Theorem \ref{thm:integrality-gap-wd+cycle-cover}, 
i.e., we show that the integrality gap of the LP-relaxation for FVS given in \Cref{fig:primal-dual-LPs} is at most $2$. 
%We prove this by designing a primal-dual algorithm with respect to the LP that achieves an approximation factor of $2$. 
We prove this by constructing a dual feasible solution $(\overline{y}, \overline{z})$ and a FVS $F$ such that the cost of $F$ is at most twice the dual objective value of $(\overline{y}, \overline{z})$. This implies that the integrality gap of the LP is at most $2$ since the LP is a relaxation for FVS. We construct such a pair of solutions via a primal-dual algorithm. 
Our primal-dual algorithm is an extension of the $2$-approximation primal-dual algorithm due to \cite{CHUDAK1998111}. We state our algorithm in \Cref{alg:primal-dual-fvs}. 

%% Primal Dual Algorithm
\begin{algorithm}[H]
\caption{Primal-Dual for FVS}\label{alg:primal-dual}
\label{alg:primal-dual-fvs}
\textbf{Input:} (1) Graph $G=(V, E)$; (2) Vertex weights $w:V\rightarrow\mathbb{R_{+}}$.\\
\textbf{Output:} Feedback vertex set $F \subseteq V$.
\begin{enumerate}
    %\item Consider Primal and Dual LPs as in \Cref{fig:primal-dual-LPs}.
    \item Initialize $(y,z) := 0$, $i := 1$, $F := \emptyset$, and $G_1 = G$.
    \item \textbf{While} the graph $G_i = (V_i, E_i)$ has a cycle:
    \begin{enumerate}
        \item Recursively remove vertices of degree $1$ from $G_i$.
        \item \textbf{If} $G_i$ has a semi-disjoint cycle $C_i$: Define the set $S_i = C_i$ as the semi-disjoint cycle.
        %increase dual variable $z_{C_i}$ until the first dual constraint for some vertex $u_i \in C_i$ becomes tight.
        \item \textbf{Else}: Define the set $S_i = V_i$ as the entire vertex set.
        %Increase dual variable $y_{V_i}$ until the first dual constraint becomes tight for some vertex $u_i \in V_i$.
        \item Increase dual variable $y_{S_i}$ until the first dual constraint becomes tight for some vertex $u_i \in S_i$.
        \item Update $F := F \cup \{u_i\}$\\
        $G_{i+1} := G_i - u_i$\\
        $i := i+1$.
    \end{enumerate}
    \item Perform reverse-delete on $F$.
    %Let $\ell$ denote number of while loop iterations. Perform reverse-delete on $F_{\ell}$ to get $F$
    \item Return $F$.
\end{enumerate}
\end{algorithm}

\Cref{alg:primal-dual-fvs} is a primal-dual algorithm based on the LP relaxation for FVS given in \Cref{fig:primal-dual-LPs}. 
The algorithm initializes with the dual feasible solution $(\overline{y}=0, \overline{z}=0)$. In the $i$th iteration of the while-loop, the algorithm selects a specific set $S_i$ of vertices (which is either a cycle $C_i$ or entire vertex set $V_i$) and increases the corresponding dual variable until a dual constraint for some vertex $v_i \in S_i$ becomes tight. It then includes the vertex $v_i$ into the FVS and removes $v_i$ from the graph. Adding $v_i$ to the candidate set $F$ is interpreted as setting the primal variable $x_{v_i} = 1$. 
Thus, \Cref{alg:primal-dual-fvs} always maintains dual feasibility and primal complementary slackness (i.e., $x_{v_i}=1$ only if the dual constraint for $v_i$ is tight). 

We now explain the reverse-delete procedure mentioned in \Cref{alg:primal-dual-fvs}. 
%The reverse-delete step then ensures that the final returned solution $F$ is a minimal FVS for $G$. 
Let $\ell$ be the number of iteration of the while loop. 
The reverse-delete procedure iteratively considers vertices in the reverse-order in which they were added into $F$, i.e. $v_{\ell}, v_{\ell-1}, \ldots v_1$. For every vertex in this order, the procedure checks whether removal of this vertex from $F$ is still a feasible FVS for $G$. If so, then the vertex is removed from $F$. We will later rely on reverse-delete to show that $F\cap S_i$ is a minimal FVS is $G[S_i]$ for every iteration $i$. 

We now observe that the algorithm terminates in polynomial time to return a feasible FVS. 

%\knote{Before going into the approximation analysis, need to state that the algorithm terminates. I.e., each iteration of the while loop indeed makes progress.} \shubhang{(done)}

% For a graph $G = (V, E)$ with vertex weights $w$, let $F$ be the solution returned by \Cref{alg:primal-dual-fvs}. The following lemma shows that \Cref{alg:primal-dual-fvs} outputs a feasible vertex cover and directly follows from the terminating condition of the while-loop in \Cref{alg:primal-dual-fvs}.

\begin{lemma}[Feasibility and Runtime]\label{lem:primal-dual-fvs-feasability}
\Cref{alg:primal-dual-fvs} returns a feasible FVS for the input graph $G$ in polynomial time.
\end{lemma}
\begin{proof}
Each step of the while loop and reverse-delete procedure can be implemented to run in polynomial time. 
In $i$th iteration of the while-loop, at least one vertex of the graph $G_i$ is removed. The empty graph has no cycles and so the while-loop will terminate in at most linear number of iterations. Furthermore, the reverse delete step considers every vertex in $F$ exactly once, and so, it also terminates in at most linear number of steps. The output set $F$ is a feasible FVS by the terminating condition of the while-loop and the fact that reverse-delete deletes a vertex only if its deletion maintains feasibility. 
\end{proof}


We now bound the approximation factor of the solution $F$ returned by \Cref{alg:primal-dual-fvs}. Let $(\overline{y}, \overline{z})$ be the dual solution constructed by \Cref{alg:primal-dual-fvs} and let $\chi^F\in \{0,1\}^V$ be the indicator vector of $F$. We note that \Cref{lem:primal-dual-fvs-feasability} implies that $\chi^F$ is a feasible solution to the primal LP. Let $\mathtt{primal}(\chi^F), \mathtt{dual}(\overline{y}, \overline{z})$ denote the objective values of the primal and dual LPs for solutions $\chi^F$ and $(\overline{y}, \overline{z})$ respectively.

%Let $\ell$ denote the number of while-loop iterations performed by \Cref{alg:primal-dual-fvs}. 
Let $I_1\subseteq [\ell]$ denote the set of iterations in which the dual variable for the entire residual graph was increased (i.e. statement (b) of \Cref{alg:primal-dual-fvs} is executed) and let  $I_2\subseteq [\ell]$ denote the set of iterations in which the dual variable for a semi-disjoint cycle in the residual graph was increased (i.e. statement (c) of \Cref{alg:primal-dual-fvs} is executed).
%Let $S_1, S_2, \ldots S_{\ell}$ be the sets chosen by \Cref{alg:primal-dual-fvs} --- we note that each $S_i$ is either a semi-disjoint cycle $C_i$ or the entire vertex set $V_i$ at the ith iteration. 
For notational convenience, we let $F_j = F \cap \{v_j\}$ and $F_{\geq j} = \cup_{k \in[j,\ell]}F_j$.
The next lemma shows that the set $S_i\cap F_{\geq i}$ is  a minimal feedback vertex set for the subgraph $G[S_i]$ for each $i \in [\ell]$. 
%induced by each of the sets considered during the execution of \Cref{alg:primal-dual-fvs}.



\begin{lemma}\label{lem:F-minimal-for-each-set-of-algorithm}
The set $S_i \cap F_{\geq i}$ is a minimal feedback vertex set for the subgraph $G[S_i]$ for each $i\in [\ell]$.
\end{lemma}
\begin{proof}
We prove by induction on $\ell-i$. 
For the base case, we consider $i = \ell$. We observe that $F_{\geq \ell} = \{v_{\ell}\}$ as otherwise, the following two observations result in a contradiction: (i) The reverse-delete step only removes $v_\ell$ from $F$ if the set $F - \{v_{\ell}\}$ is a feasible feedback vertex set for $G$; and (ii) if $F - \{v_{\ell}\}$ were a FVS for $G$, then the algorithm would have terminated after the $(\ell - 1)^{th}$ iteration. We note that $F_\ell$ is a FVS for $G_\ell$ as there was no $(\ell+1)^{th}$ iteration. If the set $S_\ell = V_\ell$ is the entire vertex set of the graph $G_\ell$, then $v_\ell \in V_\ell$.
%and is a minimal FVS for $G_\ell = G[V_\ell]$. 
Alternatively, if the set $S_\ell$ is a semi-disjoint cycle $C_\ell$, then $v_\ell \in C_\ell$, as otherwise the cycle $C_\ell$ would remain in $G_{\ell+1}$ contradicting that there were only $\ell$ iterations. In both scenarios, the set $S_{\ell}\cap F_{\geq\ell}$ is a minimal FVS for $G[S_\ell]$.

For the inductive case, assume that $0 \leq i < \ell$. By the inductive hypothesis, we have that the set $S_j\cap F_{\ge j}$ is a minimal FVS for the graph $G[S_j]$ for all $\ell \geq j > i$. We consider two cases based on whether the set $S_i$ is a semi-disjoint cycle of $G_i$ or the entire vertex set of $G_i$.

%\knote{I think all $j$ should be replaced with $i$ below.} 
First, we consider the case where $S_i$ is a semi-disjoint cycle $C_i$. We need to show that $|F\cap C_i|=1$. 
We observe that $1\leq |F_{\geq i}| \leq 2$: The lower bound follows from the fact that the set $F_{\geq i}$ is a feasible FVS for $G_i$ and thus intersects all cycles in at least one vertex. The upper bound holds by the following three observations: (i) The vertex $v_i$ is the first vertex of the set $C_i$ to be selected into $F$ by the algorithm; (ii) If the vertex $v_i$ is the pivot vertex, then all vertices in $C_i - \{v_i\}$ get recursively removed and thus cannot be selected into $F$ in any subsequent iteration after $i$; and (iii) If the vertex $v_i$ is not the pivot vertex, then all vertices in $C_i - \{v_i\}$ except the pivot vertex get recursively removed due to step (a) and thus, only the pivot vertex can be included into $F$ in some subsequent iteration after $i$. We note that if scenario (ii) happens, then $F_{\geq i}\cap C_i = \{v_i\}$ and hence, $|F\cap C_i|=1$. Alternatively, suppose that scenario (iii) happens. If the pivot is never selected into $F$, then once again we have $F_{\geq i}\cap C_i = \{v_i\}$ and hence, $|F\cap C_i|=1$. Otherwise the pivot is selected into $F$ and the reverse-delete step processes the pivot vertex before processing $v_i$. If the reverse-delete step removes the pivot vertex from $F$, then once again we have $F_{\geq i}\cap C_i = \{v_i\}$ and hence, $|F\cap C_i|=1$. Otherwise, the reverse-delete step does not remove the pivot from $F$. Consider the iteration of reverse-delete that processes $v_i$---we need to show that reverse-delete will indeed remove $v_i$ in this iteration. We observe that the set $\{v_k: k\in [1, i-1]\}$ has not been processed by reverse-delete and this set intersects every cycle that is not entirely contained in $G_i$ (this is true since $G_i$ is the residual graph after removal of $\{v_1, \ldots v_{i-1}\}$ along with recursively removing degree-1 vertices in each step). Since the pivot for $C_i$ is already in $F$ and the cycle $C_i$ is semi-disjoint in $G_i$, the reverse-delete step must remove the vertex $v_i$ from $F$ as $v_i$ intersects no cycles other than those already intersected by $F - \{v_i\}$. In particular, $F_{\geq i}\cap C_i $ is a singleton set consisting of the pivot vertex for $C_i$. Thus, in all cases we have that $|F_{\geq i}\cap C_i| = 1$. It follows that $F_i$ is a minimal FVS for the cycle $G[C_i]$.

Next, we consider the case when $S_i = V_i$ is the entire vertex set of the subgraph $G_i$. 
Here, we consider the set of all cycles that $v_i$ intersects in $G$ and partition them into two types: (1) Cycles completely contained in $G_i$; and (2) cycles that include at least one vertex from $V-V_i$. 
Since $F_{\geq i}$ is a feasible FVS for $G_i$, we have that $F_{\geq i}$ intersects all cycles of the first type.
Furthermore, since the vertices in $V- V_i$ do not exist in $G_i$, the set $\{v_1,\ldots , v_{i-1}\}$ intersect all cycles of the second type. We recall that the set $F_{\geq i+1}$ is a minimal FVS for $G_{i+1}$. 

We consider the first subcase where the  set $F_{\geq i+1}$ is an FVS for the graph $G_i$. Then, the set $F_{\geq i+1}$ intersects all cycles contained in $G_{i}$. In particular, it intersects all cycles of type (1). Thus, the set $\{v_1,\ldots, v_{i-1}\} \cup F_{\geq i+1}$ is a feasible FVS for the original graph $G$. Due to this, the reverse-delete step removes the vertex $v_i$ from $F$ and we have that $F_{\geq i} = F_{\geq i+1}$. Furthermore, by the inductive hypothesis we have that $F_{\geq i+1}$ is a minimal FVS for $G_{i+1}$. Thus, the set $F_i$ is a minimal FVS for $G_i = G[S_i]$.

We next consider the remaining subcase where the  set $F_{\geq i+1}$ is not a FVS for the graph $G_i$. In particular, there must be a cycle of the first type that $v_i$ intersects, but none of the vertices of $F_{\geq i+1}$ intersect. In particular, the set $\{v_1,\ldots, v_{i-1}\} \cup F_{\geq i+1}$ is a not feasible FVS for the original graph $G$ as the set does not intersect all type (1) cycles. Thus, the reverse-delete step cannot remove $v_i$ from $F$. By the induction hypothesis, we have that $F_{\geq i+1}$ is a minimal FVS for $G_{\geq i+1}$. Consequently, none of these vertices can be removed from $F_{\geq i}$ as the resulting set would not be a feasible FVS. It follows that $F_{\geq i}$ is a minimal FVS for $G_i = G[S_i]$.
\end{proof}


We show in \Cref{lemma:WD-CC-LP-integrality-gap} that the solution $F$ constructed by the algorithm has cost at most twice the objective value of the dual feasible solution $(\overline{y}, \overline{z})$ constructed by the algorithm. We recall that $\chi^F\in \{0,1\}^V$ is the indicator vector of the set $F$ and $\mathtt{primal}(\chi^F), \mathtt{dual}(\overline{y}, \overline{z})$ denote the objective values of the primal and dual LPs for solutions $\chi^F$ and $(\overline{y}, \overline{z})$ respectively. 

\begin{lemma}[Integrality gap bound]\label{lemma:WD-CC-LP-integrality-gap}
%\Cref{alg:primal-dual-fvs} is a $2$-approximation algorithm for \textsc{Feedback Vertex Set}.
%Let $\overline{x}$ and $(\overline{y}, \overline{z})$ be primal and dual optimum solutions for 
We have that 
\[
\mathtt{primal}(\chi^F) 
%\sum_{u\in F}c_u
\le  2\cdot\mathtt{dual} (\overline{y}, \overline{z}).
\]
\end{lemma}
\begin{proof} We have that 
\begin{align*}
    \mathtt{primal}(\chi^F)& = \sum_{v\in F} c_v&\\
    & = \sum_{v\in F}\left(\sum_{S\subseteq V:v\in S}(d_S(v) - 1)\overline{y}_S + \sum_{C\in \calC:v \in C}\overline{z}_C\right)&\\
    &= \sum_{S\subseteq V}\overline{y}_S\sum_{v \in F\cap S}(d_S(v) - 1) + \sum_{C\in\calC}\overline{z}_C\sum_{v \in F\cap C}1&\\
    &= \sum_{i\in I_1}\overline{y}_{S_i}\sum_{v \in F\cap S_i} (d_{S_i}(v) - 1) + \sum_{i \in I_2}\overline{z}_{C_i}|F \cap C_i|&\\
    &\leq 2\sum_{i \in I_1}\overline{y}_{S_i}b(S_i) + \sum_{i \in I_2}\overline{z}_{C_i} &\\
    &\leq 2\cdot\mathtt{dual} (\overline{y}, \overline{z}).&
\end{align*}
Here, the second equality follows by the primal complementary slackness maintained by \Cref{alg:primal-dual-fvs}. The fourth equality follows by the fact that all dual variables that were not incremented during some iteration of \Cref{alg:primal-dual-fvs} have value zero. The first inequality follows by \Cref{lem:F-minimal-for-each-set-of-algorithm} and \Cref{lem:minimal-fvs-is-2apx}.
\end{proof}

\Cref{lemma:WD-CC-LP-integrality-gap} completes the proof of \Cref{lemma:weak-density-cycle-cover-integrality-gap}
%\Cref{thm:integrality-gap-wd+cycle-cover} 
since $\mathtt{dual} (\overline{y}, \overline{z})\le \mathtt{primal}(\chi^F)$ by weak duality and the fact that the ILP $\min\{\sum_{u\in V}c_u x_u: x\in \weakdensitypolyhedron(G)\cap \cyclecoverpolyhedron(G)\cap \Z^V\}$ is an ILP formulation for FVS (whose LP-relaxation is given in \Cref{fig:primal-dual-LPs}). It also proves that the approximation factor of the solution $F$ returned by \Cref{alg:primal-dual-fvs} is at most $2$. 

\subsection{Polynomial-sized formulation of $\cyclecoverpolyhedron(G)$}
\label{sec:cycle-cover-poly-sized-formulation}
In this section, we show that $\cyclecoverpolyhedron(G)$ can be expressed using polynomial number of variables and constraints. This is folklore, but we include its proof for the sake of completeness.

\begin{lemma}\label{lemma:cycle-cover-poly-sized-lp}
Let $G=(V, E)$ and $E':=\{e\in E:\ \exists \text{ cycle in $G$ containing } e\}$. The cycle cover polyhedron $\cyclecoverpolyhedron(G)$ is the projection of the following polyhedron to $x$ variables:   
\begin{equation}\label{eqn:distance-cycle-cover}
    \distancebasedcyclecoverpolyhedron(G) := \left\{ (x,d)\in \R^V_{\ge 0} \times \R^{V\times E'}_{\ge 0}:\ \begin{array}{l}
{d_s^e = 0} \hfill {\qquad \forall e=st\in E'} \\
 {d_t^e + x_s \ge 1} \hfill \qquad \forall e=st\in E' \\
 {d_a^e + x_b \ge d_b^e }\hfill \qquad \forall ab\in E,\ e\in E'
  \end{array}\right\}, 
\end{equation}
Consequently, $\cyclecoverpolyhedron(G)$ admits a polynomial-sized description. 
\end{lemma}
\begin{proof}
Let $\projecteddistancebasedcyclecoverpolyhedron(G)$ be the projection of $\distancebasedcyclecoverpolyhedron(G)$ to the $x$ variables. We prove that $\projecteddistancebasedcyclecoverpolyhedron(G)=\cyclecoverpolyhedron(G)$. 

We first show that $\projecteddistancebasedcyclecoverpolyhedron(G)\subseteq \cyclecoverpolyhedron(G)$. Let $x\in \projecteddistancebasedcyclecoverpolyhedron(G)$. Let $d\in \R^{V\times E'}_{\ge 0}$ be a vector such that $(x,d)\in \distancebasedcyclecoverpolyhedron(G)$. It suffices to show that for every cycle $C$ in $G$, we have that $\sum_{u\in V(C)}x_u\ge 1$. Let $C$ be a cycle in $G$. Fix $e=st\in E(C)$. Then, $e\in E'$ and hence, $d^e_s=0$, and $d^e_t + x_s\ge 1$. Let $a_1=s, a_2, a_3, \ldots, a_{r-1}, a_r=t$ be the vertices of the cycle in that order. Then, we have the constraint $d^e_{a_i}+x_{a_{i+1}}\ge d^e_{a_{i+1}}$ for every $i\in [r-1]$. Adding these constraints gives us that $\sum_{i=2}^r x_{a_i}\ge d^e_{a_r}=d^e_t\ge 1-x_{s}=1-x_{a_1}$. Thus, the constraint $\sum_{i=1}^r x_{a_i}\ge 1$ holds. 

Next, we show that $\projecteddistancebasedcyclecoverpolyhedron(G)\supseteq \cyclecoverpolyhedron(G)$. Let $x\in \cyclecoverpolyhedron(G)$. Let $d\in \R^{V\times E'}_{\ge 0}$ be defined as follows: for each $e=st\in E'$, let 
\[
d^e_u:=\min\left\{\sum_{v\in P-\{s\}}x_v:\ P \text{ is a path from $s$ to $u$ and $P\neq \{s, t\}$}\right\}.
\]
We show that $(x, d)\in \distancebasedcyclecoverpolyhedron(G)$. The vector $(x,d)$ is non-negative by definition. Let $e=st\in E'$. Then,  $d^e_s=0$ by definition. 
Moreover, $d^e_t+x_s\ge 1$ due to the following reasoning: let $P$ be a path from $s$ to $t$ such that $d^e_t=\sum_{v\in P-\{s\}}x_v$ and $P\neq \{s, t\}$. Then, $P$ concatenated with the edge $st$ forms a cycle $C$ and $d^e_t + x_s = x_s+\sum_{v\in P-\{s\}}x_v = \sum_{u\in V(C)}x_u \ge 1$. 
Finally, $d^e_a + x_b \ge d^e_b$ for every edge $ab\in E$ due to the following reasoning: for an edge $ab\in E$, let $P$ be a path from $s$ to $a$ such that $d^e_a=\sum_{v\in P-\{s\}}x_v$. Then, $P$ concatenated with $b$ is a path from $s$ to $b$ and consequently, $d^e_b\le \sum_{v\in P+\{b\}-\{s\}}x_v$ by definition.  
\end{proof}
