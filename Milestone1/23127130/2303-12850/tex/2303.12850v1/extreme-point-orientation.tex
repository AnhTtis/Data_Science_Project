\section{Extreme Point Property of Orientation Polyhedron}\label{sec:orientation-polytope-extreme-point}
In this section, we prove \Cref{thm:orientation-polyhedron-extreme-point}. 
We recall that an extreme point of $\orientationpolyhedron(G)$ is said to be {\em minimal} if we cannot lower any single variable, keeping the others unchanged, while maintaining feasibility. %If we assume that $c(v) > 0$ for all vertices $v$, every optimal solution to \eqref{eq:pseudoforest} is automatically minimal with respect to the $x$-variables. To obtain an extreme optimal solution that is also minimal with respect to the $y$-variables, we minimize the sum of all $y$-variables, $\sum_{v \in V(G)} \sum_{e \in \delta(v)} y(e,v)$, within the optimal face.
Before proceeding with the proof of \Cref{thm:orientation-polyhedron-extreme-point}, we establish a few lemmas. In the lemmas below, we always assume that $G$ is not a pseudoforest, $(\bar{x},\bar{y})$ is a minimal extreme point of $\orientationpolyhedron(G)$ and, aiming toward a contradiction, $\bar{x}(v) < 1/3$ for all vertices $v \in V(G)$.

For a graph $G$, we denote its edge-vertex incidence graph by $H$. Thus, $V(H) = V(G) \cup E(G)$ and $E(H) = \{ev : e \in E(G),\ v \in V(G),\ e \in \delta(v)\}$. We note that $|V(H)| = |V(G)| + |E(G)|$ and $|E(H)| = 2 |E(G)|$. The \emph{support graph} of $\bar{y}$ is the subgraph of the incidence graph $H$ whose vertices are all the vertices of the incidence graph $H$ and whose edges are those for which $\bar{y}_{e,v} > 0$. Denoting this subgraph by $H(\bar{y})$, we have $V(H(\bar{y})) = V(H) = V(G) \cup E(G)$ and $E(H(\bar{y})) = \{ev \in E(H) : \bar{y}_{e,v} > 0\}$.

\begin{lemma}
\label{lem:acyclic}
The support graph $H(\bar{y})$ is a forest.
\end{lemma}


\begin{proof}
Towards a contradiction, suppose that there exists a cycle in $H(\bar{y})$. Since $H(\bar{y})$ is bipartite, the edge set of this cycle can be partitioned into two matchings $M_1$ and $M_2$. Given some $\varepsilon > 0$, we define two points $\bar{y}^{\pm} \in \mathbb{R}^{E(H)}$ by letting $\bar{y}^{\pm} := \bar{y} \pm \varepsilon \chi^{M_1} \mp \varepsilon \chi^{M_2}$. If $\varepsilon$ is sufficiently small, both $(\bar{x},\bar{y}^+)$ and $(\bar{x},\bar{y}^-)$ are feasible (that is, belong to the orientation polytope), which contradicts the fact that $(\bar{x},\bar{y}) = \frac{1}{2} (\bar{x},\bar{y}^+) + \frac{1}{2} (\bar{x},\bar{y}^-)$ is extreme.
\end{proof}

\begin{lemma}
\label{lem:num_components}
The number of (connected) components of $H(\bar{y})$ is $|V(G)|-|E(G)|+m_0$, where $m_0$ denotes the number of edges $ev \in E(H)$ such that $\bar{y}_{e,v} = 0$.
\end{lemma}


\begin{proof}
We have that $|V(H(\bar{y}))|=|V(G)|+|E(G)|$ and $|E(H(\bar{y}))|=2|E(G)|-m_0$ by definition.  
By Lemma~\ref{lem:acyclic}, the number of components of $H(\bar{y})$ is 
%
\begin{align*}
|V(H(\bar{y}))| - |E(H(\bar{y}))| 
  &= (|V(G)| + |E(G)|) - (2 |E(G)| - m_0)\\
  &= |V(G)| - |E(G)| + m_0\,.
\end{align*}
\end{proof}

\begin{lemma}
\label{lem:edges}
Every edge $e = vw \in E(G)$, we have $\bar{y}_{e,v} > 0$ or $\bar{y}_{e,w} > 0$ (or both) and $\bar{x}_v + \bar{x}_w + \bar{y}_{e,v} + \bar{y}_{e,w} = 1$.
\end{lemma}


\begin{proof}
If both $\bar{y}_{e,v} = 0$ and $\bar{y}_{e,w} = 0$ then, since $(\bar{x},\bar{y})$ is feasible, we have $\bar{x}_v + \bar{x}_w \geq 1$. Hence, $\bar{x}_v \geq 1/2$ or $\bar{x}_w \geq 1/2$, a contradiction.

For the second part, toward a contradiction, suppose that $\bar{x}_v + \bar{x}_w + \bar{y}_{e,v} + \bar{y}_{e,w} > 1$. We may assume (by symmetry) that $\bar{y}_{e,v} > 0$. Then, slightly decreasing $\bar{y}_{e,v}$ preserves the feasibility of $(\bar{x},\bar{y})$. This contradicts the minimality of $(\bar{x},\bar{y})$.
\end{proof}

Now consider a component $T$ of the support graph $H(\bar{y})$. By Lemma~\ref{lem:acyclic}, the component $T$ is a tree. We define the \emph{defect} of $T$ as the number vertices $v \in V(T) \cap V(G)$ such that $\bar{x}_v > 0$, plus the number of vertices $v \in V(T) \cap V(G)$ such that $\bar{x}_v + \sum_{e \in \delta(v)} \bar{y}_{e,v} < 1$. Below, we denote this quantity by $\mathrm{defect}(T)$.

We say that component $T$ is \emph{tight} if $\bar{x}_v + \sum_{e \in \delta(v)} \bar{y}_{e,v} = 1$ for all vertices $v \in V(T) \cap V(G)$. Tight components play an important role in our analysis. We seek a tight component with extra properties, dubbed `interesting' (Lemma~\ref{lem:exists_interesting_comp} below states that such tight components exist).

\begin{lemma}
\label{lem:tight}
Let $T$ be a tight component of $H(\bar{y})$. Suppose that some vertex $v \in V(G)$ is a leaf of $T$. Then $T$ has exactly two vertices and $\bar{x}_w = 0$ for the other vertex $w \in V(G)$ that is incident to the unique edge of $G$ in $T$.
\end{lemma}


\begin{proof}
Since $T$ is tight, we have $\bar{x}_v + \sum_{e \in \delta(v)} \bar{y}_{e,v} = 1$. If $\bar{y}_{e,v} = 0$ for all $e \in \delta(v)$, we get $\bar{x}_v = 1$, a contradiction. Hence, $T$ has at least two vertices and we have $\bar{y}_{e,v} = 0$ for all $e \in \delta(v)$ except for precisely one edge, say $f = vw$. From $\bar{x}_v + \bar{y}_{f,v} = 1$ and $\bar{x}_v + \bar{x}_w + \bar{y}_{f,v} + \bar{y}_{f,w}$ (see Lemma~\ref{lem:edges}), we get $\bar{x}_w + \bar{y}_{f,w} = 0$, which implies $\bar{x}_w = 0$ and $\bar{y}_{f,w} = 0$. The result follows.
\end{proof}


We will call a tight component $T$ with exactly two vertices as a \emph{dyad}. Let $T$ be a dyad, say, with $V(T) = \{v,f\}$ where $v \in V(G)$ and $f = vw \in E(G)$. We note that $x_v + \sum_{e \in \delta(v)} y_{e,v} = 1$ follows from the tight constraints $x_v + x_w + y_{f,v} + y_{f,w} = 1$, $x_w = 0$, $y_{e,v} = 0$ for all $e \in \delta(v) \setminus \{f\}$ and $y_{f,w} = 0$. %Note that there are two types of dyads: those with defect $0$ and those with defect $1$. 

\begin{lemma} \label{lem:avg_defect_nonsimple}
Let $T_1$, \ldots, $T_k$ denote the components of $H(\bar{y})$ and let $d$ denote the number of dyads. We have
%
\begin{equation}
\label{eq:counting}
\sum_{i=1}^{k} \mathrm{defect}(T_i) \leq k-d\,.
\end{equation}
\end{lemma}


\begin{proof}
The number of variables defining 
%ambient dimension 
the orientation polytope is $|V(G)| + |E(H)| = |V(G)| + 2 |E(G)|$. By Lemma~\ref{lem:edges}, we can write the total number of constraints that are tight at $(\bar{x},\bar{y})$ as $|E(G)| + n_\mathrm{tight} + n_0 + m_0$, where $n_\mathrm{tight}$ denotes the number of vertices $v \in V(G)$ such that $\bar{x}_v + \sum_{e \in \delta(v)} \bar{y}_{e,v} = 1$, $n_0$ denotes the number of vertices $v \in V(G)$ such that $\bar{x}_v = 0$, and $m_0$ (as above) denotes the number of edges $ev \in E(H)$ such that $\bar{y}_{e,v} = 0$. For every dyad $T$, the tight constraint $\bar{x}_v + \sum_{e \in \delta(v)} \bar{y}_{e,v} = 1$ follows from the other tight constraints, where $v$ is the unique vertex of $G$ in $T$. Since $(\bar{x},\bar{y})$ is an extreme point, we have that the number of variables is at most the number of tight constraints. Hence, 
%
\begin{align*}
|V(G)| + 2 |E(G)| &\leq |E(G)| + n_\mathrm{tight} + n_0 + m_0 - d.
\end{align*}
Rewriting the above gives
\begin{align*}
|V(G)| - n_\mathrm{tight} + |V(G)| - n_0 &\leq |V(G)| - |E(G)| + m_0 - d\,.
\end{align*}
%
Inequality \eqref{eq:counting} follows from this and Lemma~\ref{lem:num_components}.
\end{proof}

We call a component $T$ of $H(\bar{y})$ \emph{interesting} if it is tight, has defect at most $1$ and is not a dyad. By Lemma~\ref{lem:tight}, interesting components $T$ have the following properties: (i) $T$ has at least three vertices, (ii) every leaf of $T$ is an edge of $G$, and (iii) $\bar{x}(v) = 0$ for all vertices $v \in V(T) \cap V(G)$ except at most one.

\begin{lemma} \label{lem:exists_interesting_comp}
At least one component of $H(\bar{y})$ is interesting.
\end{lemma}


\begin{proof}
First, we observe that if $\bar{x}_v = 0$ for all vertices $v \in V(G)$, then $G$ is a pseudoforest, which contradicts our hypothesis. Hence, there exists a vertex $r \in V(G)$ with $\bar{x}_r > 0$. 

Let $T_1$, \ldots, $T_k$ denote the components of $H(\bar{y})$. %We apply Lemma~\ref{lem:avg_defect_nonsimple}. 
If some $T_i$ which is not a dyad has $\mathrm{defect}(T_i) = 0$, then $T_i$ is interesting. Hence, we may assume that $\mathrm{defect}(T_i) \geq 1$ for all components that are not dyads. %By~\eqref{eq:counting}, 
By \Cref{lem:avg_defect_nonsimple}, 
this implies $\mathrm{defect}(T_i) = 0$ for all dyads and $\mathrm{defect}(T_i) = 1$ for all the other components. Then, the unique component containing $r$ is interesting.
\end{proof}

We now restate and prove \Cref{thm:orientation-polyhedron-extreme-point}.
\thmOrientationPolyhedronExtremePoint*
\begin{proof}
%[Proof of \Cref{thm:orientation-polyhedron-extreme-point}]
Aiming toward a contradiction, suppose that $\bar{x}_v < 1/3$ for all vertices $v \in V(G)$. Hence, each one of the above lemmas apply. 
Let $T$ denote an interesting component of $H(\bar{y})$, which exists by Lemma~\ref{lem:exists_interesting_comp}. Since it is interesting, $T$ has at least two vertices and each leaf of $T$ is an edge of $G$. 

Let $r \in V(T) \cap V(G)$ denote any vertex such that $\bar{x}_v = 0$ for all vertices $v \in V(T) \cap V(G)$ distinct from $r$. We call $r$ the \emph{root} of $T$. For $w \in V(G)$, let $d_T(w)$ denote the number of edges $e \in V(T) \cap E(G)$ that are incident to $w$. (This is a slight abuse of notation since $d_T(w)$ is also defined for vertices $w \in V(G)$ outside of $T$.) We let $N(T)$ denote the set of vertices $w \in V(G)$ such that $d_T(w) > 0$ and $w \notin V(T)$. Finally, we let $\mathcal{L}(T) \subseteq E(G)$ denote the leaf set of $T$. 

By Lemma~\ref{lem:edges} and since $T$ is an interesting component of the support graph $H(\bar{y})$, we have
%
\begin{align*}
&|V(T) \cap E(G)| - |V(T) \cap V(G)|\\
&= \sum_{e = vw \in V(T) \cap E(G)} \left(\bar{x}_v + \bar{x}_w + \bar{y}_{e,v} + \bar{y}_{e,w}\right) 
- \sum_{v \in V(T) \cap V(G)} \left(\bar{x}_v + \sum_{e \in \delta(v)} \bar{y}_{e,v}\right)\\
&= (d_T(r) - 1) \bar{x}_r + \sum_{w \in N(T)} d_T(w) \bar{x}_w\,.
\end{align*}
%
We notice that $|V(T) \cap E(G)| - |V(T) \cap V(G)| = |\mathcal{L}(T)| - 1$. Hence, from the equations above, we get
%
\begin{equation}
\label{eq:component}
(d_T(r) - 1) \bar{x}_r + \sum_{w \in N(T)} d_T(w) \bar{x}_w = |\mathcal{L}(T)|-1\,.
\end{equation}

Let $M := (d_T(r)-1) + \sum_{w \in N(T)} d_T(w) = (d_T(r)-1) + |\mathcal{L}(T)|$ denote the sum of the coefficients in the left-hand side of \eqref{eq:component}. We note that $d_T(r) \leq |\mathcal{L}(T)|$. This implies $M \leq 2|\mathcal{L}(T)| - 1$. Thus,
$$
\frac{1}{M} \left( (d_T(r) - 1) \bar{x}_r + \sum_{w \in N(T)} d_T(w) \bar{x}_w\right) \geq \frac{|\mathcal{L}(T)|-1}{2|\mathcal{L}(T)|-1} \geq \frac{1}{3}\,.
$$
This is a contradiction since the left-hand side is an average of values that are all strictly less than $1/3$.
\end{proof}
