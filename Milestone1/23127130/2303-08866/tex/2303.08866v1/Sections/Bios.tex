\begin{IEEEbiography}
[{\includegraphics[width=1in, height=1.25in, clip, keepaspectratio]{Images/author_photos/Ian_headshot.jpg}}]
{Ian E. Nielsen} is a Research Assistant, Teaching Assistant, and PhD candidate at Rowan University studying Electrical and Computer Engineering. He graduated Cum Laude and received his B.S. in Electrical and Computer Engineering at Rowan University in 2020. He is currently a recipient of the GAANN Teaching Fellowship under the U.S. Department of Education, as of January 2021. His current research is focused on robust machine learning and how it relates to explainable artificial intelligence. His work mainly focuses on computer vision and cancer diagnosis tasks. He conducts research as part of Rowan’s Artificial Intelligence Lab (RAIL). He currently coordinates a group of 19 undergraduate student researchers alongside Dr. Ravi Ramachandran though the Rowan University engineering clinic program. Since the start of his graduate education, he has taught machine learning using PyTorch and Python to dozens of students through this program. His tutorial on robust explainability was recently published in the IEEE Signal Processing Magazine.
\end{IEEEbiography}

\begin{IEEEbiography}
[{\includegraphics[width=1in, height=1.25in, clip, keepaspectratio]{Images/author_photos/ravi_02.jpg}}]
{Ravi P. Ramachandran} (SM’08) received his B.Eng. degree (with great distinction) from Concordia University in 1984, his M.Eng. degree from McGill University in 1986 and his Ph.D. degree from McGill University in 1990. From October 1990 to December 1992, he worked at the Speech Research Department at AT\&T Bell Laboratories. From January 1993 to August 1997, he was a Research Assistant Professor at Rutgers University. He was also a Senior Speech Scientist at T-Netix from July 1996 to August 1997. Since September 1997, he is with the Department of Electrical and Computer Engineering at Rowan University, where he has been a Professor since September 2006. He has served as a consultant to T-Netix, Avenir Inc., Motorola and FocalCool. From September 2002 to September 2005, he was an Associate Editor for the IEEE Transactions on Speech and Audio Processing and was on the Speech Technical Committee for the IEEE Signal Processing society. From September 2000 to December 2015, he was on the Editorial Board of the IEEE Circuits and Systems Magazine.  Since May 2002, he has been on the Digital Signal Processing Technical Committee for the IEEE Circuits and Systems society. Since May 2012, he has been on the Education and Outreach Technical Committee for the IEEE Circuits and Systems Society. He is presently an Associate Editor for the journal Circuits, Systems and Signal Processing. His research interests are in digital signal processing, speech processing, biometrics, pattern recognition, machine learning and filter design.
\end{IEEEbiography}

\begin{IEEEbiography}
[{\includegraphics[width=1in, height=1.25in, clip, keepaspectratio]{Images/author_photos/nidhal.jpg}}]
{Nidhal Carla Bouaynaya} holds a Ph.D. in Electrical and Computer Engineering (ECE) and an M.S. in Pure Mathematics from the University of Illinois at Chicago. She is a Professor of ECE and the Director of Rowan’s Artificial Intelligence Lab (RAIL). She is currently serving as the Associate Dean for Research and Graduate Studies of the Henry M. Rowan College of Engineering. Previously, she was a faculty member with the University of Arkansas at Little Rock.

Her research interests are in big data analytics, machine learning, artificial intelligence and mathematical optimization. She co-authored more than 100 referred journal articles, book chapters and conference proceedings, such as IEEE Transactions on Pattern Analysis and Machine Intelligence, IEEE Signal Processing Letters, IEEE Signal Processing Magazine and PLOS Medicine. Dr. Bouaynaya won numerous Best Paper Awards, the most recent was at the 2019 IEEE International Workshop on Machine Learning for Signal Processing. She is also the winner of the Top algorithm at the 2016 Multinomial Brain Tumor Segmentation Challenge (BRATS). Dr. Bouaynaya has been honored with numerous research and teaching awards, including the Rowan Research Achievement Award in 2017 and The University of Arkansas at Little Rock Faculty Excellence Award in Research.
Her research is primarily funded by the National Science Foundation (NSF CCF, NSF ACI, NSF DUE, NSF I-Corps, NSF ECCS, NSF OAC and NSF HRD), The National Institutes of Health (NIH), US. Department of Education (USED), New Jersey Department of Transportation (NJ DoT), US. Department of Agriculture (USDA), the Federal Aviation Administration (FAA), Lockheed Martin Inc. and other industry. She is also interested in entrepreneurial endeavors. In 2017, she Co-founded and is Chief Executive Officer (CEO) of MRIMATH, LLC, a start-up company that uses artificial intelligence to improve patient oncology outcome and treatment response. MRIMath is funded by the NIH SBIR Program. 
\end{IEEEbiography}

\begin{IEEEbiography}
[{\includegraphics[width=1in, height=1.25in, clip, keepaspectratio]{Images/author_photos/grasool.png}}]
{Ghulam Rasool} (M'2014) is an Assistant Member in the Department of Machine Learning at the H. Lee Moffitt Cancer Center \& Research Institute, Tampa, FL. He received a B.S. in Mechanical Engineering from the National University of Sciences and Technology (NUST), Pakistan, in 2000, an M.S. in Computer Engineering from the Center for Advanced Studies in Engineering (CASE), Pakistan, in 2010, and a Ph.D. in Systems Engineering from the University of Arkansas at Little Rock in 2014. He was a postdoctoral fellow with the Rehabilitation Institute of Chicago and Northwestern University from 2014 to 2016. Before joining Moffitt, he was an Assistant Professor at the Department of Electrical and Computer Engineering at Rowan University. His current research focuses on building trustworthy multimodal machine learning and artificial intelligence model for cancer diagnosis, treatment planning, and risk assessment. His research efforts are currently funded by two National Science Foundation awards (NSF) awards. Previously his research was supported by the National Institute of Health (NIH), U.S. Department of Education, NSF, the New Jersey Health Foundation (NJHF), Google, NVIDIA, and Lockheed Martin, Inc. His work on Bayesian machine learning won the Best Student Award at the 2019 IEEE Machine Learning for Signal Processing Workshop.
\end{IEEEbiography}