\documentclass{amsart}
% CVPR 2022 Paper Template
% based on the CVPR template provided by Ming-Ming Cheng (https://github.com/MCG-NKU/CVPR_Template)
% modified and extended by Stefan Roth (stefan.roth@NOSPAMtu-darmstadt.de)

\documentclass[10pt,twocolumn,letterpaper]{article}

%%%%%%%%% PAPER TYPE  - PLEASE UPDATE FOR FINAL VERSION
%\usepackage[review]{cvpr}      % To produce the REVIEW version
\usepackage{cvpr}              % To produce the CAMERA-READY version
%\usepackage[pagenumbers]{cvpr} % To force page numbers, e.g. for an arXiv version

% Include other packages here, before hyperref.
\usepackage{graphicx}
\usepackage{amsmath}
\usepackage{amssymb}
\usepackage{booktabs}


% It is strongly recommended to use hyperref, especially for the review version.
% hyperref with option pagebackref eases the reviewers' job.
% Please disable hyperref *only* if you encounter grave issues, e.g. with the
% file validation for the camera-ready version.
%
% If you comment hyperref and then uncomment it, you should delete
% ReviewTempalte.aux before re-running LaTeX.
% (Or just hit 'q' on the first LaTeX run, let it finish, and you
%  should be clear).
\usepackage[pagebackref,breaklinks,colorlinks]{hyperref}


% Support for easy cross-referencing
\usepackage[capitalize]{cleveref}
\crefname{section}{Sec.}{Secs.}
\Crefname{section}{Section}{Sections}
\Crefname{table}{Table}{Tables}
\crefname{table}{Tab.}{Tabs.}


%%%%%%%%% PAPER ID  - PLEASE UPDATE
\def\cvprPaperID{2200} % *** Enter the CVPR Paper ID here
\def\confName{CVPR}
\def\confYear{2023}


% \usepackage[latin1]{inputenc}
\usepackage[british]{babel}
\usepackage[all]{xy}
\usepackage{amscd}
\usepackage{amssymb}
\usepackage{amsthm}
\usepackage{enumitem}
\usepackage{mathrsfs,bbm}
\usepackage{xcolor,graphicx}
\usepackage{graphics}
\usepackage{soul}
\usepackage{comment}
\usepackage[all]{xy}
\usepackage{amscd}
\usepackage{amssymb,amsmath,latexsym}
\usepackage{amsthm}
\usepackage{enumitem}
\usepackage{mathrsfs,bbm}
\usepackage{dsfont}
\usepackage{tikz-cd}
\usepackage[T1]{fontenc}
\usepackage[utf8]{inputenc}  
 %
%%%%%%%%%%%%%%%%%%%%%%%%%%%%%%%%%%
%pagestyle
%%%%%%%%%%%%%%%%%%%%%%%%%%%%%%%%%%
%\pagestyle{plain}
\textwidth=430pt
\headsep=.7cm
\evensidemargin=15pt
\oddsidemargin=15pt
\leftmargin=0cm
\rightmargin=0cm
%%
%%%%%%%%%%%%%%%%%%%%%%%
\newcommand*\fixitem {\item[]%
  \refstepcounter{enumi}\hskip-\leftmargin\labelenumi\hskip\labelsep}
\newtheorem*{mainthm}{Main Theorem}
\newtheorem*{mainthm1}{Theorem}
\newtheorem*{maincor}{Corollary}
\usepackage[colorlinks=true]{hyperref}
\DeclareMathOperator{\Forall}{\forall}
\DeclareMathOperator{\Exists}{\exists}
\DeclareMathOperator{\ord}{ord}
\newcommand{\phiD}{\varphi_D}
\newcommand{\phiDI}{\varphi_{\mathbf{D}_I}}
\newcommand{\phiDIj}{\varphi_{\mathbf{D}_I (j)}}
\newcommand{\phiH}{\varphi_H}
\newcommand{\phiTimes}{\phiD \otimes \phiH}
\newcommand{\phiTimesDI}{\varphi_{\mathbf{D}_I} \otimes \phiH}
\newcommand{\R}{\mathscr{A}}
\newcommand{\X}{\mathscr{X}}
\newcommand{\Xf}{\mathscr{X}_{(k_0 ,i)}[r_0]}
\newcommand{\Xfr}{\mathscr{X}_{(k_0,i)}[r]}
\newcommand{\hotimes}{\widehat{\otimes}}
\newcommand{\C}{\mathbb{C}_p}
\newcommand{\V}{\mathscr{V}}
\newcommand{\B}{\mathscr{B}}
\newcommand{\dualD}{\mathfrak{D}}
\newcommand{\Dg}{\mathbf{D}}
\newcommand{\DD}{\mathcal{D}^0}
\newcommand{\DDg}{\mathcal{D}}
\newcommand{\DV}{\mathcal{D}}
\newcommand{\W}{\mathscr{W}_N}
\newcommand{\Ao}{\mathbf{A}^\circ}
\newcommand{\AoK}{\mathbf{A}^\circ_{\K}}
\newcommand{\AK}{\mathbf{A}_{/\K}}
\newcommand{\OOO}{\mathscr{A}^\circ}
\newcommand{\K}{\mathcal{K}} 
\newcommand{\OK}{\mathcal{O}_{\K}}
\newcommand{\varprojlog}[1]{\underleftarrow{\log\!^{#1}}}
\newcommand{\T}{\mathscr{T}}
\newcommand{\TT}{\mathbf{T}}
\newcommand{\VV}{\mathbf{V}}
\newcommand{\HH}{\mathcal{H}}
\newcommand{\hh}{\mathcal{H}^+}
\newcommand{\HG}[2]{\mathcal{H}_{#1}(#2)}
\newcommand{\hhl}{\mathcal{H}^{+,[l]}}
\newcommand{\hhj}{\mathcal{H}^{+,[j]}}
\newcommand{\hhjj}{\mathcal{H}^{+,[l,l']}}
\newcommand{\GS}{G_{\mathbb{Q},S}}
\newcommand{\Rf}{R_{(k_0 ,i)}[r_0]}
\newcommand{\Rfr}{R_{(k_0 ,i)}[r]}
\newcommand{\parT}{\langle T\rangle}
\newcommand{\Zf}{Z_{(k_0 ,i)}[r_0]}
\newcommand{\Zfr}{\mathscr{Z}_{(k_0 ,i)}[r]}
\newcommand{\ZFf}{\mathscr{Z}_{(k_0 ,i)}[r_0]}
\newcommand{\ZFfr}{\mathscr{Z}_{(k_0 ,i)}[r]}
\newcommand{\ZF}{\mathscr{Z}}


\begin{document}

%%%%%%%%% TITLE - PLEASE UPDATE
\title{\ours: Neural 3D Relightable Faces}


\author{Anurag Ranjan$^\symknight$ \; \; \;
Kwang Moo Yi$^{\symknight\symrook}$ \; \; \;
Jen-Hao Rick Chang$^\symknight$ \; \; \;
Oncel Tuzel$^\symknight$ \\
$^\symknight$Apple \; \; \;
$^\symrook$The University of British Columbia
}

\twocolumn[{%
\renewcommand\twocolumn[1][]{#1}%
\maketitle
\begin{center}
    \newcommand{\teaserwidth}{\textwidth}
\vspace{-0.3cm}
   \includegraphics[width=0.96\linewidth]{figs/teaser4.jpg} \\
  \includegraphics[width=0.96\linewidth]{figs/teaser5.jpg}
%   }
   \captionof{figure}{Generated samples from our model. \textbf{Left}: 3D reconstruction visualization. \textbf{Center and Right}: Rendered faces using 2 different illumination conditions under 3 different poses. Illumination visualization using spherical harmonics~\cite{ramamoorthi2001efficient}.
}
\label{fig:teaser}
\end{center}%
}]

\maketitle


%%%%%%%%% ABSTRACT


Over the past few years, there has been a significant amount of research focused on studying the ReLU activation function, with the aim of achieving neural network convergence through over-parametrization. However, recent developments in the field of Large Language Models (LLMs) have sparked interest in the use of exponential activation functions, specifically in the attention mechanism.

Mathematically, we define the neural function $F: \R^{d \times m} \times  \mathbb{R}^d \rightarrow \mathbb{R}$ using an exponential activation function. Given a set of data points with labels $\{(x_1, y_1), (x_2, y_2), \dots, (x_n, y_n)\} \subset \mathbb{R}^d \times \mathbb{R}$ where $n$ denotes the number of the data. Here $F(W(t),x)$ can be expressed as $F(W(t),x) := \sum_{r=1}^m a_r \exp(\langle w_r, x \rangle)$, where $m$ represents the number of neurons, and $w_r(t)$ are weights at time $t$. It's standard in literature that $a_r$ are the fixed weights and it's never changed during the training. We initialize the weights $W(0) \in \mathbb{R}^{d \times m}$ with random Gaussian distributions, such that $w_r(0) \sim \mathcal{N}(0, I_d)$ and initialize $a_r$ from random sign distribution for each $r \in [m]$.

Using the gradient descent algorithm, we can find a weight $W(T)$ such that $\| F(W(T), X) - y \|_2 \leq \epsilon$ holds with probability $1-\delta$, where $\epsilon \in (0,0.1)$ and $m = \Omega(n^{2+o(1)}\log(n/\delta))$. To optimize the over-parametrization bound $m$, we employ several tight analysis techniques from previous studies [Song and Yang arXiv 2019, Munteanu, Omlor, Song and Woodruff ICML 2022]. 

 



%%%%%%%%% BODY TEXT
\section{Introduction}
\label{sec:introduction}
% \begin{itemize}
%     % Diffusion of FL
%     \item {\st{Diffusion of FL}}
%     % Security threats to FL
%     \item {\st{Security threats to FL with particular focus on model poisoning}}
%     % Limitations of existing countermeasures
%     \item {\st{Current countermeasures (e.g., KRUM) and their limitations}}
%     % Proposed method and its advantages
%     \item {\st{Intuitive description of the proposed method and its difference (i.e., advantages) w.r.t. state of the art}}
%     % Main contributions
%     \item {\st{Summary of the main contributions of this work}}
%     % Paper's structure and organization
%     \item {\st{Paper's structure and organization}}
% \end{itemize}

% Diffusion of FL
Recently, {\em federated learning} (FL) has emerged as the leading paradigm for training distributed, large-scale, and privacy-preserving machine learning (ML) systems~\cite{mcmahan2017googleai,mcmahan2017aistats}. 
The core idea of FL is to allow multiple edge clients to collaboratively train a shared, global model without disclosing their local private training data.
%Specifically, an FL system consists of a central server and many edge clients; 
A typical FL round involves the following steps: {\em(i)} the server randomly picks some clients and sends them the current, global model; {\em(ii)} each selected client locally trains its model with its own private data; then, it sends the resulting local model to the server;\footnote{Whenever we refer to global/local model, we mean global/local model {\em parameters}.} {\em(iii)} the server updates the global model by computing an \emph{aggregation function}, usually the average (FedAvg), on the local models received from clients.
% \begin{enumerate}
%     \item[{\em(i)}] the server sends the current, global model to the clients and appoints some of them for training;
%     \item[{\em(ii)}] each selected client locally trains its copy of the global model with its own private data; then, it sends the resulting local model back to the server;\footnote{Whenever we refer to global/local model, we mean global/local model {\em parameters}.}
%     \item[{\em(iii)}] the server updates the global model by computing an \emph{aggregation function} on the local models received from clients (by default, the average, also referred to as FedAvg~\cite{mcmahan2017aistats}).
% \end{enumerate}
This process goes on until the global model converges. %(e.g., after a certain number of rounds or other similar stopping criteria).
%\\
% The advantages of FL over the traditional, centralized learning paradigm are undoubtedly clear in terms of flexibility/scalability (clients can join/disconnect from the FL network dynamically), network communications (only model weights\footnote{We will use \textit{parameters} and \textit{weights} interchangeably.} are exchanged between clients and server), and privacy (each client's private training data is kept local at the client's end and not uploaded to the server).
\\
% Security threats to FL
%However, the growing adoption of FL also raises security concerns~\cite{costa2022covert}, particularly about its confidentiality, integrity, and availability.
Although its advantages over standard ML, FL also raises security concerns~\cite{costa2022covert}. %, particularly about its confidentiality, integrity, and availability~\cite{costa2022covert}.
% OLD, LONG VERSION
% Indeed, some work deals with privacy leakage that may expose the local data of some clients~\cite{melis2019sp}. 
% A large body of work, instead, investigates attacks that usually aim to detriment the predictive accuracy of the learned global model. For instance, \emph{data poisoning} attacks achieve this goal by letting an adversary pollute the training set of some corrupt FL clients with maliciously crafted examples~\cite{jagielski2018sp}.
% Similarly, in \emph{model poisoning} the attacker attempts to tweak the global model weights~\cite{bhagoji2019pmlr} by directly perturbing the local model's weights of some infected FL clients before these are sent to the central server for aggregation, usually via so-called Byzantine attacks. 
% It turns out that Byzantine model poisoning attacks severely impact standard FedAvg; therefore, more robust aggregation functions must be designed to make FL systems secure.
Here, we focus on \emph{untargeted model poisoning} attacks~\cite{bhagoji2019pmlr}, where an adversary attempts to tweak the global model weights %\footnote{We will use the terms \textit{parameters} and \textit{weights} interchangeably.} 
by directly perturbing the local model's parameters of some infected clients before these are sent to the central server for aggregation.
In doing so, the adversary aims to jeopardize the global model \textit{indiscriminately} at inference time.
Such model poisoning attacks severely impact standard FedAvg; therefore, more robust aggregation functions must be designed to secure FL systems.
\\
% In this paper, we focus on designing a novel robust aggregation scheme at the server's end to contrast the effect of Byzantine model poisoning attacks.
%
% Current countermeasures and their limitations
%Several countermeasures have been proposed in the literature to combat model poisoning attacks on FL systems.
% Some methods use simple statistics more robust than plain average to smooth the impact of malicious updates (e.g., Trimmed Mean and FedMedian~\cite{yin2018icml}). 
% Other defenses implement outlier detection techniques to discard malicious updates from the aggregation performed at the server's end. Those are either based on heuristics (e.g., Krum/Multi-Krum~\cite{blanchard2017nips} and Bulyan~\cite{mhamdi2018pmlr}) or data-driven approaches (e.g., K-means clustering~\cite{shen2016acm} or DnC via spectral analysis~\cite{shejwalkar2021ndss}). 
% Finally, some strategies rely on a centralized ``source of trust'' to spot potential malicious updates (e.g., FLTrust~\cite{cao2020fltrust}).
% Several countermeasures have been proposed in the literature to combat model poisoning attacks on FL systems, i.e., to discard possible malicious local updates from the aggregation performed at the server's end. 
% These techniques range from simple statistics more robust than plain average (e.g., Trimmed Mean and FedMedian~\cite{yin2018icml}) to outlier detection heuristics (e.g., Krum/Multi-Krum~\cite{blanchard2017nips} and Bulyan~\cite{mhamdi2018pmlr}) or data-driven approaches (e.g., spectral analysis via K-means clustering~\cite{shen2016acm} or spectral analysis), or methods based on ``source of trust'' (e.g., FLTrust~\cite{cao2020fltrust}).
% OLD, LONG VERSION
%Several countermeasures have been proposed in the literature to combat Byzantine model poisoning attacks on FL systems.
% Descriptive statistics
% For example, Trimmed Mean and FedMedian aggregate local model updates using more robust statistics than standard average~\cite{yin2018icml}.
%
% % Heuristics for outlier detection
% Many existing Byzantine-resilient strategies implement some outlier detection heuristics to discard the model updates sent by potentially malicious clients from the input of the aggregation function.
% One of the most popular heuristics is Krum~\cite{blanchard2017nips}.
% This strategy tries to mitigate the impact of Byzantine attacks by selecting as a global model the local model with the smallest sum of Euclidean distances to {\em all} the other local models.
% Although powerful, Krum requires the server to know (or, at least, estimate) the number of malicious FL clients upfront, which is generally impossible in a realistic attack scenario. %
% Moreover, Krum may become ineffective for complex, high-dimensional model parameter spaces due to the curse of dimensionality.
% Bulyan~\cite{mhamdi2018pmlr} tries to overcome this issue by combining Krum with a variant of Trimmed Mean.
% % Data-driven outlier detection
% Other strategies use data-driven outlier detection techniques -- e.g., via K-means clustering~\cite{shen2016acm} -- to spot potential malicious local model updates. 
% %For instance, Shen et al. propose to cluster local model updates with K-means and thus identify outliers.
%
% % Other techniques
% As far as the server is concerned, any local model received can be from a potential malicious client. 
% FLTrust~\cite{cao2020fltrust} assumes the server acts as a client, i.e., trains a local model on an additional {\em trustworthy} dataset at the server's end and compares it against all the local models from other clients. 
% This way, the server can rely on some ``source of trust'' when discarding potentially malicious clients.
%\\
% Limitations of existing Byzantine-resilient strategies
Unfortunately, existing defense mechanisms either rely on simple heuristics (e.g., Trimmed Mean and FedMedian by~\cite{yin2018icml}) or need strong and unrealistic assumptions to work effectively (e.g., foreknowledge or estimation of the number of malicious clients in the FL system, as for Krum/Multi-Krum~\cite{blanchard2017nips} and Bulyan~\cite{mhamdi2018pmlr}, which, however, cannot exceed a fixed threshold).
Furthermore, outlier detection methods using K-means clustering~\cite{shen2016acm} or spectral analysis like DnC~\cite{shejwalkar2021ndss} do not directly consider the temporal evolution of local model updates received.
Finally, strategies like FLTrust~\cite{cao2020fltrust} require the server to collect its own dataset and act as a proper client, thereby altering the standard FL protocol.
\\
% OLD, LONG VERSION
% Overall, existing Byzantine-resilient strategies are either simple heuristics (e.g., FedMedian) or, if they are more complex, they rely on strong and unrealistic assumptions to work effectively (e.g., knowing the number of malicious clients in the FL system in advance, as for Krum and alike).
% Furthermore, data-driven outlier detection methods do not consider the temporary evolution of local model updates received (e.g., K-means clustering). 
% Finally, strategies like FLTrust requires the server to collect its own dataset and act as a proper client, thereby altering the standard FL protocol.
%
% Description of the proposed method
This work introduces a novel pre-aggregation \textit{filter} robust to untargeted model poisoning attacks. Notably, this filter $(i)$ operates without requiring prior knowledge or constraints on the number of malicious clients and $(ii)$ inherently integrates temporal dependencies. 
The FL server can employ this filter as a preprocessing step before applying \textit{any} aggregation function, be it standard like FedAvg or robust like Krum or Bulyan.
Specifically, we formulate the problem of identifying corrupted updates as a multidimensional (i.e., matrix-valued) time series anomaly detection task. 
The key idea is that legitimate local updates, resulting from well-calibrated iterative procedures like stochastic gradient descent (SGD) with an appropriate learning rate, show \textit{higher predictability} compared to malicious updates. This hypothesis stems from the fact that the sequence of gradients (thus, model parameters) observed during legitimate training exhibit regular patterns, as validated in Section~\ref{subsec:intuition}. %until convergence. 
%This regularity may be more pronounced for smooth convex loss functions, but it can still be captured within an appropriate time window, even for more complex and convoluted loss surfaces. 
%We provide evidence of this claim in Appendix~B, where we show that the average mutual information (i.e., ``predictability''), calculated over pairs of legitimate model updates sent at different FL rounds, is significantly higher than the corresponding computation for a malicious client.
\\
Inspired by the matrix autoregressive (MAR) framework for multidimensional time series forecasting~\cite{chen2021je}, we propose the FLANDERS ({\em \textbf{F}ederated \textbf{L}earning meets \textbf{AN}omaly \textbf{DE}tection for a \textbf{R}obust and \textbf{S}ecure}) filter.
The main advantages of FLANDERS over existing strategies like FLDetector~\cite{zhao2020multivariate} are its resilience to large-scale attacks, where $50\%$ or more FL participants are hostile, and the capability of working under realistic non-iid scenarios.
We attribute such a capability to two key factors: $(i)$ FLANDERS works without knowing a priori the ratio of corrupted clients, and $(ii)$ it embodies temporal dependencies between intra- and inter-client updates, quickly recognizing local model drifts caused by evil players. Below, we summarize our main contributions:

\begin{itemize}
\item[{\em(i)}]
We provide empirical evidence that the sequence of models sent by legitimate clients is more predictable than those of malicious participants performing untargeted model poisoning attacks.
\\
\item[{\em(ii)}] 
We introduce FLANDERS, the first pre-aggregation filter for FL robust to untargeted model poisoning based on multidimensional time series anomaly detection.
\\
\item[{\em(iii)}] 
We integrate FLANDERS into Flower,\footnote{\scriptsize{\url{https://flower.dev/}}} a popular FL simulation framework for reproducibility.
\\
\item[{\em(iv)}] 
We show that FLANDERS improves the robustness of the existing aggregation methods under multiple settings: different datasets, client's data distribution (non-iid), models, and attack scenarios.
\\
\item[{\em(v)}] 
We publicly release all the implementation code of FLANDERS along with our experiments.\footnote{\scriptsize{\url{https://anonymous.4open.science/r/flanders_exp-7EEB}}}
\end{itemize}

% Paper's structure and organization
The remainder of the paper is structured as follows. %some related work and the current state-of-the-art solutions to security issues that FL entails. 
Section~\ref{sec:background} covers background and preliminaries. 
In Section~\ref{sec:related}, we discuss related work.
Section~\ref{sec:problem} and Section~\ref{sec:method} describe the problem formulation and the method proposed. % to tackle it. 
Section~\ref{sec:experiments} gathers experimental results. %, and Section~\ref{sec:limitations} discusses some limitations of this work.
Finally, we conclude in Section~\ref{sec:conclusion}.
 %discusses the limitations of this work and draws future research directions.
%reports conclusions and draws perspectives for future research directions.

%%%%%%% OLD %%%%%%%
%to overcome the resilience of Byzantine failures in distributed Stochastic Gradient Descent computations. 
% The strength of Krum is its time complexity, which is linear in the gradient dimension. 
% However, the robustness of the approach is guaranteed for gradient-based learning applications only when the majority of the clients are not compromised. 
% Besides, the aggregation mechanism of Krum, as well as that of similar methods, is robust from a coarse-grained perspective and does not provide solutions to errors and perturbations that may occur at inference time.
%A related approach to~\cite{blanchard2017nips} is the work of Su et al.~\cite{su2016dc}. Here, the authors propose an iterated approximate agreement to tackle a multi-layer scenario attacked by Byzantine agents. 
%However, the method works efficiently on the sole discrete context and it is inapplicable to continuous state environments.
%\gabri{Maybe, we should just talk about the main limitations of existing countermeasures without digging into their details (or, we can just mention Krum as this is the most popular one). I will move the description of all these methods to the Related Work section.}
\section{Related work}
\noindent \textbf{Video foundation models.}
With sufficient computational power and an abundant source of data, there have been attempts to build a single large-scale foundation model that can be adapted to diverse downstream tasks.
Along with the success of foundations models in the natural language processing domain~\cite{brown2020language,chen2021evaluating,devlin2019bert} and in computer vision~\cite{bertasius2021space,jia2021scaling,radford2021learning}, video data has become another data type of interest, as it has grown in scale due to numerous internet video-sharing platforms.
Accordingly, several methods to train a video foundation model have been proposed.
Due to the innate multi-modality of video data, \textit{i.e.}, a combination of visual $\cdot$ vocal $\cdot$ textual context, most works have centered around the variations of the cross-modal attention mechanism \cite{akbari2021vatt,bertasius2021space,gabeur2020multi,luo2020univl,neimark2021video,tan2021look,wei2020multi,yang2021taco}.
In addition, as most video data lack proper labels or descriptions, contrastive learning methods were studied to learn meaningful feature representations or enhance video-text alignment in a self-supervised manner \cite{akbari2021vatt,kuang2021video,luo2020univl,yang2021taco}.

More specifically, MERLOT \cite{zellers2021merlot} proposed a multi-modal representation learning method for visual commonsense reasoning, which also performed well in twelve video reasoning tasks.
VATT \cite{akbari2021vatt} introduced a multi-modal learning method via contrastive learning. 
The pre-trained model performed well in a variety of vision tasks from image classification to video action recognition and zero-shot video retrieval.
Another representative work, UniVL \cite{luo2020univl} proposed a straightforward pre-training method with auxiliary loss functions. 
After fine-tuning on a specific task, the pre-trained model performed outstandingly in a wide range of tasks of text-to-video retrieval, action segmentation, action step localization, video sentiment analysis, and video captioning.
Other foundation models for multiple video tasks include \cite{li2020hero,sun2019learning,sun2019videobert,zhu2020actbert,fu2021violet,wang2022all}. 

\noindent \textbf{Auxiliary learning.}
In order to enhance the performance of one or a multitude of primary tasks, auxiliary learning methods can be incorporated.
\cite{ruder2017overview} introduced Multi-task learning (MTL) to the deep neural networks by training a single model with multiple task losses to assist learning on the main task.
Such a method is generally adapted to pre-train the foundation models in the self-supervised manner~\cite{li2020hero,sun2019learning,sun2019videobert,zhu2020actbert,fu2021violet,wang2022all}.
However, these various pretext task losses used in the pre-training phase are ignored in the fine-tuning phase, and only the primary task loss is minimized.

Recently, meta-learning methods have been introduced for auxiliary learning.
\cite{liu2019self,navon2020auxiliary,shu2019meta} proposed a meta-learning method in which the model learns auxiliary tasks to generalize well to unseen data. 
In these settings, a separate subset of data is held out as the primary task, while the others are used as auxiliary tasks that aid the primary task's performance.
Similar methods were adopted for computer vision tasks such as semantic segmentation \cite{xu2021leveraging}.
Other domain applications include navigation tasks with reinforcement learning \cite{ye2021auxiliary}, or self-supervised learning methods on graph data \cite{hwang2020self}.
\section{Method}
\label{sec:method}

% \ml{``Inconsistent'' to ``large variation''}

% In this section, we propose our methods based on the observations in Section \ref{sec:motivation}.
In this section, we propose two techniques to further enhance the strong baseline to capture the variation of activation distributions better.
We first introduce spatial re-scaling to adapt the network to pixel-to-pixel variation.
We then propose channel-wise shifting and re-scaling to better capture the channel-to-channel variation.
Meanwhile, as both of the two methods are image-dependent, the image-to-image variation can be captured naturally.
By combining the two methods with our strong baseline, we build our enhanced BNN for SR, named EBSR.

% Because the activation distributions among pixels, channels and images have large variations \red{**are highly inconsistent} in SR networks, we introduce spatial re-scaling to adapt to pixel-wise variations and channel shift and re-scaling to adapt to channel-wise variations. And both of them are image-dependent to adapt to image-wise variations, which means during inference our network re-scales and shifts the distributions of activations flexibly for different input images. Based on these methods, we build an enhanced binary neural network for image super-resolution (EBSR).

% According to [3], the difference of activation magnitudes indicates different scaling factors are needed for each pixel.

\subsection{Spatial Re-scaling}
% It is better to use different scaling factors for different pixels to reduce the quantization error and retain more detailed information for image super-resolution. 

% \ml{In the main method, we do not need to introduce the previous works but can focus on introducing our own method. Channel rescaling in Real-to-binary Net is not relevant in this context.}

% Re-scaling the output of binary convolutions was proposed at the birth of BNN in XNOR-Net \cite{rastegari2016xnor} to reduce quantization error and improve accuracy for image classification tasks.
% It is computed as below:
% \begin{equation}
% \mathcal{A} * \mathcal{W} \approx(\operatorname{sign}(\mathcal{A}) \circledast \operatorname{sign}(\mathcal{W})) \odot \mathcal{K} \alpha
% \label{eq:xnor-net rescale}
% \end{equation}
% where $\circledast$ denotes the binary convolution and $\odot$ denotes the element-wise multiplication.
% $\mathcal{A}$, $\mathcal{W}$, $\alpha$, and $\mathcal{K}$ denote the activation, weight, weight scaling factor, and activation scaling factor, respectively.
%  Later in XNOR-Net++ \cite{bulat2019xnor}, Bulat et al. fuse the activation and weight scaling factors into a single one that is learned end-to-end based on gradients and this improves the classification accuracy on ImageNet dataset.

% % It is computed as Eq.~\ref{eq:xnor-net rescale}, where $\circledast$ denotes 
% %  the binary convolution and $\odot$ denotes the element-wise multiplication. The binary convolution of $\mathcal{A}$ and $\mathcal{W}$ is rescaled by the weight scaling factor $\alpha$ and the activation scaling factor $\mathcal{K}$, both of which are calculated analytically.


% \zc{Similarly, you should explain the meaning of A, W and the operators $\circledast$ in the formula}
% Then in Real-to-binary Net \cite{martinez2020training}, Martinez et al. used a data-driven channel re-scaling module that takes the pre-convolution activations as input to predict the activation scaling factor. Unlike that in XNOR-Net++ \cite{bulat2019xnor}, these scaling factors are not fixed during inference but rather inferred from data. By doing this, they further improved the classification accuracy on ImageNet over XNOR-Net++. 
As is shown in Figure \ref{fig:pixel}, activation distributions have large pixel-to-pixel variation in SR networks
and the difference of activation magnitudes indicates different scaling factors are preferred for different pixels.
Inspired by \cite{martinez2020training}, we propose spatial re-scaling to better adapt the network to the spatial variation
of activation distributions in SR networks.
% fit the various pixel-wise distributions in SR networks.
We take the real-valued activations $A$ before convolution as input and predict pixel-wise scaling factors $S(A)$, which re-scale the binary convolution output. Spatial re-scaling process can be formulated as follows:
\begin{equation}
A * W \approx(\operatorname{sign}(A) \circledast \operatorname{sign}(W)) \odot \alpha \odot S(A)
\label{eq:spatial rescale}
\end{equation}
where $\circledast$ denotes 
the binary convolution and $\odot$ denotes the element-wise multiplication. $A$, $W$, $\alpha$, and $S\left(A\right)$ denote real-valued activations, weights, the scaling factor of weights, and the spatial-wise scaling factor of activations respectively. $S\left(A\right) \in \mathbb{R}^{1\times H\times W}$ can be calculated with a convolution and a sigmoid function.
% as $\sigma\left( CONV\left(A\right)\right)$. 
As shown in Figure \ref{fig:method}(a), real-valued activations first go through a convolution layer,
which has an input channel of $C$ and an output channel of 1, 
and then pass through a sigmoid function to produce the scaling factors $S(A)$ along the spatial dimension.
During inference, the scaling factor will change dynamically according to different input feature maps.
By re-scaling binary convolution output using $S(A)$, we can reduce the quantization error and the original pixel-wise information in FP activation
will be preserved much better.
Spatial re-scaling leads to a large PSNR improvement of 0.24 dB (from 30.30 dB to 31.54 dB) on Set5 and 0.22 dB (from 25.09 dB to 25.31 dB)
on Urban100 compared with our strong baseline. 

\subsection{Channel-wise Shifting and Re-scaling}

\begin{table}[!tb]
\centering
\caption{Comparison between whether to fuse channel-wise shifting and re-scaling or not based on our baseline with spatial re-scaling. }
\label{tab:fusing}

\scalebox{0.65}{
\begin{tabular}{c|cc|cc|cc}
\hline
\multirow{2}{*}{Method}     & \multirow{2}{*}{OPs} & \multirow{2}{*}{Params} & \multicolumn{2}{c|}{Set5} & \multicolumn{2}{c}{Urban100} \\ \cline{4-7} 
                            &                      &                         & PSNR        & SSIM        & PSNR          & SSIM         \\ \hline
Baseline + spatial re-scale & 2.16G                & 0.05M                   & 31.54       & 0.883       & 25.31         & 0.759        \\
+ channel-wise shift and re-scale             & 2.34G                & 0.09M                   & 31.61       & 0.885       & 25.35         & 0.761        \\
+ w/ fusing                   & 2.27G                & 0.08M                   & \textbf{31.64}       & \textbf{0.885}       & \textbf{25.36}         & \textbf{0.761}        \\ \hline
\end{tabular}
}
\end{table}

In SR networks, activation distributions exhibit larger channel-to-channel variation (Figure \ref{fig:chl}).
Both the mean and magnitude of the activation distributions vary significantly across channels.
% Thus we use channel-wise shifting and re-scaling to adapt to various channel-wise distributions. 
\cite{martinez2020training} has proposed the data-driven channel re-scaling, 
but our method differs from them in further introducing data-driven thresholds to handle the channel-wise variation of both mean and magnitude.
Since the blocks to generate the scaling factors and thresholds are very similar, we further propose to fuse them into one module.
% and fusing channel-wise shifting and re-scaling into one module.
We evaluate the effect of fusing the two blocks in Table \ref{tab:fusing}.
With channel-wise shifting and re-scaling fused, our models have fewer operations and parameters overhead and slightly higher performance.

For the specific process, we take the real-valued activations as input and predict different thresholds and scaling factors for each channel. They are also image dependent, e.g., $\beta_{i}$ in Eq.\ref{eq:act_binarize} is no longer fixed during inference but generated according to different input feature maps. Channel-wise shifting and re-scaling can be formulated as follows:
\begin{equation}
A * W \approx(\operatorname{sign}(A-C_s(A)) \circledast \operatorname{sign}(W)) \odot \alpha \odot C_r(A)
\label{eq:channel-wise_shift_and_rescale}
\end{equation}
where $\circledast$ denotes 
the binary convolution and $\odot$ denotes the element-wise multiplication. $C_s(A), C_r(A) \in \mathbb{R}^{C\times1\times1}$ denote the channel-wise threshold and scaling factor, respectively. 
We show the block diagram in Figure \ref{fig:method}(b).
The real-valued input feature map is first squeezed to a ${C\times1\times1}$ vector by a global average pooling (GAP) layer.
The subsequent fully connected layers and ReLU learn the channel-wise information and output a ${2C\times1\times1}$ vector.
Then the ${2C\times1\times1}$ vector is split into two ${C\times1\times1}$ vectors.
We use the first $C$ channels as the channel-wise bias and pass the last $C$ channels through a sigmoid layer 
as the channel-wise scaling factor, which are used to shift the real-valued activations and re-scale the binary convolution output, respectively. 


% \ml{We can mention previously, channel-wise re-scale has been proposed. We propose to fuse them. Add the comparison between fuse v.s. no fuse.}

\begin{figure}[!tbp]%
  \centering
    \includegraphics[width=0.4\textwidth]{fig/methods.png}
  
% \subfloat[channel-wise shifting\&re-scale]{
%     \label{subfig:channel-wise shifting and re-scale}
%     \includegraphics[width=0.2\textwidth]{fig/chl shift and rescale.png}
%   }

  \caption{Block diagram for spatial re-scaling, and channel-wise shifting and re-scaling.} 
  % Input A is the real-valued activation tensor and C, H, and W denote its dimension. GAP stands for global average pooling. The reduction ratio r is set to 16 for a better trade-off between the performance and the number of operations and parameters.}
  \label{fig:method}
\end{figure}


\subsection{Network Structure}

Combining the spatial re-scaling and the channel-wise shifting and re-scaling methods, we construct the enhanced convolution layer (E-Conv).
Then we build our EBSR model based on E-Conv.
In Figure \ref{fig:E-conv}, we compare the binary convolution layer used in the baseline network and our proposed E-Conv.
We use spatial and channel-wise scaling factors to re-scale the binary convolution output,
and use channel-wise shifting to learn appropriate thresholds for each channel before binarization.
The scaling factors and threshold used in E-Conv are learnable and depend on the real-valued input activations.
In this way, our proposed EBSR can adapt to pixel-to-pixel, channel-to-channel, and image-to-image variations
to reduce the large binarization error and preserve more details.
% In this way, our proposed E-Conv reduces the large quantization error caused by binarization and keeps the original information of input feature maps to a large extent.


\begin{figure}[!tb]%
  \centering

    \includegraphics[width=0.5\textwidth]{fig/E-conv.png}

  \caption{Comparison of (a) the binary convolution layer with a skip connection used in our baseline network and (b) the proposed E-Conv.}
  \label{fig:E-conv}
\end{figure}


Figure \ref{fig:network} shows the basic block based on the E-Conv and our EBSR composed of the basic blocks. Following existing works, the convolution layers in the head and tail modules are not binarized. We choose the lightweight EDSR which has 16 basic blocks and 64 channels, and EDSR which has 32 basic blocks and 256 channels as our backbones, which correspond to EBSR-light and EBSR, respectively.

\begin{figure}[!tb]%
  \centering
  {
    \includegraphics[width=0.35\textwidth]{fig/network.png}
  }
  
  \caption{The structure of our proposed EBSR.  Convolution layers in purple are real-valued vanilla 3x3 convolutions.}
  \label{fig:network}
\end{figure}
We present in section~\ref{ssec:faces} an application of PnP-HVAE on face images, using a pretrained state-of-the-art hierarchical VAE. 
Next, we study the application of our framework to natural images. To that end, we introduce  in section~\ref{ssec:patchVDVAE}  a patch hierachical VAE architecture, that is able to model natural images of different resolutions. In section~\ref{ssec:app_nat}, we provide deblurring, super-resolution and inpainting experiments to demonstrate the relevance of the proposed method.

Additional results are presented in Appendix~\ref{app:add}. All experiments can be reproduced using the code available at \url{https://github.com/jprost76/PnP-HVAE}.



\subsection{Face Image restoration (FFHQ)}\label{ssec:faces}
We first demonstrate the effectiveness of PnP-HVAE on highly structured data, by performing face image restoration.
Latent variable generative models can accurately model structured images such as face images \cite{karras2019style,vahdat2020nvae,child2021very,kingma2018glow}, and then be used to produce high quality restoration of such data. 
In our experiments, we use the VDVAE model of~\cite{child2021very}, pre-trained on the FFHQ dataset~\cite{karras2019style}, as our hierarchical VAE prior.
VDVAE has $L=66$ latent variable groups in its hierarchy and generates images at resolution $256\times256$.

We compare PnP-HVAE with the intermediate layer optimization algorithm (ILO)~\cite{daras2021intermediate} that is based on a different class of generative models than HVAE. ILO is a GAN inversion method which optimizes the image latent code along with the intermediate layer representation of a StyleGAN to generate an image consistent with a degraded observation.
We use the official implementation of ILO, along with a StyleGAN2 model~\cite{karras2020analyzing, stylegan2pytorch}, that was trained for 550k iterations on images of resolution $256\times256$ from FFHQ.  
As VDVAE and StyleGAN models are not trained on the same train-test split of FFHQ, we chose to evaluate the methods on a subset of 100 images from the CelebA dataset~\cite{liu2018large}. 
For super-resolution, the degradation model corresponds to the application of a gaussian low-pass filter followed by a $\times 4$ sub-sampling, and the addition of a gaussian white noise with $\sigma=3$.
For the deblurring, we considered motion blur and  gaussian kernels, both with a noise level $\sigma=8$. %

We provide quantitative comparisons in table~\ref{table:comp_ILO}, along with a visual comparison of the results in figure~\ref{fig:face_restoration}.
PnP-HVAE has the best  PSNR and SSIM results for all the considered restoration tasks, while ILO provides better results  for the perceptual distance.
By jointly optimizing the image and its latent variable, PnP-HVAE provides  results that are both realistic and consistent with the degraded observation.
On the other hand,  ILO  only optimizes on an extended latent space. This method generates  sharp and realistic images with better LPIPS scores,   
but the results lack  of consistency with respect to the observation, which explains the overall lower PSNR performance. 






\subsection{PatchVDVAE: a HVAE for natural images}\label{ssec:patchVDVAE}
Available generative models in the literature operate on images of  fixed resolutions and
are either restrained to datasets of limited diversity, or even to registered face images~\cite{kingma2018glow,child2021very, vahdat2020nvae, karras2019style}, or requiring additional class information~\cite{brock2018large, dhariwal2021diffusion, song2020score, luhman2022optimizing}.
Fitting an unconditional model on natural images appears to be a more difficult task, as their resolution can change, and their content is highly diverse.
The complexity of the problem can be reduced by learning a prior model on patches of reduced dimension. 
For image restoration problems, the patch model can be reused on images of higher dimensions~\cite{zoran2011learning,prost2021learning,altekruger2022patchnr}. When the model is a full CNN, the prior on the set of the  patches can  be computed efficiently by applying the network on the full image~\cite{prost2021learning}.

We thus introduce  patchVDVAE, a fully convolutional hierarchical VAE.
Contrary to existing HVAE models whose resolution is constrained by the constant tensor at the input of the top-down block, patchVDVAE can generate images of different resolutions by controlling the dimension of the input latent. 
This amounts to defining a prior on patches whose dimension corresponds to the receptive field of the VAE. A similar model is used for image denoising in~\cite{prakash2021interpretable}.

 
For PatchVDVAE architecture, we use the same bottom-up and top-down blocks as VDVAE~\cite{child2021very}, and replace the constant trainable input in the first top-down block by a latent variable, to make the model fully convolutional (details on the  architecture are given in Appendix~\ref{app:details}). 
The training dataset is composed of $128\times 128$ patches extracted from a combination of DIV2K~\cite{agustsson2017ntire} and Flickr2K~\cite{Lim_2017_CVPR_workshops} datasets.
We perform data augmentation by extracting  patches at $3$ resolutions: HR-images and $\times 2$ and $\times 4$ downscaled images. 
The model is trained for $7.10^5$ iterations with a batch size of $64$. Following the recommendation of~\cite{hazami2022efficient}, we use Adamax optimizer with an exponential moving average and gradient smoothing of the variance.
We set the decoder model to be a gaussian with diagonal covariance, as in~\cite{luhman2022optimizing}.
PatchVDVAE is fully convolutional and can generate images of dimension that are multiples of $64$ as illustrated by
figure~\ref{fig:vdvae}.

\newlength{\patchwidth}
\setlength{\patchwidth}{0.135\columnwidth}
\begin{figure}[!ht]
    \centering
    \begin{subfigure}[t]{.34\columnwidth}\hspace{0.1cm}
        \setlength{\tabcolsep}{0.02pt}
\renewcommand{\arraystretch}{0}
        \begin{tabular}{*{2}{p{1.03\patchwidth}}}
            \includegraphics[width=\patchwidth]{figures_arxiv/patchVDVAE/samples/generated/64x64/setup-5-image-0018.png} &
            \includegraphics[width=\patchwidth]{figures_arxiv/patchVDVAE/samples/generated/64x64/setup-5-image-0016.png} \\
            \includegraphics[width=\patchwidth]{figures_arxiv/patchVDVAE/samples/generated/64x64/setup-5-image-0008.png} &
            \includegraphics[width=\patchwidth]{figures_arxiv/patchVDVAE/samples/generated/64x64/setup-5-image-0019.png}   
        \end{tabular}
    \end{subfigure}\hspace{-0.15cm}
    \begin{subfigure}[t]{.64\columnwidth}
\begin{tabular}{cc}\vspace{-0.1cm}
\includegraphics[width=2\patchwidth]{figures_arxiv/patchVDVAE/samples/generated/256x256/setup-2-image-0009.png}&
        \includegraphics[width=2\patchwidth]{figures_arxiv/patchVDVAE/samples/generated/256x256/setup-2-image-0002.png}\end{tabular}

    \end{subfigure}
    \caption{\label{fig:vdvae} Left: $64\times64$ patches samples from our patchVDVAE model trained on patches from natural images.
    Right: PatchVDVAE is fully convolutional and it can generate images of higher resolution (here: $128\times128$).\vspace{-0.2cm}}
\end{figure}

\subsection{Natural images restoration}\label{ssec:app_nat}
We  evaluate PnP-HVAE on natural image restoration.
For each task, we report the average value of the PSNR, the SSIM, and the LPIPS metrics on $20$ images from the test set of the BSD dataset~\cite{MartinFTM01}.\\


\noindent
{\bf Image deblurring.}
In the experiments, we consider $2$ gaussian kernels and $2$ motion blur kernels from~\cite{levin2009understanding}, with $3$ different noise levels 
$\sigma \in \{2.55, 7.65, 12.75\}$.
As a baseline we consider  EPLL~\cite{zoran2011learning}, which learns a prior on image patches with a gaussian mixture model.
We also compare PnP-HVAE  with PnP-MMO and GS-PnP, $2$ competing convergent Plug-and-Play methods based on CNN denoisers.
PnP-MMO~\cite{pesquet2021learning} restricts the denoiser to be contraction in order to guarantee the convergence of the PnP forward-backard algorithm. GS-PnP~\cite{hurault2022gradient} considers a gradient step denoiser and reaches state-of-the-art performances of non converging methods~\cite{zhang2021plug}.
We set the temperature $\tau$  in our method as $0.95$, $0.8$ and $0.6$ for noise levels $2.55$, $7.65$ and $12.75$ respectively, and we let it run for a maximum of $50$ iterations. 
For the three compared methods we use the official implementations and pre-trained models provided by the respective authors. 
Details on the choice of hyperparameters for the concurrent methods are provided in the Appendix~\ref{app:details}
Figure~\ref{fig:deblurring_bsd} illustrates that our method provides correct deblurring results. 

According to table~\ref{tab:deb}, the performance of PnP-HVAE is between those of EPLL and GS-PnP and it outperforms PnP-MMO for large noise levels.\\

\begin{table}
\begin{center}\footnotesize
    \begin{tabular}{>{\centering}m{.3cm}*{5}{c}}
    $\sigma$ &Method & PSNR$\uparrow$ & SSIM$\uparrow$ & LPIPS$\downarrow$  \\ 
    \hline
    \multirow{4}{*}{\vcell{$2.55$}}
    & PnP-HVAE & $27.75$ & $0.79$ & $0.31$\\
    & GS-PNP \cite{hurault2022gradient} & $\mathbf{29.59}$ & $\mathbf{0.84}$ & $\mathbf{0.22}$\\
    & EPLL \cite{zoran2011learning} & $26.49$ & $0.71$ & $0.36$\\ 
    & PnP-MMO \cite{pesquet2021learning} & $\underbar{29.50}$ & $\underbar{0.83}$ & $\underbar{0.20}$ \\ \hline
    \multirow{4}{*}{\vcell{$7.65$}}
    & PnP-HVAE & $\underbar{26.36}$ & $\underbar{0.72}$ & $\underbar{0.40}$\\
    & GS-PNP \cite{hurault2022gradient} & $\mathbf{27.33}$ & $\mathbf{0.77}$ & $\mathbf{0.31}$\\
    & EPLL \cite{zoran2011learning} & $24.04$ & $0.66$ & $0.45$ \\ 
    & PnP-MMO \cite{pesquet2021learning} & $25.34$ & $0.69$ & $0.34$\\
    \hline
    \multirow{4}{*}{\vcell{$12.75$}}
    & PnP-HVAE & $\underbar{25.12}$ & $\mathbf{0.73}$ & $\underbar{0.47}$\\
    & GS-PNP \cite{hurault2022gradient} & $\mathbf{26.32}$ & $\mathbf{0.73}$ & $\mathbf{0.37}$\\
    & EPLL \cite{zoran2011learning} & $23.28$ & $0.61$ & $0.51$ \\ 
    & PnP-MMO \cite{pesquet2021learning} & $22.42$ & $0.53$& $0.54$ \\
    \hline
    &\vspace*{-.3cm}\\
            \multicolumn{2}{c}{Blur and motion kernels}& \multicolumn{3}{c}{
        \includegraphics*[scale=1]{figures_arxiv/kernels/4.png}\;\includegraphics*[scale=1]{figures_arxiv/kernels/7.png}\;\includegraphics*[scale=1]{figures_arxiv/kernels/9.png}\;\includegraphics*[scale=1]{figures_arxiv/kernels/11.png}} 
    \end{tabular}
        \caption{\label{tab:deb}Comparison  of PnP-HVAE  and other restoration methods on deblurring. Results are averaged on $4$ kernels.\vspace{-0.2cm}}% on image deblurring.}
    \end{center}
\end{table}

\begin{figure}
    
    \begin{subfigure}[h]{\linewidth}
        \centering
        \includegraphics*[width=\columnwidth]{figures_arxiv/deb_s255_k7.pdf}\vspace{-0.1cm}
        \caption{Gaussian blur, $\sigma=2.55$}
    \end{subfigure}
    \begin{subfigure}[h]{\linewidth}
        \centering
        \includegraphics*[width=\columnwidth]{figures_arxiv/deb_s765_k11.pdf}\vspace{-0.1cm}
        \caption{Motion blur, $\sigma=7.65$}
    \end{subfigure}\vspace*{-0.1cm}
    \caption{\label{fig:deblurring_bsd} Natural image deblurring\vspace{-0.1cm}}
\end{figure}

\noindent {\bf Effect of the temperature.}
PnP-HVAE gives control on the temperature of the prior over the latent space.
In figure~\ref{fig:temp_effect}, we illustrate that reducing the temperature increases the strength of the regularization prior. In this example the tuning $\tau=0.7$ produces the best performance.\\
\begin{figure}[!ht]
   
    \includegraphics[width=\columnwidth]{figures_arxiv/demo_temp.pdf}\vspace{-0.15cm}
    \caption{ \label{fig:temp_effect} Effect of the temperature in PnP-VAE on a deblurring problem, with $\sigma=7.65$.\vspace{-0.15cm}}
\end{figure}


\noindent
{\bf Image inpainting.}
Next we consider the task of noisy image inpainting. 
We compose a test-set of 10 images from the validation set of BSD~\cite{MartinFTM01} and we create masks
  by occluding diverse objects of small size in the images. 
A gaussian white noise with $\sigma=3$ is added to the images.
As a comparaison, we still consider GS-PnP and EPLL.
For PnP-HVAE, the temperature is set to $\tau=0.6$, and the algorithm is run for a maximum of $200$ iterations, unless the residual $||\x_{k+1}-\x_k||$ is on a plateau.
We provide on Table~\ref{tab:inpainting_bsd} the distortion metrics with the ground truth, as well as a visual
\begin{table}



\begin{center}
    \begin{tabular}{cccc}
        & PSNR$\uparrow$ & SSIM$\uparrow$ &LPIPS$\downarrow$ \\\hline
        PnP-HVAE  & $\mathbf{29.54}$ & $\mathbf{0.93}$ & $\mathbf{0.06}$\\
        GS-PNP & $28.52$ & $\mathbf{0.93}$ & $0.09$\\
        EPLL & $\underline{29.16}$ & $\mathbf{0.93}$ & $\mathbf{0.06}$\\
    \end{tabular}
    \caption{\label{tab:inpainting_bsd}Quantitative evaluation for inpainting on BSD.}
    \end{center}
\end{table}
comparison on figure~\ref{fig:inpainting_bsd}. 
With its hierarchical structure,  PnP-HVAE outperforms the compared methods. \vspace{0.05cm}



\begin{figure}[!h]
    \includegraphics[width=\columnwidth]{figures_arxiv/demo_inp_bsd2.pdf}\vspace{-0.1cm}
    \caption{\label{fig:inpainting_bsd}Natural image inpainting\vspace{-0.3cm}}
\end{figure}











\section{Conclusion}\label{sec:conclusion}
In this work, we focus on addressing the fundamental challenge of OOD detection tasks, which is how to fully understand the semantic discrepancy between the ID/OOD samples. We reveal that the key to success in the realistic SCOOD task is to allocate as many ID samples in the unlabeled set correctly as possible. To this end, we propose a novel uncertainty-aware optimal transport scheme that introduces class-specific energy scores as guidance for effective label assignment. Experimental results show that our method achieves better performance than previous state-of-the-art methods on SCOOD benchmarks.

\textbf{Limitations.} In addition to temperature scaling, other techniques such as feature clipping applied in ReAct~\cite{sun2021react} also enhance the performance of energy score, so how to obtain an OOD score that best fits the SCOOD task can be further explored. Moreover, a setting highly related to SCOOD has been proposed in \cite{katz2022training} and formulated as a constrained optimization problem. We will also theoretically analyze these practical OOD settings in our feature work.

% \section*{Acknowledgments}
\textbf{Acknowledgments.} 
This work is supported by National Key R\&D Program of China under Grant 2020AAA0105701, National Natural Science Foundation of China (NSFC) under Grants 61872327, Major Special Science and Technology Project of Anhui, National Natural Science Foundation of China (62033012) and Ant Group through Ant Research Intern Program.


%%%%%%%%% REFERENCES
{\small
\bibliographystyle{ieee_fullname}
\bibliography{egbib}
}

%\clearpage

\section{Appendix for Proofs}

\paragraph{Proof of Theorem \ref{thm:main}.}

\begin{proof}
\label{proof:main}
Our proof has two steps. In Step 1, we will show that SimCLR is equivalent to minimizing the cross entropy loss defined in Eqn.~(\ref{eqn:cross-entropy}). 
In Step 2, we will show  that minimizing the cross-entropy loss 
is equivalent to spectral clustering on $\bfpi$. 
Combining the two steps together, we have proved our theorem. 

\textbf{Step 1: } SimCLR is equivalent to minimizing the cross entropy loss.

The cross-entropy loss takes expectation over 
$\bfW_\bfX\sim \mathbb{P}(\cdot ; \bfpi)$, 
which means $\bfW_\bfX$ has exactly one non-zero entry in each row $i$. By Lemma~\ref{lem:multinomial}, we know every row $i$ of $\bfW_\bfX$ is independent of other rows. Moreover, 
$\bfW_{\bfX,i}\sim \mathcal{M}(1, \bfpi_i/\sum_j \bfpi_{i,j})=\mathcal{M}(1, \bfpi_i)$, because $\bfpi_i$ itself is a probability distribution.
Similarly, we know $\bfW_\bfZ$ also has the row-independent property by sampling over $\mathbb{P}(\cdot;\bfK_\bfZ)$.
Therefore, by Lemma~\ref{lem:cross_split}, we know Eqn.~(\ref{eqn:cross-entropy}) is equivalent to:
\[
 -\sum_{i=1}^n \mathbb{E}_{\bfW_{\bfX,i}}[\log \mathbb{P}(\bfW_{\bfZ,i}=\bfW_{\bfX,i};\bfK_\bfZ)],
\]

This expression takes expectation over $\bfW_{\bfX,i}$ for the given row $i$. Notice that 
$\bfW_{\bfX,i}$ has exactly one non-zero entry, which equals $1$ (same for $\bfW_{\bfZ,i}$). 
As a result
we expand the above expression to be:
\begin{equation}
 -\sum_{i=1}^n \sum_{j\neq i} \Pr(\bfW_{\bfX,i,j}=1)\log \Pr(\bfW_{\bfZ,i,j}=1).
\label{eqn:detailed-expansion}    
\end{equation}


By Lemma~\ref{lem:multinomial}, $\Pr(\bfW_{\bfZ,i,j}=1)=\bfK_{\bfZ,i,j}/\|\bfK_{\bfZ,i}\|_1$ for $j\neq i$. Recall that $\bfK_\bfZ=(k(\bfZ_i-\bfZ_j))_{(i,j)\in[n]^2}$, which means 
$\bfK_{\bfZ,i,j}/\|\bfK_{\bfZ,i}\|_1=\frac{\exp(-\|\bfZ_i-\bfZ_j\|^2/{2\tau})}{\sum_{k\neq i}
\exp(-\|\bfZ_i-\bfZ_k\|^2/{2\tau})
}$ for $j\neq i$, when $k$ is the Gaussian kernel with variance $\tau$. 

Notice that $\bfZ_i=f(\bfX_i)$, so we know
\begin{equation}
-\log \Pr(\bfW_{\bfZ,i,j}=1)=
-\log \frac{\exp(-\|f(\bfX_i)-f(\bfX_j)\|^2/{2\tau})}{\sum_{k\neq i}
\exp(-\|f(\bfX_i)-f(\bfX_k)\|^2/{2\tau}),
}
\label{eqn:infonce-equivalence}    
\end{equation}


The right hand side is exactly the InfoNCE loss defined in Eqn.~(\ref{eqn:infonce}).
Inserting Eqn.~(\ref{eqn:infonce-equivalence}) into Eqn.~(\ref{eqn:detailed-expansion}), we get the SimCLR algorithm, which first samples augmentation pairs $(i,j)$ with $\Pr(\bfW_{\bfX,i,j}=1)$ for each row $i$, and then optimize the InfoNCE loss. 

\textbf{Step 2: } minimizing the cross entropy loss 
is equivalent to spectral clustering on $\bfpi$.


By Lemma~\ref{lem:convert_to_spectral}, we may further convert the loss to 
\begin{equation}
\label{eqn:main-theorem-repul-attr}
\min_{\bfZ}
-\sum_{(i,j)\in [n]^2} \mathbf{P}_{i,j}
\log k (\bfZ_i-\bfZ_j)+\log \mathbf{R}(\bfZ).
\end{equation}
Since $k$ is the Gaussian kernel, this reduces to \[
\min_\bfZ \mathrm{tr}(\bfZ^\top \mathbf{L}(\bfpi) \bfZ)
+\log \mathbf{R}(\bfZ),
\]

where we use the fact that $\mathbb{E}_{\bfW_\bfX\sim \mathbb{P}(\cdot; \bfpi)}[\mathbf{L}(\bfW_\bfX)]
=\mathbf{L}(\bfpi)
$, because the Laplacian operator is linear and $
\mathbb{E}_{\bfW_\bfX\sim \mathbb{P}(\cdot; \bfpi)}(\bfW_\bfX)=\bfpi
$.
\end{proof}

\paragraph{Proof of Theorem \ref{thm:clip}.}
\begin{proof}
Since $\bfW_\bfX\sim \mathbb{P}(\cdot;\bfpi_{\mathbf{A}, \mathbf{B}})$, we know 
$\bfW_\bfX$ has exactly one non-zero entry in each row, denoting the pair that got sampled. 
A notable difference compared to the previous proof is we now have $n_\mathcal{A}+n_\mathcal{B}$ objects in our graph. CLIP deals with this by taking a mini-batch of size $2N$, 
such that $n_\mathcal{A}=n_\mathcal{B}=N$, and adding the $2N$ InfoNCE losses together. We label the objects in $\mathcal{A}$ as $[n_\mathcal{A}]$, and the objects in $\mathcal{B}$ as $\{n_\mathcal{A}+1, \cdots, n_\mathcal{A}+n_\mathcal{B}\}$. 

Notice that $\bfpi_{\mathbf{A}, \mathbf{B}}$ is a bipartite graph, so the edges of objects in $\mathcal{A}$ will only connect to object in $\mathcal{B}$ and vice versa. We can define the similarity matrix in $\cZ$ as $\bfK_\bfZ$, 
where $\bfK_\bfZ(i, j+n_\mathcal{A})=\bfK_\bfZ(j+n_\mathcal{A},i)= k(\bfZ_i-\bfZ_j)$ for $i\in [n_\mathcal{A}], j\in [n_\mathcal{B}]$, and otherwise we set $\bfK_\bfZ(i,j)=0$. 
The rest is same as the previous proof. 
\end{proof}

\paragraph{Proof of Theorem \ref{thm:exponential}.}

\begin{proof}
\label{proof:exponential}
Since the objective function consists of a linear term combined with an entropy regularization, which is a strongly concave function, the maximization problem is a convex optimization problem. Owing to the implicit constraints provided by the entropy function, the problem is equivalent to having only the equality constraint. We then introduce the Lagrangian multiplier $\lambda$ and obtain the following relaxed problem:

$$
\widetilde{E}(\boldsymbol{\alpha})=\psi_{1}-\sum_{i=1}^n \alpha_{i} \psi_{i}+\tau \sum_{i=1}^n \alpha_{i}\log \alpha_{i}+\lambda\left(\boldsymbol{\alpha}^{\top} \mathbf{1}_n-1\right).
$$

As the relaxed problem is unconstrained, taking the derivative with respect to $\alpha_{i}$ yields

$$
\frac{\partial \widetilde{E}(\boldsymbol{\alpha})}{\partial \alpha_{i}}=-\psi_{i}+\tau\left(\log \alpha_{i}+\alpha_{i} \frac{1}{\alpha_{i}}\right)+\lambda=0.
$$

Solving the above equation implies that $\alpha_{i}$ takes the form
$
\alpha_{i}=\exp \left(\frac{1}{\tau} \psi_{i}\right) \exp \left(\frac{-\lambda}{\tau}-1\right).
$ Since $\alpha_{i}$ lies on the probability simplex, the optimal $\alpha_{i}$ is explicitly given by
$
\alpha^{*}_{i}=\frac{\exp \left(\frac{1}{\tau} \psi_{i}\right)}{\sum_{i^{\prime}=1}^n \exp \left(\frac{1}{\tau} \psi_{i^{\prime}}\right)} .
$ Substituting the optimal point into the objective function, we obtain
$$
\begin{aligned}
E\left(\boldsymbol{\alpha}^*\right)  &=\psi_1-\sum_{i=1}^n \frac{\exp \left(\frac{1}{\tau} \psi_{i}\right)}{\sum_{i^{\prime}=1}^n \exp \left(\frac{1}{\tau} \psi_{i^{\prime}}\right)} \psi_{i}+\tau \sum_{i=1}^n \frac{\exp \left(\frac{1}{\tau} \psi_{i}\right)}{\sum_{i^{\prime}=1}^n \exp \left(\frac{1}{\tau} \psi_{i^{\prime}}\right)}\log \frac{\exp \left(\frac{1}{\tau} \psi_{i}\right)}{\sum_{i^{\prime}=1}^n \exp \left(\frac{1}{\tau} \psi_{i^{\prime}}\right)} \\
& =\psi_1 - \tau \log \left(\sum_{i=1}^n \exp \left(\frac{1}{\tau} \psi_{i}\right)\right).
\end{aligned}
$$
Thus, the Lagrangian dual function is given by
\begin{equation*}
-E\left(\boldsymbol{\alpha}^*\right)= -\tau \log \frac{\exp \left(\frac{1}{\tau} \psi_{1}\right)}{\sum_{i=1}^n \exp \left(\frac{1}{\tau} \psi_{i}\right)}.\qedhere
\end{equation*}
\end{proof}



\section{More on Experiments} \label{section: experiment_details}

\paragraph{CIFAR-10 and CIFAR-100} CIFAR-10 ~\citep{krizhevsky2009learning} and CIFAR-100 ~\citep{krizhevsky2009learning} are well-known classic image classification datasets. Both CIFAR-10 and CIFAR-100 contain a total of 60k $32 \times 32$ labeled images of different classes, with 50k for training and 10k for testing. CIFAR-10 is similar to CIFAR-100, except there are 10 different classes in CIFAR-10 and 100 classes in CIFAR-100.

\paragraph{TinyImageNet} TinyImageNet ~\citep{le2015tiny} is a subset of ImageNet ~\citep{deng2009imagenet}. There are 200 different object classes in TinyImageNet, with 500 training images, 50 validation images, and 50 test images for each class. All the images in TinyImageNet are colored and labeled with a size of $64 \times 64$.

\textbf{Pseudo-code.} Algorithm \ref{alg:Training Procedure} presents the pseudo-code for our empirical training procedure.

\begin{algorithm}[!htbp]
\caption{Training Procedure}
\label{alg:Training Procedure}
\begin{algorithmic}[1]
\REQUIRE trainable encoder network $f$, batch size $N$, augmentation strategy \textit{aug}, loss function $L$ with hyperparameters \textit{args}
\FOR {sampled minibatch ${x_i}_{i=1}^N$}
\FORALL{$i \in { 1, ..., N }$}
\STATE draw two augmentations $t_i = \textit{aug}\left(x_i\right) $, $t_i' = \textit{aug}\left(x_i\right) $
\STATE $z_i = f\left(t_i\right)$, $z_i' = f\left(t_i'\right)$
\ENDFOR
\STATE compute loss $\mathcal{L} = L(N, z, z', \textit{args})$
\STATE update encoder network $f$ to minimize $\mathcal{L}$
\ENDFOR
\STATE \textbf{Return} encoder network $f$
\end{algorithmic}
\end{algorithm}

We also provide the pseudo-code for our core loss function used in the training procedure in Algorithm \ref{alg:Core loss}. The pseudo-code is almost identical to SimCLR's loss function, with the exception of an extra parameter $\gamma$.

\begin{algorithm}[!htbp]
\caption{Core loss function $\mathcal{C}$}
\label{alg:Core loss}
\begin{algorithmic}[1]
\REQUIRE batch size $N$, two encoded minibatches $z_1, z_2$, $\gamma$, temperature $\tau$
\STATE $z = \textit{concat}\left(z_1, z_2\right)$
\FOR {$i \in {1, ..., 2N }, j \in {1, ..., 2N}$ }
\STATE $s_{i,j} = \Vert z_i - z_j \Vert_2^{\gamma}$
\ENDFOR
\STATE \textbf{define} $l(i, j)$ \textbf{as} $l(i, j) = - \log \frac{exp\left(s_{i,j}/\tau \right)}{\sum_{k=1}^{2N} \mathbf{1}{[k \ne i]} exp\left(s{i, j} / \tau \right)} $
\STATE \textbf{Return} $\frac{1}{2N} \sum_{k=1}^N\left[l(i, i+N) + l(i+N, i)\right]$
\end{algorithmic}
\end{algorithm}

Utilizing the core loss function $\mathcal{C}$, we can define all kernel loss functions used in our experiments in Table \ref{table: loss definition}. For all $z_i \in z$ with even dimensions $n$, we define $z_{L_i} = z_i\left[0:n/2\right]$ and $z_{R_i} = z_i\left[n/2:n\right]$.

\begin{table}[ht]
\centering
\begin{tabular}{{@{}l|l@{}}}
Kernel  &  Loss function \\ \midrule
Laplacian & $\mathcal{C}\left(N, z, z', \gamma=1, \tau\right)$\\ \midrule
Sum       & $\lambda * \mathcal{C}\left(N, z, z', \gamma=1, \tau_1\right) + (1-\lambda) * \mathcal{C}\left(N, z, z', \gamma=2, \tau_2\right)$  \\ \midrule
Concatenation Sum&$\lambda * \mathcal{C}\left(N, z_L, z'_L, \gamma=1, \tau_1\right) + (1-\lambda) * \mathcal{C}\left(N, z_R, z'_R, \gamma=2, \tau_2\right)$\\ \midrule
$\gamma = 0.5$ & $\mathcal{C}\left(N, z, z', \gamma=0.5, \tau\right)$          \\ 

\end{tabular}

\caption{Definition of kernel loss functions in our experiments}
\label {table: loss definition}
\end{table}

\textbf{Baselines.} We reproduce the SimCLR algorithm using PyTorch Lightning~\citep{PytorchLightning}.

\textbf{Encoder details.}
The encoder $f$ consists of a backbone network and a projection network. We employ ResNet50~\citep{ResNet} as the backbone and a 2-layer MLP (connected by a batch normalization~\citep{ioffe2015batch} layer and a ReLU \cite{nair2010rectified} layer) with hidden dimensions 2048 and output dimensions 128 (or 256 in the concatenation kernel case).

\textbf{Encoder hyperparameter tuning.}
For each encoder training case, we randomly sample 500 hyperparameter groups (sample details are shown in Table \ref{table: Hyperparameter sample}) and train these samples simultaneously using Ray Tune ~\citep{RayTune}, with the ASHA scheduler~\citep{li2018massively}. Ultimately, the hyperparameter group that maximizes the online validation accuracy (integrated in PyTorch Lightning) within 5000 validation steps is chosen for the given encoder training case.

\begin{table}[ht]
\centering

\begin{tabular}{@{}l|l|l@{}}
\midrule
Hyperparameter  & Sample Range & Sample Strategy \\ \midrule
start learning rate & $\left[10^{-2}, 10\right]$ & log uniform \\ \midrule
$\lambda$       & $\left[0, 1\right]$ & uniform \\ \midrule
$\tau$, $\tau_1$, $\tau_2$ & $\left[0, 1\right]$ & log uniform \\ \midrule
\end{tabular}

\caption{Hyperparameters sample strategy}
\label {table: Hyperparameter sample}
\end{table}

\textbf{Encoder training.} 
We train each encoder using the LARS optimizer~\citep{LARSOptimizer}, LambdaLR Scheduler in PyTorch, momentum 0.9, weight decay $10^{-6}$, batch size 256, and the aforementioned hyperparameters for 400 epochs on a single A-100 GPU.

\textbf{Image transformation.} The image transformation strategy, including augmentation, is identical to the default transformation strategy provided by PyTorch Lightning.

\textbf{Linear evaluation.}
The linear head is trained using the SGD optimizer with a cosine learning rate scheduler, batch size 64, and weight decay $10^{-6}$ for 100 epochs. The learning rate starts at $0.3$ and ends at $0$.

\textbf{Moco Experiments.} We also tested our method based on MoCo~\citep{he2019moco}. The results are summarized in Table \ref{tab:results-moco}. Here we choose ResNet18~\citep{ResNet} as the backbone and set a temperature of $0.1$ as default. For our simple sum kernel, we set $\lambda=0.8$. The results show that our method outperforms the original MoCo method.

\begin{table}[thb]
\centering
\caption{MoCo Experiment Results on CIFAR-10 and CIFAR-100.}
\label{tab:results-moco}
\resizebox{\textwidth}{!}{%
\begin{tabular}{@{}c|ccc|ccc@{}}
\toprule
\multirow{3}{*}{Method} & \multicolumn{3}{c|}{CIFAR-10} & \multicolumn{3}{c}{CIFAR-100} \\ \cmidrule(lr){2-4} \cmidrule(lr){5-7} 
                        & 200 epochs & 400 epochs    & 1000 epochs   & 200 epochs & 400 epochs & 1000 epochs         \\ \midrule
MoCo (repro.)         & $76.41 \pm 0.12$    & $80.01 \pm 0.15$          & $84.45 \pm 0.08$    & $\mathbf{47.02 \pm 0.11}$ & $52.50 \pm 0.07$ & $57.62 \pm 0.15$            \\
\midrule
Laplacian Kernel        & ${78.09 \pm 0.10}$    & $\mathbf{83.85 \pm 0.09}$          & $\mathbf{88.34 \pm 0.16}$    & $46.12 \pm 0.22$   & $53.44 \pm 0.17$ & $59.10 \pm 0.14$        \\
Simple Sum Kernel & $\mathbf{78.12 \pm 0.15}$   & $83.23 \pm 0.18$ & $87.50 \pm 0.20$ & $46.65 \pm 0.06$ & $\mathbf{53.62 \pm 0.19}$ & $\mathbf{59.83 \pm 0.12}$\\
\bottomrule
\end{tabular}
}
\end{table}



\section{More Experiments on Synthetic Data}


Consider a scenario with $n$ clusters, each containing $k$ vertices. Let the probability of vertices $u$ and $v$ from the same cluster belonging to $\bfpi$ be $p$. Conversely, for vertices $u$ and $v$ from different clusters, let the probability of belonging to $\pi$ be $q$. We generate the graph $\bfpi$ randomly, based on $p$ and $q$. We experiment with values of $k=100$ and $n=6$ for ease of visualization, embedding all points in a two-dimensional space. Each vertex's initial position originates from a normal distribution. In each iteration, we sample a subgraph of $\bfpi$ uniformly, ensuring each vertex has an out-degree of $1$. We then optimize the corresponding vectors using InfoNCE loss with an SGD optimizer and iterate until convergence. Our experimental setup consists of an SGD learning rate of $1$, an InfoNCE loss temperature of $0.5$, and a batch size of $50$. We evaluate two scenarios with different $p$ and $q$ values: $p=1$, $q=0$, and $p=0.75$, $q=0.2$. The results of these experiments are visualized in Figure \ref{fig:vis-spectral-cluster}. The obtained embeddings exhibit the hallmark pattern of spectral clustering of graph $\bfpi$.

\begin{figure}[!tb]
\centering
\subfigure{
\includegraphics[width=1\textwidth]{Figures/cluster_pi.png}
\label{fig:vis-cluster}
}
\subfigure{
\includegraphics[width=1\textwidth]{Figures/noised_cluster_pi.png}
\label{fig:vis-noised-cluster}
}
\caption{Visualizations of the optimization process using InfoNCE Loss on the vectors corresponding to $\bfpi$. Points of identical color belong to the same cluster within $\bfpi$. To showcase the internal structure of $\bfpi$, we randomly select 10 vertices from each cluster to display the edge distribution of $\bfpi$.}
\label{fig:vis-spectral-cluster}
\end{figure}



\end{document}


\title[Division rings for virtually compact special groups]{Division rings for group algebras of virtually compact special groups}
\author{Sam P.~ Fisher}
\author{Pablo S\'anchez-Peralta}
\date{March 2023}

\begin{document}

\maketitle


\begin{abstract}
    Let $k * G$ be a crossed product of a division ring $k$ and a torsion-free virtually compact special group $G$. We embed $k * G$ into a division ring $\mathcal D$ and use this to confirm a recent conjecture of Kielak and Linton. In the case where $G$ is locally indicable, we prove that $\mathcal D$ is Hughes-free.
    
    If $H$ is a torsion-free one-relator group, let $\overline{kH}$ be the division ring containing $kH$ constructed by Lewin and Lewin. We prove that $\overline{kH}$ is Hughes-free whenever a Hughes-free $kH$-division ring exists. This is always the case when $k$ is of characteristic zero; in positive characteristic, our previous result implies this happens when $H$ is virtually compact special.
\end{abstract}


\section{Introduction}


A group $G$ is said to satisfy the \textit{strong Atiyah conjecture} (over $\Q$) if for every free proper cocompact $G$-CW complex $X$, the $L^2$-Betti numbers satisfy $\mathrm{lcm}(G) \cdot b_n^{(2)}(X;G) \in \Z$, where $\mathrm{lcm}(G)$ denotes the least common multiple of the orders of finite subgroups of $G$. We take the statement to be vacuous when $\mathrm{lcm}(G) = \infty$. The strong Atiyah conjecture is open in general, but it has been established for many classes of groups, including residually (torsion-free solvable) groups \cite{SchickIntegrality}, virtually compact special groups \cite{Schreve_AtiyahVCS}, and locally indicable groups \cite{JaikinLopez_Atiyah}.

In the case where $G$ is torsion-free, the strong Atiyah conjecture over $\Q$ is equivalent to the existence of a division ring $\mathcal D_{\Q G}$ such that $\Q G \subseteq \mathcal D_{\Q G} \subseteq \mathcal U(G)$, where $\mathcal U(G)$ is the algebra of operators affiliated to the von Neumann algebra $\mathcal N(G)$. In this case, we always take $\mathcal D_{\Q G}$ to be the division closure of $\Q G$ inside $\mathcal U(G)$, and we call this $\mathcal D_{\Q G}$ the \textit{Linnell skew-field} of $G$. The $L^2$-Betti numbers of $G$ can be computed as $b_n^{(2)}(G) = \dim_{\mathcal D_{\Q G}} H_n(G; \mathcal D_{\Q G})$ (see \cite[Lemma 10.39]{Luck02}). The definitions of $\mathcal N(G)$ and $\mathcal U(G)$ will not be important in this paper, what we want to emphasize is that the Atiyah conjecture is equivalent to the embedding of $\Q G$ in some division ring and that this division ring can be used to compute the $L^2$-Betti numbers of the group.

This point of view on $L^2$-Betti numbers was generalised by Henneke and Kielak in \cite{HennekeKielak_agrarian}, where they consider arbitrary embeddings of integral group rings $\Z G$ into a division ring $\mathcal D$, called agrarian embeddings, and study the agrarian homology $H_\bullet(G, \mathcal D)$ and the resulting agrarian Betti numbers $b_n^\mathcal D(G) := \dim_{\mathcal D} H_n(G; \mathcal D)$. In \cite{JaikinZapirain2020THEUO}, Jaikin-Zapirain showed that many groups $G$ have the property that for all division rings $k$, the group algebra $kG$ embeds into a Hughes-free division ring $\mathcal D_{kG}$ (see \cref{sec:prelims}) and when $k = \Q$ we have $\mathcal D_{kG} = \mathcal D_{\Q G}$. Putting $k = \mathbb F_p$ for a prime $p$ and taking $\mathcal D_{\mathbb F_pG}$-agrarian homology, this leads us to natural mod $p$ analogues of $L^2$-Betti numbers. For instance, in \cite{Fisher_improved} the Betti numbers $b_n^{\mathcal D_{kG}}$ were shown to have many analogous properties to those of $L^2$-Betti numbers, in particular in how they control finiteness properties of kernels of algebraic fibrations and in \cite{FisherHughesLeary_artin} and \cite{AOS_hyperbolization} they were related to the mod $p$ homology growth of $G$.

Very recently, Kielak and Linton proved the following embedding result for virtually compact special hyperbolic groups.
%
\begin{thm}[{\cite[Theorem 1.10]{KielakLinton_FbyZ}}]\label{thm:KLmain}
    Let $H$ be hyperbolic and virtually compact special with $\cd_\Q(H) \geqslant 2$. Then, there exists a finite-index subgroup $L \leqslant H$ and a map of short exact sequences
    \[
        \begin{tikzcd}
            1 \arrow[r] & K \arrow[d, hook] \arrow[r] & L \arrow[d, hook] \arrow[r] & \Z \arrow[d, Rightarrow, no head] \arrow[r] & 1 \\
            1 \arrow[r] & N \arrow[r]                 & G \arrow[r]                 & \Z \arrow[r]                                & 1
        \end{tikzcd}
    \]
    such that
    \begin{enumerate}[label = (\roman*)]
        \item $G$ is hyperbolic, compact special, and contains $L$ as a quasi-convex subgroup.
        \item $\cd_\Q(G) = \cd_\Q(H)$.
        \item $N$ is finitely generated.
        \item If $b_p^{(2)}(H) = 0$ for all $2 \leqslant p \leqslant n$, then $N$ is of type $\FP_n(\Q)$.
        \item If $b_p^{(2)}(H) = 0$ for all $p \geqslant 2$, then $\cd_\Q(N) = \cd_\Q(H) - 1$.
    \end{enumerate}
\end{thm}
%
As a consequence of this result, they are able to show among other things that one-relator groups with torsion are virtually free-by-cyclic \cite[Corollary 1.3]{KielakLinton_FbyZ}, confirming a conjecture of Baumslag. Much of Kielak and Linton's paper is written in the full generality of agrarian homology. However, to prove \cref{thm:KLmain}, they need to restrict themselves to $L^2$-homology as they make crucial use of Schreve's result that virtually compact special groups satisfy the Atiyah conjecture \cite{Schreve_AtiyahVCS}. As a consequence, for a torsion-free virtually compact special group $G$, Kielak and Linton can embed the group algebra $\Q G$ into its Linnell-skew field $\mathcal D_{\Q G}$ and use it in homological arguments to prove \cref{thm:KLmain}. At the end of their paper, Kielak and Linton conjecture \cite[Conjecture 6.7]{KielakLinton_FbyZ} that \cref{thm:KLmain} remains true when every instance of $\Q$ is replaced with an arbitrary field $k$ and every $L^2$-Betti number $b_i^{(2)}$ is replaced with the agrarian Betti number $b_i^{\mathcal D_{kG}}$.

The main obstacle to proving \cite[Conjecture 6.7]{KielakLinton_FbyZ} was not having access to a division ring embedding the group algebra $kG$ of a torsion-free virtually compact special group $G$. Our first main result remedies this, building on work of Linnell--Schick \cite{LinnellSchick_AtiyahExt} and Schreve \cite{Schreve_AtiyahVCS}.

\begin{thm}[{\cref{cor:VCSinField}}]\label{thm:VCSinField_intro}
    If $G$ is a torsion-free virtually compact special group and $k$ is a division ring, then any crossed product $k * G$ embeds into a division ring $\mathcal D$. Moreover, if $G$ is locally indicable, then $\mathcal D$ is Hughes-free as a $k * G$-division ring.
\end{thm}

The authors are not aware of the following immediate corollary appearing anywhere in the literature.

\begin{cor}
    If $G$ is a torsion-free virtually compact special group and $k$ is a division ring, then any crossed product $k * G$ satisfies Kaplansky's zero divisor conjecture.
\end{cor}

Using the embedding $kG \hookrightarrow \mathcal D$ given by \cref{thm:VCSinField_intro}, where $G$ is torsion-free virtually compact special, we can reprove \cref{thm:KLmain} over arbitrary fields, thus confirming \cite[Conjecture 6.7]{KielakLinton_FbyZ}.

\begin{thm}[{\cref{thm:KLmain_agr}}]
    Let $k$ be any division ring. Then \cref{thm:KLmain} remains true when every instance of $\Q$ is replaced by $k$ and every $L^2$-Betti number $b_i^{(2)}$ is replaced by the agrarian Betti number $b_i^{\mathcal D_{kG}}$.
\end{thm}

Moreover, Brodski\u{\i} showed that torsion-free one-relator groups are locally indicable \cite{BrodskiiOR}, so \cref{thm:VCSinField_intro} implies the next corollary.

\begin{cor}
    For every virtually compact special one-relator group $G$ and every division ring $k$, any crossed product $k * G$ embeds into a Hughes-free division ring $\mathcal D_{k * G}$.
\end{cor}

In \cite[Theorem 8.2]{Linton_ORH}, Linton showed that every one-relator group $G$ with negative immersions is virtually compact special, and in \cite[Theorem 1.3]{LouderWilton_NegativeImmersions} Louder and Wilton characterise the one-relator groups with negative immersions as those with a presentation $G = F/\llangle R \rrangle$, where $R$ is a cyclically reduced word of primitivity rank $> 2$ (see \cref{def:PR}) in the free group $F$. Since $R$ being of primitivity rank $1$ implies that $G$ has torsion, the only case where we don't know whether $kG$ embeds into a Hughes-free division ring is when $R$ has primitivity rank $2$. Note that in characteristic zero, every torsion-free one-relator group embeds into a Hughes-free division ring \cite[Corollary 1.4]{JaikinLopez_Atiyah}.

Recently, Jaikin-Zapirain showed that if $k$ is a field of characteristic zero and $G$ is a torsion-free one-relator group, then the group algebra $kG$ is coherent (here \textit{coherent} means that every finitely generated ideal is finitely presented) \cite[Theorem 1.1]{Jaikin_oneRelCoherence}. Building on this breakthrough, Linton proved that one-relator groups are coherent \cite{Linton_oneRelCoherent}, confirming a conjecture of Baumslag \cite{Baumslag_OneRelProblems}. In Jaikin-Zapirain's result, the only reason one must assume that $k$ is of characteristic zero is that in this case a Hughes-free division ring $\mathcal D_{kG}$ is known to exist. Thus, combining our construction of $\mathcal D_{kG}$ for $\operatorname{char} k > 0$ and Jaikin-Zapirain's arguments, we obtain the following.

\begin{cor}[\cref{cor:coherence}]
    Let $G$ be a virtually compact special one-relator group. Then the group ring $kG$ is coherent for any division ring $k$.
\end{cor}

By the uniqueness of Hughes-free division rings, we also conclude that every one-relator group with negative immersions embeds into a unique Linnell ring, as defined in \cite[Section 2.3]{Jaikin_oneRelCoherence}. This implies the next corollary.

\begin{cor}
    \cite[Conjecture 1]{Jaikin_oneRelCoherence} holds for virtually compact special locally indicable groups. In particular, it holds for virtually compact special one-relator groups.
\end{cor}

In \cite{LewinLewinORTF}, Lewin and Lewin showed that every group algebra $kG$ of a one-relator group embeds into a division ring $\overline{kG}$. It is thus natural to compare our construction of Hughes-free division rings to the Lewin--Lewin construction. Indeed, in the final section we prove the following general result.

\begin{thm}[{\cref{thm:LL_HF}}]
    Let $G$ be a torsion-free one-relator group and let $k$ be a division ring such that there exists a Hughes-free division ring $\mathcal D_{kG}$. Then $\overline{kG} \cong \mathcal D_{kG}$ as $kG$-division rings.
\end{thm}

As a consequence, the Lewin--Lewin construction $\overline{kG}$ is Hughes-free for all $k$ when $G$ has negative immersions, and it is always Hughes-free when $k$ is of characteristic $0$.


\subsection*{Organisation of the paper}

In \cref{sec:prelims} we recall some preliminaries, including compact special groups and Hughes-free division rings. In \cref{sec:divRings} we prove that group algebras of torsion-free virtually compact special groups embed into a division ring and in \cref{sec:KLconj} we use this to confirm \cite[Conjecture 6.7]{KielakLinton_FbyZ}. Finally, we prove that the Lewin--Lewin construction is Hughes-free whenever a Hughes-free division ring exists in \cref{sec:LL}.



\subsection*{Acknowledgments} 

The authors are grateful to Andrei Jaikin-Zapirain for many helpful conversations. The first author is supported by the National Science and Engineering Research Council (NSERC) [ref.~no.~567804-2022] and the European Research Council (ERC) under the European Union's Horizon 2020 research and innovation programme (Grant agreement No. 850930). The second author is supported by PID2020-114032GB-I00 of the Ministry of Science and Innovation of Spain.




\section{Preliminaries}\label{sec:prelims}

\begin{conv}
    Rings are associative and unital, and ring homomorphisms preserve the unit. If $G$ is a group we will write $e$ for the trivial element.
\end{conv}

\subsection{Special groups}

Special cube complexes were introduced by Haglund and Wise in \cite{HaglundWise_special} as a class of cube complexes that avoid certain pathological hyperplane arrangements. One of the central features of a compact special cube complex $X$ is that it admits a $\pi_1$-injective combinatorial local isometry $X \looparrowright S_\Gamma$ to the Salvetti complex $S_\Gamma$ of the finitely generated right-angled Artin group (RAAG) $A_\Gamma$ (the graph $\Gamma$ is the \textit{hyperplane graph} of $X$) \cite[Theorem 1.1]{HaglundWise_special}. Thus, fundamental groups of compact special cube complexes inherit many remarkable algebraic properties from RAAGs; for example they are subgroups of $\SL_n(\Z)$, and in particular are residually finite.

Throughout the article, we will refer to groups that are fundamental groups of compact special cube complexes as \textit{compact special groups}. Agol's Theorem \cite[Theorem 1.1]{AgolHaken} states that a hyperbolic group $G$ acting properly and cocompactly on a $\mathrm{CAT}(0)$ cube complex $X$ contains a subgroup $H$ of finite index such that $X/H$ is compact special. In this sense, virtually compact special groups are abundant.


\subsection{Hughes-free division rings}

Let $k$ be a division ring and let $G$ be a group. 

A ring $S$ is $G$-{\it graded} if $S=\bigoplus_{g\in G} S_g$ where $S_g$ is an additive subgroup for every $g\in G$, and $S_gS_h\subseteq S_{gh}$ for all $g,h\in G$. If $S_g$ contains an invertible element $u_g$ for each $g\in G$, then we say that $S$ is a \textit{crossed product} of $S_e$ and $G$ and we shall denote it by $S=S_e * G$. Note that the usual group ring $RG$ of a group $G$ with $R$ a ring is an example of a crossed product.

An $R$-{\it ring} is a pair $(S,\varphi)$ where $\varphi \colon R\rightarrow S$ is a homomorphism. We will often omit $\varphi$ if it is clear from the context. An $R$-division ring $\varphi\colon R\rightarrow \mathcal{D}$ is called {\it epic} if $\varphi(R)$ generates $\mathcal{D}$ as a division ring.

\begin{defn}
    Let $G$ be a locally indicable group and let $k$ be a division ring. We say that a $k * G$-division ring $\varphi \colon k * G \rightarrow \mathcal D$ is \textit{Hughes-free} if it is epic and the following condition is satisfied: for every non-trivial finitely generated subgroup $H \leqslant G$ and every non-trivial map $f \colon H \rightarrow \Z$, the set $\{\varphi(t)^i : i \in \Z\}$ is linearly independent over the division closure of $k[\ker f]$ in $\mathcal D$, where $t \in H$ is any element such that $f(t) = 1$. In this situation, we will also say that $\mathcal D$ is a \textit{Hughes-free division ring of fractions} of $k * G$.
\end{defn}

\begin{rem}
    Note that a Hughes-free map $\varphi \colon k * G \rightarrow \mathcal D$ is always an embedding. Indeed, Gr\"ater showed that Hughes-free embeddings are \textit{strongly} Hughes-free \cite[Corollary 8.3]{Grater20}. This means that for every $H \leqslant G$ finitely generated, every $N \trianglelefteqslant H$, and any transversal $T$ of $N$ in $H$, the set $\varphi(T)$ is linearly independent over the division closure of $k * N$ in $\mathcal D$. In particular, the set $\varphi(G)$ is linearly independent over $\varphi(k) \subseteq \mathcal D$, which implies that $\varphi$ is injective.
\end{rem}

Ian Hughes showed that when Hughes-free division rings exist, they are unique up to $k * G$-isomorphism \cite{HughesDivRings1970}. Thus, when it exists, we will always denote the Hughes-free division ring of $k * G$ by $\mathcal D_{k * G}$. If $H$ is a subgroup of $G$, note that the division closure of $k * H$ in $\mathcal D_{k*G}$ is Hughes-free as a $k*H$-division ring, and therefore we have a natural inclusion $\mathcal D_{k*H} \subseteq \mathcal D_{k*G}$. Since RAAGs are residually (torsion-free nilpotent) (\cite{Droms_thesis} and \cite{DuchampKrob_RAAGsRTFN}), so are their subgroups, and in particular so are compact special groups. Thus, Hughes-free division rings exist for compact special groups by the following result of Jaikin-Zapirain.

\begin{thm}[{\cite[Theorem 1.1]{JaikinZapirain2020THEUO}}]
    Let $G$ be a locally indicable amenable group, a residually (torsion-free nilpotent) group, or a free-by-cyclic group. Then $\mathcal D_{k * G}$ exists and it is universal.
\end{thm}

See \cref{subsec:slyv} for a definition of universality. We now prove two general Lemmas about Hughes-free division rings which will be useful to us in the later sections.

\begin{lem}\label{lem:twisted_ext}
    Let $G$ be a group and let $H \trianglelefteqslant G$ be a locally indicable normal subgroup. If $k * H$ has a Hughes-free division ring embedding $\varphi \colon k*H \hookrightarrow \mathcal D_{k*H}$, then we can form $\mathcal D_{k*H} * [G/H]$ and there is a natural embedding $k*G \cong (k*H) * [G/H] \hookrightarrow \mathcal D_{k*H} * [G/H]$.
\end{lem}

\begin{proof}
    The only potential obstruction to extending the crossed product structure of $(k*H) * [G/H]$ to $\mathcal D_{k*H} * [G/H]$ is extending the conjugation action of $G$ on $H$ to a $G$-action on all of $\mathcal D_{k*H}$. This is not a problem, however, because Hughes-free division rings are unique up to $k*H$-isomorphism. In more detail, let $\alpha \colon H \rightarrow H$ be any automorphism of $H$, and by abuse of notation write $\alpha$ for the induced automorphism of $k*H$. Then $\varphi$ and $\varphi \circ \alpha$ are both Hughes-free embeddings of $k * H$, and by uniqueness of Hughes-free embeddings, $\alpha$ extends to an automorphism $\alpha' \colon \mathcal D_{k*H} \rightarrow \mathcal D_{k*H}$ such that the diagram
    \[
        \begin{tikzcd}
            k*H \arrow[d, "\varphi", hook] \arrow[r, "\alpha", hook] & k*H \arrow[d, "\varphi", hook] \\
            \mathcal D_{k*H} \arrow[r, "\alpha'", hook] & \mathcal D_{k*H}              
        \end{tikzcd}
    \]
    commutes. \qedhere
\end{proof}

\begin{lem}\label{lem:HF_fi}
    Let $G$ be a locally indicable group and let $H \trianglelefteqslant G$ be a subgroup of finite index. If $\mathcal{D}_{k*H}$ is a Hughes-free division $k*H$-ring of fractions and $\mathcal D_{k*H} * [G/H]$ is a division ring, then $\mathcal D_{k*H} * [G/H]$ is a Hughes-free $k*G$-division ring of fractions.
\end{lem}

\begin{proof}
    Let $L$ be a finitely generated subgroup of $G$ and $N \trianglelefteqslant L$ such that $L/N=\langle tN \rangle \cong \Z$. Since $H\cap N\trianglelefteqslant L$, according to Lemma \ref{lem:twisted_ext} we can form $\mathcal{D}_{k*[H\cap N]} * [L/(H\cap N)]$. Moreover, we have the following natural isomorphism of crossed products
    \begin{align}\label{cross_prod_isom}
        \mathcal{D}_{k*[H\cap N]} * [L/(H\cap N)]&\cong (\mathcal{D}_{k*[H\cap N]} * [N/(H\cap N)]) * [L/N]
    \end{align}
    where $\mathcal{D}_{k*[H\cap N]} * [N/(H\cap N)]$ is obtained by similar considerations. 
    
    Now observe that since the automorphism of $k*[H\cap N]$ used in the construction of the crossed product ring is just a restriction of the conjugation automorphism of $k*H$, we have a natural embedding $\mathcal{D}_{k*[H\cap N]} * [N/(H\cap N)]\hookrightarrow \mathcal D_{k*H} * [G/H]$. Thus $\mathcal{D}_{k*[H\cap N]} * [N/(H\cap N)]$ is a domain because so is $\mathcal D_{k*H} * [G/H]$. Moreover, $|N:H\cap N| \leq |G:H|$ which is finite; so $\mathcal{D}_{k*[H\cap N]} * [N/(H\cap N)]$ is actually a division ring into which $kN$ embeds. 
    
    Let $\mathcal{D}_N$ and $\mathcal{D}_L$ be the division closures of $k*N$ and $k*L$ in $\mathcal{D}_{k*H} * [G/H]$, respectively. Then $\mathcal{D}_{k*[H\cap N]} * [N/(H\cap N)]\cong \mathcal{D}_{N}$ and by (\ref{cross_prod_isom}) the subring of $\mathcal{D}_L$ generated by $\mathcal{D}_N, t$ and $t^{-1}$ has a crossed product ring structure. This shows the $\mathcal{D}_{N}$-independence of the elements $\{t^i| i\in \Z\}$. \qedhere
\end{proof}





\subsection{Agrarian homology}

Let $R$ be a ring, let $G$ be a group and let $\mathcal D$ be a division ring. If the group algebra $RG$ embeds into $\mathcal D$, then we say that $G$ is \textit{$\mathcal D$-agrarian over $R$} and that the embedding $RG \hookrightarrow \mathcal D$ is an \textit{agrarian embedding}. A nice immediate consequence of having an agrarian embedding is that $RG$ satisfies Kaplansky's conjecture on zero-divisors. Note that, as far as the authors are aware, there is not a single example of a torsion-free group and a division ring $k$ such that the group algebra $kG$ does \textit{not} have an agrarian embedding.

Suppose that $G$ is $\mathcal D$-agrarian over $R$. Then $\mathcal D$ is an $RG$-bimodule and we can define the \textit{$\mathcal D$-homology and cohomology} of $G$ by
\[
    H_\bullet (G; \mathcal D) \quad \text{and} \quad H^\bullet (G; \mathcal D)
\]
and the \textit{$\mathcal D$-Betti numbers} by
\[
    b_p^\mathcal D(G) = \dim_{\mathcal D} H_p(G; \mathcal D) \quad \text{and} \quad b_\mathcal D^p(G) = \dim_\mathcal D H^p (G; \mathcal D).
\]
The theory of homological agrarian Betti numbers was introduced by Henneke--Kielak in \cite{HennekeKielak_agrarian} in the case $R = \Z$ and was studied over other fields $R$ in the case where $\mathcal D$ is Hughes-free in \cite{Fisher_improved}. However, in this article we will be mostly concerned with agrarian cohomology. Thanks to \cite[Lemma 2.2]{KielakLinton_FbyZ}, $b_p^\mathcal D(G) = b^p_\mathcal D(G)$ whenever these quantities are finite (which occurs if, for instance, $G$ is of type $\F_\infty$) and therefore we do not need to worry about the distinction between cohomological and homological $\mathcal D$-Betti numbers.

The following is the central example of agrarian homology.

\begin{ex}
    Let $G$ be a torsion-free group satisfying the strong Atiyah conjecture. Then $\Q G$ embeds into a division ring $\mathcal D_{\Q G}$ called the \textit{Linnell skew-field} (\cite{LinnellDivRings93} and \cite[Lemma 10.39]{Luck02}) and the agrarian Betti numbers $b_p^{\mathcal D_{\Q G}}(G)$ are equal to the $L^2$-Betti numbers $b_p^{(2)}(G)$. 
\end{ex}


\begin{prop}\label{prop:props_agr}
    Let $G$ be a group and let $k$ be a division ring such that $kG$ embeds into a division ring $\mathcal D$. Then
    \begin{enumerate}[label = (\roman*)]
        \item if $G$ is non-trivial, then $b_0^\mathcal D(G) = 0$;
        \item if $G$ is the fundamental group of a compact aspherical complex $X$, then $\chi(G) = \sum_{i = 0}^\infty b_i^\mathcal D(G)$;
        \item if $\mathcal D = \mathcal D_{kG}$ is Hughes-free, then for every finite index subgroup $H \leqslant G$ we have $|G:H| \cdot b_p^{\mathcal D_{kG}}(G) = b_p^{\mathcal D_{kH}} (H)$ for all $p$.
    \end{enumerate}
\end{prop}

\begin{proof}
    (i) This follows from considering the partial free resolution 
    \[
        \bigoplus_{g \in G} kG \xrightarrow{\bigoplus_{g \in G} (g - 1)} kG \xrightarrow{\alpha} k \rightarrow 0
    \]
    of the trivial $kG$-module $k$, and tensoring with $\mathcal D$ over $kG$. Note that $\alpha$ is the augmentation map.

    (ii) This is proved as usual, i.e.~ if $C_\bullet(\widetilde X; k)$ is the CW chain complex of $\widetilde X$ with coefficients in $k$, then we use the fact that $\dim_\mathcal D \mathcal D \otimes_{kG} C_n(\widetilde X)$ is the number of $n$-cells in $X$.

    (iii) This is \cite[Lemma 6.3]{Fisher_improved}. \qedhere
\end{proof}



\subsection{Sylvester matrix rank functions}\label{subsec:slyv}

Let $R$ be a ring. A \textit{Sylvester matrix rank function} $\rk$ on $R$ is a function that assigns a non-negative real number to each matrix over $R$ and satisfies the following conditions:
\begin{enumerate}
    \item $\rk(A)=0$ if $A$ is any zero matrix and $\rk(1)=1$;
    \item $\rk(AB)\leq \min\{\rk(A),\rk(B)\}$ for any matrices $A$ and $B$ which can be multiplied;
    \item $\rk\left(\begin{array}{cc}
         A & 0 \\
         0 & B
    \end{array}\right)=\rk(A)+\rk(B)$ for any matrices $A$ and $B$;
    \item $\rk\left(\begin{array}{cc}
         A & C \\
         0 & B
    \end{array}\right)\geq \rk(A)+\rk(B)$ for any matrices $A,B$ and $C$ of appropriate sizes.
\end{enumerate}

We denote by $\mathbb{P}(R)$ the set of Sylvester matrix rank functions on $R$. Note that a ring homomorphism $\varphi\colon R\rightarrow S$ induces a map $\varphi^{\#}\colon \mathbb{P}(S)\rightarrow \mathbb{P}(R)$, that is, we can pull back any rank function $\rk$ on $S$ to a rank function $\varphi^{\#}(\rk)$ on $R$ by just setting
\[
\varphi^{\#}(\rk)(A):=\rk(\varphi(A))
\]
for every matrix $A$ over $R$. We shall often abuse the notation and write $\rk$ instead of $\varphi^{\#}(\rk)$ when it is clear that we speak about the rank function on $R$.

A division ring $\mathcal{D}$ has a unique Sylvester matrix rank function which we denote by $\rk_{\mathcal{D}}$. Any Sylvester matrix rank function $\rk$ on $R$ that only takes integer values comes from a division ring by a result of P.~ Malcolmson \cite{Malcolmson}. Furthermore, there is a one-to-one correspondence between integer-valued rank functions and epic $R$-division rings.

\begin{lem}[{\cite[Corollary 3.1.15]{DLopezAlvarezThesis}}]\label{lem:equal_rk}
    Let $R$ be a ring, let $\mathcal{D}$, $\mathcal{E}$ be two epic $R$-division rings. Then $\mathcal{D}$ and $\mathcal{E}$ are $R$-isomorphic if and only if for every matrix $A$ over $R$ the induced rank functions on $R$ satisfy
    \[
        \rk_{\mathcal{D}}(A)=\rk_{\mathcal{E}}(A).
    \]
\end{lem}

We denote the set of integer-valued rank functions on a ring $R$ by $\mathbb{P}_{div}(R)$.

Given two Sylvester matrix rank functions on $R$, $\rk_1$ and $\rk_2$, we will write $\rk_1\leq \rk_2$ if for every matrix $A$ over $R$, $\rk_1(A)\leq \rk_2(A)$. This partial order structure shall play a key role.

A central notion in this theory is that of a universal $R$-division ring for a given ring $R$ (see, for instance, \cite[Section 7.2]{cohn06FIR}). In the language of Sylvester matrix rank functions, an epic $R$-division ring $\mathcal{D}$ is \textit{universal} if for every $R$-division ring $\mathcal{E}$, $\rk_{\mathcal{D}} \geqslant \rk_{\mathcal{E}}$. Note that the universal epic $R$-division ring, if it exists, is unique up to $R$-isomorphism and we denote it by $U(R)$.  





\section{Division rings for finite extensions of groups with the factorisation property} \label{sec:divRings}

The following definition was introduced by Schreve in \cite{Schreve_AtiyahVCS}, where he used it to show that virtually compact special groups satisfy the strong Atiyah conjecture. This is a strengthening of the \textit{enough torsion-free quotients} property introduced by Linnell and Schick in \cite{LinnellSchick_AtiyahExt}, which they used to study when the Atiyah conjecture passes from a subgroup to a finite index overgroup.

\begin{defn}
    A group $G$ has the \textit{factorisation property} if every homomorphism from $G$ to a finite group factors through a torsion-free elementary amenable quotient.
\end{defn}

The same proof as in \cite[Lemma 4.52]{LinnellSchick_AtiyahExt} works for the next result.

\begin{lem}\label{lem:split_crit}
    Let $Q$ be a finite group in the exact sequence
    \[
        1\rightarrow H\rightarrow G\xrightarrow{\pi} Q\rightarrow 1.
    \]
    Assume that among all normal finite index subgroups of $G$, there is a cofinal system $U_i \trianglelefteqslant G$ with $U_i\subseteq H$, such that for each $i$, the homomorphism $\pi_i$ in 
    \[
        G\xrightarrow{p_i}G/U_i\xrightarrow{\pi_i}Q
    \]
    has a splitting $s_i\colon Q\rightarrow G/U_i$. Then the profinite completion map $\widehat{\pi}\colon \widehat{G}\rightarrow Q$ has a splitting $Q\rightarrow \widehat{G}$.
\end{lem}

For the reader's convenience, we also record the following statement.

\begin{lem}[{\cite[Lemma 3.5]{Schreve_AtiyahVCS}}]\label{lem:collection}
    Suppose $H$ is finitely generated and has the factorisation property. Then there exists a collection $\mathcal{U}$ of subgroups $U\trianglelefteqslant H$ such that every finite index subgroup of $H$ contains a subgroup in $\mathcal{U}$, if $U\in \mathcal{U}$ then $H/U$ is torsion-free and elementary amenable, and if $U,V\in \mathcal{U}$, then $U\cap V\in \mathcal{U}$.
\end{lem}

If $\mathcal U$ is a collection of subgroups of $H \leqslant G$, we shall denote by $\mathcal{U}^G$ the collection $\{U^G\}_{U\in \mathcal{U}}$ where $U^G=\bigcap_{g\in G}gUg^{-1}$.


We restate \cite[Lemma 3.6]{Schreve_AtiyahVCS} with a slight variation.

\begin{lem}\label{lem:split}
    Let $Q$ be a finite group and let $1\rightarrow H\rightarrow G\xrightarrow{\pi} Q\rightarrow 1$ be an exact sequence of groups. Assume $H$ is finitely generated and has the factorisation property. Let $\mathcal{U}$ be a collection of normal subgroups of $H$ as in Lemma \ref{lem:collection}, and define $\mathcal{U}^G$ as above. If each $G/U^G$ has torsion, then there is a non-trivial subgroup $Q_0\leqslant Q$ splitting back to $\widehat{G}_0 \leqslant \widehat{G}$, where $G_0=\pi^{-1}(Q_0)$.
\end{lem}

\begin{proof}
    Since $H/U^G$ is torsion-free, the image of the torsion subgroup of $G/U^G$ in $Q$ has the same order. Denote by $a_U$ the finite collection of non-trivial subgroups of $Q$ which are images of finite subgroups of $G/U^G$ for some $U\in \mathcal{U}$.
    
    Note that if $U,V\in\mathcal{U}$, then each finite subgroup of $G/(U^G\cap V^G)$ maps isomorphically to a finite subgroup of $G/U^G$, since $G/(U^G\cap V^G)$ surjects onto $G/U^G$ with kernel contained in $H/(U\cap V)^G$ which is torsion-free, and similarly for $G/V^G$. In particular, $a_U\cap a_V$ contains $a_{U\cap V}$ and hence it is non-empty. However, $Q$ is finite; thus there is $Q_0\in \bigcap_{U\in \mathcal{U}}a_U$.
    
    This way $Q_0$ is a non-trivial finite group splitting back to $G/U^G$ for all $U^G \in \mathcal{U}^G$. Thus, $Q_0$ splits to the quotient $G/J$ for all $U^G\trianglelefteqslant J\trianglelefteqslant H$. But each finite index $J\trianglelefteqslant H$ contains some $U^G\in\mathcal{U}^G$ by \cref{lem:collection}; so $Q_0$ splits to each finite quotient $G/K$. Therefore, according to Lemma \ref{lem:split_crit}, $Q_0$ splits to $\widehat{G}_0$. \qedhere
\end{proof}

Recall that a group $G$ is \textit{good} (in the sense of Serre) if the restriction 
\[
    H_c^\bullet(\widehat G; M) \rightarrow H^\bullet (G;M)
\]
is an isomorphism for every finite $G$-module $M$, where $\widehat G$ denotes the profinite completion of $G$, and the cohomology on the left is continuous cohomology (see, for example, \cite{Serre_GaloisCohom}).

\begin{thm}\label{thm:USUBGP}
    Let $H$ be a finitely generated good group with the factorisation property and of finite cohomological dimension, and let $1 \rightarrow H \rightarrow G \rightarrow Q \rightarrow 1$ be a group extension with $G$ torsion-free and $Q$ finite. Then there is a normal subgroup $U \trianglelefteqslant G$ such that $U \leqslant H$ and $G/U$ is torsion-free and elementary amenable.
\end{thm}

\begin{proof}
    Since $G$ is torsion-free and $H$ is a finite index subgroup, according to \cite[Th\'eor\`eme 1, Section 1.7]{Serre_CohomologyFI} $\cd(H)=\cd(G)$. Moreover, goodness passes through finite index by \cite[Lemma 3.2]{GrunewaldJaikin-ZapirainZalesskii_goodness}. Hence $\cd(\widehat{G})$ is finite; and so $\widehat{G}$ is torsion-free. Therefore, Lemma \ref{lem:split} ends the proof. Indeed, if all quotients $G/U^G$ have torsion, then the splitting of Lemma \ref{lem:split} contradicts the fact that $\widehat{G}$ is torsion-free. Note that $G/U^G$ is elementary amenable due to the exact sequence $1\rightarrow H/U^G\rightarrow G/U^G\rightarrow Q\rightarrow 1$. \qedhere
\end{proof}







\begin{thm}\label{thm:VCS_field}
    Let $H$ be a finitely generated good group with the factorisation property and of finite cohomological dimension, and let $1 \rightarrow H \rightarrow G \rightarrow Q \rightarrow 1$ be a group extension with $G$ torsion-free and $Q$ finite. If $k$ is a division ring such that there is a Hughes-free embedding of $k * H$ into $\mathcal D_{k * H}$, then $k * G$ embeds into a division ring.
\end{thm}

\begin{proof}
    By \cref{thm:USUBGP}, there is a normal subgroup $U \trianglelefteqslant G$ such that $U \leqslant H$ and $G/U$ is torsion-free and elementary amenable. By \cref{lem:twisted_ext}, we can form each of the following rings:
    \[
        \mathcal D_{k * U} * [H/U], \quad \mathcal D_{k * U} * [G/U], \quad \mathcal D_{k * H} * [G/H].
    \]
    Since $H/U$ and $G/U$ are elementary amenable, according to \cite[Lemma 2.5]{LinnellSchick_AtiyahExt} $\mathcal D_{k * U} * [H/U]$ and $\mathcal D_{k * U} * [G/U]$ are Ore domains and the diagram
    \[
        \begin{tikzcd}
            {\mathcal D_{k * U} * [H/U]} \arrow[r, hook] \arrow[d, hook] & {\mathcal D_{k * U} * [G/U]} \arrow[d, hook] \\
            {\Ore(\mathcal D_{k * U} * [H/U])} \arrow[r, hook] &{\Ore(\mathcal D_{k * U} * [G/U])}          
        \end{tikzcd}
    \]
    commutes. By Hughes-freeness of $\mathcal D_{k * H}$, the map $\mathcal D_{k * U} * [H/U] \rightarrow \mathcal D_{k * H}$ is an injection. This implies that $\Ore(\mathcal D_{k * U} * [H/U]) \cong \mathcal D_{k * H}$ by the universal property of Ore localisation.

    Consider the following diagram
    \[
    \begin{tikzcd}[column sep = small]
        \mathcal D_{k * H} \arrow[r, "\cong", no head] \arrow[d, hook] & {\Ore(\mathcal D_{k * U} * [H/U])} \arrow[r, hook] \arrow[d, hook] & {\Ore(\mathcal D_{k * U} * [G/U])} \arrow[d, "\cong", no head] \\
        {\mathcal D_{k * H} *[G/H]} \arrow[r, "\cong", no head]        & {\Ore(\mathcal D_{k * U} * [H/U]) * [G/H]} \arrow[r, "\cong", no head]        & {\Ore((\mathcal D_{k * U}*[H/U])*[G/H])} \nospacepunct{.}                    
    \end{tikzcd}
    \]
    The left two vertical maps are the obvious inclusions and the right vertical map is a standard isomorphism of crossed products. The two left isomorphisms come from the isomorphism $\Ore(\mathcal D_{k * U} * [H/U]) \cong \mathcal D_{k * H}$ discussed above. For the bottom right isomorphism, it is not hard to show that the natural map
    \[
        \Ore(\mathcal D_{k * U} * [H/U]) * [G/H] \rightarrow \Ore((\mathcal D_{k * U}*[H/U])*[G/H]),
    \]
    is injective. Therefore $\Ore(\mathcal D_{k * U} * [H/U]) * [G/H]$ a domain, which implies it is a division ring since $G/H$ is finite. This proves that $\mathcal D_{k * H} * [G/H]$ is a division ring, which clearly contains $k * G$. \qedhere
\end{proof}



\begin{cor}\label{cor:VCSinField}
    If $G$ is a torsion-free virtually compact special group, then any crossed product $k * G$ embeds into a division ring $\mathcal D$. Moreover, if $H$ is a normal, finite-index, compact special subgroup of $G$, then the diagram
    \[
        \begin{tikzcd}
            k * H \arrow[r, hook] \arrow[d, hook] & {k * H * [G/H]} \arrow[d, hook] \arrow[r, "\cong", no head] & k * G \arrow[d, hook] \\
            \mathcal D_{k * H} \arrow[r, hook]    & {\mathcal D_{k * H} * [G/H]} \arrow[r, "\cong", no head]    & \mathcal D        
        \end{tikzcd}
    \]
    commutes. Moreover, if $G$ is locally indicable, then $\mathcal D = \mathcal D_{k * G}$.
\end{cor}

\begin{proof}
    Compact special groups are good and have the factorisation property \cite[Corollary 4.3]{Schreve_AtiyahVCS}. Moreover, compact special groups are finitely generated and have finite cohomological dimension, since they have finite classifying space. The claim then follows from \cref{thm:VCS_field}. The second claim follows from the proof of \cref{thm:VCS_field}. The final statement follows from \cref{lem:HF_fi}. \qedhere
\end{proof}

\begin{cor}
    Let $G$ be a torsion-free virtually compact special group and let $k$ be a division ring. Then the Kaplansky zero divisor conjecture holds for any crossed product $k * G$. \qedhere
\end{cor}

In \cite[Theorem 1.1]{Jaikin_oneRelCoherence}, Jaikin-Zapirain proves that $kG$ is coherent whenever $k$ is of characteristic $0$ and $G$ is a torsion-free one-relator group. The only time the assumption that $k$ is of characteristic $0$ is used is to ensure that a Hughes-free division ring $\mathcal D_{kG}$ exists. From Jaikin-Zapirain's arguments and the existence of $\mathcal D_{kG}$ we prove coherence of more one-relator group algebras.

\begin{cor}\label{cor:coherence}
    Let $G$ be a virtually compact special torsion-free one-relator group and let $k$ be any division ring. Then the group algebra $kG$ is coherent.
\end{cor}

Thanks to the work of Linton \cite[Theorem 8.2]{Linton_ORH}, we know that all one-relator groups with negative immersions are virtually compact special. Moreover, Louder and Wilton proved that $G$ has negative immersions if and only if the defining word of $G$ has primitivity rank at least $3$ (see \cref{sec:LL} for more details). In this sense, most one-relator groups are virtually compact special.





\section{The Kielak--Linton conjecture} \label{sec:KLconj}

Let $G$ be a torsion-free virtually compact special group. Since $G$ may not be locally indicable, a Hughes-free division ring $\mathcal D_{kG}$ may not exist. However, by \cref{prop:props_agr}, we may simply define $b_p^{\mathcal D_{kG}}(G) = \frac{1}{|G:H|} b_p^{\mathcal D_{kH}}(H)$, where $H \leqslant G$ is a compact special subgroup of finite-index. We record the following easy observation that will be useful later.

\begin{lem}\label{lem:eitherField}
    Let $k$ be a division ring and let $G$ be a torsion-free virtually compact special group and let $\mathcal D$ be the division ring containing $kG$ constructed in \cref{thm:VCS_field}. Then $b_p^\mathcal D(G) = b_p^{\mathcal D_{kG}}(G)$ for all $p$.
\end{lem}

\begin{proof}
    Let $H \trianglelefteqslant G$ be a compact special subgroup of finite index. The claim then follows quickly from the definition above and the fact that $\mathcal D \cong \mathcal D_{kH} * [G/H] \cong \bigoplus_{|G:H|}\mathcal D_{kH}$ as $\mathcal D_{kH}$-modules. \qedhere
\end{proof}


\begin{lem}\label{lem:pair_of_free}
    Let $G$ be a non-elementary hyperbolic group, let $k$ be a division ring, and suppose there is an embedding of $kG$ into a division ring $\mathcal D$. Then there exists non-isomorphic quasi-convex free subgroups $A,B \leqslant G$ such that $A$ is malnormal, $A \cap B^g = \{1\}$ for all $g \in G$, and the restriction
    \[
        H^1(G; \mathcal D) \rightarrow H^1(B; \mathcal D)
    \]
    is an isomorphism.
\end{lem}

\begin{proof}
    This is \cite[Corollary 5.7]{KielakLinton_FbyZ}, and the proof is similar: it only relies on an ``agrarian Freiheitssatz" which they prove for arbitrary agrarian embeddings \cite[Theorem 3.1]{KielakLinton_FbyZ}. \qedhere
\end{proof}

\begin{lem}\label{lem:agr_ind}
    Let $G = A*_C$ where $A$ and $C$ are locally indicable and finitely generated. Moreover, suppose that $k$ is such that $\mathcal D_{kA}$ exists and $kG$ embeds into a division ring $\mathcal D \supseteq \mathcal D_{kA}$ making the diagram
    \[
        \begin{tikzcd}
            kA \arrow[r, hook] \arrow[d, hook] & kG \arrow[d, hook] \\
            \mathcal D_{kA} \arrow[r, hook]    & \mathcal D
        \end{tikzcd}
    \]
    commute. If the restriction $H^1(A; \mathcal D_{kA}) \rightarrow H^1(C; \mathcal D_{kA})$ is surjective, then the restriction
    \[
        H^2(G; \mathcal D) \rightarrow H^2(A; \mathcal D)
    \]
    is injective.
\end{lem}

\begin{proof}
    Let $H \leqslant G$ and write $\mathcal D(H)$ for the division closure of $kH$ in $\mathcal D$. Then the proof is the same as in \cite[Proposition 4.8]{KielakLinton_FbyZ}, where one must replace every occurrence of $\mathcal D_{\Q H}$ with $\mathcal D(H)$. \qedhere
\end{proof}

The following proposition corresponds to Proposition 6.4 of \cite{KielakLinton_FbyZ}, where the result is proven in the case $k = \Q$. The obstacle Kielak and Linton faced in their paper was the fact that they didn't have access to division rings containing group rings of torsion-free virtually compact special groups.

\begin{prop}\label{prop:HNN}
    Let $H$ be non-free, torsion-free, hyperbolic, and compact special, and suppose that $b_1^{\mathcal D_{kH}}(H) \neq 0$. Then there is a hyperbolic and virtually compact special HNN extension $G = H*_F$ such that the embeddings of $F$ are quasi-convex in $G$ and such that
    \[
        b_p^{\mathcal D_{kG}}(G) = \begin{cases}
            0 & \text{if} \ p = 1 \\
            b_p^{\mathcal D_{kH}}(H) & \text{if} \ p \neq 1.
        \end{cases}
    \]
    Moreover, $H$ is quasi-convex in $G$ and $\cd_k(G) = \cd_k(H)$.
\end{prop}

\begin{proof}
    By \cref{lem:pair_of_free}, there is a pair of isomorphic free quasi-convex subgroups $A,B \leqslant H$ such that $A$ is malnormal and intersects every conjugate of $B$ trivially and the restriction 
    \[
        H^1(H; \mathcal D_{kH}) \rightarrow H^1(B; \mathcal D_{kH})
    \]
    is an isomorphism.
    
    Now let $f \colon A \rightarrow B$ be any isomorphism and let $G = H*_A$ be the corresponding HNN extension. By \cite[Theorem 6.3]{KielakLinton_FbyZ}, $G$ is virtually compact special. Moreover, since $G$ is the HNN extension of a torsion-free group, it is also torsion-free, and therefore $kG$ embeds in a division ring $\mathcal D$ by \cref{cor:VCSinField}. From \cite[Theorem 3.1]{BieriHNN}, there is a long exact sequence
    \[
        \cdots \rightarrow H^p(G; \mathcal D) \rightarrow H^p(H; \mathcal D) \rightarrow H^p(A; \mathcal D) \rightarrow H^{p+1}(G; \mathcal D) \rightarrow \cdots.
    \]
    Since $A$ is free, $H^p(A; \mathcal D) = 0$ for $p \geqslant 2$, and this immediately implies that $b_p^{\mathcal D_{kG}}(G) = b_p^{\mathcal D_{kH}}(H)$ for $p \geqslant 3$, where we have used \cref{lem:eitherField}.

    The interesting portion of the long exact sequence is then
    \[
        0 \rightarrow H^1(G; \mathcal D) \rightarrow H^1(H; \mathcal D) \rightarrow H^1(A; \mathcal D) \rightarrow H^2(G; \mathcal D) \rightarrow H^2(H; \mathcal D) \rightarrow 0,
    \]
    whence we obtain the equation
    \begin{align*}
        0 &= b_1^{\mathcal D_{kG}}(G) - b_1^{\mathcal D_{kH}}(H) + b_1^{\mathcal D_{kA}}(A) - b_2^{\mathcal D_{kG}}(G) + b_2^{\mathcal D_{kH}}(H) \\
        &= b_1^{\mathcal D_{kG}}(G) - b_2^{\mathcal D_{kG}}(G) + b_2^{\mathcal D_{kH}}(H).
    \end{align*}
    But $b_2^{\mathcal D_{kG}}(G) \leqslant b_2^{\mathcal D_{kH}}(H)$ by \cref{lem:agr_ind}, so we must have $b_1^{\mathcal D_{kG}}(G) = 0$ and $b_2^{\mathcal D_{kG}}(G) = b_2^{\mathcal D_{kH}}(H)$.

    The claim about cohomological dimensions follows exactly as in the proof of \cite[Proposition 6.4]{KielakLinton_FbyZ}.
\end{proof}

As a consequence, we can reprove \cite[Theorem 1.10]{KielakLinton_FbyZ} over arbitrary fields, thus confirming \cite[Conjecture 6.7]{KielakLinton_FbyZ}.

\begin{thm}\label{thm:KLmain_agr}
    Let $k$ be a division ring and let $H$ be hyperbolic, virtually compact special, and suppose that $\cd_k(H) \geqslant 2$. Then, there exists a finite-index subgroup $L \leqslant H$ and a map of short exact sequences
    \[
        \begin{tikzcd}
            1 \arrow[r] & K \arrow[d, hook] \arrow[r] & L \arrow[d, hook] \arrow[r] & \Z \arrow[d, Rightarrow, no head] \arrow[r] & 1 \\
            1 \arrow[r] & N \arrow[r]                 & G \arrow[r]                 & \Z \arrow[r]                                & 1
        \end{tikzcd}
    \]
    such that
    \begin{enumerate}[label = (\roman*)]
        \item $G$ is hyperbolic, compact special, and contains $L$ as a quasi-convex subgroup.
        \item $\cd_k(G) = \cd_k(H)$.
        \item $N$ is finitely generated.
        \item If $b_p^{\mathcal D_{kH}}(H) = 0$ for all $2 \leqslant p \leqslant n$, then $N$ is of type $\FP_n(k)$.
        \item If $b_p^{\mathcal D_{kH}}(H) = 0$ for all $p \geqslant 2$, then $\cd_k(N) = \cd_k(H) - 1$.
    \end{enumerate}
\end{thm}


\begin{proof}
    Now that we have established \cref{prop:HNN} (Kielak and Linton prove the $L^2$ case in \cite[Proposition 6.4]{KielakLinton_FbyZ}), the proof is very similar to the one in \cite{KielakLinton_FbyZ}; consequently, we only show which parts of the $L^2$-theory are used in the proof and provide references for the corresponding statements in the agrarian setting.

    First, Kielak and Linton use the fact that an infinite group has vanishing $0$th $L^2$-Betti number; this is also true in the agrarian setting by \cref{prop:props_agr}(i). Next, they use the fact that $L^2$-Betti numbers scale with the index when passing to finite index subgroups, which is true for agrarian Betti numbers with Hughes-free coefficients by \cref{prop:props_agr}(iii). Finally, they quote \cite[Theorem A]{Fisher_improved} (see \cite[Theorem 6.1]{KielakLinton_FbyZ}), which relates the $L^2$-Betti numbers of compact special groups (and more generally of RFRS groups) to virtual fibring with kernels of type $\FP_n(\Q)$. Luckily, the analogous result holds with agrarian Betti numbers and finiteness properties over arbitrary fields \cite[Theorem B]{Fisher_improved}. \qedhere
\end{proof}


\section{Lewin--Lewin Division Ring and the Hughes-free Condition} \label{sec:LL}

In \cite{LewinLewinORTF} J.~ Lewin and T.~ Lewin showed that for every division ring $k$ and every torsion-free one-relator group $G$, the group algebra $kG$ can be embedded in a division ring, which we denote by $\overline{kG}$ following their notation. They pointed out that they did not know whether $\overline{kG}$ is a universal $kG$-division ring of fractions. However, what they already knew is that if there were a universal $kG$-division ring of fractions, then it would be $\overline{kG}$. In this section we give evidence in this direction by showing that the Lewin--Lewin division ring is Hughes-free for most torsion-free one-relator groups (the meaning of ``most" will be made precise below).


Key algebraic structures in this argument of Lewin--Lewin are firs and semifirs. A non-zero ring $R$ is a {\it free ideal ring} or {\it fir} if every left and every right ideal is a free $R$-module of unique rank. {\it Semifirs} are defined similarly, the only difference being that only ask that finitely generated left and right ideals are free of unique rank (this concept turns out to be left-right symmetric, unlike that of fir). We will make heavy use of the following two powerful theorems of Cohn.

\begin{thm}[\cite{Cohnfreeproducts}]\label{lem:2semifir_semifir}
    The coproduct of a family of semifirs over a common sub-division ring is again a semifir.
\end{thm}

\begin{thm}[{\cite[Corollary 7.5.14]{cohn06FIR}}]
    Every semifir has a universal division ring of fractions.
\end{thm}

We also record two useful results for future reference.

\begin{lem}[\cite{Cohnfreeprodskewfields}]\label{lem:univ_equality}
    Let $R_1,R_2$ be semifirs with a common division subring $\mathcal{D}$. Then $U(R_1 *_{\mathcal{D}} R_2)\cong U(R_1 *_{\mathcal{D}} U(R_2))$.
\end{lem}

\begin{prop}[{\cite[Theorem 8.1]{JaikinLopez_Atiyah}}]\label{prop:rk_max_HF}
    Let $G$ be a locally indicable group $G$ and assume there exists a Hughes-free $kG$-division ring $\mathcal{D}_{kG}$. Then the Sylvester matrix rank function $\rk_{\mathcal{D}_{kG}}$ is maximal in $\mathbb{P}_{div}(kG).$
\end{prop}

\begin{rem}
    Note that \cref{prop:rk_max_HF} does not imply that Hughes-free division rings are necessarily universal, since different rank functions may not be comparable.
\end{rem}

We now prove two general lemmas about Hughes-free division rings which will be used right afterwards.

\begin{lem}\label{lem:HF_amalg}
    Let $G$ be a locally indicable group which is the amalgam of $G_1$ and $G_2$ over $H$. If $\mathcal{D}_{kG}$ exists, then
    \[
        U(\mathcal{D}_{kG_1} *_{\mathcal{D}_{kH}} \mathcal{D}_{kG_2})\cong \mathcal{D}_{kG}
    \]
    as $kG$-rings.
\end{lem}

\begin{proof}
    First note that by uniqueness of Hughes-free division rings, the coproduct $\mathcal{D}_{kG_1} *_{\mathcal{D}_{kH}} \mathcal{D}_{kG_2}$ is well-defined. Moreover, it is a semifir and thus has a universal division ring. On the other hand, we have natural embeddings of $\mathcal{D}_{kG_1}$, $\mathcal{D}_{kH}$, $\mathcal{D}_{kG_2}$ into $\mathcal{D}_{kG}$. Thus by universal property of coproducts we have the following commutative diagram
    \[
        \begin{tikzcd}
            kG \arrow[r, hook] \arrow[d, hook] & {\mathcal{D}_{kG_1} *_{\mathcal{D}_{kH}} \mathcal{D}_{kG_2}} \arrow[d, hook] \arrow[dl]\\
            \mathcal{D}_{kG}   & U(\mathcal{D}_{kG_1} *_{\mathcal{D}_{kH}} \mathcal{D}_{kG_2}) \nospacepunct{.}  
        \end{tikzcd}
    \]
    Let $\rk_{U(G)}$ and $\rk_G$ denote the Sylvester matrix rank functions on $\mathcal D_{kG_1} *_{\mathcal D_{kH}} \mathcal D_{kG_2}$ corresponding to $U(\mathcal{D}_{kG_1} *_{\mathcal{D}_{kH}} \mathcal{D}_{kG_2})$ and $\mathcal{D}_{kG}$, respectively. Then $\rk_{U(G)}\geq \rk_{G}$ by universality. 
    
    On the other hand, $\mathcal{D}_{kG}$ is Hughes-free and hence $\rk_{G}$ is a maximal rank function on $kG$. Thus, $\rk_{U(G)} = \rk_{G}$ on $kG$ by \cref{prop:rk_max_HF}. Since $kG$ embeds in both division rings epically, they are $kG$-isomorphic by \cref{lem:equal_rk}. \qedhere
\end{proof}



\begin{lem}\label{lem:HF_cyclic}
    Let $G$ be a locally indicable group and $N\trianglelefteqslant G$ be a normal subgroup such that $G/N=\langle tN \rangle \cong \Z$. If $\mathcal{D}_{kG}$ is a Hughes-free $kG$-division ring of fractions, then 
    $$\mathcal{D}_{kG}\cong \Ore(\mathcal{D}_{kN}[t^{\pm 1};\tau])$$
    as $kG$-rings.
\end{lem}

\begin{proof}
    First note that since $\mathcal{D}_{kG}$ is a Hughes-free division ring, the integers powers of $t$ are $\mathcal{D}_{kN}$-linearly independent. Hence, if we set $S$ for the subring of $\mathcal{D}_{kG}$ generated by $\mathcal{D}_{kN}, t$ and $t^{-1}$, we get an isomorphism $S\cong \mathcal{D}_N[t^{\pm 1};\tau]$ where $\tau$ denotes the twisted action of $t$ on $\mathcal D_{kN}$ induced by conjugation. Note that $S$ is an Ore domain. Thus by universal property of localization we have the following commutative diagram
    \[
        \begin{tikzcd}
             \Ore(\mathcal{D}_N[t^{\pm 1};\tau]) \arrow[rr, hook] &&  \mathcal{D}_{kG}\\
            &kG \arrow[ul, hook'] \arrow[ur, hook]   &    
        \end{tikzcd}
    \]
    Finally, since both embeddings are epic, we conclude that $\mathcal{D}_{kG}$ and $\Ore(\mathcal{D}_N[t^{\pm 1};\tau])$ are $kG$-isomorphic.\qedhere
\end{proof}

We are ready to show the main theorem of this section.

\begin{thm}\label{thm:LL_HF}
    Let $G$ be a torsion-free one-relator group and let $k$ be a division ring such that there exists the Hughes-free $kG$-division ring of fractions $\mathcal{D}_{kG}$. Then 
    $$\overline{kG}\cong \mathcal{D}_{kG}$$
    as $kG$-rings.
\end{thm}

\begin{proof}
    The plan of the proof is to follow the Lewin--Lewin construction of $\overline{kG}$ and use our assumption that $\mathcal D_{kG}$ exists to prove that $\overline{kG} \cong \mathcal D_{kG}$. Crucially, we recall Brodski\u{\i}'s result that torsion-free one-relator groups are locally indicable \cite{BrodskiiOR}. Let $G=\langle a_1, a_2,\ldots \mid R \rangle$ be a torsion-free one-relator group, where $R$ is a cyclically reduced word in the generators $a_i$. 
    
    We argue, as in \cite{LewinLewinORTF}, by induction on the complexity of $R$, where the complexity is defined to be the length of $R$ minus the number of generators appearing in $R$. If $R$ has complexity $0$, then $G$ is free, in which case the Lewin--Lewin construction is just the universal division ring of fractions $U(kG)$ (which exists due to \cite{Cohnfreeproducts}). But also $U(kG) \cong \mathcal D_{kG}$ for $G$ free (see, e.g., \cite[Theorem 1.1]{JaikinZapirain2020THEUO}).
    
    Now assume the complexity is greater than zero. We also assume that every generator appears in $R$. Otherwise $G$ decomposes as a free product $H * F$, where $H = \langle a_1, \dots, a_n \mid R \rangle$ with $R$ involving all the generators $a_1, \dots, a_n$ and $F$ is a free group. In this case, the Lewin--Lewin construction is $\overline{kG} = U(\overline{kH} * U(kF))$. Assuming the result for $H$, we conclude that $\overline{kG} \cong \mathcal D_{kG}$ by \cref{lem:HF_amalg}. We also assume that $R$ involves a generator of exponent sum zero, and explain how to reduce to this case at the end of the proof. Without loss of generality, suppose that $t = a_1$ has exponent sum zero, and let $N$ be the normal subgroup generated by the elements $a_i$ for $i \geqslant 2$. Note that $N$ splits as a line of groups
    \[
        \cdots * N_{i-1} *_{A_{i-1,i}} N_i *_{A_{i,i+1}} N_{i+1} * \cdots 
    \]
    where each $N_i$ is a one-relator group with relator of complexity strictly less than that of $R$ and each $A_{i-1,i}$ is a free group (see \cite[Section 5]{LewinLewinORTF}). By induction, each ring $kN_i$ embeds into a Hughes-free division ring $\mathcal D_{kN_i}$, and by uniqueness of Hughes-free division rings we can form the following line of division rings
    \[
        C = \cdots * \mathcal D_{kN_{i-1}} *_{\mathcal D_{kA_{i-1,i}}} \mathcal D_{kN_i} *_{\mathcal D_{kA_{i,i+1}}} \mathcal D_{kN_{i+1}} * \cdots 
    \]
    which is a semifir by Cohn's theorem, and it naturally contains the group ring
    \[
        kN \cong \cdots * kN_{i-1} *_{kA_{i-1,i}} kN_i *_{kA_{i,i+1}} kN_{i+1} * \cdots.
    \]
    There is thus a universal division ring of fractions $U(C)$. We can also construct a division ring $\mathcal D$ embedding $kN$ as the direct limit of the system 
    \[
        \{U(\mathcal{D}_{kN_{-i}} *_{\mathcal{D}_{kA_{-i,-i+1}}}\ldots *_{\mathcal{D}_{kA_{i-1,i}}} \mathcal{D}_{kN_{i}})\}_{i\in \N}
    \]
    of Hughes-free division rings by Lemma \ref{lem:HF_amalg}. Note $\mathcal D$ is Hughes-free as a $kN$-division ring and therefore $\mathcal D = \mathcal D_{kN}$ coincides with the division closure of $kN$ in $\mathcal D_{kG}$. Thus, we have the following commutative diagram
    \[
        \begin{tikzcd}
            kN \arrow[r, hook] \arrow[d, hook] & C \arrow[d, hook]\arrow[dl, hook]\\
            \mathcal{D}_{kN}   & U(C)    
        \end{tikzcd}
    \]
    So again combining universality and maximality of rank functions first over $C$ and then over $kN$ as in the proof of \cref{lem:HF_amalg}, it follows that $\mathcal{D}_{kN}$ and $U(C)$ are $kN$-isomorphic. In particular, $U(C)$ is Hughes-free. Finally, the Lewin--Lewin division ring is the Ore localization of $U(C)[t^{\pm 1};\tau]$, which according to Lemma \ref{lem:HF_cyclic} is $kG$-isomorphic to the Hughes-free $kG$-division ring of fractions $\mathcal{D}_{kG}$. This concludes the proof of the case where $R$ involves a generator with exponent sum zero.

    Suppose now that $R$ involves no generator with exponent sum zero. If this is the case, then by \cite[Proposition 2]{LewinLewinORTF} the one relator group $H := G * \Z$ can be given a one-relator presentation where the cyclically reduced word has complexity strictly less than that of $R$. By induction, we conclude that $\overline{kH} = \mathcal D_{kH}$. Lewin--Lewin set $\overline{kG}$ to be the division closure of $kG$ inside $\overline{kH}$, and this is just $\mathcal D_{kG}$. \qedhere 
\end{proof}

\begin{rem}
    The above proof shows that the Lewin--Lewin construction can be further extended to crossed products $k * G$.
\end{rem}

The existence of such Hughes-free division ring for locally indicable groups over a characteristic $0$ field was proved in \cite[Corollary 1.4]{JaikinLopez_Atiyah} as a corollary of the strong Atiyah conjecture over $\mathbb C$ for locally indicable groups. Thus, in characteristic zero, we conclude that the Lewin--Lewin construction is Hughes-free.

\begin{cor}
    Let $G$ be a torsion-free one-relator group and let $k$ be a field of characteristic $0$. Then $\overline{kG}$ is Hughes-free.
\end{cor}

\begin{defn}\label{def:PR}
    The \textit{primitivity rank} of a word $w$ in a free group $F$ is the minimal rank of a subgroup $K$ containing $w$ such that $w$ is imprimitive in $K$ (we define the primitivity rank to be $\infty$ if there is no such $K$).
\end{defn}

Note that words of primitivity rank $1$ are exactly the proper powers. Over arbitrary coefficients, we conclude the following.

\begin{cor}
    Let $G = F/\llangle R \rrangle$ be a torsion-free one-relator group, and suppose $R$ has primitivity rank at least $3$. Then $\overline{kG}$ is Hughes-free for any division ring $k$.
\end{cor}

\begin{proof}
    Louder and Wilton proved that $G$ has negative immersions if and only if $R$ has primitivity rank at least $3$ \cite[Theorem 1.3]{LouderWilton_NegativeImmersions} and Linton showed that one-relator groups with negative immersions are virtually compact special \cite[Theorem 8.2]{Linton_ORH}. Thus, $\mathcal D_{kG}$ exists by \cref{cor:VCSinField} and \cref{lem:HF_fi}, and by \cref{thm:LL_HF} $\overline{kG} \cong \mathcal{D}_{kG}$.
\end{proof}

We conclude with the following natural question. By the previous corollaries, it is settled in the affirmative in characteristic $0$, and only the primitivity rank $2$ case remains in characteristic $p > 0$.

\begin{q}
    Is the Lewin--Lewin construction always Hughes-free?
\end{q}







\bibliographystyle{alpha}
\bibliography{bib}

\end{document}




In order to show the above theorem we heavily use the well known fact that torsion-free one-relator groups are locally indicable established in \cite{BrodskiiOR}. To begin with we recall the construction of the Lewin--Lewin division ring. Let 
\[
    G=\langle a_1, a_2,\ldots| R(a_1,\ldots, a_n)\rangle
\]
be a torsion-free one-relator group. Assume that all generators appear in $R$, for otherwise we have $G = H * F$ where $H=\langle a_1,\ldots, a_n| R(a_1,\ldots, a_n)\rangle$. However, for any division ring $k$ and a free group $F$, crossed products $kF$ are coproducts of principal ideal domains, and hence semifirs; thus taking crossed product of semifirs, they are reduced to show an embedding for $H$. Nevertheless, this does not present a real difficult for us, since  by hypothesis the group $G$ we shall be working with is already known to have a Hughes-free $kG$-division ring of fractions; and hence by Lemma \ref{lem:HF_amalg} the Lewin--Lewin division ring is Hughes-free too. So let us keep on with this assumption and work by induction on complexity as in \cite{LewinLewinORTF}.

If $R(a_1,\ldots, a_n)$ has complexity zero, then $G$ is a free group and $\overline{kG}$ is given by the semifir structure, which is Hughes-free by \cite[Theorem 1 \& 2]{Lewinfree} .

On the other hand, if the complexity of $R$ is greater than zero, then the existence of such division ring for $kG$ splits in two cases: either $G$ has a relator with exponent sum zero on some letter or not. The former case reduces to the first as follows. Since $G$ is a torsion-free group, $R(a_1,\ldots, a_n)$ is not a proper power. So according to \cite[Proposition 2]{LewinLewinORTF} there is a torsion-free one-realtor group $H=G * G_1$ with $G_1=\langle t \rangle$ cyclic, defining relator of exponent sum $0$ on $t$ and appropriate complexity in the sense that $\mathcal{D}_{kH}$ exists once the zero exponent case is done. Therefore, they set $\overline{kG}$ as the division closure of $kG$ inside $\mathcal{D}_{kH}$. However, since Hughes-free is a property that passes to subgroups, we are once again reduced to show that the Lewin--Lewin division ring is Hughes-free for groups with defining relator having exponent sum zero on some letter.

For the exponent sum zero case we shall give a Hughes-free division ring, with a construction in the same spirit as in Lewin--Lewin, and show that it is in fact isomorphic to $\overline{kG}$. First assume that the complexity of $R$ is greater than zero, that it also has exponent sum zero on some letter and that torsion-free one-relator groups whose defining relator has smaller complexity than that of $R$ embed into a Hughes-free $kG$-division ring of fractions. Although the stronger assumption of Hughes-free is not originally stated in \cite{LewinLewinORTF}, it can be done by the complexity zero case. Now we relabel the generators as $t,x,y,\ldots, z$ so that $R(t,x,y,\ldots, z)$ has exponent sum zero on $t$. Let $N$ be the normal closure in $G$ of $\{x,y,\ldots,z\}$. Then $N$ can be seen as the direct limit of a tree of groups $(\mathcal{N},T)$, where $T$ is a tree whose vertices are subgroups $N_i$ of less complexity, one for each integer $i$, and with an edge joining $i$ and $i+1$ by a free group $A_{i,i+1}$ for each $i$ (see \cite[Section 5]{LewinLewinORTF}). Since our group is locally indicable, the uniqueness of the Hughes-free division ring allows us to form a tree of division rings $(\overline{\mathcal{N}},T)$ given by
$$\begin{tikzcd}
            &\mathcal{D}_{kN_{i-1}} && \mathcal{D}_{kN_{i}}&&\mathcal{D}_{kN_{i+1}}& \\
            \ldots &kN_{i-1}\arrow[u,hook]&\mathcal{D}_{kA_{i-1,i}}\arrow[ul,hook']\arrow[ur,hook]& kN_{i}\arrow[u,hook]&\mathcal{D}_{kA_{i,i+1}}\arrow[ur,hook]\arrow[ul,hook']&kN_{i+1}\arrow[u,hook]&\ldots \\
            &&kA_{i-1,i}\arrow[ur,hook]\arrow[ul,hook']\arrow[u,hook]&&kA_{i,i+1}\arrow[ur,hook]\arrow[ul,hook']\arrow[u,hook]&&
        \end{tikzcd}$$
Now we define $C$ as the direct limit of the system $(\overline{\mathcal{N}},T)$, which is a semifir in which $kN$ embeds by \cite[Theorem 1]{LewinLewinORTF}. Thus $kN$ embeds into the universal division ring $U(C)$ given by the semifir structure. However, by \cref{lem:univ_equality} we can also construct $\mathcal{D}$ as the direct limit of the system $\{U(\mathcal{D}_{kN_{-i}} *_{\mathcal{D}_{kA_{-i,-i+1}}}\ldots *_{\mathcal{D}_{kA_{i-1,i}}} \mathcal{D}_{kN_{i}})\}_{i\in \N}$ of Hughes-free division rings due to Lemma \ref{lem:HF_amalg}. Note also that $\mathcal{D}$ is a Hughes-free division $kN$-ring of fractions and hence it is $kN$-isomorphic to $\mathcal{D}_{kN}$, the division closure of $kN$ in $\mathcal{D}_{kG}$. Thus we have the following commutative diagram
$$\begin{tikzcd}
            kN \arrow[r, hook] \arrow[d, hook] & C \arrow[d, hook]\arrow[dl, hook]\\
            \mathcal{D}_{kN}   & U(C)    
        \end{tikzcd}$$
So again combining universality and maximality of rank functions first over $C$ and then over $kN$ as in the proof of \cref{lem:HF_amalg}, it follows that $\mathcal{D}_{kN}$ and $U(C)$ are $kN$-isomorphic. In particular, $U(C)$ is Hughes-free. Finally, the Lewin--Lewin division ring is the Ore localization of $U(C)[t^{\pm 1};\tau]$, which according to Lemma \ref{lem:HF_cyclic} is $kG$-isomorphic to the Hughes-free $kG$-division ring of fractions $\mathcal{D}_{kG}$. 
This concludes the proof of \cref{thm:LL_HF}.