\documentclass{amsart}
% CVPR 2022 Paper Template
% based on the CVPR template provided by Ming-Ming Cheng (https://github.com/MCG-NKU/CVPR_Template)
% modified and extended by Stefan Roth (stefan.roth@NOSPAMtu-darmstadt.de)

\documentclass[10pt,twocolumn,letterpaper]{article}

%%%%%%%%% PAPER TYPE  - PLEASE UPDATE FOR FINAL VERSION
%\usepackage[review]{cvpr}      % To produce the REVIEW version
\usepackage{cvpr}              % To produce the CAMERA-READY version
%\usepackage[pagenumbers]{cvpr} % To force page numbers, e.g. for an arXiv version

% Include other packages here, before hyperref.
\usepackage{graphicx}
\usepackage{amsmath}
\usepackage{amssymb}
\usepackage{booktabs}


% It is strongly recommended to use hyperref, especially for the review version.
% hyperref with option pagebackref eases the reviewers' job.
% Please disable hyperref *only* if you encounter grave issues, e.g. with the
% file validation for the camera-ready version.
%
% If you comment hyperref and then uncomment it, you should delete
% ReviewTempalte.aux before re-running LaTeX.
% (Or just hit 'q' on the first LaTeX run, let it finish, and you
%  should be clear).
\usepackage[pagebackref,breaklinks,colorlinks]{hyperref}


% Support for easy cross-referencing
\usepackage[capitalize]{cleveref}
\crefname{section}{Sec.}{Secs.}
\Crefname{section}{Section}{Sections}
\Crefname{table}{Table}{Tables}
\crefname{table}{Tab.}{Tabs.}


%%%%%%%%% PAPER ID  - PLEASE UPDATE
\def\cvprPaperID{2200} % *** Enter the CVPR Paper ID here
\def\confName{CVPR}
\def\confYear{2023}


% \usepackage[latin1]{inputenc}
\usepackage[british]{babel}
\usepackage[all]{xy}
\usepackage{amscd}
\usepackage{amssymb}
\usepackage{amsthm}
\usepackage{enumitem}
\usepackage{mathrsfs,bbm}
\usepackage{xcolor,graphicx}
\usepackage{graphics}
\usepackage{soul}
\usepackage{comment}
\usepackage[all]{xy}
\usepackage{amscd}
\usepackage{amssymb,amsmath,latexsym}
\usepackage{amsthm}
\usepackage{enumitem}
\usepackage{mathrsfs,bbm}
\usepackage{dsfont}
\usepackage{tikz-cd}
\usepackage[T1]{fontenc}
\usepackage[utf8]{inputenc}  
 %
%%%%%%%%%%%%%%%%%%%%%%%%%%%%%%%%%%
%pagestyle
%%%%%%%%%%%%%%%%%%%%%%%%%%%%%%%%%%
%\pagestyle{plain}
\textwidth=430pt
\headsep=.7cm
\evensidemargin=15pt
\oddsidemargin=15pt
\leftmargin=0cm
\rightmargin=0cm
%%
%%%%%%%%%%%%%%%%%%%%%%%
\newcommand*\fixitem {\item[]%
  \refstepcounter{enumi}\hskip-\leftmargin\labelenumi\hskip\labelsep}
\newtheorem*{mainthm}{Main Theorem}
\newtheorem*{mainthm1}{Theorem}
\newtheorem*{maincor}{Corollary}
\usepackage[colorlinks=true]{hyperref}
\DeclareMathOperator{\Forall}{\forall}
\DeclareMathOperator{\Exists}{\exists}
\DeclareMathOperator{\ord}{ord}
\newcommand{\phiD}{\varphi_D}
\newcommand{\phiDI}{\varphi_{\mathbf{D}_I}}
\newcommand{\phiDIj}{\varphi_{\mathbf{D}_I (j)}}
\newcommand{\phiH}{\varphi_H}
\newcommand{\phiTimes}{\phiD \otimes \phiH}
\newcommand{\phiTimesDI}{\varphi_{\mathbf{D}_I} \otimes \phiH}
\newcommand{\R}{\mathscr{A}}
\newcommand{\X}{\mathscr{X}}
\newcommand{\Xf}{\mathscr{X}_{(k_0 ,i)}[r_0]}
\newcommand{\Xfr}{\mathscr{X}_{(k_0,i)}[r]}
\newcommand{\hotimes}{\widehat{\otimes}}
\newcommand{\C}{\mathbb{C}_p}
\newcommand{\V}{\mathscr{V}}
\newcommand{\B}{\mathscr{B}}
\newcommand{\dualD}{\mathfrak{D}}
\newcommand{\Dg}{\mathbf{D}}
\newcommand{\DD}{\mathcal{D}^0}
\newcommand{\DDg}{\mathcal{D}}
\newcommand{\DV}{\mathcal{D}}
\newcommand{\W}{\mathscr{W}_N}
\newcommand{\Ao}{\mathbf{A}^\circ}
\newcommand{\AoK}{\mathbf{A}^\circ_{\K}}
\newcommand{\AK}{\mathbf{A}_{/\K}}
\newcommand{\OOO}{\mathscr{A}^\circ}
\newcommand{\K}{\mathcal{K}} 
\newcommand{\OK}{\mathcal{O}_{\K}}
\newcommand{\varprojlog}[1]{\underleftarrow{\log\!^{#1}}}
\newcommand{\T}{\mathscr{T}}
\newcommand{\TT}{\mathbf{T}}
\newcommand{\VV}{\mathbf{V}}
\newcommand{\HH}{\mathcal{H}}
\newcommand{\hh}{\mathcal{H}^+}
\newcommand{\HG}[2]{\mathcal{H}_{#1}(#2)}
\newcommand{\hhl}{\mathcal{H}^{+,[l]}}
\newcommand{\hhj}{\mathcal{H}^{+,[j]}}
\newcommand{\hhjj}{\mathcal{H}^{+,[l,l']}}
\newcommand{\GS}{G_{\mathbb{Q},S}}
\newcommand{\Rf}{R_{(k_0 ,i)}[r_0]}
\newcommand{\Rfr}{R_{(k_0 ,i)}[r]}
\newcommand{\parT}{\langle T\rangle}
\newcommand{\Zf}{Z_{(k_0 ,i)}[r_0]}
\newcommand{\Zfr}{\mathscr{Z}_{(k_0 ,i)}[r]}
\newcommand{\ZFf}{\mathscr{Z}_{(k_0 ,i)}[r_0]}
\newcommand{\ZFfr}{\mathscr{Z}_{(k_0 ,i)}[r]}
\newcommand{\ZF}{\mathscr{Z}}


\begin{document}

%%%%%%%%% TITLE - PLEASE UPDATE
\title{\ours: Neural 3D Relightable Faces}


\author{Anurag Ranjan$^\symknight$ \; \; \;
Kwang Moo Yi$^{\symknight\symrook}$ \; \; \;
Jen-Hao Rick Chang$^\symknight$ \; \; \;
Oncel Tuzel$^\symknight$ \\
$^\symknight$Apple \; \; \;
$^\symrook$The University of British Columbia
}

\twocolumn[{%
\renewcommand\twocolumn[1][]{#1}%
\maketitle
\begin{center}
    \newcommand{\teaserwidth}{\textwidth}
\vspace{-0.3cm}
   \includegraphics[width=0.96\linewidth]{figs/teaser4.jpg} \\
  \includegraphics[width=0.96\linewidth]{figs/teaser5.jpg}
%   }
   \captionof{figure}{Generated samples from our model. \textbf{Left}: 3D reconstruction visualization. \textbf{Center and Right}: Rendered faces using 2 different illumination conditions under 3 different poses. Illumination visualization using spherical harmonics~\cite{ramamoorthi2001efficient}.
}
\label{fig:teaser}
\end{center}%
}]

\maketitle


%%%%%%%%% ABSTRACT


Over the past few years, there has been a significant amount of research focused on studying the ReLU activation function, with the aim of achieving neural network convergence through over-parametrization. However, recent developments in the field of Large Language Models (LLMs) have sparked interest in the use of exponential activation functions, specifically in the attention mechanism.

Mathematically, we define the neural function $F: \R^{d \times m} \times  \mathbb{R}^d \rightarrow \mathbb{R}$ using an exponential activation function. Given a set of data points with labels $\{(x_1, y_1), (x_2, y_2), \dots, (x_n, y_n)\} \subset \mathbb{R}^d \times \mathbb{R}$ where $n$ denotes the number of the data. Here $F(W(t),x)$ can be expressed as $F(W(t),x) := \sum_{r=1}^m a_r \exp(\langle w_r, x \rangle)$, where $m$ represents the number of neurons, and $w_r(t)$ are weights at time $t$. It's standard in literature that $a_r$ are the fixed weights and it's never changed during the training. We initialize the weights $W(0) \in \mathbb{R}^{d \times m}$ with random Gaussian distributions, such that $w_r(0) \sim \mathcal{N}(0, I_d)$ and initialize $a_r$ from random sign distribution for each $r \in [m]$.

Using the gradient descent algorithm, we can find a weight $W(T)$ such that $\| F(W(T), X) - y \|_2 \leq \epsilon$ holds with probability $1-\delta$, where $\epsilon \in (0,0.1)$ and $m = \Omega(n^{2+o(1)}\log(n/\delta))$. To optimize the over-parametrization bound $m$, we employ several tight analysis techniques from previous studies [Song and Yang arXiv 2019, Munteanu, Omlor, Song and Woodruff ICML 2022]. 

 



%%%%%%%%% BODY TEXT
\section{Introduction}
\label{sec:introduction}
% \begin{itemize}
%     % Diffusion of FL
%     \item {\st{Diffusion of FL}}
%     % Security threats to FL
%     \item {\st{Security threats to FL with particular focus on model poisoning}}
%     % Limitations of existing countermeasures
%     \item {\st{Current countermeasures (e.g., KRUM) and their limitations}}
%     % Proposed method and its advantages
%     \item {\st{Intuitive description of the proposed method and its difference (i.e., advantages) w.r.t. state of the art}}
%     % Main contributions
%     \item {\st{Summary of the main contributions of this work}}
%     % Paper's structure and organization
%     \item {\st{Paper's structure and organization}}
% \end{itemize}

% Diffusion of FL
Recently, {\em federated learning} (FL) has emerged as the leading paradigm for training distributed, large-scale, and privacy-preserving machine learning (ML) systems~\cite{mcmahan2017googleai,mcmahan2017aistats}. 
The core idea of FL is to allow multiple edge clients to collaboratively train a shared, global model without disclosing their local private training data.
%Specifically, an FL system consists of a central server and many edge clients; 
A typical FL round involves the following steps: {\em(i)} the server randomly picks some clients and sends them the current, global model; {\em(ii)} each selected client locally trains its model with its own private data; then, it sends the resulting local model to the server;\footnote{Whenever we refer to global/local model, we mean global/local model {\em parameters}.} {\em(iii)} the server updates the global model by computing an \emph{aggregation function}, usually the average (FedAvg), on the local models received from clients.
% \begin{enumerate}
%     \item[{\em(i)}] the server sends the current, global model to the clients and appoints some of them for training;
%     \item[{\em(ii)}] each selected client locally trains its copy of the global model with its own private data; then, it sends the resulting local model back to the server;\footnote{Whenever we refer to global/local model, we mean global/local model {\em parameters}.}
%     \item[{\em(iii)}] the server updates the global model by computing an \emph{aggregation function} on the local models received from clients (by default, the average, also referred to as FedAvg~\cite{mcmahan2017aistats}).
% \end{enumerate}
This process goes on until the global model converges. %(e.g., after a certain number of rounds or other similar stopping criteria).
%\\
% The advantages of FL over the traditional, centralized learning paradigm are undoubtedly clear in terms of flexibility/scalability (clients can join/disconnect from the FL network dynamically), network communications (only model weights\footnote{We will use \textit{parameters} and \textit{weights} interchangeably.} are exchanged between clients and server), and privacy (each client's private training data is kept local at the client's end and not uploaded to the server).
\\
% Security threats to FL
%However, the growing adoption of FL also raises security concerns~\cite{costa2022covert}, particularly about its confidentiality, integrity, and availability.
Although its advantages over standard ML, FL also raises security concerns~\cite{costa2022covert}. %, particularly about its confidentiality, integrity, and availability~\cite{costa2022covert}.
% OLD, LONG VERSION
% Indeed, some work deals with privacy leakage that may expose the local data of some clients~\cite{melis2019sp}. 
% A large body of work, instead, investigates attacks that usually aim to detriment the predictive accuracy of the learned global model. For instance, \emph{data poisoning} attacks achieve this goal by letting an adversary pollute the training set of some corrupt FL clients with maliciously crafted examples~\cite{jagielski2018sp}.
% Similarly, in \emph{model poisoning} the attacker attempts to tweak the global model weights~\cite{bhagoji2019pmlr} by directly perturbing the local model's weights of some infected FL clients before these are sent to the central server for aggregation, usually via so-called Byzantine attacks. 
% It turns out that Byzantine model poisoning attacks severely impact standard FedAvg; therefore, more robust aggregation functions must be designed to make FL systems secure.
Here, we focus on \emph{untargeted model poisoning} attacks~\cite{bhagoji2019pmlr}, where an adversary attempts to tweak the global model weights %\footnote{We will use the terms \textit{parameters} and \textit{weights} interchangeably.} 
by directly perturbing the local model's parameters of some infected clients before these are sent to the central server for aggregation.
In doing so, the adversary aims to jeopardize the global model \textit{indiscriminately} at inference time.
Such model poisoning attacks severely impact standard FedAvg; therefore, more robust aggregation functions must be designed to secure FL systems.
\\
% In this paper, we focus on designing a novel robust aggregation scheme at the server's end to contrast the effect of Byzantine model poisoning attacks.
%
% Current countermeasures and their limitations
%Several countermeasures have been proposed in the literature to combat model poisoning attacks on FL systems.
% Some methods use simple statistics more robust than plain average to smooth the impact of malicious updates (e.g., Trimmed Mean and FedMedian~\cite{yin2018icml}). 
% Other defenses implement outlier detection techniques to discard malicious updates from the aggregation performed at the server's end. Those are either based on heuristics (e.g., Krum/Multi-Krum~\cite{blanchard2017nips} and Bulyan~\cite{mhamdi2018pmlr}) or data-driven approaches (e.g., K-means clustering~\cite{shen2016acm} or DnC via spectral analysis~\cite{shejwalkar2021ndss}). 
% Finally, some strategies rely on a centralized ``source of trust'' to spot potential malicious updates (e.g., FLTrust~\cite{cao2020fltrust}).
% Several countermeasures have been proposed in the literature to combat model poisoning attacks on FL systems, i.e., to discard possible malicious local updates from the aggregation performed at the server's end. 
% These techniques range from simple statistics more robust than plain average (e.g., Trimmed Mean and FedMedian~\cite{yin2018icml}) to outlier detection heuristics (e.g., Krum/Multi-Krum~\cite{blanchard2017nips} and Bulyan~\cite{mhamdi2018pmlr}) or data-driven approaches (e.g., spectral analysis via K-means clustering~\cite{shen2016acm} or spectral analysis), or methods based on ``source of trust'' (e.g., FLTrust~\cite{cao2020fltrust}).
% OLD, LONG VERSION
%Several countermeasures have been proposed in the literature to combat Byzantine model poisoning attacks on FL systems.
% Descriptive statistics
% For example, Trimmed Mean and FedMedian aggregate local model updates using more robust statistics than standard average~\cite{yin2018icml}.
%
% % Heuristics for outlier detection
% Many existing Byzantine-resilient strategies implement some outlier detection heuristics to discard the model updates sent by potentially malicious clients from the input of the aggregation function.
% One of the most popular heuristics is Krum~\cite{blanchard2017nips}.
% This strategy tries to mitigate the impact of Byzantine attacks by selecting as a global model the local model with the smallest sum of Euclidean distances to {\em all} the other local models.
% Although powerful, Krum requires the server to know (or, at least, estimate) the number of malicious FL clients upfront, which is generally impossible in a realistic attack scenario. %
% Moreover, Krum may become ineffective for complex, high-dimensional model parameter spaces due to the curse of dimensionality.
% Bulyan~\cite{mhamdi2018pmlr} tries to overcome this issue by combining Krum with a variant of Trimmed Mean.
% % Data-driven outlier detection
% Other strategies use data-driven outlier detection techniques -- e.g., via K-means clustering~\cite{shen2016acm} -- to spot potential malicious local model updates. 
% %For instance, Shen et al. propose to cluster local model updates with K-means and thus identify outliers.
%
% % Other techniques
% As far as the server is concerned, any local model received can be from a potential malicious client. 
% FLTrust~\cite{cao2020fltrust} assumes the server acts as a client, i.e., trains a local model on an additional {\em trustworthy} dataset at the server's end and compares it against all the local models from other clients. 
% This way, the server can rely on some ``source of trust'' when discarding potentially malicious clients.
%\\
% Limitations of existing Byzantine-resilient strategies
Unfortunately, existing defense mechanisms either rely on simple heuristics (e.g., Trimmed Mean and FedMedian by~\cite{yin2018icml}) or need strong and unrealistic assumptions to work effectively (e.g., foreknowledge or estimation of the number of malicious clients in the FL system, as for Krum/Multi-Krum~\cite{blanchard2017nips} and Bulyan~\cite{mhamdi2018pmlr}, which, however, cannot exceed a fixed threshold).
Furthermore, outlier detection methods using K-means clustering~\cite{shen2016acm} or spectral analysis like DnC~\cite{shejwalkar2021ndss} do not directly consider the temporal evolution of local model updates received.
Finally, strategies like FLTrust~\cite{cao2020fltrust} require the server to collect its own dataset and act as a proper client, thereby altering the standard FL protocol.
\\
% OLD, LONG VERSION
% Overall, existing Byzantine-resilient strategies are either simple heuristics (e.g., FedMedian) or, if they are more complex, they rely on strong and unrealistic assumptions to work effectively (e.g., knowing the number of malicious clients in the FL system in advance, as for Krum and alike).
% Furthermore, data-driven outlier detection methods do not consider the temporary evolution of local model updates received (e.g., K-means clustering). 
% Finally, strategies like FLTrust requires the server to collect its own dataset and act as a proper client, thereby altering the standard FL protocol.
%
% Description of the proposed method
This work introduces a novel pre-aggregation \textit{filter} robust to untargeted model poisoning attacks. Notably, this filter $(i)$ operates without requiring prior knowledge or constraints on the number of malicious clients and $(ii)$ inherently integrates temporal dependencies. 
The FL server can employ this filter as a preprocessing step before applying \textit{any} aggregation function, be it standard like FedAvg or robust like Krum or Bulyan.
Specifically, we formulate the problem of identifying corrupted updates as a multidimensional (i.e., matrix-valued) time series anomaly detection task. 
The key idea is that legitimate local updates, resulting from well-calibrated iterative procedures like stochastic gradient descent (SGD) with an appropriate learning rate, show \textit{higher predictability} compared to malicious updates. This hypothesis stems from the fact that the sequence of gradients (thus, model parameters) observed during legitimate training exhibit regular patterns, as validated in Section~\ref{subsec:intuition}. %until convergence. 
%This regularity may be more pronounced for smooth convex loss functions, but it can still be captured within an appropriate time window, even for more complex and convoluted loss surfaces. 
%We provide evidence of this claim in Appendix~B, where we show that the average mutual information (i.e., ``predictability''), calculated over pairs of legitimate model updates sent at different FL rounds, is significantly higher than the corresponding computation for a malicious client.
\\
Inspired by the matrix autoregressive (MAR) framework for multidimensional time series forecasting~\cite{chen2021je}, we propose the FLANDERS ({\em \textbf{F}ederated \textbf{L}earning meets \textbf{AN}omaly \textbf{DE}tection for a \textbf{R}obust and \textbf{S}ecure}) filter.
The main advantages of FLANDERS over existing strategies like FLDetector~\cite{zhao2020multivariate} are its resilience to large-scale attacks, where $50\%$ or more FL participants are hostile, and the capability of working under realistic non-iid scenarios.
We attribute such a capability to two key factors: $(i)$ FLANDERS works without knowing a priori the ratio of corrupted clients, and $(ii)$ it embodies temporal dependencies between intra- and inter-client updates, quickly recognizing local model drifts caused by evil players. Below, we summarize our main contributions:

\begin{itemize}
\item[{\em(i)}]
We provide empirical evidence that the sequence of models sent by legitimate clients is more predictable than those of malicious participants performing untargeted model poisoning attacks.
\\
\item[{\em(ii)}] 
We introduce FLANDERS, the first pre-aggregation filter for FL robust to untargeted model poisoning based on multidimensional time series anomaly detection.
\\
\item[{\em(iii)}] 
We integrate FLANDERS into Flower,\footnote{\scriptsize{\url{https://flower.dev/}}} a popular FL simulation framework for reproducibility.
\\
\item[{\em(iv)}] 
We show that FLANDERS improves the robustness of the existing aggregation methods under multiple settings: different datasets, client's data distribution (non-iid), models, and attack scenarios.
\\
\item[{\em(v)}] 
We publicly release all the implementation code of FLANDERS along with our experiments.\footnote{\scriptsize{\url{https://anonymous.4open.science/r/flanders_exp-7EEB}}}
\end{itemize}

% Paper's structure and organization
The remainder of the paper is structured as follows. %some related work and the current state-of-the-art solutions to security issues that FL entails. 
Section~\ref{sec:background} covers background and preliminaries. 
In Section~\ref{sec:related}, we discuss related work.
Section~\ref{sec:problem} and Section~\ref{sec:method} describe the problem formulation and the method proposed. % to tackle it. 
Section~\ref{sec:experiments} gathers experimental results. %, and Section~\ref{sec:limitations} discusses some limitations of this work.
Finally, we conclude in Section~\ref{sec:conclusion}.
 %discusses the limitations of this work and draws future research directions.
%reports conclusions and draws perspectives for future research directions.

%%%%%%% OLD %%%%%%%
%to overcome the resilience of Byzantine failures in distributed Stochastic Gradient Descent computations. 
% The strength of Krum is its time complexity, which is linear in the gradient dimension. 
% However, the robustness of the approach is guaranteed for gradient-based learning applications only when the majority of the clients are not compromised. 
% Besides, the aggregation mechanism of Krum, as well as that of similar methods, is robust from a coarse-grained perspective and does not provide solutions to errors and perturbations that may occur at inference time.
%A related approach to~\cite{blanchard2017nips} is the work of Su et al.~\cite{su2016dc}. Here, the authors propose an iterated approximate agreement to tackle a multi-layer scenario attacked by Byzantine agents. 
%However, the method works efficiently on the sole discrete context and it is inapplicable to continuous state environments.
%\gabri{Maybe, we should just talk about the main limitations of existing countermeasures without digging into their details (or, we can just mention Krum as this is the most popular one). I will move the description of all these methods to the Related Work section.}
\section{Related work}
\noindent \textbf{Video foundation models.}
With sufficient computational power and an abundant source of data, there have been attempts to build a single large-scale foundation model that can be adapted to diverse downstream tasks.
Along with the success of foundations models in the natural language processing domain~\cite{brown2020language,chen2021evaluating,devlin2019bert} and in computer vision~\cite{bertasius2021space,jia2021scaling,radford2021learning}, video data has become another data type of interest, as it has grown in scale due to numerous internet video-sharing platforms.
Accordingly, several methods to train a video foundation model have been proposed.
Due to the innate multi-modality of video data, \textit{i.e.}, a combination of visual $\cdot$ vocal $\cdot$ textual context, most works have centered around the variations of the cross-modal attention mechanism \cite{akbari2021vatt,bertasius2021space,gabeur2020multi,luo2020univl,neimark2021video,tan2021look,wei2020multi,yang2021taco}.
In addition, as most video data lack proper labels or descriptions, contrastive learning methods were studied to learn meaningful feature representations or enhance video-text alignment in a self-supervised manner \cite{akbari2021vatt,kuang2021video,luo2020univl,yang2021taco}.

More specifically, MERLOT \cite{zellers2021merlot} proposed a multi-modal representation learning method for visual commonsense reasoning, which also performed well in twelve video reasoning tasks.
VATT \cite{akbari2021vatt} introduced a multi-modal learning method via contrastive learning. 
The pre-trained model performed well in a variety of vision tasks from image classification to video action recognition and zero-shot video retrieval.
Another representative work, UniVL \cite{luo2020univl} proposed a straightforward pre-training method with auxiliary loss functions. 
After fine-tuning on a specific task, the pre-trained model performed outstandingly in a wide range of tasks of text-to-video retrieval, action segmentation, action step localization, video sentiment analysis, and video captioning.
Other foundation models for multiple video tasks include \cite{li2020hero,sun2019learning,sun2019videobert,zhu2020actbert,fu2021violet,wang2022all}. 

\noindent \textbf{Auxiliary learning.}
In order to enhance the performance of one or a multitude of primary tasks, auxiliary learning methods can be incorporated.
\cite{ruder2017overview} introduced Multi-task learning (MTL) to the deep neural networks by training a single model with multiple task losses to assist learning on the main task.
Such a method is generally adapted to pre-train the foundation models in the self-supervised manner~\cite{li2020hero,sun2019learning,sun2019videobert,zhu2020actbert,fu2021violet,wang2022all}.
However, these various pretext task losses used in the pre-training phase are ignored in the fine-tuning phase, and only the primary task loss is minimized.

Recently, meta-learning methods have been introduced for auxiliary learning.
\cite{liu2019self,navon2020auxiliary,shu2019meta} proposed a meta-learning method in which the model learns auxiliary tasks to generalize well to unseen data. 
In these settings, a separate subset of data is held out as the primary task, while the others are used as auxiliary tasks that aid the primary task's performance.
Similar methods were adopted for computer vision tasks such as semantic segmentation \cite{xu2021leveraging}.
Other domain applications include navigation tasks with reinforcement learning \cite{ye2021auxiliary}, or self-supervised learning methods on graph data \cite{hwang2020self}.
\section{Method}
\label{sec:method}

% \ml{``Inconsistent'' to ``large variation''}

% In this section, we propose our methods based on the observations in Section \ref{sec:motivation}.
In this section, we propose two techniques to further enhance the strong baseline to capture the variation of activation distributions better.
We first introduce spatial re-scaling to adapt the network to pixel-to-pixel variation.
We then propose channel-wise shifting and re-scaling to better capture the channel-to-channel variation.
Meanwhile, as both of the two methods are image-dependent, the image-to-image variation can be captured naturally.
By combining the two methods with our strong baseline, we build our enhanced BNN for SR, named EBSR.

% Because the activation distributions among pixels, channels and images have large variations \red{**are highly inconsistent} in SR networks, we introduce spatial re-scaling to adapt to pixel-wise variations and channel shift and re-scaling to adapt to channel-wise variations. And both of them are image-dependent to adapt to image-wise variations, which means during inference our network re-scales and shifts the distributions of activations flexibly for different input images. Based on these methods, we build an enhanced binary neural network for image super-resolution (EBSR).

% According to [3], the difference of activation magnitudes indicates different scaling factors are needed for each pixel.

\subsection{Spatial Re-scaling}
% It is better to use different scaling factors for different pixels to reduce the quantization error and retain more detailed information for image super-resolution. 

% \ml{In the main method, we do not need to introduce the previous works but can focus on introducing our own method. Channel rescaling in Real-to-binary Net is not relevant in this context.}

% Re-scaling the output of binary convolutions was proposed at the birth of BNN in XNOR-Net \cite{rastegari2016xnor} to reduce quantization error and improve accuracy for image classification tasks.
% It is computed as below:
% \begin{equation}
% \mathcal{A} * \mathcal{W} \approx(\operatorname{sign}(\mathcal{A}) \circledast \operatorname{sign}(\mathcal{W})) \odot \mathcal{K} \alpha
% \label{eq:xnor-net rescale}
% \end{equation}
% where $\circledast$ denotes the binary convolution and $\odot$ denotes the element-wise multiplication.
% $\mathcal{A}$, $\mathcal{W}$, $\alpha$, and $\mathcal{K}$ denote the activation, weight, weight scaling factor, and activation scaling factor, respectively.
%  Later in XNOR-Net++ \cite{bulat2019xnor}, Bulat et al. fuse the activation and weight scaling factors into a single one that is learned end-to-end based on gradients and this improves the classification accuracy on ImageNet dataset.

% % It is computed as Eq.~\ref{eq:xnor-net rescale}, where $\circledast$ denotes 
% %  the binary convolution and $\odot$ denotes the element-wise multiplication. The binary convolution of $\mathcal{A}$ and $\mathcal{W}$ is rescaled by the weight scaling factor $\alpha$ and the activation scaling factor $\mathcal{K}$, both of which are calculated analytically.


% \zc{Similarly, you should explain the meaning of A, W and the operators $\circledast$ in the formula}
% Then in Real-to-binary Net \cite{martinez2020training}, Martinez et al. used a data-driven channel re-scaling module that takes the pre-convolution activations as input to predict the activation scaling factor. Unlike that in XNOR-Net++ \cite{bulat2019xnor}, these scaling factors are not fixed during inference but rather inferred from data. By doing this, they further improved the classification accuracy on ImageNet over XNOR-Net++. 
As is shown in Figure \ref{fig:pixel}, activation distributions have large pixel-to-pixel variation in SR networks
and the difference of activation magnitudes indicates different scaling factors are preferred for different pixels.
Inspired by \cite{martinez2020training}, we propose spatial re-scaling to better adapt the network to the spatial variation
of activation distributions in SR networks.
% fit the various pixel-wise distributions in SR networks.
We take the real-valued activations $A$ before convolution as input and predict pixel-wise scaling factors $S(A)$, which re-scale the binary convolution output. Spatial re-scaling process can be formulated as follows:
\begin{equation}
A * W \approx(\operatorname{sign}(A) \circledast \operatorname{sign}(W)) \odot \alpha \odot S(A)
\label{eq:spatial rescale}
\end{equation}
where $\circledast$ denotes 
the binary convolution and $\odot$ denotes the element-wise multiplication. $A$, $W$, $\alpha$, and $S\left(A\right)$ denote real-valued activations, weights, the scaling factor of weights, and the spatial-wise scaling factor of activations respectively. $S\left(A\right) \in \mathbb{R}^{1\times H\times W}$ can be calculated with a convolution and a sigmoid function.
% as $\sigma\left( CONV\left(A\right)\right)$. 
As shown in Figure \ref{fig:method}(a), real-valued activations first go through a convolution layer,
which has an input channel of $C$ and an output channel of 1, 
and then pass through a sigmoid function to produce the scaling factors $S(A)$ along the spatial dimension.
During inference, the scaling factor will change dynamically according to different input feature maps.
By re-scaling binary convolution output using $S(A)$, we can reduce the quantization error and the original pixel-wise information in FP activation
will be preserved much better.
Spatial re-scaling leads to a large PSNR improvement of 0.24 dB (from 30.30 dB to 31.54 dB) on Set5 and 0.22 dB (from 25.09 dB to 25.31 dB)
on Urban100 compared with our strong baseline. 

\subsection{Channel-wise Shifting and Re-scaling}

\begin{table}[!tb]
\centering
\caption{Comparison between whether to fuse channel-wise shifting and re-scaling or not based on our baseline with spatial re-scaling. }
\label{tab:fusing}

\scalebox{0.65}{
\begin{tabular}{c|cc|cc|cc}
\hline
\multirow{2}{*}{Method}     & \multirow{2}{*}{OPs} & \multirow{2}{*}{Params} & \multicolumn{2}{c|}{Set5} & \multicolumn{2}{c}{Urban100} \\ \cline{4-7} 
                            &                      &                         & PSNR        & SSIM        & PSNR          & SSIM         \\ \hline
Baseline + spatial re-scale & 2.16G                & 0.05M                   & 31.54       & 0.883       & 25.31         & 0.759        \\
+ channel-wise shift and re-scale             & 2.34G                & 0.09M                   & 31.61       & 0.885       & 25.35         & 0.761        \\
+ w/ fusing                   & 2.27G                & 0.08M                   & \textbf{31.64}       & \textbf{0.885}       & \textbf{25.36}         & \textbf{0.761}        \\ \hline
\end{tabular}
}
\end{table}

In SR networks, activation distributions exhibit larger channel-to-channel variation (Figure \ref{fig:chl}).
Both the mean and magnitude of the activation distributions vary significantly across channels.
% Thus we use channel-wise shifting and re-scaling to adapt to various channel-wise distributions. 
\cite{martinez2020training} has proposed the data-driven channel re-scaling, 
but our method differs from them in further introducing data-driven thresholds to handle the channel-wise variation of both mean and magnitude.
Since the blocks to generate the scaling factors and thresholds are very similar, we further propose to fuse them into one module.
% and fusing channel-wise shifting and re-scaling into one module.
We evaluate the effect of fusing the two blocks in Table \ref{tab:fusing}.
With channel-wise shifting and re-scaling fused, our models have fewer operations and parameters overhead and slightly higher performance.

For the specific process, we take the real-valued activations as input and predict different thresholds and scaling factors for each channel. They are also image dependent, e.g., $\beta_{i}$ in Eq.\ref{eq:act_binarize} is no longer fixed during inference but generated according to different input feature maps. Channel-wise shifting and re-scaling can be formulated as follows:
\begin{equation}
A * W \approx(\operatorname{sign}(A-C_s(A)) \circledast \operatorname{sign}(W)) \odot \alpha \odot C_r(A)
\label{eq:channel-wise_shift_and_rescale}
\end{equation}
where $\circledast$ denotes 
the binary convolution and $\odot$ denotes the element-wise multiplication. $C_s(A), C_r(A) \in \mathbb{R}^{C\times1\times1}$ denote the channel-wise threshold and scaling factor, respectively. 
We show the block diagram in Figure \ref{fig:method}(b).
The real-valued input feature map is first squeezed to a ${C\times1\times1}$ vector by a global average pooling (GAP) layer.
The subsequent fully connected layers and ReLU learn the channel-wise information and output a ${2C\times1\times1}$ vector.
Then the ${2C\times1\times1}$ vector is split into two ${C\times1\times1}$ vectors.
We use the first $C$ channels as the channel-wise bias and pass the last $C$ channels through a sigmoid layer 
as the channel-wise scaling factor, which are used to shift the real-valued activations and re-scale the binary convolution output, respectively. 


% \ml{We can mention previously, channel-wise re-scale has been proposed. We propose to fuse them. Add the comparison between fuse v.s. no fuse.}

\begin{figure}[!tbp]%
  \centering
    \includegraphics[width=0.4\textwidth]{fig/methods.png}
  
% \subfloat[channel-wise shifting\&re-scale]{
%     \label{subfig:channel-wise shifting and re-scale}
%     \includegraphics[width=0.2\textwidth]{fig/chl shift and rescale.png}
%   }

  \caption{Block diagram for spatial re-scaling, and channel-wise shifting and re-scaling.} 
  % Input A is the real-valued activation tensor and C, H, and W denote its dimension. GAP stands for global average pooling. The reduction ratio r is set to 16 for a better trade-off between the performance and the number of operations and parameters.}
  \label{fig:method}
\end{figure}


\subsection{Network Structure}

Combining the spatial re-scaling and the channel-wise shifting and re-scaling methods, we construct the enhanced convolution layer (E-Conv).
Then we build our EBSR model based on E-Conv.
In Figure \ref{fig:E-conv}, we compare the binary convolution layer used in the baseline network and our proposed E-Conv.
We use spatial and channel-wise scaling factors to re-scale the binary convolution output,
and use channel-wise shifting to learn appropriate thresholds for each channel before binarization.
The scaling factors and threshold used in E-Conv are learnable and depend on the real-valued input activations.
In this way, our proposed EBSR can adapt to pixel-to-pixel, channel-to-channel, and image-to-image variations
to reduce the large binarization error and preserve more details.
% In this way, our proposed E-Conv reduces the large quantization error caused by binarization and keeps the original information of input feature maps to a large extent.


\begin{figure}[!tb]%
  \centering

    \includegraphics[width=0.5\textwidth]{fig/E-conv.png}

  \caption{Comparison of (a) the binary convolution layer with a skip connection used in our baseline network and (b) the proposed E-Conv.}
  \label{fig:E-conv}
\end{figure}


Figure \ref{fig:network} shows the basic block based on the E-Conv and our EBSR composed of the basic blocks. Following existing works, the convolution layers in the head and tail modules are not binarized. We choose the lightweight EDSR which has 16 basic blocks and 64 channels, and EDSR which has 32 basic blocks and 256 channels as our backbones, which correspond to EBSR-light and EBSR, respectively.

\begin{figure}[!tb]%
  \centering
  {
    \includegraphics[width=0.35\textwidth]{fig/network.png}
  }
  
  \caption{The structure of our proposed EBSR.  Convolution layers in purple are real-valued vanilla 3x3 convolutions.}
  \label{fig:network}
\end{figure}
We present in section~\ref{ssec:faces} an application of PnP-HVAE on face images, using a pretrained state-of-the-art hierarchical VAE. 
Next, we study the application of our framework to natural images. To that end, we introduce  in section~\ref{ssec:patchVDVAE}  a patch hierachical VAE architecture, that is able to model natural images of different resolutions. In section~\ref{ssec:app_nat}, we provide deblurring, super-resolution and inpainting experiments to demonstrate the relevance of the proposed method.

Additional results are presented in Appendix~\ref{app:add}. All experiments can be reproduced using the code available at \url{https://github.com/jprost76/PnP-HVAE}.



\subsection{Face Image restoration (FFHQ)}\label{ssec:faces}
We first demonstrate the effectiveness of PnP-HVAE on highly structured data, by performing face image restoration.
Latent variable generative models can accurately model structured images such as face images \cite{karras2019style,vahdat2020nvae,child2021very,kingma2018glow}, and then be used to produce high quality restoration of such data. 
In our experiments, we use the VDVAE model of~\cite{child2021very}, pre-trained on the FFHQ dataset~\cite{karras2019style}, as our hierarchical VAE prior.
VDVAE has $L=66$ latent variable groups in its hierarchy and generates images at resolution $256\times256$.

We compare PnP-HVAE with the intermediate layer optimization algorithm (ILO)~\cite{daras2021intermediate} that is based on a different class of generative models than HVAE. ILO is a GAN inversion method which optimizes the image latent code along with the intermediate layer representation of a StyleGAN to generate an image consistent with a degraded observation.
We use the official implementation of ILO, along with a StyleGAN2 model~\cite{karras2020analyzing, stylegan2pytorch}, that was trained for 550k iterations on images of resolution $256\times256$ from FFHQ.  
As VDVAE and StyleGAN models are not trained on the same train-test split of FFHQ, we chose to evaluate the methods on a subset of 100 images from the CelebA dataset~\cite{liu2018large}. 
For super-resolution, the degradation model corresponds to the application of a gaussian low-pass filter followed by a $\times 4$ sub-sampling, and the addition of a gaussian white noise with $\sigma=3$.
For the deblurring, we considered motion blur and  gaussian kernels, both with a noise level $\sigma=8$. %

We provide quantitative comparisons in table~\ref{table:comp_ILO}, along with a visual comparison of the results in figure~\ref{fig:face_restoration}.
PnP-HVAE has the best  PSNR and SSIM results for all the considered restoration tasks, while ILO provides better results  for the perceptual distance.
By jointly optimizing the image and its latent variable, PnP-HVAE provides  results that are both realistic and consistent with the degraded observation.
On the other hand,  ILO  only optimizes on an extended latent space. This method generates  sharp and realistic images with better LPIPS scores,   
but the results lack  of consistency with respect to the observation, which explains the overall lower PSNR performance. 






\subsection{PatchVDVAE: a HVAE for natural images}\label{ssec:patchVDVAE}
Available generative models in the literature operate on images of  fixed resolutions and
are either restrained to datasets of limited diversity, or even to registered face images~\cite{kingma2018glow,child2021very, vahdat2020nvae, karras2019style}, or requiring additional class information~\cite{brock2018large, dhariwal2021diffusion, song2020score, luhman2022optimizing}.
Fitting an unconditional model on natural images appears to be a more difficult task, as their resolution can change, and their content is highly diverse.
The complexity of the problem can be reduced by learning a prior model on patches of reduced dimension. 
For image restoration problems, the patch model can be reused on images of higher dimensions~\cite{zoran2011learning,prost2021learning,altekruger2022patchnr}. When the model is a full CNN, the prior on the set of the  patches can  be computed efficiently by applying the network on the full image~\cite{prost2021learning}.

We thus introduce  patchVDVAE, a fully convolutional hierarchical VAE.
Contrary to existing HVAE models whose resolution is constrained by the constant tensor at the input of the top-down block, patchVDVAE can generate images of different resolutions by controlling the dimension of the input latent. 
This amounts to defining a prior on patches whose dimension corresponds to the receptive field of the VAE. A similar model is used for image denoising in~\cite{prakash2021interpretable}.

 
For PatchVDVAE architecture, we use the same bottom-up and top-down blocks as VDVAE~\cite{child2021very}, and replace the constant trainable input in the first top-down block by a latent variable, to make the model fully convolutional (details on the  architecture are given in Appendix~\ref{app:details}). 
The training dataset is composed of $128\times 128$ patches extracted from a combination of DIV2K~\cite{agustsson2017ntire} and Flickr2K~\cite{Lim_2017_CVPR_workshops} datasets.
We perform data augmentation by extracting  patches at $3$ resolutions: HR-images and $\times 2$ and $\times 4$ downscaled images. 
The model is trained for $7.10^5$ iterations with a batch size of $64$. Following the recommendation of~\cite{hazami2022efficient}, we use Adamax optimizer with an exponential moving average and gradient smoothing of the variance.
We set the decoder model to be a gaussian with diagonal covariance, as in~\cite{luhman2022optimizing}.
PatchVDVAE is fully convolutional and can generate images of dimension that are multiples of $64$ as illustrated by
figure~\ref{fig:vdvae}.

\newlength{\patchwidth}
\setlength{\patchwidth}{0.135\columnwidth}
\begin{figure}[!ht]
    \centering
    \begin{subfigure}[t]{.34\columnwidth}\hspace{0.1cm}
        \setlength{\tabcolsep}{0.02pt}
\renewcommand{\arraystretch}{0}
        \begin{tabular}{*{2}{p{1.03\patchwidth}}}
            \includegraphics[width=\patchwidth]{figures_arxiv/patchVDVAE/samples/generated/64x64/setup-5-image-0018.png} &
            \includegraphics[width=\patchwidth]{figures_arxiv/patchVDVAE/samples/generated/64x64/setup-5-image-0016.png} \\
            \includegraphics[width=\patchwidth]{figures_arxiv/patchVDVAE/samples/generated/64x64/setup-5-image-0008.png} &
            \includegraphics[width=\patchwidth]{figures_arxiv/patchVDVAE/samples/generated/64x64/setup-5-image-0019.png}   
        \end{tabular}
    \end{subfigure}\hspace{-0.15cm}
    \begin{subfigure}[t]{.64\columnwidth}
\begin{tabular}{cc}\vspace{-0.1cm}
\includegraphics[width=2\patchwidth]{figures_arxiv/patchVDVAE/samples/generated/256x256/setup-2-image-0009.png}&
        \includegraphics[width=2\patchwidth]{figures_arxiv/patchVDVAE/samples/generated/256x256/setup-2-image-0002.png}\end{tabular}

    \end{subfigure}
    \caption{\label{fig:vdvae} Left: $64\times64$ patches samples from our patchVDVAE model trained on patches from natural images.
    Right: PatchVDVAE is fully convolutional and it can generate images of higher resolution (here: $128\times128$).\vspace{-0.2cm}}
\end{figure}

\subsection{Natural images restoration}\label{ssec:app_nat}
We  evaluate PnP-HVAE on natural image restoration.
For each task, we report the average value of the PSNR, the SSIM, and the LPIPS metrics on $20$ images from the test set of the BSD dataset~\cite{MartinFTM01}.\\


\noindent
{\bf Image deblurring.}
In the experiments, we consider $2$ gaussian kernels and $2$ motion blur kernels from~\cite{levin2009understanding}, with $3$ different noise levels 
$\sigma \in \{2.55, 7.65, 12.75\}$.
As a baseline we consider  EPLL~\cite{zoran2011learning}, which learns a prior on image patches with a gaussian mixture model.
We also compare PnP-HVAE  with PnP-MMO and GS-PnP, $2$ competing convergent Plug-and-Play methods based on CNN denoisers.
PnP-MMO~\cite{pesquet2021learning} restricts the denoiser to be contraction in order to guarantee the convergence of the PnP forward-backard algorithm. GS-PnP~\cite{hurault2022gradient} considers a gradient step denoiser and reaches state-of-the-art performances of non converging methods~\cite{zhang2021plug}.
We set the temperature $\tau$  in our method as $0.95$, $0.8$ and $0.6$ for noise levels $2.55$, $7.65$ and $12.75$ respectively, and we let it run for a maximum of $50$ iterations. 
For the three compared methods we use the official implementations and pre-trained models provided by the respective authors. 
Details on the choice of hyperparameters for the concurrent methods are provided in the Appendix~\ref{app:details}
Figure~\ref{fig:deblurring_bsd} illustrates that our method provides correct deblurring results. 

According to table~\ref{tab:deb}, the performance of PnP-HVAE is between those of EPLL and GS-PnP and it outperforms PnP-MMO for large noise levels.\\

\begin{table}
\begin{center}\footnotesize
    \begin{tabular}{>{\centering}m{.3cm}*{5}{c}}
    $\sigma$ &Method & PSNR$\uparrow$ & SSIM$\uparrow$ & LPIPS$\downarrow$  \\ 
    \hline
    \multirow{4}{*}{\vcell{$2.55$}}
    & PnP-HVAE & $27.75$ & $0.79$ & $0.31$\\
    & GS-PNP \cite{hurault2022gradient} & $\mathbf{29.59}$ & $\mathbf{0.84}$ & $\mathbf{0.22}$\\
    & EPLL \cite{zoran2011learning} & $26.49$ & $0.71$ & $0.36$\\ 
    & PnP-MMO \cite{pesquet2021learning} & $\underbar{29.50}$ & $\underbar{0.83}$ & $\underbar{0.20}$ \\ \hline
    \multirow{4}{*}{\vcell{$7.65$}}
    & PnP-HVAE & $\underbar{26.36}$ & $\underbar{0.72}$ & $\underbar{0.40}$\\
    & GS-PNP \cite{hurault2022gradient} & $\mathbf{27.33}$ & $\mathbf{0.77}$ & $\mathbf{0.31}$\\
    & EPLL \cite{zoran2011learning} & $24.04$ & $0.66$ & $0.45$ \\ 
    & PnP-MMO \cite{pesquet2021learning} & $25.34$ & $0.69$ & $0.34$\\
    \hline
    \multirow{4}{*}{\vcell{$12.75$}}
    & PnP-HVAE & $\underbar{25.12}$ & $\mathbf{0.73}$ & $\underbar{0.47}$\\
    & GS-PNP \cite{hurault2022gradient} & $\mathbf{26.32}$ & $\mathbf{0.73}$ & $\mathbf{0.37}$\\
    & EPLL \cite{zoran2011learning} & $23.28$ & $0.61$ & $0.51$ \\ 
    & PnP-MMO \cite{pesquet2021learning} & $22.42$ & $0.53$& $0.54$ \\
    \hline
    &\vspace*{-.3cm}\\
            \multicolumn{2}{c}{Blur and motion kernels}& \multicolumn{3}{c}{
        \includegraphics*[scale=1]{figures_arxiv/kernels/4.png}\;\includegraphics*[scale=1]{figures_arxiv/kernels/7.png}\;\includegraphics*[scale=1]{figures_arxiv/kernels/9.png}\;\includegraphics*[scale=1]{figures_arxiv/kernels/11.png}} 
    \end{tabular}
        \caption{\label{tab:deb}Comparison  of PnP-HVAE  and other restoration methods on deblurring. Results are averaged on $4$ kernels.\vspace{-0.2cm}}% on image deblurring.}
    \end{center}
\end{table}

\begin{figure}
    
    \begin{subfigure}[h]{\linewidth}
        \centering
        \includegraphics*[width=\columnwidth]{figures_arxiv/deb_s255_k7.pdf}\vspace{-0.1cm}
        \caption{Gaussian blur, $\sigma=2.55$}
    \end{subfigure}
    \begin{subfigure}[h]{\linewidth}
        \centering
        \includegraphics*[width=\columnwidth]{figures_arxiv/deb_s765_k11.pdf}\vspace{-0.1cm}
        \caption{Motion blur, $\sigma=7.65$}
    \end{subfigure}\vspace*{-0.1cm}
    \caption{\label{fig:deblurring_bsd} Natural image deblurring\vspace{-0.1cm}}
\end{figure}

\noindent {\bf Effect of the temperature.}
PnP-HVAE gives control on the temperature of the prior over the latent space.
In figure~\ref{fig:temp_effect}, we illustrate that reducing the temperature increases the strength of the regularization prior. In this example the tuning $\tau=0.7$ produces the best performance.\\
\begin{figure}[!ht]
   
    \includegraphics[width=\columnwidth]{figures_arxiv/demo_temp.pdf}\vspace{-0.15cm}
    \caption{ \label{fig:temp_effect} Effect of the temperature in PnP-VAE on a deblurring problem, with $\sigma=7.65$.\vspace{-0.15cm}}
\end{figure}


\noindent
{\bf Image inpainting.}
Next we consider the task of noisy image inpainting. 
We compose a test-set of 10 images from the validation set of BSD~\cite{MartinFTM01} and we create masks
  by occluding diverse objects of small size in the images. 
A gaussian white noise with $\sigma=3$ is added to the images.
As a comparaison, we still consider GS-PnP and EPLL.
For PnP-HVAE, the temperature is set to $\tau=0.6$, and the algorithm is run for a maximum of $200$ iterations, unless the residual $||\x_{k+1}-\x_k||$ is on a plateau.
We provide on Table~\ref{tab:inpainting_bsd} the distortion metrics with the ground truth, as well as a visual
\begin{table}



\begin{center}
    \begin{tabular}{cccc}
        & PSNR$\uparrow$ & SSIM$\uparrow$ &LPIPS$\downarrow$ \\\hline
        PnP-HVAE  & $\mathbf{29.54}$ & $\mathbf{0.93}$ & $\mathbf{0.06}$\\
        GS-PNP & $28.52$ & $\mathbf{0.93}$ & $0.09$\\
        EPLL & $\underline{29.16}$ & $\mathbf{0.93}$ & $\mathbf{0.06}$\\
    \end{tabular}
    \caption{\label{tab:inpainting_bsd}Quantitative evaluation for inpainting on BSD.}
    \end{center}
\end{table}
comparison on figure~\ref{fig:inpainting_bsd}. 
With its hierarchical structure,  PnP-HVAE outperforms the compared methods. \vspace{0.05cm}



\begin{figure}[!h]
    \includegraphics[width=\columnwidth]{figures_arxiv/demo_inp_bsd2.pdf}\vspace{-0.1cm}
    \caption{\label{fig:inpainting_bsd}Natural image inpainting\vspace{-0.3cm}}
\end{figure}











\section{Conclusion}\label{sec:conclusion}
In this work, we focus on addressing the fundamental challenge of OOD detection tasks, which is how to fully understand the semantic discrepancy between the ID/OOD samples. We reveal that the key to success in the realistic SCOOD task is to allocate as many ID samples in the unlabeled set correctly as possible. To this end, we propose a novel uncertainty-aware optimal transport scheme that introduces class-specific energy scores as guidance for effective label assignment. Experimental results show that our method achieves better performance than previous state-of-the-art methods on SCOOD benchmarks.

\textbf{Limitations.} In addition to temperature scaling, other techniques such as feature clipping applied in ReAct~\cite{sun2021react} also enhance the performance of energy score, so how to obtain an OOD score that best fits the SCOOD task can be further explored. Moreover, a setting highly related to SCOOD has been proposed in \cite{katz2022training} and formulated as a constrained optimization problem. We will also theoretically analyze these practical OOD settings in our feature work.

% \section*{Acknowledgments}
\textbf{Acknowledgments.} 
This work is supported by National Key R\&D Program of China under Grant 2020AAA0105701, National Natural Science Foundation of China (NSFC) under Grants 61872327, Major Special Science and Technology Project of Anhui, National Natural Science Foundation of China (62033012) and Ant Group through Ant Research Intern Program.


%%%%%%%%% REFERENCES
{\small
\bibliographystyle{ieee_fullname}
\bibliography{egbib}
}

%\clearpage

\section{Appendix for Proofs}

\paragraph{Proof of Theorem \ref{thm:main}.}

\begin{proof}
\label{proof:main}
Our proof has two steps. In Step 1, we will show that SimCLR is equivalent to minimizing the cross entropy loss defined in Eqn.~(\ref{eqn:cross-entropy}). 
In Step 2, we will show  that minimizing the cross-entropy loss 
is equivalent to spectral clustering on $\bfpi$. 
Combining the two steps together, we have proved our theorem. 

\textbf{Step 1: } SimCLR is equivalent to minimizing the cross entropy loss.

The cross-entropy loss takes expectation over 
$\bfW_\bfX\sim \mathbb{P}(\cdot ; \bfpi)$, 
which means $\bfW_\bfX$ has exactly one non-zero entry in each row $i$. By Lemma~\ref{lem:multinomial}, we know every row $i$ of $\bfW_\bfX$ is independent of other rows. Moreover, 
$\bfW_{\bfX,i}\sim \mathcal{M}(1, \bfpi_i/\sum_j \bfpi_{i,j})=\mathcal{M}(1, \bfpi_i)$, because $\bfpi_i$ itself is a probability distribution.
Similarly, we know $\bfW_\bfZ$ also has the row-independent property by sampling over $\mathbb{P}(\cdot;\bfK_\bfZ)$.
Therefore, by Lemma~\ref{lem:cross_split}, we know Eqn.~(\ref{eqn:cross-entropy}) is equivalent to:
\[
 -\sum_{i=1}^n \mathbb{E}_{\bfW_{\bfX,i}}[\log \mathbb{P}(\bfW_{\bfZ,i}=\bfW_{\bfX,i};\bfK_\bfZ)],
\]

This expression takes expectation over $\bfW_{\bfX,i}$ for the given row $i$. Notice that 
$\bfW_{\bfX,i}$ has exactly one non-zero entry, which equals $1$ (same for $\bfW_{\bfZ,i}$). 
As a result
we expand the above expression to be:
\begin{equation}
 -\sum_{i=1}^n \sum_{j\neq i} \Pr(\bfW_{\bfX,i,j}=1)\log \Pr(\bfW_{\bfZ,i,j}=1).
\label{eqn:detailed-expansion}    
\end{equation}


By Lemma~\ref{lem:multinomial}, $\Pr(\bfW_{\bfZ,i,j}=1)=\bfK_{\bfZ,i,j}/\|\bfK_{\bfZ,i}\|_1$ for $j\neq i$. Recall that $\bfK_\bfZ=(k(\bfZ_i-\bfZ_j))_{(i,j)\in[n]^2}$, which means 
$\bfK_{\bfZ,i,j}/\|\bfK_{\bfZ,i}\|_1=\frac{\exp(-\|\bfZ_i-\bfZ_j\|^2/{2\tau})}{\sum_{k\neq i}
\exp(-\|\bfZ_i-\bfZ_k\|^2/{2\tau})
}$ for $j\neq i$, when $k$ is the Gaussian kernel with variance $\tau$. 

Notice that $\bfZ_i=f(\bfX_i)$, so we know
\begin{equation}
-\log \Pr(\bfW_{\bfZ,i,j}=1)=
-\log \frac{\exp(-\|f(\bfX_i)-f(\bfX_j)\|^2/{2\tau})}{\sum_{k\neq i}
\exp(-\|f(\bfX_i)-f(\bfX_k)\|^2/{2\tau}),
}
\label{eqn:infonce-equivalence}    
\end{equation}


The right hand side is exactly the InfoNCE loss defined in Eqn.~(\ref{eqn:infonce}).
Inserting Eqn.~(\ref{eqn:infonce-equivalence}) into Eqn.~(\ref{eqn:detailed-expansion}), we get the SimCLR algorithm, which first samples augmentation pairs $(i,j)$ with $\Pr(\bfW_{\bfX,i,j}=1)$ for each row $i$, and then optimize the InfoNCE loss. 

\textbf{Step 2: } minimizing the cross entropy loss 
is equivalent to spectral clustering on $\bfpi$.


By Lemma~\ref{lem:convert_to_spectral}, we may further convert the loss to 
\begin{equation}
\label{eqn:main-theorem-repul-attr}
\min_{\bfZ}
-\sum_{(i,j)\in [n]^2} \mathbf{P}_{i,j}
\log k (\bfZ_i-\bfZ_j)+\log \mathbf{R}(\bfZ).
\end{equation}
Since $k$ is the Gaussian kernel, this reduces to \[
\min_\bfZ \mathrm{tr}(\bfZ^\top \mathbf{L}(\bfpi) \bfZ)
+\log \mathbf{R}(\bfZ),
\]

where we use the fact that $\mathbb{E}_{\bfW_\bfX\sim \mathbb{P}(\cdot; \bfpi)}[\mathbf{L}(\bfW_\bfX)]
=\mathbf{L}(\bfpi)
$, because the Laplacian operator is linear and $
\mathbb{E}_{\bfW_\bfX\sim \mathbb{P}(\cdot; \bfpi)}(\bfW_\bfX)=\bfpi
$.
\end{proof}

\paragraph{Proof of Theorem \ref{thm:clip}.}
\begin{proof}
Since $\bfW_\bfX\sim \mathbb{P}(\cdot;\bfpi_{\mathbf{A}, \mathbf{B}})$, we know 
$\bfW_\bfX$ has exactly one non-zero entry in each row, denoting the pair that got sampled. 
A notable difference compared to the previous proof is we now have $n_\mathcal{A}+n_\mathcal{B}$ objects in our graph. CLIP deals with this by taking a mini-batch of size $2N$, 
such that $n_\mathcal{A}=n_\mathcal{B}=N$, and adding the $2N$ InfoNCE losses together. We label the objects in $\mathcal{A}$ as $[n_\mathcal{A}]$, and the objects in $\mathcal{B}$ as $\{n_\mathcal{A}+1, \cdots, n_\mathcal{A}+n_\mathcal{B}\}$. 

Notice that $\bfpi_{\mathbf{A}, \mathbf{B}}$ is a bipartite graph, so the edges of objects in $\mathcal{A}$ will only connect to object in $\mathcal{B}$ and vice versa. We can define the similarity matrix in $\cZ$ as $\bfK_\bfZ$, 
where $\bfK_\bfZ(i, j+n_\mathcal{A})=\bfK_\bfZ(j+n_\mathcal{A},i)= k(\bfZ_i-\bfZ_j)$ for $i\in [n_\mathcal{A}], j\in [n_\mathcal{B}]$, and otherwise we set $\bfK_\bfZ(i,j)=0$. 
The rest is same as the previous proof. 
\end{proof}

\paragraph{Proof of Theorem \ref{thm:exponential}.}

\begin{proof}
\label{proof:exponential}
Since the objective function consists of a linear term combined with an entropy regularization, which is a strongly concave function, the maximization problem is a convex optimization problem. Owing to the implicit constraints provided by the entropy function, the problem is equivalent to having only the equality constraint. We then introduce the Lagrangian multiplier $\lambda$ and obtain the following relaxed problem:

$$
\widetilde{E}(\boldsymbol{\alpha})=\psi_{1}-\sum_{i=1}^n \alpha_{i} \psi_{i}+\tau \sum_{i=1}^n \alpha_{i}\log \alpha_{i}+\lambda\left(\boldsymbol{\alpha}^{\top} \mathbf{1}_n-1\right).
$$

As the relaxed problem is unconstrained, taking the derivative with respect to $\alpha_{i}$ yields

$$
\frac{\partial \widetilde{E}(\boldsymbol{\alpha})}{\partial \alpha_{i}}=-\psi_{i}+\tau\left(\log \alpha_{i}+\alpha_{i} \frac{1}{\alpha_{i}}\right)+\lambda=0.
$$

Solving the above equation implies that $\alpha_{i}$ takes the form
$
\alpha_{i}=\exp \left(\frac{1}{\tau} \psi_{i}\right) \exp \left(\frac{-\lambda}{\tau}-1\right).
$ Since $\alpha_{i}$ lies on the probability simplex, the optimal $\alpha_{i}$ is explicitly given by
$
\alpha^{*}_{i}=\frac{\exp \left(\frac{1}{\tau} \psi_{i}\right)}{\sum_{i^{\prime}=1}^n \exp \left(\frac{1}{\tau} \psi_{i^{\prime}}\right)} .
$ Substituting the optimal point into the objective function, we obtain
$$
\begin{aligned}
E\left(\boldsymbol{\alpha}^*\right)  &=\psi_1-\sum_{i=1}^n \frac{\exp \left(\frac{1}{\tau} \psi_{i}\right)}{\sum_{i^{\prime}=1}^n \exp \left(\frac{1}{\tau} \psi_{i^{\prime}}\right)} \psi_{i}+\tau \sum_{i=1}^n \frac{\exp \left(\frac{1}{\tau} \psi_{i}\right)}{\sum_{i^{\prime}=1}^n \exp \left(\frac{1}{\tau} \psi_{i^{\prime}}\right)}\log \frac{\exp \left(\frac{1}{\tau} \psi_{i}\right)}{\sum_{i^{\prime}=1}^n \exp \left(\frac{1}{\tau} \psi_{i^{\prime}}\right)} \\
& =\psi_1 - \tau \log \left(\sum_{i=1}^n \exp \left(\frac{1}{\tau} \psi_{i}\right)\right).
\end{aligned}
$$
Thus, the Lagrangian dual function is given by
\begin{equation*}
-E\left(\boldsymbol{\alpha}^*\right)= -\tau \log \frac{\exp \left(\frac{1}{\tau} \psi_{1}\right)}{\sum_{i=1}^n \exp \left(\frac{1}{\tau} \psi_{i}\right)}.\qedhere
\end{equation*}
\end{proof}



\section{More on Experiments} \label{section: experiment_details}

\paragraph{CIFAR-10 and CIFAR-100} CIFAR-10 ~\citep{krizhevsky2009learning} and CIFAR-100 ~\citep{krizhevsky2009learning} are well-known classic image classification datasets. Both CIFAR-10 and CIFAR-100 contain a total of 60k $32 \times 32$ labeled images of different classes, with 50k for training and 10k for testing. CIFAR-10 is similar to CIFAR-100, except there are 10 different classes in CIFAR-10 and 100 classes in CIFAR-100.

\paragraph{TinyImageNet} TinyImageNet ~\citep{le2015tiny} is a subset of ImageNet ~\citep{deng2009imagenet}. There are 200 different object classes in TinyImageNet, with 500 training images, 50 validation images, and 50 test images for each class. All the images in TinyImageNet are colored and labeled with a size of $64 \times 64$.

\textbf{Pseudo-code.} Algorithm \ref{alg:Training Procedure} presents the pseudo-code for our empirical training procedure.

\begin{algorithm}[!htbp]
\caption{Training Procedure}
\label{alg:Training Procedure}
\begin{algorithmic}[1]
\REQUIRE trainable encoder network $f$, batch size $N$, augmentation strategy \textit{aug}, loss function $L$ with hyperparameters \textit{args}
\FOR {sampled minibatch ${x_i}_{i=1}^N$}
\FORALL{$i \in { 1, ..., N }$}
\STATE draw two augmentations $t_i = \textit{aug}\left(x_i\right) $, $t_i' = \textit{aug}\left(x_i\right) $
\STATE $z_i = f\left(t_i\right)$, $z_i' = f\left(t_i'\right)$
\ENDFOR
\STATE compute loss $\mathcal{L} = L(N, z, z', \textit{args})$
\STATE update encoder network $f$ to minimize $\mathcal{L}$
\ENDFOR
\STATE \textbf{Return} encoder network $f$
\end{algorithmic}
\end{algorithm}

We also provide the pseudo-code for our core loss function used in the training procedure in Algorithm \ref{alg:Core loss}. The pseudo-code is almost identical to SimCLR's loss function, with the exception of an extra parameter $\gamma$.

\begin{algorithm}[!htbp]
\caption{Core loss function $\mathcal{C}$}
\label{alg:Core loss}
\begin{algorithmic}[1]
\REQUIRE batch size $N$, two encoded minibatches $z_1, z_2$, $\gamma$, temperature $\tau$
\STATE $z = \textit{concat}\left(z_1, z_2\right)$
\FOR {$i \in {1, ..., 2N }, j \in {1, ..., 2N}$ }
\STATE $s_{i,j} = \Vert z_i - z_j \Vert_2^{\gamma}$
\ENDFOR
\STATE \textbf{define} $l(i, j)$ \textbf{as} $l(i, j) = - \log \frac{exp\left(s_{i,j}/\tau \right)}{\sum_{k=1}^{2N} \mathbf{1}{[k \ne i]} exp\left(s{i, j} / \tau \right)} $
\STATE \textbf{Return} $\frac{1}{2N} \sum_{k=1}^N\left[l(i, i+N) + l(i+N, i)\right]$
\end{algorithmic}
\end{algorithm}

Utilizing the core loss function $\mathcal{C}$, we can define all kernel loss functions used in our experiments in Table \ref{table: loss definition}. For all $z_i \in z$ with even dimensions $n$, we define $z_{L_i} = z_i\left[0:n/2\right]$ and $z_{R_i} = z_i\left[n/2:n\right]$.

\begin{table}[ht]
\centering
\begin{tabular}{{@{}l|l@{}}}
Kernel  &  Loss function \\ \midrule
Laplacian & $\mathcal{C}\left(N, z, z', \gamma=1, \tau\right)$\\ \midrule
Sum       & $\lambda * \mathcal{C}\left(N, z, z', \gamma=1, \tau_1\right) + (1-\lambda) * \mathcal{C}\left(N, z, z', \gamma=2, \tau_2\right)$  \\ \midrule
Concatenation Sum&$\lambda * \mathcal{C}\left(N, z_L, z'_L, \gamma=1, \tau_1\right) + (1-\lambda) * \mathcal{C}\left(N, z_R, z'_R, \gamma=2, \tau_2\right)$\\ \midrule
$\gamma = 0.5$ & $\mathcal{C}\left(N, z, z', \gamma=0.5, \tau\right)$          \\ 

\end{tabular}

\caption{Definition of kernel loss functions in our experiments}
\label {table: loss definition}
\end{table}

\textbf{Baselines.} We reproduce the SimCLR algorithm using PyTorch Lightning~\citep{PytorchLightning}.

\textbf{Encoder details.}
The encoder $f$ consists of a backbone network and a projection network. We employ ResNet50~\citep{ResNet} as the backbone and a 2-layer MLP (connected by a batch normalization~\citep{ioffe2015batch} layer and a ReLU \cite{nair2010rectified} layer) with hidden dimensions 2048 and output dimensions 128 (or 256 in the concatenation kernel case).

\textbf{Encoder hyperparameter tuning.}
For each encoder training case, we randomly sample 500 hyperparameter groups (sample details are shown in Table \ref{table: Hyperparameter sample}) and train these samples simultaneously using Ray Tune ~\citep{RayTune}, with the ASHA scheduler~\citep{li2018massively}. Ultimately, the hyperparameter group that maximizes the online validation accuracy (integrated in PyTorch Lightning) within 5000 validation steps is chosen for the given encoder training case.

\begin{table}[ht]
\centering

\begin{tabular}{@{}l|l|l@{}}
\midrule
Hyperparameter  & Sample Range & Sample Strategy \\ \midrule
start learning rate & $\left[10^{-2}, 10\right]$ & log uniform \\ \midrule
$\lambda$       & $\left[0, 1\right]$ & uniform \\ \midrule
$\tau$, $\tau_1$, $\tau_2$ & $\left[0, 1\right]$ & log uniform \\ \midrule
\end{tabular}

\caption{Hyperparameters sample strategy}
\label {table: Hyperparameter sample}
\end{table}

\textbf{Encoder training.} 
We train each encoder using the LARS optimizer~\citep{LARSOptimizer}, LambdaLR Scheduler in PyTorch, momentum 0.9, weight decay $10^{-6}$, batch size 256, and the aforementioned hyperparameters for 400 epochs on a single A-100 GPU.

\textbf{Image transformation.} The image transformation strategy, including augmentation, is identical to the default transformation strategy provided by PyTorch Lightning.

\textbf{Linear evaluation.}
The linear head is trained using the SGD optimizer with a cosine learning rate scheduler, batch size 64, and weight decay $10^{-6}$ for 100 epochs. The learning rate starts at $0.3$ and ends at $0$.

\textbf{Moco Experiments.} We also tested our method based on MoCo~\citep{he2019moco}. The results are summarized in Table \ref{tab:results-moco}. Here we choose ResNet18~\citep{ResNet} as the backbone and set a temperature of $0.1$ as default. For our simple sum kernel, we set $\lambda=0.8$. The results show that our method outperforms the original MoCo method.

\begin{table}[thb]
\centering
\caption{MoCo Experiment Results on CIFAR-10 and CIFAR-100.}
\label{tab:results-moco}
\resizebox{\textwidth}{!}{%
\begin{tabular}{@{}c|ccc|ccc@{}}
\toprule
\multirow{3}{*}{Method} & \multicolumn{3}{c|}{CIFAR-10} & \multicolumn{3}{c}{CIFAR-100} \\ \cmidrule(lr){2-4} \cmidrule(lr){5-7} 
                        & 200 epochs & 400 epochs    & 1000 epochs   & 200 epochs & 400 epochs & 1000 epochs         \\ \midrule
MoCo (repro.)         & $76.41 \pm 0.12$    & $80.01 \pm 0.15$          & $84.45 \pm 0.08$    & $\mathbf{47.02 \pm 0.11}$ & $52.50 \pm 0.07$ & $57.62 \pm 0.15$            \\
\midrule
Laplacian Kernel        & ${78.09 \pm 0.10}$    & $\mathbf{83.85 \pm 0.09}$          & $\mathbf{88.34 \pm 0.16}$    & $46.12 \pm 0.22$   & $53.44 \pm 0.17$ & $59.10 \pm 0.14$        \\
Simple Sum Kernel & $\mathbf{78.12 \pm 0.15}$   & $83.23 \pm 0.18$ & $87.50 \pm 0.20$ & $46.65 \pm 0.06$ & $\mathbf{53.62 \pm 0.19}$ & $\mathbf{59.83 \pm 0.12}$\\
\bottomrule
\end{tabular}
}
\end{table}



\section{More Experiments on Synthetic Data}


Consider a scenario with $n$ clusters, each containing $k$ vertices. Let the probability of vertices $u$ and $v$ from the same cluster belonging to $\bfpi$ be $p$. Conversely, for vertices $u$ and $v$ from different clusters, let the probability of belonging to $\pi$ be $q$. We generate the graph $\bfpi$ randomly, based on $p$ and $q$. We experiment with values of $k=100$ and $n=6$ for ease of visualization, embedding all points in a two-dimensional space. Each vertex's initial position originates from a normal distribution. In each iteration, we sample a subgraph of $\bfpi$ uniformly, ensuring each vertex has an out-degree of $1$. We then optimize the corresponding vectors using InfoNCE loss with an SGD optimizer and iterate until convergence. Our experimental setup consists of an SGD learning rate of $1$, an InfoNCE loss temperature of $0.5$, and a batch size of $50$. We evaluate two scenarios with different $p$ and $q$ values: $p=1$, $q=0$, and $p=0.75$, $q=0.2$. The results of these experiments are visualized in Figure \ref{fig:vis-spectral-cluster}. The obtained embeddings exhibit the hallmark pattern of spectral clustering of graph $\bfpi$.

\begin{figure}[!tb]
\centering
\subfigure{
\includegraphics[width=1\textwidth]{Figures/cluster_pi.png}
\label{fig:vis-cluster}
}
\subfigure{
\includegraphics[width=1\textwidth]{Figures/noised_cluster_pi.png}
\label{fig:vis-noised-cluster}
}
\caption{Visualizations of the optimization process using InfoNCE Loss on the vectors corresponding to $\bfpi$. Points of identical color belong to the same cluster within $\bfpi$. To showcase the internal structure of $\bfpi$, we randomly select 10 vertices from each cluster to display the edge distribution of $\bfpi$.}
\label{fig:vis-spectral-cluster}
\end{figure}



\end{document}


\title[Division rings for virtually compact special and $3$-manifold groups]{Division rings for group algebras of virtually compact special groups and $3$-manifold groups}
\author{Sam P.~ Fisher}
\address[S.~P.~Fisher]{University of Oxford, Oxford, OX2 6GG, UK}
\email{sam.fisher@maths.ox.ac.uk}

\author{Pablo S\'anchez-Peralta}
\address[P.~ S\'anchez-Peralta]{Universidad Aut\'onoma de Madrid, Madrid, Spain}
\email{pablo.sanchezperalta@uam.es}


\begin{document}

\maketitle


\begin{abstract}
    Let $k$ be a division ring and let $G$ be either a torsion-free virtually compact special group or a torsion-free $3$-manifold group. We embed the group algebra $kG$ in a division ring and prove that the embedding is Hughes-free whenever $G$ is locally indicable. In particular, we prove that Kaplansky's zerodivisor conjecture holds for all group algebras of torsion-free $3$-manifold groups. The embedding is also used to confirm a conjecture of Kielak and Linton. Thanks to the work of Jaikin-Zapirain, another consequence of the embedding is that $kG$ is coherent whenever $G$ is a virtually compact special one-relator group.
    
    If $G$ is a torsion-free one-relator group, let $\overline{kG}$ be the division ring containing $kG$ constructed by Lewin and Lewin. We prove that $\overline{kG}$ is Hughes-free whenever a Hughes-free $kG$-division ring exists. This is always the case when $k$ is of characteristic zero; in positive characteristic, our previous result implies this happens when $G$ is virtually compact special.
\end{abstract}




\section{Introduction}

The Kaplansky zerodivisor conjecture asserts that $kG$ is a domain, where $k$ is a division ring and $G$ is a torsion-free group. One of the first approaches to the zerodivisor conjecture was to embed $kG$ in a division ring; this strategy was successfully implemented by Mal'cev and Neumann, independently, in the case where $G$ is bi-orderable \cite{Malcev_series, Neumann_series}, where the group algebra is embedded into the Mal'cev--Neumann division ring of power series with well-ordered supports. Note that while the Kaplansky zerodivisor conjecture is still wide open, there are also no counterexamples to the following, a priori stronger, conjecture.

\begin{conj}\label{conj:divRing}
    If $G$ is a torsion-free group and $k$ is a division ring, then $kG$ embeds into a division ring.
\end{conj}

Note that an even stronger form of \cref{conj:divRing} has been formulated by Jaikin-Zapirain in \cite[Conjecture 1]{Jaikin_oneRelCoherence}, where one asks for $kG$ to embed in a unique \textit{Linnell ring} (see \cref{rem:linnell} for a definition of the Linnell property).

When trying to prove the zerodivisor conjecture in characteristic zero, there are many analytic tools at our disposal. Most notably, one can use the well-developed arsenal of $L^2$-techniques to prove the strong Atiyah conjecture for a torsion-free group $G$, which Linnell showed to be equivalent to the statement that the division closure of $\Q G$ in $\mathcal U(G)$ is a division ring, where $\mathcal U(G)$ denotes the algebra of operators affiliated to the $G$-equivariant bounded operators on $L^2(G)$ (\cite{LinnellDivRings93}, and see \cite[Section 10]{Luck02} for relevant background). Thus, the strong Atiyah conjecture for a torsion-free group $G$ implies \cref{conj:divRing} when $k = \C$. The strong Atiyah conjecture is known for many notable classes of groups, including
\begin{enumerate}
    \item torsion-free groups in Linnell's class $\mathcal C$ \cite[Theorem 1.5]{LinnellDivRings93}, defined as the smallest class of groups containing all free groups and closed under elementary amenable extensions and directed unions. Linnell's class $\mathcal C$ was recently shown to contain all $3$-manifold groups by Kielak and Linton \cite{KielakLinton_3mfldAtiyah};
    \item the class of locally indicable groups, due to Jaikin-Zapirain and L\'opez-\'Alvarez \cite{JaikinLopez_Atiyah};
    \item the class of virtually compact special groups, as defined by Haglund and Wise \cite{HaglundWise_special}, which are groups that are virtually the fundamental group of a compact cube complex that avoids certain pathological hyperplane arrangements. This is due to Schreve \cite{Schreve_AtiyahVCS}.
\end{enumerate}
Our list is far from extensive, and we refer the reader to Jaikin-Zapirain's survey \cite{Jaikin_l2survey} for a good account of what is known about the Atiyah conjecture.


The purpose of this article is to extend some of these embedding results beyond characteristic zero. Because of the lack of a suitable analogue of $\mathcal U(G)$ in positive characteristic, the methods we employ are necessarily more algebraic. A central concept will be that of a \textit{Hughes-free embedding}: if $G$ is locally indicable, we say that an embedding $\varphi \colon kG \hookrightarrow \mathcal D_{kG}$ is Hughes free if $\mathcal D_{kG}$ is a division ring generated by the image of $kG$ and it satisfies the following linear independence condition:
\begin{enumerate}
    \item[(HF)] whenever $H \leqslant G$ is finitely generated and $K \trianglelefteqslant H$ is such that $H/K = \langle tK \rangle \cong \Z$, then the set $\{\varphi(t^i) : i \in \Z\}$ is linearly independent over the division closure of $k[\ker \varphi]$ in $\mathcal D_{kG}$.
\end{enumerate}
In this case, $\mathcal D_{kG}$ is called a \textit{Hughes-free division ring} for $kG$. This notion will be recalled in more detail in \cref{sec:prelims}. The power of Hughes-free embeddings comes from Hughes' Theorem \cite{HughesDivRings1970}, which states that when a Hughes-free division ring $\mathcal D_{kG}$ exists, it is unique up to $kG$-isomorphism. Many classes of locally indicable groups are known to be \textit{Hughes-free embeddable} (i.e.~all of their group algebras have a Hughes-free embedding) and it is conjectured that group algebras of all locally indicable groups have Hughes-free embeddings that are universal in a suitable sense (see \cite[Theorem 1.1, Corollary 1.3, Conjecture 1]{JaikinZapirain2020THEUO}).

Our main result is the following.

\begin{thm}\label{thm:main}
    Let $k$ be a division ring. If $G$ is torsion-free and is either 
    \begin{enumerate}[label=(\arabic*)]
        \item\label{item:vcs} (\cref{cor:VCSinField}) virtually the fundamental group of a compact special cube complex;
        \item\label{item:3} (\cref{thm:3mfld,thm:nonori3mfld}) a finitely generated fundamental group of a $3$-manifold;
    \end{enumerate}
    then $kG$ embeds in a division ring. Moreover, if $G$ is locally indicable, then $kG$ embeds in a Hughes-free division ring and the finite generation assumption is not necessary in \ref{item:3}.
\end{thm}

Our proof of \ref{item:vcs} builds heavily on Schreve's proof of the strong Atiyah conjecture for virtually compact special groups \cite{Schreve_AtiyahVCS}, where he introduced the \textit{factorisation property}, the key tool which allows us to produce the embeddings. The first main ingredient in the proof of \ref{item:3} is a result of Friedl--Schreve--Tillmann \cite[Theorem 3.3]{FriedlSchreveTillmann_ThurstonFox} which, together with \cite[Theorem 1.1]{KielakLinton_3mfldAtiyah}, allows us to show that the fundamental group of any irreducible $3$-manifold that is not a closed graph manifold has the factorization property and that the group algebras of their fundamental groups embed in division rings (see \cref{thm:nonGraphDivRings}). The second main ingredient is the graph of rings construction studied in \cref{sec:GraphsRings}, which is what is needed to cover the case where $M$ is a closed graph manifold; namely, we prove the following combination theorem.

\begin{thm}[\cref{thm:graphOfHF}]
    Let $k$ be a division ring and let $G = \mathcal G_\Gamma = (G_v, G_e)$ be a graph of groups such that there is a Hughes-free embedding $kG_v \hookrightarrow \mathcal D_{kG_v}$ for every vertex $v$ of $\Gamma$. Then $kG$ embeds into a division ring.
\end{thm}

The case where $G$ is the fundamental group of a non-orientable $3$-manifold is not significantly different and is handled in \cref{sec:appendix} in order to keep the exposition as straightforward as possible.

The following corollary of \cref{thm:main} is immediate. We do not need to assume that the $3$-manifold groups are finitely generated since the zerodivisor conjecture can be verified locally.

\begin{cor}\label{cor:3mfldKap}
    Group algebras of torsion-free $3$-manifold groups satisfy Kaplansky's zerodivisor conjecture. 
\end{cor}

Note that Aschenbrenner--Friedl--Wilton asked whether $\Z G$ satisfies the zerodivisor conjecture when $G$ is the fundamental group of an irreducible, orientable, manifold with empty or toroidal boundary \cite[Question 7.2.6(2)]{AFW_3mfldBook}; this was confirmed by Kielak--Linton with their proof of the Atiyah conjecture for $3$-manifold groups \cite[Corollary 1.2]{KielakLinton_3mfldAtiyah} and \cref{cor:3mfldKap} extends this to positive characteristic. Of course, our result also implies the zerodivisor conjecture for torsion-free virtually compact special groups, however this can be deduced from the fact that they have the factorisation property by \cite[Corollary 4.3]{Schreve_AtiyahVCS} and \cite[Theorem 3.7]{FriedlSchreveTillmann_ThurstonFox}. Indeed since, virtually compact special groups are residually finite and have the factorisation property, they are fully residually (torsion-free elementary amenable), and group algebras of torsion-free elementary amenable groups satisfy the zerodivisor conjecture \cite{KropLinnellMoody_TFEAisOre}.


It is interesting to remark that in contrast to \cref{cor:3mfldKap}, Gardam's counterexample to Kaplansky's unit conjecture was $\mathbb F_2 G$, where $G \cong \Z^3 \rtimes (\Z/2 \oplus \Z/2)$ is isomorphic to the fundamental group of the Hantzsche-Wendt manifold, a flat $3$-manifold \cite{Gardam_units}. The result was extended by Murray who showed that the group algebra $\mathbb F_p G$ also contains non-trivial units for every prime $p$ \cite{murray2021counterexamples}.




\subsection*{Consequences of the existence of \texorpdfstring{$\mathcal D_{kG}$}{}}

Recently, Jaikin-Zapirain showed that if $k$ is a field of characteristic zero and $G$ is a torsion-free one-relator group, then the group algebra $kG$ is coherent (here \textit{coherent} means that every finitely generated ideal is finitely presented) \cite[Theorem 1.1]{Jaikin_oneRelCoherence}. Building on this breakthrough, Linton proved that one-relator groups are coherent \cite{Linton_oneRelCoherent}, confirming a conjecture of Baumslag \cite{Baumslag_OneRelProblems}. In Jaikin-Zapirain's result, the only reason one must assume that $k$ is of characteristic zero is that in this case the Atiyah conjecture for one-relator groups \cite{JaikinLopez_Atiyah} implies that a Hughes-free division ring $\mathcal D_{kG}$ exists. Thus, combining our construction of $\mathcal D_{kG}$ for $\operatorname{char}(k) > 0$ with Jaikin-Zapirain's arguments, we obtain the following.

\begin{cor}[\cref{cor:coherence}]
    Let $G$ be a virtually compact special one-relator group. Then the group ring $kG$ is coherent for any division ring $k$.
\end{cor}

Many one-relator groups are virtually compact special: in \cite[Theorem 8.2]{Linton_ORH}, Linton showed that every one-relator group $G$ with negative immersions is virtually compact special, and in \cite[Theorem 1.3]{LouderWilton_NegativeImmersions} Louder and Wilton characterise the one-relator groups with negative immersions as those with a presentation $G = F/\llangle w \rrangle$, where $w$ is a cyclically reduced word of primitivity rank $> 2$ (see \cref{def:PR}) in the free group $F$. Since $w$ being of primitivity rank $1$ implies that $G$ has torsion, the only case where we don't know whether $kG$ embeds into a Hughes-free division ring is when $w$ has primitivity rank $2$. The question as to the coherence of $\Z G$, even in the virtually compact special case, remains open.

In another direction, an embedding $kG \hookrightarrow \mathcal D_{kG}$ gives the Hughes-free division ring $\mathcal D_{kG}$ the structure of a $kG$-module and thus we can use it to compute $H_\bullet(G; \mathcal D_{kG})$ and $b_n^{\mathcal D_{kG}}(G) := \dim_{\mathcal D_{kG}} H_n(G; \mathcal D_{kG})$. These are examples of \textit{agrarian invariants} of the group $G$, which were first introduced and studied by Henneke and Kielak in \cite{HennekeKielak_agrarian}. When $k = \Q$ and $G$ is locally indicable, then $\mathcal D_{\Q G}$ always exists and coincides with the division closure of $\Q G$ in $\mathcal U(G)$ mentioned earlier. It then follows that $b_n^{(2)}(G) = b_n^{\mathcal D_{\Q G}}(G)$, so the $L^2$-Betti numbers of $G$ are examples of agrarian invariants.  If $k$ is a field of positive characteristic, then we think of the Betti numbers $b_n^{\mathcal D_{kG}}(G)$ as mod $p$ analogues of the usual $L^2$-Betti numbers. Indeed, in \cite{Fisher_improved} the Betti numbers $b_n^{\mathcal D_{kG}}(G)$ were shown to have many analogous properties to those of $L^2$-Betti numbers, in particular in how they control finiteness properties of kernels of algebraic fibrations and in \cite{FisherHughesLeary_artin} and \cite{AOS_hyperbolization} they were related to the mod $p$ homology growth of $G$.

Recently, Kielak and Linton proved the following embedding theorem for hyperbolic virtually compact special groups.

\begin{thm}[{\cite[Theorem 1.11]{KielakLinton_FbyZ}}]\label{thm:KLmain}
    Let $H$ be hyperbolic and virtually compact special with $\cd_\Q(H) \geqslant 2$. Then, there exists a finite index subgroup $L \leqslant H$ and a map of short exact sequences
    \[
        \begin{tikzcd}
            1 \arrow[r] & K \arrow[d, hook] \arrow[r] & L \arrow[d, hook] \arrow[r] & \Z \arrow[d, Rightarrow, no head] \arrow[r] & 1 \\
            1 \arrow[r] & N \arrow[r]                 & G \arrow[r]                 & \Z \arrow[r]                                & 1
        \end{tikzcd}
    \]
    such that
    \begin{enumerate}[label = (\arabic*)]
        \item $G$ is hyperbolic, compact special, and contains $L$ as a quasi-convex subgroup.
        \item $\cd_\Q(G) = \cd_\Q(H)$.
        \item $N$ is finitely generated.
        \item If $b_p^{(2)}(H) = 0$ for all $2 \leqslant p \leqslant n$, then $N$ is of type $\FP_n(\Q)$.
        \item If $b_p^{(2)}(H) = 0$ for all $p \geqslant 2$, then $\cd_\Q(N) = \cd_\Q(H) - 1$.
    \end{enumerate}
\end{thm}

As a consequence of this result, they are able to show among other things that one-relator groups with torsion are virtually free-by-cyclic \cite[Corollary 1.3]{KielakLinton_FbyZ}, confirming a conjecture of Baumslag. Much of Kielak and Linton's paper is written in the full generality of agrarian homology. However, to prove \cref{thm:KLmain}, they need to restrict themselves to $L^2$-homology as they make crucial use of Schreve's result that virtually compact special groups satisfy the Atiyah conjecture \cite{Schreve_AtiyahVCS}. As a consequence, for a torsion-free virtually compact special group $G$, Kielak and Linton can embed the group algebra $\Q G$ into its Linnell-skew field $\mathcal D_{\Q G}$ and use it in homological arguments to prove \cref{thm:KLmain}. They conjecture \cite[Conjecture 6.7]{KielakLinton_FbyZ} that \cref{thm:KLmain} remains true when every instance of $\Q$ is replaced with an arbitrary field $k$ and every $L^2$-Betti number $b_i^{(2)}$ is replaced with the agrarian Betti number $b_i^{\mathcal D_{kG}}$. Our construction of $\mathcal D_{kG}$ for $G$ torsion-free virtually compact special confirms their conjecture.

\begin{thm}[{\cref{thm:KLmain_agr}}]
    Let $k$ be a division ring. \cref{thm:KLmain} remains true when every instance of $\Q$ is replaced by $k$ and every $L^2$-Betti number $b_i^{(2)}$ is replaced by the agrarian Betti number $b_i^{\mathcal D_{kG}}$.
\end{thm}



\subsection*{Comparison with the Lewin--Lewin division ring}

In \cite{LewinLewinORTF}, Jacques and Tekla Lewin proved that if $k$ is a division ring and $G$ is a torsion-free one-relator group, then $kG$ embeds in a division ring, which we will denote $\overline{kG}$ and call the Lewin--Lewin division ring. This was the first proof that group algebras of torsion-free one-relator groups satisfy the Kaplansky zerodivisor conjecture; Brodski\u{\i}'s result stating that torsion-free one-relator groups are locally indicable \cite{BrodskiiOR} gives another proof. 

With the proof of the Atiyah conjecture for locally indicable groups, we know that group algebras of torsion-free one-relator groups have Hughes-free embeddings in characteristic zero and \cref{thm:main} shows that there are Hughes-free embeddings of virtually compact special torsion-free one-relator groups in positive characteristic as well. It is thus natural to compare the constructions of Hughes-free division rings with the Lewin--Lewin division ring. In the final section, we prove the following.


\begin{thm}[\cref{thm:LL_HF}]
    Let $G$ be a torsion-free one-relator group and let $k$ be a division ring such that $kG$ embeds into a Hughes-free division ring $\mathcal D_{kG}$. Then $\overline{kG} \cong \mathcal D_{kG}$ as $kG$-division rings.
\end{thm}


\subsection*{Organisation of the paper}

In \cref{sec:prelims} we recall some notions that will appear throughout the paper and prove some preliminary results about Hughes-free division rings. In \cref{sec:GraphsRings} we study graphs of rings and their relationship to the group algebra of a graph of groups; the results proved in this section are geared towards proving that the group algebra of a graph manifold embeds in a division ring. In \cref{sec:divRings} we use Schreve's factorisation property to prove that group algebras of torsion-free virtually compact special groups embed into a division ring. In \cref{sec:3mflds}, we show that the group algebra of a torsion-free fundamental group of an orientable $3$-manifold embeds in a division ring; this builds on the results of the two previous sections. In \cref{sec:KLconj} we use our construction of a division ring embedding $kG$ for $G$ torsion-free virtually compact special to confirm \cite[Conjecture 6.7]{KielakLinton_FbyZ}. In \cref{sec:LL}, we prove that the Lewin--Lewin construction of a division ring embedding for the group algebra of a torsion-free one-relator group is Hughes-free whenever a Hughes-free division ring exists. In \cref{sec:appendix}, we adapt the proof of the zerodivisor conjecture for orientable $3$-manifold groups to the non-orientable case.



\subsection*{Acknowledgments} 

We are grateful to Andrei Jaikin-Zapirain for many helpful conversations. We would also like to thank Piotr Przytycki and Henry Wilton for answering questions about $3$-manifold topology, and Dawid Kielak and Kevin Schreve for valuable comments on our article. The first author is supported by the National Science and Engineering Research Council (NSERC) [ref.~no.~567804-2022] and the European Research Council (ERC) under the European Union's Horizon 2020 research and innovation programme (Grant agreement No. 850930). The second author is supported by PID2020-114032GB-I00 of the Ministry of Science and Innovation of Spain.




\section{Preliminaries}\label{sec:prelims}

Throughout, rings are assumed to be associative and unital, and ring homomorphisms preserve the unit.

\subsection{Special groups}

Special cube complexes were introduced by Haglund and Wise in \cite{HaglundWise_special} as a class of cube complexes that avoid certain pathological hyperplane arrangements. One of the core features of a compact special cube complex $X$ is that it admits a $\pi_1$-injective combinatorial local isometry $X \looparrowright S_\Gamma$ to the Salvetti complex $S_\Gamma$ of the finitely generated right-angled Artin group (RAAG) $A_\Gamma$ (the graph $\Gamma$ is the \textit{hyperplane graph} of $X$) \cite[Theorem 1.1]{HaglundWise_special}. Thus, fundamental groups of compact special cube complexes inherit many remarkable algebraic properties from RAAGs; for example they are subgroups of $\SL_n(\Z)$, and in particular are residually finite.

Throughout the article, we will refer to groups that are fundamental groups of compact special cube complexes as \textit{compact special groups}. Agol's Theorem \cite[Theorem 1.1]{AgolHaken} states that a hyperbolic group $G$ acting properly and cocompactly on a $\mathrm{CAT}(0)$ cube complex $X$ contains a subgroup $H$ of finite index such that $X/H$ is compact special. In this sense, virtually compact special groups are abundant.


\subsection{Hughes-free division rings}

Let $k$ be a division ring and let $G$ be a group. 

A ring $S$ is $G$-{\it graded} if $S = \bigoplus_{g\in G} S_g$ as an additive group, where $S_g$ is an additive subgroup for every $g\in G$, and $S_g S_h \subseteq S_{gh}$ for all $g,h \in G$. If $S_g$ contains an invertible element $u_g$ for each $g\in G$, then we say that $S$ is a \textit{crossed product} of $S_e$ and $G$ and we shall denote it by $S=S_e * G$. Note that the usual group ring $RG$ of a group $G$ with $R$ a ring is an example of a crossed product.

An $R$-{\it ring} is a pair $(S,\varphi)$ where $\varphi \colon R\rightarrow S$ is a homomorphism. We will often omit $\varphi$ if it is clear from the context. An $R$-division ring $\varphi\colon R\rightarrow \mathcal{D}$ is called {\it epic} if $\varphi(R)$ generates $\mathcal{D}$ as a division ring.

\begin{defn}
    Let $G$ be a locally indicable group and let $k$ be a division ring. We say that a $k * G$-division ring $\varphi \colon k * G \rightarrow \mathcal D$ is \textit{Hughes-free} if it is epic and the following linear independence condition is satisfied:
    \begin{enumerate}
        \item[(HF)] whenever $H \leqslant G$ is finitely generated and $K \trianglelefteqslant H$ is such that $H/K = \langle tK \rangle \cong \Z$, then the set $\{\varphi(t^i) : i \in \Z\}$ is linearly independent over the division closure of $k[\ker \varphi]$ in $\mathcal D_{kG}$.
    \end{enumerate}
    In this situation, we will also say that $\mathcal D$ is a \textit{Hughes-free division ring of fractions} of $k * G$.
\end{defn}

\begin{rem}\label{rem:linnell}
    Note that a Hughes-free map $\varphi \colon k * G \rightarrow \mathcal D$ is always an embedding. Indeed, Gr\"ater showed that Hughes-free division rings are in fact \textit{Linnell rings} \cite[Corollary 8.3]{Grater20} (see also \cite[Proposition 2.2]{Jaikin_oneRelCoherence}). This means that the following stronger linear independence condition is satisfied:
    \begin{enumerate}
        \item[(L)] Let $H \leqslant G$ be any subgroup and let $T$ be a right transversal for $H$ in $G$. Then $T$ is linearly independent over the division closure of $kH$ in $\mathcal D_{kH}$.
    \end{enumerate}
    In particular, the set $\varphi(G)$ is linearly independent over $\varphi(k) \subseteq \mathcal D$, which implies that $\varphi$ is injective. 
    
    For a torsion-free group $G$ satisfying the Atiyah conjecture, the embedding of $\Q G$ into its division closure in $\mathcal U(G)$ is strongly Hughes-free. We emphasize that Hughes-free embeddings are defined only for locally indicable groups, but a strongly Hughes-free embedding can exist for the group algebra of any torsion-free group.
\end{rem}

Ian Hughes showed that when Hughes-free division rings exist, they are unique up to $k * G$-isomorphism \cite{HughesDivRings1970}. Thus, when it exists, we will always denote the Hughes-free division ring of $k * G$ by $\mathcal D_{k * G}$. If $H$ is a subgroup of $G$, note that the division closure of $k * H$ in $\mathcal D_{k*G}$ is Hughes-free as a $k*H$-division ring, and therefore we have a natural inclusion $\mathcal D_{k*H} \subseteq \mathcal D_{k*G}$. Since RAAGs are residually (torsion-free nilpotent) (\cite{Droms_thesis} and \cite{DuchampKrob_RAAGsRTFN}), so are their subgroups, and in particular so are compact special groups. Thus, Hughes-free division rings exist for compact special groups by the following result of Jaikin-Zapirain.

\begin{thm}[{\cite[Theorem 1.1]{JaikinZapirain2020THEUO}}]
    Let $G$ be a locally indicable amenable group, a residually (torsion-free nilpotent) group, or a free-by-cyclic group. Then $\mathcal D_{k * G}$ exists and it is universal.
\end{thm}

We refer the reader to \cref{subsec:slyv} for a definition of universality. We now prove two general Lemmas about Hughes-free division rings which will be useful to us in the later sections.

\begin{lem}\label{lem:twisted_ext}
    Let $G$ be a group and let $H \trianglelefteqslant G$ be a locally indicable normal subgroup. If $k * H$ has a Hughes-free division ring embedding $\varphi \colon k*H \hookrightarrow \mathcal D_{k*H}$, then we can form $\mathcal D_{k*H} * [G/H]$ and there is a natural embedding $k*G \cong (k*H) * [G/H] \hookrightarrow \mathcal D_{k*H} * [G/H]$.
\end{lem}

\begin{proof}
    The only potential obstruction to extending the crossed product structure of $(k*H) * [G/H]$ to $\mathcal D_{k*H} * [G/H]$ is extending the conjugation action of $G$ on $H$ to a $G$-action on all of $\mathcal D_{k*H}$. This is not a problem, however, because Hughes-free division rings are unique up to $k*H$-isomorphism. In more detail, let $\alpha \colon H \rightarrow H$ be any automorphism of $H$, and by abuse of notation write $\alpha$ for the induced automorphism of $k*H$. Then $\varphi$ and $\varphi \circ \alpha$ are both Hughes-free embeddings of $k * H$, and by uniqueness of Hughes-free embeddings, $\alpha$ extends to an automorphism $\alpha' \colon \mathcal D_{k*H} \rightarrow \mathcal D_{k*H}$ such that the diagram
    \[
        \begin{tikzcd}
            k*H \arrow[d, "\varphi", hook] \arrow[r, "\alpha", hook] & k*H \arrow[d, "\varphi", hook] \\
            \mathcal D_{k*H} \arrow[r, "\alpha'", hook] & \mathcal D_{k*H}              
        \end{tikzcd}
    \]
    commutes. \qedhere
\end{proof}

The following lemma will be key throughout the article, as it allows us to pass the Hughes-free property up to finite index overgroups.

\begin{lem}\label{lem:HF_fi}
    Let $G$ be a locally indicable group and let $H \trianglelefteqslant G$ be a subgroup of finite index. If $\mathcal{D}_{k*H}$ is a Hughes-free division $k*H$-ring of fractions and $\mathcal D_{k*H} * [G/H]$ is a domain, then $\mathcal D_{k*H} * [G/H]$ is a Hughes-free $k*G$-division ring of fractions.
\end{lem}

\begin{proof}
    Let $L$ be a finitely generated subgroup of $G$ and $N \trianglelefteqslant L$ such that $L/N=\langle tN \rangle \cong \Z$. Since $H\cap N\trianglelefteqslant L$, according to Lemma \ref{lem:twisted_ext} we can form $\mathcal{D}_{k*[H\cap N]} * [L/(H\cap N)]$. Moreover, we have the following natural isomorphism of crossed products
    \begin{align}\label{cross_prod_isom}
        \mathcal{D}_{k*[H\cap N]} * [L/(H\cap N)]&\cong (\mathcal{D}_{k*[H\cap N]} * [N/(H\cap N)]) * [L/N]
    \end{align}
    where $\mathcal{D}_{k*[H\cap N]} * [N/(H\cap N)]$ is obtained by similar considerations. 
    
    Now observe that since the automorphism of $k*[H\cap N]$ used in the construction of the crossed product ring is just a restriction of the conjugation automorphism of $k*H$, we have a natural embedding $\mathcal{D}_{k*[H\cap N]} * [N/(H\cap N)]\hookrightarrow \mathcal D_{k*H} * [G/H]$. Thus $\mathcal{D}_{k*[H\cap N]} * [N/(H\cap N)]$ is a domain. Moreover, $|N:H\cap N| \leqslant |G:H| < \infty$, so $\mathcal{D}_{k*[H\cap N]} * [N/(H\cap N)]$ is actually a division ring into which $kN$ embeds. 
    
    Let $\mathcal{D}_N$ and $\mathcal{D}_L$ be the division closures of $k*N$ and $k*L$ in $\mathcal{D}_{k*H} * [G/H]$, respectively. Then $\mathcal{D}_{k*[H\cap N]} * [N/(H\cap N)]\cong \mathcal{D}_{N}$ and by (\ref{cross_prod_isom}) the subring of $\mathcal{D}_L$ generated by $\mathcal{D}_N, t$ and $t^{-1}$ has a crossed product ring structure. This shows the $\mathcal{D}_{N}$-independence of the elements $\{t^i : i\in \Z\}$. \qedhere
\end{proof}



\subsection{Agrarian homology}

Let $R$ be a ring, let $G$ be a group, and let $\mathcal D$ be a division ring. If the group algebra $RG$ embeds into $\mathcal D$, then we say that $G$ is \textit{$\mathcal D$-agrarian over $R$} and that the embedding $RG \hookrightarrow \mathcal D$ is an \textit{agrarian embedding}. A nice immediate consequence of having an agrarian embedding is that $RG$ satisfies Kaplansky's conjecture on zerodivisors. As far as the authors are aware, there is not a single example of a torsion-free group and a division ring $k$ such that the group algebra $kG$ does \textit{not} have an agrarian embedding.

Suppose that $G$ is $\mathcal D$-agrarian over $R$. Then $\mathcal D$ is an $RG$-bimodule and we can define the \textit{$\mathcal D$-homology and cohomology} of $G$ by
\[
    H_\bullet (G; \mathcal D) \quad \text{and} \quad H^\bullet (G; \mathcal D)
\]
and the \textit{$\mathcal D$-Betti numbers} by
\[
    b_p^\mathcal D(G) = \dim_{\mathcal D} H_p(G; \mathcal D) \quad \text{and} \quad b_\mathcal D^p(G) = \dim_\mathcal D H^p (G; \mathcal D).
\]
The theory of homological agrarian Betti numbers was introduced by Henneke--Kielak in \cite{HennekeKielak_agrarian} in the case $R = \Z$ and was studied over other fields $R$ in the case where $\mathcal D$ is Hughes-free in \cite{Fisher_improved}. However, in this article we will be mostly concerned with agrarian cohomology. Thanks to \cite[Lemma 2.2]{KielakLinton_FbyZ}, $b_p^\mathcal D(G) = b^p_\mathcal D(G)$ whenever these quantities are finite (which occurs if, for instance, $G$ is of type $\F_\infty$ or more generally of type $\FP_\infty(R)$) and therefore we do not need to worry about the distinction between cohomological and homological $\mathcal D$-Betti numbers.

The following is the central example of agrarian homology.

\begin{ex}
    Let $G$ be a torsion-free group satisfying the strong Atiyah conjecture. Then $\Q G$ embeds into a division ring $\mathcal D_{\Q G}$ called the \textit{Linnell skew-field} (\cite{LinnellDivRings93} and \cite[Lemma 10.39]{Luck02}) and the agrarian Betti numbers $b_p^{\mathcal D_{\Q G}}(G)$ are equal to the $L^2$-Betti numbers $b_p^{(2)}(G)$. 
\end{ex}


\begin{prop}\label{prop:props_agr}
    Let $G$ be a group and let $k$ be a division ring such that $kG$ embeds into a division ring $\mathcal D$. Then
    \begin{enumerate}[label = (\arabic*)]
        \item\label{item:0} if $G$ is non-trivial, then $b_0^\mathcal D(G) = 0$;
        \item\label{item:euler} if $G$ is the fundamental group of a compact aspherical CW complex $X$, then $\chi(G) = \sum_{i = 0}^\infty b_i^\mathcal D(G)$;
        \item\label{item:scaling} if $\mathcal D = \mathcal D_{kG}$ is Hughes-free, then for every finite index subgroup $H \leqslant G$ we have $|G:H| \cdot b_p^{\mathcal D_{kG}}(G) = b_p^{\mathcal D_{kH}} (H)$ for all $p$.
    \end{enumerate}
\end{prop}

\begin{proof}
    \ref{item:0} This follows from considering the partial free resolution 
    \[
        \bigoplus_{g \in G} kG \xrightarrow{\bigoplus_{g \in G} (g - 1)} kG \xrightarrow{\alpha} k \rightarrow 0
    \]
    of the trivial $kG$-module $k$, and tensoring with $\mathcal D$ over $kG$. Here, $\alpha$ denotes the augmentation map.

    \ref{item:euler} This is proved as usual, i.e.~if $C_\bullet(\widetilde X; k)$ is the CW chain complex of $\widetilde X$ with coefficients in $k$, then we use the rank-nullity theorem from linear algebra and the fact that $\dim_\mathcal D \mathcal D \otimes_{kG} C_n(\widetilde X)$ is the number of $n$-cells in $X$.

    \ref{item:scaling} This essentially follows from the strong Hughes-freeness theorem of Gr\"ater \cite[Corollary 8.3]{Grater20}. For a detailed proof, see \cite[Lemma 6.3]{Fisher_improved}. \qedhere
\end{proof}

\begin{rem}
    \cref{item:scaling} is of \cref{prop:props_agr} is particularly useful, as it allows us to consistently define agrarian Betti numbers for groups $G$ containing a finite index subgroup $H$ such that $kH$ has a Hughes-free embedding.
\end{rem}



\subsection{Sylvester matrix rank functions}\label{subsec:slyv}

Let $R$ be a ring. A \textit{Sylvester matrix rank function} $\rk$ on $R$ is a function that assigns a non-negative real number to each matrix over $R$ and satisfies the following conditions:
\begin{enumerate}
    \item $\rk(A)=0$ if $A$ is any zero matrix and $\rk(1)=1$;
    \item $\rk(AB)\leq \min\{\rk(A),\rk(B)\}$ for any matrices $A$ and $B$ which can be multiplied;
    \item $\rk\begin{psmallmatrix}
         A & 0 \\
         0 & B
    \end{psmallmatrix}=\rk(A)+\rk(B)$ for any matrices $A$ and $B$;
    \item $\rk \begin{psmallmatrix}
         A & C \\
         0 & B
    \end{psmallmatrix} \geq \rk(A)+\rk(B)$ for any matrices $A,B$ and $C$ of appropriate sizes.
\end{enumerate}

We denote by $\mathbb{P}(R)$ the set of Sylvester matrix rank functions on $R$. Note that a ring homomorphism $\varphi\colon R\rightarrow S$ induces a map $\varphi^{\#}\colon \mathbb{P}(S)\rightarrow \mathbb{P}(R)$, that is, we can pull back any rank function $\rk$ on $S$ to a rank function $\varphi^{\#}(\rk)$ on $R$ by setting
\[
\varphi^{\#}(\rk)(A):=\rk(\varphi(A))
\]
for every matrix $A$ over $R$. We shall often abuse notation and write $\rk$ instead of $\varphi^{\#}(\rk)$ when it is clear that we are referring to the rank function on $R$.

A division ring $\mathcal{D}$ has a unique Sylvester matrix rank function which we denote by $\rk_{\mathcal{D}}$. Any Sylvester matrix rank function $\rk$ on $R$ that only takes integer values comes from a division ring by a result of P.~ Malcolmson \cite{Malcolmson}. Furthermore, there is a one-to-one correspondence between integer-valued rank functions and epic $R$-division rings.

\begin{lem}[{\cite[Corollary 3.1.15]{DLopezAlvarezThesis}}]\label{lem:equal_rk}
    Let $R$ be a ring, let $\mathcal{D}$, $\mathcal{E}$ be two epic $R$-division rings. Then $\mathcal{D}$ and $\mathcal{E}$ are $R$-isomorphic if and only if for every matrix $A$ over $R$ the induced rank functions on $R$ satisfy
    \[
        \rk_{\mathcal{D}}(A)=\rk_{\mathcal{E}}(A).
    \]
\end{lem}

We denote the set of integer-valued rank functions on a ring $R$ by $\mathbb{P}_{div}(R)$.

Given two Sylvester matrix rank functions on $R$, $\rk_1$ and $\rk_2$, we will write $\rk_1\leq \rk_2$ if for every matrix $A$ over $R$, $\rk_1(A)\leq \rk_2(A)$. This partial order structure shall play a key role.

A central notion in this theory is that of a universal $R$-division ring for a given ring $R$ (see, for instance, \cite[Section 7.2]{cohn06FIR}). In the language of Sylvester matrix rank functions, an epic $R$-division ring $\mathcal{D}$ is \textit{universal} if for every $R$-division ring $\mathcal{E}$, $\rk_{\mathcal{D}} \geqslant \rk_{\mathcal{E}}$. Note that a universal epic $R$-division ring, if it exists, is unique up to $R$-isomorphism and we denote it by $U(R)$.  





\section{Graphs of rings}\label{sec:GraphsRings}

In this section we introduce the notion of a \textit{graph of rings} and prove some of their basic properties. The amalgamated product of rings over a common subring has been studied extensively (see, for instance, \cite{cohn06FIR}) and the HNN extension of rings was defined and studied by Dicks in \cite{Dicks_HNN}. The upshot of this section is \cref{thm:graphOfHF}, which states that group algebras of graphs of Hughes-free embeddable groups embed in a division ring. Our motivation for defining graphs of rings is to prove the Kaplansky zerodivisor conjecture for group algebras of fundamental groups of graph manifolds, which are the compact, irreducible $3$-manifolds for which the factorisation property is not known. Moreover, a tree of rings will appear in \cref{sec:LL} when studying the Lewin--Lewin division ring.

We define graphs of rings in complete analogy with graphs of groups. We take graphs to be connected and oriented, with $\overline{e}$ denoting the same edge as $e$ but with the opposite orientation. Every edge $e$ has an origin vertex $o(e)$ and a terminus vertex $t(e)$ such that $o(e) = t(\overline e)$. Graphs are allowed to have loops and multiple edges.

\begin{defn}[The graph of rings with respect to a spanning tree]\label{def:treegraph}
    Let $\Gamma$ be a graph and let $T$ be a spanning tree. For each vertex $v$ of $\Gamma$ we have a \textit{vertex ring} $R_v$ and for each edge $e$ of $\Gamma$ we have an \textit{edge ring} $R_e$ and we impose $R_e = R_{\overline{e}}$ for every edge $e$. Moreover, for each (directed) edge $e$ there is an injective ring homomorphism $\varphi_e \colon R_e \rightarrow R_{t(e)}$. Then the \textit{graph of rings} $\mathscr R_{\Gamma, T} = (R_v, R_e)$ is the ring defined as follows:
    \begin{enumerate}
        \item for each edge of $e$ of $\Gamma$ we introduce formal symbols $t_e$ and $t_e\inv$;
        \item $\mathscr R_{\Gamma, T}$ is generated by the vertex rings $R_v$ and the elements $t_e, t_e\inv$ and subjected to the relations
        \begin{itemize}
            \item $t_{\overline e} = t_e\inv$;
            \item $t_e \varphi_{\overline{e}}(r) t_e\inv = \varphi_e(r)$ for all $r \in R_e$;
            \item if $e \in T$, then $t_e = 1$. 
        \end{itemize}
    \end{enumerate}
\end{defn}

Define $\mathscr R_\Gamma^*$ in the same way as $\mathscr R_{\Gamma, T}$, except drop the relations $t_e = 1$ if $e \in T$. There is a canonical quotient map $\pi_T \colon \mathscr R_T^* \rightarrow \mathscr R_{\Gamma, T}$.

\begin{defn}[The based graph of rings]\label{def:basedgraph}
    We retain all the notations of \cref{def:treegraph}. Fix a base vertex $v_0 \in \Gamma$. We say that an element of $\mathscr R_\Gamma^*$ is a \textit{loop element} if it is of the form $r_0 t_{e_1} r_1 t_{e_2} \cdots t_{e_n} r_n$ and
    \begin{enumerate}
        \item $r_0 \in R_{o(e_1)}$;
        \item $r_i \in R_{t(e_i)}$ for all $1 \leqslant i \leqslant n-1$ ;
        \item $t(e_i) = o(e_{i+1})$ for all $i$;
        \item $o(e_1) = t(e_n) = v_0$.
    \end{enumerate}
    We then define $\mathscr R_{\Gamma, v_0}$ to be the subring of $\mathscr R_\Gamma^*$ generated by the loop elements. Since the product of loop elements is clearly a loop element, $\mathscr R_{\Gamma,v_0}$ consists of the elements of $\mathscr R_\Gamma^*$ that can be expressed as sums of loop elements.
\end{defn}

\begin{rem}
    When defining a ring with generators and relations, we are quotienting a freely generated ring by an ideal. Thus, with these definitions, we of course run the risk that $\mathscr R_\Gamma^*, \mathscr R_{\Gamma, T}$, or $\mathscr R_{\Gamma, v_0}$ is zero, and that we have lost all information about the vertex and edge rings. This never happens in the graph of groups construction, but not much can be said for a general graph of rings. In the situations of interest, however, we will see that this does not happen, and that the vertex rings inject into the graph of rings (see \cref{lem:graphofgrouprings,prop:kGinjects}) as one would hope.
\end{rem}


The following result is the analogue of \cite[Ch.~1, \S 5.2, Proposition 20]{Serre_arbres}. In particular, it implies that the isomorphism types of $\mathscr R_{\Gamma,T}$ and $\mathscr R_{\Gamma,v_0}$ are independent of the choices of $T$ and $v_0$, respectively. We will thus simplify the notation and denote the graph of rings by $\mathscr R_\Gamma$.

\begin{prop}\label{prop:GORiso}
    Restricting the canonical projection $\pi_T \colon \mathscr R_\Gamma^* \rightarrow \mathscr R_{\Gamma, T}$ induces an isomorphism $\alpha := \pi_T|_{\mathscr R_{\Gamma, v_0}} \colon \mathscr R_{\Gamma, v_0} \rightarrow \mathscr R_{\Gamma, T}$.
\end{prop}

\begin{proof}
    The proof is analogous to that of \cite[Ch.~1, \S 5.2, Proposition 20]{Serre_arbres}, to which we refer the reader for more details. For every vertex $v$ of $\Gamma$, let $c_v = e_1 \cdots e_n$ be the geodesic path from $v_0$ to $v$ in $T$ and let $\gamma_v = t_{e_1} \cdots t_{e_n}$ be the corresponding element of $\mathscr R_\Gamma^*$. Put $x' = \gamma_v x \gamma_v\inv$ whenever $x \in R_v$ and $t_e' = \gamma_{o(e)} t_e \gamma_{t(e)}\inv$ for every edge $e$ of $\Gamma$. It is straightforward to show that the assignment $\beta(x) = x'$ and $\beta(t_e) = t_e'$ induces a well-defined homomorphism $\beta \colon \mathscr R_{\Gamma, T} \rightarrow \mathscr R_{\Gamma, v_0}$ such that $\alpha \circ \beta = \id$ and $\beta \circ \alpha = \id$. \qedhere
\end{proof}


When $G$ decomposes as graph of groups $\mathscr G_\Gamma$, the group ring $kG$ decomposes as a graph of rings in the expected way.

\begin{lem}\label{lem:graphofgrouprings}
    Let $G = \mathscr G_\Gamma = (G_v, G_e)$ be a graph of groups and let $k$ be a ring. Then the group ring $kG$ decomposes as a graph of rings $\mathscr R_\Gamma = (kG_v, kG_e)$, where the edge maps $kG_e \rightarrow kG_{t(e)}$ are induced by the edge maps $G_e \rightarrow G_{t(e)}$ of the graph of groups.
\end{lem}

\begin{proof}
    Let $v_0$ be a vertex in $\Gamma$; we work with the based graph of rings presentation for $\mathscr R_\Gamma$. Define a homomorphism $\alpha \colon kG \rightarrow \mathscr R_\Gamma$ as follows. Write $g \in G$ as a loop element $g_1 e_1 g_2 e_2 \cdots e_n g_n$ and put $\alpha(g) = g_1 e_1 g_2 e_2 \cdots e_n g_n \in \mathscr R_\Gamma$. This defines a homomorphism of $G$ into the unit group $\mathscr R_\Gamma^\times$, so $\alpha$ extends to a homomorphism $kG \rightarrow \mathscr R_\Gamma$ by $k$-linearity.

    On the other hand if we put $\beta(g_1 e_1 g_2 e_2 \cdots e_n g_n) = g_1 e_1 g_2 e_2 \cdots e_n g_n \in kG$ for a loop element $g_1 e_1 g_2 e_2 \cdots e_n g_n$, we also obtain a well-defined homomorphism $\beta \colon \mathscr R_\Gamma \rightarrow kG$, since the relations in $\mathscr R_\Gamma$ hold in $kG$ (by the based graph of groups presentation for $G$). \qedhere
\end{proof}


Let $k$ be a division ring and let $G = \mathscr G_\Gamma = (G_v, G_e)$ be a graph of locally indicable groups such that there is a Hughes-free embedding $G_v \hookrightarrow \mathcal D_{kG_v}$ for every vertex $v$ of $\Gamma$. Since the division closure of the image of $kG_e$ in both $\mathcal D_{kG_{o(e)}}$ and $\mathcal D_{kG_{v(e)}}$ are Hughes-free as $kG_e$-division rings, they are $kG_e$ isomorphic by uniqueness of Hughes-free division rings \cite{HughesDivRings1970}. Thus, we can form the \textit{graph of division rings} on $\Gamma$ with vertex division rings $\mathcal D_{kG_v}$ and edge division rings $\mathcal D_{kG_e}$; we denote it by $\mathscr{DG}_\Gamma$. Our next goal is to prove that $k\mathscr G_\Gamma$ embeds into $\mathscr{DG}_\Gamma$. For this, we will need the following normal form theorem.

\begin{thm}[{\cite[Theorems 34(i) and 35(i)]{Dicks_HNN}}]\label{thm:NF}
    \begin{enumerate}[label = (\arabic*)]
        \item\label{item:amalgNF} Let $B$ and $C$ be rings containing a common subring $A$ such that $B$ (resp.~$C$) is free as a left $A$-module with basis $\{1\} \sqcup X$ (resp.~$\{1\} \sqcup Y$). Then the amalgam $B *_A C$ is free as a left $B$-module on the set of sequences of strings $y_1 x_1 y_2 x_2 \cdots$ with $x_i \in X$ and $y_i \in Y$ not beginning with an element of $X$ and including the empty sequence.
        %
        \item\label{item:HNNNF} Let $B*_A$ be an HNN extension of rings with stable letter $t$ such that $B$ is free as a left $A$-module under both edge maps, with bases $\{1\} \sqcup X$ and $\{1\} \sqcup Y$.  Then $B*_A$ is free as a left $B$-module on the set of linked expressions constructed from 
        \[
            \begin{matrix}
                \ominus \ X  \ \oplus & \ominus \ Xt^{-1} \cup \{t\inv\} \ \ominus \\
                \oplus \ tY \cup \{t\} \ \oplus & \oplus \ tYt\inv \ \ominus
            \end{matrix}
        \]
        not beginning with an element of $X$ or $Xt\inv$ and including the empty sequence.
    \end{enumerate}
\end{thm}

A \textit{linked expression} is a word $a_1 a_2 a_3 \cdots$ such that if $a_i$ belongs to a set with a $\oplus$ (resp.~$\ominus$) to its right, then $a_{i+1}$ must belong to a set with a $\oplus$ (resp.~$\ominus$) to its left; we refer to \cite{Dicks_HNN} for a precise definition. Note that \ref{item:amalgNF} is deduced from earlier work of Cohn \cite{Cohnfreeproducts} or \cite{Bergman_Coprod}.

In the following lemma and its proof, all transversals that appear are assumed to contain the relevant group's unit element.

\begin{prop}\label{prop:kGinjects}
    Let $k$ be a division ring, let $G = \mathscr G_\Gamma = (G_v, G_e)$ be a graph of locally indicable groups such that there is a Hughes-free embedding $kG_v \hookrightarrow \mathcal D_{kG_v}$ for every vertex $v$ of $\Gamma$. Then the natural map $k\mathscr G_\Gamma \rightarrow \mathscr{DG}_\Gamma$ is an embedding.
\end{prop}

\begin{proof}
    First assume that $\Gamma$ is finite. We simultaneously prove the following pair of statements by induction on the number of edges in $\Gamma$:
    \begin{enumerate}
        \item $k\mathscr G_\Gamma \rightarrow \mathscr{DG}_\Gamma$ is an embedding, and
        \item for any vertex $v$ of $\Gamma$, there is a right transversal $T$ of $G_v$ in $G$ such that the image of $T$ in $\mathscr{DG}_\Gamma$ is linearly independent over $\mathcal D_{kG_v}$.
    \end{enumerate}
    If $\Gamma$ has no edges, then it consists of a single vertex and the claims are trivial. Now suppose that $\Gamma$ has at least one edge. Let $v$ be a vertex of $\Gamma$ and let $e$ be an edge such that $o(e) = v$. Assume that $\Gamma \smallsetminus e$ is disconnected with connected components $\Gamma_1$ and $\Gamma_2$, where $v \in \Gamma_1$. By induction, $k\mathscr G_{\Gamma_1}$ embeds in $\mathscr{DG}_{\Gamma_1}$ and there is a right transversal $T_1$ of $G_v$ in $\mathscr G_{\Gamma_1}$ which remains linearly independent over $\mathcal D_{kG_v}$. Let $S_1$ be a right transversal for (the image of) $G_e$ in $G_v$. By Gr\"ater's result that Hughes-free division rings are in fact Linnell rings \cite[Corollary 8.3]{Grater20} (see also \cite[Proposition 2.2]{Jaikin_oneRelCoherence}), we obtain that $S_1$ is also linearly independent over $\mathcal D_{kG_e}$ in $\mathcal D_{kG_v}$. Thus, $S_1 T_1$ is a right transversal for $G_e$ in $\mathscr G_{\Gamma_1}$ and it is linearly independent over $\mathcal D_{kG_e}$ in $\mathscr{DG}_{\Gamma_1}$.

    Moreover, we also have that $k\mathscr G_{\Gamma_2}$ embeds in $\mathscr{DG}_{\Gamma_2}$. By a similar argument, there is a transversal $T_2$ for $G_{t(e)}$ in $\mathscr G_{\Gamma_2}$ and a transversal $S_2$ for $G_e$ in $G_{t(e)}$ such that $T_2$ and $S_2 T_2$ are linearly independent over $\mathcal D_{kG_{t(e)}}$ and $\mathcal D_{kG_e}$, respectively.

    Let $X$ be the set of alternating expressions of the form $y_1 x_1 y_2 x_2 \cdots$ with $x_i \in S_1 T_1$ and $y_i \in S_2 T_2$ not beginning with an element of $S_1 T_1$. Note that $X$ is a right transversal for $\mathscr G_{\Gamma_1}$ in $\mathscr G$. By \cref{thm:NF}\ref{item:amalgNF}, we have that $kG$ embeds in $\mathscr{DG}_\Gamma$ and $X$ is linearly independent over $\mathscr{DG}_{\Gamma_1}$. To complete the induction, note that $T_1 X$ is a right transversal for $G_v$ in $\mathscr G_\Gamma$; by linear independence of $T_1$ and $X$ over $\mathcal D_{kG_v}$ and $\mathscr{DG}_{\Gamma_1}$, respectively, we conclude that $T_1 X$ is linearly independent over $\mathcal D_{kG_v}$.

    The case where $\Gamma \smallsetminus e$ is connected is proved similarly using \cref{thm:NF}\ref{item:HNNNF}; we omit the proof.

    We now drop the assumption that $\Gamma$ is finite. For a contradiction, assume that $k \mathscr G_\Gamma \rightarrow \mathscr{DG}_\Gamma$ is not injective. Let $x$ be a non-trivial element of the kernel and let $\Gamma' \subseteq \Gamma$ on which $x$ is supported. The image of $x$ in $\mathscr{DG}_\Gamma$ will be a finite linear combination of relators, which are supported in some finite subgraph $\Gamma'' \subseteq \Gamma$. Enlarging $\Gamma'$ and $\Gamma''$ if necessary, we may assume that $\Gamma' = \Gamma''$. But then $x$ is a non-trivial element of the kernel of $k\mathscr G_{\Gamma'} \rightarrow \mathscr{DG}_{\Gamma'}$, a contradiction. \qedhere
\end{proof}

The main result of this section now follows easily.

\begin{thm}\label{thm:graphOfHF}
    Let $k$ be a division ring and let $G = \mathscr G_\Gamma = (G_v, G_e)$ be a graph of locally indicable groups such that there is a Hughes-free embedding $kG_v \hookrightarrow \mathcal D_{kG_v}$ for every vertex $v$ of $\Gamma$. Then $kG$ embeds into a division ring.
\end{thm}

\begin{proof}
    We will prove that $\mathscr{DG}_\Gamma$ is a semifir, namely that all of its finitely generated left (or right) ideals are free of unique rank. The result then follows from \cref{prop:kGinjects} and Cohn's theorem stating that every semifir embeds into a division ring \cite[Corollary 7.5.14]{cohn06FIR}.

    We begin with the case that $\Gamma$ is finite. This follows by induction on the number of edges, using the facts that amalgams of semifirs over a division ring and HNN extensions of a semifir over a division ring are still semifirs (\cite{Cohnfreeproducts} and \cite[Theorems 34(ii) and 35(ii)]{Dicks_HNN}. 
    
    If $\Gamma$ is infinite, then the result follows since $\mathscr{DG}_\Gamma$ is the colimit of the semifirs $\mathscr{DG}_{\Gamma'}$ with $\Gamma'$ finite (see \cite[\S1.1 Exercise 3]{Cohn_FreeRingsRelations}). \qedhere
\end{proof}


\begin{q}
    Is the embedding constructed in \cref{thm:graphOfHF} Hughes-free when the graph of groups $\mathscr G_\Gamma$ is locally indicable?
\end{q}












\section{Division rings for groups with the factorisation property} \label{sec:divRings}

The following definition was introduced by Schreve in \cite{Schreve_AtiyahVCS}, where he used it to show that virtually compact special groups satisfy the strong Atiyah conjecture. This is a strengthening of the \textit{enough torsion-free quotients} property introduced by Linnell and Schick in \cite{LinnellSchick_AtiyahExt}, which they used to study when the Atiyah conjecture passes from a subgroup to a finite index overgroup.

\begin{defn}
    A group $G$ has the \textit{factorisation property} if for and every homomorphism $\alpha \colon G \rightarrow Q$ with $Q$ finite, there there is a torsion-free elementary amenable group $E$ such that $\alpha$ factors as $G \rightarrow A \rightarrow Q$.
\end{defn}



Recall that a group $G$ is \textit{good} (in the sense of Serre) if the restriction 
\[
    H_c^\bullet(\widehat G; M) \rightarrow H^\bullet (G;M)
\]
is an isomorphism for every finite $G$-module $M$, where $\widehat G$ denotes the profinite completion of $G$, and the cohomology on the left is continuous cohomology (see, for example, \cite{Serre_GaloisCohom}).


\begin{thm}\label{thm:VCS_field}
    Let $H$ be a finitely generated good group with the factorisation property and of finite cohomological dimension, and let $1 \rightarrow H \rightarrow G \rightarrow Q \rightarrow 1$ be a group extension with $G$ torsion-free and $Q$ finite. If $k$ is a division ring such that there is a Hughes-free embedding of $k * H$ into $\mathcal D_{k * H}$, then $k * G$ embeds into a division ring.
\end{thm}

\begin{proof}
    By \cite[Theorem 3.7]{FriedlSchreveTillmann_ThurstonFox}, $G$ has the factorisation property and therefore there is a normal subgroup $U \trianglelefteqslant G$ such that $U \leqslant H$ and $G/U$ is torsion-free and elementary amenable. By \cref{lem:twisted_ext}, we can form each of the following rings:
    \[
        \mathcal D_{k * U} * [H/U], \quad \mathcal D_{k * U} * [G/U], \quad \mathcal D_{k * H} * [G/H].
    \]
    Since $H/U$ and $G/U$ are torsion-free elementary amenable, according to \cite[Lemma 2.5]{LinnellSchick_AtiyahExt} $\mathcal D_{k * U} * [H/U]$ and $\mathcal D_{k * U} * [G/U]$ are Ore domains and so the diagram
    \[
        \begin{tikzcd}
            {\mathcal D_{k * U} * [H/U]} \arrow[r, hook] \arrow[d, hook] & {\mathcal D_{k * U} * [G/U]} \arrow[d, hook] \\
            {\Ore(\mathcal D_{k * U} * [H/U])} \arrow[r, hook] &{\Ore(\mathcal D_{k * U} * [G/U])}          
        \end{tikzcd}
    \]
    commutes. By Hughes-freeness of $\mathcal D_{k * H}$, the map $\mathcal D_{k * U} * [H/U] \rightarrow \mathcal D_{k * H}$ is an injection. This implies that $\Ore(\mathcal D_{k * U} * [H/U]) \cong \mathcal D_{k * H}$ by the universal property of Ore localisation.

    Consider the following diagram:
    \[
    \begin{tikzcd}[column sep = small]
        \mathcal D_{k * H} \arrow[r, "\cong", no head] \arrow[d, hook] & {\Ore(\mathcal D_{k * U} * [H/U])} \arrow[r, hook] \arrow[d, hook] & {\Ore(\mathcal D_{k * U} * [G/U])} \arrow[d, "\cong", no head] \\
        {\mathcal D_{k * H} *[G/H]} \arrow[r, "\cong", no head]        & {\Ore(\mathcal D_{k * U} * [H/U]) * [G/H]} \arrow[r, "\cong", no head]        & {\Ore((\mathcal D_{k * U}*[H/U])*[G/H])} \nospacepunct{.}                    
    \end{tikzcd}
    \]
    The left and middle vertical maps are the obvious inclusions and the right vertical map is a standard isomorphism of crossed products. The two left isomorphisms come from the isomorphism $\Ore(\mathcal D_{k * U} * [H/U]) \cong \mathcal D_{k * H}$ discussed above. For the bottom right isomorphism, it is not hard to show that the natural map
    \[
        \Ore(\mathcal D_{k * U} * [H/U]) * [G/H] \rightarrow \Ore((\mathcal D_{k * U}*[H/U])*[G/H]),
    \]
    is injective. Therefore $\Ore(\mathcal D_{k * U} * [H/U]) * [G/H]$ a domain, which implies it is a division ring since $G/H$ is finite. This proves that $\mathcal D_{k * H} * [G/H]$ is a division ring, which clearly contains $k * G$. \qedhere
\end{proof}



\begin{cor}\label{cor:VCSinField}
    If $G$ is a torsion-free virtually compact special group, then any crossed product $k * G$ embeds into a division ring $\mathcal D$. Moreover, if $H$ is a normal, finite index, compact special subgroup of $G$, then the diagram
    \[
        \begin{tikzcd}
            k * H \arrow[r, hook] \arrow[d, hook] & {k * H * [G/H]} \arrow[d, hook] \arrow[r, "\cong", no head] & k * G \arrow[d, hook] \\
            \mathcal D_{k * H} \arrow[r, hook]    & {\mathcal D_{k * H} * [G/H]} \arrow[r, "\cong", no head]    & \mathcal D        
        \end{tikzcd}
    \]
    commutes. Moreover, if $G$ is locally indicable, then $\mathcal D = \mathcal D_{k * G}$.
\end{cor}

\begin{proof}
    Compact special groups are good and have the factorisation property \cite[Corollary 4.3]{Schreve_AtiyahVCS}. Moreover, compact special groups are finitely generated and have finite cohomological dimension, since they have finite classifying space. The claim then follows from the proof of \cref{thm:VCS_field}. The last statement about Hughes-freeness follows from \cref{lem:HF_fi}. \qedhere
\end{proof}

\begin{cor}
    Let $G$ be a torsion-free virtually compact special group and let $k$ be a division ring. Then the Kaplansky zero divisor conjecture holds for any crossed product $k * G$. \qedhere
\end{cor}

In \cite[Theorem 1.1]{Jaikin_oneRelCoherence}, Jaikin-Zapirain proves that $kG$ is coherent whenever $k$ is of characteristic $0$ and $G$ is a torsion-free one-relator group. The only reason the assumption that $k$ is of characteristic $0$ is needed is to ensure that a Hughes-free division ring $\mathcal D_{kG}$ exists, which follows from the fact that one-relator groups satisfy the Atiyah conjecture \cite{JaikinLopez_Atiyah}. From Jaikin-Zapirain's arguments and the existence of $\mathcal D_{kG}$ we prove coherence of more one-relator group algebras in positive characteristic.

\begin{cor}\label{cor:coherence}
    Let $G$ be a virtually compact special torsion-free one-relator group and let $k$ be any division ring. Then the group algebra $kG$ is coherent.
\end{cor}

Thanks to the work of Linton \cite[Theorem 8.2]{Linton_ORH}, we know that all one-relator groups with negative immersions are virtually compact special. Moreover, Louder and Wilton proved that $G$ has negative immersions if and only if the defining word of $G$ has primitivity rank at least $3$ (see \cref{def:PR}).


\section{Division rings for \texorpdfstring{$3$}{3}-manifold groups}\label{sec:3mflds}

Manifolds are always assumed to be connected. The goal of this section is to prove the following embedding theorem.

\begin{thm}\label{thm:3mfld}
    Let $G$ be the fundamental group of an orientable $3$-manifold $M$, let $k$ be a division ring, and assume that $G$ is torsion-free. If $G$ is finitely generated, then $kG$ embeds into a division ring. If $G$ is locally indicable (and not necessarily finitely generated), then $kG$ embeds into a Hughes-free division ring.
\end{thm}

\begin{rem}
    \cref{thm:3mfld} remains true if we drop the assumption of orientability, though we will first prove the theorem as stated here in order to keep the exposition as straightforward as possible. In \cref{sec:appendix}, we will show how the methods used here can be extended to the non-orientable case by using the non-orientable versions of the Prime and JSJ Decomposition Theorems.
\end{rem}

\begin{cor}\label{cor:3mfldKapl}
    Let $M$ be an orientable $3$-manifold with torsion-free fundamental group $G$ and let $k$ be a division ring. Then $kG$ satisfies Kaplansky's zerodivisor conjecture.
\end{cor}

\begin{proof}
    Kaplansky's zerodivisor conjecture is a local condition, so we only need to check that it holds for finitely generated subgroups of $G$. But these subgroups are just torsion-free, finitely generated $3$-manifold groups, for which the Kaplansky zerodivisor conjecture holds by \cref{thm:3mfld}. \qedhere
\end{proof}

\begin{rem}
    The Atiayh conjecture is now known for all $3$-manifold groups thanks to the work of Friedl--L\"uck and Kielak--Linton \cite{FriedlLuck_euler,KielakLinton_3mfldAtiyah}, building on the results on $3$-manifold fibring of Agol, Liu, Przytycki, and Wise \cite{AgolHaken,Liu_npcGraph,PrzWise_GraphSpecial,PrzWise_MixedSpecial}. It follows that \cref{thm:3mfld,cor:3mfldKapl} are already known to hold when $k = \C$.
\end{rem}

We offer the following extension of \cite[Theorem 3.3]{FriedlSchreveTillmann_ThurstonFox}, where it is shown the $3$-manifold groups below have the factorisation property provided the boundary is empty or toroidal. We also show that in all cases where the factorisation property is satisfied, then the group algebra embeds into a division ring.

\begin{thm}\label{thm:nonGraphDivRings}
    Let $k$ be a division ring and let $M$ be a compact, irreducible, aspherical $3$-manifold with (possibly empty) incompressible boundary. Suppose that
    \begin{enumerate}[label=(\arabic*)]
        \item $M$ has non-empty boundary,
        \item $M$ is not a closed graph manifold, or
        \item $M$ is a non-positively curved graph manifold or is Nil, Sol, or Seifert fibred.
    \end{enumerate}
    Then $G = \pi_1(M)$ has the factorisation property and $kG$ embeds in a division ring. Moreover, if $G$ is locally indicable then $kG$ embeds in a Hughes-free division ring.
\end{thm}

\begin{proof}
    First assume that $M$ has non-empty incompressible boundary. Since $M$ is aspherical, $M$ is a $K(G,1)$ space and therefore $\cd_\Q(G) < 3$, implying that there is a finite index subgroup $H \trianglelefteqslant G$ that is free-by-cyclic by \cite[Theorem 3.3]{KielakLinton_3mfldAtiyah}. By \cite[Lemma 2.4]{Schreve_AtiyahVCS} free-by-cyclic groups have the factorisation property, and \cite[Proposition 3.7]{FriedlSchreveTillmann_ThurstonFox} implies that $G$ also has the factorisation property. The subgroup $H$ is Hughes-free embeddable by \cite[Theorem 1.1]{JaikinZapirain2020THEUO}. By \cref{thm:VCS_field}, $kG$ can be embedded in a division ring, and by \cref{lem:HF_fi} the embedding is Hughes-free if $G$ is locally indicable.

    If $M$ is a closed non-graph manifold, then $M$ is virtually fibred by the work of Agol and Przytycki--Wise \cite{AgolHaken,PrzWise_MixedSpecial}. In particular, $G$ has a surface-by-cyclic subgroup of finite index (where the surface may have nontrivial boundary). Hence, $G$ has the factorisation property by \cite[Propositions 3.6 and 3.7]{FriedlSchreveTillmann_ThurstonFox}. In the case where $G$ contains a free-by-cyclic group, then, as we showed in the previous paragraph, $kG$ can be embedded in a division ring, and the embedding can be made Hughes-free if $G$ is locally indicable. Now suppose $G$ contains a finite index subgroup of the form $S \rtimes \Z$, where $S$ is the fundamental group of a closed surface. Note that $S$ is itself free-by-cyclic, so it is Hughes-free embeddable. Therefore $S \rtimes \Z$ is Hughes-free embeddable by \cite{HughesDivRings1970_2}. Thus, we conclude that $kG$ can be embedded in a division ring (which is Hughes-free if $G$ is locally indicable) by \cref{thm:VCS_field} and \cref{lem:HF_fi}.

    If $M$ is a non-positively curved graph manifold, then it is virtually fibred over the circle by a theorem of Svetlov \cite{Svetlov_npcGraph} (which was strengthened by Liu \cite{Liu_npcGraph}). The desired conclusions then follow from the same arguments as above. If $M$ is Nil, Sol, or Seifert fibred, then $G$ has the factorisation property by \cite[Lemmas 3.8 and 3.9]{FriedlSchreveTillmann_ThurstonFox}. The fundamental groups of Nil and Sol manifolds are torsion-free elementary amenable, and therefore $kG$ is an Ore domain, so $\Ore(kG)$ is Hughes-free.

    Finally, we treat the case where $M$ is a closed Seifert fibred manifold. The argument is inspired by \cite[Theorem 3.2(3), argument (c)]{FriedlLuck_euler}. By \cite[Flowchart 1, (C.19)]{AFW_3mfldBook}, there is a locally indicable subgroup $H \trianglelefteqslant G$ of finite index. Let $N \rightarrow M$ be the corresponding finite cover. Let $\varphi \colon H \rightarrow \Z$ be a map, and let $\widetilde N \rightarrow N$ be the infinite cyclic cover. Using the Scott Core Theorem \cite{Scott_core}, we write $\widetilde N = \bigcup_i N_i$ as the a directed union of $\pi_1$-injective Seifert fibred manifolds $N_i$ with boundary. As we have seen above, each $k[\pi_1(N_i)]$ embeds in a Hughes-free division ring $\mathcal D_{k [\pi_1(N_i)]}$. It then follows that $k[\pi_1(\widetilde N)] = \bigcup_i k[\pi_1(N_i)]$ embeds in the Hughes-free division ring 
    \[
        \mathcal D_{k[\pi_1(\widetilde N)]} = \bigcup_i \mathcal D_{k[\pi_1(N_i)]}.
    \]
    Then $H \cong \pi_1(\widetilde N) \rtimes \Z$ is Hughes-free embeddable by \cite{HughesDivRings1970_2}. Since $G$ has the factorisation property, $kG$ embeds in a division ring by \cref{thm:VCS_field}, which is Hughes-free if $G$ is locally indicable by \cref{lem:HF_fi}. \qedhere
\end{proof}


The factorisation property is not known to hold for the fundamental group of a general closed graph manifold, so we take a different approach than the one used in \cref{thm:nonGraphDivRings}, which uses the graph of rings construction introduced in \cref{sec:GraphsRings}.


\begin{thm}\label{thm:graphMfld}
    Let $k$ be a division ring and let $M$ be a closed graph manifold with torsion-free fundamental group $G$. Then $kG$ embeds into a division ring. If $G$ is locally indicable, then $kG$ embeds into a Hughes-free division ring.
\end{thm}

\begin{proof}
    By definition, the JSJ decomposition of $M$ has only Seifert fibred pieces (see \cite[Theorem 1.6.1]{AFW_3mfldBook} for a statement of the JSJ Decomposition Theorem). If $M$ has only one JSJ component, then the claim follows from \cref{thm:nonGraphDivRings}.

    Now suppose that $M$ has JSJ components $M_1, \dots, M_n$ with $n \geqslant 2$. The manifolds $M_i$ are all Seifert fibred manifolds with nonempty toroidal boundary. By \cite[Flowchart 1, (C.19)]{AFW_3mfldBook}, $\pi_1(M_i)$ is locally indicable for each $i$, and therefore $k[\pi_1(M_i)]$ embeds into a Hughes-free division ring $\mathcal D_{k[\pi_1(M_i)]}$ by \cref{thm:nonGraphDivRings}. Since $G$ decomposes as a graph of groups with Hughes-free embeddable vertex groups, \cref{thm:graphOfHF} implies that $kG$ embeds into a division ring.

    We do not know whether the division ring obtained from \cref{thm:graphOfHF} is Hughes-free when $G$ is locally indicable, however an argument similar to the one given at the end of the proof of \cref{thm:nonGraphDivRings} works. Let $\varphi \colon G \rightarrow \Z$ be any homomorphism and let $\widetilde M \rightarrow M$ be the corresponding infinite cyclic covering. We can write $\widetilde M$ as a directed union of $\pi_1$-injective graph manifolds with boundary, whose fundamental groups are all Hughes-free embeddable. Then $\pi_1(\widetilde M)$ is Hughes-free embeddable, and therefore so is $\pi_1(\widetilde M) \rtimes \Z \cong G$ by \cite{HughesDivRings1970_2}. \qedhere
\end{proof}


The final ingredient we will need to prove \cref{thm:3mfld} is the following form of the Prime Decomposition Theorem \cite[Theorem 1.2.1 and Lemma 1.4.2]{AFW_3mfldBook}. We refer the reader to the proof of \cite[Proposition 3.1]{KielakLinton_3mfldAtiyah}, where it is shown that the conclusion holds virtually. However, the need to pass to a finite index subgroup is only due to the fact that the assumptions of orientability and torsion-freeness are dropped.

\begin{prop}\label{prop:3decomp}
    Let $M$ be an orientable $3$-manifold whose fundamental group $G$ is finitely generated and torsion-free. Then there is a free group $F$ and finitely many compact, orientable, irreducible, aspherical $3$-manifolds $M_1, \dots, M_n$ each with (possibly empty) incompressible boundary such that
    \[
        G \cong F * \pi_1(M_1) * \cdots * \pi_1(M_n).
    \]
\end{prop}




We are now ready to prove the main theorem.

\begin{proof}[Proof of \cref{thm:3mfld}]
    Let $G \cong F * \pi_1(M_1) * \cdots * \pi_1(M_n)$ be the decomposition given by \cref{prop:3decomp}. The group algebra of $kF$ embeds into $\mathcal D_{kF}$, and the group algebras of the fundamental groups $\pi_1(M_i)$ all embed into division rings $\mathcal D_i$ by \cref{thm:nonGraphDivRings,thm:graphMfld}. Thus, $kG$ embeds into the amalgam of $\mathcal D_{kF}$ and the division rings $\mathcal D_{k[\pi_1(M_i)]}$ over the common sub-division ring $k$. By \cref{thm:2semifir_semifir,thm:semifirDivRing}, this amalgam in turn embeds into a division ring.

    Now suppose that $G$ is locally indicable. Then the group algebras $k[\pi_1(M_i)]$ all embed into Hughes-free division rings by \cref{thm:nonGraphDivRings,thm:graphMfld}. Then $kG$ embeds into a Hughes-free division ring by \cite[Corollary 6.13(iv)]{JSanchezThesis}. In the case where $G$ is locally indicable but not necessarily finitely generated, then every finitely generated subgroup of $G$ is a finitely generated locally indicable $3$-manifold and is thus Hughes-free embeddable. But then $G$ is the direct union of Hughes-free embeddable groups and is therefore itself Hughes-free embeddable. \qedhere
\end{proof}
























\section{The Kielak--Linton conjecture} \label{sec:KLconj}

Let $G$ be a torsion-free virtually compact special group. Since $G$ is not necessarily locally indicable, a Hughes-free division ring $\mathcal D_{kG}$ may not exist. However, by \cref{prop:props_agr}, we may simply define $b_p^{\mathcal D_{kG}}(G) := \frac{1}{|G:H|} b_p^{\mathcal D_{kH}}(H)$, where $H \leqslant G$ is a compact special subgroup of finite index. We record the following easy observation that will be useful later.

\begin{lem}\label{lem:eitherField}
    Let $k$ be a division ring and let $G$ be a torsion-free virtually compact special group and let $\mathcal D$ be the division ring containing $kG$ constructed in \cref{thm:VCS_field}. Then $b_p^\mathcal D(G) = b_p^{\mathcal D_{kG}}(G)$ for all $p$.
\end{lem}

\begin{proof}
    Let $H \trianglelefteqslant G$ be a compact special subgroup of finite index. The claim then follows quickly from the definition above and the fact that $\mathcal D \cong \mathcal D_{kH} * [G/H] \cong \bigoplus_{|G:H|}\mathcal D_{kH}$ as $\mathcal D_{kH}$-modules. \qedhere
\end{proof}





\begin{lem}\label{lem:pair_of_free}
    Let $G$ be a non-elementary hyperbolic group, let $k$ be a division ring, and suppose there is an embedding of $kG$ into a division ring $\mathcal D$. Then there exists non-isomorphic quasi-convex free subgroups $A,B \leqslant G$ such that $A$ is malnormal, $A \cap B^g = \{1\}$ for all $g \in G$, and the restriction
    \[
        H^1(G; \mathcal D) \rightarrow H^1(B; \mathcal D)
    \]
    is an isomorphism.
\end{lem}

\begin{proof}
    This is \cite[Corollary 5.7]{KielakLinton_FbyZ}, and the proof is similar: it only relies on an ``agrarian Freiheitssatz" which they prove for arbitrary agrarian embeddings \cite[Theorem 3.1]{KielakLinton_FbyZ}. \qedhere
\end{proof}

\begin{lem}\label{lem:agr_ind}
    Let $G = A*_C$ where $A$ and $C$ are locally indicable and finitely generated. Moreover, suppose that $k$ is such that $\mathcal D_{kA}$ exists and $kG$ embeds into a division ring $\mathcal D \supseteq \mathcal D_{kA}$ making the diagram
    \[
        \begin{tikzcd}
            kA \arrow[r, hook] \arrow[d, hook] & kG \arrow[d, hook] \\
            \mathcal D_{kA} \arrow[r, hook]    & \mathcal D
        \end{tikzcd}
    \]
    commute. If the restriction $H^1(A; \mathcal D_{kA}) \rightarrow H^1(C; \mathcal D_{kA})$ is surjective, then the restriction
    \[
        H^2(G; \mathcal D) \rightarrow H^2(A; \mathcal D)
    \]
    is injective.
\end{lem}

\begin{proof}
    Let $H \leqslant G$ and write $\mathcal D(H)$ for the division closure of $kH$ in $\mathcal D$. Then the proof is the same as in \cite[Proposition 4.8]{KielakLinton_FbyZ}, where one must replace every occurrence of $\mathcal D_{\Q H}$ with $\mathcal D(H)$. \qedhere
\end{proof}

The following proposition corresponds to Proposition 6.4 of \cite{KielakLinton_FbyZ}, where the result is proven in the case $k = \Q$. The obstacle Kielak and Linton faced in their paper was the fact that they didn't have access to division rings containing group rings of torsion-free virtually compact special groups. We repeat their proof, since this is the crucial step where the existence of a division ring embedding the group algebra of a torsion-free virtually compact special group is used.

\begin{prop}\label{prop:HNN}
    Let $H$ be non-free, torsion-free, hyperbolic, and compact special, and suppose that $b_1^{\mathcal D_{kH}}(H) \neq 0$. Then there is a hyperbolic and virtually compact special HNN extension $G = H*_F$ such that the embeddings of $F$ are quasi-convex in $G$ and such that
    \[
        b_p^{\mathcal D_{kG}}(G) = \begin{cases}
            0 & \text{if} \ p = 1 \\
            b_p^{\mathcal D_{kH}}(H) & \text{if} \ p \neq 1.
        \end{cases}
    \]
    Moreover, $H$ is quasi-convex in $G$ and $\cd_k(G) = \cd_k(H)$.
\end{prop}

\begin{proof}
    By \cref{lem:pair_of_free}, there is a pair of isomorphic free quasi-convex subgroups $A,B \leqslant H$ such that $A$ is malnormal and intersects every conjugate of $B$ trivially and the restriction 
    \[
        H^1(H; \mathcal D_{kH}) \rightarrow H^1(B; \mathcal D_{kH})
    \]
    is an isomorphism.
    
    Now let $f \colon A \rightarrow B$ be any isomorphism and let $G = H*_A$ be the corresponding HNN extension. By \cite[Theorem 6.3]{KielakLinton_FbyZ}, $G$ is virtually compact special. Moreover, since $G$ is the HNN extension of a torsion-free group, it is also torsion-free, and therefore $kG$ embeds in a division ring $\mathcal D$ by \cref{cor:VCSinField}. From \cite[Theorem 3.1]{BieriHNN}, there is a long exact sequence
    \[
        \cdots \rightarrow H^p(G; \mathcal D) \rightarrow H^p(H; \mathcal D) \rightarrow H^p(A; \mathcal D) \rightarrow H^{p+1}(G; \mathcal D) \rightarrow \cdots.
    \]
    Since $A$ is free, $H^p(A; \mathcal D) = 0$ for $p \geqslant 2$, and this immediately implies that $b_p^{\mathcal D_{kG}}(G) = b_p^{\mathcal D_{kH}}(H)$ for $p \geqslant 3$, where we have used \cref{lem:eitherField}.

    The interesting portion of the long exact sequence is then
    \[
        0 \rightarrow H^1(G; \mathcal D) \rightarrow H^1(H; \mathcal D) \rightarrow H^1(A; \mathcal D) \rightarrow H^2(G; \mathcal D) \rightarrow H^2(H; \mathcal D) \rightarrow 0,
    \]
    whence we obtain the equation
    \begin{align*}
        0 &= b_1^{\mathcal D_{kG}}(G) - b_1^{\mathcal D_{kH}}(H) + b_1^{\mathcal D_{kA}}(A) - b_2^{\mathcal D_{kG}}(G) + b_2^{\mathcal D_{kH}}(H) \\
        &= b_1^{\mathcal D_{kG}}(G) - b_2^{\mathcal D_{kG}}(G) + b_2^{\mathcal D_{kH}}(H).
    \end{align*}
    But $b_2^{\mathcal D_{kG}}(G) \leqslant b_2^{\mathcal D_{kH}}(H)$ by \cref{lem:agr_ind}, so we must have $b_1^{\mathcal D_{kG}}(G) = 0$ and $b_2^{\mathcal D_{kG}}(G) = b_2^{\mathcal D_{kH}}(H)$.

    The claim about cohomological dimensions follows exactly as in the proof of \cite[Proposition 6.4]{KielakLinton_FbyZ}.
\end{proof}

As a consequence, we can reprove \cite[Theorem 1.10]{KielakLinton_FbyZ} over arbitrary fields, thus confirming \cite[Conjecture 6.7]{KielakLinton_FbyZ}.

\begin{thm}\label{thm:KLmain_agr}
    Let $k$ be a division ring and let $H$ be hyperbolic, virtually compact special, and suppose that $\cd_k(H) \geqslant 2$. Then, there exists a finite index subgroup $L \leqslant H$ and a map of short exact sequences
    \[
        \begin{tikzcd}
            1 \arrow[r] & K \arrow[d, hook] \arrow[r] & L \arrow[d, hook] \arrow[r] & \Z \arrow[d, Rightarrow, no head] \arrow[r] & 1 \\
            1 \arrow[r] & N \arrow[r]                 & G \arrow[r]                 & \Z \arrow[r]                                & 1
        \end{tikzcd}
    \]
    such that
    \begin{enumerate}[label = (\arabic*)]
        \item $G$ is hyperbolic, compact special, and contains $L$ as a quasi-convex subgroup.
        \item $\cd_k(G) = \cd_k(H)$.
        \item $N$ is finitely generated.
        \item If $b_p^{\mathcal D_{kH}}(H) = 0$ for all $2 \leqslant p \leqslant n$, then $N$ is of type $\FP_n(k)$.
        \item If $b_p^{\mathcal D_{kH}}(H) = 0$ for all $p \geqslant 2$, then $\cd_k(N) = \cd_k(H) - 1$.
    \end{enumerate}
\end{thm}


\begin{proof}
    Now that we have established \cref{prop:HNN} (Kielak and Linton prove the $L^2$ case in \cite[Proposition 6.4]{KielakLinton_FbyZ}), the proof is very similar to the one in \cite{KielakLinton_FbyZ}; consequently, we only highlight which parts of $L^2$-theory are used in the proof and provide references for the corresponding statements in the agrarian setting.

    First, Kielak--Linton use the fact that an infinite group has vanishing $0$th $L^2$-Betti number; this is also true in the agrarian setting by \cref{prop:props_agr}\ref{item:0}. Next, they use the fact that $L^2$-Betti numbers scale with the index when passing to finite index subgroups, which is true for agrarian Betti numbers with Hughes-free coefficients by \cref{prop:props_agr}\ref{item:scaling}. Finally, they quote \cite[Theorem A]{Fisher_improved} (see \cite[Theorem 6.1]{KielakLinton_FbyZ}), which relates the $L^2$-Betti numbers of compact special groups (and more generally of RFRS groups) to virtual fibring with kernels of type $\FP_n(\Q)$. Luckily, the analogous result holds with agrarian Betti numbers and finiteness properties over arbitrary fields \cite[Theorem B]{Fisher_improved}. \qedhere
\end{proof}


\section{Hughes-freeness of the Lewin--Lewin division ring} \label{sec:LL}

In \cite{LewinLewinORTF} J.~ Lewin and T.~ Lewin showed that for every division ring $k$ and every torsion-free one-relator group $G$, the group algebra $kG$ can be embedded in a division ring, which we denote by $\overline{kG}$ following their notation. They pointed out that they did not know whether $\overline{kG}$ is a universal $kG$-division ring of fractions. However, what they already knew is that if there were a universal $kG$-division ring of fractions, then it would be $\overline{kG}$. In this section we give evidence in this direction by showing that the Lewin--Lewin division ring is Hughes-free for virtually compact special one-relator groups, and if $k$ is of characteristic zero then it is always Hughes-free.


Key algebraic structures in the argument of Lewin--Lewin are firs and semifirs. A nonzero ring $R$ is a {\it free ideal ring} (or {\it fir}) if every left and every right ideal is a free $R$-module of unique rank. {\it Semifirs} are defined similarly, the only difference being that only ask that finitely generated left and right ideals are free of unique rank. We will make heavy use of the following two powerful theorems of Cohn.

\begin{thm}[\cite{Cohnfreeproducts_3}]\label{thm:2semifir_semifir}
    The coproduct of a family of semifirs over a common sub-division ring is again a semifir.
\end{thm}

\begin{thm}[{\cite[Corollary 7.5.14]{cohn06FIR}}]\label{thm:semifirDivRing}
    Every semifir embeds into a universal division ring of fractions.
\end{thm}

We also record two useful results for future reference.

\begin{lem}[\cite{Cohnfreeprodskewfields}]\label{lem:univ_equality}
    Let $R_1,R_2$ be semifirs with a common division subring $\mathcal{D}$. Then $U(R_1 *_{\mathcal{D}} R_2)\cong U(R_1 *_{\mathcal{D}} U(R_2))$.
\end{lem}

\begin{prop}[{\cite[Theorem 8.1]{JaikinLopez_Atiyah}}]\label{prop:rk_max_HF}
    Let $G$ be a locally indicable group $G$ and assume there exists a Hughes-free $kG$-division ring $\mathcal{D}_{kG}$. Then the Sylvester matrix rank function $\rk_{\mathcal{D}_{kG}}$ is maximal in $\mathbb{P}_{div}(kG).$
\end{prop}

\begin{rem}
    Note that \cref{prop:rk_max_HF} does not imply that Hughes-free division rings are necessarily universal, since different rank functions may not be comparable.
\end{rem}

We now prove two general lemmas about Hughes-free division rings which will be used right afterwards.

\begin{lem}\label{lem:HF_amalg}
    Let $G$ be a locally indicable group which is the amalgam of $G_1$ and $G_2$ over $H$. If $\mathcal{D}_{kG}$ exists, then
    \[
        U(\mathcal{D}_{kG_1} *_{\mathcal{D}_{kH}} \mathcal{D}_{kG_2})\cong \mathcal{D}_{kG}
    \]
    as $kG$-rings.
\end{lem}

\begin{proof}
    First note that by uniqueness of Hughes-free division rings, the coproduct $\mathcal{D}_{kG_1} *_{\mathcal{D}_{kH}} \mathcal{D}_{kG_2}$ is well-defined. Moreover, it is a semifir and thus has a universal division ring. On the other hand, we have natural embeddings of $\mathcal{D}_{kG_1}$, $\mathcal{D}_{kH}$, $\mathcal{D}_{kG_2}$ into $\mathcal{D}_{kG}$. Thus by universal property of coproducts we have the following commutative diagram
    \[
        \begin{tikzcd}
            kG \arrow[r, hook] \arrow[d, hook] & {\mathcal{D}_{kG_1} *_{\mathcal{D}_{kH}} \mathcal{D}_{kG_2}} \arrow[d, hook] \arrow[dl]\\
            \mathcal{D}_{kG}   & U(\mathcal{D}_{kG_1} *_{\mathcal{D}_{kH}} \mathcal{D}_{kG_2}) \nospacepunct{.}  
        \end{tikzcd}
    \]
    Let $\rk_{U(G)}$ and $\rk_G$ denote the Sylvester matrix rank functions on $\mathcal D_{kG_1} *_{\mathcal D_{kH}} \mathcal D_{kG_2}$ corresponding to $U(\mathcal{D}_{kG_1} *_{\mathcal{D}_{kH}} \mathcal{D}_{kG_2})$ and $\mathcal{D}_{kG}$, respectively. Then $\rk_{U(G)}\geq \rk_{G}$ by universality. 
    
    On the other hand, $\mathcal{D}_{kG}$ is Hughes-free and hence $\rk_{G}$ is a maximal rank function on $kG$. Thus, $\rk_{U(G)} = \rk_{G}$ on $kG$ by \cref{prop:rk_max_HF}. Since $kG$ embeds in both division rings epically, they are $kG$-isomorphic by \cref{lem:equal_rk}. \qedhere
\end{proof}



\begin{lem}\label{lem:HF_cyclic}
    Let $G$ be a locally indicable group and $N\trianglelefteqslant G$ be a normal subgroup such that $G/N=\langle tN \rangle \cong \Z$. If $\mathcal{D}_{kG}$ is a Hughes-free $kG$-division ring of fractions, then 
    $$\mathcal{D}_{kG}\cong \Ore(\mathcal{D}_{kN}[t^{\pm 1};\tau])$$
    as $kG$-rings.
\end{lem}

\begin{proof}
    First note that since $\mathcal{D}_{kG}$ is a Hughes-free division ring, the integers powers of $t$ are $\mathcal{D}_{kN}$-linearly independent. Hence, if we set $S$ for the subring of $\mathcal{D}_{kG}$ generated by $\mathcal{D}_{kN}, t$ and $t^{-1}$, we get an isomorphism $S\cong \mathcal{D}_N[t^{\pm 1};\tau]$ where $\tau$ denotes the twisted action of $t$ on $\mathcal D_{kN}$ induced by conjugation. Note that $S$ is an Ore domain. Thus by universal property of localization we have the following commutative diagram
    \[
        \begin{tikzcd}
             \Ore(\mathcal{D}_N[t^{\pm 1};\tau]) \arrow[rr, hook] &&  \mathcal{D}_{kG}\\
            &kG \arrow[ul, hook'] \arrow[ur, hook]   &    
        \end{tikzcd}
    \]
    Finally, since both embeddings are epic, we conclude that $\mathcal{D}_{kG}$ and $\Ore(\mathcal{D}_N[t^{\pm 1};\tau])$ are $kG$-isomorphic.\qedhere
\end{proof}

We are ready to show the main theorem of this section.

\begin{thm}\label{thm:LL_HF}
    Let $G$ be a torsion-free one-relator group and let $k$ be a division ring such that $kG$ embeds into a Hughes-free division ring $\mathcal D_{kG}$. Then $\overline{kG} \cong \mathcal D_{kG}$ as $kG$-division rings.
\end{thm}

\begin{proof}
    The plan of the proof is to follow the Lewin--Lewin construction of $\overline{kG}$ and use our assumption that $\mathcal D_{kG}$ exists to prove that $\overline{kG} \cong \mathcal D_{kG}$. Crucially, we recall Brodski\u{\i}'s result that torsion-free one-relator groups are locally indicable \cite{BrodskiiOR}. Let $G=\langle a_1, a_2,\ldots \mid R \rangle$ be a torsion-free one-relator group, where $R$ is a cyclically reduced word in the generators $a_i$. 
    
    We argue, as in \cite{LewinLewinORTF}, by induction on the complexity of $R$, where the \textit{complexity} of $R$ is defined to be the length of $R$ minus the number of generators appearing in $R$. If $R$ has complexity $0$, then $G$ is free, in which case the Lewin--Lewin construction is just the universal division ring of fractions $U(kG)$ (which exists due to \cite{Cohnfreeproducts_3}). But also $U(kG) \cong \mathcal D_{kG}$ for $G$ free (see, e.g., \cite[Theorem 1.1]{JaikinZapirain2020THEUO}).
    
    Now assume the complexity is greater than zero. We also assume that every generator appears in $R$. Otherwise $G$ decomposes as a free product $H * F$, where $H = \langle a_1, \dots, a_n \mid R \rangle$ with $R$ involving all the generators $a_1, \dots, a_n$ and $F$ is a free group. In this case, the Lewin--Lewin construction is $\overline{kG} = U(\overline{kH} *_k U(kF))$. Assuming the result for $H$, we conclude that $\overline{kG} \cong \mathcal D_{kG}$ by \cref{lem:HF_amalg}. We also assume that $R$ involves a generator of exponent sum zero, and explain how to reduce to this case at the end of the proof. Without loss of generality, suppose that $t = a_1$ has exponent sum zero, and let $N$ be the normal subgroup generated by the elements $a_i$ for $i \geqslant 2$. Note that $N$ splits as a line of groups
    \[
        \cdots * N_{i-1} *_{A_{i-1,i}} N_i *_{A_{i,i+1}} N_{i+1} * \cdots 
    \]
    where each $N_i$ is a one-relator group with relator of complexity strictly less than that of $R$ and each $A_{i-1,i}$ is a free group (see \cite[Section 5]{LewinLewinORTF}). By induction, each ring $kN_i$ embeds into a Hughes-free division ring $\mathcal D_{kN_i}$, and by uniqueness of Hughes-free division rings we can form the following line of division rings
    \[
        C = \cdots * \mathcal D_{kN_{i-1}} *_{\mathcal D_{kA_{i-1,i}}} \mathcal D_{kN_i} *_{\mathcal D_{kA_{i,i+1}}} \mathcal D_{kN_{i+1}} * \cdots 
    \]
    which is a semifir by Cohn's theorem, and it naturally contains the group ring
    \[
        kN \cong \cdots * kN_{i-1} *_{kA_{i-1,i}} kN_i *_{kA_{i,i+1}} kN_{i+1} * \cdots.
    \]
    There is thus a universal division ring of fractions $U(C)$. We can also construct a division ring $\mathcal D$ embedding $kN$ as the direct limit of the system 
    \[
        \{U(\mathcal{D}_{kN_{-i}} *_{\mathcal{D}_{kA_{-i,-i+1}}}\ldots *_{\mathcal{D}_{kA_{i-1,i}}} \mathcal{D}_{kN_{i}})\}_{i\in \N}
    \]
    of Hughes-free division rings by Lemma \ref{lem:HF_amalg}. Note $\mathcal D$ is Hughes-free as a $kN$-division ring and therefore $\mathcal D = \mathcal D_{kN}$ coincides with the division closure of $kN$ in $\mathcal D_{kG}$. Thus, we have the following commutative diagram
    \[
        \begin{tikzcd}
            kN \arrow[r, hook] \arrow[d, hook] & C \arrow[d, hook]\arrow[dl, hook]\\
            \mathcal{D}_{kN}   & U(C)    
        \end{tikzcd}
    \]
    So again combining universality and maximality of rank functions first over $C$ and then over $kN$ as in the proof of \cref{lem:HF_amalg}, it follows that $\mathcal{D}_{kN}$ and $U(C)$ are $kN$-isomorphic. In particular, $U(C)$ is Hughes-free. Finally, the Lewin--Lewin division ring is the Ore localization of $U(C)[t^{\pm 1};\tau]$, which according to Lemma \ref{lem:HF_cyclic} is $kG$-isomorphic to the Hughes-free $kG$-division ring of fractions $\mathcal{D}_{kG}$. This concludes the proof of the case where $R$ involves a generator with exponent sum zero.

    Suppose now that $R$ involves no generator with exponent sum zero. If this is the case, then by \cite[Proposition 2]{LewinLewinORTF} the one relator group $H := G * \Z$ can be given a one-relator presentation where the cyclically reduced word has complexity strictly less than that of $R$. By induction, we conclude that $\overline{kH} = \mathcal D_{kH}$. Lewin--Lewin set $\overline{kG}$ to be the division closure of $kG$ inside $\overline{kH}$, and this is just $\mathcal D_{kG}$. \qedhere 
\end{proof}

\begin{rem}
    The above proof shows that the Lewin--Lewin construction can be further extended to crossed products $k * G$.
\end{rem}

The existence of such Hughes-free division ring for locally indicable groups over a characteristic $0$ field was proved in \cite[Corollary 1.4]{JaikinLopez_Atiyah} as a corollary of the strong Atiyah conjecture over $\mathbb C$ for locally indicable groups. Thus, in characteristic zero, we conclude that the Lewin--Lewin construction is Hughes-free.

\begin{cor}\label{cor:LLHF}
    Let $G$ be a torsion-free one-relator group and let $k$ be a field of characteristic $0$. Then $\overline{kG}$ is Hughes-free. If $k$ is a general division ring, then $\overline{kG}$ is Hughes-free when $G$ is virtually compact special.
\end{cor}

In a sense made precise below, most one-relator groups are virtually compact special. The following definition is due to Puder \cite{Puder_PR}.

\begin{defn}\label{def:PR}
    The \textit{primitivity rank} of a word $w$ in a free group $F$ is the minimal rank of a subgroup $K$ containing $w$ such that $w$ is imprimitive in $K$ (we define the primitivity rank to be $\infty$ if there is no such $K$).
\end{defn}

Note that words of primitivity rank $1$ are exactly the proper powers. Let $G$ be a one-relator group with presentation $G = F/\llangle R \rrangle$. Louder and Wilton proved that $G$ has negative immersions if and only if $R$ has primitivity rank at least $3$ \cite[Theorem 1.3]{LouderWilton_NegativeImmersions} and Linton showed that one-relator groups with negative immersions are virtually compact special \cite[Theorem 8.2]{Linton_ORH}. Thus, $\mathcal D_{kG}$ exists by \cref{cor:VCSinField} and \cref{lem:HF_fi}, and by \cref{thm:LL_HF} $\overline{kG} \cong \mathcal{D}_{kG}$ whenever $R$ is of primitivity rank at least $3$.

We conclude with the following natural question. By the previous remarks, it is settled in the affirmative in characteristic $0$, and only the primitivity rank $2$ case remains in characteristic $p > 0$.

\begin{q}
    Is the Lewin--Lewin construction always Hughes-free?
\end{q}





\appendix \section{The Kaplansky zerodivisor conjecture for non-orientable \texorpdfstring{$3$}{}-manifold groups} \label{sec:appendix}


To prove Kaplansky's zerodivisor conjecture for group algebras of torsion-free orientable $3$-manifolds groups, we made heavy use of the Prime and JSJ Decomposition Theorems, which are usually stated for orientable $3$-manifolds. These theorems have been extended to non-orientable $3$-manifolds by Epstein \cite{Epstein_nonoriprime} and Bonahon--Siebenmann \cite{BonahonSiebenmann_nonoriJSJ}, respectively, and we explain here how to apply them to prove that the group algebra of any torsion-free $3$-manifold group satisfies Kaplansky's zerodivisor conjecture.

We will be using the statements of the non-orientable decomposition theorems as found in Bonahon's survey \cite[Theorems 3.1, 3.2, and 3.4]{Bonahon_3mfldsSurvey}, so we take some time to ensure that our definitions are in line with his. The non-orientable Prime Decomposition Theorem tells us how to cut a $3$-manifold along essential spheres and projective planes, while the non-orientable JSJ Decomposition tells us how to cut a $3$-manifold containing no essential spheres or projective planes along essential $2$-tori and Klein bottles. An \textit{essential sphere} in a $3$-manifold $M$ is an embedded copy of $S^2$ such that neither component of $M \smallsetminus S^2$ is homeomorphic to a $3$-ball. An \textit{essential projective plane} in a $3$-manifold $M$ is an embedded $2$-sided copy of $\R P^2$. We say that a $3$-manifold $M$ is \textit{irreducible} if does not contain any essential spheres or projective planes. Since $G = \pi_1(M)$ is always assumed to be torsion-free, we will not have to worry about the projective planes as the next lemma shows. We include a short proof for the reader's convenience.

\begin{lem}\label{lem:noProj}
    Let $M$ be a $3$-manifold and let $\R P^2 \hookrightarrow M$ be an embedding. Then the induced homomorphism $\pi_1(\R P^2) \rightarrow \pi_1(M)$ is injective.
\end{lem}

\begin{proof}
    The embedded $\R P^2$ lifts to a disjoint collection $\Sigma$ of $2$-spheres and projective planes in the universal cover $\widetilde M$. If $\Sigma$ contains a copy of $S^2$, then we are done, because then a loop in $M$ representing the generator of $\R P^2$ has a lift to a non-closed path in $\widetilde M$. If $\Sigma$ contains a copy of $\R P^2$, then the Scott Core Theorem implies that there is a compact, simply connected $3$-manifold containing a copy of $\R P^2$. Simply-connected manifold are orientable, and therefore so are their boundaries. Thus we can fill the boundary to conclude that there is an embedding of $\R P^2$ into a simply-connected, closed $3$-manifold. By the Poincar\'e Conjecture, we have thus produced an embedding $\R P^2 \hookrightarrow S^3$, a contradiction. \qedhere
\end{proof}

An \textit{essential torus} in a $3$-Manifold $M$ is an embedded $\pi_1$-injective copy of $T^2$. Bonahon defines as \textit{essential Klein bottle} to be an embedded copy of the Klein bottle $K$ such that the composition $T^2 \rightarrow K \hookrightarrow M$ is $\pi_1$-injective, where $T^2 \rightarrow K$ is the orientation double cover. We will use the following as the definition of an essential Klein bottle.

\begin{lem}
    An embedded Klein bottle $\iota \colon K \hookrightarrow M$ is essential if and only if $\iota$ is $\pi_1$-injective.
\end{lem}

\begin{proof}
    If $\iota$ is $\pi_1$-injective, then clearly $K$ is essential in $M$. Conversely, since $T^2 \rightarrow K \hookrightarrow M$ is $\pi_1$-injective, either $\iota_* \pi_1(K)$ contains $\Z^2$ as an index $2$ subgroup in which case $\iota$ is $\pi_1$-injective, or $\iota_* \pi_1(K) \cong \Z^2$. But $H_1(K) \cong \Z \oplus \Z/2$, which rules out the second case. \qedhere
\end{proof}

The following lemma is well-known; we include a proof because we have not seen it appear without the assumption of orientability. In the orientable case, the lemma essentially follows from \cite[Theorem 6.1]{Howie_locIndGps}.

\begin{lem}\label{lem:locInd3mfld}
    Let $M$ be a compact, irreducible $3$-manifold whose boundary $\partial M$ contains a component of Euler characteristic $\leqslant 0$. Then $\pi_1(M)$ is locally indicable.
\end{lem}

\begin{proof}
    In the proof, all homology groups will take coefficients in $\Q$. We first prove that $M$ has $b_1(M) > 0$. If $M$ is orientable, then the Half Lives-Half Dies Lemma states that the kernel of the inclusion induced map $H_1(\partial M) \rightarrow  H_1(M)$ has dimension equal to $\frac{1}{2} \dim H_1(\partial M)$. In particular, $b_1(M) > 0$. If $M$ is not orientable, then let $p \colon \widetilde M \rightarrow M$ be the orientation double cover, and let $\tau \colon H_1(M) \hookrightarrow H_1(\widetilde M)$ be the (injective) transfer homomorphism. The following diagram commutes
    \[
        \begin{tikzcd}
            H_1(\partial M) \arrow[d, "\tau", hook] \arrow[r] & H_1(M) \arrow[d, "\tau", hook] \\
            H_1(\partial \widetilde{M}) \arrow[r]            & H_1(\widetilde M) \nospacepunct{,}
        \end{tikzcd}
    \]
    and bottom map is nonzero by the Half Lives-Half Dies Lemma. Moreover, the the image of $\tau \colon H_1(\partial M) \hookrightarrow H_1(\partial \widetilde M)$ is not contained in the kernel of $H_1(\partial \widetilde M) \rightarrow H_1(\widetilde M)$, which is nontrivial on each non-spherical boundary component. We conclude that $H_1(\partial M) \rightarrow H_1(M)$ is nonzero, and therefore $b_1(M) > 0$. We have thus shown that $\pi_1(M)$ has a homomorphism to $\Z$.

    Let $G \leqslant \pi_1(M)$ be a finitely generated group. If $G$ is of finite-index in $\pi_1(M)$, then the injectivity of the transfer homomorphism implies that $b_1(G) > 0$. Now suppose that $G$ is of infinite index and let $N \rightarrow M$ be the corresponding non-compact covering space. Let $L$ be a core of $N$ obtained by the Scott Core Theorem. Note that $L$ necessarily contains some non-($S^2$ or $\R P^2$) boundary components, and therefore by the work above $b_1(G) = b_1(N) = b_1(L) > 0$. \qedhere
\end{proof}



\begin{thm}
    Let $M$ be a compact and irreducible $3$-manifold with torsion-free fundamental group $G$. Then $kG$ embeds in a division ring, which is Hughes-free if $G$ is locally indicable.
\end{thm}

\begin{proof}
    By the non-orientable version of the JSJ Decomposition Theorem due to Bonahon--Siebenmann \cite[Splitting Theorem 1]{BonahonSiebenmann_nonoriJSJ} (see \cite[Theorem 3.4]{Bonahon_3mfldsSurvey} for the form in which we are using the theorem), $M$ can be cut along two-sided essential tori and Klein bottles such that every component either is Seifert fibred or contains no essential $2$-tori or Klein bottles. Let $M_1, \dots, M_n$ be the JSJ components and $G_1, \dots, G_n$ their fundamental groups. Each $G_i$ has a finite index subgroup having the factorisation property by \cref{thm:nonGraphDivRings} and therefore has the factorisation property by \cite[Theorem 3.7]{FriedlSchreveTillmann_ThurstonFox}. Moreover, each $G_i$ has a finite index locally indicable subgroup \cite[Flowchart 1, (C.19)]{AFW_3mfldBook} whose group ring is and Hughes-free embeddable by \cref{thm:3mfld}. Thus, by \cref{thm:VCS_field}, each $kG_i$ embeds into a division ring $\mathcal D_i$. Thus, if the JSJ decomposition only has one component, we are done.

    Now suppose that $n \geqslant 2$. Then each JSJ component has a non-empty boundary whose components are all either tori or Klein bottles. Then each fundamental group $\pi_1(M_i)$ is locally indicable (\cref{lem:locInd3mfld}), has the factorisation property (previous paragraph), and has a finite-index locally indicable subgroup $H_i$ such that $kH_i$ has a Hughes-free embedding (\cref{thm:3mfld}). Thus, each group algebra $k[\pi_1(M_i)]$ has a Hughes-free embedding by \cref{lem:HF_fi}. By \cref{thm:graphOfHF}, we conclude that $kG$ embeds in a division ring.
    
    If $G$ is locally indicable, then the proof that the embedding into a division ring can be made Hughes-free is the same as the one given in the orientable case (see the proof of \cref{thm:graphMfld}). \qedhere
\end{proof}


We now conclude using the Prime Decomposition Theorem for non-orientable manifolds, due to Epstein \cite[Theorem 1.1]{Epstein_nonoriprime}.


\begin{thm}\label{thm:nonori3mfld}
    \cref{thm:3mfld,cor:3mfldKapl} hold without the assumption of orientability.
\end{thm}

\begin{proof}
    By the non-orientable Prime Decomposition Theorem (see \cite[Theorems 3.1 and 3.2]{Bonahon_3mfldsSurvey}, $M$ can be cut along finitely many essential, $2$-sided copies of $S^2$ or $\R P^2$, such that each of the components (after cutting) is irreducible or simply connected. By \cref{lem:noProj}, the assumption that $G$ is torsion-free implies that in fact there are no $\R P^2$'s and that $G$ decomposes as the free product of the fundamental groups of the irreducible pieces. We now conclude as in the proof of \cref{thm:3mfld}.
\end{proof}







\bibliographystyle{alpha}
\bibliography{bib}

\end{document}