%

\section{Introduction}
\label{sec:intro}

The solvability of distributed tasks is a fundamental problem in the theory of 
distributed computing, and 
several proof methods for showing unsolvability have been developed.  
The most classical is based on the valency argument \cite{FischerLynchPaterson85,Herlihy91}: 
If we assume the solvability of a task, 
it implies a contradictory set of outputs to be produced along concurrent execution paths.
Another method is the one that uses the topological model \cite{Book:2013:HerlihyKozlovRajsbaum}:
In the topological method, the computation of a distributed task or a
protocol is modeled by a function over simplicial complexes representing
the nondeterministic sets of states of concurrently running agents, and
the unsolvability of a task is demonstrated by a topological
inconsistency that is implied by a hypothetical existence of a
solution to the task.

In addition to these precursors,
a new method, called the \keywd{logical method}, has recently been proposed by 
Goubault, Ledent, and Rajsbaum \cite{Inf:2021:GoubaultLedentRajsbaum}. 
They proposed to discuss the structure of distributed computation 
in a Kripke model of epistemic knowledge, which is derived from the topological model
of simplicial complex. They formulated the distributed task and protocol using the so-called 
product update models, which originate from the study of dynamic epistemic logic \cite{DitmarschHoekKooi:DELbook08}.
The logical method provides a novel way to show the unsolvability of distributed tasks:
The unsolvability follows from a \keywd{logical obstruction}, i.e., 
a formula that describes the reason for the unsolvability in the formal
language of epistemic logic. 

Their logical method, however, only applies
to the tasks and protocols where the possible failure of distributed agents is insignificant
for the discussion of unsolvability.
The semantics of epistemic logic is given by the so-called \keywd{epsistemic models}, 
where an epistemic model is a Kripke model whose possible worlds are structured 
by equivalence relations over them.
They also showed that the epistemic models used in the logical method
and the pure simplicial complexes used in the topological method are isomorphic models. 
The purity means that each facet is of the same dimension, that is, 
each possible global state of a distributed system consists of 
the same number of live agents. 

This implies that the epistemic models in \cite{Inf:2021:GoubaultLedentRajsbaum} 
are unaware of failures, since we need impure simplicial complexes to model 
`dead' agents that are missing from a facet.
Despite the innocence of failure, the epistemic models can 
argue certain significant unsolvability results, including those for
the consensus task and $k$-set agreement tasks by a wait-free protocol 
in an asynchronous environment, 
where a `dead' agent can be regarded as just infinitely slow in execution.
(See Section~1.1.3 of \cite{PhD:2019:Ledent} for further discussion.)

Later in \cite{STACS22:GoubaultLedentRajsbaum}, the same authors devised 
\keywd{partial epistemic models}, whose possible worlds are structured 
by partial equivalence relations (PERs), 
which are subordinate equivalence relations that are not necessarily reflexive,
and argued that partial epistemic models are appropriate for 
modeling the possible failure of agents.
%
Partial epistemic models inherit several virtues from epistemic models.
They are isomorphic to the topological model of impure simplicial complexes;
They provide the semantics for the epistemic logic with axiom system~$\AxiomKBn$, 
as so do epistemic models for axiom system~$\AxiomSn$ \cite{Book:2004:JosephMoses,DitmarschHoekKooi:DELbook08};
Furthermore, they enjoy the knowledge gain property, 
which states that the knowledge expressed by an adequate class of formulas 
never increases along a morphism over the models. 


%
%
%
%
%
%
%
%

However, the framework of logical method has not been presented for partial epistemic models in~\cite{STACS22:GoubaultLedentRajsbaum}. 
The notion of product update relevant to partial epistemic models is needed for the definition of task solvability, but they did not present it.

%
%

In this paper, we introduce the notion of \keywd{partial product update}, 
which refines the original product update \cite{Inf:2021:GoubaultLedentRajsbaum}.
Using partial product update models, we give a logical definition of 
task solvability and thereby provide the logical method for proving task unsolvability.
Furthermore we present a concrete logical obstruction 
and show that the consensus task is not solvable by 
the synchronous message passing protocol \cite{S1571:2001:HerlihyRajsbaumTuttle}. 

%
%
The partial product update model refines the original model to 
allow coherent definitions of distributed tasks and protocols
in a distributed environment in which the exact set of dead agents is detectable. 
%
In a product update model $\Prod{\cplI}{\actMf{A}}$ of \cite{Inf:2021:GoubaultLedentRajsbaum}, 
a task or a protocol is specified by a set of products $(X,t)$ where an action $t$ in 
$\actMf{A}$ represents a possible output for the input $X$ in the input model~$\cplI$. 
In contrast, in partial epistemic models, an action $t$ 
should be associated with not necessarily a single input 
but with a set of inputs that are indistinguishable by the agents that are alive in $t$. 
For this, we define a partial product update model $\ImProd{\cplI}{\actMf{A}}$
as a set of products of the form $(\eqClass{X}{t},t)$, where $\eqClass{X}{t}$ is
an appropriate equivalence class of $X$ determined with respect to 
the set of agents that are alive in $t$. 
    
%
%
%
%
%
%

%
Using partial product update models, we define task solvability by 
the existence of a morphism that mediates the partial product update model of a protocol and that of a task.
We will show that this definition is 
equivalent to the topological definition of task solvability
in the following sense: 
Given a task and a protocol as functions over simplicial complexes,
we derive an action model and a partial product update model for each of them. 
Then a simplicial map that defines the solvability in the topological model 
exists if and only if a mediating morphism 
over partial epistemic models exists. 
This definition of task solvability using partial product updates 
refines the one using product updates so that it allows a detectable set of failed agents.
%
%
Furthermore, likewise in \cite{Inf:2021:GoubaultLedentRajsbaum}, 
it provides the logical method that allows a logical obstruction 
to show the unsolvability of a task.

%
%
We demonstrate the use of logical obstruction in partial product update models
by showing that the consensus task is unsolvable by the single-round
synchronous message passing protocol. 
In the synchronous message passing protocol \cite{S1571:2001:HerlihyRajsbaumTuttle},
each agent sends copies of its local value to other agents synchronously. 
Unlike wait-free protocols in an asynchronous environment, 
a crash of an agent in a synchronous environment is detectable by other live agents, 
since the failure of message delivery from a dead agent can be detected within a bounded period of time. 
This gives rise to an impure simplicial complex model, as depicted in Figure~\ref{fig:impureSMP3}.
In order to argue unsolvability for such an impure simplicial complex in
partial epistemic model, we construct a partial product update from an action model 
whose actions are represented by posets of rank at most $1$. 
We present a concrete epistemic logic formula that serves as a logical 
obstruction that refutes the solvability of the consensus task.

\begin{figure}[t]
    \centering
    \includegraphics[scale=0.5]{./smpcplx.pdf}
    \caption{An impure simplicial complex of the synchronous message passing protocol for~$3$ agents,
    with $0$-dimensional facets being omitted}
    \label{fig:impureSMP3}
\end{figure}


We note that 
the previous work \cite{S1571:2001:HerlihyRajsbaumTuttle}
analyzes more precisely the (un)solvability of $k$-set agreement tasks
by the $r$-round synchronous message passing protocol, for varying $k$ and $r$,  
using the topological method.
In the present paper, we only demonstrate the unsolvability of 
the consensus task (i.e., the $1$-set agreement task) by the single-round 
synchronous message passing protocol. 
To show the other unsolvability results, we would need to further refine
the logical method presented in this paper,  as it has been done for the
logical method in asynchronous environments,   e.g.,
\cite{Nishida:Msc20,YagiNishimura:arXiv20}.

The rest of the paper is organized into the following sections.
%
Section~\ref{sec:partialEpiModel} reviews the two previous models of distributed computing, 
namely, the topological models and the epistemic models.
%
In Section~\ref{sec:PPU}, we introduce the notion of partial product update and
give the definition of task solvability in partial epistemic models.
We also show that this definition of task solvability is 
equivalent to the standard one given in terms of the topological method.
(The formal proof of the equivalence is given in Appendix~\ref{sec:solvablEq}.)
Section~\ref{sec:SMPactionmodel} defines the partial product update model for the 
synchronous message passing protocol, where the action model consists of actions 
that are represented by posets of rank at most $1$. 
%
Section~\ref{sec:obstructionSMP} gives a concrete logical obstruction 
and proves that the consensus task is not solvable by the single-round 
synchronous message passing protocol.
%
Finally Section~\ref{sec:conclusion} concludes the paper 
and indicates the direction of future research.

