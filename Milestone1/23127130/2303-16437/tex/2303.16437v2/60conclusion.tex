%

\section{Conclusion and Future Work}
\label{sec:conclusion}

We have proposed the notion of partial product update models that is suitable 
for modeling distributed tasks and protocols in partial epistemic models,
in which agents may die. 
Using partial product update models, 
we defined task solvability in partial epistemic models,
thereby providing the logical method that allows a logical 
obstruction to prove the unsolvability of a distributed task.  
This logical method defined for partial product update models
extends the original product update proposed for epistemic models:
Given a pair of an input model $\cplI$ and an action model $\actMf{A}$, 
a partial product update model $\ImProd{\cplI}{\actMf{A}}$ refines 
the original product update model up to indistinguishability by the set of live agents.
The unsolvability of a task is then proved 
by a logical obstruction.
%
We have presented a concrete formula of epistemic logic to show that the consensus task is
unsolvable by the single-round synchronous message passing protocol. 
We have shown that the formula is indeed a logical obstruction, 
where the partial product update model for the synchronous message passing protocol 
is constructed from an action model whose actions are posets of rank at most $1$.

We have demonstrated an unsolvability result, but only for the consensus task and the single-round execution of the protocol. 
In contrast, the topological method can prove more general unsolvability results 
for $k$-set agreement tasks and multiple-round execution of the protocol \cite{S1571:2001:HerlihyRajsbaumTuttle}. 
It is an interesting topic to pursue how to obtain these generalized results 
using the logical method. 

The notion of partial product update itself could be of interest from the perspective of 
dynamic epistemic logic \cite{DitmarschHoekKooi:DELbook08}. 
The original product update was developed 
in the context of dynamic updates of Kripke models: 
In terms of logic with epistemic action modalities, 
$\cplI, X \models [\actMf{A}] \varphi$ iff
$\cplI, X \models \precond(t)$ implies
$\Prod{\cplI}{\actMf{A}}, (X,t) \models \varphi$. 
It would be interesting 
to figure out an appropriate logical interpretation of partial product update. 
