%
\section*{Appendix}
\section{Equivalence of The Two Task Solvabilities}
\label{sec:solvablEq}

%
%
%
%

%
\subsection{The topological definition of task solvability}
\label{subsec:toptasksolve}

%
In the topological method for distributed computing, tasks and protocols are defined
by means of carrier maps, where a carrier map is a color-preserving function that associates each 
\emph{simplex} of an input complex with a subcomplex of an output complex \cite{Book:2013:HerlihyKozlovRajsbaum}.
However, the partial epistemic models in this paper 
are built around facets, not simplexes of arbitrary dimension, 
which implies that partial product update models cannot cover 
all aspects of carrier maps. 
(Partial product update models assume that all agents participate in the computation anyway, 
though some of them may crash during the execution.)
%
%
%

For this reason, in this paper we define the topological specification of
a task or a protocol by a suboptimal function, which we call a
\keywd{facet map}, that associates each facet of the input complex with a
set of facets of the output complex.

In what follows, we assume the set of vertexes $V$ of a complex $\cplC=\anglpair{V,S,\coloring}$
is a finite subset of $\Ag\times\Value$ and the coloring map 
is defined by $\coloring\bigl((a,v)\bigr)=a$ for each $(a,v)\in V$. 
%
%
With this assumption, we can easily translate
a complex $\anglpair{V,S,\coloring}$ into the corresponding
input simplicial model $\anglpair{V,S,\coloring,\labSM}$ given in Definition~\ref{def:inputsimplicialmodel}, 
by defining $\labSM(X)=\{\Pinput{a}{v} \mid (a,v)\in X\}$,
and similarly for the reverse translation.
In this way, 
we will confuse a complex with an input simplicial model.


%
%
%
%
%
%
%
%
%
%
%

%
%
%
%
%
%
%
%
%
%
%
%
%

%
%
%

%
%
%
%
%

%
%

%
%
%
%
%
%
%
%
%
%
%
%
%
%
%
%
%

\begin{definition}
    \label{def:facetmap}
    Given complexes 
    %
    $\cplC=\anglpair{V,S,\coloring}$ and $\cplD=\anglpair{V',S',\coloring'}$, 
    a function $\Theta: \Facet(\cplC) \to 2^{\Facet(\cplD)}$ is  
    called a \keywd{facet map} if it satisfies the following conditions.
    \begin{itemize}
        \item $\Theta$ is color-preserving, that is, 
        $\displaystyle\bigcup_{Y\in \Theta(X)}\coloring'(Y)\subseteq \coloring(X)$ for every facet $X\in \Facet(\cplC)$;
        \item $\Theta$ is surjective, i.e., 
         $\Facet(\cplD) = \bigcup_{X\in\Facet(\cplC)} \Psi(X)$.
        %
    \end{itemize} 


    In order to define a task and a protocol, 
    let $\cplI=\anglpair{V^{\cplI}, S^{\cplI}, \coloring^{\cplI}}$ be the common input complex of them.
    %
    \label{def:simplicialtask}
    A task is defined by a triple $\anglpair{\cplI,\cplO,\Delta}$, which we call a \keywd{simplicial task},
    where 
    %
        %
        %
        %
        %
        $\cplO=\anglpair{V^{\cplO}, S^{\cplO}, \coloring^{\cplO}}$ is an output complex and 
        $\Delta:\func{\Facet(\cplI)}{\PowerSet{\Facet(\cplO)}}$ is a facet map.
        %
        %
        %
        %
        %
        %
    %
%
%
%
%
%
%
%
%
%
%
%
%
%
%
%
%
%
%
%
%
%
%
%
%
%
%
%
    \label{def:simplicialprotocol}
    A protocol is defined by a triple $\anglpair{\cplI,\cplP,\Psi}$, which we call a \keywd{simplicial protocol},
    where 
    %
    $\cplP=\anglpair{V^{\cplP}, S^{\cplP}, \coloring^{\cplP}}$ is an output complex and 
    $\Psi:\func{\Facet(\cplI)}{\PowerSet{\Facet(\cplP)}}$ is a facet map satisfying the following condition:

    %
        %
        %
        %
        %
        \begin{align} \label{eq:carriercont} & 
            \begin{minipage}{.85\textwidth}
                %
                %
                $\coloring^{\cplP}(Y\cap Y')\subseteq \coloring^{\cplI} (X\cap X')$,
                for any $X,X'\in \Facet(\cplI)$ and $Y, Y' \in \Facet(\cplP)$  such that  \\
                $Y\in \Psi(X)$ and $Y'\in \Psi(X')$. 
                %
                %
            \end{minipage}
        \end{align}
        %
            %
            %
            %
            %
            %
            %
            %
        %
    %
\end{definition}

The condition~\eqref{eq:carriercont} is a natural requirement for the protocol:
If an agent~$a$ observes the same output of the protocol in two facets $Y$ and $Y'$, 
the agent~$a$ must also have agreed on the input in two facets $X$ and $X'$, 
where  $Y$ (resp., $Y'$) represents a possible global state of the system 
that is reachable from the global state of the initial input represented by $X$ (resp., $X'$). 

%



%
%

%
%
%
%
%
\begin{definition}\label{def:topsolvability}
    %
    We say 
    a protocol $\anglpair{\cplI, \cplP, \Psi}$ solves a task $\anglpair{\cplI, \cplO, \Delta}$,  
    if there exists a \keywd{descision map}, i.e., 
    a chromatic simplicial map $\DeltaTop: \cplP \to \cplO$ that satisfies
    %
     %
    %
    \begin{align} \label{eq:topsolve} &
    \begin{minipage}{.85\textwidth}
        %
        %
            $\DeltaTop (\Psi(X)) \subseteq \Delta(X)$ for every $X \in \Facet(\cplI)$,  
                   %
        %
    \end{minipage}          
    \end{align}
    where  
    $\DeltaTop (\Psi(X)) = \{ \DeltaTop(Y) \mid Y \in \Psi(X) \}$, and
    the inclusion $\mathcal{S} \subseteq \mathcal{S'}$
    means that for any $X\in \mathcal{S}$
    there exists $X'\in\mathcal{S'}$ such that $X\subseteq X'$. 
\end{definition}


%
%

%
%
%
%
%

%
%
%
%
%
%
%

%
%
%


%
%

%
%
%

%
%
%

%

%
%

%
%
%
%
%
%
%
%
%
%
%
%
%
%
%
%





%
%
%
\subsection{Equivalence of two task solvabilities}\label{subsec:solvablEq}

%
%

We will show that the above topological task solvability using simplicial complexes 
can be translated to that using partial product update models 
(Definition~\ref{subsec:tasksolv}).
For this, we introduce a translation function $\actionkappa$ as below.

%
%
%
%
%
%


%
%
%
%
%
%
%
%

%

\begin{definition}\label{def:actionkappa}
    %
    %
    %
    %
    %
    %
    %
    Let $\cplI=\anglpair{V^{\cplI}, S^{\cplI}, \coloring^{\cplI}, \labSM^{\cplI}}$ be an input simplicial model.   
    We define a translation function $\actionkappa$ that assigns an action model for 
    a simplicial task or protocol $\anglpair{\cplI, \cplD, \Theta}$
    by 
    \[ \actionkappa(\anglpair{\cplI, \cplD, \Theta})= \actMf{A} \]
    where $\actMf{A}=\anglpair{A, \sim^A, \precond^A}$ is an action model consisting of:
    \begin{itemize}
        \item The set of actions $A = \Facet(\cplD)$;
        \item The family of indistinguishability relations defined by   
        $X\sim_a^A Y$ iff $a\in \coloring^{A} (X\cap Y)$;
        \item The precondition defined by 
        $\precond^A (Y)=\bigvee\{\bigwedge \labSM^{\cplI} (X)\mid X\in \Facet(\cplI),  Y\in \Theta(X)\}$.
    \end{itemize}
\end{definition}

The translation function $\actionkappa$ is an augmentation of 
the functor $\kappa$, which has been introduced 
in \cite{STACS22:GoubaultLedentRajsbaum} to show the association of 
simplicial complexes with their corresponding partial epistemic frames,
for the purpose of establishing the categorical equivalence of these two structures.
As such, $\actionkappa$ preserves the partial epistemic frame $\anglpair{\Facet(\cplD),\sim^{\cplD}}$
induced from $\cplD$. Furthermore, the precondition 
$\precond^A$ gives the condition for $Y\in \Facet(\cplD)$ to be
a possible output of $\Theta$ for an input $X \in \Facet(\cplI)$, that is,  
$\cplI,  X\models \precond^A (Y)$ if and only if $Y\in \Theta(X)$. 
In this sense, $\actionkappa$ associates an action model that is equivalent to a given 
simplicial task or protocol. 

%
%
%
%


%




%

%
%
%
%

Particularly for an action model $\actMf{P}$ that is translated from a simplicial protocol $\anglpair{\cplI,\cplP,\Phi}$, 
the partial product update model $\ImProd{\cplI}{\actMf{P}}$ has a partial epistemic frame 
that is isomorphic to that of $\actMf{P}$.
%
%
\begin{proposition}\label{prop:protocolEq}
    Let $\cplI=\anglpair{V^{\cplI}, S^{\cplI}, \coloring^{\cplI}, \labSM^{\cplI}}$ be an input simplicial model,
    %
    $\anglpair{\cplI, \cplP, \Psi}$ be 
    a simplicial protocol,  and $\actMf{P}=\actionkappa(\anglpair{\cplI, \smplMf{P}, \Psi})$
    be the translated action model. Then 
    the action model $\actMf{P}$ and the partial product update model $\ImProd{\cplI}{\actMf{P}}$ 
    have isomorphic partial epistemic frames.
\end{proposition}

\begin{proof}
%
%
%
%
%
%
%
Let $\actMf{P}=\anglpair{P, \sim^P, \precond^P}$
and $\ImProd{\cplI}{\ActSMP} = \anglpair{W^{\ImProd{\cplI}{\ActSMP}}, \relK{\ImProd{\cplI}{\ActSMP}}{}, L^{\ImProd{\cplI}{\ActSMP}}}$. 
%
%
%
We define a pair of maps 
$f: P \to W^{\ImProd{\cplI}{\ActSMP}}$ 
and 
$g: W^{\ImProd{\cplI}{\ActSMP}} \to P$ 
by 
\[
    f(Y)=(\preEqClass{Y}{X},  Y) ~~ \text{and} ~~ g\bigl((\preEqClass{Y}{X}, Y)\bigr)= Y,  
    \]
where $X,Y \in \Facet(\cplI)$. 

We show that $f$ is well-defined, that is, 
the set $\preEqClass{Y}{X}$ is not affected by the choice of $X$. 
Suppose $(\preEqClass{Y}{X},  Y), (\preEqClass{Y}{X'},  Y) \in W^{\ImProd{\cplI}{\ActSMP}}$. 
By the definition of $\preEqClass{Y}{-}$, we have $\cplI,X\models \precond^P(Y)$ and
$\cplI,X' \models \precond^P(Y)$, which implies $Y\in \Psi(X)$ and $Y\in \Psi(X')$. 
Then, by the condition~\eqref{eq:carriercont} of the simplicial protocol, 
$X\relK{\cplI}{a}X'$ for all $a\in \AliveSet{Y}$ and
therefore $\preEqClass{Y}{X}=\preEqClass{Y}{X'}$. 

%
%
%
%

%
It is easy to see that $f$ and $g$ are inverses of each other, namely,  
$f\circ g=\ident{W^{\ImProd{\cplI}{\ActSMP}}}$ and $g\circ f =\ident{P}$.
Furthermore they preserve the indistinguishability relation. 
For $f$, suppose $Y\relK{P}{a} Y'$, where 
$f(Y)=(\preEqClass{Y}{X},Y)$ and $f(Y')=(\preEqClass{Y'}{X'},Y')$. 
By a similar discussion above, it follows that $Y\in \Psi(X)$ and $Y'\in \Psi(X')$
from the definition of $\preEqClass{Y}{-}$. 
This implies that $X \relK{\cplI}{a} X'$ and therefore 
$f(Y)\relK{\ImProd{\cplI}{\actMf{P}}}{a} f'(Y')$.
For the converse, 
suppose $\preEqClass{Y}{X}\relK{\ImProd{\cplI}{\actMf{P}}}{a} \preEqClass{Y'}{X'}$. 
Then, $Y \relK{\cplI}{a} Y'$ and therefore $g(Y) \relK{\cplI}{a} g(Y')$. 

%
%
%
%
%
%
%
%
%
%
%

%
%
%
%

%
%
%
The isomorphisms $f$ and $g$ can be regarded as morphisms over partial epistemic frames, 
defining 
$\hat{f}(Y)=\sat{\AliveSet{Y}}{f(Y)}$ and 
$\hat{g}\bigl((\preEqClass{Y}{X}, Y)\bigr)=\bigsat{\AliveSet{Y}}{g\bigl((\preEqClass{Y}{X}, Y)\bigr)}$. 
Hence $\anglpair{P,\sim^P}$ and $\ImProd{\cplI}{\actMf{P}}$ are isomorphic partial epistemic frames.
%
%
%
\end{proof}

We show the task solvability defined using simplicial complexes
implies that defined using partial product update models. 
%
That is, the existence of a morphism $\dectop$ %
in Definition~\ref{def:kripkesolvability} 
implies the existence of a morphism $\deckrip$ %
in Definition~\ref{def:topsolvability}, and vice versa.

%
%
%


%
%
%
%

\begin{lemma}\label{lem:kriptosimpldecision}
    Suppose a simplicial task $\anglpair{\cplI, \cplP, \Psi}$ 
    and a simplicial protocol $\anglpair{\cplI, \cplO, \Delta}$ are given, where
    $\cplI=\anglpair{V^{\cplI}, S^{\cplI}, \coloring^{\cplI}, \labSM^{\cplI}}$ is an input simplicial model.
    Let $\actMf{P}=\actionkappa(\anglpair{\cplI, \cplP, \Psi})$ and 
    $\actMf{T}=\actionkappa(\anglpair{\cplI, \cplO, \Delta})$ be the action models for the protocol and the task, respectively. 
    If there exists a morphism over partial epistemic models $\deckrip:\func{\ImProd{\cplI}{\actMf{P}}}{\ImProd{\cplI}{\actMf{T}}}$ 
    that satisfies the condition~\eqref{eq:kripsolve} in Definition~\ref{def:kripkesolvability}, 
    there exists a simplicial map $\dectop:\func{\cplP}{\cplO}$ that satisfies the condition~\eqref{eq:topsolve}
    in Definition~\ref{def:topsolvability}.
\end{lemma}
\begin{proof}
%
%
Let $\actMf{P}=\anglpair{P, \sim^P, \precond^P}$ and $\actMf{T}=\anglpair{T, \sim^T, \precond^T}$
and also let $\cplP = \anglpair{V^{\cplP}, S^{\cplP}, \coloring^{\cplP}}$
and $\cplO = \anglpair{V^{\cplO}, S^{\cplO}, \coloring^{\cplO}}$.

%
Using $\deckrip$, 
we define a map $\dectop:V^{\cplP} \to V^{\cplO}$ by
\[
    \dectop(v) = u, 
    \]
where $v$ and $u$ is any pair of vertexes such that
$\coloring^{\cplP}(v) = \coloring^{\cplO}(u)$
and furthermore 
$v\in X$, $u\in Y$, and 
$(\preEqClass{Y}{Z'}, Y) \in \deckrip\bigl((\preEqClass{X}{Z}, X)\bigr)$ holds 
for some $X\in\Facet(\actMf{P})$, $Y\in\Facet(\actMf{O})$, and $Z, Z'\in \Facet(\cplI)$. 

We show $\coloring^{\cplP}(v)$ is defined for every $v\in V^{\cplP}$. 
Suppose $X$ is a facet satisfying $v\in X$. 
By the surjectivity of $\Psi$, there exists $Z\in\Facet(\cplI)$ such that 
$X\in \Psi(Z)$. This implies $(\preEqClass{X}{Z}, X)$ is defined.
%
%
%
%
Furthermore we have $\AliveSet{X}\subseteq\AliveSet{Y}$,   
since $\deckrip$ preserves the indistinguishability, $\AliveSet{X}\subseteq\AliveSet{Y}$. 
%
Let us define $\coloring^{\cplP}(v) =u$, 
where  $u\in Y$ be the vertex such that $\coloring^{\cplO}(u)=\coloring^{\cplP}(v)$. 
We show this uniquely defines $u$.
Suppose $u_1, u_2 \in V^{\cplO}$ are vertexes given by the definition, that is,
$\coloring^{\cplP}(v) = \coloring^{\cplO}(u_1)= \coloring^{\cplO}(u_2)$ and
furthermore there exists $X_1, X_2\in\Facet(\cplP)$, 
$Y_1, Y_2\in\Facet(\cplO)$, and
$Z_1, Z_2, Z_1', Z_2' \in\Facet(\cplI)$ such that 
$v\in X_i$, $u_i\in Y_i$, and 
$(\preEqClass{Y_i}{Z_i'}, Y_i) \in \deckrip\bigl((\preEqClass{X_i}{Z_i}, X_i)\bigr)$
for each $i=1,2$.  
By the definition of~$\deckrip$, we 
have $\AliveSet{X_i}\subseteq\AliveSet{Y_i}$ ($i=1,2$) and hence 
$\AliveSet{X_1\cap X_2}\subseteq \AliveSet{Y_1\cap Y_2}$. 
Since $v\in X_1 \cap X_2$,  
$\coloring^{\cplP}(v) = \coloring^{\cplO}(u_1)= \coloring^{\cplO}(u_2) \in \AliveSet{Y_1\cap Y_2}$.
This implies $u_1, u_2\in Y_1 \cap Y_2$ and hence $u_1=u_2$. 

To show that $\dectop$ defined above is a simplicial map, suppose $X = \{v_0,\ldots,v_m\}\in \Facet(\cplP)$. 
By the surjectivity of $\Psi$, 
there exists $Z\in \Facet(\cplI)$ such that $X\in\Psi(Z)$. 
This implies $(\preEqClass{X}{Z}, X) \in \Facet(\ImProd{\cplI}{\actMf{P}})$. 
By the condition~\eqref{eq:kripsolve} for $\deckrip$, 
$(\preEqClass{Y}{Z'}, Y)\in \deckrip\bigl((\preEqClass{X}{Z}, X)\bigr)$
for some $Y\in \Facet(\cplO)$ and $Z'\in \Facet(\cplI)$. 
By the definition of~$\dectop$, for each $v_i$, we have
$\dectop(v_i)=u_i$ where $u_i$ is the unique vertex satisfying $u_i\in Y$ 
and $\coloring^{\cplO}(u_i)= \coloring^{\cplP}(v_i)$. 
Therefore $\dectop(X)\subseteq Y\in \Facet(\cplO)$.

Let us show that $\dectop$ also satisfies \eqref{eq:topsolve}. 
Suppose $X\in\Facet(\cplI)$ and also $Y\in\Psi(X)$.
The latter implies $\cplI,X\models \precond^{\actMf{P}}(Y)$
and therefore 
$(\preEqClass{Y}{X}, Y)$
is defined. 
Then by~\eqref{eq:kripsolve}, there exists    
$(\preEqClass{Y'}{X'}, Y')\in \deckrip\bigl((\preEqClass{Y}{X}, Y)\bigr)$
such that $\preEqClass{Y}{X}\subseteq \preEqClass{Y'}{X'}$. 
Since $X\in  \preEqClass{Y}{X}\subseteq \preEqClass{Y'}{X'}$, 
$\preEqClass{Y'}{X'}=\preEqClass{Y'}{X'}$ 
and henceforth $(\preEqClass{Y'}{X'}, Y')\in \Facet(\ImProd{\cplI}{\actMf{T}})$.
This implies that $\cplI, X\models\precond^{\actMf{T}}(Y')$ and hence 
$Y'\in \Delta(X)$. 
By the definition of~$\dectop$, $\dectop(X)\subseteq Y'$. 
Therefore $\dectop(\Psi(X))\subseteq \Delta(X)$. 
\end{proof}

%
%
%
%
%
%
%
%
%
%
%
%
%
%
%
%
%
%
%
%
%
%
%
%
%
%
%
%
%
%
%
%
%
%
%
%
%
%
%
%
%
%
%
%
%
%
%
%
%
%
%
%
%
%
%
%
%
%
%
%
%
%
%
%
%
%
%
%
%
%
%
%
%
%
%
%
%
%
%
%
%
%
%
%
%
%
%
%
%
%
%
%
%
%
%


The converse direction is proven as follows. 

%
%
%
\begin{lemma}\label{lem:smpltokripdecision}
    Let $\actMf{P}=\actionkappa(\anglpair{\cplI, \cplP, \Psi})$ and  
    $\actMf{T}=\actionkappa(\anglpair{\cplI, \cplO, \Delta})$ be 
    the translated action model for a simplicial protocol $\anglpair{\cplI, \cplP, \Psi}$ 
    and a simplicial task $\anglpair{\cplI, \cplO, \Delta}$, respectively. 
    If there exists a simplicial map $\dectop:\func{\cplP}{\cplO}$ that satisfies \eqref{eq:topsolve}, 
    then there exists a morphism over partial epistemic models $\deckrip:\func{\ImProd{\cplI}{\actMf{P}}}{\ImProd{\cplI}{\actMf{T}}}$ that satisfies \eqref{eq:kripsolve}.
\end{lemma}
\begin{proof}
%
%
%
%
Let $\actMf{P}=\anglpair{P, \sim^P, \precond^P}$ and $\actMf{T}=\anglpair{T, \sim^T, \precond^T}$.
Using $\dectop$, we define $\deckrip:\func{\ImProd{\cplI}{\actMf{P}}}{\ImProd{\cplI}{\actMf{T}}}$ by
\[ 
    \deckrip \bigl((\preEqClass{Y}{X}, Y)\bigr)= \bigsat{\AliveSet{Y}}{(\preEqClass{Z}{X}, Z)}, 
    \]
where $Z\in T$, $\dectop(Y)\subseteq Z$, $\AliveSet{Z}\subseteq\AliveSet{X}$, and 
$\cplI, X\models \precond^T (Z)$. 

There exists $Z\in T$ that satisfies this condition. 
By the definition of $\preEqClass{Y}{X}$, we have 
$\cplI, X\models \precond^P (Y)$, which implies $Y\in \Psi(X)$.
By the condition~\eqref{def:topsolvability}, there exists $Z\in\Delta(X)$ 
such that $\dectop(Y)\subseteq Z$. This means $Z\in T$ and $\cplI, X\models \precond^T(Z)$. 
Furthermore, $\AliveSet{Z}\subseteq\AliveSet{X}$ because $\Delta$ is color-preserving.

The definition of $\deckrip$ is not affected by the choice of $Z$. 
To show this, suppose $Z_1, Z_2 \in T$ such that 
$\dectop(Y)\subseteq Z_i$, $\AliveSet{Z_i}\subseteq\AliveSet{X}$, and $\cplI, X\models \precond^T (Z_i)$ 
($i=1,2$). 
From $\dectop(Y)\subseteq Z_i$ ($i=1,2$), we have
$\coloring^T(Z_1\cap Z_2) \supseteq  \coloring^T(\dectop(Y)) =\coloring^P(Y)$. 
Hence $Z_1 \sim_{\AliveSet{Y}}^T Z_2$. 
This implies $\preEqClass{Z_1}{X} \sim_{\AliveSet{Y}}^{\ImProd{\cplI}{\actMf{T}}} \preEqClass{Z_2}{X}$
and therefore  $\bigsat{\AliveSet{Y}}{(\preEqClass{Z_1}{X}, Z_1)} =\bigsat{\AliveSet{Y}}{(\preEqClass{Z_2}{X}, Z_2)}$.
 

%
%
%
%
%
%
%
%

%
%
%
%
%

%
%
%

%
%

%

%

We show that $\deckrip$ is a morphism over partial epistemic models. 
%
    %
Obviously it satisfies the saturation property by definition. 
%
%
To show that $\deckrip$ preserves the indistinguishability, 
suppose $(\preEqClass{Y_1}{X_1}, Y_1)\sim_a^{\ImProd{\cplI}{\actMf{P}}} (\preEqClass{Y_2}{X_2}, Y_2)$.
%
This implies $X_1\relK{\cplI}{a} X_2$ and $Y_1 \sim_a^P Y_2$, and in particular 
%
$a\in \coloring^{\cplP} (Y_1 \cap Y_2)$. 
For all $(\preEqClass{Z_1}{X_1}, Z_1)\in \deckrip\bigl((\preEqClass{Y_1}{X_1}, Y_1)\bigr)$ 
and $(\preEqClass{Z_2}{X_2}, Z_2)\in \deckrip\bigl((\preEqClass{Y_2}{X_2}, Y_2)\bigr)$, 
we have $Z_1\relK{T}{\AliveSet{Y_1\cap Y_2}}Z_2$ by the definition of $\deckrip$. 
%
In particular, $Z_1\relK{T}{a} Z_2$. 
Hence $(\preEqClass{Z_1}{X_1}, Z_1)\relK{\ImProd{\cplI}{\actMf{T}}}{a} (\preEqClass{Z_2}{X_2}, Z_2)$. 
%
%
%
%
For the preservation of atomic propositions, 
suppose 
$(\preEqClass{Y}{X}, Y)\in \ImProd{\cplI}{\actMf{P}}$, 
$(\preEqClass{Z}{X}, Z)\in \deckrip \bigl((\preEqClass{Y}{X}, Y)\bigr)$, and 
$a\in \AliveSet{Y}$. 
%
By definition, $X\relK{\cplI}{a}X'$ holds for every $X'\in \preEqClass{Y}{X}$. 
This implies that   
$L^{\ImProd{\cplI}{\actMf{P}}}\bigl((\preEqClass{Y}{X}, Y)\bigr) \cap \AtomProps_a =\labSM^{\cplI} (X)\cap\AtomProps_a$.
Similarly, 
$L^{\ImProd{\cplI}{\actMf{T}}}\bigl((\preEqClass{Z}{X}, Z)\bigr) \cap \AtomProps_a =\labSM^{\cplI} (X) \cap \AtomProps_a$. 
Therefore $L^{\ImProd{\cplI}{\actMf{P}}}\bigl((\preEqClass{Y}{X}, Y)\bigr) \cap \AtomProps_a = 
L^{\ImProd{\cplI}{\actMf{T}}}\bigl((\preEqClass{Z}{X}, Z)\bigr) \cap \AtomProps_a$. 

%
%
%



%
%
%
%
%
%
%
%
%

%
%
%

Finally, let us show that $\deckrip$ satisfies the condition~\eqref{eq:kripsolve}. 
Suppose 
$\deckrip\bigl((\preEqClass{Y}{X}, Y)\bigr)=$ $
\satop_{\AliveSet{Y}}\bigl((\preEqClass{Z}{X}$, $Z)\bigr)$
%
where $Z\in T$, $\dectop(Y) \subseteq Z$, $\AliveSet{Z}\subseteq\AliveSet{X}=\Ag$.
%
Assume $X'\in \preEqClass{Y}{X}$.
By definition, 
$X \sim_{\AliveSet{Y}}^{\cplI} X'$ and $\cplI, X'\models \precond^P (Y)$. 
The latter implies $Y\in \Psi(X')$ and thus $\dectop(Y)\in \dectop(\Psi(X'))$. 
Since $\dectop(\Psi(X'))\subseteq \Delta(X')$ by~\eqref{eq:topsolve}, 
there exists $Z'\in T$ such that $\dectop(Y)\subseteq Z'$ and $Z'\in \Delta(X')$. 
The former implies $\dectop(Y)\subseteq Z\cap Z'$ and thus $Z \relK{T}{\AliveSet{Y}}Z'$;
The latter implies $\cplI,X'\models \precond^T(Z')$ and hence $\preEqClass{Z'}{X'}$.
By these, we obtain 
$(\preEqClass{Z}{X},Z) \sim_{\AliveSet{Y}}^{\ImProd{\cplI}{\actMf{T}}} 
(\preEqClass{Z'}{X'},Z')$. 
Therefore we have $(\preEqClass{Z'}{X'},Z') \in \deckrip \bigl((\preEqClass{Y}{X}, Y)\bigr)$,
where $X' \in \preEqClass{Z'}{X'}$ trivially holds. 


%
%

%
%
%
%
%
%
%
%
%
%
%
%
%
%
%
%
%
%
%
%
%
%
%
%
%
%
%
\end{proof}


%
%
%
\begin{theorem}
    Suppose a simplicial task $\anglpair{\cplI, \cplP, \Psi}$ 
    and a simplicial protocol $\anglpair{\cplI, \cplO, \Delta}$ are given, where
    $\cplI=\anglpair{V^{\cplI}, S^{\cplI}, \coloring^{\cplI}, \labSM^{\cplI}}$ is an input simplicial model.
    Let $\actMf{P}=\actionkappa(\anglpair{\cplI, \cplP, \Psi})$ and 
    $\actMf{T}=\actionkappa(\anglpair{\cplI, \cplO, \Delta})$.
    Then 
    a task $\anglpair{\cplI, \cplO, \Delta}$  is solvable by
    a protocol $\anglpair{\cplI, \cplP, \Psi}$ 
    if and only if $\actMf{T}$ is solvable by $\actMf{P}$.
    %
\end{theorem}

%
%
%
\begin{proof}
    Follows from Lemma \ref{lem:kriptosimpldecision} and \ref{lem:smpltokripdecision}.
\end{proof}
