\clearpage
\newpage
\appendix
\label{sec:appendix}

\subsection{Experiment Details}
\label{sec:exp_details}
For simplicity, we secure the joystick, bottle, and book to the table. This mimics having another manipulator keep the object in place while the hand manipulates the object. The bowls and cups are not secured, making the problem of unstacking much more difficult.

To ensure fair evaluation, we start each method with the object in the same  configuration for each index trial. This corresponds to 10 different starting positions, each of which is used at the start of each baseline run.

\subsection{Model Details}
\label{sec:model_details}

Here we provide additional details about our method and baselines for easier reproduction.

For all image-based models, we normalize the inputs based on the mean and standard deviation of the data seen during training. For the tactile-based models, we normalize the inputs to be within the range $[0,1]$.

\subsubsection{BYOL Details}
The complete list of BYOL hyperparameters has been provided in Table~\ref{tab:hyperparams}. We take the model with the lowest training loss out of all the epochs.

\begin{table}[h]
    \begin{center}
    \setlength{\tabcolsep}{6pt}
    \renewcommand{\arraystretch}{1.5}
    \begin{tabular}{ c c } 
        \hline
        Parameter & Value \\
        \hline
                Optimizer          & Adam\\
                Learning rate      & $1e^{-3}$\\
                Weight decay   & $1e^{-5}$\\
                Max epochs &  1000\\
                Batch size (Tactile) & 1024 \\
                Batch size (Image) & 64 \\
                Aug. (Tactile) & Gaussian Blur (3x3) (1.0, 2.0) \ $p=0.5$ \\
                                    & Random Resize Crop (0.9, 1.0)  \ $p=0.5$ \\
                Aug. (Image) & Color Jitter (0.8, 0.8, 0.8, 0.2) \ $p=0.2$ \\
                & Gaussian Blur (3x3) (1.0, 2.0) \ $p=0.2$ \\
                                    & Random Grayscale \ $p=0.2$ \\
        \hline
    \end{tabular}
    \end{center}
    \caption{BYOL Hyperparameters.}
    \label{tab:hyperparams}
\end{table}

\subsubsection{Nearest Neighbors Details}
We give equal weight to visual and tactile distances for all of the tasks except bottle cap, where tactile and image features were given weights of 2 and 1, respectively. We do this because the quality of the neighbors on image data was poor and emphasizing the tactile data slightly vastly improves performance.

While executing NN imitation, we keep a buffer of recently executed neighbors that we call the reject buffer. Given a new observation, we pick the first nearest neighbor not in the reject buffer. This prevents the policy from getting stuck in loops if a chain of neighbors and actions are cyclical.
We set the reject buffer size to 10 for every task except Joystick Pulling, which is set to 3. 
The buffer, combined with the 2cm spatial subsampling are critical for the success of NN policies.

\subsubsection{BC Details}
We train BC end-to-end using standard MSE loss on the actions with the same learning rate as BYOL and a batch size of 64.
\subsubsection{NN-Torque Details}
Our hand does not have torque sensors, but is actuated by torque targets from a low-level PD position controller. We use the torque targets as a proxy for torque information since the desired torque will be higher when the finger is in contact with an object, but trying to move further inside, and lower when it is not in contact. 

\subsubsection{PCA Details}
We run PCA on the tactile play data and take the top 100 components for use as features. The captured variance is about 95\% and the entire explained variance ratio can be seen in Figure \ref{fig:ap_var}. By visualizing the reconstructions (Figure \ref{fig:ap_pca}), we can see that it retains a majority of the tactile information. 




\begin{figure}
    \centering
    \includegraphics[width=0.5\textwidth]{ap_figures/var.pdf}
    \caption{Explained variance ratio for PCA on the play tactile data. Most variance is captured in the first 100 components.}
    \label{fig:ap_var}
\end{figure}

\begin{figure*}
    \centering
    \begin{tabular}{@{}c@{}}
        \includegraphics[width=0.49\textwidth]{ap_figures/16451.pdf}
        \label{fig:ap_pca1}
    \end{tabular}
    \begin{tabular}{@{}c@{}}
        \includegraphics[width=0.49\textwidth]{ap_figures/27451.pdf}
        \label{fig:ap_pca2}
    \end{tabular}
    \caption{Tactile data and the PCA reconstruction of two using 100 components for two tactile readings. Most of the information is preserved, but we can see minor differences in magnitude and offset.}
    \label{fig:ap_pca}
\end{figure*}

\subsubsection{Shuffled Pad Details}
For this experiment, we permute the position of the 15 4x4 tactile sensors using the same permutation for both pretraining and deployment. This ensures that we're inputting the same data from each sensor to each location in the tactile image, but does not leverage the spatial locations of the pads on the hand. If spatial layout had no effect, we would expect no difference in the performance between this and \method.

\subsection{Additional Rollouts}
\label{sec:ap_rollouts}
We visualize extra rollouts for each task in Figures \ref{fig:ap_bottle}-\ref{fig:ap_joystick}.

\subsection{Tactile Image Visualization}
We visualize tactile images for each task in Figures \ref{fig:ap_tact_bottle}-\ref{fig:ap_tact_joystic}.


\begin{figure*}
    \centering
    \includegraphics[width=\textwidth]{ap_figures/bottle_rollouts.pdf}
    \caption{Additional rollouts for the Bottle Opening task.}
    \label{fig:ap_bottle}
\end{figure*}

\begin{figure*}
    \centering
    \includegraphics[width=\textwidth]{ap_figures/book_rollouts.pdf}
    \caption{Additional rollouts for the Book Opening task.}
    \label{fig:ap_book}
\end{figure*}

\begin{figure*}
    \centering
    \includegraphics[width=\textwidth]{ap_figures/bowl_rollouts.pdf}
    \caption{Additional rollouts for the Bowl Unstacking task.}
    \label{fig:ap_bowl}
\end{figure*}

\begin{figure*}
    \centering
    \includegraphics[width=\textwidth]{ap_figures/cup_rollouts.pdf}
    \caption{Additional rollouts for the Cup Unstacking task.}
    \label{fig:ap_cup}
\end{figure*}

\begin{figure*}[ht]
    \centering
    \includegraphics[width=\textwidth]{ap_figures/joystick_rollouts.pdf}
    \caption{Additional rollouts for the Joystick Pulling task.}
    \label{fig:ap_joystick}
\end{figure*}

\begin{figure*}
    \centering
    \includegraphics[width=0.9\textwidth]{ap_figures/bottle_tdex.pdf}
    \caption{Tactile Image for the Bottle Opening task.}
    \label{fig:ap_tact_bottle}
\end{figure*}

\begin{figure*}
    \centering
    \includegraphics[width=0.9\textwidth]{ap_figures/book_tdex.pdf}
    \caption{Tactile Image for the Book Opening task.}
    \label{fig:ap_tact_book}
\end{figure*}

\begin{figure*}
    \centering
    \includegraphics[width=0.9\textwidth]{ap_figures/bowl_tdex.pdf}
    \caption{Tactile Image for the Bowl Unstacking task.}
    \label{fig:ap_tact_bowl}
\end{figure*}

\begin{figure*}
    \centering
    \includegraphics[width=0.9\textwidth]{ap_figures/cup_tdex.pdf}
    \caption{Tactile Image for the Cup Unstacking task.}
    \label{fig:ap_tact_cup}
\end{figure*}

\begin{figure*}
    \centering
    \includegraphics[width=0.9\textwidth]{ap_figures/joystic_tdex.pdf}
    \caption{Tactile Image for the Joystick Pulling task.}
    \label{fig:ap_tact_joystic}
\end{figure*}