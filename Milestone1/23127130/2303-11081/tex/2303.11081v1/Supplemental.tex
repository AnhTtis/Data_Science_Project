
%% bare_adv.tex
%% V1.4b
%% 2015/08/26
%% by Michael Shell
%% See: 
%% http://www.michaelshell.org/
%% for current contact information.
%%
%% This is a skeleton file demonstrating the advanced use of IEEEtran.cls
%% (requires IEEEtran.cls version 1.8b or later) with an IEEE Computer
%% Society journal paper.
%%
%% Support sites:
%% http://www.michaelshell.org/tex/ieeetran/
%% http://www.ctan.org/pkg/ieeetran
%% and
%% http://www.ieee.org/

%%*************************************************************************
%% Legal Notice:
%% This code is offered as-is without any warranty either expressed or
%% implied; without even the implied warranty of MERCHANTABILITY or
%% FITNESS FOR A PARTICULAR PURPOSE! 
%% User assumes all risk.
%% In no event shall the IEEE or any contributor to this code be liable for
%% any damages or losses, including, but not limited to, incidental,
%% consequential, or any other damages, resulting from the use or misuse
%% of any information contained here.
%%
%% All comments are the opinions of their respective authors and are not
%% necessarily endorsed by the IEEE.
%%
%% This work is distributed under the LaTeX Project Public License (LPPL)
%% ( http://www.latex-project.org/ ) version 1.3, and may be freely used,
%% distributed and modified. A copy of the LPPL, version 1.3, is included
%% in the base LaTeX documentation of all distributions of LaTeX released
%% 2003/12/01 or later.
%% Retain all contribution notices and credits.
%% ** Modified files should be clearly indicated as such, including  **
%% ** renaming them and changing author support contact information. **
%%*************************************************************************


% *** Authors should verify (and, if needed, correct) their LaTeX system  ***
% *** with the testflow diagnostic prior to trusting their LaTeX platform ***
% *** with production work. The IEEE's font choices and paper sizes can   ***
% *** trigger bugs that do not appear when using other class files.       ***                          ***
% The testflow support page is at:
% http://www.michaelshell.org/tex/testflow/


% IEEEtran V1.7 and later provides for these CLASSINPUT macros to allow the
% user to reprogram some IEEEtran.cls defaults if needed. These settings
% override the internal defaults of IEEEtran.cls regardless of which class
% options are used. Do not use these unless you have good reason to do so as
% they can result in nonIEEE compliant documents. User beware. ;)
%
%\newcommand{\CLASSINPUTbaselinestretch}{1.0} % baselinestretch
%\newcommand{\CLASSINPUTinnersidemargin}{1in} % inner side margin
%\newcommand{\CLASSINPUToutersidemargin}{1in} % outer side margin
%\newcommand{\CLASSINPUTtoptextmargin}{1in}   % top text margin
%\newcommand{\CLASSINPUTbottomtextmargin}{1in}% bottom text margin




%
\documentclass[10pt,journal,compsoc]{IEEEtran}
% If IEEEtran.cls has not been installed into the LaTeX system files,
% manually specify the path to it like:
% \documentclass[10pt,journal,compsoc]{../sty/IEEEtran}


% For Computer Society journals, IEEEtran defaults to the use of 
% Palatino/Palladio as is done in IEEE Computer Society journals.
% To go back to Times Roman, you can use this code:
%\renewcommand{\rmdefault}{ptm}\selectfont





% Some very useful LaTeX packages include:
% (uncomment the ones you want to load)



% *** MISC UTILITY PACKAGES ***
%
%\usepackage{ifpdf}
% Heiko Oberdiek's ifpdf.sty is very useful if you need conditional
% compilation based on whether the output is pdf or dvi.
% usage:
% \ifpdf
%   % pdf code
% \else
%   % dvi code
% \fi
% The latest version of ifpdf.sty can be obtained from:
% http://www.ctan.org/pkg/ifpdf
% Also, note that IEEEtran.cls V1.7 and later provides a builtin
% \ifCLASSINFOpdf conditional that works the same way.
% When switching from latex to pdflatex and vice-versa, the compiler may
% have to be run twice to clear warning/error messages.






% *** CITATION PACKAGES ***
%
\ifCLASSOPTIONcompsoc
  % The IEEE Computer Society needs nocompress option
  % requires cite.sty v4.0 or later (November 2003)
  \usepackage[nocompress]{cite}
\else
  % normal IEEE
  \usepackage{cite}
\fi
\usepackage{wrapfig}
\usepackage{amsmath}
\usepackage{amsfonts}
\usepackage{bm}
\usepackage{subcaption}
\usepackage{mathtools}
\usepackage{amsthm}
\usepackage{amssymb}
\usepackage{multirow}
\usepackage{xcolor}
\usepackage{booktabs}
\usepackage{nicefrac}
\usepackage{url}
\usepackage{hyperref}
\usepackage{graphicx}
\usepackage{enumitem}
\usepackage{bbm}
% \usepackage{natbib}

\usepackage{algorithm}
\usepackage{algorithmic}



% cite.sty was written by Donald Arseneau
% V1.6 and later of IEEEtran pre-defines the format of the cite.sty package
% \cite{} output to follow that of the IEEE. Loading the cite package will
% result in citation numbers being automatically sorted and properly
% "compressed/ranged". e.g., [1], [9], [2], [7], [5], [6] without using
% cite.sty will become [1], [2], [5]--[7], [9] using cite.sty. cite.sty's
% \cite will automatically add leading space, if needed. Use cite.sty's
% noadjust option (cite.sty V3.8 and later) if you want to turn this off
% such as if a citation ever needs to be enclosed in parenthesis.
% cite.sty is already installed on most LaTeX systems. Be sure and use
% version 5.0 (2009-03-20) and later if using hyperref.sty.
% The latest version can be obtained at:
% http://www.ctan.org/pkg/cite
% The documentation is contained in the cite.sty file itself.
%
% Note that some packages require special options to format as the Computer
% Society requires. In particular, Computer Society  papers do not use
% compressed citation ranges as is done in typical IEEE papers
% (e.g., [1]-[4]). Instead, they list every citation separately in order
% (e.g., [1], [2], [3], [4]). To get the latter we need to load the cite
% package with the nocompress option which is supported by cite.sty v4.0
% and later.





% *** GRAPHICS RELATED PACKAGES ***
%
\ifCLASSINFOpdf
  % \usepackage[pdftex]{graphicx}
  % declare the path(s) where your graphic files are
  % \graphicspath{{../pdf/}{../jpeg/}}
  % and their extensions so you won't have to specify these with
  % every instance of \includegraphics
  % \DeclareGraphicsExtensions{.pdf,.jpeg,.png}
\else
  % or other class option (dvipsone, dvipdf, if not using dvips). graphicx
  % will default to the driver specified in the system graphics.cfg if no
  % driver is specified.
  % \usepackage[dvips]{graphicx}
  % declare the path(s) where your graphic files are
  % \graphicspath{{../eps/}}
  % and their extensions so you won't have to specify these with
  % every instance of \includegraphics
  % \DeclareGraphicsExtensions{.eps}
\fi
% graphicx was written by David Carlisle and Sebastian Rahtz. It is
% required if you want graphics, photos, etc. graphicx.sty is already
% installed on most LaTeX systems. The latest version and documentation
% can be obtained at: 
% http://www.ctan.org/pkg/graphicx
% Another good source of documentation is "Using Imported Graphics in
% LaTeX2e" by Keith Reckdahl which can be found at:
% http://www.ctan.org/pkg/epslatex
%
% latex, and pdflatex in dvi mode, support graphics in encapsulated
% postscript (.eps) format. pdflatex in pdf mode supports graphics
% in .pdf, .jpeg, .png and .mps (metapost) formats. Users should ensure
% that all non-photo figures use a vector format (.eps, .pdf, .mps) and
% not a bitmapped formats (.jpeg, .png). The IEEE frowns on bitmapped formats
% which can result in "jaggedy"/blurry rendering of lines and letters as
% well as large increases in file sizes.
%
% You can find documentation about the pdfTeX application at:
% http://www.tug.org/applications/pdftex





% *** MATH PACKAGES ***
%
%\usepackage{amsmath}
% A popular package from the American Mathematical Society that provides
% many useful and powerful commands for dealing with mathematics.
%
% Note that the amsmath package sets \interdisplaylinepenalty to 10000
% thus preventing page breaks from occurring within multiline equations. Use:
%\interdisplaylinepenalty=2500
% after loading amsmath to restore such page breaks as IEEEtran.cls normally
% does. amsmath.sty is already installed on most LaTeX systems. The latest
% version and documentation can be obtained at:
% http://www.ctan.org/pkg/amsmath





% *** SPECIALIZED LIST PACKAGES ***
%\usepackage{acronym}
% acronym.sty was written by Tobias Oetiker. This package provides tools for
% managing documents with large numbers of acronyms. (You don't *have* to
% use this package - unless you have a lot of acronyms, you may feel that
% such package management of them is bit of an overkill.)
% Do note that the acronym environment (which lists acronyms) will have a
% problem when used under IEEEtran.cls because acronym.sty relies on the
% description list environment - which IEEEtran.cls has customized for
% producing IEEE style lists. A workaround is to declared the longest
% label width via the IEEEtran.cls \IEEEiedlistdecl global control:
%
% \renewcommand{\IEEEiedlistdecl}{\IEEEsetlabelwidth{SONET}}
% \begin{acronym}
%
% \end{acronym}
% \renewcommand{\IEEEiedlistdecl}{\relax}% remember to reset \IEEEiedlistdecl
%
% instead of using the acronym environment's optional argument.
% The latest version and documentation can be obtained at:
% http://www.ctan.org/pkg/acronym


%\usepackage{algorithmic}
% algorithmic.sty was written by Peter Williams and Rogerio Brito.
% This package provides an algorithmic environment fo describing algorithms.
% You can use the algorithmic environment in-text or within a figure
% environment to provide for a floating algorithm. Do NOT use the algorithm
% floating environment provided by algorithm.sty (by the same authors) or
% algorithm2e.sty (by Christophe Fiorio) as the IEEE does not use dedicated
% algorithm float types and packages that provide these will not provide
% correct IEEE style captions. The latest version and documentation of
% algorithmic.sty can be obtained at:
% http://www.ctan.org/pkg/algorithms
% Also of interest may be the (relatively newer and more customizable)
% algorithmicx.sty package by Szasz Janos:
% http://www.ctan.org/pkg/algorithmicx




% *** ALIGNMENT PACKAGES ***
%
\usepackage{array}
% Frank Mittelbach's and David Carlisle's array.sty patches and improves
% the standard LaTeX2e array and tabular environments to provide better
% appearance and additional user controls. As the default LaTeX2e table
% generation code is lacking to the point of almost being broken with
% respect to the quality of the end results, all users are strongly
% advised to use an enhanced (at the very least that provided by array.sty)
% set of table tools. array.sty is already installed on most systems. The
% latest version and documentation can be obtained at:
% http://www.ctan.org/pkg/array


%\usepackage{mdwmath}
%\usepackage{mdwtab}
% Also highly recommended is Mark Wooding's extremely powerful MDW tools,
% especially mdwmath.sty and mdwtab.sty which are used to format equations
% and tables, respectively. The MDWtools set is already installed on most
% LaTeX systems. The lastest version and documentation is available at:
% http://www.ctan.org/pkg/mdwtools


% IEEEtran contains the IEEEeqnarray family of commands that can be used to
% generate multiline equations as well as matrices, tables, etc., of high
% quality.


%\usepackage{eqparbox}
% Also of notable interest is Scott Pakin's eqparbox package for creating
% (automatically sized) equal width boxes - aka "natural width parboxes".
% Available at:
% http://www.ctan.org/pkg/eqparbox




% *** SUBFIGURE PACKAGES ***
%\ifCLASSOPTIONcompsoc
%  \usepackage[caption=false,font=footnotesize,labelfont=sf,textfont=sf]{subfig}
%\else
%  \usepackage[caption=false,font=footnotesize]{subfig}
%\fi
% subfig.sty, written by Steven Douglas Cochran, is the modern replacement
% for subfigure.sty, the latter of which is no longer maintained and is
% incompatible with some LaTeX packages including fixltx2e. However,
% subfig.sty requires and automatically loads Axel Sommerfeldt's caption.sty
% which will override IEEEtran.cls' handling of captions and this will result
% in non-IEEE style figure/table captions. To prevent this problem, be sure
% and invoke subfig.sty's "caption=false" package option (available since
% subfig.sty version 1.3, 2005/06/28) as this is will preserve IEEEtran.cls
% handling of captions.
% Note that the Computer Society format requires a sans serif font rather
% than the serif font used in traditional IEEE formatting and thus the need
% to invoke different subfig.sty package options depending on whether
% compsoc mode has been enabled.
%
% The latest version and documentation of subfig.sty can be obtained at:
% http://www.ctan.org/pkg/subfig




% *** FLOAT PACKAGES ***
%
%\usepackage{fixltx2e}
% fixltx2e, the successor to the earlier fix2col.sty, was written by
% Frank Mittelbach and David Carlisle. This package corrects a few problems
% in the LaTeX2e kernel, the most notable of which is that in current
% LaTeX2e releases, the ordering of single and double column floats is not
% guaranteed to be preserved. Thus, an unpatched LaTeX2e can allow a
% single column figure to be placed prior to an earlier double column
% figure.
% Be aware that LaTeX2e kernels dated 2015 and later have fixltx2e.sty's
% corrections already built into the system in which case a warning will
% be issued if an attempt is made to load fixltx2e.sty as it is no longer
% needed.
% The latest version and documentation can be found at:
% http://www.ctan.org/pkg/fixltx2e


%\usepackage{stfloats}
% stfloats.sty was written by Sigitas Tolusis. This package gives LaTeX2e
% the ability to do double column floats at the bottom of the page as well
% as the top. (e.g., "\begin{figure*}[!b]" is not normally possible in
% LaTeX2e). It also provides a command:
%\fnbelowfloat
% to enable the placement of footnotes below bottom floats (the standard
% LaTeX2e kernel puts them above bottom floats). This is an invasive package
% which rewrites many portions of the LaTeX2e float routines. It may not work
% with other packages that modify the LaTeX2e float routines. The latest
% version and documentation can be obtained at:
% http://www.ctan.org/pkg/stfloats
% Do not use the stfloats baselinefloat ability as the IEEE does not allow
% \baselineskip to stretch. Authors submitting work to the IEEE should note
% that the IEEE rarely uses double column equations and that authors should try
% to avoid such use. Do not be tempted to use the cuted.sty or midfloat.sty
% packages (also by Sigitas Tolusis) as the IEEE does not format its papers in
% such ways.
% Do not attempt to use stfloats with fixltx2e as they are incompatible.
% Instead, use Morten Hogholm'a dblfloatfix which combines the features
% of both fixltx2e and stfloats:
%
% \usepackage{dblfloatfix}
% The latest version can be found at:
% http://www.ctan.org/pkg/dblfloatfix


%\ifCLASSOPTIONcaptionsoff
%  \usepackage[nomarkers]{endfloat}
% \let\MYoriglatexcaption\caption
% \renewcommand{\caption}[2][\relax]{\MYoriglatexcaption[#2]{#2}}
%\fi
% endfloat.sty was written by James Darrell McCauley, Jeff Goldberg and 
% Axel Sommerfeldt. This package may be useful when used in conjunction with 
% IEEEtran.cls'  captionsoff option. Some IEEE journals/societies require that
% submissions have lists of figures/tables at the end of the paper and that
% figures/tables without any captions are placed on a page by themselves at
% the end of the document. If needed, the draftcls IEEEtran class option or
% \CLASSINPUTbaselinestretch interface can be used to increase the line
% spacing as well. Be sure and use the nomarkers option of endfloat to
% prevent endfloat from "marking" where the figures would have been placed
% in the text. The two hack lines of code above are a slight modification of
% that suggested by in the endfloat docs (section 8.4.1) to ensure that
% the full captions always appear in the list of figures/tables - even if
% the user used the short optional argument of \caption[]{}.
% IEEE papers do not typically make use of \caption[]'s optional argument,
% so this should not be an issue. A similar trick can be used to disable
% captions of packages such as subfig.sty that lack options to turn off
% the subcaptions:
% For subfig.sty:
% \let\MYorigsubfloat\subfloat
% \renewcommand{\subfloat}[2][\relax]{\MYorigsubfloat[]{#2}}
% However, the above trick will not work if both optional arguments of
% the \subfloat command are used. Furthermore, there needs to be a
% description of each subfigure *somewhere* and endfloat does not add
% subfigure captions to its list of figures. Thus, the best approach is to
% avoid the use of subfigure captions (many IEEE journals avoid them anyway)
% and instead reference/explain all the subfigures within the main caption.
% The latest version of endfloat.sty and its documentation can obtained at:
% http://www.ctan.org/pkg/endfloat
%
% The IEEEtran \ifCLASSOPTIONcaptionsoff conditional can also be used
% later in the document, say, to conditionally put the References on a 
% page by themselves.





% *** PDF, URL AND HYPERLINK PACKAGES ***
%
%\usepackage{url}
% url.sty was written by Donald Arseneau. It provides better support for
% handling and breaking URLs. url.sty is already installed on most LaTeX
% systems. The latest version and documentation can be obtained at:
% http://www.ctan.org/pkg/url
% Basically, \url{my_url_here}.


% NOTE: PDF thumbnail features are not required in IEEE papers
%       and their use requires extra complexity and work.
%\ifCLASSINFOpdf
%  \usepackage[pdftex]{thumbpdf}
%\else
%  \usepackage[dvips]{thumbpdf}
%\fi
% thumbpdf.sty and its companion Perl utility were written by Heiko Oberdiek.
% It allows the user a way to produce PDF documents that contain fancy
% thumbnail images of each of the pages (which tools like acrobat reader can
% utilize). This is possible even when using dvi->ps->pdf workflow if the
% correct thumbpdf driver options are used. thumbpdf.sty incorporates the
% file containing the PDF thumbnail information (filename.tpm is used with
% dvips, filename.tpt is used with pdftex, where filename is the base name of
% your tex document) into the final ps or pdf output document. An external
% utility, the thumbpdf *Perl script* is needed to make these .tpm or .tpt
% thumbnail files from a .ps or .pdf version of the document (which obviously
% does not yet contain pdf thumbnails). Thus, one does a:
% 
% thumbpdf filename.pdf 
%
% to make a filename.tpt, and:
%
% thumbpdf --mode dvips filename.ps
%
% to make a filename.tpm which will then be loaded into the document by
% thumbpdf.sty the NEXT time the document is compiled (by pdflatex or
% latex->dvips->ps2pdf). Users must be careful to regenerate the .tpt and/or
% .tpm files if the main document changes and then to recompile the
% document to incorporate the revised thumbnails to ensure that thumbnails
% match the actual pages. It is easy to forget to do this!
% 
% Unix systems come with a Perl interpreter. However, MS Windows users
% will usually have to install a Perl interpreter so that the thumbpdf
% script can be run. The Ghostscript PS/PDF interpreter is also required.
% See the thumbpdf docs for details. The latest version and documentation
% can be obtained at.
% http://www.ctan.org/pkg/thumbpdf


% NOTE: PDF hyperlink and bookmark features are not required in IEEE
%       papers and their use requires extra complexity and work.
% *** IF USING HYPERREF BE SURE AND CHANGE THE EXAMPLE PDF ***
% *** TITLE/SUBJECT/AUTHOR/KEYWORDS INFO BELOW!!           ***
\newcommand{\embV}{\mathbf{V}}
\newcommand{\embU}{\mathbf{U}}
\newcommand{\embh}{\mathbf{h}}
\newcommand{\embH}{\mathbf{H}}
\newcommand{\embm}{\mathbf{m}}
\newcommand{\embX}{\mathbf{X}}
\newcommand{\embx}{\mathbf{x}}
\newcommand{\embe}{\mathbf{e}}
\newcommand{\embg}{\mathbf{g}}

\newcommand{\hisX}{\overline{\mathbf{X}}}
\newcommand{\hisV}{\overline{\mathbf{V}}}
\newcommand{\hisH}{\overline{\mathbf{H}}}
\newcommand{\temH}{\widehat{\mathbf{H}}}
\newcommand{\temV}{\widehat{\mathbf{V}}}

\newcommand{\vecx}{\mathbf{x}}
\newcommand{\vecy}{\mathbf{y}}
\newcommand{\vecd}{\mathbf{d}}
\newcommand{\vecM}{\mathbf{M}}

\newcommand{\rmd}{\,\mathrm{d}}

\newcommand{\update}{u}
\newcommand{\aggregate}{\oplus}
\newcommand{\loss}{\mathcal{L}}
\newcommand{\inbatch}{\mathcal{V}_{\mathcal{B}}}
\newcommand{\compensation}{\mathbf{C}}
\newcommand{\jacobian}{\mathbf{J}}
\newcommand{\dropout}{\mathbf{Dropout}}
\newcommand{\neighbor}[1]{\mathcal{N}(#1)}
\newcommand{\kneighbor}[2]{\mathcal{N}^{#2}(#1)}

\newcommand{\mr}[2]{\multirow{#1}{*}{#2}}
\newcommand{\mc}[3]{\multicolumn{#1}{#2}{#3}}
\newcommand{\udfsection}[1]{\noindent\textbf{#1}\, }

\newtheorem{assumption}{Assumption}
\newtheorem{lemma}{Lemma}
\newtheorem{proposition}{Proposition}
\newtheorem{theorem}{Theorem}


\newif\ifproof\prooftrue
% \newif\ifproof\prooffalse 

\newcommand{\showproof}[1]{\ifproof{#1}\else{}\fi}



%\newif\ifupdate\updatetrue
\newif\ifupdate\updatefalse 
\newcommand{\modify}[2]{\ifupdate{#1}\else{\color{red}#2}\fi}

\newcommand\MYhyperrefoptions{bookmarks=true,bookmarksnumbered=true,
pdfpagemode={UseOutlines},plainpages=false,pdfpagelabels=true,
colorlinks=true,linkcolor={black},citecolor={black},urlcolor={black},
pdftitle={Bare Demo of IEEEtran.cls for Computer Society Journals},%<!CHANGE!
pdfsubject={Typesetting},%<!CHANGE!
pdfauthor={Michael D. Shell},%<!CHANGE!
pdfkeywords={Computer Society, IEEEtran, journal, LaTeX, paper,
             template}}%<^!CHANGE!
%\ifCLASSINFOpdf
%\usepackage[\MYhyperrefoptions,pdftex]{hyperref}
%\else
%\usepackage[\MYhyperrefoptions,breaklinks=true,dvips]{hyperref}
%\usepackage{breakurl}
%\fi
% One significant drawback of using hyperref under DVI output is that the
% LaTeX compiler cannot break URLs across lines or pages as can be done
% under pdfLaTeX's PDF output via the hyperref pdftex driver. This is
% probably the single most important capability distinction between the
% DVI and PDF output. Perhaps surprisingly, all the other PDF features
% (PDF bookmarks, thumbnails, etc.) can be preserved in
% .tex->.dvi->.ps->.pdf workflow if the respective packages/scripts are
% loaded/invoked with the correct driver options (dvips, etc.). 
% As most IEEE papers use URLs sparingly (mainly in the references), this
% may not be as big an issue as with other publications.
%
% That said, Vilar Camara Neto created his breakurl.sty package which
% permits hyperref to easily break URLs even in dvi mode.
% Note that breakurl, unlike most other packages, must be loaded
% AFTER hyperref. The latest version of breakurl and its documentation can
% be obtained at:
% http://www.ctan.org/pkg/breakurl
% breakurl.sty is not for use under pdflatex pdf mode.
%
% The advanced features offer by hyperref.sty are not required for IEEE
% submission, so users should weigh these features against the added
% complexity of use.
% The package options above demonstrate how to enable PDF bookmarks
% (a type of table of contents viewable in Acrobat Reader) as well as
% PDF document information (title, subject, author and keywords) that is
% viewable in Acrobat reader's Document_Properties menu. PDF document
% information is also used extensively to automate the cataloging of PDF
% documents. The above set of options ensures that hyperlinks will not be
% colored in the text and thus will not be visible in the printed page,
% but will be active on "mouse over". USING COLORS OR OTHER HIGHLIGHTING
% OF HYPERLINKS CAN RESULT IN DOCUMENT REJECTION BY THE IEEE, especially if
% these appear on the "printed" page. IF IN DOUBT, ASK THE RELEVANT
% SUBMISSION EDITOR. You may need to add the option hypertexnames=false if
% you used duplicate equation numbers, etc., but this should not be needed
% in normal IEEE work.
% The latest version of hyperref and its documentation can be obtained at:
% http://www.ctan.org/pkg/hyperref





% *** Do not adjust lengths that control margins, column widths, etc. ***
% *** Do not use packages that alter fonts (such as pslatex).         ***
% There should be no need to do such things with IEEEtran.cls V1.6 and later.
% (Unless specifically asked to do so by the journal or conference you plan
% to submit to, of course. )


% correct bad hyphenation here
\hyphenation{op-tical net-works semi-conduc-tor}


\begin{document}
%
% paper title
% Titles are generally capitalized except for words such as a, an, and, as,
% at, but, by, for, in, nor, of, on, or, the, to and up, which are usually
% not capitalized unless they are the first or last word of the title.
% Linebreaks \\ can be used within to get better formatting as desired.
% Do not put math or special symbols in the title.
\title{LMC: Fast Training of Convolutional Graph Neural Networks and Recurrent Graph Neural Networks with Provable Convergence \\ Appendix}
%
%
% author names and IEEE memberships
% note positions of commas and nonbreaking spaces ( ~ ) LaTeX will not break
% a structure at a ~ so this keeps an author's name from being broken across
% two lines.
% use \thanks{} to gain access to the first footnote area
% a separate \thanks must be used for each paragraph as LaTeX2e's \thanks
% was not built to handle multiple paragraphs
%
%
%\IEEEcompsocitemizethanks is a special \thanks that produces the bulleted
% lists the Computer Society journals use for "first footnote" author
% affiliations. Use \IEEEcompsocthanksitem which works much like \item
% for each affiliation group. When not in compsoc mode,
% \IEEEcompsocitemizethanks becomes like \thanks and
% \IEEEcompsocthanksitem becomes a line break with idention. This
% facilitates dual compilation, although admittedly the differences in the
% desired content of \author between the different types of papers makes a
% one-size-fits-all approach a daunting prospect. For instance, compsoc 
% journal papers have the author affiliations above the "Manuscript
% received ..."  text while in non-compsoc journals this is reversed. Sigh.

\author{Jie~Wang,~\IEEEmembership{Senior Member,~IEEE,} Zhihao~Shi, Xize~Liang,\\Shuiwang~Ji,~\IEEEmembership{Senior Member,~IEEE,} and~Feng~Wu,~\IEEEmembership{Fellow,~IEEE}% <-this % stops a space
\IEEEcompsocitemizethanks{\IEEEcompsocthanksitem M. Shell was with the Department
of Electrical and Computer Engineering, Georgia Institute of Technology, Atlanta,
GA, 30332.\protect\\
% note need leading \protect in front of \\ to get a newline within \thanks as
% \\ is fragile and will error, could use \hfil\break instead.
E-mail: see http://www.michaelshell.org/contact.html
\IEEEcompsocthanksitem J. Doe and J. Doe are with Anonymous University.}% <-this % stops a space
\thanks{Manuscript received April 19, 2005; revised August 26, 2015.}}

% note the % following the last \IEEEmembership and also \thanks - 
% these prevent an unwanted space from occurring between the last author name
% and the end of the author line. i.e., if you had this:
% 
% \author{....lastname \thanks{...} \thanks{...} }
%                     ^------------^------------^----Do not want these spaces!
%
% a space would be appended to the last name and could cause every name on that
% line to be shifted left slightly. This is one of those "LaTeX things". For
% instance, "\textbf{A} \textbf{B}" will typeset as "A B" not "AB". To get
% "AB" then you have to do: "\textbf{A}\textbf{B}"
% \thanks is no different in this regard, so shield the last } of each \thanks
% that ends a line with a % and do not let a space in before the next \thanks.
% Spaces after \IEEEmembership other than the last one are OK (and needed) as
% you are supposed to have spaces between the names. For what it is worth,
% this is a minor point as most people would not even notice if the said evil
% space somehow managed to creep in.



% The paper headers
\markboth{Journal of \LaTeX\ Class Files,~Vol.~14, No.~8, August~2015}%
{Shell \MakeLowercase{\textit{et al.}}: Bare Advanced Demo of IEEEtran.cls for IEEE Computer Society Journals}
% The only time the second header will appear is for the odd numbered pages
% after the title page when using the twoside option.
% 
% *** Note that you probably will NOT want to include the author's ***
% *** name in the headers of peer review papers.                   ***
% You can use \ifCLASSOPTIONpeerreview for conditional compilation here if
% you desire.



% The publisher's ID mark at the bottom of the page is less important with
% Computer Society journal papers as those publications place the marks
% outside of the main text columns and, therefore, unlike regular IEEE
% journals, the available text space is not reduced by their presence.
% If you want to put a publisher's ID mark on the page you can do it like
% this:
%\IEEEpubid{0000--0000/00\$00.00~\copyright~2015 IEEE}
% or like this to get the Computer Society new two part style.
%\IEEEpubid{\makebox[\columnwidth]{\hfill 0000--0000/00/\$00.00~\copyright~2015 IEEE}%
%\hspace{\columnsep}\makebox[\columnwidth]{Published by the IEEE Computer Society\hfill}}
% Remember, if you use this you must call \IEEEpubidadjcol in the second
% column for its text to clear the IEEEpubid mark (Computer Society journal
% papers don't need this extra clearance.)



% use for special paper notices
%\IEEEspecialpapernotice{(Invited Paper)}



% for Computer Society papers, we must declare the abstract and index terms
% PRIOR to the title within the \IEEEtitleabstractindextext IEEEtran
% command as these need to go into the title area created by \maketitle.
% As a general rule, do not put math, special symbols or citations
% in the abstract or keywords.


% make the title area
\maketitle


% To allow for easy dual compilation without having to reenter the
% abstract/keywords data, the \IEEEtitleabstractindextext text will
% not be used in maketitle, but will appear (i.e., to be "transported")
% here as \IEEEdisplaynontitleabstractindextext when compsoc mode
% is not selected <OR> if conference mode is selected - because compsoc
% conference papers position the abstract like regular (non-compsoc)
% papers do!
\IEEEdisplaynontitleabstractindextext
% \IEEEdisplaynontitleabstractindextext has no effect when using
% compsoc under a non-conference mode.


% For peer review papers, you can put extra information on the cover
% page as needed:
% \ifCLASSOPTIONpeerreview
% \begin{center} \bfseries EDICS Category: 3-BBND \end{center}
% \fi
%
% For peerreview papers, this IEEEtran command inserts a page break and
% creates the second title. It will be ignored for other modes.
\IEEEpeerreviewmaketitle



\appendices
\section{More Details about Experiments}
In this section, we introduce more details about our experiments, including datasets, training and evaluation protocols, and implementations.

\subsection{Datasets}
We evaluate LMC on three small datasets, including Cora, Citeseer, and PubMed from Planetoid \cite{planetoid}, and five large datasets, PPI, Reddit \cite{graphsage}, AMAZON \cite{amazon}, Ogbn-arxiv, and Ogbn-products \cite{ogb}.
All of the datasets do not contain personally identifiable information or offensive content.
Table \ref{tab:datasets} shows the summary statistics of the datasets.
Details about the datasets are as follows.
\begin{itemize}[leftmargin=10mm]
    \item Cora, Citeseer, and PubMed are directed citation networks. Each node indicates a paper with the corresponding bag-of-words features and each directed edge indicates that one paper cites another one. The task is to classify academic papers into different subjects.
    \item PPI contains 24 protein-protein interaction graphs. Each graph corresponds to a human tissue. Each node indicates a protein with positional gene sets, motif gene sets and immunological signatures as node features. Edges represent interactions between proteins. The task is to classify protein functions.
    (proteins) and edges (interactions).
    \item REDDIT is a post-to-post graph constructed from Reddit. Each node indicates a post and each edge between posts indicates that the same user comments on both. The task is to classify Reddit posts into different communities based on (1) the GloVe CommonCrawl word vectors \cite{glove} of the post titles and comments, (2) the post’s scores, and (3) the numbers of comments made on the posts.
    \item AMAZON is an Amazon product co-purchasing network. Each node indicates a product and each edge between two products indicates that the products are purchased together. The task is to predict product types \modify{}{without node features based on rare labeled nodes}. We set the training set fraction be 0.06\% in experiments.
    \item Ogbn-arxiv is a directed citation network between all Computer Science (CS) arXiv papers indexed by MAG \cite{mag}. Each node is an arXiv paper and each directed edge indicates that one paper cites another one. The task is to classify unlabeled arXiv papers into different primary categories based on labeled papers and node features, which are computed by averaging word2vec \cite{word2vec} embeddings of words in papers' title and abstract.
    \item Ogbn-product is a large Amazon product co-purchasing network \modify{}{with rich node features}. Each node indicates a product and each edge between two products indicates that the products are purchased together. The task is to predict product types based on low-dimensional bag-of-words features of product descriptions processed by Principal Component Analysis.
\end{itemize}

\begin{table}[htbp]
  \begin{center}
    \caption{Statistics of the datasets used in our experiments.
    }\label{tab:datasets}
    \vspace{5pt}
    \scalebox{1}{
    \begin{tabular}{ccccc}
    \toprule
    \textbf{Dataset} & \textbf{\#Graphs} & \textbf{\#Classes} &\textbf{Total \#Nodes} & \textbf{Total \#Edges}  \\
      \midrule
      \midrule
      Cora & 1 & 7  & 2,708 & 5,278\\
      Citeseer & 1 & 6  & 3,327 & 4,552 \\
      PubMed & 1 & 3  & 19,717 & 44,324 \\
      \midrule
      PPI & 24 & 121 & 56,944 & 793,632 \\
      REDDIT & 1 & 41 & 232,965 & 11,606,919   \\
      AMAZON & 1 & 58 & 334,863 & 925,872   \\
      Ogbn-arxiv & 1 & 40  & 169,343 & 1,157,799  \\
      Ogbn-product & 1 & 47  & 2,449,029 & 61,859,076 \\ % 61,859,076
      \bottomrule
    \end{tabular}
    }
  \end{center}
\end{table} 




% you can choose not to have a title for an appendix
% if you want by leaving the argument blank
\subsection{Training and Evaluation Protocols}

We run all the experiments on a single GeForce RTX 2080 Ti (11 GB). All the models are implemented in Pytorch \cite{pytorch} and PyTorch Geometric \cite{pyg} based on the official implementation of \cite{gas}\footnote{\url{https://github.com/rusty1s/pyg_autoscale}. The owner does not mention the license.} and \cite{ignn}\footnote{\url{https://github.com/SwiftieH/IGNN}, licensed under the MIT License.}.

\udfsection{Data Splitting.} We use the data splitting strategies following previous works \cite{gas, ignn}.


% ogbn-arxiv is a semi-supervised dataset and is split based on the publication dates of the papers (nodes). Specifically, the training dataset consists of 90,941 papers published until 2017; the validation dataset consists of 29,799 papers published in 2018; the test dataset consists of 48,603 papers published since 2019. Other datasets are supervised datasets and are split randomly. Specifically, PATTERN consists of 10K training, 2K validation, and 2K test graphs; ZINC consists of 10K training, 1K validation, and 1K test graphs; CIFAR10 consists of 45K training, 5K validation, and 10K test graphs; MNIST consists of 55K training, 5K validation, and 10K test graphs.


\subsection{Implementation Details and Hyperparameters}



\subsubsection{Hyperparameters} 

We report the embedding dimensions, the number of partitions, sampled clusters per mini-batch, and learning rates (LR) for each dataset in Table \ref{tab:hyperprameters}.
For the small datasets, we partition each graph into ten subgraphs. For the large datasets, we select the number of partitions and sampled clusters per mini-batch to avoid running out of memory on GPUs.
For a fair comparison, we use the hyperparameters for each training method.
We provide other hyperparameters in the \textit{json} file in our codes.


\begin{table}[h]
  \centering
  \caption{%
  \textbf{The hyperprameters used in the experiments.}
  }\label{tab:hyperprameters}
  \setlength{\tabcolsep}{5pt}
  \resizebox{1.0\linewidth}{!}{%
  \begin{tabular}{lcccccc}
    \toprule
    \textbf{Dataset}  & \textbf{Dimensions} & \textbf{Partitions} & \textbf{Clusters} & \textbf{LR (METIS partition)} & \textbf{LR (random partition)} \\
    \midrule
    \textsc{Cora} & 128 & 10 & 2 & 0.003 & 0.001\\
    \textsc{CiteSeer} & 128 & 10 & 2 & 0.01 & 0.001 \\
    \textsc{PubMed} & 128 & 10 & 2 & 0.003 & 0.0003 \\
    \textsc{PPI} & 1024 & 10 & 2 & 0.01 & - \\
    \textsc{REDDIT} & 256 & 200 & 100 & 0.003 & - \\
    \textsc{AMAZON} & 64 & 40 & 1 & 0.003 & - \\
    \textsc{Ogbn-arxiv} & 256 & 80 & 10 & 0.003 & - \\
    \textsc{Ogbn-products} & 256 & 1500 & 10 & 0.001 & - \\
    \bottomrule
  \end{tabular}
  }
\end{table}


\subsubsection{An Efficient Implement of LMC for RecGCN}

The equilibrium equations of RecGCN (see Equation (1) in the main text) are
\begin{align*}
    \embH  = \sigma(  \mathbf{W} \embH   \hat{\mathbf{A}} + \mathbf{b}(\embX) ),
\end{align*}
where the nonlinear activation function $\sigma(\cdot)$ is the ReLU activation $\sigma(\cdot)=\max(\cdot,0)$, the function $\mathbf{b}(\embX) = \mathbf{P}\embX +\mathbf{c}$ is an affine function with parameters $\mathbf{P} \in \mathbb{R}^{d \times d_{x}},\mathbf{c} \in \mathbb{R}^{d}$
, the matrix $\mathbf{\hat{A}} = (\mathbf{D}+\mathbf{I})^{-1/2} (\mathbf{A}+\mathbf{I})(\mathbf{D}+\mathbf{I})^{-1/2}$ is the normalized adjacency matrix with self-loops, and $\mathbf{D}$ is the degree matrix.



Let $\mathbf{Z} = \mathbf{W} \embH   \hat{\mathbf{A}} + \mathbf{b}(\embX)$.
The Jacobian vector-product is $\langle \vec{\embV}, \nabla_{\embH} \vec{f}_{\theta} \rangle$ \cite{ignn} is
\begin{align*}
     \mathbf{W}^{\top} (\sigma'(\mathbf{Z}) \odot \embV) \hat{\mathbf{A}}^{\top}.
\end{align*}

%Notice that the aggregation matrix of RecGCN $\mathbf{\hat{A}}$ is symmetric.
For two nodes $v_i,v_j$ such that $v_i \in \mathcal{N}(v_j)$, we have
\begin{align*}
    \vec{\embV}_i^{\top} \frac{\partial [f_{\theta}]_i}{\partial \embh_{j}} = \mathbf{W}^{\top} \left(\sigma({\mathbf{Z}}_i)' \odot {\embV}_i\right) \mathbf{\hat{A}}_{ij},
\end{align*}
which is only depend on the nodes $v_i,v_j$ rather than the 2-hop neighbors of $v_j$.


Therefore, by additionally storing the historical auxiliary variables $\mathbf{Z}$, we implement LMC based on 1-hop neighbors, which employs $\mathcal{O}(n_{\max} |\inbatch| d)$ GPU memory in backward passes.

\subsubsection{Implement of Baslines}

For a fair comparison, we implement GAS by setting the gradient compensation $C_b$ to be zero. We implement CLUSTER by removing edges between partitioned subgraphs and then running GAS based on them.

\subsubsection{Normalization Technique}

In Section 4 in the main text, we assume that the subgraph $\inbatch$ is uniformly sampled from $\mathcal{V}$ and the corresponding set of labeled nodes {\small$\mathcal{V}_{L_{\mathcal{B}}} = \inbatch \cap \mathcal{V}_{L}$} is uniformly sampled from {\small$\mathcal{V}_{L}$}. To enforce the assumption, we use the normalization technique to reweight Equations (5) and (6) in the main text.



Suppose we partition the whole graph $\mathcal{V}$ into $b$ parts $\{\mathcal{V}_{\mathcal{B}_i}\}_{i=1}^b$ and then uniformly sample $c$ clusters without replacement to construct subgraph $\inbatch$. By the normalization technique, Equation (5) becomes
\begin{align}
    \mathbf{g}_w(\inbatch) = \frac{b|\mathcal{V}_{L_{\mathcal{B}}}|}{c|\mathcal{V}_L|} \frac{1}{|\mathcal{V}_{L_{\mathcal{B}}}|} \sum_{v_j \in \mathcal{V}_{L_{\mathcal{B}}}}\nabla_{w} l_w(\embh_j ,y_j),
\end{align}
where $\frac{b|\mathcal{V}_{L_{\mathcal{B}}}|}{c|\mathcal{V}_L|}$ is the corresponding weight. Similarly, Equation (6) becomes
\begin{align}
    \mathbf{g}_{\theta}(\inbatch) = \frac{b|\inbatch|}{c|\mathcal{V}|} \frac{|\mathcal{V}|}{|\inbatch|} \sum_{v_j \in \inbatch} \nabla_{\theta} \update(\embh_j ,\embm_{\neighbor{v_j}}  ,\embx_j) \embV_{j} \label{eqn:sgd_theta},
\end{align}
where $\frac{b|\inbatch|}{c|\mathcal{V}|}$ is the corresponding weight.


\subsection{Randomly Using Dropout at Different Training Steps}


In this section, we propose a trick to handle the randomness introduced by dropout \cite{dropout}. Let $\dropout(\embX) = \frac{1}{1-p} \mathbf{M} \circ \embX$ be the dropout operation, where $\mathbf{M}_{ij} \sim \text{Bern}(1-p)$ are i.i.d Bernoulli random variables, and $\circ$ is the element-wise product.


As shown by \cite{vrgcn}, with dropout \cite{dropout} of input features $\mathbf{X}$, historical embeddings and auxiliary variables become random variables, leading to inaccurate compensation messages.
Specifically, the solutions to Equations (1) and (3) in the main text are the function of the dropout features $\frac{1}{1-p} \mathbf{M}^{(k)} \circ \embX$ at the training step $k$, while the dropout features become $\frac{1}{1-p} \mathbf{M}^{(k+1)} \circ \embX$ at the training step $k+1$ (we assume that the learning rate $\eta=0$ to simplify the analysis).

\modify{}{
As the historical information under the dropout operation at the training step $k$ may be very inaccurate at the training step $k+1$, \cite{vrgcn} propose to first compute the random embeddings with dropout and the mean embeddings without dropout, and then use the mean embeddings to update historical information.
However, simultaneously computing two versions of embeddings in RecGNNs leads to double computational costs of solving Equations (1) and (3).
We thus propose to randomly use the dropout operation at different training steps and update the historical information when the dropout operation is invalid. Specifically, at each training step $k$, we either update the historical information without the dropout operation with probability $q$ or use the dropout operation without updating the historical information. We set $q=0.5$ in all experiments. Due to the trick, the historical embeddings and auxiliary variables depend on the stable input features $\embX$ rather than the random dropout features $\frac{1}{1-p} \mathbf{M} \circ \embX$. The trick can reduce overfitting and control variate for dropout efficiently.
}






\section{Message Passing}

In the GNN literature, message passing usually refers to neural message passing \cite{mpnn}, a framework in which vector messages are exchanged between nodes and updated using neural networks.


Overall, message passing is an iterative method and follows an \textit{aggregate} and \textit{update} scheme. In each iteration, the representation of each node $v_i$ is \textit{updated} using information \textit{aggregated} from $v_i$'s neighborhood $\mathcal{N}(v_i)$. Formally, the $k$-th iteration process \cite{grl} can be written as
\begin{align}
    \embm_{\neighbor{v_i}}^{(l)}&=\aggregate^{(l)}\left( \left\{g^{(l)}(\embh_j^{(l-1)})|v_j \in\neighbor{v_i}\right\}\right)\label{eqn:aggregate},\\
    \embh_i^{(l)}&=\update^{(l)}\left(\embh_i^{(l-1)}, \embm_{\neighbor{v_i}}^{(l)},\embx_i\right)\label{eqn:update},
\end{align}
where {\small $g^{(l)}$} is the function generating \textit{individual messages} for each neighborhood of $i$, {\small $\aggregate^{(l)}$} is the aggregation function mapping from a set of messages to the final message {\small $\embm_{\neighbor{v_i}}^{(l)}$}, and {\small $\update^{(l)}$} is the update function combining the message {\small $\embm_{\neighbor{v_i}}^{(l)}$} with the previous embedding {\small $\embh_i^{(l-1)}$} to obtain a new embedding for $i$.
To decrease memory costs, RecGNNs use the same generation function {\small $g^{(l)}$}, the aggregation function {\small $\aggregate^{(l)}$}, and the update function {\small $\update^{(l)}$} in each iteration, so that the message passing iteration only saves the latest embeddings {\small $\embh_i^{(K)}$} rather than all historical embeddings {\small $\{\embh_i^{(l)}\}_{l=1}^{K}$} \cite{ignn, deq}.



\section{Well-posedness Conditions of RecGNNs}\label{sec:well-posedness}



In this section, we provide tractable well-posedness conditions \cite{ignn} of RecGNNs to ensure the existence and uniqueness of the solution to Equation (1) in the main text.


The Equation (1)  in the main text of RecGNNs with the message passing functions in GCN can be formulated as
\begin{align*}
    \embH   = f_{\theta}(\embH  ;\embX) = \sigma(  \mathbf{W} \embH   \hat{\mathbf{A}} + \mathbf{b}(\embX) ),
\end{align*}
where $\mathbf{\hat{A}}$ is the aggregation matrix,  $\sigma(\cdot)$ is the nonlinear activation function, and $\mathbf{b}(\embX) $ is an affine function to encode the input features.
The Perron-Frobenius (PF) sufficient condition for well-posedness \cite{idl} requires the activation function $\sigma$ is component-wise non-expansive and $\lambda_{pf}(|\hat{\mathbf{A}}^{\top} \otimes \mathbf{W}|) = \lambda_{pf}(\hat{\mathbf{A}})\lambda_{pf}(|\mathbf{W}|) < 1$
, where $\otimes$ is the Kronecker product.
As pointed out in \cite{ignn}, if the Perron-Frobenius (PF) sufficient condition holds, the solution $\embH  $ can be achieved by iterating Equation (1) in the main text to convergence.



We further discuss how to enforce the non-convex constraint $\lambda_{pf}(|\mathbf{W}|) < (\lambda_{pf}(\hat{\mathbf{A}}))^{-1}$.
As $\lambda_{pf}(|\mathbf{W}|) \leq \| \mathbf{W} \|_p$ holds for induced norms $\| \cdot \|_p$, we enforce the stricter condition $\| \mathbf{W} \|_p < \lambda_{pf}(\hat{\mathbf{A}})^{-1}$. As pointed out in \cite{ignn}, if $p=1$ or $p=\infty$, we efficiently implement $\| \mathbf{W} \|_p < 1$ by projection. We follow the implementation in experiments and view $p$ as a hyperparameter.

\section{Examples about Long-Range Dependencies}

We provide an example inspired by label propagation algorithm.
In a citation network, if a paper $A_1$ cites another paper $A_2$, then they are likely to belong to the same area.
Recursively following citation relations between papers $A_i$ and $A_{i+1}$, we deduce that $A_1$ and $A_{K}$ are dependent for any positive integer $K$ while the dependency decreases as $K$ grows (as it is a hidden Markov process).
% When $K$ is large, the dependency between the papers $A_1$ and $A_{K}$ is a long-range dependency.
For a large $K$, the long-range dependency implies that papers $A_1$ and $A_{K}$ are likely to belong to the same area, although the confident is usually lower than that provided by $A_{2}$.
Anyway, the long-range dependency is still helpful to denoise and predict the area of $A_1$, when the features of $A_1,A_2,\dots,A_{K-1}$ are noisy and even missing


\section{Computational Complexity}


We summarize the computational complexity in Table \ref{tab:complexity}, where $n_{\max}$ is the maximum of neighborhoods, $K$ is the number of message passing iterations, $\mathcal{V}_{\mathcal{B}}$ is a set of nodes in a sampled mini-batch, $d$ is the embedding dimension, $\mathcal{V}$ is the set of nodes in the whole graph, and $\mathcal{E}$ is the set of edges in the whole graph.

\begin{table}[htbp]
    \centering
    \caption{
    Time and space complexity of message passing based GNNs.
    }
    \label{tab:complexity}
    % \vspace{-0.5mm}
    \begin{tabular}{ccc}
    \toprule
        \textbf{Method} & \textbf{Time complexity per gradient update}  &\textbf{Memory per gradient update}   \\
        \midrule
        GD and naive SGD  & $\mathcal{O}(K(|\mathcal{E}|d+|\mathcal{V}| d^2))$ & $\mathcal{O}(|\mathcal{V}| d)$ \\
        CLUSTER \cite{cluster_gcn}  & $\mathcal{O}( K(|\inbatch|^2d+|\inbatch| d^2) )$ & $\mathcal{O}(|\inbatch| d)$ \\
        GAS \cite{gas} & $\mathcal{O}( K(|\inbatch|^2d+|\inbatch| d^2) )$ & $\mathcal{O}( n_{\max} |\inbatch| d)$ \\
        \midrule
        LMC for RecGCN & $\mathcal{O}( K(|\inbatch|^2d+|\inbatch| d^2) )$ & $\mathcal{O}(n_{\max} |\inbatch| d)$ \\
        LMC for general RecGNNs & $\mathcal{O}( K(|\inbatch|^2d+|\inbatch| d^2) )$ & $\mathcal{O}(n_{\max}^2 |\inbatch| d)$ \\
    \bottomrule
    \end{tabular}
    % \vspace{-3mm}
\end{table}

\section{Proof}

\subsection{Proof of Theorem 1}
\begin{proof}
    As the labeled nodes $\mathcal{V}_{L_{\mathcal{B}}} = \inbatch \cap \mathcal{V}_{L}$ is uniformly sampled from $\mathcal{V}_{L}$, the expectation of $ \mathbf{g}_w(\inbatch)$ is
    \begin{align*}
        \mathbb{E}[\mathbf{g}_w(\inbatch)] &=\mathbb{E}[ \frac{1}{|\mathcal{V}_{L_{\mathcal{B}}}|} \sum_{v_j \in \mathcal{V}_{L_{\mathcal{B}}}}\nabla_{w} l_w(\embh_j ,y_j)]\\
        &= \nabla_{w} \mathbb{E}[ l_w(\embh_j ,y_j)]\\
        &= \nabla_{w} \loss.
     \end{align*}
     As the subgraph $\inbatch$ is uniformly sampled from $\mathcal{V}$, the expectation of $\mathbf{g}_{\theta_l}(\inbatch)$ is
    \begin{align*}
        \mathbb{E}[\mathbf{g}_{\theta_l}(\inbatch)] &= \mathbb{E}[\frac{|\mathcal{V}|}{|\mathcal{V}_\mathcal{B}|}\sum_{v_j\in\mathcal{V}_\mathcal{B}}\left(\nabla_{\theta_l}u^{(l)}(\embh^{(l-1)}_j, \embm^{(l-1)}_{\neighbor{v_j}}, \embx_j)\right)] \embV_{j}^{(l)} \\
        &= |\mathcal{V}| \mathbb{E}[ \nabla_{\theta_l}u^{(l)}(\embh^{(l-1)}_j, \embm^{(l-1)}_{\neighbor{v_j}}, \embx_j) ] \embV_{j}^{(l)} ]\\
        &=|\mathcal{V}| \frac{1}{|\mathcal{V}|} \sum_{v_j \in \mathcal{V}} \nabla_{\theta_l}u^{(l)}(\embh^{(l-1)}_j, \embm^{(l-1)}_{\neighbor{v_j}}, \embx_j) \embV_{j}^{(l)}\\
        &= \sum_{v_j \in \mathcal{V}} \nabla_{\theta_l}u^{(l)}(\embh^{(l-1)}_j, \embm^{(l-1)}_{\neighbor{v_j}}, \embx_j) \embV_{j}^{(l)}\\
        &= \nabla_{\theta_l} \loss,\,\forall\,l=1,\ldots,L.
    \end{align*}
\end{proof}


\subsection{Proof of Theorem 2}

\begin{proof}
    As the labeled nodes $\mathcal{V}_{L_{\mathcal{B}}} = \inbatch \cap \mathcal{V}_{L}$ is uniformly sampled from $\mathcal{V}_{L}$, the expectation of $ \mathbf{g}_w(\inbatch)$ is
    \begin{align*}
        \mathbb{E}[\mathbf{g}_w(\inbatch)] &=\mathbb{E}[ \frac{1}{|\mathcal{V}_{L_{\mathcal{B}}}|} \sum_{v_j \in \mathcal{V}_{L_{\mathcal{B}}}}\nabla_{w} l_w(\embh_j ,y_j)]\\
        &= \nabla_{w} \mathbb{E}[ l_w(\embh_j ,y_j)]\\
        &= \nabla_{w} \loss.
     \end{align*}
 As the subgraph $\inbatch$ is uniformly sampled from $\mathcal{V}$, the expectation of $\mathbf{g}_{\theta}(\inbatch)$ is
 \begin{align*}
     \mathbb{E}[\mathbf{g}_{\theta}(\inbatch)] &= \mathbb{E}[\frac{|\mathcal{V}|}{|\inbatch|} \sum_{v_j \in \inbatch} \nabla_{\theta} \update(\embh_j ,\embm_{\neighbor{v_j}}  ,\embx_j) ] \embV_{j} \\
     &= |\mathcal{V}| \mathbb{E}[ \nabla_{\theta} \update(\embh_j ,\embm_{\neighbor{v_j}}  ,\embx_j) ] \embV_{j} ]\\
     &=|\mathcal{V}| \frac{1}{|\mathcal{V}|} \sum_{v_j \in \mathcal{V}} \nabla_{\theta} \update(\embh_j ,\embm_{\neighbor{v_j}}  ,\embx_j) \embV_{j}\\
     &= \sum_{v_j \in \mathcal{V}} \nabla_{\theta} \update(\embh_j ,\embm_{\neighbor{v_j}}  ,\embx_j) \embV_{j}\\
     &= \nabla_{\theta} \loss.
 \end{align*}
\end{proof}


\subsection{Proof of Lemma 1 in the main text}
We first show that for a single layer ConvGNN, if the inputs of LMC and the exact forward pass are close to each other, then the approximation error is close to zero.

\begin{lemma}\label{prop:singlelayer}
    For a single layer ConvGNN, we denote the inputs we feed to the forward passing model of LMC and the exact forward passing model at the $k$-th iteration by $\hisH_k^{(0)}$ and $\embH_k^{(0)}$, respectively. Suppose that
    \begin{enumerate}
        \item the message passing function $f_{\theta^{(1)}}(\embH^{(0)})$ is $\gamma$-Lipschitz for parameter $\theta^{(1)}$ and $\embH^{(0)}$,

        \item the gradients $\widetilde{\embg}_{\theta^{(1)}}(\theta_k^{(1)})$ are bounded by $G$ for $k\in\mathbb{N}^*$,

        \item the differences are bounded by $\varepsilon_0>0$, i.e.,
        \begin{align*}
            &\|\hisH_k^{(0)}-\embH_k^{(0)}\|_F < \varepsilon_0,\,\,\forall\, k\in\mathbb{N}^*,\\
            & \|\embH_{k_1}^{(0)} - \embH_{k_2}^{(0)}\|_F < \varepsilon_0,\,\,\forall\, k_1,k_2\in\mathbb{N}^*,
        \end{align*}

        \item the differences between temporary embeddings and exact embeddings are dominated by those between historical embeddings and exact embeddings, i.e.,
        \begin{align*}
            \|\widehat{\embH}_{k}^{(0)} - \embH_{k}^{(0)}\|_F \leq C \|\overline{\embH}_{k}^{(0)} - \embH_{k}^{(0)}\|_F< C\varepsilon_0,\,\,\forall\,k\in\mathbb{N}^*,
        \end{align*}
    \end{enumerate}
    then for any $0<\delta<1$, by letting 
    \begin{align*}
        \eta \leq \frac{\varepsilon_0}{G(1-\log_{\frac{B}{B-1}} (\frac{1-\delta}{B-1}))},    
    \end{align*}
    we have
    \begin{align*}
        P\left(d_{h,k}^{(1)}>(C+3)\varepsilon_0\gamma\sqrt{B}\right)<\delta,\,\,\forall k\in\mathbb{N}^*,
    \end{align*}
    where $B$ is the number of subgraphs.
\end{lemma}

\begin{proof}
    Without loss of generality, we assume that the sampled subgraph at the $k$-th iteration is $\mathcal{V}_1$ and the most recent sampling to $\mathcal{V}_{j}$ is in the $\alpha_{k,j}$-th iteration, $j=1,\ldots, B$. Then we have
    \begin{align*}
        &[\hisH_k^{(1)}]_{\mathcal{V}_{1}} = [f_{\theta_k^{(1)}}(\temH_k^{(0)})]_{\mathcal{V}_{1}}, \,\, [\embH_k^{(1)}]_{\mathcal{V}_{1}}=[f_{\theta_k^{(1)}}(\embH_k^{(0)})]_{\mathcal{V}_{1}},\\
        &[\hisH_k^{(1)}]_{\mathcal{V}_{j}} = [f_{\theta_{\alpha_{k,j}}^{(1)}}(\temH_{\alpha_{k,j}}^{(0)})]_{\mathcal{V}_{j}}, \,\,[\embH_k^{(1)}]_{\mathcal{V}_{j}} = [f_{\theta_k^{(1)}}(\embH_k^{(0)})]_{\mathcal{V}_{j}},\\ 
        &j=2,\ldots, B.
    \end{align*}
    Hence we have
    \begin{align*}
        &\|\hisH_k^{(1)} - \embH_k^{(1)}\|_F^2\\
        ={}&\sum_{j=1}^B\|[\hisH_k^{(1)}]_{\mathcal{V}_{j}}-[\embH_k^{(1)}]_{\mathcal{V}_{j}}\|_F^2\\
        ={}& \|[f_{\theta_k^{(1)}}(\temH_k^{(0)})]_{\mathcal{V}_{1}} - [f_{\theta_k^{(1)}}(\embH_k^{(0)})]_{\mathcal{V}_{1}}\|_F^2\\
        &+\sum_{j=2}^B\| [f_{\theta_{\alpha_{k,j}}^{(1)}}(\temH_{\alpha_{k,j}}^{(0)})]_{\mathcal{V}_{j}} - [f_{\theta_k^{(1)}}(\embH_k^{(0)})]_{\mathcal{V}_{j}} \|_F^2\\
        \leq{}& \|f_{\theta_k^{(1)}}(\temH_k^{(0)}) - f_{\theta_k^{(1)}}(\embH_k^{(0)})\|_F^2\\
        &+\sum_{j=2}^B\| f_{\theta_{\alpha_{k,j}}^{(1)}}(\temH_{\alpha_{k,j}}^{(0)}) - f_{\theta_k^{(1)}}(\embH_k^{(0)}) \|_F^2\\
        \leq{}& \gamma^2C^2\varepsilon_0^2+\sum_{j=2}^B (\|f_{\theta_{\alpha_{k,j}}^{(1)}}(\temH_k^{(0)}) - f_{\theta_{\alpha_{k,j}}^{(1)}}(\embH_{\alpha_{k,j}}^{(0)})\|_F \\
        &\quad\quad\quad\quad\quad +\|f_{\theta_{\alpha_{k,j}}^{(1)}}(\embH_{\alpha_{k,j}}^{(0)}) - f_{\theta_{\alpha_{k,j}}^{(1)}}(\embH_k^{(0)})\|_F\\
        &\quad\quad\quad\quad\quad+\|f_{\theta_{\alpha_{k,j}}^{(1)}}(\embH_{k}^{(0)}) - f_{\theta_k^{(1)}}(\embH_k^{(0)})\|_F)^2\\
        \leq{}& \gamma^2C^2\varepsilon_0^2 + \gamma^2\sum_{j=2}^{B} (\|\temH_k^{(0)} - \embH_{\alpha_{k,j}}^{(0)}\|_F + \|\embH_{\alpha_{k,j}}^{(0)} - \embH_k^{(0)}\|_F\\
        &\quad\quad\quad\quad\quad\quad\quad +\|\theta_{\alpha_{k,j}}^{(1)} - \theta_k^{(1)}\|)^2\\
        \leq{}& \gamma^2C^2\varepsilon_0^2 + \gamma^2 \sum_{j=2}^B ((C+2)\varepsilon_0 + \|\theta_{\alpha_{k,j}}^{(1)} - \theta_k^{(1)}\|)^2\\
        \leq{}& \gamma^2 \sum_{j=1}^B ((C+2)\varepsilon_0+\|\theta_{\alpha_{k,j}}^{(1)} - \theta_k^{(1)}\|)^2
    \end{align*}
    Note that $\alpha_{k,1}=k$. Thus, for $\varepsilon=(C+3)\varepsilon_0\gamma\sqrt{B}$, we have
    \begin{align*}
        &P\left(\|\hisH_k^{(1)} - \embH_k^{(1)}\|_F^2>\varepsilon^2\right)\\
        ={}& 1-P\left(\|\hisH_k^{(1)} - \embH_k^{(1)}\|_F^2 \leq \varepsilon^2\right)\\
        \leq{}& 1-P\left(\sum_{j=1}^B ((C+2)\varepsilon_0+\|\theta_{\alpha_{k,j}}^{(1)} - \theta_k^{(1)}\|)^2 \leq (C+3)^2\varepsilon_0^2B\right).
    \end{align*}
    Next we show that
    \begin{align*}
        P\left(\sum_{j=1}^B ((C+2)\varepsilon_0+\|\theta_{\alpha_{k,j}}^{(1)} - \theta_k^{(1)}\|)^2 \leq (C+3)^2\varepsilon_0^2B\right) > \delta.
    \end{align*}
    Since the gradients $\widetilde{\embg}_{\theta^{(1)}}(\theta_k^{(1)})$ are bounded by $G>0$, then we have $\|\theta_{\alpha_{k,j}}^{(1)} - \theta_k^{(1)}\| \leq (k - \alpha_{k,j}) \eta G$, hence
    \begin{align*}
        &P\left(\sum_{j=1}^B ((C+2)\varepsilon_0+\|\theta_{\alpha_{k,j}}^{(1)} - \theta_k^{(1)}\|)^2 \leq (C+3)^2\varepsilon_0^2B\right)\\
        \geq{}& P\left(\sum_{j=1}^B((C+2)\varepsilon_0 + (k - \alpha_{k,j}) \eta G)^2 \leq (C+3)^2\varepsilon_0^2B\right)\\
        \geq{}& P\left(\bigcap_{j=1}^B \left\{((C+2)\varepsilon_0+(k - \alpha_{k,j})\eta G)^2 \leq (C+3)^2\varepsilon_0^2\right\}\right)\\
        ={}& P\left(\bigcap_{j=2}^B \left\{\alpha_{k,j} \geq k-\frac{\varepsilon_0}{\eta G} \right\}\right)\\
        ={}& 1 - P\left(\bigcup_{j=2}^B\left\{\alpha_{k,j} < k-\frac{\varepsilon_0}{\eta G} \right\}\right)\\
        \geq{}& 1-\sum_{j=2}^B P\left(\alpha_{k,j} < k-\frac{\varepsilon_0}{\eta G}\right).\\
    \end{align*}
    Since we sample a subgraph from $\{\mathcal{V}_j\}_{j=1}^B$ randomly at each iteration, $\alpha_{k,j} < k-\frac{\varepsilon_0}{\eta G}$ means that all the sampled subgraphs from the $\lceil k-\frac{\varepsilon_0}{\eta G} \rceil$-th to the $k$-th iterations are not $\mathcal{V}_j$. Thus, we have
    \begin{align*}
        &P\left(\alpha_{k,j} < k-\frac{\varepsilon_0}{\eta G}\right)\\
        ={}& (1-\frac{1}{B})^{\lfloor\frac{\varepsilon_0}{\eta G}\rfloor},\,\,\forall j=2,\ldots,L.
    \end{align*}
    Hence by letting
    \begin{align*}
        \eta \leq \frac{\varepsilon_0}{G(1-\log_{\frac{B}{B-1}} (\frac{1-\delta}{B-1}))}
    \end{align*}
    we have
    \begin{align*}
        &P\left(\sum_{j=1}^B ((C+2)\varepsilon_0+\|\theta_{\alpha_{k,j}}^{(1)} - \theta_k^{(1)}\|)^2 \leq (C+3)^2\varepsilon_0^2B\right)\\
        \geq{}& 1 - \sum_{j=2}^B (1-\frac{1}{B})^{\lfloor\frac{\varepsilon_0}{\eta G}\rfloor}\\
        ={}& 1-(B-1)(1-\frac{1}{B})^{\lfloor\frac{\varepsilon_0}{\eta G}\rfloor}\\
        >{}& \delta,
    \end{align*}
    which leads to
    \begin{align*}
        P\left(d_{h,k}^{(1)}>(C+3)\varepsilon_0\gamma\sqrt{B}\right) < \delta,\,\,\forall\,k\in\mathbb{N}^*.
    \end{align*}
\end{proof}

By Lemma \ref{prop:singlelayer} we show that for a single layer ConvGNN, if the inputs of the forward passing model of LMC and those of the exact forward passing model are the same, the approximation error of node embeddings $d_{h,k}^{(1)}$ converges to zero in probability since $\varepsilon_0=0$ in this case.

% \begin{lemma}
%     Given an $L$-layer ConvGNN, suppose that for $l=1,\ldots, L$ we have
%     \begin{enumerate}
%         \item the message passing function $f_{\theta^{(l)}}(\embH^{(l-1)})$ is $\gamma$-Lipschitz for parameter $\theta^{(l)}$ and $\embH^{(l-1)}$,

%         \item the gradients $\widetilde{\embg}_{\theta^{(l)}}(\theta_k^{(l)})$ are bounded by $G$ for $k\in\mathbb{N}^*$,

%         \item the learning rate $\eta_k$ is monotonically non-increasing and can be arbitrarily small when $k$ is large enough,
%     \end{enumerate}
%     then we have
%     \begin{align*}
%         d_{h,k}^{(l)} \stackrel{P}{\longrightarrow} 0\,\,{\rm as}\,\, k\to\infty,
%     \end{align*}
%     i.e., for any $\varepsilon > 0, 0<\delta<1$, there exists $K^{(l)}\in\mathbb{N}^*$ such that
%     \begin{align*}
%         P(d_{h,k}^{(l)} \geq \varepsilon) < \delta,\,\,\forall k>K^{(l)}.
%     \end{align*}
% \end{lemma}

\subsubsection{Proof of Lemma 1 in the main text}

\begin{proof}
    For any $\varepsilon>0$ and $\delta>0$, since the inputs we feed to the forward passing model of LMC and the exact forward passing model are both the node features $\embX$, the distance between the inputs are $d_{h,k}^{(0)}=0$. Thus, we have
    \begin{align*}
        P\left( d_{h,k}^{(0)} > \frac{\varepsilon}{((C+3)\gamma\sqrt{B})^L} \right) = 0 < \delta.
    \end{align*}
    By Lemma \ref{prop:singlelayer} we know that by letting
    \begin{align*}
        \eta \leq \frac{\frac{\varepsilon}{((C+3)\gamma\sqrt{B})^L}}{G(1-\log_{\frac{B}{B-1}} (\frac{1-\delta}{B-1}))}
    \end{align*}
    we have
    \begin{align*}
        P\left( d_{h,k}^{(1)} > \frac{\varepsilon}{((C+3)\gamma\sqrt{B})^{L-1}} \right)< \delta.
    \end{align*}
    By Lemma \ref{prop:singlelayer} again, letting
    \begin{align*}
        \eta \leq \min\{\frac{\frac{\varepsilon}{((C+3)\gamma\sqrt{B})^{L-1}}}{G(1-\log_{\frac{B}{B-1}} (\frac{1-\delta}{B-1}))},\frac{\frac{\varepsilon}{((C+3)\gamma\sqrt{B})^L}}{G(1-\log_{\frac{B}{B-1}} (\frac{1-\delta}{B-1}))}\}
    \end{align*}
    leads to
    \begin{align*}
        P\left( d_{h,k}^{(2)} > \frac{\varepsilon}{((C+3)\gamma\sqrt{B})^{L-2}} \right)< \delta.
    \end{align*}
    And so on, we know that by letting
    \begin{align*}
        \eta \leq \min_{l\in\{1,L\}} \frac{\frac{\varepsilon}{((C+3)\gamma\sqrt{B})^{l}}}{G(1-\log_{\frac{B}{B-1}} (\frac{1-\delta}{B-1}))}
    \end{align*}
    we have
    \begin{align*}
        P\left( d_{h,k}^{(L)} > \varepsilon \right)< \delta.
    \end{align*}
\end{proof}

\subsection{Proof of Lemma 2 in the main text}
We first show that for a two-layer ConvGNN, if the inputs of the backward passing model of LMC and those of the exact backward passing model are close to each other, then the approximation error is close to zero.

\begin{lemma}\label{prop:singlelayer_v}
    For a two-layer ConvGNN, we denote the inputs we feed to the backward passing model of LMC and the exact backward passing model by $\hisV_k^{(2)}$ and $\embV_k^{(2)}$, respectively. Suppose that
    \begin{enumerate}
        \item the mapping $\phi_{\theta^{(2)}}(\embV^{(2)})$ is $\gamma$-Lipschitz for parameter $\theta^{(2)}$ and $\embV^{(2)}$,

        \item the gradients $\widetilde{\embg}_{\theta^{(2)}}(\theta_k^{(2)})$ are bounded by $G$ for $k\in\mathbb{N}^*$,

        \item the differences are bounded by $\varepsilon_0>0$, i.e.,
        \begin{align*}
            &\|\hisV_k^{(2)}-\embV_k^{(2)}\|_F < \varepsilon_0,\,\,\forall\, k\in\mathbb{N}^*,\\
            &\|\embV_{k_1}^{(2)} - \embV_{k_2}^{(2)}\| < \varepsilon_0,\,\,\forall\, k_1,k_2\in\mathbb{N}^*,
        \end{align*}

        \item the differences between temporary embeddings and exact embeddings are dominated by those between historical embeddings and exact embeddings, i.e.,
        \begin{align*}
            \|\temV_k^{(2)} - \embV_k^{(2)}\|_F \leq C\|\hisV_k^{(2)} - \embV_k^{(2)}\|_F < C\varepsilon_0,\,\,\forall\, k\in\mathbb{N}^*,
        \end{align*}
    \end{enumerate}
    then for any $0<\delta<1$, by letting
    \begin{align*}
        \eta \leq \frac{\varepsilon_0}{G(1-\log_{\frac{B}{B-1}} (\frac{1-\delta}{B-1}))},
    \end{align*}
    we have
    \begin{align*}
        P\left(d_{v,k}^{(1)} \geq (C+3)\varepsilon_0\gamma\sqrt{B} \right)<\delta,\,\,\forall k>K,
    \end{align*}
    where $B$ is the number of subgraphs.
\end{lemma}

\begin{proof}
    Without loss of generality, we assume that the sampled subgraph at the $k$-th iteration is $\mathcal{V}_1$ and the most recent sampling to $\mathcal{V}_{j}$ is in the $\alpha_{k,j}$-th iteration, $j=1,\ldots, B$. Then we have
    \begin{align*}
        &[\hisV_k^{(1)}]_{\mathcal{V}_{1}} = [\phi_{\theta_k^{(2)}}(\temV_k^{(2)})]_{\mathcal{V}_{1}}, \,\, [\embV_k^{(1)}]_{\mathcal{V}_{1}}=[\phi_{\theta_k^{(2)}}(\embV_k^{(2)})]_{\mathcal{V}_{1}},\\
        &[\hisV_k^{(1)}]_{\mathcal{V}_{j}} = [\phi_{\theta_{\alpha_{k,j}}^{(2)}}(\temV_{\alpha_{k,j}}^{(2)})]_{\mathcal{V}_{j}}, \,\,[\embV_k^{(1)}]_{\mathcal{V}_{j}} = [\phi_{\theta_k^{(2)}}(\embV_k^{(2)})]_{\mathcal{V}_{j}},\\ 
        &j=2,\ldots, B.
    \end{align*}
    Hence we have
    \begin{align*}
        &\|\hisV_k^{(1)} - \embV_k^{(1)}\|_F^2\\
        ={}&\sum_{j=1}^B\|[\hisV_k^{(1)}]_{\mathcal{V}_{j}}-[\embV_k^{(1)}]_{\mathcal{V}_{j}}\|_F^2\\
        ={}& \|[\phi_{\theta_k^{(2)}}(\temV_k^{(2)})]_{\mathcal{V}_{1}} - [\phi_{\theta_k^{(2)}}(\embV_k^{(2)})]_{\mathcal{V}_{1}}\|_F^2\\
        &+\sum_{j=2}^B\| [\phi_{\theta_{\alpha_{k,j}}^{(2)}}(\temV_{\alpha_{k,j}}^{(2)})]_{\mathcal{V}_{j}} - [\phi_{\theta_k^{(2)}}(\embV_k^{(2)})]_{\mathcal{V}_{j}} \|_F^2\\
        \leq{}& \|\phi_{\theta_k^{(2)}}(\temV_k^{(2)}) - \phi_{\theta_k^{(2)}}(\embV_k^{(2)})\|_F^2\\
        &+\sum_{j=2}^B\| \phi_{\theta_{\alpha_{k,j}}^{(2)}}(\temV_{\alpha_{k,j}}^{(2)}) - \phi_{\theta_k^{(2)}}(\embV_k^{(2)}) \|_F^2\\
        \leq{}& \gamma^2C^2\varepsilon_0^2+\sum_{j=2}^B (\|\phi_{\theta_{\alpha_{k,j}}^{(2)}}(\temV_k^{(2)}) - \phi_{\theta_{\alpha_{k,j}}^{(2)}}(\embV_{\alpha_{k,j}}^{(2)})\|_F \\
        &\quad\quad\quad\quad\quad +\|\phi_{\theta_{\alpha_{k,j}}^{(2)}}(\embV_{\alpha_{k,j}}^{(2)}) - \phi_{\theta_{\alpha_{k,j}}^{(2)}}(\embV_k^{(2)})\|_F\\
        &\quad\quad\quad\quad\quad+\|\phi_{\theta_{\alpha_{k,j}}^{(2)}}(\embV_{k}^{(2)}) - \phi_{\theta_k^{(2)}}(\embV_k^{(2)})\|_F)^2\\
        \leq{}& \gamma^2C^2\varepsilon_0^2 + \gamma^2\sum_{j=2}^{B} (\|\temV_k^{(2)} - \embV_{\alpha_{k,j}}^{(2)}\|_F + \|\embV_{\alpha_{k,j}}^{(2)} - \embV_k^{(2)}\|_F\\
        &\quad\quad\quad\quad\quad\quad\quad +\|\theta_{\alpha_{k,j}}^{(2)} - \theta_k^{(2)}\|)^2\\
        \leq{}& \gamma^2C^2\varepsilon_0^2 + \gamma^2 \sum_{j=2}^B ((C+2)\varepsilon_0 + \|\theta_{\alpha_{k,j}}^{(2)} - \theta_k^{(2)}\|)^2\\
        \leq{}& \gamma^2 \sum_{j=1}^B ((C+2)\varepsilon_0+\|\theta_{\alpha_{k,j}}^{(2)} - \theta_k^{(2)}\|)^2
    \end{align*}
    Note that $\alpha_{k,1}=k$. Thus, for $\varepsilon = (C+3)\varepsilon_0\gamma\sqrt{B}$, we have
    \begin{align*}
        &P\left(\|\hisV_k^{(1)} - \embV_k^{(1)}\|_F^2>\varepsilon^2\right)\\
        ={}& 1-P\left(\|\hisV_k^{(1)} - \embV_k^{(1)}\|_F^2 \leq \varepsilon^2\right)\\
        \leq{}& 1-P\left(\sum_{j=1}^B ((C+2)\varepsilon_0+\|\theta_{\alpha_{k,j}}^{(2)} - \theta_k^{(2)}\|)^2 \leq (C+3)^2\varepsilon_0^2 B \right).
    \end{align*}
    Next we show that
    \begin{align*}
        P\left(\sum_{j=1}^B ((C+2)\varepsilon_0+\|\theta_{\alpha_{k,j}}^{(2)} - \theta_k^{(2)}\|)^2 \leq (C+3)^2\varepsilon_0^2 B\right) > 1 - \delta.
    \end{align*}
    Since the gradients $\widetilde{\embg}_{\theta^{(2)}}(\theta_k^{(2)})$ are bounded by $G>0$, then we have $\|\theta_{\alpha_{k,j}}^{(2)} - \theta_k^{(2)}\| \leq (k - \alpha_{k,j}) \eta G$, hence
    \begin{align*}
        &P\left(\sum_{j=1}^B ((C+2)\varepsilon_0+\|\theta_{\alpha_{k,j}}^{(2)} - \theta_k^{(2)}\|)^2 \leq (C+3)^2\varepsilon_0^2 B\right)\\
        \geq{}& P\left(\sum_{j=1}^B((C+2)\varepsilon_0 + (k - \alpha_{k,j}) \eta G)^2 \leq (C+3)^2\varepsilon_0^2 B\right)\\
        \geq{}& P\left(\bigcap_{j=1}^B \left\{((C+2)\varepsilon_0+(k - \alpha_{k,j})\eta G)^2 \leq (C+3)^2\varepsilon_0^2\right\}\right)\\
        ={}& P\left(\bigcap_{j=2}^B \left\{\alpha_{k,j} \geq k-\frac{\varepsilon_0}{\eta G} \right\}\right)\\
        ={}& 1 - P\left(\bigcup_{j=2}^B\left\{\alpha_{k,j} < k-\frac{\varepsilon_0}{\eta G} \right\}\right)\\
        \geq{}& 1-\sum_{j=2}^B P\left(\alpha_{k,j} < k-\frac{\varepsilon_0}{\eta G}\right).\\
    \end{align*}
    Since we sample a subgraph from $\{\mathcal{V}_j\}_{j=1}^B$ randomly at each iteration, $\alpha_{k,j} < k-\frac{\varepsilon_0}{\eta G}$ means that all the sampled subgraphs from the $\lceil k-\frac{\varepsilon_0}{\eta G} \rceil$-th to the $k$-th iterations are not $\mathcal{V}_j$. Thus, we have
    \begin{align*}
        &P\left(\alpha_{k,j} < k-\frac{\varepsilon_0}{\eta G}\right)\\
        ={}& (1-\frac{1}{B})^{\lfloor\frac{\varepsilon_0}{\eta G}\rfloor},\,\,\forall j=2,\ldots,L.
    \end{align*}
    Hence by letting
    \begin{align*}
        \eta \leq \frac{\varepsilon_0}{G(1-\log_{\frac{B}{B-1}} (\frac{1-\delta}{B-1}))}
    \end{align*}
    we have
    \begin{align*}
        &P\left(\sum_{j=1}^B ((C+2)\varepsilon_0+\|\theta_{\alpha_{k,j}}^{(2)} - \theta_k^{(2)}\|)^2 \leq (C+3)^2\varepsilon_0^2B\right)\\
        \geq{}& 1 - \sum_{j=2}^B (1-\frac{1}{B})^{\lfloor\frac{\varepsilon_0}{\eta G}\rfloor}\\
        ={}& 1-(B-1)(1-\frac{1}{B})^{\lfloor\frac{\varepsilon_0}{\eta G}\rfloor}\\
        >{}& \delta,
    \end{align*}
    which leads to
    \begin{align*}
        P\left(d_{v,k}^{(1)}\geq (C+3) \varepsilon_0\gamma\sqrt{B}\right)<\delta.
    \end{align*}
\end{proof}

By Lemma \ref{prop:singlelayer_v} we show that for a two-layer ConvGNN, if the inputs of the backward passing model of LMC and those of the exact backward passing model are the same, the approximation error of auxiliary variables $d_{v,k}^{(1)}$ converges to zero in probability since $\varepsilon_0=0$ in this case.

\subsubsection{Proof of Lemma 2 in the main text}

\begin{proof}
    For any $\varepsilon>0$ and $\delta>0$, since the inputs we feed to the backward passing model of LMC and the exact backward passing model are both $\embV^{(L)}=\frac{\partial \loss}{\partial \embH}$, the distance between the inputs are $d_{v,k}^{(L)}=0$. Thus, we have
    \begin{align*}
        P\left( d_{v,k}^{(L)} > \frac{\varepsilon}{((C+3)\gamma\sqrt{B})^{L-1}} \right) = 0 < \delta.
    \end{align*}
    By Lemma \ref{prop:singlelayer} we know that by letting
    \begin{align*}
        \eta \leq \frac{\frac{\varepsilon}{((C+3)\gamma\sqrt{B})^{L-1}}}{G(1-\log_{\frac{B}{B-1}} (\frac{1-\delta}{B-1}))}
    \end{align*}
    we have
    \begin{align*}
        P\left( d_{v,k}^{(L-1)} > \frac{\varepsilon}{((C+3)\gamma\sqrt{B})^{L-2}} \right)< \delta.
    \end{align*}
    By Lemma \ref{prop:singlelayer} again, letting
    \begin{align*}
        \eta \leq \min\{\frac{\frac{\varepsilon}{((C+3)\gamma\sqrt{B})^{L-2}}}{G(1-\log_{\frac{B}{B-1}} (\frac{1-\delta}{B-1}))},\frac{\frac{\varepsilon}{((C+3)\gamma\sqrt{B})^{L-1}}}{G(1-\log_{\frac{B}{B-1}} (\frac{1-\delta}{B-1}))}\}
    \end{align*}
    leads to
    \begin{align*}
        P\left( d_{h,k}^{(2)} > \frac{\varepsilon}{((C+3)\gamma\sqrt{B})^{L-2}} \right)< \delta.
    \end{align*}
    And so on, we know that by letting
    \begin{align*}
        \eta \leq \min_{l\in\{1,L-1\}} \frac{\frac{\varepsilon}{((C+3)\gamma\sqrt{B})^{l}}}{G(1-\log_{\frac{B}{B-1}} (\frac{1-\delta}{B-1}))}
    \end{align*}
    we have
    \begin{align*}
        P\left( d_{v,k}^{(1)} > \varepsilon \right)< \delta.
    \end{align*}
\end{proof}

\subsection{Proof of Lemmas 3 and 4}


\setcounter{lemma}{2}
We first introduce some useful lemmas.

\begin{lemma}\label{lemma:cfpi}
    Let the transition function $f:\mathbb{R}^d \rightarrow \mathbb{R}^d$ be $\gamma$-contraction. Consider the block-coordinate fixed-point algorithm with the partition of coordinates to $n$ blocks $\mathbf{I} = (S_1,S_2,\dots,S_n) \in \mathbb{R}^{d \times d}$ and an update rule
    \begin{align*}
        \vecx_{k+1} = (\mathbf{I}- \alpha \vecM^{(k)}) \vecx_k + \alpha \vecM^{(k)} f(\vecx_k)),
    \end{align*}
    where $ \alpha \in (0,1]$ and the stochastic matrix $\vecM^{(k)}$ chosen independently and uniformly from $\{(0,\dots,S_j,\dots,0)|j=1,\dots n\}$ indicates the updated coordinates at the $k$-th iteration. Then, we have
    \begin{align*}
        \mathbb{E}[\| \vecx_{k+1} - \vecx   \|_2^2] \leq \left( 1 - \frac{\alpha}{n}(1-\gamma^2) \right)\| \vecx_k - \vecx   \|_2^2.
    \end{align*}
    Moreover,
    \begin{align*}
        \mathbb{E}[\| \vecx_{k+1} - \vecx   \|_2^2] \leq \left( 1 - \frac{\alpha}{n}(1-\gamma^2) \right)^{k}\| \vecx_{1} - \vecx   \|_2^2.
    \end{align*}
\end{lemma}




% lemma:cfpi
\begin{proof}
    As $\vecM^{(k)}$ is chosen uniformly from the set $\{(0,\dots,S_i,\dots,0)|i=1,\dots n\}$, we have
    \begin{align*}
        \mathbb{E}[\vecM^{(k)}] = \frac{\mathbf{I}}{n}.
    \end{align*}
    We first compute the conditional expectation
    \begin{align*}
        &\mathbb{E}[ \| \vecx_{k+1} - \vecx   \|_2^2 | \vecx_k]\\ 
        ={}& \mathbb{E}[ \| \vecx_{k+1} - \vecx_k + \vecx_k - \vecx   \|_2^2 | \vecx_k] \\
        ={}& \mathbb{E}[ \| \vecx_{k+1} - \vecx_k \|_2^2 | \vecx_k] + \| \vecx_k - \vecx   \|_2^2 \\
        &+ 2 \langle \mathbb{E}[\vecx_{k+1}| \vecx_k]  - \vecx_k ,  \vecx_k - \vecx   \rangle \\
        ={}& \| \vecx_k - \vecx   \|_2^2 + \alpha^2 \mathbb{E}[ \| \vecM^{(k)}(\vecx_k - f(\vecx_k)) \|_2^2 | \vecx_k] \\
        &+ 2 \frac{\alpha}{n} \langle f(\vecx_k)  - \vecx_k ,  \vecx_k - \vecx   \rangle.
    \end{align*}
    Notice that 
    \begin{align*}
        &\mathbb{E}[ \| \vecM^{(k)}(\vecx_k - f(\vecx_k)) \|_2^2 | \vecx_k]\\
        ={}& \frac{1}{n} \sum_{j=1}^n \| (0,\dots,S_i,\dots,0)( \vecx_k - f(\vecx_k)) \|_2^2\\
        ={}&\frac{1}{n} \| \vecx_k - f(\vecx_k)\|_2^2
    \end{align*}
    and
    \begin{align*}
        &2 \langle f(\vecx_k)  - \vecx_k ,  \vecx_k - \vecx   \rangle\\
        ={}&\|f(\vecx_k) - \vecx   \|_2^2 - \|f(\vecx_k)  - \vecx_k\|_2^2 - \|\vecx_k - \vecx  \|_2^2.
    \end{align*}
    Combining the above equalities and the contraction property of $f$, we have
    \begin{align*}
        &\mathbb{E}[ \| \vecx_{k+1} - \vecx   \|_2^2 | \vecx_k]\\
        ={}& (1-\frac{\alpha}{n})\| \vecx_k - \vecx   \|_2^2 - \alpha \frac{1-\alpha}{n} \| \vecx_k - f(\vecx_k)\|_2^2 \\
        &+\frac{\alpha}{n} (\|f(\vecx_k) - \vecx   \|_2^2) \\
        \leq{}& (1-\frac{\alpha}{n})\| \vecx_k - \vecx   \|_2^2 - \alpha \frac{1-\alpha}{n} \| \vecx_k - f(\vecx_k)\|_2^2\\ 
        &+\frac{\alpha \gamma^2}{n} (\|\vecx_k - \vecx   \|_2^2) \\
        \leq{}& (1-\frac{\alpha}{n}(1-\gamma^2)) \| \vecx_k - f(\vecx_k)\|_2^2.
    \end{align*}
    By the law of total expectation, we have
    \begin{align*}
        \mathbb{E}[ \| \vecx_{k+1} - \vecx   \|_2^2] \leq (1-\frac{\alpha}{n}(1-\gamma^2) \mathbb{E}[ \| \vecx_k - \vecx   \|_2^2 ].
    \end{align*}
    Finally, we recursively deduce that
    \begin{align*}
        \mathbb{E}[\| \vecx_{k+1} - \vecx   \|_2^2] \leq \left( 1 - \frac{\alpha}{n}(1-\gamma^2) \right)^{k} \| \vecx_{1} - \vecx   \|_2^2.
    \end{align*}
\end{proof}


By Lemma \ref{lemma:cfpi}, we can deduce that the expectation of the approximate errors for the representations $d^{(k)}_h = \sqrt{ \mathbb{E}[\|\overline{\embH}^{(k)} - [\embH ]^{(k)}\|_F^2]}$ and the auxiliary variables $d^{(k)}_v = \sqrt{ \mathbb{E}[\|\overline{\embV}^{(k)} - \embV ^{(k)}\|_F^2]}$ decrease if the parameters $\theta,w$ change slow. Moreover, we can set the learning rate $\eta$ to be a such small value that the parameters $\theta,w$ change slowly.


\begin{lemma}\label{lemma:amp_cfpi}
    Suppose that the message passing function $f_{\theta}$ is $L$-Lipschitz for parameter $\theta$ and $\gamma$-contraction for $\embH$. Let $d^{(k)}_h = \sqrt{ \mathbb{E}[\|\overline{\embH}^{(k)} - [\embH ]^{(k)}\|_F^2]}$. If $\|\theta^{(k+1)}-\theta^{(k)}\|_F \leq \varepsilon$ at the $k$-th iteration, then we have
    \begin{align*}
        d^{(k+1)}_{h} \leq \rho d^{(k)}_h + K \varepsilon,
    \end{align*}
    where $\rho = \sqrt{(1-\frac{\alpha}{I}(1-\gamma^2))}$ and $K = \frac{L}{1-\gamma}$.
\end{lemma}



% lemma:amp_cfpi
\begin{proof}
    According to Lemma \ref{lemma:cfpi}, we have
    \begin{align*}
        &\mathbb{E}[\|\overline{\embH}^{(k+1)} - [\embH ]^{(k+1)}\|_F^2]\\
        \leq{}& \rho^2\mathbb{E}[\|\overline{\embH}^{(k)} - [\embH ]^{(k+1)}\|_F^2] \\
        \leq{}& \rho^2\mathbb{E}[(\|\overline{\embH}^{(k)} - [\embH ]^{(k)}\|_F + \|[\embH ]^{(k)} - [\embH ]^{(k+1)}\|_F )^2].
    \end{align*}
    As the transition function $f_{\theta}$ is  $L$-Lipschitz, we have
    \begin{align*}
        &\|[\embH ]^{(k)} - [\embH ]^{(k+1)}\|_F \\
        ={}& \|f_{\theta^{(k)}}([\embH ]^{(k)}) - f_{\theta^{(k+1)}}[\embH ]^{(k+1)}\|_F \\
        \leq{}& \|f_{\theta^{(k)}}([\embH ]^{(k)}) - f_{\theta^{(k)}}[\embH ]^{(k+1)}\|_F\\
        &+\|f_{\theta^{(k)}}([\embH ]^{(k+1)}) - f_{\theta^{(k+1)}}[\embH ]^{(k+1)}\|_F\\
        \leq{}& \gamma \|[\embH ]^{(k)} - [\embH ]^{(k+1)}\|_F + L\|\theta^{(k+1)}-\theta^{(k)}\|_F.
    \end{align*}
    By rearranging the terms, we have
    \begin{align*}
        \|[\embH ]^{(k)} - [\embH ]^{(k+1)}\|_F \leq \frac{L}{1-\gamma} \|\theta^{(k+1)}-\theta^{(k)}\|_F\leq K \varepsilon.
    \end{align*}
    Combining the above inequalities, we have
    \begin{align*}
        &\mathbb{E}[\|\overline{\embH}^{(k+1)} - [\embH ]^{(k+1)}\|_F^2]\\
        \leq{}& \rho^2 \mathbb{E}[(\|\overline{\embH}^{(k)} - [\embH ]^{(k)}\|_F + K \varepsilon)^2]\\
        ={}& \rho^2 (\mathbb{E}[\|\overline{\embH}^{(k)} - [\embH ]^{(k)}\|_F^2]\\
        &+ 2 \mathbb{E}[\|\overline{\embH}^{(k)} - [\embH ]^{(k)}\|_F] K \varepsilon  + (K \varepsilon)^2)\\
        \leq{}& \rho^2 (\mathbb{E}[\|\overline{\embH}^{(k)} - [\embH ]^{(k)}\|_F^2]\\
        &+ 2 \sqrt{\mathbb{E}[\|\overline{\embH}^{(k)} - [\embH ]^{(k)}\|_F^2} K \varepsilon  + (K \varepsilon)^2)\\
        ={}& \rho^2 (\sqrt{\mathbb{E}[\|\overline{\embH}^{(k)} - [\embH ]^{(k)}\|_F^2} + K\varepsilon)^2.
    \end{align*}
    The claims follows immediately.
\end{proof}




\begin{lemma}\label{lemma:amp_cfpi_b}
    Suppose that the mapping $\phi_{\theta,w}(\embV) = \nabla_{\vec{\embH}} f_{\theta} \embV + \nabla_{\vec{\embH}} \loss$ is $L$-Lipschitz for parameters $\theta,w$ and $\gamma$-contraction for $\embV$. Let $d^{(k)}_v = \sqrt{ \mathbb{E}[\|\overline{\embV}^{(k)} - \embV ^{(k)}\|_F^2]}$. If $\|\theta^{(k+1)}-\theta^{(k)}\|_F \leq \varepsilon$ and $\|w^{(k+1)}-w^{(k)}\|_F \leq \varepsilon$ at the $k$-th iteration, then we have
    \begin{align*}
        d^{(k+1)}_{v} \leq \rho d^{(k)}_v + K \varepsilon,
    \end{align*}
    where $\rho = \sqrt{(1-\frac{\alpha}{I}(1-\gamma^2))}$ and $K = \frac{2L}{1-\gamma}$.
\end{lemma}


% lemma:amp_cfpi_b
\begin{proof}
    According to Lemma \ref{lemma:cfpi}, we have
    \begin{align*}
        &\mathbb{E}[\|\overline{\embV}^{(k+1)} - [\embV ]^{(k+1)}\|_F^2]\\
        \leq{}& \rho^2\mathbb{E}[\|\overline{\embV}^{(k)} - [\embV ]^{(k+1)}\|_F^2] \\
        \leq{}& \rho^2\mathbb{E}[(\|\overline{\embV}^{(k)} - \embV ^{(k)}\|_F + \|\embV ^{(k)} - [\embV ]^{(k+1)}\|_F )^2].
    \end{align*}
    As the transition function $f_{\theta}$ is  $L$-Lipschitz, we have
    \begin{align*}
        &\|\embV ^{(k)} - [\embV ]^{(k+1)}\|_F\\
        ={}& \|\phi_{\theta^{(k)},w^{(k)}}(\embV ^{(k)}) - \phi_{\theta^{(k+1)},w^{(k+1)}}[\embV ]^{(k+1)}\|_F \\
        \leq{}& \|\phi_{\theta^{(k)},w^{(k)}}(\embV ^{(k)}) - \phi_{\theta^{(k)},w^{(k)}}[\embV ]^{(k+1)}\|_F \\
        &+ \|\phi_{\theta^{(k)},w^{(k)}}([\embV ]^{(k+1)}) - \phi_{\theta^{(k+1)},w^{(k+1)}}[\embV ]^{(k+1)}\|_F\\
        \leq{}& \gamma \|\embV ^{(k)} - [\embV ]^{(k+1)}\|_F\\
        &+ L\sqrt{\|\theta^{(k+1)}-\theta^{(k)}\|_F^2+ \|w-w^{(k)}\|_F^2}\\
        \leq{}& \gamma \|\embV ^{(k)} - [\embV ]^{(k+1)}\|_F\\
        &+ L(\|\theta^{(k+1)}-\theta^{(k)}\|_F+ \|w-w^{(k)}\|_F).
    \end{align*}
    By rearranging the terms, we have
    \begin{align*}
        &\|\embV ^{(k)} - [\embV ]^{(k+1)}\|_F \\
        \leq{}& \frac{L}{1-\gamma} (\|\theta^{(k+1)}-\theta^{(k)}\|_F+ \|w-w^{(k)}\|_F)\\
        \leq{}& K \varepsilon.
    \end{align*}
    Combining the above inequalities, we have
    \begin{align*}
        &\mathbb{E}[\|\overline{\embV}^{(k+1)} - [\embV ]^{(k+1)}\|_F^2]\\
        \leq{}& \rho^2 \mathbb{E}[(\|\overline{\embV}^{(k)} - \embV ^{(k)}\|_F + K \varepsilon)^2]\\
        ={}& \rho^2 (\mathbb{E}[\|\overline{\embV}^{(k)} - \embV ^{(k)}\|_F^2]\\
        &+ 2 \mathbb{E}[\|\overline{\embV}^{(k)} - \embV ^{(k)}\|_F] K \varepsilon + (K \varepsilon)^2)\\
        \leq{} & \rho^2 (\mathbb{E}[\|\overline{\embV}^{(k)} - \embV ^{(k)}\|_F^2]\\
        &+ 2 \sqrt{\mathbb{E}[\|\overline{\embV}^{(k)} - \embV ^{(k)}\|_F^2} K \varepsilon + (K \varepsilon)^2)\\
        ={}& \rho^2 (\sqrt{\mathbb{E}[\|\overline{\embV}^{(k)} - \embV ^{(k)}\|_F^2} + K\varepsilon)^2.
    \end{align*}
    The claims follows immediately.
\end{proof}




\subsubsection{Proof of Lemma 1}
% prop:convergence
\begin{proof}
    The update rule in LMC implies that
    \begin{align*}
        \|\theta^{(k+1)} - \theta^{(k)}\|_F &=  \eta \|\widetilde{\mathbf{g}}_{\theta}(\theta^{(k)})\|_F \leq \eta B.
    \end{align*}
    According to Lemma \ref{lemma:amp_cfpi}, we have
    \begin{align*}
        d_h^{(k+1)} - \frac{KB}{1-\rho} \eta &\leq \rho( d_h^{(k)} - \frac{KB}{1-\rho} \eta)\\
        &\leq \cdots\\
        &\leq  \rho^k (  d_h^{(1)} - \frac{KB}{1-\rho} \eta)\\
        &\leq \rho^k B
    \end{align*}
    The claim follows immediately.
\end{proof}



\subsubsection{Proof of Lemma 2}
% prop:convergence_b
\begin{proof}
    The update rule in LMC implies that
    \begin{align*}
        \|\theta^{(k+1)} - \theta^{(k)}\|_F &=  \eta \|\widetilde{\mathbf{g}}_{\theta}(\theta^{(k)})\|_F \leq \eta B
    \end{align*}
    and
    \begin{align*}
        \|w^{(k+1)} - w^{(k)}\|_F &=  \eta \|\widetilde{\mathbf{g}}_{w}(w^{(k)})\|_F \leq \eta B.
    \end{align*}
    According to Lemma \ref{lemma:amp_cfpi_b}, we have
    \begin{align*}
        d_v^{(k+1)} - \frac{KB}{1-\rho} \eta &\leq \rho( d_v^{(k)} - \frac{KB}{1-\rho} \eta)\\
        &\leq \cdots\\
        &\leq \rho^k (  d_v^{(1)} - \frac{KB}{1-\rho} \eta)\\
        &\leq \rho^k B
    \end{align*}
    The claim follows immediately.
\end{proof}


\subsection{Proof of Theorems 3 and 4}

\subsubsection{Sufficient conditions for convergence}

We first give sufficient conditions for convergence.
\begin{lemma}\label{lemma:suff_conv}
    Suppose that function $f:\mathbb{R}^{n} \to \mathbb{R}$ is continuously differentiable. Consider an optimization algorithm with any bounded initialization $\vecx_1$ and an update rule in the form of
    \begin{align*}
        \vecx_{k+1} = \vecx_{k} - \eta \vecd(\vecx_k),
    \end{align*}
    where $\eta>0$ is the learning rate and $\vecd(\vecx_k)$ is the estimated gradient that can be seen as a stochastic vector depending on $\vecx_k$. Let the estimation error of the gradient be $\Delta_{k} = \vecd(\vecx_k) - \nabla f(\vecx_k) $. Suppose that
    \begin{enumerate}
        \item the optimal value $f^*  = \inf_{\vecx} f(\vecx)$ is bounded; \label{con:1}
        
        \item the gradient of $f$ is $\gamma$-Lipschitz, i.e., \label{con:2}
        \begin{align*}
            \|\nabla f(\vecy) - \nabla f(\vecx)\|_2 \leq \gamma\|\vecy - \vecx\|_2,\,\forall\,\vecx,\vecy \in \mathbb{R}^{n};
        \end{align*}
        
        \item there exists $G_0>0$ that does not depend on $\eta$ such that
        \label{con:3}
        \begin{align*}
            \mathbb{E}[\|\Delta_{k}\|_2^2] \leq G_0,\,\forall\, k\in\mathbb{N}^*;
        \end{align*}

        \item there exists $N\in\mathbb{N}^*$ such that
        \begin{align*}
            |\mathbb{E}[\langle \nabla f(\vecx_k),\Delta_{k} \rangle]|<\frac{1}{\sqrt{N}},\,\,\forall\,k\in\mathbb{N}^*;
        \end{align*}
    \end{enumerate}
    then by letting $\eta=\min\{\frac{1}{\gamma},\frac{1}{\sqrt{N}}\}$, we have
    \begin{align*}
        \mathbb{E}[ \|\nabla f(\vecx_{R})\|_2^2] \leq \frac{2(f(\vecx_1) - f^*)+\gamma G_0 + 1}{\sqrt{N}}=O(\frac{1}{\sqrt{N}}),
    \end{align*}
     where $R$ is chosen uniformly from $[N]$.
\end{lemma}

\begin{proof}
    As the gradient of $f$ is $\gamma$-Lipschitz, we have
    \begin{align*}
    	f(\vecy)={}&f(\vecx)+\int_{\vecx}^{\vecy}\nabla f(\mathbf{z})\rmd\mathbf{z}\\
    	={}&f(\vecx)+\int_0^1\langle\nabla f(\vecx+t(\vecy-\vecx)), \vecy-\vecx\rangle \rmd t\\
    	={}&f(\vecx)+\langle\nabla f(\vecx),\vecy-\vecx\rangle\\
    	&+\int_0^1\langle\nabla f(\vecx+t(\vecy-\vecx))-\nabla f(\vecx), \vecy-\vecx\rangle \rmd t\\
    	\leq{}&f(\vecx)+\langle\nabla f(\vecx),\vecy-\vecx\rangle\\
    	&+\int_0^1\|\nabla f(\vecx+t(\vecy-\vecx))-\nabla f(\vecx)\|_2\| \vecy-\vecx\|_2 \rmd t\\
    	\leq{}&f(\vecx)+\langle\nabla f(\vecx),\vecy-\vecx\rangle+\int_0^1\gamma t\|\vecy-\vecx\|^2_2\rmd t\\
    	\leq{}&f(\vecx)+\langle\nabla f(\vecx), \vecy-\vecx\rangle+\frac{\gamma}{2}\|\vecy-\vecx\|_2^2,
    \end{align*}
    Then, we have
    \begin{align*}
        &f(\vecx_{k+1})\\ \leq{}&f(\vecx_k) + \langle  \nabla f(\vecx_k), \vecx_{k+1} - \vecx_k \rangle+\frac{\gamma}{2}\|\vecx_{k+1}-\vecx_k\|_2^2 \\
        ={}& f(\vecx_k) - \eta \langle \nabla f(\vecx_k), \vecd(\vecx_k) \rangle+ \frac{\eta^2 \gamma}{2}\|\vecd(\vecx_k)\|_2^2 \\
        ={}& f(\vecx_k) - \eta \langle \nabla f(\vecx_k), \Delta_{k} \rangle- \eta \|\nabla f(\vecx_k)\|_2^2\\
        &+ \frac{\eta^2 \gamma}{2}(\|\Delta_{k}\|_2^2+\|\nabla f(\vecx_k)\|_2^2 +2\langle \Delta_{k}, \nabla f(\vecx_k) \rangle)\\
        ={}& f(\vecx_k)  - \eta (1-\eta \gamma) \langle \nabla f(\vecx_k), \Delta_{k} \rangle\\
        &- \eta (1-\frac{\eta \gamma}{2}) \|\nabla f(\vecx_k)\|_2^2 + \frac{\eta^2 \gamma}{2}\|\Delta_{k}\|_2^2.
    \end{align*}
    By taking expectation of both sides, we have
    \begin{align*}
        \mathbb{E}[f(\vecx_{k+1})] \leq{}&\mathbb{E}[f(\vecx_k)] - \eta (1-\eta \gamma)  \mathbb{E}[\langle \nabla f(\vecx_k), \Delta_{k} \rangle]\\
        &- \eta (1-\frac{\eta \gamma}{2}) \mathbb{E}[ \|\nabla f(\vecx_k)\|_2^2]+ \frac{\eta^2 \gamma}{2}\mathbb{E}[\|\Delta_{k}\|_2^2].
    \end{align*}
    By summing up the above inequalities for $k\in[N]$ and dividing both sides by $N \eta(1-\frac{\eta \gamma}{2})$, we have
    \begin{align*}
        &\frac{\sum_{k=1}^{N} \mathbb{E}[ \|\nabla f(\vecx_{k})\|_2^2]}{N}\\
        \leq{}& \frac{f(\vecx_{1}) - \mathbb{E}[f(\vecx_{N})]}{N \eta(1-\frac{\eta \gamma}{2}) } + \frac{\eta \gamma}{2-\eta \gamma} \frac{\sum_{k=1}^N \mathbb{E}[\|\Delta_{k}\|_2^2]}{N}\\
        &- \frac{(1-\eta \gamma)}{(1-\frac{\eta \gamma}{2})} \frac{\sum_{k=1}^{N} \mathbb{E}[\langle \nabla f(\vecx_{k}), \Delta_{k} \rangle]}{N} \\
        \leq{}& \frac{f(\vecx_{1}) - f^* }{N \eta(1-\frac{\eta \gamma}{2}) } + \frac{\eta \gamma}{2-\eta \gamma} \frac{\sum_{k=1}^N \mathbb{E}[\|\Delta_{k}\|_2^2]}{N}\\
        &+ \frac{\sum_{k=1}^{N} |\mathbb{E}[\langle \nabla f(\vecx_{k}), \Delta_{k} \rangle]|}{N},
    \end{align*}
    where the second inequality comes from $\eta \gamma>0$ and $f(\vecx_k) \geq f^* $. According to the above conditions, we have
    \begin{align*}
        &\frac{\sum_{k=1}^{N} \mathbb{E}[ \|\nabla f(\vecx_{k})\|_2^2]}{N}\\
        \leq{}& \frac{f(\vecx_{1}) - f^* }{N \eta(1-\frac{\eta \gamma}{2}) } + \frac{\eta \gamma}{2 - \eta \gamma}G+\frac{1}{\sqrt{N}}.
    \end{align*}
    Notice that
    \begin{align*}
        \mathbb{E}[ \|\nabla f(\vecx_{R})\|_2^2] &= \mathbb{E}_R[\mathbb{E}\|[\nabla f(\vecx_{R})\|_2^2\mid R]]\\
        &=\frac{\sum_{k=1}^{N} \mathbb{E}[ \|\nabla f(\vecx_{k})\|_2^2]}{N},
    \end{align*}
    where $R$ is uniformly chosen from $[N]$, hence we have
    \begin{align*}
        &\mathbb{E}[ \|\nabla f(\vecx_{R})\|_2^2]\\
        \leq{}&\frac{f(\vecx_{1}) - f^* }{N \eta(1-\frac{\eta \gamma}{2}) } + \frac{\eta \gamma}{2 - \eta \gamma}G+\frac{1}{\sqrt{N}}.
    \end{align*}
    By letting $\eta=\min\{\frac{1}{\gamma}, \frac{1}{\sqrt{N}}\}$, we have
    \begin{align*}
        &\mathbb{E}[ \|\nabla f(\vecx_{R})\|_2^2]\\
        \leq{}& \frac{2(f(\vecx_{1})-f^*)}{\sqrt{N}}+\frac{\gamma G}{\sqrt{N}} + \frac{1}{\sqrt{N}}\\
        ={}& \frac{2(f(\vecx_{1})-f^*)+\gamma G + 1}{\sqrt{N}}\\
        ={}& O(\frac{1}{\sqrt{N}}).
    \end{align*}
\end{proof}



\subsubsection{Proof of Theorem 3}

Given an $L$-layer ConvGNN, forllowing \cite{vrgcn}, we directly assume that:
\begin{enumerate}
    \item the optimal value
    \begin{align*}
        \mathcal{L}^*=\inf\limits_{w,\theta^{(1)},\ldots,\theta^{(L)}} \mathcal{L}
    \end{align*}
    is bounded by $G>0$;

    \item the gradients of $\mathcal{L}$ with respect to parameters $w$ and $\theta^{(l)}$, i.e.,
    \begin{align*}
        \nabla_{w}\mathcal{L},\,\nabla_{\theta^{(l)}}\mathcal{L}
    \end{align*}
    are $\gamma$-Lipschitz for $\forall l\in[L]$.
\end{enumerate}
We denote the mini-batch gradients computed by LMC with normallization techniques by
\begin{align*}
    \widetilde{\embg}_w(w_k)=\frac{B}{|\mathcal{V}_{L}|} \sum_{v_j\in\mathcal{V}_{L_{\mathcal{B}_k}}} \nabla_w \ell_{w_k}([\hisH_k]_j, y_j)
\end{align*}
and
\begin{align*}
    \widetilde{\embg}_{\theta^{(l)}}(\theta^{(l)}_k) = B\sum_{v_j\in\mathcal{V}_{\mathcal{B}_k}} (\nabla_{\theta^{(l)}} u_j(\hisH_k^{(l-1)};\theta^{(l)}_k))[\hisV^{(l)}_k]_j,
\end{align*}
where $B$ is the number of subgraphs by partitions and we omit the sampled subgraph $\mathcal{V}_{\mathcal{B}_k}$ when writing $\widetilde{\embg}_w(w_k)$ and $\widetilde{\embg}_{\theta^{(l)}}(\theta^{(l)}_k)$. We denote the exact mini-batch gradients with normalization techniques by
\begin{align*}
    \embg_w(w_k)=\frac{B}{|\mathcal{V}_{L}|} \sum_{v_j\in\mathcal{V}_{L_{\mathcal{B}_k}}} \nabla_w \ell_{w_k}([\embH_k]_j, y_j)
\end{align*}
and
\begin{align*}
    \embg_{\theta^{(l)}}(\theta^{(l)}_k) = B\sum_{v_j\in\mathcal{V}_{\mathcal{B}_k}} (\nabla_{\theta^{(l)}} u_j(\embH_k^{(l-1)};\theta^{(l)}_k))[\embV^{(l)}_k]_j,
\end{align*}
The approximation error of gradients is denoted by
\begin{align*}
    \Delta_{w,k} \triangleq \widetilde{\embg}_w(w_k;\mathcal{V}_{\mathcal{B}_k}) - \nabla_w \loss(w_k)
\end{align*}
and
\begin{align*}
    \Delta_{\theta^{(l)},k} \triangleq \widetilde{\embg}_{\theta^{(l)}}(\theta_k^{(l)};\mathcal{V}_{\mathcal{B}_k}) - \nabla_{\theta^{(l)}} \loss(\theta_k^{(l)}).
\end{align*}
To show the convergence of LMC by Lemma \ref{lemma:suff_conv}, it suffices to show that
\begin{enumerate}[resume]
    \item there exists $G_0>0$ that does not depend on $\eta$ such that
    \begin{align*}
        &\mathbb{E}[\|\Delta_{w,k}\|_2^2]\leq G_0,\,\,\forall\,k\in\mathbb{N}^*,\\
        &\mathbb{E}[\|\Delta_{\theta^{(l)},k}\|_2^2]\leq G_0,\,\,\forall\,l\in[L],\,k\in\mathbb{N}^*;
    \end{align*}

    \item for any $N\in\mathbb{N}^*$, we have
    \begin{align*}
        &|\mathbb{E}[\langle \nabla_{w}\loss, \Delta_{w,k} \rangle]| \leq \frac{1}{\sqrt{N}},\,\,\forall\,k\in\mathbb{N}^*,\\
        &|\mathbb{E}[\langle \nabla_{\theta^{(l)}}\loss, \Delta_{\theta^{(l)},k} \rangle]| \leq \frac{1}{\sqrt{N}},\,\,\forall\,l\in[L],\,k\in\mathbb{N}^*
    \end{align*}
    by letting
    \begin{align*}
        \eta \leq \eta_0 \triangleq \min \{\eta_{w,0}, \eta_{\theta,0}\},
    \end{align*}
    where
    \begin{align*}
        &\eta_{w,0} \triangleq \min_{l\in\{1,L\}} \frac{\frac{1}{2G^2B\gamma\sqrt{N} ((C+3)\gamma \sqrt{B} )^l } }{1-\log_{\frac{B}{B-1}} (\frac{4G^2B\gamma \sqrt{N}-1}{4G^2B(B-1)\gamma \sqrt{N} }) },\\
        &\eta_{\theta,0} \triangleq \min\{ \eta_{h,0}, \eta_{v,0} \},\\
        &\eta_{h,0} \triangleq \min_{l\in\{1,L\}} \frac{\frac{1}{4G^3B|\mathcal{V}|\gamma\sqrt{N} ((C+3)\gamma \sqrt{B} )^l } }{1-\log_{\frac{B}{B-1}} (\frac{8G^3B|\mathcal{V}|\gamma \sqrt{N}-1}{8G^3B(B-1)|\mathcal{V}|\gamma \sqrt{N} }) },\\
        &\eta_{v,0} \triangleq \min_{l\in\{1,L-1\}} \frac{\frac{1}{4G^3B|\mathcal{V}|\sqrt{N} ((C+3)\gamma \sqrt{B} )^l } }{1-\log_{\frac{B}{B-1}} (\frac{8G^3B|\mathcal{V}| \sqrt{N}-1}{8G^3B(B-1)|\mathcal{V}| \sqrt{N} }) }.
    \end{align*}
\end{enumerate}

\begin{lemma}
    If $\|\widetilde{\mathbf{g}}_w(w_k)\|_2$ and $\|\nabla_{w}\loss(w_k)\|_2$ are bounded by $G>0$ for $k\in\mathbb{N}^*$, then we have
    \begin{align*}
        \mathbb{E}[\|\Delta_{w,k}\|_2^2] \leq G_0 \triangleq 4G^2,\,\,\forall\,k\in\mathbb{N}^*.
    \end{align*}
\end{lemma}
\begin{proof}
    We have
    \begin{align*}
        &\mathbb{E}[\|\Delta_{w,k}\|_2^2]\\
        ={}& \mathbb{E}[\|\widetilde{\mathbf{g}}_w(w_k) - \nabla_{w}\loss(w_k)\|_2^2]\\
        \leq{}& 2(\mathbb{E}[\|\widetilde{\mathbf{g}}_w(w_k)\|_2^2] + \mathbb{E}[\|\nabla_{w}\loss(w_k)\|_2^2])\\
        \leq{}& 4G^2.
    \end{align*}
\end{proof}

\begin{lemma}
    If $\|\widetilde{\mathbf{g}}_{\theta^{(l)}}(\theta^{(l)}_k)\|_2$ and $\|\nabla_{\theta^{(l)}}\loss(\theta^{(l)}_k)\|_2$ are bounded by $G>0$ for $k\in\mathbb{N}^*$ and $l\in[L]$, then we have
    \begin{align*}
        \mathbb{E}[\|\Delta_{\theta^{(l)},k}\|_2^2] \leq G_0 \triangleq 4G^2, \,\,\forall\,l\in[L],\,k\in\mathbb{N}^*.
    \end{align*}
\end{lemma}
\begin{proof}
    We have
    \begin{align*}
        &\mathbb{E}[\|\Delta_{\theta^{(l)},k}\|_2^2]\\
        ={}& \mathbb{E}[\|\widetilde{\mathbf{g}}_{\theta^{(l)}}(\theta^{(l)}_k) - \nabla_{\theta^{(l)}}\loss(\theta^{(l)}_k)\|_2^2]\\
        \leq{}& 2(\mathbb{E}[\|\widetilde{\mathbf{g}}_{\theta^{(l)}}(\theta^{(l)}_k)\|_2^2] + \mathbb{E}[\|\nabla_{\theta^{(l)}}\loss(\theta^{(l)}_k)\|_2^2])\\
        \leq{}& 4G^2.
    \end{align*}
\end{proof}


\begin{lemma}
    Suppose some conditions. For any $N\in\mathbb{N}^*$, we have
    \begin{align*}
        |\mathbb{E}[\langle \nabla_{w}\loss, \Delta_{w,k} \rangle]| \leq \frac{1}{\sqrt{N}},\,\,\forall\,k\in\mathbb{N}^*
    \end{align*}
    by letting
    \begin{align*}
        \eta \leq \eta_{w,0} \triangleq \min_{l\in\{1,L\}} \frac{\frac{1}{2G^2B\gamma\sqrt{N} ((C+3)\gamma \sqrt{B} )^l } }{1-\log_{\frac{B}{B-1}} (\frac{4G^2B\gamma \sqrt{N}-1}{4G^2B(B-1)\gamma \sqrt{N} }) }.
    \end{align*}
\end{lemma}
\begin{proof}
    We have
    \begin{align*}
        &|\mathbb{E}[\langle \nabla_{w}\loss, \Delta_{w,k} \rangle]|\\
        ={}& |\mathbb{E}[\langle \nabla_{w}\loss, \widetilde{\mathbf{g}}_w(w_k) - \mathbf{g}_w(w_k) \rangle]|\\
        \leq{}& \mathbb{E}[\|\nabla_{w}\loss\|_2 \|\widetilde{\mathbf{g}}_w(w_k) - \mathbf{g}_w(w_k) \|_2]\\
        \leq{}& \frac{GB}{|\mathcal{V}_L|}\mathbb{E}[\sum_{v_j \in \mathcal{V}_{L_{\mathcal{B}_k}]}} \|\nabla_{w}\ell_{w_k}([\hisH_k]_j,y_j) - \nabla_{w}\ell_{w_k}([\embH_k]_j,y_j)\|_2]\\
        \leq{}& \frac{GB\gamma}{|\mathcal{V}_L|} \mathbb{E}[\sum_{v_j \in \mathcal{V}_{L_{\mathcal{B}_k}]}}\|[\hisH_k]_j-[\embH_k]_j\|_2]\\
        \leq{}& \frac{GB\gamma}{|\mathcal{V}_L|} \mathbb{E}[\sum_{v_j \in \mathcal{V}_{L_{\mathcal{B}_k}]}}\|\hisH_k-\embH_k\|_F]\\
        \leq{}& \frac{GB\gamma}{|\mathcal{V}_L|} \cdot |\mathcal{V}_L|\cdot\mathbb{E}[\|\hisH_k-\embH_k\|_F]\\
        ={}& GB\gamma \cdot \mathbb{E}[\| \hisH_k - \embH_k \|_F].
    \end{align*}
    For any $N\in\mathbb{N}^*$, let $\varepsilon_0=\frac{1}{2GB\gamma\sqrt{N}}>0$ and $\delta_0=\frac{1}{4G^2B\gamma\sqrt{N}}>0$, by Lemma 1 in the main text we know that
    \begin{align*}
        P(\|\hisH_k - \embH_k\|_F > \varepsilon_0) < \delta_0,\,\,\forall k\in\mathbb{N}^*,\,\,{\rm if}\,\,\eta\leq\eta_{w,0}.
    \end{align*}
    Therefore, by letting $\eta\leq\eta_{w,0}$ we have
    \begin{align*}
        &\mathbb{E}[\| \hisH_k - \embH_k \|_F]\\
        ={}& \mathbb{E}[\| \hisH_k - \embH_k \|_F \cdot \mathbbm{1}_{\{\| \hisH_k - \embH_k \|_F<\varepsilon_0}\} ]\\
        &+ \mathbb{E}[\| \hisH_k - \embH_k \|_F \cdot \mathbbm{1}_{\{\| \hisH_k - \embH_k \|_F\geq\varepsilon_0\}} ]\\
        \leq{}& \varepsilon_0 + 2G\delta_0\\
        ={}& \frac{1}{GB\gamma\sqrt{N}},
    \end{align*}
    which leads to
    \begin{align*}
        |\mathbb{E}[\langle \nabla_{w}\loss, \Delta_{w,k} \rangle]| \leq \frac{1}{\sqrt{N}},\,\,\forall\,k\in\mathbb{N}^*.
    \end{align*}
\end{proof}

\begin{lemma}
    Suppose some conditions. For any $N\in\mathbb{N}^*$, we have
    \begin{align*}
        |\mathbb{E}[\langle \nabla_{\theta^{(l)}}\loss, \Delta_{\theta^{(l)},k} \rangle]| \leq \frac{1}{\sqrt{N}},\,\,\forall\, l\in[L],\,k\in\mathbb{N}^*
    \end{align*}
    by letting
    \begin{align*}
        \eta\leq \eta_{\theta,0} \triangleq \min\{ \eta_{h,0}, \eta_{v,0} \},
    \end{align*}
    where
    \begin{align*}
        \eta_{h,0} \triangleq \min_{l\in\{1,L\}} \frac{\frac{1}{4G^3B|\mathcal{V}|\gamma\sqrt{N} ((C+3)\gamma \sqrt{B} )^l } }{1-\log_{\frac{B}{B-1}} (\frac{8G^3B|\mathcal{V}|\gamma \sqrt{N}-1}{8G^3B(B-1)|\mathcal{V}|\gamma \sqrt{N} }) }
    \end{align*}
    and
    \begin{align*}
        \eta_{v,0} \triangleq \min_{l\in\{1,L-1\}} \frac{\frac{1}{4G^3B|\mathcal{V}|\sqrt{N} ((C+3)\gamma \sqrt{B} )^l } }{1-\log_{\frac{B}{B-1}} (\frac{8G^3B|\mathcal{V}| \sqrt{N}-1}{8G^3B(B-1)|\mathcal{V}| \sqrt{N} }) }
    \end{align*}
\end{lemma}
\begin{proof}
    We have
    \begin{align*}
        &|\mathbb{E}[\langle \nabla_{\theta^{(l)}}\loss, \Delta_{\theta^{(l)},k} \rangle]|\\
        ={}& |\mathbb{E}[\langle \nabla_{\theta^{(l)}}\loss, \widetilde{\mathbf{g}}_{\theta^{(l)}}(\theta^{(l)}_k) - \mathbf{g}_{\theta^{(l)}}(\theta^{(l)}_k) \rangle]|\\
        \leq{}& \mathbb{E}[\|\nabla_{\theta^{(l)}}\loss\|_2 \|\widetilde{\mathbf{g}}_{\theta^{(l)}}(\theta^{(l)}_k) - \mathbf{g}_{\theta^{(l)}}(\theta^{(l)}_k) \|_2]\\
        \leq{}& GB\mathbb{E}[\sum_{v_j\in\mathcal{V}_{\mathcal{B}_k}} \|(\nabla_{\theta^{(l)}} u_j(\hisH_k^{(l-1)};\theta^{(l)}_k))[\hisV^{(l)}_k]_j\\
        &\quad\quad\quad\quad\quad\quad-(\nabla_{\theta^{(l)}} u_j(\embH_k^{(l-1)};\theta^{(l)}_k))[\embV^{(l)}_k]_j\|_2]\\
        \leq{}& GB\mathbb{E}[\sum_{v_j\in\mathcal{V}_{\mathcal{B}_k}} \|(\nabla_{\theta^{(l)}} u_j(\hisH_k^{(l-1)};\theta^{(l)}_k))[\hisV^{(l)}_k]_j\\
        &\quad\quad\quad\quad\quad\quad-(\nabla_{\theta^{(l)}} u_j(\hisH_k^{(l-1)};\theta^{(l)}_k))[\embV^{(l)}_k]_j\\
        &\quad\quad\quad\quad\quad\quad+(\nabla_{\theta^{(l)}} u_j(\hisH_k^{(l-1)};\theta^{(l)}_k))[\embV^{(l)}_k]_j\\
        &\quad\quad\quad\quad\quad\quad-(\nabla_{\theta^{(l)}} u_j(\embH_k^{(l-1)};\theta^{(l)}_k))[\embV^{(l)}_k]_j\|_2]\\
        \leq{}& G^2B\mathbb{E}[\sum_{v_j\in\mathcal{V}_{\mathcal{B}_k}} \|[\hisV_k^{(l)}]_j-[\embV_k^{(l)}]_j\|_2 + \gamma \|[\hisH_k^{(l)}]_j-[\embH_k^{(l)}]_j\|_2]\\
        \leq{}& G^2B\mathbb{E}[\sum_{v_j\in\mathcal{V}_{\mathcal{B}_k}} \|\hisV_k^{(l)}-\embV_k^{(l)}\|_2 + \gamma \|\hisH_k^{(l)}-\embH_k^{(l)}\|_2]\\
        \leq{}& G^2B|\mathcal{V}|\cdot\mathbb{E}[\|\hisV_k^{(l)}-\embV_k^{(l)}\|_2] + G^2B|\mathcal{V}|\gamma\cdot \mathbb{E}[\|\hisH_k^{(l)}-\embH_k^{(l)}\|_2].
    \end{align*}
    For any $N\in\mathbb{N}^*$, let $\varepsilon_1=\frac{1}{4G^2B|\mathcal{V}|\gamma \sqrt{N} }$, $\varepsilon_2=\frac{1}{4G^2B|\mathcal{V}| \sqrt{N} }$, $\delta_1=\frac{1}{8G^3B|\mathcal{V}|\gamma \sqrt{N} }$, and $\delta_2=\frac{1}{8G^3B|\mathcal{V}|\sqrt{N} }$, by Lemma 1 in the main text we know that
    \begin{align*}
        &P(\|\hisH_k^{(l)} - \embH_k^{(l)}\|_F > \varepsilon_1) < \delta_1,\,\,\forall k\in\mathbb{N}^*,\,l\in[L],\\
        &P(\|\hisV_k^{(l)} - \embV_k^{(l)}\|_F > \varepsilon_2) < \delta_2,\,\,\forall k\in\mathbb{N}^*,\,l\in[L],\,\,{\rm if}\,\,\eta\leq\eta_{\theta,0}.
    \end{align*}
    Therefore, if we let $\eta\leq\eta_{\theta,0}$, then we have
    \begin{align*}
        &\mathbb{E}[\| \hisH^{(l)}_k - \embH^{(l)}_k \|_F]\\
        ={}& \mathbb{E}[\| \hisH^{(l)}_k - \embH^{(l)}_k \|_F \cdot \mathbbm{1}_{\{\| \hisH^{(l)}_k - \embH^{(l)}_k \|_F<\varepsilon_1}\} ]\\
        &+ \mathbb{E}[\| \hisH^{(l)}_k - \embH^{(l)}_k \|_F \cdot \mathbbm{1}_{\{\| \hisH^{(l)}_k - \embH^{(l)}_k \|_F\geq\varepsilon_1\}} ]\\
        \leq{}& \varepsilon_1 + 2G\delta_1\\
        = {}& \frac{\varepsilon}{2G^2B|\mathcal{V}|\gamma}
    \end{align*}
    and
    \begin{align*}
        &\mathbb{E}[\| \hisV^{(l)}_k - \embV^{(l)}_k \|_F]\\
        ={}& \mathbb{E}[\| \hisV^{(l)}_k - \embV^{(l)}_k \|_F \cdot \mathbbm{1}_{\{\| \hisV^{(l)}_k - \embV^{(l)}_k \|_F<\varepsilon_1}\} ]\\
        &+ \mathbb{E}[\| \hisV^{(l)}_k - \embV^{(l)}_k \|_F \cdot \mathbbm{1}_{\{\| \hisV^{(l)}_k - \embV^{(l)}_k \|_F\geq\varepsilon_1\}} ]\\
        \leq{}& \varepsilon_2 + 2G\delta_2\\
        \leq{}& \frac{\varepsilon}{2G^2B|\mathcal{V}|},
    \end{align*}
    which leads to
    \begin{align*}
        |\mathbb{E}[\langle \nabla_{\theta^{(l)}}\loss, \Delta_{\theta^{(l)},k} \rangle]| \leq \varepsilon,\,\,\forall\,k\in\mathbb{N}^*. 
    \end{align*}  
\end{proof}


\subsubsection{Proof of Theorem 4}

Following \cite{vrgcn}, we directly assume that conditions \ref{con:1} and \ref{con:2} hold. We then derive that the conditions \ref{con:3} and \ref{con:4} hold for the mini-batch gradients
\begin{align*}
    \widetilde{\mathbf{g}}_w(w^{(k)}) = \frac{1}{|\mathcal{V}_{L_{\mathcal{B}}}^{(k)}|} \sum_{v_j \in \mathcal{V}_{L_{\mathcal{B}}}^{(k)}}\nabla_{w} l_w(\widetilde{\embh}_j,y_j)
\end{align*}
and
\begin{align*}
    \widetilde{\mathbf{g}}_{\theta}(\theta^{(k)}) = \frac{|\mathcal{V}|}{|\inbatch^{(k)}|} \sum_{v_j \in \inbatch^{(k)}} \nabla_{\theta} \update(\widetilde{\embh}_j,\overline{\embm}_{\neighbor{v_j}},\embx_j) \widetilde{\embV}_j
\end{align*}
computed by LMC, where we omit the superscript $(k)$ of $\widetilde{\embh}$, $\overline{\embm}$, $\widetilde{\embV}$, and $\inbatch^{(k)}$ is the mini-batch of nodes at $k$-th iteration. Denote the exact mini-batch gradients by
\begin{align*}
    \mathbf{g}_w(w^{(k)}) = \frac{1}{|\mathcal{V}_{L_{\mathcal{B}}}^{(k)}|} \sum_{v_j \in \mathcal{V}_{L_{\mathcal{B}}}^{(k)}}\nabla_{w} l_w(\embh_j ,y_j)
\end{align*}
and
\begin{align*}
    \mathbf{g}_{\theta}(\theta^{(k)}) = \frac{|\mathcal{V}|}{|\inbatch^{(k)}|} \sum_{v_j \in \inbatch^{(k)}} \nabla_{\theta} \update(\embh_j ,\embm_{\neighbor{v_j}}  ,\embx_j) \embV_j.
\end{align*}


\begin{lemma}\label{lemma:w_amp_variance}
    If $\|\widetilde{\mathbf{g}}_w(w^{(k)})\|_2$ and $\|\nabla_{w} \mathcal{L}\|_2$ are bounded by $G>0$,
    then we have
    \begin{align*}
        \mathbb{E}[\| \Delta_w^{(k)} \|_2^2] \leq B_1,
    \end{align*}
    where $B_1=4G^2$ and $\Delta_w^{(k)}=\widetilde{\mathbf{g}}_w(w^{(k)}) - \nabla_{w} \loss$ is the estimation error of the gradient of $\loss$ with respect to $w$ at $k$-th iteration.
\end{lemma}
\begin{proof}
    By 
    \begin{align*}
        \mathbb{E}[(X_1+X_2)^2] \leq 2(\mathbb{E}[X_1^2]+\mathbb{E}[X_2^2]),
    \end{align*}
    we have
    \begin{align*}
        &\mathbb{E}[\| \Delta_w^{(k)} \|_2^2]\\
        ={}&\mathbb{E}[\| \widetilde{\mathbf{g}}_w(w^{(k)}) - \nabla_{w} \mathcal{L} \|_2^2]\\
        \leq{}& 2( \mathbb{E}[ \| \widetilde{\mathbf{g}}_w(w^{(k)})\|_2^2] +  \mathbb{E}[ \|\nabla_{w} \mathcal{L}\|_2^2])\\
        \leq{}& 4 G^2.
    \end{align*}
\end{proof}




\begin{lemma}\label{lemma:theta_amp_variance}
    If $\|\widetilde{\mathbf{g}}_{\theta}(\theta^{(k)})\|_2$ and $\|\nabla_{\theta} \mathcal{L}\|_2$ are bounded by $G>0$, then we have
    \begin{align*}
        \mathbb{E}[\| \Delta_\theta^{(k)} \|_2^2] \leq B_1,
    \end{align*}
    where $B_1=4G^2$ and $\Delta_\theta^{(k)}=\widetilde{\mathbf{g}}_\theta(\theta^{(k)}) - \nabla_{\theta} \loss$ is the estimation error of the gradient of $\loss$ with respect to $\theta$ at $k$-th iteration.
\end{lemma}
\begin{proof}
    By 
    \begin{align*}
        \mathbb{E}[(X_1+X_2)^2] \leq 2(\mathbb{E}[X_1^2]+\mathbb{E}[X_2^2]),
    \end{align*}
    we have
    \begin{align*}
        &\mathbb{E}[\| \Delta_\theta^{(k)} \|_2^2]\\
        ={}&\mathbb{E}[\| \widetilde{\mathbf{g}}_\theta(\theta^{(k)}) - \nabla_{\theta} \mathcal{L} \|_2^2]\\
        \leq{}& 2( \mathbb{E}[ \| \widetilde{\mathbf{g}}_\theta(\theta^{(k)})\|_2^2] +  \mathbb{E}[ \|\nabla_{\theta} \mathcal{L}\|_2^2])\\
        \leq{}& 4 G^2.
    \end{align*}
\end{proof}




\begin{lemma}\label{lemma:w_amp_inner_product}
    Suppose that the conditions in Lemma \ref{lemma:amp_cfpi} hold, i.e., $f_\theta$ is $L$-Lipschitz for parameter $\theta$ and $\gamma$-contraction for $\embH$. Suppose that $\nabla_{w} l_w(\embh,y)$ is $L$-Lipschitz, and $\|\widetilde{\mathbf{g}}_w(w^{(k)})\|_2, \|\nabla_{w} \mathcal{L}\|_2$ are bounded by $G>0$, then there exists $B_2 \geq \max\{ LG^2, \frac{KLG^2}{1-\rho}\}$ such that
    \begin{align*}
        |\mathbb{E}[\langle \nabla_{w} \mathcal{L}, \Delta_w^{(k)} \rangle]| \leq \eta B_2 + \rho^{k-1} B_2,
    \end{align*}
    where $\rho=\sqrt{(1-(1-\gamma^2)/b)}<1$, $K=\frac{2L}{1-\gamma}$, $b$ is the number of partition subgraphs, and $\Delta_w^{(k)}=\widetilde{\mathbf{g}}_w(w^{(k)}) - \nabla_{w} \loss$ is the estimation error of the gradient of $\loss$ with respect to $w$ at $k$-th iteration.
\end{lemma}
\begin{proof}
    We have
    \begin{align*}
        &|\mathbb{E}[\langle \nabla_{w} \mathcal{L}, \Delta_w^{(k)} \rangle]|\\
        ={}&|\mathbb{E}[\langle \nabla_{w} \mathcal{L}, \widetilde{\mathbf{g}}_w(w^{(k)}) - \nabla_{w} \mathcal{L} \rangle]|\\
        ={}&|\mathbb{E}[\langle \nabla_{w} \mathcal{L}, \widetilde{\mathbf{g}}_w(w^{(k)}) - \mathbf{g}_w(w^{(k)}) \rangle]|\\
        \leq{}& \mathbb{E}[\| \nabla_{w} \mathcal{L}\|_2 \| \widetilde{\mathbf{g}}_w(w^{(k)}) - \mathbf{g}_w(w^{(k)}) \|_2]\\
        \leq{}& G \mathbb{E}[ \frac{1}{|\mathcal{V}^{(k)}_{L_\mathcal{B}}|}\|\sum_{v_j \in \mathcal{V}^{(k)}_{L_\mathcal{B}}}\nabla_w l_w(\widetilde{\embh}_j,y_j)- \nabla_w l_w(\embh_j,y_j)\|_2]\\
        \leq{}& G L \mathbb{E}[ \|\overline{\embH}_{\mathcal{V}_{\mathcal{B}_j}}^{(k)}-\embH _{\mathcal{V}_{\mathcal{B}_j}}^{(k)}\|_F]\\
        \leq{}& G L \mathbb{E}[ \|\overline{\embH}^{(k)} - \embH^{(k)}\|_F]\\
        \leq{}& G L \sqrt{\mathbb{E}[ \|\overline{\embH}^{(k)}-\embH^{(k)}\|_F^2]}
    \end{align*}
    According to Lemma 1 in the main text, we have
    \begin{align*}
        |\mathbb{E}[\langle \nabla_{w} \mathcal{L}, \Delta_w^{(k)} \rangle]| &\leq G L (\frac{KG}{1-\rho}\eta + \rho^{k-1}G)\\
        &=\eta \frac{KLG^2}{1-\rho}+\rho^{k-1}LG^2\\
        &\leq \eta B_2 + \rho^{k-1} B_2.
    \end{align*}
\end{proof}





\begin{lemma}\label{lemma:theta_amp_inner_product}
    Suppose that the conditions in Lemmas \ref{lemma:amp_cfpi} and \ref{lemma:amp_cfpi_b} hold. Suppose that the gradient $\nabla_{\vec{\embH}} \update(\embh_j,\embm_{\neighbor{v_j}},\embx_j)$ is $L$-Lipschitz.
    % Let the in-batch error of the steady-states be $\varepsilon_h^{(k)} = \max_j \|\embH_{\mathcal{V}_{\mathcal{B}_j}}^{(k)}-[\embH _{\mathcal{V}_{\mathcal{B}_j}}]^{(k)}\|_F$ and the upstream gradients be $\varepsilon_v^{(k)} = \max_j \|\embV_{\mathcal{V}_{\mathcal{B}_j}}^{(k)}-[\embV _{\mathcal{V}_{\mathcal{B}_j}}]^{(k)}\|_F$, respectively.
    The norms of gradients $\|\embV\|_F$ and $\|\nabla_{\theta}\update(\embh_j,\embm_{\neighbor{v_j}},\embx_j)\|_F$ are bounded by $G$. Then, there exist constants $\hat{B} \geq \max\{ (1+L|\mathcal{V}|)B^2, (1+L|\mathcal{V}|)\frac{KB^2}{1-\rho}\}$ such that
    \begin{align*}
        |\mathbb{E}[\langle \nabla_{\theta} \mathcal{L}, \widetilde{\mathbf{g}}_{\theta}(\theta^{(k)}) - \nabla_{\theta} \mathcal{L} \rangle]| \leq \hat{B} \rho^{k-1} + \hat{B} \eta,
    \end{align*}
    where $|\mathcal{V}|$ is the number of nodes in the graph.
\end{lemma}
\begin{proof}
    As $\|\mathbf{A}_1\mathbf{a}_1-\mathbf{A}_2\mathbf{a}_2\|_2 \leq \|\mathbf{A}_1\|_F\|\mathbf{a}_1-\mathbf{a}_2\|_2+\|\mathbf{A}_1-\mathbf{A}_2\|_F\|\mathbf{a}_2\|_2$,
    we can bound $\| \widetilde{\mathbf{g}}_{\theta}(\theta^{(k)}) - \mathbf{g}_{\theta}(\theta^{(k)}) \|_2$ by
    \begin{align*}
         &\| \widetilde{\mathbf{g}}_{\theta}(\theta^{(k)}) - \mathbf{g}_{\theta}(\theta^{(k)}) \|_2\\
         ={}& \frac{|\mathcal{V}|}{|\inbatch^{(k)}|} \| \sum_{v_j \in \inbatch^{(k)}} (\nabla_{\theta} \update(\widetilde{\embh}_j,\overline{\embm}_{\neighbor{v_j}},\embx_j) \widetilde{\embV}_j)\\
         &\quad\quad\quad\quad\quad\quad\quad-\nabla_{\theta} \update(\embh_j,\embm_{\neighbor{v_j}},\embx_j) \embV_j\|_2\\
         \leq{}& \frac{|\mathcal{V}|}{|\inbatch^{(k)}|} \sum_{v_j \in \inbatch^{(k)}} \|\nabla_{\theta} \update(\widetilde{\embh}_j,\overline{\embm}_{\neighbor{v_j}},\embx_j) \widetilde{\embV}_j\\
         &\quad\quad\quad\quad\quad\quad\quad-\nabla_{\theta} \update(\embh_j,\embm_{\neighbor{v_j}},\embx_j) \embV_j\|_2\\
         \leq{}& |\mathcal{V}|\max_{v_j \in \inbatch^{(k)}} \| \nabla_{\theta} \update(\widetilde{\embh}_j,\overline{\embm}_{\neighbor{v_j}},\embx_j) \widetilde{\embV}_j\\
         &\quad\,\,\,\quad\quad\quad\quad-\nabla_{\theta} \update(\embh_j,\embm_{\neighbor{v_j}},\embx_j) \embV_j\|_2\\
         \leq{}& |\mathcal{V}|\max_{v_j \in \inbatch}\{ \| \nabla_{\theta} \update(\widetilde{\embh}_j,\overline{\embm}_{\neighbor{v_j}},\embx_j)  \|_F\| \widetilde{\embV}_j - \embV_j\|_2 \\
         &\quad\quad\quad\quad\,+\|\embV_j\|_2\| \nabla_{\theta} \update(\widetilde{\embh}_j,\overline{\embm}_{\neighbor{v_j}},\embx_j)\\
         &\quad\,\,\,\,\quad\quad\quad\,\,\,\quad\quad\quad\quad- \nabla_{\theta} \update(\embh_j,\embm_{\neighbor{v_j}},\embx_j) \|_F\}\\
         \leq{}& \max_{v_j \in \inbatch^{(k)}} \{ \| \nabla_{\theta} \update(\widetilde{\embh}_j^{(k)},\overline{\embm}_{\neighbor{v_j}}^{(k)},\embx_j)  \|_F\| \overline{\embV}_j^{(k)} - \embV_j^{(k)} \|_2\\
         &+GL \max(\| \overline{\embh}_j - \embh_j \|_F,\| \widetilde{\embh}_j - \embh_j \|_F) \}\\
         \leq{}&  B \|\overline{\embV}^{(k)} - \embV^{(k)}\|_F + BL|\mathcal{V}|\|\overline{\embH}^{(k)} -  \embH^{(k)}\|_F.
    \end{align*}
    By Lemmas 1 and 2 in the main text, we have
    \begin{align*}
        \mathbb{E}[\|\overline{\embV}^{(k)} - \embV ^{(k)}\|_F] \leq d_v^{(k+1)} \leq \rho^{k-1}B + \frac{KB}{1-\rho} \eta,\\
        \mathbb{E}[\|\overline{\embH}^{(k)} -  \embH^{(k)}\|_F] \leq d_h^{(k+1)} \leq \rho^{k-1}B + \frac{KB}{1-\rho} \eta.
    \end{align*}
    By taking the expectation, we have
    \begin{align*}
        &\mathbb{E}[\| \widetilde{\mathbf{g}}_{\theta}(\theta^{(k)}) - \mathbf{g}_{\theta}(\theta^{(k)}) \|_2]\\
        \leq{}& B(\rho^{k-1}B+\frac{KB}{1-\rho}\eta)+BL|\mathcal{V}|(\rho^{k-1}B+\frac{KB}{1-\rho}\eta)\\
        ={}& \hat{B} \rho^{k-1} + \hat{B} \eta.
    \end{align*}
\end{proof}






According to Lemma \ref{lemma:w_amp_variance}, \ref{lemma:theta_amp_variance}, \ref{lemma:w_amp_inner_product}, and \ref{lemma:theta_amp_inner_product}, the conditions in Lemma \ref{lemma:sgd} hold. Theorem 2 in the main text follows immediately.



\section{More Experiments}


In this section, we report more experiments on the prediction performance for more baselines including the state-of-the-art RecGNNs (SSE \cite{sse} and IGNN \cite{ignn}) and \modify{}{our implemented RecGCNs trained by} full-batch gradient descent (GD), CLUSTER \cite{deq_gcn}, GraphSAINT \cite{graphsaint}, and GAS.



We evaluate these methods on the protein-protein interaction dataset (PPI), the Reddit dataset \cite{graphsage}, the Amazon product co-purchasing network dataset (AMAZON) \cite{amazon}, and open graph benchmarkings (Ogbn-arxiv and Ogbn-products) \cite{ogb} in Table \ref{tab:largegraph}.
GraphSAINT suffers from the out-of-memory issue in the evaluation process on Ogbn-products.
LMC outperform GD, GraphSAINT, and CLUSTER by a large margin and achieve the comparable prediction performance against GAS.
Moreover, Table 1 in the main text demonstrates that LMC enjoys much faster convergence than GAS, showing the effectiveness of LMC.
Therefore, LMC can accelerate the training of RecGNNs without sacrificing accuracy.





% When using GD, RecGCN suffers from out-of-memory issues on PPI and Ogbn-products.
% Moreover, IGNN, which is also trained by GD, also suffer from OOM on ogbn-products.
% Therefore, GD suffers from severe OOM issues when training RecGNNs on large graphs.



\begin{table}[h]
  \centering
  \caption{%
    \textbf{Performance on large graph datasets.}
    OOM denotes the out-of-memory issue.
  }\label{tab:largegraph}
  \setlength{\tabcolsep}{2pt}
  \resizebox{\linewidth}{!}{%
  \begin{tabular}{llcccccc}
    \toprule
    \mc{2}{l}{\footnotesize{\textbf{\#\,nodes}}} & \footnotesize{57K}  & \footnotesize{230K}  & \footnotesize{335K} & \footnotesize{169K} & \footnotesize{2.4M} \\[-0.1cm]
    \mc{2}{l}{\footnotesize{\textbf{\#\,edges}}} & \footnotesize{794K}  &  \footnotesize{11.6M} & \footnotesize{926K} & \footnotesize{1.2M} & \footnotesize{61.9M} \\[-0.05cm]
    \mc{2}{l}{\textbf{Method}} & \textsc{PPI} & \textsc{REDDIT} & \textsc{Amazon}& \texttt{Ogbn-arxiv} & \texttt{Ogbn-products} \\
    \midrule
    &\textsc{SSE}         & 83.60          & ---              & 85.08                  & ---                  & --- \\
    &\textsc{IGNN}        & 97.56          & 94.95            & 85.31              & 70.88              & OOM \\
    \midrule
    % \mr{4}{\rotatebox{90}{RecGCN}}
    %& \\ [-0.33cm]
    & \textsc{GD}   & OOM   & 96.35   & 86.61 & 68.43   & OOM \\
    & \textsc{GraphSAINT}   & 61.14   & 96.60   & 43.53 & 70.93   & OOM \\
    & \textsc{CLUSTER}   & 97.37     & 94.49 & 85.69 & 69.23 & 76.32 \\ 
    & \textsc{GAS}   & 98.88   & 96.44 & 88.34 & 71.92   & 77.32 \\ 
    & \textsc{LMC}   & \textbf{98.99}     & \textbf{96.58} & \textbf{88.84} & \textbf{72.64}   & \textbf{77.73} \\ % [0.125cm]
    \bottomrule
  \end{tabular}
 }
\end{table}


% Although both CLUSTER, GAS, and LMC are scalable, the prediction performance of LMC outperforms that of the subgraph-sampling methods on all datasets. As shown in Section \ref{sec:lrp}, subgraph-sampling methods may hurt meaningful topological structures and long-range patterns in graphs, which leads to a sub-optimal performance.

% \begin{table*}[t]
% \centering
% \caption{Classification accuracies for the missing-vector setting.}
% \begin{tabular}{cccccccc}
% \hline
%     & \multicolumn{2}{c}{\textbf{Cora}}       & \multicolumn{2}{c}{\textbf{CiteSeer}}   & \multicolumn{2}{c}{\textbf{PubMed}}     \\ 
%     & \textbf{0\%}    & \textbf{100\%}      & \textbf{0\%}    & \textbf{100\%}      & \textbf{0\%}    & \textbf{100\%}    \\ \hline
%     \textbf{ConvGCN}                    & 82.5$\pm$ 1.2     & 58.8 $\pm$ 3.5    & 69.5 $\pm$ 2.1    & 31.3 $\pm$ 2.7    & 77.9 $\pm$ 1.4 & 44.9 $\pm$ 4.4  \\ 
%     % \textbf{RecGCN+CLUSTER}      & 81.5 $\pm$ 1.0    & 64.0 $\pm$ 4.6    & 71.4 $\pm$ 0.6    & 36.4 $\pm$ 4.1    & 78.3 $\pm$ 0.7   & 44.8 $\pm$ 3.2 \\
%     \textbf{RecGCN+CLUSTER}      & 80.9 $\pm$ 1.4    & 64.5 $\pm$ 7.8    & 69.7 $\pm$ 2.7    & 47.7 $\pm$ 1.1    & 78.5 $\pm$ 0.7   & 49.4 $\pm$ 8.1 \\
%     \textbf{RecGCN+GAS}      & 82.0 $\pm$ 0.7    & 68.6 $\pm$ 7.8    & 70.3 $\pm$ 1.3    & 48.7 $\pm$ 3.0    & 78.7 $\pm$ 0.5   & 55.8 $\pm$ 9.8 \\
%     % \textbf{RecGCN+LMC}           & \textbf{82.6 $\pm$ 0.6}    & \textbf{64.8 $\pm$ 7.8}    & \textbf{71.7 $\pm$ 1.1}    & \textbf{39.9 $\pm$ 4.6}    & \textbf{79.1 $\pm$ 0.5}   & \textbf{50.8 $\pm$ 7.0} \\\hline
%     \textbf{RecGCN+LMC}           & \textbf{82.6 $\pm$ 0.6}    & \textbf{70.6 $\pm$ 6.1}    & \textbf{70.9 $\pm$ 1.8}    & \textbf{51.0 $\pm$ 1.4}    & \textbf{79.7 $\pm$ 0.6}   & \textbf{58.0 $\pm$ 13.8} \\\hline % ready
%     \textbf{ConvGAT}                    & 82.3 $\pm$ 2.3    & 65.3 $\pm$ 2.1    & 69.3 $\pm$ 1.6    & 42.8 $\pm$ 1.6    & 77.4 $\pm$ 0.5   & 63.1 $\pm$ 0.7 \\
%     % \textbf{RecGAT+CLUSTER}      & 81.4 $\pm$ 0.9 & 64.3 $\pm$ 2.6    & \textbf{70.7 $\pm$ 0.9}        & 44.5 $\pm$ 4.0      & 77.9 $\pm$ 0.7     & 44.9 $\pm$ 2.7 \\
%     \textbf{RecGAT+CLUSTER}      & 81.0 $\pm$ 0.5 & 67.5 $\pm$ 1.8    & 70.4 $\pm$ 1.0        & 43.8 $\pm$ 6.2      & 78.5 $\pm$ 0.2     & 45.7 $\pm$ 2.2 \\
%     \textbf{RecGAT+GAS}      & 81.5 $\pm$ 1.8 & 62.8 $\pm$ 6.1    & 70.9 $\pm$ 1.2        & 47.8 $\pm$ 3.1      & 79.1 $\pm$ 0.3     & 63.8 $\pm$ 2.1 \\
%     % \textbf{RecGAT+LMC}           & \textbf{82.5$\pm$ 0.7} & \textbf{69.4$\pm$ 1.0}  & \textbf{70.7 $\pm$ 0.6} & \textbf{51.0 $\pm$ 1.7} & \textbf{79.5 $\pm$ 0.8} & \textbf{67.4 $\pm$ 2.1} \\ \hline
%     \textbf{RecGAT+LMC}           & \textbf{83.2$\pm$ 0.5} & \textbf{70.6 $\pm$ 2.4}  & \textbf{71.4 $\pm$ 0.7} & \textbf{51.3 $\pm$ 2.3} & \textbf{79.9 $\pm$ 0.8} & \textbf{65.2 $\pm$ 3.7} \\ \hline
% \end{tabular}
% \vspace{-4mm}
% \label{tab:missing_vector}
% \end{table*}

% \subsection{LMC learns long-range interactions}\label{sec:lrp}


% % In Section \ref{sec:exp_fast}, we find that LMC, CLUSTER, and GAS are much more memory and time-efficient than naive SGD, while the prediction performances of CLUSTER and GAS are weaker than that of naive SGD.
% In this section, we evaluate the ability to learn long-range dependencies of  CLUSTER, GAS, and LMC.
% % % We further demonstrate that LMC and naive SGD preserve long-range patterns in graph and learn long-range dependencies but SUBGRAPH do not.



% % \subsubsection{The synthetic Chains dataset}


% % We first evaluate naive SGD, LMC, CLUSTER, and GAS on the synthetic Chains dataset \cite{ignn}. We aims to classify nodes in a length-$l$ chain which the class information about is only provided as the feature in the end node. Thus, to classify all nodes in a chain, the GNN models must aggregate the information from at least $(l-1)$-th hop neighbors. We use a training set, validation set, and testing set with 100, 100, and 200 nodes, respectively.
% % In practice, to scale to a large graph under finite memory consumption, we partition off it to the enough small subgraphs. The number of the partitioned subgraphs may be large. Therefore, we fix the length of chain as $20$ and plot the prediction performance with respect to the number of subgraph partitions in Figure \ref{fig:chains}.



% % \begin{figure}[htbp]
% % \centering % <-- added
% %   \includegraphics[width=200pt]{imgs/exp_sec2/chains.pdf}
% %   \caption{Micro-$F_1$ (\%) performance with respect to the number of the partitioned subgraphs}\label{fig:chains}
% % \end{figure}


% % When the number of subgraph partitions is greater than $20$, the METIS partitions divide a chain into many subgraphs. As CLUSTER and GAS ignore edges among the subgraphs, it can not aggregate the information from different subgraphs. Therefore, CLUSTER and GAS fail to give meaningful predictions when the number of subgraph partitions is large enough. LMC and standard SGD can aggregate the information from different subgraphs. Thus, LMC and standard SGD can preserve the long-range patterns in a large graph, even if the large graph is partitioned to many small subgraphs.






% % \subsubsection{The real-world graphs under missing-vector setting}


% % We evaluate CLUSTER, GAS, and LMC in real-world graphs. 
% As the node prediction in most standard node classification benchmarks mainly depends on short-range dependencies, we follow the missing feature setting \cite{pairnorm} to impose long-range dependencies in real-world graphs. We remove the feature vectors in a subset $\mathcal{M}$ of the unlabeled nodes $\mathcal{V}_u$. We denote the missing fraction by $p = \frac{|\mathcal{M}|}{|\mathcal{V}_u|}$. Under the missing feature setting, RecGNNs require a larger number of propagation steps to recover the effective feature representation for the missing feature nodes.




% % We use three standard node classification datasets: Cora, Citeseer, and PubMed \cite{planetoid} with the same data splits as \citet{gcn}.
%  We report the two node feature setups: the 0\% setup and the 100\% setup for each dataset in Table \ref{tab:missing_vector},.
% % We partition each graph to $10$ subgraphs.
% The models trained by LMC outperform CLUSTER and GAS on all datasets, as CLUSTER and GAS do not consider external connectivity among partitioned subgraphs in backward passes, which leads to a sub-optimal prediction performance of RecGNNs.




% We find that RecGCN and RecGAT outperform GCN and GAT respectively. Thus, RecGNNs can learn long-range dependencies better than ConvGNNs. Notice that we do not introduce the additional normalization methods \cite{pairnorm, lipschitznorm} to further alleviate over-smoothing, as these methods are mainly designed for ConvGNNs. Developing the normalization methods for RecGNNs is promising and we left it for future work.


\section{Potential Societal Impacts}


In this paper, we propose  novel and scalable training algorithm for RecGNNs.
This work is promising in many practical and important scenarios such as search engine, recommendation systems, biological networks, and molecular property prediction.
% The application of this work may make our daily life more convenient and efficient.
Nonetheless, this work may have some potential risks. For example, using this work in search engine and recommendation systems to over-mine the behavior of users may cause undesirable privacy disclosure. 

\bibliographystyle{IEEEtran}
\bibliography{LMC@TPAMI}

\end{document}


