\section{Introduction}

% =========================================   OLD  ==========================================

% yw / the paragraphs are good. i wonder if it will helpful to expand a little into the core challenges and our fixes / proposal etc. for example, we can follow a sequence idea as folloes.

  % Modern recommendation systems have large impacts on the structural evolution of social networks, which further impacts downstream information sharing and resource allocation.
  % 
  %  then expand on this sentence in the paragraph
  
  % However, these impacts of recommendation systems on social networks can be challenging to assess quantitatively even in simulation settings: these impacts can be non-stationary, persistent, or delayed, which makes standard evaluation metrics of A/B tests or longitudinal studies unsuitable. 

% expand on this sentence in the second paragraph, especially the challenges of existing work.
  
  
%
  % To this end, we propose a dynamic network formation model and its accompanying metrics for assessing impacts of recommendation systems on networks. The core idea is to extend the Jackson-Rogers model to including an additional phase of ``connection due to recommendations''; we then identify the conditions (stationary; transient; immediate) under which the commonly adopted metrics can produce valid estimates of the impact on networks due to recommendation algorithms. 

% expand on it; maybe give some intuition of why the model choice etc?

  
  % We finally corroborate the proposal with empirical studies, showing that the validity of empirical metrics relies heavily on the conditions we identify.

  % expand on main findings of the empirical studies.


  % in general, intro is good. we could emphasize a little the challenges that existing work suffers from and how our proposal resolves it.

  

 %Understanding how algorithmic recommendations impact societal networks is essential as social networks play a large role in mediating access to resources and in the diffusion of information. Therefore changes in the network structure can lead to material changes to people's opinions, resources and general well-being. Modern recommendations systems have a large influence on structural evolution of social graphs as well as downstream outcomes. Recommendations can be a polarizing force, promoting links between like-minded individuals. They can lead to network segregation which in turn may correspond to limited access to resources to less connected minorities. At the same time algorithmic recommendations present an opportunity to increase network integration by promoting and maintaining diverse links and surfacing connections that users would not naturally be exposed to.

% Generally recommendation systems can impact social networks directly in the case of link recommendation as in follow and friend recommendations on social networking sites. Recommendation systems can also impact social networks implicitly through content recommendation; for instance in an academic setting citation networks are impacted by paper recommendation settings; in professional networks job recommendations can mediate the formation of professional links between individuals; and in social networks recommendation of new content such as posts increase exposure of content creators.

%Measuring the impact of recommendations on social ties is hard to do in academic settings since proper evaluations of recommendation impacts require access to a production environment. Even in that case A/B testing through separation of nodes into treatment and control is difficult since networks have... umm.. network effects which makes the evaluation hard. Similarly, longitudinal studies can also have limitations for two reasons: first, the effects of recommendation can be persistent and second the evolution of the network sans recommendation can also be non-stationary.

%Currently, the primary approach for studying the feedback dynamics between recommendation and network has been in simulation; with the general goal of explaining phenomenons like polarization, distributional inequalities etc by finding salient features of networks and of recommendations that contribute to these effects.

% =========================================   NEW   ==========================================
Link recommendation algorithms such as Facebook's "People You May Know", Twitter's "Who to Follow" and LinkedIn's "Recommended for You" have an ever-increasing influence on the evolution of social networks, with some accounts crediting over 50\% of links to algorithmic recommendations \citep{PeopleYouMayKnow}.
This can cause downstream effects on information flow, opinion dynamics, and resource allocation. For instance, recommendations can be a polarizing force increasing network segregation, which in turn may re-inforce opinion echo-chambers \citep{cinus2022effect} or restrict access to information and resources for less connected communities \citep{dandekar_biased_2013, auletta_impact_2022, santos_link_2021}. At the same time, recommendation systems can also surface “long-range” connections between nodes that would not otherwise be exposed to each other, and thus, increase network integration by promoting and maintaining diverse links \citep{rajkumar_causal_2022, dangelo_recommending_2019}. Given the ubiquity of algorithmic recommendations on social media, studying their impacts is seeing increased academic and regulatory interest. However, such studies are challenging for a variety of reasons, such as lack of normative framing \citep{daly2010network} and limited access to large-scale platforms. In this work, we explore a more foundational evaluation challenges stemming from the fact that social networks have underlying dynamics that interfere with algorithmic recommendations.
%One challenge is that link recommendations lack a normative framing for what counts as desirable outcomes \citep{daly2010network}. Normative issues aside, 
% Normative issues aside, it is also challenging to assess the impacts of link recommendation algorithms on well-defined structural properties of networks, such as clustering coefficient, homophily and statistics of degree distributions. 

Existing real-world evaluations of link recommendation algorithms rely on A/B tests (experimental) or longitudinal data (observational). However, both experimental and observational evaluations can yield misleading conclusions. The validity of A/B tests relies on the Stable Unit Treatment Value Assumption (SUTVA) \citep{imbens_causal_2015, gui2015network}, which is violated in  case of network interference. Observational studies may fail to assess causal impacts as longitudinal evaluation lacks counterfactual measurements on what the network evolution would have been without algorithmic recommendations. These challenges have motivated the use of simulation-based evaluations of link recommendations. In simulation studies, one can make explicit network modeling assumptions and then evaluate the impact of recommenders. Despite a growing number of works in this space \citep{stoica_algorithmic_2018, cinus2022effect, fabbri_effect_2020, fabbri2022exposure, ferrara2022link, santos_link_2021}, 
existing simulation-based evaluation falls short of providing insights into the mechanism through which recommendations impact social networks. Existing simulation studies primarily consider static networks, and thus do not take into account feedback loops between link recommendation and organic network evolution.

% yw / above, can we include one or a few sentences giving a quick summary of what the punchline of Figure 1.  (or we can refer to Figure 1 when the corresponding results are discussed in the intro.) 

% yw / separately, let's include the punchline of Figure 1 as part of the caption. (so that readers who only look at the figure will know what they should take away from the figure.)

In dynamic settings, the main challenge is to measure impacts relative to a "baseline" network. Given this, existing works often measure the effects relative to the initial networks; they are rarely measured relative to a plausible counterfactual based on the natural evolution of the graph without link recommendations. Such evaluations can lead to qualitatively wrong conclusions. For instance, \citet{abebe_effect_2022} shows that triadic closure -- the most common type of friend-of-friend recommendations -- can reduce segregation with respect to the initial network before intervention. However, triadic closure can, at the same time, increase network segregation in relative terms with respect to a natural evolution dynamic which assumes the addition of random edges. 


In this work, we initiate a study to explore \emph{dynamic} impacts of link recommendations through simulations. We find that link recommendations can have surprising delayed and indirect effects on the structural properties of networks. 
%Figure \ref{fig:sketch_effects} sketches where these effects come up in link recommendation evaluation. \pz{The transition from the figure to "For instance" here feels not very natural.}
%Further, evaluations that do not consider the feedback loop between recommendations and organic growth cannot capture the impacts of recommendations to the full extent.
For instance, the effects of friend-of-friend recommendations can vary in the short-term and long-term: it may alleviate degree inequality   in the short term, but increase degree inequality in the long term;  the perceived increase in degree inequality can even be significantly more pronounced after the recommendations are discontinued. Moreover, we demonstrate that the effects of link recommendations can persist in the network even after recommendations have been discontinued. This phenomenon is due to the fact that recommendations impact the network in two ways: directly through the creation of algorithmic edges, and indirectly by influencing the natural growth dynamics. Indirect effects amplify the direct effects, contributing to the persistent impact of link recommendation algorithms. A stylized illustration of the indirect and delayed effects can be found in Figure~\ref{fig:sketch_effects}.

% typically in A/B studies on networks we claim that network interference plays an important role -- however in this setup we can investigate the additional biases due to dynamics

\begin{figure}
    \centering
    \includegraphics[trim={11cm 2cm 6cm 5cm}, clip, width=0.85\linewidth]{figures/sketch_updated.pdf}
    \caption{Delayed and indirect effects: The image displays two counterfactual evolutions of the network. The solid blue line represents the trajectory for an intervention interval $[\underline{t}, \overline{t}]$ in which the full network receives algorithmic recommendations. The dashed black trajectory corresponds to a counterfactual network evolution without recommendations. The total causal effect of recommendation at time $t$; $\text{Effect}_t$ can be computed as the difference between the counterfactual trajectories at time $t$ (solid blue line and dashed black line). The delayed effect at some time $T\geq \overline{t}$ is the difference:  $\text{Effect}_T - \text{Effect}_{\overline{t}}$. 
    %The temporal evolution of the network in the presence of link recommendations is affected \emph{directly} by the addition of algorithmic edges; but also \emph{indirectly} as the addition of algorithmic edges biases the natural growth dynamics. 
 The solid purple trajectory illustrates the counterfactual evolution of the network in which the indirect effects are removed. The difference between the purple and dashed curve captures the direct effects and the difference between the blue and purple line captures the indirect effects.} 
    \label{fig:sketch_effects}
\end{figure}

%A limitation of current simulation evaluations is that they do not consider dynamics of natural growth. For instance an increase in homophily with respect to the initial network could actually correspond to a relative decrease in homophily with respect to the network which evolves organically. Furthermore there is a feedback loop between algorithmic growth and natural evolution. By essentially, assuming a static counter-factual current simulation studies are not accounting for indirect effects in which the presence of algorithmic edges mediates the creation of organic links. Our work addresses this limitation by explicitly modeling natural growth and measuring recommendation impacts with respect to a valid counterfactual. Moreover we measure indirect effects in our experimental setup.


% yw / below maybe it is worth clarifying the relationship between evaluation and experimental findings. or shall we highlight the findings first.

\paragraph{Contributions.}
%The contributions of our work are concerned with improving methodology of simulation based analyses by proposing a flexible modeling approach which incorporates many desirable properties of real networks and most importantly distinguish between organic/natural growth/evolution and algorithmically aided one (through recs).
%This work addresses methodological issues of simulation based evaluation by explicitly modeling natural  evolution of the dynamics. 
%We model recommendations as interventions in the network and argue that their impact needs to be assessed counterfactually with respect to the natural growth. Furthermore we study the feedback loop between algorithmic growth and natural growth, and distinguish between direct and indirect effects. 
We enumerate our contribution along modeling, evaluation, and experimental findings.
\begin{itemize}
\item\emph{Modeling:} We propose a dynamic network formation model which extends upon the Jackson-Rogers model~\citep{jackson2007meeting} to incorporate algorithmic recommendations. Unlike the classic model, our dynamic model includes not only the  original two phases ---dubbed "meeting strangers" and "meeting friends"---but also a third phase of ``meeting recommendations.'' Additionally, we consider latent node representations, enabling us to model community structure and node activity levels in a flexible manner.
\item \emph{Evaluation:} %We follow the current literature in measuring the impact of recommendation via changes to structural properties of the network such as clustering coefficient, group homophily and Gini coefficient of the degree distribution. 
We monitor the progression of network metrics over different intervention windows. We compare the immediate impacts observed during intervention with the delayed impacts observed after the intervention has ended. Further, we measure the indirect impacts that recommendations have on network properties. To tease out the indirect impact, we compare the observed network evolution with a counterfactual baseline network that discounts the influence of recommendations on natural growth.
% \pz{this counterfactual network is a bit hard to understand when reading this}
\item \emph{Experimental findings:} 
Our study reveals diverse qualitative patterns for delayed and indirect effects; we find that different durations of the intervention and/or different times of measurement can lead to drastically different conclusions about \emph{"How do recommendations impact networks?."} Furthermore, we find that indirect effects can be substantial; they can significantly amplify the impact of recommendations and persist even after the intervention has ended.
\end{itemize}
% \subsubsection*{Organization}
% The rest of the paper is structured as follows: In Section  \ref{sec:related} we briefly survey related literature. In Section \ref{subsec:model} we present our dynamic evolution model discuss its salient properties. Further in \ref{subsec:evaluation} we discuss evaluation. In section \ref{sec:experiments} we present our experimental setup and findings. We conclude with discussion and future work in Sections \ref{sec:discussion}-\ref{sec:future}

