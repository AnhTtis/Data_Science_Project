
\section{Counterfactual evaluation}\label{sec:experiments}
We simulate counterfactual scenarios by applying different recommendation interventions for various time periods. Although accessing counterfactuals in reality is infeasible, these simulations offer insights into dynamic and temporal phenomena and provide grounded foundation for subsequent sections.
\subsection{Setup}
The results in sections \ref{subsec:delayed} and \ref{subsec:indirect} consider a simple experimental setup to illustrate delayed and indirect effects of \fof{} and \latent{}\footnote{For \latent{} recommendations we use softmax temperature $\beta = 10$} recommenders. In the baseline setup we consider two equally sized communities. We sample latent embeddings for the nodes independently from $\mathcal{N}(\mu_c, \sigma I)$, with $\mu_1 = [0,1]$ and $\mu_2 = [1, 0]$; for both communities the variance of the embeddings is set to $\sigma = 0.05$. We sample 50 nodes in each group and initialize edges by connecting pairs of nodes $i-j$ with probability proportional to the inner product of their embeddings. Then, for each node in the network, we consider their neighbors at distance 2 and connect to them with probability $p_1=0.05$. This results in a slightly homophilic initial network with homophily $h1=h2\approx 0.1$.  Upon initialization, at each time step $5$, new nodes arrive. For natural growth we consider $N_s = N_f = 100$ and the connection probability of connecting to a candidate node in the "Meeting Friends" phase: $p_2= 0.05$. Further, we assume a behavioral model where nodes accept recommended edges with a constant probability, $0.5$.
%We consider a strictly growing network where at each time step, $5$ new nodes arrive and neither nodes nor edges are removed. 
At each time step we measure structural metrics of the network such as clustering coefficient, Gini coefficient and homophily. We repeat each trajectory for 5 random seeds and report average results along with confidence bands.
\footnote{Code for reproducing experimental results can be found \href{https://anonymous.4open.science/r/delayed-indirect-CB60/README.md}{here}.}

In section \ref{sec:group_structure} we vary these assumptions by considering majority-minority structure, differentiated homophily, and within group heterogeneity. Finally in Appendix \ref{app:rewire_acc} and \ref{app:rec_variants}, we consider additional behavioral assumptions, different variants of the underlying dynamics and modifications to the recommendation algorithms. 


%\mc{In Appendix \ref{app} we demonstrate that our finding are robust over varying choices of parameters. Further we perform ablation studies to further understand the impact of each component of our model}
%For our experimentation, we average our results over multiple different seeds to marginalize any randomization effects. We chose to run our simulations in 2-dim with two separate groups centered at $(1, 0)$ and $(0, 1)$, with initial group size of 50. For our probability function we chose $a = 2, b = 5$ and for our hazard function we chose $c = 0.0001, d = 0.08$; these values were determined experimentally.

% -some metrics are stationary over time ie. result of algorithmic choice
% -does the network forget the effect of recommendation (run natural then recommend then rerun natural)?
% Stationary definition for our purposes
% We note that most of the initialization settings did not matter 

\subsection{Delayed effects}\label{subsec:delayed}
%In this section we characterize the delayed impacts and the indirect effects for the \fof{} and \latent{} recommenders. 
\begin{figure}
    \centering
    \includegraphics[width=0.9\linewidth]{figures/delayed_effects.pdf}
    \caption{Trajectory of homophily, global clustering and Gini coefficient for \latent{} and \fof{} recommendations applied over varying intervention intervals. The solid black line represents the evolution of the metric under natural growth dynamics. The blue lines represent trajectories for the \latent{} recommender whereas the orange lines correspond to \fof.  The shaded area corresponds to the 95\% confidence intervals calculated over 5 independent trajectories.}
    \label{fig:gini_delayed}
\end{figure}

We find that affinity and neighborhood based recommenders have opposite long term effects on homophily and clustering coefficients. \latent{} leads to increases in homophily as well as global clustering whereas \fof{} recommendations decrease them. Both recommenders have diminishing delayed impacts with respect to homophily and clustering as upon end of the intervention the trajectory of the metric regresses to the counterfactual natural growth trajectory. Meanwhile, effects with respect to the Gini coefficient are qualitatively different. \latent{} recommendations lead to increases in degree inequality both in the short and in the long term. Conversely, \fof{} recommendations decrease the Gini coefficient in the short term but increase it in the long term. Figure \ref{fig:gini_delayed} illustrates these findings.

\paragraph{Clustering and homophily.}
In the case of \latent{} recommendations, if nodes $i$, $j$, and $k$ are similar in embedding space, it is likely they are from the same community and all three pairs of edges have been proposed as recommendations, leading to a strong bias for closing triangles within the community. This results in increased homophily and clustering, especially with the behavioral assumption that nodes accept links based on embedding similarity (see Fig. \ref{fig:bevarior_variants_c, fig:bevarior_variants_d}). The bias is further exacerbated for high-temperature $\beta$. Conversely, with a low $\beta$ value, which corresponds to a more stochastic recommender, it is less likely that all three edges of a given triplet will close, thus slightly lowering the homophily value compared to the high-$\beta$ case.
%\pz{This sentence is incomplete} -- fixed

% In the natural growth dynamic, phase 1 is biased in favor of within community nodes whereas phase 2 is less biased. 
The \fof{} recommender intervention mimics the "Meeting Friends" phase of natural growth, which has a bias towards forming cross-community links, resulting in decreased homophily. Decreased clustering occurs because random \fof{} recommenders lack the bias of connecting nodes with a large common neighborhood, unlike neighborhood-based models such as Adamic-Adar \citep{adamic2003friends}, which favor links with nodes with a large number of common neighborhoods and thus increase clustering (see Fig. \ref{fig:rec_variants}).

%Both recommender systems have diminishing impacts over time. After the intervention ends, the metrics trend back towards the baseline of natural growth, but eventually asymptote away from it. . 


\paragraph{Gini coefficient.}
The Gini coefficient for recommendations shows amplifying delayed effects. The natural growth trend exhibits a slight increase in inequality over time. \latent{} recommendations exacerbate the wealth gap between popular and unpopular nodes through a bias towards nodes with high embedding norms, creating the "rich-get-richer" effect. \fof{} recommendations initially reduce degree inequality as they are less biased towards popular nodes.
Surprisingly, once recommendations stop, the Gini coefficient increases dramatically for both recommenders, particularly for \fof{}. This is due to a "relative-rich-minority" effect caused by differences in edge density between 'older' and 'newer' nodes. Over time, as natural growth is continuing to take place, the set of "rich" nodes becomes relatively smaller, resulting in higher inequality as measured by the Gini coefficient. If a rewiring behavioral model is assumed  (see Fig. \ref{fig:bevarior_variants_b, fig:bevarior_variants_d}), where nodes receiving recommendations do not change their degree, the delayed impacts diminish, similar to the case of clustering and homophily.


% In the case of the clustering coefficient, we instead compare \adad{} and \fof{}. They are both neighborhood-based recommendations that consider recommendation candidate nodes at distance 2. While \fof{} recommendations have no preference among the neighbors of neighbors, \adad{} chose among the neighbors based on the Adamic-Adar index \citep{adamic2003friends}. For a pair of nodes $i,j$ the index is defined as the sum of the inverse logarithmic degree of their common-neighbors: $A(i,j) = \sum_{k\in \mathcal{N}(i)\cup \mathcal{N}(j)}\frac{1}{\log d_i}$. Despite a lot of similarities between these recommendation algorithms, we see that they have starkly different impacts on the clustering of the graph. Figure \ref{fig:clustering_adad_fof} shows that under natural dynamics the natural growth process keeps the clustering coefficient nearly constant. \fof{} recommendations first decrease the clustering and then increase it slightly. The initial decrease can be attributed to the fact that, every time a new edge is added, it closes a number of triangles and opens a number of wedges. In the beginning when the network is not as dense, the addition of a random triadic closure on average opens more wedges than it closes triangles, thus leading to a decrease in clustering.
% % yw / i can't parse the sentence above
% When the network becomes denser due to the addition of algorithmic edges, triadic closures are now likely to close more triangles than open wedges leading to increases in the clustering coefficient. On the other hand, recommendations based on Adamic Adar dramatically increase the clustering coefficient. Here we have two key observations: the dramatic increase and relatively low persistence of the effect. The reason \adad increases the clustering coefficient is that it has a propensity to choose pairs of nodes that have a large common neighborhood.  these recs close more triangle than they open wedges. These effects are not as persistent because ...
% % yw / i can't parse the sentence above either...
% % sl / add something here?




% \begin{figure}
%     \centering
%     \includegraphics[width=\linewidth]{WWW/figures/temporal_dynamics.png}
%     \caption{Comparison of temporal dynamics}
%     \label{fig:my_label}
% \end{figure}

\subsection{Indirect effects}\label{subsec:indirect}
The "Meeting Friends" phase occurs in feedback loop with algorithmic recommendations, leading to the mediation phenomenon illustrated in Figure \ref{fig:direct_indirect}. We find that mediated edges are common, long-lasting and biased. For instance, in the case of \latent{} recommendations, the mediated edges are more likely to connect nodes from the same community, resulting in significant effects on homophily. 
%The bias in mediated edges leads to significant effects on homophily measures. \pz{The sentence after "for instance
 %is hard to read}\mc{this is weird I remember writing a full sentence about this ...}

%We further argue that the observed strong indirect effects on metrics of network integration is due to the relative bias in the bichromaticism of mediated edges.
%\pz{This sentence is a little hard to read. But I haven't come up with a better way to say it} \mc{I think this sentence was written a bit to hastingly last time; I like the overall idea of having the first paragraph be a summary paragraph. I think the way to go is to split this convoluted sentence into 2-3 simpler statements}

%For these metrics, we also find that the indirect effects for different recommenders are very different. 
% Among other robust findings is that the strength of indirect effects is a relatively small component of the immediate effect but a large component of the delayed \paragraph{}
\begin{figure}
    \centering
    \includegraphics[width = 0.9\linewidth]{figures/mediation_constant_False_False.pdf}
    \caption{Indirect effects: Solid black line corresponds to natural growth dynamics. The blue and orange lines represent trajectories for \latent{} and \fof{} recommenders, respectively. Shaded areas corresponds to 95\% confidence bands.  Left plot illustrates the prevalence of algorithmically mediated edges in the "Meeting Friends" phase of the natural growth dynamics. Dashed and dotted lines correspond to various intervention intervals. Center plot illustrates the bias in the mediated edges relative to unmediated ones. Solid lines correspond to fraction of bicromatic edges among unmediated edges created in the "Meeting Friends" phase; the dashed line records the fraction of bicromatic mediated edges. Right plot shows the trajectory of homophily metric. Dotted lines correspond to the trajectory of the metric for the unmediated counterfactual evolution of the network.
    %The first row shows the prevalence, persistance and bias of mediated edges. Solid black line corresponds to natural growth dynamics. The blue and orange lines represent trajectories for \latent{} and \fof{}, respectively. Left plot shows the fraction of mediated edges among the total edges created in the "Meeting Friends" phase.  Lighter lines correspond to shorter while darker lines correspond to longer intervention windows. Right plot shows the fraction of bichromatic edges among mediated (dashed line) and unmediated edges (solid line) when the intervention window is 50-200. The second row shows indirect effects of \latent{} on homophily. Left plot shows the trajectories of homophily. The solid black line is the evolution under natural growth dynamics. The solid blue lines represent trajectories under algorithmic growth, while the dashed lines correspond to the unmediated counterfactual. Right plot shows the relative magnitude of indirect effects over time. Lighter lines correspond to shorter while darker lines correspond to longer interventions. \pz{need update}
    }
    \label{fig:prevalence_persistance_bias}
\end{figure}
\paragraph{Prevalence, persistence and bias of mediated edges.}
 Mediated edges play a substantial role in the "Meeting Friends" stage of natural growth dynamics. Figure \ref{fig:prevalence_persistance_bias} shows that the proportion of mediated edges increases with the duration of the intervention period and remains constant even after recommendations have stopped for both \latent{} and \fof{} recommenders. The bias of mediated edges in terms of bichromaticism is distinct for each recommender, relative to the proportion of bichromatic edges under natural growth.
For instance, about a third of unmediated "Meeting Friends" edges are bichromatic for both recommenders. However, the proportion of bichromatic edges among mediated edges is lower for \latent{} and higher for \fof{}. Our findings about homophily in Section \ref{subsec:delayed} suggest that algorithmic edges are biased towards monochromatic for \latent{} and bichromatic for \fof{}. This experiment supports the hypothesis that recommendations have a compounding effect by inducing biases in the natural network evolution.
 %the fraction of mediated edges among the total edges created during phase 2 keep increasing from 0\% to around 30\% along with the recommendation, which shows that the prevalence of mediated edges, and thus the indirect effect, gets increasingly pronounced within the intervention window. Even after we stop the intervention, the fraction still keeps at about the same level or increases slightly from when the intervention is stopped. It shows that in the long run, the indirect effect is still significant and prevalent, so it is persistent over time even after recommendation is stopped. 
 
 \paragraph{Counterfactual measurements of indirect effects.}

 
%We find that recommendations have indirect effects on some metrics more than others, specifically the metrics of network integration. While the fractions of mediated edges among the total edges during phase 2 are very similar for \latent{} and \fof{}, the two recommendations have the opposite indirect effects on the network integration. from the same group, which increases homophily and decreases the fraction of bichromatic edges, while \fof{} is more likely to recommend nodes from the other group, which decreases homophily and increases the fraction of bichromatic edges. While the two recommenders have direct effects on the structural properties of the network, they also indirectly impact the network evolution through the properties of the mediated edges.
% \begin{figure}
%     \centering
%     \includegraphics[width = 0.7\linewidth]{WWW/figures/indirect_homophily.png}
%     \caption{Indirect effects of \latent{} on homophily. Left plot shows the trajectories of homophily. The solid black line is the evolution under natural growth dynamics. The solid blue lines represent trajectories under algorithmic growth, while the dashed lines correspond to the unmediated counterfactual. Right plot shows the relative magnitude of indirect effects over time. Lighter lines correspond to shorter while darker lines correspond to longer interventions.}
%     \label{fig:homophily_indirect}
% \end{figure}
To measure the indirect effects on structural metrics, we apply the counterfactural procedure from Section \ref{subsec:indirect_methodology} and isolate the direct effects.  We find that without mediation, structural metrics trend faster to their natural evolution. Figure \ref{fig:prevalence_persistance_bias} shows the comparison of homophily under natural growth, algorithmic growth, and unmediated algorithmic growth. The difference between the latter two quantifies the magnitude of indirect effects. Additionally, indirect effects grow relatively stronger over time, even after recommendations end, highlighting the persistence of indirect impacts.

% \pz{With the updated figure, we can hardly see the indirect effects increase over time, or the ratio of indirect to total effects}

% \mc{here I decided to de-emphasize that, because I though that 4 figures would not fit properly -- but they totally fit -- so I think we'll bring it back}


%The total effects are measured by the difference between the overall effects and the natural growth dynamics, and the indirect effects are measured by subtracting the direct effects from the overall effects. On the left side of Figure \ref{fig:homophily_indirect}, homophily increases because of the impact of \latent{}. Furthermore, homophily under total effects is larger than under direct effects only, which indicates that indirect effects amplify the homophilous effects brought by \latent{}. Also, even after the recommendation is stopped, the trajectory under direct effect has a higher speed of converging with the trajectory under natural growth. However, with indirect effects, the homophilous effect brought by the recommendation stays longer and more persistent. The right side of Figure \ref{fig:homophily_indirect} shows that the fraction of indirect effects in the total effects become larger along with the intervention. When the intervention is stopped, its fraction becomes even larger.




% This figure also illustrates that our 'de-mediation'  counterfactual is legit

% Here we want to make 3 points: 1st: effects persist through indirect edges -- (smth to show that they are a significant portion of the phase 2 edges)

% 2nd. point is that recommendations have indirect effects on some metrics more than others. And we have not found evidence in our experiments that the strength of indirect effects is as dependent on the choice of recommender.

% 3rd. point is the indirect effects shine once the recommendations are stopped. Thus proving amplification due to indirect effects.


\subsection{Impact of group structure}\label{sec:group_structure}
Previous studies have highlighted the unequal impact of recommendations on different communities, especially when they are divided into majority and minority groups and display varying levels of homophily \citep{stoica_algorithmic_2018, ferrara2022link, fabbri_effect_2020}. Here, we examine the effects of differential homophily between majority and minority groups as well as within-community heterogeneity.
%Existing works have emphasized the disparate impacts that recommendations have on communities, particularly in cases when they are separated into minority and majority groups as well as when they display differentiated homophily (one group is more homophilous than the other)\citep{stoica_algorithmic_2018, ferrara2022link, fabbri_effect_2020}. Here we consider several prototypical networks for which we investigate the role of differential homophily between 
majority and minority group and the role of within-community heterogeneity. %\pz{For each setting below, should we be more specific about the experiment settings (i.e. specific values of mu and sigma for each group)? Cuz we are very specific about them in Section 4.1 Setup}  --> will add in figure caption
 
 \paragraph{Community heterogeneity.}
We examine the impacts of homogeneity and heterogeneity in the latent representation of nodes on the results of link recommendation. We model within-group heterogeneity by varying the variance of the latent embeddings. In the heterogeneous setting, we set the variance of the embedding distribution to $\sigma^2 = 0.1$, and to $\sigma^2 = 0.01$ in the homogeneous setting; in the extreme case when the variance is 0 this recovers the JR variant of \citep{abebe_effect_2022}. 
%The key difference between homogeneous and heterogeneous settings is how recommendations affect the clustering of the network for the \latent{} recommender. 
Figure \ref{fig:homo_clustering} shows that when there is high within-group heterogeneity, \latent{} recommendations greatly increase the global clustering coefficient, while  for high within-group homogeneity, \latent{} recommenders have the opposite effect, reducing global clustering. This unexpected phenomenon is due to \latent{} recommendations favoring nodes with high embedding norms. For any existing edge $i-j$, there is a disproportionately high chance that both $i$ and $j$ will receive recommendations concentrated on a small subset of large-normed nodes, creating closed triangles. In the homogeneous setting, such 'collisions' are less likely to occur, as the probability of two nodes independently being recommended with the same node is lower. These findings highlight the importance of investigating not only between-group differences but also within-group differences.
 \begin{figure}
     \centering
     \includegraphics[width = 0.9\linewidth]{figures/heterogeneity_effects_False_False_True.pdf}
     \caption{Effects of within-group heterogeity. Black trajectory  corresponds to the natural evolution of the global clustering metric. Blue and orange line indicate the trajectory for \latent{} and \fof{} recommender. Left plot: two heterogenous groups. Center plot: two homogenous group. Right plot: one homogenous and one heterogeneous group. }
     \label{fig:homo_clustering}
 \end{figure}
 %Latent recommender ultimately makes global clustering higher than natural growth, but the patterns for heterophilous and homophilous groups are very different. FoF causes global clustering to be higher for the highly heterophilous group, but this effect is reversed or not apparent in the highly homophilous group. Surprisingly the homogeneity and bifrac are not very different between the two settings. We see different directions of intervention effects for latent-based and FoF, but as we discovered, it is related to densification.

 \paragraph{Homophilic minority and heterophilic majority.}
In previous studies, it has been observed that homophilic minority groups receive an excessive amount of exposure from recommendations, leading to increased disparities in homophily between groups. We simulate the network evolution for a majority fraction of $60\%$, $\mu_1 = [0,1]$ and $\mu_2 = [1.2, 1]$ and find that our results support this conclusion for \latent{} recommendations. In contrast, Figure \ref{fig:minority_majority} shows that while \fof{} increases homophily for the majority, it decreases it significantly for the minority. The \fof{} recommender amplifies the visibility of the majority more than that of the minority. This phenomenon can be explained by the heterophilic nature of the majority group, where most of its direct connections are with minority nodes. Though the minority group is homophilic, there is a larger probability that a majority node $i$ is recommended with majority nodes at distance 2, because $i$'s neighbors' neighbors that are in the minority group are more likely to also be direct neighbors of $i$.
% This results in a larger probability of finding majority nodes at distance 2 from a given node $i$, despite having direct minority neighbors and a homophilic minority group. This is because $i$'s neighbors' neighbors that are in the minority are more likely to be direct neighbors of $i$. 
%\mc{i feel like this paragraph is a bit too convoluted} \pz{I modified the last sentences a little bit, but maybe it becomes more wordy...}
\begin{figure}
    \centering
    \includegraphics[width = \linewidth]{figures/minority_majority_effects_majority_heterophily_constant_False_False_True.pdf}
    \caption{Effect of homophily: Trajectory of average degree and community homophily for \latent{} and \fof{} recommenders for the heterophilic majority and homophilic minority. }
    \label{fig:minority_majority}
\end{figure}

% \paragraph{Behavioral model}
% \paragraph{Densification hypothesis}
% With edge removal, the two recommenders have about the same average degree. However, the latent recommender generates a much larger degree variance. This difference is even larger for heterophilous groups. 
% Note: with edge removal, we no longer see the delayed effect for gini, as gini immediately increases with latent intervention, but gini for FoF still has some weird patterns

% With edge removal, we no longer see the delayed effect for gini, as gini immediately increases with latent intervention. But under FoF, it looks like intervention has almost no impact on gini. The pattern of global clustering is also very different between latent and FoF recommenders. 

% \paragraph{Role of dying edges}
% No delayed effects

% Are indirect effects attenuating or amplifying effects of rec?
% \begin{figure}
% \includegraphics[width=0.5\linewidth]{WWW/figures/indirect_nat_growth.png}
% \caption{The orange and yellow line are the edges added due to natural growth phase 2 (Meeting Friends). As we can see more than a quarter of the edges added by natural growth are mediated by our algorithms. This shows that indirect effects are quite substantial. Moreover at time t=100 the intervention stops however the number of mediated edges continue to grow linearly. Does this make sense??}
% \end{figure} \label{fig:indirect_p2}

% \begin{figure*}
%     \centering
%     \includegraphics[width=0.7\textwidth]{WWW/figures/indirect_effects_full.png}
%     \caption{Caption}
%     \label{fig:my_label}
% \end{figure*}

\subsection{Evaluation Biases.}
As we do not have access to the full range of counterfactual measurements, we often resort to either longitudinal or A/B evaluations. However, these methods have their limitations and may fail to accurately capture the impacts of recommendations. In this section, we explore the potential biases and limitations of these evaluation procedures.

 \paragraph{Longitudinal evaluation.}
Simulating a longitudinal evaluation mimics an observational study. In this setting, a single trajectory is observed. The estimated total effect of link recommendations is the temporal difference between the value of the metric before and after the intervention. If the underlying dynamics for a metric are stationary, as is the case for homophily in Figure \ref{fig:gini_delayed}, then a longitudinal evaluation is unbiased. Conversely, when under natural dynamics a metric is non-stationary, a naive comparison between the initial network at time $\underline{t}$ and the network at time $T$ could yield qualitatively misleading estimates. For instance, in Figure \ref{fig:gini_delayed}, when measuring the impact of \fof{} on clustering coefficient, a longitudinal measurement would compare the solid orange trajectory between time $\underline{t}=50$ and $T=400$, thus over-estimating about double the size of the true effects. Furthermore, the naive observational measurement for \latent{} would incorrectly suggest a negative effect on network clustering, when in reality it has a positive impact compared to the natural evolution. 

%To avoid these biases, researchers can develop models that can estimate the evolution of the metric under natural dynamics.

  %Note that even though \latent{} has a positive effect on clustering compared with the natural growth dynamic, a naivethat recommendations decrease clustering. This highlights the importance of identifying the proper counterfactual baseline. \pz{should we say a naive comparison is between the network at t=50 when the intervention starts, and the network at t=100 or t=150} \mc{Really good point: I was actually thinking whether we should plot the difference between true effect and 'naive' effect; as a means of illustration.} 
%Under the natural growth dynamic, the homophily value increases slightly over time, whereas global clustering coefficient first decreases then asymptotes to an equilibrium of $\approx 0.17$. 
 \paragraph{A/B evaluation.}
A/B evaluations in dynamic networks become complex due to changes of network structure over time caused by natural dynamics and algorithmic interventions. We simulate A/B tests aimed at estimating causal effects of \latent{} and \fof{} link recommendations. We perform random treatment ($p=0.5$) on node assignment for clustering and Gini coefficient and community-based treatment assignment for measuring homophily. 
%For each metric to compute naive estimators that simply aggregate over the nodes in each group as well as more sophisticated estimators that aim to reduce network interference by discounting recommended edges for estimates of the control value and double counting edges for estimates of the treatment value. 
The estimates in Figure~\ref{fig:ab} are adjusted for network interference. Detailed computation of the adjusted estimates as well as the corresponding plots for the naive cases can be found in Appendix \ref{app:ab}. 
The quality of A/B estimates varies based on the metric and intervention. For example, A/B tests overestimate the impact on homophily for \latent{} but underestimate it for \fof{}. The estimation for clustering is accurate for \latent{} but greatly underestimated for \fof{}. For the Gini coefficient, both \latent{} and \fof{} interventions underestimate the metric for the treatment group while accurately estimating the metric's evolution under natural growth for the control group. Accross most settings, the quality of the metric deteriorates over time, further supporting the existence of dynamic effects that compound network interference effects.


%One approach to addressing this challenge is to use causal inference methods that control for the confounding effects of network evolution, allowing for a more accurate assessment of the impact of the recommendation algorithm.

\begin{figure}
    \centering
    \begin{subfigure}[b]{\linewidth}
    \centering
    \includegraphics[width=\linewidth]{figures/ab_conclusion_constant_False_False_embedding_True.pdf}
    \caption{\latent{} recommender}
    \label{fig:ab_latent}
    \end{subfigure}

    \begin{subfigure}[b]{\linewidth}
    \centering
    \includegraphics[width=\linewidth]{figures/ab_conclusion_constant_False_False_random_fof_True.pdf}
    \caption{\fof{} recommender}
    \label{fig:ab_fof}
    \end{subfigure}
    \caption{A/B Evaluations: Solid lines correspond to ground truth counterfactual evaluation of homophily, clustering and Gini coefficient. Dashed lines correspond to trajectories estimated from running an A/B test.}
    \label{fig:ab}
\end{figure}