\section{Related Work}\label{sec:related}
%We review recent works that investigate the impacts that link recommendations have on social networks.
\paragraph{Simulation studies.}
Simulation studies typically analyze the impact of link recommendations on static networks. Some works focus on a single round and examine biases observed in proposed recommendations, finding that homophily --- the preference for within-group links --- leads to exposure bias toward the more homophilous group, even if it's a minority \citep{karimi2018homophily, fabbri_effect_2020, espin2022inequality, stoica_algorithmic_2018}.  Similarly, \citep{fabbri2022exposure, ferrara2022link} find that homophily and node degree are strong indicators of which nodes will receive disproportionate visibility. Other works make explicit behavioral assumptions on how nodes accept link recommendations and consider cumulative effects of link recommendations over multiple rounds. These studies reveal algorithmic amplification of biases via "rich get richer" effects and increases in observed homophily over time \citep{fabbri2022exposure, ferrara2022link}. A shortcoming of existing simulation based evaluations is that they implicitly assume that the addition of algorithmic edges is the only change in the network. Conversely, in our work we make explicit modelling assumptions about the underlying network dynamics. This evaluation setup allows us to measure the impact of link recommendations with respect to counterfactual natural evolution. Furthermore, by emphasizing underlying temporal dynamics we can pose more subtle evaluation questions such as: "How does the effect of algorithmic intervention fade over time once recommendations are stopped?" or "How do algorithmic recommendations bias the underlying network growth?". 
%\mc{add this work too: https://blog.ml.cmu.edu/2022/09/02/long-term-dynamics-of-fairness-intervention-in-connection-recommender-systems/} \mc{ferarra: The analysis of visibility measures visibility changes with respect to network prior to rec. Given that more homophilic minorities have a larger than statistical parity share of the high visibility nodes, it can be said that rec systems actually are a correcting force -- undoing the biases of the natural network.}

% A number of simulation studies analyze the impact of link recommendations on social networks. Typically these works employ static networks either synthethic or derived from real network data. The most common modelling assumption is that all or a part of the nodes receive recommendations. Some works looks only at one round of recommendations and investigate exposure bias. For instance some works have concluded that homophily--the property that nodes have a preference for in-group links- leads to exposure bias in favor of the more homophilous group\citep{karimi2018homophily, fabbri_effect_2020, espin2022inequality}, even if the homophilous group is a minority \citep{stoica_algorithmic_2018}. Further these works show the amplification of biases via rich get richer effects Similarly, \citep{fabbri2022exposure, ferrara2022link} find that homophily and node degree are strong indicators of which nodes will receive disproportionate visibility for most link recommendation algorithms. However, investigating disparities with respect to either a single or a few recommendation steps does not give a complete picture of the longer-term consequences of algorithmic interventions in networks. In contrast, our work takes a longer-term perspective and highlights the importance of dynamical modeling and measurement in the evaluation of link recommendations' impacts.
% \mc{ Previous simulation based evaluations of link recommendations consider static networks which lack underlying natural network dynamics and thus cannot give rise to the evaluation phenomena studied here. Our work shows that these phenomena can significantly skew the performance of link recommenders under plausible network evolution models. } \mc{smth like:  We show that these effects are not solely due to algorithmically added edges but also a result of recommendations impacting natural network dynamics. Existing empirical evaluations on social media platforms measure immediate effects when evaluating link recommendation and are not explicitly measuring whether algorithmic interventions have a sustained effect.}
%In general, the recommendation systems discussed display rich-get-richer effects that benefit high-centrality nodes. Such analyses implicate link recommendation algorithms in the amplification bias and inequality of networks.
\paragraph{Platform studies.}
There is limited publicly available research evaluating link recommendation algorithms on real social networking platforms. In one experimental study \citep{daly2010network}, several recommendation algorithms were compared on IBM's SocialBlue network and found to reduce group homophily. The study also revealed that friend-of-friend recommendations had the highest rate of acceptance but the lowest level of edge activity. On the other hand, \citep{rajkumar_causal_2022} found that recommending more distant connections or "weak ties" through LinkedIn's "People You May Know" algorithm led to higher transmission of job opportunities.
In observational settings, a longitudinal study comparing the Twitter network before and after the introduction of the "Who To Follow" recommender in 2010 \citep{su2016effect} found that while recommendations increased the number of connections for all users, the highest gains were achieved by the most popular nodes. Conversely, a study comparing links formed naturally and links formed via recommendations on Flickr and Tumblr \citep{aiello_evolution_2017} found that the recommended links were more diverse and less biased towards popular users. So far, existing observational platform studies provide limited understanding of the underlying mechanisms and lack access to counterfactual network evolution. A/B tests, on the other hand, may produce wrong estimates of the effects due to network interference. Our counterfactual simulations allow us to articulate in stylized settings the source of bias in both longitudinal and A/B evaluations.
%and illustrate the increasing interference effects over time. Our simulation-based approach also enables us to model different aspects and run ablation tests to understand the impact of each component on broader dynamics, filling the gap left by longitudinal evaluations.
\paragraph{Theoretical investigations.}
The interaction between homophily and friend-of-friend recommendations was studied theoretically in a number of works. Under choice homophily, which captures the setting when nodes preferentially accept recommendations to within group nodes,
\citep{stoica_algorithmic_2018, asikainen2020cumulative} show that such interventions lead to more exposure gains for the homophilous group which exacerbates observed homophily. In contrast, \citep{abebe_effect_2022} shows that, when the closure of triangles via friend-of-friend recommendations is not biased in favor of in-group edges, recommendations can in fact improve network integration. In our work, we investigate friend-of-friend recommendations and their impacts on network segregation and show that the impacts further depend on the length of intervention as well as the time of measurement.

\paragraph{Delayed algorithmic impacts.} Temporal dynamics associated with algorithmic interventions have been previously studied in the context of fairness in Machine Learning. These works showcase broad settings where algorithmic interventions designed to improve fairness \citep{liu2018delayed, daly2010network, akpinar2022long}, robustness \citep{milli2019social} or diversity \citep{curmei2022towards} in the short term, lead to the opposite effect in the long run. Our work uncovers similar surprising temporal dynamics in the case of link recommendations.


% Link recommendations have been studied for a  long time: structural/neighborhood model and global/affinity based models.

% Utility based recs: \citep{li_utility-based_2016}  Many works have since been proposed to make social rec systems model utility of links and fairness implication \citep{dangelo_recommending_2019}
% \citep{fabbri_exposure_2021} only considers additions of edges whereas \citep{ferrara2022link} aims to control for densification effects via link recommendation re-wireing rather than addition. Both model assume implicitly a stationary counterfactual.
% \citep{stoica2018algorithmic}: Instagram: Observational study: they finding that replacing natural evolution with one that was algorithmically induced. They have natural growth and pure algorithmic growth but not the necessary combination of the two. They find the so called ceiling effect and show the recommendation is biased against the majority group of women on Instagram. They further explain this via the fact that the minority if more homophilous and thus receives a disproportionate share of the recommendations.

% Further works have focused on downstream impacts: <<----- this should go to intro
% Other works focus primarily on understanding the downstream impact of recommendations on opinion dynamic, polarization and radicalization; as well as likelihood of job transmission \citep{rajkumar_causal_2022}

% % sometimes people analyze impact they recs via their downstream impacts
% Dynamics of polarization in networks: \citep{dandekar2013biased}

% Polarization effects of recommendation in online social networks: \citep{santos2021link}

% Opinion dynamics: \citep{auletta_impact_2022, cinus2022effect}