
\section{Discussion and Future work}\label{sec:discussion}
%\mc{add to conclusion that Gini coefficient is a bad metric especially as is not decomposable, meaning that it cannot be calculated for subgroups of the population and then aggregated to obtain the overall coefficient. This makes it difficult to interpret the results and determine the sources of inequality. Finally, the Gini coefficient is highly sensitive to outliers, meaning that a single extreme value can have a large impact on the overall result}
In this study, we explored the dynamic effects of link recommendations on network evolution through simulations. Our proposed extension of the Jackson-Rogers model provides insight into the impact of link recommendations on network structure. Emphasizing the importance of temporal dynamics and measurement timing, our simulations revealed surprising and persistent effects of link recommendations on network structure. 

Using synthetic data and simple network evolution models and recommendation algorithms, we answered "what-if" scenarios in a controlled setting, providing a valuable first step in understanding these effects in real networks. Our results showed that link recommendations can have delayed and indirect impacts on network structure, with long-lasting effects even after recommendations have ceased, which result in significant cascading indirect effects over time. This highlights the need for further research in evaluating link recommendation algorithms in dynamic networks.

Finally, our study sheds light on the potential biases that can arise in evaluating link recommendation algorithms in dynamic networks. We find that evaluating metrics longitudinally or using A/B tests can result in biased estimates when the underlying network dynamics are not stationary. This highlights the need for advanced estimation procedures that consider both network interference and underlying dynamic effects to accurately assess the impacts of link recommendation algorithms.

We identify several opportunities for future research. To improve the validity of our conclusions, one avenue is to validate modeling assumptions against real-world networks. Refining the modeling assumptions, such as allowing for non-recommender-driven edge creation between existing nodes, could better reflect reality and potentially result in greater indirect effects. Another important area of interest is to further develop methods for measuring direct and indirect effects in different network formation models.

A second direction of future work is to study the downstream impacts of link recommendation. Modeling the node embeddings as a dynamic property, influenced by the local neighborhood through biased assimilation \citep{dandekar_biased_2013} or mere exposure effects \citep{curmei2022towards}, could provide insight into how opinion formation on networks is influenced by recommendations. Finally, examining recommendation scenarios where network edges in the social network are formed by content recommendations, rather than user recommendations, which is typical of social media sites like Instagram,  is a promising area for investigation.
\newpage

%There are, however, some limitations regarding our current evaluation scheme. One limitation is that we may be potentially underestimating the impact of indirect effects for at least two reasons. First, we are only counting the 'first-order' indirect effects: we only consider the indirect impacts due to the edges created by link recommendations. However, there are second order indirect effects due to mediated edges, and even higher order effects due to second order effects. Secondly, in our model, natural growth occurs only upon node arrival. In contrast, a model with spontaneous edge creation after the arrival of nodes might be both more realistic as well as exhibit even more indirect effects.

%This finding tells us more about the idiosyncrasy and sensitivity of the Gini coefficient to distributional changes and point to the inadequacy of using this metric for assessing inequality impacts. \yw{I dont understand how this sentence fit into the paragraph above. probably need to smooth things out here.}

% Overall, our results lead to substantially different conclusions from those in the existing literature, especially around the effect of link recommendations on network evolution. \yw{do we give explicitly citation here? or do we just say we illustrate the sensitivity of these conclusions to duration of recommendation and time of measurement?} By developing a simulation framework for dynamic settings, along with counterfactual evaluation methodologies, we provide a novel approach to evaluating link recommendations on dynamic networks.\yw{I guess we no longer claim this as contribution? maybe fix it to say it highlights how existing evaluating paradigms may fall short and how simulation can help etc etc? make it consistent with abstract / intro}
%\propchange{TODO: Add smth about delayed impacts -- either here or in intro (e.g. Lydia's work)}
%Whereas previous works have a 'strawman' de-facto natural growth assumption - namely nothing changes -- or we add random edges -- here we say that even without recommendations this dynamic model is reasonable -- i.e. the growth model is not a post-facto consideration
% Currently, we consider the effects of recommendation systems on between (...).
% Lower timescale
% Removal of death function
% Increase size of graphs by multiple magnitudes
% We would like to expand our network representation to include not just social networks, but also content recommendation. --> content vs user, analysis of diversity/homogeneity
% There are a few interesting directions to explore in the future.
% One direction concerns validating and refining model assumptions. Since our conclusions are specific to the dynamic network model we employ, it is important to assess the validity of the modeling assumptions in real-world network formation processes and 
% the sensitivity of our conclusions to the modeling assumptions; these assessments could further suggest how the modeling assumptions can be refined. Along this line, another important question concerns the external validity of our findings. For example, we observe consistently in our simulations that homophilic majority groups become more homophilic due to triadic closure. In principle, however, group homophily need not only emerge from triadic closure but can evolve from other factors. Whether our findings would generalize to other models of homophily remains an important open question.

% there is more than one way to model group homophily; an important question pertaining to generalization is if our broad findings can be extended to other models.

% The second direction of future work is to assess the influence of preference dynamics. Social networks are often used to model opinion formation, and an interesting avenue for research is to consider the impact of preference shifts in our context. For instance, biased assimilation \citep{liu2018delayed} or mere exposure effects \citep{curmei2022towards} could be integrated into our network model as shifts of the nodes' internal embedding, which could be framed as a function of new links. Another related direction of interest is to study and model the recommendation scenarios where updates to underlying social networks are indirect and mediated by content recommendations. In the case of implicit networks, we can distinguish between network features and content features, which are fed into the recommendation algorithm. Such a scenario could be modeled by generalizing our model with the addition of "content" nodes and "user" nodes. Having developed a flexible framework, we hope that the questions we posed can help future work in developing the appropriate models for simulating, measuring, and evaluating the effects of link recommendations.

% \yw{ It is probably less important but I feel the last paragraph can probably be made more coherent and easier to read. I feel I can't easily parse the last paragraph, though it could be me being too tired to read...}

%Given that the model we consider here is so simple there could be value in working out the theoretical details of it

%\propchange{there are all of these new models trying to make link recommendations fairer -- less biased -- looking to understand their delayed impact -- akin to delayed impacts for FairML ...}

%Since our conclusions are specific to the dynamic network model we employ, it is important to assess the validity of the modeling assumptions in real-world network formation processes

%We identify several avenues for future work. Since our conclusions are specific to the dynamic network model we employ, one such direction is to validate and improve the assumptions of our model with respect to the evolution of real-world networks. This could lead to refinements in the modeling assumptions,  for instance modifying the dynamics to allow for non-recommender driven edge creation between exiting nodes (not only at node arrival) could better reflect reality and potentially exhibit even greater indirect effects. Further developing more genereral ways to measure direct and indirect effects is a matter of considerable interest as our current setup we had a very clean notion of indirect effects through mediation organic edge creation in the "Meeting Friends" phase. The same mechanism may not hold for other network formation models.

%The second direction of future work is to assess the influence of preference dynamics. Social networks are often used to model opinion formation, and an interesting avenue for research is to consider the impact of preference shifts in our context. For instance, biased assimilation \citep{liu2018delayed} or mere exposure effects \citep{curmei2022towards} could be integrated into our network model as shifts of the nodes' internal embedding, which could be framed as a function of new links. Another related direction of interest is to study and model the recommendation scenarios where updates to underlying social networks are indirect and mediated by content recommendations. In the case of implicit networks, we can distinguish between network features and content features, which are fed into the recommendation algorithm. Such a scenario could be modeled by generalizing our model with the addition of "content" nodes and "user" nodes. Having developed a flexible framework, we hope that the questions we posed can help future work in developing the appropriate models for simulating, measuring, and evaluating the effects of link recommendations

%Another direction of future work is to explore the impact of preference dynamics in our model. Since social networks are often used to model opinion formation, it would be valuable to consider the influence of shifts in preferences, such as biased assimilation or mere exposure effects, on the internal embedding of nodes in our network model. Another related area of research is to investigate the role of content recommendations in network evolution. This could be modeled by adding "content" and "user" nodes to our current framework. Our proposed framework provides a flexible foundation for future work to develop more sophisticated models for simulating, measuring, and evaluating the impact of link recommendations.





